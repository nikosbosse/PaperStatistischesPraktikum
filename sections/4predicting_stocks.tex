\chapter{Predicting Stocks}\label{ch:predictions}


\section{Predictions Using Machine Learning}

\section{Predictions Using Time Series}
\subsection{Idea/Process and Evaluation}
irgendwas in der Richtung: wir benutzen Time Series Modelle, machen Predictions und gucken uns am Ende dann den Mean Squared Error an. Sinnvollerweise immer die ersten 10 Perioden verwerfen, um den MSE vergleichbar zu machen zwischen allen Gruppen, auch denen, bei denen die ersten paar Perioden nicht definiert sind. 	
In-Sample vs. Out of Sample Prediction?
Do we need a Training, Validation and Test Period? Probably yes, since the respective next value was also used for validation, not only testing (we choose the model that minimizes MSE)

\subsection{Data Preparation}
% Überlegung: Data Preparation Steps hier oder bei Data?

We can clearly see that the Data exhibits a trend. As Figure \ref{fig:cum_sd_all} shows the variance of the time series is not constant over time. 

\begin{figure}[h]
    \centering
    \begin{adjustbox}{width=.9\textwidth,center}
    %% Creator: Matplotlib, PGF backend
%%
%% To include the figure in your LaTeX document, write
%%   \input{<filename>.pgf}
%%
%% Make sure the required packages are loaded in your preamble
%%   \usepackage{pgf}
%%
%% Figures using additional raster images can only be included by \input if
%% they are in the same directory as the main LaTeX file. For loading figures
%% from other directories you can use the `import` package
%%   \usepackage{import}
%% and then include the figures with
%%   \import{<path to file>}{<filename>.pgf}
%%
%% Matplotlib used the following preamble
%%   \usepackage{fontspec}
%%   \setmainfont{DejaVuSerif.ttf}[Path=/opt/tljh/user/lib/python3.6/site-packages/matplotlib/mpl-data/fonts/ttf/]
%%   \setsansfont{DejaVuSans.ttf}[Path=/opt/tljh/user/lib/python3.6/site-packages/matplotlib/mpl-data/fonts/ttf/]
%%   \setmonofont{DejaVuSansMono.ttf}[Path=/opt/tljh/user/lib/python3.6/site-packages/matplotlib/mpl-data/fonts/ttf/]
%%
\begingroup%
\makeatletter%
\begin{pgfpicture}%
\pgfpathrectangle{\pgfpointorigin}{\pgfqpoint{6.863921in}{3.474064in}}%
\pgfusepath{use as bounding box, clip}%
\begin{pgfscope}%
\pgfsetbuttcap%
\pgfsetmiterjoin%
\definecolor{currentfill}{rgb}{1.000000,1.000000,1.000000}%
\pgfsetfillcolor{currentfill}%
\pgfsetlinewidth{0.000000pt}%
\definecolor{currentstroke}{rgb}{1.000000,1.000000,1.000000}%
\pgfsetstrokecolor{currentstroke}%
\pgfsetdash{}{0pt}%
\pgfpathmoveto{\pgfqpoint{0.000000in}{0.000000in}}%
\pgfpathlineto{\pgfqpoint{6.863921in}{0.000000in}}%
\pgfpathlineto{\pgfqpoint{6.863921in}{3.474064in}}%
\pgfpathlineto{\pgfqpoint{0.000000in}{3.474064in}}%
\pgfpathclose%
\pgfusepath{fill}%
\end{pgfscope}%
\begin{pgfscope}%
\pgfsetbuttcap%
\pgfsetmiterjoin%
\definecolor{currentfill}{rgb}{0.917647,0.917647,0.949020}%
\pgfsetfillcolor{currentfill}%
\pgfsetlinewidth{0.000000pt}%
\definecolor{currentstroke}{rgb}{0.000000,0.000000,0.000000}%
\pgfsetstrokecolor{currentstroke}%
\pgfsetstrokeopacity{0.000000}%
\pgfsetdash{}{0pt}%
\pgfpathmoveto{\pgfqpoint{0.563921in}{0.521603in}}%
\pgfpathlineto{\pgfqpoint{6.763921in}{0.521603in}}%
\pgfpathlineto{\pgfqpoint{6.763921in}{3.164103in}}%
\pgfpathlineto{\pgfqpoint{0.563921in}{3.164103in}}%
\pgfpathclose%
\pgfusepath{fill}%
\end{pgfscope}%
\begin{pgfscope}%
\pgfpathrectangle{\pgfqpoint{0.563921in}{0.521603in}}{\pgfqpoint{6.200000in}{2.642500in}}%
\pgfusepath{clip}%
\pgfsetroundcap%
\pgfsetroundjoin%
\pgfsetlinewidth{0.803000pt}%
\definecolor{currentstroke}{rgb}{1.000000,1.000000,1.000000}%
\pgfsetstrokecolor{currentstroke}%
\pgfsetdash{}{0pt}%
\pgfpathmoveto{\pgfqpoint{0.840585in}{0.521603in}}%
\pgfpathlineto{\pgfqpoint{0.840585in}{3.164103in}}%
\pgfusepath{stroke}%
\end{pgfscope}%
\begin{pgfscope}%
\definecolor{textcolor}{rgb}{0.150000,0.150000,0.150000}%
\pgfsetstrokecolor{textcolor}%
\pgfsetfillcolor{textcolor}%
\pgftext[x=0.840585in,y=0.424381in,,top]{\color{textcolor}\rmfamily\fontsize{10.000000}{12.000000}\selectfont 2012}%
\end{pgfscope}%
\begin{pgfscope}%
\pgfpathrectangle{\pgfqpoint{0.563921in}{0.521603in}}{\pgfqpoint{6.200000in}{2.642500in}}%
\pgfusepath{clip}%
\pgfsetroundcap%
\pgfsetroundjoin%
\pgfsetlinewidth{0.803000pt}%
\definecolor{currentstroke}{rgb}{1.000000,1.000000,1.000000}%
\pgfsetstrokecolor{currentstroke}%
\pgfsetdash{}{0pt}%
\pgfpathmoveto{\pgfqpoint{1.783845in}{0.521603in}}%
\pgfpathlineto{\pgfqpoint{1.783845in}{3.164103in}}%
\pgfusepath{stroke}%
\end{pgfscope}%
\begin{pgfscope}%
\definecolor{textcolor}{rgb}{0.150000,0.150000,0.150000}%
\pgfsetstrokecolor{textcolor}%
\pgfsetfillcolor{textcolor}%
\pgftext[x=1.783845in,y=0.424381in,,top]{\color{textcolor}\rmfamily\fontsize{10.000000}{12.000000}\selectfont 2013}%
\end{pgfscope}%
\begin{pgfscope}%
\pgfpathrectangle{\pgfqpoint{0.563921in}{0.521603in}}{\pgfqpoint{6.200000in}{2.642500in}}%
\pgfusepath{clip}%
\pgfsetroundcap%
\pgfsetroundjoin%
\pgfsetlinewidth{0.803000pt}%
\definecolor{currentstroke}{rgb}{1.000000,1.000000,1.000000}%
\pgfsetstrokecolor{currentstroke}%
\pgfsetdash{}{0pt}%
\pgfpathmoveto{\pgfqpoint{2.724527in}{0.521603in}}%
\pgfpathlineto{\pgfqpoint{2.724527in}{3.164103in}}%
\pgfusepath{stroke}%
\end{pgfscope}%
\begin{pgfscope}%
\definecolor{textcolor}{rgb}{0.150000,0.150000,0.150000}%
\pgfsetstrokecolor{textcolor}%
\pgfsetfillcolor{textcolor}%
\pgftext[x=2.724527in,y=0.424381in,,top]{\color{textcolor}\rmfamily\fontsize{10.000000}{12.000000}\selectfont 2014}%
\end{pgfscope}%
\begin{pgfscope}%
\pgfpathrectangle{\pgfqpoint{0.563921in}{0.521603in}}{\pgfqpoint{6.200000in}{2.642500in}}%
\pgfusepath{clip}%
\pgfsetroundcap%
\pgfsetroundjoin%
\pgfsetlinewidth{0.803000pt}%
\definecolor{currentstroke}{rgb}{1.000000,1.000000,1.000000}%
\pgfsetstrokecolor{currentstroke}%
\pgfsetdash{}{0pt}%
\pgfpathmoveto{\pgfqpoint{3.665210in}{0.521603in}}%
\pgfpathlineto{\pgfqpoint{3.665210in}{3.164103in}}%
\pgfusepath{stroke}%
\end{pgfscope}%
\begin{pgfscope}%
\definecolor{textcolor}{rgb}{0.150000,0.150000,0.150000}%
\pgfsetstrokecolor{textcolor}%
\pgfsetfillcolor{textcolor}%
\pgftext[x=3.665210in,y=0.424381in,,top]{\color{textcolor}\rmfamily\fontsize{10.000000}{12.000000}\selectfont 2015}%
\end{pgfscope}%
\begin{pgfscope}%
\pgfpathrectangle{\pgfqpoint{0.563921in}{0.521603in}}{\pgfqpoint{6.200000in}{2.642500in}}%
\pgfusepath{clip}%
\pgfsetroundcap%
\pgfsetroundjoin%
\pgfsetlinewidth{0.803000pt}%
\definecolor{currentstroke}{rgb}{1.000000,1.000000,1.000000}%
\pgfsetstrokecolor{currentstroke}%
\pgfsetdash{}{0pt}%
\pgfpathmoveto{\pgfqpoint{4.605892in}{0.521603in}}%
\pgfpathlineto{\pgfqpoint{4.605892in}{3.164103in}}%
\pgfusepath{stroke}%
\end{pgfscope}%
\begin{pgfscope}%
\definecolor{textcolor}{rgb}{0.150000,0.150000,0.150000}%
\pgfsetstrokecolor{textcolor}%
\pgfsetfillcolor{textcolor}%
\pgftext[x=4.605892in,y=0.424381in,,top]{\color{textcolor}\rmfamily\fontsize{10.000000}{12.000000}\selectfont 2016}%
\end{pgfscope}%
\begin{pgfscope}%
\pgfpathrectangle{\pgfqpoint{0.563921in}{0.521603in}}{\pgfqpoint{6.200000in}{2.642500in}}%
\pgfusepath{clip}%
\pgfsetroundcap%
\pgfsetroundjoin%
\pgfsetlinewidth{0.803000pt}%
\definecolor{currentstroke}{rgb}{1.000000,1.000000,1.000000}%
\pgfsetstrokecolor{currentstroke}%
\pgfsetdash{}{0pt}%
\pgfpathmoveto{\pgfqpoint{5.549152in}{0.521603in}}%
\pgfpathlineto{\pgfqpoint{5.549152in}{3.164103in}}%
\pgfusepath{stroke}%
\end{pgfscope}%
\begin{pgfscope}%
\definecolor{textcolor}{rgb}{0.150000,0.150000,0.150000}%
\pgfsetstrokecolor{textcolor}%
\pgfsetfillcolor{textcolor}%
\pgftext[x=5.549152in,y=0.424381in,,top]{\color{textcolor}\rmfamily\fontsize{10.000000}{12.000000}\selectfont 2017}%
\end{pgfscope}%
\begin{pgfscope}%
\pgfpathrectangle{\pgfqpoint{0.563921in}{0.521603in}}{\pgfqpoint{6.200000in}{2.642500in}}%
\pgfusepath{clip}%
\pgfsetroundcap%
\pgfsetroundjoin%
\pgfsetlinewidth{0.803000pt}%
\definecolor{currentstroke}{rgb}{1.000000,1.000000,1.000000}%
\pgfsetstrokecolor{currentstroke}%
\pgfsetdash{}{0pt}%
\pgfpathmoveto{\pgfqpoint{6.489835in}{0.521603in}}%
\pgfpathlineto{\pgfqpoint{6.489835in}{3.164103in}}%
\pgfusepath{stroke}%
\end{pgfscope}%
\begin{pgfscope}%
\definecolor{textcolor}{rgb}{0.150000,0.150000,0.150000}%
\pgfsetstrokecolor{textcolor}%
\pgfsetfillcolor{textcolor}%
\pgftext[x=6.489835in,y=0.424381in,,top]{\color{textcolor}\rmfamily\fontsize{10.000000}{12.000000}\selectfont 2018}%
\end{pgfscope}%
\begin{pgfscope}%
\definecolor{textcolor}{rgb}{0.150000,0.150000,0.150000}%
\pgfsetstrokecolor{textcolor}%
\pgfsetfillcolor{textcolor}%
\pgftext[x=3.663921in,y=0.234413in,,top]{\color{textcolor}\rmfamily\fontsize{10.000000}{12.000000}\selectfont Time t}%
\end{pgfscope}%
\begin{pgfscope}%
\pgfpathrectangle{\pgfqpoint{0.563921in}{0.521603in}}{\pgfqpoint{6.200000in}{2.642500in}}%
\pgfusepath{clip}%
\pgfsetroundcap%
\pgfsetroundjoin%
\pgfsetlinewidth{0.803000pt}%
\definecolor{currentstroke}{rgb}{1.000000,1.000000,1.000000}%
\pgfsetstrokecolor{currentstroke}%
\pgfsetdash{}{0pt}%
\pgfpathmoveto{\pgfqpoint{0.563921in}{0.641717in}}%
\pgfpathlineto{\pgfqpoint{6.763921in}{0.641717in}}%
\pgfusepath{stroke}%
\end{pgfscope}%
\begin{pgfscope}%
\definecolor{textcolor}{rgb}{0.150000,0.150000,0.150000}%
\pgfsetstrokecolor{textcolor}%
\pgfsetfillcolor{textcolor}%
\pgftext[x=0.378334in,y=0.588955in,left,base]{\color{textcolor}\rmfamily\fontsize{10.000000}{12.000000}\selectfont 0}%
\end{pgfscope}%
\begin{pgfscope}%
\pgfpathrectangle{\pgfqpoint{0.563921in}{0.521603in}}{\pgfqpoint{6.200000in}{2.642500in}}%
\pgfusepath{clip}%
\pgfsetroundcap%
\pgfsetroundjoin%
\pgfsetlinewidth{0.803000pt}%
\definecolor{currentstroke}{rgb}{1.000000,1.000000,1.000000}%
\pgfsetstrokecolor{currentstroke}%
\pgfsetdash{}{0pt}%
\pgfpathmoveto{\pgfqpoint{0.563921in}{0.939662in}}%
\pgfpathlineto{\pgfqpoint{6.763921in}{0.939662in}}%
\pgfusepath{stroke}%
\end{pgfscope}%
\begin{pgfscope}%
\definecolor{textcolor}{rgb}{0.150000,0.150000,0.150000}%
\pgfsetstrokecolor{textcolor}%
\pgfsetfillcolor{textcolor}%
\pgftext[x=0.378334in,y=0.886901in,left,base]{\color{textcolor}\rmfamily\fontsize{10.000000}{12.000000}\selectfont 5}%
\end{pgfscope}%
\begin{pgfscope}%
\pgfpathrectangle{\pgfqpoint{0.563921in}{0.521603in}}{\pgfqpoint{6.200000in}{2.642500in}}%
\pgfusepath{clip}%
\pgfsetroundcap%
\pgfsetroundjoin%
\pgfsetlinewidth{0.803000pt}%
\definecolor{currentstroke}{rgb}{1.000000,1.000000,1.000000}%
\pgfsetstrokecolor{currentstroke}%
\pgfsetdash{}{0pt}%
\pgfpathmoveto{\pgfqpoint{0.563921in}{1.237607in}}%
\pgfpathlineto{\pgfqpoint{6.763921in}{1.237607in}}%
\pgfusepath{stroke}%
\end{pgfscope}%
\begin{pgfscope}%
\definecolor{textcolor}{rgb}{0.150000,0.150000,0.150000}%
\pgfsetstrokecolor{textcolor}%
\pgfsetfillcolor{textcolor}%
\pgftext[x=0.289968in,y=1.184846in,left,base]{\color{textcolor}\rmfamily\fontsize{10.000000}{12.000000}\selectfont 10}%
\end{pgfscope}%
\begin{pgfscope}%
\pgfpathrectangle{\pgfqpoint{0.563921in}{0.521603in}}{\pgfqpoint{6.200000in}{2.642500in}}%
\pgfusepath{clip}%
\pgfsetroundcap%
\pgfsetroundjoin%
\pgfsetlinewidth{0.803000pt}%
\definecolor{currentstroke}{rgb}{1.000000,1.000000,1.000000}%
\pgfsetstrokecolor{currentstroke}%
\pgfsetdash{}{0pt}%
\pgfpathmoveto{\pgfqpoint{0.563921in}{1.535553in}}%
\pgfpathlineto{\pgfqpoint{6.763921in}{1.535553in}}%
\pgfusepath{stroke}%
\end{pgfscope}%
\begin{pgfscope}%
\definecolor{textcolor}{rgb}{0.150000,0.150000,0.150000}%
\pgfsetstrokecolor{textcolor}%
\pgfsetfillcolor{textcolor}%
\pgftext[x=0.289968in,y=1.482791in,left,base]{\color{textcolor}\rmfamily\fontsize{10.000000}{12.000000}\selectfont 15}%
\end{pgfscope}%
\begin{pgfscope}%
\pgfpathrectangle{\pgfqpoint{0.563921in}{0.521603in}}{\pgfqpoint{6.200000in}{2.642500in}}%
\pgfusepath{clip}%
\pgfsetroundcap%
\pgfsetroundjoin%
\pgfsetlinewidth{0.803000pt}%
\definecolor{currentstroke}{rgb}{1.000000,1.000000,1.000000}%
\pgfsetstrokecolor{currentstroke}%
\pgfsetdash{}{0pt}%
\pgfpathmoveto{\pgfqpoint{0.563921in}{1.833498in}}%
\pgfpathlineto{\pgfqpoint{6.763921in}{1.833498in}}%
\pgfusepath{stroke}%
\end{pgfscope}%
\begin{pgfscope}%
\definecolor{textcolor}{rgb}{0.150000,0.150000,0.150000}%
\pgfsetstrokecolor{textcolor}%
\pgfsetfillcolor{textcolor}%
\pgftext[x=0.289968in,y=1.780736in,left,base]{\color{textcolor}\rmfamily\fontsize{10.000000}{12.000000}\selectfont 20}%
\end{pgfscope}%
\begin{pgfscope}%
\pgfpathrectangle{\pgfqpoint{0.563921in}{0.521603in}}{\pgfqpoint{6.200000in}{2.642500in}}%
\pgfusepath{clip}%
\pgfsetroundcap%
\pgfsetroundjoin%
\pgfsetlinewidth{0.803000pt}%
\definecolor{currentstroke}{rgb}{1.000000,1.000000,1.000000}%
\pgfsetstrokecolor{currentstroke}%
\pgfsetdash{}{0pt}%
\pgfpathmoveto{\pgfqpoint{0.563921in}{2.131443in}}%
\pgfpathlineto{\pgfqpoint{6.763921in}{2.131443in}}%
\pgfusepath{stroke}%
\end{pgfscope}%
\begin{pgfscope}%
\definecolor{textcolor}{rgb}{0.150000,0.150000,0.150000}%
\pgfsetstrokecolor{textcolor}%
\pgfsetfillcolor{textcolor}%
\pgftext[x=0.289968in,y=2.078682in,left,base]{\color{textcolor}\rmfamily\fontsize{10.000000}{12.000000}\selectfont 25}%
\end{pgfscope}%
\begin{pgfscope}%
\pgfpathrectangle{\pgfqpoint{0.563921in}{0.521603in}}{\pgfqpoint{6.200000in}{2.642500in}}%
\pgfusepath{clip}%
\pgfsetroundcap%
\pgfsetroundjoin%
\pgfsetlinewidth{0.803000pt}%
\definecolor{currentstroke}{rgb}{1.000000,1.000000,1.000000}%
\pgfsetstrokecolor{currentstroke}%
\pgfsetdash{}{0pt}%
\pgfpathmoveto{\pgfqpoint{0.563921in}{2.429388in}}%
\pgfpathlineto{\pgfqpoint{6.763921in}{2.429388in}}%
\pgfusepath{stroke}%
\end{pgfscope}%
\begin{pgfscope}%
\definecolor{textcolor}{rgb}{0.150000,0.150000,0.150000}%
\pgfsetstrokecolor{textcolor}%
\pgfsetfillcolor{textcolor}%
\pgftext[x=0.289968in,y=2.376627in,left,base]{\color{textcolor}\rmfamily\fontsize{10.000000}{12.000000}\selectfont 30}%
\end{pgfscope}%
\begin{pgfscope}%
\pgfpathrectangle{\pgfqpoint{0.563921in}{0.521603in}}{\pgfqpoint{6.200000in}{2.642500in}}%
\pgfusepath{clip}%
\pgfsetroundcap%
\pgfsetroundjoin%
\pgfsetlinewidth{0.803000pt}%
\definecolor{currentstroke}{rgb}{1.000000,1.000000,1.000000}%
\pgfsetstrokecolor{currentstroke}%
\pgfsetdash{}{0pt}%
\pgfpathmoveto{\pgfqpoint{0.563921in}{2.727334in}}%
\pgfpathlineto{\pgfqpoint{6.763921in}{2.727334in}}%
\pgfusepath{stroke}%
\end{pgfscope}%
\begin{pgfscope}%
\definecolor{textcolor}{rgb}{0.150000,0.150000,0.150000}%
\pgfsetstrokecolor{textcolor}%
\pgfsetfillcolor{textcolor}%
\pgftext[x=0.289968in,y=2.674572in,left,base]{\color{textcolor}\rmfamily\fontsize{10.000000}{12.000000}\selectfont 35}%
\end{pgfscope}%
\begin{pgfscope}%
\pgfpathrectangle{\pgfqpoint{0.563921in}{0.521603in}}{\pgfqpoint{6.200000in}{2.642500in}}%
\pgfusepath{clip}%
\pgfsetroundcap%
\pgfsetroundjoin%
\pgfsetlinewidth{0.803000pt}%
\definecolor{currentstroke}{rgb}{1.000000,1.000000,1.000000}%
\pgfsetstrokecolor{currentstroke}%
\pgfsetdash{}{0pt}%
\pgfpathmoveto{\pgfqpoint{0.563921in}{3.025279in}}%
\pgfpathlineto{\pgfqpoint{6.763921in}{3.025279in}}%
\pgfusepath{stroke}%
\end{pgfscope}%
\begin{pgfscope}%
\definecolor{textcolor}{rgb}{0.150000,0.150000,0.150000}%
\pgfsetstrokecolor{textcolor}%
\pgfsetfillcolor{textcolor}%
\pgftext[x=0.289968in,y=2.972517in,left,base]{\color{textcolor}\rmfamily\fontsize{10.000000}{12.000000}\selectfont 40}%
\end{pgfscope}%
\begin{pgfscope}%
\definecolor{textcolor}{rgb}{0.150000,0.150000,0.150000}%
\pgfsetstrokecolor{textcolor}%
\pgfsetfillcolor{textcolor}%
\pgftext[x=0.234413in,y=1.842853in,,bottom,rotate=90.000000]{\color{textcolor}\rmfamily\fontsize{10.000000}{12.000000}\selectfont Standard Deviation}%
\end{pgfscope}%
\begin{pgfscope}%
\pgfpathrectangle{\pgfqpoint{0.563921in}{0.521603in}}{\pgfqpoint{6.200000in}{2.642500in}}%
\pgfusepath{clip}%
\pgfsetroundcap%
\pgfsetroundjoin%
\pgfsetlinewidth{1.505625pt}%
\definecolor{currentstroke}{rgb}{0.121569,0.466667,0.705882}%
\pgfsetstrokecolor{currentstroke}%
\pgfsetdash{}{0pt}%
\pgfpathmoveto{\pgfqpoint{0.845739in}{0.641717in}}%
\pgfpathlineto{\pgfqpoint{0.848317in}{0.658700in}}%
\pgfpathlineto{\pgfqpoint{0.850894in}{0.655601in}}%
\pgfpathlineto{\pgfqpoint{0.853471in}{0.657178in}}%
\pgfpathlineto{\pgfqpoint{0.861203in}{0.655876in}}%
\pgfpathlineto{\pgfqpoint{0.863780in}{0.658219in}}%
\pgfpathlineto{\pgfqpoint{0.866357in}{0.657038in}}%
\pgfpathlineto{\pgfqpoint{0.868934in}{0.657822in}}%
\pgfpathlineto{\pgfqpoint{0.871512in}{0.657520in}}%
\pgfpathlineto{\pgfqpoint{0.881820in}{0.657714in}}%
\pgfpathlineto{\pgfqpoint{0.884398in}{0.664412in}}%
\pgfpathlineto{\pgfqpoint{0.886975in}{0.674516in}}%
\pgfpathlineto{\pgfqpoint{0.889552in}{0.678938in}}%
\pgfpathlineto{\pgfqpoint{0.897284in}{0.681434in}}%
\pgfpathlineto{\pgfqpoint{0.899861in}{0.684693in}}%
\pgfpathlineto{\pgfqpoint{0.902438in}{0.689764in}}%
\pgfpathlineto{\pgfqpoint{0.905015in}{0.699797in}}%
\pgfpathlineto{\pgfqpoint{0.907593in}{0.705818in}}%
\pgfpathlineto{\pgfqpoint{0.920479in}{0.712517in}}%
\pgfpathlineto{\pgfqpoint{0.925633in}{0.716874in}}%
\pgfpathlineto{\pgfqpoint{0.933365in}{0.718223in}}%
\pgfpathlineto{\pgfqpoint{0.938519in}{0.721714in}}%
\pgfpathlineto{\pgfqpoint{0.941096in}{0.723123in}}%
\pgfpathlineto{\pgfqpoint{0.943674in}{0.722600in}}%
\pgfpathlineto{\pgfqpoint{0.959137in}{0.724998in}}%
\pgfpathlineto{\pgfqpoint{0.961714in}{0.725513in}}%
\pgfpathlineto{\pgfqpoint{0.974600in}{0.726569in}}%
\pgfpathlineto{\pgfqpoint{0.979754in}{0.728341in}}%
\pgfpathlineto{\pgfqpoint{0.992641in}{0.728982in}}%
\pgfpathlineto{\pgfqpoint{1.008104in}{0.727125in}}%
\pgfpathlineto{\pgfqpoint{1.015835in}{0.724771in}}%
\pgfpathlineto{\pgfqpoint{1.023567in}{0.724558in}}%
\pgfpathlineto{\pgfqpoint{1.028722in}{0.727060in}}%
\pgfpathlineto{\pgfqpoint{1.031299in}{0.730154in}}%
\pgfpathlineto{\pgfqpoint{1.033876in}{0.732095in}}%
\pgfpathlineto{\pgfqpoint{1.049339in}{0.736017in}}%
\pgfpathlineto{\pgfqpoint{1.051916in}{0.736007in}}%
\pgfpathlineto{\pgfqpoint{1.080266in}{0.738117in}}%
\pgfpathlineto{\pgfqpoint{1.085420in}{0.736871in}}%
\pgfpathlineto{\pgfqpoint{1.106038in}{0.735102in}}%
\pgfpathlineto{\pgfqpoint{1.116347in}{0.733921in}}%
\pgfpathlineto{\pgfqpoint{1.124078in}{0.732187in}}%
\pgfpathlineto{\pgfqpoint{1.136964in}{0.731630in}}%
\pgfpathlineto{\pgfqpoint{1.142119in}{0.732548in}}%
\pgfpathlineto{\pgfqpoint{1.152428in}{0.733541in}}%
\pgfpathlineto{\pgfqpoint{1.157582in}{0.734353in}}%
\pgfpathlineto{\pgfqpoint{1.170468in}{0.733340in}}%
\pgfpathlineto{\pgfqpoint{1.178200in}{0.731841in}}%
\pgfpathlineto{\pgfqpoint{1.193663in}{0.731199in}}%
\pgfpathlineto{\pgfqpoint{1.196240in}{0.731943in}}%
\pgfpathlineto{\pgfqpoint{1.214281in}{0.731766in}}%
\pgfpathlineto{\pgfqpoint{1.229744in}{0.731411in}}%
\pgfpathlineto{\pgfqpoint{1.232321in}{0.732498in}}%
\pgfpathlineto{\pgfqpoint{1.240053in}{0.733725in}}%
\pgfpathlineto{\pgfqpoint{1.242630in}{0.734979in}}%
\pgfpathlineto{\pgfqpoint{1.258093in}{0.733729in}}%
\pgfpathlineto{\pgfqpoint{1.268402in}{0.732423in}}%
\pgfpathlineto{\pgfqpoint{1.283866in}{0.731618in}}%
\pgfpathlineto{\pgfqpoint{1.286443in}{0.731290in}}%
\pgfpathlineto{\pgfqpoint{1.304483in}{0.730991in}}%
\pgfpathlineto{\pgfqpoint{1.312215in}{0.731656in}}%
\pgfpathlineto{\pgfqpoint{1.314792in}{0.732580in}}%
\pgfpathlineto{\pgfqpoint{1.322524in}{0.733766in}}%
\pgfpathlineto{\pgfqpoint{1.335410in}{0.734017in}}%
\pgfpathlineto{\pgfqpoint{1.340564in}{0.733473in}}%
\pgfpathlineto{\pgfqpoint{1.350873in}{0.733751in}}%
\pgfpathlineto{\pgfqpoint{1.356027in}{0.736834in}}%
\pgfpathlineto{\pgfqpoint{1.358605in}{0.737646in}}%
\pgfpathlineto{\pgfqpoint{1.371491in}{0.738087in}}%
\pgfpathlineto{\pgfqpoint{1.374068in}{0.739209in}}%
\pgfpathlineto{\pgfqpoint{1.376645in}{0.741173in}}%
\pgfpathlineto{\pgfqpoint{1.384377in}{0.742688in}}%
\pgfpathlineto{\pgfqpoint{1.389531in}{0.745424in}}%
\pgfpathlineto{\pgfqpoint{1.392108in}{0.745986in}}%
\pgfpathlineto{\pgfqpoint{1.394686in}{0.747591in}}%
\pgfpathlineto{\pgfqpoint{1.402417in}{0.748914in}}%
\pgfpathlineto{\pgfqpoint{1.412726in}{0.754756in}}%
\pgfpathlineto{\pgfqpoint{1.420458in}{0.756498in}}%
\pgfpathlineto{\pgfqpoint{1.425612in}{0.759785in}}%
\pgfpathlineto{\pgfqpoint{1.430767in}{0.765275in}}%
\pgfpathlineto{\pgfqpoint{1.438498in}{0.767742in}}%
\pgfpathlineto{\pgfqpoint{1.448807in}{0.774412in}}%
\pgfpathlineto{\pgfqpoint{1.456539in}{0.775953in}}%
\pgfpathlineto{\pgfqpoint{1.466848in}{0.780839in}}%
\pgfpathlineto{\pgfqpoint{1.477156in}{0.781629in}}%
\pgfpathlineto{\pgfqpoint{1.479734in}{0.782426in}}%
\pgfpathlineto{\pgfqpoint{1.484888in}{0.785367in}}%
\pgfpathlineto{\pgfqpoint{1.497774in}{0.786314in}}%
\pgfpathlineto{\pgfqpoint{1.500351in}{0.787103in}}%
\pgfpathlineto{\pgfqpoint{1.502929in}{0.788951in}}%
\pgfpathlineto{\pgfqpoint{1.510660in}{0.790605in}}%
\pgfpathlineto{\pgfqpoint{1.520969in}{0.795969in}}%
\pgfpathlineto{\pgfqpoint{1.531278in}{0.798212in}}%
\pgfpathlineto{\pgfqpoint{1.539010in}{0.800370in}}%
\pgfpathlineto{\pgfqpoint{1.546741in}{0.801352in}}%
\pgfpathlineto{\pgfqpoint{1.554473in}{0.805040in}}%
\pgfpathlineto{\pgfqpoint{1.557050in}{0.806751in}}%
\pgfpathlineto{\pgfqpoint{1.575091in}{0.811491in}}%
\pgfpathlineto{\pgfqpoint{1.582822in}{0.811984in}}%
\pgfpathlineto{\pgfqpoint{1.593131in}{0.816090in}}%
\pgfpathlineto{\pgfqpoint{1.603440in}{0.815990in}}%
\pgfpathlineto{\pgfqpoint{1.611172in}{0.814736in}}%
\pgfpathlineto{\pgfqpoint{1.626635in}{0.813970in}}%
\pgfpathlineto{\pgfqpoint{1.629212in}{0.813594in}}%
\pgfpathlineto{\pgfqpoint{1.644675in}{0.812425in}}%
\pgfpathlineto{\pgfqpoint{1.647253in}{0.812050in}}%
\pgfpathlineto{\pgfqpoint{1.660139in}{0.810978in}}%
\pgfpathlineto{\pgfqpoint{1.665293in}{0.810218in}}%
\pgfpathlineto{\pgfqpoint{1.716837in}{0.808676in}}%
\pgfpathlineto{\pgfqpoint{1.729723in}{0.809852in}}%
\pgfpathlineto{\pgfqpoint{1.737455in}{0.811119in}}%
\pgfpathlineto{\pgfqpoint{1.745187in}{0.811641in}}%
\pgfpathlineto{\pgfqpoint{1.755495in}{0.814283in}}%
\pgfpathlineto{\pgfqpoint{1.781268in}{0.816010in}}%
\pgfpathlineto{\pgfqpoint{1.788999in}{0.817960in}}%
\pgfpathlineto{\pgfqpoint{1.791576in}{0.819160in}}%
\pgfpathlineto{\pgfqpoint{1.799308in}{0.820379in}}%
\pgfpathlineto{\pgfqpoint{1.804462in}{0.823129in}}%
\pgfpathlineto{\pgfqpoint{1.809617in}{0.826288in}}%
\pgfpathlineto{\pgfqpoint{1.817349in}{0.828031in}}%
\pgfpathlineto{\pgfqpoint{1.825080in}{0.833800in}}%
\pgfpathlineto{\pgfqpoint{1.827657in}{0.836154in}}%
\pgfpathlineto{\pgfqpoint{1.837966in}{0.838751in}}%
\pgfpathlineto{\pgfqpoint{1.845698in}{0.847037in}}%
\pgfpathlineto{\pgfqpoint{1.853430in}{0.850028in}}%
\pgfpathlineto{\pgfqpoint{1.858584in}{0.856477in}}%
\pgfpathlineto{\pgfqpoint{1.863738in}{0.862285in}}%
\pgfpathlineto{\pgfqpoint{1.871470in}{0.864917in}}%
\pgfpathlineto{\pgfqpoint{1.879202in}{0.874458in}}%
\pgfpathlineto{\pgfqpoint{1.881779in}{0.877728in}}%
\pgfpathlineto{\pgfqpoint{1.889510in}{0.880885in}}%
\pgfpathlineto{\pgfqpoint{1.899819in}{0.894540in}}%
\pgfpathlineto{\pgfqpoint{1.910128in}{0.898382in}}%
\pgfpathlineto{\pgfqpoint{1.917860in}{0.907689in}}%
\pgfpathlineto{\pgfqpoint{1.925591in}{0.909998in}}%
\pgfpathlineto{\pgfqpoint{1.930746in}{0.915511in}}%
\pgfpathlineto{\pgfqpoint{1.935900in}{0.921600in}}%
\pgfpathlineto{\pgfqpoint{1.943632in}{0.924257in}}%
\pgfpathlineto{\pgfqpoint{1.953941in}{0.937004in}}%
\pgfpathlineto{\pgfqpoint{1.961672in}{0.940440in}}%
\pgfpathlineto{\pgfqpoint{1.971981in}{0.953094in}}%
\pgfpathlineto{\pgfqpoint{1.979713in}{0.955958in}}%
\pgfpathlineto{\pgfqpoint{1.990022in}{0.967013in}}%
\pgfpathlineto{\pgfqpoint{1.997753in}{0.969468in}}%
\pgfpathlineto{\pgfqpoint{2.005485in}{0.977360in}}%
\pgfpathlineto{\pgfqpoint{2.015794in}{0.979781in}}%
\pgfpathlineto{\pgfqpoint{2.023526in}{0.987149in}}%
\pgfpathlineto{\pgfqpoint{2.026103in}{0.989415in}}%
\pgfpathlineto{\pgfqpoint{2.033834in}{0.991626in}}%
\pgfpathlineto{\pgfqpoint{2.038989in}{0.996708in}}%
\pgfpathlineto{\pgfqpoint{2.044143in}{1.002423in}}%
\pgfpathlineto{\pgfqpoint{2.051875in}{1.004399in}}%
\pgfpathlineto{\pgfqpoint{2.059606in}{1.009888in}}%
\pgfpathlineto{\pgfqpoint{2.062184in}{1.011699in}}%
\pgfpathlineto{\pgfqpoint{2.069915in}{1.013506in}}%
\pgfpathlineto{\pgfqpoint{2.075070in}{1.018226in}}%
\pgfpathlineto{\pgfqpoint{2.080224in}{1.020745in}}%
\pgfpathlineto{\pgfqpoint{2.087956in}{1.021834in}}%
\pgfpathlineto{\pgfqpoint{2.095687in}{1.026012in}}%
\pgfpathlineto{\pgfqpoint{2.098265in}{1.028180in}}%
\pgfpathlineto{\pgfqpoint{2.105996in}{1.030307in}}%
\pgfpathlineto{\pgfqpoint{2.113728in}{1.037323in}}%
\pgfpathlineto{\pgfqpoint{2.116305in}{1.040177in}}%
\pgfpathlineto{\pgfqpoint{2.124037in}{1.042983in}}%
\pgfpathlineto{\pgfqpoint{2.134346in}{1.054545in}}%
\pgfpathlineto{\pgfqpoint{2.142077in}{1.057500in}}%
\pgfpathlineto{\pgfqpoint{2.149809in}{1.065793in}}%
\pgfpathlineto{\pgfqpoint{2.152386in}{1.068257in}}%
\pgfpathlineto{\pgfqpoint{2.162695in}{1.071116in}}%
\pgfpathlineto{\pgfqpoint{2.170427in}{1.078742in}}%
\pgfpathlineto{\pgfqpoint{2.178158in}{1.081084in}}%
\pgfpathlineto{\pgfqpoint{2.183313in}{1.084874in}}%
\pgfpathlineto{\pgfqpoint{2.185890in}{1.086540in}}%
\pgfpathlineto{\pgfqpoint{2.188467in}{1.088887in}}%
\pgfpathlineto{\pgfqpoint{2.196199in}{1.091101in}}%
\pgfpathlineto{\pgfqpoint{2.206508in}{1.099103in}}%
\pgfpathlineto{\pgfqpoint{2.214239in}{1.101466in}}%
\pgfpathlineto{\pgfqpoint{2.219394in}{1.106293in}}%
\pgfpathlineto{\pgfqpoint{2.224548in}{1.109252in}}%
\pgfpathlineto{\pgfqpoint{2.232280in}{1.110295in}}%
\pgfpathlineto{\pgfqpoint{2.237434in}{1.113014in}}%
\pgfpathlineto{\pgfqpoint{2.242589in}{1.116052in}}%
\pgfpathlineto{\pgfqpoint{2.250320in}{1.117414in}}%
\pgfpathlineto{\pgfqpoint{2.260629in}{1.121831in}}%
\pgfpathlineto{\pgfqpoint{2.268361in}{1.123822in}}%
\pgfpathlineto{\pgfqpoint{2.278670in}{1.133675in}}%
\pgfpathlineto{\pgfqpoint{2.286401in}{1.136247in}}%
\pgfpathlineto{\pgfqpoint{2.296710in}{1.146498in}}%
\pgfpathlineto{\pgfqpoint{2.304442in}{1.149278in}}%
\pgfpathlineto{\pgfqpoint{2.314751in}{1.160362in}}%
\pgfpathlineto{\pgfqpoint{2.322482in}{1.163013in}}%
\pgfpathlineto{\pgfqpoint{2.332791in}{1.174477in}}%
\pgfpathlineto{\pgfqpoint{2.340523in}{1.177277in}}%
\pgfpathlineto{\pgfqpoint{2.350832in}{1.188218in}}%
\pgfpathlineto{\pgfqpoint{2.358563in}{1.190929in}}%
\pgfpathlineto{\pgfqpoint{2.363718in}{1.196094in}}%
\pgfpathlineto{\pgfqpoint{2.368872in}{1.199933in}}%
\pgfpathlineto{\pgfqpoint{2.376604in}{1.201747in}}%
\pgfpathlineto{\pgfqpoint{2.386912in}{1.208141in}}%
\pgfpathlineto{\pgfqpoint{2.394644in}{1.209683in}}%
\pgfpathlineto{\pgfqpoint{2.404953in}{1.214649in}}%
\pgfpathlineto{\pgfqpoint{2.415262in}{1.215858in}}%
\pgfpathlineto{\pgfqpoint{2.422993in}{1.220383in}}%
\pgfpathlineto{\pgfqpoint{2.430725in}{1.222281in}}%
\pgfpathlineto{\pgfqpoint{2.441034in}{1.231063in}}%
\pgfpathlineto{\pgfqpoint{2.448766in}{1.233412in}}%
\pgfpathlineto{\pgfqpoint{2.459074in}{1.243965in}}%
\pgfpathlineto{\pgfqpoint{2.466806in}{1.246609in}}%
\pgfpathlineto{\pgfqpoint{2.477115in}{1.256172in}}%
\pgfpathlineto{\pgfqpoint{2.484847in}{1.258227in}}%
\pgfpathlineto{\pgfqpoint{2.495155in}{1.265972in}}%
\pgfpathlineto{\pgfqpoint{2.502887in}{1.267732in}}%
\pgfpathlineto{\pgfqpoint{2.510619in}{1.272465in}}%
\pgfpathlineto{\pgfqpoint{2.513196in}{1.274580in}}%
\pgfpathlineto{\pgfqpoint{2.520928in}{1.276824in}}%
\pgfpathlineto{\pgfqpoint{2.526082in}{1.280761in}}%
\pgfpathlineto{\pgfqpoint{2.531236in}{1.285663in}}%
\pgfpathlineto{\pgfqpoint{2.538968in}{1.288189in}}%
\pgfpathlineto{\pgfqpoint{2.549277in}{1.298399in}}%
\pgfpathlineto{\pgfqpoint{2.557009in}{1.301151in}}%
\pgfpathlineto{\pgfqpoint{2.567317in}{1.312301in}}%
\pgfpathlineto{\pgfqpoint{2.575049in}{1.315202in}}%
\pgfpathlineto{\pgfqpoint{2.585358in}{1.326996in}}%
\pgfpathlineto{\pgfqpoint{2.593089in}{1.330097in}}%
\pgfpathlineto{\pgfqpoint{2.603398in}{1.343264in}}%
\pgfpathlineto{\pgfqpoint{2.611130in}{1.346687in}}%
\pgfpathlineto{\pgfqpoint{2.621439in}{1.360419in}}%
\pgfpathlineto{\pgfqpoint{2.629170in}{1.364046in}}%
\pgfpathlineto{\pgfqpoint{2.634325in}{1.371986in}}%
\pgfpathlineto{\pgfqpoint{2.639479in}{1.376051in}}%
\pgfpathlineto{\pgfqpoint{2.647211in}{1.378627in}}%
\pgfpathlineto{\pgfqpoint{2.657520in}{1.388132in}}%
\pgfpathlineto{\pgfqpoint{2.665251in}{1.390745in}}%
\pgfpathlineto{\pgfqpoint{2.672983in}{1.397367in}}%
\pgfpathlineto{\pgfqpoint{2.675560in}{1.399410in}}%
\pgfpathlineto{\pgfqpoint{2.683292in}{1.401679in}}%
\pgfpathlineto{\pgfqpoint{2.685869in}{1.404740in}}%
\pgfpathlineto{\pgfqpoint{2.693601in}{1.417294in}}%
\pgfpathlineto{\pgfqpoint{2.701332in}{1.421488in}}%
\pgfpathlineto{\pgfqpoint{2.703910in}{1.425671in}}%
\pgfpathlineto{\pgfqpoint{2.709064in}{1.430137in}}%
\pgfpathlineto{\pgfqpoint{2.711641in}{1.434823in}}%
\pgfpathlineto{\pgfqpoint{2.719373in}{1.439459in}}%
\pgfpathlineto{\pgfqpoint{2.721950in}{1.444253in}}%
\pgfpathlineto{\pgfqpoint{2.727105in}{1.448421in}}%
\pgfpathlineto{\pgfqpoint{2.729682in}{1.452614in}}%
\pgfpathlineto{\pgfqpoint{2.737413in}{1.456544in}}%
\pgfpathlineto{\pgfqpoint{2.745145in}{1.467506in}}%
\pgfpathlineto{\pgfqpoint{2.747722in}{1.470895in}}%
\pgfpathlineto{\pgfqpoint{2.755454in}{1.473906in}}%
\pgfpathlineto{\pgfqpoint{2.765763in}{1.488379in}}%
\pgfpathlineto{\pgfqpoint{2.776072in}{1.491689in}}%
\pgfpathlineto{\pgfqpoint{2.781226in}{1.497587in}}%
\pgfpathlineto{\pgfqpoint{2.783803in}{1.499431in}}%
\pgfpathlineto{\pgfqpoint{2.791535in}{1.501034in}}%
\pgfpathlineto{\pgfqpoint{2.801844in}{1.507354in}}%
\pgfpathlineto{\pgfqpoint{2.809575in}{1.508102in}}%
\pgfpathlineto{\pgfqpoint{2.817307in}{1.511937in}}%
\pgfpathlineto{\pgfqpoint{2.819884in}{1.513618in}}%
\pgfpathlineto{\pgfqpoint{2.827616in}{1.515174in}}%
\pgfpathlineto{\pgfqpoint{2.837925in}{1.522484in}}%
\pgfpathlineto{\pgfqpoint{2.848234in}{1.524451in}}%
\pgfpathlineto{\pgfqpoint{2.855965in}{1.529924in}}%
\pgfpathlineto{\pgfqpoint{2.863697in}{1.531877in}}%
\pgfpathlineto{\pgfqpoint{2.874006in}{1.540551in}}%
\pgfpathlineto{\pgfqpoint{2.881737in}{1.542394in}}%
\pgfpathlineto{\pgfqpoint{2.892046in}{1.550578in}}%
\pgfpathlineto{\pgfqpoint{2.899778in}{1.552544in}}%
\pgfpathlineto{\pgfqpoint{2.907509in}{1.557525in}}%
\pgfpathlineto{\pgfqpoint{2.910087in}{1.558814in}}%
\pgfpathlineto{\pgfqpoint{2.917818in}{1.560471in}}%
\pgfpathlineto{\pgfqpoint{2.928127in}{1.566996in}}%
\pgfpathlineto{\pgfqpoint{2.935859in}{1.568589in}}%
\pgfpathlineto{\pgfqpoint{2.946168in}{1.575510in}}%
\pgfpathlineto{\pgfqpoint{2.953899in}{1.577545in}}%
\pgfpathlineto{\pgfqpoint{2.964208in}{1.585789in}}%
\pgfpathlineto{\pgfqpoint{2.971940in}{1.587511in}}%
\pgfpathlineto{\pgfqpoint{2.982249in}{1.594143in}}%
\pgfpathlineto{\pgfqpoint{2.989980in}{1.595586in}}%
\pgfpathlineto{\pgfqpoint{2.995135in}{1.599136in}}%
\pgfpathlineto{\pgfqpoint{2.997712in}{1.601257in}}%
\pgfpathlineto{\pgfqpoint{3.008021in}{1.603509in}}%
\pgfpathlineto{\pgfqpoint{3.015752in}{1.609736in}}%
\pgfpathlineto{\pgfqpoint{3.018330in}{1.611559in}}%
\pgfpathlineto{\pgfqpoint{3.026061in}{1.613469in}}%
\pgfpathlineto{\pgfqpoint{3.031216in}{1.617589in}}%
\pgfpathlineto{\pgfqpoint{3.036370in}{1.622345in}}%
\pgfpathlineto{\pgfqpoint{3.044102in}{1.624712in}}%
\pgfpathlineto{\pgfqpoint{3.051833in}{1.631571in}}%
\pgfpathlineto{\pgfqpoint{3.054411in}{1.633966in}}%
\pgfpathlineto{\pgfqpoint{3.062142in}{1.636595in}}%
\pgfpathlineto{\pgfqpoint{3.072451in}{1.645965in}}%
\pgfpathlineto{\pgfqpoint{3.080183in}{1.648270in}}%
\pgfpathlineto{\pgfqpoint{3.090491in}{1.656975in}}%
\pgfpathlineto{\pgfqpoint{3.100800in}{1.659273in}}%
\pgfpathlineto{\pgfqpoint{3.108532in}{1.666372in}}%
\pgfpathlineto{\pgfqpoint{3.116264in}{1.668722in}}%
\pgfpathlineto{\pgfqpoint{3.126572in}{1.678602in}}%
\pgfpathlineto{\pgfqpoint{3.134304in}{1.681342in}}%
\pgfpathlineto{\pgfqpoint{3.142036in}{1.688802in}}%
\pgfpathlineto{\pgfqpoint{3.144613in}{1.691108in}}%
\pgfpathlineto{\pgfqpoint{3.152345in}{1.693385in}}%
\pgfpathlineto{\pgfqpoint{3.162653in}{1.702981in}}%
\pgfpathlineto{\pgfqpoint{3.170385in}{1.705268in}}%
\pgfpathlineto{\pgfqpoint{3.180694in}{1.713851in}}%
\pgfpathlineto{\pgfqpoint{3.188426in}{1.715898in}}%
\pgfpathlineto{\pgfqpoint{3.196157in}{1.722774in}}%
\pgfpathlineto{\pgfqpoint{3.206466in}{1.724998in}}%
\pgfpathlineto{\pgfqpoint{3.216775in}{1.733346in}}%
\pgfpathlineto{\pgfqpoint{3.224507in}{1.735481in}}%
\pgfpathlineto{\pgfqpoint{3.234815in}{1.743768in}}%
\pgfpathlineto{\pgfqpoint{3.242547in}{1.745701in}}%
\pgfpathlineto{\pgfqpoint{3.252856in}{1.753671in}}%
\pgfpathlineto{\pgfqpoint{3.260588in}{1.755693in}}%
\pgfpathlineto{\pgfqpoint{3.268319in}{1.760537in}}%
\pgfpathlineto{\pgfqpoint{3.270896in}{1.761756in}}%
\pgfpathlineto{\pgfqpoint{3.278628in}{1.763044in}}%
\pgfpathlineto{\pgfqpoint{3.288937in}{1.767627in}}%
\pgfpathlineto{\pgfqpoint{3.296668in}{1.768837in}}%
\pgfpathlineto{\pgfqpoint{3.306977in}{1.774142in}}%
\pgfpathlineto{\pgfqpoint{3.314709in}{1.775748in}}%
\pgfpathlineto{\pgfqpoint{3.325018in}{1.782505in}}%
\pgfpathlineto{\pgfqpoint{3.332749in}{1.784231in}}%
\pgfpathlineto{\pgfqpoint{3.343058in}{1.790674in}}%
\pgfpathlineto{\pgfqpoint{3.353367in}{1.792260in}}%
\pgfpathlineto{\pgfqpoint{3.361099in}{1.796801in}}%
\pgfpathlineto{\pgfqpoint{3.368830in}{1.798420in}}%
\pgfpathlineto{\pgfqpoint{3.379139in}{1.804474in}}%
\pgfpathlineto{\pgfqpoint{3.386871in}{1.805971in}}%
\pgfpathlineto{\pgfqpoint{3.397180in}{1.812586in}}%
\pgfpathlineto{\pgfqpoint{3.404911in}{1.814173in}}%
\pgfpathlineto{\pgfqpoint{3.415220in}{1.819312in}}%
\pgfpathlineto{\pgfqpoint{3.422952in}{1.820408in}}%
\pgfpathlineto{\pgfqpoint{3.428106in}{1.822182in}}%
\pgfpathlineto{\pgfqpoint{3.433261in}{1.823707in}}%
\pgfpathlineto{\pgfqpoint{3.443570in}{1.825122in}}%
\pgfpathlineto{\pgfqpoint{3.448724in}{1.826720in}}%
\pgfpathlineto{\pgfqpoint{3.479651in}{1.829627in}}%
\pgfpathlineto{\pgfqpoint{3.482228in}{1.830257in}}%
\pgfpathlineto{\pgfqpoint{3.487382in}{1.833254in}}%
\pgfpathlineto{\pgfqpoint{3.495114in}{1.835079in}}%
\pgfpathlineto{\pgfqpoint{3.505423in}{1.843526in}}%
\pgfpathlineto{\pgfqpoint{3.513154in}{1.845774in}}%
\pgfpathlineto{\pgfqpoint{3.523463in}{1.856025in}}%
\pgfpathlineto{\pgfqpoint{3.531195in}{1.858804in}}%
\pgfpathlineto{\pgfqpoint{3.541504in}{1.869885in}}%
\pgfpathlineto{\pgfqpoint{3.549235in}{1.872699in}}%
\pgfpathlineto{\pgfqpoint{3.559544in}{1.884664in}}%
\pgfpathlineto{\pgfqpoint{3.567276in}{1.887662in}}%
\pgfpathlineto{\pgfqpoint{3.572430in}{1.893076in}}%
\pgfpathlineto{\pgfqpoint{3.577585in}{1.896032in}}%
\pgfpathlineto{\pgfqpoint{3.585316in}{1.898676in}}%
\pgfpathlineto{\pgfqpoint{3.595625in}{1.911225in}}%
\pgfpathlineto{\pgfqpoint{3.603357in}{1.914156in}}%
\pgfpathlineto{\pgfqpoint{3.611088in}{1.922138in}}%
\pgfpathlineto{\pgfqpoint{3.613666in}{1.924433in}}%
\pgfpathlineto{\pgfqpoint{3.621397in}{1.926673in}}%
\pgfpathlineto{\pgfqpoint{3.626552in}{1.931922in}}%
\pgfpathlineto{\pgfqpoint{3.631706in}{1.938739in}}%
\pgfpathlineto{\pgfqpoint{3.639438in}{1.942404in}}%
\pgfpathlineto{\pgfqpoint{3.644592in}{1.949523in}}%
\pgfpathlineto{\pgfqpoint{3.649747in}{1.952934in}}%
\pgfpathlineto{\pgfqpoint{3.657478in}{1.956386in}}%
\pgfpathlineto{\pgfqpoint{3.662633in}{1.962703in}}%
\pgfpathlineto{\pgfqpoint{3.667787in}{1.965671in}}%
\pgfpathlineto{\pgfqpoint{3.675519in}{1.968093in}}%
\pgfpathlineto{\pgfqpoint{3.680673in}{1.972577in}}%
\pgfpathlineto{\pgfqpoint{3.685828in}{1.977910in}}%
\pgfpathlineto{\pgfqpoint{3.693559in}{1.980286in}}%
\pgfpathlineto{\pgfqpoint{3.703868in}{1.989496in}}%
\pgfpathlineto{\pgfqpoint{3.714177in}{1.991939in}}%
\pgfpathlineto{\pgfqpoint{3.721909in}{2.000056in}}%
\pgfpathlineto{\pgfqpoint{3.729640in}{2.002728in}}%
\pgfpathlineto{\pgfqpoint{3.739949in}{2.013086in}}%
\pgfpathlineto{\pgfqpoint{3.747681in}{2.015678in}}%
\pgfpathlineto{\pgfqpoint{3.757990in}{2.026615in}}%
\pgfpathlineto{\pgfqpoint{3.765721in}{2.029161in}}%
\pgfpathlineto{\pgfqpoint{3.776030in}{2.039926in}}%
\pgfpathlineto{\pgfqpoint{3.786339in}{2.042789in}}%
\pgfpathlineto{\pgfqpoint{3.794070in}{2.051498in}}%
\pgfpathlineto{\pgfqpoint{3.801802in}{2.054533in}}%
\pgfpathlineto{\pgfqpoint{3.812111in}{2.066582in}}%
\pgfpathlineto{\pgfqpoint{3.819843in}{2.069735in}}%
\pgfpathlineto{\pgfqpoint{3.827574in}{2.077884in}}%
\pgfpathlineto{\pgfqpoint{3.830151in}{2.080133in}}%
\pgfpathlineto{\pgfqpoint{3.837883in}{2.082620in}}%
\pgfpathlineto{\pgfqpoint{3.848192in}{2.090630in}}%
\pgfpathlineto{\pgfqpoint{3.855924in}{2.093010in}}%
\pgfpathlineto{\pgfqpoint{3.866232in}{2.102112in}}%
\pgfpathlineto{\pgfqpoint{3.873964in}{2.104395in}}%
\pgfpathlineto{\pgfqpoint{3.879118in}{2.108528in}}%
\pgfpathlineto{\pgfqpoint{3.884273in}{2.112252in}}%
\pgfpathlineto{\pgfqpoint{3.892005in}{2.114447in}}%
\pgfpathlineto{\pgfqpoint{3.899736in}{2.120020in}}%
\pgfpathlineto{\pgfqpoint{3.910045in}{2.122087in}}%
\pgfpathlineto{\pgfqpoint{3.920354in}{2.130723in}}%
\pgfpathlineto{\pgfqpoint{3.928086in}{2.132754in}}%
\pgfpathlineto{\pgfqpoint{3.938394in}{2.140376in}}%
\pgfpathlineto{\pgfqpoint{3.946126in}{2.142186in}}%
\pgfpathlineto{\pgfqpoint{3.953858in}{2.146995in}}%
\pgfpathlineto{\pgfqpoint{3.956435in}{2.148184in}}%
\pgfpathlineto{\pgfqpoint{3.964167in}{2.149268in}}%
\pgfpathlineto{\pgfqpoint{3.974475in}{2.153286in}}%
\pgfpathlineto{\pgfqpoint{3.982207in}{2.154429in}}%
\pgfpathlineto{\pgfqpoint{3.992516in}{2.158796in}}%
\pgfpathlineto{\pgfqpoint{4.000247in}{2.159981in}}%
\pgfpathlineto{\pgfqpoint{4.007979in}{2.163958in}}%
\pgfpathlineto{\pgfqpoint{4.010556in}{2.165443in}}%
\pgfpathlineto{\pgfqpoint{4.018288in}{2.166873in}}%
\pgfpathlineto{\pgfqpoint{4.028597in}{2.172459in}}%
\pgfpathlineto{\pgfqpoint{4.038906in}{2.173619in}}%
\pgfpathlineto{\pgfqpoint{4.046637in}{2.177261in}}%
\pgfpathlineto{\pgfqpoint{4.054369in}{2.178330in}}%
\pgfpathlineto{\pgfqpoint{4.064678in}{2.182422in}}%
\pgfpathlineto{\pgfqpoint{4.072409in}{2.183241in}}%
\pgfpathlineto{\pgfqpoint{4.082718in}{2.187121in}}%
\pgfpathlineto{\pgfqpoint{4.093027in}{2.188618in}}%
\pgfpathlineto{\pgfqpoint{4.100759in}{2.191422in}}%
\pgfpathlineto{\pgfqpoint{4.108490in}{2.192446in}}%
\pgfpathlineto{\pgfqpoint{4.116222in}{2.194981in}}%
\pgfpathlineto{\pgfqpoint{4.118799in}{2.195753in}}%
\pgfpathlineto{\pgfqpoint{4.129108in}{2.196805in}}%
\pgfpathlineto{\pgfqpoint{4.134263in}{2.198054in}}%
\pgfpathlineto{\pgfqpoint{4.147149in}{2.199244in}}%
\pgfpathlineto{\pgfqpoint{4.170343in}{2.203350in}}%
\pgfpathlineto{\pgfqpoint{4.172921in}{2.204016in}}%
\pgfpathlineto{\pgfqpoint{4.183230in}{2.205294in}}%
\pgfpathlineto{\pgfqpoint{4.188384in}{2.205975in}}%
\pgfpathlineto{\pgfqpoint{4.245083in}{2.208071in}}%
\pgfpathlineto{\pgfqpoint{4.260546in}{2.207880in}}%
\pgfpathlineto{\pgfqpoint{4.281164in}{2.206115in}}%
\pgfpathlineto{\pgfqpoint{4.291472in}{2.205406in}}%
\pgfpathlineto{\pgfqpoint{4.299204in}{2.204415in}}%
\pgfpathlineto{\pgfqpoint{4.312090in}{2.203866in}}%
\pgfpathlineto{\pgfqpoint{4.317245in}{2.203206in}}%
\pgfpathlineto{\pgfqpoint{4.332708in}{2.202238in}}%
\pgfpathlineto{\pgfqpoint{4.335285in}{2.201844in}}%
\pgfpathlineto{\pgfqpoint{4.345594in}{2.201009in}}%
\pgfpathlineto{\pgfqpoint{4.353326in}{2.199665in}}%
\pgfpathlineto{\pgfqpoint{4.368789in}{2.198263in}}%
\pgfpathlineto{\pgfqpoint{4.371366in}{2.198028in}}%
\pgfpathlineto{\pgfqpoint{4.399715in}{2.198265in}}%
\pgfpathlineto{\pgfqpoint{4.420333in}{2.198571in}}%
\pgfpathlineto{\pgfqpoint{4.425488in}{2.199567in}}%
\pgfpathlineto{\pgfqpoint{4.435796in}{2.200653in}}%
\pgfpathlineto{\pgfqpoint{4.443528in}{2.202470in}}%
\pgfpathlineto{\pgfqpoint{4.453837in}{2.203916in}}%
\pgfpathlineto{\pgfqpoint{4.461569in}{2.205983in}}%
\pgfpathlineto{\pgfqpoint{4.471877in}{2.207112in}}%
\pgfpathlineto{\pgfqpoint{4.479609in}{2.208666in}}%
\pgfpathlineto{\pgfqpoint{4.489918in}{2.209767in}}%
\pgfpathlineto{\pgfqpoint{4.497649in}{2.211751in}}%
\pgfpathlineto{\pgfqpoint{4.507958in}{2.213045in}}%
\pgfpathlineto{\pgfqpoint{4.510536in}{2.213642in}}%
\pgfpathlineto{\pgfqpoint{4.549194in}{2.218805in}}%
\pgfpathlineto{\pgfqpoint{4.559503in}{2.219716in}}%
\pgfpathlineto{\pgfqpoint{4.585275in}{2.219751in}}%
\pgfpathlineto{\pgfqpoint{4.618778in}{2.219838in}}%
\pgfpathlineto{\pgfqpoint{4.623933in}{2.219046in}}%
\pgfpathlineto{\pgfqpoint{4.634242in}{2.218248in}}%
\pgfpathlineto{\pgfqpoint{4.641973in}{2.216932in}}%
\pgfpathlineto{\pgfqpoint{4.654859in}{2.215929in}}%
\pgfpathlineto{\pgfqpoint{4.660014in}{2.214991in}}%
\pgfpathlineto{\pgfqpoint{4.678054in}{2.214024in}}%
\pgfpathlineto{\pgfqpoint{4.696095in}{2.214471in}}%
\pgfpathlineto{\pgfqpoint{4.711558in}{2.215300in}}%
\pgfpathlineto{\pgfqpoint{4.714135in}{2.215597in}}%
\pgfpathlineto{\pgfqpoint{4.727021in}{2.216453in}}%
\pgfpathlineto{\pgfqpoint{4.732176in}{2.217367in}}%
\pgfpathlineto{\pgfqpoint{4.745062in}{2.218798in}}%
\pgfpathlineto{\pgfqpoint{4.750216in}{2.219952in}}%
\pgfpathlineto{\pgfqpoint{4.760525in}{2.221038in}}%
\pgfpathlineto{\pgfqpoint{4.768257in}{2.222925in}}%
\pgfpathlineto{\pgfqpoint{4.778566in}{2.224272in}}%
\pgfpathlineto{\pgfqpoint{4.786297in}{2.226311in}}%
\pgfpathlineto{\pgfqpoint{4.796606in}{2.227865in}}%
\pgfpathlineto{\pgfqpoint{4.804338in}{2.230584in}}%
\pgfpathlineto{\pgfqpoint{4.812069in}{2.231545in}}%
\pgfpathlineto{\pgfqpoint{4.819801in}{2.234326in}}%
\pgfpathlineto{\pgfqpoint{4.830110in}{2.235378in}}%
\pgfpathlineto{\pgfqpoint{4.840419in}{2.239544in}}%
\pgfpathlineto{\pgfqpoint{4.848150in}{2.240576in}}%
\pgfpathlineto{\pgfqpoint{4.858459in}{2.244706in}}%
\pgfpathlineto{\pgfqpoint{4.866191in}{2.245723in}}%
\pgfpathlineto{\pgfqpoint{4.876500in}{2.250301in}}%
\pgfpathlineto{\pgfqpoint{4.884231in}{2.251524in}}%
\pgfpathlineto{\pgfqpoint{4.894540in}{2.256162in}}%
\pgfpathlineto{\pgfqpoint{4.904849in}{2.258178in}}%
\pgfpathlineto{\pgfqpoint{4.912581in}{2.261179in}}%
\pgfpathlineto{\pgfqpoint{4.920312in}{2.262242in}}%
\pgfpathlineto{\pgfqpoint{4.930621in}{2.266317in}}%
\pgfpathlineto{\pgfqpoint{4.938353in}{2.267366in}}%
\pgfpathlineto{\pgfqpoint{4.948662in}{2.271846in}}%
\pgfpathlineto{\pgfqpoint{4.956393in}{2.272944in}}%
\pgfpathlineto{\pgfqpoint{4.966702in}{2.276512in}}%
\pgfpathlineto{\pgfqpoint{4.974434in}{2.277406in}}%
\pgfpathlineto{\pgfqpoint{4.984743in}{2.281842in}}%
\pgfpathlineto{\pgfqpoint{4.995051in}{2.282874in}}%
\pgfpathlineto{\pgfqpoint{5.002783in}{2.285949in}}%
\pgfpathlineto{\pgfqpoint{5.010515in}{2.287095in}}%
\pgfpathlineto{\pgfqpoint{5.020824in}{2.291678in}}%
\pgfpathlineto{\pgfqpoint{5.028555in}{2.292570in}}%
\pgfpathlineto{\pgfqpoint{5.038864in}{2.296518in}}%
\pgfpathlineto{\pgfqpoint{5.046596in}{2.297696in}}%
\pgfpathlineto{\pgfqpoint{5.056905in}{2.302424in}}%
\pgfpathlineto{\pgfqpoint{5.064636in}{2.303268in}}%
\pgfpathlineto{\pgfqpoint{5.072368in}{2.306882in}}%
\pgfpathlineto{\pgfqpoint{5.074945in}{2.308352in}}%
\pgfpathlineto{\pgfqpoint{5.085254in}{2.309822in}}%
\pgfpathlineto{\pgfqpoint{5.092986in}{2.314208in}}%
\pgfpathlineto{\pgfqpoint{5.100717in}{2.315850in}}%
\pgfpathlineto{\pgfqpoint{5.111026in}{2.323092in}}%
\pgfpathlineto{\pgfqpoint{5.118758in}{2.324963in}}%
\pgfpathlineto{\pgfqpoint{5.129067in}{2.332251in}}%
\pgfpathlineto{\pgfqpoint{5.136798in}{2.333942in}}%
\pgfpathlineto{\pgfqpoint{5.147107in}{2.340088in}}%
\pgfpathlineto{\pgfqpoint{5.154839in}{2.341624in}}%
\pgfpathlineto{\pgfqpoint{5.165148in}{2.347682in}}%
\pgfpathlineto{\pgfqpoint{5.172879in}{2.349202in}}%
\pgfpathlineto{\pgfqpoint{5.183188in}{2.355553in}}%
\pgfpathlineto{\pgfqpoint{5.190920in}{2.357199in}}%
\pgfpathlineto{\pgfqpoint{5.201228in}{2.363619in}}%
\pgfpathlineto{\pgfqpoint{5.208960in}{2.365191in}}%
\pgfpathlineto{\pgfqpoint{5.219269in}{2.371560in}}%
\pgfpathlineto{\pgfqpoint{5.227001in}{2.373214in}}%
\pgfpathlineto{\pgfqpoint{5.237309in}{2.379588in}}%
\pgfpathlineto{\pgfqpoint{5.247618in}{2.381196in}}%
\pgfpathlineto{\pgfqpoint{5.255350in}{2.385498in}}%
\pgfpathlineto{\pgfqpoint{5.263082in}{2.386883in}}%
\pgfpathlineto{\pgfqpoint{5.273390in}{2.391654in}}%
\pgfpathlineto{\pgfqpoint{5.281122in}{2.392899in}}%
\pgfpathlineto{\pgfqpoint{5.291431in}{2.398402in}}%
\pgfpathlineto{\pgfqpoint{5.299163in}{2.399557in}}%
\pgfpathlineto{\pgfqpoint{5.309471in}{2.404243in}}%
\pgfpathlineto{\pgfqpoint{5.319780in}{2.406120in}}%
\pgfpathlineto{\pgfqpoint{5.327512in}{2.408572in}}%
\pgfpathlineto{\pgfqpoint{5.337821in}{2.410005in}}%
\pgfpathlineto{\pgfqpoint{5.345552in}{2.412049in}}%
\pgfpathlineto{\pgfqpoint{5.355861in}{2.413401in}}%
\pgfpathlineto{\pgfqpoint{5.363593in}{2.415339in}}%
\pgfpathlineto{\pgfqpoint{5.379056in}{2.417327in}}%
\pgfpathlineto{\pgfqpoint{5.381633in}{2.417717in}}%
\pgfpathlineto{\pgfqpoint{5.394519in}{2.418757in}}%
\pgfpathlineto{\pgfqpoint{5.399674in}{2.419614in}}%
\pgfpathlineto{\pgfqpoint{5.409983in}{2.420906in}}%
\pgfpathlineto{\pgfqpoint{5.417714in}{2.423375in}}%
\pgfpathlineto{\pgfqpoint{5.425446in}{2.424164in}}%
\pgfpathlineto{\pgfqpoint{5.435755in}{2.427520in}}%
\pgfpathlineto{\pgfqpoint{5.446064in}{2.429041in}}%
\pgfpathlineto{\pgfqpoint{5.448641in}{2.429830in}}%
\pgfpathlineto{\pgfqpoint{5.469259in}{2.433838in}}%
\pgfpathlineto{\pgfqpoint{5.471836in}{2.434614in}}%
\pgfpathlineto{\pgfqpoint{5.482145in}{2.436063in}}%
\pgfpathlineto{\pgfqpoint{5.489876in}{2.439230in}}%
\pgfpathlineto{\pgfqpoint{5.497608in}{2.440479in}}%
\pgfpathlineto{\pgfqpoint{5.507917in}{2.444725in}}%
\pgfpathlineto{\pgfqpoint{5.515648in}{2.445839in}}%
\pgfpathlineto{\pgfqpoint{5.525957in}{2.450409in}}%
\pgfpathlineto{\pgfqpoint{5.536266in}{2.451550in}}%
\pgfpathlineto{\pgfqpoint{5.543998in}{2.454818in}}%
\pgfpathlineto{\pgfqpoint{5.554307in}{2.455876in}}%
\pgfpathlineto{\pgfqpoint{5.562038in}{2.459028in}}%
\pgfpathlineto{\pgfqpoint{5.569770in}{2.460012in}}%
\pgfpathlineto{\pgfqpoint{5.580079in}{2.463917in}}%
\pgfpathlineto{\pgfqpoint{5.590388in}{2.464877in}}%
\pgfpathlineto{\pgfqpoint{5.598119in}{2.467992in}}%
\pgfpathlineto{\pgfqpoint{5.608428in}{2.469869in}}%
\pgfpathlineto{\pgfqpoint{5.616160in}{2.472600in}}%
\pgfpathlineto{\pgfqpoint{5.626469in}{2.474157in}}%
\pgfpathlineto{\pgfqpoint{5.634200in}{2.476402in}}%
\pgfpathlineto{\pgfqpoint{5.644509in}{2.477959in}}%
\pgfpathlineto{\pgfqpoint{5.652241in}{2.480778in}}%
\pgfpathlineto{\pgfqpoint{5.659972in}{2.481907in}}%
\pgfpathlineto{\pgfqpoint{5.670281in}{2.487019in}}%
\pgfpathlineto{\pgfqpoint{5.680590in}{2.488372in}}%
\pgfpathlineto{\pgfqpoint{5.688322in}{2.493194in}}%
\pgfpathlineto{\pgfqpoint{5.696053in}{2.494788in}}%
\pgfpathlineto{\pgfqpoint{5.703785in}{2.499954in}}%
\pgfpathlineto{\pgfqpoint{5.706362in}{2.501708in}}%
\pgfpathlineto{\pgfqpoint{5.714094in}{2.503420in}}%
\pgfpathlineto{\pgfqpoint{5.724403in}{2.510518in}}%
\pgfpathlineto{\pgfqpoint{5.732134in}{2.512401in}}%
\pgfpathlineto{\pgfqpoint{5.742443in}{2.519694in}}%
\pgfpathlineto{\pgfqpoint{5.750175in}{2.521656in}}%
\pgfpathlineto{\pgfqpoint{5.760484in}{2.529084in}}%
\pgfpathlineto{\pgfqpoint{5.768215in}{2.530794in}}%
\pgfpathlineto{\pgfqpoint{5.778524in}{2.537759in}}%
\pgfpathlineto{\pgfqpoint{5.786256in}{2.539453in}}%
\pgfpathlineto{\pgfqpoint{5.796565in}{2.545956in}}%
\pgfpathlineto{\pgfqpoint{5.804296in}{2.547535in}}%
\pgfpathlineto{\pgfqpoint{5.812028in}{2.552177in}}%
\pgfpathlineto{\pgfqpoint{5.822337in}{2.553777in}}%
\pgfpathlineto{\pgfqpoint{5.832646in}{2.560202in}}%
\pgfpathlineto{\pgfqpoint{5.840377in}{2.562066in}}%
\pgfpathlineto{\pgfqpoint{5.850686in}{2.569848in}}%
\pgfpathlineto{\pgfqpoint{5.858418in}{2.571714in}}%
\pgfpathlineto{\pgfqpoint{5.868727in}{2.580218in}}%
\pgfpathlineto{\pgfqpoint{5.876458in}{2.582351in}}%
\pgfpathlineto{\pgfqpoint{5.886767in}{2.590242in}}%
\pgfpathlineto{\pgfqpoint{5.894499in}{2.592245in}}%
\pgfpathlineto{\pgfqpoint{5.904807in}{2.599909in}}%
\pgfpathlineto{\pgfqpoint{5.912539in}{2.602035in}}%
\pgfpathlineto{\pgfqpoint{5.922848in}{2.610561in}}%
\pgfpathlineto{\pgfqpoint{5.933157in}{2.612969in}}%
\pgfpathlineto{\pgfqpoint{5.940888in}{2.620847in}}%
\pgfpathlineto{\pgfqpoint{5.948620in}{2.623541in}}%
\pgfpathlineto{\pgfqpoint{5.958929in}{2.634051in}}%
\pgfpathlineto{\pgfqpoint{5.966661in}{2.636788in}}%
\pgfpathlineto{\pgfqpoint{5.976969in}{2.648895in}}%
\pgfpathlineto{\pgfqpoint{5.984701in}{2.652116in}}%
\pgfpathlineto{\pgfqpoint{5.995010in}{2.664644in}}%
\pgfpathlineto{\pgfqpoint{6.002742in}{2.667717in}}%
\pgfpathlineto{\pgfqpoint{6.013050in}{2.678608in}}%
\pgfpathlineto{\pgfqpoint{6.020782in}{2.681362in}}%
\pgfpathlineto{\pgfqpoint{6.025936in}{2.684097in}}%
\pgfpathlineto{\pgfqpoint{6.031091in}{2.689363in}}%
\pgfpathlineto{\pgfqpoint{6.038823in}{2.692126in}}%
\pgfpathlineto{\pgfqpoint{6.049131in}{2.703213in}}%
\pgfpathlineto{\pgfqpoint{6.056863in}{2.706017in}}%
\pgfpathlineto{\pgfqpoint{6.067172in}{2.717132in}}%
\pgfpathlineto{\pgfqpoint{6.074904in}{2.719727in}}%
\pgfpathlineto{\pgfqpoint{6.085212in}{2.726683in}}%
\pgfpathlineto{\pgfqpoint{6.092944in}{2.728524in}}%
\pgfpathlineto{\pgfqpoint{6.100676in}{2.734980in}}%
\pgfpathlineto{\pgfqpoint{6.103253in}{2.737292in}}%
\pgfpathlineto{\pgfqpoint{6.110984in}{2.739576in}}%
\pgfpathlineto{\pgfqpoint{6.121293in}{2.748245in}}%
\pgfpathlineto{\pgfqpoint{6.129025in}{2.750475in}}%
\pgfpathlineto{\pgfqpoint{6.139334in}{2.758874in}}%
\pgfpathlineto{\pgfqpoint{6.147065in}{2.760839in}}%
\pgfpathlineto{\pgfqpoint{6.157374in}{2.768442in}}%
\pgfpathlineto{\pgfqpoint{6.165106in}{2.770297in}}%
\pgfpathlineto{\pgfqpoint{6.175415in}{2.777985in}}%
\pgfpathlineto{\pgfqpoint{6.185724in}{2.779692in}}%
\pgfpathlineto{\pgfqpoint{6.193455in}{2.785453in}}%
\pgfpathlineto{\pgfqpoint{6.201187in}{2.787781in}}%
\pgfpathlineto{\pgfqpoint{6.211496in}{2.797457in}}%
\pgfpathlineto{\pgfqpoint{6.219227in}{2.800078in}}%
\pgfpathlineto{\pgfqpoint{6.226959in}{2.807319in}}%
\pgfpathlineto{\pgfqpoint{6.229536in}{2.809651in}}%
\pgfpathlineto{\pgfqpoint{6.237268in}{2.811971in}}%
\pgfpathlineto{\pgfqpoint{6.247577in}{2.820921in}}%
\pgfpathlineto{\pgfqpoint{6.255308in}{2.823354in}}%
\pgfpathlineto{\pgfqpoint{6.265617in}{2.834041in}}%
\pgfpathlineto{\pgfqpoint{6.273349in}{2.836740in}}%
\pgfpathlineto{\pgfqpoint{6.283658in}{2.847574in}}%
\pgfpathlineto{\pgfqpoint{6.291389in}{2.850386in}}%
\pgfpathlineto{\pgfqpoint{6.301698in}{2.861655in}}%
\pgfpathlineto{\pgfqpoint{6.309430in}{2.864645in}}%
\pgfpathlineto{\pgfqpoint{6.319739in}{2.881428in}}%
\pgfpathlineto{\pgfqpoint{6.327470in}{2.885189in}}%
\pgfpathlineto{\pgfqpoint{6.337779in}{2.900127in}}%
\pgfpathlineto{\pgfqpoint{6.345511in}{2.903735in}}%
\pgfpathlineto{\pgfqpoint{6.355820in}{2.917508in}}%
\pgfpathlineto{\pgfqpoint{6.363551in}{2.920847in}}%
\pgfpathlineto{\pgfqpoint{6.373860in}{2.934293in}}%
\pgfpathlineto{\pgfqpoint{6.381592in}{2.937850in}}%
\pgfpathlineto{\pgfqpoint{6.386746in}{2.945269in}}%
\pgfpathlineto{\pgfqpoint{6.391901in}{2.948872in}}%
\pgfpathlineto{\pgfqpoint{6.399632in}{2.952702in}}%
\pgfpathlineto{\pgfqpoint{6.407364in}{2.965501in}}%
\pgfpathlineto{\pgfqpoint{6.409941in}{2.969942in}}%
\pgfpathlineto{\pgfqpoint{6.417673in}{2.974175in}}%
\pgfpathlineto{\pgfqpoint{6.427982in}{2.990750in}}%
\pgfpathlineto{\pgfqpoint{6.435713in}{2.994889in}}%
\pgfpathlineto{\pgfqpoint{6.446022in}{3.010745in}}%
\pgfpathlineto{\pgfqpoint{6.453754in}{3.014708in}}%
\pgfpathlineto{\pgfqpoint{6.464063in}{3.029576in}}%
\pgfpathlineto{\pgfqpoint{6.474371in}{3.033182in}}%
\pgfpathlineto{\pgfqpoint{6.482103in}{3.043990in}}%
\pgfpathlineto{\pgfqpoint{6.482103in}{3.043990in}}%
\pgfusepath{stroke}%
\end{pgfscope}%
\begin{pgfscope}%
\pgfpathrectangle{\pgfqpoint{0.563921in}{0.521603in}}{\pgfqpoint{6.200000in}{2.642500in}}%
\pgfusepath{clip}%
\pgfsetroundcap%
\pgfsetroundjoin%
\pgfsetlinewidth{1.505625pt}%
\definecolor{currentstroke}{rgb}{1.000000,0.498039,0.054902}%
\pgfsetstrokecolor{currentstroke}%
\pgfsetdash{}{0pt}%
\pgfpathmoveto{\pgfqpoint{0.845739in}{0.641717in}}%
\pgfpathlineto{\pgfqpoint{0.848317in}{0.642611in}}%
\pgfpathlineto{\pgfqpoint{0.850894in}{0.656202in}}%
\pgfpathlineto{\pgfqpoint{0.853471in}{0.654748in}}%
\pgfpathlineto{\pgfqpoint{0.861203in}{0.653373in}}%
\pgfpathlineto{\pgfqpoint{0.863780in}{0.653781in}}%
\pgfpathlineto{\pgfqpoint{0.866357in}{0.656485in}}%
\pgfpathlineto{\pgfqpoint{0.868934in}{0.666224in}}%
\pgfpathlineto{\pgfqpoint{0.871512in}{0.671444in}}%
\pgfpathlineto{\pgfqpoint{0.881820in}{0.678235in}}%
\pgfpathlineto{\pgfqpoint{0.886975in}{0.691486in}}%
\pgfpathlineto{\pgfqpoint{0.897284in}{0.689184in}}%
\pgfpathlineto{\pgfqpoint{0.899861in}{0.687584in}}%
\pgfpathlineto{\pgfqpoint{0.902438in}{0.687569in}}%
\pgfpathlineto{\pgfqpoint{0.907593in}{0.686010in}}%
\pgfpathlineto{\pgfqpoint{0.917901in}{0.684751in}}%
\pgfpathlineto{\pgfqpoint{0.920479in}{0.685833in}}%
\pgfpathlineto{\pgfqpoint{0.923056in}{0.688687in}}%
\pgfpathlineto{\pgfqpoint{0.925633in}{0.696231in}}%
\pgfpathlineto{\pgfqpoint{0.933365in}{0.699730in}}%
\pgfpathlineto{\pgfqpoint{0.935942in}{0.703718in}}%
\pgfpathlineto{\pgfqpoint{0.938519in}{0.705206in}}%
\pgfpathlineto{\pgfqpoint{0.941096in}{0.708489in}}%
\pgfpathlineto{\pgfqpoint{0.943674in}{0.709648in}}%
\pgfpathlineto{\pgfqpoint{0.956560in}{0.712167in}}%
\pgfpathlineto{\pgfqpoint{0.961714in}{0.717672in}}%
\pgfpathlineto{\pgfqpoint{0.972023in}{0.719708in}}%
\pgfpathlineto{\pgfqpoint{0.987486in}{0.728898in}}%
\pgfpathlineto{\pgfqpoint{0.990063in}{0.731394in}}%
\pgfpathlineto{\pgfqpoint{0.992641in}{0.731887in}}%
\pgfpathlineto{\pgfqpoint{0.995218in}{0.733458in}}%
\pgfpathlineto{\pgfqpoint{0.997795in}{0.733848in}}%
\pgfpathlineto{\pgfqpoint{1.005527in}{0.734120in}}%
\pgfpathlineto{\pgfqpoint{1.010681in}{0.732687in}}%
\pgfpathlineto{\pgfqpoint{1.023567in}{0.733091in}}%
\pgfpathlineto{\pgfqpoint{1.026144in}{0.735010in}}%
\pgfpathlineto{\pgfqpoint{1.033876in}{0.753113in}}%
\pgfpathlineto{\pgfqpoint{1.041608in}{0.759563in}}%
\pgfpathlineto{\pgfqpoint{1.049339in}{0.773502in}}%
\pgfpathlineto{\pgfqpoint{1.051916in}{0.777479in}}%
\pgfpathlineto{\pgfqpoint{1.059648in}{0.783911in}}%
\pgfpathlineto{\pgfqpoint{1.067380in}{0.797977in}}%
\pgfpathlineto{\pgfqpoint{1.069957in}{0.800878in}}%
\pgfpathlineto{\pgfqpoint{1.077689in}{0.803742in}}%
\pgfpathlineto{\pgfqpoint{1.080266in}{0.807217in}}%
\pgfpathlineto{\pgfqpoint{1.082843in}{0.808969in}}%
\pgfpathlineto{\pgfqpoint{1.085420in}{0.811509in}}%
\pgfpathlineto{\pgfqpoint{1.100883in}{0.813612in}}%
\pgfpathlineto{\pgfqpoint{1.103461in}{0.815481in}}%
\pgfpathlineto{\pgfqpoint{1.106038in}{0.816387in}}%
\pgfpathlineto{\pgfqpoint{1.113770in}{0.817757in}}%
\pgfpathlineto{\pgfqpoint{1.118924in}{0.820740in}}%
\pgfpathlineto{\pgfqpoint{1.124078in}{0.822137in}}%
\pgfpathlineto{\pgfqpoint{1.134387in}{0.823216in}}%
\pgfpathlineto{\pgfqpoint{1.139542in}{0.827415in}}%
\pgfpathlineto{\pgfqpoint{1.142119in}{0.830322in}}%
\pgfpathlineto{\pgfqpoint{1.149851in}{0.833073in}}%
\pgfpathlineto{\pgfqpoint{1.157582in}{0.842727in}}%
\pgfpathlineto{\pgfqpoint{1.160159in}{0.844597in}}%
\pgfpathlineto{\pgfqpoint{1.167891in}{0.846340in}}%
\pgfpathlineto{\pgfqpoint{1.173045in}{0.848740in}}%
\pgfpathlineto{\pgfqpoint{1.178200in}{0.850586in}}%
\pgfpathlineto{\pgfqpoint{1.188509in}{0.850427in}}%
\pgfpathlineto{\pgfqpoint{1.193663in}{0.848928in}}%
\pgfpathlineto{\pgfqpoint{1.196240in}{0.847902in}}%
\pgfpathlineto{\pgfqpoint{1.206549in}{0.846248in}}%
\pgfpathlineto{\pgfqpoint{1.214281in}{0.843643in}}%
\pgfpathlineto{\pgfqpoint{1.224590in}{0.842915in}}%
\pgfpathlineto{\pgfqpoint{1.232321in}{0.840266in}}%
\pgfpathlineto{\pgfqpoint{1.240053in}{0.839414in}}%
\pgfpathlineto{\pgfqpoint{1.250362in}{0.835978in}}%
\pgfpathlineto{\pgfqpoint{1.260671in}{0.834472in}}%
\pgfpathlineto{\pgfqpoint{1.268402in}{0.832132in}}%
\pgfpathlineto{\pgfqpoint{1.281288in}{0.830581in}}%
\pgfpathlineto{\pgfqpoint{1.286443in}{0.829422in}}%
\pgfpathlineto{\pgfqpoint{1.296752in}{0.828092in}}%
\pgfpathlineto{\pgfqpoint{1.304483in}{0.827044in}}%
\pgfpathlineto{\pgfqpoint{1.314792in}{0.828009in}}%
\pgfpathlineto{\pgfqpoint{1.322524in}{0.828678in}}%
\pgfpathlineto{\pgfqpoint{1.335410in}{0.828664in}}%
\pgfpathlineto{\pgfqpoint{1.340564in}{0.828039in}}%
\pgfpathlineto{\pgfqpoint{1.353450in}{0.828188in}}%
\pgfpathlineto{\pgfqpoint{1.358605in}{0.826964in}}%
\pgfpathlineto{\pgfqpoint{1.368914in}{0.825694in}}%
\pgfpathlineto{\pgfqpoint{1.374068in}{0.824873in}}%
\pgfpathlineto{\pgfqpoint{1.386954in}{0.824569in}}%
\pgfpathlineto{\pgfqpoint{1.394686in}{0.823297in}}%
\pgfpathlineto{\pgfqpoint{1.407572in}{0.822411in}}%
\pgfpathlineto{\pgfqpoint{1.412726in}{0.821339in}}%
\pgfpathlineto{\pgfqpoint{1.423035in}{0.820266in}}%
\pgfpathlineto{\pgfqpoint{1.430767in}{0.819196in}}%
\pgfpathlineto{\pgfqpoint{1.441076in}{0.818317in}}%
\pgfpathlineto{\pgfqpoint{1.448807in}{0.817120in}}%
\pgfpathlineto{\pgfqpoint{1.461693in}{0.816224in}}%
\pgfpathlineto{\pgfqpoint{1.466848in}{0.815780in}}%
\pgfpathlineto{\pgfqpoint{1.479734in}{0.815426in}}%
\pgfpathlineto{\pgfqpoint{1.484888in}{0.814871in}}%
\pgfpathlineto{\pgfqpoint{1.513237in}{0.814350in}}%
\pgfpathlineto{\pgfqpoint{1.528701in}{0.813704in}}%
\pgfpathlineto{\pgfqpoint{1.539010in}{0.812094in}}%
\pgfpathlineto{\pgfqpoint{1.600863in}{0.809671in}}%
\pgfpathlineto{\pgfqpoint{1.611172in}{0.808059in}}%
\pgfpathlineto{\pgfqpoint{1.629212in}{0.806990in}}%
\pgfpathlineto{\pgfqpoint{1.642098in}{0.805949in}}%
\pgfpathlineto{\pgfqpoint{1.647253in}{0.805189in}}%
\pgfpathlineto{\pgfqpoint{1.673025in}{0.803397in}}%
\pgfpathlineto{\pgfqpoint{1.683333in}{0.802356in}}%
\pgfpathlineto{\pgfqpoint{1.696220in}{0.801352in}}%
\pgfpathlineto{\pgfqpoint{1.701374in}{0.800663in}}%
\pgfpathlineto{\pgfqpoint{1.714260in}{0.799669in}}%
\pgfpathlineto{\pgfqpoint{1.719414in}{0.799043in}}%
\pgfpathlineto{\pgfqpoint{1.747764in}{0.797524in}}%
\pgfpathlineto{\pgfqpoint{1.755495in}{0.796876in}}%
\pgfpathlineto{\pgfqpoint{1.791576in}{0.796254in}}%
\pgfpathlineto{\pgfqpoint{1.801885in}{0.797123in}}%
\pgfpathlineto{\pgfqpoint{1.827657in}{0.801912in}}%
\pgfpathlineto{\pgfqpoint{1.871470in}{0.803088in}}%
\pgfpathlineto{\pgfqpoint{1.879202in}{0.804928in}}%
\pgfpathlineto{\pgfqpoint{1.881779in}{0.805741in}}%
\pgfpathlineto{\pgfqpoint{1.889510in}{0.806607in}}%
\pgfpathlineto{\pgfqpoint{1.899819in}{0.810036in}}%
\pgfpathlineto{\pgfqpoint{1.912705in}{0.811650in}}%
\pgfpathlineto{\pgfqpoint{1.917860in}{0.813183in}}%
\pgfpathlineto{\pgfqpoint{1.928169in}{0.814577in}}%
\pgfpathlineto{\pgfqpoint{1.935900in}{0.816944in}}%
\pgfpathlineto{\pgfqpoint{1.943632in}{0.817897in}}%
\pgfpathlineto{\pgfqpoint{1.953941in}{0.824121in}}%
\pgfpathlineto{\pgfqpoint{1.961672in}{0.826045in}}%
\pgfpathlineto{\pgfqpoint{1.971981in}{0.833330in}}%
\pgfpathlineto{\pgfqpoint{1.979713in}{0.835160in}}%
\pgfpathlineto{\pgfqpoint{1.987445in}{0.840063in}}%
\pgfpathlineto{\pgfqpoint{1.990022in}{0.841932in}}%
\pgfpathlineto{\pgfqpoint{1.997753in}{0.843708in}}%
\pgfpathlineto{\pgfqpoint{2.005485in}{0.850358in}}%
\pgfpathlineto{\pgfqpoint{2.015794in}{0.852468in}}%
\pgfpathlineto{\pgfqpoint{2.020948in}{0.856366in}}%
\pgfpathlineto{\pgfqpoint{2.026103in}{0.859407in}}%
\pgfpathlineto{\pgfqpoint{2.033834in}{0.860712in}}%
\pgfpathlineto{\pgfqpoint{2.044143in}{0.865554in}}%
\pgfpathlineto{\pgfqpoint{2.051875in}{0.866252in}}%
\pgfpathlineto{\pgfqpoint{2.059606in}{0.868710in}}%
\pgfpathlineto{\pgfqpoint{2.062184in}{0.870428in}}%
\pgfpathlineto{\pgfqpoint{2.069915in}{0.871890in}}%
\pgfpathlineto{\pgfqpoint{2.080224in}{0.879006in}}%
\pgfpathlineto{\pgfqpoint{2.087956in}{0.880669in}}%
\pgfpathlineto{\pgfqpoint{2.095687in}{0.886719in}}%
\pgfpathlineto{\pgfqpoint{2.098265in}{0.889308in}}%
\pgfpathlineto{\pgfqpoint{2.105996in}{0.891765in}}%
\pgfpathlineto{\pgfqpoint{2.116305in}{0.901319in}}%
\pgfpathlineto{\pgfqpoint{2.124037in}{0.903405in}}%
\pgfpathlineto{\pgfqpoint{2.131768in}{0.912450in}}%
\pgfpathlineto{\pgfqpoint{2.134346in}{0.915863in}}%
\pgfpathlineto{\pgfqpoint{2.142077in}{0.919721in}}%
\pgfpathlineto{\pgfqpoint{2.149809in}{0.931238in}}%
\pgfpathlineto{\pgfqpoint{2.152386in}{0.935143in}}%
\pgfpathlineto{\pgfqpoint{2.162695in}{0.939395in}}%
\pgfpathlineto{\pgfqpoint{2.170427in}{0.951129in}}%
\pgfpathlineto{\pgfqpoint{2.178158in}{0.955126in}}%
\pgfpathlineto{\pgfqpoint{2.188467in}{0.969904in}}%
\pgfpathlineto{\pgfqpoint{2.196199in}{0.974301in}}%
\pgfpathlineto{\pgfqpoint{2.201353in}{0.980462in}}%
\pgfpathlineto{\pgfqpoint{2.203930in}{0.983273in}}%
\pgfpathlineto{\pgfqpoint{2.206508in}{0.985217in}}%
\pgfpathlineto{\pgfqpoint{2.214239in}{0.987424in}}%
\pgfpathlineto{\pgfqpoint{2.219394in}{0.992263in}}%
\pgfpathlineto{\pgfqpoint{2.224548in}{0.995917in}}%
\pgfpathlineto{\pgfqpoint{2.232280in}{0.997359in}}%
\pgfpathlineto{\pgfqpoint{2.237434in}{1.001117in}}%
\pgfpathlineto{\pgfqpoint{2.242589in}{1.005654in}}%
\pgfpathlineto{\pgfqpoint{2.250320in}{1.008103in}}%
\pgfpathlineto{\pgfqpoint{2.255475in}{1.012320in}}%
\pgfpathlineto{\pgfqpoint{2.260629in}{1.014965in}}%
\pgfpathlineto{\pgfqpoint{2.268361in}{1.017829in}}%
\pgfpathlineto{\pgfqpoint{2.273515in}{1.023354in}}%
\pgfpathlineto{\pgfqpoint{2.278670in}{1.029154in}}%
\pgfpathlineto{\pgfqpoint{2.286401in}{1.032049in}}%
\pgfpathlineto{\pgfqpoint{2.291556in}{1.037464in}}%
\pgfpathlineto{\pgfqpoint{2.296710in}{1.040502in}}%
\pgfpathlineto{\pgfqpoint{2.304442in}{1.042177in}}%
\pgfpathlineto{\pgfqpoint{2.312173in}{1.047451in}}%
\pgfpathlineto{\pgfqpoint{2.314751in}{1.049196in}}%
\pgfpathlineto{\pgfqpoint{2.322482in}{1.050833in}}%
\pgfpathlineto{\pgfqpoint{2.332791in}{1.057165in}}%
\pgfpathlineto{\pgfqpoint{2.340523in}{1.058921in}}%
\pgfpathlineto{\pgfqpoint{2.350832in}{1.065390in}}%
\pgfpathlineto{\pgfqpoint{2.358563in}{1.066900in}}%
\pgfpathlineto{\pgfqpoint{2.366295in}{1.071434in}}%
\pgfpathlineto{\pgfqpoint{2.368872in}{1.072772in}}%
\pgfpathlineto{\pgfqpoint{2.376604in}{1.073889in}}%
\pgfpathlineto{\pgfqpoint{2.386912in}{1.077723in}}%
\pgfpathlineto{\pgfqpoint{2.399799in}{1.079676in}}%
\pgfpathlineto{\pgfqpoint{2.404953in}{1.080751in}}%
\pgfpathlineto{\pgfqpoint{2.415262in}{1.081343in}}%
\pgfpathlineto{\pgfqpoint{2.422993in}{1.083654in}}%
\pgfpathlineto{\pgfqpoint{2.430725in}{1.084448in}}%
\pgfpathlineto{\pgfqpoint{2.441034in}{1.088715in}}%
\pgfpathlineto{\pgfqpoint{2.448766in}{1.089859in}}%
\pgfpathlineto{\pgfqpoint{2.459074in}{1.095924in}}%
\pgfpathlineto{\pgfqpoint{2.466806in}{1.097168in}}%
\pgfpathlineto{\pgfqpoint{2.477115in}{1.101643in}}%
\pgfpathlineto{\pgfqpoint{2.484847in}{1.102605in}}%
\pgfpathlineto{\pgfqpoint{2.490001in}{1.104436in}}%
\pgfpathlineto{\pgfqpoint{2.495155in}{1.105832in}}%
\pgfpathlineto{\pgfqpoint{2.508041in}{1.106971in}}%
\pgfpathlineto{\pgfqpoint{2.513196in}{1.108611in}}%
\pgfpathlineto{\pgfqpoint{2.523505in}{1.110460in}}%
\pgfpathlineto{\pgfqpoint{2.526082in}{1.111499in}}%
\pgfpathlineto{\pgfqpoint{2.531236in}{1.115445in}}%
\pgfpathlineto{\pgfqpoint{2.538968in}{1.117383in}}%
\pgfpathlineto{\pgfqpoint{2.549277in}{1.125765in}}%
\pgfpathlineto{\pgfqpoint{2.557009in}{1.128190in}}%
\pgfpathlineto{\pgfqpoint{2.564740in}{1.135348in}}%
\pgfpathlineto{\pgfqpoint{2.567317in}{1.137475in}}%
\pgfpathlineto{\pgfqpoint{2.575049in}{1.139514in}}%
\pgfpathlineto{\pgfqpoint{2.585358in}{1.147149in}}%
\pgfpathlineto{\pgfqpoint{2.593089in}{1.148926in}}%
\pgfpathlineto{\pgfqpoint{2.603398in}{1.156181in}}%
\pgfpathlineto{\pgfqpoint{2.611130in}{1.158065in}}%
\pgfpathlineto{\pgfqpoint{2.621439in}{1.166129in}}%
\pgfpathlineto{\pgfqpoint{2.629170in}{1.168444in}}%
\pgfpathlineto{\pgfqpoint{2.634325in}{1.173366in}}%
\pgfpathlineto{\pgfqpoint{2.639479in}{1.175899in}}%
\pgfpathlineto{\pgfqpoint{2.647211in}{1.178264in}}%
\pgfpathlineto{\pgfqpoint{2.657520in}{1.187019in}}%
\pgfpathlineto{\pgfqpoint{2.665251in}{1.189347in}}%
\pgfpathlineto{\pgfqpoint{2.670406in}{1.193354in}}%
\pgfpathlineto{\pgfqpoint{2.675560in}{1.196750in}}%
\pgfpathlineto{\pgfqpoint{2.683292in}{1.198648in}}%
\pgfpathlineto{\pgfqpoint{2.688446in}{1.202627in}}%
\pgfpathlineto{\pgfqpoint{2.693601in}{1.207434in}}%
\pgfpathlineto{\pgfqpoint{2.701332in}{1.210125in}}%
\pgfpathlineto{\pgfqpoint{2.703910in}{1.212879in}}%
\pgfpathlineto{\pgfqpoint{2.709064in}{1.215680in}}%
\pgfpathlineto{\pgfqpoint{2.711641in}{1.218490in}}%
\pgfpathlineto{\pgfqpoint{2.719373in}{1.221370in}}%
\pgfpathlineto{\pgfqpoint{2.721950in}{1.224515in}}%
\pgfpathlineto{\pgfqpoint{2.727105in}{1.227277in}}%
\pgfpathlineto{\pgfqpoint{2.729682in}{1.230078in}}%
\pgfpathlineto{\pgfqpoint{2.737413in}{1.232830in}}%
\pgfpathlineto{\pgfqpoint{2.747722in}{1.243008in}}%
\pgfpathlineto{\pgfqpoint{2.755454in}{1.245003in}}%
\pgfpathlineto{\pgfqpoint{2.763185in}{1.251330in}}%
\pgfpathlineto{\pgfqpoint{2.765763in}{1.254167in}}%
\pgfpathlineto{\pgfqpoint{2.776072in}{1.256872in}}%
\pgfpathlineto{\pgfqpoint{2.781226in}{1.261971in}}%
\pgfpathlineto{\pgfqpoint{2.783803in}{1.263748in}}%
\pgfpathlineto{\pgfqpoint{2.791535in}{1.265254in}}%
\pgfpathlineto{\pgfqpoint{2.801844in}{1.271296in}}%
\pgfpathlineto{\pgfqpoint{2.809575in}{1.272204in}}%
\pgfpathlineto{\pgfqpoint{2.817307in}{1.275700in}}%
\pgfpathlineto{\pgfqpoint{2.819884in}{1.277310in}}%
\pgfpathlineto{\pgfqpoint{2.827616in}{1.279175in}}%
\pgfpathlineto{\pgfqpoint{2.837925in}{1.286861in}}%
\pgfpathlineto{\pgfqpoint{2.848234in}{1.288756in}}%
\pgfpathlineto{\pgfqpoint{2.855965in}{1.294223in}}%
\pgfpathlineto{\pgfqpoint{2.863697in}{1.296220in}}%
\pgfpathlineto{\pgfqpoint{2.874006in}{1.304305in}}%
\pgfpathlineto{\pgfqpoint{2.881737in}{1.306199in}}%
\pgfpathlineto{\pgfqpoint{2.892046in}{1.316154in}}%
\pgfpathlineto{\pgfqpoint{2.899778in}{1.318717in}}%
\pgfpathlineto{\pgfqpoint{2.907509in}{1.325308in}}%
\pgfpathlineto{\pgfqpoint{2.910087in}{1.327029in}}%
\pgfpathlineto{\pgfqpoint{2.917818in}{1.328946in}}%
\pgfpathlineto{\pgfqpoint{2.928127in}{1.336499in}}%
\pgfpathlineto{\pgfqpoint{2.935859in}{1.338264in}}%
\pgfpathlineto{\pgfqpoint{2.946168in}{1.344562in}}%
\pgfpathlineto{\pgfqpoint{2.953899in}{1.346056in}}%
\pgfpathlineto{\pgfqpoint{2.964208in}{1.352290in}}%
\pgfpathlineto{\pgfqpoint{2.971940in}{1.353184in}}%
\pgfpathlineto{\pgfqpoint{2.982249in}{1.356515in}}%
\pgfpathlineto{\pgfqpoint{2.992557in}{1.357964in}}%
\pgfpathlineto{\pgfqpoint{2.997712in}{1.359682in}}%
\pgfpathlineto{\pgfqpoint{3.008021in}{1.360505in}}%
\pgfpathlineto{\pgfqpoint{3.018330in}{1.363996in}}%
\pgfpathlineto{\pgfqpoint{3.026061in}{1.364831in}}%
\pgfpathlineto{\pgfqpoint{3.036370in}{1.368189in}}%
\pgfpathlineto{\pgfqpoint{3.046679in}{1.369688in}}%
\pgfpathlineto{\pgfqpoint{3.054411in}{1.372603in}}%
\pgfpathlineto{\pgfqpoint{3.062142in}{1.373733in}}%
\pgfpathlineto{\pgfqpoint{3.069874in}{1.376489in}}%
\pgfpathlineto{\pgfqpoint{3.072451in}{1.377263in}}%
\pgfpathlineto{\pgfqpoint{3.082760in}{1.378817in}}%
\pgfpathlineto{\pgfqpoint{3.090491in}{1.381360in}}%
\pgfpathlineto{\pgfqpoint{3.100800in}{1.382660in}}%
\pgfpathlineto{\pgfqpoint{3.108532in}{1.386433in}}%
\pgfpathlineto{\pgfqpoint{3.116264in}{1.387762in}}%
\pgfpathlineto{\pgfqpoint{3.123995in}{1.391781in}}%
\pgfpathlineto{\pgfqpoint{3.126572in}{1.393565in}}%
\pgfpathlineto{\pgfqpoint{3.134304in}{1.395449in}}%
\pgfpathlineto{\pgfqpoint{3.144613in}{1.402395in}}%
\pgfpathlineto{\pgfqpoint{3.152345in}{1.403982in}}%
\pgfpathlineto{\pgfqpoint{3.162653in}{1.410742in}}%
\pgfpathlineto{\pgfqpoint{3.170385in}{1.412441in}}%
\pgfpathlineto{\pgfqpoint{3.180694in}{1.418462in}}%
\pgfpathlineto{\pgfqpoint{3.188426in}{1.419978in}}%
\pgfpathlineto{\pgfqpoint{3.196157in}{1.424733in}}%
\pgfpathlineto{\pgfqpoint{3.206466in}{1.426276in}}%
\pgfpathlineto{\pgfqpoint{3.216775in}{1.431896in}}%
\pgfpathlineto{\pgfqpoint{3.224507in}{1.433276in}}%
\pgfpathlineto{\pgfqpoint{3.234815in}{1.438258in}}%
\pgfpathlineto{\pgfqpoint{3.242547in}{1.439332in}}%
\pgfpathlineto{\pgfqpoint{3.252856in}{1.443560in}}%
\pgfpathlineto{\pgfqpoint{3.260588in}{1.444446in}}%
\pgfpathlineto{\pgfqpoint{3.268319in}{1.446448in}}%
\pgfpathlineto{\pgfqpoint{3.288937in}{1.447886in}}%
\pgfpathlineto{\pgfqpoint{3.301823in}{1.448717in}}%
\pgfpathlineto{\pgfqpoint{3.306977in}{1.449212in}}%
\pgfpathlineto{\pgfqpoint{3.317286in}{1.449837in}}%
\pgfpathlineto{\pgfqpoint{3.325018in}{1.451118in}}%
\pgfpathlineto{\pgfqpoint{3.335327in}{1.452066in}}%
\pgfpathlineto{\pgfqpoint{3.343058in}{1.453416in}}%
\pgfpathlineto{\pgfqpoint{3.355944in}{1.454439in}}%
\pgfpathlineto{\pgfqpoint{3.361099in}{1.455373in}}%
\pgfpathlineto{\pgfqpoint{3.376562in}{1.456681in}}%
\pgfpathlineto{\pgfqpoint{3.415220in}{1.460057in}}%
\pgfpathlineto{\pgfqpoint{3.459033in}{1.460988in}}%
\pgfpathlineto{\pgfqpoint{3.482228in}{1.459799in}}%
\pgfpathlineto{\pgfqpoint{3.500268in}{1.460450in}}%
\pgfpathlineto{\pgfqpoint{3.518309in}{1.462770in}}%
\pgfpathlineto{\pgfqpoint{3.523463in}{1.463924in}}%
\pgfpathlineto{\pgfqpoint{3.533772in}{1.465090in}}%
\pgfpathlineto{\pgfqpoint{3.541504in}{1.466494in}}%
\pgfpathlineto{\pgfqpoint{3.554390in}{1.467648in}}%
\pgfpathlineto{\pgfqpoint{3.559544in}{1.468419in}}%
\pgfpathlineto{\pgfqpoint{3.569853in}{1.469315in}}%
\pgfpathlineto{\pgfqpoint{3.577585in}{1.470343in}}%
\pgfpathlineto{\pgfqpoint{3.587893in}{1.471485in}}%
\pgfpathlineto{\pgfqpoint{3.613666in}{1.475762in}}%
\pgfpathlineto{\pgfqpoint{3.626552in}{1.476471in}}%
\pgfpathlineto{\pgfqpoint{3.631706in}{1.477591in}}%
\pgfpathlineto{\pgfqpoint{3.642015in}{1.478836in}}%
\pgfpathlineto{\pgfqpoint{3.644592in}{1.479489in}}%
\pgfpathlineto{\pgfqpoint{3.667787in}{1.482520in}}%
\pgfpathlineto{\pgfqpoint{3.683250in}{1.483566in}}%
\pgfpathlineto{\pgfqpoint{3.685828in}{1.483842in}}%
\pgfpathlineto{\pgfqpoint{3.716754in}{1.484083in}}%
\pgfpathlineto{\pgfqpoint{3.809534in}{1.476952in}}%
\pgfpathlineto{\pgfqpoint{3.812111in}{1.476625in}}%
\pgfpathlineto{\pgfqpoint{3.824997in}{1.475638in}}%
\pgfpathlineto{\pgfqpoint{3.830151in}{1.474916in}}%
\pgfpathlineto{\pgfqpoint{3.840460in}{1.474130in}}%
\pgfpathlineto{\pgfqpoint{3.848192in}{1.473035in}}%
\pgfpathlineto{\pgfqpoint{3.861078in}{1.472047in}}%
\pgfpathlineto{\pgfqpoint{3.866232in}{1.471447in}}%
\pgfpathlineto{\pgfqpoint{3.879118in}{1.470438in}}%
\pgfpathlineto{\pgfqpoint{3.884273in}{1.469566in}}%
\pgfpathlineto{\pgfqpoint{3.897159in}{1.468290in}}%
\pgfpathlineto{\pgfqpoint{3.899736in}{1.467909in}}%
\pgfpathlineto{\pgfqpoint{3.912622in}{1.467103in}}%
\pgfpathlineto{\pgfqpoint{3.920354in}{1.465912in}}%
\pgfpathlineto{\pgfqpoint{3.933240in}{1.464759in}}%
\pgfpathlineto{\pgfqpoint{3.956435in}{1.461841in}}%
\pgfpathlineto{\pgfqpoint{3.966744in}{1.460972in}}%
\pgfpathlineto{\pgfqpoint{3.974475in}{1.459669in}}%
\pgfpathlineto{\pgfqpoint{3.984784in}{1.458835in}}%
\pgfpathlineto{\pgfqpoint{3.992516in}{1.457616in}}%
\pgfpathlineto{\pgfqpoint{4.005402in}{1.456500in}}%
\pgfpathlineto{\pgfqpoint{4.010556in}{1.455821in}}%
\pgfpathlineto{\pgfqpoint{4.023442in}{1.454844in}}%
\pgfpathlineto{\pgfqpoint{4.028597in}{1.454206in}}%
\pgfpathlineto{\pgfqpoint{4.044060in}{1.453157in}}%
\pgfpathlineto{\pgfqpoint{4.046637in}{1.452795in}}%
\pgfpathlineto{\pgfqpoint{4.059523in}{1.451742in}}%
\pgfpathlineto{\pgfqpoint{4.064678in}{1.451000in}}%
\pgfpathlineto{\pgfqpoint{4.077564in}{1.449895in}}%
\pgfpathlineto{\pgfqpoint{4.082718in}{1.449192in}}%
\pgfpathlineto{\pgfqpoint{4.095604in}{1.448115in}}%
\pgfpathlineto{\pgfqpoint{4.100759in}{1.447453in}}%
\pgfpathlineto{\pgfqpoint{4.113645in}{1.446529in}}%
\pgfpathlineto{\pgfqpoint{4.118799in}{1.445830in}}%
\pgfpathlineto{\pgfqpoint{4.131685in}{1.444646in}}%
\pgfpathlineto{\pgfqpoint{4.134263in}{1.444261in}}%
\pgfpathlineto{\pgfqpoint{4.147149in}{1.443461in}}%
\pgfpathlineto{\pgfqpoint{4.154880in}{1.442206in}}%
\pgfpathlineto{\pgfqpoint{4.167766in}{1.441097in}}%
\pgfpathlineto{\pgfqpoint{4.172921in}{1.440381in}}%
\pgfpathlineto{\pgfqpoint{4.185807in}{1.439312in}}%
\pgfpathlineto{\pgfqpoint{4.190961in}{1.438485in}}%
\pgfpathlineto{\pgfqpoint{4.201270in}{1.437621in}}%
\pgfpathlineto{\pgfqpoint{4.209002in}{1.436365in}}%
\pgfpathlineto{\pgfqpoint{4.219311in}{1.435527in}}%
\pgfpathlineto{\pgfqpoint{4.227042in}{1.434348in}}%
\pgfpathlineto{\pgfqpoint{4.239928in}{1.433508in}}%
\pgfpathlineto{\pgfqpoint{4.245083in}{1.432923in}}%
\pgfpathlineto{\pgfqpoint{4.257969in}{1.432071in}}%
\pgfpathlineto{\pgfqpoint{4.263123in}{1.431323in}}%
\pgfpathlineto{\pgfqpoint{4.273432in}{1.430471in}}%
\pgfpathlineto{\pgfqpoint{4.281164in}{1.429269in}}%
\pgfpathlineto{\pgfqpoint{4.291472in}{1.428451in}}%
\pgfpathlineto{\pgfqpoint{4.299204in}{1.427201in}}%
\pgfpathlineto{\pgfqpoint{4.312090in}{1.426380in}}%
\pgfpathlineto{\pgfqpoint{4.317245in}{1.425564in}}%
\pgfpathlineto{\pgfqpoint{4.330131in}{1.424385in}}%
\pgfpathlineto{\pgfqpoint{4.335285in}{1.423606in}}%
\pgfpathlineto{\pgfqpoint{4.348171in}{1.422420in}}%
\pgfpathlineto{\pgfqpoint{4.353326in}{1.421606in}}%
\pgfpathlineto{\pgfqpoint{4.366212in}{1.420368in}}%
\pgfpathlineto{\pgfqpoint{4.371366in}{1.419550in}}%
\pgfpathlineto{\pgfqpoint{4.384252in}{1.418439in}}%
\pgfpathlineto{\pgfqpoint{4.389407in}{1.417713in}}%
\pgfpathlineto{\pgfqpoint{4.402293in}{1.416598in}}%
\pgfpathlineto{\pgfqpoint{4.407447in}{1.415866in}}%
\pgfpathlineto{\pgfqpoint{4.420333in}{1.414764in}}%
\pgfpathlineto{\pgfqpoint{4.425488in}{1.413966in}}%
\pgfpathlineto{\pgfqpoint{4.438374in}{1.412777in}}%
\pgfpathlineto{\pgfqpoint{4.443528in}{1.411983in}}%
\pgfpathlineto{\pgfqpoint{4.456414in}{1.410799in}}%
\pgfpathlineto{\pgfqpoint{4.461569in}{1.410014in}}%
\pgfpathlineto{\pgfqpoint{4.474455in}{1.408830in}}%
\pgfpathlineto{\pgfqpoint{4.479609in}{1.408047in}}%
\pgfpathlineto{\pgfqpoint{4.492495in}{1.406877in}}%
\pgfpathlineto{\pgfqpoint{4.497649in}{1.406096in}}%
\pgfpathlineto{\pgfqpoint{4.510536in}{1.404932in}}%
\pgfpathlineto{\pgfqpoint{4.544039in}{1.401889in}}%
\pgfpathlineto{\pgfqpoint{4.551771in}{1.400799in}}%
\pgfpathlineto{\pgfqpoint{4.564657in}{1.399705in}}%
\pgfpathlineto{\pgfqpoint{4.569811in}{1.399011in}}%
\pgfpathlineto{\pgfqpoint{4.582697in}{1.397956in}}%
\pgfpathlineto{\pgfqpoint{4.585275in}{1.397591in}}%
\pgfpathlineto{\pgfqpoint{4.600738in}{1.396499in}}%
\pgfpathlineto{\pgfqpoint{4.603315in}{1.396143in}}%
\pgfpathlineto{\pgfqpoint{4.621356in}{1.395050in}}%
\pgfpathlineto{\pgfqpoint{4.641973in}{1.393769in}}%
\pgfpathlineto{\pgfqpoint{4.667746in}{1.393686in}}%
\pgfpathlineto{\pgfqpoint{4.711558in}{1.398090in}}%
\pgfpathlineto{\pgfqpoint{4.714135in}{1.398474in}}%
\pgfpathlineto{\pgfqpoint{4.739907in}{1.399776in}}%
\pgfpathlineto{\pgfqpoint{4.760525in}{1.400782in}}%
\pgfpathlineto{\pgfqpoint{4.778566in}{1.400737in}}%
\pgfpathlineto{\pgfqpoint{4.837842in}{1.399130in}}%
\pgfpathlineto{\pgfqpoint{4.858459in}{1.398502in}}%
\pgfpathlineto{\pgfqpoint{4.873922in}{1.397854in}}%
\pgfpathlineto{\pgfqpoint{4.966702in}{1.391064in}}%
\pgfpathlineto{\pgfqpoint{4.979588in}{1.390278in}}%
\pgfpathlineto{\pgfqpoint{4.984743in}{1.389718in}}%
\pgfpathlineto{\pgfqpoint{5.000206in}{1.388841in}}%
\pgfpathlineto{\pgfqpoint{5.020824in}{1.387134in}}%
\pgfpathlineto{\pgfqpoint{5.051750in}{1.385649in}}%
\pgfpathlineto{\pgfqpoint{5.056905in}{1.385313in}}%
\pgfpathlineto{\pgfqpoint{5.085254in}{1.384793in}}%
\pgfpathlineto{\pgfqpoint{5.118758in}{1.383006in}}%
\pgfpathlineto{\pgfqpoint{5.147107in}{1.380700in}}%
\pgfpathlineto{\pgfqpoint{5.162570in}{1.379725in}}%
\pgfpathlineto{\pgfqpoint{5.175456in}{1.378888in}}%
\pgfpathlineto{\pgfqpoint{5.201228in}{1.376694in}}%
\pgfpathlineto{\pgfqpoint{5.216692in}{1.375612in}}%
\pgfpathlineto{\pgfqpoint{5.237309in}{1.374005in}}%
\pgfpathlineto{\pgfqpoint{5.252773in}{1.373164in}}%
\pgfpathlineto{\pgfqpoint{5.265659in}{1.372375in}}%
\pgfpathlineto{\pgfqpoint{5.291431in}{1.370474in}}%
\pgfpathlineto{\pgfqpoint{5.306894in}{1.369539in}}%
\pgfpathlineto{\pgfqpoint{5.324935in}{1.368389in}}%
\pgfpathlineto{\pgfqpoint{5.345552in}{1.367421in}}%
\pgfpathlineto{\pgfqpoint{5.361016in}{1.366706in}}%
\pgfpathlineto{\pgfqpoint{5.363593in}{1.366415in}}%
\pgfpathlineto{\pgfqpoint{5.376479in}{1.365559in}}%
\pgfpathlineto{\pgfqpoint{5.381633in}{1.364993in}}%
\pgfpathlineto{\pgfqpoint{5.397097in}{1.363907in}}%
\pgfpathlineto{\pgfqpoint{5.417714in}{1.362193in}}%
\pgfpathlineto{\pgfqpoint{5.430600in}{1.361353in}}%
\pgfpathlineto{\pgfqpoint{5.435755in}{1.360781in}}%
\pgfpathlineto{\pgfqpoint{5.461527in}{1.359393in}}%
\pgfpathlineto{\pgfqpoint{5.471836in}{1.358278in}}%
\pgfpathlineto{\pgfqpoint{5.487299in}{1.357243in}}%
\pgfpathlineto{\pgfqpoint{5.507917in}{1.355778in}}%
\pgfpathlineto{\pgfqpoint{5.523380in}{1.354842in}}%
\pgfpathlineto{\pgfqpoint{5.525957in}{1.354610in}}%
\pgfpathlineto{\pgfqpoint{5.541421in}{1.353886in}}%
\pgfpathlineto{\pgfqpoint{5.543998in}{1.353639in}}%
\pgfpathlineto{\pgfqpoint{5.569770in}{1.352589in}}%
\pgfpathlineto{\pgfqpoint{5.580079in}{1.351867in}}%
\pgfpathlineto{\pgfqpoint{5.605851in}{1.350946in}}%
\pgfpathlineto{\pgfqpoint{5.641932in}{1.349244in}}%
\pgfpathlineto{\pgfqpoint{5.665127in}{1.348376in}}%
\pgfpathlineto{\pgfqpoint{5.696053in}{1.347810in}}%
\pgfpathlineto{\pgfqpoint{5.809451in}{1.344053in}}%
\pgfpathlineto{\pgfqpoint{5.812028in}{1.343859in}}%
\pgfpathlineto{\pgfqpoint{5.866149in}{1.342422in}}%
\pgfpathlineto{\pgfqpoint{5.922848in}{1.340057in}}%
\pgfpathlineto{\pgfqpoint{6.010473in}{1.339147in}}%
\pgfpathlineto{\pgfqpoint{6.219227in}{1.347803in}}%
\pgfpathlineto{\pgfqpoint{6.229536in}{1.349030in}}%
\pgfpathlineto{\pgfqpoint{6.242422in}{1.349979in}}%
\pgfpathlineto{\pgfqpoint{6.247577in}{1.350793in}}%
\pgfpathlineto{\pgfqpoint{6.257886in}{1.351691in}}%
\pgfpathlineto{\pgfqpoint{6.265617in}{1.353123in}}%
\pgfpathlineto{\pgfqpoint{6.275926in}{1.354160in}}%
\pgfpathlineto{\pgfqpoint{6.283658in}{1.355776in}}%
\pgfpathlineto{\pgfqpoint{6.293967in}{1.356797in}}%
\pgfpathlineto{\pgfqpoint{6.301698in}{1.358354in}}%
\pgfpathlineto{\pgfqpoint{6.312007in}{1.359539in}}%
\pgfpathlineto{\pgfqpoint{6.319739in}{1.361731in}}%
\pgfpathlineto{\pgfqpoint{6.330048in}{1.363219in}}%
\pgfpathlineto{\pgfqpoint{6.337779in}{1.365613in}}%
\pgfpathlineto{\pgfqpoint{6.348088in}{1.367155in}}%
\pgfpathlineto{\pgfqpoint{6.355820in}{1.369054in}}%
\pgfpathlineto{\pgfqpoint{6.366129in}{1.370261in}}%
\pgfpathlineto{\pgfqpoint{6.373860in}{1.371982in}}%
\pgfpathlineto{\pgfqpoint{6.384169in}{1.373216in}}%
\pgfpathlineto{\pgfqpoint{6.386746in}{1.373803in}}%
\pgfpathlineto{\pgfqpoint{6.402209in}{1.375608in}}%
\pgfpathlineto{\pgfqpoint{6.409941in}{1.378155in}}%
\pgfpathlineto{\pgfqpoint{6.417673in}{1.379096in}}%
\pgfpathlineto{\pgfqpoint{6.427982in}{1.382795in}}%
\pgfpathlineto{\pgfqpoint{6.435713in}{1.383751in}}%
\pgfpathlineto{\pgfqpoint{6.446022in}{1.387274in}}%
\pgfpathlineto{\pgfqpoint{6.453754in}{1.388268in}}%
\pgfpathlineto{\pgfqpoint{6.464063in}{1.391882in}}%
\pgfpathlineto{\pgfqpoint{6.474371in}{1.392764in}}%
\pgfpathlineto{\pgfqpoint{6.482103in}{1.395586in}}%
\pgfpathlineto{\pgfqpoint{6.482103in}{1.395586in}}%
\pgfusepath{stroke}%
\end{pgfscope}%
\begin{pgfscope}%
\pgfpathrectangle{\pgfqpoint{0.563921in}{0.521603in}}{\pgfqpoint{6.200000in}{2.642500in}}%
\pgfusepath{clip}%
\pgfsetroundcap%
\pgfsetroundjoin%
\pgfsetlinewidth{1.505625pt}%
\definecolor{currentstroke}{rgb}{0.172549,0.627451,0.172549}%
\pgfsetstrokecolor{currentstroke}%
\pgfsetdash{}{0pt}%
\pgfpathmoveto{\pgfqpoint{0.845739in}{0.641717in}}%
\pgfpathlineto{\pgfqpoint{0.848317in}{0.646186in}}%
\pgfpathlineto{\pgfqpoint{0.850894in}{0.645797in}}%
\pgfpathlineto{\pgfqpoint{0.853471in}{0.646472in}}%
\pgfpathlineto{\pgfqpoint{0.861203in}{0.649087in}}%
\pgfpathlineto{\pgfqpoint{0.863780in}{0.648748in}}%
\pgfpathlineto{\pgfqpoint{0.868934in}{0.650005in}}%
\pgfpathlineto{\pgfqpoint{0.881820in}{0.649413in}}%
\pgfpathlineto{\pgfqpoint{0.889552in}{0.651848in}}%
\pgfpathlineto{\pgfqpoint{0.935942in}{0.650998in}}%
\pgfpathlineto{\pgfqpoint{0.941096in}{0.651392in}}%
\pgfpathlineto{\pgfqpoint{0.953982in}{0.651004in}}%
\pgfpathlineto{\pgfqpoint{0.961714in}{0.651138in}}%
\pgfpathlineto{\pgfqpoint{0.974600in}{0.652163in}}%
\pgfpathlineto{\pgfqpoint{0.979754in}{0.653092in}}%
\pgfpathlineto{\pgfqpoint{1.023567in}{0.653098in}}%
\pgfpathlineto{\pgfqpoint{1.028722in}{0.655233in}}%
\pgfpathlineto{\pgfqpoint{1.033876in}{0.659190in}}%
\pgfpathlineto{\pgfqpoint{1.044185in}{0.661788in}}%
\pgfpathlineto{\pgfqpoint{1.049339in}{0.663127in}}%
\pgfpathlineto{\pgfqpoint{1.082843in}{0.666623in}}%
\pgfpathlineto{\pgfqpoint{1.136964in}{0.664768in}}%
\pgfpathlineto{\pgfqpoint{1.209126in}{0.663750in}}%
\pgfpathlineto{\pgfqpoint{1.232321in}{0.663311in}}%
\pgfpathlineto{\pgfqpoint{1.250362in}{0.663638in}}%
\pgfpathlineto{\pgfqpoint{1.276134in}{0.663776in}}%
\pgfpathlineto{\pgfqpoint{1.286443in}{0.664486in}}%
\pgfpathlineto{\pgfqpoint{1.299329in}{0.665051in}}%
\pgfpathlineto{\pgfqpoint{1.301906in}{0.665475in}}%
\pgfpathlineto{\pgfqpoint{1.304483in}{0.666532in}}%
\pgfpathlineto{\pgfqpoint{1.335410in}{0.668448in}}%
\pgfpathlineto{\pgfqpoint{1.353450in}{0.668482in}}%
\pgfpathlineto{\pgfqpoint{1.371491in}{0.668981in}}%
\pgfpathlineto{\pgfqpoint{1.394686in}{0.672655in}}%
\pgfpathlineto{\pgfqpoint{1.404995in}{0.673929in}}%
\pgfpathlineto{\pgfqpoint{1.412726in}{0.675643in}}%
\pgfpathlineto{\pgfqpoint{1.425612in}{0.676966in}}%
\pgfpathlineto{\pgfqpoint{1.430767in}{0.677831in}}%
\pgfpathlineto{\pgfqpoint{1.459116in}{0.679641in}}%
\pgfpathlineto{\pgfqpoint{1.466848in}{0.680174in}}%
\pgfpathlineto{\pgfqpoint{1.482311in}{0.680832in}}%
\pgfpathlineto{\pgfqpoint{1.484888in}{0.681434in}}%
\pgfpathlineto{\pgfqpoint{1.495197in}{0.682499in}}%
\pgfpathlineto{\pgfqpoint{1.502929in}{0.684836in}}%
\pgfpathlineto{\pgfqpoint{1.510660in}{0.685580in}}%
\pgfpathlineto{\pgfqpoint{1.520969in}{0.689437in}}%
\pgfpathlineto{\pgfqpoint{1.531278in}{0.691146in}}%
\pgfpathlineto{\pgfqpoint{1.536432in}{0.692843in}}%
\pgfpathlineto{\pgfqpoint{1.539010in}{0.693830in}}%
\pgfpathlineto{\pgfqpoint{1.546741in}{0.694837in}}%
\pgfpathlineto{\pgfqpoint{1.557050in}{0.698835in}}%
\pgfpathlineto{\pgfqpoint{1.567359in}{0.700412in}}%
\pgfpathlineto{\pgfqpoint{1.575091in}{0.702106in}}%
\pgfpathlineto{\pgfqpoint{1.585399in}{0.703308in}}%
\pgfpathlineto{\pgfqpoint{1.593131in}{0.704932in}}%
\pgfpathlineto{\pgfqpoint{1.629212in}{0.705200in}}%
\pgfpathlineto{\pgfqpoint{1.660139in}{0.704924in}}%
\pgfpathlineto{\pgfqpoint{1.678179in}{0.704367in}}%
\pgfpathlineto{\pgfqpoint{1.737455in}{0.704397in}}%
\pgfpathlineto{\pgfqpoint{1.763227in}{0.704437in}}%
\pgfpathlineto{\pgfqpoint{1.825080in}{0.704058in}}%
\pgfpathlineto{\pgfqpoint{1.827657in}{0.704245in}}%
\pgfpathlineto{\pgfqpoint{1.845698in}{0.704984in}}%
\pgfpathlineto{\pgfqpoint{1.861161in}{0.705998in}}%
\pgfpathlineto{\pgfqpoint{1.863738in}{0.706314in}}%
\pgfpathlineto{\pgfqpoint{1.879202in}{0.707301in}}%
\pgfpathlineto{\pgfqpoint{1.915283in}{0.711182in}}%
\pgfpathlineto{\pgfqpoint{1.917860in}{0.711714in}}%
\pgfpathlineto{\pgfqpoint{1.928169in}{0.712424in}}%
\pgfpathlineto{\pgfqpoint{1.935900in}{0.713752in}}%
\pgfpathlineto{\pgfqpoint{1.946209in}{0.714707in}}%
\pgfpathlineto{\pgfqpoint{1.953941in}{0.716353in}}%
\pgfpathlineto{\pgfqpoint{1.966827in}{0.717693in}}%
\pgfpathlineto{\pgfqpoint{1.971981in}{0.718577in}}%
\pgfpathlineto{\pgfqpoint{1.984867in}{0.719632in}}%
\pgfpathlineto{\pgfqpoint{1.990022in}{0.720294in}}%
\pgfpathlineto{\pgfqpoint{2.018371in}{0.721869in}}%
\pgfpathlineto{\pgfqpoint{2.051875in}{0.723973in}}%
\pgfpathlineto{\pgfqpoint{2.062184in}{0.724327in}}%
\pgfpathlineto{\pgfqpoint{2.105996in}{0.724332in}}%
\pgfpathlineto{\pgfqpoint{2.144655in}{0.726071in}}%
\pgfpathlineto{\pgfqpoint{2.152386in}{0.726835in}}%
\pgfpathlineto{\pgfqpoint{2.178158in}{0.727882in}}%
\pgfpathlineto{\pgfqpoint{2.196199in}{0.728895in}}%
\pgfpathlineto{\pgfqpoint{2.224548in}{0.730713in}}%
\pgfpathlineto{\pgfqpoint{2.255475in}{0.731533in}}%
\pgfpathlineto{\pgfqpoint{2.286401in}{0.732725in}}%
\pgfpathlineto{\pgfqpoint{2.304442in}{0.733999in}}%
\pgfpathlineto{\pgfqpoint{2.314751in}{0.735461in}}%
\pgfpathlineto{\pgfqpoint{2.327637in}{0.736348in}}%
\pgfpathlineto{\pgfqpoint{2.332791in}{0.737013in}}%
\pgfpathlineto{\pgfqpoint{2.348254in}{0.738042in}}%
\pgfpathlineto{\pgfqpoint{2.366295in}{0.739034in}}%
\pgfpathlineto{\pgfqpoint{2.386912in}{0.739747in}}%
\pgfpathlineto{\pgfqpoint{2.451343in}{0.740844in}}%
\pgfpathlineto{\pgfqpoint{2.459074in}{0.741503in}}%
\pgfpathlineto{\pgfqpoint{2.487424in}{0.742676in}}%
\pgfpathlineto{\pgfqpoint{2.502887in}{0.743254in}}%
\pgfpathlineto{\pgfqpoint{2.528659in}{0.744438in}}%
\pgfpathlineto{\pgfqpoint{2.531236in}{0.744831in}}%
\pgfpathlineto{\pgfqpoint{2.544122in}{0.746253in}}%
\pgfpathlineto{\pgfqpoint{2.549277in}{0.747145in}}%
\pgfpathlineto{\pgfqpoint{2.559586in}{0.748123in}}%
\pgfpathlineto{\pgfqpoint{2.567317in}{0.749680in}}%
\pgfpathlineto{\pgfqpoint{2.577626in}{0.750714in}}%
\pgfpathlineto{\pgfqpoint{2.585358in}{0.752523in}}%
\pgfpathlineto{\pgfqpoint{2.595667in}{0.753775in}}%
\pgfpathlineto{\pgfqpoint{2.603398in}{0.755644in}}%
\pgfpathlineto{\pgfqpoint{2.613707in}{0.756848in}}%
\pgfpathlineto{\pgfqpoint{2.621439in}{0.758495in}}%
\pgfpathlineto{\pgfqpoint{2.631748in}{0.759451in}}%
\pgfpathlineto{\pgfqpoint{2.639479in}{0.760379in}}%
\pgfpathlineto{\pgfqpoint{2.652365in}{0.761649in}}%
\pgfpathlineto{\pgfqpoint{2.675560in}{0.764698in}}%
\pgfpathlineto{\pgfqpoint{2.685869in}{0.765595in}}%
\pgfpathlineto{\pgfqpoint{2.693601in}{0.767215in}}%
\pgfpathlineto{\pgfqpoint{2.703910in}{0.768354in}}%
\pgfpathlineto{\pgfqpoint{2.765763in}{0.775984in}}%
\pgfpathlineto{\pgfqpoint{2.899778in}{0.779611in}}%
\pgfpathlineto{\pgfqpoint{2.910087in}{0.780062in}}%
\pgfpathlineto{\pgfqpoint{2.943590in}{0.780970in}}%
\pgfpathlineto{\pgfqpoint{2.977094in}{0.782232in}}%
\pgfpathlineto{\pgfqpoint{3.031216in}{0.784609in}}%
\pgfpathlineto{\pgfqpoint{3.046679in}{0.785340in}}%
\pgfpathlineto{\pgfqpoint{3.072451in}{0.786811in}}%
\pgfpathlineto{\pgfqpoint{3.105955in}{0.788065in}}%
\pgfpathlineto{\pgfqpoint{3.126572in}{0.789137in}}%
\pgfpathlineto{\pgfqpoint{3.154922in}{0.790552in}}%
\pgfpathlineto{\pgfqpoint{3.180694in}{0.791810in}}%
\pgfpathlineto{\pgfqpoint{3.214198in}{0.792841in}}%
\pgfpathlineto{\pgfqpoint{3.242547in}{0.793739in}}%
\pgfpathlineto{\pgfqpoint{3.270896in}{0.794033in}}%
\pgfpathlineto{\pgfqpoint{3.322441in}{0.794535in}}%
\pgfpathlineto{\pgfqpoint{3.361099in}{0.795046in}}%
\pgfpathlineto{\pgfqpoint{3.482228in}{0.795299in}}%
\pgfpathlineto{\pgfqpoint{3.536349in}{0.795893in}}%
\pgfpathlineto{\pgfqpoint{3.595625in}{0.797221in}}%
\pgfpathlineto{\pgfqpoint{3.685828in}{0.796790in}}%
\pgfpathlineto{\pgfqpoint{3.755412in}{0.795583in}}%
\pgfpathlineto{\pgfqpoint{3.794070in}{0.795314in}}%
\pgfpathlineto{\pgfqpoint{3.920354in}{0.795760in}}%
\pgfpathlineto{\pgfqpoint{3.969321in}{0.797467in}}%
\pgfpathlineto{\pgfqpoint{4.010556in}{0.798906in}}%
\pgfpathlineto{\pgfqpoint{4.041483in}{0.799939in}}%
\pgfpathlineto{\pgfqpoint{4.064678in}{0.800817in}}%
\pgfpathlineto{\pgfqpoint{4.095604in}{0.801754in}}%
\pgfpathlineto{\pgfqpoint{4.118799in}{0.802628in}}%
\pgfpathlineto{\pgfqpoint{4.281164in}{0.803621in}}%
\pgfpathlineto{\pgfqpoint{4.389407in}{0.803431in}}%
\pgfpathlineto{\pgfqpoint{4.407447in}{0.804344in}}%
\pgfpathlineto{\pgfqpoint{4.422910in}{0.805427in}}%
\pgfpathlineto{\pgfqpoint{4.425488in}{0.805744in}}%
\pgfpathlineto{\pgfqpoint{4.438374in}{0.806666in}}%
\pgfpathlineto{\pgfqpoint{4.443528in}{0.807197in}}%
\pgfpathlineto{\pgfqpoint{4.456414in}{0.808106in}}%
\pgfpathlineto{\pgfqpoint{4.461569in}{0.808762in}}%
\pgfpathlineto{\pgfqpoint{4.471877in}{0.809446in}}%
\pgfpathlineto{\pgfqpoint{4.479609in}{0.810609in}}%
\pgfpathlineto{\pgfqpoint{4.492495in}{0.811758in}}%
\pgfpathlineto{\pgfqpoint{4.497649in}{0.812528in}}%
\pgfpathlineto{\pgfqpoint{4.510536in}{0.813687in}}%
\pgfpathlineto{\pgfqpoint{4.546617in}{0.816709in}}%
\pgfpathlineto{\pgfqpoint{4.551771in}{0.817409in}}%
\pgfpathlineto{\pgfqpoint{4.564657in}{0.818466in}}%
\pgfpathlineto{\pgfqpoint{4.569811in}{0.819185in}}%
\pgfpathlineto{\pgfqpoint{4.582697in}{0.820318in}}%
\pgfpathlineto{\pgfqpoint{4.585275in}{0.820716in}}%
\pgfpathlineto{\pgfqpoint{4.598161in}{0.821566in}}%
\pgfpathlineto{\pgfqpoint{4.603315in}{0.822407in}}%
\pgfpathlineto{\pgfqpoint{4.618778in}{0.823463in}}%
\pgfpathlineto{\pgfqpoint{4.639396in}{0.824410in}}%
\pgfpathlineto{\pgfqpoint{4.660014in}{0.825054in}}%
\pgfpathlineto{\pgfqpoint{4.688363in}{0.825929in}}%
\pgfpathlineto{\pgfqpoint{4.714135in}{0.826884in}}%
\pgfpathlineto{\pgfqpoint{4.739907in}{0.827765in}}%
\pgfpathlineto{\pgfqpoint{4.814647in}{0.833403in}}%
\pgfpathlineto{\pgfqpoint{4.819801in}{0.834103in}}%
\pgfpathlineto{\pgfqpoint{4.832687in}{0.834880in}}%
\pgfpathlineto{\pgfqpoint{4.840419in}{0.836151in}}%
\pgfpathlineto{\pgfqpoint{4.853305in}{0.837135in}}%
\pgfpathlineto{\pgfqpoint{4.876500in}{0.839235in}}%
\pgfpathlineto{\pgfqpoint{4.889386in}{0.840189in}}%
\pgfpathlineto{\pgfqpoint{4.894540in}{0.840767in}}%
\pgfpathlineto{\pgfqpoint{4.907426in}{0.841612in}}%
\pgfpathlineto{\pgfqpoint{4.912581in}{0.842165in}}%
\pgfpathlineto{\pgfqpoint{4.940930in}{0.843712in}}%
\pgfpathlineto{\pgfqpoint{4.956393in}{0.844478in}}%
\pgfpathlineto{\pgfqpoint{4.984743in}{0.845964in}}%
\pgfpathlineto{\pgfqpoint{5.010515in}{0.846887in}}%
\pgfpathlineto{\pgfqpoint{5.056905in}{0.849903in}}%
\pgfpathlineto{\pgfqpoint{5.069791in}{0.850423in}}%
\pgfpathlineto{\pgfqpoint{5.074945in}{0.851029in}}%
\pgfpathlineto{\pgfqpoint{5.090408in}{0.851979in}}%
\pgfpathlineto{\pgfqpoint{5.092986in}{0.852344in}}%
\pgfpathlineto{\pgfqpoint{5.105872in}{0.853452in}}%
\pgfpathlineto{\pgfqpoint{5.111026in}{0.854280in}}%
\pgfpathlineto{\pgfqpoint{5.123912in}{0.855540in}}%
\pgfpathlineto{\pgfqpoint{5.129067in}{0.856256in}}%
\pgfpathlineto{\pgfqpoint{5.144530in}{0.857322in}}%
\pgfpathlineto{\pgfqpoint{5.165148in}{0.858745in}}%
\pgfpathlineto{\pgfqpoint{5.180611in}{0.859713in}}%
\pgfpathlineto{\pgfqpoint{5.201228in}{0.861124in}}%
\pgfpathlineto{\pgfqpoint{5.216692in}{0.862029in}}%
\pgfpathlineto{\pgfqpoint{5.237309in}{0.863373in}}%
\pgfpathlineto{\pgfqpoint{5.268236in}{0.864470in}}%
\pgfpathlineto{\pgfqpoint{5.309471in}{0.865841in}}%
\pgfpathlineto{\pgfqpoint{5.415137in}{0.867698in}}%
\pgfpathlineto{\pgfqpoint{5.435755in}{0.868651in}}%
\pgfpathlineto{\pgfqpoint{5.464104in}{0.869789in}}%
\pgfpathlineto{\pgfqpoint{5.562038in}{0.875350in}}%
\pgfpathlineto{\pgfqpoint{5.592965in}{0.876672in}}%
\pgfpathlineto{\pgfqpoint{5.605851in}{0.877053in}}%
\pgfpathlineto{\pgfqpoint{5.649663in}{0.878107in}}%
\pgfpathlineto{\pgfqpoint{5.760484in}{0.880796in}}%
\pgfpathlineto{\pgfqpoint{5.804296in}{0.881646in}}%
\pgfpathlineto{\pgfqpoint{5.832646in}{0.882234in}}%
\pgfpathlineto{\pgfqpoint{5.933157in}{0.882468in}}%
\pgfpathlineto{\pgfqpoint{5.976969in}{0.882394in}}%
\pgfpathlineto{\pgfqpoint{6.031091in}{0.882105in}}%
\pgfpathlineto{\pgfqpoint{6.100676in}{0.881034in}}%
\pgfpathlineto{\pgfqpoint{6.211496in}{0.878718in}}%
\pgfpathlineto{\pgfqpoint{6.263040in}{0.877660in}}%
\pgfpathlineto{\pgfqpoint{6.319739in}{0.876455in}}%
\pgfpathlineto{\pgfqpoint{6.404787in}{0.876523in}}%
\pgfpathlineto{\pgfqpoint{6.482103in}{0.877577in}}%
\pgfpathlineto{\pgfqpoint{6.482103in}{0.877577in}}%
\pgfusepath{stroke}%
\end{pgfscope}%
\begin{pgfscope}%
\pgfpathrectangle{\pgfqpoint{0.563921in}{0.521603in}}{\pgfqpoint{6.200000in}{2.642500in}}%
\pgfusepath{clip}%
\pgfsetroundcap%
\pgfsetroundjoin%
\pgfsetlinewidth{1.505625pt}%
\definecolor{currentstroke}{rgb}{0.839216,0.152941,0.156863}%
\pgfsetstrokecolor{currentstroke}%
\pgfsetdash{}{0pt}%
\pgfpathmoveto{\pgfqpoint{0.845739in}{0.641717in}}%
\pgfpathlineto{\pgfqpoint{0.848317in}{0.655422in}}%
\pgfpathlineto{\pgfqpoint{0.850894in}{0.658811in}}%
\pgfpathlineto{\pgfqpoint{0.853471in}{0.657284in}}%
\pgfpathlineto{\pgfqpoint{0.863780in}{0.658051in}}%
\pgfpathlineto{\pgfqpoint{0.866357in}{0.659514in}}%
\pgfpathlineto{\pgfqpoint{0.868934in}{0.659704in}}%
\pgfpathlineto{\pgfqpoint{0.871512in}{0.658998in}}%
\pgfpathlineto{\pgfqpoint{0.881820in}{0.658654in}}%
\pgfpathlineto{\pgfqpoint{0.884398in}{0.657905in}}%
\pgfpathlineto{\pgfqpoint{0.886975in}{0.657753in}}%
\pgfpathlineto{\pgfqpoint{0.889552in}{0.661881in}}%
\pgfpathlineto{\pgfqpoint{0.897284in}{0.666631in}}%
\pgfpathlineto{\pgfqpoint{0.899861in}{0.670761in}}%
\pgfpathlineto{\pgfqpoint{0.902438in}{0.673459in}}%
\pgfpathlineto{\pgfqpoint{0.907593in}{0.675500in}}%
\pgfpathlineto{\pgfqpoint{0.915324in}{0.676056in}}%
\pgfpathlineto{\pgfqpoint{0.923056in}{0.675251in}}%
\pgfpathlineto{\pgfqpoint{0.925633in}{0.675946in}}%
\pgfpathlineto{\pgfqpoint{0.938519in}{0.677104in}}%
\pgfpathlineto{\pgfqpoint{0.943674in}{0.677541in}}%
\pgfpathlineto{\pgfqpoint{0.959137in}{0.677399in}}%
\pgfpathlineto{\pgfqpoint{0.961714in}{0.678468in}}%
\pgfpathlineto{\pgfqpoint{0.974600in}{0.678711in}}%
\pgfpathlineto{\pgfqpoint{0.979754in}{0.678169in}}%
\pgfpathlineto{\pgfqpoint{1.008104in}{0.677418in}}%
\pgfpathlineto{\pgfqpoint{1.023567in}{0.677060in}}%
\pgfpathlineto{\pgfqpoint{1.033876in}{0.679369in}}%
\pgfpathlineto{\pgfqpoint{1.044185in}{0.680509in}}%
\pgfpathlineto{\pgfqpoint{1.069957in}{0.685428in}}%
\pgfpathlineto{\pgfqpoint{1.085420in}{0.687203in}}%
\pgfpathlineto{\pgfqpoint{1.103461in}{0.687789in}}%
\pgfpathlineto{\pgfqpoint{1.121501in}{0.688944in}}%
\pgfpathlineto{\pgfqpoint{1.136964in}{0.688271in}}%
\pgfpathlineto{\pgfqpoint{1.152428in}{0.689859in}}%
\pgfpathlineto{\pgfqpoint{1.157582in}{0.691364in}}%
\pgfpathlineto{\pgfqpoint{1.173045in}{0.690795in}}%
\pgfpathlineto{\pgfqpoint{1.178200in}{0.690419in}}%
\pgfpathlineto{\pgfqpoint{1.214281in}{0.689728in}}%
\pgfpathlineto{\pgfqpoint{1.232321in}{0.689910in}}%
\pgfpathlineto{\pgfqpoint{1.247785in}{0.690351in}}%
\pgfpathlineto{\pgfqpoint{1.268402in}{0.689318in}}%
\pgfpathlineto{\pgfqpoint{1.301906in}{0.688366in}}%
\pgfpathlineto{\pgfqpoint{1.314792in}{0.687818in}}%
\pgfpathlineto{\pgfqpoint{1.335410in}{0.687670in}}%
\pgfpathlineto{\pgfqpoint{1.340564in}{0.688441in}}%
\pgfpathlineto{\pgfqpoint{1.376645in}{0.689271in}}%
\pgfpathlineto{\pgfqpoint{1.404995in}{0.688564in}}%
\pgfpathlineto{\pgfqpoint{1.430767in}{0.687470in}}%
\pgfpathlineto{\pgfqpoint{1.448807in}{0.687446in}}%
\pgfpathlineto{\pgfqpoint{1.461693in}{0.688065in}}%
\pgfpathlineto{\pgfqpoint{1.466848in}{0.688736in}}%
\pgfpathlineto{\pgfqpoint{1.482311in}{0.689542in}}%
\pgfpathlineto{\pgfqpoint{1.495197in}{0.692087in}}%
\pgfpathlineto{\pgfqpoint{1.502929in}{0.694862in}}%
\pgfpathlineto{\pgfqpoint{1.513237in}{0.696493in}}%
\pgfpathlineto{\pgfqpoint{1.520969in}{0.699066in}}%
\pgfpathlineto{\pgfqpoint{1.528701in}{0.700098in}}%
\pgfpathlineto{\pgfqpoint{1.536432in}{0.703049in}}%
\pgfpathlineto{\pgfqpoint{1.539010in}{0.704018in}}%
\pgfpathlineto{\pgfqpoint{1.549318in}{0.705679in}}%
\pgfpathlineto{\pgfqpoint{1.557050in}{0.708376in}}%
\pgfpathlineto{\pgfqpoint{1.564782in}{0.709241in}}%
\pgfpathlineto{\pgfqpoint{1.575091in}{0.714406in}}%
\pgfpathlineto{\pgfqpoint{1.585399in}{0.716305in}}%
\pgfpathlineto{\pgfqpoint{1.593131in}{0.719722in}}%
\pgfpathlineto{\pgfqpoint{1.600863in}{0.720851in}}%
\pgfpathlineto{\pgfqpoint{1.611172in}{0.724529in}}%
\pgfpathlineto{\pgfqpoint{1.626635in}{0.725936in}}%
\pgfpathlineto{\pgfqpoint{1.629212in}{0.726548in}}%
\pgfpathlineto{\pgfqpoint{1.639521in}{0.727769in}}%
\pgfpathlineto{\pgfqpoint{1.647253in}{0.730846in}}%
\pgfpathlineto{\pgfqpoint{1.654984in}{0.731845in}}%
\pgfpathlineto{\pgfqpoint{1.665293in}{0.736961in}}%
\pgfpathlineto{\pgfqpoint{1.673025in}{0.738062in}}%
\pgfpathlineto{\pgfqpoint{1.678179in}{0.741079in}}%
\pgfpathlineto{\pgfqpoint{1.683333in}{0.742352in}}%
\pgfpathlineto{\pgfqpoint{1.691065in}{0.743493in}}%
\pgfpathlineto{\pgfqpoint{1.701374in}{0.747944in}}%
\pgfpathlineto{\pgfqpoint{1.711683in}{0.750003in}}%
\pgfpathlineto{\pgfqpoint{1.719414in}{0.752482in}}%
\pgfpathlineto{\pgfqpoint{1.732301in}{0.754283in}}%
\pgfpathlineto{\pgfqpoint{1.737455in}{0.755369in}}%
\pgfpathlineto{\pgfqpoint{1.752918in}{0.756856in}}%
\pgfpathlineto{\pgfqpoint{1.755495in}{0.757249in}}%
\pgfpathlineto{\pgfqpoint{1.770959in}{0.758537in}}%
\pgfpathlineto{\pgfqpoint{1.773536in}{0.759073in}}%
\pgfpathlineto{\pgfqpoint{1.809617in}{0.760598in}}%
\pgfpathlineto{\pgfqpoint{1.840543in}{0.760987in}}%
\pgfpathlineto{\pgfqpoint{1.853430in}{0.761586in}}%
\pgfpathlineto{\pgfqpoint{1.889510in}{0.762742in}}%
\pgfpathlineto{\pgfqpoint{1.964250in}{0.763954in}}%
\pgfpathlineto{\pgfqpoint{1.990022in}{0.763881in}}%
\pgfpathlineto{\pgfqpoint{2.038989in}{0.763439in}}%
\pgfpathlineto{\pgfqpoint{2.129191in}{0.759755in}}%
\pgfpathlineto{\pgfqpoint{2.152386in}{0.758658in}}%
\pgfpathlineto{\pgfqpoint{2.196199in}{0.757522in}}%
\pgfpathlineto{\pgfqpoint{2.242589in}{0.755885in}}%
\pgfpathlineto{\pgfqpoint{2.276092in}{0.754717in}}%
\pgfpathlineto{\pgfqpoint{2.296710in}{0.753888in}}%
\pgfpathlineto{\pgfqpoint{2.325059in}{0.752935in}}%
\pgfpathlineto{\pgfqpoint{2.350832in}{0.751880in}}%
\pgfpathlineto{\pgfqpoint{2.397221in}{0.750569in}}%
\pgfpathlineto{\pgfqpoint{2.441034in}{0.749209in}}%
\pgfpathlineto{\pgfqpoint{2.474538in}{0.748160in}}%
\pgfpathlineto{\pgfqpoint{2.513196in}{0.746893in}}%
\pgfpathlineto{\pgfqpoint{2.564740in}{0.745574in}}%
\pgfpathlineto{\pgfqpoint{2.603398in}{0.744809in}}%
\pgfpathlineto{\pgfqpoint{2.649788in}{0.744043in}}%
\pgfpathlineto{\pgfqpoint{2.675560in}{0.743639in}}%
\pgfpathlineto{\pgfqpoint{2.719373in}{0.743821in}}%
\pgfpathlineto{\pgfqpoint{2.742568in}{0.744180in}}%
\pgfpathlineto{\pgfqpoint{2.763185in}{0.744817in}}%
\pgfpathlineto{\pgfqpoint{2.783803in}{0.744890in}}%
\pgfpathlineto{\pgfqpoint{2.863697in}{0.743896in}}%
\pgfpathlineto{\pgfqpoint{2.935859in}{0.743433in}}%
\pgfpathlineto{\pgfqpoint{2.974517in}{0.744494in}}%
\pgfpathlineto{\pgfqpoint{2.982249in}{0.744992in}}%
\pgfpathlineto{\pgfqpoint{3.008021in}{0.746023in}}%
\pgfpathlineto{\pgfqpoint{3.018330in}{0.746691in}}%
\pgfpathlineto{\pgfqpoint{3.046679in}{0.747597in}}%
\pgfpathlineto{\pgfqpoint{3.072451in}{0.748582in}}%
\pgfpathlineto{\pgfqpoint{3.103378in}{0.749462in}}%
\pgfpathlineto{\pgfqpoint{3.142036in}{0.753001in}}%
\pgfpathlineto{\pgfqpoint{3.144613in}{0.753771in}}%
\pgfpathlineto{\pgfqpoint{3.154922in}{0.755336in}}%
\pgfpathlineto{\pgfqpoint{3.162653in}{0.757673in}}%
\pgfpathlineto{\pgfqpoint{3.170385in}{0.758471in}}%
\pgfpathlineto{\pgfqpoint{3.180694in}{0.762111in}}%
\pgfpathlineto{\pgfqpoint{3.188426in}{0.763022in}}%
\pgfpathlineto{\pgfqpoint{3.196157in}{0.765784in}}%
\pgfpathlineto{\pgfqpoint{3.206466in}{0.766677in}}%
\pgfpathlineto{\pgfqpoint{3.216775in}{0.770152in}}%
\pgfpathlineto{\pgfqpoint{3.224507in}{0.771104in}}%
\pgfpathlineto{\pgfqpoint{3.227084in}{0.772099in}}%
\pgfpathlineto{\pgfqpoint{3.232238in}{0.775524in}}%
\pgfpathlineto{\pgfqpoint{3.234815in}{0.777031in}}%
\pgfpathlineto{\pgfqpoint{3.242547in}{0.778627in}}%
\pgfpathlineto{\pgfqpoint{3.252856in}{0.785215in}}%
\pgfpathlineto{\pgfqpoint{3.260588in}{0.786721in}}%
\pgfpathlineto{\pgfqpoint{3.270896in}{0.792300in}}%
\pgfpathlineto{\pgfqpoint{3.278628in}{0.793643in}}%
\pgfpathlineto{\pgfqpoint{3.288937in}{0.797703in}}%
\pgfpathlineto{\pgfqpoint{3.296668in}{0.798752in}}%
\pgfpathlineto{\pgfqpoint{3.304400in}{0.802358in}}%
\pgfpathlineto{\pgfqpoint{3.306977in}{0.803641in}}%
\pgfpathlineto{\pgfqpoint{3.314709in}{0.804970in}}%
\pgfpathlineto{\pgfqpoint{3.325018in}{0.810444in}}%
\pgfpathlineto{\pgfqpoint{3.332749in}{0.811783in}}%
\pgfpathlineto{\pgfqpoint{3.343058in}{0.816937in}}%
\pgfpathlineto{\pgfqpoint{3.353367in}{0.818127in}}%
\pgfpathlineto{\pgfqpoint{3.361099in}{0.821797in}}%
\pgfpathlineto{\pgfqpoint{3.368830in}{0.823111in}}%
\pgfpathlineto{\pgfqpoint{3.379139in}{0.827781in}}%
\pgfpathlineto{\pgfqpoint{3.386871in}{0.828832in}}%
\pgfpathlineto{\pgfqpoint{3.397180in}{0.833300in}}%
\pgfpathlineto{\pgfqpoint{3.404911in}{0.834325in}}%
\pgfpathlineto{\pgfqpoint{3.415220in}{0.838049in}}%
\pgfpathlineto{\pgfqpoint{3.422952in}{0.839060in}}%
\pgfpathlineto{\pgfqpoint{3.430684in}{0.841556in}}%
\pgfpathlineto{\pgfqpoint{3.433261in}{0.842354in}}%
\pgfpathlineto{\pgfqpoint{3.443570in}{0.843837in}}%
\pgfpathlineto{\pgfqpoint{3.451301in}{0.845754in}}%
\pgfpathlineto{\pgfqpoint{3.464187in}{0.846842in}}%
\pgfpathlineto{\pgfqpoint{3.484805in}{0.849175in}}%
\pgfpathlineto{\pgfqpoint{3.487382in}{0.849739in}}%
\pgfpathlineto{\pgfqpoint{3.497691in}{0.850949in}}%
\pgfpathlineto{\pgfqpoint{3.505423in}{0.852750in}}%
\pgfpathlineto{\pgfqpoint{3.515732in}{0.854228in}}%
\pgfpathlineto{\pgfqpoint{3.523463in}{0.856146in}}%
\pgfpathlineto{\pgfqpoint{3.533772in}{0.857255in}}%
\pgfpathlineto{\pgfqpoint{3.541504in}{0.859065in}}%
\pgfpathlineto{\pgfqpoint{3.551813in}{0.860515in}}%
\pgfpathlineto{\pgfqpoint{3.577585in}{0.867446in}}%
\pgfpathlineto{\pgfqpoint{3.585316in}{0.868606in}}%
\pgfpathlineto{\pgfqpoint{3.595625in}{0.873425in}}%
\pgfpathlineto{\pgfqpoint{3.603357in}{0.874527in}}%
\pgfpathlineto{\pgfqpoint{3.613666in}{0.878339in}}%
\pgfpathlineto{\pgfqpoint{3.623974in}{0.879900in}}%
\pgfpathlineto{\pgfqpoint{3.631706in}{0.882608in}}%
\pgfpathlineto{\pgfqpoint{3.639438in}{0.883614in}}%
\pgfpathlineto{\pgfqpoint{3.644592in}{0.885684in}}%
\pgfpathlineto{\pgfqpoint{3.649747in}{0.886724in}}%
\pgfpathlineto{\pgfqpoint{3.660055in}{0.888564in}}%
\pgfpathlineto{\pgfqpoint{3.662633in}{0.889352in}}%
\pgfpathlineto{\pgfqpoint{3.683250in}{0.893011in}}%
\pgfpathlineto{\pgfqpoint{3.685828in}{0.893837in}}%
\pgfpathlineto{\pgfqpoint{3.696136in}{0.895400in}}%
\pgfpathlineto{\pgfqpoint{3.703868in}{0.897584in}}%
\pgfpathlineto{\pgfqpoint{3.714177in}{0.898263in}}%
\pgfpathlineto{\pgfqpoint{3.721909in}{0.900508in}}%
\pgfpathlineto{\pgfqpoint{3.752835in}{0.903448in}}%
\pgfpathlineto{\pgfqpoint{3.757990in}{0.904111in}}%
\pgfpathlineto{\pgfqpoint{3.770876in}{0.905006in}}%
\pgfpathlineto{\pgfqpoint{3.776030in}{0.905792in}}%
\pgfpathlineto{\pgfqpoint{3.791493in}{0.907007in}}%
\pgfpathlineto{\pgfqpoint{3.794070in}{0.907405in}}%
\pgfpathlineto{\pgfqpoint{3.806957in}{0.908453in}}%
\pgfpathlineto{\pgfqpoint{3.812111in}{0.909006in}}%
\pgfpathlineto{\pgfqpoint{3.824997in}{0.910033in}}%
\pgfpathlineto{\pgfqpoint{3.830151in}{0.910565in}}%
\pgfpathlineto{\pgfqpoint{3.879118in}{0.911233in}}%
\pgfpathlineto{\pgfqpoint{3.899736in}{0.911471in}}%
\pgfpathlineto{\pgfqpoint{3.933240in}{0.912065in}}%
\pgfpathlineto{\pgfqpoint{3.953858in}{0.912950in}}%
\pgfpathlineto{\pgfqpoint{4.026020in}{0.916312in}}%
\pgfpathlineto{\pgfqpoint{4.028597in}{0.916529in}}%
\pgfpathlineto{\pgfqpoint{4.044060in}{0.917219in}}%
\pgfpathlineto{\pgfqpoint{4.046637in}{0.917531in}}%
\pgfpathlineto{\pgfqpoint{4.100759in}{0.918848in}}%
\pgfpathlineto{\pgfqpoint{4.147149in}{0.918919in}}%
\pgfpathlineto{\pgfqpoint{4.317245in}{0.914412in}}%
\pgfpathlineto{\pgfqpoint{4.366212in}{0.913422in}}%
\pgfpathlineto{\pgfqpoint{4.384252in}{0.913535in}}%
\pgfpathlineto{\pgfqpoint{4.451260in}{0.916828in}}%
\pgfpathlineto{\pgfqpoint{4.461569in}{0.917749in}}%
\pgfpathlineto{\pgfqpoint{4.495072in}{0.918942in}}%
\pgfpathlineto{\pgfqpoint{4.497649in}{0.919218in}}%
\pgfpathlineto{\pgfqpoint{4.523422in}{0.920498in}}%
\pgfpathlineto{\pgfqpoint{4.533730in}{0.921580in}}%
\pgfpathlineto{\pgfqpoint{4.549194in}{0.922670in}}%
\pgfpathlineto{\pgfqpoint{4.569811in}{0.924176in}}%
\pgfpathlineto{\pgfqpoint{4.582697in}{0.924903in}}%
\pgfpathlineto{\pgfqpoint{4.585275in}{0.925172in}}%
\pgfpathlineto{\pgfqpoint{4.600738in}{0.926006in}}%
\pgfpathlineto{\pgfqpoint{4.603315in}{0.926222in}}%
\pgfpathlineto{\pgfqpoint{4.657437in}{0.926606in}}%
\pgfpathlineto{\pgfqpoint{4.750216in}{0.924619in}}%
\pgfpathlineto{\pgfqpoint{4.804338in}{0.924550in}}%
\pgfpathlineto{\pgfqpoint{4.930621in}{0.925049in}}%
\pgfpathlineto{\pgfqpoint{5.028555in}{0.924786in}}%
\pgfpathlineto{\pgfqpoint{5.108449in}{0.926598in}}%
\pgfpathlineto{\pgfqpoint{5.129067in}{0.927950in}}%
\pgfpathlineto{\pgfqpoint{5.144530in}{0.928780in}}%
\pgfpathlineto{\pgfqpoint{5.165148in}{0.929980in}}%
\pgfpathlineto{\pgfqpoint{5.190920in}{0.931246in}}%
\pgfpathlineto{\pgfqpoint{5.219269in}{0.933364in}}%
\pgfpathlineto{\pgfqpoint{5.232155in}{0.934177in}}%
\pgfpathlineto{\pgfqpoint{5.237309in}{0.934769in}}%
\pgfpathlineto{\pgfqpoint{5.252773in}{0.935760in}}%
\pgfpathlineto{\pgfqpoint{5.322357in}{0.942590in}}%
\pgfpathlineto{\pgfqpoint{5.327512in}{0.943469in}}%
\pgfpathlineto{\pgfqpoint{5.340398in}{0.944599in}}%
\pgfpathlineto{\pgfqpoint{5.345552in}{0.945294in}}%
\pgfpathlineto{\pgfqpoint{5.515648in}{0.953451in}}%
\pgfpathlineto{\pgfqpoint{5.525957in}{0.954638in}}%
\pgfpathlineto{\pgfqpoint{5.554307in}{0.955942in}}%
\pgfpathlineto{\pgfqpoint{5.580079in}{0.957939in}}%
\pgfpathlineto{\pgfqpoint{5.595542in}{0.958692in}}%
\pgfpathlineto{\pgfqpoint{5.616160in}{0.960513in}}%
\pgfpathlineto{\pgfqpoint{5.641932in}{0.961981in}}%
\pgfpathlineto{\pgfqpoint{5.652241in}{0.962752in}}%
\pgfpathlineto{\pgfqpoint{5.667704in}{0.963552in}}%
\pgfpathlineto{\pgfqpoint{5.670281in}{0.963781in}}%
\pgfpathlineto{\pgfqpoint{5.696053in}{0.964865in}}%
\pgfpathlineto{\pgfqpoint{5.716671in}{0.965935in}}%
\pgfpathlineto{\pgfqpoint{5.742443in}{0.967070in}}%
\pgfpathlineto{\pgfqpoint{5.773370in}{0.968136in}}%
\pgfpathlineto{\pgfqpoint{5.812028in}{0.969914in}}%
\pgfpathlineto{\pgfqpoint{5.832646in}{0.970701in}}%
\pgfpathlineto{\pgfqpoint{5.848109in}{0.971601in}}%
\pgfpathlineto{\pgfqpoint{5.886767in}{0.973691in}}%
\pgfpathlineto{\pgfqpoint{5.917694in}{0.974792in}}%
\pgfpathlineto{\pgfqpoint{5.935734in}{0.975473in}}%
\pgfpathlineto{\pgfqpoint{5.958929in}{0.976634in}}%
\pgfpathlineto{\pgfqpoint{6.113562in}{0.979861in}}%
\pgfpathlineto{\pgfqpoint{6.134179in}{0.980768in}}%
\pgfpathlineto{\pgfqpoint{6.165106in}{0.981359in}}%
\pgfpathlineto{\pgfqpoint{6.224382in}{0.983424in}}%
\pgfpathlineto{\pgfqpoint{6.247577in}{0.984953in}}%
\pgfpathlineto{\pgfqpoint{6.260463in}{0.985972in}}%
\pgfpathlineto{\pgfqpoint{6.265617in}{0.986694in}}%
\pgfpathlineto{\pgfqpoint{6.278503in}{0.987772in}}%
\pgfpathlineto{\pgfqpoint{6.283658in}{0.988456in}}%
\pgfpathlineto{\pgfqpoint{6.296544in}{0.989589in}}%
\pgfpathlineto{\pgfqpoint{6.301698in}{0.990390in}}%
\pgfpathlineto{\pgfqpoint{6.312007in}{0.991287in}}%
\pgfpathlineto{\pgfqpoint{6.330048in}{0.994639in}}%
\pgfpathlineto{\pgfqpoint{6.337779in}{0.997696in}}%
\pgfpathlineto{\pgfqpoint{6.345511in}{0.998734in}}%
\pgfpathlineto{\pgfqpoint{6.355820in}{1.002660in}}%
\pgfpathlineto{\pgfqpoint{6.363551in}{1.003560in}}%
\pgfpathlineto{\pgfqpoint{6.373860in}{1.006962in}}%
\pgfpathlineto{\pgfqpoint{6.384169in}{1.008505in}}%
\pgfpathlineto{\pgfqpoint{6.386746in}{1.009257in}}%
\pgfpathlineto{\pgfqpoint{6.404787in}{1.012159in}}%
\pgfpathlineto{\pgfqpoint{6.409941in}{1.013646in}}%
\pgfpathlineto{\pgfqpoint{6.420250in}{1.014962in}}%
\pgfpathlineto{\pgfqpoint{6.427982in}{1.016723in}}%
\pgfpathlineto{\pgfqpoint{6.438290in}{1.017921in}}%
\pgfpathlineto{\pgfqpoint{6.446022in}{1.019763in}}%
\pgfpathlineto{\pgfqpoint{6.453754in}{1.020623in}}%
\pgfpathlineto{\pgfqpoint{6.464063in}{1.024346in}}%
\pgfpathlineto{\pgfqpoint{6.474371in}{1.025164in}}%
\pgfpathlineto{\pgfqpoint{6.482103in}{1.027621in}}%
\pgfpathlineto{\pgfqpoint{6.482103in}{1.027621in}}%
\pgfusepath{stroke}%
\end{pgfscope}%
\begin{pgfscope}%
\pgfpathrectangle{\pgfqpoint{0.563921in}{0.521603in}}{\pgfqpoint{6.200000in}{2.642500in}}%
\pgfusepath{clip}%
\pgfsetroundcap%
\pgfsetroundjoin%
\pgfsetlinewidth{1.505625pt}%
\definecolor{currentstroke}{rgb}{0.580392,0.403922,0.741176}%
\pgfsetstrokecolor{currentstroke}%
\pgfsetdash{}{0pt}%
\pgfpathmoveto{\pgfqpoint{0.845739in}{0.641717in}}%
\pgfpathlineto{\pgfqpoint{0.848317in}{0.651251in}}%
\pgfpathlineto{\pgfqpoint{0.850894in}{0.651656in}}%
\pgfpathlineto{\pgfqpoint{0.853471in}{0.659581in}}%
\pgfpathlineto{\pgfqpoint{0.861203in}{0.660025in}}%
\pgfpathlineto{\pgfqpoint{0.866357in}{0.657500in}}%
\pgfpathlineto{\pgfqpoint{0.871512in}{0.655647in}}%
\pgfpathlineto{\pgfqpoint{0.884398in}{0.654476in}}%
\pgfpathlineto{\pgfqpoint{0.889552in}{0.653479in}}%
\pgfpathlineto{\pgfqpoint{0.902438in}{0.652975in}}%
\pgfpathlineto{\pgfqpoint{0.905015in}{0.653947in}}%
\pgfpathlineto{\pgfqpoint{0.915324in}{0.654728in}}%
\pgfpathlineto{\pgfqpoint{0.917901in}{0.655999in}}%
\pgfpathlineto{\pgfqpoint{0.925633in}{0.656042in}}%
\pgfpathlineto{\pgfqpoint{0.941096in}{0.655691in}}%
\pgfpathlineto{\pgfqpoint{0.943674in}{0.656902in}}%
\pgfpathlineto{\pgfqpoint{0.953982in}{0.658355in}}%
\pgfpathlineto{\pgfqpoint{0.956560in}{0.658885in}}%
\pgfpathlineto{\pgfqpoint{1.023567in}{0.657710in}}%
\pgfpathlineto{\pgfqpoint{1.028722in}{0.658256in}}%
\pgfpathlineto{\pgfqpoint{1.062225in}{0.658674in}}%
\pgfpathlineto{\pgfqpoint{1.080266in}{0.663740in}}%
\pgfpathlineto{\pgfqpoint{1.085420in}{0.663870in}}%
\pgfpathlineto{\pgfqpoint{1.100883in}{0.664058in}}%
\pgfpathlineto{\pgfqpoint{1.103461in}{0.664220in}}%
\pgfpathlineto{\pgfqpoint{1.106038in}{0.665201in}}%
\pgfpathlineto{\pgfqpoint{1.116347in}{0.665526in}}%
\pgfpathlineto{\pgfqpoint{1.124078in}{0.668825in}}%
\pgfpathlineto{\pgfqpoint{1.136964in}{0.669767in}}%
\pgfpathlineto{\pgfqpoint{1.149851in}{0.669355in}}%
\pgfpathlineto{\pgfqpoint{1.191086in}{0.669482in}}%
\pgfpathlineto{\pgfqpoint{1.196240in}{0.670465in}}%
\pgfpathlineto{\pgfqpoint{1.206549in}{0.671243in}}%
\pgfpathlineto{\pgfqpoint{1.214281in}{0.672654in}}%
\pgfpathlineto{\pgfqpoint{1.224590in}{0.673197in}}%
\pgfpathlineto{\pgfqpoint{1.232321in}{0.676148in}}%
\pgfpathlineto{\pgfqpoint{1.265825in}{0.679121in}}%
\pgfpathlineto{\pgfqpoint{1.268402in}{0.680240in}}%
\pgfpathlineto{\pgfqpoint{1.276134in}{0.681634in}}%
\pgfpathlineto{\pgfqpoint{1.283866in}{0.686715in}}%
\pgfpathlineto{\pgfqpoint{1.286443in}{0.688085in}}%
\pgfpathlineto{\pgfqpoint{1.294174in}{0.689227in}}%
\pgfpathlineto{\pgfqpoint{1.299329in}{0.691601in}}%
\pgfpathlineto{\pgfqpoint{1.304483in}{0.695012in}}%
\pgfpathlineto{\pgfqpoint{1.312215in}{0.697443in}}%
\pgfpathlineto{\pgfqpoint{1.314792in}{0.699760in}}%
\pgfpathlineto{\pgfqpoint{1.319947in}{0.701605in}}%
\pgfpathlineto{\pgfqpoint{1.322524in}{0.703195in}}%
\pgfpathlineto{\pgfqpoint{1.330255in}{0.704831in}}%
\pgfpathlineto{\pgfqpoint{1.340564in}{0.711491in}}%
\pgfpathlineto{\pgfqpoint{1.348296in}{0.713328in}}%
\pgfpathlineto{\pgfqpoint{1.358605in}{0.722348in}}%
\pgfpathlineto{\pgfqpoint{1.371491in}{0.724542in}}%
\pgfpathlineto{\pgfqpoint{1.376645in}{0.728018in}}%
\pgfpathlineto{\pgfqpoint{1.384377in}{0.729890in}}%
\pgfpathlineto{\pgfqpoint{1.392108in}{0.734019in}}%
\pgfpathlineto{\pgfqpoint{1.394686in}{0.735337in}}%
\pgfpathlineto{\pgfqpoint{1.404995in}{0.737086in}}%
\pgfpathlineto{\pgfqpoint{1.412726in}{0.739235in}}%
\pgfpathlineto{\pgfqpoint{1.423035in}{0.740654in}}%
\pgfpathlineto{\pgfqpoint{1.430767in}{0.741957in}}%
\pgfpathlineto{\pgfqpoint{1.446230in}{0.742842in}}%
\pgfpathlineto{\pgfqpoint{1.448807in}{0.743235in}}%
\pgfpathlineto{\pgfqpoint{1.464270in}{0.744332in}}%
\pgfpathlineto{\pgfqpoint{1.466848in}{0.744593in}}%
\pgfpathlineto{\pgfqpoint{1.482311in}{0.745363in}}%
\pgfpathlineto{\pgfqpoint{1.484888in}{0.745765in}}%
\pgfpathlineto{\pgfqpoint{1.495197in}{0.746804in}}%
\pgfpathlineto{\pgfqpoint{1.567359in}{0.759547in}}%
\pgfpathlineto{\pgfqpoint{1.585399in}{0.760932in}}%
\pgfpathlineto{\pgfqpoint{1.587977in}{0.762311in}}%
\pgfpathlineto{\pgfqpoint{1.593131in}{0.766396in}}%
\pgfpathlineto{\pgfqpoint{1.603440in}{0.769183in}}%
\pgfpathlineto{\pgfqpoint{1.611172in}{0.772402in}}%
\pgfpathlineto{\pgfqpoint{1.624058in}{0.773356in}}%
\pgfpathlineto{\pgfqpoint{1.629212in}{0.775556in}}%
\pgfpathlineto{\pgfqpoint{1.636944in}{0.776414in}}%
\pgfpathlineto{\pgfqpoint{1.642098in}{0.777969in}}%
\pgfpathlineto{\pgfqpoint{1.647253in}{0.778749in}}%
\pgfpathlineto{\pgfqpoint{1.678179in}{0.780611in}}%
\pgfpathlineto{\pgfqpoint{1.716837in}{0.784437in}}%
\pgfpathlineto{\pgfqpoint{1.719414in}{0.785070in}}%
\pgfpathlineto{\pgfqpoint{1.727146in}{0.785745in}}%
\pgfpathlineto{\pgfqpoint{1.737455in}{0.788684in}}%
\pgfpathlineto{\pgfqpoint{1.747764in}{0.790095in}}%
\pgfpathlineto{\pgfqpoint{1.755495in}{0.791679in}}%
\pgfpathlineto{\pgfqpoint{1.781268in}{0.793225in}}%
\pgfpathlineto{\pgfqpoint{1.788999in}{0.794276in}}%
\pgfpathlineto{\pgfqpoint{1.791576in}{0.795043in}}%
\pgfpathlineto{\pgfqpoint{1.801885in}{0.796425in}}%
\pgfpathlineto{\pgfqpoint{1.809617in}{0.799117in}}%
\pgfpathlineto{\pgfqpoint{1.817349in}{0.800154in}}%
\pgfpathlineto{\pgfqpoint{1.827657in}{0.804391in}}%
\pgfpathlineto{\pgfqpoint{1.837966in}{0.805353in}}%
\pgfpathlineto{\pgfqpoint{1.845698in}{0.808817in}}%
\pgfpathlineto{\pgfqpoint{1.853430in}{0.810043in}}%
\pgfpathlineto{\pgfqpoint{1.861161in}{0.814184in}}%
\pgfpathlineto{\pgfqpoint{1.863738in}{0.815515in}}%
\pgfpathlineto{\pgfqpoint{1.871470in}{0.816785in}}%
\pgfpathlineto{\pgfqpoint{1.881779in}{0.823270in}}%
\pgfpathlineto{\pgfqpoint{1.889510in}{0.824906in}}%
\pgfpathlineto{\pgfqpoint{1.899819in}{0.831805in}}%
\pgfpathlineto{\pgfqpoint{1.910128in}{0.833917in}}%
\pgfpathlineto{\pgfqpoint{1.917860in}{0.839720in}}%
\pgfpathlineto{\pgfqpoint{1.925591in}{0.841297in}}%
\pgfpathlineto{\pgfqpoint{1.935900in}{0.848315in}}%
\pgfpathlineto{\pgfqpoint{1.943632in}{0.850383in}}%
\pgfpathlineto{\pgfqpoint{1.953941in}{0.859105in}}%
\pgfpathlineto{\pgfqpoint{1.961672in}{0.861460in}}%
\pgfpathlineto{\pgfqpoint{1.971981in}{0.871033in}}%
\pgfpathlineto{\pgfqpoint{1.979713in}{0.873262in}}%
\pgfpathlineto{\pgfqpoint{1.990022in}{0.882430in}}%
\pgfpathlineto{\pgfqpoint{1.997753in}{0.884761in}}%
\pgfpathlineto{\pgfqpoint{2.005485in}{0.893428in}}%
\pgfpathlineto{\pgfqpoint{2.015794in}{0.896500in}}%
\pgfpathlineto{\pgfqpoint{2.023526in}{0.905886in}}%
\pgfpathlineto{\pgfqpoint{2.026103in}{0.908703in}}%
\pgfpathlineto{\pgfqpoint{2.033834in}{0.911065in}}%
\pgfpathlineto{\pgfqpoint{2.044143in}{0.921655in}}%
\pgfpathlineto{\pgfqpoint{2.051875in}{0.923976in}}%
\pgfpathlineto{\pgfqpoint{2.062184in}{0.936011in}}%
\pgfpathlineto{\pgfqpoint{2.069915in}{0.939302in}}%
\pgfpathlineto{\pgfqpoint{2.077647in}{0.948967in}}%
\pgfpathlineto{\pgfqpoint{2.080224in}{0.952078in}}%
\pgfpathlineto{\pgfqpoint{2.087956in}{0.955311in}}%
\pgfpathlineto{\pgfqpoint{2.098265in}{0.966780in}}%
\pgfpathlineto{\pgfqpoint{2.105996in}{0.969335in}}%
\pgfpathlineto{\pgfqpoint{2.116305in}{0.980200in}}%
\pgfpathlineto{\pgfqpoint{2.124037in}{0.982916in}}%
\pgfpathlineto{\pgfqpoint{2.134346in}{0.995721in}}%
\pgfpathlineto{\pgfqpoint{2.142077in}{0.998936in}}%
\pgfpathlineto{\pgfqpoint{2.149809in}{1.008511in}}%
\pgfpathlineto{\pgfqpoint{2.152386in}{1.011266in}}%
\pgfpathlineto{\pgfqpoint{2.162695in}{1.014252in}}%
\pgfpathlineto{\pgfqpoint{2.170427in}{1.020536in}}%
\pgfpathlineto{\pgfqpoint{2.178158in}{1.022378in}}%
\pgfpathlineto{\pgfqpoint{2.188467in}{1.028962in}}%
\pgfpathlineto{\pgfqpoint{2.196199in}{1.030779in}}%
\pgfpathlineto{\pgfqpoint{2.206508in}{1.037111in}}%
\pgfpathlineto{\pgfqpoint{2.214239in}{1.038921in}}%
\pgfpathlineto{\pgfqpoint{2.219394in}{1.042451in}}%
\pgfpathlineto{\pgfqpoint{2.224548in}{1.044488in}}%
\pgfpathlineto{\pgfqpoint{2.232280in}{1.045895in}}%
\pgfpathlineto{\pgfqpoint{2.237434in}{1.049484in}}%
\pgfpathlineto{\pgfqpoint{2.242589in}{1.053007in}}%
\pgfpathlineto{\pgfqpoint{2.250320in}{1.054818in}}%
\pgfpathlineto{\pgfqpoint{2.255475in}{1.058378in}}%
\pgfpathlineto{\pgfqpoint{2.260629in}{1.060448in}}%
\pgfpathlineto{\pgfqpoint{2.268361in}{1.062697in}}%
\pgfpathlineto{\pgfqpoint{2.278670in}{1.072320in}}%
\pgfpathlineto{\pgfqpoint{2.286401in}{1.074912in}}%
\pgfpathlineto{\pgfqpoint{2.296710in}{1.085262in}}%
\pgfpathlineto{\pgfqpoint{2.304442in}{1.088212in}}%
\pgfpathlineto{\pgfqpoint{2.314751in}{1.099834in}}%
\pgfpathlineto{\pgfqpoint{2.322482in}{1.102821in}}%
\pgfpathlineto{\pgfqpoint{2.332791in}{1.114878in}}%
\pgfpathlineto{\pgfqpoint{2.340523in}{1.117789in}}%
\pgfpathlineto{\pgfqpoint{2.348254in}{1.126061in}}%
\pgfpathlineto{\pgfqpoint{2.350832in}{1.128375in}}%
\pgfpathlineto{\pgfqpoint{2.358563in}{1.130557in}}%
\pgfpathlineto{\pgfqpoint{2.363718in}{1.134756in}}%
\pgfpathlineto{\pgfqpoint{2.368872in}{1.137657in}}%
\pgfpathlineto{\pgfqpoint{2.376604in}{1.139303in}}%
\pgfpathlineto{\pgfqpoint{2.384335in}{1.143092in}}%
\pgfpathlineto{\pgfqpoint{2.386912in}{1.144348in}}%
\pgfpathlineto{\pgfqpoint{2.397221in}{1.146179in}}%
\pgfpathlineto{\pgfqpoint{2.404953in}{1.148624in}}%
\pgfpathlineto{\pgfqpoint{2.415262in}{1.149400in}}%
\pgfpathlineto{\pgfqpoint{2.422993in}{1.151997in}}%
\pgfpathlineto{\pgfqpoint{2.430725in}{1.152937in}}%
\pgfpathlineto{\pgfqpoint{2.441034in}{1.157557in}}%
\pgfpathlineto{\pgfqpoint{2.448766in}{1.158712in}}%
\pgfpathlineto{\pgfqpoint{2.459074in}{1.163694in}}%
\pgfpathlineto{\pgfqpoint{2.469383in}{1.165684in}}%
\pgfpathlineto{\pgfqpoint{2.477115in}{1.167660in}}%
\pgfpathlineto{\pgfqpoint{2.487424in}{1.168972in}}%
\pgfpathlineto{\pgfqpoint{2.495155in}{1.170860in}}%
\pgfpathlineto{\pgfqpoint{2.508041in}{1.172195in}}%
\pgfpathlineto{\pgfqpoint{2.513196in}{1.173891in}}%
\pgfpathlineto{\pgfqpoint{2.520928in}{1.174934in}}%
\pgfpathlineto{\pgfqpoint{2.528659in}{1.178678in}}%
\pgfpathlineto{\pgfqpoint{2.531236in}{1.180014in}}%
\pgfpathlineto{\pgfqpoint{2.538968in}{1.181247in}}%
\pgfpathlineto{\pgfqpoint{2.549277in}{1.186823in}}%
\pgfpathlineto{\pgfqpoint{2.557009in}{1.188206in}}%
\pgfpathlineto{\pgfqpoint{2.567317in}{1.194030in}}%
\pgfpathlineto{\pgfqpoint{2.575049in}{1.195450in}}%
\pgfpathlineto{\pgfqpoint{2.585358in}{1.201041in}}%
\pgfpathlineto{\pgfqpoint{2.593089in}{1.202623in}}%
\pgfpathlineto{\pgfqpoint{2.603398in}{1.208368in}}%
\pgfpathlineto{\pgfqpoint{2.611130in}{1.209856in}}%
\pgfpathlineto{\pgfqpoint{2.621439in}{1.216409in}}%
\pgfpathlineto{\pgfqpoint{2.629170in}{1.218206in}}%
\pgfpathlineto{\pgfqpoint{2.634325in}{1.221478in}}%
\pgfpathlineto{\pgfqpoint{2.639479in}{1.223009in}}%
\pgfpathlineto{\pgfqpoint{2.647211in}{1.224448in}}%
\pgfpathlineto{\pgfqpoint{2.657520in}{1.229633in}}%
\pgfpathlineto{\pgfqpoint{2.665251in}{1.231016in}}%
\pgfpathlineto{\pgfqpoint{2.670406in}{1.233436in}}%
\pgfpathlineto{\pgfqpoint{2.675560in}{1.234997in}}%
\pgfpathlineto{\pgfqpoint{2.685869in}{1.236446in}}%
\pgfpathlineto{\pgfqpoint{2.693601in}{1.239125in}}%
\pgfpathlineto{\pgfqpoint{2.703910in}{1.240796in}}%
\pgfpathlineto{\pgfqpoint{2.739991in}{1.247400in}}%
\pgfpathlineto{\pgfqpoint{2.747722in}{1.250796in}}%
\pgfpathlineto{\pgfqpoint{2.755454in}{1.251895in}}%
\pgfpathlineto{\pgfqpoint{2.765763in}{1.256363in}}%
\pgfpathlineto{\pgfqpoint{2.776072in}{1.257321in}}%
\pgfpathlineto{\pgfqpoint{2.781226in}{1.259052in}}%
\pgfpathlineto{\pgfqpoint{2.783803in}{1.259486in}}%
\pgfpathlineto{\pgfqpoint{2.827616in}{1.261711in}}%
\pgfpathlineto{\pgfqpoint{2.837925in}{1.264267in}}%
\pgfpathlineto{\pgfqpoint{2.850811in}{1.265291in}}%
\pgfpathlineto{\pgfqpoint{2.855965in}{1.266388in}}%
\pgfpathlineto{\pgfqpoint{2.866274in}{1.267324in}}%
\pgfpathlineto{\pgfqpoint{2.874006in}{1.268850in}}%
\pgfpathlineto{\pgfqpoint{2.881737in}{1.269349in}}%
\pgfpathlineto{\pgfqpoint{2.892046in}{1.272068in}}%
\pgfpathlineto{\pgfqpoint{2.902355in}{1.273501in}}%
\pgfpathlineto{\pgfqpoint{2.910087in}{1.275457in}}%
\pgfpathlineto{\pgfqpoint{2.920395in}{1.276943in}}%
\pgfpathlineto{\pgfqpoint{2.928127in}{1.279380in}}%
\pgfpathlineto{\pgfqpoint{2.935859in}{1.280266in}}%
\pgfpathlineto{\pgfqpoint{2.946168in}{1.284986in}}%
\pgfpathlineto{\pgfqpoint{2.953899in}{1.286287in}}%
\pgfpathlineto{\pgfqpoint{2.964208in}{1.291344in}}%
\pgfpathlineto{\pgfqpoint{2.971940in}{1.292524in}}%
\pgfpathlineto{\pgfqpoint{2.982249in}{1.296958in}}%
\pgfpathlineto{\pgfqpoint{2.989980in}{1.297965in}}%
\pgfpathlineto{\pgfqpoint{2.997712in}{1.301762in}}%
\pgfpathlineto{\pgfqpoint{3.008021in}{1.303170in}}%
\pgfpathlineto{\pgfqpoint{3.018330in}{1.308677in}}%
\pgfpathlineto{\pgfqpoint{3.026061in}{1.310235in}}%
\pgfpathlineto{\pgfqpoint{3.036370in}{1.315790in}}%
\pgfpathlineto{\pgfqpoint{3.044102in}{1.317056in}}%
\pgfpathlineto{\pgfqpoint{3.054411in}{1.322281in}}%
\pgfpathlineto{\pgfqpoint{3.062142in}{1.323563in}}%
\pgfpathlineto{\pgfqpoint{3.072451in}{1.328727in}}%
\pgfpathlineto{\pgfqpoint{3.080183in}{1.329962in}}%
\pgfpathlineto{\pgfqpoint{3.090491in}{1.335118in}}%
\pgfpathlineto{\pgfqpoint{3.100800in}{1.336425in}}%
\pgfpathlineto{\pgfqpoint{3.108532in}{1.340282in}}%
\pgfpathlineto{\pgfqpoint{3.116264in}{1.341750in}}%
\pgfpathlineto{\pgfqpoint{3.126572in}{1.347941in}}%
\pgfpathlineto{\pgfqpoint{3.134304in}{1.349504in}}%
\pgfpathlineto{\pgfqpoint{3.142036in}{1.354153in}}%
\pgfpathlineto{\pgfqpoint{3.144613in}{1.355546in}}%
\pgfpathlineto{\pgfqpoint{3.152345in}{1.356913in}}%
\pgfpathlineto{\pgfqpoint{3.160076in}{1.361126in}}%
\pgfpathlineto{\pgfqpoint{3.162653in}{1.362891in}}%
\pgfpathlineto{\pgfqpoint{3.170385in}{1.364552in}}%
\pgfpathlineto{\pgfqpoint{3.180694in}{1.371361in}}%
\pgfpathlineto{\pgfqpoint{3.188426in}{1.372919in}}%
\pgfpathlineto{\pgfqpoint{3.196157in}{1.378029in}}%
\pgfpathlineto{\pgfqpoint{3.206466in}{1.379824in}}%
\pgfpathlineto{\pgfqpoint{3.216775in}{1.386312in}}%
\pgfpathlineto{\pgfqpoint{3.224507in}{1.387852in}}%
\pgfpathlineto{\pgfqpoint{3.232238in}{1.390917in}}%
\pgfpathlineto{\pgfqpoint{3.234815in}{1.391895in}}%
\pgfpathlineto{\pgfqpoint{3.242547in}{1.392794in}}%
\pgfpathlineto{\pgfqpoint{3.252856in}{1.396813in}}%
\pgfpathlineto{\pgfqpoint{3.260588in}{1.397776in}}%
\pgfpathlineto{\pgfqpoint{3.270896in}{1.401028in}}%
\pgfpathlineto{\pgfqpoint{3.281205in}{1.402346in}}%
\pgfpathlineto{\pgfqpoint{3.288937in}{1.404481in}}%
\pgfpathlineto{\pgfqpoint{3.299246in}{1.405945in}}%
\pgfpathlineto{\pgfqpoint{3.306977in}{1.408359in}}%
\pgfpathlineto{\pgfqpoint{3.314709in}{1.409280in}}%
\pgfpathlineto{\pgfqpoint{3.325018in}{1.413304in}}%
\pgfpathlineto{\pgfqpoint{3.332749in}{1.414337in}}%
\pgfpathlineto{\pgfqpoint{3.343058in}{1.418432in}}%
\pgfpathlineto{\pgfqpoint{3.353367in}{1.419436in}}%
\pgfpathlineto{\pgfqpoint{3.361099in}{1.422645in}}%
\pgfpathlineto{\pgfqpoint{3.368830in}{1.423701in}}%
\pgfpathlineto{\pgfqpoint{3.379139in}{1.428060in}}%
\pgfpathlineto{\pgfqpoint{3.386871in}{1.429156in}}%
\pgfpathlineto{\pgfqpoint{3.397180in}{1.434595in}}%
\pgfpathlineto{\pgfqpoint{3.404911in}{1.436067in}}%
\pgfpathlineto{\pgfqpoint{3.415220in}{1.441664in}}%
\pgfpathlineto{\pgfqpoint{3.422952in}{1.442891in}}%
\pgfpathlineto{\pgfqpoint{3.428106in}{1.445034in}}%
\pgfpathlineto{\pgfqpoint{3.433261in}{1.446899in}}%
\pgfpathlineto{\pgfqpoint{3.443570in}{1.448532in}}%
\pgfpathlineto{\pgfqpoint{3.448724in}{1.450111in}}%
\pgfpathlineto{\pgfqpoint{3.451301in}{1.450637in}}%
\pgfpathlineto{\pgfqpoint{3.477073in}{1.451977in}}%
\pgfpathlineto{\pgfqpoint{3.484805in}{1.453516in}}%
\pgfpathlineto{\pgfqpoint{3.487382in}{1.454200in}}%
\pgfpathlineto{\pgfqpoint{3.495114in}{1.454982in}}%
\pgfpathlineto{\pgfqpoint{3.502845in}{1.457896in}}%
\pgfpathlineto{\pgfqpoint{3.505423in}{1.459095in}}%
\pgfpathlineto{\pgfqpoint{3.513154in}{1.460244in}}%
\pgfpathlineto{\pgfqpoint{3.523463in}{1.465356in}}%
\pgfpathlineto{\pgfqpoint{3.531195in}{1.466628in}}%
\pgfpathlineto{\pgfqpoint{3.541504in}{1.471561in}}%
\pgfpathlineto{\pgfqpoint{3.549235in}{1.472715in}}%
\pgfpathlineto{\pgfqpoint{3.559544in}{1.477378in}}%
\pgfpathlineto{\pgfqpoint{3.567276in}{1.478398in}}%
\pgfpathlineto{\pgfqpoint{3.572430in}{1.480430in}}%
\pgfpathlineto{\pgfqpoint{3.577585in}{1.481591in}}%
\pgfpathlineto{\pgfqpoint{3.585316in}{1.482715in}}%
\pgfpathlineto{\pgfqpoint{3.595625in}{1.487146in}}%
\pgfpathlineto{\pgfqpoint{3.603357in}{1.488286in}}%
\pgfpathlineto{\pgfqpoint{3.611088in}{1.491114in}}%
\pgfpathlineto{\pgfqpoint{3.613666in}{1.491759in}}%
\pgfpathlineto{\pgfqpoint{3.623974in}{1.492820in}}%
\pgfpathlineto{\pgfqpoint{3.644592in}{1.497095in}}%
\pgfpathlineto{\pgfqpoint{3.680673in}{1.501935in}}%
\pgfpathlineto{\pgfqpoint{3.685828in}{1.503290in}}%
\pgfpathlineto{\pgfqpoint{3.696136in}{1.504423in}}%
\pgfpathlineto{\pgfqpoint{3.703868in}{1.505763in}}%
\pgfpathlineto{\pgfqpoint{3.716754in}{1.506299in}}%
\pgfpathlineto{\pgfqpoint{3.721909in}{1.507064in}}%
\pgfpathlineto{\pgfqpoint{3.737372in}{1.508228in}}%
\pgfpathlineto{\pgfqpoint{3.776030in}{1.509988in}}%
\pgfpathlineto{\pgfqpoint{3.804379in}{1.510863in}}%
\pgfpathlineto{\pgfqpoint{3.837883in}{1.513345in}}%
\pgfpathlineto{\pgfqpoint{3.873964in}{1.514907in}}%
\pgfpathlineto{\pgfqpoint{3.884273in}{1.515479in}}%
\pgfpathlineto{\pgfqpoint{4.026020in}{1.519635in}}%
\pgfpathlineto{\pgfqpoint{4.041483in}{1.520071in}}%
\pgfpathlineto{\pgfqpoint{4.056946in}{1.520415in}}%
\pgfpathlineto{\pgfqpoint{4.074987in}{1.520333in}}%
\pgfpathlineto{\pgfqpoint{4.219311in}{1.519921in}}%
\pgfpathlineto{\pgfqpoint{4.239928in}{1.519722in}}%
\pgfpathlineto{\pgfqpoint{4.288895in}{1.517689in}}%
\pgfpathlineto{\pgfqpoint{4.299204in}{1.516410in}}%
\pgfpathlineto{\pgfqpoint{4.314667in}{1.515516in}}%
\pgfpathlineto{\pgfqpoint{4.317245in}{1.515216in}}%
\pgfpathlineto{\pgfqpoint{4.332708in}{1.514223in}}%
\pgfpathlineto{\pgfqpoint{4.353326in}{1.512391in}}%
\pgfpathlineto{\pgfqpoint{4.368789in}{1.511177in}}%
\pgfpathlineto{\pgfqpoint{4.381675in}{1.510406in}}%
\pgfpathlineto{\pgfqpoint{4.402293in}{1.509146in}}%
\pgfpathlineto{\pgfqpoint{4.422910in}{1.508651in}}%
\pgfpathlineto{\pgfqpoint{4.492495in}{1.510251in}}%
\pgfpathlineto{\pgfqpoint{4.515690in}{1.511305in}}%
\pgfpathlineto{\pgfqpoint{4.533730in}{1.512031in}}%
\pgfpathlineto{\pgfqpoint{4.562080in}{1.513324in}}%
\pgfpathlineto{\pgfqpoint{4.569811in}{1.514051in}}%
\pgfpathlineto{\pgfqpoint{4.585275in}{1.514793in}}%
\pgfpathlineto{\pgfqpoint{4.636819in}{1.515272in}}%
\pgfpathlineto{\pgfqpoint{4.678054in}{1.514767in}}%
\pgfpathlineto{\pgfqpoint{4.742485in}{1.517349in}}%
\pgfpathlineto{\pgfqpoint{4.819801in}{1.526835in}}%
\pgfpathlineto{\pgfqpoint{4.832687in}{1.527892in}}%
\pgfpathlineto{\pgfqpoint{4.840419in}{1.529478in}}%
\pgfpathlineto{\pgfqpoint{4.850728in}{1.530516in}}%
\pgfpathlineto{\pgfqpoint{4.858459in}{1.532165in}}%
\pgfpathlineto{\pgfqpoint{4.868768in}{1.533252in}}%
\pgfpathlineto{\pgfqpoint{4.876500in}{1.535018in}}%
\pgfpathlineto{\pgfqpoint{4.884231in}{1.535677in}}%
\pgfpathlineto{\pgfqpoint{4.894540in}{1.539071in}}%
\pgfpathlineto{\pgfqpoint{4.904849in}{1.540726in}}%
\pgfpathlineto{\pgfqpoint{4.912581in}{1.542979in}}%
\pgfpathlineto{\pgfqpoint{4.922890in}{1.544506in}}%
\pgfpathlineto{\pgfqpoint{4.930621in}{1.546740in}}%
\pgfpathlineto{\pgfqpoint{4.938353in}{1.547571in}}%
\pgfpathlineto{\pgfqpoint{4.948662in}{1.551016in}}%
\pgfpathlineto{\pgfqpoint{4.958971in}{1.552698in}}%
\pgfpathlineto{\pgfqpoint{4.966702in}{1.554974in}}%
\pgfpathlineto{\pgfqpoint{4.977011in}{1.556459in}}%
\pgfpathlineto{\pgfqpoint{4.984743in}{1.558827in}}%
\pgfpathlineto{\pgfqpoint{4.995051in}{1.559573in}}%
\pgfpathlineto{\pgfqpoint{5.002783in}{1.562123in}}%
\pgfpathlineto{\pgfqpoint{5.010515in}{1.563120in}}%
\pgfpathlineto{\pgfqpoint{5.020824in}{1.567277in}}%
\pgfpathlineto{\pgfqpoint{5.028555in}{1.568340in}}%
\pgfpathlineto{\pgfqpoint{5.038864in}{1.572396in}}%
\pgfpathlineto{\pgfqpoint{5.046596in}{1.573408in}}%
\pgfpathlineto{\pgfqpoint{5.056905in}{1.577356in}}%
\pgfpathlineto{\pgfqpoint{5.064636in}{1.578342in}}%
\pgfpathlineto{\pgfqpoint{5.072368in}{1.582131in}}%
\pgfpathlineto{\pgfqpoint{5.074945in}{1.583545in}}%
\pgfpathlineto{\pgfqpoint{5.085254in}{1.585051in}}%
\pgfpathlineto{\pgfqpoint{5.092986in}{1.589644in}}%
\pgfpathlineto{\pgfqpoint{5.100717in}{1.591186in}}%
\pgfpathlineto{\pgfqpoint{5.111026in}{1.597304in}}%
\pgfpathlineto{\pgfqpoint{5.118758in}{1.598824in}}%
\pgfpathlineto{\pgfqpoint{5.129067in}{1.605662in}}%
\pgfpathlineto{\pgfqpoint{5.136798in}{1.607321in}}%
\pgfpathlineto{\pgfqpoint{5.147107in}{1.613867in}}%
\pgfpathlineto{\pgfqpoint{5.154839in}{1.615533in}}%
\pgfpathlineto{\pgfqpoint{5.165148in}{1.621617in}}%
\pgfpathlineto{\pgfqpoint{5.172879in}{1.623060in}}%
\pgfpathlineto{\pgfqpoint{5.183188in}{1.628676in}}%
\pgfpathlineto{\pgfqpoint{5.190920in}{1.629952in}}%
\pgfpathlineto{\pgfqpoint{5.201228in}{1.634426in}}%
\pgfpathlineto{\pgfqpoint{5.208960in}{1.635454in}}%
\pgfpathlineto{\pgfqpoint{5.219269in}{1.639452in}}%
\pgfpathlineto{\pgfqpoint{5.227001in}{1.640525in}}%
\pgfpathlineto{\pgfqpoint{5.237309in}{1.644545in}}%
\pgfpathlineto{\pgfqpoint{5.247618in}{1.645575in}}%
\pgfpathlineto{\pgfqpoint{5.255350in}{1.648472in}}%
\pgfpathlineto{\pgfqpoint{5.265659in}{1.650259in}}%
\pgfpathlineto{\pgfqpoint{5.273390in}{1.652871in}}%
\pgfpathlineto{\pgfqpoint{5.281122in}{1.653686in}}%
\pgfpathlineto{\pgfqpoint{5.291431in}{1.657266in}}%
\pgfpathlineto{\pgfqpoint{5.299163in}{1.658069in}}%
\pgfpathlineto{\pgfqpoint{5.309471in}{1.661472in}}%
\pgfpathlineto{\pgfqpoint{5.317203in}{1.662336in}}%
\pgfpathlineto{\pgfqpoint{5.327512in}{1.665809in}}%
\pgfpathlineto{\pgfqpoint{5.337821in}{1.667479in}}%
\pgfpathlineto{\pgfqpoint{5.345552in}{1.669763in}}%
\pgfpathlineto{\pgfqpoint{5.358438in}{1.671632in}}%
\pgfpathlineto{\pgfqpoint{5.363593in}{1.672580in}}%
\pgfpathlineto{\pgfqpoint{5.373902in}{1.673469in}}%
\pgfpathlineto{\pgfqpoint{5.381633in}{1.675066in}}%
\pgfpathlineto{\pgfqpoint{5.391942in}{1.676181in}}%
\pgfpathlineto{\pgfqpoint{5.399674in}{1.677698in}}%
\pgfpathlineto{\pgfqpoint{5.409983in}{1.678953in}}%
\pgfpathlineto{\pgfqpoint{5.417714in}{1.681385in}}%
\pgfpathlineto{\pgfqpoint{5.428023in}{1.682551in}}%
\pgfpathlineto{\pgfqpoint{5.435755in}{1.684257in}}%
\pgfpathlineto{\pgfqpoint{5.461527in}{1.686403in}}%
\pgfpathlineto{\pgfqpoint{5.471836in}{1.687629in}}%
\pgfpathlineto{\pgfqpoint{5.487299in}{1.688767in}}%
\pgfpathlineto{\pgfqpoint{5.523380in}{1.693802in}}%
\pgfpathlineto{\pgfqpoint{5.525957in}{1.694336in}}%
\pgfpathlineto{\pgfqpoint{5.538843in}{1.695336in}}%
\pgfpathlineto{\pgfqpoint{5.543998in}{1.696306in}}%
\pgfpathlineto{\pgfqpoint{5.556884in}{1.697318in}}%
\pgfpathlineto{\pgfqpoint{5.562038in}{1.698430in}}%
\pgfpathlineto{\pgfqpoint{5.572347in}{1.699484in}}%
\pgfpathlineto{\pgfqpoint{5.580079in}{1.700744in}}%
\pgfpathlineto{\pgfqpoint{5.592965in}{1.701589in}}%
\pgfpathlineto{\pgfqpoint{5.598119in}{1.702352in}}%
\pgfpathlineto{\pgfqpoint{5.613582in}{1.703483in}}%
\pgfpathlineto{\pgfqpoint{5.616160in}{1.703807in}}%
\pgfpathlineto{\pgfqpoint{5.629046in}{1.704737in}}%
\pgfpathlineto{\pgfqpoint{5.634200in}{1.705395in}}%
\pgfpathlineto{\pgfqpoint{5.647086in}{1.706339in}}%
\pgfpathlineto{\pgfqpoint{5.685744in}{1.712017in}}%
\pgfpathlineto{\pgfqpoint{5.688322in}{1.712990in}}%
\pgfpathlineto{\pgfqpoint{5.696053in}{1.713932in}}%
\pgfpathlineto{\pgfqpoint{5.706362in}{1.717964in}}%
\pgfpathlineto{\pgfqpoint{5.714094in}{1.718987in}}%
\pgfpathlineto{\pgfqpoint{5.724403in}{1.723462in}}%
\pgfpathlineto{\pgfqpoint{5.732134in}{1.724705in}}%
\pgfpathlineto{\pgfqpoint{5.742443in}{1.730127in}}%
\pgfpathlineto{\pgfqpoint{5.750175in}{1.731461in}}%
\pgfpathlineto{\pgfqpoint{5.760484in}{1.736107in}}%
\pgfpathlineto{\pgfqpoint{5.768215in}{1.737220in}}%
\pgfpathlineto{\pgfqpoint{5.778524in}{1.741349in}}%
\pgfpathlineto{\pgfqpoint{5.786256in}{1.742349in}}%
\pgfpathlineto{\pgfqpoint{5.796565in}{1.746357in}}%
\pgfpathlineto{\pgfqpoint{5.804296in}{1.747308in}}%
\pgfpathlineto{\pgfqpoint{5.812028in}{1.750256in}}%
\pgfpathlineto{\pgfqpoint{5.824914in}{1.752040in}}%
\pgfpathlineto{\pgfqpoint{5.832646in}{1.754222in}}%
\pgfpathlineto{\pgfqpoint{5.840377in}{1.755029in}}%
\pgfpathlineto{\pgfqpoint{5.850686in}{1.758408in}}%
\pgfpathlineto{\pgfqpoint{5.858418in}{1.759228in}}%
\pgfpathlineto{\pgfqpoint{5.868727in}{1.762557in}}%
\pgfpathlineto{\pgfqpoint{5.879035in}{1.764143in}}%
\pgfpathlineto{\pgfqpoint{5.886767in}{1.766494in}}%
\pgfpathlineto{\pgfqpoint{5.894499in}{1.767550in}}%
\pgfpathlineto{\pgfqpoint{5.904807in}{1.771764in}}%
\pgfpathlineto{\pgfqpoint{5.912539in}{1.772818in}}%
\pgfpathlineto{\pgfqpoint{5.922848in}{1.777152in}}%
\pgfpathlineto{\pgfqpoint{5.933157in}{1.778236in}}%
\pgfpathlineto{\pgfqpoint{5.940888in}{1.781924in}}%
\pgfpathlineto{\pgfqpoint{5.948620in}{1.783252in}}%
\pgfpathlineto{\pgfqpoint{5.958929in}{1.788693in}}%
\pgfpathlineto{\pgfqpoint{5.966661in}{1.790115in}}%
\pgfpathlineto{\pgfqpoint{5.976969in}{1.796164in}}%
\pgfpathlineto{\pgfqpoint{5.984701in}{1.797750in}}%
\pgfpathlineto{\pgfqpoint{5.995010in}{1.804501in}}%
\pgfpathlineto{\pgfqpoint{6.002742in}{1.806254in}}%
\pgfpathlineto{\pgfqpoint{6.013050in}{1.812158in}}%
\pgfpathlineto{\pgfqpoint{6.025936in}{1.815038in}}%
\pgfpathlineto{\pgfqpoint{6.031091in}{1.817767in}}%
\pgfpathlineto{\pgfqpoint{6.038823in}{1.819041in}}%
\pgfpathlineto{\pgfqpoint{6.049131in}{1.824223in}}%
\pgfpathlineto{\pgfqpoint{6.056863in}{1.825518in}}%
\pgfpathlineto{\pgfqpoint{6.067172in}{1.831729in}}%
\pgfpathlineto{\pgfqpoint{6.074904in}{1.833063in}}%
\pgfpathlineto{\pgfqpoint{6.085212in}{1.837820in}}%
\pgfpathlineto{\pgfqpoint{6.092944in}{1.839103in}}%
\pgfpathlineto{\pgfqpoint{6.103253in}{1.844203in}}%
\pgfpathlineto{\pgfqpoint{6.110984in}{1.845469in}}%
\pgfpathlineto{\pgfqpoint{6.121293in}{1.850499in}}%
\pgfpathlineto{\pgfqpoint{6.129025in}{1.851787in}}%
\pgfpathlineto{\pgfqpoint{6.139334in}{1.856828in}}%
\pgfpathlineto{\pgfqpoint{6.147065in}{1.858086in}}%
\pgfpathlineto{\pgfqpoint{6.157374in}{1.862996in}}%
\pgfpathlineto{\pgfqpoint{6.165106in}{1.864160in}}%
\pgfpathlineto{\pgfqpoint{6.175415in}{1.868745in}}%
\pgfpathlineto{\pgfqpoint{6.185724in}{1.869747in}}%
\pgfpathlineto{\pgfqpoint{6.193455in}{1.873034in}}%
\pgfpathlineto{\pgfqpoint{6.201187in}{1.874267in}}%
\pgfpathlineto{\pgfqpoint{6.211496in}{1.879218in}}%
\pgfpathlineto{\pgfqpoint{6.219227in}{1.880597in}}%
\pgfpathlineto{\pgfqpoint{6.226959in}{1.884234in}}%
\pgfpathlineto{\pgfqpoint{6.229536in}{1.885285in}}%
\pgfpathlineto{\pgfqpoint{6.237268in}{1.886315in}}%
\pgfpathlineto{\pgfqpoint{6.247577in}{1.890073in}}%
\pgfpathlineto{\pgfqpoint{6.255308in}{1.891085in}}%
\pgfpathlineto{\pgfqpoint{6.265617in}{1.895568in}}%
\pgfpathlineto{\pgfqpoint{6.273349in}{1.896722in}}%
\pgfpathlineto{\pgfqpoint{6.283658in}{1.902060in}}%
\pgfpathlineto{\pgfqpoint{6.291389in}{1.903392in}}%
\pgfpathlineto{\pgfqpoint{6.301698in}{1.910439in}}%
\pgfpathlineto{\pgfqpoint{6.309430in}{1.912370in}}%
\pgfpathlineto{\pgfqpoint{6.319739in}{1.919402in}}%
\pgfpathlineto{\pgfqpoint{6.327470in}{1.920980in}}%
\pgfpathlineto{\pgfqpoint{6.337779in}{1.927173in}}%
\pgfpathlineto{\pgfqpoint{6.345511in}{1.928698in}}%
\pgfpathlineto{\pgfqpoint{6.355820in}{1.934892in}}%
\pgfpathlineto{\pgfqpoint{6.363551in}{1.936383in}}%
\pgfpathlineto{\pgfqpoint{6.373860in}{1.942005in}}%
\pgfpathlineto{\pgfqpoint{6.381592in}{1.943323in}}%
\pgfpathlineto{\pgfqpoint{6.386746in}{1.945925in}}%
\pgfpathlineto{\pgfqpoint{6.391901in}{1.947232in}}%
\pgfpathlineto{\pgfqpoint{6.399632in}{1.948607in}}%
\pgfpathlineto{\pgfqpoint{6.409941in}{1.954577in}}%
\pgfpathlineto{\pgfqpoint{6.417673in}{1.955993in}}%
\pgfpathlineto{\pgfqpoint{6.427982in}{1.962015in}}%
\pgfpathlineto{\pgfqpoint{6.435713in}{1.963569in}}%
\pgfpathlineto{\pgfqpoint{6.446022in}{1.970135in}}%
\pgfpathlineto{\pgfqpoint{6.453754in}{1.971709in}}%
\pgfpathlineto{\pgfqpoint{6.464063in}{1.977696in}}%
\pgfpathlineto{\pgfqpoint{6.474371in}{1.979103in}}%
\pgfpathlineto{\pgfqpoint{6.482103in}{1.983336in}}%
\pgfpathlineto{\pgfqpoint{6.482103in}{1.983336in}}%
\pgfusepath{stroke}%
\end{pgfscope}%
\begin{pgfscope}%
\pgfpathrectangle{\pgfqpoint{0.563921in}{0.521603in}}{\pgfqpoint{6.200000in}{2.642500in}}%
\pgfusepath{clip}%
\pgfsetroundcap%
\pgfsetroundjoin%
\pgfsetlinewidth{1.505625pt}%
\definecolor{currentstroke}{rgb}{0.549020,0.337255,0.294118}%
\pgfsetstrokecolor{currentstroke}%
\pgfsetdash{}{0pt}%
\pgfpathmoveto{\pgfqpoint{0.845739in}{0.641717in}}%
\pgfpathlineto{\pgfqpoint{0.848317in}{0.642313in}}%
\pgfpathlineto{\pgfqpoint{0.850894in}{0.648196in}}%
\pgfpathlineto{\pgfqpoint{0.853471in}{0.650933in}}%
\pgfpathlineto{\pgfqpoint{0.861203in}{0.649962in}}%
\pgfpathlineto{\pgfqpoint{0.863780in}{0.650863in}}%
\pgfpathlineto{\pgfqpoint{0.866357in}{0.658682in}}%
\pgfpathlineto{\pgfqpoint{0.868934in}{0.660441in}}%
\pgfpathlineto{\pgfqpoint{0.871512in}{0.661209in}}%
\pgfpathlineto{\pgfqpoint{0.889552in}{0.659387in}}%
\pgfpathlineto{\pgfqpoint{0.897284in}{0.661694in}}%
\pgfpathlineto{\pgfqpoint{0.899861in}{0.666279in}}%
\pgfpathlineto{\pgfqpoint{0.905015in}{0.667984in}}%
\pgfpathlineto{\pgfqpoint{0.915324in}{0.679311in}}%
\pgfpathlineto{\pgfqpoint{0.917901in}{0.685767in}}%
\pgfpathlineto{\pgfqpoint{0.920479in}{0.689581in}}%
\pgfpathlineto{\pgfqpoint{0.923056in}{0.692003in}}%
\pgfpathlineto{\pgfqpoint{0.925633in}{0.695852in}}%
\pgfpathlineto{\pgfqpoint{0.938519in}{0.696841in}}%
\pgfpathlineto{\pgfqpoint{0.943674in}{0.695946in}}%
\pgfpathlineto{\pgfqpoint{0.951405in}{0.695219in}}%
\pgfpathlineto{\pgfqpoint{0.961714in}{0.692096in}}%
\pgfpathlineto{\pgfqpoint{0.974600in}{0.690777in}}%
\pgfpathlineto{\pgfqpoint{0.979754in}{0.693258in}}%
\pgfpathlineto{\pgfqpoint{0.987486in}{0.694469in}}%
\pgfpathlineto{\pgfqpoint{0.992641in}{0.699748in}}%
\pgfpathlineto{\pgfqpoint{0.997795in}{0.700708in}}%
\pgfpathlineto{\pgfqpoint{1.015835in}{0.702981in}}%
\pgfpathlineto{\pgfqpoint{1.023567in}{0.704499in}}%
\pgfpathlineto{\pgfqpoint{1.028722in}{0.707548in}}%
\pgfpathlineto{\pgfqpoint{1.031299in}{0.708533in}}%
\pgfpathlineto{\pgfqpoint{1.033876in}{0.708876in}}%
\pgfpathlineto{\pgfqpoint{1.046762in}{0.709498in}}%
\pgfpathlineto{\pgfqpoint{1.051916in}{0.710242in}}%
\pgfpathlineto{\pgfqpoint{1.067380in}{0.710350in}}%
\pgfpathlineto{\pgfqpoint{1.098306in}{0.709607in}}%
\pgfpathlineto{\pgfqpoint{1.106038in}{0.708231in}}%
\pgfpathlineto{\pgfqpoint{1.149851in}{0.706863in}}%
\pgfpathlineto{\pgfqpoint{1.155005in}{0.707244in}}%
\pgfpathlineto{\pgfqpoint{1.160159in}{0.706895in}}%
\pgfpathlineto{\pgfqpoint{1.178200in}{0.706962in}}%
\pgfpathlineto{\pgfqpoint{1.206549in}{0.707929in}}%
\pgfpathlineto{\pgfqpoint{1.214281in}{0.710530in}}%
\pgfpathlineto{\pgfqpoint{1.224590in}{0.710941in}}%
\pgfpathlineto{\pgfqpoint{1.229744in}{0.712656in}}%
\pgfpathlineto{\pgfqpoint{1.232321in}{0.714141in}}%
\pgfpathlineto{\pgfqpoint{1.240053in}{0.715688in}}%
\pgfpathlineto{\pgfqpoint{1.242630in}{0.717372in}}%
\pgfpathlineto{\pgfqpoint{1.245207in}{0.718370in}}%
\pgfpathlineto{\pgfqpoint{1.263248in}{0.719781in}}%
\pgfpathlineto{\pgfqpoint{1.268402in}{0.719898in}}%
\pgfpathlineto{\pgfqpoint{1.278711in}{0.720784in}}%
\pgfpathlineto{\pgfqpoint{1.283866in}{0.725189in}}%
\pgfpathlineto{\pgfqpoint{1.286443in}{0.727435in}}%
\pgfpathlineto{\pgfqpoint{1.294174in}{0.730094in}}%
\pgfpathlineto{\pgfqpoint{1.296752in}{0.732628in}}%
\pgfpathlineto{\pgfqpoint{1.301906in}{0.735681in}}%
\pgfpathlineto{\pgfqpoint{1.304483in}{0.736290in}}%
\pgfpathlineto{\pgfqpoint{1.319947in}{0.737867in}}%
\pgfpathlineto{\pgfqpoint{1.322524in}{0.738356in}}%
\pgfpathlineto{\pgfqpoint{1.340564in}{0.738626in}}%
\pgfpathlineto{\pgfqpoint{1.366336in}{0.736989in}}%
\pgfpathlineto{\pgfqpoint{1.376645in}{0.735867in}}%
\pgfpathlineto{\pgfqpoint{1.389531in}{0.735099in}}%
\pgfpathlineto{\pgfqpoint{1.394686in}{0.734717in}}%
\pgfpathlineto{\pgfqpoint{1.407572in}{0.735268in}}%
\pgfpathlineto{\pgfqpoint{1.412726in}{0.735991in}}%
\pgfpathlineto{\pgfqpoint{1.425612in}{0.736796in}}%
\pgfpathlineto{\pgfqpoint{1.430767in}{0.737602in}}%
\pgfpathlineto{\pgfqpoint{1.446230in}{0.738654in}}%
\pgfpathlineto{\pgfqpoint{1.448807in}{0.738996in}}%
\pgfpathlineto{\pgfqpoint{1.464270in}{0.740152in}}%
\pgfpathlineto{\pgfqpoint{1.466848in}{0.740505in}}%
\pgfpathlineto{\pgfqpoint{1.479734in}{0.741305in}}%
\pgfpathlineto{\pgfqpoint{1.484888in}{0.743040in}}%
\pgfpathlineto{\pgfqpoint{1.495197in}{0.744663in}}%
\pgfpathlineto{\pgfqpoint{1.500351in}{0.746316in}}%
\pgfpathlineto{\pgfqpoint{1.502929in}{0.747456in}}%
\pgfpathlineto{\pgfqpoint{1.510660in}{0.748606in}}%
\pgfpathlineto{\pgfqpoint{1.520969in}{0.753074in}}%
\pgfpathlineto{\pgfqpoint{1.528701in}{0.754317in}}%
\pgfpathlineto{\pgfqpoint{1.536432in}{0.757221in}}%
\pgfpathlineto{\pgfqpoint{1.539010in}{0.758109in}}%
\pgfpathlineto{\pgfqpoint{1.549318in}{0.759584in}}%
\pgfpathlineto{\pgfqpoint{1.557050in}{0.761984in}}%
\pgfpathlineto{\pgfqpoint{1.585399in}{0.764509in}}%
\pgfpathlineto{\pgfqpoint{1.593131in}{0.766838in}}%
\pgfpathlineto{\pgfqpoint{1.606017in}{0.767683in}}%
\pgfpathlineto{\pgfqpoint{1.611172in}{0.769542in}}%
\pgfpathlineto{\pgfqpoint{1.626635in}{0.770857in}}%
\pgfpathlineto{\pgfqpoint{1.629212in}{0.771465in}}%
\pgfpathlineto{\pgfqpoint{1.654984in}{0.772303in}}%
\pgfpathlineto{\pgfqpoint{1.673025in}{0.771857in}}%
\pgfpathlineto{\pgfqpoint{1.693642in}{0.773940in}}%
\pgfpathlineto{\pgfqpoint{1.701374in}{0.775665in}}%
\pgfpathlineto{\pgfqpoint{1.714260in}{0.777103in}}%
\pgfpathlineto{\pgfqpoint{1.719414in}{0.778482in}}%
\pgfpathlineto{\pgfqpoint{1.727146in}{0.779183in}}%
\pgfpathlineto{\pgfqpoint{1.737455in}{0.782043in}}%
\pgfpathlineto{\pgfqpoint{1.750341in}{0.783379in}}%
\pgfpathlineto{\pgfqpoint{1.755495in}{0.783955in}}%
\pgfpathlineto{\pgfqpoint{1.809617in}{0.785051in}}%
\pgfpathlineto{\pgfqpoint{1.822503in}{0.786063in}}%
\pgfpathlineto{\pgfqpoint{1.827657in}{0.787021in}}%
\pgfpathlineto{\pgfqpoint{1.837966in}{0.787529in}}%
\pgfpathlineto{\pgfqpoint{1.843121in}{0.788895in}}%
\pgfpathlineto{\pgfqpoint{1.845698in}{0.790672in}}%
\pgfpathlineto{\pgfqpoint{1.853430in}{0.792663in}}%
\pgfpathlineto{\pgfqpoint{1.863738in}{0.803210in}}%
\pgfpathlineto{\pgfqpoint{1.871470in}{0.805618in}}%
\pgfpathlineto{\pgfqpoint{1.881779in}{0.815972in}}%
\pgfpathlineto{\pgfqpoint{1.889510in}{0.818306in}}%
\pgfpathlineto{\pgfqpoint{1.899819in}{0.828291in}}%
\pgfpathlineto{\pgfqpoint{1.910128in}{0.831073in}}%
\pgfpathlineto{\pgfqpoint{1.917860in}{0.838450in}}%
\pgfpathlineto{\pgfqpoint{1.925591in}{0.840277in}}%
\pgfpathlineto{\pgfqpoint{1.935900in}{0.847912in}}%
\pgfpathlineto{\pgfqpoint{1.943632in}{0.849836in}}%
\pgfpathlineto{\pgfqpoint{1.953941in}{0.857733in}}%
\pgfpathlineto{\pgfqpoint{1.961672in}{0.859703in}}%
\pgfpathlineto{\pgfqpoint{1.971981in}{0.866492in}}%
\pgfpathlineto{\pgfqpoint{1.979713in}{0.867821in}}%
\pgfpathlineto{\pgfqpoint{1.990022in}{0.874433in}}%
\pgfpathlineto{\pgfqpoint{1.997753in}{0.875787in}}%
\pgfpathlineto{\pgfqpoint{2.005485in}{0.880179in}}%
\pgfpathlineto{\pgfqpoint{2.015794in}{0.881763in}}%
\pgfpathlineto{\pgfqpoint{2.023526in}{0.887221in}}%
\pgfpathlineto{\pgfqpoint{2.026103in}{0.888852in}}%
\pgfpathlineto{\pgfqpoint{2.033834in}{0.890648in}}%
\pgfpathlineto{\pgfqpoint{2.038989in}{0.894106in}}%
\pgfpathlineto{\pgfqpoint{2.044143in}{0.898223in}}%
\pgfpathlineto{\pgfqpoint{2.051875in}{0.900133in}}%
\pgfpathlineto{\pgfqpoint{2.062184in}{0.908078in}}%
\pgfpathlineto{\pgfqpoint{2.069915in}{0.910428in}}%
\pgfpathlineto{\pgfqpoint{2.072493in}{0.913176in}}%
\pgfpathlineto{\pgfqpoint{2.080224in}{0.916068in}}%
\pgfpathlineto{\pgfqpoint{2.090533in}{0.918066in}}%
\pgfpathlineto{\pgfqpoint{2.098265in}{0.921276in}}%
\pgfpathlineto{\pgfqpoint{2.105996in}{0.922333in}}%
\pgfpathlineto{\pgfqpoint{2.116305in}{0.927018in}}%
\pgfpathlineto{\pgfqpoint{2.124037in}{0.928208in}}%
\pgfpathlineto{\pgfqpoint{2.134346in}{0.934495in}}%
\pgfpathlineto{\pgfqpoint{2.142077in}{0.935703in}}%
\pgfpathlineto{\pgfqpoint{2.149809in}{0.938950in}}%
\pgfpathlineto{\pgfqpoint{2.152386in}{0.940913in}}%
\pgfpathlineto{\pgfqpoint{2.162695in}{0.942517in}}%
\pgfpathlineto{\pgfqpoint{2.170427in}{0.945108in}}%
\pgfpathlineto{\pgfqpoint{2.180735in}{0.946409in}}%
\pgfpathlineto{\pgfqpoint{2.224548in}{0.954400in}}%
\pgfpathlineto{\pgfqpoint{2.234857in}{0.955075in}}%
\pgfpathlineto{\pgfqpoint{2.242589in}{0.956429in}}%
\pgfpathlineto{\pgfqpoint{2.252897in}{0.957632in}}%
\pgfpathlineto{\pgfqpoint{2.255475in}{0.958290in}}%
\pgfpathlineto{\pgfqpoint{2.270938in}{0.960409in}}%
\pgfpathlineto{\pgfqpoint{2.276092in}{0.962383in}}%
\pgfpathlineto{\pgfqpoint{2.278670in}{0.963652in}}%
\pgfpathlineto{\pgfqpoint{2.286401in}{0.964893in}}%
\pgfpathlineto{\pgfqpoint{2.296710in}{0.969304in}}%
\pgfpathlineto{\pgfqpoint{2.304442in}{0.970530in}}%
\pgfpathlineto{\pgfqpoint{2.314751in}{0.974577in}}%
\pgfpathlineto{\pgfqpoint{2.322482in}{0.975457in}}%
\pgfpathlineto{\pgfqpoint{2.332791in}{0.979585in}}%
\pgfpathlineto{\pgfqpoint{2.340523in}{0.980687in}}%
\pgfpathlineto{\pgfqpoint{2.350832in}{0.985376in}}%
\pgfpathlineto{\pgfqpoint{2.358563in}{0.986443in}}%
\pgfpathlineto{\pgfqpoint{2.366295in}{0.989239in}}%
\pgfpathlineto{\pgfqpoint{2.368872in}{0.989891in}}%
\pgfpathlineto{\pgfqpoint{2.379181in}{0.991037in}}%
\pgfpathlineto{\pgfqpoint{2.386912in}{0.992784in}}%
\pgfpathlineto{\pgfqpoint{2.438457in}{0.995570in}}%
\pgfpathlineto{\pgfqpoint{2.441034in}{0.995948in}}%
\pgfpathlineto{\pgfqpoint{2.451343in}{0.997003in}}%
\pgfpathlineto{\pgfqpoint{2.459074in}{0.998501in}}%
\pgfpathlineto{\pgfqpoint{2.487424in}{0.999381in}}%
\pgfpathlineto{\pgfqpoint{2.510619in}{0.999154in}}%
\pgfpathlineto{\pgfqpoint{2.544122in}{1.002373in}}%
\pgfpathlineto{\pgfqpoint{2.549277in}{1.003444in}}%
\pgfpathlineto{\pgfqpoint{2.557009in}{1.004140in}}%
\pgfpathlineto{\pgfqpoint{2.564740in}{1.006332in}}%
\pgfpathlineto{\pgfqpoint{2.567317in}{1.006965in}}%
\pgfpathlineto{\pgfqpoint{2.577626in}{1.008285in}}%
\pgfpathlineto{\pgfqpoint{2.585358in}{1.010834in}}%
\pgfpathlineto{\pgfqpoint{2.593089in}{1.011609in}}%
\pgfpathlineto{\pgfqpoint{2.600821in}{1.014608in}}%
\pgfpathlineto{\pgfqpoint{2.603398in}{1.015846in}}%
\pgfpathlineto{\pgfqpoint{2.611130in}{1.017012in}}%
\pgfpathlineto{\pgfqpoint{2.621439in}{1.021603in}}%
\pgfpathlineto{\pgfqpoint{2.629170in}{1.022869in}}%
\pgfpathlineto{\pgfqpoint{2.634325in}{1.024967in}}%
\pgfpathlineto{\pgfqpoint{2.639479in}{1.025948in}}%
\pgfpathlineto{\pgfqpoint{2.647211in}{1.026755in}}%
\pgfpathlineto{\pgfqpoint{2.657520in}{1.030078in}}%
\pgfpathlineto{\pgfqpoint{2.667829in}{1.031900in}}%
\pgfpathlineto{\pgfqpoint{2.672983in}{1.033313in}}%
\pgfpathlineto{\pgfqpoint{2.675560in}{1.033873in}}%
\pgfpathlineto{\pgfqpoint{2.685869in}{1.034656in}}%
\pgfpathlineto{\pgfqpoint{2.693601in}{1.036112in}}%
\pgfpathlineto{\pgfqpoint{2.709064in}{1.037262in}}%
\pgfpathlineto{\pgfqpoint{2.711641in}{1.037714in}}%
\pgfpathlineto{\pgfqpoint{2.783803in}{1.041486in}}%
\pgfpathlineto{\pgfqpoint{2.809575in}{1.041081in}}%
\pgfpathlineto{\pgfqpoint{2.819884in}{1.040537in}}%
\pgfpathlineto{\pgfqpoint{2.943590in}{1.040644in}}%
\pgfpathlineto{\pgfqpoint{2.992557in}{1.042783in}}%
\pgfpathlineto{\pgfqpoint{2.997712in}{1.043322in}}%
\pgfpathlineto{\pgfqpoint{3.015752in}{1.044224in}}%
\pgfpathlineto{\pgfqpoint{3.036370in}{1.046560in}}%
\pgfpathlineto{\pgfqpoint{3.049256in}{1.047444in}}%
\pgfpathlineto{\pgfqpoint{3.054411in}{1.048167in}}%
\pgfpathlineto{\pgfqpoint{3.144613in}{1.051024in}}%
\pgfpathlineto{\pgfqpoint{3.216775in}{1.051802in}}%
\pgfpathlineto{\pgfqpoint{3.304400in}{1.054044in}}%
\pgfpathlineto{\pgfqpoint{3.392025in}{1.061992in}}%
\pgfpathlineto{\pgfqpoint{3.397180in}{1.062837in}}%
\pgfpathlineto{\pgfqpoint{3.407489in}{1.063738in}}%
\pgfpathlineto{\pgfqpoint{3.415220in}{1.065090in}}%
\pgfpathlineto{\pgfqpoint{3.430684in}{1.066350in}}%
\pgfpathlineto{\pgfqpoint{3.433261in}{1.066677in}}%
\pgfpathlineto{\pgfqpoint{3.446147in}{1.067593in}}%
\pgfpathlineto{\pgfqpoint{3.461610in}{1.068852in}}%
\pgfpathlineto{\pgfqpoint{3.495114in}{1.071992in}}%
\pgfpathlineto{\pgfqpoint{3.505423in}{1.074766in}}%
\pgfpathlineto{\pgfqpoint{3.513154in}{1.075521in}}%
\pgfpathlineto{\pgfqpoint{3.523463in}{1.079330in}}%
\pgfpathlineto{\pgfqpoint{3.531195in}{1.080339in}}%
\pgfpathlineto{\pgfqpoint{3.541504in}{1.084018in}}%
\pgfpathlineto{\pgfqpoint{3.551813in}{1.085517in}}%
\pgfpathlineto{\pgfqpoint{3.559544in}{1.088000in}}%
\pgfpathlineto{\pgfqpoint{3.567276in}{1.088749in}}%
\pgfpathlineto{\pgfqpoint{3.572430in}{1.090423in}}%
\pgfpathlineto{\pgfqpoint{3.577585in}{1.091474in}}%
\pgfpathlineto{\pgfqpoint{3.585316in}{1.092465in}}%
\pgfpathlineto{\pgfqpoint{3.595625in}{1.096592in}}%
\pgfpathlineto{\pgfqpoint{3.603357in}{1.097638in}}%
\pgfpathlineto{\pgfqpoint{3.613666in}{1.101403in}}%
\pgfpathlineto{\pgfqpoint{3.623974in}{1.103000in}}%
\pgfpathlineto{\pgfqpoint{3.631706in}{1.106302in}}%
\pgfpathlineto{\pgfqpoint{3.639438in}{1.107534in}}%
\pgfpathlineto{\pgfqpoint{3.644592in}{1.110157in}}%
\pgfpathlineto{\pgfqpoint{3.649747in}{1.111490in}}%
\pgfpathlineto{\pgfqpoint{3.657478in}{1.112686in}}%
\pgfpathlineto{\pgfqpoint{3.662633in}{1.114790in}}%
\pgfpathlineto{\pgfqpoint{3.680673in}{1.117973in}}%
\pgfpathlineto{\pgfqpoint{3.685828in}{1.119694in}}%
\pgfpathlineto{\pgfqpoint{3.696136in}{1.121241in}}%
\pgfpathlineto{\pgfqpoint{3.703868in}{1.123618in}}%
\pgfpathlineto{\pgfqpoint{3.714177in}{1.124502in}}%
\pgfpathlineto{\pgfqpoint{3.721909in}{1.127220in}}%
\pgfpathlineto{\pgfqpoint{3.776030in}{1.131952in}}%
\pgfpathlineto{\pgfqpoint{3.794070in}{1.132949in}}%
\pgfpathlineto{\pgfqpoint{3.809534in}{1.133880in}}%
\pgfpathlineto{\pgfqpoint{3.827574in}{1.134814in}}%
\pgfpathlineto{\pgfqpoint{3.969321in}{1.134957in}}%
\pgfpathlineto{\pgfqpoint{3.992516in}{1.134174in}}%
\pgfpathlineto{\pgfqpoint{4.038906in}{1.132943in}}%
\pgfpathlineto{\pgfqpoint{4.093027in}{1.130250in}}%
\pgfpathlineto{\pgfqpoint{4.134263in}{1.128581in}}%
\pgfpathlineto{\pgfqpoint{4.206424in}{1.127276in}}%
\pgfpathlineto{\pgfqpoint{4.227042in}{1.125868in}}%
\pgfpathlineto{\pgfqpoint{4.242505in}{1.124950in}}%
\pgfpathlineto{\pgfqpoint{4.263123in}{1.123442in}}%
\pgfpathlineto{\pgfqpoint{4.278586in}{1.122511in}}%
\pgfpathlineto{\pgfqpoint{4.291472in}{1.121797in}}%
\pgfpathlineto{\pgfqpoint{4.299204in}{1.121121in}}%
\pgfpathlineto{\pgfqpoint{4.324976in}{1.120100in}}%
\pgfpathlineto{\pgfqpoint{4.353326in}{1.117959in}}%
\pgfpathlineto{\pgfqpoint{4.368789in}{1.116955in}}%
\pgfpathlineto{\pgfqpoint{4.389407in}{1.115474in}}%
\pgfpathlineto{\pgfqpoint{4.404870in}{1.114507in}}%
\pgfpathlineto{\pgfqpoint{4.425488in}{1.113174in}}%
\pgfpathlineto{\pgfqpoint{4.451260in}{1.112136in}}%
\pgfpathlineto{\pgfqpoint{4.510536in}{1.108772in}}%
\pgfpathlineto{\pgfqpoint{4.585275in}{1.106314in}}%
\pgfpathlineto{\pgfqpoint{4.621356in}{1.105567in}}%
\pgfpathlineto{\pgfqpoint{4.641973in}{1.104443in}}%
\pgfpathlineto{\pgfqpoint{4.706404in}{1.103931in}}%
\pgfpathlineto{\pgfqpoint{4.750216in}{1.104639in}}%
\pgfpathlineto{\pgfqpoint{4.768257in}{1.105062in}}%
\pgfpathlineto{\pgfqpoint{4.837842in}{1.106830in}}%
\pgfpathlineto{\pgfqpoint{4.858459in}{1.107672in}}%
\pgfpathlineto{\pgfqpoint{5.031132in}{1.110366in}}%
\pgfpathlineto{\pgfqpoint{5.108449in}{1.114368in}}%
\pgfpathlineto{\pgfqpoint{5.111026in}{1.114671in}}%
\pgfpathlineto{\pgfqpoint{5.123912in}{1.115572in}}%
\pgfpathlineto{\pgfqpoint{5.139375in}{1.116798in}}%
\pgfpathlineto{\pgfqpoint{5.196074in}{1.122200in}}%
\pgfpathlineto{\pgfqpoint{5.201228in}{1.123034in}}%
\pgfpathlineto{\pgfqpoint{5.211537in}{1.123820in}}%
\pgfpathlineto{\pgfqpoint{5.219269in}{1.125096in}}%
\pgfpathlineto{\pgfqpoint{5.232155in}{1.126384in}}%
\pgfpathlineto{\pgfqpoint{5.237309in}{1.127319in}}%
\pgfpathlineto{\pgfqpoint{5.250196in}{1.128252in}}%
\pgfpathlineto{\pgfqpoint{5.255350in}{1.128973in}}%
\pgfpathlineto{\pgfqpoint{5.268236in}{1.130137in}}%
\pgfpathlineto{\pgfqpoint{5.273390in}{1.131000in}}%
\pgfpathlineto{\pgfqpoint{5.283699in}{1.131923in}}%
\pgfpathlineto{\pgfqpoint{5.291431in}{1.133221in}}%
\pgfpathlineto{\pgfqpoint{5.301740in}{1.134061in}}%
\pgfpathlineto{\pgfqpoint{5.309471in}{1.135565in}}%
\pgfpathlineto{\pgfqpoint{5.319780in}{1.136447in}}%
\pgfpathlineto{\pgfqpoint{5.327512in}{1.137964in}}%
\pgfpathlineto{\pgfqpoint{5.340398in}{1.139303in}}%
\pgfpathlineto{\pgfqpoint{5.345552in}{1.140122in}}%
\pgfpathlineto{\pgfqpoint{5.373902in}{1.141950in}}%
\pgfpathlineto{\pgfqpoint{5.381633in}{1.142969in}}%
\pgfpathlineto{\pgfqpoint{5.394519in}{1.143940in}}%
\pgfpathlineto{\pgfqpoint{5.399674in}{1.144449in}}%
\pgfpathlineto{\pgfqpoint{5.453795in}{1.146195in}}%
\pgfpathlineto{\pgfqpoint{5.497608in}{1.147111in}}%
\pgfpathlineto{\pgfqpoint{5.515648in}{1.147899in}}%
\pgfpathlineto{\pgfqpoint{5.543998in}{1.148978in}}%
\pgfpathlineto{\pgfqpoint{5.574924in}{1.149856in}}%
\pgfpathlineto{\pgfqpoint{5.605851in}{1.151271in}}%
\pgfpathlineto{\pgfqpoint{5.616160in}{1.152581in}}%
\pgfpathlineto{\pgfqpoint{5.629046in}{1.153576in}}%
\pgfpathlineto{\pgfqpoint{5.634200in}{1.154282in}}%
\pgfpathlineto{\pgfqpoint{5.644509in}{1.154998in}}%
\pgfpathlineto{\pgfqpoint{5.652241in}{1.156196in}}%
\pgfpathlineto{\pgfqpoint{5.662550in}{1.156950in}}%
\pgfpathlineto{\pgfqpoint{5.670281in}{1.158759in}}%
\pgfpathlineto{\pgfqpoint{5.683167in}{1.160046in}}%
\pgfpathlineto{\pgfqpoint{5.688322in}{1.161242in}}%
\pgfpathlineto{\pgfqpoint{5.698630in}{1.162408in}}%
\pgfpathlineto{\pgfqpoint{5.706362in}{1.164149in}}%
\pgfpathlineto{\pgfqpoint{5.716671in}{1.165181in}}%
\pgfpathlineto{\pgfqpoint{5.724403in}{1.166758in}}%
\pgfpathlineto{\pgfqpoint{5.734711in}{1.167900in}}%
\pgfpathlineto{\pgfqpoint{5.742443in}{1.169624in}}%
\pgfpathlineto{\pgfqpoint{5.752752in}{1.170747in}}%
\pgfpathlineto{\pgfqpoint{5.760484in}{1.172310in}}%
\pgfpathlineto{\pgfqpoint{5.770792in}{1.173317in}}%
\pgfpathlineto{\pgfqpoint{5.778524in}{1.174715in}}%
\pgfpathlineto{\pgfqpoint{5.788833in}{1.175573in}}%
\pgfpathlineto{\pgfqpoint{5.796565in}{1.176787in}}%
\pgfpathlineto{\pgfqpoint{5.806873in}{1.177604in}}%
\pgfpathlineto{\pgfqpoint{5.812028in}{1.178491in}}%
\pgfpathlineto{\pgfqpoint{5.824914in}{1.179433in}}%
\pgfpathlineto{\pgfqpoint{5.832646in}{1.180671in}}%
\pgfpathlineto{\pgfqpoint{5.845532in}{1.181877in}}%
\pgfpathlineto{\pgfqpoint{5.850686in}{1.182453in}}%
\pgfpathlineto{\pgfqpoint{5.876458in}{1.183786in}}%
\pgfpathlineto{\pgfqpoint{5.904807in}{1.185560in}}%
\pgfpathlineto{\pgfqpoint{5.920271in}{1.186364in}}%
\pgfpathlineto{\pgfqpoint{5.922848in}{1.186612in}}%
\pgfpathlineto{\pgfqpoint{5.938311in}{1.187467in}}%
\pgfpathlineto{\pgfqpoint{5.940888in}{1.187796in}}%
\pgfpathlineto{\pgfqpoint{5.953775in}{1.188811in}}%
\pgfpathlineto{\pgfqpoint{5.976969in}{1.191027in}}%
\pgfpathlineto{\pgfqpoint{5.989855in}{1.192171in}}%
\pgfpathlineto{\pgfqpoint{5.995010in}{1.192871in}}%
\pgfpathlineto{\pgfqpoint{6.028514in}{1.194952in}}%
\pgfpathlineto{\pgfqpoint{6.038823in}{1.195394in}}%
\pgfpathlineto{\pgfqpoint{6.080058in}{1.198683in}}%
\pgfpathlineto{\pgfqpoint{6.085212in}{1.199582in}}%
\pgfpathlineto{\pgfqpoint{6.095521in}{1.200549in}}%
\pgfpathlineto{\pgfqpoint{6.103253in}{1.201959in}}%
\pgfpathlineto{\pgfqpoint{6.113562in}{1.202987in}}%
\pgfpathlineto{\pgfqpoint{6.121293in}{1.204548in}}%
\pgfpathlineto{\pgfqpoint{6.131602in}{1.205617in}}%
\pgfpathlineto{\pgfqpoint{6.139334in}{1.207296in}}%
\pgfpathlineto{\pgfqpoint{6.149643in}{1.208480in}}%
\pgfpathlineto{\pgfqpoint{6.157374in}{1.210140in}}%
\pgfpathlineto{\pgfqpoint{6.167683in}{1.211233in}}%
\pgfpathlineto{\pgfqpoint{6.175415in}{1.212817in}}%
\pgfpathlineto{\pgfqpoint{6.188301in}{1.213936in}}%
\pgfpathlineto{\pgfqpoint{6.193455in}{1.215074in}}%
\pgfpathlineto{\pgfqpoint{6.203764in}{1.216335in}}%
\pgfpathlineto{\pgfqpoint{6.211496in}{1.218137in}}%
\pgfpathlineto{\pgfqpoint{6.221805in}{1.219358in}}%
\pgfpathlineto{\pgfqpoint{6.229536in}{1.221043in}}%
\pgfpathlineto{\pgfqpoint{6.239845in}{1.222090in}}%
\pgfpathlineto{\pgfqpoint{6.247577in}{1.223271in}}%
\pgfpathlineto{\pgfqpoint{6.257886in}{1.224192in}}%
\pgfpathlineto{\pgfqpoint{6.265617in}{1.225625in}}%
\pgfpathlineto{\pgfqpoint{6.278503in}{1.226930in}}%
\pgfpathlineto{\pgfqpoint{6.283658in}{1.227913in}}%
\pgfpathlineto{\pgfqpoint{6.293967in}{1.228941in}}%
\pgfpathlineto{\pgfqpoint{6.301698in}{1.230140in}}%
\pgfpathlineto{\pgfqpoint{6.330048in}{1.231306in}}%
\pgfpathlineto{\pgfqpoint{6.422827in}{1.237576in}}%
\pgfpathlineto{\pgfqpoint{6.427982in}{1.238228in}}%
\pgfpathlineto{\pgfqpoint{6.440868in}{1.239214in}}%
\pgfpathlineto{\pgfqpoint{6.446022in}{1.240007in}}%
\pgfpathlineto{\pgfqpoint{6.458908in}{1.241232in}}%
\pgfpathlineto{\pgfqpoint{6.464063in}{1.242068in}}%
\pgfpathlineto{\pgfqpoint{6.476949in}{1.242949in}}%
\pgfpathlineto{\pgfqpoint{6.482103in}{1.243782in}}%
\pgfpathlineto{\pgfqpoint{6.482103in}{1.243782in}}%
\pgfusepath{stroke}%
\end{pgfscope}%
\begin{pgfscope}%
\pgfpathrectangle{\pgfqpoint{0.563921in}{0.521603in}}{\pgfqpoint{6.200000in}{2.642500in}}%
\pgfusepath{clip}%
\pgfsetroundcap%
\pgfsetroundjoin%
\pgfsetlinewidth{1.505625pt}%
\definecolor{currentstroke}{rgb}{0.890196,0.466667,0.760784}%
\pgfsetstrokecolor{currentstroke}%
\pgfsetdash{}{0pt}%
\pgfpathmoveto{\pgfqpoint{0.845739in}{0.641717in}}%
\pgfpathlineto{\pgfqpoint{0.848317in}{0.651251in}}%
\pgfpathlineto{\pgfqpoint{0.850894in}{0.656324in}}%
\pgfpathlineto{\pgfqpoint{0.853471in}{0.662793in}}%
\pgfpathlineto{\pgfqpoint{0.861203in}{0.662053in}}%
\pgfpathlineto{\pgfqpoint{0.866357in}{0.689930in}}%
\pgfpathlineto{\pgfqpoint{0.868934in}{0.700406in}}%
\pgfpathlineto{\pgfqpoint{0.871512in}{0.698570in}}%
\pgfpathlineto{\pgfqpoint{0.881820in}{0.701328in}}%
\pgfpathlineto{\pgfqpoint{0.884398in}{0.705959in}}%
\pgfpathlineto{\pgfqpoint{0.886975in}{0.706574in}}%
\pgfpathlineto{\pgfqpoint{0.889552in}{0.705084in}}%
\pgfpathlineto{\pgfqpoint{0.897284in}{0.704029in}}%
\pgfpathlineto{\pgfqpoint{0.899861in}{0.706023in}}%
\pgfpathlineto{\pgfqpoint{0.902438in}{0.706786in}}%
\pgfpathlineto{\pgfqpoint{0.907593in}{0.706564in}}%
\pgfpathlineto{\pgfqpoint{0.915324in}{0.706432in}}%
\pgfpathlineto{\pgfqpoint{0.917901in}{0.708281in}}%
\pgfpathlineto{\pgfqpoint{0.920479in}{0.717828in}}%
\pgfpathlineto{\pgfqpoint{0.923056in}{0.723834in}}%
\pgfpathlineto{\pgfqpoint{0.925633in}{0.732820in}}%
\pgfpathlineto{\pgfqpoint{0.933365in}{0.737724in}}%
\pgfpathlineto{\pgfqpoint{0.935942in}{0.740638in}}%
\pgfpathlineto{\pgfqpoint{0.938519in}{0.747744in}}%
\pgfpathlineto{\pgfqpoint{0.941096in}{0.761658in}}%
\pgfpathlineto{\pgfqpoint{0.943674in}{0.771253in}}%
\pgfpathlineto{\pgfqpoint{0.951405in}{0.784017in}}%
\pgfpathlineto{\pgfqpoint{0.953982in}{0.793284in}}%
\pgfpathlineto{\pgfqpoint{0.959137in}{0.800632in}}%
\pgfpathlineto{\pgfqpoint{0.961714in}{0.805359in}}%
\pgfpathlineto{\pgfqpoint{0.972023in}{0.809715in}}%
\pgfpathlineto{\pgfqpoint{0.974600in}{0.813361in}}%
\pgfpathlineto{\pgfqpoint{0.979754in}{0.818161in}}%
\pgfpathlineto{\pgfqpoint{0.990063in}{0.820896in}}%
\pgfpathlineto{\pgfqpoint{0.995218in}{0.824039in}}%
\pgfpathlineto{\pgfqpoint{0.997795in}{0.826024in}}%
\pgfpathlineto{\pgfqpoint{1.005527in}{0.826036in}}%
\pgfpathlineto{\pgfqpoint{1.008104in}{0.824393in}}%
\pgfpathlineto{\pgfqpoint{1.010681in}{0.823601in}}%
\pgfpathlineto{\pgfqpoint{1.023567in}{0.824220in}}%
\pgfpathlineto{\pgfqpoint{1.031299in}{0.835928in}}%
\pgfpathlineto{\pgfqpoint{1.033876in}{0.837224in}}%
\pgfpathlineto{\pgfqpoint{1.041608in}{0.837409in}}%
\pgfpathlineto{\pgfqpoint{1.046762in}{0.835169in}}%
\pgfpathlineto{\pgfqpoint{1.051916in}{0.832110in}}%
\pgfpathlineto{\pgfqpoint{1.059648in}{0.831345in}}%
\pgfpathlineto{\pgfqpoint{1.064802in}{0.828831in}}%
\pgfpathlineto{\pgfqpoint{1.069957in}{0.826515in}}%
\pgfpathlineto{\pgfqpoint{1.077689in}{0.825416in}}%
\pgfpathlineto{\pgfqpoint{1.085420in}{0.821520in}}%
\pgfpathlineto{\pgfqpoint{1.098306in}{0.819531in}}%
\pgfpathlineto{\pgfqpoint{1.106038in}{0.816021in}}%
\pgfpathlineto{\pgfqpoint{1.113770in}{0.814905in}}%
\pgfpathlineto{\pgfqpoint{1.124078in}{0.810335in}}%
\pgfpathlineto{\pgfqpoint{1.131810in}{0.809356in}}%
\pgfpathlineto{\pgfqpoint{1.142119in}{0.805502in}}%
\pgfpathlineto{\pgfqpoint{1.149851in}{0.804560in}}%
\pgfpathlineto{\pgfqpoint{1.160159in}{0.800976in}}%
\pgfpathlineto{\pgfqpoint{1.173045in}{0.800244in}}%
\pgfpathlineto{\pgfqpoint{1.178200in}{0.800628in}}%
\pgfpathlineto{\pgfqpoint{1.188509in}{0.802479in}}%
\pgfpathlineto{\pgfqpoint{1.191086in}{0.804031in}}%
\pgfpathlineto{\pgfqpoint{1.193663in}{0.806827in}}%
\pgfpathlineto{\pgfqpoint{1.196240in}{0.810494in}}%
\pgfpathlineto{\pgfqpoint{1.203972in}{0.812549in}}%
\pgfpathlineto{\pgfqpoint{1.214281in}{0.819915in}}%
\pgfpathlineto{\pgfqpoint{1.224590in}{0.820430in}}%
\pgfpathlineto{\pgfqpoint{1.229744in}{0.822916in}}%
\pgfpathlineto{\pgfqpoint{1.232321in}{0.825668in}}%
\pgfpathlineto{\pgfqpoint{1.240053in}{0.828814in}}%
\pgfpathlineto{\pgfqpoint{1.242630in}{0.832347in}}%
\pgfpathlineto{\pgfqpoint{1.245207in}{0.833351in}}%
\pgfpathlineto{\pgfqpoint{1.258093in}{0.833539in}}%
\pgfpathlineto{\pgfqpoint{1.268402in}{0.835591in}}%
\pgfpathlineto{\pgfqpoint{1.278711in}{0.835093in}}%
\pgfpathlineto{\pgfqpoint{1.283866in}{0.834793in}}%
\pgfpathlineto{\pgfqpoint{1.286443in}{0.834676in}}%
\pgfpathlineto{\pgfqpoint{1.299329in}{0.836063in}}%
\pgfpathlineto{\pgfqpoint{1.301906in}{0.837172in}}%
\pgfpathlineto{\pgfqpoint{1.304483in}{0.836863in}}%
\pgfpathlineto{\pgfqpoint{1.332833in}{0.836505in}}%
\pgfpathlineto{\pgfqpoint{1.340564in}{0.839063in}}%
\pgfpathlineto{\pgfqpoint{1.353450in}{0.839588in}}%
\pgfpathlineto{\pgfqpoint{1.358605in}{0.839104in}}%
\pgfpathlineto{\pgfqpoint{1.366336in}{0.839461in}}%
\pgfpathlineto{\pgfqpoint{1.371491in}{0.841072in}}%
\pgfpathlineto{\pgfqpoint{1.376645in}{0.841404in}}%
\pgfpathlineto{\pgfqpoint{1.392108in}{0.840488in}}%
\pgfpathlineto{\pgfqpoint{1.394686in}{0.839863in}}%
\pgfpathlineto{\pgfqpoint{1.404995in}{0.838609in}}%
\pgfpathlineto{\pgfqpoint{1.412726in}{0.836712in}}%
\pgfpathlineto{\pgfqpoint{1.423035in}{0.835463in}}%
\pgfpathlineto{\pgfqpoint{1.430767in}{0.834146in}}%
\pgfpathlineto{\pgfqpoint{1.441076in}{0.833429in}}%
\pgfpathlineto{\pgfqpoint{1.448807in}{0.832339in}}%
\pgfpathlineto{\pgfqpoint{1.461693in}{0.831773in}}%
\pgfpathlineto{\pgfqpoint{1.466848in}{0.830988in}}%
\pgfpathlineto{\pgfqpoint{1.479734in}{0.829923in}}%
\pgfpathlineto{\pgfqpoint{1.484888in}{0.829132in}}%
\pgfpathlineto{\pgfqpoint{1.495197in}{0.828172in}}%
\pgfpathlineto{\pgfqpoint{1.500351in}{0.827545in}}%
\pgfpathlineto{\pgfqpoint{1.502929in}{0.827971in}}%
\pgfpathlineto{\pgfqpoint{1.520969in}{0.828225in}}%
\pgfpathlineto{\pgfqpoint{1.531278in}{0.827471in}}%
\pgfpathlineto{\pgfqpoint{1.539010in}{0.826041in}}%
\pgfpathlineto{\pgfqpoint{1.549318in}{0.825103in}}%
\pgfpathlineto{\pgfqpoint{1.557050in}{0.823750in}}%
\pgfpathlineto{\pgfqpoint{1.569936in}{0.822476in}}%
\pgfpathlineto{\pgfqpoint{1.575091in}{0.821767in}}%
\pgfpathlineto{\pgfqpoint{1.585399in}{0.820950in}}%
\pgfpathlineto{\pgfqpoint{1.593131in}{0.819738in}}%
\pgfpathlineto{\pgfqpoint{1.603440in}{0.818882in}}%
\pgfpathlineto{\pgfqpoint{1.611172in}{0.817616in}}%
\pgfpathlineto{\pgfqpoint{1.626635in}{0.816851in}}%
\pgfpathlineto{\pgfqpoint{1.629212in}{0.816440in}}%
\pgfpathlineto{\pgfqpoint{1.644675in}{0.815071in}}%
\pgfpathlineto{\pgfqpoint{1.662716in}{0.813640in}}%
\pgfpathlineto{\pgfqpoint{1.683333in}{0.811933in}}%
\pgfpathlineto{\pgfqpoint{1.719414in}{0.810946in}}%
\pgfpathlineto{\pgfqpoint{1.747764in}{0.811425in}}%
\pgfpathlineto{\pgfqpoint{1.755495in}{0.813294in}}%
\pgfpathlineto{\pgfqpoint{1.781268in}{0.814574in}}%
\pgfpathlineto{\pgfqpoint{1.786422in}{0.815501in}}%
\pgfpathlineto{\pgfqpoint{1.791576in}{0.817816in}}%
\pgfpathlineto{\pgfqpoint{1.801885in}{0.819570in}}%
\pgfpathlineto{\pgfqpoint{1.809617in}{0.823000in}}%
\pgfpathlineto{\pgfqpoint{1.817349in}{0.824466in}}%
\pgfpathlineto{\pgfqpoint{1.827657in}{0.830730in}}%
\pgfpathlineto{\pgfqpoint{1.837966in}{0.832777in}}%
\pgfpathlineto{\pgfqpoint{1.843121in}{0.837752in}}%
\pgfpathlineto{\pgfqpoint{1.845698in}{0.840839in}}%
\pgfpathlineto{\pgfqpoint{1.853430in}{0.843839in}}%
\pgfpathlineto{\pgfqpoint{1.858584in}{0.849184in}}%
\pgfpathlineto{\pgfqpoint{1.861161in}{0.850853in}}%
\pgfpathlineto{\pgfqpoint{1.863738in}{0.853572in}}%
\pgfpathlineto{\pgfqpoint{1.871470in}{0.856060in}}%
\pgfpathlineto{\pgfqpoint{1.881779in}{0.865618in}}%
\pgfpathlineto{\pgfqpoint{1.889510in}{0.867826in}}%
\pgfpathlineto{\pgfqpoint{1.899819in}{0.877551in}}%
\pgfpathlineto{\pgfqpoint{1.910128in}{0.880386in}}%
\pgfpathlineto{\pgfqpoint{1.917860in}{0.887089in}}%
\pgfpathlineto{\pgfqpoint{1.925591in}{0.888545in}}%
\pgfpathlineto{\pgfqpoint{1.930746in}{0.892563in}}%
\pgfpathlineto{\pgfqpoint{1.935900in}{0.896763in}}%
\pgfpathlineto{\pgfqpoint{1.943632in}{0.898310in}}%
\pgfpathlineto{\pgfqpoint{1.953941in}{0.907870in}}%
\pgfpathlineto{\pgfqpoint{1.961672in}{0.910763in}}%
\pgfpathlineto{\pgfqpoint{1.971981in}{0.922217in}}%
\pgfpathlineto{\pgfqpoint{1.979713in}{0.924705in}}%
\pgfpathlineto{\pgfqpoint{1.990022in}{0.934733in}}%
\pgfpathlineto{\pgfqpoint{1.997753in}{0.936959in}}%
\pgfpathlineto{\pgfqpoint{2.005485in}{0.943475in}}%
\pgfpathlineto{\pgfqpoint{2.015794in}{0.945560in}}%
\pgfpathlineto{\pgfqpoint{2.026103in}{0.953591in}}%
\pgfpathlineto{\pgfqpoint{2.033834in}{0.955760in}}%
\pgfpathlineto{\pgfqpoint{2.041566in}{0.963649in}}%
\pgfpathlineto{\pgfqpoint{2.044143in}{0.966377in}}%
\pgfpathlineto{\pgfqpoint{2.051875in}{0.968237in}}%
\pgfpathlineto{\pgfqpoint{2.057029in}{0.971975in}}%
\pgfpathlineto{\pgfqpoint{2.062184in}{0.974745in}}%
\pgfpathlineto{\pgfqpoint{2.069915in}{0.976476in}}%
\pgfpathlineto{\pgfqpoint{2.077647in}{0.980297in}}%
\pgfpathlineto{\pgfqpoint{2.080224in}{0.981199in}}%
\pgfpathlineto{\pgfqpoint{2.087956in}{0.982209in}}%
\pgfpathlineto{\pgfqpoint{2.098265in}{0.986287in}}%
\pgfpathlineto{\pgfqpoint{2.105996in}{0.987726in}}%
\pgfpathlineto{\pgfqpoint{2.113728in}{0.992655in}}%
\pgfpathlineto{\pgfqpoint{2.116305in}{0.994522in}}%
\pgfpathlineto{\pgfqpoint{2.124037in}{0.996260in}}%
\pgfpathlineto{\pgfqpoint{2.134346in}{1.004856in}}%
\pgfpathlineto{\pgfqpoint{2.142077in}{1.007345in}}%
\pgfpathlineto{\pgfqpoint{2.147232in}{1.011955in}}%
\pgfpathlineto{\pgfqpoint{2.152386in}{1.015351in}}%
\pgfpathlineto{\pgfqpoint{2.162695in}{1.017220in}}%
\pgfpathlineto{\pgfqpoint{2.170427in}{1.022063in}}%
\pgfpathlineto{\pgfqpoint{2.178158in}{1.023593in}}%
\pgfpathlineto{\pgfqpoint{2.183313in}{1.025741in}}%
\pgfpathlineto{\pgfqpoint{2.188467in}{1.027816in}}%
\pgfpathlineto{\pgfqpoint{2.196199in}{1.028949in}}%
\pgfpathlineto{\pgfqpoint{2.206508in}{1.032855in}}%
\pgfpathlineto{\pgfqpoint{2.214239in}{1.034112in}}%
\pgfpathlineto{\pgfqpoint{2.219394in}{1.036680in}}%
\pgfpathlineto{\pgfqpoint{2.224548in}{1.037826in}}%
\pgfpathlineto{\pgfqpoint{2.234857in}{1.038689in}}%
\pgfpathlineto{\pgfqpoint{2.242589in}{1.040818in}}%
\pgfpathlineto{\pgfqpoint{2.252897in}{1.042671in}}%
\pgfpathlineto{\pgfqpoint{2.260629in}{1.045117in}}%
\pgfpathlineto{\pgfqpoint{2.268361in}{1.046651in}}%
\pgfpathlineto{\pgfqpoint{2.276092in}{1.052371in}}%
\pgfpathlineto{\pgfqpoint{2.278670in}{1.054602in}}%
\pgfpathlineto{\pgfqpoint{2.286401in}{1.056922in}}%
\pgfpathlineto{\pgfqpoint{2.294133in}{1.064099in}}%
\pgfpathlineto{\pgfqpoint{2.296710in}{1.066963in}}%
\pgfpathlineto{\pgfqpoint{2.304442in}{1.069656in}}%
\pgfpathlineto{\pgfqpoint{2.314751in}{1.083739in}}%
\pgfpathlineto{\pgfqpoint{2.322482in}{1.087132in}}%
\pgfpathlineto{\pgfqpoint{2.330214in}{1.098017in}}%
\pgfpathlineto{\pgfqpoint{2.332791in}{1.102101in}}%
\pgfpathlineto{\pgfqpoint{2.340523in}{1.105706in}}%
\pgfpathlineto{\pgfqpoint{2.350832in}{1.118353in}}%
\pgfpathlineto{\pgfqpoint{2.358563in}{1.121306in}}%
\pgfpathlineto{\pgfqpoint{2.363718in}{1.127469in}}%
\pgfpathlineto{\pgfqpoint{2.368872in}{1.131890in}}%
\pgfpathlineto{\pgfqpoint{2.376604in}{1.133941in}}%
\pgfpathlineto{\pgfqpoint{2.386912in}{1.141666in}}%
\pgfpathlineto{\pgfqpoint{2.394644in}{1.143485in}}%
\pgfpathlineto{\pgfqpoint{2.404953in}{1.148491in}}%
\pgfpathlineto{\pgfqpoint{2.415262in}{1.150283in}}%
\pgfpathlineto{\pgfqpoint{2.422993in}{1.156067in}}%
\pgfpathlineto{\pgfqpoint{2.430725in}{1.158207in}}%
\pgfpathlineto{\pgfqpoint{2.438457in}{1.166745in}}%
\pgfpathlineto{\pgfqpoint{2.441034in}{1.169813in}}%
\pgfpathlineto{\pgfqpoint{2.448766in}{1.173229in}}%
\pgfpathlineto{\pgfqpoint{2.459074in}{1.187621in}}%
\pgfpathlineto{\pgfqpoint{2.466806in}{1.190674in}}%
\pgfpathlineto{\pgfqpoint{2.477115in}{1.202591in}}%
\pgfpathlineto{\pgfqpoint{2.484847in}{1.204988in}}%
\pgfpathlineto{\pgfqpoint{2.490001in}{1.208893in}}%
\pgfpathlineto{\pgfqpoint{2.495155in}{1.211703in}}%
\pgfpathlineto{\pgfqpoint{2.502887in}{1.213091in}}%
\pgfpathlineto{\pgfqpoint{2.508041in}{1.215320in}}%
\pgfpathlineto{\pgfqpoint{2.513196in}{1.219026in}}%
\pgfpathlineto{\pgfqpoint{2.520928in}{1.220939in}}%
\pgfpathlineto{\pgfqpoint{2.528659in}{1.226436in}}%
\pgfpathlineto{\pgfqpoint{2.531236in}{1.228439in}}%
\pgfpathlineto{\pgfqpoint{2.538968in}{1.230388in}}%
\pgfpathlineto{\pgfqpoint{2.549277in}{1.237191in}}%
\pgfpathlineto{\pgfqpoint{2.557009in}{1.238670in}}%
\pgfpathlineto{\pgfqpoint{2.567317in}{1.244843in}}%
\pgfpathlineto{\pgfqpoint{2.575049in}{1.246638in}}%
\pgfpathlineto{\pgfqpoint{2.585358in}{1.253652in}}%
\pgfpathlineto{\pgfqpoint{2.593089in}{1.255356in}}%
\pgfpathlineto{\pgfqpoint{2.600821in}{1.260459in}}%
\pgfpathlineto{\pgfqpoint{2.603398in}{1.262309in}}%
\pgfpathlineto{\pgfqpoint{2.611130in}{1.264253in}}%
\pgfpathlineto{\pgfqpoint{2.621439in}{1.272083in}}%
\pgfpathlineto{\pgfqpoint{2.629170in}{1.274161in}}%
\pgfpathlineto{\pgfqpoint{2.634325in}{1.278644in}}%
\pgfpathlineto{\pgfqpoint{2.639479in}{1.280759in}}%
\pgfpathlineto{\pgfqpoint{2.647211in}{1.282729in}}%
\pgfpathlineto{\pgfqpoint{2.657520in}{1.289866in}}%
\pgfpathlineto{\pgfqpoint{2.665251in}{1.291901in}}%
\pgfpathlineto{\pgfqpoint{2.670406in}{1.295342in}}%
\pgfpathlineto{\pgfqpoint{2.675560in}{1.297891in}}%
\pgfpathlineto{\pgfqpoint{2.683292in}{1.299292in}}%
\pgfpathlineto{\pgfqpoint{2.688446in}{1.302121in}}%
\pgfpathlineto{\pgfqpoint{2.693601in}{1.305390in}}%
\pgfpathlineto{\pgfqpoint{2.701332in}{1.307139in}}%
\pgfpathlineto{\pgfqpoint{2.703910in}{1.309000in}}%
\pgfpathlineto{\pgfqpoint{2.709064in}{1.311091in}}%
\pgfpathlineto{\pgfqpoint{2.711641in}{1.313180in}}%
\pgfpathlineto{\pgfqpoint{2.719373in}{1.315226in}}%
\pgfpathlineto{\pgfqpoint{2.721950in}{1.317475in}}%
\pgfpathlineto{\pgfqpoint{2.727105in}{1.319421in}}%
\pgfpathlineto{\pgfqpoint{2.729682in}{1.321428in}}%
\pgfpathlineto{\pgfqpoint{2.737413in}{1.323389in}}%
\pgfpathlineto{\pgfqpoint{2.747722in}{1.331714in}}%
\pgfpathlineto{\pgfqpoint{2.755454in}{1.333558in}}%
\pgfpathlineto{\pgfqpoint{2.763185in}{1.339459in}}%
\pgfpathlineto{\pgfqpoint{2.765763in}{1.341493in}}%
\pgfpathlineto{\pgfqpoint{2.776072in}{1.343661in}}%
\pgfpathlineto{\pgfqpoint{2.781226in}{1.348124in}}%
\pgfpathlineto{\pgfqpoint{2.783803in}{1.349605in}}%
\pgfpathlineto{\pgfqpoint{2.791535in}{1.351438in}}%
\pgfpathlineto{\pgfqpoint{2.801844in}{1.358689in}}%
\pgfpathlineto{\pgfqpoint{2.812153in}{1.360701in}}%
\pgfpathlineto{\pgfqpoint{2.817307in}{1.362383in}}%
\pgfpathlineto{\pgfqpoint{2.819884in}{1.363523in}}%
\pgfpathlineto{\pgfqpoint{2.827616in}{1.364793in}}%
\pgfpathlineto{\pgfqpoint{2.837925in}{1.371156in}}%
\pgfpathlineto{\pgfqpoint{2.848234in}{1.372820in}}%
\pgfpathlineto{\pgfqpoint{2.855965in}{1.378298in}}%
\pgfpathlineto{\pgfqpoint{2.863697in}{1.380401in}}%
\pgfpathlineto{\pgfqpoint{2.874006in}{1.388512in}}%
\pgfpathlineto{\pgfqpoint{2.881737in}{1.390550in}}%
\pgfpathlineto{\pgfqpoint{2.892046in}{1.399246in}}%
\pgfpathlineto{\pgfqpoint{2.899778in}{1.401359in}}%
\pgfpathlineto{\pgfqpoint{2.907509in}{1.405965in}}%
\pgfpathlineto{\pgfqpoint{2.910087in}{1.407107in}}%
\pgfpathlineto{\pgfqpoint{2.917818in}{1.408527in}}%
\pgfpathlineto{\pgfqpoint{2.928127in}{1.413957in}}%
\pgfpathlineto{\pgfqpoint{2.935859in}{1.415197in}}%
\pgfpathlineto{\pgfqpoint{2.946168in}{1.420561in}}%
\pgfpathlineto{\pgfqpoint{2.953899in}{1.422237in}}%
\pgfpathlineto{\pgfqpoint{2.964208in}{1.430243in}}%
\pgfpathlineto{\pgfqpoint{2.971940in}{1.431714in}}%
\pgfpathlineto{\pgfqpoint{2.982249in}{1.436926in}}%
\pgfpathlineto{\pgfqpoint{2.989980in}{1.438132in}}%
\pgfpathlineto{\pgfqpoint{2.995135in}{1.441145in}}%
\pgfpathlineto{\pgfqpoint{2.997712in}{1.442893in}}%
\pgfpathlineto{\pgfqpoint{3.008021in}{1.444576in}}%
\pgfpathlineto{\pgfqpoint{3.018330in}{1.451451in}}%
\pgfpathlineto{\pgfqpoint{3.026061in}{1.452950in}}%
\pgfpathlineto{\pgfqpoint{3.036370in}{1.458551in}}%
\pgfpathlineto{\pgfqpoint{3.044102in}{1.459841in}}%
\pgfpathlineto{\pgfqpoint{3.054411in}{1.464821in}}%
\pgfpathlineto{\pgfqpoint{3.062142in}{1.466323in}}%
\pgfpathlineto{\pgfqpoint{3.069874in}{1.470297in}}%
\pgfpathlineto{\pgfqpoint{3.072451in}{1.471238in}}%
\pgfpathlineto{\pgfqpoint{3.082760in}{1.472957in}}%
\pgfpathlineto{\pgfqpoint{3.090491in}{1.475894in}}%
\pgfpathlineto{\pgfqpoint{3.100800in}{1.476950in}}%
\pgfpathlineto{\pgfqpoint{3.108532in}{1.480190in}}%
\pgfpathlineto{\pgfqpoint{3.116264in}{1.481423in}}%
\pgfpathlineto{\pgfqpoint{3.126572in}{1.486588in}}%
\pgfpathlineto{\pgfqpoint{3.134304in}{1.488161in}}%
\pgfpathlineto{\pgfqpoint{3.142036in}{1.491904in}}%
\pgfpathlineto{\pgfqpoint{3.144613in}{1.492943in}}%
\pgfpathlineto{\pgfqpoint{3.152345in}{1.493983in}}%
\pgfpathlineto{\pgfqpoint{3.162653in}{1.498358in}}%
\pgfpathlineto{\pgfqpoint{3.172962in}{1.500206in}}%
\pgfpathlineto{\pgfqpoint{3.180694in}{1.502809in}}%
\pgfpathlineto{\pgfqpoint{3.188426in}{1.503573in}}%
\pgfpathlineto{\pgfqpoint{3.196157in}{1.505849in}}%
\pgfpathlineto{\pgfqpoint{3.209043in}{1.507104in}}%
\pgfpathlineto{\pgfqpoint{3.216775in}{1.508681in}}%
\pgfpathlineto{\pgfqpoint{3.227084in}{1.509935in}}%
\pgfpathlineto{\pgfqpoint{3.234815in}{1.511343in}}%
\pgfpathlineto{\pgfqpoint{3.252856in}{1.512009in}}%
\pgfpathlineto{\pgfqpoint{3.265742in}{1.511700in}}%
\pgfpathlineto{\pgfqpoint{3.288937in}{1.509646in}}%
\pgfpathlineto{\pgfqpoint{3.304400in}{1.508696in}}%
\pgfpathlineto{\pgfqpoint{3.306977in}{1.508485in}}%
\pgfpathlineto{\pgfqpoint{3.335327in}{1.508914in}}%
\pgfpathlineto{\pgfqpoint{3.376562in}{1.508884in}}%
\pgfpathlineto{\pgfqpoint{3.410066in}{1.507926in}}%
\pgfpathlineto{\pgfqpoint{3.415220in}{1.507352in}}%
\pgfpathlineto{\pgfqpoint{3.428106in}{1.506485in}}%
\pgfpathlineto{\pgfqpoint{3.433261in}{1.505797in}}%
\pgfpathlineto{\pgfqpoint{3.443570in}{1.505018in}}%
\pgfpathlineto{\pgfqpoint{3.469342in}{1.501085in}}%
\pgfpathlineto{\pgfqpoint{3.479651in}{1.500214in}}%
\pgfpathlineto{\pgfqpoint{3.487382in}{1.499048in}}%
\pgfpathlineto{\pgfqpoint{3.551813in}{1.496923in}}%
\pgfpathlineto{\pgfqpoint{3.572430in}{1.497259in}}%
\pgfpathlineto{\pgfqpoint{3.603357in}{1.498169in}}%
\pgfpathlineto{\pgfqpoint{3.613666in}{1.499595in}}%
\pgfpathlineto{\pgfqpoint{3.626552in}{1.500448in}}%
\pgfpathlineto{\pgfqpoint{3.631706in}{1.501615in}}%
\pgfpathlineto{\pgfqpoint{3.642015in}{1.502946in}}%
\pgfpathlineto{\pgfqpoint{3.644592in}{1.503575in}}%
\pgfpathlineto{\pgfqpoint{3.667787in}{1.506232in}}%
\pgfpathlineto{\pgfqpoint{3.683250in}{1.507175in}}%
\pgfpathlineto{\pgfqpoint{3.685828in}{1.507487in}}%
\pgfpathlineto{\pgfqpoint{3.701291in}{1.508687in}}%
\pgfpathlineto{\pgfqpoint{3.703868in}{1.509177in}}%
\pgfpathlineto{\pgfqpoint{3.714177in}{1.509798in}}%
\pgfpathlineto{\pgfqpoint{3.721909in}{1.512355in}}%
\pgfpathlineto{\pgfqpoint{3.732217in}{1.513840in}}%
\pgfpathlineto{\pgfqpoint{3.739949in}{1.515170in}}%
\pgfpathlineto{\pgfqpoint{3.750258in}{1.516360in}}%
\pgfpathlineto{\pgfqpoint{3.757990in}{1.518562in}}%
\pgfpathlineto{\pgfqpoint{3.768298in}{1.520022in}}%
\pgfpathlineto{\pgfqpoint{3.776030in}{1.522531in}}%
\pgfpathlineto{\pgfqpoint{3.786339in}{1.523516in}}%
\pgfpathlineto{\pgfqpoint{3.794070in}{1.527014in}}%
\pgfpathlineto{\pgfqpoint{3.801802in}{1.528247in}}%
\pgfpathlineto{\pgfqpoint{3.812111in}{1.532720in}}%
\pgfpathlineto{\pgfqpoint{3.819843in}{1.533845in}}%
\pgfpathlineto{\pgfqpoint{3.830151in}{1.537185in}}%
\pgfpathlineto{\pgfqpoint{3.843038in}{1.539158in}}%
\pgfpathlineto{\pgfqpoint{3.848192in}{1.540605in}}%
\pgfpathlineto{\pgfqpoint{3.858501in}{1.542020in}}%
\pgfpathlineto{\pgfqpoint{3.866232in}{1.544029in}}%
\pgfpathlineto{\pgfqpoint{3.876541in}{1.545165in}}%
\pgfpathlineto{\pgfqpoint{3.899736in}{1.547851in}}%
\pgfpathlineto{\pgfqpoint{3.912622in}{1.548714in}}%
\pgfpathlineto{\pgfqpoint{3.920354in}{1.550082in}}%
\pgfpathlineto{\pgfqpoint{3.930663in}{1.550886in}}%
\pgfpathlineto{\pgfqpoint{3.938394in}{1.551893in}}%
\pgfpathlineto{\pgfqpoint{3.951280in}{1.552861in}}%
\pgfpathlineto{\pgfqpoint{3.956435in}{1.553476in}}%
\pgfpathlineto{\pgfqpoint{3.984784in}{1.554849in}}%
\pgfpathlineto{\pgfqpoint{4.002825in}{1.556344in}}%
\pgfpathlineto{\pgfqpoint{4.028597in}{1.560129in}}%
\pgfpathlineto{\pgfqpoint{4.041483in}{1.560805in}}%
\pgfpathlineto{\pgfqpoint{4.046637in}{1.561498in}}%
\pgfpathlineto{\pgfqpoint{4.059523in}{1.562561in}}%
\pgfpathlineto{\pgfqpoint{4.064678in}{1.563166in}}%
\pgfpathlineto{\pgfqpoint{4.077564in}{1.563962in}}%
\pgfpathlineto{\pgfqpoint{4.082718in}{1.564682in}}%
\pgfpathlineto{\pgfqpoint{4.108490in}{1.565501in}}%
\pgfpathlineto{\pgfqpoint{4.118799in}{1.565717in}}%
\pgfpathlineto{\pgfqpoint{4.144571in}{1.564931in}}%
\pgfpathlineto{\pgfqpoint{4.172921in}{1.563503in}}%
\pgfpathlineto{\pgfqpoint{4.183230in}{1.562895in}}%
\pgfpathlineto{\pgfqpoint{4.190961in}{1.561452in}}%
\pgfpathlineto{\pgfqpoint{4.201270in}{1.560449in}}%
\pgfpathlineto{\pgfqpoint{4.209002in}{1.558992in}}%
\pgfpathlineto{\pgfqpoint{4.219311in}{1.558002in}}%
\pgfpathlineto{\pgfqpoint{4.227042in}{1.556511in}}%
\pgfpathlineto{\pgfqpoint{4.237351in}{1.555527in}}%
\pgfpathlineto{\pgfqpoint{4.245083in}{1.554061in}}%
\pgfpathlineto{\pgfqpoint{4.255392in}{1.553105in}}%
\pgfpathlineto{\pgfqpoint{4.263123in}{1.551634in}}%
\pgfpathlineto{\pgfqpoint{4.273432in}{1.550773in}}%
\pgfpathlineto{\pgfqpoint{4.281164in}{1.549345in}}%
\pgfpathlineto{\pgfqpoint{4.291472in}{1.548456in}}%
\pgfpathlineto{\pgfqpoint{4.299204in}{1.547094in}}%
\pgfpathlineto{\pgfqpoint{4.312090in}{1.546160in}}%
\pgfpathlineto{\pgfqpoint{4.317245in}{1.545240in}}%
\pgfpathlineto{\pgfqpoint{4.327553in}{1.544313in}}%
\pgfpathlineto{\pgfqpoint{4.335285in}{1.542918in}}%
\pgfpathlineto{\pgfqpoint{4.348171in}{1.541740in}}%
\pgfpathlineto{\pgfqpoint{4.353326in}{1.541044in}}%
\pgfpathlineto{\pgfqpoint{4.366212in}{1.539932in}}%
\pgfpathlineto{\pgfqpoint{4.371366in}{1.539134in}}%
\pgfpathlineto{\pgfqpoint{4.381675in}{1.538222in}}%
\pgfpathlineto{\pgfqpoint{4.389407in}{1.536816in}}%
\pgfpathlineto{\pgfqpoint{4.399715in}{1.535877in}}%
\pgfpathlineto{\pgfqpoint{4.407447in}{1.534514in}}%
\pgfpathlineto{\pgfqpoint{4.417756in}{1.533602in}}%
\pgfpathlineto{\pgfqpoint{4.425488in}{1.532288in}}%
\pgfpathlineto{\pgfqpoint{4.435796in}{1.531415in}}%
\pgfpathlineto{\pgfqpoint{4.443528in}{1.530086in}}%
\pgfpathlineto{\pgfqpoint{4.453837in}{1.529222in}}%
\pgfpathlineto{\pgfqpoint{4.461569in}{1.527975in}}%
\pgfpathlineto{\pgfqpoint{4.471877in}{1.527111in}}%
\pgfpathlineto{\pgfqpoint{4.479609in}{1.525797in}}%
\pgfpathlineto{\pgfqpoint{4.489918in}{1.524914in}}%
\pgfpathlineto{\pgfqpoint{4.497649in}{1.523606in}}%
\pgfpathlineto{\pgfqpoint{4.507958in}{1.522731in}}%
\pgfpathlineto{\pgfqpoint{4.515690in}{1.521847in}}%
\pgfpathlineto{\pgfqpoint{4.525999in}{1.520959in}}%
\pgfpathlineto{\pgfqpoint{4.533730in}{1.519624in}}%
\pgfpathlineto{\pgfqpoint{4.544039in}{1.518740in}}%
\pgfpathlineto{\pgfqpoint{4.551771in}{1.517424in}}%
\pgfpathlineto{\pgfqpoint{4.562080in}{1.516557in}}%
\pgfpathlineto{\pgfqpoint{4.569811in}{1.515257in}}%
\pgfpathlineto{\pgfqpoint{4.580120in}{1.514387in}}%
\pgfpathlineto{\pgfqpoint{4.585275in}{1.513520in}}%
\pgfpathlineto{\pgfqpoint{4.598161in}{1.512661in}}%
\pgfpathlineto{\pgfqpoint{4.603315in}{1.511802in}}%
\pgfpathlineto{\pgfqpoint{4.616201in}{1.510939in}}%
\pgfpathlineto{\pgfqpoint{4.623933in}{1.509711in}}%
\pgfpathlineto{\pgfqpoint{4.636819in}{1.508566in}}%
\pgfpathlineto{\pgfqpoint{4.641973in}{1.507928in}}%
\pgfpathlineto{\pgfqpoint{4.657437in}{1.507149in}}%
\pgfpathlineto{\pgfqpoint{4.660014in}{1.506853in}}%
\pgfpathlineto{\pgfqpoint{4.672900in}{1.506052in}}%
\pgfpathlineto{\pgfqpoint{4.685786in}{1.505097in}}%
\pgfpathlineto{\pgfqpoint{4.714135in}{1.502426in}}%
\pgfpathlineto{\pgfqpoint{4.727021in}{1.501779in}}%
\pgfpathlineto{\pgfqpoint{4.732176in}{1.501070in}}%
\pgfpathlineto{\pgfqpoint{4.745062in}{1.499849in}}%
\pgfpathlineto{\pgfqpoint{4.750216in}{1.499073in}}%
\pgfpathlineto{\pgfqpoint{4.763102in}{1.497858in}}%
\pgfpathlineto{\pgfqpoint{4.768257in}{1.497061in}}%
\pgfpathlineto{\pgfqpoint{4.781143in}{1.495876in}}%
\pgfpathlineto{\pgfqpoint{4.786297in}{1.495082in}}%
\pgfpathlineto{\pgfqpoint{4.799183in}{1.493907in}}%
\pgfpathlineto{\pgfqpoint{4.804338in}{1.493178in}}%
\pgfpathlineto{\pgfqpoint{4.817224in}{1.492089in}}%
\pgfpathlineto{\pgfqpoint{4.819801in}{1.491728in}}%
\pgfpathlineto{\pgfqpoint{4.835264in}{1.490686in}}%
\pgfpathlineto{\pgfqpoint{4.840419in}{1.490002in}}%
\pgfpathlineto{\pgfqpoint{4.853305in}{1.488988in}}%
\pgfpathlineto{\pgfqpoint{4.858459in}{1.488354in}}%
\pgfpathlineto{\pgfqpoint{4.886809in}{1.486752in}}%
\pgfpathlineto{\pgfqpoint{4.912581in}{1.485179in}}%
\pgfpathlineto{\pgfqpoint{4.925467in}{1.484343in}}%
\pgfpathlineto{\pgfqpoint{4.930621in}{1.483701in}}%
\pgfpathlineto{\pgfqpoint{4.943507in}{1.482781in}}%
\pgfpathlineto{\pgfqpoint{4.948662in}{1.482150in}}%
\pgfpathlineto{\pgfqpoint{4.961548in}{1.481191in}}%
\pgfpathlineto{\pgfqpoint{4.966702in}{1.480510in}}%
\pgfpathlineto{\pgfqpoint{4.979588in}{1.479542in}}%
\pgfpathlineto{\pgfqpoint{4.984743in}{1.478933in}}%
\pgfpathlineto{\pgfqpoint{5.000206in}{1.478021in}}%
\pgfpathlineto{\pgfqpoint{5.002783in}{1.477714in}}%
\pgfpathlineto{\pgfqpoint{5.018246in}{1.476622in}}%
\pgfpathlineto{\pgfqpoint{5.038864in}{1.474903in}}%
\pgfpathlineto{\pgfqpoint{5.054327in}{1.473813in}}%
\pgfpathlineto{\pgfqpoint{5.056905in}{1.473482in}}%
\pgfpathlineto{\pgfqpoint{5.072368in}{1.472252in}}%
\pgfpathlineto{\pgfqpoint{5.074945in}{1.472009in}}%
\pgfpathlineto{\pgfqpoint{5.100717in}{1.470748in}}%
\pgfpathlineto{\pgfqpoint{5.121335in}{1.469772in}}%
\pgfpathlineto{\pgfqpoint{5.147107in}{1.468886in}}%
\pgfpathlineto{\pgfqpoint{5.263082in}{1.466918in}}%
\pgfpathlineto{\pgfqpoint{5.291431in}{1.464637in}}%
\pgfpathlineto{\pgfqpoint{5.306894in}{1.463656in}}%
\pgfpathlineto{\pgfqpoint{5.327512in}{1.462152in}}%
\pgfpathlineto{\pgfqpoint{5.340398in}{1.461251in}}%
\pgfpathlineto{\pgfqpoint{5.345552in}{1.460679in}}%
\pgfpathlineto{\pgfqpoint{5.358438in}{1.459811in}}%
\pgfpathlineto{\pgfqpoint{5.363593in}{1.459208in}}%
\pgfpathlineto{\pgfqpoint{5.379056in}{1.458112in}}%
\pgfpathlineto{\pgfqpoint{5.399674in}{1.456622in}}%
\pgfpathlineto{\pgfqpoint{5.453795in}{1.455629in}}%
\pgfpathlineto{\pgfqpoint{5.520803in}{1.456162in}}%
\pgfpathlineto{\pgfqpoint{5.543998in}{1.456692in}}%
\pgfpathlineto{\pgfqpoint{5.572347in}{1.457499in}}%
\pgfpathlineto{\pgfqpoint{5.598119in}{1.458098in}}%
\pgfpathlineto{\pgfqpoint{5.662550in}{1.459207in}}%
\pgfpathlineto{\pgfqpoint{5.688322in}{1.460532in}}%
\pgfpathlineto{\pgfqpoint{5.703785in}{1.461414in}}%
\pgfpathlineto{\pgfqpoint{5.724403in}{1.462399in}}%
\pgfpathlineto{\pgfqpoint{5.742443in}{1.463353in}}%
\pgfpathlineto{\pgfqpoint{5.788833in}{1.465328in}}%
\pgfpathlineto{\pgfqpoint{5.812028in}{1.466531in}}%
\pgfpathlineto{\pgfqpoint{5.830068in}{1.467371in}}%
\pgfpathlineto{\pgfqpoint{5.863572in}{1.471764in}}%
\pgfpathlineto{\pgfqpoint{5.868727in}{1.473151in}}%
\pgfpathlineto{\pgfqpoint{5.879035in}{1.474541in}}%
\pgfpathlineto{\pgfqpoint{5.886767in}{1.476551in}}%
\pgfpathlineto{\pgfqpoint{5.897076in}{1.477893in}}%
\pgfpathlineto{\pgfqpoint{5.904807in}{1.479784in}}%
\pgfpathlineto{\pgfqpoint{5.915116in}{1.481305in}}%
\pgfpathlineto{\pgfqpoint{5.922848in}{1.483605in}}%
\pgfpathlineto{\pgfqpoint{5.935734in}{1.485003in}}%
\pgfpathlineto{\pgfqpoint{5.940888in}{1.486491in}}%
\pgfpathlineto{\pgfqpoint{5.951197in}{1.487757in}}%
\pgfpathlineto{\pgfqpoint{5.958929in}{1.489542in}}%
\pgfpathlineto{\pgfqpoint{5.969238in}{1.490698in}}%
\pgfpathlineto{\pgfqpoint{5.976969in}{1.492503in}}%
\pgfpathlineto{\pgfqpoint{5.987278in}{1.493879in}}%
\pgfpathlineto{\pgfqpoint{5.995010in}{1.496003in}}%
\pgfpathlineto{\pgfqpoint{6.005319in}{1.497356in}}%
\pgfpathlineto{\pgfqpoint{6.013050in}{1.499443in}}%
\pgfpathlineto{\pgfqpoint{6.025936in}{1.500871in}}%
\pgfpathlineto{\pgfqpoint{6.031091in}{1.502260in}}%
\pgfpathlineto{\pgfqpoint{6.041400in}{1.503794in}}%
\pgfpathlineto{\pgfqpoint{6.049131in}{1.506092in}}%
\pgfpathlineto{\pgfqpoint{6.059440in}{1.507527in}}%
\pgfpathlineto{\pgfqpoint{6.067172in}{1.509689in}}%
\pgfpathlineto{\pgfqpoint{6.080058in}{1.511410in}}%
\pgfpathlineto{\pgfqpoint{6.085212in}{1.512246in}}%
\pgfpathlineto{\pgfqpoint{6.095521in}{1.513068in}}%
\pgfpathlineto{\pgfqpoint{6.103253in}{1.514724in}}%
\pgfpathlineto{\pgfqpoint{6.116139in}{1.515791in}}%
\pgfpathlineto{\pgfqpoint{6.129025in}{1.516664in}}%
\pgfpathlineto{\pgfqpoint{6.157374in}{1.519064in}}%
\pgfpathlineto{\pgfqpoint{6.167683in}{1.519712in}}%
\pgfpathlineto{\pgfqpoint{6.175415in}{1.521035in}}%
\pgfpathlineto{\pgfqpoint{6.226959in}{1.521635in}}%
\pgfpathlineto{\pgfqpoint{6.291389in}{1.526856in}}%
\pgfpathlineto{\pgfqpoint{6.301698in}{1.528591in}}%
\pgfpathlineto{\pgfqpoint{6.314584in}{1.529900in}}%
\pgfpathlineto{\pgfqpoint{6.319739in}{1.530717in}}%
\pgfpathlineto{\pgfqpoint{6.330048in}{1.531562in}}%
\pgfpathlineto{\pgfqpoint{6.337779in}{1.533018in}}%
\pgfpathlineto{\pgfqpoint{6.348088in}{1.533938in}}%
\pgfpathlineto{\pgfqpoint{6.355820in}{1.535015in}}%
\pgfpathlineto{\pgfqpoint{6.368706in}{1.535892in}}%
\pgfpathlineto{\pgfqpoint{6.373860in}{1.536482in}}%
\pgfpathlineto{\pgfqpoint{6.399632in}{1.537829in}}%
\pgfpathlineto{\pgfqpoint{6.425404in}{1.541454in}}%
\pgfpathlineto{\pgfqpoint{6.427982in}{1.542064in}}%
\pgfpathlineto{\pgfqpoint{6.438290in}{1.543356in}}%
\pgfpathlineto{\pgfqpoint{6.446022in}{1.545560in}}%
\pgfpathlineto{\pgfqpoint{6.453754in}{1.546431in}}%
\pgfpathlineto{\pgfqpoint{6.464063in}{1.549994in}}%
\pgfpathlineto{\pgfqpoint{6.474371in}{1.550880in}}%
\pgfpathlineto{\pgfqpoint{6.482103in}{1.553657in}}%
\pgfpathlineto{\pgfqpoint{6.482103in}{1.553657in}}%
\pgfusepath{stroke}%
\end{pgfscope}%
\begin{pgfscope}%
\pgfpathrectangle{\pgfqpoint{0.563921in}{0.521603in}}{\pgfqpoint{6.200000in}{2.642500in}}%
\pgfusepath{clip}%
\pgfsetroundcap%
\pgfsetroundjoin%
\pgfsetlinewidth{1.505625pt}%
\definecolor{currentstroke}{rgb}{0.498039,0.498039,0.498039}%
\pgfsetstrokecolor{currentstroke}%
\pgfsetdash{}{0pt}%
\pgfpathmoveto{\pgfqpoint{0.845739in}{0.641717in}}%
\pgfpathlineto{\pgfqpoint{0.848317in}{0.652443in}}%
\pgfpathlineto{\pgfqpoint{0.850894in}{0.655308in}}%
\pgfpathlineto{\pgfqpoint{0.853471in}{0.656194in}}%
\pgfpathlineto{\pgfqpoint{0.861203in}{0.655643in}}%
\pgfpathlineto{\pgfqpoint{0.863780in}{0.654449in}}%
\pgfpathlineto{\pgfqpoint{0.889552in}{0.653292in}}%
\pgfpathlineto{\pgfqpoint{0.897284in}{0.653691in}}%
\pgfpathlineto{\pgfqpoint{0.907593in}{0.666884in}}%
\pgfpathlineto{\pgfqpoint{0.925633in}{0.668074in}}%
\pgfpathlineto{\pgfqpoint{0.956560in}{0.665628in}}%
\pgfpathlineto{\pgfqpoint{0.961714in}{0.664992in}}%
\pgfpathlineto{\pgfqpoint{0.977177in}{0.664086in}}%
\pgfpathlineto{\pgfqpoint{0.979754in}{0.663791in}}%
\pgfpathlineto{\pgfqpoint{1.010681in}{0.663094in}}%
\pgfpathlineto{\pgfqpoint{1.015835in}{0.663967in}}%
\pgfpathlineto{\pgfqpoint{1.023567in}{0.664577in}}%
\pgfpathlineto{\pgfqpoint{1.033876in}{0.667417in}}%
\pgfpathlineto{\pgfqpoint{1.044185in}{0.668762in}}%
\pgfpathlineto{\pgfqpoint{1.049339in}{0.670046in}}%
\pgfpathlineto{\pgfqpoint{1.059648in}{0.670345in}}%
\pgfpathlineto{\pgfqpoint{1.069957in}{0.669614in}}%
\pgfpathlineto{\pgfqpoint{1.098306in}{0.669303in}}%
\pgfpathlineto{\pgfqpoint{1.134387in}{0.669593in}}%
\pgfpathlineto{\pgfqpoint{1.136964in}{0.670324in}}%
\pgfpathlineto{\pgfqpoint{1.142119in}{0.673480in}}%
\pgfpathlineto{\pgfqpoint{1.149851in}{0.675124in}}%
\pgfpathlineto{\pgfqpoint{1.157582in}{0.680097in}}%
\pgfpathlineto{\pgfqpoint{1.160159in}{0.681052in}}%
\pgfpathlineto{\pgfqpoint{1.167891in}{0.682254in}}%
\pgfpathlineto{\pgfqpoint{1.178200in}{0.686717in}}%
\pgfpathlineto{\pgfqpoint{1.185931in}{0.687900in}}%
\pgfpathlineto{\pgfqpoint{1.196240in}{0.693154in}}%
\pgfpathlineto{\pgfqpoint{1.203972in}{0.694367in}}%
\pgfpathlineto{\pgfqpoint{1.214281in}{0.698635in}}%
\pgfpathlineto{\pgfqpoint{1.224590in}{0.699883in}}%
\pgfpathlineto{\pgfqpoint{1.232321in}{0.702264in}}%
\pgfpathlineto{\pgfqpoint{1.242630in}{0.703483in}}%
\pgfpathlineto{\pgfqpoint{1.250362in}{0.706600in}}%
\pgfpathlineto{\pgfqpoint{1.258093in}{0.708022in}}%
\pgfpathlineto{\pgfqpoint{1.265825in}{0.713642in}}%
\pgfpathlineto{\pgfqpoint{1.268402in}{0.715585in}}%
\pgfpathlineto{\pgfqpoint{1.276134in}{0.717663in}}%
\pgfpathlineto{\pgfqpoint{1.281288in}{0.720927in}}%
\pgfpathlineto{\pgfqpoint{1.286443in}{0.724051in}}%
\pgfpathlineto{\pgfqpoint{1.294174in}{0.725484in}}%
\pgfpathlineto{\pgfqpoint{1.304483in}{0.731590in}}%
\pgfpathlineto{\pgfqpoint{1.312215in}{0.733665in}}%
\pgfpathlineto{\pgfqpoint{1.314792in}{0.735638in}}%
\pgfpathlineto{\pgfqpoint{1.319947in}{0.737460in}}%
\pgfpathlineto{\pgfqpoint{1.322524in}{0.739209in}}%
\pgfpathlineto{\pgfqpoint{1.330255in}{0.741119in}}%
\pgfpathlineto{\pgfqpoint{1.340564in}{0.748244in}}%
\pgfpathlineto{\pgfqpoint{1.348296in}{0.750105in}}%
\pgfpathlineto{\pgfqpoint{1.353450in}{0.754305in}}%
\pgfpathlineto{\pgfqpoint{1.358605in}{0.756463in}}%
\pgfpathlineto{\pgfqpoint{1.371491in}{0.758497in}}%
\pgfpathlineto{\pgfqpoint{1.376645in}{0.760458in}}%
\pgfpathlineto{\pgfqpoint{1.384377in}{0.761527in}}%
\pgfpathlineto{\pgfqpoint{1.392108in}{0.764541in}}%
\pgfpathlineto{\pgfqpoint{1.394686in}{0.765203in}}%
\pgfpathlineto{\pgfqpoint{1.404995in}{0.766557in}}%
\pgfpathlineto{\pgfqpoint{1.412726in}{0.768131in}}%
\pgfpathlineto{\pgfqpoint{1.425612in}{0.769391in}}%
\pgfpathlineto{\pgfqpoint{1.430767in}{0.770022in}}%
\pgfpathlineto{\pgfqpoint{1.443653in}{0.769969in}}%
\pgfpathlineto{\pgfqpoint{1.448807in}{0.769746in}}%
\pgfpathlineto{\pgfqpoint{1.482311in}{0.769734in}}%
\pgfpathlineto{\pgfqpoint{1.497774in}{0.770697in}}%
\pgfpathlineto{\pgfqpoint{1.502929in}{0.771684in}}%
\pgfpathlineto{\pgfqpoint{1.513237in}{0.772373in}}%
\pgfpathlineto{\pgfqpoint{1.520969in}{0.774062in}}%
\pgfpathlineto{\pgfqpoint{1.531278in}{0.775230in}}%
\pgfpathlineto{\pgfqpoint{1.539010in}{0.776837in}}%
\pgfpathlineto{\pgfqpoint{1.549318in}{0.777961in}}%
\pgfpathlineto{\pgfqpoint{1.554473in}{0.779756in}}%
\pgfpathlineto{\pgfqpoint{1.557050in}{0.780954in}}%
\pgfpathlineto{\pgfqpoint{1.567359in}{0.782625in}}%
\pgfpathlineto{\pgfqpoint{1.572513in}{0.783579in}}%
\pgfpathlineto{\pgfqpoint{1.582822in}{0.783915in}}%
\pgfpathlineto{\pgfqpoint{1.587977in}{0.784150in}}%
\pgfpathlineto{\pgfqpoint{1.593131in}{0.784959in}}%
\pgfpathlineto{\pgfqpoint{1.642098in}{0.785673in}}%
\pgfpathlineto{\pgfqpoint{1.665293in}{0.783779in}}%
\pgfpathlineto{\pgfqpoint{1.729723in}{0.782377in}}%
\pgfpathlineto{\pgfqpoint{1.745187in}{0.782402in}}%
\pgfpathlineto{\pgfqpoint{1.863738in}{0.778461in}}%
\pgfpathlineto{\pgfqpoint{1.917860in}{0.779533in}}%
\pgfpathlineto{\pgfqpoint{1.928169in}{0.780169in}}%
\pgfpathlineto{\pgfqpoint{1.951364in}{0.784097in}}%
\pgfpathlineto{\pgfqpoint{1.953941in}{0.784877in}}%
\pgfpathlineto{\pgfqpoint{1.961672in}{0.785595in}}%
\pgfpathlineto{\pgfqpoint{1.971981in}{0.788789in}}%
\pgfpathlineto{\pgfqpoint{1.979713in}{0.789715in}}%
\pgfpathlineto{\pgfqpoint{1.990022in}{0.793345in}}%
\pgfpathlineto{\pgfqpoint{1.997753in}{0.794306in}}%
\pgfpathlineto{\pgfqpoint{2.005485in}{0.797099in}}%
\pgfpathlineto{\pgfqpoint{2.015794in}{0.797999in}}%
\pgfpathlineto{\pgfqpoint{2.026103in}{0.801569in}}%
\pgfpathlineto{\pgfqpoint{2.033834in}{0.802604in}}%
\pgfpathlineto{\pgfqpoint{2.041566in}{0.806045in}}%
\pgfpathlineto{\pgfqpoint{2.044143in}{0.807465in}}%
\pgfpathlineto{\pgfqpoint{2.051875in}{0.808776in}}%
\pgfpathlineto{\pgfqpoint{2.062184in}{0.814009in}}%
\pgfpathlineto{\pgfqpoint{2.069915in}{0.815723in}}%
\pgfpathlineto{\pgfqpoint{2.080224in}{0.823040in}}%
\pgfpathlineto{\pgfqpoint{2.087956in}{0.825026in}}%
\pgfpathlineto{\pgfqpoint{2.093110in}{0.828656in}}%
\pgfpathlineto{\pgfqpoint{2.098265in}{0.831729in}}%
\pgfpathlineto{\pgfqpoint{2.105996in}{0.833035in}}%
\pgfpathlineto{\pgfqpoint{2.116305in}{0.839084in}}%
\pgfpathlineto{\pgfqpoint{2.124037in}{0.840407in}}%
\pgfpathlineto{\pgfqpoint{2.134346in}{0.846402in}}%
\pgfpathlineto{\pgfqpoint{2.142077in}{0.847647in}}%
\pgfpathlineto{\pgfqpoint{2.152386in}{0.851302in}}%
\pgfpathlineto{\pgfqpoint{2.185890in}{0.853830in}}%
\pgfpathlineto{\pgfqpoint{2.188467in}{0.854307in}}%
\pgfpathlineto{\pgfqpoint{2.201353in}{0.855668in}}%
\pgfpathlineto{\pgfqpoint{2.206508in}{0.856799in}}%
\pgfpathlineto{\pgfqpoint{2.216816in}{0.858014in}}%
\pgfpathlineto{\pgfqpoint{2.224548in}{0.858845in}}%
\pgfpathlineto{\pgfqpoint{2.234857in}{0.859474in}}%
\pgfpathlineto{\pgfqpoint{2.242589in}{0.860841in}}%
\pgfpathlineto{\pgfqpoint{2.252897in}{0.861652in}}%
\pgfpathlineto{\pgfqpoint{2.260629in}{0.862688in}}%
\pgfpathlineto{\pgfqpoint{2.270938in}{0.863861in}}%
\pgfpathlineto{\pgfqpoint{2.278670in}{0.865350in}}%
\pgfpathlineto{\pgfqpoint{2.288978in}{0.866083in}}%
\pgfpathlineto{\pgfqpoint{2.296710in}{0.867195in}}%
\pgfpathlineto{\pgfqpoint{2.309596in}{0.868309in}}%
\pgfpathlineto{\pgfqpoint{2.314751in}{0.869213in}}%
\pgfpathlineto{\pgfqpoint{2.366295in}{0.872525in}}%
\pgfpathlineto{\pgfqpoint{2.430725in}{0.870682in}}%
\pgfpathlineto{\pgfqpoint{2.441034in}{0.870210in}}%
\pgfpathlineto{\pgfqpoint{2.477115in}{0.869738in}}%
\pgfpathlineto{\pgfqpoint{2.520928in}{0.868374in}}%
\pgfpathlineto{\pgfqpoint{2.528659in}{0.868271in}}%
\pgfpathlineto{\pgfqpoint{2.567317in}{0.871773in}}%
\pgfpathlineto{\pgfqpoint{2.600821in}{0.873855in}}%
\pgfpathlineto{\pgfqpoint{2.621439in}{0.875250in}}%
\pgfpathlineto{\pgfqpoint{2.683292in}{0.876126in}}%
\pgfpathlineto{\pgfqpoint{2.701332in}{0.875902in}}%
\pgfpathlineto{\pgfqpoint{2.755454in}{0.875847in}}%
\pgfpathlineto{\pgfqpoint{2.765763in}{0.875741in}}%
\pgfpathlineto{\pgfqpoint{2.809575in}{0.875018in}}%
\pgfpathlineto{\pgfqpoint{2.837925in}{0.873980in}}%
\pgfpathlineto{\pgfqpoint{2.889469in}{0.872631in}}%
\pgfpathlineto{\pgfqpoint{2.946168in}{0.870717in}}%
\pgfpathlineto{\pgfqpoint{3.031216in}{0.869317in}}%
\pgfpathlineto{\pgfqpoint{3.054411in}{0.869083in}}%
\pgfpathlineto{\pgfqpoint{3.100800in}{0.869578in}}%
\pgfpathlineto{\pgfqpoint{3.126572in}{0.870397in}}%
\pgfpathlineto{\pgfqpoint{3.214198in}{0.872007in}}%
\pgfpathlineto{\pgfqpoint{3.234815in}{0.873328in}}%
\pgfpathlineto{\pgfqpoint{3.250279in}{0.874302in}}%
\pgfpathlineto{\pgfqpoint{3.268319in}{0.875760in}}%
\pgfpathlineto{\pgfqpoint{3.288937in}{0.876202in}}%
\pgfpathlineto{\pgfqpoint{3.353367in}{0.876795in}}%
\pgfpathlineto{\pgfqpoint{3.376562in}{0.877181in}}%
\pgfpathlineto{\pgfqpoint{3.397180in}{0.877400in}}%
\pgfpathlineto{\pgfqpoint{3.531195in}{0.879886in}}%
\pgfpathlineto{\pgfqpoint{3.559544in}{0.881514in}}%
\pgfpathlineto{\pgfqpoint{3.613666in}{0.881494in}}%
\pgfpathlineto{\pgfqpoint{3.770876in}{0.879107in}}%
\pgfpathlineto{\pgfqpoint{3.794070in}{0.879409in}}%
\pgfpathlineto{\pgfqpoint{3.899736in}{0.880273in}}%
\pgfpathlineto{\pgfqpoint{3.951280in}{0.881154in}}%
\pgfpathlineto{\pgfqpoint{3.992516in}{0.882782in}}%
\pgfpathlineto{\pgfqpoint{4.093027in}{0.883607in}}%
\pgfpathlineto{\pgfqpoint{4.154880in}{0.882571in}}%
\pgfpathlineto{\pgfqpoint{4.216733in}{0.881719in}}%
\pgfpathlineto{\pgfqpoint{4.263123in}{0.881085in}}%
\pgfpathlineto{\pgfqpoint{4.314667in}{0.879743in}}%
\pgfpathlineto{\pgfqpoint{4.407447in}{0.876813in}}%
\pgfpathlineto{\pgfqpoint{4.492495in}{0.875158in}}%
\pgfpathlineto{\pgfqpoint{4.569811in}{0.873472in}}%
\pgfpathlineto{\pgfqpoint{4.685786in}{0.872971in}}%
\pgfpathlineto{\pgfqpoint{4.714135in}{0.874893in}}%
\pgfpathlineto{\pgfqpoint{4.729599in}{0.875571in}}%
\pgfpathlineto{\pgfqpoint{4.750216in}{0.877105in}}%
\pgfpathlineto{\pgfqpoint{4.763102in}{0.877996in}}%
\pgfpathlineto{\pgfqpoint{4.768257in}{0.878655in}}%
\pgfpathlineto{\pgfqpoint{4.781143in}{0.879774in}}%
\pgfpathlineto{\pgfqpoint{4.786297in}{0.880526in}}%
\pgfpathlineto{\pgfqpoint{4.796606in}{0.881305in}}%
\pgfpathlineto{\pgfqpoint{4.804338in}{0.882693in}}%
\pgfpathlineto{\pgfqpoint{4.817224in}{0.884001in}}%
\pgfpathlineto{\pgfqpoint{4.819801in}{0.884471in}}%
\pgfpathlineto{\pgfqpoint{4.832687in}{0.885439in}}%
\pgfpathlineto{\pgfqpoint{4.840419in}{0.886971in}}%
\pgfpathlineto{\pgfqpoint{4.850728in}{0.888023in}}%
\pgfpathlineto{\pgfqpoint{4.858459in}{0.889228in}}%
\pgfpathlineto{\pgfqpoint{4.871345in}{0.890133in}}%
\pgfpathlineto{\pgfqpoint{4.891963in}{0.891785in}}%
\pgfpathlineto{\pgfqpoint{4.912581in}{0.893136in}}%
\pgfpathlineto{\pgfqpoint{4.938353in}{0.894477in}}%
\pgfpathlineto{\pgfqpoint{4.948662in}{0.895443in}}%
\pgfpathlineto{\pgfqpoint{5.000206in}{0.897463in}}%
\pgfpathlineto{\pgfqpoint{5.049173in}{0.901709in}}%
\pgfpathlineto{\pgfqpoint{5.056905in}{0.903157in}}%
\pgfpathlineto{\pgfqpoint{5.067213in}{0.904192in}}%
\pgfpathlineto{\pgfqpoint{5.074945in}{0.906030in}}%
\pgfpathlineto{\pgfqpoint{5.087831in}{0.907462in}}%
\pgfpathlineto{\pgfqpoint{5.092986in}{0.908762in}}%
\pgfpathlineto{\pgfqpoint{5.103294in}{0.910060in}}%
\pgfpathlineto{\pgfqpoint{5.111026in}{0.912043in}}%
\pgfpathlineto{\pgfqpoint{5.121335in}{0.913328in}}%
\pgfpathlineto{\pgfqpoint{5.129067in}{0.915185in}}%
\pgfpathlineto{\pgfqpoint{5.139375in}{0.916328in}}%
\pgfpathlineto{\pgfqpoint{5.147107in}{0.917967in}}%
\pgfpathlineto{\pgfqpoint{5.159993in}{0.919258in}}%
\pgfpathlineto{\pgfqpoint{5.165148in}{0.920040in}}%
\pgfpathlineto{\pgfqpoint{5.178034in}{0.921173in}}%
\pgfpathlineto{\pgfqpoint{5.183188in}{0.921930in}}%
\pgfpathlineto{\pgfqpoint{5.198651in}{0.923193in}}%
\pgfpathlineto{\pgfqpoint{5.219269in}{0.924745in}}%
\pgfpathlineto{\pgfqpoint{5.234732in}{0.925727in}}%
\pgfpathlineto{\pgfqpoint{5.237309in}{0.926007in}}%
\pgfpathlineto{\pgfqpoint{5.252773in}{0.927015in}}%
\pgfpathlineto{\pgfqpoint{5.273390in}{0.928181in}}%
\pgfpathlineto{\pgfqpoint{5.288854in}{0.928888in}}%
\pgfpathlineto{\pgfqpoint{5.304317in}{0.929758in}}%
\pgfpathlineto{\pgfqpoint{5.319780in}{0.930468in}}%
\pgfpathlineto{\pgfqpoint{5.358438in}{0.931653in}}%
\pgfpathlineto{\pgfqpoint{5.389365in}{0.931663in}}%
\pgfpathlineto{\pgfqpoint{5.453795in}{0.931432in}}%
\pgfpathlineto{\pgfqpoint{5.497608in}{0.932770in}}%
\pgfpathlineto{\pgfqpoint{5.525957in}{0.935011in}}%
\pgfpathlineto{\pgfqpoint{5.541421in}{0.935937in}}%
\pgfpathlineto{\pgfqpoint{5.543998in}{0.936224in}}%
\pgfpathlineto{\pgfqpoint{5.556884in}{0.936986in}}%
\pgfpathlineto{\pgfqpoint{5.562038in}{0.937690in}}%
\pgfpathlineto{\pgfqpoint{5.577501in}{0.938750in}}%
\pgfpathlineto{\pgfqpoint{5.580079in}{0.939005in}}%
\pgfpathlineto{\pgfqpoint{5.598119in}{0.940005in}}%
\pgfpathlineto{\pgfqpoint{5.634200in}{0.940603in}}%
\pgfpathlineto{\pgfqpoint{5.696053in}{0.941056in}}%
\pgfpathlineto{\pgfqpoint{5.778524in}{0.942324in}}%
\pgfpathlineto{\pgfqpoint{5.868727in}{0.942140in}}%
\pgfpathlineto{\pgfqpoint{5.912539in}{0.941333in}}%
\pgfpathlineto{\pgfqpoint{5.953775in}{0.940618in}}%
\pgfpathlineto{\pgfqpoint{6.085212in}{0.937878in}}%
\pgfpathlineto{\pgfqpoint{6.245000in}{0.939179in}}%
\pgfpathlineto{\pgfqpoint{6.291389in}{0.940050in}}%
\pgfpathlineto{\pgfqpoint{6.348088in}{0.940686in}}%
\pgfpathlineto{\pgfqpoint{6.399632in}{0.940051in}}%
\pgfpathlineto{\pgfqpoint{6.422827in}{0.941173in}}%
\pgfpathlineto{\pgfqpoint{6.456331in}{0.943534in}}%
\pgfpathlineto{\pgfqpoint{6.464063in}{0.944445in}}%
\pgfpathlineto{\pgfqpoint{6.479526in}{0.945416in}}%
\pgfpathlineto{\pgfqpoint{6.482103in}{0.945709in}}%
\pgfpathlineto{\pgfqpoint{6.482103in}{0.945709in}}%
\pgfusepath{stroke}%
\end{pgfscope}%
\begin{pgfscope}%
\pgfpathrectangle{\pgfqpoint{0.563921in}{0.521603in}}{\pgfqpoint{6.200000in}{2.642500in}}%
\pgfusepath{clip}%
\pgfsetroundcap%
\pgfsetroundjoin%
\pgfsetlinewidth{1.505625pt}%
\definecolor{currentstroke}{rgb}{0.737255,0.741176,0.133333}%
\pgfsetstrokecolor{currentstroke}%
\pgfsetdash{}{0pt}%
\pgfpathmoveto{\pgfqpoint{0.845739in}{0.641717in}}%
\pgfpathlineto{\pgfqpoint{0.848317in}{0.653933in}}%
\pgfpathlineto{\pgfqpoint{0.850894in}{0.651739in}}%
\pgfpathlineto{\pgfqpoint{0.853471in}{0.653065in}}%
\pgfpathlineto{\pgfqpoint{0.863780in}{0.656339in}}%
\pgfpathlineto{\pgfqpoint{0.866357in}{0.657935in}}%
\pgfpathlineto{\pgfqpoint{0.871512in}{0.656214in}}%
\pgfpathlineto{\pgfqpoint{0.881820in}{0.656987in}}%
\pgfpathlineto{\pgfqpoint{0.884398in}{0.659578in}}%
\pgfpathlineto{\pgfqpoint{0.889552in}{0.658794in}}%
\pgfpathlineto{\pgfqpoint{0.897284in}{0.659164in}}%
\pgfpathlineto{\pgfqpoint{0.907593in}{0.657250in}}%
\pgfpathlineto{\pgfqpoint{0.920479in}{0.657087in}}%
\pgfpathlineto{\pgfqpoint{0.925633in}{0.666625in}}%
\pgfpathlineto{\pgfqpoint{0.933365in}{0.670374in}}%
\pgfpathlineto{\pgfqpoint{0.935942in}{0.672749in}}%
\pgfpathlineto{\pgfqpoint{0.938519in}{0.676044in}}%
\pgfpathlineto{\pgfqpoint{0.943674in}{0.691156in}}%
\pgfpathlineto{\pgfqpoint{0.951405in}{0.695499in}}%
\pgfpathlineto{\pgfqpoint{0.956560in}{0.706877in}}%
\pgfpathlineto{\pgfqpoint{0.961714in}{0.713774in}}%
\pgfpathlineto{\pgfqpoint{0.972023in}{0.716230in}}%
\pgfpathlineto{\pgfqpoint{0.979754in}{0.726012in}}%
\pgfpathlineto{\pgfqpoint{0.987486in}{0.728351in}}%
\pgfpathlineto{\pgfqpoint{0.990063in}{0.731584in}}%
\pgfpathlineto{\pgfqpoint{0.997795in}{0.735445in}}%
\pgfpathlineto{\pgfqpoint{1.010681in}{0.736817in}}%
\pgfpathlineto{\pgfqpoint{1.015835in}{0.738670in}}%
\pgfpathlineto{\pgfqpoint{1.026144in}{0.739602in}}%
\pgfpathlineto{\pgfqpoint{1.041608in}{0.741138in}}%
\pgfpathlineto{\pgfqpoint{1.049339in}{0.741326in}}%
\pgfpathlineto{\pgfqpoint{1.051916in}{0.741812in}}%
\pgfpathlineto{\pgfqpoint{1.085420in}{0.745347in}}%
\pgfpathlineto{\pgfqpoint{1.098306in}{0.745152in}}%
\pgfpathlineto{\pgfqpoint{1.100883in}{0.744887in}}%
\pgfpathlineto{\pgfqpoint{1.118924in}{0.747420in}}%
\pgfpathlineto{\pgfqpoint{1.124078in}{0.747832in}}%
\pgfpathlineto{\pgfqpoint{1.136964in}{0.747585in}}%
\pgfpathlineto{\pgfqpoint{1.142119in}{0.748648in}}%
\pgfpathlineto{\pgfqpoint{1.157582in}{0.749064in}}%
\pgfpathlineto{\pgfqpoint{1.160159in}{0.748650in}}%
\pgfpathlineto{\pgfqpoint{1.170468in}{0.747928in}}%
\pgfpathlineto{\pgfqpoint{1.178200in}{0.746731in}}%
\pgfpathlineto{\pgfqpoint{1.191086in}{0.745452in}}%
\pgfpathlineto{\pgfqpoint{1.196240in}{0.744416in}}%
\pgfpathlineto{\pgfqpoint{1.240053in}{0.741590in}}%
\pgfpathlineto{\pgfqpoint{1.250362in}{0.740051in}}%
\pgfpathlineto{\pgfqpoint{1.263248in}{0.738963in}}%
\pgfpathlineto{\pgfqpoint{1.268402in}{0.738400in}}%
\pgfpathlineto{\pgfqpoint{1.283866in}{0.738495in}}%
\pgfpathlineto{\pgfqpoint{1.286443in}{0.739059in}}%
\pgfpathlineto{\pgfqpoint{1.301906in}{0.739627in}}%
\pgfpathlineto{\pgfqpoint{1.319947in}{0.741993in}}%
\pgfpathlineto{\pgfqpoint{1.322524in}{0.742409in}}%
\pgfpathlineto{\pgfqpoint{1.348296in}{0.743319in}}%
\pgfpathlineto{\pgfqpoint{1.356027in}{0.745027in}}%
\pgfpathlineto{\pgfqpoint{1.358605in}{0.745347in}}%
\pgfpathlineto{\pgfqpoint{1.374068in}{0.745779in}}%
\pgfpathlineto{\pgfqpoint{1.376645in}{0.746511in}}%
\pgfpathlineto{\pgfqpoint{1.389531in}{0.748565in}}%
\pgfpathlineto{\pgfqpoint{1.394686in}{0.750083in}}%
\pgfpathlineto{\pgfqpoint{1.404995in}{0.751653in}}%
\pgfpathlineto{\pgfqpoint{1.410149in}{0.752868in}}%
\pgfpathlineto{\pgfqpoint{1.412726in}{0.753334in}}%
\pgfpathlineto{\pgfqpoint{1.423035in}{0.754221in}}%
\pgfpathlineto{\pgfqpoint{1.430767in}{0.755761in}}%
\pgfpathlineto{\pgfqpoint{1.459116in}{0.757492in}}%
\pgfpathlineto{\pgfqpoint{1.466848in}{0.758128in}}%
\pgfpathlineto{\pgfqpoint{1.482311in}{0.758944in}}%
\pgfpathlineto{\pgfqpoint{1.484888in}{0.759313in}}%
\pgfpathlineto{\pgfqpoint{1.495197in}{0.760112in}}%
\pgfpathlineto{\pgfqpoint{1.502929in}{0.762717in}}%
\pgfpathlineto{\pgfqpoint{1.513237in}{0.764189in}}%
\pgfpathlineto{\pgfqpoint{1.520969in}{0.766611in}}%
\pgfpathlineto{\pgfqpoint{1.531278in}{0.767940in}}%
\pgfpathlineto{\pgfqpoint{1.549318in}{0.771256in}}%
\pgfpathlineto{\pgfqpoint{1.557050in}{0.774829in}}%
\pgfpathlineto{\pgfqpoint{1.567359in}{0.776559in}}%
\pgfpathlineto{\pgfqpoint{1.575091in}{0.779255in}}%
\pgfpathlineto{\pgfqpoint{1.582822in}{0.780275in}}%
\pgfpathlineto{\pgfqpoint{1.593131in}{0.784821in}}%
\pgfpathlineto{\pgfqpoint{1.606017in}{0.786517in}}%
\pgfpathlineto{\pgfqpoint{1.611172in}{0.787670in}}%
\pgfpathlineto{\pgfqpoint{1.624058in}{0.788308in}}%
\pgfpathlineto{\pgfqpoint{1.629212in}{0.790676in}}%
\pgfpathlineto{\pgfqpoint{1.636944in}{0.791549in}}%
\pgfpathlineto{\pgfqpoint{1.644675in}{0.794474in}}%
\pgfpathlineto{\pgfqpoint{1.647253in}{0.795409in}}%
\pgfpathlineto{\pgfqpoint{1.657561in}{0.797183in}}%
\pgfpathlineto{\pgfqpoint{1.675602in}{0.802070in}}%
\pgfpathlineto{\pgfqpoint{1.678179in}{0.803401in}}%
\pgfpathlineto{\pgfqpoint{1.683333in}{0.804886in}}%
\pgfpathlineto{\pgfqpoint{1.691065in}{0.806213in}}%
\pgfpathlineto{\pgfqpoint{1.701374in}{0.811544in}}%
\pgfpathlineto{\pgfqpoint{1.709106in}{0.812866in}}%
\pgfpathlineto{\pgfqpoint{1.719414in}{0.817589in}}%
\pgfpathlineto{\pgfqpoint{1.727146in}{0.818748in}}%
\pgfpathlineto{\pgfqpoint{1.734878in}{0.821836in}}%
\pgfpathlineto{\pgfqpoint{1.737455in}{0.822689in}}%
\pgfpathlineto{\pgfqpoint{1.745187in}{0.823766in}}%
\pgfpathlineto{\pgfqpoint{1.755495in}{0.828519in}}%
\pgfpathlineto{\pgfqpoint{1.768381in}{0.830652in}}%
\pgfpathlineto{\pgfqpoint{1.773536in}{0.832395in}}%
\pgfpathlineto{\pgfqpoint{1.781268in}{0.833491in}}%
\pgfpathlineto{\pgfqpoint{1.786422in}{0.834982in}}%
\pgfpathlineto{\pgfqpoint{1.791576in}{0.838028in}}%
\pgfpathlineto{\pgfqpoint{1.799308in}{0.839702in}}%
\pgfpathlineto{\pgfqpoint{1.809617in}{0.847322in}}%
\pgfpathlineto{\pgfqpoint{1.817349in}{0.849095in}}%
\pgfpathlineto{\pgfqpoint{1.827657in}{0.855340in}}%
\pgfpathlineto{\pgfqpoint{1.837966in}{0.856736in}}%
\pgfpathlineto{\pgfqpoint{1.845698in}{0.860869in}}%
\pgfpathlineto{\pgfqpoint{1.853430in}{0.861876in}}%
\pgfpathlineto{\pgfqpoint{1.863738in}{0.865943in}}%
\pgfpathlineto{\pgfqpoint{1.871470in}{0.866869in}}%
\pgfpathlineto{\pgfqpoint{1.881779in}{0.871137in}}%
\pgfpathlineto{\pgfqpoint{1.892088in}{0.872748in}}%
\pgfpathlineto{\pgfqpoint{1.899819in}{0.875289in}}%
\pgfpathlineto{\pgfqpoint{1.912705in}{0.876939in}}%
\pgfpathlineto{\pgfqpoint{1.917860in}{0.878837in}}%
\pgfpathlineto{\pgfqpoint{1.925591in}{0.879547in}}%
\pgfpathlineto{\pgfqpoint{1.935900in}{0.883099in}}%
\pgfpathlineto{\pgfqpoint{1.943632in}{0.884016in}}%
\pgfpathlineto{\pgfqpoint{1.953941in}{0.887863in}}%
\pgfpathlineto{\pgfqpoint{1.964250in}{0.889653in}}%
\pgfpathlineto{\pgfqpoint{1.971981in}{0.891982in}}%
\pgfpathlineto{\pgfqpoint{1.982290in}{0.893105in}}%
\pgfpathlineto{\pgfqpoint{1.990022in}{0.895126in}}%
\pgfpathlineto{\pgfqpoint{1.997753in}{0.896148in}}%
\pgfpathlineto{\pgfqpoint{2.005485in}{0.900288in}}%
\pgfpathlineto{\pgfqpoint{2.015794in}{0.901499in}}%
\pgfpathlineto{\pgfqpoint{2.023526in}{0.904759in}}%
\pgfpathlineto{\pgfqpoint{2.026103in}{0.905705in}}%
\pgfpathlineto{\pgfqpoint{2.036412in}{0.907603in}}%
\pgfpathlineto{\pgfqpoint{2.044143in}{0.910604in}}%
\pgfpathlineto{\pgfqpoint{2.051875in}{0.911178in}}%
\pgfpathlineto{\pgfqpoint{2.062184in}{0.913833in}}%
\pgfpathlineto{\pgfqpoint{2.069915in}{0.914496in}}%
\pgfpathlineto{\pgfqpoint{2.080224in}{0.918021in}}%
\pgfpathlineto{\pgfqpoint{2.087956in}{0.918913in}}%
\pgfpathlineto{\pgfqpoint{2.095687in}{0.921988in}}%
\pgfpathlineto{\pgfqpoint{2.098265in}{0.923757in}}%
\pgfpathlineto{\pgfqpoint{2.105996in}{0.925433in}}%
\pgfpathlineto{\pgfqpoint{2.116305in}{0.931892in}}%
\pgfpathlineto{\pgfqpoint{2.124037in}{0.933419in}}%
\pgfpathlineto{\pgfqpoint{2.134346in}{0.940512in}}%
\pgfpathlineto{\pgfqpoint{2.142077in}{0.942226in}}%
\pgfpathlineto{\pgfqpoint{2.149809in}{0.946771in}}%
\pgfpathlineto{\pgfqpoint{2.152386in}{0.948273in}}%
\pgfpathlineto{\pgfqpoint{2.162695in}{0.949713in}}%
\pgfpathlineto{\pgfqpoint{2.170427in}{0.953676in}}%
\pgfpathlineto{\pgfqpoint{2.178158in}{0.955034in}}%
\pgfpathlineto{\pgfqpoint{2.188467in}{0.959941in}}%
\pgfpathlineto{\pgfqpoint{2.196199in}{0.961351in}}%
\pgfpathlineto{\pgfqpoint{2.206508in}{0.966224in}}%
\pgfpathlineto{\pgfqpoint{2.214239in}{0.967524in}}%
\pgfpathlineto{\pgfqpoint{2.221971in}{0.971174in}}%
\pgfpathlineto{\pgfqpoint{2.224548in}{0.972195in}}%
\pgfpathlineto{\pgfqpoint{2.232280in}{0.973120in}}%
\pgfpathlineto{\pgfqpoint{2.242589in}{0.977792in}}%
\pgfpathlineto{\pgfqpoint{2.250320in}{0.979099in}}%
\pgfpathlineto{\pgfqpoint{2.255475in}{0.981785in}}%
\pgfpathlineto{\pgfqpoint{2.260629in}{0.983441in}}%
\pgfpathlineto{\pgfqpoint{2.268361in}{0.984869in}}%
\pgfpathlineto{\pgfqpoint{2.278670in}{0.990512in}}%
\pgfpathlineto{\pgfqpoint{2.286401in}{0.991981in}}%
\pgfpathlineto{\pgfqpoint{2.296710in}{0.997548in}}%
\pgfpathlineto{\pgfqpoint{2.304442in}{0.998971in}}%
\pgfpathlineto{\pgfqpoint{2.314751in}{1.004358in}}%
\pgfpathlineto{\pgfqpoint{2.322482in}{1.005725in}}%
\pgfpathlineto{\pgfqpoint{2.325059in}{1.007035in}}%
\pgfpathlineto{\pgfqpoint{2.366295in}{1.013948in}}%
\pgfpathlineto{\pgfqpoint{2.404953in}{1.017602in}}%
\pgfpathlineto{\pgfqpoint{2.420416in}{1.018583in}}%
\pgfpathlineto{\pgfqpoint{2.422993in}{1.018908in}}%
\pgfpathlineto{\pgfqpoint{2.433302in}{1.019995in}}%
\pgfpathlineto{\pgfqpoint{2.441034in}{1.022342in}}%
\pgfpathlineto{\pgfqpoint{2.448766in}{1.023241in}}%
\pgfpathlineto{\pgfqpoint{2.456497in}{1.026545in}}%
\pgfpathlineto{\pgfqpoint{2.459074in}{1.027954in}}%
\pgfpathlineto{\pgfqpoint{2.466806in}{1.029181in}}%
\pgfpathlineto{\pgfqpoint{2.477115in}{1.033142in}}%
\pgfpathlineto{\pgfqpoint{2.484847in}{1.034001in}}%
\pgfpathlineto{\pgfqpoint{2.495155in}{1.037317in}}%
\pgfpathlineto{\pgfqpoint{2.508041in}{1.038716in}}%
\pgfpathlineto{\pgfqpoint{2.513196in}{1.040212in}}%
\pgfpathlineto{\pgfqpoint{2.523505in}{1.041848in}}%
\pgfpathlineto{\pgfqpoint{2.531236in}{1.045142in}}%
\pgfpathlineto{\pgfqpoint{2.538968in}{1.046318in}}%
\pgfpathlineto{\pgfqpoint{2.549277in}{1.051140in}}%
\pgfpathlineto{\pgfqpoint{2.557009in}{1.052413in}}%
\pgfpathlineto{\pgfqpoint{2.564740in}{1.055893in}}%
\pgfpathlineto{\pgfqpoint{2.567317in}{1.056886in}}%
\pgfpathlineto{\pgfqpoint{2.575049in}{1.057724in}}%
\pgfpathlineto{\pgfqpoint{2.585358in}{1.061223in}}%
\pgfpathlineto{\pgfqpoint{2.595667in}{1.063014in}}%
\pgfpathlineto{\pgfqpoint{2.603398in}{1.066155in}}%
\pgfpathlineto{\pgfqpoint{2.613707in}{1.067978in}}%
\pgfpathlineto{\pgfqpoint{2.621439in}{1.070931in}}%
\pgfpathlineto{\pgfqpoint{2.629170in}{1.071958in}}%
\pgfpathlineto{\pgfqpoint{2.634325in}{1.074166in}}%
\pgfpathlineto{\pgfqpoint{2.652365in}{1.078317in}}%
\pgfpathlineto{\pgfqpoint{2.657520in}{1.080165in}}%
\pgfpathlineto{\pgfqpoint{2.667829in}{1.081846in}}%
\pgfpathlineto{\pgfqpoint{2.675560in}{1.085030in}}%
\pgfpathlineto{\pgfqpoint{2.683292in}{1.086181in}}%
\pgfpathlineto{\pgfqpoint{2.693601in}{1.092281in}}%
\pgfpathlineto{\pgfqpoint{2.701332in}{1.093925in}}%
\pgfpathlineto{\pgfqpoint{2.703910in}{1.095592in}}%
\pgfpathlineto{\pgfqpoint{2.709064in}{1.097325in}}%
\pgfpathlineto{\pgfqpoint{2.711641in}{1.099003in}}%
\pgfpathlineto{\pgfqpoint{2.719373in}{1.100734in}}%
\pgfpathlineto{\pgfqpoint{2.721950in}{1.102553in}}%
\pgfpathlineto{\pgfqpoint{2.727105in}{1.104247in}}%
\pgfpathlineto{\pgfqpoint{2.729682in}{1.105928in}}%
\pgfpathlineto{\pgfqpoint{2.737413in}{1.107510in}}%
\pgfpathlineto{\pgfqpoint{2.747722in}{1.114073in}}%
\pgfpathlineto{\pgfqpoint{2.755454in}{1.115507in}}%
\pgfpathlineto{\pgfqpoint{2.765763in}{1.122478in}}%
\pgfpathlineto{\pgfqpoint{2.776072in}{1.124572in}}%
\pgfpathlineto{\pgfqpoint{2.781226in}{1.128526in}}%
\pgfpathlineto{\pgfqpoint{2.783803in}{1.129929in}}%
\pgfpathlineto{\pgfqpoint{2.791535in}{1.131059in}}%
\pgfpathlineto{\pgfqpoint{2.801844in}{1.135927in}}%
\pgfpathlineto{\pgfqpoint{2.809575in}{1.136862in}}%
\pgfpathlineto{\pgfqpoint{2.819884in}{1.141300in}}%
\pgfpathlineto{\pgfqpoint{2.827616in}{1.142511in}}%
\pgfpathlineto{\pgfqpoint{2.837925in}{1.148185in}}%
\pgfpathlineto{\pgfqpoint{2.848234in}{1.149684in}}%
\pgfpathlineto{\pgfqpoint{2.855965in}{1.153694in}}%
\pgfpathlineto{\pgfqpoint{2.863697in}{1.155135in}}%
\pgfpathlineto{\pgfqpoint{2.874006in}{1.160768in}}%
\pgfpathlineto{\pgfqpoint{2.881737in}{1.161896in}}%
\pgfpathlineto{\pgfqpoint{2.892046in}{1.166852in}}%
\pgfpathlineto{\pgfqpoint{2.899778in}{1.168109in}}%
\pgfpathlineto{\pgfqpoint{2.907509in}{1.171688in}}%
\pgfpathlineto{\pgfqpoint{2.910087in}{1.172673in}}%
\pgfpathlineto{\pgfqpoint{2.917818in}{1.173783in}}%
\pgfpathlineto{\pgfqpoint{2.928127in}{1.178124in}}%
\pgfpathlineto{\pgfqpoint{2.938436in}{1.179856in}}%
\pgfpathlineto{\pgfqpoint{2.946168in}{1.181786in}}%
\pgfpathlineto{\pgfqpoint{2.956476in}{1.183094in}}%
\pgfpathlineto{\pgfqpoint{2.964208in}{1.184702in}}%
\pgfpathlineto{\pgfqpoint{2.992557in}{1.186024in}}%
\pgfpathlineto{\pgfqpoint{2.997712in}{1.186747in}}%
\pgfpathlineto{\pgfqpoint{3.010598in}{1.187511in}}%
\pgfpathlineto{\pgfqpoint{3.018330in}{1.188289in}}%
\pgfpathlineto{\pgfqpoint{3.033793in}{1.188966in}}%
\pgfpathlineto{\pgfqpoint{3.069874in}{1.192177in}}%
\pgfpathlineto{\pgfqpoint{3.072451in}{1.192543in}}%
\pgfpathlineto{\pgfqpoint{3.085337in}{1.193587in}}%
\pgfpathlineto{\pgfqpoint{3.090491in}{1.194342in}}%
\pgfpathlineto{\pgfqpoint{3.103378in}{1.195300in}}%
\pgfpathlineto{\pgfqpoint{3.108532in}{1.196277in}}%
\pgfpathlineto{\pgfqpoint{3.121418in}{1.197449in}}%
\pgfpathlineto{\pgfqpoint{3.126572in}{1.198241in}}%
\pgfpathlineto{\pgfqpoint{3.136881in}{1.199067in}}%
\pgfpathlineto{\pgfqpoint{3.144613in}{1.200132in}}%
\pgfpathlineto{\pgfqpoint{3.157499in}{1.201061in}}%
\pgfpathlineto{\pgfqpoint{3.162653in}{1.201611in}}%
\pgfpathlineto{\pgfqpoint{3.178117in}{1.202550in}}%
\pgfpathlineto{\pgfqpoint{3.227084in}{1.207492in}}%
\pgfpathlineto{\pgfqpoint{3.234815in}{1.209044in}}%
\pgfpathlineto{\pgfqpoint{3.245124in}{1.210053in}}%
\pgfpathlineto{\pgfqpoint{3.252856in}{1.211475in}}%
\pgfpathlineto{\pgfqpoint{3.268319in}{1.212551in}}%
\pgfpathlineto{\pgfqpoint{3.283782in}{1.213374in}}%
\pgfpathlineto{\pgfqpoint{3.322441in}{1.216036in}}%
\pgfpathlineto{\pgfqpoint{3.325018in}{1.216377in}}%
\pgfpathlineto{\pgfqpoint{3.337904in}{1.217427in}}%
\pgfpathlineto{\pgfqpoint{3.343058in}{1.217940in}}%
\pgfpathlineto{\pgfqpoint{3.358522in}{1.218774in}}%
\pgfpathlineto{\pgfqpoint{3.379139in}{1.220421in}}%
\pgfpathlineto{\pgfqpoint{3.392025in}{1.221269in}}%
\pgfpathlineto{\pgfqpoint{3.397180in}{1.221847in}}%
\pgfpathlineto{\pgfqpoint{3.430684in}{1.223309in}}%
\pgfpathlineto{\pgfqpoint{3.448724in}{1.223868in}}%
\pgfpathlineto{\pgfqpoint{3.484805in}{1.224118in}}%
\pgfpathlineto{\pgfqpoint{3.502845in}{1.225677in}}%
\pgfpathlineto{\pgfqpoint{3.505423in}{1.226664in}}%
\pgfpathlineto{\pgfqpoint{3.513154in}{1.227646in}}%
\pgfpathlineto{\pgfqpoint{3.520886in}{1.231226in}}%
\pgfpathlineto{\pgfqpoint{3.523463in}{1.232593in}}%
\pgfpathlineto{\pgfqpoint{3.531195in}{1.233857in}}%
\pgfpathlineto{\pgfqpoint{3.541504in}{1.239067in}}%
\pgfpathlineto{\pgfqpoint{3.549235in}{1.240338in}}%
\pgfpathlineto{\pgfqpoint{3.559544in}{1.245565in}}%
\pgfpathlineto{\pgfqpoint{3.567276in}{1.246975in}}%
\pgfpathlineto{\pgfqpoint{3.572430in}{1.249976in}}%
\pgfpathlineto{\pgfqpoint{3.577585in}{1.251496in}}%
\pgfpathlineto{\pgfqpoint{3.585316in}{1.253000in}}%
\pgfpathlineto{\pgfqpoint{3.595625in}{1.259456in}}%
\pgfpathlineto{\pgfqpoint{3.603357in}{1.261101in}}%
\pgfpathlineto{\pgfqpoint{3.613666in}{1.267274in}}%
\pgfpathlineto{\pgfqpoint{3.621397in}{1.268593in}}%
\pgfpathlineto{\pgfqpoint{3.626552in}{1.271167in}}%
\pgfpathlineto{\pgfqpoint{3.631706in}{1.274223in}}%
\pgfpathlineto{\pgfqpoint{3.639438in}{1.275784in}}%
\pgfpathlineto{\pgfqpoint{3.644592in}{1.279052in}}%
\pgfpathlineto{\pgfqpoint{3.649747in}{1.280672in}}%
\pgfpathlineto{\pgfqpoint{3.657478in}{1.282231in}}%
\pgfpathlineto{\pgfqpoint{3.662633in}{1.285142in}}%
\pgfpathlineto{\pgfqpoint{3.667787in}{1.286646in}}%
\pgfpathlineto{\pgfqpoint{3.675519in}{1.287909in}}%
\pgfpathlineto{\pgfqpoint{3.685828in}{1.293118in}}%
\pgfpathlineto{\pgfqpoint{3.693559in}{1.294361in}}%
\pgfpathlineto{\pgfqpoint{3.701291in}{1.297658in}}%
\pgfpathlineto{\pgfqpoint{3.703868in}{1.298684in}}%
\pgfpathlineto{\pgfqpoint{3.714177in}{1.299768in}}%
\pgfpathlineto{\pgfqpoint{3.721909in}{1.303060in}}%
\pgfpathlineto{\pgfqpoint{3.732217in}{1.304946in}}%
\pgfpathlineto{\pgfqpoint{3.739949in}{1.307341in}}%
\pgfpathlineto{\pgfqpoint{3.747681in}{1.308304in}}%
\pgfpathlineto{\pgfqpoint{3.755412in}{1.312222in}}%
\pgfpathlineto{\pgfqpoint{3.757990in}{1.313574in}}%
\pgfpathlineto{\pgfqpoint{3.765721in}{1.314842in}}%
\pgfpathlineto{\pgfqpoint{3.773453in}{1.318925in}}%
\pgfpathlineto{\pgfqpoint{3.776030in}{1.320380in}}%
\pgfpathlineto{\pgfqpoint{3.786339in}{1.321871in}}%
\pgfpathlineto{\pgfqpoint{3.794070in}{1.326222in}}%
\pgfpathlineto{\pgfqpoint{3.801802in}{1.327749in}}%
\pgfpathlineto{\pgfqpoint{3.812111in}{1.333680in}}%
\pgfpathlineto{\pgfqpoint{3.819843in}{1.335353in}}%
\pgfpathlineto{\pgfqpoint{3.830151in}{1.341127in}}%
\pgfpathlineto{\pgfqpoint{3.837883in}{1.342478in}}%
\pgfpathlineto{\pgfqpoint{3.848192in}{1.347047in}}%
\pgfpathlineto{\pgfqpoint{3.855924in}{1.348266in}}%
\pgfpathlineto{\pgfqpoint{3.866232in}{1.352813in}}%
\pgfpathlineto{\pgfqpoint{3.873964in}{1.353964in}}%
\pgfpathlineto{\pgfqpoint{3.881696in}{1.356998in}}%
\pgfpathlineto{\pgfqpoint{3.884273in}{1.357918in}}%
\pgfpathlineto{\pgfqpoint{3.892005in}{1.358846in}}%
\pgfpathlineto{\pgfqpoint{3.899736in}{1.361454in}}%
\pgfpathlineto{\pgfqpoint{3.910045in}{1.362299in}}%
\pgfpathlineto{\pgfqpoint{3.920354in}{1.366140in}}%
\pgfpathlineto{\pgfqpoint{3.930663in}{1.367849in}}%
\pgfpathlineto{\pgfqpoint{3.938394in}{1.370267in}}%
\pgfpathlineto{\pgfqpoint{3.948703in}{1.371793in}}%
\pgfpathlineto{\pgfqpoint{3.956435in}{1.375017in}}%
\pgfpathlineto{\pgfqpoint{3.964167in}{1.375996in}}%
\pgfpathlineto{\pgfqpoint{3.974475in}{1.379547in}}%
\pgfpathlineto{\pgfqpoint{3.984784in}{1.381118in}}%
\pgfpathlineto{\pgfqpoint{3.992516in}{1.383971in}}%
\pgfpathlineto{\pgfqpoint{4.000247in}{1.385124in}}%
\pgfpathlineto{\pgfqpoint{4.010556in}{1.389812in}}%
\pgfpathlineto{\pgfqpoint{4.018288in}{1.391068in}}%
\pgfpathlineto{\pgfqpoint{4.028597in}{1.395862in}}%
\pgfpathlineto{\pgfqpoint{4.038906in}{1.396888in}}%
\pgfpathlineto{\pgfqpoint{4.046637in}{1.400186in}}%
\pgfpathlineto{\pgfqpoint{4.054369in}{1.401253in}}%
\pgfpathlineto{\pgfqpoint{4.064678in}{1.405222in}}%
\pgfpathlineto{\pgfqpoint{4.072409in}{1.406083in}}%
\pgfpathlineto{\pgfqpoint{4.082718in}{1.410157in}}%
\pgfpathlineto{\pgfqpoint{4.090450in}{1.411089in}}%
\pgfpathlineto{\pgfqpoint{4.100759in}{1.414950in}}%
\pgfpathlineto{\pgfqpoint{4.108490in}{1.415911in}}%
\pgfpathlineto{\pgfqpoint{4.118799in}{1.419607in}}%
\pgfpathlineto{\pgfqpoint{4.129108in}{1.420982in}}%
\pgfpathlineto{\pgfqpoint{4.134263in}{1.422576in}}%
\pgfpathlineto{\pgfqpoint{4.147149in}{1.424126in}}%
\pgfpathlineto{\pgfqpoint{4.154880in}{1.426244in}}%
\pgfpathlineto{\pgfqpoint{4.162612in}{1.427171in}}%
\pgfpathlineto{\pgfqpoint{4.172921in}{1.431220in}}%
\pgfpathlineto{\pgfqpoint{4.180652in}{1.432497in}}%
\pgfpathlineto{\pgfqpoint{4.190961in}{1.437500in}}%
\pgfpathlineto{\pgfqpoint{4.198693in}{1.438882in}}%
\pgfpathlineto{\pgfqpoint{4.209002in}{1.445252in}}%
\pgfpathlineto{\pgfqpoint{4.216733in}{1.446837in}}%
\pgfpathlineto{\pgfqpoint{4.227042in}{1.452405in}}%
\pgfpathlineto{\pgfqpoint{4.234774in}{1.453756in}}%
\pgfpathlineto{\pgfqpoint{4.245083in}{1.458807in}}%
\pgfpathlineto{\pgfqpoint{4.252814in}{1.460134in}}%
\pgfpathlineto{\pgfqpoint{4.263123in}{1.464922in}}%
\pgfpathlineto{\pgfqpoint{4.273432in}{1.466017in}}%
\pgfpathlineto{\pgfqpoint{4.281164in}{1.468915in}}%
\pgfpathlineto{\pgfqpoint{4.291472in}{1.470452in}}%
\pgfpathlineto{\pgfqpoint{4.299204in}{1.472590in}}%
\pgfpathlineto{\pgfqpoint{4.312090in}{1.474061in}}%
\pgfpathlineto{\pgfqpoint{4.317245in}{1.475598in}}%
\pgfpathlineto{\pgfqpoint{4.327553in}{1.477069in}}%
\pgfpathlineto{\pgfqpoint{4.335285in}{1.479337in}}%
\pgfpathlineto{\pgfqpoint{4.345594in}{1.480811in}}%
\pgfpathlineto{\pgfqpoint{4.353326in}{1.483020in}}%
\pgfpathlineto{\pgfqpoint{4.363634in}{1.483953in}}%
\pgfpathlineto{\pgfqpoint{4.371366in}{1.485945in}}%
\pgfpathlineto{\pgfqpoint{4.381675in}{1.487662in}}%
\pgfpathlineto{\pgfqpoint{4.389407in}{1.490691in}}%
\pgfpathlineto{\pgfqpoint{4.397138in}{1.491832in}}%
\pgfpathlineto{\pgfqpoint{4.407447in}{1.496394in}}%
\pgfpathlineto{\pgfqpoint{4.415179in}{1.497732in}}%
\pgfpathlineto{\pgfqpoint{4.425488in}{1.502684in}}%
\pgfpathlineto{\pgfqpoint{4.433219in}{1.504126in}}%
\pgfpathlineto{\pgfqpoint{4.443528in}{1.509776in}}%
\pgfpathlineto{\pgfqpoint{4.451260in}{1.510830in}}%
\pgfpathlineto{\pgfqpoint{4.461569in}{1.516521in}}%
\pgfpathlineto{\pgfqpoint{4.469300in}{1.517857in}}%
\pgfpathlineto{\pgfqpoint{4.479609in}{1.523534in}}%
\pgfpathlineto{\pgfqpoint{4.487341in}{1.524946in}}%
\pgfpathlineto{\pgfqpoint{4.497649in}{1.530996in}}%
\pgfpathlineto{\pgfqpoint{4.505381in}{1.532509in}}%
\pgfpathlineto{\pgfqpoint{4.510536in}{1.535400in}}%
\pgfpathlineto{\pgfqpoint{4.515690in}{1.536851in}}%
\pgfpathlineto{\pgfqpoint{4.523422in}{1.538195in}}%
\pgfpathlineto{\pgfqpoint{4.533730in}{1.543678in}}%
\pgfpathlineto{\pgfqpoint{4.541462in}{1.545043in}}%
\pgfpathlineto{\pgfqpoint{4.551771in}{1.549638in}}%
\pgfpathlineto{\pgfqpoint{4.559503in}{1.550668in}}%
\pgfpathlineto{\pgfqpoint{4.569811in}{1.555381in}}%
\pgfpathlineto{\pgfqpoint{4.577543in}{1.556410in}}%
\pgfpathlineto{\pgfqpoint{4.585275in}{1.559745in}}%
\pgfpathlineto{\pgfqpoint{4.595584in}{1.560893in}}%
\pgfpathlineto{\pgfqpoint{4.603315in}{1.564245in}}%
\pgfpathlineto{\pgfqpoint{4.613624in}{1.565077in}}%
\pgfpathlineto{\pgfqpoint{4.621356in}{1.567382in}}%
\pgfpathlineto{\pgfqpoint{4.623933in}{1.567941in}}%
\pgfpathlineto{\pgfqpoint{4.634242in}{1.569304in}}%
\pgfpathlineto{\pgfqpoint{4.641973in}{1.570946in}}%
\pgfpathlineto{\pgfqpoint{4.654859in}{1.571721in}}%
\pgfpathlineto{\pgfqpoint{4.660014in}{1.572658in}}%
\pgfpathlineto{\pgfqpoint{4.672900in}{1.573894in}}%
\pgfpathlineto{\pgfqpoint{4.685786in}{1.575428in}}%
\pgfpathlineto{\pgfqpoint{4.696095in}{1.577566in}}%
\pgfpathlineto{\pgfqpoint{4.711558in}{1.578316in}}%
\pgfpathlineto{\pgfqpoint{4.714135in}{1.578624in}}%
\pgfpathlineto{\pgfqpoint{4.727021in}{1.579416in}}%
\pgfpathlineto{\pgfqpoint{4.732176in}{1.580179in}}%
\pgfpathlineto{\pgfqpoint{4.745062in}{1.581532in}}%
\pgfpathlineto{\pgfqpoint{4.750216in}{1.582521in}}%
\pgfpathlineto{\pgfqpoint{4.760525in}{1.583586in}}%
\pgfpathlineto{\pgfqpoint{4.768257in}{1.585283in}}%
\pgfpathlineto{\pgfqpoint{4.783720in}{1.586523in}}%
\pgfpathlineto{\pgfqpoint{4.786297in}{1.586883in}}%
\pgfpathlineto{\pgfqpoint{4.799183in}{1.588014in}}%
\pgfpathlineto{\pgfqpoint{4.804338in}{1.588992in}}%
\pgfpathlineto{\pgfqpoint{4.814647in}{1.589944in}}%
\pgfpathlineto{\pgfqpoint{4.819801in}{1.590989in}}%
\pgfpathlineto{\pgfqpoint{4.832687in}{1.592138in}}%
\pgfpathlineto{\pgfqpoint{4.840419in}{1.594406in}}%
\pgfpathlineto{\pgfqpoint{4.850728in}{1.595977in}}%
\pgfpathlineto{\pgfqpoint{4.858459in}{1.598447in}}%
\pgfpathlineto{\pgfqpoint{4.866191in}{1.599250in}}%
\pgfpathlineto{\pgfqpoint{4.876500in}{1.603104in}}%
\pgfpathlineto{\pgfqpoint{4.884231in}{1.604229in}}%
\pgfpathlineto{\pgfqpoint{4.894540in}{1.608278in}}%
\pgfpathlineto{\pgfqpoint{4.904849in}{1.609874in}}%
\pgfpathlineto{\pgfqpoint{4.912581in}{1.612121in}}%
\pgfpathlineto{\pgfqpoint{4.922890in}{1.613586in}}%
\pgfpathlineto{\pgfqpoint{4.930621in}{1.615638in}}%
\pgfpathlineto{\pgfqpoint{4.938353in}{1.616386in}}%
\pgfpathlineto{\pgfqpoint{4.948662in}{1.619236in}}%
\pgfpathlineto{\pgfqpoint{4.958971in}{1.620556in}}%
\pgfpathlineto{\pgfqpoint{4.966702in}{1.622517in}}%
\pgfpathlineto{\pgfqpoint{4.974434in}{1.623153in}}%
\pgfpathlineto{\pgfqpoint{4.984743in}{1.626363in}}%
\pgfpathlineto{\pgfqpoint{4.995051in}{1.627126in}}%
\pgfpathlineto{\pgfqpoint{5.002783in}{1.629588in}}%
\pgfpathlineto{\pgfqpoint{5.010515in}{1.630475in}}%
\pgfpathlineto{\pgfqpoint{5.020824in}{1.634115in}}%
\pgfpathlineto{\pgfqpoint{5.031132in}{1.635483in}}%
\pgfpathlineto{\pgfqpoint{5.038864in}{1.637349in}}%
\pgfpathlineto{\pgfqpoint{5.049173in}{1.638496in}}%
\pgfpathlineto{\pgfqpoint{5.056905in}{1.640031in}}%
\pgfpathlineto{\pgfqpoint{5.067213in}{1.640700in}}%
\pgfpathlineto{\pgfqpoint{5.074945in}{1.641872in}}%
\pgfpathlineto{\pgfqpoint{5.090408in}{1.642870in}}%
\pgfpathlineto{\pgfqpoint{5.092986in}{1.643341in}}%
\pgfpathlineto{\pgfqpoint{5.103294in}{1.644360in}}%
\pgfpathlineto{\pgfqpoint{5.111026in}{1.646089in}}%
\pgfpathlineto{\pgfqpoint{5.121335in}{1.647312in}}%
\pgfpathlineto{\pgfqpoint{5.129067in}{1.649322in}}%
\pgfpathlineto{\pgfqpoint{5.139375in}{1.650542in}}%
\pgfpathlineto{\pgfqpoint{5.147107in}{1.652320in}}%
\pgfpathlineto{\pgfqpoint{5.157416in}{1.653440in}}%
\pgfpathlineto{\pgfqpoint{5.165148in}{1.655371in}}%
\pgfpathlineto{\pgfqpoint{5.175456in}{1.656730in}}%
\pgfpathlineto{\pgfqpoint{5.183188in}{1.658726in}}%
\pgfpathlineto{\pgfqpoint{5.193497in}{1.660188in}}%
\pgfpathlineto{\pgfqpoint{5.201228in}{1.662306in}}%
\pgfpathlineto{\pgfqpoint{5.211537in}{1.663765in}}%
\pgfpathlineto{\pgfqpoint{5.219269in}{1.665795in}}%
\pgfpathlineto{\pgfqpoint{5.229578in}{1.667232in}}%
\pgfpathlineto{\pgfqpoint{5.237309in}{1.669453in}}%
\pgfpathlineto{\pgfqpoint{5.247618in}{1.670279in}}%
\pgfpathlineto{\pgfqpoint{5.255350in}{1.672730in}}%
\pgfpathlineto{\pgfqpoint{5.265659in}{1.674328in}}%
\pgfpathlineto{\pgfqpoint{5.273390in}{1.676555in}}%
\pgfpathlineto{\pgfqpoint{5.283699in}{1.678124in}}%
\pgfpathlineto{\pgfqpoint{5.291431in}{1.680583in}}%
\pgfpathlineto{\pgfqpoint{5.301740in}{1.682030in}}%
\pgfpathlineto{\pgfqpoint{5.309471in}{1.684305in}}%
\pgfpathlineto{\pgfqpoint{5.319780in}{1.685845in}}%
\pgfpathlineto{\pgfqpoint{5.327512in}{1.688223in}}%
\pgfpathlineto{\pgfqpoint{5.337821in}{1.689691in}}%
\pgfpathlineto{\pgfqpoint{5.345552in}{1.691772in}}%
\pgfpathlineto{\pgfqpoint{5.355861in}{1.693093in}}%
\pgfpathlineto{\pgfqpoint{5.363593in}{1.695207in}}%
\pgfpathlineto{\pgfqpoint{5.373902in}{1.696608in}}%
\pgfpathlineto{\pgfqpoint{5.381633in}{1.698535in}}%
\pgfpathlineto{\pgfqpoint{5.391942in}{1.699820in}}%
\pgfpathlineto{\pgfqpoint{5.399674in}{1.701387in}}%
\pgfpathlineto{\pgfqpoint{5.409983in}{1.702735in}}%
\pgfpathlineto{\pgfqpoint{5.417714in}{1.704653in}}%
\pgfpathlineto{\pgfqpoint{5.428023in}{1.705379in}}%
\pgfpathlineto{\pgfqpoint{5.435755in}{1.706927in}}%
\pgfpathlineto{\pgfqpoint{5.448641in}{1.708399in}}%
\pgfpathlineto{\pgfqpoint{5.471836in}{1.710283in}}%
\pgfpathlineto{\pgfqpoint{5.484722in}{1.711212in}}%
\pgfpathlineto{\pgfqpoint{5.489876in}{1.711985in}}%
\pgfpathlineto{\pgfqpoint{5.502762in}{1.713077in}}%
\pgfpathlineto{\pgfqpoint{5.507917in}{1.713793in}}%
\pgfpathlineto{\pgfqpoint{5.520803in}{1.714716in}}%
\pgfpathlineto{\pgfqpoint{5.525957in}{1.715301in}}%
\pgfpathlineto{\pgfqpoint{5.541421in}{1.716221in}}%
\pgfpathlineto{\pgfqpoint{5.543998in}{1.716506in}}%
\pgfpathlineto{\pgfqpoint{5.556884in}{1.717295in}}%
\pgfpathlineto{\pgfqpoint{5.562038in}{1.718320in}}%
\pgfpathlineto{\pgfqpoint{5.572347in}{1.719316in}}%
\pgfpathlineto{\pgfqpoint{5.580079in}{1.720776in}}%
\pgfpathlineto{\pgfqpoint{5.592965in}{1.721734in}}%
\pgfpathlineto{\pgfqpoint{5.598119in}{1.722731in}}%
\pgfpathlineto{\pgfqpoint{5.608428in}{1.723843in}}%
\pgfpathlineto{\pgfqpoint{5.616160in}{1.725691in}}%
\pgfpathlineto{\pgfqpoint{5.626469in}{1.726848in}}%
\pgfpathlineto{\pgfqpoint{5.667704in}{1.735675in}}%
\pgfpathlineto{\pgfqpoint{5.670281in}{1.736538in}}%
\pgfpathlineto{\pgfqpoint{5.680590in}{1.737433in}}%
\pgfpathlineto{\pgfqpoint{5.688322in}{1.740147in}}%
\pgfpathlineto{\pgfqpoint{5.696053in}{1.741032in}}%
\pgfpathlineto{\pgfqpoint{5.706362in}{1.744712in}}%
\pgfpathlineto{\pgfqpoint{5.714094in}{1.745652in}}%
\pgfpathlineto{\pgfqpoint{5.724403in}{1.749457in}}%
\pgfpathlineto{\pgfqpoint{5.732134in}{1.750470in}}%
\pgfpathlineto{\pgfqpoint{5.742443in}{1.754419in}}%
\pgfpathlineto{\pgfqpoint{5.752752in}{1.756158in}}%
\pgfpathlineto{\pgfqpoint{5.760484in}{1.758792in}}%
\pgfpathlineto{\pgfqpoint{5.768215in}{1.759669in}}%
\pgfpathlineto{\pgfqpoint{5.778524in}{1.763167in}}%
\pgfpathlineto{\pgfqpoint{5.788833in}{1.764895in}}%
\pgfpathlineto{\pgfqpoint{5.796565in}{1.767433in}}%
\pgfpathlineto{\pgfqpoint{5.806873in}{1.769090in}}%
\pgfpathlineto{\pgfqpoint{5.812028in}{1.770720in}}%
\pgfpathlineto{\pgfqpoint{5.822337in}{1.771609in}}%
\pgfpathlineto{\pgfqpoint{5.832646in}{1.775325in}}%
\pgfpathlineto{\pgfqpoint{5.840377in}{1.776355in}}%
\pgfpathlineto{\pgfqpoint{5.850686in}{1.780376in}}%
\pgfpathlineto{\pgfqpoint{5.858418in}{1.781335in}}%
\pgfpathlineto{\pgfqpoint{5.868727in}{1.785501in}}%
\pgfpathlineto{\pgfqpoint{5.876458in}{1.786491in}}%
\pgfpathlineto{\pgfqpoint{5.886767in}{1.790516in}}%
\pgfpathlineto{\pgfqpoint{5.894499in}{1.791588in}}%
\pgfpathlineto{\pgfqpoint{5.904807in}{1.795566in}}%
\pgfpathlineto{\pgfqpoint{5.912539in}{1.796636in}}%
\pgfpathlineto{\pgfqpoint{5.922848in}{1.801286in}}%
\pgfpathlineto{\pgfqpoint{5.933157in}{1.802446in}}%
\pgfpathlineto{\pgfqpoint{5.940888in}{1.806115in}}%
\pgfpathlineto{\pgfqpoint{5.948620in}{1.807407in}}%
\pgfpathlineto{\pgfqpoint{5.958929in}{1.812213in}}%
\pgfpathlineto{\pgfqpoint{5.966661in}{1.813228in}}%
\pgfpathlineto{\pgfqpoint{5.976969in}{1.817621in}}%
\pgfpathlineto{\pgfqpoint{5.984701in}{1.818715in}}%
\pgfpathlineto{\pgfqpoint{5.995010in}{1.822989in}}%
\pgfpathlineto{\pgfqpoint{6.002742in}{1.824094in}}%
\pgfpathlineto{\pgfqpoint{6.013050in}{1.828346in}}%
\pgfpathlineto{\pgfqpoint{6.025936in}{1.830264in}}%
\pgfpathlineto{\pgfqpoint{6.031091in}{1.832152in}}%
\pgfpathlineto{\pgfqpoint{6.038823in}{1.833205in}}%
\pgfpathlineto{\pgfqpoint{6.049131in}{1.837699in}}%
\pgfpathlineto{\pgfqpoint{6.056863in}{1.838867in}}%
\pgfpathlineto{\pgfqpoint{6.067172in}{1.844003in}}%
\pgfpathlineto{\pgfqpoint{6.074904in}{1.845441in}}%
\pgfpathlineto{\pgfqpoint{6.085212in}{1.850971in}}%
\pgfpathlineto{\pgfqpoint{6.092944in}{1.852309in}}%
\pgfpathlineto{\pgfqpoint{6.103253in}{1.858064in}}%
\pgfpathlineto{\pgfqpoint{6.110984in}{1.859538in}}%
\pgfpathlineto{\pgfqpoint{6.121293in}{1.865016in}}%
\pgfpathlineto{\pgfqpoint{6.129025in}{1.866491in}}%
\pgfpathlineto{\pgfqpoint{6.139334in}{1.872643in}}%
\pgfpathlineto{\pgfqpoint{6.147065in}{1.874218in}}%
\pgfpathlineto{\pgfqpoint{6.157374in}{1.880530in}}%
\pgfpathlineto{\pgfqpoint{6.165106in}{1.882119in}}%
\pgfpathlineto{\pgfqpoint{6.175415in}{1.888388in}}%
\pgfpathlineto{\pgfqpoint{6.185724in}{1.889875in}}%
\pgfpathlineto{\pgfqpoint{6.193455in}{1.894570in}}%
\pgfpathlineto{\pgfqpoint{6.201187in}{1.896309in}}%
\pgfpathlineto{\pgfqpoint{6.211496in}{1.903081in}}%
\pgfpathlineto{\pgfqpoint{6.219227in}{1.904663in}}%
\pgfpathlineto{\pgfqpoint{6.229536in}{1.911093in}}%
\pgfpathlineto{\pgfqpoint{6.237268in}{1.912486in}}%
\pgfpathlineto{\pgfqpoint{6.247577in}{1.918386in}}%
\pgfpathlineto{\pgfqpoint{6.255308in}{1.919953in}}%
\pgfpathlineto{\pgfqpoint{6.265617in}{1.926340in}}%
\pgfpathlineto{\pgfqpoint{6.273349in}{1.928010in}}%
\pgfpathlineto{\pgfqpoint{6.283658in}{1.935015in}}%
\pgfpathlineto{\pgfqpoint{6.291389in}{1.936762in}}%
\pgfpathlineto{\pgfqpoint{6.301698in}{1.943383in}}%
\pgfpathlineto{\pgfqpoint{6.309430in}{1.945025in}}%
\pgfpathlineto{\pgfqpoint{6.319739in}{1.952162in}}%
\pgfpathlineto{\pgfqpoint{6.327470in}{1.953988in}}%
\pgfpathlineto{\pgfqpoint{6.337779in}{1.961505in}}%
\pgfpathlineto{\pgfqpoint{6.345511in}{1.963461in}}%
\pgfpathlineto{\pgfqpoint{6.355820in}{1.971281in}}%
\pgfpathlineto{\pgfqpoint{6.363551in}{1.973147in}}%
\pgfpathlineto{\pgfqpoint{6.373860in}{1.980338in}}%
\pgfpathlineto{\pgfqpoint{6.381592in}{1.982053in}}%
\pgfpathlineto{\pgfqpoint{6.386746in}{1.985668in}}%
\pgfpathlineto{\pgfqpoint{6.391901in}{1.987537in}}%
\pgfpathlineto{\pgfqpoint{6.399632in}{1.989436in}}%
\pgfpathlineto{\pgfqpoint{6.409941in}{1.996690in}}%
\pgfpathlineto{\pgfqpoint{6.417673in}{1.998128in}}%
\pgfpathlineto{\pgfqpoint{6.427982in}{2.004868in}}%
\pgfpathlineto{\pgfqpoint{6.435713in}{2.006688in}}%
\pgfpathlineto{\pgfqpoint{6.446022in}{2.014244in}}%
\pgfpathlineto{\pgfqpoint{6.453754in}{2.016128in}}%
\pgfpathlineto{\pgfqpoint{6.464063in}{2.023161in}}%
\pgfpathlineto{\pgfqpoint{6.474371in}{2.024956in}}%
\pgfpathlineto{\pgfqpoint{6.482103in}{2.030593in}}%
\pgfpathlineto{\pgfqpoint{6.482103in}{2.030593in}}%
\pgfusepath{stroke}%
\end{pgfscope}%
\begin{pgfscope}%
\pgfpathrectangle{\pgfqpoint{0.563921in}{0.521603in}}{\pgfqpoint{6.200000in}{2.642500in}}%
\pgfusepath{clip}%
\pgfsetroundcap%
\pgfsetroundjoin%
\pgfsetlinewidth{1.505625pt}%
\definecolor{currentstroke}{rgb}{0.090196,0.745098,0.811765}%
\pgfsetstrokecolor{currentstroke}%
\pgfsetdash{}{0pt}%
\pgfpathmoveto{\pgfqpoint{0.845739in}{0.641717in}}%
\pgfpathlineto{\pgfqpoint{0.850894in}{0.667793in}}%
\pgfpathlineto{\pgfqpoint{0.853471in}{0.674543in}}%
\pgfpathlineto{\pgfqpoint{0.861203in}{0.673799in}}%
\pgfpathlineto{\pgfqpoint{0.863780in}{0.671929in}}%
\pgfpathlineto{\pgfqpoint{0.866357in}{0.672040in}}%
\pgfpathlineto{\pgfqpoint{0.868934in}{0.671504in}}%
\pgfpathlineto{\pgfqpoint{0.871512in}{0.672660in}}%
\pgfpathlineto{\pgfqpoint{0.881820in}{0.672656in}}%
\pgfpathlineto{\pgfqpoint{0.884398in}{0.671216in}}%
\pgfpathlineto{\pgfqpoint{0.889552in}{0.669705in}}%
\pgfpathlineto{\pgfqpoint{0.902438in}{0.667795in}}%
\pgfpathlineto{\pgfqpoint{0.907593in}{0.666479in}}%
\pgfpathlineto{\pgfqpoint{0.920479in}{0.664980in}}%
\pgfpathlineto{\pgfqpoint{0.923056in}{0.664584in}}%
\pgfpathlineto{\pgfqpoint{0.925633in}{0.666039in}}%
\pgfpathlineto{\pgfqpoint{0.933365in}{0.669259in}}%
\pgfpathlineto{\pgfqpoint{0.941096in}{0.685017in}}%
\pgfpathlineto{\pgfqpoint{0.943674in}{0.688665in}}%
\pgfpathlineto{\pgfqpoint{0.951405in}{0.692931in}}%
\pgfpathlineto{\pgfqpoint{0.953982in}{0.695645in}}%
\pgfpathlineto{\pgfqpoint{0.959137in}{0.698610in}}%
\pgfpathlineto{\pgfqpoint{0.961714in}{0.700611in}}%
\pgfpathlineto{\pgfqpoint{0.987486in}{0.704062in}}%
\pgfpathlineto{\pgfqpoint{0.995218in}{0.708346in}}%
\pgfpathlineto{\pgfqpoint{0.997795in}{0.709880in}}%
\pgfpathlineto{\pgfqpoint{1.008104in}{0.712454in}}%
\pgfpathlineto{\pgfqpoint{1.013258in}{0.712944in}}%
\pgfpathlineto{\pgfqpoint{1.015835in}{0.713601in}}%
\pgfpathlineto{\pgfqpoint{1.023567in}{0.714318in}}%
\pgfpathlineto{\pgfqpoint{1.026144in}{0.718382in}}%
\pgfpathlineto{\pgfqpoint{1.031299in}{0.722755in}}%
\pgfpathlineto{\pgfqpoint{1.033876in}{0.724084in}}%
\pgfpathlineto{\pgfqpoint{1.041608in}{0.725688in}}%
\pgfpathlineto{\pgfqpoint{1.051916in}{0.730161in}}%
\pgfpathlineto{\pgfqpoint{1.059648in}{0.732594in}}%
\pgfpathlineto{\pgfqpoint{1.062225in}{0.734409in}}%
\pgfpathlineto{\pgfqpoint{1.064802in}{0.735153in}}%
\pgfpathlineto{\pgfqpoint{1.067380in}{0.735286in}}%
\pgfpathlineto{\pgfqpoint{1.069957in}{0.736226in}}%
\pgfpathlineto{\pgfqpoint{1.082843in}{0.737304in}}%
\pgfpathlineto{\pgfqpoint{1.085420in}{0.737305in}}%
\pgfpathlineto{\pgfqpoint{1.098306in}{0.736055in}}%
\pgfpathlineto{\pgfqpoint{1.106038in}{0.734247in}}%
\pgfpathlineto{\pgfqpoint{1.121501in}{0.732512in}}%
\pgfpathlineto{\pgfqpoint{1.124078in}{0.732104in}}%
\pgfpathlineto{\pgfqpoint{1.142119in}{0.731241in}}%
\pgfpathlineto{\pgfqpoint{1.152428in}{0.731676in}}%
\pgfpathlineto{\pgfqpoint{1.170468in}{0.733274in}}%
\pgfpathlineto{\pgfqpoint{1.178200in}{0.738292in}}%
\pgfpathlineto{\pgfqpoint{1.185931in}{0.739608in}}%
\pgfpathlineto{\pgfqpoint{1.191086in}{0.741755in}}%
\pgfpathlineto{\pgfqpoint{1.196240in}{0.742150in}}%
\pgfpathlineto{\pgfqpoint{1.211704in}{0.743409in}}%
\pgfpathlineto{\pgfqpoint{1.227167in}{0.745674in}}%
\pgfpathlineto{\pgfqpoint{1.229744in}{0.746879in}}%
\pgfpathlineto{\pgfqpoint{1.242630in}{0.747587in}}%
\pgfpathlineto{\pgfqpoint{1.250362in}{0.750791in}}%
\pgfpathlineto{\pgfqpoint{1.258093in}{0.751733in}}%
\pgfpathlineto{\pgfqpoint{1.265825in}{0.756335in}}%
\pgfpathlineto{\pgfqpoint{1.268402in}{0.758151in}}%
\pgfpathlineto{\pgfqpoint{1.276134in}{0.759879in}}%
\pgfpathlineto{\pgfqpoint{1.286443in}{0.767480in}}%
\pgfpathlineto{\pgfqpoint{1.294174in}{0.768458in}}%
\pgfpathlineto{\pgfqpoint{1.299329in}{0.771688in}}%
\pgfpathlineto{\pgfqpoint{1.304483in}{0.775601in}}%
\pgfpathlineto{\pgfqpoint{1.312215in}{0.777923in}}%
\pgfpathlineto{\pgfqpoint{1.314792in}{0.780012in}}%
\pgfpathlineto{\pgfqpoint{1.319947in}{0.781581in}}%
\pgfpathlineto{\pgfqpoint{1.322524in}{0.783024in}}%
\pgfpathlineto{\pgfqpoint{1.332833in}{0.785210in}}%
\pgfpathlineto{\pgfqpoint{1.337987in}{0.786754in}}%
\pgfpathlineto{\pgfqpoint{1.340564in}{0.788045in}}%
\pgfpathlineto{\pgfqpoint{1.348296in}{0.789050in}}%
\pgfpathlineto{\pgfqpoint{1.356027in}{0.794737in}}%
\pgfpathlineto{\pgfqpoint{1.358605in}{0.795997in}}%
\pgfpathlineto{\pgfqpoint{1.366336in}{0.796814in}}%
\pgfpathlineto{\pgfqpoint{1.371491in}{0.798634in}}%
\pgfpathlineto{\pgfqpoint{1.376645in}{0.802468in}}%
\pgfpathlineto{\pgfqpoint{1.384377in}{0.804250in}}%
\pgfpathlineto{\pgfqpoint{1.394686in}{0.809064in}}%
\pgfpathlineto{\pgfqpoint{1.402417in}{0.810443in}}%
\pgfpathlineto{\pgfqpoint{1.412726in}{0.816276in}}%
\pgfpathlineto{\pgfqpoint{1.420458in}{0.817519in}}%
\pgfpathlineto{\pgfqpoint{1.428189in}{0.821128in}}%
\pgfpathlineto{\pgfqpoint{1.430767in}{0.822549in}}%
\pgfpathlineto{\pgfqpoint{1.441076in}{0.824765in}}%
\pgfpathlineto{\pgfqpoint{1.448807in}{0.826894in}}%
\pgfpathlineto{\pgfqpoint{1.459116in}{0.828334in}}%
\pgfpathlineto{\pgfqpoint{1.464270in}{0.829773in}}%
\pgfpathlineto{\pgfqpoint{1.466848in}{0.830332in}}%
\pgfpathlineto{\pgfqpoint{1.477156in}{0.830955in}}%
\pgfpathlineto{\pgfqpoint{1.482311in}{0.833869in}}%
\pgfpathlineto{\pgfqpoint{1.484888in}{0.835481in}}%
\pgfpathlineto{\pgfqpoint{1.492620in}{0.836907in}}%
\pgfpathlineto{\pgfqpoint{1.500351in}{0.841670in}}%
\pgfpathlineto{\pgfqpoint{1.502929in}{0.843372in}}%
\pgfpathlineto{\pgfqpoint{1.510660in}{0.844900in}}%
\pgfpathlineto{\pgfqpoint{1.515815in}{0.847991in}}%
\pgfpathlineto{\pgfqpoint{1.520969in}{0.851309in}}%
\pgfpathlineto{\pgfqpoint{1.528701in}{0.853016in}}%
\pgfpathlineto{\pgfqpoint{1.536432in}{0.856910in}}%
\pgfpathlineto{\pgfqpoint{1.539010in}{0.858089in}}%
\pgfpathlineto{\pgfqpoint{1.549318in}{0.859941in}}%
\pgfpathlineto{\pgfqpoint{1.557050in}{0.863657in}}%
\pgfpathlineto{\pgfqpoint{1.587977in}{0.867741in}}%
\pgfpathlineto{\pgfqpoint{1.593131in}{0.869228in}}%
\pgfpathlineto{\pgfqpoint{1.611172in}{0.870185in}}%
\pgfpathlineto{\pgfqpoint{1.647253in}{0.869299in}}%
\pgfpathlineto{\pgfqpoint{1.657561in}{0.868422in}}%
\pgfpathlineto{\pgfqpoint{1.665293in}{0.867030in}}%
\pgfpathlineto{\pgfqpoint{1.737455in}{0.863496in}}%
\pgfpathlineto{\pgfqpoint{1.750341in}{0.863580in}}%
\pgfpathlineto{\pgfqpoint{1.755495in}{0.863936in}}%
\pgfpathlineto{\pgfqpoint{1.788999in}{0.864388in}}%
\pgfpathlineto{\pgfqpoint{1.791576in}{0.865011in}}%
\pgfpathlineto{\pgfqpoint{1.819926in}{0.866262in}}%
\pgfpathlineto{\pgfqpoint{1.840543in}{0.869543in}}%
\pgfpathlineto{\pgfqpoint{1.845698in}{0.871825in}}%
\pgfpathlineto{\pgfqpoint{1.853430in}{0.873000in}}%
\pgfpathlineto{\pgfqpoint{1.863738in}{0.877025in}}%
\pgfpathlineto{\pgfqpoint{1.871470in}{0.877916in}}%
\pgfpathlineto{\pgfqpoint{1.881779in}{0.882066in}}%
\pgfpathlineto{\pgfqpoint{1.889510in}{0.883152in}}%
\pgfpathlineto{\pgfqpoint{1.899819in}{0.887777in}}%
\pgfpathlineto{\pgfqpoint{1.910128in}{0.889107in}}%
\pgfpathlineto{\pgfqpoint{1.917860in}{0.891487in}}%
\pgfpathlineto{\pgfqpoint{1.928169in}{0.892637in}}%
\pgfpathlineto{\pgfqpoint{1.935900in}{0.895225in}}%
\pgfpathlineto{\pgfqpoint{1.943632in}{0.896387in}}%
\pgfpathlineto{\pgfqpoint{1.953941in}{0.902009in}}%
\pgfpathlineto{\pgfqpoint{1.961672in}{0.903732in}}%
\pgfpathlineto{\pgfqpoint{1.971981in}{0.909943in}}%
\pgfpathlineto{\pgfqpoint{1.979713in}{0.911192in}}%
\pgfpathlineto{\pgfqpoint{1.990022in}{0.915629in}}%
\pgfpathlineto{\pgfqpoint{1.997753in}{0.916566in}}%
\pgfpathlineto{\pgfqpoint{2.005485in}{0.919657in}}%
\pgfpathlineto{\pgfqpoint{2.015794in}{0.920667in}}%
\pgfpathlineto{\pgfqpoint{2.026103in}{0.925537in}}%
\pgfpathlineto{\pgfqpoint{2.033834in}{0.927152in}}%
\pgfpathlineto{\pgfqpoint{2.038989in}{0.930894in}}%
\pgfpathlineto{\pgfqpoint{2.044143in}{0.935202in}}%
\pgfpathlineto{\pgfqpoint{2.051875in}{0.936666in}}%
\pgfpathlineto{\pgfqpoint{2.062184in}{0.944933in}}%
\pgfpathlineto{\pgfqpoint{2.069915in}{0.947394in}}%
\pgfpathlineto{\pgfqpoint{2.077647in}{0.954678in}}%
\pgfpathlineto{\pgfqpoint{2.080224in}{0.956883in}}%
\pgfpathlineto{\pgfqpoint{2.087956in}{0.959498in}}%
\pgfpathlineto{\pgfqpoint{2.095687in}{0.967397in}}%
\pgfpathlineto{\pgfqpoint{2.098265in}{0.970545in}}%
\pgfpathlineto{\pgfqpoint{2.105996in}{0.973737in}}%
\pgfpathlineto{\pgfqpoint{2.116305in}{0.988295in}}%
\pgfpathlineto{\pgfqpoint{2.124037in}{0.992106in}}%
\pgfpathlineto{\pgfqpoint{2.131768in}{1.002880in}}%
\pgfpathlineto{\pgfqpoint{2.134346in}{1.006046in}}%
\pgfpathlineto{\pgfqpoint{2.142077in}{1.008961in}}%
\pgfpathlineto{\pgfqpoint{2.152386in}{1.019167in}}%
\pgfpathlineto{\pgfqpoint{2.162695in}{1.022031in}}%
\pgfpathlineto{\pgfqpoint{2.167849in}{1.026705in}}%
\pgfpathlineto{\pgfqpoint{2.170427in}{1.028205in}}%
\pgfpathlineto{\pgfqpoint{2.178158in}{1.029898in}}%
\pgfpathlineto{\pgfqpoint{2.183313in}{1.033164in}}%
\pgfpathlineto{\pgfqpoint{2.188467in}{1.036481in}}%
\pgfpathlineto{\pgfqpoint{2.196199in}{1.038042in}}%
\pgfpathlineto{\pgfqpoint{2.206508in}{1.043912in}}%
\pgfpathlineto{\pgfqpoint{2.214239in}{1.045538in}}%
\pgfpathlineto{\pgfqpoint{2.219394in}{1.048899in}}%
\pgfpathlineto{\pgfqpoint{2.224548in}{1.050837in}}%
\pgfpathlineto{\pgfqpoint{2.232280in}{1.051801in}}%
\pgfpathlineto{\pgfqpoint{2.242589in}{1.056213in}}%
\pgfpathlineto{\pgfqpoint{2.250320in}{1.057468in}}%
\pgfpathlineto{\pgfqpoint{2.255475in}{1.059664in}}%
\pgfpathlineto{\pgfqpoint{2.270938in}{1.063637in}}%
\pgfpathlineto{\pgfqpoint{2.276092in}{1.066864in}}%
\pgfpathlineto{\pgfqpoint{2.278670in}{1.068790in}}%
\pgfpathlineto{\pgfqpoint{2.286401in}{1.070383in}}%
\pgfpathlineto{\pgfqpoint{2.296710in}{1.075886in}}%
\pgfpathlineto{\pgfqpoint{2.304442in}{1.076976in}}%
\pgfpathlineto{\pgfqpoint{2.314751in}{1.081409in}}%
\pgfpathlineto{\pgfqpoint{2.322482in}{1.082465in}}%
\pgfpathlineto{\pgfqpoint{2.330214in}{1.085625in}}%
\pgfpathlineto{\pgfqpoint{2.332791in}{1.087082in}}%
\pgfpathlineto{\pgfqpoint{2.340523in}{1.088392in}}%
\pgfpathlineto{\pgfqpoint{2.345677in}{1.091186in}}%
\pgfpathlineto{\pgfqpoint{2.350832in}{1.093305in}}%
\pgfpathlineto{\pgfqpoint{2.361140in}{1.094805in}}%
\pgfpathlineto{\pgfqpoint{2.366295in}{1.095960in}}%
\pgfpathlineto{\pgfqpoint{2.368872in}{1.096340in}}%
\pgfpathlineto{\pgfqpoint{2.384335in}{1.097430in}}%
\pgfpathlineto{\pgfqpoint{2.386912in}{1.097705in}}%
\pgfpathlineto{\pgfqpoint{2.430725in}{1.099255in}}%
\pgfpathlineto{\pgfqpoint{2.435880in}{1.100248in}}%
\pgfpathlineto{\pgfqpoint{2.441034in}{1.102265in}}%
\pgfpathlineto{\pgfqpoint{2.448766in}{1.103437in}}%
\pgfpathlineto{\pgfqpoint{2.456497in}{1.106642in}}%
\pgfpathlineto{\pgfqpoint{2.459074in}{1.107353in}}%
\pgfpathlineto{\pgfqpoint{2.469383in}{1.108571in}}%
\pgfpathlineto{\pgfqpoint{2.477115in}{1.110566in}}%
\pgfpathlineto{\pgfqpoint{2.487424in}{1.111741in}}%
\pgfpathlineto{\pgfqpoint{2.495155in}{1.113484in}}%
\pgfpathlineto{\pgfqpoint{2.508041in}{1.114808in}}%
\pgfpathlineto{\pgfqpoint{2.513196in}{1.116301in}}%
\pgfpathlineto{\pgfqpoint{2.523505in}{1.118035in}}%
\pgfpathlineto{\pgfqpoint{2.531236in}{1.120554in}}%
\pgfpathlineto{\pgfqpoint{2.538968in}{1.121562in}}%
\pgfpathlineto{\pgfqpoint{2.549277in}{1.126507in}}%
\pgfpathlineto{\pgfqpoint{2.557009in}{1.127731in}}%
\pgfpathlineto{\pgfqpoint{2.567317in}{1.132266in}}%
\pgfpathlineto{\pgfqpoint{2.575049in}{1.133376in}}%
\pgfpathlineto{\pgfqpoint{2.585358in}{1.137358in}}%
\pgfpathlineto{\pgfqpoint{2.595667in}{1.139138in}}%
\pgfpathlineto{\pgfqpoint{2.603398in}{1.142674in}}%
\pgfpathlineto{\pgfqpoint{2.611130in}{1.143787in}}%
\pgfpathlineto{\pgfqpoint{2.621439in}{1.148202in}}%
\pgfpathlineto{\pgfqpoint{2.629170in}{1.149294in}}%
\pgfpathlineto{\pgfqpoint{2.634325in}{1.151941in}}%
\pgfpathlineto{\pgfqpoint{2.652365in}{1.156522in}}%
\pgfpathlineto{\pgfqpoint{2.657520in}{1.158935in}}%
\pgfpathlineto{\pgfqpoint{2.665251in}{1.160171in}}%
\pgfpathlineto{\pgfqpoint{2.672983in}{1.163643in}}%
\pgfpathlineto{\pgfqpoint{2.675560in}{1.164698in}}%
\pgfpathlineto{\pgfqpoint{2.683292in}{1.165916in}}%
\pgfpathlineto{\pgfqpoint{2.688446in}{1.168691in}}%
\pgfpathlineto{\pgfqpoint{2.693601in}{1.171935in}}%
\pgfpathlineto{\pgfqpoint{2.701332in}{1.173658in}}%
\pgfpathlineto{\pgfqpoint{2.703910in}{1.175491in}}%
\pgfpathlineto{\pgfqpoint{2.709064in}{1.177484in}}%
\pgfpathlineto{\pgfqpoint{2.711641in}{1.179387in}}%
\pgfpathlineto{\pgfqpoint{2.719373in}{1.181724in}}%
\pgfpathlineto{\pgfqpoint{2.721950in}{1.184071in}}%
\pgfpathlineto{\pgfqpoint{2.727105in}{1.186355in}}%
\pgfpathlineto{\pgfqpoint{2.729682in}{1.188570in}}%
\pgfpathlineto{\pgfqpoint{2.737413in}{1.190687in}}%
\pgfpathlineto{\pgfqpoint{2.745145in}{1.196649in}}%
\pgfpathlineto{\pgfqpoint{2.747722in}{1.198564in}}%
\pgfpathlineto{\pgfqpoint{2.755454in}{1.199997in}}%
\pgfpathlineto{\pgfqpoint{2.765763in}{1.206344in}}%
\pgfpathlineto{\pgfqpoint{2.776072in}{1.207879in}}%
\pgfpathlineto{\pgfqpoint{2.781226in}{1.211247in}}%
\pgfpathlineto{\pgfqpoint{2.783803in}{1.212435in}}%
\pgfpathlineto{\pgfqpoint{2.791535in}{1.213521in}}%
\pgfpathlineto{\pgfqpoint{2.801844in}{1.217940in}}%
\pgfpathlineto{\pgfqpoint{2.809575in}{1.218589in}}%
\pgfpathlineto{\pgfqpoint{2.814730in}{1.220318in}}%
\pgfpathlineto{\pgfqpoint{2.819884in}{1.223578in}}%
\pgfpathlineto{\pgfqpoint{2.827616in}{1.225486in}}%
\pgfpathlineto{\pgfqpoint{2.837925in}{1.233906in}}%
\pgfpathlineto{\pgfqpoint{2.848234in}{1.236263in}}%
\pgfpathlineto{\pgfqpoint{2.855965in}{1.243039in}}%
\pgfpathlineto{\pgfqpoint{2.863697in}{1.245552in}}%
\pgfpathlineto{\pgfqpoint{2.874006in}{1.254995in}}%
\pgfpathlineto{\pgfqpoint{2.881737in}{1.257077in}}%
\pgfpathlineto{\pgfqpoint{2.892046in}{1.267903in}}%
\pgfpathlineto{\pgfqpoint{2.899778in}{1.270404in}}%
\pgfpathlineto{\pgfqpoint{2.910087in}{1.278919in}}%
\pgfpathlineto{\pgfqpoint{2.917818in}{1.281154in}}%
\pgfpathlineto{\pgfqpoint{2.925550in}{1.287533in}}%
\pgfpathlineto{\pgfqpoint{2.928127in}{1.289451in}}%
\pgfpathlineto{\pgfqpoint{2.935859in}{1.291175in}}%
\pgfpathlineto{\pgfqpoint{2.946168in}{1.297452in}}%
\pgfpathlineto{\pgfqpoint{2.953899in}{1.299199in}}%
\pgfpathlineto{\pgfqpoint{2.964208in}{1.306998in}}%
\pgfpathlineto{\pgfqpoint{2.971940in}{1.308473in}}%
\pgfpathlineto{\pgfqpoint{2.979671in}{1.312851in}}%
\pgfpathlineto{\pgfqpoint{2.982249in}{1.313906in}}%
\pgfpathlineto{\pgfqpoint{2.989980in}{1.315048in}}%
\pgfpathlineto{\pgfqpoint{2.997712in}{1.319041in}}%
\pgfpathlineto{\pgfqpoint{3.008021in}{1.320382in}}%
\pgfpathlineto{\pgfqpoint{3.018330in}{1.325676in}}%
\pgfpathlineto{\pgfqpoint{3.026061in}{1.326732in}}%
\pgfpathlineto{\pgfqpoint{3.036370in}{1.331942in}}%
\pgfpathlineto{\pgfqpoint{3.044102in}{1.333515in}}%
\pgfpathlineto{\pgfqpoint{3.054411in}{1.339646in}}%
\pgfpathlineto{\pgfqpoint{3.062142in}{1.341355in}}%
\pgfpathlineto{\pgfqpoint{3.069874in}{1.345661in}}%
\pgfpathlineto{\pgfqpoint{3.072451in}{1.346955in}}%
\pgfpathlineto{\pgfqpoint{3.080183in}{1.348345in}}%
\pgfpathlineto{\pgfqpoint{3.087914in}{1.352855in}}%
\pgfpathlineto{\pgfqpoint{3.090491in}{1.354582in}}%
\pgfpathlineto{\pgfqpoint{3.100800in}{1.356367in}}%
\pgfpathlineto{\pgfqpoint{3.108532in}{1.361691in}}%
\pgfpathlineto{\pgfqpoint{3.116264in}{1.363501in}}%
\pgfpathlineto{\pgfqpoint{3.126572in}{1.370634in}}%
\pgfpathlineto{\pgfqpoint{3.134304in}{1.372573in}}%
\pgfpathlineto{\pgfqpoint{3.142036in}{1.377468in}}%
\pgfpathlineto{\pgfqpoint{3.144613in}{1.378872in}}%
\pgfpathlineto{\pgfqpoint{3.152345in}{1.380345in}}%
\pgfpathlineto{\pgfqpoint{3.162653in}{1.386116in}}%
\pgfpathlineto{\pgfqpoint{3.170385in}{1.387432in}}%
\pgfpathlineto{\pgfqpoint{3.178117in}{1.391748in}}%
\pgfpathlineto{\pgfqpoint{3.180694in}{1.393426in}}%
\pgfpathlineto{\pgfqpoint{3.188426in}{1.395164in}}%
\pgfpathlineto{\pgfqpoint{3.196157in}{1.400718in}}%
\pgfpathlineto{\pgfqpoint{3.206466in}{1.402535in}}%
\pgfpathlineto{\pgfqpoint{3.216775in}{1.409708in}}%
\pgfpathlineto{\pgfqpoint{3.224507in}{1.411467in}}%
\pgfpathlineto{\pgfqpoint{3.234815in}{1.417564in}}%
\pgfpathlineto{\pgfqpoint{3.242547in}{1.419076in}}%
\pgfpathlineto{\pgfqpoint{3.252856in}{1.425357in}}%
\pgfpathlineto{\pgfqpoint{3.260588in}{1.427026in}}%
\pgfpathlineto{\pgfqpoint{3.270896in}{1.432925in}}%
\pgfpathlineto{\pgfqpoint{3.278628in}{1.434534in}}%
\pgfpathlineto{\pgfqpoint{3.288937in}{1.440320in}}%
\pgfpathlineto{\pgfqpoint{3.296668in}{1.441900in}}%
\pgfpathlineto{\pgfqpoint{3.306977in}{1.448510in}}%
\pgfpathlineto{\pgfqpoint{3.314709in}{1.450418in}}%
\pgfpathlineto{\pgfqpoint{3.325018in}{1.458027in}}%
\pgfpathlineto{\pgfqpoint{3.332749in}{1.459915in}}%
\pgfpathlineto{\pgfqpoint{3.343058in}{1.467127in}}%
\pgfpathlineto{\pgfqpoint{3.353367in}{1.469000in}}%
\pgfpathlineto{\pgfqpoint{3.361099in}{1.474451in}}%
\pgfpathlineto{\pgfqpoint{3.368830in}{1.476217in}}%
\pgfpathlineto{\pgfqpoint{3.379139in}{1.482585in}}%
\pgfpathlineto{\pgfqpoint{3.386871in}{1.484201in}}%
\pgfpathlineto{\pgfqpoint{3.397180in}{1.490693in}}%
\pgfpathlineto{\pgfqpoint{3.404911in}{1.492121in}}%
\pgfpathlineto{\pgfqpoint{3.415220in}{1.497346in}}%
\pgfpathlineto{\pgfqpoint{3.422952in}{1.498649in}}%
\pgfpathlineto{\pgfqpoint{3.430684in}{1.502066in}}%
\pgfpathlineto{\pgfqpoint{3.433261in}{1.503274in}}%
\pgfpathlineto{\pgfqpoint{3.440992in}{1.504487in}}%
\pgfpathlineto{\pgfqpoint{3.451301in}{1.508329in}}%
\pgfpathlineto{\pgfqpoint{3.461610in}{1.509545in}}%
\pgfpathlineto{\pgfqpoint{3.479651in}{1.512609in}}%
\pgfpathlineto{\pgfqpoint{3.487382in}{1.515670in}}%
\pgfpathlineto{\pgfqpoint{3.495114in}{1.516742in}}%
\pgfpathlineto{\pgfqpoint{3.505423in}{1.521896in}}%
\pgfpathlineto{\pgfqpoint{3.513154in}{1.523356in}}%
\pgfpathlineto{\pgfqpoint{3.523463in}{1.528601in}}%
\pgfpathlineto{\pgfqpoint{3.531195in}{1.529744in}}%
\pgfpathlineto{\pgfqpoint{3.541504in}{1.534489in}}%
\pgfpathlineto{\pgfqpoint{3.549235in}{1.535665in}}%
\pgfpathlineto{\pgfqpoint{3.559544in}{1.539815in}}%
\pgfpathlineto{\pgfqpoint{3.567276in}{1.540967in}}%
\pgfpathlineto{\pgfqpoint{3.572430in}{1.543555in}}%
\pgfpathlineto{\pgfqpoint{3.577585in}{1.544930in}}%
\pgfpathlineto{\pgfqpoint{3.585316in}{1.546320in}}%
\pgfpathlineto{\pgfqpoint{3.595625in}{1.552141in}}%
\pgfpathlineto{\pgfqpoint{3.603357in}{1.553622in}}%
\pgfpathlineto{\pgfqpoint{3.613666in}{1.558759in}}%
\pgfpathlineto{\pgfqpoint{3.621397in}{1.559955in}}%
\pgfpathlineto{\pgfqpoint{3.629129in}{1.563681in}}%
\pgfpathlineto{\pgfqpoint{3.631706in}{1.565100in}}%
\pgfpathlineto{\pgfqpoint{3.639438in}{1.566687in}}%
\pgfpathlineto{\pgfqpoint{3.644592in}{1.569925in}}%
\pgfpathlineto{\pgfqpoint{3.649747in}{1.571591in}}%
\pgfpathlineto{\pgfqpoint{3.657478in}{1.573312in}}%
\pgfpathlineto{\pgfqpoint{3.662633in}{1.576414in}}%
\pgfpathlineto{\pgfqpoint{3.667787in}{1.577852in}}%
\pgfpathlineto{\pgfqpoint{3.675519in}{1.579099in}}%
\pgfpathlineto{\pgfqpoint{3.685828in}{1.584412in}}%
\pgfpathlineto{\pgfqpoint{3.693559in}{1.585880in}}%
\pgfpathlineto{\pgfqpoint{3.703868in}{1.591784in}}%
\pgfpathlineto{\pgfqpoint{3.714177in}{1.593231in}}%
\pgfpathlineto{\pgfqpoint{3.721909in}{1.597482in}}%
\pgfpathlineto{\pgfqpoint{3.729640in}{1.598914in}}%
\pgfpathlineto{\pgfqpoint{3.737372in}{1.602508in}}%
\pgfpathlineto{\pgfqpoint{3.739949in}{1.603412in}}%
\pgfpathlineto{\pgfqpoint{3.747681in}{1.604420in}}%
\pgfpathlineto{\pgfqpoint{3.750258in}{1.605683in}}%
\pgfpathlineto{\pgfqpoint{3.757990in}{1.612719in}}%
\pgfpathlineto{\pgfqpoint{3.765721in}{1.614986in}}%
\pgfpathlineto{\pgfqpoint{3.776030in}{1.624593in}}%
\pgfpathlineto{\pgfqpoint{3.786339in}{1.627138in}}%
\pgfpathlineto{\pgfqpoint{3.794070in}{1.634640in}}%
\pgfpathlineto{\pgfqpoint{3.801802in}{1.637240in}}%
\pgfpathlineto{\pgfqpoint{3.812111in}{1.647263in}}%
\pgfpathlineto{\pgfqpoint{3.819843in}{1.649901in}}%
\pgfpathlineto{\pgfqpoint{3.827574in}{1.657574in}}%
\pgfpathlineto{\pgfqpoint{3.830151in}{1.659812in}}%
\pgfpathlineto{\pgfqpoint{3.837883in}{1.662249in}}%
\pgfpathlineto{\pgfqpoint{3.845615in}{1.669073in}}%
\pgfpathlineto{\pgfqpoint{3.848192in}{1.671615in}}%
\pgfpathlineto{\pgfqpoint{3.855924in}{1.674282in}}%
\pgfpathlineto{\pgfqpoint{3.866232in}{1.684929in}}%
\pgfpathlineto{\pgfqpoint{3.873964in}{1.687621in}}%
\pgfpathlineto{\pgfqpoint{3.881696in}{1.694475in}}%
\pgfpathlineto{\pgfqpoint{3.884273in}{1.696679in}}%
\pgfpathlineto{\pgfqpoint{3.892005in}{1.698959in}}%
\pgfpathlineto{\pgfqpoint{3.899736in}{1.705404in}}%
\pgfpathlineto{\pgfqpoint{3.910045in}{1.707545in}}%
\pgfpathlineto{\pgfqpoint{3.920354in}{1.716372in}}%
\pgfpathlineto{\pgfqpoint{3.928086in}{1.718554in}}%
\pgfpathlineto{\pgfqpoint{3.938394in}{1.727455in}}%
\pgfpathlineto{\pgfqpoint{3.946126in}{1.729796in}}%
\pgfpathlineto{\pgfqpoint{3.956435in}{1.739169in}}%
\pgfpathlineto{\pgfqpoint{3.964167in}{1.741702in}}%
\pgfpathlineto{\pgfqpoint{3.974475in}{1.751401in}}%
\pgfpathlineto{\pgfqpoint{3.982207in}{1.753966in}}%
\pgfpathlineto{\pgfqpoint{3.992516in}{1.763430in}}%
\pgfpathlineto{\pgfqpoint{4.000247in}{1.765562in}}%
\pgfpathlineto{\pgfqpoint{4.010556in}{1.774511in}}%
\pgfpathlineto{\pgfqpoint{4.018288in}{1.776799in}}%
\pgfpathlineto{\pgfqpoint{4.028597in}{1.785791in}}%
\pgfpathlineto{\pgfqpoint{4.038906in}{1.787876in}}%
\pgfpathlineto{\pgfqpoint{4.046637in}{1.794431in}}%
\pgfpathlineto{\pgfqpoint{4.054369in}{1.796657in}}%
\pgfpathlineto{\pgfqpoint{4.064678in}{1.805226in}}%
\pgfpathlineto{\pgfqpoint{4.072409in}{1.807156in}}%
\pgfpathlineto{\pgfqpoint{4.082718in}{1.814990in}}%
\pgfpathlineto{\pgfqpoint{4.090450in}{1.816963in}}%
\pgfpathlineto{\pgfqpoint{4.100759in}{1.825709in}}%
\pgfpathlineto{\pgfqpoint{4.108490in}{1.828046in}}%
\pgfpathlineto{\pgfqpoint{4.118799in}{1.837695in}}%
\pgfpathlineto{\pgfqpoint{4.126531in}{1.839885in}}%
\pgfpathlineto{\pgfqpoint{4.134263in}{1.847211in}}%
\pgfpathlineto{\pgfqpoint{4.144571in}{1.849778in}}%
\pgfpathlineto{\pgfqpoint{4.154880in}{1.860076in}}%
\pgfpathlineto{\pgfqpoint{4.162612in}{1.862879in}}%
\pgfpathlineto{\pgfqpoint{4.172921in}{1.874150in}}%
\pgfpathlineto{\pgfqpoint{4.180652in}{1.877066in}}%
\pgfpathlineto{\pgfqpoint{4.190961in}{1.888245in}}%
\pgfpathlineto{\pgfqpoint{4.198693in}{1.890875in}}%
\pgfpathlineto{\pgfqpoint{4.209002in}{1.901933in}}%
\pgfpathlineto{\pgfqpoint{4.216733in}{1.904862in}}%
\pgfpathlineto{\pgfqpoint{4.219311in}{1.907850in}}%
\pgfpathlineto{\pgfqpoint{4.227042in}{1.912136in}}%
\pgfpathlineto{\pgfqpoint{4.234774in}{1.913712in}}%
\pgfpathlineto{\pgfqpoint{4.245083in}{1.918404in}}%
\pgfpathlineto{\pgfqpoint{4.252814in}{1.919724in}}%
\pgfpathlineto{\pgfqpoint{4.257969in}{1.921864in}}%
\pgfpathlineto{\pgfqpoint{4.263123in}{1.922708in}}%
\pgfpathlineto{\pgfqpoint{4.276009in}{1.923383in}}%
\pgfpathlineto{\pgfqpoint{4.281164in}{1.924660in}}%
\pgfpathlineto{\pgfqpoint{4.291472in}{1.925656in}}%
\pgfpathlineto{\pgfqpoint{4.299204in}{1.927339in}}%
\pgfpathlineto{\pgfqpoint{4.312090in}{1.928665in}}%
\pgfpathlineto{\pgfqpoint{4.317245in}{1.930071in}}%
\pgfpathlineto{\pgfqpoint{4.327553in}{1.931469in}}%
\pgfpathlineto{\pgfqpoint{4.335285in}{1.933541in}}%
\pgfpathlineto{\pgfqpoint{4.345594in}{1.934778in}}%
\pgfpathlineto{\pgfqpoint{4.353326in}{1.936099in}}%
\pgfpathlineto{\pgfqpoint{4.363634in}{1.936686in}}%
\pgfpathlineto{\pgfqpoint{4.371366in}{1.938374in}}%
\pgfpathlineto{\pgfqpoint{4.381675in}{1.939673in}}%
\pgfpathlineto{\pgfqpoint{4.389407in}{1.941765in}}%
\pgfpathlineto{\pgfqpoint{4.397138in}{1.942610in}}%
\pgfpathlineto{\pgfqpoint{4.407447in}{1.946209in}}%
\pgfpathlineto{\pgfqpoint{4.415179in}{1.947310in}}%
\pgfpathlineto{\pgfqpoint{4.422910in}{1.951037in}}%
\pgfpathlineto{\pgfqpoint{4.425488in}{1.952465in}}%
\pgfpathlineto{\pgfqpoint{4.433219in}{1.953929in}}%
\pgfpathlineto{\pgfqpoint{4.443528in}{1.959982in}}%
\pgfpathlineto{\pgfqpoint{4.451260in}{1.961558in}}%
\pgfpathlineto{\pgfqpoint{4.461569in}{1.967488in}}%
\pgfpathlineto{\pgfqpoint{4.469300in}{1.969162in}}%
\pgfpathlineto{\pgfqpoint{4.479609in}{1.975684in}}%
\pgfpathlineto{\pgfqpoint{4.487341in}{1.977255in}}%
\pgfpathlineto{\pgfqpoint{4.495072in}{1.982464in}}%
\pgfpathlineto{\pgfqpoint{4.497649in}{1.984442in}}%
\pgfpathlineto{\pgfqpoint{4.505381in}{1.986336in}}%
\pgfpathlineto{\pgfqpoint{4.510536in}{1.989846in}}%
\pgfpathlineto{\pgfqpoint{4.515690in}{1.991255in}}%
\pgfpathlineto{\pgfqpoint{4.523422in}{1.992495in}}%
\pgfpathlineto{\pgfqpoint{4.533730in}{1.997540in}}%
\pgfpathlineto{\pgfqpoint{4.541462in}{1.998776in}}%
\pgfpathlineto{\pgfqpoint{4.551771in}{2.002639in}}%
\pgfpathlineto{\pgfqpoint{4.559503in}{2.003502in}}%
\pgfpathlineto{\pgfqpoint{4.569811in}{2.007649in}}%
\pgfpathlineto{\pgfqpoint{4.580120in}{2.008880in}}%
\pgfpathlineto{\pgfqpoint{4.585275in}{2.009949in}}%
\pgfpathlineto{\pgfqpoint{4.598161in}{2.011226in}}%
\pgfpathlineto{\pgfqpoint{4.603315in}{2.012265in}}%
\pgfpathlineto{\pgfqpoint{4.641973in}{2.013265in}}%
\pgfpathlineto{\pgfqpoint{4.670323in}{2.012251in}}%
\pgfpathlineto{\pgfqpoint{4.696095in}{2.010746in}}%
\pgfpathlineto{\pgfqpoint{4.708981in}{2.009740in}}%
\pgfpathlineto{\pgfqpoint{4.714135in}{2.009036in}}%
\pgfpathlineto{\pgfqpoint{4.747639in}{2.007713in}}%
\pgfpathlineto{\pgfqpoint{4.763102in}{2.007265in}}%
\pgfpathlineto{\pgfqpoint{4.814647in}{2.007001in}}%
\pgfpathlineto{\pgfqpoint{4.832687in}{2.006718in}}%
\pgfpathlineto{\pgfqpoint{4.858459in}{2.006238in}}%
\pgfpathlineto{\pgfqpoint{4.886809in}{2.006260in}}%
\pgfpathlineto{\pgfqpoint{4.930621in}{2.009726in}}%
\pgfpathlineto{\pgfqpoint{4.964125in}{2.010628in}}%
\pgfpathlineto{\pgfqpoint{5.018246in}{2.009907in}}%
\pgfpathlineto{\pgfqpoint{5.038864in}{2.009294in}}%
\pgfpathlineto{\pgfqpoint{5.056905in}{2.008823in}}%
\pgfpathlineto{\pgfqpoint{5.105872in}{2.007663in}}%
\pgfpathlineto{\pgfqpoint{5.136798in}{2.007226in}}%
\pgfpathlineto{\pgfqpoint{5.183188in}{2.004748in}}%
\pgfpathlineto{\pgfqpoint{5.208960in}{2.003753in}}%
\pgfpathlineto{\pgfqpoint{5.237309in}{2.001674in}}%
\pgfpathlineto{\pgfqpoint{5.252773in}{2.000848in}}%
\pgfpathlineto{\pgfqpoint{5.255350in}{2.000525in}}%
\pgfpathlineto{\pgfqpoint{5.268236in}{1.999598in}}%
\pgfpathlineto{\pgfqpoint{5.273390in}{1.998958in}}%
\pgfpathlineto{\pgfqpoint{5.286276in}{1.998010in}}%
\pgfpathlineto{\pgfqpoint{5.309471in}{1.995758in}}%
\pgfpathlineto{\pgfqpoint{5.322357in}{1.994797in}}%
\pgfpathlineto{\pgfqpoint{5.327512in}{1.994167in}}%
\pgfpathlineto{\pgfqpoint{5.340398in}{1.993153in}}%
\pgfpathlineto{\pgfqpoint{5.345552in}{1.992437in}}%
\pgfpathlineto{\pgfqpoint{5.358438in}{1.991372in}}%
\pgfpathlineto{\pgfqpoint{5.363593in}{1.990735in}}%
\pgfpathlineto{\pgfqpoint{5.376479in}{1.989844in}}%
\pgfpathlineto{\pgfqpoint{5.399674in}{1.987711in}}%
\pgfpathlineto{\pgfqpoint{5.487299in}{1.985234in}}%
\pgfpathlineto{\pgfqpoint{5.500185in}{1.985812in}}%
\pgfpathlineto{\pgfqpoint{5.543998in}{1.988333in}}%
\pgfpathlineto{\pgfqpoint{5.559461in}{1.989290in}}%
\pgfpathlineto{\pgfqpoint{5.562038in}{1.989724in}}%
\pgfpathlineto{\pgfqpoint{5.572347in}{1.990511in}}%
\pgfpathlineto{\pgfqpoint{5.580079in}{1.991673in}}%
\pgfpathlineto{\pgfqpoint{5.595542in}{1.992718in}}%
\pgfpathlineto{\pgfqpoint{5.598119in}{1.993055in}}%
\pgfpathlineto{\pgfqpoint{5.611005in}{1.994056in}}%
\pgfpathlineto{\pgfqpoint{5.652241in}{1.999365in}}%
\pgfpathlineto{\pgfqpoint{5.662550in}{2.000239in}}%
\pgfpathlineto{\pgfqpoint{5.670281in}{2.001593in}}%
\pgfpathlineto{\pgfqpoint{5.683167in}{2.002453in}}%
\pgfpathlineto{\pgfqpoint{5.688322in}{2.003300in}}%
\pgfpathlineto{\pgfqpoint{5.698630in}{2.004153in}}%
\pgfpathlineto{\pgfqpoint{5.706362in}{2.005560in}}%
\pgfpathlineto{\pgfqpoint{5.716671in}{2.006462in}}%
\pgfpathlineto{\pgfqpoint{5.724403in}{2.007829in}}%
\pgfpathlineto{\pgfqpoint{5.734711in}{2.008847in}}%
\pgfpathlineto{\pgfqpoint{5.742443in}{2.010331in}}%
\pgfpathlineto{\pgfqpoint{5.752752in}{2.011364in}}%
\pgfpathlineto{\pgfqpoint{5.760484in}{2.012879in}}%
\pgfpathlineto{\pgfqpoint{5.770792in}{2.013946in}}%
\pgfpathlineto{\pgfqpoint{5.778524in}{2.015586in}}%
\pgfpathlineto{\pgfqpoint{5.788833in}{2.016677in}}%
\pgfpathlineto{\pgfqpoint{5.796565in}{2.018247in}}%
\pgfpathlineto{\pgfqpoint{5.806873in}{2.019267in}}%
\pgfpathlineto{\pgfqpoint{5.812028in}{2.020321in}}%
\pgfpathlineto{\pgfqpoint{5.824914in}{2.021472in}}%
\pgfpathlineto{\pgfqpoint{5.832646in}{2.023241in}}%
\pgfpathlineto{\pgfqpoint{5.842954in}{2.024422in}}%
\pgfpathlineto{\pgfqpoint{5.850686in}{2.026406in}}%
\pgfpathlineto{\pgfqpoint{5.860995in}{2.027572in}}%
\pgfpathlineto{\pgfqpoint{5.868727in}{2.028773in}}%
\pgfpathlineto{\pgfqpoint{5.881613in}{2.029866in}}%
\pgfpathlineto{\pgfqpoint{5.886767in}{2.030430in}}%
\pgfpathlineto{\pgfqpoint{5.935734in}{2.032496in}}%
\pgfpathlineto{\pgfqpoint{5.948620in}{2.032932in}}%
\pgfpathlineto{\pgfqpoint{6.020782in}{2.034081in}}%
\pgfpathlineto{\pgfqpoint{6.080058in}{2.035079in}}%
\pgfpathlineto{\pgfqpoint{6.085212in}{2.035627in}}%
\pgfpathlineto{\pgfqpoint{6.129025in}{2.036618in}}%
\pgfpathlineto{\pgfqpoint{6.167683in}{2.035542in}}%
\pgfpathlineto{\pgfqpoint{6.201187in}{2.034251in}}%
\pgfpathlineto{\pgfqpoint{6.221805in}{2.032928in}}%
\pgfpathlineto{\pgfqpoint{6.247577in}{2.031294in}}%
\pgfpathlineto{\pgfqpoint{6.275926in}{2.030126in}}%
\pgfpathlineto{\pgfqpoint{6.301698in}{2.028339in}}%
\pgfpathlineto{\pgfqpoint{6.317161in}{2.027452in}}%
\pgfpathlineto{\pgfqpoint{6.337779in}{2.026123in}}%
\pgfpathlineto{\pgfqpoint{6.409941in}{2.024894in}}%
\pgfpathlineto{\pgfqpoint{6.456331in}{2.026731in}}%
\pgfpathlineto{\pgfqpoint{6.464063in}{2.027334in}}%
\pgfpathlineto{\pgfqpoint{6.482103in}{2.027842in}}%
\pgfpathlineto{\pgfqpoint{6.482103in}{2.027842in}}%
\pgfusepath{stroke}%
\end{pgfscope}%
\begin{pgfscope}%
\pgfsetrectcap%
\pgfsetmiterjoin%
\pgfsetlinewidth{0.803000pt}%
\definecolor{currentstroke}{rgb}{1.000000,1.000000,1.000000}%
\pgfsetstrokecolor{currentstroke}%
\pgfsetdash{}{0pt}%
\pgfpathmoveto{\pgfqpoint{0.563921in}{0.521603in}}%
\pgfpathlineto{\pgfqpoint{0.563921in}{3.164103in}}%
\pgfusepath{stroke}%
\end{pgfscope}%
\begin{pgfscope}%
\pgfsetrectcap%
\pgfsetmiterjoin%
\pgfsetlinewidth{0.803000pt}%
\definecolor{currentstroke}{rgb}{1.000000,1.000000,1.000000}%
\pgfsetstrokecolor{currentstroke}%
\pgfsetdash{}{0pt}%
\pgfpathmoveto{\pgfqpoint{6.763921in}{0.521603in}}%
\pgfpathlineto{\pgfqpoint{6.763921in}{3.164103in}}%
\pgfusepath{stroke}%
\end{pgfscope}%
\begin{pgfscope}%
\pgfsetrectcap%
\pgfsetmiterjoin%
\pgfsetlinewidth{0.803000pt}%
\definecolor{currentstroke}{rgb}{1.000000,1.000000,1.000000}%
\pgfsetstrokecolor{currentstroke}%
\pgfsetdash{}{0pt}%
\pgfpathmoveto{\pgfqpoint{0.563921in}{0.521603in}}%
\pgfpathlineto{\pgfqpoint{6.763921in}{0.521603in}}%
\pgfusepath{stroke}%
\end{pgfscope}%
\begin{pgfscope}%
\pgfsetrectcap%
\pgfsetmiterjoin%
\pgfsetlinewidth{0.803000pt}%
\definecolor{currentstroke}{rgb}{1.000000,1.000000,1.000000}%
\pgfsetstrokecolor{currentstroke}%
\pgfsetdash{}{0pt}%
\pgfpathmoveto{\pgfqpoint{0.563921in}{3.164103in}}%
\pgfpathlineto{\pgfqpoint{6.763921in}{3.164103in}}%
\pgfusepath{stroke}%
\end{pgfscope}%
\begin{pgfscope}%
\definecolor{textcolor}{rgb}{0.150000,0.150000,0.150000}%
\pgfsetstrokecolor{textcolor}%
\pgfsetfillcolor{textcolor}%
\pgftext[x=3.663921in,y=3.247437in,,base]{\color{textcolor}\rmfamily\fontsize{12.000000}{14.400000}\selectfont 'Cumulative' Standard Deviation of Stock Prices}%
\end{pgfscope}%
\begin{pgfscope}%
\pgfsetbuttcap%
\pgfsetmiterjoin%
\definecolor{currentfill}{rgb}{0.917647,0.917647,0.949020}%
\pgfsetfillcolor{currentfill}%
\pgfsetfillopacity{0.800000}%
\pgfsetlinewidth{1.003750pt}%
\definecolor{currentstroke}{rgb}{0.800000,0.800000,0.800000}%
\pgfsetstrokecolor{currentstroke}%
\pgfsetstrokeopacity{0.800000}%
\pgfsetdash{}{0pt}%
\pgfpathmoveto{\pgfqpoint{0.661143in}{1.014420in}}%
\pgfpathlineto{\pgfqpoint{1.532224in}{1.014420in}}%
\pgfpathquadraticcurveto{\pgfqpoint{1.560001in}{1.014420in}}{\pgfqpoint{1.560001in}{1.042198in}}%
\pgfpathlineto{\pgfqpoint{1.560001in}{3.066881in}}%
\pgfpathquadraticcurveto{\pgfqpoint{1.560001in}{3.094659in}}{\pgfqpoint{1.532224in}{3.094659in}}%
\pgfpathlineto{\pgfqpoint{0.661143in}{3.094659in}}%
\pgfpathquadraticcurveto{\pgfqpoint{0.633366in}{3.094659in}}{\pgfqpoint{0.633366in}{3.066881in}}%
\pgfpathlineto{\pgfqpoint{0.633366in}{1.042198in}}%
\pgfpathquadraticcurveto{\pgfqpoint{0.633366in}{1.014420in}}{\pgfqpoint{0.661143in}{1.014420in}}%
\pgfpathclose%
\pgfusepath{stroke,fill}%
\end{pgfscope}%
\begin{pgfscope}%
\pgfsetroundcap%
\pgfsetroundjoin%
\pgfsetlinewidth{1.505625pt}%
\definecolor{currentstroke}{rgb}{0.121569,0.466667,0.705882}%
\pgfsetstrokecolor{currentstroke}%
\pgfsetdash{}{0pt}%
\pgfpathmoveto{\pgfqpoint{0.688921in}{2.982191in}}%
\pgfpathlineto{\pgfqpoint{0.966699in}{2.982191in}}%
\pgfusepath{stroke}%
\end{pgfscope}%
\begin{pgfscope}%
\definecolor{textcolor}{rgb}{0.150000,0.150000,0.150000}%
\pgfsetstrokecolor{textcolor}%
\pgfsetfillcolor{textcolor}%
\pgftext[x=1.077810in,y=2.933580in,left,base]{\color{textcolor}\rmfamily\fontsize{10.000000}{12.000000}\selectfont MMM}%
\end{pgfscope}%
\begin{pgfscope}%
\pgfsetroundcap%
\pgfsetroundjoin%
\pgfsetlinewidth{1.505625pt}%
\definecolor{currentstroke}{rgb}{1.000000,0.498039,0.054902}%
\pgfsetstrokecolor{currentstroke}%
\pgfsetdash{}{0pt}%
\pgfpathmoveto{\pgfqpoint{0.688921in}{2.778334in}}%
\pgfpathlineto{\pgfqpoint{0.966699in}{2.778334in}}%
\pgfusepath{stroke}%
\end{pgfscope}%
\begin{pgfscope}%
\definecolor{textcolor}{rgb}{0.150000,0.150000,0.150000}%
\pgfsetstrokecolor{textcolor}%
\pgfsetfillcolor{textcolor}%
\pgftext[x=1.077810in,y=2.729723in,left,base]{\color{textcolor}\rmfamily\fontsize{10.000000}{12.000000}\selectfont AXP}%
\end{pgfscope}%
\begin{pgfscope}%
\pgfsetroundcap%
\pgfsetroundjoin%
\pgfsetlinewidth{1.505625pt}%
\definecolor{currentstroke}{rgb}{0.172549,0.627451,0.172549}%
\pgfsetstrokecolor{currentstroke}%
\pgfsetdash{}{0pt}%
\pgfpathmoveto{\pgfqpoint{0.688921in}{2.574477in}}%
\pgfpathlineto{\pgfqpoint{0.966699in}{2.574477in}}%
\pgfusepath{stroke}%
\end{pgfscope}%
\begin{pgfscope}%
\definecolor{textcolor}{rgb}{0.150000,0.150000,0.150000}%
\pgfsetstrokecolor{textcolor}%
\pgfsetfillcolor{textcolor}%
\pgftext[x=1.077810in,y=2.525866in,left,base]{\color{textcolor}\rmfamily\fontsize{10.000000}{12.000000}\selectfont GE}%
\end{pgfscope}%
\begin{pgfscope}%
\pgfsetroundcap%
\pgfsetroundjoin%
\pgfsetlinewidth{1.505625pt}%
\definecolor{currentstroke}{rgb}{0.839216,0.152941,0.156863}%
\pgfsetstrokecolor{currentstroke}%
\pgfsetdash{}{0pt}%
\pgfpathmoveto{\pgfqpoint{0.688921in}{2.370620in}}%
\pgfpathlineto{\pgfqpoint{0.966699in}{2.370620in}}%
\pgfusepath{stroke}%
\end{pgfscope}%
\begin{pgfscope}%
\definecolor{textcolor}{rgb}{0.150000,0.150000,0.150000}%
\pgfsetstrokecolor{textcolor}%
\pgfsetfillcolor{textcolor}%
\pgftext[x=1.077810in,y=2.322009in,left,base]{\color{textcolor}\rmfamily\fontsize{10.000000}{12.000000}\selectfont INTC}%
\end{pgfscope}%
\begin{pgfscope}%
\pgfsetroundcap%
\pgfsetroundjoin%
\pgfsetlinewidth{1.505625pt}%
\definecolor{currentstroke}{rgb}{0.580392,0.403922,0.741176}%
\pgfsetstrokecolor{currentstroke}%
\pgfsetdash{}{0pt}%
\pgfpathmoveto{\pgfqpoint{0.688921in}{2.166762in}}%
\pgfpathlineto{\pgfqpoint{0.966699in}{2.166762in}}%
\pgfusepath{stroke}%
\end{pgfscope}%
\begin{pgfscope}%
\definecolor{textcolor}{rgb}{0.150000,0.150000,0.150000}%
\pgfsetstrokecolor{textcolor}%
\pgfsetfillcolor{textcolor}%
\pgftext[x=1.077810in,y=2.118151in,left,base]{\color{textcolor}\rmfamily\fontsize{10.000000}{12.000000}\selectfont JNJ}%
\end{pgfscope}%
\begin{pgfscope}%
\pgfsetroundcap%
\pgfsetroundjoin%
\pgfsetlinewidth{1.505625pt}%
\definecolor{currentstroke}{rgb}{0.549020,0.337255,0.294118}%
\pgfsetstrokecolor{currentstroke}%
\pgfsetdash{}{0pt}%
\pgfpathmoveto{\pgfqpoint{0.688921in}{1.962905in}}%
\pgfpathlineto{\pgfqpoint{0.966699in}{1.962905in}}%
\pgfusepath{stroke}%
\end{pgfscope}%
\begin{pgfscope}%
\definecolor{textcolor}{rgb}{0.150000,0.150000,0.150000}%
\pgfsetstrokecolor{textcolor}%
\pgfsetfillcolor{textcolor}%
\pgftext[x=1.077810in,y=1.914294in,left,base]{\color{textcolor}\rmfamily\fontsize{10.000000}{12.000000}\selectfont PG}%
\end{pgfscope}%
\begin{pgfscope}%
\pgfsetroundcap%
\pgfsetroundjoin%
\pgfsetlinewidth{1.505625pt}%
\definecolor{currentstroke}{rgb}{0.890196,0.466667,0.760784}%
\pgfsetstrokecolor{currentstroke}%
\pgfsetdash{}{0pt}%
\pgfpathmoveto{\pgfqpoint{0.688921in}{1.759048in}}%
\pgfpathlineto{\pgfqpoint{0.966699in}{1.759048in}}%
\pgfusepath{stroke}%
\end{pgfscope}%
\begin{pgfscope}%
\definecolor{textcolor}{rgb}{0.150000,0.150000,0.150000}%
\pgfsetstrokecolor{textcolor}%
\pgfsetfillcolor{textcolor}%
\pgftext[x=1.077810in,y=1.710437in,left,base]{\color{textcolor}\rmfamily\fontsize{10.000000}{12.000000}\selectfont UTX}%
\end{pgfscope}%
\begin{pgfscope}%
\pgfsetroundcap%
\pgfsetroundjoin%
\pgfsetlinewidth{1.505625pt}%
\definecolor{currentstroke}{rgb}{0.498039,0.498039,0.498039}%
\pgfsetstrokecolor{currentstroke}%
\pgfsetdash{}{0pt}%
\pgfpathmoveto{\pgfqpoint{0.688921in}{1.555191in}}%
\pgfpathlineto{\pgfqpoint{0.966699in}{1.555191in}}%
\pgfusepath{stroke}%
\end{pgfscope}%
\begin{pgfscope}%
\definecolor{textcolor}{rgb}{0.150000,0.150000,0.150000}%
\pgfsetstrokecolor{textcolor}%
\pgfsetfillcolor{textcolor}%
\pgftext[x=1.077810in,y=1.506580in,left,base]{\color{textcolor}\rmfamily\fontsize{10.000000}{12.000000}\selectfont VZ}%
\end{pgfscope}%
\begin{pgfscope}%
\pgfsetroundcap%
\pgfsetroundjoin%
\pgfsetlinewidth{1.505625pt}%
\definecolor{currentstroke}{rgb}{0.737255,0.741176,0.133333}%
\pgfsetstrokecolor{currentstroke}%
\pgfsetdash{}{0pt}%
\pgfpathmoveto{\pgfqpoint{0.688921in}{1.351334in}}%
\pgfpathlineto{\pgfqpoint{0.966699in}{1.351334in}}%
\pgfusepath{stroke}%
\end{pgfscope}%
\begin{pgfscope}%
\definecolor{textcolor}{rgb}{0.150000,0.150000,0.150000}%
\pgfsetstrokecolor{textcolor}%
\pgfsetfillcolor{textcolor}%
\pgftext[x=1.077810in,y=1.302722in,left,base]{\color{textcolor}\rmfamily\fontsize{10.000000}{12.000000}\selectfont V}%
\end{pgfscope}%
\begin{pgfscope}%
\pgfsetroundcap%
\pgfsetroundjoin%
\pgfsetlinewidth{1.505625pt}%
\definecolor{currentstroke}{rgb}{0.090196,0.745098,0.811765}%
\pgfsetstrokecolor{currentstroke}%
\pgfsetdash{}{0pt}%
\pgfpathmoveto{\pgfqpoint{0.688921in}{1.147476in}}%
\pgfpathlineto{\pgfqpoint{0.966699in}{1.147476in}}%
\pgfusepath{stroke}%
\end{pgfscope}%
\begin{pgfscope}%
\definecolor{textcolor}{rgb}{0.150000,0.150000,0.150000}%
\pgfsetstrokecolor{textcolor}%
\pgfsetfillcolor{textcolor}%
\pgftext[x=1.077810in,y=1.098865in,left,base]{\color{textcolor}\rmfamily\fontsize{10.000000}{12.000000}\selectfont DIS}%
\end{pgfscope}%
\end{pgfpicture}%
\makeatother%
\endgroup%

    \end{adjustbox}  
    \caption{Standard deviation for the time series of stock prices. The value of the graph at point t is calculated as the standard deviation of all recorded values of the respective stocks up to that point t.}
    \label{fig:cum_sd_all}
\end{figure}{}

We take the log of the data in order to stabilize the variance and convert the exponential trend to a linear trend. 

ALTERNATIVE: calculate returns and plot variance of returns. 

We look at autocorrelation and partial autocorrelation. 
\begin{figure}[h]
    \centering
    \begin{adjustbox}{width=.9\textwidth,center}
    %% Creator: Matplotlib, PGF backend
%%
%% To include the figure in your LaTeX document, write
%%   \input{<filename>.pgf}
%%
%% Make sure the required packages are loaded in your preamble
%%   \usepackage{pgf}
%%
%% Figures using additional raster images can only be included by \input if
%% they are in the same directory as the main LaTeX file. For loading figures
%% from other directories you can use the `import` package
%%   \usepackage{import}
%% and then include the figures with
%%   \import{<path to file>}{<filename>.pgf}
%%
%% Matplotlib used the following preamble
%%   \usepackage{fontspec}
%%   \setmainfont{DejaVuSerif.ttf}[Path=/opt/tljh/user/lib/python3.6/site-packages/matplotlib/mpl-data/fonts/ttf/]
%%   \setsansfont{DejaVuSans.ttf}[Path=/opt/tljh/user/lib/python3.6/site-packages/matplotlib/mpl-data/fonts/ttf/]
%%   \setmonofont{DejaVuSansMono.ttf}[Path=/opt/tljh/user/lib/python3.6/site-packages/matplotlib/mpl-data/fonts/ttf/]
%%
\begingroup%
\makeatletter%
\begin{pgfpicture}%
\pgfpathrectangle{\pgfpointorigin}{\pgfqpoint{17.000000in}{20.000000in}}%
\pgfusepath{use as bounding box, clip}%
\begin{pgfscope}%
\pgfsetbuttcap%
\pgfsetmiterjoin%
\definecolor{currentfill}{rgb}{1.000000,1.000000,1.000000}%
\pgfsetfillcolor{currentfill}%
\pgfsetlinewidth{0.000000pt}%
\definecolor{currentstroke}{rgb}{1.000000,1.000000,1.000000}%
\pgfsetstrokecolor{currentstroke}%
\pgfsetdash{}{0pt}%
\pgfpathmoveto{\pgfqpoint{0.000000in}{0.000000in}}%
\pgfpathlineto{\pgfqpoint{17.000000in}{0.000000in}}%
\pgfpathlineto{\pgfqpoint{17.000000in}{20.000000in}}%
\pgfpathlineto{\pgfqpoint{0.000000in}{20.000000in}}%
\pgfpathclose%
\pgfusepath{fill}%
\end{pgfscope}%
\begin{pgfscope}%
\pgfsetbuttcap%
\pgfsetmiterjoin%
\definecolor{currentfill}{rgb}{0.917647,0.917647,0.949020}%
\pgfsetfillcolor{currentfill}%
\pgfsetlinewidth{0.000000pt}%
\definecolor{currentstroke}{rgb}{0.000000,0.000000,0.000000}%
\pgfsetstrokecolor{currentstroke}%
\pgfsetstrokeopacity{0.000000}%
\pgfsetdash{}{0pt}%
\pgfpathmoveto{\pgfqpoint{2.125000in}{16.722093in}}%
\pgfpathlineto{\pgfqpoint{7.614583in}{16.722093in}}%
\pgfpathlineto{\pgfqpoint{7.614583in}{17.600000in}}%
\pgfpathlineto{\pgfqpoint{2.125000in}{17.600000in}}%
\pgfpathclose%
\pgfusepath{fill}%
\end{pgfscope}%
\begin{pgfscope}%
\pgfpathrectangle{\pgfqpoint{2.125000in}{16.722093in}}{\pgfqpoint{5.489583in}{0.877907in}}%
\pgfusepath{clip}%
\pgfsetroundcap%
\pgfsetroundjoin%
\pgfsetlinewidth{0.803000pt}%
\definecolor{currentstroke}{rgb}{1.000000,1.000000,1.000000}%
\pgfsetstrokecolor{currentstroke}%
\pgfsetdash{}{0pt}%
\pgfpathmoveto{\pgfqpoint{2.374527in}{16.722093in}}%
\pgfpathlineto{\pgfqpoint{2.374527in}{17.600000in}}%
\pgfusepath{stroke}%
\end{pgfscope}%
\begin{pgfscope}%
\definecolor{textcolor}{rgb}{0.150000,0.150000,0.150000}%
\pgfsetstrokecolor{textcolor}%
\pgfsetfillcolor{textcolor}%
\pgftext[x=2.374527in,y=16.624871in,,top]{\color{textcolor}\rmfamily\fontsize{14.000000}{16.800000}\selectfont 0}%
\end{pgfscope}%
\begin{pgfscope}%
\pgfpathrectangle{\pgfqpoint{2.125000in}{16.722093in}}{\pgfqpoint{5.489583in}{0.877907in}}%
\pgfusepath{clip}%
\pgfsetroundcap%
\pgfsetroundjoin%
\pgfsetlinewidth{0.803000pt}%
\definecolor{currentstroke}{rgb}{1.000000,1.000000,1.000000}%
\pgfsetstrokecolor{currentstroke}%
\pgfsetdash{}{0pt}%
\pgfpathmoveto{\pgfqpoint{2.990641in}{16.722093in}}%
\pgfpathlineto{\pgfqpoint{2.990641in}{17.600000in}}%
\pgfusepath{stroke}%
\end{pgfscope}%
\begin{pgfscope}%
\definecolor{textcolor}{rgb}{0.150000,0.150000,0.150000}%
\pgfsetstrokecolor{textcolor}%
\pgfsetfillcolor{textcolor}%
\pgftext[x=2.990641in,y=16.624871in,,top]{\color{textcolor}\rmfamily\fontsize{14.000000}{16.800000}\selectfont 5}%
\end{pgfscope}%
\begin{pgfscope}%
\pgfpathrectangle{\pgfqpoint{2.125000in}{16.722093in}}{\pgfqpoint{5.489583in}{0.877907in}}%
\pgfusepath{clip}%
\pgfsetroundcap%
\pgfsetroundjoin%
\pgfsetlinewidth{0.803000pt}%
\definecolor{currentstroke}{rgb}{1.000000,1.000000,1.000000}%
\pgfsetstrokecolor{currentstroke}%
\pgfsetdash{}{0pt}%
\pgfpathmoveto{\pgfqpoint{3.606756in}{16.722093in}}%
\pgfpathlineto{\pgfqpoint{3.606756in}{17.600000in}}%
\pgfusepath{stroke}%
\end{pgfscope}%
\begin{pgfscope}%
\definecolor{textcolor}{rgb}{0.150000,0.150000,0.150000}%
\pgfsetstrokecolor{textcolor}%
\pgfsetfillcolor{textcolor}%
\pgftext[x=3.606756in,y=16.624871in,,top]{\color{textcolor}\rmfamily\fontsize{14.000000}{16.800000}\selectfont 10}%
\end{pgfscope}%
\begin{pgfscope}%
\pgfpathrectangle{\pgfqpoint{2.125000in}{16.722093in}}{\pgfqpoint{5.489583in}{0.877907in}}%
\pgfusepath{clip}%
\pgfsetroundcap%
\pgfsetroundjoin%
\pgfsetlinewidth{0.803000pt}%
\definecolor{currentstroke}{rgb}{1.000000,1.000000,1.000000}%
\pgfsetstrokecolor{currentstroke}%
\pgfsetdash{}{0pt}%
\pgfpathmoveto{\pgfqpoint{4.222871in}{16.722093in}}%
\pgfpathlineto{\pgfqpoint{4.222871in}{17.600000in}}%
\pgfusepath{stroke}%
\end{pgfscope}%
\begin{pgfscope}%
\definecolor{textcolor}{rgb}{0.150000,0.150000,0.150000}%
\pgfsetstrokecolor{textcolor}%
\pgfsetfillcolor{textcolor}%
\pgftext[x=4.222871in,y=16.624871in,,top]{\color{textcolor}\rmfamily\fontsize{14.000000}{16.800000}\selectfont 15}%
\end{pgfscope}%
\begin{pgfscope}%
\pgfpathrectangle{\pgfqpoint{2.125000in}{16.722093in}}{\pgfqpoint{5.489583in}{0.877907in}}%
\pgfusepath{clip}%
\pgfsetroundcap%
\pgfsetroundjoin%
\pgfsetlinewidth{0.803000pt}%
\definecolor{currentstroke}{rgb}{1.000000,1.000000,1.000000}%
\pgfsetstrokecolor{currentstroke}%
\pgfsetdash{}{0pt}%
\pgfpathmoveto{\pgfqpoint{4.838986in}{16.722093in}}%
\pgfpathlineto{\pgfqpoint{4.838986in}{17.600000in}}%
\pgfusepath{stroke}%
\end{pgfscope}%
\begin{pgfscope}%
\definecolor{textcolor}{rgb}{0.150000,0.150000,0.150000}%
\pgfsetstrokecolor{textcolor}%
\pgfsetfillcolor{textcolor}%
\pgftext[x=4.838986in,y=16.624871in,,top]{\color{textcolor}\rmfamily\fontsize{14.000000}{16.800000}\selectfont 20}%
\end{pgfscope}%
\begin{pgfscope}%
\pgfpathrectangle{\pgfqpoint{2.125000in}{16.722093in}}{\pgfqpoint{5.489583in}{0.877907in}}%
\pgfusepath{clip}%
\pgfsetroundcap%
\pgfsetroundjoin%
\pgfsetlinewidth{0.803000pt}%
\definecolor{currentstroke}{rgb}{1.000000,1.000000,1.000000}%
\pgfsetstrokecolor{currentstroke}%
\pgfsetdash{}{0pt}%
\pgfpathmoveto{\pgfqpoint{5.455101in}{16.722093in}}%
\pgfpathlineto{\pgfqpoint{5.455101in}{17.600000in}}%
\pgfusepath{stroke}%
\end{pgfscope}%
\begin{pgfscope}%
\definecolor{textcolor}{rgb}{0.150000,0.150000,0.150000}%
\pgfsetstrokecolor{textcolor}%
\pgfsetfillcolor{textcolor}%
\pgftext[x=5.455101in,y=16.624871in,,top]{\color{textcolor}\rmfamily\fontsize{14.000000}{16.800000}\selectfont 25}%
\end{pgfscope}%
\begin{pgfscope}%
\pgfpathrectangle{\pgfqpoint{2.125000in}{16.722093in}}{\pgfqpoint{5.489583in}{0.877907in}}%
\pgfusepath{clip}%
\pgfsetroundcap%
\pgfsetroundjoin%
\pgfsetlinewidth{0.803000pt}%
\definecolor{currentstroke}{rgb}{1.000000,1.000000,1.000000}%
\pgfsetstrokecolor{currentstroke}%
\pgfsetdash{}{0pt}%
\pgfpathmoveto{\pgfqpoint{6.071216in}{16.722093in}}%
\pgfpathlineto{\pgfqpoint{6.071216in}{17.600000in}}%
\pgfusepath{stroke}%
\end{pgfscope}%
\begin{pgfscope}%
\definecolor{textcolor}{rgb}{0.150000,0.150000,0.150000}%
\pgfsetstrokecolor{textcolor}%
\pgfsetfillcolor{textcolor}%
\pgftext[x=6.071216in,y=16.624871in,,top]{\color{textcolor}\rmfamily\fontsize{14.000000}{16.800000}\selectfont 30}%
\end{pgfscope}%
\begin{pgfscope}%
\pgfpathrectangle{\pgfqpoint{2.125000in}{16.722093in}}{\pgfqpoint{5.489583in}{0.877907in}}%
\pgfusepath{clip}%
\pgfsetroundcap%
\pgfsetroundjoin%
\pgfsetlinewidth{0.803000pt}%
\definecolor{currentstroke}{rgb}{1.000000,1.000000,1.000000}%
\pgfsetstrokecolor{currentstroke}%
\pgfsetdash{}{0pt}%
\pgfpathmoveto{\pgfqpoint{6.687330in}{16.722093in}}%
\pgfpathlineto{\pgfqpoint{6.687330in}{17.600000in}}%
\pgfusepath{stroke}%
\end{pgfscope}%
\begin{pgfscope}%
\definecolor{textcolor}{rgb}{0.150000,0.150000,0.150000}%
\pgfsetstrokecolor{textcolor}%
\pgfsetfillcolor{textcolor}%
\pgftext[x=6.687330in,y=16.624871in,,top]{\color{textcolor}\rmfamily\fontsize{14.000000}{16.800000}\selectfont 35}%
\end{pgfscope}%
\begin{pgfscope}%
\pgfpathrectangle{\pgfqpoint{2.125000in}{16.722093in}}{\pgfqpoint{5.489583in}{0.877907in}}%
\pgfusepath{clip}%
\pgfsetroundcap%
\pgfsetroundjoin%
\pgfsetlinewidth{0.803000pt}%
\definecolor{currentstroke}{rgb}{1.000000,1.000000,1.000000}%
\pgfsetstrokecolor{currentstroke}%
\pgfsetdash{}{0pt}%
\pgfpathmoveto{\pgfqpoint{7.303445in}{16.722093in}}%
\pgfpathlineto{\pgfqpoint{7.303445in}{17.600000in}}%
\pgfusepath{stroke}%
\end{pgfscope}%
\begin{pgfscope}%
\definecolor{textcolor}{rgb}{0.150000,0.150000,0.150000}%
\pgfsetstrokecolor{textcolor}%
\pgfsetfillcolor{textcolor}%
\pgftext[x=7.303445in,y=16.624871in,,top]{\color{textcolor}\rmfamily\fontsize{14.000000}{16.800000}\selectfont 40}%
\end{pgfscope}%
\begin{pgfscope}%
\pgfpathrectangle{\pgfqpoint{2.125000in}{16.722093in}}{\pgfqpoint{5.489583in}{0.877907in}}%
\pgfusepath{clip}%
\pgfsetroundcap%
\pgfsetroundjoin%
\pgfsetlinewidth{0.803000pt}%
\definecolor{currentstroke}{rgb}{1.000000,1.000000,1.000000}%
\pgfsetstrokecolor{currentstroke}%
\pgfsetdash{}{0pt}%
\pgfpathmoveto{\pgfqpoint{2.125000in}{17.000124in}}%
\pgfpathlineto{\pgfqpoint{7.614583in}{17.000124in}}%
\pgfusepath{stroke}%
\end{pgfscope}%
\begin{pgfscope}%
\definecolor{textcolor}{rgb}{0.150000,0.150000,0.150000}%
\pgfsetstrokecolor{textcolor}%
\pgfsetfillcolor{textcolor}%
\pgftext[x=1.904066in,y=16.926258in,left,base]{\color{textcolor}\rmfamily\fontsize{14.000000}{16.800000}\selectfont 0}%
\end{pgfscope}%
\begin{pgfscope}%
\pgfpathrectangle{\pgfqpoint{2.125000in}{16.722093in}}{\pgfqpoint{5.489583in}{0.877907in}}%
\pgfusepath{clip}%
\pgfsetroundcap%
\pgfsetroundjoin%
\pgfsetlinewidth{0.803000pt}%
\definecolor{currentstroke}{rgb}{1.000000,1.000000,1.000000}%
\pgfsetstrokecolor{currentstroke}%
\pgfsetdash{}{0pt}%
\pgfpathmoveto{\pgfqpoint{2.125000in}{17.560095in}}%
\pgfpathlineto{\pgfqpoint{7.614583in}{17.560095in}}%
\pgfusepath{stroke}%
\end{pgfscope}%
\begin{pgfscope}%
\definecolor{textcolor}{rgb}{0.150000,0.150000,0.150000}%
\pgfsetstrokecolor{textcolor}%
\pgfsetfillcolor{textcolor}%
\pgftext[x=1.904066in,y=17.486229in,left,base]{\color{textcolor}\rmfamily\fontsize{14.000000}{16.800000}\selectfont 1}%
\end{pgfscope}%
\begin{pgfscope}%
\pgfpathrectangle{\pgfqpoint{2.125000in}{16.722093in}}{\pgfqpoint{5.489583in}{0.877907in}}%
\pgfusepath{clip}%
\pgfsetbuttcap%
\pgfsetroundjoin%
\definecolor{currentfill}{rgb}{0.121569,0.466667,0.705882}%
\pgfsetfillcolor{currentfill}%
\pgfsetfillopacity{0.250000}%
\pgfsetlinewidth{1.003750pt}%
\definecolor{currentstroke}{rgb}{1.000000,1.000000,1.000000}%
\pgfsetstrokecolor{currentstroke}%
\pgfsetstrokeopacity{0.250000}%
\pgfsetdash{}{0pt}%
\pgfpathmoveto{\pgfqpoint{2.436138in}{17.028377in}}%
\pgfpathlineto{\pgfqpoint{2.436138in}{16.971870in}}%
\pgfpathlineto{\pgfqpoint{2.620972in}{16.951276in}}%
\pgfpathlineto{\pgfqpoint{2.744195in}{16.937151in}}%
\pgfpathlineto{\pgfqpoint{2.867418in}{16.925718in}}%
\pgfpathlineto{\pgfqpoint{2.990641in}{16.915872in}}%
\pgfpathlineto{\pgfqpoint{3.113864in}{16.907108in}}%
\pgfpathlineto{\pgfqpoint{3.237087in}{16.899142in}}%
\pgfpathlineto{\pgfqpoint{3.360310in}{16.891800in}}%
\pgfpathlineto{\pgfqpoint{3.483533in}{16.884962in}}%
\pgfpathlineto{\pgfqpoint{3.606756in}{16.878543in}}%
\pgfpathlineto{\pgfqpoint{3.729979in}{16.872481in}}%
\pgfpathlineto{\pgfqpoint{3.853202in}{16.866724in}}%
\pgfpathlineto{\pgfqpoint{3.976425in}{16.861236in}}%
\pgfpathlineto{\pgfqpoint{4.099648in}{16.855984in}}%
\pgfpathlineto{\pgfqpoint{4.222871in}{16.850945in}}%
\pgfpathlineto{\pgfqpoint{4.346094in}{16.846096in}}%
\pgfpathlineto{\pgfqpoint{4.469317in}{16.841422in}}%
\pgfpathlineto{\pgfqpoint{4.592540in}{16.836905in}}%
\pgfpathlineto{\pgfqpoint{4.715763in}{16.832534in}}%
\pgfpathlineto{\pgfqpoint{4.838986in}{16.828296in}}%
\pgfpathlineto{\pgfqpoint{4.962209in}{16.824184in}}%
\pgfpathlineto{\pgfqpoint{5.085432in}{16.820188in}}%
\pgfpathlineto{\pgfqpoint{5.208655in}{16.816301in}}%
\pgfpathlineto{\pgfqpoint{5.331878in}{16.812515in}}%
\pgfpathlineto{\pgfqpoint{5.455101in}{16.808823in}}%
\pgfpathlineto{\pgfqpoint{5.578324in}{16.805221in}}%
\pgfpathlineto{\pgfqpoint{5.701547in}{16.801703in}}%
\pgfpathlineto{\pgfqpoint{5.824770in}{16.798264in}}%
\pgfpathlineto{\pgfqpoint{5.947993in}{16.794901in}}%
\pgfpathlineto{\pgfqpoint{6.071216in}{16.791610in}}%
\pgfpathlineto{\pgfqpoint{6.194439in}{16.788387in}}%
\pgfpathlineto{\pgfqpoint{6.317662in}{16.785229in}}%
\pgfpathlineto{\pgfqpoint{6.440885in}{16.782133in}}%
\pgfpathlineto{\pgfqpoint{6.564108in}{16.779096in}}%
\pgfpathlineto{\pgfqpoint{6.687330in}{16.776116in}}%
\pgfpathlineto{\pgfqpoint{6.810553in}{16.773191in}}%
\pgfpathlineto{\pgfqpoint{6.933776in}{16.770318in}}%
\pgfpathlineto{\pgfqpoint{7.056999in}{16.767496in}}%
\pgfpathlineto{\pgfqpoint{7.180222in}{16.764724in}}%
\pgfpathlineto{\pgfqpoint{7.365057in}{16.761998in}}%
\pgfpathlineto{\pgfqpoint{7.365057in}{17.238250in}}%
\pgfpathlineto{\pgfqpoint{7.365057in}{17.238250in}}%
\pgfpathlineto{\pgfqpoint{7.180222in}{17.235524in}}%
\pgfpathlineto{\pgfqpoint{7.056999in}{17.232751in}}%
\pgfpathlineto{\pgfqpoint{6.933776in}{17.229929in}}%
\pgfpathlineto{\pgfqpoint{6.810553in}{17.227057in}}%
\pgfpathlineto{\pgfqpoint{6.687330in}{17.224131in}}%
\pgfpathlineto{\pgfqpoint{6.564108in}{17.221151in}}%
\pgfpathlineto{\pgfqpoint{6.440885in}{17.218114in}}%
\pgfpathlineto{\pgfqpoint{6.317662in}{17.215018in}}%
\pgfpathlineto{\pgfqpoint{6.194439in}{17.211860in}}%
\pgfpathlineto{\pgfqpoint{6.071216in}{17.208637in}}%
\pgfpathlineto{\pgfqpoint{5.947993in}{17.205346in}}%
\pgfpathlineto{\pgfqpoint{5.824770in}{17.201983in}}%
\pgfpathlineto{\pgfqpoint{5.701547in}{17.198545in}}%
\pgfpathlineto{\pgfqpoint{5.578324in}{17.195026in}}%
\pgfpathlineto{\pgfqpoint{5.455101in}{17.191424in}}%
\pgfpathlineto{\pgfqpoint{5.331878in}{17.187733in}}%
\pgfpathlineto{\pgfqpoint{5.208655in}{17.183947in}}%
\pgfpathlineto{\pgfqpoint{5.085432in}{17.180059in}}%
\pgfpathlineto{\pgfqpoint{4.962209in}{17.176064in}}%
\pgfpathlineto{\pgfqpoint{4.838986in}{17.171951in}}%
\pgfpathlineto{\pgfqpoint{4.715763in}{17.167714in}}%
\pgfpathlineto{\pgfqpoint{4.592540in}{17.163342in}}%
\pgfpathlineto{\pgfqpoint{4.469317in}{17.158826in}}%
\pgfpathlineto{\pgfqpoint{4.346094in}{17.154151in}}%
\pgfpathlineto{\pgfqpoint{4.222871in}{17.149303in}}%
\pgfpathlineto{\pgfqpoint{4.099648in}{17.144263in}}%
\pgfpathlineto{\pgfqpoint{3.976425in}{17.139012in}}%
\pgfpathlineto{\pgfqpoint{3.853202in}{17.133523in}}%
\pgfpathlineto{\pgfqpoint{3.729979in}{17.127767in}}%
\pgfpathlineto{\pgfqpoint{3.606756in}{17.121704in}}%
\pgfpathlineto{\pgfqpoint{3.483533in}{17.115286in}}%
\pgfpathlineto{\pgfqpoint{3.360310in}{17.108448in}}%
\pgfpathlineto{\pgfqpoint{3.237087in}{17.101105in}}%
\pgfpathlineto{\pgfqpoint{3.113864in}{17.093140in}}%
\pgfpathlineto{\pgfqpoint{2.990641in}{17.084375in}}%
\pgfpathlineto{\pgfqpoint{2.867418in}{17.074530in}}%
\pgfpathlineto{\pgfqpoint{2.744195in}{17.063096in}}%
\pgfpathlineto{\pgfqpoint{2.620972in}{17.048972in}}%
\pgfpathlineto{\pgfqpoint{2.436138in}{17.028377in}}%
\pgfpathclose%
\pgfusepath{stroke,fill}%
\end{pgfscope}%
\begin{pgfscope}%
\pgfpathrectangle{\pgfqpoint{2.125000in}{16.722093in}}{\pgfqpoint{5.489583in}{0.877907in}}%
\pgfusepath{clip}%
\pgfsetbuttcap%
\pgfsetroundjoin%
\pgfsetlinewidth{1.505625pt}%
\definecolor{currentstroke}{rgb}{0.000000,0.000000,0.000000}%
\pgfsetstrokecolor{currentstroke}%
\pgfsetdash{}{0pt}%
\pgfpathmoveto{\pgfqpoint{2.374527in}{17.000124in}}%
\pgfpathlineto{\pgfqpoint{2.374527in}{17.560095in}}%
\pgfusepath{stroke}%
\end{pgfscope}%
\begin{pgfscope}%
\pgfpathrectangle{\pgfqpoint{2.125000in}{16.722093in}}{\pgfqpoint{5.489583in}{0.877907in}}%
\pgfusepath{clip}%
\pgfsetbuttcap%
\pgfsetroundjoin%
\pgfsetlinewidth{1.505625pt}%
\definecolor{currentstroke}{rgb}{0.000000,0.000000,0.000000}%
\pgfsetstrokecolor{currentstroke}%
\pgfsetdash{}{0pt}%
\pgfpathmoveto{\pgfqpoint{2.497749in}{17.000124in}}%
\pgfpathlineto{\pgfqpoint{2.497749in}{17.558578in}}%
\pgfusepath{stroke}%
\end{pgfscope}%
\begin{pgfscope}%
\pgfpathrectangle{\pgfqpoint{2.125000in}{16.722093in}}{\pgfqpoint{5.489583in}{0.877907in}}%
\pgfusepath{clip}%
\pgfsetbuttcap%
\pgfsetroundjoin%
\pgfsetlinewidth{1.505625pt}%
\definecolor{currentstroke}{rgb}{0.000000,0.000000,0.000000}%
\pgfsetstrokecolor{currentstroke}%
\pgfsetdash{}{0pt}%
\pgfpathmoveto{\pgfqpoint{2.620972in}{17.000124in}}%
\pgfpathlineto{\pgfqpoint{2.620972in}{17.557087in}}%
\pgfusepath{stroke}%
\end{pgfscope}%
\begin{pgfscope}%
\pgfpathrectangle{\pgfqpoint{2.125000in}{16.722093in}}{\pgfqpoint{5.489583in}{0.877907in}}%
\pgfusepath{clip}%
\pgfsetbuttcap%
\pgfsetroundjoin%
\pgfsetlinewidth{1.505625pt}%
\definecolor{currentstroke}{rgb}{0.000000,0.000000,0.000000}%
\pgfsetstrokecolor{currentstroke}%
\pgfsetdash{}{0pt}%
\pgfpathmoveto{\pgfqpoint{2.744195in}{17.000124in}}%
\pgfpathlineto{\pgfqpoint{2.744195in}{17.555564in}}%
\pgfusepath{stroke}%
\end{pgfscope}%
\begin{pgfscope}%
\pgfpathrectangle{\pgfqpoint{2.125000in}{16.722093in}}{\pgfqpoint{5.489583in}{0.877907in}}%
\pgfusepath{clip}%
\pgfsetbuttcap%
\pgfsetroundjoin%
\pgfsetlinewidth{1.505625pt}%
\definecolor{currentstroke}{rgb}{0.000000,0.000000,0.000000}%
\pgfsetstrokecolor{currentstroke}%
\pgfsetdash{}{0pt}%
\pgfpathmoveto{\pgfqpoint{2.867418in}{17.000124in}}%
\pgfpathlineto{\pgfqpoint{2.867418in}{17.554022in}}%
\pgfusepath{stroke}%
\end{pgfscope}%
\begin{pgfscope}%
\pgfpathrectangle{\pgfqpoint{2.125000in}{16.722093in}}{\pgfqpoint{5.489583in}{0.877907in}}%
\pgfusepath{clip}%
\pgfsetbuttcap%
\pgfsetroundjoin%
\pgfsetlinewidth{1.505625pt}%
\definecolor{currentstroke}{rgb}{0.000000,0.000000,0.000000}%
\pgfsetstrokecolor{currentstroke}%
\pgfsetdash{}{0pt}%
\pgfpathmoveto{\pgfqpoint{2.990641in}{17.000124in}}%
\pgfpathlineto{\pgfqpoint{2.990641in}{17.552526in}}%
\pgfusepath{stroke}%
\end{pgfscope}%
\begin{pgfscope}%
\pgfpathrectangle{\pgfqpoint{2.125000in}{16.722093in}}{\pgfqpoint{5.489583in}{0.877907in}}%
\pgfusepath{clip}%
\pgfsetbuttcap%
\pgfsetroundjoin%
\pgfsetlinewidth{1.505625pt}%
\definecolor{currentstroke}{rgb}{0.000000,0.000000,0.000000}%
\pgfsetstrokecolor{currentstroke}%
\pgfsetdash{}{0pt}%
\pgfpathmoveto{\pgfqpoint{3.113864in}{17.000124in}}%
\pgfpathlineto{\pgfqpoint{3.113864in}{17.551044in}}%
\pgfusepath{stroke}%
\end{pgfscope}%
\begin{pgfscope}%
\pgfpathrectangle{\pgfqpoint{2.125000in}{16.722093in}}{\pgfqpoint{5.489583in}{0.877907in}}%
\pgfusepath{clip}%
\pgfsetbuttcap%
\pgfsetroundjoin%
\pgfsetlinewidth{1.505625pt}%
\definecolor{currentstroke}{rgb}{0.000000,0.000000,0.000000}%
\pgfsetstrokecolor{currentstroke}%
\pgfsetdash{}{0pt}%
\pgfpathmoveto{\pgfqpoint{3.237087in}{17.000124in}}%
\pgfpathlineto{\pgfqpoint{3.237087in}{17.549523in}}%
\pgfusepath{stroke}%
\end{pgfscope}%
\begin{pgfscope}%
\pgfpathrectangle{\pgfqpoint{2.125000in}{16.722093in}}{\pgfqpoint{5.489583in}{0.877907in}}%
\pgfusepath{clip}%
\pgfsetbuttcap%
\pgfsetroundjoin%
\pgfsetlinewidth{1.505625pt}%
\definecolor{currentstroke}{rgb}{0.000000,0.000000,0.000000}%
\pgfsetstrokecolor{currentstroke}%
\pgfsetdash{}{0pt}%
\pgfpathmoveto{\pgfqpoint{3.360310in}{17.000124in}}%
\pgfpathlineto{\pgfqpoint{3.360310in}{17.547995in}}%
\pgfusepath{stroke}%
\end{pgfscope}%
\begin{pgfscope}%
\pgfpathrectangle{\pgfqpoint{2.125000in}{16.722093in}}{\pgfqpoint{5.489583in}{0.877907in}}%
\pgfusepath{clip}%
\pgfsetbuttcap%
\pgfsetroundjoin%
\pgfsetlinewidth{1.505625pt}%
\definecolor{currentstroke}{rgb}{0.000000,0.000000,0.000000}%
\pgfsetstrokecolor{currentstroke}%
\pgfsetdash{}{0pt}%
\pgfpathmoveto{\pgfqpoint{3.483533in}{17.000124in}}%
\pgfpathlineto{\pgfqpoint{3.483533in}{17.546426in}}%
\pgfusepath{stroke}%
\end{pgfscope}%
\begin{pgfscope}%
\pgfpathrectangle{\pgfqpoint{2.125000in}{16.722093in}}{\pgfqpoint{5.489583in}{0.877907in}}%
\pgfusepath{clip}%
\pgfsetbuttcap%
\pgfsetroundjoin%
\pgfsetlinewidth{1.505625pt}%
\definecolor{currentstroke}{rgb}{0.000000,0.000000,0.000000}%
\pgfsetstrokecolor{currentstroke}%
\pgfsetdash{}{0pt}%
\pgfpathmoveto{\pgfqpoint{3.606756in}{17.000124in}}%
\pgfpathlineto{\pgfqpoint{3.606756in}{17.544892in}}%
\pgfusepath{stroke}%
\end{pgfscope}%
\begin{pgfscope}%
\pgfpathrectangle{\pgfqpoint{2.125000in}{16.722093in}}{\pgfqpoint{5.489583in}{0.877907in}}%
\pgfusepath{clip}%
\pgfsetbuttcap%
\pgfsetroundjoin%
\pgfsetlinewidth{1.505625pt}%
\definecolor{currentstroke}{rgb}{0.000000,0.000000,0.000000}%
\pgfsetstrokecolor{currentstroke}%
\pgfsetdash{}{0pt}%
\pgfpathmoveto{\pgfqpoint{3.729979in}{17.000124in}}%
\pgfpathlineto{\pgfqpoint{3.729979in}{17.543391in}}%
\pgfusepath{stroke}%
\end{pgfscope}%
\begin{pgfscope}%
\pgfpathrectangle{\pgfqpoint{2.125000in}{16.722093in}}{\pgfqpoint{5.489583in}{0.877907in}}%
\pgfusepath{clip}%
\pgfsetbuttcap%
\pgfsetroundjoin%
\pgfsetlinewidth{1.505625pt}%
\definecolor{currentstroke}{rgb}{0.000000,0.000000,0.000000}%
\pgfsetstrokecolor{currentstroke}%
\pgfsetdash{}{0pt}%
\pgfpathmoveto{\pgfqpoint{3.853202in}{17.000124in}}%
\pgfpathlineto{\pgfqpoint{3.853202in}{17.541902in}}%
\pgfusepath{stroke}%
\end{pgfscope}%
\begin{pgfscope}%
\pgfpathrectangle{\pgfqpoint{2.125000in}{16.722093in}}{\pgfqpoint{5.489583in}{0.877907in}}%
\pgfusepath{clip}%
\pgfsetbuttcap%
\pgfsetroundjoin%
\pgfsetlinewidth{1.505625pt}%
\definecolor{currentstroke}{rgb}{0.000000,0.000000,0.000000}%
\pgfsetstrokecolor{currentstroke}%
\pgfsetdash{}{0pt}%
\pgfpathmoveto{\pgfqpoint{3.976425in}{17.000124in}}%
\pgfpathlineto{\pgfqpoint{3.976425in}{17.540435in}}%
\pgfusepath{stroke}%
\end{pgfscope}%
\begin{pgfscope}%
\pgfpathrectangle{\pgfqpoint{2.125000in}{16.722093in}}{\pgfqpoint{5.489583in}{0.877907in}}%
\pgfusepath{clip}%
\pgfsetbuttcap%
\pgfsetroundjoin%
\pgfsetlinewidth{1.505625pt}%
\definecolor{currentstroke}{rgb}{0.000000,0.000000,0.000000}%
\pgfsetstrokecolor{currentstroke}%
\pgfsetdash{}{0pt}%
\pgfpathmoveto{\pgfqpoint{4.099648in}{17.000124in}}%
\pgfpathlineto{\pgfqpoint{4.099648in}{17.538939in}}%
\pgfusepath{stroke}%
\end{pgfscope}%
\begin{pgfscope}%
\pgfpathrectangle{\pgfqpoint{2.125000in}{16.722093in}}{\pgfqpoint{5.489583in}{0.877907in}}%
\pgfusepath{clip}%
\pgfsetbuttcap%
\pgfsetroundjoin%
\pgfsetlinewidth{1.505625pt}%
\definecolor{currentstroke}{rgb}{0.000000,0.000000,0.000000}%
\pgfsetstrokecolor{currentstroke}%
\pgfsetdash{}{0pt}%
\pgfpathmoveto{\pgfqpoint{4.222871in}{17.000124in}}%
\pgfpathlineto{\pgfqpoint{4.222871in}{17.537452in}}%
\pgfusepath{stroke}%
\end{pgfscope}%
\begin{pgfscope}%
\pgfpathrectangle{\pgfqpoint{2.125000in}{16.722093in}}{\pgfqpoint{5.489583in}{0.877907in}}%
\pgfusepath{clip}%
\pgfsetbuttcap%
\pgfsetroundjoin%
\pgfsetlinewidth{1.505625pt}%
\definecolor{currentstroke}{rgb}{0.000000,0.000000,0.000000}%
\pgfsetstrokecolor{currentstroke}%
\pgfsetdash{}{0pt}%
\pgfpathmoveto{\pgfqpoint{4.346094in}{17.000124in}}%
\pgfpathlineto{\pgfqpoint{4.346094in}{17.535968in}}%
\pgfusepath{stroke}%
\end{pgfscope}%
\begin{pgfscope}%
\pgfpathrectangle{\pgfqpoint{2.125000in}{16.722093in}}{\pgfqpoint{5.489583in}{0.877907in}}%
\pgfusepath{clip}%
\pgfsetbuttcap%
\pgfsetroundjoin%
\pgfsetlinewidth{1.505625pt}%
\definecolor{currentstroke}{rgb}{0.000000,0.000000,0.000000}%
\pgfsetstrokecolor{currentstroke}%
\pgfsetdash{}{0pt}%
\pgfpathmoveto{\pgfqpoint{4.469317in}{17.000124in}}%
\pgfpathlineto{\pgfqpoint{4.469317in}{17.534527in}}%
\pgfusepath{stroke}%
\end{pgfscope}%
\begin{pgfscope}%
\pgfpathrectangle{\pgfqpoint{2.125000in}{16.722093in}}{\pgfqpoint{5.489583in}{0.877907in}}%
\pgfusepath{clip}%
\pgfsetbuttcap%
\pgfsetroundjoin%
\pgfsetlinewidth{1.505625pt}%
\definecolor{currentstroke}{rgb}{0.000000,0.000000,0.000000}%
\pgfsetstrokecolor{currentstroke}%
\pgfsetdash{}{0pt}%
\pgfpathmoveto{\pgfqpoint{4.592540in}{17.000124in}}%
\pgfpathlineto{\pgfqpoint{4.592540in}{17.533072in}}%
\pgfusepath{stroke}%
\end{pgfscope}%
\begin{pgfscope}%
\pgfpathrectangle{\pgfqpoint{2.125000in}{16.722093in}}{\pgfqpoint{5.489583in}{0.877907in}}%
\pgfusepath{clip}%
\pgfsetbuttcap%
\pgfsetroundjoin%
\pgfsetlinewidth{1.505625pt}%
\definecolor{currentstroke}{rgb}{0.000000,0.000000,0.000000}%
\pgfsetstrokecolor{currentstroke}%
\pgfsetdash{}{0pt}%
\pgfpathmoveto{\pgfqpoint{4.715763in}{17.000124in}}%
\pgfpathlineto{\pgfqpoint{4.715763in}{17.531613in}}%
\pgfusepath{stroke}%
\end{pgfscope}%
\begin{pgfscope}%
\pgfpathrectangle{\pgfqpoint{2.125000in}{16.722093in}}{\pgfqpoint{5.489583in}{0.877907in}}%
\pgfusepath{clip}%
\pgfsetbuttcap%
\pgfsetroundjoin%
\pgfsetlinewidth{1.505625pt}%
\definecolor{currentstroke}{rgb}{0.000000,0.000000,0.000000}%
\pgfsetstrokecolor{currentstroke}%
\pgfsetdash{}{0pt}%
\pgfpathmoveto{\pgfqpoint{4.838986in}{17.000124in}}%
\pgfpathlineto{\pgfqpoint{4.838986in}{17.530118in}}%
\pgfusepath{stroke}%
\end{pgfscope}%
\begin{pgfscope}%
\pgfpathrectangle{\pgfqpoint{2.125000in}{16.722093in}}{\pgfqpoint{5.489583in}{0.877907in}}%
\pgfusepath{clip}%
\pgfsetbuttcap%
\pgfsetroundjoin%
\pgfsetlinewidth{1.505625pt}%
\definecolor{currentstroke}{rgb}{0.000000,0.000000,0.000000}%
\pgfsetstrokecolor{currentstroke}%
\pgfsetdash{}{0pt}%
\pgfpathmoveto{\pgfqpoint{4.962209in}{17.000124in}}%
\pgfpathlineto{\pgfqpoint{4.962209in}{17.528611in}}%
\pgfusepath{stroke}%
\end{pgfscope}%
\begin{pgfscope}%
\pgfpathrectangle{\pgfqpoint{2.125000in}{16.722093in}}{\pgfqpoint{5.489583in}{0.877907in}}%
\pgfusepath{clip}%
\pgfsetbuttcap%
\pgfsetroundjoin%
\pgfsetlinewidth{1.505625pt}%
\definecolor{currentstroke}{rgb}{0.000000,0.000000,0.000000}%
\pgfsetstrokecolor{currentstroke}%
\pgfsetdash{}{0pt}%
\pgfpathmoveto{\pgfqpoint{5.085432in}{17.000124in}}%
\pgfpathlineto{\pgfqpoint{5.085432in}{17.527124in}}%
\pgfusepath{stroke}%
\end{pgfscope}%
\begin{pgfscope}%
\pgfpathrectangle{\pgfqpoint{2.125000in}{16.722093in}}{\pgfqpoint{5.489583in}{0.877907in}}%
\pgfusepath{clip}%
\pgfsetbuttcap%
\pgfsetroundjoin%
\pgfsetlinewidth{1.505625pt}%
\definecolor{currentstroke}{rgb}{0.000000,0.000000,0.000000}%
\pgfsetstrokecolor{currentstroke}%
\pgfsetdash{}{0pt}%
\pgfpathmoveto{\pgfqpoint{5.208655in}{17.000124in}}%
\pgfpathlineto{\pgfqpoint{5.208655in}{17.525681in}}%
\pgfusepath{stroke}%
\end{pgfscope}%
\begin{pgfscope}%
\pgfpathrectangle{\pgfqpoint{2.125000in}{16.722093in}}{\pgfqpoint{5.489583in}{0.877907in}}%
\pgfusepath{clip}%
\pgfsetbuttcap%
\pgfsetroundjoin%
\pgfsetlinewidth{1.505625pt}%
\definecolor{currentstroke}{rgb}{0.000000,0.000000,0.000000}%
\pgfsetstrokecolor{currentstroke}%
\pgfsetdash{}{0pt}%
\pgfpathmoveto{\pgfqpoint{5.331878in}{17.000124in}}%
\pgfpathlineto{\pgfqpoint{5.331878in}{17.524252in}}%
\pgfusepath{stroke}%
\end{pgfscope}%
\begin{pgfscope}%
\pgfpathrectangle{\pgfqpoint{2.125000in}{16.722093in}}{\pgfqpoint{5.489583in}{0.877907in}}%
\pgfusepath{clip}%
\pgfsetbuttcap%
\pgfsetroundjoin%
\pgfsetlinewidth{1.505625pt}%
\definecolor{currentstroke}{rgb}{0.000000,0.000000,0.000000}%
\pgfsetstrokecolor{currentstroke}%
\pgfsetdash{}{0pt}%
\pgfpathmoveto{\pgfqpoint{5.455101in}{17.000124in}}%
\pgfpathlineto{\pgfqpoint{5.455101in}{17.522852in}}%
\pgfusepath{stroke}%
\end{pgfscope}%
\begin{pgfscope}%
\pgfpathrectangle{\pgfqpoint{2.125000in}{16.722093in}}{\pgfqpoint{5.489583in}{0.877907in}}%
\pgfusepath{clip}%
\pgfsetbuttcap%
\pgfsetroundjoin%
\pgfsetlinewidth{1.505625pt}%
\definecolor{currentstroke}{rgb}{0.000000,0.000000,0.000000}%
\pgfsetstrokecolor{currentstroke}%
\pgfsetdash{}{0pt}%
\pgfpathmoveto{\pgfqpoint{5.578324in}{17.000124in}}%
\pgfpathlineto{\pgfqpoint{5.578324in}{17.521471in}}%
\pgfusepath{stroke}%
\end{pgfscope}%
\begin{pgfscope}%
\pgfpathrectangle{\pgfqpoint{2.125000in}{16.722093in}}{\pgfqpoint{5.489583in}{0.877907in}}%
\pgfusepath{clip}%
\pgfsetbuttcap%
\pgfsetroundjoin%
\pgfsetlinewidth{1.505625pt}%
\definecolor{currentstroke}{rgb}{0.000000,0.000000,0.000000}%
\pgfsetstrokecolor{currentstroke}%
\pgfsetdash{}{0pt}%
\pgfpathmoveto{\pgfqpoint{5.701547in}{17.000124in}}%
\pgfpathlineto{\pgfqpoint{5.701547in}{17.520063in}}%
\pgfusepath{stroke}%
\end{pgfscope}%
\begin{pgfscope}%
\pgfpathrectangle{\pgfqpoint{2.125000in}{16.722093in}}{\pgfqpoint{5.489583in}{0.877907in}}%
\pgfusepath{clip}%
\pgfsetbuttcap%
\pgfsetroundjoin%
\pgfsetlinewidth{1.505625pt}%
\definecolor{currentstroke}{rgb}{0.000000,0.000000,0.000000}%
\pgfsetstrokecolor{currentstroke}%
\pgfsetdash{}{0pt}%
\pgfpathmoveto{\pgfqpoint{5.824770in}{17.000124in}}%
\pgfpathlineto{\pgfqpoint{5.824770in}{17.518648in}}%
\pgfusepath{stroke}%
\end{pgfscope}%
\begin{pgfscope}%
\pgfpathrectangle{\pgfqpoint{2.125000in}{16.722093in}}{\pgfqpoint{5.489583in}{0.877907in}}%
\pgfusepath{clip}%
\pgfsetbuttcap%
\pgfsetroundjoin%
\pgfsetlinewidth{1.505625pt}%
\definecolor{currentstroke}{rgb}{0.000000,0.000000,0.000000}%
\pgfsetstrokecolor{currentstroke}%
\pgfsetdash{}{0pt}%
\pgfpathmoveto{\pgfqpoint{5.947993in}{17.000124in}}%
\pgfpathlineto{\pgfqpoint{5.947993in}{17.517275in}}%
\pgfusepath{stroke}%
\end{pgfscope}%
\begin{pgfscope}%
\pgfpathrectangle{\pgfqpoint{2.125000in}{16.722093in}}{\pgfqpoint{5.489583in}{0.877907in}}%
\pgfusepath{clip}%
\pgfsetbuttcap%
\pgfsetroundjoin%
\pgfsetlinewidth{1.505625pt}%
\definecolor{currentstroke}{rgb}{0.000000,0.000000,0.000000}%
\pgfsetstrokecolor{currentstroke}%
\pgfsetdash{}{0pt}%
\pgfpathmoveto{\pgfqpoint{6.071216in}{17.000124in}}%
\pgfpathlineto{\pgfqpoint{6.071216in}{17.515893in}}%
\pgfusepath{stroke}%
\end{pgfscope}%
\begin{pgfscope}%
\pgfpathrectangle{\pgfqpoint{2.125000in}{16.722093in}}{\pgfqpoint{5.489583in}{0.877907in}}%
\pgfusepath{clip}%
\pgfsetbuttcap%
\pgfsetroundjoin%
\pgfsetlinewidth{1.505625pt}%
\definecolor{currentstroke}{rgb}{0.000000,0.000000,0.000000}%
\pgfsetstrokecolor{currentstroke}%
\pgfsetdash{}{0pt}%
\pgfpathmoveto{\pgfqpoint{6.194439in}{17.000124in}}%
\pgfpathlineto{\pgfqpoint{6.194439in}{17.514547in}}%
\pgfusepath{stroke}%
\end{pgfscope}%
\begin{pgfscope}%
\pgfpathrectangle{\pgfqpoint{2.125000in}{16.722093in}}{\pgfqpoint{5.489583in}{0.877907in}}%
\pgfusepath{clip}%
\pgfsetbuttcap%
\pgfsetroundjoin%
\pgfsetlinewidth{1.505625pt}%
\definecolor{currentstroke}{rgb}{0.000000,0.000000,0.000000}%
\pgfsetstrokecolor{currentstroke}%
\pgfsetdash{}{0pt}%
\pgfpathmoveto{\pgfqpoint{6.317662in}{17.000124in}}%
\pgfpathlineto{\pgfqpoint{6.317662in}{17.513191in}}%
\pgfusepath{stroke}%
\end{pgfscope}%
\begin{pgfscope}%
\pgfpathrectangle{\pgfqpoint{2.125000in}{16.722093in}}{\pgfqpoint{5.489583in}{0.877907in}}%
\pgfusepath{clip}%
\pgfsetbuttcap%
\pgfsetroundjoin%
\pgfsetlinewidth{1.505625pt}%
\definecolor{currentstroke}{rgb}{0.000000,0.000000,0.000000}%
\pgfsetstrokecolor{currentstroke}%
\pgfsetdash{}{0pt}%
\pgfpathmoveto{\pgfqpoint{6.440885in}{17.000124in}}%
\pgfpathlineto{\pgfqpoint{6.440885in}{17.511844in}}%
\pgfusepath{stroke}%
\end{pgfscope}%
\begin{pgfscope}%
\pgfpathrectangle{\pgfqpoint{2.125000in}{16.722093in}}{\pgfqpoint{5.489583in}{0.877907in}}%
\pgfusepath{clip}%
\pgfsetbuttcap%
\pgfsetroundjoin%
\pgfsetlinewidth{1.505625pt}%
\definecolor{currentstroke}{rgb}{0.000000,0.000000,0.000000}%
\pgfsetstrokecolor{currentstroke}%
\pgfsetdash{}{0pt}%
\pgfpathmoveto{\pgfqpoint{6.564108in}{17.000124in}}%
\pgfpathlineto{\pgfqpoint{6.564108in}{17.510514in}}%
\pgfusepath{stroke}%
\end{pgfscope}%
\begin{pgfscope}%
\pgfpathrectangle{\pgfqpoint{2.125000in}{16.722093in}}{\pgfqpoint{5.489583in}{0.877907in}}%
\pgfusepath{clip}%
\pgfsetbuttcap%
\pgfsetroundjoin%
\pgfsetlinewidth{1.505625pt}%
\definecolor{currentstroke}{rgb}{0.000000,0.000000,0.000000}%
\pgfsetstrokecolor{currentstroke}%
\pgfsetdash{}{0pt}%
\pgfpathmoveto{\pgfqpoint{6.687330in}{17.000124in}}%
\pgfpathlineto{\pgfqpoint{6.687330in}{17.509144in}}%
\pgfusepath{stroke}%
\end{pgfscope}%
\begin{pgfscope}%
\pgfpathrectangle{\pgfqpoint{2.125000in}{16.722093in}}{\pgfqpoint{5.489583in}{0.877907in}}%
\pgfusepath{clip}%
\pgfsetbuttcap%
\pgfsetroundjoin%
\pgfsetlinewidth{1.505625pt}%
\definecolor{currentstroke}{rgb}{0.000000,0.000000,0.000000}%
\pgfsetstrokecolor{currentstroke}%
\pgfsetdash{}{0pt}%
\pgfpathmoveto{\pgfqpoint{6.810553in}{17.000124in}}%
\pgfpathlineto{\pgfqpoint{6.810553in}{17.507760in}}%
\pgfusepath{stroke}%
\end{pgfscope}%
\begin{pgfscope}%
\pgfpathrectangle{\pgfqpoint{2.125000in}{16.722093in}}{\pgfqpoint{5.489583in}{0.877907in}}%
\pgfusepath{clip}%
\pgfsetbuttcap%
\pgfsetroundjoin%
\pgfsetlinewidth{1.505625pt}%
\definecolor{currentstroke}{rgb}{0.000000,0.000000,0.000000}%
\pgfsetstrokecolor{currentstroke}%
\pgfsetdash{}{0pt}%
\pgfpathmoveto{\pgfqpoint{6.933776in}{17.000124in}}%
\pgfpathlineto{\pgfqpoint{6.933776in}{17.506375in}}%
\pgfusepath{stroke}%
\end{pgfscope}%
\begin{pgfscope}%
\pgfpathrectangle{\pgfqpoint{2.125000in}{16.722093in}}{\pgfqpoint{5.489583in}{0.877907in}}%
\pgfusepath{clip}%
\pgfsetbuttcap%
\pgfsetroundjoin%
\pgfsetlinewidth{1.505625pt}%
\definecolor{currentstroke}{rgb}{0.000000,0.000000,0.000000}%
\pgfsetstrokecolor{currentstroke}%
\pgfsetdash{}{0pt}%
\pgfpathmoveto{\pgfqpoint{7.056999in}{17.000124in}}%
\pgfpathlineto{\pgfqpoint{7.056999in}{17.505002in}}%
\pgfusepath{stroke}%
\end{pgfscope}%
\begin{pgfscope}%
\pgfpathrectangle{\pgfqpoint{2.125000in}{16.722093in}}{\pgfqpoint{5.489583in}{0.877907in}}%
\pgfusepath{clip}%
\pgfsetbuttcap%
\pgfsetroundjoin%
\pgfsetlinewidth{1.505625pt}%
\definecolor{currentstroke}{rgb}{0.000000,0.000000,0.000000}%
\pgfsetstrokecolor{currentstroke}%
\pgfsetdash{}{0pt}%
\pgfpathmoveto{\pgfqpoint{7.180222in}{17.000124in}}%
\pgfpathlineto{\pgfqpoint{7.180222in}{17.503608in}}%
\pgfusepath{stroke}%
\end{pgfscope}%
\begin{pgfscope}%
\pgfpathrectangle{\pgfqpoint{2.125000in}{16.722093in}}{\pgfqpoint{5.489583in}{0.877907in}}%
\pgfusepath{clip}%
\pgfsetbuttcap%
\pgfsetroundjoin%
\pgfsetlinewidth{1.505625pt}%
\definecolor{currentstroke}{rgb}{0.000000,0.000000,0.000000}%
\pgfsetstrokecolor{currentstroke}%
\pgfsetdash{}{0pt}%
\pgfpathmoveto{\pgfqpoint{7.303445in}{17.000124in}}%
\pgfpathlineto{\pgfqpoint{7.303445in}{17.502202in}}%
\pgfusepath{stroke}%
\end{pgfscope}%
\begin{pgfscope}%
\pgfpathrectangle{\pgfqpoint{2.125000in}{16.722093in}}{\pgfqpoint{5.489583in}{0.877907in}}%
\pgfusepath{clip}%
\pgfsetroundcap%
\pgfsetroundjoin%
\pgfsetlinewidth{1.505625pt}%
\definecolor{currentstroke}{rgb}{0.121569,0.466667,0.705882}%
\pgfsetstrokecolor{currentstroke}%
\pgfsetdash{}{0pt}%
\pgfpathmoveto{\pgfqpoint{2.125000in}{17.000124in}}%
\pgfpathlineto{\pgfqpoint{7.614583in}{17.000124in}}%
\pgfusepath{stroke}%
\end{pgfscope}%
\begin{pgfscope}%
\pgfpathrectangle{\pgfqpoint{2.125000in}{16.722093in}}{\pgfqpoint{5.489583in}{0.877907in}}%
\pgfusepath{clip}%
\pgfsetbuttcap%
\pgfsetroundjoin%
\definecolor{currentfill}{rgb}{0.121569,0.466667,0.705882}%
\pgfsetfillcolor{currentfill}%
\pgfsetlinewidth{1.003750pt}%
\definecolor{currentstroke}{rgb}{0.121569,0.466667,0.705882}%
\pgfsetstrokecolor{currentstroke}%
\pgfsetdash{}{0pt}%
\pgfsys@defobject{currentmarker}{\pgfqpoint{-0.034722in}{-0.034722in}}{\pgfqpoint{0.034722in}{0.034722in}}{%
\pgfpathmoveto{\pgfqpoint{0.000000in}{-0.034722in}}%
\pgfpathcurveto{\pgfqpoint{0.009208in}{-0.034722in}}{\pgfqpoint{0.018041in}{-0.031064in}}{\pgfqpoint{0.024552in}{-0.024552in}}%
\pgfpathcurveto{\pgfqpoint{0.031064in}{-0.018041in}}{\pgfqpoint{0.034722in}{-0.009208in}}{\pgfqpoint{0.034722in}{0.000000in}}%
\pgfpathcurveto{\pgfqpoint{0.034722in}{0.009208in}}{\pgfqpoint{0.031064in}{0.018041in}}{\pgfqpoint{0.024552in}{0.024552in}}%
\pgfpathcurveto{\pgfqpoint{0.018041in}{0.031064in}}{\pgfqpoint{0.009208in}{0.034722in}}{\pgfqpoint{0.000000in}{0.034722in}}%
\pgfpathcurveto{\pgfqpoint{-0.009208in}{0.034722in}}{\pgfqpoint{-0.018041in}{0.031064in}}{\pgfqpoint{-0.024552in}{0.024552in}}%
\pgfpathcurveto{\pgfqpoint{-0.031064in}{0.018041in}}{\pgfqpoint{-0.034722in}{0.009208in}}{\pgfqpoint{-0.034722in}{0.000000in}}%
\pgfpathcurveto{\pgfqpoint{-0.034722in}{-0.009208in}}{\pgfqpoint{-0.031064in}{-0.018041in}}{\pgfqpoint{-0.024552in}{-0.024552in}}%
\pgfpathcurveto{\pgfqpoint{-0.018041in}{-0.031064in}}{\pgfqpoint{-0.009208in}{-0.034722in}}{\pgfqpoint{0.000000in}{-0.034722in}}%
\pgfpathclose%
\pgfusepath{stroke,fill}%
}%
\begin{pgfscope}%
\pgfsys@transformshift{2.374527in}{17.560095in}%
\pgfsys@useobject{currentmarker}{}%
\end{pgfscope}%
\begin{pgfscope}%
\pgfsys@transformshift{2.497749in}{17.558578in}%
\pgfsys@useobject{currentmarker}{}%
\end{pgfscope}%
\begin{pgfscope}%
\pgfsys@transformshift{2.620972in}{17.557087in}%
\pgfsys@useobject{currentmarker}{}%
\end{pgfscope}%
\begin{pgfscope}%
\pgfsys@transformshift{2.744195in}{17.555564in}%
\pgfsys@useobject{currentmarker}{}%
\end{pgfscope}%
\begin{pgfscope}%
\pgfsys@transformshift{2.867418in}{17.554022in}%
\pgfsys@useobject{currentmarker}{}%
\end{pgfscope}%
\begin{pgfscope}%
\pgfsys@transformshift{2.990641in}{17.552526in}%
\pgfsys@useobject{currentmarker}{}%
\end{pgfscope}%
\begin{pgfscope}%
\pgfsys@transformshift{3.113864in}{17.551044in}%
\pgfsys@useobject{currentmarker}{}%
\end{pgfscope}%
\begin{pgfscope}%
\pgfsys@transformshift{3.237087in}{17.549523in}%
\pgfsys@useobject{currentmarker}{}%
\end{pgfscope}%
\begin{pgfscope}%
\pgfsys@transformshift{3.360310in}{17.547995in}%
\pgfsys@useobject{currentmarker}{}%
\end{pgfscope}%
\begin{pgfscope}%
\pgfsys@transformshift{3.483533in}{17.546426in}%
\pgfsys@useobject{currentmarker}{}%
\end{pgfscope}%
\begin{pgfscope}%
\pgfsys@transformshift{3.606756in}{17.544892in}%
\pgfsys@useobject{currentmarker}{}%
\end{pgfscope}%
\begin{pgfscope}%
\pgfsys@transformshift{3.729979in}{17.543391in}%
\pgfsys@useobject{currentmarker}{}%
\end{pgfscope}%
\begin{pgfscope}%
\pgfsys@transformshift{3.853202in}{17.541902in}%
\pgfsys@useobject{currentmarker}{}%
\end{pgfscope}%
\begin{pgfscope}%
\pgfsys@transformshift{3.976425in}{17.540435in}%
\pgfsys@useobject{currentmarker}{}%
\end{pgfscope}%
\begin{pgfscope}%
\pgfsys@transformshift{4.099648in}{17.538939in}%
\pgfsys@useobject{currentmarker}{}%
\end{pgfscope}%
\begin{pgfscope}%
\pgfsys@transformshift{4.222871in}{17.537452in}%
\pgfsys@useobject{currentmarker}{}%
\end{pgfscope}%
\begin{pgfscope}%
\pgfsys@transformshift{4.346094in}{17.535968in}%
\pgfsys@useobject{currentmarker}{}%
\end{pgfscope}%
\begin{pgfscope}%
\pgfsys@transformshift{4.469317in}{17.534527in}%
\pgfsys@useobject{currentmarker}{}%
\end{pgfscope}%
\begin{pgfscope}%
\pgfsys@transformshift{4.592540in}{17.533072in}%
\pgfsys@useobject{currentmarker}{}%
\end{pgfscope}%
\begin{pgfscope}%
\pgfsys@transformshift{4.715763in}{17.531613in}%
\pgfsys@useobject{currentmarker}{}%
\end{pgfscope}%
\begin{pgfscope}%
\pgfsys@transformshift{4.838986in}{17.530118in}%
\pgfsys@useobject{currentmarker}{}%
\end{pgfscope}%
\begin{pgfscope}%
\pgfsys@transformshift{4.962209in}{17.528611in}%
\pgfsys@useobject{currentmarker}{}%
\end{pgfscope}%
\begin{pgfscope}%
\pgfsys@transformshift{5.085432in}{17.527124in}%
\pgfsys@useobject{currentmarker}{}%
\end{pgfscope}%
\begin{pgfscope}%
\pgfsys@transformshift{5.208655in}{17.525681in}%
\pgfsys@useobject{currentmarker}{}%
\end{pgfscope}%
\begin{pgfscope}%
\pgfsys@transformshift{5.331878in}{17.524252in}%
\pgfsys@useobject{currentmarker}{}%
\end{pgfscope}%
\begin{pgfscope}%
\pgfsys@transformshift{5.455101in}{17.522852in}%
\pgfsys@useobject{currentmarker}{}%
\end{pgfscope}%
\begin{pgfscope}%
\pgfsys@transformshift{5.578324in}{17.521471in}%
\pgfsys@useobject{currentmarker}{}%
\end{pgfscope}%
\begin{pgfscope}%
\pgfsys@transformshift{5.701547in}{17.520063in}%
\pgfsys@useobject{currentmarker}{}%
\end{pgfscope}%
\begin{pgfscope}%
\pgfsys@transformshift{5.824770in}{17.518648in}%
\pgfsys@useobject{currentmarker}{}%
\end{pgfscope}%
\begin{pgfscope}%
\pgfsys@transformshift{5.947993in}{17.517275in}%
\pgfsys@useobject{currentmarker}{}%
\end{pgfscope}%
\begin{pgfscope}%
\pgfsys@transformshift{6.071216in}{17.515893in}%
\pgfsys@useobject{currentmarker}{}%
\end{pgfscope}%
\begin{pgfscope}%
\pgfsys@transformshift{6.194439in}{17.514547in}%
\pgfsys@useobject{currentmarker}{}%
\end{pgfscope}%
\begin{pgfscope}%
\pgfsys@transformshift{6.317662in}{17.513191in}%
\pgfsys@useobject{currentmarker}{}%
\end{pgfscope}%
\begin{pgfscope}%
\pgfsys@transformshift{6.440885in}{17.511844in}%
\pgfsys@useobject{currentmarker}{}%
\end{pgfscope}%
\begin{pgfscope}%
\pgfsys@transformshift{6.564108in}{17.510514in}%
\pgfsys@useobject{currentmarker}{}%
\end{pgfscope}%
\begin{pgfscope}%
\pgfsys@transformshift{6.687330in}{17.509144in}%
\pgfsys@useobject{currentmarker}{}%
\end{pgfscope}%
\begin{pgfscope}%
\pgfsys@transformshift{6.810553in}{17.507760in}%
\pgfsys@useobject{currentmarker}{}%
\end{pgfscope}%
\begin{pgfscope}%
\pgfsys@transformshift{6.933776in}{17.506375in}%
\pgfsys@useobject{currentmarker}{}%
\end{pgfscope}%
\begin{pgfscope}%
\pgfsys@transformshift{7.056999in}{17.505002in}%
\pgfsys@useobject{currentmarker}{}%
\end{pgfscope}%
\begin{pgfscope}%
\pgfsys@transformshift{7.180222in}{17.503608in}%
\pgfsys@useobject{currentmarker}{}%
\end{pgfscope}%
\begin{pgfscope}%
\pgfsys@transformshift{7.303445in}{17.502202in}%
\pgfsys@useobject{currentmarker}{}%
\end{pgfscope}%
\end{pgfscope}%
\begin{pgfscope}%
\pgfsetrectcap%
\pgfsetmiterjoin%
\pgfsetlinewidth{0.803000pt}%
\definecolor{currentstroke}{rgb}{1.000000,1.000000,1.000000}%
\pgfsetstrokecolor{currentstroke}%
\pgfsetdash{}{0pt}%
\pgfpathmoveto{\pgfqpoint{2.125000in}{16.722093in}}%
\pgfpathlineto{\pgfqpoint{2.125000in}{17.600000in}}%
\pgfusepath{stroke}%
\end{pgfscope}%
\begin{pgfscope}%
\pgfsetrectcap%
\pgfsetmiterjoin%
\pgfsetlinewidth{0.803000pt}%
\definecolor{currentstroke}{rgb}{1.000000,1.000000,1.000000}%
\pgfsetstrokecolor{currentstroke}%
\pgfsetdash{}{0pt}%
\pgfpathmoveto{\pgfqpoint{7.614583in}{16.722093in}}%
\pgfpathlineto{\pgfqpoint{7.614583in}{17.600000in}}%
\pgfusepath{stroke}%
\end{pgfscope}%
\begin{pgfscope}%
\pgfsetrectcap%
\pgfsetmiterjoin%
\pgfsetlinewidth{0.803000pt}%
\definecolor{currentstroke}{rgb}{1.000000,1.000000,1.000000}%
\pgfsetstrokecolor{currentstroke}%
\pgfsetdash{}{0pt}%
\pgfpathmoveto{\pgfqpoint{2.125000in}{16.722093in}}%
\pgfpathlineto{\pgfqpoint{7.614583in}{16.722093in}}%
\pgfusepath{stroke}%
\end{pgfscope}%
\begin{pgfscope}%
\pgfsetrectcap%
\pgfsetmiterjoin%
\pgfsetlinewidth{0.803000pt}%
\definecolor{currentstroke}{rgb}{1.000000,1.000000,1.000000}%
\pgfsetstrokecolor{currentstroke}%
\pgfsetdash{}{0pt}%
\pgfpathmoveto{\pgfqpoint{2.125000in}{17.600000in}}%
\pgfpathlineto{\pgfqpoint{7.614583in}{17.600000in}}%
\pgfusepath{stroke}%
\end{pgfscope}%
\begin{pgfscope}%
\definecolor{textcolor}{rgb}{0.150000,0.150000,0.150000}%
\pgfsetstrokecolor{textcolor}%
\pgfsetfillcolor{textcolor}%
\pgftext[x=4.869792in,y=17.683333in,,base]{\color{textcolor}\rmfamily\fontsize{16.800000}{20.160000}\selectfont Autocorrelation}%
\end{pgfscope}%
\begin{pgfscope}%
\pgfsetbuttcap%
\pgfsetmiterjoin%
\definecolor{currentfill}{rgb}{0.917647,0.917647,0.949020}%
\pgfsetfillcolor{currentfill}%
\pgfsetlinewidth{0.000000pt}%
\definecolor{currentstroke}{rgb}{0.000000,0.000000,0.000000}%
\pgfsetstrokecolor{currentstroke}%
\pgfsetstrokeopacity{0.000000}%
\pgfsetdash{}{0pt}%
\pgfpathmoveto{\pgfqpoint{9.810417in}{16.722093in}}%
\pgfpathlineto{\pgfqpoint{15.300000in}{16.722093in}}%
\pgfpathlineto{\pgfqpoint{15.300000in}{17.600000in}}%
\pgfpathlineto{\pgfqpoint{9.810417in}{17.600000in}}%
\pgfpathclose%
\pgfusepath{fill}%
\end{pgfscope}%
\begin{pgfscope}%
\pgfpathrectangle{\pgfqpoint{9.810417in}{16.722093in}}{\pgfqpoint{5.489583in}{0.877907in}}%
\pgfusepath{clip}%
\pgfsetroundcap%
\pgfsetroundjoin%
\pgfsetlinewidth{0.803000pt}%
\definecolor{currentstroke}{rgb}{1.000000,1.000000,1.000000}%
\pgfsetstrokecolor{currentstroke}%
\pgfsetdash{}{0pt}%
\pgfpathmoveto{\pgfqpoint{10.059943in}{16.722093in}}%
\pgfpathlineto{\pgfqpoint{10.059943in}{17.600000in}}%
\pgfusepath{stroke}%
\end{pgfscope}%
\begin{pgfscope}%
\definecolor{textcolor}{rgb}{0.150000,0.150000,0.150000}%
\pgfsetstrokecolor{textcolor}%
\pgfsetfillcolor{textcolor}%
\pgftext[x=10.059943in,y=16.624871in,,top]{\color{textcolor}\rmfamily\fontsize{14.000000}{16.800000}\selectfont 0}%
\end{pgfscope}%
\begin{pgfscope}%
\pgfpathrectangle{\pgfqpoint{9.810417in}{16.722093in}}{\pgfqpoint{5.489583in}{0.877907in}}%
\pgfusepath{clip}%
\pgfsetroundcap%
\pgfsetroundjoin%
\pgfsetlinewidth{0.803000pt}%
\definecolor{currentstroke}{rgb}{1.000000,1.000000,1.000000}%
\pgfsetstrokecolor{currentstroke}%
\pgfsetdash{}{0pt}%
\pgfpathmoveto{\pgfqpoint{10.676058in}{16.722093in}}%
\pgfpathlineto{\pgfqpoint{10.676058in}{17.600000in}}%
\pgfusepath{stroke}%
\end{pgfscope}%
\begin{pgfscope}%
\definecolor{textcolor}{rgb}{0.150000,0.150000,0.150000}%
\pgfsetstrokecolor{textcolor}%
\pgfsetfillcolor{textcolor}%
\pgftext[x=10.676058in,y=16.624871in,,top]{\color{textcolor}\rmfamily\fontsize{14.000000}{16.800000}\selectfont 5}%
\end{pgfscope}%
\begin{pgfscope}%
\pgfpathrectangle{\pgfqpoint{9.810417in}{16.722093in}}{\pgfqpoint{5.489583in}{0.877907in}}%
\pgfusepath{clip}%
\pgfsetroundcap%
\pgfsetroundjoin%
\pgfsetlinewidth{0.803000pt}%
\definecolor{currentstroke}{rgb}{1.000000,1.000000,1.000000}%
\pgfsetstrokecolor{currentstroke}%
\pgfsetdash{}{0pt}%
\pgfpathmoveto{\pgfqpoint{11.292173in}{16.722093in}}%
\pgfpathlineto{\pgfqpoint{11.292173in}{17.600000in}}%
\pgfusepath{stroke}%
\end{pgfscope}%
\begin{pgfscope}%
\definecolor{textcolor}{rgb}{0.150000,0.150000,0.150000}%
\pgfsetstrokecolor{textcolor}%
\pgfsetfillcolor{textcolor}%
\pgftext[x=11.292173in,y=16.624871in,,top]{\color{textcolor}\rmfamily\fontsize{14.000000}{16.800000}\selectfont 10}%
\end{pgfscope}%
\begin{pgfscope}%
\pgfpathrectangle{\pgfqpoint{9.810417in}{16.722093in}}{\pgfqpoint{5.489583in}{0.877907in}}%
\pgfusepath{clip}%
\pgfsetroundcap%
\pgfsetroundjoin%
\pgfsetlinewidth{0.803000pt}%
\definecolor{currentstroke}{rgb}{1.000000,1.000000,1.000000}%
\pgfsetstrokecolor{currentstroke}%
\pgfsetdash{}{0pt}%
\pgfpathmoveto{\pgfqpoint{11.908288in}{16.722093in}}%
\pgfpathlineto{\pgfqpoint{11.908288in}{17.600000in}}%
\pgfusepath{stroke}%
\end{pgfscope}%
\begin{pgfscope}%
\definecolor{textcolor}{rgb}{0.150000,0.150000,0.150000}%
\pgfsetstrokecolor{textcolor}%
\pgfsetfillcolor{textcolor}%
\pgftext[x=11.908288in,y=16.624871in,,top]{\color{textcolor}\rmfamily\fontsize{14.000000}{16.800000}\selectfont 15}%
\end{pgfscope}%
\begin{pgfscope}%
\pgfpathrectangle{\pgfqpoint{9.810417in}{16.722093in}}{\pgfqpoint{5.489583in}{0.877907in}}%
\pgfusepath{clip}%
\pgfsetroundcap%
\pgfsetroundjoin%
\pgfsetlinewidth{0.803000pt}%
\definecolor{currentstroke}{rgb}{1.000000,1.000000,1.000000}%
\pgfsetstrokecolor{currentstroke}%
\pgfsetdash{}{0pt}%
\pgfpathmoveto{\pgfqpoint{12.524403in}{16.722093in}}%
\pgfpathlineto{\pgfqpoint{12.524403in}{17.600000in}}%
\pgfusepath{stroke}%
\end{pgfscope}%
\begin{pgfscope}%
\definecolor{textcolor}{rgb}{0.150000,0.150000,0.150000}%
\pgfsetstrokecolor{textcolor}%
\pgfsetfillcolor{textcolor}%
\pgftext[x=12.524403in,y=16.624871in,,top]{\color{textcolor}\rmfamily\fontsize{14.000000}{16.800000}\selectfont 20}%
\end{pgfscope}%
\begin{pgfscope}%
\pgfpathrectangle{\pgfqpoint{9.810417in}{16.722093in}}{\pgfqpoint{5.489583in}{0.877907in}}%
\pgfusepath{clip}%
\pgfsetroundcap%
\pgfsetroundjoin%
\pgfsetlinewidth{0.803000pt}%
\definecolor{currentstroke}{rgb}{1.000000,1.000000,1.000000}%
\pgfsetstrokecolor{currentstroke}%
\pgfsetdash{}{0pt}%
\pgfpathmoveto{\pgfqpoint{13.140517in}{16.722093in}}%
\pgfpathlineto{\pgfqpoint{13.140517in}{17.600000in}}%
\pgfusepath{stroke}%
\end{pgfscope}%
\begin{pgfscope}%
\definecolor{textcolor}{rgb}{0.150000,0.150000,0.150000}%
\pgfsetstrokecolor{textcolor}%
\pgfsetfillcolor{textcolor}%
\pgftext[x=13.140517in,y=16.624871in,,top]{\color{textcolor}\rmfamily\fontsize{14.000000}{16.800000}\selectfont 25}%
\end{pgfscope}%
\begin{pgfscope}%
\pgfpathrectangle{\pgfqpoint{9.810417in}{16.722093in}}{\pgfqpoint{5.489583in}{0.877907in}}%
\pgfusepath{clip}%
\pgfsetroundcap%
\pgfsetroundjoin%
\pgfsetlinewidth{0.803000pt}%
\definecolor{currentstroke}{rgb}{1.000000,1.000000,1.000000}%
\pgfsetstrokecolor{currentstroke}%
\pgfsetdash{}{0pt}%
\pgfpathmoveto{\pgfqpoint{13.756632in}{16.722093in}}%
\pgfpathlineto{\pgfqpoint{13.756632in}{17.600000in}}%
\pgfusepath{stroke}%
\end{pgfscope}%
\begin{pgfscope}%
\definecolor{textcolor}{rgb}{0.150000,0.150000,0.150000}%
\pgfsetstrokecolor{textcolor}%
\pgfsetfillcolor{textcolor}%
\pgftext[x=13.756632in,y=16.624871in,,top]{\color{textcolor}\rmfamily\fontsize{14.000000}{16.800000}\selectfont 30}%
\end{pgfscope}%
\begin{pgfscope}%
\pgfpathrectangle{\pgfqpoint{9.810417in}{16.722093in}}{\pgfqpoint{5.489583in}{0.877907in}}%
\pgfusepath{clip}%
\pgfsetroundcap%
\pgfsetroundjoin%
\pgfsetlinewidth{0.803000pt}%
\definecolor{currentstroke}{rgb}{1.000000,1.000000,1.000000}%
\pgfsetstrokecolor{currentstroke}%
\pgfsetdash{}{0pt}%
\pgfpathmoveto{\pgfqpoint{14.372747in}{16.722093in}}%
\pgfpathlineto{\pgfqpoint{14.372747in}{17.600000in}}%
\pgfusepath{stroke}%
\end{pgfscope}%
\begin{pgfscope}%
\definecolor{textcolor}{rgb}{0.150000,0.150000,0.150000}%
\pgfsetstrokecolor{textcolor}%
\pgfsetfillcolor{textcolor}%
\pgftext[x=14.372747in,y=16.624871in,,top]{\color{textcolor}\rmfamily\fontsize{14.000000}{16.800000}\selectfont 35}%
\end{pgfscope}%
\begin{pgfscope}%
\pgfpathrectangle{\pgfqpoint{9.810417in}{16.722093in}}{\pgfqpoint{5.489583in}{0.877907in}}%
\pgfusepath{clip}%
\pgfsetroundcap%
\pgfsetroundjoin%
\pgfsetlinewidth{0.803000pt}%
\definecolor{currentstroke}{rgb}{1.000000,1.000000,1.000000}%
\pgfsetstrokecolor{currentstroke}%
\pgfsetdash{}{0pt}%
\pgfpathmoveto{\pgfqpoint{14.988862in}{16.722093in}}%
\pgfpathlineto{\pgfqpoint{14.988862in}{17.600000in}}%
\pgfusepath{stroke}%
\end{pgfscope}%
\begin{pgfscope}%
\definecolor{textcolor}{rgb}{0.150000,0.150000,0.150000}%
\pgfsetstrokecolor{textcolor}%
\pgfsetfillcolor{textcolor}%
\pgftext[x=14.988862in,y=16.624871in,,top]{\color{textcolor}\rmfamily\fontsize{14.000000}{16.800000}\selectfont 40}%
\end{pgfscope}%
\begin{pgfscope}%
\pgfpathrectangle{\pgfqpoint{9.810417in}{16.722093in}}{\pgfqpoint{5.489583in}{0.877907in}}%
\pgfusepath{clip}%
\pgfsetroundcap%
\pgfsetroundjoin%
\pgfsetlinewidth{0.803000pt}%
\definecolor{currentstroke}{rgb}{1.000000,1.000000,1.000000}%
\pgfsetstrokecolor{currentstroke}%
\pgfsetdash{}{0pt}%
\pgfpathmoveto{\pgfqpoint{9.810417in}{16.800332in}}%
\pgfpathlineto{\pgfqpoint{15.300000in}{16.800332in}}%
\pgfusepath{stroke}%
\end{pgfscope}%
\begin{pgfscope}%
\definecolor{textcolor}{rgb}{0.150000,0.150000,0.150000}%
\pgfsetstrokecolor{textcolor}%
\pgfsetfillcolor{textcolor}%
\pgftext[x=9.589483in,y=16.726466in,left,base]{\color{textcolor}\rmfamily\fontsize{14.000000}{16.800000}\selectfont 0}%
\end{pgfscope}%
\begin{pgfscope}%
\pgfpathrectangle{\pgfqpoint{9.810417in}{16.722093in}}{\pgfqpoint{5.489583in}{0.877907in}}%
\pgfusepath{clip}%
\pgfsetroundcap%
\pgfsetroundjoin%
\pgfsetlinewidth{0.803000pt}%
\definecolor{currentstroke}{rgb}{1.000000,1.000000,1.000000}%
\pgfsetstrokecolor{currentstroke}%
\pgfsetdash{}{0pt}%
\pgfpathmoveto{\pgfqpoint{9.810417in}{17.560095in}}%
\pgfpathlineto{\pgfqpoint{15.300000in}{17.560095in}}%
\pgfusepath{stroke}%
\end{pgfscope}%
\begin{pgfscope}%
\definecolor{textcolor}{rgb}{0.150000,0.150000,0.150000}%
\pgfsetstrokecolor{textcolor}%
\pgfsetfillcolor{textcolor}%
\pgftext[x=9.589483in,y=17.486229in,left,base]{\color{textcolor}\rmfamily\fontsize{14.000000}{16.800000}\selectfont 1}%
\end{pgfscope}%
\begin{pgfscope}%
\pgfpathrectangle{\pgfqpoint{9.810417in}{16.722093in}}{\pgfqpoint{5.489583in}{0.877907in}}%
\pgfusepath{clip}%
\pgfsetbuttcap%
\pgfsetroundjoin%
\definecolor{currentfill}{rgb}{0.121569,0.466667,0.705882}%
\pgfsetfillcolor{currentfill}%
\pgfsetfillopacity{0.250000}%
\pgfsetlinewidth{1.003750pt}%
\definecolor{currentstroke}{rgb}{1.000000,1.000000,1.000000}%
\pgfsetstrokecolor{currentstroke}%
\pgfsetstrokeopacity{0.250000}%
\pgfsetdash{}{0pt}%
\pgfpathmoveto{\pgfqpoint{10.121555in}{16.838665in}}%
\pgfpathlineto{\pgfqpoint{10.121555in}{16.761998in}}%
\pgfpathlineto{\pgfqpoint{10.306389in}{16.761998in}}%
\pgfpathlineto{\pgfqpoint{10.429612in}{16.761998in}}%
\pgfpathlineto{\pgfqpoint{10.552835in}{16.761998in}}%
\pgfpathlineto{\pgfqpoint{10.676058in}{16.761998in}}%
\pgfpathlineto{\pgfqpoint{10.799281in}{16.761998in}}%
\pgfpathlineto{\pgfqpoint{10.922504in}{16.761998in}}%
\pgfpathlineto{\pgfqpoint{11.045727in}{16.761998in}}%
\pgfpathlineto{\pgfqpoint{11.168950in}{16.761998in}}%
\pgfpathlineto{\pgfqpoint{11.292173in}{16.761998in}}%
\pgfpathlineto{\pgfqpoint{11.415396in}{16.761998in}}%
\pgfpathlineto{\pgfqpoint{11.538619in}{16.761998in}}%
\pgfpathlineto{\pgfqpoint{11.661842in}{16.761998in}}%
\pgfpathlineto{\pgfqpoint{11.785065in}{16.761998in}}%
\pgfpathlineto{\pgfqpoint{11.908288in}{16.761998in}}%
\pgfpathlineto{\pgfqpoint{12.031511in}{16.761998in}}%
\pgfpathlineto{\pgfqpoint{12.154734in}{16.761998in}}%
\pgfpathlineto{\pgfqpoint{12.277957in}{16.761998in}}%
\pgfpathlineto{\pgfqpoint{12.401180in}{16.761998in}}%
\pgfpathlineto{\pgfqpoint{12.524403in}{16.761998in}}%
\pgfpathlineto{\pgfqpoint{12.647626in}{16.761998in}}%
\pgfpathlineto{\pgfqpoint{12.770849in}{16.761998in}}%
\pgfpathlineto{\pgfqpoint{12.894072in}{16.761998in}}%
\pgfpathlineto{\pgfqpoint{13.017294in}{16.761998in}}%
\pgfpathlineto{\pgfqpoint{13.140517in}{16.761998in}}%
\pgfpathlineto{\pgfqpoint{13.263740in}{16.761998in}}%
\pgfpathlineto{\pgfqpoint{13.386963in}{16.761998in}}%
\pgfpathlineto{\pgfqpoint{13.510186in}{16.761998in}}%
\pgfpathlineto{\pgfqpoint{13.633409in}{16.761998in}}%
\pgfpathlineto{\pgfqpoint{13.756632in}{16.761998in}}%
\pgfpathlineto{\pgfqpoint{13.879855in}{16.761998in}}%
\pgfpathlineto{\pgfqpoint{14.003078in}{16.761998in}}%
\pgfpathlineto{\pgfqpoint{14.126301in}{16.761998in}}%
\pgfpathlineto{\pgfqpoint{14.249524in}{16.761998in}}%
\pgfpathlineto{\pgfqpoint{14.372747in}{16.761998in}}%
\pgfpathlineto{\pgfqpoint{14.495970in}{16.761998in}}%
\pgfpathlineto{\pgfqpoint{14.619193in}{16.761998in}}%
\pgfpathlineto{\pgfqpoint{14.742416in}{16.761998in}}%
\pgfpathlineto{\pgfqpoint{14.865639in}{16.761998in}}%
\pgfpathlineto{\pgfqpoint{15.050473in}{16.761998in}}%
\pgfpathlineto{\pgfqpoint{15.050473in}{16.838665in}}%
\pgfpathlineto{\pgfqpoint{15.050473in}{16.838665in}}%
\pgfpathlineto{\pgfqpoint{14.865639in}{16.838665in}}%
\pgfpathlineto{\pgfqpoint{14.742416in}{16.838665in}}%
\pgfpathlineto{\pgfqpoint{14.619193in}{16.838665in}}%
\pgfpathlineto{\pgfqpoint{14.495970in}{16.838665in}}%
\pgfpathlineto{\pgfqpoint{14.372747in}{16.838665in}}%
\pgfpathlineto{\pgfqpoint{14.249524in}{16.838665in}}%
\pgfpathlineto{\pgfqpoint{14.126301in}{16.838665in}}%
\pgfpathlineto{\pgfqpoint{14.003078in}{16.838665in}}%
\pgfpathlineto{\pgfqpoint{13.879855in}{16.838665in}}%
\pgfpathlineto{\pgfqpoint{13.756632in}{16.838665in}}%
\pgfpathlineto{\pgfqpoint{13.633409in}{16.838665in}}%
\pgfpathlineto{\pgfqpoint{13.510186in}{16.838665in}}%
\pgfpathlineto{\pgfqpoint{13.386963in}{16.838665in}}%
\pgfpathlineto{\pgfqpoint{13.263740in}{16.838665in}}%
\pgfpathlineto{\pgfqpoint{13.140517in}{16.838665in}}%
\pgfpathlineto{\pgfqpoint{13.017294in}{16.838665in}}%
\pgfpathlineto{\pgfqpoint{12.894072in}{16.838665in}}%
\pgfpathlineto{\pgfqpoint{12.770849in}{16.838665in}}%
\pgfpathlineto{\pgfqpoint{12.647626in}{16.838665in}}%
\pgfpathlineto{\pgfqpoint{12.524403in}{16.838665in}}%
\pgfpathlineto{\pgfqpoint{12.401180in}{16.838665in}}%
\pgfpathlineto{\pgfqpoint{12.277957in}{16.838665in}}%
\pgfpathlineto{\pgfqpoint{12.154734in}{16.838665in}}%
\pgfpathlineto{\pgfqpoint{12.031511in}{16.838665in}}%
\pgfpathlineto{\pgfqpoint{11.908288in}{16.838665in}}%
\pgfpathlineto{\pgfqpoint{11.785065in}{16.838665in}}%
\pgfpathlineto{\pgfqpoint{11.661842in}{16.838665in}}%
\pgfpathlineto{\pgfqpoint{11.538619in}{16.838665in}}%
\pgfpathlineto{\pgfqpoint{11.415396in}{16.838665in}}%
\pgfpathlineto{\pgfqpoint{11.292173in}{16.838665in}}%
\pgfpathlineto{\pgfqpoint{11.168950in}{16.838665in}}%
\pgfpathlineto{\pgfqpoint{11.045727in}{16.838665in}}%
\pgfpathlineto{\pgfqpoint{10.922504in}{16.838665in}}%
\pgfpathlineto{\pgfqpoint{10.799281in}{16.838665in}}%
\pgfpathlineto{\pgfqpoint{10.676058in}{16.838665in}}%
\pgfpathlineto{\pgfqpoint{10.552835in}{16.838665in}}%
\pgfpathlineto{\pgfqpoint{10.429612in}{16.838665in}}%
\pgfpathlineto{\pgfqpoint{10.306389in}{16.838665in}}%
\pgfpathlineto{\pgfqpoint{10.121555in}{16.838665in}}%
\pgfpathclose%
\pgfusepath{stroke,fill}%
\end{pgfscope}%
\begin{pgfscope}%
\pgfpathrectangle{\pgfqpoint{9.810417in}{16.722093in}}{\pgfqpoint{5.489583in}{0.877907in}}%
\pgfusepath{clip}%
\pgfsetbuttcap%
\pgfsetroundjoin%
\pgfsetlinewidth{1.505625pt}%
\definecolor{currentstroke}{rgb}{0.000000,0.000000,0.000000}%
\pgfsetstrokecolor{currentstroke}%
\pgfsetdash{}{0pt}%
\pgfpathmoveto{\pgfqpoint{10.059943in}{16.800332in}}%
\pgfpathlineto{\pgfqpoint{10.059943in}{17.560095in}}%
\pgfusepath{stroke}%
\end{pgfscope}%
\begin{pgfscope}%
\pgfpathrectangle{\pgfqpoint{9.810417in}{16.722093in}}{\pgfqpoint{5.489583in}{0.877907in}}%
\pgfusepath{clip}%
\pgfsetbuttcap%
\pgfsetroundjoin%
\pgfsetlinewidth{1.505625pt}%
\definecolor{currentstroke}{rgb}{0.000000,0.000000,0.000000}%
\pgfsetstrokecolor{currentstroke}%
\pgfsetdash{}{0pt}%
\pgfpathmoveto{\pgfqpoint{10.183166in}{16.800332in}}%
\pgfpathlineto{\pgfqpoint{10.183166in}{17.558539in}}%
\pgfusepath{stroke}%
\end{pgfscope}%
\begin{pgfscope}%
\pgfpathrectangle{\pgfqpoint{9.810417in}{16.722093in}}{\pgfqpoint{5.489583in}{0.877907in}}%
\pgfusepath{clip}%
\pgfsetbuttcap%
\pgfsetroundjoin%
\pgfsetlinewidth{1.505625pt}%
\definecolor{currentstroke}{rgb}{0.000000,0.000000,0.000000}%
\pgfsetstrokecolor{currentstroke}%
\pgfsetdash{}{0pt}%
\pgfpathmoveto{\pgfqpoint{10.306389in}{16.800332in}}%
\pgfpathlineto{\pgfqpoint{10.306389in}{16.807931in}}%
\pgfusepath{stroke}%
\end{pgfscope}%
\begin{pgfscope}%
\pgfpathrectangle{\pgfqpoint{9.810417in}{16.722093in}}{\pgfqpoint{5.489583in}{0.877907in}}%
\pgfusepath{clip}%
\pgfsetbuttcap%
\pgfsetroundjoin%
\pgfsetlinewidth{1.505625pt}%
\definecolor{currentstroke}{rgb}{0.000000,0.000000,0.000000}%
\pgfsetstrokecolor{currentstroke}%
\pgfsetdash{}{0pt}%
\pgfpathmoveto{\pgfqpoint{10.429612in}{16.800332in}}%
\pgfpathlineto{\pgfqpoint{10.429612in}{16.788182in}}%
\pgfusepath{stroke}%
\end{pgfscope}%
\begin{pgfscope}%
\pgfpathrectangle{\pgfqpoint{9.810417in}{16.722093in}}{\pgfqpoint{5.489583in}{0.877907in}}%
\pgfusepath{clip}%
\pgfsetbuttcap%
\pgfsetroundjoin%
\pgfsetlinewidth{1.505625pt}%
\definecolor{currentstroke}{rgb}{0.000000,0.000000,0.000000}%
\pgfsetstrokecolor{currentstroke}%
\pgfsetdash{}{0pt}%
\pgfpathmoveto{\pgfqpoint{10.552835in}{16.800332in}}%
\pgfpathlineto{\pgfqpoint{10.552835in}{16.792765in}}%
\pgfusepath{stroke}%
\end{pgfscope}%
\begin{pgfscope}%
\pgfpathrectangle{\pgfqpoint{9.810417in}{16.722093in}}{\pgfqpoint{5.489583in}{0.877907in}}%
\pgfusepath{clip}%
\pgfsetbuttcap%
\pgfsetroundjoin%
\pgfsetlinewidth{1.505625pt}%
\definecolor{currentstroke}{rgb}{0.000000,0.000000,0.000000}%
\pgfsetstrokecolor{currentstroke}%
\pgfsetdash{}{0pt}%
\pgfpathmoveto{\pgfqpoint{10.676058in}{16.800332in}}%
\pgfpathlineto{\pgfqpoint{10.676058in}{16.814088in}}%
\pgfusepath{stroke}%
\end{pgfscope}%
\begin{pgfscope}%
\pgfpathrectangle{\pgfqpoint{9.810417in}{16.722093in}}{\pgfqpoint{5.489583in}{0.877907in}}%
\pgfusepath{clip}%
\pgfsetbuttcap%
\pgfsetroundjoin%
\pgfsetlinewidth{1.505625pt}%
\definecolor{currentstroke}{rgb}{0.000000,0.000000,0.000000}%
\pgfsetstrokecolor{currentstroke}%
\pgfsetdash{}{0pt}%
\pgfpathmoveto{\pgfqpoint{10.799281in}{16.800332in}}%
\pgfpathlineto{\pgfqpoint{10.799281in}{16.804443in}}%
\pgfusepath{stroke}%
\end{pgfscope}%
\begin{pgfscope}%
\pgfpathrectangle{\pgfqpoint{9.810417in}{16.722093in}}{\pgfqpoint{5.489583in}{0.877907in}}%
\pgfusepath{clip}%
\pgfsetbuttcap%
\pgfsetroundjoin%
\pgfsetlinewidth{1.505625pt}%
\definecolor{currentstroke}{rgb}{0.000000,0.000000,0.000000}%
\pgfsetstrokecolor{currentstroke}%
\pgfsetdash{}{0pt}%
\pgfpathmoveto{\pgfqpoint{10.922504in}{16.800332in}}%
\pgfpathlineto{\pgfqpoint{10.922504in}{16.785665in}}%
\pgfusepath{stroke}%
\end{pgfscope}%
\begin{pgfscope}%
\pgfpathrectangle{\pgfqpoint{9.810417in}{16.722093in}}{\pgfqpoint{5.489583in}{0.877907in}}%
\pgfusepath{clip}%
\pgfsetbuttcap%
\pgfsetroundjoin%
\pgfsetlinewidth{1.505625pt}%
\definecolor{currentstroke}{rgb}{0.000000,0.000000,0.000000}%
\pgfsetstrokecolor{currentstroke}%
\pgfsetdash{}{0pt}%
\pgfpathmoveto{\pgfqpoint{11.045727in}{16.800332in}}%
\pgfpathlineto{\pgfqpoint{11.045727in}{16.796228in}}%
\pgfusepath{stroke}%
\end{pgfscope}%
\begin{pgfscope}%
\pgfpathrectangle{\pgfqpoint{9.810417in}{16.722093in}}{\pgfqpoint{5.489583in}{0.877907in}}%
\pgfusepath{clip}%
\pgfsetbuttcap%
\pgfsetroundjoin%
\pgfsetlinewidth{1.505625pt}%
\definecolor{currentstroke}{rgb}{0.000000,0.000000,0.000000}%
\pgfsetstrokecolor{currentstroke}%
\pgfsetdash{}{0pt}%
\pgfpathmoveto{\pgfqpoint{11.168950in}{16.800332in}}%
\pgfpathlineto{\pgfqpoint{11.168950in}{16.785658in}}%
\pgfusepath{stroke}%
\end{pgfscope}%
\begin{pgfscope}%
\pgfpathrectangle{\pgfqpoint{9.810417in}{16.722093in}}{\pgfqpoint{5.489583in}{0.877907in}}%
\pgfusepath{clip}%
\pgfsetbuttcap%
\pgfsetroundjoin%
\pgfsetlinewidth{1.505625pt}%
\definecolor{currentstroke}{rgb}{0.000000,0.000000,0.000000}%
\pgfsetstrokecolor{currentstroke}%
\pgfsetdash{}{0pt}%
\pgfpathmoveto{\pgfqpoint{11.292173in}{16.800332in}}%
\pgfpathlineto{\pgfqpoint{11.292173in}{16.810830in}}%
\pgfusepath{stroke}%
\end{pgfscope}%
\begin{pgfscope}%
\pgfpathrectangle{\pgfqpoint{9.810417in}{16.722093in}}{\pgfqpoint{5.489583in}{0.877907in}}%
\pgfusepath{clip}%
\pgfsetbuttcap%
\pgfsetroundjoin%
\pgfsetlinewidth{1.505625pt}%
\definecolor{currentstroke}{rgb}{0.000000,0.000000,0.000000}%
\pgfsetstrokecolor{currentstroke}%
\pgfsetdash{}{0pt}%
\pgfpathmoveto{\pgfqpoint{11.415396in}{16.800332in}}%
\pgfpathlineto{\pgfqpoint{11.415396in}{16.810565in}}%
\pgfusepath{stroke}%
\end{pgfscope}%
\begin{pgfscope}%
\pgfpathrectangle{\pgfqpoint{9.810417in}{16.722093in}}{\pgfqpoint{5.489583in}{0.877907in}}%
\pgfusepath{clip}%
\pgfsetbuttcap%
\pgfsetroundjoin%
\pgfsetlinewidth{1.505625pt}%
\definecolor{currentstroke}{rgb}{0.000000,0.000000,0.000000}%
\pgfsetstrokecolor{currentstroke}%
\pgfsetdash{}{0pt}%
\pgfpathmoveto{\pgfqpoint{11.538619in}{16.800332in}}%
\pgfpathlineto{\pgfqpoint{11.538619in}{16.802471in}}%
\pgfusepath{stroke}%
\end{pgfscope}%
\begin{pgfscope}%
\pgfpathrectangle{\pgfqpoint{9.810417in}{16.722093in}}{\pgfqpoint{5.489583in}{0.877907in}}%
\pgfusepath{clip}%
\pgfsetbuttcap%
\pgfsetroundjoin%
\pgfsetlinewidth{1.505625pt}%
\definecolor{currentstroke}{rgb}{0.000000,0.000000,0.000000}%
\pgfsetstrokecolor{currentstroke}%
\pgfsetdash{}{0pt}%
\pgfpathmoveto{\pgfqpoint{11.661842in}{16.800332in}}%
\pgfpathlineto{\pgfqpoint{11.661842in}{16.806130in}}%
\pgfusepath{stroke}%
\end{pgfscope}%
\begin{pgfscope}%
\pgfpathrectangle{\pgfqpoint{9.810417in}{16.722093in}}{\pgfqpoint{5.489583in}{0.877907in}}%
\pgfusepath{clip}%
\pgfsetbuttcap%
\pgfsetroundjoin%
\pgfsetlinewidth{1.505625pt}%
\definecolor{currentstroke}{rgb}{0.000000,0.000000,0.000000}%
\pgfsetstrokecolor{currentstroke}%
\pgfsetdash{}{0pt}%
\pgfpathmoveto{\pgfqpoint{11.785065in}{16.800332in}}%
\pgfpathlineto{\pgfqpoint{11.785065in}{16.789038in}}%
\pgfusepath{stroke}%
\end{pgfscope}%
\begin{pgfscope}%
\pgfpathrectangle{\pgfqpoint{9.810417in}{16.722093in}}{\pgfqpoint{5.489583in}{0.877907in}}%
\pgfusepath{clip}%
\pgfsetbuttcap%
\pgfsetroundjoin%
\pgfsetlinewidth{1.505625pt}%
\definecolor{currentstroke}{rgb}{0.000000,0.000000,0.000000}%
\pgfsetstrokecolor{currentstroke}%
\pgfsetdash{}{0pt}%
\pgfpathmoveto{\pgfqpoint{11.908288in}{16.800332in}}%
\pgfpathlineto{\pgfqpoint{11.908288in}{16.803164in}}%
\pgfusepath{stroke}%
\end{pgfscope}%
\begin{pgfscope}%
\pgfpathrectangle{\pgfqpoint{9.810417in}{16.722093in}}{\pgfqpoint{5.489583in}{0.877907in}}%
\pgfusepath{clip}%
\pgfsetbuttcap%
\pgfsetroundjoin%
\pgfsetlinewidth{1.505625pt}%
\definecolor{currentstroke}{rgb}{0.000000,0.000000,0.000000}%
\pgfsetstrokecolor{currentstroke}%
\pgfsetdash{}{0pt}%
\pgfpathmoveto{\pgfqpoint{12.031511in}{16.800332in}}%
\pgfpathlineto{\pgfqpoint{12.031511in}{16.799989in}}%
\pgfusepath{stroke}%
\end{pgfscope}%
\begin{pgfscope}%
\pgfpathrectangle{\pgfqpoint{9.810417in}{16.722093in}}{\pgfqpoint{5.489583in}{0.877907in}}%
\pgfusepath{clip}%
\pgfsetbuttcap%
\pgfsetroundjoin%
\pgfsetlinewidth{1.505625pt}%
\definecolor{currentstroke}{rgb}{0.000000,0.000000,0.000000}%
\pgfsetstrokecolor{currentstroke}%
\pgfsetdash{}{0pt}%
\pgfpathmoveto{\pgfqpoint{12.154734in}{16.800332in}}%
\pgfpathlineto{\pgfqpoint{12.154734in}{16.813690in}}%
\pgfusepath{stroke}%
\end{pgfscope}%
\begin{pgfscope}%
\pgfpathrectangle{\pgfqpoint{9.810417in}{16.722093in}}{\pgfqpoint{5.489583in}{0.877907in}}%
\pgfusepath{clip}%
\pgfsetbuttcap%
\pgfsetroundjoin%
\pgfsetlinewidth{1.505625pt}%
\definecolor{currentstroke}{rgb}{0.000000,0.000000,0.000000}%
\pgfsetstrokecolor{currentstroke}%
\pgfsetdash{}{0pt}%
\pgfpathmoveto{\pgfqpoint{12.277957in}{16.800332in}}%
\pgfpathlineto{\pgfqpoint{12.277957in}{16.793479in}}%
\pgfusepath{stroke}%
\end{pgfscope}%
\begin{pgfscope}%
\pgfpathrectangle{\pgfqpoint{9.810417in}{16.722093in}}{\pgfqpoint{5.489583in}{0.877907in}}%
\pgfusepath{clip}%
\pgfsetbuttcap%
\pgfsetroundjoin%
\pgfsetlinewidth{1.505625pt}%
\definecolor{currentstroke}{rgb}{0.000000,0.000000,0.000000}%
\pgfsetstrokecolor{currentstroke}%
\pgfsetdash{}{0pt}%
\pgfpathmoveto{\pgfqpoint{12.401180in}{16.800332in}}%
\pgfpathlineto{\pgfqpoint{12.401180in}{16.796786in}}%
\pgfusepath{stroke}%
\end{pgfscope}%
\begin{pgfscope}%
\pgfpathrectangle{\pgfqpoint{9.810417in}{16.722093in}}{\pgfqpoint{5.489583in}{0.877907in}}%
\pgfusepath{clip}%
\pgfsetbuttcap%
\pgfsetroundjoin%
\pgfsetlinewidth{1.505625pt}%
\definecolor{currentstroke}{rgb}{0.000000,0.000000,0.000000}%
\pgfsetstrokecolor{currentstroke}%
\pgfsetdash{}{0pt}%
\pgfpathmoveto{\pgfqpoint{12.524403in}{16.800332in}}%
\pgfpathlineto{\pgfqpoint{12.524403in}{16.787424in}}%
\pgfusepath{stroke}%
\end{pgfscope}%
\begin{pgfscope}%
\pgfpathrectangle{\pgfqpoint{9.810417in}{16.722093in}}{\pgfqpoint{5.489583in}{0.877907in}}%
\pgfusepath{clip}%
\pgfsetbuttcap%
\pgfsetroundjoin%
\pgfsetlinewidth{1.505625pt}%
\definecolor{currentstroke}{rgb}{0.000000,0.000000,0.000000}%
\pgfsetstrokecolor{currentstroke}%
\pgfsetdash{}{0pt}%
\pgfpathmoveto{\pgfqpoint{12.647626in}{16.800332in}}%
\pgfpathlineto{\pgfqpoint{12.647626in}{16.795431in}}%
\pgfusepath{stroke}%
\end{pgfscope}%
\begin{pgfscope}%
\pgfpathrectangle{\pgfqpoint{9.810417in}{16.722093in}}{\pgfqpoint{5.489583in}{0.877907in}}%
\pgfusepath{clip}%
\pgfsetbuttcap%
\pgfsetroundjoin%
\pgfsetlinewidth{1.505625pt}%
\definecolor{currentstroke}{rgb}{0.000000,0.000000,0.000000}%
\pgfsetstrokecolor{currentstroke}%
\pgfsetdash{}{0pt}%
\pgfpathmoveto{\pgfqpoint{12.770849in}{16.800332in}}%
\pgfpathlineto{\pgfqpoint{12.770849in}{16.806691in}}%
\pgfusepath{stroke}%
\end{pgfscope}%
\begin{pgfscope}%
\pgfpathrectangle{\pgfqpoint{9.810417in}{16.722093in}}{\pgfqpoint{5.489583in}{0.877907in}}%
\pgfusepath{clip}%
\pgfsetbuttcap%
\pgfsetroundjoin%
\pgfsetlinewidth{1.505625pt}%
\definecolor{currentstroke}{rgb}{0.000000,0.000000,0.000000}%
\pgfsetstrokecolor{currentstroke}%
\pgfsetdash{}{0pt}%
\pgfpathmoveto{\pgfqpoint{12.894072in}{16.800332in}}%
\pgfpathlineto{\pgfqpoint{12.894072in}{16.813541in}}%
\pgfusepath{stroke}%
\end{pgfscope}%
\begin{pgfscope}%
\pgfpathrectangle{\pgfqpoint{9.810417in}{16.722093in}}{\pgfqpoint{5.489583in}{0.877907in}}%
\pgfusepath{clip}%
\pgfsetbuttcap%
\pgfsetroundjoin%
\pgfsetlinewidth{1.505625pt}%
\definecolor{currentstroke}{rgb}{0.000000,0.000000,0.000000}%
\pgfsetstrokecolor{currentstroke}%
\pgfsetdash{}{0pt}%
\pgfpathmoveto{\pgfqpoint{13.017294in}{16.800332in}}%
\pgfpathlineto{\pgfqpoint{13.017294in}{16.803383in}}%
\pgfusepath{stroke}%
\end{pgfscope}%
\begin{pgfscope}%
\pgfpathrectangle{\pgfqpoint{9.810417in}{16.722093in}}{\pgfqpoint{5.489583in}{0.877907in}}%
\pgfusepath{clip}%
\pgfsetbuttcap%
\pgfsetroundjoin%
\pgfsetlinewidth{1.505625pt}%
\definecolor{currentstroke}{rgb}{0.000000,0.000000,0.000000}%
\pgfsetstrokecolor{currentstroke}%
\pgfsetdash{}{0pt}%
\pgfpathmoveto{\pgfqpoint{13.140517in}{16.800332in}}%
\pgfpathlineto{\pgfqpoint{13.140517in}{16.807568in}}%
\pgfusepath{stroke}%
\end{pgfscope}%
\begin{pgfscope}%
\pgfpathrectangle{\pgfqpoint{9.810417in}{16.722093in}}{\pgfqpoint{5.489583in}{0.877907in}}%
\pgfusepath{clip}%
\pgfsetbuttcap%
\pgfsetroundjoin%
\pgfsetlinewidth{1.505625pt}%
\definecolor{currentstroke}{rgb}{0.000000,0.000000,0.000000}%
\pgfsetstrokecolor{currentstroke}%
\pgfsetdash{}{0pt}%
\pgfpathmoveto{\pgfqpoint{13.263740in}{16.800332in}}%
\pgfpathlineto{\pgfqpoint{13.263740in}{16.806899in}}%
\pgfusepath{stroke}%
\end{pgfscope}%
\begin{pgfscope}%
\pgfpathrectangle{\pgfqpoint{9.810417in}{16.722093in}}{\pgfqpoint{5.489583in}{0.877907in}}%
\pgfusepath{clip}%
\pgfsetbuttcap%
\pgfsetroundjoin%
\pgfsetlinewidth{1.505625pt}%
\definecolor{currentstroke}{rgb}{0.000000,0.000000,0.000000}%
\pgfsetstrokecolor{currentstroke}%
\pgfsetdash{}{0pt}%
\pgfpathmoveto{\pgfqpoint{13.386963in}{16.800332in}}%
\pgfpathlineto{\pgfqpoint{13.386963in}{16.790691in}}%
\pgfusepath{stroke}%
\end{pgfscope}%
\begin{pgfscope}%
\pgfpathrectangle{\pgfqpoint{9.810417in}{16.722093in}}{\pgfqpoint{5.489583in}{0.877907in}}%
\pgfusepath{clip}%
\pgfsetbuttcap%
\pgfsetroundjoin%
\pgfsetlinewidth{1.505625pt}%
\definecolor{currentstroke}{rgb}{0.000000,0.000000,0.000000}%
\pgfsetstrokecolor{currentstroke}%
\pgfsetdash{}{0pt}%
\pgfpathmoveto{\pgfqpoint{13.510186in}{16.800332in}}%
\pgfpathlineto{\pgfqpoint{13.510186in}{16.796733in}}%
\pgfusepath{stroke}%
\end{pgfscope}%
\begin{pgfscope}%
\pgfpathrectangle{\pgfqpoint{9.810417in}{16.722093in}}{\pgfqpoint{5.489583in}{0.877907in}}%
\pgfusepath{clip}%
\pgfsetbuttcap%
\pgfsetroundjoin%
\pgfsetlinewidth{1.505625pt}%
\definecolor{currentstroke}{rgb}{0.000000,0.000000,0.000000}%
\pgfsetstrokecolor{currentstroke}%
\pgfsetdash{}{0pt}%
\pgfpathmoveto{\pgfqpoint{13.633409in}{16.800332in}}%
\pgfpathlineto{\pgfqpoint{13.633409in}{16.813213in}}%
\pgfusepath{stroke}%
\end{pgfscope}%
\begin{pgfscope}%
\pgfpathrectangle{\pgfqpoint{9.810417in}{16.722093in}}{\pgfqpoint{5.489583in}{0.877907in}}%
\pgfusepath{clip}%
\pgfsetbuttcap%
\pgfsetroundjoin%
\pgfsetlinewidth{1.505625pt}%
\definecolor{currentstroke}{rgb}{0.000000,0.000000,0.000000}%
\pgfsetstrokecolor{currentstroke}%
\pgfsetdash{}{0pt}%
\pgfpathmoveto{\pgfqpoint{13.756632in}{16.800332in}}%
\pgfpathlineto{\pgfqpoint{13.756632in}{16.795599in}}%
\pgfusepath{stroke}%
\end{pgfscope}%
\begin{pgfscope}%
\pgfpathrectangle{\pgfqpoint{9.810417in}{16.722093in}}{\pgfqpoint{5.489583in}{0.877907in}}%
\pgfusepath{clip}%
\pgfsetbuttcap%
\pgfsetroundjoin%
\pgfsetlinewidth{1.505625pt}%
\definecolor{currentstroke}{rgb}{0.000000,0.000000,0.000000}%
\pgfsetstrokecolor{currentstroke}%
\pgfsetdash{}{0pt}%
\pgfpathmoveto{\pgfqpoint{13.879855in}{16.800332in}}%
\pgfpathlineto{\pgfqpoint{13.879855in}{16.809830in}}%
\pgfusepath{stroke}%
\end{pgfscope}%
\begin{pgfscope}%
\pgfpathrectangle{\pgfqpoint{9.810417in}{16.722093in}}{\pgfqpoint{5.489583in}{0.877907in}}%
\pgfusepath{clip}%
\pgfsetbuttcap%
\pgfsetroundjoin%
\pgfsetlinewidth{1.505625pt}%
\definecolor{currentstroke}{rgb}{0.000000,0.000000,0.000000}%
\pgfsetstrokecolor{currentstroke}%
\pgfsetdash{}{0pt}%
\pgfpathmoveto{\pgfqpoint{14.003078in}{16.800332in}}%
\pgfpathlineto{\pgfqpoint{14.003078in}{16.795792in}}%
\pgfusepath{stroke}%
\end{pgfscope}%
\begin{pgfscope}%
\pgfpathrectangle{\pgfqpoint{9.810417in}{16.722093in}}{\pgfqpoint{5.489583in}{0.877907in}}%
\pgfusepath{clip}%
\pgfsetbuttcap%
\pgfsetroundjoin%
\pgfsetlinewidth{1.505625pt}%
\definecolor{currentstroke}{rgb}{0.000000,0.000000,0.000000}%
\pgfsetstrokecolor{currentstroke}%
\pgfsetdash{}{0pt}%
\pgfpathmoveto{\pgfqpoint{14.126301in}{16.800332in}}%
\pgfpathlineto{\pgfqpoint{14.126301in}{16.803019in}}%
\pgfusepath{stroke}%
\end{pgfscope}%
\begin{pgfscope}%
\pgfpathrectangle{\pgfqpoint{9.810417in}{16.722093in}}{\pgfqpoint{5.489583in}{0.877907in}}%
\pgfusepath{clip}%
\pgfsetbuttcap%
\pgfsetroundjoin%
\pgfsetlinewidth{1.505625pt}%
\definecolor{currentstroke}{rgb}{0.000000,0.000000,0.000000}%
\pgfsetstrokecolor{currentstroke}%
\pgfsetdash{}{0pt}%
\pgfpathmoveto{\pgfqpoint{14.249524in}{16.800332in}}%
\pgfpathlineto{\pgfqpoint{14.249524in}{16.805453in}}%
\pgfusepath{stroke}%
\end{pgfscope}%
\begin{pgfscope}%
\pgfpathrectangle{\pgfqpoint{9.810417in}{16.722093in}}{\pgfqpoint{5.489583in}{0.877907in}}%
\pgfusepath{clip}%
\pgfsetbuttcap%
\pgfsetroundjoin%
\pgfsetlinewidth{1.505625pt}%
\definecolor{currentstroke}{rgb}{0.000000,0.000000,0.000000}%
\pgfsetstrokecolor{currentstroke}%
\pgfsetdash{}{0pt}%
\pgfpathmoveto{\pgfqpoint{14.372747in}{16.800332in}}%
\pgfpathlineto{\pgfqpoint{14.372747in}{16.786284in}}%
\pgfusepath{stroke}%
\end{pgfscope}%
\begin{pgfscope}%
\pgfpathrectangle{\pgfqpoint{9.810417in}{16.722093in}}{\pgfqpoint{5.489583in}{0.877907in}}%
\pgfusepath{clip}%
\pgfsetbuttcap%
\pgfsetroundjoin%
\pgfsetlinewidth{1.505625pt}%
\definecolor{currentstroke}{rgb}{0.000000,0.000000,0.000000}%
\pgfsetstrokecolor{currentstroke}%
\pgfsetdash{}{0pt}%
\pgfpathmoveto{\pgfqpoint{14.495970in}{16.800332in}}%
\pgfpathlineto{\pgfqpoint{14.495970in}{16.793159in}}%
\pgfusepath{stroke}%
\end{pgfscope}%
\begin{pgfscope}%
\pgfpathrectangle{\pgfqpoint{9.810417in}{16.722093in}}{\pgfqpoint{5.489583in}{0.877907in}}%
\pgfusepath{clip}%
\pgfsetbuttcap%
\pgfsetroundjoin%
\pgfsetlinewidth{1.505625pt}%
\definecolor{currentstroke}{rgb}{0.000000,0.000000,0.000000}%
\pgfsetstrokecolor{currentstroke}%
\pgfsetdash{}{0pt}%
\pgfpathmoveto{\pgfqpoint{14.619193in}{16.800332in}}%
\pgfpathlineto{\pgfqpoint{14.619193in}{16.798215in}}%
\pgfusepath{stroke}%
\end{pgfscope}%
\begin{pgfscope}%
\pgfpathrectangle{\pgfqpoint{9.810417in}{16.722093in}}{\pgfqpoint{5.489583in}{0.877907in}}%
\pgfusepath{clip}%
\pgfsetbuttcap%
\pgfsetroundjoin%
\pgfsetlinewidth{1.505625pt}%
\definecolor{currentstroke}{rgb}{0.000000,0.000000,0.000000}%
\pgfsetstrokecolor{currentstroke}%
\pgfsetdash{}{0pt}%
\pgfpathmoveto{\pgfqpoint{14.742416in}{16.800332in}}%
\pgfpathlineto{\pgfqpoint{14.742416in}{16.804587in}}%
\pgfusepath{stroke}%
\end{pgfscope}%
\begin{pgfscope}%
\pgfpathrectangle{\pgfqpoint{9.810417in}{16.722093in}}{\pgfqpoint{5.489583in}{0.877907in}}%
\pgfusepath{clip}%
\pgfsetbuttcap%
\pgfsetroundjoin%
\pgfsetlinewidth{1.505625pt}%
\definecolor{currentstroke}{rgb}{0.000000,0.000000,0.000000}%
\pgfsetstrokecolor{currentstroke}%
\pgfsetdash{}{0pt}%
\pgfpathmoveto{\pgfqpoint{14.865639in}{16.800332in}}%
\pgfpathlineto{\pgfqpoint{14.865639in}{16.791916in}}%
\pgfusepath{stroke}%
\end{pgfscope}%
\begin{pgfscope}%
\pgfpathrectangle{\pgfqpoint{9.810417in}{16.722093in}}{\pgfqpoint{5.489583in}{0.877907in}}%
\pgfusepath{clip}%
\pgfsetbuttcap%
\pgfsetroundjoin%
\pgfsetlinewidth{1.505625pt}%
\definecolor{currentstroke}{rgb}{0.000000,0.000000,0.000000}%
\pgfsetstrokecolor{currentstroke}%
\pgfsetdash{}{0pt}%
\pgfpathmoveto{\pgfqpoint{14.988862in}{16.800332in}}%
\pgfpathlineto{\pgfqpoint{14.988862in}{16.794590in}}%
\pgfusepath{stroke}%
\end{pgfscope}%
\begin{pgfscope}%
\pgfpathrectangle{\pgfqpoint{9.810417in}{16.722093in}}{\pgfqpoint{5.489583in}{0.877907in}}%
\pgfusepath{clip}%
\pgfsetroundcap%
\pgfsetroundjoin%
\pgfsetlinewidth{1.505625pt}%
\definecolor{currentstroke}{rgb}{0.121569,0.466667,0.705882}%
\pgfsetstrokecolor{currentstroke}%
\pgfsetdash{}{0pt}%
\pgfpathmoveto{\pgfqpoint{9.810417in}{16.800332in}}%
\pgfpathlineto{\pgfqpoint{15.300000in}{16.800332in}}%
\pgfusepath{stroke}%
\end{pgfscope}%
\begin{pgfscope}%
\pgfpathrectangle{\pgfqpoint{9.810417in}{16.722093in}}{\pgfqpoint{5.489583in}{0.877907in}}%
\pgfusepath{clip}%
\pgfsetbuttcap%
\pgfsetroundjoin%
\definecolor{currentfill}{rgb}{0.121569,0.466667,0.705882}%
\pgfsetfillcolor{currentfill}%
\pgfsetlinewidth{1.003750pt}%
\definecolor{currentstroke}{rgb}{0.121569,0.466667,0.705882}%
\pgfsetstrokecolor{currentstroke}%
\pgfsetdash{}{0pt}%
\pgfsys@defobject{currentmarker}{\pgfqpoint{-0.034722in}{-0.034722in}}{\pgfqpoint{0.034722in}{0.034722in}}{%
\pgfpathmoveto{\pgfqpoint{0.000000in}{-0.034722in}}%
\pgfpathcurveto{\pgfqpoint{0.009208in}{-0.034722in}}{\pgfqpoint{0.018041in}{-0.031064in}}{\pgfqpoint{0.024552in}{-0.024552in}}%
\pgfpathcurveto{\pgfqpoint{0.031064in}{-0.018041in}}{\pgfqpoint{0.034722in}{-0.009208in}}{\pgfqpoint{0.034722in}{0.000000in}}%
\pgfpathcurveto{\pgfqpoint{0.034722in}{0.009208in}}{\pgfqpoint{0.031064in}{0.018041in}}{\pgfqpoint{0.024552in}{0.024552in}}%
\pgfpathcurveto{\pgfqpoint{0.018041in}{0.031064in}}{\pgfqpoint{0.009208in}{0.034722in}}{\pgfqpoint{0.000000in}{0.034722in}}%
\pgfpathcurveto{\pgfqpoint{-0.009208in}{0.034722in}}{\pgfqpoint{-0.018041in}{0.031064in}}{\pgfqpoint{-0.024552in}{0.024552in}}%
\pgfpathcurveto{\pgfqpoint{-0.031064in}{0.018041in}}{\pgfqpoint{-0.034722in}{0.009208in}}{\pgfqpoint{-0.034722in}{0.000000in}}%
\pgfpathcurveto{\pgfqpoint{-0.034722in}{-0.009208in}}{\pgfqpoint{-0.031064in}{-0.018041in}}{\pgfqpoint{-0.024552in}{-0.024552in}}%
\pgfpathcurveto{\pgfqpoint{-0.018041in}{-0.031064in}}{\pgfqpoint{-0.009208in}{-0.034722in}}{\pgfqpoint{0.000000in}{-0.034722in}}%
\pgfpathclose%
\pgfusepath{stroke,fill}%
}%
\begin{pgfscope}%
\pgfsys@transformshift{10.059943in}{17.560095in}%
\pgfsys@useobject{currentmarker}{}%
\end{pgfscope}%
\begin{pgfscope}%
\pgfsys@transformshift{10.183166in}{17.558539in}%
\pgfsys@useobject{currentmarker}{}%
\end{pgfscope}%
\begin{pgfscope}%
\pgfsys@transformshift{10.306389in}{16.807931in}%
\pgfsys@useobject{currentmarker}{}%
\end{pgfscope}%
\begin{pgfscope}%
\pgfsys@transformshift{10.429612in}{16.788182in}%
\pgfsys@useobject{currentmarker}{}%
\end{pgfscope}%
\begin{pgfscope}%
\pgfsys@transformshift{10.552835in}{16.792765in}%
\pgfsys@useobject{currentmarker}{}%
\end{pgfscope}%
\begin{pgfscope}%
\pgfsys@transformshift{10.676058in}{16.814088in}%
\pgfsys@useobject{currentmarker}{}%
\end{pgfscope}%
\begin{pgfscope}%
\pgfsys@transformshift{10.799281in}{16.804443in}%
\pgfsys@useobject{currentmarker}{}%
\end{pgfscope}%
\begin{pgfscope}%
\pgfsys@transformshift{10.922504in}{16.785665in}%
\pgfsys@useobject{currentmarker}{}%
\end{pgfscope}%
\begin{pgfscope}%
\pgfsys@transformshift{11.045727in}{16.796228in}%
\pgfsys@useobject{currentmarker}{}%
\end{pgfscope}%
\begin{pgfscope}%
\pgfsys@transformshift{11.168950in}{16.785658in}%
\pgfsys@useobject{currentmarker}{}%
\end{pgfscope}%
\begin{pgfscope}%
\pgfsys@transformshift{11.292173in}{16.810830in}%
\pgfsys@useobject{currentmarker}{}%
\end{pgfscope}%
\begin{pgfscope}%
\pgfsys@transformshift{11.415396in}{16.810565in}%
\pgfsys@useobject{currentmarker}{}%
\end{pgfscope}%
\begin{pgfscope}%
\pgfsys@transformshift{11.538619in}{16.802471in}%
\pgfsys@useobject{currentmarker}{}%
\end{pgfscope}%
\begin{pgfscope}%
\pgfsys@transformshift{11.661842in}{16.806130in}%
\pgfsys@useobject{currentmarker}{}%
\end{pgfscope}%
\begin{pgfscope}%
\pgfsys@transformshift{11.785065in}{16.789038in}%
\pgfsys@useobject{currentmarker}{}%
\end{pgfscope}%
\begin{pgfscope}%
\pgfsys@transformshift{11.908288in}{16.803164in}%
\pgfsys@useobject{currentmarker}{}%
\end{pgfscope}%
\begin{pgfscope}%
\pgfsys@transformshift{12.031511in}{16.799989in}%
\pgfsys@useobject{currentmarker}{}%
\end{pgfscope}%
\begin{pgfscope}%
\pgfsys@transformshift{12.154734in}{16.813690in}%
\pgfsys@useobject{currentmarker}{}%
\end{pgfscope}%
\begin{pgfscope}%
\pgfsys@transformshift{12.277957in}{16.793479in}%
\pgfsys@useobject{currentmarker}{}%
\end{pgfscope}%
\begin{pgfscope}%
\pgfsys@transformshift{12.401180in}{16.796786in}%
\pgfsys@useobject{currentmarker}{}%
\end{pgfscope}%
\begin{pgfscope}%
\pgfsys@transformshift{12.524403in}{16.787424in}%
\pgfsys@useobject{currentmarker}{}%
\end{pgfscope}%
\begin{pgfscope}%
\pgfsys@transformshift{12.647626in}{16.795431in}%
\pgfsys@useobject{currentmarker}{}%
\end{pgfscope}%
\begin{pgfscope}%
\pgfsys@transformshift{12.770849in}{16.806691in}%
\pgfsys@useobject{currentmarker}{}%
\end{pgfscope}%
\begin{pgfscope}%
\pgfsys@transformshift{12.894072in}{16.813541in}%
\pgfsys@useobject{currentmarker}{}%
\end{pgfscope}%
\begin{pgfscope}%
\pgfsys@transformshift{13.017294in}{16.803383in}%
\pgfsys@useobject{currentmarker}{}%
\end{pgfscope}%
\begin{pgfscope}%
\pgfsys@transformshift{13.140517in}{16.807568in}%
\pgfsys@useobject{currentmarker}{}%
\end{pgfscope}%
\begin{pgfscope}%
\pgfsys@transformshift{13.263740in}{16.806899in}%
\pgfsys@useobject{currentmarker}{}%
\end{pgfscope}%
\begin{pgfscope}%
\pgfsys@transformshift{13.386963in}{16.790691in}%
\pgfsys@useobject{currentmarker}{}%
\end{pgfscope}%
\begin{pgfscope}%
\pgfsys@transformshift{13.510186in}{16.796733in}%
\pgfsys@useobject{currentmarker}{}%
\end{pgfscope}%
\begin{pgfscope}%
\pgfsys@transformshift{13.633409in}{16.813213in}%
\pgfsys@useobject{currentmarker}{}%
\end{pgfscope}%
\begin{pgfscope}%
\pgfsys@transformshift{13.756632in}{16.795599in}%
\pgfsys@useobject{currentmarker}{}%
\end{pgfscope}%
\begin{pgfscope}%
\pgfsys@transformshift{13.879855in}{16.809830in}%
\pgfsys@useobject{currentmarker}{}%
\end{pgfscope}%
\begin{pgfscope}%
\pgfsys@transformshift{14.003078in}{16.795792in}%
\pgfsys@useobject{currentmarker}{}%
\end{pgfscope}%
\begin{pgfscope}%
\pgfsys@transformshift{14.126301in}{16.803019in}%
\pgfsys@useobject{currentmarker}{}%
\end{pgfscope}%
\begin{pgfscope}%
\pgfsys@transformshift{14.249524in}{16.805453in}%
\pgfsys@useobject{currentmarker}{}%
\end{pgfscope}%
\begin{pgfscope}%
\pgfsys@transformshift{14.372747in}{16.786284in}%
\pgfsys@useobject{currentmarker}{}%
\end{pgfscope}%
\begin{pgfscope}%
\pgfsys@transformshift{14.495970in}{16.793159in}%
\pgfsys@useobject{currentmarker}{}%
\end{pgfscope}%
\begin{pgfscope}%
\pgfsys@transformshift{14.619193in}{16.798215in}%
\pgfsys@useobject{currentmarker}{}%
\end{pgfscope}%
\begin{pgfscope}%
\pgfsys@transformshift{14.742416in}{16.804587in}%
\pgfsys@useobject{currentmarker}{}%
\end{pgfscope}%
\begin{pgfscope}%
\pgfsys@transformshift{14.865639in}{16.791916in}%
\pgfsys@useobject{currentmarker}{}%
\end{pgfscope}%
\begin{pgfscope}%
\pgfsys@transformshift{14.988862in}{16.794590in}%
\pgfsys@useobject{currentmarker}{}%
\end{pgfscope}%
\end{pgfscope}%
\begin{pgfscope}%
\pgfsetrectcap%
\pgfsetmiterjoin%
\pgfsetlinewidth{0.803000pt}%
\definecolor{currentstroke}{rgb}{1.000000,1.000000,1.000000}%
\pgfsetstrokecolor{currentstroke}%
\pgfsetdash{}{0pt}%
\pgfpathmoveto{\pgfqpoint{9.810417in}{16.722093in}}%
\pgfpathlineto{\pgfqpoint{9.810417in}{17.600000in}}%
\pgfusepath{stroke}%
\end{pgfscope}%
\begin{pgfscope}%
\pgfsetrectcap%
\pgfsetmiterjoin%
\pgfsetlinewidth{0.803000pt}%
\definecolor{currentstroke}{rgb}{1.000000,1.000000,1.000000}%
\pgfsetstrokecolor{currentstroke}%
\pgfsetdash{}{0pt}%
\pgfpathmoveto{\pgfqpoint{15.300000in}{16.722093in}}%
\pgfpathlineto{\pgfqpoint{15.300000in}{17.600000in}}%
\pgfusepath{stroke}%
\end{pgfscope}%
\begin{pgfscope}%
\pgfsetrectcap%
\pgfsetmiterjoin%
\pgfsetlinewidth{0.803000pt}%
\definecolor{currentstroke}{rgb}{1.000000,1.000000,1.000000}%
\pgfsetstrokecolor{currentstroke}%
\pgfsetdash{}{0pt}%
\pgfpathmoveto{\pgfqpoint{9.810417in}{16.722093in}}%
\pgfpathlineto{\pgfqpoint{15.300000in}{16.722093in}}%
\pgfusepath{stroke}%
\end{pgfscope}%
\begin{pgfscope}%
\pgfsetrectcap%
\pgfsetmiterjoin%
\pgfsetlinewidth{0.803000pt}%
\definecolor{currentstroke}{rgb}{1.000000,1.000000,1.000000}%
\pgfsetstrokecolor{currentstroke}%
\pgfsetdash{}{0pt}%
\pgfpathmoveto{\pgfqpoint{9.810417in}{17.600000in}}%
\pgfpathlineto{\pgfqpoint{15.300000in}{17.600000in}}%
\pgfusepath{stroke}%
\end{pgfscope}%
\begin{pgfscope}%
\definecolor{textcolor}{rgb}{0.150000,0.150000,0.150000}%
\pgfsetstrokecolor{textcolor}%
\pgfsetfillcolor{textcolor}%
\pgftext[x=12.555208in,y=17.683333in,,base]{\color{textcolor}\rmfamily\fontsize{16.800000}{20.160000}\selectfont Partial Autocorrelation}%
\end{pgfscope}%
\begin{pgfscope}%
\pgfsetbuttcap%
\pgfsetmiterjoin%
\definecolor{currentfill}{rgb}{0.917647,0.917647,0.949020}%
\pgfsetfillcolor{currentfill}%
\pgfsetlinewidth{0.000000pt}%
\definecolor{currentstroke}{rgb}{0.000000,0.000000,0.000000}%
\pgfsetstrokecolor{currentstroke}%
\pgfsetstrokeopacity{0.000000}%
\pgfsetdash{}{0pt}%
\pgfpathmoveto{\pgfqpoint{2.125000in}{15.141860in}}%
\pgfpathlineto{\pgfqpoint{7.614583in}{15.141860in}}%
\pgfpathlineto{\pgfqpoint{7.614583in}{16.019767in}}%
\pgfpathlineto{\pgfqpoint{2.125000in}{16.019767in}}%
\pgfpathclose%
\pgfusepath{fill}%
\end{pgfscope}%
\begin{pgfscope}%
\pgfpathrectangle{\pgfqpoint{2.125000in}{15.141860in}}{\pgfqpoint{5.489583in}{0.877907in}}%
\pgfusepath{clip}%
\pgfsetroundcap%
\pgfsetroundjoin%
\pgfsetlinewidth{0.803000pt}%
\definecolor{currentstroke}{rgb}{1.000000,1.000000,1.000000}%
\pgfsetstrokecolor{currentstroke}%
\pgfsetdash{}{0pt}%
\pgfpathmoveto{\pgfqpoint{2.374527in}{15.141860in}}%
\pgfpathlineto{\pgfqpoint{2.374527in}{16.019767in}}%
\pgfusepath{stroke}%
\end{pgfscope}%
\begin{pgfscope}%
\definecolor{textcolor}{rgb}{0.150000,0.150000,0.150000}%
\pgfsetstrokecolor{textcolor}%
\pgfsetfillcolor{textcolor}%
\pgftext[x=2.374527in,y=15.044638in,,top]{\color{textcolor}\rmfamily\fontsize{14.000000}{16.800000}\selectfont 0}%
\end{pgfscope}%
\begin{pgfscope}%
\pgfpathrectangle{\pgfqpoint{2.125000in}{15.141860in}}{\pgfqpoint{5.489583in}{0.877907in}}%
\pgfusepath{clip}%
\pgfsetroundcap%
\pgfsetroundjoin%
\pgfsetlinewidth{0.803000pt}%
\definecolor{currentstroke}{rgb}{1.000000,1.000000,1.000000}%
\pgfsetstrokecolor{currentstroke}%
\pgfsetdash{}{0pt}%
\pgfpathmoveto{\pgfqpoint{2.990641in}{15.141860in}}%
\pgfpathlineto{\pgfqpoint{2.990641in}{16.019767in}}%
\pgfusepath{stroke}%
\end{pgfscope}%
\begin{pgfscope}%
\definecolor{textcolor}{rgb}{0.150000,0.150000,0.150000}%
\pgfsetstrokecolor{textcolor}%
\pgfsetfillcolor{textcolor}%
\pgftext[x=2.990641in,y=15.044638in,,top]{\color{textcolor}\rmfamily\fontsize{14.000000}{16.800000}\selectfont 5}%
\end{pgfscope}%
\begin{pgfscope}%
\pgfpathrectangle{\pgfqpoint{2.125000in}{15.141860in}}{\pgfqpoint{5.489583in}{0.877907in}}%
\pgfusepath{clip}%
\pgfsetroundcap%
\pgfsetroundjoin%
\pgfsetlinewidth{0.803000pt}%
\definecolor{currentstroke}{rgb}{1.000000,1.000000,1.000000}%
\pgfsetstrokecolor{currentstroke}%
\pgfsetdash{}{0pt}%
\pgfpathmoveto{\pgfqpoint{3.606756in}{15.141860in}}%
\pgfpathlineto{\pgfqpoint{3.606756in}{16.019767in}}%
\pgfusepath{stroke}%
\end{pgfscope}%
\begin{pgfscope}%
\definecolor{textcolor}{rgb}{0.150000,0.150000,0.150000}%
\pgfsetstrokecolor{textcolor}%
\pgfsetfillcolor{textcolor}%
\pgftext[x=3.606756in,y=15.044638in,,top]{\color{textcolor}\rmfamily\fontsize{14.000000}{16.800000}\selectfont 10}%
\end{pgfscope}%
\begin{pgfscope}%
\pgfpathrectangle{\pgfqpoint{2.125000in}{15.141860in}}{\pgfqpoint{5.489583in}{0.877907in}}%
\pgfusepath{clip}%
\pgfsetroundcap%
\pgfsetroundjoin%
\pgfsetlinewidth{0.803000pt}%
\definecolor{currentstroke}{rgb}{1.000000,1.000000,1.000000}%
\pgfsetstrokecolor{currentstroke}%
\pgfsetdash{}{0pt}%
\pgfpathmoveto{\pgfqpoint{4.222871in}{15.141860in}}%
\pgfpathlineto{\pgfqpoint{4.222871in}{16.019767in}}%
\pgfusepath{stroke}%
\end{pgfscope}%
\begin{pgfscope}%
\definecolor{textcolor}{rgb}{0.150000,0.150000,0.150000}%
\pgfsetstrokecolor{textcolor}%
\pgfsetfillcolor{textcolor}%
\pgftext[x=4.222871in,y=15.044638in,,top]{\color{textcolor}\rmfamily\fontsize{14.000000}{16.800000}\selectfont 15}%
\end{pgfscope}%
\begin{pgfscope}%
\pgfpathrectangle{\pgfqpoint{2.125000in}{15.141860in}}{\pgfqpoint{5.489583in}{0.877907in}}%
\pgfusepath{clip}%
\pgfsetroundcap%
\pgfsetroundjoin%
\pgfsetlinewidth{0.803000pt}%
\definecolor{currentstroke}{rgb}{1.000000,1.000000,1.000000}%
\pgfsetstrokecolor{currentstroke}%
\pgfsetdash{}{0pt}%
\pgfpathmoveto{\pgfqpoint{4.838986in}{15.141860in}}%
\pgfpathlineto{\pgfqpoint{4.838986in}{16.019767in}}%
\pgfusepath{stroke}%
\end{pgfscope}%
\begin{pgfscope}%
\definecolor{textcolor}{rgb}{0.150000,0.150000,0.150000}%
\pgfsetstrokecolor{textcolor}%
\pgfsetfillcolor{textcolor}%
\pgftext[x=4.838986in,y=15.044638in,,top]{\color{textcolor}\rmfamily\fontsize{14.000000}{16.800000}\selectfont 20}%
\end{pgfscope}%
\begin{pgfscope}%
\pgfpathrectangle{\pgfqpoint{2.125000in}{15.141860in}}{\pgfqpoint{5.489583in}{0.877907in}}%
\pgfusepath{clip}%
\pgfsetroundcap%
\pgfsetroundjoin%
\pgfsetlinewidth{0.803000pt}%
\definecolor{currentstroke}{rgb}{1.000000,1.000000,1.000000}%
\pgfsetstrokecolor{currentstroke}%
\pgfsetdash{}{0pt}%
\pgfpathmoveto{\pgfqpoint{5.455101in}{15.141860in}}%
\pgfpathlineto{\pgfqpoint{5.455101in}{16.019767in}}%
\pgfusepath{stroke}%
\end{pgfscope}%
\begin{pgfscope}%
\definecolor{textcolor}{rgb}{0.150000,0.150000,0.150000}%
\pgfsetstrokecolor{textcolor}%
\pgfsetfillcolor{textcolor}%
\pgftext[x=5.455101in,y=15.044638in,,top]{\color{textcolor}\rmfamily\fontsize{14.000000}{16.800000}\selectfont 25}%
\end{pgfscope}%
\begin{pgfscope}%
\pgfpathrectangle{\pgfqpoint{2.125000in}{15.141860in}}{\pgfqpoint{5.489583in}{0.877907in}}%
\pgfusepath{clip}%
\pgfsetroundcap%
\pgfsetroundjoin%
\pgfsetlinewidth{0.803000pt}%
\definecolor{currentstroke}{rgb}{1.000000,1.000000,1.000000}%
\pgfsetstrokecolor{currentstroke}%
\pgfsetdash{}{0pt}%
\pgfpathmoveto{\pgfqpoint{6.071216in}{15.141860in}}%
\pgfpathlineto{\pgfqpoint{6.071216in}{16.019767in}}%
\pgfusepath{stroke}%
\end{pgfscope}%
\begin{pgfscope}%
\definecolor{textcolor}{rgb}{0.150000,0.150000,0.150000}%
\pgfsetstrokecolor{textcolor}%
\pgfsetfillcolor{textcolor}%
\pgftext[x=6.071216in,y=15.044638in,,top]{\color{textcolor}\rmfamily\fontsize{14.000000}{16.800000}\selectfont 30}%
\end{pgfscope}%
\begin{pgfscope}%
\pgfpathrectangle{\pgfqpoint{2.125000in}{15.141860in}}{\pgfqpoint{5.489583in}{0.877907in}}%
\pgfusepath{clip}%
\pgfsetroundcap%
\pgfsetroundjoin%
\pgfsetlinewidth{0.803000pt}%
\definecolor{currentstroke}{rgb}{1.000000,1.000000,1.000000}%
\pgfsetstrokecolor{currentstroke}%
\pgfsetdash{}{0pt}%
\pgfpathmoveto{\pgfqpoint{6.687330in}{15.141860in}}%
\pgfpathlineto{\pgfqpoint{6.687330in}{16.019767in}}%
\pgfusepath{stroke}%
\end{pgfscope}%
\begin{pgfscope}%
\definecolor{textcolor}{rgb}{0.150000,0.150000,0.150000}%
\pgfsetstrokecolor{textcolor}%
\pgfsetfillcolor{textcolor}%
\pgftext[x=6.687330in,y=15.044638in,,top]{\color{textcolor}\rmfamily\fontsize{14.000000}{16.800000}\selectfont 35}%
\end{pgfscope}%
\begin{pgfscope}%
\pgfpathrectangle{\pgfqpoint{2.125000in}{15.141860in}}{\pgfqpoint{5.489583in}{0.877907in}}%
\pgfusepath{clip}%
\pgfsetroundcap%
\pgfsetroundjoin%
\pgfsetlinewidth{0.803000pt}%
\definecolor{currentstroke}{rgb}{1.000000,1.000000,1.000000}%
\pgfsetstrokecolor{currentstroke}%
\pgfsetdash{}{0pt}%
\pgfpathmoveto{\pgfqpoint{7.303445in}{15.141860in}}%
\pgfpathlineto{\pgfqpoint{7.303445in}{16.019767in}}%
\pgfusepath{stroke}%
\end{pgfscope}%
\begin{pgfscope}%
\definecolor{textcolor}{rgb}{0.150000,0.150000,0.150000}%
\pgfsetstrokecolor{textcolor}%
\pgfsetfillcolor{textcolor}%
\pgftext[x=7.303445in,y=15.044638in,,top]{\color{textcolor}\rmfamily\fontsize{14.000000}{16.800000}\selectfont 40}%
\end{pgfscope}%
\begin{pgfscope}%
\pgfpathrectangle{\pgfqpoint{2.125000in}{15.141860in}}{\pgfqpoint{5.489583in}{0.877907in}}%
\pgfusepath{clip}%
\pgfsetroundcap%
\pgfsetroundjoin%
\pgfsetlinewidth{0.803000pt}%
\definecolor{currentstroke}{rgb}{1.000000,1.000000,1.000000}%
\pgfsetstrokecolor{currentstroke}%
\pgfsetdash{}{0pt}%
\pgfpathmoveto{\pgfqpoint{2.125000in}{15.412639in}}%
\pgfpathlineto{\pgfqpoint{7.614583in}{15.412639in}}%
\pgfusepath{stroke}%
\end{pgfscope}%
\begin{pgfscope}%
\definecolor{textcolor}{rgb}{0.150000,0.150000,0.150000}%
\pgfsetstrokecolor{textcolor}%
\pgfsetfillcolor{textcolor}%
\pgftext[x=1.904066in,y=15.338773in,left,base]{\color{textcolor}\rmfamily\fontsize{14.000000}{16.800000}\selectfont 0}%
\end{pgfscope}%
\begin{pgfscope}%
\pgfpathrectangle{\pgfqpoint{2.125000in}{15.141860in}}{\pgfqpoint{5.489583in}{0.877907in}}%
\pgfusepath{clip}%
\pgfsetroundcap%
\pgfsetroundjoin%
\pgfsetlinewidth{0.803000pt}%
\definecolor{currentstroke}{rgb}{1.000000,1.000000,1.000000}%
\pgfsetstrokecolor{currentstroke}%
\pgfsetdash{}{0pt}%
\pgfpathmoveto{\pgfqpoint{2.125000in}{15.979863in}}%
\pgfpathlineto{\pgfqpoint{7.614583in}{15.979863in}}%
\pgfusepath{stroke}%
\end{pgfscope}%
\begin{pgfscope}%
\definecolor{textcolor}{rgb}{0.150000,0.150000,0.150000}%
\pgfsetstrokecolor{textcolor}%
\pgfsetfillcolor{textcolor}%
\pgftext[x=1.904066in,y=15.905996in,left,base]{\color{textcolor}\rmfamily\fontsize{14.000000}{16.800000}\selectfont 1}%
\end{pgfscope}%
\begin{pgfscope}%
\pgfpathrectangle{\pgfqpoint{2.125000in}{15.141860in}}{\pgfqpoint{5.489583in}{0.877907in}}%
\pgfusepath{clip}%
\pgfsetbuttcap%
\pgfsetroundjoin%
\definecolor{currentfill}{rgb}{0.121569,0.466667,0.705882}%
\pgfsetfillcolor{currentfill}%
\pgfsetfillopacity{0.250000}%
\pgfsetlinewidth{1.003750pt}%
\definecolor{currentstroke}{rgb}{1.000000,1.000000,1.000000}%
\pgfsetstrokecolor{currentstroke}%
\pgfsetstrokeopacity{0.250000}%
\pgfsetdash{}{0pt}%
\pgfpathmoveto{\pgfqpoint{2.436138in}{15.441259in}}%
\pgfpathlineto{\pgfqpoint{2.436138in}{15.384020in}}%
\pgfpathlineto{\pgfqpoint{2.620972in}{15.363240in}}%
\pgfpathlineto{\pgfqpoint{2.744195in}{15.349046in}}%
\pgfpathlineto{\pgfqpoint{2.867418in}{15.337597in}}%
\pgfpathlineto{\pgfqpoint{2.990641in}{15.327775in}}%
\pgfpathlineto{\pgfqpoint{3.113864in}{15.319064in}}%
\pgfpathlineto{\pgfqpoint{3.237087in}{15.311174in}}%
\pgfpathlineto{\pgfqpoint{3.360310in}{15.303926in}}%
\pgfpathlineto{\pgfqpoint{3.483533in}{15.297196in}}%
\pgfpathlineto{\pgfqpoint{3.606756in}{15.290901in}}%
\pgfpathlineto{\pgfqpoint{3.729979in}{15.284972in}}%
\pgfpathlineto{\pgfqpoint{3.853202in}{15.279358in}}%
\pgfpathlineto{\pgfqpoint{3.976425in}{15.274018in}}%
\pgfpathlineto{\pgfqpoint{4.099648in}{15.268926in}}%
\pgfpathlineto{\pgfqpoint{4.222871in}{15.264057in}}%
\pgfpathlineto{\pgfqpoint{4.346094in}{15.259392in}}%
\pgfpathlineto{\pgfqpoint{4.469317in}{15.254909in}}%
\pgfpathlineto{\pgfqpoint{4.592540in}{15.250595in}}%
\pgfpathlineto{\pgfqpoint{4.715763in}{15.246437in}}%
\pgfpathlineto{\pgfqpoint{4.838986in}{15.242425in}}%
\pgfpathlineto{\pgfqpoint{4.962209in}{15.238545in}}%
\pgfpathlineto{\pgfqpoint{5.085432in}{15.234788in}}%
\pgfpathlineto{\pgfqpoint{5.208655in}{15.231147in}}%
\pgfpathlineto{\pgfqpoint{5.331878in}{15.227612in}}%
\pgfpathlineto{\pgfqpoint{5.455101in}{15.224176in}}%
\pgfpathlineto{\pgfqpoint{5.578324in}{15.220833in}}%
\pgfpathlineto{\pgfqpoint{5.701547in}{15.217578in}}%
\pgfpathlineto{\pgfqpoint{5.824770in}{15.214408in}}%
\pgfpathlineto{\pgfqpoint{5.947993in}{15.211319in}}%
\pgfpathlineto{\pgfqpoint{6.071216in}{15.208307in}}%
\pgfpathlineto{\pgfqpoint{6.194439in}{15.205370in}}%
\pgfpathlineto{\pgfqpoint{6.317662in}{15.202503in}}%
\pgfpathlineto{\pgfqpoint{6.440885in}{15.199702in}}%
\pgfpathlineto{\pgfqpoint{6.564108in}{15.196964in}}%
\pgfpathlineto{\pgfqpoint{6.687330in}{15.194288in}}%
\pgfpathlineto{\pgfqpoint{6.810553in}{15.191672in}}%
\pgfpathlineto{\pgfqpoint{6.933776in}{15.189115in}}%
\pgfpathlineto{\pgfqpoint{7.056999in}{15.186614in}}%
\pgfpathlineto{\pgfqpoint{7.180222in}{15.184165in}}%
\pgfpathlineto{\pgfqpoint{7.365057in}{15.181765in}}%
\pgfpathlineto{\pgfqpoint{7.365057in}{15.643514in}}%
\pgfpathlineto{\pgfqpoint{7.365057in}{15.643514in}}%
\pgfpathlineto{\pgfqpoint{7.180222in}{15.641114in}}%
\pgfpathlineto{\pgfqpoint{7.056999in}{15.638665in}}%
\pgfpathlineto{\pgfqpoint{6.933776in}{15.636164in}}%
\pgfpathlineto{\pgfqpoint{6.810553in}{15.633607in}}%
\pgfpathlineto{\pgfqpoint{6.687330in}{15.630991in}}%
\pgfpathlineto{\pgfqpoint{6.564108in}{15.628315in}}%
\pgfpathlineto{\pgfqpoint{6.440885in}{15.625577in}}%
\pgfpathlineto{\pgfqpoint{6.317662in}{15.622776in}}%
\pgfpathlineto{\pgfqpoint{6.194439in}{15.619909in}}%
\pgfpathlineto{\pgfqpoint{6.071216in}{15.616972in}}%
\pgfpathlineto{\pgfqpoint{5.947993in}{15.613960in}}%
\pgfpathlineto{\pgfqpoint{5.824770in}{15.610871in}}%
\pgfpathlineto{\pgfqpoint{5.701547in}{15.607700in}}%
\pgfpathlineto{\pgfqpoint{5.578324in}{15.604446in}}%
\pgfpathlineto{\pgfqpoint{5.455101in}{15.601103in}}%
\pgfpathlineto{\pgfqpoint{5.331878in}{15.597667in}}%
\pgfpathlineto{\pgfqpoint{5.208655in}{15.594132in}}%
\pgfpathlineto{\pgfqpoint{5.085432in}{15.590490in}}%
\pgfpathlineto{\pgfqpoint{4.962209in}{15.586734in}}%
\pgfpathlineto{\pgfqpoint{4.838986in}{15.582854in}}%
\pgfpathlineto{\pgfqpoint{4.715763in}{15.578842in}}%
\pgfpathlineto{\pgfqpoint{4.592540in}{15.574684in}}%
\pgfpathlineto{\pgfqpoint{4.469317in}{15.570369in}}%
\pgfpathlineto{\pgfqpoint{4.346094in}{15.565887in}}%
\pgfpathlineto{\pgfqpoint{4.222871in}{15.561222in}}%
\pgfpathlineto{\pgfqpoint{4.099648in}{15.556353in}}%
\pgfpathlineto{\pgfqpoint{3.976425in}{15.551261in}}%
\pgfpathlineto{\pgfqpoint{3.853202in}{15.545921in}}%
\pgfpathlineto{\pgfqpoint{3.729979in}{15.540307in}}%
\pgfpathlineto{\pgfqpoint{3.606756in}{15.534378in}}%
\pgfpathlineto{\pgfqpoint{3.483533in}{15.528082in}}%
\pgfpathlineto{\pgfqpoint{3.360310in}{15.521353in}}%
\pgfpathlineto{\pgfqpoint{3.237087in}{15.514104in}}%
\pgfpathlineto{\pgfqpoint{3.113864in}{15.506214in}}%
\pgfpathlineto{\pgfqpoint{2.990641in}{15.497504in}}%
\pgfpathlineto{\pgfqpoint{2.867418in}{15.487682in}}%
\pgfpathlineto{\pgfqpoint{2.744195in}{15.476233in}}%
\pgfpathlineto{\pgfqpoint{2.620972in}{15.462039in}}%
\pgfpathlineto{\pgfqpoint{2.436138in}{15.441259in}}%
\pgfpathclose%
\pgfusepath{stroke,fill}%
\end{pgfscope}%
\begin{pgfscope}%
\pgfpathrectangle{\pgfqpoint{2.125000in}{15.141860in}}{\pgfqpoint{5.489583in}{0.877907in}}%
\pgfusepath{clip}%
\pgfsetbuttcap%
\pgfsetroundjoin%
\pgfsetlinewidth{1.505625pt}%
\definecolor{currentstroke}{rgb}{0.000000,0.000000,0.000000}%
\pgfsetstrokecolor{currentstroke}%
\pgfsetdash{}{0pt}%
\pgfpathmoveto{\pgfqpoint{2.374527in}{15.412639in}}%
\pgfpathlineto{\pgfqpoint{2.374527in}{15.979863in}}%
\pgfusepath{stroke}%
\end{pgfscope}%
\begin{pgfscope}%
\pgfpathrectangle{\pgfqpoint{2.125000in}{15.141860in}}{\pgfqpoint{5.489583in}{0.877907in}}%
\pgfusepath{clip}%
\pgfsetbuttcap%
\pgfsetroundjoin%
\pgfsetlinewidth{1.505625pt}%
\definecolor{currentstroke}{rgb}{0.000000,0.000000,0.000000}%
\pgfsetstrokecolor{currentstroke}%
\pgfsetdash{}{0pt}%
\pgfpathmoveto{\pgfqpoint{2.497749in}{15.412639in}}%
\pgfpathlineto{\pgfqpoint{2.497749in}{15.976931in}}%
\pgfusepath{stroke}%
\end{pgfscope}%
\begin{pgfscope}%
\pgfpathrectangle{\pgfqpoint{2.125000in}{15.141860in}}{\pgfqpoint{5.489583in}{0.877907in}}%
\pgfusepath{clip}%
\pgfsetbuttcap%
\pgfsetroundjoin%
\pgfsetlinewidth{1.505625pt}%
\definecolor{currentstroke}{rgb}{0.000000,0.000000,0.000000}%
\pgfsetstrokecolor{currentstroke}%
\pgfsetdash{}{0pt}%
\pgfpathmoveto{\pgfqpoint{2.620972in}{15.412639in}}%
\pgfpathlineto{\pgfqpoint{2.620972in}{15.973898in}}%
\pgfusepath{stroke}%
\end{pgfscope}%
\begin{pgfscope}%
\pgfpathrectangle{\pgfqpoint{2.125000in}{15.141860in}}{\pgfqpoint{5.489583in}{0.877907in}}%
\pgfusepath{clip}%
\pgfsetbuttcap%
\pgfsetroundjoin%
\pgfsetlinewidth{1.505625pt}%
\definecolor{currentstroke}{rgb}{0.000000,0.000000,0.000000}%
\pgfsetstrokecolor{currentstroke}%
\pgfsetdash{}{0pt}%
\pgfpathmoveto{\pgfqpoint{2.744195in}{15.412639in}}%
\pgfpathlineto{\pgfqpoint{2.744195in}{15.970976in}}%
\pgfusepath{stroke}%
\end{pgfscope}%
\begin{pgfscope}%
\pgfpathrectangle{\pgfqpoint{2.125000in}{15.141860in}}{\pgfqpoint{5.489583in}{0.877907in}}%
\pgfusepath{clip}%
\pgfsetbuttcap%
\pgfsetroundjoin%
\pgfsetlinewidth{1.505625pt}%
\definecolor{currentstroke}{rgb}{0.000000,0.000000,0.000000}%
\pgfsetstrokecolor{currentstroke}%
\pgfsetdash{}{0pt}%
\pgfpathmoveto{\pgfqpoint{2.867418in}{15.412639in}}%
\pgfpathlineto{\pgfqpoint{2.867418in}{15.968054in}}%
\pgfusepath{stroke}%
\end{pgfscope}%
\begin{pgfscope}%
\pgfpathrectangle{\pgfqpoint{2.125000in}{15.141860in}}{\pgfqpoint{5.489583in}{0.877907in}}%
\pgfusepath{clip}%
\pgfsetbuttcap%
\pgfsetroundjoin%
\pgfsetlinewidth{1.505625pt}%
\definecolor{currentstroke}{rgb}{0.000000,0.000000,0.000000}%
\pgfsetstrokecolor{currentstroke}%
\pgfsetdash{}{0pt}%
\pgfpathmoveto{\pgfqpoint{2.990641in}{15.412639in}}%
\pgfpathlineto{\pgfqpoint{2.990641in}{15.965165in}}%
\pgfusepath{stroke}%
\end{pgfscope}%
\begin{pgfscope}%
\pgfpathrectangle{\pgfqpoint{2.125000in}{15.141860in}}{\pgfqpoint{5.489583in}{0.877907in}}%
\pgfusepath{clip}%
\pgfsetbuttcap%
\pgfsetroundjoin%
\pgfsetlinewidth{1.505625pt}%
\definecolor{currentstroke}{rgb}{0.000000,0.000000,0.000000}%
\pgfsetstrokecolor{currentstroke}%
\pgfsetdash{}{0pt}%
\pgfpathmoveto{\pgfqpoint{3.113864in}{15.412639in}}%
\pgfpathlineto{\pgfqpoint{3.113864in}{15.962414in}}%
\pgfusepath{stroke}%
\end{pgfscope}%
\begin{pgfscope}%
\pgfpathrectangle{\pgfqpoint{2.125000in}{15.141860in}}{\pgfqpoint{5.489583in}{0.877907in}}%
\pgfusepath{clip}%
\pgfsetbuttcap%
\pgfsetroundjoin%
\pgfsetlinewidth{1.505625pt}%
\definecolor{currentstroke}{rgb}{0.000000,0.000000,0.000000}%
\pgfsetstrokecolor{currentstroke}%
\pgfsetdash{}{0pt}%
\pgfpathmoveto{\pgfqpoint{3.237087in}{15.412639in}}%
\pgfpathlineto{\pgfqpoint{3.237087in}{15.959647in}}%
\pgfusepath{stroke}%
\end{pgfscope}%
\begin{pgfscope}%
\pgfpathrectangle{\pgfqpoint{2.125000in}{15.141860in}}{\pgfqpoint{5.489583in}{0.877907in}}%
\pgfusepath{clip}%
\pgfsetbuttcap%
\pgfsetroundjoin%
\pgfsetlinewidth{1.505625pt}%
\definecolor{currentstroke}{rgb}{0.000000,0.000000,0.000000}%
\pgfsetstrokecolor{currentstroke}%
\pgfsetdash{}{0pt}%
\pgfpathmoveto{\pgfqpoint{3.360310in}{15.412639in}}%
\pgfpathlineto{\pgfqpoint{3.360310in}{15.956958in}}%
\pgfusepath{stroke}%
\end{pgfscope}%
\begin{pgfscope}%
\pgfpathrectangle{\pgfqpoint{2.125000in}{15.141860in}}{\pgfqpoint{5.489583in}{0.877907in}}%
\pgfusepath{clip}%
\pgfsetbuttcap%
\pgfsetroundjoin%
\pgfsetlinewidth{1.505625pt}%
\definecolor{currentstroke}{rgb}{0.000000,0.000000,0.000000}%
\pgfsetstrokecolor{currentstroke}%
\pgfsetdash{}{0pt}%
\pgfpathmoveto{\pgfqpoint{3.483533in}{15.412639in}}%
\pgfpathlineto{\pgfqpoint{3.483533in}{15.954179in}}%
\pgfusepath{stroke}%
\end{pgfscope}%
\begin{pgfscope}%
\pgfpathrectangle{\pgfqpoint{2.125000in}{15.141860in}}{\pgfqpoint{5.489583in}{0.877907in}}%
\pgfusepath{clip}%
\pgfsetbuttcap%
\pgfsetroundjoin%
\pgfsetlinewidth{1.505625pt}%
\definecolor{currentstroke}{rgb}{0.000000,0.000000,0.000000}%
\pgfsetstrokecolor{currentstroke}%
\pgfsetdash{}{0pt}%
\pgfpathmoveto{\pgfqpoint{3.606756in}{15.412639in}}%
\pgfpathlineto{\pgfqpoint{3.606756in}{15.951552in}}%
\pgfusepath{stroke}%
\end{pgfscope}%
\begin{pgfscope}%
\pgfpathrectangle{\pgfqpoint{2.125000in}{15.141860in}}{\pgfqpoint{5.489583in}{0.877907in}}%
\pgfusepath{clip}%
\pgfsetbuttcap%
\pgfsetroundjoin%
\pgfsetlinewidth{1.505625pt}%
\definecolor{currentstroke}{rgb}{0.000000,0.000000,0.000000}%
\pgfsetstrokecolor{currentstroke}%
\pgfsetdash{}{0pt}%
\pgfpathmoveto{\pgfqpoint{3.729979in}{15.412639in}}%
\pgfpathlineto{\pgfqpoint{3.729979in}{15.949071in}}%
\pgfusepath{stroke}%
\end{pgfscope}%
\begin{pgfscope}%
\pgfpathrectangle{\pgfqpoint{2.125000in}{15.141860in}}{\pgfqpoint{5.489583in}{0.877907in}}%
\pgfusepath{clip}%
\pgfsetbuttcap%
\pgfsetroundjoin%
\pgfsetlinewidth{1.505625pt}%
\definecolor{currentstroke}{rgb}{0.000000,0.000000,0.000000}%
\pgfsetstrokecolor{currentstroke}%
\pgfsetdash{}{0pt}%
\pgfpathmoveto{\pgfqpoint{3.853202in}{15.412639in}}%
\pgfpathlineto{\pgfqpoint{3.853202in}{15.946640in}}%
\pgfusepath{stroke}%
\end{pgfscope}%
\begin{pgfscope}%
\pgfpathrectangle{\pgfqpoint{2.125000in}{15.141860in}}{\pgfqpoint{5.489583in}{0.877907in}}%
\pgfusepath{clip}%
\pgfsetbuttcap%
\pgfsetroundjoin%
\pgfsetlinewidth{1.505625pt}%
\definecolor{currentstroke}{rgb}{0.000000,0.000000,0.000000}%
\pgfsetstrokecolor{currentstroke}%
\pgfsetdash{}{0pt}%
\pgfpathmoveto{\pgfqpoint{3.976425in}{15.412639in}}%
\pgfpathlineto{\pgfqpoint{3.976425in}{15.944043in}}%
\pgfusepath{stroke}%
\end{pgfscope}%
\begin{pgfscope}%
\pgfpathrectangle{\pgfqpoint{2.125000in}{15.141860in}}{\pgfqpoint{5.489583in}{0.877907in}}%
\pgfusepath{clip}%
\pgfsetbuttcap%
\pgfsetroundjoin%
\pgfsetlinewidth{1.505625pt}%
\definecolor{currentstroke}{rgb}{0.000000,0.000000,0.000000}%
\pgfsetstrokecolor{currentstroke}%
\pgfsetdash{}{0pt}%
\pgfpathmoveto{\pgfqpoint{4.099648in}{15.412639in}}%
\pgfpathlineto{\pgfqpoint{4.099648in}{15.941321in}}%
\pgfusepath{stroke}%
\end{pgfscope}%
\begin{pgfscope}%
\pgfpathrectangle{\pgfqpoint{2.125000in}{15.141860in}}{\pgfqpoint{5.489583in}{0.877907in}}%
\pgfusepath{clip}%
\pgfsetbuttcap%
\pgfsetroundjoin%
\pgfsetlinewidth{1.505625pt}%
\definecolor{currentstroke}{rgb}{0.000000,0.000000,0.000000}%
\pgfsetstrokecolor{currentstroke}%
\pgfsetdash{}{0pt}%
\pgfpathmoveto{\pgfqpoint{4.222871in}{15.412639in}}%
\pgfpathlineto{\pgfqpoint{4.222871in}{15.938536in}}%
\pgfusepath{stroke}%
\end{pgfscope}%
\begin{pgfscope}%
\pgfpathrectangle{\pgfqpoint{2.125000in}{15.141860in}}{\pgfqpoint{5.489583in}{0.877907in}}%
\pgfusepath{clip}%
\pgfsetbuttcap%
\pgfsetroundjoin%
\pgfsetlinewidth{1.505625pt}%
\definecolor{currentstroke}{rgb}{0.000000,0.000000,0.000000}%
\pgfsetstrokecolor{currentstroke}%
\pgfsetdash{}{0pt}%
\pgfpathmoveto{\pgfqpoint{4.346094in}{15.412639in}}%
\pgfpathlineto{\pgfqpoint{4.346094in}{15.935889in}}%
\pgfusepath{stroke}%
\end{pgfscope}%
\begin{pgfscope}%
\pgfpathrectangle{\pgfqpoint{2.125000in}{15.141860in}}{\pgfqpoint{5.489583in}{0.877907in}}%
\pgfusepath{clip}%
\pgfsetbuttcap%
\pgfsetroundjoin%
\pgfsetlinewidth{1.505625pt}%
\definecolor{currentstroke}{rgb}{0.000000,0.000000,0.000000}%
\pgfsetstrokecolor{currentstroke}%
\pgfsetdash{}{0pt}%
\pgfpathmoveto{\pgfqpoint{4.469317in}{15.412639in}}%
\pgfpathlineto{\pgfqpoint{4.469317in}{15.933211in}}%
\pgfusepath{stroke}%
\end{pgfscope}%
\begin{pgfscope}%
\pgfpathrectangle{\pgfqpoint{2.125000in}{15.141860in}}{\pgfqpoint{5.489583in}{0.877907in}}%
\pgfusepath{clip}%
\pgfsetbuttcap%
\pgfsetroundjoin%
\pgfsetlinewidth{1.505625pt}%
\definecolor{currentstroke}{rgb}{0.000000,0.000000,0.000000}%
\pgfsetstrokecolor{currentstroke}%
\pgfsetdash{}{0pt}%
\pgfpathmoveto{\pgfqpoint{4.592540in}{15.412639in}}%
\pgfpathlineto{\pgfqpoint{4.592540in}{15.930379in}}%
\pgfusepath{stroke}%
\end{pgfscope}%
\begin{pgfscope}%
\pgfpathrectangle{\pgfqpoint{2.125000in}{15.141860in}}{\pgfqpoint{5.489583in}{0.877907in}}%
\pgfusepath{clip}%
\pgfsetbuttcap%
\pgfsetroundjoin%
\pgfsetlinewidth{1.505625pt}%
\definecolor{currentstroke}{rgb}{0.000000,0.000000,0.000000}%
\pgfsetstrokecolor{currentstroke}%
\pgfsetdash{}{0pt}%
\pgfpathmoveto{\pgfqpoint{4.715763in}{15.412639in}}%
\pgfpathlineto{\pgfqpoint{4.715763in}{15.927532in}}%
\pgfusepath{stroke}%
\end{pgfscope}%
\begin{pgfscope}%
\pgfpathrectangle{\pgfqpoint{2.125000in}{15.141860in}}{\pgfqpoint{5.489583in}{0.877907in}}%
\pgfusepath{clip}%
\pgfsetbuttcap%
\pgfsetroundjoin%
\pgfsetlinewidth{1.505625pt}%
\definecolor{currentstroke}{rgb}{0.000000,0.000000,0.000000}%
\pgfsetstrokecolor{currentstroke}%
\pgfsetdash{}{0pt}%
\pgfpathmoveto{\pgfqpoint{4.838986in}{15.412639in}}%
\pgfpathlineto{\pgfqpoint{4.838986in}{15.924879in}}%
\pgfusepath{stroke}%
\end{pgfscope}%
\begin{pgfscope}%
\pgfpathrectangle{\pgfqpoint{2.125000in}{15.141860in}}{\pgfqpoint{5.489583in}{0.877907in}}%
\pgfusepath{clip}%
\pgfsetbuttcap%
\pgfsetroundjoin%
\pgfsetlinewidth{1.505625pt}%
\definecolor{currentstroke}{rgb}{0.000000,0.000000,0.000000}%
\pgfsetstrokecolor{currentstroke}%
\pgfsetdash{}{0pt}%
\pgfpathmoveto{\pgfqpoint{4.962209in}{15.412639in}}%
\pgfpathlineto{\pgfqpoint{4.962209in}{15.922206in}}%
\pgfusepath{stroke}%
\end{pgfscope}%
\begin{pgfscope}%
\pgfpathrectangle{\pgfqpoint{2.125000in}{15.141860in}}{\pgfqpoint{5.489583in}{0.877907in}}%
\pgfusepath{clip}%
\pgfsetbuttcap%
\pgfsetroundjoin%
\pgfsetlinewidth{1.505625pt}%
\definecolor{currentstroke}{rgb}{0.000000,0.000000,0.000000}%
\pgfsetstrokecolor{currentstroke}%
\pgfsetdash{}{0pt}%
\pgfpathmoveto{\pgfqpoint{5.085432in}{15.412639in}}%
\pgfpathlineto{\pgfqpoint{5.085432in}{15.919575in}}%
\pgfusepath{stroke}%
\end{pgfscope}%
\begin{pgfscope}%
\pgfpathrectangle{\pgfqpoint{2.125000in}{15.141860in}}{\pgfqpoint{5.489583in}{0.877907in}}%
\pgfusepath{clip}%
\pgfsetbuttcap%
\pgfsetroundjoin%
\pgfsetlinewidth{1.505625pt}%
\definecolor{currentstroke}{rgb}{0.000000,0.000000,0.000000}%
\pgfsetstrokecolor{currentstroke}%
\pgfsetdash{}{0pt}%
\pgfpathmoveto{\pgfqpoint{5.208655in}{15.412639in}}%
\pgfpathlineto{\pgfqpoint{5.208655in}{15.917107in}}%
\pgfusepath{stroke}%
\end{pgfscope}%
\begin{pgfscope}%
\pgfpathrectangle{\pgfqpoint{2.125000in}{15.141860in}}{\pgfqpoint{5.489583in}{0.877907in}}%
\pgfusepath{clip}%
\pgfsetbuttcap%
\pgfsetroundjoin%
\pgfsetlinewidth{1.505625pt}%
\definecolor{currentstroke}{rgb}{0.000000,0.000000,0.000000}%
\pgfsetstrokecolor{currentstroke}%
\pgfsetdash{}{0pt}%
\pgfpathmoveto{\pgfqpoint{5.331878in}{15.412639in}}%
\pgfpathlineto{\pgfqpoint{5.331878in}{15.914701in}}%
\pgfusepath{stroke}%
\end{pgfscope}%
\begin{pgfscope}%
\pgfpathrectangle{\pgfqpoint{2.125000in}{15.141860in}}{\pgfqpoint{5.489583in}{0.877907in}}%
\pgfusepath{clip}%
\pgfsetbuttcap%
\pgfsetroundjoin%
\pgfsetlinewidth{1.505625pt}%
\definecolor{currentstroke}{rgb}{0.000000,0.000000,0.000000}%
\pgfsetstrokecolor{currentstroke}%
\pgfsetdash{}{0pt}%
\pgfpathmoveto{\pgfqpoint{5.455101in}{15.412639in}}%
\pgfpathlineto{\pgfqpoint{5.455101in}{15.912328in}}%
\pgfusepath{stroke}%
\end{pgfscope}%
\begin{pgfscope}%
\pgfpathrectangle{\pgfqpoint{2.125000in}{15.141860in}}{\pgfqpoint{5.489583in}{0.877907in}}%
\pgfusepath{clip}%
\pgfsetbuttcap%
\pgfsetroundjoin%
\pgfsetlinewidth{1.505625pt}%
\definecolor{currentstroke}{rgb}{0.000000,0.000000,0.000000}%
\pgfsetstrokecolor{currentstroke}%
\pgfsetdash{}{0pt}%
\pgfpathmoveto{\pgfqpoint{5.578324in}{15.412639in}}%
\pgfpathlineto{\pgfqpoint{5.578324in}{15.909927in}}%
\pgfusepath{stroke}%
\end{pgfscope}%
\begin{pgfscope}%
\pgfpathrectangle{\pgfqpoint{2.125000in}{15.141860in}}{\pgfqpoint{5.489583in}{0.877907in}}%
\pgfusepath{clip}%
\pgfsetbuttcap%
\pgfsetroundjoin%
\pgfsetlinewidth{1.505625pt}%
\definecolor{currentstroke}{rgb}{0.000000,0.000000,0.000000}%
\pgfsetstrokecolor{currentstroke}%
\pgfsetdash{}{0pt}%
\pgfpathmoveto{\pgfqpoint{5.701547in}{15.412639in}}%
\pgfpathlineto{\pgfqpoint{5.701547in}{15.907513in}}%
\pgfusepath{stroke}%
\end{pgfscope}%
\begin{pgfscope}%
\pgfpathrectangle{\pgfqpoint{2.125000in}{15.141860in}}{\pgfqpoint{5.489583in}{0.877907in}}%
\pgfusepath{clip}%
\pgfsetbuttcap%
\pgfsetroundjoin%
\pgfsetlinewidth{1.505625pt}%
\definecolor{currentstroke}{rgb}{0.000000,0.000000,0.000000}%
\pgfsetstrokecolor{currentstroke}%
\pgfsetdash{}{0pt}%
\pgfpathmoveto{\pgfqpoint{5.824770in}{15.412639in}}%
\pgfpathlineto{\pgfqpoint{5.824770in}{15.905015in}}%
\pgfusepath{stroke}%
\end{pgfscope}%
\begin{pgfscope}%
\pgfpathrectangle{\pgfqpoint{2.125000in}{15.141860in}}{\pgfqpoint{5.489583in}{0.877907in}}%
\pgfusepath{clip}%
\pgfsetbuttcap%
\pgfsetroundjoin%
\pgfsetlinewidth{1.505625pt}%
\definecolor{currentstroke}{rgb}{0.000000,0.000000,0.000000}%
\pgfsetstrokecolor{currentstroke}%
\pgfsetdash{}{0pt}%
\pgfpathmoveto{\pgfqpoint{5.947993in}{15.412639in}}%
\pgfpathlineto{\pgfqpoint{5.947993in}{15.902482in}}%
\pgfusepath{stroke}%
\end{pgfscope}%
\begin{pgfscope}%
\pgfpathrectangle{\pgfqpoint{2.125000in}{15.141860in}}{\pgfqpoint{5.489583in}{0.877907in}}%
\pgfusepath{clip}%
\pgfsetbuttcap%
\pgfsetroundjoin%
\pgfsetlinewidth{1.505625pt}%
\definecolor{currentstroke}{rgb}{0.000000,0.000000,0.000000}%
\pgfsetstrokecolor{currentstroke}%
\pgfsetdash{}{0pt}%
\pgfpathmoveto{\pgfqpoint{6.071216in}{15.412639in}}%
\pgfpathlineto{\pgfqpoint{6.071216in}{15.899967in}}%
\pgfusepath{stroke}%
\end{pgfscope}%
\begin{pgfscope}%
\pgfpathrectangle{\pgfqpoint{2.125000in}{15.141860in}}{\pgfqpoint{5.489583in}{0.877907in}}%
\pgfusepath{clip}%
\pgfsetbuttcap%
\pgfsetroundjoin%
\pgfsetlinewidth{1.505625pt}%
\definecolor{currentstroke}{rgb}{0.000000,0.000000,0.000000}%
\pgfsetstrokecolor{currentstroke}%
\pgfsetdash{}{0pt}%
\pgfpathmoveto{\pgfqpoint{6.194439in}{15.412639in}}%
\pgfpathlineto{\pgfqpoint{6.194439in}{15.897447in}}%
\pgfusepath{stroke}%
\end{pgfscope}%
\begin{pgfscope}%
\pgfpathrectangle{\pgfqpoint{2.125000in}{15.141860in}}{\pgfqpoint{5.489583in}{0.877907in}}%
\pgfusepath{clip}%
\pgfsetbuttcap%
\pgfsetroundjoin%
\pgfsetlinewidth{1.505625pt}%
\definecolor{currentstroke}{rgb}{0.000000,0.000000,0.000000}%
\pgfsetstrokecolor{currentstroke}%
\pgfsetdash{}{0pt}%
\pgfpathmoveto{\pgfqpoint{6.317662in}{15.412639in}}%
\pgfpathlineto{\pgfqpoint{6.317662in}{15.895061in}}%
\pgfusepath{stroke}%
\end{pgfscope}%
\begin{pgfscope}%
\pgfpathrectangle{\pgfqpoint{2.125000in}{15.141860in}}{\pgfqpoint{5.489583in}{0.877907in}}%
\pgfusepath{clip}%
\pgfsetbuttcap%
\pgfsetroundjoin%
\pgfsetlinewidth{1.505625pt}%
\definecolor{currentstroke}{rgb}{0.000000,0.000000,0.000000}%
\pgfsetstrokecolor{currentstroke}%
\pgfsetdash{}{0pt}%
\pgfpathmoveto{\pgfqpoint{6.440885in}{15.412639in}}%
\pgfpathlineto{\pgfqpoint{6.440885in}{15.892727in}}%
\pgfusepath{stroke}%
\end{pgfscope}%
\begin{pgfscope}%
\pgfpathrectangle{\pgfqpoint{2.125000in}{15.141860in}}{\pgfqpoint{5.489583in}{0.877907in}}%
\pgfusepath{clip}%
\pgfsetbuttcap%
\pgfsetroundjoin%
\pgfsetlinewidth{1.505625pt}%
\definecolor{currentstroke}{rgb}{0.000000,0.000000,0.000000}%
\pgfsetstrokecolor{currentstroke}%
\pgfsetdash{}{0pt}%
\pgfpathmoveto{\pgfqpoint{6.564108in}{15.412639in}}%
\pgfpathlineto{\pgfqpoint{6.564108in}{15.890261in}}%
\pgfusepath{stroke}%
\end{pgfscope}%
\begin{pgfscope}%
\pgfpathrectangle{\pgfqpoint{2.125000in}{15.141860in}}{\pgfqpoint{5.489583in}{0.877907in}}%
\pgfusepath{clip}%
\pgfsetbuttcap%
\pgfsetroundjoin%
\pgfsetlinewidth{1.505625pt}%
\definecolor{currentstroke}{rgb}{0.000000,0.000000,0.000000}%
\pgfsetstrokecolor{currentstroke}%
\pgfsetdash{}{0pt}%
\pgfpathmoveto{\pgfqpoint{6.687330in}{15.412639in}}%
\pgfpathlineto{\pgfqpoint{6.687330in}{15.887721in}}%
\pgfusepath{stroke}%
\end{pgfscope}%
\begin{pgfscope}%
\pgfpathrectangle{\pgfqpoint{2.125000in}{15.141860in}}{\pgfqpoint{5.489583in}{0.877907in}}%
\pgfusepath{clip}%
\pgfsetbuttcap%
\pgfsetroundjoin%
\pgfsetlinewidth{1.505625pt}%
\definecolor{currentstroke}{rgb}{0.000000,0.000000,0.000000}%
\pgfsetstrokecolor{currentstroke}%
\pgfsetdash{}{0pt}%
\pgfpathmoveto{\pgfqpoint{6.810553in}{15.412639in}}%
\pgfpathlineto{\pgfqpoint{6.810553in}{15.885132in}}%
\pgfusepath{stroke}%
\end{pgfscope}%
\begin{pgfscope}%
\pgfpathrectangle{\pgfqpoint{2.125000in}{15.141860in}}{\pgfqpoint{5.489583in}{0.877907in}}%
\pgfusepath{clip}%
\pgfsetbuttcap%
\pgfsetroundjoin%
\pgfsetlinewidth{1.505625pt}%
\definecolor{currentstroke}{rgb}{0.000000,0.000000,0.000000}%
\pgfsetstrokecolor{currentstroke}%
\pgfsetdash{}{0pt}%
\pgfpathmoveto{\pgfqpoint{6.933776in}{15.412639in}}%
\pgfpathlineto{\pgfqpoint{6.933776in}{15.882594in}}%
\pgfusepath{stroke}%
\end{pgfscope}%
\begin{pgfscope}%
\pgfpathrectangle{\pgfqpoint{2.125000in}{15.141860in}}{\pgfqpoint{5.489583in}{0.877907in}}%
\pgfusepath{clip}%
\pgfsetbuttcap%
\pgfsetroundjoin%
\pgfsetlinewidth{1.505625pt}%
\definecolor{currentstroke}{rgb}{0.000000,0.000000,0.000000}%
\pgfsetstrokecolor{currentstroke}%
\pgfsetdash{}{0pt}%
\pgfpathmoveto{\pgfqpoint{7.056999in}{15.412639in}}%
\pgfpathlineto{\pgfqpoint{7.056999in}{15.880217in}}%
\pgfusepath{stroke}%
\end{pgfscope}%
\begin{pgfscope}%
\pgfpathrectangle{\pgfqpoint{2.125000in}{15.141860in}}{\pgfqpoint{5.489583in}{0.877907in}}%
\pgfusepath{clip}%
\pgfsetbuttcap%
\pgfsetroundjoin%
\pgfsetlinewidth{1.505625pt}%
\definecolor{currentstroke}{rgb}{0.000000,0.000000,0.000000}%
\pgfsetstrokecolor{currentstroke}%
\pgfsetdash{}{0pt}%
\pgfpathmoveto{\pgfqpoint{7.180222in}{15.412639in}}%
\pgfpathlineto{\pgfqpoint{7.180222in}{15.877894in}}%
\pgfusepath{stroke}%
\end{pgfscope}%
\begin{pgfscope}%
\pgfpathrectangle{\pgfqpoint{2.125000in}{15.141860in}}{\pgfqpoint{5.489583in}{0.877907in}}%
\pgfusepath{clip}%
\pgfsetbuttcap%
\pgfsetroundjoin%
\pgfsetlinewidth{1.505625pt}%
\definecolor{currentstroke}{rgb}{0.000000,0.000000,0.000000}%
\pgfsetstrokecolor{currentstroke}%
\pgfsetdash{}{0pt}%
\pgfpathmoveto{\pgfqpoint{7.303445in}{15.412639in}}%
\pgfpathlineto{\pgfqpoint{7.303445in}{15.875623in}}%
\pgfusepath{stroke}%
\end{pgfscope}%
\begin{pgfscope}%
\pgfpathrectangle{\pgfqpoint{2.125000in}{15.141860in}}{\pgfqpoint{5.489583in}{0.877907in}}%
\pgfusepath{clip}%
\pgfsetroundcap%
\pgfsetroundjoin%
\pgfsetlinewidth{1.505625pt}%
\definecolor{currentstroke}{rgb}{0.121569,0.466667,0.705882}%
\pgfsetstrokecolor{currentstroke}%
\pgfsetdash{}{0pt}%
\pgfpathmoveto{\pgfqpoint{2.125000in}{15.412639in}}%
\pgfpathlineto{\pgfqpoint{7.614583in}{15.412639in}}%
\pgfusepath{stroke}%
\end{pgfscope}%
\begin{pgfscope}%
\pgfpathrectangle{\pgfqpoint{2.125000in}{15.141860in}}{\pgfqpoint{5.489583in}{0.877907in}}%
\pgfusepath{clip}%
\pgfsetbuttcap%
\pgfsetroundjoin%
\definecolor{currentfill}{rgb}{0.121569,0.466667,0.705882}%
\pgfsetfillcolor{currentfill}%
\pgfsetlinewidth{1.003750pt}%
\definecolor{currentstroke}{rgb}{0.121569,0.466667,0.705882}%
\pgfsetstrokecolor{currentstroke}%
\pgfsetdash{}{0pt}%
\pgfsys@defobject{currentmarker}{\pgfqpoint{-0.034722in}{-0.034722in}}{\pgfqpoint{0.034722in}{0.034722in}}{%
\pgfpathmoveto{\pgfqpoint{0.000000in}{-0.034722in}}%
\pgfpathcurveto{\pgfqpoint{0.009208in}{-0.034722in}}{\pgfqpoint{0.018041in}{-0.031064in}}{\pgfqpoint{0.024552in}{-0.024552in}}%
\pgfpathcurveto{\pgfqpoint{0.031064in}{-0.018041in}}{\pgfqpoint{0.034722in}{-0.009208in}}{\pgfqpoint{0.034722in}{0.000000in}}%
\pgfpathcurveto{\pgfqpoint{0.034722in}{0.009208in}}{\pgfqpoint{0.031064in}{0.018041in}}{\pgfqpoint{0.024552in}{0.024552in}}%
\pgfpathcurveto{\pgfqpoint{0.018041in}{0.031064in}}{\pgfqpoint{0.009208in}{0.034722in}}{\pgfqpoint{0.000000in}{0.034722in}}%
\pgfpathcurveto{\pgfqpoint{-0.009208in}{0.034722in}}{\pgfqpoint{-0.018041in}{0.031064in}}{\pgfqpoint{-0.024552in}{0.024552in}}%
\pgfpathcurveto{\pgfqpoint{-0.031064in}{0.018041in}}{\pgfqpoint{-0.034722in}{0.009208in}}{\pgfqpoint{-0.034722in}{0.000000in}}%
\pgfpathcurveto{\pgfqpoint{-0.034722in}{-0.009208in}}{\pgfqpoint{-0.031064in}{-0.018041in}}{\pgfqpoint{-0.024552in}{-0.024552in}}%
\pgfpathcurveto{\pgfqpoint{-0.018041in}{-0.031064in}}{\pgfqpoint{-0.009208in}{-0.034722in}}{\pgfqpoint{0.000000in}{-0.034722in}}%
\pgfpathclose%
\pgfusepath{stroke,fill}%
}%
\begin{pgfscope}%
\pgfsys@transformshift{2.374527in}{15.979863in}%
\pgfsys@useobject{currentmarker}{}%
\end{pgfscope}%
\begin{pgfscope}%
\pgfsys@transformshift{2.497749in}{15.976931in}%
\pgfsys@useobject{currentmarker}{}%
\end{pgfscope}%
\begin{pgfscope}%
\pgfsys@transformshift{2.620972in}{15.973898in}%
\pgfsys@useobject{currentmarker}{}%
\end{pgfscope}%
\begin{pgfscope}%
\pgfsys@transformshift{2.744195in}{15.970976in}%
\pgfsys@useobject{currentmarker}{}%
\end{pgfscope}%
\begin{pgfscope}%
\pgfsys@transformshift{2.867418in}{15.968054in}%
\pgfsys@useobject{currentmarker}{}%
\end{pgfscope}%
\begin{pgfscope}%
\pgfsys@transformshift{2.990641in}{15.965165in}%
\pgfsys@useobject{currentmarker}{}%
\end{pgfscope}%
\begin{pgfscope}%
\pgfsys@transformshift{3.113864in}{15.962414in}%
\pgfsys@useobject{currentmarker}{}%
\end{pgfscope}%
\begin{pgfscope}%
\pgfsys@transformshift{3.237087in}{15.959647in}%
\pgfsys@useobject{currentmarker}{}%
\end{pgfscope}%
\begin{pgfscope}%
\pgfsys@transformshift{3.360310in}{15.956958in}%
\pgfsys@useobject{currentmarker}{}%
\end{pgfscope}%
\begin{pgfscope}%
\pgfsys@transformshift{3.483533in}{15.954179in}%
\pgfsys@useobject{currentmarker}{}%
\end{pgfscope}%
\begin{pgfscope}%
\pgfsys@transformshift{3.606756in}{15.951552in}%
\pgfsys@useobject{currentmarker}{}%
\end{pgfscope}%
\begin{pgfscope}%
\pgfsys@transformshift{3.729979in}{15.949071in}%
\pgfsys@useobject{currentmarker}{}%
\end{pgfscope}%
\begin{pgfscope}%
\pgfsys@transformshift{3.853202in}{15.946640in}%
\pgfsys@useobject{currentmarker}{}%
\end{pgfscope}%
\begin{pgfscope}%
\pgfsys@transformshift{3.976425in}{15.944043in}%
\pgfsys@useobject{currentmarker}{}%
\end{pgfscope}%
\begin{pgfscope}%
\pgfsys@transformshift{4.099648in}{15.941321in}%
\pgfsys@useobject{currentmarker}{}%
\end{pgfscope}%
\begin{pgfscope}%
\pgfsys@transformshift{4.222871in}{15.938536in}%
\pgfsys@useobject{currentmarker}{}%
\end{pgfscope}%
\begin{pgfscope}%
\pgfsys@transformshift{4.346094in}{15.935889in}%
\pgfsys@useobject{currentmarker}{}%
\end{pgfscope}%
\begin{pgfscope}%
\pgfsys@transformshift{4.469317in}{15.933211in}%
\pgfsys@useobject{currentmarker}{}%
\end{pgfscope}%
\begin{pgfscope}%
\pgfsys@transformshift{4.592540in}{15.930379in}%
\pgfsys@useobject{currentmarker}{}%
\end{pgfscope}%
\begin{pgfscope}%
\pgfsys@transformshift{4.715763in}{15.927532in}%
\pgfsys@useobject{currentmarker}{}%
\end{pgfscope}%
\begin{pgfscope}%
\pgfsys@transformshift{4.838986in}{15.924879in}%
\pgfsys@useobject{currentmarker}{}%
\end{pgfscope}%
\begin{pgfscope}%
\pgfsys@transformshift{4.962209in}{15.922206in}%
\pgfsys@useobject{currentmarker}{}%
\end{pgfscope}%
\begin{pgfscope}%
\pgfsys@transformshift{5.085432in}{15.919575in}%
\pgfsys@useobject{currentmarker}{}%
\end{pgfscope}%
\begin{pgfscope}%
\pgfsys@transformshift{5.208655in}{15.917107in}%
\pgfsys@useobject{currentmarker}{}%
\end{pgfscope}%
\begin{pgfscope}%
\pgfsys@transformshift{5.331878in}{15.914701in}%
\pgfsys@useobject{currentmarker}{}%
\end{pgfscope}%
\begin{pgfscope}%
\pgfsys@transformshift{5.455101in}{15.912328in}%
\pgfsys@useobject{currentmarker}{}%
\end{pgfscope}%
\begin{pgfscope}%
\pgfsys@transformshift{5.578324in}{15.909927in}%
\pgfsys@useobject{currentmarker}{}%
\end{pgfscope}%
\begin{pgfscope}%
\pgfsys@transformshift{5.701547in}{15.907513in}%
\pgfsys@useobject{currentmarker}{}%
\end{pgfscope}%
\begin{pgfscope}%
\pgfsys@transformshift{5.824770in}{15.905015in}%
\pgfsys@useobject{currentmarker}{}%
\end{pgfscope}%
\begin{pgfscope}%
\pgfsys@transformshift{5.947993in}{15.902482in}%
\pgfsys@useobject{currentmarker}{}%
\end{pgfscope}%
\begin{pgfscope}%
\pgfsys@transformshift{6.071216in}{15.899967in}%
\pgfsys@useobject{currentmarker}{}%
\end{pgfscope}%
\begin{pgfscope}%
\pgfsys@transformshift{6.194439in}{15.897447in}%
\pgfsys@useobject{currentmarker}{}%
\end{pgfscope}%
\begin{pgfscope}%
\pgfsys@transformshift{6.317662in}{15.895061in}%
\pgfsys@useobject{currentmarker}{}%
\end{pgfscope}%
\begin{pgfscope}%
\pgfsys@transformshift{6.440885in}{15.892727in}%
\pgfsys@useobject{currentmarker}{}%
\end{pgfscope}%
\begin{pgfscope}%
\pgfsys@transformshift{6.564108in}{15.890261in}%
\pgfsys@useobject{currentmarker}{}%
\end{pgfscope}%
\begin{pgfscope}%
\pgfsys@transformshift{6.687330in}{15.887721in}%
\pgfsys@useobject{currentmarker}{}%
\end{pgfscope}%
\begin{pgfscope}%
\pgfsys@transformshift{6.810553in}{15.885132in}%
\pgfsys@useobject{currentmarker}{}%
\end{pgfscope}%
\begin{pgfscope}%
\pgfsys@transformshift{6.933776in}{15.882594in}%
\pgfsys@useobject{currentmarker}{}%
\end{pgfscope}%
\begin{pgfscope}%
\pgfsys@transformshift{7.056999in}{15.880217in}%
\pgfsys@useobject{currentmarker}{}%
\end{pgfscope}%
\begin{pgfscope}%
\pgfsys@transformshift{7.180222in}{15.877894in}%
\pgfsys@useobject{currentmarker}{}%
\end{pgfscope}%
\begin{pgfscope}%
\pgfsys@transformshift{7.303445in}{15.875623in}%
\pgfsys@useobject{currentmarker}{}%
\end{pgfscope}%
\end{pgfscope}%
\begin{pgfscope}%
\pgfsetrectcap%
\pgfsetmiterjoin%
\pgfsetlinewidth{0.803000pt}%
\definecolor{currentstroke}{rgb}{1.000000,1.000000,1.000000}%
\pgfsetstrokecolor{currentstroke}%
\pgfsetdash{}{0pt}%
\pgfpathmoveto{\pgfqpoint{2.125000in}{15.141860in}}%
\pgfpathlineto{\pgfqpoint{2.125000in}{16.019767in}}%
\pgfusepath{stroke}%
\end{pgfscope}%
\begin{pgfscope}%
\pgfsetrectcap%
\pgfsetmiterjoin%
\pgfsetlinewidth{0.803000pt}%
\definecolor{currentstroke}{rgb}{1.000000,1.000000,1.000000}%
\pgfsetstrokecolor{currentstroke}%
\pgfsetdash{}{0pt}%
\pgfpathmoveto{\pgfqpoint{7.614583in}{15.141860in}}%
\pgfpathlineto{\pgfqpoint{7.614583in}{16.019767in}}%
\pgfusepath{stroke}%
\end{pgfscope}%
\begin{pgfscope}%
\pgfsetrectcap%
\pgfsetmiterjoin%
\pgfsetlinewidth{0.803000pt}%
\definecolor{currentstroke}{rgb}{1.000000,1.000000,1.000000}%
\pgfsetstrokecolor{currentstroke}%
\pgfsetdash{}{0pt}%
\pgfpathmoveto{\pgfqpoint{2.125000in}{15.141860in}}%
\pgfpathlineto{\pgfqpoint{7.614583in}{15.141860in}}%
\pgfusepath{stroke}%
\end{pgfscope}%
\begin{pgfscope}%
\pgfsetrectcap%
\pgfsetmiterjoin%
\pgfsetlinewidth{0.803000pt}%
\definecolor{currentstroke}{rgb}{1.000000,1.000000,1.000000}%
\pgfsetstrokecolor{currentstroke}%
\pgfsetdash{}{0pt}%
\pgfpathmoveto{\pgfqpoint{2.125000in}{16.019767in}}%
\pgfpathlineto{\pgfqpoint{7.614583in}{16.019767in}}%
\pgfusepath{stroke}%
\end{pgfscope}%
\begin{pgfscope}%
\definecolor{textcolor}{rgb}{0.150000,0.150000,0.150000}%
\pgfsetstrokecolor{textcolor}%
\pgfsetfillcolor{textcolor}%
\pgftext[x=4.869792in,y=16.103101in,,base]{\color{textcolor}\rmfamily\fontsize{16.800000}{20.160000}\selectfont Autocorrelation}%
\end{pgfscope}%
\begin{pgfscope}%
\pgfsetbuttcap%
\pgfsetmiterjoin%
\definecolor{currentfill}{rgb}{0.917647,0.917647,0.949020}%
\pgfsetfillcolor{currentfill}%
\pgfsetlinewidth{0.000000pt}%
\definecolor{currentstroke}{rgb}{0.000000,0.000000,0.000000}%
\pgfsetstrokecolor{currentstroke}%
\pgfsetstrokeopacity{0.000000}%
\pgfsetdash{}{0pt}%
\pgfpathmoveto{\pgfqpoint{9.810417in}{15.141860in}}%
\pgfpathlineto{\pgfqpoint{15.300000in}{15.141860in}}%
\pgfpathlineto{\pgfqpoint{15.300000in}{16.019767in}}%
\pgfpathlineto{\pgfqpoint{9.810417in}{16.019767in}}%
\pgfpathclose%
\pgfusepath{fill}%
\end{pgfscope}%
\begin{pgfscope}%
\pgfpathrectangle{\pgfqpoint{9.810417in}{15.141860in}}{\pgfqpoint{5.489583in}{0.877907in}}%
\pgfusepath{clip}%
\pgfsetroundcap%
\pgfsetroundjoin%
\pgfsetlinewidth{0.803000pt}%
\definecolor{currentstroke}{rgb}{1.000000,1.000000,1.000000}%
\pgfsetstrokecolor{currentstroke}%
\pgfsetdash{}{0pt}%
\pgfpathmoveto{\pgfqpoint{10.059943in}{15.141860in}}%
\pgfpathlineto{\pgfqpoint{10.059943in}{16.019767in}}%
\pgfusepath{stroke}%
\end{pgfscope}%
\begin{pgfscope}%
\definecolor{textcolor}{rgb}{0.150000,0.150000,0.150000}%
\pgfsetstrokecolor{textcolor}%
\pgfsetfillcolor{textcolor}%
\pgftext[x=10.059943in,y=15.044638in,,top]{\color{textcolor}\rmfamily\fontsize{14.000000}{16.800000}\selectfont 0}%
\end{pgfscope}%
\begin{pgfscope}%
\pgfpathrectangle{\pgfqpoint{9.810417in}{15.141860in}}{\pgfqpoint{5.489583in}{0.877907in}}%
\pgfusepath{clip}%
\pgfsetroundcap%
\pgfsetroundjoin%
\pgfsetlinewidth{0.803000pt}%
\definecolor{currentstroke}{rgb}{1.000000,1.000000,1.000000}%
\pgfsetstrokecolor{currentstroke}%
\pgfsetdash{}{0pt}%
\pgfpathmoveto{\pgfqpoint{10.676058in}{15.141860in}}%
\pgfpathlineto{\pgfqpoint{10.676058in}{16.019767in}}%
\pgfusepath{stroke}%
\end{pgfscope}%
\begin{pgfscope}%
\definecolor{textcolor}{rgb}{0.150000,0.150000,0.150000}%
\pgfsetstrokecolor{textcolor}%
\pgfsetfillcolor{textcolor}%
\pgftext[x=10.676058in,y=15.044638in,,top]{\color{textcolor}\rmfamily\fontsize{14.000000}{16.800000}\selectfont 5}%
\end{pgfscope}%
\begin{pgfscope}%
\pgfpathrectangle{\pgfqpoint{9.810417in}{15.141860in}}{\pgfqpoint{5.489583in}{0.877907in}}%
\pgfusepath{clip}%
\pgfsetroundcap%
\pgfsetroundjoin%
\pgfsetlinewidth{0.803000pt}%
\definecolor{currentstroke}{rgb}{1.000000,1.000000,1.000000}%
\pgfsetstrokecolor{currentstroke}%
\pgfsetdash{}{0pt}%
\pgfpathmoveto{\pgfqpoint{11.292173in}{15.141860in}}%
\pgfpathlineto{\pgfqpoint{11.292173in}{16.019767in}}%
\pgfusepath{stroke}%
\end{pgfscope}%
\begin{pgfscope}%
\definecolor{textcolor}{rgb}{0.150000,0.150000,0.150000}%
\pgfsetstrokecolor{textcolor}%
\pgfsetfillcolor{textcolor}%
\pgftext[x=11.292173in,y=15.044638in,,top]{\color{textcolor}\rmfamily\fontsize{14.000000}{16.800000}\selectfont 10}%
\end{pgfscope}%
\begin{pgfscope}%
\pgfpathrectangle{\pgfqpoint{9.810417in}{15.141860in}}{\pgfqpoint{5.489583in}{0.877907in}}%
\pgfusepath{clip}%
\pgfsetroundcap%
\pgfsetroundjoin%
\pgfsetlinewidth{0.803000pt}%
\definecolor{currentstroke}{rgb}{1.000000,1.000000,1.000000}%
\pgfsetstrokecolor{currentstroke}%
\pgfsetdash{}{0pt}%
\pgfpathmoveto{\pgfqpoint{11.908288in}{15.141860in}}%
\pgfpathlineto{\pgfqpoint{11.908288in}{16.019767in}}%
\pgfusepath{stroke}%
\end{pgfscope}%
\begin{pgfscope}%
\definecolor{textcolor}{rgb}{0.150000,0.150000,0.150000}%
\pgfsetstrokecolor{textcolor}%
\pgfsetfillcolor{textcolor}%
\pgftext[x=11.908288in,y=15.044638in,,top]{\color{textcolor}\rmfamily\fontsize{14.000000}{16.800000}\selectfont 15}%
\end{pgfscope}%
\begin{pgfscope}%
\pgfpathrectangle{\pgfqpoint{9.810417in}{15.141860in}}{\pgfqpoint{5.489583in}{0.877907in}}%
\pgfusepath{clip}%
\pgfsetroundcap%
\pgfsetroundjoin%
\pgfsetlinewidth{0.803000pt}%
\definecolor{currentstroke}{rgb}{1.000000,1.000000,1.000000}%
\pgfsetstrokecolor{currentstroke}%
\pgfsetdash{}{0pt}%
\pgfpathmoveto{\pgfqpoint{12.524403in}{15.141860in}}%
\pgfpathlineto{\pgfqpoint{12.524403in}{16.019767in}}%
\pgfusepath{stroke}%
\end{pgfscope}%
\begin{pgfscope}%
\definecolor{textcolor}{rgb}{0.150000,0.150000,0.150000}%
\pgfsetstrokecolor{textcolor}%
\pgfsetfillcolor{textcolor}%
\pgftext[x=12.524403in,y=15.044638in,,top]{\color{textcolor}\rmfamily\fontsize{14.000000}{16.800000}\selectfont 20}%
\end{pgfscope}%
\begin{pgfscope}%
\pgfpathrectangle{\pgfqpoint{9.810417in}{15.141860in}}{\pgfqpoint{5.489583in}{0.877907in}}%
\pgfusepath{clip}%
\pgfsetroundcap%
\pgfsetroundjoin%
\pgfsetlinewidth{0.803000pt}%
\definecolor{currentstroke}{rgb}{1.000000,1.000000,1.000000}%
\pgfsetstrokecolor{currentstroke}%
\pgfsetdash{}{0pt}%
\pgfpathmoveto{\pgfqpoint{13.140517in}{15.141860in}}%
\pgfpathlineto{\pgfqpoint{13.140517in}{16.019767in}}%
\pgfusepath{stroke}%
\end{pgfscope}%
\begin{pgfscope}%
\definecolor{textcolor}{rgb}{0.150000,0.150000,0.150000}%
\pgfsetstrokecolor{textcolor}%
\pgfsetfillcolor{textcolor}%
\pgftext[x=13.140517in,y=15.044638in,,top]{\color{textcolor}\rmfamily\fontsize{14.000000}{16.800000}\selectfont 25}%
\end{pgfscope}%
\begin{pgfscope}%
\pgfpathrectangle{\pgfqpoint{9.810417in}{15.141860in}}{\pgfqpoint{5.489583in}{0.877907in}}%
\pgfusepath{clip}%
\pgfsetroundcap%
\pgfsetroundjoin%
\pgfsetlinewidth{0.803000pt}%
\definecolor{currentstroke}{rgb}{1.000000,1.000000,1.000000}%
\pgfsetstrokecolor{currentstroke}%
\pgfsetdash{}{0pt}%
\pgfpathmoveto{\pgfqpoint{13.756632in}{15.141860in}}%
\pgfpathlineto{\pgfqpoint{13.756632in}{16.019767in}}%
\pgfusepath{stroke}%
\end{pgfscope}%
\begin{pgfscope}%
\definecolor{textcolor}{rgb}{0.150000,0.150000,0.150000}%
\pgfsetstrokecolor{textcolor}%
\pgfsetfillcolor{textcolor}%
\pgftext[x=13.756632in,y=15.044638in,,top]{\color{textcolor}\rmfamily\fontsize{14.000000}{16.800000}\selectfont 30}%
\end{pgfscope}%
\begin{pgfscope}%
\pgfpathrectangle{\pgfqpoint{9.810417in}{15.141860in}}{\pgfqpoint{5.489583in}{0.877907in}}%
\pgfusepath{clip}%
\pgfsetroundcap%
\pgfsetroundjoin%
\pgfsetlinewidth{0.803000pt}%
\definecolor{currentstroke}{rgb}{1.000000,1.000000,1.000000}%
\pgfsetstrokecolor{currentstroke}%
\pgfsetdash{}{0pt}%
\pgfpathmoveto{\pgfqpoint{14.372747in}{15.141860in}}%
\pgfpathlineto{\pgfqpoint{14.372747in}{16.019767in}}%
\pgfusepath{stroke}%
\end{pgfscope}%
\begin{pgfscope}%
\definecolor{textcolor}{rgb}{0.150000,0.150000,0.150000}%
\pgfsetstrokecolor{textcolor}%
\pgfsetfillcolor{textcolor}%
\pgftext[x=14.372747in,y=15.044638in,,top]{\color{textcolor}\rmfamily\fontsize{14.000000}{16.800000}\selectfont 35}%
\end{pgfscope}%
\begin{pgfscope}%
\pgfpathrectangle{\pgfqpoint{9.810417in}{15.141860in}}{\pgfqpoint{5.489583in}{0.877907in}}%
\pgfusepath{clip}%
\pgfsetroundcap%
\pgfsetroundjoin%
\pgfsetlinewidth{0.803000pt}%
\definecolor{currentstroke}{rgb}{1.000000,1.000000,1.000000}%
\pgfsetstrokecolor{currentstroke}%
\pgfsetdash{}{0pt}%
\pgfpathmoveto{\pgfqpoint{14.988862in}{15.141860in}}%
\pgfpathlineto{\pgfqpoint{14.988862in}{16.019767in}}%
\pgfusepath{stroke}%
\end{pgfscope}%
\begin{pgfscope}%
\definecolor{textcolor}{rgb}{0.150000,0.150000,0.150000}%
\pgfsetstrokecolor{textcolor}%
\pgfsetfillcolor{textcolor}%
\pgftext[x=14.988862in,y=15.044638in,,top]{\color{textcolor}\rmfamily\fontsize{14.000000}{16.800000}\selectfont 40}%
\end{pgfscope}%
\begin{pgfscope}%
\pgfpathrectangle{\pgfqpoint{9.810417in}{15.141860in}}{\pgfqpoint{5.489583in}{0.877907in}}%
\pgfusepath{clip}%
\pgfsetroundcap%
\pgfsetroundjoin%
\pgfsetlinewidth{0.803000pt}%
\definecolor{currentstroke}{rgb}{1.000000,1.000000,1.000000}%
\pgfsetstrokecolor{currentstroke}%
\pgfsetdash{}{0pt}%
\pgfpathmoveto{\pgfqpoint{9.810417in}{15.220099in}}%
\pgfpathlineto{\pgfqpoint{15.300000in}{15.220099in}}%
\pgfusepath{stroke}%
\end{pgfscope}%
\begin{pgfscope}%
\definecolor{textcolor}{rgb}{0.150000,0.150000,0.150000}%
\pgfsetstrokecolor{textcolor}%
\pgfsetfillcolor{textcolor}%
\pgftext[x=9.589483in,y=15.146233in,left,base]{\color{textcolor}\rmfamily\fontsize{14.000000}{16.800000}\selectfont 0}%
\end{pgfscope}%
\begin{pgfscope}%
\pgfpathrectangle{\pgfqpoint{9.810417in}{15.141860in}}{\pgfqpoint{5.489583in}{0.877907in}}%
\pgfusepath{clip}%
\pgfsetroundcap%
\pgfsetroundjoin%
\pgfsetlinewidth{0.803000pt}%
\definecolor{currentstroke}{rgb}{1.000000,1.000000,1.000000}%
\pgfsetstrokecolor{currentstroke}%
\pgfsetdash{}{0pt}%
\pgfpathmoveto{\pgfqpoint{9.810417in}{15.979863in}}%
\pgfpathlineto{\pgfqpoint{15.300000in}{15.979863in}}%
\pgfusepath{stroke}%
\end{pgfscope}%
\begin{pgfscope}%
\definecolor{textcolor}{rgb}{0.150000,0.150000,0.150000}%
\pgfsetstrokecolor{textcolor}%
\pgfsetfillcolor{textcolor}%
\pgftext[x=9.589483in,y=15.905996in,left,base]{\color{textcolor}\rmfamily\fontsize{14.000000}{16.800000}\selectfont 1}%
\end{pgfscope}%
\begin{pgfscope}%
\pgfpathrectangle{\pgfqpoint{9.810417in}{15.141860in}}{\pgfqpoint{5.489583in}{0.877907in}}%
\pgfusepath{clip}%
\pgfsetbuttcap%
\pgfsetroundjoin%
\definecolor{currentfill}{rgb}{0.121569,0.466667,0.705882}%
\pgfsetfillcolor{currentfill}%
\pgfsetfillopacity{0.250000}%
\pgfsetlinewidth{1.003750pt}%
\definecolor{currentstroke}{rgb}{1.000000,1.000000,1.000000}%
\pgfsetstrokecolor{currentstroke}%
\pgfsetstrokeopacity{0.250000}%
\pgfsetdash{}{0pt}%
\pgfpathmoveto{\pgfqpoint{10.121555in}{15.258433in}}%
\pgfpathlineto{\pgfqpoint{10.121555in}{15.181765in}}%
\pgfpathlineto{\pgfqpoint{10.306389in}{15.181765in}}%
\pgfpathlineto{\pgfqpoint{10.429612in}{15.181765in}}%
\pgfpathlineto{\pgfqpoint{10.552835in}{15.181765in}}%
\pgfpathlineto{\pgfqpoint{10.676058in}{15.181765in}}%
\pgfpathlineto{\pgfqpoint{10.799281in}{15.181765in}}%
\pgfpathlineto{\pgfqpoint{10.922504in}{15.181765in}}%
\pgfpathlineto{\pgfqpoint{11.045727in}{15.181765in}}%
\pgfpathlineto{\pgfqpoint{11.168950in}{15.181765in}}%
\pgfpathlineto{\pgfqpoint{11.292173in}{15.181765in}}%
\pgfpathlineto{\pgfqpoint{11.415396in}{15.181765in}}%
\pgfpathlineto{\pgfqpoint{11.538619in}{15.181765in}}%
\pgfpathlineto{\pgfqpoint{11.661842in}{15.181765in}}%
\pgfpathlineto{\pgfqpoint{11.785065in}{15.181765in}}%
\pgfpathlineto{\pgfqpoint{11.908288in}{15.181765in}}%
\pgfpathlineto{\pgfqpoint{12.031511in}{15.181765in}}%
\pgfpathlineto{\pgfqpoint{12.154734in}{15.181765in}}%
\pgfpathlineto{\pgfqpoint{12.277957in}{15.181765in}}%
\pgfpathlineto{\pgfqpoint{12.401180in}{15.181765in}}%
\pgfpathlineto{\pgfqpoint{12.524403in}{15.181765in}}%
\pgfpathlineto{\pgfqpoint{12.647626in}{15.181765in}}%
\pgfpathlineto{\pgfqpoint{12.770849in}{15.181765in}}%
\pgfpathlineto{\pgfqpoint{12.894072in}{15.181765in}}%
\pgfpathlineto{\pgfqpoint{13.017294in}{15.181765in}}%
\pgfpathlineto{\pgfqpoint{13.140517in}{15.181765in}}%
\pgfpathlineto{\pgfqpoint{13.263740in}{15.181765in}}%
\pgfpathlineto{\pgfqpoint{13.386963in}{15.181765in}}%
\pgfpathlineto{\pgfqpoint{13.510186in}{15.181765in}}%
\pgfpathlineto{\pgfqpoint{13.633409in}{15.181765in}}%
\pgfpathlineto{\pgfqpoint{13.756632in}{15.181765in}}%
\pgfpathlineto{\pgfqpoint{13.879855in}{15.181765in}}%
\pgfpathlineto{\pgfqpoint{14.003078in}{15.181765in}}%
\pgfpathlineto{\pgfqpoint{14.126301in}{15.181765in}}%
\pgfpathlineto{\pgfqpoint{14.249524in}{15.181765in}}%
\pgfpathlineto{\pgfqpoint{14.372747in}{15.181765in}}%
\pgfpathlineto{\pgfqpoint{14.495970in}{15.181765in}}%
\pgfpathlineto{\pgfqpoint{14.619193in}{15.181765in}}%
\pgfpathlineto{\pgfqpoint{14.742416in}{15.181765in}}%
\pgfpathlineto{\pgfqpoint{14.865639in}{15.181765in}}%
\pgfpathlineto{\pgfqpoint{15.050473in}{15.181765in}}%
\pgfpathlineto{\pgfqpoint{15.050473in}{15.258433in}}%
\pgfpathlineto{\pgfqpoint{15.050473in}{15.258433in}}%
\pgfpathlineto{\pgfqpoint{14.865639in}{15.258433in}}%
\pgfpathlineto{\pgfqpoint{14.742416in}{15.258433in}}%
\pgfpathlineto{\pgfqpoint{14.619193in}{15.258433in}}%
\pgfpathlineto{\pgfqpoint{14.495970in}{15.258433in}}%
\pgfpathlineto{\pgfqpoint{14.372747in}{15.258433in}}%
\pgfpathlineto{\pgfqpoint{14.249524in}{15.258433in}}%
\pgfpathlineto{\pgfqpoint{14.126301in}{15.258433in}}%
\pgfpathlineto{\pgfqpoint{14.003078in}{15.258433in}}%
\pgfpathlineto{\pgfqpoint{13.879855in}{15.258433in}}%
\pgfpathlineto{\pgfqpoint{13.756632in}{15.258433in}}%
\pgfpathlineto{\pgfqpoint{13.633409in}{15.258433in}}%
\pgfpathlineto{\pgfqpoint{13.510186in}{15.258433in}}%
\pgfpathlineto{\pgfqpoint{13.386963in}{15.258433in}}%
\pgfpathlineto{\pgfqpoint{13.263740in}{15.258433in}}%
\pgfpathlineto{\pgfqpoint{13.140517in}{15.258433in}}%
\pgfpathlineto{\pgfqpoint{13.017294in}{15.258433in}}%
\pgfpathlineto{\pgfqpoint{12.894072in}{15.258433in}}%
\pgfpathlineto{\pgfqpoint{12.770849in}{15.258433in}}%
\pgfpathlineto{\pgfqpoint{12.647626in}{15.258433in}}%
\pgfpathlineto{\pgfqpoint{12.524403in}{15.258433in}}%
\pgfpathlineto{\pgfqpoint{12.401180in}{15.258433in}}%
\pgfpathlineto{\pgfqpoint{12.277957in}{15.258433in}}%
\pgfpathlineto{\pgfqpoint{12.154734in}{15.258433in}}%
\pgfpathlineto{\pgfqpoint{12.031511in}{15.258433in}}%
\pgfpathlineto{\pgfqpoint{11.908288in}{15.258433in}}%
\pgfpathlineto{\pgfqpoint{11.785065in}{15.258433in}}%
\pgfpathlineto{\pgfqpoint{11.661842in}{15.258433in}}%
\pgfpathlineto{\pgfqpoint{11.538619in}{15.258433in}}%
\pgfpathlineto{\pgfqpoint{11.415396in}{15.258433in}}%
\pgfpathlineto{\pgfqpoint{11.292173in}{15.258433in}}%
\pgfpathlineto{\pgfqpoint{11.168950in}{15.258433in}}%
\pgfpathlineto{\pgfqpoint{11.045727in}{15.258433in}}%
\pgfpathlineto{\pgfqpoint{10.922504in}{15.258433in}}%
\pgfpathlineto{\pgfqpoint{10.799281in}{15.258433in}}%
\pgfpathlineto{\pgfqpoint{10.676058in}{15.258433in}}%
\pgfpathlineto{\pgfqpoint{10.552835in}{15.258433in}}%
\pgfpathlineto{\pgfqpoint{10.429612in}{15.258433in}}%
\pgfpathlineto{\pgfqpoint{10.306389in}{15.258433in}}%
\pgfpathlineto{\pgfqpoint{10.121555in}{15.258433in}}%
\pgfpathclose%
\pgfusepath{stroke,fill}%
\end{pgfscope}%
\begin{pgfscope}%
\pgfpathrectangle{\pgfqpoint{9.810417in}{15.141860in}}{\pgfqpoint{5.489583in}{0.877907in}}%
\pgfusepath{clip}%
\pgfsetbuttcap%
\pgfsetroundjoin%
\pgfsetlinewidth{1.505625pt}%
\definecolor{currentstroke}{rgb}{0.000000,0.000000,0.000000}%
\pgfsetstrokecolor{currentstroke}%
\pgfsetdash{}{0pt}%
\pgfpathmoveto{\pgfqpoint{10.059943in}{15.220099in}}%
\pgfpathlineto{\pgfqpoint{10.059943in}{15.979863in}}%
\pgfusepath{stroke}%
\end{pgfscope}%
\begin{pgfscope}%
\pgfpathrectangle{\pgfqpoint{9.810417in}{15.141860in}}{\pgfqpoint{5.489583in}{0.877907in}}%
\pgfusepath{clip}%
\pgfsetbuttcap%
\pgfsetroundjoin%
\pgfsetlinewidth{1.505625pt}%
\definecolor{currentstroke}{rgb}{0.000000,0.000000,0.000000}%
\pgfsetstrokecolor{currentstroke}%
\pgfsetdash{}{0pt}%
\pgfpathmoveto{\pgfqpoint{10.183166in}{15.220099in}}%
\pgfpathlineto{\pgfqpoint{10.183166in}{15.976437in}}%
\pgfusepath{stroke}%
\end{pgfscope}%
\begin{pgfscope}%
\pgfpathrectangle{\pgfqpoint{9.810417in}{15.141860in}}{\pgfqpoint{5.489583in}{0.877907in}}%
\pgfusepath{clip}%
\pgfsetbuttcap%
\pgfsetroundjoin%
\pgfsetlinewidth{1.505625pt}%
\definecolor{currentstroke}{rgb}{0.000000,0.000000,0.000000}%
\pgfsetstrokecolor{currentstroke}%
\pgfsetdash{}{0pt}%
\pgfpathmoveto{\pgfqpoint{10.306389in}{15.220099in}}%
\pgfpathlineto{\pgfqpoint{10.306389in}{15.202667in}}%
\pgfusepath{stroke}%
\end{pgfscope}%
\begin{pgfscope}%
\pgfpathrectangle{\pgfqpoint{9.810417in}{15.141860in}}{\pgfqpoint{5.489583in}{0.877907in}}%
\pgfusepath{clip}%
\pgfsetbuttcap%
\pgfsetroundjoin%
\pgfsetlinewidth{1.505625pt}%
\definecolor{currentstroke}{rgb}{0.000000,0.000000,0.000000}%
\pgfsetstrokecolor{currentstroke}%
\pgfsetdash{}{0pt}%
\pgfpathmoveto{\pgfqpoint{10.429612in}{15.220099in}}%
\pgfpathlineto{\pgfqpoint{10.429612in}{15.234885in}}%
\pgfusepath{stroke}%
\end{pgfscope}%
\begin{pgfscope}%
\pgfpathrectangle{\pgfqpoint{9.810417in}{15.141860in}}{\pgfqpoint{5.489583in}{0.877907in}}%
\pgfusepath{clip}%
\pgfsetbuttcap%
\pgfsetroundjoin%
\pgfsetlinewidth{1.505625pt}%
\definecolor{currentstroke}{rgb}{0.000000,0.000000,0.000000}%
\pgfsetstrokecolor{currentstroke}%
\pgfsetdash{}{0pt}%
\pgfpathmoveto{\pgfqpoint{10.552835in}{15.220099in}}%
\pgfpathlineto{\pgfqpoint{10.552835in}{15.216929in}}%
\pgfusepath{stroke}%
\end{pgfscope}%
\begin{pgfscope}%
\pgfpathrectangle{\pgfqpoint{9.810417in}{15.141860in}}{\pgfqpoint{5.489583in}{0.877907in}}%
\pgfusepath{clip}%
\pgfsetbuttcap%
\pgfsetroundjoin%
\pgfsetlinewidth{1.505625pt}%
\definecolor{currentstroke}{rgb}{0.000000,0.000000,0.000000}%
\pgfsetstrokecolor{currentstroke}%
\pgfsetdash{}{0pt}%
\pgfpathmoveto{\pgfqpoint{10.676058in}{15.220099in}}%
\pgfpathlineto{\pgfqpoint{10.676058in}{15.223376in}}%
\pgfusepath{stroke}%
\end{pgfscope}%
\begin{pgfscope}%
\pgfpathrectangle{\pgfqpoint{9.810417in}{15.141860in}}{\pgfqpoint{5.489583in}{0.877907in}}%
\pgfusepath{clip}%
\pgfsetbuttcap%
\pgfsetroundjoin%
\pgfsetlinewidth{1.505625pt}%
\definecolor{currentstroke}{rgb}{0.000000,0.000000,0.000000}%
\pgfsetstrokecolor{currentstroke}%
\pgfsetdash{}{0pt}%
\pgfpathmoveto{\pgfqpoint{10.799281in}{15.220099in}}%
\pgfpathlineto{\pgfqpoint{10.799281in}{15.238183in}}%
\pgfusepath{stroke}%
\end{pgfscope}%
\begin{pgfscope}%
\pgfpathrectangle{\pgfqpoint{9.810417in}{15.141860in}}{\pgfqpoint{5.489583in}{0.877907in}}%
\pgfusepath{clip}%
\pgfsetbuttcap%
\pgfsetroundjoin%
\pgfsetlinewidth{1.505625pt}%
\definecolor{currentstroke}{rgb}{0.000000,0.000000,0.000000}%
\pgfsetstrokecolor{currentstroke}%
\pgfsetdash{}{0pt}%
\pgfpathmoveto{\pgfqpoint{10.922504in}{15.220099in}}%
\pgfpathlineto{\pgfqpoint{10.922504in}{15.215006in}}%
\pgfusepath{stroke}%
\end{pgfscope}%
\begin{pgfscope}%
\pgfpathrectangle{\pgfqpoint{9.810417in}{15.141860in}}{\pgfqpoint{5.489583in}{0.877907in}}%
\pgfusepath{clip}%
\pgfsetbuttcap%
\pgfsetroundjoin%
\pgfsetlinewidth{1.505625pt}%
\definecolor{currentstroke}{rgb}{0.000000,0.000000,0.000000}%
\pgfsetstrokecolor{currentstroke}%
\pgfsetdash{}{0pt}%
\pgfpathmoveto{\pgfqpoint{11.045727in}{15.220099in}}%
\pgfpathlineto{\pgfqpoint{11.045727in}{15.230690in}}%
\pgfusepath{stroke}%
\end{pgfscope}%
\begin{pgfscope}%
\pgfpathrectangle{\pgfqpoint{9.810417in}{15.141860in}}{\pgfqpoint{5.489583in}{0.877907in}}%
\pgfusepath{clip}%
\pgfsetbuttcap%
\pgfsetroundjoin%
\pgfsetlinewidth{1.505625pt}%
\definecolor{currentstroke}{rgb}{0.000000,0.000000,0.000000}%
\pgfsetstrokecolor{currentstroke}%
\pgfsetdash{}{0pt}%
\pgfpathmoveto{\pgfqpoint{11.168950in}{15.220099in}}%
\pgfpathlineto{\pgfqpoint{11.168950in}{15.203849in}}%
\pgfusepath{stroke}%
\end{pgfscope}%
\begin{pgfscope}%
\pgfpathrectangle{\pgfqpoint{9.810417in}{15.141860in}}{\pgfqpoint{5.489583in}{0.877907in}}%
\pgfusepath{clip}%
\pgfsetbuttcap%
\pgfsetroundjoin%
\pgfsetlinewidth{1.505625pt}%
\definecolor{currentstroke}{rgb}{0.000000,0.000000,0.000000}%
\pgfsetstrokecolor{currentstroke}%
\pgfsetdash{}{0pt}%
\pgfpathmoveto{\pgfqpoint{11.292173in}{15.220099in}}%
\pgfpathlineto{\pgfqpoint{11.292173in}{15.242116in}}%
\pgfusepath{stroke}%
\end{pgfscope}%
\begin{pgfscope}%
\pgfpathrectangle{\pgfqpoint{9.810417in}{15.141860in}}{\pgfqpoint{5.489583in}{0.877907in}}%
\pgfusepath{clip}%
\pgfsetbuttcap%
\pgfsetroundjoin%
\pgfsetlinewidth{1.505625pt}%
\definecolor{currentstroke}{rgb}{0.000000,0.000000,0.000000}%
\pgfsetstrokecolor{currentstroke}%
\pgfsetdash{}{0pt}%
\pgfpathmoveto{\pgfqpoint{11.415396in}{15.220099in}}%
\pgfpathlineto{\pgfqpoint{11.415396in}{15.238844in}}%
\pgfusepath{stroke}%
\end{pgfscope}%
\begin{pgfscope}%
\pgfpathrectangle{\pgfqpoint{9.810417in}{15.141860in}}{\pgfqpoint{5.489583in}{0.877907in}}%
\pgfusepath{clip}%
\pgfsetbuttcap%
\pgfsetroundjoin%
\pgfsetlinewidth{1.505625pt}%
\definecolor{currentstroke}{rgb}{0.000000,0.000000,0.000000}%
\pgfsetstrokecolor{currentstroke}%
\pgfsetdash{}{0pt}%
\pgfpathmoveto{\pgfqpoint{11.538619in}{15.220099in}}%
\pgfpathlineto{\pgfqpoint{11.538619in}{15.225891in}}%
\pgfusepath{stroke}%
\end{pgfscope}%
\begin{pgfscope}%
\pgfpathrectangle{\pgfqpoint{9.810417in}{15.141860in}}{\pgfqpoint{5.489583in}{0.877907in}}%
\pgfusepath{clip}%
\pgfsetbuttcap%
\pgfsetroundjoin%
\pgfsetlinewidth{1.505625pt}%
\definecolor{currentstroke}{rgb}{0.000000,0.000000,0.000000}%
\pgfsetstrokecolor{currentstroke}%
\pgfsetdash{}{0pt}%
\pgfpathmoveto{\pgfqpoint{11.661842in}{15.220099in}}%
\pgfpathlineto{\pgfqpoint{11.661842in}{15.194351in}}%
\pgfusepath{stroke}%
\end{pgfscope}%
\begin{pgfscope}%
\pgfpathrectangle{\pgfqpoint{9.810417in}{15.141860in}}{\pgfqpoint{5.489583in}{0.877907in}}%
\pgfusepath{clip}%
\pgfsetbuttcap%
\pgfsetroundjoin%
\pgfsetlinewidth{1.505625pt}%
\definecolor{currentstroke}{rgb}{0.000000,0.000000,0.000000}%
\pgfsetstrokecolor{currentstroke}%
\pgfsetdash{}{0pt}%
\pgfpathmoveto{\pgfqpoint{11.785065in}{15.220099in}}%
\pgfpathlineto{\pgfqpoint{11.785065in}{15.199830in}}%
\pgfusepath{stroke}%
\end{pgfscope}%
\begin{pgfscope}%
\pgfpathrectangle{\pgfqpoint{9.810417in}{15.141860in}}{\pgfqpoint{5.489583in}{0.877907in}}%
\pgfusepath{clip}%
\pgfsetbuttcap%
\pgfsetroundjoin%
\pgfsetlinewidth{1.505625pt}%
\definecolor{currentstroke}{rgb}{0.000000,0.000000,0.000000}%
\pgfsetstrokecolor{currentstroke}%
\pgfsetdash{}{0pt}%
\pgfpathmoveto{\pgfqpoint{11.908288in}{15.220099in}}%
\pgfpathlineto{\pgfqpoint{11.908288in}{15.209845in}}%
\pgfusepath{stroke}%
\end{pgfscope}%
\begin{pgfscope}%
\pgfpathrectangle{\pgfqpoint{9.810417in}{15.141860in}}{\pgfqpoint{5.489583in}{0.877907in}}%
\pgfusepath{clip}%
\pgfsetbuttcap%
\pgfsetroundjoin%
\pgfsetlinewidth{1.505625pt}%
\definecolor{currentstroke}{rgb}{0.000000,0.000000,0.000000}%
\pgfsetstrokecolor{currentstroke}%
\pgfsetdash{}{0pt}%
\pgfpathmoveto{\pgfqpoint{12.031511in}{15.220099in}}%
\pgfpathlineto{\pgfqpoint{12.031511in}{15.239118in}}%
\pgfusepath{stroke}%
\end{pgfscope}%
\begin{pgfscope}%
\pgfpathrectangle{\pgfqpoint{9.810417in}{15.141860in}}{\pgfqpoint{5.489583in}{0.877907in}}%
\pgfusepath{clip}%
\pgfsetbuttcap%
\pgfsetroundjoin%
\pgfsetlinewidth{1.505625pt}%
\definecolor{currentstroke}{rgb}{0.000000,0.000000,0.000000}%
\pgfsetstrokecolor{currentstroke}%
\pgfsetdash{}{0pt}%
\pgfpathmoveto{\pgfqpoint{12.154734in}{15.220099in}}%
\pgfpathlineto{\pgfqpoint{12.154734in}{15.212988in}}%
\pgfusepath{stroke}%
\end{pgfscope}%
\begin{pgfscope}%
\pgfpathrectangle{\pgfqpoint{9.810417in}{15.141860in}}{\pgfqpoint{5.489583in}{0.877907in}}%
\pgfusepath{clip}%
\pgfsetbuttcap%
\pgfsetroundjoin%
\pgfsetlinewidth{1.505625pt}%
\definecolor{currentstroke}{rgb}{0.000000,0.000000,0.000000}%
\pgfsetstrokecolor{currentstroke}%
\pgfsetdash{}{0pt}%
\pgfpathmoveto{\pgfqpoint{12.277957in}{15.220099in}}%
\pgfpathlineto{\pgfqpoint{12.277957in}{15.194128in}}%
\pgfusepath{stroke}%
\end{pgfscope}%
\begin{pgfscope}%
\pgfpathrectangle{\pgfqpoint{9.810417in}{15.141860in}}{\pgfqpoint{5.489583in}{0.877907in}}%
\pgfusepath{clip}%
\pgfsetbuttcap%
\pgfsetroundjoin%
\pgfsetlinewidth{1.505625pt}%
\definecolor{currentstroke}{rgb}{0.000000,0.000000,0.000000}%
\pgfsetstrokecolor{currentstroke}%
\pgfsetdash{}{0pt}%
\pgfpathmoveto{\pgfqpoint{12.401180in}{15.220099in}}%
\pgfpathlineto{\pgfqpoint{12.401180in}{15.215222in}}%
\pgfusepath{stroke}%
\end{pgfscope}%
\begin{pgfscope}%
\pgfpathrectangle{\pgfqpoint{9.810417in}{15.141860in}}{\pgfqpoint{5.489583in}{0.877907in}}%
\pgfusepath{clip}%
\pgfsetbuttcap%
\pgfsetroundjoin%
\pgfsetlinewidth{1.505625pt}%
\definecolor{currentstroke}{rgb}{0.000000,0.000000,0.000000}%
\pgfsetstrokecolor{currentstroke}%
\pgfsetdash{}{0pt}%
\pgfpathmoveto{\pgfqpoint{12.524403in}{15.220099in}}%
\pgfpathlineto{\pgfqpoint{12.524403in}{15.246563in}}%
\pgfusepath{stroke}%
\end{pgfscope}%
\begin{pgfscope}%
\pgfpathrectangle{\pgfqpoint{9.810417in}{15.141860in}}{\pgfqpoint{5.489583in}{0.877907in}}%
\pgfusepath{clip}%
\pgfsetbuttcap%
\pgfsetroundjoin%
\pgfsetlinewidth{1.505625pt}%
\definecolor{currentstroke}{rgb}{0.000000,0.000000,0.000000}%
\pgfsetstrokecolor{currentstroke}%
\pgfsetdash{}{0pt}%
\pgfpathmoveto{\pgfqpoint{12.647626in}{15.220099in}}%
\pgfpathlineto{\pgfqpoint{12.647626in}{15.216437in}}%
\pgfusepath{stroke}%
\end{pgfscope}%
\begin{pgfscope}%
\pgfpathrectangle{\pgfqpoint{9.810417in}{15.141860in}}{\pgfqpoint{5.489583in}{0.877907in}}%
\pgfusepath{clip}%
\pgfsetbuttcap%
\pgfsetroundjoin%
\pgfsetlinewidth{1.505625pt}%
\definecolor{currentstroke}{rgb}{0.000000,0.000000,0.000000}%
\pgfsetstrokecolor{currentstroke}%
\pgfsetdash{}{0pt}%
\pgfpathmoveto{\pgfqpoint{12.770849in}{15.220099in}}%
\pgfpathlineto{\pgfqpoint{12.770849in}{15.224132in}}%
\pgfusepath{stroke}%
\end{pgfscope}%
\begin{pgfscope}%
\pgfpathrectangle{\pgfqpoint{9.810417in}{15.141860in}}{\pgfqpoint{5.489583in}{0.877907in}}%
\pgfusepath{clip}%
\pgfsetbuttcap%
\pgfsetroundjoin%
\pgfsetlinewidth{1.505625pt}%
\definecolor{currentstroke}{rgb}{0.000000,0.000000,0.000000}%
\pgfsetstrokecolor{currentstroke}%
\pgfsetdash{}{0pt}%
\pgfpathmoveto{\pgfqpoint{12.894072in}{15.220099in}}%
\pgfpathlineto{\pgfqpoint{12.894072in}{15.239570in}}%
\pgfusepath{stroke}%
\end{pgfscope}%
\begin{pgfscope}%
\pgfpathrectangle{\pgfqpoint{9.810417in}{15.141860in}}{\pgfqpoint{5.489583in}{0.877907in}}%
\pgfusepath{clip}%
\pgfsetbuttcap%
\pgfsetroundjoin%
\pgfsetlinewidth{1.505625pt}%
\definecolor{currentstroke}{rgb}{0.000000,0.000000,0.000000}%
\pgfsetstrokecolor{currentstroke}%
\pgfsetdash{}{0pt}%
\pgfpathmoveto{\pgfqpoint{13.017294in}{15.220099in}}%
\pgfpathlineto{\pgfqpoint{13.017294in}{15.224306in}}%
\pgfusepath{stroke}%
\end{pgfscope}%
\begin{pgfscope}%
\pgfpathrectangle{\pgfqpoint{9.810417in}{15.141860in}}{\pgfqpoint{5.489583in}{0.877907in}}%
\pgfusepath{clip}%
\pgfsetbuttcap%
\pgfsetroundjoin%
\pgfsetlinewidth{1.505625pt}%
\definecolor{currentstroke}{rgb}{0.000000,0.000000,0.000000}%
\pgfsetstrokecolor{currentstroke}%
\pgfsetdash{}{0pt}%
\pgfpathmoveto{\pgfqpoint{13.140517in}{15.220099in}}%
\pgfpathlineto{\pgfqpoint{13.140517in}{15.226160in}}%
\pgfusepath{stroke}%
\end{pgfscope}%
\begin{pgfscope}%
\pgfpathrectangle{\pgfqpoint{9.810417in}{15.141860in}}{\pgfqpoint{5.489583in}{0.877907in}}%
\pgfusepath{clip}%
\pgfsetbuttcap%
\pgfsetroundjoin%
\pgfsetlinewidth{1.505625pt}%
\definecolor{currentstroke}{rgb}{0.000000,0.000000,0.000000}%
\pgfsetstrokecolor{currentstroke}%
\pgfsetdash{}{0pt}%
\pgfpathmoveto{\pgfqpoint{13.263740in}{15.220099in}}%
\pgfpathlineto{\pgfqpoint{13.263740in}{15.215944in}}%
\pgfusepath{stroke}%
\end{pgfscope}%
\begin{pgfscope}%
\pgfpathrectangle{\pgfqpoint{9.810417in}{15.141860in}}{\pgfqpoint{5.489583in}{0.877907in}}%
\pgfusepath{clip}%
\pgfsetbuttcap%
\pgfsetroundjoin%
\pgfsetlinewidth{1.505625pt}%
\definecolor{currentstroke}{rgb}{0.000000,0.000000,0.000000}%
\pgfsetstrokecolor{currentstroke}%
\pgfsetdash{}{0pt}%
\pgfpathmoveto{\pgfqpoint{13.386963in}{15.220099in}}%
\pgfpathlineto{\pgfqpoint{13.386963in}{15.218044in}}%
\pgfusepath{stroke}%
\end{pgfscope}%
\begin{pgfscope}%
\pgfpathrectangle{\pgfqpoint{9.810417in}{15.141860in}}{\pgfqpoint{5.489583in}{0.877907in}}%
\pgfusepath{clip}%
\pgfsetbuttcap%
\pgfsetroundjoin%
\pgfsetlinewidth{1.505625pt}%
\definecolor{currentstroke}{rgb}{0.000000,0.000000,0.000000}%
\pgfsetstrokecolor{currentstroke}%
\pgfsetdash{}{0pt}%
\pgfpathmoveto{\pgfqpoint{13.510186in}{15.220099in}}%
\pgfpathlineto{\pgfqpoint{13.510186in}{15.202795in}}%
\pgfusepath{stroke}%
\end{pgfscope}%
\begin{pgfscope}%
\pgfpathrectangle{\pgfqpoint{9.810417in}{15.141860in}}{\pgfqpoint{5.489583in}{0.877907in}}%
\pgfusepath{clip}%
\pgfsetbuttcap%
\pgfsetroundjoin%
\pgfsetlinewidth{1.505625pt}%
\definecolor{currentstroke}{rgb}{0.000000,0.000000,0.000000}%
\pgfsetstrokecolor{currentstroke}%
\pgfsetdash{}{0pt}%
\pgfpathmoveto{\pgfqpoint{13.633409in}{15.220099in}}%
\pgfpathlineto{\pgfqpoint{13.633409in}{15.214413in}}%
\pgfusepath{stroke}%
\end{pgfscope}%
\begin{pgfscope}%
\pgfpathrectangle{\pgfqpoint{9.810417in}{15.141860in}}{\pgfqpoint{5.489583in}{0.877907in}}%
\pgfusepath{clip}%
\pgfsetbuttcap%
\pgfsetroundjoin%
\pgfsetlinewidth{1.505625pt}%
\definecolor{currentstroke}{rgb}{0.000000,0.000000,0.000000}%
\pgfsetstrokecolor{currentstroke}%
\pgfsetdash{}{0pt}%
\pgfpathmoveto{\pgfqpoint{13.756632in}{15.220099in}}%
\pgfpathlineto{\pgfqpoint{13.756632in}{15.223549in}}%
\pgfusepath{stroke}%
\end{pgfscope}%
\begin{pgfscope}%
\pgfpathrectangle{\pgfqpoint{9.810417in}{15.141860in}}{\pgfqpoint{5.489583in}{0.877907in}}%
\pgfusepath{clip}%
\pgfsetbuttcap%
\pgfsetroundjoin%
\pgfsetlinewidth{1.505625pt}%
\definecolor{currentstroke}{rgb}{0.000000,0.000000,0.000000}%
\pgfsetstrokecolor{currentstroke}%
\pgfsetdash{}{0pt}%
\pgfpathmoveto{\pgfqpoint{13.879855in}{15.220099in}}%
\pgfpathlineto{\pgfqpoint{13.879855in}{15.218953in}}%
\pgfusepath{stroke}%
\end{pgfscope}%
\begin{pgfscope}%
\pgfpathrectangle{\pgfqpoint{9.810417in}{15.141860in}}{\pgfqpoint{5.489583in}{0.877907in}}%
\pgfusepath{clip}%
\pgfsetbuttcap%
\pgfsetroundjoin%
\pgfsetlinewidth{1.505625pt}%
\definecolor{currentstroke}{rgb}{0.000000,0.000000,0.000000}%
\pgfsetstrokecolor{currentstroke}%
\pgfsetdash{}{0pt}%
\pgfpathmoveto{\pgfqpoint{14.003078in}{15.220099in}}%
\pgfpathlineto{\pgfqpoint{14.003078in}{15.238695in}}%
\pgfusepath{stroke}%
\end{pgfscope}%
\begin{pgfscope}%
\pgfpathrectangle{\pgfqpoint{9.810417in}{15.141860in}}{\pgfqpoint{5.489583in}{0.877907in}}%
\pgfusepath{clip}%
\pgfsetbuttcap%
\pgfsetroundjoin%
\pgfsetlinewidth{1.505625pt}%
\definecolor{currentstroke}{rgb}{0.000000,0.000000,0.000000}%
\pgfsetstrokecolor{currentstroke}%
\pgfsetdash{}{0pt}%
\pgfpathmoveto{\pgfqpoint{14.126301in}{15.220099in}}%
\pgfpathlineto{\pgfqpoint{14.126301in}{15.223453in}}%
\pgfusepath{stroke}%
\end{pgfscope}%
\begin{pgfscope}%
\pgfpathrectangle{\pgfqpoint{9.810417in}{15.141860in}}{\pgfqpoint{5.489583in}{0.877907in}}%
\pgfusepath{clip}%
\pgfsetbuttcap%
\pgfsetroundjoin%
\pgfsetlinewidth{1.505625pt}%
\definecolor{currentstroke}{rgb}{0.000000,0.000000,0.000000}%
\pgfsetstrokecolor{currentstroke}%
\pgfsetdash{}{0pt}%
\pgfpathmoveto{\pgfqpoint{14.249524in}{15.220099in}}%
\pgfpathlineto{\pgfqpoint{14.249524in}{15.198821in}}%
\pgfusepath{stroke}%
\end{pgfscope}%
\begin{pgfscope}%
\pgfpathrectangle{\pgfqpoint{9.810417in}{15.141860in}}{\pgfqpoint{5.489583in}{0.877907in}}%
\pgfusepath{clip}%
\pgfsetbuttcap%
\pgfsetroundjoin%
\pgfsetlinewidth{1.505625pt}%
\definecolor{currentstroke}{rgb}{0.000000,0.000000,0.000000}%
\pgfsetstrokecolor{currentstroke}%
\pgfsetdash{}{0pt}%
\pgfpathmoveto{\pgfqpoint{14.372747in}{15.220099in}}%
\pgfpathlineto{\pgfqpoint{14.372747in}{15.208401in}}%
\pgfusepath{stroke}%
\end{pgfscope}%
\begin{pgfscope}%
\pgfpathrectangle{\pgfqpoint{9.810417in}{15.141860in}}{\pgfqpoint{5.489583in}{0.877907in}}%
\pgfusepath{clip}%
\pgfsetbuttcap%
\pgfsetroundjoin%
\pgfsetlinewidth{1.505625pt}%
\definecolor{currentstroke}{rgb}{0.000000,0.000000,0.000000}%
\pgfsetstrokecolor{currentstroke}%
\pgfsetdash{}{0pt}%
\pgfpathmoveto{\pgfqpoint{14.495970in}{15.220099in}}%
\pgfpathlineto{\pgfqpoint{14.495970in}{15.208474in}}%
\pgfusepath{stroke}%
\end{pgfscope}%
\begin{pgfscope}%
\pgfpathrectangle{\pgfqpoint{9.810417in}{15.141860in}}{\pgfqpoint{5.489583in}{0.877907in}}%
\pgfusepath{clip}%
\pgfsetbuttcap%
\pgfsetroundjoin%
\pgfsetlinewidth{1.505625pt}%
\definecolor{currentstroke}{rgb}{0.000000,0.000000,0.000000}%
\pgfsetstrokecolor{currentstroke}%
\pgfsetdash{}{0pt}%
\pgfpathmoveto{\pgfqpoint{14.619193in}{15.220099in}}%
\pgfpathlineto{\pgfqpoint{14.619193in}{15.223827in}}%
\pgfusepath{stroke}%
\end{pgfscope}%
\begin{pgfscope}%
\pgfpathrectangle{\pgfqpoint{9.810417in}{15.141860in}}{\pgfqpoint{5.489583in}{0.877907in}}%
\pgfusepath{clip}%
\pgfsetbuttcap%
\pgfsetroundjoin%
\pgfsetlinewidth{1.505625pt}%
\definecolor{currentstroke}{rgb}{0.000000,0.000000,0.000000}%
\pgfsetstrokecolor{currentstroke}%
\pgfsetdash{}{0pt}%
\pgfpathmoveto{\pgfqpoint{14.742416in}{15.220099in}}%
\pgfpathlineto{\pgfqpoint{14.742416in}{15.241676in}}%
\pgfusepath{stroke}%
\end{pgfscope}%
\begin{pgfscope}%
\pgfpathrectangle{\pgfqpoint{9.810417in}{15.141860in}}{\pgfqpoint{5.489583in}{0.877907in}}%
\pgfusepath{clip}%
\pgfsetbuttcap%
\pgfsetroundjoin%
\pgfsetlinewidth{1.505625pt}%
\definecolor{currentstroke}{rgb}{0.000000,0.000000,0.000000}%
\pgfsetstrokecolor{currentstroke}%
\pgfsetdash{}{0pt}%
\pgfpathmoveto{\pgfqpoint{14.865639in}{15.220099in}}%
\pgfpathlineto{\pgfqpoint{14.865639in}{15.225734in}}%
\pgfusepath{stroke}%
\end{pgfscope}%
\begin{pgfscope}%
\pgfpathrectangle{\pgfqpoint{9.810417in}{15.141860in}}{\pgfqpoint{5.489583in}{0.877907in}}%
\pgfusepath{clip}%
\pgfsetbuttcap%
\pgfsetroundjoin%
\pgfsetlinewidth{1.505625pt}%
\definecolor{currentstroke}{rgb}{0.000000,0.000000,0.000000}%
\pgfsetstrokecolor{currentstroke}%
\pgfsetdash{}{0pt}%
\pgfpathmoveto{\pgfqpoint{14.988862in}{15.220099in}}%
\pgfpathlineto{\pgfqpoint{14.988862in}{15.225291in}}%
\pgfusepath{stroke}%
\end{pgfscope}%
\begin{pgfscope}%
\pgfpathrectangle{\pgfqpoint{9.810417in}{15.141860in}}{\pgfqpoint{5.489583in}{0.877907in}}%
\pgfusepath{clip}%
\pgfsetroundcap%
\pgfsetroundjoin%
\pgfsetlinewidth{1.505625pt}%
\definecolor{currentstroke}{rgb}{0.121569,0.466667,0.705882}%
\pgfsetstrokecolor{currentstroke}%
\pgfsetdash{}{0pt}%
\pgfpathmoveto{\pgfqpoint{9.810417in}{15.220099in}}%
\pgfpathlineto{\pgfqpoint{15.300000in}{15.220099in}}%
\pgfusepath{stroke}%
\end{pgfscope}%
\begin{pgfscope}%
\pgfpathrectangle{\pgfqpoint{9.810417in}{15.141860in}}{\pgfqpoint{5.489583in}{0.877907in}}%
\pgfusepath{clip}%
\pgfsetbuttcap%
\pgfsetroundjoin%
\definecolor{currentfill}{rgb}{0.121569,0.466667,0.705882}%
\pgfsetfillcolor{currentfill}%
\pgfsetlinewidth{1.003750pt}%
\definecolor{currentstroke}{rgb}{0.121569,0.466667,0.705882}%
\pgfsetstrokecolor{currentstroke}%
\pgfsetdash{}{0pt}%
\pgfsys@defobject{currentmarker}{\pgfqpoint{-0.034722in}{-0.034722in}}{\pgfqpoint{0.034722in}{0.034722in}}{%
\pgfpathmoveto{\pgfqpoint{0.000000in}{-0.034722in}}%
\pgfpathcurveto{\pgfqpoint{0.009208in}{-0.034722in}}{\pgfqpoint{0.018041in}{-0.031064in}}{\pgfqpoint{0.024552in}{-0.024552in}}%
\pgfpathcurveto{\pgfqpoint{0.031064in}{-0.018041in}}{\pgfqpoint{0.034722in}{-0.009208in}}{\pgfqpoint{0.034722in}{0.000000in}}%
\pgfpathcurveto{\pgfqpoint{0.034722in}{0.009208in}}{\pgfqpoint{0.031064in}{0.018041in}}{\pgfqpoint{0.024552in}{0.024552in}}%
\pgfpathcurveto{\pgfqpoint{0.018041in}{0.031064in}}{\pgfqpoint{0.009208in}{0.034722in}}{\pgfqpoint{0.000000in}{0.034722in}}%
\pgfpathcurveto{\pgfqpoint{-0.009208in}{0.034722in}}{\pgfqpoint{-0.018041in}{0.031064in}}{\pgfqpoint{-0.024552in}{0.024552in}}%
\pgfpathcurveto{\pgfqpoint{-0.031064in}{0.018041in}}{\pgfqpoint{-0.034722in}{0.009208in}}{\pgfqpoint{-0.034722in}{0.000000in}}%
\pgfpathcurveto{\pgfqpoint{-0.034722in}{-0.009208in}}{\pgfqpoint{-0.031064in}{-0.018041in}}{\pgfqpoint{-0.024552in}{-0.024552in}}%
\pgfpathcurveto{\pgfqpoint{-0.018041in}{-0.031064in}}{\pgfqpoint{-0.009208in}{-0.034722in}}{\pgfqpoint{0.000000in}{-0.034722in}}%
\pgfpathclose%
\pgfusepath{stroke,fill}%
}%
\begin{pgfscope}%
\pgfsys@transformshift{10.059943in}{15.979863in}%
\pgfsys@useobject{currentmarker}{}%
\end{pgfscope}%
\begin{pgfscope}%
\pgfsys@transformshift{10.183166in}{15.976437in}%
\pgfsys@useobject{currentmarker}{}%
\end{pgfscope}%
\begin{pgfscope}%
\pgfsys@transformshift{10.306389in}{15.202667in}%
\pgfsys@useobject{currentmarker}{}%
\end{pgfscope}%
\begin{pgfscope}%
\pgfsys@transformshift{10.429612in}{15.234885in}%
\pgfsys@useobject{currentmarker}{}%
\end{pgfscope}%
\begin{pgfscope}%
\pgfsys@transformshift{10.552835in}{15.216929in}%
\pgfsys@useobject{currentmarker}{}%
\end{pgfscope}%
\begin{pgfscope}%
\pgfsys@transformshift{10.676058in}{15.223376in}%
\pgfsys@useobject{currentmarker}{}%
\end{pgfscope}%
\begin{pgfscope}%
\pgfsys@transformshift{10.799281in}{15.238183in}%
\pgfsys@useobject{currentmarker}{}%
\end{pgfscope}%
\begin{pgfscope}%
\pgfsys@transformshift{10.922504in}{15.215006in}%
\pgfsys@useobject{currentmarker}{}%
\end{pgfscope}%
\begin{pgfscope}%
\pgfsys@transformshift{11.045727in}{15.230690in}%
\pgfsys@useobject{currentmarker}{}%
\end{pgfscope}%
\begin{pgfscope}%
\pgfsys@transformshift{11.168950in}{15.203849in}%
\pgfsys@useobject{currentmarker}{}%
\end{pgfscope}%
\begin{pgfscope}%
\pgfsys@transformshift{11.292173in}{15.242116in}%
\pgfsys@useobject{currentmarker}{}%
\end{pgfscope}%
\begin{pgfscope}%
\pgfsys@transformshift{11.415396in}{15.238844in}%
\pgfsys@useobject{currentmarker}{}%
\end{pgfscope}%
\begin{pgfscope}%
\pgfsys@transformshift{11.538619in}{15.225891in}%
\pgfsys@useobject{currentmarker}{}%
\end{pgfscope}%
\begin{pgfscope}%
\pgfsys@transformshift{11.661842in}{15.194351in}%
\pgfsys@useobject{currentmarker}{}%
\end{pgfscope}%
\begin{pgfscope}%
\pgfsys@transformshift{11.785065in}{15.199830in}%
\pgfsys@useobject{currentmarker}{}%
\end{pgfscope}%
\begin{pgfscope}%
\pgfsys@transformshift{11.908288in}{15.209845in}%
\pgfsys@useobject{currentmarker}{}%
\end{pgfscope}%
\begin{pgfscope}%
\pgfsys@transformshift{12.031511in}{15.239118in}%
\pgfsys@useobject{currentmarker}{}%
\end{pgfscope}%
\begin{pgfscope}%
\pgfsys@transformshift{12.154734in}{15.212988in}%
\pgfsys@useobject{currentmarker}{}%
\end{pgfscope}%
\begin{pgfscope}%
\pgfsys@transformshift{12.277957in}{15.194128in}%
\pgfsys@useobject{currentmarker}{}%
\end{pgfscope}%
\begin{pgfscope}%
\pgfsys@transformshift{12.401180in}{15.215222in}%
\pgfsys@useobject{currentmarker}{}%
\end{pgfscope}%
\begin{pgfscope}%
\pgfsys@transformshift{12.524403in}{15.246563in}%
\pgfsys@useobject{currentmarker}{}%
\end{pgfscope}%
\begin{pgfscope}%
\pgfsys@transformshift{12.647626in}{15.216437in}%
\pgfsys@useobject{currentmarker}{}%
\end{pgfscope}%
\begin{pgfscope}%
\pgfsys@transformshift{12.770849in}{15.224132in}%
\pgfsys@useobject{currentmarker}{}%
\end{pgfscope}%
\begin{pgfscope}%
\pgfsys@transformshift{12.894072in}{15.239570in}%
\pgfsys@useobject{currentmarker}{}%
\end{pgfscope}%
\begin{pgfscope}%
\pgfsys@transformshift{13.017294in}{15.224306in}%
\pgfsys@useobject{currentmarker}{}%
\end{pgfscope}%
\begin{pgfscope}%
\pgfsys@transformshift{13.140517in}{15.226160in}%
\pgfsys@useobject{currentmarker}{}%
\end{pgfscope}%
\begin{pgfscope}%
\pgfsys@transformshift{13.263740in}{15.215944in}%
\pgfsys@useobject{currentmarker}{}%
\end{pgfscope}%
\begin{pgfscope}%
\pgfsys@transformshift{13.386963in}{15.218044in}%
\pgfsys@useobject{currentmarker}{}%
\end{pgfscope}%
\begin{pgfscope}%
\pgfsys@transformshift{13.510186in}{15.202795in}%
\pgfsys@useobject{currentmarker}{}%
\end{pgfscope}%
\begin{pgfscope}%
\pgfsys@transformshift{13.633409in}{15.214413in}%
\pgfsys@useobject{currentmarker}{}%
\end{pgfscope}%
\begin{pgfscope}%
\pgfsys@transformshift{13.756632in}{15.223549in}%
\pgfsys@useobject{currentmarker}{}%
\end{pgfscope}%
\begin{pgfscope}%
\pgfsys@transformshift{13.879855in}{15.218953in}%
\pgfsys@useobject{currentmarker}{}%
\end{pgfscope}%
\begin{pgfscope}%
\pgfsys@transformshift{14.003078in}{15.238695in}%
\pgfsys@useobject{currentmarker}{}%
\end{pgfscope}%
\begin{pgfscope}%
\pgfsys@transformshift{14.126301in}{15.223453in}%
\pgfsys@useobject{currentmarker}{}%
\end{pgfscope}%
\begin{pgfscope}%
\pgfsys@transformshift{14.249524in}{15.198821in}%
\pgfsys@useobject{currentmarker}{}%
\end{pgfscope}%
\begin{pgfscope}%
\pgfsys@transformshift{14.372747in}{15.208401in}%
\pgfsys@useobject{currentmarker}{}%
\end{pgfscope}%
\begin{pgfscope}%
\pgfsys@transformshift{14.495970in}{15.208474in}%
\pgfsys@useobject{currentmarker}{}%
\end{pgfscope}%
\begin{pgfscope}%
\pgfsys@transformshift{14.619193in}{15.223827in}%
\pgfsys@useobject{currentmarker}{}%
\end{pgfscope}%
\begin{pgfscope}%
\pgfsys@transformshift{14.742416in}{15.241676in}%
\pgfsys@useobject{currentmarker}{}%
\end{pgfscope}%
\begin{pgfscope}%
\pgfsys@transformshift{14.865639in}{15.225734in}%
\pgfsys@useobject{currentmarker}{}%
\end{pgfscope}%
\begin{pgfscope}%
\pgfsys@transformshift{14.988862in}{15.225291in}%
\pgfsys@useobject{currentmarker}{}%
\end{pgfscope}%
\end{pgfscope}%
\begin{pgfscope}%
\pgfsetrectcap%
\pgfsetmiterjoin%
\pgfsetlinewidth{0.803000pt}%
\definecolor{currentstroke}{rgb}{1.000000,1.000000,1.000000}%
\pgfsetstrokecolor{currentstroke}%
\pgfsetdash{}{0pt}%
\pgfpathmoveto{\pgfqpoint{9.810417in}{15.141860in}}%
\pgfpathlineto{\pgfqpoint{9.810417in}{16.019767in}}%
\pgfusepath{stroke}%
\end{pgfscope}%
\begin{pgfscope}%
\pgfsetrectcap%
\pgfsetmiterjoin%
\pgfsetlinewidth{0.803000pt}%
\definecolor{currentstroke}{rgb}{1.000000,1.000000,1.000000}%
\pgfsetstrokecolor{currentstroke}%
\pgfsetdash{}{0pt}%
\pgfpathmoveto{\pgfqpoint{15.300000in}{15.141860in}}%
\pgfpathlineto{\pgfqpoint{15.300000in}{16.019767in}}%
\pgfusepath{stroke}%
\end{pgfscope}%
\begin{pgfscope}%
\pgfsetrectcap%
\pgfsetmiterjoin%
\pgfsetlinewidth{0.803000pt}%
\definecolor{currentstroke}{rgb}{1.000000,1.000000,1.000000}%
\pgfsetstrokecolor{currentstroke}%
\pgfsetdash{}{0pt}%
\pgfpathmoveto{\pgfqpoint{9.810417in}{15.141860in}}%
\pgfpathlineto{\pgfqpoint{15.300000in}{15.141860in}}%
\pgfusepath{stroke}%
\end{pgfscope}%
\begin{pgfscope}%
\pgfsetrectcap%
\pgfsetmiterjoin%
\pgfsetlinewidth{0.803000pt}%
\definecolor{currentstroke}{rgb}{1.000000,1.000000,1.000000}%
\pgfsetstrokecolor{currentstroke}%
\pgfsetdash{}{0pt}%
\pgfpathmoveto{\pgfqpoint{9.810417in}{16.019767in}}%
\pgfpathlineto{\pgfqpoint{15.300000in}{16.019767in}}%
\pgfusepath{stroke}%
\end{pgfscope}%
\begin{pgfscope}%
\definecolor{textcolor}{rgb}{0.150000,0.150000,0.150000}%
\pgfsetstrokecolor{textcolor}%
\pgfsetfillcolor{textcolor}%
\pgftext[x=12.555208in,y=16.103101in,,base]{\color{textcolor}\rmfamily\fontsize{16.800000}{20.160000}\selectfont Partial Autocorrelation}%
\end{pgfscope}%
\begin{pgfscope}%
\pgfsetbuttcap%
\pgfsetmiterjoin%
\definecolor{currentfill}{rgb}{0.917647,0.917647,0.949020}%
\pgfsetfillcolor{currentfill}%
\pgfsetlinewidth{0.000000pt}%
\definecolor{currentstroke}{rgb}{0.000000,0.000000,0.000000}%
\pgfsetstrokecolor{currentstroke}%
\pgfsetstrokeopacity{0.000000}%
\pgfsetdash{}{0pt}%
\pgfpathmoveto{\pgfqpoint{2.125000in}{13.561628in}}%
\pgfpathlineto{\pgfqpoint{7.614583in}{13.561628in}}%
\pgfpathlineto{\pgfqpoint{7.614583in}{14.439535in}}%
\pgfpathlineto{\pgfqpoint{2.125000in}{14.439535in}}%
\pgfpathclose%
\pgfusepath{fill}%
\end{pgfscope}%
\begin{pgfscope}%
\pgfpathrectangle{\pgfqpoint{2.125000in}{13.561628in}}{\pgfqpoint{5.489583in}{0.877907in}}%
\pgfusepath{clip}%
\pgfsetroundcap%
\pgfsetroundjoin%
\pgfsetlinewidth{0.803000pt}%
\definecolor{currentstroke}{rgb}{1.000000,1.000000,1.000000}%
\pgfsetstrokecolor{currentstroke}%
\pgfsetdash{}{0pt}%
\pgfpathmoveto{\pgfqpoint{2.374527in}{13.561628in}}%
\pgfpathlineto{\pgfqpoint{2.374527in}{14.439535in}}%
\pgfusepath{stroke}%
\end{pgfscope}%
\begin{pgfscope}%
\definecolor{textcolor}{rgb}{0.150000,0.150000,0.150000}%
\pgfsetstrokecolor{textcolor}%
\pgfsetfillcolor{textcolor}%
\pgftext[x=2.374527in,y=13.464406in,,top]{\color{textcolor}\rmfamily\fontsize{14.000000}{16.800000}\selectfont 0}%
\end{pgfscope}%
\begin{pgfscope}%
\pgfpathrectangle{\pgfqpoint{2.125000in}{13.561628in}}{\pgfqpoint{5.489583in}{0.877907in}}%
\pgfusepath{clip}%
\pgfsetroundcap%
\pgfsetroundjoin%
\pgfsetlinewidth{0.803000pt}%
\definecolor{currentstroke}{rgb}{1.000000,1.000000,1.000000}%
\pgfsetstrokecolor{currentstroke}%
\pgfsetdash{}{0pt}%
\pgfpathmoveto{\pgfqpoint{2.990641in}{13.561628in}}%
\pgfpathlineto{\pgfqpoint{2.990641in}{14.439535in}}%
\pgfusepath{stroke}%
\end{pgfscope}%
\begin{pgfscope}%
\definecolor{textcolor}{rgb}{0.150000,0.150000,0.150000}%
\pgfsetstrokecolor{textcolor}%
\pgfsetfillcolor{textcolor}%
\pgftext[x=2.990641in,y=13.464406in,,top]{\color{textcolor}\rmfamily\fontsize{14.000000}{16.800000}\selectfont 5}%
\end{pgfscope}%
\begin{pgfscope}%
\pgfpathrectangle{\pgfqpoint{2.125000in}{13.561628in}}{\pgfqpoint{5.489583in}{0.877907in}}%
\pgfusepath{clip}%
\pgfsetroundcap%
\pgfsetroundjoin%
\pgfsetlinewidth{0.803000pt}%
\definecolor{currentstroke}{rgb}{1.000000,1.000000,1.000000}%
\pgfsetstrokecolor{currentstroke}%
\pgfsetdash{}{0pt}%
\pgfpathmoveto{\pgfqpoint{3.606756in}{13.561628in}}%
\pgfpathlineto{\pgfqpoint{3.606756in}{14.439535in}}%
\pgfusepath{stroke}%
\end{pgfscope}%
\begin{pgfscope}%
\definecolor{textcolor}{rgb}{0.150000,0.150000,0.150000}%
\pgfsetstrokecolor{textcolor}%
\pgfsetfillcolor{textcolor}%
\pgftext[x=3.606756in,y=13.464406in,,top]{\color{textcolor}\rmfamily\fontsize{14.000000}{16.800000}\selectfont 10}%
\end{pgfscope}%
\begin{pgfscope}%
\pgfpathrectangle{\pgfqpoint{2.125000in}{13.561628in}}{\pgfqpoint{5.489583in}{0.877907in}}%
\pgfusepath{clip}%
\pgfsetroundcap%
\pgfsetroundjoin%
\pgfsetlinewidth{0.803000pt}%
\definecolor{currentstroke}{rgb}{1.000000,1.000000,1.000000}%
\pgfsetstrokecolor{currentstroke}%
\pgfsetdash{}{0pt}%
\pgfpathmoveto{\pgfqpoint{4.222871in}{13.561628in}}%
\pgfpathlineto{\pgfqpoint{4.222871in}{14.439535in}}%
\pgfusepath{stroke}%
\end{pgfscope}%
\begin{pgfscope}%
\definecolor{textcolor}{rgb}{0.150000,0.150000,0.150000}%
\pgfsetstrokecolor{textcolor}%
\pgfsetfillcolor{textcolor}%
\pgftext[x=4.222871in,y=13.464406in,,top]{\color{textcolor}\rmfamily\fontsize{14.000000}{16.800000}\selectfont 15}%
\end{pgfscope}%
\begin{pgfscope}%
\pgfpathrectangle{\pgfqpoint{2.125000in}{13.561628in}}{\pgfqpoint{5.489583in}{0.877907in}}%
\pgfusepath{clip}%
\pgfsetroundcap%
\pgfsetroundjoin%
\pgfsetlinewidth{0.803000pt}%
\definecolor{currentstroke}{rgb}{1.000000,1.000000,1.000000}%
\pgfsetstrokecolor{currentstroke}%
\pgfsetdash{}{0pt}%
\pgfpathmoveto{\pgfqpoint{4.838986in}{13.561628in}}%
\pgfpathlineto{\pgfqpoint{4.838986in}{14.439535in}}%
\pgfusepath{stroke}%
\end{pgfscope}%
\begin{pgfscope}%
\definecolor{textcolor}{rgb}{0.150000,0.150000,0.150000}%
\pgfsetstrokecolor{textcolor}%
\pgfsetfillcolor{textcolor}%
\pgftext[x=4.838986in,y=13.464406in,,top]{\color{textcolor}\rmfamily\fontsize{14.000000}{16.800000}\selectfont 20}%
\end{pgfscope}%
\begin{pgfscope}%
\pgfpathrectangle{\pgfqpoint{2.125000in}{13.561628in}}{\pgfqpoint{5.489583in}{0.877907in}}%
\pgfusepath{clip}%
\pgfsetroundcap%
\pgfsetroundjoin%
\pgfsetlinewidth{0.803000pt}%
\definecolor{currentstroke}{rgb}{1.000000,1.000000,1.000000}%
\pgfsetstrokecolor{currentstroke}%
\pgfsetdash{}{0pt}%
\pgfpathmoveto{\pgfqpoint{5.455101in}{13.561628in}}%
\pgfpathlineto{\pgfqpoint{5.455101in}{14.439535in}}%
\pgfusepath{stroke}%
\end{pgfscope}%
\begin{pgfscope}%
\definecolor{textcolor}{rgb}{0.150000,0.150000,0.150000}%
\pgfsetstrokecolor{textcolor}%
\pgfsetfillcolor{textcolor}%
\pgftext[x=5.455101in,y=13.464406in,,top]{\color{textcolor}\rmfamily\fontsize{14.000000}{16.800000}\selectfont 25}%
\end{pgfscope}%
\begin{pgfscope}%
\pgfpathrectangle{\pgfqpoint{2.125000in}{13.561628in}}{\pgfqpoint{5.489583in}{0.877907in}}%
\pgfusepath{clip}%
\pgfsetroundcap%
\pgfsetroundjoin%
\pgfsetlinewidth{0.803000pt}%
\definecolor{currentstroke}{rgb}{1.000000,1.000000,1.000000}%
\pgfsetstrokecolor{currentstroke}%
\pgfsetdash{}{0pt}%
\pgfpathmoveto{\pgfqpoint{6.071216in}{13.561628in}}%
\pgfpathlineto{\pgfqpoint{6.071216in}{14.439535in}}%
\pgfusepath{stroke}%
\end{pgfscope}%
\begin{pgfscope}%
\definecolor{textcolor}{rgb}{0.150000,0.150000,0.150000}%
\pgfsetstrokecolor{textcolor}%
\pgfsetfillcolor{textcolor}%
\pgftext[x=6.071216in,y=13.464406in,,top]{\color{textcolor}\rmfamily\fontsize{14.000000}{16.800000}\selectfont 30}%
\end{pgfscope}%
\begin{pgfscope}%
\pgfpathrectangle{\pgfqpoint{2.125000in}{13.561628in}}{\pgfqpoint{5.489583in}{0.877907in}}%
\pgfusepath{clip}%
\pgfsetroundcap%
\pgfsetroundjoin%
\pgfsetlinewidth{0.803000pt}%
\definecolor{currentstroke}{rgb}{1.000000,1.000000,1.000000}%
\pgfsetstrokecolor{currentstroke}%
\pgfsetdash{}{0pt}%
\pgfpathmoveto{\pgfqpoint{6.687330in}{13.561628in}}%
\pgfpathlineto{\pgfqpoint{6.687330in}{14.439535in}}%
\pgfusepath{stroke}%
\end{pgfscope}%
\begin{pgfscope}%
\definecolor{textcolor}{rgb}{0.150000,0.150000,0.150000}%
\pgfsetstrokecolor{textcolor}%
\pgfsetfillcolor{textcolor}%
\pgftext[x=6.687330in,y=13.464406in,,top]{\color{textcolor}\rmfamily\fontsize{14.000000}{16.800000}\selectfont 35}%
\end{pgfscope}%
\begin{pgfscope}%
\pgfpathrectangle{\pgfqpoint{2.125000in}{13.561628in}}{\pgfqpoint{5.489583in}{0.877907in}}%
\pgfusepath{clip}%
\pgfsetroundcap%
\pgfsetroundjoin%
\pgfsetlinewidth{0.803000pt}%
\definecolor{currentstroke}{rgb}{1.000000,1.000000,1.000000}%
\pgfsetstrokecolor{currentstroke}%
\pgfsetdash{}{0pt}%
\pgfpathmoveto{\pgfqpoint{7.303445in}{13.561628in}}%
\pgfpathlineto{\pgfqpoint{7.303445in}{14.439535in}}%
\pgfusepath{stroke}%
\end{pgfscope}%
\begin{pgfscope}%
\definecolor{textcolor}{rgb}{0.150000,0.150000,0.150000}%
\pgfsetstrokecolor{textcolor}%
\pgfsetfillcolor{textcolor}%
\pgftext[x=7.303445in,y=13.464406in,,top]{\color{textcolor}\rmfamily\fontsize{14.000000}{16.800000}\selectfont 40}%
\end{pgfscope}%
\begin{pgfscope}%
\pgfpathrectangle{\pgfqpoint{2.125000in}{13.561628in}}{\pgfqpoint{5.489583in}{0.877907in}}%
\pgfusepath{clip}%
\pgfsetroundcap%
\pgfsetroundjoin%
\pgfsetlinewidth{0.803000pt}%
\definecolor{currentstroke}{rgb}{1.000000,1.000000,1.000000}%
\pgfsetstrokecolor{currentstroke}%
\pgfsetdash{}{0pt}%
\pgfpathmoveto{\pgfqpoint{2.125000in}{13.835489in}}%
\pgfpathlineto{\pgfqpoint{7.614583in}{13.835489in}}%
\pgfusepath{stroke}%
\end{pgfscope}%
\begin{pgfscope}%
\definecolor{textcolor}{rgb}{0.150000,0.150000,0.150000}%
\pgfsetstrokecolor{textcolor}%
\pgfsetfillcolor{textcolor}%
\pgftext[x=1.904066in,y=13.761623in,left,base]{\color{textcolor}\rmfamily\fontsize{14.000000}{16.800000}\selectfont 0}%
\end{pgfscope}%
\begin{pgfscope}%
\pgfpathrectangle{\pgfqpoint{2.125000in}{13.561628in}}{\pgfqpoint{5.489583in}{0.877907in}}%
\pgfusepath{clip}%
\pgfsetroundcap%
\pgfsetroundjoin%
\pgfsetlinewidth{0.803000pt}%
\definecolor{currentstroke}{rgb}{1.000000,1.000000,1.000000}%
\pgfsetstrokecolor{currentstroke}%
\pgfsetdash{}{0pt}%
\pgfpathmoveto{\pgfqpoint{2.125000in}{14.399630in}}%
\pgfpathlineto{\pgfqpoint{7.614583in}{14.399630in}}%
\pgfusepath{stroke}%
\end{pgfscope}%
\begin{pgfscope}%
\definecolor{textcolor}{rgb}{0.150000,0.150000,0.150000}%
\pgfsetstrokecolor{textcolor}%
\pgfsetfillcolor{textcolor}%
\pgftext[x=1.904066in,y=14.325764in,left,base]{\color{textcolor}\rmfamily\fontsize{14.000000}{16.800000}\selectfont 1}%
\end{pgfscope}%
\begin{pgfscope}%
\pgfpathrectangle{\pgfqpoint{2.125000in}{13.561628in}}{\pgfqpoint{5.489583in}{0.877907in}}%
\pgfusepath{clip}%
\pgfsetbuttcap%
\pgfsetroundjoin%
\definecolor{currentfill}{rgb}{0.121569,0.466667,0.705882}%
\pgfsetfillcolor{currentfill}%
\pgfsetfillopacity{0.250000}%
\pgfsetlinewidth{1.003750pt}%
\definecolor{currentstroke}{rgb}{1.000000,1.000000,1.000000}%
\pgfsetstrokecolor{currentstroke}%
\pgfsetstrokeopacity{0.250000}%
\pgfsetdash{}{0pt}%
\pgfpathmoveto{\pgfqpoint{2.436138in}{13.863953in}}%
\pgfpathlineto{\pgfqpoint{2.436138in}{13.807025in}}%
\pgfpathlineto{\pgfqpoint{2.620972in}{13.786321in}}%
\pgfpathlineto{\pgfqpoint{2.744195in}{13.772152in}}%
\pgfpathlineto{\pgfqpoint{2.867418in}{13.760703in}}%
\pgfpathlineto{\pgfqpoint{2.990641in}{13.750862in}}%
\pgfpathlineto{\pgfqpoint{3.113864in}{13.742117in}}%
\pgfpathlineto{\pgfqpoint{3.237087in}{13.734184in}}%
\pgfpathlineto{\pgfqpoint{3.360310in}{13.726883in}}%
\pgfpathlineto{\pgfqpoint{3.483533in}{13.720091in}}%
\pgfpathlineto{\pgfqpoint{3.606756in}{13.713725in}}%
\pgfpathlineto{\pgfqpoint{3.729979in}{13.707718in}}%
\pgfpathlineto{\pgfqpoint{3.853202in}{13.702024in}}%
\pgfpathlineto{\pgfqpoint{3.976425in}{13.696604in}}%
\pgfpathlineto{\pgfqpoint{4.099648in}{13.691427in}}%
\pgfpathlineto{\pgfqpoint{4.222871in}{13.686471in}}%
\pgfpathlineto{\pgfqpoint{4.346094in}{13.681716in}}%
\pgfpathlineto{\pgfqpoint{4.469317in}{13.677141in}}%
\pgfpathlineto{\pgfqpoint{4.592540in}{13.672732in}}%
\pgfpathlineto{\pgfqpoint{4.715763in}{13.668477in}}%
\pgfpathlineto{\pgfqpoint{4.838986in}{13.664363in}}%
\pgfpathlineto{\pgfqpoint{4.962209in}{13.660381in}}%
\pgfpathlineto{\pgfqpoint{5.085432in}{13.656520in}}%
\pgfpathlineto{\pgfqpoint{5.208655in}{13.652772in}}%
\pgfpathlineto{\pgfqpoint{5.331878in}{13.649130in}}%
\pgfpathlineto{\pgfqpoint{5.455101in}{13.645587in}}%
\pgfpathlineto{\pgfqpoint{5.578324in}{13.642140in}}%
\pgfpathlineto{\pgfqpoint{5.701547in}{13.638782in}}%
\pgfpathlineto{\pgfqpoint{5.824770in}{13.635509in}}%
\pgfpathlineto{\pgfqpoint{5.947993in}{13.632317in}}%
\pgfpathlineto{\pgfqpoint{6.071216in}{13.629203in}}%
\pgfpathlineto{\pgfqpoint{6.194439in}{13.626163in}}%
\pgfpathlineto{\pgfqpoint{6.317662in}{13.623195in}}%
\pgfpathlineto{\pgfqpoint{6.440885in}{13.620293in}}%
\pgfpathlineto{\pgfqpoint{6.564108in}{13.617452in}}%
\pgfpathlineto{\pgfqpoint{6.687330in}{13.614667in}}%
\pgfpathlineto{\pgfqpoint{6.810553in}{13.611937in}}%
\pgfpathlineto{\pgfqpoint{6.933776in}{13.609260in}}%
\pgfpathlineto{\pgfqpoint{7.056999in}{13.606635in}}%
\pgfpathlineto{\pgfqpoint{7.180222in}{13.604060in}}%
\pgfpathlineto{\pgfqpoint{7.365057in}{13.601533in}}%
\pgfpathlineto{\pgfqpoint{7.365057in}{14.069445in}}%
\pgfpathlineto{\pgfqpoint{7.365057in}{14.069445in}}%
\pgfpathlineto{\pgfqpoint{7.180222in}{14.066918in}}%
\pgfpathlineto{\pgfqpoint{7.056999in}{14.064343in}}%
\pgfpathlineto{\pgfqpoint{6.933776in}{14.061718in}}%
\pgfpathlineto{\pgfqpoint{6.810553in}{14.059041in}}%
\pgfpathlineto{\pgfqpoint{6.687330in}{14.056311in}}%
\pgfpathlineto{\pgfqpoint{6.564108in}{14.053526in}}%
\pgfpathlineto{\pgfqpoint{6.440885in}{14.050685in}}%
\pgfpathlineto{\pgfqpoint{6.317662in}{14.047783in}}%
\pgfpathlineto{\pgfqpoint{6.194439in}{14.044815in}}%
\pgfpathlineto{\pgfqpoint{6.071216in}{14.041775in}}%
\pgfpathlineto{\pgfqpoint{5.947993in}{14.038661in}}%
\pgfpathlineto{\pgfqpoint{5.824770in}{14.035469in}}%
\pgfpathlineto{\pgfqpoint{5.701547in}{14.032197in}}%
\pgfpathlineto{\pgfqpoint{5.578324in}{14.028838in}}%
\pgfpathlineto{\pgfqpoint{5.455101in}{14.025391in}}%
\pgfpathlineto{\pgfqpoint{5.331878in}{14.021849in}}%
\pgfpathlineto{\pgfqpoint{5.208655in}{14.018206in}}%
\pgfpathlineto{\pgfqpoint{5.085432in}{14.014458in}}%
\pgfpathlineto{\pgfqpoint{4.962209in}{14.010597in}}%
\pgfpathlineto{\pgfqpoint{4.838986in}{14.006615in}}%
\pgfpathlineto{\pgfqpoint{4.715763in}{14.002501in}}%
\pgfpathlineto{\pgfqpoint{4.592540in}{13.998246in}}%
\pgfpathlineto{\pgfqpoint{4.469317in}{13.993837in}}%
\pgfpathlineto{\pgfqpoint{4.346094in}{13.989262in}}%
\pgfpathlineto{\pgfqpoint{4.222871in}{13.984507in}}%
\pgfpathlineto{\pgfqpoint{4.099648in}{13.979551in}}%
\pgfpathlineto{\pgfqpoint{3.976425in}{13.974374in}}%
\pgfpathlineto{\pgfqpoint{3.853202in}{13.968954in}}%
\pgfpathlineto{\pgfqpoint{3.729979in}{13.963260in}}%
\pgfpathlineto{\pgfqpoint{3.606756in}{13.957253in}}%
\pgfpathlineto{\pgfqpoint{3.483533in}{13.950887in}}%
\pgfpathlineto{\pgfqpoint{3.360310in}{13.944095in}}%
\pgfpathlineto{\pgfqpoint{3.237087in}{13.936794in}}%
\pgfpathlineto{\pgfqpoint{3.113864in}{13.928861in}}%
\pgfpathlineto{\pgfqpoint{2.990641in}{13.920116in}}%
\pgfpathlineto{\pgfqpoint{2.867418in}{13.910275in}}%
\pgfpathlineto{\pgfqpoint{2.744195in}{13.898826in}}%
\pgfpathlineto{\pgfqpoint{2.620972in}{13.884658in}}%
\pgfpathlineto{\pgfqpoint{2.436138in}{13.863953in}}%
\pgfpathclose%
\pgfusepath{stroke,fill}%
\end{pgfscope}%
\begin{pgfscope}%
\pgfpathrectangle{\pgfqpoint{2.125000in}{13.561628in}}{\pgfqpoint{5.489583in}{0.877907in}}%
\pgfusepath{clip}%
\pgfsetbuttcap%
\pgfsetroundjoin%
\pgfsetlinewidth{1.505625pt}%
\definecolor{currentstroke}{rgb}{0.000000,0.000000,0.000000}%
\pgfsetstrokecolor{currentstroke}%
\pgfsetdash{}{0pt}%
\pgfpathmoveto{\pgfqpoint{2.374527in}{13.835489in}}%
\pgfpathlineto{\pgfqpoint{2.374527in}{14.399630in}}%
\pgfusepath{stroke}%
\end{pgfscope}%
\begin{pgfscope}%
\pgfpathrectangle{\pgfqpoint{2.125000in}{13.561628in}}{\pgfqpoint{5.489583in}{0.877907in}}%
\pgfusepath{clip}%
\pgfsetbuttcap%
\pgfsetroundjoin%
\pgfsetlinewidth{1.505625pt}%
\definecolor{currentstroke}{rgb}{0.000000,0.000000,0.000000}%
\pgfsetstrokecolor{currentstroke}%
\pgfsetdash{}{0pt}%
\pgfpathmoveto{\pgfqpoint{2.497749in}{13.835489in}}%
\pgfpathlineto{\pgfqpoint{2.497749in}{14.397362in}}%
\pgfusepath{stroke}%
\end{pgfscope}%
\begin{pgfscope}%
\pgfpathrectangle{\pgfqpoint{2.125000in}{13.561628in}}{\pgfqpoint{5.489583in}{0.877907in}}%
\pgfusepath{clip}%
\pgfsetbuttcap%
\pgfsetroundjoin%
\pgfsetlinewidth{1.505625pt}%
\definecolor{currentstroke}{rgb}{0.000000,0.000000,0.000000}%
\pgfsetstrokecolor{currentstroke}%
\pgfsetdash{}{0pt}%
\pgfpathmoveto{\pgfqpoint{2.620972in}{13.835489in}}%
\pgfpathlineto{\pgfqpoint{2.620972in}{14.395030in}}%
\pgfusepath{stroke}%
\end{pgfscope}%
\begin{pgfscope}%
\pgfpathrectangle{\pgfqpoint{2.125000in}{13.561628in}}{\pgfqpoint{5.489583in}{0.877907in}}%
\pgfusepath{clip}%
\pgfsetbuttcap%
\pgfsetroundjoin%
\pgfsetlinewidth{1.505625pt}%
\definecolor{currentstroke}{rgb}{0.000000,0.000000,0.000000}%
\pgfsetstrokecolor{currentstroke}%
\pgfsetdash{}{0pt}%
\pgfpathmoveto{\pgfqpoint{2.744195in}{13.835489in}}%
\pgfpathlineto{\pgfqpoint{2.744195in}{14.392807in}}%
\pgfusepath{stroke}%
\end{pgfscope}%
\begin{pgfscope}%
\pgfpathrectangle{\pgfqpoint{2.125000in}{13.561628in}}{\pgfqpoint{5.489583in}{0.877907in}}%
\pgfusepath{clip}%
\pgfsetbuttcap%
\pgfsetroundjoin%
\pgfsetlinewidth{1.505625pt}%
\definecolor{currentstroke}{rgb}{0.000000,0.000000,0.000000}%
\pgfsetstrokecolor{currentstroke}%
\pgfsetdash{}{0pt}%
\pgfpathmoveto{\pgfqpoint{2.867418in}{13.835489in}}%
\pgfpathlineto{\pgfqpoint{2.867418in}{14.390575in}}%
\pgfusepath{stroke}%
\end{pgfscope}%
\begin{pgfscope}%
\pgfpathrectangle{\pgfqpoint{2.125000in}{13.561628in}}{\pgfqpoint{5.489583in}{0.877907in}}%
\pgfusepath{clip}%
\pgfsetbuttcap%
\pgfsetroundjoin%
\pgfsetlinewidth{1.505625pt}%
\definecolor{currentstroke}{rgb}{0.000000,0.000000,0.000000}%
\pgfsetstrokecolor{currentstroke}%
\pgfsetdash{}{0pt}%
\pgfpathmoveto{\pgfqpoint{2.990641in}{13.835489in}}%
\pgfpathlineto{\pgfqpoint{2.990641in}{14.388411in}}%
\pgfusepath{stroke}%
\end{pgfscope}%
\begin{pgfscope}%
\pgfpathrectangle{\pgfqpoint{2.125000in}{13.561628in}}{\pgfqpoint{5.489583in}{0.877907in}}%
\pgfusepath{clip}%
\pgfsetbuttcap%
\pgfsetroundjoin%
\pgfsetlinewidth{1.505625pt}%
\definecolor{currentstroke}{rgb}{0.000000,0.000000,0.000000}%
\pgfsetstrokecolor{currentstroke}%
\pgfsetdash{}{0pt}%
\pgfpathmoveto{\pgfqpoint{3.113864in}{13.835489in}}%
\pgfpathlineto{\pgfqpoint{3.113864in}{14.386256in}}%
\pgfusepath{stroke}%
\end{pgfscope}%
\begin{pgfscope}%
\pgfpathrectangle{\pgfqpoint{2.125000in}{13.561628in}}{\pgfqpoint{5.489583in}{0.877907in}}%
\pgfusepath{clip}%
\pgfsetbuttcap%
\pgfsetroundjoin%
\pgfsetlinewidth{1.505625pt}%
\definecolor{currentstroke}{rgb}{0.000000,0.000000,0.000000}%
\pgfsetstrokecolor{currentstroke}%
\pgfsetdash{}{0pt}%
\pgfpathmoveto{\pgfqpoint{3.237087in}{13.835489in}}%
\pgfpathlineto{\pgfqpoint{3.237087in}{14.384126in}}%
\pgfusepath{stroke}%
\end{pgfscope}%
\begin{pgfscope}%
\pgfpathrectangle{\pgfqpoint{2.125000in}{13.561628in}}{\pgfqpoint{5.489583in}{0.877907in}}%
\pgfusepath{clip}%
\pgfsetbuttcap%
\pgfsetroundjoin%
\pgfsetlinewidth{1.505625pt}%
\definecolor{currentstroke}{rgb}{0.000000,0.000000,0.000000}%
\pgfsetstrokecolor{currentstroke}%
\pgfsetdash{}{0pt}%
\pgfpathmoveto{\pgfqpoint{3.360310in}{13.835489in}}%
\pgfpathlineto{\pgfqpoint{3.360310in}{14.382124in}}%
\pgfusepath{stroke}%
\end{pgfscope}%
\begin{pgfscope}%
\pgfpathrectangle{\pgfqpoint{2.125000in}{13.561628in}}{\pgfqpoint{5.489583in}{0.877907in}}%
\pgfusepath{clip}%
\pgfsetbuttcap%
\pgfsetroundjoin%
\pgfsetlinewidth{1.505625pt}%
\definecolor{currentstroke}{rgb}{0.000000,0.000000,0.000000}%
\pgfsetstrokecolor{currentstroke}%
\pgfsetdash{}{0pt}%
\pgfpathmoveto{\pgfqpoint{3.483533in}{13.835489in}}%
\pgfpathlineto{\pgfqpoint{3.483533in}{14.380057in}}%
\pgfusepath{stroke}%
\end{pgfscope}%
\begin{pgfscope}%
\pgfpathrectangle{\pgfqpoint{2.125000in}{13.561628in}}{\pgfqpoint{5.489583in}{0.877907in}}%
\pgfusepath{clip}%
\pgfsetbuttcap%
\pgfsetroundjoin%
\pgfsetlinewidth{1.505625pt}%
\definecolor{currentstroke}{rgb}{0.000000,0.000000,0.000000}%
\pgfsetstrokecolor{currentstroke}%
\pgfsetdash{}{0pt}%
\pgfpathmoveto{\pgfqpoint{3.606756in}{13.835489in}}%
\pgfpathlineto{\pgfqpoint{3.606756in}{14.378057in}}%
\pgfusepath{stroke}%
\end{pgfscope}%
\begin{pgfscope}%
\pgfpathrectangle{\pgfqpoint{2.125000in}{13.561628in}}{\pgfqpoint{5.489583in}{0.877907in}}%
\pgfusepath{clip}%
\pgfsetbuttcap%
\pgfsetroundjoin%
\pgfsetlinewidth{1.505625pt}%
\definecolor{currentstroke}{rgb}{0.000000,0.000000,0.000000}%
\pgfsetstrokecolor{currentstroke}%
\pgfsetdash{}{0pt}%
\pgfpathmoveto{\pgfqpoint{3.729979in}{13.835489in}}%
\pgfpathlineto{\pgfqpoint{3.729979in}{14.376005in}}%
\pgfusepath{stroke}%
\end{pgfscope}%
\begin{pgfscope}%
\pgfpathrectangle{\pgfqpoint{2.125000in}{13.561628in}}{\pgfqpoint{5.489583in}{0.877907in}}%
\pgfusepath{clip}%
\pgfsetbuttcap%
\pgfsetroundjoin%
\pgfsetlinewidth{1.505625pt}%
\definecolor{currentstroke}{rgb}{0.000000,0.000000,0.000000}%
\pgfsetstrokecolor{currentstroke}%
\pgfsetdash{}{0pt}%
\pgfpathmoveto{\pgfqpoint{3.853202in}{13.835489in}}%
\pgfpathlineto{\pgfqpoint{3.853202in}{14.373965in}}%
\pgfusepath{stroke}%
\end{pgfscope}%
\begin{pgfscope}%
\pgfpathrectangle{\pgfqpoint{2.125000in}{13.561628in}}{\pgfqpoint{5.489583in}{0.877907in}}%
\pgfusepath{clip}%
\pgfsetbuttcap%
\pgfsetroundjoin%
\pgfsetlinewidth{1.505625pt}%
\definecolor{currentstroke}{rgb}{0.000000,0.000000,0.000000}%
\pgfsetstrokecolor{currentstroke}%
\pgfsetdash{}{0pt}%
\pgfpathmoveto{\pgfqpoint{3.976425in}{13.835489in}}%
\pgfpathlineto{\pgfqpoint{3.976425in}{14.371866in}}%
\pgfusepath{stroke}%
\end{pgfscope}%
\begin{pgfscope}%
\pgfpathrectangle{\pgfqpoint{2.125000in}{13.561628in}}{\pgfqpoint{5.489583in}{0.877907in}}%
\pgfusepath{clip}%
\pgfsetbuttcap%
\pgfsetroundjoin%
\pgfsetlinewidth{1.505625pt}%
\definecolor{currentstroke}{rgb}{0.000000,0.000000,0.000000}%
\pgfsetstrokecolor{currentstroke}%
\pgfsetdash{}{0pt}%
\pgfpathmoveto{\pgfqpoint{4.099648in}{13.835489in}}%
\pgfpathlineto{\pgfqpoint{4.099648in}{14.369584in}}%
\pgfusepath{stroke}%
\end{pgfscope}%
\begin{pgfscope}%
\pgfpathrectangle{\pgfqpoint{2.125000in}{13.561628in}}{\pgfqpoint{5.489583in}{0.877907in}}%
\pgfusepath{clip}%
\pgfsetbuttcap%
\pgfsetroundjoin%
\pgfsetlinewidth{1.505625pt}%
\definecolor{currentstroke}{rgb}{0.000000,0.000000,0.000000}%
\pgfsetstrokecolor{currentstroke}%
\pgfsetdash{}{0pt}%
\pgfpathmoveto{\pgfqpoint{4.222871in}{13.835489in}}%
\pgfpathlineto{\pgfqpoint{4.222871in}{14.367296in}}%
\pgfusepath{stroke}%
\end{pgfscope}%
\begin{pgfscope}%
\pgfpathrectangle{\pgfqpoint{2.125000in}{13.561628in}}{\pgfqpoint{5.489583in}{0.877907in}}%
\pgfusepath{clip}%
\pgfsetbuttcap%
\pgfsetroundjoin%
\pgfsetlinewidth{1.505625pt}%
\definecolor{currentstroke}{rgb}{0.000000,0.000000,0.000000}%
\pgfsetstrokecolor{currentstroke}%
\pgfsetdash{}{0pt}%
\pgfpathmoveto{\pgfqpoint{4.346094in}{13.835489in}}%
\pgfpathlineto{\pgfqpoint{4.346094in}{14.365092in}}%
\pgfusepath{stroke}%
\end{pgfscope}%
\begin{pgfscope}%
\pgfpathrectangle{\pgfqpoint{2.125000in}{13.561628in}}{\pgfqpoint{5.489583in}{0.877907in}}%
\pgfusepath{clip}%
\pgfsetbuttcap%
\pgfsetroundjoin%
\pgfsetlinewidth{1.505625pt}%
\definecolor{currentstroke}{rgb}{0.000000,0.000000,0.000000}%
\pgfsetstrokecolor{currentstroke}%
\pgfsetdash{}{0pt}%
\pgfpathmoveto{\pgfqpoint{4.469317in}{13.835489in}}%
\pgfpathlineto{\pgfqpoint{4.469317in}{14.362772in}}%
\pgfusepath{stroke}%
\end{pgfscope}%
\begin{pgfscope}%
\pgfpathrectangle{\pgfqpoint{2.125000in}{13.561628in}}{\pgfqpoint{5.489583in}{0.877907in}}%
\pgfusepath{clip}%
\pgfsetbuttcap%
\pgfsetroundjoin%
\pgfsetlinewidth{1.505625pt}%
\definecolor{currentstroke}{rgb}{0.000000,0.000000,0.000000}%
\pgfsetstrokecolor{currentstroke}%
\pgfsetdash{}{0pt}%
\pgfpathmoveto{\pgfqpoint{4.592540in}{13.835489in}}%
\pgfpathlineto{\pgfqpoint{4.592540in}{14.360499in}}%
\pgfusepath{stroke}%
\end{pgfscope}%
\begin{pgfscope}%
\pgfpathrectangle{\pgfqpoint{2.125000in}{13.561628in}}{\pgfqpoint{5.489583in}{0.877907in}}%
\pgfusepath{clip}%
\pgfsetbuttcap%
\pgfsetroundjoin%
\pgfsetlinewidth{1.505625pt}%
\definecolor{currentstroke}{rgb}{0.000000,0.000000,0.000000}%
\pgfsetstrokecolor{currentstroke}%
\pgfsetdash{}{0pt}%
\pgfpathmoveto{\pgfqpoint{4.715763in}{13.835489in}}%
\pgfpathlineto{\pgfqpoint{4.715763in}{14.358170in}}%
\pgfusepath{stroke}%
\end{pgfscope}%
\begin{pgfscope}%
\pgfpathrectangle{\pgfqpoint{2.125000in}{13.561628in}}{\pgfqpoint{5.489583in}{0.877907in}}%
\pgfusepath{clip}%
\pgfsetbuttcap%
\pgfsetroundjoin%
\pgfsetlinewidth{1.505625pt}%
\definecolor{currentstroke}{rgb}{0.000000,0.000000,0.000000}%
\pgfsetstrokecolor{currentstroke}%
\pgfsetdash{}{0pt}%
\pgfpathmoveto{\pgfqpoint{4.838986in}{13.835489in}}%
\pgfpathlineto{\pgfqpoint{4.838986in}{14.355879in}}%
\pgfusepath{stroke}%
\end{pgfscope}%
\begin{pgfscope}%
\pgfpathrectangle{\pgfqpoint{2.125000in}{13.561628in}}{\pgfqpoint{5.489583in}{0.877907in}}%
\pgfusepath{clip}%
\pgfsetbuttcap%
\pgfsetroundjoin%
\pgfsetlinewidth{1.505625pt}%
\definecolor{currentstroke}{rgb}{0.000000,0.000000,0.000000}%
\pgfsetstrokecolor{currentstroke}%
\pgfsetdash{}{0pt}%
\pgfpathmoveto{\pgfqpoint{4.962209in}{13.835489in}}%
\pgfpathlineto{\pgfqpoint{4.962209in}{14.353680in}}%
\pgfusepath{stroke}%
\end{pgfscope}%
\begin{pgfscope}%
\pgfpathrectangle{\pgfqpoint{2.125000in}{13.561628in}}{\pgfqpoint{5.489583in}{0.877907in}}%
\pgfusepath{clip}%
\pgfsetbuttcap%
\pgfsetroundjoin%
\pgfsetlinewidth{1.505625pt}%
\definecolor{currentstroke}{rgb}{0.000000,0.000000,0.000000}%
\pgfsetstrokecolor{currentstroke}%
\pgfsetdash{}{0pt}%
\pgfpathmoveto{\pgfqpoint{5.085432in}{13.835489in}}%
\pgfpathlineto{\pgfqpoint{5.085432in}{14.351460in}}%
\pgfusepath{stroke}%
\end{pgfscope}%
\begin{pgfscope}%
\pgfpathrectangle{\pgfqpoint{2.125000in}{13.561628in}}{\pgfqpoint{5.489583in}{0.877907in}}%
\pgfusepath{clip}%
\pgfsetbuttcap%
\pgfsetroundjoin%
\pgfsetlinewidth{1.505625pt}%
\definecolor{currentstroke}{rgb}{0.000000,0.000000,0.000000}%
\pgfsetstrokecolor{currentstroke}%
\pgfsetdash{}{0pt}%
\pgfpathmoveto{\pgfqpoint{5.208655in}{13.835489in}}%
\pgfpathlineto{\pgfqpoint{5.208655in}{14.349347in}}%
\pgfusepath{stroke}%
\end{pgfscope}%
\begin{pgfscope}%
\pgfpathrectangle{\pgfqpoint{2.125000in}{13.561628in}}{\pgfqpoint{5.489583in}{0.877907in}}%
\pgfusepath{clip}%
\pgfsetbuttcap%
\pgfsetroundjoin%
\pgfsetlinewidth{1.505625pt}%
\definecolor{currentstroke}{rgb}{0.000000,0.000000,0.000000}%
\pgfsetstrokecolor{currentstroke}%
\pgfsetdash{}{0pt}%
\pgfpathmoveto{\pgfqpoint{5.331878in}{13.835489in}}%
\pgfpathlineto{\pgfqpoint{5.331878in}{14.347147in}}%
\pgfusepath{stroke}%
\end{pgfscope}%
\begin{pgfscope}%
\pgfpathrectangle{\pgfqpoint{2.125000in}{13.561628in}}{\pgfqpoint{5.489583in}{0.877907in}}%
\pgfusepath{clip}%
\pgfsetbuttcap%
\pgfsetroundjoin%
\pgfsetlinewidth{1.505625pt}%
\definecolor{currentstroke}{rgb}{0.000000,0.000000,0.000000}%
\pgfsetstrokecolor{currentstroke}%
\pgfsetdash{}{0pt}%
\pgfpathmoveto{\pgfqpoint{5.455101in}{13.835489in}}%
\pgfpathlineto{\pgfqpoint{5.455101in}{14.344897in}}%
\pgfusepath{stroke}%
\end{pgfscope}%
\begin{pgfscope}%
\pgfpathrectangle{\pgfqpoint{2.125000in}{13.561628in}}{\pgfqpoint{5.489583in}{0.877907in}}%
\pgfusepath{clip}%
\pgfsetbuttcap%
\pgfsetroundjoin%
\pgfsetlinewidth{1.505625pt}%
\definecolor{currentstroke}{rgb}{0.000000,0.000000,0.000000}%
\pgfsetstrokecolor{currentstroke}%
\pgfsetdash{}{0pt}%
\pgfpathmoveto{\pgfqpoint{5.578324in}{13.835489in}}%
\pgfpathlineto{\pgfqpoint{5.578324in}{14.342708in}}%
\pgfusepath{stroke}%
\end{pgfscope}%
\begin{pgfscope}%
\pgfpathrectangle{\pgfqpoint{2.125000in}{13.561628in}}{\pgfqpoint{5.489583in}{0.877907in}}%
\pgfusepath{clip}%
\pgfsetbuttcap%
\pgfsetroundjoin%
\pgfsetlinewidth{1.505625pt}%
\definecolor{currentstroke}{rgb}{0.000000,0.000000,0.000000}%
\pgfsetstrokecolor{currentstroke}%
\pgfsetdash{}{0pt}%
\pgfpathmoveto{\pgfqpoint{5.701547in}{13.835489in}}%
\pgfpathlineto{\pgfqpoint{5.701547in}{14.340452in}}%
\pgfusepath{stroke}%
\end{pgfscope}%
\begin{pgfscope}%
\pgfpathrectangle{\pgfqpoint{2.125000in}{13.561628in}}{\pgfqpoint{5.489583in}{0.877907in}}%
\pgfusepath{clip}%
\pgfsetbuttcap%
\pgfsetroundjoin%
\pgfsetlinewidth{1.505625pt}%
\definecolor{currentstroke}{rgb}{0.000000,0.000000,0.000000}%
\pgfsetstrokecolor{currentstroke}%
\pgfsetdash{}{0pt}%
\pgfpathmoveto{\pgfqpoint{5.824770in}{13.835489in}}%
\pgfpathlineto{\pgfqpoint{5.824770in}{14.338206in}}%
\pgfusepath{stroke}%
\end{pgfscope}%
\begin{pgfscope}%
\pgfpathrectangle{\pgfqpoint{2.125000in}{13.561628in}}{\pgfqpoint{5.489583in}{0.877907in}}%
\pgfusepath{clip}%
\pgfsetbuttcap%
\pgfsetroundjoin%
\pgfsetlinewidth{1.505625pt}%
\definecolor{currentstroke}{rgb}{0.000000,0.000000,0.000000}%
\pgfsetstrokecolor{currentstroke}%
\pgfsetdash{}{0pt}%
\pgfpathmoveto{\pgfqpoint{5.947993in}{13.835489in}}%
\pgfpathlineto{\pgfqpoint{5.947993in}{14.335944in}}%
\pgfusepath{stroke}%
\end{pgfscope}%
\begin{pgfscope}%
\pgfpathrectangle{\pgfqpoint{2.125000in}{13.561628in}}{\pgfqpoint{5.489583in}{0.877907in}}%
\pgfusepath{clip}%
\pgfsetbuttcap%
\pgfsetroundjoin%
\pgfsetlinewidth{1.505625pt}%
\definecolor{currentstroke}{rgb}{0.000000,0.000000,0.000000}%
\pgfsetstrokecolor{currentstroke}%
\pgfsetdash{}{0pt}%
\pgfpathmoveto{\pgfqpoint{6.071216in}{13.835489in}}%
\pgfpathlineto{\pgfqpoint{6.071216in}{14.333612in}}%
\pgfusepath{stroke}%
\end{pgfscope}%
\begin{pgfscope}%
\pgfpathrectangle{\pgfqpoint{2.125000in}{13.561628in}}{\pgfqpoint{5.489583in}{0.877907in}}%
\pgfusepath{clip}%
\pgfsetbuttcap%
\pgfsetroundjoin%
\pgfsetlinewidth{1.505625pt}%
\definecolor{currentstroke}{rgb}{0.000000,0.000000,0.000000}%
\pgfsetstrokecolor{currentstroke}%
\pgfsetdash{}{0pt}%
\pgfpathmoveto{\pgfqpoint{6.194439in}{13.835489in}}%
\pgfpathlineto{\pgfqpoint{6.194439in}{14.331311in}}%
\pgfusepath{stroke}%
\end{pgfscope}%
\begin{pgfscope}%
\pgfpathrectangle{\pgfqpoint{2.125000in}{13.561628in}}{\pgfqpoint{5.489583in}{0.877907in}}%
\pgfusepath{clip}%
\pgfsetbuttcap%
\pgfsetroundjoin%
\pgfsetlinewidth{1.505625pt}%
\definecolor{currentstroke}{rgb}{0.000000,0.000000,0.000000}%
\pgfsetstrokecolor{currentstroke}%
\pgfsetdash{}{0pt}%
\pgfpathmoveto{\pgfqpoint{6.317662in}{13.835489in}}%
\pgfpathlineto{\pgfqpoint{6.317662in}{14.329089in}}%
\pgfusepath{stroke}%
\end{pgfscope}%
\begin{pgfscope}%
\pgfpathrectangle{\pgfqpoint{2.125000in}{13.561628in}}{\pgfqpoint{5.489583in}{0.877907in}}%
\pgfusepath{clip}%
\pgfsetbuttcap%
\pgfsetroundjoin%
\pgfsetlinewidth{1.505625pt}%
\definecolor{currentstroke}{rgb}{0.000000,0.000000,0.000000}%
\pgfsetstrokecolor{currentstroke}%
\pgfsetdash{}{0pt}%
\pgfpathmoveto{\pgfqpoint{6.440885in}{13.835489in}}%
\pgfpathlineto{\pgfqpoint{6.440885in}{14.327145in}}%
\pgfusepath{stroke}%
\end{pgfscope}%
\begin{pgfscope}%
\pgfpathrectangle{\pgfqpoint{2.125000in}{13.561628in}}{\pgfqpoint{5.489583in}{0.877907in}}%
\pgfusepath{clip}%
\pgfsetbuttcap%
\pgfsetroundjoin%
\pgfsetlinewidth{1.505625pt}%
\definecolor{currentstroke}{rgb}{0.000000,0.000000,0.000000}%
\pgfsetstrokecolor{currentstroke}%
\pgfsetdash{}{0pt}%
\pgfpathmoveto{\pgfqpoint{6.564108in}{13.835489in}}%
\pgfpathlineto{\pgfqpoint{6.564108in}{14.325463in}}%
\pgfusepath{stroke}%
\end{pgfscope}%
\begin{pgfscope}%
\pgfpathrectangle{\pgfqpoint{2.125000in}{13.561628in}}{\pgfqpoint{5.489583in}{0.877907in}}%
\pgfusepath{clip}%
\pgfsetbuttcap%
\pgfsetroundjoin%
\pgfsetlinewidth{1.505625pt}%
\definecolor{currentstroke}{rgb}{0.000000,0.000000,0.000000}%
\pgfsetstrokecolor{currentstroke}%
\pgfsetdash{}{0pt}%
\pgfpathmoveto{\pgfqpoint{6.687330in}{13.835489in}}%
\pgfpathlineto{\pgfqpoint{6.687330in}{14.323599in}}%
\pgfusepath{stroke}%
\end{pgfscope}%
\begin{pgfscope}%
\pgfpathrectangle{\pgfqpoint{2.125000in}{13.561628in}}{\pgfqpoint{5.489583in}{0.877907in}}%
\pgfusepath{clip}%
\pgfsetbuttcap%
\pgfsetroundjoin%
\pgfsetlinewidth{1.505625pt}%
\definecolor{currentstroke}{rgb}{0.000000,0.000000,0.000000}%
\pgfsetstrokecolor{currentstroke}%
\pgfsetdash{}{0pt}%
\pgfpathmoveto{\pgfqpoint{6.810553in}{13.835489in}}%
\pgfpathlineto{\pgfqpoint{6.810553in}{14.321759in}}%
\pgfusepath{stroke}%
\end{pgfscope}%
\begin{pgfscope}%
\pgfpathrectangle{\pgfqpoint{2.125000in}{13.561628in}}{\pgfqpoint{5.489583in}{0.877907in}}%
\pgfusepath{clip}%
\pgfsetbuttcap%
\pgfsetroundjoin%
\pgfsetlinewidth{1.505625pt}%
\definecolor{currentstroke}{rgb}{0.000000,0.000000,0.000000}%
\pgfsetstrokecolor{currentstroke}%
\pgfsetdash{}{0pt}%
\pgfpathmoveto{\pgfqpoint{6.933776in}{13.835489in}}%
\pgfpathlineto{\pgfqpoint{6.933776in}{14.319873in}}%
\pgfusepath{stroke}%
\end{pgfscope}%
\begin{pgfscope}%
\pgfpathrectangle{\pgfqpoint{2.125000in}{13.561628in}}{\pgfqpoint{5.489583in}{0.877907in}}%
\pgfusepath{clip}%
\pgfsetbuttcap%
\pgfsetroundjoin%
\pgfsetlinewidth{1.505625pt}%
\definecolor{currentstroke}{rgb}{0.000000,0.000000,0.000000}%
\pgfsetstrokecolor{currentstroke}%
\pgfsetdash{}{0pt}%
\pgfpathmoveto{\pgfqpoint{7.056999in}{13.835489in}}%
\pgfpathlineto{\pgfqpoint{7.056999in}{14.317976in}}%
\pgfusepath{stroke}%
\end{pgfscope}%
\begin{pgfscope}%
\pgfpathrectangle{\pgfqpoint{2.125000in}{13.561628in}}{\pgfqpoint{5.489583in}{0.877907in}}%
\pgfusepath{clip}%
\pgfsetbuttcap%
\pgfsetroundjoin%
\pgfsetlinewidth{1.505625pt}%
\definecolor{currentstroke}{rgb}{0.000000,0.000000,0.000000}%
\pgfsetstrokecolor{currentstroke}%
\pgfsetdash{}{0pt}%
\pgfpathmoveto{\pgfqpoint{7.180222in}{13.835489in}}%
\pgfpathlineto{\pgfqpoint{7.180222in}{14.316161in}}%
\pgfusepath{stroke}%
\end{pgfscope}%
\begin{pgfscope}%
\pgfpathrectangle{\pgfqpoint{2.125000in}{13.561628in}}{\pgfqpoint{5.489583in}{0.877907in}}%
\pgfusepath{clip}%
\pgfsetbuttcap%
\pgfsetroundjoin%
\pgfsetlinewidth{1.505625pt}%
\definecolor{currentstroke}{rgb}{0.000000,0.000000,0.000000}%
\pgfsetstrokecolor{currentstroke}%
\pgfsetdash{}{0pt}%
\pgfpathmoveto{\pgfqpoint{7.303445in}{13.835489in}}%
\pgfpathlineto{\pgfqpoint{7.303445in}{14.314258in}}%
\pgfusepath{stroke}%
\end{pgfscope}%
\begin{pgfscope}%
\pgfpathrectangle{\pgfqpoint{2.125000in}{13.561628in}}{\pgfqpoint{5.489583in}{0.877907in}}%
\pgfusepath{clip}%
\pgfsetroundcap%
\pgfsetroundjoin%
\pgfsetlinewidth{1.505625pt}%
\definecolor{currentstroke}{rgb}{0.121569,0.466667,0.705882}%
\pgfsetstrokecolor{currentstroke}%
\pgfsetdash{}{0pt}%
\pgfpathmoveto{\pgfqpoint{2.125000in}{13.835489in}}%
\pgfpathlineto{\pgfqpoint{7.614583in}{13.835489in}}%
\pgfusepath{stroke}%
\end{pgfscope}%
\begin{pgfscope}%
\pgfpathrectangle{\pgfqpoint{2.125000in}{13.561628in}}{\pgfqpoint{5.489583in}{0.877907in}}%
\pgfusepath{clip}%
\pgfsetbuttcap%
\pgfsetroundjoin%
\definecolor{currentfill}{rgb}{0.121569,0.466667,0.705882}%
\pgfsetfillcolor{currentfill}%
\pgfsetlinewidth{1.003750pt}%
\definecolor{currentstroke}{rgb}{0.121569,0.466667,0.705882}%
\pgfsetstrokecolor{currentstroke}%
\pgfsetdash{}{0pt}%
\pgfsys@defobject{currentmarker}{\pgfqpoint{-0.034722in}{-0.034722in}}{\pgfqpoint{0.034722in}{0.034722in}}{%
\pgfpathmoveto{\pgfqpoint{0.000000in}{-0.034722in}}%
\pgfpathcurveto{\pgfqpoint{0.009208in}{-0.034722in}}{\pgfqpoint{0.018041in}{-0.031064in}}{\pgfqpoint{0.024552in}{-0.024552in}}%
\pgfpathcurveto{\pgfqpoint{0.031064in}{-0.018041in}}{\pgfqpoint{0.034722in}{-0.009208in}}{\pgfqpoint{0.034722in}{0.000000in}}%
\pgfpathcurveto{\pgfqpoint{0.034722in}{0.009208in}}{\pgfqpoint{0.031064in}{0.018041in}}{\pgfqpoint{0.024552in}{0.024552in}}%
\pgfpathcurveto{\pgfqpoint{0.018041in}{0.031064in}}{\pgfqpoint{0.009208in}{0.034722in}}{\pgfqpoint{0.000000in}{0.034722in}}%
\pgfpathcurveto{\pgfqpoint{-0.009208in}{0.034722in}}{\pgfqpoint{-0.018041in}{0.031064in}}{\pgfqpoint{-0.024552in}{0.024552in}}%
\pgfpathcurveto{\pgfqpoint{-0.031064in}{0.018041in}}{\pgfqpoint{-0.034722in}{0.009208in}}{\pgfqpoint{-0.034722in}{0.000000in}}%
\pgfpathcurveto{\pgfqpoint{-0.034722in}{-0.009208in}}{\pgfqpoint{-0.031064in}{-0.018041in}}{\pgfqpoint{-0.024552in}{-0.024552in}}%
\pgfpathcurveto{\pgfqpoint{-0.018041in}{-0.031064in}}{\pgfqpoint{-0.009208in}{-0.034722in}}{\pgfqpoint{0.000000in}{-0.034722in}}%
\pgfpathclose%
\pgfusepath{stroke,fill}%
}%
\begin{pgfscope}%
\pgfsys@transformshift{2.374527in}{14.399630in}%
\pgfsys@useobject{currentmarker}{}%
\end{pgfscope}%
\begin{pgfscope}%
\pgfsys@transformshift{2.497749in}{14.397362in}%
\pgfsys@useobject{currentmarker}{}%
\end{pgfscope}%
\begin{pgfscope}%
\pgfsys@transformshift{2.620972in}{14.395030in}%
\pgfsys@useobject{currentmarker}{}%
\end{pgfscope}%
\begin{pgfscope}%
\pgfsys@transformshift{2.744195in}{14.392807in}%
\pgfsys@useobject{currentmarker}{}%
\end{pgfscope}%
\begin{pgfscope}%
\pgfsys@transformshift{2.867418in}{14.390575in}%
\pgfsys@useobject{currentmarker}{}%
\end{pgfscope}%
\begin{pgfscope}%
\pgfsys@transformshift{2.990641in}{14.388411in}%
\pgfsys@useobject{currentmarker}{}%
\end{pgfscope}%
\begin{pgfscope}%
\pgfsys@transformshift{3.113864in}{14.386256in}%
\pgfsys@useobject{currentmarker}{}%
\end{pgfscope}%
\begin{pgfscope}%
\pgfsys@transformshift{3.237087in}{14.384126in}%
\pgfsys@useobject{currentmarker}{}%
\end{pgfscope}%
\begin{pgfscope}%
\pgfsys@transformshift{3.360310in}{14.382124in}%
\pgfsys@useobject{currentmarker}{}%
\end{pgfscope}%
\begin{pgfscope}%
\pgfsys@transformshift{3.483533in}{14.380057in}%
\pgfsys@useobject{currentmarker}{}%
\end{pgfscope}%
\begin{pgfscope}%
\pgfsys@transformshift{3.606756in}{14.378057in}%
\pgfsys@useobject{currentmarker}{}%
\end{pgfscope}%
\begin{pgfscope}%
\pgfsys@transformshift{3.729979in}{14.376005in}%
\pgfsys@useobject{currentmarker}{}%
\end{pgfscope}%
\begin{pgfscope}%
\pgfsys@transformshift{3.853202in}{14.373965in}%
\pgfsys@useobject{currentmarker}{}%
\end{pgfscope}%
\begin{pgfscope}%
\pgfsys@transformshift{3.976425in}{14.371866in}%
\pgfsys@useobject{currentmarker}{}%
\end{pgfscope}%
\begin{pgfscope}%
\pgfsys@transformshift{4.099648in}{14.369584in}%
\pgfsys@useobject{currentmarker}{}%
\end{pgfscope}%
\begin{pgfscope}%
\pgfsys@transformshift{4.222871in}{14.367296in}%
\pgfsys@useobject{currentmarker}{}%
\end{pgfscope}%
\begin{pgfscope}%
\pgfsys@transformshift{4.346094in}{14.365092in}%
\pgfsys@useobject{currentmarker}{}%
\end{pgfscope}%
\begin{pgfscope}%
\pgfsys@transformshift{4.469317in}{14.362772in}%
\pgfsys@useobject{currentmarker}{}%
\end{pgfscope}%
\begin{pgfscope}%
\pgfsys@transformshift{4.592540in}{14.360499in}%
\pgfsys@useobject{currentmarker}{}%
\end{pgfscope}%
\begin{pgfscope}%
\pgfsys@transformshift{4.715763in}{14.358170in}%
\pgfsys@useobject{currentmarker}{}%
\end{pgfscope}%
\begin{pgfscope}%
\pgfsys@transformshift{4.838986in}{14.355879in}%
\pgfsys@useobject{currentmarker}{}%
\end{pgfscope}%
\begin{pgfscope}%
\pgfsys@transformshift{4.962209in}{14.353680in}%
\pgfsys@useobject{currentmarker}{}%
\end{pgfscope}%
\begin{pgfscope}%
\pgfsys@transformshift{5.085432in}{14.351460in}%
\pgfsys@useobject{currentmarker}{}%
\end{pgfscope}%
\begin{pgfscope}%
\pgfsys@transformshift{5.208655in}{14.349347in}%
\pgfsys@useobject{currentmarker}{}%
\end{pgfscope}%
\begin{pgfscope}%
\pgfsys@transformshift{5.331878in}{14.347147in}%
\pgfsys@useobject{currentmarker}{}%
\end{pgfscope}%
\begin{pgfscope}%
\pgfsys@transformshift{5.455101in}{14.344897in}%
\pgfsys@useobject{currentmarker}{}%
\end{pgfscope}%
\begin{pgfscope}%
\pgfsys@transformshift{5.578324in}{14.342708in}%
\pgfsys@useobject{currentmarker}{}%
\end{pgfscope}%
\begin{pgfscope}%
\pgfsys@transformshift{5.701547in}{14.340452in}%
\pgfsys@useobject{currentmarker}{}%
\end{pgfscope}%
\begin{pgfscope}%
\pgfsys@transformshift{5.824770in}{14.338206in}%
\pgfsys@useobject{currentmarker}{}%
\end{pgfscope}%
\begin{pgfscope}%
\pgfsys@transformshift{5.947993in}{14.335944in}%
\pgfsys@useobject{currentmarker}{}%
\end{pgfscope}%
\begin{pgfscope}%
\pgfsys@transformshift{6.071216in}{14.333612in}%
\pgfsys@useobject{currentmarker}{}%
\end{pgfscope}%
\begin{pgfscope}%
\pgfsys@transformshift{6.194439in}{14.331311in}%
\pgfsys@useobject{currentmarker}{}%
\end{pgfscope}%
\begin{pgfscope}%
\pgfsys@transformshift{6.317662in}{14.329089in}%
\pgfsys@useobject{currentmarker}{}%
\end{pgfscope}%
\begin{pgfscope}%
\pgfsys@transformshift{6.440885in}{14.327145in}%
\pgfsys@useobject{currentmarker}{}%
\end{pgfscope}%
\begin{pgfscope}%
\pgfsys@transformshift{6.564108in}{14.325463in}%
\pgfsys@useobject{currentmarker}{}%
\end{pgfscope}%
\begin{pgfscope}%
\pgfsys@transformshift{6.687330in}{14.323599in}%
\pgfsys@useobject{currentmarker}{}%
\end{pgfscope}%
\begin{pgfscope}%
\pgfsys@transformshift{6.810553in}{14.321759in}%
\pgfsys@useobject{currentmarker}{}%
\end{pgfscope}%
\begin{pgfscope}%
\pgfsys@transformshift{6.933776in}{14.319873in}%
\pgfsys@useobject{currentmarker}{}%
\end{pgfscope}%
\begin{pgfscope}%
\pgfsys@transformshift{7.056999in}{14.317976in}%
\pgfsys@useobject{currentmarker}{}%
\end{pgfscope}%
\begin{pgfscope}%
\pgfsys@transformshift{7.180222in}{14.316161in}%
\pgfsys@useobject{currentmarker}{}%
\end{pgfscope}%
\begin{pgfscope}%
\pgfsys@transformshift{7.303445in}{14.314258in}%
\pgfsys@useobject{currentmarker}{}%
\end{pgfscope}%
\end{pgfscope}%
\begin{pgfscope}%
\pgfsetrectcap%
\pgfsetmiterjoin%
\pgfsetlinewidth{0.803000pt}%
\definecolor{currentstroke}{rgb}{1.000000,1.000000,1.000000}%
\pgfsetstrokecolor{currentstroke}%
\pgfsetdash{}{0pt}%
\pgfpathmoveto{\pgfqpoint{2.125000in}{13.561628in}}%
\pgfpathlineto{\pgfqpoint{2.125000in}{14.439535in}}%
\pgfusepath{stroke}%
\end{pgfscope}%
\begin{pgfscope}%
\pgfsetrectcap%
\pgfsetmiterjoin%
\pgfsetlinewidth{0.803000pt}%
\definecolor{currentstroke}{rgb}{1.000000,1.000000,1.000000}%
\pgfsetstrokecolor{currentstroke}%
\pgfsetdash{}{0pt}%
\pgfpathmoveto{\pgfqpoint{7.614583in}{13.561628in}}%
\pgfpathlineto{\pgfqpoint{7.614583in}{14.439535in}}%
\pgfusepath{stroke}%
\end{pgfscope}%
\begin{pgfscope}%
\pgfsetrectcap%
\pgfsetmiterjoin%
\pgfsetlinewidth{0.803000pt}%
\definecolor{currentstroke}{rgb}{1.000000,1.000000,1.000000}%
\pgfsetstrokecolor{currentstroke}%
\pgfsetdash{}{0pt}%
\pgfpathmoveto{\pgfqpoint{2.125000in}{13.561628in}}%
\pgfpathlineto{\pgfqpoint{7.614583in}{13.561628in}}%
\pgfusepath{stroke}%
\end{pgfscope}%
\begin{pgfscope}%
\pgfsetrectcap%
\pgfsetmiterjoin%
\pgfsetlinewidth{0.803000pt}%
\definecolor{currentstroke}{rgb}{1.000000,1.000000,1.000000}%
\pgfsetstrokecolor{currentstroke}%
\pgfsetdash{}{0pt}%
\pgfpathmoveto{\pgfqpoint{2.125000in}{14.439535in}}%
\pgfpathlineto{\pgfqpoint{7.614583in}{14.439535in}}%
\pgfusepath{stroke}%
\end{pgfscope}%
\begin{pgfscope}%
\definecolor{textcolor}{rgb}{0.150000,0.150000,0.150000}%
\pgfsetstrokecolor{textcolor}%
\pgfsetfillcolor{textcolor}%
\pgftext[x=4.869792in,y=14.522868in,,base]{\color{textcolor}\rmfamily\fontsize{16.800000}{20.160000}\selectfont Autocorrelation}%
\end{pgfscope}%
\begin{pgfscope}%
\pgfsetbuttcap%
\pgfsetmiterjoin%
\definecolor{currentfill}{rgb}{0.917647,0.917647,0.949020}%
\pgfsetfillcolor{currentfill}%
\pgfsetlinewidth{0.000000pt}%
\definecolor{currentstroke}{rgb}{0.000000,0.000000,0.000000}%
\pgfsetstrokecolor{currentstroke}%
\pgfsetstrokeopacity{0.000000}%
\pgfsetdash{}{0pt}%
\pgfpathmoveto{\pgfqpoint{9.810417in}{13.561628in}}%
\pgfpathlineto{\pgfqpoint{15.300000in}{13.561628in}}%
\pgfpathlineto{\pgfqpoint{15.300000in}{14.439535in}}%
\pgfpathlineto{\pgfqpoint{9.810417in}{14.439535in}}%
\pgfpathclose%
\pgfusepath{fill}%
\end{pgfscope}%
\begin{pgfscope}%
\pgfpathrectangle{\pgfqpoint{9.810417in}{13.561628in}}{\pgfqpoint{5.489583in}{0.877907in}}%
\pgfusepath{clip}%
\pgfsetroundcap%
\pgfsetroundjoin%
\pgfsetlinewidth{0.803000pt}%
\definecolor{currentstroke}{rgb}{1.000000,1.000000,1.000000}%
\pgfsetstrokecolor{currentstroke}%
\pgfsetdash{}{0pt}%
\pgfpathmoveto{\pgfqpoint{10.059943in}{13.561628in}}%
\pgfpathlineto{\pgfqpoint{10.059943in}{14.439535in}}%
\pgfusepath{stroke}%
\end{pgfscope}%
\begin{pgfscope}%
\definecolor{textcolor}{rgb}{0.150000,0.150000,0.150000}%
\pgfsetstrokecolor{textcolor}%
\pgfsetfillcolor{textcolor}%
\pgftext[x=10.059943in,y=13.464406in,,top]{\color{textcolor}\rmfamily\fontsize{14.000000}{16.800000}\selectfont 0}%
\end{pgfscope}%
\begin{pgfscope}%
\pgfpathrectangle{\pgfqpoint{9.810417in}{13.561628in}}{\pgfqpoint{5.489583in}{0.877907in}}%
\pgfusepath{clip}%
\pgfsetroundcap%
\pgfsetroundjoin%
\pgfsetlinewidth{0.803000pt}%
\definecolor{currentstroke}{rgb}{1.000000,1.000000,1.000000}%
\pgfsetstrokecolor{currentstroke}%
\pgfsetdash{}{0pt}%
\pgfpathmoveto{\pgfqpoint{10.676058in}{13.561628in}}%
\pgfpathlineto{\pgfqpoint{10.676058in}{14.439535in}}%
\pgfusepath{stroke}%
\end{pgfscope}%
\begin{pgfscope}%
\definecolor{textcolor}{rgb}{0.150000,0.150000,0.150000}%
\pgfsetstrokecolor{textcolor}%
\pgfsetfillcolor{textcolor}%
\pgftext[x=10.676058in,y=13.464406in,,top]{\color{textcolor}\rmfamily\fontsize{14.000000}{16.800000}\selectfont 5}%
\end{pgfscope}%
\begin{pgfscope}%
\pgfpathrectangle{\pgfqpoint{9.810417in}{13.561628in}}{\pgfqpoint{5.489583in}{0.877907in}}%
\pgfusepath{clip}%
\pgfsetroundcap%
\pgfsetroundjoin%
\pgfsetlinewidth{0.803000pt}%
\definecolor{currentstroke}{rgb}{1.000000,1.000000,1.000000}%
\pgfsetstrokecolor{currentstroke}%
\pgfsetdash{}{0pt}%
\pgfpathmoveto{\pgfqpoint{11.292173in}{13.561628in}}%
\pgfpathlineto{\pgfqpoint{11.292173in}{14.439535in}}%
\pgfusepath{stroke}%
\end{pgfscope}%
\begin{pgfscope}%
\definecolor{textcolor}{rgb}{0.150000,0.150000,0.150000}%
\pgfsetstrokecolor{textcolor}%
\pgfsetfillcolor{textcolor}%
\pgftext[x=11.292173in,y=13.464406in,,top]{\color{textcolor}\rmfamily\fontsize{14.000000}{16.800000}\selectfont 10}%
\end{pgfscope}%
\begin{pgfscope}%
\pgfpathrectangle{\pgfqpoint{9.810417in}{13.561628in}}{\pgfqpoint{5.489583in}{0.877907in}}%
\pgfusepath{clip}%
\pgfsetroundcap%
\pgfsetroundjoin%
\pgfsetlinewidth{0.803000pt}%
\definecolor{currentstroke}{rgb}{1.000000,1.000000,1.000000}%
\pgfsetstrokecolor{currentstroke}%
\pgfsetdash{}{0pt}%
\pgfpathmoveto{\pgfqpoint{11.908288in}{13.561628in}}%
\pgfpathlineto{\pgfqpoint{11.908288in}{14.439535in}}%
\pgfusepath{stroke}%
\end{pgfscope}%
\begin{pgfscope}%
\definecolor{textcolor}{rgb}{0.150000,0.150000,0.150000}%
\pgfsetstrokecolor{textcolor}%
\pgfsetfillcolor{textcolor}%
\pgftext[x=11.908288in,y=13.464406in,,top]{\color{textcolor}\rmfamily\fontsize{14.000000}{16.800000}\selectfont 15}%
\end{pgfscope}%
\begin{pgfscope}%
\pgfpathrectangle{\pgfqpoint{9.810417in}{13.561628in}}{\pgfqpoint{5.489583in}{0.877907in}}%
\pgfusepath{clip}%
\pgfsetroundcap%
\pgfsetroundjoin%
\pgfsetlinewidth{0.803000pt}%
\definecolor{currentstroke}{rgb}{1.000000,1.000000,1.000000}%
\pgfsetstrokecolor{currentstroke}%
\pgfsetdash{}{0pt}%
\pgfpathmoveto{\pgfqpoint{12.524403in}{13.561628in}}%
\pgfpathlineto{\pgfqpoint{12.524403in}{14.439535in}}%
\pgfusepath{stroke}%
\end{pgfscope}%
\begin{pgfscope}%
\definecolor{textcolor}{rgb}{0.150000,0.150000,0.150000}%
\pgfsetstrokecolor{textcolor}%
\pgfsetfillcolor{textcolor}%
\pgftext[x=12.524403in,y=13.464406in,,top]{\color{textcolor}\rmfamily\fontsize{14.000000}{16.800000}\selectfont 20}%
\end{pgfscope}%
\begin{pgfscope}%
\pgfpathrectangle{\pgfqpoint{9.810417in}{13.561628in}}{\pgfqpoint{5.489583in}{0.877907in}}%
\pgfusepath{clip}%
\pgfsetroundcap%
\pgfsetroundjoin%
\pgfsetlinewidth{0.803000pt}%
\definecolor{currentstroke}{rgb}{1.000000,1.000000,1.000000}%
\pgfsetstrokecolor{currentstroke}%
\pgfsetdash{}{0pt}%
\pgfpathmoveto{\pgfqpoint{13.140517in}{13.561628in}}%
\pgfpathlineto{\pgfqpoint{13.140517in}{14.439535in}}%
\pgfusepath{stroke}%
\end{pgfscope}%
\begin{pgfscope}%
\definecolor{textcolor}{rgb}{0.150000,0.150000,0.150000}%
\pgfsetstrokecolor{textcolor}%
\pgfsetfillcolor{textcolor}%
\pgftext[x=13.140517in,y=13.464406in,,top]{\color{textcolor}\rmfamily\fontsize{14.000000}{16.800000}\selectfont 25}%
\end{pgfscope}%
\begin{pgfscope}%
\pgfpathrectangle{\pgfqpoint{9.810417in}{13.561628in}}{\pgfqpoint{5.489583in}{0.877907in}}%
\pgfusepath{clip}%
\pgfsetroundcap%
\pgfsetroundjoin%
\pgfsetlinewidth{0.803000pt}%
\definecolor{currentstroke}{rgb}{1.000000,1.000000,1.000000}%
\pgfsetstrokecolor{currentstroke}%
\pgfsetdash{}{0pt}%
\pgfpathmoveto{\pgfqpoint{13.756632in}{13.561628in}}%
\pgfpathlineto{\pgfqpoint{13.756632in}{14.439535in}}%
\pgfusepath{stroke}%
\end{pgfscope}%
\begin{pgfscope}%
\definecolor{textcolor}{rgb}{0.150000,0.150000,0.150000}%
\pgfsetstrokecolor{textcolor}%
\pgfsetfillcolor{textcolor}%
\pgftext[x=13.756632in,y=13.464406in,,top]{\color{textcolor}\rmfamily\fontsize{14.000000}{16.800000}\selectfont 30}%
\end{pgfscope}%
\begin{pgfscope}%
\pgfpathrectangle{\pgfqpoint{9.810417in}{13.561628in}}{\pgfqpoint{5.489583in}{0.877907in}}%
\pgfusepath{clip}%
\pgfsetroundcap%
\pgfsetroundjoin%
\pgfsetlinewidth{0.803000pt}%
\definecolor{currentstroke}{rgb}{1.000000,1.000000,1.000000}%
\pgfsetstrokecolor{currentstroke}%
\pgfsetdash{}{0pt}%
\pgfpathmoveto{\pgfqpoint{14.372747in}{13.561628in}}%
\pgfpathlineto{\pgfqpoint{14.372747in}{14.439535in}}%
\pgfusepath{stroke}%
\end{pgfscope}%
\begin{pgfscope}%
\definecolor{textcolor}{rgb}{0.150000,0.150000,0.150000}%
\pgfsetstrokecolor{textcolor}%
\pgfsetfillcolor{textcolor}%
\pgftext[x=14.372747in,y=13.464406in,,top]{\color{textcolor}\rmfamily\fontsize{14.000000}{16.800000}\selectfont 35}%
\end{pgfscope}%
\begin{pgfscope}%
\pgfpathrectangle{\pgfqpoint{9.810417in}{13.561628in}}{\pgfqpoint{5.489583in}{0.877907in}}%
\pgfusepath{clip}%
\pgfsetroundcap%
\pgfsetroundjoin%
\pgfsetlinewidth{0.803000pt}%
\definecolor{currentstroke}{rgb}{1.000000,1.000000,1.000000}%
\pgfsetstrokecolor{currentstroke}%
\pgfsetdash{}{0pt}%
\pgfpathmoveto{\pgfqpoint{14.988862in}{13.561628in}}%
\pgfpathlineto{\pgfqpoint{14.988862in}{14.439535in}}%
\pgfusepath{stroke}%
\end{pgfscope}%
\begin{pgfscope}%
\definecolor{textcolor}{rgb}{0.150000,0.150000,0.150000}%
\pgfsetstrokecolor{textcolor}%
\pgfsetfillcolor{textcolor}%
\pgftext[x=14.988862in,y=13.464406in,,top]{\color{textcolor}\rmfamily\fontsize{14.000000}{16.800000}\selectfont 40}%
\end{pgfscope}%
\begin{pgfscope}%
\pgfpathrectangle{\pgfqpoint{9.810417in}{13.561628in}}{\pgfqpoint{5.489583in}{0.877907in}}%
\pgfusepath{clip}%
\pgfsetroundcap%
\pgfsetroundjoin%
\pgfsetlinewidth{0.803000pt}%
\definecolor{currentstroke}{rgb}{1.000000,1.000000,1.000000}%
\pgfsetstrokecolor{currentstroke}%
\pgfsetdash{}{0pt}%
\pgfpathmoveto{\pgfqpoint{9.810417in}{13.639867in}}%
\pgfpathlineto{\pgfqpoint{15.300000in}{13.639867in}}%
\pgfusepath{stroke}%
\end{pgfscope}%
\begin{pgfscope}%
\definecolor{textcolor}{rgb}{0.150000,0.150000,0.150000}%
\pgfsetstrokecolor{textcolor}%
\pgfsetfillcolor{textcolor}%
\pgftext[x=9.589483in,y=13.566000in,left,base]{\color{textcolor}\rmfamily\fontsize{14.000000}{16.800000}\selectfont 0}%
\end{pgfscope}%
\begin{pgfscope}%
\pgfpathrectangle{\pgfqpoint{9.810417in}{13.561628in}}{\pgfqpoint{5.489583in}{0.877907in}}%
\pgfusepath{clip}%
\pgfsetroundcap%
\pgfsetroundjoin%
\pgfsetlinewidth{0.803000pt}%
\definecolor{currentstroke}{rgb}{1.000000,1.000000,1.000000}%
\pgfsetstrokecolor{currentstroke}%
\pgfsetdash{}{0pt}%
\pgfpathmoveto{\pgfqpoint{9.810417in}{14.399630in}}%
\pgfpathlineto{\pgfqpoint{15.300000in}{14.399630in}}%
\pgfusepath{stroke}%
\end{pgfscope}%
\begin{pgfscope}%
\definecolor{textcolor}{rgb}{0.150000,0.150000,0.150000}%
\pgfsetstrokecolor{textcolor}%
\pgfsetfillcolor{textcolor}%
\pgftext[x=9.589483in,y=14.325764in,left,base]{\color{textcolor}\rmfamily\fontsize{14.000000}{16.800000}\selectfont 1}%
\end{pgfscope}%
\begin{pgfscope}%
\pgfpathrectangle{\pgfqpoint{9.810417in}{13.561628in}}{\pgfqpoint{5.489583in}{0.877907in}}%
\pgfusepath{clip}%
\pgfsetbuttcap%
\pgfsetroundjoin%
\definecolor{currentfill}{rgb}{0.121569,0.466667,0.705882}%
\pgfsetfillcolor{currentfill}%
\pgfsetfillopacity{0.250000}%
\pgfsetlinewidth{1.003750pt}%
\definecolor{currentstroke}{rgb}{1.000000,1.000000,1.000000}%
\pgfsetstrokecolor{currentstroke}%
\pgfsetstrokeopacity{0.250000}%
\pgfsetdash{}{0pt}%
\pgfpathmoveto{\pgfqpoint{10.121555in}{13.678200in}}%
\pgfpathlineto{\pgfqpoint{10.121555in}{13.601533in}}%
\pgfpathlineto{\pgfqpoint{10.306389in}{13.601533in}}%
\pgfpathlineto{\pgfqpoint{10.429612in}{13.601533in}}%
\pgfpathlineto{\pgfqpoint{10.552835in}{13.601533in}}%
\pgfpathlineto{\pgfqpoint{10.676058in}{13.601533in}}%
\pgfpathlineto{\pgfqpoint{10.799281in}{13.601533in}}%
\pgfpathlineto{\pgfqpoint{10.922504in}{13.601533in}}%
\pgfpathlineto{\pgfqpoint{11.045727in}{13.601533in}}%
\pgfpathlineto{\pgfqpoint{11.168950in}{13.601533in}}%
\pgfpathlineto{\pgfqpoint{11.292173in}{13.601533in}}%
\pgfpathlineto{\pgfqpoint{11.415396in}{13.601533in}}%
\pgfpathlineto{\pgfqpoint{11.538619in}{13.601533in}}%
\pgfpathlineto{\pgfqpoint{11.661842in}{13.601533in}}%
\pgfpathlineto{\pgfqpoint{11.785065in}{13.601533in}}%
\pgfpathlineto{\pgfqpoint{11.908288in}{13.601533in}}%
\pgfpathlineto{\pgfqpoint{12.031511in}{13.601533in}}%
\pgfpathlineto{\pgfqpoint{12.154734in}{13.601533in}}%
\pgfpathlineto{\pgfqpoint{12.277957in}{13.601533in}}%
\pgfpathlineto{\pgfqpoint{12.401180in}{13.601533in}}%
\pgfpathlineto{\pgfqpoint{12.524403in}{13.601533in}}%
\pgfpathlineto{\pgfqpoint{12.647626in}{13.601533in}}%
\pgfpathlineto{\pgfqpoint{12.770849in}{13.601533in}}%
\pgfpathlineto{\pgfqpoint{12.894072in}{13.601533in}}%
\pgfpathlineto{\pgfqpoint{13.017294in}{13.601533in}}%
\pgfpathlineto{\pgfqpoint{13.140517in}{13.601533in}}%
\pgfpathlineto{\pgfqpoint{13.263740in}{13.601533in}}%
\pgfpathlineto{\pgfqpoint{13.386963in}{13.601533in}}%
\pgfpathlineto{\pgfqpoint{13.510186in}{13.601533in}}%
\pgfpathlineto{\pgfqpoint{13.633409in}{13.601533in}}%
\pgfpathlineto{\pgfqpoint{13.756632in}{13.601533in}}%
\pgfpathlineto{\pgfqpoint{13.879855in}{13.601533in}}%
\pgfpathlineto{\pgfqpoint{14.003078in}{13.601533in}}%
\pgfpathlineto{\pgfqpoint{14.126301in}{13.601533in}}%
\pgfpathlineto{\pgfqpoint{14.249524in}{13.601533in}}%
\pgfpathlineto{\pgfqpoint{14.372747in}{13.601533in}}%
\pgfpathlineto{\pgfqpoint{14.495970in}{13.601533in}}%
\pgfpathlineto{\pgfqpoint{14.619193in}{13.601533in}}%
\pgfpathlineto{\pgfqpoint{14.742416in}{13.601533in}}%
\pgfpathlineto{\pgfqpoint{14.865639in}{13.601533in}}%
\pgfpathlineto{\pgfqpoint{15.050473in}{13.601533in}}%
\pgfpathlineto{\pgfqpoint{15.050473in}{13.678200in}}%
\pgfpathlineto{\pgfqpoint{15.050473in}{13.678200in}}%
\pgfpathlineto{\pgfqpoint{14.865639in}{13.678200in}}%
\pgfpathlineto{\pgfqpoint{14.742416in}{13.678200in}}%
\pgfpathlineto{\pgfqpoint{14.619193in}{13.678200in}}%
\pgfpathlineto{\pgfqpoint{14.495970in}{13.678200in}}%
\pgfpathlineto{\pgfqpoint{14.372747in}{13.678200in}}%
\pgfpathlineto{\pgfqpoint{14.249524in}{13.678200in}}%
\pgfpathlineto{\pgfqpoint{14.126301in}{13.678200in}}%
\pgfpathlineto{\pgfqpoint{14.003078in}{13.678200in}}%
\pgfpathlineto{\pgfqpoint{13.879855in}{13.678200in}}%
\pgfpathlineto{\pgfqpoint{13.756632in}{13.678200in}}%
\pgfpathlineto{\pgfqpoint{13.633409in}{13.678200in}}%
\pgfpathlineto{\pgfqpoint{13.510186in}{13.678200in}}%
\pgfpathlineto{\pgfqpoint{13.386963in}{13.678200in}}%
\pgfpathlineto{\pgfqpoint{13.263740in}{13.678200in}}%
\pgfpathlineto{\pgfqpoint{13.140517in}{13.678200in}}%
\pgfpathlineto{\pgfqpoint{13.017294in}{13.678200in}}%
\pgfpathlineto{\pgfqpoint{12.894072in}{13.678200in}}%
\pgfpathlineto{\pgfqpoint{12.770849in}{13.678200in}}%
\pgfpathlineto{\pgfqpoint{12.647626in}{13.678200in}}%
\pgfpathlineto{\pgfqpoint{12.524403in}{13.678200in}}%
\pgfpathlineto{\pgfqpoint{12.401180in}{13.678200in}}%
\pgfpathlineto{\pgfqpoint{12.277957in}{13.678200in}}%
\pgfpathlineto{\pgfqpoint{12.154734in}{13.678200in}}%
\pgfpathlineto{\pgfqpoint{12.031511in}{13.678200in}}%
\pgfpathlineto{\pgfqpoint{11.908288in}{13.678200in}}%
\pgfpathlineto{\pgfqpoint{11.785065in}{13.678200in}}%
\pgfpathlineto{\pgfqpoint{11.661842in}{13.678200in}}%
\pgfpathlineto{\pgfqpoint{11.538619in}{13.678200in}}%
\pgfpathlineto{\pgfqpoint{11.415396in}{13.678200in}}%
\pgfpathlineto{\pgfqpoint{11.292173in}{13.678200in}}%
\pgfpathlineto{\pgfqpoint{11.168950in}{13.678200in}}%
\pgfpathlineto{\pgfqpoint{11.045727in}{13.678200in}}%
\pgfpathlineto{\pgfqpoint{10.922504in}{13.678200in}}%
\pgfpathlineto{\pgfqpoint{10.799281in}{13.678200in}}%
\pgfpathlineto{\pgfqpoint{10.676058in}{13.678200in}}%
\pgfpathlineto{\pgfqpoint{10.552835in}{13.678200in}}%
\pgfpathlineto{\pgfqpoint{10.429612in}{13.678200in}}%
\pgfpathlineto{\pgfqpoint{10.306389in}{13.678200in}}%
\pgfpathlineto{\pgfqpoint{10.121555in}{13.678200in}}%
\pgfpathclose%
\pgfusepath{stroke,fill}%
\end{pgfscope}%
\begin{pgfscope}%
\pgfpathrectangle{\pgfqpoint{9.810417in}{13.561628in}}{\pgfqpoint{5.489583in}{0.877907in}}%
\pgfusepath{clip}%
\pgfsetbuttcap%
\pgfsetroundjoin%
\pgfsetlinewidth{1.505625pt}%
\definecolor{currentstroke}{rgb}{0.000000,0.000000,0.000000}%
\pgfsetstrokecolor{currentstroke}%
\pgfsetdash{}{0pt}%
\pgfpathmoveto{\pgfqpoint{10.059943in}{13.639867in}}%
\pgfpathlineto{\pgfqpoint{10.059943in}{14.399630in}}%
\pgfusepath{stroke}%
\end{pgfscope}%
\begin{pgfscope}%
\pgfpathrectangle{\pgfqpoint{9.810417in}{13.561628in}}{\pgfqpoint{5.489583in}{0.877907in}}%
\pgfusepath{clip}%
\pgfsetbuttcap%
\pgfsetroundjoin%
\pgfsetlinewidth{1.505625pt}%
\definecolor{currentstroke}{rgb}{0.000000,0.000000,0.000000}%
\pgfsetstrokecolor{currentstroke}%
\pgfsetdash{}{0pt}%
\pgfpathmoveto{\pgfqpoint{10.183166in}{13.639867in}}%
\pgfpathlineto{\pgfqpoint{10.183166in}{14.397078in}}%
\pgfusepath{stroke}%
\end{pgfscope}%
\begin{pgfscope}%
\pgfpathrectangle{\pgfqpoint{9.810417in}{13.561628in}}{\pgfqpoint{5.489583in}{0.877907in}}%
\pgfusepath{clip}%
\pgfsetbuttcap%
\pgfsetroundjoin%
\pgfsetlinewidth{1.505625pt}%
\definecolor{currentstroke}{rgb}{0.000000,0.000000,0.000000}%
\pgfsetstrokecolor{currentstroke}%
\pgfsetdash{}{0pt}%
\pgfpathmoveto{\pgfqpoint{10.306389in}{13.639867in}}%
\pgfpathlineto{\pgfqpoint{10.306389in}{13.625105in}}%
\pgfusepath{stroke}%
\end{pgfscope}%
\begin{pgfscope}%
\pgfpathrectangle{\pgfqpoint{9.810417in}{13.561628in}}{\pgfqpoint{5.489583in}{0.877907in}}%
\pgfusepath{clip}%
\pgfsetbuttcap%
\pgfsetroundjoin%
\pgfsetlinewidth{1.505625pt}%
\definecolor{currentstroke}{rgb}{0.000000,0.000000,0.000000}%
\pgfsetstrokecolor{currentstroke}%
\pgfsetdash{}{0pt}%
\pgfpathmoveto{\pgfqpoint{10.429612in}{13.639867in}}%
\pgfpathlineto{\pgfqpoint{10.429612in}{13.660246in}}%
\pgfusepath{stroke}%
\end{pgfscope}%
\begin{pgfscope}%
\pgfpathrectangle{\pgfqpoint{9.810417in}{13.561628in}}{\pgfqpoint{5.489583in}{0.877907in}}%
\pgfusepath{clip}%
\pgfsetbuttcap%
\pgfsetroundjoin%
\pgfsetlinewidth{1.505625pt}%
\definecolor{currentstroke}{rgb}{0.000000,0.000000,0.000000}%
\pgfsetstrokecolor{currentstroke}%
\pgfsetdash{}{0pt}%
\pgfpathmoveto{\pgfqpoint{10.552835in}{13.639867in}}%
\pgfpathlineto{\pgfqpoint{10.552835in}{13.635487in}}%
\pgfusepath{stroke}%
\end{pgfscope}%
\begin{pgfscope}%
\pgfpathrectangle{\pgfqpoint{9.810417in}{13.561628in}}{\pgfqpoint{5.489583in}{0.877907in}}%
\pgfusepath{clip}%
\pgfsetbuttcap%
\pgfsetroundjoin%
\pgfsetlinewidth{1.505625pt}%
\definecolor{currentstroke}{rgb}{0.000000,0.000000,0.000000}%
\pgfsetstrokecolor{currentstroke}%
\pgfsetdash{}{0pt}%
\pgfpathmoveto{\pgfqpoint{10.676058in}{13.639867in}}%
\pgfpathlineto{\pgfqpoint{10.676058in}{13.652440in}}%
\pgfusepath{stroke}%
\end{pgfscope}%
\begin{pgfscope}%
\pgfpathrectangle{\pgfqpoint{9.810417in}{13.561628in}}{\pgfqpoint{5.489583in}{0.877907in}}%
\pgfusepath{clip}%
\pgfsetbuttcap%
\pgfsetroundjoin%
\pgfsetlinewidth{1.505625pt}%
\definecolor{currentstroke}{rgb}{0.000000,0.000000,0.000000}%
\pgfsetstrokecolor{currentstroke}%
\pgfsetdash{}{0pt}%
\pgfpathmoveto{\pgfqpoint{10.799281in}{13.639867in}}%
\pgfpathlineto{\pgfqpoint{10.799281in}{13.639411in}}%
\pgfusepath{stroke}%
\end{pgfscope}%
\begin{pgfscope}%
\pgfpathrectangle{\pgfqpoint{9.810417in}{13.561628in}}{\pgfqpoint{5.489583in}{0.877907in}}%
\pgfusepath{clip}%
\pgfsetbuttcap%
\pgfsetroundjoin%
\pgfsetlinewidth{1.505625pt}%
\definecolor{currentstroke}{rgb}{0.000000,0.000000,0.000000}%
\pgfsetstrokecolor{currentstroke}%
\pgfsetdash{}{0pt}%
\pgfpathmoveto{\pgfqpoint{10.922504in}{13.639867in}}%
\pgfpathlineto{\pgfqpoint{10.922504in}{13.644059in}}%
\pgfusepath{stroke}%
\end{pgfscope}%
\begin{pgfscope}%
\pgfpathrectangle{\pgfqpoint{9.810417in}{13.561628in}}{\pgfqpoint{5.489583in}{0.877907in}}%
\pgfusepath{clip}%
\pgfsetbuttcap%
\pgfsetroundjoin%
\pgfsetlinewidth{1.505625pt}%
\definecolor{currentstroke}{rgb}{0.000000,0.000000,0.000000}%
\pgfsetstrokecolor{currentstroke}%
\pgfsetdash{}{0pt}%
\pgfpathmoveto{\pgfqpoint{11.045727in}{13.639867in}}%
\pgfpathlineto{\pgfqpoint{11.045727in}{13.663785in}}%
\pgfusepath{stroke}%
\end{pgfscope}%
\begin{pgfscope}%
\pgfpathrectangle{\pgfqpoint{9.810417in}{13.561628in}}{\pgfqpoint{5.489583in}{0.877907in}}%
\pgfusepath{clip}%
\pgfsetbuttcap%
\pgfsetroundjoin%
\pgfsetlinewidth{1.505625pt}%
\definecolor{currentstroke}{rgb}{0.000000,0.000000,0.000000}%
\pgfsetstrokecolor{currentstroke}%
\pgfsetdash{}{0pt}%
\pgfpathmoveto{\pgfqpoint{11.168950in}{13.639867in}}%
\pgfpathlineto{\pgfqpoint{11.168950in}{13.624725in}}%
\pgfusepath{stroke}%
\end{pgfscope}%
\begin{pgfscope}%
\pgfpathrectangle{\pgfqpoint{9.810417in}{13.561628in}}{\pgfqpoint{5.489583in}{0.877907in}}%
\pgfusepath{clip}%
\pgfsetbuttcap%
\pgfsetroundjoin%
\pgfsetlinewidth{1.505625pt}%
\definecolor{currentstroke}{rgb}{0.000000,0.000000,0.000000}%
\pgfsetstrokecolor{currentstroke}%
\pgfsetdash{}{0pt}%
\pgfpathmoveto{\pgfqpoint{11.292173in}{13.639867in}}%
\pgfpathlineto{\pgfqpoint{11.292173in}{13.653613in}}%
\pgfusepath{stroke}%
\end{pgfscope}%
\begin{pgfscope}%
\pgfpathrectangle{\pgfqpoint{9.810417in}{13.561628in}}{\pgfqpoint{5.489583in}{0.877907in}}%
\pgfusepath{clip}%
\pgfsetbuttcap%
\pgfsetroundjoin%
\pgfsetlinewidth{1.505625pt}%
\definecolor{currentstroke}{rgb}{0.000000,0.000000,0.000000}%
\pgfsetstrokecolor{currentstroke}%
\pgfsetdash{}{0pt}%
\pgfpathmoveto{\pgfqpoint{11.415396in}{13.639867in}}%
\pgfpathlineto{\pgfqpoint{11.415396in}{13.626753in}}%
\pgfusepath{stroke}%
\end{pgfscope}%
\begin{pgfscope}%
\pgfpathrectangle{\pgfqpoint{9.810417in}{13.561628in}}{\pgfqpoint{5.489583in}{0.877907in}}%
\pgfusepath{clip}%
\pgfsetbuttcap%
\pgfsetroundjoin%
\pgfsetlinewidth{1.505625pt}%
\definecolor{currentstroke}{rgb}{0.000000,0.000000,0.000000}%
\pgfsetstrokecolor{currentstroke}%
\pgfsetdash{}{0pt}%
\pgfpathmoveto{\pgfqpoint{11.538619in}{13.639867in}}%
\pgfpathlineto{\pgfqpoint{11.538619in}{13.642618in}}%
\pgfusepath{stroke}%
\end{pgfscope}%
\begin{pgfscope}%
\pgfpathrectangle{\pgfqpoint{9.810417in}{13.561628in}}{\pgfqpoint{5.489583in}{0.877907in}}%
\pgfusepath{clip}%
\pgfsetbuttcap%
\pgfsetroundjoin%
\pgfsetlinewidth{1.505625pt}%
\definecolor{currentstroke}{rgb}{0.000000,0.000000,0.000000}%
\pgfsetstrokecolor{currentstroke}%
\pgfsetdash{}{0pt}%
\pgfpathmoveto{\pgfqpoint{11.661842in}{13.639867in}}%
\pgfpathlineto{\pgfqpoint{11.661842in}{13.625213in}}%
\pgfusepath{stroke}%
\end{pgfscope}%
\begin{pgfscope}%
\pgfpathrectangle{\pgfqpoint{9.810417in}{13.561628in}}{\pgfqpoint{5.489583in}{0.877907in}}%
\pgfusepath{clip}%
\pgfsetbuttcap%
\pgfsetroundjoin%
\pgfsetlinewidth{1.505625pt}%
\definecolor{currentstroke}{rgb}{0.000000,0.000000,0.000000}%
\pgfsetstrokecolor{currentstroke}%
\pgfsetdash{}{0pt}%
\pgfpathmoveto{\pgfqpoint{11.785065in}{13.639867in}}%
\pgfpathlineto{\pgfqpoint{11.785065in}{13.602345in}}%
\pgfusepath{stroke}%
\end{pgfscope}%
\begin{pgfscope}%
\pgfpathrectangle{\pgfqpoint{9.810417in}{13.561628in}}{\pgfqpoint{5.489583in}{0.877907in}}%
\pgfusepath{clip}%
\pgfsetbuttcap%
\pgfsetroundjoin%
\pgfsetlinewidth{1.505625pt}%
\definecolor{currentstroke}{rgb}{0.000000,0.000000,0.000000}%
\pgfsetstrokecolor{currentstroke}%
\pgfsetdash{}{0pt}%
\pgfpathmoveto{\pgfqpoint{11.908288in}{13.639867in}}%
\pgfpathlineto{\pgfqpoint{11.908288in}{13.637704in}}%
\pgfusepath{stroke}%
\end{pgfscope}%
\begin{pgfscope}%
\pgfpathrectangle{\pgfqpoint{9.810417in}{13.561628in}}{\pgfqpoint{5.489583in}{0.877907in}}%
\pgfusepath{clip}%
\pgfsetbuttcap%
\pgfsetroundjoin%
\pgfsetlinewidth{1.505625pt}%
\definecolor{currentstroke}{rgb}{0.000000,0.000000,0.000000}%
\pgfsetstrokecolor{currentstroke}%
\pgfsetdash{}{0pt}%
\pgfpathmoveto{\pgfqpoint{12.031511in}{13.639867in}}%
\pgfpathlineto{\pgfqpoint{12.031511in}{13.652483in}}%
\pgfusepath{stroke}%
\end{pgfscope}%
\begin{pgfscope}%
\pgfpathrectangle{\pgfqpoint{9.810417in}{13.561628in}}{\pgfqpoint{5.489583in}{0.877907in}}%
\pgfusepath{clip}%
\pgfsetbuttcap%
\pgfsetroundjoin%
\pgfsetlinewidth{1.505625pt}%
\definecolor{currentstroke}{rgb}{0.000000,0.000000,0.000000}%
\pgfsetstrokecolor{currentstroke}%
\pgfsetdash{}{0pt}%
\pgfpathmoveto{\pgfqpoint{12.154734in}{13.639867in}}%
\pgfpathlineto{\pgfqpoint{12.154734in}{13.614715in}}%
\pgfusepath{stroke}%
\end{pgfscope}%
\begin{pgfscope}%
\pgfpathrectangle{\pgfqpoint{9.810417in}{13.561628in}}{\pgfqpoint{5.489583in}{0.877907in}}%
\pgfusepath{clip}%
\pgfsetbuttcap%
\pgfsetroundjoin%
\pgfsetlinewidth{1.505625pt}%
\definecolor{currentstroke}{rgb}{0.000000,0.000000,0.000000}%
\pgfsetstrokecolor{currentstroke}%
\pgfsetdash{}{0pt}%
\pgfpathmoveto{\pgfqpoint{12.277957in}{13.639867in}}%
\pgfpathlineto{\pgfqpoint{12.277957in}{13.646557in}}%
\pgfusepath{stroke}%
\end{pgfscope}%
\begin{pgfscope}%
\pgfpathrectangle{\pgfqpoint{9.810417in}{13.561628in}}{\pgfqpoint{5.489583in}{0.877907in}}%
\pgfusepath{clip}%
\pgfsetbuttcap%
\pgfsetroundjoin%
\pgfsetlinewidth{1.505625pt}%
\definecolor{currentstroke}{rgb}{0.000000,0.000000,0.000000}%
\pgfsetstrokecolor{currentstroke}%
\pgfsetdash{}{0pt}%
\pgfpathmoveto{\pgfqpoint{12.401180in}{13.639867in}}%
\pgfpathlineto{\pgfqpoint{12.401180in}{13.625305in}}%
\pgfusepath{stroke}%
\end{pgfscope}%
\begin{pgfscope}%
\pgfpathrectangle{\pgfqpoint{9.810417in}{13.561628in}}{\pgfqpoint{5.489583in}{0.877907in}}%
\pgfusepath{clip}%
\pgfsetbuttcap%
\pgfsetroundjoin%
\pgfsetlinewidth{1.505625pt}%
\definecolor{currentstroke}{rgb}{0.000000,0.000000,0.000000}%
\pgfsetstrokecolor{currentstroke}%
\pgfsetdash{}{0pt}%
\pgfpathmoveto{\pgfqpoint{12.524403in}{13.639867in}}%
\pgfpathlineto{\pgfqpoint{12.524403in}{13.645622in}}%
\pgfusepath{stroke}%
\end{pgfscope}%
\begin{pgfscope}%
\pgfpathrectangle{\pgfqpoint{9.810417in}{13.561628in}}{\pgfqpoint{5.489583in}{0.877907in}}%
\pgfusepath{clip}%
\pgfsetbuttcap%
\pgfsetroundjoin%
\pgfsetlinewidth{1.505625pt}%
\definecolor{currentstroke}{rgb}{0.000000,0.000000,0.000000}%
\pgfsetstrokecolor{currentstroke}%
\pgfsetdash{}{0pt}%
\pgfpathmoveto{\pgfqpoint{12.647626in}{13.639867in}}%
\pgfpathlineto{\pgfqpoint{12.647626in}{13.653251in}}%
\pgfusepath{stroke}%
\end{pgfscope}%
\begin{pgfscope}%
\pgfpathrectangle{\pgfqpoint{9.810417in}{13.561628in}}{\pgfqpoint{5.489583in}{0.877907in}}%
\pgfusepath{clip}%
\pgfsetbuttcap%
\pgfsetroundjoin%
\pgfsetlinewidth{1.505625pt}%
\definecolor{currentstroke}{rgb}{0.000000,0.000000,0.000000}%
\pgfsetstrokecolor{currentstroke}%
\pgfsetdash{}{0pt}%
\pgfpathmoveto{\pgfqpoint{12.770849in}{13.639867in}}%
\pgfpathlineto{\pgfqpoint{12.770849in}{13.634705in}}%
\pgfusepath{stroke}%
\end{pgfscope}%
\begin{pgfscope}%
\pgfpathrectangle{\pgfqpoint{9.810417in}{13.561628in}}{\pgfqpoint{5.489583in}{0.877907in}}%
\pgfusepath{clip}%
\pgfsetbuttcap%
\pgfsetroundjoin%
\pgfsetlinewidth{1.505625pt}%
\definecolor{currentstroke}{rgb}{0.000000,0.000000,0.000000}%
\pgfsetstrokecolor{currentstroke}%
\pgfsetdash{}{0pt}%
\pgfpathmoveto{\pgfqpoint{12.894072in}{13.639867in}}%
\pgfpathlineto{\pgfqpoint{12.894072in}{13.660459in}}%
\pgfusepath{stroke}%
\end{pgfscope}%
\begin{pgfscope}%
\pgfpathrectangle{\pgfqpoint{9.810417in}{13.561628in}}{\pgfqpoint{5.489583in}{0.877907in}}%
\pgfusepath{clip}%
\pgfsetbuttcap%
\pgfsetroundjoin%
\pgfsetlinewidth{1.505625pt}%
\definecolor{currentstroke}{rgb}{0.000000,0.000000,0.000000}%
\pgfsetstrokecolor{currentstroke}%
\pgfsetdash{}{0pt}%
\pgfpathmoveto{\pgfqpoint{13.017294in}{13.639867in}}%
\pgfpathlineto{\pgfqpoint{13.017294in}{13.618121in}}%
\pgfusepath{stroke}%
\end{pgfscope}%
\begin{pgfscope}%
\pgfpathrectangle{\pgfqpoint{9.810417in}{13.561628in}}{\pgfqpoint{5.489583in}{0.877907in}}%
\pgfusepath{clip}%
\pgfsetbuttcap%
\pgfsetroundjoin%
\pgfsetlinewidth{1.505625pt}%
\definecolor{currentstroke}{rgb}{0.000000,0.000000,0.000000}%
\pgfsetstrokecolor{currentstroke}%
\pgfsetdash{}{0pt}%
\pgfpathmoveto{\pgfqpoint{13.140517in}{13.639867in}}%
\pgfpathlineto{\pgfqpoint{13.140517in}{13.632358in}}%
\pgfusepath{stroke}%
\end{pgfscope}%
\begin{pgfscope}%
\pgfpathrectangle{\pgfqpoint{9.810417in}{13.561628in}}{\pgfqpoint{5.489583in}{0.877907in}}%
\pgfusepath{clip}%
\pgfsetbuttcap%
\pgfsetroundjoin%
\pgfsetlinewidth{1.505625pt}%
\definecolor{currentstroke}{rgb}{0.000000,0.000000,0.000000}%
\pgfsetstrokecolor{currentstroke}%
\pgfsetdash{}{0pt}%
\pgfpathmoveto{\pgfqpoint{13.263740in}{13.639867in}}%
\pgfpathlineto{\pgfqpoint{13.263740in}{13.648504in}}%
\pgfusepath{stroke}%
\end{pgfscope}%
\begin{pgfscope}%
\pgfpathrectangle{\pgfqpoint{9.810417in}{13.561628in}}{\pgfqpoint{5.489583in}{0.877907in}}%
\pgfusepath{clip}%
\pgfsetbuttcap%
\pgfsetroundjoin%
\pgfsetlinewidth{1.505625pt}%
\definecolor{currentstroke}{rgb}{0.000000,0.000000,0.000000}%
\pgfsetstrokecolor{currentstroke}%
\pgfsetdash{}{0pt}%
\pgfpathmoveto{\pgfqpoint{13.386963in}{13.639867in}}%
\pgfpathlineto{\pgfqpoint{13.386963in}{13.628033in}}%
\pgfusepath{stroke}%
\end{pgfscope}%
\begin{pgfscope}%
\pgfpathrectangle{\pgfqpoint{9.810417in}{13.561628in}}{\pgfqpoint{5.489583in}{0.877907in}}%
\pgfusepath{clip}%
\pgfsetbuttcap%
\pgfsetroundjoin%
\pgfsetlinewidth{1.505625pt}%
\definecolor{currentstroke}{rgb}{0.000000,0.000000,0.000000}%
\pgfsetstrokecolor{currentstroke}%
\pgfsetdash{}{0pt}%
\pgfpathmoveto{\pgfqpoint{13.510186in}{13.639867in}}%
\pgfpathlineto{\pgfqpoint{13.510186in}{13.640320in}}%
\pgfusepath{stroke}%
\end{pgfscope}%
\begin{pgfscope}%
\pgfpathrectangle{\pgfqpoint{9.810417in}{13.561628in}}{\pgfqpoint{5.489583in}{0.877907in}}%
\pgfusepath{clip}%
\pgfsetbuttcap%
\pgfsetroundjoin%
\pgfsetlinewidth{1.505625pt}%
\definecolor{currentstroke}{rgb}{0.000000,0.000000,0.000000}%
\pgfsetstrokecolor{currentstroke}%
\pgfsetdash{}{0pt}%
\pgfpathmoveto{\pgfqpoint{13.633409in}{13.639867in}}%
\pgfpathlineto{\pgfqpoint{13.633409in}{13.632564in}}%
\pgfusepath{stroke}%
\end{pgfscope}%
\begin{pgfscope}%
\pgfpathrectangle{\pgfqpoint{9.810417in}{13.561628in}}{\pgfqpoint{5.489583in}{0.877907in}}%
\pgfusepath{clip}%
\pgfsetbuttcap%
\pgfsetroundjoin%
\pgfsetlinewidth{1.505625pt}%
\definecolor{currentstroke}{rgb}{0.000000,0.000000,0.000000}%
\pgfsetstrokecolor{currentstroke}%
\pgfsetdash{}{0pt}%
\pgfpathmoveto{\pgfqpoint{13.756632in}{13.639867in}}%
\pgfpathlineto{\pgfqpoint{13.756632in}{13.627775in}}%
\pgfusepath{stroke}%
\end{pgfscope}%
\begin{pgfscope}%
\pgfpathrectangle{\pgfqpoint{9.810417in}{13.561628in}}{\pgfqpoint{5.489583in}{0.877907in}}%
\pgfusepath{clip}%
\pgfsetbuttcap%
\pgfsetroundjoin%
\pgfsetlinewidth{1.505625pt}%
\definecolor{currentstroke}{rgb}{0.000000,0.000000,0.000000}%
\pgfsetstrokecolor{currentstroke}%
\pgfsetdash{}{0pt}%
\pgfpathmoveto{\pgfqpoint{13.879855in}{13.639867in}}%
\pgfpathlineto{\pgfqpoint{13.879855in}{13.641695in}}%
\pgfusepath{stroke}%
\end{pgfscope}%
\begin{pgfscope}%
\pgfpathrectangle{\pgfqpoint{9.810417in}{13.561628in}}{\pgfqpoint{5.489583in}{0.877907in}}%
\pgfusepath{clip}%
\pgfsetbuttcap%
\pgfsetroundjoin%
\pgfsetlinewidth{1.505625pt}%
\definecolor{currentstroke}{rgb}{0.000000,0.000000,0.000000}%
\pgfsetstrokecolor{currentstroke}%
\pgfsetdash{}{0pt}%
\pgfpathmoveto{\pgfqpoint{14.003078in}{13.639867in}}%
\pgfpathlineto{\pgfqpoint{14.003078in}{13.654167in}}%
\pgfusepath{stroke}%
\end{pgfscope}%
\begin{pgfscope}%
\pgfpathrectangle{\pgfqpoint{9.810417in}{13.561628in}}{\pgfqpoint{5.489583in}{0.877907in}}%
\pgfusepath{clip}%
\pgfsetbuttcap%
\pgfsetroundjoin%
\pgfsetlinewidth{1.505625pt}%
\definecolor{currentstroke}{rgb}{0.000000,0.000000,0.000000}%
\pgfsetstrokecolor{currentstroke}%
\pgfsetdash{}{0pt}%
\pgfpathmoveto{\pgfqpoint{14.126301in}{13.639867in}}%
\pgfpathlineto{\pgfqpoint{14.126301in}{13.694911in}}%
\pgfusepath{stroke}%
\end{pgfscope}%
\begin{pgfscope}%
\pgfpathrectangle{\pgfqpoint{9.810417in}{13.561628in}}{\pgfqpoint{5.489583in}{0.877907in}}%
\pgfusepath{clip}%
\pgfsetbuttcap%
\pgfsetroundjoin%
\pgfsetlinewidth{1.505625pt}%
\definecolor{currentstroke}{rgb}{0.000000,0.000000,0.000000}%
\pgfsetstrokecolor{currentstroke}%
\pgfsetdash{}{0pt}%
\pgfpathmoveto{\pgfqpoint{14.249524in}{13.639867in}}%
\pgfpathlineto{\pgfqpoint{14.249524in}{13.687956in}}%
\pgfusepath{stroke}%
\end{pgfscope}%
\begin{pgfscope}%
\pgfpathrectangle{\pgfqpoint{9.810417in}{13.561628in}}{\pgfqpoint{5.489583in}{0.877907in}}%
\pgfusepath{clip}%
\pgfsetbuttcap%
\pgfsetroundjoin%
\pgfsetlinewidth{1.505625pt}%
\definecolor{currentstroke}{rgb}{0.000000,0.000000,0.000000}%
\pgfsetstrokecolor{currentstroke}%
\pgfsetdash{}{0pt}%
\pgfpathmoveto{\pgfqpoint{14.372747in}{13.639867in}}%
\pgfpathlineto{\pgfqpoint{14.372747in}{13.603316in}}%
\pgfusepath{stroke}%
\end{pgfscope}%
\begin{pgfscope}%
\pgfpathrectangle{\pgfqpoint{9.810417in}{13.561628in}}{\pgfqpoint{5.489583in}{0.877907in}}%
\pgfusepath{clip}%
\pgfsetbuttcap%
\pgfsetroundjoin%
\pgfsetlinewidth{1.505625pt}%
\definecolor{currentstroke}{rgb}{0.000000,0.000000,0.000000}%
\pgfsetstrokecolor{currentstroke}%
\pgfsetdash{}{0pt}%
\pgfpathmoveto{\pgfqpoint{14.495970in}{13.639867in}}%
\pgfpathlineto{\pgfqpoint{14.495970in}{13.645697in}}%
\pgfusepath{stroke}%
\end{pgfscope}%
\begin{pgfscope}%
\pgfpathrectangle{\pgfqpoint{9.810417in}{13.561628in}}{\pgfqpoint{5.489583in}{0.877907in}}%
\pgfusepath{clip}%
\pgfsetbuttcap%
\pgfsetroundjoin%
\pgfsetlinewidth{1.505625pt}%
\definecolor{currentstroke}{rgb}{0.000000,0.000000,0.000000}%
\pgfsetstrokecolor{currentstroke}%
\pgfsetdash{}{0pt}%
\pgfpathmoveto{\pgfqpoint{14.619193in}{13.639867in}}%
\pgfpathlineto{\pgfqpoint{14.619193in}{13.628356in}}%
\pgfusepath{stroke}%
\end{pgfscope}%
\begin{pgfscope}%
\pgfpathrectangle{\pgfqpoint{9.810417in}{13.561628in}}{\pgfqpoint{5.489583in}{0.877907in}}%
\pgfusepath{clip}%
\pgfsetbuttcap%
\pgfsetroundjoin%
\pgfsetlinewidth{1.505625pt}%
\definecolor{currentstroke}{rgb}{0.000000,0.000000,0.000000}%
\pgfsetstrokecolor{currentstroke}%
\pgfsetdash{}{0pt}%
\pgfpathmoveto{\pgfqpoint{14.742416in}{13.639867in}}%
\pgfpathlineto{\pgfqpoint{14.742416in}{13.641580in}}%
\pgfusepath{stroke}%
\end{pgfscope}%
\begin{pgfscope}%
\pgfpathrectangle{\pgfqpoint{9.810417in}{13.561628in}}{\pgfqpoint{5.489583in}{0.877907in}}%
\pgfusepath{clip}%
\pgfsetbuttcap%
\pgfsetroundjoin%
\pgfsetlinewidth{1.505625pt}%
\definecolor{currentstroke}{rgb}{0.000000,0.000000,0.000000}%
\pgfsetstrokecolor{currentstroke}%
\pgfsetdash{}{0pt}%
\pgfpathmoveto{\pgfqpoint{14.865639in}{13.639867in}}%
\pgfpathlineto{\pgfqpoint{14.865639in}{13.652100in}}%
\pgfusepath{stroke}%
\end{pgfscope}%
\begin{pgfscope}%
\pgfpathrectangle{\pgfqpoint{9.810417in}{13.561628in}}{\pgfqpoint{5.489583in}{0.877907in}}%
\pgfusepath{clip}%
\pgfsetbuttcap%
\pgfsetroundjoin%
\pgfsetlinewidth{1.505625pt}%
\definecolor{currentstroke}{rgb}{0.000000,0.000000,0.000000}%
\pgfsetstrokecolor{currentstroke}%
\pgfsetdash{}{0pt}%
\pgfpathmoveto{\pgfqpoint{14.988862in}{13.639867in}}%
\pgfpathlineto{\pgfqpoint{14.988862in}{13.625585in}}%
\pgfusepath{stroke}%
\end{pgfscope}%
\begin{pgfscope}%
\pgfpathrectangle{\pgfqpoint{9.810417in}{13.561628in}}{\pgfqpoint{5.489583in}{0.877907in}}%
\pgfusepath{clip}%
\pgfsetroundcap%
\pgfsetroundjoin%
\pgfsetlinewidth{1.505625pt}%
\definecolor{currentstroke}{rgb}{0.121569,0.466667,0.705882}%
\pgfsetstrokecolor{currentstroke}%
\pgfsetdash{}{0pt}%
\pgfpathmoveto{\pgfqpoint{9.810417in}{13.639867in}}%
\pgfpathlineto{\pgfqpoint{15.300000in}{13.639867in}}%
\pgfusepath{stroke}%
\end{pgfscope}%
\begin{pgfscope}%
\pgfpathrectangle{\pgfqpoint{9.810417in}{13.561628in}}{\pgfqpoint{5.489583in}{0.877907in}}%
\pgfusepath{clip}%
\pgfsetbuttcap%
\pgfsetroundjoin%
\definecolor{currentfill}{rgb}{0.121569,0.466667,0.705882}%
\pgfsetfillcolor{currentfill}%
\pgfsetlinewidth{1.003750pt}%
\definecolor{currentstroke}{rgb}{0.121569,0.466667,0.705882}%
\pgfsetstrokecolor{currentstroke}%
\pgfsetdash{}{0pt}%
\pgfsys@defobject{currentmarker}{\pgfqpoint{-0.034722in}{-0.034722in}}{\pgfqpoint{0.034722in}{0.034722in}}{%
\pgfpathmoveto{\pgfqpoint{0.000000in}{-0.034722in}}%
\pgfpathcurveto{\pgfqpoint{0.009208in}{-0.034722in}}{\pgfqpoint{0.018041in}{-0.031064in}}{\pgfqpoint{0.024552in}{-0.024552in}}%
\pgfpathcurveto{\pgfqpoint{0.031064in}{-0.018041in}}{\pgfqpoint{0.034722in}{-0.009208in}}{\pgfqpoint{0.034722in}{0.000000in}}%
\pgfpathcurveto{\pgfqpoint{0.034722in}{0.009208in}}{\pgfqpoint{0.031064in}{0.018041in}}{\pgfqpoint{0.024552in}{0.024552in}}%
\pgfpathcurveto{\pgfqpoint{0.018041in}{0.031064in}}{\pgfqpoint{0.009208in}{0.034722in}}{\pgfqpoint{0.000000in}{0.034722in}}%
\pgfpathcurveto{\pgfqpoint{-0.009208in}{0.034722in}}{\pgfqpoint{-0.018041in}{0.031064in}}{\pgfqpoint{-0.024552in}{0.024552in}}%
\pgfpathcurveto{\pgfqpoint{-0.031064in}{0.018041in}}{\pgfqpoint{-0.034722in}{0.009208in}}{\pgfqpoint{-0.034722in}{0.000000in}}%
\pgfpathcurveto{\pgfqpoint{-0.034722in}{-0.009208in}}{\pgfqpoint{-0.031064in}{-0.018041in}}{\pgfqpoint{-0.024552in}{-0.024552in}}%
\pgfpathcurveto{\pgfqpoint{-0.018041in}{-0.031064in}}{\pgfqpoint{-0.009208in}{-0.034722in}}{\pgfqpoint{0.000000in}{-0.034722in}}%
\pgfpathclose%
\pgfusepath{stroke,fill}%
}%
\begin{pgfscope}%
\pgfsys@transformshift{10.059943in}{14.399630in}%
\pgfsys@useobject{currentmarker}{}%
\end{pgfscope}%
\begin{pgfscope}%
\pgfsys@transformshift{10.183166in}{14.397078in}%
\pgfsys@useobject{currentmarker}{}%
\end{pgfscope}%
\begin{pgfscope}%
\pgfsys@transformshift{10.306389in}{13.625105in}%
\pgfsys@useobject{currentmarker}{}%
\end{pgfscope}%
\begin{pgfscope}%
\pgfsys@transformshift{10.429612in}{13.660246in}%
\pgfsys@useobject{currentmarker}{}%
\end{pgfscope}%
\begin{pgfscope}%
\pgfsys@transformshift{10.552835in}{13.635487in}%
\pgfsys@useobject{currentmarker}{}%
\end{pgfscope}%
\begin{pgfscope}%
\pgfsys@transformshift{10.676058in}{13.652440in}%
\pgfsys@useobject{currentmarker}{}%
\end{pgfscope}%
\begin{pgfscope}%
\pgfsys@transformshift{10.799281in}{13.639411in}%
\pgfsys@useobject{currentmarker}{}%
\end{pgfscope}%
\begin{pgfscope}%
\pgfsys@transformshift{10.922504in}{13.644059in}%
\pgfsys@useobject{currentmarker}{}%
\end{pgfscope}%
\begin{pgfscope}%
\pgfsys@transformshift{11.045727in}{13.663785in}%
\pgfsys@useobject{currentmarker}{}%
\end{pgfscope}%
\begin{pgfscope}%
\pgfsys@transformshift{11.168950in}{13.624725in}%
\pgfsys@useobject{currentmarker}{}%
\end{pgfscope}%
\begin{pgfscope}%
\pgfsys@transformshift{11.292173in}{13.653613in}%
\pgfsys@useobject{currentmarker}{}%
\end{pgfscope}%
\begin{pgfscope}%
\pgfsys@transformshift{11.415396in}{13.626753in}%
\pgfsys@useobject{currentmarker}{}%
\end{pgfscope}%
\begin{pgfscope}%
\pgfsys@transformshift{11.538619in}{13.642618in}%
\pgfsys@useobject{currentmarker}{}%
\end{pgfscope}%
\begin{pgfscope}%
\pgfsys@transformshift{11.661842in}{13.625213in}%
\pgfsys@useobject{currentmarker}{}%
\end{pgfscope}%
\begin{pgfscope}%
\pgfsys@transformshift{11.785065in}{13.602345in}%
\pgfsys@useobject{currentmarker}{}%
\end{pgfscope}%
\begin{pgfscope}%
\pgfsys@transformshift{11.908288in}{13.637704in}%
\pgfsys@useobject{currentmarker}{}%
\end{pgfscope}%
\begin{pgfscope}%
\pgfsys@transformshift{12.031511in}{13.652483in}%
\pgfsys@useobject{currentmarker}{}%
\end{pgfscope}%
\begin{pgfscope}%
\pgfsys@transformshift{12.154734in}{13.614715in}%
\pgfsys@useobject{currentmarker}{}%
\end{pgfscope}%
\begin{pgfscope}%
\pgfsys@transformshift{12.277957in}{13.646557in}%
\pgfsys@useobject{currentmarker}{}%
\end{pgfscope}%
\begin{pgfscope}%
\pgfsys@transformshift{12.401180in}{13.625305in}%
\pgfsys@useobject{currentmarker}{}%
\end{pgfscope}%
\begin{pgfscope}%
\pgfsys@transformshift{12.524403in}{13.645622in}%
\pgfsys@useobject{currentmarker}{}%
\end{pgfscope}%
\begin{pgfscope}%
\pgfsys@transformshift{12.647626in}{13.653251in}%
\pgfsys@useobject{currentmarker}{}%
\end{pgfscope}%
\begin{pgfscope}%
\pgfsys@transformshift{12.770849in}{13.634705in}%
\pgfsys@useobject{currentmarker}{}%
\end{pgfscope}%
\begin{pgfscope}%
\pgfsys@transformshift{12.894072in}{13.660459in}%
\pgfsys@useobject{currentmarker}{}%
\end{pgfscope}%
\begin{pgfscope}%
\pgfsys@transformshift{13.017294in}{13.618121in}%
\pgfsys@useobject{currentmarker}{}%
\end{pgfscope}%
\begin{pgfscope}%
\pgfsys@transformshift{13.140517in}{13.632358in}%
\pgfsys@useobject{currentmarker}{}%
\end{pgfscope}%
\begin{pgfscope}%
\pgfsys@transformshift{13.263740in}{13.648504in}%
\pgfsys@useobject{currentmarker}{}%
\end{pgfscope}%
\begin{pgfscope}%
\pgfsys@transformshift{13.386963in}{13.628033in}%
\pgfsys@useobject{currentmarker}{}%
\end{pgfscope}%
\begin{pgfscope}%
\pgfsys@transformshift{13.510186in}{13.640320in}%
\pgfsys@useobject{currentmarker}{}%
\end{pgfscope}%
\begin{pgfscope}%
\pgfsys@transformshift{13.633409in}{13.632564in}%
\pgfsys@useobject{currentmarker}{}%
\end{pgfscope}%
\begin{pgfscope}%
\pgfsys@transformshift{13.756632in}{13.627775in}%
\pgfsys@useobject{currentmarker}{}%
\end{pgfscope}%
\begin{pgfscope}%
\pgfsys@transformshift{13.879855in}{13.641695in}%
\pgfsys@useobject{currentmarker}{}%
\end{pgfscope}%
\begin{pgfscope}%
\pgfsys@transformshift{14.003078in}{13.654167in}%
\pgfsys@useobject{currentmarker}{}%
\end{pgfscope}%
\begin{pgfscope}%
\pgfsys@transformshift{14.126301in}{13.694911in}%
\pgfsys@useobject{currentmarker}{}%
\end{pgfscope}%
\begin{pgfscope}%
\pgfsys@transformshift{14.249524in}{13.687956in}%
\pgfsys@useobject{currentmarker}{}%
\end{pgfscope}%
\begin{pgfscope}%
\pgfsys@transformshift{14.372747in}{13.603316in}%
\pgfsys@useobject{currentmarker}{}%
\end{pgfscope}%
\begin{pgfscope}%
\pgfsys@transformshift{14.495970in}{13.645697in}%
\pgfsys@useobject{currentmarker}{}%
\end{pgfscope}%
\begin{pgfscope}%
\pgfsys@transformshift{14.619193in}{13.628356in}%
\pgfsys@useobject{currentmarker}{}%
\end{pgfscope}%
\begin{pgfscope}%
\pgfsys@transformshift{14.742416in}{13.641580in}%
\pgfsys@useobject{currentmarker}{}%
\end{pgfscope}%
\begin{pgfscope}%
\pgfsys@transformshift{14.865639in}{13.652100in}%
\pgfsys@useobject{currentmarker}{}%
\end{pgfscope}%
\begin{pgfscope}%
\pgfsys@transformshift{14.988862in}{13.625585in}%
\pgfsys@useobject{currentmarker}{}%
\end{pgfscope}%
\end{pgfscope}%
\begin{pgfscope}%
\pgfsetrectcap%
\pgfsetmiterjoin%
\pgfsetlinewidth{0.803000pt}%
\definecolor{currentstroke}{rgb}{1.000000,1.000000,1.000000}%
\pgfsetstrokecolor{currentstroke}%
\pgfsetdash{}{0pt}%
\pgfpathmoveto{\pgfqpoint{9.810417in}{13.561628in}}%
\pgfpathlineto{\pgfqpoint{9.810417in}{14.439535in}}%
\pgfusepath{stroke}%
\end{pgfscope}%
\begin{pgfscope}%
\pgfsetrectcap%
\pgfsetmiterjoin%
\pgfsetlinewidth{0.803000pt}%
\definecolor{currentstroke}{rgb}{1.000000,1.000000,1.000000}%
\pgfsetstrokecolor{currentstroke}%
\pgfsetdash{}{0pt}%
\pgfpathmoveto{\pgfqpoint{15.300000in}{13.561628in}}%
\pgfpathlineto{\pgfqpoint{15.300000in}{14.439535in}}%
\pgfusepath{stroke}%
\end{pgfscope}%
\begin{pgfscope}%
\pgfsetrectcap%
\pgfsetmiterjoin%
\pgfsetlinewidth{0.803000pt}%
\definecolor{currentstroke}{rgb}{1.000000,1.000000,1.000000}%
\pgfsetstrokecolor{currentstroke}%
\pgfsetdash{}{0pt}%
\pgfpathmoveto{\pgfqpoint{9.810417in}{13.561628in}}%
\pgfpathlineto{\pgfqpoint{15.300000in}{13.561628in}}%
\pgfusepath{stroke}%
\end{pgfscope}%
\begin{pgfscope}%
\pgfsetrectcap%
\pgfsetmiterjoin%
\pgfsetlinewidth{0.803000pt}%
\definecolor{currentstroke}{rgb}{1.000000,1.000000,1.000000}%
\pgfsetstrokecolor{currentstroke}%
\pgfsetdash{}{0pt}%
\pgfpathmoveto{\pgfqpoint{9.810417in}{14.439535in}}%
\pgfpathlineto{\pgfqpoint{15.300000in}{14.439535in}}%
\pgfusepath{stroke}%
\end{pgfscope}%
\begin{pgfscope}%
\definecolor{textcolor}{rgb}{0.150000,0.150000,0.150000}%
\pgfsetstrokecolor{textcolor}%
\pgfsetfillcolor{textcolor}%
\pgftext[x=12.555208in,y=14.522868in,,base]{\color{textcolor}\rmfamily\fontsize{16.800000}{20.160000}\selectfont Partial Autocorrelation}%
\end{pgfscope}%
\begin{pgfscope}%
\pgfsetbuttcap%
\pgfsetmiterjoin%
\definecolor{currentfill}{rgb}{0.917647,0.917647,0.949020}%
\pgfsetfillcolor{currentfill}%
\pgfsetlinewidth{0.000000pt}%
\definecolor{currentstroke}{rgb}{0.000000,0.000000,0.000000}%
\pgfsetstrokecolor{currentstroke}%
\pgfsetstrokeopacity{0.000000}%
\pgfsetdash{}{0pt}%
\pgfpathmoveto{\pgfqpoint{2.125000in}{11.981395in}}%
\pgfpathlineto{\pgfqpoint{7.614583in}{11.981395in}}%
\pgfpathlineto{\pgfqpoint{7.614583in}{12.859302in}}%
\pgfpathlineto{\pgfqpoint{2.125000in}{12.859302in}}%
\pgfpathclose%
\pgfusepath{fill}%
\end{pgfscope}%
\begin{pgfscope}%
\pgfpathrectangle{\pgfqpoint{2.125000in}{11.981395in}}{\pgfqpoint{5.489583in}{0.877907in}}%
\pgfusepath{clip}%
\pgfsetroundcap%
\pgfsetroundjoin%
\pgfsetlinewidth{0.803000pt}%
\definecolor{currentstroke}{rgb}{1.000000,1.000000,1.000000}%
\pgfsetstrokecolor{currentstroke}%
\pgfsetdash{}{0pt}%
\pgfpathmoveto{\pgfqpoint{2.374527in}{11.981395in}}%
\pgfpathlineto{\pgfqpoint{2.374527in}{12.859302in}}%
\pgfusepath{stroke}%
\end{pgfscope}%
\begin{pgfscope}%
\definecolor{textcolor}{rgb}{0.150000,0.150000,0.150000}%
\pgfsetstrokecolor{textcolor}%
\pgfsetfillcolor{textcolor}%
\pgftext[x=2.374527in,y=11.884173in,,top]{\color{textcolor}\rmfamily\fontsize{14.000000}{16.800000}\selectfont 0}%
\end{pgfscope}%
\begin{pgfscope}%
\pgfpathrectangle{\pgfqpoint{2.125000in}{11.981395in}}{\pgfqpoint{5.489583in}{0.877907in}}%
\pgfusepath{clip}%
\pgfsetroundcap%
\pgfsetroundjoin%
\pgfsetlinewidth{0.803000pt}%
\definecolor{currentstroke}{rgb}{1.000000,1.000000,1.000000}%
\pgfsetstrokecolor{currentstroke}%
\pgfsetdash{}{0pt}%
\pgfpathmoveto{\pgfqpoint{2.990641in}{11.981395in}}%
\pgfpathlineto{\pgfqpoint{2.990641in}{12.859302in}}%
\pgfusepath{stroke}%
\end{pgfscope}%
\begin{pgfscope}%
\definecolor{textcolor}{rgb}{0.150000,0.150000,0.150000}%
\pgfsetstrokecolor{textcolor}%
\pgfsetfillcolor{textcolor}%
\pgftext[x=2.990641in,y=11.884173in,,top]{\color{textcolor}\rmfamily\fontsize{14.000000}{16.800000}\selectfont 5}%
\end{pgfscope}%
\begin{pgfscope}%
\pgfpathrectangle{\pgfqpoint{2.125000in}{11.981395in}}{\pgfqpoint{5.489583in}{0.877907in}}%
\pgfusepath{clip}%
\pgfsetroundcap%
\pgfsetroundjoin%
\pgfsetlinewidth{0.803000pt}%
\definecolor{currentstroke}{rgb}{1.000000,1.000000,1.000000}%
\pgfsetstrokecolor{currentstroke}%
\pgfsetdash{}{0pt}%
\pgfpathmoveto{\pgfqpoint{3.606756in}{11.981395in}}%
\pgfpathlineto{\pgfqpoint{3.606756in}{12.859302in}}%
\pgfusepath{stroke}%
\end{pgfscope}%
\begin{pgfscope}%
\definecolor{textcolor}{rgb}{0.150000,0.150000,0.150000}%
\pgfsetstrokecolor{textcolor}%
\pgfsetfillcolor{textcolor}%
\pgftext[x=3.606756in,y=11.884173in,,top]{\color{textcolor}\rmfamily\fontsize{14.000000}{16.800000}\selectfont 10}%
\end{pgfscope}%
\begin{pgfscope}%
\pgfpathrectangle{\pgfqpoint{2.125000in}{11.981395in}}{\pgfqpoint{5.489583in}{0.877907in}}%
\pgfusepath{clip}%
\pgfsetroundcap%
\pgfsetroundjoin%
\pgfsetlinewidth{0.803000pt}%
\definecolor{currentstroke}{rgb}{1.000000,1.000000,1.000000}%
\pgfsetstrokecolor{currentstroke}%
\pgfsetdash{}{0pt}%
\pgfpathmoveto{\pgfqpoint{4.222871in}{11.981395in}}%
\pgfpathlineto{\pgfqpoint{4.222871in}{12.859302in}}%
\pgfusepath{stroke}%
\end{pgfscope}%
\begin{pgfscope}%
\definecolor{textcolor}{rgb}{0.150000,0.150000,0.150000}%
\pgfsetstrokecolor{textcolor}%
\pgfsetfillcolor{textcolor}%
\pgftext[x=4.222871in,y=11.884173in,,top]{\color{textcolor}\rmfamily\fontsize{14.000000}{16.800000}\selectfont 15}%
\end{pgfscope}%
\begin{pgfscope}%
\pgfpathrectangle{\pgfqpoint{2.125000in}{11.981395in}}{\pgfqpoint{5.489583in}{0.877907in}}%
\pgfusepath{clip}%
\pgfsetroundcap%
\pgfsetroundjoin%
\pgfsetlinewidth{0.803000pt}%
\definecolor{currentstroke}{rgb}{1.000000,1.000000,1.000000}%
\pgfsetstrokecolor{currentstroke}%
\pgfsetdash{}{0pt}%
\pgfpathmoveto{\pgfqpoint{4.838986in}{11.981395in}}%
\pgfpathlineto{\pgfqpoint{4.838986in}{12.859302in}}%
\pgfusepath{stroke}%
\end{pgfscope}%
\begin{pgfscope}%
\definecolor{textcolor}{rgb}{0.150000,0.150000,0.150000}%
\pgfsetstrokecolor{textcolor}%
\pgfsetfillcolor{textcolor}%
\pgftext[x=4.838986in,y=11.884173in,,top]{\color{textcolor}\rmfamily\fontsize{14.000000}{16.800000}\selectfont 20}%
\end{pgfscope}%
\begin{pgfscope}%
\pgfpathrectangle{\pgfqpoint{2.125000in}{11.981395in}}{\pgfqpoint{5.489583in}{0.877907in}}%
\pgfusepath{clip}%
\pgfsetroundcap%
\pgfsetroundjoin%
\pgfsetlinewidth{0.803000pt}%
\definecolor{currentstroke}{rgb}{1.000000,1.000000,1.000000}%
\pgfsetstrokecolor{currentstroke}%
\pgfsetdash{}{0pt}%
\pgfpathmoveto{\pgfqpoint{5.455101in}{11.981395in}}%
\pgfpathlineto{\pgfqpoint{5.455101in}{12.859302in}}%
\pgfusepath{stroke}%
\end{pgfscope}%
\begin{pgfscope}%
\definecolor{textcolor}{rgb}{0.150000,0.150000,0.150000}%
\pgfsetstrokecolor{textcolor}%
\pgfsetfillcolor{textcolor}%
\pgftext[x=5.455101in,y=11.884173in,,top]{\color{textcolor}\rmfamily\fontsize{14.000000}{16.800000}\selectfont 25}%
\end{pgfscope}%
\begin{pgfscope}%
\pgfpathrectangle{\pgfqpoint{2.125000in}{11.981395in}}{\pgfqpoint{5.489583in}{0.877907in}}%
\pgfusepath{clip}%
\pgfsetroundcap%
\pgfsetroundjoin%
\pgfsetlinewidth{0.803000pt}%
\definecolor{currentstroke}{rgb}{1.000000,1.000000,1.000000}%
\pgfsetstrokecolor{currentstroke}%
\pgfsetdash{}{0pt}%
\pgfpathmoveto{\pgfqpoint{6.071216in}{11.981395in}}%
\pgfpathlineto{\pgfqpoint{6.071216in}{12.859302in}}%
\pgfusepath{stroke}%
\end{pgfscope}%
\begin{pgfscope}%
\definecolor{textcolor}{rgb}{0.150000,0.150000,0.150000}%
\pgfsetstrokecolor{textcolor}%
\pgfsetfillcolor{textcolor}%
\pgftext[x=6.071216in,y=11.884173in,,top]{\color{textcolor}\rmfamily\fontsize{14.000000}{16.800000}\selectfont 30}%
\end{pgfscope}%
\begin{pgfscope}%
\pgfpathrectangle{\pgfqpoint{2.125000in}{11.981395in}}{\pgfqpoint{5.489583in}{0.877907in}}%
\pgfusepath{clip}%
\pgfsetroundcap%
\pgfsetroundjoin%
\pgfsetlinewidth{0.803000pt}%
\definecolor{currentstroke}{rgb}{1.000000,1.000000,1.000000}%
\pgfsetstrokecolor{currentstroke}%
\pgfsetdash{}{0pt}%
\pgfpathmoveto{\pgfqpoint{6.687330in}{11.981395in}}%
\pgfpathlineto{\pgfqpoint{6.687330in}{12.859302in}}%
\pgfusepath{stroke}%
\end{pgfscope}%
\begin{pgfscope}%
\definecolor{textcolor}{rgb}{0.150000,0.150000,0.150000}%
\pgfsetstrokecolor{textcolor}%
\pgfsetfillcolor{textcolor}%
\pgftext[x=6.687330in,y=11.884173in,,top]{\color{textcolor}\rmfamily\fontsize{14.000000}{16.800000}\selectfont 35}%
\end{pgfscope}%
\begin{pgfscope}%
\pgfpathrectangle{\pgfqpoint{2.125000in}{11.981395in}}{\pgfqpoint{5.489583in}{0.877907in}}%
\pgfusepath{clip}%
\pgfsetroundcap%
\pgfsetroundjoin%
\pgfsetlinewidth{0.803000pt}%
\definecolor{currentstroke}{rgb}{1.000000,1.000000,1.000000}%
\pgfsetstrokecolor{currentstroke}%
\pgfsetdash{}{0pt}%
\pgfpathmoveto{\pgfqpoint{7.303445in}{11.981395in}}%
\pgfpathlineto{\pgfqpoint{7.303445in}{12.859302in}}%
\pgfusepath{stroke}%
\end{pgfscope}%
\begin{pgfscope}%
\definecolor{textcolor}{rgb}{0.150000,0.150000,0.150000}%
\pgfsetstrokecolor{textcolor}%
\pgfsetfillcolor{textcolor}%
\pgftext[x=7.303445in,y=11.884173in,,top]{\color{textcolor}\rmfamily\fontsize{14.000000}{16.800000}\selectfont 40}%
\end{pgfscope}%
\begin{pgfscope}%
\pgfpathrectangle{\pgfqpoint{2.125000in}{11.981395in}}{\pgfqpoint{5.489583in}{0.877907in}}%
\pgfusepath{clip}%
\pgfsetroundcap%
\pgfsetroundjoin%
\pgfsetlinewidth{0.803000pt}%
\definecolor{currentstroke}{rgb}{1.000000,1.000000,1.000000}%
\pgfsetstrokecolor{currentstroke}%
\pgfsetdash{}{0pt}%
\pgfpathmoveto{\pgfqpoint{2.125000in}{12.257086in}}%
\pgfpathlineto{\pgfqpoint{7.614583in}{12.257086in}}%
\pgfusepath{stroke}%
\end{pgfscope}%
\begin{pgfscope}%
\definecolor{textcolor}{rgb}{0.150000,0.150000,0.150000}%
\pgfsetstrokecolor{textcolor}%
\pgfsetfillcolor{textcolor}%
\pgftext[x=1.904066in,y=12.183220in,left,base]{\color{textcolor}\rmfamily\fontsize{14.000000}{16.800000}\selectfont 0}%
\end{pgfscope}%
\begin{pgfscope}%
\pgfpathrectangle{\pgfqpoint{2.125000in}{11.981395in}}{\pgfqpoint{5.489583in}{0.877907in}}%
\pgfusepath{clip}%
\pgfsetroundcap%
\pgfsetroundjoin%
\pgfsetlinewidth{0.803000pt}%
\definecolor{currentstroke}{rgb}{1.000000,1.000000,1.000000}%
\pgfsetstrokecolor{currentstroke}%
\pgfsetdash{}{0pt}%
\pgfpathmoveto{\pgfqpoint{2.125000in}{12.819397in}}%
\pgfpathlineto{\pgfqpoint{7.614583in}{12.819397in}}%
\pgfusepath{stroke}%
\end{pgfscope}%
\begin{pgfscope}%
\definecolor{textcolor}{rgb}{0.150000,0.150000,0.150000}%
\pgfsetstrokecolor{textcolor}%
\pgfsetfillcolor{textcolor}%
\pgftext[x=1.904066in,y=12.745531in,left,base]{\color{textcolor}\rmfamily\fontsize{14.000000}{16.800000}\selectfont 1}%
\end{pgfscope}%
\begin{pgfscope}%
\pgfpathrectangle{\pgfqpoint{2.125000in}{11.981395in}}{\pgfqpoint{5.489583in}{0.877907in}}%
\pgfusepath{clip}%
\pgfsetbuttcap%
\pgfsetroundjoin%
\definecolor{currentfill}{rgb}{0.121569,0.466667,0.705882}%
\pgfsetfillcolor{currentfill}%
\pgfsetfillopacity{0.250000}%
\pgfsetlinewidth{1.003750pt}%
\definecolor{currentstroke}{rgb}{1.000000,1.000000,1.000000}%
\pgfsetstrokecolor{currentstroke}%
\pgfsetstrokeopacity{0.250000}%
\pgfsetdash{}{0pt}%
\pgfpathmoveto{\pgfqpoint{2.436138in}{12.285457in}}%
\pgfpathlineto{\pgfqpoint{2.436138in}{12.228715in}}%
\pgfpathlineto{\pgfqpoint{2.620972in}{12.208063in}}%
\pgfpathlineto{\pgfqpoint{2.744195in}{12.193916in}}%
\pgfpathlineto{\pgfqpoint{2.867418in}{12.182476in}}%
\pgfpathlineto{\pgfqpoint{2.990641in}{12.172636in}}%
\pgfpathlineto{\pgfqpoint{3.113864in}{12.163888in}}%
\pgfpathlineto{\pgfqpoint{3.237087in}{12.155948in}}%
\pgfpathlineto{\pgfqpoint{3.360310in}{12.148639in}}%
\pgfpathlineto{\pgfqpoint{3.483533in}{12.141840in}}%
\pgfpathlineto{\pgfqpoint{3.606756in}{12.135467in}}%
\pgfpathlineto{\pgfqpoint{3.729979in}{12.129451in}}%
\pgfpathlineto{\pgfqpoint{3.853202in}{12.123743in}}%
\pgfpathlineto{\pgfqpoint{3.976425in}{12.118306in}}%
\pgfpathlineto{\pgfqpoint{4.099648in}{12.113108in}}%
\pgfpathlineto{\pgfqpoint{4.222871in}{12.108123in}}%
\pgfpathlineto{\pgfqpoint{4.346094in}{12.103329in}}%
\pgfpathlineto{\pgfqpoint{4.469317in}{12.098707in}}%
\pgfpathlineto{\pgfqpoint{4.592540in}{12.094240in}}%
\pgfpathlineto{\pgfqpoint{4.715763in}{12.089919in}}%
\pgfpathlineto{\pgfqpoint{4.838986in}{12.085734in}}%
\pgfpathlineto{\pgfqpoint{4.962209in}{12.081678in}}%
\pgfpathlineto{\pgfqpoint{5.085432in}{12.077741in}}%
\pgfpathlineto{\pgfqpoint{5.208655in}{12.073915in}}%
\pgfpathlineto{\pgfqpoint{5.331878in}{12.070194in}}%
\pgfpathlineto{\pgfqpoint{5.455101in}{12.066574in}}%
\pgfpathlineto{\pgfqpoint{5.578324in}{12.063046in}}%
\pgfpathlineto{\pgfqpoint{5.701547in}{12.059608in}}%
\pgfpathlineto{\pgfqpoint{5.824770in}{12.056252in}}%
\pgfpathlineto{\pgfqpoint{5.947993in}{12.052976in}}%
\pgfpathlineto{\pgfqpoint{6.071216in}{12.049774in}}%
\pgfpathlineto{\pgfqpoint{6.194439in}{12.046645in}}%
\pgfpathlineto{\pgfqpoint{6.317662in}{12.043585in}}%
\pgfpathlineto{\pgfqpoint{6.440885in}{12.040592in}}%
\pgfpathlineto{\pgfqpoint{6.564108in}{12.037662in}}%
\pgfpathlineto{\pgfqpoint{6.687330in}{12.034792in}}%
\pgfpathlineto{\pgfqpoint{6.810553in}{12.031981in}}%
\pgfpathlineto{\pgfqpoint{6.933776in}{12.029229in}}%
\pgfpathlineto{\pgfqpoint{7.056999in}{12.026533in}}%
\pgfpathlineto{\pgfqpoint{7.180222in}{12.023891in}}%
\pgfpathlineto{\pgfqpoint{7.365057in}{12.021300in}}%
\pgfpathlineto{\pgfqpoint{7.365057in}{12.492872in}}%
\pgfpathlineto{\pgfqpoint{7.365057in}{12.492872in}}%
\pgfpathlineto{\pgfqpoint{7.180222in}{12.490281in}}%
\pgfpathlineto{\pgfqpoint{7.056999in}{12.487639in}}%
\pgfpathlineto{\pgfqpoint{6.933776in}{12.484943in}}%
\pgfpathlineto{\pgfqpoint{6.810553in}{12.482190in}}%
\pgfpathlineto{\pgfqpoint{6.687330in}{12.479380in}}%
\pgfpathlineto{\pgfqpoint{6.564108in}{12.476510in}}%
\pgfpathlineto{\pgfqpoint{6.440885in}{12.473580in}}%
\pgfpathlineto{\pgfqpoint{6.317662in}{12.470586in}}%
\pgfpathlineto{\pgfqpoint{6.194439in}{12.467527in}}%
\pgfpathlineto{\pgfqpoint{6.071216in}{12.464398in}}%
\pgfpathlineto{\pgfqpoint{5.947993in}{12.461196in}}%
\pgfpathlineto{\pgfqpoint{5.824770in}{12.457919in}}%
\pgfpathlineto{\pgfqpoint{5.701547in}{12.454564in}}%
\pgfpathlineto{\pgfqpoint{5.578324in}{12.451125in}}%
\pgfpathlineto{\pgfqpoint{5.455101in}{12.447598in}}%
\pgfpathlineto{\pgfqpoint{5.331878in}{12.443977in}}%
\pgfpathlineto{\pgfqpoint{5.208655in}{12.440257in}}%
\pgfpathlineto{\pgfqpoint{5.085432in}{12.436431in}}%
\pgfpathlineto{\pgfqpoint{4.962209in}{12.432494in}}%
\pgfpathlineto{\pgfqpoint{4.838986in}{12.428438in}}%
\pgfpathlineto{\pgfqpoint{4.715763in}{12.424253in}}%
\pgfpathlineto{\pgfqpoint{4.592540in}{12.419932in}}%
\pgfpathlineto{\pgfqpoint{4.469317in}{12.415465in}}%
\pgfpathlineto{\pgfqpoint{4.346094in}{12.410843in}}%
\pgfpathlineto{\pgfqpoint{4.222871in}{12.406049in}}%
\pgfpathlineto{\pgfqpoint{4.099648in}{12.401063in}}%
\pgfpathlineto{\pgfqpoint{3.976425in}{12.395865in}}%
\pgfpathlineto{\pgfqpoint{3.853202in}{12.390428in}}%
\pgfpathlineto{\pgfqpoint{3.729979in}{12.384721in}}%
\pgfpathlineto{\pgfqpoint{3.606756in}{12.378705in}}%
\pgfpathlineto{\pgfqpoint{3.483533in}{12.372332in}}%
\pgfpathlineto{\pgfqpoint{3.360310in}{12.365533in}}%
\pgfpathlineto{\pgfqpoint{3.237087in}{12.358224in}}%
\pgfpathlineto{\pgfqpoint{3.113864in}{12.350284in}}%
\pgfpathlineto{\pgfqpoint{2.990641in}{12.341536in}}%
\pgfpathlineto{\pgfqpoint{2.867418in}{12.331696in}}%
\pgfpathlineto{\pgfqpoint{2.744195in}{12.320256in}}%
\pgfpathlineto{\pgfqpoint{2.620972in}{12.306109in}}%
\pgfpathlineto{\pgfqpoint{2.436138in}{12.285457in}}%
\pgfpathclose%
\pgfusepath{stroke,fill}%
\end{pgfscope}%
\begin{pgfscope}%
\pgfpathrectangle{\pgfqpoint{2.125000in}{11.981395in}}{\pgfqpoint{5.489583in}{0.877907in}}%
\pgfusepath{clip}%
\pgfsetbuttcap%
\pgfsetroundjoin%
\pgfsetlinewidth{1.505625pt}%
\definecolor{currentstroke}{rgb}{0.000000,0.000000,0.000000}%
\pgfsetstrokecolor{currentstroke}%
\pgfsetdash{}{0pt}%
\pgfpathmoveto{\pgfqpoint{2.374527in}{12.257086in}}%
\pgfpathlineto{\pgfqpoint{2.374527in}{12.819397in}}%
\pgfusepath{stroke}%
\end{pgfscope}%
\begin{pgfscope}%
\pgfpathrectangle{\pgfqpoint{2.125000in}{11.981395in}}{\pgfqpoint{5.489583in}{0.877907in}}%
\pgfusepath{clip}%
\pgfsetbuttcap%
\pgfsetroundjoin%
\pgfsetlinewidth{1.505625pt}%
\definecolor{currentstroke}{rgb}{0.000000,0.000000,0.000000}%
\pgfsetstrokecolor{currentstroke}%
\pgfsetdash{}{0pt}%
\pgfpathmoveto{\pgfqpoint{2.497749in}{12.257086in}}%
\pgfpathlineto{\pgfqpoint{2.497749in}{12.817378in}}%
\pgfusepath{stroke}%
\end{pgfscope}%
\begin{pgfscope}%
\pgfpathrectangle{\pgfqpoint{2.125000in}{11.981395in}}{\pgfqpoint{5.489583in}{0.877907in}}%
\pgfusepath{clip}%
\pgfsetbuttcap%
\pgfsetroundjoin%
\pgfsetlinewidth{1.505625pt}%
\definecolor{currentstroke}{rgb}{0.000000,0.000000,0.000000}%
\pgfsetstrokecolor{currentstroke}%
\pgfsetdash{}{0pt}%
\pgfpathmoveto{\pgfqpoint{2.620972in}{12.257086in}}%
\pgfpathlineto{\pgfqpoint{2.620972in}{12.815418in}}%
\pgfusepath{stroke}%
\end{pgfscope}%
\begin{pgfscope}%
\pgfpathrectangle{\pgfqpoint{2.125000in}{11.981395in}}{\pgfqpoint{5.489583in}{0.877907in}}%
\pgfusepath{clip}%
\pgfsetbuttcap%
\pgfsetroundjoin%
\pgfsetlinewidth{1.505625pt}%
\definecolor{currentstroke}{rgb}{0.000000,0.000000,0.000000}%
\pgfsetstrokecolor{currentstroke}%
\pgfsetdash{}{0pt}%
\pgfpathmoveto{\pgfqpoint{2.744195in}{12.257086in}}%
\pgfpathlineto{\pgfqpoint{2.744195in}{12.813480in}}%
\pgfusepath{stroke}%
\end{pgfscope}%
\begin{pgfscope}%
\pgfpathrectangle{\pgfqpoint{2.125000in}{11.981395in}}{\pgfqpoint{5.489583in}{0.877907in}}%
\pgfusepath{clip}%
\pgfsetbuttcap%
\pgfsetroundjoin%
\pgfsetlinewidth{1.505625pt}%
\definecolor{currentstroke}{rgb}{0.000000,0.000000,0.000000}%
\pgfsetstrokecolor{currentstroke}%
\pgfsetdash{}{0pt}%
\pgfpathmoveto{\pgfqpoint{2.867418in}{12.257086in}}%
\pgfpathlineto{\pgfqpoint{2.867418in}{12.811532in}}%
\pgfusepath{stroke}%
\end{pgfscope}%
\begin{pgfscope}%
\pgfpathrectangle{\pgfqpoint{2.125000in}{11.981395in}}{\pgfqpoint{5.489583in}{0.877907in}}%
\pgfusepath{clip}%
\pgfsetbuttcap%
\pgfsetroundjoin%
\pgfsetlinewidth{1.505625pt}%
\definecolor{currentstroke}{rgb}{0.000000,0.000000,0.000000}%
\pgfsetstrokecolor{currentstroke}%
\pgfsetdash{}{0pt}%
\pgfpathmoveto{\pgfqpoint{2.990641in}{12.257086in}}%
\pgfpathlineto{\pgfqpoint{2.990641in}{12.809570in}}%
\pgfusepath{stroke}%
\end{pgfscope}%
\begin{pgfscope}%
\pgfpathrectangle{\pgfqpoint{2.125000in}{11.981395in}}{\pgfqpoint{5.489583in}{0.877907in}}%
\pgfusepath{clip}%
\pgfsetbuttcap%
\pgfsetroundjoin%
\pgfsetlinewidth{1.505625pt}%
\definecolor{currentstroke}{rgb}{0.000000,0.000000,0.000000}%
\pgfsetstrokecolor{currentstroke}%
\pgfsetdash{}{0pt}%
\pgfpathmoveto{\pgfqpoint{3.113864in}{12.257086in}}%
\pgfpathlineto{\pgfqpoint{3.113864in}{12.807608in}}%
\pgfusepath{stroke}%
\end{pgfscope}%
\begin{pgfscope}%
\pgfpathrectangle{\pgfqpoint{2.125000in}{11.981395in}}{\pgfqpoint{5.489583in}{0.877907in}}%
\pgfusepath{clip}%
\pgfsetbuttcap%
\pgfsetroundjoin%
\pgfsetlinewidth{1.505625pt}%
\definecolor{currentstroke}{rgb}{0.000000,0.000000,0.000000}%
\pgfsetstrokecolor{currentstroke}%
\pgfsetdash{}{0pt}%
\pgfpathmoveto{\pgfqpoint{3.237087in}{12.257086in}}%
\pgfpathlineto{\pgfqpoint{3.237087in}{12.805615in}}%
\pgfusepath{stroke}%
\end{pgfscope}%
\begin{pgfscope}%
\pgfpathrectangle{\pgfqpoint{2.125000in}{11.981395in}}{\pgfqpoint{5.489583in}{0.877907in}}%
\pgfusepath{clip}%
\pgfsetbuttcap%
\pgfsetroundjoin%
\pgfsetlinewidth{1.505625pt}%
\definecolor{currentstroke}{rgb}{0.000000,0.000000,0.000000}%
\pgfsetstrokecolor{currentstroke}%
\pgfsetdash{}{0pt}%
\pgfpathmoveto{\pgfqpoint{3.360310in}{12.257086in}}%
\pgfpathlineto{\pgfqpoint{3.360310in}{12.803613in}}%
\pgfusepath{stroke}%
\end{pgfscope}%
\begin{pgfscope}%
\pgfpathrectangle{\pgfqpoint{2.125000in}{11.981395in}}{\pgfqpoint{5.489583in}{0.877907in}}%
\pgfusepath{clip}%
\pgfsetbuttcap%
\pgfsetroundjoin%
\pgfsetlinewidth{1.505625pt}%
\definecolor{currentstroke}{rgb}{0.000000,0.000000,0.000000}%
\pgfsetstrokecolor{currentstroke}%
\pgfsetdash{}{0pt}%
\pgfpathmoveto{\pgfqpoint{3.483533in}{12.257086in}}%
\pgfpathlineto{\pgfqpoint{3.483533in}{12.801601in}}%
\pgfusepath{stroke}%
\end{pgfscope}%
\begin{pgfscope}%
\pgfpathrectangle{\pgfqpoint{2.125000in}{11.981395in}}{\pgfqpoint{5.489583in}{0.877907in}}%
\pgfusepath{clip}%
\pgfsetbuttcap%
\pgfsetroundjoin%
\pgfsetlinewidth{1.505625pt}%
\definecolor{currentstroke}{rgb}{0.000000,0.000000,0.000000}%
\pgfsetstrokecolor{currentstroke}%
\pgfsetdash{}{0pt}%
\pgfpathmoveto{\pgfqpoint{3.606756in}{12.257086in}}%
\pgfpathlineto{\pgfqpoint{3.606756in}{12.799768in}}%
\pgfusepath{stroke}%
\end{pgfscope}%
\begin{pgfscope}%
\pgfpathrectangle{\pgfqpoint{2.125000in}{11.981395in}}{\pgfqpoint{5.489583in}{0.877907in}}%
\pgfusepath{clip}%
\pgfsetbuttcap%
\pgfsetroundjoin%
\pgfsetlinewidth{1.505625pt}%
\definecolor{currentstroke}{rgb}{0.000000,0.000000,0.000000}%
\pgfsetstrokecolor{currentstroke}%
\pgfsetdash{}{0pt}%
\pgfpathmoveto{\pgfqpoint{3.729979in}{12.257086in}}%
\pgfpathlineto{\pgfqpoint{3.729979in}{12.797987in}}%
\pgfusepath{stroke}%
\end{pgfscope}%
\begin{pgfscope}%
\pgfpathrectangle{\pgfqpoint{2.125000in}{11.981395in}}{\pgfqpoint{5.489583in}{0.877907in}}%
\pgfusepath{clip}%
\pgfsetbuttcap%
\pgfsetroundjoin%
\pgfsetlinewidth{1.505625pt}%
\definecolor{currentstroke}{rgb}{0.000000,0.000000,0.000000}%
\pgfsetstrokecolor{currentstroke}%
\pgfsetdash{}{0pt}%
\pgfpathmoveto{\pgfqpoint{3.853202in}{12.257086in}}%
\pgfpathlineto{\pgfqpoint{3.853202in}{12.796160in}}%
\pgfusepath{stroke}%
\end{pgfscope}%
\begin{pgfscope}%
\pgfpathrectangle{\pgfqpoint{2.125000in}{11.981395in}}{\pgfqpoint{5.489583in}{0.877907in}}%
\pgfusepath{clip}%
\pgfsetbuttcap%
\pgfsetroundjoin%
\pgfsetlinewidth{1.505625pt}%
\definecolor{currentstroke}{rgb}{0.000000,0.000000,0.000000}%
\pgfsetstrokecolor{currentstroke}%
\pgfsetdash{}{0pt}%
\pgfpathmoveto{\pgfqpoint{3.976425in}{12.257086in}}%
\pgfpathlineto{\pgfqpoint{3.976425in}{12.794368in}}%
\pgfusepath{stroke}%
\end{pgfscope}%
\begin{pgfscope}%
\pgfpathrectangle{\pgfqpoint{2.125000in}{11.981395in}}{\pgfqpoint{5.489583in}{0.877907in}}%
\pgfusepath{clip}%
\pgfsetbuttcap%
\pgfsetroundjoin%
\pgfsetlinewidth{1.505625pt}%
\definecolor{currentstroke}{rgb}{0.000000,0.000000,0.000000}%
\pgfsetstrokecolor{currentstroke}%
\pgfsetdash{}{0pt}%
\pgfpathmoveto{\pgfqpoint{4.099648in}{12.257086in}}%
\pgfpathlineto{\pgfqpoint{4.099648in}{12.792655in}}%
\pgfusepath{stroke}%
\end{pgfscope}%
\begin{pgfscope}%
\pgfpathrectangle{\pgfqpoint{2.125000in}{11.981395in}}{\pgfqpoint{5.489583in}{0.877907in}}%
\pgfusepath{clip}%
\pgfsetbuttcap%
\pgfsetroundjoin%
\pgfsetlinewidth{1.505625pt}%
\definecolor{currentstroke}{rgb}{0.000000,0.000000,0.000000}%
\pgfsetstrokecolor{currentstroke}%
\pgfsetdash{}{0pt}%
\pgfpathmoveto{\pgfqpoint{4.222871in}{12.257086in}}%
\pgfpathlineto{\pgfqpoint{4.222871in}{12.790969in}}%
\pgfusepath{stroke}%
\end{pgfscope}%
\begin{pgfscope}%
\pgfpathrectangle{\pgfqpoint{2.125000in}{11.981395in}}{\pgfqpoint{5.489583in}{0.877907in}}%
\pgfusepath{clip}%
\pgfsetbuttcap%
\pgfsetroundjoin%
\pgfsetlinewidth{1.505625pt}%
\definecolor{currentstroke}{rgb}{0.000000,0.000000,0.000000}%
\pgfsetstrokecolor{currentstroke}%
\pgfsetdash{}{0pt}%
\pgfpathmoveto{\pgfqpoint{4.346094in}{12.257086in}}%
\pgfpathlineto{\pgfqpoint{4.346094in}{12.789432in}}%
\pgfusepath{stroke}%
\end{pgfscope}%
\begin{pgfscope}%
\pgfpathrectangle{\pgfqpoint{2.125000in}{11.981395in}}{\pgfqpoint{5.489583in}{0.877907in}}%
\pgfusepath{clip}%
\pgfsetbuttcap%
\pgfsetroundjoin%
\pgfsetlinewidth{1.505625pt}%
\definecolor{currentstroke}{rgb}{0.000000,0.000000,0.000000}%
\pgfsetstrokecolor{currentstroke}%
\pgfsetdash{}{0pt}%
\pgfpathmoveto{\pgfqpoint{4.469317in}{12.257086in}}%
\pgfpathlineto{\pgfqpoint{4.469317in}{12.787934in}}%
\pgfusepath{stroke}%
\end{pgfscope}%
\begin{pgfscope}%
\pgfpathrectangle{\pgfqpoint{2.125000in}{11.981395in}}{\pgfqpoint{5.489583in}{0.877907in}}%
\pgfusepath{clip}%
\pgfsetbuttcap%
\pgfsetroundjoin%
\pgfsetlinewidth{1.505625pt}%
\definecolor{currentstroke}{rgb}{0.000000,0.000000,0.000000}%
\pgfsetstrokecolor{currentstroke}%
\pgfsetdash{}{0pt}%
\pgfpathmoveto{\pgfqpoint{4.592540in}{12.257086in}}%
\pgfpathlineto{\pgfqpoint{4.592540in}{12.786344in}}%
\pgfusepath{stroke}%
\end{pgfscope}%
\begin{pgfscope}%
\pgfpathrectangle{\pgfqpoint{2.125000in}{11.981395in}}{\pgfqpoint{5.489583in}{0.877907in}}%
\pgfusepath{clip}%
\pgfsetbuttcap%
\pgfsetroundjoin%
\pgfsetlinewidth{1.505625pt}%
\definecolor{currentstroke}{rgb}{0.000000,0.000000,0.000000}%
\pgfsetstrokecolor{currentstroke}%
\pgfsetdash{}{0pt}%
\pgfpathmoveto{\pgfqpoint{4.715763in}{12.257086in}}%
\pgfpathlineto{\pgfqpoint{4.715763in}{12.784548in}}%
\pgfusepath{stroke}%
\end{pgfscope}%
\begin{pgfscope}%
\pgfpathrectangle{\pgfqpoint{2.125000in}{11.981395in}}{\pgfqpoint{5.489583in}{0.877907in}}%
\pgfusepath{clip}%
\pgfsetbuttcap%
\pgfsetroundjoin%
\pgfsetlinewidth{1.505625pt}%
\definecolor{currentstroke}{rgb}{0.000000,0.000000,0.000000}%
\pgfsetstrokecolor{currentstroke}%
\pgfsetdash{}{0pt}%
\pgfpathmoveto{\pgfqpoint{4.838986in}{12.257086in}}%
\pgfpathlineto{\pgfqpoint{4.838986in}{12.782705in}}%
\pgfusepath{stroke}%
\end{pgfscope}%
\begin{pgfscope}%
\pgfpathrectangle{\pgfqpoint{2.125000in}{11.981395in}}{\pgfqpoint{5.489583in}{0.877907in}}%
\pgfusepath{clip}%
\pgfsetbuttcap%
\pgfsetroundjoin%
\pgfsetlinewidth{1.505625pt}%
\definecolor{currentstroke}{rgb}{0.000000,0.000000,0.000000}%
\pgfsetstrokecolor{currentstroke}%
\pgfsetdash{}{0pt}%
\pgfpathmoveto{\pgfqpoint{4.962209in}{12.257086in}}%
\pgfpathlineto{\pgfqpoint{4.962209in}{12.780851in}}%
\pgfusepath{stroke}%
\end{pgfscope}%
\begin{pgfscope}%
\pgfpathrectangle{\pgfqpoint{2.125000in}{11.981395in}}{\pgfqpoint{5.489583in}{0.877907in}}%
\pgfusepath{clip}%
\pgfsetbuttcap%
\pgfsetroundjoin%
\pgfsetlinewidth{1.505625pt}%
\definecolor{currentstroke}{rgb}{0.000000,0.000000,0.000000}%
\pgfsetstrokecolor{currentstroke}%
\pgfsetdash{}{0pt}%
\pgfpathmoveto{\pgfqpoint{5.085432in}{12.257086in}}%
\pgfpathlineto{\pgfqpoint{5.085432in}{12.779013in}}%
\pgfusepath{stroke}%
\end{pgfscope}%
\begin{pgfscope}%
\pgfpathrectangle{\pgfqpoint{2.125000in}{11.981395in}}{\pgfqpoint{5.489583in}{0.877907in}}%
\pgfusepath{clip}%
\pgfsetbuttcap%
\pgfsetroundjoin%
\pgfsetlinewidth{1.505625pt}%
\definecolor{currentstroke}{rgb}{0.000000,0.000000,0.000000}%
\pgfsetstrokecolor{currentstroke}%
\pgfsetdash{}{0pt}%
\pgfpathmoveto{\pgfqpoint{5.208655in}{12.257086in}}%
\pgfpathlineto{\pgfqpoint{5.208655in}{12.777090in}}%
\pgfusepath{stroke}%
\end{pgfscope}%
\begin{pgfscope}%
\pgfpathrectangle{\pgfqpoint{2.125000in}{11.981395in}}{\pgfqpoint{5.489583in}{0.877907in}}%
\pgfusepath{clip}%
\pgfsetbuttcap%
\pgfsetroundjoin%
\pgfsetlinewidth{1.505625pt}%
\definecolor{currentstroke}{rgb}{0.000000,0.000000,0.000000}%
\pgfsetstrokecolor{currentstroke}%
\pgfsetdash{}{0pt}%
\pgfpathmoveto{\pgfqpoint{5.331878in}{12.257086in}}%
\pgfpathlineto{\pgfqpoint{5.331878in}{12.775157in}}%
\pgfusepath{stroke}%
\end{pgfscope}%
\begin{pgfscope}%
\pgfpathrectangle{\pgfqpoint{2.125000in}{11.981395in}}{\pgfqpoint{5.489583in}{0.877907in}}%
\pgfusepath{clip}%
\pgfsetbuttcap%
\pgfsetroundjoin%
\pgfsetlinewidth{1.505625pt}%
\definecolor{currentstroke}{rgb}{0.000000,0.000000,0.000000}%
\pgfsetstrokecolor{currentstroke}%
\pgfsetdash{}{0pt}%
\pgfpathmoveto{\pgfqpoint{5.455101in}{12.257086in}}%
\pgfpathlineto{\pgfqpoint{5.455101in}{12.773228in}}%
\pgfusepath{stroke}%
\end{pgfscope}%
\begin{pgfscope}%
\pgfpathrectangle{\pgfqpoint{2.125000in}{11.981395in}}{\pgfqpoint{5.489583in}{0.877907in}}%
\pgfusepath{clip}%
\pgfsetbuttcap%
\pgfsetroundjoin%
\pgfsetlinewidth{1.505625pt}%
\definecolor{currentstroke}{rgb}{0.000000,0.000000,0.000000}%
\pgfsetstrokecolor{currentstroke}%
\pgfsetdash{}{0pt}%
\pgfpathmoveto{\pgfqpoint{5.578324in}{12.257086in}}%
\pgfpathlineto{\pgfqpoint{5.578324in}{12.771310in}}%
\pgfusepath{stroke}%
\end{pgfscope}%
\begin{pgfscope}%
\pgfpathrectangle{\pgfqpoint{2.125000in}{11.981395in}}{\pgfqpoint{5.489583in}{0.877907in}}%
\pgfusepath{clip}%
\pgfsetbuttcap%
\pgfsetroundjoin%
\pgfsetlinewidth{1.505625pt}%
\definecolor{currentstroke}{rgb}{0.000000,0.000000,0.000000}%
\pgfsetstrokecolor{currentstroke}%
\pgfsetdash{}{0pt}%
\pgfpathmoveto{\pgfqpoint{5.701547in}{12.257086in}}%
\pgfpathlineto{\pgfqpoint{5.701547in}{12.769439in}}%
\pgfusepath{stroke}%
\end{pgfscope}%
\begin{pgfscope}%
\pgfpathrectangle{\pgfqpoint{2.125000in}{11.981395in}}{\pgfqpoint{5.489583in}{0.877907in}}%
\pgfusepath{clip}%
\pgfsetbuttcap%
\pgfsetroundjoin%
\pgfsetlinewidth{1.505625pt}%
\definecolor{currentstroke}{rgb}{0.000000,0.000000,0.000000}%
\pgfsetstrokecolor{currentstroke}%
\pgfsetdash{}{0pt}%
\pgfpathmoveto{\pgfqpoint{5.824770in}{12.257086in}}%
\pgfpathlineto{\pgfqpoint{5.824770in}{12.767589in}}%
\pgfusepath{stroke}%
\end{pgfscope}%
\begin{pgfscope}%
\pgfpathrectangle{\pgfqpoint{2.125000in}{11.981395in}}{\pgfqpoint{5.489583in}{0.877907in}}%
\pgfusepath{clip}%
\pgfsetbuttcap%
\pgfsetroundjoin%
\pgfsetlinewidth{1.505625pt}%
\definecolor{currentstroke}{rgb}{0.000000,0.000000,0.000000}%
\pgfsetstrokecolor{currentstroke}%
\pgfsetdash{}{0pt}%
\pgfpathmoveto{\pgfqpoint{5.947993in}{12.257086in}}%
\pgfpathlineto{\pgfqpoint{5.947993in}{12.765726in}}%
\pgfusepath{stroke}%
\end{pgfscope}%
\begin{pgfscope}%
\pgfpathrectangle{\pgfqpoint{2.125000in}{11.981395in}}{\pgfqpoint{5.489583in}{0.877907in}}%
\pgfusepath{clip}%
\pgfsetbuttcap%
\pgfsetroundjoin%
\pgfsetlinewidth{1.505625pt}%
\definecolor{currentstroke}{rgb}{0.000000,0.000000,0.000000}%
\pgfsetstrokecolor{currentstroke}%
\pgfsetdash{}{0pt}%
\pgfpathmoveto{\pgfqpoint{6.071216in}{12.257086in}}%
\pgfpathlineto{\pgfqpoint{6.071216in}{12.763788in}}%
\pgfusepath{stroke}%
\end{pgfscope}%
\begin{pgfscope}%
\pgfpathrectangle{\pgfqpoint{2.125000in}{11.981395in}}{\pgfqpoint{5.489583in}{0.877907in}}%
\pgfusepath{clip}%
\pgfsetbuttcap%
\pgfsetroundjoin%
\pgfsetlinewidth{1.505625pt}%
\definecolor{currentstroke}{rgb}{0.000000,0.000000,0.000000}%
\pgfsetstrokecolor{currentstroke}%
\pgfsetdash{}{0pt}%
\pgfpathmoveto{\pgfqpoint{6.194439in}{12.257086in}}%
\pgfpathlineto{\pgfqpoint{6.194439in}{12.761813in}}%
\pgfusepath{stroke}%
\end{pgfscope}%
\begin{pgfscope}%
\pgfpathrectangle{\pgfqpoint{2.125000in}{11.981395in}}{\pgfqpoint{5.489583in}{0.877907in}}%
\pgfusepath{clip}%
\pgfsetbuttcap%
\pgfsetroundjoin%
\pgfsetlinewidth{1.505625pt}%
\definecolor{currentstroke}{rgb}{0.000000,0.000000,0.000000}%
\pgfsetstrokecolor{currentstroke}%
\pgfsetdash{}{0pt}%
\pgfpathmoveto{\pgfqpoint{6.317662in}{12.257086in}}%
\pgfpathlineto{\pgfqpoint{6.317662in}{12.759876in}}%
\pgfusepath{stroke}%
\end{pgfscope}%
\begin{pgfscope}%
\pgfpathrectangle{\pgfqpoint{2.125000in}{11.981395in}}{\pgfqpoint{5.489583in}{0.877907in}}%
\pgfusepath{clip}%
\pgfsetbuttcap%
\pgfsetroundjoin%
\pgfsetlinewidth{1.505625pt}%
\definecolor{currentstroke}{rgb}{0.000000,0.000000,0.000000}%
\pgfsetstrokecolor{currentstroke}%
\pgfsetdash{}{0pt}%
\pgfpathmoveto{\pgfqpoint{6.440885in}{12.257086in}}%
\pgfpathlineto{\pgfqpoint{6.440885in}{12.757979in}}%
\pgfusepath{stroke}%
\end{pgfscope}%
\begin{pgfscope}%
\pgfpathrectangle{\pgfqpoint{2.125000in}{11.981395in}}{\pgfqpoint{5.489583in}{0.877907in}}%
\pgfusepath{clip}%
\pgfsetbuttcap%
\pgfsetroundjoin%
\pgfsetlinewidth{1.505625pt}%
\definecolor{currentstroke}{rgb}{0.000000,0.000000,0.000000}%
\pgfsetstrokecolor{currentstroke}%
\pgfsetdash{}{0pt}%
\pgfpathmoveto{\pgfqpoint{6.564108in}{12.257086in}}%
\pgfpathlineto{\pgfqpoint{6.564108in}{12.756069in}}%
\pgfusepath{stroke}%
\end{pgfscope}%
\begin{pgfscope}%
\pgfpathrectangle{\pgfqpoint{2.125000in}{11.981395in}}{\pgfqpoint{5.489583in}{0.877907in}}%
\pgfusepath{clip}%
\pgfsetbuttcap%
\pgfsetroundjoin%
\pgfsetlinewidth{1.505625pt}%
\definecolor{currentstroke}{rgb}{0.000000,0.000000,0.000000}%
\pgfsetstrokecolor{currentstroke}%
\pgfsetdash{}{0pt}%
\pgfpathmoveto{\pgfqpoint{6.687330in}{12.257086in}}%
\pgfpathlineto{\pgfqpoint{6.687330in}{12.754041in}}%
\pgfusepath{stroke}%
\end{pgfscope}%
\begin{pgfscope}%
\pgfpathrectangle{\pgfqpoint{2.125000in}{11.981395in}}{\pgfqpoint{5.489583in}{0.877907in}}%
\pgfusepath{clip}%
\pgfsetbuttcap%
\pgfsetroundjoin%
\pgfsetlinewidth{1.505625pt}%
\definecolor{currentstroke}{rgb}{0.000000,0.000000,0.000000}%
\pgfsetstrokecolor{currentstroke}%
\pgfsetdash{}{0pt}%
\pgfpathmoveto{\pgfqpoint{6.810553in}{12.257086in}}%
\pgfpathlineto{\pgfqpoint{6.810553in}{12.751921in}}%
\pgfusepath{stroke}%
\end{pgfscope}%
\begin{pgfscope}%
\pgfpathrectangle{\pgfqpoint{2.125000in}{11.981395in}}{\pgfqpoint{5.489583in}{0.877907in}}%
\pgfusepath{clip}%
\pgfsetbuttcap%
\pgfsetroundjoin%
\pgfsetlinewidth{1.505625pt}%
\definecolor{currentstroke}{rgb}{0.000000,0.000000,0.000000}%
\pgfsetstrokecolor{currentstroke}%
\pgfsetdash{}{0pt}%
\pgfpathmoveto{\pgfqpoint{6.933776in}{12.257086in}}%
\pgfpathlineto{\pgfqpoint{6.933776in}{12.749796in}}%
\pgfusepath{stroke}%
\end{pgfscope}%
\begin{pgfscope}%
\pgfpathrectangle{\pgfqpoint{2.125000in}{11.981395in}}{\pgfqpoint{5.489583in}{0.877907in}}%
\pgfusepath{clip}%
\pgfsetbuttcap%
\pgfsetroundjoin%
\pgfsetlinewidth{1.505625pt}%
\definecolor{currentstroke}{rgb}{0.000000,0.000000,0.000000}%
\pgfsetstrokecolor{currentstroke}%
\pgfsetdash{}{0pt}%
\pgfpathmoveto{\pgfqpoint{7.056999in}{12.257086in}}%
\pgfpathlineto{\pgfqpoint{7.056999in}{12.747644in}}%
\pgfusepath{stroke}%
\end{pgfscope}%
\begin{pgfscope}%
\pgfpathrectangle{\pgfqpoint{2.125000in}{11.981395in}}{\pgfqpoint{5.489583in}{0.877907in}}%
\pgfusepath{clip}%
\pgfsetbuttcap%
\pgfsetroundjoin%
\pgfsetlinewidth{1.505625pt}%
\definecolor{currentstroke}{rgb}{0.000000,0.000000,0.000000}%
\pgfsetstrokecolor{currentstroke}%
\pgfsetdash{}{0pt}%
\pgfpathmoveto{\pgfqpoint{7.180222in}{12.257086in}}%
\pgfpathlineto{\pgfqpoint{7.180222in}{12.745580in}}%
\pgfusepath{stroke}%
\end{pgfscope}%
\begin{pgfscope}%
\pgfpathrectangle{\pgfqpoint{2.125000in}{11.981395in}}{\pgfqpoint{5.489583in}{0.877907in}}%
\pgfusepath{clip}%
\pgfsetbuttcap%
\pgfsetroundjoin%
\pgfsetlinewidth{1.505625pt}%
\definecolor{currentstroke}{rgb}{0.000000,0.000000,0.000000}%
\pgfsetstrokecolor{currentstroke}%
\pgfsetdash{}{0pt}%
\pgfpathmoveto{\pgfqpoint{7.303445in}{12.257086in}}%
\pgfpathlineto{\pgfqpoint{7.303445in}{12.743478in}}%
\pgfusepath{stroke}%
\end{pgfscope}%
\begin{pgfscope}%
\pgfpathrectangle{\pgfqpoint{2.125000in}{11.981395in}}{\pgfqpoint{5.489583in}{0.877907in}}%
\pgfusepath{clip}%
\pgfsetroundcap%
\pgfsetroundjoin%
\pgfsetlinewidth{1.505625pt}%
\definecolor{currentstroke}{rgb}{0.121569,0.466667,0.705882}%
\pgfsetstrokecolor{currentstroke}%
\pgfsetdash{}{0pt}%
\pgfpathmoveto{\pgfqpoint{2.125000in}{12.257086in}}%
\pgfpathlineto{\pgfqpoint{7.614583in}{12.257086in}}%
\pgfusepath{stroke}%
\end{pgfscope}%
\begin{pgfscope}%
\pgfpathrectangle{\pgfqpoint{2.125000in}{11.981395in}}{\pgfqpoint{5.489583in}{0.877907in}}%
\pgfusepath{clip}%
\pgfsetbuttcap%
\pgfsetroundjoin%
\definecolor{currentfill}{rgb}{0.121569,0.466667,0.705882}%
\pgfsetfillcolor{currentfill}%
\pgfsetlinewidth{1.003750pt}%
\definecolor{currentstroke}{rgb}{0.121569,0.466667,0.705882}%
\pgfsetstrokecolor{currentstroke}%
\pgfsetdash{}{0pt}%
\pgfsys@defobject{currentmarker}{\pgfqpoint{-0.034722in}{-0.034722in}}{\pgfqpoint{0.034722in}{0.034722in}}{%
\pgfpathmoveto{\pgfqpoint{0.000000in}{-0.034722in}}%
\pgfpathcurveto{\pgfqpoint{0.009208in}{-0.034722in}}{\pgfqpoint{0.018041in}{-0.031064in}}{\pgfqpoint{0.024552in}{-0.024552in}}%
\pgfpathcurveto{\pgfqpoint{0.031064in}{-0.018041in}}{\pgfqpoint{0.034722in}{-0.009208in}}{\pgfqpoint{0.034722in}{0.000000in}}%
\pgfpathcurveto{\pgfqpoint{0.034722in}{0.009208in}}{\pgfqpoint{0.031064in}{0.018041in}}{\pgfqpoint{0.024552in}{0.024552in}}%
\pgfpathcurveto{\pgfqpoint{0.018041in}{0.031064in}}{\pgfqpoint{0.009208in}{0.034722in}}{\pgfqpoint{0.000000in}{0.034722in}}%
\pgfpathcurveto{\pgfqpoint{-0.009208in}{0.034722in}}{\pgfqpoint{-0.018041in}{0.031064in}}{\pgfqpoint{-0.024552in}{0.024552in}}%
\pgfpathcurveto{\pgfqpoint{-0.031064in}{0.018041in}}{\pgfqpoint{-0.034722in}{0.009208in}}{\pgfqpoint{-0.034722in}{0.000000in}}%
\pgfpathcurveto{\pgfqpoint{-0.034722in}{-0.009208in}}{\pgfqpoint{-0.031064in}{-0.018041in}}{\pgfqpoint{-0.024552in}{-0.024552in}}%
\pgfpathcurveto{\pgfqpoint{-0.018041in}{-0.031064in}}{\pgfqpoint{-0.009208in}{-0.034722in}}{\pgfqpoint{0.000000in}{-0.034722in}}%
\pgfpathclose%
\pgfusepath{stroke,fill}%
}%
\begin{pgfscope}%
\pgfsys@transformshift{2.374527in}{12.819397in}%
\pgfsys@useobject{currentmarker}{}%
\end{pgfscope}%
\begin{pgfscope}%
\pgfsys@transformshift{2.497749in}{12.817378in}%
\pgfsys@useobject{currentmarker}{}%
\end{pgfscope}%
\begin{pgfscope}%
\pgfsys@transformshift{2.620972in}{12.815418in}%
\pgfsys@useobject{currentmarker}{}%
\end{pgfscope}%
\begin{pgfscope}%
\pgfsys@transformshift{2.744195in}{12.813480in}%
\pgfsys@useobject{currentmarker}{}%
\end{pgfscope}%
\begin{pgfscope}%
\pgfsys@transformshift{2.867418in}{12.811532in}%
\pgfsys@useobject{currentmarker}{}%
\end{pgfscope}%
\begin{pgfscope}%
\pgfsys@transformshift{2.990641in}{12.809570in}%
\pgfsys@useobject{currentmarker}{}%
\end{pgfscope}%
\begin{pgfscope}%
\pgfsys@transformshift{3.113864in}{12.807608in}%
\pgfsys@useobject{currentmarker}{}%
\end{pgfscope}%
\begin{pgfscope}%
\pgfsys@transformshift{3.237087in}{12.805615in}%
\pgfsys@useobject{currentmarker}{}%
\end{pgfscope}%
\begin{pgfscope}%
\pgfsys@transformshift{3.360310in}{12.803613in}%
\pgfsys@useobject{currentmarker}{}%
\end{pgfscope}%
\begin{pgfscope}%
\pgfsys@transformshift{3.483533in}{12.801601in}%
\pgfsys@useobject{currentmarker}{}%
\end{pgfscope}%
\begin{pgfscope}%
\pgfsys@transformshift{3.606756in}{12.799768in}%
\pgfsys@useobject{currentmarker}{}%
\end{pgfscope}%
\begin{pgfscope}%
\pgfsys@transformshift{3.729979in}{12.797987in}%
\pgfsys@useobject{currentmarker}{}%
\end{pgfscope}%
\begin{pgfscope}%
\pgfsys@transformshift{3.853202in}{12.796160in}%
\pgfsys@useobject{currentmarker}{}%
\end{pgfscope}%
\begin{pgfscope}%
\pgfsys@transformshift{3.976425in}{12.794368in}%
\pgfsys@useobject{currentmarker}{}%
\end{pgfscope}%
\begin{pgfscope}%
\pgfsys@transformshift{4.099648in}{12.792655in}%
\pgfsys@useobject{currentmarker}{}%
\end{pgfscope}%
\begin{pgfscope}%
\pgfsys@transformshift{4.222871in}{12.790969in}%
\pgfsys@useobject{currentmarker}{}%
\end{pgfscope}%
\begin{pgfscope}%
\pgfsys@transformshift{4.346094in}{12.789432in}%
\pgfsys@useobject{currentmarker}{}%
\end{pgfscope}%
\begin{pgfscope}%
\pgfsys@transformshift{4.469317in}{12.787934in}%
\pgfsys@useobject{currentmarker}{}%
\end{pgfscope}%
\begin{pgfscope}%
\pgfsys@transformshift{4.592540in}{12.786344in}%
\pgfsys@useobject{currentmarker}{}%
\end{pgfscope}%
\begin{pgfscope}%
\pgfsys@transformshift{4.715763in}{12.784548in}%
\pgfsys@useobject{currentmarker}{}%
\end{pgfscope}%
\begin{pgfscope}%
\pgfsys@transformshift{4.838986in}{12.782705in}%
\pgfsys@useobject{currentmarker}{}%
\end{pgfscope}%
\begin{pgfscope}%
\pgfsys@transformshift{4.962209in}{12.780851in}%
\pgfsys@useobject{currentmarker}{}%
\end{pgfscope}%
\begin{pgfscope}%
\pgfsys@transformshift{5.085432in}{12.779013in}%
\pgfsys@useobject{currentmarker}{}%
\end{pgfscope}%
\begin{pgfscope}%
\pgfsys@transformshift{5.208655in}{12.777090in}%
\pgfsys@useobject{currentmarker}{}%
\end{pgfscope}%
\begin{pgfscope}%
\pgfsys@transformshift{5.331878in}{12.775157in}%
\pgfsys@useobject{currentmarker}{}%
\end{pgfscope}%
\begin{pgfscope}%
\pgfsys@transformshift{5.455101in}{12.773228in}%
\pgfsys@useobject{currentmarker}{}%
\end{pgfscope}%
\begin{pgfscope}%
\pgfsys@transformshift{5.578324in}{12.771310in}%
\pgfsys@useobject{currentmarker}{}%
\end{pgfscope}%
\begin{pgfscope}%
\pgfsys@transformshift{5.701547in}{12.769439in}%
\pgfsys@useobject{currentmarker}{}%
\end{pgfscope}%
\begin{pgfscope}%
\pgfsys@transformshift{5.824770in}{12.767589in}%
\pgfsys@useobject{currentmarker}{}%
\end{pgfscope}%
\begin{pgfscope}%
\pgfsys@transformshift{5.947993in}{12.765726in}%
\pgfsys@useobject{currentmarker}{}%
\end{pgfscope}%
\begin{pgfscope}%
\pgfsys@transformshift{6.071216in}{12.763788in}%
\pgfsys@useobject{currentmarker}{}%
\end{pgfscope}%
\begin{pgfscope}%
\pgfsys@transformshift{6.194439in}{12.761813in}%
\pgfsys@useobject{currentmarker}{}%
\end{pgfscope}%
\begin{pgfscope}%
\pgfsys@transformshift{6.317662in}{12.759876in}%
\pgfsys@useobject{currentmarker}{}%
\end{pgfscope}%
\begin{pgfscope}%
\pgfsys@transformshift{6.440885in}{12.757979in}%
\pgfsys@useobject{currentmarker}{}%
\end{pgfscope}%
\begin{pgfscope}%
\pgfsys@transformshift{6.564108in}{12.756069in}%
\pgfsys@useobject{currentmarker}{}%
\end{pgfscope}%
\begin{pgfscope}%
\pgfsys@transformshift{6.687330in}{12.754041in}%
\pgfsys@useobject{currentmarker}{}%
\end{pgfscope}%
\begin{pgfscope}%
\pgfsys@transformshift{6.810553in}{12.751921in}%
\pgfsys@useobject{currentmarker}{}%
\end{pgfscope}%
\begin{pgfscope}%
\pgfsys@transformshift{6.933776in}{12.749796in}%
\pgfsys@useobject{currentmarker}{}%
\end{pgfscope}%
\begin{pgfscope}%
\pgfsys@transformshift{7.056999in}{12.747644in}%
\pgfsys@useobject{currentmarker}{}%
\end{pgfscope}%
\begin{pgfscope}%
\pgfsys@transformshift{7.180222in}{12.745580in}%
\pgfsys@useobject{currentmarker}{}%
\end{pgfscope}%
\begin{pgfscope}%
\pgfsys@transformshift{7.303445in}{12.743478in}%
\pgfsys@useobject{currentmarker}{}%
\end{pgfscope}%
\end{pgfscope}%
\begin{pgfscope}%
\pgfsetrectcap%
\pgfsetmiterjoin%
\pgfsetlinewidth{0.803000pt}%
\definecolor{currentstroke}{rgb}{1.000000,1.000000,1.000000}%
\pgfsetstrokecolor{currentstroke}%
\pgfsetdash{}{0pt}%
\pgfpathmoveto{\pgfqpoint{2.125000in}{11.981395in}}%
\pgfpathlineto{\pgfqpoint{2.125000in}{12.859302in}}%
\pgfusepath{stroke}%
\end{pgfscope}%
\begin{pgfscope}%
\pgfsetrectcap%
\pgfsetmiterjoin%
\pgfsetlinewidth{0.803000pt}%
\definecolor{currentstroke}{rgb}{1.000000,1.000000,1.000000}%
\pgfsetstrokecolor{currentstroke}%
\pgfsetdash{}{0pt}%
\pgfpathmoveto{\pgfqpoint{7.614583in}{11.981395in}}%
\pgfpathlineto{\pgfqpoint{7.614583in}{12.859302in}}%
\pgfusepath{stroke}%
\end{pgfscope}%
\begin{pgfscope}%
\pgfsetrectcap%
\pgfsetmiterjoin%
\pgfsetlinewidth{0.803000pt}%
\definecolor{currentstroke}{rgb}{1.000000,1.000000,1.000000}%
\pgfsetstrokecolor{currentstroke}%
\pgfsetdash{}{0pt}%
\pgfpathmoveto{\pgfqpoint{2.125000in}{11.981395in}}%
\pgfpathlineto{\pgfqpoint{7.614583in}{11.981395in}}%
\pgfusepath{stroke}%
\end{pgfscope}%
\begin{pgfscope}%
\pgfsetrectcap%
\pgfsetmiterjoin%
\pgfsetlinewidth{0.803000pt}%
\definecolor{currentstroke}{rgb}{1.000000,1.000000,1.000000}%
\pgfsetstrokecolor{currentstroke}%
\pgfsetdash{}{0pt}%
\pgfpathmoveto{\pgfqpoint{2.125000in}{12.859302in}}%
\pgfpathlineto{\pgfqpoint{7.614583in}{12.859302in}}%
\pgfusepath{stroke}%
\end{pgfscope}%
\begin{pgfscope}%
\definecolor{textcolor}{rgb}{0.150000,0.150000,0.150000}%
\pgfsetstrokecolor{textcolor}%
\pgfsetfillcolor{textcolor}%
\pgftext[x=4.869792in,y=12.942636in,,base]{\color{textcolor}\rmfamily\fontsize{16.800000}{20.160000}\selectfont Autocorrelation}%
\end{pgfscope}%
\begin{pgfscope}%
\pgfsetbuttcap%
\pgfsetmiterjoin%
\definecolor{currentfill}{rgb}{0.917647,0.917647,0.949020}%
\pgfsetfillcolor{currentfill}%
\pgfsetlinewidth{0.000000pt}%
\definecolor{currentstroke}{rgb}{0.000000,0.000000,0.000000}%
\pgfsetstrokecolor{currentstroke}%
\pgfsetstrokeopacity{0.000000}%
\pgfsetdash{}{0pt}%
\pgfpathmoveto{\pgfqpoint{9.810417in}{11.981395in}}%
\pgfpathlineto{\pgfqpoint{15.300000in}{11.981395in}}%
\pgfpathlineto{\pgfqpoint{15.300000in}{12.859302in}}%
\pgfpathlineto{\pgfqpoint{9.810417in}{12.859302in}}%
\pgfpathclose%
\pgfusepath{fill}%
\end{pgfscope}%
\begin{pgfscope}%
\pgfpathrectangle{\pgfqpoint{9.810417in}{11.981395in}}{\pgfqpoint{5.489583in}{0.877907in}}%
\pgfusepath{clip}%
\pgfsetroundcap%
\pgfsetroundjoin%
\pgfsetlinewidth{0.803000pt}%
\definecolor{currentstroke}{rgb}{1.000000,1.000000,1.000000}%
\pgfsetstrokecolor{currentstroke}%
\pgfsetdash{}{0pt}%
\pgfpathmoveto{\pgfqpoint{10.059943in}{11.981395in}}%
\pgfpathlineto{\pgfqpoint{10.059943in}{12.859302in}}%
\pgfusepath{stroke}%
\end{pgfscope}%
\begin{pgfscope}%
\definecolor{textcolor}{rgb}{0.150000,0.150000,0.150000}%
\pgfsetstrokecolor{textcolor}%
\pgfsetfillcolor{textcolor}%
\pgftext[x=10.059943in,y=11.884173in,,top]{\color{textcolor}\rmfamily\fontsize{14.000000}{16.800000}\selectfont 0}%
\end{pgfscope}%
\begin{pgfscope}%
\pgfpathrectangle{\pgfqpoint{9.810417in}{11.981395in}}{\pgfqpoint{5.489583in}{0.877907in}}%
\pgfusepath{clip}%
\pgfsetroundcap%
\pgfsetroundjoin%
\pgfsetlinewidth{0.803000pt}%
\definecolor{currentstroke}{rgb}{1.000000,1.000000,1.000000}%
\pgfsetstrokecolor{currentstroke}%
\pgfsetdash{}{0pt}%
\pgfpathmoveto{\pgfqpoint{10.676058in}{11.981395in}}%
\pgfpathlineto{\pgfqpoint{10.676058in}{12.859302in}}%
\pgfusepath{stroke}%
\end{pgfscope}%
\begin{pgfscope}%
\definecolor{textcolor}{rgb}{0.150000,0.150000,0.150000}%
\pgfsetstrokecolor{textcolor}%
\pgfsetfillcolor{textcolor}%
\pgftext[x=10.676058in,y=11.884173in,,top]{\color{textcolor}\rmfamily\fontsize{14.000000}{16.800000}\selectfont 5}%
\end{pgfscope}%
\begin{pgfscope}%
\pgfpathrectangle{\pgfqpoint{9.810417in}{11.981395in}}{\pgfqpoint{5.489583in}{0.877907in}}%
\pgfusepath{clip}%
\pgfsetroundcap%
\pgfsetroundjoin%
\pgfsetlinewidth{0.803000pt}%
\definecolor{currentstroke}{rgb}{1.000000,1.000000,1.000000}%
\pgfsetstrokecolor{currentstroke}%
\pgfsetdash{}{0pt}%
\pgfpathmoveto{\pgfqpoint{11.292173in}{11.981395in}}%
\pgfpathlineto{\pgfqpoint{11.292173in}{12.859302in}}%
\pgfusepath{stroke}%
\end{pgfscope}%
\begin{pgfscope}%
\definecolor{textcolor}{rgb}{0.150000,0.150000,0.150000}%
\pgfsetstrokecolor{textcolor}%
\pgfsetfillcolor{textcolor}%
\pgftext[x=11.292173in,y=11.884173in,,top]{\color{textcolor}\rmfamily\fontsize{14.000000}{16.800000}\selectfont 10}%
\end{pgfscope}%
\begin{pgfscope}%
\pgfpathrectangle{\pgfqpoint{9.810417in}{11.981395in}}{\pgfqpoint{5.489583in}{0.877907in}}%
\pgfusepath{clip}%
\pgfsetroundcap%
\pgfsetroundjoin%
\pgfsetlinewidth{0.803000pt}%
\definecolor{currentstroke}{rgb}{1.000000,1.000000,1.000000}%
\pgfsetstrokecolor{currentstroke}%
\pgfsetdash{}{0pt}%
\pgfpathmoveto{\pgfqpoint{11.908288in}{11.981395in}}%
\pgfpathlineto{\pgfqpoint{11.908288in}{12.859302in}}%
\pgfusepath{stroke}%
\end{pgfscope}%
\begin{pgfscope}%
\definecolor{textcolor}{rgb}{0.150000,0.150000,0.150000}%
\pgfsetstrokecolor{textcolor}%
\pgfsetfillcolor{textcolor}%
\pgftext[x=11.908288in,y=11.884173in,,top]{\color{textcolor}\rmfamily\fontsize{14.000000}{16.800000}\selectfont 15}%
\end{pgfscope}%
\begin{pgfscope}%
\pgfpathrectangle{\pgfqpoint{9.810417in}{11.981395in}}{\pgfqpoint{5.489583in}{0.877907in}}%
\pgfusepath{clip}%
\pgfsetroundcap%
\pgfsetroundjoin%
\pgfsetlinewidth{0.803000pt}%
\definecolor{currentstroke}{rgb}{1.000000,1.000000,1.000000}%
\pgfsetstrokecolor{currentstroke}%
\pgfsetdash{}{0pt}%
\pgfpathmoveto{\pgfqpoint{12.524403in}{11.981395in}}%
\pgfpathlineto{\pgfqpoint{12.524403in}{12.859302in}}%
\pgfusepath{stroke}%
\end{pgfscope}%
\begin{pgfscope}%
\definecolor{textcolor}{rgb}{0.150000,0.150000,0.150000}%
\pgfsetstrokecolor{textcolor}%
\pgfsetfillcolor{textcolor}%
\pgftext[x=12.524403in,y=11.884173in,,top]{\color{textcolor}\rmfamily\fontsize{14.000000}{16.800000}\selectfont 20}%
\end{pgfscope}%
\begin{pgfscope}%
\pgfpathrectangle{\pgfqpoint{9.810417in}{11.981395in}}{\pgfqpoint{5.489583in}{0.877907in}}%
\pgfusepath{clip}%
\pgfsetroundcap%
\pgfsetroundjoin%
\pgfsetlinewidth{0.803000pt}%
\definecolor{currentstroke}{rgb}{1.000000,1.000000,1.000000}%
\pgfsetstrokecolor{currentstroke}%
\pgfsetdash{}{0pt}%
\pgfpathmoveto{\pgfqpoint{13.140517in}{11.981395in}}%
\pgfpathlineto{\pgfqpoint{13.140517in}{12.859302in}}%
\pgfusepath{stroke}%
\end{pgfscope}%
\begin{pgfscope}%
\definecolor{textcolor}{rgb}{0.150000,0.150000,0.150000}%
\pgfsetstrokecolor{textcolor}%
\pgfsetfillcolor{textcolor}%
\pgftext[x=13.140517in,y=11.884173in,,top]{\color{textcolor}\rmfamily\fontsize{14.000000}{16.800000}\selectfont 25}%
\end{pgfscope}%
\begin{pgfscope}%
\pgfpathrectangle{\pgfqpoint{9.810417in}{11.981395in}}{\pgfqpoint{5.489583in}{0.877907in}}%
\pgfusepath{clip}%
\pgfsetroundcap%
\pgfsetroundjoin%
\pgfsetlinewidth{0.803000pt}%
\definecolor{currentstroke}{rgb}{1.000000,1.000000,1.000000}%
\pgfsetstrokecolor{currentstroke}%
\pgfsetdash{}{0pt}%
\pgfpathmoveto{\pgfqpoint{13.756632in}{11.981395in}}%
\pgfpathlineto{\pgfqpoint{13.756632in}{12.859302in}}%
\pgfusepath{stroke}%
\end{pgfscope}%
\begin{pgfscope}%
\definecolor{textcolor}{rgb}{0.150000,0.150000,0.150000}%
\pgfsetstrokecolor{textcolor}%
\pgfsetfillcolor{textcolor}%
\pgftext[x=13.756632in,y=11.884173in,,top]{\color{textcolor}\rmfamily\fontsize{14.000000}{16.800000}\selectfont 30}%
\end{pgfscope}%
\begin{pgfscope}%
\pgfpathrectangle{\pgfqpoint{9.810417in}{11.981395in}}{\pgfqpoint{5.489583in}{0.877907in}}%
\pgfusepath{clip}%
\pgfsetroundcap%
\pgfsetroundjoin%
\pgfsetlinewidth{0.803000pt}%
\definecolor{currentstroke}{rgb}{1.000000,1.000000,1.000000}%
\pgfsetstrokecolor{currentstroke}%
\pgfsetdash{}{0pt}%
\pgfpathmoveto{\pgfqpoint{14.372747in}{11.981395in}}%
\pgfpathlineto{\pgfqpoint{14.372747in}{12.859302in}}%
\pgfusepath{stroke}%
\end{pgfscope}%
\begin{pgfscope}%
\definecolor{textcolor}{rgb}{0.150000,0.150000,0.150000}%
\pgfsetstrokecolor{textcolor}%
\pgfsetfillcolor{textcolor}%
\pgftext[x=14.372747in,y=11.884173in,,top]{\color{textcolor}\rmfamily\fontsize{14.000000}{16.800000}\selectfont 35}%
\end{pgfscope}%
\begin{pgfscope}%
\pgfpathrectangle{\pgfqpoint{9.810417in}{11.981395in}}{\pgfqpoint{5.489583in}{0.877907in}}%
\pgfusepath{clip}%
\pgfsetroundcap%
\pgfsetroundjoin%
\pgfsetlinewidth{0.803000pt}%
\definecolor{currentstroke}{rgb}{1.000000,1.000000,1.000000}%
\pgfsetstrokecolor{currentstroke}%
\pgfsetdash{}{0pt}%
\pgfpathmoveto{\pgfqpoint{14.988862in}{11.981395in}}%
\pgfpathlineto{\pgfqpoint{14.988862in}{12.859302in}}%
\pgfusepath{stroke}%
\end{pgfscope}%
\begin{pgfscope}%
\definecolor{textcolor}{rgb}{0.150000,0.150000,0.150000}%
\pgfsetstrokecolor{textcolor}%
\pgfsetfillcolor{textcolor}%
\pgftext[x=14.988862in,y=11.884173in,,top]{\color{textcolor}\rmfamily\fontsize{14.000000}{16.800000}\selectfont 40}%
\end{pgfscope}%
\begin{pgfscope}%
\pgfpathrectangle{\pgfqpoint{9.810417in}{11.981395in}}{\pgfqpoint{5.489583in}{0.877907in}}%
\pgfusepath{clip}%
\pgfsetroundcap%
\pgfsetroundjoin%
\pgfsetlinewidth{0.803000pt}%
\definecolor{currentstroke}{rgb}{1.000000,1.000000,1.000000}%
\pgfsetstrokecolor{currentstroke}%
\pgfsetdash{}{0pt}%
\pgfpathmoveto{\pgfqpoint{9.810417in}{12.069362in}}%
\pgfpathlineto{\pgfqpoint{15.300000in}{12.069362in}}%
\pgfusepath{stroke}%
\end{pgfscope}%
\begin{pgfscope}%
\definecolor{textcolor}{rgb}{0.150000,0.150000,0.150000}%
\pgfsetstrokecolor{textcolor}%
\pgfsetfillcolor{textcolor}%
\pgftext[x=9.589483in,y=11.995495in,left,base]{\color{textcolor}\rmfamily\fontsize{14.000000}{16.800000}\selectfont 0}%
\end{pgfscope}%
\begin{pgfscope}%
\pgfpathrectangle{\pgfqpoint{9.810417in}{11.981395in}}{\pgfqpoint{5.489583in}{0.877907in}}%
\pgfusepath{clip}%
\pgfsetroundcap%
\pgfsetroundjoin%
\pgfsetlinewidth{0.803000pt}%
\definecolor{currentstroke}{rgb}{1.000000,1.000000,1.000000}%
\pgfsetstrokecolor{currentstroke}%
\pgfsetdash{}{0pt}%
\pgfpathmoveto{\pgfqpoint{9.810417in}{12.819397in}}%
\pgfpathlineto{\pgfqpoint{15.300000in}{12.819397in}}%
\pgfusepath{stroke}%
\end{pgfscope}%
\begin{pgfscope}%
\definecolor{textcolor}{rgb}{0.150000,0.150000,0.150000}%
\pgfsetstrokecolor{textcolor}%
\pgfsetfillcolor{textcolor}%
\pgftext[x=9.589483in,y=12.745531in,left,base]{\color{textcolor}\rmfamily\fontsize{14.000000}{16.800000}\selectfont 1}%
\end{pgfscope}%
\begin{pgfscope}%
\pgfpathrectangle{\pgfqpoint{9.810417in}{11.981395in}}{\pgfqpoint{5.489583in}{0.877907in}}%
\pgfusepath{clip}%
\pgfsetbuttcap%
\pgfsetroundjoin%
\definecolor{currentfill}{rgb}{0.121569,0.466667,0.705882}%
\pgfsetfillcolor{currentfill}%
\pgfsetfillopacity{0.250000}%
\pgfsetlinewidth{1.003750pt}%
\definecolor{currentstroke}{rgb}{1.000000,1.000000,1.000000}%
\pgfsetstrokecolor{currentstroke}%
\pgfsetstrokeopacity{0.250000}%
\pgfsetdash{}{0pt}%
\pgfpathmoveto{\pgfqpoint{10.121555in}{12.107205in}}%
\pgfpathlineto{\pgfqpoint{10.121555in}{12.031519in}}%
\pgfpathlineto{\pgfqpoint{10.306389in}{12.031519in}}%
\pgfpathlineto{\pgfqpoint{10.429612in}{12.031519in}}%
\pgfpathlineto{\pgfqpoint{10.552835in}{12.031519in}}%
\pgfpathlineto{\pgfqpoint{10.676058in}{12.031519in}}%
\pgfpathlineto{\pgfqpoint{10.799281in}{12.031519in}}%
\pgfpathlineto{\pgfqpoint{10.922504in}{12.031519in}}%
\pgfpathlineto{\pgfqpoint{11.045727in}{12.031519in}}%
\pgfpathlineto{\pgfqpoint{11.168950in}{12.031519in}}%
\pgfpathlineto{\pgfqpoint{11.292173in}{12.031519in}}%
\pgfpathlineto{\pgfqpoint{11.415396in}{12.031519in}}%
\pgfpathlineto{\pgfqpoint{11.538619in}{12.031519in}}%
\pgfpathlineto{\pgfqpoint{11.661842in}{12.031519in}}%
\pgfpathlineto{\pgfqpoint{11.785065in}{12.031519in}}%
\pgfpathlineto{\pgfqpoint{11.908288in}{12.031519in}}%
\pgfpathlineto{\pgfqpoint{12.031511in}{12.031519in}}%
\pgfpathlineto{\pgfqpoint{12.154734in}{12.031519in}}%
\pgfpathlineto{\pgfqpoint{12.277957in}{12.031519in}}%
\pgfpathlineto{\pgfqpoint{12.401180in}{12.031519in}}%
\pgfpathlineto{\pgfqpoint{12.524403in}{12.031519in}}%
\pgfpathlineto{\pgfqpoint{12.647626in}{12.031519in}}%
\pgfpathlineto{\pgfqpoint{12.770849in}{12.031519in}}%
\pgfpathlineto{\pgfqpoint{12.894072in}{12.031519in}}%
\pgfpathlineto{\pgfqpoint{13.017294in}{12.031519in}}%
\pgfpathlineto{\pgfqpoint{13.140517in}{12.031519in}}%
\pgfpathlineto{\pgfqpoint{13.263740in}{12.031519in}}%
\pgfpathlineto{\pgfqpoint{13.386963in}{12.031519in}}%
\pgfpathlineto{\pgfqpoint{13.510186in}{12.031519in}}%
\pgfpathlineto{\pgfqpoint{13.633409in}{12.031519in}}%
\pgfpathlineto{\pgfqpoint{13.756632in}{12.031519in}}%
\pgfpathlineto{\pgfqpoint{13.879855in}{12.031519in}}%
\pgfpathlineto{\pgfqpoint{14.003078in}{12.031519in}}%
\pgfpathlineto{\pgfqpoint{14.126301in}{12.031519in}}%
\pgfpathlineto{\pgfqpoint{14.249524in}{12.031519in}}%
\pgfpathlineto{\pgfqpoint{14.372747in}{12.031519in}}%
\pgfpathlineto{\pgfqpoint{14.495970in}{12.031519in}}%
\pgfpathlineto{\pgfqpoint{14.619193in}{12.031519in}}%
\pgfpathlineto{\pgfqpoint{14.742416in}{12.031519in}}%
\pgfpathlineto{\pgfqpoint{14.865639in}{12.031519in}}%
\pgfpathlineto{\pgfqpoint{15.050473in}{12.031519in}}%
\pgfpathlineto{\pgfqpoint{15.050473in}{12.107205in}}%
\pgfpathlineto{\pgfqpoint{15.050473in}{12.107205in}}%
\pgfpathlineto{\pgfqpoint{14.865639in}{12.107205in}}%
\pgfpathlineto{\pgfqpoint{14.742416in}{12.107205in}}%
\pgfpathlineto{\pgfqpoint{14.619193in}{12.107205in}}%
\pgfpathlineto{\pgfqpoint{14.495970in}{12.107205in}}%
\pgfpathlineto{\pgfqpoint{14.372747in}{12.107205in}}%
\pgfpathlineto{\pgfqpoint{14.249524in}{12.107205in}}%
\pgfpathlineto{\pgfqpoint{14.126301in}{12.107205in}}%
\pgfpathlineto{\pgfqpoint{14.003078in}{12.107205in}}%
\pgfpathlineto{\pgfqpoint{13.879855in}{12.107205in}}%
\pgfpathlineto{\pgfqpoint{13.756632in}{12.107205in}}%
\pgfpathlineto{\pgfqpoint{13.633409in}{12.107205in}}%
\pgfpathlineto{\pgfqpoint{13.510186in}{12.107205in}}%
\pgfpathlineto{\pgfqpoint{13.386963in}{12.107205in}}%
\pgfpathlineto{\pgfqpoint{13.263740in}{12.107205in}}%
\pgfpathlineto{\pgfqpoint{13.140517in}{12.107205in}}%
\pgfpathlineto{\pgfqpoint{13.017294in}{12.107205in}}%
\pgfpathlineto{\pgfqpoint{12.894072in}{12.107205in}}%
\pgfpathlineto{\pgfqpoint{12.770849in}{12.107205in}}%
\pgfpathlineto{\pgfqpoint{12.647626in}{12.107205in}}%
\pgfpathlineto{\pgfqpoint{12.524403in}{12.107205in}}%
\pgfpathlineto{\pgfqpoint{12.401180in}{12.107205in}}%
\pgfpathlineto{\pgfqpoint{12.277957in}{12.107205in}}%
\pgfpathlineto{\pgfqpoint{12.154734in}{12.107205in}}%
\pgfpathlineto{\pgfqpoint{12.031511in}{12.107205in}}%
\pgfpathlineto{\pgfqpoint{11.908288in}{12.107205in}}%
\pgfpathlineto{\pgfqpoint{11.785065in}{12.107205in}}%
\pgfpathlineto{\pgfqpoint{11.661842in}{12.107205in}}%
\pgfpathlineto{\pgfqpoint{11.538619in}{12.107205in}}%
\pgfpathlineto{\pgfqpoint{11.415396in}{12.107205in}}%
\pgfpathlineto{\pgfqpoint{11.292173in}{12.107205in}}%
\pgfpathlineto{\pgfqpoint{11.168950in}{12.107205in}}%
\pgfpathlineto{\pgfqpoint{11.045727in}{12.107205in}}%
\pgfpathlineto{\pgfqpoint{10.922504in}{12.107205in}}%
\pgfpathlineto{\pgfqpoint{10.799281in}{12.107205in}}%
\pgfpathlineto{\pgfqpoint{10.676058in}{12.107205in}}%
\pgfpathlineto{\pgfqpoint{10.552835in}{12.107205in}}%
\pgfpathlineto{\pgfqpoint{10.429612in}{12.107205in}}%
\pgfpathlineto{\pgfqpoint{10.306389in}{12.107205in}}%
\pgfpathlineto{\pgfqpoint{10.121555in}{12.107205in}}%
\pgfpathclose%
\pgfusepath{stroke,fill}%
\end{pgfscope}%
\begin{pgfscope}%
\pgfpathrectangle{\pgfqpoint{9.810417in}{11.981395in}}{\pgfqpoint{5.489583in}{0.877907in}}%
\pgfusepath{clip}%
\pgfsetbuttcap%
\pgfsetroundjoin%
\pgfsetlinewidth{1.505625pt}%
\definecolor{currentstroke}{rgb}{0.000000,0.000000,0.000000}%
\pgfsetstrokecolor{currentstroke}%
\pgfsetdash{}{0pt}%
\pgfpathmoveto{\pgfqpoint{10.059943in}{12.069362in}}%
\pgfpathlineto{\pgfqpoint{10.059943in}{12.819397in}}%
\pgfusepath{stroke}%
\end{pgfscope}%
\begin{pgfscope}%
\pgfpathrectangle{\pgfqpoint{9.810417in}{11.981395in}}{\pgfqpoint{5.489583in}{0.877907in}}%
\pgfusepath{clip}%
\pgfsetbuttcap%
\pgfsetroundjoin%
\pgfsetlinewidth{1.505625pt}%
\definecolor{currentstroke}{rgb}{0.000000,0.000000,0.000000}%
\pgfsetstrokecolor{currentstroke}%
\pgfsetdash{}{0pt}%
\pgfpathmoveto{\pgfqpoint{10.183166in}{12.069362in}}%
\pgfpathlineto{\pgfqpoint{10.183166in}{12.817200in}}%
\pgfusepath{stroke}%
\end{pgfscope}%
\begin{pgfscope}%
\pgfpathrectangle{\pgfqpoint{9.810417in}{11.981395in}}{\pgfqpoint{5.489583in}{0.877907in}}%
\pgfusepath{clip}%
\pgfsetbuttcap%
\pgfsetroundjoin%
\pgfsetlinewidth{1.505625pt}%
\definecolor{currentstroke}{rgb}{0.000000,0.000000,0.000000}%
\pgfsetstrokecolor{currentstroke}%
\pgfsetdash{}{0pt}%
\pgfpathmoveto{\pgfqpoint{10.306389in}{12.069362in}}%
\pgfpathlineto{\pgfqpoint{10.306389in}{12.081146in}}%
\pgfusepath{stroke}%
\end{pgfscope}%
\begin{pgfscope}%
\pgfpathrectangle{\pgfqpoint{9.810417in}{11.981395in}}{\pgfqpoint{5.489583in}{0.877907in}}%
\pgfusepath{clip}%
\pgfsetbuttcap%
\pgfsetroundjoin%
\pgfsetlinewidth{1.505625pt}%
\definecolor{currentstroke}{rgb}{0.000000,0.000000,0.000000}%
\pgfsetstrokecolor{currentstroke}%
\pgfsetdash{}{0pt}%
\pgfpathmoveto{\pgfqpoint{10.429612in}{12.069362in}}%
\pgfpathlineto{\pgfqpoint{10.429612in}{12.073290in}}%
\pgfusepath{stroke}%
\end{pgfscope}%
\begin{pgfscope}%
\pgfpathrectangle{\pgfqpoint{9.810417in}{11.981395in}}{\pgfqpoint{5.489583in}{0.877907in}}%
\pgfusepath{clip}%
\pgfsetbuttcap%
\pgfsetroundjoin%
\pgfsetlinewidth{1.505625pt}%
\definecolor{currentstroke}{rgb}{0.000000,0.000000,0.000000}%
\pgfsetstrokecolor{currentstroke}%
\pgfsetdash{}{0pt}%
\pgfpathmoveto{\pgfqpoint{10.552835in}{12.069362in}}%
\pgfpathlineto{\pgfqpoint{10.552835in}{12.065606in}}%
\pgfusepath{stroke}%
\end{pgfscope}%
\begin{pgfscope}%
\pgfpathrectangle{\pgfqpoint{9.810417in}{11.981395in}}{\pgfqpoint{5.489583in}{0.877907in}}%
\pgfusepath{clip}%
\pgfsetbuttcap%
\pgfsetroundjoin%
\pgfsetlinewidth{1.505625pt}%
\definecolor{currentstroke}{rgb}{0.000000,0.000000,0.000000}%
\pgfsetstrokecolor{currentstroke}%
\pgfsetdash{}{0pt}%
\pgfpathmoveto{\pgfqpoint{10.676058in}{12.069362in}}%
\pgfpathlineto{\pgfqpoint{10.676058in}{12.064613in}}%
\pgfusepath{stroke}%
\end{pgfscope}%
\begin{pgfscope}%
\pgfpathrectangle{\pgfqpoint{9.810417in}{11.981395in}}{\pgfqpoint{5.489583in}{0.877907in}}%
\pgfusepath{clip}%
\pgfsetbuttcap%
\pgfsetroundjoin%
\pgfsetlinewidth{1.505625pt}%
\definecolor{currentstroke}{rgb}{0.000000,0.000000,0.000000}%
\pgfsetstrokecolor{currentstroke}%
\pgfsetdash{}{0pt}%
\pgfpathmoveto{\pgfqpoint{10.799281in}{12.069362in}}%
\pgfpathlineto{\pgfqpoint{10.799281in}{12.067475in}}%
\pgfusepath{stroke}%
\end{pgfscope}%
\begin{pgfscope}%
\pgfpathrectangle{\pgfqpoint{9.810417in}{11.981395in}}{\pgfqpoint{5.489583in}{0.877907in}}%
\pgfusepath{clip}%
\pgfsetbuttcap%
\pgfsetroundjoin%
\pgfsetlinewidth{1.505625pt}%
\definecolor{currentstroke}{rgb}{0.000000,0.000000,0.000000}%
\pgfsetstrokecolor{currentstroke}%
\pgfsetdash{}{0pt}%
\pgfpathmoveto{\pgfqpoint{10.922504in}{12.069362in}}%
\pgfpathlineto{\pgfqpoint{10.922504in}{12.060884in}}%
\pgfusepath{stroke}%
\end{pgfscope}%
\begin{pgfscope}%
\pgfpathrectangle{\pgfqpoint{9.810417in}{11.981395in}}{\pgfqpoint{5.489583in}{0.877907in}}%
\pgfusepath{clip}%
\pgfsetbuttcap%
\pgfsetroundjoin%
\pgfsetlinewidth{1.505625pt}%
\definecolor{currentstroke}{rgb}{0.000000,0.000000,0.000000}%
\pgfsetstrokecolor{currentstroke}%
\pgfsetdash{}{0pt}%
\pgfpathmoveto{\pgfqpoint{11.045727in}{12.069362in}}%
\pgfpathlineto{\pgfqpoint{11.045727in}{12.065368in}}%
\pgfusepath{stroke}%
\end{pgfscope}%
\begin{pgfscope}%
\pgfpathrectangle{\pgfqpoint{9.810417in}{11.981395in}}{\pgfqpoint{5.489583in}{0.877907in}}%
\pgfusepath{clip}%
\pgfsetbuttcap%
\pgfsetroundjoin%
\pgfsetlinewidth{1.505625pt}%
\definecolor{currentstroke}{rgb}{0.000000,0.000000,0.000000}%
\pgfsetstrokecolor{currentstroke}%
\pgfsetdash{}{0pt}%
\pgfpathmoveto{\pgfqpoint{11.168950in}{12.069362in}}%
\pgfpathlineto{\pgfqpoint{11.168950in}{12.065133in}}%
\pgfusepath{stroke}%
\end{pgfscope}%
\begin{pgfscope}%
\pgfpathrectangle{\pgfqpoint{9.810417in}{11.981395in}}{\pgfqpoint{5.489583in}{0.877907in}}%
\pgfusepath{clip}%
\pgfsetbuttcap%
\pgfsetroundjoin%
\pgfsetlinewidth{1.505625pt}%
\definecolor{currentstroke}{rgb}{0.000000,0.000000,0.000000}%
\pgfsetstrokecolor{currentstroke}%
\pgfsetdash{}{0pt}%
\pgfpathmoveto{\pgfqpoint{11.292173in}{12.069362in}}%
\pgfpathlineto{\pgfqpoint{11.292173in}{12.108848in}}%
\pgfusepath{stroke}%
\end{pgfscope}%
\begin{pgfscope}%
\pgfpathrectangle{\pgfqpoint{9.810417in}{11.981395in}}{\pgfqpoint{5.489583in}{0.877907in}}%
\pgfusepath{clip}%
\pgfsetbuttcap%
\pgfsetroundjoin%
\pgfsetlinewidth{1.505625pt}%
\definecolor{currentstroke}{rgb}{0.000000,0.000000,0.000000}%
\pgfsetstrokecolor{currentstroke}%
\pgfsetdash{}{0pt}%
\pgfpathmoveto{\pgfqpoint{11.415396in}{12.069362in}}%
\pgfpathlineto{\pgfqpoint{11.415396in}{12.081383in}}%
\pgfusepath{stroke}%
\end{pgfscope}%
\begin{pgfscope}%
\pgfpathrectangle{\pgfqpoint{9.810417in}{11.981395in}}{\pgfqpoint{5.489583in}{0.877907in}}%
\pgfusepath{clip}%
\pgfsetbuttcap%
\pgfsetroundjoin%
\pgfsetlinewidth{1.505625pt}%
\definecolor{currentstroke}{rgb}{0.000000,0.000000,0.000000}%
\pgfsetstrokecolor{currentstroke}%
\pgfsetdash{}{0pt}%
\pgfpathmoveto{\pgfqpoint{11.538619in}{12.069362in}}%
\pgfpathlineto{\pgfqpoint{11.538619in}{12.058489in}}%
\pgfusepath{stroke}%
\end{pgfscope}%
\begin{pgfscope}%
\pgfpathrectangle{\pgfqpoint{9.810417in}{11.981395in}}{\pgfqpoint{5.489583in}{0.877907in}}%
\pgfusepath{clip}%
\pgfsetbuttcap%
\pgfsetroundjoin%
\pgfsetlinewidth{1.505625pt}%
\definecolor{currentstroke}{rgb}{0.000000,0.000000,0.000000}%
\pgfsetstrokecolor{currentstroke}%
\pgfsetdash{}{0pt}%
\pgfpathmoveto{\pgfqpoint{11.661842in}{12.069362in}}%
\pgfpathlineto{\pgfqpoint{11.661842in}{12.075366in}}%
\pgfusepath{stroke}%
\end{pgfscope}%
\begin{pgfscope}%
\pgfpathrectangle{\pgfqpoint{9.810417in}{11.981395in}}{\pgfqpoint{5.489583in}{0.877907in}}%
\pgfusepath{clip}%
\pgfsetbuttcap%
\pgfsetroundjoin%
\pgfsetlinewidth{1.505625pt}%
\definecolor{currentstroke}{rgb}{0.000000,0.000000,0.000000}%
\pgfsetstrokecolor{currentstroke}%
\pgfsetdash{}{0pt}%
\pgfpathmoveto{\pgfqpoint{11.785065in}{12.069362in}}%
\pgfpathlineto{\pgfqpoint{11.785065in}{12.085929in}}%
\pgfusepath{stroke}%
\end{pgfscope}%
\begin{pgfscope}%
\pgfpathrectangle{\pgfqpoint{9.810417in}{11.981395in}}{\pgfqpoint{5.489583in}{0.877907in}}%
\pgfusepath{clip}%
\pgfsetbuttcap%
\pgfsetroundjoin%
\pgfsetlinewidth{1.505625pt}%
\definecolor{currentstroke}{rgb}{0.000000,0.000000,0.000000}%
\pgfsetstrokecolor{currentstroke}%
\pgfsetdash{}{0pt}%
\pgfpathmoveto{\pgfqpoint{11.908288in}{12.069362in}}%
\pgfpathlineto{\pgfqpoint{11.908288in}{12.074803in}}%
\pgfusepath{stroke}%
\end{pgfscope}%
\begin{pgfscope}%
\pgfpathrectangle{\pgfqpoint{9.810417in}{11.981395in}}{\pgfqpoint{5.489583in}{0.877907in}}%
\pgfusepath{clip}%
\pgfsetbuttcap%
\pgfsetroundjoin%
\pgfsetlinewidth{1.505625pt}%
\definecolor{currentstroke}{rgb}{0.000000,0.000000,0.000000}%
\pgfsetstrokecolor{currentstroke}%
\pgfsetdash{}{0pt}%
\pgfpathmoveto{\pgfqpoint{12.031511in}{12.069362in}}%
\pgfpathlineto{\pgfqpoint{12.031511in}{12.102232in}}%
\pgfusepath{stroke}%
\end{pgfscope}%
\begin{pgfscope}%
\pgfpathrectangle{\pgfqpoint{9.810417in}{11.981395in}}{\pgfqpoint{5.489583in}{0.877907in}}%
\pgfusepath{clip}%
\pgfsetbuttcap%
\pgfsetroundjoin%
\pgfsetlinewidth{1.505625pt}%
\definecolor{currentstroke}{rgb}{0.000000,0.000000,0.000000}%
\pgfsetstrokecolor{currentstroke}%
\pgfsetdash{}{0pt}%
\pgfpathmoveto{\pgfqpoint{12.154734in}{12.069362in}}%
\pgfpathlineto{\pgfqpoint{12.154734in}{12.077707in}}%
\pgfusepath{stroke}%
\end{pgfscope}%
\begin{pgfscope}%
\pgfpathrectangle{\pgfqpoint{9.810417in}{11.981395in}}{\pgfqpoint{5.489583in}{0.877907in}}%
\pgfusepath{clip}%
\pgfsetbuttcap%
\pgfsetroundjoin%
\pgfsetlinewidth{1.505625pt}%
\definecolor{currentstroke}{rgb}{0.000000,0.000000,0.000000}%
\pgfsetstrokecolor{currentstroke}%
\pgfsetdash{}{0pt}%
\pgfpathmoveto{\pgfqpoint{12.277957in}{12.069362in}}%
\pgfpathlineto{\pgfqpoint{12.277957in}{12.047248in}}%
\pgfusepath{stroke}%
\end{pgfscope}%
\begin{pgfscope}%
\pgfpathrectangle{\pgfqpoint{9.810417in}{11.981395in}}{\pgfqpoint{5.489583in}{0.877907in}}%
\pgfusepath{clip}%
\pgfsetbuttcap%
\pgfsetroundjoin%
\pgfsetlinewidth{1.505625pt}%
\definecolor{currentstroke}{rgb}{0.000000,0.000000,0.000000}%
\pgfsetstrokecolor{currentstroke}%
\pgfsetdash{}{0pt}%
\pgfpathmoveto{\pgfqpoint{12.401180in}{12.069362in}}%
\pgfpathlineto{\pgfqpoint{12.401180in}{12.021300in}}%
\pgfusepath{stroke}%
\end{pgfscope}%
\begin{pgfscope}%
\pgfpathrectangle{\pgfqpoint{9.810417in}{11.981395in}}{\pgfqpoint{5.489583in}{0.877907in}}%
\pgfusepath{clip}%
\pgfsetbuttcap%
\pgfsetroundjoin%
\pgfsetlinewidth{1.505625pt}%
\definecolor{currentstroke}{rgb}{0.000000,0.000000,0.000000}%
\pgfsetstrokecolor{currentstroke}%
\pgfsetdash{}{0pt}%
\pgfpathmoveto{\pgfqpoint{12.524403in}{12.069362in}}%
\pgfpathlineto{\pgfqpoint{12.524403in}{12.055468in}}%
\pgfusepath{stroke}%
\end{pgfscope}%
\begin{pgfscope}%
\pgfpathrectangle{\pgfqpoint{9.810417in}{11.981395in}}{\pgfqpoint{5.489583in}{0.877907in}}%
\pgfusepath{clip}%
\pgfsetbuttcap%
\pgfsetroundjoin%
\pgfsetlinewidth{1.505625pt}%
\definecolor{currentstroke}{rgb}{0.000000,0.000000,0.000000}%
\pgfsetstrokecolor{currentstroke}%
\pgfsetdash{}{0pt}%
\pgfpathmoveto{\pgfqpoint{12.647626in}{12.069362in}}%
\pgfpathlineto{\pgfqpoint{12.647626in}{12.063044in}}%
\pgfusepath{stroke}%
\end{pgfscope}%
\begin{pgfscope}%
\pgfpathrectangle{\pgfqpoint{9.810417in}{11.981395in}}{\pgfqpoint{5.489583in}{0.877907in}}%
\pgfusepath{clip}%
\pgfsetbuttcap%
\pgfsetroundjoin%
\pgfsetlinewidth{1.505625pt}%
\definecolor{currentstroke}{rgb}{0.000000,0.000000,0.000000}%
\pgfsetstrokecolor{currentstroke}%
\pgfsetdash{}{0pt}%
\pgfpathmoveto{\pgfqpoint{12.770849in}{12.069362in}}%
\pgfpathlineto{\pgfqpoint{12.770849in}{12.071533in}}%
\pgfusepath{stroke}%
\end{pgfscope}%
\begin{pgfscope}%
\pgfpathrectangle{\pgfqpoint{9.810417in}{11.981395in}}{\pgfqpoint{5.489583in}{0.877907in}}%
\pgfusepath{clip}%
\pgfsetbuttcap%
\pgfsetroundjoin%
\pgfsetlinewidth{1.505625pt}%
\definecolor{currentstroke}{rgb}{0.000000,0.000000,0.000000}%
\pgfsetstrokecolor{currentstroke}%
\pgfsetdash{}{0pt}%
\pgfpathmoveto{\pgfqpoint{12.894072in}{12.069362in}}%
\pgfpathlineto{\pgfqpoint{12.894072in}{12.050384in}}%
\pgfusepath{stroke}%
\end{pgfscope}%
\begin{pgfscope}%
\pgfpathrectangle{\pgfqpoint{9.810417in}{11.981395in}}{\pgfqpoint{5.489583in}{0.877907in}}%
\pgfusepath{clip}%
\pgfsetbuttcap%
\pgfsetroundjoin%
\pgfsetlinewidth{1.505625pt}%
\definecolor{currentstroke}{rgb}{0.000000,0.000000,0.000000}%
\pgfsetstrokecolor{currentstroke}%
\pgfsetdash{}{0pt}%
\pgfpathmoveto{\pgfqpoint{13.017294in}{12.069362in}}%
\pgfpathlineto{\pgfqpoint{13.017294in}{12.066209in}}%
\pgfusepath{stroke}%
\end{pgfscope}%
\begin{pgfscope}%
\pgfpathrectangle{\pgfqpoint{9.810417in}{11.981395in}}{\pgfqpoint{5.489583in}{0.877907in}}%
\pgfusepath{clip}%
\pgfsetbuttcap%
\pgfsetroundjoin%
\pgfsetlinewidth{1.505625pt}%
\definecolor{currentstroke}{rgb}{0.000000,0.000000,0.000000}%
\pgfsetstrokecolor{currentstroke}%
\pgfsetdash{}{0pt}%
\pgfpathmoveto{\pgfqpoint{13.140517in}{12.069362in}}%
\pgfpathlineto{\pgfqpoint{13.140517in}{12.072928in}}%
\pgfusepath{stroke}%
\end{pgfscope}%
\begin{pgfscope}%
\pgfpathrectangle{\pgfqpoint{9.810417in}{11.981395in}}{\pgfqpoint{5.489583in}{0.877907in}}%
\pgfusepath{clip}%
\pgfsetbuttcap%
\pgfsetroundjoin%
\pgfsetlinewidth{1.505625pt}%
\definecolor{currentstroke}{rgb}{0.000000,0.000000,0.000000}%
\pgfsetstrokecolor{currentstroke}%
\pgfsetdash{}{0pt}%
\pgfpathmoveto{\pgfqpoint{13.263740in}{12.069362in}}%
\pgfpathlineto{\pgfqpoint{13.263740in}{12.073318in}}%
\pgfusepath{stroke}%
\end{pgfscope}%
\begin{pgfscope}%
\pgfpathrectangle{\pgfqpoint{9.810417in}{11.981395in}}{\pgfqpoint{5.489583in}{0.877907in}}%
\pgfusepath{clip}%
\pgfsetbuttcap%
\pgfsetroundjoin%
\pgfsetlinewidth{1.505625pt}%
\definecolor{currentstroke}{rgb}{0.000000,0.000000,0.000000}%
\pgfsetstrokecolor{currentstroke}%
\pgfsetdash{}{0pt}%
\pgfpathmoveto{\pgfqpoint{13.386963in}{12.069362in}}%
\pgfpathlineto{\pgfqpoint{13.386963in}{12.077699in}}%
\pgfusepath{stroke}%
\end{pgfscope}%
\begin{pgfscope}%
\pgfpathrectangle{\pgfqpoint{9.810417in}{11.981395in}}{\pgfqpoint{5.489583in}{0.877907in}}%
\pgfusepath{clip}%
\pgfsetbuttcap%
\pgfsetroundjoin%
\pgfsetlinewidth{1.505625pt}%
\definecolor{currentstroke}{rgb}{0.000000,0.000000,0.000000}%
\pgfsetstrokecolor{currentstroke}%
\pgfsetdash{}{0pt}%
\pgfpathmoveto{\pgfqpoint{13.510186in}{12.069362in}}%
\pgfpathlineto{\pgfqpoint{13.510186in}{12.068278in}}%
\pgfusepath{stroke}%
\end{pgfscope}%
\begin{pgfscope}%
\pgfpathrectangle{\pgfqpoint{9.810417in}{11.981395in}}{\pgfqpoint{5.489583in}{0.877907in}}%
\pgfusepath{clip}%
\pgfsetbuttcap%
\pgfsetroundjoin%
\pgfsetlinewidth{1.505625pt}%
\definecolor{currentstroke}{rgb}{0.000000,0.000000,0.000000}%
\pgfsetstrokecolor{currentstroke}%
\pgfsetdash{}{0pt}%
\pgfpathmoveto{\pgfqpoint{13.633409in}{12.069362in}}%
\pgfpathlineto{\pgfqpoint{13.633409in}{12.065080in}}%
\pgfusepath{stroke}%
\end{pgfscope}%
\begin{pgfscope}%
\pgfpathrectangle{\pgfqpoint{9.810417in}{11.981395in}}{\pgfqpoint{5.489583in}{0.877907in}}%
\pgfusepath{clip}%
\pgfsetbuttcap%
\pgfsetroundjoin%
\pgfsetlinewidth{1.505625pt}%
\definecolor{currentstroke}{rgb}{0.000000,0.000000,0.000000}%
\pgfsetstrokecolor{currentstroke}%
\pgfsetdash{}{0pt}%
\pgfpathmoveto{\pgfqpoint{13.756632in}{12.069362in}}%
\pgfpathlineto{\pgfqpoint{13.756632in}{12.051108in}}%
\pgfusepath{stroke}%
\end{pgfscope}%
\begin{pgfscope}%
\pgfpathrectangle{\pgfqpoint{9.810417in}{11.981395in}}{\pgfqpoint{5.489583in}{0.877907in}}%
\pgfusepath{clip}%
\pgfsetbuttcap%
\pgfsetroundjoin%
\pgfsetlinewidth{1.505625pt}%
\definecolor{currentstroke}{rgb}{0.000000,0.000000,0.000000}%
\pgfsetstrokecolor{currentstroke}%
\pgfsetdash{}{0pt}%
\pgfpathmoveto{\pgfqpoint{13.879855in}{12.069362in}}%
\pgfpathlineto{\pgfqpoint{13.879855in}{12.058356in}}%
\pgfusepath{stroke}%
\end{pgfscope}%
\begin{pgfscope}%
\pgfpathrectangle{\pgfqpoint{9.810417in}{11.981395in}}{\pgfqpoint{5.489583in}{0.877907in}}%
\pgfusepath{clip}%
\pgfsetbuttcap%
\pgfsetroundjoin%
\pgfsetlinewidth{1.505625pt}%
\definecolor{currentstroke}{rgb}{0.000000,0.000000,0.000000}%
\pgfsetstrokecolor{currentstroke}%
\pgfsetdash{}{0pt}%
\pgfpathmoveto{\pgfqpoint{14.003078in}{12.069362in}}%
\pgfpathlineto{\pgfqpoint{14.003078in}{12.072160in}}%
\pgfusepath{stroke}%
\end{pgfscope}%
\begin{pgfscope}%
\pgfpathrectangle{\pgfqpoint{9.810417in}{11.981395in}}{\pgfqpoint{5.489583in}{0.877907in}}%
\pgfusepath{clip}%
\pgfsetbuttcap%
\pgfsetroundjoin%
\pgfsetlinewidth{1.505625pt}%
\definecolor{currentstroke}{rgb}{0.000000,0.000000,0.000000}%
\pgfsetstrokecolor{currentstroke}%
\pgfsetdash{}{0pt}%
\pgfpathmoveto{\pgfqpoint{14.126301in}{12.069362in}}%
\pgfpathlineto{\pgfqpoint{14.126301in}{12.072771in}}%
\pgfusepath{stroke}%
\end{pgfscope}%
\begin{pgfscope}%
\pgfpathrectangle{\pgfqpoint{9.810417in}{11.981395in}}{\pgfqpoint{5.489583in}{0.877907in}}%
\pgfusepath{clip}%
\pgfsetbuttcap%
\pgfsetroundjoin%
\pgfsetlinewidth{1.505625pt}%
\definecolor{currentstroke}{rgb}{0.000000,0.000000,0.000000}%
\pgfsetstrokecolor{currentstroke}%
\pgfsetdash{}{0pt}%
\pgfpathmoveto{\pgfqpoint{14.249524in}{12.069362in}}%
\pgfpathlineto{\pgfqpoint{14.249524in}{12.060944in}}%
\pgfusepath{stroke}%
\end{pgfscope}%
\begin{pgfscope}%
\pgfpathrectangle{\pgfqpoint{9.810417in}{11.981395in}}{\pgfqpoint{5.489583in}{0.877907in}}%
\pgfusepath{clip}%
\pgfsetbuttcap%
\pgfsetroundjoin%
\pgfsetlinewidth{1.505625pt}%
\definecolor{currentstroke}{rgb}{0.000000,0.000000,0.000000}%
\pgfsetstrokecolor{currentstroke}%
\pgfsetdash{}{0pt}%
\pgfpathmoveto{\pgfqpoint{14.372747in}{12.069362in}}%
\pgfpathlineto{\pgfqpoint{14.372747in}{12.038156in}}%
\pgfusepath{stroke}%
\end{pgfscope}%
\begin{pgfscope}%
\pgfpathrectangle{\pgfqpoint{9.810417in}{11.981395in}}{\pgfqpoint{5.489583in}{0.877907in}}%
\pgfusepath{clip}%
\pgfsetbuttcap%
\pgfsetroundjoin%
\pgfsetlinewidth{1.505625pt}%
\definecolor{currentstroke}{rgb}{0.000000,0.000000,0.000000}%
\pgfsetstrokecolor{currentstroke}%
\pgfsetdash{}{0pt}%
\pgfpathmoveto{\pgfqpoint{14.495970in}{12.069362in}}%
\pgfpathlineto{\pgfqpoint{14.495970in}{12.046751in}}%
\pgfusepath{stroke}%
\end{pgfscope}%
\begin{pgfscope}%
\pgfpathrectangle{\pgfqpoint{9.810417in}{11.981395in}}{\pgfqpoint{5.489583in}{0.877907in}}%
\pgfusepath{clip}%
\pgfsetbuttcap%
\pgfsetroundjoin%
\pgfsetlinewidth{1.505625pt}%
\definecolor{currentstroke}{rgb}{0.000000,0.000000,0.000000}%
\pgfsetstrokecolor{currentstroke}%
\pgfsetdash{}{0pt}%
\pgfpathmoveto{\pgfqpoint{14.619193in}{12.069362in}}%
\pgfpathlineto{\pgfqpoint{14.619193in}{12.069568in}}%
\pgfusepath{stroke}%
\end{pgfscope}%
\begin{pgfscope}%
\pgfpathrectangle{\pgfqpoint{9.810417in}{11.981395in}}{\pgfqpoint{5.489583in}{0.877907in}}%
\pgfusepath{clip}%
\pgfsetbuttcap%
\pgfsetroundjoin%
\pgfsetlinewidth{1.505625pt}%
\definecolor{currentstroke}{rgb}{0.000000,0.000000,0.000000}%
\pgfsetstrokecolor{currentstroke}%
\pgfsetdash{}{0pt}%
\pgfpathmoveto{\pgfqpoint{14.742416in}{12.069362in}}%
\pgfpathlineto{\pgfqpoint{14.742416in}{12.060900in}}%
\pgfusepath{stroke}%
\end{pgfscope}%
\begin{pgfscope}%
\pgfpathrectangle{\pgfqpoint{9.810417in}{11.981395in}}{\pgfqpoint{5.489583in}{0.877907in}}%
\pgfusepath{clip}%
\pgfsetbuttcap%
\pgfsetroundjoin%
\pgfsetlinewidth{1.505625pt}%
\definecolor{currentstroke}{rgb}{0.000000,0.000000,0.000000}%
\pgfsetstrokecolor{currentstroke}%
\pgfsetdash{}{0pt}%
\pgfpathmoveto{\pgfqpoint{14.865639in}{12.069362in}}%
\pgfpathlineto{\pgfqpoint{14.865639in}{12.087149in}}%
\pgfusepath{stroke}%
\end{pgfscope}%
\begin{pgfscope}%
\pgfpathrectangle{\pgfqpoint{9.810417in}{11.981395in}}{\pgfqpoint{5.489583in}{0.877907in}}%
\pgfusepath{clip}%
\pgfsetbuttcap%
\pgfsetroundjoin%
\pgfsetlinewidth{1.505625pt}%
\definecolor{currentstroke}{rgb}{0.000000,0.000000,0.000000}%
\pgfsetstrokecolor{currentstroke}%
\pgfsetdash{}{0pt}%
\pgfpathmoveto{\pgfqpoint{14.988862in}{12.069362in}}%
\pgfpathlineto{\pgfqpoint{14.988862in}{12.059867in}}%
\pgfusepath{stroke}%
\end{pgfscope}%
\begin{pgfscope}%
\pgfpathrectangle{\pgfqpoint{9.810417in}{11.981395in}}{\pgfqpoint{5.489583in}{0.877907in}}%
\pgfusepath{clip}%
\pgfsetroundcap%
\pgfsetroundjoin%
\pgfsetlinewidth{1.505625pt}%
\definecolor{currentstroke}{rgb}{0.121569,0.466667,0.705882}%
\pgfsetstrokecolor{currentstroke}%
\pgfsetdash{}{0pt}%
\pgfpathmoveto{\pgfqpoint{9.810417in}{12.069362in}}%
\pgfpathlineto{\pgfqpoint{15.300000in}{12.069362in}}%
\pgfusepath{stroke}%
\end{pgfscope}%
\begin{pgfscope}%
\pgfpathrectangle{\pgfqpoint{9.810417in}{11.981395in}}{\pgfqpoint{5.489583in}{0.877907in}}%
\pgfusepath{clip}%
\pgfsetbuttcap%
\pgfsetroundjoin%
\definecolor{currentfill}{rgb}{0.121569,0.466667,0.705882}%
\pgfsetfillcolor{currentfill}%
\pgfsetlinewidth{1.003750pt}%
\definecolor{currentstroke}{rgb}{0.121569,0.466667,0.705882}%
\pgfsetstrokecolor{currentstroke}%
\pgfsetdash{}{0pt}%
\pgfsys@defobject{currentmarker}{\pgfqpoint{-0.034722in}{-0.034722in}}{\pgfqpoint{0.034722in}{0.034722in}}{%
\pgfpathmoveto{\pgfqpoint{0.000000in}{-0.034722in}}%
\pgfpathcurveto{\pgfqpoint{0.009208in}{-0.034722in}}{\pgfqpoint{0.018041in}{-0.031064in}}{\pgfqpoint{0.024552in}{-0.024552in}}%
\pgfpathcurveto{\pgfqpoint{0.031064in}{-0.018041in}}{\pgfqpoint{0.034722in}{-0.009208in}}{\pgfqpoint{0.034722in}{0.000000in}}%
\pgfpathcurveto{\pgfqpoint{0.034722in}{0.009208in}}{\pgfqpoint{0.031064in}{0.018041in}}{\pgfqpoint{0.024552in}{0.024552in}}%
\pgfpathcurveto{\pgfqpoint{0.018041in}{0.031064in}}{\pgfqpoint{0.009208in}{0.034722in}}{\pgfqpoint{0.000000in}{0.034722in}}%
\pgfpathcurveto{\pgfqpoint{-0.009208in}{0.034722in}}{\pgfqpoint{-0.018041in}{0.031064in}}{\pgfqpoint{-0.024552in}{0.024552in}}%
\pgfpathcurveto{\pgfqpoint{-0.031064in}{0.018041in}}{\pgfqpoint{-0.034722in}{0.009208in}}{\pgfqpoint{-0.034722in}{0.000000in}}%
\pgfpathcurveto{\pgfqpoint{-0.034722in}{-0.009208in}}{\pgfqpoint{-0.031064in}{-0.018041in}}{\pgfqpoint{-0.024552in}{-0.024552in}}%
\pgfpathcurveto{\pgfqpoint{-0.018041in}{-0.031064in}}{\pgfqpoint{-0.009208in}{-0.034722in}}{\pgfqpoint{0.000000in}{-0.034722in}}%
\pgfpathclose%
\pgfusepath{stroke,fill}%
}%
\begin{pgfscope}%
\pgfsys@transformshift{10.059943in}{12.819397in}%
\pgfsys@useobject{currentmarker}{}%
\end{pgfscope}%
\begin{pgfscope}%
\pgfsys@transformshift{10.183166in}{12.817200in}%
\pgfsys@useobject{currentmarker}{}%
\end{pgfscope}%
\begin{pgfscope}%
\pgfsys@transformshift{10.306389in}{12.081146in}%
\pgfsys@useobject{currentmarker}{}%
\end{pgfscope}%
\begin{pgfscope}%
\pgfsys@transformshift{10.429612in}{12.073290in}%
\pgfsys@useobject{currentmarker}{}%
\end{pgfscope}%
\begin{pgfscope}%
\pgfsys@transformshift{10.552835in}{12.065606in}%
\pgfsys@useobject{currentmarker}{}%
\end{pgfscope}%
\begin{pgfscope}%
\pgfsys@transformshift{10.676058in}{12.064613in}%
\pgfsys@useobject{currentmarker}{}%
\end{pgfscope}%
\begin{pgfscope}%
\pgfsys@transformshift{10.799281in}{12.067475in}%
\pgfsys@useobject{currentmarker}{}%
\end{pgfscope}%
\begin{pgfscope}%
\pgfsys@transformshift{10.922504in}{12.060884in}%
\pgfsys@useobject{currentmarker}{}%
\end{pgfscope}%
\begin{pgfscope}%
\pgfsys@transformshift{11.045727in}{12.065368in}%
\pgfsys@useobject{currentmarker}{}%
\end{pgfscope}%
\begin{pgfscope}%
\pgfsys@transformshift{11.168950in}{12.065133in}%
\pgfsys@useobject{currentmarker}{}%
\end{pgfscope}%
\begin{pgfscope}%
\pgfsys@transformshift{11.292173in}{12.108848in}%
\pgfsys@useobject{currentmarker}{}%
\end{pgfscope}%
\begin{pgfscope}%
\pgfsys@transformshift{11.415396in}{12.081383in}%
\pgfsys@useobject{currentmarker}{}%
\end{pgfscope}%
\begin{pgfscope}%
\pgfsys@transformshift{11.538619in}{12.058489in}%
\pgfsys@useobject{currentmarker}{}%
\end{pgfscope}%
\begin{pgfscope}%
\pgfsys@transformshift{11.661842in}{12.075366in}%
\pgfsys@useobject{currentmarker}{}%
\end{pgfscope}%
\begin{pgfscope}%
\pgfsys@transformshift{11.785065in}{12.085929in}%
\pgfsys@useobject{currentmarker}{}%
\end{pgfscope}%
\begin{pgfscope}%
\pgfsys@transformshift{11.908288in}{12.074803in}%
\pgfsys@useobject{currentmarker}{}%
\end{pgfscope}%
\begin{pgfscope}%
\pgfsys@transformshift{12.031511in}{12.102232in}%
\pgfsys@useobject{currentmarker}{}%
\end{pgfscope}%
\begin{pgfscope}%
\pgfsys@transformshift{12.154734in}{12.077707in}%
\pgfsys@useobject{currentmarker}{}%
\end{pgfscope}%
\begin{pgfscope}%
\pgfsys@transformshift{12.277957in}{12.047248in}%
\pgfsys@useobject{currentmarker}{}%
\end{pgfscope}%
\begin{pgfscope}%
\pgfsys@transformshift{12.401180in}{12.021300in}%
\pgfsys@useobject{currentmarker}{}%
\end{pgfscope}%
\begin{pgfscope}%
\pgfsys@transformshift{12.524403in}{12.055468in}%
\pgfsys@useobject{currentmarker}{}%
\end{pgfscope}%
\begin{pgfscope}%
\pgfsys@transformshift{12.647626in}{12.063044in}%
\pgfsys@useobject{currentmarker}{}%
\end{pgfscope}%
\begin{pgfscope}%
\pgfsys@transformshift{12.770849in}{12.071533in}%
\pgfsys@useobject{currentmarker}{}%
\end{pgfscope}%
\begin{pgfscope}%
\pgfsys@transformshift{12.894072in}{12.050384in}%
\pgfsys@useobject{currentmarker}{}%
\end{pgfscope}%
\begin{pgfscope}%
\pgfsys@transformshift{13.017294in}{12.066209in}%
\pgfsys@useobject{currentmarker}{}%
\end{pgfscope}%
\begin{pgfscope}%
\pgfsys@transformshift{13.140517in}{12.072928in}%
\pgfsys@useobject{currentmarker}{}%
\end{pgfscope}%
\begin{pgfscope}%
\pgfsys@transformshift{13.263740in}{12.073318in}%
\pgfsys@useobject{currentmarker}{}%
\end{pgfscope}%
\begin{pgfscope}%
\pgfsys@transformshift{13.386963in}{12.077699in}%
\pgfsys@useobject{currentmarker}{}%
\end{pgfscope}%
\begin{pgfscope}%
\pgfsys@transformshift{13.510186in}{12.068278in}%
\pgfsys@useobject{currentmarker}{}%
\end{pgfscope}%
\begin{pgfscope}%
\pgfsys@transformshift{13.633409in}{12.065080in}%
\pgfsys@useobject{currentmarker}{}%
\end{pgfscope}%
\begin{pgfscope}%
\pgfsys@transformshift{13.756632in}{12.051108in}%
\pgfsys@useobject{currentmarker}{}%
\end{pgfscope}%
\begin{pgfscope}%
\pgfsys@transformshift{13.879855in}{12.058356in}%
\pgfsys@useobject{currentmarker}{}%
\end{pgfscope}%
\begin{pgfscope}%
\pgfsys@transformshift{14.003078in}{12.072160in}%
\pgfsys@useobject{currentmarker}{}%
\end{pgfscope}%
\begin{pgfscope}%
\pgfsys@transformshift{14.126301in}{12.072771in}%
\pgfsys@useobject{currentmarker}{}%
\end{pgfscope}%
\begin{pgfscope}%
\pgfsys@transformshift{14.249524in}{12.060944in}%
\pgfsys@useobject{currentmarker}{}%
\end{pgfscope}%
\begin{pgfscope}%
\pgfsys@transformshift{14.372747in}{12.038156in}%
\pgfsys@useobject{currentmarker}{}%
\end{pgfscope}%
\begin{pgfscope}%
\pgfsys@transformshift{14.495970in}{12.046751in}%
\pgfsys@useobject{currentmarker}{}%
\end{pgfscope}%
\begin{pgfscope}%
\pgfsys@transformshift{14.619193in}{12.069568in}%
\pgfsys@useobject{currentmarker}{}%
\end{pgfscope}%
\begin{pgfscope}%
\pgfsys@transformshift{14.742416in}{12.060900in}%
\pgfsys@useobject{currentmarker}{}%
\end{pgfscope}%
\begin{pgfscope}%
\pgfsys@transformshift{14.865639in}{12.087149in}%
\pgfsys@useobject{currentmarker}{}%
\end{pgfscope}%
\begin{pgfscope}%
\pgfsys@transformshift{14.988862in}{12.059867in}%
\pgfsys@useobject{currentmarker}{}%
\end{pgfscope}%
\end{pgfscope}%
\begin{pgfscope}%
\pgfsetrectcap%
\pgfsetmiterjoin%
\pgfsetlinewidth{0.803000pt}%
\definecolor{currentstroke}{rgb}{1.000000,1.000000,1.000000}%
\pgfsetstrokecolor{currentstroke}%
\pgfsetdash{}{0pt}%
\pgfpathmoveto{\pgfqpoint{9.810417in}{11.981395in}}%
\pgfpathlineto{\pgfqpoint{9.810417in}{12.859302in}}%
\pgfusepath{stroke}%
\end{pgfscope}%
\begin{pgfscope}%
\pgfsetrectcap%
\pgfsetmiterjoin%
\pgfsetlinewidth{0.803000pt}%
\definecolor{currentstroke}{rgb}{1.000000,1.000000,1.000000}%
\pgfsetstrokecolor{currentstroke}%
\pgfsetdash{}{0pt}%
\pgfpathmoveto{\pgfqpoint{15.300000in}{11.981395in}}%
\pgfpathlineto{\pgfqpoint{15.300000in}{12.859302in}}%
\pgfusepath{stroke}%
\end{pgfscope}%
\begin{pgfscope}%
\pgfsetrectcap%
\pgfsetmiterjoin%
\pgfsetlinewidth{0.803000pt}%
\definecolor{currentstroke}{rgb}{1.000000,1.000000,1.000000}%
\pgfsetstrokecolor{currentstroke}%
\pgfsetdash{}{0pt}%
\pgfpathmoveto{\pgfqpoint{9.810417in}{11.981395in}}%
\pgfpathlineto{\pgfqpoint{15.300000in}{11.981395in}}%
\pgfusepath{stroke}%
\end{pgfscope}%
\begin{pgfscope}%
\pgfsetrectcap%
\pgfsetmiterjoin%
\pgfsetlinewidth{0.803000pt}%
\definecolor{currentstroke}{rgb}{1.000000,1.000000,1.000000}%
\pgfsetstrokecolor{currentstroke}%
\pgfsetdash{}{0pt}%
\pgfpathmoveto{\pgfqpoint{9.810417in}{12.859302in}}%
\pgfpathlineto{\pgfqpoint{15.300000in}{12.859302in}}%
\pgfusepath{stroke}%
\end{pgfscope}%
\begin{pgfscope}%
\definecolor{textcolor}{rgb}{0.150000,0.150000,0.150000}%
\pgfsetstrokecolor{textcolor}%
\pgfsetfillcolor{textcolor}%
\pgftext[x=12.555208in,y=12.942636in,,base]{\color{textcolor}\rmfamily\fontsize{16.800000}{20.160000}\selectfont Partial Autocorrelation}%
\end{pgfscope}%
\begin{pgfscope}%
\pgfsetbuttcap%
\pgfsetmiterjoin%
\definecolor{currentfill}{rgb}{0.917647,0.917647,0.949020}%
\pgfsetfillcolor{currentfill}%
\pgfsetlinewidth{0.000000pt}%
\definecolor{currentstroke}{rgb}{0.000000,0.000000,0.000000}%
\pgfsetstrokecolor{currentstroke}%
\pgfsetstrokeopacity{0.000000}%
\pgfsetdash{}{0pt}%
\pgfpathmoveto{\pgfqpoint{2.125000in}{10.401163in}}%
\pgfpathlineto{\pgfqpoint{7.614583in}{10.401163in}}%
\pgfpathlineto{\pgfqpoint{7.614583in}{11.279070in}}%
\pgfpathlineto{\pgfqpoint{2.125000in}{11.279070in}}%
\pgfpathclose%
\pgfusepath{fill}%
\end{pgfscope}%
\begin{pgfscope}%
\pgfpathrectangle{\pgfqpoint{2.125000in}{10.401163in}}{\pgfqpoint{5.489583in}{0.877907in}}%
\pgfusepath{clip}%
\pgfsetroundcap%
\pgfsetroundjoin%
\pgfsetlinewidth{0.803000pt}%
\definecolor{currentstroke}{rgb}{1.000000,1.000000,1.000000}%
\pgfsetstrokecolor{currentstroke}%
\pgfsetdash{}{0pt}%
\pgfpathmoveto{\pgfqpoint{2.374527in}{10.401163in}}%
\pgfpathlineto{\pgfqpoint{2.374527in}{11.279070in}}%
\pgfusepath{stroke}%
\end{pgfscope}%
\begin{pgfscope}%
\definecolor{textcolor}{rgb}{0.150000,0.150000,0.150000}%
\pgfsetstrokecolor{textcolor}%
\pgfsetfillcolor{textcolor}%
\pgftext[x=2.374527in,y=10.303941in,,top]{\color{textcolor}\rmfamily\fontsize{14.000000}{16.800000}\selectfont 0}%
\end{pgfscope}%
\begin{pgfscope}%
\pgfpathrectangle{\pgfqpoint{2.125000in}{10.401163in}}{\pgfqpoint{5.489583in}{0.877907in}}%
\pgfusepath{clip}%
\pgfsetroundcap%
\pgfsetroundjoin%
\pgfsetlinewidth{0.803000pt}%
\definecolor{currentstroke}{rgb}{1.000000,1.000000,1.000000}%
\pgfsetstrokecolor{currentstroke}%
\pgfsetdash{}{0pt}%
\pgfpathmoveto{\pgfqpoint{2.990641in}{10.401163in}}%
\pgfpathlineto{\pgfqpoint{2.990641in}{11.279070in}}%
\pgfusepath{stroke}%
\end{pgfscope}%
\begin{pgfscope}%
\definecolor{textcolor}{rgb}{0.150000,0.150000,0.150000}%
\pgfsetstrokecolor{textcolor}%
\pgfsetfillcolor{textcolor}%
\pgftext[x=2.990641in,y=10.303941in,,top]{\color{textcolor}\rmfamily\fontsize{14.000000}{16.800000}\selectfont 5}%
\end{pgfscope}%
\begin{pgfscope}%
\pgfpathrectangle{\pgfqpoint{2.125000in}{10.401163in}}{\pgfqpoint{5.489583in}{0.877907in}}%
\pgfusepath{clip}%
\pgfsetroundcap%
\pgfsetroundjoin%
\pgfsetlinewidth{0.803000pt}%
\definecolor{currentstroke}{rgb}{1.000000,1.000000,1.000000}%
\pgfsetstrokecolor{currentstroke}%
\pgfsetdash{}{0pt}%
\pgfpathmoveto{\pgfqpoint{3.606756in}{10.401163in}}%
\pgfpathlineto{\pgfqpoint{3.606756in}{11.279070in}}%
\pgfusepath{stroke}%
\end{pgfscope}%
\begin{pgfscope}%
\definecolor{textcolor}{rgb}{0.150000,0.150000,0.150000}%
\pgfsetstrokecolor{textcolor}%
\pgfsetfillcolor{textcolor}%
\pgftext[x=3.606756in,y=10.303941in,,top]{\color{textcolor}\rmfamily\fontsize{14.000000}{16.800000}\selectfont 10}%
\end{pgfscope}%
\begin{pgfscope}%
\pgfpathrectangle{\pgfqpoint{2.125000in}{10.401163in}}{\pgfqpoint{5.489583in}{0.877907in}}%
\pgfusepath{clip}%
\pgfsetroundcap%
\pgfsetroundjoin%
\pgfsetlinewidth{0.803000pt}%
\definecolor{currentstroke}{rgb}{1.000000,1.000000,1.000000}%
\pgfsetstrokecolor{currentstroke}%
\pgfsetdash{}{0pt}%
\pgfpathmoveto{\pgfqpoint{4.222871in}{10.401163in}}%
\pgfpathlineto{\pgfqpoint{4.222871in}{11.279070in}}%
\pgfusepath{stroke}%
\end{pgfscope}%
\begin{pgfscope}%
\definecolor{textcolor}{rgb}{0.150000,0.150000,0.150000}%
\pgfsetstrokecolor{textcolor}%
\pgfsetfillcolor{textcolor}%
\pgftext[x=4.222871in,y=10.303941in,,top]{\color{textcolor}\rmfamily\fontsize{14.000000}{16.800000}\selectfont 15}%
\end{pgfscope}%
\begin{pgfscope}%
\pgfpathrectangle{\pgfqpoint{2.125000in}{10.401163in}}{\pgfqpoint{5.489583in}{0.877907in}}%
\pgfusepath{clip}%
\pgfsetroundcap%
\pgfsetroundjoin%
\pgfsetlinewidth{0.803000pt}%
\definecolor{currentstroke}{rgb}{1.000000,1.000000,1.000000}%
\pgfsetstrokecolor{currentstroke}%
\pgfsetdash{}{0pt}%
\pgfpathmoveto{\pgfqpoint{4.838986in}{10.401163in}}%
\pgfpathlineto{\pgfqpoint{4.838986in}{11.279070in}}%
\pgfusepath{stroke}%
\end{pgfscope}%
\begin{pgfscope}%
\definecolor{textcolor}{rgb}{0.150000,0.150000,0.150000}%
\pgfsetstrokecolor{textcolor}%
\pgfsetfillcolor{textcolor}%
\pgftext[x=4.838986in,y=10.303941in,,top]{\color{textcolor}\rmfamily\fontsize{14.000000}{16.800000}\selectfont 20}%
\end{pgfscope}%
\begin{pgfscope}%
\pgfpathrectangle{\pgfqpoint{2.125000in}{10.401163in}}{\pgfqpoint{5.489583in}{0.877907in}}%
\pgfusepath{clip}%
\pgfsetroundcap%
\pgfsetroundjoin%
\pgfsetlinewidth{0.803000pt}%
\definecolor{currentstroke}{rgb}{1.000000,1.000000,1.000000}%
\pgfsetstrokecolor{currentstroke}%
\pgfsetdash{}{0pt}%
\pgfpathmoveto{\pgfqpoint{5.455101in}{10.401163in}}%
\pgfpathlineto{\pgfqpoint{5.455101in}{11.279070in}}%
\pgfusepath{stroke}%
\end{pgfscope}%
\begin{pgfscope}%
\definecolor{textcolor}{rgb}{0.150000,0.150000,0.150000}%
\pgfsetstrokecolor{textcolor}%
\pgfsetfillcolor{textcolor}%
\pgftext[x=5.455101in,y=10.303941in,,top]{\color{textcolor}\rmfamily\fontsize{14.000000}{16.800000}\selectfont 25}%
\end{pgfscope}%
\begin{pgfscope}%
\pgfpathrectangle{\pgfqpoint{2.125000in}{10.401163in}}{\pgfqpoint{5.489583in}{0.877907in}}%
\pgfusepath{clip}%
\pgfsetroundcap%
\pgfsetroundjoin%
\pgfsetlinewidth{0.803000pt}%
\definecolor{currentstroke}{rgb}{1.000000,1.000000,1.000000}%
\pgfsetstrokecolor{currentstroke}%
\pgfsetdash{}{0pt}%
\pgfpathmoveto{\pgfqpoint{6.071216in}{10.401163in}}%
\pgfpathlineto{\pgfqpoint{6.071216in}{11.279070in}}%
\pgfusepath{stroke}%
\end{pgfscope}%
\begin{pgfscope}%
\definecolor{textcolor}{rgb}{0.150000,0.150000,0.150000}%
\pgfsetstrokecolor{textcolor}%
\pgfsetfillcolor{textcolor}%
\pgftext[x=6.071216in,y=10.303941in,,top]{\color{textcolor}\rmfamily\fontsize{14.000000}{16.800000}\selectfont 30}%
\end{pgfscope}%
\begin{pgfscope}%
\pgfpathrectangle{\pgfqpoint{2.125000in}{10.401163in}}{\pgfqpoint{5.489583in}{0.877907in}}%
\pgfusepath{clip}%
\pgfsetroundcap%
\pgfsetroundjoin%
\pgfsetlinewidth{0.803000pt}%
\definecolor{currentstroke}{rgb}{1.000000,1.000000,1.000000}%
\pgfsetstrokecolor{currentstroke}%
\pgfsetdash{}{0pt}%
\pgfpathmoveto{\pgfqpoint{6.687330in}{10.401163in}}%
\pgfpathlineto{\pgfqpoint{6.687330in}{11.279070in}}%
\pgfusepath{stroke}%
\end{pgfscope}%
\begin{pgfscope}%
\definecolor{textcolor}{rgb}{0.150000,0.150000,0.150000}%
\pgfsetstrokecolor{textcolor}%
\pgfsetfillcolor{textcolor}%
\pgftext[x=6.687330in,y=10.303941in,,top]{\color{textcolor}\rmfamily\fontsize{14.000000}{16.800000}\selectfont 35}%
\end{pgfscope}%
\begin{pgfscope}%
\pgfpathrectangle{\pgfqpoint{2.125000in}{10.401163in}}{\pgfqpoint{5.489583in}{0.877907in}}%
\pgfusepath{clip}%
\pgfsetroundcap%
\pgfsetroundjoin%
\pgfsetlinewidth{0.803000pt}%
\definecolor{currentstroke}{rgb}{1.000000,1.000000,1.000000}%
\pgfsetstrokecolor{currentstroke}%
\pgfsetdash{}{0pt}%
\pgfpathmoveto{\pgfqpoint{7.303445in}{10.401163in}}%
\pgfpathlineto{\pgfqpoint{7.303445in}{11.279070in}}%
\pgfusepath{stroke}%
\end{pgfscope}%
\begin{pgfscope}%
\definecolor{textcolor}{rgb}{0.150000,0.150000,0.150000}%
\pgfsetstrokecolor{textcolor}%
\pgfsetfillcolor{textcolor}%
\pgftext[x=7.303445in,y=10.303941in,,top]{\color{textcolor}\rmfamily\fontsize{14.000000}{16.800000}\selectfont 40}%
\end{pgfscope}%
\begin{pgfscope}%
\pgfpathrectangle{\pgfqpoint{2.125000in}{10.401163in}}{\pgfqpoint{5.489583in}{0.877907in}}%
\pgfusepath{clip}%
\pgfsetroundcap%
\pgfsetroundjoin%
\pgfsetlinewidth{0.803000pt}%
\definecolor{currentstroke}{rgb}{1.000000,1.000000,1.000000}%
\pgfsetstrokecolor{currentstroke}%
\pgfsetdash{}{0pt}%
\pgfpathmoveto{\pgfqpoint{2.125000in}{10.679420in}}%
\pgfpathlineto{\pgfqpoint{7.614583in}{10.679420in}}%
\pgfusepath{stroke}%
\end{pgfscope}%
\begin{pgfscope}%
\definecolor{textcolor}{rgb}{0.150000,0.150000,0.150000}%
\pgfsetstrokecolor{textcolor}%
\pgfsetfillcolor{textcolor}%
\pgftext[x=1.904066in,y=10.605554in,left,base]{\color{textcolor}\rmfamily\fontsize{14.000000}{16.800000}\selectfont 0}%
\end{pgfscope}%
\begin{pgfscope}%
\pgfpathrectangle{\pgfqpoint{2.125000in}{10.401163in}}{\pgfqpoint{5.489583in}{0.877907in}}%
\pgfusepath{clip}%
\pgfsetroundcap%
\pgfsetroundjoin%
\pgfsetlinewidth{0.803000pt}%
\definecolor{currentstroke}{rgb}{1.000000,1.000000,1.000000}%
\pgfsetstrokecolor{currentstroke}%
\pgfsetdash{}{0pt}%
\pgfpathmoveto{\pgfqpoint{2.125000in}{11.239165in}}%
\pgfpathlineto{\pgfqpoint{7.614583in}{11.239165in}}%
\pgfusepath{stroke}%
\end{pgfscope}%
\begin{pgfscope}%
\definecolor{textcolor}{rgb}{0.150000,0.150000,0.150000}%
\pgfsetstrokecolor{textcolor}%
\pgfsetfillcolor{textcolor}%
\pgftext[x=1.904066in,y=11.165299in,left,base]{\color{textcolor}\rmfamily\fontsize{14.000000}{16.800000}\selectfont 1}%
\end{pgfscope}%
\begin{pgfscope}%
\pgfpathrectangle{\pgfqpoint{2.125000in}{10.401163in}}{\pgfqpoint{5.489583in}{0.877907in}}%
\pgfusepath{clip}%
\pgfsetbuttcap%
\pgfsetroundjoin%
\definecolor{currentfill}{rgb}{0.121569,0.466667,0.705882}%
\pgfsetfillcolor{currentfill}%
\pgfsetfillopacity{0.250000}%
\pgfsetlinewidth{1.003750pt}%
\definecolor{currentstroke}{rgb}{1.000000,1.000000,1.000000}%
\pgfsetstrokecolor{currentstroke}%
\pgfsetstrokeopacity{0.250000}%
\pgfsetdash{}{0pt}%
\pgfpathmoveto{\pgfqpoint{2.436138in}{10.707662in}}%
\pgfpathlineto{\pgfqpoint{2.436138in}{10.651178in}}%
\pgfpathlineto{\pgfqpoint{2.620972in}{10.630586in}}%
\pgfpathlineto{\pgfqpoint{2.744195in}{10.616461in}}%
\pgfpathlineto{\pgfqpoint{2.867418in}{10.605024in}}%
\pgfpathlineto{\pgfqpoint{2.990641in}{10.595174in}}%
\pgfpathlineto{\pgfqpoint{3.113864in}{10.586404in}}%
\pgfpathlineto{\pgfqpoint{3.237087in}{10.578434in}}%
\pgfpathlineto{\pgfqpoint{3.360310in}{10.571087in}}%
\pgfpathlineto{\pgfqpoint{3.483533in}{10.564244in}}%
\pgfpathlineto{\pgfqpoint{3.606756in}{10.557819in}}%
\pgfpathlineto{\pgfqpoint{3.729979in}{10.551750in}}%
\pgfpathlineto{\pgfqpoint{3.853202in}{10.545987in}}%
\pgfpathlineto{\pgfqpoint{3.976425in}{10.540492in}}%
\pgfpathlineto{\pgfqpoint{4.099648in}{10.535235in}}%
\pgfpathlineto{\pgfqpoint{4.222871in}{10.530190in}}%
\pgfpathlineto{\pgfqpoint{4.346094in}{10.525336in}}%
\pgfpathlineto{\pgfqpoint{4.469317in}{10.520655in}}%
\pgfpathlineto{\pgfqpoint{4.592540in}{10.516132in}}%
\pgfpathlineto{\pgfqpoint{4.715763in}{10.511753in}}%
\pgfpathlineto{\pgfqpoint{4.838986in}{10.507508in}}%
\pgfpathlineto{\pgfqpoint{4.962209in}{10.503386in}}%
\pgfpathlineto{\pgfqpoint{5.085432in}{10.499379in}}%
\pgfpathlineto{\pgfqpoint{5.208655in}{10.495481in}}%
\pgfpathlineto{\pgfqpoint{5.331878in}{10.491684in}}%
\pgfpathlineto{\pgfqpoint{5.455101in}{10.487982in}}%
\pgfpathlineto{\pgfqpoint{5.578324in}{10.484369in}}%
\pgfpathlineto{\pgfqpoint{5.701547in}{10.480840in}}%
\pgfpathlineto{\pgfqpoint{5.824770in}{10.477391in}}%
\pgfpathlineto{\pgfqpoint{5.947993in}{10.474019in}}%
\pgfpathlineto{\pgfqpoint{6.071216in}{10.470718in}}%
\pgfpathlineto{\pgfqpoint{6.194439in}{10.467487in}}%
\pgfpathlineto{\pgfqpoint{6.317662in}{10.464322in}}%
\pgfpathlineto{\pgfqpoint{6.440885in}{10.461220in}}%
\pgfpathlineto{\pgfqpoint{6.564108in}{10.458179in}}%
\pgfpathlineto{\pgfqpoint{6.687330in}{10.455195in}}%
\pgfpathlineto{\pgfqpoint{6.810553in}{10.452266in}}%
\pgfpathlineto{\pgfqpoint{6.933776in}{10.449392in}}%
\pgfpathlineto{\pgfqpoint{7.056999in}{10.446569in}}%
\pgfpathlineto{\pgfqpoint{7.180222in}{10.443795in}}%
\pgfpathlineto{\pgfqpoint{7.365057in}{10.441068in}}%
\pgfpathlineto{\pgfqpoint{7.365057in}{10.917773in}}%
\pgfpathlineto{\pgfqpoint{7.365057in}{10.917773in}}%
\pgfpathlineto{\pgfqpoint{7.180222in}{10.915046in}}%
\pgfpathlineto{\pgfqpoint{7.056999in}{10.912272in}}%
\pgfpathlineto{\pgfqpoint{6.933776in}{10.909449in}}%
\pgfpathlineto{\pgfqpoint{6.810553in}{10.906574in}}%
\pgfpathlineto{\pgfqpoint{6.687330in}{10.903646in}}%
\pgfpathlineto{\pgfqpoint{6.564108in}{10.900662in}}%
\pgfpathlineto{\pgfqpoint{6.440885in}{10.897621in}}%
\pgfpathlineto{\pgfqpoint{6.317662in}{10.894519in}}%
\pgfpathlineto{\pgfqpoint{6.194439in}{10.891354in}}%
\pgfpathlineto{\pgfqpoint{6.071216in}{10.888122in}}%
\pgfpathlineto{\pgfqpoint{5.947993in}{10.884822in}}%
\pgfpathlineto{\pgfqpoint{5.824770in}{10.881449in}}%
\pgfpathlineto{\pgfqpoint{5.701547in}{10.878001in}}%
\pgfpathlineto{\pgfqpoint{5.578324in}{10.874472in}}%
\pgfpathlineto{\pgfqpoint{5.455101in}{10.870859in}}%
\pgfpathlineto{\pgfqpoint{5.331878in}{10.867157in}}%
\pgfpathlineto{\pgfqpoint{5.208655in}{10.863360in}}%
\pgfpathlineto{\pgfqpoint{5.085432in}{10.859461in}}%
\pgfpathlineto{\pgfqpoint{4.962209in}{10.855455in}}%
\pgfpathlineto{\pgfqpoint{4.838986in}{10.851333in}}%
\pgfpathlineto{\pgfqpoint{4.715763in}{10.847088in}}%
\pgfpathlineto{\pgfqpoint{4.592540in}{10.842709in}}%
\pgfpathlineto{\pgfqpoint{4.469317in}{10.838185in}}%
\pgfpathlineto{\pgfqpoint{4.346094in}{10.833504in}}%
\pgfpathlineto{\pgfqpoint{4.222871in}{10.828650in}}%
\pgfpathlineto{\pgfqpoint{4.099648in}{10.823606in}}%
\pgfpathlineto{\pgfqpoint{3.976425in}{10.818349in}}%
\pgfpathlineto{\pgfqpoint{3.853202in}{10.812854in}}%
\pgfpathlineto{\pgfqpoint{3.729979in}{10.807091in}}%
\pgfpathlineto{\pgfqpoint{3.606756in}{10.801022in}}%
\pgfpathlineto{\pgfqpoint{3.483533in}{10.794597in}}%
\pgfpathlineto{\pgfqpoint{3.360310in}{10.787753in}}%
\pgfpathlineto{\pgfqpoint{3.237087in}{10.780406in}}%
\pgfpathlineto{\pgfqpoint{3.113864in}{10.772436in}}%
\pgfpathlineto{\pgfqpoint{2.990641in}{10.763667in}}%
\pgfpathlineto{\pgfqpoint{2.867418in}{10.753817in}}%
\pgfpathlineto{\pgfqpoint{2.744195in}{10.742380in}}%
\pgfpathlineto{\pgfqpoint{2.620972in}{10.728254in}}%
\pgfpathlineto{\pgfqpoint{2.436138in}{10.707662in}}%
\pgfpathclose%
\pgfusepath{stroke,fill}%
\end{pgfscope}%
\begin{pgfscope}%
\pgfpathrectangle{\pgfqpoint{2.125000in}{10.401163in}}{\pgfqpoint{5.489583in}{0.877907in}}%
\pgfusepath{clip}%
\pgfsetbuttcap%
\pgfsetroundjoin%
\pgfsetlinewidth{1.505625pt}%
\definecolor{currentstroke}{rgb}{0.000000,0.000000,0.000000}%
\pgfsetstrokecolor{currentstroke}%
\pgfsetdash{}{0pt}%
\pgfpathmoveto{\pgfqpoint{2.374527in}{10.679420in}}%
\pgfpathlineto{\pgfqpoint{2.374527in}{11.239165in}}%
\pgfusepath{stroke}%
\end{pgfscope}%
\begin{pgfscope}%
\pgfpathrectangle{\pgfqpoint{2.125000in}{10.401163in}}{\pgfqpoint{5.489583in}{0.877907in}}%
\pgfusepath{clip}%
\pgfsetbuttcap%
\pgfsetroundjoin%
\pgfsetlinewidth{1.505625pt}%
\definecolor{currentstroke}{rgb}{0.000000,0.000000,0.000000}%
\pgfsetstrokecolor{currentstroke}%
\pgfsetdash{}{0pt}%
\pgfpathmoveto{\pgfqpoint{2.497749in}{10.679420in}}%
\pgfpathlineto{\pgfqpoint{2.497749in}{11.237753in}}%
\pgfusepath{stroke}%
\end{pgfscope}%
\begin{pgfscope}%
\pgfpathrectangle{\pgfqpoint{2.125000in}{10.401163in}}{\pgfqpoint{5.489583in}{0.877907in}}%
\pgfusepath{clip}%
\pgfsetbuttcap%
\pgfsetroundjoin%
\pgfsetlinewidth{1.505625pt}%
\definecolor{currentstroke}{rgb}{0.000000,0.000000,0.000000}%
\pgfsetstrokecolor{currentstroke}%
\pgfsetdash{}{0pt}%
\pgfpathmoveto{\pgfqpoint{2.620972in}{10.679420in}}%
\pgfpathlineto{\pgfqpoint{2.620972in}{11.236331in}}%
\pgfusepath{stroke}%
\end{pgfscope}%
\begin{pgfscope}%
\pgfpathrectangle{\pgfqpoint{2.125000in}{10.401163in}}{\pgfqpoint{5.489583in}{0.877907in}}%
\pgfusepath{clip}%
\pgfsetbuttcap%
\pgfsetroundjoin%
\pgfsetlinewidth{1.505625pt}%
\definecolor{currentstroke}{rgb}{0.000000,0.000000,0.000000}%
\pgfsetstrokecolor{currentstroke}%
\pgfsetdash{}{0pt}%
\pgfpathmoveto{\pgfqpoint{2.744195in}{10.679420in}}%
\pgfpathlineto{\pgfqpoint{2.744195in}{11.234897in}}%
\pgfusepath{stroke}%
\end{pgfscope}%
\begin{pgfscope}%
\pgfpathrectangle{\pgfqpoint{2.125000in}{10.401163in}}{\pgfqpoint{5.489583in}{0.877907in}}%
\pgfusepath{clip}%
\pgfsetbuttcap%
\pgfsetroundjoin%
\pgfsetlinewidth{1.505625pt}%
\definecolor{currentstroke}{rgb}{0.000000,0.000000,0.000000}%
\pgfsetstrokecolor{currentstroke}%
\pgfsetdash{}{0pt}%
\pgfpathmoveto{\pgfqpoint{2.867418in}{10.679420in}}%
\pgfpathlineto{\pgfqpoint{2.867418in}{11.233421in}}%
\pgfusepath{stroke}%
\end{pgfscope}%
\begin{pgfscope}%
\pgfpathrectangle{\pgfqpoint{2.125000in}{10.401163in}}{\pgfqpoint{5.489583in}{0.877907in}}%
\pgfusepath{clip}%
\pgfsetbuttcap%
\pgfsetroundjoin%
\pgfsetlinewidth{1.505625pt}%
\definecolor{currentstroke}{rgb}{0.000000,0.000000,0.000000}%
\pgfsetstrokecolor{currentstroke}%
\pgfsetdash{}{0pt}%
\pgfpathmoveto{\pgfqpoint{2.990641in}{10.679420in}}%
\pgfpathlineto{\pgfqpoint{2.990641in}{11.231974in}}%
\pgfusepath{stroke}%
\end{pgfscope}%
\begin{pgfscope}%
\pgfpathrectangle{\pgfqpoint{2.125000in}{10.401163in}}{\pgfqpoint{5.489583in}{0.877907in}}%
\pgfusepath{clip}%
\pgfsetbuttcap%
\pgfsetroundjoin%
\pgfsetlinewidth{1.505625pt}%
\definecolor{currentstroke}{rgb}{0.000000,0.000000,0.000000}%
\pgfsetstrokecolor{currentstroke}%
\pgfsetdash{}{0pt}%
\pgfpathmoveto{\pgfqpoint{3.113864in}{10.679420in}}%
\pgfpathlineto{\pgfqpoint{3.113864in}{11.230498in}}%
\pgfusepath{stroke}%
\end{pgfscope}%
\begin{pgfscope}%
\pgfpathrectangle{\pgfqpoint{2.125000in}{10.401163in}}{\pgfqpoint{5.489583in}{0.877907in}}%
\pgfusepath{clip}%
\pgfsetbuttcap%
\pgfsetroundjoin%
\pgfsetlinewidth{1.505625pt}%
\definecolor{currentstroke}{rgb}{0.000000,0.000000,0.000000}%
\pgfsetstrokecolor{currentstroke}%
\pgfsetdash{}{0pt}%
\pgfpathmoveto{\pgfqpoint{3.237087in}{10.679420in}}%
\pgfpathlineto{\pgfqpoint{3.237087in}{11.229019in}}%
\pgfusepath{stroke}%
\end{pgfscope}%
\begin{pgfscope}%
\pgfpathrectangle{\pgfqpoint{2.125000in}{10.401163in}}{\pgfqpoint{5.489583in}{0.877907in}}%
\pgfusepath{clip}%
\pgfsetbuttcap%
\pgfsetroundjoin%
\pgfsetlinewidth{1.505625pt}%
\definecolor{currentstroke}{rgb}{0.000000,0.000000,0.000000}%
\pgfsetstrokecolor{currentstroke}%
\pgfsetdash{}{0pt}%
\pgfpathmoveto{\pgfqpoint{3.360310in}{10.679420in}}%
\pgfpathlineto{\pgfqpoint{3.360310in}{11.227536in}}%
\pgfusepath{stroke}%
\end{pgfscope}%
\begin{pgfscope}%
\pgfpathrectangle{\pgfqpoint{2.125000in}{10.401163in}}{\pgfqpoint{5.489583in}{0.877907in}}%
\pgfusepath{clip}%
\pgfsetbuttcap%
\pgfsetroundjoin%
\pgfsetlinewidth{1.505625pt}%
\definecolor{currentstroke}{rgb}{0.000000,0.000000,0.000000}%
\pgfsetstrokecolor{currentstroke}%
\pgfsetdash{}{0pt}%
\pgfpathmoveto{\pgfqpoint{3.483533in}{10.679420in}}%
\pgfpathlineto{\pgfqpoint{3.483533in}{11.226034in}}%
\pgfusepath{stroke}%
\end{pgfscope}%
\begin{pgfscope}%
\pgfpathrectangle{\pgfqpoint{2.125000in}{10.401163in}}{\pgfqpoint{5.489583in}{0.877907in}}%
\pgfusepath{clip}%
\pgfsetbuttcap%
\pgfsetroundjoin%
\pgfsetlinewidth{1.505625pt}%
\definecolor{currentstroke}{rgb}{0.000000,0.000000,0.000000}%
\pgfsetstrokecolor{currentstroke}%
\pgfsetdash{}{0pt}%
\pgfpathmoveto{\pgfqpoint{3.606756in}{10.679420in}}%
\pgfpathlineto{\pgfqpoint{3.606756in}{11.224535in}}%
\pgfusepath{stroke}%
\end{pgfscope}%
\begin{pgfscope}%
\pgfpathrectangle{\pgfqpoint{2.125000in}{10.401163in}}{\pgfqpoint{5.489583in}{0.877907in}}%
\pgfusepath{clip}%
\pgfsetbuttcap%
\pgfsetroundjoin%
\pgfsetlinewidth{1.505625pt}%
\definecolor{currentstroke}{rgb}{0.000000,0.000000,0.000000}%
\pgfsetstrokecolor{currentstroke}%
\pgfsetdash{}{0pt}%
\pgfpathmoveto{\pgfqpoint{3.729979in}{10.679420in}}%
\pgfpathlineto{\pgfqpoint{3.729979in}{11.223062in}}%
\pgfusepath{stroke}%
\end{pgfscope}%
\begin{pgfscope}%
\pgfpathrectangle{\pgfqpoint{2.125000in}{10.401163in}}{\pgfqpoint{5.489583in}{0.877907in}}%
\pgfusepath{clip}%
\pgfsetbuttcap%
\pgfsetroundjoin%
\pgfsetlinewidth{1.505625pt}%
\definecolor{currentstroke}{rgb}{0.000000,0.000000,0.000000}%
\pgfsetstrokecolor{currentstroke}%
\pgfsetdash{}{0pt}%
\pgfpathmoveto{\pgfqpoint{3.853202in}{10.679420in}}%
\pgfpathlineto{\pgfqpoint{3.853202in}{11.221579in}}%
\pgfusepath{stroke}%
\end{pgfscope}%
\begin{pgfscope}%
\pgfpathrectangle{\pgfqpoint{2.125000in}{10.401163in}}{\pgfqpoint{5.489583in}{0.877907in}}%
\pgfusepath{clip}%
\pgfsetbuttcap%
\pgfsetroundjoin%
\pgfsetlinewidth{1.505625pt}%
\definecolor{currentstroke}{rgb}{0.000000,0.000000,0.000000}%
\pgfsetstrokecolor{currentstroke}%
\pgfsetdash{}{0pt}%
\pgfpathmoveto{\pgfqpoint{3.976425in}{10.679420in}}%
\pgfpathlineto{\pgfqpoint{3.976425in}{11.220089in}}%
\pgfusepath{stroke}%
\end{pgfscope}%
\begin{pgfscope}%
\pgfpathrectangle{\pgfqpoint{2.125000in}{10.401163in}}{\pgfqpoint{5.489583in}{0.877907in}}%
\pgfusepath{clip}%
\pgfsetbuttcap%
\pgfsetroundjoin%
\pgfsetlinewidth{1.505625pt}%
\definecolor{currentstroke}{rgb}{0.000000,0.000000,0.000000}%
\pgfsetstrokecolor{currentstroke}%
\pgfsetdash{}{0pt}%
\pgfpathmoveto{\pgfqpoint{4.099648in}{10.679420in}}%
\pgfpathlineto{\pgfqpoint{4.099648in}{11.218607in}}%
\pgfusepath{stroke}%
\end{pgfscope}%
\begin{pgfscope}%
\pgfpathrectangle{\pgfqpoint{2.125000in}{10.401163in}}{\pgfqpoint{5.489583in}{0.877907in}}%
\pgfusepath{clip}%
\pgfsetbuttcap%
\pgfsetroundjoin%
\pgfsetlinewidth{1.505625pt}%
\definecolor{currentstroke}{rgb}{0.000000,0.000000,0.000000}%
\pgfsetstrokecolor{currentstroke}%
\pgfsetdash{}{0pt}%
\pgfpathmoveto{\pgfqpoint{4.222871in}{10.679420in}}%
\pgfpathlineto{\pgfqpoint{4.222871in}{11.217158in}}%
\pgfusepath{stroke}%
\end{pgfscope}%
\begin{pgfscope}%
\pgfpathrectangle{\pgfqpoint{2.125000in}{10.401163in}}{\pgfqpoint{5.489583in}{0.877907in}}%
\pgfusepath{clip}%
\pgfsetbuttcap%
\pgfsetroundjoin%
\pgfsetlinewidth{1.505625pt}%
\definecolor{currentstroke}{rgb}{0.000000,0.000000,0.000000}%
\pgfsetstrokecolor{currentstroke}%
\pgfsetdash{}{0pt}%
\pgfpathmoveto{\pgfqpoint{4.346094in}{10.679420in}}%
\pgfpathlineto{\pgfqpoint{4.346094in}{11.215743in}}%
\pgfusepath{stroke}%
\end{pgfscope}%
\begin{pgfscope}%
\pgfpathrectangle{\pgfqpoint{2.125000in}{10.401163in}}{\pgfqpoint{5.489583in}{0.877907in}}%
\pgfusepath{clip}%
\pgfsetbuttcap%
\pgfsetroundjoin%
\pgfsetlinewidth{1.505625pt}%
\definecolor{currentstroke}{rgb}{0.000000,0.000000,0.000000}%
\pgfsetstrokecolor{currentstroke}%
\pgfsetdash{}{0pt}%
\pgfpathmoveto{\pgfqpoint{4.469317in}{10.679420in}}%
\pgfpathlineto{\pgfqpoint{4.469317in}{11.214345in}}%
\pgfusepath{stroke}%
\end{pgfscope}%
\begin{pgfscope}%
\pgfpathrectangle{\pgfqpoint{2.125000in}{10.401163in}}{\pgfqpoint{5.489583in}{0.877907in}}%
\pgfusepath{clip}%
\pgfsetbuttcap%
\pgfsetroundjoin%
\pgfsetlinewidth{1.505625pt}%
\definecolor{currentstroke}{rgb}{0.000000,0.000000,0.000000}%
\pgfsetstrokecolor{currentstroke}%
\pgfsetdash{}{0pt}%
\pgfpathmoveto{\pgfqpoint{4.592540in}{10.679420in}}%
\pgfpathlineto{\pgfqpoint{4.592540in}{11.212926in}}%
\pgfusepath{stroke}%
\end{pgfscope}%
\begin{pgfscope}%
\pgfpathrectangle{\pgfqpoint{2.125000in}{10.401163in}}{\pgfqpoint{5.489583in}{0.877907in}}%
\pgfusepath{clip}%
\pgfsetbuttcap%
\pgfsetroundjoin%
\pgfsetlinewidth{1.505625pt}%
\definecolor{currentstroke}{rgb}{0.000000,0.000000,0.000000}%
\pgfsetstrokecolor{currentstroke}%
\pgfsetdash{}{0pt}%
\pgfpathmoveto{\pgfqpoint{4.715763in}{10.679420in}}%
\pgfpathlineto{\pgfqpoint{4.715763in}{11.211537in}}%
\pgfusepath{stroke}%
\end{pgfscope}%
\begin{pgfscope}%
\pgfpathrectangle{\pgfqpoint{2.125000in}{10.401163in}}{\pgfqpoint{5.489583in}{0.877907in}}%
\pgfusepath{clip}%
\pgfsetbuttcap%
\pgfsetroundjoin%
\pgfsetlinewidth{1.505625pt}%
\definecolor{currentstroke}{rgb}{0.000000,0.000000,0.000000}%
\pgfsetstrokecolor{currentstroke}%
\pgfsetdash{}{0pt}%
\pgfpathmoveto{\pgfqpoint{4.838986in}{10.679420in}}%
\pgfpathlineto{\pgfqpoint{4.838986in}{11.210151in}}%
\pgfusepath{stroke}%
\end{pgfscope}%
\begin{pgfscope}%
\pgfpathrectangle{\pgfqpoint{2.125000in}{10.401163in}}{\pgfqpoint{5.489583in}{0.877907in}}%
\pgfusepath{clip}%
\pgfsetbuttcap%
\pgfsetroundjoin%
\pgfsetlinewidth{1.505625pt}%
\definecolor{currentstroke}{rgb}{0.000000,0.000000,0.000000}%
\pgfsetstrokecolor{currentstroke}%
\pgfsetdash{}{0pt}%
\pgfpathmoveto{\pgfqpoint{4.962209in}{10.679420in}}%
\pgfpathlineto{\pgfqpoint{4.962209in}{11.208759in}}%
\pgfusepath{stroke}%
\end{pgfscope}%
\begin{pgfscope}%
\pgfpathrectangle{\pgfqpoint{2.125000in}{10.401163in}}{\pgfqpoint{5.489583in}{0.877907in}}%
\pgfusepath{clip}%
\pgfsetbuttcap%
\pgfsetroundjoin%
\pgfsetlinewidth{1.505625pt}%
\definecolor{currentstroke}{rgb}{0.000000,0.000000,0.000000}%
\pgfsetstrokecolor{currentstroke}%
\pgfsetdash{}{0pt}%
\pgfpathmoveto{\pgfqpoint{5.085432in}{10.679420in}}%
\pgfpathlineto{\pgfqpoint{5.085432in}{11.207344in}}%
\pgfusepath{stroke}%
\end{pgfscope}%
\begin{pgfscope}%
\pgfpathrectangle{\pgfqpoint{2.125000in}{10.401163in}}{\pgfqpoint{5.489583in}{0.877907in}}%
\pgfusepath{clip}%
\pgfsetbuttcap%
\pgfsetroundjoin%
\pgfsetlinewidth{1.505625pt}%
\definecolor{currentstroke}{rgb}{0.000000,0.000000,0.000000}%
\pgfsetstrokecolor{currentstroke}%
\pgfsetdash{}{0pt}%
\pgfpathmoveto{\pgfqpoint{5.208655in}{10.679420in}}%
\pgfpathlineto{\pgfqpoint{5.208655in}{11.205919in}}%
\pgfusepath{stroke}%
\end{pgfscope}%
\begin{pgfscope}%
\pgfpathrectangle{\pgfqpoint{2.125000in}{10.401163in}}{\pgfqpoint{5.489583in}{0.877907in}}%
\pgfusepath{clip}%
\pgfsetbuttcap%
\pgfsetroundjoin%
\pgfsetlinewidth{1.505625pt}%
\definecolor{currentstroke}{rgb}{0.000000,0.000000,0.000000}%
\pgfsetstrokecolor{currentstroke}%
\pgfsetdash{}{0pt}%
\pgfpathmoveto{\pgfqpoint{5.331878in}{10.679420in}}%
\pgfpathlineto{\pgfqpoint{5.331878in}{11.204502in}}%
\pgfusepath{stroke}%
\end{pgfscope}%
\begin{pgfscope}%
\pgfpathrectangle{\pgfqpoint{2.125000in}{10.401163in}}{\pgfqpoint{5.489583in}{0.877907in}}%
\pgfusepath{clip}%
\pgfsetbuttcap%
\pgfsetroundjoin%
\pgfsetlinewidth{1.505625pt}%
\definecolor{currentstroke}{rgb}{0.000000,0.000000,0.000000}%
\pgfsetstrokecolor{currentstroke}%
\pgfsetdash{}{0pt}%
\pgfpathmoveto{\pgfqpoint{5.455101in}{10.679420in}}%
\pgfpathlineto{\pgfqpoint{5.455101in}{11.203108in}}%
\pgfusepath{stroke}%
\end{pgfscope}%
\begin{pgfscope}%
\pgfpathrectangle{\pgfqpoint{2.125000in}{10.401163in}}{\pgfqpoint{5.489583in}{0.877907in}}%
\pgfusepath{clip}%
\pgfsetbuttcap%
\pgfsetroundjoin%
\pgfsetlinewidth{1.505625pt}%
\definecolor{currentstroke}{rgb}{0.000000,0.000000,0.000000}%
\pgfsetstrokecolor{currentstroke}%
\pgfsetdash{}{0pt}%
\pgfpathmoveto{\pgfqpoint{5.578324in}{10.679420in}}%
\pgfpathlineto{\pgfqpoint{5.578324in}{11.201736in}}%
\pgfusepath{stroke}%
\end{pgfscope}%
\begin{pgfscope}%
\pgfpathrectangle{\pgfqpoint{2.125000in}{10.401163in}}{\pgfqpoint{5.489583in}{0.877907in}}%
\pgfusepath{clip}%
\pgfsetbuttcap%
\pgfsetroundjoin%
\pgfsetlinewidth{1.505625pt}%
\definecolor{currentstroke}{rgb}{0.000000,0.000000,0.000000}%
\pgfsetstrokecolor{currentstroke}%
\pgfsetdash{}{0pt}%
\pgfpathmoveto{\pgfqpoint{5.701547in}{10.679420in}}%
\pgfpathlineto{\pgfqpoint{5.701547in}{11.200343in}}%
\pgfusepath{stroke}%
\end{pgfscope}%
\begin{pgfscope}%
\pgfpathrectangle{\pgfqpoint{2.125000in}{10.401163in}}{\pgfqpoint{5.489583in}{0.877907in}}%
\pgfusepath{clip}%
\pgfsetbuttcap%
\pgfsetroundjoin%
\pgfsetlinewidth{1.505625pt}%
\definecolor{currentstroke}{rgb}{0.000000,0.000000,0.000000}%
\pgfsetstrokecolor{currentstroke}%
\pgfsetdash{}{0pt}%
\pgfpathmoveto{\pgfqpoint{5.824770in}{10.679420in}}%
\pgfpathlineto{\pgfqpoint{5.824770in}{11.198933in}}%
\pgfusepath{stroke}%
\end{pgfscope}%
\begin{pgfscope}%
\pgfpathrectangle{\pgfqpoint{2.125000in}{10.401163in}}{\pgfqpoint{5.489583in}{0.877907in}}%
\pgfusepath{clip}%
\pgfsetbuttcap%
\pgfsetroundjoin%
\pgfsetlinewidth{1.505625pt}%
\definecolor{currentstroke}{rgb}{0.000000,0.000000,0.000000}%
\pgfsetstrokecolor{currentstroke}%
\pgfsetdash{}{0pt}%
\pgfpathmoveto{\pgfqpoint{5.947993in}{10.679420in}}%
\pgfpathlineto{\pgfqpoint{5.947993in}{11.197525in}}%
\pgfusepath{stroke}%
\end{pgfscope}%
\begin{pgfscope}%
\pgfpathrectangle{\pgfqpoint{2.125000in}{10.401163in}}{\pgfqpoint{5.489583in}{0.877907in}}%
\pgfusepath{clip}%
\pgfsetbuttcap%
\pgfsetroundjoin%
\pgfsetlinewidth{1.505625pt}%
\definecolor{currentstroke}{rgb}{0.000000,0.000000,0.000000}%
\pgfsetstrokecolor{currentstroke}%
\pgfsetdash{}{0pt}%
\pgfpathmoveto{\pgfqpoint{6.071216in}{10.679420in}}%
\pgfpathlineto{\pgfqpoint{6.071216in}{11.196102in}}%
\pgfusepath{stroke}%
\end{pgfscope}%
\begin{pgfscope}%
\pgfpathrectangle{\pgfqpoint{2.125000in}{10.401163in}}{\pgfqpoint{5.489583in}{0.877907in}}%
\pgfusepath{clip}%
\pgfsetbuttcap%
\pgfsetroundjoin%
\pgfsetlinewidth{1.505625pt}%
\definecolor{currentstroke}{rgb}{0.000000,0.000000,0.000000}%
\pgfsetstrokecolor{currentstroke}%
\pgfsetdash{}{0pt}%
\pgfpathmoveto{\pgfqpoint{6.194439in}{10.679420in}}%
\pgfpathlineto{\pgfqpoint{6.194439in}{11.194646in}}%
\pgfusepath{stroke}%
\end{pgfscope}%
\begin{pgfscope}%
\pgfpathrectangle{\pgfqpoint{2.125000in}{10.401163in}}{\pgfqpoint{5.489583in}{0.877907in}}%
\pgfusepath{clip}%
\pgfsetbuttcap%
\pgfsetroundjoin%
\pgfsetlinewidth{1.505625pt}%
\definecolor{currentstroke}{rgb}{0.000000,0.000000,0.000000}%
\pgfsetstrokecolor{currentstroke}%
\pgfsetdash{}{0pt}%
\pgfpathmoveto{\pgfqpoint{6.317662in}{10.679420in}}%
\pgfpathlineto{\pgfqpoint{6.317662in}{11.193211in}}%
\pgfusepath{stroke}%
\end{pgfscope}%
\begin{pgfscope}%
\pgfpathrectangle{\pgfqpoint{2.125000in}{10.401163in}}{\pgfqpoint{5.489583in}{0.877907in}}%
\pgfusepath{clip}%
\pgfsetbuttcap%
\pgfsetroundjoin%
\pgfsetlinewidth{1.505625pt}%
\definecolor{currentstroke}{rgb}{0.000000,0.000000,0.000000}%
\pgfsetstrokecolor{currentstroke}%
\pgfsetdash{}{0pt}%
\pgfpathmoveto{\pgfqpoint{6.440885in}{10.679420in}}%
\pgfpathlineto{\pgfqpoint{6.440885in}{11.191786in}}%
\pgfusepath{stroke}%
\end{pgfscope}%
\begin{pgfscope}%
\pgfpathrectangle{\pgfqpoint{2.125000in}{10.401163in}}{\pgfqpoint{5.489583in}{0.877907in}}%
\pgfusepath{clip}%
\pgfsetbuttcap%
\pgfsetroundjoin%
\pgfsetlinewidth{1.505625pt}%
\definecolor{currentstroke}{rgb}{0.000000,0.000000,0.000000}%
\pgfsetstrokecolor{currentstroke}%
\pgfsetdash{}{0pt}%
\pgfpathmoveto{\pgfqpoint{6.564108in}{10.679420in}}%
\pgfpathlineto{\pgfqpoint{6.564108in}{11.190375in}}%
\pgfusepath{stroke}%
\end{pgfscope}%
\begin{pgfscope}%
\pgfpathrectangle{\pgfqpoint{2.125000in}{10.401163in}}{\pgfqpoint{5.489583in}{0.877907in}}%
\pgfusepath{clip}%
\pgfsetbuttcap%
\pgfsetroundjoin%
\pgfsetlinewidth{1.505625pt}%
\definecolor{currentstroke}{rgb}{0.000000,0.000000,0.000000}%
\pgfsetstrokecolor{currentstroke}%
\pgfsetdash{}{0pt}%
\pgfpathmoveto{\pgfqpoint{6.687330in}{10.679420in}}%
\pgfpathlineto{\pgfqpoint{6.687330in}{11.188930in}}%
\pgfusepath{stroke}%
\end{pgfscope}%
\begin{pgfscope}%
\pgfpathrectangle{\pgfqpoint{2.125000in}{10.401163in}}{\pgfqpoint{5.489583in}{0.877907in}}%
\pgfusepath{clip}%
\pgfsetbuttcap%
\pgfsetroundjoin%
\pgfsetlinewidth{1.505625pt}%
\definecolor{currentstroke}{rgb}{0.000000,0.000000,0.000000}%
\pgfsetstrokecolor{currentstroke}%
\pgfsetdash{}{0pt}%
\pgfpathmoveto{\pgfqpoint{6.810553in}{10.679420in}}%
\pgfpathlineto{\pgfqpoint{6.810553in}{11.187480in}}%
\pgfusepath{stroke}%
\end{pgfscope}%
\begin{pgfscope}%
\pgfpathrectangle{\pgfqpoint{2.125000in}{10.401163in}}{\pgfqpoint{5.489583in}{0.877907in}}%
\pgfusepath{clip}%
\pgfsetbuttcap%
\pgfsetroundjoin%
\pgfsetlinewidth{1.505625pt}%
\definecolor{currentstroke}{rgb}{0.000000,0.000000,0.000000}%
\pgfsetstrokecolor{currentstroke}%
\pgfsetdash{}{0pt}%
\pgfpathmoveto{\pgfqpoint{6.933776in}{10.679420in}}%
\pgfpathlineto{\pgfqpoint{6.933776in}{11.186056in}}%
\pgfusepath{stroke}%
\end{pgfscope}%
\begin{pgfscope}%
\pgfpathrectangle{\pgfqpoint{2.125000in}{10.401163in}}{\pgfqpoint{5.489583in}{0.877907in}}%
\pgfusepath{clip}%
\pgfsetbuttcap%
\pgfsetroundjoin%
\pgfsetlinewidth{1.505625pt}%
\definecolor{currentstroke}{rgb}{0.000000,0.000000,0.000000}%
\pgfsetstrokecolor{currentstroke}%
\pgfsetdash{}{0pt}%
\pgfpathmoveto{\pgfqpoint{7.056999in}{10.679420in}}%
\pgfpathlineto{\pgfqpoint{7.056999in}{11.184643in}}%
\pgfusepath{stroke}%
\end{pgfscope}%
\begin{pgfscope}%
\pgfpathrectangle{\pgfqpoint{2.125000in}{10.401163in}}{\pgfqpoint{5.489583in}{0.877907in}}%
\pgfusepath{clip}%
\pgfsetbuttcap%
\pgfsetroundjoin%
\pgfsetlinewidth{1.505625pt}%
\definecolor{currentstroke}{rgb}{0.000000,0.000000,0.000000}%
\pgfsetstrokecolor{currentstroke}%
\pgfsetdash{}{0pt}%
\pgfpathmoveto{\pgfqpoint{7.180222in}{10.679420in}}%
\pgfpathlineto{\pgfqpoint{7.180222in}{11.183260in}}%
\pgfusepath{stroke}%
\end{pgfscope}%
\begin{pgfscope}%
\pgfpathrectangle{\pgfqpoint{2.125000in}{10.401163in}}{\pgfqpoint{5.489583in}{0.877907in}}%
\pgfusepath{clip}%
\pgfsetbuttcap%
\pgfsetroundjoin%
\pgfsetlinewidth{1.505625pt}%
\definecolor{currentstroke}{rgb}{0.000000,0.000000,0.000000}%
\pgfsetstrokecolor{currentstroke}%
\pgfsetdash{}{0pt}%
\pgfpathmoveto{\pgfqpoint{7.303445in}{10.679420in}}%
\pgfpathlineto{\pgfqpoint{7.303445in}{11.181878in}}%
\pgfusepath{stroke}%
\end{pgfscope}%
\begin{pgfscope}%
\pgfpathrectangle{\pgfqpoint{2.125000in}{10.401163in}}{\pgfqpoint{5.489583in}{0.877907in}}%
\pgfusepath{clip}%
\pgfsetroundcap%
\pgfsetroundjoin%
\pgfsetlinewidth{1.505625pt}%
\definecolor{currentstroke}{rgb}{0.121569,0.466667,0.705882}%
\pgfsetstrokecolor{currentstroke}%
\pgfsetdash{}{0pt}%
\pgfpathmoveto{\pgfqpoint{2.125000in}{10.679420in}}%
\pgfpathlineto{\pgfqpoint{7.614583in}{10.679420in}}%
\pgfusepath{stroke}%
\end{pgfscope}%
\begin{pgfscope}%
\pgfpathrectangle{\pgfqpoint{2.125000in}{10.401163in}}{\pgfqpoint{5.489583in}{0.877907in}}%
\pgfusepath{clip}%
\pgfsetbuttcap%
\pgfsetroundjoin%
\definecolor{currentfill}{rgb}{0.121569,0.466667,0.705882}%
\pgfsetfillcolor{currentfill}%
\pgfsetlinewidth{1.003750pt}%
\definecolor{currentstroke}{rgb}{0.121569,0.466667,0.705882}%
\pgfsetstrokecolor{currentstroke}%
\pgfsetdash{}{0pt}%
\pgfsys@defobject{currentmarker}{\pgfqpoint{-0.034722in}{-0.034722in}}{\pgfqpoint{0.034722in}{0.034722in}}{%
\pgfpathmoveto{\pgfqpoint{0.000000in}{-0.034722in}}%
\pgfpathcurveto{\pgfqpoint{0.009208in}{-0.034722in}}{\pgfqpoint{0.018041in}{-0.031064in}}{\pgfqpoint{0.024552in}{-0.024552in}}%
\pgfpathcurveto{\pgfqpoint{0.031064in}{-0.018041in}}{\pgfqpoint{0.034722in}{-0.009208in}}{\pgfqpoint{0.034722in}{0.000000in}}%
\pgfpathcurveto{\pgfqpoint{0.034722in}{0.009208in}}{\pgfqpoint{0.031064in}{0.018041in}}{\pgfqpoint{0.024552in}{0.024552in}}%
\pgfpathcurveto{\pgfqpoint{0.018041in}{0.031064in}}{\pgfqpoint{0.009208in}{0.034722in}}{\pgfqpoint{0.000000in}{0.034722in}}%
\pgfpathcurveto{\pgfqpoint{-0.009208in}{0.034722in}}{\pgfqpoint{-0.018041in}{0.031064in}}{\pgfqpoint{-0.024552in}{0.024552in}}%
\pgfpathcurveto{\pgfqpoint{-0.031064in}{0.018041in}}{\pgfqpoint{-0.034722in}{0.009208in}}{\pgfqpoint{-0.034722in}{0.000000in}}%
\pgfpathcurveto{\pgfqpoint{-0.034722in}{-0.009208in}}{\pgfqpoint{-0.031064in}{-0.018041in}}{\pgfqpoint{-0.024552in}{-0.024552in}}%
\pgfpathcurveto{\pgfqpoint{-0.018041in}{-0.031064in}}{\pgfqpoint{-0.009208in}{-0.034722in}}{\pgfqpoint{0.000000in}{-0.034722in}}%
\pgfpathclose%
\pgfusepath{stroke,fill}%
}%
\begin{pgfscope}%
\pgfsys@transformshift{2.374527in}{11.239165in}%
\pgfsys@useobject{currentmarker}{}%
\end{pgfscope}%
\begin{pgfscope}%
\pgfsys@transformshift{2.497749in}{11.237753in}%
\pgfsys@useobject{currentmarker}{}%
\end{pgfscope}%
\begin{pgfscope}%
\pgfsys@transformshift{2.620972in}{11.236331in}%
\pgfsys@useobject{currentmarker}{}%
\end{pgfscope}%
\begin{pgfscope}%
\pgfsys@transformshift{2.744195in}{11.234897in}%
\pgfsys@useobject{currentmarker}{}%
\end{pgfscope}%
\begin{pgfscope}%
\pgfsys@transformshift{2.867418in}{11.233421in}%
\pgfsys@useobject{currentmarker}{}%
\end{pgfscope}%
\begin{pgfscope}%
\pgfsys@transformshift{2.990641in}{11.231974in}%
\pgfsys@useobject{currentmarker}{}%
\end{pgfscope}%
\begin{pgfscope}%
\pgfsys@transformshift{3.113864in}{11.230498in}%
\pgfsys@useobject{currentmarker}{}%
\end{pgfscope}%
\begin{pgfscope}%
\pgfsys@transformshift{3.237087in}{11.229019in}%
\pgfsys@useobject{currentmarker}{}%
\end{pgfscope}%
\begin{pgfscope}%
\pgfsys@transformshift{3.360310in}{11.227536in}%
\pgfsys@useobject{currentmarker}{}%
\end{pgfscope}%
\begin{pgfscope}%
\pgfsys@transformshift{3.483533in}{11.226034in}%
\pgfsys@useobject{currentmarker}{}%
\end{pgfscope}%
\begin{pgfscope}%
\pgfsys@transformshift{3.606756in}{11.224535in}%
\pgfsys@useobject{currentmarker}{}%
\end{pgfscope}%
\begin{pgfscope}%
\pgfsys@transformshift{3.729979in}{11.223062in}%
\pgfsys@useobject{currentmarker}{}%
\end{pgfscope}%
\begin{pgfscope}%
\pgfsys@transformshift{3.853202in}{11.221579in}%
\pgfsys@useobject{currentmarker}{}%
\end{pgfscope}%
\begin{pgfscope}%
\pgfsys@transformshift{3.976425in}{11.220089in}%
\pgfsys@useobject{currentmarker}{}%
\end{pgfscope}%
\begin{pgfscope}%
\pgfsys@transformshift{4.099648in}{11.218607in}%
\pgfsys@useobject{currentmarker}{}%
\end{pgfscope}%
\begin{pgfscope}%
\pgfsys@transformshift{4.222871in}{11.217158in}%
\pgfsys@useobject{currentmarker}{}%
\end{pgfscope}%
\begin{pgfscope}%
\pgfsys@transformshift{4.346094in}{11.215743in}%
\pgfsys@useobject{currentmarker}{}%
\end{pgfscope}%
\begin{pgfscope}%
\pgfsys@transformshift{4.469317in}{11.214345in}%
\pgfsys@useobject{currentmarker}{}%
\end{pgfscope}%
\begin{pgfscope}%
\pgfsys@transformshift{4.592540in}{11.212926in}%
\pgfsys@useobject{currentmarker}{}%
\end{pgfscope}%
\begin{pgfscope}%
\pgfsys@transformshift{4.715763in}{11.211537in}%
\pgfsys@useobject{currentmarker}{}%
\end{pgfscope}%
\begin{pgfscope}%
\pgfsys@transformshift{4.838986in}{11.210151in}%
\pgfsys@useobject{currentmarker}{}%
\end{pgfscope}%
\begin{pgfscope}%
\pgfsys@transformshift{4.962209in}{11.208759in}%
\pgfsys@useobject{currentmarker}{}%
\end{pgfscope}%
\begin{pgfscope}%
\pgfsys@transformshift{5.085432in}{11.207344in}%
\pgfsys@useobject{currentmarker}{}%
\end{pgfscope}%
\begin{pgfscope}%
\pgfsys@transformshift{5.208655in}{11.205919in}%
\pgfsys@useobject{currentmarker}{}%
\end{pgfscope}%
\begin{pgfscope}%
\pgfsys@transformshift{5.331878in}{11.204502in}%
\pgfsys@useobject{currentmarker}{}%
\end{pgfscope}%
\begin{pgfscope}%
\pgfsys@transformshift{5.455101in}{11.203108in}%
\pgfsys@useobject{currentmarker}{}%
\end{pgfscope}%
\begin{pgfscope}%
\pgfsys@transformshift{5.578324in}{11.201736in}%
\pgfsys@useobject{currentmarker}{}%
\end{pgfscope}%
\begin{pgfscope}%
\pgfsys@transformshift{5.701547in}{11.200343in}%
\pgfsys@useobject{currentmarker}{}%
\end{pgfscope}%
\begin{pgfscope}%
\pgfsys@transformshift{5.824770in}{11.198933in}%
\pgfsys@useobject{currentmarker}{}%
\end{pgfscope}%
\begin{pgfscope}%
\pgfsys@transformshift{5.947993in}{11.197525in}%
\pgfsys@useobject{currentmarker}{}%
\end{pgfscope}%
\begin{pgfscope}%
\pgfsys@transformshift{6.071216in}{11.196102in}%
\pgfsys@useobject{currentmarker}{}%
\end{pgfscope}%
\begin{pgfscope}%
\pgfsys@transformshift{6.194439in}{11.194646in}%
\pgfsys@useobject{currentmarker}{}%
\end{pgfscope}%
\begin{pgfscope}%
\pgfsys@transformshift{6.317662in}{11.193211in}%
\pgfsys@useobject{currentmarker}{}%
\end{pgfscope}%
\begin{pgfscope}%
\pgfsys@transformshift{6.440885in}{11.191786in}%
\pgfsys@useobject{currentmarker}{}%
\end{pgfscope}%
\begin{pgfscope}%
\pgfsys@transformshift{6.564108in}{11.190375in}%
\pgfsys@useobject{currentmarker}{}%
\end{pgfscope}%
\begin{pgfscope}%
\pgfsys@transformshift{6.687330in}{11.188930in}%
\pgfsys@useobject{currentmarker}{}%
\end{pgfscope}%
\begin{pgfscope}%
\pgfsys@transformshift{6.810553in}{11.187480in}%
\pgfsys@useobject{currentmarker}{}%
\end{pgfscope}%
\begin{pgfscope}%
\pgfsys@transformshift{6.933776in}{11.186056in}%
\pgfsys@useobject{currentmarker}{}%
\end{pgfscope}%
\begin{pgfscope}%
\pgfsys@transformshift{7.056999in}{11.184643in}%
\pgfsys@useobject{currentmarker}{}%
\end{pgfscope}%
\begin{pgfscope}%
\pgfsys@transformshift{7.180222in}{11.183260in}%
\pgfsys@useobject{currentmarker}{}%
\end{pgfscope}%
\begin{pgfscope}%
\pgfsys@transformshift{7.303445in}{11.181878in}%
\pgfsys@useobject{currentmarker}{}%
\end{pgfscope}%
\end{pgfscope}%
\begin{pgfscope}%
\pgfsetrectcap%
\pgfsetmiterjoin%
\pgfsetlinewidth{0.803000pt}%
\definecolor{currentstroke}{rgb}{1.000000,1.000000,1.000000}%
\pgfsetstrokecolor{currentstroke}%
\pgfsetdash{}{0pt}%
\pgfpathmoveto{\pgfqpoint{2.125000in}{10.401163in}}%
\pgfpathlineto{\pgfqpoint{2.125000in}{11.279070in}}%
\pgfusepath{stroke}%
\end{pgfscope}%
\begin{pgfscope}%
\pgfsetrectcap%
\pgfsetmiterjoin%
\pgfsetlinewidth{0.803000pt}%
\definecolor{currentstroke}{rgb}{1.000000,1.000000,1.000000}%
\pgfsetstrokecolor{currentstroke}%
\pgfsetdash{}{0pt}%
\pgfpathmoveto{\pgfqpoint{7.614583in}{10.401163in}}%
\pgfpathlineto{\pgfqpoint{7.614583in}{11.279070in}}%
\pgfusepath{stroke}%
\end{pgfscope}%
\begin{pgfscope}%
\pgfsetrectcap%
\pgfsetmiterjoin%
\pgfsetlinewidth{0.803000pt}%
\definecolor{currentstroke}{rgb}{1.000000,1.000000,1.000000}%
\pgfsetstrokecolor{currentstroke}%
\pgfsetdash{}{0pt}%
\pgfpathmoveto{\pgfqpoint{2.125000in}{10.401163in}}%
\pgfpathlineto{\pgfqpoint{7.614583in}{10.401163in}}%
\pgfusepath{stroke}%
\end{pgfscope}%
\begin{pgfscope}%
\pgfsetrectcap%
\pgfsetmiterjoin%
\pgfsetlinewidth{0.803000pt}%
\definecolor{currentstroke}{rgb}{1.000000,1.000000,1.000000}%
\pgfsetstrokecolor{currentstroke}%
\pgfsetdash{}{0pt}%
\pgfpathmoveto{\pgfqpoint{2.125000in}{11.279070in}}%
\pgfpathlineto{\pgfqpoint{7.614583in}{11.279070in}}%
\pgfusepath{stroke}%
\end{pgfscope}%
\begin{pgfscope}%
\definecolor{textcolor}{rgb}{0.150000,0.150000,0.150000}%
\pgfsetstrokecolor{textcolor}%
\pgfsetfillcolor{textcolor}%
\pgftext[x=4.869792in,y=11.362403in,,base]{\color{textcolor}\rmfamily\fontsize{16.800000}{20.160000}\selectfont Autocorrelation}%
\end{pgfscope}%
\begin{pgfscope}%
\pgfsetbuttcap%
\pgfsetmiterjoin%
\definecolor{currentfill}{rgb}{0.917647,0.917647,0.949020}%
\pgfsetfillcolor{currentfill}%
\pgfsetlinewidth{0.000000pt}%
\definecolor{currentstroke}{rgb}{0.000000,0.000000,0.000000}%
\pgfsetstrokecolor{currentstroke}%
\pgfsetstrokeopacity{0.000000}%
\pgfsetdash{}{0pt}%
\pgfpathmoveto{\pgfqpoint{9.810417in}{10.401163in}}%
\pgfpathlineto{\pgfqpoint{15.300000in}{10.401163in}}%
\pgfpathlineto{\pgfqpoint{15.300000in}{11.279070in}}%
\pgfpathlineto{\pgfqpoint{9.810417in}{11.279070in}}%
\pgfpathclose%
\pgfusepath{fill}%
\end{pgfscope}%
\begin{pgfscope}%
\pgfpathrectangle{\pgfqpoint{9.810417in}{10.401163in}}{\pgfqpoint{5.489583in}{0.877907in}}%
\pgfusepath{clip}%
\pgfsetroundcap%
\pgfsetroundjoin%
\pgfsetlinewidth{0.803000pt}%
\definecolor{currentstroke}{rgb}{1.000000,1.000000,1.000000}%
\pgfsetstrokecolor{currentstroke}%
\pgfsetdash{}{0pt}%
\pgfpathmoveto{\pgfqpoint{10.059943in}{10.401163in}}%
\pgfpathlineto{\pgfqpoint{10.059943in}{11.279070in}}%
\pgfusepath{stroke}%
\end{pgfscope}%
\begin{pgfscope}%
\definecolor{textcolor}{rgb}{0.150000,0.150000,0.150000}%
\pgfsetstrokecolor{textcolor}%
\pgfsetfillcolor{textcolor}%
\pgftext[x=10.059943in,y=10.303941in,,top]{\color{textcolor}\rmfamily\fontsize{14.000000}{16.800000}\selectfont 0}%
\end{pgfscope}%
\begin{pgfscope}%
\pgfpathrectangle{\pgfqpoint{9.810417in}{10.401163in}}{\pgfqpoint{5.489583in}{0.877907in}}%
\pgfusepath{clip}%
\pgfsetroundcap%
\pgfsetroundjoin%
\pgfsetlinewidth{0.803000pt}%
\definecolor{currentstroke}{rgb}{1.000000,1.000000,1.000000}%
\pgfsetstrokecolor{currentstroke}%
\pgfsetdash{}{0pt}%
\pgfpathmoveto{\pgfqpoint{10.676058in}{10.401163in}}%
\pgfpathlineto{\pgfqpoint{10.676058in}{11.279070in}}%
\pgfusepath{stroke}%
\end{pgfscope}%
\begin{pgfscope}%
\definecolor{textcolor}{rgb}{0.150000,0.150000,0.150000}%
\pgfsetstrokecolor{textcolor}%
\pgfsetfillcolor{textcolor}%
\pgftext[x=10.676058in,y=10.303941in,,top]{\color{textcolor}\rmfamily\fontsize{14.000000}{16.800000}\selectfont 5}%
\end{pgfscope}%
\begin{pgfscope}%
\pgfpathrectangle{\pgfqpoint{9.810417in}{10.401163in}}{\pgfqpoint{5.489583in}{0.877907in}}%
\pgfusepath{clip}%
\pgfsetroundcap%
\pgfsetroundjoin%
\pgfsetlinewidth{0.803000pt}%
\definecolor{currentstroke}{rgb}{1.000000,1.000000,1.000000}%
\pgfsetstrokecolor{currentstroke}%
\pgfsetdash{}{0pt}%
\pgfpathmoveto{\pgfqpoint{11.292173in}{10.401163in}}%
\pgfpathlineto{\pgfqpoint{11.292173in}{11.279070in}}%
\pgfusepath{stroke}%
\end{pgfscope}%
\begin{pgfscope}%
\definecolor{textcolor}{rgb}{0.150000,0.150000,0.150000}%
\pgfsetstrokecolor{textcolor}%
\pgfsetfillcolor{textcolor}%
\pgftext[x=11.292173in,y=10.303941in,,top]{\color{textcolor}\rmfamily\fontsize{14.000000}{16.800000}\selectfont 10}%
\end{pgfscope}%
\begin{pgfscope}%
\pgfpathrectangle{\pgfqpoint{9.810417in}{10.401163in}}{\pgfqpoint{5.489583in}{0.877907in}}%
\pgfusepath{clip}%
\pgfsetroundcap%
\pgfsetroundjoin%
\pgfsetlinewidth{0.803000pt}%
\definecolor{currentstroke}{rgb}{1.000000,1.000000,1.000000}%
\pgfsetstrokecolor{currentstroke}%
\pgfsetdash{}{0pt}%
\pgfpathmoveto{\pgfqpoint{11.908288in}{10.401163in}}%
\pgfpathlineto{\pgfqpoint{11.908288in}{11.279070in}}%
\pgfusepath{stroke}%
\end{pgfscope}%
\begin{pgfscope}%
\definecolor{textcolor}{rgb}{0.150000,0.150000,0.150000}%
\pgfsetstrokecolor{textcolor}%
\pgfsetfillcolor{textcolor}%
\pgftext[x=11.908288in,y=10.303941in,,top]{\color{textcolor}\rmfamily\fontsize{14.000000}{16.800000}\selectfont 15}%
\end{pgfscope}%
\begin{pgfscope}%
\pgfpathrectangle{\pgfqpoint{9.810417in}{10.401163in}}{\pgfqpoint{5.489583in}{0.877907in}}%
\pgfusepath{clip}%
\pgfsetroundcap%
\pgfsetroundjoin%
\pgfsetlinewidth{0.803000pt}%
\definecolor{currentstroke}{rgb}{1.000000,1.000000,1.000000}%
\pgfsetstrokecolor{currentstroke}%
\pgfsetdash{}{0pt}%
\pgfpathmoveto{\pgfqpoint{12.524403in}{10.401163in}}%
\pgfpathlineto{\pgfqpoint{12.524403in}{11.279070in}}%
\pgfusepath{stroke}%
\end{pgfscope}%
\begin{pgfscope}%
\definecolor{textcolor}{rgb}{0.150000,0.150000,0.150000}%
\pgfsetstrokecolor{textcolor}%
\pgfsetfillcolor{textcolor}%
\pgftext[x=12.524403in,y=10.303941in,,top]{\color{textcolor}\rmfamily\fontsize{14.000000}{16.800000}\selectfont 20}%
\end{pgfscope}%
\begin{pgfscope}%
\pgfpathrectangle{\pgfqpoint{9.810417in}{10.401163in}}{\pgfqpoint{5.489583in}{0.877907in}}%
\pgfusepath{clip}%
\pgfsetroundcap%
\pgfsetroundjoin%
\pgfsetlinewidth{0.803000pt}%
\definecolor{currentstroke}{rgb}{1.000000,1.000000,1.000000}%
\pgfsetstrokecolor{currentstroke}%
\pgfsetdash{}{0pt}%
\pgfpathmoveto{\pgfqpoint{13.140517in}{10.401163in}}%
\pgfpathlineto{\pgfqpoint{13.140517in}{11.279070in}}%
\pgfusepath{stroke}%
\end{pgfscope}%
\begin{pgfscope}%
\definecolor{textcolor}{rgb}{0.150000,0.150000,0.150000}%
\pgfsetstrokecolor{textcolor}%
\pgfsetfillcolor{textcolor}%
\pgftext[x=13.140517in,y=10.303941in,,top]{\color{textcolor}\rmfamily\fontsize{14.000000}{16.800000}\selectfont 25}%
\end{pgfscope}%
\begin{pgfscope}%
\pgfpathrectangle{\pgfqpoint{9.810417in}{10.401163in}}{\pgfqpoint{5.489583in}{0.877907in}}%
\pgfusepath{clip}%
\pgfsetroundcap%
\pgfsetroundjoin%
\pgfsetlinewidth{0.803000pt}%
\definecolor{currentstroke}{rgb}{1.000000,1.000000,1.000000}%
\pgfsetstrokecolor{currentstroke}%
\pgfsetdash{}{0pt}%
\pgfpathmoveto{\pgfqpoint{13.756632in}{10.401163in}}%
\pgfpathlineto{\pgfqpoint{13.756632in}{11.279070in}}%
\pgfusepath{stroke}%
\end{pgfscope}%
\begin{pgfscope}%
\definecolor{textcolor}{rgb}{0.150000,0.150000,0.150000}%
\pgfsetstrokecolor{textcolor}%
\pgfsetfillcolor{textcolor}%
\pgftext[x=13.756632in,y=10.303941in,,top]{\color{textcolor}\rmfamily\fontsize{14.000000}{16.800000}\selectfont 30}%
\end{pgfscope}%
\begin{pgfscope}%
\pgfpathrectangle{\pgfqpoint{9.810417in}{10.401163in}}{\pgfqpoint{5.489583in}{0.877907in}}%
\pgfusepath{clip}%
\pgfsetroundcap%
\pgfsetroundjoin%
\pgfsetlinewidth{0.803000pt}%
\definecolor{currentstroke}{rgb}{1.000000,1.000000,1.000000}%
\pgfsetstrokecolor{currentstroke}%
\pgfsetdash{}{0pt}%
\pgfpathmoveto{\pgfqpoint{14.372747in}{10.401163in}}%
\pgfpathlineto{\pgfqpoint{14.372747in}{11.279070in}}%
\pgfusepath{stroke}%
\end{pgfscope}%
\begin{pgfscope}%
\definecolor{textcolor}{rgb}{0.150000,0.150000,0.150000}%
\pgfsetstrokecolor{textcolor}%
\pgfsetfillcolor{textcolor}%
\pgftext[x=14.372747in,y=10.303941in,,top]{\color{textcolor}\rmfamily\fontsize{14.000000}{16.800000}\selectfont 35}%
\end{pgfscope}%
\begin{pgfscope}%
\pgfpathrectangle{\pgfqpoint{9.810417in}{10.401163in}}{\pgfqpoint{5.489583in}{0.877907in}}%
\pgfusepath{clip}%
\pgfsetroundcap%
\pgfsetroundjoin%
\pgfsetlinewidth{0.803000pt}%
\definecolor{currentstroke}{rgb}{1.000000,1.000000,1.000000}%
\pgfsetstrokecolor{currentstroke}%
\pgfsetdash{}{0pt}%
\pgfpathmoveto{\pgfqpoint{14.988862in}{10.401163in}}%
\pgfpathlineto{\pgfqpoint{14.988862in}{11.279070in}}%
\pgfusepath{stroke}%
\end{pgfscope}%
\begin{pgfscope}%
\definecolor{textcolor}{rgb}{0.150000,0.150000,0.150000}%
\pgfsetstrokecolor{textcolor}%
\pgfsetfillcolor{textcolor}%
\pgftext[x=14.988862in,y=10.303941in,,top]{\color{textcolor}\rmfamily\fontsize{14.000000}{16.800000}\selectfont 40}%
\end{pgfscope}%
\begin{pgfscope}%
\pgfpathrectangle{\pgfqpoint{9.810417in}{10.401163in}}{\pgfqpoint{5.489583in}{0.877907in}}%
\pgfusepath{clip}%
\pgfsetroundcap%
\pgfsetroundjoin%
\pgfsetlinewidth{0.803000pt}%
\definecolor{currentstroke}{rgb}{1.000000,1.000000,1.000000}%
\pgfsetstrokecolor{currentstroke}%
\pgfsetdash{}{0pt}%
\pgfpathmoveto{\pgfqpoint{9.810417in}{10.479401in}}%
\pgfpathlineto{\pgfqpoint{15.300000in}{10.479401in}}%
\pgfusepath{stroke}%
\end{pgfscope}%
\begin{pgfscope}%
\definecolor{textcolor}{rgb}{0.150000,0.150000,0.150000}%
\pgfsetstrokecolor{textcolor}%
\pgfsetfillcolor{textcolor}%
\pgftext[x=9.589483in,y=10.405535in,left,base]{\color{textcolor}\rmfamily\fontsize{14.000000}{16.800000}\selectfont 0}%
\end{pgfscope}%
\begin{pgfscope}%
\pgfpathrectangle{\pgfqpoint{9.810417in}{10.401163in}}{\pgfqpoint{5.489583in}{0.877907in}}%
\pgfusepath{clip}%
\pgfsetroundcap%
\pgfsetroundjoin%
\pgfsetlinewidth{0.803000pt}%
\definecolor{currentstroke}{rgb}{1.000000,1.000000,1.000000}%
\pgfsetstrokecolor{currentstroke}%
\pgfsetdash{}{0pt}%
\pgfpathmoveto{\pgfqpoint{9.810417in}{11.239165in}}%
\pgfpathlineto{\pgfqpoint{15.300000in}{11.239165in}}%
\pgfusepath{stroke}%
\end{pgfscope}%
\begin{pgfscope}%
\definecolor{textcolor}{rgb}{0.150000,0.150000,0.150000}%
\pgfsetstrokecolor{textcolor}%
\pgfsetfillcolor{textcolor}%
\pgftext[x=9.589483in,y=11.165299in,left,base]{\color{textcolor}\rmfamily\fontsize{14.000000}{16.800000}\selectfont 1}%
\end{pgfscope}%
\begin{pgfscope}%
\pgfpathrectangle{\pgfqpoint{9.810417in}{10.401163in}}{\pgfqpoint{5.489583in}{0.877907in}}%
\pgfusepath{clip}%
\pgfsetbuttcap%
\pgfsetroundjoin%
\definecolor{currentfill}{rgb}{0.121569,0.466667,0.705882}%
\pgfsetfillcolor{currentfill}%
\pgfsetfillopacity{0.250000}%
\pgfsetlinewidth{1.003750pt}%
\definecolor{currentstroke}{rgb}{1.000000,1.000000,1.000000}%
\pgfsetstrokecolor{currentstroke}%
\pgfsetstrokeopacity{0.250000}%
\pgfsetdash{}{0pt}%
\pgfpathmoveto{\pgfqpoint{10.121555in}{10.517735in}}%
\pgfpathlineto{\pgfqpoint{10.121555in}{10.441068in}}%
\pgfpathlineto{\pgfqpoint{10.306389in}{10.441068in}}%
\pgfpathlineto{\pgfqpoint{10.429612in}{10.441068in}}%
\pgfpathlineto{\pgfqpoint{10.552835in}{10.441068in}}%
\pgfpathlineto{\pgfqpoint{10.676058in}{10.441068in}}%
\pgfpathlineto{\pgfqpoint{10.799281in}{10.441068in}}%
\pgfpathlineto{\pgfqpoint{10.922504in}{10.441068in}}%
\pgfpathlineto{\pgfqpoint{11.045727in}{10.441068in}}%
\pgfpathlineto{\pgfqpoint{11.168950in}{10.441068in}}%
\pgfpathlineto{\pgfqpoint{11.292173in}{10.441068in}}%
\pgfpathlineto{\pgfqpoint{11.415396in}{10.441068in}}%
\pgfpathlineto{\pgfqpoint{11.538619in}{10.441068in}}%
\pgfpathlineto{\pgfqpoint{11.661842in}{10.441068in}}%
\pgfpathlineto{\pgfqpoint{11.785065in}{10.441068in}}%
\pgfpathlineto{\pgfqpoint{11.908288in}{10.441068in}}%
\pgfpathlineto{\pgfqpoint{12.031511in}{10.441068in}}%
\pgfpathlineto{\pgfqpoint{12.154734in}{10.441068in}}%
\pgfpathlineto{\pgfqpoint{12.277957in}{10.441068in}}%
\pgfpathlineto{\pgfqpoint{12.401180in}{10.441068in}}%
\pgfpathlineto{\pgfqpoint{12.524403in}{10.441068in}}%
\pgfpathlineto{\pgfqpoint{12.647626in}{10.441068in}}%
\pgfpathlineto{\pgfqpoint{12.770849in}{10.441068in}}%
\pgfpathlineto{\pgfqpoint{12.894072in}{10.441068in}}%
\pgfpathlineto{\pgfqpoint{13.017294in}{10.441068in}}%
\pgfpathlineto{\pgfqpoint{13.140517in}{10.441068in}}%
\pgfpathlineto{\pgfqpoint{13.263740in}{10.441068in}}%
\pgfpathlineto{\pgfqpoint{13.386963in}{10.441068in}}%
\pgfpathlineto{\pgfqpoint{13.510186in}{10.441068in}}%
\pgfpathlineto{\pgfqpoint{13.633409in}{10.441068in}}%
\pgfpathlineto{\pgfqpoint{13.756632in}{10.441068in}}%
\pgfpathlineto{\pgfqpoint{13.879855in}{10.441068in}}%
\pgfpathlineto{\pgfqpoint{14.003078in}{10.441068in}}%
\pgfpathlineto{\pgfqpoint{14.126301in}{10.441068in}}%
\pgfpathlineto{\pgfqpoint{14.249524in}{10.441068in}}%
\pgfpathlineto{\pgfqpoint{14.372747in}{10.441068in}}%
\pgfpathlineto{\pgfqpoint{14.495970in}{10.441068in}}%
\pgfpathlineto{\pgfqpoint{14.619193in}{10.441068in}}%
\pgfpathlineto{\pgfqpoint{14.742416in}{10.441068in}}%
\pgfpathlineto{\pgfqpoint{14.865639in}{10.441068in}}%
\pgfpathlineto{\pgfqpoint{15.050473in}{10.441068in}}%
\pgfpathlineto{\pgfqpoint{15.050473in}{10.517735in}}%
\pgfpathlineto{\pgfqpoint{15.050473in}{10.517735in}}%
\pgfpathlineto{\pgfqpoint{14.865639in}{10.517735in}}%
\pgfpathlineto{\pgfqpoint{14.742416in}{10.517735in}}%
\pgfpathlineto{\pgfqpoint{14.619193in}{10.517735in}}%
\pgfpathlineto{\pgfqpoint{14.495970in}{10.517735in}}%
\pgfpathlineto{\pgfqpoint{14.372747in}{10.517735in}}%
\pgfpathlineto{\pgfqpoint{14.249524in}{10.517735in}}%
\pgfpathlineto{\pgfqpoint{14.126301in}{10.517735in}}%
\pgfpathlineto{\pgfqpoint{14.003078in}{10.517735in}}%
\pgfpathlineto{\pgfqpoint{13.879855in}{10.517735in}}%
\pgfpathlineto{\pgfqpoint{13.756632in}{10.517735in}}%
\pgfpathlineto{\pgfqpoint{13.633409in}{10.517735in}}%
\pgfpathlineto{\pgfqpoint{13.510186in}{10.517735in}}%
\pgfpathlineto{\pgfqpoint{13.386963in}{10.517735in}}%
\pgfpathlineto{\pgfqpoint{13.263740in}{10.517735in}}%
\pgfpathlineto{\pgfqpoint{13.140517in}{10.517735in}}%
\pgfpathlineto{\pgfqpoint{13.017294in}{10.517735in}}%
\pgfpathlineto{\pgfqpoint{12.894072in}{10.517735in}}%
\pgfpathlineto{\pgfqpoint{12.770849in}{10.517735in}}%
\pgfpathlineto{\pgfqpoint{12.647626in}{10.517735in}}%
\pgfpathlineto{\pgfqpoint{12.524403in}{10.517735in}}%
\pgfpathlineto{\pgfqpoint{12.401180in}{10.517735in}}%
\pgfpathlineto{\pgfqpoint{12.277957in}{10.517735in}}%
\pgfpathlineto{\pgfqpoint{12.154734in}{10.517735in}}%
\pgfpathlineto{\pgfqpoint{12.031511in}{10.517735in}}%
\pgfpathlineto{\pgfqpoint{11.908288in}{10.517735in}}%
\pgfpathlineto{\pgfqpoint{11.785065in}{10.517735in}}%
\pgfpathlineto{\pgfqpoint{11.661842in}{10.517735in}}%
\pgfpathlineto{\pgfqpoint{11.538619in}{10.517735in}}%
\pgfpathlineto{\pgfqpoint{11.415396in}{10.517735in}}%
\pgfpathlineto{\pgfqpoint{11.292173in}{10.517735in}}%
\pgfpathlineto{\pgfqpoint{11.168950in}{10.517735in}}%
\pgfpathlineto{\pgfqpoint{11.045727in}{10.517735in}}%
\pgfpathlineto{\pgfqpoint{10.922504in}{10.517735in}}%
\pgfpathlineto{\pgfqpoint{10.799281in}{10.517735in}}%
\pgfpathlineto{\pgfqpoint{10.676058in}{10.517735in}}%
\pgfpathlineto{\pgfqpoint{10.552835in}{10.517735in}}%
\pgfpathlineto{\pgfqpoint{10.429612in}{10.517735in}}%
\pgfpathlineto{\pgfqpoint{10.306389in}{10.517735in}}%
\pgfpathlineto{\pgfqpoint{10.121555in}{10.517735in}}%
\pgfpathclose%
\pgfusepath{stroke,fill}%
\end{pgfscope}%
\begin{pgfscope}%
\pgfpathrectangle{\pgfqpoint{9.810417in}{10.401163in}}{\pgfqpoint{5.489583in}{0.877907in}}%
\pgfusepath{clip}%
\pgfsetbuttcap%
\pgfsetroundjoin%
\pgfsetlinewidth{1.505625pt}%
\definecolor{currentstroke}{rgb}{0.000000,0.000000,0.000000}%
\pgfsetstrokecolor{currentstroke}%
\pgfsetdash{}{0pt}%
\pgfpathmoveto{\pgfqpoint{10.059943in}{10.479401in}}%
\pgfpathlineto{\pgfqpoint{10.059943in}{11.239165in}}%
\pgfusepath{stroke}%
\end{pgfscope}%
\begin{pgfscope}%
\pgfpathrectangle{\pgfqpoint{9.810417in}{10.401163in}}{\pgfqpoint{5.489583in}{0.877907in}}%
\pgfusepath{clip}%
\pgfsetbuttcap%
\pgfsetroundjoin%
\pgfsetlinewidth{1.505625pt}%
\definecolor{currentstroke}{rgb}{0.000000,0.000000,0.000000}%
\pgfsetstrokecolor{currentstroke}%
\pgfsetdash{}{0pt}%
\pgfpathmoveto{\pgfqpoint{10.183166in}{10.479401in}}%
\pgfpathlineto{\pgfqpoint{10.183166in}{11.237752in}}%
\pgfusepath{stroke}%
\end{pgfscope}%
\begin{pgfscope}%
\pgfpathrectangle{\pgfqpoint{9.810417in}{10.401163in}}{\pgfqpoint{5.489583in}{0.877907in}}%
\pgfusepath{clip}%
\pgfsetbuttcap%
\pgfsetroundjoin%
\pgfsetlinewidth{1.505625pt}%
\definecolor{currentstroke}{rgb}{0.000000,0.000000,0.000000}%
\pgfsetstrokecolor{currentstroke}%
\pgfsetdash{}{0pt}%
\pgfpathmoveto{\pgfqpoint{10.306389in}{10.479401in}}%
\pgfpathlineto{\pgfqpoint{10.306389in}{10.474090in}}%
\pgfusepath{stroke}%
\end{pgfscope}%
\begin{pgfscope}%
\pgfpathrectangle{\pgfqpoint{9.810417in}{10.401163in}}{\pgfqpoint{5.489583in}{0.877907in}}%
\pgfusepath{clip}%
\pgfsetbuttcap%
\pgfsetroundjoin%
\pgfsetlinewidth{1.505625pt}%
\definecolor{currentstroke}{rgb}{0.000000,0.000000,0.000000}%
\pgfsetstrokecolor{currentstroke}%
\pgfsetdash{}{0pt}%
\pgfpathmoveto{\pgfqpoint{10.429612in}{10.479401in}}%
\pgfpathlineto{\pgfqpoint{10.429612in}{10.474101in}}%
\pgfusepath{stroke}%
\end{pgfscope}%
\begin{pgfscope}%
\pgfpathrectangle{\pgfqpoint{9.810417in}{10.401163in}}{\pgfqpoint{5.489583in}{0.877907in}}%
\pgfusepath{clip}%
\pgfsetbuttcap%
\pgfsetroundjoin%
\pgfsetlinewidth{1.505625pt}%
\definecolor{currentstroke}{rgb}{0.000000,0.000000,0.000000}%
\pgfsetstrokecolor{currentstroke}%
\pgfsetdash{}{0pt}%
\pgfpathmoveto{\pgfqpoint{10.552835in}{10.479401in}}%
\pgfpathlineto{\pgfqpoint{10.552835in}{10.462792in}}%
\pgfusepath{stroke}%
\end{pgfscope}%
\begin{pgfscope}%
\pgfpathrectangle{\pgfqpoint{9.810417in}{10.401163in}}{\pgfqpoint{5.489583in}{0.877907in}}%
\pgfusepath{clip}%
\pgfsetbuttcap%
\pgfsetroundjoin%
\pgfsetlinewidth{1.505625pt}%
\definecolor{currentstroke}{rgb}{0.000000,0.000000,0.000000}%
\pgfsetstrokecolor{currentstroke}%
\pgfsetdash{}{0pt}%
\pgfpathmoveto{\pgfqpoint{10.676058in}{10.479401in}}%
\pgfpathlineto{\pgfqpoint{10.676058in}{10.488750in}}%
\pgfusepath{stroke}%
\end{pgfscope}%
\begin{pgfscope}%
\pgfpathrectangle{\pgfqpoint{9.810417in}{10.401163in}}{\pgfqpoint{5.489583in}{0.877907in}}%
\pgfusepath{clip}%
\pgfsetbuttcap%
\pgfsetroundjoin%
\pgfsetlinewidth{1.505625pt}%
\definecolor{currentstroke}{rgb}{0.000000,0.000000,0.000000}%
\pgfsetstrokecolor{currentstroke}%
\pgfsetdash{}{0pt}%
\pgfpathmoveto{\pgfqpoint{10.799281in}{10.479401in}}%
\pgfpathlineto{\pgfqpoint{10.799281in}{10.467843in}}%
\pgfusepath{stroke}%
\end{pgfscope}%
\begin{pgfscope}%
\pgfpathrectangle{\pgfqpoint{9.810417in}{10.401163in}}{\pgfqpoint{5.489583in}{0.877907in}}%
\pgfusepath{clip}%
\pgfsetbuttcap%
\pgfsetroundjoin%
\pgfsetlinewidth{1.505625pt}%
\definecolor{currentstroke}{rgb}{0.000000,0.000000,0.000000}%
\pgfsetstrokecolor{currentstroke}%
\pgfsetdash{}{0pt}%
\pgfpathmoveto{\pgfqpoint{10.922504in}{10.479401in}}%
\pgfpathlineto{\pgfqpoint{10.922504in}{10.477460in}}%
\pgfusepath{stroke}%
\end{pgfscope}%
\begin{pgfscope}%
\pgfpathrectangle{\pgfqpoint{9.810417in}{10.401163in}}{\pgfqpoint{5.489583in}{0.877907in}}%
\pgfusepath{clip}%
\pgfsetbuttcap%
\pgfsetroundjoin%
\pgfsetlinewidth{1.505625pt}%
\definecolor{currentstroke}{rgb}{0.000000,0.000000,0.000000}%
\pgfsetstrokecolor{currentstroke}%
\pgfsetdash{}{0pt}%
\pgfpathmoveto{\pgfqpoint{11.045727in}{10.479401in}}%
\pgfpathlineto{\pgfqpoint{11.045727in}{10.475896in}}%
\pgfusepath{stroke}%
\end{pgfscope}%
\begin{pgfscope}%
\pgfpathrectangle{\pgfqpoint{9.810417in}{10.401163in}}{\pgfqpoint{5.489583in}{0.877907in}}%
\pgfusepath{clip}%
\pgfsetbuttcap%
\pgfsetroundjoin%
\pgfsetlinewidth{1.505625pt}%
\definecolor{currentstroke}{rgb}{0.000000,0.000000,0.000000}%
\pgfsetstrokecolor{currentstroke}%
\pgfsetdash{}{0pt}%
\pgfpathmoveto{\pgfqpoint{11.168950in}{10.479401in}}%
\pgfpathlineto{\pgfqpoint{11.168950in}{10.471942in}}%
\pgfusepath{stroke}%
\end{pgfscope}%
\begin{pgfscope}%
\pgfpathrectangle{\pgfqpoint{9.810417in}{10.401163in}}{\pgfqpoint{5.489583in}{0.877907in}}%
\pgfusepath{clip}%
\pgfsetbuttcap%
\pgfsetroundjoin%
\pgfsetlinewidth{1.505625pt}%
\definecolor{currentstroke}{rgb}{0.000000,0.000000,0.000000}%
\pgfsetstrokecolor{currentstroke}%
\pgfsetdash{}{0pt}%
\pgfpathmoveto{\pgfqpoint{11.292173in}{10.479401in}}%
\pgfpathlineto{\pgfqpoint{11.292173in}{10.479226in}}%
\pgfusepath{stroke}%
\end{pgfscope}%
\begin{pgfscope}%
\pgfpathrectangle{\pgfqpoint{9.810417in}{10.401163in}}{\pgfqpoint{5.489583in}{0.877907in}}%
\pgfusepath{clip}%
\pgfsetbuttcap%
\pgfsetroundjoin%
\pgfsetlinewidth{1.505625pt}%
\definecolor{currentstroke}{rgb}{0.000000,0.000000,0.000000}%
\pgfsetstrokecolor{currentstroke}%
\pgfsetdash{}{0pt}%
\pgfpathmoveto{\pgfqpoint{11.415396in}{10.479401in}}%
\pgfpathlineto{\pgfqpoint{11.415396in}{10.487917in}}%
\pgfusepath{stroke}%
\end{pgfscope}%
\begin{pgfscope}%
\pgfpathrectangle{\pgfqpoint{9.810417in}{10.401163in}}{\pgfqpoint{5.489583in}{0.877907in}}%
\pgfusepath{clip}%
\pgfsetbuttcap%
\pgfsetroundjoin%
\pgfsetlinewidth{1.505625pt}%
\definecolor{currentstroke}{rgb}{0.000000,0.000000,0.000000}%
\pgfsetstrokecolor{currentstroke}%
\pgfsetdash{}{0pt}%
\pgfpathmoveto{\pgfqpoint{11.538619in}{10.479401in}}%
\pgfpathlineto{\pgfqpoint{11.538619in}{10.474265in}}%
\pgfusepath{stroke}%
\end{pgfscope}%
\begin{pgfscope}%
\pgfpathrectangle{\pgfqpoint{9.810417in}{10.401163in}}{\pgfqpoint{5.489583in}{0.877907in}}%
\pgfusepath{clip}%
\pgfsetbuttcap%
\pgfsetroundjoin%
\pgfsetlinewidth{1.505625pt}%
\definecolor{currentstroke}{rgb}{0.000000,0.000000,0.000000}%
\pgfsetstrokecolor{currentstroke}%
\pgfsetdash{}{0pt}%
\pgfpathmoveto{\pgfqpoint{11.661842in}{10.479401in}}%
\pgfpathlineto{\pgfqpoint{11.661842in}{10.475457in}}%
\pgfusepath{stroke}%
\end{pgfscope}%
\begin{pgfscope}%
\pgfpathrectangle{\pgfqpoint{9.810417in}{10.401163in}}{\pgfqpoint{5.489583in}{0.877907in}}%
\pgfusepath{clip}%
\pgfsetbuttcap%
\pgfsetroundjoin%
\pgfsetlinewidth{1.505625pt}%
\definecolor{currentstroke}{rgb}{0.000000,0.000000,0.000000}%
\pgfsetstrokecolor{currentstroke}%
\pgfsetdash{}{0pt}%
\pgfpathmoveto{\pgfqpoint{11.785065in}{10.479401in}}%
\pgfpathlineto{\pgfqpoint{11.785065in}{10.480883in}}%
\pgfusepath{stroke}%
\end{pgfscope}%
\begin{pgfscope}%
\pgfpathrectangle{\pgfqpoint{9.810417in}{10.401163in}}{\pgfqpoint{5.489583in}{0.877907in}}%
\pgfusepath{clip}%
\pgfsetbuttcap%
\pgfsetroundjoin%
\pgfsetlinewidth{1.505625pt}%
\definecolor{currentstroke}{rgb}{0.000000,0.000000,0.000000}%
\pgfsetstrokecolor{currentstroke}%
\pgfsetdash{}{0pt}%
\pgfpathmoveto{\pgfqpoint{11.908288in}{10.479401in}}%
\pgfpathlineto{\pgfqpoint{11.908288in}{10.491011in}}%
\pgfusepath{stroke}%
\end{pgfscope}%
\begin{pgfscope}%
\pgfpathrectangle{\pgfqpoint{9.810417in}{10.401163in}}{\pgfqpoint{5.489583in}{0.877907in}}%
\pgfusepath{clip}%
\pgfsetbuttcap%
\pgfsetroundjoin%
\pgfsetlinewidth{1.505625pt}%
\definecolor{currentstroke}{rgb}{0.000000,0.000000,0.000000}%
\pgfsetstrokecolor{currentstroke}%
\pgfsetdash{}{0pt}%
\pgfpathmoveto{\pgfqpoint{12.031511in}{10.479401in}}%
\pgfpathlineto{\pgfqpoint{12.031511in}{10.490082in}}%
\pgfusepath{stroke}%
\end{pgfscope}%
\begin{pgfscope}%
\pgfpathrectangle{\pgfqpoint{9.810417in}{10.401163in}}{\pgfqpoint{5.489583in}{0.877907in}}%
\pgfusepath{clip}%
\pgfsetbuttcap%
\pgfsetroundjoin%
\pgfsetlinewidth{1.505625pt}%
\definecolor{currentstroke}{rgb}{0.000000,0.000000,0.000000}%
\pgfsetstrokecolor{currentstroke}%
\pgfsetdash{}{0pt}%
\pgfpathmoveto{\pgfqpoint{12.154734in}{10.479401in}}%
\pgfpathlineto{\pgfqpoint{12.154734in}{10.483925in}}%
\pgfusepath{stroke}%
\end{pgfscope}%
\begin{pgfscope}%
\pgfpathrectangle{\pgfqpoint{9.810417in}{10.401163in}}{\pgfqpoint{5.489583in}{0.877907in}}%
\pgfusepath{clip}%
\pgfsetbuttcap%
\pgfsetroundjoin%
\pgfsetlinewidth{1.505625pt}%
\definecolor{currentstroke}{rgb}{0.000000,0.000000,0.000000}%
\pgfsetstrokecolor{currentstroke}%
\pgfsetdash{}{0pt}%
\pgfpathmoveto{\pgfqpoint{12.277957in}{10.479401in}}%
\pgfpathlineto{\pgfqpoint{12.277957in}{10.470171in}}%
\pgfusepath{stroke}%
\end{pgfscope}%
\begin{pgfscope}%
\pgfpathrectangle{\pgfqpoint{9.810417in}{10.401163in}}{\pgfqpoint{5.489583in}{0.877907in}}%
\pgfusepath{clip}%
\pgfsetbuttcap%
\pgfsetroundjoin%
\pgfsetlinewidth{1.505625pt}%
\definecolor{currentstroke}{rgb}{0.000000,0.000000,0.000000}%
\pgfsetstrokecolor{currentstroke}%
\pgfsetdash{}{0pt}%
\pgfpathmoveto{\pgfqpoint{12.401180in}{10.479401in}}%
\pgfpathlineto{\pgfqpoint{12.401180in}{10.488543in}}%
\pgfusepath{stroke}%
\end{pgfscope}%
\begin{pgfscope}%
\pgfpathrectangle{\pgfqpoint{9.810417in}{10.401163in}}{\pgfqpoint{5.489583in}{0.877907in}}%
\pgfusepath{clip}%
\pgfsetbuttcap%
\pgfsetroundjoin%
\pgfsetlinewidth{1.505625pt}%
\definecolor{currentstroke}{rgb}{0.000000,0.000000,0.000000}%
\pgfsetstrokecolor{currentstroke}%
\pgfsetdash{}{0pt}%
\pgfpathmoveto{\pgfqpoint{12.524403in}{10.479401in}}%
\pgfpathlineto{\pgfqpoint{12.524403in}{10.479145in}}%
\pgfusepath{stroke}%
\end{pgfscope}%
\begin{pgfscope}%
\pgfpathrectangle{\pgfqpoint{9.810417in}{10.401163in}}{\pgfqpoint{5.489583in}{0.877907in}}%
\pgfusepath{clip}%
\pgfsetbuttcap%
\pgfsetroundjoin%
\pgfsetlinewidth{1.505625pt}%
\definecolor{currentstroke}{rgb}{0.000000,0.000000,0.000000}%
\pgfsetstrokecolor{currentstroke}%
\pgfsetdash{}{0pt}%
\pgfpathmoveto{\pgfqpoint{12.647626in}{10.479401in}}%
\pgfpathlineto{\pgfqpoint{12.647626in}{10.476369in}}%
\pgfusepath{stroke}%
\end{pgfscope}%
\begin{pgfscope}%
\pgfpathrectangle{\pgfqpoint{9.810417in}{10.401163in}}{\pgfqpoint{5.489583in}{0.877907in}}%
\pgfusepath{clip}%
\pgfsetbuttcap%
\pgfsetroundjoin%
\pgfsetlinewidth{1.505625pt}%
\definecolor{currentstroke}{rgb}{0.000000,0.000000,0.000000}%
\pgfsetstrokecolor{currentstroke}%
\pgfsetdash{}{0pt}%
\pgfpathmoveto{\pgfqpoint{12.770849in}{10.479401in}}%
\pgfpathlineto{\pgfqpoint{12.770849in}{10.468494in}}%
\pgfusepath{stroke}%
\end{pgfscope}%
\begin{pgfscope}%
\pgfpathrectangle{\pgfqpoint{9.810417in}{10.401163in}}{\pgfqpoint{5.489583in}{0.877907in}}%
\pgfusepath{clip}%
\pgfsetbuttcap%
\pgfsetroundjoin%
\pgfsetlinewidth{1.505625pt}%
\definecolor{currentstroke}{rgb}{0.000000,0.000000,0.000000}%
\pgfsetstrokecolor{currentstroke}%
\pgfsetdash{}{0pt}%
\pgfpathmoveto{\pgfqpoint{12.894072in}{10.479401in}}%
\pgfpathlineto{\pgfqpoint{12.894072in}{10.474579in}}%
\pgfusepath{stroke}%
\end{pgfscope}%
\begin{pgfscope}%
\pgfpathrectangle{\pgfqpoint{9.810417in}{10.401163in}}{\pgfqpoint{5.489583in}{0.877907in}}%
\pgfusepath{clip}%
\pgfsetbuttcap%
\pgfsetroundjoin%
\pgfsetlinewidth{1.505625pt}%
\definecolor{currentstroke}{rgb}{0.000000,0.000000,0.000000}%
\pgfsetstrokecolor{currentstroke}%
\pgfsetdash{}{0pt}%
\pgfpathmoveto{\pgfqpoint{13.017294in}{10.479401in}}%
\pgfpathlineto{\pgfqpoint{13.017294in}{10.480993in}}%
\pgfusepath{stroke}%
\end{pgfscope}%
\begin{pgfscope}%
\pgfpathrectangle{\pgfqpoint{9.810417in}{10.401163in}}{\pgfqpoint{5.489583in}{0.877907in}}%
\pgfusepath{clip}%
\pgfsetbuttcap%
\pgfsetroundjoin%
\pgfsetlinewidth{1.505625pt}%
\definecolor{currentstroke}{rgb}{0.000000,0.000000,0.000000}%
\pgfsetstrokecolor{currentstroke}%
\pgfsetdash{}{0pt}%
\pgfpathmoveto{\pgfqpoint{13.140517in}{10.479401in}}%
\pgfpathlineto{\pgfqpoint{13.140517in}{10.486893in}}%
\pgfusepath{stroke}%
\end{pgfscope}%
\begin{pgfscope}%
\pgfpathrectangle{\pgfqpoint{9.810417in}{10.401163in}}{\pgfqpoint{5.489583in}{0.877907in}}%
\pgfusepath{clip}%
\pgfsetbuttcap%
\pgfsetroundjoin%
\pgfsetlinewidth{1.505625pt}%
\definecolor{currentstroke}{rgb}{0.000000,0.000000,0.000000}%
\pgfsetstrokecolor{currentstroke}%
\pgfsetdash{}{0pt}%
\pgfpathmoveto{\pgfqpoint{13.263740in}{10.479401in}}%
\pgfpathlineto{\pgfqpoint{13.263740in}{10.486707in}}%
\pgfusepath{stroke}%
\end{pgfscope}%
\begin{pgfscope}%
\pgfpathrectangle{\pgfqpoint{9.810417in}{10.401163in}}{\pgfqpoint{5.489583in}{0.877907in}}%
\pgfusepath{clip}%
\pgfsetbuttcap%
\pgfsetroundjoin%
\pgfsetlinewidth{1.505625pt}%
\definecolor{currentstroke}{rgb}{0.000000,0.000000,0.000000}%
\pgfsetstrokecolor{currentstroke}%
\pgfsetdash{}{0pt}%
\pgfpathmoveto{\pgfqpoint{13.386963in}{10.479401in}}%
\pgfpathlineto{\pgfqpoint{13.386963in}{10.470054in}}%
\pgfusepath{stroke}%
\end{pgfscope}%
\begin{pgfscope}%
\pgfpathrectangle{\pgfqpoint{9.810417in}{10.401163in}}{\pgfqpoint{5.489583in}{0.877907in}}%
\pgfusepath{clip}%
\pgfsetbuttcap%
\pgfsetroundjoin%
\pgfsetlinewidth{1.505625pt}%
\definecolor{currentstroke}{rgb}{0.000000,0.000000,0.000000}%
\pgfsetstrokecolor{currentstroke}%
\pgfsetdash{}{0pt}%
\pgfpathmoveto{\pgfqpoint{13.510186in}{10.479401in}}%
\pgfpathlineto{\pgfqpoint{13.510186in}{10.471396in}}%
\pgfusepath{stroke}%
\end{pgfscope}%
\begin{pgfscope}%
\pgfpathrectangle{\pgfqpoint{9.810417in}{10.401163in}}{\pgfqpoint{5.489583in}{0.877907in}}%
\pgfusepath{clip}%
\pgfsetbuttcap%
\pgfsetroundjoin%
\pgfsetlinewidth{1.505625pt}%
\definecolor{currentstroke}{rgb}{0.000000,0.000000,0.000000}%
\pgfsetstrokecolor{currentstroke}%
\pgfsetdash{}{0pt}%
\pgfpathmoveto{\pgfqpoint{13.633409in}{10.479401in}}%
\pgfpathlineto{\pgfqpoint{13.633409in}{10.479059in}}%
\pgfusepath{stroke}%
\end{pgfscope}%
\begin{pgfscope}%
\pgfpathrectangle{\pgfqpoint{9.810417in}{10.401163in}}{\pgfqpoint{5.489583in}{0.877907in}}%
\pgfusepath{clip}%
\pgfsetbuttcap%
\pgfsetroundjoin%
\pgfsetlinewidth{1.505625pt}%
\definecolor{currentstroke}{rgb}{0.000000,0.000000,0.000000}%
\pgfsetstrokecolor{currentstroke}%
\pgfsetdash{}{0pt}%
\pgfpathmoveto{\pgfqpoint{13.756632in}{10.479401in}}%
\pgfpathlineto{\pgfqpoint{13.756632in}{10.474079in}}%
\pgfusepath{stroke}%
\end{pgfscope}%
\begin{pgfscope}%
\pgfpathrectangle{\pgfqpoint{9.810417in}{10.401163in}}{\pgfqpoint{5.489583in}{0.877907in}}%
\pgfusepath{clip}%
\pgfsetbuttcap%
\pgfsetroundjoin%
\pgfsetlinewidth{1.505625pt}%
\definecolor{currentstroke}{rgb}{0.000000,0.000000,0.000000}%
\pgfsetstrokecolor{currentstroke}%
\pgfsetdash{}{0pt}%
\pgfpathmoveto{\pgfqpoint{13.879855in}{10.479401in}}%
\pgfpathlineto{\pgfqpoint{13.879855in}{10.465764in}}%
\pgfusepath{stroke}%
\end{pgfscope}%
\begin{pgfscope}%
\pgfpathrectangle{\pgfqpoint{9.810417in}{10.401163in}}{\pgfqpoint{5.489583in}{0.877907in}}%
\pgfusepath{clip}%
\pgfsetbuttcap%
\pgfsetroundjoin%
\pgfsetlinewidth{1.505625pt}%
\definecolor{currentstroke}{rgb}{0.000000,0.000000,0.000000}%
\pgfsetstrokecolor{currentstroke}%
\pgfsetdash{}{0pt}%
\pgfpathmoveto{\pgfqpoint{14.003078in}{10.479401in}}%
\pgfpathlineto{\pgfqpoint{14.003078in}{10.485554in}}%
\pgfusepath{stroke}%
\end{pgfscope}%
\begin{pgfscope}%
\pgfpathrectangle{\pgfqpoint{9.810417in}{10.401163in}}{\pgfqpoint{5.489583in}{0.877907in}}%
\pgfusepath{clip}%
\pgfsetbuttcap%
\pgfsetroundjoin%
\pgfsetlinewidth{1.505625pt}%
\definecolor{currentstroke}{rgb}{0.000000,0.000000,0.000000}%
\pgfsetstrokecolor{currentstroke}%
\pgfsetdash{}{0pt}%
\pgfpathmoveto{\pgfqpoint{14.126301in}{10.479401in}}%
\pgfpathlineto{\pgfqpoint{14.126301in}{10.482125in}}%
\pgfusepath{stroke}%
\end{pgfscope}%
\begin{pgfscope}%
\pgfpathrectangle{\pgfqpoint{9.810417in}{10.401163in}}{\pgfqpoint{5.489583in}{0.877907in}}%
\pgfusepath{clip}%
\pgfsetbuttcap%
\pgfsetroundjoin%
\pgfsetlinewidth{1.505625pt}%
\definecolor{currentstroke}{rgb}{0.000000,0.000000,0.000000}%
\pgfsetstrokecolor{currentstroke}%
\pgfsetdash{}{0pt}%
\pgfpathmoveto{\pgfqpoint{14.249524in}{10.479401in}}%
\pgfpathlineto{\pgfqpoint{14.249524in}{10.484175in}}%
\pgfusepath{stroke}%
\end{pgfscope}%
\begin{pgfscope}%
\pgfpathrectangle{\pgfqpoint{9.810417in}{10.401163in}}{\pgfqpoint{5.489583in}{0.877907in}}%
\pgfusepath{clip}%
\pgfsetbuttcap%
\pgfsetroundjoin%
\pgfsetlinewidth{1.505625pt}%
\definecolor{currentstroke}{rgb}{0.000000,0.000000,0.000000}%
\pgfsetstrokecolor{currentstroke}%
\pgfsetdash{}{0pt}%
\pgfpathmoveto{\pgfqpoint{14.372747in}{10.479401in}}%
\pgfpathlineto{\pgfqpoint{14.372747in}{10.465506in}}%
\pgfusepath{stroke}%
\end{pgfscope}%
\begin{pgfscope}%
\pgfpathrectangle{\pgfqpoint{9.810417in}{10.401163in}}{\pgfqpoint{5.489583in}{0.877907in}}%
\pgfusepath{clip}%
\pgfsetbuttcap%
\pgfsetroundjoin%
\pgfsetlinewidth{1.505625pt}%
\definecolor{currentstroke}{rgb}{0.000000,0.000000,0.000000}%
\pgfsetstrokecolor{currentstroke}%
\pgfsetdash{}{0pt}%
\pgfpathmoveto{\pgfqpoint{14.495970in}{10.479401in}}%
\pgfpathlineto{\pgfqpoint{14.495970in}{10.476061in}}%
\pgfusepath{stroke}%
\end{pgfscope}%
\begin{pgfscope}%
\pgfpathrectangle{\pgfqpoint{9.810417in}{10.401163in}}{\pgfqpoint{5.489583in}{0.877907in}}%
\pgfusepath{clip}%
\pgfsetbuttcap%
\pgfsetroundjoin%
\pgfsetlinewidth{1.505625pt}%
\definecolor{currentstroke}{rgb}{0.000000,0.000000,0.000000}%
\pgfsetstrokecolor{currentstroke}%
\pgfsetdash{}{0pt}%
\pgfpathmoveto{\pgfqpoint{14.619193in}{10.479401in}}%
\pgfpathlineto{\pgfqpoint{14.619193in}{10.487083in}}%
\pgfusepath{stroke}%
\end{pgfscope}%
\begin{pgfscope}%
\pgfpathrectangle{\pgfqpoint{9.810417in}{10.401163in}}{\pgfqpoint{5.489583in}{0.877907in}}%
\pgfusepath{clip}%
\pgfsetbuttcap%
\pgfsetroundjoin%
\pgfsetlinewidth{1.505625pt}%
\definecolor{currentstroke}{rgb}{0.000000,0.000000,0.000000}%
\pgfsetstrokecolor{currentstroke}%
\pgfsetdash{}{0pt}%
\pgfpathmoveto{\pgfqpoint{14.742416in}{10.479401in}}%
\pgfpathlineto{\pgfqpoint{14.742416in}{10.482670in}}%
\pgfusepath{stroke}%
\end{pgfscope}%
\begin{pgfscope}%
\pgfpathrectangle{\pgfqpoint{9.810417in}{10.401163in}}{\pgfqpoint{5.489583in}{0.877907in}}%
\pgfusepath{clip}%
\pgfsetbuttcap%
\pgfsetroundjoin%
\pgfsetlinewidth{1.505625pt}%
\definecolor{currentstroke}{rgb}{0.000000,0.000000,0.000000}%
\pgfsetstrokecolor{currentstroke}%
\pgfsetdash{}{0pt}%
\pgfpathmoveto{\pgfqpoint{14.865639in}{10.479401in}}%
\pgfpathlineto{\pgfqpoint{14.865639in}{10.489412in}}%
\pgfusepath{stroke}%
\end{pgfscope}%
\begin{pgfscope}%
\pgfpathrectangle{\pgfqpoint{9.810417in}{10.401163in}}{\pgfqpoint{5.489583in}{0.877907in}}%
\pgfusepath{clip}%
\pgfsetbuttcap%
\pgfsetroundjoin%
\pgfsetlinewidth{1.505625pt}%
\definecolor{currentstroke}{rgb}{0.000000,0.000000,0.000000}%
\pgfsetstrokecolor{currentstroke}%
\pgfsetdash{}{0pt}%
\pgfpathmoveto{\pgfqpoint{14.988862in}{10.479401in}}%
\pgfpathlineto{\pgfqpoint{14.988862in}{10.478444in}}%
\pgfusepath{stroke}%
\end{pgfscope}%
\begin{pgfscope}%
\pgfpathrectangle{\pgfqpoint{9.810417in}{10.401163in}}{\pgfqpoint{5.489583in}{0.877907in}}%
\pgfusepath{clip}%
\pgfsetroundcap%
\pgfsetroundjoin%
\pgfsetlinewidth{1.505625pt}%
\definecolor{currentstroke}{rgb}{0.121569,0.466667,0.705882}%
\pgfsetstrokecolor{currentstroke}%
\pgfsetdash{}{0pt}%
\pgfpathmoveto{\pgfqpoint{9.810417in}{10.479401in}}%
\pgfpathlineto{\pgfqpoint{15.300000in}{10.479401in}}%
\pgfusepath{stroke}%
\end{pgfscope}%
\begin{pgfscope}%
\pgfpathrectangle{\pgfqpoint{9.810417in}{10.401163in}}{\pgfqpoint{5.489583in}{0.877907in}}%
\pgfusepath{clip}%
\pgfsetbuttcap%
\pgfsetroundjoin%
\definecolor{currentfill}{rgb}{0.121569,0.466667,0.705882}%
\pgfsetfillcolor{currentfill}%
\pgfsetlinewidth{1.003750pt}%
\definecolor{currentstroke}{rgb}{0.121569,0.466667,0.705882}%
\pgfsetstrokecolor{currentstroke}%
\pgfsetdash{}{0pt}%
\pgfsys@defobject{currentmarker}{\pgfqpoint{-0.034722in}{-0.034722in}}{\pgfqpoint{0.034722in}{0.034722in}}{%
\pgfpathmoveto{\pgfqpoint{0.000000in}{-0.034722in}}%
\pgfpathcurveto{\pgfqpoint{0.009208in}{-0.034722in}}{\pgfqpoint{0.018041in}{-0.031064in}}{\pgfqpoint{0.024552in}{-0.024552in}}%
\pgfpathcurveto{\pgfqpoint{0.031064in}{-0.018041in}}{\pgfqpoint{0.034722in}{-0.009208in}}{\pgfqpoint{0.034722in}{0.000000in}}%
\pgfpathcurveto{\pgfqpoint{0.034722in}{0.009208in}}{\pgfqpoint{0.031064in}{0.018041in}}{\pgfqpoint{0.024552in}{0.024552in}}%
\pgfpathcurveto{\pgfqpoint{0.018041in}{0.031064in}}{\pgfqpoint{0.009208in}{0.034722in}}{\pgfqpoint{0.000000in}{0.034722in}}%
\pgfpathcurveto{\pgfqpoint{-0.009208in}{0.034722in}}{\pgfqpoint{-0.018041in}{0.031064in}}{\pgfqpoint{-0.024552in}{0.024552in}}%
\pgfpathcurveto{\pgfqpoint{-0.031064in}{0.018041in}}{\pgfqpoint{-0.034722in}{0.009208in}}{\pgfqpoint{-0.034722in}{0.000000in}}%
\pgfpathcurveto{\pgfqpoint{-0.034722in}{-0.009208in}}{\pgfqpoint{-0.031064in}{-0.018041in}}{\pgfqpoint{-0.024552in}{-0.024552in}}%
\pgfpathcurveto{\pgfqpoint{-0.018041in}{-0.031064in}}{\pgfqpoint{-0.009208in}{-0.034722in}}{\pgfqpoint{0.000000in}{-0.034722in}}%
\pgfpathclose%
\pgfusepath{stroke,fill}%
}%
\begin{pgfscope}%
\pgfsys@transformshift{10.059943in}{11.239165in}%
\pgfsys@useobject{currentmarker}{}%
\end{pgfscope}%
\begin{pgfscope}%
\pgfsys@transformshift{10.183166in}{11.237752in}%
\pgfsys@useobject{currentmarker}{}%
\end{pgfscope}%
\begin{pgfscope}%
\pgfsys@transformshift{10.306389in}{10.474090in}%
\pgfsys@useobject{currentmarker}{}%
\end{pgfscope}%
\begin{pgfscope}%
\pgfsys@transformshift{10.429612in}{10.474101in}%
\pgfsys@useobject{currentmarker}{}%
\end{pgfscope}%
\begin{pgfscope}%
\pgfsys@transformshift{10.552835in}{10.462792in}%
\pgfsys@useobject{currentmarker}{}%
\end{pgfscope}%
\begin{pgfscope}%
\pgfsys@transformshift{10.676058in}{10.488750in}%
\pgfsys@useobject{currentmarker}{}%
\end{pgfscope}%
\begin{pgfscope}%
\pgfsys@transformshift{10.799281in}{10.467843in}%
\pgfsys@useobject{currentmarker}{}%
\end{pgfscope}%
\begin{pgfscope}%
\pgfsys@transformshift{10.922504in}{10.477460in}%
\pgfsys@useobject{currentmarker}{}%
\end{pgfscope}%
\begin{pgfscope}%
\pgfsys@transformshift{11.045727in}{10.475896in}%
\pgfsys@useobject{currentmarker}{}%
\end{pgfscope}%
\begin{pgfscope}%
\pgfsys@transformshift{11.168950in}{10.471942in}%
\pgfsys@useobject{currentmarker}{}%
\end{pgfscope}%
\begin{pgfscope}%
\pgfsys@transformshift{11.292173in}{10.479226in}%
\pgfsys@useobject{currentmarker}{}%
\end{pgfscope}%
\begin{pgfscope}%
\pgfsys@transformshift{11.415396in}{10.487917in}%
\pgfsys@useobject{currentmarker}{}%
\end{pgfscope}%
\begin{pgfscope}%
\pgfsys@transformshift{11.538619in}{10.474265in}%
\pgfsys@useobject{currentmarker}{}%
\end{pgfscope}%
\begin{pgfscope}%
\pgfsys@transformshift{11.661842in}{10.475457in}%
\pgfsys@useobject{currentmarker}{}%
\end{pgfscope}%
\begin{pgfscope}%
\pgfsys@transformshift{11.785065in}{10.480883in}%
\pgfsys@useobject{currentmarker}{}%
\end{pgfscope}%
\begin{pgfscope}%
\pgfsys@transformshift{11.908288in}{10.491011in}%
\pgfsys@useobject{currentmarker}{}%
\end{pgfscope}%
\begin{pgfscope}%
\pgfsys@transformshift{12.031511in}{10.490082in}%
\pgfsys@useobject{currentmarker}{}%
\end{pgfscope}%
\begin{pgfscope}%
\pgfsys@transformshift{12.154734in}{10.483925in}%
\pgfsys@useobject{currentmarker}{}%
\end{pgfscope}%
\begin{pgfscope}%
\pgfsys@transformshift{12.277957in}{10.470171in}%
\pgfsys@useobject{currentmarker}{}%
\end{pgfscope}%
\begin{pgfscope}%
\pgfsys@transformshift{12.401180in}{10.488543in}%
\pgfsys@useobject{currentmarker}{}%
\end{pgfscope}%
\begin{pgfscope}%
\pgfsys@transformshift{12.524403in}{10.479145in}%
\pgfsys@useobject{currentmarker}{}%
\end{pgfscope}%
\begin{pgfscope}%
\pgfsys@transformshift{12.647626in}{10.476369in}%
\pgfsys@useobject{currentmarker}{}%
\end{pgfscope}%
\begin{pgfscope}%
\pgfsys@transformshift{12.770849in}{10.468494in}%
\pgfsys@useobject{currentmarker}{}%
\end{pgfscope}%
\begin{pgfscope}%
\pgfsys@transformshift{12.894072in}{10.474579in}%
\pgfsys@useobject{currentmarker}{}%
\end{pgfscope}%
\begin{pgfscope}%
\pgfsys@transformshift{13.017294in}{10.480993in}%
\pgfsys@useobject{currentmarker}{}%
\end{pgfscope}%
\begin{pgfscope}%
\pgfsys@transformshift{13.140517in}{10.486893in}%
\pgfsys@useobject{currentmarker}{}%
\end{pgfscope}%
\begin{pgfscope}%
\pgfsys@transformshift{13.263740in}{10.486707in}%
\pgfsys@useobject{currentmarker}{}%
\end{pgfscope}%
\begin{pgfscope}%
\pgfsys@transformshift{13.386963in}{10.470054in}%
\pgfsys@useobject{currentmarker}{}%
\end{pgfscope}%
\begin{pgfscope}%
\pgfsys@transformshift{13.510186in}{10.471396in}%
\pgfsys@useobject{currentmarker}{}%
\end{pgfscope}%
\begin{pgfscope}%
\pgfsys@transformshift{13.633409in}{10.479059in}%
\pgfsys@useobject{currentmarker}{}%
\end{pgfscope}%
\begin{pgfscope}%
\pgfsys@transformshift{13.756632in}{10.474079in}%
\pgfsys@useobject{currentmarker}{}%
\end{pgfscope}%
\begin{pgfscope}%
\pgfsys@transformshift{13.879855in}{10.465764in}%
\pgfsys@useobject{currentmarker}{}%
\end{pgfscope}%
\begin{pgfscope}%
\pgfsys@transformshift{14.003078in}{10.485554in}%
\pgfsys@useobject{currentmarker}{}%
\end{pgfscope}%
\begin{pgfscope}%
\pgfsys@transformshift{14.126301in}{10.482125in}%
\pgfsys@useobject{currentmarker}{}%
\end{pgfscope}%
\begin{pgfscope}%
\pgfsys@transformshift{14.249524in}{10.484175in}%
\pgfsys@useobject{currentmarker}{}%
\end{pgfscope}%
\begin{pgfscope}%
\pgfsys@transformshift{14.372747in}{10.465506in}%
\pgfsys@useobject{currentmarker}{}%
\end{pgfscope}%
\begin{pgfscope}%
\pgfsys@transformshift{14.495970in}{10.476061in}%
\pgfsys@useobject{currentmarker}{}%
\end{pgfscope}%
\begin{pgfscope}%
\pgfsys@transformshift{14.619193in}{10.487083in}%
\pgfsys@useobject{currentmarker}{}%
\end{pgfscope}%
\begin{pgfscope}%
\pgfsys@transformshift{14.742416in}{10.482670in}%
\pgfsys@useobject{currentmarker}{}%
\end{pgfscope}%
\begin{pgfscope}%
\pgfsys@transformshift{14.865639in}{10.489412in}%
\pgfsys@useobject{currentmarker}{}%
\end{pgfscope}%
\begin{pgfscope}%
\pgfsys@transformshift{14.988862in}{10.478444in}%
\pgfsys@useobject{currentmarker}{}%
\end{pgfscope}%
\end{pgfscope}%
\begin{pgfscope}%
\pgfsetrectcap%
\pgfsetmiterjoin%
\pgfsetlinewidth{0.803000pt}%
\definecolor{currentstroke}{rgb}{1.000000,1.000000,1.000000}%
\pgfsetstrokecolor{currentstroke}%
\pgfsetdash{}{0pt}%
\pgfpathmoveto{\pgfqpoint{9.810417in}{10.401163in}}%
\pgfpathlineto{\pgfqpoint{9.810417in}{11.279070in}}%
\pgfusepath{stroke}%
\end{pgfscope}%
\begin{pgfscope}%
\pgfsetrectcap%
\pgfsetmiterjoin%
\pgfsetlinewidth{0.803000pt}%
\definecolor{currentstroke}{rgb}{1.000000,1.000000,1.000000}%
\pgfsetstrokecolor{currentstroke}%
\pgfsetdash{}{0pt}%
\pgfpathmoveto{\pgfqpoint{15.300000in}{10.401163in}}%
\pgfpathlineto{\pgfqpoint{15.300000in}{11.279070in}}%
\pgfusepath{stroke}%
\end{pgfscope}%
\begin{pgfscope}%
\pgfsetrectcap%
\pgfsetmiterjoin%
\pgfsetlinewidth{0.803000pt}%
\definecolor{currentstroke}{rgb}{1.000000,1.000000,1.000000}%
\pgfsetstrokecolor{currentstroke}%
\pgfsetdash{}{0pt}%
\pgfpathmoveto{\pgfqpoint{9.810417in}{10.401163in}}%
\pgfpathlineto{\pgfqpoint{15.300000in}{10.401163in}}%
\pgfusepath{stroke}%
\end{pgfscope}%
\begin{pgfscope}%
\pgfsetrectcap%
\pgfsetmiterjoin%
\pgfsetlinewidth{0.803000pt}%
\definecolor{currentstroke}{rgb}{1.000000,1.000000,1.000000}%
\pgfsetstrokecolor{currentstroke}%
\pgfsetdash{}{0pt}%
\pgfpathmoveto{\pgfqpoint{9.810417in}{11.279070in}}%
\pgfpathlineto{\pgfqpoint{15.300000in}{11.279070in}}%
\pgfusepath{stroke}%
\end{pgfscope}%
\begin{pgfscope}%
\definecolor{textcolor}{rgb}{0.150000,0.150000,0.150000}%
\pgfsetstrokecolor{textcolor}%
\pgfsetfillcolor{textcolor}%
\pgftext[x=12.555208in,y=11.362403in,,base]{\color{textcolor}\rmfamily\fontsize{16.800000}{20.160000}\selectfont Partial Autocorrelation}%
\end{pgfscope}%
\begin{pgfscope}%
\pgfsetbuttcap%
\pgfsetmiterjoin%
\definecolor{currentfill}{rgb}{0.917647,0.917647,0.949020}%
\pgfsetfillcolor{currentfill}%
\pgfsetlinewidth{0.000000pt}%
\definecolor{currentstroke}{rgb}{0.000000,0.000000,0.000000}%
\pgfsetstrokecolor{currentstroke}%
\pgfsetstrokeopacity{0.000000}%
\pgfsetdash{}{0pt}%
\pgfpathmoveto{\pgfqpoint{2.125000in}{8.820930in}}%
\pgfpathlineto{\pgfqpoint{7.614583in}{8.820930in}}%
\pgfpathlineto{\pgfqpoint{7.614583in}{9.698837in}}%
\pgfpathlineto{\pgfqpoint{2.125000in}{9.698837in}}%
\pgfpathclose%
\pgfusepath{fill}%
\end{pgfscope}%
\begin{pgfscope}%
\pgfpathrectangle{\pgfqpoint{2.125000in}{8.820930in}}{\pgfqpoint{5.489583in}{0.877907in}}%
\pgfusepath{clip}%
\pgfsetroundcap%
\pgfsetroundjoin%
\pgfsetlinewidth{0.803000pt}%
\definecolor{currentstroke}{rgb}{1.000000,1.000000,1.000000}%
\pgfsetstrokecolor{currentstroke}%
\pgfsetdash{}{0pt}%
\pgfpathmoveto{\pgfqpoint{2.374527in}{8.820930in}}%
\pgfpathlineto{\pgfqpoint{2.374527in}{9.698837in}}%
\pgfusepath{stroke}%
\end{pgfscope}%
\begin{pgfscope}%
\definecolor{textcolor}{rgb}{0.150000,0.150000,0.150000}%
\pgfsetstrokecolor{textcolor}%
\pgfsetfillcolor{textcolor}%
\pgftext[x=2.374527in,y=8.723708in,,top]{\color{textcolor}\rmfamily\fontsize{14.000000}{16.800000}\selectfont 0}%
\end{pgfscope}%
\begin{pgfscope}%
\pgfpathrectangle{\pgfqpoint{2.125000in}{8.820930in}}{\pgfqpoint{5.489583in}{0.877907in}}%
\pgfusepath{clip}%
\pgfsetroundcap%
\pgfsetroundjoin%
\pgfsetlinewidth{0.803000pt}%
\definecolor{currentstroke}{rgb}{1.000000,1.000000,1.000000}%
\pgfsetstrokecolor{currentstroke}%
\pgfsetdash{}{0pt}%
\pgfpathmoveto{\pgfqpoint{2.990641in}{8.820930in}}%
\pgfpathlineto{\pgfqpoint{2.990641in}{9.698837in}}%
\pgfusepath{stroke}%
\end{pgfscope}%
\begin{pgfscope}%
\definecolor{textcolor}{rgb}{0.150000,0.150000,0.150000}%
\pgfsetstrokecolor{textcolor}%
\pgfsetfillcolor{textcolor}%
\pgftext[x=2.990641in,y=8.723708in,,top]{\color{textcolor}\rmfamily\fontsize{14.000000}{16.800000}\selectfont 5}%
\end{pgfscope}%
\begin{pgfscope}%
\pgfpathrectangle{\pgfqpoint{2.125000in}{8.820930in}}{\pgfqpoint{5.489583in}{0.877907in}}%
\pgfusepath{clip}%
\pgfsetroundcap%
\pgfsetroundjoin%
\pgfsetlinewidth{0.803000pt}%
\definecolor{currentstroke}{rgb}{1.000000,1.000000,1.000000}%
\pgfsetstrokecolor{currentstroke}%
\pgfsetdash{}{0pt}%
\pgfpathmoveto{\pgfqpoint{3.606756in}{8.820930in}}%
\pgfpathlineto{\pgfqpoint{3.606756in}{9.698837in}}%
\pgfusepath{stroke}%
\end{pgfscope}%
\begin{pgfscope}%
\definecolor{textcolor}{rgb}{0.150000,0.150000,0.150000}%
\pgfsetstrokecolor{textcolor}%
\pgfsetfillcolor{textcolor}%
\pgftext[x=3.606756in,y=8.723708in,,top]{\color{textcolor}\rmfamily\fontsize{14.000000}{16.800000}\selectfont 10}%
\end{pgfscope}%
\begin{pgfscope}%
\pgfpathrectangle{\pgfqpoint{2.125000in}{8.820930in}}{\pgfqpoint{5.489583in}{0.877907in}}%
\pgfusepath{clip}%
\pgfsetroundcap%
\pgfsetroundjoin%
\pgfsetlinewidth{0.803000pt}%
\definecolor{currentstroke}{rgb}{1.000000,1.000000,1.000000}%
\pgfsetstrokecolor{currentstroke}%
\pgfsetdash{}{0pt}%
\pgfpathmoveto{\pgfqpoint{4.222871in}{8.820930in}}%
\pgfpathlineto{\pgfqpoint{4.222871in}{9.698837in}}%
\pgfusepath{stroke}%
\end{pgfscope}%
\begin{pgfscope}%
\definecolor{textcolor}{rgb}{0.150000,0.150000,0.150000}%
\pgfsetstrokecolor{textcolor}%
\pgfsetfillcolor{textcolor}%
\pgftext[x=4.222871in,y=8.723708in,,top]{\color{textcolor}\rmfamily\fontsize{14.000000}{16.800000}\selectfont 15}%
\end{pgfscope}%
\begin{pgfscope}%
\pgfpathrectangle{\pgfqpoint{2.125000in}{8.820930in}}{\pgfqpoint{5.489583in}{0.877907in}}%
\pgfusepath{clip}%
\pgfsetroundcap%
\pgfsetroundjoin%
\pgfsetlinewidth{0.803000pt}%
\definecolor{currentstroke}{rgb}{1.000000,1.000000,1.000000}%
\pgfsetstrokecolor{currentstroke}%
\pgfsetdash{}{0pt}%
\pgfpathmoveto{\pgfqpoint{4.838986in}{8.820930in}}%
\pgfpathlineto{\pgfqpoint{4.838986in}{9.698837in}}%
\pgfusepath{stroke}%
\end{pgfscope}%
\begin{pgfscope}%
\definecolor{textcolor}{rgb}{0.150000,0.150000,0.150000}%
\pgfsetstrokecolor{textcolor}%
\pgfsetfillcolor{textcolor}%
\pgftext[x=4.838986in,y=8.723708in,,top]{\color{textcolor}\rmfamily\fontsize{14.000000}{16.800000}\selectfont 20}%
\end{pgfscope}%
\begin{pgfscope}%
\pgfpathrectangle{\pgfqpoint{2.125000in}{8.820930in}}{\pgfqpoint{5.489583in}{0.877907in}}%
\pgfusepath{clip}%
\pgfsetroundcap%
\pgfsetroundjoin%
\pgfsetlinewidth{0.803000pt}%
\definecolor{currentstroke}{rgb}{1.000000,1.000000,1.000000}%
\pgfsetstrokecolor{currentstroke}%
\pgfsetdash{}{0pt}%
\pgfpathmoveto{\pgfqpoint{5.455101in}{8.820930in}}%
\pgfpathlineto{\pgfqpoint{5.455101in}{9.698837in}}%
\pgfusepath{stroke}%
\end{pgfscope}%
\begin{pgfscope}%
\definecolor{textcolor}{rgb}{0.150000,0.150000,0.150000}%
\pgfsetstrokecolor{textcolor}%
\pgfsetfillcolor{textcolor}%
\pgftext[x=5.455101in,y=8.723708in,,top]{\color{textcolor}\rmfamily\fontsize{14.000000}{16.800000}\selectfont 25}%
\end{pgfscope}%
\begin{pgfscope}%
\pgfpathrectangle{\pgfqpoint{2.125000in}{8.820930in}}{\pgfqpoint{5.489583in}{0.877907in}}%
\pgfusepath{clip}%
\pgfsetroundcap%
\pgfsetroundjoin%
\pgfsetlinewidth{0.803000pt}%
\definecolor{currentstroke}{rgb}{1.000000,1.000000,1.000000}%
\pgfsetstrokecolor{currentstroke}%
\pgfsetdash{}{0pt}%
\pgfpathmoveto{\pgfqpoint{6.071216in}{8.820930in}}%
\pgfpathlineto{\pgfqpoint{6.071216in}{9.698837in}}%
\pgfusepath{stroke}%
\end{pgfscope}%
\begin{pgfscope}%
\definecolor{textcolor}{rgb}{0.150000,0.150000,0.150000}%
\pgfsetstrokecolor{textcolor}%
\pgfsetfillcolor{textcolor}%
\pgftext[x=6.071216in,y=8.723708in,,top]{\color{textcolor}\rmfamily\fontsize{14.000000}{16.800000}\selectfont 30}%
\end{pgfscope}%
\begin{pgfscope}%
\pgfpathrectangle{\pgfqpoint{2.125000in}{8.820930in}}{\pgfqpoint{5.489583in}{0.877907in}}%
\pgfusepath{clip}%
\pgfsetroundcap%
\pgfsetroundjoin%
\pgfsetlinewidth{0.803000pt}%
\definecolor{currentstroke}{rgb}{1.000000,1.000000,1.000000}%
\pgfsetstrokecolor{currentstroke}%
\pgfsetdash{}{0pt}%
\pgfpathmoveto{\pgfqpoint{6.687330in}{8.820930in}}%
\pgfpathlineto{\pgfqpoint{6.687330in}{9.698837in}}%
\pgfusepath{stroke}%
\end{pgfscope}%
\begin{pgfscope}%
\definecolor{textcolor}{rgb}{0.150000,0.150000,0.150000}%
\pgfsetstrokecolor{textcolor}%
\pgfsetfillcolor{textcolor}%
\pgftext[x=6.687330in,y=8.723708in,,top]{\color{textcolor}\rmfamily\fontsize{14.000000}{16.800000}\selectfont 35}%
\end{pgfscope}%
\begin{pgfscope}%
\pgfpathrectangle{\pgfqpoint{2.125000in}{8.820930in}}{\pgfqpoint{5.489583in}{0.877907in}}%
\pgfusepath{clip}%
\pgfsetroundcap%
\pgfsetroundjoin%
\pgfsetlinewidth{0.803000pt}%
\definecolor{currentstroke}{rgb}{1.000000,1.000000,1.000000}%
\pgfsetstrokecolor{currentstroke}%
\pgfsetdash{}{0pt}%
\pgfpathmoveto{\pgfqpoint{7.303445in}{8.820930in}}%
\pgfpathlineto{\pgfqpoint{7.303445in}{9.698837in}}%
\pgfusepath{stroke}%
\end{pgfscope}%
\begin{pgfscope}%
\definecolor{textcolor}{rgb}{0.150000,0.150000,0.150000}%
\pgfsetstrokecolor{textcolor}%
\pgfsetfillcolor{textcolor}%
\pgftext[x=7.303445in,y=8.723708in,,top]{\color{textcolor}\rmfamily\fontsize{14.000000}{16.800000}\selectfont 40}%
\end{pgfscope}%
\begin{pgfscope}%
\pgfpathrectangle{\pgfqpoint{2.125000in}{8.820930in}}{\pgfqpoint{5.489583in}{0.877907in}}%
\pgfusepath{clip}%
\pgfsetroundcap%
\pgfsetroundjoin%
\pgfsetlinewidth{0.803000pt}%
\definecolor{currentstroke}{rgb}{1.000000,1.000000,1.000000}%
\pgfsetstrokecolor{currentstroke}%
\pgfsetdash{}{0pt}%
\pgfpathmoveto{\pgfqpoint{2.125000in}{9.096129in}}%
\pgfpathlineto{\pgfqpoint{7.614583in}{9.096129in}}%
\pgfusepath{stroke}%
\end{pgfscope}%
\begin{pgfscope}%
\definecolor{textcolor}{rgb}{0.150000,0.150000,0.150000}%
\pgfsetstrokecolor{textcolor}%
\pgfsetfillcolor{textcolor}%
\pgftext[x=1.904066in,y=9.022263in,left,base]{\color{textcolor}\rmfamily\fontsize{14.000000}{16.800000}\selectfont 0}%
\end{pgfscope}%
\begin{pgfscope}%
\pgfpathrectangle{\pgfqpoint{2.125000in}{8.820930in}}{\pgfqpoint{5.489583in}{0.877907in}}%
\pgfusepath{clip}%
\pgfsetroundcap%
\pgfsetroundjoin%
\pgfsetlinewidth{0.803000pt}%
\definecolor{currentstroke}{rgb}{1.000000,1.000000,1.000000}%
\pgfsetstrokecolor{currentstroke}%
\pgfsetdash{}{0pt}%
\pgfpathmoveto{\pgfqpoint{2.125000in}{9.658932in}}%
\pgfpathlineto{\pgfqpoint{7.614583in}{9.658932in}}%
\pgfusepath{stroke}%
\end{pgfscope}%
\begin{pgfscope}%
\definecolor{textcolor}{rgb}{0.150000,0.150000,0.150000}%
\pgfsetstrokecolor{textcolor}%
\pgfsetfillcolor{textcolor}%
\pgftext[x=1.904066in,y=9.585066in,left,base]{\color{textcolor}\rmfamily\fontsize{14.000000}{16.800000}\selectfont 1}%
\end{pgfscope}%
\begin{pgfscope}%
\pgfpathrectangle{\pgfqpoint{2.125000in}{8.820930in}}{\pgfqpoint{5.489583in}{0.877907in}}%
\pgfusepath{clip}%
\pgfsetbuttcap%
\pgfsetroundjoin%
\definecolor{currentfill}{rgb}{0.121569,0.466667,0.705882}%
\pgfsetfillcolor{currentfill}%
\pgfsetfillopacity{0.250000}%
\pgfsetlinewidth{1.003750pt}%
\definecolor{currentstroke}{rgb}{1.000000,1.000000,1.000000}%
\pgfsetstrokecolor{currentstroke}%
\pgfsetstrokeopacity{0.250000}%
\pgfsetdash{}{0pt}%
\pgfpathmoveto{\pgfqpoint{2.436138in}{9.124525in}}%
\pgfpathlineto{\pgfqpoint{2.436138in}{9.067732in}}%
\pgfpathlineto{\pgfqpoint{2.620972in}{9.047062in}}%
\pgfpathlineto{\pgfqpoint{2.744195in}{9.032904in}}%
\pgfpathlineto{\pgfqpoint{2.867418in}{9.021455in}}%
\pgfpathlineto{\pgfqpoint{2.990641in}{9.011608in}}%
\pgfpathlineto{\pgfqpoint{3.113864in}{9.002850in}}%
\pgfpathlineto{\pgfqpoint{3.237087in}{8.994898in}}%
\pgfpathlineto{\pgfqpoint{3.360310in}{8.987579in}}%
\pgfpathlineto{\pgfqpoint{3.483533in}{8.980771in}}%
\pgfpathlineto{\pgfqpoint{3.606756in}{8.974389in}}%
\pgfpathlineto{\pgfqpoint{3.729979in}{8.968367in}}%
\pgfpathlineto{\pgfqpoint{3.853202in}{8.962655in}}%
\pgfpathlineto{\pgfqpoint{3.976425in}{8.957214in}}%
\pgfpathlineto{\pgfqpoint{4.099648in}{8.952012in}}%
\pgfpathlineto{\pgfqpoint{4.222871in}{8.947026in}}%
\pgfpathlineto{\pgfqpoint{4.346094in}{8.942235in}}%
\pgfpathlineto{\pgfqpoint{4.469317in}{8.937620in}}%
\pgfpathlineto{\pgfqpoint{4.592540in}{8.933167in}}%
\pgfpathlineto{\pgfqpoint{4.715763in}{8.928867in}}%
\pgfpathlineto{\pgfqpoint{4.838986in}{8.924707in}}%
\pgfpathlineto{\pgfqpoint{4.962209in}{8.920678in}}%
\pgfpathlineto{\pgfqpoint{5.085432in}{8.916769in}}%
\pgfpathlineto{\pgfqpoint{5.208655in}{8.912974in}}%
\pgfpathlineto{\pgfqpoint{5.331878in}{8.909285in}}%
\pgfpathlineto{\pgfqpoint{5.455101in}{8.905696in}}%
\pgfpathlineto{\pgfqpoint{5.578324in}{8.902202in}}%
\pgfpathlineto{\pgfqpoint{5.701547in}{8.898795in}}%
\pgfpathlineto{\pgfqpoint{5.824770in}{8.895473in}}%
\pgfpathlineto{\pgfqpoint{5.947993in}{8.892229in}}%
\pgfpathlineto{\pgfqpoint{6.071216in}{8.889062in}}%
\pgfpathlineto{\pgfqpoint{6.194439in}{8.885967in}}%
\pgfpathlineto{\pgfqpoint{6.317662in}{8.882942in}}%
\pgfpathlineto{\pgfqpoint{6.440885in}{8.879983in}}%
\pgfpathlineto{\pgfqpoint{6.564108in}{8.877086in}}%
\pgfpathlineto{\pgfqpoint{6.687330in}{8.874248in}}%
\pgfpathlineto{\pgfqpoint{6.810553in}{8.871466in}}%
\pgfpathlineto{\pgfqpoint{6.933776in}{8.868736in}}%
\pgfpathlineto{\pgfqpoint{7.056999in}{8.866056in}}%
\pgfpathlineto{\pgfqpoint{7.180222in}{8.863423in}}%
\pgfpathlineto{\pgfqpoint{7.365057in}{8.860835in}}%
\pgfpathlineto{\pgfqpoint{7.365057in}{9.331422in}}%
\pgfpathlineto{\pgfqpoint{7.365057in}{9.331422in}}%
\pgfpathlineto{\pgfqpoint{7.180222in}{9.328834in}}%
\pgfpathlineto{\pgfqpoint{7.056999in}{9.326201in}}%
\pgfpathlineto{\pgfqpoint{6.933776in}{9.323521in}}%
\pgfpathlineto{\pgfqpoint{6.810553in}{9.320791in}}%
\pgfpathlineto{\pgfqpoint{6.687330in}{9.318010in}}%
\pgfpathlineto{\pgfqpoint{6.564108in}{9.315172in}}%
\pgfpathlineto{\pgfqpoint{6.440885in}{9.312275in}}%
\pgfpathlineto{\pgfqpoint{6.317662in}{9.309315in}}%
\pgfpathlineto{\pgfqpoint{6.194439in}{9.306290in}}%
\pgfpathlineto{\pgfqpoint{6.071216in}{9.303196in}}%
\pgfpathlineto{\pgfqpoint{5.947993in}{9.300028in}}%
\pgfpathlineto{\pgfqpoint{5.824770in}{9.296784in}}%
\pgfpathlineto{\pgfqpoint{5.701547in}{9.293462in}}%
\pgfpathlineto{\pgfqpoint{5.578324in}{9.290056in}}%
\pgfpathlineto{\pgfqpoint{5.455101in}{9.286561in}}%
\pgfpathlineto{\pgfqpoint{5.331878in}{9.282972in}}%
\pgfpathlineto{\pgfqpoint{5.208655in}{9.279283in}}%
\pgfpathlineto{\pgfqpoint{5.085432in}{9.275488in}}%
\pgfpathlineto{\pgfqpoint{4.962209in}{9.271580in}}%
\pgfpathlineto{\pgfqpoint{4.838986in}{9.267550in}}%
\pgfpathlineto{\pgfqpoint{4.715763in}{9.263391in}}%
\pgfpathlineto{\pgfqpoint{4.592540in}{9.259090in}}%
\pgfpathlineto{\pgfqpoint{4.469317in}{9.254638in}}%
\pgfpathlineto{\pgfqpoint{4.346094in}{9.250022in}}%
\pgfpathlineto{\pgfqpoint{4.222871in}{9.245231in}}%
\pgfpathlineto{\pgfqpoint{4.099648in}{9.240245in}}%
\pgfpathlineto{\pgfqpoint{3.976425in}{9.235043in}}%
\pgfpathlineto{\pgfqpoint{3.853202in}{9.229602in}}%
\pgfpathlineto{\pgfqpoint{3.729979in}{9.223890in}}%
\pgfpathlineto{\pgfqpoint{3.606756in}{9.217868in}}%
\pgfpathlineto{\pgfqpoint{3.483533in}{9.211486in}}%
\pgfpathlineto{\pgfqpoint{3.360310in}{9.204678in}}%
\pgfpathlineto{\pgfqpoint{3.237087in}{9.197359in}}%
\pgfpathlineto{\pgfqpoint{3.113864in}{9.189408in}}%
\pgfpathlineto{\pgfqpoint{2.990641in}{9.180649in}}%
\pgfpathlineto{\pgfqpoint{2.867418in}{9.170802in}}%
\pgfpathlineto{\pgfqpoint{2.744195in}{9.159353in}}%
\pgfpathlineto{\pgfqpoint{2.620972in}{9.145195in}}%
\pgfpathlineto{\pgfqpoint{2.436138in}{9.124525in}}%
\pgfpathclose%
\pgfusepath{stroke,fill}%
\end{pgfscope}%
\begin{pgfscope}%
\pgfpathrectangle{\pgfqpoint{2.125000in}{8.820930in}}{\pgfqpoint{5.489583in}{0.877907in}}%
\pgfusepath{clip}%
\pgfsetbuttcap%
\pgfsetroundjoin%
\pgfsetlinewidth{1.505625pt}%
\definecolor{currentstroke}{rgb}{0.000000,0.000000,0.000000}%
\pgfsetstrokecolor{currentstroke}%
\pgfsetdash{}{0pt}%
\pgfpathmoveto{\pgfqpoint{2.374527in}{9.096129in}}%
\pgfpathlineto{\pgfqpoint{2.374527in}{9.658932in}}%
\pgfusepath{stroke}%
\end{pgfscope}%
\begin{pgfscope}%
\pgfpathrectangle{\pgfqpoint{2.125000in}{8.820930in}}{\pgfqpoint{5.489583in}{0.877907in}}%
\pgfusepath{clip}%
\pgfsetbuttcap%
\pgfsetroundjoin%
\pgfsetlinewidth{1.505625pt}%
\definecolor{currentstroke}{rgb}{0.000000,0.000000,0.000000}%
\pgfsetstrokecolor{currentstroke}%
\pgfsetdash{}{0pt}%
\pgfpathmoveto{\pgfqpoint{2.497749in}{9.096129in}}%
\pgfpathlineto{\pgfqpoint{2.497749in}{9.656919in}}%
\pgfusepath{stroke}%
\end{pgfscope}%
\begin{pgfscope}%
\pgfpathrectangle{\pgfqpoint{2.125000in}{8.820930in}}{\pgfqpoint{5.489583in}{0.877907in}}%
\pgfusepath{clip}%
\pgfsetbuttcap%
\pgfsetroundjoin%
\pgfsetlinewidth{1.505625pt}%
\definecolor{currentstroke}{rgb}{0.000000,0.000000,0.000000}%
\pgfsetstrokecolor{currentstroke}%
\pgfsetdash{}{0pt}%
\pgfpathmoveto{\pgfqpoint{2.620972in}{9.096129in}}%
\pgfpathlineto{\pgfqpoint{2.620972in}{9.654932in}}%
\pgfusepath{stroke}%
\end{pgfscope}%
\begin{pgfscope}%
\pgfpathrectangle{\pgfqpoint{2.125000in}{8.820930in}}{\pgfqpoint{5.489583in}{0.877907in}}%
\pgfusepath{clip}%
\pgfsetbuttcap%
\pgfsetroundjoin%
\pgfsetlinewidth{1.505625pt}%
\definecolor{currentstroke}{rgb}{0.000000,0.000000,0.000000}%
\pgfsetstrokecolor{currentstroke}%
\pgfsetdash{}{0pt}%
\pgfpathmoveto{\pgfqpoint{2.744195in}{9.096129in}}%
\pgfpathlineto{\pgfqpoint{2.744195in}{9.652972in}}%
\pgfusepath{stroke}%
\end{pgfscope}%
\begin{pgfscope}%
\pgfpathrectangle{\pgfqpoint{2.125000in}{8.820930in}}{\pgfqpoint{5.489583in}{0.877907in}}%
\pgfusepath{clip}%
\pgfsetbuttcap%
\pgfsetroundjoin%
\pgfsetlinewidth{1.505625pt}%
\definecolor{currentstroke}{rgb}{0.000000,0.000000,0.000000}%
\pgfsetstrokecolor{currentstroke}%
\pgfsetdash{}{0pt}%
\pgfpathmoveto{\pgfqpoint{2.867418in}{9.096129in}}%
\pgfpathlineto{\pgfqpoint{2.867418in}{9.651012in}}%
\pgfusepath{stroke}%
\end{pgfscope}%
\begin{pgfscope}%
\pgfpathrectangle{\pgfqpoint{2.125000in}{8.820930in}}{\pgfqpoint{5.489583in}{0.877907in}}%
\pgfusepath{clip}%
\pgfsetbuttcap%
\pgfsetroundjoin%
\pgfsetlinewidth{1.505625pt}%
\definecolor{currentstroke}{rgb}{0.000000,0.000000,0.000000}%
\pgfsetstrokecolor{currentstroke}%
\pgfsetdash{}{0pt}%
\pgfpathmoveto{\pgfqpoint{2.990641in}{9.096129in}}%
\pgfpathlineto{\pgfqpoint{2.990641in}{9.649181in}}%
\pgfusepath{stroke}%
\end{pgfscope}%
\begin{pgfscope}%
\pgfpathrectangle{\pgfqpoint{2.125000in}{8.820930in}}{\pgfqpoint{5.489583in}{0.877907in}}%
\pgfusepath{clip}%
\pgfsetbuttcap%
\pgfsetroundjoin%
\pgfsetlinewidth{1.505625pt}%
\definecolor{currentstroke}{rgb}{0.000000,0.000000,0.000000}%
\pgfsetstrokecolor{currentstroke}%
\pgfsetdash{}{0pt}%
\pgfpathmoveto{\pgfqpoint{3.113864in}{9.096129in}}%
\pgfpathlineto{\pgfqpoint{3.113864in}{9.647275in}}%
\pgfusepath{stroke}%
\end{pgfscope}%
\begin{pgfscope}%
\pgfpathrectangle{\pgfqpoint{2.125000in}{8.820930in}}{\pgfqpoint{5.489583in}{0.877907in}}%
\pgfusepath{clip}%
\pgfsetbuttcap%
\pgfsetroundjoin%
\pgfsetlinewidth{1.505625pt}%
\definecolor{currentstroke}{rgb}{0.000000,0.000000,0.000000}%
\pgfsetstrokecolor{currentstroke}%
\pgfsetdash{}{0pt}%
\pgfpathmoveto{\pgfqpoint{3.237087in}{9.096129in}}%
\pgfpathlineto{\pgfqpoint{3.237087in}{9.645299in}}%
\pgfusepath{stroke}%
\end{pgfscope}%
\begin{pgfscope}%
\pgfpathrectangle{\pgfqpoint{2.125000in}{8.820930in}}{\pgfqpoint{5.489583in}{0.877907in}}%
\pgfusepath{clip}%
\pgfsetbuttcap%
\pgfsetroundjoin%
\pgfsetlinewidth{1.505625pt}%
\definecolor{currentstroke}{rgb}{0.000000,0.000000,0.000000}%
\pgfsetstrokecolor{currentstroke}%
\pgfsetdash{}{0pt}%
\pgfpathmoveto{\pgfqpoint{3.360310in}{9.096129in}}%
\pgfpathlineto{\pgfqpoint{3.360310in}{9.643302in}}%
\pgfusepath{stroke}%
\end{pgfscope}%
\begin{pgfscope}%
\pgfpathrectangle{\pgfqpoint{2.125000in}{8.820930in}}{\pgfqpoint{5.489583in}{0.877907in}}%
\pgfusepath{clip}%
\pgfsetbuttcap%
\pgfsetroundjoin%
\pgfsetlinewidth{1.505625pt}%
\definecolor{currentstroke}{rgb}{0.000000,0.000000,0.000000}%
\pgfsetstrokecolor{currentstroke}%
\pgfsetdash{}{0pt}%
\pgfpathmoveto{\pgfqpoint{3.483533in}{9.096129in}}%
\pgfpathlineto{\pgfqpoint{3.483533in}{9.641283in}}%
\pgfusepath{stroke}%
\end{pgfscope}%
\begin{pgfscope}%
\pgfpathrectangle{\pgfqpoint{2.125000in}{8.820930in}}{\pgfqpoint{5.489583in}{0.877907in}}%
\pgfusepath{clip}%
\pgfsetbuttcap%
\pgfsetroundjoin%
\pgfsetlinewidth{1.505625pt}%
\definecolor{currentstroke}{rgb}{0.000000,0.000000,0.000000}%
\pgfsetstrokecolor{currentstroke}%
\pgfsetdash{}{0pt}%
\pgfpathmoveto{\pgfqpoint{3.606756in}{9.096129in}}%
\pgfpathlineto{\pgfqpoint{3.606756in}{9.639340in}}%
\pgfusepath{stroke}%
\end{pgfscope}%
\begin{pgfscope}%
\pgfpathrectangle{\pgfqpoint{2.125000in}{8.820930in}}{\pgfqpoint{5.489583in}{0.877907in}}%
\pgfusepath{clip}%
\pgfsetbuttcap%
\pgfsetroundjoin%
\pgfsetlinewidth{1.505625pt}%
\definecolor{currentstroke}{rgb}{0.000000,0.000000,0.000000}%
\pgfsetstrokecolor{currentstroke}%
\pgfsetdash{}{0pt}%
\pgfpathmoveto{\pgfqpoint{3.729979in}{9.096129in}}%
\pgfpathlineto{\pgfqpoint{3.729979in}{9.637500in}}%
\pgfusepath{stroke}%
\end{pgfscope}%
\begin{pgfscope}%
\pgfpathrectangle{\pgfqpoint{2.125000in}{8.820930in}}{\pgfqpoint{5.489583in}{0.877907in}}%
\pgfusepath{clip}%
\pgfsetbuttcap%
\pgfsetroundjoin%
\pgfsetlinewidth{1.505625pt}%
\definecolor{currentstroke}{rgb}{0.000000,0.000000,0.000000}%
\pgfsetstrokecolor{currentstroke}%
\pgfsetdash{}{0pt}%
\pgfpathmoveto{\pgfqpoint{3.853202in}{9.096129in}}%
\pgfpathlineto{\pgfqpoint{3.853202in}{9.635664in}}%
\pgfusepath{stroke}%
\end{pgfscope}%
\begin{pgfscope}%
\pgfpathrectangle{\pgfqpoint{2.125000in}{8.820930in}}{\pgfqpoint{5.489583in}{0.877907in}}%
\pgfusepath{clip}%
\pgfsetbuttcap%
\pgfsetroundjoin%
\pgfsetlinewidth{1.505625pt}%
\definecolor{currentstroke}{rgb}{0.000000,0.000000,0.000000}%
\pgfsetstrokecolor{currentstroke}%
\pgfsetdash{}{0pt}%
\pgfpathmoveto{\pgfqpoint{3.976425in}{9.096129in}}%
\pgfpathlineto{\pgfqpoint{3.976425in}{9.633884in}}%
\pgfusepath{stroke}%
\end{pgfscope}%
\begin{pgfscope}%
\pgfpathrectangle{\pgfqpoint{2.125000in}{8.820930in}}{\pgfqpoint{5.489583in}{0.877907in}}%
\pgfusepath{clip}%
\pgfsetbuttcap%
\pgfsetroundjoin%
\pgfsetlinewidth{1.505625pt}%
\definecolor{currentstroke}{rgb}{0.000000,0.000000,0.000000}%
\pgfsetstrokecolor{currentstroke}%
\pgfsetdash{}{0pt}%
\pgfpathmoveto{\pgfqpoint{4.099648in}{9.096129in}}%
\pgfpathlineto{\pgfqpoint{4.099648in}{9.632011in}}%
\pgfusepath{stroke}%
\end{pgfscope}%
\begin{pgfscope}%
\pgfpathrectangle{\pgfqpoint{2.125000in}{8.820930in}}{\pgfqpoint{5.489583in}{0.877907in}}%
\pgfusepath{clip}%
\pgfsetbuttcap%
\pgfsetroundjoin%
\pgfsetlinewidth{1.505625pt}%
\definecolor{currentstroke}{rgb}{0.000000,0.000000,0.000000}%
\pgfsetstrokecolor{currentstroke}%
\pgfsetdash{}{0pt}%
\pgfpathmoveto{\pgfqpoint{4.222871in}{9.096129in}}%
\pgfpathlineto{\pgfqpoint{4.222871in}{9.630098in}}%
\pgfusepath{stroke}%
\end{pgfscope}%
\begin{pgfscope}%
\pgfpathrectangle{\pgfqpoint{2.125000in}{8.820930in}}{\pgfqpoint{5.489583in}{0.877907in}}%
\pgfusepath{clip}%
\pgfsetbuttcap%
\pgfsetroundjoin%
\pgfsetlinewidth{1.505625pt}%
\definecolor{currentstroke}{rgb}{0.000000,0.000000,0.000000}%
\pgfsetstrokecolor{currentstroke}%
\pgfsetdash{}{0pt}%
\pgfpathmoveto{\pgfqpoint{4.346094in}{9.096129in}}%
\pgfpathlineto{\pgfqpoint{4.346094in}{9.628279in}}%
\pgfusepath{stroke}%
\end{pgfscope}%
\begin{pgfscope}%
\pgfpathrectangle{\pgfqpoint{2.125000in}{8.820930in}}{\pgfqpoint{5.489583in}{0.877907in}}%
\pgfusepath{clip}%
\pgfsetbuttcap%
\pgfsetroundjoin%
\pgfsetlinewidth{1.505625pt}%
\definecolor{currentstroke}{rgb}{0.000000,0.000000,0.000000}%
\pgfsetstrokecolor{currentstroke}%
\pgfsetdash{}{0pt}%
\pgfpathmoveto{\pgfqpoint{4.469317in}{9.096129in}}%
\pgfpathlineto{\pgfqpoint{4.469317in}{9.626323in}}%
\pgfusepath{stroke}%
\end{pgfscope}%
\begin{pgfscope}%
\pgfpathrectangle{\pgfqpoint{2.125000in}{8.820930in}}{\pgfqpoint{5.489583in}{0.877907in}}%
\pgfusepath{clip}%
\pgfsetbuttcap%
\pgfsetroundjoin%
\pgfsetlinewidth{1.505625pt}%
\definecolor{currentstroke}{rgb}{0.000000,0.000000,0.000000}%
\pgfsetstrokecolor{currentstroke}%
\pgfsetdash{}{0pt}%
\pgfpathmoveto{\pgfqpoint{4.592540in}{9.096129in}}%
\pgfpathlineto{\pgfqpoint{4.592540in}{9.624287in}}%
\pgfusepath{stroke}%
\end{pgfscope}%
\begin{pgfscope}%
\pgfpathrectangle{\pgfqpoint{2.125000in}{8.820930in}}{\pgfqpoint{5.489583in}{0.877907in}}%
\pgfusepath{clip}%
\pgfsetbuttcap%
\pgfsetroundjoin%
\pgfsetlinewidth{1.505625pt}%
\definecolor{currentstroke}{rgb}{0.000000,0.000000,0.000000}%
\pgfsetstrokecolor{currentstroke}%
\pgfsetdash{}{0pt}%
\pgfpathmoveto{\pgfqpoint{4.715763in}{9.096129in}}%
\pgfpathlineto{\pgfqpoint{4.715763in}{9.622164in}}%
\pgfusepath{stroke}%
\end{pgfscope}%
\begin{pgfscope}%
\pgfpathrectangle{\pgfqpoint{2.125000in}{8.820930in}}{\pgfqpoint{5.489583in}{0.877907in}}%
\pgfusepath{clip}%
\pgfsetbuttcap%
\pgfsetroundjoin%
\pgfsetlinewidth{1.505625pt}%
\definecolor{currentstroke}{rgb}{0.000000,0.000000,0.000000}%
\pgfsetstrokecolor{currentstroke}%
\pgfsetdash{}{0pt}%
\pgfpathmoveto{\pgfqpoint{4.838986in}{9.096129in}}%
\pgfpathlineto{\pgfqpoint{4.838986in}{9.620059in}}%
\pgfusepath{stroke}%
\end{pgfscope}%
\begin{pgfscope}%
\pgfpathrectangle{\pgfqpoint{2.125000in}{8.820930in}}{\pgfqpoint{5.489583in}{0.877907in}}%
\pgfusepath{clip}%
\pgfsetbuttcap%
\pgfsetroundjoin%
\pgfsetlinewidth{1.505625pt}%
\definecolor{currentstroke}{rgb}{0.000000,0.000000,0.000000}%
\pgfsetstrokecolor{currentstroke}%
\pgfsetdash{}{0pt}%
\pgfpathmoveto{\pgfqpoint{4.962209in}{9.096129in}}%
\pgfpathlineto{\pgfqpoint{4.962209in}{9.618016in}}%
\pgfusepath{stroke}%
\end{pgfscope}%
\begin{pgfscope}%
\pgfpathrectangle{\pgfqpoint{2.125000in}{8.820930in}}{\pgfqpoint{5.489583in}{0.877907in}}%
\pgfusepath{clip}%
\pgfsetbuttcap%
\pgfsetroundjoin%
\pgfsetlinewidth{1.505625pt}%
\definecolor{currentstroke}{rgb}{0.000000,0.000000,0.000000}%
\pgfsetstrokecolor{currentstroke}%
\pgfsetdash{}{0pt}%
\pgfpathmoveto{\pgfqpoint{5.085432in}{9.096129in}}%
\pgfpathlineto{\pgfqpoint{5.085432in}{9.615977in}}%
\pgfusepath{stroke}%
\end{pgfscope}%
\begin{pgfscope}%
\pgfpathrectangle{\pgfqpoint{2.125000in}{8.820930in}}{\pgfqpoint{5.489583in}{0.877907in}}%
\pgfusepath{clip}%
\pgfsetbuttcap%
\pgfsetroundjoin%
\pgfsetlinewidth{1.505625pt}%
\definecolor{currentstroke}{rgb}{0.000000,0.000000,0.000000}%
\pgfsetstrokecolor{currentstroke}%
\pgfsetdash{}{0pt}%
\pgfpathmoveto{\pgfqpoint{5.208655in}{9.096129in}}%
\pgfpathlineto{\pgfqpoint{5.208655in}{9.613882in}}%
\pgfusepath{stroke}%
\end{pgfscope}%
\begin{pgfscope}%
\pgfpathrectangle{\pgfqpoint{2.125000in}{8.820930in}}{\pgfqpoint{5.489583in}{0.877907in}}%
\pgfusepath{clip}%
\pgfsetbuttcap%
\pgfsetroundjoin%
\pgfsetlinewidth{1.505625pt}%
\definecolor{currentstroke}{rgb}{0.000000,0.000000,0.000000}%
\pgfsetstrokecolor{currentstroke}%
\pgfsetdash{}{0pt}%
\pgfpathmoveto{\pgfqpoint{5.331878in}{9.096129in}}%
\pgfpathlineto{\pgfqpoint{5.331878in}{9.611825in}}%
\pgfusepath{stroke}%
\end{pgfscope}%
\begin{pgfscope}%
\pgfpathrectangle{\pgfqpoint{2.125000in}{8.820930in}}{\pgfqpoint{5.489583in}{0.877907in}}%
\pgfusepath{clip}%
\pgfsetbuttcap%
\pgfsetroundjoin%
\pgfsetlinewidth{1.505625pt}%
\definecolor{currentstroke}{rgb}{0.000000,0.000000,0.000000}%
\pgfsetstrokecolor{currentstroke}%
\pgfsetdash{}{0pt}%
\pgfpathmoveto{\pgfqpoint{5.455101in}{9.096129in}}%
\pgfpathlineto{\pgfqpoint{5.455101in}{9.609737in}}%
\pgfusepath{stroke}%
\end{pgfscope}%
\begin{pgfscope}%
\pgfpathrectangle{\pgfqpoint{2.125000in}{8.820930in}}{\pgfqpoint{5.489583in}{0.877907in}}%
\pgfusepath{clip}%
\pgfsetbuttcap%
\pgfsetroundjoin%
\pgfsetlinewidth{1.505625pt}%
\definecolor{currentstroke}{rgb}{0.000000,0.000000,0.000000}%
\pgfsetstrokecolor{currentstroke}%
\pgfsetdash{}{0pt}%
\pgfpathmoveto{\pgfqpoint{5.578324in}{9.096129in}}%
\pgfpathlineto{\pgfqpoint{5.578324in}{9.607754in}}%
\pgfusepath{stroke}%
\end{pgfscope}%
\begin{pgfscope}%
\pgfpathrectangle{\pgfqpoint{2.125000in}{8.820930in}}{\pgfqpoint{5.489583in}{0.877907in}}%
\pgfusepath{clip}%
\pgfsetbuttcap%
\pgfsetroundjoin%
\pgfsetlinewidth{1.505625pt}%
\definecolor{currentstroke}{rgb}{0.000000,0.000000,0.000000}%
\pgfsetstrokecolor{currentstroke}%
\pgfsetdash{}{0pt}%
\pgfpathmoveto{\pgfqpoint{5.701547in}{9.096129in}}%
\pgfpathlineto{\pgfqpoint{5.701547in}{9.605766in}}%
\pgfusepath{stroke}%
\end{pgfscope}%
\begin{pgfscope}%
\pgfpathrectangle{\pgfqpoint{2.125000in}{8.820930in}}{\pgfqpoint{5.489583in}{0.877907in}}%
\pgfusepath{clip}%
\pgfsetbuttcap%
\pgfsetroundjoin%
\pgfsetlinewidth{1.505625pt}%
\definecolor{currentstroke}{rgb}{0.000000,0.000000,0.000000}%
\pgfsetstrokecolor{currentstroke}%
\pgfsetdash{}{0pt}%
\pgfpathmoveto{\pgfqpoint{5.824770in}{9.096129in}}%
\pgfpathlineto{\pgfqpoint{5.824770in}{9.603797in}}%
\pgfusepath{stroke}%
\end{pgfscope}%
\begin{pgfscope}%
\pgfpathrectangle{\pgfqpoint{2.125000in}{8.820930in}}{\pgfqpoint{5.489583in}{0.877907in}}%
\pgfusepath{clip}%
\pgfsetbuttcap%
\pgfsetroundjoin%
\pgfsetlinewidth{1.505625pt}%
\definecolor{currentstroke}{rgb}{0.000000,0.000000,0.000000}%
\pgfsetstrokecolor{currentstroke}%
\pgfsetdash{}{0pt}%
\pgfpathmoveto{\pgfqpoint{5.947993in}{9.096129in}}%
\pgfpathlineto{\pgfqpoint{5.947993in}{9.601791in}}%
\pgfusepath{stroke}%
\end{pgfscope}%
\begin{pgfscope}%
\pgfpathrectangle{\pgfqpoint{2.125000in}{8.820930in}}{\pgfqpoint{5.489583in}{0.877907in}}%
\pgfusepath{clip}%
\pgfsetbuttcap%
\pgfsetroundjoin%
\pgfsetlinewidth{1.505625pt}%
\definecolor{currentstroke}{rgb}{0.000000,0.000000,0.000000}%
\pgfsetstrokecolor{currentstroke}%
\pgfsetdash{}{0pt}%
\pgfpathmoveto{\pgfqpoint{6.071216in}{9.096129in}}%
\pgfpathlineto{\pgfqpoint{6.071216in}{9.599677in}}%
\pgfusepath{stroke}%
\end{pgfscope}%
\begin{pgfscope}%
\pgfpathrectangle{\pgfqpoint{2.125000in}{8.820930in}}{\pgfqpoint{5.489583in}{0.877907in}}%
\pgfusepath{clip}%
\pgfsetbuttcap%
\pgfsetroundjoin%
\pgfsetlinewidth{1.505625pt}%
\definecolor{currentstroke}{rgb}{0.000000,0.000000,0.000000}%
\pgfsetstrokecolor{currentstroke}%
\pgfsetdash{}{0pt}%
\pgfpathmoveto{\pgfqpoint{6.194439in}{9.096129in}}%
\pgfpathlineto{\pgfqpoint{6.194439in}{9.597661in}}%
\pgfusepath{stroke}%
\end{pgfscope}%
\begin{pgfscope}%
\pgfpathrectangle{\pgfqpoint{2.125000in}{8.820930in}}{\pgfqpoint{5.489583in}{0.877907in}}%
\pgfusepath{clip}%
\pgfsetbuttcap%
\pgfsetroundjoin%
\pgfsetlinewidth{1.505625pt}%
\definecolor{currentstroke}{rgb}{0.000000,0.000000,0.000000}%
\pgfsetstrokecolor{currentstroke}%
\pgfsetdash{}{0pt}%
\pgfpathmoveto{\pgfqpoint{6.317662in}{9.096129in}}%
\pgfpathlineto{\pgfqpoint{6.317662in}{9.595690in}}%
\pgfusepath{stroke}%
\end{pgfscope}%
\begin{pgfscope}%
\pgfpathrectangle{\pgfqpoint{2.125000in}{8.820930in}}{\pgfqpoint{5.489583in}{0.877907in}}%
\pgfusepath{clip}%
\pgfsetbuttcap%
\pgfsetroundjoin%
\pgfsetlinewidth{1.505625pt}%
\definecolor{currentstroke}{rgb}{0.000000,0.000000,0.000000}%
\pgfsetstrokecolor{currentstroke}%
\pgfsetdash{}{0pt}%
\pgfpathmoveto{\pgfqpoint{6.440885in}{9.096129in}}%
\pgfpathlineto{\pgfqpoint{6.440885in}{9.593759in}}%
\pgfusepath{stroke}%
\end{pgfscope}%
\begin{pgfscope}%
\pgfpathrectangle{\pgfqpoint{2.125000in}{8.820930in}}{\pgfqpoint{5.489583in}{0.877907in}}%
\pgfusepath{clip}%
\pgfsetbuttcap%
\pgfsetroundjoin%
\pgfsetlinewidth{1.505625pt}%
\definecolor{currentstroke}{rgb}{0.000000,0.000000,0.000000}%
\pgfsetstrokecolor{currentstroke}%
\pgfsetdash{}{0pt}%
\pgfpathmoveto{\pgfqpoint{6.564108in}{9.096129in}}%
\pgfpathlineto{\pgfqpoint{6.564108in}{9.591879in}}%
\pgfusepath{stroke}%
\end{pgfscope}%
\begin{pgfscope}%
\pgfpathrectangle{\pgfqpoint{2.125000in}{8.820930in}}{\pgfqpoint{5.489583in}{0.877907in}}%
\pgfusepath{clip}%
\pgfsetbuttcap%
\pgfsetroundjoin%
\pgfsetlinewidth{1.505625pt}%
\definecolor{currentstroke}{rgb}{0.000000,0.000000,0.000000}%
\pgfsetstrokecolor{currentstroke}%
\pgfsetdash{}{0pt}%
\pgfpathmoveto{\pgfqpoint{6.687330in}{9.096129in}}%
\pgfpathlineto{\pgfqpoint{6.687330in}{9.590064in}}%
\pgfusepath{stroke}%
\end{pgfscope}%
\begin{pgfscope}%
\pgfpathrectangle{\pgfqpoint{2.125000in}{8.820930in}}{\pgfqpoint{5.489583in}{0.877907in}}%
\pgfusepath{clip}%
\pgfsetbuttcap%
\pgfsetroundjoin%
\pgfsetlinewidth{1.505625pt}%
\definecolor{currentstroke}{rgb}{0.000000,0.000000,0.000000}%
\pgfsetstrokecolor{currentstroke}%
\pgfsetdash{}{0pt}%
\pgfpathmoveto{\pgfqpoint{6.810553in}{9.096129in}}%
\pgfpathlineto{\pgfqpoint{6.810553in}{9.588455in}}%
\pgfusepath{stroke}%
\end{pgfscope}%
\begin{pgfscope}%
\pgfpathrectangle{\pgfqpoint{2.125000in}{8.820930in}}{\pgfqpoint{5.489583in}{0.877907in}}%
\pgfusepath{clip}%
\pgfsetbuttcap%
\pgfsetroundjoin%
\pgfsetlinewidth{1.505625pt}%
\definecolor{currentstroke}{rgb}{0.000000,0.000000,0.000000}%
\pgfsetstrokecolor{currentstroke}%
\pgfsetdash{}{0pt}%
\pgfpathmoveto{\pgfqpoint{6.933776in}{9.096129in}}%
\pgfpathlineto{\pgfqpoint{6.933776in}{9.586843in}}%
\pgfusepath{stroke}%
\end{pgfscope}%
\begin{pgfscope}%
\pgfpathrectangle{\pgfqpoint{2.125000in}{8.820930in}}{\pgfqpoint{5.489583in}{0.877907in}}%
\pgfusepath{clip}%
\pgfsetbuttcap%
\pgfsetroundjoin%
\pgfsetlinewidth{1.505625pt}%
\definecolor{currentstroke}{rgb}{0.000000,0.000000,0.000000}%
\pgfsetstrokecolor{currentstroke}%
\pgfsetdash{}{0pt}%
\pgfpathmoveto{\pgfqpoint{7.056999in}{9.096129in}}%
\pgfpathlineto{\pgfqpoint{7.056999in}{9.585317in}}%
\pgfusepath{stroke}%
\end{pgfscope}%
\begin{pgfscope}%
\pgfpathrectangle{\pgfqpoint{2.125000in}{8.820930in}}{\pgfqpoint{5.489583in}{0.877907in}}%
\pgfusepath{clip}%
\pgfsetbuttcap%
\pgfsetroundjoin%
\pgfsetlinewidth{1.505625pt}%
\definecolor{currentstroke}{rgb}{0.000000,0.000000,0.000000}%
\pgfsetstrokecolor{currentstroke}%
\pgfsetdash{}{0pt}%
\pgfpathmoveto{\pgfqpoint{7.180222in}{9.096129in}}%
\pgfpathlineto{\pgfqpoint{7.180222in}{9.583876in}}%
\pgfusepath{stroke}%
\end{pgfscope}%
\begin{pgfscope}%
\pgfpathrectangle{\pgfqpoint{2.125000in}{8.820930in}}{\pgfqpoint{5.489583in}{0.877907in}}%
\pgfusepath{clip}%
\pgfsetbuttcap%
\pgfsetroundjoin%
\pgfsetlinewidth{1.505625pt}%
\definecolor{currentstroke}{rgb}{0.000000,0.000000,0.000000}%
\pgfsetstrokecolor{currentstroke}%
\pgfsetdash{}{0pt}%
\pgfpathmoveto{\pgfqpoint{7.303445in}{9.096129in}}%
\pgfpathlineto{\pgfqpoint{7.303445in}{9.582497in}}%
\pgfusepath{stroke}%
\end{pgfscope}%
\begin{pgfscope}%
\pgfpathrectangle{\pgfqpoint{2.125000in}{8.820930in}}{\pgfqpoint{5.489583in}{0.877907in}}%
\pgfusepath{clip}%
\pgfsetroundcap%
\pgfsetroundjoin%
\pgfsetlinewidth{1.505625pt}%
\definecolor{currentstroke}{rgb}{0.121569,0.466667,0.705882}%
\pgfsetstrokecolor{currentstroke}%
\pgfsetdash{}{0pt}%
\pgfpathmoveto{\pgfqpoint{2.125000in}{9.096129in}}%
\pgfpathlineto{\pgfqpoint{7.614583in}{9.096129in}}%
\pgfusepath{stroke}%
\end{pgfscope}%
\begin{pgfscope}%
\pgfpathrectangle{\pgfqpoint{2.125000in}{8.820930in}}{\pgfqpoint{5.489583in}{0.877907in}}%
\pgfusepath{clip}%
\pgfsetbuttcap%
\pgfsetroundjoin%
\definecolor{currentfill}{rgb}{0.121569,0.466667,0.705882}%
\pgfsetfillcolor{currentfill}%
\pgfsetlinewidth{1.003750pt}%
\definecolor{currentstroke}{rgb}{0.121569,0.466667,0.705882}%
\pgfsetstrokecolor{currentstroke}%
\pgfsetdash{}{0pt}%
\pgfsys@defobject{currentmarker}{\pgfqpoint{-0.034722in}{-0.034722in}}{\pgfqpoint{0.034722in}{0.034722in}}{%
\pgfpathmoveto{\pgfqpoint{0.000000in}{-0.034722in}}%
\pgfpathcurveto{\pgfqpoint{0.009208in}{-0.034722in}}{\pgfqpoint{0.018041in}{-0.031064in}}{\pgfqpoint{0.024552in}{-0.024552in}}%
\pgfpathcurveto{\pgfqpoint{0.031064in}{-0.018041in}}{\pgfqpoint{0.034722in}{-0.009208in}}{\pgfqpoint{0.034722in}{0.000000in}}%
\pgfpathcurveto{\pgfqpoint{0.034722in}{0.009208in}}{\pgfqpoint{0.031064in}{0.018041in}}{\pgfqpoint{0.024552in}{0.024552in}}%
\pgfpathcurveto{\pgfqpoint{0.018041in}{0.031064in}}{\pgfqpoint{0.009208in}{0.034722in}}{\pgfqpoint{0.000000in}{0.034722in}}%
\pgfpathcurveto{\pgfqpoint{-0.009208in}{0.034722in}}{\pgfqpoint{-0.018041in}{0.031064in}}{\pgfqpoint{-0.024552in}{0.024552in}}%
\pgfpathcurveto{\pgfqpoint{-0.031064in}{0.018041in}}{\pgfqpoint{-0.034722in}{0.009208in}}{\pgfqpoint{-0.034722in}{0.000000in}}%
\pgfpathcurveto{\pgfqpoint{-0.034722in}{-0.009208in}}{\pgfqpoint{-0.031064in}{-0.018041in}}{\pgfqpoint{-0.024552in}{-0.024552in}}%
\pgfpathcurveto{\pgfqpoint{-0.018041in}{-0.031064in}}{\pgfqpoint{-0.009208in}{-0.034722in}}{\pgfqpoint{0.000000in}{-0.034722in}}%
\pgfpathclose%
\pgfusepath{stroke,fill}%
}%
\begin{pgfscope}%
\pgfsys@transformshift{2.374527in}{9.658932in}%
\pgfsys@useobject{currentmarker}{}%
\end{pgfscope}%
\begin{pgfscope}%
\pgfsys@transformshift{2.497749in}{9.656919in}%
\pgfsys@useobject{currentmarker}{}%
\end{pgfscope}%
\begin{pgfscope}%
\pgfsys@transformshift{2.620972in}{9.654932in}%
\pgfsys@useobject{currentmarker}{}%
\end{pgfscope}%
\begin{pgfscope}%
\pgfsys@transformshift{2.744195in}{9.652972in}%
\pgfsys@useobject{currentmarker}{}%
\end{pgfscope}%
\begin{pgfscope}%
\pgfsys@transformshift{2.867418in}{9.651012in}%
\pgfsys@useobject{currentmarker}{}%
\end{pgfscope}%
\begin{pgfscope}%
\pgfsys@transformshift{2.990641in}{9.649181in}%
\pgfsys@useobject{currentmarker}{}%
\end{pgfscope}%
\begin{pgfscope}%
\pgfsys@transformshift{3.113864in}{9.647275in}%
\pgfsys@useobject{currentmarker}{}%
\end{pgfscope}%
\begin{pgfscope}%
\pgfsys@transformshift{3.237087in}{9.645299in}%
\pgfsys@useobject{currentmarker}{}%
\end{pgfscope}%
\begin{pgfscope}%
\pgfsys@transformshift{3.360310in}{9.643302in}%
\pgfsys@useobject{currentmarker}{}%
\end{pgfscope}%
\begin{pgfscope}%
\pgfsys@transformshift{3.483533in}{9.641283in}%
\pgfsys@useobject{currentmarker}{}%
\end{pgfscope}%
\begin{pgfscope}%
\pgfsys@transformshift{3.606756in}{9.639340in}%
\pgfsys@useobject{currentmarker}{}%
\end{pgfscope}%
\begin{pgfscope}%
\pgfsys@transformshift{3.729979in}{9.637500in}%
\pgfsys@useobject{currentmarker}{}%
\end{pgfscope}%
\begin{pgfscope}%
\pgfsys@transformshift{3.853202in}{9.635664in}%
\pgfsys@useobject{currentmarker}{}%
\end{pgfscope}%
\begin{pgfscope}%
\pgfsys@transformshift{3.976425in}{9.633884in}%
\pgfsys@useobject{currentmarker}{}%
\end{pgfscope}%
\begin{pgfscope}%
\pgfsys@transformshift{4.099648in}{9.632011in}%
\pgfsys@useobject{currentmarker}{}%
\end{pgfscope}%
\begin{pgfscope}%
\pgfsys@transformshift{4.222871in}{9.630098in}%
\pgfsys@useobject{currentmarker}{}%
\end{pgfscope}%
\begin{pgfscope}%
\pgfsys@transformshift{4.346094in}{9.628279in}%
\pgfsys@useobject{currentmarker}{}%
\end{pgfscope}%
\begin{pgfscope}%
\pgfsys@transformshift{4.469317in}{9.626323in}%
\pgfsys@useobject{currentmarker}{}%
\end{pgfscope}%
\begin{pgfscope}%
\pgfsys@transformshift{4.592540in}{9.624287in}%
\pgfsys@useobject{currentmarker}{}%
\end{pgfscope}%
\begin{pgfscope}%
\pgfsys@transformshift{4.715763in}{9.622164in}%
\pgfsys@useobject{currentmarker}{}%
\end{pgfscope}%
\begin{pgfscope}%
\pgfsys@transformshift{4.838986in}{9.620059in}%
\pgfsys@useobject{currentmarker}{}%
\end{pgfscope}%
\begin{pgfscope}%
\pgfsys@transformshift{4.962209in}{9.618016in}%
\pgfsys@useobject{currentmarker}{}%
\end{pgfscope}%
\begin{pgfscope}%
\pgfsys@transformshift{5.085432in}{9.615977in}%
\pgfsys@useobject{currentmarker}{}%
\end{pgfscope}%
\begin{pgfscope}%
\pgfsys@transformshift{5.208655in}{9.613882in}%
\pgfsys@useobject{currentmarker}{}%
\end{pgfscope}%
\begin{pgfscope}%
\pgfsys@transformshift{5.331878in}{9.611825in}%
\pgfsys@useobject{currentmarker}{}%
\end{pgfscope}%
\begin{pgfscope}%
\pgfsys@transformshift{5.455101in}{9.609737in}%
\pgfsys@useobject{currentmarker}{}%
\end{pgfscope}%
\begin{pgfscope}%
\pgfsys@transformshift{5.578324in}{9.607754in}%
\pgfsys@useobject{currentmarker}{}%
\end{pgfscope}%
\begin{pgfscope}%
\pgfsys@transformshift{5.701547in}{9.605766in}%
\pgfsys@useobject{currentmarker}{}%
\end{pgfscope}%
\begin{pgfscope}%
\pgfsys@transformshift{5.824770in}{9.603797in}%
\pgfsys@useobject{currentmarker}{}%
\end{pgfscope}%
\begin{pgfscope}%
\pgfsys@transformshift{5.947993in}{9.601791in}%
\pgfsys@useobject{currentmarker}{}%
\end{pgfscope}%
\begin{pgfscope}%
\pgfsys@transformshift{6.071216in}{9.599677in}%
\pgfsys@useobject{currentmarker}{}%
\end{pgfscope}%
\begin{pgfscope}%
\pgfsys@transformshift{6.194439in}{9.597661in}%
\pgfsys@useobject{currentmarker}{}%
\end{pgfscope}%
\begin{pgfscope}%
\pgfsys@transformshift{6.317662in}{9.595690in}%
\pgfsys@useobject{currentmarker}{}%
\end{pgfscope}%
\begin{pgfscope}%
\pgfsys@transformshift{6.440885in}{9.593759in}%
\pgfsys@useobject{currentmarker}{}%
\end{pgfscope}%
\begin{pgfscope}%
\pgfsys@transformshift{6.564108in}{9.591879in}%
\pgfsys@useobject{currentmarker}{}%
\end{pgfscope}%
\begin{pgfscope}%
\pgfsys@transformshift{6.687330in}{9.590064in}%
\pgfsys@useobject{currentmarker}{}%
\end{pgfscope}%
\begin{pgfscope}%
\pgfsys@transformshift{6.810553in}{9.588455in}%
\pgfsys@useobject{currentmarker}{}%
\end{pgfscope}%
\begin{pgfscope}%
\pgfsys@transformshift{6.933776in}{9.586843in}%
\pgfsys@useobject{currentmarker}{}%
\end{pgfscope}%
\begin{pgfscope}%
\pgfsys@transformshift{7.056999in}{9.585317in}%
\pgfsys@useobject{currentmarker}{}%
\end{pgfscope}%
\begin{pgfscope}%
\pgfsys@transformshift{7.180222in}{9.583876in}%
\pgfsys@useobject{currentmarker}{}%
\end{pgfscope}%
\begin{pgfscope}%
\pgfsys@transformshift{7.303445in}{9.582497in}%
\pgfsys@useobject{currentmarker}{}%
\end{pgfscope}%
\end{pgfscope}%
\begin{pgfscope}%
\pgfsetrectcap%
\pgfsetmiterjoin%
\pgfsetlinewidth{0.803000pt}%
\definecolor{currentstroke}{rgb}{1.000000,1.000000,1.000000}%
\pgfsetstrokecolor{currentstroke}%
\pgfsetdash{}{0pt}%
\pgfpathmoveto{\pgfqpoint{2.125000in}{8.820930in}}%
\pgfpathlineto{\pgfqpoint{2.125000in}{9.698837in}}%
\pgfusepath{stroke}%
\end{pgfscope}%
\begin{pgfscope}%
\pgfsetrectcap%
\pgfsetmiterjoin%
\pgfsetlinewidth{0.803000pt}%
\definecolor{currentstroke}{rgb}{1.000000,1.000000,1.000000}%
\pgfsetstrokecolor{currentstroke}%
\pgfsetdash{}{0pt}%
\pgfpathmoveto{\pgfqpoint{7.614583in}{8.820930in}}%
\pgfpathlineto{\pgfqpoint{7.614583in}{9.698837in}}%
\pgfusepath{stroke}%
\end{pgfscope}%
\begin{pgfscope}%
\pgfsetrectcap%
\pgfsetmiterjoin%
\pgfsetlinewidth{0.803000pt}%
\definecolor{currentstroke}{rgb}{1.000000,1.000000,1.000000}%
\pgfsetstrokecolor{currentstroke}%
\pgfsetdash{}{0pt}%
\pgfpathmoveto{\pgfqpoint{2.125000in}{8.820930in}}%
\pgfpathlineto{\pgfqpoint{7.614583in}{8.820930in}}%
\pgfusepath{stroke}%
\end{pgfscope}%
\begin{pgfscope}%
\pgfsetrectcap%
\pgfsetmiterjoin%
\pgfsetlinewidth{0.803000pt}%
\definecolor{currentstroke}{rgb}{1.000000,1.000000,1.000000}%
\pgfsetstrokecolor{currentstroke}%
\pgfsetdash{}{0pt}%
\pgfpathmoveto{\pgfqpoint{2.125000in}{9.698837in}}%
\pgfpathlineto{\pgfqpoint{7.614583in}{9.698837in}}%
\pgfusepath{stroke}%
\end{pgfscope}%
\begin{pgfscope}%
\definecolor{textcolor}{rgb}{0.150000,0.150000,0.150000}%
\pgfsetstrokecolor{textcolor}%
\pgfsetfillcolor{textcolor}%
\pgftext[x=4.869792in,y=9.782171in,,base]{\color{textcolor}\rmfamily\fontsize{16.800000}{20.160000}\selectfont Autocorrelation}%
\end{pgfscope}%
\begin{pgfscope}%
\pgfsetbuttcap%
\pgfsetmiterjoin%
\definecolor{currentfill}{rgb}{0.917647,0.917647,0.949020}%
\pgfsetfillcolor{currentfill}%
\pgfsetlinewidth{0.000000pt}%
\definecolor{currentstroke}{rgb}{0.000000,0.000000,0.000000}%
\pgfsetstrokecolor{currentstroke}%
\pgfsetstrokeopacity{0.000000}%
\pgfsetdash{}{0pt}%
\pgfpathmoveto{\pgfqpoint{9.810417in}{8.820930in}}%
\pgfpathlineto{\pgfqpoint{15.300000in}{8.820930in}}%
\pgfpathlineto{\pgfqpoint{15.300000in}{9.698837in}}%
\pgfpathlineto{\pgfqpoint{9.810417in}{9.698837in}}%
\pgfpathclose%
\pgfusepath{fill}%
\end{pgfscope}%
\begin{pgfscope}%
\pgfpathrectangle{\pgfqpoint{9.810417in}{8.820930in}}{\pgfqpoint{5.489583in}{0.877907in}}%
\pgfusepath{clip}%
\pgfsetroundcap%
\pgfsetroundjoin%
\pgfsetlinewidth{0.803000pt}%
\definecolor{currentstroke}{rgb}{1.000000,1.000000,1.000000}%
\pgfsetstrokecolor{currentstroke}%
\pgfsetdash{}{0pt}%
\pgfpathmoveto{\pgfqpoint{10.059943in}{8.820930in}}%
\pgfpathlineto{\pgfqpoint{10.059943in}{9.698837in}}%
\pgfusepath{stroke}%
\end{pgfscope}%
\begin{pgfscope}%
\definecolor{textcolor}{rgb}{0.150000,0.150000,0.150000}%
\pgfsetstrokecolor{textcolor}%
\pgfsetfillcolor{textcolor}%
\pgftext[x=10.059943in,y=8.723708in,,top]{\color{textcolor}\rmfamily\fontsize{14.000000}{16.800000}\selectfont 0}%
\end{pgfscope}%
\begin{pgfscope}%
\pgfpathrectangle{\pgfqpoint{9.810417in}{8.820930in}}{\pgfqpoint{5.489583in}{0.877907in}}%
\pgfusepath{clip}%
\pgfsetroundcap%
\pgfsetroundjoin%
\pgfsetlinewidth{0.803000pt}%
\definecolor{currentstroke}{rgb}{1.000000,1.000000,1.000000}%
\pgfsetstrokecolor{currentstroke}%
\pgfsetdash{}{0pt}%
\pgfpathmoveto{\pgfqpoint{10.676058in}{8.820930in}}%
\pgfpathlineto{\pgfqpoint{10.676058in}{9.698837in}}%
\pgfusepath{stroke}%
\end{pgfscope}%
\begin{pgfscope}%
\definecolor{textcolor}{rgb}{0.150000,0.150000,0.150000}%
\pgfsetstrokecolor{textcolor}%
\pgfsetfillcolor{textcolor}%
\pgftext[x=10.676058in,y=8.723708in,,top]{\color{textcolor}\rmfamily\fontsize{14.000000}{16.800000}\selectfont 5}%
\end{pgfscope}%
\begin{pgfscope}%
\pgfpathrectangle{\pgfqpoint{9.810417in}{8.820930in}}{\pgfqpoint{5.489583in}{0.877907in}}%
\pgfusepath{clip}%
\pgfsetroundcap%
\pgfsetroundjoin%
\pgfsetlinewidth{0.803000pt}%
\definecolor{currentstroke}{rgb}{1.000000,1.000000,1.000000}%
\pgfsetstrokecolor{currentstroke}%
\pgfsetdash{}{0pt}%
\pgfpathmoveto{\pgfqpoint{11.292173in}{8.820930in}}%
\pgfpathlineto{\pgfqpoint{11.292173in}{9.698837in}}%
\pgfusepath{stroke}%
\end{pgfscope}%
\begin{pgfscope}%
\definecolor{textcolor}{rgb}{0.150000,0.150000,0.150000}%
\pgfsetstrokecolor{textcolor}%
\pgfsetfillcolor{textcolor}%
\pgftext[x=11.292173in,y=8.723708in,,top]{\color{textcolor}\rmfamily\fontsize{14.000000}{16.800000}\selectfont 10}%
\end{pgfscope}%
\begin{pgfscope}%
\pgfpathrectangle{\pgfqpoint{9.810417in}{8.820930in}}{\pgfqpoint{5.489583in}{0.877907in}}%
\pgfusepath{clip}%
\pgfsetroundcap%
\pgfsetroundjoin%
\pgfsetlinewidth{0.803000pt}%
\definecolor{currentstroke}{rgb}{1.000000,1.000000,1.000000}%
\pgfsetstrokecolor{currentstroke}%
\pgfsetdash{}{0pt}%
\pgfpathmoveto{\pgfqpoint{11.908288in}{8.820930in}}%
\pgfpathlineto{\pgfqpoint{11.908288in}{9.698837in}}%
\pgfusepath{stroke}%
\end{pgfscope}%
\begin{pgfscope}%
\definecolor{textcolor}{rgb}{0.150000,0.150000,0.150000}%
\pgfsetstrokecolor{textcolor}%
\pgfsetfillcolor{textcolor}%
\pgftext[x=11.908288in,y=8.723708in,,top]{\color{textcolor}\rmfamily\fontsize{14.000000}{16.800000}\selectfont 15}%
\end{pgfscope}%
\begin{pgfscope}%
\pgfpathrectangle{\pgfqpoint{9.810417in}{8.820930in}}{\pgfqpoint{5.489583in}{0.877907in}}%
\pgfusepath{clip}%
\pgfsetroundcap%
\pgfsetroundjoin%
\pgfsetlinewidth{0.803000pt}%
\definecolor{currentstroke}{rgb}{1.000000,1.000000,1.000000}%
\pgfsetstrokecolor{currentstroke}%
\pgfsetdash{}{0pt}%
\pgfpathmoveto{\pgfqpoint{12.524403in}{8.820930in}}%
\pgfpathlineto{\pgfqpoint{12.524403in}{9.698837in}}%
\pgfusepath{stroke}%
\end{pgfscope}%
\begin{pgfscope}%
\definecolor{textcolor}{rgb}{0.150000,0.150000,0.150000}%
\pgfsetstrokecolor{textcolor}%
\pgfsetfillcolor{textcolor}%
\pgftext[x=12.524403in,y=8.723708in,,top]{\color{textcolor}\rmfamily\fontsize{14.000000}{16.800000}\selectfont 20}%
\end{pgfscope}%
\begin{pgfscope}%
\pgfpathrectangle{\pgfqpoint{9.810417in}{8.820930in}}{\pgfqpoint{5.489583in}{0.877907in}}%
\pgfusepath{clip}%
\pgfsetroundcap%
\pgfsetroundjoin%
\pgfsetlinewidth{0.803000pt}%
\definecolor{currentstroke}{rgb}{1.000000,1.000000,1.000000}%
\pgfsetstrokecolor{currentstroke}%
\pgfsetdash{}{0pt}%
\pgfpathmoveto{\pgfqpoint{13.140517in}{8.820930in}}%
\pgfpathlineto{\pgfqpoint{13.140517in}{9.698837in}}%
\pgfusepath{stroke}%
\end{pgfscope}%
\begin{pgfscope}%
\definecolor{textcolor}{rgb}{0.150000,0.150000,0.150000}%
\pgfsetstrokecolor{textcolor}%
\pgfsetfillcolor{textcolor}%
\pgftext[x=13.140517in,y=8.723708in,,top]{\color{textcolor}\rmfamily\fontsize{14.000000}{16.800000}\selectfont 25}%
\end{pgfscope}%
\begin{pgfscope}%
\pgfpathrectangle{\pgfqpoint{9.810417in}{8.820930in}}{\pgfqpoint{5.489583in}{0.877907in}}%
\pgfusepath{clip}%
\pgfsetroundcap%
\pgfsetroundjoin%
\pgfsetlinewidth{0.803000pt}%
\definecolor{currentstroke}{rgb}{1.000000,1.000000,1.000000}%
\pgfsetstrokecolor{currentstroke}%
\pgfsetdash{}{0pt}%
\pgfpathmoveto{\pgfqpoint{13.756632in}{8.820930in}}%
\pgfpathlineto{\pgfqpoint{13.756632in}{9.698837in}}%
\pgfusepath{stroke}%
\end{pgfscope}%
\begin{pgfscope}%
\definecolor{textcolor}{rgb}{0.150000,0.150000,0.150000}%
\pgfsetstrokecolor{textcolor}%
\pgfsetfillcolor{textcolor}%
\pgftext[x=13.756632in,y=8.723708in,,top]{\color{textcolor}\rmfamily\fontsize{14.000000}{16.800000}\selectfont 30}%
\end{pgfscope}%
\begin{pgfscope}%
\pgfpathrectangle{\pgfqpoint{9.810417in}{8.820930in}}{\pgfqpoint{5.489583in}{0.877907in}}%
\pgfusepath{clip}%
\pgfsetroundcap%
\pgfsetroundjoin%
\pgfsetlinewidth{0.803000pt}%
\definecolor{currentstroke}{rgb}{1.000000,1.000000,1.000000}%
\pgfsetstrokecolor{currentstroke}%
\pgfsetdash{}{0pt}%
\pgfpathmoveto{\pgfqpoint{14.372747in}{8.820930in}}%
\pgfpathlineto{\pgfqpoint{14.372747in}{9.698837in}}%
\pgfusepath{stroke}%
\end{pgfscope}%
\begin{pgfscope}%
\definecolor{textcolor}{rgb}{0.150000,0.150000,0.150000}%
\pgfsetstrokecolor{textcolor}%
\pgfsetfillcolor{textcolor}%
\pgftext[x=14.372747in,y=8.723708in,,top]{\color{textcolor}\rmfamily\fontsize{14.000000}{16.800000}\selectfont 35}%
\end{pgfscope}%
\begin{pgfscope}%
\pgfpathrectangle{\pgfqpoint{9.810417in}{8.820930in}}{\pgfqpoint{5.489583in}{0.877907in}}%
\pgfusepath{clip}%
\pgfsetroundcap%
\pgfsetroundjoin%
\pgfsetlinewidth{0.803000pt}%
\definecolor{currentstroke}{rgb}{1.000000,1.000000,1.000000}%
\pgfsetstrokecolor{currentstroke}%
\pgfsetdash{}{0pt}%
\pgfpathmoveto{\pgfqpoint{14.988862in}{8.820930in}}%
\pgfpathlineto{\pgfqpoint{14.988862in}{9.698837in}}%
\pgfusepath{stroke}%
\end{pgfscope}%
\begin{pgfscope}%
\definecolor{textcolor}{rgb}{0.150000,0.150000,0.150000}%
\pgfsetstrokecolor{textcolor}%
\pgfsetfillcolor{textcolor}%
\pgftext[x=14.988862in,y=8.723708in,,top]{\color{textcolor}\rmfamily\fontsize{14.000000}{16.800000}\selectfont 40}%
\end{pgfscope}%
\begin{pgfscope}%
\pgfpathrectangle{\pgfqpoint{9.810417in}{8.820930in}}{\pgfqpoint{5.489583in}{0.877907in}}%
\pgfusepath{clip}%
\pgfsetroundcap%
\pgfsetroundjoin%
\pgfsetlinewidth{0.803000pt}%
\definecolor{currentstroke}{rgb}{1.000000,1.000000,1.000000}%
\pgfsetstrokecolor{currentstroke}%
\pgfsetdash{}{0pt}%
\pgfpathmoveto{\pgfqpoint{9.810417in}{8.899169in}}%
\pgfpathlineto{\pgfqpoint{15.300000in}{8.899169in}}%
\pgfusepath{stroke}%
\end{pgfscope}%
\begin{pgfscope}%
\definecolor{textcolor}{rgb}{0.150000,0.150000,0.150000}%
\pgfsetstrokecolor{textcolor}%
\pgfsetfillcolor{textcolor}%
\pgftext[x=9.589483in,y=8.825303in,left,base]{\color{textcolor}\rmfamily\fontsize{14.000000}{16.800000}\selectfont 0}%
\end{pgfscope}%
\begin{pgfscope}%
\pgfpathrectangle{\pgfqpoint{9.810417in}{8.820930in}}{\pgfqpoint{5.489583in}{0.877907in}}%
\pgfusepath{clip}%
\pgfsetroundcap%
\pgfsetroundjoin%
\pgfsetlinewidth{0.803000pt}%
\definecolor{currentstroke}{rgb}{1.000000,1.000000,1.000000}%
\pgfsetstrokecolor{currentstroke}%
\pgfsetdash{}{0pt}%
\pgfpathmoveto{\pgfqpoint{9.810417in}{9.658932in}}%
\pgfpathlineto{\pgfqpoint{15.300000in}{9.658932in}}%
\pgfusepath{stroke}%
\end{pgfscope}%
\begin{pgfscope}%
\definecolor{textcolor}{rgb}{0.150000,0.150000,0.150000}%
\pgfsetstrokecolor{textcolor}%
\pgfsetfillcolor{textcolor}%
\pgftext[x=9.589483in,y=9.585066in,left,base]{\color{textcolor}\rmfamily\fontsize{14.000000}{16.800000}\selectfont 1}%
\end{pgfscope}%
\begin{pgfscope}%
\pgfpathrectangle{\pgfqpoint{9.810417in}{8.820930in}}{\pgfqpoint{5.489583in}{0.877907in}}%
\pgfusepath{clip}%
\pgfsetbuttcap%
\pgfsetroundjoin%
\definecolor{currentfill}{rgb}{0.121569,0.466667,0.705882}%
\pgfsetfillcolor{currentfill}%
\pgfsetfillopacity{0.250000}%
\pgfsetlinewidth{1.003750pt}%
\definecolor{currentstroke}{rgb}{1.000000,1.000000,1.000000}%
\pgfsetstrokecolor{currentstroke}%
\pgfsetstrokeopacity{0.250000}%
\pgfsetdash{}{0pt}%
\pgfpathmoveto{\pgfqpoint{10.121555in}{8.937503in}}%
\pgfpathlineto{\pgfqpoint{10.121555in}{8.860835in}}%
\pgfpathlineto{\pgfqpoint{10.306389in}{8.860835in}}%
\pgfpathlineto{\pgfqpoint{10.429612in}{8.860835in}}%
\pgfpathlineto{\pgfqpoint{10.552835in}{8.860835in}}%
\pgfpathlineto{\pgfqpoint{10.676058in}{8.860835in}}%
\pgfpathlineto{\pgfqpoint{10.799281in}{8.860835in}}%
\pgfpathlineto{\pgfqpoint{10.922504in}{8.860835in}}%
\pgfpathlineto{\pgfqpoint{11.045727in}{8.860835in}}%
\pgfpathlineto{\pgfqpoint{11.168950in}{8.860835in}}%
\pgfpathlineto{\pgfqpoint{11.292173in}{8.860835in}}%
\pgfpathlineto{\pgfqpoint{11.415396in}{8.860835in}}%
\pgfpathlineto{\pgfqpoint{11.538619in}{8.860835in}}%
\pgfpathlineto{\pgfqpoint{11.661842in}{8.860835in}}%
\pgfpathlineto{\pgfqpoint{11.785065in}{8.860835in}}%
\pgfpathlineto{\pgfqpoint{11.908288in}{8.860835in}}%
\pgfpathlineto{\pgfqpoint{12.031511in}{8.860835in}}%
\pgfpathlineto{\pgfqpoint{12.154734in}{8.860835in}}%
\pgfpathlineto{\pgfqpoint{12.277957in}{8.860835in}}%
\pgfpathlineto{\pgfqpoint{12.401180in}{8.860835in}}%
\pgfpathlineto{\pgfqpoint{12.524403in}{8.860835in}}%
\pgfpathlineto{\pgfqpoint{12.647626in}{8.860835in}}%
\pgfpathlineto{\pgfqpoint{12.770849in}{8.860835in}}%
\pgfpathlineto{\pgfqpoint{12.894072in}{8.860835in}}%
\pgfpathlineto{\pgfqpoint{13.017294in}{8.860835in}}%
\pgfpathlineto{\pgfqpoint{13.140517in}{8.860835in}}%
\pgfpathlineto{\pgfqpoint{13.263740in}{8.860835in}}%
\pgfpathlineto{\pgfqpoint{13.386963in}{8.860835in}}%
\pgfpathlineto{\pgfqpoint{13.510186in}{8.860835in}}%
\pgfpathlineto{\pgfqpoint{13.633409in}{8.860835in}}%
\pgfpathlineto{\pgfqpoint{13.756632in}{8.860835in}}%
\pgfpathlineto{\pgfqpoint{13.879855in}{8.860835in}}%
\pgfpathlineto{\pgfqpoint{14.003078in}{8.860835in}}%
\pgfpathlineto{\pgfqpoint{14.126301in}{8.860835in}}%
\pgfpathlineto{\pgfqpoint{14.249524in}{8.860835in}}%
\pgfpathlineto{\pgfqpoint{14.372747in}{8.860835in}}%
\pgfpathlineto{\pgfqpoint{14.495970in}{8.860835in}}%
\pgfpathlineto{\pgfqpoint{14.619193in}{8.860835in}}%
\pgfpathlineto{\pgfqpoint{14.742416in}{8.860835in}}%
\pgfpathlineto{\pgfqpoint{14.865639in}{8.860835in}}%
\pgfpathlineto{\pgfqpoint{15.050473in}{8.860835in}}%
\pgfpathlineto{\pgfqpoint{15.050473in}{8.937503in}}%
\pgfpathlineto{\pgfqpoint{15.050473in}{8.937503in}}%
\pgfpathlineto{\pgfqpoint{14.865639in}{8.937503in}}%
\pgfpathlineto{\pgfqpoint{14.742416in}{8.937503in}}%
\pgfpathlineto{\pgfqpoint{14.619193in}{8.937503in}}%
\pgfpathlineto{\pgfqpoint{14.495970in}{8.937503in}}%
\pgfpathlineto{\pgfqpoint{14.372747in}{8.937503in}}%
\pgfpathlineto{\pgfqpoint{14.249524in}{8.937503in}}%
\pgfpathlineto{\pgfqpoint{14.126301in}{8.937503in}}%
\pgfpathlineto{\pgfqpoint{14.003078in}{8.937503in}}%
\pgfpathlineto{\pgfqpoint{13.879855in}{8.937503in}}%
\pgfpathlineto{\pgfqpoint{13.756632in}{8.937503in}}%
\pgfpathlineto{\pgfqpoint{13.633409in}{8.937503in}}%
\pgfpathlineto{\pgfqpoint{13.510186in}{8.937503in}}%
\pgfpathlineto{\pgfqpoint{13.386963in}{8.937503in}}%
\pgfpathlineto{\pgfqpoint{13.263740in}{8.937503in}}%
\pgfpathlineto{\pgfqpoint{13.140517in}{8.937503in}}%
\pgfpathlineto{\pgfqpoint{13.017294in}{8.937503in}}%
\pgfpathlineto{\pgfqpoint{12.894072in}{8.937503in}}%
\pgfpathlineto{\pgfqpoint{12.770849in}{8.937503in}}%
\pgfpathlineto{\pgfqpoint{12.647626in}{8.937503in}}%
\pgfpathlineto{\pgfqpoint{12.524403in}{8.937503in}}%
\pgfpathlineto{\pgfqpoint{12.401180in}{8.937503in}}%
\pgfpathlineto{\pgfqpoint{12.277957in}{8.937503in}}%
\pgfpathlineto{\pgfqpoint{12.154734in}{8.937503in}}%
\pgfpathlineto{\pgfqpoint{12.031511in}{8.937503in}}%
\pgfpathlineto{\pgfqpoint{11.908288in}{8.937503in}}%
\pgfpathlineto{\pgfqpoint{11.785065in}{8.937503in}}%
\pgfpathlineto{\pgfqpoint{11.661842in}{8.937503in}}%
\pgfpathlineto{\pgfqpoint{11.538619in}{8.937503in}}%
\pgfpathlineto{\pgfqpoint{11.415396in}{8.937503in}}%
\pgfpathlineto{\pgfqpoint{11.292173in}{8.937503in}}%
\pgfpathlineto{\pgfqpoint{11.168950in}{8.937503in}}%
\pgfpathlineto{\pgfqpoint{11.045727in}{8.937503in}}%
\pgfpathlineto{\pgfqpoint{10.922504in}{8.937503in}}%
\pgfpathlineto{\pgfqpoint{10.799281in}{8.937503in}}%
\pgfpathlineto{\pgfqpoint{10.676058in}{8.937503in}}%
\pgfpathlineto{\pgfqpoint{10.552835in}{8.937503in}}%
\pgfpathlineto{\pgfqpoint{10.429612in}{8.937503in}}%
\pgfpathlineto{\pgfqpoint{10.306389in}{8.937503in}}%
\pgfpathlineto{\pgfqpoint{10.121555in}{8.937503in}}%
\pgfpathclose%
\pgfusepath{stroke,fill}%
\end{pgfscope}%
\begin{pgfscope}%
\pgfpathrectangle{\pgfqpoint{9.810417in}{8.820930in}}{\pgfqpoint{5.489583in}{0.877907in}}%
\pgfusepath{clip}%
\pgfsetbuttcap%
\pgfsetroundjoin%
\pgfsetlinewidth{1.505625pt}%
\definecolor{currentstroke}{rgb}{0.000000,0.000000,0.000000}%
\pgfsetstrokecolor{currentstroke}%
\pgfsetdash{}{0pt}%
\pgfpathmoveto{\pgfqpoint{10.059943in}{8.899169in}}%
\pgfpathlineto{\pgfqpoint{10.059943in}{9.658932in}}%
\pgfusepath{stroke}%
\end{pgfscope}%
\begin{pgfscope}%
\pgfpathrectangle{\pgfqpoint{9.810417in}{8.820930in}}{\pgfqpoint{5.489583in}{0.877907in}}%
\pgfusepath{clip}%
\pgfsetbuttcap%
\pgfsetroundjoin%
\pgfsetlinewidth{1.505625pt}%
\definecolor{currentstroke}{rgb}{0.000000,0.000000,0.000000}%
\pgfsetstrokecolor{currentstroke}%
\pgfsetdash{}{0pt}%
\pgfpathmoveto{\pgfqpoint{10.183166in}{8.899169in}}%
\pgfpathlineto{\pgfqpoint{10.183166in}{9.656717in}}%
\pgfusepath{stroke}%
\end{pgfscope}%
\begin{pgfscope}%
\pgfpathrectangle{\pgfqpoint{9.810417in}{8.820930in}}{\pgfqpoint{5.489583in}{0.877907in}}%
\pgfusepath{clip}%
\pgfsetbuttcap%
\pgfsetroundjoin%
\pgfsetlinewidth{1.505625pt}%
\definecolor{currentstroke}{rgb}{0.000000,0.000000,0.000000}%
\pgfsetstrokecolor{currentstroke}%
\pgfsetdash{}{0pt}%
\pgfpathmoveto{\pgfqpoint{10.306389in}{8.899169in}}%
\pgfpathlineto{\pgfqpoint{10.306389in}{8.903534in}}%
\pgfusepath{stroke}%
\end{pgfscope}%
\begin{pgfscope}%
\pgfpathrectangle{\pgfqpoint{9.810417in}{8.820930in}}{\pgfqpoint{5.489583in}{0.877907in}}%
\pgfusepath{clip}%
\pgfsetbuttcap%
\pgfsetroundjoin%
\pgfsetlinewidth{1.505625pt}%
\definecolor{currentstroke}{rgb}{0.000000,0.000000,0.000000}%
\pgfsetstrokecolor{currentstroke}%
\pgfsetdash{}{0pt}%
\pgfpathmoveto{\pgfqpoint{10.429612in}{8.899169in}}%
\pgfpathlineto{\pgfqpoint{10.429612in}{8.904148in}}%
\pgfusepath{stroke}%
\end{pgfscope}%
\begin{pgfscope}%
\pgfpathrectangle{\pgfqpoint{9.810417in}{8.820930in}}{\pgfqpoint{5.489583in}{0.877907in}}%
\pgfusepath{clip}%
\pgfsetbuttcap%
\pgfsetroundjoin%
\pgfsetlinewidth{1.505625pt}%
\definecolor{currentstroke}{rgb}{0.000000,0.000000,0.000000}%
\pgfsetstrokecolor{currentstroke}%
\pgfsetdash{}{0pt}%
\pgfpathmoveto{\pgfqpoint{10.552835in}{8.899169in}}%
\pgfpathlineto{\pgfqpoint{10.552835in}{8.897394in}}%
\pgfusepath{stroke}%
\end{pgfscope}%
\begin{pgfscope}%
\pgfpathrectangle{\pgfqpoint{9.810417in}{8.820930in}}{\pgfqpoint{5.489583in}{0.877907in}}%
\pgfusepath{clip}%
\pgfsetbuttcap%
\pgfsetroundjoin%
\pgfsetlinewidth{1.505625pt}%
\definecolor{currentstroke}{rgb}{0.000000,0.000000,0.000000}%
\pgfsetstrokecolor{currentstroke}%
\pgfsetdash{}{0pt}%
\pgfpathmoveto{\pgfqpoint{10.676058in}{8.899169in}}%
\pgfpathlineto{\pgfqpoint{10.676058in}{8.927893in}}%
\pgfusepath{stroke}%
\end{pgfscope}%
\begin{pgfscope}%
\pgfpathrectangle{\pgfqpoint{9.810417in}{8.820930in}}{\pgfqpoint{5.489583in}{0.877907in}}%
\pgfusepath{clip}%
\pgfsetbuttcap%
\pgfsetroundjoin%
\pgfsetlinewidth{1.505625pt}%
\definecolor{currentstroke}{rgb}{0.000000,0.000000,0.000000}%
\pgfsetstrokecolor{currentstroke}%
\pgfsetdash{}{0pt}%
\pgfpathmoveto{\pgfqpoint{10.799281in}{8.899169in}}%
\pgfpathlineto{\pgfqpoint{10.799281in}{8.880456in}}%
\pgfusepath{stroke}%
\end{pgfscope}%
\begin{pgfscope}%
\pgfpathrectangle{\pgfqpoint{9.810417in}{8.820930in}}{\pgfqpoint{5.489583in}{0.877907in}}%
\pgfusepath{clip}%
\pgfsetbuttcap%
\pgfsetroundjoin%
\pgfsetlinewidth{1.505625pt}%
\definecolor{currentstroke}{rgb}{0.000000,0.000000,0.000000}%
\pgfsetstrokecolor{currentstroke}%
\pgfsetdash{}{0pt}%
\pgfpathmoveto{\pgfqpoint{10.922504in}{8.899169in}}%
\pgfpathlineto{\pgfqpoint{10.922504in}{8.881465in}}%
\pgfusepath{stroke}%
\end{pgfscope}%
\begin{pgfscope}%
\pgfpathrectangle{\pgfqpoint{9.810417in}{8.820930in}}{\pgfqpoint{5.489583in}{0.877907in}}%
\pgfusepath{clip}%
\pgfsetbuttcap%
\pgfsetroundjoin%
\pgfsetlinewidth{1.505625pt}%
\definecolor{currentstroke}{rgb}{0.000000,0.000000,0.000000}%
\pgfsetstrokecolor{currentstroke}%
\pgfsetdash{}{0pt}%
\pgfpathmoveto{\pgfqpoint{11.045727in}{8.899169in}}%
\pgfpathlineto{\pgfqpoint{11.045727in}{8.892266in}}%
\pgfusepath{stroke}%
\end{pgfscope}%
\begin{pgfscope}%
\pgfpathrectangle{\pgfqpoint{9.810417in}{8.820930in}}{\pgfqpoint{5.489583in}{0.877907in}}%
\pgfusepath{clip}%
\pgfsetbuttcap%
\pgfsetroundjoin%
\pgfsetlinewidth{1.505625pt}%
\definecolor{currentstroke}{rgb}{0.000000,0.000000,0.000000}%
\pgfsetstrokecolor{currentstroke}%
\pgfsetdash{}{0pt}%
\pgfpathmoveto{\pgfqpoint{11.168950in}{8.899169in}}%
\pgfpathlineto{\pgfqpoint{11.168950in}{8.893383in}}%
\pgfusepath{stroke}%
\end{pgfscope}%
\begin{pgfscope}%
\pgfpathrectangle{\pgfqpoint{9.810417in}{8.820930in}}{\pgfqpoint{5.489583in}{0.877907in}}%
\pgfusepath{clip}%
\pgfsetbuttcap%
\pgfsetroundjoin%
\pgfsetlinewidth{1.505625pt}%
\definecolor{currentstroke}{rgb}{0.000000,0.000000,0.000000}%
\pgfsetstrokecolor{currentstroke}%
\pgfsetdash{}{0pt}%
\pgfpathmoveto{\pgfqpoint{11.292173in}{8.899169in}}%
\pgfpathlineto{\pgfqpoint{11.292173in}{8.913614in}}%
\pgfusepath{stroke}%
\end{pgfscope}%
\begin{pgfscope}%
\pgfpathrectangle{\pgfqpoint{9.810417in}{8.820930in}}{\pgfqpoint{5.489583in}{0.877907in}}%
\pgfusepath{clip}%
\pgfsetbuttcap%
\pgfsetroundjoin%
\pgfsetlinewidth{1.505625pt}%
\definecolor{currentstroke}{rgb}{0.000000,0.000000,0.000000}%
\pgfsetstrokecolor{currentstroke}%
\pgfsetdash{}{0pt}%
\pgfpathmoveto{\pgfqpoint{11.415396in}{8.899169in}}%
\pgfpathlineto{\pgfqpoint{11.415396in}{8.920884in}}%
\pgfusepath{stroke}%
\end{pgfscope}%
\begin{pgfscope}%
\pgfpathrectangle{\pgfqpoint{9.810417in}{8.820930in}}{\pgfqpoint{5.489583in}{0.877907in}}%
\pgfusepath{clip}%
\pgfsetbuttcap%
\pgfsetroundjoin%
\pgfsetlinewidth{1.505625pt}%
\definecolor{currentstroke}{rgb}{0.000000,0.000000,0.000000}%
\pgfsetstrokecolor{currentstroke}%
\pgfsetdash{}{0pt}%
\pgfpathmoveto{\pgfqpoint{11.538619in}{8.899169in}}%
\pgfpathlineto{\pgfqpoint{11.538619in}{8.899769in}}%
\pgfusepath{stroke}%
\end{pgfscope}%
\begin{pgfscope}%
\pgfpathrectangle{\pgfqpoint{9.810417in}{8.820930in}}{\pgfqpoint{5.489583in}{0.877907in}}%
\pgfusepath{clip}%
\pgfsetbuttcap%
\pgfsetroundjoin%
\pgfsetlinewidth{1.505625pt}%
\definecolor{currentstroke}{rgb}{0.000000,0.000000,0.000000}%
\pgfsetstrokecolor{currentstroke}%
\pgfsetdash{}{0pt}%
\pgfpathmoveto{\pgfqpoint{11.661842in}{8.899169in}}%
\pgfpathlineto{\pgfqpoint{11.661842in}{8.911209in}}%
\pgfusepath{stroke}%
\end{pgfscope}%
\begin{pgfscope}%
\pgfpathrectangle{\pgfqpoint{9.810417in}{8.820930in}}{\pgfqpoint{5.489583in}{0.877907in}}%
\pgfusepath{clip}%
\pgfsetbuttcap%
\pgfsetroundjoin%
\pgfsetlinewidth{1.505625pt}%
\definecolor{currentstroke}{rgb}{0.000000,0.000000,0.000000}%
\pgfsetstrokecolor{currentstroke}%
\pgfsetdash{}{0pt}%
\pgfpathmoveto{\pgfqpoint{11.785065in}{8.899169in}}%
\pgfpathlineto{\pgfqpoint{11.785065in}{8.877984in}}%
\pgfusepath{stroke}%
\end{pgfscope}%
\begin{pgfscope}%
\pgfpathrectangle{\pgfqpoint{9.810417in}{8.820930in}}{\pgfqpoint{5.489583in}{0.877907in}}%
\pgfusepath{clip}%
\pgfsetbuttcap%
\pgfsetroundjoin%
\pgfsetlinewidth{1.505625pt}%
\definecolor{currentstroke}{rgb}{0.000000,0.000000,0.000000}%
\pgfsetstrokecolor{currentstroke}%
\pgfsetdash{}{0pt}%
\pgfpathmoveto{\pgfqpoint{11.908288in}{8.899169in}}%
\pgfpathlineto{\pgfqpoint{11.908288in}{8.889088in}}%
\pgfusepath{stroke}%
\end{pgfscope}%
\begin{pgfscope}%
\pgfpathrectangle{\pgfqpoint{9.810417in}{8.820930in}}{\pgfqpoint{5.489583in}{0.877907in}}%
\pgfusepath{clip}%
\pgfsetbuttcap%
\pgfsetroundjoin%
\pgfsetlinewidth{1.505625pt}%
\definecolor{currentstroke}{rgb}{0.000000,0.000000,0.000000}%
\pgfsetstrokecolor{currentstroke}%
\pgfsetdash{}{0pt}%
\pgfpathmoveto{\pgfqpoint{12.031511in}{8.899169in}}%
\pgfpathlineto{\pgfqpoint{12.031511in}{8.917679in}}%
\pgfusepath{stroke}%
\end{pgfscope}%
\begin{pgfscope}%
\pgfpathrectangle{\pgfqpoint{9.810417in}{8.820930in}}{\pgfqpoint{5.489583in}{0.877907in}}%
\pgfusepath{clip}%
\pgfsetbuttcap%
\pgfsetroundjoin%
\pgfsetlinewidth{1.505625pt}%
\definecolor{currentstroke}{rgb}{0.000000,0.000000,0.000000}%
\pgfsetstrokecolor{currentstroke}%
\pgfsetdash{}{0pt}%
\pgfpathmoveto{\pgfqpoint{12.154734in}{8.899169in}}%
\pgfpathlineto{\pgfqpoint{12.154734in}{8.865466in}}%
\pgfusepath{stroke}%
\end{pgfscope}%
\begin{pgfscope}%
\pgfpathrectangle{\pgfqpoint{9.810417in}{8.820930in}}{\pgfqpoint{5.489583in}{0.877907in}}%
\pgfusepath{clip}%
\pgfsetbuttcap%
\pgfsetroundjoin%
\pgfsetlinewidth{1.505625pt}%
\definecolor{currentstroke}{rgb}{0.000000,0.000000,0.000000}%
\pgfsetstrokecolor{currentstroke}%
\pgfsetdash{}{0pt}%
\pgfpathmoveto{\pgfqpoint{12.277957in}{8.899169in}}%
\pgfpathlineto{\pgfqpoint{12.277957in}{8.875761in}}%
\pgfusepath{stroke}%
\end{pgfscope}%
\begin{pgfscope}%
\pgfpathrectangle{\pgfqpoint{9.810417in}{8.820930in}}{\pgfqpoint{5.489583in}{0.877907in}}%
\pgfusepath{clip}%
\pgfsetbuttcap%
\pgfsetroundjoin%
\pgfsetlinewidth{1.505625pt}%
\definecolor{currentstroke}{rgb}{0.000000,0.000000,0.000000}%
\pgfsetstrokecolor{currentstroke}%
\pgfsetdash{}{0pt}%
\pgfpathmoveto{\pgfqpoint{12.401180in}{8.899169in}}%
\pgfpathlineto{\pgfqpoint{12.401180in}{8.875551in}}%
\pgfusepath{stroke}%
\end{pgfscope}%
\begin{pgfscope}%
\pgfpathrectangle{\pgfqpoint{9.810417in}{8.820930in}}{\pgfqpoint{5.489583in}{0.877907in}}%
\pgfusepath{clip}%
\pgfsetbuttcap%
\pgfsetroundjoin%
\pgfsetlinewidth{1.505625pt}%
\definecolor{currentstroke}{rgb}{0.000000,0.000000,0.000000}%
\pgfsetstrokecolor{currentstroke}%
\pgfsetdash{}{0pt}%
\pgfpathmoveto{\pgfqpoint{12.524403in}{8.899169in}}%
\pgfpathlineto{\pgfqpoint{12.524403in}{8.904916in}}%
\pgfusepath{stroke}%
\end{pgfscope}%
\begin{pgfscope}%
\pgfpathrectangle{\pgfqpoint{9.810417in}{8.820930in}}{\pgfqpoint{5.489583in}{0.877907in}}%
\pgfusepath{clip}%
\pgfsetbuttcap%
\pgfsetroundjoin%
\pgfsetlinewidth{1.505625pt}%
\definecolor{currentstroke}{rgb}{0.000000,0.000000,0.000000}%
\pgfsetstrokecolor{currentstroke}%
\pgfsetdash{}{0pt}%
\pgfpathmoveto{\pgfqpoint{12.647626in}{8.899169in}}%
\pgfpathlineto{\pgfqpoint{12.647626in}{8.909892in}}%
\pgfusepath{stroke}%
\end{pgfscope}%
\begin{pgfscope}%
\pgfpathrectangle{\pgfqpoint{9.810417in}{8.820930in}}{\pgfqpoint{5.489583in}{0.877907in}}%
\pgfusepath{clip}%
\pgfsetbuttcap%
\pgfsetroundjoin%
\pgfsetlinewidth{1.505625pt}%
\definecolor{currentstroke}{rgb}{0.000000,0.000000,0.000000}%
\pgfsetstrokecolor{currentstroke}%
\pgfsetdash{}{0pt}%
\pgfpathmoveto{\pgfqpoint{12.770849in}{8.899169in}}%
\pgfpathlineto{\pgfqpoint{12.770849in}{8.898822in}}%
\pgfusepath{stroke}%
\end{pgfscope}%
\begin{pgfscope}%
\pgfpathrectangle{\pgfqpoint{9.810417in}{8.820930in}}{\pgfqpoint{5.489583in}{0.877907in}}%
\pgfusepath{clip}%
\pgfsetbuttcap%
\pgfsetroundjoin%
\pgfsetlinewidth{1.505625pt}%
\definecolor{currentstroke}{rgb}{0.000000,0.000000,0.000000}%
\pgfsetstrokecolor{currentstroke}%
\pgfsetdash{}{0pt}%
\pgfpathmoveto{\pgfqpoint{12.894072in}{8.899169in}}%
\pgfpathlineto{\pgfqpoint{12.894072in}{8.884663in}}%
\pgfusepath{stroke}%
\end{pgfscope}%
\begin{pgfscope}%
\pgfpathrectangle{\pgfqpoint{9.810417in}{8.820930in}}{\pgfqpoint{5.489583in}{0.877907in}}%
\pgfusepath{clip}%
\pgfsetbuttcap%
\pgfsetroundjoin%
\pgfsetlinewidth{1.505625pt}%
\definecolor{currentstroke}{rgb}{0.000000,0.000000,0.000000}%
\pgfsetstrokecolor{currentstroke}%
\pgfsetdash{}{0pt}%
\pgfpathmoveto{\pgfqpoint{13.017294in}{8.899169in}}%
\pgfpathlineto{\pgfqpoint{13.017294in}{8.907107in}}%
\pgfusepath{stroke}%
\end{pgfscope}%
\begin{pgfscope}%
\pgfpathrectangle{\pgfqpoint{9.810417in}{8.820930in}}{\pgfqpoint{5.489583in}{0.877907in}}%
\pgfusepath{clip}%
\pgfsetbuttcap%
\pgfsetroundjoin%
\pgfsetlinewidth{1.505625pt}%
\definecolor{currentstroke}{rgb}{0.000000,0.000000,0.000000}%
\pgfsetstrokecolor{currentstroke}%
\pgfsetdash{}{0pt}%
\pgfpathmoveto{\pgfqpoint{13.140517in}{8.899169in}}%
\pgfpathlineto{\pgfqpoint{13.140517in}{8.892804in}}%
\pgfusepath{stroke}%
\end{pgfscope}%
\begin{pgfscope}%
\pgfpathrectangle{\pgfqpoint{9.810417in}{8.820930in}}{\pgfqpoint{5.489583in}{0.877907in}}%
\pgfusepath{clip}%
\pgfsetbuttcap%
\pgfsetroundjoin%
\pgfsetlinewidth{1.505625pt}%
\definecolor{currentstroke}{rgb}{0.000000,0.000000,0.000000}%
\pgfsetstrokecolor{currentstroke}%
\pgfsetdash{}{0pt}%
\pgfpathmoveto{\pgfqpoint{13.263740in}{8.899169in}}%
\pgfpathlineto{\pgfqpoint{13.263740in}{8.921495in}}%
\pgfusepath{stroke}%
\end{pgfscope}%
\begin{pgfscope}%
\pgfpathrectangle{\pgfqpoint{9.810417in}{8.820930in}}{\pgfqpoint{5.489583in}{0.877907in}}%
\pgfusepath{clip}%
\pgfsetbuttcap%
\pgfsetroundjoin%
\pgfsetlinewidth{1.505625pt}%
\definecolor{currentstroke}{rgb}{0.000000,0.000000,0.000000}%
\pgfsetstrokecolor{currentstroke}%
\pgfsetdash{}{0pt}%
\pgfpathmoveto{\pgfqpoint{13.386963in}{8.899169in}}%
\pgfpathlineto{\pgfqpoint{13.386963in}{8.892523in}}%
\pgfusepath{stroke}%
\end{pgfscope}%
\begin{pgfscope}%
\pgfpathrectangle{\pgfqpoint{9.810417in}{8.820930in}}{\pgfqpoint{5.489583in}{0.877907in}}%
\pgfusepath{clip}%
\pgfsetbuttcap%
\pgfsetroundjoin%
\pgfsetlinewidth{1.505625pt}%
\definecolor{currentstroke}{rgb}{0.000000,0.000000,0.000000}%
\pgfsetstrokecolor{currentstroke}%
\pgfsetdash{}{0pt}%
\pgfpathmoveto{\pgfqpoint{13.510186in}{8.899169in}}%
\pgfpathlineto{\pgfqpoint{13.510186in}{8.902054in}}%
\pgfusepath{stroke}%
\end{pgfscope}%
\begin{pgfscope}%
\pgfpathrectangle{\pgfqpoint{9.810417in}{8.820930in}}{\pgfqpoint{5.489583in}{0.877907in}}%
\pgfusepath{clip}%
\pgfsetbuttcap%
\pgfsetroundjoin%
\pgfsetlinewidth{1.505625pt}%
\definecolor{currentstroke}{rgb}{0.000000,0.000000,0.000000}%
\pgfsetstrokecolor{currentstroke}%
\pgfsetdash{}{0pt}%
\pgfpathmoveto{\pgfqpoint{13.633409in}{8.899169in}}%
\pgfpathlineto{\pgfqpoint{13.633409in}{8.884742in}}%
\pgfusepath{stroke}%
\end{pgfscope}%
\begin{pgfscope}%
\pgfpathrectangle{\pgfqpoint{9.810417in}{8.820930in}}{\pgfqpoint{5.489583in}{0.877907in}}%
\pgfusepath{clip}%
\pgfsetbuttcap%
\pgfsetroundjoin%
\pgfsetlinewidth{1.505625pt}%
\definecolor{currentstroke}{rgb}{0.000000,0.000000,0.000000}%
\pgfsetstrokecolor{currentstroke}%
\pgfsetdash{}{0pt}%
\pgfpathmoveto{\pgfqpoint{13.756632in}{8.899169in}}%
\pgfpathlineto{\pgfqpoint{13.756632in}{8.874818in}}%
\pgfusepath{stroke}%
\end{pgfscope}%
\begin{pgfscope}%
\pgfpathrectangle{\pgfqpoint{9.810417in}{8.820930in}}{\pgfqpoint{5.489583in}{0.877907in}}%
\pgfusepath{clip}%
\pgfsetbuttcap%
\pgfsetroundjoin%
\pgfsetlinewidth{1.505625pt}%
\definecolor{currentstroke}{rgb}{0.000000,0.000000,0.000000}%
\pgfsetstrokecolor{currentstroke}%
\pgfsetdash{}{0pt}%
\pgfpathmoveto{\pgfqpoint{13.879855in}{8.899169in}}%
\pgfpathlineto{\pgfqpoint{13.879855in}{8.921886in}}%
\pgfusepath{stroke}%
\end{pgfscope}%
\begin{pgfscope}%
\pgfpathrectangle{\pgfqpoint{9.810417in}{8.820930in}}{\pgfqpoint{5.489583in}{0.877907in}}%
\pgfusepath{clip}%
\pgfsetbuttcap%
\pgfsetroundjoin%
\pgfsetlinewidth{1.505625pt}%
\definecolor{currentstroke}{rgb}{0.000000,0.000000,0.000000}%
\pgfsetstrokecolor{currentstroke}%
\pgfsetdash{}{0pt}%
\pgfpathmoveto{\pgfqpoint{14.003078in}{8.899169in}}%
\pgfpathlineto{\pgfqpoint{14.003078in}{8.907410in}}%
\pgfusepath{stroke}%
\end{pgfscope}%
\begin{pgfscope}%
\pgfpathrectangle{\pgfqpoint{9.810417in}{8.820930in}}{\pgfqpoint{5.489583in}{0.877907in}}%
\pgfusepath{clip}%
\pgfsetbuttcap%
\pgfsetroundjoin%
\pgfsetlinewidth{1.505625pt}%
\definecolor{currentstroke}{rgb}{0.000000,0.000000,0.000000}%
\pgfsetstrokecolor{currentstroke}%
\pgfsetdash{}{0pt}%
\pgfpathmoveto{\pgfqpoint{14.126301in}{8.899169in}}%
\pgfpathlineto{\pgfqpoint{14.126301in}{8.906983in}}%
\pgfusepath{stroke}%
\end{pgfscope}%
\begin{pgfscope}%
\pgfpathrectangle{\pgfqpoint{9.810417in}{8.820930in}}{\pgfqpoint{5.489583in}{0.877907in}}%
\pgfusepath{clip}%
\pgfsetbuttcap%
\pgfsetroundjoin%
\pgfsetlinewidth{1.505625pt}%
\definecolor{currentstroke}{rgb}{0.000000,0.000000,0.000000}%
\pgfsetstrokecolor{currentstroke}%
\pgfsetdash{}{0pt}%
\pgfpathmoveto{\pgfqpoint{14.249524in}{8.899169in}}%
\pgfpathlineto{\pgfqpoint{14.249524in}{8.907925in}}%
\pgfusepath{stroke}%
\end{pgfscope}%
\begin{pgfscope}%
\pgfpathrectangle{\pgfqpoint{9.810417in}{8.820930in}}{\pgfqpoint{5.489583in}{0.877907in}}%
\pgfusepath{clip}%
\pgfsetbuttcap%
\pgfsetroundjoin%
\pgfsetlinewidth{1.505625pt}%
\definecolor{currentstroke}{rgb}{0.000000,0.000000,0.000000}%
\pgfsetstrokecolor{currentstroke}%
\pgfsetdash{}{0pt}%
\pgfpathmoveto{\pgfqpoint{14.372747in}{8.899169in}}%
\pgfpathlineto{\pgfqpoint{14.372747in}{8.919555in}}%
\pgfusepath{stroke}%
\end{pgfscope}%
\begin{pgfscope}%
\pgfpathrectangle{\pgfqpoint{9.810417in}{8.820930in}}{\pgfqpoint{5.489583in}{0.877907in}}%
\pgfusepath{clip}%
\pgfsetbuttcap%
\pgfsetroundjoin%
\pgfsetlinewidth{1.505625pt}%
\definecolor{currentstroke}{rgb}{0.000000,0.000000,0.000000}%
\pgfsetstrokecolor{currentstroke}%
\pgfsetdash{}{0pt}%
\pgfpathmoveto{\pgfqpoint{14.495970in}{8.899169in}}%
\pgfpathlineto{\pgfqpoint{14.495970in}{8.952129in}}%
\pgfusepath{stroke}%
\end{pgfscope}%
\begin{pgfscope}%
\pgfpathrectangle{\pgfqpoint{9.810417in}{8.820930in}}{\pgfqpoint{5.489583in}{0.877907in}}%
\pgfusepath{clip}%
\pgfsetbuttcap%
\pgfsetroundjoin%
\pgfsetlinewidth{1.505625pt}%
\definecolor{currentstroke}{rgb}{0.000000,0.000000,0.000000}%
\pgfsetstrokecolor{currentstroke}%
\pgfsetdash{}{0pt}%
\pgfpathmoveto{\pgfqpoint{14.619193in}{8.899169in}}%
\pgfpathlineto{\pgfqpoint{14.619193in}{8.897277in}}%
\pgfusepath{stroke}%
\end{pgfscope}%
\begin{pgfscope}%
\pgfpathrectangle{\pgfqpoint{9.810417in}{8.820930in}}{\pgfqpoint{5.489583in}{0.877907in}}%
\pgfusepath{clip}%
\pgfsetbuttcap%
\pgfsetroundjoin%
\pgfsetlinewidth{1.505625pt}%
\definecolor{currentstroke}{rgb}{0.000000,0.000000,0.000000}%
\pgfsetstrokecolor{currentstroke}%
\pgfsetdash{}{0pt}%
\pgfpathmoveto{\pgfqpoint{14.742416in}{8.899169in}}%
\pgfpathlineto{\pgfqpoint{14.742416in}{8.917881in}}%
\pgfusepath{stroke}%
\end{pgfscope}%
\begin{pgfscope}%
\pgfpathrectangle{\pgfqpoint{9.810417in}{8.820930in}}{\pgfqpoint{5.489583in}{0.877907in}}%
\pgfusepath{clip}%
\pgfsetbuttcap%
\pgfsetroundjoin%
\pgfsetlinewidth{1.505625pt}%
\definecolor{currentstroke}{rgb}{0.000000,0.000000,0.000000}%
\pgfsetstrokecolor{currentstroke}%
\pgfsetdash{}{0pt}%
\pgfpathmoveto{\pgfqpoint{14.865639in}{8.899169in}}%
\pgfpathlineto{\pgfqpoint{14.865639in}{8.916638in}}%
\pgfusepath{stroke}%
\end{pgfscope}%
\begin{pgfscope}%
\pgfpathrectangle{\pgfqpoint{9.810417in}{8.820930in}}{\pgfqpoint{5.489583in}{0.877907in}}%
\pgfusepath{clip}%
\pgfsetbuttcap%
\pgfsetroundjoin%
\pgfsetlinewidth{1.505625pt}%
\definecolor{currentstroke}{rgb}{0.000000,0.000000,0.000000}%
\pgfsetstrokecolor{currentstroke}%
\pgfsetdash{}{0pt}%
\pgfpathmoveto{\pgfqpoint{14.988862in}{8.899169in}}%
\pgfpathlineto{\pgfqpoint{14.988862in}{8.914839in}}%
\pgfusepath{stroke}%
\end{pgfscope}%
\begin{pgfscope}%
\pgfpathrectangle{\pgfqpoint{9.810417in}{8.820930in}}{\pgfqpoint{5.489583in}{0.877907in}}%
\pgfusepath{clip}%
\pgfsetroundcap%
\pgfsetroundjoin%
\pgfsetlinewidth{1.505625pt}%
\definecolor{currentstroke}{rgb}{0.121569,0.466667,0.705882}%
\pgfsetstrokecolor{currentstroke}%
\pgfsetdash{}{0pt}%
\pgfpathmoveto{\pgfqpoint{9.810417in}{8.899169in}}%
\pgfpathlineto{\pgfqpoint{15.300000in}{8.899169in}}%
\pgfusepath{stroke}%
\end{pgfscope}%
\begin{pgfscope}%
\pgfpathrectangle{\pgfqpoint{9.810417in}{8.820930in}}{\pgfqpoint{5.489583in}{0.877907in}}%
\pgfusepath{clip}%
\pgfsetbuttcap%
\pgfsetroundjoin%
\definecolor{currentfill}{rgb}{0.121569,0.466667,0.705882}%
\pgfsetfillcolor{currentfill}%
\pgfsetlinewidth{1.003750pt}%
\definecolor{currentstroke}{rgb}{0.121569,0.466667,0.705882}%
\pgfsetstrokecolor{currentstroke}%
\pgfsetdash{}{0pt}%
\pgfsys@defobject{currentmarker}{\pgfqpoint{-0.034722in}{-0.034722in}}{\pgfqpoint{0.034722in}{0.034722in}}{%
\pgfpathmoveto{\pgfqpoint{0.000000in}{-0.034722in}}%
\pgfpathcurveto{\pgfqpoint{0.009208in}{-0.034722in}}{\pgfqpoint{0.018041in}{-0.031064in}}{\pgfqpoint{0.024552in}{-0.024552in}}%
\pgfpathcurveto{\pgfqpoint{0.031064in}{-0.018041in}}{\pgfqpoint{0.034722in}{-0.009208in}}{\pgfqpoint{0.034722in}{0.000000in}}%
\pgfpathcurveto{\pgfqpoint{0.034722in}{0.009208in}}{\pgfqpoint{0.031064in}{0.018041in}}{\pgfqpoint{0.024552in}{0.024552in}}%
\pgfpathcurveto{\pgfqpoint{0.018041in}{0.031064in}}{\pgfqpoint{0.009208in}{0.034722in}}{\pgfqpoint{0.000000in}{0.034722in}}%
\pgfpathcurveto{\pgfqpoint{-0.009208in}{0.034722in}}{\pgfqpoint{-0.018041in}{0.031064in}}{\pgfqpoint{-0.024552in}{0.024552in}}%
\pgfpathcurveto{\pgfqpoint{-0.031064in}{0.018041in}}{\pgfqpoint{-0.034722in}{0.009208in}}{\pgfqpoint{-0.034722in}{0.000000in}}%
\pgfpathcurveto{\pgfqpoint{-0.034722in}{-0.009208in}}{\pgfqpoint{-0.031064in}{-0.018041in}}{\pgfqpoint{-0.024552in}{-0.024552in}}%
\pgfpathcurveto{\pgfqpoint{-0.018041in}{-0.031064in}}{\pgfqpoint{-0.009208in}{-0.034722in}}{\pgfqpoint{0.000000in}{-0.034722in}}%
\pgfpathclose%
\pgfusepath{stroke,fill}%
}%
\begin{pgfscope}%
\pgfsys@transformshift{10.059943in}{9.658932in}%
\pgfsys@useobject{currentmarker}{}%
\end{pgfscope}%
\begin{pgfscope}%
\pgfsys@transformshift{10.183166in}{9.656717in}%
\pgfsys@useobject{currentmarker}{}%
\end{pgfscope}%
\begin{pgfscope}%
\pgfsys@transformshift{10.306389in}{8.903534in}%
\pgfsys@useobject{currentmarker}{}%
\end{pgfscope}%
\begin{pgfscope}%
\pgfsys@transformshift{10.429612in}{8.904148in}%
\pgfsys@useobject{currentmarker}{}%
\end{pgfscope}%
\begin{pgfscope}%
\pgfsys@transformshift{10.552835in}{8.897394in}%
\pgfsys@useobject{currentmarker}{}%
\end{pgfscope}%
\begin{pgfscope}%
\pgfsys@transformshift{10.676058in}{8.927893in}%
\pgfsys@useobject{currentmarker}{}%
\end{pgfscope}%
\begin{pgfscope}%
\pgfsys@transformshift{10.799281in}{8.880456in}%
\pgfsys@useobject{currentmarker}{}%
\end{pgfscope}%
\begin{pgfscope}%
\pgfsys@transformshift{10.922504in}{8.881465in}%
\pgfsys@useobject{currentmarker}{}%
\end{pgfscope}%
\begin{pgfscope}%
\pgfsys@transformshift{11.045727in}{8.892266in}%
\pgfsys@useobject{currentmarker}{}%
\end{pgfscope}%
\begin{pgfscope}%
\pgfsys@transformshift{11.168950in}{8.893383in}%
\pgfsys@useobject{currentmarker}{}%
\end{pgfscope}%
\begin{pgfscope}%
\pgfsys@transformshift{11.292173in}{8.913614in}%
\pgfsys@useobject{currentmarker}{}%
\end{pgfscope}%
\begin{pgfscope}%
\pgfsys@transformshift{11.415396in}{8.920884in}%
\pgfsys@useobject{currentmarker}{}%
\end{pgfscope}%
\begin{pgfscope}%
\pgfsys@transformshift{11.538619in}{8.899769in}%
\pgfsys@useobject{currentmarker}{}%
\end{pgfscope}%
\begin{pgfscope}%
\pgfsys@transformshift{11.661842in}{8.911209in}%
\pgfsys@useobject{currentmarker}{}%
\end{pgfscope}%
\begin{pgfscope}%
\pgfsys@transformshift{11.785065in}{8.877984in}%
\pgfsys@useobject{currentmarker}{}%
\end{pgfscope}%
\begin{pgfscope}%
\pgfsys@transformshift{11.908288in}{8.889088in}%
\pgfsys@useobject{currentmarker}{}%
\end{pgfscope}%
\begin{pgfscope}%
\pgfsys@transformshift{12.031511in}{8.917679in}%
\pgfsys@useobject{currentmarker}{}%
\end{pgfscope}%
\begin{pgfscope}%
\pgfsys@transformshift{12.154734in}{8.865466in}%
\pgfsys@useobject{currentmarker}{}%
\end{pgfscope}%
\begin{pgfscope}%
\pgfsys@transformshift{12.277957in}{8.875761in}%
\pgfsys@useobject{currentmarker}{}%
\end{pgfscope}%
\begin{pgfscope}%
\pgfsys@transformshift{12.401180in}{8.875551in}%
\pgfsys@useobject{currentmarker}{}%
\end{pgfscope}%
\begin{pgfscope}%
\pgfsys@transformshift{12.524403in}{8.904916in}%
\pgfsys@useobject{currentmarker}{}%
\end{pgfscope}%
\begin{pgfscope}%
\pgfsys@transformshift{12.647626in}{8.909892in}%
\pgfsys@useobject{currentmarker}{}%
\end{pgfscope}%
\begin{pgfscope}%
\pgfsys@transformshift{12.770849in}{8.898822in}%
\pgfsys@useobject{currentmarker}{}%
\end{pgfscope}%
\begin{pgfscope}%
\pgfsys@transformshift{12.894072in}{8.884663in}%
\pgfsys@useobject{currentmarker}{}%
\end{pgfscope}%
\begin{pgfscope}%
\pgfsys@transformshift{13.017294in}{8.907107in}%
\pgfsys@useobject{currentmarker}{}%
\end{pgfscope}%
\begin{pgfscope}%
\pgfsys@transformshift{13.140517in}{8.892804in}%
\pgfsys@useobject{currentmarker}{}%
\end{pgfscope}%
\begin{pgfscope}%
\pgfsys@transformshift{13.263740in}{8.921495in}%
\pgfsys@useobject{currentmarker}{}%
\end{pgfscope}%
\begin{pgfscope}%
\pgfsys@transformshift{13.386963in}{8.892523in}%
\pgfsys@useobject{currentmarker}{}%
\end{pgfscope}%
\begin{pgfscope}%
\pgfsys@transformshift{13.510186in}{8.902054in}%
\pgfsys@useobject{currentmarker}{}%
\end{pgfscope}%
\begin{pgfscope}%
\pgfsys@transformshift{13.633409in}{8.884742in}%
\pgfsys@useobject{currentmarker}{}%
\end{pgfscope}%
\begin{pgfscope}%
\pgfsys@transformshift{13.756632in}{8.874818in}%
\pgfsys@useobject{currentmarker}{}%
\end{pgfscope}%
\begin{pgfscope}%
\pgfsys@transformshift{13.879855in}{8.921886in}%
\pgfsys@useobject{currentmarker}{}%
\end{pgfscope}%
\begin{pgfscope}%
\pgfsys@transformshift{14.003078in}{8.907410in}%
\pgfsys@useobject{currentmarker}{}%
\end{pgfscope}%
\begin{pgfscope}%
\pgfsys@transformshift{14.126301in}{8.906983in}%
\pgfsys@useobject{currentmarker}{}%
\end{pgfscope}%
\begin{pgfscope}%
\pgfsys@transformshift{14.249524in}{8.907925in}%
\pgfsys@useobject{currentmarker}{}%
\end{pgfscope}%
\begin{pgfscope}%
\pgfsys@transformshift{14.372747in}{8.919555in}%
\pgfsys@useobject{currentmarker}{}%
\end{pgfscope}%
\begin{pgfscope}%
\pgfsys@transformshift{14.495970in}{8.952129in}%
\pgfsys@useobject{currentmarker}{}%
\end{pgfscope}%
\begin{pgfscope}%
\pgfsys@transformshift{14.619193in}{8.897277in}%
\pgfsys@useobject{currentmarker}{}%
\end{pgfscope}%
\begin{pgfscope}%
\pgfsys@transformshift{14.742416in}{8.917881in}%
\pgfsys@useobject{currentmarker}{}%
\end{pgfscope}%
\begin{pgfscope}%
\pgfsys@transformshift{14.865639in}{8.916638in}%
\pgfsys@useobject{currentmarker}{}%
\end{pgfscope}%
\begin{pgfscope}%
\pgfsys@transformshift{14.988862in}{8.914839in}%
\pgfsys@useobject{currentmarker}{}%
\end{pgfscope}%
\end{pgfscope}%
\begin{pgfscope}%
\pgfsetrectcap%
\pgfsetmiterjoin%
\pgfsetlinewidth{0.803000pt}%
\definecolor{currentstroke}{rgb}{1.000000,1.000000,1.000000}%
\pgfsetstrokecolor{currentstroke}%
\pgfsetdash{}{0pt}%
\pgfpathmoveto{\pgfqpoint{9.810417in}{8.820930in}}%
\pgfpathlineto{\pgfqpoint{9.810417in}{9.698837in}}%
\pgfusepath{stroke}%
\end{pgfscope}%
\begin{pgfscope}%
\pgfsetrectcap%
\pgfsetmiterjoin%
\pgfsetlinewidth{0.803000pt}%
\definecolor{currentstroke}{rgb}{1.000000,1.000000,1.000000}%
\pgfsetstrokecolor{currentstroke}%
\pgfsetdash{}{0pt}%
\pgfpathmoveto{\pgfqpoint{15.300000in}{8.820930in}}%
\pgfpathlineto{\pgfqpoint{15.300000in}{9.698837in}}%
\pgfusepath{stroke}%
\end{pgfscope}%
\begin{pgfscope}%
\pgfsetrectcap%
\pgfsetmiterjoin%
\pgfsetlinewidth{0.803000pt}%
\definecolor{currentstroke}{rgb}{1.000000,1.000000,1.000000}%
\pgfsetstrokecolor{currentstroke}%
\pgfsetdash{}{0pt}%
\pgfpathmoveto{\pgfqpoint{9.810417in}{8.820930in}}%
\pgfpathlineto{\pgfqpoint{15.300000in}{8.820930in}}%
\pgfusepath{stroke}%
\end{pgfscope}%
\begin{pgfscope}%
\pgfsetrectcap%
\pgfsetmiterjoin%
\pgfsetlinewidth{0.803000pt}%
\definecolor{currentstroke}{rgb}{1.000000,1.000000,1.000000}%
\pgfsetstrokecolor{currentstroke}%
\pgfsetdash{}{0pt}%
\pgfpathmoveto{\pgfqpoint{9.810417in}{9.698837in}}%
\pgfpathlineto{\pgfqpoint{15.300000in}{9.698837in}}%
\pgfusepath{stroke}%
\end{pgfscope}%
\begin{pgfscope}%
\definecolor{textcolor}{rgb}{0.150000,0.150000,0.150000}%
\pgfsetstrokecolor{textcolor}%
\pgfsetfillcolor{textcolor}%
\pgftext[x=12.555208in,y=9.782171in,,base]{\color{textcolor}\rmfamily\fontsize{16.800000}{20.160000}\selectfont Partial Autocorrelation}%
\end{pgfscope}%
\begin{pgfscope}%
\pgfsetbuttcap%
\pgfsetmiterjoin%
\definecolor{currentfill}{rgb}{0.917647,0.917647,0.949020}%
\pgfsetfillcolor{currentfill}%
\pgfsetlinewidth{0.000000pt}%
\definecolor{currentstroke}{rgb}{0.000000,0.000000,0.000000}%
\pgfsetstrokecolor{currentstroke}%
\pgfsetstrokeopacity{0.000000}%
\pgfsetdash{}{0pt}%
\pgfpathmoveto{\pgfqpoint{2.125000in}{7.240698in}}%
\pgfpathlineto{\pgfqpoint{7.614583in}{7.240698in}}%
\pgfpathlineto{\pgfqpoint{7.614583in}{8.118605in}}%
\pgfpathlineto{\pgfqpoint{2.125000in}{8.118605in}}%
\pgfpathclose%
\pgfusepath{fill}%
\end{pgfscope}%
\begin{pgfscope}%
\pgfpathrectangle{\pgfqpoint{2.125000in}{7.240698in}}{\pgfqpoint{5.489583in}{0.877907in}}%
\pgfusepath{clip}%
\pgfsetroundcap%
\pgfsetroundjoin%
\pgfsetlinewidth{0.803000pt}%
\definecolor{currentstroke}{rgb}{1.000000,1.000000,1.000000}%
\pgfsetstrokecolor{currentstroke}%
\pgfsetdash{}{0pt}%
\pgfpathmoveto{\pgfqpoint{2.374527in}{7.240698in}}%
\pgfpathlineto{\pgfqpoint{2.374527in}{8.118605in}}%
\pgfusepath{stroke}%
\end{pgfscope}%
\begin{pgfscope}%
\definecolor{textcolor}{rgb}{0.150000,0.150000,0.150000}%
\pgfsetstrokecolor{textcolor}%
\pgfsetfillcolor{textcolor}%
\pgftext[x=2.374527in,y=7.143475in,,top]{\color{textcolor}\rmfamily\fontsize{14.000000}{16.800000}\selectfont 0}%
\end{pgfscope}%
\begin{pgfscope}%
\pgfpathrectangle{\pgfqpoint{2.125000in}{7.240698in}}{\pgfqpoint{5.489583in}{0.877907in}}%
\pgfusepath{clip}%
\pgfsetroundcap%
\pgfsetroundjoin%
\pgfsetlinewidth{0.803000pt}%
\definecolor{currentstroke}{rgb}{1.000000,1.000000,1.000000}%
\pgfsetstrokecolor{currentstroke}%
\pgfsetdash{}{0pt}%
\pgfpathmoveto{\pgfqpoint{2.990641in}{7.240698in}}%
\pgfpathlineto{\pgfqpoint{2.990641in}{8.118605in}}%
\pgfusepath{stroke}%
\end{pgfscope}%
\begin{pgfscope}%
\definecolor{textcolor}{rgb}{0.150000,0.150000,0.150000}%
\pgfsetstrokecolor{textcolor}%
\pgfsetfillcolor{textcolor}%
\pgftext[x=2.990641in,y=7.143475in,,top]{\color{textcolor}\rmfamily\fontsize{14.000000}{16.800000}\selectfont 5}%
\end{pgfscope}%
\begin{pgfscope}%
\pgfpathrectangle{\pgfqpoint{2.125000in}{7.240698in}}{\pgfqpoint{5.489583in}{0.877907in}}%
\pgfusepath{clip}%
\pgfsetroundcap%
\pgfsetroundjoin%
\pgfsetlinewidth{0.803000pt}%
\definecolor{currentstroke}{rgb}{1.000000,1.000000,1.000000}%
\pgfsetstrokecolor{currentstroke}%
\pgfsetdash{}{0pt}%
\pgfpathmoveto{\pgfqpoint{3.606756in}{7.240698in}}%
\pgfpathlineto{\pgfqpoint{3.606756in}{8.118605in}}%
\pgfusepath{stroke}%
\end{pgfscope}%
\begin{pgfscope}%
\definecolor{textcolor}{rgb}{0.150000,0.150000,0.150000}%
\pgfsetstrokecolor{textcolor}%
\pgfsetfillcolor{textcolor}%
\pgftext[x=3.606756in,y=7.143475in,,top]{\color{textcolor}\rmfamily\fontsize{14.000000}{16.800000}\selectfont 10}%
\end{pgfscope}%
\begin{pgfscope}%
\pgfpathrectangle{\pgfqpoint{2.125000in}{7.240698in}}{\pgfqpoint{5.489583in}{0.877907in}}%
\pgfusepath{clip}%
\pgfsetroundcap%
\pgfsetroundjoin%
\pgfsetlinewidth{0.803000pt}%
\definecolor{currentstroke}{rgb}{1.000000,1.000000,1.000000}%
\pgfsetstrokecolor{currentstroke}%
\pgfsetdash{}{0pt}%
\pgfpathmoveto{\pgfqpoint{4.222871in}{7.240698in}}%
\pgfpathlineto{\pgfqpoint{4.222871in}{8.118605in}}%
\pgfusepath{stroke}%
\end{pgfscope}%
\begin{pgfscope}%
\definecolor{textcolor}{rgb}{0.150000,0.150000,0.150000}%
\pgfsetstrokecolor{textcolor}%
\pgfsetfillcolor{textcolor}%
\pgftext[x=4.222871in,y=7.143475in,,top]{\color{textcolor}\rmfamily\fontsize{14.000000}{16.800000}\selectfont 15}%
\end{pgfscope}%
\begin{pgfscope}%
\pgfpathrectangle{\pgfqpoint{2.125000in}{7.240698in}}{\pgfqpoint{5.489583in}{0.877907in}}%
\pgfusepath{clip}%
\pgfsetroundcap%
\pgfsetroundjoin%
\pgfsetlinewidth{0.803000pt}%
\definecolor{currentstroke}{rgb}{1.000000,1.000000,1.000000}%
\pgfsetstrokecolor{currentstroke}%
\pgfsetdash{}{0pt}%
\pgfpathmoveto{\pgfqpoint{4.838986in}{7.240698in}}%
\pgfpathlineto{\pgfqpoint{4.838986in}{8.118605in}}%
\pgfusepath{stroke}%
\end{pgfscope}%
\begin{pgfscope}%
\definecolor{textcolor}{rgb}{0.150000,0.150000,0.150000}%
\pgfsetstrokecolor{textcolor}%
\pgfsetfillcolor{textcolor}%
\pgftext[x=4.838986in,y=7.143475in,,top]{\color{textcolor}\rmfamily\fontsize{14.000000}{16.800000}\selectfont 20}%
\end{pgfscope}%
\begin{pgfscope}%
\pgfpathrectangle{\pgfqpoint{2.125000in}{7.240698in}}{\pgfqpoint{5.489583in}{0.877907in}}%
\pgfusepath{clip}%
\pgfsetroundcap%
\pgfsetroundjoin%
\pgfsetlinewidth{0.803000pt}%
\definecolor{currentstroke}{rgb}{1.000000,1.000000,1.000000}%
\pgfsetstrokecolor{currentstroke}%
\pgfsetdash{}{0pt}%
\pgfpathmoveto{\pgfqpoint{5.455101in}{7.240698in}}%
\pgfpathlineto{\pgfqpoint{5.455101in}{8.118605in}}%
\pgfusepath{stroke}%
\end{pgfscope}%
\begin{pgfscope}%
\definecolor{textcolor}{rgb}{0.150000,0.150000,0.150000}%
\pgfsetstrokecolor{textcolor}%
\pgfsetfillcolor{textcolor}%
\pgftext[x=5.455101in,y=7.143475in,,top]{\color{textcolor}\rmfamily\fontsize{14.000000}{16.800000}\selectfont 25}%
\end{pgfscope}%
\begin{pgfscope}%
\pgfpathrectangle{\pgfqpoint{2.125000in}{7.240698in}}{\pgfqpoint{5.489583in}{0.877907in}}%
\pgfusepath{clip}%
\pgfsetroundcap%
\pgfsetroundjoin%
\pgfsetlinewidth{0.803000pt}%
\definecolor{currentstroke}{rgb}{1.000000,1.000000,1.000000}%
\pgfsetstrokecolor{currentstroke}%
\pgfsetdash{}{0pt}%
\pgfpathmoveto{\pgfqpoint{6.071216in}{7.240698in}}%
\pgfpathlineto{\pgfqpoint{6.071216in}{8.118605in}}%
\pgfusepath{stroke}%
\end{pgfscope}%
\begin{pgfscope}%
\definecolor{textcolor}{rgb}{0.150000,0.150000,0.150000}%
\pgfsetstrokecolor{textcolor}%
\pgfsetfillcolor{textcolor}%
\pgftext[x=6.071216in,y=7.143475in,,top]{\color{textcolor}\rmfamily\fontsize{14.000000}{16.800000}\selectfont 30}%
\end{pgfscope}%
\begin{pgfscope}%
\pgfpathrectangle{\pgfqpoint{2.125000in}{7.240698in}}{\pgfqpoint{5.489583in}{0.877907in}}%
\pgfusepath{clip}%
\pgfsetroundcap%
\pgfsetroundjoin%
\pgfsetlinewidth{0.803000pt}%
\definecolor{currentstroke}{rgb}{1.000000,1.000000,1.000000}%
\pgfsetstrokecolor{currentstroke}%
\pgfsetdash{}{0pt}%
\pgfpathmoveto{\pgfqpoint{6.687330in}{7.240698in}}%
\pgfpathlineto{\pgfqpoint{6.687330in}{8.118605in}}%
\pgfusepath{stroke}%
\end{pgfscope}%
\begin{pgfscope}%
\definecolor{textcolor}{rgb}{0.150000,0.150000,0.150000}%
\pgfsetstrokecolor{textcolor}%
\pgfsetfillcolor{textcolor}%
\pgftext[x=6.687330in,y=7.143475in,,top]{\color{textcolor}\rmfamily\fontsize{14.000000}{16.800000}\selectfont 35}%
\end{pgfscope}%
\begin{pgfscope}%
\pgfpathrectangle{\pgfqpoint{2.125000in}{7.240698in}}{\pgfqpoint{5.489583in}{0.877907in}}%
\pgfusepath{clip}%
\pgfsetroundcap%
\pgfsetroundjoin%
\pgfsetlinewidth{0.803000pt}%
\definecolor{currentstroke}{rgb}{1.000000,1.000000,1.000000}%
\pgfsetstrokecolor{currentstroke}%
\pgfsetdash{}{0pt}%
\pgfpathmoveto{\pgfqpoint{7.303445in}{7.240698in}}%
\pgfpathlineto{\pgfqpoint{7.303445in}{8.118605in}}%
\pgfusepath{stroke}%
\end{pgfscope}%
\begin{pgfscope}%
\definecolor{textcolor}{rgb}{0.150000,0.150000,0.150000}%
\pgfsetstrokecolor{textcolor}%
\pgfsetfillcolor{textcolor}%
\pgftext[x=7.303445in,y=7.143475in,,top]{\color{textcolor}\rmfamily\fontsize{14.000000}{16.800000}\selectfont 40}%
\end{pgfscope}%
\begin{pgfscope}%
\pgfpathrectangle{\pgfqpoint{2.125000in}{7.240698in}}{\pgfqpoint{5.489583in}{0.877907in}}%
\pgfusepath{clip}%
\pgfsetroundcap%
\pgfsetroundjoin%
\pgfsetlinewidth{0.803000pt}%
\definecolor{currentstroke}{rgb}{1.000000,1.000000,1.000000}%
\pgfsetstrokecolor{currentstroke}%
\pgfsetdash{}{0pt}%
\pgfpathmoveto{\pgfqpoint{2.125000in}{7.514385in}}%
\pgfpathlineto{\pgfqpoint{7.614583in}{7.514385in}}%
\pgfusepath{stroke}%
\end{pgfscope}%
\begin{pgfscope}%
\definecolor{textcolor}{rgb}{0.150000,0.150000,0.150000}%
\pgfsetstrokecolor{textcolor}%
\pgfsetfillcolor{textcolor}%
\pgftext[x=1.904066in,y=7.440518in,left,base]{\color{textcolor}\rmfamily\fontsize{14.000000}{16.800000}\selectfont 0}%
\end{pgfscope}%
\begin{pgfscope}%
\pgfpathrectangle{\pgfqpoint{2.125000in}{7.240698in}}{\pgfqpoint{5.489583in}{0.877907in}}%
\pgfusepath{clip}%
\pgfsetroundcap%
\pgfsetroundjoin%
\pgfsetlinewidth{0.803000pt}%
\definecolor{currentstroke}{rgb}{1.000000,1.000000,1.000000}%
\pgfsetstrokecolor{currentstroke}%
\pgfsetdash{}{0pt}%
\pgfpathmoveto{\pgfqpoint{2.125000in}{8.078700in}}%
\pgfpathlineto{\pgfqpoint{7.614583in}{8.078700in}}%
\pgfusepath{stroke}%
\end{pgfscope}%
\begin{pgfscope}%
\definecolor{textcolor}{rgb}{0.150000,0.150000,0.150000}%
\pgfsetstrokecolor{textcolor}%
\pgfsetfillcolor{textcolor}%
\pgftext[x=1.904066in,y=8.004834in,left,base]{\color{textcolor}\rmfamily\fontsize{14.000000}{16.800000}\selectfont 1}%
\end{pgfscope}%
\begin{pgfscope}%
\pgfpathrectangle{\pgfqpoint{2.125000in}{7.240698in}}{\pgfqpoint{5.489583in}{0.877907in}}%
\pgfusepath{clip}%
\pgfsetbuttcap%
\pgfsetroundjoin%
\definecolor{currentfill}{rgb}{0.121569,0.466667,0.705882}%
\pgfsetfillcolor{currentfill}%
\pgfsetfillopacity{0.250000}%
\pgfsetlinewidth{1.003750pt}%
\definecolor{currentstroke}{rgb}{1.000000,1.000000,1.000000}%
\pgfsetstrokecolor{currentstroke}%
\pgfsetstrokeopacity{0.250000}%
\pgfsetdash{}{0pt}%
\pgfpathmoveto{\pgfqpoint{2.436138in}{7.542857in}}%
\pgfpathlineto{\pgfqpoint{2.436138in}{7.485912in}}%
\pgfpathlineto{\pgfqpoint{2.620972in}{7.465211in}}%
\pgfpathlineto{\pgfqpoint{2.744195in}{7.451052in}}%
\pgfpathlineto{\pgfqpoint{2.867418in}{7.439618in}}%
\pgfpathlineto{\pgfqpoint{2.990641in}{7.429800in}}%
\pgfpathlineto{\pgfqpoint{3.113864in}{7.421083in}}%
\pgfpathlineto{\pgfqpoint{3.237087in}{7.413177in}}%
\pgfpathlineto{\pgfqpoint{3.360310in}{7.405902in}}%
\pgfpathlineto{\pgfqpoint{3.483533in}{7.399140in}}%
\pgfpathlineto{\pgfqpoint{3.606756in}{7.392806in}}%
\pgfpathlineto{\pgfqpoint{3.729979in}{7.386838in}}%
\pgfpathlineto{\pgfqpoint{3.853202in}{7.381183in}}%
\pgfpathlineto{\pgfqpoint{3.976425in}{7.375803in}}%
\pgfpathlineto{\pgfqpoint{4.099648in}{7.370670in}}%
\pgfpathlineto{\pgfqpoint{4.222871in}{7.365755in}}%
\pgfpathlineto{\pgfqpoint{4.346094in}{7.361035in}}%
\pgfpathlineto{\pgfqpoint{4.469317in}{7.356492in}}%
\pgfpathlineto{\pgfqpoint{4.592540in}{7.352111in}}%
\pgfpathlineto{\pgfqpoint{4.715763in}{7.347878in}}%
\pgfpathlineto{\pgfqpoint{4.838986in}{7.343783in}}%
\pgfpathlineto{\pgfqpoint{4.962209in}{7.339816in}}%
\pgfpathlineto{\pgfqpoint{5.085432in}{7.335968in}}%
\pgfpathlineto{\pgfqpoint{5.208655in}{7.332231in}}%
\pgfpathlineto{\pgfqpoint{5.331878in}{7.328596in}}%
\pgfpathlineto{\pgfqpoint{5.455101in}{7.325059in}}%
\pgfpathlineto{\pgfqpoint{5.578324in}{7.321613in}}%
\pgfpathlineto{\pgfqpoint{5.701547in}{7.318253in}}%
\pgfpathlineto{\pgfqpoint{5.824770in}{7.314971in}}%
\pgfpathlineto{\pgfqpoint{5.947993in}{7.311765in}}%
\pgfpathlineto{\pgfqpoint{6.071216in}{7.308630in}}%
\pgfpathlineto{\pgfqpoint{6.194439in}{7.305564in}}%
\pgfpathlineto{\pgfqpoint{6.317662in}{7.302565in}}%
\pgfpathlineto{\pgfqpoint{6.440885in}{7.299627in}}%
\pgfpathlineto{\pgfqpoint{6.564108in}{7.296749in}}%
\pgfpathlineto{\pgfqpoint{6.687330in}{7.293925in}}%
\pgfpathlineto{\pgfqpoint{6.810553in}{7.291156in}}%
\pgfpathlineto{\pgfqpoint{6.933776in}{7.288440in}}%
\pgfpathlineto{\pgfqpoint{7.056999in}{7.285778in}}%
\pgfpathlineto{\pgfqpoint{7.180222in}{7.283166in}}%
\pgfpathlineto{\pgfqpoint{7.365057in}{7.280603in}}%
\pgfpathlineto{\pgfqpoint{7.365057in}{7.748166in}}%
\pgfpathlineto{\pgfqpoint{7.365057in}{7.748166in}}%
\pgfpathlineto{\pgfqpoint{7.180222in}{7.745603in}}%
\pgfpathlineto{\pgfqpoint{7.056999in}{7.742991in}}%
\pgfpathlineto{\pgfqpoint{6.933776in}{7.740329in}}%
\pgfpathlineto{\pgfqpoint{6.810553in}{7.737613in}}%
\pgfpathlineto{\pgfqpoint{6.687330in}{7.734844in}}%
\pgfpathlineto{\pgfqpoint{6.564108in}{7.732020in}}%
\pgfpathlineto{\pgfqpoint{6.440885in}{7.729142in}}%
\pgfpathlineto{\pgfqpoint{6.317662in}{7.726204in}}%
\pgfpathlineto{\pgfqpoint{6.194439in}{7.723205in}}%
\pgfpathlineto{\pgfqpoint{6.071216in}{7.720139in}}%
\pgfpathlineto{\pgfqpoint{5.947993in}{7.717004in}}%
\pgfpathlineto{\pgfqpoint{5.824770in}{7.713798in}}%
\pgfpathlineto{\pgfqpoint{5.701547in}{7.710516in}}%
\pgfpathlineto{\pgfqpoint{5.578324in}{7.707156in}}%
\pgfpathlineto{\pgfqpoint{5.455101in}{7.703710in}}%
\pgfpathlineto{\pgfqpoint{5.331878in}{7.700173in}}%
\pgfpathlineto{\pgfqpoint{5.208655in}{7.696538in}}%
\pgfpathlineto{\pgfqpoint{5.085432in}{7.692801in}}%
\pgfpathlineto{\pgfqpoint{4.962209in}{7.688953in}}%
\pgfpathlineto{\pgfqpoint{4.838986in}{7.684986in}}%
\pgfpathlineto{\pgfqpoint{4.715763in}{7.680891in}}%
\pgfpathlineto{\pgfqpoint{4.592540in}{7.676658in}}%
\pgfpathlineto{\pgfqpoint{4.469317in}{7.672277in}}%
\pgfpathlineto{\pgfqpoint{4.346094in}{7.667734in}}%
\pgfpathlineto{\pgfqpoint{4.222871in}{7.663014in}}%
\pgfpathlineto{\pgfqpoint{4.099648in}{7.658099in}}%
\pgfpathlineto{\pgfqpoint{3.976425in}{7.652966in}}%
\pgfpathlineto{\pgfqpoint{3.853202in}{7.647586in}}%
\pgfpathlineto{\pgfqpoint{3.729979in}{7.641931in}}%
\pgfpathlineto{\pgfqpoint{3.606756in}{7.635963in}}%
\pgfpathlineto{\pgfqpoint{3.483533in}{7.629629in}}%
\pgfpathlineto{\pgfqpoint{3.360310in}{7.622867in}}%
\pgfpathlineto{\pgfqpoint{3.237087in}{7.615592in}}%
\pgfpathlineto{\pgfqpoint{3.113864in}{7.607686in}}%
\pgfpathlineto{\pgfqpoint{2.990641in}{7.598969in}}%
\pgfpathlineto{\pgfqpoint{2.867418in}{7.589151in}}%
\pgfpathlineto{\pgfqpoint{2.744195in}{7.577717in}}%
\pgfpathlineto{\pgfqpoint{2.620972in}{7.563558in}}%
\pgfpathlineto{\pgfqpoint{2.436138in}{7.542857in}}%
\pgfpathclose%
\pgfusepath{stroke,fill}%
\end{pgfscope}%
\begin{pgfscope}%
\pgfpathrectangle{\pgfqpoint{2.125000in}{7.240698in}}{\pgfqpoint{5.489583in}{0.877907in}}%
\pgfusepath{clip}%
\pgfsetbuttcap%
\pgfsetroundjoin%
\pgfsetlinewidth{1.505625pt}%
\definecolor{currentstroke}{rgb}{0.000000,0.000000,0.000000}%
\pgfsetstrokecolor{currentstroke}%
\pgfsetdash{}{0pt}%
\pgfpathmoveto{\pgfqpoint{2.374527in}{7.514385in}}%
\pgfpathlineto{\pgfqpoint{2.374527in}{8.078700in}}%
\pgfusepath{stroke}%
\end{pgfscope}%
\begin{pgfscope}%
\pgfpathrectangle{\pgfqpoint{2.125000in}{7.240698in}}{\pgfqpoint{5.489583in}{0.877907in}}%
\pgfusepath{clip}%
\pgfsetbuttcap%
\pgfsetroundjoin%
\pgfsetlinewidth{1.505625pt}%
\definecolor{currentstroke}{rgb}{0.000000,0.000000,0.000000}%
\pgfsetstrokecolor{currentstroke}%
\pgfsetdash{}{0pt}%
\pgfpathmoveto{\pgfqpoint{2.497749in}{7.514385in}}%
\pgfpathlineto{\pgfqpoint{2.497749in}{8.076247in}}%
\pgfusepath{stroke}%
\end{pgfscope}%
\begin{pgfscope}%
\pgfpathrectangle{\pgfqpoint{2.125000in}{7.240698in}}{\pgfqpoint{5.489583in}{0.877907in}}%
\pgfusepath{clip}%
\pgfsetbuttcap%
\pgfsetroundjoin%
\pgfsetlinewidth{1.505625pt}%
\definecolor{currentstroke}{rgb}{0.000000,0.000000,0.000000}%
\pgfsetstrokecolor{currentstroke}%
\pgfsetdash{}{0pt}%
\pgfpathmoveto{\pgfqpoint{2.620972in}{7.514385in}}%
\pgfpathlineto{\pgfqpoint{2.620972in}{8.073754in}}%
\pgfusepath{stroke}%
\end{pgfscope}%
\begin{pgfscope}%
\pgfpathrectangle{\pgfqpoint{2.125000in}{7.240698in}}{\pgfqpoint{5.489583in}{0.877907in}}%
\pgfusepath{clip}%
\pgfsetbuttcap%
\pgfsetroundjoin%
\pgfsetlinewidth{1.505625pt}%
\definecolor{currentstroke}{rgb}{0.000000,0.000000,0.000000}%
\pgfsetstrokecolor{currentstroke}%
\pgfsetdash{}{0pt}%
\pgfpathmoveto{\pgfqpoint{2.744195in}{7.514385in}}%
\pgfpathlineto{\pgfqpoint{2.744195in}{8.071266in}}%
\pgfusepath{stroke}%
\end{pgfscope}%
\begin{pgfscope}%
\pgfpathrectangle{\pgfqpoint{2.125000in}{7.240698in}}{\pgfqpoint{5.489583in}{0.877907in}}%
\pgfusepath{clip}%
\pgfsetbuttcap%
\pgfsetroundjoin%
\pgfsetlinewidth{1.505625pt}%
\definecolor{currentstroke}{rgb}{0.000000,0.000000,0.000000}%
\pgfsetstrokecolor{currentstroke}%
\pgfsetdash{}{0pt}%
\pgfpathmoveto{\pgfqpoint{2.867418in}{7.514385in}}%
\pgfpathlineto{\pgfqpoint{2.867418in}{8.068713in}}%
\pgfusepath{stroke}%
\end{pgfscope}%
\begin{pgfscope}%
\pgfpathrectangle{\pgfqpoint{2.125000in}{7.240698in}}{\pgfqpoint{5.489583in}{0.877907in}}%
\pgfusepath{clip}%
\pgfsetbuttcap%
\pgfsetroundjoin%
\pgfsetlinewidth{1.505625pt}%
\definecolor{currentstroke}{rgb}{0.000000,0.000000,0.000000}%
\pgfsetstrokecolor{currentstroke}%
\pgfsetdash{}{0pt}%
\pgfpathmoveto{\pgfqpoint{2.990641in}{7.514385in}}%
\pgfpathlineto{\pgfqpoint{2.990641in}{8.066264in}}%
\pgfusepath{stroke}%
\end{pgfscope}%
\begin{pgfscope}%
\pgfpathrectangle{\pgfqpoint{2.125000in}{7.240698in}}{\pgfqpoint{5.489583in}{0.877907in}}%
\pgfusepath{clip}%
\pgfsetbuttcap%
\pgfsetroundjoin%
\pgfsetlinewidth{1.505625pt}%
\definecolor{currentstroke}{rgb}{0.000000,0.000000,0.000000}%
\pgfsetstrokecolor{currentstroke}%
\pgfsetdash{}{0pt}%
\pgfpathmoveto{\pgfqpoint{3.113864in}{7.514385in}}%
\pgfpathlineto{\pgfqpoint{3.113864in}{8.063966in}}%
\pgfusepath{stroke}%
\end{pgfscope}%
\begin{pgfscope}%
\pgfpathrectangle{\pgfqpoint{2.125000in}{7.240698in}}{\pgfqpoint{5.489583in}{0.877907in}}%
\pgfusepath{clip}%
\pgfsetbuttcap%
\pgfsetroundjoin%
\pgfsetlinewidth{1.505625pt}%
\definecolor{currentstroke}{rgb}{0.000000,0.000000,0.000000}%
\pgfsetstrokecolor{currentstroke}%
\pgfsetdash{}{0pt}%
\pgfpathmoveto{\pgfqpoint{3.237087in}{7.514385in}}%
\pgfpathlineto{\pgfqpoint{3.237087in}{8.061742in}}%
\pgfusepath{stroke}%
\end{pgfscope}%
\begin{pgfscope}%
\pgfpathrectangle{\pgfqpoint{2.125000in}{7.240698in}}{\pgfqpoint{5.489583in}{0.877907in}}%
\pgfusepath{clip}%
\pgfsetbuttcap%
\pgfsetroundjoin%
\pgfsetlinewidth{1.505625pt}%
\definecolor{currentstroke}{rgb}{0.000000,0.000000,0.000000}%
\pgfsetstrokecolor{currentstroke}%
\pgfsetdash{}{0pt}%
\pgfpathmoveto{\pgfqpoint{3.360310in}{7.514385in}}%
\pgfpathlineto{\pgfqpoint{3.360310in}{8.059523in}}%
\pgfusepath{stroke}%
\end{pgfscope}%
\begin{pgfscope}%
\pgfpathrectangle{\pgfqpoint{2.125000in}{7.240698in}}{\pgfqpoint{5.489583in}{0.877907in}}%
\pgfusepath{clip}%
\pgfsetbuttcap%
\pgfsetroundjoin%
\pgfsetlinewidth{1.505625pt}%
\definecolor{currentstroke}{rgb}{0.000000,0.000000,0.000000}%
\pgfsetstrokecolor{currentstroke}%
\pgfsetdash{}{0pt}%
\pgfpathmoveto{\pgfqpoint{3.483533in}{7.514385in}}%
\pgfpathlineto{\pgfqpoint{3.483533in}{8.057148in}}%
\pgfusepath{stroke}%
\end{pgfscope}%
\begin{pgfscope}%
\pgfpathrectangle{\pgfqpoint{2.125000in}{7.240698in}}{\pgfqpoint{5.489583in}{0.877907in}}%
\pgfusepath{clip}%
\pgfsetbuttcap%
\pgfsetroundjoin%
\pgfsetlinewidth{1.505625pt}%
\definecolor{currentstroke}{rgb}{0.000000,0.000000,0.000000}%
\pgfsetstrokecolor{currentstroke}%
\pgfsetdash{}{0pt}%
\pgfpathmoveto{\pgfqpoint{3.606756in}{7.514385in}}%
\pgfpathlineto{\pgfqpoint{3.606756in}{8.054778in}}%
\pgfusepath{stroke}%
\end{pgfscope}%
\begin{pgfscope}%
\pgfpathrectangle{\pgfqpoint{2.125000in}{7.240698in}}{\pgfqpoint{5.489583in}{0.877907in}}%
\pgfusepath{clip}%
\pgfsetbuttcap%
\pgfsetroundjoin%
\pgfsetlinewidth{1.505625pt}%
\definecolor{currentstroke}{rgb}{0.000000,0.000000,0.000000}%
\pgfsetstrokecolor{currentstroke}%
\pgfsetdash{}{0pt}%
\pgfpathmoveto{\pgfqpoint{3.729979in}{7.514385in}}%
\pgfpathlineto{\pgfqpoint{3.729979in}{8.052557in}}%
\pgfusepath{stroke}%
\end{pgfscope}%
\begin{pgfscope}%
\pgfpathrectangle{\pgfqpoint{2.125000in}{7.240698in}}{\pgfqpoint{5.489583in}{0.877907in}}%
\pgfusepath{clip}%
\pgfsetbuttcap%
\pgfsetroundjoin%
\pgfsetlinewidth{1.505625pt}%
\definecolor{currentstroke}{rgb}{0.000000,0.000000,0.000000}%
\pgfsetstrokecolor{currentstroke}%
\pgfsetdash{}{0pt}%
\pgfpathmoveto{\pgfqpoint{3.853202in}{7.514385in}}%
\pgfpathlineto{\pgfqpoint{3.853202in}{8.050251in}}%
\pgfusepath{stroke}%
\end{pgfscope}%
\begin{pgfscope}%
\pgfpathrectangle{\pgfqpoint{2.125000in}{7.240698in}}{\pgfqpoint{5.489583in}{0.877907in}}%
\pgfusepath{clip}%
\pgfsetbuttcap%
\pgfsetroundjoin%
\pgfsetlinewidth{1.505625pt}%
\definecolor{currentstroke}{rgb}{0.000000,0.000000,0.000000}%
\pgfsetstrokecolor{currentstroke}%
\pgfsetdash{}{0pt}%
\pgfpathmoveto{\pgfqpoint{3.976425in}{7.514385in}}%
\pgfpathlineto{\pgfqpoint{3.976425in}{8.047897in}}%
\pgfusepath{stroke}%
\end{pgfscope}%
\begin{pgfscope}%
\pgfpathrectangle{\pgfqpoint{2.125000in}{7.240698in}}{\pgfqpoint{5.489583in}{0.877907in}}%
\pgfusepath{clip}%
\pgfsetbuttcap%
\pgfsetroundjoin%
\pgfsetlinewidth{1.505625pt}%
\definecolor{currentstroke}{rgb}{0.000000,0.000000,0.000000}%
\pgfsetstrokecolor{currentstroke}%
\pgfsetdash{}{0pt}%
\pgfpathmoveto{\pgfqpoint{4.099648in}{7.514385in}}%
\pgfpathlineto{\pgfqpoint{4.099648in}{8.045634in}}%
\pgfusepath{stroke}%
\end{pgfscope}%
\begin{pgfscope}%
\pgfpathrectangle{\pgfqpoint{2.125000in}{7.240698in}}{\pgfqpoint{5.489583in}{0.877907in}}%
\pgfusepath{clip}%
\pgfsetbuttcap%
\pgfsetroundjoin%
\pgfsetlinewidth{1.505625pt}%
\definecolor{currentstroke}{rgb}{0.000000,0.000000,0.000000}%
\pgfsetstrokecolor{currentstroke}%
\pgfsetdash{}{0pt}%
\pgfpathmoveto{\pgfqpoint{4.222871in}{7.514385in}}%
\pgfpathlineto{\pgfqpoint{4.222871in}{8.043489in}}%
\pgfusepath{stroke}%
\end{pgfscope}%
\begin{pgfscope}%
\pgfpathrectangle{\pgfqpoint{2.125000in}{7.240698in}}{\pgfqpoint{5.489583in}{0.877907in}}%
\pgfusepath{clip}%
\pgfsetbuttcap%
\pgfsetroundjoin%
\pgfsetlinewidth{1.505625pt}%
\definecolor{currentstroke}{rgb}{0.000000,0.000000,0.000000}%
\pgfsetstrokecolor{currentstroke}%
\pgfsetdash{}{0pt}%
\pgfpathmoveto{\pgfqpoint{4.346094in}{7.514385in}}%
\pgfpathlineto{\pgfqpoint{4.346094in}{8.041350in}}%
\pgfusepath{stroke}%
\end{pgfscope}%
\begin{pgfscope}%
\pgfpathrectangle{\pgfqpoint{2.125000in}{7.240698in}}{\pgfqpoint{5.489583in}{0.877907in}}%
\pgfusepath{clip}%
\pgfsetbuttcap%
\pgfsetroundjoin%
\pgfsetlinewidth{1.505625pt}%
\definecolor{currentstroke}{rgb}{0.000000,0.000000,0.000000}%
\pgfsetstrokecolor{currentstroke}%
\pgfsetdash{}{0pt}%
\pgfpathmoveto{\pgfqpoint{4.469317in}{7.514385in}}%
\pgfpathlineto{\pgfqpoint{4.469317in}{8.039282in}}%
\pgfusepath{stroke}%
\end{pgfscope}%
\begin{pgfscope}%
\pgfpathrectangle{\pgfqpoint{2.125000in}{7.240698in}}{\pgfqpoint{5.489583in}{0.877907in}}%
\pgfusepath{clip}%
\pgfsetbuttcap%
\pgfsetroundjoin%
\pgfsetlinewidth{1.505625pt}%
\definecolor{currentstroke}{rgb}{0.000000,0.000000,0.000000}%
\pgfsetstrokecolor{currentstroke}%
\pgfsetdash{}{0pt}%
\pgfpathmoveto{\pgfqpoint{4.592540in}{7.514385in}}%
\pgfpathlineto{\pgfqpoint{4.592540in}{8.037195in}}%
\pgfusepath{stroke}%
\end{pgfscope}%
\begin{pgfscope}%
\pgfpathrectangle{\pgfqpoint{2.125000in}{7.240698in}}{\pgfqpoint{5.489583in}{0.877907in}}%
\pgfusepath{clip}%
\pgfsetbuttcap%
\pgfsetroundjoin%
\pgfsetlinewidth{1.505625pt}%
\definecolor{currentstroke}{rgb}{0.000000,0.000000,0.000000}%
\pgfsetstrokecolor{currentstroke}%
\pgfsetdash{}{0pt}%
\pgfpathmoveto{\pgfqpoint{4.715763in}{7.514385in}}%
\pgfpathlineto{\pgfqpoint{4.715763in}{8.035070in}}%
\pgfusepath{stroke}%
\end{pgfscope}%
\begin{pgfscope}%
\pgfpathrectangle{\pgfqpoint{2.125000in}{7.240698in}}{\pgfqpoint{5.489583in}{0.877907in}}%
\pgfusepath{clip}%
\pgfsetbuttcap%
\pgfsetroundjoin%
\pgfsetlinewidth{1.505625pt}%
\definecolor{currentstroke}{rgb}{0.000000,0.000000,0.000000}%
\pgfsetstrokecolor{currentstroke}%
\pgfsetdash{}{0pt}%
\pgfpathmoveto{\pgfqpoint{4.838986in}{7.514385in}}%
\pgfpathlineto{\pgfqpoint{4.838986in}{8.033007in}}%
\pgfusepath{stroke}%
\end{pgfscope}%
\begin{pgfscope}%
\pgfpathrectangle{\pgfqpoint{2.125000in}{7.240698in}}{\pgfqpoint{5.489583in}{0.877907in}}%
\pgfusepath{clip}%
\pgfsetbuttcap%
\pgfsetroundjoin%
\pgfsetlinewidth{1.505625pt}%
\definecolor{currentstroke}{rgb}{0.000000,0.000000,0.000000}%
\pgfsetstrokecolor{currentstroke}%
\pgfsetdash{}{0pt}%
\pgfpathmoveto{\pgfqpoint{4.962209in}{7.514385in}}%
\pgfpathlineto{\pgfqpoint{4.962209in}{8.030903in}}%
\pgfusepath{stroke}%
\end{pgfscope}%
\begin{pgfscope}%
\pgfpathrectangle{\pgfqpoint{2.125000in}{7.240698in}}{\pgfqpoint{5.489583in}{0.877907in}}%
\pgfusepath{clip}%
\pgfsetbuttcap%
\pgfsetroundjoin%
\pgfsetlinewidth{1.505625pt}%
\definecolor{currentstroke}{rgb}{0.000000,0.000000,0.000000}%
\pgfsetstrokecolor{currentstroke}%
\pgfsetdash{}{0pt}%
\pgfpathmoveto{\pgfqpoint{5.085432in}{7.514385in}}%
\pgfpathlineto{\pgfqpoint{5.085432in}{8.028828in}}%
\pgfusepath{stroke}%
\end{pgfscope}%
\begin{pgfscope}%
\pgfpathrectangle{\pgfqpoint{2.125000in}{7.240698in}}{\pgfqpoint{5.489583in}{0.877907in}}%
\pgfusepath{clip}%
\pgfsetbuttcap%
\pgfsetroundjoin%
\pgfsetlinewidth{1.505625pt}%
\definecolor{currentstroke}{rgb}{0.000000,0.000000,0.000000}%
\pgfsetstrokecolor{currentstroke}%
\pgfsetdash{}{0pt}%
\pgfpathmoveto{\pgfqpoint{5.208655in}{7.514385in}}%
\pgfpathlineto{\pgfqpoint{5.208655in}{8.026882in}}%
\pgfusepath{stroke}%
\end{pgfscope}%
\begin{pgfscope}%
\pgfpathrectangle{\pgfqpoint{2.125000in}{7.240698in}}{\pgfqpoint{5.489583in}{0.877907in}}%
\pgfusepath{clip}%
\pgfsetbuttcap%
\pgfsetroundjoin%
\pgfsetlinewidth{1.505625pt}%
\definecolor{currentstroke}{rgb}{0.000000,0.000000,0.000000}%
\pgfsetstrokecolor{currentstroke}%
\pgfsetdash{}{0pt}%
\pgfpathmoveto{\pgfqpoint{5.331878in}{7.514385in}}%
\pgfpathlineto{\pgfqpoint{5.331878in}{8.024897in}}%
\pgfusepath{stroke}%
\end{pgfscope}%
\begin{pgfscope}%
\pgfpathrectangle{\pgfqpoint{2.125000in}{7.240698in}}{\pgfqpoint{5.489583in}{0.877907in}}%
\pgfusepath{clip}%
\pgfsetbuttcap%
\pgfsetroundjoin%
\pgfsetlinewidth{1.505625pt}%
\definecolor{currentstroke}{rgb}{0.000000,0.000000,0.000000}%
\pgfsetstrokecolor{currentstroke}%
\pgfsetdash{}{0pt}%
\pgfpathmoveto{\pgfqpoint{5.455101in}{7.514385in}}%
\pgfpathlineto{\pgfqpoint{5.455101in}{8.022899in}}%
\pgfusepath{stroke}%
\end{pgfscope}%
\begin{pgfscope}%
\pgfpathrectangle{\pgfqpoint{2.125000in}{7.240698in}}{\pgfqpoint{5.489583in}{0.877907in}}%
\pgfusepath{clip}%
\pgfsetbuttcap%
\pgfsetroundjoin%
\pgfsetlinewidth{1.505625pt}%
\definecolor{currentstroke}{rgb}{0.000000,0.000000,0.000000}%
\pgfsetstrokecolor{currentstroke}%
\pgfsetdash{}{0pt}%
\pgfpathmoveto{\pgfqpoint{5.578324in}{7.514385in}}%
\pgfpathlineto{\pgfqpoint{5.578324in}{8.021028in}}%
\pgfusepath{stroke}%
\end{pgfscope}%
\begin{pgfscope}%
\pgfpathrectangle{\pgfqpoint{2.125000in}{7.240698in}}{\pgfqpoint{5.489583in}{0.877907in}}%
\pgfusepath{clip}%
\pgfsetbuttcap%
\pgfsetroundjoin%
\pgfsetlinewidth{1.505625pt}%
\definecolor{currentstroke}{rgb}{0.000000,0.000000,0.000000}%
\pgfsetstrokecolor{currentstroke}%
\pgfsetdash{}{0pt}%
\pgfpathmoveto{\pgfqpoint{5.701547in}{7.514385in}}%
\pgfpathlineto{\pgfqpoint{5.701547in}{8.019284in}}%
\pgfusepath{stroke}%
\end{pgfscope}%
\begin{pgfscope}%
\pgfpathrectangle{\pgfqpoint{2.125000in}{7.240698in}}{\pgfqpoint{5.489583in}{0.877907in}}%
\pgfusepath{clip}%
\pgfsetbuttcap%
\pgfsetroundjoin%
\pgfsetlinewidth{1.505625pt}%
\definecolor{currentstroke}{rgb}{0.000000,0.000000,0.000000}%
\pgfsetstrokecolor{currentstroke}%
\pgfsetdash{}{0pt}%
\pgfpathmoveto{\pgfqpoint{5.824770in}{7.514385in}}%
\pgfpathlineto{\pgfqpoint{5.824770in}{8.017557in}}%
\pgfusepath{stroke}%
\end{pgfscope}%
\begin{pgfscope}%
\pgfpathrectangle{\pgfqpoint{2.125000in}{7.240698in}}{\pgfqpoint{5.489583in}{0.877907in}}%
\pgfusepath{clip}%
\pgfsetbuttcap%
\pgfsetroundjoin%
\pgfsetlinewidth{1.505625pt}%
\definecolor{currentstroke}{rgb}{0.000000,0.000000,0.000000}%
\pgfsetstrokecolor{currentstroke}%
\pgfsetdash{}{0pt}%
\pgfpathmoveto{\pgfqpoint{5.947993in}{7.514385in}}%
\pgfpathlineto{\pgfqpoint{5.947993in}{8.015860in}}%
\pgfusepath{stroke}%
\end{pgfscope}%
\begin{pgfscope}%
\pgfpathrectangle{\pgfqpoint{2.125000in}{7.240698in}}{\pgfqpoint{5.489583in}{0.877907in}}%
\pgfusepath{clip}%
\pgfsetbuttcap%
\pgfsetroundjoin%
\pgfsetlinewidth{1.505625pt}%
\definecolor{currentstroke}{rgb}{0.000000,0.000000,0.000000}%
\pgfsetstrokecolor{currentstroke}%
\pgfsetdash{}{0pt}%
\pgfpathmoveto{\pgfqpoint{6.071216in}{7.514385in}}%
\pgfpathlineto{\pgfqpoint{6.071216in}{8.014035in}}%
\pgfusepath{stroke}%
\end{pgfscope}%
\begin{pgfscope}%
\pgfpathrectangle{\pgfqpoint{2.125000in}{7.240698in}}{\pgfqpoint{5.489583in}{0.877907in}}%
\pgfusepath{clip}%
\pgfsetbuttcap%
\pgfsetroundjoin%
\pgfsetlinewidth{1.505625pt}%
\definecolor{currentstroke}{rgb}{0.000000,0.000000,0.000000}%
\pgfsetstrokecolor{currentstroke}%
\pgfsetdash{}{0pt}%
\pgfpathmoveto{\pgfqpoint{6.194439in}{7.514385in}}%
\pgfpathlineto{\pgfqpoint{6.194439in}{8.012172in}}%
\pgfusepath{stroke}%
\end{pgfscope}%
\begin{pgfscope}%
\pgfpathrectangle{\pgfqpoint{2.125000in}{7.240698in}}{\pgfqpoint{5.489583in}{0.877907in}}%
\pgfusepath{clip}%
\pgfsetbuttcap%
\pgfsetroundjoin%
\pgfsetlinewidth{1.505625pt}%
\definecolor{currentstroke}{rgb}{0.000000,0.000000,0.000000}%
\pgfsetstrokecolor{currentstroke}%
\pgfsetdash{}{0pt}%
\pgfpathmoveto{\pgfqpoint{6.317662in}{7.514385in}}%
\pgfpathlineto{\pgfqpoint{6.317662in}{8.010449in}}%
\pgfusepath{stroke}%
\end{pgfscope}%
\begin{pgfscope}%
\pgfpathrectangle{\pgfqpoint{2.125000in}{7.240698in}}{\pgfqpoint{5.489583in}{0.877907in}}%
\pgfusepath{clip}%
\pgfsetbuttcap%
\pgfsetroundjoin%
\pgfsetlinewidth{1.505625pt}%
\definecolor{currentstroke}{rgb}{0.000000,0.000000,0.000000}%
\pgfsetstrokecolor{currentstroke}%
\pgfsetdash{}{0pt}%
\pgfpathmoveto{\pgfqpoint{6.440885in}{7.514385in}}%
\pgfpathlineto{\pgfqpoint{6.440885in}{8.008842in}}%
\pgfusepath{stroke}%
\end{pgfscope}%
\begin{pgfscope}%
\pgfpathrectangle{\pgfqpoint{2.125000in}{7.240698in}}{\pgfqpoint{5.489583in}{0.877907in}}%
\pgfusepath{clip}%
\pgfsetbuttcap%
\pgfsetroundjoin%
\pgfsetlinewidth{1.505625pt}%
\definecolor{currentstroke}{rgb}{0.000000,0.000000,0.000000}%
\pgfsetstrokecolor{currentstroke}%
\pgfsetdash{}{0pt}%
\pgfpathmoveto{\pgfqpoint{6.564108in}{7.514385in}}%
\pgfpathlineto{\pgfqpoint{6.564108in}{8.007270in}}%
\pgfusepath{stroke}%
\end{pgfscope}%
\begin{pgfscope}%
\pgfpathrectangle{\pgfqpoint{2.125000in}{7.240698in}}{\pgfqpoint{5.489583in}{0.877907in}}%
\pgfusepath{clip}%
\pgfsetbuttcap%
\pgfsetroundjoin%
\pgfsetlinewidth{1.505625pt}%
\definecolor{currentstroke}{rgb}{0.000000,0.000000,0.000000}%
\pgfsetstrokecolor{currentstroke}%
\pgfsetdash{}{0pt}%
\pgfpathmoveto{\pgfqpoint{6.687330in}{7.514385in}}%
\pgfpathlineto{\pgfqpoint{6.687330in}{8.005645in}}%
\pgfusepath{stroke}%
\end{pgfscope}%
\begin{pgfscope}%
\pgfpathrectangle{\pgfqpoint{2.125000in}{7.240698in}}{\pgfqpoint{5.489583in}{0.877907in}}%
\pgfusepath{clip}%
\pgfsetbuttcap%
\pgfsetroundjoin%
\pgfsetlinewidth{1.505625pt}%
\definecolor{currentstroke}{rgb}{0.000000,0.000000,0.000000}%
\pgfsetstrokecolor{currentstroke}%
\pgfsetdash{}{0pt}%
\pgfpathmoveto{\pgfqpoint{6.810553in}{7.514385in}}%
\pgfpathlineto{\pgfqpoint{6.810553in}{8.003833in}}%
\pgfusepath{stroke}%
\end{pgfscope}%
\begin{pgfscope}%
\pgfpathrectangle{\pgfqpoint{2.125000in}{7.240698in}}{\pgfqpoint{5.489583in}{0.877907in}}%
\pgfusepath{clip}%
\pgfsetbuttcap%
\pgfsetroundjoin%
\pgfsetlinewidth{1.505625pt}%
\definecolor{currentstroke}{rgb}{0.000000,0.000000,0.000000}%
\pgfsetstrokecolor{currentstroke}%
\pgfsetdash{}{0pt}%
\pgfpathmoveto{\pgfqpoint{6.933776in}{7.514385in}}%
\pgfpathlineto{\pgfqpoint{6.933776in}{8.001976in}}%
\pgfusepath{stroke}%
\end{pgfscope}%
\begin{pgfscope}%
\pgfpathrectangle{\pgfqpoint{2.125000in}{7.240698in}}{\pgfqpoint{5.489583in}{0.877907in}}%
\pgfusepath{clip}%
\pgfsetbuttcap%
\pgfsetroundjoin%
\pgfsetlinewidth{1.505625pt}%
\definecolor{currentstroke}{rgb}{0.000000,0.000000,0.000000}%
\pgfsetstrokecolor{currentstroke}%
\pgfsetdash{}{0pt}%
\pgfpathmoveto{\pgfqpoint{7.056999in}{7.514385in}}%
\pgfpathlineto{\pgfqpoint{7.056999in}{8.000079in}}%
\pgfusepath{stroke}%
\end{pgfscope}%
\begin{pgfscope}%
\pgfpathrectangle{\pgfqpoint{2.125000in}{7.240698in}}{\pgfqpoint{5.489583in}{0.877907in}}%
\pgfusepath{clip}%
\pgfsetbuttcap%
\pgfsetroundjoin%
\pgfsetlinewidth{1.505625pt}%
\definecolor{currentstroke}{rgb}{0.000000,0.000000,0.000000}%
\pgfsetstrokecolor{currentstroke}%
\pgfsetdash{}{0pt}%
\pgfpathmoveto{\pgfqpoint{7.180222in}{7.514385in}}%
\pgfpathlineto{\pgfqpoint{7.180222in}{7.998204in}}%
\pgfusepath{stroke}%
\end{pgfscope}%
\begin{pgfscope}%
\pgfpathrectangle{\pgfqpoint{2.125000in}{7.240698in}}{\pgfqpoint{5.489583in}{0.877907in}}%
\pgfusepath{clip}%
\pgfsetbuttcap%
\pgfsetroundjoin%
\pgfsetlinewidth{1.505625pt}%
\definecolor{currentstroke}{rgb}{0.000000,0.000000,0.000000}%
\pgfsetstrokecolor{currentstroke}%
\pgfsetdash{}{0pt}%
\pgfpathmoveto{\pgfqpoint{7.303445in}{7.514385in}}%
\pgfpathlineto{\pgfqpoint{7.303445in}{7.996364in}}%
\pgfusepath{stroke}%
\end{pgfscope}%
\begin{pgfscope}%
\pgfpathrectangle{\pgfqpoint{2.125000in}{7.240698in}}{\pgfqpoint{5.489583in}{0.877907in}}%
\pgfusepath{clip}%
\pgfsetroundcap%
\pgfsetroundjoin%
\pgfsetlinewidth{1.505625pt}%
\definecolor{currentstroke}{rgb}{0.121569,0.466667,0.705882}%
\pgfsetstrokecolor{currentstroke}%
\pgfsetdash{}{0pt}%
\pgfpathmoveto{\pgfqpoint{2.125000in}{7.514385in}}%
\pgfpathlineto{\pgfqpoint{7.614583in}{7.514385in}}%
\pgfusepath{stroke}%
\end{pgfscope}%
\begin{pgfscope}%
\pgfpathrectangle{\pgfqpoint{2.125000in}{7.240698in}}{\pgfqpoint{5.489583in}{0.877907in}}%
\pgfusepath{clip}%
\pgfsetbuttcap%
\pgfsetroundjoin%
\definecolor{currentfill}{rgb}{0.121569,0.466667,0.705882}%
\pgfsetfillcolor{currentfill}%
\pgfsetlinewidth{1.003750pt}%
\definecolor{currentstroke}{rgb}{0.121569,0.466667,0.705882}%
\pgfsetstrokecolor{currentstroke}%
\pgfsetdash{}{0pt}%
\pgfsys@defobject{currentmarker}{\pgfqpoint{-0.034722in}{-0.034722in}}{\pgfqpoint{0.034722in}{0.034722in}}{%
\pgfpathmoveto{\pgfqpoint{0.000000in}{-0.034722in}}%
\pgfpathcurveto{\pgfqpoint{0.009208in}{-0.034722in}}{\pgfqpoint{0.018041in}{-0.031064in}}{\pgfqpoint{0.024552in}{-0.024552in}}%
\pgfpathcurveto{\pgfqpoint{0.031064in}{-0.018041in}}{\pgfqpoint{0.034722in}{-0.009208in}}{\pgfqpoint{0.034722in}{0.000000in}}%
\pgfpathcurveto{\pgfqpoint{0.034722in}{0.009208in}}{\pgfqpoint{0.031064in}{0.018041in}}{\pgfqpoint{0.024552in}{0.024552in}}%
\pgfpathcurveto{\pgfqpoint{0.018041in}{0.031064in}}{\pgfqpoint{0.009208in}{0.034722in}}{\pgfqpoint{0.000000in}{0.034722in}}%
\pgfpathcurveto{\pgfqpoint{-0.009208in}{0.034722in}}{\pgfqpoint{-0.018041in}{0.031064in}}{\pgfqpoint{-0.024552in}{0.024552in}}%
\pgfpathcurveto{\pgfqpoint{-0.031064in}{0.018041in}}{\pgfqpoint{-0.034722in}{0.009208in}}{\pgfqpoint{-0.034722in}{0.000000in}}%
\pgfpathcurveto{\pgfqpoint{-0.034722in}{-0.009208in}}{\pgfqpoint{-0.031064in}{-0.018041in}}{\pgfqpoint{-0.024552in}{-0.024552in}}%
\pgfpathcurveto{\pgfqpoint{-0.018041in}{-0.031064in}}{\pgfqpoint{-0.009208in}{-0.034722in}}{\pgfqpoint{0.000000in}{-0.034722in}}%
\pgfpathclose%
\pgfusepath{stroke,fill}%
}%
\begin{pgfscope}%
\pgfsys@transformshift{2.374527in}{8.078700in}%
\pgfsys@useobject{currentmarker}{}%
\end{pgfscope}%
\begin{pgfscope}%
\pgfsys@transformshift{2.497749in}{8.076247in}%
\pgfsys@useobject{currentmarker}{}%
\end{pgfscope}%
\begin{pgfscope}%
\pgfsys@transformshift{2.620972in}{8.073754in}%
\pgfsys@useobject{currentmarker}{}%
\end{pgfscope}%
\begin{pgfscope}%
\pgfsys@transformshift{2.744195in}{8.071266in}%
\pgfsys@useobject{currentmarker}{}%
\end{pgfscope}%
\begin{pgfscope}%
\pgfsys@transformshift{2.867418in}{8.068713in}%
\pgfsys@useobject{currentmarker}{}%
\end{pgfscope}%
\begin{pgfscope}%
\pgfsys@transformshift{2.990641in}{8.066264in}%
\pgfsys@useobject{currentmarker}{}%
\end{pgfscope}%
\begin{pgfscope}%
\pgfsys@transformshift{3.113864in}{8.063966in}%
\pgfsys@useobject{currentmarker}{}%
\end{pgfscope}%
\begin{pgfscope}%
\pgfsys@transformshift{3.237087in}{8.061742in}%
\pgfsys@useobject{currentmarker}{}%
\end{pgfscope}%
\begin{pgfscope}%
\pgfsys@transformshift{3.360310in}{8.059523in}%
\pgfsys@useobject{currentmarker}{}%
\end{pgfscope}%
\begin{pgfscope}%
\pgfsys@transformshift{3.483533in}{8.057148in}%
\pgfsys@useobject{currentmarker}{}%
\end{pgfscope}%
\begin{pgfscope}%
\pgfsys@transformshift{3.606756in}{8.054778in}%
\pgfsys@useobject{currentmarker}{}%
\end{pgfscope}%
\begin{pgfscope}%
\pgfsys@transformshift{3.729979in}{8.052557in}%
\pgfsys@useobject{currentmarker}{}%
\end{pgfscope}%
\begin{pgfscope}%
\pgfsys@transformshift{3.853202in}{8.050251in}%
\pgfsys@useobject{currentmarker}{}%
\end{pgfscope}%
\begin{pgfscope}%
\pgfsys@transformshift{3.976425in}{8.047897in}%
\pgfsys@useobject{currentmarker}{}%
\end{pgfscope}%
\begin{pgfscope}%
\pgfsys@transformshift{4.099648in}{8.045634in}%
\pgfsys@useobject{currentmarker}{}%
\end{pgfscope}%
\begin{pgfscope}%
\pgfsys@transformshift{4.222871in}{8.043489in}%
\pgfsys@useobject{currentmarker}{}%
\end{pgfscope}%
\begin{pgfscope}%
\pgfsys@transformshift{4.346094in}{8.041350in}%
\pgfsys@useobject{currentmarker}{}%
\end{pgfscope}%
\begin{pgfscope}%
\pgfsys@transformshift{4.469317in}{8.039282in}%
\pgfsys@useobject{currentmarker}{}%
\end{pgfscope}%
\begin{pgfscope}%
\pgfsys@transformshift{4.592540in}{8.037195in}%
\pgfsys@useobject{currentmarker}{}%
\end{pgfscope}%
\begin{pgfscope}%
\pgfsys@transformshift{4.715763in}{8.035070in}%
\pgfsys@useobject{currentmarker}{}%
\end{pgfscope}%
\begin{pgfscope}%
\pgfsys@transformshift{4.838986in}{8.033007in}%
\pgfsys@useobject{currentmarker}{}%
\end{pgfscope}%
\begin{pgfscope}%
\pgfsys@transformshift{4.962209in}{8.030903in}%
\pgfsys@useobject{currentmarker}{}%
\end{pgfscope}%
\begin{pgfscope}%
\pgfsys@transformshift{5.085432in}{8.028828in}%
\pgfsys@useobject{currentmarker}{}%
\end{pgfscope}%
\begin{pgfscope}%
\pgfsys@transformshift{5.208655in}{8.026882in}%
\pgfsys@useobject{currentmarker}{}%
\end{pgfscope}%
\begin{pgfscope}%
\pgfsys@transformshift{5.331878in}{8.024897in}%
\pgfsys@useobject{currentmarker}{}%
\end{pgfscope}%
\begin{pgfscope}%
\pgfsys@transformshift{5.455101in}{8.022899in}%
\pgfsys@useobject{currentmarker}{}%
\end{pgfscope}%
\begin{pgfscope}%
\pgfsys@transformshift{5.578324in}{8.021028in}%
\pgfsys@useobject{currentmarker}{}%
\end{pgfscope}%
\begin{pgfscope}%
\pgfsys@transformshift{5.701547in}{8.019284in}%
\pgfsys@useobject{currentmarker}{}%
\end{pgfscope}%
\begin{pgfscope}%
\pgfsys@transformshift{5.824770in}{8.017557in}%
\pgfsys@useobject{currentmarker}{}%
\end{pgfscope}%
\begin{pgfscope}%
\pgfsys@transformshift{5.947993in}{8.015860in}%
\pgfsys@useobject{currentmarker}{}%
\end{pgfscope}%
\begin{pgfscope}%
\pgfsys@transformshift{6.071216in}{8.014035in}%
\pgfsys@useobject{currentmarker}{}%
\end{pgfscope}%
\begin{pgfscope}%
\pgfsys@transformshift{6.194439in}{8.012172in}%
\pgfsys@useobject{currentmarker}{}%
\end{pgfscope}%
\begin{pgfscope}%
\pgfsys@transformshift{6.317662in}{8.010449in}%
\pgfsys@useobject{currentmarker}{}%
\end{pgfscope}%
\begin{pgfscope}%
\pgfsys@transformshift{6.440885in}{8.008842in}%
\pgfsys@useobject{currentmarker}{}%
\end{pgfscope}%
\begin{pgfscope}%
\pgfsys@transformshift{6.564108in}{8.007270in}%
\pgfsys@useobject{currentmarker}{}%
\end{pgfscope}%
\begin{pgfscope}%
\pgfsys@transformshift{6.687330in}{8.005645in}%
\pgfsys@useobject{currentmarker}{}%
\end{pgfscope}%
\begin{pgfscope}%
\pgfsys@transformshift{6.810553in}{8.003833in}%
\pgfsys@useobject{currentmarker}{}%
\end{pgfscope}%
\begin{pgfscope}%
\pgfsys@transformshift{6.933776in}{8.001976in}%
\pgfsys@useobject{currentmarker}{}%
\end{pgfscope}%
\begin{pgfscope}%
\pgfsys@transformshift{7.056999in}{8.000079in}%
\pgfsys@useobject{currentmarker}{}%
\end{pgfscope}%
\begin{pgfscope}%
\pgfsys@transformshift{7.180222in}{7.998204in}%
\pgfsys@useobject{currentmarker}{}%
\end{pgfscope}%
\begin{pgfscope}%
\pgfsys@transformshift{7.303445in}{7.996364in}%
\pgfsys@useobject{currentmarker}{}%
\end{pgfscope}%
\end{pgfscope}%
\begin{pgfscope}%
\pgfsetrectcap%
\pgfsetmiterjoin%
\pgfsetlinewidth{0.803000pt}%
\definecolor{currentstroke}{rgb}{1.000000,1.000000,1.000000}%
\pgfsetstrokecolor{currentstroke}%
\pgfsetdash{}{0pt}%
\pgfpathmoveto{\pgfqpoint{2.125000in}{7.240698in}}%
\pgfpathlineto{\pgfqpoint{2.125000in}{8.118605in}}%
\pgfusepath{stroke}%
\end{pgfscope}%
\begin{pgfscope}%
\pgfsetrectcap%
\pgfsetmiterjoin%
\pgfsetlinewidth{0.803000pt}%
\definecolor{currentstroke}{rgb}{1.000000,1.000000,1.000000}%
\pgfsetstrokecolor{currentstroke}%
\pgfsetdash{}{0pt}%
\pgfpathmoveto{\pgfqpoint{7.614583in}{7.240698in}}%
\pgfpathlineto{\pgfqpoint{7.614583in}{8.118605in}}%
\pgfusepath{stroke}%
\end{pgfscope}%
\begin{pgfscope}%
\pgfsetrectcap%
\pgfsetmiterjoin%
\pgfsetlinewidth{0.803000pt}%
\definecolor{currentstroke}{rgb}{1.000000,1.000000,1.000000}%
\pgfsetstrokecolor{currentstroke}%
\pgfsetdash{}{0pt}%
\pgfpathmoveto{\pgfqpoint{2.125000in}{7.240698in}}%
\pgfpathlineto{\pgfqpoint{7.614583in}{7.240698in}}%
\pgfusepath{stroke}%
\end{pgfscope}%
\begin{pgfscope}%
\pgfsetrectcap%
\pgfsetmiterjoin%
\pgfsetlinewidth{0.803000pt}%
\definecolor{currentstroke}{rgb}{1.000000,1.000000,1.000000}%
\pgfsetstrokecolor{currentstroke}%
\pgfsetdash{}{0pt}%
\pgfpathmoveto{\pgfqpoint{2.125000in}{8.118605in}}%
\pgfpathlineto{\pgfqpoint{7.614583in}{8.118605in}}%
\pgfusepath{stroke}%
\end{pgfscope}%
\begin{pgfscope}%
\definecolor{textcolor}{rgb}{0.150000,0.150000,0.150000}%
\pgfsetstrokecolor{textcolor}%
\pgfsetfillcolor{textcolor}%
\pgftext[x=4.869792in,y=8.201938in,,base]{\color{textcolor}\rmfamily\fontsize{16.800000}{20.160000}\selectfont Autocorrelation}%
\end{pgfscope}%
\begin{pgfscope}%
\pgfsetbuttcap%
\pgfsetmiterjoin%
\definecolor{currentfill}{rgb}{0.917647,0.917647,0.949020}%
\pgfsetfillcolor{currentfill}%
\pgfsetlinewidth{0.000000pt}%
\definecolor{currentstroke}{rgb}{0.000000,0.000000,0.000000}%
\pgfsetstrokecolor{currentstroke}%
\pgfsetstrokeopacity{0.000000}%
\pgfsetdash{}{0pt}%
\pgfpathmoveto{\pgfqpoint{9.810417in}{7.240698in}}%
\pgfpathlineto{\pgfqpoint{15.300000in}{7.240698in}}%
\pgfpathlineto{\pgfqpoint{15.300000in}{8.118605in}}%
\pgfpathlineto{\pgfqpoint{9.810417in}{8.118605in}}%
\pgfpathclose%
\pgfusepath{fill}%
\end{pgfscope}%
\begin{pgfscope}%
\pgfpathrectangle{\pgfqpoint{9.810417in}{7.240698in}}{\pgfqpoint{5.489583in}{0.877907in}}%
\pgfusepath{clip}%
\pgfsetroundcap%
\pgfsetroundjoin%
\pgfsetlinewidth{0.803000pt}%
\definecolor{currentstroke}{rgb}{1.000000,1.000000,1.000000}%
\pgfsetstrokecolor{currentstroke}%
\pgfsetdash{}{0pt}%
\pgfpathmoveto{\pgfqpoint{10.059943in}{7.240698in}}%
\pgfpathlineto{\pgfqpoint{10.059943in}{8.118605in}}%
\pgfusepath{stroke}%
\end{pgfscope}%
\begin{pgfscope}%
\definecolor{textcolor}{rgb}{0.150000,0.150000,0.150000}%
\pgfsetstrokecolor{textcolor}%
\pgfsetfillcolor{textcolor}%
\pgftext[x=10.059943in,y=7.143475in,,top]{\color{textcolor}\rmfamily\fontsize{14.000000}{16.800000}\selectfont 0}%
\end{pgfscope}%
\begin{pgfscope}%
\pgfpathrectangle{\pgfqpoint{9.810417in}{7.240698in}}{\pgfqpoint{5.489583in}{0.877907in}}%
\pgfusepath{clip}%
\pgfsetroundcap%
\pgfsetroundjoin%
\pgfsetlinewidth{0.803000pt}%
\definecolor{currentstroke}{rgb}{1.000000,1.000000,1.000000}%
\pgfsetstrokecolor{currentstroke}%
\pgfsetdash{}{0pt}%
\pgfpathmoveto{\pgfqpoint{10.676058in}{7.240698in}}%
\pgfpathlineto{\pgfqpoint{10.676058in}{8.118605in}}%
\pgfusepath{stroke}%
\end{pgfscope}%
\begin{pgfscope}%
\definecolor{textcolor}{rgb}{0.150000,0.150000,0.150000}%
\pgfsetstrokecolor{textcolor}%
\pgfsetfillcolor{textcolor}%
\pgftext[x=10.676058in,y=7.143475in,,top]{\color{textcolor}\rmfamily\fontsize{14.000000}{16.800000}\selectfont 5}%
\end{pgfscope}%
\begin{pgfscope}%
\pgfpathrectangle{\pgfqpoint{9.810417in}{7.240698in}}{\pgfqpoint{5.489583in}{0.877907in}}%
\pgfusepath{clip}%
\pgfsetroundcap%
\pgfsetroundjoin%
\pgfsetlinewidth{0.803000pt}%
\definecolor{currentstroke}{rgb}{1.000000,1.000000,1.000000}%
\pgfsetstrokecolor{currentstroke}%
\pgfsetdash{}{0pt}%
\pgfpathmoveto{\pgfqpoint{11.292173in}{7.240698in}}%
\pgfpathlineto{\pgfqpoint{11.292173in}{8.118605in}}%
\pgfusepath{stroke}%
\end{pgfscope}%
\begin{pgfscope}%
\definecolor{textcolor}{rgb}{0.150000,0.150000,0.150000}%
\pgfsetstrokecolor{textcolor}%
\pgfsetfillcolor{textcolor}%
\pgftext[x=11.292173in,y=7.143475in,,top]{\color{textcolor}\rmfamily\fontsize{14.000000}{16.800000}\selectfont 10}%
\end{pgfscope}%
\begin{pgfscope}%
\pgfpathrectangle{\pgfqpoint{9.810417in}{7.240698in}}{\pgfqpoint{5.489583in}{0.877907in}}%
\pgfusepath{clip}%
\pgfsetroundcap%
\pgfsetroundjoin%
\pgfsetlinewidth{0.803000pt}%
\definecolor{currentstroke}{rgb}{1.000000,1.000000,1.000000}%
\pgfsetstrokecolor{currentstroke}%
\pgfsetdash{}{0pt}%
\pgfpathmoveto{\pgfqpoint{11.908288in}{7.240698in}}%
\pgfpathlineto{\pgfqpoint{11.908288in}{8.118605in}}%
\pgfusepath{stroke}%
\end{pgfscope}%
\begin{pgfscope}%
\definecolor{textcolor}{rgb}{0.150000,0.150000,0.150000}%
\pgfsetstrokecolor{textcolor}%
\pgfsetfillcolor{textcolor}%
\pgftext[x=11.908288in,y=7.143475in,,top]{\color{textcolor}\rmfamily\fontsize{14.000000}{16.800000}\selectfont 15}%
\end{pgfscope}%
\begin{pgfscope}%
\pgfpathrectangle{\pgfqpoint{9.810417in}{7.240698in}}{\pgfqpoint{5.489583in}{0.877907in}}%
\pgfusepath{clip}%
\pgfsetroundcap%
\pgfsetroundjoin%
\pgfsetlinewidth{0.803000pt}%
\definecolor{currentstroke}{rgb}{1.000000,1.000000,1.000000}%
\pgfsetstrokecolor{currentstroke}%
\pgfsetdash{}{0pt}%
\pgfpathmoveto{\pgfqpoint{12.524403in}{7.240698in}}%
\pgfpathlineto{\pgfqpoint{12.524403in}{8.118605in}}%
\pgfusepath{stroke}%
\end{pgfscope}%
\begin{pgfscope}%
\definecolor{textcolor}{rgb}{0.150000,0.150000,0.150000}%
\pgfsetstrokecolor{textcolor}%
\pgfsetfillcolor{textcolor}%
\pgftext[x=12.524403in,y=7.143475in,,top]{\color{textcolor}\rmfamily\fontsize{14.000000}{16.800000}\selectfont 20}%
\end{pgfscope}%
\begin{pgfscope}%
\pgfpathrectangle{\pgfqpoint{9.810417in}{7.240698in}}{\pgfqpoint{5.489583in}{0.877907in}}%
\pgfusepath{clip}%
\pgfsetroundcap%
\pgfsetroundjoin%
\pgfsetlinewidth{0.803000pt}%
\definecolor{currentstroke}{rgb}{1.000000,1.000000,1.000000}%
\pgfsetstrokecolor{currentstroke}%
\pgfsetdash{}{0pt}%
\pgfpathmoveto{\pgfqpoint{13.140517in}{7.240698in}}%
\pgfpathlineto{\pgfqpoint{13.140517in}{8.118605in}}%
\pgfusepath{stroke}%
\end{pgfscope}%
\begin{pgfscope}%
\definecolor{textcolor}{rgb}{0.150000,0.150000,0.150000}%
\pgfsetstrokecolor{textcolor}%
\pgfsetfillcolor{textcolor}%
\pgftext[x=13.140517in,y=7.143475in,,top]{\color{textcolor}\rmfamily\fontsize{14.000000}{16.800000}\selectfont 25}%
\end{pgfscope}%
\begin{pgfscope}%
\pgfpathrectangle{\pgfqpoint{9.810417in}{7.240698in}}{\pgfqpoint{5.489583in}{0.877907in}}%
\pgfusepath{clip}%
\pgfsetroundcap%
\pgfsetroundjoin%
\pgfsetlinewidth{0.803000pt}%
\definecolor{currentstroke}{rgb}{1.000000,1.000000,1.000000}%
\pgfsetstrokecolor{currentstroke}%
\pgfsetdash{}{0pt}%
\pgfpathmoveto{\pgfqpoint{13.756632in}{7.240698in}}%
\pgfpathlineto{\pgfqpoint{13.756632in}{8.118605in}}%
\pgfusepath{stroke}%
\end{pgfscope}%
\begin{pgfscope}%
\definecolor{textcolor}{rgb}{0.150000,0.150000,0.150000}%
\pgfsetstrokecolor{textcolor}%
\pgfsetfillcolor{textcolor}%
\pgftext[x=13.756632in,y=7.143475in,,top]{\color{textcolor}\rmfamily\fontsize{14.000000}{16.800000}\selectfont 30}%
\end{pgfscope}%
\begin{pgfscope}%
\pgfpathrectangle{\pgfqpoint{9.810417in}{7.240698in}}{\pgfqpoint{5.489583in}{0.877907in}}%
\pgfusepath{clip}%
\pgfsetroundcap%
\pgfsetroundjoin%
\pgfsetlinewidth{0.803000pt}%
\definecolor{currentstroke}{rgb}{1.000000,1.000000,1.000000}%
\pgfsetstrokecolor{currentstroke}%
\pgfsetdash{}{0pt}%
\pgfpathmoveto{\pgfqpoint{14.372747in}{7.240698in}}%
\pgfpathlineto{\pgfqpoint{14.372747in}{8.118605in}}%
\pgfusepath{stroke}%
\end{pgfscope}%
\begin{pgfscope}%
\definecolor{textcolor}{rgb}{0.150000,0.150000,0.150000}%
\pgfsetstrokecolor{textcolor}%
\pgfsetfillcolor{textcolor}%
\pgftext[x=14.372747in,y=7.143475in,,top]{\color{textcolor}\rmfamily\fontsize{14.000000}{16.800000}\selectfont 35}%
\end{pgfscope}%
\begin{pgfscope}%
\pgfpathrectangle{\pgfqpoint{9.810417in}{7.240698in}}{\pgfqpoint{5.489583in}{0.877907in}}%
\pgfusepath{clip}%
\pgfsetroundcap%
\pgfsetroundjoin%
\pgfsetlinewidth{0.803000pt}%
\definecolor{currentstroke}{rgb}{1.000000,1.000000,1.000000}%
\pgfsetstrokecolor{currentstroke}%
\pgfsetdash{}{0pt}%
\pgfpathmoveto{\pgfqpoint{14.988862in}{7.240698in}}%
\pgfpathlineto{\pgfqpoint{14.988862in}{8.118605in}}%
\pgfusepath{stroke}%
\end{pgfscope}%
\begin{pgfscope}%
\definecolor{textcolor}{rgb}{0.150000,0.150000,0.150000}%
\pgfsetstrokecolor{textcolor}%
\pgfsetfillcolor{textcolor}%
\pgftext[x=14.988862in,y=7.143475in,,top]{\color{textcolor}\rmfamily\fontsize{14.000000}{16.800000}\selectfont 40}%
\end{pgfscope}%
\begin{pgfscope}%
\pgfpathrectangle{\pgfqpoint{9.810417in}{7.240698in}}{\pgfqpoint{5.489583in}{0.877907in}}%
\pgfusepath{clip}%
\pgfsetroundcap%
\pgfsetroundjoin%
\pgfsetlinewidth{0.803000pt}%
\definecolor{currentstroke}{rgb}{1.000000,1.000000,1.000000}%
\pgfsetstrokecolor{currentstroke}%
\pgfsetdash{}{0pt}%
\pgfpathmoveto{\pgfqpoint{9.810417in}{7.318936in}}%
\pgfpathlineto{\pgfqpoint{15.300000in}{7.318936in}}%
\pgfusepath{stroke}%
\end{pgfscope}%
\begin{pgfscope}%
\definecolor{textcolor}{rgb}{0.150000,0.150000,0.150000}%
\pgfsetstrokecolor{textcolor}%
\pgfsetfillcolor{textcolor}%
\pgftext[x=9.589483in,y=7.245070in,left,base]{\color{textcolor}\rmfamily\fontsize{14.000000}{16.800000}\selectfont 0}%
\end{pgfscope}%
\begin{pgfscope}%
\pgfpathrectangle{\pgfqpoint{9.810417in}{7.240698in}}{\pgfqpoint{5.489583in}{0.877907in}}%
\pgfusepath{clip}%
\pgfsetroundcap%
\pgfsetroundjoin%
\pgfsetlinewidth{0.803000pt}%
\definecolor{currentstroke}{rgb}{1.000000,1.000000,1.000000}%
\pgfsetstrokecolor{currentstroke}%
\pgfsetdash{}{0pt}%
\pgfpathmoveto{\pgfqpoint{9.810417in}{8.078700in}}%
\pgfpathlineto{\pgfqpoint{15.300000in}{8.078700in}}%
\pgfusepath{stroke}%
\end{pgfscope}%
\begin{pgfscope}%
\definecolor{textcolor}{rgb}{0.150000,0.150000,0.150000}%
\pgfsetstrokecolor{textcolor}%
\pgfsetfillcolor{textcolor}%
\pgftext[x=9.589483in,y=8.004834in,left,base]{\color{textcolor}\rmfamily\fontsize{14.000000}{16.800000}\selectfont 1}%
\end{pgfscope}%
\begin{pgfscope}%
\pgfpathrectangle{\pgfqpoint{9.810417in}{7.240698in}}{\pgfqpoint{5.489583in}{0.877907in}}%
\pgfusepath{clip}%
\pgfsetbuttcap%
\pgfsetroundjoin%
\definecolor{currentfill}{rgb}{0.121569,0.466667,0.705882}%
\pgfsetfillcolor{currentfill}%
\pgfsetfillopacity{0.250000}%
\pgfsetlinewidth{1.003750pt}%
\definecolor{currentstroke}{rgb}{1.000000,1.000000,1.000000}%
\pgfsetstrokecolor{currentstroke}%
\pgfsetstrokeopacity{0.250000}%
\pgfsetdash{}{0pt}%
\pgfpathmoveto{\pgfqpoint{10.121555in}{7.357270in}}%
\pgfpathlineto{\pgfqpoint{10.121555in}{7.280603in}}%
\pgfpathlineto{\pgfqpoint{10.306389in}{7.280603in}}%
\pgfpathlineto{\pgfqpoint{10.429612in}{7.280603in}}%
\pgfpathlineto{\pgfqpoint{10.552835in}{7.280603in}}%
\pgfpathlineto{\pgfqpoint{10.676058in}{7.280603in}}%
\pgfpathlineto{\pgfqpoint{10.799281in}{7.280603in}}%
\pgfpathlineto{\pgfqpoint{10.922504in}{7.280603in}}%
\pgfpathlineto{\pgfqpoint{11.045727in}{7.280603in}}%
\pgfpathlineto{\pgfqpoint{11.168950in}{7.280603in}}%
\pgfpathlineto{\pgfqpoint{11.292173in}{7.280603in}}%
\pgfpathlineto{\pgfqpoint{11.415396in}{7.280603in}}%
\pgfpathlineto{\pgfqpoint{11.538619in}{7.280603in}}%
\pgfpathlineto{\pgfqpoint{11.661842in}{7.280603in}}%
\pgfpathlineto{\pgfqpoint{11.785065in}{7.280603in}}%
\pgfpathlineto{\pgfqpoint{11.908288in}{7.280603in}}%
\pgfpathlineto{\pgfqpoint{12.031511in}{7.280603in}}%
\pgfpathlineto{\pgfqpoint{12.154734in}{7.280603in}}%
\pgfpathlineto{\pgfqpoint{12.277957in}{7.280603in}}%
\pgfpathlineto{\pgfqpoint{12.401180in}{7.280603in}}%
\pgfpathlineto{\pgfqpoint{12.524403in}{7.280603in}}%
\pgfpathlineto{\pgfqpoint{12.647626in}{7.280603in}}%
\pgfpathlineto{\pgfqpoint{12.770849in}{7.280603in}}%
\pgfpathlineto{\pgfqpoint{12.894072in}{7.280603in}}%
\pgfpathlineto{\pgfqpoint{13.017294in}{7.280603in}}%
\pgfpathlineto{\pgfqpoint{13.140517in}{7.280603in}}%
\pgfpathlineto{\pgfqpoint{13.263740in}{7.280603in}}%
\pgfpathlineto{\pgfqpoint{13.386963in}{7.280603in}}%
\pgfpathlineto{\pgfqpoint{13.510186in}{7.280603in}}%
\pgfpathlineto{\pgfqpoint{13.633409in}{7.280603in}}%
\pgfpathlineto{\pgfqpoint{13.756632in}{7.280603in}}%
\pgfpathlineto{\pgfqpoint{13.879855in}{7.280603in}}%
\pgfpathlineto{\pgfqpoint{14.003078in}{7.280603in}}%
\pgfpathlineto{\pgfqpoint{14.126301in}{7.280603in}}%
\pgfpathlineto{\pgfqpoint{14.249524in}{7.280603in}}%
\pgfpathlineto{\pgfqpoint{14.372747in}{7.280603in}}%
\pgfpathlineto{\pgfqpoint{14.495970in}{7.280603in}}%
\pgfpathlineto{\pgfqpoint{14.619193in}{7.280603in}}%
\pgfpathlineto{\pgfqpoint{14.742416in}{7.280603in}}%
\pgfpathlineto{\pgfqpoint{14.865639in}{7.280603in}}%
\pgfpathlineto{\pgfqpoint{15.050473in}{7.280603in}}%
\pgfpathlineto{\pgfqpoint{15.050473in}{7.357270in}}%
\pgfpathlineto{\pgfqpoint{15.050473in}{7.357270in}}%
\pgfpathlineto{\pgfqpoint{14.865639in}{7.357270in}}%
\pgfpathlineto{\pgfqpoint{14.742416in}{7.357270in}}%
\pgfpathlineto{\pgfqpoint{14.619193in}{7.357270in}}%
\pgfpathlineto{\pgfqpoint{14.495970in}{7.357270in}}%
\pgfpathlineto{\pgfqpoint{14.372747in}{7.357270in}}%
\pgfpathlineto{\pgfqpoint{14.249524in}{7.357270in}}%
\pgfpathlineto{\pgfqpoint{14.126301in}{7.357270in}}%
\pgfpathlineto{\pgfqpoint{14.003078in}{7.357270in}}%
\pgfpathlineto{\pgfqpoint{13.879855in}{7.357270in}}%
\pgfpathlineto{\pgfqpoint{13.756632in}{7.357270in}}%
\pgfpathlineto{\pgfqpoint{13.633409in}{7.357270in}}%
\pgfpathlineto{\pgfqpoint{13.510186in}{7.357270in}}%
\pgfpathlineto{\pgfqpoint{13.386963in}{7.357270in}}%
\pgfpathlineto{\pgfqpoint{13.263740in}{7.357270in}}%
\pgfpathlineto{\pgfqpoint{13.140517in}{7.357270in}}%
\pgfpathlineto{\pgfqpoint{13.017294in}{7.357270in}}%
\pgfpathlineto{\pgfqpoint{12.894072in}{7.357270in}}%
\pgfpathlineto{\pgfqpoint{12.770849in}{7.357270in}}%
\pgfpathlineto{\pgfqpoint{12.647626in}{7.357270in}}%
\pgfpathlineto{\pgfqpoint{12.524403in}{7.357270in}}%
\pgfpathlineto{\pgfqpoint{12.401180in}{7.357270in}}%
\pgfpathlineto{\pgfqpoint{12.277957in}{7.357270in}}%
\pgfpathlineto{\pgfqpoint{12.154734in}{7.357270in}}%
\pgfpathlineto{\pgfqpoint{12.031511in}{7.357270in}}%
\pgfpathlineto{\pgfqpoint{11.908288in}{7.357270in}}%
\pgfpathlineto{\pgfqpoint{11.785065in}{7.357270in}}%
\pgfpathlineto{\pgfqpoint{11.661842in}{7.357270in}}%
\pgfpathlineto{\pgfqpoint{11.538619in}{7.357270in}}%
\pgfpathlineto{\pgfqpoint{11.415396in}{7.357270in}}%
\pgfpathlineto{\pgfqpoint{11.292173in}{7.357270in}}%
\pgfpathlineto{\pgfqpoint{11.168950in}{7.357270in}}%
\pgfpathlineto{\pgfqpoint{11.045727in}{7.357270in}}%
\pgfpathlineto{\pgfqpoint{10.922504in}{7.357270in}}%
\pgfpathlineto{\pgfqpoint{10.799281in}{7.357270in}}%
\pgfpathlineto{\pgfqpoint{10.676058in}{7.357270in}}%
\pgfpathlineto{\pgfqpoint{10.552835in}{7.357270in}}%
\pgfpathlineto{\pgfqpoint{10.429612in}{7.357270in}}%
\pgfpathlineto{\pgfqpoint{10.306389in}{7.357270in}}%
\pgfpathlineto{\pgfqpoint{10.121555in}{7.357270in}}%
\pgfpathclose%
\pgfusepath{stroke,fill}%
\end{pgfscope}%
\begin{pgfscope}%
\pgfpathrectangle{\pgfqpoint{9.810417in}{7.240698in}}{\pgfqpoint{5.489583in}{0.877907in}}%
\pgfusepath{clip}%
\pgfsetbuttcap%
\pgfsetroundjoin%
\pgfsetlinewidth{1.505625pt}%
\definecolor{currentstroke}{rgb}{0.000000,0.000000,0.000000}%
\pgfsetstrokecolor{currentstroke}%
\pgfsetdash{}{0pt}%
\pgfpathmoveto{\pgfqpoint{10.059943in}{7.318936in}}%
\pgfpathlineto{\pgfqpoint{10.059943in}{8.078700in}}%
\pgfusepath{stroke}%
\end{pgfscope}%
\begin{pgfscope}%
\pgfpathrectangle{\pgfqpoint{9.810417in}{7.240698in}}{\pgfqpoint{5.489583in}{0.877907in}}%
\pgfusepath{clip}%
\pgfsetbuttcap%
\pgfsetroundjoin%
\pgfsetlinewidth{1.505625pt}%
\definecolor{currentstroke}{rgb}{0.000000,0.000000,0.000000}%
\pgfsetstrokecolor{currentstroke}%
\pgfsetdash{}{0pt}%
\pgfpathmoveto{\pgfqpoint{10.183166in}{7.318936in}}%
\pgfpathlineto{\pgfqpoint{10.183166in}{8.075900in}}%
\pgfusepath{stroke}%
\end{pgfscope}%
\begin{pgfscope}%
\pgfpathrectangle{\pgfqpoint{9.810417in}{7.240698in}}{\pgfqpoint{5.489583in}{0.877907in}}%
\pgfusepath{clip}%
\pgfsetbuttcap%
\pgfsetroundjoin%
\pgfsetlinewidth{1.505625pt}%
\definecolor{currentstroke}{rgb}{0.000000,0.000000,0.000000}%
\pgfsetstrokecolor{currentstroke}%
\pgfsetdash{}{0pt}%
\pgfpathmoveto{\pgfqpoint{10.306389in}{7.318936in}}%
\pgfpathlineto{\pgfqpoint{10.306389in}{7.309470in}}%
\pgfusepath{stroke}%
\end{pgfscope}%
\begin{pgfscope}%
\pgfpathrectangle{\pgfqpoint{9.810417in}{7.240698in}}{\pgfqpoint{5.489583in}{0.877907in}}%
\pgfusepath{clip}%
\pgfsetbuttcap%
\pgfsetroundjoin%
\pgfsetlinewidth{1.505625pt}%
\definecolor{currentstroke}{rgb}{0.000000,0.000000,0.000000}%
\pgfsetstrokecolor{currentstroke}%
\pgfsetdash{}{0pt}%
\pgfpathmoveto{\pgfqpoint{10.429612in}{7.318936in}}%
\pgfpathlineto{\pgfqpoint{10.429612in}{7.318066in}}%
\pgfusepath{stroke}%
\end{pgfscope}%
\begin{pgfscope}%
\pgfpathrectangle{\pgfqpoint{9.810417in}{7.240698in}}{\pgfqpoint{5.489583in}{0.877907in}}%
\pgfusepath{clip}%
\pgfsetbuttcap%
\pgfsetroundjoin%
\pgfsetlinewidth{1.505625pt}%
\definecolor{currentstroke}{rgb}{0.000000,0.000000,0.000000}%
\pgfsetstrokecolor{currentstroke}%
\pgfsetdash{}{0pt}%
\pgfpathmoveto{\pgfqpoint{10.552835in}{7.318936in}}%
\pgfpathlineto{\pgfqpoint{10.552835in}{7.305082in}}%
\pgfusepath{stroke}%
\end{pgfscope}%
\begin{pgfscope}%
\pgfpathrectangle{\pgfqpoint{9.810417in}{7.240698in}}{\pgfqpoint{5.489583in}{0.877907in}}%
\pgfusepath{clip}%
\pgfsetbuttcap%
\pgfsetroundjoin%
\pgfsetlinewidth{1.505625pt}%
\definecolor{currentstroke}{rgb}{0.000000,0.000000,0.000000}%
\pgfsetstrokecolor{currentstroke}%
\pgfsetdash{}{0pt}%
\pgfpathmoveto{\pgfqpoint{10.676058in}{7.318936in}}%
\pgfpathlineto{\pgfqpoint{10.676058in}{7.336369in}}%
\pgfusepath{stroke}%
\end{pgfscope}%
\begin{pgfscope}%
\pgfpathrectangle{\pgfqpoint{9.810417in}{7.240698in}}{\pgfqpoint{5.489583in}{0.877907in}}%
\pgfusepath{clip}%
\pgfsetbuttcap%
\pgfsetroundjoin%
\pgfsetlinewidth{1.505625pt}%
\definecolor{currentstroke}{rgb}{0.000000,0.000000,0.000000}%
\pgfsetstrokecolor{currentstroke}%
\pgfsetdash{}{0pt}%
\pgfpathmoveto{\pgfqpoint{10.799281in}{7.318936in}}%
\pgfpathlineto{\pgfqpoint{10.799281in}{7.344185in}}%
\pgfusepath{stroke}%
\end{pgfscope}%
\begin{pgfscope}%
\pgfpathrectangle{\pgfqpoint{9.810417in}{7.240698in}}{\pgfqpoint{5.489583in}{0.877907in}}%
\pgfusepath{clip}%
\pgfsetbuttcap%
\pgfsetroundjoin%
\pgfsetlinewidth{1.505625pt}%
\definecolor{currentstroke}{rgb}{0.000000,0.000000,0.000000}%
\pgfsetstrokecolor{currentstroke}%
\pgfsetdash{}{0pt}%
\pgfpathmoveto{\pgfqpoint{10.922504in}{7.318936in}}%
\pgfpathlineto{\pgfqpoint{10.922504in}{7.330536in}}%
\pgfusepath{stroke}%
\end{pgfscope}%
\begin{pgfscope}%
\pgfpathrectangle{\pgfqpoint{9.810417in}{7.240698in}}{\pgfqpoint{5.489583in}{0.877907in}}%
\pgfusepath{clip}%
\pgfsetbuttcap%
\pgfsetroundjoin%
\pgfsetlinewidth{1.505625pt}%
\definecolor{currentstroke}{rgb}{0.000000,0.000000,0.000000}%
\pgfsetstrokecolor{currentstroke}%
\pgfsetdash{}{0pt}%
\pgfpathmoveto{\pgfqpoint{11.045727in}{7.318936in}}%
\pgfpathlineto{\pgfqpoint{11.045727in}{7.317247in}}%
\pgfusepath{stroke}%
\end{pgfscope}%
\begin{pgfscope}%
\pgfpathrectangle{\pgfqpoint{9.810417in}{7.240698in}}{\pgfqpoint{5.489583in}{0.877907in}}%
\pgfusepath{clip}%
\pgfsetbuttcap%
\pgfsetroundjoin%
\pgfsetlinewidth{1.505625pt}%
\definecolor{currentstroke}{rgb}{0.000000,0.000000,0.000000}%
\pgfsetstrokecolor{currentstroke}%
\pgfsetdash{}{0pt}%
\pgfpathmoveto{\pgfqpoint{11.168950in}{7.318936in}}%
\pgfpathlineto{\pgfqpoint{11.168950in}{7.287925in}}%
\pgfusepath{stroke}%
\end{pgfscope}%
\begin{pgfscope}%
\pgfpathrectangle{\pgfqpoint{9.810417in}{7.240698in}}{\pgfqpoint{5.489583in}{0.877907in}}%
\pgfusepath{clip}%
\pgfsetbuttcap%
\pgfsetroundjoin%
\pgfsetlinewidth{1.505625pt}%
\definecolor{currentstroke}{rgb}{0.000000,0.000000,0.000000}%
\pgfsetstrokecolor{currentstroke}%
\pgfsetdash{}{0pt}%
\pgfpathmoveto{\pgfqpoint{11.292173in}{7.318936in}}%
\pgfpathlineto{\pgfqpoint{11.292173in}{7.319602in}}%
\pgfusepath{stroke}%
\end{pgfscope}%
\begin{pgfscope}%
\pgfpathrectangle{\pgfqpoint{9.810417in}{7.240698in}}{\pgfqpoint{5.489583in}{0.877907in}}%
\pgfusepath{clip}%
\pgfsetbuttcap%
\pgfsetroundjoin%
\pgfsetlinewidth{1.505625pt}%
\definecolor{currentstroke}{rgb}{0.000000,0.000000,0.000000}%
\pgfsetstrokecolor{currentstroke}%
\pgfsetdash{}{0pt}%
\pgfpathmoveto{\pgfqpoint{11.415396in}{7.318936in}}%
\pgfpathlineto{\pgfqpoint{11.415396in}{7.345977in}}%
\pgfusepath{stroke}%
\end{pgfscope}%
\begin{pgfscope}%
\pgfpathrectangle{\pgfqpoint{9.810417in}{7.240698in}}{\pgfqpoint{5.489583in}{0.877907in}}%
\pgfusepath{clip}%
\pgfsetbuttcap%
\pgfsetroundjoin%
\pgfsetlinewidth{1.505625pt}%
\definecolor{currentstroke}{rgb}{0.000000,0.000000,0.000000}%
\pgfsetstrokecolor{currentstroke}%
\pgfsetdash{}{0pt}%
\pgfpathmoveto{\pgfqpoint{11.538619in}{7.318936in}}%
\pgfpathlineto{\pgfqpoint{11.538619in}{7.303012in}}%
\pgfusepath{stroke}%
\end{pgfscope}%
\begin{pgfscope}%
\pgfpathrectangle{\pgfqpoint{9.810417in}{7.240698in}}{\pgfqpoint{5.489583in}{0.877907in}}%
\pgfusepath{clip}%
\pgfsetbuttcap%
\pgfsetroundjoin%
\pgfsetlinewidth{1.505625pt}%
\definecolor{currentstroke}{rgb}{0.000000,0.000000,0.000000}%
\pgfsetstrokecolor{currentstroke}%
\pgfsetdash{}{0pt}%
\pgfpathmoveto{\pgfqpoint{11.661842in}{7.318936in}}%
\pgfpathlineto{\pgfqpoint{11.661842in}{7.307171in}}%
\pgfusepath{stroke}%
\end{pgfscope}%
\begin{pgfscope}%
\pgfpathrectangle{\pgfqpoint{9.810417in}{7.240698in}}{\pgfqpoint{5.489583in}{0.877907in}}%
\pgfusepath{clip}%
\pgfsetbuttcap%
\pgfsetroundjoin%
\pgfsetlinewidth{1.505625pt}%
\definecolor{currentstroke}{rgb}{0.000000,0.000000,0.000000}%
\pgfsetstrokecolor{currentstroke}%
\pgfsetdash{}{0pt}%
\pgfpathmoveto{\pgfqpoint{11.785065in}{7.318936in}}%
\pgfpathlineto{\pgfqpoint{11.785065in}{7.331168in}}%
\pgfusepath{stroke}%
\end{pgfscope}%
\begin{pgfscope}%
\pgfpathrectangle{\pgfqpoint{9.810417in}{7.240698in}}{\pgfqpoint{5.489583in}{0.877907in}}%
\pgfusepath{clip}%
\pgfsetbuttcap%
\pgfsetroundjoin%
\pgfsetlinewidth{1.505625pt}%
\definecolor{currentstroke}{rgb}{0.000000,0.000000,0.000000}%
\pgfsetstrokecolor{currentstroke}%
\pgfsetdash{}{0pt}%
\pgfpathmoveto{\pgfqpoint{11.908288in}{7.318936in}}%
\pgfpathlineto{\pgfqpoint{11.908288in}{7.339705in}}%
\pgfusepath{stroke}%
\end{pgfscope}%
\begin{pgfscope}%
\pgfpathrectangle{\pgfqpoint{9.810417in}{7.240698in}}{\pgfqpoint{5.489583in}{0.877907in}}%
\pgfusepath{clip}%
\pgfsetbuttcap%
\pgfsetroundjoin%
\pgfsetlinewidth{1.505625pt}%
\definecolor{currentstroke}{rgb}{0.000000,0.000000,0.000000}%
\pgfsetstrokecolor{currentstroke}%
\pgfsetdash{}{0pt}%
\pgfpathmoveto{\pgfqpoint{12.031511in}{7.318936in}}%
\pgfpathlineto{\pgfqpoint{12.031511in}{7.319448in}}%
\pgfusepath{stroke}%
\end{pgfscope}%
\begin{pgfscope}%
\pgfpathrectangle{\pgfqpoint{9.810417in}{7.240698in}}{\pgfqpoint{5.489583in}{0.877907in}}%
\pgfusepath{clip}%
\pgfsetbuttcap%
\pgfsetroundjoin%
\pgfsetlinewidth{1.505625pt}%
\definecolor{currentstroke}{rgb}{0.000000,0.000000,0.000000}%
\pgfsetstrokecolor{currentstroke}%
\pgfsetdash{}{0pt}%
\pgfpathmoveto{\pgfqpoint{12.154734in}{7.318936in}}%
\pgfpathlineto{\pgfqpoint{12.154734in}{7.330533in}}%
\pgfusepath{stroke}%
\end{pgfscope}%
\begin{pgfscope}%
\pgfpathrectangle{\pgfqpoint{9.810417in}{7.240698in}}{\pgfqpoint{5.489583in}{0.877907in}}%
\pgfusepath{clip}%
\pgfsetbuttcap%
\pgfsetroundjoin%
\pgfsetlinewidth{1.505625pt}%
\definecolor{currentstroke}{rgb}{0.000000,0.000000,0.000000}%
\pgfsetstrokecolor{currentstroke}%
\pgfsetdash{}{0pt}%
\pgfpathmoveto{\pgfqpoint{12.277957in}{7.318936in}}%
\pgfpathlineto{\pgfqpoint{12.277957in}{7.311897in}}%
\pgfusepath{stroke}%
\end{pgfscope}%
\begin{pgfscope}%
\pgfpathrectangle{\pgfqpoint{9.810417in}{7.240698in}}{\pgfqpoint{5.489583in}{0.877907in}}%
\pgfusepath{clip}%
\pgfsetbuttcap%
\pgfsetroundjoin%
\pgfsetlinewidth{1.505625pt}%
\definecolor{currentstroke}{rgb}{0.000000,0.000000,0.000000}%
\pgfsetstrokecolor{currentstroke}%
\pgfsetdash{}{0pt}%
\pgfpathmoveto{\pgfqpoint{12.401180in}{7.318936in}}%
\pgfpathlineto{\pgfqpoint{12.401180in}{7.310123in}}%
\pgfusepath{stroke}%
\end{pgfscope}%
\begin{pgfscope}%
\pgfpathrectangle{\pgfqpoint{9.810417in}{7.240698in}}{\pgfqpoint{5.489583in}{0.877907in}}%
\pgfusepath{clip}%
\pgfsetbuttcap%
\pgfsetroundjoin%
\pgfsetlinewidth{1.505625pt}%
\definecolor{currentstroke}{rgb}{0.000000,0.000000,0.000000}%
\pgfsetstrokecolor{currentstroke}%
\pgfsetdash{}{0pt}%
\pgfpathmoveto{\pgfqpoint{12.524403in}{7.318936in}}%
\pgfpathlineto{\pgfqpoint{12.524403in}{7.331737in}}%
\pgfusepath{stroke}%
\end{pgfscope}%
\begin{pgfscope}%
\pgfpathrectangle{\pgfqpoint{9.810417in}{7.240698in}}{\pgfqpoint{5.489583in}{0.877907in}}%
\pgfusepath{clip}%
\pgfsetbuttcap%
\pgfsetroundjoin%
\pgfsetlinewidth{1.505625pt}%
\definecolor{currentstroke}{rgb}{0.000000,0.000000,0.000000}%
\pgfsetstrokecolor{currentstroke}%
\pgfsetdash{}{0pt}%
\pgfpathmoveto{\pgfqpoint{12.647626in}{7.318936in}}%
\pgfpathlineto{\pgfqpoint{12.647626in}{7.312940in}}%
\pgfusepath{stroke}%
\end{pgfscope}%
\begin{pgfscope}%
\pgfpathrectangle{\pgfqpoint{9.810417in}{7.240698in}}{\pgfqpoint{5.489583in}{0.877907in}}%
\pgfusepath{clip}%
\pgfsetbuttcap%
\pgfsetroundjoin%
\pgfsetlinewidth{1.505625pt}%
\definecolor{currentstroke}{rgb}{0.000000,0.000000,0.000000}%
\pgfsetstrokecolor{currentstroke}%
\pgfsetdash{}{0pt}%
\pgfpathmoveto{\pgfqpoint{12.770849in}{7.318936in}}%
\pgfpathlineto{\pgfqpoint{12.770849in}{7.321245in}}%
\pgfusepath{stroke}%
\end{pgfscope}%
\begin{pgfscope}%
\pgfpathrectangle{\pgfqpoint{9.810417in}{7.240698in}}{\pgfqpoint{5.489583in}{0.877907in}}%
\pgfusepath{clip}%
\pgfsetbuttcap%
\pgfsetroundjoin%
\pgfsetlinewidth{1.505625pt}%
\definecolor{currentstroke}{rgb}{0.000000,0.000000,0.000000}%
\pgfsetstrokecolor{currentstroke}%
\pgfsetdash{}{0pt}%
\pgfpathmoveto{\pgfqpoint{12.894072in}{7.318936in}}%
\pgfpathlineto{\pgfqpoint{12.894072in}{7.338549in}}%
\pgfusepath{stroke}%
\end{pgfscope}%
\begin{pgfscope}%
\pgfpathrectangle{\pgfqpoint{9.810417in}{7.240698in}}{\pgfqpoint{5.489583in}{0.877907in}}%
\pgfusepath{clip}%
\pgfsetbuttcap%
\pgfsetroundjoin%
\pgfsetlinewidth{1.505625pt}%
\definecolor{currentstroke}{rgb}{0.000000,0.000000,0.000000}%
\pgfsetstrokecolor{currentstroke}%
\pgfsetdash{}{0pt}%
\pgfpathmoveto{\pgfqpoint{13.017294in}{7.318936in}}%
\pgfpathlineto{\pgfqpoint{13.017294in}{7.311332in}}%
\pgfusepath{stroke}%
\end{pgfscope}%
\begin{pgfscope}%
\pgfpathrectangle{\pgfqpoint{9.810417in}{7.240698in}}{\pgfqpoint{5.489583in}{0.877907in}}%
\pgfusepath{clip}%
\pgfsetbuttcap%
\pgfsetroundjoin%
\pgfsetlinewidth{1.505625pt}%
\definecolor{currentstroke}{rgb}{0.000000,0.000000,0.000000}%
\pgfsetstrokecolor{currentstroke}%
\pgfsetdash{}{0pt}%
\pgfpathmoveto{\pgfqpoint{13.140517in}{7.318936in}}%
\pgfpathlineto{\pgfqpoint{13.140517in}{7.315351in}}%
\pgfusepath{stroke}%
\end{pgfscope}%
\begin{pgfscope}%
\pgfpathrectangle{\pgfqpoint{9.810417in}{7.240698in}}{\pgfqpoint{5.489583in}{0.877907in}}%
\pgfusepath{clip}%
\pgfsetbuttcap%
\pgfsetroundjoin%
\pgfsetlinewidth{1.505625pt}%
\definecolor{currentstroke}{rgb}{0.000000,0.000000,0.000000}%
\pgfsetstrokecolor{currentstroke}%
\pgfsetdash{}{0pt}%
\pgfpathmoveto{\pgfqpoint{13.263740in}{7.318936in}}%
\pgfpathlineto{\pgfqpoint{13.263740in}{7.339863in}}%
\pgfusepath{stroke}%
\end{pgfscope}%
\begin{pgfscope}%
\pgfpathrectangle{\pgfqpoint{9.810417in}{7.240698in}}{\pgfqpoint{5.489583in}{0.877907in}}%
\pgfusepath{clip}%
\pgfsetbuttcap%
\pgfsetroundjoin%
\pgfsetlinewidth{1.505625pt}%
\definecolor{currentstroke}{rgb}{0.000000,0.000000,0.000000}%
\pgfsetstrokecolor{currentstroke}%
\pgfsetdash{}{0pt}%
\pgfpathmoveto{\pgfqpoint{13.386963in}{7.318936in}}%
\pgfpathlineto{\pgfqpoint{13.386963in}{7.343497in}}%
\pgfusepath{stroke}%
\end{pgfscope}%
\begin{pgfscope}%
\pgfpathrectangle{\pgfqpoint{9.810417in}{7.240698in}}{\pgfqpoint{5.489583in}{0.877907in}}%
\pgfusepath{clip}%
\pgfsetbuttcap%
\pgfsetroundjoin%
\pgfsetlinewidth{1.505625pt}%
\definecolor{currentstroke}{rgb}{0.000000,0.000000,0.000000}%
\pgfsetstrokecolor{currentstroke}%
\pgfsetdash{}{0pt}%
\pgfpathmoveto{\pgfqpoint{13.510186in}{7.318936in}}%
\pgfpathlineto{\pgfqpoint{13.510186in}{7.320560in}}%
\pgfusepath{stroke}%
\end{pgfscope}%
\begin{pgfscope}%
\pgfpathrectangle{\pgfqpoint{9.810417in}{7.240698in}}{\pgfqpoint{5.489583in}{0.877907in}}%
\pgfusepath{clip}%
\pgfsetbuttcap%
\pgfsetroundjoin%
\pgfsetlinewidth{1.505625pt}%
\definecolor{currentstroke}{rgb}{0.000000,0.000000,0.000000}%
\pgfsetstrokecolor{currentstroke}%
\pgfsetdash{}{0pt}%
\pgfpathmoveto{\pgfqpoint{13.633409in}{7.318936in}}%
\pgfpathlineto{\pgfqpoint{13.633409in}{7.322852in}}%
\pgfusepath{stroke}%
\end{pgfscope}%
\begin{pgfscope}%
\pgfpathrectangle{\pgfqpoint{9.810417in}{7.240698in}}{\pgfqpoint{5.489583in}{0.877907in}}%
\pgfusepath{clip}%
\pgfsetbuttcap%
\pgfsetroundjoin%
\pgfsetlinewidth{1.505625pt}%
\definecolor{currentstroke}{rgb}{0.000000,0.000000,0.000000}%
\pgfsetstrokecolor{currentstroke}%
\pgfsetdash{}{0pt}%
\pgfpathmoveto{\pgfqpoint{13.756632in}{7.318936in}}%
\pgfpathlineto{\pgfqpoint{13.756632in}{7.294712in}}%
\pgfusepath{stroke}%
\end{pgfscope}%
\begin{pgfscope}%
\pgfpathrectangle{\pgfqpoint{9.810417in}{7.240698in}}{\pgfqpoint{5.489583in}{0.877907in}}%
\pgfusepath{clip}%
\pgfsetbuttcap%
\pgfsetroundjoin%
\pgfsetlinewidth{1.505625pt}%
\definecolor{currentstroke}{rgb}{0.000000,0.000000,0.000000}%
\pgfsetstrokecolor{currentstroke}%
\pgfsetdash{}{0pt}%
\pgfpathmoveto{\pgfqpoint{13.879855in}{7.318936in}}%
\pgfpathlineto{\pgfqpoint{13.879855in}{7.311488in}}%
\pgfusepath{stroke}%
\end{pgfscope}%
\begin{pgfscope}%
\pgfpathrectangle{\pgfqpoint{9.810417in}{7.240698in}}{\pgfqpoint{5.489583in}{0.877907in}}%
\pgfusepath{clip}%
\pgfsetbuttcap%
\pgfsetroundjoin%
\pgfsetlinewidth{1.505625pt}%
\definecolor{currentstroke}{rgb}{0.000000,0.000000,0.000000}%
\pgfsetstrokecolor{currentstroke}%
\pgfsetdash{}{0pt}%
\pgfpathmoveto{\pgfqpoint{14.003078in}{7.318936in}}%
\pgfpathlineto{\pgfqpoint{14.003078in}{7.347219in}}%
\pgfusepath{stroke}%
\end{pgfscope}%
\begin{pgfscope}%
\pgfpathrectangle{\pgfqpoint{9.810417in}{7.240698in}}{\pgfqpoint{5.489583in}{0.877907in}}%
\pgfusepath{clip}%
\pgfsetbuttcap%
\pgfsetroundjoin%
\pgfsetlinewidth{1.505625pt}%
\definecolor{currentstroke}{rgb}{0.000000,0.000000,0.000000}%
\pgfsetstrokecolor{currentstroke}%
\pgfsetdash{}{0pt}%
\pgfpathmoveto{\pgfqpoint{14.126301in}{7.318936in}}%
\pgfpathlineto{\pgfqpoint{14.126301in}{7.343185in}}%
\pgfusepath{stroke}%
\end{pgfscope}%
\begin{pgfscope}%
\pgfpathrectangle{\pgfqpoint{9.810417in}{7.240698in}}{\pgfqpoint{5.489583in}{0.877907in}}%
\pgfusepath{clip}%
\pgfsetbuttcap%
\pgfsetroundjoin%
\pgfsetlinewidth{1.505625pt}%
\definecolor{currentstroke}{rgb}{0.000000,0.000000,0.000000}%
\pgfsetstrokecolor{currentstroke}%
\pgfsetdash{}{0pt}%
\pgfpathmoveto{\pgfqpoint{14.249524in}{7.318936in}}%
\pgfpathlineto{\pgfqpoint{14.249524in}{7.320056in}}%
\pgfusepath{stroke}%
\end{pgfscope}%
\begin{pgfscope}%
\pgfpathrectangle{\pgfqpoint{9.810417in}{7.240698in}}{\pgfqpoint{5.489583in}{0.877907in}}%
\pgfusepath{clip}%
\pgfsetbuttcap%
\pgfsetroundjoin%
\pgfsetlinewidth{1.505625pt}%
\definecolor{currentstroke}{rgb}{0.000000,0.000000,0.000000}%
\pgfsetstrokecolor{currentstroke}%
\pgfsetdash{}{0pt}%
\pgfpathmoveto{\pgfqpoint{14.372747in}{7.318936in}}%
\pgfpathlineto{\pgfqpoint{14.372747in}{7.301917in}}%
\pgfusepath{stroke}%
\end{pgfscope}%
\begin{pgfscope}%
\pgfpathrectangle{\pgfqpoint{9.810417in}{7.240698in}}{\pgfqpoint{5.489583in}{0.877907in}}%
\pgfusepath{clip}%
\pgfsetbuttcap%
\pgfsetroundjoin%
\pgfsetlinewidth{1.505625pt}%
\definecolor{currentstroke}{rgb}{0.000000,0.000000,0.000000}%
\pgfsetstrokecolor{currentstroke}%
\pgfsetdash{}{0pt}%
\pgfpathmoveto{\pgfqpoint{14.495970in}{7.318936in}}%
\pgfpathlineto{\pgfqpoint{14.495970in}{7.284857in}}%
\pgfusepath{stroke}%
\end{pgfscope}%
\begin{pgfscope}%
\pgfpathrectangle{\pgfqpoint{9.810417in}{7.240698in}}{\pgfqpoint{5.489583in}{0.877907in}}%
\pgfusepath{clip}%
\pgfsetbuttcap%
\pgfsetroundjoin%
\pgfsetlinewidth{1.505625pt}%
\definecolor{currentstroke}{rgb}{0.000000,0.000000,0.000000}%
\pgfsetstrokecolor{currentstroke}%
\pgfsetdash{}{0pt}%
\pgfpathmoveto{\pgfqpoint{14.619193in}{7.318936in}}%
\pgfpathlineto{\pgfqpoint{14.619193in}{7.313659in}}%
\pgfusepath{stroke}%
\end{pgfscope}%
\begin{pgfscope}%
\pgfpathrectangle{\pgfqpoint{9.810417in}{7.240698in}}{\pgfqpoint{5.489583in}{0.877907in}}%
\pgfusepath{clip}%
\pgfsetbuttcap%
\pgfsetroundjoin%
\pgfsetlinewidth{1.505625pt}%
\definecolor{currentstroke}{rgb}{0.000000,0.000000,0.000000}%
\pgfsetstrokecolor{currentstroke}%
\pgfsetdash{}{0pt}%
\pgfpathmoveto{\pgfqpoint{14.742416in}{7.318936in}}%
\pgfpathlineto{\pgfqpoint{14.742416in}{7.314069in}}%
\pgfusepath{stroke}%
\end{pgfscope}%
\begin{pgfscope}%
\pgfpathrectangle{\pgfqpoint{9.810417in}{7.240698in}}{\pgfqpoint{5.489583in}{0.877907in}}%
\pgfusepath{clip}%
\pgfsetbuttcap%
\pgfsetroundjoin%
\pgfsetlinewidth{1.505625pt}%
\definecolor{currentstroke}{rgb}{0.000000,0.000000,0.000000}%
\pgfsetstrokecolor{currentstroke}%
\pgfsetdash{}{0pt}%
\pgfpathmoveto{\pgfqpoint{14.865639in}{7.318936in}}%
\pgfpathlineto{\pgfqpoint{14.865639in}{7.325345in}}%
\pgfusepath{stroke}%
\end{pgfscope}%
\begin{pgfscope}%
\pgfpathrectangle{\pgfqpoint{9.810417in}{7.240698in}}{\pgfqpoint{5.489583in}{0.877907in}}%
\pgfusepath{clip}%
\pgfsetbuttcap%
\pgfsetroundjoin%
\pgfsetlinewidth{1.505625pt}%
\definecolor{currentstroke}{rgb}{0.000000,0.000000,0.000000}%
\pgfsetstrokecolor{currentstroke}%
\pgfsetdash{}{0pt}%
\pgfpathmoveto{\pgfqpoint{14.988862in}{7.318936in}}%
\pgfpathlineto{\pgfqpoint{14.988862in}{7.318718in}}%
\pgfusepath{stroke}%
\end{pgfscope}%
\begin{pgfscope}%
\pgfpathrectangle{\pgfqpoint{9.810417in}{7.240698in}}{\pgfqpoint{5.489583in}{0.877907in}}%
\pgfusepath{clip}%
\pgfsetroundcap%
\pgfsetroundjoin%
\pgfsetlinewidth{1.505625pt}%
\definecolor{currentstroke}{rgb}{0.121569,0.466667,0.705882}%
\pgfsetstrokecolor{currentstroke}%
\pgfsetdash{}{0pt}%
\pgfpathmoveto{\pgfqpoint{9.810417in}{7.318936in}}%
\pgfpathlineto{\pgfqpoint{15.300000in}{7.318936in}}%
\pgfusepath{stroke}%
\end{pgfscope}%
\begin{pgfscope}%
\pgfpathrectangle{\pgfqpoint{9.810417in}{7.240698in}}{\pgfqpoint{5.489583in}{0.877907in}}%
\pgfusepath{clip}%
\pgfsetbuttcap%
\pgfsetroundjoin%
\definecolor{currentfill}{rgb}{0.121569,0.466667,0.705882}%
\pgfsetfillcolor{currentfill}%
\pgfsetlinewidth{1.003750pt}%
\definecolor{currentstroke}{rgb}{0.121569,0.466667,0.705882}%
\pgfsetstrokecolor{currentstroke}%
\pgfsetdash{}{0pt}%
\pgfsys@defobject{currentmarker}{\pgfqpoint{-0.034722in}{-0.034722in}}{\pgfqpoint{0.034722in}{0.034722in}}{%
\pgfpathmoveto{\pgfqpoint{0.000000in}{-0.034722in}}%
\pgfpathcurveto{\pgfqpoint{0.009208in}{-0.034722in}}{\pgfqpoint{0.018041in}{-0.031064in}}{\pgfqpoint{0.024552in}{-0.024552in}}%
\pgfpathcurveto{\pgfqpoint{0.031064in}{-0.018041in}}{\pgfqpoint{0.034722in}{-0.009208in}}{\pgfqpoint{0.034722in}{0.000000in}}%
\pgfpathcurveto{\pgfqpoint{0.034722in}{0.009208in}}{\pgfqpoint{0.031064in}{0.018041in}}{\pgfqpoint{0.024552in}{0.024552in}}%
\pgfpathcurveto{\pgfqpoint{0.018041in}{0.031064in}}{\pgfqpoint{0.009208in}{0.034722in}}{\pgfqpoint{0.000000in}{0.034722in}}%
\pgfpathcurveto{\pgfqpoint{-0.009208in}{0.034722in}}{\pgfqpoint{-0.018041in}{0.031064in}}{\pgfqpoint{-0.024552in}{0.024552in}}%
\pgfpathcurveto{\pgfqpoint{-0.031064in}{0.018041in}}{\pgfqpoint{-0.034722in}{0.009208in}}{\pgfqpoint{-0.034722in}{0.000000in}}%
\pgfpathcurveto{\pgfqpoint{-0.034722in}{-0.009208in}}{\pgfqpoint{-0.031064in}{-0.018041in}}{\pgfqpoint{-0.024552in}{-0.024552in}}%
\pgfpathcurveto{\pgfqpoint{-0.018041in}{-0.031064in}}{\pgfqpoint{-0.009208in}{-0.034722in}}{\pgfqpoint{0.000000in}{-0.034722in}}%
\pgfpathclose%
\pgfusepath{stroke,fill}%
}%
\begin{pgfscope}%
\pgfsys@transformshift{10.059943in}{8.078700in}%
\pgfsys@useobject{currentmarker}{}%
\end{pgfscope}%
\begin{pgfscope}%
\pgfsys@transformshift{10.183166in}{8.075900in}%
\pgfsys@useobject{currentmarker}{}%
\end{pgfscope}%
\begin{pgfscope}%
\pgfsys@transformshift{10.306389in}{7.309470in}%
\pgfsys@useobject{currentmarker}{}%
\end{pgfscope}%
\begin{pgfscope}%
\pgfsys@transformshift{10.429612in}{7.318066in}%
\pgfsys@useobject{currentmarker}{}%
\end{pgfscope}%
\begin{pgfscope}%
\pgfsys@transformshift{10.552835in}{7.305082in}%
\pgfsys@useobject{currentmarker}{}%
\end{pgfscope}%
\begin{pgfscope}%
\pgfsys@transformshift{10.676058in}{7.336369in}%
\pgfsys@useobject{currentmarker}{}%
\end{pgfscope}%
\begin{pgfscope}%
\pgfsys@transformshift{10.799281in}{7.344185in}%
\pgfsys@useobject{currentmarker}{}%
\end{pgfscope}%
\begin{pgfscope}%
\pgfsys@transformshift{10.922504in}{7.330536in}%
\pgfsys@useobject{currentmarker}{}%
\end{pgfscope}%
\begin{pgfscope}%
\pgfsys@transformshift{11.045727in}{7.317247in}%
\pgfsys@useobject{currentmarker}{}%
\end{pgfscope}%
\begin{pgfscope}%
\pgfsys@transformshift{11.168950in}{7.287925in}%
\pgfsys@useobject{currentmarker}{}%
\end{pgfscope}%
\begin{pgfscope}%
\pgfsys@transformshift{11.292173in}{7.319602in}%
\pgfsys@useobject{currentmarker}{}%
\end{pgfscope}%
\begin{pgfscope}%
\pgfsys@transformshift{11.415396in}{7.345977in}%
\pgfsys@useobject{currentmarker}{}%
\end{pgfscope}%
\begin{pgfscope}%
\pgfsys@transformshift{11.538619in}{7.303012in}%
\pgfsys@useobject{currentmarker}{}%
\end{pgfscope}%
\begin{pgfscope}%
\pgfsys@transformshift{11.661842in}{7.307171in}%
\pgfsys@useobject{currentmarker}{}%
\end{pgfscope}%
\begin{pgfscope}%
\pgfsys@transformshift{11.785065in}{7.331168in}%
\pgfsys@useobject{currentmarker}{}%
\end{pgfscope}%
\begin{pgfscope}%
\pgfsys@transformshift{11.908288in}{7.339705in}%
\pgfsys@useobject{currentmarker}{}%
\end{pgfscope}%
\begin{pgfscope}%
\pgfsys@transformshift{12.031511in}{7.319448in}%
\pgfsys@useobject{currentmarker}{}%
\end{pgfscope}%
\begin{pgfscope}%
\pgfsys@transformshift{12.154734in}{7.330533in}%
\pgfsys@useobject{currentmarker}{}%
\end{pgfscope}%
\begin{pgfscope}%
\pgfsys@transformshift{12.277957in}{7.311897in}%
\pgfsys@useobject{currentmarker}{}%
\end{pgfscope}%
\begin{pgfscope}%
\pgfsys@transformshift{12.401180in}{7.310123in}%
\pgfsys@useobject{currentmarker}{}%
\end{pgfscope}%
\begin{pgfscope}%
\pgfsys@transformshift{12.524403in}{7.331737in}%
\pgfsys@useobject{currentmarker}{}%
\end{pgfscope}%
\begin{pgfscope}%
\pgfsys@transformshift{12.647626in}{7.312940in}%
\pgfsys@useobject{currentmarker}{}%
\end{pgfscope}%
\begin{pgfscope}%
\pgfsys@transformshift{12.770849in}{7.321245in}%
\pgfsys@useobject{currentmarker}{}%
\end{pgfscope}%
\begin{pgfscope}%
\pgfsys@transformshift{12.894072in}{7.338549in}%
\pgfsys@useobject{currentmarker}{}%
\end{pgfscope}%
\begin{pgfscope}%
\pgfsys@transformshift{13.017294in}{7.311332in}%
\pgfsys@useobject{currentmarker}{}%
\end{pgfscope}%
\begin{pgfscope}%
\pgfsys@transformshift{13.140517in}{7.315351in}%
\pgfsys@useobject{currentmarker}{}%
\end{pgfscope}%
\begin{pgfscope}%
\pgfsys@transformshift{13.263740in}{7.339863in}%
\pgfsys@useobject{currentmarker}{}%
\end{pgfscope}%
\begin{pgfscope}%
\pgfsys@transformshift{13.386963in}{7.343497in}%
\pgfsys@useobject{currentmarker}{}%
\end{pgfscope}%
\begin{pgfscope}%
\pgfsys@transformshift{13.510186in}{7.320560in}%
\pgfsys@useobject{currentmarker}{}%
\end{pgfscope}%
\begin{pgfscope}%
\pgfsys@transformshift{13.633409in}{7.322852in}%
\pgfsys@useobject{currentmarker}{}%
\end{pgfscope}%
\begin{pgfscope}%
\pgfsys@transformshift{13.756632in}{7.294712in}%
\pgfsys@useobject{currentmarker}{}%
\end{pgfscope}%
\begin{pgfscope}%
\pgfsys@transformshift{13.879855in}{7.311488in}%
\pgfsys@useobject{currentmarker}{}%
\end{pgfscope}%
\begin{pgfscope}%
\pgfsys@transformshift{14.003078in}{7.347219in}%
\pgfsys@useobject{currentmarker}{}%
\end{pgfscope}%
\begin{pgfscope}%
\pgfsys@transformshift{14.126301in}{7.343185in}%
\pgfsys@useobject{currentmarker}{}%
\end{pgfscope}%
\begin{pgfscope}%
\pgfsys@transformshift{14.249524in}{7.320056in}%
\pgfsys@useobject{currentmarker}{}%
\end{pgfscope}%
\begin{pgfscope}%
\pgfsys@transformshift{14.372747in}{7.301917in}%
\pgfsys@useobject{currentmarker}{}%
\end{pgfscope}%
\begin{pgfscope}%
\pgfsys@transformshift{14.495970in}{7.284857in}%
\pgfsys@useobject{currentmarker}{}%
\end{pgfscope}%
\begin{pgfscope}%
\pgfsys@transformshift{14.619193in}{7.313659in}%
\pgfsys@useobject{currentmarker}{}%
\end{pgfscope}%
\begin{pgfscope}%
\pgfsys@transformshift{14.742416in}{7.314069in}%
\pgfsys@useobject{currentmarker}{}%
\end{pgfscope}%
\begin{pgfscope}%
\pgfsys@transformshift{14.865639in}{7.325345in}%
\pgfsys@useobject{currentmarker}{}%
\end{pgfscope}%
\begin{pgfscope}%
\pgfsys@transformshift{14.988862in}{7.318718in}%
\pgfsys@useobject{currentmarker}{}%
\end{pgfscope}%
\end{pgfscope}%
\begin{pgfscope}%
\pgfsetrectcap%
\pgfsetmiterjoin%
\pgfsetlinewidth{0.803000pt}%
\definecolor{currentstroke}{rgb}{1.000000,1.000000,1.000000}%
\pgfsetstrokecolor{currentstroke}%
\pgfsetdash{}{0pt}%
\pgfpathmoveto{\pgfqpoint{9.810417in}{7.240698in}}%
\pgfpathlineto{\pgfqpoint{9.810417in}{8.118605in}}%
\pgfusepath{stroke}%
\end{pgfscope}%
\begin{pgfscope}%
\pgfsetrectcap%
\pgfsetmiterjoin%
\pgfsetlinewidth{0.803000pt}%
\definecolor{currentstroke}{rgb}{1.000000,1.000000,1.000000}%
\pgfsetstrokecolor{currentstroke}%
\pgfsetdash{}{0pt}%
\pgfpathmoveto{\pgfqpoint{15.300000in}{7.240698in}}%
\pgfpathlineto{\pgfqpoint{15.300000in}{8.118605in}}%
\pgfusepath{stroke}%
\end{pgfscope}%
\begin{pgfscope}%
\pgfsetrectcap%
\pgfsetmiterjoin%
\pgfsetlinewidth{0.803000pt}%
\definecolor{currentstroke}{rgb}{1.000000,1.000000,1.000000}%
\pgfsetstrokecolor{currentstroke}%
\pgfsetdash{}{0pt}%
\pgfpathmoveto{\pgfqpoint{9.810417in}{7.240698in}}%
\pgfpathlineto{\pgfqpoint{15.300000in}{7.240698in}}%
\pgfusepath{stroke}%
\end{pgfscope}%
\begin{pgfscope}%
\pgfsetrectcap%
\pgfsetmiterjoin%
\pgfsetlinewidth{0.803000pt}%
\definecolor{currentstroke}{rgb}{1.000000,1.000000,1.000000}%
\pgfsetstrokecolor{currentstroke}%
\pgfsetdash{}{0pt}%
\pgfpathmoveto{\pgfqpoint{9.810417in}{8.118605in}}%
\pgfpathlineto{\pgfqpoint{15.300000in}{8.118605in}}%
\pgfusepath{stroke}%
\end{pgfscope}%
\begin{pgfscope}%
\definecolor{textcolor}{rgb}{0.150000,0.150000,0.150000}%
\pgfsetstrokecolor{textcolor}%
\pgfsetfillcolor{textcolor}%
\pgftext[x=12.555208in,y=8.201938in,,base]{\color{textcolor}\rmfamily\fontsize{16.800000}{20.160000}\selectfont Partial Autocorrelation}%
\end{pgfscope}%
\begin{pgfscope}%
\pgfsetbuttcap%
\pgfsetmiterjoin%
\definecolor{currentfill}{rgb}{0.917647,0.917647,0.949020}%
\pgfsetfillcolor{currentfill}%
\pgfsetlinewidth{0.000000pt}%
\definecolor{currentstroke}{rgb}{0.000000,0.000000,0.000000}%
\pgfsetstrokecolor{currentstroke}%
\pgfsetstrokeopacity{0.000000}%
\pgfsetdash{}{0pt}%
\pgfpathmoveto{\pgfqpoint{2.125000in}{5.660465in}}%
\pgfpathlineto{\pgfqpoint{7.614583in}{5.660465in}}%
\pgfpathlineto{\pgfqpoint{7.614583in}{6.538372in}}%
\pgfpathlineto{\pgfqpoint{2.125000in}{6.538372in}}%
\pgfpathclose%
\pgfusepath{fill}%
\end{pgfscope}%
\begin{pgfscope}%
\pgfpathrectangle{\pgfqpoint{2.125000in}{5.660465in}}{\pgfqpoint{5.489583in}{0.877907in}}%
\pgfusepath{clip}%
\pgfsetroundcap%
\pgfsetroundjoin%
\pgfsetlinewidth{0.803000pt}%
\definecolor{currentstroke}{rgb}{1.000000,1.000000,1.000000}%
\pgfsetstrokecolor{currentstroke}%
\pgfsetdash{}{0pt}%
\pgfpathmoveto{\pgfqpoint{2.374527in}{5.660465in}}%
\pgfpathlineto{\pgfqpoint{2.374527in}{6.538372in}}%
\pgfusepath{stroke}%
\end{pgfscope}%
\begin{pgfscope}%
\definecolor{textcolor}{rgb}{0.150000,0.150000,0.150000}%
\pgfsetstrokecolor{textcolor}%
\pgfsetfillcolor{textcolor}%
\pgftext[x=2.374527in,y=5.563243in,,top]{\color{textcolor}\rmfamily\fontsize{14.000000}{16.800000}\selectfont 0}%
\end{pgfscope}%
\begin{pgfscope}%
\pgfpathrectangle{\pgfqpoint{2.125000in}{5.660465in}}{\pgfqpoint{5.489583in}{0.877907in}}%
\pgfusepath{clip}%
\pgfsetroundcap%
\pgfsetroundjoin%
\pgfsetlinewidth{0.803000pt}%
\definecolor{currentstroke}{rgb}{1.000000,1.000000,1.000000}%
\pgfsetstrokecolor{currentstroke}%
\pgfsetdash{}{0pt}%
\pgfpathmoveto{\pgfqpoint{2.990641in}{5.660465in}}%
\pgfpathlineto{\pgfqpoint{2.990641in}{6.538372in}}%
\pgfusepath{stroke}%
\end{pgfscope}%
\begin{pgfscope}%
\definecolor{textcolor}{rgb}{0.150000,0.150000,0.150000}%
\pgfsetstrokecolor{textcolor}%
\pgfsetfillcolor{textcolor}%
\pgftext[x=2.990641in,y=5.563243in,,top]{\color{textcolor}\rmfamily\fontsize{14.000000}{16.800000}\selectfont 5}%
\end{pgfscope}%
\begin{pgfscope}%
\pgfpathrectangle{\pgfqpoint{2.125000in}{5.660465in}}{\pgfqpoint{5.489583in}{0.877907in}}%
\pgfusepath{clip}%
\pgfsetroundcap%
\pgfsetroundjoin%
\pgfsetlinewidth{0.803000pt}%
\definecolor{currentstroke}{rgb}{1.000000,1.000000,1.000000}%
\pgfsetstrokecolor{currentstroke}%
\pgfsetdash{}{0pt}%
\pgfpathmoveto{\pgfqpoint{3.606756in}{5.660465in}}%
\pgfpathlineto{\pgfqpoint{3.606756in}{6.538372in}}%
\pgfusepath{stroke}%
\end{pgfscope}%
\begin{pgfscope}%
\definecolor{textcolor}{rgb}{0.150000,0.150000,0.150000}%
\pgfsetstrokecolor{textcolor}%
\pgfsetfillcolor{textcolor}%
\pgftext[x=3.606756in,y=5.563243in,,top]{\color{textcolor}\rmfamily\fontsize{14.000000}{16.800000}\selectfont 10}%
\end{pgfscope}%
\begin{pgfscope}%
\pgfpathrectangle{\pgfqpoint{2.125000in}{5.660465in}}{\pgfqpoint{5.489583in}{0.877907in}}%
\pgfusepath{clip}%
\pgfsetroundcap%
\pgfsetroundjoin%
\pgfsetlinewidth{0.803000pt}%
\definecolor{currentstroke}{rgb}{1.000000,1.000000,1.000000}%
\pgfsetstrokecolor{currentstroke}%
\pgfsetdash{}{0pt}%
\pgfpathmoveto{\pgfqpoint{4.222871in}{5.660465in}}%
\pgfpathlineto{\pgfqpoint{4.222871in}{6.538372in}}%
\pgfusepath{stroke}%
\end{pgfscope}%
\begin{pgfscope}%
\definecolor{textcolor}{rgb}{0.150000,0.150000,0.150000}%
\pgfsetstrokecolor{textcolor}%
\pgfsetfillcolor{textcolor}%
\pgftext[x=4.222871in,y=5.563243in,,top]{\color{textcolor}\rmfamily\fontsize{14.000000}{16.800000}\selectfont 15}%
\end{pgfscope}%
\begin{pgfscope}%
\pgfpathrectangle{\pgfqpoint{2.125000in}{5.660465in}}{\pgfqpoint{5.489583in}{0.877907in}}%
\pgfusepath{clip}%
\pgfsetroundcap%
\pgfsetroundjoin%
\pgfsetlinewidth{0.803000pt}%
\definecolor{currentstroke}{rgb}{1.000000,1.000000,1.000000}%
\pgfsetstrokecolor{currentstroke}%
\pgfsetdash{}{0pt}%
\pgfpathmoveto{\pgfqpoint{4.838986in}{5.660465in}}%
\pgfpathlineto{\pgfqpoint{4.838986in}{6.538372in}}%
\pgfusepath{stroke}%
\end{pgfscope}%
\begin{pgfscope}%
\definecolor{textcolor}{rgb}{0.150000,0.150000,0.150000}%
\pgfsetstrokecolor{textcolor}%
\pgfsetfillcolor{textcolor}%
\pgftext[x=4.838986in,y=5.563243in,,top]{\color{textcolor}\rmfamily\fontsize{14.000000}{16.800000}\selectfont 20}%
\end{pgfscope}%
\begin{pgfscope}%
\pgfpathrectangle{\pgfqpoint{2.125000in}{5.660465in}}{\pgfqpoint{5.489583in}{0.877907in}}%
\pgfusepath{clip}%
\pgfsetroundcap%
\pgfsetroundjoin%
\pgfsetlinewidth{0.803000pt}%
\definecolor{currentstroke}{rgb}{1.000000,1.000000,1.000000}%
\pgfsetstrokecolor{currentstroke}%
\pgfsetdash{}{0pt}%
\pgfpathmoveto{\pgfqpoint{5.455101in}{5.660465in}}%
\pgfpathlineto{\pgfqpoint{5.455101in}{6.538372in}}%
\pgfusepath{stroke}%
\end{pgfscope}%
\begin{pgfscope}%
\definecolor{textcolor}{rgb}{0.150000,0.150000,0.150000}%
\pgfsetstrokecolor{textcolor}%
\pgfsetfillcolor{textcolor}%
\pgftext[x=5.455101in,y=5.563243in,,top]{\color{textcolor}\rmfamily\fontsize{14.000000}{16.800000}\selectfont 25}%
\end{pgfscope}%
\begin{pgfscope}%
\pgfpathrectangle{\pgfqpoint{2.125000in}{5.660465in}}{\pgfqpoint{5.489583in}{0.877907in}}%
\pgfusepath{clip}%
\pgfsetroundcap%
\pgfsetroundjoin%
\pgfsetlinewidth{0.803000pt}%
\definecolor{currentstroke}{rgb}{1.000000,1.000000,1.000000}%
\pgfsetstrokecolor{currentstroke}%
\pgfsetdash{}{0pt}%
\pgfpathmoveto{\pgfqpoint{6.071216in}{5.660465in}}%
\pgfpathlineto{\pgfqpoint{6.071216in}{6.538372in}}%
\pgfusepath{stroke}%
\end{pgfscope}%
\begin{pgfscope}%
\definecolor{textcolor}{rgb}{0.150000,0.150000,0.150000}%
\pgfsetstrokecolor{textcolor}%
\pgfsetfillcolor{textcolor}%
\pgftext[x=6.071216in,y=5.563243in,,top]{\color{textcolor}\rmfamily\fontsize{14.000000}{16.800000}\selectfont 30}%
\end{pgfscope}%
\begin{pgfscope}%
\pgfpathrectangle{\pgfqpoint{2.125000in}{5.660465in}}{\pgfqpoint{5.489583in}{0.877907in}}%
\pgfusepath{clip}%
\pgfsetroundcap%
\pgfsetroundjoin%
\pgfsetlinewidth{0.803000pt}%
\definecolor{currentstroke}{rgb}{1.000000,1.000000,1.000000}%
\pgfsetstrokecolor{currentstroke}%
\pgfsetdash{}{0pt}%
\pgfpathmoveto{\pgfqpoint{6.687330in}{5.660465in}}%
\pgfpathlineto{\pgfqpoint{6.687330in}{6.538372in}}%
\pgfusepath{stroke}%
\end{pgfscope}%
\begin{pgfscope}%
\definecolor{textcolor}{rgb}{0.150000,0.150000,0.150000}%
\pgfsetstrokecolor{textcolor}%
\pgfsetfillcolor{textcolor}%
\pgftext[x=6.687330in,y=5.563243in,,top]{\color{textcolor}\rmfamily\fontsize{14.000000}{16.800000}\selectfont 35}%
\end{pgfscope}%
\begin{pgfscope}%
\pgfpathrectangle{\pgfqpoint{2.125000in}{5.660465in}}{\pgfqpoint{5.489583in}{0.877907in}}%
\pgfusepath{clip}%
\pgfsetroundcap%
\pgfsetroundjoin%
\pgfsetlinewidth{0.803000pt}%
\definecolor{currentstroke}{rgb}{1.000000,1.000000,1.000000}%
\pgfsetstrokecolor{currentstroke}%
\pgfsetdash{}{0pt}%
\pgfpathmoveto{\pgfqpoint{7.303445in}{5.660465in}}%
\pgfpathlineto{\pgfqpoint{7.303445in}{6.538372in}}%
\pgfusepath{stroke}%
\end{pgfscope}%
\begin{pgfscope}%
\definecolor{textcolor}{rgb}{0.150000,0.150000,0.150000}%
\pgfsetstrokecolor{textcolor}%
\pgfsetfillcolor{textcolor}%
\pgftext[x=7.303445in,y=5.563243in,,top]{\color{textcolor}\rmfamily\fontsize{14.000000}{16.800000}\selectfont 40}%
\end{pgfscope}%
\begin{pgfscope}%
\pgfpathrectangle{\pgfqpoint{2.125000in}{5.660465in}}{\pgfqpoint{5.489583in}{0.877907in}}%
\pgfusepath{clip}%
\pgfsetroundcap%
\pgfsetroundjoin%
\pgfsetlinewidth{0.803000pt}%
\definecolor{currentstroke}{rgb}{1.000000,1.000000,1.000000}%
\pgfsetstrokecolor{currentstroke}%
\pgfsetdash{}{0pt}%
\pgfpathmoveto{\pgfqpoint{2.125000in}{5.928493in}}%
\pgfpathlineto{\pgfqpoint{7.614583in}{5.928493in}}%
\pgfusepath{stroke}%
\end{pgfscope}%
\begin{pgfscope}%
\definecolor{textcolor}{rgb}{0.150000,0.150000,0.150000}%
\pgfsetstrokecolor{textcolor}%
\pgfsetfillcolor{textcolor}%
\pgftext[x=1.904066in,y=5.854627in,left,base]{\color{textcolor}\rmfamily\fontsize{14.000000}{16.800000}\selectfont 0}%
\end{pgfscope}%
\begin{pgfscope}%
\pgfpathrectangle{\pgfqpoint{2.125000in}{5.660465in}}{\pgfqpoint{5.489583in}{0.877907in}}%
\pgfusepath{clip}%
\pgfsetroundcap%
\pgfsetroundjoin%
\pgfsetlinewidth{0.803000pt}%
\definecolor{currentstroke}{rgb}{1.000000,1.000000,1.000000}%
\pgfsetstrokecolor{currentstroke}%
\pgfsetdash{}{0pt}%
\pgfpathmoveto{\pgfqpoint{2.125000in}{6.498467in}}%
\pgfpathlineto{\pgfqpoint{7.614583in}{6.498467in}}%
\pgfusepath{stroke}%
\end{pgfscope}%
\begin{pgfscope}%
\definecolor{textcolor}{rgb}{0.150000,0.150000,0.150000}%
\pgfsetstrokecolor{textcolor}%
\pgfsetfillcolor{textcolor}%
\pgftext[x=1.904066in,y=6.424601in,left,base]{\color{textcolor}\rmfamily\fontsize{14.000000}{16.800000}\selectfont 1}%
\end{pgfscope}%
\begin{pgfscope}%
\pgfpathrectangle{\pgfqpoint{2.125000in}{5.660465in}}{\pgfqpoint{5.489583in}{0.877907in}}%
\pgfusepath{clip}%
\pgfsetbuttcap%
\pgfsetroundjoin%
\definecolor{currentfill}{rgb}{0.121569,0.466667,0.705882}%
\pgfsetfillcolor{currentfill}%
\pgfsetfillopacity{0.250000}%
\pgfsetlinewidth{1.003750pt}%
\definecolor{currentstroke}{rgb}{1.000000,1.000000,1.000000}%
\pgfsetstrokecolor{currentstroke}%
\pgfsetstrokeopacity{0.250000}%
\pgfsetdash{}{0pt}%
\pgfpathmoveto{\pgfqpoint{2.436138in}{5.957251in}}%
\pgfpathlineto{\pgfqpoint{2.436138in}{5.899735in}}%
\pgfpathlineto{\pgfqpoint{2.620972in}{5.878864in}}%
\pgfpathlineto{\pgfqpoint{2.744195in}{5.864620in}}%
\pgfpathlineto{\pgfqpoint{2.867418in}{5.853147in}}%
\pgfpathlineto{\pgfqpoint{2.990641in}{5.843319in}}%
\pgfpathlineto{\pgfqpoint{3.113864in}{5.834615in}}%
\pgfpathlineto{\pgfqpoint{3.237087in}{5.826747in}}%
\pgfpathlineto{\pgfqpoint{3.360310in}{5.819535in}}%
\pgfpathlineto{\pgfqpoint{3.483533in}{5.812856in}}%
\pgfpathlineto{\pgfqpoint{3.606756in}{5.806622in}}%
\pgfpathlineto{\pgfqpoint{3.729979in}{5.800761in}}%
\pgfpathlineto{\pgfqpoint{3.853202in}{5.795223in}}%
\pgfpathlineto{\pgfqpoint{3.976425in}{5.789972in}}%
\pgfpathlineto{\pgfqpoint{4.099648in}{5.784973in}}%
\pgfpathlineto{\pgfqpoint{4.222871in}{5.780204in}}%
\pgfpathlineto{\pgfqpoint{4.346094in}{5.775643in}}%
\pgfpathlineto{\pgfqpoint{4.469317in}{5.771269in}}%
\pgfpathlineto{\pgfqpoint{4.592540in}{5.767067in}}%
\pgfpathlineto{\pgfqpoint{4.715763in}{5.763026in}}%
\pgfpathlineto{\pgfqpoint{4.838986in}{5.759133in}}%
\pgfpathlineto{\pgfqpoint{4.962209in}{5.755375in}}%
\pgfpathlineto{\pgfqpoint{5.085432in}{5.751745in}}%
\pgfpathlineto{\pgfqpoint{5.208655in}{5.748231in}}%
\pgfpathlineto{\pgfqpoint{5.331878in}{5.744823in}}%
\pgfpathlineto{\pgfqpoint{5.455101in}{5.741514in}}%
\pgfpathlineto{\pgfqpoint{5.578324in}{5.738298in}}%
\pgfpathlineto{\pgfqpoint{5.701547in}{5.735166in}}%
\pgfpathlineto{\pgfqpoint{5.824770in}{5.732114in}}%
\pgfpathlineto{\pgfqpoint{5.947993in}{5.729137in}}%
\pgfpathlineto{\pgfqpoint{6.071216in}{5.726232in}}%
\pgfpathlineto{\pgfqpoint{6.194439in}{5.723394in}}%
\pgfpathlineto{\pgfqpoint{6.317662in}{5.720619in}}%
\pgfpathlineto{\pgfqpoint{6.440885in}{5.717904in}}%
\pgfpathlineto{\pgfqpoint{6.564108in}{5.715245in}}%
\pgfpathlineto{\pgfqpoint{6.687330in}{5.712639in}}%
\pgfpathlineto{\pgfqpoint{6.810553in}{5.710086in}}%
\pgfpathlineto{\pgfqpoint{6.933776in}{5.707584in}}%
\pgfpathlineto{\pgfqpoint{7.056999in}{5.705132in}}%
\pgfpathlineto{\pgfqpoint{7.180222in}{5.702727in}}%
\pgfpathlineto{\pgfqpoint{7.365057in}{5.700370in}}%
\pgfpathlineto{\pgfqpoint{7.365057in}{6.156617in}}%
\pgfpathlineto{\pgfqpoint{7.365057in}{6.156617in}}%
\pgfpathlineto{\pgfqpoint{7.180222in}{6.154260in}}%
\pgfpathlineto{\pgfqpoint{7.056999in}{6.151855in}}%
\pgfpathlineto{\pgfqpoint{6.933776in}{6.149403in}}%
\pgfpathlineto{\pgfqpoint{6.810553in}{6.146901in}}%
\pgfpathlineto{\pgfqpoint{6.687330in}{6.144348in}}%
\pgfpathlineto{\pgfqpoint{6.564108in}{6.141742in}}%
\pgfpathlineto{\pgfqpoint{6.440885in}{6.139083in}}%
\pgfpathlineto{\pgfqpoint{6.317662in}{6.136367in}}%
\pgfpathlineto{\pgfqpoint{6.194439in}{6.133593in}}%
\pgfpathlineto{\pgfqpoint{6.071216in}{6.130755in}}%
\pgfpathlineto{\pgfqpoint{5.947993in}{6.127850in}}%
\pgfpathlineto{\pgfqpoint{5.824770in}{6.124873in}}%
\pgfpathlineto{\pgfqpoint{5.701547in}{6.121820in}}%
\pgfpathlineto{\pgfqpoint{5.578324in}{6.118689in}}%
\pgfpathlineto{\pgfqpoint{5.455101in}{6.115473in}}%
\pgfpathlineto{\pgfqpoint{5.331878in}{6.112164in}}%
\pgfpathlineto{\pgfqpoint{5.208655in}{6.108756in}}%
\pgfpathlineto{\pgfqpoint{5.085432in}{6.105242in}}%
\pgfpathlineto{\pgfqpoint{4.962209in}{6.101611in}}%
\pgfpathlineto{\pgfqpoint{4.838986in}{6.097854in}}%
\pgfpathlineto{\pgfqpoint{4.715763in}{6.093961in}}%
\pgfpathlineto{\pgfqpoint{4.592540in}{6.089919in}}%
\pgfpathlineto{\pgfqpoint{4.469317in}{6.085718in}}%
\pgfpathlineto{\pgfqpoint{4.346094in}{6.081344in}}%
\pgfpathlineto{\pgfqpoint{4.222871in}{6.076783in}}%
\pgfpathlineto{\pgfqpoint{4.099648in}{6.072014in}}%
\pgfpathlineto{\pgfqpoint{3.976425in}{6.067015in}}%
\pgfpathlineto{\pgfqpoint{3.853202in}{6.061763in}}%
\pgfpathlineto{\pgfqpoint{3.729979in}{6.056226in}}%
\pgfpathlineto{\pgfqpoint{3.606756in}{6.050365in}}%
\pgfpathlineto{\pgfqpoint{3.483533in}{6.044131in}}%
\pgfpathlineto{\pgfqpoint{3.360310in}{6.037452in}}%
\pgfpathlineto{\pgfqpoint{3.237087in}{6.030240in}}%
\pgfpathlineto{\pgfqpoint{3.113864in}{6.022372in}}%
\pgfpathlineto{\pgfqpoint{2.990641in}{6.013667in}}%
\pgfpathlineto{\pgfqpoint{2.867418in}{6.003840in}}%
\pgfpathlineto{\pgfqpoint{2.744195in}{5.992367in}}%
\pgfpathlineto{\pgfqpoint{2.620972in}{5.978122in}}%
\pgfpathlineto{\pgfqpoint{2.436138in}{5.957251in}}%
\pgfpathclose%
\pgfusepath{stroke,fill}%
\end{pgfscope}%
\begin{pgfscope}%
\pgfpathrectangle{\pgfqpoint{2.125000in}{5.660465in}}{\pgfqpoint{5.489583in}{0.877907in}}%
\pgfusepath{clip}%
\pgfsetbuttcap%
\pgfsetroundjoin%
\pgfsetlinewidth{1.505625pt}%
\definecolor{currentstroke}{rgb}{0.000000,0.000000,0.000000}%
\pgfsetstrokecolor{currentstroke}%
\pgfsetdash{}{0pt}%
\pgfpathmoveto{\pgfqpoint{2.374527in}{5.928493in}}%
\pgfpathlineto{\pgfqpoint{2.374527in}{6.498467in}}%
\pgfusepath{stroke}%
\end{pgfscope}%
\begin{pgfscope}%
\pgfpathrectangle{\pgfqpoint{2.125000in}{5.660465in}}{\pgfqpoint{5.489583in}{0.877907in}}%
\pgfusepath{clip}%
\pgfsetbuttcap%
\pgfsetroundjoin%
\pgfsetlinewidth{1.505625pt}%
\definecolor{currentstroke}{rgb}{0.000000,0.000000,0.000000}%
\pgfsetstrokecolor{currentstroke}%
\pgfsetdash{}{0pt}%
\pgfpathmoveto{\pgfqpoint{2.497749in}{5.928493in}}%
\pgfpathlineto{\pgfqpoint{2.497749in}{6.495353in}}%
\pgfusepath{stroke}%
\end{pgfscope}%
\begin{pgfscope}%
\pgfpathrectangle{\pgfqpoint{2.125000in}{5.660465in}}{\pgfqpoint{5.489583in}{0.877907in}}%
\pgfusepath{clip}%
\pgfsetbuttcap%
\pgfsetroundjoin%
\pgfsetlinewidth{1.505625pt}%
\definecolor{currentstroke}{rgb}{0.000000,0.000000,0.000000}%
\pgfsetstrokecolor{currentstroke}%
\pgfsetdash{}{0pt}%
\pgfpathmoveto{\pgfqpoint{2.620972in}{5.928493in}}%
\pgfpathlineto{\pgfqpoint{2.620972in}{6.492011in}}%
\pgfusepath{stroke}%
\end{pgfscope}%
\begin{pgfscope}%
\pgfpathrectangle{\pgfqpoint{2.125000in}{5.660465in}}{\pgfqpoint{5.489583in}{0.877907in}}%
\pgfusepath{clip}%
\pgfsetbuttcap%
\pgfsetroundjoin%
\pgfsetlinewidth{1.505625pt}%
\definecolor{currentstroke}{rgb}{0.000000,0.000000,0.000000}%
\pgfsetstrokecolor{currentstroke}%
\pgfsetdash{}{0pt}%
\pgfpathmoveto{\pgfqpoint{2.744195in}{5.928493in}}%
\pgfpathlineto{\pgfqpoint{2.744195in}{6.488593in}}%
\pgfusepath{stroke}%
\end{pgfscope}%
\begin{pgfscope}%
\pgfpathrectangle{\pgfqpoint{2.125000in}{5.660465in}}{\pgfqpoint{5.489583in}{0.877907in}}%
\pgfusepath{clip}%
\pgfsetbuttcap%
\pgfsetroundjoin%
\pgfsetlinewidth{1.505625pt}%
\definecolor{currentstroke}{rgb}{0.000000,0.000000,0.000000}%
\pgfsetstrokecolor{currentstroke}%
\pgfsetdash{}{0pt}%
\pgfpathmoveto{\pgfqpoint{2.867418in}{5.928493in}}%
\pgfpathlineto{\pgfqpoint{2.867418in}{6.485133in}}%
\pgfusepath{stroke}%
\end{pgfscope}%
\begin{pgfscope}%
\pgfpathrectangle{\pgfqpoint{2.125000in}{5.660465in}}{\pgfqpoint{5.489583in}{0.877907in}}%
\pgfusepath{clip}%
\pgfsetbuttcap%
\pgfsetroundjoin%
\pgfsetlinewidth{1.505625pt}%
\definecolor{currentstroke}{rgb}{0.000000,0.000000,0.000000}%
\pgfsetstrokecolor{currentstroke}%
\pgfsetdash{}{0pt}%
\pgfpathmoveto{\pgfqpoint{2.990641in}{5.928493in}}%
\pgfpathlineto{\pgfqpoint{2.990641in}{6.481784in}}%
\pgfusepath{stroke}%
\end{pgfscope}%
\begin{pgfscope}%
\pgfpathrectangle{\pgfqpoint{2.125000in}{5.660465in}}{\pgfqpoint{5.489583in}{0.877907in}}%
\pgfusepath{clip}%
\pgfsetbuttcap%
\pgfsetroundjoin%
\pgfsetlinewidth{1.505625pt}%
\definecolor{currentstroke}{rgb}{0.000000,0.000000,0.000000}%
\pgfsetstrokecolor{currentstroke}%
\pgfsetdash{}{0pt}%
\pgfpathmoveto{\pgfqpoint{3.113864in}{5.928493in}}%
\pgfpathlineto{\pgfqpoint{3.113864in}{6.478315in}}%
\pgfusepath{stroke}%
\end{pgfscope}%
\begin{pgfscope}%
\pgfpathrectangle{\pgfqpoint{2.125000in}{5.660465in}}{\pgfqpoint{5.489583in}{0.877907in}}%
\pgfusepath{clip}%
\pgfsetbuttcap%
\pgfsetroundjoin%
\pgfsetlinewidth{1.505625pt}%
\definecolor{currentstroke}{rgb}{0.000000,0.000000,0.000000}%
\pgfsetstrokecolor{currentstroke}%
\pgfsetdash{}{0pt}%
\pgfpathmoveto{\pgfqpoint{3.237087in}{5.928493in}}%
\pgfpathlineto{\pgfqpoint{3.237087in}{6.474795in}}%
\pgfusepath{stroke}%
\end{pgfscope}%
\begin{pgfscope}%
\pgfpathrectangle{\pgfqpoint{2.125000in}{5.660465in}}{\pgfqpoint{5.489583in}{0.877907in}}%
\pgfusepath{clip}%
\pgfsetbuttcap%
\pgfsetroundjoin%
\pgfsetlinewidth{1.505625pt}%
\definecolor{currentstroke}{rgb}{0.000000,0.000000,0.000000}%
\pgfsetstrokecolor{currentstroke}%
\pgfsetdash{}{0pt}%
\pgfpathmoveto{\pgfqpoint{3.360310in}{5.928493in}}%
\pgfpathlineto{\pgfqpoint{3.360310in}{6.471287in}}%
\pgfusepath{stroke}%
\end{pgfscope}%
\begin{pgfscope}%
\pgfpathrectangle{\pgfqpoint{2.125000in}{5.660465in}}{\pgfqpoint{5.489583in}{0.877907in}}%
\pgfusepath{clip}%
\pgfsetbuttcap%
\pgfsetroundjoin%
\pgfsetlinewidth{1.505625pt}%
\definecolor{currentstroke}{rgb}{0.000000,0.000000,0.000000}%
\pgfsetstrokecolor{currentstroke}%
\pgfsetdash{}{0pt}%
\pgfpathmoveto{\pgfqpoint{3.483533in}{5.928493in}}%
\pgfpathlineto{\pgfqpoint{3.483533in}{6.467787in}}%
\pgfusepath{stroke}%
\end{pgfscope}%
\begin{pgfscope}%
\pgfpathrectangle{\pgfqpoint{2.125000in}{5.660465in}}{\pgfqpoint{5.489583in}{0.877907in}}%
\pgfusepath{clip}%
\pgfsetbuttcap%
\pgfsetroundjoin%
\pgfsetlinewidth{1.505625pt}%
\definecolor{currentstroke}{rgb}{0.000000,0.000000,0.000000}%
\pgfsetstrokecolor{currentstroke}%
\pgfsetdash{}{0pt}%
\pgfpathmoveto{\pgfqpoint{3.606756in}{5.928493in}}%
\pgfpathlineto{\pgfqpoint{3.606756in}{6.464512in}}%
\pgfusepath{stroke}%
\end{pgfscope}%
\begin{pgfscope}%
\pgfpathrectangle{\pgfqpoint{2.125000in}{5.660465in}}{\pgfqpoint{5.489583in}{0.877907in}}%
\pgfusepath{clip}%
\pgfsetbuttcap%
\pgfsetroundjoin%
\pgfsetlinewidth{1.505625pt}%
\definecolor{currentstroke}{rgb}{0.000000,0.000000,0.000000}%
\pgfsetstrokecolor{currentstroke}%
\pgfsetdash{}{0pt}%
\pgfpathmoveto{\pgfqpoint{3.729979in}{5.928493in}}%
\pgfpathlineto{\pgfqpoint{3.729979in}{6.461297in}}%
\pgfusepath{stroke}%
\end{pgfscope}%
\begin{pgfscope}%
\pgfpathrectangle{\pgfqpoint{2.125000in}{5.660465in}}{\pgfqpoint{5.489583in}{0.877907in}}%
\pgfusepath{clip}%
\pgfsetbuttcap%
\pgfsetroundjoin%
\pgfsetlinewidth{1.505625pt}%
\definecolor{currentstroke}{rgb}{0.000000,0.000000,0.000000}%
\pgfsetstrokecolor{currentstroke}%
\pgfsetdash{}{0pt}%
\pgfpathmoveto{\pgfqpoint{3.853202in}{5.928493in}}%
\pgfpathlineto{\pgfqpoint{3.853202in}{6.457978in}}%
\pgfusepath{stroke}%
\end{pgfscope}%
\begin{pgfscope}%
\pgfpathrectangle{\pgfqpoint{2.125000in}{5.660465in}}{\pgfqpoint{5.489583in}{0.877907in}}%
\pgfusepath{clip}%
\pgfsetbuttcap%
\pgfsetroundjoin%
\pgfsetlinewidth{1.505625pt}%
\definecolor{currentstroke}{rgb}{0.000000,0.000000,0.000000}%
\pgfsetstrokecolor{currentstroke}%
\pgfsetdash{}{0pt}%
\pgfpathmoveto{\pgfqpoint{3.976425in}{5.928493in}}%
\pgfpathlineto{\pgfqpoint{3.976425in}{6.454706in}}%
\pgfusepath{stroke}%
\end{pgfscope}%
\begin{pgfscope}%
\pgfpathrectangle{\pgfqpoint{2.125000in}{5.660465in}}{\pgfqpoint{5.489583in}{0.877907in}}%
\pgfusepath{clip}%
\pgfsetbuttcap%
\pgfsetroundjoin%
\pgfsetlinewidth{1.505625pt}%
\definecolor{currentstroke}{rgb}{0.000000,0.000000,0.000000}%
\pgfsetstrokecolor{currentstroke}%
\pgfsetdash{}{0pt}%
\pgfpathmoveto{\pgfqpoint{4.099648in}{5.928493in}}%
\pgfpathlineto{\pgfqpoint{4.099648in}{6.451288in}}%
\pgfusepath{stroke}%
\end{pgfscope}%
\begin{pgfscope}%
\pgfpathrectangle{\pgfqpoint{2.125000in}{5.660465in}}{\pgfqpoint{5.489583in}{0.877907in}}%
\pgfusepath{clip}%
\pgfsetbuttcap%
\pgfsetroundjoin%
\pgfsetlinewidth{1.505625pt}%
\definecolor{currentstroke}{rgb}{0.000000,0.000000,0.000000}%
\pgfsetstrokecolor{currentstroke}%
\pgfsetdash{}{0pt}%
\pgfpathmoveto{\pgfqpoint{4.222871in}{5.928493in}}%
\pgfpathlineto{\pgfqpoint{4.222871in}{6.447915in}}%
\pgfusepath{stroke}%
\end{pgfscope}%
\begin{pgfscope}%
\pgfpathrectangle{\pgfqpoint{2.125000in}{5.660465in}}{\pgfqpoint{5.489583in}{0.877907in}}%
\pgfusepath{clip}%
\pgfsetbuttcap%
\pgfsetroundjoin%
\pgfsetlinewidth{1.505625pt}%
\definecolor{currentstroke}{rgb}{0.000000,0.000000,0.000000}%
\pgfsetstrokecolor{currentstroke}%
\pgfsetdash{}{0pt}%
\pgfpathmoveto{\pgfqpoint{4.346094in}{5.928493in}}%
\pgfpathlineto{\pgfqpoint{4.346094in}{6.444638in}}%
\pgfusepath{stroke}%
\end{pgfscope}%
\begin{pgfscope}%
\pgfpathrectangle{\pgfqpoint{2.125000in}{5.660465in}}{\pgfqpoint{5.489583in}{0.877907in}}%
\pgfusepath{clip}%
\pgfsetbuttcap%
\pgfsetroundjoin%
\pgfsetlinewidth{1.505625pt}%
\definecolor{currentstroke}{rgb}{0.000000,0.000000,0.000000}%
\pgfsetstrokecolor{currentstroke}%
\pgfsetdash{}{0pt}%
\pgfpathmoveto{\pgfqpoint{4.469317in}{5.928493in}}%
\pgfpathlineto{\pgfqpoint{4.469317in}{6.441261in}}%
\pgfusepath{stroke}%
\end{pgfscope}%
\begin{pgfscope}%
\pgfpathrectangle{\pgfqpoint{2.125000in}{5.660465in}}{\pgfqpoint{5.489583in}{0.877907in}}%
\pgfusepath{clip}%
\pgfsetbuttcap%
\pgfsetroundjoin%
\pgfsetlinewidth{1.505625pt}%
\definecolor{currentstroke}{rgb}{0.000000,0.000000,0.000000}%
\pgfsetstrokecolor{currentstroke}%
\pgfsetdash{}{0pt}%
\pgfpathmoveto{\pgfqpoint{4.592540in}{5.928493in}}%
\pgfpathlineto{\pgfqpoint{4.592540in}{6.437869in}}%
\pgfusepath{stroke}%
\end{pgfscope}%
\begin{pgfscope}%
\pgfpathrectangle{\pgfqpoint{2.125000in}{5.660465in}}{\pgfqpoint{5.489583in}{0.877907in}}%
\pgfusepath{clip}%
\pgfsetbuttcap%
\pgfsetroundjoin%
\pgfsetlinewidth{1.505625pt}%
\definecolor{currentstroke}{rgb}{0.000000,0.000000,0.000000}%
\pgfsetstrokecolor{currentstroke}%
\pgfsetdash{}{0pt}%
\pgfpathmoveto{\pgfqpoint{4.715763in}{5.928493in}}%
\pgfpathlineto{\pgfqpoint{4.715763in}{6.434492in}}%
\pgfusepath{stroke}%
\end{pgfscope}%
\begin{pgfscope}%
\pgfpathrectangle{\pgfqpoint{2.125000in}{5.660465in}}{\pgfqpoint{5.489583in}{0.877907in}}%
\pgfusepath{clip}%
\pgfsetbuttcap%
\pgfsetroundjoin%
\pgfsetlinewidth{1.505625pt}%
\definecolor{currentstroke}{rgb}{0.000000,0.000000,0.000000}%
\pgfsetstrokecolor{currentstroke}%
\pgfsetdash{}{0pt}%
\pgfpathmoveto{\pgfqpoint{4.838986in}{5.928493in}}%
\pgfpathlineto{\pgfqpoint{4.838986in}{6.431243in}}%
\pgfusepath{stroke}%
\end{pgfscope}%
\begin{pgfscope}%
\pgfpathrectangle{\pgfqpoint{2.125000in}{5.660465in}}{\pgfqpoint{5.489583in}{0.877907in}}%
\pgfusepath{clip}%
\pgfsetbuttcap%
\pgfsetroundjoin%
\pgfsetlinewidth{1.505625pt}%
\definecolor{currentstroke}{rgb}{0.000000,0.000000,0.000000}%
\pgfsetstrokecolor{currentstroke}%
\pgfsetdash{}{0pt}%
\pgfpathmoveto{\pgfqpoint{4.962209in}{5.928493in}}%
\pgfpathlineto{\pgfqpoint{4.962209in}{6.427989in}}%
\pgfusepath{stroke}%
\end{pgfscope}%
\begin{pgfscope}%
\pgfpathrectangle{\pgfqpoint{2.125000in}{5.660465in}}{\pgfqpoint{5.489583in}{0.877907in}}%
\pgfusepath{clip}%
\pgfsetbuttcap%
\pgfsetroundjoin%
\pgfsetlinewidth{1.505625pt}%
\definecolor{currentstroke}{rgb}{0.000000,0.000000,0.000000}%
\pgfsetstrokecolor{currentstroke}%
\pgfsetdash{}{0pt}%
\pgfpathmoveto{\pgfqpoint{5.085432in}{5.928493in}}%
\pgfpathlineto{\pgfqpoint{5.085432in}{6.424890in}}%
\pgfusepath{stroke}%
\end{pgfscope}%
\begin{pgfscope}%
\pgfpathrectangle{\pgfqpoint{2.125000in}{5.660465in}}{\pgfqpoint{5.489583in}{0.877907in}}%
\pgfusepath{clip}%
\pgfsetbuttcap%
\pgfsetroundjoin%
\pgfsetlinewidth{1.505625pt}%
\definecolor{currentstroke}{rgb}{0.000000,0.000000,0.000000}%
\pgfsetstrokecolor{currentstroke}%
\pgfsetdash{}{0pt}%
\pgfpathmoveto{\pgfqpoint{5.208655in}{5.928493in}}%
\pgfpathlineto{\pgfqpoint{5.208655in}{6.422039in}}%
\pgfusepath{stroke}%
\end{pgfscope}%
\begin{pgfscope}%
\pgfpathrectangle{\pgfqpoint{2.125000in}{5.660465in}}{\pgfqpoint{5.489583in}{0.877907in}}%
\pgfusepath{clip}%
\pgfsetbuttcap%
\pgfsetroundjoin%
\pgfsetlinewidth{1.505625pt}%
\definecolor{currentstroke}{rgb}{0.000000,0.000000,0.000000}%
\pgfsetstrokecolor{currentstroke}%
\pgfsetdash{}{0pt}%
\pgfpathmoveto{\pgfqpoint{5.331878in}{5.928493in}}%
\pgfpathlineto{\pgfqpoint{5.331878in}{6.419286in}}%
\pgfusepath{stroke}%
\end{pgfscope}%
\begin{pgfscope}%
\pgfpathrectangle{\pgfqpoint{2.125000in}{5.660465in}}{\pgfqpoint{5.489583in}{0.877907in}}%
\pgfusepath{clip}%
\pgfsetbuttcap%
\pgfsetroundjoin%
\pgfsetlinewidth{1.505625pt}%
\definecolor{currentstroke}{rgb}{0.000000,0.000000,0.000000}%
\pgfsetstrokecolor{currentstroke}%
\pgfsetdash{}{0pt}%
\pgfpathmoveto{\pgfqpoint{5.455101in}{5.928493in}}%
\pgfpathlineto{\pgfqpoint{5.455101in}{6.416623in}}%
\pgfusepath{stroke}%
\end{pgfscope}%
\begin{pgfscope}%
\pgfpathrectangle{\pgfqpoint{2.125000in}{5.660465in}}{\pgfqpoint{5.489583in}{0.877907in}}%
\pgfusepath{clip}%
\pgfsetbuttcap%
\pgfsetroundjoin%
\pgfsetlinewidth{1.505625pt}%
\definecolor{currentstroke}{rgb}{0.000000,0.000000,0.000000}%
\pgfsetstrokecolor{currentstroke}%
\pgfsetdash{}{0pt}%
\pgfpathmoveto{\pgfqpoint{5.578324in}{5.928493in}}%
\pgfpathlineto{\pgfqpoint{5.578324in}{6.414160in}}%
\pgfusepath{stroke}%
\end{pgfscope}%
\begin{pgfscope}%
\pgfpathrectangle{\pgfqpoint{2.125000in}{5.660465in}}{\pgfqpoint{5.489583in}{0.877907in}}%
\pgfusepath{clip}%
\pgfsetbuttcap%
\pgfsetroundjoin%
\pgfsetlinewidth{1.505625pt}%
\definecolor{currentstroke}{rgb}{0.000000,0.000000,0.000000}%
\pgfsetstrokecolor{currentstroke}%
\pgfsetdash{}{0pt}%
\pgfpathmoveto{\pgfqpoint{5.701547in}{5.928493in}}%
\pgfpathlineto{\pgfqpoint{5.701547in}{6.411886in}}%
\pgfusepath{stroke}%
\end{pgfscope}%
\begin{pgfscope}%
\pgfpathrectangle{\pgfqpoint{2.125000in}{5.660465in}}{\pgfqpoint{5.489583in}{0.877907in}}%
\pgfusepath{clip}%
\pgfsetbuttcap%
\pgfsetroundjoin%
\pgfsetlinewidth{1.505625pt}%
\definecolor{currentstroke}{rgb}{0.000000,0.000000,0.000000}%
\pgfsetstrokecolor{currentstroke}%
\pgfsetdash{}{0pt}%
\pgfpathmoveto{\pgfqpoint{5.824770in}{5.928493in}}%
\pgfpathlineto{\pgfqpoint{5.824770in}{6.409473in}}%
\pgfusepath{stroke}%
\end{pgfscope}%
\begin{pgfscope}%
\pgfpathrectangle{\pgfqpoint{2.125000in}{5.660465in}}{\pgfqpoint{5.489583in}{0.877907in}}%
\pgfusepath{clip}%
\pgfsetbuttcap%
\pgfsetroundjoin%
\pgfsetlinewidth{1.505625pt}%
\definecolor{currentstroke}{rgb}{0.000000,0.000000,0.000000}%
\pgfsetstrokecolor{currentstroke}%
\pgfsetdash{}{0pt}%
\pgfpathmoveto{\pgfqpoint{5.947993in}{5.928493in}}%
\pgfpathlineto{\pgfqpoint{5.947993in}{6.407214in}}%
\pgfusepath{stroke}%
\end{pgfscope}%
\begin{pgfscope}%
\pgfpathrectangle{\pgfqpoint{2.125000in}{5.660465in}}{\pgfqpoint{5.489583in}{0.877907in}}%
\pgfusepath{clip}%
\pgfsetbuttcap%
\pgfsetroundjoin%
\pgfsetlinewidth{1.505625pt}%
\definecolor{currentstroke}{rgb}{0.000000,0.000000,0.000000}%
\pgfsetstrokecolor{currentstroke}%
\pgfsetdash{}{0pt}%
\pgfpathmoveto{\pgfqpoint{6.071216in}{5.928493in}}%
\pgfpathlineto{\pgfqpoint{6.071216in}{6.405011in}}%
\pgfusepath{stroke}%
\end{pgfscope}%
\begin{pgfscope}%
\pgfpathrectangle{\pgfqpoint{2.125000in}{5.660465in}}{\pgfqpoint{5.489583in}{0.877907in}}%
\pgfusepath{clip}%
\pgfsetbuttcap%
\pgfsetroundjoin%
\pgfsetlinewidth{1.505625pt}%
\definecolor{currentstroke}{rgb}{0.000000,0.000000,0.000000}%
\pgfsetstrokecolor{currentstroke}%
\pgfsetdash{}{0pt}%
\pgfpathmoveto{\pgfqpoint{6.194439in}{5.928493in}}%
\pgfpathlineto{\pgfqpoint{6.194439in}{6.402876in}}%
\pgfusepath{stroke}%
\end{pgfscope}%
\begin{pgfscope}%
\pgfpathrectangle{\pgfqpoint{2.125000in}{5.660465in}}{\pgfqpoint{5.489583in}{0.877907in}}%
\pgfusepath{clip}%
\pgfsetbuttcap%
\pgfsetroundjoin%
\pgfsetlinewidth{1.505625pt}%
\definecolor{currentstroke}{rgb}{0.000000,0.000000,0.000000}%
\pgfsetstrokecolor{currentstroke}%
\pgfsetdash{}{0pt}%
\pgfpathmoveto{\pgfqpoint{6.317662in}{5.928493in}}%
\pgfpathlineto{\pgfqpoint{6.317662in}{6.400928in}}%
\pgfusepath{stroke}%
\end{pgfscope}%
\begin{pgfscope}%
\pgfpathrectangle{\pgfqpoint{2.125000in}{5.660465in}}{\pgfqpoint{5.489583in}{0.877907in}}%
\pgfusepath{clip}%
\pgfsetbuttcap%
\pgfsetroundjoin%
\pgfsetlinewidth{1.505625pt}%
\definecolor{currentstroke}{rgb}{0.000000,0.000000,0.000000}%
\pgfsetstrokecolor{currentstroke}%
\pgfsetdash{}{0pt}%
\pgfpathmoveto{\pgfqpoint{6.440885in}{5.928493in}}%
\pgfpathlineto{\pgfqpoint{6.440885in}{6.398997in}}%
\pgfusepath{stroke}%
\end{pgfscope}%
\begin{pgfscope}%
\pgfpathrectangle{\pgfqpoint{2.125000in}{5.660465in}}{\pgfqpoint{5.489583in}{0.877907in}}%
\pgfusepath{clip}%
\pgfsetbuttcap%
\pgfsetroundjoin%
\pgfsetlinewidth{1.505625pt}%
\definecolor{currentstroke}{rgb}{0.000000,0.000000,0.000000}%
\pgfsetstrokecolor{currentstroke}%
\pgfsetdash{}{0pt}%
\pgfpathmoveto{\pgfqpoint{6.564108in}{5.928493in}}%
\pgfpathlineto{\pgfqpoint{6.564108in}{6.397107in}}%
\pgfusepath{stroke}%
\end{pgfscope}%
\begin{pgfscope}%
\pgfpathrectangle{\pgfqpoint{2.125000in}{5.660465in}}{\pgfqpoint{5.489583in}{0.877907in}}%
\pgfusepath{clip}%
\pgfsetbuttcap%
\pgfsetroundjoin%
\pgfsetlinewidth{1.505625pt}%
\definecolor{currentstroke}{rgb}{0.000000,0.000000,0.000000}%
\pgfsetstrokecolor{currentstroke}%
\pgfsetdash{}{0pt}%
\pgfpathmoveto{\pgfqpoint{6.687330in}{5.928493in}}%
\pgfpathlineto{\pgfqpoint{6.687330in}{6.395153in}}%
\pgfusepath{stroke}%
\end{pgfscope}%
\begin{pgfscope}%
\pgfpathrectangle{\pgfqpoint{2.125000in}{5.660465in}}{\pgfqpoint{5.489583in}{0.877907in}}%
\pgfusepath{clip}%
\pgfsetbuttcap%
\pgfsetroundjoin%
\pgfsetlinewidth{1.505625pt}%
\definecolor{currentstroke}{rgb}{0.000000,0.000000,0.000000}%
\pgfsetstrokecolor{currentstroke}%
\pgfsetdash{}{0pt}%
\pgfpathmoveto{\pgfqpoint{6.810553in}{5.928493in}}%
\pgfpathlineto{\pgfqpoint{6.810553in}{6.393100in}}%
\pgfusepath{stroke}%
\end{pgfscope}%
\begin{pgfscope}%
\pgfpathrectangle{\pgfqpoint{2.125000in}{5.660465in}}{\pgfqpoint{5.489583in}{0.877907in}}%
\pgfusepath{clip}%
\pgfsetbuttcap%
\pgfsetroundjoin%
\pgfsetlinewidth{1.505625pt}%
\definecolor{currentstroke}{rgb}{0.000000,0.000000,0.000000}%
\pgfsetstrokecolor{currentstroke}%
\pgfsetdash{}{0pt}%
\pgfpathmoveto{\pgfqpoint{6.933776in}{5.928493in}}%
\pgfpathlineto{\pgfqpoint{6.933776in}{6.391090in}}%
\pgfusepath{stroke}%
\end{pgfscope}%
\begin{pgfscope}%
\pgfpathrectangle{\pgfqpoint{2.125000in}{5.660465in}}{\pgfqpoint{5.489583in}{0.877907in}}%
\pgfusepath{clip}%
\pgfsetbuttcap%
\pgfsetroundjoin%
\pgfsetlinewidth{1.505625pt}%
\definecolor{currentstroke}{rgb}{0.000000,0.000000,0.000000}%
\pgfsetstrokecolor{currentstroke}%
\pgfsetdash{}{0pt}%
\pgfpathmoveto{\pgfqpoint{7.056999in}{5.928493in}}%
\pgfpathlineto{\pgfqpoint{7.056999in}{6.389099in}}%
\pgfusepath{stroke}%
\end{pgfscope}%
\begin{pgfscope}%
\pgfpathrectangle{\pgfqpoint{2.125000in}{5.660465in}}{\pgfqpoint{5.489583in}{0.877907in}}%
\pgfusepath{clip}%
\pgfsetbuttcap%
\pgfsetroundjoin%
\pgfsetlinewidth{1.505625pt}%
\definecolor{currentstroke}{rgb}{0.000000,0.000000,0.000000}%
\pgfsetstrokecolor{currentstroke}%
\pgfsetdash{}{0pt}%
\pgfpathmoveto{\pgfqpoint{7.180222in}{5.928493in}}%
\pgfpathlineto{\pgfqpoint{7.180222in}{6.386862in}}%
\pgfusepath{stroke}%
\end{pgfscope}%
\begin{pgfscope}%
\pgfpathrectangle{\pgfqpoint{2.125000in}{5.660465in}}{\pgfqpoint{5.489583in}{0.877907in}}%
\pgfusepath{clip}%
\pgfsetbuttcap%
\pgfsetroundjoin%
\pgfsetlinewidth{1.505625pt}%
\definecolor{currentstroke}{rgb}{0.000000,0.000000,0.000000}%
\pgfsetstrokecolor{currentstroke}%
\pgfsetdash{}{0pt}%
\pgfpathmoveto{\pgfqpoint{7.303445in}{5.928493in}}%
\pgfpathlineto{\pgfqpoint{7.303445in}{6.384645in}}%
\pgfusepath{stroke}%
\end{pgfscope}%
\begin{pgfscope}%
\pgfpathrectangle{\pgfqpoint{2.125000in}{5.660465in}}{\pgfqpoint{5.489583in}{0.877907in}}%
\pgfusepath{clip}%
\pgfsetroundcap%
\pgfsetroundjoin%
\pgfsetlinewidth{1.505625pt}%
\definecolor{currentstroke}{rgb}{0.121569,0.466667,0.705882}%
\pgfsetstrokecolor{currentstroke}%
\pgfsetdash{}{0pt}%
\pgfpathmoveto{\pgfqpoint{2.125000in}{5.928493in}}%
\pgfpathlineto{\pgfqpoint{7.614583in}{5.928493in}}%
\pgfusepath{stroke}%
\end{pgfscope}%
\begin{pgfscope}%
\pgfpathrectangle{\pgfqpoint{2.125000in}{5.660465in}}{\pgfqpoint{5.489583in}{0.877907in}}%
\pgfusepath{clip}%
\pgfsetbuttcap%
\pgfsetroundjoin%
\definecolor{currentfill}{rgb}{0.121569,0.466667,0.705882}%
\pgfsetfillcolor{currentfill}%
\pgfsetlinewidth{1.003750pt}%
\definecolor{currentstroke}{rgb}{0.121569,0.466667,0.705882}%
\pgfsetstrokecolor{currentstroke}%
\pgfsetdash{}{0pt}%
\pgfsys@defobject{currentmarker}{\pgfqpoint{-0.034722in}{-0.034722in}}{\pgfqpoint{0.034722in}{0.034722in}}{%
\pgfpathmoveto{\pgfqpoint{0.000000in}{-0.034722in}}%
\pgfpathcurveto{\pgfqpoint{0.009208in}{-0.034722in}}{\pgfqpoint{0.018041in}{-0.031064in}}{\pgfqpoint{0.024552in}{-0.024552in}}%
\pgfpathcurveto{\pgfqpoint{0.031064in}{-0.018041in}}{\pgfqpoint{0.034722in}{-0.009208in}}{\pgfqpoint{0.034722in}{0.000000in}}%
\pgfpathcurveto{\pgfqpoint{0.034722in}{0.009208in}}{\pgfqpoint{0.031064in}{0.018041in}}{\pgfqpoint{0.024552in}{0.024552in}}%
\pgfpathcurveto{\pgfqpoint{0.018041in}{0.031064in}}{\pgfqpoint{0.009208in}{0.034722in}}{\pgfqpoint{0.000000in}{0.034722in}}%
\pgfpathcurveto{\pgfqpoint{-0.009208in}{0.034722in}}{\pgfqpoint{-0.018041in}{0.031064in}}{\pgfqpoint{-0.024552in}{0.024552in}}%
\pgfpathcurveto{\pgfqpoint{-0.031064in}{0.018041in}}{\pgfqpoint{-0.034722in}{0.009208in}}{\pgfqpoint{-0.034722in}{0.000000in}}%
\pgfpathcurveto{\pgfqpoint{-0.034722in}{-0.009208in}}{\pgfqpoint{-0.031064in}{-0.018041in}}{\pgfqpoint{-0.024552in}{-0.024552in}}%
\pgfpathcurveto{\pgfqpoint{-0.018041in}{-0.031064in}}{\pgfqpoint{-0.009208in}{-0.034722in}}{\pgfqpoint{0.000000in}{-0.034722in}}%
\pgfpathclose%
\pgfusepath{stroke,fill}%
}%
\begin{pgfscope}%
\pgfsys@transformshift{2.374527in}{6.498467in}%
\pgfsys@useobject{currentmarker}{}%
\end{pgfscope}%
\begin{pgfscope}%
\pgfsys@transformshift{2.497749in}{6.495353in}%
\pgfsys@useobject{currentmarker}{}%
\end{pgfscope}%
\begin{pgfscope}%
\pgfsys@transformshift{2.620972in}{6.492011in}%
\pgfsys@useobject{currentmarker}{}%
\end{pgfscope}%
\begin{pgfscope}%
\pgfsys@transformshift{2.744195in}{6.488593in}%
\pgfsys@useobject{currentmarker}{}%
\end{pgfscope}%
\begin{pgfscope}%
\pgfsys@transformshift{2.867418in}{6.485133in}%
\pgfsys@useobject{currentmarker}{}%
\end{pgfscope}%
\begin{pgfscope}%
\pgfsys@transformshift{2.990641in}{6.481784in}%
\pgfsys@useobject{currentmarker}{}%
\end{pgfscope}%
\begin{pgfscope}%
\pgfsys@transformshift{3.113864in}{6.478315in}%
\pgfsys@useobject{currentmarker}{}%
\end{pgfscope}%
\begin{pgfscope}%
\pgfsys@transformshift{3.237087in}{6.474795in}%
\pgfsys@useobject{currentmarker}{}%
\end{pgfscope}%
\begin{pgfscope}%
\pgfsys@transformshift{3.360310in}{6.471287in}%
\pgfsys@useobject{currentmarker}{}%
\end{pgfscope}%
\begin{pgfscope}%
\pgfsys@transformshift{3.483533in}{6.467787in}%
\pgfsys@useobject{currentmarker}{}%
\end{pgfscope}%
\begin{pgfscope}%
\pgfsys@transformshift{3.606756in}{6.464512in}%
\pgfsys@useobject{currentmarker}{}%
\end{pgfscope}%
\begin{pgfscope}%
\pgfsys@transformshift{3.729979in}{6.461297in}%
\pgfsys@useobject{currentmarker}{}%
\end{pgfscope}%
\begin{pgfscope}%
\pgfsys@transformshift{3.853202in}{6.457978in}%
\pgfsys@useobject{currentmarker}{}%
\end{pgfscope}%
\begin{pgfscope}%
\pgfsys@transformshift{3.976425in}{6.454706in}%
\pgfsys@useobject{currentmarker}{}%
\end{pgfscope}%
\begin{pgfscope}%
\pgfsys@transformshift{4.099648in}{6.451288in}%
\pgfsys@useobject{currentmarker}{}%
\end{pgfscope}%
\begin{pgfscope}%
\pgfsys@transformshift{4.222871in}{6.447915in}%
\pgfsys@useobject{currentmarker}{}%
\end{pgfscope}%
\begin{pgfscope}%
\pgfsys@transformshift{4.346094in}{6.444638in}%
\pgfsys@useobject{currentmarker}{}%
\end{pgfscope}%
\begin{pgfscope}%
\pgfsys@transformshift{4.469317in}{6.441261in}%
\pgfsys@useobject{currentmarker}{}%
\end{pgfscope}%
\begin{pgfscope}%
\pgfsys@transformshift{4.592540in}{6.437869in}%
\pgfsys@useobject{currentmarker}{}%
\end{pgfscope}%
\begin{pgfscope}%
\pgfsys@transformshift{4.715763in}{6.434492in}%
\pgfsys@useobject{currentmarker}{}%
\end{pgfscope}%
\begin{pgfscope}%
\pgfsys@transformshift{4.838986in}{6.431243in}%
\pgfsys@useobject{currentmarker}{}%
\end{pgfscope}%
\begin{pgfscope}%
\pgfsys@transformshift{4.962209in}{6.427989in}%
\pgfsys@useobject{currentmarker}{}%
\end{pgfscope}%
\begin{pgfscope}%
\pgfsys@transformshift{5.085432in}{6.424890in}%
\pgfsys@useobject{currentmarker}{}%
\end{pgfscope}%
\begin{pgfscope}%
\pgfsys@transformshift{5.208655in}{6.422039in}%
\pgfsys@useobject{currentmarker}{}%
\end{pgfscope}%
\begin{pgfscope}%
\pgfsys@transformshift{5.331878in}{6.419286in}%
\pgfsys@useobject{currentmarker}{}%
\end{pgfscope}%
\begin{pgfscope}%
\pgfsys@transformshift{5.455101in}{6.416623in}%
\pgfsys@useobject{currentmarker}{}%
\end{pgfscope}%
\begin{pgfscope}%
\pgfsys@transformshift{5.578324in}{6.414160in}%
\pgfsys@useobject{currentmarker}{}%
\end{pgfscope}%
\begin{pgfscope}%
\pgfsys@transformshift{5.701547in}{6.411886in}%
\pgfsys@useobject{currentmarker}{}%
\end{pgfscope}%
\begin{pgfscope}%
\pgfsys@transformshift{5.824770in}{6.409473in}%
\pgfsys@useobject{currentmarker}{}%
\end{pgfscope}%
\begin{pgfscope}%
\pgfsys@transformshift{5.947993in}{6.407214in}%
\pgfsys@useobject{currentmarker}{}%
\end{pgfscope}%
\begin{pgfscope}%
\pgfsys@transformshift{6.071216in}{6.405011in}%
\pgfsys@useobject{currentmarker}{}%
\end{pgfscope}%
\begin{pgfscope}%
\pgfsys@transformshift{6.194439in}{6.402876in}%
\pgfsys@useobject{currentmarker}{}%
\end{pgfscope}%
\begin{pgfscope}%
\pgfsys@transformshift{6.317662in}{6.400928in}%
\pgfsys@useobject{currentmarker}{}%
\end{pgfscope}%
\begin{pgfscope}%
\pgfsys@transformshift{6.440885in}{6.398997in}%
\pgfsys@useobject{currentmarker}{}%
\end{pgfscope}%
\begin{pgfscope}%
\pgfsys@transformshift{6.564108in}{6.397107in}%
\pgfsys@useobject{currentmarker}{}%
\end{pgfscope}%
\begin{pgfscope}%
\pgfsys@transformshift{6.687330in}{6.395153in}%
\pgfsys@useobject{currentmarker}{}%
\end{pgfscope}%
\begin{pgfscope}%
\pgfsys@transformshift{6.810553in}{6.393100in}%
\pgfsys@useobject{currentmarker}{}%
\end{pgfscope}%
\begin{pgfscope}%
\pgfsys@transformshift{6.933776in}{6.391090in}%
\pgfsys@useobject{currentmarker}{}%
\end{pgfscope}%
\begin{pgfscope}%
\pgfsys@transformshift{7.056999in}{6.389099in}%
\pgfsys@useobject{currentmarker}{}%
\end{pgfscope}%
\begin{pgfscope}%
\pgfsys@transformshift{7.180222in}{6.386862in}%
\pgfsys@useobject{currentmarker}{}%
\end{pgfscope}%
\begin{pgfscope}%
\pgfsys@transformshift{7.303445in}{6.384645in}%
\pgfsys@useobject{currentmarker}{}%
\end{pgfscope}%
\end{pgfscope}%
\begin{pgfscope}%
\pgfsetrectcap%
\pgfsetmiterjoin%
\pgfsetlinewidth{0.803000pt}%
\definecolor{currentstroke}{rgb}{1.000000,1.000000,1.000000}%
\pgfsetstrokecolor{currentstroke}%
\pgfsetdash{}{0pt}%
\pgfpathmoveto{\pgfqpoint{2.125000in}{5.660465in}}%
\pgfpathlineto{\pgfqpoint{2.125000in}{6.538372in}}%
\pgfusepath{stroke}%
\end{pgfscope}%
\begin{pgfscope}%
\pgfsetrectcap%
\pgfsetmiterjoin%
\pgfsetlinewidth{0.803000pt}%
\definecolor{currentstroke}{rgb}{1.000000,1.000000,1.000000}%
\pgfsetstrokecolor{currentstroke}%
\pgfsetdash{}{0pt}%
\pgfpathmoveto{\pgfqpoint{7.614583in}{5.660465in}}%
\pgfpathlineto{\pgfqpoint{7.614583in}{6.538372in}}%
\pgfusepath{stroke}%
\end{pgfscope}%
\begin{pgfscope}%
\pgfsetrectcap%
\pgfsetmiterjoin%
\pgfsetlinewidth{0.803000pt}%
\definecolor{currentstroke}{rgb}{1.000000,1.000000,1.000000}%
\pgfsetstrokecolor{currentstroke}%
\pgfsetdash{}{0pt}%
\pgfpathmoveto{\pgfqpoint{2.125000in}{5.660465in}}%
\pgfpathlineto{\pgfqpoint{7.614583in}{5.660465in}}%
\pgfusepath{stroke}%
\end{pgfscope}%
\begin{pgfscope}%
\pgfsetrectcap%
\pgfsetmiterjoin%
\pgfsetlinewidth{0.803000pt}%
\definecolor{currentstroke}{rgb}{1.000000,1.000000,1.000000}%
\pgfsetstrokecolor{currentstroke}%
\pgfsetdash{}{0pt}%
\pgfpathmoveto{\pgfqpoint{2.125000in}{6.538372in}}%
\pgfpathlineto{\pgfqpoint{7.614583in}{6.538372in}}%
\pgfusepath{stroke}%
\end{pgfscope}%
\begin{pgfscope}%
\definecolor{textcolor}{rgb}{0.150000,0.150000,0.150000}%
\pgfsetstrokecolor{textcolor}%
\pgfsetfillcolor{textcolor}%
\pgftext[x=4.869792in,y=6.621705in,,base]{\color{textcolor}\rmfamily\fontsize{16.800000}{20.160000}\selectfont Autocorrelation}%
\end{pgfscope}%
\begin{pgfscope}%
\pgfsetbuttcap%
\pgfsetmiterjoin%
\definecolor{currentfill}{rgb}{0.917647,0.917647,0.949020}%
\pgfsetfillcolor{currentfill}%
\pgfsetlinewidth{0.000000pt}%
\definecolor{currentstroke}{rgb}{0.000000,0.000000,0.000000}%
\pgfsetstrokecolor{currentstroke}%
\pgfsetstrokeopacity{0.000000}%
\pgfsetdash{}{0pt}%
\pgfpathmoveto{\pgfqpoint{9.810417in}{5.660465in}}%
\pgfpathlineto{\pgfqpoint{15.300000in}{5.660465in}}%
\pgfpathlineto{\pgfqpoint{15.300000in}{6.538372in}}%
\pgfpathlineto{\pgfqpoint{9.810417in}{6.538372in}}%
\pgfpathclose%
\pgfusepath{fill}%
\end{pgfscope}%
\begin{pgfscope}%
\pgfpathrectangle{\pgfqpoint{9.810417in}{5.660465in}}{\pgfqpoint{5.489583in}{0.877907in}}%
\pgfusepath{clip}%
\pgfsetroundcap%
\pgfsetroundjoin%
\pgfsetlinewidth{0.803000pt}%
\definecolor{currentstroke}{rgb}{1.000000,1.000000,1.000000}%
\pgfsetstrokecolor{currentstroke}%
\pgfsetdash{}{0pt}%
\pgfpathmoveto{\pgfqpoint{10.059943in}{5.660465in}}%
\pgfpathlineto{\pgfqpoint{10.059943in}{6.538372in}}%
\pgfusepath{stroke}%
\end{pgfscope}%
\begin{pgfscope}%
\definecolor{textcolor}{rgb}{0.150000,0.150000,0.150000}%
\pgfsetstrokecolor{textcolor}%
\pgfsetfillcolor{textcolor}%
\pgftext[x=10.059943in,y=5.563243in,,top]{\color{textcolor}\rmfamily\fontsize{14.000000}{16.800000}\selectfont 0}%
\end{pgfscope}%
\begin{pgfscope}%
\pgfpathrectangle{\pgfqpoint{9.810417in}{5.660465in}}{\pgfqpoint{5.489583in}{0.877907in}}%
\pgfusepath{clip}%
\pgfsetroundcap%
\pgfsetroundjoin%
\pgfsetlinewidth{0.803000pt}%
\definecolor{currentstroke}{rgb}{1.000000,1.000000,1.000000}%
\pgfsetstrokecolor{currentstroke}%
\pgfsetdash{}{0pt}%
\pgfpathmoveto{\pgfqpoint{10.676058in}{5.660465in}}%
\pgfpathlineto{\pgfqpoint{10.676058in}{6.538372in}}%
\pgfusepath{stroke}%
\end{pgfscope}%
\begin{pgfscope}%
\definecolor{textcolor}{rgb}{0.150000,0.150000,0.150000}%
\pgfsetstrokecolor{textcolor}%
\pgfsetfillcolor{textcolor}%
\pgftext[x=10.676058in,y=5.563243in,,top]{\color{textcolor}\rmfamily\fontsize{14.000000}{16.800000}\selectfont 5}%
\end{pgfscope}%
\begin{pgfscope}%
\pgfpathrectangle{\pgfqpoint{9.810417in}{5.660465in}}{\pgfqpoint{5.489583in}{0.877907in}}%
\pgfusepath{clip}%
\pgfsetroundcap%
\pgfsetroundjoin%
\pgfsetlinewidth{0.803000pt}%
\definecolor{currentstroke}{rgb}{1.000000,1.000000,1.000000}%
\pgfsetstrokecolor{currentstroke}%
\pgfsetdash{}{0pt}%
\pgfpathmoveto{\pgfqpoint{11.292173in}{5.660465in}}%
\pgfpathlineto{\pgfqpoint{11.292173in}{6.538372in}}%
\pgfusepath{stroke}%
\end{pgfscope}%
\begin{pgfscope}%
\definecolor{textcolor}{rgb}{0.150000,0.150000,0.150000}%
\pgfsetstrokecolor{textcolor}%
\pgfsetfillcolor{textcolor}%
\pgftext[x=11.292173in,y=5.563243in,,top]{\color{textcolor}\rmfamily\fontsize{14.000000}{16.800000}\selectfont 10}%
\end{pgfscope}%
\begin{pgfscope}%
\pgfpathrectangle{\pgfqpoint{9.810417in}{5.660465in}}{\pgfqpoint{5.489583in}{0.877907in}}%
\pgfusepath{clip}%
\pgfsetroundcap%
\pgfsetroundjoin%
\pgfsetlinewidth{0.803000pt}%
\definecolor{currentstroke}{rgb}{1.000000,1.000000,1.000000}%
\pgfsetstrokecolor{currentstroke}%
\pgfsetdash{}{0pt}%
\pgfpathmoveto{\pgfqpoint{11.908288in}{5.660465in}}%
\pgfpathlineto{\pgfqpoint{11.908288in}{6.538372in}}%
\pgfusepath{stroke}%
\end{pgfscope}%
\begin{pgfscope}%
\definecolor{textcolor}{rgb}{0.150000,0.150000,0.150000}%
\pgfsetstrokecolor{textcolor}%
\pgfsetfillcolor{textcolor}%
\pgftext[x=11.908288in,y=5.563243in,,top]{\color{textcolor}\rmfamily\fontsize{14.000000}{16.800000}\selectfont 15}%
\end{pgfscope}%
\begin{pgfscope}%
\pgfpathrectangle{\pgfqpoint{9.810417in}{5.660465in}}{\pgfqpoint{5.489583in}{0.877907in}}%
\pgfusepath{clip}%
\pgfsetroundcap%
\pgfsetroundjoin%
\pgfsetlinewidth{0.803000pt}%
\definecolor{currentstroke}{rgb}{1.000000,1.000000,1.000000}%
\pgfsetstrokecolor{currentstroke}%
\pgfsetdash{}{0pt}%
\pgfpathmoveto{\pgfqpoint{12.524403in}{5.660465in}}%
\pgfpathlineto{\pgfqpoint{12.524403in}{6.538372in}}%
\pgfusepath{stroke}%
\end{pgfscope}%
\begin{pgfscope}%
\definecolor{textcolor}{rgb}{0.150000,0.150000,0.150000}%
\pgfsetstrokecolor{textcolor}%
\pgfsetfillcolor{textcolor}%
\pgftext[x=12.524403in,y=5.563243in,,top]{\color{textcolor}\rmfamily\fontsize{14.000000}{16.800000}\selectfont 20}%
\end{pgfscope}%
\begin{pgfscope}%
\pgfpathrectangle{\pgfqpoint{9.810417in}{5.660465in}}{\pgfqpoint{5.489583in}{0.877907in}}%
\pgfusepath{clip}%
\pgfsetroundcap%
\pgfsetroundjoin%
\pgfsetlinewidth{0.803000pt}%
\definecolor{currentstroke}{rgb}{1.000000,1.000000,1.000000}%
\pgfsetstrokecolor{currentstroke}%
\pgfsetdash{}{0pt}%
\pgfpathmoveto{\pgfqpoint{13.140517in}{5.660465in}}%
\pgfpathlineto{\pgfqpoint{13.140517in}{6.538372in}}%
\pgfusepath{stroke}%
\end{pgfscope}%
\begin{pgfscope}%
\definecolor{textcolor}{rgb}{0.150000,0.150000,0.150000}%
\pgfsetstrokecolor{textcolor}%
\pgfsetfillcolor{textcolor}%
\pgftext[x=13.140517in,y=5.563243in,,top]{\color{textcolor}\rmfamily\fontsize{14.000000}{16.800000}\selectfont 25}%
\end{pgfscope}%
\begin{pgfscope}%
\pgfpathrectangle{\pgfqpoint{9.810417in}{5.660465in}}{\pgfqpoint{5.489583in}{0.877907in}}%
\pgfusepath{clip}%
\pgfsetroundcap%
\pgfsetroundjoin%
\pgfsetlinewidth{0.803000pt}%
\definecolor{currentstroke}{rgb}{1.000000,1.000000,1.000000}%
\pgfsetstrokecolor{currentstroke}%
\pgfsetdash{}{0pt}%
\pgfpathmoveto{\pgfqpoint{13.756632in}{5.660465in}}%
\pgfpathlineto{\pgfqpoint{13.756632in}{6.538372in}}%
\pgfusepath{stroke}%
\end{pgfscope}%
\begin{pgfscope}%
\definecolor{textcolor}{rgb}{0.150000,0.150000,0.150000}%
\pgfsetstrokecolor{textcolor}%
\pgfsetfillcolor{textcolor}%
\pgftext[x=13.756632in,y=5.563243in,,top]{\color{textcolor}\rmfamily\fontsize{14.000000}{16.800000}\selectfont 30}%
\end{pgfscope}%
\begin{pgfscope}%
\pgfpathrectangle{\pgfqpoint{9.810417in}{5.660465in}}{\pgfqpoint{5.489583in}{0.877907in}}%
\pgfusepath{clip}%
\pgfsetroundcap%
\pgfsetroundjoin%
\pgfsetlinewidth{0.803000pt}%
\definecolor{currentstroke}{rgb}{1.000000,1.000000,1.000000}%
\pgfsetstrokecolor{currentstroke}%
\pgfsetdash{}{0pt}%
\pgfpathmoveto{\pgfqpoint{14.372747in}{5.660465in}}%
\pgfpathlineto{\pgfqpoint{14.372747in}{6.538372in}}%
\pgfusepath{stroke}%
\end{pgfscope}%
\begin{pgfscope}%
\definecolor{textcolor}{rgb}{0.150000,0.150000,0.150000}%
\pgfsetstrokecolor{textcolor}%
\pgfsetfillcolor{textcolor}%
\pgftext[x=14.372747in,y=5.563243in,,top]{\color{textcolor}\rmfamily\fontsize{14.000000}{16.800000}\selectfont 35}%
\end{pgfscope}%
\begin{pgfscope}%
\pgfpathrectangle{\pgfqpoint{9.810417in}{5.660465in}}{\pgfqpoint{5.489583in}{0.877907in}}%
\pgfusepath{clip}%
\pgfsetroundcap%
\pgfsetroundjoin%
\pgfsetlinewidth{0.803000pt}%
\definecolor{currentstroke}{rgb}{1.000000,1.000000,1.000000}%
\pgfsetstrokecolor{currentstroke}%
\pgfsetdash{}{0pt}%
\pgfpathmoveto{\pgfqpoint{14.988862in}{5.660465in}}%
\pgfpathlineto{\pgfqpoint{14.988862in}{6.538372in}}%
\pgfusepath{stroke}%
\end{pgfscope}%
\begin{pgfscope}%
\definecolor{textcolor}{rgb}{0.150000,0.150000,0.150000}%
\pgfsetstrokecolor{textcolor}%
\pgfsetfillcolor{textcolor}%
\pgftext[x=14.988862in,y=5.563243in,,top]{\color{textcolor}\rmfamily\fontsize{14.000000}{16.800000}\selectfont 40}%
\end{pgfscope}%
\begin{pgfscope}%
\pgfpathrectangle{\pgfqpoint{9.810417in}{5.660465in}}{\pgfqpoint{5.489583in}{0.877907in}}%
\pgfusepath{clip}%
\pgfsetroundcap%
\pgfsetroundjoin%
\pgfsetlinewidth{0.803000pt}%
\definecolor{currentstroke}{rgb}{1.000000,1.000000,1.000000}%
\pgfsetstrokecolor{currentstroke}%
\pgfsetdash{}{0pt}%
\pgfpathmoveto{\pgfqpoint{9.810417in}{5.738704in}}%
\pgfpathlineto{\pgfqpoint{15.300000in}{5.738704in}}%
\pgfusepath{stroke}%
\end{pgfscope}%
\begin{pgfscope}%
\definecolor{textcolor}{rgb}{0.150000,0.150000,0.150000}%
\pgfsetstrokecolor{textcolor}%
\pgfsetfillcolor{textcolor}%
\pgftext[x=9.589483in,y=5.664838in,left,base]{\color{textcolor}\rmfamily\fontsize{14.000000}{16.800000}\selectfont 0}%
\end{pgfscope}%
\begin{pgfscope}%
\pgfpathrectangle{\pgfqpoint{9.810417in}{5.660465in}}{\pgfqpoint{5.489583in}{0.877907in}}%
\pgfusepath{clip}%
\pgfsetroundcap%
\pgfsetroundjoin%
\pgfsetlinewidth{0.803000pt}%
\definecolor{currentstroke}{rgb}{1.000000,1.000000,1.000000}%
\pgfsetstrokecolor{currentstroke}%
\pgfsetdash{}{0pt}%
\pgfpathmoveto{\pgfqpoint{9.810417in}{6.498467in}}%
\pgfpathlineto{\pgfqpoint{15.300000in}{6.498467in}}%
\pgfusepath{stroke}%
\end{pgfscope}%
\begin{pgfscope}%
\definecolor{textcolor}{rgb}{0.150000,0.150000,0.150000}%
\pgfsetstrokecolor{textcolor}%
\pgfsetfillcolor{textcolor}%
\pgftext[x=9.589483in,y=6.424601in,left,base]{\color{textcolor}\rmfamily\fontsize{14.000000}{16.800000}\selectfont 1}%
\end{pgfscope}%
\begin{pgfscope}%
\pgfpathrectangle{\pgfqpoint{9.810417in}{5.660465in}}{\pgfqpoint{5.489583in}{0.877907in}}%
\pgfusepath{clip}%
\pgfsetbuttcap%
\pgfsetroundjoin%
\definecolor{currentfill}{rgb}{0.121569,0.466667,0.705882}%
\pgfsetfillcolor{currentfill}%
\pgfsetfillopacity{0.250000}%
\pgfsetlinewidth{1.003750pt}%
\definecolor{currentstroke}{rgb}{1.000000,1.000000,1.000000}%
\pgfsetstrokecolor{currentstroke}%
\pgfsetstrokeopacity{0.250000}%
\pgfsetdash{}{0pt}%
\pgfpathmoveto{\pgfqpoint{10.121555in}{5.777038in}}%
\pgfpathlineto{\pgfqpoint{10.121555in}{5.700370in}}%
\pgfpathlineto{\pgfqpoint{10.306389in}{5.700370in}}%
\pgfpathlineto{\pgfqpoint{10.429612in}{5.700370in}}%
\pgfpathlineto{\pgfqpoint{10.552835in}{5.700370in}}%
\pgfpathlineto{\pgfqpoint{10.676058in}{5.700370in}}%
\pgfpathlineto{\pgfqpoint{10.799281in}{5.700370in}}%
\pgfpathlineto{\pgfqpoint{10.922504in}{5.700370in}}%
\pgfpathlineto{\pgfqpoint{11.045727in}{5.700370in}}%
\pgfpathlineto{\pgfqpoint{11.168950in}{5.700370in}}%
\pgfpathlineto{\pgfqpoint{11.292173in}{5.700370in}}%
\pgfpathlineto{\pgfqpoint{11.415396in}{5.700370in}}%
\pgfpathlineto{\pgfqpoint{11.538619in}{5.700370in}}%
\pgfpathlineto{\pgfqpoint{11.661842in}{5.700370in}}%
\pgfpathlineto{\pgfqpoint{11.785065in}{5.700370in}}%
\pgfpathlineto{\pgfqpoint{11.908288in}{5.700370in}}%
\pgfpathlineto{\pgfqpoint{12.031511in}{5.700370in}}%
\pgfpathlineto{\pgfqpoint{12.154734in}{5.700370in}}%
\pgfpathlineto{\pgfqpoint{12.277957in}{5.700370in}}%
\pgfpathlineto{\pgfqpoint{12.401180in}{5.700370in}}%
\pgfpathlineto{\pgfqpoint{12.524403in}{5.700370in}}%
\pgfpathlineto{\pgfqpoint{12.647626in}{5.700370in}}%
\pgfpathlineto{\pgfqpoint{12.770849in}{5.700370in}}%
\pgfpathlineto{\pgfqpoint{12.894072in}{5.700370in}}%
\pgfpathlineto{\pgfqpoint{13.017294in}{5.700370in}}%
\pgfpathlineto{\pgfqpoint{13.140517in}{5.700370in}}%
\pgfpathlineto{\pgfqpoint{13.263740in}{5.700370in}}%
\pgfpathlineto{\pgfqpoint{13.386963in}{5.700370in}}%
\pgfpathlineto{\pgfqpoint{13.510186in}{5.700370in}}%
\pgfpathlineto{\pgfqpoint{13.633409in}{5.700370in}}%
\pgfpathlineto{\pgfqpoint{13.756632in}{5.700370in}}%
\pgfpathlineto{\pgfqpoint{13.879855in}{5.700370in}}%
\pgfpathlineto{\pgfqpoint{14.003078in}{5.700370in}}%
\pgfpathlineto{\pgfqpoint{14.126301in}{5.700370in}}%
\pgfpathlineto{\pgfqpoint{14.249524in}{5.700370in}}%
\pgfpathlineto{\pgfqpoint{14.372747in}{5.700370in}}%
\pgfpathlineto{\pgfqpoint{14.495970in}{5.700370in}}%
\pgfpathlineto{\pgfqpoint{14.619193in}{5.700370in}}%
\pgfpathlineto{\pgfqpoint{14.742416in}{5.700370in}}%
\pgfpathlineto{\pgfqpoint{14.865639in}{5.700370in}}%
\pgfpathlineto{\pgfqpoint{15.050473in}{5.700370in}}%
\pgfpathlineto{\pgfqpoint{15.050473in}{5.777038in}}%
\pgfpathlineto{\pgfqpoint{15.050473in}{5.777038in}}%
\pgfpathlineto{\pgfqpoint{14.865639in}{5.777038in}}%
\pgfpathlineto{\pgfqpoint{14.742416in}{5.777038in}}%
\pgfpathlineto{\pgfqpoint{14.619193in}{5.777038in}}%
\pgfpathlineto{\pgfqpoint{14.495970in}{5.777038in}}%
\pgfpathlineto{\pgfqpoint{14.372747in}{5.777038in}}%
\pgfpathlineto{\pgfqpoint{14.249524in}{5.777038in}}%
\pgfpathlineto{\pgfqpoint{14.126301in}{5.777038in}}%
\pgfpathlineto{\pgfqpoint{14.003078in}{5.777038in}}%
\pgfpathlineto{\pgfqpoint{13.879855in}{5.777038in}}%
\pgfpathlineto{\pgfqpoint{13.756632in}{5.777038in}}%
\pgfpathlineto{\pgfqpoint{13.633409in}{5.777038in}}%
\pgfpathlineto{\pgfqpoint{13.510186in}{5.777038in}}%
\pgfpathlineto{\pgfqpoint{13.386963in}{5.777038in}}%
\pgfpathlineto{\pgfqpoint{13.263740in}{5.777038in}}%
\pgfpathlineto{\pgfqpoint{13.140517in}{5.777038in}}%
\pgfpathlineto{\pgfqpoint{13.017294in}{5.777038in}}%
\pgfpathlineto{\pgfqpoint{12.894072in}{5.777038in}}%
\pgfpathlineto{\pgfqpoint{12.770849in}{5.777038in}}%
\pgfpathlineto{\pgfqpoint{12.647626in}{5.777038in}}%
\pgfpathlineto{\pgfqpoint{12.524403in}{5.777038in}}%
\pgfpathlineto{\pgfqpoint{12.401180in}{5.777038in}}%
\pgfpathlineto{\pgfqpoint{12.277957in}{5.777038in}}%
\pgfpathlineto{\pgfqpoint{12.154734in}{5.777038in}}%
\pgfpathlineto{\pgfqpoint{12.031511in}{5.777038in}}%
\pgfpathlineto{\pgfqpoint{11.908288in}{5.777038in}}%
\pgfpathlineto{\pgfqpoint{11.785065in}{5.777038in}}%
\pgfpathlineto{\pgfqpoint{11.661842in}{5.777038in}}%
\pgfpathlineto{\pgfqpoint{11.538619in}{5.777038in}}%
\pgfpathlineto{\pgfqpoint{11.415396in}{5.777038in}}%
\pgfpathlineto{\pgfqpoint{11.292173in}{5.777038in}}%
\pgfpathlineto{\pgfqpoint{11.168950in}{5.777038in}}%
\pgfpathlineto{\pgfqpoint{11.045727in}{5.777038in}}%
\pgfpathlineto{\pgfqpoint{10.922504in}{5.777038in}}%
\pgfpathlineto{\pgfqpoint{10.799281in}{5.777038in}}%
\pgfpathlineto{\pgfqpoint{10.676058in}{5.777038in}}%
\pgfpathlineto{\pgfqpoint{10.552835in}{5.777038in}}%
\pgfpathlineto{\pgfqpoint{10.429612in}{5.777038in}}%
\pgfpathlineto{\pgfqpoint{10.306389in}{5.777038in}}%
\pgfpathlineto{\pgfqpoint{10.121555in}{5.777038in}}%
\pgfpathclose%
\pgfusepath{stroke,fill}%
\end{pgfscope}%
\begin{pgfscope}%
\pgfpathrectangle{\pgfqpoint{9.810417in}{5.660465in}}{\pgfqpoint{5.489583in}{0.877907in}}%
\pgfusepath{clip}%
\pgfsetbuttcap%
\pgfsetroundjoin%
\pgfsetlinewidth{1.505625pt}%
\definecolor{currentstroke}{rgb}{0.000000,0.000000,0.000000}%
\pgfsetstrokecolor{currentstroke}%
\pgfsetdash{}{0pt}%
\pgfpathmoveto{\pgfqpoint{10.059943in}{5.738704in}}%
\pgfpathlineto{\pgfqpoint{10.059943in}{6.498467in}}%
\pgfusepath{stroke}%
\end{pgfscope}%
\begin{pgfscope}%
\pgfpathrectangle{\pgfqpoint{9.810417in}{5.660465in}}{\pgfqpoint{5.489583in}{0.877907in}}%
\pgfusepath{clip}%
\pgfsetbuttcap%
\pgfsetroundjoin%
\pgfsetlinewidth{1.505625pt}%
\definecolor{currentstroke}{rgb}{0.000000,0.000000,0.000000}%
\pgfsetstrokecolor{currentstroke}%
\pgfsetdash{}{0pt}%
\pgfpathmoveto{\pgfqpoint{10.183166in}{5.738704in}}%
\pgfpathlineto{\pgfqpoint{10.183166in}{6.494818in}}%
\pgfusepath{stroke}%
\end{pgfscope}%
\begin{pgfscope}%
\pgfpathrectangle{\pgfqpoint{9.810417in}{5.660465in}}{\pgfqpoint{5.489583in}{0.877907in}}%
\pgfusepath{clip}%
\pgfsetbuttcap%
\pgfsetroundjoin%
\pgfsetlinewidth{1.505625pt}%
\definecolor{currentstroke}{rgb}{0.000000,0.000000,0.000000}%
\pgfsetstrokecolor{currentstroke}%
\pgfsetdash{}{0pt}%
\pgfpathmoveto{\pgfqpoint{10.306389in}{5.738704in}}%
\pgfpathlineto{\pgfqpoint{10.306389in}{5.704530in}}%
\pgfusepath{stroke}%
\end{pgfscope}%
\begin{pgfscope}%
\pgfpathrectangle{\pgfqpoint{9.810417in}{5.660465in}}{\pgfqpoint{5.489583in}{0.877907in}}%
\pgfusepath{clip}%
\pgfsetbuttcap%
\pgfsetroundjoin%
\pgfsetlinewidth{1.505625pt}%
\definecolor{currentstroke}{rgb}{0.000000,0.000000,0.000000}%
\pgfsetstrokecolor{currentstroke}%
\pgfsetdash{}{0pt}%
\pgfpathmoveto{\pgfqpoint{10.429612in}{5.738704in}}%
\pgfpathlineto{\pgfqpoint{10.429612in}{5.726947in}}%
\pgfusepath{stroke}%
\end{pgfscope}%
\begin{pgfscope}%
\pgfpathrectangle{\pgfqpoint{9.810417in}{5.660465in}}{\pgfqpoint{5.489583in}{0.877907in}}%
\pgfusepath{clip}%
\pgfsetbuttcap%
\pgfsetroundjoin%
\pgfsetlinewidth{1.505625pt}%
\definecolor{currentstroke}{rgb}{0.000000,0.000000,0.000000}%
\pgfsetstrokecolor{currentstroke}%
\pgfsetdash{}{0pt}%
\pgfpathmoveto{\pgfqpoint{10.552835in}{5.738704in}}%
\pgfpathlineto{\pgfqpoint{10.552835in}{5.731162in}}%
\pgfusepath{stroke}%
\end{pgfscope}%
\begin{pgfscope}%
\pgfpathrectangle{\pgfqpoint{9.810417in}{5.660465in}}{\pgfqpoint{5.489583in}{0.877907in}}%
\pgfusepath{clip}%
\pgfsetbuttcap%
\pgfsetroundjoin%
\pgfsetlinewidth{1.505625pt}%
\definecolor{currentstroke}{rgb}{0.000000,0.000000,0.000000}%
\pgfsetstrokecolor{currentstroke}%
\pgfsetdash{}{0pt}%
\pgfpathmoveto{\pgfqpoint{10.676058in}{5.738704in}}%
\pgfpathlineto{\pgfqpoint{10.676058in}{5.752214in}}%
\pgfusepath{stroke}%
\end{pgfscope}%
\begin{pgfscope}%
\pgfpathrectangle{\pgfqpoint{9.810417in}{5.660465in}}{\pgfqpoint{5.489583in}{0.877907in}}%
\pgfusepath{clip}%
\pgfsetbuttcap%
\pgfsetroundjoin%
\pgfsetlinewidth{1.505625pt}%
\definecolor{currentstroke}{rgb}{0.000000,0.000000,0.000000}%
\pgfsetstrokecolor{currentstroke}%
\pgfsetdash{}{0pt}%
\pgfpathmoveto{\pgfqpoint{10.799281in}{5.738704in}}%
\pgfpathlineto{\pgfqpoint{10.799281in}{5.718116in}}%
\pgfusepath{stroke}%
\end{pgfscope}%
\begin{pgfscope}%
\pgfpathrectangle{\pgfqpoint{9.810417in}{5.660465in}}{\pgfqpoint{5.489583in}{0.877907in}}%
\pgfusepath{clip}%
\pgfsetbuttcap%
\pgfsetroundjoin%
\pgfsetlinewidth{1.505625pt}%
\definecolor{currentstroke}{rgb}{0.000000,0.000000,0.000000}%
\pgfsetstrokecolor{currentstroke}%
\pgfsetdash{}{0pt}%
\pgfpathmoveto{\pgfqpoint{10.922504in}{5.738704in}}%
\pgfpathlineto{\pgfqpoint{10.922504in}{5.730071in}}%
\pgfusepath{stroke}%
\end{pgfscope}%
\begin{pgfscope}%
\pgfpathrectangle{\pgfqpoint{9.810417in}{5.660465in}}{\pgfqpoint{5.489583in}{0.877907in}}%
\pgfusepath{clip}%
\pgfsetbuttcap%
\pgfsetroundjoin%
\pgfsetlinewidth{1.505625pt}%
\definecolor{currentstroke}{rgb}{0.000000,0.000000,0.000000}%
\pgfsetstrokecolor{currentstroke}%
\pgfsetdash{}{0pt}%
\pgfpathmoveto{\pgfqpoint{11.045727in}{5.738704in}}%
\pgfpathlineto{\pgfqpoint{11.045727in}{5.738515in}}%
\pgfusepath{stroke}%
\end{pgfscope}%
\begin{pgfscope}%
\pgfpathrectangle{\pgfqpoint{9.810417in}{5.660465in}}{\pgfqpoint{5.489583in}{0.877907in}}%
\pgfusepath{clip}%
\pgfsetbuttcap%
\pgfsetroundjoin%
\pgfsetlinewidth{1.505625pt}%
\definecolor{currentstroke}{rgb}{0.000000,0.000000,0.000000}%
\pgfsetstrokecolor{currentstroke}%
\pgfsetdash{}{0pt}%
\pgfpathmoveto{\pgfqpoint{11.168950in}{5.738704in}}%
\pgfpathlineto{\pgfqpoint{11.168950in}{5.737618in}}%
\pgfusepath{stroke}%
\end{pgfscope}%
\begin{pgfscope}%
\pgfpathrectangle{\pgfqpoint{9.810417in}{5.660465in}}{\pgfqpoint{5.489583in}{0.877907in}}%
\pgfusepath{clip}%
\pgfsetbuttcap%
\pgfsetroundjoin%
\pgfsetlinewidth{1.505625pt}%
\definecolor{currentstroke}{rgb}{0.000000,0.000000,0.000000}%
\pgfsetstrokecolor{currentstroke}%
\pgfsetdash{}{0pt}%
\pgfpathmoveto{\pgfqpoint{11.292173in}{5.738704in}}%
\pgfpathlineto{\pgfqpoint{11.292173in}{5.766994in}}%
\pgfusepath{stroke}%
\end{pgfscope}%
\begin{pgfscope}%
\pgfpathrectangle{\pgfqpoint{9.810417in}{5.660465in}}{\pgfqpoint{5.489583in}{0.877907in}}%
\pgfusepath{clip}%
\pgfsetbuttcap%
\pgfsetroundjoin%
\pgfsetlinewidth{1.505625pt}%
\definecolor{currentstroke}{rgb}{0.000000,0.000000,0.000000}%
\pgfsetstrokecolor{currentstroke}%
\pgfsetdash{}{0pt}%
\pgfpathmoveto{\pgfqpoint{11.415396in}{5.738704in}}%
\pgfpathlineto{\pgfqpoint{11.415396in}{5.742105in}}%
\pgfusepath{stroke}%
\end{pgfscope}%
\begin{pgfscope}%
\pgfpathrectangle{\pgfqpoint{9.810417in}{5.660465in}}{\pgfqpoint{5.489583in}{0.877907in}}%
\pgfusepath{clip}%
\pgfsetbuttcap%
\pgfsetroundjoin%
\pgfsetlinewidth{1.505625pt}%
\definecolor{currentstroke}{rgb}{0.000000,0.000000,0.000000}%
\pgfsetstrokecolor{currentstroke}%
\pgfsetdash{}{0pt}%
\pgfpathmoveto{\pgfqpoint{11.538619in}{5.738704in}}%
\pgfpathlineto{\pgfqpoint{11.538619in}{5.720343in}}%
\pgfusepath{stroke}%
\end{pgfscope}%
\begin{pgfscope}%
\pgfpathrectangle{\pgfqpoint{9.810417in}{5.660465in}}{\pgfqpoint{5.489583in}{0.877907in}}%
\pgfusepath{clip}%
\pgfsetbuttcap%
\pgfsetroundjoin%
\pgfsetlinewidth{1.505625pt}%
\definecolor{currentstroke}{rgb}{0.000000,0.000000,0.000000}%
\pgfsetstrokecolor{currentstroke}%
\pgfsetdash{}{0pt}%
\pgfpathmoveto{\pgfqpoint{11.661842in}{5.738704in}}%
\pgfpathlineto{\pgfqpoint{11.661842in}{5.743483in}}%
\pgfusepath{stroke}%
\end{pgfscope}%
\begin{pgfscope}%
\pgfpathrectangle{\pgfqpoint{9.810417in}{5.660465in}}{\pgfqpoint{5.489583in}{0.877907in}}%
\pgfusepath{clip}%
\pgfsetbuttcap%
\pgfsetroundjoin%
\pgfsetlinewidth{1.505625pt}%
\definecolor{currentstroke}{rgb}{0.000000,0.000000,0.000000}%
\pgfsetstrokecolor{currentstroke}%
\pgfsetdash{}{0pt}%
\pgfpathmoveto{\pgfqpoint{11.785065in}{5.738704in}}%
\pgfpathlineto{\pgfqpoint{11.785065in}{5.716537in}}%
\pgfusepath{stroke}%
\end{pgfscope}%
\begin{pgfscope}%
\pgfpathrectangle{\pgfqpoint{9.810417in}{5.660465in}}{\pgfqpoint{5.489583in}{0.877907in}}%
\pgfusepath{clip}%
\pgfsetbuttcap%
\pgfsetroundjoin%
\pgfsetlinewidth{1.505625pt}%
\definecolor{currentstroke}{rgb}{0.000000,0.000000,0.000000}%
\pgfsetstrokecolor{currentstroke}%
\pgfsetdash{}{0pt}%
\pgfpathmoveto{\pgfqpoint{11.908288in}{5.738704in}}%
\pgfpathlineto{\pgfqpoint{11.908288in}{5.742957in}}%
\pgfusepath{stroke}%
\end{pgfscope}%
\begin{pgfscope}%
\pgfpathrectangle{\pgfqpoint{9.810417in}{5.660465in}}{\pgfqpoint{5.489583in}{0.877907in}}%
\pgfusepath{clip}%
\pgfsetbuttcap%
\pgfsetroundjoin%
\pgfsetlinewidth{1.505625pt}%
\definecolor{currentstroke}{rgb}{0.000000,0.000000,0.000000}%
\pgfsetstrokecolor{currentstroke}%
\pgfsetdash{}{0pt}%
\pgfpathmoveto{\pgfqpoint{12.031511in}{5.738704in}}%
\pgfpathlineto{\pgfqpoint{12.031511in}{5.748041in}}%
\pgfusepath{stroke}%
\end{pgfscope}%
\begin{pgfscope}%
\pgfpathrectangle{\pgfqpoint{9.810417in}{5.660465in}}{\pgfqpoint{5.489583in}{0.877907in}}%
\pgfusepath{clip}%
\pgfsetbuttcap%
\pgfsetroundjoin%
\pgfsetlinewidth{1.505625pt}%
\definecolor{currentstroke}{rgb}{0.000000,0.000000,0.000000}%
\pgfsetstrokecolor{currentstroke}%
\pgfsetdash{}{0pt}%
\pgfpathmoveto{\pgfqpoint{12.154734in}{5.738704in}}%
\pgfpathlineto{\pgfqpoint{12.154734in}{5.722072in}}%
\pgfusepath{stroke}%
\end{pgfscope}%
\begin{pgfscope}%
\pgfpathrectangle{\pgfqpoint{9.810417in}{5.660465in}}{\pgfqpoint{5.489583in}{0.877907in}}%
\pgfusepath{clip}%
\pgfsetbuttcap%
\pgfsetroundjoin%
\pgfsetlinewidth{1.505625pt}%
\definecolor{currentstroke}{rgb}{0.000000,0.000000,0.000000}%
\pgfsetstrokecolor{currentstroke}%
\pgfsetdash{}{0pt}%
\pgfpathmoveto{\pgfqpoint{12.277957in}{5.738704in}}%
\pgfpathlineto{\pgfqpoint{12.277957in}{5.733855in}}%
\pgfusepath{stroke}%
\end{pgfscope}%
\begin{pgfscope}%
\pgfpathrectangle{\pgfqpoint{9.810417in}{5.660465in}}{\pgfqpoint{5.489583in}{0.877907in}}%
\pgfusepath{clip}%
\pgfsetbuttcap%
\pgfsetroundjoin%
\pgfsetlinewidth{1.505625pt}%
\definecolor{currentstroke}{rgb}{0.000000,0.000000,0.000000}%
\pgfsetstrokecolor{currentstroke}%
\pgfsetdash{}{0pt}%
\pgfpathmoveto{\pgfqpoint{12.401180in}{5.738704in}}%
\pgfpathlineto{\pgfqpoint{12.401180in}{5.740818in}}%
\pgfusepath{stroke}%
\end{pgfscope}%
\begin{pgfscope}%
\pgfpathrectangle{\pgfqpoint{9.810417in}{5.660465in}}{\pgfqpoint{5.489583in}{0.877907in}}%
\pgfusepath{clip}%
\pgfsetbuttcap%
\pgfsetroundjoin%
\pgfsetlinewidth{1.505625pt}%
\definecolor{currentstroke}{rgb}{0.000000,0.000000,0.000000}%
\pgfsetstrokecolor{currentstroke}%
\pgfsetdash{}{0pt}%
\pgfpathmoveto{\pgfqpoint{12.524403in}{5.738704in}}%
\pgfpathlineto{\pgfqpoint{12.524403in}{5.755199in}}%
\pgfusepath{stroke}%
\end{pgfscope}%
\begin{pgfscope}%
\pgfpathrectangle{\pgfqpoint{9.810417in}{5.660465in}}{\pgfqpoint{5.489583in}{0.877907in}}%
\pgfusepath{clip}%
\pgfsetbuttcap%
\pgfsetroundjoin%
\pgfsetlinewidth{1.505625pt}%
\definecolor{currentstroke}{rgb}{0.000000,0.000000,0.000000}%
\pgfsetstrokecolor{currentstroke}%
\pgfsetdash{}{0pt}%
\pgfpathmoveto{\pgfqpoint{12.647626in}{5.738704in}}%
\pgfpathlineto{\pgfqpoint{12.647626in}{5.731481in}}%
\pgfusepath{stroke}%
\end{pgfscope}%
\begin{pgfscope}%
\pgfpathrectangle{\pgfqpoint{9.810417in}{5.660465in}}{\pgfqpoint{5.489583in}{0.877907in}}%
\pgfusepath{clip}%
\pgfsetbuttcap%
\pgfsetroundjoin%
\pgfsetlinewidth{1.505625pt}%
\definecolor{currentstroke}{rgb}{0.000000,0.000000,0.000000}%
\pgfsetstrokecolor{currentstroke}%
\pgfsetdash{}{0pt}%
\pgfpathmoveto{\pgfqpoint{12.770849in}{5.738704in}}%
\pgfpathlineto{\pgfqpoint{12.770849in}{5.758217in}}%
\pgfusepath{stroke}%
\end{pgfscope}%
\begin{pgfscope}%
\pgfpathrectangle{\pgfqpoint{9.810417in}{5.660465in}}{\pgfqpoint{5.489583in}{0.877907in}}%
\pgfusepath{clip}%
\pgfsetbuttcap%
\pgfsetroundjoin%
\pgfsetlinewidth{1.505625pt}%
\definecolor{currentstroke}{rgb}{0.000000,0.000000,0.000000}%
\pgfsetstrokecolor{currentstroke}%
\pgfsetdash{}{0pt}%
\pgfpathmoveto{\pgfqpoint{12.894072in}{5.738704in}}%
\pgfpathlineto{\pgfqpoint{12.894072in}{5.768832in}}%
\pgfusepath{stroke}%
\end{pgfscope}%
\begin{pgfscope}%
\pgfpathrectangle{\pgfqpoint{9.810417in}{5.660465in}}{\pgfqpoint{5.489583in}{0.877907in}}%
\pgfusepath{clip}%
\pgfsetbuttcap%
\pgfsetroundjoin%
\pgfsetlinewidth{1.505625pt}%
\definecolor{currentstroke}{rgb}{0.000000,0.000000,0.000000}%
\pgfsetstrokecolor{currentstroke}%
\pgfsetdash{}{0pt}%
\pgfpathmoveto{\pgfqpoint{13.017294in}{5.738704in}}%
\pgfpathlineto{\pgfqpoint{13.017294in}{5.747465in}}%
\pgfusepath{stroke}%
\end{pgfscope}%
\begin{pgfscope}%
\pgfpathrectangle{\pgfqpoint{9.810417in}{5.660465in}}{\pgfqpoint{5.489583in}{0.877907in}}%
\pgfusepath{clip}%
\pgfsetbuttcap%
\pgfsetroundjoin%
\pgfsetlinewidth{1.505625pt}%
\definecolor{currentstroke}{rgb}{0.000000,0.000000,0.000000}%
\pgfsetstrokecolor{currentstroke}%
\pgfsetdash{}{0pt}%
\pgfpathmoveto{\pgfqpoint{13.140517in}{5.738704in}}%
\pgfpathlineto{\pgfqpoint{13.140517in}{5.748047in}}%
\pgfusepath{stroke}%
\end{pgfscope}%
\begin{pgfscope}%
\pgfpathrectangle{\pgfqpoint{9.810417in}{5.660465in}}{\pgfqpoint{5.489583in}{0.877907in}}%
\pgfusepath{clip}%
\pgfsetbuttcap%
\pgfsetroundjoin%
\pgfsetlinewidth{1.505625pt}%
\definecolor{currentstroke}{rgb}{0.000000,0.000000,0.000000}%
\pgfsetstrokecolor{currentstroke}%
\pgfsetdash{}{0pt}%
\pgfpathmoveto{\pgfqpoint{13.263740in}{5.738704in}}%
\pgfpathlineto{\pgfqpoint{13.263740in}{5.762313in}}%
\pgfusepath{stroke}%
\end{pgfscope}%
\begin{pgfscope}%
\pgfpathrectangle{\pgfqpoint{9.810417in}{5.660465in}}{\pgfqpoint{5.489583in}{0.877907in}}%
\pgfusepath{clip}%
\pgfsetbuttcap%
\pgfsetroundjoin%
\pgfsetlinewidth{1.505625pt}%
\definecolor{currentstroke}{rgb}{0.000000,0.000000,0.000000}%
\pgfsetstrokecolor{currentstroke}%
\pgfsetdash{}{0pt}%
\pgfpathmoveto{\pgfqpoint{13.386963in}{5.738704in}}%
\pgfpathlineto{\pgfqpoint{13.386963in}{5.761192in}}%
\pgfusepath{stroke}%
\end{pgfscope}%
\begin{pgfscope}%
\pgfpathrectangle{\pgfqpoint{9.810417in}{5.660465in}}{\pgfqpoint{5.489583in}{0.877907in}}%
\pgfusepath{clip}%
\pgfsetbuttcap%
\pgfsetroundjoin%
\pgfsetlinewidth{1.505625pt}%
\definecolor{currentstroke}{rgb}{0.000000,0.000000,0.000000}%
\pgfsetstrokecolor{currentstroke}%
\pgfsetdash{}{0pt}%
\pgfpathmoveto{\pgfqpoint{13.510186in}{5.738704in}}%
\pgfpathlineto{\pgfqpoint{13.510186in}{5.713273in}}%
\pgfusepath{stroke}%
\end{pgfscope}%
\begin{pgfscope}%
\pgfpathrectangle{\pgfqpoint{9.810417in}{5.660465in}}{\pgfqpoint{5.489583in}{0.877907in}}%
\pgfusepath{clip}%
\pgfsetbuttcap%
\pgfsetroundjoin%
\pgfsetlinewidth{1.505625pt}%
\definecolor{currentstroke}{rgb}{0.000000,0.000000,0.000000}%
\pgfsetstrokecolor{currentstroke}%
\pgfsetdash{}{0pt}%
\pgfpathmoveto{\pgfqpoint{13.633409in}{5.738704in}}%
\pgfpathlineto{\pgfqpoint{13.633409in}{5.759513in}}%
\pgfusepath{stroke}%
\end{pgfscope}%
\begin{pgfscope}%
\pgfpathrectangle{\pgfqpoint{9.810417in}{5.660465in}}{\pgfqpoint{5.489583in}{0.877907in}}%
\pgfusepath{clip}%
\pgfsetbuttcap%
\pgfsetroundjoin%
\pgfsetlinewidth{1.505625pt}%
\definecolor{currentstroke}{rgb}{0.000000,0.000000,0.000000}%
\pgfsetstrokecolor{currentstroke}%
\pgfsetdash{}{0pt}%
\pgfpathmoveto{\pgfqpoint{13.756632in}{5.738704in}}%
\pgfpathlineto{\pgfqpoint{13.756632in}{5.744728in}}%
\pgfusepath{stroke}%
\end{pgfscope}%
\begin{pgfscope}%
\pgfpathrectangle{\pgfqpoint{9.810417in}{5.660465in}}{\pgfqpoint{5.489583in}{0.877907in}}%
\pgfusepath{clip}%
\pgfsetbuttcap%
\pgfsetroundjoin%
\pgfsetlinewidth{1.505625pt}%
\definecolor{currentstroke}{rgb}{0.000000,0.000000,0.000000}%
\pgfsetstrokecolor{currentstroke}%
\pgfsetdash{}{0pt}%
\pgfpathmoveto{\pgfqpoint{13.879855in}{5.738704in}}%
\pgfpathlineto{\pgfqpoint{13.879855in}{5.746406in}}%
\pgfusepath{stroke}%
\end{pgfscope}%
\begin{pgfscope}%
\pgfpathrectangle{\pgfqpoint{9.810417in}{5.660465in}}{\pgfqpoint{5.489583in}{0.877907in}}%
\pgfusepath{clip}%
\pgfsetbuttcap%
\pgfsetroundjoin%
\pgfsetlinewidth{1.505625pt}%
\definecolor{currentstroke}{rgb}{0.000000,0.000000,0.000000}%
\pgfsetstrokecolor{currentstroke}%
\pgfsetdash{}{0pt}%
\pgfpathmoveto{\pgfqpoint{14.003078in}{5.738704in}}%
\pgfpathlineto{\pgfqpoint{14.003078in}{5.762792in}}%
\pgfusepath{stroke}%
\end{pgfscope}%
\begin{pgfscope}%
\pgfpathrectangle{\pgfqpoint{9.810417in}{5.660465in}}{\pgfqpoint{5.489583in}{0.877907in}}%
\pgfusepath{clip}%
\pgfsetbuttcap%
\pgfsetroundjoin%
\pgfsetlinewidth{1.505625pt}%
\definecolor{currentstroke}{rgb}{0.000000,0.000000,0.000000}%
\pgfsetstrokecolor{currentstroke}%
\pgfsetdash{}{0pt}%
\pgfpathmoveto{\pgfqpoint{14.126301in}{5.738704in}}%
\pgfpathlineto{\pgfqpoint{14.126301in}{5.738581in}}%
\pgfusepath{stroke}%
\end{pgfscope}%
\begin{pgfscope}%
\pgfpathrectangle{\pgfqpoint{9.810417in}{5.660465in}}{\pgfqpoint{5.489583in}{0.877907in}}%
\pgfusepath{clip}%
\pgfsetbuttcap%
\pgfsetroundjoin%
\pgfsetlinewidth{1.505625pt}%
\definecolor{currentstroke}{rgb}{0.000000,0.000000,0.000000}%
\pgfsetstrokecolor{currentstroke}%
\pgfsetdash{}{0pt}%
\pgfpathmoveto{\pgfqpoint{14.249524in}{5.738704in}}%
\pgfpathlineto{\pgfqpoint{14.249524in}{5.742813in}}%
\pgfusepath{stroke}%
\end{pgfscope}%
\begin{pgfscope}%
\pgfpathrectangle{\pgfqpoint{9.810417in}{5.660465in}}{\pgfqpoint{5.489583in}{0.877907in}}%
\pgfusepath{clip}%
\pgfsetbuttcap%
\pgfsetroundjoin%
\pgfsetlinewidth{1.505625pt}%
\definecolor{currentstroke}{rgb}{0.000000,0.000000,0.000000}%
\pgfsetstrokecolor{currentstroke}%
\pgfsetdash{}{0pt}%
\pgfpathmoveto{\pgfqpoint{14.372747in}{5.738704in}}%
\pgfpathlineto{\pgfqpoint{14.372747in}{5.728299in}}%
\pgfusepath{stroke}%
\end{pgfscope}%
\begin{pgfscope}%
\pgfpathrectangle{\pgfqpoint{9.810417in}{5.660465in}}{\pgfqpoint{5.489583in}{0.877907in}}%
\pgfusepath{clip}%
\pgfsetbuttcap%
\pgfsetroundjoin%
\pgfsetlinewidth{1.505625pt}%
\definecolor{currentstroke}{rgb}{0.000000,0.000000,0.000000}%
\pgfsetstrokecolor{currentstroke}%
\pgfsetdash{}{0pt}%
\pgfpathmoveto{\pgfqpoint{14.495970in}{5.738704in}}%
\pgfpathlineto{\pgfqpoint{14.495970in}{5.724068in}}%
\pgfusepath{stroke}%
\end{pgfscope}%
\begin{pgfscope}%
\pgfpathrectangle{\pgfqpoint{9.810417in}{5.660465in}}{\pgfqpoint{5.489583in}{0.877907in}}%
\pgfusepath{clip}%
\pgfsetbuttcap%
\pgfsetroundjoin%
\pgfsetlinewidth{1.505625pt}%
\definecolor{currentstroke}{rgb}{0.000000,0.000000,0.000000}%
\pgfsetstrokecolor{currentstroke}%
\pgfsetdash{}{0pt}%
\pgfpathmoveto{\pgfqpoint{14.619193in}{5.738704in}}%
\pgfpathlineto{\pgfqpoint{14.619193in}{5.741766in}}%
\pgfusepath{stroke}%
\end{pgfscope}%
\begin{pgfscope}%
\pgfpathrectangle{\pgfqpoint{9.810417in}{5.660465in}}{\pgfqpoint{5.489583in}{0.877907in}}%
\pgfusepath{clip}%
\pgfsetbuttcap%
\pgfsetroundjoin%
\pgfsetlinewidth{1.505625pt}%
\definecolor{currentstroke}{rgb}{0.000000,0.000000,0.000000}%
\pgfsetstrokecolor{currentstroke}%
\pgfsetdash{}{0pt}%
\pgfpathmoveto{\pgfqpoint{14.742416in}{5.738704in}}%
\pgfpathlineto{\pgfqpoint{14.742416in}{5.740102in}}%
\pgfusepath{stroke}%
\end{pgfscope}%
\begin{pgfscope}%
\pgfpathrectangle{\pgfqpoint{9.810417in}{5.660465in}}{\pgfqpoint{5.489583in}{0.877907in}}%
\pgfusepath{clip}%
\pgfsetbuttcap%
\pgfsetroundjoin%
\pgfsetlinewidth{1.505625pt}%
\definecolor{currentstroke}{rgb}{0.000000,0.000000,0.000000}%
\pgfsetstrokecolor{currentstroke}%
\pgfsetdash{}{0pt}%
\pgfpathmoveto{\pgfqpoint{14.865639in}{5.738704in}}%
\pgfpathlineto{\pgfqpoint{14.865639in}{5.701217in}}%
\pgfusepath{stroke}%
\end{pgfscope}%
\begin{pgfscope}%
\pgfpathrectangle{\pgfqpoint{9.810417in}{5.660465in}}{\pgfqpoint{5.489583in}{0.877907in}}%
\pgfusepath{clip}%
\pgfsetbuttcap%
\pgfsetroundjoin%
\pgfsetlinewidth{1.505625pt}%
\definecolor{currentstroke}{rgb}{0.000000,0.000000,0.000000}%
\pgfsetstrokecolor{currentstroke}%
\pgfsetdash{}{0pt}%
\pgfpathmoveto{\pgfqpoint{14.988862in}{5.738704in}}%
\pgfpathlineto{\pgfqpoint{14.988862in}{5.740465in}}%
\pgfusepath{stroke}%
\end{pgfscope}%
\begin{pgfscope}%
\pgfpathrectangle{\pgfqpoint{9.810417in}{5.660465in}}{\pgfqpoint{5.489583in}{0.877907in}}%
\pgfusepath{clip}%
\pgfsetroundcap%
\pgfsetroundjoin%
\pgfsetlinewidth{1.505625pt}%
\definecolor{currentstroke}{rgb}{0.121569,0.466667,0.705882}%
\pgfsetstrokecolor{currentstroke}%
\pgfsetdash{}{0pt}%
\pgfpathmoveto{\pgfqpoint{9.810417in}{5.738704in}}%
\pgfpathlineto{\pgfqpoint{15.300000in}{5.738704in}}%
\pgfusepath{stroke}%
\end{pgfscope}%
\begin{pgfscope}%
\pgfpathrectangle{\pgfqpoint{9.810417in}{5.660465in}}{\pgfqpoint{5.489583in}{0.877907in}}%
\pgfusepath{clip}%
\pgfsetbuttcap%
\pgfsetroundjoin%
\definecolor{currentfill}{rgb}{0.121569,0.466667,0.705882}%
\pgfsetfillcolor{currentfill}%
\pgfsetlinewidth{1.003750pt}%
\definecolor{currentstroke}{rgb}{0.121569,0.466667,0.705882}%
\pgfsetstrokecolor{currentstroke}%
\pgfsetdash{}{0pt}%
\pgfsys@defobject{currentmarker}{\pgfqpoint{-0.034722in}{-0.034722in}}{\pgfqpoint{0.034722in}{0.034722in}}{%
\pgfpathmoveto{\pgfqpoint{0.000000in}{-0.034722in}}%
\pgfpathcurveto{\pgfqpoint{0.009208in}{-0.034722in}}{\pgfqpoint{0.018041in}{-0.031064in}}{\pgfqpoint{0.024552in}{-0.024552in}}%
\pgfpathcurveto{\pgfqpoint{0.031064in}{-0.018041in}}{\pgfqpoint{0.034722in}{-0.009208in}}{\pgfqpoint{0.034722in}{0.000000in}}%
\pgfpathcurveto{\pgfqpoint{0.034722in}{0.009208in}}{\pgfqpoint{0.031064in}{0.018041in}}{\pgfqpoint{0.024552in}{0.024552in}}%
\pgfpathcurveto{\pgfqpoint{0.018041in}{0.031064in}}{\pgfqpoint{0.009208in}{0.034722in}}{\pgfqpoint{0.000000in}{0.034722in}}%
\pgfpathcurveto{\pgfqpoint{-0.009208in}{0.034722in}}{\pgfqpoint{-0.018041in}{0.031064in}}{\pgfqpoint{-0.024552in}{0.024552in}}%
\pgfpathcurveto{\pgfqpoint{-0.031064in}{0.018041in}}{\pgfqpoint{-0.034722in}{0.009208in}}{\pgfqpoint{-0.034722in}{0.000000in}}%
\pgfpathcurveto{\pgfqpoint{-0.034722in}{-0.009208in}}{\pgfqpoint{-0.031064in}{-0.018041in}}{\pgfqpoint{-0.024552in}{-0.024552in}}%
\pgfpathcurveto{\pgfqpoint{-0.018041in}{-0.031064in}}{\pgfqpoint{-0.009208in}{-0.034722in}}{\pgfqpoint{0.000000in}{-0.034722in}}%
\pgfpathclose%
\pgfusepath{stroke,fill}%
}%
\begin{pgfscope}%
\pgfsys@transformshift{10.059943in}{6.498467in}%
\pgfsys@useobject{currentmarker}{}%
\end{pgfscope}%
\begin{pgfscope}%
\pgfsys@transformshift{10.183166in}{6.494818in}%
\pgfsys@useobject{currentmarker}{}%
\end{pgfscope}%
\begin{pgfscope}%
\pgfsys@transformshift{10.306389in}{5.704530in}%
\pgfsys@useobject{currentmarker}{}%
\end{pgfscope}%
\begin{pgfscope}%
\pgfsys@transformshift{10.429612in}{5.726947in}%
\pgfsys@useobject{currentmarker}{}%
\end{pgfscope}%
\begin{pgfscope}%
\pgfsys@transformshift{10.552835in}{5.731162in}%
\pgfsys@useobject{currentmarker}{}%
\end{pgfscope}%
\begin{pgfscope}%
\pgfsys@transformshift{10.676058in}{5.752214in}%
\pgfsys@useobject{currentmarker}{}%
\end{pgfscope}%
\begin{pgfscope}%
\pgfsys@transformshift{10.799281in}{5.718116in}%
\pgfsys@useobject{currentmarker}{}%
\end{pgfscope}%
\begin{pgfscope}%
\pgfsys@transformshift{10.922504in}{5.730071in}%
\pgfsys@useobject{currentmarker}{}%
\end{pgfscope}%
\begin{pgfscope}%
\pgfsys@transformshift{11.045727in}{5.738515in}%
\pgfsys@useobject{currentmarker}{}%
\end{pgfscope}%
\begin{pgfscope}%
\pgfsys@transformshift{11.168950in}{5.737618in}%
\pgfsys@useobject{currentmarker}{}%
\end{pgfscope}%
\begin{pgfscope}%
\pgfsys@transformshift{11.292173in}{5.766994in}%
\pgfsys@useobject{currentmarker}{}%
\end{pgfscope}%
\begin{pgfscope}%
\pgfsys@transformshift{11.415396in}{5.742105in}%
\pgfsys@useobject{currentmarker}{}%
\end{pgfscope}%
\begin{pgfscope}%
\pgfsys@transformshift{11.538619in}{5.720343in}%
\pgfsys@useobject{currentmarker}{}%
\end{pgfscope}%
\begin{pgfscope}%
\pgfsys@transformshift{11.661842in}{5.743483in}%
\pgfsys@useobject{currentmarker}{}%
\end{pgfscope}%
\begin{pgfscope}%
\pgfsys@transformshift{11.785065in}{5.716537in}%
\pgfsys@useobject{currentmarker}{}%
\end{pgfscope}%
\begin{pgfscope}%
\pgfsys@transformshift{11.908288in}{5.742957in}%
\pgfsys@useobject{currentmarker}{}%
\end{pgfscope}%
\begin{pgfscope}%
\pgfsys@transformshift{12.031511in}{5.748041in}%
\pgfsys@useobject{currentmarker}{}%
\end{pgfscope}%
\begin{pgfscope}%
\pgfsys@transformshift{12.154734in}{5.722072in}%
\pgfsys@useobject{currentmarker}{}%
\end{pgfscope}%
\begin{pgfscope}%
\pgfsys@transformshift{12.277957in}{5.733855in}%
\pgfsys@useobject{currentmarker}{}%
\end{pgfscope}%
\begin{pgfscope}%
\pgfsys@transformshift{12.401180in}{5.740818in}%
\pgfsys@useobject{currentmarker}{}%
\end{pgfscope}%
\begin{pgfscope}%
\pgfsys@transformshift{12.524403in}{5.755199in}%
\pgfsys@useobject{currentmarker}{}%
\end{pgfscope}%
\begin{pgfscope}%
\pgfsys@transformshift{12.647626in}{5.731481in}%
\pgfsys@useobject{currentmarker}{}%
\end{pgfscope}%
\begin{pgfscope}%
\pgfsys@transformshift{12.770849in}{5.758217in}%
\pgfsys@useobject{currentmarker}{}%
\end{pgfscope}%
\begin{pgfscope}%
\pgfsys@transformshift{12.894072in}{5.768832in}%
\pgfsys@useobject{currentmarker}{}%
\end{pgfscope}%
\begin{pgfscope}%
\pgfsys@transformshift{13.017294in}{5.747465in}%
\pgfsys@useobject{currentmarker}{}%
\end{pgfscope}%
\begin{pgfscope}%
\pgfsys@transformshift{13.140517in}{5.748047in}%
\pgfsys@useobject{currentmarker}{}%
\end{pgfscope}%
\begin{pgfscope}%
\pgfsys@transformshift{13.263740in}{5.762313in}%
\pgfsys@useobject{currentmarker}{}%
\end{pgfscope}%
\begin{pgfscope}%
\pgfsys@transformshift{13.386963in}{5.761192in}%
\pgfsys@useobject{currentmarker}{}%
\end{pgfscope}%
\begin{pgfscope}%
\pgfsys@transformshift{13.510186in}{5.713273in}%
\pgfsys@useobject{currentmarker}{}%
\end{pgfscope}%
\begin{pgfscope}%
\pgfsys@transformshift{13.633409in}{5.759513in}%
\pgfsys@useobject{currentmarker}{}%
\end{pgfscope}%
\begin{pgfscope}%
\pgfsys@transformshift{13.756632in}{5.744728in}%
\pgfsys@useobject{currentmarker}{}%
\end{pgfscope}%
\begin{pgfscope}%
\pgfsys@transformshift{13.879855in}{5.746406in}%
\pgfsys@useobject{currentmarker}{}%
\end{pgfscope}%
\begin{pgfscope}%
\pgfsys@transformshift{14.003078in}{5.762792in}%
\pgfsys@useobject{currentmarker}{}%
\end{pgfscope}%
\begin{pgfscope}%
\pgfsys@transformshift{14.126301in}{5.738581in}%
\pgfsys@useobject{currentmarker}{}%
\end{pgfscope}%
\begin{pgfscope}%
\pgfsys@transformshift{14.249524in}{5.742813in}%
\pgfsys@useobject{currentmarker}{}%
\end{pgfscope}%
\begin{pgfscope}%
\pgfsys@transformshift{14.372747in}{5.728299in}%
\pgfsys@useobject{currentmarker}{}%
\end{pgfscope}%
\begin{pgfscope}%
\pgfsys@transformshift{14.495970in}{5.724068in}%
\pgfsys@useobject{currentmarker}{}%
\end{pgfscope}%
\begin{pgfscope}%
\pgfsys@transformshift{14.619193in}{5.741766in}%
\pgfsys@useobject{currentmarker}{}%
\end{pgfscope}%
\begin{pgfscope}%
\pgfsys@transformshift{14.742416in}{5.740102in}%
\pgfsys@useobject{currentmarker}{}%
\end{pgfscope}%
\begin{pgfscope}%
\pgfsys@transformshift{14.865639in}{5.701217in}%
\pgfsys@useobject{currentmarker}{}%
\end{pgfscope}%
\begin{pgfscope}%
\pgfsys@transformshift{14.988862in}{5.740465in}%
\pgfsys@useobject{currentmarker}{}%
\end{pgfscope}%
\end{pgfscope}%
\begin{pgfscope}%
\pgfsetrectcap%
\pgfsetmiterjoin%
\pgfsetlinewidth{0.803000pt}%
\definecolor{currentstroke}{rgb}{1.000000,1.000000,1.000000}%
\pgfsetstrokecolor{currentstroke}%
\pgfsetdash{}{0pt}%
\pgfpathmoveto{\pgfqpoint{9.810417in}{5.660465in}}%
\pgfpathlineto{\pgfqpoint{9.810417in}{6.538372in}}%
\pgfusepath{stroke}%
\end{pgfscope}%
\begin{pgfscope}%
\pgfsetrectcap%
\pgfsetmiterjoin%
\pgfsetlinewidth{0.803000pt}%
\definecolor{currentstroke}{rgb}{1.000000,1.000000,1.000000}%
\pgfsetstrokecolor{currentstroke}%
\pgfsetdash{}{0pt}%
\pgfpathmoveto{\pgfqpoint{15.300000in}{5.660465in}}%
\pgfpathlineto{\pgfqpoint{15.300000in}{6.538372in}}%
\pgfusepath{stroke}%
\end{pgfscope}%
\begin{pgfscope}%
\pgfsetrectcap%
\pgfsetmiterjoin%
\pgfsetlinewidth{0.803000pt}%
\definecolor{currentstroke}{rgb}{1.000000,1.000000,1.000000}%
\pgfsetstrokecolor{currentstroke}%
\pgfsetdash{}{0pt}%
\pgfpathmoveto{\pgfqpoint{9.810417in}{5.660465in}}%
\pgfpathlineto{\pgfqpoint{15.300000in}{5.660465in}}%
\pgfusepath{stroke}%
\end{pgfscope}%
\begin{pgfscope}%
\pgfsetrectcap%
\pgfsetmiterjoin%
\pgfsetlinewidth{0.803000pt}%
\definecolor{currentstroke}{rgb}{1.000000,1.000000,1.000000}%
\pgfsetstrokecolor{currentstroke}%
\pgfsetdash{}{0pt}%
\pgfpathmoveto{\pgfqpoint{9.810417in}{6.538372in}}%
\pgfpathlineto{\pgfqpoint{15.300000in}{6.538372in}}%
\pgfusepath{stroke}%
\end{pgfscope}%
\begin{pgfscope}%
\definecolor{textcolor}{rgb}{0.150000,0.150000,0.150000}%
\pgfsetstrokecolor{textcolor}%
\pgfsetfillcolor{textcolor}%
\pgftext[x=12.555208in,y=6.621705in,,base]{\color{textcolor}\rmfamily\fontsize{16.800000}{20.160000}\selectfont Partial Autocorrelation}%
\end{pgfscope}%
\begin{pgfscope}%
\pgfsetbuttcap%
\pgfsetmiterjoin%
\definecolor{currentfill}{rgb}{0.917647,0.917647,0.949020}%
\pgfsetfillcolor{currentfill}%
\pgfsetlinewidth{0.000000pt}%
\definecolor{currentstroke}{rgb}{0.000000,0.000000,0.000000}%
\pgfsetstrokecolor{currentstroke}%
\pgfsetstrokeopacity{0.000000}%
\pgfsetdash{}{0pt}%
\pgfpathmoveto{\pgfqpoint{2.125000in}{4.080233in}}%
\pgfpathlineto{\pgfqpoint{7.614583in}{4.080233in}}%
\pgfpathlineto{\pgfqpoint{7.614583in}{4.958140in}}%
\pgfpathlineto{\pgfqpoint{2.125000in}{4.958140in}}%
\pgfpathclose%
\pgfusepath{fill}%
\end{pgfscope}%
\begin{pgfscope}%
\pgfpathrectangle{\pgfqpoint{2.125000in}{4.080233in}}{\pgfqpoint{5.489583in}{0.877907in}}%
\pgfusepath{clip}%
\pgfsetroundcap%
\pgfsetroundjoin%
\pgfsetlinewidth{0.803000pt}%
\definecolor{currentstroke}{rgb}{1.000000,1.000000,1.000000}%
\pgfsetstrokecolor{currentstroke}%
\pgfsetdash{}{0pt}%
\pgfpathmoveto{\pgfqpoint{2.374527in}{4.080233in}}%
\pgfpathlineto{\pgfqpoint{2.374527in}{4.958140in}}%
\pgfusepath{stroke}%
\end{pgfscope}%
\begin{pgfscope}%
\definecolor{textcolor}{rgb}{0.150000,0.150000,0.150000}%
\pgfsetstrokecolor{textcolor}%
\pgfsetfillcolor{textcolor}%
\pgftext[x=2.374527in,y=3.983010in,,top]{\color{textcolor}\rmfamily\fontsize{14.000000}{16.800000}\selectfont 0}%
\end{pgfscope}%
\begin{pgfscope}%
\pgfpathrectangle{\pgfqpoint{2.125000in}{4.080233in}}{\pgfqpoint{5.489583in}{0.877907in}}%
\pgfusepath{clip}%
\pgfsetroundcap%
\pgfsetroundjoin%
\pgfsetlinewidth{0.803000pt}%
\definecolor{currentstroke}{rgb}{1.000000,1.000000,1.000000}%
\pgfsetstrokecolor{currentstroke}%
\pgfsetdash{}{0pt}%
\pgfpathmoveto{\pgfqpoint{2.990641in}{4.080233in}}%
\pgfpathlineto{\pgfqpoint{2.990641in}{4.958140in}}%
\pgfusepath{stroke}%
\end{pgfscope}%
\begin{pgfscope}%
\definecolor{textcolor}{rgb}{0.150000,0.150000,0.150000}%
\pgfsetstrokecolor{textcolor}%
\pgfsetfillcolor{textcolor}%
\pgftext[x=2.990641in,y=3.983010in,,top]{\color{textcolor}\rmfamily\fontsize{14.000000}{16.800000}\selectfont 5}%
\end{pgfscope}%
\begin{pgfscope}%
\pgfpathrectangle{\pgfqpoint{2.125000in}{4.080233in}}{\pgfqpoint{5.489583in}{0.877907in}}%
\pgfusepath{clip}%
\pgfsetroundcap%
\pgfsetroundjoin%
\pgfsetlinewidth{0.803000pt}%
\definecolor{currentstroke}{rgb}{1.000000,1.000000,1.000000}%
\pgfsetstrokecolor{currentstroke}%
\pgfsetdash{}{0pt}%
\pgfpathmoveto{\pgfqpoint{3.606756in}{4.080233in}}%
\pgfpathlineto{\pgfqpoint{3.606756in}{4.958140in}}%
\pgfusepath{stroke}%
\end{pgfscope}%
\begin{pgfscope}%
\definecolor{textcolor}{rgb}{0.150000,0.150000,0.150000}%
\pgfsetstrokecolor{textcolor}%
\pgfsetfillcolor{textcolor}%
\pgftext[x=3.606756in,y=3.983010in,,top]{\color{textcolor}\rmfamily\fontsize{14.000000}{16.800000}\selectfont 10}%
\end{pgfscope}%
\begin{pgfscope}%
\pgfpathrectangle{\pgfqpoint{2.125000in}{4.080233in}}{\pgfqpoint{5.489583in}{0.877907in}}%
\pgfusepath{clip}%
\pgfsetroundcap%
\pgfsetroundjoin%
\pgfsetlinewidth{0.803000pt}%
\definecolor{currentstroke}{rgb}{1.000000,1.000000,1.000000}%
\pgfsetstrokecolor{currentstroke}%
\pgfsetdash{}{0pt}%
\pgfpathmoveto{\pgfqpoint{4.222871in}{4.080233in}}%
\pgfpathlineto{\pgfqpoint{4.222871in}{4.958140in}}%
\pgfusepath{stroke}%
\end{pgfscope}%
\begin{pgfscope}%
\definecolor{textcolor}{rgb}{0.150000,0.150000,0.150000}%
\pgfsetstrokecolor{textcolor}%
\pgfsetfillcolor{textcolor}%
\pgftext[x=4.222871in,y=3.983010in,,top]{\color{textcolor}\rmfamily\fontsize{14.000000}{16.800000}\selectfont 15}%
\end{pgfscope}%
\begin{pgfscope}%
\pgfpathrectangle{\pgfqpoint{2.125000in}{4.080233in}}{\pgfqpoint{5.489583in}{0.877907in}}%
\pgfusepath{clip}%
\pgfsetroundcap%
\pgfsetroundjoin%
\pgfsetlinewidth{0.803000pt}%
\definecolor{currentstroke}{rgb}{1.000000,1.000000,1.000000}%
\pgfsetstrokecolor{currentstroke}%
\pgfsetdash{}{0pt}%
\pgfpathmoveto{\pgfqpoint{4.838986in}{4.080233in}}%
\pgfpathlineto{\pgfqpoint{4.838986in}{4.958140in}}%
\pgfusepath{stroke}%
\end{pgfscope}%
\begin{pgfscope}%
\definecolor{textcolor}{rgb}{0.150000,0.150000,0.150000}%
\pgfsetstrokecolor{textcolor}%
\pgfsetfillcolor{textcolor}%
\pgftext[x=4.838986in,y=3.983010in,,top]{\color{textcolor}\rmfamily\fontsize{14.000000}{16.800000}\selectfont 20}%
\end{pgfscope}%
\begin{pgfscope}%
\pgfpathrectangle{\pgfqpoint{2.125000in}{4.080233in}}{\pgfqpoint{5.489583in}{0.877907in}}%
\pgfusepath{clip}%
\pgfsetroundcap%
\pgfsetroundjoin%
\pgfsetlinewidth{0.803000pt}%
\definecolor{currentstroke}{rgb}{1.000000,1.000000,1.000000}%
\pgfsetstrokecolor{currentstroke}%
\pgfsetdash{}{0pt}%
\pgfpathmoveto{\pgfqpoint{5.455101in}{4.080233in}}%
\pgfpathlineto{\pgfqpoint{5.455101in}{4.958140in}}%
\pgfusepath{stroke}%
\end{pgfscope}%
\begin{pgfscope}%
\definecolor{textcolor}{rgb}{0.150000,0.150000,0.150000}%
\pgfsetstrokecolor{textcolor}%
\pgfsetfillcolor{textcolor}%
\pgftext[x=5.455101in,y=3.983010in,,top]{\color{textcolor}\rmfamily\fontsize{14.000000}{16.800000}\selectfont 25}%
\end{pgfscope}%
\begin{pgfscope}%
\pgfpathrectangle{\pgfqpoint{2.125000in}{4.080233in}}{\pgfqpoint{5.489583in}{0.877907in}}%
\pgfusepath{clip}%
\pgfsetroundcap%
\pgfsetroundjoin%
\pgfsetlinewidth{0.803000pt}%
\definecolor{currentstroke}{rgb}{1.000000,1.000000,1.000000}%
\pgfsetstrokecolor{currentstroke}%
\pgfsetdash{}{0pt}%
\pgfpathmoveto{\pgfqpoint{6.071216in}{4.080233in}}%
\pgfpathlineto{\pgfqpoint{6.071216in}{4.958140in}}%
\pgfusepath{stroke}%
\end{pgfscope}%
\begin{pgfscope}%
\definecolor{textcolor}{rgb}{0.150000,0.150000,0.150000}%
\pgfsetstrokecolor{textcolor}%
\pgfsetfillcolor{textcolor}%
\pgftext[x=6.071216in,y=3.983010in,,top]{\color{textcolor}\rmfamily\fontsize{14.000000}{16.800000}\selectfont 30}%
\end{pgfscope}%
\begin{pgfscope}%
\pgfpathrectangle{\pgfqpoint{2.125000in}{4.080233in}}{\pgfqpoint{5.489583in}{0.877907in}}%
\pgfusepath{clip}%
\pgfsetroundcap%
\pgfsetroundjoin%
\pgfsetlinewidth{0.803000pt}%
\definecolor{currentstroke}{rgb}{1.000000,1.000000,1.000000}%
\pgfsetstrokecolor{currentstroke}%
\pgfsetdash{}{0pt}%
\pgfpathmoveto{\pgfqpoint{6.687330in}{4.080233in}}%
\pgfpathlineto{\pgfqpoint{6.687330in}{4.958140in}}%
\pgfusepath{stroke}%
\end{pgfscope}%
\begin{pgfscope}%
\definecolor{textcolor}{rgb}{0.150000,0.150000,0.150000}%
\pgfsetstrokecolor{textcolor}%
\pgfsetfillcolor{textcolor}%
\pgftext[x=6.687330in,y=3.983010in,,top]{\color{textcolor}\rmfamily\fontsize{14.000000}{16.800000}\selectfont 35}%
\end{pgfscope}%
\begin{pgfscope}%
\pgfpathrectangle{\pgfqpoint{2.125000in}{4.080233in}}{\pgfqpoint{5.489583in}{0.877907in}}%
\pgfusepath{clip}%
\pgfsetroundcap%
\pgfsetroundjoin%
\pgfsetlinewidth{0.803000pt}%
\definecolor{currentstroke}{rgb}{1.000000,1.000000,1.000000}%
\pgfsetstrokecolor{currentstroke}%
\pgfsetdash{}{0pt}%
\pgfpathmoveto{\pgfqpoint{7.303445in}{4.080233in}}%
\pgfpathlineto{\pgfqpoint{7.303445in}{4.958140in}}%
\pgfusepath{stroke}%
\end{pgfscope}%
\begin{pgfscope}%
\definecolor{textcolor}{rgb}{0.150000,0.150000,0.150000}%
\pgfsetstrokecolor{textcolor}%
\pgfsetfillcolor{textcolor}%
\pgftext[x=7.303445in,y=3.983010in,,top]{\color{textcolor}\rmfamily\fontsize{14.000000}{16.800000}\selectfont 40}%
\end{pgfscope}%
\begin{pgfscope}%
\pgfpathrectangle{\pgfqpoint{2.125000in}{4.080233in}}{\pgfqpoint{5.489583in}{0.877907in}}%
\pgfusepath{clip}%
\pgfsetroundcap%
\pgfsetroundjoin%
\pgfsetlinewidth{0.803000pt}%
\definecolor{currentstroke}{rgb}{1.000000,1.000000,1.000000}%
\pgfsetstrokecolor{currentstroke}%
\pgfsetdash{}{0pt}%
\pgfpathmoveto{\pgfqpoint{2.125000in}{4.357721in}}%
\pgfpathlineto{\pgfqpoint{7.614583in}{4.357721in}}%
\pgfusepath{stroke}%
\end{pgfscope}%
\begin{pgfscope}%
\definecolor{textcolor}{rgb}{0.150000,0.150000,0.150000}%
\pgfsetstrokecolor{textcolor}%
\pgfsetfillcolor{textcolor}%
\pgftext[x=1.904066in,y=4.283855in,left,base]{\color{textcolor}\rmfamily\fontsize{14.000000}{16.800000}\selectfont 0}%
\end{pgfscope}%
\begin{pgfscope}%
\pgfpathrectangle{\pgfqpoint{2.125000in}{4.080233in}}{\pgfqpoint{5.489583in}{0.877907in}}%
\pgfusepath{clip}%
\pgfsetroundcap%
\pgfsetroundjoin%
\pgfsetlinewidth{0.803000pt}%
\definecolor{currentstroke}{rgb}{1.000000,1.000000,1.000000}%
\pgfsetstrokecolor{currentstroke}%
\pgfsetdash{}{0pt}%
\pgfpathmoveto{\pgfqpoint{2.125000in}{4.918235in}}%
\pgfpathlineto{\pgfqpoint{7.614583in}{4.918235in}}%
\pgfusepath{stroke}%
\end{pgfscope}%
\begin{pgfscope}%
\definecolor{textcolor}{rgb}{0.150000,0.150000,0.150000}%
\pgfsetstrokecolor{textcolor}%
\pgfsetfillcolor{textcolor}%
\pgftext[x=1.904066in,y=4.844369in,left,base]{\color{textcolor}\rmfamily\fontsize{14.000000}{16.800000}\selectfont 1}%
\end{pgfscope}%
\begin{pgfscope}%
\pgfpathrectangle{\pgfqpoint{2.125000in}{4.080233in}}{\pgfqpoint{5.489583in}{0.877907in}}%
\pgfusepath{clip}%
\pgfsetbuttcap%
\pgfsetroundjoin%
\definecolor{currentfill}{rgb}{0.121569,0.466667,0.705882}%
\pgfsetfillcolor{currentfill}%
\pgfsetfillopacity{0.250000}%
\pgfsetlinewidth{1.003750pt}%
\definecolor{currentstroke}{rgb}{1.000000,1.000000,1.000000}%
\pgfsetstrokecolor{currentstroke}%
\pgfsetstrokeopacity{0.250000}%
\pgfsetdash{}{0pt}%
\pgfpathmoveto{\pgfqpoint{2.436138in}{4.386002in}}%
\pgfpathlineto{\pgfqpoint{2.436138in}{4.329441in}}%
\pgfpathlineto{\pgfqpoint{2.620972in}{4.308834in}}%
\pgfpathlineto{\pgfqpoint{2.744195in}{4.294707in}}%
\pgfpathlineto{\pgfqpoint{2.867418in}{4.283273in}}%
\pgfpathlineto{\pgfqpoint{2.990641in}{4.273430in}}%
\pgfpathlineto{\pgfqpoint{3.113864in}{4.264670in}}%
\pgfpathlineto{\pgfqpoint{3.237087in}{4.256711in}}%
\pgfpathlineto{\pgfqpoint{3.360310in}{4.249376in}}%
\pgfpathlineto{\pgfqpoint{3.483533in}{4.242545in}}%
\pgfpathlineto{\pgfqpoint{3.606756in}{4.236134in}}%
\pgfpathlineto{\pgfqpoint{3.729979in}{4.230080in}}%
\pgfpathlineto{\pgfqpoint{3.853202in}{4.224333in}}%
\pgfpathlineto{\pgfqpoint{3.976425in}{4.218856in}}%
\pgfpathlineto{\pgfqpoint{4.099648in}{4.213618in}}%
\pgfpathlineto{\pgfqpoint{4.222871in}{4.208595in}}%
\pgfpathlineto{\pgfqpoint{4.346094in}{4.203765in}}%
\pgfpathlineto{\pgfqpoint{4.469317in}{4.199109in}}%
\pgfpathlineto{\pgfqpoint{4.592540in}{4.194612in}}%
\pgfpathlineto{\pgfqpoint{4.715763in}{4.190261in}}%
\pgfpathlineto{\pgfqpoint{4.838986in}{4.186045in}}%
\pgfpathlineto{\pgfqpoint{4.962209in}{4.181954in}}%
\pgfpathlineto{\pgfqpoint{5.085432in}{4.177980in}}%
\pgfpathlineto{\pgfqpoint{5.208655in}{4.174115in}}%
\pgfpathlineto{\pgfqpoint{5.331878in}{4.170351in}}%
\pgfpathlineto{\pgfqpoint{5.455101in}{4.166682in}}%
\pgfpathlineto{\pgfqpoint{5.578324in}{4.163102in}}%
\pgfpathlineto{\pgfqpoint{5.701547in}{4.159607in}}%
\pgfpathlineto{\pgfqpoint{5.824770in}{4.156191in}}%
\pgfpathlineto{\pgfqpoint{5.947993in}{4.152850in}}%
\pgfpathlineto{\pgfqpoint{6.071216in}{4.149580in}}%
\pgfpathlineto{\pgfqpoint{6.194439in}{4.146377in}}%
\pgfpathlineto{\pgfqpoint{6.317662in}{4.143239in}}%
\pgfpathlineto{\pgfqpoint{6.440885in}{4.140161in}}%
\pgfpathlineto{\pgfqpoint{6.564108in}{4.137143in}}%
\pgfpathlineto{\pgfqpoint{6.687330in}{4.134180in}}%
\pgfpathlineto{\pgfqpoint{6.810553in}{4.131272in}}%
\pgfpathlineto{\pgfqpoint{6.933776in}{4.128416in}}%
\pgfpathlineto{\pgfqpoint{7.056999in}{4.125609in}}%
\pgfpathlineto{\pgfqpoint{7.180222in}{4.122850in}}%
\pgfpathlineto{\pgfqpoint{7.365057in}{4.120137in}}%
\pgfpathlineto{\pgfqpoint{7.365057in}{4.595305in}}%
\pgfpathlineto{\pgfqpoint{7.365057in}{4.595305in}}%
\pgfpathlineto{\pgfqpoint{7.180222in}{4.592592in}}%
\pgfpathlineto{\pgfqpoint{7.056999in}{4.589834in}}%
\pgfpathlineto{\pgfqpoint{6.933776in}{4.587027in}}%
\pgfpathlineto{\pgfqpoint{6.810553in}{4.584170in}}%
\pgfpathlineto{\pgfqpoint{6.687330in}{4.581262in}}%
\pgfpathlineto{\pgfqpoint{6.564108in}{4.578300in}}%
\pgfpathlineto{\pgfqpoint{6.440885in}{4.575281in}}%
\pgfpathlineto{\pgfqpoint{6.317662in}{4.572204in}}%
\pgfpathlineto{\pgfqpoint{6.194439in}{4.569065in}}%
\pgfpathlineto{\pgfqpoint{6.071216in}{4.565862in}}%
\pgfpathlineto{\pgfqpoint{5.947993in}{4.562592in}}%
\pgfpathlineto{\pgfqpoint{5.824770in}{4.559251in}}%
\pgfpathlineto{\pgfqpoint{5.701547in}{4.555835in}}%
\pgfpathlineto{\pgfqpoint{5.578324in}{4.552340in}}%
\pgfpathlineto{\pgfqpoint{5.455101in}{4.548761in}}%
\pgfpathlineto{\pgfqpoint{5.331878in}{4.545092in}}%
\pgfpathlineto{\pgfqpoint{5.208655in}{4.541328in}}%
\pgfpathlineto{\pgfqpoint{5.085432in}{4.537462in}}%
\pgfpathlineto{\pgfqpoint{4.962209in}{4.533488in}}%
\pgfpathlineto{\pgfqpoint{4.838986in}{4.529398in}}%
\pgfpathlineto{\pgfqpoint{4.715763in}{4.525181in}}%
\pgfpathlineto{\pgfqpoint{4.592540in}{4.520830in}}%
\pgfpathlineto{\pgfqpoint{4.469317in}{4.516333in}}%
\pgfpathlineto{\pgfqpoint{4.346094in}{4.511677in}}%
\pgfpathlineto{\pgfqpoint{4.222871in}{4.506847in}}%
\pgfpathlineto{\pgfqpoint{4.099648in}{4.501824in}}%
\pgfpathlineto{\pgfqpoint{3.976425in}{4.496586in}}%
\pgfpathlineto{\pgfqpoint{3.853202in}{4.491109in}}%
\pgfpathlineto{\pgfqpoint{3.729979in}{4.485362in}}%
\pgfpathlineto{\pgfqpoint{3.606756in}{4.479308in}}%
\pgfpathlineto{\pgfqpoint{3.483533in}{4.472898in}}%
\pgfpathlineto{\pgfqpoint{3.360310in}{4.466067in}}%
\pgfpathlineto{\pgfqpoint{3.237087in}{4.458732in}}%
\pgfpathlineto{\pgfqpoint{3.113864in}{4.450772in}}%
\pgfpathlineto{\pgfqpoint{2.990641in}{4.442013in}}%
\pgfpathlineto{\pgfqpoint{2.867418in}{4.432169in}}%
\pgfpathlineto{\pgfqpoint{2.744195in}{4.420736in}}%
\pgfpathlineto{\pgfqpoint{2.620972in}{4.406609in}}%
\pgfpathlineto{\pgfqpoint{2.436138in}{4.386002in}}%
\pgfpathclose%
\pgfusepath{stroke,fill}%
\end{pgfscope}%
\begin{pgfscope}%
\pgfpathrectangle{\pgfqpoint{2.125000in}{4.080233in}}{\pgfqpoint{5.489583in}{0.877907in}}%
\pgfusepath{clip}%
\pgfsetbuttcap%
\pgfsetroundjoin%
\pgfsetlinewidth{1.505625pt}%
\definecolor{currentstroke}{rgb}{0.000000,0.000000,0.000000}%
\pgfsetstrokecolor{currentstroke}%
\pgfsetdash{}{0pt}%
\pgfpathmoveto{\pgfqpoint{2.374527in}{4.357721in}}%
\pgfpathlineto{\pgfqpoint{2.374527in}{4.918235in}}%
\pgfusepath{stroke}%
\end{pgfscope}%
\begin{pgfscope}%
\pgfpathrectangle{\pgfqpoint{2.125000in}{4.080233in}}{\pgfqpoint{5.489583in}{0.877907in}}%
\pgfusepath{clip}%
\pgfsetbuttcap%
\pgfsetroundjoin%
\pgfsetlinewidth{1.505625pt}%
\definecolor{currentstroke}{rgb}{0.000000,0.000000,0.000000}%
\pgfsetstrokecolor{currentstroke}%
\pgfsetdash{}{0pt}%
\pgfpathmoveto{\pgfqpoint{2.497749in}{4.357721in}}%
\pgfpathlineto{\pgfqpoint{2.497749in}{4.916588in}}%
\pgfusepath{stroke}%
\end{pgfscope}%
\begin{pgfscope}%
\pgfpathrectangle{\pgfqpoint{2.125000in}{4.080233in}}{\pgfqpoint{5.489583in}{0.877907in}}%
\pgfusepath{clip}%
\pgfsetbuttcap%
\pgfsetroundjoin%
\pgfsetlinewidth{1.505625pt}%
\definecolor{currentstroke}{rgb}{0.000000,0.000000,0.000000}%
\pgfsetstrokecolor{currentstroke}%
\pgfsetdash{}{0pt}%
\pgfpathmoveto{\pgfqpoint{2.620972in}{4.357721in}}%
\pgfpathlineto{\pgfqpoint{2.620972in}{4.914936in}}%
\pgfusepath{stroke}%
\end{pgfscope}%
\begin{pgfscope}%
\pgfpathrectangle{\pgfqpoint{2.125000in}{4.080233in}}{\pgfqpoint{5.489583in}{0.877907in}}%
\pgfusepath{clip}%
\pgfsetbuttcap%
\pgfsetroundjoin%
\pgfsetlinewidth{1.505625pt}%
\definecolor{currentstroke}{rgb}{0.000000,0.000000,0.000000}%
\pgfsetstrokecolor{currentstroke}%
\pgfsetdash{}{0pt}%
\pgfpathmoveto{\pgfqpoint{2.744195in}{4.357721in}}%
\pgfpathlineto{\pgfqpoint{2.744195in}{4.913320in}}%
\pgfusepath{stroke}%
\end{pgfscope}%
\begin{pgfscope}%
\pgfpathrectangle{\pgfqpoint{2.125000in}{4.080233in}}{\pgfqpoint{5.489583in}{0.877907in}}%
\pgfusepath{clip}%
\pgfsetbuttcap%
\pgfsetroundjoin%
\pgfsetlinewidth{1.505625pt}%
\definecolor{currentstroke}{rgb}{0.000000,0.000000,0.000000}%
\pgfsetstrokecolor{currentstroke}%
\pgfsetdash{}{0pt}%
\pgfpathmoveto{\pgfqpoint{2.867418in}{4.357721in}}%
\pgfpathlineto{\pgfqpoint{2.867418in}{4.911705in}}%
\pgfusepath{stroke}%
\end{pgfscope}%
\begin{pgfscope}%
\pgfpathrectangle{\pgfqpoint{2.125000in}{4.080233in}}{\pgfqpoint{5.489583in}{0.877907in}}%
\pgfusepath{clip}%
\pgfsetbuttcap%
\pgfsetroundjoin%
\pgfsetlinewidth{1.505625pt}%
\definecolor{currentstroke}{rgb}{0.000000,0.000000,0.000000}%
\pgfsetstrokecolor{currentstroke}%
\pgfsetdash{}{0pt}%
\pgfpathmoveto{\pgfqpoint{2.990641in}{4.357721in}}%
\pgfpathlineto{\pgfqpoint{2.990641in}{4.910102in}}%
\pgfusepath{stroke}%
\end{pgfscope}%
\begin{pgfscope}%
\pgfpathrectangle{\pgfqpoint{2.125000in}{4.080233in}}{\pgfqpoint{5.489583in}{0.877907in}}%
\pgfusepath{clip}%
\pgfsetbuttcap%
\pgfsetroundjoin%
\pgfsetlinewidth{1.505625pt}%
\definecolor{currentstroke}{rgb}{0.000000,0.000000,0.000000}%
\pgfsetstrokecolor{currentstroke}%
\pgfsetdash{}{0pt}%
\pgfpathmoveto{\pgfqpoint{3.113864in}{4.357721in}}%
\pgfpathlineto{\pgfqpoint{3.113864in}{4.908515in}}%
\pgfusepath{stroke}%
\end{pgfscope}%
\begin{pgfscope}%
\pgfpathrectangle{\pgfqpoint{2.125000in}{4.080233in}}{\pgfqpoint{5.489583in}{0.877907in}}%
\pgfusepath{clip}%
\pgfsetbuttcap%
\pgfsetroundjoin%
\pgfsetlinewidth{1.505625pt}%
\definecolor{currentstroke}{rgb}{0.000000,0.000000,0.000000}%
\pgfsetstrokecolor{currentstroke}%
\pgfsetdash{}{0pt}%
\pgfpathmoveto{\pgfqpoint{3.237087in}{4.357721in}}%
\pgfpathlineto{\pgfqpoint{3.237087in}{4.906909in}}%
\pgfusepath{stroke}%
\end{pgfscope}%
\begin{pgfscope}%
\pgfpathrectangle{\pgfqpoint{2.125000in}{4.080233in}}{\pgfqpoint{5.489583in}{0.877907in}}%
\pgfusepath{clip}%
\pgfsetbuttcap%
\pgfsetroundjoin%
\pgfsetlinewidth{1.505625pt}%
\definecolor{currentstroke}{rgb}{0.000000,0.000000,0.000000}%
\pgfsetstrokecolor{currentstroke}%
\pgfsetdash{}{0pt}%
\pgfpathmoveto{\pgfqpoint{3.360310in}{4.357721in}}%
\pgfpathlineto{\pgfqpoint{3.360310in}{4.905343in}}%
\pgfusepath{stroke}%
\end{pgfscope}%
\begin{pgfscope}%
\pgfpathrectangle{\pgfqpoint{2.125000in}{4.080233in}}{\pgfqpoint{5.489583in}{0.877907in}}%
\pgfusepath{clip}%
\pgfsetbuttcap%
\pgfsetroundjoin%
\pgfsetlinewidth{1.505625pt}%
\definecolor{currentstroke}{rgb}{0.000000,0.000000,0.000000}%
\pgfsetstrokecolor{currentstroke}%
\pgfsetdash{}{0pt}%
\pgfpathmoveto{\pgfqpoint{3.483533in}{4.357721in}}%
\pgfpathlineto{\pgfqpoint{3.483533in}{4.903709in}}%
\pgfusepath{stroke}%
\end{pgfscope}%
\begin{pgfscope}%
\pgfpathrectangle{\pgfqpoint{2.125000in}{4.080233in}}{\pgfqpoint{5.489583in}{0.877907in}}%
\pgfusepath{clip}%
\pgfsetbuttcap%
\pgfsetroundjoin%
\pgfsetlinewidth{1.505625pt}%
\definecolor{currentstroke}{rgb}{0.000000,0.000000,0.000000}%
\pgfsetstrokecolor{currentstroke}%
\pgfsetdash{}{0pt}%
\pgfpathmoveto{\pgfqpoint{3.606756in}{4.357721in}}%
\pgfpathlineto{\pgfqpoint{3.606756in}{4.902111in}}%
\pgfusepath{stroke}%
\end{pgfscope}%
\begin{pgfscope}%
\pgfpathrectangle{\pgfqpoint{2.125000in}{4.080233in}}{\pgfqpoint{5.489583in}{0.877907in}}%
\pgfusepath{clip}%
\pgfsetbuttcap%
\pgfsetroundjoin%
\pgfsetlinewidth{1.505625pt}%
\definecolor{currentstroke}{rgb}{0.000000,0.000000,0.000000}%
\pgfsetstrokecolor{currentstroke}%
\pgfsetdash{}{0pt}%
\pgfpathmoveto{\pgfqpoint{3.729979in}{4.357721in}}%
\pgfpathlineto{\pgfqpoint{3.729979in}{4.900528in}}%
\pgfusepath{stroke}%
\end{pgfscope}%
\begin{pgfscope}%
\pgfpathrectangle{\pgfqpoint{2.125000in}{4.080233in}}{\pgfqpoint{5.489583in}{0.877907in}}%
\pgfusepath{clip}%
\pgfsetbuttcap%
\pgfsetroundjoin%
\pgfsetlinewidth{1.505625pt}%
\definecolor{currentstroke}{rgb}{0.000000,0.000000,0.000000}%
\pgfsetstrokecolor{currentstroke}%
\pgfsetdash{}{0pt}%
\pgfpathmoveto{\pgfqpoint{3.853202in}{4.357721in}}%
\pgfpathlineto{\pgfqpoint{3.853202in}{4.898906in}}%
\pgfusepath{stroke}%
\end{pgfscope}%
\begin{pgfscope}%
\pgfpathrectangle{\pgfqpoint{2.125000in}{4.080233in}}{\pgfqpoint{5.489583in}{0.877907in}}%
\pgfusepath{clip}%
\pgfsetbuttcap%
\pgfsetroundjoin%
\pgfsetlinewidth{1.505625pt}%
\definecolor{currentstroke}{rgb}{0.000000,0.000000,0.000000}%
\pgfsetstrokecolor{currentstroke}%
\pgfsetdash{}{0pt}%
\pgfpathmoveto{\pgfqpoint{3.976425in}{4.357721in}}%
\pgfpathlineto{\pgfqpoint{3.976425in}{4.897248in}}%
\pgfusepath{stroke}%
\end{pgfscope}%
\begin{pgfscope}%
\pgfpathrectangle{\pgfqpoint{2.125000in}{4.080233in}}{\pgfqpoint{5.489583in}{0.877907in}}%
\pgfusepath{clip}%
\pgfsetbuttcap%
\pgfsetroundjoin%
\pgfsetlinewidth{1.505625pt}%
\definecolor{currentstroke}{rgb}{0.000000,0.000000,0.000000}%
\pgfsetstrokecolor{currentstroke}%
\pgfsetdash{}{0pt}%
\pgfpathmoveto{\pgfqpoint{4.099648in}{4.357721in}}%
\pgfpathlineto{\pgfqpoint{4.099648in}{4.895586in}}%
\pgfusepath{stroke}%
\end{pgfscope}%
\begin{pgfscope}%
\pgfpathrectangle{\pgfqpoint{2.125000in}{4.080233in}}{\pgfqpoint{5.489583in}{0.877907in}}%
\pgfusepath{clip}%
\pgfsetbuttcap%
\pgfsetroundjoin%
\pgfsetlinewidth{1.505625pt}%
\definecolor{currentstroke}{rgb}{0.000000,0.000000,0.000000}%
\pgfsetstrokecolor{currentstroke}%
\pgfsetdash{}{0pt}%
\pgfpathmoveto{\pgfqpoint{4.222871in}{4.357721in}}%
\pgfpathlineto{\pgfqpoint{4.222871in}{4.893948in}}%
\pgfusepath{stroke}%
\end{pgfscope}%
\begin{pgfscope}%
\pgfpathrectangle{\pgfqpoint{2.125000in}{4.080233in}}{\pgfqpoint{5.489583in}{0.877907in}}%
\pgfusepath{clip}%
\pgfsetbuttcap%
\pgfsetroundjoin%
\pgfsetlinewidth{1.505625pt}%
\definecolor{currentstroke}{rgb}{0.000000,0.000000,0.000000}%
\pgfsetstrokecolor{currentstroke}%
\pgfsetdash{}{0pt}%
\pgfpathmoveto{\pgfqpoint{4.346094in}{4.357721in}}%
\pgfpathlineto{\pgfqpoint{4.346094in}{4.892356in}}%
\pgfusepath{stroke}%
\end{pgfscope}%
\begin{pgfscope}%
\pgfpathrectangle{\pgfqpoint{2.125000in}{4.080233in}}{\pgfqpoint{5.489583in}{0.877907in}}%
\pgfusepath{clip}%
\pgfsetbuttcap%
\pgfsetroundjoin%
\pgfsetlinewidth{1.505625pt}%
\definecolor{currentstroke}{rgb}{0.000000,0.000000,0.000000}%
\pgfsetstrokecolor{currentstroke}%
\pgfsetdash{}{0pt}%
\pgfpathmoveto{\pgfqpoint{4.469317in}{4.357721in}}%
\pgfpathlineto{\pgfqpoint{4.469317in}{4.890790in}}%
\pgfusepath{stroke}%
\end{pgfscope}%
\begin{pgfscope}%
\pgfpathrectangle{\pgfqpoint{2.125000in}{4.080233in}}{\pgfqpoint{5.489583in}{0.877907in}}%
\pgfusepath{clip}%
\pgfsetbuttcap%
\pgfsetroundjoin%
\pgfsetlinewidth{1.505625pt}%
\definecolor{currentstroke}{rgb}{0.000000,0.000000,0.000000}%
\pgfsetstrokecolor{currentstroke}%
\pgfsetdash{}{0pt}%
\pgfpathmoveto{\pgfqpoint{4.592540in}{4.357721in}}%
\pgfpathlineto{\pgfqpoint{4.592540in}{4.889232in}}%
\pgfusepath{stroke}%
\end{pgfscope}%
\begin{pgfscope}%
\pgfpathrectangle{\pgfqpoint{2.125000in}{4.080233in}}{\pgfqpoint{5.489583in}{0.877907in}}%
\pgfusepath{clip}%
\pgfsetbuttcap%
\pgfsetroundjoin%
\pgfsetlinewidth{1.505625pt}%
\definecolor{currentstroke}{rgb}{0.000000,0.000000,0.000000}%
\pgfsetstrokecolor{currentstroke}%
\pgfsetdash{}{0pt}%
\pgfpathmoveto{\pgfqpoint{4.715763in}{4.357721in}}%
\pgfpathlineto{\pgfqpoint{4.715763in}{4.887672in}}%
\pgfusepath{stroke}%
\end{pgfscope}%
\begin{pgfscope}%
\pgfpathrectangle{\pgfqpoint{2.125000in}{4.080233in}}{\pgfqpoint{5.489583in}{0.877907in}}%
\pgfusepath{clip}%
\pgfsetbuttcap%
\pgfsetroundjoin%
\pgfsetlinewidth{1.505625pt}%
\definecolor{currentstroke}{rgb}{0.000000,0.000000,0.000000}%
\pgfsetstrokecolor{currentstroke}%
\pgfsetdash{}{0pt}%
\pgfpathmoveto{\pgfqpoint{4.838986in}{4.357721in}}%
\pgfpathlineto{\pgfqpoint{4.838986in}{4.886071in}}%
\pgfusepath{stroke}%
\end{pgfscope}%
\begin{pgfscope}%
\pgfpathrectangle{\pgfqpoint{2.125000in}{4.080233in}}{\pgfqpoint{5.489583in}{0.877907in}}%
\pgfusepath{clip}%
\pgfsetbuttcap%
\pgfsetroundjoin%
\pgfsetlinewidth{1.505625pt}%
\definecolor{currentstroke}{rgb}{0.000000,0.000000,0.000000}%
\pgfsetstrokecolor{currentstroke}%
\pgfsetdash{}{0pt}%
\pgfpathmoveto{\pgfqpoint{4.962209in}{4.357721in}}%
\pgfpathlineto{\pgfqpoint{4.962209in}{4.884462in}}%
\pgfusepath{stroke}%
\end{pgfscope}%
\begin{pgfscope}%
\pgfpathrectangle{\pgfqpoint{2.125000in}{4.080233in}}{\pgfqpoint{5.489583in}{0.877907in}}%
\pgfusepath{clip}%
\pgfsetbuttcap%
\pgfsetroundjoin%
\pgfsetlinewidth{1.505625pt}%
\definecolor{currentstroke}{rgb}{0.000000,0.000000,0.000000}%
\pgfsetstrokecolor{currentstroke}%
\pgfsetdash{}{0pt}%
\pgfpathmoveto{\pgfqpoint{5.085432in}{4.357721in}}%
\pgfpathlineto{\pgfqpoint{5.085432in}{4.882956in}}%
\pgfusepath{stroke}%
\end{pgfscope}%
\begin{pgfscope}%
\pgfpathrectangle{\pgfqpoint{2.125000in}{4.080233in}}{\pgfqpoint{5.489583in}{0.877907in}}%
\pgfusepath{clip}%
\pgfsetbuttcap%
\pgfsetroundjoin%
\pgfsetlinewidth{1.505625pt}%
\definecolor{currentstroke}{rgb}{0.000000,0.000000,0.000000}%
\pgfsetstrokecolor{currentstroke}%
\pgfsetdash{}{0pt}%
\pgfpathmoveto{\pgfqpoint{5.208655in}{4.357721in}}%
\pgfpathlineto{\pgfqpoint{5.208655in}{4.881433in}}%
\pgfusepath{stroke}%
\end{pgfscope}%
\begin{pgfscope}%
\pgfpathrectangle{\pgfqpoint{2.125000in}{4.080233in}}{\pgfqpoint{5.489583in}{0.877907in}}%
\pgfusepath{clip}%
\pgfsetbuttcap%
\pgfsetroundjoin%
\pgfsetlinewidth{1.505625pt}%
\definecolor{currentstroke}{rgb}{0.000000,0.000000,0.000000}%
\pgfsetstrokecolor{currentstroke}%
\pgfsetdash{}{0pt}%
\pgfpathmoveto{\pgfqpoint{5.331878in}{4.357721in}}%
\pgfpathlineto{\pgfqpoint{5.331878in}{4.879925in}}%
\pgfusepath{stroke}%
\end{pgfscope}%
\begin{pgfscope}%
\pgfpathrectangle{\pgfqpoint{2.125000in}{4.080233in}}{\pgfqpoint{5.489583in}{0.877907in}}%
\pgfusepath{clip}%
\pgfsetbuttcap%
\pgfsetroundjoin%
\pgfsetlinewidth{1.505625pt}%
\definecolor{currentstroke}{rgb}{0.000000,0.000000,0.000000}%
\pgfsetstrokecolor{currentstroke}%
\pgfsetdash{}{0pt}%
\pgfpathmoveto{\pgfqpoint{5.455101in}{4.357721in}}%
\pgfpathlineto{\pgfqpoint{5.455101in}{4.878413in}}%
\pgfusepath{stroke}%
\end{pgfscope}%
\begin{pgfscope}%
\pgfpathrectangle{\pgfqpoint{2.125000in}{4.080233in}}{\pgfqpoint{5.489583in}{0.877907in}}%
\pgfusepath{clip}%
\pgfsetbuttcap%
\pgfsetroundjoin%
\pgfsetlinewidth{1.505625pt}%
\definecolor{currentstroke}{rgb}{0.000000,0.000000,0.000000}%
\pgfsetstrokecolor{currentstroke}%
\pgfsetdash{}{0pt}%
\pgfpathmoveto{\pgfqpoint{5.578324in}{4.357721in}}%
\pgfpathlineto{\pgfqpoint{5.578324in}{4.876938in}}%
\pgfusepath{stroke}%
\end{pgfscope}%
\begin{pgfscope}%
\pgfpathrectangle{\pgfqpoint{2.125000in}{4.080233in}}{\pgfqpoint{5.489583in}{0.877907in}}%
\pgfusepath{clip}%
\pgfsetbuttcap%
\pgfsetroundjoin%
\pgfsetlinewidth{1.505625pt}%
\definecolor{currentstroke}{rgb}{0.000000,0.000000,0.000000}%
\pgfsetstrokecolor{currentstroke}%
\pgfsetdash{}{0pt}%
\pgfpathmoveto{\pgfqpoint{5.701547in}{4.357721in}}%
\pgfpathlineto{\pgfqpoint{5.701547in}{4.875525in}}%
\pgfusepath{stroke}%
\end{pgfscope}%
\begin{pgfscope}%
\pgfpathrectangle{\pgfqpoint{2.125000in}{4.080233in}}{\pgfqpoint{5.489583in}{0.877907in}}%
\pgfusepath{clip}%
\pgfsetbuttcap%
\pgfsetroundjoin%
\pgfsetlinewidth{1.505625pt}%
\definecolor{currentstroke}{rgb}{0.000000,0.000000,0.000000}%
\pgfsetstrokecolor{currentstroke}%
\pgfsetdash{}{0pt}%
\pgfpathmoveto{\pgfqpoint{5.824770in}{4.357721in}}%
\pgfpathlineto{\pgfqpoint{5.824770in}{4.874151in}}%
\pgfusepath{stroke}%
\end{pgfscope}%
\begin{pgfscope}%
\pgfpathrectangle{\pgfqpoint{2.125000in}{4.080233in}}{\pgfqpoint{5.489583in}{0.877907in}}%
\pgfusepath{clip}%
\pgfsetbuttcap%
\pgfsetroundjoin%
\pgfsetlinewidth{1.505625pt}%
\definecolor{currentstroke}{rgb}{0.000000,0.000000,0.000000}%
\pgfsetstrokecolor{currentstroke}%
\pgfsetdash{}{0pt}%
\pgfpathmoveto{\pgfqpoint{5.947993in}{4.357721in}}%
\pgfpathlineto{\pgfqpoint{5.947993in}{4.872767in}}%
\pgfusepath{stroke}%
\end{pgfscope}%
\begin{pgfscope}%
\pgfpathrectangle{\pgfqpoint{2.125000in}{4.080233in}}{\pgfqpoint{5.489583in}{0.877907in}}%
\pgfusepath{clip}%
\pgfsetbuttcap%
\pgfsetroundjoin%
\pgfsetlinewidth{1.505625pt}%
\definecolor{currentstroke}{rgb}{0.000000,0.000000,0.000000}%
\pgfsetstrokecolor{currentstroke}%
\pgfsetdash{}{0pt}%
\pgfpathmoveto{\pgfqpoint{6.071216in}{4.357721in}}%
\pgfpathlineto{\pgfqpoint{6.071216in}{4.871404in}}%
\pgfusepath{stroke}%
\end{pgfscope}%
\begin{pgfscope}%
\pgfpathrectangle{\pgfqpoint{2.125000in}{4.080233in}}{\pgfqpoint{5.489583in}{0.877907in}}%
\pgfusepath{clip}%
\pgfsetbuttcap%
\pgfsetroundjoin%
\pgfsetlinewidth{1.505625pt}%
\definecolor{currentstroke}{rgb}{0.000000,0.000000,0.000000}%
\pgfsetstrokecolor{currentstroke}%
\pgfsetdash{}{0pt}%
\pgfpathmoveto{\pgfqpoint{6.194439in}{4.357721in}}%
\pgfpathlineto{\pgfqpoint{6.194439in}{4.870086in}}%
\pgfusepath{stroke}%
\end{pgfscope}%
\begin{pgfscope}%
\pgfpathrectangle{\pgfqpoint{2.125000in}{4.080233in}}{\pgfqpoint{5.489583in}{0.877907in}}%
\pgfusepath{clip}%
\pgfsetbuttcap%
\pgfsetroundjoin%
\pgfsetlinewidth{1.505625pt}%
\definecolor{currentstroke}{rgb}{0.000000,0.000000,0.000000}%
\pgfsetstrokecolor{currentstroke}%
\pgfsetdash{}{0pt}%
\pgfpathmoveto{\pgfqpoint{6.317662in}{4.357721in}}%
\pgfpathlineto{\pgfqpoint{6.317662in}{4.868728in}}%
\pgfusepath{stroke}%
\end{pgfscope}%
\begin{pgfscope}%
\pgfpathrectangle{\pgfqpoint{2.125000in}{4.080233in}}{\pgfqpoint{5.489583in}{0.877907in}}%
\pgfusepath{clip}%
\pgfsetbuttcap%
\pgfsetroundjoin%
\pgfsetlinewidth{1.505625pt}%
\definecolor{currentstroke}{rgb}{0.000000,0.000000,0.000000}%
\pgfsetstrokecolor{currentstroke}%
\pgfsetdash{}{0pt}%
\pgfpathmoveto{\pgfqpoint{6.440885in}{4.357721in}}%
\pgfpathlineto{\pgfqpoint{6.440885in}{4.867418in}}%
\pgfusepath{stroke}%
\end{pgfscope}%
\begin{pgfscope}%
\pgfpathrectangle{\pgfqpoint{2.125000in}{4.080233in}}{\pgfqpoint{5.489583in}{0.877907in}}%
\pgfusepath{clip}%
\pgfsetbuttcap%
\pgfsetroundjoin%
\pgfsetlinewidth{1.505625pt}%
\definecolor{currentstroke}{rgb}{0.000000,0.000000,0.000000}%
\pgfsetstrokecolor{currentstroke}%
\pgfsetdash{}{0pt}%
\pgfpathmoveto{\pgfqpoint{6.564108in}{4.357721in}}%
\pgfpathlineto{\pgfqpoint{6.564108in}{4.866053in}}%
\pgfusepath{stroke}%
\end{pgfscope}%
\begin{pgfscope}%
\pgfpathrectangle{\pgfqpoint{2.125000in}{4.080233in}}{\pgfqpoint{5.489583in}{0.877907in}}%
\pgfusepath{clip}%
\pgfsetbuttcap%
\pgfsetroundjoin%
\pgfsetlinewidth{1.505625pt}%
\definecolor{currentstroke}{rgb}{0.000000,0.000000,0.000000}%
\pgfsetstrokecolor{currentstroke}%
\pgfsetdash{}{0pt}%
\pgfpathmoveto{\pgfqpoint{6.687330in}{4.357721in}}%
\pgfpathlineto{\pgfqpoint{6.687330in}{4.864713in}}%
\pgfusepath{stroke}%
\end{pgfscope}%
\begin{pgfscope}%
\pgfpathrectangle{\pgfqpoint{2.125000in}{4.080233in}}{\pgfqpoint{5.489583in}{0.877907in}}%
\pgfusepath{clip}%
\pgfsetbuttcap%
\pgfsetroundjoin%
\pgfsetlinewidth{1.505625pt}%
\definecolor{currentstroke}{rgb}{0.000000,0.000000,0.000000}%
\pgfsetstrokecolor{currentstroke}%
\pgfsetdash{}{0pt}%
\pgfpathmoveto{\pgfqpoint{6.810553in}{4.357721in}}%
\pgfpathlineto{\pgfqpoint{6.810553in}{4.863378in}}%
\pgfusepath{stroke}%
\end{pgfscope}%
\begin{pgfscope}%
\pgfpathrectangle{\pgfqpoint{2.125000in}{4.080233in}}{\pgfqpoint{5.489583in}{0.877907in}}%
\pgfusepath{clip}%
\pgfsetbuttcap%
\pgfsetroundjoin%
\pgfsetlinewidth{1.505625pt}%
\definecolor{currentstroke}{rgb}{0.000000,0.000000,0.000000}%
\pgfsetstrokecolor{currentstroke}%
\pgfsetdash{}{0pt}%
\pgfpathmoveto{\pgfqpoint{6.933776in}{4.357721in}}%
\pgfpathlineto{\pgfqpoint{6.933776in}{4.862067in}}%
\pgfusepath{stroke}%
\end{pgfscope}%
\begin{pgfscope}%
\pgfpathrectangle{\pgfqpoint{2.125000in}{4.080233in}}{\pgfqpoint{5.489583in}{0.877907in}}%
\pgfusepath{clip}%
\pgfsetbuttcap%
\pgfsetroundjoin%
\pgfsetlinewidth{1.505625pt}%
\definecolor{currentstroke}{rgb}{0.000000,0.000000,0.000000}%
\pgfsetstrokecolor{currentstroke}%
\pgfsetdash{}{0pt}%
\pgfpathmoveto{\pgfqpoint{7.056999in}{4.357721in}}%
\pgfpathlineto{\pgfqpoint{7.056999in}{4.860738in}}%
\pgfusepath{stroke}%
\end{pgfscope}%
\begin{pgfscope}%
\pgfpathrectangle{\pgfqpoint{2.125000in}{4.080233in}}{\pgfqpoint{5.489583in}{0.877907in}}%
\pgfusepath{clip}%
\pgfsetbuttcap%
\pgfsetroundjoin%
\pgfsetlinewidth{1.505625pt}%
\definecolor{currentstroke}{rgb}{0.000000,0.000000,0.000000}%
\pgfsetstrokecolor{currentstroke}%
\pgfsetdash{}{0pt}%
\pgfpathmoveto{\pgfqpoint{7.180222in}{4.357721in}}%
\pgfpathlineto{\pgfqpoint{7.180222in}{4.859445in}}%
\pgfusepath{stroke}%
\end{pgfscope}%
\begin{pgfscope}%
\pgfpathrectangle{\pgfqpoint{2.125000in}{4.080233in}}{\pgfqpoint{5.489583in}{0.877907in}}%
\pgfusepath{clip}%
\pgfsetbuttcap%
\pgfsetroundjoin%
\pgfsetlinewidth{1.505625pt}%
\definecolor{currentstroke}{rgb}{0.000000,0.000000,0.000000}%
\pgfsetstrokecolor{currentstroke}%
\pgfsetdash{}{0pt}%
\pgfpathmoveto{\pgfqpoint{7.303445in}{4.357721in}}%
\pgfpathlineto{\pgfqpoint{7.303445in}{4.858133in}}%
\pgfusepath{stroke}%
\end{pgfscope}%
\begin{pgfscope}%
\pgfpathrectangle{\pgfqpoint{2.125000in}{4.080233in}}{\pgfqpoint{5.489583in}{0.877907in}}%
\pgfusepath{clip}%
\pgfsetroundcap%
\pgfsetroundjoin%
\pgfsetlinewidth{1.505625pt}%
\definecolor{currentstroke}{rgb}{0.121569,0.466667,0.705882}%
\pgfsetstrokecolor{currentstroke}%
\pgfsetdash{}{0pt}%
\pgfpathmoveto{\pgfqpoint{2.125000in}{4.357721in}}%
\pgfpathlineto{\pgfqpoint{7.614583in}{4.357721in}}%
\pgfusepath{stroke}%
\end{pgfscope}%
\begin{pgfscope}%
\pgfpathrectangle{\pgfqpoint{2.125000in}{4.080233in}}{\pgfqpoint{5.489583in}{0.877907in}}%
\pgfusepath{clip}%
\pgfsetbuttcap%
\pgfsetroundjoin%
\definecolor{currentfill}{rgb}{0.121569,0.466667,0.705882}%
\pgfsetfillcolor{currentfill}%
\pgfsetlinewidth{1.003750pt}%
\definecolor{currentstroke}{rgb}{0.121569,0.466667,0.705882}%
\pgfsetstrokecolor{currentstroke}%
\pgfsetdash{}{0pt}%
\pgfsys@defobject{currentmarker}{\pgfqpoint{-0.034722in}{-0.034722in}}{\pgfqpoint{0.034722in}{0.034722in}}{%
\pgfpathmoveto{\pgfqpoint{0.000000in}{-0.034722in}}%
\pgfpathcurveto{\pgfqpoint{0.009208in}{-0.034722in}}{\pgfqpoint{0.018041in}{-0.031064in}}{\pgfqpoint{0.024552in}{-0.024552in}}%
\pgfpathcurveto{\pgfqpoint{0.031064in}{-0.018041in}}{\pgfqpoint{0.034722in}{-0.009208in}}{\pgfqpoint{0.034722in}{0.000000in}}%
\pgfpathcurveto{\pgfqpoint{0.034722in}{0.009208in}}{\pgfqpoint{0.031064in}{0.018041in}}{\pgfqpoint{0.024552in}{0.024552in}}%
\pgfpathcurveto{\pgfqpoint{0.018041in}{0.031064in}}{\pgfqpoint{0.009208in}{0.034722in}}{\pgfqpoint{0.000000in}{0.034722in}}%
\pgfpathcurveto{\pgfqpoint{-0.009208in}{0.034722in}}{\pgfqpoint{-0.018041in}{0.031064in}}{\pgfqpoint{-0.024552in}{0.024552in}}%
\pgfpathcurveto{\pgfqpoint{-0.031064in}{0.018041in}}{\pgfqpoint{-0.034722in}{0.009208in}}{\pgfqpoint{-0.034722in}{0.000000in}}%
\pgfpathcurveto{\pgfqpoint{-0.034722in}{-0.009208in}}{\pgfqpoint{-0.031064in}{-0.018041in}}{\pgfqpoint{-0.024552in}{-0.024552in}}%
\pgfpathcurveto{\pgfqpoint{-0.018041in}{-0.031064in}}{\pgfqpoint{-0.009208in}{-0.034722in}}{\pgfqpoint{0.000000in}{-0.034722in}}%
\pgfpathclose%
\pgfusepath{stroke,fill}%
}%
\begin{pgfscope}%
\pgfsys@transformshift{2.374527in}{4.918235in}%
\pgfsys@useobject{currentmarker}{}%
\end{pgfscope}%
\begin{pgfscope}%
\pgfsys@transformshift{2.497749in}{4.916588in}%
\pgfsys@useobject{currentmarker}{}%
\end{pgfscope}%
\begin{pgfscope}%
\pgfsys@transformshift{2.620972in}{4.914936in}%
\pgfsys@useobject{currentmarker}{}%
\end{pgfscope}%
\begin{pgfscope}%
\pgfsys@transformshift{2.744195in}{4.913320in}%
\pgfsys@useobject{currentmarker}{}%
\end{pgfscope}%
\begin{pgfscope}%
\pgfsys@transformshift{2.867418in}{4.911705in}%
\pgfsys@useobject{currentmarker}{}%
\end{pgfscope}%
\begin{pgfscope}%
\pgfsys@transformshift{2.990641in}{4.910102in}%
\pgfsys@useobject{currentmarker}{}%
\end{pgfscope}%
\begin{pgfscope}%
\pgfsys@transformshift{3.113864in}{4.908515in}%
\pgfsys@useobject{currentmarker}{}%
\end{pgfscope}%
\begin{pgfscope}%
\pgfsys@transformshift{3.237087in}{4.906909in}%
\pgfsys@useobject{currentmarker}{}%
\end{pgfscope}%
\begin{pgfscope}%
\pgfsys@transformshift{3.360310in}{4.905343in}%
\pgfsys@useobject{currentmarker}{}%
\end{pgfscope}%
\begin{pgfscope}%
\pgfsys@transformshift{3.483533in}{4.903709in}%
\pgfsys@useobject{currentmarker}{}%
\end{pgfscope}%
\begin{pgfscope}%
\pgfsys@transformshift{3.606756in}{4.902111in}%
\pgfsys@useobject{currentmarker}{}%
\end{pgfscope}%
\begin{pgfscope}%
\pgfsys@transformshift{3.729979in}{4.900528in}%
\pgfsys@useobject{currentmarker}{}%
\end{pgfscope}%
\begin{pgfscope}%
\pgfsys@transformshift{3.853202in}{4.898906in}%
\pgfsys@useobject{currentmarker}{}%
\end{pgfscope}%
\begin{pgfscope}%
\pgfsys@transformshift{3.976425in}{4.897248in}%
\pgfsys@useobject{currentmarker}{}%
\end{pgfscope}%
\begin{pgfscope}%
\pgfsys@transformshift{4.099648in}{4.895586in}%
\pgfsys@useobject{currentmarker}{}%
\end{pgfscope}%
\begin{pgfscope}%
\pgfsys@transformshift{4.222871in}{4.893948in}%
\pgfsys@useobject{currentmarker}{}%
\end{pgfscope}%
\begin{pgfscope}%
\pgfsys@transformshift{4.346094in}{4.892356in}%
\pgfsys@useobject{currentmarker}{}%
\end{pgfscope}%
\begin{pgfscope}%
\pgfsys@transformshift{4.469317in}{4.890790in}%
\pgfsys@useobject{currentmarker}{}%
\end{pgfscope}%
\begin{pgfscope}%
\pgfsys@transformshift{4.592540in}{4.889232in}%
\pgfsys@useobject{currentmarker}{}%
\end{pgfscope}%
\begin{pgfscope}%
\pgfsys@transformshift{4.715763in}{4.887672in}%
\pgfsys@useobject{currentmarker}{}%
\end{pgfscope}%
\begin{pgfscope}%
\pgfsys@transformshift{4.838986in}{4.886071in}%
\pgfsys@useobject{currentmarker}{}%
\end{pgfscope}%
\begin{pgfscope}%
\pgfsys@transformshift{4.962209in}{4.884462in}%
\pgfsys@useobject{currentmarker}{}%
\end{pgfscope}%
\begin{pgfscope}%
\pgfsys@transformshift{5.085432in}{4.882956in}%
\pgfsys@useobject{currentmarker}{}%
\end{pgfscope}%
\begin{pgfscope}%
\pgfsys@transformshift{5.208655in}{4.881433in}%
\pgfsys@useobject{currentmarker}{}%
\end{pgfscope}%
\begin{pgfscope}%
\pgfsys@transformshift{5.331878in}{4.879925in}%
\pgfsys@useobject{currentmarker}{}%
\end{pgfscope}%
\begin{pgfscope}%
\pgfsys@transformshift{5.455101in}{4.878413in}%
\pgfsys@useobject{currentmarker}{}%
\end{pgfscope}%
\begin{pgfscope}%
\pgfsys@transformshift{5.578324in}{4.876938in}%
\pgfsys@useobject{currentmarker}{}%
\end{pgfscope}%
\begin{pgfscope}%
\pgfsys@transformshift{5.701547in}{4.875525in}%
\pgfsys@useobject{currentmarker}{}%
\end{pgfscope}%
\begin{pgfscope}%
\pgfsys@transformshift{5.824770in}{4.874151in}%
\pgfsys@useobject{currentmarker}{}%
\end{pgfscope}%
\begin{pgfscope}%
\pgfsys@transformshift{5.947993in}{4.872767in}%
\pgfsys@useobject{currentmarker}{}%
\end{pgfscope}%
\begin{pgfscope}%
\pgfsys@transformshift{6.071216in}{4.871404in}%
\pgfsys@useobject{currentmarker}{}%
\end{pgfscope}%
\begin{pgfscope}%
\pgfsys@transformshift{6.194439in}{4.870086in}%
\pgfsys@useobject{currentmarker}{}%
\end{pgfscope}%
\begin{pgfscope}%
\pgfsys@transformshift{6.317662in}{4.868728in}%
\pgfsys@useobject{currentmarker}{}%
\end{pgfscope}%
\begin{pgfscope}%
\pgfsys@transformshift{6.440885in}{4.867418in}%
\pgfsys@useobject{currentmarker}{}%
\end{pgfscope}%
\begin{pgfscope}%
\pgfsys@transformshift{6.564108in}{4.866053in}%
\pgfsys@useobject{currentmarker}{}%
\end{pgfscope}%
\begin{pgfscope}%
\pgfsys@transformshift{6.687330in}{4.864713in}%
\pgfsys@useobject{currentmarker}{}%
\end{pgfscope}%
\begin{pgfscope}%
\pgfsys@transformshift{6.810553in}{4.863378in}%
\pgfsys@useobject{currentmarker}{}%
\end{pgfscope}%
\begin{pgfscope}%
\pgfsys@transformshift{6.933776in}{4.862067in}%
\pgfsys@useobject{currentmarker}{}%
\end{pgfscope}%
\begin{pgfscope}%
\pgfsys@transformshift{7.056999in}{4.860738in}%
\pgfsys@useobject{currentmarker}{}%
\end{pgfscope}%
\begin{pgfscope}%
\pgfsys@transformshift{7.180222in}{4.859445in}%
\pgfsys@useobject{currentmarker}{}%
\end{pgfscope}%
\begin{pgfscope}%
\pgfsys@transformshift{7.303445in}{4.858133in}%
\pgfsys@useobject{currentmarker}{}%
\end{pgfscope}%
\end{pgfscope}%
\begin{pgfscope}%
\pgfsetrectcap%
\pgfsetmiterjoin%
\pgfsetlinewidth{0.803000pt}%
\definecolor{currentstroke}{rgb}{1.000000,1.000000,1.000000}%
\pgfsetstrokecolor{currentstroke}%
\pgfsetdash{}{0pt}%
\pgfpathmoveto{\pgfqpoint{2.125000in}{4.080233in}}%
\pgfpathlineto{\pgfqpoint{2.125000in}{4.958140in}}%
\pgfusepath{stroke}%
\end{pgfscope}%
\begin{pgfscope}%
\pgfsetrectcap%
\pgfsetmiterjoin%
\pgfsetlinewidth{0.803000pt}%
\definecolor{currentstroke}{rgb}{1.000000,1.000000,1.000000}%
\pgfsetstrokecolor{currentstroke}%
\pgfsetdash{}{0pt}%
\pgfpathmoveto{\pgfqpoint{7.614583in}{4.080233in}}%
\pgfpathlineto{\pgfqpoint{7.614583in}{4.958140in}}%
\pgfusepath{stroke}%
\end{pgfscope}%
\begin{pgfscope}%
\pgfsetrectcap%
\pgfsetmiterjoin%
\pgfsetlinewidth{0.803000pt}%
\definecolor{currentstroke}{rgb}{1.000000,1.000000,1.000000}%
\pgfsetstrokecolor{currentstroke}%
\pgfsetdash{}{0pt}%
\pgfpathmoveto{\pgfqpoint{2.125000in}{4.080233in}}%
\pgfpathlineto{\pgfqpoint{7.614583in}{4.080233in}}%
\pgfusepath{stroke}%
\end{pgfscope}%
\begin{pgfscope}%
\pgfsetrectcap%
\pgfsetmiterjoin%
\pgfsetlinewidth{0.803000pt}%
\definecolor{currentstroke}{rgb}{1.000000,1.000000,1.000000}%
\pgfsetstrokecolor{currentstroke}%
\pgfsetdash{}{0pt}%
\pgfpathmoveto{\pgfqpoint{2.125000in}{4.958140in}}%
\pgfpathlineto{\pgfqpoint{7.614583in}{4.958140in}}%
\pgfusepath{stroke}%
\end{pgfscope}%
\begin{pgfscope}%
\definecolor{textcolor}{rgb}{0.150000,0.150000,0.150000}%
\pgfsetstrokecolor{textcolor}%
\pgfsetfillcolor{textcolor}%
\pgftext[x=4.869792in,y=5.041473in,,base]{\color{textcolor}\rmfamily\fontsize{16.800000}{20.160000}\selectfont Autocorrelation}%
\end{pgfscope}%
\begin{pgfscope}%
\pgfsetbuttcap%
\pgfsetmiterjoin%
\definecolor{currentfill}{rgb}{0.917647,0.917647,0.949020}%
\pgfsetfillcolor{currentfill}%
\pgfsetlinewidth{0.000000pt}%
\definecolor{currentstroke}{rgb}{0.000000,0.000000,0.000000}%
\pgfsetstrokecolor{currentstroke}%
\pgfsetstrokeopacity{0.000000}%
\pgfsetdash{}{0pt}%
\pgfpathmoveto{\pgfqpoint{9.810417in}{4.080233in}}%
\pgfpathlineto{\pgfqpoint{15.300000in}{4.080233in}}%
\pgfpathlineto{\pgfqpoint{15.300000in}{4.958140in}}%
\pgfpathlineto{\pgfqpoint{9.810417in}{4.958140in}}%
\pgfpathclose%
\pgfusepath{fill}%
\end{pgfscope}%
\begin{pgfscope}%
\pgfpathrectangle{\pgfqpoint{9.810417in}{4.080233in}}{\pgfqpoint{5.489583in}{0.877907in}}%
\pgfusepath{clip}%
\pgfsetroundcap%
\pgfsetroundjoin%
\pgfsetlinewidth{0.803000pt}%
\definecolor{currentstroke}{rgb}{1.000000,1.000000,1.000000}%
\pgfsetstrokecolor{currentstroke}%
\pgfsetdash{}{0pt}%
\pgfpathmoveto{\pgfqpoint{10.059943in}{4.080233in}}%
\pgfpathlineto{\pgfqpoint{10.059943in}{4.958140in}}%
\pgfusepath{stroke}%
\end{pgfscope}%
\begin{pgfscope}%
\definecolor{textcolor}{rgb}{0.150000,0.150000,0.150000}%
\pgfsetstrokecolor{textcolor}%
\pgfsetfillcolor{textcolor}%
\pgftext[x=10.059943in,y=3.983010in,,top]{\color{textcolor}\rmfamily\fontsize{14.000000}{16.800000}\selectfont 0}%
\end{pgfscope}%
\begin{pgfscope}%
\pgfpathrectangle{\pgfqpoint{9.810417in}{4.080233in}}{\pgfqpoint{5.489583in}{0.877907in}}%
\pgfusepath{clip}%
\pgfsetroundcap%
\pgfsetroundjoin%
\pgfsetlinewidth{0.803000pt}%
\definecolor{currentstroke}{rgb}{1.000000,1.000000,1.000000}%
\pgfsetstrokecolor{currentstroke}%
\pgfsetdash{}{0pt}%
\pgfpathmoveto{\pgfqpoint{10.676058in}{4.080233in}}%
\pgfpathlineto{\pgfqpoint{10.676058in}{4.958140in}}%
\pgfusepath{stroke}%
\end{pgfscope}%
\begin{pgfscope}%
\definecolor{textcolor}{rgb}{0.150000,0.150000,0.150000}%
\pgfsetstrokecolor{textcolor}%
\pgfsetfillcolor{textcolor}%
\pgftext[x=10.676058in,y=3.983010in,,top]{\color{textcolor}\rmfamily\fontsize{14.000000}{16.800000}\selectfont 5}%
\end{pgfscope}%
\begin{pgfscope}%
\pgfpathrectangle{\pgfqpoint{9.810417in}{4.080233in}}{\pgfqpoint{5.489583in}{0.877907in}}%
\pgfusepath{clip}%
\pgfsetroundcap%
\pgfsetroundjoin%
\pgfsetlinewidth{0.803000pt}%
\definecolor{currentstroke}{rgb}{1.000000,1.000000,1.000000}%
\pgfsetstrokecolor{currentstroke}%
\pgfsetdash{}{0pt}%
\pgfpathmoveto{\pgfqpoint{11.292173in}{4.080233in}}%
\pgfpathlineto{\pgfqpoint{11.292173in}{4.958140in}}%
\pgfusepath{stroke}%
\end{pgfscope}%
\begin{pgfscope}%
\definecolor{textcolor}{rgb}{0.150000,0.150000,0.150000}%
\pgfsetstrokecolor{textcolor}%
\pgfsetfillcolor{textcolor}%
\pgftext[x=11.292173in,y=3.983010in,,top]{\color{textcolor}\rmfamily\fontsize{14.000000}{16.800000}\selectfont 10}%
\end{pgfscope}%
\begin{pgfscope}%
\pgfpathrectangle{\pgfqpoint{9.810417in}{4.080233in}}{\pgfqpoint{5.489583in}{0.877907in}}%
\pgfusepath{clip}%
\pgfsetroundcap%
\pgfsetroundjoin%
\pgfsetlinewidth{0.803000pt}%
\definecolor{currentstroke}{rgb}{1.000000,1.000000,1.000000}%
\pgfsetstrokecolor{currentstroke}%
\pgfsetdash{}{0pt}%
\pgfpathmoveto{\pgfqpoint{11.908288in}{4.080233in}}%
\pgfpathlineto{\pgfqpoint{11.908288in}{4.958140in}}%
\pgfusepath{stroke}%
\end{pgfscope}%
\begin{pgfscope}%
\definecolor{textcolor}{rgb}{0.150000,0.150000,0.150000}%
\pgfsetstrokecolor{textcolor}%
\pgfsetfillcolor{textcolor}%
\pgftext[x=11.908288in,y=3.983010in,,top]{\color{textcolor}\rmfamily\fontsize{14.000000}{16.800000}\selectfont 15}%
\end{pgfscope}%
\begin{pgfscope}%
\pgfpathrectangle{\pgfqpoint{9.810417in}{4.080233in}}{\pgfqpoint{5.489583in}{0.877907in}}%
\pgfusepath{clip}%
\pgfsetroundcap%
\pgfsetroundjoin%
\pgfsetlinewidth{0.803000pt}%
\definecolor{currentstroke}{rgb}{1.000000,1.000000,1.000000}%
\pgfsetstrokecolor{currentstroke}%
\pgfsetdash{}{0pt}%
\pgfpathmoveto{\pgfqpoint{12.524403in}{4.080233in}}%
\pgfpathlineto{\pgfqpoint{12.524403in}{4.958140in}}%
\pgfusepath{stroke}%
\end{pgfscope}%
\begin{pgfscope}%
\definecolor{textcolor}{rgb}{0.150000,0.150000,0.150000}%
\pgfsetstrokecolor{textcolor}%
\pgfsetfillcolor{textcolor}%
\pgftext[x=12.524403in,y=3.983010in,,top]{\color{textcolor}\rmfamily\fontsize{14.000000}{16.800000}\selectfont 20}%
\end{pgfscope}%
\begin{pgfscope}%
\pgfpathrectangle{\pgfqpoint{9.810417in}{4.080233in}}{\pgfqpoint{5.489583in}{0.877907in}}%
\pgfusepath{clip}%
\pgfsetroundcap%
\pgfsetroundjoin%
\pgfsetlinewidth{0.803000pt}%
\definecolor{currentstroke}{rgb}{1.000000,1.000000,1.000000}%
\pgfsetstrokecolor{currentstroke}%
\pgfsetdash{}{0pt}%
\pgfpathmoveto{\pgfqpoint{13.140517in}{4.080233in}}%
\pgfpathlineto{\pgfqpoint{13.140517in}{4.958140in}}%
\pgfusepath{stroke}%
\end{pgfscope}%
\begin{pgfscope}%
\definecolor{textcolor}{rgb}{0.150000,0.150000,0.150000}%
\pgfsetstrokecolor{textcolor}%
\pgfsetfillcolor{textcolor}%
\pgftext[x=13.140517in,y=3.983010in,,top]{\color{textcolor}\rmfamily\fontsize{14.000000}{16.800000}\selectfont 25}%
\end{pgfscope}%
\begin{pgfscope}%
\pgfpathrectangle{\pgfqpoint{9.810417in}{4.080233in}}{\pgfqpoint{5.489583in}{0.877907in}}%
\pgfusepath{clip}%
\pgfsetroundcap%
\pgfsetroundjoin%
\pgfsetlinewidth{0.803000pt}%
\definecolor{currentstroke}{rgb}{1.000000,1.000000,1.000000}%
\pgfsetstrokecolor{currentstroke}%
\pgfsetdash{}{0pt}%
\pgfpathmoveto{\pgfqpoint{13.756632in}{4.080233in}}%
\pgfpathlineto{\pgfqpoint{13.756632in}{4.958140in}}%
\pgfusepath{stroke}%
\end{pgfscope}%
\begin{pgfscope}%
\definecolor{textcolor}{rgb}{0.150000,0.150000,0.150000}%
\pgfsetstrokecolor{textcolor}%
\pgfsetfillcolor{textcolor}%
\pgftext[x=13.756632in,y=3.983010in,,top]{\color{textcolor}\rmfamily\fontsize{14.000000}{16.800000}\selectfont 30}%
\end{pgfscope}%
\begin{pgfscope}%
\pgfpathrectangle{\pgfqpoint{9.810417in}{4.080233in}}{\pgfqpoint{5.489583in}{0.877907in}}%
\pgfusepath{clip}%
\pgfsetroundcap%
\pgfsetroundjoin%
\pgfsetlinewidth{0.803000pt}%
\definecolor{currentstroke}{rgb}{1.000000,1.000000,1.000000}%
\pgfsetstrokecolor{currentstroke}%
\pgfsetdash{}{0pt}%
\pgfpathmoveto{\pgfqpoint{14.372747in}{4.080233in}}%
\pgfpathlineto{\pgfqpoint{14.372747in}{4.958140in}}%
\pgfusepath{stroke}%
\end{pgfscope}%
\begin{pgfscope}%
\definecolor{textcolor}{rgb}{0.150000,0.150000,0.150000}%
\pgfsetstrokecolor{textcolor}%
\pgfsetfillcolor{textcolor}%
\pgftext[x=14.372747in,y=3.983010in,,top]{\color{textcolor}\rmfamily\fontsize{14.000000}{16.800000}\selectfont 35}%
\end{pgfscope}%
\begin{pgfscope}%
\pgfpathrectangle{\pgfqpoint{9.810417in}{4.080233in}}{\pgfqpoint{5.489583in}{0.877907in}}%
\pgfusepath{clip}%
\pgfsetroundcap%
\pgfsetroundjoin%
\pgfsetlinewidth{0.803000pt}%
\definecolor{currentstroke}{rgb}{1.000000,1.000000,1.000000}%
\pgfsetstrokecolor{currentstroke}%
\pgfsetdash{}{0pt}%
\pgfpathmoveto{\pgfqpoint{14.988862in}{4.080233in}}%
\pgfpathlineto{\pgfqpoint{14.988862in}{4.958140in}}%
\pgfusepath{stroke}%
\end{pgfscope}%
\begin{pgfscope}%
\definecolor{textcolor}{rgb}{0.150000,0.150000,0.150000}%
\pgfsetstrokecolor{textcolor}%
\pgfsetfillcolor{textcolor}%
\pgftext[x=14.988862in,y=3.983010in,,top]{\color{textcolor}\rmfamily\fontsize{14.000000}{16.800000}\selectfont 40}%
\end{pgfscope}%
\begin{pgfscope}%
\pgfpathrectangle{\pgfqpoint{9.810417in}{4.080233in}}{\pgfqpoint{5.489583in}{0.877907in}}%
\pgfusepath{clip}%
\pgfsetroundcap%
\pgfsetroundjoin%
\pgfsetlinewidth{0.803000pt}%
\definecolor{currentstroke}{rgb}{1.000000,1.000000,1.000000}%
\pgfsetstrokecolor{currentstroke}%
\pgfsetdash{}{0pt}%
\pgfpathmoveto{\pgfqpoint{9.810417in}{4.158471in}}%
\pgfpathlineto{\pgfqpoint{15.300000in}{4.158471in}}%
\pgfusepath{stroke}%
\end{pgfscope}%
\begin{pgfscope}%
\definecolor{textcolor}{rgb}{0.150000,0.150000,0.150000}%
\pgfsetstrokecolor{textcolor}%
\pgfsetfillcolor{textcolor}%
\pgftext[x=9.589483in,y=4.084605in,left,base]{\color{textcolor}\rmfamily\fontsize{14.000000}{16.800000}\selectfont 0}%
\end{pgfscope}%
\begin{pgfscope}%
\pgfpathrectangle{\pgfqpoint{9.810417in}{4.080233in}}{\pgfqpoint{5.489583in}{0.877907in}}%
\pgfusepath{clip}%
\pgfsetroundcap%
\pgfsetroundjoin%
\pgfsetlinewidth{0.803000pt}%
\definecolor{currentstroke}{rgb}{1.000000,1.000000,1.000000}%
\pgfsetstrokecolor{currentstroke}%
\pgfsetdash{}{0pt}%
\pgfpathmoveto{\pgfqpoint{9.810417in}{4.918235in}}%
\pgfpathlineto{\pgfqpoint{15.300000in}{4.918235in}}%
\pgfusepath{stroke}%
\end{pgfscope}%
\begin{pgfscope}%
\definecolor{textcolor}{rgb}{0.150000,0.150000,0.150000}%
\pgfsetstrokecolor{textcolor}%
\pgfsetfillcolor{textcolor}%
\pgftext[x=9.589483in,y=4.844369in,left,base]{\color{textcolor}\rmfamily\fontsize{14.000000}{16.800000}\selectfont 1}%
\end{pgfscope}%
\begin{pgfscope}%
\pgfpathrectangle{\pgfqpoint{9.810417in}{4.080233in}}{\pgfqpoint{5.489583in}{0.877907in}}%
\pgfusepath{clip}%
\pgfsetbuttcap%
\pgfsetroundjoin%
\definecolor{currentfill}{rgb}{0.121569,0.466667,0.705882}%
\pgfsetfillcolor{currentfill}%
\pgfsetfillopacity{0.250000}%
\pgfsetlinewidth{1.003750pt}%
\definecolor{currentstroke}{rgb}{1.000000,1.000000,1.000000}%
\pgfsetstrokecolor{currentstroke}%
\pgfsetstrokeopacity{0.250000}%
\pgfsetdash{}{0pt}%
\pgfpathmoveto{\pgfqpoint{10.121555in}{4.196805in}}%
\pgfpathlineto{\pgfqpoint{10.121555in}{4.120137in}}%
\pgfpathlineto{\pgfqpoint{10.306389in}{4.120137in}}%
\pgfpathlineto{\pgfqpoint{10.429612in}{4.120137in}}%
\pgfpathlineto{\pgfqpoint{10.552835in}{4.120137in}}%
\pgfpathlineto{\pgfqpoint{10.676058in}{4.120137in}}%
\pgfpathlineto{\pgfqpoint{10.799281in}{4.120137in}}%
\pgfpathlineto{\pgfqpoint{10.922504in}{4.120137in}}%
\pgfpathlineto{\pgfqpoint{11.045727in}{4.120137in}}%
\pgfpathlineto{\pgfqpoint{11.168950in}{4.120137in}}%
\pgfpathlineto{\pgfqpoint{11.292173in}{4.120137in}}%
\pgfpathlineto{\pgfqpoint{11.415396in}{4.120137in}}%
\pgfpathlineto{\pgfqpoint{11.538619in}{4.120137in}}%
\pgfpathlineto{\pgfqpoint{11.661842in}{4.120137in}}%
\pgfpathlineto{\pgfqpoint{11.785065in}{4.120137in}}%
\pgfpathlineto{\pgfqpoint{11.908288in}{4.120137in}}%
\pgfpathlineto{\pgfqpoint{12.031511in}{4.120137in}}%
\pgfpathlineto{\pgfqpoint{12.154734in}{4.120137in}}%
\pgfpathlineto{\pgfqpoint{12.277957in}{4.120137in}}%
\pgfpathlineto{\pgfqpoint{12.401180in}{4.120137in}}%
\pgfpathlineto{\pgfqpoint{12.524403in}{4.120137in}}%
\pgfpathlineto{\pgfqpoint{12.647626in}{4.120137in}}%
\pgfpathlineto{\pgfqpoint{12.770849in}{4.120137in}}%
\pgfpathlineto{\pgfqpoint{12.894072in}{4.120137in}}%
\pgfpathlineto{\pgfqpoint{13.017294in}{4.120137in}}%
\pgfpathlineto{\pgfqpoint{13.140517in}{4.120137in}}%
\pgfpathlineto{\pgfqpoint{13.263740in}{4.120137in}}%
\pgfpathlineto{\pgfqpoint{13.386963in}{4.120137in}}%
\pgfpathlineto{\pgfqpoint{13.510186in}{4.120137in}}%
\pgfpathlineto{\pgfqpoint{13.633409in}{4.120137in}}%
\pgfpathlineto{\pgfqpoint{13.756632in}{4.120137in}}%
\pgfpathlineto{\pgfqpoint{13.879855in}{4.120137in}}%
\pgfpathlineto{\pgfqpoint{14.003078in}{4.120137in}}%
\pgfpathlineto{\pgfqpoint{14.126301in}{4.120137in}}%
\pgfpathlineto{\pgfqpoint{14.249524in}{4.120137in}}%
\pgfpathlineto{\pgfqpoint{14.372747in}{4.120137in}}%
\pgfpathlineto{\pgfqpoint{14.495970in}{4.120137in}}%
\pgfpathlineto{\pgfqpoint{14.619193in}{4.120137in}}%
\pgfpathlineto{\pgfqpoint{14.742416in}{4.120137in}}%
\pgfpathlineto{\pgfqpoint{14.865639in}{4.120137in}}%
\pgfpathlineto{\pgfqpoint{15.050473in}{4.120137in}}%
\pgfpathlineto{\pgfqpoint{15.050473in}{4.196805in}}%
\pgfpathlineto{\pgfqpoint{15.050473in}{4.196805in}}%
\pgfpathlineto{\pgfqpoint{14.865639in}{4.196805in}}%
\pgfpathlineto{\pgfqpoint{14.742416in}{4.196805in}}%
\pgfpathlineto{\pgfqpoint{14.619193in}{4.196805in}}%
\pgfpathlineto{\pgfqpoint{14.495970in}{4.196805in}}%
\pgfpathlineto{\pgfqpoint{14.372747in}{4.196805in}}%
\pgfpathlineto{\pgfqpoint{14.249524in}{4.196805in}}%
\pgfpathlineto{\pgfqpoint{14.126301in}{4.196805in}}%
\pgfpathlineto{\pgfqpoint{14.003078in}{4.196805in}}%
\pgfpathlineto{\pgfqpoint{13.879855in}{4.196805in}}%
\pgfpathlineto{\pgfqpoint{13.756632in}{4.196805in}}%
\pgfpathlineto{\pgfqpoint{13.633409in}{4.196805in}}%
\pgfpathlineto{\pgfqpoint{13.510186in}{4.196805in}}%
\pgfpathlineto{\pgfqpoint{13.386963in}{4.196805in}}%
\pgfpathlineto{\pgfqpoint{13.263740in}{4.196805in}}%
\pgfpathlineto{\pgfqpoint{13.140517in}{4.196805in}}%
\pgfpathlineto{\pgfqpoint{13.017294in}{4.196805in}}%
\pgfpathlineto{\pgfqpoint{12.894072in}{4.196805in}}%
\pgfpathlineto{\pgfqpoint{12.770849in}{4.196805in}}%
\pgfpathlineto{\pgfqpoint{12.647626in}{4.196805in}}%
\pgfpathlineto{\pgfqpoint{12.524403in}{4.196805in}}%
\pgfpathlineto{\pgfqpoint{12.401180in}{4.196805in}}%
\pgfpathlineto{\pgfqpoint{12.277957in}{4.196805in}}%
\pgfpathlineto{\pgfqpoint{12.154734in}{4.196805in}}%
\pgfpathlineto{\pgfqpoint{12.031511in}{4.196805in}}%
\pgfpathlineto{\pgfqpoint{11.908288in}{4.196805in}}%
\pgfpathlineto{\pgfqpoint{11.785065in}{4.196805in}}%
\pgfpathlineto{\pgfqpoint{11.661842in}{4.196805in}}%
\pgfpathlineto{\pgfqpoint{11.538619in}{4.196805in}}%
\pgfpathlineto{\pgfqpoint{11.415396in}{4.196805in}}%
\pgfpathlineto{\pgfqpoint{11.292173in}{4.196805in}}%
\pgfpathlineto{\pgfqpoint{11.168950in}{4.196805in}}%
\pgfpathlineto{\pgfqpoint{11.045727in}{4.196805in}}%
\pgfpathlineto{\pgfqpoint{10.922504in}{4.196805in}}%
\pgfpathlineto{\pgfqpoint{10.799281in}{4.196805in}}%
\pgfpathlineto{\pgfqpoint{10.676058in}{4.196805in}}%
\pgfpathlineto{\pgfqpoint{10.552835in}{4.196805in}}%
\pgfpathlineto{\pgfqpoint{10.429612in}{4.196805in}}%
\pgfpathlineto{\pgfqpoint{10.306389in}{4.196805in}}%
\pgfpathlineto{\pgfqpoint{10.121555in}{4.196805in}}%
\pgfpathclose%
\pgfusepath{stroke,fill}%
\end{pgfscope}%
\begin{pgfscope}%
\pgfpathrectangle{\pgfqpoint{9.810417in}{4.080233in}}{\pgfqpoint{5.489583in}{0.877907in}}%
\pgfusepath{clip}%
\pgfsetbuttcap%
\pgfsetroundjoin%
\pgfsetlinewidth{1.505625pt}%
\definecolor{currentstroke}{rgb}{0.000000,0.000000,0.000000}%
\pgfsetstrokecolor{currentstroke}%
\pgfsetdash{}{0pt}%
\pgfpathmoveto{\pgfqpoint{10.059943in}{4.158471in}}%
\pgfpathlineto{\pgfqpoint{10.059943in}{4.918235in}}%
\pgfusepath{stroke}%
\end{pgfscope}%
\begin{pgfscope}%
\pgfpathrectangle{\pgfqpoint{9.810417in}{4.080233in}}{\pgfqpoint{5.489583in}{0.877907in}}%
\pgfusepath{clip}%
\pgfsetbuttcap%
\pgfsetroundjoin%
\pgfsetlinewidth{1.505625pt}%
\definecolor{currentstroke}{rgb}{0.000000,0.000000,0.000000}%
\pgfsetstrokecolor{currentstroke}%
\pgfsetdash{}{0pt}%
\pgfpathmoveto{\pgfqpoint{10.183166in}{4.158471in}}%
\pgfpathlineto{\pgfqpoint{10.183166in}{4.916505in}}%
\pgfusepath{stroke}%
\end{pgfscope}%
\begin{pgfscope}%
\pgfpathrectangle{\pgfqpoint{9.810417in}{4.080233in}}{\pgfqpoint{5.489583in}{0.877907in}}%
\pgfusepath{clip}%
\pgfsetbuttcap%
\pgfsetroundjoin%
\pgfsetlinewidth{1.505625pt}%
\definecolor{currentstroke}{rgb}{0.000000,0.000000,0.000000}%
\pgfsetstrokecolor{currentstroke}%
\pgfsetdash{}{0pt}%
\pgfpathmoveto{\pgfqpoint{10.306389in}{4.158471in}}%
\pgfpathlineto{\pgfqpoint{10.306389in}{4.155629in}}%
\pgfusepath{stroke}%
\end{pgfscope}%
\begin{pgfscope}%
\pgfpathrectangle{\pgfqpoint{9.810417in}{4.080233in}}{\pgfqpoint{5.489583in}{0.877907in}}%
\pgfusepath{clip}%
\pgfsetbuttcap%
\pgfsetroundjoin%
\pgfsetlinewidth{1.505625pt}%
\definecolor{currentstroke}{rgb}{0.000000,0.000000,0.000000}%
\pgfsetstrokecolor{currentstroke}%
\pgfsetdash{}{0pt}%
\pgfpathmoveto{\pgfqpoint{10.429612in}{4.158471in}}%
\pgfpathlineto{\pgfqpoint{10.429612in}{4.167532in}}%
\pgfusepath{stroke}%
\end{pgfscope}%
\begin{pgfscope}%
\pgfpathrectangle{\pgfqpoint{9.810417in}{4.080233in}}{\pgfqpoint{5.489583in}{0.877907in}}%
\pgfusepath{clip}%
\pgfsetbuttcap%
\pgfsetroundjoin%
\pgfsetlinewidth{1.505625pt}%
\definecolor{currentstroke}{rgb}{0.000000,0.000000,0.000000}%
\pgfsetstrokecolor{currentstroke}%
\pgfsetdash{}{0pt}%
\pgfpathmoveto{\pgfqpoint{10.552835in}{4.158471in}}%
\pgfpathlineto{\pgfqpoint{10.552835in}{4.157868in}}%
\pgfusepath{stroke}%
\end{pgfscope}%
\begin{pgfscope}%
\pgfpathrectangle{\pgfqpoint{9.810417in}{4.080233in}}{\pgfqpoint{5.489583in}{0.877907in}}%
\pgfusepath{clip}%
\pgfsetbuttcap%
\pgfsetroundjoin%
\pgfsetlinewidth{1.505625pt}%
\definecolor{currentstroke}{rgb}{0.000000,0.000000,0.000000}%
\pgfsetstrokecolor{currentstroke}%
\pgfsetdash{}{0pt}%
\pgfpathmoveto{\pgfqpoint{10.676058in}{4.158471in}}%
\pgfpathlineto{\pgfqpoint{10.676058in}{4.160529in}}%
\pgfusepath{stroke}%
\end{pgfscope}%
\begin{pgfscope}%
\pgfpathrectangle{\pgfqpoint{9.810417in}{4.080233in}}{\pgfqpoint{5.489583in}{0.877907in}}%
\pgfusepath{clip}%
\pgfsetbuttcap%
\pgfsetroundjoin%
\pgfsetlinewidth{1.505625pt}%
\definecolor{currentstroke}{rgb}{0.000000,0.000000,0.000000}%
\pgfsetstrokecolor{currentstroke}%
\pgfsetdash{}{0pt}%
\pgfpathmoveto{\pgfqpoint{10.799281in}{4.158471in}}%
\pgfpathlineto{\pgfqpoint{10.799281in}{4.161884in}}%
\pgfusepath{stroke}%
\end{pgfscope}%
\begin{pgfscope}%
\pgfpathrectangle{\pgfqpoint{9.810417in}{4.080233in}}{\pgfqpoint{5.489583in}{0.877907in}}%
\pgfusepath{clip}%
\pgfsetbuttcap%
\pgfsetroundjoin%
\pgfsetlinewidth{1.505625pt}%
\definecolor{currentstroke}{rgb}{0.000000,0.000000,0.000000}%
\pgfsetstrokecolor{currentstroke}%
\pgfsetdash{}{0pt}%
\pgfpathmoveto{\pgfqpoint{10.922504in}{4.158471in}}%
\pgfpathlineto{\pgfqpoint{10.922504in}{4.151472in}}%
\pgfusepath{stroke}%
\end{pgfscope}%
\begin{pgfscope}%
\pgfpathrectangle{\pgfqpoint{9.810417in}{4.080233in}}{\pgfqpoint{5.489583in}{0.877907in}}%
\pgfusepath{clip}%
\pgfsetbuttcap%
\pgfsetroundjoin%
\pgfsetlinewidth{1.505625pt}%
\definecolor{currentstroke}{rgb}{0.000000,0.000000,0.000000}%
\pgfsetstrokecolor{currentstroke}%
\pgfsetdash{}{0pt}%
\pgfpathmoveto{\pgfqpoint{11.045727in}{4.158471in}}%
\pgfpathlineto{\pgfqpoint{11.045727in}{4.169515in}}%
\pgfusepath{stroke}%
\end{pgfscope}%
\begin{pgfscope}%
\pgfpathrectangle{\pgfqpoint{9.810417in}{4.080233in}}{\pgfqpoint{5.489583in}{0.877907in}}%
\pgfusepath{clip}%
\pgfsetbuttcap%
\pgfsetroundjoin%
\pgfsetlinewidth{1.505625pt}%
\definecolor{currentstroke}{rgb}{0.000000,0.000000,0.000000}%
\pgfsetstrokecolor{currentstroke}%
\pgfsetdash{}{0pt}%
\pgfpathmoveto{\pgfqpoint{11.168950in}{4.158471in}}%
\pgfpathlineto{\pgfqpoint{11.168950in}{4.136413in}}%
\pgfusepath{stroke}%
\end{pgfscope}%
\begin{pgfscope}%
\pgfpathrectangle{\pgfqpoint{9.810417in}{4.080233in}}{\pgfqpoint{5.489583in}{0.877907in}}%
\pgfusepath{clip}%
\pgfsetbuttcap%
\pgfsetroundjoin%
\pgfsetlinewidth{1.505625pt}%
\definecolor{currentstroke}{rgb}{0.000000,0.000000,0.000000}%
\pgfsetstrokecolor{currentstroke}%
\pgfsetdash{}{0pt}%
\pgfpathmoveto{\pgfqpoint{11.292173in}{4.158471in}}%
\pgfpathlineto{\pgfqpoint{11.292173in}{4.168312in}}%
\pgfusepath{stroke}%
\end{pgfscope}%
\begin{pgfscope}%
\pgfpathrectangle{\pgfqpoint{9.810417in}{4.080233in}}{\pgfqpoint{5.489583in}{0.877907in}}%
\pgfusepath{clip}%
\pgfsetbuttcap%
\pgfsetroundjoin%
\pgfsetlinewidth{1.505625pt}%
\definecolor{currentstroke}{rgb}{0.000000,0.000000,0.000000}%
\pgfsetstrokecolor{currentstroke}%
\pgfsetdash{}{0pt}%
\pgfpathmoveto{\pgfqpoint{11.415396in}{4.158471in}}%
\pgfpathlineto{\pgfqpoint{11.415396in}{4.161032in}}%
\pgfusepath{stroke}%
\end{pgfscope}%
\begin{pgfscope}%
\pgfpathrectangle{\pgfqpoint{9.810417in}{4.080233in}}{\pgfqpoint{5.489583in}{0.877907in}}%
\pgfusepath{clip}%
\pgfsetbuttcap%
\pgfsetroundjoin%
\pgfsetlinewidth{1.505625pt}%
\definecolor{currentstroke}{rgb}{0.000000,0.000000,0.000000}%
\pgfsetstrokecolor{currentstroke}%
\pgfsetdash{}{0pt}%
\pgfpathmoveto{\pgfqpoint{11.538619in}{4.158471in}}%
\pgfpathlineto{\pgfqpoint{11.538619in}{4.145539in}}%
\pgfusepath{stroke}%
\end{pgfscope}%
\begin{pgfscope}%
\pgfpathrectangle{\pgfqpoint{9.810417in}{4.080233in}}{\pgfqpoint{5.489583in}{0.877907in}}%
\pgfusepath{clip}%
\pgfsetbuttcap%
\pgfsetroundjoin%
\pgfsetlinewidth{1.505625pt}%
\definecolor{currentstroke}{rgb}{0.000000,0.000000,0.000000}%
\pgfsetstrokecolor{currentstroke}%
\pgfsetdash{}{0pt}%
\pgfpathmoveto{\pgfqpoint{11.661842in}{4.158471in}}%
\pgfpathlineto{\pgfqpoint{11.661842in}{4.146648in}}%
\pgfusepath{stroke}%
\end{pgfscope}%
\begin{pgfscope}%
\pgfpathrectangle{\pgfqpoint{9.810417in}{4.080233in}}{\pgfqpoint{5.489583in}{0.877907in}}%
\pgfusepath{clip}%
\pgfsetbuttcap%
\pgfsetroundjoin%
\pgfsetlinewidth{1.505625pt}%
\definecolor{currentstroke}{rgb}{0.000000,0.000000,0.000000}%
\pgfsetstrokecolor{currentstroke}%
\pgfsetdash{}{0pt}%
\pgfpathmoveto{\pgfqpoint{11.785065in}{4.158471in}}%
\pgfpathlineto{\pgfqpoint{11.785065in}{4.155147in}}%
\pgfusepath{stroke}%
\end{pgfscope}%
\begin{pgfscope}%
\pgfpathrectangle{\pgfqpoint{9.810417in}{4.080233in}}{\pgfqpoint{5.489583in}{0.877907in}}%
\pgfusepath{clip}%
\pgfsetbuttcap%
\pgfsetroundjoin%
\pgfsetlinewidth{1.505625pt}%
\definecolor{currentstroke}{rgb}{0.000000,0.000000,0.000000}%
\pgfsetstrokecolor{currentstroke}%
\pgfsetdash{}{0pt}%
\pgfpathmoveto{\pgfqpoint{11.908288in}{4.158471in}}%
\pgfpathlineto{\pgfqpoint{11.908288in}{4.164772in}}%
\pgfusepath{stroke}%
\end{pgfscope}%
\begin{pgfscope}%
\pgfpathrectangle{\pgfqpoint{9.810417in}{4.080233in}}{\pgfqpoint{5.489583in}{0.877907in}}%
\pgfusepath{clip}%
\pgfsetbuttcap%
\pgfsetroundjoin%
\pgfsetlinewidth{1.505625pt}%
\definecolor{currentstroke}{rgb}{0.000000,0.000000,0.000000}%
\pgfsetstrokecolor{currentstroke}%
\pgfsetdash{}{0pt}%
\pgfpathmoveto{\pgfqpoint{12.031511in}{4.158471in}}%
\pgfpathlineto{\pgfqpoint{12.031511in}{4.170073in}}%
\pgfusepath{stroke}%
\end{pgfscope}%
\begin{pgfscope}%
\pgfpathrectangle{\pgfqpoint{9.810417in}{4.080233in}}{\pgfqpoint{5.489583in}{0.877907in}}%
\pgfusepath{clip}%
\pgfsetbuttcap%
\pgfsetroundjoin%
\pgfsetlinewidth{1.505625pt}%
\definecolor{currentstroke}{rgb}{0.000000,0.000000,0.000000}%
\pgfsetstrokecolor{currentstroke}%
\pgfsetdash{}{0pt}%
\pgfpathmoveto{\pgfqpoint{12.154734in}{4.158471in}}%
\pgfpathlineto{\pgfqpoint{12.154734in}{4.165321in}}%
\pgfusepath{stroke}%
\end{pgfscope}%
\begin{pgfscope}%
\pgfpathrectangle{\pgfqpoint{9.810417in}{4.080233in}}{\pgfqpoint{5.489583in}{0.877907in}}%
\pgfusepath{clip}%
\pgfsetbuttcap%
\pgfsetroundjoin%
\pgfsetlinewidth{1.505625pt}%
\definecolor{currentstroke}{rgb}{0.000000,0.000000,0.000000}%
\pgfsetstrokecolor{currentstroke}%
\pgfsetdash{}{0pt}%
\pgfpathmoveto{\pgfqpoint{12.277957in}{4.158471in}}%
\pgfpathlineto{\pgfqpoint{12.277957in}{4.159858in}}%
\pgfusepath{stroke}%
\end{pgfscope}%
\begin{pgfscope}%
\pgfpathrectangle{\pgfqpoint{9.810417in}{4.080233in}}{\pgfqpoint{5.489583in}{0.877907in}}%
\pgfusepath{clip}%
\pgfsetbuttcap%
\pgfsetroundjoin%
\pgfsetlinewidth{1.505625pt}%
\definecolor{currentstroke}{rgb}{0.000000,0.000000,0.000000}%
\pgfsetstrokecolor{currentstroke}%
\pgfsetdash{}{0pt}%
\pgfpathmoveto{\pgfqpoint{12.401180in}{4.158471in}}%
\pgfpathlineto{\pgfqpoint{12.401180in}{4.156250in}}%
\pgfusepath{stroke}%
\end{pgfscope}%
\begin{pgfscope}%
\pgfpathrectangle{\pgfqpoint{9.810417in}{4.080233in}}{\pgfqpoint{5.489583in}{0.877907in}}%
\pgfusepath{clip}%
\pgfsetbuttcap%
\pgfsetroundjoin%
\pgfsetlinewidth{1.505625pt}%
\definecolor{currentstroke}{rgb}{0.000000,0.000000,0.000000}%
\pgfsetstrokecolor{currentstroke}%
\pgfsetdash{}{0pt}%
\pgfpathmoveto{\pgfqpoint{12.524403in}{4.158471in}}%
\pgfpathlineto{\pgfqpoint{12.524403in}{4.145457in}}%
\pgfusepath{stroke}%
\end{pgfscope}%
\begin{pgfscope}%
\pgfpathrectangle{\pgfqpoint{9.810417in}{4.080233in}}{\pgfqpoint{5.489583in}{0.877907in}}%
\pgfusepath{clip}%
\pgfsetbuttcap%
\pgfsetroundjoin%
\pgfsetlinewidth{1.505625pt}%
\definecolor{currentstroke}{rgb}{0.000000,0.000000,0.000000}%
\pgfsetstrokecolor{currentstroke}%
\pgfsetdash{}{0pt}%
\pgfpathmoveto{\pgfqpoint{12.647626in}{4.158471in}}%
\pgfpathlineto{\pgfqpoint{12.647626in}{4.155106in}}%
\pgfusepath{stroke}%
\end{pgfscope}%
\begin{pgfscope}%
\pgfpathrectangle{\pgfqpoint{9.810417in}{4.080233in}}{\pgfqpoint{5.489583in}{0.877907in}}%
\pgfusepath{clip}%
\pgfsetbuttcap%
\pgfsetroundjoin%
\pgfsetlinewidth{1.505625pt}%
\definecolor{currentstroke}{rgb}{0.000000,0.000000,0.000000}%
\pgfsetstrokecolor{currentstroke}%
\pgfsetdash{}{0pt}%
\pgfpathmoveto{\pgfqpoint{12.770849in}{4.158471in}}%
\pgfpathlineto{\pgfqpoint{12.770849in}{4.187559in}}%
\pgfusepath{stroke}%
\end{pgfscope}%
\begin{pgfscope}%
\pgfpathrectangle{\pgfqpoint{9.810417in}{4.080233in}}{\pgfqpoint{5.489583in}{0.877907in}}%
\pgfusepath{clip}%
\pgfsetbuttcap%
\pgfsetroundjoin%
\pgfsetlinewidth{1.505625pt}%
\definecolor{currentstroke}{rgb}{0.000000,0.000000,0.000000}%
\pgfsetstrokecolor{currentstroke}%
\pgfsetdash{}{0pt}%
\pgfpathmoveto{\pgfqpoint{12.894072in}{4.158471in}}%
\pgfpathlineto{\pgfqpoint{12.894072in}{4.151936in}}%
\pgfusepath{stroke}%
\end{pgfscope}%
\begin{pgfscope}%
\pgfpathrectangle{\pgfqpoint{9.810417in}{4.080233in}}{\pgfqpoint{5.489583in}{0.877907in}}%
\pgfusepath{clip}%
\pgfsetbuttcap%
\pgfsetroundjoin%
\pgfsetlinewidth{1.505625pt}%
\definecolor{currentstroke}{rgb}{0.000000,0.000000,0.000000}%
\pgfsetstrokecolor{currentstroke}%
\pgfsetdash{}{0pt}%
\pgfpathmoveto{\pgfqpoint{13.017294in}{4.158471in}}%
\pgfpathlineto{\pgfqpoint{13.017294in}{4.162514in}}%
\pgfusepath{stroke}%
\end{pgfscope}%
\begin{pgfscope}%
\pgfpathrectangle{\pgfqpoint{9.810417in}{4.080233in}}{\pgfqpoint{5.489583in}{0.877907in}}%
\pgfusepath{clip}%
\pgfsetbuttcap%
\pgfsetroundjoin%
\pgfsetlinewidth{1.505625pt}%
\definecolor{currentstroke}{rgb}{0.000000,0.000000,0.000000}%
\pgfsetstrokecolor{currentstroke}%
\pgfsetdash{}{0pt}%
\pgfpathmoveto{\pgfqpoint{13.140517in}{4.158471in}}%
\pgfpathlineto{\pgfqpoint{13.140517in}{4.155848in}}%
\pgfusepath{stroke}%
\end{pgfscope}%
\begin{pgfscope}%
\pgfpathrectangle{\pgfqpoint{9.810417in}{4.080233in}}{\pgfqpoint{5.489583in}{0.877907in}}%
\pgfusepath{clip}%
\pgfsetbuttcap%
\pgfsetroundjoin%
\pgfsetlinewidth{1.505625pt}%
\definecolor{currentstroke}{rgb}{0.000000,0.000000,0.000000}%
\pgfsetstrokecolor{currentstroke}%
\pgfsetdash{}{0pt}%
\pgfpathmoveto{\pgfqpoint{13.263740in}{4.158471in}}%
\pgfpathlineto{\pgfqpoint{13.263740in}{4.168743in}}%
\pgfusepath{stroke}%
\end{pgfscope}%
\begin{pgfscope}%
\pgfpathrectangle{\pgfqpoint{9.810417in}{4.080233in}}{\pgfqpoint{5.489583in}{0.877907in}}%
\pgfusepath{clip}%
\pgfsetbuttcap%
\pgfsetroundjoin%
\pgfsetlinewidth{1.505625pt}%
\definecolor{currentstroke}{rgb}{0.000000,0.000000,0.000000}%
\pgfsetstrokecolor{currentstroke}%
\pgfsetdash{}{0pt}%
\pgfpathmoveto{\pgfqpoint{13.386963in}{4.158471in}}%
\pgfpathlineto{\pgfqpoint{13.386963in}{4.175848in}}%
\pgfusepath{stroke}%
\end{pgfscope}%
\begin{pgfscope}%
\pgfpathrectangle{\pgfqpoint{9.810417in}{4.080233in}}{\pgfqpoint{5.489583in}{0.877907in}}%
\pgfusepath{clip}%
\pgfsetbuttcap%
\pgfsetroundjoin%
\pgfsetlinewidth{1.505625pt}%
\definecolor{currentstroke}{rgb}{0.000000,0.000000,0.000000}%
\pgfsetstrokecolor{currentstroke}%
\pgfsetdash{}{0pt}%
\pgfpathmoveto{\pgfqpoint{13.510186in}{4.158471in}}%
\pgfpathlineto{\pgfqpoint{13.510186in}{4.168638in}}%
\pgfusepath{stroke}%
\end{pgfscope}%
\begin{pgfscope}%
\pgfpathrectangle{\pgfqpoint{9.810417in}{4.080233in}}{\pgfqpoint{5.489583in}{0.877907in}}%
\pgfusepath{clip}%
\pgfsetbuttcap%
\pgfsetroundjoin%
\pgfsetlinewidth{1.505625pt}%
\definecolor{currentstroke}{rgb}{0.000000,0.000000,0.000000}%
\pgfsetstrokecolor{currentstroke}%
\pgfsetdash{}{0pt}%
\pgfpathmoveto{\pgfqpoint{13.633409in}{4.158471in}}%
\pgfpathlineto{\pgfqpoint{13.633409in}{4.155242in}}%
\pgfusepath{stroke}%
\end{pgfscope}%
\begin{pgfscope}%
\pgfpathrectangle{\pgfqpoint{9.810417in}{4.080233in}}{\pgfqpoint{5.489583in}{0.877907in}}%
\pgfusepath{clip}%
\pgfsetbuttcap%
\pgfsetroundjoin%
\pgfsetlinewidth{1.505625pt}%
\definecolor{currentstroke}{rgb}{0.000000,0.000000,0.000000}%
\pgfsetstrokecolor{currentstroke}%
\pgfsetdash{}{0pt}%
\pgfpathmoveto{\pgfqpoint{13.756632in}{4.158471in}}%
\pgfpathlineto{\pgfqpoint{13.756632in}{4.161979in}}%
\pgfusepath{stroke}%
\end{pgfscope}%
\begin{pgfscope}%
\pgfpathrectangle{\pgfqpoint{9.810417in}{4.080233in}}{\pgfqpoint{5.489583in}{0.877907in}}%
\pgfusepath{clip}%
\pgfsetbuttcap%
\pgfsetroundjoin%
\pgfsetlinewidth{1.505625pt}%
\definecolor{currentstroke}{rgb}{0.000000,0.000000,0.000000}%
\pgfsetstrokecolor{currentstroke}%
\pgfsetdash{}{0pt}%
\pgfpathmoveto{\pgfqpoint{13.879855in}{4.158471in}}%
\pgfpathlineto{\pgfqpoint{13.879855in}{4.173305in}}%
\pgfusepath{stroke}%
\end{pgfscope}%
\begin{pgfscope}%
\pgfpathrectangle{\pgfqpoint{9.810417in}{4.080233in}}{\pgfqpoint{5.489583in}{0.877907in}}%
\pgfusepath{clip}%
\pgfsetbuttcap%
\pgfsetroundjoin%
\pgfsetlinewidth{1.505625pt}%
\definecolor{currentstroke}{rgb}{0.000000,0.000000,0.000000}%
\pgfsetstrokecolor{currentstroke}%
\pgfsetdash{}{0pt}%
\pgfpathmoveto{\pgfqpoint{14.003078in}{4.158471in}}%
\pgfpathlineto{\pgfqpoint{14.003078in}{4.146073in}}%
\pgfusepath{stroke}%
\end{pgfscope}%
\begin{pgfscope}%
\pgfpathrectangle{\pgfqpoint{9.810417in}{4.080233in}}{\pgfqpoint{5.489583in}{0.877907in}}%
\pgfusepath{clip}%
\pgfsetbuttcap%
\pgfsetroundjoin%
\pgfsetlinewidth{1.505625pt}%
\definecolor{currentstroke}{rgb}{0.000000,0.000000,0.000000}%
\pgfsetstrokecolor{currentstroke}%
\pgfsetdash{}{0pt}%
\pgfpathmoveto{\pgfqpoint{14.126301in}{4.158471in}}%
\pgfpathlineto{\pgfqpoint{14.126301in}{4.171898in}}%
\pgfusepath{stroke}%
\end{pgfscope}%
\begin{pgfscope}%
\pgfpathrectangle{\pgfqpoint{9.810417in}{4.080233in}}{\pgfqpoint{5.489583in}{0.877907in}}%
\pgfusepath{clip}%
\pgfsetbuttcap%
\pgfsetroundjoin%
\pgfsetlinewidth{1.505625pt}%
\definecolor{currentstroke}{rgb}{0.000000,0.000000,0.000000}%
\pgfsetstrokecolor{currentstroke}%
\pgfsetdash{}{0pt}%
\pgfpathmoveto{\pgfqpoint{14.249524in}{4.158471in}}%
\pgfpathlineto{\pgfqpoint{14.249524in}{4.139357in}}%
\pgfusepath{stroke}%
\end{pgfscope}%
\begin{pgfscope}%
\pgfpathrectangle{\pgfqpoint{9.810417in}{4.080233in}}{\pgfqpoint{5.489583in}{0.877907in}}%
\pgfusepath{clip}%
\pgfsetbuttcap%
\pgfsetroundjoin%
\pgfsetlinewidth{1.505625pt}%
\definecolor{currentstroke}{rgb}{0.000000,0.000000,0.000000}%
\pgfsetstrokecolor{currentstroke}%
\pgfsetdash{}{0pt}%
\pgfpathmoveto{\pgfqpoint{14.372747in}{4.158471in}}%
\pgfpathlineto{\pgfqpoint{14.372747in}{4.164160in}}%
\pgfusepath{stroke}%
\end{pgfscope}%
\begin{pgfscope}%
\pgfpathrectangle{\pgfqpoint{9.810417in}{4.080233in}}{\pgfqpoint{5.489583in}{0.877907in}}%
\pgfusepath{clip}%
\pgfsetbuttcap%
\pgfsetroundjoin%
\pgfsetlinewidth{1.505625pt}%
\definecolor{currentstroke}{rgb}{0.000000,0.000000,0.000000}%
\pgfsetstrokecolor{currentstroke}%
\pgfsetdash{}{0pt}%
\pgfpathmoveto{\pgfqpoint{14.495970in}{4.158471in}}%
\pgfpathlineto{\pgfqpoint{14.495970in}{4.158660in}}%
\pgfusepath{stroke}%
\end{pgfscope}%
\begin{pgfscope}%
\pgfpathrectangle{\pgfqpoint{9.810417in}{4.080233in}}{\pgfqpoint{5.489583in}{0.877907in}}%
\pgfusepath{clip}%
\pgfsetbuttcap%
\pgfsetroundjoin%
\pgfsetlinewidth{1.505625pt}%
\definecolor{currentstroke}{rgb}{0.000000,0.000000,0.000000}%
\pgfsetstrokecolor{currentstroke}%
\pgfsetdash{}{0pt}%
\pgfpathmoveto{\pgfqpoint{14.619193in}{4.158471in}}%
\pgfpathlineto{\pgfqpoint{14.619193in}{4.165907in}}%
\pgfusepath{stroke}%
\end{pgfscope}%
\begin{pgfscope}%
\pgfpathrectangle{\pgfqpoint{9.810417in}{4.080233in}}{\pgfqpoint{5.489583in}{0.877907in}}%
\pgfusepath{clip}%
\pgfsetbuttcap%
\pgfsetroundjoin%
\pgfsetlinewidth{1.505625pt}%
\definecolor{currentstroke}{rgb}{0.000000,0.000000,0.000000}%
\pgfsetstrokecolor{currentstroke}%
\pgfsetdash{}{0pt}%
\pgfpathmoveto{\pgfqpoint{14.742416in}{4.158471in}}%
\pgfpathlineto{\pgfqpoint{14.742416in}{4.151998in}}%
\pgfusepath{stroke}%
\end{pgfscope}%
\begin{pgfscope}%
\pgfpathrectangle{\pgfqpoint{9.810417in}{4.080233in}}{\pgfqpoint{5.489583in}{0.877907in}}%
\pgfusepath{clip}%
\pgfsetbuttcap%
\pgfsetroundjoin%
\pgfsetlinewidth{1.505625pt}%
\definecolor{currentstroke}{rgb}{0.000000,0.000000,0.000000}%
\pgfsetstrokecolor{currentstroke}%
\pgfsetdash{}{0pt}%
\pgfpathmoveto{\pgfqpoint{14.865639in}{4.158471in}}%
\pgfpathlineto{\pgfqpoint{14.865639in}{4.166518in}}%
\pgfusepath{stroke}%
\end{pgfscope}%
\begin{pgfscope}%
\pgfpathrectangle{\pgfqpoint{9.810417in}{4.080233in}}{\pgfqpoint{5.489583in}{0.877907in}}%
\pgfusepath{clip}%
\pgfsetbuttcap%
\pgfsetroundjoin%
\pgfsetlinewidth{1.505625pt}%
\definecolor{currentstroke}{rgb}{0.000000,0.000000,0.000000}%
\pgfsetstrokecolor{currentstroke}%
\pgfsetdash{}{0pt}%
\pgfpathmoveto{\pgfqpoint{14.988862in}{4.158471in}}%
\pgfpathlineto{\pgfqpoint{14.988862in}{4.152790in}}%
\pgfusepath{stroke}%
\end{pgfscope}%
\begin{pgfscope}%
\pgfpathrectangle{\pgfqpoint{9.810417in}{4.080233in}}{\pgfqpoint{5.489583in}{0.877907in}}%
\pgfusepath{clip}%
\pgfsetroundcap%
\pgfsetroundjoin%
\pgfsetlinewidth{1.505625pt}%
\definecolor{currentstroke}{rgb}{0.121569,0.466667,0.705882}%
\pgfsetstrokecolor{currentstroke}%
\pgfsetdash{}{0pt}%
\pgfpathmoveto{\pgfqpoint{9.810417in}{4.158471in}}%
\pgfpathlineto{\pgfqpoint{15.300000in}{4.158471in}}%
\pgfusepath{stroke}%
\end{pgfscope}%
\begin{pgfscope}%
\pgfpathrectangle{\pgfqpoint{9.810417in}{4.080233in}}{\pgfqpoint{5.489583in}{0.877907in}}%
\pgfusepath{clip}%
\pgfsetbuttcap%
\pgfsetroundjoin%
\definecolor{currentfill}{rgb}{0.121569,0.466667,0.705882}%
\pgfsetfillcolor{currentfill}%
\pgfsetlinewidth{1.003750pt}%
\definecolor{currentstroke}{rgb}{0.121569,0.466667,0.705882}%
\pgfsetstrokecolor{currentstroke}%
\pgfsetdash{}{0pt}%
\pgfsys@defobject{currentmarker}{\pgfqpoint{-0.034722in}{-0.034722in}}{\pgfqpoint{0.034722in}{0.034722in}}{%
\pgfpathmoveto{\pgfqpoint{0.000000in}{-0.034722in}}%
\pgfpathcurveto{\pgfqpoint{0.009208in}{-0.034722in}}{\pgfqpoint{0.018041in}{-0.031064in}}{\pgfqpoint{0.024552in}{-0.024552in}}%
\pgfpathcurveto{\pgfqpoint{0.031064in}{-0.018041in}}{\pgfqpoint{0.034722in}{-0.009208in}}{\pgfqpoint{0.034722in}{0.000000in}}%
\pgfpathcurveto{\pgfqpoint{0.034722in}{0.009208in}}{\pgfqpoint{0.031064in}{0.018041in}}{\pgfqpoint{0.024552in}{0.024552in}}%
\pgfpathcurveto{\pgfqpoint{0.018041in}{0.031064in}}{\pgfqpoint{0.009208in}{0.034722in}}{\pgfqpoint{0.000000in}{0.034722in}}%
\pgfpathcurveto{\pgfqpoint{-0.009208in}{0.034722in}}{\pgfqpoint{-0.018041in}{0.031064in}}{\pgfqpoint{-0.024552in}{0.024552in}}%
\pgfpathcurveto{\pgfqpoint{-0.031064in}{0.018041in}}{\pgfqpoint{-0.034722in}{0.009208in}}{\pgfqpoint{-0.034722in}{0.000000in}}%
\pgfpathcurveto{\pgfqpoint{-0.034722in}{-0.009208in}}{\pgfqpoint{-0.031064in}{-0.018041in}}{\pgfqpoint{-0.024552in}{-0.024552in}}%
\pgfpathcurveto{\pgfqpoint{-0.018041in}{-0.031064in}}{\pgfqpoint{-0.009208in}{-0.034722in}}{\pgfqpoint{0.000000in}{-0.034722in}}%
\pgfpathclose%
\pgfusepath{stroke,fill}%
}%
\begin{pgfscope}%
\pgfsys@transformshift{10.059943in}{4.918235in}%
\pgfsys@useobject{currentmarker}{}%
\end{pgfscope}%
\begin{pgfscope}%
\pgfsys@transformshift{10.183166in}{4.916505in}%
\pgfsys@useobject{currentmarker}{}%
\end{pgfscope}%
\begin{pgfscope}%
\pgfsys@transformshift{10.306389in}{4.155629in}%
\pgfsys@useobject{currentmarker}{}%
\end{pgfscope}%
\begin{pgfscope}%
\pgfsys@transformshift{10.429612in}{4.167532in}%
\pgfsys@useobject{currentmarker}{}%
\end{pgfscope}%
\begin{pgfscope}%
\pgfsys@transformshift{10.552835in}{4.157868in}%
\pgfsys@useobject{currentmarker}{}%
\end{pgfscope}%
\begin{pgfscope}%
\pgfsys@transformshift{10.676058in}{4.160529in}%
\pgfsys@useobject{currentmarker}{}%
\end{pgfscope}%
\begin{pgfscope}%
\pgfsys@transformshift{10.799281in}{4.161884in}%
\pgfsys@useobject{currentmarker}{}%
\end{pgfscope}%
\begin{pgfscope}%
\pgfsys@transformshift{10.922504in}{4.151472in}%
\pgfsys@useobject{currentmarker}{}%
\end{pgfscope}%
\begin{pgfscope}%
\pgfsys@transformshift{11.045727in}{4.169515in}%
\pgfsys@useobject{currentmarker}{}%
\end{pgfscope}%
\begin{pgfscope}%
\pgfsys@transformshift{11.168950in}{4.136413in}%
\pgfsys@useobject{currentmarker}{}%
\end{pgfscope}%
\begin{pgfscope}%
\pgfsys@transformshift{11.292173in}{4.168312in}%
\pgfsys@useobject{currentmarker}{}%
\end{pgfscope}%
\begin{pgfscope}%
\pgfsys@transformshift{11.415396in}{4.161032in}%
\pgfsys@useobject{currentmarker}{}%
\end{pgfscope}%
\begin{pgfscope}%
\pgfsys@transformshift{11.538619in}{4.145539in}%
\pgfsys@useobject{currentmarker}{}%
\end{pgfscope}%
\begin{pgfscope}%
\pgfsys@transformshift{11.661842in}{4.146648in}%
\pgfsys@useobject{currentmarker}{}%
\end{pgfscope}%
\begin{pgfscope}%
\pgfsys@transformshift{11.785065in}{4.155147in}%
\pgfsys@useobject{currentmarker}{}%
\end{pgfscope}%
\begin{pgfscope}%
\pgfsys@transformshift{11.908288in}{4.164772in}%
\pgfsys@useobject{currentmarker}{}%
\end{pgfscope}%
\begin{pgfscope}%
\pgfsys@transformshift{12.031511in}{4.170073in}%
\pgfsys@useobject{currentmarker}{}%
\end{pgfscope}%
\begin{pgfscope}%
\pgfsys@transformshift{12.154734in}{4.165321in}%
\pgfsys@useobject{currentmarker}{}%
\end{pgfscope}%
\begin{pgfscope}%
\pgfsys@transformshift{12.277957in}{4.159858in}%
\pgfsys@useobject{currentmarker}{}%
\end{pgfscope}%
\begin{pgfscope}%
\pgfsys@transformshift{12.401180in}{4.156250in}%
\pgfsys@useobject{currentmarker}{}%
\end{pgfscope}%
\begin{pgfscope}%
\pgfsys@transformshift{12.524403in}{4.145457in}%
\pgfsys@useobject{currentmarker}{}%
\end{pgfscope}%
\begin{pgfscope}%
\pgfsys@transformshift{12.647626in}{4.155106in}%
\pgfsys@useobject{currentmarker}{}%
\end{pgfscope}%
\begin{pgfscope}%
\pgfsys@transformshift{12.770849in}{4.187559in}%
\pgfsys@useobject{currentmarker}{}%
\end{pgfscope}%
\begin{pgfscope}%
\pgfsys@transformshift{12.894072in}{4.151936in}%
\pgfsys@useobject{currentmarker}{}%
\end{pgfscope}%
\begin{pgfscope}%
\pgfsys@transformshift{13.017294in}{4.162514in}%
\pgfsys@useobject{currentmarker}{}%
\end{pgfscope}%
\begin{pgfscope}%
\pgfsys@transformshift{13.140517in}{4.155848in}%
\pgfsys@useobject{currentmarker}{}%
\end{pgfscope}%
\begin{pgfscope}%
\pgfsys@transformshift{13.263740in}{4.168743in}%
\pgfsys@useobject{currentmarker}{}%
\end{pgfscope}%
\begin{pgfscope}%
\pgfsys@transformshift{13.386963in}{4.175848in}%
\pgfsys@useobject{currentmarker}{}%
\end{pgfscope}%
\begin{pgfscope}%
\pgfsys@transformshift{13.510186in}{4.168638in}%
\pgfsys@useobject{currentmarker}{}%
\end{pgfscope}%
\begin{pgfscope}%
\pgfsys@transformshift{13.633409in}{4.155242in}%
\pgfsys@useobject{currentmarker}{}%
\end{pgfscope}%
\begin{pgfscope}%
\pgfsys@transformshift{13.756632in}{4.161979in}%
\pgfsys@useobject{currentmarker}{}%
\end{pgfscope}%
\begin{pgfscope}%
\pgfsys@transformshift{13.879855in}{4.173305in}%
\pgfsys@useobject{currentmarker}{}%
\end{pgfscope}%
\begin{pgfscope}%
\pgfsys@transformshift{14.003078in}{4.146073in}%
\pgfsys@useobject{currentmarker}{}%
\end{pgfscope}%
\begin{pgfscope}%
\pgfsys@transformshift{14.126301in}{4.171898in}%
\pgfsys@useobject{currentmarker}{}%
\end{pgfscope}%
\begin{pgfscope}%
\pgfsys@transformshift{14.249524in}{4.139357in}%
\pgfsys@useobject{currentmarker}{}%
\end{pgfscope}%
\begin{pgfscope}%
\pgfsys@transformshift{14.372747in}{4.164160in}%
\pgfsys@useobject{currentmarker}{}%
\end{pgfscope}%
\begin{pgfscope}%
\pgfsys@transformshift{14.495970in}{4.158660in}%
\pgfsys@useobject{currentmarker}{}%
\end{pgfscope}%
\begin{pgfscope}%
\pgfsys@transformshift{14.619193in}{4.165907in}%
\pgfsys@useobject{currentmarker}{}%
\end{pgfscope}%
\begin{pgfscope}%
\pgfsys@transformshift{14.742416in}{4.151998in}%
\pgfsys@useobject{currentmarker}{}%
\end{pgfscope}%
\begin{pgfscope}%
\pgfsys@transformshift{14.865639in}{4.166518in}%
\pgfsys@useobject{currentmarker}{}%
\end{pgfscope}%
\begin{pgfscope}%
\pgfsys@transformshift{14.988862in}{4.152790in}%
\pgfsys@useobject{currentmarker}{}%
\end{pgfscope}%
\end{pgfscope}%
\begin{pgfscope}%
\pgfsetrectcap%
\pgfsetmiterjoin%
\pgfsetlinewidth{0.803000pt}%
\definecolor{currentstroke}{rgb}{1.000000,1.000000,1.000000}%
\pgfsetstrokecolor{currentstroke}%
\pgfsetdash{}{0pt}%
\pgfpathmoveto{\pgfqpoint{9.810417in}{4.080233in}}%
\pgfpathlineto{\pgfqpoint{9.810417in}{4.958140in}}%
\pgfusepath{stroke}%
\end{pgfscope}%
\begin{pgfscope}%
\pgfsetrectcap%
\pgfsetmiterjoin%
\pgfsetlinewidth{0.803000pt}%
\definecolor{currentstroke}{rgb}{1.000000,1.000000,1.000000}%
\pgfsetstrokecolor{currentstroke}%
\pgfsetdash{}{0pt}%
\pgfpathmoveto{\pgfqpoint{15.300000in}{4.080233in}}%
\pgfpathlineto{\pgfqpoint{15.300000in}{4.958140in}}%
\pgfusepath{stroke}%
\end{pgfscope}%
\begin{pgfscope}%
\pgfsetrectcap%
\pgfsetmiterjoin%
\pgfsetlinewidth{0.803000pt}%
\definecolor{currentstroke}{rgb}{1.000000,1.000000,1.000000}%
\pgfsetstrokecolor{currentstroke}%
\pgfsetdash{}{0pt}%
\pgfpathmoveto{\pgfqpoint{9.810417in}{4.080233in}}%
\pgfpathlineto{\pgfqpoint{15.300000in}{4.080233in}}%
\pgfusepath{stroke}%
\end{pgfscope}%
\begin{pgfscope}%
\pgfsetrectcap%
\pgfsetmiterjoin%
\pgfsetlinewidth{0.803000pt}%
\definecolor{currentstroke}{rgb}{1.000000,1.000000,1.000000}%
\pgfsetstrokecolor{currentstroke}%
\pgfsetdash{}{0pt}%
\pgfpathmoveto{\pgfqpoint{9.810417in}{4.958140in}}%
\pgfpathlineto{\pgfqpoint{15.300000in}{4.958140in}}%
\pgfusepath{stroke}%
\end{pgfscope}%
\begin{pgfscope}%
\definecolor{textcolor}{rgb}{0.150000,0.150000,0.150000}%
\pgfsetstrokecolor{textcolor}%
\pgfsetfillcolor{textcolor}%
\pgftext[x=12.555208in,y=5.041473in,,base]{\color{textcolor}\rmfamily\fontsize{16.800000}{20.160000}\selectfont Partial Autocorrelation}%
\end{pgfscope}%
\begin{pgfscope}%
\pgfsetbuttcap%
\pgfsetmiterjoin%
\definecolor{currentfill}{rgb}{0.917647,0.917647,0.949020}%
\pgfsetfillcolor{currentfill}%
\pgfsetlinewidth{0.000000pt}%
\definecolor{currentstroke}{rgb}{0.000000,0.000000,0.000000}%
\pgfsetstrokecolor{currentstroke}%
\pgfsetstrokeopacity{0.000000}%
\pgfsetdash{}{0pt}%
\pgfpathmoveto{\pgfqpoint{2.125000in}{2.500000in}}%
\pgfpathlineto{\pgfqpoint{7.614583in}{2.500000in}}%
\pgfpathlineto{\pgfqpoint{7.614583in}{3.377907in}}%
\pgfpathlineto{\pgfqpoint{2.125000in}{3.377907in}}%
\pgfpathclose%
\pgfusepath{fill}%
\end{pgfscope}%
\begin{pgfscope}%
\pgfpathrectangle{\pgfqpoint{2.125000in}{2.500000in}}{\pgfqpoint{5.489583in}{0.877907in}}%
\pgfusepath{clip}%
\pgfsetroundcap%
\pgfsetroundjoin%
\pgfsetlinewidth{0.803000pt}%
\definecolor{currentstroke}{rgb}{1.000000,1.000000,1.000000}%
\pgfsetstrokecolor{currentstroke}%
\pgfsetdash{}{0pt}%
\pgfpathmoveto{\pgfqpoint{2.374527in}{2.500000in}}%
\pgfpathlineto{\pgfqpoint{2.374527in}{3.377907in}}%
\pgfusepath{stroke}%
\end{pgfscope}%
\begin{pgfscope}%
\definecolor{textcolor}{rgb}{0.150000,0.150000,0.150000}%
\pgfsetstrokecolor{textcolor}%
\pgfsetfillcolor{textcolor}%
\pgftext[x=2.374527in,y=2.402778in,,top]{\color{textcolor}\rmfamily\fontsize{14.000000}{16.800000}\selectfont 0}%
\end{pgfscope}%
\begin{pgfscope}%
\pgfpathrectangle{\pgfqpoint{2.125000in}{2.500000in}}{\pgfqpoint{5.489583in}{0.877907in}}%
\pgfusepath{clip}%
\pgfsetroundcap%
\pgfsetroundjoin%
\pgfsetlinewidth{0.803000pt}%
\definecolor{currentstroke}{rgb}{1.000000,1.000000,1.000000}%
\pgfsetstrokecolor{currentstroke}%
\pgfsetdash{}{0pt}%
\pgfpathmoveto{\pgfqpoint{2.990641in}{2.500000in}}%
\pgfpathlineto{\pgfqpoint{2.990641in}{3.377907in}}%
\pgfusepath{stroke}%
\end{pgfscope}%
\begin{pgfscope}%
\definecolor{textcolor}{rgb}{0.150000,0.150000,0.150000}%
\pgfsetstrokecolor{textcolor}%
\pgfsetfillcolor{textcolor}%
\pgftext[x=2.990641in,y=2.402778in,,top]{\color{textcolor}\rmfamily\fontsize{14.000000}{16.800000}\selectfont 5}%
\end{pgfscope}%
\begin{pgfscope}%
\pgfpathrectangle{\pgfqpoint{2.125000in}{2.500000in}}{\pgfqpoint{5.489583in}{0.877907in}}%
\pgfusepath{clip}%
\pgfsetroundcap%
\pgfsetroundjoin%
\pgfsetlinewidth{0.803000pt}%
\definecolor{currentstroke}{rgb}{1.000000,1.000000,1.000000}%
\pgfsetstrokecolor{currentstroke}%
\pgfsetdash{}{0pt}%
\pgfpathmoveto{\pgfqpoint{3.606756in}{2.500000in}}%
\pgfpathlineto{\pgfqpoint{3.606756in}{3.377907in}}%
\pgfusepath{stroke}%
\end{pgfscope}%
\begin{pgfscope}%
\definecolor{textcolor}{rgb}{0.150000,0.150000,0.150000}%
\pgfsetstrokecolor{textcolor}%
\pgfsetfillcolor{textcolor}%
\pgftext[x=3.606756in,y=2.402778in,,top]{\color{textcolor}\rmfamily\fontsize{14.000000}{16.800000}\selectfont 10}%
\end{pgfscope}%
\begin{pgfscope}%
\pgfpathrectangle{\pgfqpoint{2.125000in}{2.500000in}}{\pgfqpoint{5.489583in}{0.877907in}}%
\pgfusepath{clip}%
\pgfsetroundcap%
\pgfsetroundjoin%
\pgfsetlinewidth{0.803000pt}%
\definecolor{currentstroke}{rgb}{1.000000,1.000000,1.000000}%
\pgfsetstrokecolor{currentstroke}%
\pgfsetdash{}{0pt}%
\pgfpathmoveto{\pgfqpoint{4.222871in}{2.500000in}}%
\pgfpathlineto{\pgfqpoint{4.222871in}{3.377907in}}%
\pgfusepath{stroke}%
\end{pgfscope}%
\begin{pgfscope}%
\definecolor{textcolor}{rgb}{0.150000,0.150000,0.150000}%
\pgfsetstrokecolor{textcolor}%
\pgfsetfillcolor{textcolor}%
\pgftext[x=4.222871in,y=2.402778in,,top]{\color{textcolor}\rmfamily\fontsize{14.000000}{16.800000}\selectfont 15}%
\end{pgfscope}%
\begin{pgfscope}%
\pgfpathrectangle{\pgfqpoint{2.125000in}{2.500000in}}{\pgfqpoint{5.489583in}{0.877907in}}%
\pgfusepath{clip}%
\pgfsetroundcap%
\pgfsetroundjoin%
\pgfsetlinewidth{0.803000pt}%
\definecolor{currentstroke}{rgb}{1.000000,1.000000,1.000000}%
\pgfsetstrokecolor{currentstroke}%
\pgfsetdash{}{0pt}%
\pgfpathmoveto{\pgfqpoint{4.838986in}{2.500000in}}%
\pgfpathlineto{\pgfqpoint{4.838986in}{3.377907in}}%
\pgfusepath{stroke}%
\end{pgfscope}%
\begin{pgfscope}%
\definecolor{textcolor}{rgb}{0.150000,0.150000,0.150000}%
\pgfsetstrokecolor{textcolor}%
\pgfsetfillcolor{textcolor}%
\pgftext[x=4.838986in,y=2.402778in,,top]{\color{textcolor}\rmfamily\fontsize{14.000000}{16.800000}\selectfont 20}%
\end{pgfscope}%
\begin{pgfscope}%
\pgfpathrectangle{\pgfqpoint{2.125000in}{2.500000in}}{\pgfqpoint{5.489583in}{0.877907in}}%
\pgfusepath{clip}%
\pgfsetroundcap%
\pgfsetroundjoin%
\pgfsetlinewidth{0.803000pt}%
\definecolor{currentstroke}{rgb}{1.000000,1.000000,1.000000}%
\pgfsetstrokecolor{currentstroke}%
\pgfsetdash{}{0pt}%
\pgfpathmoveto{\pgfqpoint{5.455101in}{2.500000in}}%
\pgfpathlineto{\pgfqpoint{5.455101in}{3.377907in}}%
\pgfusepath{stroke}%
\end{pgfscope}%
\begin{pgfscope}%
\definecolor{textcolor}{rgb}{0.150000,0.150000,0.150000}%
\pgfsetstrokecolor{textcolor}%
\pgfsetfillcolor{textcolor}%
\pgftext[x=5.455101in,y=2.402778in,,top]{\color{textcolor}\rmfamily\fontsize{14.000000}{16.800000}\selectfont 25}%
\end{pgfscope}%
\begin{pgfscope}%
\pgfpathrectangle{\pgfqpoint{2.125000in}{2.500000in}}{\pgfqpoint{5.489583in}{0.877907in}}%
\pgfusepath{clip}%
\pgfsetroundcap%
\pgfsetroundjoin%
\pgfsetlinewidth{0.803000pt}%
\definecolor{currentstroke}{rgb}{1.000000,1.000000,1.000000}%
\pgfsetstrokecolor{currentstroke}%
\pgfsetdash{}{0pt}%
\pgfpathmoveto{\pgfqpoint{6.071216in}{2.500000in}}%
\pgfpathlineto{\pgfqpoint{6.071216in}{3.377907in}}%
\pgfusepath{stroke}%
\end{pgfscope}%
\begin{pgfscope}%
\definecolor{textcolor}{rgb}{0.150000,0.150000,0.150000}%
\pgfsetstrokecolor{textcolor}%
\pgfsetfillcolor{textcolor}%
\pgftext[x=6.071216in,y=2.402778in,,top]{\color{textcolor}\rmfamily\fontsize{14.000000}{16.800000}\selectfont 30}%
\end{pgfscope}%
\begin{pgfscope}%
\pgfpathrectangle{\pgfqpoint{2.125000in}{2.500000in}}{\pgfqpoint{5.489583in}{0.877907in}}%
\pgfusepath{clip}%
\pgfsetroundcap%
\pgfsetroundjoin%
\pgfsetlinewidth{0.803000pt}%
\definecolor{currentstroke}{rgb}{1.000000,1.000000,1.000000}%
\pgfsetstrokecolor{currentstroke}%
\pgfsetdash{}{0pt}%
\pgfpathmoveto{\pgfqpoint{6.687330in}{2.500000in}}%
\pgfpathlineto{\pgfqpoint{6.687330in}{3.377907in}}%
\pgfusepath{stroke}%
\end{pgfscope}%
\begin{pgfscope}%
\definecolor{textcolor}{rgb}{0.150000,0.150000,0.150000}%
\pgfsetstrokecolor{textcolor}%
\pgfsetfillcolor{textcolor}%
\pgftext[x=6.687330in,y=2.402778in,,top]{\color{textcolor}\rmfamily\fontsize{14.000000}{16.800000}\selectfont 35}%
\end{pgfscope}%
\begin{pgfscope}%
\pgfpathrectangle{\pgfqpoint{2.125000in}{2.500000in}}{\pgfqpoint{5.489583in}{0.877907in}}%
\pgfusepath{clip}%
\pgfsetroundcap%
\pgfsetroundjoin%
\pgfsetlinewidth{0.803000pt}%
\definecolor{currentstroke}{rgb}{1.000000,1.000000,1.000000}%
\pgfsetstrokecolor{currentstroke}%
\pgfsetdash{}{0pt}%
\pgfpathmoveto{\pgfqpoint{7.303445in}{2.500000in}}%
\pgfpathlineto{\pgfqpoint{7.303445in}{3.377907in}}%
\pgfusepath{stroke}%
\end{pgfscope}%
\begin{pgfscope}%
\definecolor{textcolor}{rgb}{0.150000,0.150000,0.150000}%
\pgfsetstrokecolor{textcolor}%
\pgfsetfillcolor{textcolor}%
\pgftext[x=7.303445in,y=2.402778in,,top]{\color{textcolor}\rmfamily\fontsize{14.000000}{16.800000}\selectfont 40}%
\end{pgfscope}%
\begin{pgfscope}%
\pgfpathrectangle{\pgfqpoint{2.125000in}{2.500000in}}{\pgfqpoint{5.489583in}{0.877907in}}%
\pgfusepath{clip}%
\pgfsetroundcap%
\pgfsetroundjoin%
\pgfsetlinewidth{0.803000pt}%
\definecolor{currentstroke}{rgb}{1.000000,1.000000,1.000000}%
\pgfsetstrokecolor{currentstroke}%
\pgfsetdash{}{0pt}%
\pgfpathmoveto{\pgfqpoint{2.125000in}{2.778370in}}%
\pgfpathlineto{\pgfqpoint{7.614583in}{2.778370in}}%
\pgfusepath{stroke}%
\end{pgfscope}%
\begin{pgfscope}%
\definecolor{textcolor}{rgb}{0.150000,0.150000,0.150000}%
\pgfsetstrokecolor{textcolor}%
\pgfsetfillcolor{textcolor}%
\pgftext[x=1.904066in,y=2.704504in,left,base]{\color{textcolor}\rmfamily\fontsize{14.000000}{16.800000}\selectfont 0}%
\end{pgfscope}%
\begin{pgfscope}%
\pgfpathrectangle{\pgfqpoint{2.125000in}{2.500000in}}{\pgfqpoint{5.489583in}{0.877907in}}%
\pgfusepath{clip}%
\pgfsetroundcap%
\pgfsetroundjoin%
\pgfsetlinewidth{0.803000pt}%
\definecolor{currentstroke}{rgb}{1.000000,1.000000,1.000000}%
\pgfsetstrokecolor{currentstroke}%
\pgfsetdash{}{0pt}%
\pgfpathmoveto{\pgfqpoint{2.125000in}{3.338002in}}%
\pgfpathlineto{\pgfqpoint{7.614583in}{3.338002in}}%
\pgfusepath{stroke}%
\end{pgfscope}%
\begin{pgfscope}%
\definecolor{textcolor}{rgb}{0.150000,0.150000,0.150000}%
\pgfsetstrokecolor{textcolor}%
\pgfsetfillcolor{textcolor}%
\pgftext[x=1.904066in,y=3.264136in,left,base]{\color{textcolor}\rmfamily\fontsize{14.000000}{16.800000}\selectfont 1}%
\end{pgfscope}%
\begin{pgfscope}%
\pgfpathrectangle{\pgfqpoint{2.125000in}{2.500000in}}{\pgfqpoint{5.489583in}{0.877907in}}%
\pgfusepath{clip}%
\pgfsetbuttcap%
\pgfsetroundjoin%
\definecolor{currentfill}{rgb}{0.121569,0.466667,0.705882}%
\pgfsetfillcolor{currentfill}%
\pgfsetfillopacity{0.250000}%
\pgfsetlinewidth{1.003750pt}%
\definecolor{currentstroke}{rgb}{1.000000,1.000000,1.000000}%
\pgfsetstrokecolor{currentstroke}%
\pgfsetstrokeopacity{0.250000}%
\pgfsetdash{}{0pt}%
\pgfpathmoveto{\pgfqpoint{2.436138in}{2.806607in}}%
\pgfpathlineto{\pgfqpoint{2.436138in}{2.750134in}}%
\pgfpathlineto{\pgfqpoint{2.620972in}{2.729551in}}%
\pgfpathlineto{\pgfqpoint{2.744195in}{2.715434in}}%
\pgfpathlineto{\pgfqpoint{2.867418in}{2.704003in}}%
\pgfpathlineto{\pgfqpoint{2.990641in}{2.694155in}}%
\pgfpathlineto{\pgfqpoint{3.113864in}{2.685387in}}%
\pgfpathlineto{\pgfqpoint{3.237087in}{2.677415in}}%
\pgfpathlineto{\pgfqpoint{3.360310in}{2.670064in}}%
\pgfpathlineto{\pgfqpoint{3.483533in}{2.663218in}}%
\pgfpathlineto{\pgfqpoint{3.606756in}{2.656790in}}%
\pgfpathlineto{\pgfqpoint{3.729979in}{2.650719in}}%
\pgfpathlineto{\pgfqpoint{3.853202in}{2.644955in}}%
\pgfpathlineto{\pgfqpoint{3.976425in}{2.639458in}}%
\pgfpathlineto{\pgfqpoint{4.099648in}{2.634199in}}%
\pgfpathlineto{\pgfqpoint{4.222871in}{2.629152in}}%
\pgfpathlineto{\pgfqpoint{4.346094in}{2.624296in}}%
\pgfpathlineto{\pgfqpoint{4.469317in}{2.619611in}}%
\pgfpathlineto{\pgfqpoint{4.592540in}{2.615085in}}%
\pgfpathlineto{\pgfqpoint{4.715763in}{2.610705in}}%
\pgfpathlineto{\pgfqpoint{4.838986in}{2.606459in}}%
\pgfpathlineto{\pgfqpoint{4.962209in}{2.602339in}}%
\pgfpathlineto{\pgfqpoint{5.085432in}{2.598335in}}%
\pgfpathlineto{\pgfqpoint{5.208655in}{2.594439in}}%
\pgfpathlineto{\pgfqpoint{5.331878in}{2.590643in}}%
\pgfpathlineto{\pgfqpoint{5.455101in}{2.586940in}}%
\pgfpathlineto{\pgfqpoint{5.578324in}{2.583326in}}%
\pgfpathlineto{\pgfqpoint{5.701547in}{2.579795in}}%
\pgfpathlineto{\pgfqpoint{5.824770in}{2.576343in}}%
\pgfpathlineto{\pgfqpoint{5.947993in}{2.572967in}}%
\pgfpathlineto{\pgfqpoint{6.071216in}{2.569661in}}%
\pgfpathlineto{\pgfqpoint{6.194439in}{2.566424in}}%
\pgfpathlineto{\pgfqpoint{6.317662in}{2.563252in}}%
\pgfpathlineto{\pgfqpoint{6.440885in}{2.560142in}}%
\pgfpathlineto{\pgfqpoint{6.564108in}{2.557091in}}%
\pgfpathlineto{\pgfqpoint{6.687330in}{2.554097in}}%
\pgfpathlineto{\pgfqpoint{6.810553in}{2.551158in}}%
\pgfpathlineto{\pgfqpoint{6.933776in}{2.548272in}}%
\pgfpathlineto{\pgfqpoint{7.056999in}{2.545435in}}%
\pgfpathlineto{\pgfqpoint{7.180222in}{2.542647in}}%
\pgfpathlineto{\pgfqpoint{7.365057in}{2.539905in}}%
\pgfpathlineto{\pgfqpoint{7.365057in}{3.016836in}}%
\pgfpathlineto{\pgfqpoint{7.365057in}{3.016836in}}%
\pgfpathlineto{\pgfqpoint{7.180222in}{3.014094in}}%
\pgfpathlineto{\pgfqpoint{7.056999in}{3.011305in}}%
\pgfpathlineto{\pgfqpoint{6.933776in}{3.008469in}}%
\pgfpathlineto{\pgfqpoint{6.810553in}{3.005582in}}%
\pgfpathlineto{\pgfqpoint{6.687330in}{3.002643in}}%
\pgfpathlineto{\pgfqpoint{6.564108in}{2.999650in}}%
\pgfpathlineto{\pgfqpoint{6.440885in}{2.996599in}}%
\pgfpathlineto{\pgfqpoint{6.317662in}{2.993489in}}%
\pgfpathlineto{\pgfqpoint{6.194439in}{2.990317in}}%
\pgfpathlineto{\pgfqpoint{6.071216in}{2.987079in}}%
\pgfpathlineto{\pgfqpoint{5.947993in}{2.983774in}}%
\pgfpathlineto{\pgfqpoint{5.824770in}{2.980398in}}%
\pgfpathlineto{\pgfqpoint{5.701547in}{2.976946in}}%
\pgfpathlineto{\pgfqpoint{5.578324in}{2.973415in}}%
\pgfpathlineto{\pgfqpoint{5.455101in}{2.969801in}}%
\pgfpathlineto{\pgfqpoint{5.331878in}{2.966098in}}%
\pgfpathlineto{\pgfqpoint{5.208655in}{2.962302in}}%
\pgfpathlineto{\pgfqpoint{5.085432in}{2.958406in}}%
\pgfpathlineto{\pgfqpoint{4.962209in}{2.954401in}}%
\pgfpathlineto{\pgfqpoint{4.838986in}{2.950281in}}%
\pgfpathlineto{\pgfqpoint{4.715763in}{2.946036in}}%
\pgfpathlineto{\pgfqpoint{4.592540in}{2.941656in}}%
\pgfpathlineto{\pgfqpoint{4.469317in}{2.937129in}}%
\pgfpathlineto{\pgfqpoint{4.346094in}{2.932445in}}%
\pgfpathlineto{\pgfqpoint{4.222871in}{2.927589in}}%
\pgfpathlineto{\pgfqpoint{4.099648in}{2.922541in}}%
\pgfpathlineto{\pgfqpoint{3.976425in}{2.917282in}}%
\pgfpathlineto{\pgfqpoint{3.853202in}{2.911786in}}%
\pgfpathlineto{\pgfqpoint{3.729979in}{2.906022in}}%
\pgfpathlineto{\pgfqpoint{3.606756in}{2.899950in}}%
\pgfpathlineto{\pgfqpoint{3.483533in}{2.893523in}}%
\pgfpathlineto{\pgfqpoint{3.360310in}{2.886676in}}%
\pgfpathlineto{\pgfqpoint{3.237087in}{2.879326in}}%
\pgfpathlineto{\pgfqpoint{3.113864in}{2.871354in}}%
\pgfpathlineto{\pgfqpoint{2.990641in}{2.862586in}}%
\pgfpathlineto{\pgfqpoint{2.867418in}{2.852738in}}%
\pgfpathlineto{\pgfqpoint{2.744195in}{2.841307in}}%
\pgfpathlineto{\pgfqpoint{2.620972in}{2.827189in}}%
\pgfpathlineto{\pgfqpoint{2.436138in}{2.806607in}}%
\pgfpathclose%
\pgfusepath{stroke,fill}%
\end{pgfscope}%
\begin{pgfscope}%
\pgfpathrectangle{\pgfqpoint{2.125000in}{2.500000in}}{\pgfqpoint{5.489583in}{0.877907in}}%
\pgfusepath{clip}%
\pgfsetbuttcap%
\pgfsetroundjoin%
\pgfsetlinewidth{1.505625pt}%
\definecolor{currentstroke}{rgb}{0.000000,0.000000,0.000000}%
\pgfsetstrokecolor{currentstroke}%
\pgfsetdash{}{0pt}%
\pgfpathmoveto{\pgfqpoint{2.374527in}{2.778370in}}%
\pgfpathlineto{\pgfqpoint{2.374527in}{3.338002in}}%
\pgfusepath{stroke}%
\end{pgfscope}%
\begin{pgfscope}%
\pgfpathrectangle{\pgfqpoint{2.125000in}{2.500000in}}{\pgfqpoint{5.489583in}{0.877907in}}%
\pgfusepath{clip}%
\pgfsetbuttcap%
\pgfsetroundjoin%
\pgfsetlinewidth{1.505625pt}%
\definecolor{currentstroke}{rgb}{0.000000,0.000000,0.000000}%
\pgfsetstrokecolor{currentstroke}%
\pgfsetdash{}{0pt}%
\pgfpathmoveto{\pgfqpoint{2.497749in}{2.778370in}}%
\pgfpathlineto{\pgfqpoint{2.497749in}{3.336501in}}%
\pgfusepath{stroke}%
\end{pgfscope}%
\begin{pgfscope}%
\pgfpathrectangle{\pgfqpoint{2.125000in}{2.500000in}}{\pgfqpoint{5.489583in}{0.877907in}}%
\pgfusepath{clip}%
\pgfsetbuttcap%
\pgfsetroundjoin%
\pgfsetlinewidth{1.505625pt}%
\definecolor{currentstroke}{rgb}{0.000000,0.000000,0.000000}%
\pgfsetstrokecolor{currentstroke}%
\pgfsetdash{}{0pt}%
\pgfpathmoveto{\pgfqpoint{2.620972in}{2.778370in}}%
\pgfpathlineto{\pgfqpoint{2.620972in}{3.335030in}}%
\pgfusepath{stroke}%
\end{pgfscope}%
\begin{pgfscope}%
\pgfpathrectangle{\pgfqpoint{2.125000in}{2.500000in}}{\pgfqpoint{5.489583in}{0.877907in}}%
\pgfusepath{clip}%
\pgfsetbuttcap%
\pgfsetroundjoin%
\pgfsetlinewidth{1.505625pt}%
\definecolor{currentstroke}{rgb}{0.000000,0.000000,0.000000}%
\pgfsetstrokecolor{currentstroke}%
\pgfsetdash{}{0pt}%
\pgfpathmoveto{\pgfqpoint{2.744195in}{2.778370in}}%
\pgfpathlineto{\pgfqpoint{2.744195in}{3.333602in}}%
\pgfusepath{stroke}%
\end{pgfscope}%
\begin{pgfscope}%
\pgfpathrectangle{\pgfqpoint{2.125000in}{2.500000in}}{\pgfqpoint{5.489583in}{0.877907in}}%
\pgfusepath{clip}%
\pgfsetbuttcap%
\pgfsetroundjoin%
\pgfsetlinewidth{1.505625pt}%
\definecolor{currentstroke}{rgb}{0.000000,0.000000,0.000000}%
\pgfsetstrokecolor{currentstroke}%
\pgfsetdash{}{0pt}%
\pgfpathmoveto{\pgfqpoint{2.867418in}{2.778370in}}%
\pgfpathlineto{\pgfqpoint{2.867418in}{3.332197in}}%
\pgfusepath{stroke}%
\end{pgfscope}%
\begin{pgfscope}%
\pgfpathrectangle{\pgfqpoint{2.125000in}{2.500000in}}{\pgfqpoint{5.489583in}{0.877907in}}%
\pgfusepath{clip}%
\pgfsetbuttcap%
\pgfsetroundjoin%
\pgfsetlinewidth{1.505625pt}%
\definecolor{currentstroke}{rgb}{0.000000,0.000000,0.000000}%
\pgfsetstrokecolor{currentstroke}%
\pgfsetdash{}{0pt}%
\pgfpathmoveto{\pgfqpoint{2.990641in}{2.778370in}}%
\pgfpathlineto{\pgfqpoint{2.990641in}{3.330797in}}%
\pgfusepath{stroke}%
\end{pgfscope}%
\begin{pgfscope}%
\pgfpathrectangle{\pgfqpoint{2.125000in}{2.500000in}}{\pgfqpoint{5.489583in}{0.877907in}}%
\pgfusepath{clip}%
\pgfsetbuttcap%
\pgfsetroundjoin%
\pgfsetlinewidth{1.505625pt}%
\definecolor{currentstroke}{rgb}{0.000000,0.000000,0.000000}%
\pgfsetstrokecolor{currentstroke}%
\pgfsetdash{}{0pt}%
\pgfpathmoveto{\pgfqpoint{3.113864in}{2.778370in}}%
\pgfpathlineto{\pgfqpoint{3.113864in}{3.329429in}}%
\pgfusepath{stroke}%
\end{pgfscope}%
\begin{pgfscope}%
\pgfpathrectangle{\pgfqpoint{2.125000in}{2.500000in}}{\pgfqpoint{5.489583in}{0.877907in}}%
\pgfusepath{clip}%
\pgfsetbuttcap%
\pgfsetroundjoin%
\pgfsetlinewidth{1.505625pt}%
\definecolor{currentstroke}{rgb}{0.000000,0.000000,0.000000}%
\pgfsetstrokecolor{currentstroke}%
\pgfsetdash{}{0pt}%
\pgfpathmoveto{\pgfqpoint{3.237087in}{2.778370in}}%
\pgfpathlineto{\pgfqpoint{3.237087in}{3.328006in}}%
\pgfusepath{stroke}%
\end{pgfscope}%
\begin{pgfscope}%
\pgfpathrectangle{\pgfqpoint{2.125000in}{2.500000in}}{\pgfqpoint{5.489583in}{0.877907in}}%
\pgfusepath{clip}%
\pgfsetbuttcap%
\pgfsetroundjoin%
\pgfsetlinewidth{1.505625pt}%
\definecolor{currentstroke}{rgb}{0.000000,0.000000,0.000000}%
\pgfsetstrokecolor{currentstroke}%
\pgfsetdash{}{0pt}%
\pgfpathmoveto{\pgfqpoint{3.360310in}{2.778370in}}%
\pgfpathlineto{\pgfqpoint{3.360310in}{3.326544in}}%
\pgfusepath{stroke}%
\end{pgfscope}%
\begin{pgfscope}%
\pgfpathrectangle{\pgfqpoint{2.125000in}{2.500000in}}{\pgfqpoint{5.489583in}{0.877907in}}%
\pgfusepath{clip}%
\pgfsetbuttcap%
\pgfsetroundjoin%
\pgfsetlinewidth{1.505625pt}%
\definecolor{currentstroke}{rgb}{0.000000,0.000000,0.000000}%
\pgfsetstrokecolor{currentstroke}%
\pgfsetdash{}{0pt}%
\pgfpathmoveto{\pgfqpoint{3.483533in}{2.778370in}}%
\pgfpathlineto{\pgfqpoint{3.483533in}{3.325041in}}%
\pgfusepath{stroke}%
\end{pgfscope}%
\begin{pgfscope}%
\pgfpathrectangle{\pgfqpoint{2.125000in}{2.500000in}}{\pgfqpoint{5.489583in}{0.877907in}}%
\pgfusepath{clip}%
\pgfsetbuttcap%
\pgfsetroundjoin%
\pgfsetlinewidth{1.505625pt}%
\definecolor{currentstroke}{rgb}{0.000000,0.000000,0.000000}%
\pgfsetstrokecolor{currentstroke}%
\pgfsetdash{}{0pt}%
\pgfpathmoveto{\pgfqpoint{3.606756in}{2.778370in}}%
\pgfpathlineto{\pgfqpoint{3.606756in}{3.323537in}}%
\pgfusepath{stroke}%
\end{pgfscope}%
\begin{pgfscope}%
\pgfpathrectangle{\pgfqpoint{2.125000in}{2.500000in}}{\pgfqpoint{5.489583in}{0.877907in}}%
\pgfusepath{clip}%
\pgfsetbuttcap%
\pgfsetroundjoin%
\pgfsetlinewidth{1.505625pt}%
\definecolor{currentstroke}{rgb}{0.000000,0.000000,0.000000}%
\pgfsetstrokecolor{currentstroke}%
\pgfsetdash{}{0pt}%
\pgfpathmoveto{\pgfqpoint{3.729979in}{2.778370in}}%
\pgfpathlineto{\pgfqpoint{3.729979in}{3.322044in}}%
\pgfusepath{stroke}%
\end{pgfscope}%
\begin{pgfscope}%
\pgfpathrectangle{\pgfqpoint{2.125000in}{2.500000in}}{\pgfqpoint{5.489583in}{0.877907in}}%
\pgfusepath{clip}%
\pgfsetbuttcap%
\pgfsetroundjoin%
\pgfsetlinewidth{1.505625pt}%
\definecolor{currentstroke}{rgb}{0.000000,0.000000,0.000000}%
\pgfsetstrokecolor{currentstroke}%
\pgfsetdash{}{0pt}%
\pgfpathmoveto{\pgfqpoint{3.853202in}{2.778370in}}%
\pgfpathlineto{\pgfqpoint{3.853202in}{3.320573in}}%
\pgfusepath{stroke}%
\end{pgfscope}%
\begin{pgfscope}%
\pgfpathrectangle{\pgfqpoint{2.125000in}{2.500000in}}{\pgfqpoint{5.489583in}{0.877907in}}%
\pgfusepath{clip}%
\pgfsetbuttcap%
\pgfsetroundjoin%
\pgfsetlinewidth{1.505625pt}%
\definecolor{currentstroke}{rgb}{0.000000,0.000000,0.000000}%
\pgfsetstrokecolor{currentstroke}%
\pgfsetdash{}{0pt}%
\pgfpathmoveto{\pgfqpoint{3.976425in}{2.778370in}}%
\pgfpathlineto{\pgfqpoint{3.976425in}{3.319111in}}%
\pgfusepath{stroke}%
\end{pgfscope}%
\begin{pgfscope}%
\pgfpathrectangle{\pgfqpoint{2.125000in}{2.500000in}}{\pgfqpoint{5.489583in}{0.877907in}}%
\pgfusepath{clip}%
\pgfsetbuttcap%
\pgfsetroundjoin%
\pgfsetlinewidth{1.505625pt}%
\definecolor{currentstroke}{rgb}{0.000000,0.000000,0.000000}%
\pgfsetstrokecolor{currentstroke}%
\pgfsetdash{}{0pt}%
\pgfpathmoveto{\pgfqpoint{4.099648in}{2.778370in}}%
\pgfpathlineto{\pgfqpoint{4.099648in}{3.317665in}}%
\pgfusepath{stroke}%
\end{pgfscope}%
\begin{pgfscope}%
\pgfpathrectangle{\pgfqpoint{2.125000in}{2.500000in}}{\pgfqpoint{5.489583in}{0.877907in}}%
\pgfusepath{clip}%
\pgfsetbuttcap%
\pgfsetroundjoin%
\pgfsetlinewidth{1.505625pt}%
\definecolor{currentstroke}{rgb}{0.000000,0.000000,0.000000}%
\pgfsetstrokecolor{currentstroke}%
\pgfsetdash{}{0pt}%
\pgfpathmoveto{\pgfqpoint{4.222871in}{2.778370in}}%
\pgfpathlineto{\pgfqpoint{4.222871in}{3.316243in}}%
\pgfusepath{stroke}%
\end{pgfscope}%
\begin{pgfscope}%
\pgfpathrectangle{\pgfqpoint{2.125000in}{2.500000in}}{\pgfqpoint{5.489583in}{0.877907in}}%
\pgfusepath{clip}%
\pgfsetbuttcap%
\pgfsetroundjoin%
\pgfsetlinewidth{1.505625pt}%
\definecolor{currentstroke}{rgb}{0.000000,0.000000,0.000000}%
\pgfsetstrokecolor{currentstroke}%
\pgfsetdash{}{0pt}%
\pgfpathmoveto{\pgfqpoint{4.346094in}{2.778370in}}%
\pgfpathlineto{\pgfqpoint{4.346094in}{3.314848in}}%
\pgfusepath{stroke}%
\end{pgfscope}%
\begin{pgfscope}%
\pgfpathrectangle{\pgfqpoint{2.125000in}{2.500000in}}{\pgfqpoint{5.489583in}{0.877907in}}%
\pgfusepath{clip}%
\pgfsetbuttcap%
\pgfsetroundjoin%
\pgfsetlinewidth{1.505625pt}%
\definecolor{currentstroke}{rgb}{0.000000,0.000000,0.000000}%
\pgfsetstrokecolor{currentstroke}%
\pgfsetdash{}{0pt}%
\pgfpathmoveto{\pgfqpoint{4.469317in}{2.778370in}}%
\pgfpathlineto{\pgfqpoint{4.469317in}{3.313441in}}%
\pgfusepath{stroke}%
\end{pgfscope}%
\begin{pgfscope}%
\pgfpathrectangle{\pgfqpoint{2.125000in}{2.500000in}}{\pgfqpoint{5.489583in}{0.877907in}}%
\pgfusepath{clip}%
\pgfsetbuttcap%
\pgfsetroundjoin%
\pgfsetlinewidth{1.505625pt}%
\definecolor{currentstroke}{rgb}{0.000000,0.000000,0.000000}%
\pgfsetstrokecolor{currentstroke}%
\pgfsetdash{}{0pt}%
\pgfpathmoveto{\pgfqpoint{4.592540in}{2.778370in}}%
\pgfpathlineto{\pgfqpoint{4.592540in}{3.311986in}}%
\pgfusepath{stroke}%
\end{pgfscope}%
\begin{pgfscope}%
\pgfpathrectangle{\pgfqpoint{2.125000in}{2.500000in}}{\pgfqpoint{5.489583in}{0.877907in}}%
\pgfusepath{clip}%
\pgfsetbuttcap%
\pgfsetroundjoin%
\pgfsetlinewidth{1.505625pt}%
\definecolor{currentstroke}{rgb}{0.000000,0.000000,0.000000}%
\pgfsetstrokecolor{currentstroke}%
\pgfsetdash{}{0pt}%
\pgfpathmoveto{\pgfqpoint{4.715763in}{2.778370in}}%
\pgfpathlineto{\pgfqpoint{4.715763in}{3.310478in}}%
\pgfusepath{stroke}%
\end{pgfscope}%
\begin{pgfscope}%
\pgfpathrectangle{\pgfqpoint{2.125000in}{2.500000in}}{\pgfqpoint{5.489583in}{0.877907in}}%
\pgfusepath{clip}%
\pgfsetbuttcap%
\pgfsetroundjoin%
\pgfsetlinewidth{1.505625pt}%
\definecolor{currentstroke}{rgb}{0.000000,0.000000,0.000000}%
\pgfsetstrokecolor{currentstroke}%
\pgfsetdash{}{0pt}%
\pgfpathmoveto{\pgfqpoint{4.838986in}{2.778370in}}%
\pgfpathlineto{\pgfqpoint{4.838986in}{3.308997in}}%
\pgfusepath{stroke}%
\end{pgfscope}%
\begin{pgfscope}%
\pgfpathrectangle{\pgfqpoint{2.125000in}{2.500000in}}{\pgfqpoint{5.489583in}{0.877907in}}%
\pgfusepath{clip}%
\pgfsetbuttcap%
\pgfsetroundjoin%
\pgfsetlinewidth{1.505625pt}%
\definecolor{currentstroke}{rgb}{0.000000,0.000000,0.000000}%
\pgfsetstrokecolor{currentstroke}%
\pgfsetdash{}{0pt}%
\pgfpathmoveto{\pgfqpoint{4.962209in}{2.778370in}}%
\pgfpathlineto{\pgfqpoint{4.962209in}{3.307567in}}%
\pgfusepath{stroke}%
\end{pgfscope}%
\begin{pgfscope}%
\pgfpathrectangle{\pgfqpoint{2.125000in}{2.500000in}}{\pgfqpoint{5.489583in}{0.877907in}}%
\pgfusepath{clip}%
\pgfsetbuttcap%
\pgfsetroundjoin%
\pgfsetlinewidth{1.505625pt}%
\definecolor{currentstroke}{rgb}{0.000000,0.000000,0.000000}%
\pgfsetstrokecolor{currentstroke}%
\pgfsetdash{}{0pt}%
\pgfpathmoveto{\pgfqpoint{5.085432in}{2.778370in}}%
\pgfpathlineto{\pgfqpoint{5.085432in}{3.306120in}}%
\pgfusepath{stroke}%
\end{pgfscope}%
\begin{pgfscope}%
\pgfpathrectangle{\pgfqpoint{2.125000in}{2.500000in}}{\pgfqpoint{5.489583in}{0.877907in}}%
\pgfusepath{clip}%
\pgfsetbuttcap%
\pgfsetroundjoin%
\pgfsetlinewidth{1.505625pt}%
\definecolor{currentstroke}{rgb}{0.000000,0.000000,0.000000}%
\pgfsetstrokecolor{currentstroke}%
\pgfsetdash{}{0pt}%
\pgfpathmoveto{\pgfqpoint{5.208655in}{2.778370in}}%
\pgfpathlineto{\pgfqpoint{5.208655in}{3.304781in}}%
\pgfusepath{stroke}%
\end{pgfscope}%
\begin{pgfscope}%
\pgfpathrectangle{\pgfqpoint{2.125000in}{2.500000in}}{\pgfqpoint{5.489583in}{0.877907in}}%
\pgfusepath{clip}%
\pgfsetbuttcap%
\pgfsetroundjoin%
\pgfsetlinewidth{1.505625pt}%
\definecolor{currentstroke}{rgb}{0.000000,0.000000,0.000000}%
\pgfsetstrokecolor{currentstroke}%
\pgfsetdash{}{0pt}%
\pgfpathmoveto{\pgfqpoint{5.331878in}{2.778370in}}%
\pgfpathlineto{\pgfqpoint{5.331878in}{3.303469in}}%
\pgfusepath{stroke}%
\end{pgfscope}%
\begin{pgfscope}%
\pgfpathrectangle{\pgfqpoint{2.125000in}{2.500000in}}{\pgfqpoint{5.489583in}{0.877907in}}%
\pgfusepath{clip}%
\pgfsetbuttcap%
\pgfsetroundjoin%
\pgfsetlinewidth{1.505625pt}%
\definecolor{currentstroke}{rgb}{0.000000,0.000000,0.000000}%
\pgfsetstrokecolor{currentstroke}%
\pgfsetdash{}{0pt}%
\pgfpathmoveto{\pgfqpoint{5.455101in}{2.778370in}}%
\pgfpathlineto{\pgfqpoint{5.455101in}{3.302157in}}%
\pgfusepath{stroke}%
\end{pgfscope}%
\begin{pgfscope}%
\pgfpathrectangle{\pgfqpoint{2.125000in}{2.500000in}}{\pgfqpoint{5.489583in}{0.877907in}}%
\pgfusepath{clip}%
\pgfsetbuttcap%
\pgfsetroundjoin%
\pgfsetlinewidth{1.505625pt}%
\definecolor{currentstroke}{rgb}{0.000000,0.000000,0.000000}%
\pgfsetstrokecolor{currentstroke}%
\pgfsetdash{}{0pt}%
\pgfpathmoveto{\pgfqpoint{5.578324in}{2.778370in}}%
\pgfpathlineto{\pgfqpoint{5.578324in}{3.300837in}}%
\pgfusepath{stroke}%
\end{pgfscope}%
\begin{pgfscope}%
\pgfpathrectangle{\pgfqpoint{2.125000in}{2.500000in}}{\pgfqpoint{5.489583in}{0.877907in}}%
\pgfusepath{clip}%
\pgfsetbuttcap%
\pgfsetroundjoin%
\pgfsetlinewidth{1.505625pt}%
\definecolor{currentstroke}{rgb}{0.000000,0.000000,0.000000}%
\pgfsetstrokecolor{currentstroke}%
\pgfsetdash{}{0pt}%
\pgfpathmoveto{\pgfqpoint{5.701547in}{2.778370in}}%
\pgfpathlineto{\pgfqpoint{5.701547in}{3.299513in}}%
\pgfusepath{stroke}%
\end{pgfscope}%
\begin{pgfscope}%
\pgfpathrectangle{\pgfqpoint{2.125000in}{2.500000in}}{\pgfqpoint{5.489583in}{0.877907in}}%
\pgfusepath{clip}%
\pgfsetbuttcap%
\pgfsetroundjoin%
\pgfsetlinewidth{1.505625pt}%
\definecolor{currentstroke}{rgb}{0.000000,0.000000,0.000000}%
\pgfsetstrokecolor{currentstroke}%
\pgfsetdash{}{0pt}%
\pgfpathmoveto{\pgfqpoint{5.824770in}{2.778370in}}%
\pgfpathlineto{\pgfqpoint{5.824770in}{3.298173in}}%
\pgfusepath{stroke}%
\end{pgfscope}%
\begin{pgfscope}%
\pgfpathrectangle{\pgfqpoint{2.125000in}{2.500000in}}{\pgfqpoint{5.489583in}{0.877907in}}%
\pgfusepath{clip}%
\pgfsetbuttcap%
\pgfsetroundjoin%
\pgfsetlinewidth{1.505625pt}%
\definecolor{currentstroke}{rgb}{0.000000,0.000000,0.000000}%
\pgfsetstrokecolor{currentstroke}%
\pgfsetdash{}{0pt}%
\pgfpathmoveto{\pgfqpoint{5.947993in}{2.778370in}}%
\pgfpathlineto{\pgfqpoint{5.947993in}{3.296870in}}%
\pgfusepath{stroke}%
\end{pgfscope}%
\begin{pgfscope}%
\pgfpathrectangle{\pgfqpoint{2.125000in}{2.500000in}}{\pgfqpoint{5.489583in}{0.877907in}}%
\pgfusepath{clip}%
\pgfsetbuttcap%
\pgfsetroundjoin%
\pgfsetlinewidth{1.505625pt}%
\definecolor{currentstroke}{rgb}{0.000000,0.000000,0.000000}%
\pgfsetstrokecolor{currentstroke}%
\pgfsetdash{}{0pt}%
\pgfpathmoveto{\pgfqpoint{6.071216in}{2.778370in}}%
\pgfpathlineto{\pgfqpoint{6.071216in}{3.295553in}}%
\pgfusepath{stroke}%
\end{pgfscope}%
\begin{pgfscope}%
\pgfpathrectangle{\pgfqpoint{2.125000in}{2.500000in}}{\pgfqpoint{5.489583in}{0.877907in}}%
\pgfusepath{clip}%
\pgfsetbuttcap%
\pgfsetroundjoin%
\pgfsetlinewidth{1.505625pt}%
\definecolor{currentstroke}{rgb}{0.000000,0.000000,0.000000}%
\pgfsetstrokecolor{currentstroke}%
\pgfsetdash{}{0pt}%
\pgfpathmoveto{\pgfqpoint{6.194439in}{2.778370in}}%
\pgfpathlineto{\pgfqpoint{6.194439in}{3.294204in}}%
\pgfusepath{stroke}%
\end{pgfscope}%
\begin{pgfscope}%
\pgfpathrectangle{\pgfqpoint{2.125000in}{2.500000in}}{\pgfqpoint{5.489583in}{0.877907in}}%
\pgfusepath{clip}%
\pgfsetbuttcap%
\pgfsetroundjoin%
\pgfsetlinewidth{1.505625pt}%
\definecolor{currentstroke}{rgb}{0.000000,0.000000,0.000000}%
\pgfsetstrokecolor{currentstroke}%
\pgfsetdash{}{0pt}%
\pgfpathmoveto{\pgfqpoint{6.317662in}{2.778370in}}%
\pgfpathlineto{\pgfqpoint{6.317662in}{3.292871in}}%
\pgfusepath{stroke}%
\end{pgfscope}%
\begin{pgfscope}%
\pgfpathrectangle{\pgfqpoint{2.125000in}{2.500000in}}{\pgfqpoint{5.489583in}{0.877907in}}%
\pgfusepath{clip}%
\pgfsetbuttcap%
\pgfsetroundjoin%
\pgfsetlinewidth{1.505625pt}%
\definecolor{currentstroke}{rgb}{0.000000,0.000000,0.000000}%
\pgfsetstrokecolor{currentstroke}%
\pgfsetdash{}{0pt}%
\pgfpathmoveto{\pgfqpoint{6.440885in}{2.778370in}}%
\pgfpathlineto{\pgfqpoint{6.440885in}{3.291550in}}%
\pgfusepath{stroke}%
\end{pgfscope}%
\begin{pgfscope}%
\pgfpathrectangle{\pgfqpoint{2.125000in}{2.500000in}}{\pgfqpoint{5.489583in}{0.877907in}}%
\pgfusepath{clip}%
\pgfsetbuttcap%
\pgfsetroundjoin%
\pgfsetlinewidth{1.505625pt}%
\definecolor{currentstroke}{rgb}{0.000000,0.000000,0.000000}%
\pgfsetstrokecolor{currentstroke}%
\pgfsetdash{}{0pt}%
\pgfpathmoveto{\pgfqpoint{6.564108in}{2.778370in}}%
\pgfpathlineto{\pgfqpoint{6.564108in}{3.290213in}}%
\pgfusepath{stroke}%
\end{pgfscope}%
\begin{pgfscope}%
\pgfpathrectangle{\pgfqpoint{2.125000in}{2.500000in}}{\pgfqpoint{5.489583in}{0.877907in}}%
\pgfusepath{clip}%
\pgfsetbuttcap%
\pgfsetroundjoin%
\pgfsetlinewidth{1.505625pt}%
\definecolor{currentstroke}{rgb}{0.000000,0.000000,0.000000}%
\pgfsetstrokecolor{currentstroke}%
\pgfsetdash{}{0pt}%
\pgfpathmoveto{\pgfqpoint{6.687330in}{2.778370in}}%
\pgfpathlineto{\pgfqpoint{6.687330in}{3.288874in}}%
\pgfusepath{stroke}%
\end{pgfscope}%
\begin{pgfscope}%
\pgfpathrectangle{\pgfqpoint{2.125000in}{2.500000in}}{\pgfqpoint{5.489583in}{0.877907in}}%
\pgfusepath{clip}%
\pgfsetbuttcap%
\pgfsetroundjoin%
\pgfsetlinewidth{1.505625pt}%
\definecolor{currentstroke}{rgb}{0.000000,0.000000,0.000000}%
\pgfsetstrokecolor{currentstroke}%
\pgfsetdash{}{0pt}%
\pgfpathmoveto{\pgfqpoint{6.810553in}{2.778370in}}%
\pgfpathlineto{\pgfqpoint{6.810553in}{3.287570in}}%
\pgfusepath{stroke}%
\end{pgfscope}%
\begin{pgfscope}%
\pgfpathrectangle{\pgfqpoint{2.125000in}{2.500000in}}{\pgfqpoint{5.489583in}{0.877907in}}%
\pgfusepath{clip}%
\pgfsetbuttcap%
\pgfsetroundjoin%
\pgfsetlinewidth{1.505625pt}%
\definecolor{currentstroke}{rgb}{0.000000,0.000000,0.000000}%
\pgfsetstrokecolor{currentstroke}%
\pgfsetdash{}{0pt}%
\pgfpathmoveto{\pgfqpoint{6.933776in}{2.778370in}}%
\pgfpathlineto{\pgfqpoint{6.933776in}{3.286257in}}%
\pgfusepath{stroke}%
\end{pgfscope}%
\begin{pgfscope}%
\pgfpathrectangle{\pgfqpoint{2.125000in}{2.500000in}}{\pgfqpoint{5.489583in}{0.877907in}}%
\pgfusepath{clip}%
\pgfsetbuttcap%
\pgfsetroundjoin%
\pgfsetlinewidth{1.505625pt}%
\definecolor{currentstroke}{rgb}{0.000000,0.000000,0.000000}%
\pgfsetstrokecolor{currentstroke}%
\pgfsetdash{}{0pt}%
\pgfpathmoveto{\pgfqpoint{7.056999in}{2.778370in}}%
\pgfpathlineto{\pgfqpoint{7.056999in}{3.284967in}}%
\pgfusepath{stroke}%
\end{pgfscope}%
\begin{pgfscope}%
\pgfpathrectangle{\pgfqpoint{2.125000in}{2.500000in}}{\pgfqpoint{5.489583in}{0.877907in}}%
\pgfusepath{clip}%
\pgfsetbuttcap%
\pgfsetroundjoin%
\pgfsetlinewidth{1.505625pt}%
\definecolor{currentstroke}{rgb}{0.000000,0.000000,0.000000}%
\pgfsetstrokecolor{currentstroke}%
\pgfsetdash{}{0pt}%
\pgfpathmoveto{\pgfqpoint{7.180222in}{2.778370in}}%
\pgfpathlineto{\pgfqpoint{7.180222in}{3.283749in}}%
\pgfusepath{stroke}%
\end{pgfscope}%
\begin{pgfscope}%
\pgfpathrectangle{\pgfqpoint{2.125000in}{2.500000in}}{\pgfqpoint{5.489583in}{0.877907in}}%
\pgfusepath{clip}%
\pgfsetbuttcap%
\pgfsetroundjoin%
\pgfsetlinewidth{1.505625pt}%
\definecolor{currentstroke}{rgb}{0.000000,0.000000,0.000000}%
\pgfsetstrokecolor{currentstroke}%
\pgfsetdash{}{0pt}%
\pgfpathmoveto{\pgfqpoint{7.303445in}{2.778370in}}%
\pgfpathlineto{\pgfqpoint{7.303445in}{3.282568in}}%
\pgfusepath{stroke}%
\end{pgfscope}%
\begin{pgfscope}%
\pgfpathrectangle{\pgfqpoint{2.125000in}{2.500000in}}{\pgfqpoint{5.489583in}{0.877907in}}%
\pgfusepath{clip}%
\pgfsetroundcap%
\pgfsetroundjoin%
\pgfsetlinewidth{1.505625pt}%
\definecolor{currentstroke}{rgb}{0.121569,0.466667,0.705882}%
\pgfsetstrokecolor{currentstroke}%
\pgfsetdash{}{0pt}%
\pgfpathmoveto{\pgfqpoint{2.125000in}{2.778370in}}%
\pgfpathlineto{\pgfqpoint{7.614583in}{2.778370in}}%
\pgfusepath{stroke}%
\end{pgfscope}%
\begin{pgfscope}%
\pgfpathrectangle{\pgfqpoint{2.125000in}{2.500000in}}{\pgfqpoint{5.489583in}{0.877907in}}%
\pgfusepath{clip}%
\pgfsetbuttcap%
\pgfsetroundjoin%
\definecolor{currentfill}{rgb}{0.121569,0.466667,0.705882}%
\pgfsetfillcolor{currentfill}%
\pgfsetlinewidth{1.003750pt}%
\definecolor{currentstroke}{rgb}{0.121569,0.466667,0.705882}%
\pgfsetstrokecolor{currentstroke}%
\pgfsetdash{}{0pt}%
\pgfsys@defobject{currentmarker}{\pgfqpoint{-0.034722in}{-0.034722in}}{\pgfqpoint{0.034722in}{0.034722in}}{%
\pgfpathmoveto{\pgfqpoint{0.000000in}{-0.034722in}}%
\pgfpathcurveto{\pgfqpoint{0.009208in}{-0.034722in}}{\pgfqpoint{0.018041in}{-0.031064in}}{\pgfqpoint{0.024552in}{-0.024552in}}%
\pgfpathcurveto{\pgfqpoint{0.031064in}{-0.018041in}}{\pgfqpoint{0.034722in}{-0.009208in}}{\pgfqpoint{0.034722in}{0.000000in}}%
\pgfpathcurveto{\pgfqpoint{0.034722in}{0.009208in}}{\pgfqpoint{0.031064in}{0.018041in}}{\pgfqpoint{0.024552in}{0.024552in}}%
\pgfpathcurveto{\pgfqpoint{0.018041in}{0.031064in}}{\pgfqpoint{0.009208in}{0.034722in}}{\pgfqpoint{0.000000in}{0.034722in}}%
\pgfpathcurveto{\pgfqpoint{-0.009208in}{0.034722in}}{\pgfqpoint{-0.018041in}{0.031064in}}{\pgfqpoint{-0.024552in}{0.024552in}}%
\pgfpathcurveto{\pgfqpoint{-0.031064in}{0.018041in}}{\pgfqpoint{-0.034722in}{0.009208in}}{\pgfqpoint{-0.034722in}{0.000000in}}%
\pgfpathcurveto{\pgfqpoint{-0.034722in}{-0.009208in}}{\pgfqpoint{-0.031064in}{-0.018041in}}{\pgfqpoint{-0.024552in}{-0.024552in}}%
\pgfpathcurveto{\pgfqpoint{-0.018041in}{-0.031064in}}{\pgfqpoint{-0.009208in}{-0.034722in}}{\pgfqpoint{0.000000in}{-0.034722in}}%
\pgfpathclose%
\pgfusepath{stroke,fill}%
}%
\begin{pgfscope}%
\pgfsys@transformshift{2.374527in}{3.338002in}%
\pgfsys@useobject{currentmarker}{}%
\end{pgfscope}%
\begin{pgfscope}%
\pgfsys@transformshift{2.497749in}{3.336501in}%
\pgfsys@useobject{currentmarker}{}%
\end{pgfscope}%
\begin{pgfscope}%
\pgfsys@transformshift{2.620972in}{3.335030in}%
\pgfsys@useobject{currentmarker}{}%
\end{pgfscope}%
\begin{pgfscope}%
\pgfsys@transformshift{2.744195in}{3.333602in}%
\pgfsys@useobject{currentmarker}{}%
\end{pgfscope}%
\begin{pgfscope}%
\pgfsys@transformshift{2.867418in}{3.332197in}%
\pgfsys@useobject{currentmarker}{}%
\end{pgfscope}%
\begin{pgfscope}%
\pgfsys@transformshift{2.990641in}{3.330797in}%
\pgfsys@useobject{currentmarker}{}%
\end{pgfscope}%
\begin{pgfscope}%
\pgfsys@transformshift{3.113864in}{3.329429in}%
\pgfsys@useobject{currentmarker}{}%
\end{pgfscope}%
\begin{pgfscope}%
\pgfsys@transformshift{3.237087in}{3.328006in}%
\pgfsys@useobject{currentmarker}{}%
\end{pgfscope}%
\begin{pgfscope}%
\pgfsys@transformshift{3.360310in}{3.326544in}%
\pgfsys@useobject{currentmarker}{}%
\end{pgfscope}%
\begin{pgfscope}%
\pgfsys@transformshift{3.483533in}{3.325041in}%
\pgfsys@useobject{currentmarker}{}%
\end{pgfscope}%
\begin{pgfscope}%
\pgfsys@transformshift{3.606756in}{3.323537in}%
\pgfsys@useobject{currentmarker}{}%
\end{pgfscope}%
\begin{pgfscope}%
\pgfsys@transformshift{3.729979in}{3.322044in}%
\pgfsys@useobject{currentmarker}{}%
\end{pgfscope}%
\begin{pgfscope}%
\pgfsys@transformshift{3.853202in}{3.320573in}%
\pgfsys@useobject{currentmarker}{}%
\end{pgfscope}%
\begin{pgfscope}%
\pgfsys@transformshift{3.976425in}{3.319111in}%
\pgfsys@useobject{currentmarker}{}%
\end{pgfscope}%
\begin{pgfscope}%
\pgfsys@transformshift{4.099648in}{3.317665in}%
\pgfsys@useobject{currentmarker}{}%
\end{pgfscope}%
\begin{pgfscope}%
\pgfsys@transformshift{4.222871in}{3.316243in}%
\pgfsys@useobject{currentmarker}{}%
\end{pgfscope}%
\begin{pgfscope}%
\pgfsys@transformshift{4.346094in}{3.314848in}%
\pgfsys@useobject{currentmarker}{}%
\end{pgfscope}%
\begin{pgfscope}%
\pgfsys@transformshift{4.469317in}{3.313441in}%
\pgfsys@useobject{currentmarker}{}%
\end{pgfscope}%
\begin{pgfscope}%
\pgfsys@transformshift{4.592540in}{3.311986in}%
\pgfsys@useobject{currentmarker}{}%
\end{pgfscope}%
\begin{pgfscope}%
\pgfsys@transformshift{4.715763in}{3.310478in}%
\pgfsys@useobject{currentmarker}{}%
\end{pgfscope}%
\begin{pgfscope}%
\pgfsys@transformshift{4.838986in}{3.308997in}%
\pgfsys@useobject{currentmarker}{}%
\end{pgfscope}%
\begin{pgfscope}%
\pgfsys@transformshift{4.962209in}{3.307567in}%
\pgfsys@useobject{currentmarker}{}%
\end{pgfscope}%
\begin{pgfscope}%
\pgfsys@transformshift{5.085432in}{3.306120in}%
\pgfsys@useobject{currentmarker}{}%
\end{pgfscope}%
\begin{pgfscope}%
\pgfsys@transformshift{5.208655in}{3.304781in}%
\pgfsys@useobject{currentmarker}{}%
\end{pgfscope}%
\begin{pgfscope}%
\pgfsys@transformshift{5.331878in}{3.303469in}%
\pgfsys@useobject{currentmarker}{}%
\end{pgfscope}%
\begin{pgfscope}%
\pgfsys@transformshift{5.455101in}{3.302157in}%
\pgfsys@useobject{currentmarker}{}%
\end{pgfscope}%
\begin{pgfscope}%
\pgfsys@transformshift{5.578324in}{3.300837in}%
\pgfsys@useobject{currentmarker}{}%
\end{pgfscope}%
\begin{pgfscope}%
\pgfsys@transformshift{5.701547in}{3.299513in}%
\pgfsys@useobject{currentmarker}{}%
\end{pgfscope}%
\begin{pgfscope}%
\pgfsys@transformshift{5.824770in}{3.298173in}%
\pgfsys@useobject{currentmarker}{}%
\end{pgfscope}%
\begin{pgfscope}%
\pgfsys@transformshift{5.947993in}{3.296870in}%
\pgfsys@useobject{currentmarker}{}%
\end{pgfscope}%
\begin{pgfscope}%
\pgfsys@transformshift{6.071216in}{3.295553in}%
\pgfsys@useobject{currentmarker}{}%
\end{pgfscope}%
\begin{pgfscope}%
\pgfsys@transformshift{6.194439in}{3.294204in}%
\pgfsys@useobject{currentmarker}{}%
\end{pgfscope}%
\begin{pgfscope}%
\pgfsys@transformshift{6.317662in}{3.292871in}%
\pgfsys@useobject{currentmarker}{}%
\end{pgfscope}%
\begin{pgfscope}%
\pgfsys@transformshift{6.440885in}{3.291550in}%
\pgfsys@useobject{currentmarker}{}%
\end{pgfscope}%
\begin{pgfscope}%
\pgfsys@transformshift{6.564108in}{3.290213in}%
\pgfsys@useobject{currentmarker}{}%
\end{pgfscope}%
\begin{pgfscope}%
\pgfsys@transformshift{6.687330in}{3.288874in}%
\pgfsys@useobject{currentmarker}{}%
\end{pgfscope}%
\begin{pgfscope}%
\pgfsys@transformshift{6.810553in}{3.287570in}%
\pgfsys@useobject{currentmarker}{}%
\end{pgfscope}%
\begin{pgfscope}%
\pgfsys@transformshift{6.933776in}{3.286257in}%
\pgfsys@useobject{currentmarker}{}%
\end{pgfscope}%
\begin{pgfscope}%
\pgfsys@transformshift{7.056999in}{3.284967in}%
\pgfsys@useobject{currentmarker}{}%
\end{pgfscope}%
\begin{pgfscope}%
\pgfsys@transformshift{7.180222in}{3.283749in}%
\pgfsys@useobject{currentmarker}{}%
\end{pgfscope}%
\begin{pgfscope}%
\pgfsys@transformshift{7.303445in}{3.282568in}%
\pgfsys@useobject{currentmarker}{}%
\end{pgfscope}%
\end{pgfscope}%
\begin{pgfscope}%
\pgfsetrectcap%
\pgfsetmiterjoin%
\pgfsetlinewidth{0.803000pt}%
\definecolor{currentstroke}{rgb}{1.000000,1.000000,1.000000}%
\pgfsetstrokecolor{currentstroke}%
\pgfsetdash{}{0pt}%
\pgfpathmoveto{\pgfqpoint{2.125000in}{2.500000in}}%
\pgfpathlineto{\pgfqpoint{2.125000in}{3.377907in}}%
\pgfusepath{stroke}%
\end{pgfscope}%
\begin{pgfscope}%
\pgfsetrectcap%
\pgfsetmiterjoin%
\pgfsetlinewidth{0.803000pt}%
\definecolor{currentstroke}{rgb}{1.000000,1.000000,1.000000}%
\pgfsetstrokecolor{currentstroke}%
\pgfsetdash{}{0pt}%
\pgfpathmoveto{\pgfqpoint{7.614583in}{2.500000in}}%
\pgfpathlineto{\pgfqpoint{7.614583in}{3.377907in}}%
\pgfusepath{stroke}%
\end{pgfscope}%
\begin{pgfscope}%
\pgfsetrectcap%
\pgfsetmiterjoin%
\pgfsetlinewidth{0.803000pt}%
\definecolor{currentstroke}{rgb}{1.000000,1.000000,1.000000}%
\pgfsetstrokecolor{currentstroke}%
\pgfsetdash{}{0pt}%
\pgfpathmoveto{\pgfqpoint{2.125000in}{2.500000in}}%
\pgfpathlineto{\pgfqpoint{7.614583in}{2.500000in}}%
\pgfusepath{stroke}%
\end{pgfscope}%
\begin{pgfscope}%
\pgfsetrectcap%
\pgfsetmiterjoin%
\pgfsetlinewidth{0.803000pt}%
\definecolor{currentstroke}{rgb}{1.000000,1.000000,1.000000}%
\pgfsetstrokecolor{currentstroke}%
\pgfsetdash{}{0pt}%
\pgfpathmoveto{\pgfqpoint{2.125000in}{3.377907in}}%
\pgfpathlineto{\pgfqpoint{7.614583in}{3.377907in}}%
\pgfusepath{stroke}%
\end{pgfscope}%
\begin{pgfscope}%
\definecolor{textcolor}{rgb}{0.150000,0.150000,0.150000}%
\pgfsetstrokecolor{textcolor}%
\pgfsetfillcolor{textcolor}%
\pgftext[x=4.869792in,y=3.461240in,,base]{\color{textcolor}\rmfamily\fontsize{16.800000}{20.160000}\selectfont Autocorrelation}%
\end{pgfscope}%
\begin{pgfscope}%
\pgfsetbuttcap%
\pgfsetmiterjoin%
\definecolor{currentfill}{rgb}{0.917647,0.917647,0.949020}%
\pgfsetfillcolor{currentfill}%
\pgfsetlinewidth{0.000000pt}%
\definecolor{currentstroke}{rgb}{0.000000,0.000000,0.000000}%
\pgfsetstrokecolor{currentstroke}%
\pgfsetstrokeopacity{0.000000}%
\pgfsetdash{}{0pt}%
\pgfpathmoveto{\pgfqpoint{9.810417in}{2.500000in}}%
\pgfpathlineto{\pgfqpoint{15.300000in}{2.500000in}}%
\pgfpathlineto{\pgfqpoint{15.300000in}{3.377907in}}%
\pgfpathlineto{\pgfqpoint{9.810417in}{3.377907in}}%
\pgfpathclose%
\pgfusepath{fill}%
\end{pgfscope}%
\begin{pgfscope}%
\pgfpathrectangle{\pgfqpoint{9.810417in}{2.500000in}}{\pgfqpoint{5.489583in}{0.877907in}}%
\pgfusepath{clip}%
\pgfsetroundcap%
\pgfsetroundjoin%
\pgfsetlinewidth{0.803000pt}%
\definecolor{currentstroke}{rgb}{1.000000,1.000000,1.000000}%
\pgfsetstrokecolor{currentstroke}%
\pgfsetdash{}{0pt}%
\pgfpathmoveto{\pgfqpoint{10.059943in}{2.500000in}}%
\pgfpathlineto{\pgfqpoint{10.059943in}{3.377907in}}%
\pgfusepath{stroke}%
\end{pgfscope}%
\begin{pgfscope}%
\definecolor{textcolor}{rgb}{0.150000,0.150000,0.150000}%
\pgfsetstrokecolor{textcolor}%
\pgfsetfillcolor{textcolor}%
\pgftext[x=10.059943in,y=2.402778in,,top]{\color{textcolor}\rmfamily\fontsize{14.000000}{16.800000}\selectfont 0}%
\end{pgfscope}%
\begin{pgfscope}%
\pgfpathrectangle{\pgfqpoint{9.810417in}{2.500000in}}{\pgfqpoint{5.489583in}{0.877907in}}%
\pgfusepath{clip}%
\pgfsetroundcap%
\pgfsetroundjoin%
\pgfsetlinewidth{0.803000pt}%
\definecolor{currentstroke}{rgb}{1.000000,1.000000,1.000000}%
\pgfsetstrokecolor{currentstroke}%
\pgfsetdash{}{0pt}%
\pgfpathmoveto{\pgfqpoint{10.676058in}{2.500000in}}%
\pgfpathlineto{\pgfqpoint{10.676058in}{3.377907in}}%
\pgfusepath{stroke}%
\end{pgfscope}%
\begin{pgfscope}%
\definecolor{textcolor}{rgb}{0.150000,0.150000,0.150000}%
\pgfsetstrokecolor{textcolor}%
\pgfsetfillcolor{textcolor}%
\pgftext[x=10.676058in,y=2.402778in,,top]{\color{textcolor}\rmfamily\fontsize{14.000000}{16.800000}\selectfont 5}%
\end{pgfscope}%
\begin{pgfscope}%
\pgfpathrectangle{\pgfqpoint{9.810417in}{2.500000in}}{\pgfqpoint{5.489583in}{0.877907in}}%
\pgfusepath{clip}%
\pgfsetroundcap%
\pgfsetroundjoin%
\pgfsetlinewidth{0.803000pt}%
\definecolor{currentstroke}{rgb}{1.000000,1.000000,1.000000}%
\pgfsetstrokecolor{currentstroke}%
\pgfsetdash{}{0pt}%
\pgfpathmoveto{\pgfqpoint{11.292173in}{2.500000in}}%
\pgfpathlineto{\pgfqpoint{11.292173in}{3.377907in}}%
\pgfusepath{stroke}%
\end{pgfscope}%
\begin{pgfscope}%
\definecolor{textcolor}{rgb}{0.150000,0.150000,0.150000}%
\pgfsetstrokecolor{textcolor}%
\pgfsetfillcolor{textcolor}%
\pgftext[x=11.292173in,y=2.402778in,,top]{\color{textcolor}\rmfamily\fontsize{14.000000}{16.800000}\selectfont 10}%
\end{pgfscope}%
\begin{pgfscope}%
\pgfpathrectangle{\pgfqpoint{9.810417in}{2.500000in}}{\pgfqpoint{5.489583in}{0.877907in}}%
\pgfusepath{clip}%
\pgfsetroundcap%
\pgfsetroundjoin%
\pgfsetlinewidth{0.803000pt}%
\definecolor{currentstroke}{rgb}{1.000000,1.000000,1.000000}%
\pgfsetstrokecolor{currentstroke}%
\pgfsetdash{}{0pt}%
\pgfpathmoveto{\pgfqpoint{11.908288in}{2.500000in}}%
\pgfpathlineto{\pgfqpoint{11.908288in}{3.377907in}}%
\pgfusepath{stroke}%
\end{pgfscope}%
\begin{pgfscope}%
\definecolor{textcolor}{rgb}{0.150000,0.150000,0.150000}%
\pgfsetstrokecolor{textcolor}%
\pgfsetfillcolor{textcolor}%
\pgftext[x=11.908288in,y=2.402778in,,top]{\color{textcolor}\rmfamily\fontsize{14.000000}{16.800000}\selectfont 15}%
\end{pgfscope}%
\begin{pgfscope}%
\pgfpathrectangle{\pgfqpoint{9.810417in}{2.500000in}}{\pgfqpoint{5.489583in}{0.877907in}}%
\pgfusepath{clip}%
\pgfsetroundcap%
\pgfsetroundjoin%
\pgfsetlinewidth{0.803000pt}%
\definecolor{currentstroke}{rgb}{1.000000,1.000000,1.000000}%
\pgfsetstrokecolor{currentstroke}%
\pgfsetdash{}{0pt}%
\pgfpathmoveto{\pgfqpoint{12.524403in}{2.500000in}}%
\pgfpathlineto{\pgfqpoint{12.524403in}{3.377907in}}%
\pgfusepath{stroke}%
\end{pgfscope}%
\begin{pgfscope}%
\definecolor{textcolor}{rgb}{0.150000,0.150000,0.150000}%
\pgfsetstrokecolor{textcolor}%
\pgfsetfillcolor{textcolor}%
\pgftext[x=12.524403in,y=2.402778in,,top]{\color{textcolor}\rmfamily\fontsize{14.000000}{16.800000}\selectfont 20}%
\end{pgfscope}%
\begin{pgfscope}%
\pgfpathrectangle{\pgfqpoint{9.810417in}{2.500000in}}{\pgfqpoint{5.489583in}{0.877907in}}%
\pgfusepath{clip}%
\pgfsetroundcap%
\pgfsetroundjoin%
\pgfsetlinewidth{0.803000pt}%
\definecolor{currentstroke}{rgb}{1.000000,1.000000,1.000000}%
\pgfsetstrokecolor{currentstroke}%
\pgfsetdash{}{0pt}%
\pgfpathmoveto{\pgfqpoint{13.140517in}{2.500000in}}%
\pgfpathlineto{\pgfqpoint{13.140517in}{3.377907in}}%
\pgfusepath{stroke}%
\end{pgfscope}%
\begin{pgfscope}%
\definecolor{textcolor}{rgb}{0.150000,0.150000,0.150000}%
\pgfsetstrokecolor{textcolor}%
\pgfsetfillcolor{textcolor}%
\pgftext[x=13.140517in,y=2.402778in,,top]{\color{textcolor}\rmfamily\fontsize{14.000000}{16.800000}\selectfont 25}%
\end{pgfscope}%
\begin{pgfscope}%
\pgfpathrectangle{\pgfqpoint{9.810417in}{2.500000in}}{\pgfqpoint{5.489583in}{0.877907in}}%
\pgfusepath{clip}%
\pgfsetroundcap%
\pgfsetroundjoin%
\pgfsetlinewidth{0.803000pt}%
\definecolor{currentstroke}{rgb}{1.000000,1.000000,1.000000}%
\pgfsetstrokecolor{currentstroke}%
\pgfsetdash{}{0pt}%
\pgfpathmoveto{\pgfqpoint{13.756632in}{2.500000in}}%
\pgfpathlineto{\pgfqpoint{13.756632in}{3.377907in}}%
\pgfusepath{stroke}%
\end{pgfscope}%
\begin{pgfscope}%
\definecolor{textcolor}{rgb}{0.150000,0.150000,0.150000}%
\pgfsetstrokecolor{textcolor}%
\pgfsetfillcolor{textcolor}%
\pgftext[x=13.756632in,y=2.402778in,,top]{\color{textcolor}\rmfamily\fontsize{14.000000}{16.800000}\selectfont 30}%
\end{pgfscope}%
\begin{pgfscope}%
\pgfpathrectangle{\pgfqpoint{9.810417in}{2.500000in}}{\pgfqpoint{5.489583in}{0.877907in}}%
\pgfusepath{clip}%
\pgfsetroundcap%
\pgfsetroundjoin%
\pgfsetlinewidth{0.803000pt}%
\definecolor{currentstroke}{rgb}{1.000000,1.000000,1.000000}%
\pgfsetstrokecolor{currentstroke}%
\pgfsetdash{}{0pt}%
\pgfpathmoveto{\pgfqpoint{14.372747in}{2.500000in}}%
\pgfpathlineto{\pgfqpoint{14.372747in}{3.377907in}}%
\pgfusepath{stroke}%
\end{pgfscope}%
\begin{pgfscope}%
\definecolor{textcolor}{rgb}{0.150000,0.150000,0.150000}%
\pgfsetstrokecolor{textcolor}%
\pgfsetfillcolor{textcolor}%
\pgftext[x=14.372747in,y=2.402778in,,top]{\color{textcolor}\rmfamily\fontsize{14.000000}{16.800000}\selectfont 35}%
\end{pgfscope}%
\begin{pgfscope}%
\pgfpathrectangle{\pgfqpoint{9.810417in}{2.500000in}}{\pgfqpoint{5.489583in}{0.877907in}}%
\pgfusepath{clip}%
\pgfsetroundcap%
\pgfsetroundjoin%
\pgfsetlinewidth{0.803000pt}%
\definecolor{currentstroke}{rgb}{1.000000,1.000000,1.000000}%
\pgfsetstrokecolor{currentstroke}%
\pgfsetdash{}{0pt}%
\pgfpathmoveto{\pgfqpoint{14.988862in}{2.500000in}}%
\pgfpathlineto{\pgfqpoint{14.988862in}{3.377907in}}%
\pgfusepath{stroke}%
\end{pgfscope}%
\begin{pgfscope}%
\definecolor{textcolor}{rgb}{0.150000,0.150000,0.150000}%
\pgfsetstrokecolor{textcolor}%
\pgfsetfillcolor{textcolor}%
\pgftext[x=14.988862in,y=2.402778in,,top]{\color{textcolor}\rmfamily\fontsize{14.000000}{16.800000}\selectfont 40}%
\end{pgfscope}%
\begin{pgfscope}%
\pgfpathrectangle{\pgfqpoint{9.810417in}{2.500000in}}{\pgfqpoint{5.489583in}{0.877907in}}%
\pgfusepath{clip}%
\pgfsetroundcap%
\pgfsetroundjoin%
\pgfsetlinewidth{0.803000pt}%
\definecolor{currentstroke}{rgb}{1.000000,1.000000,1.000000}%
\pgfsetstrokecolor{currentstroke}%
\pgfsetdash{}{0pt}%
\pgfpathmoveto{\pgfqpoint{9.810417in}{2.578239in}}%
\pgfpathlineto{\pgfqpoint{15.300000in}{2.578239in}}%
\pgfusepath{stroke}%
\end{pgfscope}%
\begin{pgfscope}%
\definecolor{textcolor}{rgb}{0.150000,0.150000,0.150000}%
\pgfsetstrokecolor{textcolor}%
\pgfsetfillcolor{textcolor}%
\pgftext[x=9.589483in,y=2.504373in,left,base]{\color{textcolor}\rmfamily\fontsize{14.000000}{16.800000}\selectfont 0}%
\end{pgfscope}%
\begin{pgfscope}%
\pgfpathrectangle{\pgfqpoint{9.810417in}{2.500000in}}{\pgfqpoint{5.489583in}{0.877907in}}%
\pgfusepath{clip}%
\pgfsetroundcap%
\pgfsetroundjoin%
\pgfsetlinewidth{0.803000pt}%
\definecolor{currentstroke}{rgb}{1.000000,1.000000,1.000000}%
\pgfsetstrokecolor{currentstroke}%
\pgfsetdash{}{0pt}%
\pgfpathmoveto{\pgfqpoint{9.810417in}{3.338002in}}%
\pgfpathlineto{\pgfqpoint{15.300000in}{3.338002in}}%
\pgfusepath{stroke}%
\end{pgfscope}%
\begin{pgfscope}%
\definecolor{textcolor}{rgb}{0.150000,0.150000,0.150000}%
\pgfsetstrokecolor{textcolor}%
\pgfsetfillcolor{textcolor}%
\pgftext[x=9.589483in,y=3.264136in,left,base]{\color{textcolor}\rmfamily\fontsize{14.000000}{16.800000}\selectfont 1}%
\end{pgfscope}%
\begin{pgfscope}%
\pgfpathrectangle{\pgfqpoint{9.810417in}{2.500000in}}{\pgfqpoint{5.489583in}{0.877907in}}%
\pgfusepath{clip}%
\pgfsetbuttcap%
\pgfsetroundjoin%
\definecolor{currentfill}{rgb}{0.121569,0.466667,0.705882}%
\pgfsetfillcolor{currentfill}%
\pgfsetfillopacity{0.250000}%
\pgfsetlinewidth{1.003750pt}%
\definecolor{currentstroke}{rgb}{1.000000,1.000000,1.000000}%
\pgfsetstrokecolor{currentstroke}%
\pgfsetstrokeopacity{0.250000}%
\pgfsetdash{}{0pt}%
\pgfpathmoveto{\pgfqpoint{10.121555in}{2.616572in}}%
\pgfpathlineto{\pgfqpoint{10.121555in}{2.539905in}}%
\pgfpathlineto{\pgfqpoint{10.306389in}{2.539905in}}%
\pgfpathlineto{\pgfqpoint{10.429612in}{2.539905in}}%
\pgfpathlineto{\pgfqpoint{10.552835in}{2.539905in}}%
\pgfpathlineto{\pgfqpoint{10.676058in}{2.539905in}}%
\pgfpathlineto{\pgfqpoint{10.799281in}{2.539905in}}%
\pgfpathlineto{\pgfqpoint{10.922504in}{2.539905in}}%
\pgfpathlineto{\pgfqpoint{11.045727in}{2.539905in}}%
\pgfpathlineto{\pgfqpoint{11.168950in}{2.539905in}}%
\pgfpathlineto{\pgfqpoint{11.292173in}{2.539905in}}%
\pgfpathlineto{\pgfqpoint{11.415396in}{2.539905in}}%
\pgfpathlineto{\pgfqpoint{11.538619in}{2.539905in}}%
\pgfpathlineto{\pgfqpoint{11.661842in}{2.539905in}}%
\pgfpathlineto{\pgfqpoint{11.785065in}{2.539905in}}%
\pgfpathlineto{\pgfqpoint{11.908288in}{2.539905in}}%
\pgfpathlineto{\pgfqpoint{12.031511in}{2.539905in}}%
\pgfpathlineto{\pgfqpoint{12.154734in}{2.539905in}}%
\pgfpathlineto{\pgfqpoint{12.277957in}{2.539905in}}%
\pgfpathlineto{\pgfqpoint{12.401180in}{2.539905in}}%
\pgfpathlineto{\pgfqpoint{12.524403in}{2.539905in}}%
\pgfpathlineto{\pgfqpoint{12.647626in}{2.539905in}}%
\pgfpathlineto{\pgfqpoint{12.770849in}{2.539905in}}%
\pgfpathlineto{\pgfqpoint{12.894072in}{2.539905in}}%
\pgfpathlineto{\pgfqpoint{13.017294in}{2.539905in}}%
\pgfpathlineto{\pgfqpoint{13.140517in}{2.539905in}}%
\pgfpathlineto{\pgfqpoint{13.263740in}{2.539905in}}%
\pgfpathlineto{\pgfqpoint{13.386963in}{2.539905in}}%
\pgfpathlineto{\pgfqpoint{13.510186in}{2.539905in}}%
\pgfpathlineto{\pgfqpoint{13.633409in}{2.539905in}}%
\pgfpathlineto{\pgfqpoint{13.756632in}{2.539905in}}%
\pgfpathlineto{\pgfqpoint{13.879855in}{2.539905in}}%
\pgfpathlineto{\pgfqpoint{14.003078in}{2.539905in}}%
\pgfpathlineto{\pgfqpoint{14.126301in}{2.539905in}}%
\pgfpathlineto{\pgfqpoint{14.249524in}{2.539905in}}%
\pgfpathlineto{\pgfqpoint{14.372747in}{2.539905in}}%
\pgfpathlineto{\pgfqpoint{14.495970in}{2.539905in}}%
\pgfpathlineto{\pgfqpoint{14.619193in}{2.539905in}}%
\pgfpathlineto{\pgfqpoint{14.742416in}{2.539905in}}%
\pgfpathlineto{\pgfqpoint{14.865639in}{2.539905in}}%
\pgfpathlineto{\pgfqpoint{15.050473in}{2.539905in}}%
\pgfpathlineto{\pgfqpoint{15.050473in}{2.616572in}}%
\pgfpathlineto{\pgfqpoint{15.050473in}{2.616572in}}%
\pgfpathlineto{\pgfqpoint{14.865639in}{2.616572in}}%
\pgfpathlineto{\pgfqpoint{14.742416in}{2.616572in}}%
\pgfpathlineto{\pgfqpoint{14.619193in}{2.616572in}}%
\pgfpathlineto{\pgfqpoint{14.495970in}{2.616572in}}%
\pgfpathlineto{\pgfqpoint{14.372747in}{2.616572in}}%
\pgfpathlineto{\pgfqpoint{14.249524in}{2.616572in}}%
\pgfpathlineto{\pgfqpoint{14.126301in}{2.616572in}}%
\pgfpathlineto{\pgfqpoint{14.003078in}{2.616572in}}%
\pgfpathlineto{\pgfqpoint{13.879855in}{2.616572in}}%
\pgfpathlineto{\pgfqpoint{13.756632in}{2.616572in}}%
\pgfpathlineto{\pgfqpoint{13.633409in}{2.616572in}}%
\pgfpathlineto{\pgfqpoint{13.510186in}{2.616572in}}%
\pgfpathlineto{\pgfqpoint{13.386963in}{2.616572in}}%
\pgfpathlineto{\pgfqpoint{13.263740in}{2.616572in}}%
\pgfpathlineto{\pgfqpoint{13.140517in}{2.616572in}}%
\pgfpathlineto{\pgfqpoint{13.017294in}{2.616572in}}%
\pgfpathlineto{\pgfqpoint{12.894072in}{2.616572in}}%
\pgfpathlineto{\pgfqpoint{12.770849in}{2.616572in}}%
\pgfpathlineto{\pgfqpoint{12.647626in}{2.616572in}}%
\pgfpathlineto{\pgfqpoint{12.524403in}{2.616572in}}%
\pgfpathlineto{\pgfqpoint{12.401180in}{2.616572in}}%
\pgfpathlineto{\pgfqpoint{12.277957in}{2.616572in}}%
\pgfpathlineto{\pgfqpoint{12.154734in}{2.616572in}}%
\pgfpathlineto{\pgfqpoint{12.031511in}{2.616572in}}%
\pgfpathlineto{\pgfqpoint{11.908288in}{2.616572in}}%
\pgfpathlineto{\pgfqpoint{11.785065in}{2.616572in}}%
\pgfpathlineto{\pgfqpoint{11.661842in}{2.616572in}}%
\pgfpathlineto{\pgfqpoint{11.538619in}{2.616572in}}%
\pgfpathlineto{\pgfqpoint{11.415396in}{2.616572in}}%
\pgfpathlineto{\pgfqpoint{11.292173in}{2.616572in}}%
\pgfpathlineto{\pgfqpoint{11.168950in}{2.616572in}}%
\pgfpathlineto{\pgfqpoint{11.045727in}{2.616572in}}%
\pgfpathlineto{\pgfqpoint{10.922504in}{2.616572in}}%
\pgfpathlineto{\pgfqpoint{10.799281in}{2.616572in}}%
\pgfpathlineto{\pgfqpoint{10.676058in}{2.616572in}}%
\pgfpathlineto{\pgfqpoint{10.552835in}{2.616572in}}%
\pgfpathlineto{\pgfqpoint{10.429612in}{2.616572in}}%
\pgfpathlineto{\pgfqpoint{10.306389in}{2.616572in}}%
\pgfpathlineto{\pgfqpoint{10.121555in}{2.616572in}}%
\pgfpathclose%
\pgfusepath{stroke,fill}%
\end{pgfscope}%
\begin{pgfscope}%
\pgfpathrectangle{\pgfqpoint{9.810417in}{2.500000in}}{\pgfqpoint{5.489583in}{0.877907in}}%
\pgfusepath{clip}%
\pgfsetbuttcap%
\pgfsetroundjoin%
\pgfsetlinewidth{1.505625pt}%
\definecolor{currentstroke}{rgb}{0.000000,0.000000,0.000000}%
\pgfsetstrokecolor{currentstroke}%
\pgfsetdash{}{0pt}%
\pgfpathmoveto{\pgfqpoint{10.059943in}{2.578239in}}%
\pgfpathlineto{\pgfqpoint{10.059943in}{3.338002in}}%
\pgfusepath{stroke}%
\end{pgfscope}%
\begin{pgfscope}%
\pgfpathrectangle{\pgfqpoint{9.810417in}{2.500000in}}{\pgfqpoint{5.489583in}{0.877907in}}%
\pgfusepath{clip}%
\pgfsetbuttcap%
\pgfsetroundjoin%
\pgfsetlinewidth{1.505625pt}%
\definecolor{currentstroke}{rgb}{0.000000,0.000000,0.000000}%
\pgfsetstrokecolor{currentstroke}%
\pgfsetdash{}{0pt}%
\pgfpathmoveto{\pgfqpoint{10.183166in}{2.578239in}}%
\pgfpathlineto{\pgfqpoint{10.183166in}{3.336467in}}%
\pgfusepath{stroke}%
\end{pgfscope}%
\begin{pgfscope}%
\pgfpathrectangle{\pgfqpoint{9.810417in}{2.500000in}}{\pgfqpoint{5.489583in}{0.877907in}}%
\pgfusepath{clip}%
\pgfsetbuttcap%
\pgfsetroundjoin%
\pgfsetlinewidth{1.505625pt}%
\definecolor{currentstroke}{rgb}{0.000000,0.000000,0.000000}%
\pgfsetstrokecolor{currentstroke}%
\pgfsetdash{}{0pt}%
\pgfpathmoveto{\pgfqpoint{10.306389in}{2.578239in}}%
\pgfpathlineto{\pgfqpoint{10.306389in}{2.587040in}}%
\pgfusepath{stroke}%
\end{pgfscope}%
\begin{pgfscope}%
\pgfpathrectangle{\pgfqpoint{9.810417in}{2.500000in}}{\pgfqpoint{5.489583in}{0.877907in}}%
\pgfusepath{clip}%
\pgfsetbuttcap%
\pgfsetroundjoin%
\pgfsetlinewidth{1.505625pt}%
\definecolor{currentstroke}{rgb}{0.000000,0.000000,0.000000}%
\pgfsetstrokecolor{currentstroke}%
\pgfsetdash{}{0pt}%
\pgfpathmoveto{\pgfqpoint{10.429612in}{2.578239in}}%
\pgfpathlineto{\pgfqpoint{10.429612in}{2.591726in}}%
\pgfusepath{stroke}%
\end{pgfscope}%
\begin{pgfscope}%
\pgfpathrectangle{\pgfqpoint{9.810417in}{2.500000in}}{\pgfqpoint{5.489583in}{0.877907in}}%
\pgfusepath{clip}%
\pgfsetbuttcap%
\pgfsetroundjoin%
\pgfsetlinewidth{1.505625pt}%
\definecolor{currentstroke}{rgb}{0.000000,0.000000,0.000000}%
\pgfsetstrokecolor{currentstroke}%
\pgfsetdash{}{0pt}%
\pgfpathmoveto{\pgfqpoint{10.552835in}{2.578239in}}%
\pgfpathlineto{\pgfqpoint{10.552835in}{2.585178in}}%
\pgfusepath{stroke}%
\end{pgfscope}%
\begin{pgfscope}%
\pgfpathrectangle{\pgfqpoint{9.810417in}{2.500000in}}{\pgfqpoint{5.489583in}{0.877907in}}%
\pgfusepath{clip}%
\pgfsetbuttcap%
\pgfsetroundjoin%
\pgfsetlinewidth{1.505625pt}%
\definecolor{currentstroke}{rgb}{0.000000,0.000000,0.000000}%
\pgfsetstrokecolor{currentstroke}%
\pgfsetdash{}{0pt}%
\pgfpathmoveto{\pgfqpoint{10.676058in}{2.578239in}}%
\pgfpathlineto{\pgfqpoint{10.676058in}{2.579114in}}%
\pgfusepath{stroke}%
\end{pgfscope}%
\begin{pgfscope}%
\pgfpathrectangle{\pgfqpoint{9.810417in}{2.500000in}}{\pgfqpoint{5.489583in}{0.877907in}}%
\pgfusepath{clip}%
\pgfsetbuttcap%
\pgfsetroundjoin%
\pgfsetlinewidth{1.505625pt}%
\definecolor{currentstroke}{rgb}{0.000000,0.000000,0.000000}%
\pgfsetstrokecolor{currentstroke}%
\pgfsetdash{}{0pt}%
\pgfpathmoveto{\pgfqpoint{10.799281in}{2.578239in}}%
\pgfpathlineto{\pgfqpoint{10.799281in}{2.588180in}}%
\pgfusepath{stroke}%
\end{pgfscope}%
\begin{pgfscope}%
\pgfpathrectangle{\pgfqpoint{9.810417in}{2.500000in}}{\pgfqpoint{5.489583in}{0.877907in}}%
\pgfusepath{clip}%
\pgfsetbuttcap%
\pgfsetroundjoin%
\pgfsetlinewidth{1.505625pt}%
\definecolor{currentstroke}{rgb}{0.000000,0.000000,0.000000}%
\pgfsetstrokecolor{currentstroke}%
\pgfsetdash{}{0pt}%
\pgfpathmoveto{\pgfqpoint{10.922504in}{2.578239in}}%
\pgfpathlineto{\pgfqpoint{10.922504in}{2.559086in}}%
\pgfusepath{stroke}%
\end{pgfscope}%
\begin{pgfscope}%
\pgfpathrectangle{\pgfqpoint{9.810417in}{2.500000in}}{\pgfqpoint{5.489583in}{0.877907in}}%
\pgfusepath{clip}%
\pgfsetbuttcap%
\pgfsetroundjoin%
\pgfsetlinewidth{1.505625pt}%
\definecolor{currentstroke}{rgb}{0.000000,0.000000,0.000000}%
\pgfsetstrokecolor{currentstroke}%
\pgfsetdash{}{0pt}%
\pgfpathmoveto{\pgfqpoint{11.045727in}{2.578239in}}%
\pgfpathlineto{\pgfqpoint{11.045727in}{2.563569in}}%
\pgfusepath{stroke}%
\end{pgfscope}%
\begin{pgfscope}%
\pgfpathrectangle{\pgfqpoint{9.810417in}{2.500000in}}{\pgfqpoint{5.489583in}{0.877907in}}%
\pgfusepath{clip}%
\pgfsetbuttcap%
\pgfsetroundjoin%
\pgfsetlinewidth{1.505625pt}%
\definecolor{currentstroke}{rgb}{0.000000,0.000000,0.000000}%
\pgfsetstrokecolor{currentstroke}%
\pgfsetdash{}{0pt}%
\pgfpathmoveto{\pgfqpoint{11.168950in}{2.578239in}}%
\pgfpathlineto{\pgfqpoint{11.168950in}{2.562190in}}%
\pgfusepath{stroke}%
\end{pgfscope}%
\begin{pgfscope}%
\pgfpathrectangle{\pgfqpoint{9.810417in}{2.500000in}}{\pgfqpoint{5.489583in}{0.877907in}}%
\pgfusepath{clip}%
\pgfsetbuttcap%
\pgfsetroundjoin%
\pgfsetlinewidth{1.505625pt}%
\definecolor{currentstroke}{rgb}{0.000000,0.000000,0.000000}%
\pgfsetstrokecolor{currentstroke}%
\pgfsetdash{}{0pt}%
\pgfpathmoveto{\pgfqpoint{11.292173in}{2.578239in}}%
\pgfpathlineto{\pgfqpoint{11.292173in}{2.575613in}}%
\pgfusepath{stroke}%
\end{pgfscope}%
\begin{pgfscope}%
\pgfpathrectangle{\pgfqpoint{9.810417in}{2.500000in}}{\pgfqpoint{5.489583in}{0.877907in}}%
\pgfusepath{clip}%
\pgfsetbuttcap%
\pgfsetroundjoin%
\pgfsetlinewidth{1.505625pt}%
\definecolor{currentstroke}{rgb}{0.000000,0.000000,0.000000}%
\pgfsetstrokecolor{currentstroke}%
\pgfsetdash{}{0pt}%
\pgfpathmoveto{\pgfqpoint{11.415396in}{2.578239in}}%
\pgfpathlineto{\pgfqpoint{11.415396in}{2.579625in}}%
\pgfusepath{stroke}%
\end{pgfscope}%
\begin{pgfscope}%
\pgfpathrectangle{\pgfqpoint{9.810417in}{2.500000in}}{\pgfqpoint{5.489583in}{0.877907in}}%
\pgfusepath{clip}%
\pgfsetbuttcap%
\pgfsetroundjoin%
\pgfsetlinewidth{1.505625pt}%
\definecolor{currentstroke}{rgb}{0.000000,0.000000,0.000000}%
\pgfsetstrokecolor{currentstroke}%
\pgfsetdash{}{0pt}%
\pgfpathmoveto{\pgfqpoint{11.538619in}{2.578239in}}%
\pgfpathlineto{\pgfqpoint{11.538619in}{2.583756in}}%
\pgfusepath{stroke}%
\end{pgfscope}%
\begin{pgfscope}%
\pgfpathrectangle{\pgfqpoint{9.810417in}{2.500000in}}{\pgfqpoint{5.489583in}{0.877907in}}%
\pgfusepath{clip}%
\pgfsetbuttcap%
\pgfsetroundjoin%
\pgfsetlinewidth{1.505625pt}%
\definecolor{currentstroke}{rgb}{0.000000,0.000000,0.000000}%
\pgfsetstrokecolor{currentstroke}%
\pgfsetdash{}{0pt}%
\pgfpathmoveto{\pgfqpoint{11.661842in}{2.578239in}}%
\pgfpathlineto{\pgfqpoint{11.661842in}{2.580273in}}%
\pgfusepath{stroke}%
\end{pgfscope}%
\begin{pgfscope}%
\pgfpathrectangle{\pgfqpoint{9.810417in}{2.500000in}}{\pgfqpoint{5.489583in}{0.877907in}}%
\pgfusepath{clip}%
\pgfsetbuttcap%
\pgfsetroundjoin%
\pgfsetlinewidth{1.505625pt}%
\definecolor{currentstroke}{rgb}{0.000000,0.000000,0.000000}%
\pgfsetstrokecolor{currentstroke}%
\pgfsetdash{}{0pt}%
\pgfpathmoveto{\pgfqpoint{11.785065in}{2.578239in}}%
\pgfpathlineto{\pgfqpoint{11.785065in}{2.583029in}}%
\pgfusepath{stroke}%
\end{pgfscope}%
\begin{pgfscope}%
\pgfpathrectangle{\pgfqpoint{9.810417in}{2.500000in}}{\pgfqpoint{5.489583in}{0.877907in}}%
\pgfusepath{clip}%
\pgfsetbuttcap%
\pgfsetroundjoin%
\pgfsetlinewidth{1.505625pt}%
\definecolor{currentstroke}{rgb}{0.000000,0.000000,0.000000}%
\pgfsetstrokecolor{currentstroke}%
\pgfsetdash{}{0pt}%
\pgfpathmoveto{\pgfqpoint{11.908288in}{2.578239in}}%
\pgfpathlineto{\pgfqpoint{11.908288in}{2.586527in}}%
\pgfusepath{stroke}%
\end{pgfscope}%
\begin{pgfscope}%
\pgfpathrectangle{\pgfqpoint{9.810417in}{2.500000in}}{\pgfqpoint{5.489583in}{0.877907in}}%
\pgfusepath{clip}%
\pgfsetbuttcap%
\pgfsetroundjoin%
\pgfsetlinewidth{1.505625pt}%
\definecolor{currentstroke}{rgb}{0.000000,0.000000,0.000000}%
\pgfsetstrokecolor{currentstroke}%
\pgfsetdash{}{0pt}%
\pgfpathmoveto{\pgfqpoint{12.031511in}{2.578239in}}%
\pgfpathlineto{\pgfqpoint{12.031511in}{2.587634in}}%
\pgfusepath{stroke}%
\end{pgfscope}%
\begin{pgfscope}%
\pgfpathrectangle{\pgfqpoint{9.810417in}{2.500000in}}{\pgfqpoint{5.489583in}{0.877907in}}%
\pgfusepath{clip}%
\pgfsetbuttcap%
\pgfsetroundjoin%
\pgfsetlinewidth{1.505625pt}%
\definecolor{currentstroke}{rgb}{0.000000,0.000000,0.000000}%
\pgfsetstrokecolor{currentstroke}%
\pgfsetdash{}{0pt}%
\pgfpathmoveto{\pgfqpoint{12.154734in}{2.578239in}}%
\pgfpathlineto{\pgfqpoint{12.154734in}{2.573724in}}%
\pgfusepath{stroke}%
\end{pgfscope}%
\begin{pgfscope}%
\pgfpathrectangle{\pgfqpoint{9.810417in}{2.500000in}}{\pgfqpoint{5.489583in}{0.877907in}}%
\pgfusepath{clip}%
\pgfsetbuttcap%
\pgfsetroundjoin%
\pgfsetlinewidth{1.505625pt}%
\definecolor{currentstroke}{rgb}{0.000000,0.000000,0.000000}%
\pgfsetstrokecolor{currentstroke}%
\pgfsetdash{}{0pt}%
\pgfpathmoveto{\pgfqpoint{12.277957in}{2.578239in}}%
\pgfpathlineto{\pgfqpoint{12.277957in}{2.560912in}}%
\pgfusepath{stroke}%
\end{pgfscope}%
\begin{pgfscope}%
\pgfpathrectangle{\pgfqpoint{9.810417in}{2.500000in}}{\pgfqpoint{5.489583in}{0.877907in}}%
\pgfusepath{clip}%
\pgfsetbuttcap%
\pgfsetroundjoin%
\pgfsetlinewidth{1.505625pt}%
\definecolor{currentstroke}{rgb}{0.000000,0.000000,0.000000}%
\pgfsetstrokecolor{currentstroke}%
\pgfsetdash{}{0pt}%
\pgfpathmoveto{\pgfqpoint{12.401180in}{2.578239in}}%
\pgfpathlineto{\pgfqpoint{12.401180in}{2.557842in}}%
\pgfusepath{stroke}%
\end{pgfscope}%
\begin{pgfscope}%
\pgfpathrectangle{\pgfqpoint{9.810417in}{2.500000in}}{\pgfqpoint{5.489583in}{0.877907in}}%
\pgfusepath{clip}%
\pgfsetbuttcap%
\pgfsetroundjoin%
\pgfsetlinewidth{1.505625pt}%
\definecolor{currentstroke}{rgb}{0.000000,0.000000,0.000000}%
\pgfsetstrokecolor{currentstroke}%
\pgfsetdash{}{0pt}%
\pgfpathmoveto{\pgfqpoint{12.524403in}{2.578239in}}%
\pgfpathlineto{\pgfqpoint{12.524403in}{2.584788in}}%
\pgfusepath{stroke}%
\end{pgfscope}%
\begin{pgfscope}%
\pgfpathrectangle{\pgfqpoint{9.810417in}{2.500000in}}{\pgfqpoint{5.489583in}{0.877907in}}%
\pgfusepath{clip}%
\pgfsetbuttcap%
\pgfsetroundjoin%
\pgfsetlinewidth{1.505625pt}%
\definecolor{currentstroke}{rgb}{0.000000,0.000000,0.000000}%
\pgfsetstrokecolor{currentstroke}%
\pgfsetdash{}{0pt}%
\pgfpathmoveto{\pgfqpoint{12.647626in}{2.578239in}}%
\pgfpathlineto{\pgfqpoint{12.647626in}{2.592913in}}%
\pgfusepath{stroke}%
\end{pgfscope}%
\begin{pgfscope}%
\pgfpathrectangle{\pgfqpoint{9.810417in}{2.500000in}}{\pgfqpoint{5.489583in}{0.877907in}}%
\pgfusepath{clip}%
\pgfsetbuttcap%
\pgfsetroundjoin%
\pgfsetlinewidth{1.505625pt}%
\definecolor{currentstroke}{rgb}{0.000000,0.000000,0.000000}%
\pgfsetstrokecolor{currentstroke}%
\pgfsetdash{}{0pt}%
\pgfpathmoveto{\pgfqpoint{12.770849in}{2.578239in}}%
\pgfpathlineto{\pgfqpoint{12.770849in}{2.570571in}}%
\pgfusepath{stroke}%
\end{pgfscope}%
\begin{pgfscope}%
\pgfpathrectangle{\pgfqpoint{9.810417in}{2.500000in}}{\pgfqpoint{5.489583in}{0.877907in}}%
\pgfusepath{clip}%
\pgfsetbuttcap%
\pgfsetroundjoin%
\pgfsetlinewidth{1.505625pt}%
\definecolor{currentstroke}{rgb}{0.000000,0.000000,0.000000}%
\pgfsetstrokecolor{currentstroke}%
\pgfsetdash{}{0pt}%
\pgfpathmoveto{\pgfqpoint{12.894072in}{2.578239in}}%
\pgfpathlineto{\pgfqpoint{12.894072in}{2.613410in}}%
\pgfusepath{stroke}%
\end{pgfscope}%
\begin{pgfscope}%
\pgfpathrectangle{\pgfqpoint{9.810417in}{2.500000in}}{\pgfqpoint{5.489583in}{0.877907in}}%
\pgfusepath{clip}%
\pgfsetbuttcap%
\pgfsetroundjoin%
\pgfsetlinewidth{1.505625pt}%
\definecolor{currentstroke}{rgb}{0.000000,0.000000,0.000000}%
\pgfsetstrokecolor{currentstroke}%
\pgfsetdash{}{0pt}%
\pgfpathmoveto{\pgfqpoint{13.017294in}{2.578239in}}%
\pgfpathlineto{\pgfqpoint{13.017294in}{2.588035in}}%
\pgfusepath{stroke}%
\end{pgfscope}%
\begin{pgfscope}%
\pgfpathrectangle{\pgfqpoint{9.810417in}{2.500000in}}{\pgfqpoint{5.489583in}{0.877907in}}%
\pgfusepath{clip}%
\pgfsetbuttcap%
\pgfsetroundjoin%
\pgfsetlinewidth{1.505625pt}%
\definecolor{currentstroke}{rgb}{0.000000,0.000000,0.000000}%
\pgfsetstrokecolor{currentstroke}%
\pgfsetdash{}{0pt}%
\pgfpathmoveto{\pgfqpoint{13.140517in}{2.578239in}}%
\pgfpathlineto{\pgfqpoint{13.140517in}{2.580547in}}%
\pgfusepath{stroke}%
\end{pgfscope}%
\begin{pgfscope}%
\pgfpathrectangle{\pgfqpoint{9.810417in}{2.500000in}}{\pgfqpoint{5.489583in}{0.877907in}}%
\pgfusepath{clip}%
\pgfsetbuttcap%
\pgfsetroundjoin%
\pgfsetlinewidth{1.505625pt}%
\definecolor{currentstroke}{rgb}{0.000000,0.000000,0.000000}%
\pgfsetstrokecolor{currentstroke}%
\pgfsetdash{}{0pt}%
\pgfpathmoveto{\pgfqpoint{13.263740in}{2.578239in}}%
\pgfpathlineto{\pgfqpoint{13.263740in}{2.577271in}}%
\pgfusepath{stroke}%
\end{pgfscope}%
\begin{pgfscope}%
\pgfpathrectangle{\pgfqpoint{9.810417in}{2.500000in}}{\pgfqpoint{5.489583in}{0.877907in}}%
\pgfusepath{clip}%
\pgfsetbuttcap%
\pgfsetroundjoin%
\pgfsetlinewidth{1.505625pt}%
\definecolor{currentstroke}{rgb}{0.000000,0.000000,0.000000}%
\pgfsetstrokecolor{currentstroke}%
\pgfsetdash{}{0pt}%
\pgfpathmoveto{\pgfqpoint{13.386963in}{2.578239in}}%
\pgfpathlineto{\pgfqpoint{13.386963in}{2.575820in}}%
\pgfusepath{stroke}%
\end{pgfscope}%
\begin{pgfscope}%
\pgfpathrectangle{\pgfqpoint{9.810417in}{2.500000in}}{\pgfqpoint{5.489583in}{0.877907in}}%
\pgfusepath{clip}%
\pgfsetbuttcap%
\pgfsetroundjoin%
\pgfsetlinewidth{1.505625pt}%
\definecolor{currentstroke}{rgb}{0.000000,0.000000,0.000000}%
\pgfsetstrokecolor{currentstroke}%
\pgfsetdash{}{0pt}%
\pgfpathmoveto{\pgfqpoint{13.510186in}{2.578239in}}%
\pgfpathlineto{\pgfqpoint{13.510186in}{2.571232in}}%
\pgfusepath{stroke}%
\end{pgfscope}%
\begin{pgfscope}%
\pgfpathrectangle{\pgfqpoint{9.810417in}{2.500000in}}{\pgfqpoint{5.489583in}{0.877907in}}%
\pgfusepath{clip}%
\pgfsetbuttcap%
\pgfsetroundjoin%
\pgfsetlinewidth{1.505625pt}%
\definecolor{currentstroke}{rgb}{0.000000,0.000000,0.000000}%
\pgfsetstrokecolor{currentstroke}%
\pgfsetdash{}{0pt}%
\pgfpathmoveto{\pgfqpoint{13.633409in}{2.578239in}}%
\pgfpathlineto{\pgfqpoint{13.633409in}{2.587742in}}%
\pgfusepath{stroke}%
\end{pgfscope}%
\begin{pgfscope}%
\pgfpathrectangle{\pgfqpoint{9.810417in}{2.500000in}}{\pgfqpoint{5.489583in}{0.877907in}}%
\pgfusepath{clip}%
\pgfsetbuttcap%
\pgfsetroundjoin%
\pgfsetlinewidth{1.505625pt}%
\definecolor{currentstroke}{rgb}{0.000000,0.000000,0.000000}%
\pgfsetstrokecolor{currentstroke}%
\pgfsetdash{}{0pt}%
\pgfpathmoveto{\pgfqpoint{13.756632in}{2.578239in}}%
\pgfpathlineto{\pgfqpoint{13.756632in}{2.570066in}}%
\pgfusepath{stroke}%
\end{pgfscope}%
\begin{pgfscope}%
\pgfpathrectangle{\pgfqpoint{9.810417in}{2.500000in}}{\pgfqpoint{5.489583in}{0.877907in}}%
\pgfusepath{clip}%
\pgfsetbuttcap%
\pgfsetroundjoin%
\pgfsetlinewidth{1.505625pt}%
\definecolor{currentstroke}{rgb}{0.000000,0.000000,0.000000}%
\pgfsetstrokecolor{currentstroke}%
\pgfsetdash{}{0pt}%
\pgfpathmoveto{\pgfqpoint{13.879855in}{2.578239in}}%
\pgfpathlineto{\pgfqpoint{13.879855in}{2.564442in}}%
\pgfusepath{stroke}%
\end{pgfscope}%
\begin{pgfscope}%
\pgfpathrectangle{\pgfqpoint{9.810417in}{2.500000in}}{\pgfqpoint{5.489583in}{0.877907in}}%
\pgfusepath{clip}%
\pgfsetbuttcap%
\pgfsetroundjoin%
\pgfsetlinewidth{1.505625pt}%
\definecolor{currentstroke}{rgb}{0.000000,0.000000,0.000000}%
\pgfsetstrokecolor{currentstroke}%
\pgfsetdash{}{0pt}%
\pgfpathmoveto{\pgfqpoint{14.003078in}{2.578239in}}%
\pgfpathlineto{\pgfqpoint{14.003078in}{2.581334in}}%
\pgfusepath{stroke}%
\end{pgfscope}%
\begin{pgfscope}%
\pgfpathrectangle{\pgfqpoint{9.810417in}{2.500000in}}{\pgfqpoint{5.489583in}{0.877907in}}%
\pgfusepath{clip}%
\pgfsetbuttcap%
\pgfsetroundjoin%
\pgfsetlinewidth{1.505625pt}%
\definecolor{currentstroke}{rgb}{0.000000,0.000000,0.000000}%
\pgfsetstrokecolor{currentstroke}%
\pgfsetdash{}{0pt}%
\pgfpathmoveto{\pgfqpoint{14.126301in}{2.578239in}}%
\pgfpathlineto{\pgfqpoint{14.126301in}{2.580231in}}%
\pgfusepath{stroke}%
\end{pgfscope}%
\begin{pgfscope}%
\pgfpathrectangle{\pgfqpoint{9.810417in}{2.500000in}}{\pgfqpoint{5.489583in}{0.877907in}}%
\pgfusepath{clip}%
\pgfsetbuttcap%
\pgfsetroundjoin%
\pgfsetlinewidth{1.505625pt}%
\definecolor{currentstroke}{rgb}{0.000000,0.000000,0.000000}%
\pgfsetstrokecolor{currentstroke}%
\pgfsetdash{}{0pt}%
\pgfpathmoveto{\pgfqpoint{14.249524in}{2.578239in}}%
\pgfpathlineto{\pgfqpoint{14.249524in}{2.573166in}}%
\pgfusepath{stroke}%
\end{pgfscope}%
\begin{pgfscope}%
\pgfpathrectangle{\pgfqpoint{9.810417in}{2.500000in}}{\pgfqpoint{5.489583in}{0.877907in}}%
\pgfusepath{clip}%
\pgfsetbuttcap%
\pgfsetroundjoin%
\pgfsetlinewidth{1.505625pt}%
\definecolor{currentstroke}{rgb}{0.000000,0.000000,0.000000}%
\pgfsetstrokecolor{currentstroke}%
\pgfsetdash{}{0pt}%
\pgfpathmoveto{\pgfqpoint{14.372747in}{2.578239in}}%
\pgfpathlineto{\pgfqpoint{14.372747in}{2.577758in}}%
\pgfusepath{stroke}%
\end{pgfscope}%
\begin{pgfscope}%
\pgfpathrectangle{\pgfqpoint{9.810417in}{2.500000in}}{\pgfqpoint{5.489583in}{0.877907in}}%
\pgfusepath{clip}%
\pgfsetbuttcap%
\pgfsetroundjoin%
\pgfsetlinewidth{1.505625pt}%
\definecolor{currentstroke}{rgb}{0.000000,0.000000,0.000000}%
\pgfsetstrokecolor{currentstroke}%
\pgfsetdash{}{0pt}%
\pgfpathmoveto{\pgfqpoint{14.495970in}{2.578239in}}%
\pgfpathlineto{\pgfqpoint{14.495970in}{2.590639in}}%
\pgfusepath{stroke}%
\end{pgfscope}%
\begin{pgfscope}%
\pgfpathrectangle{\pgfqpoint{9.810417in}{2.500000in}}{\pgfqpoint{5.489583in}{0.877907in}}%
\pgfusepath{clip}%
\pgfsetbuttcap%
\pgfsetroundjoin%
\pgfsetlinewidth{1.505625pt}%
\definecolor{currentstroke}{rgb}{0.000000,0.000000,0.000000}%
\pgfsetstrokecolor{currentstroke}%
\pgfsetdash{}{0pt}%
\pgfpathmoveto{\pgfqpoint{14.619193in}{2.578239in}}%
\pgfpathlineto{\pgfqpoint{14.619193in}{2.575078in}}%
\pgfusepath{stroke}%
\end{pgfscope}%
\begin{pgfscope}%
\pgfpathrectangle{\pgfqpoint{9.810417in}{2.500000in}}{\pgfqpoint{5.489583in}{0.877907in}}%
\pgfusepath{clip}%
\pgfsetbuttcap%
\pgfsetroundjoin%
\pgfsetlinewidth{1.505625pt}%
\definecolor{currentstroke}{rgb}{0.000000,0.000000,0.000000}%
\pgfsetstrokecolor{currentstroke}%
\pgfsetdash{}{0pt}%
\pgfpathmoveto{\pgfqpoint{14.742416in}{2.578239in}}%
\pgfpathlineto{\pgfqpoint{14.742416in}{2.585655in}}%
\pgfusepath{stroke}%
\end{pgfscope}%
\begin{pgfscope}%
\pgfpathrectangle{\pgfqpoint{9.810417in}{2.500000in}}{\pgfqpoint{5.489583in}{0.877907in}}%
\pgfusepath{clip}%
\pgfsetbuttcap%
\pgfsetroundjoin%
\pgfsetlinewidth{1.505625pt}%
\definecolor{currentstroke}{rgb}{0.000000,0.000000,0.000000}%
\pgfsetstrokecolor{currentstroke}%
\pgfsetdash{}{0pt}%
\pgfpathmoveto{\pgfqpoint{14.865639in}{2.578239in}}%
\pgfpathlineto{\pgfqpoint{14.865639in}{2.602179in}}%
\pgfusepath{stroke}%
\end{pgfscope}%
\begin{pgfscope}%
\pgfpathrectangle{\pgfqpoint{9.810417in}{2.500000in}}{\pgfqpoint{5.489583in}{0.877907in}}%
\pgfusepath{clip}%
\pgfsetbuttcap%
\pgfsetroundjoin%
\pgfsetlinewidth{1.505625pt}%
\definecolor{currentstroke}{rgb}{0.000000,0.000000,0.000000}%
\pgfsetstrokecolor{currentstroke}%
\pgfsetdash{}{0pt}%
\pgfpathmoveto{\pgfqpoint{14.988862in}{2.578239in}}%
\pgfpathlineto{\pgfqpoint{14.988862in}{2.589722in}}%
\pgfusepath{stroke}%
\end{pgfscope}%
\begin{pgfscope}%
\pgfpathrectangle{\pgfqpoint{9.810417in}{2.500000in}}{\pgfqpoint{5.489583in}{0.877907in}}%
\pgfusepath{clip}%
\pgfsetroundcap%
\pgfsetroundjoin%
\pgfsetlinewidth{1.505625pt}%
\definecolor{currentstroke}{rgb}{0.121569,0.466667,0.705882}%
\pgfsetstrokecolor{currentstroke}%
\pgfsetdash{}{0pt}%
\pgfpathmoveto{\pgfqpoint{9.810417in}{2.578239in}}%
\pgfpathlineto{\pgfqpoint{15.300000in}{2.578239in}}%
\pgfusepath{stroke}%
\end{pgfscope}%
\begin{pgfscope}%
\pgfpathrectangle{\pgfqpoint{9.810417in}{2.500000in}}{\pgfqpoint{5.489583in}{0.877907in}}%
\pgfusepath{clip}%
\pgfsetbuttcap%
\pgfsetroundjoin%
\definecolor{currentfill}{rgb}{0.121569,0.466667,0.705882}%
\pgfsetfillcolor{currentfill}%
\pgfsetlinewidth{1.003750pt}%
\definecolor{currentstroke}{rgb}{0.121569,0.466667,0.705882}%
\pgfsetstrokecolor{currentstroke}%
\pgfsetdash{}{0pt}%
\pgfsys@defobject{currentmarker}{\pgfqpoint{-0.034722in}{-0.034722in}}{\pgfqpoint{0.034722in}{0.034722in}}{%
\pgfpathmoveto{\pgfqpoint{0.000000in}{-0.034722in}}%
\pgfpathcurveto{\pgfqpoint{0.009208in}{-0.034722in}}{\pgfqpoint{0.018041in}{-0.031064in}}{\pgfqpoint{0.024552in}{-0.024552in}}%
\pgfpathcurveto{\pgfqpoint{0.031064in}{-0.018041in}}{\pgfqpoint{0.034722in}{-0.009208in}}{\pgfqpoint{0.034722in}{0.000000in}}%
\pgfpathcurveto{\pgfqpoint{0.034722in}{0.009208in}}{\pgfqpoint{0.031064in}{0.018041in}}{\pgfqpoint{0.024552in}{0.024552in}}%
\pgfpathcurveto{\pgfqpoint{0.018041in}{0.031064in}}{\pgfqpoint{0.009208in}{0.034722in}}{\pgfqpoint{0.000000in}{0.034722in}}%
\pgfpathcurveto{\pgfqpoint{-0.009208in}{0.034722in}}{\pgfqpoint{-0.018041in}{0.031064in}}{\pgfqpoint{-0.024552in}{0.024552in}}%
\pgfpathcurveto{\pgfqpoint{-0.031064in}{0.018041in}}{\pgfqpoint{-0.034722in}{0.009208in}}{\pgfqpoint{-0.034722in}{0.000000in}}%
\pgfpathcurveto{\pgfqpoint{-0.034722in}{-0.009208in}}{\pgfqpoint{-0.031064in}{-0.018041in}}{\pgfqpoint{-0.024552in}{-0.024552in}}%
\pgfpathcurveto{\pgfqpoint{-0.018041in}{-0.031064in}}{\pgfqpoint{-0.009208in}{-0.034722in}}{\pgfqpoint{0.000000in}{-0.034722in}}%
\pgfpathclose%
\pgfusepath{stroke,fill}%
}%
\begin{pgfscope}%
\pgfsys@transformshift{10.059943in}{3.338002in}%
\pgfsys@useobject{currentmarker}{}%
\end{pgfscope}%
\begin{pgfscope}%
\pgfsys@transformshift{10.183166in}{3.336467in}%
\pgfsys@useobject{currentmarker}{}%
\end{pgfscope}%
\begin{pgfscope}%
\pgfsys@transformshift{10.306389in}{2.587040in}%
\pgfsys@useobject{currentmarker}{}%
\end{pgfscope}%
\begin{pgfscope}%
\pgfsys@transformshift{10.429612in}{2.591726in}%
\pgfsys@useobject{currentmarker}{}%
\end{pgfscope}%
\begin{pgfscope}%
\pgfsys@transformshift{10.552835in}{2.585178in}%
\pgfsys@useobject{currentmarker}{}%
\end{pgfscope}%
\begin{pgfscope}%
\pgfsys@transformshift{10.676058in}{2.579114in}%
\pgfsys@useobject{currentmarker}{}%
\end{pgfscope}%
\begin{pgfscope}%
\pgfsys@transformshift{10.799281in}{2.588180in}%
\pgfsys@useobject{currentmarker}{}%
\end{pgfscope}%
\begin{pgfscope}%
\pgfsys@transformshift{10.922504in}{2.559086in}%
\pgfsys@useobject{currentmarker}{}%
\end{pgfscope}%
\begin{pgfscope}%
\pgfsys@transformshift{11.045727in}{2.563569in}%
\pgfsys@useobject{currentmarker}{}%
\end{pgfscope}%
\begin{pgfscope}%
\pgfsys@transformshift{11.168950in}{2.562190in}%
\pgfsys@useobject{currentmarker}{}%
\end{pgfscope}%
\begin{pgfscope}%
\pgfsys@transformshift{11.292173in}{2.575613in}%
\pgfsys@useobject{currentmarker}{}%
\end{pgfscope}%
\begin{pgfscope}%
\pgfsys@transformshift{11.415396in}{2.579625in}%
\pgfsys@useobject{currentmarker}{}%
\end{pgfscope}%
\begin{pgfscope}%
\pgfsys@transformshift{11.538619in}{2.583756in}%
\pgfsys@useobject{currentmarker}{}%
\end{pgfscope}%
\begin{pgfscope}%
\pgfsys@transformshift{11.661842in}{2.580273in}%
\pgfsys@useobject{currentmarker}{}%
\end{pgfscope}%
\begin{pgfscope}%
\pgfsys@transformshift{11.785065in}{2.583029in}%
\pgfsys@useobject{currentmarker}{}%
\end{pgfscope}%
\begin{pgfscope}%
\pgfsys@transformshift{11.908288in}{2.586527in}%
\pgfsys@useobject{currentmarker}{}%
\end{pgfscope}%
\begin{pgfscope}%
\pgfsys@transformshift{12.031511in}{2.587634in}%
\pgfsys@useobject{currentmarker}{}%
\end{pgfscope}%
\begin{pgfscope}%
\pgfsys@transformshift{12.154734in}{2.573724in}%
\pgfsys@useobject{currentmarker}{}%
\end{pgfscope}%
\begin{pgfscope}%
\pgfsys@transformshift{12.277957in}{2.560912in}%
\pgfsys@useobject{currentmarker}{}%
\end{pgfscope}%
\begin{pgfscope}%
\pgfsys@transformshift{12.401180in}{2.557842in}%
\pgfsys@useobject{currentmarker}{}%
\end{pgfscope}%
\begin{pgfscope}%
\pgfsys@transformshift{12.524403in}{2.584788in}%
\pgfsys@useobject{currentmarker}{}%
\end{pgfscope}%
\begin{pgfscope}%
\pgfsys@transformshift{12.647626in}{2.592913in}%
\pgfsys@useobject{currentmarker}{}%
\end{pgfscope}%
\begin{pgfscope}%
\pgfsys@transformshift{12.770849in}{2.570571in}%
\pgfsys@useobject{currentmarker}{}%
\end{pgfscope}%
\begin{pgfscope}%
\pgfsys@transformshift{12.894072in}{2.613410in}%
\pgfsys@useobject{currentmarker}{}%
\end{pgfscope}%
\begin{pgfscope}%
\pgfsys@transformshift{13.017294in}{2.588035in}%
\pgfsys@useobject{currentmarker}{}%
\end{pgfscope}%
\begin{pgfscope}%
\pgfsys@transformshift{13.140517in}{2.580547in}%
\pgfsys@useobject{currentmarker}{}%
\end{pgfscope}%
\begin{pgfscope}%
\pgfsys@transformshift{13.263740in}{2.577271in}%
\pgfsys@useobject{currentmarker}{}%
\end{pgfscope}%
\begin{pgfscope}%
\pgfsys@transformshift{13.386963in}{2.575820in}%
\pgfsys@useobject{currentmarker}{}%
\end{pgfscope}%
\begin{pgfscope}%
\pgfsys@transformshift{13.510186in}{2.571232in}%
\pgfsys@useobject{currentmarker}{}%
\end{pgfscope}%
\begin{pgfscope}%
\pgfsys@transformshift{13.633409in}{2.587742in}%
\pgfsys@useobject{currentmarker}{}%
\end{pgfscope}%
\begin{pgfscope}%
\pgfsys@transformshift{13.756632in}{2.570066in}%
\pgfsys@useobject{currentmarker}{}%
\end{pgfscope}%
\begin{pgfscope}%
\pgfsys@transformshift{13.879855in}{2.564442in}%
\pgfsys@useobject{currentmarker}{}%
\end{pgfscope}%
\begin{pgfscope}%
\pgfsys@transformshift{14.003078in}{2.581334in}%
\pgfsys@useobject{currentmarker}{}%
\end{pgfscope}%
\begin{pgfscope}%
\pgfsys@transformshift{14.126301in}{2.580231in}%
\pgfsys@useobject{currentmarker}{}%
\end{pgfscope}%
\begin{pgfscope}%
\pgfsys@transformshift{14.249524in}{2.573166in}%
\pgfsys@useobject{currentmarker}{}%
\end{pgfscope}%
\begin{pgfscope}%
\pgfsys@transformshift{14.372747in}{2.577758in}%
\pgfsys@useobject{currentmarker}{}%
\end{pgfscope}%
\begin{pgfscope}%
\pgfsys@transformshift{14.495970in}{2.590639in}%
\pgfsys@useobject{currentmarker}{}%
\end{pgfscope}%
\begin{pgfscope}%
\pgfsys@transformshift{14.619193in}{2.575078in}%
\pgfsys@useobject{currentmarker}{}%
\end{pgfscope}%
\begin{pgfscope}%
\pgfsys@transformshift{14.742416in}{2.585655in}%
\pgfsys@useobject{currentmarker}{}%
\end{pgfscope}%
\begin{pgfscope}%
\pgfsys@transformshift{14.865639in}{2.602179in}%
\pgfsys@useobject{currentmarker}{}%
\end{pgfscope}%
\begin{pgfscope}%
\pgfsys@transformshift{14.988862in}{2.589722in}%
\pgfsys@useobject{currentmarker}{}%
\end{pgfscope}%
\end{pgfscope}%
\begin{pgfscope}%
\pgfsetrectcap%
\pgfsetmiterjoin%
\pgfsetlinewidth{0.803000pt}%
\definecolor{currentstroke}{rgb}{1.000000,1.000000,1.000000}%
\pgfsetstrokecolor{currentstroke}%
\pgfsetdash{}{0pt}%
\pgfpathmoveto{\pgfqpoint{9.810417in}{2.500000in}}%
\pgfpathlineto{\pgfqpoint{9.810417in}{3.377907in}}%
\pgfusepath{stroke}%
\end{pgfscope}%
\begin{pgfscope}%
\pgfsetrectcap%
\pgfsetmiterjoin%
\pgfsetlinewidth{0.803000pt}%
\definecolor{currentstroke}{rgb}{1.000000,1.000000,1.000000}%
\pgfsetstrokecolor{currentstroke}%
\pgfsetdash{}{0pt}%
\pgfpathmoveto{\pgfqpoint{15.300000in}{2.500000in}}%
\pgfpathlineto{\pgfqpoint{15.300000in}{3.377907in}}%
\pgfusepath{stroke}%
\end{pgfscope}%
\begin{pgfscope}%
\pgfsetrectcap%
\pgfsetmiterjoin%
\pgfsetlinewidth{0.803000pt}%
\definecolor{currentstroke}{rgb}{1.000000,1.000000,1.000000}%
\pgfsetstrokecolor{currentstroke}%
\pgfsetdash{}{0pt}%
\pgfpathmoveto{\pgfqpoint{9.810417in}{2.500000in}}%
\pgfpathlineto{\pgfqpoint{15.300000in}{2.500000in}}%
\pgfusepath{stroke}%
\end{pgfscope}%
\begin{pgfscope}%
\pgfsetrectcap%
\pgfsetmiterjoin%
\pgfsetlinewidth{0.803000pt}%
\definecolor{currentstroke}{rgb}{1.000000,1.000000,1.000000}%
\pgfsetstrokecolor{currentstroke}%
\pgfsetdash{}{0pt}%
\pgfpathmoveto{\pgfqpoint{9.810417in}{3.377907in}}%
\pgfpathlineto{\pgfqpoint{15.300000in}{3.377907in}}%
\pgfusepath{stroke}%
\end{pgfscope}%
\begin{pgfscope}%
\definecolor{textcolor}{rgb}{0.150000,0.150000,0.150000}%
\pgfsetstrokecolor{textcolor}%
\pgfsetfillcolor{textcolor}%
\pgftext[x=12.555208in,y=3.461240in,,base]{\color{textcolor}\rmfamily\fontsize{16.800000}{20.160000}\selectfont Partial Autocorrelation}%
\end{pgfscope}%
\end{pgfpicture}%
\makeatother%
\endgroup%

    \end{adjustbox}  
    \caption{Autocorrelation and partial autocorrelation for the log of the adjusted closing prices for all stocks}
    \label{fig:acf_pacf_log_adjclose}
\end{figure}{}






Figure2: Plot of Log closing prices.
%\input{figrues.pgf}

Also ACF and PACF plots
\begin{figure}[h]
    \centering
    \begin{adjustbox}{width=.9\textwidth,center}
    %% Creator: Matplotlib, PGF backend
%%
%% To include the figure in your LaTeX document, write
%%   \input{<filename>.pgf}
%%
%% Make sure the required packages are loaded in your preamble
%%   \usepackage{pgf}
%%
%% Figures using additional raster images can only be included by \input if
%% they are in the same directory as the main LaTeX file. For loading figures
%% from other directories you can use the `import` package
%%   \usepackage{import}
%% and then include the figures with
%%   \import{<path to file>}{<filename>.pgf}
%%
%% Matplotlib used the following preamble
%%   \usepackage{fontspec}
%%   \setmainfont{DejaVuSerif.ttf}[Path=/opt/tljh/user/lib/python3.6/site-packages/matplotlib/mpl-data/fonts/ttf/]
%%   \setsansfont{DejaVuSans.ttf}[Path=/opt/tljh/user/lib/python3.6/site-packages/matplotlib/mpl-data/fonts/ttf/]
%%   \setmonofont{DejaVuSansMono.ttf}[Path=/opt/tljh/user/lib/python3.6/site-packages/matplotlib/mpl-data/fonts/ttf/]
%%
\begingroup%
\makeatletter%
\begin{pgfpicture}%
\pgfpathrectangle{\pgfpointorigin}{\pgfqpoint{15.000000in}{6.000000in}}%
\pgfusepath{use as bounding box, clip}%
\begin{pgfscope}%
\pgfsetbuttcap%
\pgfsetmiterjoin%
\definecolor{currentfill}{rgb}{1.000000,1.000000,1.000000}%
\pgfsetfillcolor{currentfill}%
\pgfsetlinewidth{0.000000pt}%
\definecolor{currentstroke}{rgb}{1.000000,1.000000,1.000000}%
\pgfsetstrokecolor{currentstroke}%
\pgfsetdash{}{0pt}%
\pgfpathmoveto{\pgfqpoint{0.000000in}{0.000000in}}%
\pgfpathlineto{\pgfqpoint{15.000000in}{0.000000in}}%
\pgfpathlineto{\pgfqpoint{15.000000in}{6.000000in}}%
\pgfpathlineto{\pgfqpoint{0.000000in}{6.000000in}}%
\pgfpathclose%
\pgfusepath{fill}%
\end{pgfscope}%
\begin{pgfscope}%
\pgfsetbuttcap%
\pgfsetmiterjoin%
\definecolor{currentfill}{rgb}{0.917647,0.917647,0.949020}%
\pgfsetfillcolor{currentfill}%
\pgfsetlinewidth{0.000000pt}%
\definecolor{currentstroke}{rgb}{0.000000,0.000000,0.000000}%
\pgfsetstrokecolor{currentstroke}%
\pgfsetstrokeopacity{0.000000}%
\pgfsetdash{}{0pt}%
\pgfpathmoveto{\pgfqpoint{1.875000in}{4.835882in}}%
\pgfpathlineto{\pgfqpoint{6.718750in}{4.835882in}}%
\pgfpathlineto{\pgfqpoint{6.718750in}{5.280000in}}%
\pgfpathlineto{\pgfqpoint{1.875000in}{5.280000in}}%
\pgfpathclose%
\pgfusepath{fill}%
\end{pgfscope}%
\begin{pgfscope}%
\pgfpathrectangle{\pgfqpoint{1.875000in}{4.835882in}}{\pgfqpoint{4.843750in}{0.444118in}}%
\pgfusepath{clip}%
\pgfsetroundcap%
\pgfsetroundjoin%
\pgfsetlinewidth{0.803000pt}%
\definecolor{currentstroke}{rgb}{1.000000,1.000000,1.000000}%
\pgfsetstrokecolor{currentstroke}%
\pgfsetdash{}{0pt}%
\pgfpathmoveto{\pgfqpoint{2.091144in}{4.835882in}}%
\pgfpathlineto{\pgfqpoint{2.091144in}{5.280000in}}%
\pgfusepath{stroke}%
\end{pgfscope}%
\begin{pgfscope}%
\definecolor{textcolor}{rgb}{0.150000,0.150000,0.150000}%
\pgfsetstrokecolor{textcolor}%
\pgfsetfillcolor{textcolor}%
\pgftext[x=2.091144in,y=4.738660in,,top]{\color{textcolor}\rmfamily\fontsize{14.000000}{16.800000}\selectfont 2012}%
\end{pgfscope}%
\begin{pgfscope}%
\pgfpathrectangle{\pgfqpoint{1.875000in}{4.835882in}}{\pgfqpoint{4.843750in}{0.444118in}}%
\pgfusepath{clip}%
\pgfsetroundcap%
\pgfsetroundjoin%
\pgfsetlinewidth{0.803000pt}%
\definecolor{currentstroke}{rgb}{1.000000,1.000000,1.000000}%
\pgfsetstrokecolor{currentstroke}%
\pgfsetdash{}{0pt}%
\pgfpathmoveto{\pgfqpoint{2.828065in}{4.835882in}}%
\pgfpathlineto{\pgfqpoint{2.828065in}{5.280000in}}%
\pgfusepath{stroke}%
\end{pgfscope}%
\begin{pgfscope}%
\definecolor{textcolor}{rgb}{0.150000,0.150000,0.150000}%
\pgfsetstrokecolor{textcolor}%
\pgfsetfillcolor{textcolor}%
\pgftext[x=2.828065in,y=4.738660in,,top]{\color{textcolor}\rmfamily\fontsize{14.000000}{16.800000}\selectfont 2013}%
\end{pgfscope}%
\begin{pgfscope}%
\pgfpathrectangle{\pgfqpoint{1.875000in}{4.835882in}}{\pgfqpoint{4.843750in}{0.444118in}}%
\pgfusepath{clip}%
\pgfsetroundcap%
\pgfsetroundjoin%
\pgfsetlinewidth{0.803000pt}%
\definecolor{currentstroke}{rgb}{1.000000,1.000000,1.000000}%
\pgfsetstrokecolor{currentstroke}%
\pgfsetdash{}{0pt}%
\pgfpathmoveto{\pgfqpoint{3.562973in}{4.835882in}}%
\pgfpathlineto{\pgfqpoint{3.562973in}{5.280000in}}%
\pgfusepath{stroke}%
\end{pgfscope}%
\begin{pgfscope}%
\definecolor{textcolor}{rgb}{0.150000,0.150000,0.150000}%
\pgfsetstrokecolor{textcolor}%
\pgfsetfillcolor{textcolor}%
\pgftext[x=3.562973in,y=4.738660in,,top]{\color{textcolor}\rmfamily\fontsize{14.000000}{16.800000}\selectfont 2014}%
\end{pgfscope}%
\begin{pgfscope}%
\pgfpathrectangle{\pgfqpoint{1.875000in}{4.835882in}}{\pgfqpoint{4.843750in}{0.444118in}}%
\pgfusepath{clip}%
\pgfsetroundcap%
\pgfsetroundjoin%
\pgfsetlinewidth{0.803000pt}%
\definecolor{currentstroke}{rgb}{1.000000,1.000000,1.000000}%
\pgfsetstrokecolor{currentstroke}%
\pgfsetdash{}{0pt}%
\pgfpathmoveto{\pgfqpoint{4.297882in}{4.835882in}}%
\pgfpathlineto{\pgfqpoint{4.297882in}{5.280000in}}%
\pgfusepath{stroke}%
\end{pgfscope}%
\begin{pgfscope}%
\definecolor{textcolor}{rgb}{0.150000,0.150000,0.150000}%
\pgfsetstrokecolor{textcolor}%
\pgfsetfillcolor{textcolor}%
\pgftext[x=4.297882in,y=4.738660in,,top]{\color{textcolor}\rmfamily\fontsize{14.000000}{16.800000}\selectfont 2015}%
\end{pgfscope}%
\begin{pgfscope}%
\pgfpathrectangle{\pgfqpoint{1.875000in}{4.835882in}}{\pgfqpoint{4.843750in}{0.444118in}}%
\pgfusepath{clip}%
\pgfsetroundcap%
\pgfsetroundjoin%
\pgfsetlinewidth{0.803000pt}%
\definecolor{currentstroke}{rgb}{1.000000,1.000000,1.000000}%
\pgfsetstrokecolor{currentstroke}%
\pgfsetdash{}{0pt}%
\pgfpathmoveto{\pgfqpoint{5.032790in}{4.835882in}}%
\pgfpathlineto{\pgfqpoint{5.032790in}{5.280000in}}%
\pgfusepath{stroke}%
\end{pgfscope}%
\begin{pgfscope}%
\definecolor{textcolor}{rgb}{0.150000,0.150000,0.150000}%
\pgfsetstrokecolor{textcolor}%
\pgfsetfillcolor{textcolor}%
\pgftext[x=5.032790in,y=4.738660in,,top]{\color{textcolor}\rmfamily\fontsize{14.000000}{16.800000}\selectfont 2016}%
\end{pgfscope}%
\begin{pgfscope}%
\pgfpathrectangle{\pgfqpoint{1.875000in}{4.835882in}}{\pgfqpoint{4.843750in}{0.444118in}}%
\pgfusepath{clip}%
\pgfsetroundcap%
\pgfsetroundjoin%
\pgfsetlinewidth{0.803000pt}%
\definecolor{currentstroke}{rgb}{1.000000,1.000000,1.000000}%
\pgfsetstrokecolor{currentstroke}%
\pgfsetdash{}{0pt}%
\pgfpathmoveto{\pgfqpoint{5.769712in}{4.835882in}}%
\pgfpathlineto{\pgfqpoint{5.769712in}{5.280000in}}%
\pgfusepath{stroke}%
\end{pgfscope}%
\begin{pgfscope}%
\definecolor{textcolor}{rgb}{0.150000,0.150000,0.150000}%
\pgfsetstrokecolor{textcolor}%
\pgfsetfillcolor{textcolor}%
\pgftext[x=5.769712in,y=4.738660in,,top]{\color{textcolor}\rmfamily\fontsize{14.000000}{16.800000}\selectfont 2017}%
\end{pgfscope}%
\begin{pgfscope}%
\pgfpathrectangle{\pgfqpoint{1.875000in}{4.835882in}}{\pgfqpoint{4.843750in}{0.444118in}}%
\pgfusepath{clip}%
\pgfsetroundcap%
\pgfsetroundjoin%
\pgfsetlinewidth{0.803000pt}%
\definecolor{currentstroke}{rgb}{1.000000,1.000000,1.000000}%
\pgfsetstrokecolor{currentstroke}%
\pgfsetdash{}{0pt}%
\pgfpathmoveto{\pgfqpoint{6.504620in}{4.835882in}}%
\pgfpathlineto{\pgfqpoint{6.504620in}{5.280000in}}%
\pgfusepath{stroke}%
\end{pgfscope}%
\begin{pgfscope}%
\definecolor{textcolor}{rgb}{0.150000,0.150000,0.150000}%
\pgfsetstrokecolor{textcolor}%
\pgfsetfillcolor{textcolor}%
\pgftext[x=6.504620in,y=4.738660in,,top]{\color{textcolor}\rmfamily\fontsize{14.000000}{16.800000}\selectfont 2018}%
\end{pgfscope}%
\begin{pgfscope}%
\pgfpathrectangle{\pgfqpoint{1.875000in}{4.835882in}}{\pgfqpoint{4.843750in}{0.444118in}}%
\pgfusepath{clip}%
\pgfsetroundcap%
\pgfsetroundjoin%
\pgfsetlinewidth{0.803000pt}%
\definecolor{currentstroke}{rgb}{1.000000,1.000000,1.000000}%
\pgfsetstrokecolor{currentstroke}%
\pgfsetdash{}{0pt}%
\pgfpathmoveto{\pgfqpoint{1.875000in}{4.932400in}}%
\pgfpathlineto{\pgfqpoint{6.718750in}{4.932400in}}%
\pgfusepath{stroke}%
\end{pgfscope}%
\begin{pgfscope}%
\definecolor{textcolor}{rgb}{0.150000,0.150000,0.150000}%
\pgfsetstrokecolor{textcolor}%
\pgfsetfillcolor{textcolor}%
\pgftext[x=1.406643in,y=4.858533in,left,base]{\color{textcolor}\rmfamily\fontsize{14.000000}{16.800000}\selectfont 100}%
\end{pgfscope}%
\begin{pgfscope}%
\pgfpathrectangle{\pgfqpoint{1.875000in}{4.835882in}}{\pgfqpoint{4.843750in}{0.444118in}}%
\pgfusepath{clip}%
\pgfsetroundcap%
\pgfsetroundjoin%
\pgfsetlinewidth{0.803000pt}%
\definecolor{currentstroke}{rgb}{1.000000,1.000000,1.000000}%
\pgfsetstrokecolor{currentstroke}%
\pgfsetdash{}{0pt}%
\pgfpathmoveto{\pgfqpoint{1.875000in}{5.178150in}}%
\pgfpathlineto{\pgfqpoint{6.718750in}{5.178150in}}%
\pgfusepath{stroke}%
\end{pgfscope}%
\begin{pgfscope}%
\definecolor{textcolor}{rgb}{0.150000,0.150000,0.150000}%
\pgfsetstrokecolor{textcolor}%
\pgfsetfillcolor{textcolor}%
\pgftext[x=1.406643in,y=5.104284in,left,base]{\color{textcolor}\rmfamily\fontsize{14.000000}{16.800000}\selectfont 200}%
\end{pgfscope}%
\begin{pgfscope}%
\pgfpathrectangle{\pgfqpoint{1.875000in}{4.835882in}}{\pgfqpoint{4.843750in}{0.444118in}}%
\pgfusepath{clip}%
\pgfsetroundcap%
\pgfsetroundjoin%
\pgfsetlinewidth{1.505625pt}%
\definecolor{currentstroke}{rgb}{0.121569,0.466667,0.705882}%
\pgfsetstrokecolor{currentstroke}%
\pgfsetdash{}{0pt}%
\pgfpathmoveto{\pgfqpoint{2.095170in}{4.856315in}}%
\pgfpathlineto{\pgfqpoint{2.097184in}{4.857716in}}%
\pgfpathlineto{\pgfqpoint{2.101211in}{4.856070in}}%
\pgfpathlineto{\pgfqpoint{2.109265in}{4.857962in}}%
\pgfpathlineto{\pgfqpoint{2.111278in}{4.856880in}}%
\pgfpathlineto{\pgfqpoint{2.113291in}{4.857913in}}%
\pgfpathlineto{\pgfqpoint{2.115305in}{4.856536in}}%
\pgfpathlineto{\pgfqpoint{2.123359in}{4.857814in}}%
\pgfpathlineto{\pgfqpoint{2.127386in}{4.861009in}}%
\pgfpathlineto{\pgfqpoint{2.129399in}{4.860714in}}%
\pgfpathlineto{\pgfqpoint{2.135439in}{4.860616in}}%
\pgfpathlineto{\pgfqpoint{2.139466in}{4.862385in}}%
\pgfpathlineto{\pgfqpoint{2.141480in}{4.864622in}}%
\pgfpathlineto{\pgfqpoint{2.151547in}{4.862852in}}%
\pgfpathlineto{\pgfqpoint{2.153560in}{4.864155in}}%
\pgfpathlineto{\pgfqpoint{2.169668in}{4.865531in}}%
\pgfpathlineto{\pgfqpoint{2.171681in}{4.863737in}}%
\pgfpathlineto{\pgfqpoint{2.177722in}{4.865531in}}%
\pgfpathlineto{\pgfqpoint{2.179735in}{4.865457in}}%
\pgfpathlineto{\pgfqpoint{2.181749in}{4.864671in}}%
\pgfpathlineto{\pgfqpoint{2.183762in}{4.865973in}}%
\pgfpathlineto{\pgfqpoint{2.185776in}{4.865777in}}%
\pgfpathlineto{\pgfqpoint{2.195843in}{4.866194in}}%
\pgfpathlineto{\pgfqpoint{2.199870in}{4.867104in}}%
\pgfpathlineto{\pgfqpoint{2.207923in}{4.866219in}}%
\pgfpathlineto{\pgfqpoint{2.213964in}{4.865703in}}%
\pgfpathlineto{\pgfqpoint{2.220004in}{4.864769in}}%
\pgfpathlineto{\pgfqpoint{2.222018in}{4.860444in}}%
\pgfpathlineto{\pgfqpoint{2.224031in}{4.861501in}}%
\pgfpathlineto{\pgfqpoint{2.226045in}{4.864056in}}%
\pgfpathlineto{\pgfqpoint{2.228058in}{4.864228in}}%
\pgfpathlineto{\pgfqpoint{2.234098in}{4.865752in}}%
\pgfpathlineto{\pgfqpoint{2.236112in}{4.868234in}}%
\pgfpathlineto{\pgfqpoint{2.238125in}{4.868480in}}%
\pgfpathlineto{\pgfqpoint{2.240139in}{4.870790in}}%
\pgfpathlineto{\pgfqpoint{2.242152in}{4.869881in}}%
\pgfpathlineto{\pgfqpoint{2.248192in}{4.870249in}}%
\pgfpathlineto{\pgfqpoint{2.254233in}{4.867866in}}%
\pgfpathlineto{\pgfqpoint{2.256246in}{4.867620in}}%
\pgfpathlineto{\pgfqpoint{2.264300in}{4.869021in}}%
\pgfpathlineto{\pgfqpoint{2.266313in}{4.867620in}}%
\pgfpathlineto{\pgfqpoint{2.270340in}{4.869168in}}%
\pgfpathlineto{\pgfqpoint{2.276381in}{4.869217in}}%
\pgfpathlineto{\pgfqpoint{2.278394in}{4.868308in}}%
\pgfpathlineto{\pgfqpoint{2.282421in}{4.865187in}}%
\pgfpathlineto{\pgfqpoint{2.290475in}{4.863270in}}%
\pgfpathlineto{\pgfqpoint{2.292488in}{4.859682in}}%
\pgfpathlineto{\pgfqpoint{2.294502in}{4.861255in}}%
\pgfpathlineto{\pgfqpoint{2.296515in}{4.864351in}}%
\pgfpathlineto{\pgfqpoint{2.298529in}{4.861968in}}%
\pgfpathlineto{\pgfqpoint{2.304569in}{4.863417in}}%
\pgfpathlineto{\pgfqpoint{2.306582in}{4.865555in}}%
\pgfpathlineto{\pgfqpoint{2.310609in}{4.864228in}}%
\pgfpathlineto{\pgfqpoint{2.312623in}{4.865629in}}%
\pgfpathlineto{\pgfqpoint{2.318663in}{4.864917in}}%
\pgfpathlineto{\pgfqpoint{2.320677in}{4.867694in}}%
\pgfpathlineto{\pgfqpoint{2.326717in}{4.869463in}}%
\pgfpathlineto{\pgfqpoint{2.338798in}{4.869537in}}%
\pgfpathlineto{\pgfqpoint{2.340811in}{4.868062in}}%
\pgfpathlineto{\pgfqpoint{2.348865in}{4.865777in}}%
\pgfpathlineto{\pgfqpoint{2.362959in}{4.862140in}}%
\pgfpathlineto{\pgfqpoint{2.364972in}{4.862484in}}%
\pgfpathlineto{\pgfqpoint{2.368999in}{4.858674in}}%
\pgfpathlineto{\pgfqpoint{2.375040in}{4.860640in}}%
\pgfpathlineto{\pgfqpoint{2.377053in}{4.859952in}}%
\pgfpathlineto{\pgfqpoint{2.379067in}{4.861353in}}%
\pgfpathlineto{\pgfqpoint{2.381080in}{4.861722in}}%
\pgfpathlineto{\pgfqpoint{2.383093in}{4.861304in}}%
\pgfpathlineto{\pgfqpoint{2.391147in}{4.863270in}}%
\pgfpathlineto{\pgfqpoint{2.393161in}{4.860616in}}%
\pgfpathlineto{\pgfqpoint{2.395174in}{4.860542in}}%
\pgfpathlineto{\pgfqpoint{2.397188in}{4.857323in}}%
\pgfpathlineto{\pgfqpoint{2.405241in}{4.856635in}}%
\pgfpathlineto{\pgfqpoint{2.407255in}{4.860985in}}%
\pgfpathlineto{\pgfqpoint{2.409268in}{4.862779in}}%
\pgfpathlineto{\pgfqpoint{2.411282in}{4.863811in}}%
\pgfpathlineto{\pgfqpoint{2.417322in}{4.862115in}}%
\pgfpathlineto{\pgfqpoint{2.419335in}{4.865310in}}%
\pgfpathlineto{\pgfqpoint{2.421349in}{4.864081in}}%
\pgfpathlineto{\pgfqpoint{2.425376in}{4.866784in}}%
\pgfpathlineto{\pgfqpoint{2.431416in}{4.866514in}}%
\pgfpathlineto{\pgfqpoint{2.433430in}{4.867571in}}%
\pgfpathlineto{\pgfqpoint{2.435443in}{4.866981in}}%
\pgfpathlineto{\pgfqpoint{2.437456in}{4.865310in}}%
\pgfpathlineto{\pgfqpoint{2.439470in}{4.865531in}}%
\pgfpathlineto{\pgfqpoint{2.445510in}{4.863516in}}%
\pgfpathlineto{\pgfqpoint{2.447524in}{4.864179in}}%
\pgfpathlineto{\pgfqpoint{2.449537in}{4.866194in}}%
\pgfpathlineto{\pgfqpoint{2.451551in}{4.866194in}}%
\pgfpathlineto{\pgfqpoint{2.453564in}{4.871232in}}%
\pgfpathlineto{\pgfqpoint{2.459604in}{4.870569in}}%
\pgfpathlineto{\pgfqpoint{2.461618in}{4.871429in}}%
\pgfpathlineto{\pgfqpoint{2.465645in}{4.871159in}}%
\pgfpathlineto{\pgfqpoint{2.467658in}{4.869979in}}%
\pgfpathlineto{\pgfqpoint{2.473699in}{4.869905in}}%
\pgfpathlineto{\pgfqpoint{2.477725in}{4.867620in}}%
\pgfpathlineto{\pgfqpoint{2.479739in}{4.864671in}}%
\pgfpathlineto{\pgfqpoint{2.481752in}{4.867079in}}%
\pgfpathlineto{\pgfqpoint{2.487793in}{4.868136in}}%
\pgfpathlineto{\pgfqpoint{2.489806in}{4.869954in}}%
\pgfpathlineto{\pgfqpoint{2.491820in}{4.873837in}}%
\pgfpathlineto{\pgfqpoint{2.493833in}{4.873764in}}%
\pgfpathlineto{\pgfqpoint{2.495846in}{4.872043in}}%
\pgfpathlineto{\pgfqpoint{2.501887in}{4.870741in}}%
\pgfpathlineto{\pgfqpoint{2.503900in}{4.868431in}}%
\pgfpathlineto{\pgfqpoint{2.505914in}{4.869487in}}%
\pgfpathlineto{\pgfqpoint{2.507927in}{4.873272in}}%
\pgfpathlineto{\pgfqpoint{2.509941in}{4.875582in}}%
\pgfpathlineto{\pgfqpoint{2.520008in}{4.874525in}}%
\pgfpathlineto{\pgfqpoint{2.522021in}{4.872215in}}%
\pgfpathlineto{\pgfqpoint{2.524035in}{4.875558in}}%
\pgfpathlineto{\pgfqpoint{2.530075in}{4.874968in}}%
\pgfpathlineto{\pgfqpoint{2.532089in}{4.875558in}}%
\pgfpathlineto{\pgfqpoint{2.536115in}{4.875336in}}%
\pgfpathlineto{\pgfqpoint{2.538129in}{4.876762in}}%
\pgfpathlineto{\pgfqpoint{2.548196in}{4.877278in}}%
\pgfpathlineto{\pgfqpoint{2.550210in}{4.879760in}}%
\pgfpathlineto{\pgfqpoint{2.552223in}{4.880792in}}%
\pgfpathlineto{\pgfqpoint{2.558263in}{4.880079in}}%
\pgfpathlineto{\pgfqpoint{2.560277in}{4.878679in}}%
\pgfpathlineto{\pgfqpoint{2.562290in}{4.878801in}}%
\pgfpathlineto{\pgfqpoint{2.564304in}{4.877352in}}%
\pgfpathlineto{\pgfqpoint{2.566317in}{4.879096in}}%
\pgfpathlineto{\pgfqpoint{2.576384in}{4.878285in}}%
\pgfpathlineto{\pgfqpoint{2.578398in}{4.876885in}}%
\pgfpathlineto{\pgfqpoint{2.580411in}{4.878629in}}%
\pgfpathlineto{\pgfqpoint{2.588465in}{4.876713in}}%
\pgfpathlineto{\pgfqpoint{2.590478in}{4.876860in}}%
\pgfpathlineto{\pgfqpoint{2.592492in}{4.880030in}}%
\pgfpathlineto{\pgfqpoint{2.594505in}{4.879072in}}%
\pgfpathlineto{\pgfqpoint{2.600546in}{4.874624in}}%
\pgfpathlineto{\pgfqpoint{2.602559in}{4.875656in}}%
\pgfpathlineto{\pgfqpoint{2.604573in}{4.874919in}}%
\pgfpathlineto{\pgfqpoint{2.606586in}{4.877499in}}%
\pgfpathlineto{\pgfqpoint{2.608599in}{4.881480in}}%
\pgfpathlineto{\pgfqpoint{2.614640in}{4.881062in}}%
\pgfpathlineto{\pgfqpoint{2.616653in}{4.880350in}}%
\pgfpathlineto{\pgfqpoint{2.620680in}{4.880669in}}%
\pgfpathlineto{\pgfqpoint{2.622694in}{4.879883in}}%
\pgfpathlineto{\pgfqpoint{2.628734in}{4.880964in}}%
\pgfpathlineto{\pgfqpoint{2.630747in}{4.879096in}}%
\pgfpathlineto{\pgfqpoint{2.632761in}{4.878605in}}%
\pgfpathlineto{\pgfqpoint{2.634774in}{4.879072in}}%
\pgfpathlineto{\pgfqpoint{2.636788in}{4.878261in}}%
\pgfpathlineto{\pgfqpoint{2.648868in}{4.882414in}}%
\pgfpathlineto{\pgfqpoint{2.650882in}{4.883520in}}%
\pgfpathlineto{\pgfqpoint{2.656922in}{4.884380in}}%
\pgfpathlineto{\pgfqpoint{2.658936in}{4.881406in}}%
\pgfpathlineto{\pgfqpoint{2.662963in}{4.879096in}}%
\pgfpathlineto{\pgfqpoint{2.671016in}{4.879023in}}%
\pgfpathlineto{\pgfqpoint{2.673030in}{4.881996in}}%
\pgfpathlineto{\pgfqpoint{2.675043in}{4.883200in}}%
\pgfpathlineto{\pgfqpoint{2.677057in}{4.883053in}}%
\pgfpathlineto{\pgfqpoint{2.679070in}{4.879342in}}%
\pgfpathlineto{\pgfqpoint{2.685110in}{4.878482in}}%
\pgfpathlineto{\pgfqpoint{2.687124in}{4.870593in}}%
\pgfpathlineto{\pgfqpoint{2.691151in}{4.868701in}}%
\pgfpathlineto{\pgfqpoint{2.693164in}{4.869143in}}%
\pgfpathlineto{\pgfqpoint{2.703232in}{4.868259in}}%
\pgfpathlineto{\pgfqpoint{2.705245in}{4.871675in}}%
\pgfpathlineto{\pgfqpoint{2.707258in}{4.871109in}}%
\pgfpathlineto{\pgfqpoint{2.713299in}{4.872412in}}%
\pgfpathlineto{\pgfqpoint{2.715312in}{4.875017in}}%
\pgfpathlineto{\pgfqpoint{2.717326in}{4.871945in}}%
\pgfpathlineto{\pgfqpoint{2.719339in}{4.870225in}}%
\pgfpathlineto{\pgfqpoint{2.721353in}{4.870765in}}%
\pgfpathlineto{\pgfqpoint{2.729406in}{4.871331in}}%
\pgfpathlineto{\pgfqpoint{2.731420in}{4.867669in}}%
\pgfpathlineto{\pgfqpoint{2.735447in}{4.870126in}}%
\pgfpathlineto{\pgfqpoint{2.741487in}{4.872338in}}%
\pgfpathlineto{\pgfqpoint{2.745514in}{4.872264in}}%
\pgfpathlineto{\pgfqpoint{2.749541in}{4.875066in}}%
\pgfpathlineto{\pgfqpoint{2.755581in}{4.874673in}}%
\pgfpathlineto{\pgfqpoint{2.757595in}{4.875115in}}%
\pgfpathlineto{\pgfqpoint{2.759608in}{4.876418in}}%
\pgfpathlineto{\pgfqpoint{2.761621in}{4.875828in}}%
\pgfpathlineto{\pgfqpoint{2.763635in}{4.876467in}}%
\pgfpathlineto{\pgfqpoint{2.771689in}{4.874722in}}%
\pgfpathlineto{\pgfqpoint{2.773702in}{4.876049in}}%
\pgfpathlineto{\pgfqpoint{2.775716in}{4.876467in}}%
\pgfpathlineto{\pgfqpoint{2.777729in}{4.877622in}}%
\pgfpathlineto{\pgfqpoint{2.783769in}{4.878384in}}%
\pgfpathlineto{\pgfqpoint{2.785783in}{4.882144in}}%
\pgfpathlineto{\pgfqpoint{2.791823in}{4.879244in}}%
\pgfpathlineto{\pgfqpoint{2.797864in}{4.880792in}}%
\pgfpathlineto{\pgfqpoint{2.799877in}{4.882512in}}%
\pgfpathlineto{\pgfqpoint{2.801890in}{4.880767in}}%
\pgfpathlineto{\pgfqpoint{2.803904in}{4.883102in}}%
\pgfpathlineto{\pgfqpoint{2.805917in}{4.880939in}}%
\pgfpathlineto{\pgfqpoint{2.815985in}{4.880890in}}%
\pgfpathlineto{\pgfqpoint{2.817998in}{4.879981in}}%
\pgfpathlineto{\pgfqpoint{2.820011in}{4.878187in}}%
\pgfpathlineto{\pgfqpoint{2.826052in}{4.880423in}}%
\pgfpathlineto{\pgfqpoint{2.830079in}{4.884454in}}%
\pgfpathlineto{\pgfqpoint{2.832092in}{4.884208in}}%
\pgfpathlineto{\pgfqpoint{2.834106in}{4.885682in}}%
\pgfpathlineto{\pgfqpoint{2.842159in}{4.885953in}}%
\pgfpathlineto{\pgfqpoint{2.844173in}{4.887845in}}%
\pgfpathlineto{\pgfqpoint{2.846186in}{4.888853in}}%
\pgfpathlineto{\pgfqpoint{2.848200in}{4.887575in}}%
\pgfpathlineto{\pgfqpoint{2.860280in}{4.891335in}}%
\pgfpathlineto{\pgfqpoint{2.862294in}{4.892711in}}%
\pgfpathlineto{\pgfqpoint{2.874375in}{4.894652in}}%
\pgfpathlineto{\pgfqpoint{2.876388in}{4.896569in}}%
\pgfpathlineto{\pgfqpoint{2.882428in}{4.896692in}}%
\pgfpathlineto{\pgfqpoint{2.884442in}{4.899125in}}%
\pgfpathlineto{\pgfqpoint{2.886455in}{4.897012in}}%
\pgfpathlineto{\pgfqpoint{2.888469in}{4.896495in}}%
\pgfpathlineto{\pgfqpoint{2.890482in}{4.898609in}}%
\pgfpathlineto{\pgfqpoint{2.896522in}{4.896962in}}%
\pgfpathlineto{\pgfqpoint{2.898536in}{4.898461in}}%
\pgfpathlineto{\pgfqpoint{2.900549in}{4.900968in}}%
\pgfpathlineto{\pgfqpoint{2.902563in}{4.899985in}}%
\pgfpathlineto{\pgfqpoint{2.904576in}{4.900894in}}%
\pgfpathlineto{\pgfqpoint{2.910617in}{4.900821in}}%
\pgfpathlineto{\pgfqpoint{2.912630in}{4.902565in}}%
\pgfpathlineto{\pgfqpoint{2.916657in}{4.902467in}}%
\pgfpathlineto{\pgfqpoint{2.918670in}{4.903426in}}%
\pgfpathlineto{\pgfqpoint{2.926724in}{4.905416in}}%
\pgfpathlineto{\pgfqpoint{2.928738in}{4.903254in}}%
\pgfpathlineto{\pgfqpoint{2.930751in}{4.902344in}}%
\pgfpathlineto{\pgfqpoint{2.932765in}{4.904065in}}%
\pgfpathlineto{\pgfqpoint{2.938805in}{4.900305in}}%
\pgfpathlineto{\pgfqpoint{2.940818in}{4.901484in}}%
\pgfpathlineto{\pgfqpoint{2.942832in}{4.904138in}}%
\pgfpathlineto{\pgfqpoint{2.944845in}{4.905023in}}%
\pgfpathlineto{\pgfqpoint{2.952899in}{4.903524in}}%
\pgfpathlineto{\pgfqpoint{2.954912in}{4.905981in}}%
\pgfpathlineto{\pgfqpoint{2.956926in}{4.906424in}}%
\pgfpathlineto{\pgfqpoint{2.958939in}{4.906153in}}%
\pgfpathlineto{\pgfqpoint{2.960953in}{4.908611in}}%
\pgfpathlineto{\pgfqpoint{2.966993in}{4.908832in}}%
\pgfpathlineto{\pgfqpoint{2.969007in}{4.907407in}}%
\pgfpathlineto{\pgfqpoint{2.971020in}{4.907308in}}%
\pgfpathlineto{\pgfqpoint{2.973033in}{4.909274in}}%
\pgfpathlineto{\pgfqpoint{2.975047in}{4.910061in}}%
\pgfpathlineto{\pgfqpoint{2.983101in}{4.907505in}}%
\pgfpathlineto{\pgfqpoint{2.985114in}{4.908513in}}%
\pgfpathlineto{\pgfqpoint{2.987128in}{4.907014in}}%
\pgfpathlineto{\pgfqpoint{2.989141in}{4.910110in}}%
\pgfpathlineto{\pgfqpoint{2.995181in}{4.907481in}}%
\pgfpathlineto{\pgfqpoint{2.997195in}{4.909373in}}%
\pgfpathlineto{\pgfqpoint{2.999208in}{4.907751in}}%
\pgfpathlineto{\pgfqpoint{3.001222in}{4.909889in}}%
\pgfpathlineto{\pgfqpoint{3.009275in}{4.908488in}}%
\pgfpathlineto{\pgfqpoint{3.011289in}{4.910331in}}%
\pgfpathlineto{\pgfqpoint{3.013302in}{4.908562in}}%
\pgfpathlineto{\pgfqpoint{3.017329in}{4.908758in}}%
\pgfpathlineto{\pgfqpoint{3.025383in}{4.909127in}}%
\pgfpathlineto{\pgfqpoint{3.027397in}{4.912789in}}%
\pgfpathlineto{\pgfqpoint{3.029410in}{4.913944in}}%
\pgfpathlineto{\pgfqpoint{3.037464in}{4.908734in}}%
\pgfpathlineto{\pgfqpoint{3.039477in}{4.909545in}}%
\pgfpathlineto{\pgfqpoint{3.043504in}{4.907087in}}%
\pgfpathlineto{\pgfqpoint{3.045518in}{4.908611in}}%
\pgfpathlineto{\pgfqpoint{3.051558in}{4.908808in}}%
\pgfpathlineto{\pgfqpoint{3.053571in}{4.912150in}}%
\pgfpathlineto{\pgfqpoint{3.055585in}{4.913157in}}%
\pgfpathlineto{\pgfqpoint{3.057598in}{4.906891in}}%
\pgfpathlineto{\pgfqpoint{3.059612in}{4.904581in}}%
\pgfpathlineto{\pgfqpoint{3.065652in}{4.904679in}}%
\pgfpathlineto{\pgfqpoint{3.067665in}{4.906522in}}%
\pgfpathlineto{\pgfqpoint{3.069679in}{4.906178in}}%
\pgfpathlineto{\pgfqpoint{3.073706in}{4.913084in}}%
\pgfpathlineto{\pgfqpoint{3.081760in}{4.913452in}}%
\pgfpathlineto{\pgfqpoint{3.083773in}{4.913845in}}%
\pgfpathlineto{\pgfqpoint{3.085786in}{4.917458in}}%
\pgfpathlineto{\pgfqpoint{3.087800in}{4.918638in}}%
\pgfpathlineto{\pgfqpoint{3.095854in}{4.918859in}}%
\pgfpathlineto{\pgfqpoint{3.097867in}{4.920776in}}%
\pgfpathlineto{\pgfqpoint{3.099881in}{4.919866in}}%
\pgfpathlineto{\pgfqpoint{3.101894in}{4.920554in}}%
\pgfpathlineto{\pgfqpoint{3.109948in}{4.921857in}}%
\pgfpathlineto{\pgfqpoint{3.113975in}{4.919817in}}%
\pgfpathlineto{\pgfqpoint{3.115988in}{4.919522in}}%
\pgfpathlineto{\pgfqpoint{3.124042in}{4.922299in}}%
\pgfpathlineto{\pgfqpoint{3.126055in}{4.921341in}}%
\pgfpathlineto{\pgfqpoint{3.128069in}{4.921955in}}%
\pgfpathlineto{\pgfqpoint{3.130082in}{4.919522in}}%
\pgfpathlineto{\pgfqpoint{3.136123in}{4.920235in}}%
\pgfpathlineto{\pgfqpoint{3.138136in}{4.918982in}}%
\pgfpathlineto{\pgfqpoint{3.140150in}{4.915959in}}%
\pgfpathlineto{\pgfqpoint{3.142163in}{4.916131in}}%
\pgfpathlineto{\pgfqpoint{3.144176in}{4.921292in}}%
\pgfpathlineto{\pgfqpoint{3.150217in}{4.920653in}}%
\pgfpathlineto{\pgfqpoint{3.152230in}{4.919424in}}%
\pgfpathlineto{\pgfqpoint{3.154244in}{4.916893in}}%
\pgfpathlineto{\pgfqpoint{3.156257in}{4.921488in}}%
\pgfpathlineto{\pgfqpoint{3.158271in}{4.921120in}}%
\pgfpathlineto{\pgfqpoint{3.164311in}{4.922987in}}%
\pgfpathlineto{\pgfqpoint{3.166324in}{4.925224in}}%
\pgfpathlineto{\pgfqpoint{3.168338in}{4.922275in}}%
\pgfpathlineto{\pgfqpoint{3.170351in}{4.916377in}}%
\pgfpathlineto{\pgfqpoint{3.172365in}{4.918072in}}%
\pgfpathlineto{\pgfqpoint{3.178405in}{4.913673in}}%
\pgfpathlineto{\pgfqpoint{3.180419in}{4.915222in}}%
\pgfpathlineto{\pgfqpoint{3.182432in}{4.918269in}}%
\pgfpathlineto{\pgfqpoint{3.184445in}{4.919449in}}%
\pgfpathlineto{\pgfqpoint{3.186459in}{4.917581in}}%
\pgfpathlineto{\pgfqpoint{3.192499in}{4.917483in}}%
\pgfpathlineto{\pgfqpoint{3.194513in}{4.916254in}}%
\pgfpathlineto{\pgfqpoint{3.196526in}{4.917777in}}%
\pgfpathlineto{\pgfqpoint{3.200553in}{4.922201in}}%
\pgfpathlineto{\pgfqpoint{3.206593in}{4.923454in}}%
\pgfpathlineto{\pgfqpoint{3.208607in}{4.926010in}}%
\pgfpathlineto{\pgfqpoint{3.210620in}{4.926182in}}%
\pgfpathlineto{\pgfqpoint{3.212634in}{4.928517in}}%
\pgfpathlineto{\pgfqpoint{3.214647in}{4.929795in}}%
\pgfpathlineto{\pgfqpoint{3.220687in}{4.929180in}}%
\pgfpathlineto{\pgfqpoint{3.222701in}{4.928124in}}%
\pgfpathlineto{\pgfqpoint{3.224714in}{4.928738in}}%
\pgfpathlineto{\pgfqpoint{3.228741in}{4.932031in}}%
\pgfpathlineto{\pgfqpoint{3.234782in}{4.932203in}}%
\pgfpathlineto{\pgfqpoint{3.236795in}{4.933211in}}%
\pgfpathlineto{\pgfqpoint{3.238808in}{4.932326in}}%
\pgfpathlineto{\pgfqpoint{3.242835in}{4.933530in}}%
\pgfpathlineto{\pgfqpoint{3.248876in}{4.932916in}}%
\pgfpathlineto{\pgfqpoint{3.250889in}{4.933383in}}%
\pgfpathlineto{\pgfqpoint{3.252903in}{4.934636in}}%
\pgfpathlineto{\pgfqpoint{3.254916in}{4.936700in}}%
\pgfpathlineto{\pgfqpoint{3.264983in}{4.934783in}}%
\pgfpathlineto{\pgfqpoint{3.266997in}{4.935471in}}%
\pgfpathlineto{\pgfqpoint{3.269010in}{4.937265in}}%
\pgfpathlineto{\pgfqpoint{3.271024in}{4.936504in}}%
\pgfpathlineto{\pgfqpoint{3.277064in}{4.936651in}}%
\pgfpathlineto{\pgfqpoint{3.279077in}{4.937265in}}%
\pgfpathlineto{\pgfqpoint{3.281091in}{4.935029in}}%
\pgfpathlineto{\pgfqpoint{3.283104in}{4.931294in}}%
\pgfpathlineto{\pgfqpoint{3.285118in}{4.931417in}}%
\pgfpathlineto{\pgfqpoint{3.293172in}{4.930384in}}%
\pgfpathlineto{\pgfqpoint{3.295185in}{4.927485in}}%
\pgfpathlineto{\pgfqpoint{3.297198in}{4.930139in}}%
\pgfpathlineto{\pgfqpoint{3.299212in}{4.929573in}}%
\pgfpathlineto{\pgfqpoint{3.305252in}{4.929426in}}%
\pgfpathlineto{\pgfqpoint{3.307266in}{4.926035in}}%
\pgfpathlineto{\pgfqpoint{3.313306in}{4.927829in}}%
\pgfpathlineto{\pgfqpoint{3.321360in}{4.927067in}}%
\pgfpathlineto{\pgfqpoint{3.323373in}{4.929893in}}%
\pgfpathlineto{\pgfqpoint{3.327400in}{4.930925in}}%
\pgfpathlineto{\pgfqpoint{3.335454in}{4.936356in}}%
\pgfpathlineto{\pgfqpoint{3.337467in}{4.938740in}}%
\pgfpathlineto{\pgfqpoint{3.339481in}{4.937634in}}%
\pgfpathlineto{\pgfqpoint{3.341494in}{4.938494in}}%
\pgfpathlineto{\pgfqpoint{3.347535in}{4.939821in}}%
\pgfpathlineto{\pgfqpoint{3.349548in}{4.941320in}}%
\pgfpathlineto{\pgfqpoint{3.351562in}{4.944196in}}%
\pgfpathlineto{\pgfqpoint{3.353575in}{4.944810in}}%
\pgfpathlineto{\pgfqpoint{3.355588in}{4.941492in}}%
\pgfpathlineto{\pgfqpoint{3.361629in}{4.943827in}}%
\pgfpathlineto{\pgfqpoint{3.363642in}{4.943163in}}%
\pgfpathlineto{\pgfqpoint{3.365656in}{4.941886in}}%
\pgfpathlineto{\pgfqpoint{3.367669in}{4.942869in}}%
\pgfpathlineto{\pgfqpoint{3.369683in}{4.941935in}}%
\pgfpathlineto{\pgfqpoint{3.375723in}{4.940214in}}%
\pgfpathlineto{\pgfqpoint{3.377736in}{4.940657in}}%
\pgfpathlineto{\pgfqpoint{3.381763in}{4.938396in}}%
\pgfpathlineto{\pgfqpoint{3.383777in}{4.940190in}}%
\pgfpathlineto{\pgfqpoint{3.389817in}{4.938691in}}%
\pgfpathlineto{\pgfqpoint{3.391830in}{4.935447in}}%
\pgfpathlineto{\pgfqpoint{3.393844in}{4.936307in}}%
\pgfpathlineto{\pgfqpoint{3.397871in}{4.942991in}}%
\pgfpathlineto{\pgfqpoint{3.403911in}{4.944441in}}%
\pgfpathlineto{\pgfqpoint{3.405925in}{4.941075in}}%
\pgfpathlineto{\pgfqpoint{3.407938in}{4.943458in}}%
\pgfpathlineto{\pgfqpoint{3.409951in}{4.947022in}}%
\pgfpathlineto{\pgfqpoint{3.420019in}{4.949528in}}%
\pgfpathlineto{\pgfqpoint{3.422032in}{4.948250in}}%
\pgfpathlineto{\pgfqpoint{3.424046in}{4.948889in}}%
\pgfpathlineto{\pgfqpoint{3.426059in}{4.950855in}}%
\pgfpathlineto{\pgfqpoint{3.434113in}{4.952674in}}%
\pgfpathlineto{\pgfqpoint{3.436126in}{4.951642in}}%
\pgfpathlineto{\pgfqpoint{3.438140in}{4.953878in}}%
\pgfpathlineto{\pgfqpoint{3.448207in}{4.954468in}}%
\pgfpathlineto{\pgfqpoint{3.450220in}{4.956606in}}%
\pgfpathlineto{\pgfqpoint{3.452234in}{4.955082in}}%
\pgfpathlineto{\pgfqpoint{3.454247in}{4.958425in}}%
\pgfpathlineto{\pgfqpoint{3.460288in}{4.958351in}}%
\pgfpathlineto{\pgfqpoint{3.464315in}{4.959702in}}%
\pgfpathlineto{\pgfqpoint{3.466328in}{4.962258in}}%
\pgfpathlineto{\pgfqpoint{3.476395in}{4.962799in}}%
\pgfpathlineto{\pgfqpoint{3.478409in}{4.962307in}}%
\pgfpathlineto{\pgfqpoint{3.480422in}{4.964814in}}%
\pgfpathlineto{\pgfqpoint{3.482436in}{4.966239in}}%
\pgfpathlineto{\pgfqpoint{3.488476in}{4.966682in}}%
\pgfpathlineto{\pgfqpoint{3.490489in}{4.968648in}}%
\pgfpathlineto{\pgfqpoint{3.492503in}{4.971523in}}%
\pgfpathlineto{\pgfqpoint{3.496530in}{4.971548in}}%
\pgfpathlineto{\pgfqpoint{3.502570in}{4.959113in}}%
\pgfpathlineto{\pgfqpoint{3.504584in}{4.956803in}}%
\pgfpathlineto{\pgfqpoint{3.506597in}{4.956508in}}%
\pgfpathlineto{\pgfqpoint{3.508610in}{4.957294in}}%
\pgfpathlineto{\pgfqpoint{3.510624in}{4.961079in}}%
\pgfpathlineto{\pgfqpoint{3.516664in}{4.961005in}}%
\pgfpathlineto{\pgfqpoint{3.520691in}{4.957171in}}%
\pgfpathlineto{\pgfqpoint{3.524718in}{4.956434in}}%
\pgfpathlineto{\pgfqpoint{3.530758in}{4.959064in}}%
\pgfpathlineto{\pgfqpoint{3.534785in}{4.976438in}}%
\pgfpathlineto{\pgfqpoint{3.538812in}{4.978404in}}%
\pgfpathlineto{\pgfqpoint{3.546866in}{4.978969in}}%
\pgfpathlineto{\pgfqpoint{3.550893in}{4.981746in}}%
\pgfpathlineto{\pgfqpoint{3.552906in}{4.984007in}}%
\pgfpathlineto{\pgfqpoint{3.558947in}{4.984155in}}%
\pgfpathlineto{\pgfqpoint{3.560960in}{4.985924in}}%
\pgfpathlineto{\pgfqpoint{3.564987in}{4.981402in}}%
\pgfpathlineto{\pgfqpoint{3.567000in}{4.982090in}}%
\pgfpathlineto{\pgfqpoint{3.573041in}{4.980345in}}%
\pgfpathlineto{\pgfqpoint{3.575054in}{4.980370in}}%
\pgfpathlineto{\pgfqpoint{3.577068in}{4.978207in}}%
\pgfpathlineto{\pgfqpoint{3.581095in}{4.977249in}}%
\pgfpathlineto{\pgfqpoint{3.587135in}{4.974054in}}%
\pgfpathlineto{\pgfqpoint{3.589148in}{4.979879in}}%
\pgfpathlineto{\pgfqpoint{3.591162in}{4.982066in}}%
\pgfpathlineto{\pgfqpoint{3.593175in}{4.981476in}}%
\pgfpathlineto{\pgfqpoint{3.595189in}{4.979657in}}%
\pgfpathlineto{\pgfqpoint{3.603242in}{4.978969in}}%
\pgfpathlineto{\pgfqpoint{3.605256in}{4.977888in}}%
\pgfpathlineto{\pgfqpoint{3.607269in}{4.974103in}}%
\pgfpathlineto{\pgfqpoint{3.609283in}{4.964519in}}%
\pgfpathlineto{\pgfqpoint{3.615323in}{4.961840in}}%
\pgfpathlineto{\pgfqpoint{3.617337in}{4.963659in}}%
\pgfpathlineto{\pgfqpoint{3.619350in}{4.964593in}}%
\pgfpathlineto{\pgfqpoint{3.621363in}{4.959899in}}%
\pgfpathlineto{\pgfqpoint{3.623377in}{4.960194in}}%
\pgfpathlineto{\pgfqpoint{3.629417in}{4.951027in}}%
\pgfpathlineto{\pgfqpoint{3.631431in}{4.957048in}}%
\pgfpathlineto{\pgfqpoint{3.633444in}{4.958425in}}%
\pgfpathlineto{\pgfqpoint{3.637471in}{4.964765in}}%
\pgfpathlineto{\pgfqpoint{3.643511in}{4.963413in}}%
\pgfpathlineto{\pgfqpoint{3.645525in}{4.966141in}}%
\pgfpathlineto{\pgfqpoint{3.647538in}{4.966829in}}%
\pgfpathlineto{\pgfqpoint{3.649552in}{4.966190in}}%
\pgfpathlineto{\pgfqpoint{3.651565in}{4.970442in}}%
\pgfpathlineto{\pgfqpoint{3.659619in}{4.969754in}}%
\pgfpathlineto{\pgfqpoint{3.661632in}{4.967075in}}%
\pgfpathlineto{\pgfqpoint{3.663646in}{4.969238in}}%
\pgfpathlineto{\pgfqpoint{3.665659in}{4.969262in}}%
\pgfpathlineto{\pgfqpoint{3.671700in}{4.970614in}}%
\pgfpathlineto{\pgfqpoint{3.673713in}{4.972162in}}%
\pgfpathlineto{\pgfqpoint{3.675727in}{4.972015in}}%
\pgfpathlineto{\pgfqpoint{3.677740in}{4.975209in}}%
\pgfpathlineto{\pgfqpoint{3.679753in}{4.976045in}}%
\pgfpathlineto{\pgfqpoint{3.685794in}{4.970638in}}%
\pgfpathlineto{\pgfqpoint{3.687807in}{4.971646in}}%
\pgfpathlineto{\pgfqpoint{3.689821in}{4.974153in}}%
\pgfpathlineto{\pgfqpoint{3.693848in}{4.974718in}}%
\pgfpathlineto{\pgfqpoint{3.699888in}{4.973514in}}%
\pgfpathlineto{\pgfqpoint{3.701901in}{4.971277in}}%
\pgfpathlineto{\pgfqpoint{3.703915in}{4.971400in}}%
\pgfpathlineto{\pgfqpoint{3.705928in}{4.967616in}}%
\pgfpathlineto{\pgfqpoint{3.707942in}{4.965502in}}%
\pgfpathlineto{\pgfqpoint{3.713982in}{4.970761in}}%
\pgfpathlineto{\pgfqpoint{3.715995in}{4.971769in}}%
\pgfpathlineto{\pgfqpoint{3.718009in}{4.968525in}}%
\pgfpathlineto{\pgfqpoint{3.720022in}{4.971105in}}%
\pgfpathlineto{\pgfqpoint{3.722036in}{4.972580in}}%
\pgfpathlineto{\pgfqpoint{3.728076in}{4.971081in}}%
\pgfpathlineto{\pgfqpoint{3.730090in}{4.974595in}}%
\pgfpathlineto{\pgfqpoint{3.732103in}{4.972506in}}%
\pgfpathlineto{\pgfqpoint{3.734116in}{4.971990in}}%
\pgfpathlineto{\pgfqpoint{3.736130in}{4.974890in}}%
\pgfpathlineto{\pgfqpoint{3.742170in}{4.978035in}}%
\pgfpathlineto{\pgfqpoint{3.744184in}{4.979903in}}%
\pgfpathlineto{\pgfqpoint{3.746197in}{4.978773in}}%
\pgfpathlineto{\pgfqpoint{3.748211in}{4.979068in}}%
\pgfpathlineto{\pgfqpoint{3.750224in}{4.978478in}}%
\pgfpathlineto{\pgfqpoint{3.756264in}{4.975332in}}%
\pgfpathlineto{\pgfqpoint{3.758278in}{4.976266in}}%
\pgfpathlineto{\pgfqpoint{3.760291in}{4.978429in}}%
\pgfpathlineto{\pgfqpoint{3.764318in}{4.971007in}}%
\pgfpathlineto{\pgfqpoint{3.770359in}{4.972654in}}%
\pgfpathlineto{\pgfqpoint{3.772372in}{4.974669in}}%
\pgfpathlineto{\pgfqpoint{3.774385in}{4.980419in}}%
\pgfpathlineto{\pgfqpoint{3.776399in}{4.982484in}}%
\pgfpathlineto{\pgfqpoint{3.786466in}{4.984966in}}%
\pgfpathlineto{\pgfqpoint{3.788480in}{4.983049in}}%
\pgfpathlineto{\pgfqpoint{3.790493in}{4.980173in}}%
\pgfpathlineto{\pgfqpoint{3.792506in}{4.979977in}}%
\pgfpathlineto{\pgfqpoint{3.800560in}{4.982090in}}%
\pgfpathlineto{\pgfqpoint{3.804587in}{4.989094in}}%
\pgfpathlineto{\pgfqpoint{3.806601in}{4.987620in}}%
\pgfpathlineto{\pgfqpoint{3.812641in}{4.988676in}}%
\pgfpathlineto{\pgfqpoint{3.814654in}{4.985826in}}%
\pgfpathlineto{\pgfqpoint{3.816668in}{4.989807in}}%
\pgfpathlineto{\pgfqpoint{3.818681in}{4.989143in}}%
\pgfpathlineto{\pgfqpoint{3.826735in}{4.993419in}}%
\pgfpathlineto{\pgfqpoint{3.828749in}{4.992584in}}%
\pgfpathlineto{\pgfqpoint{3.832775in}{4.989463in}}%
\pgfpathlineto{\pgfqpoint{3.840829in}{4.990937in}}%
\pgfpathlineto{\pgfqpoint{3.842843in}{4.987841in}}%
\pgfpathlineto{\pgfqpoint{3.844856in}{4.990618in}}%
\pgfpathlineto{\pgfqpoint{3.846870in}{4.989881in}}%
\pgfpathlineto{\pgfqpoint{3.848883in}{4.991675in}}%
\pgfpathlineto{\pgfqpoint{3.858950in}{4.992289in}}%
\pgfpathlineto{\pgfqpoint{3.860964in}{4.994378in}}%
\pgfpathlineto{\pgfqpoint{3.862977in}{4.994722in}}%
\pgfpathlineto{\pgfqpoint{3.869017in}{4.994206in}}%
\pgfpathlineto{\pgfqpoint{3.871031in}{4.995459in}}%
\pgfpathlineto{\pgfqpoint{3.873044in}{4.994083in}}%
\pgfpathlineto{\pgfqpoint{3.875058in}{4.997229in}}%
\pgfpathlineto{\pgfqpoint{3.877071in}{4.999219in}}%
\pgfpathlineto{\pgfqpoint{3.883112in}{5.000694in}}%
\pgfpathlineto{\pgfqpoint{3.887138in}{4.998728in}}%
\pgfpathlineto{\pgfqpoint{3.889152in}{4.995828in}}%
\pgfpathlineto{\pgfqpoint{3.891165in}{4.996467in}}%
\pgfpathlineto{\pgfqpoint{3.897206in}{4.996368in}}%
\pgfpathlineto{\pgfqpoint{3.903246in}{4.998900in}}%
\pgfpathlineto{\pgfqpoint{3.905260in}{5.000350in}}%
\pgfpathlineto{\pgfqpoint{3.911300in}{4.998040in}}%
\pgfpathlineto{\pgfqpoint{3.913313in}{4.995877in}}%
\pgfpathlineto{\pgfqpoint{3.915327in}{4.997081in}}%
\pgfpathlineto{\pgfqpoint{3.919354in}{4.997179in}}%
\pgfpathlineto{\pgfqpoint{3.925394in}{4.996196in}}%
\pgfpathlineto{\pgfqpoint{3.927407in}{4.999047in}}%
\pgfpathlineto{\pgfqpoint{3.929421in}{5.000497in}}%
\pgfpathlineto{\pgfqpoint{3.931434in}{5.000890in}}%
\pgfpathlineto{\pgfqpoint{3.943515in}{4.999317in}}%
\pgfpathlineto{\pgfqpoint{3.945528in}{4.997622in}}%
\pgfpathlineto{\pgfqpoint{3.947542in}{4.998506in}}%
\pgfpathlineto{\pgfqpoint{3.955596in}{5.000128in}}%
\pgfpathlineto{\pgfqpoint{3.957609in}{5.002537in}}%
\pgfpathlineto{\pgfqpoint{3.959623in}{4.996860in}}%
\pgfpathlineto{\pgfqpoint{3.961636in}{4.999637in}}%
\pgfpathlineto{\pgfqpoint{3.967676in}{4.998506in}}%
\pgfpathlineto{\pgfqpoint{3.969690in}{5.000276in}}%
\pgfpathlineto{\pgfqpoint{3.971703in}{4.999317in}}%
\pgfpathlineto{\pgfqpoint{3.973717in}{5.000300in}}%
\pgfpathlineto{\pgfqpoint{3.975730in}{5.000276in}}%
\pgfpathlineto{\pgfqpoint{3.981771in}{5.001161in}}%
\pgfpathlineto{\pgfqpoint{3.983784in}{4.997892in}}%
\pgfpathlineto{\pgfqpoint{3.985797in}{4.997278in}}%
\pgfpathlineto{\pgfqpoint{3.987811in}{4.991134in}}%
\pgfpathlineto{\pgfqpoint{3.989824in}{4.989438in}}%
\pgfpathlineto{\pgfqpoint{3.995865in}{4.990814in}}%
\pgfpathlineto{\pgfqpoint{3.997878in}{4.988726in}}%
\pgfpathlineto{\pgfqpoint{4.001905in}{4.987325in}}%
\pgfpathlineto{\pgfqpoint{4.003918in}{4.991036in}}%
\pgfpathlineto{\pgfqpoint{4.009959in}{4.990446in}}%
\pgfpathlineto{\pgfqpoint{4.011972in}{4.991109in}}%
\pgfpathlineto{\pgfqpoint{4.013986in}{4.993100in}}%
\pgfpathlineto{\pgfqpoint{4.015999in}{4.994255in}}%
\pgfpathlineto{\pgfqpoint{4.018013in}{4.993346in}}%
\pgfpathlineto{\pgfqpoint{4.024053in}{4.998113in}}%
\pgfpathlineto{\pgfqpoint{4.026066in}{4.998384in}}%
\pgfpathlineto{\pgfqpoint{4.028080in}{5.001013in}}%
\pgfpathlineto{\pgfqpoint{4.030093in}{5.000792in}}%
\pgfpathlineto{\pgfqpoint{4.032107in}{4.999981in}}%
\pgfpathlineto{\pgfqpoint{4.038147in}{5.001234in}}%
\pgfpathlineto{\pgfqpoint{4.040160in}{5.001013in}}%
\pgfpathlineto{\pgfqpoint{4.042174in}{4.999563in}}%
\pgfpathlineto{\pgfqpoint{4.054255in}{5.000178in}}%
\pgfpathlineto{\pgfqpoint{4.058282in}{4.998850in}}%
\pgfpathlineto{\pgfqpoint{4.060295in}{5.000128in}}%
\pgfpathlineto{\pgfqpoint{4.066335in}{5.001627in}}%
\pgfpathlineto{\pgfqpoint{4.068349in}{5.000694in}}%
\pgfpathlineto{\pgfqpoint{4.070362in}{5.001087in}}%
\pgfpathlineto{\pgfqpoint{4.074389in}{4.999563in}}%
\pgfpathlineto{\pgfqpoint{4.082443in}{5.001554in}}%
\pgfpathlineto{\pgfqpoint{4.084456in}{5.002684in}}%
\pgfpathlineto{\pgfqpoint{4.086470in}{5.005879in}}%
\pgfpathlineto{\pgfqpoint{4.088483in}{5.005559in}}%
\pgfpathlineto{\pgfqpoint{4.094524in}{5.003421in}}%
\pgfpathlineto{\pgfqpoint{4.096537in}{5.000497in}}%
\pgfpathlineto{\pgfqpoint{4.098550in}{5.001627in}}%
\pgfpathlineto{\pgfqpoint{4.100564in}{4.996393in}}%
\pgfpathlineto{\pgfqpoint{4.108618in}{4.995779in}}%
\pgfpathlineto{\pgfqpoint{4.110631in}{4.994648in}}%
\pgfpathlineto{\pgfqpoint{4.112645in}{4.989217in}}%
\pgfpathlineto{\pgfqpoint{4.114658in}{4.988111in}}%
\pgfpathlineto{\pgfqpoint{4.116671in}{4.991380in}}%
\pgfpathlineto{\pgfqpoint{4.122712in}{4.991748in}}%
\pgfpathlineto{\pgfqpoint{4.124725in}{4.985973in}}%
\pgfpathlineto{\pgfqpoint{4.126739in}{4.994083in}}%
\pgfpathlineto{\pgfqpoint{4.128752in}{4.988062in}}%
\pgfpathlineto{\pgfqpoint{4.130766in}{4.977593in}}%
\pgfpathlineto{\pgfqpoint{4.136806in}{4.975578in}}%
\pgfpathlineto{\pgfqpoint{4.138819in}{4.978355in}}%
\pgfpathlineto{\pgfqpoint{4.140833in}{4.978453in}}%
\pgfpathlineto{\pgfqpoint{4.142846in}{4.980247in}}%
\pgfpathlineto{\pgfqpoint{4.144860in}{4.985359in}}%
\pgfpathlineto{\pgfqpoint{4.150900in}{4.985801in}}%
\pgfpathlineto{\pgfqpoint{4.152914in}{4.993026in}}%
\pgfpathlineto{\pgfqpoint{4.154927in}{4.988726in}}%
\pgfpathlineto{\pgfqpoint{4.156940in}{5.001996in}}%
\pgfpathlineto{\pgfqpoint{4.158954in}{5.009688in}}%
\pgfpathlineto{\pgfqpoint{4.164994in}{5.011802in}}%
\pgfpathlineto{\pgfqpoint{4.167008in}{5.015045in}}%
\pgfpathlineto{\pgfqpoint{4.169021in}{5.014947in}}%
\pgfpathlineto{\pgfqpoint{4.171035in}{5.017306in}}%
\pgfpathlineto{\pgfqpoint{4.173048in}{5.020943in}}%
\pgfpathlineto{\pgfqpoint{4.179088in}{5.019911in}}%
\pgfpathlineto{\pgfqpoint{4.181102in}{5.023720in}}%
\pgfpathlineto{\pgfqpoint{4.185129in}{5.025908in}}%
\pgfpathlineto{\pgfqpoint{4.187142in}{5.026989in}}%
\pgfpathlineto{\pgfqpoint{4.193182in}{5.029397in}}%
\pgfpathlineto{\pgfqpoint{4.195196in}{5.028193in}}%
\pgfpathlineto{\pgfqpoint{4.201236in}{5.031978in}}%
\pgfpathlineto{\pgfqpoint{4.207277in}{5.031634in}}%
\pgfpathlineto{\pgfqpoint{4.209290in}{5.034804in}}%
\pgfpathlineto{\pgfqpoint{4.211303in}{5.033796in}}%
\pgfpathlineto{\pgfqpoint{4.215330in}{5.036696in}}%
\pgfpathlineto{\pgfqpoint{4.221371in}{5.036082in}}%
\pgfpathlineto{\pgfqpoint{4.223384in}{5.032125in}}%
\pgfpathlineto{\pgfqpoint{4.225398in}{5.032666in}}%
\pgfpathlineto{\pgfqpoint{4.229425in}{5.036549in}}%
\pgfpathlineto{\pgfqpoint{4.235465in}{5.032322in}}%
\pgfpathlineto{\pgfqpoint{4.237478in}{5.037654in}}%
\pgfpathlineto{\pgfqpoint{4.239492in}{5.041292in}}%
\pgfpathlineto{\pgfqpoint{4.243519in}{5.041316in}}%
\pgfpathlineto{\pgfqpoint{4.249559in}{5.038392in}}%
\pgfpathlineto{\pgfqpoint{4.251572in}{5.038171in}}%
\pgfpathlineto{\pgfqpoint{4.253586in}{5.032518in}}%
\pgfpathlineto{\pgfqpoint{4.255599in}{5.034484in}}%
\pgfpathlineto{\pgfqpoint{4.257613in}{5.030061in}}%
\pgfpathlineto{\pgfqpoint{4.263653in}{5.029471in}}%
\pgfpathlineto{\pgfqpoint{4.265667in}{5.034288in}}%
\pgfpathlineto{\pgfqpoint{4.267680in}{5.037654in}}%
\pgfpathlineto{\pgfqpoint{4.269693in}{5.047927in}}%
\pgfpathlineto{\pgfqpoint{4.271707in}{5.048320in}}%
\pgfpathlineto{\pgfqpoint{4.277747in}{5.052252in}}%
\pgfpathlineto{\pgfqpoint{4.279761in}{5.051367in}}%
\pgfpathlineto{\pgfqpoint{4.281774in}{5.051564in}}%
\pgfpathlineto{\pgfqpoint{4.285801in}{5.050040in}}%
\pgfpathlineto{\pgfqpoint{4.291841in}{5.051023in}}%
\pgfpathlineto{\pgfqpoint{4.293855in}{5.049106in}}%
\pgfpathlineto{\pgfqpoint{4.295868in}{5.045789in}}%
\pgfpathlineto{\pgfqpoint{4.299895in}{5.045224in}}%
\pgfpathlineto{\pgfqpoint{4.305936in}{5.037138in}}%
\pgfpathlineto{\pgfqpoint{4.307949in}{5.033403in}}%
\pgfpathlineto{\pgfqpoint{4.309962in}{5.035910in}}%
\pgfpathlineto{\pgfqpoint{4.311976in}{5.044290in}}%
\pgfpathlineto{\pgfqpoint{4.313989in}{5.039891in}}%
\pgfpathlineto{\pgfqpoint{4.322043in}{5.037704in}}%
\pgfpathlineto{\pgfqpoint{4.324057in}{5.036008in}}%
\pgfpathlineto{\pgfqpoint{4.326070in}{5.035615in}}%
\pgfpathlineto{\pgfqpoint{4.328083in}{5.040726in}}%
\pgfpathlineto{\pgfqpoint{4.336137in}{5.040603in}}%
\pgfpathlineto{\pgfqpoint{4.338151in}{5.041906in}}%
\pgfpathlineto{\pgfqpoint{4.340164in}{5.049229in}}%
\pgfpathlineto{\pgfqpoint{4.342178in}{5.045150in}}%
\pgfpathlineto{\pgfqpoint{4.348218in}{5.045617in}}%
\pgfpathlineto{\pgfqpoint{4.350231in}{5.044290in}}%
\pgfpathlineto{\pgfqpoint{4.352245in}{5.044953in}}%
\pgfpathlineto{\pgfqpoint{4.354258in}{5.049745in}}%
\pgfpathlineto{\pgfqpoint{4.356272in}{5.041390in}}%
\pgfpathlineto{\pgfqpoint{4.362312in}{5.046010in}}%
\pgfpathlineto{\pgfqpoint{4.364325in}{5.049328in}}%
\pgfpathlineto{\pgfqpoint{4.366339in}{5.046870in}}%
\pgfpathlineto{\pgfqpoint{4.368352in}{5.050679in}}%
\pgfpathlineto{\pgfqpoint{4.376406in}{5.046944in}}%
\pgfpathlineto{\pgfqpoint{4.378420in}{5.049082in}}%
\pgfpathlineto{\pgfqpoint{4.380433in}{5.048394in}}%
\pgfpathlineto{\pgfqpoint{4.382447in}{5.051515in}}%
\pgfpathlineto{\pgfqpoint{4.384460in}{5.051589in}}%
\pgfpathlineto{\pgfqpoint{4.394527in}{5.054808in}}%
\pgfpathlineto{\pgfqpoint{4.396541in}{5.054120in}}%
\pgfpathlineto{\pgfqpoint{4.398554in}{5.056381in}}%
\pgfpathlineto{\pgfqpoint{4.406608in}{5.058592in}}%
\pgfpathlineto{\pgfqpoint{4.408621in}{5.058076in}}%
\pgfpathlineto{\pgfqpoint{4.410635in}{5.059674in}}%
\pgfpathlineto{\pgfqpoint{4.412648in}{5.057560in}}%
\pgfpathlineto{\pgfqpoint{4.418689in}{5.061615in}}%
\pgfpathlineto{\pgfqpoint{4.420702in}{5.056725in}}%
\pgfpathlineto{\pgfqpoint{4.422715in}{5.054267in}}%
\pgfpathlineto{\pgfqpoint{4.424729in}{5.055176in}}%
\pgfpathlineto{\pgfqpoint{4.426742in}{5.048123in}}%
\pgfpathlineto{\pgfqpoint{4.432783in}{5.052547in}}%
\pgfpathlineto{\pgfqpoint{4.434796in}{5.043503in}}%
\pgfpathlineto{\pgfqpoint{4.436810in}{5.042324in}}%
\pgfpathlineto{\pgfqpoint{4.438823in}{5.048369in}}%
\pgfpathlineto{\pgfqpoint{4.440836in}{5.044560in}}%
\pgfpathlineto{\pgfqpoint{4.446877in}{5.052178in}}%
\pgfpathlineto{\pgfqpoint{4.448890in}{5.047829in}}%
\pgfpathlineto{\pgfqpoint{4.450904in}{5.052768in}}%
\pgfpathlineto{\pgfqpoint{4.452917in}{5.050974in}}%
\pgfpathlineto{\pgfqpoint{4.454931in}{5.052793in}}%
\pgfpathlineto{\pgfqpoint{4.460971in}{5.052006in}}%
\pgfpathlineto{\pgfqpoint{4.462984in}{5.052424in}}%
\pgfpathlineto{\pgfqpoint{4.464998in}{5.044462in}}%
\pgfpathlineto{\pgfqpoint{4.467011in}{5.044216in}}%
\pgfpathlineto{\pgfqpoint{4.475065in}{5.051834in}}%
\pgfpathlineto{\pgfqpoint{4.477079in}{5.049426in}}%
\pgfpathlineto{\pgfqpoint{4.479092in}{5.044069in}}%
\pgfpathlineto{\pgfqpoint{4.481105in}{5.044683in}}%
\pgfpathlineto{\pgfqpoint{4.491173in}{5.052178in}}%
\pgfpathlineto{\pgfqpoint{4.493186in}{5.052326in}}%
\pgfpathlineto{\pgfqpoint{4.497213in}{5.054071in}}%
\pgfpathlineto{\pgfqpoint{4.503253in}{5.051367in}}%
\pgfpathlineto{\pgfqpoint{4.505267in}{5.051711in}}%
\pgfpathlineto{\pgfqpoint{4.507280in}{5.052694in}}%
\pgfpathlineto{\pgfqpoint{4.509294in}{5.051490in}}%
\pgfpathlineto{\pgfqpoint{4.511307in}{5.042299in}}%
\pgfpathlineto{\pgfqpoint{4.517347in}{5.048443in}}%
\pgfpathlineto{\pgfqpoint{4.519361in}{5.047337in}}%
\pgfpathlineto{\pgfqpoint{4.521374in}{5.048812in}}%
\pgfpathlineto{\pgfqpoint{4.523388in}{5.037777in}}%
\pgfpathlineto{\pgfqpoint{4.525401in}{5.036303in}}%
\pgfpathlineto{\pgfqpoint{4.531442in}{5.034091in}}%
\pgfpathlineto{\pgfqpoint{4.533455in}{5.034853in}}%
\pgfpathlineto{\pgfqpoint{4.535468in}{5.031855in}}%
\pgfpathlineto{\pgfqpoint{4.537482in}{5.030601in}}%
\pgfpathlineto{\pgfqpoint{4.539495in}{5.033428in}}%
\pgfpathlineto{\pgfqpoint{4.545536in}{5.036377in}}%
\pgfpathlineto{\pgfqpoint{4.547549in}{5.033944in}}%
\pgfpathlineto{\pgfqpoint{4.549563in}{5.033329in}}%
\pgfpathlineto{\pgfqpoint{4.551576in}{5.035467in}}%
\pgfpathlineto{\pgfqpoint{4.553590in}{5.039842in}}%
\pgfpathlineto{\pgfqpoint{4.559630in}{5.038318in}}%
\pgfpathlineto{\pgfqpoint{4.561643in}{5.038687in}}%
\pgfpathlineto{\pgfqpoint{4.563657in}{5.041414in}}%
\pgfpathlineto{\pgfqpoint{4.565670in}{5.045445in}}%
\pgfpathlineto{\pgfqpoint{4.567684in}{5.045789in}}%
\pgfpathlineto{\pgfqpoint{4.573724in}{5.044830in}}%
\pgfpathlineto{\pgfqpoint{4.575737in}{5.045396in}}%
\pgfpathlineto{\pgfqpoint{4.577751in}{5.044953in}}%
\pgfpathlineto{\pgfqpoint{4.579764in}{5.045322in}}%
\pgfpathlineto{\pgfqpoint{4.581778in}{5.042914in}}%
\pgfpathlineto{\pgfqpoint{4.589832in}{5.039842in}}%
\pgfpathlineto{\pgfqpoint{4.591845in}{5.043012in}}%
\pgfpathlineto{\pgfqpoint{4.593858in}{5.042668in}}%
\pgfpathlineto{\pgfqpoint{4.595872in}{5.038711in}}%
\pgfpathlineto{\pgfqpoint{4.603926in}{5.038711in}}%
\pgfpathlineto{\pgfqpoint{4.605939in}{5.041120in}}%
\pgfpathlineto{\pgfqpoint{4.609966in}{5.034288in}}%
\pgfpathlineto{\pgfqpoint{4.616006in}{5.033206in}}%
\pgfpathlineto{\pgfqpoint{4.618020in}{5.034067in}}%
\pgfpathlineto{\pgfqpoint{4.620033in}{5.038637in}}%
\pgfpathlineto{\pgfqpoint{4.622047in}{5.040456in}}%
\pgfpathlineto{\pgfqpoint{4.624060in}{5.036426in}}%
\pgfpathlineto{\pgfqpoint{4.630101in}{5.031609in}}%
\pgfpathlineto{\pgfqpoint{4.634127in}{5.033993in}}%
\pgfpathlineto{\pgfqpoint{4.636141in}{5.039915in}}%
\pgfpathlineto{\pgfqpoint{4.638154in}{5.038441in}}%
\pgfpathlineto{\pgfqpoint{4.646208in}{5.040407in}}%
\pgfpathlineto{\pgfqpoint{4.650235in}{5.031732in}}%
\pgfpathlineto{\pgfqpoint{4.652248in}{5.034312in}}%
\pgfpathlineto{\pgfqpoint{4.658289in}{5.027358in}}%
\pgfpathlineto{\pgfqpoint{4.660302in}{5.028144in}}%
\pgfpathlineto{\pgfqpoint{4.662316in}{5.031118in}}%
\pgfpathlineto{\pgfqpoint{4.664329in}{5.030528in}}%
\pgfpathlineto{\pgfqpoint{4.672383in}{5.029864in}}%
\pgfpathlineto{\pgfqpoint{4.674396in}{5.030651in}}%
\pgfpathlineto{\pgfqpoint{4.676410in}{5.024630in}}%
\pgfpathlineto{\pgfqpoint{4.680437in}{5.029569in}}%
\pgfpathlineto{\pgfqpoint{4.688490in}{5.033624in}}%
\pgfpathlineto{\pgfqpoint{4.690504in}{5.031904in}}%
\pgfpathlineto{\pgfqpoint{4.692517in}{5.034484in}}%
\pgfpathlineto{\pgfqpoint{4.694531in}{5.033428in}}%
\pgfpathlineto{\pgfqpoint{4.700571in}{5.034386in}}%
\pgfpathlineto{\pgfqpoint{4.702585in}{5.031363in}}%
\pgfpathlineto{\pgfqpoint{4.704598in}{5.030601in}}%
\pgfpathlineto{\pgfqpoint{4.706612in}{5.017528in}}%
\pgfpathlineto{\pgfqpoint{4.714665in}{5.015758in}}%
\pgfpathlineto{\pgfqpoint{4.716679in}{5.021091in}}%
\pgfpathlineto{\pgfqpoint{4.720706in}{5.022098in}}%
\pgfpathlineto{\pgfqpoint{4.722719in}{5.021582in}}%
\pgfpathlineto{\pgfqpoint{4.728759in}{5.018879in}}%
\pgfpathlineto{\pgfqpoint{4.730773in}{5.019518in}}%
\pgfpathlineto{\pgfqpoint{4.732786in}{5.020870in}}%
\pgfpathlineto{\pgfqpoint{4.734800in}{5.016987in}}%
\pgfpathlineto{\pgfqpoint{4.736813in}{5.016176in}}%
\pgfpathlineto{\pgfqpoint{4.742854in}{5.021361in}}%
\pgfpathlineto{\pgfqpoint{4.744867in}{5.015267in}}%
\pgfpathlineto{\pgfqpoint{4.746880in}{5.015340in}}%
\pgfpathlineto{\pgfqpoint{4.748894in}{5.012858in}}%
\pgfpathlineto{\pgfqpoint{4.750907in}{5.014824in}}%
\pgfpathlineto{\pgfqpoint{4.756948in}{5.016938in}}%
\pgfpathlineto{\pgfqpoint{4.758961in}{5.014873in}}%
\pgfpathlineto{\pgfqpoint{4.760975in}{5.011728in}}%
\pgfpathlineto{\pgfqpoint{4.762988in}{5.006125in}}%
\pgfpathlineto{\pgfqpoint{4.771042in}{4.995484in}}%
\pgfpathlineto{\pgfqpoint{4.773055in}{4.993395in}}%
\pgfpathlineto{\pgfqpoint{4.775069in}{5.004699in}}%
\pgfpathlineto{\pgfqpoint{4.777082in}{5.007329in}}%
\pgfpathlineto{\pgfqpoint{4.779096in}{5.008042in}}%
\pgfpathlineto{\pgfqpoint{4.785136in}{5.003421in}}%
\pgfpathlineto{\pgfqpoint{4.787149in}{4.995312in}}%
\pgfpathlineto{\pgfqpoint{4.789163in}{5.001406in}}%
\pgfpathlineto{\pgfqpoint{4.791176in}{5.002537in}}%
\pgfpathlineto{\pgfqpoint{4.793190in}{4.998285in}}%
\pgfpathlineto{\pgfqpoint{4.801244in}{5.006297in}}%
\pgfpathlineto{\pgfqpoint{4.803257in}{5.000571in}}%
\pgfpathlineto{\pgfqpoint{4.805270in}{5.000374in}}%
\pgfpathlineto{\pgfqpoint{4.807284in}{5.001455in}}%
\pgfpathlineto{\pgfqpoint{4.813324in}{5.000448in}}%
\pgfpathlineto{\pgfqpoint{4.815338in}{5.006665in}}%
\pgfpathlineto{\pgfqpoint{4.817351in}{5.008017in}}%
\pgfpathlineto{\pgfqpoint{4.819365in}{5.005215in}}%
\pgfpathlineto{\pgfqpoint{4.821378in}{4.997794in}}%
\pgfpathlineto{\pgfqpoint{4.827418in}{4.998728in}}%
\pgfpathlineto{\pgfqpoint{4.829432in}{4.994230in}}%
\pgfpathlineto{\pgfqpoint{4.833459in}{4.993247in}}%
\pgfpathlineto{\pgfqpoint{4.835472in}{4.997671in}}%
\pgfpathlineto{\pgfqpoint{4.841512in}{4.995041in}}%
\pgfpathlineto{\pgfqpoint{4.843526in}{5.002094in}}%
\pgfpathlineto{\pgfqpoint{4.845539in}{5.002586in}}%
\pgfpathlineto{\pgfqpoint{4.847553in}{5.000423in}}%
\pgfpathlineto{\pgfqpoint{4.849566in}{5.005781in}}%
\pgfpathlineto{\pgfqpoint{4.855607in}{5.012735in}}%
\pgfpathlineto{\pgfqpoint{4.857620in}{5.011556in}}%
\pgfpathlineto{\pgfqpoint{4.861647in}{5.019764in}}%
\pgfpathlineto{\pgfqpoint{4.863660in}{5.020698in}}%
\pgfpathlineto{\pgfqpoint{4.869701in}{5.021066in}}%
\pgfpathlineto{\pgfqpoint{4.873728in}{5.017183in}}%
\pgfpathlineto{\pgfqpoint{4.875741in}{5.019125in}}%
\pgfpathlineto{\pgfqpoint{4.877755in}{5.018044in}}%
\pgfpathlineto{\pgfqpoint{4.883795in}{5.016471in}}%
\pgfpathlineto{\pgfqpoint{4.885808in}{5.018953in}}%
\pgfpathlineto{\pgfqpoint{4.887822in}{5.020526in}}%
\pgfpathlineto{\pgfqpoint{4.889835in}{5.034312in}}%
\pgfpathlineto{\pgfqpoint{4.891849in}{5.033845in}}%
\pgfpathlineto{\pgfqpoint{4.899902in}{5.035934in}}%
\pgfpathlineto{\pgfqpoint{4.901916in}{5.038932in}}%
\pgfpathlineto{\pgfqpoint{4.905943in}{5.036991in}}%
\pgfpathlineto{\pgfqpoint{4.911983in}{5.043036in}}%
\pgfpathlineto{\pgfqpoint{4.913997in}{5.040431in}}%
\pgfpathlineto{\pgfqpoint{4.920037in}{5.041562in}}%
\pgfpathlineto{\pgfqpoint{4.926077in}{5.037581in}}%
\pgfpathlineto{\pgfqpoint{4.928091in}{5.037851in}}%
\pgfpathlineto{\pgfqpoint{4.930104in}{5.041095in}}%
\pgfpathlineto{\pgfqpoint{4.932118in}{5.034976in}}%
\pgfpathlineto{\pgfqpoint{4.934131in}{5.033526in}}%
\pgfpathlineto{\pgfqpoint{4.940171in}{5.039227in}}%
\pgfpathlineto{\pgfqpoint{4.942185in}{5.036598in}}%
\pgfpathlineto{\pgfqpoint{4.946212in}{5.041931in}}%
\pgfpathlineto{\pgfqpoint{4.948225in}{5.043208in}}%
\pgfpathlineto{\pgfqpoint{4.954266in}{5.042545in}}%
\pgfpathlineto{\pgfqpoint{4.956279in}{5.040505in}}%
\pgfpathlineto{\pgfqpoint{4.958292in}{5.040235in}}%
\pgfpathlineto{\pgfqpoint{4.962319in}{5.041046in}}%
\pgfpathlineto{\pgfqpoint{4.968360in}{5.037900in}}%
\pgfpathlineto{\pgfqpoint{4.970373in}{5.038613in}}%
\pgfpathlineto{\pgfqpoint{4.974400in}{5.032961in}}%
\pgfpathlineto{\pgfqpoint{4.976413in}{5.041562in}}%
\pgfpathlineto{\pgfqpoint{4.982454in}{5.040505in}}%
\pgfpathlineto{\pgfqpoint{4.986481in}{5.036696in}}%
\pgfpathlineto{\pgfqpoint{4.988494in}{5.039989in}}%
\pgfpathlineto{\pgfqpoint{4.990508in}{5.033968in}}%
\pgfpathlineto{\pgfqpoint{4.996548in}{5.040235in}}%
\pgfpathlineto{\pgfqpoint{4.998561in}{5.018928in}}%
\pgfpathlineto{\pgfqpoint{5.000575in}{5.023008in}}%
\pgfpathlineto{\pgfqpoint{5.002588in}{5.020550in}}%
\pgfpathlineto{\pgfqpoint{5.004602in}{5.016225in}}%
\pgfpathlineto{\pgfqpoint{5.010642in}{5.017478in}}%
\pgfpathlineto{\pgfqpoint{5.012655in}{5.020821in}}%
\pgfpathlineto{\pgfqpoint{5.014669in}{5.025736in}}%
\pgfpathlineto{\pgfqpoint{5.024736in}{5.025858in}}%
\pgfpathlineto{\pgfqpoint{5.026750in}{5.029618in}}%
\pgfpathlineto{\pgfqpoint{5.038830in}{5.016004in}}%
\pgfpathlineto{\pgfqpoint{5.040844in}{5.017429in}}%
\pgfpathlineto{\pgfqpoint{5.044871in}{5.002881in}}%
\pgfpathlineto{\pgfqpoint{5.046884in}{5.001799in}}%
\pgfpathlineto{\pgfqpoint{5.052924in}{5.001726in}}%
\pgfpathlineto{\pgfqpoint{5.054938in}{5.002635in}}%
\pgfpathlineto{\pgfqpoint{5.056951in}{4.997818in}}%
\pgfpathlineto{\pgfqpoint{5.058965in}{5.003348in}}%
\pgfpathlineto{\pgfqpoint{5.060978in}{4.997769in}}%
\pgfpathlineto{\pgfqpoint{5.069032in}{4.997081in}}%
\pgfpathlineto{\pgfqpoint{5.071045in}{4.993886in}}%
\pgfpathlineto{\pgfqpoint{5.073059in}{4.995680in}}%
\pgfpathlineto{\pgfqpoint{5.075072in}{4.999612in}}%
\pgfpathlineto{\pgfqpoint{5.081113in}{4.995238in}}%
\pgfpathlineto{\pgfqpoint{5.083126in}{5.011408in}}%
\pgfpathlineto{\pgfqpoint{5.085140in}{5.013153in}}%
\pgfpathlineto{\pgfqpoint{5.087153in}{5.017110in}}%
\pgfpathlineto{\pgfqpoint{5.089166in}{5.025367in}}%
\pgfpathlineto{\pgfqpoint{5.097220in}{5.018363in}}%
\pgfpathlineto{\pgfqpoint{5.099234in}{5.028783in}}%
\pgfpathlineto{\pgfqpoint{5.101247in}{5.030847in}}%
\pgfpathlineto{\pgfqpoint{5.103261in}{5.030921in}}%
\pgfpathlineto{\pgfqpoint{5.109301in}{5.031879in}}%
\pgfpathlineto{\pgfqpoint{5.111314in}{5.033772in}}%
\pgfpathlineto{\pgfqpoint{5.113328in}{5.031093in}}%
\pgfpathlineto{\pgfqpoint{5.115341in}{5.027063in}}%
\pgfpathlineto{\pgfqpoint{5.117355in}{5.034509in}}%
\pgfpathlineto{\pgfqpoint{5.125409in}{5.038048in}}%
\pgfpathlineto{\pgfqpoint{5.127422in}{5.040530in}}%
\pgfpathlineto{\pgfqpoint{5.129435in}{5.040874in}}%
\pgfpathlineto{\pgfqpoint{5.131449in}{5.040137in}}%
\pgfpathlineto{\pgfqpoint{5.137489in}{5.042963in}}%
\pgfpathlineto{\pgfqpoint{5.139503in}{5.039498in}}%
\pgfpathlineto{\pgfqpoint{5.141516in}{5.041881in}}%
\pgfpathlineto{\pgfqpoint{5.143530in}{5.045863in}}%
\pgfpathlineto{\pgfqpoint{5.145543in}{5.044241in}}%
\pgfpathlineto{\pgfqpoint{5.151583in}{5.041070in}}%
\pgfpathlineto{\pgfqpoint{5.153597in}{5.047239in}}%
\pgfpathlineto{\pgfqpoint{5.157624in}{5.046747in}}%
\pgfpathlineto{\pgfqpoint{5.159637in}{5.048320in}}%
\pgfpathlineto{\pgfqpoint{5.165677in}{5.049451in}}%
\pgfpathlineto{\pgfqpoint{5.169704in}{5.048369in}}%
\pgfpathlineto{\pgfqpoint{5.171718in}{5.047902in}}%
\pgfpathlineto{\pgfqpoint{5.173731in}{5.052400in}}%
\pgfpathlineto{\pgfqpoint{5.179772in}{5.052252in}}%
\pgfpathlineto{\pgfqpoint{5.185812in}{5.056626in}}%
\pgfpathlineto{\pgfqpoint{5.187825in}{5.060190in}}%
\pgfpathlineto{\pgfqpoint{5.199906in}{5.058224in}}%
\pgfpathlineto{\pgfqpoint{5.207960in}{5.062352in}}%
\pgfpathlineto{\pgfqpoint{5.209973in}{5.058076in}}%
\pgfpathlineto{\pgfqpoint{5.211987in}{5.063409in}}%
\pgfpathlineto{\pgfqpoint{5.214000in}{5.063139in}}%
\pgfpathlineto{\pgfqpoint{5.216014in}{5.065179in}}%
\pgfpathlineto{\pgfqpoint{5.224067in}{5.061271in}}%
\pgfpathlineto{\pgfqpoint{5.226081in}{5.063532in}}%
\pgfpathlineto{\pgfqpoint{5.228094in}{5.064343in}}%
\pgfpathlineto{\pgfqpoint{5.230108in}{5.063139in}}%
\pgfpathlineto{\pgfqpoint{5.236148in}{5.063040in}}%
\pgfpathlineto{\pgfqpoint{5.238162in}{5.066309in}}%
\pgfpathlineto{\pgfqpoint{5.240175in}{5.067562in}}%
\pgfpathlineto{\pgfqpoint{5.242188in}{5.066579in}}%
\pgfpathlineto{\pgfqpoint{5.244202in}{5.067980in}}%
\pgfpathlineto{\pgfqpoint{5.252256in}{5.069946in}}%
\pgfpathlineto{\pgfqpoint{5.254269in}{5.068349in}}%
\pgfpathlineto{\pgfqpoint{5.256283in}{5.067710in}}%
\pgfpathlineto{\pgfqpoint{5.258296in}{5.067710in}}%
\pgfpathlineto{\pgfqpoint{5.264336in}{5.067095in}}%
\pgfpathlineto{\pgfqpoint{5.266350in}{5.062107in}}%
\pgfpathlineto{\pgfqpoint{5.268363in}{5.065695in}}%
\pgfpathlineto{\pgfqpoint{5.270377in}{5.063974in}}%
\pgfpathlineto{\pgfqpoint{5.278431in}{5.066997in}}%
\pgfpathlineto{\pgfqpoint{5.280444in}{5.066162in}}%
\pgfpathlineto{\pgfqpoint{5.282457in}{5.064343in}}%
\pgfpathlineto{\pgfqpoint{5.284471in}{5.065719in}}%
\pgfpathlineto{\pgfqpoint{5.286484in}{5.068250in}}%
\pgfpathlineto{\pgfqpoint{5.292525in}{5.067415in}}%
\pgfpathlineto{\pgfqpoint{5.294538in}{5.071347in}}%
\pgfpathlineto{\pgfqpoint{5.296552in}{5.070216in}}%
\pgfpathlineto{\pgfqpoint{5.298565in}{5.071077in}}%
\pgfpathlineto{\pgfqpoint{5.300578in}{5.066948in}}%
\pgfpathlineto{\pgfqpoint{5.306619in}{5.069725in}}%
\pgfpathlineto{\pgfqpoint{5.308632in}{5.065793in}}%
\pgfpathlineto{\pgfqpoint{5.310646in}{5.066063in}}%
\pgfpathlineto{\pgfqpoint{5.312659in}{5.062180in}}%
\pgfpathlineto{\pgfqpoint{5.314673in}{5.061959in}}%
\pgfpathlineto{\pgfqpoint{5.320713in}{5.064466in}}%
\pgfpathlineto{\pgfqpoint{5.322726in}{5.070192in}}%
\pgfpathlineto{\pgfqpoint{5.324740in}{5.073288in}}%
\pgfpathlineto{\pgfqpoint{5.326753in}{5.070831in}}%
\pgfpathlineto{\pgfqpoint{5.328767in}{5.070782in}}%
\pgfpathlineto{\pgfqpoint{5.336820in}{5.069479in}}%
\pgfpathlineto{\pgfqpoint{5.338834in}{5.070315in}}%
\pgfpathlineto{\pgfqpoint{5.340847in}{5.068840in}}%
\pgfpathlineto{\pgfqpoint{5.342861in}{5.069627in}}%
\pgfpathlineto{\pgfqpoint{5.352928in}{5.076532in}}%
\pgfpathlineto{\pgfqpoint{5.356955in}{5.070020in}}%
\pgfpathlineto{\pgfqpoint{5.362995in}{5.066776in}}%
\pgfpathlineto{\pgfqpoint{5.365009in}{5.067882in}}%
\pgfpathlineto{\pgfqpoint{5.367022in}{5.068300in}}%
\pgfpathlineto{\pgfqpoint{5.369036in}{5.073141in}}%
\pgfpathlineto{\pgfqpoint{5.371049in}{5.070905in}}%
\pgfpathlineto{\pgfqpoint{5.377089in}{5.076114in}}%
\pgfpathlineto{\pgfqpoint{5.379103in}{5.076434in}}%
\pgfpathlineto{\pgfqpoint{5.381116in}{5.076114in}}%
\pgfpathlineto{\pgfqpoint{5.383130in}{5.082651in}}%
\pgfpathlineto{\pgfqpoint{5.385143in}{5.071298in}}%
\pgfpathlineto{\pgfqpoint{5.391184in}{5.066923in}}%
\pgfpathlineto{\pgfqpoint{5.395210in}{5.076753in}}%
\pgfpathlineto{\pgfqpoint{5.397224in}{5.084961in}}%
\pgfpathlineto{\pgfqpoint{5.399237in}{5.085895in}}%
\pgfpathlineto{\pgfqpoint{5.407291in}{5.086116in}}%
\pgfpathlineto{\pgfqpoint{5.411318in}{5.084372in}}%
\pgfpathlineto{\pgfqpoint{5.413331in}{5.089508in}}%
\pgfpathlineto{\pgfqpoint{5.419372in}{5.091523in}}%
\pgfpathlineto{\pgfqpoint{5.421385in}{5.094128in}}%
\pgfpathlineto{\pgfqpoint{5.423399in}{5.094251in}}%
\pgfpathlineto{\pgfqpoint{5.425412in}{5.098109in}}%
\pgfpathlineto{\pgfqpoint{5.427426in}{5.099240in}}%
\pgfpathlineto{\pgfqpoint{5.435479in}{5.098871in}}%
\pgfpathlineto{\pgfqpoint{5.437493in}{5.099289in}}%
\pgfpathlineto{\pgfqpoint{5.439506in}{5.096635in}}%
\pgfpathlineto{\pgfqpoint{5.441520in}{5.097052in}}%
\pgfpathlineto{\pgfqpoint{5.447560in}{5.095209in}}%
\pgfpathlineto{\pgfqpoint{5.449574in}{5.090737in}}%
\pgfpathlineto{\pgfqpoint{5.451587in}{5.092113in}}%
\pgfpathlineto{\pgfqpoint{5.453600in}{5.091449in}}%
\pgfpathlineto{\pgfqpoint{5.455614in}{5.092309in}}%
\pgfpathlineto{\pgfqpoint{5.465681in}{5.092359in}}%
\pgfpathlineto{\pgfqpoint{5.467695in}{5.091081in}}%
\pgfpathlineto{\pgfqpoint{5.469708in}{5.092801in}}%
\pgfpathlineto{\pgfqpoint{5.479775in}{5.093366in}}%
\pgfpathlineto{\pgfqpoint{5.481789in}{5.098330in}}%
\pgfpathlineto{\pgfqpoint{5.483802in}{5.096659in}}%
\pgfpathlineto{\pgfqpoint{5.489842in}{5.097323in}}%
\pgfpathlineto{\pgfqpoint{5.491856in}{5.094349in}}%
\pgfpathlineto{\pgfqpoint{5.493869in}{5.098306in}}%
\pgfpathlineto{\pgfqpoint{5.495883in}{5.096733in}}%
\pgfpathlineto{\pgfqpoint{5.497896in}{5.097716in}}%
\pgfpathlineto{\pgfqpoint{5.503937in}{5.096463in}}%
\pgfpathlineto{\pgfqpoint{5.505950in}{5.098085in}}%
\pgfpathlineto{\pgfqpoint{5.507964in}{5.097249in}}%
\pgfpathlineto{\pgfqpoint{5.509977in}{5.097642in}}%
\pgfpathlineto{\pgfqpoint{5.511990in}{5.097249in}}%
\pgfpathlineto{\pgfqpoint{5.518031in}{5.099756in}}%
\pgfpathlineto{\pgfqpoint{5.520044in}{5.098945in}}%
\pgfpathlineto{\pgfqpoint{5.522058in}{5.096856in}}%
\pgfpathlineto{\pgfqpoint{5.526085in}{5.100444in}}%
\pgfpathlineto{\pgfqpoint{5.534138in}{5.099657in}}%
\pgfpathlineto{\pgfqpoint{5.536152in}{5.098035in}}%
\pgfpathlineto{\pgfqpoint{5.538165in}{5.098822in}}%
\pgfpathlineto{\pgfqpoint{5.540179in}{5.088623in}}%
\pgfpathlineto{\pgfqpoint{5.546219in}{5.094251in}}%
\pgfpathlineto{\pgfqpoint{5.548232in}{5.089606in}}%
\pgfpathlineto{\pgfqpoint{5.550246in}{5.088574in}}%
\pgfpathlineto{\pgfqpoint{5.552259in}{5.090786in}}%
\pgfpathlineto{\pgfqpoint{5.554273in}{5.087296in}}%
\pgfpathlineto{\pgfqpoint{5.560313in}{5.091105in}}%
\pgfpathlineto{\pgfqpoint{5.562327in}{5.093120in}}%
\pgfpathlineto{\pgfqpoint{5.564340in}{5.097667in}}%
\pgfpathlineto{\pgfqpoint{5.566353in}{5.098257in}}%
\pgfpathlineto{\pgfqpoint{5.568367in}{5.092629in}}%
\pgfpathlineto{\pgfqpoint{5.574407in}{5.089336in}}%
\pgfpathlineto{\pgfqpoint{5.576421in}{5.090122in}}%
\pgfpathlineto{\pgfqpoint{5.578434in}{5.093096in}}%
\pgfpathlineto{\pgfqpoint{5.580448in}{5.088009in}}%
\pgfpathlineto{\pgfqpoint{5.582461in}{5.089975in}}%
\pgfpathlineto{\pgfqpoint{5.588501in}{5.087247in}}%
\pgfpathlineto{\pgfqpoint{5.590515in}{5.079555in}}%
\pgfpathlineto{\pgfqpoint{5.592528in}{5.081177in}}%
\pgfpathlineto{\pgfqpoint{5.594542in}{5.079457in}}%
\pgfpathlineto{\pgfqpoint{5.596555in}{5.078769in}}%
\pgfpathlineto{\pgfqpoint{5.602596in}{5.078326in}}%
\pgfpathlineto{\pgfqpoint{5.604609in}{5.074984in}}%
\pgfpathlineto{\pgfqpoint{5.606622in}{5.075082in}}%
\pgfpathlineto{\pgfqpoint{5.610649in}{5.076458in}}%
\pgfpathlineto{\pgfqpoint{5.622730in}{5.075402in}}%
\pgfpathlineto{\pgfqpoint{5.624743in}{5.074566in}}%
\pgfpathlineto{\pgfqpoint{5.630784in}{5.078621in}}%
\pgfpathlineto{\pgfqpoint{5.632797in}{5.067095in}}%
\pgfpathlineto{\pgfqpoint{5.634811in}{5.067734in}}%
\pgfpathlineto{\pgfqpoint{5.636824in}{5.066014in}}%
\pgfpathlineto{\pgfqpoint{5.638838in}{5.066039in}}%
\pgfpathlineto{\pgfqpoint{5.644878in}{5.064957in}}%
\pgfpathlineto{\pgfqpoint{5.646891in}{5.062549in}}%
\pgfpathlineto{\pgfqpoint{5.650918in}{5.068472in}}%
\pgfpathlineto{\pgfqpoint{5.652932in}{5.067661in}}%
\pgfpathlineto{\pgfqpoint{5.660985in}{5.078080in}}%
\pgfpathlineto{\pgfqpoint{5.662999in}{5.076606in}}%
\pgfpathlineto{\pgfqpoint{5.665012in}{5.085502in}}%
\pgfpathlineto{\pgfqpoint{5.667026in}{5.087345in}}%
\pgfpathlineto{\pgfqpoint{5.673066in}{5.082381in}}%
\pgfpathlineto{\pgfqpoint{5.675080in}{5.085428in}}%
\pgfpathlineto{\pgfqpoint{5.677093in}{5.082848in}}%
\pgfpathlineto{\pgfqpoint{5.679107in}{5.084593in}}%
\pgfpathlineto{\pgfqpoint{5.681120in}{5.085035in}}%
\pgfpathlineto{\pgfqpoint{5.687160in}{5.081718in}}%
\pgfpathlineto{\pgfqpoint{5.691187in}{5.083389in}}%
\pgfpathlineto{\pgfqpoint{5.695214in}{5.086239in}}%
\pgfpathlineto{\pgfqpoint{5.701254in}{5.083978in}}%
\pgfpathlineto{\pgfqpoint{5.703268in}{5.084593in}}%
\pgfpathlineto{\pgfqpoint{5.705281in}{5.082209in}}%
\pgfpathlineto{\pgfqpoint{5.707295in}{5.084273in}}%
\pgfpathlineto{\pgfqpoint{5.717362in}{5.082356in}}%
\pgfpathlineto{\pgfqpoint{5.719375in}{5.092137in}}%
\pgfpathlineto{\pgfqpoint{5.721389in}{5.091744in}}%
\pgfpathlineto{\pgfqpoint{5.723402in}{5.097765in}}%
\pgfpathlineto{\pgfqpoint{5.729443in}{5.100419in}}%
\pgfpathlineto{\pgfqpoint{5.731456in}{5.098551in}}%
\pgfpathlineto{\pgfqpoint{5.733470in}{5.093415in}}%
\pgfpathlineto{\pgfqpoint{5.735483in}{5.092064in}}%
\pgfpathlineto{\pgfqpoint{5.737496in}{5.095381in}}%
\pgfpathlineto{\pgfqpoint{5.743537in}{5.097003in}}%
\pgfpathlineto{\pgfqpoint{5.745550in}{5.098134in}}%
\pgfpathlineto{\pgfqpoint{5.747564in}{5.097618in}}%
\pgfpathlineto{\pgfqpoint{5.749577in}{5.099412in}}%
\pgfpathlineto{\pgfqpoint{5.751591in}{5.098355in}}%
\pgfpathlineto{\pgfqpoint{5.759644in}{5.098748in}}%
\pgfpathlineto{\pgfqpoint{5.761658in}{5.096831in}}%
\pgfpathlineto{\pgfqpoint{5.765685in}{5.097937in}}%
\pgfpathlineto{\pgfqpoint{5.773739in}{5.096757in}}%
\pgfpathlineto{\pgfqpoint{5.775752in}{5.097372in}}%
\pgfpathlineto{\pgfqpoint{5.777765in}{5.095971in}}%
\pgfpathlineto{\pgfqpoint{5.779779in}{5.097175in}}%
\pgfpathlineto{\pgfqpoint{5.785819in}{5.094963in}}%
\pgfpathlineto{\pgfqpoint{5.787833in}{5.093366in}}%
\pgfpathlineto{\pgfqpoint{5.789846in}{5.096389in}}%
\pgfpathlineto{\pgfqpoint{5.791860in}{5.095357in}}%
\pgfpathlineto{\pgfqpoint{5.801927in}{5.094939in}}%
\pgfpathlineto{\pgfqpoint{5.803940in}{5.097765in}}%
\pgfpathlineto{\pgfqpoint{5.805954in}{5.098207in}}%
\pgfpathlineto{\pgfqpoint{5.807967in}{5.097765in}}%
\pgfpathlineto{\pgfqpoint{5.814007in}{5.097814in}}%
\pgfpathlineto{\pgfqpoint{5.816021in}{5.091965in}}%
\pgfpathlineto{\pgfqpoint{5.818034in}{5.093710in}}%
\pgfpathlineto{\pgfqpoint{5.820048in}{5.093907in}}%
\pgfpathlineto{\pgfqpoint{5.822061in}{5.095430in}}%
\pgfpathlineto{\pgfqpoint{5.830115in}{5.089311in}}%
\pgfpathlineto{\pgfqpoint{5.832129in}{5.090122in}}%
\pgfpathlineto{\pgfqpoint{5.834142in}{5.087837in}}%
\pgfpathlineto{\pgfqpoint{5.836155in}{5.089827in}}%
\pgfpathlineto{\pgfqpoint{5.842196in}{5.089950in}}%
\pgfpathlineto{\pgfqpoint{5.844209in}{5.091474in}}%
\pgfpathlineto{\pgfqpoint{5.848236in}{5.097052in}}%
\pgfpathlineto{\pgfqpoint{5.850250in}{5.098945in}}%
\pgfpathlineto{\pgfqpoint{5.858303in}{5.104916in}}%
\pgfpathlineto{\pgfqpoint{5.862330in}{5.111847in}}%
\pgfpathlineto{\pgfqpoint{5.864344in}{5.110790in}}%
\pgfpathlineto{\pgfqpoint{5.872397in}{5.111748in}}%
\pgfpathlineto{\pgfqpoint{5.874411in}{5.117941in}}%
\pgfpathlineto{\pgfqpoint{5.876424in}{5.120620in}}%
\pgfpathlineto{\pgfqpoint{5.878438in}{5.121111in}}%
\pgfpathlineto{\pgfqpoint{5.884478in}{5.119956in}}%
\pgfpathlineto{\pgfqpoint{5.886492in}{5.118654in}}%
\pgfpathlineto{\pgfqpoint{5.888505in}{5.126813in}}%
\pgfpathlineto{\pgfqpoint{5.890518in}{5.126862in}}%
\pgfpathlineto{\pgfqpoint{5.892532in}{5.125535in}}%
\pgfpathlineto{\pgfqpoint{5.898572in}{5.124576in}}%
\pgfpathlineto{\pgfqpoint{5.902599in}{5.126002in}}%
\pgfpathlineto{\pgfqpoint{5.904613in}{5.126886in}}%
\pgfpathlineto{\pgfqpoint{5.906626in}{5.129934in}}%
\pgfpathlineto{\pgfqpoint{5.912666in}{5.130646in}}%
\pgfpathlineto{\pgfqpoint{5.914680in}{5.127894in}}%
\pgfpathlineto{\pgfqpoint{5.916693in}{5.129909in}}%
\pgfpathlineto{\pgfqpoint{5.918707in}{5.127845in}}%
\pgfpathlineto{\pgfqpoint{5.920720in}{5.132588in}}%
\pgfpathlineto{\pgfqpoint{5.926761in}{5.134112in}}%
\pgfpathlineto{\pgfqpoint{5.928774in}{5.132072in}}%
\pgfpathlineto{\pgfqpoint{5.932801in}{5.132096in}}%
\pgfpathlineto{\pgfqpoint{5.934814in}{5.130622in}}%
\pgfpathlineto{\pgfqpoint{5.940855in}{5.128115in}}%
\pgfpathlineto{\pgfqpoint{5.942868in}{5.129418in}}%
\pgfpathlineto{\pgfqpoint{5.944882in}{5.128705in}}%
\pgfpathlineto{\pgfqpoint{5.946895in}{5.130081in}}%
\pgfpathlineto{\pgfqpoint{5.948908in}{5.130204in}}%
\pgfpathlineto{\pgfqpoint{5.954949in}{5.128803in}}%
\pgfpathlineto{\pgfqpoint{5.956962in}{5.127575in}}%
\pgfpathlineto{\pgfqpoint{5.958976in}{5.127673in}}%
\pgfpathlineto{\pgfqpoint{5.960989in}{5.126862in}}%
\pgfpathlineto{\pgfqpoint{5.963003in}{5.127108in}}%
\pgfpathlineto{\pgfqpoint{5.969043in}{5.126444in}}%
\pgfpathlineto{\pgfqpoint{5.971056in}{5.127280in}}%
\pgfpathlineto{\pgfqpoint{5.973070in}{5.126420in}}%
\pgfpathlineto{\pgfqpoint{5.975083in}{5.123987in}}%
\pgfpathlineto{\pgfqpoint{5.983137in}{5.127968in}}%
\pgfpathlineto{\pgfqpoint{5.987164in}{5.126739in}}%
\pgfpathlineto{\pgfqpoint{5.989177in}{5.129811in}}%
\pgfpathlineto{\pgfqpoint{5.991191in}{5.130597in}}%
\pgfpathlineto{\pgfqpoint{5.999245in}{5.139027in}}%
\pgfpathlineto{\pgfqpoint{6.001258in}{5.138707in}}%
\pgfpathlineto{\pgfqpoint{6.003272in}{5.141287in}}%
\pgfpathlineto{\pgfqpoint{6.011325in}{5.138265in}}%
\pgfpathlineto{\pgfqpoint{6.013339in}{5.142270in}}%
\pgfpathlineto{\pgfqpoint{6.015352in}{5.144826in}}%
\pgfpathlineto{\pgfqpoint{6.017366in}{5.149176in}}%
\pgfpathlineto{\pgfqpoint{6.019379in}{5.148906in}}%
\pgfpathlineto{\pgfqpoint{6.025419in}{5.147185in}}%
\pgfpathlineto{\pgfqpoint{6.027433in}{5.145588in}}%
\pgfpathlineto{\pgfqpoint{6.029446in}{5.142516in}}%
\pgfpathlineto{\pgfqpoint{6.031460in}{5.142762in}}%
\pgfpathlineto{\pgfqpoint{6.033473in}{5.142369in}}%
\pgfpathlineto{\pgfqpoint{6.041527in}{5.145465in}}%
\pgfpathlineto{\pgfqpoint{6.043540in}{5.141042in}}%
\pgfpathlineto{\pgfqpoint{6.047567in}{5.143278in}}%
\pgfpathlineto{\pgfqpoint{6.053608in}{5.149594in}}%
\pgfpathlineto{\pgfqpoint{6.055621in}{5.147775in}}%
\pgfpathlineto{\pgfqpoint{6.057635in}{5.147235in}}%
\pgfpathlineto{\pgfqpoint{6.059648in}{5.152002in}}%
\pgfpathlineto{\pgfqpoint{6.061661in}{5.154632in}}%
\pgfpathlineto{\pgfqpoint{6.069715in}{5.158760in}}%
\pgfpathlineto{\pgfqpoint{6.071729in}{5.163503in}}%
\pgfpathlineto{\pgfqpoint{6.073742in}{5.163233in}}%
\pgfpathlineto{\pgfqpoint{6.075756in}{5.168713in}}%
\pgfpathlineto{\pgfqpoint{6.081796in}{5.167583in}}%
\pgfpathlineto{\pgfqpoint{6.083809in}{5.165690in}}%
\pgfpathlineto{\pgfqpoint{6.085823in}{5.164757in}}%
\pgfpathlineto{\pgfqpoint{6.089850in}{5.169229in}}%
\pgfpathlineto{\pgfqpoint{6.095890in}{5.170311in}}%
\pgfpathlineto{\pgfqpoint{6.097904in}{5.174316in}}%
\pgfpathlineto{\pgfqpoint{6.099917in}{5.176430in}}%
\pgfpathlineto{\pgfqpoint{6.101930in}{5.179403in}}%
\pgfpathlineto{\pgfqpoint{6.103944in}{5.183950in}}%
\pgfpathlineto{\pgfqpoint{6.111998in}{5.184245in}}%
\pgfpathlineto{\pgfqpoint{6.116025in}{5.181566in}}%
\pgfpathlineto{\pgfqpoint{6.118038in}{5.183163in}}%
\pgfpathlineto{\pgfqpoint{6.124078in}{5.182475in}}%
\pgfpathlineto{\pgfqpoint{6.126092in}{5.175889in}}%
\pgfpathlineto{\pgfqpoint{6.128105in}{5.177855in}}%
\pgfpathlineto{\pgfqpoint{6.130119in}{5.171392in}}%
\pgfpathlineto{\pgfqpoint{6.132132in}{5.172178in}}%
\pgfpathlineto{\pgfqpoint{6.138172in}{5.176012in}}%
\pgfpathlineto{\pgfqpoint{6.142199in}{5.175840in}}%
\pgfpathlineto{\pgfqpoint{6.144213in}{5.171785in}}%
\pgfpathlineto{\pgfqpoint{6.146226in}{5.175447in}}%
\pgfpathlineto{\pgfqpoint{6.152267in}{5.177536in}}%
\pgfpathlineto{\pgfqpoint{6.154280in}{5.175594in}}%
\pgfpathlineto{\pgfqpoint{6.156294in}{5.179428in}}%
\pgfpathlineto{\pgfqpoint{6.158307in}{5.178936in}}%
\pgfpathlineto{\pgfqpoint{6.160320in}{5.180534in}}%
\pgfpathlineto{\pgfqpoint{6.166361in}{5.180313in}}%
\pgfpathlineto{\pgfqpoint{6.168374in}{5.179452in}}%
\pgfpathlineto{\pgfqpoint{6.170388in}{5.181296in}}%
\pgfpathlineto{\pgfqpoint{6.172401in}{5.182107in}}%
\pgfpathlineto{\pgfqpoint{6.174415in}{5.179108in}}%
\pgfpathlineto{\pgfqpoint{6.180455in}{5.176405in}}%
\pgfpathlineto{\pgfqpoint{6.182468in}{5.151658in}}%
\pgfpathlineto{\pgfqpoint{6.184482in}{5.150823in}}%
\pgfpathlineto{\pgfqpoint{6.186495in}{5.153206in}}%
\pgfpathlineto{\pgfqpoint{6.188509in}{5.152420in}}%
\pgfpathlineto{\pgfqpoint{6.194549in}{5.155811in}}%
\pgfpathlineto{\pgfqpoint{6.200589in}{5.170851in}}%
\pgfpathlineto{\pgfqpoint{6.202603in}{5.170925in}}%
\pgfpathlineto{\pgfqpoint{6.208643in}{5.170433in}}%
\pgfpathlineto{\pgfqpoint{6.210657in}{5.168074in}}%
\pgfpathlineto{\pgfqpoint{6.212670in}{5.168197in}}%
\pgfpathlineto{\pgfqpoint{6.216697in}{5.167017in}}%
\pgfpathlineto{\pgfqpoint{6.222737in}{5.170261in}}%
\pgfpathlineto{\pgfqpoint{6.224751in}{5.169819in}}%
\pgfpathlineto{\pgfqpoint{6.226764in}{5.171539in}}%
\pgfpathlineto{\pgfqpoint{6.228778in}{5.165543in}}%
\pgfpathlineto{\pgfqpoint{6.230791in}{5.161316in}}%
\pgfpathlineto{\pgfqpoint{6.236831in}{5.163700in}}%
\pgfpathlineto{\pgfqpoint{6.238845in}{5.165813in}}%
\pgfpathlineto{\pgfqpoint{6.240858in}{5.162029in}}%
\pgfpathlineto{\pgfqpoint{6.242872in}{5.160726in}}%
\pgfpathlineto{\pgfqpoint{6.244885in}{5.160751in}}%
\pgfpathlineto{\pgfqpoint{6.250926in}{5.161537in}}%
\pgfpathlineto{\pgfqpoint{6.252939in}{5.162619in}}%
\pgfpathlineto{\pgfqpoint{6.256966in}{5.165887in}}%
\pgfpathlineto{\pgfqpoint{6.258979in}{5.164118in}}%
\pgfpathlineto{\pgfqpoint{6.267033in}{5.158048in}}%
\pgfpathlineto{\pgfqpoint{6.269047in}{5.160579in}}%
\pgfpathlineto{\pgfqpoint{6.271060in}{5.166231in}}%
\pgfpathlineto{\pgfqpoint{6.279114in}{5.178175in}}%
\pgfpathlineto{\pgfqpoint{6.283141in}{5.178027in}}%
\pgfpathlineto{\pgfqpoint{6.287168in}{5.187071in}}%
\pgfpathlineto{\pgfqpoint{6.293208in}{5.188029in}}%
\pgfpathlineto{\pgfqpoint{6.295221in}{5.187562in}}%
\pgfpathlineto{\pgfqpoint{6.297235in}{5.180411in}}%
\pgfpathlineto{\pgfqpoint{6.299248in}{5.180239in}}%
\pgfpathlineto{\pgfqpoint{6.301262in}{5.181001in}}%
\pgfpathlineto{\pgfqpoint{6.307302in}{5.180927in}}%
\pgfpathlineto{\pgfqpoint{6.309316in}{5.181640in}}%
\pgfpathlineto{\pgfqpoint{6.311329in}{5.178003in}}%
\pgfpathlineto{\pgfqpoint{6.313342in}{5.178150in}}%
\pgfpathlineto{\pgfqpoint{6.315356in}{5.178986in}}%
\pgfpathlineto{\pgfqpoint{6.321396in}{5.185695in}}%
\pgfpathlineto{\pgfqpoint{6.325423in}{5.194517in}}%
\pgfpathlineto{\pgfqpoint{6.327437in}{5.194148in}}%
\pgfpathlineto{\pgfqpoint{6.329450in}{5.194517in}}%
\pgfpathlineto{\pgfqpoint{6.337504in}{5.195058in}}%
\pgfpathlineto{\pgfqpoint{6.339517in}{5.194492in}}%
\pgfpathlineto{\pgfqpoint{6.341531in}{5.197024in}}%
\pgfpathlineto{\pgfqpoint{6.343544in}{5.197319in}}%
\pgfpathlineto{\pgfqpoint{6.349584in}{5.199678in}}%
\pgfpathlineto{\pgfqpoint{6.351598in}{5.197392in}}%
\pgfpathlineto{\pgfqpoint{6.353611in}{5.198621in}}%
\pgfpathlineto{\pgfqpoint{6.355625in}{5.200882in}}%
\pgfpathlineto{\pgfqpoint{6.357638in}{5.205772in}}%
\pgfpathlineto{\pgfqpoint{6.363679in}{5.206313in}}%
\pgfpathlineto{\pgfqpoint{6.365692in}{5.237032in}}%
\pgfpathlineto{\pgfqpoint{6.367705in}{5.244134in}}%
\pgfpathlineto{\pgfqpoint{6.369719in}{5.233026in}}%
\pgfpathlineto{\pgfqpoint{6.371732in}{5.237253in}}%
\pgfpathlineto{\pgfqpoint{6.379786in}{5.226563in}}%
\pgfpathlineto{\pgfqpoint{6.381800in}{5.226538in}}%
\pgfpathlineto{\pgfqpoint{6.383813in}{5.231355in}}%
\pgfpathlineto{\pgfqpoint{6.385827in}{5.231330in}}%
\pgfpathlineto{\pgfqpoint{6.391867in}{5.226858in}}%
\pgfpathlineto{\pgfqpoint{6.395894in}{5.225727in}}%
\pgfpathlineto{\pgfqpoint{6.397907in}{5.222360in}}%
\pgfpathlineto{\pgfqpoint{6.399921in}{5.220149in}}%
\pgfpathlineto{\pgfqpoint{6.405961in}{5.221943in}}%
\pgfpathlineto{\pgfqpoint{6.407974in}{5.224548in}}%
\pgfpathlineto{\pgfqpoint{6.409988in}{5.220026in}}%
\pgfpathlineto{\pgfqpoint{6.412001in}{5.224769in}}%
\pgfpathlineto{\pgfqpoint{6.414015in}{5.224621in}}%
\pgfpathlineto{\pgfqpoint{6.420055in}{5.229610in}}%
\pgfpathlineto{\pgfqpoint{6.422069in}{5.235729in}}%
\pgfpathlineto{\pgfqpoint{6.424082in}{5.232584in}}%
\pgfpathlineto{\pgfqpoint{6.428109in}{5.232092in}}%
\pgfpathlineto{\pgfqpoint{6.434149in}{5.238285in}}%
\pgfpathlineto{\pgfqpoint{6.436163in}{5.242119in}}%
\pgfpathlineto{\pgfqpoint{6.438176in}{5.247722in}}%
\pgfpathlineto{\pgfqpoint{6.440190in}{5.259813in}}%
\pgfpathlineto{\pgfqpoint{6.442203in}{5.255144in}}%
\pgfpathlineto{\pgfqpoint{6.448243in}{5.250671in}}%
\pgfpathlineto{\pgfqpoint{6.450257in}{5.248312in}}%
\pgfpathlineto{\pgfqpoint{6.452270in}{5.249295in}}%
\pgfpathlineto{\pgfqpoint{6.454284in}{5.253128in}}%
\pgfpathlineto{\pgfqpoint{6.456297in}{5.248017in}}%
\pgfpathlineto{\pgfqpoint{6.462337in}{5.250769in}}%
\pgfpathlineto{\pgfqpoint{6.464351in}{5.244355in}}%
\pgfpathlineto{\pgfqpoint{6.466364in}{5.250327in}}%
\pgfpathlineto{\pgfqpoint{6.468378in}{5.247894in}}%
\pgfpathlineto{\pgfqpoint{6.470391in}{5.247697in}}%
\pgfpathlineto{\pgfqpoint{6.478445in}{5.248631in}}%
\pgfpathlineto{\pgfqpoint{6.480459in}{5.245166in}}%
\pgfpathlineto{\pgfqpoint{6.482472in}{5.240128in}}%
\pgfpathlineto{\pgfqpoint{6.484485in}{5.240005in}}%
\pgfpathlineto{\pgfqpoint{6.492539in}{5.241701in}}%
\pgfpathlineto{\pgfqpoint{6.494553in}{5.243470in}}%
\pgfpathlineto{\pgfqpoint{6.498580in}{5.241504in}}%
\pgfpathlineto{\pgfqpoint{6.498580in}{5.241504in}}%
\pgfusepath{stroke}%
\end{pgfscope}%
\begin{pgfscope}%
\pgfsetrectcap%
\pgfsetmiterjoin%
\pgfsetlinewidth{0.803000pt}%
\definecolor{currentstroke}{rgb}{1.000000,1.000000,1.000000}%
\pgfsetstrokecolor{currentstroke}%
\pgfsetdash{}{0pt}%
\pgfpathmoveto{\pgfqpoint{1.875000in}{4.835882in}}%
\pgfpathlineto{\pgfqpoint{1.875000in}{5.280000in}}%
\pgfusepath{stroke}%
\end{pgfscope}%
\begin{pgfscope}%
\pgfsetrectcap%
\pgfsetmiterjoin%
\pgfsetlinewidth{0.803000pt}%
\definecolor{currentstroke}{rgb}{1.000000,1.000000,1.000000}%
\pgfsetstrokecolor{currentstroke}%
\pgfsetdash{}{0pt}%
\pgfpathmoveto{\pgfqpoint{6.718750in}{4.835882in}}%
\pgfpathlineto{\pgfqpoint{6.718750in}{5.280000in}}%
\pgfusepath{stroke}%
\end{pgfscope}%
\begin{pgfscope}%
\pgfsetrectcap%
\pgfsetmiterjoin%
\pgfsetlinewidth{0.803000pt}%
\definecolor{currentstroke}{rgb}{1.000000,1.000000,1.000000}%
\pgfsetstrokecolor{currentstroke}%
\pgfsetdash{}{0pt}%
\pgfpathmoveto{\pgfqpoint{1.875000in}{4.835882in}}%
\pgfpathlineto{\pgfqpoint{6.718750in}{4.835882in}}%
\pgfusepath{stroke}%
\end{pgfscope}%
\begin{pgfscope}%
\pgfsetrectcap%
\pgfsetmiterjoin%
\pgfsetlinewidth{0.803000pt}%
\definecolor{currentstroke}{rgb}{1.000000,1.000000,1.000000}%
\pgfsetstrokecolor{currentstroke}%
\pgfsetdash{}{0pt}%
\pgfpathmoveto{\pgfqpoint{1.875000in}{5.280000in}}%
\pgfpathlineto{\pgfqpoint{6.718750in}{5.280000in}}%
\pgfusepath{stroke}%
\end{pgfscope}%
\begin{pgfscope}%
\definecolor{textcolor}{rgb}{0.150000,0.150000,0.150000}%
\pgfsetstrokecolor{textcolor}%
\pgfsetfillcolor{textcolor}%
\pgftext[x=4.296875in,y=5.363333in,,base]{\color{textcolor}\rmfamily\fontsize{16.800000}{20.160000}\selectfont MMM}%
\end{pgfscope}%
\begin{pgfscope}%
\pgfsetbuttcap%
\pgfsetmiterjoin%
\definecolor{currentfill}{rgb}{0.917647,0.917647,0.949020}%
\pgfsetfillcolor{currentfill}%
\pgfsetlinewidth{0.000000pt}%
\definecolor{currentstroke}{rgb}{0.000000,0.000000,0.000000}%
\pgfsetstrokecolor{currentstroke}%
\pgfsetstrokeopacity{0.000000}%
\pgfsetdash{}{0pt}%
\pgfpathmoveto{\pgfqpoint{8.656250in}{4.835882in}}%
\pgfpathlineto{\pgfqpoint{13.500000in}{4.835882in}}%
\pgfpathlineto{\pgfqpoint{13.500000in}{5.280000in}}%
\pgfpathlineto{\pgfqpoint{8.656250in}{5.280000in}}%
\pgfpathclose%
\pgfusepath{fill}%
\end{pgfscope}%
\begin{pgfscope}%
\pgfpathrectangle{\pgfqpoint{8.656250in}{4.835882in}}{\pgfqpoint{4.843750in}{0.444118in}}%
\pgfusepath{clip}%
\pgfsetroundcap%
\pgfsetroundjoin%
\pgfsetlinewidth{0.803000pt}%
\definecolor{currentstroke}{rgb}{1.000000,1.000000,1.000000}%
\pgfsetstrokecolor{currentstroke}%
\pgfsetdash{}{0pt}%
\pgfpathmoveto{\pgfqpoint{8.872394in}{4.835882in}}%
\pgfpathlineto{\pgfqpoint{8.872394in}{5.280000in}}%
\pgfusepath{stroke}%
\end{pgfscope}%
\begin{pgfscope}%
\definecolor{textcolor}{rgb}{0.150000,0.150000,0.150000}%
\pgfsetstrokecolor{textcolor}%
\pgfsetfillcolor{textcolor}%
\pgftext[x=8.872394in,y=4.738660in,,top]{\color{textcolor}\rmfamily\fontsize{14.000000}{16.800000}\selectfont 2012}%
\end{pgfscope}%
\begin{pgfscope}%
\pgfpathrectangle{\pgfqpoint{8.656250in}{4.835882in}}{\pgfqpoint{4.843750in}{0.444118in}}%
\pgfusepath{clip}%
\pgfsetroundcap%
\pgfsetroundjoin%
\pgfsetlinewidth{0.803000pt}%
\definecolor{currentstroke}{rgb}{1.000000,1.000000,1.000000}%
\pgfsetstrokecolor{currentstroke}%
\pgfsetdash{}{0pt}%
\pgfpathmoveto{\pgfqpoint{9.609315in}{4.835882in}}%
\pgfpathlineto{\pgfqpoint{9.609315in}{5.280000in}}%
\pgfusepath{stroke}%
\end{pgfscope}%
\begin{pgfscope}%
\definecolor{textcolor}{rgb}{0.150000,0.150000,0.150000}%
\pgfsetstrokecolor{textcolor}%
\pgfsetfillcolor{textcolor}%
\pgftext[x=9.609315in,y=4.738660in,,top]{\color{textcolor}\rmfamily\fontsize{14.000000}{16.800000}\selectfont 2013}%
\end{pgfscope}%
\begin{pgfscope}%
\pgfpathrectangle{\pgfqpoint{8.656250in}{4.835882in}}{\pgfqpoint{4.843750in}{0.444118in}}%
\pgfusepath{clip}%
\pgfsetroundcap%
\pgfsetroundjoin%
\pgfsetlinewidth{0.803000pt}%
\definecolor{currentstroke}{rgb}{1.000000,1.000000,1.000000}%
\pgfsetstrokecolor{currentstroke}%
\pgfsetdash{}{0pt}%
\pgfpathmoveto{\pgfqpoint{10.344223in}{4.835882in}}%
\pgfpathlineto{\pgfqpoint{10.344223in}{5.280000in}}%
\pgfusepath{stroke}%
\end{pgfscope}%
\begin{pgfscope}%
\definecolor{textcolor}{rgb}{0.150000,0.150000,0.150000}%
\pgfsetstrokecolor{textcolor}%
\pgfsetfillcolor{textcolor}%
\pgftext[x=10.344223in,y=4.738660in,,top]{\color{textcolor}\rmfamily\fontsize{14.000000}{16.800000}\selectfont 2014}%
\end{pgfscope}%
\begin{pgfscope}%
\pgfpathrectangle{\pgfqpoint{8.656250in}{4.835882in}}{\pgfqpoint{4.843750in}{0.444118in}}%
\pgfusepath{clip}%
\pgfsetroundcap%
\pgfsetroundjoin%
\pgfsetlinewidth{0.803000pt}%
\definecolor{currentstroke}{rgb}{1.000000,1.000000,1.000000}%
\pgfsetstrokecolor{currentstroke}%
\pgfsetdash{}{0pt}%
\pgfpathmoveto{\pgfqpoint{11.079132in}{4.835882in}}%
\pgfpathlineto{\pgfqpoint{11.079132in}{5.280000in}}%
\pgfusepath{stroke}%
\end{pgfscope}%
\begin{pgfscope}%
\definecolor{textcolor}{rgb}{0.150000,0.150000,0.150000}%
\pgfsetstrokecolor{textcolor}%
\pgfsetfillcolor{textcolor}%
\pgftext[x=11.079132in,y=4.738660in,,top]{\color{textcolor}\rmfamily\fontsize{14.000000}{16.800000}\selectfont 2015}%
\end{pgfscope}%
\begin{pgfscope}%
\pgfpathrectangle{\pgfqpoint{8.656250in}{4.835882in}}{\pgfqpoint{4.843750in}{0.444118in}}%
\pgfusepath{clip}%
\pgfsetroundcap%
\pgfsetroundjoin%
\pgfsetlinewidth{0.803000pt}%
\definecolor{currentstroke}{rgb}{1.000000,1.000000,1.000000}%
\pgfsetstrokecolor{currentstroke}%
\pgfsetdash{}{0pt}%
\pgfpathmoveto{\pgfqpoint{11.814040in}{4.835882in}}%
\pgfpathlineto{\pgfqpoint{11.814040in}{5.280000in}}%
\pgfusepath{stroke}%
\end{pgfscope}%
\begin{pgfscope}%
\definecolor{textcolor}{rgb}{0.150000,0.150000,0.150000}%
\pgfsetstrokecolor{textcolor}%
\pgfsetfillcolor{textcolor}%
\pgftext[x=11.814040in,y=4.738660in,,top]{\color{textcolor}\rmfamily\fontsize{14.000000}{16.800000}\selectfont 2016}%
\end{pgfscope}%
\begin{pgfscope}%
\pgfpathrectangle{\pgfqpoint{8.656250in}{4.835882in}}{\pgfqpoint{4.843750in}{0.444118in}}%
\pgfusepath{clip}%
\pgfsetroundcap%
\pgfsetroundjoin%
\pgfsetlinewidth{0.803000pt}%
\definecolor{currentstroke}{rgb}{1.000000,1.000000,1.000000}%
\pgfsetstrokecolor{currentstroke}%
\pgfsetdash{}{0pt}%
\pgfpathmoveto{\pgfqpoint{12.550962in}{4.835882in}}%
\pgfpathlineto{\pgfqpoint{12.550962in}{5.280000in}}%
\pgfusepath{stroke}%
\end{pgfscope}%
\begin{pgfscope}%
\definecolor{textcolor}{rgb}{0.150000,0.150000,0.150000}%
\pgfsetstrokecolor{textcolor}%
\pgfsetfillcolor{textcolor}%
\pgftext[x=12.550962in,y=4.738660in,,top]{\color{textcolor}\rmfamily\fontsize{14.000000}{16.800000}\selectfont 2017}%
\end{pgfscope}%
\begin{pgfscope}%
\pgfpathrectangle{\pgfqpoint{8.656250in}{4.835882in}}{\pgfqpoint{4.843750in}{0.444118in}}%
\pgfusepath{clip}%
\pgfsetroundcap%
\pgfsetroundjoin%
\pgfsetlinewidth{0.803000pt}%
\definecolor{currentstroke}{rgb}{1.000000,1.000000,1.000000}%
\pgfsetstrokecolor{currentstroke}%
\pgfsetdash{}{0pt}%
\pgfpathmoveto{\pgfqpoint{13.285870in}{4.835882in}}%
\pgfpathlineto{\pgfqpoint{13.285870in}{5.280000in}}%
\pgfusepath{stroke}%
\end{pgfscope}%
\begin{pgfscope}%
\definecolor{textcolor}{rgb}{0.150000,0.150000,0.150000}%
\pgfsetstrokecolor{textcolor}%
\pgfsetfillcolor{textcolor}%
\pgftext[x=13.285870in,y=4.738660in,,top]{\color{textcolor}\rmfamily\fontsize{14.000000}{16.800000}\selectfont 2018}%
\end{pgfscope}%
\begin{pgfscope}%
\pgfpathrectangle{\pgfqpoint{8.656250in}{4.835882in}}{\pgfqpoint{4.843750in}{0.444118in}}%
\pgfusepath{clip}%
\pgfsetroundcap%
\pgfsetroundjoin%
\pgfsetlinewidth{0.803000pt}%
\definecolor{currentstroke}{rgb}{1.000000,1.000000,1.000000}%
\pgfsetstrokecolor{currentstroke}%
\pgfsetdash{}{0pt}%
\pgfpathmoveto{\pgfqpoint{8.656250in}{4.906369in}}%
\pgfpathlineto{\pgfqpoint{13.500000in}{4.906369in}}%
\pgfusepath{stroke}%
\end{pgfscope}%
\begin{pgfscope}%
\definecolor{textcolor}{rgb}{0.150000,0.150000,0.150000}%
\pgfsetstrokecolor{textcolor}%
\pgfsetfillcolor{textcolor}%
\pgftext[x=8.311605in,y=4.832503in,left,base]{\color{textcolor}\rmfamily\fontsize{14.000000}{16.800000}\selectfont 50}%
\end{pgfscope}%
\begin{pgfscope}%
\pgfpathrectangle{\pgfqpoint{8.656250in}{4.835882in}}{\pgfqpoint{4.843750in}{0.444118in}}%
\pgfusepath{clip}%
\pgfsetroundcap%
\pgfsetroundjoin%
\pgfsetlinewidth{0.803000pt}%
\definecolor{currentstroke}{rgb}{1.000000,1.000000,1.000000}%
\pgfsetstrokecolor{currentstroke}%
\pgfsetdash{}{0pt}%
\pgfpathmoveto{\pgfqpoint{8.656250in}{5.093495in}}%
\pgfpathlineto{\pgfqpoint{13.500000in}{5.093495in}}%
\pgfusepath{stroke}%
\end{pgfscope}%
\begin{pgfscope}%
\definecolor{textcolor}{rgb}{0.150000,0.150000,0.150000}%
\pgfsetstrokecolor{textcolor}%
\pgfsetfillcolor{textcolor}%
\pgftext[x=8.311605in,y=5.019629in,left,base]{\color{textcolor}\rmfamily\fontsize{14.000000}{16.800000}\selectfont 75}%
\end{pgfscope}%
\begin{pgfscope}%
\pgfpathrectangle{\pgfqpoint{8.656250in}{4.835882in}}{\pgfqpoint{4.843750in}{0.444118in}}%
\pgfusepath{clip}%
\pgfsetroundcap%
\pgfsetroundjoin%
\pgfsetlinewidth{1.505625pt}%
\definecolor{currentstroke}{rgb}{0.121569,0.466667,0.705882}%
\pgfsetstrokecolor{currentstroke}%
\pgfsetdash{}{0pt}%
\pgfpathmoveto{\pgfqpoint{8.876420in}{4.856070in}}%
\pgfpathlineto{\pgfqpoint{8.878434in}{4.856294in}}%
\pgfpathlineto{\pgfqpoint{8.880447in}{4.860037in}}%
\pgfpathlineto{\pgfqpoint{8.882461in}{4.856444in}}%
\pgfpathlineto{\pgfqpoint{8.888501in}{4.857267in}}%
\pgfpathlineto{\pgfqpoint{8.892528in}{4.861010in}}%
\pgfpathlineto{\pgfqpoint{8.894541in}{4.865725in}}%
\pgfpathlineto{\pgfqpoint{8.904609in}{4.869543in}}%
\pgfpathlineto{\pgfqpoint{8.908636in}{4.874483in}}%
\pgfpathlineto{\pgfqpoint{8.910649in}{4.868345in}}%
\pgfpathlineto{\pgfqpoint{8.918703in}{4.862881in}}%
\pgfpathlineto{\pgfqpoint{8.920716in}{4.869243in}}%
\pgfpathlineto{\pgfqpoint{8.924743in}{4.867073in}}%
\pgfpathlineto{\pgfqpoint{8.930784in}{4.862207in}}%
\pgfpathlineto{\pgfqpoint{8.932797in}{4.869019in}}%
\pgfpathlineto{\pgfqpoint{8.936824in}{4.875905in}}%
\pgfpathlineto{\pgfqpoint{8.938837in}{4.883240in}}%
\pgfpathlineto{\pgfqpoint{8.944878in}{4.880246in}}%
\pgfpathlineto{\pgfqpoint{8.946891in}{4.882417in}}%
\pgfpathlineto{\pgfqpoint{8.948905in}{4.879123in}}%
\pgfpathlineto{\pgfqpoint{8.950918in}{4.883614in}}%
\pgfpathlineto{\pgfqpoint{8.952931in}{4.880246in}}%
\pgfpathlineto{\pgfqpoint{8.958972in}{4.881968in}}%
\pgfpathlineto{\pgfqpoint{8.960985in}{4.881294in}}%
\pgfpathlineto{\pgfqpoint{8.962999in}{4.878300in}}%
\pgfpathlineto{\pgfqpoint{8.965012in}{4.887357in}}%
\pgfpathlineto{\pgfqpoint{8.977093in}{4.886833in}}%
\pgfpathlineto{\pgfqpoint{8.979106in}{4.886010in}}%
\pgfpathlineto{\pgfqpoint{8.981120in}{4.890501in}}%
\pgfpathlineto{\pgfqpoint{8.987160in}{4.896189in}}%
\pgfpathlineto{\pgfqpoint{8.989173in}{4.893345in}}%
\pgfpathlineto{\pgfqpoint{8.991187in}{4.887507in}}%
\pgfpathlineto{\pgfqpoint{8.993200in}{4.892073in}}%
\pgfpathlineto{\pgfqpoint{8.995214in}{4.888180in}}%
\pgfpathlineto{\pgfqpoint{9.001254in}{4.888031in}}%
\pgfpathlineto{\pgfqpoint{9.003268in}{4.879797in}}%
\pgfpathlineto{\pgfqpoint{9.009308in}{4.889603in}}%
\pgfpathlineto{\pgfqpoint{9.015348in}{4.886683in}}%
\pgfpathlineto{\pgfqpoint{9.017362in}{4.896638in}}%
\pgfpathlineto{\pgfqpoint{9.019375in}{4.909438in}}%
\pgfpathlineto{\pgfqpoint{9.021389in}{4.913255in}}%
\pgfpathlineto{\pgfqpoint{9.023402in}{4.912133in}}%
\pgfpathlineto{\pgfqpoint{9.029442in}{4.916923in}}%
\pgfpathlineto{\pgfqpoint{9.031456in}{4.914528in}}%
\pgfpathlineto{\pgfqpoint{9.033469in}{4.915426in}}%
\pgfpathlineto{\pgfqpoint{9.035483in}{4.917073in}}%
\pgfpathlineto{\pgfqpoint{9.037496in}{4.916773in}}%
\pgfpathlineto{\pgfqpoint{9.043537in}{4.926279in}}%
\pgfpathlineto{\pgfqpoint{9.045550in}{4.923360in}}%
\pgfpathlineto{\pgfqpoint{9.047563in}{4.928974in}}%
\pgfpathlineto{\pgfqpoint{9.049577in}{4.921115in}}%
\pgfpathlineto{\pgfqpoint{9.051590in}{4.920890in}}%
\pgfpathlineto{\pgfqpoint{9.057631in}{4.921938in}}%
\pgfpathlineto{\pgfqpoint{9.059644in}{4.925830in}}%
\pgfpathlineto{\pgfqpoint{9.061658in}{4.918944in}}%
\pgfpathlineto{\pgfqpoint{9.063671in}{4.923884in}}%
\pgfpathlineto{\pgfqpoint{9.071725in}{4.917522in}}%
\pgfpathlineto{\pgfqpoint{9.073738in}{4.910935in}}%
\pgfpathlineto{\pgfqpoint{9.075752in}{4.916100in}}%
\pgfpathlineto{\pgfqpoint{9.077765in}{4.923435in}}%
\pgfpathlineto{\pgfqpoint{9.079779in}{4.918345in}}%
\pgfpathlineto{\pgfqpoint{9.085819in}{4.922013in}}%
\pgfpathlineto{\pgfqpoint{9.087832in}{4.924408in}}%
\pgfpathlineto{\pgfqpoint{9.089846in}{4.923435in}}%
\pgfpathlineto{\pgfqpoint{9.091859in}{4.920291in}}%
\pgfpathlineto{\pgfqpoint{9.093873in}{4.919468in}}%
\pgfpathlineto{\pgfqpoint{9.099913in}{4.918644in}}%
\pgfpathlineto{\pgfqpoint{9.101927in}{4.920665in}}%
\pgfpathlineto{\pgfqpoint{9.103940in}{4.929348in}}%
\pgfpathlineto{\pgfqpoint{9.107967in}{4.937806in}}%
\pgfpathlineto{\pgfqpoint{9.114007in}{4.938106in}}%
\pgfpathlineto{\pgfqpoint{9.116021in}{4.943195in}}%
\pgfpathlineto{\pgfqpoint{9.118034in}{4.943794in}}%
\pgfpathlineto{\pgfqpoint{9.120048in}{4.942447in}}%
\pgfpathlineto{\pgfqpoint{9.122061in}{4.937357in}}%
\pgfpathlineto{\pgfqpoint{9.128101in}{4.937357in}}%
\pgfpathlineto{\pgfqpoint{9.130115in}{4.935935in}}%
\pgfpathlineto{\pgfqpoint{9.132128in}{4.933016in}}%
\pgfpathlineto{\pgfqpoint{9.134142in}{4.932791in}}%
\pgfpathlineto{\pgfqpoint{9.136155in}{4.934288in}}%
\pgfpathlineto{\pgfqpoint{9.144209in}{4.922761in}}%
\pgfpathlineto{\pgfqpoint{9.146222in}{4.919094in}}%
\pgfpathlineto{\pgfqpoint{9.148236in}{4.907342in}}%
\pgfpathlineto{\pgfqpoint{9.150249in}{4.905695in}}%
\pgfpathlineto{\pgfqpoint{9.156290in}{4.911908in}}%
\pgfpathlineto{\pgfqpoint{9.158303in}{4.912058in}}%
\pgfpathlineto{\pgfqpoint{9.160317in}{4.909513in}}%
\pgfpathlineto{\pgfqpoint{9.162330in}{4.911983in}}%
\pgfpathlineto{\pgfqpoint{9.164343in}{4.908390in}}%
\pgfpathlineto{\pgfqpoint{9.172397in}{4.913480in}}%
\pgfpathlineto{\pgfqpoint{9.174411in}{4.906070in}}%
\pgfpathlineto{\pgfqpoint{9.176424in}{4.908540in}}%
\pgfpathlineto{\pgfqpoint{9.178438in}{4.892372in}}%
\pgfpathlineto{\pgfqpoint{9.184478in}{4.892896in}}%
\pgfpathlineto{\pgfqpoint{9.186491in}{4.896115in}}%
\pgfpathlineto{\pgfqpoint{9.188505in}{4.905546in}}%
\pgfpathlineto{\pgfqpoint{9.190518in}{4.904573in}}%
\pgfpathlineto{\pgfqpoint{9.192532in}{4.908764in}}%
\pgfpathlineto{\pgfqpoint{9.198572in}{4.903674in}}%
\pgfpathlineto{\pgfqpoint{9.200585in}{4.912956in}}%
\pgfpathlineto{\pgfqpoint{9.202599in}{4.903674in}}%
\pgfpathlineto{\pgfqpoint{9.204612in}{4.903300in}}%
\pgfpathlineto{\pgfqpoint{9.206626in}{4.911609in}}%
\pgfpathlineto{\pgfqpoint{9.212666in}{4.908689in}}%
\pgfpathlineto{\pgfqpoint{9.214680in}{4.916025in}}%
\pgfpathlineto{\pgfqpoint{9.216693in}{4.919393in}}%
\pgfpathlineto{\pgfqpoint{9.218706in}{4.911833in}}%
\pgfpathlineto{\pgfqpoint{9.220720in}{4.915052in}}%
\pgfpathlineto{\pgfqpoint{9.226760in}{4.910037in}}%
\pgfpathlineto{\pgfqpoint{9.228774in}{4.910486in}}%
\pgfpathlineto{\pgfqpoint{9.230787in}{4.915650in}}%
\pgfpathlineto{\pgfqpoint{9.232801in}{4.914378in}}%
\pgfpathlineto{\pgfqpoint{9.234814in}{4.924633in}}%
\pgfpathlineto{\pgfqpoint{9.240854in}{4.930246in}}%
\pgfpathlineto{\pgfqpoint{9.242868in}{4.934064in}}%
\pgfpathlineto{\pgfqpoint{9.246895in}{4.933016in}}%
\pgfpathlineto{\pgfqpoint{9.248908in}{4.928749in}}%
\pgfpathlineto{\pgfqpoint{9.256962in}{4.927177in}}%
\pgfpathlineto{\pgfqpoint{9.258975in}{4.925082in}}%
\pgfpathlineto{\pgfqpoint{9.260989in}{4.917297in}}%
\pgfpathlineto{\pgfqpoint{9.263002in}{4.924034in}}%
\pgfpathlineto{\pgfqpoint{9.269043in}{4.928824in}}%
\pgfpathlineto{\pgfqpoint{9.271056in}{4.929124in}}%
\pgfpathlineto{\pgfqpoint{9.273070in}{4.926504in}}%
\pgfpathlineto{\pgfqpoint{9.275083in}{4.912582in}}%
\pgfpathlineto{\pgfqpoint{9.277096in}{4.909737in}}%
\pgfpathlineto{\pgfqpoint{9.285150in}{4.908465in}}%
\pgfpathlineto{\pgfqpoint{9.287164in}{4.911309in}}%
\pgfpathlineto{\pgfqpoint{9.289177in}{4.922911in}}%
\pgfpathlineto{\pgfqpoint{9.291191in}{4.928076in}}%
\pgfpathlineto{\pgfqpoint{9.297231in}{4.926504in}}%
\pgfpathlineto{\pgfqpoint{9.299244in}{4.922537in}}%
\pgfpathlineto{\pgfqpoint{9.301258in}{4.916399in}}%
\pgfpathlineto{\pgfqpoint{9.303271in}{4.914303in}}%
\pgfpathlineto{\pgfqpoint{9.305285in}{4.921863in}}%
\pgfpathlineto{\pgfqpoint{9.311325in}{4.918420in}}%
\pgfpathlineto{\pgfqpoint{9.313339in}{4.922088in}}%
\pgfpathlineto{\pgfqpoint{9.315352in}{4.923884in}}%
\pgfpathlineto{\pgfqpoint{9.317365in}{4.914153in}}%
\pgfpathlineto{\pgfqpoint{9.319379in}{4.909962in}}%
\pgfpathlineto{\pgfqpoint{9.325419in}{4.911833in}}%
\pgfpathlineto{\pgfqpoint{9.327433in}{4.911534in}}%
\pgfpathlineto{\pgfqpoint{9.331460in}{4.920291in}}%
\pgfpathlineto{\pgfqpoint{9.333473in}{4.921713in}}%
\pgfpathlineto{\pgfqpoint{9.341527in}{4.915052in}}%
\pgfpathlineto{\pgfqpoint{9.343540in}{4.916549in}}%
\pgfpathlineto{\pgfqpoint{9.345554in}{4.913854in}}%
\pgfpathlineto{\pgfqpoint{9.347567in}{4.921040in}}%
\pgfpathlineto{\pgfqpoint{9.353607in}{4.920591in}}%
\pgfpathlineto{\pgfqpoint{9.355621in}{4.921564in}}%
\pgfpathlineto{\pgfqpoint{9.357634in}{4.920591in}}%
\pgfpathlineto{\pgfqpoint{9.359648in}{4.918944in}}%
\pgfpathlineto{\pgfqpoint{9.361661in}{4.926579in}}%
\pgfpathlineto{\pgfqpoint{9.369715in}{4.928674in}}%
\pgfpathlineto{\pgfqpoint{9.371728in}{4.919019in}}%
\pgfpathlineto{\pgfqpoint{9.375755in}{4.922686in}}%
\pgfpathlineto{\pgfqpoint{9.381796in}{4.921264in}}%
\pgfpathlineto{\pgfqpoint{9.383809in}{4.919318in}}%
\pgfpathlineto{\pgfqpoint{9.385823in}{4.919618in}}%
\pgfpathlineto{\pgfqpoint{9.387836in}{4.931594in}}%
\pgfpathlineto{\pgfqpoint{9.389849in}{4.933091in}}%
\pgfpathlineto{\pgfqpoint{9.395890in}{4.932492in}}%
\pgfpathlineto{\pgfqpoint{9.397903in}{4.929049in}}%
\pgfpathlineto{\pgfqpoint{9.399917in}{4.929049in}}%
\pgfpathlineto{\pgfqpoint{9.401930in}{4.926803in}}%
\pgfpathlineto{\pgfqpoint{9.403944in}{4.923585in}}%
\pgfpathlineto{\pgfqpoint{9.409984in}{4.922312in}}%
\pgfpathlineto{\pgfqpoint{9.411997in}{4.918644in}}%
\pgfpathlineto{\pgfqpoint{9.414011in}{4.911908in}}%
\pgfpathlineto{\pgfqpoint{9.418038in}{4.916773in}}%
\pgfpathlineto{\pgfqpoint{9.424078in}{4.922686in}}%
\pgfpathlineto{\pgfqpoint{9.426092in}{4.918944in}}%
\pgfpathlineto{\pgfqpoint{9.428105in}{4.921863in}}%
\pgfpathlineto{\pgfqpoint{9.430118in}{4.928375in}}%
\pgfpathlineto{\pgfqpoint{9.432132in}{4.929722in}}%
\pgfpathlineto{\pgfqpoint{9.438172in}{4.931444in}}%
\pgfpathlineto{\pgfqpoint{9.440186in}{4.927552in}}%
\pgfpathlineto{\pgfqpoint{9.442199in}{4.925680in}}%
\pgfpathlineto{\pgfqpoint{9.444213in}{4.929124in}}%
\pgfpathlineto{\pgfqpoint{9.446226in}{4.925156in}}%
\pgfpathlineto{\pgfqpoint{9.452266in}{4.923136in}}%
\pgfpathlineto{\pgfqpoint{9.454280in}{4.930171in}}%
\pgfpathlineto{\pgfqpoint{9.456293in}{4.935186in}}%
\pgfpathlineto{\pgfqpoint{9.458307in}{4.923285in}}%
\pgfpathlineto{\pgfqpoint{9.460320in}{4.918195in}}%
\pgfpathlineto{\pgfqpoint{9.466360in}{4.917372in}}%
\pgfpathlineto{\pgfqpoint{9.468374in}{4.908091in}}%
\pgfpathlineto{\pgfqpoint{9.470387in}{4.907118in}}%
\pgfpathlineto{\pgfqpoint{9.474414in}{4.910635in}}%
\pgfpathlineto{\pgfqpoint{9.484482in}{4.912133in}}%
\pgfpathlineto{\pgfqpoint{9.486495in}{4.918121in}}%
\pgfpathlineto{\pgfqpoint{9.494549in}{4.914752in}}%
\pgfpathlineto{\pgfqpoint{9.496562in}{4.920366in}}%
\pgfpathlineto{\pgfqpoint{9.498576in}{4.909438in}}%
\pgfpathlineto{\pgfqpoint{9.500589in}{4.909363in}}%
\pgfpathlineto{\pgfqpoint{9.502603in}{4.911159in}}%
\pgfpathlineto{\pgfqpoint{9.508643in}{4.909064in}}%
\pgfpathlineto{\pgfqpoint{9.510656in}{4.903749in}}%
\pgfpathlineto{\pgfqpoint{9.512670in}{4.896339in}}%
\pgfpathlineto{\pgfqpoint{9.514683in}{4.896264in}}%
\pgfpathlineto{\pgfqpoint{9.516697in}{4.900755in}}%
\pgfpathlineto{\pgfqpoint{9.522737in}{4.907118in}}%
\pgfpathlineto{\pgfqpoint{9.524750in}{4.911384in}}%
\pgfpathlineto{\pgfqpoint{9.526764in}{4.912207in}}%
\pgfpathlineto{\pgfqpoint{9.530791in}{4.915800in}}%
\pgfpathlineto{\pgfqpoint{9.536831in}{4.910186in}}%
\pgfpathlineto{\pgfqpoint{9.538845in}{4.901728in}}%
\pgfpathlineto{\pgfqpoint{9.540858in}{4.908989in}}%
\pgfpathlineto{\pgfqpoint{9.542871in}{4.911758in}}%
\pgfpathlineto{\pgfqpoint{9.544885in}{4.911609in}}%
\pgfpathlineto{\pgfqpoint{9.550925in}{4.912282in}}%
\pgfpathlineto{\pgfqpoint{9.552939in}{4.911234in}}%
\pgfpathlineto{\pgfqpoint{9.554952in}{4.915052in}}%
\pgfpathlineto{\pgfqpoint{9.556966in}{4.913106in}}%
\pgfpathlineto{\pgfqpoint{9.558979in}{4.916474in}}%
\pgfpathlineto{\pgfqpoint{9.565019in}{4.917447in}}%
\pgfpathlineto{\pgfqpoint{9.567033in}{4.919543in}}%
\pgfpathlineto{\pgfqpoint{9.569046in}{4.923659in}}%
\pgfpathlineto{\pgfqpoint{9.571060in}{4.924034in}}%
\pgfpathlineto{\pgfqpoint{9.573073in}{4.916698in}}%
\pgfpathlineto{\pgfqpoint{9.579114in}{4.920965in}}%
\pgfpathlineto{\pgfqpoint{9.581127in}{4.924707in}}%
\pgfpathlineto{\pgfqpoint{9.583140in}{4.917671in}}%
\pgfpathlineto{\pgfqpoint{9.585154in}{4.921863in}}%
\pgfpathlineto{\pgfqpoint{9.587167in}{4.923510in}}%
\pgfpathlineto{\pgfqpoint{9.593208in}{4.922686in}}%
\pgfpathlineto{\pgfqpoint{9.597235in}{4.919992in}}%
\pgfpathlineto{\pgfqpoint{9.599248in}{4.916998in}}%
\pgfpathlineto{\pgfqpoint{9.601261in}{4.916848in}}%
\pgfpathlineto{\pgfqpoint{9.607302in}{4.922387in}}%
\pgfpathlineto{\pgfqpoint{9.611329in}{4.932417in}}%
\pgfpathlineto{\pgfqpoint{9.613342in}{4.934064in}}%
\pgfpathlineto{\pgfqpoint{9.615356in}{4.938255in}}%
\pgfpathlineto{\pgfqpoint{9.621396in}{4.939977in}}%
\pgfpathlineto{\pgfqpoint{9.623409in}{4.942297in}}%
\pgfpathlineto{\pgfqpoint{9.625423in}{4.942671in}}%
\pgfpathlineto{\pgfqpoint{9.629450in}{4.949333in}}%
\pgfpathlineto{\pgfqpoint{9.635490in}{4.949109in}}%
\pgfpathlineto{\pgfqpoint{9.637504in}{4.946115in}}%
\pgfpathlineto{\pgfqpoint{9.639517in}{4.945142in}}%
\pgfpathlineto{\pgfqpoint{9.641530in}{4.945965in}}%
\pgfpathlineto{\pgfqpoint{9.643544in}{4.939378in}}%
\pgfpathlineto{\pgfqpoint{9.651598in}{4.936833in}}%
\pgfpathlineto{\pgfqpoint{9.653611in}{4.933615in}}%
\pgfpathlineto{\pgfqpoint{9.655625in}{4.936159in}}%
\pgfpathlineto{\pgfqpoint{9.657638in}{4.937507in}}%
\pgfpathlineto{\pgfqpoint{9.663678in}{4.934588in}}%
\pgfpathlineto{\pgfqpoint{9.665692in}{4.937133in}}%
\pgfpathlineto{\pgfqpoint{9.667705in}{4.936309in}}%
\pgfpathlineto{\pgfqpoint{9.669719in}{4.932791in}}%
\pgfpathlineto{\pgfqpoint{9.671732in}{4.940276in}}%
\pgfpathlineto{\pgfqpoint{9.677772in}{4.936908in}}%
\pgfpathlineto{\pgfqpoint{9.679786in}{4.945366in}}%
\pgfpathlineto{\pgfqpoint{9.681799in}{4.944692in}}%
\pgfpathlineto{\pgfqpoint{9.683813in}{4.955471in}}%
\pgfpathlineto{\pgfqpoint{9.685826in}{4.953151in}}%
\pgfpathlineto{\pgfqpoint{9.691867in}{4.954423in}}%
\pgfpathlineto{\pgfqpoint{9.693880in}{4.955920in}}%
\pgfpathlineto{\pgfqpoint{9.695893in}{4.955172in}}%
\pgfpathlineto{\pgfqpoint{9.697907in}{4.956818in}}%
\pgfpathlineto{\pgfqpoint{9.699920in}{4.952402in}}%
\pgfpathlineto{\pgfqpoint{9.707974in}{4.956444in}}%
\pgfpathlineto{\pgfqpoint{9.712001in}{4.951429in}}%
\pgfpathlineto{\pgfqpoint{9.714015in}{4.958390in}}%
\pgfpathlineto{\pgfqpoint{9.722068in}{4.953974in}}%
\pgfpathlineto{\pgfqpoint{9.724082in}{4.957941in}}%
\pgfpathlineto{\pgfqpoint{9.726095in}{4.955546in}}%
\pgfpathlineto{\pgfqpoint{9.728109in}{4.957043in}}%
\pgfpathlineto{\pgfqpoint{9.734149in}{4.960561in}}%
\pgfpathlineto{\pgfqpoint{9.736162in}{4.968944in}}%
\pgfpathlineto{\pgfqpoint{9.738176in}{4.972612in}}%
\pgfpathlineto{\pgfqpoint{9.740189in}{4.971938in}}%
\pgfpathlineto{\pgfqpoint{9.742203in}{4.972911in}}%
\pgfpathlineto{\pgfqpoint{9.748243in}{4.978375in}}%
\pgfpathlineto{\pgfqpoint{9.750257in}{4.976953in}}%
\pgfpathlineto{\pgfqpoint{9.754283in}{4.977552in}}%
\pgfpathlineto{\pgfqpoint{9.756297in}{4.982417in}}%
\pgfpathlineto{\pgfqpoint{9.762337in}{4.980246in}}%
\pgfpathlineto{\pgfqpoint{9.764351in}{4.975830in}}%
\pgfpathlineto{\pgfqpoint{9.766364in}{4.981743in}}%
\pgfpathlineto{\pgfqpoint{9.768378in}{4.977776in}}%
\pgfpathlineto{\pgfqpoint{9.770391in}{4.983240in}}%
\pgfpathlineto{\pgfqpoint{9.776431in}{4.982492in}}%
\pgfpathlineto{\pgfqpoint{9.778445in}{4.989752in}}%
\pgfpathlineto{\pgfqpoint{9.780458in}{4.989603in}}%
\pgfpathlineto{\pgfqpoint{9.782472in}{4.991699in}}%
\pgfpathlineto{\pgfqpoint{9.790525in}{4.990351in}}%
\pgfpathlineto{\pgfqpoint{9.792539in}{4.992971in}}%
\pgfpathlineto{\pgfqpoint{9.794552in}{4.984812in}}%
\pgfpathlineto{\pgfqpoint{9.796566in}{4.988106in}}%
\pgfpathlineto{\pgfqpoint{9.798579in}{4.978300in}}%
\pgfpathlineto{\pgfqpoint{9.804620in}{4.980396in}}%
\pgfpathlineto{\pgfqpoint{9.806633in}{4.977776in}}%
\pgfpathlineto{\pgfqpoint{9.808647in}{4.979049in}}%
\pgfpathlineto{\pgfqpoint{9.810660in}{4.981369in}}%
\pgfpathlineto{\pgfqpoint{9.812673in}{4.980920in}}%
\pgfpathlineto{\pgfqpoint{9.818714in}{4.970142in}}%
\pgfpathlineto{\pgfqpoint{9.820727in}{4.973510in}}%
\pgfpathlineto{\pgfqpoint{9.822741in}{4.970366in}}%
\pgfpathlineto{\pgfqpoint{9.824754in}{4.976579in}}%
\pgfpathlineto{\pgfqpoint{9.826768in}{4.991549in}}%
\pgfpathlineto{\pgfqpoint{9.832808in}{4.987657in}}%
\pgfpathlineto{\pgfqpoint{9.834821in}{4.993046in}}%
\pgfpathlineto{\pgfqpoint{9.836835in}{4.992821in}}%
\pgfpathlineto{\pgfqpoint{9.838848in}{4.997687in}}%
\pgfpathlineto{\pgfqpoint{9.840862in}{4.995067in}}%
\pgfpathlineto{\pgfqpoint{9.846902in}{4.994243in}}%
\pgfpathlineto{\pgfqpoint{9.848915in}{4.999558in}}%
\pgfpathlineto{\pgfqpoint{9.850929in}{4.998660in}}%
\pgfpathlineto{\pgfqpoint{9.854956in}{5.011983in}}%
\pgfpathlineto{\pgfqpoint{9.860996in}{5.010860in}}%
\pgfpathlineto{\pgfqpoint{9.863010in}{5.012058in}}%
\pgfpathlineto{\pgfqpoint{9.865023in}{5.012582in}}%
\pgfpathlineto{\pgfqpoint{9.875090in}{5.008989in}}%
\pgfpathlineto{\pgfqpoint{9.879117in}{5.029423in}}%
\pgfpathlineto{\pgfqpoint{9.881131in}{5.025681in}}%
\pgfpathlineto{\pgfqpoint{9.883144in}{5.033166in}}%
\pgfpathlineto{\pgfqpoint{9.889184in}{5.040501in}}%
\pgfpathlineto{\pgfqpoint{9.891198in}{5.045366in}}%
\pgfpathlineto{\pgfqpoint{9.893211in}{5.040800in}}%
\pgfpathlineto{\pgfqpoint{9.895225in}{5.042522in}}%
\pgfpathlineto{\pgfqpoint{9.897238in}{5.046489in}}%
\pgfpathlineto{\pgfqpoint{9.905292in}{5.052552in}}%
\pgfpathlineto{\pgfqpoint{9.907305in}{5.050306in}}%
\pgfpathlineto{\pgfqpoint{9.909319in}{5.052402in}}%
\pgfpathlineto{\pgfqpoint{9.911332in}{5.049483in}}%
\pgfpathlineto{\pgfqpoint{9.917373in}{5.054648in}}%
\pgfpathlineto{\pgfqpoint{9.919386in}{5.051878in}}%
\pgfpathlineto{\pgfqpoint{9.921400in}{5.042971in}}%
\pgfpathlineto{\pgfqpoint{9.925426in}{5.065351in}}%
\pgfpathlineto{\pgfqpoint{9.931467in}{5.066998in}}%
\pgfpathlineto{\pgfqpoint{9.935494in}{5.042672in}}%
\pgfpathlineto{\pgfqpoint{9.937507in}{5.046040in}}%
\pgfpathlineto{\pgfqpoint{9.939521in}{5.030770in}}%
\pgfpathlineto{\pgfqpoint{9.945561in}{5.036684in}}%
\pgfpathlineto{\pgfqpoint{9.947574in}{5.044468in}}%
\pgfpathlineto{\pgfqpoint{9.949588in}{5.039378in}}%
\pgfpathlineto{\pgfqpoint{9.951601in}{5.030396in}}%
\pgfpathlineto{\pgfqpoint{9.953615in}{5.033091in}}%
\pgfpathlineto{\pgfqpoint{9.959655in}{5.024184in}}%
\pgfpathlineto{\pgfqpoint{9.961669in}{5.032492in}}%
\pgfpathlineto{\pgfqpoint{9.963682in}{5.037058in}}%
\pgfpathlineto{\pgfqpoint{9.965695in}{5.045441in}}%
\pgfpathlineto{\pgfqpoint{9.967709in}{5.042971in}}%
\pgfpathlineto{\pgfqpoint{9.973749in}{5.048959in}}%
\pgfpathlineto{\pgfqpoint{9.975763in}{5.043570in}}%
\pgfpathlineto{\pgfqpoint{9.977776in}{5.043270in}}%
\pgfpathlineto{\pgfqpoint{9.981803in}{5.055172in}}%
\pgfpathlineto{\pgfqpoint{9.987843in}{5.060187in}}%
\pgfpathlineto{\pgfqpoint{9.989857in}{5.064079in}}%
\pgfpathlineto{\pgfqpoint{9.991870in}{5.055321in}}%
\pgfpathlineto{\pgfqpoint{9.993884in}{5.059438in}}%
\pgfpathlineto{\pgfqpoint{9.995897in}{5.069019in}}%
\pgfpathlineto{\pgfqpoint{10.001937in}{5.065726in}}%
\pgfpathlineto{\pgfqpoint{10.003951in}{5.068570in}}%
\pgfpathlineto{\pgfqpoint{10.005964in}{5.058540in}}%
\pgfpathlineto{\pgfqpoint{10.007978in}{5.039378in}}%
\pgfpathlineto{\pgfqpoint{10.009991in}{5.039753in}}%
\pgfpathlineto{\pgfqpoint{10.016032in}{5.044244in}}%
\pgfpathlineto{\pgfqpoint{10.018045in}{5.042148in}}%
\pgfpathlineto{\pgfqpoint{10.020058in}{5.048660in}}%
\pgfpathlineto{\pgfqpoint{10.022072in}{5.051429in}}%
\pgfpathlineto{\pgfqpoint{10.024085in}{5.048510in}}%
\pgfpathlineto{\pgfqpoint{10.030126in}{5.046564in}}%
\pgfpathlineto{\pgfqpoint{10.032139in}{5.047462in}}%
\pgfpathlineto{\pgfqpoint{10.034153in}{5.037732in}}%
\pgfpathlineto{\pgfqpoint{10.036166in}{5.050456in}}%
\pgfpathlineto{\pgfqpoint{10.044220in}{5.052552in}}%
\pgfpathlineto{\pgfqpoint{10.046233in}{5.051729in}}%
\pgfpathlineto{\pgfqpoint{10.048247in}{5.047163in}}%
\pgfpathlineto{\pgfqpoint{10.050260in}{5.054348in}}%
\pgfpathlineto{\pgfqpoint{10.052274in}{5.049633in}}%
\pgfpathlineto{\pgfqpoint{10.058314in}{5.049558in}}%
\pgfpathlineto{\pgfqpoint{10.060327in}{5.054573in}}%
\pgfpathlineto{\pgfqpoint{10.062341in}{5.052178in}}%
\pgfpathlineto{\pgfqpoint{10.064354in}{5.045366in}}%
\pgfpathlineto{\pgfqpoint{10.066368in}{5.047312in}}%
\pgfpathlineto{\pgfqpoint{10.072408in}{5.041624in}}%
\pgfpathlineto{\pgfqpoint{10.074422in}{5.041175in}}%
\pgfpathlineto{\pgfqpoint{10.076435in}{5.035486in}}%
\pgfpathlineto{\pgfqpoint{10.078448in}{5.038330in}}%
\pgfpathlineto{\pgfqpoint{10.080462in}{5.036908in}}%
\pgfpathlineto{\pgfqpoint{10.086502in}{5.036684in}}%
\pgfpathlineto{\pgfqpoint{10.088516in}{5.025007in}}%
\pgfpathlineto{\pgfqpoint{10.090529in}{5.025681in}}%
\pgfpathlineto{\pgfqpoint{10.092543in}{5.027103in}}%
\pgfpathlineto{\pgfqpoint{10.094556in}{5.025007in}}%
\pgfpathlineto{\pgfqpoint{10.102610in}{5.028525in}}%
\pgfpathlineto{\pgfqpoint{10.106637in}{5.037956in}}%
\pgfpathlineto{\pgfqpoint{10.108650in}{5.034887in}}%
\pgfpathlineto{\pgfqpoint{10.114691in}{5.036983in}}%
\pgfpathlineto{\pgfqpoint{10.116704in}{5.043420in}}%
\pgfpathlineto{\pgfqpoint{10.118717in}{5.047387in}}%
\pgfpathlineto{\pgfqpoint{10.120731in}{5.048136in}}%
\pgfpathlineto{\pgfqpoint{10.122744in}{5.048211in}}%
\pgfpathlineto{\pgfqpoint{10.128785in}{5.050306in}}%
\pgfpathlineto{\pgfqpoint{10.130798in}{5.059363in}}%
\pgfpathlineto{\pgfqpoint{10.132812in}{5.064303in}}%
\pgfpathlineto{\pgfqpoint{10.134825in}{5.063780in}}%
\pgfpathlineto{\pgfqpoint{10.136838in}{5.062058in}}%
\pgfpathlineto{\pgfqpoint{10.144892in}{5.053525in}}%
\pgfpathlineto{\pgfqpoint{10.146906in}{5.052926in}}%
\pgfpathlineto{\pgfqpoint{10.148919in}{5.055247in}}%
\pgfpathlineto{\pgfqpoint{10.150933in}{5.052253in}}%
\pgfpathlineto{\pgfqpoint{10.156973in}{5.049708in}}%
\pgfpathlineto{\pgfqpoint{10.158986in}{5.052552in}}%
\pgfpathlineto{\pgfqpoint{10.161000in}{5.044842in}}%
\pgfpathlineto{\pgfqpoint{10.163013in}{5.041025in}}%
\pgfpathlineto{\pgfqpoint{10.165027in}{5.042971in}}%
\pgfpathlineto{\pgfqpoint{10.171067in}{5.033540in}}%
\pgfpathlineto{\pgfqpoint{10.173080in}{5.028600in}}%
\pgfpathlineto{\pgfqpoint{10.175094in}{5.028600in}}%
\pgfpathlineto{\pgfqpoint{10.177107in}{5.045366in}}%
\pgfpathlineto{\pgfqpoint{10.179121in}{5.050456in}}%
\pgfpathlineto{\pgfqpoint{10.185161in}{5.055097in}}%
\pgfpathlineto{\pgfqpoint{10.187175in}{5.049483in}}%
\pgfpathlineto{\pgfqpoint{10.189188in}{5.056818in}}%
\pgfpathlineto{\pgfqpoint{10.191201in}{5.083690in}}%
\pgfpathlineto{\pgfqpoint{10.193215in}{5.085711in}}%
\pgfpathlineto{\pgfqpoint{10.199255in}{5.084887in}}%
\pgfpathlineto{\pgfqpoint{10.201269in}{5.088106in}}%
\pgfpathlineto{\pgfqpoint{10.203282in}{5.086459in}}%
\pgfpathlineto{\pgfqpoint{10.205296in}{5.088330in}}%
\pgfpathlineto{\pgfqpoint{10.207309in}{5.100082in}}%
\pgfpathlineto{\pgfqpoint{10.213349in}{5.101055in}}%
\pgfpathlineto{\pgfqpoint{10.215363in}{5.107268in}}%
\pgfpathlineto{\pgfqpoint{10.217376in}{5.103450in}}%
\pgfpathlineto{\pgfqpoint{10.219390in}{5.094468in}}%
\pgfpathlineto{\pgfqpoint{10.221403in}{5.096938in}}%
\pgfpathlineto{\pgfqpoint{10.229457in}{5.095142in}}%
\pgfpathlineto{\pgfqpoint{10.231470in}{5.096714in}}%
\pgfpathlineto{\pgfqpoint{10.233484in}{5.088405in}}%
\pgfpathlineto{\pgfqpoint{10.235497in}{5.094318in}}%
\pgfpathlineto{\pgfqpoint{10.241538in}{5.091774in}}%
\pgfpathlineto{\pgfqpoint{10.243551in}{5.089229in}}%
\pgfpathlineto{\pgfqpoint{10.247578in}{5.095142in}}%
\pgfpathlineto{\pgfqpoint{10.249591in}{5.101354in}}%
\pgfpathlineto{\pgfqpoint{10.255632in}{5.098211in}}%
\pgfpathlineto{\pgfqpoint{10.257645in}{5.099034in}}%
\pgfpathlineto{\pgfqpoint{10.259659in}{5.097612in}}%
\pgfpathlineto{\pgfqpoint{10.261672in}{5.109289in}}%
\pgfpathlineto{\pgfqpoint{10.263686in}{5.108615in}}%
\pgfpathlineto{\pgfqpoint{10.269726in}{5.113705in}}%
\pgfpathlineto{\pgfqpoint{10.273753in}{5.120516in}}%
\pgfpathlineto{\pgfqpoint{10.277780in}{5.122013in}}%
\pgfpathlineto{\pgfqpoint{10.283820in}{5.118495in}}%
\pgfpathlineto{\pgfqpoint{10.285834in}{5.113780in}}%
\pgfpathlineto{\pgfqpoint{10.287847in}{5.113031in}}%
\pgfpathlineto{\pgfqpoint{10.289860in}{5.113331in}}%
\pgfpathlineto{\pgfqpoint{10.291874in}{5.122911in}}%
\pgfpathlineto{\pgfqpoint{10.297914in}{5.121639in}}%
\pgfpathlineto{\pgfqpoint{10.299928in}{5.118495in}}%
\pgfpathlineto{\pgfqpoint{10.301941in}{5.108914in}}%
\pgfpathlineto{\pgfqpoint{10.303955in}{5.104798in}}%
\pgfpathlineto{\pgfqpoint{10.305968in}{5.107417in}}%
\pgfpathlineto{\pgfqpoint{10.312008in}{5.113181in}}%
\pgfpathlineto{\pgfqpoint{10.314022in}{5.110336in}}%
\pgfpathlineto{\pgfqpoint{10.316035in}{5.123286in}}%
\pgfpathlineto{\pgfqpoint{10.318049in}{5.126205in}}%
\pgfpathlineto{\pgfqpoint{10.320062in}{5.134139in}}%
\pgfpathlineto{\pgfqpoint{10.326102in}{5.139229in}}%
\pgfpathlineto{\pgfqpoint{10.328116in}{5.141849in}}%
\pgfpathlineto{\pgfqpoint{10.340197in}{5.148061in}}%
\pgfpathlineto{\pgfqpoint{10.342210in}{5.155846in}}%
\pgfpathlineto{\pgfqpoint{10.346237in}{5.147088in}}%
\pgfpathlineto{\pgfqpoint{10.348250in}{5.149109in}}%
\pgfpathlineto{\pgfqpoint{10.354291in}{5.148810in}}%
\pgfpathlineto{\pgfqpoint{10.356304in}{5.146489in}}%
\pgfpathlineto{\pgfqpoint{10.358318in}{5.148361in}}%
\pgfpathlineto{\pgfqpoint{10.360331in}{5.144768in}}%
\pgfpathlineto{\pgfqpoint{10.362345in}{5.142447in}}%
\pgfpathlineto{\pgfqpoint{10.368385in}{5.131744in}}%
\pgfpathlineto{\pgfqpoint{10.370398in}{5.132642in}}%
\pgfpathlineto{\pgfqpoint{10.372412in}{5.140426in}}%
\pgfpathlineto{\pgfqpoint{10.374425in}{5.137133in}}%
\pgfpathlineto{\pgfqpoint{10.376439in}{5.159139in}}%
\pgfpathlineto{\pgfqpoint{10.384492in}{5.156519in}}%
\pgfpathlineto{\pgfqpoint{10.386506in}{5.160262in}}%
\pgfpathlineto{\pgfqpoint{10.390533in}{5.131444in}}%
\pgfpathlineto{\pgfqpoint{10.396573in}{5.122986in}}%
\pgfpathlineto{\pgfqpoint{10.398587in}{5.129274in}}%
\pgfpathlineto{\pgfqpoint{10.400600in}{5.121789in}}%
\pgfpathlineto{\pgfqpoint{10.402613in}{5.129124in}}%
\pgfpathlineto{\pgfqpoint{10.404627in}{5.118121in}}%
\pgfpathlineto{\pgfqpoint{10.410667in}{5.103226in}}%
\pgfpathlineto{\pgfqpoint{10.412681in}{5.111085in}}%
\pgfpathlineto{\pgfqpoint{10.414694in}{5.109139in}}%
\pgfpathlineto{\pgfqpoint{10.416708in}{5.122762in}}%
\pgfpathlineto{\pgfqpoint{10.418721in}{5.131744in}}%
\pgfpathlineto{\pgfqpoint{10.424761in}{5.141025in}}%
\pgfpathlineto{\pgfqpoint{10.426775in}{5.142672in}}%
\pgfpathlineto{\pgfqpoint{10.428788in}{5.145666in}}%
\pgfpathlineto{\pgfqpoint{10.430802in}{5.144843in}}%
\pgfpathlineto{\pgfqpoint{10.432815in}{5.145591in}}%
\pgfpathlineto{\pgfqpoint{10.440869in}{5.145666in}}%
\pgfpathlineto{\pgfqpoint{10.442882in}{5.144468in}}%
\pgfpathlineto{\pgfqpoint{10.444896in}{5.145666in}}%
\pgfpathlineto{\pgfqpoint{10.446909in}{5.143869in}}%
\pgfpathlineto{\pgfqpoint{10.452950in}{5.151878in}}%
\pgfpathlineto{\pgfqpoint{10.454963in}{5.151804in}}%
\pgfpathlineto{\pgfqpoint{10.456977in}{5.150606in}}%
\pgfpathlineto{\pgfqpoint{10.458990in}{5.154423in}}%
\pgfpathlineto{\pgfqpoint{10.461003in}{5.161310in}}%
\pgfpathlineto{\pgfqpoint{10.467044in}{5.152253in}}%
\pgfpathlineto{\pgfqpoint{10.469057in}{5.170441in}}%
\pgfpathlineto{\pgfqpoint{10.471071in}{5.167223in}}%
\pgfpathlineto{\pgfqpoint{10.473084in}{5.176729in}}%
\pgfpathlineto{\pgfqpoint{10.475098in}{5.179049in}}%
\pgfpathlineto{\pgfqpoint{10.481138in}{5.177926in}}%
\pgfpathlineto{\pgfqpoint{10.483151in}{5.174483in}}%
\pgfpathlineto{\pgfqpoint{10.485165in}{5.172462in}}%
\pgfpathlineto{\pgfqpoint{10.487178in}{5.157193in}}%
\pgfpathlineto{\pgfqpoint{10.489192in}{5.153600in}}%
\pgfpathlineto{\pgfqpoint{10.497245in}{5.163405in}}%
\pgfpathlineto{\pgfqpoint{10.499259in}{5.157492in}}%
\pgfpathlineto{\pgfqpoint{10.501272in}{5.164079in}}%
\pgfpathlineto{\pgfqpoint{10.511340in}{5.158241in}}%
\pgfpathlineto{\pgfqpoint{10.513353in}{5.150082in}}%
\pgfpathlineto{\pgfqpoint{10.515366in}{5.152402in}}%
\pgfpathlineto{\pgfqpoint{10.517380in}{5.155621in}}%
\pgfpathlineto{\pgfqpoint{10.523420in}{5.152627in}}%
\pgfpathlineto{\pgfqpoint{10.525434in}{5.160486in}}%
\pgfpathlineto{\pgfqpoint{10.527447in}{5.156819in}}%
\pgfpathlineto{\pgfqpoint{10.529461in}{5.160786in}}%
\pgfpathlineto{\pgfqpoint{10.531474in}{5.148286in}}%
\pgfpathlineto{\pgfqpoint{10.537514in}{5.130546in}}%
\pgfpathlineto{\pgfqpoint{10.539528in}{5.129798in}}%
\pgfpathlineto{\pgfqpoint{10.541541in}{5.145217in}}%
\pgfpathlineto{\pgfqpoint{10.543555in}{5.121938in}}%
\pgfpathlineto{\pgfqpoint{10.545568in}{5.116325in}}%
\pgfpathlineto{\pgfqpoint{10.551609in}{5.122911in}}%
\pgfpathlineto{\pgfqpoint{10.553622in}{5.126654in}}%
\pgfpathlineto{\pgfqpoint{10.555635in}{5.136085in}}%
\pgfpathlineto{\pgfqpoint{10.557649in}{5.127926in}}%
\pgfpathlineto{\pgfqpoint{10.565703in}{5.130995in}}%
\pgfpathlineto{\pgfqpoint{10.567716in}{5.133765in}}%
\pgfpathlineto{\pgfqpoint{10.569730in}{5.134214in}}%
\pgfpathlineto{\pgfqpoint{10.571743in}{5.136160in}}%
\pgfpathlineto{\pgfqpoint{10.573756in}{5.133540in}}%
\pgfpathlineto{\pgfqpoint{10.579797in}{5.133690in}}%
\pgfpathlineto{\pgfqpoint{10.581810in}{5.138780in}}%
\pgfpathlineto{\pgfqpoint{10.583824in}{5.136235in}}%
\pgfpathlineto{\pgfqpoint{10.585837in}{5.132043in}}%
\pgfpathlineto{\pgfqpoint{10.587851in}{5.132792in}}%
\pgfpathlineto{\pgfqpoint{10.593891in}{5.135786in}}%
\pgfpathlineto{\pgfqpoint{10.595904in}{5.127777in}}%
\pgfpathlineto{\pgfqpoint{10.597918in}{5.140052in}}%
\pgfpathlineto{\pgfqpoint{10.599931in}{5.144468in}}%
\pgfpathlineto{\pgfqpoint{10.601945in}{5.146040in}}%
\pgfpathlineto{\pgfqpoint{10.607985in}{5.151654in}}%
\pgfpathlineto{\pgfqpoint{10.612012in}{5.143420in}}%
\pgfpathlineto{\pgfqpoint{10.614025in}{5.137432in}}%
\pgfpathlineto{\pgfqpoint{10.616039in}{5.136759in}}%
\pgfpathlineto{\pgfqpoint{10.622079in}{5.141025in}}%
\pgfpathlineto{\pgfqpoint{10.624093in}{5.133914in}}%
\pgfpathlineto{\pgfqpoint{10.626106in}{5.139229in}}%
\pgfpathlineto{\pgfqpoint{10.628120in}{5.141250in}}%
\pgfpathlineto{\pgfqpoint{10.638187in}{5.163630in}}%
\pgfpathlineto{\pgfqpoint{10.640200in}{5.161384in}}%
\pgfpathlineto{\pgfqpoint{10.644227in}{5.164379in}}%
\pgfpathlineto{\pgfqpoint{10.650267in}{5.167073in}}%
\pgfpathlineto{\pgfqpoint{10.652281in}{5.165950in}}%
\pgfpathlineto{\pgfqpoint{10.654294in}{5.166549in}}%
\pgfpathlineto{\pgfqpoint{10.656308in}{5.173361in}}%
\pgfpathlineto{\pgfqpoint{10.658321in}{5.187956in}}%
\pgfpathlineto{\pgfqpoint{10.664362in}{5.192522in}}%
\pgfpathlineto{\pgfqpoint{10.670402in}{5.186908in}}%
\pgfpathlineto{\pgfqpoint{10.672415in}{5.187507in}}%
\pgfpathlineto{\pgfqpoint{10.678456in}{5.184214in}}%
\pgfpathlineto{\pgfqpoint{10.680469in}{5.186235in}}%
\pgfpathlineto{\pgfqpoint{10.682483in}{5.192447in}}%
\pgfpathlineto{\pgfqpoint{10.684496in}{5.189079in}}%
\pgfpathlineto{\pgfqpoint{10.686510in}{5.192298in}}%
\pgfpathlineto{\pgfqpoint{10.692550in}{5.192073in}}%
\pgfpathlineto{\pgfqpoint{10.694563in}{5.184663in}}%
\pgfpathlineto{\pgfqpoint{10.696577in}{5.186085in}}%
\pgfpathlineto{\pgfqpoint{10.698590in}{5.183765in}}%
\pgfpathlineto{\pgfqpoint{10.700604in}{5.188106in}}%
\pgfpathlineto{\pgfqpoint{10.706644in}{5.187657in}}%
\pgfpathlineto{\pgfqpoint{10.708657in}{5.191025in}}%
\pgfpathlineto{\pgfqpoint{10.710671in}{5.190052in}}%
\pgfpathlineto{\pgfqpoint{10.712684in}{5.194394in}}%
\pgfpathlineto{\pgfqpoint{10.720738in}{5.191325in}}%
\pgfpathlineto{\pgfqpoint{10.722752in}{5.185636in}}%
\pgfpathlineto{\pgfqpoint{10.724765in}{5.188555in}}%
\pgfpathlineto{\pgfqpoint{10.726778in}{5.186609in}}%
\pgfpathlineto{\pgfqpoint{10.738859in}{5.186684in}}%
\pgfpathlineto{\pgfqpoint{10.740873in}{5.176429in}}%
\pgfpathlineto{\pgfqpoint{10.742886in}{5.180172in}}%
\pgfpathlineto{\pgfqpoint{10.748926in}{5.175681in}}%
\pgfpathlineto{\pgfqpoint{10.750940in}{5.179124in}}%
\pgfpathlineto{\pgfqpoint{10.754967in}{5.177552in}}%
\pgfpathlineto{\pgfqpoint{10.756980in}{5.169094in}}%
\pgfpathlineto{\pgfqpoint{10.763021in}{5.168645in}}%
\pgfpathlineto{\pgfqpoint{10.765034in}{5.167597in}}%
\pgfpathlineto{\pgfqpoint{10.767047in}{5.162058in}}%
\pgfpathlineto{\pgfqpoint{10.769061in}{5.141849in}}%
\pgfpathlineto{\pgfqpoint{10.771074in}{5.131295in}}%
\pgfpathlineto{\pgfqpoint{10.777115in}{5.134438in}}%
\pgfpathlineto{\pgfqpoint{10.779128in}{5.130546in}}%
\pgfpathlineto{\pgfqpoint{10.781142in}{5.130771in}}%
\pgfpathlineto{\pgfqpoint{10.783155in}{5.128151in}}%
\pgfpathlineto{\pgfqpoint{10.785168in}{5.138181in}}%
\pgfpathlineto{\pgfqpoint{10.791209in}{5.134588in}}%
\pgfpathlineto{\pgfqpoint{10.793222in}{5.135486in}}%
\pgfpathlineto{\pgfqpoint{10.795236in}{5.137732in}}%
\pgfpathlineto{\pgfqpoint{10.797249in}{5.136834in}}%
\pgfpathlineto{\pgfqpoint{10.799263in}{5.132193in}}%
\pgfpathlineto{\pgfqpoint{10.805303in}{5.136085in}}%
\pgfpathlineto{\pgfqpoint{10.807316in}{5.142747in}}%
\pgfpathlineto{\pgfqpoint{10.809330in}{5.145292in}}%
\pgfpathlineto{\pgfqpoint{10.811343in}{5.149783in}}%
\pgfpathlineto{\pgfqpoint{10.813357in}{5.147986in}}%
\pgfpathlineto{\pgfqpoint{10.819397in}{5.152926in}}%
\pgfpathlineto{\pgfqpoint{10.821410in}{5.149858in}}%
\pgfpathlineto{\pgfqpoint{10.823424in}{5.150456in}}%
\pgfpathlineto{\pgfqpoint{10.825437in}{5.148959in}}%
\pgfpathlineto{\pgfqpoint{10.827451in}{5.152627in}}%
\pgfpathlineto{\pgfqpoint{10.835505in}{5.153825in}}%
\pgfpathlineto{\pgfqpoint{10.837518in}{5.156744in}}%
\pgfpathlineto{\pgfqpoint{10.839532in}{5.153301in}}%
\pgfpathlineto{\pgfqpoint{10.841545in}{5.153001in}}%
\pgfpathlineto{\pgfqpoint{10.847585in}{5.148361in}}%
\pgfpathlineto{\pgfqpoint{10.849599in}{5.141175in}}%
\pgfpathlineto{\pgfqpoint{10.851612in}{5.144693in}}%
\pgfpathlineto{\pgfqpoint{10.853626in}{5.144768in}}%
\pgfpathlineto{\pgfqpoint{10.855639in}{5.139378in}}%
\pgfpathlineto{\pgfqpoint{10.861679in}{5.137582in}}%
\pgfpathlineto{\pgfqpoint{10.867720in}{5.156444in}}%
\pgfpathlineto{\pgfqpoint{10.869733in}{5.153675in}}%
\pgfpathlineto{\pgfqpoint{10.875774in}{5.148735in}}%
\pgfpathlineto{\pgfqpoint{10.877787in}{5.144019in}}%
\pgfpathlineto{\pgfqpoint{10.879800in}{5.145516in}}%
\pgfpathlineto{\pgfqpoint{10.881814in}{5.133316in}}%
\pgfpathlineto{\pgfqpoint{10.883827in}{5.144468in}}%
\pgfpathlineto{\pgfqpoint{10.889868in}{5.141549in}}%
\pgfpathlineto{\pgfqpoint{10.891881in}{5.138705in}}%
\pgfpathlineto{\pgfqpoint{10.893895in}{5.128450in}}%
\pgfpathlineto{\pgfqpoint{10.895908in}{5.128899in}}%
\pgfpathlineto{\pgfqpoint{10.897921in}{5.137881in}}%
\pgfpathlineto{\pgfqpoint{10.903962in}{5.136983in}}%
\pgfpathlineto{\pgfqpoint{10.905975in}{5.125307in}}%
\pgfpathlineto{\pgfqpoint{10.907989in}{5.139603in}}%
\pgfpathlineto{\pgfqpoint{10.910002in}{5.129049in}}%
\pgfpathlineto{\pgfqpoint{10.912016in}{5.122762in}}%
\pgfpathlineto{\pgfqpoint{10.918056in}{5.107417in}}%
\pgfpathlineto{\pgfqpoint{10.920069in}{5.107118in}}%
\pgfpathlineto{\pgfqpoint{10.922083in}{5.094543in}}%
\pgfpathlineto{\pgfqpoint{10.924096in}{5.089753in}}%
\pgfpathlineto{\pgfqpoint{10.926110in}{5.105995in}}%
\pgfpathlineto{\pgfqpoint{10.932150in}{5.115950in}}%
\pgfpathlineto{\pgfqpoint{10.934164in}{5.127328in}}%
\pgfpathlineto{\pgfqpoint{10.936177in}{5.115651in}}%
\pgfpathlineto{\pgfqpoint{10.938190in}{5.127103in}}%
\pgfpathlineto{\pgfqpoint{10.940204in}{5.132567in}}%
\pgfpathlineto{\pgfqpoint{10.946244in}{5.134214in}}%
\pgfpathlineto{\pgfqpoint{10.948258in}{5.143795in}}%
\pgfpathlineto{\pgfqpoint{10.952285in}{5.148810in}}%
\pgfpathlineto{\pgfqpoint{10.954298in}{5.157268in}}%
\pgfpathlineto{\pgfqpoint{10.960338in}{5.163480in}}%
\pgfpathlineto{\pgfqpoint{10.962352in}{5.167223in}}%
\pgfpathlineto{\pgfqpoint{10.964365in}{5.174408in}}%
\pgfpathlineto{\pgfqpoint{10.966379in}{5.168570in}}%
\pgfpathlineto{\pgfqpoint{10.968392in}{5.173286in}}%
\pgfpathlineto{\pgfqpoint{10.974432in}{5.174259in}}%
\pgfpathlineto{\pgfqpoint{10.976446in}{5.169693in}}%
\pgfpathlineto{\pgfqpoint{10.978459in}{5.168346in}}%
\pgfpathlineto{\pgfqpoint{10.982486in}{5.162283in}}%
\pgfpathlineto{\pgfqpoint{10.988527in}{5.158465in}}%
\pgfpathlineto{\pgfqpoint{10.990540in}{5.161609in}}%
\pgfpathlineto{\pgfqpoint{10.992553in}{5.161085in}}%
\pgfpathlineto{\pgfqpoint{10.994567in}{5.161908in}}%
\pgfpathlineto{\pgfqpoint{10.996580in}{5.160337in}}%
\pgfpathlineto{\pgfqpoint{11.002621in}{5.164828in}}%
\pgfpathlineto{\pgfqpoint{11.004634in}{5.167298in}}%
\pgfpathlineto{\pgfqpoint{11.006648in}{5.167672in}}%
\pgfpathlineto{\pgfqpoint{11.010675in}{5.174408in}}%
\pgfpathlineto{\pgfqpoint{11.016715in}{5.172462in}}%
\pgfpathlineto{\pgfqpoint{11.018728in}{5.178450in}}%
\pgfpathlineto{\pgfqpoint{11.020742in}{5.166250in}}%
\pgfpathlineto{\pgfqpoint{11.022755in}{5.170292in}}%
\pgfpathlineto{\pgfqpoint{11.024769in}{5.175980in}}%
\pgfpathlineto{\pgfqpoint{11.030809in}{5.182343in}}%
\pgfpathlineto{\pgfqpoint{11.032822in}{5.180920in}}%
\pgfpathlineto{\pgfqpoint{11.034836in}{5.176729in}}%
\pgfpathlineto{\pgfqpoint{11.036849in}{5.179349in}}%
\pgfpathlineto{\pgfqpoint{11.038863in}{5.164528in}}%
\pgfpathlineto{\pgfqpoint{11.044903in}{5.157867in}}%
\pgfpathlineto{\pgfqpoint{11.046917in}{5.145366in}}%
\pgfpathlineto{\pgfqpoint{11.050943in}{5.179648in}}%
\pgfpathlineto{\pgfqpoint{11.052957in}{5.177777in}}%
\pgfpathlineto{\pgfqpoint{11.063024in}{5.185935in}}%
\pgfpathlineto{\pgfqpoint{11.067051in}{5.187432in}}%
\pgfpathlineto{\pgfqpoint{11.075105in}{5.187283in}}%
\pgfpathlineto{\pgfqpoint{11.077118in}{5.178750in}}%
\pgfpathlineto{\pgfqpoint{11.081145in}{5.178600in}}%
\pgfpathlineto{\pgfqpoint{11.087186in}{5.161459in}}%
\pgfpathlineto{\pgfqpoint{11.089199in}{5.148061in}}%
\pgfpathlineto{\pgfqpoint{11.091212in}{5.161534in}}%
\pgfpathlineto{\pgfqpoint{11.093226in}{5.170441in}}%
\pgfpathlineto{\pgfqpoint{11.095239in}{5.162358in}}%
\pgfpathlineto{\pgfqpoint{11.103293in}{5.154049in}}%
\pgfpathlineto{\pgfqpoint{11.105307in}{5.139004in}}%
\pgfpathlineto{\pgfqpoint{11.107320in}{5.130696in}}%
\pgfpathlineto{\pgfqpoint{11.109333in}{5.131819in}}%
\pgfpathlineto{\pgfqpoint{11.119401in}{5.143196in}}%
\pgfpathlineto{\pgfqpoint{11.121414in}{5.120217in}}%
\pgfpathlineto{\pgfqpoint{11.129468in}{5.112732in}}%
\pgfpathlineto{\pgfqpoint{11.133495in}{5.101804in}}%
\pgfpathlineto{\pgfqpoint{11.135508in}{5.103675in}}%
\pgfpathlineto{\pgfqpoint{11.137522in}{5.094543in}}%
\pgfpathlineto{\pgfqpoint{11.143562in}{5.104573in}}%
\pgfpathlineto{\pgfqpoint{11.145575in}{5.115726in}}%
\pgfpathlineto{\pgfqpoint{11.147589in}{5.114902in}}%
\pgfpathlineto{\pgfqpoint{11.149602in}{5.122687in}}%
\pgfpathlineto{\pgfqpoint{11.151616in}{5.124633in}}%
\pgfpathlineto{\pgfqpoint{11.157656in}{5.124483in}}%
\pgfpathlineto{\pgfqpoint{11.159670in}{5.130471in}}%
\pgfpathlineto{\pgfqpoint{11.161683in}{5.131594in}}%
\pgfpathlineto{\pgfqpoint{11.163697in}{5.093046in}}%
\pgfpathlineto{\pgfqpoint{11.165710in}{5.076354in}}%
\pgfpathlineto{\pgfqpoint{11.173764in}{5.083315in}}%
\pgfpathlineto{\pgfqpoint{11.175777in}{5.088181in}}%
\pgfpathlineto{\pgfqpoint{11.177791in}{5.078600in}}%
\pgfpathlineto{\pgfqpoint{11.179804in}{5.088555in}}%
\pgfpathlineto{\pgfqpoint{11.185844in}{5.091848in}}%
\pgfpathlineto{\pgfqpoint{11.187858in}{5.095741in}}%
\pgfpathlineto{\pgfqpoint{11.191885in}{5.112357in}}%
\pgfpathlineto{\pgfqpoint{11.193898in}{5.100830in}}%
\pgfpathlineto{\pgfqpoint{11.199939in}{5.103899in}}%
\pgfpathlineto{\pgfqpoint{11.201952in}{5.103076in}}%
\pgfpathlineto{\pgfqpoint{11.203965in}{5.094094in}}%
\pgfpathlineto{\pgfqpoint{11.205979in}{5.097762in}}%
\pgfpathlineto{\pgfqpoint{11.207992in}{5.091923in}}%
\pgfpathlineto{\pgfqpoint{11.214033in}{5.093271in}}%
\pgfpathlineto{\pgfqpoint{11.216046in}{5.083540in}}%
\pgfpathlineto{\pgfqpoint{11.218060in}{5.085935in}}%
\pgfpathlineto{\pgfqpoint{11.220073in}{5.100606in}}%
\pgfpathlineto{\pgfqpoint{11.222086in}{5.093944in}}%
\pgfpathlineto{\pgfqpoint{11.228127in}{5.100157in}}%
\pgfpathlineto{\pgfqpoint{11.230140in}{5.097088in}}%
\pgfpathlineto{\pgfqpoint{11.232154in}{5.102702in}}%
\pgfpathlineto{\pgfqpoint{11.234167in}{5.100456in}}%
\pgfpathlineto{\pgfqpoint{11.236181in}{5.108540in}}%
\pgfpathlineto{\pgfqpoint{11.242221in}{5.105097in}}%
\pgfpathlineto{\pgfqpoint{11.244234in}{5.099259in}}%
\pgfpathlineto{\pgfqpoint{11.246248in}{5.090501in}}%
\pgfpathlineto{\pgfqpoint{11.248261in}{5.079124in}}%
\pgfpathlineto{\pgfqpoint{11.250275in}{5.075606in}}%
\pgfpathlineto{\pgfqpoint{11.256315in}{5.076130in}}%
\pgfpathlineto{\pgfqpoint{11.258329in}{5.078450in}}%
\pgfpathlineto{\pgfqpoint{11.262355in}{5.089528in}}%
\pgfpathlineto{\pgfqpoint{11.270409in}{5.089004in}}%
\pgfpathlineto{\pgfqpoint{11.272423in}{5.079947in}}%
\pgfpathlineto{\pgfqpoint{11.276450in}{5.085711in}}%
\pgfpathlineto{\pgfqpoint{11.278463in}{5.088705in}}%
\pgfpathlineto{\pgfqpoint{11.284503in}{5.086459in}}%
\pgfpathlineto{\pgfqpoint{11.288530in}{5.089827in}}%
\pgfpathlineto{\pgfqpoint{11.290544in}{5.097986in}}%
\pgfpathlineto{\pgfqpoint{11.292557in}{5.072836in}}%
\pgfpathlineto{\pgfqpoint{11.298597in}{5.072238in}}%
\pgfpathlineto{\pgfqpoint{11.300611in}{5.072612in}}%
\pgfpathlineto{\pgfqpoint{11.302624in}{5.080696in}}%
\pgfpathlineto{\pgfqpoint{11.304638in}{5.078675in}}%
\pgfpathlineto{\pgfqpoint{11.312692in}{5.074034in}}%
\pgfpathlineto{\pgfqpoint{11.314705in}{5.074034in}}%
\pgfpathlineto{\pgfqpoint{11.316718in}{5.071714in}}%
\pgfpathlineto{\pgfqpoint{11.320745in}{5.075381in}}%
\pgfpathlineto{\pgfqpoint{11.326786in}{5.079573in}}%
\pgfpathlineto{\pgfqpoint{11.328799in}{5.076504in}}%
\pgfpathlineto{\pgfqpoint{11.330813in}{5.076579in}}%
\pgfpathlineto{\pgfqpoint{11.334840in}{5.084438in}}%
\pgfpathlineto{\pgfqpoint{11.340880in}{5.089528in}}%
\pgfpathlineto{\pgfqpoint{11.342893in}{5.085187in}}%
\pgfpathlineto{\pgfqpoint{11.346920in}{5.096863in}}%
\pgfpathlineto{\pgfqpoint{11.348934in}{5.093121in}}%
\pgfpathlineto{\pgfqpoint{11.354974in}{5.092821in}}%
\pgfpathlineto{\pgfqpoint{11.356987in}{5.101130in}}%
\pgfpathlineto{\pgfqpoint{11.359001in}{5.098286in}}%
\pgfpathlineto{\pgfqpoint{11.361014in}{5.096938in}}%
\pgfpathlineto{\pgfqpoint{11.363028in}{5.100307in}}%
\pgfpathlineto{\pgfqpoint{11.371082in}{5.092896in}}%
\pgfpathlineto{\pgfqpoint{11.375108in}{5.092223in}}%
\pgfpathlineto{\pgfqpoint{11.377122in}{5.089678in}}%
\pgfpathlineto{\pgfqpoint{11.383162in}{5.087881in}}%
\pgfpathlineto{\pgfqpoint{11.387189in}{5.095816in}}%
\pgfpathlineto{\pgfqpoint{11.389203in}{5.087133in}}%
\pgfpathlineto{\pgfqpoint{11.391216in}{5.087357in}}%
\pgfpathlineto{\pgfqpoint{11.397256in}{5.083241in}}%
\pgfpathlineto{\pgfqpoint{11.399270in}{5.085786in}}%
\pgfpathlineto{\pgfqpoint{11.401283in}{5.092896in}}%
\pgfpathlineto{\pgfqpoint{11.403297in}{5.093720in}}%
\pgfpathlineto{\pgfqpoint{11.405310in}{5.088330in}}%
\pgfpathlineto{\pgfqpoint{11.411351in}{5.086384in}}%
\pgfpathlineto{\pgfqpoint{11.413364in}{5.087133in}}%
\pgfpathlineto{\pgfqpoint{11.415377in}{5.093869in}}%
\pgfpathlineto{\pgfqpoint{11.417391in}{5.097013in}}%
\pgfpathlineto{\pgfqpoint{11.419404in}{5.092896in}}%
\pgfpathlineto{\pgfqpoint{11.425445in}{5.100307in}}%
\pgfpathlineto{\pgfqpoint{11.427458in}{5.101130in}}%
\pgfpathlineto{\pgfqpoint{11.429472in}{5.096789in}}%
\pgfpathlineto{\pgfqpoint{11.431485in}{5.090726in}}%
\pgfpathlineto{\pgfqpoint{11.433498in}{5.090726in}}%
\pgfpathlineto{\pgfqpoint{11.439539in}{5.076205in}}%
\pgfpathlineto{\pgfqpoint{11.441552in}{5.077702in}}%
\pgfpathlineto{\pgfqpoint{11.443566in}{5.082492in}}%
\pgfpathlineto{\pgfqpoint{11.445579in}{5.081145in}}%
\pgfpathlineto{\pgfqpoint{11.453633in}{5.076803in}}%
\pgfpathlineto{\pgfqpoint{11.455646in}{5.076504in}}%
\pgfpathlineto{\pgfqpoint{11.457660in}{5.065127in}}%
\pgfpathlineto{\pgfqpoint{11.459673in}{5.067971in}}%
\pgfpathlineto{\pgfqpoint{11.461687in}{5.074783in}}%
\pgfpathlineto{\pgfqpoint{11.469740in}{5.086459in}}%
\pgfpathlineto{\pgfqpoint{11.471754in}{5.083465in}}%
\pgfpathlineto{\pgfqpoint{11.475781in}{5.088181in}}%
\pgfpathlineto{\pgfqpoint{11.481821in}{5.088780in}}%
\pgfpathlineto{\pgfqpoint{11.483835in}{5.086309in}}%
\pgfpathlineto{\pgfqpoint{11.485848in}{5.086609in}}%
\pgfpathlineto{\pgfqpoint{11.487862in}{5.072687in}}%
\pgfpathlineto{\pgfqpoint{11.489875in}{5.064902in}}%
\pgfpathlineto{\pgfqpoint{11.495915in}{5.058016in}}%
\pgfpathlineto{\pgfqpoint{11.497929in}{5.059363in}}%
\pgfpathlineto{\pgfqpoint{11.501956in}{5.066474in}}%
\pgfpathlineto{\pgfqpoint{11.503969in}{5.066025in}}%
\pgfpathlineto{\pgfqpoint{11.510009in}{5.065501in}}%
\pgfpathlineto{\pgfqpoint{11.512023in}{5.063630in}}%
\pgfpathlineto{\pgfqpoint{11.514036in}{5.062732in}}%
\pgfpathlineto{\pgfqpoint{11.516050in}{5.058615in}}%
\pgfpathlineto{\pgfqpoint{11.518063in}{5.091699in}}%
\pgfpathlineto{\pgfqpoint{11.524104in}{5.102552in}}%
\pgfpathlineto{\pgfqpoint{11.526117in}{5.103076in}}%
\pgfpathlineto{\pgfqpoint{11.530144in}{5.099034in}}%
\pgfpathlineto{\pgfqpoint{11.532157in}{5.100082in}}%
\pgfpathlineto{\pgfqpoint{11.538198in}{5.100681in}}%
\pgfpathlineto{\pgfqpoint{11.540211in}{5.102402in}}%
\pgfpathlineto{\pgfqpoint{11.542225in}{5.100381in}}%
\pgfpathlineto{\pgfqpoint{11.546251in}{5.072836in}}%
\pgfpathlineto{\pgfqpoint{11.552292in}{5.056145in}}%
\pgfpathlineto{\pgfqpoint{11.554305in}{5.048211in}}%
\pgfpathlineto{\pgfqpoint{11.556319in}{5.063031in}}%
\pgfpathlineto{\pgfqpoint{11.558332in}{5.071789in}}%
\pgfpathlineto{\pgfqpoint{11.560346in}{5.070142in}}%
\pgfpathlineto{\pgfqpoint{11.566386in}{5.070666in}}%
\pgfpathlineto{\pgfqpoint{11.568399in}{5.051579in}}%
\pgfpathlineto{\pgfqpoint{11.570413in}{5.058166in}}%
\pgfpathlineto{\pgfqpoint{11.572426in}{5.060411in}}%
\pgfpathlineto{\pgfqpoint{11.574440in}{5.052103in}}%
\pgfpathlineto{\pgfqpoint{11.582494in}{5.062058in}}%
\pgfpathlineto{\pgfqpoint{11.584507in}{5.059438in}}%
\pgfpathlineto{\pgfqpoint{11.588534in}{5.062133in}}%
\pgfpathlineto{\pgfqpoint{11.594574in}{5.059588in}}%
\pgfpathlineto{\pgfqpoint{11.596588in}{5.069094in}}%
\pgfpathlineto{\pgfqpoint{11.598601in}{5.074857in}}%
\pgfpathlineto{\pgfqpoint{11.600615in}{5.073061in}}%
\pgfpathlineto{\pgfqpoint{11.602628in}{5.065277in}}%
\pgfpathlineto{\pgfqpoint{11.608668in}{5.070815in}}%
\pgfpathlineto{\pgfqpoint{11.610682in}{5.063555in}}%
\pgfpathlineto{\pgfqpoint{11.612695in}{5.063031in}}%
\pgfpathlineto{\pgfqpoint{11.614709in}{5.056444in}}%
\pgfpathlineto{\pgfqpoint{11.616722in}{5.059288in}}%
\pgfpathlineto{\pgfqpoint{11.622762in}{5.046938in}}%
\pgfpathlineto{\pgfqpoint{11.624776in}{5.045291in}}%
\pgfpathlineto{\pgfqpoint{11.626789in}{5.052477in}}%
\pgfpathlineto{\pgfqpoint{11.628803in}{5.050830in}}%
\pgfpathlineto{\pgfqpoint{11.630816in}{5.054423in}}%
\pgfpathlineto{\pgfqpoint{11.636857in}{5.074708in}}%
\pgfpathlineto{\pgfqpoint{11.638870in}{5.072013in}}%
\pgfpathlineto{\pgfqpoint{11.640884in}{5.075980in}}%
\pgfpathlineto{\pgfqpoint{11.642897in}{5.075980in}}%
\pgfpathlineto{\pgfqpoint{11.644910in}{5.077028in}}%
\pgfpathlineto{\pgfqpoint{11.650951in}{5.076878in}}%
\pgfpathlineto{\pgfqpoint{11.652964in}{5.071863in}}%
\pgfpathlineto{\pgfqpoint{11.654978in}{5.068794in}}%
\pgfpathlineto{\pgfqpoint{11.659005in}{5.076130in}}%
\pgfpathlineto{\pgfqpoint{11.667058in}{5.074333in}}%
\pgfpathlineto{\pgfqpoint{11.669072in}{5.071265in}}%
\pgfpathlineto{\pgfqpoint{11.671085in}{5.042971in}}%
\pgfpathlineto{\pgfqpoint{11.673099in}{5.057717in}}%
\pgfpathlineto{\pgfqpoint{11.681152in}{5.053600in}}%
\pgfpathlineto{\pgfqpoint{11.683166in}{5.056744in}}%
\pgfpathlineto{\pgfqpoint{11.685179in}{5.055097in}}%
\pgfpathlineto{\pgfqpoint{11.687193in}{5.048360in}}%
\pgfpathlineto{\pgfqpoint{11.693233in}{5.053151in}}%
\pgfpathlineto{\pgfqpoint{11.697260in}{5.054049in}}%
\pgfpathlineto{\pgfqpoint{11.699273in}{5.053076in}}%
\pgfpathlineto{\pgfqpoint{11.701287in}{5.055621in}}%
\pgfpathlineto{\pgfqpoint{11.707327in}{5.049483in}}%
\pgfpathlineto{\pgfqpoint{11.709341in}{5.049109in}}%
\pgfpathlineto{\pgfqpoint{11.711354in}{5.045890in}}%
\pgfpathlineto{\pgfqpoint{11.715381in}{5.033839in}}%
\pgfpathlineto{\pgfqpoint{11.721421in}{5.037058in}}%
\pgfpathlineto{\pgfqpoint{11.723435in}{5.033166in}}%
\pgfpathlineto{\pgfqpoint{11.725448in}{5.040501in}}%
\pgfpathlineto{\pgfqpoint{11.727462in}{5.044693in}}%
\pgfpathlineto{\pgfqpoint{11.729475in}{5.042372in}}%
\pgfpathlineto{\pgfqpoint{11.735516in}{5.041175in}}%
\pgfpathlineto{\pgfqpoint{11.737529in}{5.036833in}}%
\pgfpathlineto{\pgfqpoint{11.743569in}{5.038405in}}%
\pgfpathlineto{\pgfqpoint{11.749610in}{5.036908in}}%
\pgfpathlineto{\pgfqpoint{11.751623in}{5.040875in}}%
\pgfpathlineto{\pgfqpoint{11.755650in}{5.028300in}}%
\pgfpathlineto{\pgfqpoint{11.757663in}{5.033091in}}%
\pgfpathlineto{\pgfqpoint{11.763704in}{5.029648in}}%
\pgfpathlineto{\pgfqpoint{11.765717in}{5.024708in}}%
\pgfpathlineto{\pgfqpoint{11.767731in}{5.024333in}}%
\pgfpathlineto{\pgfqpoint{11.769744in}{5.026130in}}%
\pgfpathlineto{\pgfqpoint{11.771758in}{5.017297in}}%
\pgfpathlineto{\pgfqpoint{11.777798in}{5.017148in}}%
\pgfpathlineto{\pgfqpoint{11.779811in}{5.026429in}}%
\pgfpathlineto{\pgfqpoint{11.781825in}{5.030396in}}%
\pgfpathlineto{\pgfqpoint{11.783838in}{5.022312in}}%
\pgfpathlineto{\pgfqpoint{11.785852in}{5.010336in}}%
\pgfpathlineto{\pgfqpoint{11.791892in}{5.014079in}}%
\pgfpathlineto{\pgfqpoint{11.793905in}{5.017223in}}%
\pgfpathlineto{\pgfqpoint{11.795919in}{5.025157in}}%
\pgfpathlineto{\pgfqpoint{11.797932in}{5.026504in}}%
\pgfpathlineto{\pgfqpoint{11.805986in}{5.023660in}}%
\pgfpathlineto{\pgfqpoint{11.808000in}{5.029199in}}%
\pgfpathlineto{\pgfqpoint{11.810013in}{5.026504in}}%
\pgfpathlineto{\pgfqpoint{11.812027in}{5.022163in}}%
\pgfpathlineto{\pgfqpoint{11.820080in}{5.008390in}}%
\pgfpathlineto{\pgfqpoint{11.822094in}{5.001055in}}%
\pgfpathlineto{\pgfqpoint{11.824107in}{4.988031in}}%
\pgfpathlineto{\pgfqpoint{11.826121in}{4.983914in}}%
\pgfpathlineto{\pgfqpoint{11.828134in}{4.982417in}}%
\pgfpathlineto{\pgfqpoint{11.834174in}{4.985411in}}%
\pgfpathlineto{\pgfqpoint{11.836188in}{4.987881in}}%
\pgfpathlineto{\pgfqpoint{11.838201in}{4.976878in}}%
\pgfpathlineto{\pgfqpoint{11.840215in}{4.980022in}}%
\pgfpathlineto{\pgfqpoint{11.842228in}{4.977327in}}%
\pgfpathlineto{\pgfqpoint{11.850282in}{4.975456in}}%
\pgfpathlineto{\pgfqpoint{11.852295in}{4.978151in}}%
\pgfpathlineto{\pgfqpoint{11.854309in}{4.975456in}}%
\pgfpathlineto{\pgfqpoint{11.856322in}{4.921788in}}%
\pgfpathlineto{\pgfqpoint{11.862363in}{4.921489in}}%
\pgfpathlineto{\pgfqpoint{11.864376in}{4.922013in}}%
\pgfpathlineto{\pgfqpoint{11.866390in}{4.917971in}}%
\pgfpathlineto{\pgfqpoint{11.868403in}{4.906369in}}%
\pgfpathlineto{\pgfqpoint{11.870416in}{4.910710in}}%
\pgfpathlineto{\pgfqpoint{11.876457in}{4.919243in}}%
\pgfpathlineto{\pgfqpoint{11.878470in}{4.911908in}}%
\pgfpathlineto{\pgfqpoint{11.880484in}{4.915052in}}%
\pgfpathlineto{\pgfqpoint{11.882497in}{4.916998in}}%
\pgfpathlineto{\pgfqpoint{11.884511in}{4.914153in}}%
\pgfpathlineto{\pgfqpoint{11.890551in}{4.902926in}}%
\pgfpathlineto{\pgfqpoint{11.892564in}{4.904573in}}%
\pgfpathlineto{\pgfqpoint{11.894578in}{4.902177in}}%
\pgfpathlineto{\pgfqpoint{11.896591in}{4.893794in}}%
\pgfpathlineto{\pgfqpoint{11.898605in}{4.904797in}}%
\pgfpathlineto{\pgfqpoint{11.906659in}{4.908465in}}%
\pgfpathlineto{\pgfqpoint{11.910685in}{4.915351in}}%
\pgfpathlineto{\pgfqpoint{11.912699in}{4.919318in}}%
\pgfpathlineto{\pgfqpoint{11.918739in}{4.925830in}}%
\pgfpathlineto{\pgfqpoint{11.922766in}{4.918794in}}%
\pgfpathlineto{\pgfqpoint{11.924780in}{4.924109in}}%
\pgfpathlineto{\pgfqpoint{11.926793in}{4.924034in}}%
\pgfpathlineto{\pgfqpoint{11.932833in}{4.925456in}}%
\pgfpathlineto{\pgfqpoint{11.934847in}{4.934064in}}%
\pgfpathlineto{\pgfqpoint{11.936860in}{4.936384in}}%
\pgfpathlineto{\pgfqpoint{11.938874in}{4.943195in}}%
\pgfpathlineto{\pgfqpoint{11.940887in}{4.944618in}}%
\pgfpathlineto{\pgfqpoint{11.946927in}{4.949633in}}%
\pgfpathlineto{\pgfqpoint{11.948941in}{4.952701in}}%
\pgfpathlineto{\pgfqpoint{11.952968in}{4.947911in}}%
\pgfpathlineto{\pgfqpoint{11.954981in}{4.952926in}}%
\pgfpathlineto{\pgfqpoint{11.961022in}{4.953675in}}%
\pgfpathlineto{\pgfqpoint{11.963035in}{4.951279in}}%
\pgfpathlineto{\pgfqpoint{11.967062in}{4.957342in}}%
\pgfpathlineto{\pgfqpoint{11.969075in}{4.965351in}}%
\pgfpathlineto{\pgfqpoint{11.975116in}{4.965276in}}%
\pgfpathlineto{\pgfqpoint{11.977129in}{4.961085in}}%
\pgfpathlineto{\pgfqpoint{11.979143in}{4.961234in}}%
\pgfpathlineto{\pgfqpoint{11.981156in}{4.960037in}}%
\pgfpathlineto{\pgfqpoint{11.989210in}{4.958689in}}%
\pgfpathlineto{\pgfqpoint{11.991223in}{4.961010in}}%
\pgfpathlineto{\pgfqpoint{11.993237in}{4.958764in}}%
\pgfpathlineto{\pgfqpoint{11.995250in}{4.966624in}}%
\pgfpathlineto{\pgfqpoint{11.997264in}{4.964528in}}%
\pgfpathlineto{\pgfqpoint{12.003304in}{4.961160in}}%
\pgfpathlineto{\pgfqpoint{12.005317in}{4.958091in}}%
\pgfpathlineto{\pgfqpoint{12.007331in}{4.957941in}}%
\pgfpathlineto{\pgfqpoint{12.009344in}{4.950680in}}%
\pgfpathlineto{\pgfqpoint{12.011358in}{4.955246in}}%
\pgfpathlineto{\pgfqpoint{12.017398in}{4.957342in}}%
\pgfpathlineto{\pgfqpoint{12.019412in}{4.963630in}}%
\pgfpathlineto{\pgfqpoint{12.021425in}{4.974184in}}%
\pgfpathlineto{\pgfqpoint{12.023438in}{4.976654in}}%
\pgfpathlineto{\pgfqpoint{12.025452in}{4.974034in}}%
\pgfpathlineto{\pgfqpoint{12.031492in}{4.977252in}}%
\pgfpathlineto{\pgfqpoint{12.033506in}{4.984213in}}%
\pgfpathlineto{\pgfqpoint{12.035519in}{4.994468in}}%
\pgfpathlineto{\pgfqpoint{12.037533in}{4.998660in}}%
\pgfpathlineto{\pgfqpoint{12.039546in}{5.000980in}}%
\pgfpathlineto{\pgfqpoint{12.045586in}{4.999333in}}%
\pgfpathlineto{\pgfqpoint{12.047600in}{5.002926in}}%
\pgfpathlineto{\pgfqpoint{12.049613in}{5.003076in}}%
\pgfpathlineto{\pgfqpoint{12.053640in}{4.997387in}}%
\pgfpathlineto{\pgfqpoint{12.059681in}{4.999184in}}%
\pgfpathlineto{\pgfqpoint{12.063707in}{4.989079in}}%
\pgfpathlineto{\pgfqpoint{12.065721in}{4.986684in}}%
\pgfpathlineto{\pgfqpoint{12.067734in}{4.990875in}}%
\pgfpathlineto{\pgfqpoint{12.073775in}{4.987133in}}%
\pgfpathlineto{\pgfqpoint{12.075788in}{4.993196in}}%
\pgfpathlineto{\pgfqpoint{12.081828in}{4.988106in}}%
\pgfpathlineto{\pgfqpoint{12.087869in}{4.987731in}}%
\pgfpathlineto{\pgfqpoint{12.089882in}{4.978450in}}%
\pgfpathlineto{\pgfqpoint{12.091896in}{4.983839in}}%
\pgfpathlineto{\pgfqpoint{12.093909in}{4.978300in}}%
\pgfpathlineto{\pgfqpoint{12.095923in}{4.986684in}}%
\pgfpathlineto{\pgfqpoint{12.101963in}{4.984363in}}%
\pgfpathlineto{\pgfqpoint{12.103976in}{4.993420in}}%
\pgfpathlineto{\pgfqpoint{12.105990in}{4.996564in}}%
\pgfpathlineto{\pgfqpoint{12.108003in}{4.995965in}}%
\pgfpathlineto{\pgfqpoint{12.110017in}{4.998061in}}%
\pgfpathlineto{\pgfqpoint{12.118070in}{4.999782in}}%
\pgfpathlineto{\pgfqpoint{12.120084in}{5.000830in}}%
\pgfpathlineto{\pgfqpoint{12.122097in}{5.004348in}}%
\pgfpathlineto{\pgfqpoint{12.124111in}{4.997836in}}%
\pgfpathlineto{\pgfqpoint{12.130151in}{5.001055in}}%
\pgfpathlineto{\pgfqpoint{12.132165in}{5.000681in}}%
\pgfpathlineto{\pgfqpoint{12.134178in}{5.003300in}}%
\pgfpathlineto{\pgfqpoint{12.136192in}{4.999408in}}%
\pgfpathlineto{\pgfqpoint{12.138205in}{4.994169in}}%
\pgfpathlineto{\pgfqpoint{12.144245in}{4.984887in}}%
\pgfpathlineto{\pgfqpoint{12.146259in}{4.966399in}}%
\pgfpathlineto{\pgfqpoint{12.148272in}{4.968869in}}%
\pgfpathlineto{\pgfqpoint{12.150286in}{4.972537in}}%
\pgfpathlineto{\pgfqpoint{12.152299in}{4.972013in}}%
\pgfpathlineto{\pgfqpoint{12.158339in}{4.975157in}}%
\pgfpathlineto{\pgfqpoint{12.160353in}{4.975082in}}%
\pgfpathlineto{\pgfqpoint{12.162366in}{4.972687in}}%
\pgfpathlineto{\pgfqpoint{12.164380in}{4.981893in}}%
\pgfpathlineto{\pgfqpoint{12.166393in}{4.959213in}}%
\pgfpathlineto{\pgfqpoint{12.172434in}{4.942222in}}%
\pgfpathlineto{\pgfqpoint{12.174447in}{4.943869in}}%
\pgfpathlineto{\pgfqpoint{12.176460in}{4.958315in}}%
\pgfpathlineto{\pgfqpoint{12.178474in}{4.966399in}}%
\pgfpathlineto{\pgfqpoint{12.180487in}{4.965875in}}%
\pgfpathlineto{\pgfqpoint{12.188541in}{4.954872in}}%
\pgfpathlineto{\pgfqpoint{12.192568in}{4.959588in}}%
\pgfpathlineto{\pgfqpoint{12.194581in}{4.971564in}}%
\pgfpathlineto{\pgfqpoint{12.200622in}{4.976354in}}%
\pgfpathlineto{\pgfqpoint{12.202635in}{4.982417in}}%
\pgfpathlineto{\pgfqpoint{12.204649in}{4.983091in}}%
\pgfpathlineto{\pgfqpoint{12.206662in}{4.986758in}}%
\pgfpathlineto{\pgfqpoint{12.208676in}{4.987956in}}%
\pgfpathlineto{\pgfqpoint{12.214716in}{4.989453in}}%
\pgfpathlineto{\pgfqpoint{12.216729in}{4.990651in}}%
\pgfpathlineto{\pgfqpoint{12.218743in}{4.992971in}}%
\pgfpathlineto{\pgfqpoint{12.220756in}{4.985486in}}%
\pgfpathlineto{\pgfqpoint{12.222770in}{4.991549in}}%
\pgfpathlineto{\pgfqpoint{12.230824in}{4.992148in}}%
\pgfpathlineto{\pgfqpoint{12.234850in}{4.995217in}}%
\pgfpathlineto{\pgfqpoint{12.236864in}{4.992821in}}%
\pgfpathlineto{\pgfqpoint{12.242904in}{4.990501in}}%
\pgfpathlineto{\pgfqpoint{12.244918in}{4.985785in}}%
\pgfpathlineto{\pgfqpoint{12.246931in}{4.988330in}}%
\pgfpathlineto{\pgfqpoint{12.248945in}{4.989079in}}%
\pgfpathlineto{\pgfqpoint{12.250958in}{5.000381in}}%
\pgfpathlineto{\pgfqpoint{12.256998in}{5.002627in}}%
\pgfpathlineto{\pgfqpoint{12.259012in}{4.999483in}}%
\pgfpathlineto{\pgfqpoint{12.261025in}{4.994842in}}%
\pgfpathlineto{\pgfqpoint{12.263039in}{5.000082in}}%
\pgfpathlineto{\pgfqpoint{12.265052in}{4.999408in}}%
\pgfpathlineto{\pgfqpoint{12.271092in}{5.001205in}}%
\pgfpathlineto{\pgfqpoint{12.273106in}{4.998809in}}%
\pgfpathlineto{\pgfqpoint{12.275119in}{5.001579in}}%
\pgfpathlineto{\pgfqpoint{12.285187in}{4.999258in}}%
\pgfpathlineto{\pgfqpoint{12.287200in}{5.001579in}}%
\pgfpathlineto{\pgfqpoint{12.289214in}{4.997836in}}%
\pgfpathlineto{\pgfqpoint{12.293240in}{4.995217in}}%
\pgfpathlineto{\pgfqpoint{12.299281in}{5.000381in}}%
\pgfpathlineto{\pgfqpoint{12.301294in}{4.999932in}}%
\pgfpathlineto{\pgfqpoint{12.303308in}{5.000830in}}%
\pgfpathlineto{\pgfqpoint{12.305321in}{4.995666in}}%
\pgfpathlineto{\pgfqpoint{12.307335in}{4.998061in}}%
\pgfpathlineto{\pgfqpoint{12.315388in}{5.002103in}}%
\pgfpathlineto{\pgfqpoint{12.317402in}{5.005172in}}%
\pgfpathlineto{\pgfqpoint{12.319415in}{5.005546in}}%
\pgfpathlineto{\pgfqpoint{12.321429in}{4.997537in}}%
\pgfpathlineto{\pgfqpoint{12.327469in}{5.003151in}}%
\pgfpathlineto{\pgfqpoint{12.329482in}{4.992073in}}%
\pgfpathlineto{\pgfqpoint{12.331496in}{4.985785in}}%
\pgfpathlineto{\pgfqpoint{12.333509in}{4.988330in}}%
\pgfpathlineto{\pgfqpoint{12.335523in}{4.987133in}}%
\pgfpathlineto{\pgfqpoint{12.341563in}{4.989977in}}%
\pgfpathlineto{\pgfqpoint{12.343577in}{4.987507in}}%
\pgfpathlineto{\pgfqpoint{12.345590in}{4.991549in}}%
\pgfpathlineto{\pgfqpoint{12.347603in}{4.994019in}}%
\pgfpathlineto{\pgfqpoint{12.349617in}{4.988480in}}%
\pgfpathlineto{\pgfqpoint{12.355657in}{4.985411in}}%
\pgfpathlineto{\pgfqpoint{12.357671in}{4.991549in}}%
\pgfpathlineto{\pgfqpoint{12.359684in}{4.991100in}}%
\pgfpathlineto{\pgfqpoint{12.361698in}{4.985037in}}%
\pgfpathlineto{\pgfqpoint{12.363711in}{4.989827in}}%
\pgfpathlineto{\pgfqpoint{12.369751in}{4.988181in}}%
\pgfpathlineto{\pgfqpoint{12.371765in}{4.988929in}}%
\pgfpathlineto{\pgfqpoint{12.373778in}{4.994393in}}%
\pgfpathlineto{\pgfqpoint{12.375792in}{4.977028in}}%
\pgfpathlineto{\pgfqpoint{12.377805in}{4.975755in}}%
\pgfpathlineto{\pgfqpoint{12.383846in}{4.976728in}}%
\pgfpathlineto{\pgfqpoint{12.385859in}{4.969318in}}%
\pgfpathlineto{\pgfqpoint{12.387872in}{4.968046in}}%
\pgfpathlineto{\pgfqpoint{12.391899in}{4.964154in}}%
\pgfpathlineto{\pgfqpoint{12.397940in}{4.962357in}}%
\pgfpathlineto{\pgfqpoint{12.399953in}{4.963704in}}%
\pgfpathlineto{\pgfqpoint{12.401967in}{4.972088in}}%
\pgfpathlineto{\pgfqpoint{12.403980in}{5.011833in}}%
\pgfpathlineto{\pgfqpoint{12.405993in}{5.015950in}}%
\pgfpathlineto{\pgfqpoint{12.412034in}{5.014004in}}%
\pgfpathlineto{\pgfqpoint{12.414047in}{5.011534in}}%
\pgfpathlineto{\pgfqpoint{12.418074in}{5.012881in}}%
\pgfpathlineto{\pgfqpoint{12.420088in}{5.009438in}}%
\pgfpathlineto{\pgfqpoint{12.426128in}{5.009214in}}%
\pgfpathlineto{\pgfqpoint{12.428141in}{5.008016in}}%
\pgfpathlineto{\pgfqpoint{12.430155in}{5.002252in}}%
\pgfpathlineto{\pgfqpoint{12.432168in}{5.001429in}}%
\pgfpathlineto{\pgfqpoint{12.434182in}{5.002702in}}%
\pgfpathlineto{\pgfqpoint{12.440222in}{5.013405in}}%
\pgfpathlineto{\pgfqpoint{12.442235in}{5.013854in}}%
\pgfpathlineto{\pgfqpoint{12.446262in}{5.035711in}}%
\pgfpathlineto{\pgfqpoint{12.448276in}{5.038555in}}%
\pgfpathlineto{\pgfqpoint{12.454316in}{5.052327in}}%
\pgfpathlineto{\pgfqpoint{12.456330in}{5.052702in}}%
\pgfpathlineto{\pgfqpoint{12.458343in}{5.047013in}}%
\pgfpathlineto{\pgfqpoint{12.460357in}{5.047687in}}%
\pgfpathlineto{\pgfqpoint{12.462370in}{5.042148in}}%
\pgfpathlineto{\pgfqpoint{12.470424in}{5.047312in}}%
\pgfpathlineto{\pgfqpoint{12.472437in}{5.055621in}}%
\pgfpathlineto{\pgfqpoint{12.476464in}{5.055471in}}%
\pgfpathlineto{\pgfqpoint{12.482504in}{5.050232in}}%
\pgfpathlineto{\pgfqpoint{12.484518in}{5.045666in}}%
\pgfpathlineto{\pgfqpoint{12.488545in}{5.053076in}}%
\pgfpathlineto{\pgfqpoint{12.490558in}{5.048285in}}%
\pgfpathlineto{\pgfqpoint{12.496599in}{5.049483in}}%
\pgfpathlineto{\pgfqpoint{12.498612in}{5.051504in}}%
\pgfpathlineto{\pgfqpoint{12.500625in}{5.065726in}}%
\pgfpathlineto{\pgfqpoint{12.502639in}{5.070217in}}%
\pgfpathlineto{\pgfqpoint{12.504652in}{5.069169in}}%
\pgfpathlineto{\pgfqpoint{12.510693in}{5.060636in}}%
\pgfpathlineto{\pgfqpoint{12.514720in}{5.064154in}}%
\pgfpathlineto{\pgfqpoint{12.516733in}{5.070441in}}%
\pgfpathlineto{\pgfqpoint{12.518746in}{5.070815in}}%
\pgfpathlineto{\pgfqpoint{12.524787in}{5.067597in}}%
\pgfpathlineto{\pgfqpoint{12.526800in}{5.071265in}}%
\pgfpathlineto{\pgfqpoint{12.528814in}{5.073136in}}%
\pgfpathlineto{\pgfqpoint{12.530827in}{5.067821in}}%
\pgfpathlineto{\pgfqpoint{12.532841in}{5.070666in}}%
\pgfpathlineto{\pgfqpoint{12.540894in}{5.070741in}}%
\pgfpathlineto{\pgfqpoint{12.544921in}{5.063106in}}%
\pgfpathlineto{\pgfqpoint{12.546935in}{5.064229in}}%
\pgfpathlineto{\pgfqpoint{12.554989in}{5.073360in}}%
\pgfpathlineto{\pgfqpoint{12.557002in}{5.082268in}}%
\pgfpathlineto{\pgfqpoint{12.559015in}{5.075456in}}%
\pgfpathlineto{\pgfqpoint{12.567069in}{5.079348in}}%
\pgfpathlineto{\pgfqpoint{12.569083in}{5.085037in}}%
\pgfpathlineto{\pgfqpoint{12.571096in}{5.086908in}}%
\pgfpathlineto{\pgfqpoint{12.573110in}{5.086684in}}%
\pgfpathlineto{\pgfqpoint{12.575123in}{5.084812in}}%
\pgfpathlineto{\pgfqpoint{12.583177in}{5.084663in}}%
\pgfpathlineto{\pgfqpoint{12.585190in}{5.091100in}}%
\pgfpathlineto{\pgfqpoint{12.587204in}{5.085336in}}%
\pgfpathlineto{\pgfqpoint{12.589217in}{5.081818in}}%
\pgfpathlineto{\pgfqpoint{12.595257in}{5.080172in}}%
\pgfpathlineto{\pgfqpoint{12.597271in}{5.090651in}}%
\pgfpathlineto{\pgfqpoint{12.599284in}{5.086759in}}%
\pgfpathlineto{\pgfqpoint{12.601298in}{5.087058in}}%
\pgfpathlineto{\pgfqpoint{12.603311in}{5.086459in}}%
\pgfpathlineto{\pgfqpoint{12.609352in}{5.089603in}}%
\pgfpathlineto{\pgfqpoint{12.611365in}{5.083091in}}%
\pgfpathlineto{\pgfqpoint{12.613379in}{5.085860in}}%
\pgfpathlineto{\pgfqpoint{12.615392in}{5.084064in}}%
\pgfpathlineto{\pgfqpoint{12.617405in}{5.095067in}}%
\pgfpathlineto{\pgfqpoint{12.625459in}{5.092747in}}%
\pgfpathlineto{\pgfqpoint{12.627473in}{5.093345in}}%
\pgfpathlineto{\pgfqpoint{12.631500in}{5.098286in}}%
\pgfpathlineto{\pgfqpoint{12.637540in}{5.101354in}}%
\pgfpathlineto{\pgfqpoint{12.639553in}{5.104947in}}%
\pgfpathlineto{\pgfqpoint{12.641567in}{5.106369in}}%
\pgfpathlineto{\pgfqpoint{12.643580in}{5.105696in}}%
\pgfpathlineto{\pgfqpoint{12.645594in}{5.107118in}}%
\pgfpathlineto{\pgfqpoint{12.653647in}{5.108989in}}%
\pgfpathlineto{\pgfqpoint{12.655661in}{5.108241in}}%
\pgfpathlineto{\pgfqpoint{12.657674in}{5.109588in}}%
\pgfpathlineto{\pgfqpoint{12.659688in}{5.107492in}}%
\pgfpathlineto{\pgfqpoint{12.665728in}{5.110411in}}%
\pgfpathlineto{\pgfqpoint{12.667742in}{5.109663in}}%
\pgfpathlineto{\pgfqpoint{12.669755in}{5.123061in}}%
\pgfpathlineto{\pgfqpoint{12.671768in}{5.109962in}}%
\pgfpathlineto{\pgfqpoint{12.673782in}{5.108390in}}%
\pgfpathlineto{\pgfqpoint{12.679822in}{5.105621in}}%
\pgfpathlineto{\pgfqpoint{12.681836in}{5.106220in}}%
\pgfpathlineto{\pgfqpoint{12.683849in}{5.102327in}}%
\pgfpathlineto{\pgfqpoint{12.685863in}{5.104199in}}%
\pgfpathlineto{\pgfqpoint{12.687876in}{5.104723in}}%
\pgfpathlineto{\pgfqpoint{12.693916in}{5.103525in}}%
\pgfpathlineto{\pgfqpoint{12.695930in}{5.106968in}}%
\pgfpathlineto{\pgfqpoint{12.697943in}{5.103675in}}%
\pgfpathlineto{\pgfqpoint{12.699957in}{5.107567in}}%
\pgfpathlineto{\pgfqpoint{12.701970in}{5.103825in}}%
\pgfpathlineto{\pgfqpoint{12.708011in}{5.100905in}}%
\pgfpathlineto{\pgfqpoint{12.710024in}{5.091399in}}%
\pgfpathlineto{\pgfqpoint{12.714051in}{5.093645in}}%
\pgfpathlineto{\pgfqpoint{12.716064in}{5.096265in}}%
\pgfpathlineto{\pgfqpoint{12.722105in}{5.091923in}}%
\pgfpathlineto{\pgfqpoint{12.724118in}{5.099408in}}%
\pgfpathlineto{\pgfqpoint{12.726132in}{5.096564in}}%
\pgfpathlineto{\pgfqpoint{12.728145in}{5.103525in}}%
\pgfpathlineto{\pgfqpoint{12.730158in}{5.102777in}}%
\pgfpathlineto{\pgfqpoint{12.736199in}{5.099034in}}%
\pgfpathlineto{\pgfqpoint{12.738212in}{5.096639in}}%
\pgfpathlineto{\pgfqpoint{12.740226in}{5.095366in}}%
\pgfpathlineto{\pgfqpoint{12.742239in}{5.096489in}}%
\pgfpathlineto{\pgfqpoint{12.744253in}{5.095441in}}%
\pgfpathlineto{\pgfqpoint{12.750293in}{5.093420in}}%
\pgfpathlineto{\pgfqpoint{12.752306in}{5.091774in}}%
\pgfpathlineto{\pgfqpoint{12.754320in}{5.087507in}}%
\pgfpathlineto{\pgfqpoint{12.756333in}{5.081145in}}%
\pgfpathlineto{\pgfqpoint{12.764387in}{5.087432in}}%
\pgfpathlineto{\pgfqpoint{12.766401in}{5.081070in}}%
\pgfpathlineto{\pgfqpoint{12.768414in}{5.079348in}}%
\pgfpathlineto{\pgfqpoint{12.770427in}{5.111759in}}%
\pgfpathlineto{\pgfqpoint{12.772441in}{5.108615in}}%
\pgfpathlineto{\pgfqpoint{12.780495in}{5.116175in}}%
\pgfpathlineto{\pgfqpoint{12.784522in}{5.114004in}}%
\pgfpathlineto{\pgfqpoint{12.786535in}{5.106145in}}%
\pgfpathlineto{\pgfqpoint{12.792575in}{5.105995in}}%
\pgfpathlineto{\pgfqpoint{12.794589in}{5.108241in}}%
\pgfpathlineto{\pgfqpoint{12.798616in}{5.099483in}}%
\pgfpathlineto{\pgfqpoint{12.800629in}{5.099408in}}%
\pgfpathlineto{\pgfqpoint{12.806669in}{5.098286in}}%
\pgfpathlineto{\pgfqpoint{12.810696in}{5.101804in}}%
\pgfpathlineto{\pgfqpoint{12.812710in}{5.096489in}}%
\pgfpathlineto{\pgfqpoint{12.814723in}{5.093420in}}%
\pgfpathlineto{\pgfqpoint{12.820764in}{5.099483in}}%
\pgfpathlineto{\pgfqpoint{12.822777in}{5.098061in}}%
\pgfpathlineto{\pgfqpoint{12.824790in}{5.085262in}}%
\pgfpathlineto{\pgfqpoint{12.826804in}{5.085336in}}%
\pgfpathlineto{\pgfqpoint{12.828817in}{5.088405in}}%
\pgfpathlineto{\pgfqpoint{12.834858in}{5.089678in}}%
\pgfpathlineto{\pgfqpoint{12.836871in}{5.091324in}}%
\pgfpathlineto{\pgfqpoint{12.838885in}{5.090726in}}%
\pgfpathlineto{\pgfqpoint{12.840898in}{5.093046in}}%
\pgfpathlineto{\pgfqpoint{12.842911in}{5.093196in}}%
\pgfpathlineto{\pgfqpoint{12.852979in}{5.089453in}}%
\pgfpathlineto{\pgfqpoint{12.854992in}{5.099109in}}%
\pgfpathlineto{\pgfqpoint{12.857006in}{5.100681in}}%
\pgfpathlineto{\pgfqpoint{12.863046in}{5.104124in}}%
\pgfpathlineto{\pgfqpoint{12.865059in}{5.103226in}}%
\pgfpathlineto{\pgfqpoint{12.867073in}{5.110187in}}%
\pgfpathlineto{\pgfqpoint{12.869086in}{5.111235in}}%
\pgfpathlineto{\pgfqpoint{12.871100in}{5.113854in}}%
\pgfpathlineto{\pgfqpoint{12.877140in}{5.112807in}}%
\pgfpathlineto{\pgfqpoint{12.879154in}{5.115875in}}%
\pgfpathlineto{\pgfqpoint{12.881167in}{5.117672in}}%
\pgfpathlineto{\pgfqpoint{12.883180in}{5.116624in}}%
\pgfpathlineto{\pgfqpoint{12.885194in}{5.122088in}}%
\pgfpathlineto{\pgfqpoint{12.891234in}{5.125232in}}%
\pgfpathlineto{\pgfqpoint{12.893248in}{5.129798in}}%
\pgfpathlineto{\pgfqpoint{12.895261in}{5.127627in}}%
\pgfpathlineto{\pgfqpoint{12.899288in}{5.127702in}}%
\pgfpathlineto{\pgfqpoint{12.905328in}{5.133016in}}%
\pgfpathlineto{\pgfqpoint{12.907342in}{5.133914in}}%
\pgfpathlineto{\pgfqpoint{12.909355in}{5.140352in}}%
\pgfpathlineto{\pgfqpoint{12.911369in}{5.137881in}}%
\pgfpathlineto{\pgfqpoint{12.913382in}{5.142298in}}%
\pgfpathlineto{\pgfqpoint{12.919422in}{5.148510in}}%
\pgfpathlineto{\pgfqpoint{12.923449in}{5.149334in}}%
\pgfpathlineto{\pgfqpoint{12.925463in}{5.140950in}}%
\pgfpathlineto{\pgfqpoint{12.927476in}{5.145292in}}%
\pgfpathlineto{\pgfqpoint{12.933517in}{5.145067in}}%
\pgfpathlineto{\pgfqpoint{12.935530in}{5.144019in}}%
\pgfpathlineto{\pgfqpoint{12.939557in}{5.152852in}}%
\pgfpathlineto{\pgfqpoint{12.941570in}{5.152178in}}%
\pgfpathlineto{\pgfqpoint{12.947611in}{5.151729in}}%
\pgfpathlineto{\pgfqpoint{12.949624in}{5.153899in}}%
\pgfpathlineto{\pgfqpoint{12.951638in}{5.156893in}}%
\pgfpathlineto{\pgfqpoint{12.953651in}{5.152702in}}%
\pgfpathlineto{\pgfqpoint{12.955665in}{5.154423in}}%
\pgfpathlineto{\pgfqpoint{12.961705in}{5.150157in}}%
\pgfpathlineto{\pgfqpoint{12.963718in}{5.153076in}}%
\pgfpathlineto{\pgfqpoint{12.965732in}{5.152328in}}%
\pgfpathlineto{\pgfqpoint{12.967745in}{5.141774in}}%
\pgfpathlineto{\pgfqpoint{12.969759in}{5.148660in}}%
\pgfpathlineto{\pgfqpoint{12.975799in}{5.151804in}}%
\pgfpathlineto{\pgfqpoint{12.979826in}{5.152328in}}%
\pgfpathlineto{\pgfqpoint{12.981839in}{5.154049in}}%
\pgfpathlineto{\pgfqpoint{12.983853in}{5.157193in}}%
\pgfpathlineto{\pgfqpoint{12.989893in}{5.156295in}}%
\pgfpathlineto{\pgfqpoint{12.991907in}{5.156968in}}%
\pgfpathlineto{\pgfqpoint{12.993920in}{5.155172in}}%
\pgfpathlineto{\pgfqpoint{12.995933in}{5.146938in}}%
\pgfpathlineto{\pgfqpoint{12.997947in}{5.144992in}}%
\pgfpathlineto{\pgfqpoint{13.003987in}{5.153525in}}%
\pgfpathlineto{\pgfqpoint{13.006001in}{5.163181in}}%
\pgfpathlineto{\pgfqpoint{13.008014in}{5.167522in}}%
\pgfpathlineto{\pgfqpoint{13.010028in}{5.158091in}}%
\pgfpathlineto{\pgfqpoint{13.012041in}{5.152926in}}%
\pgfpathlineto{\pgfqpoint{13.022108in}{5.152178in}}%
\pgfpathlineto{\pgfqpoint{13.026135in}{5.153525in}}%
\pgfpathlineto{\pgfqpoint{13.034189in}{5.153151in}}%
\pgfpathlineto{\pgfqpoint{13.036202in}{5.155172in}}%
\pgfpathlineto{\pgfqpoint{13.038216in}{5.158091in}}%
\pgfpathlineto{\pgfqpoint{13.040229in}{5.158390in}}%
\pgfpathlineto{\pgfqpoint{13.048283in}{5.152402in}}%
\pgfpathlineto{\pgfqpoint{13.050297in}{5.151804in}}%
\pgfpathlineto{\pgfqpoint{13.052310in}{5.145816in}}%
\pgfpathlineto{\pgfqpoint{13.054323in}{5.144693in}}%
\pgfpathlineto{\pgfqpoint{13.060364in}{5.155172in}}%
\pgfpathlineto{\pgfqpoint{13.062377in}{5.161384in}}%
\pgfpathlineto{\pgfqpoint{13.064391in}{5.162058in}}%
\pgfpathlineto{\pgfqpoint{13.066404in}{5.158765in}}%
\pgfpathlineto{\pgfqpoint{13.068418in}{5.164603in}}%
\pgfpathlineto{\pgfqpoint{13.074458in}{5.170965in}}%
\pgfpathlineto{\pgfqpoint{13.076471in}{5.179199in}}%
\pgfpathlineto{\pgfqpoint{13.078485in}{5.175082in}}%
\pgfpathlineto{\pgfqpoint{13.082512in}{5.174858in}}%
\pgfpathlineto{\pgfqpoint{13.088552in}{5.173435in}}%
\pgfpathlineto{\pgfqpoint{13.090566in}{5.176953in}}%
\pgfpathlineto{\pgfqpoint{13.094592in}{5.187507in}}%
\pgfpathlineto{\pgfqpoint{13.096606in}{5.189828in}}%
\pgfpathlineto{\pgfqpoint{13.102646in}{5.190426in}}%
\pgfpathlineto{\pgfqpoint{13.104660in}{5.196864in}}%
\pgfpathlineto{\pgfqpoint{13.106673in}{5.193795in}}%
\pgfpathlineto{\pgfqpoint{13.110700in}{5.200307in}}%
\pgfpathlineto{\pgfqpoint{13.116740in}{5.201355in}}%
\pgfpathlineto{\pgfqpoint{13.118754in}{5.202777in}}%
\pgfpathlineto{\pgfqpoint{13.120767in}{5.203301in}}%
\pgfpathlineto{\pgfqpoint{13.122781in}{5.200756in}}%
\pgfpathlineto{\pgfqpoint{13.124794in}{5.209888in}}%
\pgfpathlineto{\pgfqpoint{13.132848in}{5.201355in}}%
\pgfpathlineto{\pgfqpoint{13.134861in}{5.204199in}}%
\pgfpathlineto{\pgfqpoint{13.136875in}{5.202852in}}%
\pgfpathlineto{\pgfqpoint{13.138888in}{5.204274in}}%
\pgfpathlineto{\pgfqpoint{13.144929in}{5.206370in}}%
\pgfpathlineto{\pgfqpoint{13.146942in}{5.217148in}}%
\pgfpathlineto{\pgfqpoint{13.148955in}{5.214753in}}%
\pgfpathlineto{\pgfqpoint{13.150969in}{5.230546in}}%
\pgfpathlineto{\pgfqpoint{13.152982in}{5.231295in}}%
\pgfpathlineto{\pgfqpoint{13.159023in}{5.225980in}}%
\pgfpathlineto{\pgfqpoint{13.161036in}{5.229274in}}%
\pgfpathlineto{\pgfqpoint{13.167077in}{5.235936in}}%
\pgfpathlineto{\pgfqpoint{13.173117in}{5.234888in}}%
\pgfpathlineto{\pgfqpoint{13.175130in}{5.228226in}}%
\pgfpathlineto{\pgfqpoint{13.177144in}{5.226430in}}%
\pgfpathlineto{\pgfqpoint{13.179157in}{5.216400in}}%
\pgfpathlineto{\pgfqpoint{13.181171in}{5.214678in}}%
\pgfpathlineto{\pgfqpoint{13.187211in}{5.217447in}}%
\pgfpathlineto{\pgfqpoint{13.189224in}{5.216474in}}%
\pgfpathlineto{\pgfqpoint{13.191238in}{5.212807in}}%
\pgfpathlineto{\pgfqpoint{13.193251in}{5.214977in}}%
\pgfpathlineto{\pgfqpoint{13.195265in}{5.215950in}}%
\pgfpathlineto{\pgfqpoint{13.201305in}{5.217822in}}%
\pgfpathlineto{\pgfqpoint{13.203319in}{5.221190in}}%
\pgfpathlineto{\pgfqpoint{13.205332in}{5.216924in}}%
\pgfpathlineto{\pgfqpoint{13.209359in}{5.214379in}}%
\pgfpathlineto{\pgfqpoint{13.215399in}{5.214304in}}%
\pgfpathlineto{\pgfqpoint{13.219426in}{5.237208in}}%
\pgfpathlineto{\pgfqpoint{13.221440in}{5.245292in}}%
\pgfpathlineto{\pgfqpoint{13.223453in}{5.246415in}}%
\pgfpathlineto{\pgfqpoint{13.229493in}{5.251729in}}%
\pgfpathlineto{\pgfqpoint{13.231507in}{5.252552in}}%
\pgfpathlineto{\pgfqpoint{13.233520in}{5.248960in}}%
\pgfpathlineto{\pgfqpoint{13.235534in}{5.251654in}}%
\pgfpathlineto{\pgfqpoint{13.237547in}{5.251430in}}%
\pgfpathlineto{\pgfqpoint{13.243587in}{5.254723in}}%
\pgfpathlineto{\pgfqpoint{13.245601in}{5.257418in}}%
\pgfpathlineto{\pgfqpoint{13.247614in}{5.245816in}}%
\pgfpathlineto{\pgfqpoint{13.249628in}{5.241175in}}%
\pgfpathlineto{\pgfqpoint{13.251641in}{5.251205in}}%
\pgfpathlineto{\pgfqpoint{13.257682in}{5.259663in}}%
\pgfpathlineto{\pgfqpoint{13.261709in}{5.251130in}}%
\pgfpathlineto{\pgfqpoint{13.263722in}{5.251055in}}%
\pgfpathlineto{\pgfqpoint{13.265735in}{5.252777in}}%
\pgfpathlineto{\pgfqpoint{13.273789in}{5.251579in}}%
\pgfpathlineto{\pgfqpoint{13.277816in}{5.259813in}}%
\pgfpathlineto{\pgfqpoint{13.279830in}{5.256969in}}%
\pgfpathlineto{\pgfqpoint{13.279830in}{5.256969in}}%
\pgfusepath{stroke}%
\end{pgfscope}%
\begin{pgfscope}%
\pgfsetrectcap%
\pgfsetmiterjoin%
\pgfsetlinewidth{0.803000pt}%
\definecolor{currentstroke}{rgb}{1.000000,1.000000,1.000000}%
\pgfsetstrokecolor{currentstroke}%
\pgfsetdash{}{0pt}%
\pgfpathmoveto{\pgfqpoint{8.656250in}{4.835882in}}%
\pgfpathlineto{\pgfqpoint{8.656250in}{5.280000in}}%
\pgfusepath{stroke}%
\end{pgfscope}%
\begin{pgfscope}%
\pgfsetrectcap%
\pgfsetmiterjoin%
\pgfsetlinewidth{0.803000pt}%
\definecolor{currentstroke}{rgb}{1.000000,1.000000,1.000000}%
\pgfsetstrokecolor{currentstroke}%
\pgfsetdash{}{0pt}%
\pgfpathmoveto{\pgfqpoint{13.500000in}{4.835882in}}%
\pgfpathlineto{\pgfqpoint{13.500000in}{5.280000in}}%
\pgfusepath{stroke}%
\end{pgfscope}%
\begin{pgfscope}%
\pgfsetrectcap%
\pgfsetmiterjoin%
\pgfsetlinewidth{0.803000pt}%
\definecolor{currentstroke}{rgb}{1.000000,1.000000,1.000000}%
\pgfsetstrokecolor{currentstroke}%
\pgfsetdash{}{0pt}%
\pgfpathmoveto{\pgfqpoint{8.656250in}{4.835882in}}%
\pgfpathlineto{\pgfqpoint{13.500000in}{4.835882in}}%
\pgfusepath{stroke}%
\end{pgfscope}%
\begin{pgfscope}%
\pgfsetrectcap%
\pgfsetmiterjoin%
\pgfsetlinewidth{0.803000pt}%
\definecolor{currentstroke}{rgb}{1.000000,1.000000,1.000000}%
\pgfsetstrokecolor{currentstroke}%
\pgfsetdash{}{0pt}%
\pgfpathmoveto{\pgfqpoint{8.656250in}{5.280000in}}%
\pgfpathlineto{\pgfqpoint{13.500000in}{5.280000in}}%
\pgfusepath{stroke}%
\end{pgfscope}%
\begin{pgfscope}%
\definecolor{textcolor}{rgb}{0.150000,0.150000,0.150000}%
\pgfsetstrokecolor{textcolor}%
\pgfsetfillcolor{textcolor}%
\pgftext[x=11.078125in,y=5.363333in,,base]{\color{textcolor}\rmfamily\fontsize{16.800000}{20.160000}\selectfont AXP}%
\end{pgfscope}%
\begin{pgfscope}%
\pgfsetbuttcap%
\pgfsetmiterjoin%
\definecolor{currentfill}{rgb}{0.917647,0.917647,0.949020}%
\pgfsetfillcolor{currentfill}%
\pgfsetlinewidth{0.000000pt}%
\definecolor{currentstroke}{rgb}{0.000000,0.000000,0.000000}%
\pgfsetstrokecolor{currentstroke}%
\pgfsetstrokeopacity{0.000000}%
\pgfsetdash{}{0pt}%
\pgfpathmoveto{\pgfqpoint{1.875000in}{3.814412in}}%
\pgfpathlineto{\pgfqpoint{6.718750in}{3.814412in}}%
\pgfpathlineto{\pgfqpoint{6.718750in}{4.258529in}}%
\pgfpathlineto{\pgfqpoint{1.875000in}{4.258529in}}%
\pgfpathclose%
\pgfusepath{fill}%
\end{pgfscope}%
\begin{pgfscope}%
\pgfpathrectangle{\pgfqpoint{1.875000in}{3.814412in}}{\pgfqpoint{4.843750in}{0.444118in}}%
\pgfusepath{clip}%
\pgfsetroundcap%
\pgfsetroundjoin%
\pgfsetlinewidth{0.803000pt}%
\definecolor{currentstroke}{rgb}{1.000000,1.000000,1.000000}%
\pgfsetstrokecolor{currentstroke}%
\pgfsetdash{}{0pt}%
\pgfpathmoveto{\pgfqpoint{2.091144in}{3.814412in}}%
\pgfpathlineto{\pgfqpoint{2.091144in}{4.258529in}}%
\pgfusepath{stroke}%
\end{pgfscope}%
\begin{pgfscope}%
\definecolor{textcolor}{rgb}{0.150000,0.150000,0.150000}%
\pgfsetstrokecolor{textcolor}%
\pgfsetfillcolor{textcolor}%
\pgftext[x=2.091144in,y=3.717190in,,top]{\color{textcolor}\rmfamily\fontsize{14.000000}{16.800000}\selectfont 2012}%
\end{pgfscope}%
\begin{pgfscope}%
\pgfpathrectangle{\pgfqpoint{1.875000in}{3.814412in}}{\pgfqpoint{4.843750in}{0.444118in}}%
\pgfusepath{clip}%
\pgfsetroundcap%
\pgfsetroundjoin%
\pgfsetlinewidth{0.803000pt}%
\definecolor{currentstroke}{rgb}{1.000000,1.000000,1.000000}%
\pgfsetstrokecolor{currentstroke}%
\pgfsetdash{}{0pt}%
\pgfpathmoveto{\pgfqpoint{2.828065in}{3.814412in}}%
\pgfpathlineto{\pgfqpoint{2.828065in}{4.258529in}}%
\pgfusepath{stroke}%
\end{pgfscope}%
\begin{pgfscope}%
\definecolor{textcolor}{rgb}{0.150000,0.150000,0.150000}%
\pgfsetstrokecolor{textcolor}%
\pgfsetfillcolor{textcolor}%
\pgftext[x=2.828065in,y=3.717190in,,top]{\color{textcolor}\rmfamily\fontsize{14.000000}{16.800000}\selectfont 2013}%
\end{pgfscope}%
\begin{pgfscope}%
\pgfpathrectangle{\pgfqpoint{1.875000in}{3.814412in}}{\pgfqpoint{4.843750in}{0.444118in}}%
\pgfusepath{clip}%
\pgfsetroundcap%
\pgfsetroundjoin%
\pgfsetlinewidth{0.803000pt}%
\definecolor{currentstroke}{rgb}{1.000000,1.000000,1.000000}%
\pgfsetstrokecolor{currentstroke}%
\pgfsetdash{}{0pt}%
\pgfpathmoveto{\pgfqpoint{3.562973in}{3.814412in}}%
\pgfpathlineto{\pgfqpoint{3.562973in}{4.258529in}}%
\pgfusepath{stroke}%
\end{pgfscope}%
\begin{pgfscope}%
\definecolor{textcolor}{rgb}{0.150000,0.150000,0.150000}%
\pgfsetstrokecolor{textcolor}%
\pgfsetfillcolor{textcolor}%
\pgftext[x=3.562973in,y=3.717190in,,top]{\color{textcolor}\rmfamily\fontsize{14.000000}{16.800000}\selectfont 2014}%
\end{pgfscope}%
\begin{pgfscope}%
\pgfpathrectangle{\pgfqpoint{1.875000in}{3.814412in}}{\pgfqpoint{4.843750in}{0.444118in}}%
\pgfusepath{clip}%
\pgfsetroundcap%
\pgfsetroundjoin%
\pgfsetlinewidth{0.803000pt}%
\definecolor{currentstroke}{rgb}{1.000000,1.000000,1.000000}%
\pgfsetstrokecolor{currentstroke}%
\pgfsetdash{}{0pt}%
\pgfpathmoveto{\pgfqpoint{4.297882in}{3.814412in}}%
\pgfpathlineto{\pgfqpoint{4.297882in}{4.258529in}}%
\pgfusepath{stroke}%
\end{pgfscope}%
\begin{pgfscope}%
\definecolor{textcolor}{rgb}{0.150000,0.150000,0.150000}%
\pgfsetstrokecolor{textcolor}%
\pgfsetfillcolor{textcolor}%
\pgftext[x=4.297882in,y=3.717190in,,top]{\color{textcolor}\rmfamily\fontsize{14.000000}{16.800000}\selectfont 2015}%
\end{pgfscope}%
\begin{pgfscope}%
\pgfpathrectangle{\pgfqpoint{1.875000in}{3.814412in}}{\pgfqpoint{4.843750in}{0.444118in}}%
\pgfusepath{clip}%
\pgfsetroundcap%
\pgfsetroundjoin%
\pgfsetlinewidth{0.803000pt}%
\definecolor{currentstroke}{rgb}{1.000000,1.000000,1.000000}%
\pgfsetstrokecolor{currentstroke}%
\pgfsetdash{}{0pt}%
\pgfpathmoveto{\pgfqpoint{5.032790in}{3.814412in}}%
\pgfpathlineto{\pgfqpoint{5.032790in}{4.258529in}}%
\pgfusepath{stroke}%
\end{pgfscope}%
\begin{pgfscope}%
\definecolor{textcolor}{rgb}{0.150000,0.150000,0.150000}%
\pgfsetstrokecolor{textcolor}%
\pgfsetfillcolor{textcolor}%
\pgftext[x=5.032790in,y=3.717190in,,top]{\color{textcolor}\rmfamily\fontsize{14.000000}{16.800000}\selectfont 2016}%
\end{pgfscope}%
\begin{pgfscope}%
\pgfpathrectangle{\pgfqpoint{1.875000in}{3.814412in}}{\pgfqpoint{4.843750in}{0.444118in}}%
\pgfusepath{clip}%
\pgfsetroundcap%
\pgfsetroundjoin%
\pgfsetlinewidth{0.803000pt}%
\definecolor{currentstroke}{rgb}{1.000000,1.000000,1.000000}%
\pgfsetstrokecolor{currentstroke}%
\pgfsetdash{}{0pt}%
\pgfpathmoveto{\pgfqpoint{5.769712in}{3.814412in}}%
\pgfpathlineto{\pgfqpoint{5.769712in}{4.258529in}}%
\pgfusepath{stroke}%
\end{pgfscope}%
\begin{pgfscope}%
\definecolor{textcolor}{rgb}{0.150000,0.150000,0.150000}%
\pgfsetstrokecolor{textcolor}%
\pgfsetfillcolor{textcolor}%
\pgftext[x=5.769712in,y=3.717190in,,top]{\color{textcolor}\rmfamily\fontsize{14.000000}{16.800000}\selectfont 2017}%
\end{pgfscope}%
\begin{pgfscope}%
\pgfpathrectangle{\pgfqpoint{1.875000in}{3.814412in}}{\pgfqpoint{4.843750in}{0.444118in}}%
\pgfusepath{clip}%
\pgfsetroundcap%
\pgfsetroundjoin%
\pgfsetlinewidth{0.803000pt}%
\definecolor{currentstroke}{rgb}{1.000000,1.000000,1.000000}%
\pgfsetstrokecolor{currentstroke}%
\pgfsetdash{}{0pt}%
\pgfpathmoveto{\pgfqpoint{6.504620in}{3.814412in}}%
\pgfpathlineto{\pgfqpoint{6.504620in}{4.258529in}}%
\pgfusepath{stroke}%
\end{pgfscope}%
\begin{pgfscope}%
\definecolor{textcolor}{rgb}{0.150000,0.150000,0.150000}%
\pgfsetstrokecolor{textcolor}%
\pgfsetfillcolor{textcolor}%
\pgftext[x=6.504620in,y=3.717190in,,top]{\color{textcolor}\rmfamily\fontsize{14.000000}{16.800000}\selectfont 2018}%
\end{pgfscope}%
\begin{pgfscope}%
\pgfpathrectangle{\pgfqpoint{1.875000in}{3.814412in}}{\pgfqpoint{4.843750in}{0.444118in}}%
\pgfusepath{clip}%
\pgfsetroundcap%
\pgfsetroundjoin%
\pgfsetlinewidth{0.803000pt}%
\definecolor{currentstroke}{rgb}{1.000000,1.000000,1.000000}%
\pgfsetstrokecolor{currentstroke}%
\pgfsetdash{}{0pt}%
\pgfpathmoveto{\pgfqpoint{1.875000in}{3.868088in}}%
\pgfpathlineto{\pgfqpoint{6.718750in}{3.868088in}}%
\pgfusepath{stroke}%
\end{pgfscope}%
\begin{pgfscope}%
\definecolor{textcolor}{rgb}{0.150000,0.150000,0.150000}%
\pgfsetstrokecolor{textcolor}%
\pgfsetfillcolor{textcolor}%
\pgftext[x=1.530355in,y=3.794222in,left,base]{\color{textcolor}\rmfamily\fontsize{14.000000}{16.800000}\selectfont 15}%
\end{pgfscope}%
\begin{pgfscope}%
\pgfpathrectangle{\pgfqpoint{1.875000in}{3.814412in}}{\pgfqpoint{4.843750in}{0.444118in}}%
\pgfusepath{clip}%
\pgfsetroundcap%
\pgfsetroundjoin%
\pgfsetlinewidth{0.803000pt}%
\definecolor{currentstroke}{rgb}{1.000000,1.000000,1.000000}%
\pgfsetstrokecolor{currentstroke}%
\pgfsetdash{}{0pt}%
\pgfpathmoveto{\pgfqpoint{1.875000in}{4.002044in}}%
\pgfpathlineto{\pgfqpoint{6.718750in}{4.002044in}}%
\pgfusepath{stroke}%
\end{pgfscope}%
\begin{pgfscope}%
\definecolor{textcolor}{rgb}{0.150000,0.150000,0.150000}%
\pgfsetstrokecolor{textcolor}%
\pgfsetfillcolor{textcolor}%
\pgftext[x=1.530355in,y=3.928178in,left,base]{\color{textcolor}\rmfamily\fontsize{14.000000}{16.800000}\selectfont 20}%
\end{pgfscope}%
\begin{pgfscope}%
\pgfpathrectangle{\pgfqpoint{1.875000in}{3.814412in}}{\pgfqpoint{4.843750in}{0.444118in}}%
\pgfusepath{clip}%
\pgfsetroundcap%
\pgfsetroundjoin%
\pgfsetlinewidth{0.803000pt}%
\definecolor{currentstroke}{rgb}{1.000000,1.000000,1.000000}%
\pgfsetstrokecolor{currentstroke}%
\pgfsetdash{}{0pt}%
\pgfpathmoveto{\pgfqpoint{1.875000in}{4.136000in}}%
\pgfpathlineto{\pgfqpoint{6.718750in}{4.136000in}}%
\pgfusepath{stroke}%
\end{pgfscope}%
\begin{pgfscope}%
\definecolor{textcolor}{rgb}{0.150000,0.150000,0.150000}%
\pgfsetstrokecolor{textcolor}%
\pgfsetfillcolor{textcolor}%
\pgftext[x=1.530355in,y=4.062134in,left,base]{\color{textcolor}\rmfamily\fontsize{14.000000}{16.800000}\selectfont 25}%
\end{pgfscope}%
\begin{pgfscope}%
\pgfpathrectangle{\pgfqpoint{1.875000in}{3.814412in}}{\pgfqpoint{4.843750in}{0.444118in}}%
\pgfusepath{clip}%
\pgfsetroundcap%
\pgfsetroundjoin%
\pgfsetlinewidth{1.505625pt}%
\definecolor{currentstroke}{rgb}{0.121569,0.466667,0.705882}%
\pgfsetstrokecolor{currentstroke}%
\pgfsetdash{}{0pt}%
\pgfpathmoveto{\pgfqpoint{2.095170in}{3.835671in}}%
\pgfpathlineto{\pgfqpoint{2.097184in}{3.839689in}}%
\pgfpathlineto{\pgfqpoint{2.099197in}{3.839421in}}%
\pgfpathlineto{\pgfqpoint{2.101211in}{3.841565in}}%
\pgfpathlineto{\pgfqpoint{2.107251in}{3.845851in}}%
\pgfpathlineto{\pgfqpoint{2.109265in}{3.842904in}}%
\pgfpathlineto{\pgfqpoint{2.111278in}{3.846119in}}%
\pgfpathlineto{\pgfqpoint{2.113291in}{3.847191in}}%
\pgfpathlineto{\pgfqpoint{2.115305in}{3.845315in}}%
\pgfpathlineto{\pgfqpoint{2.123359in}{3.843440in}}%
\pgfpathlineto{\pgfqpoint{2.125372in}{3.849066in}}%
\pgfpathlineto{\pgfqpoint{2.127386in}{3.851477in}}%
\pgfpathlineto{\pgfqpoint{2.129399in}{3.851477in}}%
\pgfpathlineto{\pgfqpoint{2.135439in}{3.847459in}}%
\pgfpathlineto{\pgfqpoint{2.137453in}{3.845315in}}%
\pgfpathlineto{\pgfqpoint{2.139466in}{3.851209in}}%
\pgfpathlineto{\pgfqpoint{2.143493in}{3.849066in}}%
\pgfpathlineto{\pgfqpoint{2.149534in}{3.846655in}}%
\pgfpathlineto{\pgfqpoint{2.151547in}{3.842636in}}%
\pgfpathlineto{\pgfqpoint{2.153560in}{3.843976in}}%
\pgfpathlineto{\pgfqpoint{2.155574in}{3.843440in}}%
\pgfpathlineto{\pgfqpoint{2.157587in}{3.849066in}}%
\pgfpathlineto{\pgfqpoint{2.163628in}{3.849602in}}%
\pgfpathlineto{\pgfqpoint{2.165641in}{3.852281in}}%
\pgfpathlineto{\pgfqpoint{2.167655in}{3.853353in}}%
\pgfpathlineto{\pgfqpoint{2.169668in}{3.851209in}}%
\pgfpathlineto{\pgfqpoint{2.171681in}{3.846119in}}%
\pgfpathlineto{\pgfqpoint{2.177722in}{3.849870in}}%
\pgfpathlineto{\pgfqpoint{2.179735in}{3.847459in}}%
\pgfpathlineto{\pgfqpoint{2.181749in}{3.843708in}}%
\pgfpathlineto{\pgfqpoint{2.185776in}{3.854157in}}%
\pgfpathlineto{\pgfqpoint{2.193829in}{3.856836in}}%
\pgfpathlineto{\pgfqpoint{2.195843in}{3.856300in}}%
\pgfpathlineto{\pgfqpoint{2.197856in}{3.858175in}}%
\pgfpathlineto{\pgfqpoint{2.205910in}{3.853353in}}%
\pgfpathlineto{\pgfqpoint{2.207923in}{3.855228in}}%
\pgfpathlineto{\pgfqpoint{2.209937in}{3.853085in}}%
\pgfpathlineto{\pgfqpoint{2.211950in}{3.854424in}}%
\pgfpathlineto{\pgfqpoint{2.213964in}{3.851477in}}%
\pgfpathlineto{\pgfqpoint{2.220004in}{3.848798in}}%
\pgfpathlineto{\pgfqpoint{2.222018in}{3.840225in}}%
\pgfpathlineto{\pgfqpoint{2.226045in}{3.852549in}}%
\pgfpathlineto{\pgfqpoint{2.228058in}{3.852817in}}%
\pgfpathlineto{\pgfqpoint{2.234098in}{3.854692in}}%
\pgfpathlineto{\pgfqpoint{2.236112in}{3.864069in}}%
\pgfpathlineto{\pgfqpoint{2.238125in}{3.868088in}}%
\pgfpathlineto{\pgfqpoint{2.240139in}{3.875589in}}%
\pgfpathlineto{\pgfqpoint{2.242152in}{3.876393in}}%
\pgfpathlineto{\pgfqpoint{2.248192in}{3.876661in}}%
\pgfpathlineto{\pgfqpoint{2.250206in}{3.873714in}}%
\pgfpathlineto{\pgfqpoint{2.252219in}{3.873714in}}%
\pgfpathlineto{\pgfqpoint{2.254233in}{3.869160in}}%
\pgfpathlineto{\pgfqpoint{2.256246in}{3.867820in}}%
\pgfpathlineto{\pgfqpoint{2.262287in}{3.873178in}}%
\pgfpathlineto{\pgfqpoint{2.264300in}{3.873178in}}%
\pgfpathlineto{\pgfqpoint{2.268327in}{3.871303in}}%
\pgfpathlineto{\pgfqpoint{2.270340in}{3.873714in}}%
\pgfpathlineto{\pgfqpoint{2.276381in}{3.872642in}}%
\pgfpathlineto{\pgfqpoint{2.278394in}{3.871571in}}%
\pgfpathlineto{\pgfqpoint{2.282421in}{3.861926in}}%
\pgfpathlineto{\pgfqpoint{2.290475in}{3.856032in}}%
\pgfpathlineto{\pgfqpoint{2.292488in}{3.846655in}}%
\pgfpathlineto{\pgfqpoint{2.296515in}{3.858175in}}%
\pgfpathlineto{\pgfqpoint{2.298529in}{3.849602in}}%
\pgfpathlineto{\pgfqpoint{2.304569in}{3.849870in}}%
\pgfpathlineto{\pgfqpoint{2.306582in}{3.858979in}}%
\pgfpathlineto{\pgfqpoint{2.308596in}{3.853889in}}%
\pgfpathlineto{\pgfqpoint{2.310609in}{3.854692in}}%
\pgfpathlineto{\pgfqpoint{2.312623in}{3.859247in}}%
\pgfpathlineto{\pgfqpoint{2.318663in}{3.853353in}}%
\pgfpathlineto{\pgfqpoint{2.320677in}{3.862998in}}%
\pgfpathlineto{\pgfqpoint{2.322690in}{3.861122in}}%
\pgfpathlineto{\pgfqpoint{2.326717in}{3.867820in}}%
\pgfpathlineto{\pgfqpoint{2.332757in}{3.863801in}}%
\pgfpathlineto{\pgfqpoint{2.334771in}{3.868088in}}%
\pgfpathlineto{\pgfqpoint{2.336784in}{3.867552in}}%
\pgfpathlineto{\pgfqpoint{2.338798in}{3.864337in}}%
\pgfpathlineto{\pgfqpoint{2.340811in}{3.858979in}}%
\pgfpathlineto{\pgfqpoint{2.346851in}{3.858443in}}%
\pgfpathlineto{\pgfqpoint{2.348865in}{3.857104in}}%
\pgfpathlineto{\pgfqpoint{2.350878in}{3.850138in}}%
\pgfpathlineto{\pgfqpoint{2.352892in}{3.853889in}}%
\pgfpathlineto{\pgfqpoint{2.354905in}{3.852281in}}%
\pgfpathlineto{\pgfqpoint{2.360945in}{3.843708in}}%
\pgfpathlineto{\pgfqpoint{2.362959in}{3.839689in}}%
\pgfpathlineto{\pgfqpoint{2.364972in}{3.852013in}}%
\pgfpathlineto{\pgfqpoint{2.366986in}{3.849602in}}%
\pgfpathlineto{\pgfqpoint{2.375040in}{3.854424in}}%
\pgfpathlineto{\pgfqpoint{2.377053in}{3.855496in}}%
\pgfpathlineto{\pgfqpoint{2.379067in}{3.855496in}}%
\pgfpathlineto{\pgfqpoint{2.381080in}{3.857104in}}%
\pgfpathlineto{\pgfqpoint{2.383093in}{3.856032in}}%
\pgfpathlineto{\pgfqpoint{2.391147in}{3.858979in}}%
\pgfpathlineto{\pgfqpoint{2.393161in}{3.852817in}}%
\pgfpathlineto{\pgfqpoint{2.395174in}{3.853889in}}%
\pgfpathlineto{\pgfqpoint{2.397188in}{3.842636in}}%
\pgfpathlineto{\pgfqpoint{2.403228in}{3.834599in}}%
\pgfpathlineto{\pgfqpoint{2.405241in}{3.836474in}}%
\pgfpathlineto{\pgfqpoint{2.407255in}{3.849602in}}%
\pgfpathlineto{\pgfqpoint{2.409268in}{3.852013in}}%
\pgfpathlineto{\pgfqpoint{2.411282in}{3.856032in}}%
\pgfpathlineto{\pgfqpoint{2.417322in}{3.854157in}}%
\pgfpathlineto{\pgfqpoint{2.419335in}{3.861658in}}%
\pgfpathlineto{\pgfqpoint{2.421349in}{3.859515in}}%
\pgfpathlineto{\pgfqpoint{2.425376in}{3.873446in}}%
\pgfpathlineto{\pgfqpoint{2.431416in}{3.868356in}}%
\pgfpathlineto{\pgfqpoint{2.433430in}{3.873446in}}%
\pgfpathlineto{\pgfqpoint{2.435443in}{3.875322in}}%
\pgfpathlineto{\pgfqpoint{2.437456in}{3.867284in}}%
\pgfpathlineto{\pgfqpoint{2.439470in}{3.872910in}}%
\pgfpathlineto{\pgfqpoint{2.445510in}{3.867016in}}%
\pgfpathlineto{\pgfqpoint{2.449537in}{3.879608in}}%
\pgfpathlineto{\pgfqpoint{2.451551in}{3.880948in}}%
\pgfpathlineto{\pgfqpoint{2.453564in}{3.894075in}}%
\pgfpathlineto{\pgfqpoint{2.459604in}{3.886842in}}%
\pgfpathlineto{\pgfqpoint{2.465645in}{3.883627in}}%
\pgfpathlineto{\pgfqpoint{2.467658in}{3.876929in}}%
\pgfpathlineto{\pgfqpoint{2.473699in}{3.877733in}}%
\pgfpathlineto{\pgfqpoint{2.475712in}{3.869160in}}%
\pgfpathlineto{\pgfqpoint{2.477725in}{3.870231in}}%
\pgfpathlineto{\pgfqpoint{2.479739in}{3.865409in}}%
\pgfpathlineto{\pgfqpoint{2.481752in}{3.872107in}}%
\pgfpathlineto{\pgfqpoint{2.487793in}{3.868356in}}%
\pgfpathlineto{\pgfqpoint{2.491820in}{3.873714in}}%
\pgfpathlineto{\pgfqpoint{2.493833in}{3.872642in}}%
\pgfpathlineto{\pgfqpoint{2.501887in}{3.878804in}}%
\pgfpathlineto{\pgfqpoint{2.503900in}{3.876125in}}%
\pgfpathlineto{\pgfqpoint{2.505914in}{3.876929in}}%
\pgfpathlineto{\pgfqpoint{2.507927in}{3.888449in}}%
\pgfpathlineto{\pgfqpoint{2.509941in}{3.895683in}}%
\pgfpathlineto{\pgfqpoint{2.520008in}{3.891932in}}%
\pgfpathlineto{\pgfqpoint{2.522021in}{3.887645in}}%
\pgfpathlineto{\pgfqpoint{2.524035in}{3.896487in}}%
\pgfpathlineto{\pgfqpoint{2.530075in}{3.896755in}}%
\pgfpathlineto{\pgfqpoint{2.532089in}{3.899969in}}%
\pgfpathlineto{\pgfqpoint{2.534102in}{3.897558in}}%
\pgfpathlineto{\pgfqpoint{2.538129in}{3.899434in}}%
\pgfpathlineto{\pgfqpoint{2.546183in}{3.896219in}}%
\pgfpathlineto{\pgfqpoint{2.548196in}{3.896487in}}%
\pgfpathlineto{\pgfqpoint{2.550210in}{3.898362in}}%
\pgfpathlineto{\pgfqpoint{2.552223in}{3.897290in}}%
\pgfpathlineto{\pgfqpoint{2.558263in}{3.895951in}}%
\pgfpathlineto{\pgfqpoint{2.562290in}{3.893004in}}%
\pgfpathlineto{\pgfqpoint{2.564304in}{3.890057in}}%
\pgfpathlineto{\pgfqpoint{2.566317in}{3.893272in}}%
\pgfpathlineto{\pgfqpoint{2.572357in}{3.894343in}}%
\pgfpathlineto{\pgfqpoint{2.574371in}{3.893540in}}%
\pgfpathlineto{\pgfqpoint{2.576384in}{3.893807in}}%
\pgfpathlineto{\pgfqpoint{2.578398in}{3.890057in}}%
\pgfpathlineto{\pgfqpoint{2.580411in}{3.891396in}}%
\pgfpathlineto{\pgfqpoint{2.588465in}{3.887378in}}%
\pgfpathlineto{\pgfqpoint{2.590478in}{3.890325in}}%
\pgfpathlineto{\pgfqpoint{2.592492in}{3.903720in}}%
\pgfpathlineto{\pgfqpoint{2.594505in}{3.909614in}}%
\pgfpathlineto{\pgfqpoint{2.600546in}{3.907203in}}%
\pgfpathlineto{\pgfqpoint{2.602559in}{3.909614in}}%
\pgfpathlineto{\pgfqpoint{2.604573in}{3.915776in}}%
\pgfpathlineto{\pgfqpoint{2.606586in}{3.918455in}}%
\pgfpathlineto{\pgfqpoint{2.608599in}{3.920063in}}%
\pgfpathlineto{\pgfqpoint{2.614640in}{3.918991in}}%
\pgfpathlineto{\pgfqpoint{2.620680in}{3.930243in}}%
\pgfpathlineto{\pgfqpoint{2.622694in}{3.932387in}}%
\pgfpathlineto{\pgfqpoint{2.630747in}{3.927832in}}%
\pgfpathlineto{\pgfqpoint{2.632761in}{3.923546in}}%
\pgfpathlineto{\pgfqpoint{2.634774in}{3.936405in}}%
\pgfpathlineto{\pgfqpoint{2.636788in}{3.936138in}}%
\pgfpathlineto{\pgfqpoint{2.642828in}{3.938013in}}%
\pgfpathlineto{\pgfqpoint{2.644842in}{3.937745in}}%
\pgfpathlineto{\pgfqpoint{2.646855in}{3.940156in}}%
\pgfpathlineto{\pgfqpoint{2.648868in}{3.940960in}}%
\pgfpathlineto{\pgfqpoint{2.650882in}{3.944443in}}%
\pgfpathlineto{\pgfqpoint{2.656922in}{3.940424in}}%
\pgfpathlineto{\pgfqpoint{2.658936in}{3.934262in}}%
\pgfpathlineto{\pgfqpoint{2.660949in}{3.930243in}}%
\pgfpathlineto{\pgfqpoint{2.662963in}{3.931851in}}%
\pgfpathlineto{\pgfqpoint{2.664976in}{3.931315in}}%
\pgfpathlineto{\pgfqpoint{2.671016in}{3.934530in}}%
\pgfpathlineto{\pgfqpoint{2.673030in}{3.934530in}}%
\pgfpathlineto{\pgfqpoint{2.675043in}{3.940156in}}%
\pgfpathlineto{\pgfqpoint{2.677057in}{3.938013in}}%
\pgfpathlineto{\pgfqpoint{2.679070in}{3.921938in}}%
\pgfpathlineto{\pgfqpoint{2.685110in}{3.915240in}}%
\pgfpathlineto{\pgfqpoint{2.687124in}{3.906399in}}%
\pgfpathlineto{\pgfqpoint{2.691151in}{3.906131in}}%
\pgfpathlineto{\pgfqpoint{2.693164in}{3.902916in}}%
\pgfpathlineto{\pgfqpoint{2.703232in}{3.901845in}}%
\pgfpathlineto{\pgfqpoint{2.705245in}{3.907739in}}%
\pgfpathlineto{\pgfqpoint{2.707258in}{3.907203in}}%
\pgfpathlineto{\pgfqpoint{2.713299in}{3.909078in}}%
\pgfpathlineto{\pgfqpoint{2.715312in}{3.912829in}}%
\pgfpathlineto{\pgfqpoint{2.717326in}{3.903452in}}%
\pgfpathlineto{\pgfqpoint{2.719339in}{3.898362in}}%
\pgfpathlineto{\pgfqpoint{2.721353in}{3.900773in}}%
\pgfpathlineto{\pgfqpoint{2.727393in}{3.898362in}}%
\pgfpathlineto{\pgfqpoint{2.729406in}{3.894075in}}%
\pgfpathlineto{\pgfqpoint{2.731420in}{3.880144in}}%
\pgfpathlineto{\pgfqpoint{2.733433in}{3.881216in}}%
\pgfpathlineto{\pgfqpoint{2.735447in}{3.883091in}}%
\pgfpathlineto{\pgfqpoint{2.741487in}{3.893540in}}%
\pgfpathlineto{\pgfqpoint{2.743500in}{3.892736in}}%
\pgfpathlineto{\pgfqpoint{2.745514in}{3.894075in}}%
\pgfpathlineto{\pgfqpoint{2.749541in}{3.901577in}}%
\pgfpathlineto{\pgfqpoint{2.755581in}{3.901845in}}%
\pgfpathlineto{\pgfqpoint{2.757595in}{3.898094in}}%
\pgfpathlineto{\pgfqpoint{2.759608in}{3.903720in}}%
\pgfpathlineto{\pgfqpoint{2.763635in}{3.903452in}}%
\pgfpathlineto{\pgfqpoint{2.769675in}{3.897022in}}%
\pgfpathlineto{\pgfqpoint{2.771689in}{3.897826in}}%
\pgfpathlineto{\pgfqpoint{2.773702in}{3.905328in}}%
\pgfpathlineto{\pgfqpoint{2.775716in}{3.908275in}}%
\pgfpathlineto{\pgfqpoint{2.777729in}{3.910150in}}%
\pgfpathlineto{\pgfqpoint{2.783769in}{3.908811in}}%
\pgfpathlineto{\pgfqpoint{2.785783in}{3.911222in}}%
\pgfpathlineto{\pgfqpoint{2.787796in}{3.916848in}}%
\pgfpathlineto{\pgfqpoint{2.789810in}{3.913633in}}%
\pgfpathlineto{\pgfqpoint{2.791823in}{3.913633in}}%
\pgfpathlineto{\pgfqpoint{2.797864in}{3.920063in}}%
\pgfpathlineto{\pgfqpoint{2.799877in}{3.914973in}}%
\pgfpathlineto{\pgfqpoint{2.801890in}{3.900773in}}%
\pgfpathlineto{\pgfqpoint{2.803904in}{3.905596in}}%
\pgfpathlineto{\pgfqpoint{2.805917in}{3.902113in}}%
\pgfpathlineto{\pgfqpoint{2.815985in}{3.899969in}}%
\pgfpathlineto{\pgfqpoint{2.817998in}{3.898094in}}%
\pgfpathlineto{\pgfqpoint{2.820011in}{3.893004in}}%
\pgfpathlineto{\pgfqpoint{2.830079in}{3.911758in}}%
\pgfpathlineto{\pgfqpoint{2.832092in}{3.906667in}}%
\pgfpathlineto{\pgfqpoint{2.834106in}{3.908811in}}%
\pgfpathlineto{\pgfqpoint{2.840146in}{3.907471in}}%
\pgfpathlineto{\pgfqpoint{2.842159in}{3.902649in}}%
\pgfpathlineto{\pgfqpoint{2.844173in}{3.903720in}}%
\pgfpathlineto{\pgfqpoint{2.846186in}{3.908275in}}%
\pgfpathlineto{\pgfqpoint{2.848200in}{3.907471in}}%
\pgfpathlineto{\pgfqpoint{2.854240in}{3.907203in}}%
\pgfpathlineto{\pgfqpoint{2.856254in}{3.908811in}}%
\pgfpathlineto{\pgfqpoint{2.858267in}{3.907203in}}%
\pgfpathlineto{\pgfqpoint{2.860280in}{3.910954in}}%
\pgfpathlineto{\pgfqpoint{2.862294in}{3.926493in}}%
\pgfpathlineto{\pgfqpoint{2.870348in}{3.925689in}}%
\pgfpathlineto{\pgfqpoint{2.872361in}{3.924349in}}%
\pgfpathlineto{\pgfqpoint{2.874375in}{3.926493in}}%
\pgfpathlineto{\pgfqpoint{2.876388in}{3.931583in}}%
\pgfpathlineto{\pgfqpoint{2.882428in}{3.935870in}}%
\pgfpathlineto{\pgfqpoint{2.884442in}{3.935870in}}%
\pgfpathlineto{\pgfqpoint{2.886455in}{3.930243in}}%
\pgfpathlineto{\pgfqpoint{2.888469in}{3.931315in}}%
\pgfpathlineto{\pgfqpoint{2.890482in}{3.938549in}}%
\pgfpathlineto{\pgfqpoint{2.896522in}{3.932119in}}%
\pgfpathlineto{\pgfqpoint{2.898536in}{3.936941in}}%
\pgfpathlineto{\pgfqpoint{2.900549in}{3.934798in}}%
\pgfpathlineto{\pgfqpoint{2.904576in}{3.935870in}}%
\pgfpathlineto{\pgfqpoint{2.910617in}{3.935066in}}%
\pgfpathlineto{\pgfqpoint{2.912630in}{3.937745in}}%
\pgfpathlineto{\pgfqpoint{2.914643in}{3.954623in}}%
\pgfpathlineto{\pgfqpoint{2.916657in}{3.954891in}}%
\pgfpathlineto{\pgfqpoint{2.918670in}{3.952480in}}%
\pgfpathlineto{\pgfqpoint{2.926724in}{3.962125in}}%
\pgfpathlineto{\pgfqpoint{2.928738in}{3.954891in}}%
\pgfpathlineto{\pgfqpoint{2.930751in}{3.955695in}}%
\pgfpathlineto{\pgfqpoint{2.932765in}{3.958642in}}%
\pgfpathlineto{\pgfqpoint{2.938805in}{3.946318in}}%
\pgfpathlineto{\pgfqpoint{2.940818in}{3.951409in}}%
\pgfpathlineto{\pgfqpoint{2.942832in}{3.958106in}}%
\pgfpathlineto{\pgfqpoint{2.944845in}{3.954891in}}%
\pgfpathlineto{\pgfqpoint{2.946859in}{3.954356in}}%
\pgfpathlineto{\pgfqpoint{2.952899in}{3.955963in}}%
\pgfpathlineto{\pgfqpoint{2.954912in}{3.962661in}}%
\pgfpathlineto{\pgfqpoint{2.956926in}{3.964536in}}%
\pgfpathlineto{\pgfqpoint{2.958939in}{3.964536in}}%
\pgfpathlineto{\pgfqpoint{2.960953in}{3.966680in}}%
\pgfpathlineto{\pgfqpoint{2.966993in}{3.963465in}}%
\pgfpathlineto{\pgfqpoint{2.969007in}{3.958910in}}%
\pgfpathlineto{\pgfqpoint{2.971020in}{3.960785in}}%
\pgfpathlineto{\pgfqpoint{2.973033in}{3.964804in}}%
\pgfpathlineto{\pgfqpoint{2.975047in}{3.959714in}}%
\pgfpathlineto{\pgfqpoint{2.981087in}{3.955695in}}%
\pgfpathlineto{\pgfqpoint{2.983101in}{3.957035in}}%
\pgfpathlineto{\pgfqpoint{2.985114in}{3.959982in}}%
\pgfpathlineto{\pgfqpoint{2.987128in}{3.956499in}}%
\pgfpathlineto{\pgfqpoint{2.989141in}{3.958106in}}%
\pgfpathlineto{\pgfqpoint{2.995181in}{3.955427in}}%
\pgfpathlineto{\pgfqpoint{2.997195in}{3.952748in}}%
\pgfpathlineto{\pgfqpoint{3.001222in}{3.952748in}}%
\pgfpathlineto{\pgfqpoint{3.009275in}{3.951944in}}%
\pgfpathlineto{\pgfqpoint{3.011289in}{3.957571in}}%
\pgfpathlineto{\pgfqpoint{3.013302in}{3.950337in}}%
\pgfpathlineto{\pgfqpoint{3.015316in}{3.951944in}}%
\pgfpathlineto{\pgfqpoint{3.017329in}{3.948997in}}%
\pgfpathlineto{\pgfqpoint{3.023370in}{3.952748in}}%
\pgfpathlineto{\pgfqpoint{3.025383in}{3.951676in}}%
\pgfpathlineto{\pgfqpoint{3.027397in}{3.962661in}}%
\pgfpathlineto{\pgfqpoint{3.029410in}{3.962661in}}%
\pgfpathlineto{\pgfqpoint{3.031423in}{3.959982in}}%
\pgfpathlineto{\pgfqpoint{3.037464in}{3.946318in}}%
\pgfpathlineto{\pgfqpoint{3.039477in}{3.952480in}}%
\pgfpathlineto{\pgfqpoint{3.041491in}{3.945247in}}%
\pgfpathlineto{\pgfqpoint{3.043504in}{3.943371in}}%
\pgfpathlineto{\pgfqpoint{3.045518in}{3.924082in}}%
\pgfpathlineto{\pgfqpoint{3.051558in}{3.915508in}}%
\pgfpathlineto{\pgfqpoint{3.053571in}{3.918723in}}%
\pgfpathlineto{\pgfqpoint{3.055585in}{3.928368in}}%
\pgfpathlineto{\pgfqpoint{3.057598in}{3.928368in}}%
\pgfpathlineto{\pgfqpoint{3.059612in}{3.933726in}}%
\pgfpathlineto{\pgfqpoint{3.067665in}{3.935334in}}%
\pgfpathlineto{\pgfqpoint{3.069679in}{3.932387in}}%
\pgfpathlineto{\pgfqpoint{3.071692in}{3.936138in}}%
\pgfpathlineto{\pgfqpoint{3.073706in}{3.941228in}}%
\pgfpathlineto{\pgfqpoint{3.079746in}{3.941496in}}%
\pgfpathlineto{\pgfqpoint{3.081760in}{3.943639in}}%
\pgfpathlineto{\pgfqpoint{3.083773in}{3.950605in}}%
\pgfpathlineto{\pgfqpoint{3.085786in}{3.945782in}}%
\pgfpathlineto{\pgfqpoint{3.087800in}{3.948194in}}%
\pgfpathlineto{\pgfqpoint{3.093840in}{3.947122in}}%
\pgfpathlineto{\pgfqpoint{3.095854in}{3.950605in}}%
\pgfpathlineto{\pgfqpoint{3.097867in}{3.955427in}}%
\pgfpathlineto{\pgfqpoint{3.099881in}{3.955963in}}%
\pgfpathlineto{\pgfqpoint{3.101894in}{3.959982in}}%
\pgfpathlineto{\pgfqpoint{3.107934in}{3.962393in}}%
\pgfpathlineto{\pgfqpoint{3.109948in}{3.964268in}}%
\pgfpathlineto{\pgfqpoint{3.111961in}{3.968555in}}%
\pgfpathlineto{\pgfqpoint{3.113975in}{3.964268in}}%
\pgfpathlineto{\pgfqpoint{3.115988in}{3.961589in}}%
\pgfpathlineto{\pgfqpoint{3.126055in}{3.963732in}}%
\pgfpathlineto{\pgfqpoint{3.128069in}{3.962929in}}%
\pgfpathlineto{\pgfqpoint{3.130082in}{3.957035in}}%
\pgfpathlineto{\pgfqpoint{3.136123in}{3.963732in}}%
\pgfpathlineto{\pgfqpoint{3.138136in}{3.964268in}}%
\pgfpathlineto{\pgfqpoint{3.140150in}{3.957035in}}%
\pgfpathlineto{\pgfqpoint{3.142163in}{3.958374in}}%
\pgfpathlineto{\pgfqpoint{3.144176in}{3.968555in}}%
\pgfpathlineto{\pgfqpoint{3.150217in}{3.966680in}}%
\pgfpathlineto{\pgfqpoint{3.152230in}{3.962661in}}%
\pgfpathlineto{\pgfqpoint{3.154244in}{3.960785in}}%
\pgfpathlineto{\pgfqpoint{3.156257in}{3.964536in}}%
\pgfpathlineto{\pgfqpoint{3.158271in}{3.961321in}}%
\pgfpathlineto{\pgfqpoint{3.164311in}{3.966680in}}%
\pgfpathlineto{\pgfqpoint{3.166324in}{3.978468in}}%
\pgfpathlineto{\pgfqpoint{3.168338in}{3.970966in}}%
\pgfpathlineto{\pgfqpoint{3.170351in}{3.959446in}}%
\pgfpathlineto{\pgfqpoint{3.172365in}{3.961857in}}%
\pgfpathlineto{\pgfqpoint{3.178405in}{3.952748in}}%
\pgfpathlineto{\pgfqpoint{3.182432in}{3.959446in}}%
\pgfpathlineto{\pgfqpoint{3.184445in}{3.961053in}}%
\pgfpathlineto{\pgfqpoint{3.186459in}{3.958374in}}%
\pgfpathlineto{\pgfqpoint{3.192499in}{3.961321in}}%
\pgfpathlineto{\pgfqpoint{3.194513in}{3.952212in}}%
\pgfpathlineto{\pgfqpoint{3.196526in}{3.952212in}}%
\pgfpathlineto{\pgfqpoint{3.200553in}{3.959178in}}%
\pgfpathlineto{\pgfqpoint{3.206593in}{3.961053in}}%
\pgfpathlineto{\pgfqpoint{3.208607in}{3.967483in}}%
\pgfpathlineto{\pgfqpoint{3.210620in}{3.965608in}}%
\pgfpathlineto{\pgfqpoint{3.212634in}{3.974181in}}%
\pgfpathlineto{\pgfqpoint{3.214647in}{3.970430in}}%
\pgfpathlineto{\pgfqpoint{3.220687in}{3.967483in}}%
\pgfpathlineto{\pgfqpoint{3.222701in}{3.963465in}}%
\pgfpathlineto{\pgfqpoint{3.226728in}{3.967483in}}%
\pgfpathlineto{\pgfqpoint{3.228741in}{3.990792in}}%
\pgfpathlineto{\pgfqpoint{3.234782in}{3.993739in}}%
\pgfpathlineto{\pgfqpoint{3.236795in}{3.990524in}}%
\pgfpathlineto{\pgfqpoint{3.238808in}{3.988648in}}%
\pgfpathlineto{\pgfqpoint{3.240822in}{3.989988in}}%
\pgfpathlineto{\pgfqpoint{3.242835in}{3.989184in}}%
\pgfpathlineto{\pgfqpoint{3.248876in}{3.985701in}}%
\pgfpathlineto{\pgfqpoint{3.250889in}{3.985701in}}%
\pgfpathlineto{\pgfqpoint{3.252903in}{3.983290in}}%
\pgfpathlineto{\pgfqpoint{3.254916in}{3.988648in}}%
\pgfpathlineto{\pgfqpoint{3.256930in}{3.990256in}}%
\pgfpathlineto{\pgfqpoint{3.262970in}{3.986505in}}%
\pgfpathlineto{\pgfqpoint{3.264983in}{3.981951in}}%
\pgfpathlineto{\pgfqpoint{3.266997in}{3.982754in}}%
\pgfpathlineto{\pgfqpoint{3.269010in}{3.982486in}}%
\pgfpathlineto{\pgfqpoint{3.271024in}{3.980611in}}%
\pgfpathlineto{\pgfqpoint{3.277064in}{3.981147in}}%
\pgfpathlineto{\pgfqpoint{3.279077in}{3.979807in}}%
\pgfpathlineto{\pgfqpoint{3.281091in}{3.976860in}}%
\pgfpathlineto{\pgfqpoint{3.285118in}{3.974449in}}%
\pgfpathlineto{\pgfqpoint{3.291158in}{3.972306in}}%
\pgfpathlineto{\pgfqpoint{3.295185in}{3.967215in}}%
\pgfpathlineto{\pgfqpoint{3.297198in}{3.970698in}}%
\pgfpathlineto{\pgfqpoint{3.299212in}{3.970698in}}%
\pgfpathlineto{\pgfqpoint{3.305252in}{3.967215in}}%
\pgfpathlineto{\pgfqpoint{3.307266in}{3.958106in}}%
\pgfpathlineto{\pgfqpoint{3.309279in}{3.958374in}}%
\pgfpathlineto{\pgfqpoint{3.311293in}{3.956499in}}%
\pgfpathlineto{\pgfqpoint{3.313306in}{3.957303in}}%
\pgfpathlineto{\pgfqpoint{3.321360in}{3.955427in}}%
\pgfpathlineto{\pgfqpoint{3.323373in}{3.957838in}}%
\pgfpathlineto{\pgfqpoint{3.327400in}{3.957571in}}%
\pgfpathlineto{\pgfqpoint{3.333441in}{3.962393in}}%
\pgfpathlineto{\pgfqpoint{3.335454in}{3.972574in}}%
\pgfpathlineto{\pgfqpoint{3.337467in}{3.977396in}}%
\pgfpathlineto{\pgfqpoint{3.339481in}{3.972306in}}%
\pgfpathlineto{\pgfqpoint{3.341494in}{3.970698in}}%
\pgfpathlineto{\pgfqpoint{3.347535in}{3.978468in}}%
\pgfpathlineto{\pgfqpoint{3.349548in}{3.984898in}}%
\pgfpathlineto{\pgfqpoint{3.351562in}{3.993739in}}%
\pgfpathlineto{\pgfqpoint{3.353575in}{3.989184in}}%
\pgfpathlineto{\pgfqpoint{3.355588in}{3.979539in}}%
\pgfpathlineto{\pgfqpoint{3.361629in}{3.985433in}}%
\pgfpathlineto{\pgfqpoint{3.363642in}{3.986237in}}%
\pgfpathlineto{\pgfqpoint{3.365656in}{3.984362in}}%
\pgfpathlineto{\pgfqpoint{3.367669in}{3.984630in}}%
\pgfpathlineto{\pgfqpoint{3.369683in}{3.980343in}}%
\pgfpathlineto{\pgfqpoint{3.375723in}{3.977128in}}%
\pgfpathlineto{\pgfqpoint{3.377736in}{3.983022in}}%
\pgfpathlineto{\pgfqpoint{3.379750in}{3.986505in}}%
\pgfpathlineto{\pgfqpoint{3.381763in}{3.981415in}}%
\pgfpathlineto{\pgfqpoint{3.383777in}{3.980343in}}%
\pgfpathlineto{\pgfqpoint{3.389817in}{3.978200in}}%
\pgfpathlineto{\pgfqpoint{3.391830in}{3.972306in}}%
\pgfpathlineto{\pgfqpoint{3.393844in}{3.970162in}}%
\pgfpathlineto{\pgfqpoint{3.395857in}{3.984630in}}%
\pgfpathlineto{\pgfqpoint{3.397871in}{3.987845in}}%
\pgfpathlineto{\pgfqpoint{3.403911in}{3.987577in}}%
\pgfpathlineto{\pgfqpoint{3.405925in}{3.983290in}}%
\pgfpathlineto{\pgfqpoint{3.407938in}{3.987041in}}%
\pgfpathlineto{\pgfqpoint{3.409951in}{3.994007in}}%
\pgfpathlineto{\pgfqpoint{3.411965in}{4.012492in}}%
\pgfpathlineto{\pgfqpoint{3.418005in}{4.025084in}}%
\pgfpathlineto{\pgfqpoint{3.420019in}{4.022673in}}%
\pgfpathlineto{\pgfqpoint{3.422032in}{4.015707in}}%
\pgfpathlineto{\pgfqpoint{3.424046in}{4.020798in}}%
\pgfpathlineto{\pgfqpoint{3.426059in}{4.019458in}}%
\pgfpathlineto{\pgfqpoint{3.432099in}{4.024013in}}%
\pgfpathlineto{\pgfqpoint{3.436126in}{4.029907in}}%
\pgfpathlineto{\pgfqpoint{3.438140in}{4.025084in}}%
\pgfpathlineto{\pgfqpoint{3.440153in}{4.033658in}}%
\pgfpathlineto{\pgfqpoint{3.448207in}{4.030978in}}%
\pgfpathlineto{\pgfqpoint{3.450220in}{4.041427in}}%
\pgfpathlineto{\pgfqpoint{3.452234in}{4.034997in}}%
\pgfpathlineto{\pgfqpoint{3.454247in}{4.044642in}}%
\pgfpathlineto{\pgfqpoint{3.460288in}{4.043838in}}%
\pgfpathlineto{\pgfqpoint{3.462301in}{4.044642in}}%
\pgfpathlineto{\pgfqpoint{3.464315in}{4.046785in}}%
\pgfpathlineto{\pgfqpoint{3.466328in}{4.043302in}}%
\pgfpathlineto{\pgfqpoint{3.468341in}{4.047857in}}%
\pgfpathlineto{\pgfqpoint{3.474382in}{4.048125in}}%
\pgfpathlineto{\pgfqpoint{3.476395in}{4.044106in}}%
\pgfpathlineto{\pgfqpoint{3.480422in}{4.041695in}}%
\pgfpathlineto{\pgfqpoint{3.482436in}{4.045178in}}%
\pgfpathlineto{\pgfqpoint{3.488476in}{4.037676in}}%
\pgfpathlineto{\pgfqpoint{3.492503in}{4.039819in}}%
\pgfpathlineto{\pgfqpoint{3.496530in}{4.036337in}}%
\pgfpathlineto{\pgfqpoint{3.502570in}{4.036337in}}%
\pgfpathlineto{\pgfqpoint{3.504584in}{4.034193in}}%
\pgfpathlineto{\pgfqpoint{3.506597in}{4.035801in}}%
\pgfpathlineto{\pgfqpoint{3.508610in}{4.031782in}}%
\pgfpathlineto{\pgfqpoint{3.510624in}{4.042231in}}%
\pgfpathlineto{\pgfqpoint{3.516664in}{4.047589in}}%
\pgfpathlineto{\pgfqpoint{3.518678in}{4.046517in}}%
\pgfpathlineto{\pgfqpoint{3.520691in}{4.034461in}}%
\pgfpathlineto{\pgfqpoint{3.522705in}{4.033658in}}%
\pgfpathlineto{\pgfqpoint{3.524718in}{4.040087in}}%
\pgfpathlineto{\pgfqpoint{3.532772in}{4.044106in}}%
\pgfpathlineto{\pgfqpoint{3.534785in}{4.052143in}}%
\pgfpathlineto{\pgfqpoint{3.536799in}{4.055090in}}%
\pgfpathlineto{\pgfqpoint{3.538812in}{4.055894in}}%
\pgfpathlineto{\pgfqpoint{3.544852in}{4.056698in}}%
\pgfpathlineto{\pgfqpoint{3.546866in}{4.061252in}}%
\pgfpathlineto{\pgfqpoint{3.550893in}{4.066075in}}%
\pgfpathlineto{\pgfqpoint{3.552906in}{4.066075in}}%
\pgfpathlineto{\pgfqpoint{3.558947in}{4.067414in}}%
\pgfpathlineto{\pgfqpoint{3.560960in}{4.070361in}}%
\pgfpathlineto{\pgfqpoint{3.564987in}{4.058841in}}%
\pgfpathlineto{\pgfqpoint{3.567000in}{4.058573in}}%
\pgfpathlineto{\pgfqpoint{3.573041in}{4.053751in}}%
\pgfpathlineto{\pgfqpoint{3.575054in}{4.054287in}}%
\pgfpathlineto{\pgfqpoint{3.577068in}{4.052679in}}%
\pgfpathlineto{\pgfqpoint{3.579081in}{4.052947in}}%
\pgfpathlineto{\pgfqpoint{3.581095in}{4.047321in}}%
\pgfpathlineto{\pgfqpoint{3.587135in}{4.042231in}}%
\pgfpathlineto{\pgfqpoint{3.589148in}{4.047589in}}%
\pgfpathlineto{\pgfqpoint{3.591162in}{4.055626in}}%
\pgfpathlineto{\pgfqpoint{3.593175in}{4.052411in}}%
\pgfpathlineto{\pgfqpoint{3.595189in}{4.039016in}}%
\pgfpathlineto{\pgfqpoint{3.603242in}{4.032854in}}%
\pgfpathlineto{\pgfqpoint{3.605256in}{4.026424in}}%
\pgfpathlineto{\pgfqpoint{3.607269in}{4.022673in}}%
\pgfpathlineto{\pgfqpoint{3.609283in}{4.003919in}}%
\pgfpathlineto{\pgfqpoint{3.615323in}{4.006598in}}%
\pgfpathlineto{\pgfqpoint{3.617337in}{4.014904in}}%
\pgfpathlineto{\pgfqpoint{3.619350in}{4.011421in}}%
\pgfpathlineto{\pgfqpoint{3.621363in}{4.015707in}}%
\pgfpathlineto{\pgfqpoint{3.623377in}{4.007938in}}%
\pgfpathlineto{\pgfqpoint{3.629417in}{3.991060in}}%
\pgfpathlineto{\pgfqpoint{3.631431in}{3.995882in}}%
\pgfpathlineto{\pgfqpoint{3.633444in}{3.994810in}}%
\pgfpathlineto{\pgfqpoint{3.635458in}{4.003919in}}%
\pgfpathlineto{\pgfqpoint{3.637471in}{4.009278in}}%
\pgfpathlineto{\pgfqpoint{3.643511in}{4.006063in}}%
\pgfpathlineto{\pgfqpoint{3.645525in}{4.014368in}}%
\pgfpathlineto{\pgfqpoint{3.647538in}{4.013564in}}%
\pgfpathlineto{\pgfqpoint{3.649552in}{4.014636in}}%
\pgfpathlineto{\pgfqpoint{3.651565in}{4.021066in}}%
\pgfpathlineto{\pgfqpoint{3.659619in}{4.019190in}}%
\pgfpathlineto{\pgfqpoint{3.661632in}{4.013564in}}%
\pgfpathlineto{\pgfqpoint{3.663646in}{4.012492in}}%
\pgfpathlineto{\pgfqpoint{3.665659in}{4.008474in}}%
\pgfpathlineto{\pgfqpoint{3.671700in}{4.015975in}}%
\pgfpathlineto{\pgfqpoint{3.673713in}{4.015707in}}%
\pgfpathlineto{\pgfqpoint{3.675727in}{4.016243in}}%
\pgfpathlineto{\pgfqpoint{3.677740in}{4.020530in}}%
\pgfpathlineto{\pgfqpoint{3.679753in}{4.019994in}}%
\pgfpathlineto{\pgfqpoint{3.685794in}{4.012492in}}%
\pgfpathlineto{\pgfqpoint{3.687807in}{4.024013in}}%
\pgfpathlineto{\pgfqpoint{3.691834in}{4.036337in}}%
\pgfpathlineto{\pgfqpoint{3.693848in}{4.034461in}}%
\pgfpathlineto{\pgfqpoint{3.699888in}{4.032318in}}%
\pgfpathlineto{\pgfqpoint{3.703915in}{4.026424in}}%
\pgfpathlineto{\pgfqpoint{3.705928in}{4.017047in}}%
\pgfpathlineto{\pgfqpoint{3.707942in}{4.012225in}}%
\pgfpathlineto{\pgfqpoint{3.713982in}{4.019190in}}%
\pgfpathlineto{\pgfqpoint{3.715995in}{4.024013in}}%
\pgfpathlineto{\pgfqpoint{3.718009in}{4.015975in}}%
\pgfpathlineto{\pgfqpoint{3.720022in}{4.015707in}}%
\pgfpathlineto{\pgfqpoint{3.722036in}{4.018387in}}%
\pgfpathlineto{\pgfqpoint{3.728076in}{4.018654in}}%
\pgfpathlineto{\pgfqpoint{3.730090in}{4.025084in}}%
\pgfpathlineto{\pgfqpoint{3.732103in}{4.023209in}}%
\pgfpathlineto{\pgfqpoint{3.734116in}{4.027496in}}%
\pgfpathlineto{\pgfqpoint{3.736130in}{4.028835in}}%
\pgfpathlineto{\pgfqpoint{3.742170in}{4.029103in}}%
\pgfpathlineto{\pgfqpoint{3.744184in}{4.028567in}}%
\pgfpathlineto{\pgfqpoint{3.748211in}{4.036605in}}%
\pgfpathlineto{\pgfqpoint{3.750224in}{4.032050in}}%
\pgfpathlineto{\pgfqpoint{3.756264in}{4.028299in}}%
\pgfpathlineto{\pgfqpoint{3.758278in}{4.026156in}}%
\pgfpathlineto{\pgfqpoint{3.760291in}{4.030443in}}%
\pgfpathlineto{\pgfqpoint{3.762305in}{4.022405in}}%
\pgfpathlineto{\pgfqpoint{3.764318in}{4.019190in}}%
\pgfpathlineto{\pgfqpoint{3.772372in}{4.027496in}}%
\pgfpathlineto{\pgfqpoint{3.774385in}{4.034193in}}%
\pgfpathlineto{\pgfqpoint{3.776399in}{4.043570in}}%
\pgfpathlineto{\pgfqpoint{3.786466in}{4.044106in}}%
\pgfpathlineto{\pgfqpoint{3.788480in}{4.040623in}}%
\pgfpathlineto{\pgfqpoint{3.790493in}{4.041427in}}%
\pgfpathlineto{\pgfqpoint{3.792506in}{4.044642in}}%
\pgfpathlineto{\pgfqpoint{3.798547in}{4.048393in}}%
\pgfpathlineto{\pgfqpoint{3.800560in}{4.048125in}}%
\pgfpathlineto{\pgfqpoint{3.802574in}{4.050804in}}%
\pgfpathlineto{\pgfqpoint{3.806601in}{4.046249in}}%
\pgfpathlineto{\pgfqpoint{3.812641in}{4.044106in}}%
\pgfpathlineto{\pgfqpoint{3.814654in}{4.035533in}}%
\pgfpathlineto{\pgfqpoint{3.816668in}{4.043034in}}%
\pgfpathlineto{\pgfqpoint{3.818681in}{4.041159in}}%
\pgfpathlineto{\pgfqpoint{3.820695in}{4.040623in}}%
\pgfpathlineto{\pgfqpoint{3.826735in}{4.050000in}}%
\pgfpathlineto{\pgfqpoint{3.828749in}{4.051608in}}%
\pgfpathlineto{\pgfqpoint{3.832775in}{4.044642in}}%
\pgfpathlineto{\pgfqpoint{3.834789in}{4.045981in}}%
\pgfpathlineto{\pgfqpoint{3.840829in}{4.044642in}}%
\pgfpathlineto{\pgfqpoint{3.842843in}{4.037944in}}%
\pgfpathlineto{\pgfqpoint{3.844856in}{4.042231in}}%
\pgfpathlineto{\pgfqpoint{3.856937in}{4.043838in}}%
\pgfpathlineto{\pgfqpoint{3.860964in}{4.047589in}}%
\pgfpathlineto{\pgfqpoint{3.862977in}{4.048661in}}%
\pgfpathlineto{\pgfqpoint{3.869017in}{4.049464in}}%
\pgfpathlineto{\pgfqpoint{3.871031in}{4.048661in}}%
\pgfpathlineto{\pgfqpoint{3.873044in}{4.043570in}}%
\pgfpathlineto{\pgfqpoint{3.875058in}{4.048125in}}%
\pgfpathlineto{\pgfqpoint{3.877071in}{4.057234in}}%
\pgfpathlineto{\pgfqpoint{3.883112in}{4.062860in}}%
\pgfpathlineto{\pgfqpoint{3.885125in}{4.062056in}}%
\pgfpathlineto{\pgfqpoint{3.889152in}{4.052411in}}%
\pgfpathlineto{\pgfqpoint{3.891165in}{4.054019in}}%
\pgfpathlineto{\pgfqpoint{3.897206in}{4.049464in}}%
\pgfpathlineto{\pgfqpoint{3.899219in}{4.050536in}}%
\pgfpathlineto{\pgfqpoint{3.901233in}{4.050804in}}%
\pgfpathlineto{\pgfqpoint{3.903246in}{4.056430in}}%
\pgfpathlineto{\pgfqpoint{3.905260in}{4.057502in}}%
\pgfpathlineto{\pgfqpoint{3.913313in}{4.048928in}}%
\pgfpathlineto{\pgfqpoint{3.917340in}{4.042499in}}%
\pgfpathlineto{\pgfqpoint{3.919354in}{4.045714in}}%
\pgfpathlineto{\pgfqpoint{3.925394in}{4.042231in}}%
\pgfpathlineto{\pgfqpoint{3.927407in}{4.044910in}}%
\pgfpathlineto{\pgfqpoint{3.931434in}{4.055090in}}%
\pgfpathlineto{\pgfqpoint{3.939488in}{4.052679in}}%
\pgfpathlineto{\pgfqpoint{3.941502in}{4.044374in}}%
\pgfpathlineto{\pgfqpoint{3.943515in}{4.043302in}}%
\pgfpathlineto{\pgfqpoint{3.945528in}{4.040623in}}%
\pgfpathlineto{\pgfqpoint{3.947542in}{4.048125in}}%
\pgfpathlineto{\pgfqpoint{3.953582in}{4.050536in}}%
\pgfpathlineto{\pgfqpoint{3.955596in}{4.049464in}}%
\pgfpathlineto{\pgfqpoint{3.957609in}{4.058573in}}%
\pgfpathlineto{\pgfqpoint{3.959623in}{4.049464in}}%
\pgfpathlineto{\pgfqpoint{3.967676in}{4.035801in}}%
\pgfpathlineto{\pgfqpoint{3.969690in}{4.036605in}}%
\pgfpathlineto{\pgfqpoint{3.971703in}{4.034193in}}%
\pgfpathlineto{\pgfqpoint{3.973717in}{4.034729in}}%
\pgfpathlineto{\pgfqpoint{3.975730in}{4.031514in}}%
\pgfpathlineto{\pgfqpoint{3.981771in}{4.027228in}}%
\pgfpathlineto{\pgfqpoint{3.983784in}{4.024013in}}%
\pgfpathlineto{\pgfqpoint{3.985797in}{4.028299in}}%
\pgfpathlineto{\pgfqpoint{3.987811in}{4.017583in}}%
\pgfpathlineto{\pgfqpoint{3.989824in}{4.021869in}}%
\pgfpathlineto{\pgfqpoint{3.995865in}{4.020262in}}%
\pgfpathlineto{\pgfqpoint{3.997878in}{4.014636in}}%
\pgfpathlineto{\pgfqpoint{3.999892in}{4.024013in}}%
\pgfpathlineto{\pgfqpoint{4.001905in}{4.025084in}}%
\pgfpathlineto{\pgfqpoint{4.003918in}{4.028835in}}%
\pgfpathlineto{\pgfqpoint{4.009959in}{4.031514in}}%
\pgfpathlineto{\pgfqpoint{4.011972in}{4.027496in}}%
\pgfpathlineto{\pgfqpoint{4.013986in}{4.032318in}}%
\pgfpathlineto{\pgfqpoint{4.015999in}{4.033658in}}%
\pgfpathlineto{\pgfqpoint{4.018013in}{4.028299in}}%
\pgfpathlineto{\pgfqpoint{4.024053in}{4.037676in}}%
\pgfpathlineto{\pgfqpoint{4.026066in}{4.037140in}}%
\pgfpathlineto{\pgfqpoint{4.028080in}{4.044106in}}%
\pgfpathlineto{\pgfqpoint{4.030093in}{4.045714in}}%
\pgfpathlineto{\pgfqpoint{4.032107in}{4.039552in}}%
\pgfpathlineto{\pgfqpoint{4.038147in}{4.040623in}}%
\pgfpathlineto{\pgfqpoint{4.040160in}{4.036337in}}%
\pgfpathlineto{\pgfqpoint{4.042174in}{4.039016in}}%
\pgfpathlineto{\pgfqpoint{4.044187in}{4.036337in}}%
\pgfpathlineto{\pgfqpoint{4.046201in}{4.035801in}}%
\pgfpathlineto{\pgfqpoint{4.054255in}{4.032854in}}%
\pgfpathlineto{\pgfqpoint{4.056268in}{4.034997in}}%
\pgfpathlineto{\pgfqpoint{4.058282in}{4.035265in}}%
\pgfpathlineto{\pgfqpoint{4.060295in}{4.038480in}}%
\pgfpathlineto{\pgfqpoint{4.066335in}{4.037944in}}%
\pgfpathlineto{\pgfqpoint{4.068349in}{4.033925in}}%
\pgfpathlineto{\pgfqpoint{4.072376in}{4.036605in}}%
\pgfpathlineto{\pgfqpoint{4.074389in}{4.033390in}}%
\pgfpathlineto{\pgfqpoint{4.080429in}{4.034461in}}%
\pgfpathlineto{\pgfqpoint{4.082443in}{4.040891in}}%
\pgfpathlineto{\pgfqpoint{4.084456in}{4.041963in}}%
\pgfpathlineto{\pgfqpoint{4.086470in}{4.045714in}}%
\pgfpathlineto{\pgfqpoint{4.088483in}{4.047321in}}%
\pgfpathlineto{\pgfqpoint{4.098550in}{4.039552in}}%
\pgfpathlineto{\pgfqpoint{4.100564in}{4.030978in}}%
\pgfpathlineto{\pgfqpoint{4.102577in}{4.032854in}}%
\pgfpathlineto{\pgfqpoint{4.108618in}{4.028031in}}%
\pgfpathlineto{\pgfqpoint{4.110631in}{4.032586in}}%
\pgfpathlineto{\pgfqpoint{4.112645in}{4.022405in}}%
\pgfpathlineto{\pgfqpoint{4.114658in}{4.021601in}}%
\pgfpathlineto{\pgfqpoint{4.116671in}{4.027763in}}%
\pgfpathlineto{\pgfqpoint{4.122712in}{4.023745in}}%
\pgfpathlineto{\pgfqpoint{4.124725in}{4.014636in}}%
\pgfpathlineto{\pgfqpoint{4.126739in}{4.024281in}}%
\pgfpathlineto{\pgfqpoint{4.130766in}{4.002848in}}%
\pgfpathlineto{\pgfqpoint{4.136806in}{3.995614in}}%
\pgfpathlineto{\pgfqpoint{4.140833in}{4.002848in}}%
\pgfpathlineto{\pgfqpoint{4.142846in}{4.002312in}}%
\pgfpathlineto{\pgfqpoint{4.144860in}{4.014904in}}%
\pgfpathlineto{\pgfqpoint{4.150900in}{4.019458in}}%
\pgfpathlineto{\pgfqpoint{4.152914in}{4.028835in}}%
\pgfpathlineto{\pgfqpoint{4.154927in}{4.023209in}}%
\pgfpathlineto{\pgfqpoint{4.158954in}{4.033122in}}%
\pgfpathlineto{\pgfqpoint{4.164994in}{4.030443in}}%
\pgfpathlineto{\pgfqpoint{4.167008in}{4.038212in}}%
\pgfpathlineto{\pgfqpoint{4.169021in}{4.033390in}}%
\pgfpathlineto{\pgfqpoint{4.171035in}{4.033658in}}%
\pgfpathlineto{\pgfqpoint{4.173048in}{4.036872in}}%
\pgfpathlineto{\pgfqpoint{4.179088in}{4.034461in}}%
\pgfpathlineto{\pgfqpoint{4.181102in}{4.034461in}}%
\pgfpathlineto{\pgfqpoint{4.183115in}{4.037140in}}%
\pgfpathlineto{\pgfqpoint{4.185129in}{4.048928in}}%
\pgfpathlineto{\pgfqpoint{4.187142in}{4.050000in}}%
\pgfpathlineto{\pgfqpoint{4.193182in}{4.051340in}}%
\pgfpathlineto{\pgfqpoint{4.195196in}{4.049464in}}%
\pgfpathlineto{\pgfqpoint{4.197209in}{4.052411in}}%
\pgfpathlineto{\pgfqpoint{4.199223in}{4.050268in}}%
\pgfpathlineto{\pgfqpoint{4.201236in}{4.051072in}}%
\pgfpathlineto{\pgfqpoint{4.207277in}{4.054555in}}%
\pgfpathlineto{\pgfqpoint{4.209290in}{4.063396in}}%
\pgfpathlineto{\pgfqpoint{4.213317in}{4.059645in}}%
\pgfpathlineto{\pgfqpoint{4.215330in}{4.062860in}}%
\pgfpathlineto{\pgfqpoint{4.221371in}{4.063128in}}%
\pgfpathlineto{\pgfqpoint{4.223384in}{4.059913in}}%
\pgfpathlineto{\pgfqpoint{4.225398in}{4.060181in}}%
\pgfpathlineto{\pgfqpoint{4.229425in}{4.051876in}}%
\pgfpathlineto{\pgfqpoint{4.235465in}{4.041427in}}%
\pgfpathlineto{\pgfqpoint{4.237478in}{4.041963in}}%
\pgfpathlineto{\pgfqpoint{4.239492in}{4.049464in}}%
\pgfpathlineto{\pgfqpoint{4.241505in}{4.043034in}}%
\pgfpathlineto{\pgfqpoint{4.243519in}{4.041159in}}%
\pgfpathlineto{\pgfqpoint{4.251572in}{4.031782in}}%
\pgfpathlineto{\pgfqpoint{4.253586in}{4.024816in}}%
\pgfpathlineto{\pgfqpoint{4.255599in}{4.028031in}}%
\pgfpathlineto{\pgfqpoint{4.257613in}{4.016511in}}%
\pgfpathlineto{\pgfqpoint{4.265667in}{4.007670in}}%
\pgfpathlineto{\pgfqpoint{4.267680in}{4.011421in}}%
\pgfpathlineto{\pgfqpoint{4.269693in}{4.027228in}}%
\pgfpathlineto{\pgfqpoint{4.271707in}{4.037944in}}%
\pgfpathlineto{\pgfqpoint{4.277747in}{4.039819in}}%
\pgfpathlineto{\pgfqpoint{4.279761in}{4.043838in}}%
\pgfpathlineto{\pgfqpoint{4.281774in}{4.042499in}}%
\pgfpathlineto{\pgfqpoint{4.291841in}{4.039819in}}%
\pgfpathlineto{\pgfqpoint{4.293855in}{4.036872in}}%
\pgfpathlineto{\pgfqpoint{4.295868in}{4.030175in}}%
\pgfpathlineto{\pgfqpoint{4.299895in}{4.025352in}}%
\pgfpathlineto{\pgfqpoint{4.305936in}{4.015172in}}%
\pgfpathlineto{\pgfqpoint{4.307949in}{4.003383in}}%
\pgfpathlineto{\pgfqpoint{4.309962in}{4.003651in}}%
\pgfpathlineto{\pgfqpoint{4.311976in}{4.010081in}}%
\pgfpathlineto{\pgfqpoint{4.313989in}{4.002312in}}%
\pgfpathlineto{\pgfqpoint{4.320030in}{4.001240in}}%
\pgfpathlineto{\pgfqpoint{4.322043in}{3.998561in}}%
\pgfpathlineto{\pgfqpoint{4.324057in}{3.996954in}}%
\pgfpathlineto{\pgfqpoint{4.326070in}{3.992399in}}%
\pgfpathlineto{\pgfqpoint{4.328083in}{3.992667in}}%
\pgfpathlineto{\pgfqpoint{4.336137in}{3.998293in}}%
\pgfpathlineto{\pgfqpoint{4.340164in}{4.007938in}}%
\pgfpathlineto{\pgfqpoint{4.342178in}{4.012492in}}%
\pgfpathlineto{\pgfqpoint{4.348218in}{4.014904in}}%
\pgfpathlineto{\pgfqpoint{4.350231in}{4.010349in}}%
\pgfpathlineto{\pgfqpoint{4.352245in}{3.998293in}}%
\pgfpathlineto{\pgfqpoint{4.354258in}{4.003651in}}%
\pgfpathlineto{\pgfqpoint{4.356272in}{3.999365in}}%
\pgfpathlineto{\pgfqpoint{4.362312in}{4.006330in}}%
\pgfpathlineto{\pgfqpoint{4.364325in}{4.012225in}}%
\pgfpathlineto{\pgfqpoint{4.366339in}{4.005259in}}%
\pgfpathlineto{\pgfqpoint{4.368352in}{4.013028in}}%
\pgfpathlineto{\pgfqpoint{4.370366in}{4.013296in}}%
\pgfpathlineto{\pgfqpoint{4.376406in}{4.015975in}}%
\pgfpathlineto{\pgfqpoint{4.378420in}{4.017851in}}%
\pgfpathlineto{\pgfqpoint{4.380433in}{4.018922in}}%
\pgfpathlineto{\pgfqpoint{4.382447in}{4.021601in}}%
\pgfpathlineto{\pgfqpoint{4.384460in}{4.027496in}}%
\pgfpathlineto{\pgfqpoint{4.392514in}{4.027763in}}%
\pgfpathlineto{\pgfqpoint{4.394527in}{4.029639in}}%
\pgfpathlineto{\pgfqpoint{4.396541in}{4.029371in}}%
\pgfpathlineto{\pgfqpoint{4.398554in}{4.033925in}}%
\pgfpathlineto{\pgfqpoint{4.404594in}{4.033122in}}%
\pgfpathlineto{\pgfqpoint{4.406608in}{4.037944in}}%
\pgfpathlineto{\pgfqpoint{4.408621in}{4.049732in}}%
\pgfpathlineto{\pgfqpoint{4.410635in}{4.049196in}}%
\pgfpathlineto{\pgfqpoint{4.412648in}{4.051608in}}%
\pgfpathlineto{\pgfqpoint{4.418689in}{4.054287in}}%
\pgfpathlineto{\pgfqpoint{4.422715in}{4.044106in}}%
\pgfpathlineto{\pgfqpoint{4.424729in}{4.047589in}}%
\pgfpathlineto{\pgfqpoint{4.426742in}{4.038748in}}%
\pgfpathlineto{\pgfqpoint{4.432783in}{4.043570in}}%
\pgfpathlineto{\pgfqpoint{4.434796in}{4.033122in}}%
\pgfpathlineto{\pgfqpoint{4.436810in}{4.033390in}}%
\pgfpathlineto{\pgfqpoint{4.438823in}{4.038212in}}%
\pgfpathlineto{\pgfqpoint{4.440836in}{4.030175in}}%
\pgfpathlineto{\pgfqpoint{4.446877in}{4.039284in}}%
\pgfpathlineto{\pgfqpoint{4.448890in}{4.036069in}}%
\pgfpathlineto{\pgfqpoint{4.450904in}{4.043570in}}%
\pgfpathlineto{\pgfqpoint{4.452917in}{4.036605in}}%
\pgfpathlineto{\pgfqpoint{4.454931in}{4.038212in}}%
\pgfpathlineto{\pgfqpoint{4.460971in}{4.039819in}}%
\pgfpathlineto{\pgfqpoint{4.462984in}{4.035265in}}%
\pgfpathlineto{\pgfqpoint{4.464998in}{4.027228in}}%
\pgfpathlineto{\pgfqpoint{4.467011in}{4.024816in}}%
\pgfpathlineto{\pgfqpoint{4.469025in}{4.026156in}}%
\pgfpathlineto{\pgfqpoint{4.475065in}{4.031782in}}%
\pgfpathlineto{\pgfqpoint{4.477079in}{4.024816in}}%
\pgfpathlineto{\pgfqpoint{4.479092in}{4.025620in}}%
\pgfpathlineto{\pgfqpoint{4.481105in}{4.027763in}}%
\pgfpathlineto{\pgfqpoint{4.489159in}{4.033390in}}%
\pgfpathlineto{\pgfqpoint{4.491173in}{4.029639in}}%
\pgfpathlineto{\pgfqpoint{4.493186in}{4.029371in}}%
\pgfpathlineto{\pgfqpoint{4.495200in}{4.045714in}}%
\pgfpathlineto{\pgfqpoint{4.497213in}{4.108137in}}%
\pgfpathlineto{\pgfqpoint{4.503253in}{4.088312in}}%
\pgfpathlineto{\pgfqpoint{4.505267in}{4.090723in}}%
\pgfpathlineto{\pgfqpoint{4.507280in}{4.084561in}}%
\pgfpathlineto{\pgfqpoint{4.509294in}{4.080542in}}%
\pgfpathlineto{\pgfqpoint{4.511307in}{4.080006in}}%
\pgfpathlineto{\pgfqpoint{4.517347in}{4.074648in}}%
\pgfpathlineto{\pgfqpoint{4.519361in}{4.065807in}}%
\pgfpathlineto{\pgfqpoint{4.521374in}{4.072237in}}%
\pgfpathlineto{\pgfqpoint{4.525401in}{4.069826in}}%
\pgfpathlineto{\pgfqpoint{4.531442in}{4.071433in}}%
\pgfpathlineto{\pgfqpoint{4.533455in}{4.077059in}}%
\pgfpathlineto{\pgfqpoint{4.537482in}{4.075988in}}%
\pgfpathlineto{\pgfqpoint{4.539495in}{4.081346in}}%
\pgfpathlineto{\pgfqpoint{4.545536in}{4.080274in}}%
\pgfpathlineto{\pgfqpoint{4.547549in}{4.072505in}}%
\pgfpathlineto{\pgfqpoint{4.549563in}{4.070094in}}%
\pgfpathlineto{\pgfqpoint{4.551576in}{4.075184in}}%
\pgfpathlineto{\pgfqpoint{4.553590in}{4.082417in}}%
\pgfpathlineto{\pgfqpoint{4.559630in}{4.072505in}}%
\pgfpathlineto{\pgfqpoint{4.561643in}{4.074916in}}%
\pgfpathlineto{\pgfqpoint{4.565670in}{4.083489in}}%
\pgfpathlineto{\pgfqpoint{4.567684in}{4.080274in}}%
\pgfpathlineto{\pgfqpoint{4.575737in}{4.082150in}}%
\pgfpathlineto{\pgfqpoint{4.577751in}{4.088579in}}%
\pgfpathlineto{\pgfqpoint{4.579764in}{4.090455in}}%
\pgfpathlineto{\pgfqpoint{4.591845in}{4.085900in}}%
\pgfpathlineto{\pgfqpoint{4.593858in}{4.088312in}}%
\pgfpathlineto{\pgfqpoint{4.595872in}{4.080274in}}%
\pgfpathlineto{\pgfqpoint{4.601912in}{4.080542in}}%
\pgfpathlineto{\pgfqpoint{4.603926in}{4.081614in}}%
\pgfpathlineto{\pgfqpoint{4.605939in}{4.086168in}}%
\pgfpathlineto{\pgfqpoint{4.607953in}{4.080006in}}%
\pgfpathlineto{\pgfqpoint{4.609966in}{4.080810in}}%
\pgfpathlineto{\pgfqpoint{4.616006in}{4.079738in}}%
\pgfpathlineto{\pgfqpoint{4.618020in}{4.081614in}}%
\pgfpathlineto{\pgfqpoint{4.620033in}{4.088312in}}%
\pgfpathlineto{\pgfqpoint{4.624060in}{4.082953in}}%
\pgfpathlineto{\pgfqpoint{4.630101in}{4.078935in}}%
\pgfpathlineto{\pgfqpoint{4.632114in}{4.079203in}}%
\pgfpathlineto{\pgfqpoint{4.634127in}{4.080274in}}%
\pgfpathlineto{\pgfqpoint{4.636141in}{4.087776in}}%
\pgfpathlineto{\pgfqpoint{4.638154in}{4.084829in}}%
\pgfpathlineto{\pgfqpoint{4.644195in}{4.088847in}}%
\pgfpathlineto{\pgfqpoint{4.646208in}{4.091794in}}%
\pgfpathlineto{\pgfqpoint{4.650235in}{4.080274in}}%
\pgfpathlineto{\pgfqpoint{4.652248in}{4.081346in}}%
\pgfpathlineto{\pgfqpoint{4.658289in}{4.071165in}}%
\pgfpathlineto{\pgfqpoint{4.660302in}{4.069558in}}%
\pgfpathlineto{\pgfqpoint{4.664329in}{4.074380in}}%
\pgfpathlineto{\pgfqpoint{4.672383in}{4.063664in}}%
\pgfpathlineto{\pgfqpoint{4.674396in}{4.067414in}}%
\pgfpathlineto{\pgfqpoint{4.676410in}{4.054287in}}%
\pgfpathlineto{\pgfqpoint{4.678423in}{4.057234in}}%
\pgfpathlineto{\pgfqpoint{4.680437in}{4.062860in}}%
\pgfpathlineto{\pgfqpoint{4.686477in}{4.067414in}}%
\pgfpathlineto{\pgfqpoint{4.688490in}{4.071701in}}%
\pgfpathlineto{\pgfqpoint{4.690504in}{4.074112in}}%
\pgfpathlineto{\pgfqpoint{4.694531in}{4.084829in}}%
\pgfpathlineto{\pgfqpoint{4.700571in}{4.082685in}}%
\pgfpathlineto{\pgfqpoint{4.706612in}{4.062592in}}%
\pgfpathlineto{\pgfqpoint{4.708625in}{4.051072in}}%
\pgfpathlineto{\pgfqpoint{4.714665in}{4.055626in}}%
\pgfpathlineto{\pgfqpoint{4.718692in}{4.062592in}}%
\pgfpathlineto{\pgfqpoint{4.720706in}{4.059377in}}%
\pgfpathlineto{\pgfqpoint{4.722719in}{4.059109in}}%
\pgfpathlineto{\pgfqpoint{4.728759in}{4.053751in}}%
\pgfpathlineto{\pgfqpoint{4.730773in}{4.054555in}}%
\pgfpathlineto{\pgfqpoint{4.732786in}{4.059109in}}%
\pgfpathlineto{\pgfqpoint{4.734800in}{4.057502in}}%
\pgfpathlineto{\pgfqpoint{4.736813in}{4.051876in}}%
\pgfpathlineto{\pgfqpoint{4.742854in}{4.062056in}}%
\pgfpathlineto{\pgfqpoint{4.744867in}{4.050000in}}%
\pgfpathlineto{\pgfqpoint{4.746880in}{4.053483in}}%
\pgfpathlineto{\pgfqpoint{4.748894in}{4.051876in}}%
\pgfpathlineto{\pgfqpoint{4.750907in}{4.058573in}}%
\pgfpathlineto{\pgfqpoint{4.756948in}{4.061520in}}%
\pgfpathlineto{\pgfqpoint{4.758961in}{4.058305in}}%
\pgfpathlineto{\pgfqpoint{4.760975in}{4.050536in}}%
\pgfpathlineto{\pgfqpoint{4.765001in}{4.024816in}}%
\pgfpathlineto{\pgfqpoint{4.771042in}{4.008206in}}%
\pgfpathlineto{\pgfqpoint{4.773055in}{3.994810in}}%
\pgfpathlineto{\pgfqpoint{4.775069in}{4.011421in}}%
\pgfpathlineto{\pgfqpoint{4.777082in}{4.034193in}}%
\pgfpathlineto{\pgfqpoint{4.779096in}{4.037676in}}%
\pgfpathlineto{\pgfqpoint{4.785136in}{4.029907in}}%
\pgfpathlineto{\pgfqpoint{4.787149in}{4.008474in}}%
\pgfpathlineto{\pgfqpoint{4.789163in}{4.024281in}}%
\pgfpathlineto{\pgfqpoint{4.791176in}{4.022941in}}%
\pgfpathlineto{\pgfqpoint{4.793190in}{4.011421in}}%
\pgfpathlineto{\pgfqpoint{4.801244in}{4.033122in}}%
\pgfpathlineto{\pgfqpoint{4.803257in}{4.023745in}}%
\pgfpathlineto{\pgfqpoint{4.805270in}{4.026692in}}%
\pgfpathlineto{\pgfqpoint{4.807284in}{4.032854in}}%
\pgfpathlineto{\pgfqpoint{4.813324in}{4.028835in}}%
\pgfpathlineto{\pgfqpoint{4.817351in}{4.055090in}}%
\pgfpathlineto{\pgfqpoint{4.819365in}{4.047053in}}%
\pgfpathlineto{\pgfqpoint{4.821378in}{4.034461in}}%
\pgfpathlineto{\pgfqpoint{4.827418in}{4.041159in}}%
\pgfpathlineto{\pgfqpoint{4.831445in}{4.042231in}}%
\pgfpathlineto{\pgfqpoint{4.833459in}{4.037140in}}%
\pgfpathlineto{\pgfqpoint{4.835472in}{4.037140in}}%
\pgfpathlineto{\pgfqpoint{4.841512in}{4.023209in}}%
\pgfpathlineto{\pgfqpoint{4.843526in}{4.029103in}}%
\pgfpathlineto{\pgfqpoint{4.845539in}{4.044106in}}%
\pgfpathlineto{\pgfqpoint{4.847553in}{4.043302in}}%
\pgfpathlineto{\pgfqpoint{4.849566in}{4.049732in}}%
\pgfpathlineto{\pgfqpoint{4.859634in}{4.102511in}}%
\pgfpathlineto{\pgfqpoint{4.861647in}{4.108405in}}%
\pgfpathlineto{\pgfqpoint{4.863660in}{4.109477in}}%
\pgfpathlineto{\pgfqpoint{4.869701in}{4.110012in}}%
\pgfpathlineto{\pgfqpoint{4.873728in}{4.098760in}}%
\pgfpathlineto{\pgfqpoint{4.875741in}{4.108405in}}%
\pgfpathlineto{\pgfqpoint{4.877755in}{4.130374in}}%
\pgfpathlineto{\pgfqpoint{4.883795in}{4.130642in}}%
\pgfpathlineto{\pgfqpoint{4.885808in}{4.125819in}}%
\pgfpathlineto{\pgfqpoint{4.887822in}{4.127427in}}%
\pgfpathlineto{\pgfqpoint{4.889835in}{4.144037in}}%
\pgfpathlineto{\pgfqpoint{4.891849in}{4.142430in}}%
\pgfpathlineto{\pgfqpoint{4.897889in}{4.143234in}}%
\pgfpathlineto{\pgfqpoint{4.901916in}{4.139751in}}%
\pgfpathlineto{\pgfqpoint{4.903929in}{4.138411in}}%
\pgfpathlineto{\pgfqpoint{4.905943in}{4.129034in}}%
\pgfpathlineto{\pgfqpoint{4.913997in}{4.144305in}}%
\pgfpathlineto{\pgfqpoint{4.916010in}{4.143234in}}%
\pgfpathlineto{\pgfqpoint{4.918023in}{4.145377in}}%
\pgfpathlineto{\pgfqpoint{4.920037in}{4.151807in}}%
\pgfpathlineto{\pgfqpoint{4.926077in}{4.148056in}}%
\pgfpathlineto{\pgfqpoint{4.928091in}{4.156361in}}%
\pgfpathlineto{\pgfqpoint{4.930104in}{4.168953in}}%
\pgfpathlineto{\pgfqpoint{4.932118in}{4.157433in}}%
\pgfpathlineto{\pgfqpoint{4.934131in}{4.160112in}}%
\pgfpathlineto{\pgfqpoint{4.940171in}{4.161987in}}%
\pgfpathlineto{\pgfqpoint{4.942185in}{4.160916in}}%
\pgfpathlineto{\pgfqpoint{4.944198in}{4.165470in}}%
\pgfpathlineto{\pgfqpoint{4.946212in}{4.159844in}}%
\pgfpathlineto{\pgfqpoint{4.948225in}{4.168685in}}%
\pgfpathlineto{\pgfqpoint{4.954266in}{4.167078in}}%
\pgfpathlineto{\pgfqpoint{4.956279in}{4.168685in}}%
\pgfpathlineto{\pgfqpoint{4.958292in}{4.161987in}}%
\pgfpathlineto{\pgfqpoint{4.962319in}{4.161987in}}%
\pgfpathlineto{\pgfqpoint{4.968360in}{4.152343in}}%
\pgfpathlineto{\pgfqpoint{4.970373in}{4.157433in}}%
\pgfpathlineto{\pgfqpoint{4.972387in}{4.152878in}}%
\pgfpathlineto{\pgfqpoint{4.974400in}{4.154218in}}%
\pgfpathlineto{\pgfqpoint{4.976413in}{4.164934in}}%
\pgfpathlineto{\pgfqpoint{4.982454in}{4.162255in}}%
\pgfpathlineto{\pgfqpoint{4.984467in}{4.157969in}}%
\pgfpathlineto{\pgfqpoint{4.986481in}{4.164399in}}%
\pgfpathlineto{\pgfqpoint{4.988494in}{4.168417in}}%
\pgfpathlineto{\pgfqpoint{4.990508in}{4.159576in}}%
\pgfpathlineto{\pgfqpoint{4.996548in}{4.159576in}}%
\pgfpathlineto{\pgfqpoint{4.998561in}{4.160916in}}%
\pgfpathlineto{\pgfqpoint{5.000575in}{4.176187in}}%
\pgfpathlineto{\pgfqpoint{5.004602in}{4.165202in}}%
\pgfpathlineto{\pgfqpoint{5.010642in}{4.168149in}}%
\pgfpathlineto{\pgfqpoint{5.012655in}{4.170025in}}%
\pgfpathlineto{\pgfqpoint{5.014669in}{4.180741in}}%
\pgfpathlineto{\pgfqpoint{5.016682in}{4.178062in}}%
\pgfpathlineto{\pgfqpoint{5.024736in}{4.179670in}}%
\pgfpathlineto{\pgfqpoint{5.026750in}{4.188243in}}%
\pgfpathlineto{\pgfqpoint{5.028763in}{4.183152in}}%
\pgfpathlineto{\pgfqpoint{5.030777in}{4.185296in}}%
\pgfpathlineto{\pgfqpoint{5.038830in}{4.175115in}}%
\pgfpathlineto{\pgfqpoint{5.040844in}{4.175919in}}%
\pgfpathlineto{\pgfqpoint{5.042857in}{4.164666in}}%
\pgfpathlineto{\pgfqpoint{5.044871in}{4.134928in}}%
\pgfpathlineto{\pgfqpoint{5.046884in}{4.123140in}}%
\pgfpathlineto{\pgfqpoint{5.054938in}{4.127427in}}%
\pgfpathlineto{\pgfqpoint{5.056951in}{4.118050in}}%
\pgfpathlineto{\pgfqpoint{5.058965in}{4.137072in}}%
\pgfpathlineto{\pgfqpoint{5.060978in}{4.123944in}}%
\pgfpathlineto{\pgfqpoint{5.069032in}{4.123944in}}%
\pgfpathlineto{\pgfqpoint{5.071045in}{4.112692in}}%
\pgfpathlineto{\pgfqpoint{5.073059in}{4.126355in}}%
\pgfpathlineto{\pgfqpoint{5.075072in}{4.118050in}}%
\pgfpathlineto{\pgfqpoint{5.081113in}{4.113495in}}%
\pgfpathlineto{\pgfqpoint{5.083126in}{4.119657in}}%
\pgfpathlineto{\pgfqpoint{5.085140in}{4.112692in}}%
\pgfpathlineto{\pgfqpoint{5.087153in}{4.117514in}}%
\pgfpathlineto{\pgfqpoint{5.089166in}{4.138143in}}%
\pgfpathlineto{\pgfqpoint{5.095207in}{4.127427in}}%
\pgfpathlineto{\pgfqpoint{5.097220in}{4.118050in}}%
\pgfpathlineto{\pgfqpoint{5.101247in}{4.139751in}}%
\pgfpathlineto{\pgfqpoint{5.103261in}{4.125015in}}%
\pgfpathlineto{\pgfqpoint{5.109301in}{4.116442in}}%
\pgfpathlineto{\pgfqpoint{5.111314in}{4.119121in}}%
\pgfpathlineto{\pgfqpoint{5.113328in}{4.119657in}}%
\pgfpathlineto{\pgfqpoint{5.115341in}{4.099832in}}%
\pgfpathlineto{\pgfqpoint{5.117355in}{4.118586in}}%
\pgfpathlineto{\pgfqpoint{5.125409in}{4.132517in}}%
\pgfpathlineto{\pgfqpoint{5.127422in}{4.143501in}}%
\pgfpathlineto{\pgfqpoint{5.129435in}{4.137607in}}%
\pgfpathlineto{\pgfqpoint{5.131449in}{4.136268in}}%
\pgfpathlineto{\pgfqpoint{5.137489in}{4.145109in}}%
\pgfpathlineto{\pgfqpoint{5.139503in}{4.140822in}}%
\pgfpathlineto{\pgfqpoint{5.141516in}{4.134660in}}%
\pgfpathlineto{\pgfqpoint{5.143530in}{4.146448in}}%
\pgfpathlineto{\pgfqpoint{5.145543in}{4.150467in}}%
\pgfpathlineto{\pgfqpoint{5.151583in}{4.144305in}}%
\pgfpathlineto{\pgfqpoint{5.153597in}{4.161452in}}%
\pgfpathlineto{\pgfqpoint{5.155610in}{4.168417in}}%
\pgfpathlineto{\pgfqpoint{5.157624in}{4.169489in}}%
\pgfpathlineto{\pgfqpoint{5.159637in}{4.175115in}}%
\pgfpathlineto{\pgfqpoint{5.165677in}{4.171096in}}%
\pgfpathlineto{\pgfqpoint{5.167691in}{4.165738in}}%
\pgfpathlineto{\pgfqpoint{5.169704in}{4.165470in}}%
\pgfpathlineto{\pgfqpoint{5.171718in}{4.163059in}}%
\pgfpathlineto{\pgfqpoint{5.173731in}{4.172168in}}%
\pgfpathlineto{\pgfqpoint{5.179772in}{4.170561in}}%
\pgfpathlineto{\pgfqpoint{5.181785in}{4.170828in}}%
\pgfpathlineto{\pgfqpoint{5.183799in}{4.168417in}}%
\pgfpathlineto{\pgfqpoint{5.185812in}{4.186635in}}%
\pgfpathlineto{\pgfqpoint{5.187825in}{4.185832in}}%
\pgfpathlineto{\pgfqpoint{5.193866in}{4.189582in}}%
\pgfpathlineto{\pgfqpoint{5.195879in}{4.189046in}}%
\pgfpathlineto{\pgfqpoint{5.199906in}{4.190118in}}%
\pgfpathlineto{\pgfqpoint{5.207960in}{4.198959in}}%
\pgfpathlineto{\pgfqpoint{5.209973in}{4.198691in}}%
\pgfpathlineto{\pgfqpoint{5.211987in}{4.206997in}}%
\pgfpathlineto{\pgfqpoint{5.214000in}{4.205925in}}%
\pgfpathlineto{\pgfqpoint{5.216014in}{4.209140in}}%
\pgfpathlineto{\pgfqpoint{5.224067in}{4.187171in}}%
\pgfpathlineto{\pgfqpoint{5.226081in}{4.185296in}}%
\pgfpathlineto{\pgfqpoint{5.228094in}{4.179134in}}%
\pgfpathlineto{\pgfqpoint{5.230108in}{4.182617in}}%
\pgfpathlineto{\pgfqpoint{5.236148in}{4.180741in}}%
\pgfpathlineto{\pgfqpoint{5.238162in}{4.183152in}}%
\pgfpathlineto{\pgfqpoint{5.240175in}{4.187171in}}%
\pgfpathlineto{\pgfqpoint{5.244202in}{4.188243in}}%
\pgfpathlineto{\pgfqpoint{5.250242in}{4.189046in}}%
\pgfpathlineto{\pgfqpoint{5.252256in}{4.191190in}}%
\pgfpathlineto{\pgfqpoint{5.254269in}{4.191190in}}%
\pgfpathlineto{\pgfqpoint{5.258296in}{4.182081in}}%
\pgfpathlineto{\pgfqpoint{5.264336in}{4.180205in}}%
\pgfpathlineto{\pgfqpoint{5.266350in}{4.185296in}}%
\pgfpathlineto{\pgfqpoint{5.268363in}{4.186099in}}%
\pgfpathlineto{\pgfqpoint{5.270377in}{4.185296in}}%
\pgfpathlineto{\pgfqpoint{5.272390in}{4.181813in}}%
\pgfpathlineto{\pgfqpoint{5.278431in}{4.185028in}}%
\pgfpathlineto{\pgfqpoint{5.280444in}{4.179134in}}%
\pgfpathlineto{\pgfqpoint{5.282457in}{4.166006in}}%
\pgfpathlineto{\pgfqpoint{5.284471in}{4.161719in}}%
\pgfpathlineto{\pgfqpoint{5.286484in}{4.167078in}}%
\pgfpathlineto{\pgfqpoint{5.292525in}{4.161184in}}%
\pgfpathlineto{\pgfqpoint{5.294538in}{4.175383in}}%
\pgfpathlineto{\pgfqpoint{5.296552in}{4.172168in}}%
\pgfpathlineto{\pgfqpoint{5.298565in}{4.166542in}}%
\pgfpathlineto{\pgfqpoint{5.300578in}{4.156093in}}%
\pgfpathlineto{\pgfqpoint{5.306619in}{4.163327in}}%
\pgfpathlineto{\pgfqpoint{5.308632in}{4.157701in}}%
\pgfpathlineto{\pgfqpoint{5.310646in}{4.155290in}}%
\pgfpathlineto{\pgfqpoint{5.312659in}{4.149395in}}%
\pgfpathlineto{\pgfqpoint{5.314673in}{4.154218in}}%
\pgfpathlineto{\pgfqpoint{5.320713in}{4.152343in}}%
\pgfpathlineto{\pgfqpoint{5.322726in}{4.160916in}}%
\pgfpathlineto{\pgfqpoint{5.324740in}{4.166542in}}%
\pgfpathlineto{\pgfqpoint{5.326753in}{4.164934in}}%
\pgfpathlineto{\pgfqpoint{5.328767in}{4.167078in}}%
\pgfpathlineto{\pgfqpoint{5.336820in}{4.169757in}}%
\pgfpathlineto{\pgfqpoint{5.338834in}{4.166810in}}%
\pgfpathlineto{\pgfqpoint{5.340847in}{4.165470in}}%
\pgfpathlineto{\pgfqpoint{5.342861in}{4.163059in}}%
\pgfpathlineto{\pgfqpoint{5.348901in}{4.167078in}}%
\pgfpathlineto{\pgfqpoint{5.350915in}{4.167613in}}%
\pgfpathlineto{\pgfqpoint{5.352928in}{4.171632in}}%
\pgfpathlineto{\pgfqpoint{5.354942in}{4.170025in}}%
\pgfpathlineto{\pgfqpoint{5.356955in}{4.165202in}}%
\pgfpathlineto{\pgfqpoint{5.362995in}{4.160380in}}%
\pgfpathlineto{\pgfqpoint{5.365009in}{4.174579in}}%
\pgfpathlineto{\pgfqpoint{5.367022in}{4.178062in}}%
\pgfpathlineto{\pgfqpoint{5.369036in}{4.184760in}}%
\pgfpathlineto{\pgfqpoint{5.371049in}{4.183688in}}%
\pgfpathlineto{\pgfqpoint{5.377089in}{4.189046in}}%
\pgfpathlineto{\pgfqpoint{5.379103in}{4.191726in}}%
\pgfpathlineto{\pgfqpoint{5.381116in}{4.187975in}}%
\pgfpathlineto{\pgfqpoint{5.383130in}{4.197620in}}%
\pgfpathlineto{\pgfqpoint{5.385143in}{4.165470in}}%
\pgfpathlineto{\pgfqpoint{5.391184in}{4.153682in}}%
\pgfpathlineto{\pgfqpoint{5.395210in}{4.182617in}}%
\pgfpathlineto{\pgfqpoint{5.397224in}{4.204317in}}%
\pgfpathlineto{\pgfqpoint{5.399237in}{4.204585in}}%
\pgfpathlineto{\pgfqpoint{5.407291in}{4.203514in}}%
\pgfpathlineto{\pgfqpoint{5.409305in}{4.210479in}}%
\pgfpathlineto{\pgfqpoint{5.411318in}{4.212355in}}%
\pgfpathlineto{\pgfqpoint{5.413331in}{4.221196in}}%
\pgfpathlineto{\pgfqpoint{5.419372in}{4.221464in}}%
\pgfpathlineto{\pgfqpoint{5.421385in}{4.222535in}}%
\pgfpathlineto{\pgfqpoint{5.423399in}{4.224947in}}%
\pgfpathlineto{\pgfqpoint{5.427426in}{4.237271in}}%
\pgfpathlineto{\pgfqpoint{5.435479in}{4.238342in}}%
\pgfpathlineto{\pgfqpoint{5.439506in}{4.230305in}}%
\pgfpathlineto{\pgfqpoint{5.441520in}{4.217981in}}%
\pgfpathlineto{\pgfqpoint{5.449574in}{4.204050in}}%
\pgfpathlineto{\pgfqpoint{5.451587in}{4.199763in}}%
\pgfpathlineto{\pgfqpoint{5.453600in}{4.198959in}}%
\pgfpathlineto{\pgfqpoint{5.455614in}{4.196280in}}%
\pgfpathlineto{\pgfqpoint{5.461654in}{4.196548in}}%
\pgfpathlineto{\pgfqpoint{5.463668in}{4.194137in}}%
\pgfpathlineto{\pgfqpoint{5.465681in}{4.196012in}}%
\pgfpathlineto{\pgfqpoint{5.467695in}{4.197084in}}%
\pgfpathlineto{\pgfqpoint{5.469708in}{4.199763in}}%
\pgfpathlineto{\pgfqpoint{5.475748in}{4.199495in}}%
\pgfpathlineto{\pgfqpoint{5.477762in}{4.200031in}}%
\pgfpathlineto{\pgfqpoint{5.479775in}{4.199495in}}%
\pgfpathlineto{\pgfqpoint{5.481789in}{4.199763in}}%
\pgfpathlineto{\pgfqpoint{5.483802in}{4.198691in}}%
\pgfpathlineto{\pgfqpoint{5.489842in}{4.198691in}}%
\pgfpathlineto{\pgfqpoint{5.491856in}{4.197620in}}%
\pgfpathlineto{\pgfqpoint{5.493869in}{4.199763in}}%
\pgfpathlineto{\pgfqpoint{5.495883in}{4.203246in}}%
\pgfpathlineto{\pgfqpoint{5.497896in}{4.198959in}}%
\pgfpathlineto{\pgfqpoint{5.503937in}{4.200567in}}%
\pgfpathlineto{\pgfqpoint{5.505950in}{4.198423in}}%
\pgfpathlineto{\pgfqpoint{5.509977in}{4.197888in}}%
\pgfpathlineto{\pgfqpoint{5.511990in}{4.198423in}}%
\pgfpathlineto{\pgfqpoint{5.518031in}{4.201638in}}%
\pgfpathlineto{\pgfqpoint{5.520044in}{4.201638in}}%
\pgfpathlineto{\pgfqpoint{5.522058in}{4.198691in}}%
\pgfpathlineto{\pgfqpoint{5.524071in}{4.197888in}}%
\pgfpathlineto{\pgfqpoint{5.526085in}{4.199763in}}%
\pgfpathlineto{\pgfqpoint{5.534138in}{4.194137in}}%
\pgfpathlineto{\pgfqpoint{5.538165in}{4.194137in}}%
\pgfpathlineto{\pgfqpoint{5.540179in}{4.172168in}}%
\pgfpathlineto{\pgfqpoint{5.546219in}{4.181009in}}%
\pgfpathlineto{\pgfqpoint{5.548232in}{4.166006in}}%
\pgfpathlineto{\pgfqpoint{5.550246in}{4.162523in}}%
\pgfpathlineto{\pgfqpoint{5.552259in}{4.169221in}}%
\pgfpathlineto{\pgfqpoint{5.554273in}{4.167613in}}%
\pgfpathlineto{\pgfqpoint{5.560313in}{4.161719in}}%
\pgfpathlineto{\pgfqpoint{5.564340in}{4.171632in}}%
\pgfpathlineto{\pgfqpoint{5.566353in}{4.176187in}}%
\pgfpathlineto{\pgfqpoint{5.568367in}{4.172436in}}%
\pgfpathlineto{\pgfqpoint{5.574407in}{4.164131in}}%
\pgfpathlineto{\pgfqpoint{5.576421in}{4.172168in}}%
\pgfpathlineto{\pgfqpoint{5.578434in}{4.172704in}}%
\pgfpathlineto{\pgfqpoint{5.580448in}{4.164131in}}%
\pgfpathlineto{\pgfqpoint{5.582461in}{4.166006in}}%
\pgfpathlineto{\pgfqpoint{5.588501in}{4.166542in}}%
\pgfpathlineto{\pgfqpoint{5.590515in}{4.163327in}}%
\pgfpathlineto{\pgfqpoint{5.592528in}{4.163327in}}%
\pgfpathlineto{\pgfqpoint{5.596555in}{4.153414in}}%
\pgfpathlineto{\pgfqpoint{5.602596in}{4.148056in}}%
\pgfpathlineto{\pgfqpoint{5.604609in}{4.149663in}}%
\pgfpathlineto{\pgfqpoint{5.606622in}{4.149128in}}%
\pgfpathlineto{\pgfqpoint{5.608636in}{4.146181in}}%
\pgfpathlineto{\pgfqpoint{5.610649in}{4.148860in}}%
\pgfpathlineto{\pgfqpoint{5.616690in}{4.148056in}}%
\pgfpathlineto{\pgfqpoint{5.618703in}{4.151003in}}%
\pgfpathlineto{\pgfqpoint{5.620717in}{4.152878in}}%
\pgfpathlineto{\pgfqpoint{5.622730in}{4.153146in}}%
\pgfpathlineto{\pgfqpoint{5.624743in}{4.151003in}}%
\pgfpathlineto{\pgfqpoint{5.630784in}{4.149663in}}%
\pgfpathlineto{\pgfqpoint{5.632797in}{4.143234in}}%
\pgfpathlineto{\pgfqpoint{5.634811in}{4.148324in}}%
\pgfpathlineto{\pgfqpoint{5.636824in}{4.142698in}}%
\pgfpathlineto{\pgfqpoint{5.638838in}{4.156629in}}%
\pgfpathlineto{\pgfqpoint{5.644878in}{4.153950in}}%
\pgfpathlineto{\pgfqpoint{5.646891in}{4.148592in}}%
\pgfpathlineto{\pgfqpoint{5.648905in}{4.139483in}}%
\pgfpathlineto{\pgfqpoint{5.650918in}{4.134392in}}%
\pgfpathlineto{\pgfqpoint{5.652932in}{4.138143in}}%
\pgfpathlineto{\pgfqpoint{5.658972in}{4.158772in}}%
\pgfpathlineto{\pgfqpoint{5.660985in}{4.161452in}}%
\pgfpathlineto{\pgfqpoint{5.662999in}{4.166274in}}%
\pgfpathlineto{\pgfqpoint{5.665012in}{4.184760in}}%
\pgfpathlineto{\pgfqpoint{5.667026in}{4.191993in}}%
\pgfpathlineto{\pgfqpoint{5.673066in}{4.187171in}}%
\pgfpathlineto{\pgfqpoint{5.675080in}{4.192797in}}%
\pgfpathlineto{\pgfqpoint{5.677093in}{4.192529in}}%
\pgfpathlineto{\pgfqpoint{5.679107in}{4.193869in}}%
\pgfpathlineto{\pgfqpoint{5.681120in}{4.190922in}}%
\pgfpathlineto{\pgfqpoint{5.687160in}{4.195744in}}%
\pgfpathlineto{\pgfqpoint{5.689174in}{4.202978in}}%
\pgfpathlineto{\pgfqpoint{5.691187in}{4.206729in}}%
\pgfpathlineto{\pgfqpoint{5.695214in}{4.209140in}}%
\pgfpathlineto{\pgfqpoint{5.701254in}{4.204585in}}%
\pgfpathlineto{\pgfqpoint{5.703268in}{4.200031in}}%
\pgfpathlineto{\pgfqpoint{5.705281in}{4.193065in}}%
\pgfpathlineto{\pgfqpoint{5.707295in}{4.208068in}}%
\pgfpathlineto{\pgfqpoint{5.709308in}{4.206729in}}%
\pgfpathlineto{\pgfqpoint{5.715349in}{4.201370in}}%
\pgfpathlineto{\pgfqpoint{5.717362in}{4.202710in}}%
\pgfpathlineto{\pgfqpoint{5.719375in}{4.212891in}}%
\pgfpathlineto{\pgfqpoint{5.721389in}{4.211283in}}%
\pgfpathlineto{\pgfqpoint{5.723402in}{4.217177in}}%
\pgfpathlineto{\pgfqpoint{5.729443in}{4.219053in}}%
\pgfpathlineto{\pgfqpoint{5.731456in}{4.216106in}}%
\pgfpathlineto{\pgfqpoint{5.735483in}{4.204853in}}%
\pgfpathlineto{\pgfqpoint{5.737496in}{4.216373in}}%
\pgfpathlineto{\pgfqpoint{5.743537in}{4.220392in}}%
\pgfpathlineto{\pgfqpoint{5.745550in}{4.228162in}}%
\pgfpathlineto{\pgfqpoint{5.747564in}{4.225482in}}%
\pgfpathlineto{\pgfqpoint{5.749577in}{4.223875in}}%
\pgfpathlineto{\pgfqpoint{5.751591in}{4.225215in}}%
\pgfpathlineto{\pgfqpoint{5.759644in}{4.225750in}}%
\pgfpathlineto{\pgfqpoint{5.761658in}{4.220928in}}%
\pgfpathlineto{\pgfqpoint{5.763671in}{4.221196in}}%
\pgfpathlineto{\pgfqpoint{5.765685in}{4.218517in}}%
\pgfpathlineto{\pgfqpoint{5.775752in}{4.220928in}}%
\pgfpathlineto{\pgfqpoint{5.777765in}{4.216641in}}%
\pgfpathlineto{\pgfqpoint{5.779779in}{4.218785in}}%
\pgfpathlineto{\pgfqpoint{5.785819in}{4.215302in}}%
\pgfpathlineto{\pgfqpoint{5.787833in}{4.213159in}}%
\pgfpathlineto{\pgfqpoint{5.789846in}{4.215570in}}%
\pgfpathlineto{\pgfqpoint{5.791860in}{4.213426in}}%
\pgfpathlineto{\pgfqpoint{5.805954in}{4.209140in}}%
\pgfpathlineto{\pgfqpoint{5.807967in}{4.193065in}}%
\pgfpathlineto{\pgfqpoint{5.814007in}{4.174579in}}%
\pgfpathlineto{\pgfqpoint{5.816021in}{4.180473in}}%
\pgfpathlineto{\pgfqpoint{5.818034in}{4.189314in}}%
\pgfpathlineto{\pgfqpoint{5.820048in}{4.187975in}}%
\pgfpathlineto{\pgfqpoint{5.822061in}{4.180741in}}%
\pgfpathlineto{\pgfqpoint{5.828102in}{4.179402in}}%
\pgfpathlineto{\pgfqpoint{5.830115in}{4.173240in}}%
\pgfpathlineto{\pgfqpoint{5.834142in}{4.172704in}}%
\pgfpathlineto{\pgfqpoint{5.836155in}{4.173240in}}%
\pgfpathlineto{\pgfqpoint{5.842196in}{4.172436in}}%
\pgfpathlineto{\pgfqpoint{5.846223in}{4.166810in}}%
\pgfpathlineto{\pgfqpoint{5.850250in}{4.173775in}}%
\pgfpathlineto{\pgfqpoint{5.856290in}{4.181277in}}%
\pgfpathlineto{\pgfqpoint{5.858303in}{4.187171in}}%
\pgfpathlineto{\pgfqpoint{5.860317in}{4.188779in}}%
\pgfpathlineto{\pgfqpoint{5.862330in}{4.191190in}}%
\pgfpathlineto{\pgfqpoint{5.864344in}{4.189314in}}%
\pgfpathlineto{\pgfqpoint{5.872397in}{4.192797in}}%
\pgfpathlineto{\pgfqpoint{5.874411in}{4.188243in}}%
\pgfpathlineto{\pgfqpoint{5.876424in}{4.186635in}}%
\pgfpathlineto{\pgfqpoint{5.878438in}{4.190654in}}%
\pgfpathlineto{\pgfqpoint{5.884478in}{4.184760in}}%
\pgfpathlineto{\pgfqpoint{5.886492in}{4.181545in}}%
\pgfpathlineto{\pgfqpoint{5.888505in}{4.190654in}}%
\pgfpathlineto{\pgfqpoint{5.890518in}{4.190654in}}%
\pgfpathlineto{\pgfqpoint{5.892532in}{4.189046in}}%
\pgfpathlineto{\pgfqpoint{5.898572in}{4.186099in}}%
\pgfpathlineto{\pgfqpoint{5.900586in}{4.182884in}}%
\pgfpathlineto{\pgfqpoint{5.902599in}{4.181277in}}%
\pgfpathlineto{\pgfqpoint{5.904613in}{4.178062in}}%
\pgfpathlineto{\pgfqpoint{5.906626in}{4.192797in}}%
\pgfpathlineto{\pgfqpoint{5.912666in}{4.182884in}}%
\pgfpathlineto{\pgfqpoint{5.914680in}{4.175115in}}%
\pgfpathlineto{\pgfqpoint{5.916693in}{4.180473in}}%
\pgfpathlineto{\pgfqpoint{5.918707in}{4.180205in}}%
\pgfpathlineto{\pgfqpoint{5.920720in}{4.183152in}}%
\pgfpathlineto{\pgfqpoint{5.926761in}{4.179937in}}%
\pgfpathlineto{\pgfqpoint{5.928774in}{4.171632in}}%
\pgfpathlineto{\pgfqpoint{5.930787in}{4.174847in}}%
\pgfpathlineto{\pgfqpoint{5.934814in}{4.179402in}}%
\pgfpathlineto{\pgfqpoint{5.940855in}{4.172704in}}%
\pgfpathlineto{\pgfqpoint{5.942868in}{4.176990in}}%
\pgfpathlineto{\pgfqpoint{5.944882in}{4.178598in}}%
\pgfpathlineto{\pgfqpoint{5.946895in}{4.183152in}}%
\pgfpathlineto{\pgfqpoint{5.948908in}{4.181277in}}%
\pgfpathlineto{\pgfqpoint{5.954949in}{4.183152in}}%
\pgfpathlineto{\pgfqpoint{5.956962in}{4.186635in}}%
\pgfpathlineto{\pgfqpoint{5.960989in}{4.184492in}}%
\pgfpathlineto{\pgfqpoint{5.963003in}{4.185832in}}%
\pgfpathlineto{\pgfqpoint{5.969043in}{4.186367in}}%
\pgfpathlineto{\pgfqpoint{5.971056in}{4.187171in}}%
\pgfpathlineto{\pgfqpoint{5.975083in}{4.175651in}}%
\pgfpathlineto{\pgfqpoint{5.983137in}{4.177526in}}%
\pgfpathlineto{\pgfqpoint{5.989177in}{4.192529in}}%
\pgfpathlineto{\pgfqpoint{5.991191in}{4.175383in}}%
\pgfpathlineto{\pgfqpoint{5.997231in}{4.175383in}}%
\pgfpathlineto{\pgfqpoint{5.999245in}{4.172972in}}%
\pgfpathlineto{\pgfqpoint{6.003272in}{4.164131in}}%
\pgfpathlineto{\pgfqpoint{6.005285in}{4.161987in}}%
\pgfpathlineto{\pgfqpoint{6.011325in}{4.160648in}}%
\pgfpathlineto{\pgfqpoint{6.013339in}{4.161987in}}%
\pgfpathlineto{\pgfqpoint{6.015352in}{4.167613in}}%
\pgfpathlineto{\pgfqpoint{6.017366in}{4.167078in}}%
\pgfpathlineto{\pgfqpoint{6.019379in}{4.167346in}}%
\pgfpathlineto{\pgfqpoint{6.025419in}{4.163863in}}%
\pgfpathlineto{\pgfqpoint{6.027433in}{4.160380in}}%
\pgfpathlineto{\pgfqpoint{6.029446in}{4.155022in}}%
\pgfpathlineto{\pgfqpoint{6.031460in}{4.159040in}}%
\pgfpathlineto{\pgfqpoint{6.033473in}{4.144573in}}%
\pgfpathlineto{\pgfqpoint{6.039514in}{4.142430in}}%
\pgfpathlineto{\pgfqpoint{6.041527in}{4.139215in}}%
\pgfpathlineto{\pgfqpoint{6.043540in}{4.123944in}}%
\pgfpathlineto{\pgfqpoint{6.045554in}{4.125551in}}%
\pgfpathlineto{\pgfqpoint{6.047567in}{4.139483in}}%
\pgfpathlineto{\pgfqpoint{6.053608in}{4.142430in}}%
\pgfpathlineto{\pgfqpoint{6.055621in}{4.144841in}}%
\pgfpathlineto{\pgfqpoint{6.059648in}{4.125819in}}%
\pgfpathlineto{\pgfqpoint{6.061661in}{4.125015in}}%
\pgfpathlineto{\pgfqpoint{6.069715in}{4.122872in}}%
\pgfpathlineto{\pgfqpoint{6.071729in}{4.123408in}}%
\pgfpathlineto{\pgfqpoint{6.073742in}{4.131445in}}%
\pgfpathlineto{\pgfqpoint{6.075756in}{4.135196in}}%
\pgfpathlineto{\pgfqpoint{6.081796in}{4.137607in}}%
\pgfpathlineto{\pgfqpoint{6.083809in}{4.136536in}}%
\pgfpathlineto{\pgfqpoint{6.085823in}{4.130374in}}%
\pgfpathlineto{\pgfqpoint{6.087836in}{4.128230in}}%
\pgfpathlineto{\pgfqpoint{6.095890in}{4.160648in}}%
\pgfpathlineto{\pgfqpoint{6.097904in}{4.148860in}}%
\pgfpathlineto{\pgfqpoint{6.099917in}{4.154754in}}%
\pgfpathlineto{\pgfqpoint{6.101930in}{4.166542in}}%
\pgfpathlineto{\pgfqpoint{6.103944in}{4.174043in}}%
\pgfpathlineto{\pgfqpoint{6.109984in}{4.169221in}}%
\pgfpathlineto{\pgfqpoint{6.111998in}{4.152878in}}%
\pgfpathlineto{\pgfqpoint{6.114011in}{4.144305in}}%
\pgfpathlineto{\pgfqpoint{6.116025in}{4.138679in}}%
\pgfpathlineto{\pgfqpoint{6.124078in}{4.140286in}}%
\pgfpathlineto{\pgfqpoint{6.126092in}{4.130374in}}%
\pgfpathlineto{\pgfqpoint{6.128105in}{4.127159in}}%
\pgfpathlineto{\pgfqpoint{6.130119in}{4.125819in}}%
\pgfpathlineto{\pgfqpoint{6.132132in}{4.125551in}}%
\pgfpathlineto{\pgfqpoint{6.138172in}{4.136268in}}%
\pgfpathlineto{\pgfqpoint{6.142199in}{4.133857in}}%
\pgfpathlineto{\pgfqpoint{6.144213in}{4.108405in}}%
\pgfpathlineto{\pgfqpoint{6.146226in}{4.104654in}}%
\pgfpathlineto{\pgfqpoint{6.152267in}{4.101975in}}%
\pgfpathlineto{\pgfqpoint{6.154280in}{4.110280in}}%
\pgfpathlineto{\pgfqpoint{6.158307in}{4.120193in}}%
\pgfpathlineto{\pgfqpoint{6.160320in}{4.119925in}}%
\pgfpathlineto{\pgfqpoint{6.166361in}{4.120997in}}%
\pgfpathlineto{\pgfqpoint{6.170388in}{4.123944in}}%
\pgfpathlineto{\pgfqpoint{6.172401in}{4.117782in}}%
\pgfpathlineto{\pgfqpoint{6.174415in}{4.098760in}}%
\pgfpathlineto{\pgfqpoint{6.180455in}{4.086972in}}%
\pgfpathlineto{\pgfqpoint{6.182468in}{4.087240in}}%
\pgfpathlineto{\pgfqpoint{6.186495in}{4.095813in}}%
\pgfpathlineto{\pgfqpoint{6.188509in}{4.089383in}}%
\pgfpathlineto{\pgfqpoint{6.194549in}{4.091259in}}%
\pgfpathlineto{\pgfqpoint{6.196562in}{4.087240in}}%
\pgfpathlineto{\pgfqpoint{6.198576in}{4.089115in}}%
\pgfpathlineto{\pgfqpoint{6.200589in}{4.095009in}}%
\pgfpathlineto{\pgfqpoint{6.202603in}{4.095545in}}%
\pgfpathlineto{\pgfqpoint{6.210657in}{4.090187in}}%
\pgfpathlineto{\pgfqpoint{6.212670in}{4.093938in}}%
\pgfpathlineto{\pgfqpoint{6.214683in}{4.083757in}}%
\pgfpathlineto{\pgfqpoint{6.216697in}{4.081346in}}%
\pgfpathlineto{\pgfqpoint{6.222737in}{4.085365in}}%
\pgfpathlineto{\pgfqpoint{6.224751in}{4.080006in}}%
\pgfpathlineto{\pgfqpoint{6.226764in}{4.078935in}}%
\pgfpathlineto{\pgfqpoint{6.228778in}{4.070361in}}%
\pgfpathlineto{\pgfqpoint{6.230791in}{4.065539in}}%
\pgfpathlineto{\pgfqpoint{6.236831in}{4.063932in}}%
\pgfpathlineto{\pgfqpoint{6.238845in}{4.066611in}}%
\pgfpathlineto{\pgfqpoint{6.240858in}{4.061520in}}%
\pgfpathlineto{\pgfqpoint{6.242872in}{4.059377in}}%
\pgfpathlineto{\pgfqpoint{6.244885in}{4.063932in}}%
\pgfpathlineto{\pgfqpoint{6.250926in}{4.063664in}}%
\pgfpathlineto{\pgfqpoint{6.252939in}{4.062860in}}%
\pgfpathlineto{\pgfqpoint{6.254952in}{4.058841in}}%
\pgfpathlineto{\pgfqpoint{6.256966in}{4.065539in}}%
\pgfpathlineto{\pgfqpoint{6.258979in}{4.080006in}}%
\pgfpathlineto{\pgfqpoint{6.267033in}{4.070361in}}%
\pgfpathlineto{\pgfqpoint{6.269047in}{4.074648in}}%
\pgfpathlineto{\pgfqpoint{6.271060in}{4.052679in}}%
\pgfpathlineto{\pgfqpoint{6.273073in}{4.047589in}}%
\pgfpathlineto{\pgfqpoint{6.279114in}{4.045178in}}%
\pgfpathlineto{\pgfqpoint{6.285154in}{4.058305in}}%
\pgfpathlineto{\pgfqpoint{6.287168in}{4.056162in}}%
\pgfpathlineto{\pgfqpoint{6.293208in}{4.069290in}}%
\pgfpathlineto{\pgfqpoint{6.295221in}{4.062860in}}%
\pgfpathlineto{\pgfqpoint{6.297235in}{4.065807in}}%
\pgfpathlineto{\pgfqpoint{6.299248in}{4.076523in}}%
\pgfpathlineto{\pgfqpoint{6.301262in}{4.079470in}}%
\pgfpathlineto{\pgfqpoint{6.307302in}{4.085365in}}%
\pgfpathlineto{\pgfqpoint{6.309316in}{4.080810in}}%
\pgfpathlineto{\pgfqpoint{6.311329in}{4.067147in}}%
\pgfpathlineto{\pgfqpoint{6.313342in}{4.063932in}}%
\pgfpathlineto{\pgfqpoint{6.315356in}{4.062324in}}%
\pgfpathlineto{\pgfqpoint{6.321396in}{4.071969in}}%
\pgfpathlineto{\pgfqpoint{6.323410in}{4.077595in}}%
\pgfpathlineto{\pgfqpoint{6.325423in}{4.069826in}}%
\pgfpathlineto{\pgfqpoint{6.327437in}{4.071165in}}%
\pgfpathlineto{\pgfqpoint{6.329450in}{4.067414in}}%
\pgfpathlineto{\pgfqpoint{6.335490in}{4.043838in}}%
\pgfpathlineto{\pgfqpoint{6.337504in}{4.042231in}}%
\pgfpathlineto{\pgfqpoint{6.339517in}{4.034997in}}%
\pgfpathlineto{\pgfqpoint{6.341531in}{4.034461in}}%
\pgfpathlineto{\pgfqpoint{6.343544in}{4.032854in}}%
\pgfpathlineto{\pgfqpoint{6.349584in}{4.042231in}}%
\pgfpathlineto{\pgfqpoint{6.351598in}{4.037944in}}%
\pgfpathlineto{\pgfqpoint{6.353611in}{4.036337in}}%
\pgfpathlineto{\pgfqpoint{6.355625in}{4.047589in}}%
\pgfpathlineto{\pgfqpoint{6.357638in}{4.053751in}}%
\pgfpathlineto{\pgfqpoint{6.365692in}{4.005795in}}%
\pgfpathlineto{\pgfqpoint{6.367705in}{3.996418in}}%
\pgfpathlineto{\pgfqpoint{6.369719in}{3.991863in}}%
\pgfpathlineto{\pgfqpoint{6.371732in}{3.978736in}}%
\pgfpathlineto{\pgfqpoint{6.377773in}{3.969359in}}%
\pgfpathlineto{\pgfqpoint{6.379786in}{3.963197in}}%
\pgfpathlineto{\pgfqpoint{6.381800in}{3.959714in}}%
\pgfpathlineto{\pgfqpoint{6.383813in}{3.957838in}}%
\pgfpathlineto{\pgfqpoint{6.385827in}{3.962661in}}%
\pgfpathlineto{\pgfqpoint{6.391867in}{3.962393in}}%
\pgfpathlineto{\pgfqpoint{6.393880in}{3.964536in}}%
\pgfpathlineto{\pgfqpoint{6.395894in}{3.962393in}}%
\pgfpathlineto{\pgfqpoint{6.397907in}{3.959178in}}%
\pgfpathlineto{\pgfqpoint{6.399921in}{3.971502in}}%
\pgfpathlineto{\pgfqpoint{6.405961in}{3.935066in}}%
\pgfpathlineto{\pgfqpoint{6.407974in}{3.907471in}}%
\pgfpathlineto{\pgfqpoint{6.409988in}{3.916312in}}%
\pgfpathlineto{\pgfqpoint{6.414015in}{3.915240in}}%
\pgfpathlineto{\pgfqpoint{6.420055in}{3.909614in}}%
\pgfpathlineto{\pgfqpoint{6.422069in}{3.905864in}}%
\pgfpathlineto{\pgfqpoint{6.424082in}{3.913633in}}%
\pgfpathlineto{\pgfqpoint{6.428109in}{3.914705in}}%
\pgfpathlineto{\pgfqpoint{6.434149in}{3.912829in}}%
\pgfpathlineto{\pgfqpoint{6.436163in}{3.920063in}}%
\pgfpathlineto{\pgfqpoint{6.438176in}{3.921938in}}%
\pgfpathlineto{\pgfqpoint{6.440190in}{3.917116in}}%
\pgfpathlineto{\pgfqpoint{6.442203in}{3.906935in}}%
\pgfpathlineto{\pgfqpoint{6.448243in}{3.908811in}}%
\pgfpathlineto{\pgfqpoint{6.450257in}{3.903988in}}%
\pgfpathlineto{\pgfqpoint{6.456297in}{3.905864in}}%
\pgfpathlineto{\pgfqpoint{6.462337in}{3.904524in}}%
\pgfpathlineto{\pgfqpoint{6.464351in}{3.910954in}}%
\pgfpathlineto{\pgfqpoint{6.468378in}{3.904256in}}%
\pgfpathlineto{\pgfqpoint{6.470391in}{3.908543in}}%
\pgfpathlineto{\pgfqpoint{6.476432in}{3.907203in}}%
\pgfpathlineto{\pgfqpoint{6.480459in}{3.899434in}}%
\pgfpathlineto{\pgfqpoint{6.484485in}{3.900773in}}%
\pgfpathlineto{\pgfqpoint{6.492539in}{3.901845in}}%
\pgfpathlineto{\pgfqpoint{6.496566in}{3.900237in}}%
\pgfpathlineto{\pgfqpoint{6.498580in}{3.902381in}}%
\pgfpathlineto{\pgfqpoint{6.498580in}{3.902381in}}%
\pgfusepath{stroke}%
\end{pgfscope}%
\begin{pgfscope}%
\pgfsetrectcap%
\pgfsetmiterjoin%
\pgfsetlinewidth{0.803000pt}%
\definecolor{currentstroke}{rgb}{1.000000,1.000000,1.000000}%
\pgfsetstrokecolor{currentstroke}%
\pgfsetdash{}{0pt}%
\pgfpathmoveto{\pgfqpoint{1.875000in}{3.814412in}}%
\pgfpathlineto{\pgfqpoint{1.875000in}{4.258529in}}%
\pgfusepath{stroke}%
\end{pgfscope}%
\begin{pgfscope}%
\pgfsetrectcap%
\pgfsetmiterjoin%
\pgfsetlinewidth{0.803000pt}%
\definecolor{currentstroke}{rgb}{1.000000,1.000000,1.000000}%
\pgfsetstrokecolor{currentstroke}%
\pgfsetdash{}{0pt}%
\pgfpathmoveto{\pgfqpoint{6.718750in}{3.814412in}}%
\pgfpathlineto{\pgfqpoint{6.718750in}{4.258529in}}%
\pgfusepath{stroke}%
\end{pgfscope}%
\begin{pgfscope}%
\pgfsetrectcap%
\pgfsetmiterjoin%
\pgfsetlinewidth{0.803000pt}%
\definecolor{currentstroke}{rgb}{1.000000,1.000000,1.000000}%
\pgfsetstrokecolor{currentstroke}%
\pgfsetdash{}{0pt}%
\pgfpathmoveto{\pgfqpoint{1.875000in}{3.814412in}}%
\pgfpathlineto{\pgfqpoint{6.718750in}{3.814412in}}%
\pgfusepath{stroke}%
\end{pgfscope}%
\begin{pgfscope}%
\pgfsetrectcap%
\pgfsetmiterjoin%
\pgfsetlinewidth{0.803000pt}%
\definecolor{currentstroke}{rgb}{1.000000,1.000000,1.000000}%
\pgfsetstrokecolor{currentstroke}%
\pgfsetdash{}{0pt}%
\pgfpathmoveto{\pgfqpoint{1.875000in}{4.258529in}}%
\pgfpathlineto{\pgfqpoint{6.718750in}{4.258529in}}%
\pgfusepath{stroke}%
\end{pgfscope}%
\begin{pgfscope}%
\definecolor{textcolor}{rgb}{0.150000,0.150000,0.150000}%
\pgfsetstrokecolor{textcolor}%
\pgfsetfillcolor{textcolor}%
\pgftext[x=4.296875in,y=4.341863in,,base]{\color{textcolor}\rmfamily\fontsize{16.800000}{20.160000}\selectfont GE}%
\end{pgfscope}%
\begin{pgfscope}%
\pgfsetbuttcap%
\pgfsetmiterjoin%
\definecolor{currentfill}{rgb}{0.917647,0.917647,0.949020}%
\pgfsetfillcolor{currentfill}%
\pgfsetlinewidth{0.000000pt}%
\definecolor{currentstroke}{rgb}{0.000000,0.000000,0.000000}%
\pgfsetstrokecolor{currentstroke}%
\pgfsetstrokeopacity{0.000000}%
\pgfsetdash{}{0pt}%
\pgfpathmoveto{\pgfqpoint{8.656250in}{3.814412in}}%
\pgfpathlineto{\pgfqpoint{13.500000in}{3.814412in}}%
\pgfpathlineto{\pgfqpoint{13.500000in}{4.258529in}}%
\pgfpathlineto{\pgfqpoint{8.656250in}{4.258529in}}%
\pgfpathclose%
\pgfusepath{fill}%
\end{pgfscope}%
\begin{pgfscope}%
\pgfpathrectangle{\pgfqpoint{8.656250in}{3.814412in}}{\pgfqpoint{4.843750in}{0.444118in}}%
\pgfusepath{clip}%
\pgfsetroundcap%
\pgfsetroundjoin%
\pgfsetlinewidth{0.803000pt}%
\definecolor{currentstroke}{rgb}{1.000000,1.000000,1.000000}%
\pgfsetstrokecolor{currentstroke}%
\pgfsetdash{}{0pt}%
\pgfpathmoveto{\pgfqpoint{8.872394in}{3.814412in}}%
\pgfpathlineto{\pgfqpoint{8.872394in}{4.258529in}}%
\pgfusepath{stroke}%
\end{pgfscope}%
\begin{pgfscope}%
\definecolor{textcolor}{rgb}{0.150000,0.150000,0.150000}%
\pgfsetstrokecolor{textcolor}%
\pgfsetfillcolor{textcolor}%
\pgftext[x=8.872394in,y=3.717190in,,top]{\color{textcolor}\rmfamily\fontsize{14.000000}{16.800000}\selectfont 2012}%
\end{pgfscope}%
\begin{pgfscope}%
\pgfpathrectangle{\pgfqpoint{8.656250in}{3.814412in}}{\pgfqpoint{4.843750in}{0.444118in}}%
\pgfusepath{clip}%
\pgfsetroundcap%
\pgfsetroundjoin%
\pgfsetlinewidth{0.803000pt}%
\definecolor{currentstroke}{rgb}{1.000000,1.000000,1.000000}%
\pgfsetstrokecolor{currentstroke}%
\pgfsetdash{}{0pt}%
\pgfpathmoveto{\pgfqpoint{9.609315in}{3.814412in}}%
\pgfpathlineto{\pgfqpoint{9.609315in}{4.258529in}}%
\pgfusepath{stroke}%
\end{pgfscope}%
\begin{pgfscope}%
\definecolor{textcolor}{rgb}{0.150000,0.150000,0.150000}%
\pgfsetstrokecolor{textcolor}%
\pgfsetfillcolor{textcolor}%
\pgftext[x=9.609315in,y=3.717190in,,top]{\color{textcolor}\rmfamily\fontsize{14.000000}{16.800000}\selectfont 2013}%
\end{pgfscope}%
\begin{pgfscope}%
\pgfpathrectangle{\pgfqpoint{8.656250in}{3.814412in}}{\pgfqpoint{4.843750in}{0.444118in}}%
\pgfusepath{clip}%
\pgfsetroundcap%
\pgfsetroundjoin%
\pgfsetlinewidth{0.803000pt}%
\definecolor{currentstroke}{rgb}{1.000000,1.000000,1.000000}%
\pgfsetstrokecolor{currentstroke}%
\pgfsetdash{}{0pt}%
\pgfpathmoveto{\pgfqpoint{10.344223in}{3.814412in}}%
\pgfpathlineto{\pgfqpoint{10.344223in}{4.258529in}}%
\pgfusepath{stroke}%
\end{pgfscope}%
\begin{pgfscope}%
\definecolor{textcolor}{rgb}{0.150000,0.150000,0.150000}%
\pgfsetstrokecolor{textcolor}%
\pgfsetfillcolor{textcolor}%
\pgftext[x=10.344223in,y=3.717190in,,top]{\color{textcolor}\rmfamily\fontsize{14.000000}{16.800000}\selectfont 2014}%
\end{pgfscope}%
\begin{pgfscope}%
\pgfpathrectangle{\pgfqpoint{8.656250in}{3.814412in}}{\pgfqpoint{4.843750in}{0.444118in}}%
\pgfusepath{clip}%
\pgfsetroundcap%
\pgfsetroundjoin%
\pgfsetlinewidth{0.803000pt}%
\definecolor{currentstroke}{rgb}{1.000000,1.000000,1.000000}%
\pgfsetstrokecolor{currentstroke}%
\pgfsetdash{}{0pt}%
\pgfpathmoveto{\pgfqpoint{11.079132in}{3.814412in}}%
\pgfpathlineto{\pgfqpoint{11.079132in}{4.258529in}}%
\pgfusepath{stroke}%
\end{pgfscope}%
\begin{pgfscope}%
\definecolor{textcolor}{rgb}{0.150000,0.150000,0.150000}%
\pgfsetstrokecolor{textcolor}%
\pgfsetfillcolor{textcolor}%
\pgftext[x=11.079132in,y=3.717190in,,top]{\color{textcolor}\rmfamily\fontsize{14.000000}{16.800000}\selectfont 2015}%
\end{pgfscope}%
\begin{pgfscope}%
\pgfpathrectangle{\pgfqpoint{8.656250in}{3.814412in}}{\pgfqpoint{4.843750in}{0.444118in}}%
\pgfusepath{clip}%
\pgfsetroundcap%
\pgfsetroundjoin%
\pgfsetlinewidth{0.803000pt}%
\definecolor{currentstroke}{rgb}{1.000000,1.000000,1.000000}%
\pgfsetstrokecolor{currentstroke}%
\pgfsetdash{}{0pt}%
\pgfpathmoveto{\pgfqpoint{11.814040in}{3.814412in}}%
\pgfpathlineto{\pgfqpoint{11.814040in}{4.258529in}}%
\pgfusepath{stroke}%
\end{pgfscope}%
\begin{pgfscope}%
\definecolor{textcolor}{rgb}{0.150000,0.150000,0.150000}%
\pgfsetstrokecolor{textcolor}%
\pgfsetfillcolor{textcolor}%
\pgftext[x=11.814040in,y=3.717190in,,top]{\color{textcolor}\rmfamily\fontsize{14.000000}{16.800000}\selectfont 2016}%
\end{pgfscope}%
\begin{pgfscope}%
\pgfpathrectangle{\pgfqpoint{8.656250in}{3.814412in}}{\pgfqpoint{4.843750in}{0.444118in}}%
\pgfusepath{clip}%
\pgfsetroundcap%
\pgfsetroundjoin%
\pgfsetlinewidth{0.803000pt}%
\definecolor{currentstroke}{rgb}{1.000000,1.000000,1.000000}%
\pgfsetstrokecolor{currentstroke}%
\pgfsetdash{}{0pt}%
\pgfpathmoveto{\pgfqpoint{12.550962in}{3.814412in}}%
\pgfpathlineto{\pgfqpoint{12.550962in}{4.258529in}}%
\pgfusepath{stroke}%
\end{pgfscope}%
\begin{pgfscope}%
\definecolor{textcolor}{rgb}{0.150000,0.150000,0.150000}%
\pgfsetstrokecolor{textcolor}%
\pgfsetfillcolor{textcolor}%
\pgftext[x=12.550962in,y=3.717190in,,top]{\color{textcolor}\rmfamily\fontsize{14.000000}{16.800000}\selectfont 2017}%
\end{pgfscope}%
\begin{pgfscope}%
\pgfpathrectangle{\pgfqpoint{8.656250in}{3.814412in}}{\pgfqpoint{4.843750in}{0.444118in}}%
\pgfusepath{clip}%
\pgfsetroundcap%
\pgfsetroundjoin%
\pgfsetlinewidth{0.803000pt}%
\definecolor{currentstroke}{rgb}{1.000000,1.000000,1.000000}%
\pgfsetstrokecolor{currentstroke}%
\pgfsetdash{}{0pt}%
\pgfpathmoveto{\pgfqpoint{13.285870in}{3.814412in}}%
\pgfpathlineto{\pgfqpoint{13.285870in}{4.258529in}}%
\pgfusepath{stroke}%
\end{pgfscope}%
\begin{pgfscope}%
\definecolor{textcolor}{rgb}{0.150000,0.150000,0.150000}%
\pgfsetstrokecolor{textcolor}%
\pgfsetfillcolor{textcolor}%
\pgftext[x=13.285870in,y=3.717190in,,top]{\color{textcolor}\rmfamily\fontsize{14.000000}{16.800000}\selectfont 2018}%
\end{pgfscope}%
\begin{pgfscope}%
\pgfpathrectangle{\pgfqpoint{8.656250in}{3.814412in}}{\pgfqpoint{4.843750in}{0.444118in}}%
\pgfusepath{clip}%
\pgfsetroundcap%
\pgfsetroundjoin%
\pgfsetlinewidth{0.803000pt}%
\definecolor{currentstroke}{rgb}{1.000000,1.000000,1.000000}%
\pgfsetstrokecolor{currentstroke}%
\pgfsetdash{}{0pt}%
\pgfpathmoveto{\pgfqpoint{8.656250in}{3.890041in}}%
\pgfpathlineto{\pgfqpoint{13.500000in}{3.890041in}}%
\pgfusepath{stroke}%
\end{pgfscope}%
\begin{pgfscope}%
\definecolor{textcolor}{rgb}{0.150000,0.150000,0.150000}%
\pgfsetstrokecolor{textcolor}%
\pgfsetfillcolor{textcolor}%
\pgftext[x=8.311605in,y=3.816175in,left,base]{\color{textcolor}\rmfamily\fontsize{14.000000}{16.800000}\selectfont 20}%
\end{pgfscope}%
\begin{pgfscope}%
\pgfpathrectangle{\pgfqpoint{8.656250in}{3.814412in}}{\pgfqpoint{4.843750in}{0.444118in}}%
\pgfusepath{clip}%
\pgfsetroundcap%
\pgfsetroundjoin%
\pgfsetlinewidth{0.803000pt}%
\definecolor{currentstroke}{rgb}{1.000000,1.000000,1.000000}%
\pgfsetstrokecolor{currentstroke}%
\pgfsetdash{}{0pt}%
\pgfpathmoveto{\pgfqpoint{8.656250in}{4.159833in}}%
\pgfpathlineto{\pgfqpoint{13.500000in}{4.159833in}}%
\pgfusepath{stroke}%
\end{pgfscope}%
\begin{pgfscope}%
\definecolor{textcolor}{rgb}{0.150000,0.150000,0.150000}%
\pgfsetstrokecolor{textcolor}%
\pgfsetfillcolor{textcolor}%
\pgftext[x=8.311605in,y=4.085967in,left,base]{\color{textcolor}\rmfamily\fontsize{14.000000}{16.800000}\selectfont 40}%
\end{pgfscope}%
\begin{pgfscope}%
\pgfpathrectangle{\pgfqpoint{8.656250in}{3.814412in}}{\pgfqpoint{4.843750in}{0.444118in}}%
\pgfusepath{clip}%
\pgfsetroundcap%
\pgfsetroundjoin%
\pgfsetlinewidth{1.505625pt}%
\definecolor{currentstroke}{rgb}{0.121569,0.466667,0.705882}%
\pgfsetstrokecolor{currentstroke}%
\pgfsetdash{}{0pt}%
\pgfpathmoveto{\pgfqpoint{8.876420in}{3.882757in}}%
\pgfpathlineto{\pgfqpoint{8.878434in}{3.888962in}}%
\pgfpathlineto{\pgfqpoint{8.880447in}{3.892065in}}%
\pgfpathlineto{\pgfqpoint{8.882461in}{3.890446in}}%
\pgfpathlineto{\pgfqpoint{8.890515in}{3.894088in}}%
\pgfpathlineto{\pgfqpoint{8.892528in}{3.896246in}}%
\pgfpathlineto{\pgfqpoint{8.894541in}{3.895707in}}%
\pgfpathlineto{\pgfqpoint{8.896555in}{3.889232in}}%
\pgfpathlineto{\pgfqpoint{8.904609in}{3.888153in}}%
\pgfpathlineto{\pgfqpoint{8.906622in}{3.891930in}}%
\pgfpathlineto{\pgfqpoint{8.908636in}{3.894493in}}%
\pgfpathlineto{\pgfqpoint{8.910649in}{3.902452in}}%
\pgfpathlineto{\pgfqpoint{8.916689in}{3.905959in}}%
\pgfpathlineto{\pgfqpoint{8.918703in}{3.907982in}}%
\pgfpathlineto{\pgfqpoint{8.920716in}{3.907982in}}%
\pgfpathlineto{\pgfqpoint{8.922730in}{3.906498in}}%
\pgfpathlineto{\pgfqpoint{8.924743in}{3.906229in}}%
\pgfpathlineto{\pgfqpoint{8.930784in}{3.906364in}}%
\pgfpathlineto{\pgfqpoint{8.932797in}{3.902856in}}%
\pgfpathlineto{\pgfqpoint{8.934810in}{3.904340in}}%
\pgfpathlineto{\pgfqpoint{8.936824in}{3.903666in}}%
\pgfpathlineto{\pgfqpoint{8.938837in}{3.908657in}}%
\pgfpathlineto{\pgfqpoint{8.944878in}{3.908387in}}%
\pgfpathlineto{\pgfqpoint{8.946891in}{3.907578in}}%
\pgfpathlineto{\pgfqpoint{8.948905in}{3.909871in}}%
\pgfpathlineto{\pgfqpoint{8.950918in}{3.909871in}}%
\pgfpathlineto{\pgfqpoint{8.952931in}{3.908252in}}%
\pgfpathlineto{\pgfqpoint{8.958972in}{3.908252in}}%
\pgfpathlineto{\pgfqpoint{8.960985in}{3.909061in}}%
\pgfpathlineto{\pgfqpoint{8.962999in}{3.906903in}}%
\pgfpathlineto{\pgfqpoint{8.965012in}{3.909601in}}%
\pgfpathlineto{\pgfqpoint{8.967026in}{3.915402in}}%
\pgfpathlineto{\pgfqpoint{8.975079in}{3.913108in}}%
\pgfpathlineto{\pgfqpoint{8.977093in}{3.908522in}}%
\pgfpathlineto{\pgfqpoint{8.979106in}{3.907712in}}%
\pgfpathlineto{\pgfqpoint{8.987160in}{3.910276in}}%
\pgfpathlineto{\pgfqpoint{8.989173in}{3.914053in}}%
\pgfpathlineto{\pgfqpoint{8.991187in}{3.910141in}}%
\pgfpathlineto{\pgfqpoint{8.993200in}{3.909871in}}%
\pgfpathlineto{\pgfqpoint{8.995214in}{3.910545in}}%
\pgfpathlineto{\pgfqpoint{9.001254in}{3.906498in}}%
\pgfpathlineto{\pgfqpoint{9.003268in}{3.907173in}}%
\pgfpathlineto{\pgfqpoint{9.005281in}{3.910410in}}%
\pgfpathlineto{\pgfqpoint{9.007295in}{3.909736in}}%
\pgfpathlineto{\pgfqpoint{9.009308in}{3.912164in}}%
\pgfpathlineto{\pgfqpoint{9.015348in}{3.911355in}}%
\pgfpathlineto{\pgfqpoint{9.017362in}{3.916751in}}%
\pgfpathlineto{\pgfqpoint{9.019375in}{3.916346in}}%
\pgfpathlineto{\pgfqpoint{9.021389in}{3.919448in}}%
\pgfpathlineto{\pgfqpoint{9.029442in}{3.919448in}}%
\pgfpathlineto{\pgfqpoint{9.033469in}{3.919853in}}%
\pgfpathlineto{\pgfqpoint{9.035483in}{3.921067in}}%
\pgfpathlineto{\pgfqpoint{9.037496in}{3.920932in}}%
\pgfpathlineto{\pgfqpoint{9.043537in}{3.924305in}}%
\pgfpathlineto{\pgfqpoint{9.045550in}{3.924305in}}%
\pgfpathlineto{\pgfqpoint{9.047563in}{3.919988in}}%
\pgfpathlineto{\pgfqpoint{9.049577in}{3.923900in}}%
\pgfpathlineto{\pgfqpoint{9.051590in}{3.923495in}}%
\pgfpathlineto{\pgfqpoint{9.057631in}{3.926328in}}%
\pgfpathlineto{\pgfqpoint{9.059644in}{3.923360in}}%
\pgfpathlineto{\pgfqpoint{9.061658in}{3.921472in}}%
\pgfpathlineto{\pgfqpoint{9.063671in}{3.922956in}}%
\pgfpathlineto{\pgfqpoint{9.071725in}{3.919583in}}%
\pgfpathlineto{\pgfqpoint{9.073738in}{3.916346in}}%
\pgfpathlineto{\pgfqpoint{9.075752in}{3.920528in}}%
\pgfpathlineto{\pgfqpoint{9.077765in}{3.927407in}}%
\pgfpathlineto{\pgfqpoint{9.079779in}{3.923226in}}%
\pgfpathlineto{\pgfqpoint{9.087832in}{3.927272in}}%
\pgfpathlineto{\pgfqpoint{9.089846in}{3.921607in}}%
\pgfpathlineto{\pgfqpoint{9.091859in}{3.918909in}}%
\pgfpathlineto{\pgfqpoint{9.093873in}{3.917965in}}%
\pgfpathlineto{\pgfqpoint{9.099913in}{3.916346in}}%
\pgfpathlineto{\pgfqpoint{9.101927in}{3.914727in}}%
\pgfpathlineto{\pgfqpoint{9.103940in}{3.920662in}}%
\pgfpathlineto{\pgfqpoint{9.105953in}{3.924574in}}%
\pgfpathlineto{\pgfqpoint{9.107967in}{3.926328in}}%
\pgfpathlineto{\pgfqpoint{9.114007in}{3.926463in}}%
\pgfpathlineto{\pgfqpoint{9.116021in}{3.932398in}}%
\pgfpathlineto{\pgfqpoint{9.118034in}{3.934961in}}%
\pgfpathlineto{\pgfqpoint{9.120048in}{3.930510in}}%
\pgfpathlineto{\pgfqpoint{9.122061in}{3.923360in}}%
\pgfpathlineto{\pgfqpoint{9.128101in}{3.921742in}}%
\pgfpathlineto{\pgfqpoint{9.130115in}{3.917560in}}%
\pgfpathlineto{\pgfqpoint{9.132128in}{3.915536in}}%
\pgfpathlineto{\pgfqpoint{9.134142in}{3.916211in}}%
\pgfpathlineto{\pgfqpoint{9.136155in}{3.920393in}}%
\pgfpathlineto{\pgfqpoint{9.144209in}{3.912299in}}%
\pgfpathlineto{\pgfqpoint{9.148236in}{3.904745in}}%
\pgfpathlineto{\pgfqpoint{9.150249in}{3.903396in}}%
\pgfpathlineto{\pgfqpoint{9.156290in}{3.904340in}}%
\pgfpathlineto{\pgfqpoint{9.158303in}{3.902991in}}%
\pgfpathlineto{\pgfqpoint{9.160317in}{3.896651in}}%
\pgfpathlineto{\pgfqpoint{9.162330in}{3.898809in}}%
\pgfpathlineto{\pgfqpoint{9.172397in}{3.903666in}}%
\pgfpathlineto{\pgfqpoint{9.174411in}{3.904070in}}%
\pgfpathlineto{\pgfqpoint{9.176424in}{3.900968in}}%
\pgfpathlineto{\pgfqpoint{9.178438in}{3.893279in}}%
\pgfpathlineto{\pgfqpoint{9.184478in}{3.892199in}}%
\pgfpathlineto{\pgfqpoint{9.186491in}{3.896516in}}%
\pgfpathlineto{\pgfqpoint{9.188505in}{3.903396in}}%
\pgfpathlineto{\pgfqpoint{9.190518in}{3.902047in}}%
\pgfpathlineto{\pgfqpoint{9.192532in}{3.907173in}}%
\pgfpathlineto{\pgfqpoint{9.198572in}{3.902586in}}%
\pgfpathlineto{\pgfqpoint{9.200585in}{3.908387in}}%
\pgfpathlineto{\pgfqpoint{9.202599in}{3.908522in}}%
\pgfpathlineto{\pgfqpoint{9.206626in}{3.917290in}}%
\pgfpathlineto{\pgfqpoint{9.212666in}{3.918099in}}%
\pgfpathlineto{\pgfqpoint{9.216693in}{3.920528in}}%
\pgfpathlineto{\pgfqpoint{9.218706in}{3.910410in}}%
\pgfpathlineto{\pgfqpoint{9.220720in}{3.912839in}}%
\pgfpathlineto{\pgfqpoint{9.226760in}{3.903261in}}%
\pgfpathlineto{\pgfqpoint{9.228774in}{3.902721in}}%
\pgfpathlineto{\pgfqpoint{9.230787in}{3.905015in}}%
\pgfpathlineto{\pgfqpoint{9.232801in}{3.900833in}}%
\pgfpathlineto{\pgfqpoint{9.234814in}{3.909736in}}%
\pgfpathlineto{\pgfqpoint{9.240854in}{3.910006in}}%
\pgfpathlineto{\pgfqpoint{9.242868in}{3.912029in}}%
\pgfpathlineto{\pgfqpoint{9.246895in}{3.908657in}}%
\pgfpathlineto{\pgfqpoint{9.248908in}{3.904475in}}%
\pgfpathlineto{\pgfqpoint{9.254949in}{3.904475in}}%
\pgfpathlineto{\pgfqpoint{9.256962in}{3.897865in}}%
\pgfpathlineto{\pgfqpoint{9.258975in}{3.896111in}}%
\pgfpathlineto{\pgfqpoint{9.260989in}{3.888962in}}%
\pgfpathlineto{\pgfqpoint{9.263002in}{3.894493in}}%
\pgfpathlineto{\pgfqpoint{9.269043in}{3.893279in}}%
\pgfpathlineto{\pgfqpoint{9.271056in}{3.895977in}}%
\pgfpathlineto{\pgfqpoint{9.273070in}{3.905015in}}%
\pgfpathlineto{\pgfqpoint{9.275083in}{3.903261in}}%
\pgfpathlineto{\pgfqpoint{9.277096in}{3.897460in}}%
\pgfpathlineto{\pgfqpoint{9.283137in}{3.894628in}}%
\pgfpathlineto{\pgfqpoint{9.285150in}{3.891930in}}%
\pgfpathlineto{\pgfqpoint{9.287164in}{3.893279in}}%
\pgfpathlineto{\pgfqpoint{9.289177in}{3.897191in}}%
\pgfpathlineto{\pgfqpoint{9.291191in}{3.902856in}}%
\pgfpathlineto{\pgfqpoint{9.299244in}{3.899349in}}%
\pgfpathlineto{\pgfqpoint{9.301258in}{3.901912in}}%
\pgfpathlineto{\pgfqpoint{9.303271in}{3.901642in}}%
\pgfpathlineto{\pgfqpoint{9.305285in}{3.907712in}}%
\pgfpathlineto{\pgfqpoint{9.311325in}{3.908522in}}%
\pgfpathlineto{\pgfqpoint{9.313339in}{3.910680in}}%
\pgfpathlineto{\pgfqpoint{9.317365in}{3.912839in}}%
\pgfpathlineto{\pgfqpoint{9.319379in}{3.914727in}}%
\pgfpathlineto{\pgfqpoint{9.325419in}{3.912704in}}%
\pgfpathlineto{\pgfqpoint{9.329446in}{3.908117in}}%
\pgfpathlineto{\pgfqpoint{9.331460in}{3.911624in}}%
\pgfpathlineto{\pgfqpoint{9.333473in}{3.908792in}}%
\pgfpathlineto{\pgfqpoint{9.339513in}{3.907712in}}%
\pgfpathlineto{\pgfqpoint{9.341527in}{3.906364in}}%
\pgfpathlineto{\pgfqpoint{9.343540in}{3.902182in}}%
\pgfpathlineto{\pgfqpoint{9.345554in}{3.894628in}}%
\pgfpathlineto{\pgfqpoint{9.347567in}{3.893144in}}%
\pgfpathlineto{\pgfqpoint{9.353607in}{3.892469in}}%
\pgfpathlineto{\pgfqpoint{9.355621in}{3.894223in}}%
\pgfpathlineto{\pgfqpoint{9.359648in}{3.886129in}}%
\pgfpathlineto{\pgfqpoint{9.361661in}{3.892334in}}%
\pgfpathlineto{\pgfqpoint{9.369715in}{3.887883in}}%
\pgfpathlineto{\pgfqpoint{9.371728in}{3.887478in}}%
\pgfpathlineto{\pgfqpoint{9.373742in}{3.895302in}}%
\pgfpathlineto{\pgfqpoint{9.375755in}{3.885320in}}%
\pgfpathlineto{\pgfqpoint{9.381796in}{3.875068in}}%
\pgfpathlineto{\pgfqpoint{9.383809in}{3.876012in}}%
\pgfpathlineto{\pgfqpoint{9.385823in}{3.874393in}}%
\pgfpathlineto{\pgfqpoint{9.387836in}{3.876282in}}%
\pgfpathlineto{\pgfqpoint{9.389849in}{3.876282in}}%
\pgfpathlineto{\pgfqpoint{9.395890in}{3.875607in}}%
\pgfpathlineto{\pgfqpoint{9.397903in}{3.876282in}}%
\pgfpathlineto{\pgfqpoint{9.399917in}{3.873854in}}%
\pgfpathlineto{\pgfqpoint{9.401930in}{3.874258in}}%
\pgfpathlineto{\pgfqpoint{9.403944in}{3.873719in}}%
\pgfpathlineto{\pgfqpoint{9.409984in}{3.870077in}}%
\pgfpathlineto{\pgfqpoint{9.411997in}{3.867244in}}%
\pgfpathlineto{\pgfqpoint{9.414011in}{3.868458in}}%
\pgfpathlineto{\pgfqpoint{9.416024in}{3.873314in}}%
\pgfpathlineto{\pgfqpoint{9.418038in}{3.868593in}}%
\pgfpathlineto{\pgfqpoint{9.426092in}{3.870481in}}%
\pgfpathlineto{\pgfqpoint{9.428105in}{3.867379in}}%
\pgfpathlineto{\pgfqpoint{9.430118in}{3.866434in}}%
\pgfpathlineto{\pgfqpoint{9.432132in}{3.868728in}}%
\pgfpathlineto{\pgfqpoint{9.438172in}{3.866974in}}%
\pgfpathlineto{\pgfqpoint{9.440186in}{3.860229in}}%
\pgfpathlineto{\pgfqpoint{9.446226in}{3.855643in}}%
\pgfpathlineto{\pgfqpoint{9.452266in}{3.858341in}}%
\pgfpathlineto{\pgfqpoint{9.454280in}{3.865220in}}%
\pgfpathlineto{\pgfqpoint{9.456293in}{3.859015in}}%
\pgfpathlineto{\pgfqpoint{9.458307in}{3.857666in}}%
\pgfpathlineto{\pgfqpoint{9.460320in}{3.853349in}}%
\pgfpathlineto{\pgfqpoint{9.466360in}{3.855373in}}%
\pgfpathlineto{\pgfqpoint{9.468374in}{3.856857in}}%
\pgfpathlineto{\pgfqpoint{9.470387in}{3.855373in}}%
\pgfpathlineto{\pgfqpoint{9.474414in}{3.860769in}}%
\pgfpathlineto{\pgfqpoint{9.484482in}{3.857261in}}%
\pgfpathlineto{\pgfqpoint{9.486495in}{3.864141in}}%
\pgfpathlineto{\pgfqpoint{9.488508in}{3.861983in}}%
\pgfpathlineto{\pgfqpoint{9.494549in}{3.861983in}}%
\pgfpathlineto{\pgfqpoint{9.496562in}{3.860769in}}%
\pgfpathlineto{\pgfqpoint{9.498576in}{3.851731in}}%
\pgfpathlineto{\pgfqpoint{9.500589in}{3.850786in}}%
\pgfpathlineto{\pgfqpoint{9.508643in}{3.850247in}}%
\pgfpathlineto{\pgfqpoint{9.510656in}{3.844716in}}%
\pgfpathlineto{\pgfqpoint{9.512670in}{3.841209in}}%
\pgfpathlineto{\pgfqpoint{9.514683in}{3.842018in}}%
\pgfpathlineto{\pgfqpoint{9.516697in}{3.843772in}}%
\pgfpathlineto{\pgfqpoint{9.522737in}{3.844446in}}%
\pgfpathlineto{\pgfqpoint{9.524750in}{3.836218in}}%
\pgfpathlineto{\pgfqpoint{9.526764in}{3.834599in}}%
\pgfpathlineto{\pgfqpoint{9.530791in}{3.838511in}}%
\pgfpathlineto{\pgfqpoint{9.538845in}{3.840939in}}%
\pgfpathlineto{\pgfqpoint{9.540858in}{3.842693in}}%
\pgfpathlineto{\pgfqpoint{9.542871in}{3.836487in}}%
\pgfpathlineto{\pgfqpoint{9.544885in}{3.836892in}}%
\pgfpathlineto{\pgfqpoint{9.550925in}{3.836622in}}%
\pgfpathlineto{\pgfqpoint{9.552939in}{3.841344in}}%
\pgfpathlineto{\pgfqpoint{9.554952in}{3.839995in}}%
\pgfpathlineto{\pgfqpoint{9.556966in}{3.843367in}}%
\pgfpathlineto{\pgfqpoint{9.558979in}{3.843367in}}%
\pgfpathlineto{\pgfqpoint{9.565019in}{3.842558in}}%
\pgfpathlineto{\pgfqpoint{9.567033in}{3.848898in}}%
\pgfpathlineto{\pgfqpoint{9.569046in}{3.849033in}}%
\pgfpathlineto{\pgfqpoint{9.571060in}{3.847144in}}%
\pgfpathlineto{\pgfqpoint{9.573073in}{3.847549in}}%
\pgfpathlineto{\pgfqpoint{9.579114in}{3.847954in}}%
\pgfpathlineto{\pgfqpoint{9.581127in}{3.852270in}}%
\pgfpathlineto{\pgfqpoint{9.583140in}{3.853889in}}%
\pgfpathlineto{\pgfqpoint{9.585154in}{3.853080in}}%
\pgfpathlineto{\pgfqpoint{9.587167in}{3.850247in}}%
\pgfpathlineto{\pgfqpoint{9.593208in}{3.848763in}}%
\pgfpathlineto{\pgfqpoint{9.597235in}{3.848898in}}%
\pgfpathlineto{\pgfqpoint{9.599248in}{3.847279in}}%
\pgfpathlineto{\pgfqpoint{9.601261in}{3.844177in}}%
\pgfpathlineto{\pgfqpoint{9.607302in}{3.848493in}}%
\pgfpathlineto{\pgfqpoint{9.611329in}{3.856992in}}%
\pgfpathlineto{\pgfqpoint{9.613342in}{3.856317in}}%
\pgfpathlineto{\pgfqpoint{9.615356in}{3.854564in}}%
\pgfpathlineto{\pgfqpoint{9.621396in}{3.855508in}}%
\pgfpathlineto{\pgfqpoint{9.623409in}{3.853754in}}%
\pgfpathlineto{\pgfqpoint{9.627436in}{3.861578in}}%
\pgfpathlineto{\pgfqpoint{9.629450in}{3.863736in}}%
\pgfpathlineto{\pgfqpoint{9.635490in}{3.863736in}}%
\pgfpathlineto{\pgfqpoint{9.637504in}{3.862522in}}%
\pgfpathlineto{\pgfqpoint{9.639517in}{3.865085in}}%
\pgfpathlineto{\pgfqpoint{9.641530in}{3.871291in}}%
\pgfpathlineto{\pgfqpoint{9.643544in}{3.855508in}}%
\pgfpathlineto{\pgfqpoint{9.653611in}{3.853889in}}%
\pgfpathlineto{\pgfqpoint{9.655625in}{3.852135in}}%
\pgfpathlineto{\pgfqpoint{9.657638in}{3.852270in}}%
\pgfpathlineto{\pgfqpoint{9.663678in}{3.853349in}}%
\pgfpathlineto{\pgfqpoint{9.665692in}{3.855778in}}%
\pgfpathlineto{\pgfqpoint{9.667705in}{3.856857in}}%
\pgfpathlineto{\pgfqpoint{9.669719in}{3.853215in}}%
\pgfpathlineto{\pgfqpoint{9.671732in}{3.856722in}}%
\pgfpathlineto{\pgfqpoint{9.677772in}{3.854564in}}%
\pgfpathlineto{\pgfqpoint{9.679786in}{3.857261in}}%
\pgfpathlineto{\pgfqpoint{9.683813in}{3.853080in}}%
\pgfpathlineto{\pgfqpoint{9.685826in}{3.855238in}}%
\pgfpathlineto{\pgfqpoint{9.691867in}{3.855508in}}%
\pgfpathlineto{\pgfqpoint{9.693880in}{3.857396in}}%
\pgfpathlineto{\pgfqpoint{9.695893in}{3.858071in}}%
\pgfpathlineto{\pgfqpoint{9.697907in}{3.857801in}}%
\pgfpathlineto{\pgfqpoint{9.699920in}{3.856587in}}%
\pgfpathlineto{\pgfqpoint{9.707974in}{3.856182in}}%
\pgfpathlineto{\pgfqpoint{9.709988in}{3.852270in}}%
\pgfpathlineto{\pgfqpoint{9.712001in}{3.846874in}}%
\pgfpathlineto{\pgfqpoint{9.714015in}{3.848763in}}%
\pgfpathlineto{\pgfqpoint{9.720055in}{3.846605in}}%
\pgfpathlineto{\pgfqpoint{9.724082in}{3.854429in}}%
\pgfpathlineto{\pgfqpoint{9.726095in}{3.853889in}}%
\pgfpathlineto{\pgfqpoint{9.728109in}{3.855508in}}%
\pgfpathlineto{\pgfqpoint{9.734149in}{3.858206in}}%
\pgfpathlineto{\pgfqpoint{9.738176in}{3.863602in}}%
\pgfpathlineto{\pgfqpoint{9.740189in}{3.865220in}}%
\pgfpathlineto{\pgfqpoint{9.742203in}{3.861713in}}%
\pgfpathlineto{\pgfqpoint{9.748243in}{3.862927in}}%
\pgfpathlineto{\pgfqpoint{9.750257in}{3.862387in}}%
\pgfpathlineto{\pgfqpoint{9.754283in}{3.862522in}}%
\pgfpathlineto{\pgfqpoint{9.756297in}{3.859420in}}%
\pgfpathlineto{\pgfqpoint{9.762337in}{3.858206in}}%
\pgfpathlineto{\pgfqpoint{9.764351in}{3.856857in}}%
\pgfpathlineto{\pgfqpoint{9.766364in}{3.857261in}}%
\pgfpathlineto{\pgfqpoint{9.768378in}{3.855643in}}%
\pgfpathlineto{\pgfqpoint{9.770391in}{3.858880in}}%
\pgfpathlineto{\pgfqpoint{9.776431in}{3.856857in}}%
\pgfpathlineto{\pgfqpoint{9.778445in}{3.863871in}}%
\pgfpathlineto{\pgfqpoint{9.782472in}{3.864681in}}%
\pgfpathlineto{\pgfqpoint{9.790525in}{3.860094in}}%
\pgfpathlineto{\pgfqpoint{9.792539in}{3.860364in}}%
\pgfpathlineto{\pgfqpoint{9.794552in}{3.855778in}}%
\pgfpathlineto{\pgfqpoint{9.796566in}{3.856857in}}%
\pgfpathlineto{\pgfqpoint{9.798579in}{3.854564in}}%
\pgfpathlineto{\pgfqpoint{9.804620in}{3.856182in}}%
\pgfpathlineto{\pgfqpoint{9.808647in}{3.869267in}}%
\pgfpathlineto{\pgfqpoint{9.810660in}{3.864546in}}%
\pgfpathlineto{\pgfqpoint{9.812673in}{3.862792in}}%
\pgfpathlineto{\pgfqpoint{9.818714in}{3.859420in}}%
\pgfpathlineto{\pgfqpoint{9.820727in}{3.865490in}}%
\pgfpathlineto{\pgfqpoint{9.822741in}{3.865625in}}%
\pgfpathlineto{\pgfqpoint{9.824754in}{3.869132in}}%
\pgfpathlineto{\pgfqpoint{9.826768in}{3.871291in}}%
\pgfpathlineto{\pgfqpoint{9.832808in}{3.876282in}}%
\pgfpathlineto{\pgfqpoint{9.834821in}{3.881812in}}%
\pgfpathlineto{\pgfqpoint{9.836835in}{3.885050in}}%
\pgfpathlineto{\pgfqpoint{9.838848in}{3.881812in}}%
\pgfpathlineto{\pgfqpoint{9.840862in}{3.882082in}}%
\pgfpathlineto{\pgfqpoint{9.846902in}{3.886129in}}%
\pgfpathlineto{\pgfqpoint{9.848915in}{3.888287in}}%
\pgfpathlineto{\pgfqpoint{9.850929in}{3.888692in}}%
\pgfpathlineto{\pgfqpoint{9.854956in}{3.890851in}}%
\pgfpathlineto{\pgfqpoint{9.860996in}{3.890311in}}%
\pgfpathlineto{\pgfqpoint{9.863010in}{3.893009in}}%
\pgfpathlineto{\pgfqpoint{9.867036in}{3.895437in}}%
\pgfpathlineto{\pgfqpoint{9.869050in}{3.896921in}}%
\pgfpathlineto{\pgfqpoint{9.875090in}{3.892199in}}%
\pgfpathlineto{\pgfqpoint{9.877104in}{3.889502in}}%
\pgfpathlineto{\pgfqpoint{9.879117in}{3.893548in}}%
\pgfpathlineto{\pgfqpoint{9.881131in}{3.890716in}}%
\pgfpathlineto{\pgfqpoint{9.883144in}{3.891795in}}%
\pgfpathlineto{\pgfqpoint{9.889184in}{3.892199in}}%
\pgfpathlineto{\pgfqpoint{9.891198in}{3.893009in}}%
\pgfpathlineto{\pgfqpoint{9.893211in}{3.892065in}}%
\pgfpathlineto{\pgfqpoint{9.895225in}{3.891930in}}%
\pgfpathlineto{\pgfqpoint{9.897238in}{3.890446in}}%
\pgfpathlineto{\pgfqpoint{9.905292in}{3.892199in}}%
\pgfpathlineto{\pgfqpoint{9.907305in}{3.894358in}}%
\pgfpathlineto{\pgfqpoint{9.909319in}{3.893683in}}%
\pgfpathlineto{\pgfqpoint{9.911332in}{3.894493in}}%
\pgfpathlineto{\pgfqpoint{9.917373in}{3.905284in}}%
\pgfpathlineto{\pgfqpoint{9.919386in}{3.906633in}}%
\pgfpathlineto{\pgfqpoint{9.921400in}{3.899214in}}%
\pgfpathlineto{\pgfqpoint{9.925426in}{3.898000in}}%
\pgfpathlineto{\pgfqpoint{9.931467in}{3.902721in}}%
\pgfpathlineto{\pgfqpoint{9.935494in}{3.896516in}}%
\pgfpathlineto{\pgfqpoint{9.937507in}{3.902452in}}%
\pgfpathlineto{\pgfqpoint{9.939521in}{3.901777in}}%
\pgfpathlineto{\pgfqpoint{9.945561in}{3.903801in}}%
\pgfpathlineto{\pgfqpoint{9.947574in}{3.907982in}}%
\pgfpathlineto{\pgfqpoint{9.949588in}{3.902586in}}%
\pgfpathlineto{\pgfqpoint{9.951601in}{3.893414in}}%
\pgfpathlineto{\pgfqpoint{9.953615in}{3.893548in}}%
\pgfpathlineto{\pgfqpoint{9.959655in}{3.886534in}}%
\pgfpathlineto{\pgfqpoint{9.961669in}{3.889906in}}%
\pgfpathlineto{\pgfqpoint{9.963682in}{3.891390in}}%
\pgfpathlineto{\pgfqpoint{9.965695in}{3.891930in}}%
\pgfpathlineto{\pgfqpoint{9.967709in}{3.893953in}}%
\pgfpathlineto{\pgfqpoint{9.975763in}{3.888153in}}%
\pgfpathlineto{\pgfqpoint{9.977776in}{3.888557in}}%
\pgfpathlineto{\pgfqpoint{9.981803in}{3.892065in}}%
\pgfpathlineto{\pgfqpoint{9.987843in}{3.882217in}}%
\pgfpathlineto{\pgfqpoint{9.989857in}{3.881678in}}%
\pgfpathlineto{\pgfqpoint{9.991870in}{3.882892in}}%
\pgfpathlineto{\pgfqpoint{9.993884in}{3.891255in}}%
\pgfpathlineto{\pgfqpoint{9.995897in}{3.890176in}}%
\pgfpathlineto{\pgfqpoint{10.001937in}{3.890716in}}%
\pgfpathlineto{\pgfqpoint{10.003951in}{3.894088in}}%
\pgfpathlineto{\pgfqpoint{10.005964in}{3.893009in}}%
\pgfpathlineto{\pgfqpoint{10.007978in}{3.882757in}}%
\pgfpathlineto{\pgfqpoint{10.009991in}{3.880464in}}%
\pgfpathlineto{\pgfqpoint{10.016032in}{3.877496in}}%
\pgfpathlineto{\pgfqpoint{10.018045in}{3.877226in}}%
\pgfpathlineto{\pgfqpoint{10.024085in}{3.883027in}}%
\pgfpathlineto{\pgfqpoint{10.030126in}{3.882757in}}%
\pgfpathlineto{\pgfqpoint{10.032139in}{3.884376in}}%
\pgfpathlineto{\pgfqpoint{10.034153in}{3.883836in}}%
\pgfpathlineto{\pgfqpoint{10.036166in}{3.882352in}}%
\pgfpathlineto{\pgfqpoint{10.038180in}{3.882487in}}%
\pgfpathlineto{\pgfqpoint{10.044220in}{3.881678in}}%
\pgfpathlineto{\pgfqpoint{10.048247in}{3.879115in}}%
\pgfpathlineto{\pgfqpoint{10.050260in}{3.876282in}}%
\pgfpathlineto{\pgfqpoint{10.058314in}{3.878440in}}%
\pgfpathlineto{\pgfqpoint{10.060327in}{3.877091in}}%
\pgfpathlineto{\pgfqpoint{10.062341in}{3.877631in}}%
\pgfpathlineto{\pgfqpoint{10.064354in}{3.871560in}}%
\pgfpathlineto{\pgfqpoint{10.066368in}{3.870077in}}%
\pgfpathlineto{\pgfqpoint{10.072408in}{3.874393in}}%
\pgfpathlineto{\pgfqpoint{10.074422in}{3.877091in}}%
\pgfpathlineto{\pgfqpoint{10.076435in}{3.873044in}}%
\pgfpathlineto{\pgfqpoint{10.078448in}{3.874123in}}%
\pgfpathlineto{\pgfqpoint{10.080462in}{3.876147in}}%
\pgfpathlineto{\pgfqpoint{10.088516in}{3.873314in}}%
\pgfpathlineto{\pgfqpoint{10.090529in}{3.874528in}}%
\pgfpathlineto{\pgfqpoint{10.092543in}{3.871830in}}%
\pgfpathlineto{\pgfqpoint{10.094556in}{3.870886in}}%
\pgfpathlineto{\pgfqpoint{10.102610in}{3.871965in}}%
\pgfpathlineto{\pgfqpoint{10.104623in}{3.878440in}}%
\pgfpathlineto{\pgfqpoint{10.106637in}{3.878035in}}%
\pgfpathlineto{\pgfqpoint{10.116704in}{3.882352in}}%
\pgfpathlineto{\pgfqpoint{10.120731in}{3.878305in}}%
\pgfpathlineto{\pgfqpoint{10.122744in}{3.887613in}}%
\pgfpathlineto{\pgfqpoint{10.128785in}{3.887073in}}%
\pgfpathlineto{\pgfqpoint{10.130798in}{3.890985in}}%
\pgfpathlineto{\pgfqpoint{10.132812in}{3.892874in}}%
\pgfpathlineto{\pgfqpoint{10.134825in}{3.893009in}}%
\pgfpathlineto{\pgfqpoint{10.136838in}{3.891390in}}%
\pgfpathlineto{\pgfqpoint{10.142879in}{3.889636in}}%
\pgfpathlineto{\pgfqpoint{10.144892in}{3.890581in}}%
\pgfpathlineto{\pgfqpoint{10.146906in}{3.890581in}}%
\pgfpathlineto{\pgfqpoint{10.148919in}{3.887208in}}%
\pgfpathlineto{\pgfqpoint{10.150933in}{3.882352in}}%
\pgfpathlineto{\pgfqpoint{10.156973in}{3.881678in}}%
\pgfpathlineto{\pgfqpoint{10.158986in}{3.880598in}}%
\pgfpathlineto{\pgfqpoint{10.161000in}{3.881273in}}%
\pgfpathlineto{\pgfqpoint{10.163013in}{3.878035in}}%
\pgfpathlineto{\pgfqpoint{10.165027in}{3.880464in}}%
\pgfpathlineto{\pgfqpoint{10.171067in}{3.880598in}}%
\pgfpathlineto{\pgfqpoint{10.173080in}{3.876686in}}%
\pgfpathlineto{\pgfqpoint{10.175094in}{3.877901in}}%
\pgfpathlineto{\pgfqpoint{10.177107in}{3.883701in}}%
\pgfpathlineto{\pgfqpoint{10.179121in}{3.885590in}}%
\pgfpathlineto{\pgfqpoint{10.185161in}{3.887748in}}%
\pgfpathlineto{\pgfqpoint{10.187175in}{3.887073in}}%
\pgfpathlineto{\pgfqpoint{10.191201in}{3.893009in}}%
\pgfpathlineto{\pgfqpoint{10.193215in}{3.892604in}}%
\pgfpathlineto{\pgfqpoint{10.199255in}{3.895572in}}%
\pgfpathlineto{\pgfqpoint{10.201269in}{3.894762in}}%
\pgfpathlineto{\pgfqpoint{10.203282in}{3.890985in}}%
\pgfpathlineto{\pgfqpoint{10.205296in}{3.891525in}}%
\pgfpathlineto{\pgfqpoint{10.207309in}{3.896786in}}%
\pgfpathlineto{\pgfqpoint{10.213349in}{3.898135in}}%
\pgfpathlineto{\pgfqpoint{10.215363in}{3.899889in}}%
\pgfpathlineto{\pgfqpoint{10.219390in}{3.899349in}}%
\pgfpathlineto{\pgfqpoint{10.221403in}{3.897730in}}%
\pgfpathlineto{\pgfqpoint{10.229457in}{3.896921in}}%
\pgfpathlineto{\pgfqpoint{10.231470in}{3.899484in}}%
\pgfpathlineto{\pgfqpoint{10.233484in}{3.897191in}}%
\pgfpathlineto{\pgfqpoint{10.241538in}{3.898540in}}%
\pgfpathlineto{\pgfqpoint{10.243551in}{3.901507in}}%
\pgfpathlineto{\pgfqpoint{10.245565in}{3.903396in}}%
\pgfpathlineto{\pgfqpoint{10.247578in}{3.900968in}}%
\pgfpathlineto{\pgfqpoint{10.249591in}{3.902586in}}%
\pgfpathlineto{\pgfqpoint{10.255632in}{3.903396in}}%
\pgfpathlineto{\pgfqpoint{10.257645in}{3.904610in}}%
\pgfpathlineto{\pgfqpoint{10.259659in}{3.902991in}}%
\pgfpathlineto{\pgfqpoint{10.261672in}{3.910680in}}%
\pgfpathlineto{\pgfqpoint{10.263686in}{3.895032in}}%
\pgfpathlineto{\pgfqpoint{10.269726in}{3.893683in}}%
\pgfpathlineto{\pgfqpoint{10.271739in}{3.892469in}}%
\pgfpathlineto{\pgfqpoint{10.273753in}{3.895437in}}%
\pgfpathlineto{\pgfqpoint{10.277780in}{3.894762in}}%
\pgfpathlineto{\pgfqpoint{10.283820in}{3.893144in}}%
\pgfpathlineto{\pgfqpoint{10.285834in}{3.891390in}}%
\pgfpathlineto{\pgfqpoint{10.287847in}{3.893548in}}%
\pgfpathlineto{\pgfqpoint{10.291874in}{3.905959in}}%
\pgfpathlineto{\pgfqpoint{10.297914in}{3.907308in}}%
\pgfpathlineto{\pgfqpoint{10.299928in}{3.905959in}}%
\pgfpathlineto{\pgfqpoint{10.301941in}{3.901372in}}%
\pgfpathlineto{\pgfqpoint{10.303955in}{3.901912in}}%
\pgfpathlineto{\pgfqpoint{10.305968in}{3.899889in}}%
\pgfpathlineto{\pgfqpoint{10.312008in}{3.901777in}}%
\pgfpathlineto{\pgfqpoint{10.314022in}{3.904205in}}%
\pgfpathlineto{\pgfqpoint{10.316035in}{3.909736in}}%
\pgfpathlineto{\pgfqpoint{10.318049in}{3.909601in}}%
\pgfpathlineto{\pgfqpoint{10.320062in}{3.908792in}}%
\pgfpathlineto{\pgfqpoint{10.328116in}{3.912973in}}%
\pgfpathlineto{\pgfqpoint{10.332143in}{3.916076in}}%
\pgfpathlineto{\pgfqpoint{10.334156in}{3.914997in}}%
\pgfpathlineto{\pgfqpoint{10.342210in}{3.919044in}}%
\pgfpathlineto{\pgfqpoint{10.346237in}{3.917155in}}%
\pgfpathlineto{\pgfqpoint{10.348250in}{3.917020in}}%
\pgfpathlineto{\pgfqpoint{10.354291in}{3.913378in}}%
\pgfpathlineto{\pgfqpoint{10.356304in}{3.914862in}}%
\pgfpathlineto{\pgfqpoint{10.360331in}{3.911624in}}%
\pgfpathlineto{\pgfqpoint{10.362345in}{3.914187in}}%
\pgfpathlineto{\pgfqpoint{10.368385in}{3.913783in}}%
\pgfpathlineto{\pgfqpoint{10.370398in}{3.925384in}}%
\pgfpathlineto{\pgfqpoint{10.372412in}{3.927272in}}%
\pgfpathlineto{\pgfqpoint{10.374425in}{3.925789in}}%
\pgfpathlineto{\pgfqpoint{10.376439in}{3.917830in}}%
\pgfpathlineto{\pgfqpoint{10.384492in}{3.914862in}}%
\pgfpathlineto{\pgfqpoint{10.386506in}{3.911624in}}%
\pgfpathlineto{\pgfqpoint{10.388519in}{3.909601in}}%
\pgfpathlineto{\pgfqpoint{10.390533in}{3.905824in}}%
\pgfpathlineto{\pgfqpoint{10.396573in}{3.904880in}}%
\pgfpathlineto{\pgfqpoint{10.398587in}{3.906903in}}%
\pgfpathlineto{\pgfqpoint{10.400600in}{3.904340in}}%
\pgfpathlineto{\pgfqpoint{10.402613in}{3.905015in}}%
\pgfpathlineto{\pgfqpoint{10.410667in}{3.895977in}}%
\pgfpathlineto{\pgfqpoint{10.414694in}{3.893548in}}%
\pgfpathlineto{\pgfqpoint{10.416708in}{3.899079in}}%
\pgfpathlineto{\pgfqpoint{10.418721in}{3.901642in}}%
\pgfpathlineto{\pgfqpoint{10.424761in}{3.902586in}}%
\pgfpathlineto{\pgfqpoint{10.426775in}{3.904610in}}%
\pgfpathlineto{\pgfqpoint{10.428788in}{3.905554in}}%
\pgfpathlineto{\pgfqpoint{10.430802in}{3.907308in}}%
\pgfpathlineto{\pgfqpoint{10.432815in}{3.907982in}}%
\pgfpathlineto{\pgfqpoint{10.440869in}{3.907982in}}%
\pgfpathlineto{\pgfqpoint{10.442882in}{3.905015in}}%
\pgfpathlineto{\pgfqpoint{10.444896in}{3.907712in}}%
\pgfpathlineto{\pgfqpoint{10.446909in}{3.904070in}}%
\pgfpathlineto{\pgfqpoint{10.452950in}{3.906498in}}%
\pgfpathlineto{\pgfqpoint{10.454963in}{3.906364in}}%
\pgfpathlineto{\pgfqpoint{10.456977in}{3.908522in}}%
\pgfpathlineto{\pgfqpoint{10.461003in}{3.907982in}}%
\pgfpathlineto{\pgfqpoint{10.467044in}{3.905015in}}%
\pgfpathlineto{\pgfqpoint{10.469057in}{3.906229in}}%
\pgfpathlineto{\pgfqpoint{10.471071in}{3.905015in}}%
\pgfpathlineto{\pgfqpoint{10.473084in}{3.906498in}}%
\pgfpathlineto{\pgfqpoint{10.475098in}{3.906633in}}%
\pgfpathlineto{\pgfqpoint{10.481138in}{3.908927in}}%
\pgfpathlineto{\pgfqpoint{10.483151in}{3.907712in}}%
\pgfpathlineto{\pgfqpoint{10.485165in}{3.907982in}}%
\pgfpathlineto{\pgfqpoint{10.487178in}{3.905824in}}%
\pgfpathlineto{\pgfqpoint{10.489192in}{3.905015in}}%
\pgfpathlineto{\pgfqpoint{10.497245in}{3.908657in}}%
\pgfpathlineto{\pgfqpoint{10.499259in}{3.911085in}}%
\pgfpathlineto{\pgfqpoint{10.501272in}{3.915806in}}%
\pgfpathlineto{\pgfqpoint{10.503286in}{3.912839in}}%
\pgfpathlineto{\pgfqpoint{10.509326in}{3.912164in}}%
\pgfpathlineto{\pgfqpoint{10.511340in}{3.916211in}}%
\pgfpathlineto{\pgfqpoint{10.515366in}{3.914457in}}%
\pgfpathlineto{\pgfqpoint{10.517380in}{3.917965in}}%
\pgfpathlineto{\pgfqpoint{10.523420in}{3.920258in}}%
\pgfpathlineto{\pgfqpoint{10.525434in}{3.922281in}}%
\pgfpathlineto{\pgfqpoint{10.527447in}{3.921202in}}%
\pgfpathlineto{\pgfqpoint{10.529461in}{3.927137in}}%
\pgfpathlineto{\pgfqpoint{10.531474in}{3.924305in}}%
\pgfpathlineto{\pgfqpoint{10.537514in}{3.928082in}}%
\pgfpathlineto{\pgfqpoint{10.539528in}{3.932938in}}%
\pgfpathlineto{\pgfqpoint{10.541541in}{3.933747in}}%
\pgfpathlineto{\pgfqpoint{10.543555in}{3.927407in}}%
\pgfpathlineto{\pgfqpoint{10.545568in}{3.924574in}}%
\pgfpathlineto{\pgfqpoint{10.551609in}{3.928891in}}%
\pgfpathlineto{\pgfqpoint{10.555635in}{3.933208in}}%
\pgfpathlineto{\pgfqpoint{10.557649in}{3.934557in}}%
\pgfpathlineto{\pgfqpoint{10.565703in}{3.933478in}}%
\pgfpathlineto{\pgfqpoint{10.569730in}{3.931184in}}%
\pgfpathlineto{\pgfqpoint{10.571743in}{3.931184in}}%
\pgfpathlineto{\pgfqpoint{10.573756in}{3.925384in}}%
\pgfpathlineto{\pgfqpoint{10.579797in}{3.926193in}}%
\pgfpathlineto{\pgfqpoint{10.583824in}{3.930375in}}%
\pgfpathlineto{\pgfqpoint{10.585837in}{3.927677in}}%
\pgfpathlineto{\pgfqpoint{10.587851in}{3.927137in}}%
\pgfpathlineto{\pgfqpoint{10.595904in}{3.927407in}}%
\pgfpathlineto{\pgfqpoint{10.597918in}{3.929296in}}%
\pgfpathlineto{\pgfqpoint{10.601945in}{3.928486in}}%
\pgfpathlineto{\pgfqpoint{10.607985in}{3.929296in}}%
\pgfpathlineto{\pgfqpoint{10.609999in}{3.930240in}}%
\pgfpathlineto{\pgfqpoint{10.612012in}{3.928891in}}%
\pgfpathlineto{\pgfqpoint{10.614025in}{3.925114in}}%
\pgfpathlineto{\pgfqpoint{10.616039in}{3.922956in}}%
\pgfpathlineto{\pgfqpoint{10.622079in}{3.925519in}}%
\pgfpathlineto{\pgfqpoint{10.624093in}{3.925519in}}%
\pgfpathlineto{\pgfqpoint{10.626106in}{3.927407in}}%
\pgfpathlineto{\pgfqpoint{10.628120in}{3.926733in}}%
\pgfpathlineto{\pgfqpoint{10.630133in}{3.928352in}}%
\pgfpathlineto{\pgfqpoint{10.638187in}{3.933343in}}%
\pgfpathlineto{\pgfqpoint{10.640200in}{3.935366in}}%
\pgfpathlineto{\pgfqpoint{10.642214in}{3.936310in}}%
\pgfpathlineto{\pgfqpoint{10.644227in}{3.940492in}}%
\pgfpathlineto{\pgfqpoint{10.650267in}{3.939818in}}%
\pgfpathlineto{\pgfqpoint{10.652281in}{3.944539in}}%
\pgfpathlineto{\pgfqpoint{10.654294in}{3.943730in}}%
\pgfpathlineto{\pgfqpoint{10.656308in}{3.944539in}}%
\pgfpathlineto{\pgfqpoint{10.658321in}{3.950474in}}%
\pgfpathlineto{\pgfqpoint{10.664362in}{3.947372in}}%
\pgfpathlineto{\pgfqpoint{10.666375in}{3.951284in}}%
\pgfpathlineto{\pgfqpoint{10.668388in}{3.947642in}}%
\pgfpathlineto{\pgfqpoint{10.670402in}{3.948046in}}%
\pgfpathlineto{\pgfqpoint{10.672415in}{3.970439in}}%
\pgfpathlineto{\pgfqpoint{10.678456in}{3.972058in}}%
\pgfpathlineto{\pgfqpoint{10.682483in}{3.971114in}}%
\pgfpathlineto{\pgfqpoint{10.686510in}{3.974216in}}%
\pgfpathlineto{\pgfqpoint{10.692550in}{3.974621in}}%
\pgfpathlineto{\pgfqpoint{10.694563in}{3.977723in}}%
\pgfpathlineto{\pgfqpoint{10.696577in}{3.982175in}}%
\pgfpathlineto{\pgfqpoint{10.698590in}{3.981096in}}%
\pgfpathlineto{\pgfqpoint{10.700604in}{3.982849in}}%
\pgfpathlineto{\pgfqpoint{10.706644in}{3.982445in}}%
\pgfpathlineto{\pgfqpoint{10.708657in}{3.983389in}}%
\pgfpathlineto{\pgfqpoint{10.710671in}{3.983389in}}%
\pgfpathlineto{\pgfqpoint{10.712684in}{3.985278in}}%
\pgfpathlineto{\pgfqpoint{10.720738in}{3.983929in}}%
\pgfpathlineto{\pgfqpoint{10.722752in}{3.981231in}}%
\pgfpathlineto{\pgfqpoint{10.724765in}{3.982310in}}%
\pgfpathlineto{\pgfqpoint{10.726778in}{3.986627in}}%
\pgfpathlineto{\pgfqpoint{10.728792in}{3.986492in}}%
\pgfpathlineto{\pgfqpoint{10.734832in}{3.989324in}}%
\pgfpathlineto{\pgfqpoint{10.736846in}{3.991888in}}%
\pgfpathlineto{\pgfqpoint{10.738859in}{4.026421in}}%
\pgfpathlineto{\pgfqpoint{10.740873in}{4.015224in}}%
\pgfpathlineto{\pgfqpoint{10.742886in}{4.015224in}}%
\pgfpathlineto{\pgfqpoint{10.748926in}{4.019541in}}%
\pgfpathlineto{\pgfqpoint{10.750940in}{4.028040in}}%
\pgfpathlineto{\pgfqpoint{10.754967in}{4.021699in}}%
\pgfpathlineto{\pgfqpoint{10.763021in}{4.021430in}}%
\pgfpathlineto{\pgfqpoint{10.765034in}{4.021025in}}%
\pgfpathlineto{\pgfqpoint{10.767047in}{4.022914in}}%
\pgfpathlineto{\pgfqpoint{10.769061in}{4.017518in}}%
\pgfpathlineto{\pgfqpoint{10.771074in}{4.015764in}}%
\pgfpathlineto{\pgfqpoint{10.777115in}{4.019406in}}%
\pgfpathlineto{\pgfqpoint{10.779128in}{4.007535in}}%
\pgfpathlineto{\pgfqpoint{10.781142in}{4.007940in}}%
\pgfpathlineto{\pgfqpoint{10.783155in}{4.005917in}}%
\pgfpathlineto{\pgfqpoint{10.785168in}{4.004972in}}%
\pgfpathlineto{\pgfqpoint{10.793222in}{4.011178in}}%
\pgfpathlineto{\pgfqpoint{10.795236in}{4.022644in}}%
\pgfpathlineto{\pgfqpoint{10.797249in}{4.020755in}}%
\pgfpathlineto{\pgfqpoint{10.799263in}{4.023453in}}%
\pgfpathlineto{\pgfqpoint{10.805303in}{4.026286in}}%
\pgfpathlineto{\pgfqpoint{10.807316in}{4.025477in}}%
\pgfpathlineto{\pgfqpoint{10.809330in}{4.027365in}}%
\pgfpathlineto{\pgfqpoint{10.811343in}{4.035054in}}%
\pgfpathlineto{\pgfqpoint{10.813357in}{4.032491in}}%
\pgfpathlineto{\pgfqpoint{10.821410in}{4.030872in}}%
\pgfpathlineto{\pgfqpoint{10.823424in}{4.030738in}}%
\pgfpathlineto{\pgfqpoint{10.825437in}{4.029119in}}%
\pgfpathlineto{\pgfqpoint{10.827451in}{4.032356in}}%
\pgfpathlineto{\pgfqpoint{10.835505in}{4.028174in}}%
\pgfpathlineto{\pgfqpoint{10.837518in}{4.028174in}}%
\pgfpathlineto{\pgfqpoint{10.839532in}{4.032221in}}%
\pgfpathlineto{\pgfqpoint{10.847585in}{4.037078in}}%
\pgfpathlineto{\pgfqpoint{10.849599in}{4.032221in}}%
\pgfpathlineto{\pgfqpoint{10.851612in}{4.033435in}}%
\pgfpathlineto{\pgfqpoint{10.853626in}{4.033435in}}%
\pgfpathlineto{\pgfqpoint{10.855639in}{4.028714in}}%
\pgfpathlineto{\pgfqpoint{10.861679in}{4.027770in}}%
\pgfpathlineto{\pgfqpoint{10.863693in}{4.032356in}}%
\pgfpathlineto{\pgfqpoint{10.865706in}{4.033031in}}%
\pgfpathlineto{\pgfqpoint{10.867720in}{4.035189in}}%
\pgfpathlineto{\pgfqpoint{10.869733in}{4.031142in}}%
\pgfpathlineto{\pgfqpoint{10.875774in}{4.029793in}}%
\pgfpathlineto{\pgfqpoint{10.877787in}{4.026421in}}%
\pgfpathlineto{\pgfqpoint{10.879800in}{4.030333in}}%
\pgfpathlineto{\pgfqpoint{10.881814in}{4.023048in}}%
\pgfpathlineto{\pgfqpoint{10.883827in}{4.024532in}}%
\pgfpathlineto{\pgfqpoint{10.889868in}{4.032086in}}%
\pgfpathlineto{\pgfqpoint{10.891881in}{4.031142in}}%
\pgfpathlineto{\pgfqpoint{10.893895in}{4.021295in}}%
\pgfpathlineto{\pgfqpoint{10.895908in}{4.015764in}}%
\pgfpathlineto{\pgfqpoint{10.897921in}{4.021834in}}%
\pgfpathlineto{\pgfqpoint{10.903962in}{4.022779in}}%
\pgfpathlineto{\pgfqpoint{10.905975in}{4.015224in}}%
\pgfpathlineto{\pgfqpoint{10.907989in}{4.024667in}}%
\pgfpathlineto{\pgfqpoint{10.910002in}{4.016978in}}%
\pgfpathlineto{\pgfqpoint{10.912016in}{3.996744in}}%
\pgfpathlineto{\pgfqpoint{10.918056in}{3.991618in}}%
\pgfpathlineto{\pgfqpoint{10.920069in}{3.999442in}}%
\pgfpathlineto{\pgfqpoint{10.922083in}{3.989324in}}%
\pgfpathlineto{\pgfqpoint{10.924096in}{3.984333in}}%
\pgfpathlineto{\pgfqpoint{10.926110in}{3.990539in}}%
\pgfpathlineto{\pgfqpoint{10.932150in}{3.992832in}}%
\pgfpathlineto{\pgfqpoint{10.934164in}{4.005107in}}%
\pgfpathlineto{\pgfqpoint{10.936177in}{4.001060in}}%
\pgfpathlineto{\pgfqpoint{10.940204in}{4.011717in}}%
\pgfpathlineto{\pgfqpoint{10.946244in}{4.011987in}}%
\pgfpathlineto{\pgfqpoint{10.948258in}{4.018327in}}%
\pgfpathlineto{\pgfqpoint{10.950271in}{4.020485in}}%
\pgfpathlineto{\pgfqpoint{10.952285in}{4.004703in}}%
\pgfpathlineto{\pgfqpoint{10.954298in}{4.021565in}}%
\pgfpathlineto{\pgfqpoint{10.960338in}{4.025072in}}%
\pgfpathlineto{\pgfqpoint{10.962352in}{4.027770in}}%
\pgfpathlineto{\pgfqpoint{10.964365in}{4.021160in}}%
\pgfpathlineto{\pgfqpoint{10.966379in}{4.021969in}}%
\pgfpathlineto{\pgfqpoint{10.968392in}{4.019136in}}%
\pgfpathlineto{\pgfqpoint{10.974432in}{4.015224in}}%
\pgfpathlineto{\pgfqpoint{10.978459in}{4.016708in}}%
\pgfpathlineto{\pgfqpoint{10.982486in}{4.023453in}}%
\pgfpathlineto{\pgfqpoint{10.988527in}{4.026960in}}%
\pgfpathlineto{\pgfqpoint{10.990540in}{4.032491in}}%
\pgfpathlineto{\pgfqpoint{10.992553in}{4.028174in}}%
\pgfpathlineto{\pgfqpoint{10.994567in}{4.047195in}}%
\pgfpathlineto{\pgfqpoint{10.996580in}{4.043013in}}%
\pgfpathlineto{\pgfqpoint{11.002621in}{4.050837in}}%
\pgfpathlineto{\pgfqpoint{11.004634in}{4.051646in}}%
\pgfpathlineto{\pgfqpoint{11.006648in}{4.058526in}}%
\pgfpathlineto{\pgfqpoint{11.010675in}{4.062708in}}%
\pgfpathlineto{\pgfqpoint{11.016715in}{4.061764in}}%
\pgfpathlineto{\pgfqpoint{11.018728in}{4.066890in}}%
\pgfpathlineto{\pgfqpoint{11.020742in}{4.064866in}}%
\pgfpathlineto{\pgfqpoint{11.022755in}{4.065136in}}%
\pgfpathlineto{\pgfqpoint{11.024769in}{4.067699in}}%
\pgfpathlineto{\pgfqpoint{11.030809in}{4.062033in}}%
\pgfpathlineto{\pgfqpoint{11.032822in}{4.058391in}}%
\pgfpathlineto{\pgfqpoint{11.034836in}{4.052860in}}%
\pgfpathlineto{\pgfqpoint{11.036849in}{4.056098in}}%
\pgfpathlineto{\pgfqpoint{11.038863in}{4.050567in}}%
\pgfpathlineto{\pgfqpoint{11.044903in}{4.046925in}}%
\pgfpathlineto{\pgfqpoint{11.046917in}{4.042608in}}%
\pgfpathlineto{\pgfqpoint{11.050943in}{4.060010in}}%
\pgfpathlineto{\pgfqpoint{11.052957in}{4.052186in}}%
\pgfpathlineto{\pgfqpoint{11.061011in}{4.064866in}}%
\pgfpathlineto{\pgfqpoint{11.063024in}{4.065001in}}%
\pgfpathlineto{\pgfqpoint{11.067051in}{4.066215in}}%
\pgfpathlineto{\pgfqpoint{11.073091in}{4.061898in}}%
\pgfpathlineto{\pgfqpoint{11.077118in}{4.051242in}}%
\pgfpathlineto{\pgfqpoint{11.081145in}{4.052051in}}%
\pgfpathlineto{\pgfqpoint{11.087186in}{4.047195in}}%
\pgfpathlineto{\pgfqpoint{11.089199in}{4.039236in}}%
\pgfpathlineto{\pgfqpoint{11.093226in}{4.055963in}}%
\pgfpathlineto{\pgfqpoint{11.095239in}{4.056907in}}%
\pgfpathlineto{\pgfqpoint{11.101280in}{4.055019in}}%
\pgfpathlineto{\pgfqpoint{11.105307in}{4.052051in}}%
\pgfpathlineto{\pgfqpoint{11.107320in}{4.050028in}}%
\pgfpathlineto{\pgfqpoint{11.109333in}{4.053130in}}%
\pgfpathlineto{\pgfqpoint{11.117387in}{4.048948in}}%
\pgfpathlineto{\pgfqpoint{11.121414in}{4.058661in}}%
\pgfpathlineto{\pgfqpoint{11.123428in}{4.053130in}}%
\pgfpathlineto{\pgfqpoint{11.129468in}{4.045576in}}%
\pgfpathlineto{\pgfqpoint{11.131481in}{4.026286in}}%
\pgfpathlineto{\pgfqpoint{11.133495in}{4.021430in}}%
\pgfpathlineto{\pgfqpoint{11.135508in}{4.026556in}}%
\pgfpathlineto{\pgfqpoint{11.137522in}{4.012661in}}%
\pgfpathlineto{\pgfqpoint{11.143562in}{4.019946in}}%
\pgfpathlineto{\pgfqpoint{11.145575in}{4.020485in}}%
\pgfpathlineto{\pgfqpoint{11.147589in}{4.022239in}}%
\pgfpathlineto{\pgfqpoint{11.149602in}{4.026286in}}%
\pgfpathlineto{\pgfqpoint{11.151616in}{4.018597in}}%
\pgfpathlineto{\pgfqpoint{11.157656in}{4.014145in}}%
\pgfpathlineto{\pgfqpoint{11.159670in}{4.023318in}}%
\pgfpathlineto{\pgfqpoint{11.161683in}{4.021565in}}%
\pgfpathlineto{\pgfqpoint{11.163697in}{4.028579in}}%
\pgfpathlineto{\pgfqpoint{11.165710in}{4.031277in}}%
\pgfpathlineto{\pgfqpoint{11.173764in}{4.035864in}}%
\pgfpathlineto{\pgfqpoint{11.175777in}{4.030198in}}%
\pgfpathlineto{\pgfqpoint{11.177791in}{4.029523in}}%
\pgfpathlineto{\pgfqpoint{11.179804in}{4.031817in}}%
\pgfpathlineto{\pgfqpoint{11.185844in}{4.024128in}}%
\pgfpathlineto{\pgfqpoint{11.187858in}{4.031817in}}%
\pgfpathlineto{\pgfqpoint{11.189871in}{4.026421in}}%
\pgfpathlineto{\pgfqpoint{11.193898in}{4.018057in}}%
\pgfpathlineto{\pgfqpoint{11.199939in}{4.027635in}}%
\pgfpathlineto{\pgfqpoint{11.203965in}{4.028444in}}%
\pgfpathlineto{\pgfqpoint{11.205979in}{4.023723in}}%
\pgfpathlineto{\pgfqpoint{11.207992in}{4.017248in}}%
\pgfpathlineto{\pgfqpoint{11.214033in}{4.011717in}}%
\pgfpathlineto{\pgfqpoint{11.216046in}{3.999442in}}%
\pgfpathlineto{\pgfqpoint{11.218060in}{4.006996in}}%
\pgfpathlineto{\pgfqpoint{11.220073in}{3.988650in}}%
\pgfpathlineto{\pgfqpoint{11.222086in}{3.990269in}}%
\pgfpathlineto{\pgfqpoint{11.228127in}{3.989055in}}%
\pgfpathlineto{\pgfqpoint{11.230140in}{3.986222in}}%
\pgfpathlineto{\pgfqpoint{11.232154in}{3.989729in}}%
\pgfpathlineto{\pgfqpoint{11.234167in}{3.987976in}}%
\pgfpathlineto{\pgfqpoint{11.236181in}{3.994855in}}%
\pgfpathlineto{\pgfqpoint{11.242221in}{3.993506in}}%
\pgfpathlineto{\pgfqpoint{11.244234in}{3.988515in}}%
\pgfpathlineto{\pgfqpoint{11.246248in}{3.977858in}}%
\pgfpathlineto{\pgfqpoint{11.248261in}{3.980017in}}%
\pgfpathlineto{\pgfqpoint{11.250275in}{4.003084in}}%
\pgfpathlineto{\pgfqpoint{11.258329in}{3.994316in}}%
\pgfpathlineto{\pgfqpoint{11.260342in}{3.988785in}}%
\pgfpathlineto{\pgfqpoint{11.262355in}{3.988785in}}%
\pgfpathlineto{\pgfqpoint{11.270409in}{3.991618in}}%
\pgfpathlineto{\pgfqpoint{11.272423in}{3.994181in}}%
\pgfpathlineto{\pgfqpoint{11.274436in}{3.994855in}}%
\pgfpathlineto{\pgfqpoint{11.276450in}{3.993911in}}%
\pgfpathlineto{\pgfqpoint{11.278463in}{4.002274in}}%
\pgfpathlineto{\pgfqpoint{11.284503in}{3.999846in}}%
\pgfpathlineto{\pgfqpoint{11.286517in}{3.997014in}}%
\pgfpathlineto{\pgfqpoint{11.288530in}{4.012931in}}%
\pgfpathlineto{\pgfqpoint{11.290544in}{4.013471in}}%
\pgfpathlineto{\pgfqpoint{11.292557in}{4.008615in}}%
\pgfpathlineto{\pgfqpoint{11.298597in}{4.011717in}}%
\pgfpathlineto{\pgfqpoint{11.300611in}{4.008210in}}%
\pgfpathlineto{\pgfqpoint{11.302624in}{4.011447in}}%
\pgfpathlineto{\pgfqpoint{11.306651in}{4.004028in}}%
\pgfpathlineto{\pgfqpoint{11.312692in}{4.009019in}}%
\pgfpathlineto{\pgfqpoint{11.314705in}{4.015224in}}%
\pgfpathlineto{\pgfqpoint{11.316718in}{4.013741in}}%
\pgfpathlineto{\pgfqpoint{11.318732in}{4.009694in}}%
\pgfpathlineto{\pgfqpoint{11.320745in}{4.020081in}}%
\pgfpathlineto{\pgfqpoint{11.326786in}{4.020216in}}%
\pgfpathlineto{\pgfqpoint{11.330813in}{4.008480in}}%
\pgfpathlineto{\pgfqpoint{11.332826in}{4.008749in}}%
\pgfpathlineto{\pgfqpoint{11.334840in}{4.015494in}}%
\pgfpathlineto{\pgfqpoint{11.340880in}{4.014145in}}%
\pgfpathlineto{\pgfqpoint{11.342893in}{4.008884in}}%
\pgfpathlineto{\pgfqpoint{11.346920in}{4.017518in}}%
\pgfpathlineto{\pgfqpoint{11.348934in}{4.017788in}}%
\pgfpathlineto{\pgfqpoint{11.354974in}{4.022779in}}%
\pgfpathlineto{\pgfqpoint{11.356987in}{4.019676in}}%
\pgfpathlineto{\pgfqpoint{11.361014in}{4.024532in}}%
\pgfpathlineto{\pgfqpoint{11.363028in}{4.023318in}}%
\pgfpathlineto{\pgfqpoint{11.371082in}{4.019136in}}%
\pgfpathlineto{\pgfqpoint{11.373095in}{4.026421in}}%
\pgfpathlineto{\pgfqpoint{11.375108in}{4.030063in}}%
\pgfpathlineto{\pgfqpoint{11.377122in}{4.035459in}}%
\pgfpathlineto{\pgfqpoint{11.383162in}{4.028849in}}%
\pgfpathlineto{\pgfqpoint{11.387189in}{4.014550in}}%
\pgfpathlineto{\pgfqpoint{11.391216in}{4.003893in}}%
\pgfpathlineto{\pgfqpoint{11.397256in}{3.997418in}}%
\pgfpathlineto{\pgfqpoint{11.399270in}{3.996744in}}%
\pgfpathlineto{\pgfqpoint{11.401283in}{4.003623in}}%
\pgfpathlineto{\pgfqpoint{11.403297in}{4.004028in}}%
\pgfpathlineto{\pgfqpoint{11.405310in}{3.997688in}}%
\pgfpathlineto{\pgfqpoint{11.411351in}{3.998497in}}%
\pgfpathlineto{\pgfqpoint{11.415377in}{4.005242in}}%
\pgfpathlineto{\pgfqpoint{11.417391in}{4.010368in}}%
\pgfpathlineto{\pgfqpoint{11.419404in}{4.006726in}}%
\pgfpathlineto{\pgfqpoint{11.425445in}{4.008884in}}%
\pgfpathlineto{\pgfqpoint{11.429472in}{4.004703in}}%
\pgfpathlineto{\pgfqpoint{11.431485in}{4.005647in}}%
\pgfpathlineto{\pgfqpoint{11.433498in}{3.994046in}}%
\pgfpathlineto{\pgfqpoint{11.439539in}{3.986357in}}%
\pgfpathlineto{\pgfqpoint{11.441552in}{3.986761in}}%
\pgfpathlineto{\pgfqpoint{11.443566in}{3.983929in}}%
\pgfpathlineto{\pgfqpoint{11.445579in}{3.988515in}}%
\pgfpathlineto{\pgfqpoint{11.455646in}{3.980556in}}%
\pgfpathlineto{\pgfqpoint{11.457660in}{3.975700in}}%
\pgfpathlineto{\pgfqpoint{11.459673in}{3.968820in}}%
\pgfpathlineto{\pgfqpoint{11.461687in}{3.971653in}}%
\pgfpathlineto{\pgfqpoint{11.467727in}{3.978398in}}%
\pgfpathlineto{\pgfqpoint{11.469740in}{3.977454in}}%
\pgfpathlineto{\pgfqpoint{11.471754in}{3.977993in}}%
\pgfpathlineto{\pgfqpoint{11.473767in}{3.980556in}}%
\pgfpathlineto{\pgfqpoint{11.475781in}{3.975295in}}%
\pgfpathlineto{\pgfqpoint{11.481821in}{3.970844in}}%
\pgfpathlineto{\pgfqpoint{11.483835in}{3.966257in}}%
\pgfpathlineto{\pgfqpoint{11.485848in}{3.964908in}}%
\pgfpathlineto{\pgfqpoint{11.487862in}{3.964908in}}%
\pgfpathlineto{\pgfqpoint{11.489875in}{3.958298in}}%
\pgfpathlineto{\pgfqpoint{11.495915in}{3.961806in}}%
\pgfpathlineto{\pgfqpoint{11.497929in}{3.969225in}}%
\pgfpathlineto{\pgfqpoint{11.499942in}{3.969765in}}%
\pgfpathlineto{\pgfqpoint{11.501956in}{3.968551in}}%
\pgfpathlineto{\pgfqpoint{11.512023in}{3.971248in}}%
\pgfpathlineto{\pgfqpoint{11.514036in}{3.974081in}}%
\pgfpathlineto{\pgfqpoint{11.518063in}{3.971114in}}%
\pgfpathlineto{\pgfqpoint{11.524104in}{3.980286in}}%
\pgfpathlineto{\pgfqpoint{11.526117in}{3.972193in}}%
\pgfpathlineto{\pgfqpoint{11.528130in}{3.978128in}}%
\pgfpathlineto{\pgfqpoint{11.530144in}{3.970979in}}%
\pgfpathlineto{\pgfqpoint{11.532157in}{3.972867in}}%
\pgfpathlineto{\pgfqpoint{11.538198in}{3.973542in}}%
\pgfpathlineto{\pgfqpoint{11.540211in}{3.971518in}}%
\pgfpathlineto{\pgfqpoint{11.542225in}{3.964773in}}%
\pgfpathlineto{\pgfqpoint{11.546251in}{3.942920in}}%
\pgfpathlineto{\pgfqpoint{11.552292in}{3.939143in}}%
\pgfpathlineto{\pgfqpoint{11.554305in}{3.934557in}}%
\pgfpathlineto{\pgfqpoint{11.556319in}{3.951958in}}%
\pgfpathlineto{\pgfqpoint{11.558332in}{3.957084in}}%
\pgfpathlineto{\pgfqpoint{11.560346in}{3.965583in}}%
\pgfpathlineto{\pgfqpoint{11.566386in}{3.966932in}}%
\pgfpathlineto{\pgfqpoint{11.568399in}{3.958298in}}%
\pgfpathlineto{\pgfqpoint{11.570413in}{3.967741in}}%
\pgfpathlineto{\pgfqpoint{11.572426in}{3.973542in}}%
\pgfpathlineto{\pgfqpoint{11.574440in}{3.966797in}}%
\pgfpathlineto{\pgfqpoint{11.582494in}{3.978668in}}%
\pgfpathlineto{\pgfqpoint{11.584507in}{3.975430in}}%
\pgfpathlineto{\pgfqpoint{11.586520in}{3.975835in}}%
\pgfpathlineto{\pgfqpoint{11.588534in}{3.978263in}}%
\pgfpathlineto{\pgfqpoint{11.594574in}{3.977319in}}%
\pgfpathlineto{\pgfqpoint{11.596588in}{3.981501in}}%
\pgfpathlineto{\pgfqpoint{11.598601in}{3.981905in}}%
\pgfpathlineto{\pgfqpoint{11.600615in}{3.981231in}}%
\pgfpathlineto{\pgfqpoint{11.602628in}{3.972867in}}%
\pgfpathlineto{\pgfqpoint{11.608668in}{3.974486in}}%
\pgfpathlineto{\pgfqpoint{11.610682in}{3.968551in}}%
\pgfpathlineto{\pgfqpoint{11.612695in}{3.969360in}}%
\pgfpathlineto{\pgfqpoint{11.614709in}{3.966257in}}%
\pgfpathlineto{\pgfqpoint{11.616722in}{3.970304in}}%
\pgfpathlineto{\pgfqpoint{11.622762in}{3.969630in}}%
\pgfpathlineto{\pgfqpoint{11.624776in}{3.975430in}}%
\pgfpathlineto{\pgfqpoint{11.626789in}{3.986357in}}%
\pgfpathlineto{\pgfqpoint{11.628803in}{3.984738in}}%
\pgfpathlineto{\pgfqpoint{11.630816in}{3.990943in}}%
\pgfpathlineto{\pgfqpoint{11.636857in}{3.999442in}}%
\pgfpathlineto{\pgfqpoint{11.640884in}{4.012931in}}%
\pgfpathlineto{\pgfqpoint{11.642897in}{4.015359in}}%
\pgfpathlineto{\pgfqpoint{11.644910in}{4.010773in}}%
\pgfpathlineto{\pgfqpoint{11.650951in}{4.011582in}}%
\pgfpathlineto{\pgfqpoint{11.652964in}{4.009559in}}%
\pgfpathlineto{\pgfqpoint{11.654978in}{4.018732in}}%
\pgfpathlineto{\pgfqpoint{11.656991in}{4.018192in}}%
\pgfpathlineto{\pgfqpoint{11.659005in}{4.021699in}}%
\pgfpathlineto{\pgfqpoint{11.665045in}{4.028309in}}%
\pgfpathlineto{\pgfqpoint{11.667058in}{4.026556in}}%
\pgfpathlineto{\pgfqpoint{11.669072in}{4.025611in}}%
\pgfpathlineto{\pgfqpoint{11.671085in}{4.038157in}}%
\pgfpathlineto{\pgfqpoint{11.673099in}{4.044227in}}%
\pgfpathlineto{\pgfqpoint{11.681152in}{4.038966in}}%
\pgfpathlineto{\pgfqpoint{11.683166in}{4.041934in}}%
\pgfpathlineto{\pgfqpoint{11.685179in}{4.033705in}}%
\pgfpathlineto{\pgfqpoint{11.687193in}{4.031682in}}%
\pgfpathlineto{\pgfqpoint{11.693233in}{4.034649in}}%
\pgfpathlineto{\pgfqpoint{11.695247in}{4.037078in}}%
\pgfpathlineto{\pgfqpoint{11.697260in}{4.038022in}}%
\pgfpathlineto{\pgfqpoint{11.701287in}{4.034245in}}%
\pgfpathlineto{\pgfqpoint{11.709341in}{4.026556in}}%
\pgfpathlineto{\pgfqpoint{11.715381in}{4.013066in}}%
\pgfpathlineto{\pgfqpoint{11.721421in}{4.012931in}}%
\pgfpathlineto{\pgfqpoint{11.725448in}{4.026016in}}%
\pgfpathlineto{\pgfqpoint{11.727462in}{4.039910in}}%
\pgfpathlineto{\pgfqpoint{11.729475in}{4.044362in}}%
\pgfpathlineto{\pgfqpoint{11.735516in}{4.042069in}}%
\pgfpathlineto{\pgfqpoint{11.737529in}{4.040585in}}%
\pgfpathlineto{\pgfqpoint{11.739542in}{4.041799in}}%
\pgfpathlineto{\pgfqpoint{11.743569in}{4.041799in}}%
\pgfpathlineto{\pgfqpoint{11.749610in}{4.045711in}}%
\pgfpathlineto{\pgfqpoint{11.751623in}{4.049623in}}%
\pgfpathlineto{\pgfqpoint{11.753637in}{4.046385in}}%
\pgfpathlineto{\pgfqpoint{11.755650in}{4.036673in}}%
\pgfpathlineto{\pgfqpoint{11.757663in}{4.047734in}}%
\pgfpathlineto{\pgfqpoint{11.763704in}{4.048274in}}%
\pgfpathlineto{\pgfqpoint{11.765717in}{4.045441in}}%
\pgfpathlineto{\pgfqpoint{11.767731in}{4.046116in}}%
\pgfpathlineto{\pgfqpoint{11.769744in}{4.045711in}}%
\pgfpathlineto{\pgfqpoint{11.771758in}{4.039506in}}%
\pgfpathlineto{\pgfqpoint{11.777798in}{4.041934in}}%
\pgfpathlineto{\pgfqpoint{11.779811in}{4.050702in}}%
\pgfpathlineto{\pgfqpoint{11.781825in}{4.052186in}}%
\pgfpathlineto{\pgfqpoint{11.783838in}{4.047330in}}%
\pgfpathlineto{\pgfqpoint{11.785852in}{4.034649in}}%
\pgfpathlineto{\pgfqpoint{11.791892in}{4.039101in}}%
\pgfpathlineto{\pgfqpoint{11.793905in}{4.045171in}}%
\pgfpathlineto{\pgfqpoint{11.795919in}{4.048409in}}%
\pgfpathlineto{\pgfqpoint{11.805986in}{4.047599in}}%
\pgfpathlineto{\pgfqpoint{11.808000in}{4.053805in}}%
\pgfpathlineto{\pgfqpoint{11.812027in}{4.041799in}}%
\pgfpathlineto{\pgfqpoint{11.822094in}{4.034110in}}%
\pgfpathlineto{\pgfqpoint{11.824107in}{4.024937in}}%
\pgfpathlineto{\pgfqpoint{11.826121in}{4.009829in}}%
\pgfpathlineto{\pgfqpoint{11.828134in}{4.005782in}}%
\pgfpathlineto{\pgfqpoint{11.834174in}{4.012527in}}%
\pgfpathlineto{\pgfqpoint{11.836188in}{4.020081in}}%
\pgfpathlineto{\pgfqpoint{11.838201in}{4.010638in}}%
\pgfpathlineto{\pgfqpoint{11.840215in}{4.020755in}}%
\pgfpathlineto{\pgfqpoint{11.842228in}{3.984333in}}%
\pgfpathlineto{\pgfqpoint{11.850282in}{3.984873in}}%
\pgfpathlineto{\pgfqpoint{11.852295in}{3.982310in}}%
\pgfpathlineto{\pgfqpoint{11.854309in}{3.983119in}}%
\pgfpathlineto{\pgfqpoint{11.856322in}{3.986492in}}%
\pgfpathlineto{\pgfqpoint{11.862363in}{3.982445in}}%
\pgfpathlineto{\pgfqpoint{11.864376in}{3.986492in}}%
\pgfpathlineto{\pgfqpoint{11.866390in}{3.985008in}}%
\pgfpathlineto{\pgfqpoint{11.868403in}{3.986896in}}%
\pgfpathlineto{\pgfqpoint{11.870416in}{3.999711in}}%
\pgfpathlineto{\pgfqpoint{11.876457in}{3.997283in}}%
\pgfpathlineto{\pgfqpoint{11.878470in}{3.984873in}}%
\pgfpathlineto{\pgfqpoint{11.880484in}{3.982310in}}%
\pgfpathlineto{\pgfqpoint{11.882497in}{3.987706in}}%
\pgfpathlineto{\pgfqpoint{11.884511in}{3.978668in}}%
\pgfpathlineto{\pgfqpoint{11.890551in}{3.975970in}}%
\pgfpathlineto{\pgfqpoint{11.892564in}{3.975835in}}%
\pgfpathlineto{\pgfqpoint{11.894578in}{3.968685in}}%
\pgfpathlineto{\pgfqpoint{11.896591in}{3.968551in}}%
\pgfpathlineto{\pgfqpoint{11.898605in}{3.973677in}}%
\pgfpathlineto{\pgfqpoint{11.906659in}{3.975430in}}%
\pgfpathlineto{\pgfqpoint{11.908672in}{3.983929in}}%
\pgfpathlineto{\pgfqpoint{11.910685in}{3.983389in}}%
\pgfpathlineto{\pgfqpoint{11.912699in}{3.974621in}}%
\pgfpathlineto{\pgfqpoint{11.918739in}{3.982445in}}%
\pgfpathlineto{\pgfqpoint{11.920753in}{3.975700in}}%
\pgfpathlineto{\pgfqpoint{11.924780in}{3.985817in}}%
\pgfpathlineto{\pgfqpoint{11.926793in}{3.988110in}}%
\pgfpathlineto{\pgfqpoint{11.932833in}{3.985413in}}%
\pgfpathlineto{\pgfqpoint{11.934847in}{3.995125in}}%
\pgfpathlineto{\pgfqpoint{11.936860in}{3.997148in}}%
\pgfpathlineto{\pgfqpoint{11.940887in}{3.998228in}}%
\pgfpathlineto{\pgfqpoint{11.946927in}{4.002140in}}%
\pgfpathlineto{\pgfqpoint{11.948941in}{3.997418in}}%
\pgfpathlineto{\pgfqpoint{11.952968in}{4.005917in}}%
\pgfpathlineto{\pgfqpoint{11.954981in}{4.012257in}}%
\pgfpathlineto{\pgfqpoint{11.961022in}{4.008210in}}%
\pgfpathlineto{\pgfqpoint{11.963035in}{4.010908in}}%
\pgfpathlineto{\pgfqpoint{11.965049in}{4.011313in}}%
\pgfpathlineto{\pgfqpoint{11.967062in}{4.014955in}}%
\pgfpathlineto{\pgfqpoint{11.969075in}{4.023588in}}%
\pgfpathlineto{\pgfqpoint{11.975116in}{4.019406in}}%
\pgfpathlineto{\pgfqpoint{11.977129in}{4.019136in}}%
\pgfpathlineto{\pgfqpoint{11.979143in}{4.015224in}}%
\pgfpathlineto{\pgfqpoint{11.981156in}{4.013741in}}%
\pgfpathlineto{\pgfqpoint{11.989210in}{4.014010in}}%
\pgfpathlineto{\pgfqpoint{11.991223in}{4.019946in}}%
\pgfpathlineto{\pgfqpoint{11.993237in}{4.023993in}}%
\pgfpathlineto{\pgfqpoint{11.995250in}{4.019541in}}%
\pgfpathlineto{\pgfqpoint{11.997264in}{4.020755in}}%
\pgfpathlineto{\pgfqpoint{12.005317in}{4.014010in}}%
\pgfpathlineto{\pgfqpoint{12.007331in}{4.016169in}}%
\pgfpathlineto{\pgfqpoint{12.009344in}{4.009694in}}%
\pgfpathlineto{\pgfqpoint{12.011358in}{4.010638in}}%
\pgfpathlineto{\pgfqpoint{12.017398in}{4.011178in}}%
\pgfpathlineto{\pgfqpoint{12.019412in}{4.013471in}}%
\pgfpathlineto{\pgfqpoint{12.021425in}{4.016843in}}%
\pgfpathlineto{\pgfqpoint{12.025452in}{4.008480in}}%
\pgfpathlineto{\pgfqpoint{12.031492in}{4.010908in}}%
\pgfpathlineto{\pgfqpoint{12.033506in}{4.010233in}}%
\pgfpathlineto{\pgfqpoint{12.035519in}{4.015224in}}%
\pgfpathlineto{\pgfqpoint{12.037533in}{4.014820in}}%
\pgfpathlineto{\pgfqpoint{12.039546in}{4.010773in}}%
\pgfpathlineto{\pgfqpoint{12.045586in}{4.007670in}}%
\pgfpathlineto{\pgfqpoint{12.047600in}{4.007805in}}%
\pgfpathlineto{\pgfqpoint{12.049613in}{4.012122in}}%
\pgfpathlineto{\pgfqpoint{12.051627in}{4.004163in}}%
\pgfpathlineto{\pgfqpoint{12.053640in}{3.993911in}}%
\pgfpathlineto{\pgfqpoint{12.059681in}{3.998093in}}%
\pgfpathlineto{\pgfqpoint{12.063707in}{3.991888in}}%
\pgfpathlineto{\pgfqpoint{12.065721in}{3.992427in}}%
\pgfpathlineto{\pgfqpoint{12.067734in}{3.994181in}}%
\pgfpathlineto{\pgfqpoint{12.073775in}{3.991213in}}%
\pgfpathlineto{\pgfqpoint{12.075788in}{3.995395in}}%
\pgfpathlineto{\pgfqpoint{12.077802in}{3.994451in}}%
\pgfpathlineto{\pgfqpoint{12.079815in}{3.990673in}}%
\pgfpathlineto{\pgfqpoint{12.087869in}{3.998632in}}%
\pgfpathlineto{\pgfqpoint{12.089882in}{3.993506in}}%
\pgfpathlineto{\pgfqpoint{12.091896in}{3.993641in}}%
\pgfpathlineto{\pgfqpoint{12.093909in}{3.989055in}}%
\pgfpathlineto{\pgfqpoint{12.095923in}{3.995530in}}%
\pgfpathlineto{\pgfqpoint{12.101963in}{3.996609in}}%
\pgfpathlineto{\pgfqpoint{12.103976in}{4.006861in}}%
\pgfpathlineto{\pgfqpoint{12.105990in}{4.011043in}}%
\pgfpathlineto{\pgfqpoint{12.110017in}{4.013201in}}%
\pgfpathlineto{\pgfqpoint{12.118070in}{4.013471in}}%
\pgfpathlineto{\pgfqpoint{12.122097in}{4.015629in}}%
\pgfpathlineto{\pgfqpoint{12.124111in}{4.013876in}}%
\pgfpathlineto{\pgfqpoint{12.130151in}{4.014685in}}%
\pgfpathlineto{\pgfqpoint{12.132165in}{4.017113in}}%
\pgfpathlineto{\pgfqpoint{12.136192in}{4.017922in}}%
\pgfpathlineto{\pgfqpoint{12.138205in}{4.019136in}}%
\pgfpathlineto{\pgfqpoint{12.144245in}{4.020755in}}%
\pgfpathlineto{\pgfqpoint{12.146259in}{4.020351in}}%
\pgfpathlineto{\pgfqpoint{12.148272in}{4.013741in}}%
\pgfpathlineto{\pgfqpoint{12.152299in}{4.015629in}}%
\pgfpathlineto{\pgfqpoint{12.160353in}{4.022644in}}%
\pgfpathlineto{\pgfqpoint{12.162366in}{4.022239in}}%
\pgfpathlineto{\pgfqpoint{12.164380in}{4.030872in}}%
\pgfpathlineto{\pgfqpoint{12.166393in}{4.013066in}}%
\pgfpathlineto{\pgfqpoint{12.172434in}{4.002679in}}%
\pgfpathlineto{\pgfqpoint{12.174447in}{4.008480in}}%
\pgfpathlineto{\pgfqpoint{12.178474in}{4.028579in}}%
\pgfpathlineto{\pgfqpoint{12.180487in}{4.027905in}}%
\pgfpathlineto{\pgfqpoint{12.188541in}{4.027095in}}%
\pgfpathlineto{\pgfqpoint{12.192568in}{4.033570in}}%
\pgfpathlineto{\pgfqpoint{12.194581in}{4.043553in}}%
\pgfpathlineto{\pgfqpoint{12.200622in}{4.048274in}}%
\pgfpathlineto{\pgfqpoint{12.202635in}{4.055154in}}%
\pgfpathlineto{\pgfqpoint{12.204649in}{4.056098in}}%
\pgfpathlineto{\pgfqpoint{12.206662in}{4.058391in}}%
\pgfpathlineto{\pgfqpoint{12.208676in}{4.056772in}}%
\pgfpathlineto{\pgfqpoint{12.214716in}{4.056638in}}%
\pgfpathlineto{\pgfqpoint{12.216729in}{4.057852in}}%
\pgfpathlineto{\pgfqpoint{12.218743in}{4.064596in}}%
\pgfpathlineto{\pgfqpoint{12.220756in}{4.046925in}}%
\pgfpathlineto{\pgfqpoint{12.222770in}{4.051781in}}%
\pgfpathlineto{\pgfqpoint{12.228810in}{4.052051in}}%
\pgfpathlineto{\pgfqpoint{12.230824in}{4.057042in}}%
\pgfpathlineto{\pgfqpoint{12.232837in}{4.053805in}}%
\pgfpathlineto{\pgfqpoint{12.234850in}{4.053130in}}%
\pgfpathlineto{\pgfqpoint{12.236864in}{4.054209in}}%
\pgfpathlineto{\pgfqpoint{12.242904in}{4.054074in}}%
\pgfpathlineto{\pgfqpoint{12.244918in}{4.050432in}}%
\pgfpathlineto{\pgfqpoint{12.246931in}{4.049893in}}%
\pgfpathlineto{\pgfqpoint{12.250958in}{4.059066in}}%
\pgfpathlineto{\pgfqpoint{12.256998in}{4.059740in}}%
\pgfpathlineto{\pgfqpoint{12.259012in}{4.058256in}}%
\pgfpathlineto{\pgfqpoint{12.261025in}{4.053400in}}%
\pgfpathlineto{\pgfqpoint{12.263039in}{4.055289in}}%
\pgfpathlineto{\pgfqpoint{12.265052in}{4.053805in}}%
\pgfpathlineto{\pgfqpoint{12.271092in}{4.058121in}}%
\pgfpathlineto{\pgfqpoint{12.273106in}{4.061898in}}%
\pgfpathlineto{\pgfqpoint{12.275119in}{4.059470in}}%
\pgfpathlineto{\pgfqpoint{12.277133in}{4.058931in}}%
\pgfpathlineto{\pgfqpoint{12.279146in}{4.062303in}}%
\pgfpathlineto{\pgfqpoint{12.287200in}{4.064327in}}%
\pgfpathlineto{\pgfqpoint{12.289214in}{4.061089in}}%
\pgfpathlineto{\pgfqpoint{12.291227in}{4.060415in}}%
\pgfpathlineto{\pgfqpoint{12.293240in}{4.062573in}}%
\pgfpathlineto{\pgfqpoint{12.299281in}{4.066215in}}%
\pgfpathlineto{\pgfqpoint{12.305321in}{4.072016in}}%
\pgfpathlineto{\pgfqpoint{12.307335in}{4.072825in}}%
\pgfpathlineto{\pgfqpoint{12.315388in}{4.078895in}}%
\pgfpathlineto{\pgfqpoint{12.317402in}{4.077546in}}%
\pgfpathlineto{\pgfqpoint{12.319415in}{4.077277in}}%
\pgfpathlineto{\pgfqpoint{12.321429in}{4.064731in}}%
\pgfpathlineto{\pgfqpoint{12.327469in}{4.072825in}}%
\pgfpathlineto{\pgfqpoint{12.329482in}{4.066890in}}%
\pgfpathlineto{\pgfqpoint{12.331496in}{4.067025in}}%
\pgfpathlineto{\pgfqpoint{12.335523in}{4.092790in}}%
\pgfpathlineto{\pgfqpoint{12.341563in}{4.086315in}}%
\pgfpathlineto{\pgfqpoint{12.343577in}{4.086045in}}%
\pgfpathlineto{\pgfqpoint{12.345590in}{4.089957in}}%
\pgfpathlineto{\pgfqpoint{12.347603in}{4.091306in}}%
\pgfpathlineto{\pgfqpoint{12.349617in}{4.086719in}}%
\pgfpathlineto{\pgfqpoint{12.355657in}{4.079975in}}%
\pgfpathlineto{\pgfqpoint{12.357671in}{4.086584in}}%
\pgfpathlineto{\pgfqpoint{12.359684in}{4.089822in}}%
\pgfpathlineto{\pgfqpoint{12.361698in}{4.088338in}}%
\pgfpathlineto{\pgfqpoint{12.363711in}{4.093734in}}%
\pgfpathlineto{\pgfqpoint{12.369751in}{4.092655in}}%
\pgfpathlineto{\pgfqpoint{12.371765in}{4.091171in}}%
\pgfpathlineto{\pgfqpoint{12.373778in}{4.096702in}}%
\pgfpathlineto{\pgfqpoint{12.375792in}{4.097781in}}%
\pgfpathlineto{\pgfqpoint{12.377805in}{4.098185in}}%
\pgfpathlineto{\pgfqpoint{12.383846in}{4.097106in}}%
\pgfpathlineto{\pgfqpoint{12.385859in}{4.087798in}}%
\pgfpathlineto{\pgfqpoint{12.389886in}{4.084021in}}%
\pgfpathlineto{\pgfqpoint{12.391899in}{4.089957in}}%
\pgfpathlineto{\pgfqpoint{12.397940in}{4.087933in}}%
\pgfpathlineto{\pgfqpoint{12.399953in}{4.093734in}}%
\pgfpathlineto{\pgfqpoint{12.401967in}{4.065676in}}%
\pgfpathlineto{\pgfqpoint{12.403980in}{4.064596in}}%
\pgfpathlineto{\pgfqpoint{12.405993in}{4.061089in}}%
\pgfpathlineto{\pgfqpoint{12.412034in}{4.062573in}}%
\pgfpathlineto{\pgfqpoint{12.418074in}{4.056907in}}%
\pgfpathlineto{\pgfqpoint{12.420088in}{4.055963in}}%
\pgfpathlineto{\pgfqpoint{12.426128in}{4.057582in}}%
\pgfpathlineto{\pgfqpoint{12.428141in}{4.053265in}}%
\pgfpathlineto{\pgfqpoint{12.430155in}{4.054209in}}%
\pgfpathlineto{\pgfqpoint{12.434182in}{4.045036in}}%
\pgfpathlineto{\pgfqpoint{12.440222in}{4.058661in}}%
\pgfpathlineto{\pgfqpoint{12.444249in}{4.059470in}}%
\pgfpathlineto{\pgfqpoint{12.446262in}{4.056233in}}%
\pgfpathlineto{\pgfqpoint{12.448276in}{4.057717in}}%
\pgfpathlineto{\pgfqpoint{12.454316in}{4.055963in}}%
\pgfpathlineto{\pgfqpoint{12.456330in}{4.061494in}}%
\pgfpathlineto{\pgfqpoint{12.458343in}{4.060549in}}%
\pgfpathlineto{\pgfqpoint{12.460357in}{4.062843in}}%
\pgfpathlineto{\pgfqpoint{12.462370in}{4.061898in}}%
\pgfpathlineto{\pgfqpoint{12.468410in}{4.062303in}}%
\pgfpathlineto{\pgfqpoint{12.470424in}{4.068643in}}%
\pgfpathlineto{\pgfqpoint{12.472437in}{4.065136in}}%
\pgfpathlineto{\pgfqpoint{12.476464in}{4.068104in}}%
\pgfpathlineto{\pgfqpoint{12.482504in}{4.069048in}}%
\pgfpathlineto{\pgfqpoint{12.484518in}{4.066485in}}%
\pgfpathlineto{\pgfqpoint{12.486531in}{4.058796in}}%
\pgfpathlineto{\pgfqpoint{12.488545in}{4.046925in}}%
\pgfpathlineto{\pgfqpoint{12.490558in}{4.051916in}}%
\pgfpathlineto{\pgfqpoint{12.496599in}{4.054884in}}%
\pgfpathlineto{\pgfqpoint{12.498612in}{4.059066in}}%
\pgfpathlineto{\pgfqpoint{12.500625in}{4.068913in}}%
\pgfpathlineto{\pgfqpoint{12.502639in}{4.071476in}}%
\pgfpathlineto{\pgfqpoint{12.504652in}{4.072151in}}%
\pgfpathlineto{\pgfqpoint{12.510693in}{4.074848in}}%
\pgfpathlineto{\pgfqpoint{12.512706in}{4.085370in}}%
\pgfpathlineto{\pgfqpoint{12.514720in}{4.082133in}}%
\pgfpathlineto{\pgfqpoint{12.516733in}{4.085235in}}%
\pgfpathlineto{\pgfqpoint{12.518746in}{4.079165in}}%
\pgfpathlineto{\pgfqpoint{12.524787in}{4.086450in}}%
\pgfpathlineto{\pgfqpoint{12.526800in}{4.090496in}}%
\pgfpathlineto{\pgfqpoint{12.528814in}{4.087664in}}%
\pgfpathlineto{\pgfqpoint{12.530827in}{4.086989in}}%
\pgfpathlineto{\pgfqpoint{12.532841in}{4.087529in}}%
\pgfpathlineto{\pgfqpoint{12.540894in}{4.088743in}}%
\pgfpathlineto{\pgfqpoint{12.542908in}{4.083212in}}%
\pgfpathlineto{\pgfqpoint{12.544921in}{4.083617in}}%
\pgfpathlineto{\pgfqpoint{12.546935in}{4.078626in}}%
\pgfpathlineto{\pgfqpoint{12.554989in}{4.082807in}}%
\pgfpathlineto{\pgfqpoint{12.557002in}{4.080379in}}%
\pgfpathlineto{\pgfqpoint{12.559015in}{4.079705in}}%
\pgfpathlineto{\pgfqpoint{12.561029in}{4.081323in}}%
\pgfpathlineto{\pgfqpoint{12.567069in}{4.082942in}}%
\pgfpathlineto{\pgfqpoint{12.569083in}{4.081998in}}%
\pgfpathlineto{\pgfqpoint{12.571096in}{4.087259in}}%
\pgfpathlineto{\pgfqpoint{12.573110in}{4.084156in}}%
\pgfpathlineto{\pgfqpoint{12.575123in}{4.085235in}}%
\pgfpathlineto{\pgfqpoint{12.583177in}{4.085370in}}%
\pgfpathlineto{\pgfqpoint{12.585190in}{4.084831in}}%
\pgfpathlineto{\pgfqpoint{12.587204in}{4.082403in}}%
\pgfpathlineto{\pgfqpoint{12.589217in}{4.087124in}}%
\pgfpathlineto{\pgfqpoint{12.595257in}{4.084966in}}%
\pgfpathlineto{\pgfqpoint{12.597271in}{4.095757in}}%
\pgfpathlineto{\pgfqpoint{12.599284in}{4.097916in}}%
\pgfpathlineto{\pgfqpoint{12.601298in}{4.094948in}}%
\pgfpathlineto{\pgfqpoint{12.603311in}{4.100209in}}%
\pgfpathlineto{\pgfqpoint{12.609352in}{4.093194in}}%
\pgfpathlineto{\pgfqpoint{12.611365in}{4.085640in}}%
\pgfpathlineto{\pgfqpoint{12.613379in}{4.081863in}}%
\pgfpathlineto{\pgfqpoint{12.615392in}{4.083886in}}%
\pgfpathlineto{\pgfqpoint{12.617405in}{4.085101in}}%
\pgfpathlineto{\pgfqpoint{12.623446in}{4.081863in}}%
\pgfpathlineto{\pgfqpoint{12.625459in}{4.082942in}}%
\pgfpathlineto{\pgfqpoint{12.627473in}{4.083347in}}%
\pgfpathlineto{\pgfqpoint{12.629486in}{4.071611in}}%
\pgfpathlineto{\pgfqpoint{12.631500in}{4.070127in}}%
\pgfpathlineto{\pgfqpoint{12.641567in}{4.079165in}}%
\pgfpathlineto{\pgfqpoint{12.643580in}{4.083752in}}%
\pgfpathlineto{\pgfqpoint{12.645594in}{4.084561in}}%
\pgfpathlineto{\pgfqpoint{12.653647in}{4.085101in}}%
\pgfpathlineto{\pgfqpoint{12.655661in}{4.079300in}}%
\pgfpathlineto{\pgfqpoint{12.657674in}{4.080784in}}%
\pgfpathlineto{\pgfqpoint{12.659688in}{4.085235in}}%
\pgfpathlineto{\pgfqpoint{12.665728in}{4.084966in}}%
\pgfpathlineto{\pgfqpoint{12.669755in}{4.077546in}}%
\pgfpathlineto{\pgfqpoint{12.673782in}{4.077142in}}%
\pgfpathlineto{\pgfqpoint{12.679822in}{4.072960in}}%
\pgfpathlineto{\pgfqpoint{12.681836in}{4.075928in}}%
\pgfpathlineto{\pgfqpoint{12.683849in}{4.073634in}}%
\pgfpathlineto{\pgfqpoint{12.685863in}{4.076197in}}%
\pgfpathlineto{\pgfqpoint{12.687876in}{4.077277in}}%
\pgfpathlineto{\pgfqpoint{12.693916in}{4.067834in}}%
\pgfpathlineto{\pgfqpoint{12.695930in}{4.068104in}}%
\pgfpathlineto{\pgfqpoint{12.697943in}{4.067025in}}%
\pgfpathlineto{\pgfqpoint{12.699957in}{4.067564in}}%
\pgfpathlineto{\pgfqpoint{12.701970in}{4.069183in}}%
\pgfpathlineto{\pgfqpoint{12.708011in}{4.071206in}}%
\pgfpathlineto{\pgfqpoint{12.710024in}{4.066215in}}%
\pgfpathlineto{\pgfqpoint{12.712037in}{4.070397in}}%
\pgfpathlineto{\pgfqpoint{12.716064in}{4.067834in}}%
\pgfpathlineto{\pgfqpoint{12.722105in}{4.070667in}}%
\pgfpathlineto{\pgfqpoint{12.724118in}{4.073365in}}%
\pgfpathlineto{\pgfqpoint{12.726132in}{4.072960in}}%
\pgfpathlineto{\pgfqpoint{12.728145in}{4.075253in}}%
\pgfpathlineto{\pgfqpoint{12.730158in}{4.079300in}}%
\pgfpathlineto{\pgfqpoint{12.736199in}{4.080514in}}%
\pgfpathlineto{\pgfqpoint{12.738212in}{4.081998in}}%
\pgfpathlineto{\pgfqpoint{12.740226in}{4.081323in}}%
\pgfpathlineto{\pgfqpoint{12.742239in}{4.078895in}}%
\pgfpathlineto{\pgfqpoint{12.744253in}{4.078895in}}%
\pgfpathlineto{\pgfqpoint{12.754320in}{4.073769in}}%
\pgfpathlineto{\pgfqpoint{12.756333in}{4.068913in}}%
\pgfpathlineto{\pgfqpoint{12.764387in}{4.071881in}}%
\pgfpathlineto{\pgfqpoint{12.766401in}{4.075523in}}%
\pgfpathlineto{\pgfqpoint{12.768414in}{4.077277in}}%
\pgfpathlineto{\pgfqpoint{12.770427in}{4.080784in}}%
\pgfpathlineto{\pgfqpoint{12.780495in}{4.089552in}}%
\pgfpathlineto{\pgfqpoint{12.782508in}{4.090361in}}%
\pgfpathlineto{\pgfqpoint{12.784522in}{4.096702in}}%
\pgfpathlineto{\pgfqpoint{12.786535in}{4.080379in}}%
\pgfpathlineto{\pgfqpoint{12.792575in}{4.082403in}}%
\pgfpathlineto{\pgfqpoint{12.794589in}{4.090766in}}%
\pgfpathlineto{\pgfqpoint{12.796602in}{4.094408in}}%
\pgfpathlineto{\pgfqpoint{12.798616in}{4.092790in}}%
\pgfpathlineto{\pgfqpoint{12.800629in}{4.092385in}}%
\pgfpathlineto{\pgfqpoint{12.806669in}{4.088743in}}%
\pgfpathlineto{\pgfqpoint{12.808683in}{4.086584in}}%
\pgfpathlineto{\pgfqpoint{12.812710in}{4.077951in}}%
\pgfpathlineto{\pgfqpoint{12.814723in}{4.075793in}}%
\pgfpathlineto{\pgfqpoint{12.820764in}{4.077142in}}%
\pgfpathlineto{\pgfqpoint{12.822777in}{4.079570in}}%
\pgfpathlineto{\pgfqpoint{12.824790in}{4.069588in}}%
\pgfpathlineto{\pgfqpoint{12.828817in}{4.074174in}}%
\pgfpathlineto{\pgfqpoint{12.836871in}{4.080109in}}%
\pgfpathlineto{\pgfqpoint{12.838885in}{4.083482in}}%
\pgfpathlineto{\pgfqpoint{12.840898in}{4.085235in}}%
\pgfpathlineto{\pgfqpoint{12.842911in}{4.085235in}}%
\pgfpathlineto{\pgfqpoint{12.854992in}{4.083482in}}%
\pgfpathlineto{\pgfqpoint{12.857006in}{4.086045in}}%
\pgfpathlineto{\pgfqpoint{12.863046in}{4.086180in}}%
\pgfpathlineto{\pgfqpoint{12.865059in}{4.083617in}}%
\pgfpathlineto{\pgfqpoint{12.867073in}{4.085235in}}%
\pgfpathlineto{\pgfqpoint{12.869086in}{4.088068in}}%
\pgfpathlineto{\pgfqpoint{12.871100in}{4.078221in}}%
\pgfpathlineto{\pgfqpoint{12.877140in}{4.078356in}}%
\pgfpathlineto{\pgfqpoint{12.879154in}{4.080379in}}%
\pgfpathlineto{\pgfqpoint{12.881167in}{4.075793in}}%
\pgfpathlineto{\pgfqpoint{12.883180in}{4.073095in}}%
\pgfpathlineto{\pgfqpoint{12.885194in}{4.071746in}}%
\pgfpathlineto{\pgfqpoint{12.891234in}{4.075658in}}%
\pgfpathlineto{\pgfqpoint{12.893248in}{4.067294in}}%
\pgfpathlineto{\pgfqpoint{12.897275in}{4.060819in}}%
\pgfpathlineto{\pgfqpoint{12.899288in}{4.058661in}}%
\pgfpathlineto{\pgfqpoint{12.905328in}{4.057177in}}%
\pgfpathlineto{\pgfqpoint{12.907342in}{4.051781in}}%
\pgfpathlineto{\pgfqpoint{12.909355in}{4.058796in}}%
\pgfpathlineto{\pgfqpoint{12.911369in}{4.050297in}}%
\pgfpathlineto{\pgfqpoint{12.913382in}{4.052860in}}%
\pgfpathlineto{\pgfqpoint{12.919422in}{4.049353in}}%
\pgfpathlineto{\pgfqpoint{12.923449in}{4.060549in}}%
\pgfpathlineto{\pgfqpoint{12.925463in}{4.051511in}}%
\pgfpathlineto{\pgfqpoint{12.927476in}{4.054749in}}%
\pgfpathlineto{\pgfqpoint{12.933517in}{4.051781in}}%
\pgfpathlineto{\pgfqpoint{12.937544in}{4.059470in}}%
\pgfpathlineto{\pgfqpoint{12.939557in}{4.059335in}}%
\pgfpathlineto{\pgfqpoint{12.941570in}{4.065001in}}%
\pgfpathlineto{\pgfqpoint{12.947611in}{4.062303in}}%
\pgfpathlineto{\pgfqpoint{12.951638in}{4.063382in}}%
\pgfpathlineto{\pgfqpoint{12.953651in}{4.065810in}}%
\pgfpathlineto{\pgfqpoint{12.955665in}{4.065541in}}%
\pgfpathlineto{\pgfqpoint{12.961705in}{4.062708in}}%
\pgfpathlineto{\pgfqpoint{12.963718in}{4.064866in}}%
\pgfpathlineto{\pgfqpoint{12.965732in}{4.065810in}}%
\pgfpathlineto{\pgfqpoint{12.967745in}{4.068643in}}%
\pgfpathlineto{\pgfqpoint{12.969759in}{4.073095in}}%
\pgfpathlineto{\pgfqpoint{12.975799in}{4.075118in}}%
\pgfpathlineto{\pgfqpoint{12.977812in}{4.086315in}}%
\pgfpathlineto{\pgfqpoint{12.979826in}{4.090092in}}%
\pgfpathlineto{\pgfqpoint{12.981839in}{4.091710in}}%
\pgfpathlineto{\pgfqpoint{12.983853in}{4.089282in}}%
\pgfpathlineto{\pgfqpoint{12.989893in}{4.090901in}}%
\pgfpathlineto{\pgfqpoint{12.991907in}{4.090631in}}%
\pgfpathlineto{\pgfqpoint{12.993920in}{4.092925in}}%
\pgfpathlineto{\pgfqpoint{12.995933in}{4.087124in}}%
\pgfpathlineto{\pgfqpoint{12.997947in}{4.083617in}}%
\pgfpathlineto{\pgfqpoint{13.003987in}{4.089687in}}%
\pgfpathlineto{\pgfqpoint{13.006001in}{4.085370in}}%
\pgfpathlineto{\pgfqpoint{13.008014in}{4.082942in}}%
\pgfpathlineto{\pgfqpoint{13.010028in}{4.074579in}}%
\pgfpathlineto{\pgfqpoint{13.012041in}{4.072555in}}%
\pgfpathlineto{\pgfqpoint{13.018081in}{4.071341in}}%
\pgfpathlineto{\pgfqpoint{13.020095in}{4.067969in}}%
\pgfpathlineto{\pgfqpoint{13.026135in}{4.068104in}}%
\pgfpathlineto{\pgfqpoint{13.032176in}{4.067969in}}%
\pgfpathlineto{\pgfqpoint{13.034189in}{4.068913in}}%
\pgfpathlineto{\pgfqpoint{13.038216in}{4.073365in}}%
\pgfpathlineto{\pgfqpoint{13.040229in}{4.073634in}}%
\pgfpathlineto{\pgfqpoint{13.048283in}{4.072690in}}%
\pgfpathlineto{\pgfqpoint{13.050297in}{4.082268in}}%
\pgfpathlineto{\pgfqpoint{13.052310in}{4.079435in}}%
\pgfpathlineto{\pgfqpoint{13.054323in}{4.074848in}}%
\pgfpathlineto{\pgfqpoint{13.060364in}{4.082403in}}%
\pgfpathlineto{\pgfqpoint{13.064391in}{4.089552in}}%
\pgfpathlineto{\pgfqpoint{13.066404in}{4.091576in}}%
\pgfpathlineto{\pgfqpoint{13.068418in}{4.098320in}}%
\pgfpathlineto{\pgfqpoint{13.074458in}{4.098320in}}%
\pgfpathlineto{\pgfqpoint{13.076471in}{4.101288in}}%
\pgfpathlineto{\pgfqpoint{13.078485in}{4.099130in}}%
\pgfpathlineto{\pgfqpoint{13.080498in}{4.100883in}}%
\pgfpathlineto{\pgfqpoint{13.088552in}{4.100344in}}%
\pgfpathlineto{\pgfqpoint{13.090566in}{4.104391in}}%
\pgfpathlineto{\pgfqpoint{13.092579in}{4.105200in}}%
\pgfpathlineto{\pgfqpoint{13.102646in}{4.124625in}}%
\pgfpathlineto{\pgfqpoint{13.104660in}{4.129077in}}%
\pgfpathlineto{\pgfqpoint{13.106673in}{4.128537in}}%
\pgfpathlineto{\pgfqpoint{13.108687in}{4.130965in}}%
\pgfpathlineto{\pgfqpoint{13.116740in}{4.135282in}}%
\pgfpathlineto{\pgfqpoint{13.118754in}{4.132449in}}%
\pgfpathlineto{\pgfqpoint{13.120767in}{4.127997in}}%
\pgfpathlineto{\pgfqpoint{13.122781in}{4.126514in}}%
\pgfpathlineto{\pgfqpoint{13.124794in}{4.132719in}}%
\pgfpathlineto{\pgfqpoint{13.132848in}{4.134338in}}%
\pgfpathlineto{\pgfqpoint{13.134861in}{4.140273in}}%
\pgfpathlineto{\pgfqpoint{13.136875in}{4.138250in}}%
\pgfpathlineto{\pgfqpoint{13.138888in}{4.142566in}}%
\pgfpathlineto{\pgfqpoint{13.146942in}{4.149311in}}%
\pgfpathlineto{\pgfqpoint{13.148955in}{4.147153in}}%
\pgfpathlineto{\pgfqpoint{13.150969in}{4.154437in}}%
\pgfpathlineto{\pgfqpoint{13.152982in}{4.193827in}}%
\pgfpathlineto{\pgfqpoint{13.159023in}{4.193422in}}%
\pgfpathlineto{\pgfqpoint{13.163050in}{4.223773in}}%
\pgfpathlineto{\pgfqpoint{13.165063in}{4.228765in}}%
\pgfpathlineto{\pgfqpoint{13.167077in}{4.218917in}}%
\pgfpathlineto{\pgfqpoint{13.173117in}{4.227146in}}%
\pgfpathlineto{\pgfqpoint{13.175130in}{4.228225in}}%
\pgfpathlineto{\pgfqpoint{13.177144in}{4.227146in}}%
\pgfpathlineto{\pgfqpoint{13.179157in}{4.222020in}}%
\pgfpathlineto{\pgfqpoint{13.181171in}{4.212577in}}%
\pgfpathlineto{\pgfqpoint{13.189224in}{4.216219in}}%
\pgfpathlineto{\pgfqpoint{13.191238in}{4.211093in}}%
\pgfpathlineto{\pgfqpoint{13.193251in}{4.213521in}}%
\pgfpathlineto{\pgfqpoint{13.195265in}{4.200302in}}%
\pgfpathlineto{\pgfqpoint{13.201305in}{4.200167in}}%
\pgfpathlineto{\pgfqpoint{13.203319in}{4.204348in}}%
\pgfpathlineto{\pgfqpoint{13.205332in}{4.200571in}}%
\pgfpathlineto{\pgfqpoint{13.209359in}{4.201785in}}%
\pgfpathlineto{\pgfqpoint{13.215399in}{4.198413in}}%
\pgfpathlineto{\pgfqpoint{13.217413in}{4.201516in}}%
\pgfpathlineto{\pgfqpoint{13.219426in}{4.191398in}}%
\pgfpathlineto{\pgfqpoint{13.221440in}{4.203000in}}%
\pgfpathlineto{\pgfqpoint{13.223453in}{4.200976in}}%
\pgfpathlineto{\pgfqpoint{13.229493in}{4.198413in}}%
\pgfpathlineto{\pgfqpoint{13.231507in}{4.184789in}}%
\pgfpathlineto{\pgfqpoint{13.233520in}{4.184923in}}%
\pgfpathlineto{\pgfqpoint{13.235534in}{4.180067in}}%
\pgfpathlineto{\pgfqpoint{13.237547in}{4.183575in}}%
\pgfpathlineto{\pgfqpoint{13.243587in}{4.187621in}}%
\pgfpathlineto{\pgfqpoint{13.245601in}{4.183305in}}%
\pgfpathlineto{\pgfqpoint{13.247614in}{4.183440in}}%
\pgfpathlineto{\pgfqpoint{13.249628in}{4.182495in}}%
\pgfpathlineto{\pgfqpoint{13.251641in}{4.199357in}}%
\pgfpathlineto{\pgfqpoint{13.257682in}{4.221480in}}%
\pgfpathlineto{\pgfqpoint{13.259695in}{4.231597in}}%
\pgfpathlineto{\pgfqpoint{13.261709in}{4.238342in}}%
\pgfpathlineto{\pgfqpoint{13.263722in}{4.227955in}}%
\pgfpathlineto{\pgfqpoint{13.265735in}{4.227146in}}%
\pgfpathlineto{\pgfqpoint{13.273789in}{4.219052in}}%
\pgfpathlineto{\pgfqpoint{13.275803in}{4.219457in}}%
\pgfpathlineto{\pgfqpoint{13.277816in}{4.220941in}}%
\pgfpathlineto{\pgfqpoint{13.279830in}{4.220131in}}%
\pgfpathlineto{\pgfqpoint{13.279830in}{4.220131in}}%
\pgfusepath{stroke}%
\end{pgfscope}%
\begin{pgfscope}%
\pgfsetrectcap%
\pgfsetmiterjoin%
\pgfsetlinewidth{0.803000pt}%
\definecolor{currentstroke}{rgb}{1.000000,1.000000,1.000000}%
\pgfsetstrokecolor{currentstroke}%
\pgfsetdash{}{0pt}%
\pgfpathmoveto{\pgfqpoint{8.656250in}{3.814412in}}%
\pgfpathlineto{\pgfqpoint{8.656250in}{4.258529in}}%
\pgfusepath{stroke}%
\end{pgfscope}%
\begin{pgfscope}%
\pgfsetrectcap%
\pgfsetmiterjoin%
\pgfsetlinewidth{0.803000pt}%
\definecolor{currentstroke}{rgb}{1.000000,1.000000,1.000000}%
\pgfsetstrokecolor{currentstroke}%
\pgfsetdash{}{0pt}%
\pgfpathmoveto{\pgfqpoint{13.500000in}{3.814412in}}%
\pgfpathlineto{\pgfqpoint{13.500000in}{4.258529in}}%
\pgfusepath{stroke}%
\end{pgfscope}%
\begin{pgfscope}%
\pgfsetrectcap%
\pgfsetmiterjoin%
\pgfsetlinewidth{0.803000pt}%
\definecolor{currentstroke}{rgb}{1.000000,1.000000,1.000000}%
\pgfsetstrokecolor{currentstroke}%
\pgfsetdash{}{0pt}%
\pgfpathmoveto{\pgfqpoint{8.656250in}{3.814412in}}%
\pgfpathlineto{\pgfqpoint{13.500000in}{3.814412in}}%
\pgfusepath{stroke}%
\end{pgfscope}%
\begin{pgfscope}%
\pgfsetrectcap%
\pgfsetmiterjoin%
\pgfsetlinewidth{0.803000pt}%
\definecolor{currentstroke}{rgb}{1.000000,1.000000,1.000000}%
\pgfsetstrokecolor{currentstroke}%
\pgfsetdash{}{0pt}%
\pgfpathmoveto{\pgfqpoint{8.656250in}{4.258529in}}%
\pgfpathlineto{\pgfqpoint{13.500000in}{4.258529in}}%
\pgfusepath{stroke}%
\end{pgfscope}%
\begin{pgfscope}%
\definecolor{textcolor}{rgb}{0.150000,0.150000,0.150000}%
\pgfsetstrokecolor{textcolor}%
\pgfsetfillcolor{textcolor}%
\pgftext[x=11.078125in,y=4.341863in,,base]{\color{textcolor}\rmfamily\fontsize{16.800000}{20.160000}\selectfont INTC}%
\end{pgfscope}%
\begin{pgfscope}%
\pgfsetbuttcap%
\pgfsetmiterjoin%
\definecolor{currentfill}{rgb}{0.917647,0.917647,0.949020}%
\pgfsetfillcolor{currentfill}%
\pgfsetlinewidth{0.000000pt}%
\definecolor{currentstroke}{rgb}{0.000000,0.000000,0.000000}%
\pgfsetstrokecolor{currentstroke}%
\pgfsetstrokeopacity{0.000000}%
\pgfsetdash{}{0pt}%
\pgfpathmoveto{\pgfqpoint{1.875000in}{2.792941in}}%
\pgfpathlineto{\pgfqpoint{6.718750in}{2.792941in}}%
\pgfpathlineto{\pgfqpoint{6.718750in}{3.237059in}}%
\pgfpathlineto{\pgfqpoint{1.875000in}{3.237059in}}%
\pgfpathclose%
\pgfusepath{fill}%
\end{pgfscope}%
\begin{pgfscope}%
\pgfpathrectangle{\pgfqpoint{1.875000in}{2.792941in}}{\pgfqpoint{4.843750in}{0.444118in}}%
\pgfusepath{clip}%
\pgfsetroundcap%
\pgfsetroundjoin%
\pgfsetlinewidth{0.803000pt}%
\definecolor{currentstroke}{rgb}{1.000000,1.000000,1.000000}%
\pgfsetstrokecolor{currentstroke}%
\pgfsetdash{}{0pt}%
\pgfpathmoveto{\pgfqpoint{2.091144in}{2.792941in}}%
\pgfpathlineto{\pgfqpoint{2.091144in}{3.237059in}}%
\pgfusepath{stroke}%
\end{pgfscope}%
\begin{pgfscope}%
\definecolor{textcolor}{rgb}{0.150000,0.150000,0.150000}%
\pgfsetstrokecolor{textcolor}%
\pgfsetfillcolor{textcolor}%
\pgftext[x=2.091144in,y=2.695719in,,top]{\color{textcolor}\rmfamily\fontsize{14.000000}{16.800000}\selectfont 2012}%
\end{pgfscope}%
\begin{pgfscope}%
\pgfpathrectangle{\pgfqpoint{1.875000in}{2.792941in}}{\pgfqpoint{4.843750in}{0.444118in}}%
\pgfusepath{clip}%
\pgfsetroundcap%
\pgfsetroundjoin%
\pgfsetlinewidth{0.803000pt}%
\definecolor{currentstroke}{rgb}{1.000000,1.000000,1.000000}%
\pgfsetstrokecolor{currentstroke}%
\pgfsetdash{}{0pt}%
\pgfpathmoveto{\pgfqpoint{2.828065in}{2.792941in}}%
\pgfpathlineto{\pgfqpoint{2.828065in}{3.237059in}}%
\pgfusepath{stroke}%
\end{pgfscope}%
\begin{pgfscope}%
\definecolor{textcolor}{rgb}{0.150000,0.150000,0.150000}%
\pgfsetstrokecolor{textcolor}%
\pgfsetfillcolor{textcolor}%
\pgftext[x=2.828065in,y=2.695719in,,top]{\color{textcolor}\rmfamily\fontsize{14.000000}{16.800000}\selectfont 2013}%
\end{pgfscope}%
\begin{pgfscope}%
\pgfpathrectangle{\pgfqpoint{1.875000in}{2.792941in}}{\pgfqpoint{4.843750in}{0.444118in}}%
\pgfusepath{clip}%
\pgfsetroundcap%
\pgfsetroundjoin%
\pgfsetlinewidth{0.803000pt}%
\definecolor{currentstroke}{rgb}{1.000000,1.000000,1.000000}%
\pgfsetstrokecolor{currentstroke}%
\pgfsetdash{}{0pt}%
\pgfpathmoveto{\pgfqpoint{3.562973in}{2.792941in}}%
\pgfpathlineto{\pgfqpoint{3.562973in}{3.237059in}}%
\pgfusepath{stroke}%
\end{pgfscope}%
\begin{pgfscope}%
\definecolor{textcolor}{rgb}{0.150000,0.150000,0.150000}%
\pgfsetstrokecolor{textcolor}%
\pgfsetfillcolor{textcolor}%
\pgftext[x=3.562973in,y=2.695719in,,top]{\color{textcolor}\rmfamily\fontsize{14.000000}{16.800000}\selectfont 2014}%
\end{pgfscope}%
\begin{pgfscope}%
\pgfpathrectangle{\pgfqpoint{1.875000in}{2.792941in}}{\pgfqpoint{4.843750in}{0.444118in}}%
\pgfusepath{clip}%
\pgfsetroundcap%
\pgfsetroundjoin%
\pgfsetlinewidth{0.803000pt}%
\definecolor{currentstroke}{rgb}{1.000000,1.000000,1.000000}%
\pgfsetstrokecolor{currentstroke}%
\pgfsetdash{}{0pt}%
\pgfpathmoveto{\pgfqpoint{4.297882in}{2.792941in}}%
\pgfpathlineto{\pgfqpoint{4.297882in}{3.237059in}}%
\pgfusepath{stroke}%
\end{pgfscope}%
\begin{pgfscope}%
\definecolor{textcolor}{rgb}{0.150000,0.150000,0.150000}%
\pgfsetstrokecolor{textcolor}%
\pgfsetfillcolor{textcolor}%
\pgftext[x=4.297882in,y=2.695719in,,top]{\color{textcolor}\rmfamily\fontsize{14.000000}{16.800000}\selectfont 2015}%
\end{pgfscope}%
\begin{pgfscope}%
\pgfpathrectangle{\pgfqpoint{1.875000in}{2.792941in}}{\pgfqpoint{4.843750in}{0.444118in}}%
\pgfusepath{clip}%
\pgfsetroundcap%
\pgfsetroundjoin%
\pgfsetlinewidth{0.803000pt}%
\definecolor{currentstroke}{rgb}{1.000000,1.000000,1.000000}%
\pgfsetstrokecolor{currentstroke}%
\pgfsetdash{}{0pt}%
\pgfpathmoveto{\pgfqpoint{5.032790in}{2.792941in}}%
\pgfpathlineto{\pgfqpoint{5.032790in}{3.237059in}}%
\pgfusepath{stroke}%
\end{pgfscope}%
\begin{pgfscope}%
\definecolor{textcolor}{rgb}{0.150000,0.150000,0.150000}%
\pgfsetstrokecolor{textcolor}%
\pgfsetfillcolor{textcolor}%
\pgftext[x=5.032790in,y=2.695719in,,top]{\color{textcolor}\rmfamily\fontsize{14.000000}{16.800000}\selectfont 2016}%
\end{pgfscope}%
\begin{pgfscope}%
\pgfpathrectangle{\pgfqpoint{1.875000in}{2.792941in}}{\pgfqpoint{4.843750in}{0.444118in}}%
\pgfusepath{clip}%
\pgfsetroundcap%
\pgfsetroundjoin%
\pgfsetlinewidth{0.803000pt}%
\definecolor{currentstroke}{rgb}{1.000000,1.000000,1.000000}%
\pgfsetstrokecolor{currentstroke}%
\pgfsetdash{}{0pt}%
\pgfpathmoveto{\pgfqpoint{5.769712in}{2.792941in}}%
\pgfpathlineto{\pgfqpoint{5.769712in}{3.237059in}}%
\pgfusepath{stroke}%
\end{pgfscope}%
\begin{pgfscope}%
\definecolor{textcolor}{rgb}{0.150000,0.150000,0.150000}%
\pgfsetstrokecolor{textcolor}%
\pgfsetfillcolor{textcolor}%
\pgftext[x=5.769712in,y=2.695719in,,top]{\color{textcolor}\rmfamily\fontsize{14.000000}{16.800000}\selectfont 2017}%
\end{pgfscope}%
\begin{pgfscope}%
\pgfpathrectangle{\pgfqpoint{1.875000in}{2.792941in}}{\pgfqpoint{4.843750in}{0.444118in}}%
\pgfusepath{clip}%
\pgfsetroundcap%
\pgfsetroundjoin%
\pgfsetlinewidth{0.803000pt}%
\definecolor{currentstroke}{rgb}{1.000000,1.000000,1.000000}%
\pgfsetstrokecolor{currentstroke}%
\pgfsetdash{}{0pt}%
\pgfpathmoveto{\pgfqpoint{6.504620in}{2.792941in}}%
\pgfpathlineto{\pgfqpoint{6.504620in}{3.237059in}}%
\pgfusepath{stroke}%
\end{pgfscope}%
\begin{pgfscope}%
\definecolor{textcolor}{rgb}{0.150000,0.150000,0.150000}%
\pgfsetstrokecolor{textcolor}%
\pgfsetfillcolor{textcolor}%
\pgftext[x=6.504620in,y=2.695719in,,top]{\color{textcolor}\rmfamily\fontsize{14.000000}{16.800000}\selectfont 2018}%
\end{pgfscope}%
\begin{pgfscope}%
\pgfpathrectangle{\pgfqpoint{1.875000in}{2.792941in}}{\pgfqpoint{4.843750in}{0.444118in}}%
\pgfusepath{clip}%
\pgfsetroundcap%
\pgfsetroundjoin%
\pgfsetlinewidth{0.803000pt}%
\definecolor{currentstroke}{rgb}{1.000000,1.000000,1.000000}%
\pgfsetstrokecolor{currentstroke}%
\pgfsetdash{}{0pt}%
\pgfpathmoveto{\pgfqpoint{1.875000in}{2.809957in}}%
\pgfpathlineto{\pgfqpoint{6.718750in}{2.809957in}}%
\pgfusepath{stroke}%
\end{pgfscope}%
\begin{pgfscope}%
\definecolor{textcolor}{rgb}{0.150000,0.150000,0.150000}%
\pgfsetstrokecolor{textcolor}%
\pgfsetfillcolor{textcolor}%
\pgftext[x=1.530355in,y=2.736091in,left,base]{\color{textcolor}\rmfamily\fontsize{14.000000}{16.800000}\selectfont 50}%
\end{pgfscope}%
\begin{pgfscope}%
\pgfpathrectangle{\pgfqpoint{1.875000in}{2.792941in}}{\pgfqpoint{4.843750in}{0.444118in}}%
\pgfusepath{clip}%
\pgfsetroundcap%
\pgfsetroundjoin%
\pgfsetlinewidth{0.803000pt}%
\definecolor{currentstroke}{rgb}{1.000000,1.000000,1.000000}%
\pgfsetstrokecolor{currentstroke}%
\pgfsetdash{}{0pt}%
\pgfpathmoveto{\pgfqpoint{1.875000in}{3.043119in}}%
\pgfpathlineto{\pgfqpoint{6.718750in}{3.043119in}}%
\pgfusepath{stroke}%
\end{pgfscope}%
\begin{pgfscope}%
\definecolor{textcolor}{rgb}{0.150000,0.150000,0.150000}%
\pgfsetstrokecolor{textcolor}%
\pgfsetfillcolor{textcolor}%
\pgftext[x=1.406643in,y=2.969253in,left,base]{\color{textcolor}\rmfamily\fontsize{14.000000}{16.800000}\selectfont 100}%
\end{pgfscope}%
\begin{pgfscope}%
\pgfpathrectangle{\pgfqpoint{1.875000in}{2.792941in}}{\pgfqpoint{4.843750in}{0.444118in}}%
\pgfusepath{clip}%
\pgfsetroundcap%
\pgfsetroundjoin%
\pgfsetlinewidth{1.505625pt}%
\definecolor{currentstroke}{rgb}{0.121569,0.466667,0.705882}%
\pgfsetstrokecolor{currentstroke}%
\pgfsetdash{}{0pt}%
\pgfpathmoveto{\pgfqpoint{2.095170in}{2.824180in}}%
\pgfpathlineto{\pgfqpoint{2.097184in}{2.822688in}}%
\pgfpathlineto{\pgfqpoint{2.099197in}{2.822408in}}%
\pgfpathlineto{\pgfqpoint{2.101211in}{2.820263in}}%
\pgfpathlineto{\pgfqpoint{2.107251in}{2.820636in}}%
\pgfpathlineto{\pgfqpoint{2.109265in}{2.821662in}}%
\pgfpathlineto{\pgfqpoint{2.111278in}{2.821382in}}%
\pgfpathlineto{\pgfqpoint{2.115305in}{2.821849in}}%
\pgfpathlineto{\pgfqpoint{2.123359in}{2.821336in}}%
\pgfpathlineto{\pgfqpoint{2.125372in}{2.821942in}}%
\pgfpathlineto{\pgfqpoint{2.127386in}{2.821615in}}%
\pgfpathlineto{\pgfqpoint{2.129399in}{2.821895in}}%
\pgfpathlineto{\pgfqpoint{2.137453in}{2.820916in}}%
\pgfpathlineto{\pgfqpoint{2.139466in}{2.821709in}}%
\pgfpathlineto{\pgfqpoint{2.141480in}{2.823527in}}%
\pgfpathlineto{\pgfqpoint{2.143493in}{2.823014in}}%
\pgfpathlineto{\pgfqpoint{2.149534in}{2.823574in}}%
\pgfpathlineto{\pgfqpoint{2.151547in}{2.824320in}}%
\pgfpathlineto{\pgfqpoint{2.155574in}{2.823108in}}%
\pgfpathlineto{\pgfqpoint{2.157587in}{2.823294in}}%
\pgfpathlineto{\pgfqpoint{2.163628in}{2.821615in}}%
\pgfpathlineto{\pgfqpoint{2.167655in}{2.821802in}}%
\pgfpathlineto{\pgfqpoint{2.171681in}{2.819377in}}%
\pgfpathlineto{\pgfqpoint{2.181749in}{2.819564in}}%
\pgfpathlineto{\pgfqpoint{2.183762in}{2.820590in}}%
\pgfpathlineto{\pgfqpoint{2.185776in}{2.820869in}}%
\pgfpathlineto{\pgfqpoint{2.205910in}{2.820963in}}%
\pgfpathlineto{\pgfqpoint{2.207923in}{2.823714in}}%
\pgfpathlineto{\pgfqpoint{2.213964in}{2.822175in}}%
\pgfpathlineto{\pgfqpoint{2.220004in}{2.822688in}}%
\pgfpathlineto{\pgfqpoint{2.222018in}{2.820590in}}%
\pgfpathlineto{\pgfqpoint{2.224031in}{2.820403in}}%
\pgfpathlineto{\pgfqpoint{2.226045in}{2.822501in}}%
\pgfpathlineto{\pgfqpoint{2.228058in}{2.822082in}}%
\pgfpathlineto{\pgfqpoint{2.236112in}{2.824320in}}%
\pgfpathlineto{\pgfqpoint{2.238125in}{2.823341in}}%
\pgfpathlineto{\pgfqpoint{2.242152in}{2.823527in}}%
\pgfpathlineto{\pgfqpoint{2.248192in}{2.823854in}}%
\pgfpathlineto{\pgfqpoint{2.254233in}{2.821009in}}%
\pgfpathlineto{\pgfqpoint{2.256246in}{2.821336in}}%
\pgfpathlineto{\pgfqpoint{2.266313in}{2.825393in}}%
\pgfpathlineto{\pgfqpoint{2.268327in}{2.825113in}}%
\pgfpathlineto{\pgfqpoint{2.270340in}{2.826698in}}%
\pgfpathlineto{\pgfqpoint{2.276381in}{2.827631in}}%
\pgfpathlineto{\pgfqpoint{2.282421in}{2.824320in}}%
\pgfpathlineto{\pgfqpoint{2.290475in}{2.822735in}}%
\pgfpathlineto{\pgfqpoint{2.292488in}{2.820030in}}%
\pgfpathlineto{\pgfqpoint{2.296515in}{2.819843in}}%
\pgfpathlineto{\pgfqpoint{2.298529in}{2.817512in}}%
\pgfpathlineto{\pgfqpoint{2.306582in}{2.820077in}}%
\pgfpathlineto{\pgfqpoint{2.308596in}{2.816439in}}%
\pgfpathlineto{\pgfqpoint{2.310609in}{2.815600in}}%
\pgfpathlineto{\pgfqpoint{2.312623in}{2.818165in}}%
\pgfpathlineto{\pgfqpoint{2.318663in}{2.816859in}}%
\pgfpathlineto{\pgfqpoint{2.320677in}{2.818398in}}%
\pgfpathlineto{\pgfqpoint{2.322690in}{2.820869in}}%
\pgfpathlineto{\pgfqpoint{2.324703in}{2.822082in}}%
\pgfpathlineto{\pgfqpoint{2.338798in}{2.824320in}}%
\pgfpathlineto{\pgfqpoint{2.340811in}{2.822082in}}%
\pgfpathlineto{\pgfqpoint{2.346851in}{2.822315in}}%
\pgfpathlineto{\pgfqpoint{2.348865in}{2.822968in}}%
\pgfpathlineto{\pgfqpoint{2.350878in}{2.820310in}}%
\pgfpathlineto{\pgfqpoint{2.352892in}{2.821429in}}%
\pgfpathlineto{\pgfqpoint{2.354905in}{2.820543in}}%
\pgfpathlineto{\pgfqpoint{2.360945in}{2.819004in}}%
\pgfpathlineto{\pgfqpoint{2.362959in}{2.817792in}}%
\pgfpathlineto{\pgfqpoint{2.364972in}{2.818165in}}%
\pgfpathlineto{\pgfqpoint{2.368999in}{2.816812in}}%
\pgfpathlineto{\pgfqpoint{2.377053in}{2.817465in}}%
\pgfpathlineto{\pgfqpoint{2.379067in}{2.816486in}}%
\pgfpathlineto{\pgfqpoint{2.381080in}{2.818165in}}%
\pgfpathlineto{\pgfqpoint{2.383093in}{2.815926in}}%
\pgfpathlineto{\pgfqpoint{2.391147in}{2.816253in}}%
\pgfpathlineto{\pgfqpoint{2.393161in}{2.814760in}}%
\pgfpathlineto{\pgfqpoint{2.395174in}{2.815600in}}%
\pgfpathlineto{\pgfqpoint{2.397188in}{2.813128in}}%
\pgfpathlineto{\pgfqpoint{2.403228in}{2.815273in}}%
\pgfpathlineto{\pgfqpoint{2.405241in}{2.814760in}}%
\pgfpathlineto{\pgfqpoint{2.407255in}{2.817045in}}%
\pgfpathlineto{\pgfqpoint{2.409268in}{2.817045in}}%
\pgfpathlineto{\pgfqpoint{2.411282in}{2.817698in}}%
\pgfpathlineto{\pgfqpoint{2.417322in}{2.814434in}}%
\pgfpathlineto{\pgfqpoint{2.419335in}{2.818118in}}%
\pgfpathlineto{\pgfqpoint{2.423362in}{2.827165in}}%
\pgfpathlineto{\pgfqpoint{2.425376in}{2.829310in}}%
\pgfpathlineto{\pgfqpoint{2.431416in}{2.830429in}}%
\pgfpathlineto{\pgfqpoint{2.435443in}{2.833087in}}%
\pgfpathlineto{\pgfqpoint{2.437456in}{2.830755in}}%
\pgfpathlineto{\pgfqpoint{2.439470in}{2.831688in}}%
\pgfpathlineto{\pgfqpoint{2.447524in}{2.830895in}}%
\pgfpathlineto{\pgfqpoint{2.449537in}{2.832481in}}%
\pgfpathlineto{\pgfqpoint{2.451551in}{2.832854in}}%
\pgfpathlineto{\pgfqpoint{2.453564in}{2.835232in}}%
\pgfpathlineto{\pgfqpoint{2.461618in}{2.837097in}}%
\pgfpathlineto{\pgfqpoint{2.467658in}{2.835559in}}%
\pgfpathlineto{\pgfqpoint{2.477725in}{2.836538in}}%
\pgfpathlineto{\pgfqpoint{2.479739in}{2.835838in}}%
\pgfpathlineto{\pgfqpoint{2.481752in}{2.839242in}}%
\pgfpathlineto{\pgfqpoint{2.487793in}{2.838636in}}%
\pgfpathlineto{\pgfqpoint{2.491820in}{2.842180in}}%
\pgfpathlineto{\pgfqpoint{2.493833in}{2.842787in}}%
\pgfpathlineto{\pgfqpoint{2.495846in}{2.839336in}}%
\pgfpathlineto{\pgfqpoint{2.501887in}{2.837331in}}%
\pgfpathlineto{\pgfqpoint{2.503900in}{2.834439in}}%
\pgfpathlineto{\pgfqpoint{2.505914in}{2.835139in}}%
\pgfpathlineto{\pgfqpoint{2.507927in}{2.839755in}}%
\pgfpathlineto{\pgfqpoint{2.509941in}{2.842740in}}%
\pgfpathlineto{\pgfqpoint{2.515981in}{2.842460in}}%
\pgfpathlineto{\pgfqpoint{2.517994in}{2.841574in}}%
\pgfpathlineto{\pgfqpoint{2.520008in}{2.842227in}}%
\pgfpathlineto{\pgfqpoint{2.522021in}{2.838636in}}%
\pgfpathlineto{\pgfqpoint{2.524035in}{2.841201in}}%
\pgfpathlineto{\pgfqpoint{2.530075in}{2.840129in}}%
\pgfpathlineto{\pgfqpoint{2.532089in}{2.838030in}}%
\pgfpathlineto{\pgfqpoint{2.536115in}{2.838170in}}%
\pgfpathlineto{\pgfqpoint{2.538129in}{2.839382in}}%
\pgfpathlineto{\pgfqpoint{2.544169in}{2.838683in}}%
\pgfpathlineto{\pgfqpoint{2.546183in}{2.839382in}}%
\pgfpathlineto{\pgfqpoint{2.552223in}{2.836165in}}%
\pgfpathlineto{\pgfqpoint{2.564304in}{2.835932in}}%
\pgfpathlineto{\pgfqpoint{2.566317in}{2.837750in}}%
\pgfpathlineto{\pgfqpoint{2.576384in}{2.836864in}}%
\pgfpathlineto{\pgfqpoint{2.578398in}{2.836258in}}%
\pgfpathlineto{\pgfqpoint{2.580411in}{2.837097in}}%
\pgfpathlineto{\pgfqpoint{2.590478in}{2.836491in}}%
\pgfpathlineto{\pgfqpoint{2.592492in}{2.838683in}}%
\pgfpathlineto{\pgfqpoint{2.594505in}{2.838823in}}%
\pgfpathlineto{\pgfqpoint{2.602559in}{2.840082in}}%
\pgfpathlineto{\pgfqpoint{2.604573in}{2.839849in}}%
\pgfpathlineto{\pgfqpoint{2.606586in}{2.843113in}}%
\pgfpathlineto{\pgfqpoint{2.608599in}{2.841108in}}%
\pgfpathlineto{\pgfqpoint{2.614640in}{2.840268in}}%
\pgfpathlineto{\pgfqpoint{2.616653in}{2.841434in}}%
\pgfpathlineto{\pgfqpoint{2.618667in}{2.841621in}}%
\pgfpathlineto{\pgfqpoint{2.622694in}{2.843393in}}%
\pgfpathlineto{\pgfqpoint{2.628734in}{2.843160in}}%
\pgfpathlineto{\pgfqpoint{2.630747in}{2.844372in}}%
\pgfpathlineto{\pgfqpoint{2.632761in}{2.843160in}}%
\pgfpathlineto{\pgfqpoint{2.646855in}{2.843160in}}%
\pgfpathlineto{\pgfqpoint{2.650882in}{2.845678in}}%
\pgfpathlineto{\pgfqpoint{2.656922in}{2.844838in}}%
\pgfpathlineto{\pgfqpoint{2.658936in}{2.840875in}}%
\pgfpathlineto{\pgfqpoint{2.664976in}{2.839196in}}%
\pgfpathlineto{\pgfqpoint{2.671016in}{2.841621in}}%
\pgfpathlineto{\pgfqpoint{2.673030in}{2.845258in}}%
\pgfpathlineto{\pgfqpoint{2.677057in}{2.856730in}}%
\pgfpathlineto{\pgfqpoint{2.679070in}{2.854211in}}%
\pgfpathlineto{\pgfqpoint{2.685110in}{2.853885in}}%
\pgfpathlineto{\pgfqpoint{2.687124in}{2.850434in}}%
\pgfpathlineto{\pgfqpoint{2.689137in}{2.849875in}}%
\pgfpathlineto{\pgfqpoint{2.691151in}{2.851414in}}%
\pgfpathlineto{\pgfqpoint{2.693164in}{2.850481in}}%
\pgfpathlineto{\pgfqpoint{2.703232in}{2.850154in}}%
\pgfpathlineto{\pgfqpoint{2.705245in}{2.852813in}}%
\pgfpathlineto{\pgfqpoint{2.707258in}{2.850481in}}%
\pgfpathlineto{\pgfqpoint{2.713299in}{2.850061in}}%
\pgfpathlineto{\pgfqpoint{2.715312in}{2.850901in}}%
\pgfpathlineto{\pgfqpoint{2.719339in}{2.845678in}}%
\pgfpathlineto{\pgfqpoint{2.721353in}{2.846517in}}%
\pgfpathlineto{\pgfqpoint{2.729406in}{2.845118in}}%
\pgfpathlineto{\pgfqpoint{2.733433in}{2.843439in}}%
\pgfpathlineto{\pgfqpoint{2.735447in}{2.843906in}}%
\pgfpathlineto{\pgfqpoint{2.741487in}{2.844092in}}%
\pgfpathlineto{\pgfqpoint{2.743500in}{2.845724in}}%
\pgfpathlineto{\pgfqpoint{2.745514in}{2.845445in}}%
\pgfpathlineto{\pgfqpoint{2.749541in}{2.847683in}}%
\pgfpathlineto{\pgfqpoint{2.757595in}{2.844745in}}%
\pgfpathlineto{\pgfqpoint{2.759608in}{2.846657in}}%
\pgfpathlineto{\pgfqpoint{2.761621in}{2.846377in}}%
\pgfpathlineto{\pgfqpoint{2.763635in}{2.848336in}}%
\pgfpathlineto{\pgfqpoint{2.769675in}{2.848056in}}%
\pgfpathlineto{\pgfqpoint{2.773702in}{2.849268in}}%
\pgfpathlineto{\pgfqpoint{2.775716in}{2.849595in}}%
\pgfpathlineto{\pgfqpoint{2.777729in}{2.851134in}}%
\pgfpathlineto{\pgfqpoint{2.783769in}{2.851740in}}%
\pgfpathlineto{\pgfqpoint{2.785783in}{2.853699in}}%
\pgfpathlineto{\pgfqpoint{2.791823in}{2.852066in}}%
\pgfpathlineto{\pgfqpoint{2.799877in}{2.853092in}}%
\pgfpathlineto{\pgfqpoint{2.801890in}{2.851833in}}%
\pgfpathlineto{\pgfqpoint{2.803904in}{2.852300in}}%
\pgfpathlineto{\pgfqpoint{2.805917in}{2.850434in}}%
\pgfpathlineto{\pgfqpoint{2.811958in}{2.849455in}}%
\pgfpathlineto{\pgfqpoint{2.815985in}{2.850061in}}%
\pgfpathlineto{\pgfqpoint{2.817998in}{2.849735in}}%
\pgfpathlineto{\pgfqpoint{2.820011in}{2.847357in}}%
\pgfpathlineto{\pgfqpoint{2.826052in}{2.849781in}}%
\pgfpathlineto{\pgfqpoint{2.830079in}{2.852673in}}%
\pgfpathlineto{\pgfqpoint{2.832092in}{2.852300in}}%
\pgfpathlineto{\pgfqpoint{2.834106in}{2.855424in}}%
\pgfpathlineto{\pgfqpoint{2.842159in}{2.854911in}}%
\pgfpathlineto{\pgfqpoint{2.848200in}{2.858548in}}%
\pgfpathlineto{\pgfqpoint{2.854240in}{2.859388in}}%
\pgfpathlineto{\pgfqpoint{2.856254in}{2.858642in}}%
\pgfpathlineto{\pgfqpoint{2.860280in}{2.860693in}}%
\pgfpathlineto{\pgfqpoint{2.862294in}{2.861999in}}%
\pgfpathlineto{\pgfqpoint{2.870348in}{2.859854in}}%
\pgfpathlineto{\pgfqpoint{2.874375in}{2.861486in}}%
\pgfpathlineto{\pgfqpoint{2.876388in}{2.864657in}}%
\pgfpathlineto{\pgfqpoint{2.882428in}{2.863491in}}%
\pgfpathlineto{\pgfqpoint{2.884442in}{2.866569in}}%
\pgfpathlineto{\pgfqpoint{2.888469in}{2.864657in}}%
\pgfpathlineto{\pgfqpoint{2.890482in}{2.865683in}}%
\pgfpathlineto{\pgfqpoint{2.896522in}{2.865403in}}%
\pgfpathlineto{\pgfqpoint{2.900549in}{2.870393in}}%
\pgfpathlineto{\pgfqpoint{2.902563in}{2.869087in}}%
\pgfpathlineto{\pgfqpoint{2.904576in}{2.870719in}}%
\pgfpathlineto{\pgfqpoint{2.910617in}{2.870486in}}%
\pgfpathlineto{\pgfqpoint{2.912630in}{2.871978in}}%
\pgfpathlineto{\pgfqpoint{2.914643in}{2.871419in}}%
\pgfpathlineto{\pgfqpoint{2.916657in}{2.872025in}}%
\pgfpathlineto{\pgfqpoint{2.918670in}{2.873377in}}%
\pgfpathlineto{\pgfqpoint{2.926724in}{2.876502in}}%
\pgfpathlineto{\pgfqpoint{2.928738in}{2.875289in}}%
\pgfpathlineto{\pgfqpoint{2.930751in}{2.876175in}}%
\pgfpathlineto{\pgfqpoint{2.932765in}{2.876129in}}%
\pgfpathlineto{\pgfqpoint{2.938805in}{2.873424in}}%
\pgfpathlineto{\pgfqpoint{2.940818in}{2.874170in}}%
\pgfpathlineto{\pgfqpoint{2.942832in}{2.876409in}}%
\pgfpathlineto{\pgfqpoint{2.944845in}{2.875569in}}%
\pgfpathlineto{\pgfqpoint{2.946859in}{2.877901in}}%
\pgfpathlineto{\pgfqpoint{2.952899in}{2.879859in}}%
\pgfpathlineto{\pgfqpoint{2.954912in}{2.881631in}}%
\pgfpathlineto{\pgfqpoint{2.956926in}{2.880605in}}%
\pgfpathlineto{\pgfqpoint{2.960953in}{2.883730in}}%
\pgfpathlineto{\pgfqpoint{2.971020in}{2.885129in}}%
\pgfpathlineto{\pgfqpoint{2.973033in}{2.887320in}}%
\pgfpathlineto{\pgfqpoint{2.975047in}{2.887647in}}%
\pgfpathlineto{\pgfqpoint{2.981087in}{2.886155in}}%
\pgfpathlineto{\pgfqpoint{2.983101in}{2.886341in}}%
\pgfpathlineto{\pgfqpoint{2.985114in}{2.888673in}}%
\pgfpathlineto{\pgfqpoint{2.987128in}{2.886947in}}%
\pgfpathlineto{\pgfqpoint{2.989141in}{2.889792in}}%
\pgfpathlineto{\pgfqpoint{2.995181in}{2.889559in}}%
\pgfpathlineto{\pgfqpoint{2.997195in}{2.894175in}}%
\pgfpathlineto{\pgfqpoint{3.001222in}{2.896833in}}%
\pgfpathlineto{\pgfqpoint{3.009275in}{2.898419in}}%
\pgfpathlineto{\pgfqpoint{3.011289in}{2.901403in}}%
\pgfpathlineto{\pgfqpoint{3.013302in}{2.898979in}}%
\pgfpathlineto{\pgfqpoint{3.015316in}{2.900284in}}%
\pgfpathlineto{\pgfqpoint{3.017329in}{2.898839in}}%
\pgfpathlineto{\pgfqpoint{3.023370in}{2.895201in}}%
\pgfpathlineto{\pgfqpoint{3.031423in}{2.901590in}}%
\pgfpathlineto{\pgfqpoint{3.037464in}{2.897533in}}%
\pgfpathlineto{\pgfqpoint{3.039477in}{2.904341in}}%
\pgfpathlineto{\pgfqpoint{3.041491in}{2.906160in}}%
\pgfpathlineto{\pgfqpoint{3.043504in}{2.903315in}}%
\pgfpathlineto{\pgfqpoint{3.045518in}{2.908445in}}%
\pgfpathlineto{\pgfqpoint{3.051558in}{2.909797in}}%
\pgfpathlineto{\pgfqpoint{3.053571in}{2.912222in}}%
\pgfpathlineto{\pgfqpoint{3.055585in}{2.908072in}}%
\pgfpathlineto{\pgfqpoint{3.057598in}{2.911336in}}%
\pgfpathlineto{\pgfqpoint{3.059612in}{2.910916in}}%
\pgfpathlineto{\pgfqpoint{3.065652in}{2.912735in}}%
\pgfpathlineto{\pgfqpoint{3.067665in}{2.911383in}}%
\pgfpathlineto{\pgfqpoint{3.069679in}{2.907419in}}%
\pgfpathlineto{\pgfqpoint{3.071692in}{2.911103in}}%
\pgfpathlineto{\pgfqpoint{3.073706in}{2.913388in}}%
\pgfpathlineto{\pgfqpoint{3.079746in}{2.909191in}}%
\pgfpathlineto{\pgfqpoint{3.081760in}{2.912549in}}%
\pgfpathlineto{\pgfqpoint{3.083773in}{2.912269in}}%
\pgfpathlineto{\pgfqpoint{3.085786in}{2.911056in}}%
\pgfpathlineto{\pgfqpoint{3.087800in}{2.913435in}}%
\pgfpathlineto{\pgfqpoint{3.093840in}{2.913808in}}%
\pgfpathlineto{\pgfqpoint{3.095854in}{2.918191in}}%
\pgfpathlineto{\pgfqpoint{3.097867in}{2.920803in}}%
\pgfpathlineto{\pgfqpoint{3.099881in}{2.920056in}}%
\pgfpathlineto{\pgfqpoint{3.101894in}{2.922575in}}%
\pgfpathlineto{\pgfqpoint{3.107934in}{2.922295in}}%
\pgfpathlineto{\pgfqpoint{3.109948in}{2.924533in}}%
\pgfpathlineto{\pgfqpoint{3.111961in}{2.924067in}}%
\pgfpathlineto{\pgfqpoint{3.115988in}{2.920150in}}%
\pgfpathlineto{\pgfqpoint{3.124042in}{2.923274in}}%
\pgfpathlineto{\pgfqpoint{3.126055in}{2.915533in}}%
\pgfpathlineto{\pgfqpoint{3.128069in}{2.916885in}}%
\pgfpathlineto{\pgfqpoint{3.130082in}{2.909704in}}%
\pgfpathlineto{\pgfqpoint{3.136123in}{2.911802in}}%
\pgfpathlineto{\pgfqpoint{3.140150in}{2.907745in}}%
\pgfpathlineto{\pgfqpoint{3.142163in}{2.910823in}}%
\pgfpathlineto{\pgfqpoint{3.144176in}{2.912595in}}%
\pgfpathlineto{\pgfqpoint{3.150217in}{2.913528in}}%
\pgfpathlineto{\pgfqpoint{3.152230in}{2.911756in}}%
\pgfpathlineto{\pgfqpoint{3.154244in}{2.907979in}}%
\pgfpathlineto{\pgfqpoint{3.156257in}{2.912595in}}%
\pgfpathlineto{\pgfqpoint{3.158271in}{2.912595in}}%
\pgfpathlineto{\pgfqpoint{3.164311in}{2.915440in}}%
\pgfpathlineto{\pgfqpoint{3.166324in}{2.918331in}}%
\pgfpathlineto{\pgfqpoint{3.168338in}{2.912409in}}%
\pgfpathlineto{\pgfqpoint{3.170351in}{2.903549in}}%
\pgfpathlineto{\pgfqpoint{3.172365in}{2.905834in}}%
\pgfpathlineto{\pgfqpoint{3.178405in}{2.911429in}}%
\pgfpathlineto{\pgfqpoint{3.180419in}{2.914367in}}%
\pgfpathlineto{\pgfqpoint{3.182432in}{2.920849in}}%
\pgfpathlineto{\pgfqpoint{3.184445in}{2.919730in}}%
\pgfpathlineto{\pgfqpoint{3.186459in}{2.916372in}}%
\pgfpathlineto{\pgfqpoint{3.192499in}{2.919404in}}%
\pgfpathlineto{\pgfqpoint{3.194513in}{2.919170in}}%
\pgfpathlineto{\pgfqpoint{3.196526in}{2.920010in}}%
\pgfpathlineto{\pgfqpoint{3.200553in}{2.924300in}}%
\pgfpathlineto{\pgfqpoint{3.208607in}{2.928310in}}%
\pgfpathlineto{\pgfqpoint{3.214647in}{2.932694in}}%
\pgfpathlineto{\pgfqpoint{3.222701in}{2.934326in}}%
\pgfpathlineto{\pgfqpoint{3.224714in}{2.933253in}}%
\pgfpathlineto{\pgfqpoint{3.226728in}{2.933440in}}%
\pgfpathlineto{\pgfqpoint{3.228741in}{2.941554in}}%
\pgfpathlineto{\pgfqpoint{3.238808in}{2.942067in}}%
\pgfpathlineto{\pgfqpoint{3.242835in}{2.943932in}}%
\pgfpathlineto{\pgfqpoint{3.248876in}{2.945424in}}%
\pgfpathlineto{\pgfqpoint{3.250889in}{2.945285in}}%
\pgfpathlineto{\pgfqpoint{3.254916in}{2.947663in}}%
\pgfpathlineto{\pgfqpoint{3.256930in}{2.950088in}}%
\pgfpathlineto{\pgfqpoint{3.262970in}{2.947756in}}%
\pgfpathlineto{\pgfqpoint{3.264983in}{2.947803in}}%
\pgfpathlineto{\pgfqpoint{3.266997in}{2.947243in}}%
\pgfpathlineto{\pgfqpoint{3.269010in}{2.945937in}}%
\pgfpathlineto{\pgfqpoint{3.271024in}{2.942067in}}%
\pgfpathlineto{\pgfqpoint{3.277064in}{2.940668in}}%
\pgfpathlineto{\pgfqpoint{3.279077in}{2.944632in}}%
\pgfpathlineto{\pgfqpoint{3.281091in}{2.935492in}}%
\pgfpathlineto{\pgfqpoint{3.283104in}{2.930968in}}%
\pgfpathlineto{\pgfqpoint{3.285118in}{2.930269in}}%
\pgfpathlineto{\pgfqpoint{3.291158in}{2.934512in}}%
\pgfpathlineto{\pgfqpoint{3.297198in}{2.925885in}}%
\pgfpathlineto{\pgfqpoint{3.299212in}{2.929103in}}%
\pgfpathlineto{\pgfqpoint{3.305252in}{2.925606in}}%
\pgfpathlineto{\pgfqpoint{3.307266in}{2.920150in}}%
\pgfpathlineto{\pgfqpoint{3.309279in}{2.921595in}}%
\pgfpathlineto{\pgfqpoint{3.311293in}{2.921782in}}%
\pgfpathlineto{\pgfqpoint{3.313306in}{2.921129in}}%
\pgfpathlineto{\pgfqpoint{3.321360in}{2.921176in}}%
\pgfpathlineto{\pgfqpoint{3.323373in}{2.923088in}}%
\pgfpathlineto{\pgfqpoint{3.333441in}{2.925699in}}%
\pgfpathlineto{\pgfqpoint{3.337467in}{2.932367in}}%
\pgfpathlineto{\pgfqpoint{3.339481in}{2.931481in}}%
\pgfpathlineto{\pgfqpoint{3.341494in}{2.929709in}}%
\pgfpathlineto{\pgfqpoint{3.349548in}{2.931668in}}%
\pgfpathlineto{\pgfqpoint{3.351562in}{2.935072in}}%
\pgfpathlineto{\pgfqpoint{3.353575in}{2.935725in}}%
\pgfpathlineto{\pgfqpoint{3.355588in}{2.934139in}}%
\pgfpathlineto{\pgfqpoint{3.361629in}{2.931808in}}%
\pgfpathlineto{\pgfqpoint{3.363642in}{2.928357in}}%
\pgfpathlineto{\pgfqpoint{3.365656in}{2.923787in}}%
\pgfpathlineto{\pgfqpoint{3.367669in}{2.923740in}}%
\pgfpathlineto{\pgfqpoint{3.369683in}{2.922388in}}%
\pgfpathlineto{\pgfqpoint{3.375723in}{2.922248in}}%
\pgfpathlineto{\pgfqpoint{3.377736in}{2.925326in}}%
\pgfpathlineto{\pgfqpoint{3.379750in}{2.924626in}}%
\pgfpathlineto{\pgfqpoint{3.381763in}{2.921782in}}%
\pgfpathlineto{\pgfqpoint{3.383777in}{2.924720in}}%
\pgfpathlineto{\pgfqpoint{3.389817in}{2.921828in}}%
\pgfpathlineto{\pgfqpoint{3.391830in}{2.917911in}}%
\pgfpathlineto{\pgfqpoint{3.393844in}{2.919310in}}%
\pgfpathlineto{\pgfqpoint{3.397871in}{2.933253in}}%
\pgfpathlineto{\pgfqpoint{3.405925in}{2.935165in}}%
\pgfpathlineto{\pgfqpoint{3.409951in}{2.943279in}}%
\pgfpathlineto{\pgfqpoint{3.411965in}{2.941927in}}%
\pgfpathlineto{\pgfqpoint{3.418005in}{2.940202in}}%
\pgfpathlineto{\pgfqpoint{3.420019in}{2.944818in}}%
\pgfpathlineto{\pgfqpoint{3.422032in}{2.943792in}}%
\pgfpathlineto{\pgfqpoint{3.424046in}{2.944772in}}%
\pgfpathlineto{\pgfqpoint{3.426059in}{2.943746in}}%
\pgfpathlineto{\pgfqpoint{3.432099in}{2.944958in}}%
\pgfpathlineto{\pgfqpoint{3.434113in}{2.947943in}}%
\pgfpathlineto{\pgfqpoint{3.436126in}{2.946544in}}%
\pgfpathlineto{\pgfqpoint{3.438140in}{2.945844in}}%
\pgfpathlineto{\pgfqpoint{3.440153in}{2.948875in}}%
\pgfpathlineto{\pgfqpoint{3.448207in}{2.946637in}}%
\pgfpathlineto{\pgfqpoint{3.450220in}{2.947523in}}%
\pgfpathlineto{\pgfqpoint{3.452234in}{2.946124in}}%
\pgfpathlineto{\pgfqpoint{3.454247in}{2.951580in}}%
\pgfpathlineto{\pgfqpoint{3.460288in}{2.952513in}}%
\pgfpathlineto{\pgfqpoint{3.462301in}{2.949621in}}%
\pgfpathlineto{\pgfqpoint{3.464315in}{2.948735in}}%
\pgfpathlineto{\pgfqpoint{3.468341in}{2.952932in}}%
\pgfpathlineto{\pgfqpoint{3.474382in}{2.952559in}}%
\pgfpathlineto{\pgfqpoint{3.476395in}{2.954798in}}%
\pgfpathlineto{\pgfqpoint{3.478409in}{2.955963in}}%
\pgfpathlineto{\pgfqpoint{3.480422in}{2.956150in}}%
\pgfpathlineto{\pgfqpoint{3.482436in}{2.958994in}}%
\pgfpathlineto{\pgfqpoint{3.488476in}{2.960533in}}%
\pgfpathlineto{\pgfqpoint{3.490489in}{2.958248in}}%
\pgfpathlineto{\pgfqpoint{3.492503in}{2.957922in}}%
\pgfpathlineto{\pgfqpoint{3.504584in}{2.953865in}}%
\pgfpathlineto{\pgfqpoint{3.506597in}{2.952513in}}%
\pgfpathlineto{\pgfqpoint{3.508610in}{2.949855in}}%
\pgfpathlineto{\pgfqpoint{3.510624in}{2.955730in}}%
\pgfpathlineto{\pgfqpoint{3.516664in}{2.955730in}}%
\pgfpathlineto{\pgfqpoint{3.518678in}{2.954564in}}%
\pgfpathlineto{\pgfqpoint{3.520691in}{2.950414in}}%
\pgfpathlineto{\pgfqpoint{3.522705in}{2.942580in}}%
\pgfpathlineto{\pgfqpoint{3.524718in}{2.943326in}}%
\pgfpathlineto{\pgfqpoint{3.530758in}{2.943419in}}%
\pgfpathlineto{\pgfqpoint{3.532772in}{2.940575in}}%
\pgfpathlineto{\pgfqpoint{3.534785in}{2.948502in}}%
\pgfpathlineto{\pgfqpoint{3.536799in}{2.945891in}}%
\pgfpathlineto{\pgfqpoint{3.538812in}{2.946310in}}%
\pgfpathlineto{\pgfqpoint{3.546866in}{2.946217in}}%
\pgfpathlineto{\pgfqpoint{3.550893in}{2.948129in}}%
\pgfpathlineto{\pgfqpoint{3.552906in}{2.947383in}}%
\pgfpathlineto{\pgfqpoint{3.558947in}{2.947150in}}%
\pgfpathlineto{\pgfqpoint{3.560960in}{2.944305in}}%
\pgfpathlineto{\pgfqpoint{3.564987in}{2.942067in}}%
\pgfpathlineto{\pgfqpoint{3.567000in}{2.945331in}}%
\pgfpathlineto{\pgfqpoint{3.573041in}{2.947290in}}%
\pgfpathlineto{\pgfqpoint{3.575054in}{2.955124in}}%
\pgfpathlineto{\pgfqpoint{3.577068in}{2.954611in}}%
\pgfpathlineto{\pgfqpoint{3.579081in}{2.956896in}}%
\pgfpathlineto{\pgfqpoint{3.581095in}{2.956943in}}%
\pgfpathlineto{\pgfqpoint{3.587135in}{2.956010in}}%
\pgfpathlineto{\pgfqpoint{3.591162in}{2.957176in}}%
\pgfpathlineto{\pgfqpoint{3.593175in}{2.956570in}}%
\pgfpathlineto{\pgfqpoint{3.595189in}{2.958248in}}%
\pgfpathlineto{\pgfqpoint{3.603242in}{2.954098in}}%
\pgfpathlineto{\pgfqpoint{3.605256in}{2.955264in}}%
\pgfpathlineto{\pgfqpoint{3.607269in}{2.948922in}}%
\pgfpathlineto{\pgfqpoint{3.609283in}{2.940388in}}%
\pgfpathlineto{\pgfqpoint{3.615323in}{2.937683in}}%
\pgfpathlineto{\pgfqpoint{3.617337in}{2.938336in}}%
\pgfpathlineto{\pgfqpoint{3.619350in}{2.933533in}}%
\pgfpathlineto{\pgfqpoint{3.621363in}{2.935911in}}%
\pgfpathlineto{\pgfqpoint{3.623377in}{2.931808in}}%
\pgfpathlineto{\pgfqpoint{3.629417in}{2.924999in}}%
\pgfpathlineto{\pgfqpoint{3.631431in}{2.924347in}}%
\pgfpathlineto{\pgfqpoint{3.633444in}{2.927005in}}%
\pgfpathlineto{\pgfqpoint{3.637471in}{2.938103in}}%
\pgfpathlineto{\pgfqpoint{3.643511in}{2.942207in}}%
\pgfpathlineto{\pgfqpoint{3.645525in}{2.949855in}}%
\pgfpathlineto{\pgfqpoint{3.647538in}{2.947663in}}%
\pgfpathlineto{\pgfqpoint{3.651565in}{2.949015in}}%
\pgfpathlineto{\pgfqpoint{3.659619in}{2.946637in}}%
\pgfpathlineto{\pgfqpoint{3.661632in}{2.944492in}}%
\pgfpathlineto{\pgfqpoint{3.663646in}{2.947476in}}%
\pgfpathlineto{\pgfqpoint{3.665659in}{2.946684in}}%
\pgfpathlineto{\pgfqpoint{3.673713in}{2.945005in}}%
\pgfpathlineto{\pgfqpoint{3.675727in}{2.945005in}}%
\pgfpathlineto{\pgfqpoint{3.677740in}{2.946031in}}%
\pgfpathlineto{\pgfqpoint{3.679753in}{2.949108in}}%
\pgfpathlineto{\pgfqpoint{3.685794in}{2.946823in}}%
\pgfpathlineto{\pgfqpoint{3.687807in}{2.954005in}}%
\pgfpathlineto{\pgfqpoint{3.689821in}{2.950974in}}%
\pgfpathlineto{\pgfqpoint{3.693848in}{2.953958in}}%
\pgfpathlineto{\pgfqpoint{3.703915in}{2.955077in}}%
\pgfpathlineto{\pgfqpoint{3.705928in}{2.952652in}}%
\pgfpathlineto{\pgfqpoint{3.707942in}{2.951906in}}%
\pgfpathlineto{\pgfqpoint{3.713982in}{2.956430in}}%
\pgfpathlineto{\pgfqpoint{3.715995in}{2.956430in}}%
\pgfpathlineto{\pgfqpoint{3.718009in}{2.955031in}}%
\pgfpathlineto{\pgfqpoint{3.720022in}{2.957176in}}%
\pgfpathlineto{\pgfqpoint{3.722036in}{2.964497in}}%
\pgfpathlineto{\pgfqpoint{3.728076in}{2.961559in}}%
\pgfpathlineto{\pgfqpoint{3.730090in}{2.970373in}}%
\pgfpathlineto{\pgfqpoint{3.732103in}{2.969020in}}%
\pgfpathlineto{\pgfqpoint{3.742170in}{2.973777in}}%
\pgfpathlineto{\pgfqpoint{3.744184in}{2.972611in}}%
\pgfpathlineto{\pgfqpoint{3.746197in}{2.973777in}}%
\pgfpathlineto{\pgfqpoint{3.750224in}{2.974570in}}%
\pgfpathlineto{\pgfqpoint{3.756264in}{2.972425in}}%
\pgfpathlineto{\pgfqpoint{3.758278in}{2.973124in}}%
\pgfpathlineto{\pgfqpoint{3.760291in}{2.976715in}}%
\pgfpathlineto{\pgfqpoint{3.762305in}{2.966969in}}%
\pgfpathlineto{\pgfqpoint{3.764318in}{2.968274in}}%
\pgfpathlineto{\pgfqpoint{3.770359in}{2.969394in}}%
\pgfpathlineto{\pgfqpoint{3.772372in}{2.977694in}}%
\pgfpathlineto{\pgfqpoint{3.774385in}{2.975875in}}%
\pgfpathlineto{\pgfqpoint{3.776399in}{2.976761in}}%
\pgfpathlineto{\pgfqpoint{3.786466in}{2.981658in}}%
\pgfpathlineto{\pgfqpoint{3.788480in}{2.981844in}}%
\pgfpathlineto{\pgfqpoint{3.792506in}{2.980072in}}%
\pgfpathlineto{\pgfqpoint{3.798547in}{2.986368in}}%
\pgfpathlineto{\pgfqpoint{3.800560in}{2.985109in}}%
\pgfpathlineto{\pgfqpoint{3.802574in}{2.986135in}}%
\pgfpathlineto{\pgfqpoint{3.804587in}{2.983103in}}%
\pgfpathlineto{\pgfqpoint{3.806601in}{2.978160in}}%
\pgfpathlineto{\pgfqpoint{3.812641in}{2.980958in}}%
\pgfpathlineto{\pgfqpoint{3.814654in}{2.978953in}}%
\pgfpathlineto{\pgfqpoint{3.816668in}{2.984642in}}%
\pgfpathlineto{\pgfqpoint{3.818681in}{2.982964in}}%
\pgfpathlineto{\pgfqpoint{3.820695in}{2.984642in}}%
\pgfpathlineto{\pgfqpoint{3.826735in}{2.983057in}}%
\pgfpathlineto{\pgfqpoint{3.828749in}{2.985062in}}%
\pgfpathlineto{\pgfqpoint{3.834789in}{2.983290in}}%
\pgfpathlineto{\pgfqpoint{3.840829in}{2.983523in}}%
\pgfpathlineto{\pgfqpoint{3.842843in}{2.981938in}}%
\pgfpathlineto{\pgfqpoint{3.844856in}{2.985528in}}%
\pgfpathlineto{\pgfqpoint{3.846870in}{2.987673in}}%
\pgfpathlineto{\pgfqpoint{3.848883in}{2.987767in}}%
\pgfpathlineto{\pgfqpoint{3.856937in}{2.987067in}}%
\pgfpathlineto{\pgfqpoint{3.858950in}{2.984969in}}%
\pgfpathlineto{\pgfqpoint{3.860964in}{2.986834in}}%
\pgfpathlineto{\pgfqpoint{3.862977in}{2.989725in}}%
\pgfpathlineto{\pgfqpoint{3.873044in}{2.994668in}}%
\pgfpathlineto{\pgfqpoint{3.875058in}{2.996860in}}%
\pgfpathlineto{\pgfqpoint{3.877071in}{2.996720in}}%
\pgfpathlineto{\pgfqpoint{3.883112in}{2.996860in}}%
\pgfpathlineto{\pgfqpoint{3.885125in}{3.000451in}}%
\pgfpathlineto{\pgfqpoint{3.889152in}{2.994062in}}%
\pgfpathlineto{\pgfqpoint{3.897206in}{2.993736in}}%
\pgfpathlineto{\pgfqpoint{3.899219in}{2.991637in}}%
\pgfpathlineto{\pgfqpoint{3.903246in}{2.999285in}}%
\pgfpathlineto{\pgfqpoint{3.905260in}{3.005207in}}%
\pgfpathlineto{\pgfqpoint{3.913313in}{3.002456in}}%
\pgfpathlineto{\pgfqpoint{3.915327in}{3.007212in}}%
\pgfpathlineto{\pgfqpoint{3.917340in}{3.006746in}}%
\pgfpathlineto{\pgfqpoint{3.919354in}{3.004088in}}%
\pgfpathlineto{\pgfqpoint{3.925394in}{3.002549in}}%
\pgfpathlineto{\pgfqpoint{3.927407in}{3.007632in}}%
\pgfpathlineto{\pgfqpoint{3.929421in}{3.007585in}}%
\pgfpathlineto{\pgfqpoint{3.931434in}{3.005813in}}%
\pgfpathlineto{\pgfqpoint{3.939488in}{3.010104in}}%
\pgfpathlineto{\pgfqpoint{3.941502in}{3.007026in}}%
\pgfpathlineto{\pgfqpoint{3.943515in}{3.008332in}}%
\pgfpathlineto{\pgfqpoint{3.945528in}{3.007352in}}%
\pgfpathlineto{\pgfqpoint{3.947542in}{3.004508in}}%
\pgfpathlineto{\pgfqpoint{3.953582in}{3.005674in}}%
\pgfpathlineto{\pgfqpoint{3.955596in}{2.997093in}}%
\pgfpathlineto{\pgfqpoint{3.957609in}{2.992803in}}%
\pgfpathlineto{\pgfqpoint{3.959623in}{2.985249in}}%
\pgfpathlineto{\pgfqpoint{3.961636in}{2.991078in}}%
\pgfpathlineto{\pgfqpoint{3.967676in}{2.988932in}}%
\pgfpathlineto{\pgfqpoint{3.969690in}{2.993829in}}%
\pgfpathlineto{\pgfqpoint{3.971703in}{2.992663in}}%
\pgfpathlineto{\pgfqpoint{3.975730in}{2.992337in}}%
\pgfpathlineto{\pgfqpoint{3.981771in}{2.992337in}}%
\pgfpathlineto{\pgfqpoint{3.983784in}{2.991730in}}%
\pgfpathlineto{\pgfqpoint{3.985797in}{2.993129in}}%
\pgfpathlineto{\pgfqpoint{3.987811in}{2.984129in}}%
\pgfpathlineto{\pgfqpoint{3.989824in}{2.983337in}}%
\pgfpathlineto{\pgfqpoint{3.995865in}{2.984409in}}%
\pgfpathlineto{\pgfqpoint{3.997878in}{2.983010in}}%
\pgfpathlineto{\pgfqpoint{3.999892in}{2.986648in}}%
\pgfpathlineto{\pgfqpoint{4.001905in}{2.983476in}}%
\pgfpathlineto{\pgfqpoint{4.003918in}{2.988140in}}%
\pgfpathlineto{\pgfqpoint{4.009959in}{2.988466in}}%
\pgfpathlineto{\pgfqpoint{4.011972in}{2.986321in}}%
\pgfpathlineto{\pgfqpoint{4.013986in}{2.990844in}}%
\pgfpathlineto{\pgfqpoint{4.015999in}{2.991964in}}%
\pgfpathlineto{\pgfqpoint{4.018013in}{2.988513in}}%
\pgfpathlineto{\pgfqpoint{4.024053in}{2.994762in}}%
\pgfpathlineto{\pgfqpoint{4.028080in}{2.996813in}}%
\pgfpathlineto{\pgfqpoint{4.030093in}{3.000824in}}%
\pgfpathlineto{\pgfqpoint{4.032107in}{2.999192in}}%
\pgfpathlineto{\pgfqpoint{4.038147in}{2.999751in}}%
\pgfpathlineto{\pgfqpoint{4.040160in}{3.000591in}}%
\pgfpathlineto{\pgfqpoint{4.044187in}{2.998585in}}%
\pgfpathlineto{\pgfqpoint{4.046201in}{3.001803in}}%
\pgfpathlineto{\pgfqpoint{4.054255in}{3.000264in}}%
\pgfpathlineto{\pgfqpoint{4.056268in}{3.001896in}}%
\pgfpathlineto{\pgfqpoint{4.058282in}{3.002269in}}%
\pgfpathlineto{\pgfqpoint{4.060295in}{3.004601in}}%
\pgfpathlineto{\pgfqpoint{4.068349in}{3.002083in}}%
\pgfpathlineto{\pgfqpoint{4.070362in}{3.006979in}}%
\pgfpathlineto{\pgfqpoint{4.072376in}{3.005161in}}%
\pgfpathlineto{\pgfqpoint{4.080429in}{3.005860in}}%
\pgfpathlineto{\pgfqpoint{4.082443in}{3.010617in}}%
\pgfpathlineto{\pgfqpoint{4.084456in}{3.011876in}}%
\pgfpathlineto{\pgfqpoint{4.086470in}{3.016632in}}%
\pgfpathlineto{\pgfqpoint{4.088483in}{3.019244in}}%
\pgfpathlineto{\pgfqpoint{4.094524in}{3.018777in}}%
\pgfpathlineto{\pgfqpoint{4.096537in}{3.017098in}}%
\pgfpathlineto{\pgfqpoint{4.098550in}{3.021902in}}%
\pgfpathlineto{\pgfqpoint{4.100564in}{3.015606in}}%
\pgfpathlineto{\pgfqpoint{4.102577in}{3.015606in}}%
\pgfpathlineto{\pgfqpoint{4.108618in}{3.013321in}}%
\pgfpathlineto{\pgfqpoint{4.110631in}{3.013508in}}%
\pgfpathlineto{\pgfqpoint{4.112645in}{3.004135in}}%
\pgfpathlineto{\pgfqpoint{4.114658in}{3.002269in}}%
\pgfpathlineto{\pgfqpoint{4.116671in}{3.007539in}}%
\pgfpathlineto{\pgfqpoint{4.122712in}{3.006420in}}%
\pgfpathlineto{\pgfqpoint{4.124725in}{2.996300in}}%
\pgfpathlineto{\pgfqpoint{4.126739in}{3.006653in}}%
\pgfpathlineto{\pgfqpoint{4.128752in}{2.995041in}}%
\pgfpathlineto{\pgfqpoint{4.130766in}{2.991544in}}%
\pgfpathlineto{\pgfqpoint{4.136806in}{2.982917in}}%
\pgfpathlineto{\pgfqpoint{4.138819in}{2.974243in}}%
\pgfpathlineto{\pgfqpoint{4.140833in}{2.979186in}}%
\pgfpathlineto{\pgfqpoint{4.142846in}{2.973311in}}%
\pgfpathlineto{\pgfqpoint{4.144860in}{2.981192in}}%
\pgfpathlineto{\pgfqpoint{4.150900in}{2.983243in}}%
\pgfpathlineto{\pgfqpoint{4.156940in}{2.997280in}}%
\pgfpathlineto{\pgfqpoint{4.158954in}{2.999332in}}%
\pgfpathlineto{\pgfqpoint{4.164994in}{3.003202in}}%
\pgfpathlineto{\pgfqpoint{4.169021in}{3.009311in}}%
\pgfpathlineto{\pgfqpoint{4.171035in}{3.015373in}}%
\pgfpathlineto{\pgfqpoint{4.173048in}{3.018404in}}%
\pgfpathlineto{\pgfqpoint{4.179088in}{3.017098in}}%
\pgfpathlineto{\pgfqpoint{4.181102in}{3.021808in}}%
\pgfpathlineto{\pgfqpoint{4.185129in}{3.023440in}}%
\pgfpathlineto{\pgfqpoint{4.187142in}{3.020130in}}%
\pgfpathlineto{\pgfqpoint{4.195196in}{3.023021in}}%
\pgfpathlineto{\pgfqpoint{4.197209in}{3.022368in}}%
\pgfpathlineto{\pgfqpoint{4.199223in}{3.023674in}}%
\pgfpathlineto{\pgfqpoint{4.201236in}{3.019943in}}%
\pgfpathlineto{\pgfqpoint{4.207277in}{3.020503in}}%
\pgfpathlineto{\pgfqpoint{4.209290in}{3.022694in}}%
\pgfpathlineto{\pgfqpoint{4.211303in}{3.022415in}}%
\pgfpathlineto{\pgfqpoint{4.213317in}{3.020036in}}%
\pgfpathlineto{\pgfqpoint{4.215330in}{3.021575in}}%
\pgfpathlineto{\pgfqpoint{4.223384in}{3.016819in}}%
\pgfpathlineto{\pgfqpoint{4.229425in}{3.023207in}}%
\pgfpathlineto{\pgfqpoint{4.235465in}{3.022275in}}%
\pgfpathlineto{\pgfqpoint{4.237478in}{3.024280in}}%
\pgfpathlineto{\pgfqpoint{4.239492in}{3.021016in}}%
\pgfpathlineto{\pgfqpoint{4.241505in}{3.020363in}}%
\pgfpathlineto{\pgfqpoint{4.243519in}{3.024280in}}%
\pgfpathlineto{\pgfqpoint{4.249559in}{3.024326in}}%
\pgfpathlineto{\pgfqpoint{4.251572in}{3.022368in}}%
\pgfpathlineto{\pgfqpoint{4.253586in}{3.014907in}}%
\pgfpathlineto{\pgfqpoint{4.255599in}{3.016912in}}%
\pgfpathlineto{\pgfqpoint{4.257613in}{3.007446in}}%
\pgfpathlineto{\pgfqpoint{4.263653in}{3.005534in}}%
\pgfpathlineto{\pgfqpoint{4.265667in}{3.000544in}}%
\pgfpathlineto{\pgfqpoint{4.267680in}{3.005953in}}%
\pgfpathlineto{\pgfqpoint{4.269693in}{3.017285in}}%
\pgfpathlineto{\pgfqpoint{4.271707in}{3.012062in}}%
\pgfpathlineto{\pgfqpoint{4.277747in}{3.016959in}}%
\pgfpathlineto{\pgfqpoint{4.279761in}{3.006839in}}%
\pgfpathlineto{\pgfqpoint{4.285801in}{3.010057in}}%
\pgfpathlineto{\pgfqpoint{4.293855in}{3.011269in}}%
\pgfpathlineto{\pgfqpoint{4.295868in}{3.008005in}}%
\pgfpathlineto{\pgfqpoint{4.299895in}{3.007819in}}%
\pgfpathlineto{\pgfqpoint{4.305936in}{3.004834in}}%
\pgfpathlineto{\pgfqpoint{4.307949in}{3.002689in}}%
\pgfpathlineto{\pgfqpoint{4.309962in}{3.012109in}}%
\pgfpathlineto{\pgfqpoint{4.311976in}{3.015513in}}%
\pgfpathlineto{\pgfqpoint{4.313989in}{3.009544in}}%
\pgfpathlineto{\pgfqpoint{4.320030in}{3.008052in}}%
\pgfpathlineto{\pgfqpoint{4.322043in}{3.008798in}}%
\pgfpathlineto{\pgfqpoint{4.324057in}{3.005674in}}%
\pgfpathlineto{\pgfqpoint{4.326070in}{2.999471in}}%
\pgfpathlineto{\pgfqpoint{4.328083in}{3.005860in}}%
\pgfpathlineto{\pgfqpoint{4.336137in}{2.994482in}}%
\pgfpathlineto{\pgfqpoint{4.338151in}{2.997000in}}%
\pgfpathlineto{\pgfqpoint{4.340164in}{3.004694in}}%
\pgfpathlineto{\pgfqpoint{4.342178in}{2.998259in}}%
\pgfpathlineto{\pgfqpoint{4.348218in}{2.998492in}}%
\pgfpathlineto{\pgfqpoint{4.350231in}{2.997793in}}%
\pgfpathlineto{\pgfqpoint{4.352245in}{2.995274in}}%
\pgfpathlineto{\pgfqpoint{4.354258in}{2.999005in}}%
\pgfpathlineto{\pgfqpoint{4.356272in}{2.989772in}}%
\pgfpathlineto{\pgfqpoint{4.362312in}{2.992616in}}%
\pgfpathlineto{\pgfqpoint{4.364325in}{2.999332in}}%
\pgfpathlineto{\pgfqpoint{4.366339in}{2.994808in}}%
\pgfpathlineto{\pgfqpoint{4.368352in}{2.999332in}}%
\pgfpathlineto{\pgfqpoint{4.370366in}{2.993736in}}%
\pgfpathlineto{\pgfqpoint{4.376406in}{2.988280in}}%
\pgfpathlineto{\pgfqpoint{4.378420in}{2.990611in}}%
\pgfpathlineto{\pgfqpoint{4.380433in}{2.990751in}}%
\pgfpathlineto{\pgfqpoint{4.382447in}{2.982730in}}%
\pgfpathlineto{\pgfqpoint{4.384460in}{2.987627in}}%
\pgfpathlineto{\pgfqpoint{4.392514in}{2.990984in}}%
\pgfpathlineto{\pgfqpoint{4.394527in}{2.989026in}}%
\pgfpathlineto{\pgfqpoint{4.396541in}{2.992057in}}%
\pgfpathlineto{\pgfqpoint{4.398554in}{2.993129in}}%
\pgfpathlineto{\pgfqpoint{4.404594in}{2.992803in}}%
\pgfpathlineto{\pgfqpoint{4.408621in}{2.997093in}}%
\pgfpathlineto{\pgfqpoint{4.410635in}{3.003715in}}%
\pgfpathlineto{\pgfqpoint{4.412648in}{3.002503in}}%
\pgfpathlineto{\pgfqpoint{4.418689in}{3.005440in}}%
\pgfpathlineto{\pgfqpoint{4.422715in}{2.998912in}}%
\pgfpathlineto{\pgfqpoint{4.424729in}{3.002549in}}%
\pgfpathlineto{\pgfqpoint{4.426742in}{2.992523in}}%
\pgfpathlineto{\pgfqpoint{4.432783in}{2.994808in}}%
\pgfpathlineto{\pgfqpoint{4.436810in}{2.985109in}}%
\pgfpathlineto{\pgfqpoint{4.438823in}{2.991357in}}%
\pgfpathlineto{\pgfqpoint{4.440836in}{2.988793in}}%
\pgfpathlineto{\pgfqpoint{4.446877in}{2.996487in}}%
\pgfpathlineto{\pgfqpoint{4.448890in}{2.991591in}}%
\pgfpathlineto{\pgfqpoint{4.450904in}{2.998119in}}%
\pgfpathlineto{\pgfqpoint{4.452917in}{2.999098in}}%
\pgfpathlineto{\pgfqpoint{4.454931in}{3.002036in}}%
\pgfpathlineto{\pgfqpoint{4.460971in}{3.004461in}}%
\pgfpathlineto{\pgfqpoint{4.462984in}{3.000218in}}%
\pgfpathlineto{\pgfqpoint{4.464998in}{2.993502in}}%
\pgfpathlineto{\pgfqpoint{4.467011in}{2.992663in}}%
\pgfpathlineto{\pgfqpoint{4.469025in}{2.993502in}}%
\pgfpathlineto{\pgfqpoint{4.475065in}{2.998492in}}%
\pgfpathlineto{\pgfqpoint{4.477079in}{2.994575in}}%
\pgfpathlineto{\pgfqpoint{4.479092in}{2.988559in}}%
\pgfpathlineto{\pgfqpoint{4.481105in}{2.990565in}}%
\pgfpathlineto{\pgfqpoint{4.489159in}{2.988559in}}%
\pgfpathlineto{\pgfqpoint{4.491173in}{2.992477in}}%
\pgfpathlineto{\pgfqpoint{4.493186in}{2.992710in}}%
\pgfpathlineto{\pgfqpoint{4.495200in}{2.997606in}}%
\pgfpathlineto{\pgfqpoint{4.497213in}{3.000637in}}%
\pgfpathlineto{\pgfqpoint{4.503253in}{2.994342in}}%
\pgfpathlineto{\pgfqpoint{4.507280in}{2.994575in}}%
\pgfpathlineto{\pgfqpoint{4.509294in}{2.991217in}}%
\pgfpathlineto{\pgfqpoint{4.511307in}{2.990331in}}%
\pgfpathlineto{\pgfqpoint{4.519361in}{2.993316in}}%
\pgfpathlineto{\pgfqpoint{4.523388in}{2.993969in}}%
\pgfpathlineto{\pgfqpoint{4.525401in}{2.996534in}}%
\pgfpathlineto{\pgfqpoint{4.531442in}{2.994482in}}%
\pgfpathlineto{\pgfqpoint{4.533455in}{2.995135in}}%
\pgfpathlineto{\pgfqpoint{4.535468in}{2.993689in}}%
\pgfpathlineto{\pgfqpoint{4.537482in}{2.988746in}}%
\pgfpathlineto{\pgfqpoint{4.539495in}{2.992616in}}%
\pgfpathlineto{\pgfqpoint{4.545536in}{2.993502in}}%
\pgfpathlineto{\pgfqpoint{4.547549in}{2.990005in}}%
\pgfpathlineto{\pgfqpoint{4.549563in}{2.988606in}}%
\pgfpathlineto{\pgfqpoint{4.551576in}{2.990658in}}%
\pgfpathlineto{\pgfqpoint{4.553590in}{2.998166in}}%
\pgfpathlineto{\pgfqpoint{4.559630in}{2.996347in}}%
\pgfpathlineto{\pgfqpoint{4.561643in}{2.994015in}}%
\pgfpathlineto{\pgfqpoint{4.563657in}{2.994342in}}%
\pgfpathlineto{\pgfqpoint{4.565670in}{2.999658in}}%
\pgfpathlineto{\pgfqpoint{4.567684in}{3.001617in}}%
\pgfpathlineto{\pgfqpoint{4.575737in}{3.008518in}}%
\pgfpathlineto{\pgfqpoint{4.579764in}{3.005207in}}%
\pgfpathlineto{\pgfqpoint{4.581778in}{3.000730in}}%
\pgfpathlineto{\pgfqpoint{4.589832in}{2.998539in}}%
\pgfpathlineto{\pgfqpoint{4.591845in}{2.999844in}}%
\pgfpathlineto{\pgfqpoint{4.593858in}{2.999891in}}%
\pgfpathlineto{\pgfqpoint{4.595872in}{2.995694in}}%
\pgfpathlineto{\pgfqpoint{4.605939in}{2.995461in}}%
\pgfpathlineto{\pgfqpoint{4.609966in}{2.989212in}}%
\pgfpathlineto{\pgfqpoint{4.616006in}{2.986554in}}%
\pgfpathlineto{\pgfqpoint{4.618020in}{2.987627in}}%
\pgfpathlineto{\pgfqpoint{4.620033in}{2.990425in}}%
\pgfpathlineto{\pgfqpoint{4.622047in}{2.991917in}}%
\pgfpathlineto{\pgfqpoint{4.624060in}{2.988186in}}%
\pgfpathlineto{\pgfqpoint{4.630101in}{2.984596in}}%
\pgfpathlineto{\pgfqpoint{4.632114in}{2.988233in}}%
\pgfpathlineto{\pgfqpoint{4.634127in}{2.989772in}}%
\pgfpathlineto{\pgfqpoint{4.636141in}{2.996580in}}%
\pgfpathlineto{\pgfqpoint{4.638154in}{2.994528in}}%
\pgfpathlineto{\pgfqpoint{4.644195in}{2.995461in}}%
\pgfpathlineto{\pgfqpoint{4.650235in}{2.991404in}}%
\pgfpathlineto{\pgfqpoint{4.652248in}{2.993596in}}%
\pgfpathlineto{\pgfqpoint{4.658289in}{2.985388in}}%
\pgfpathlineto{\pgfqpoint{4.660302in}{2.984456in}}%
\pgfpathlineto{\pgfqpoint{4.662316in}{2.988699in}}%
\pgfpathlineto{\pgfqpoint{4.672383in}{2.987580in}}%
\pgfpathlineto{\pgfqpoint{4.674396in}{2.990565in}}%
\pgfpathlineto{\pgfqpoint{4.676410in}{2.985761in}}%
\pgfpathlineto{\pgfqpoint{4.678423in}{2.988420in}}%
\pgfpathlineto{\pgfqpoint{4.680437in}{2.993129in}}%
\pgfpathlineto{\pgfqpoint{4.686477in}{2.996207in}}%
\pgfpathlineto{\pgfqpoint{4.688490in}{2.994155in}}%
\pgfpathlineto{\pgfqpoint{4.692517in}{2.999751in}}%
\pgfpathlineto{\pgfqpoint{4.694531in}{2.995414in}}%
\pgfpathlineto{\pgfqpoint{4.702585in}{2.996534in}}%
\pgfpathlineto{\pgfqpoint{4.704598in}{2.995834in}}%
\pgfpathlineto{\pgfqpoint{4.706612in}{2.995927in}}%
\pgfpathlineto{\pgfqpoint{4.708625in}{2.991544in}}%
\pgfpathlineto{\pgfqpoint{4.714665in}{2.987907in}}%
\pgfpathlineto{\pgfqpoint{4.718692in}{2.993969in}}%
\pgfpathlineto{\pgfqpoint{4.720706in}{2.994435in}}%
\pgfpathlineto{\pgfqpoint{4.722719in}{2.995974in}}%
\pgfpathlineto{\pgfqpoint{4.728759in}{2.995181in}}%
\pgfpathlineto{\pgfqpoint{4.730773in}{2.994249in}}%
\pgfpathlineto{\pgfqpoint{4.732786in}{2.997280in}}%
\pgfpathlineto{\pgfqpoint{4.734800in}{2.991217in}}%
\pgfpathlineto{\pgfqpoint{4.736813in}{2.990285in}}%
\pgfpathlineto{\pgfqpoint{4.742854in}{2.994202in}}%
\pgfpathlineto{\pgfqpoint{4.744867in}{2.990938in}}%
\pgfpathlineto{\pgfqpoint{4.748894in}{2.988746in}}%
\pgfpathlineto{\pgfqpoint{4.756948in}{2.994575in}}%
\pgfpathlineto{\pgfqpoint{4.758961in}{2.992477in}}%
\pgfpathlineto{\pgfqpoint{4.760975in}{2.992197in}}%
\pgfpathlineto{\pgfqpoint{4.762988in}{2.990052in}}%
\pgfpathlineto{\pgfqpoint{4.765001in}{2.979559in}}%
\pgfpathlineto{\pgfqpoint{4.771042in}{2.968041in}}%
\pgfpathlineto{\pgfqpoint{4.773055in}{2.959228in}}%
\pgfpathlineto{\pgfqpoint{4.775069in}{2.977694in}}%
\pgfpathlineto{\pgfqpoint{4.777082in}{2.982357in}}%
\pgfpathlineto{\pgfqpoint{4.779096in}{2.977927in}}%
\pgfpathlineto{\pgfqpoint{4.785136in}{2.972938in}}%
\pgfpathlineto{\pgfqpoint{4.787149in}{2.964963in}}%
\pgfpathlineto{\pgfqpoint{4.789163in}{2.970280in}}%
\pgfpathlineto{\pgfqpoint{4.791176in}{2.967295in}}%
\pgfpathlineto{\pgfqpoint{4.793190in}{2.961653in}}%
\pgfpathlineto{\pgfqpoint{4.801244in}{2.972751in}}%
\pgfpathlineto{\pgfqpoint{4.803257in}{2.965523in}}%
\pgfpathlineto{\pgfqpoint{4.805270in}{2.967621in}}%
\pgfpathlineto{\pgfqpoint{4.807284in}{2.968507in}}%
\pgfpathlineto{\pgfqpoint{4.813324in}{2.970140in}}%
\pgfpathlineto{\pgfqpoint{4.815338in}{2.974710in}}%
\pgfpathlineto{\pgfqpoint{4.819365in}{2.976528in}}%
\pgfpathlineto{\pgfqpoint{4.821378in}{2.970419in}}%
\pgfpathlineto{\pgfqpoint{4.827418in}{2.969347in}}%
\pgfpathlineto{\pgfqpoint{4.829432in}{2.969813in}}%
\pgfpathlineto{\pgfqpoint{4.831445in}{2.968741in}}%
\pgfpathlineto{\pgfqpoint{4.833459in}{2.966596in}}%
\pgfpathlineto{\pgfqpoint{4.835472in}{2.960347in}}%
\pgfpathlineto{\pgfqpoint{4.841512in}{2.961932in}}%
\pgfpathlineto{\pgfqpoint{4.843526in}{2.968974in}}%
\pgfpathlineto{\pgfqpoint{4.845539in}{2.970280in}}%
\pgfpathlineto{\pgfqpoint{4.847553in}{2.969487in}}%
\pgfpathlineto{\pgfqpoint{4.849566in}{2.972704in}}%
\pgfpathlineto{\pgfqpoint{4.855607in}{2.976202in}}%
\pgfpathlineto{\pgfqpoint{4.857620in}{2.970559in}}%
\pgfpathlineto{\pgfqpoint{4.859634in}{2.977134in}}%
\pgfpathlineto{\pgfqpoint{4.861647in}{2.977554in}}%
\pgfpathlineto{\pgfqpoint{4.863660in}{2.978767in}}%
\pgfpathlineto{\pgfqpoint{4.869701in}{2.981378in}}%
\pgfpathlineto{\pgfqpoint{4.871714in}{2.979140in}}%
\pgfpathlineto{\pgfqpoint{4.873728in}{2.975223in}}%
\pgfpathlineto{\pgfqpoint{4.875741in}{2.986274in}}%
\pgfpathlineto{\pgfqpoint{4.877755in}{2.990891in}}%
\pgfpathlineto{\pgfqpoint{4.883795in}{2.989632in}}%
\pgfpathlineto{\pgfqpoint{4.885808in}{2.988140in}}%
\pgfpathlineto{\pgfqpoint{4.887822in}{2.988326in}}%
\pgfpathlineto{\pgfqpoint{4.889835in}{2.996300in}}%
\pgfpathlineto{\pgfqpoint{4.891849in}{2.999611in}}%
\pgfpathlineto{\pgfqpoint{4.897889in}{2.998026in}}%
\pgfpathlineto{\pgfqpoint{4.901916in}{3.000311in}}%
\pgfpathlineto{\pgfqpoint{4.903929in}{3.004088in}}%
\pgfpathlineto{\pgfqpoint{4.905943in}{3.002642in}}%
\pgfpathlineto{\pgfqpoint{4.911983in}{3.007446in}}%
\pgfpathlineto{\pgfqpoint{4.913997in}{3.006560in}}%
\pgfpathlineto{\pgfqpoint{4.916010in}{3.006466in}}%
\pgfpathlineto{\pgfqpoint{4.918023in}{3.008098in}}%
\pgfpathlineto{\pgfqpoint{4.926077in}{3.001850in}}%
\pgfpathlineto{\pgfqpoint{4.930104in}{3.006140in}}%
\pgfpathlineto{\pgfqpoint{4.932118in}{2.999425in}}%
\pgfpathlineto{\pgfqpoint{4.934131in}{2.997793in}}%
\pgfpathlineto{\pgfqpoint{4.942185in}{3.004601in}}%
\pgfpathlineto{\pgfqpoint{4.944198in}{3.009544in}}%
\pgfpathlineto{\pgfqpoint{4.946212in}{3.008798in}}%
\pgfpathlineto{\pgfqpoint{4.948225in}{3.011922in}}%
\pgfpathlineto{\pgfqpoint{4.954266in}{3.013181in}}%
\pgfpathlineto{\pgfqpoint{4.956279in}{3.010057in}}%
\pgfpathlineto{\pgfqpoint{4.958292in}{3.009731in}}%
\pgfpathlineto{\pgfqpoint{4.962319in}{3.011456in}}%
\pgfpathlineto{\pgfqpoint{4.968360in}{3.006653in}}%
\pgfpathlineto{\pgfqpoint{4.970373in}{3.011409in}}%
\pgfpathlineto{\pgfqpoint{4.972387in}{3.010104in}}%
\pgfpathlineto{\pgfqpoint{4.974400in}{3.004741in}}%
\pgfpathlineto{\pgfqpoint{4.976413in}{3.013927in}}%
\pgfpathlineto{\pgfqpoint{4.982454in}{3.015513in}}%
\pgfpathlineto{\pgfqpoint{4.984467in}{3.011689in}}%
\pgfpathlineto{\pgfqpoint{4.986481in}{3.010523in}}%
\pgfpathlineto{\pgfqpoint{4.988494in}{3.012622in}}%
\pgfpathlineto{\pgfqpoint{4.990508in}{3.008518in}}%
\pgfpathlineto{\pgfqpoint{4.996548in}{3.010523in}}%
\pgfpathlineto{\pgfqpoint{4.998561in}{3.018917in}}%
\pgfpathlineto{\pgfqpoint{5.000575in}{3.023674in}}%
\pgfpathlineto{\pgfqpoint{5.004602in}{3.009684in}}%
\pgfpathlineto{\pgfqpoint{5.010642in}{3.008098in}}%
\pgfpathlineto{\pgfqpoint{5.012655in}{3.012902in}}%
\pgfpathlineto{\pgfqpoint{5.014669in}{3.016212in}}%
\pgfpathlineto{\pgfqpoint{5.016682in}{3.017192in}}%
\pgfpathlineto{\pgfqpoint{5.024736in}{3.015093in}}%
\pgfpathlineto{\pgfqpoint{5.026750in}{3.018497in}}%
\pgfpathlineto{\pgfqpoint{5.028763in}{3.017472in}}%
\pgfpathlineto{\pgfqpoint{5.030777in}{3.012948in}}%
\pgfpathlineto{\pgfqpoint{5.038830in}{3.003435in}}%
\pgfpathlineto{\pgfqpoint{5.040844in}{3.005207in}}%
\pgfpathlineto{\pgfqpoint{5.042857in}{3.003062in}}%
\pgfpathlineto{\pgfqpoint{5.046884in}{2.993596in}}%
\pgfpathlineto{\pgfqpoint{5.052924in}{2.991078in}}%
\pgfpathlineto{\pgfqpoint{5.054938in}{2.993922in}}%
\pgfpathlineto{\pgfqpoint{5.056951in}{2.988746in}}%
\pgfpathlineto{\pgfqpoint{5.058965in}{2.996673in}}%
\pgfpathlineto{\pgfqpoint{5.060978in}{2.988653in}}%
\pgfpathlineto{\pgfqpoint{5.069032in}{2.990798in}}%
\pgfpathlineto{\pgfqpoint{5.071045in}{2.983337in}}%
\pgfpathlineto{\pgfqpoint{5.073059in}{2.984176in}}%
\pgfpathlineto{\pgfqpoint{5.075072in}{2.987627in}}%
\pgfpathlineto{\pgfqpoint{5.081113in}{2.986135in}}%
\pgfpathlineto{\pgfqpoint{5.083126in}{3.006420in}}%
\pgfpathlineto{\pgfqpoint{5.085140in}{3.010570in}}%
\pgfpathlineto{\pgfqpoint{5.087153in}{3.011036in}}%
\pgfpathlineto{\pgfqpoint{5.089166in}{3.020269in}}%
\pgfpathlineto{\pgfqpoint{5.095207in}{3.019943in}}%
\pgfpathlineto{\pgfqpoint{5.097220in}{3.015886in}}%
\pgfpathlineto{\pgfqpoint{5.099234in}{3.018964in}}%
\pgfpathlineto{\pgfqpoint{5.101247in}{3.017984in}}%
\pgfpathlineto{\pgfqpoint{5.103261in}{3.003715in}}%
\pgfpathlineto{\pgfqpoint{5.109301in}{3.009917in}}%
\pgfpathlineto{\pgfqpoint{5.111314in}{3.009777in}}%
\pgfpathlineto{\pgfqpoint{5.113328in}{3.008751in}}%
\pgfpathlineto{\pgfqpoint{5.115341in}{3.008611in}}%
\pgfpathlineto{\pgfqpoint{5.127422in}{3.012016in}}%
\pgfpathlineto{\pgfqpoint{5.129435in}{3.019430in}}%
\pgfpathlineto{\pgfqpoint{5.131449in}{3.022275in}}%
\pgfpathlineto{\pgfqpoint{5.137489in}{3.024793in}}%
\pgfpathlineto{\pgfqpoint{5.139503in}{3.021948in}}%
\pgfpathlineto{\pgfqpoint{5.141516in}{3.025679in}}%
\pgfpathlineto{\pgfqpoint{5.143530in}{3.031788in}}%
\pgfpathlineto{\pgfqpoint{5.145543in}{3.029223in}}%
\pgfpathlineto{\pgfqpoint{5.151583in}{3.026751in}}%
\pgfpathlineto{\pgfqpoint{5.153597in}{3.035378in}}%
\pgfpathlineto{\pgfqpoint{5.155610in}{3.034586in}}%
\pgfpathlineto{\pgfqpoint{5.157624in}{3.032907in}}%
\pgfpathlineto{\pgfqpoint{5.159637in}{3.032301in}}%
\pgfpathlineto{\pgfqpoint{5.165677in}{3.033327in}}%
\pgfpathlineto{\pgfqpoint{5.167691in}{3.031088in}}%
\pgfpathlineto{\pgfqpoint{5.169704in}{3.033746in}}%
\pgfpathlineto{\pgfqpoint{5.171718in}{3.035005in}}%
\pgfpathlineto{\pgfqpoint{5.173731in}{3.037477in}}%
\pgfpathlineto{\pgfqpoint{5.181785in}{3.037663in}}%
\pgfpathlineto{\pgfqpoint{5.183799in}{3.036171in}}%
\pgfpathlineto{\pgfqpoint{5.185812in}{3.033327in}}%
\pgfpathlineto{\pgfqpoint{5.187825in}{3.036544in}}%
\pgfpathlineto{\pgfqpoint{5.193866in}{3.035612in}}%
\pgfpathlineto{\pgfqpoint{5.195879in}{3.036031in}}%
\pgfpathlineto{\pgfqpoint{5.197893in}{3.040834in}}%
\pgfpathlineto{\pgfqpoint{5.199906in}{3.040042in}}%
\pgfpathlineto{\pgfqpoint{5.207960in}{3.039669in}}%
\pgfpathlineto{\pgfqpoint{5.209973in}{3.043586in}}%
\pgfpathlineto{\pgfqpoint{5.211987in}{3.042886in}}%
\pgfpathlineto{\pgfqpoint{5.214000in}{3.039529in}}%
\pgfpathlineto{\pgfqpoint{5.216014in}{3.043772in}}%
\pgfpathlineto{\pgfqpoint{5.222054in}{3.041207in}}%
\pgfpathlineto{\pgfqpoint{5.226081in}{3.044751in}}%
\pgfpathlineto{\pgfqpoint{5.230108in}{3.043399in}}%
\pgfpathlineto{\pgfqpoint{5.236148in}{3.042840in}}%
\pgfpathlineto{\pgfqpoint{5.238162in}{3.045544in}}%
\pgfpathlineto{\pgfqpoint{5.240175in}{3.046710in}}%
\pgfpathlineto{\pgfqpoint{5.242188in}{3.046570in}}%
\pgfpathlineto{\pgfqpoint{5.244202in}{3.048016in}}%
\pgfpathlineto{\pgfqpoint{5.250242in}{3.051233in}}%
\pgfpathlineto{\pgfqpoint{5.252256in}{3.058695in}}%
\pgfpathlineto{\pgfqpoint{5.254269in}{3.062565in}}%
\pgfpathlineto{\pgfqpoint{5.256283in}{3.062518in}}%
\pgfpathlineto{\pgfqpoint{5.258296in}{3.061446in}}%
\pgfpathlineto{\pgfqpoint{5.264336in}{3.062239in}}%
\pgfpathlineto{\pgfqpoint{5.266350in}{3.059674in}}%
\pgfpathlineto{\pgfqpoint{5.268363in}{3.059114in}}%
\pgfpathlineto{\pgfqpoint{5.272390in}{3.056130in}}%
\pgfpathlineto{\pgfqpoint{5.278431in}{3.059021in}}%
\pgfpathlineto{\pgfqpoint{5.280444in}{3.058741in}}%
\pgfpathlineto{\pgfqpoint{5.282457in}{3.056736in}}%
\pgfpathlineto{\pgfqpoint{5.284471in}{3.059581in}}%
\pgfpathlineto{\pgfqpoint{5.286484in}{3.058974in}}%
\pgfpathlineto{\pgfqpoint{5.292525in}{3.063171in}}%
\pgfpathlineto{\pgfqpoint{5.294538in}{3.067228in}}%
\pgfpathlineto{\pgfqpoint{5.296552in}{3.065923in}}%
\pgfpathlineto{\pgfqpoint{5.298565in}{3.065363in}}%
\pgfpathlineto{\pgfqpoint{5.300578in}{3.062472in}}%
\pgfpathlineto{\pgfqpoint{5.306619in}{3.066249in}}%
\pgfpathlineto{\pgfqpoint{5.308632in}{3.063638in}}%
\pgfpathlineto{\pgfqpoint{5.310646in}{3.062612in}}%
\pgfpathlineto{\pgfqpoint{5.312659in}{3.059441in}}%
\pgfpathlineto{\pgfqpoint{5.314673in}{3.061959in}}%
\pgfpathlineto{\pgfqpoint{5.320713in}{3.059907in}}%
\pgfpathlineto{\pgfqpoint{5.324740in}{3.065037in}}%
\pgfpathlineto{\pgfqpoint{5.326753in}{3.063078in}}%
\pgfpathlineto{\pgfqpoint{5.328767in}{3.063777in}}%
\pgfpathlineto{\pgfqpoint{5.336820in}{3.062192in}}%
\pgfpathlineto{\pgfqpoint{5.338834in}{3.062565in}}%
\pgfpathlineto{\pgfqpoint{5.340847in}{3.069933in}}%
\pgfpathlineto{\pgfqpoint{5.342861in}{3.071099in}}%
\pgfpathlineto{\pgfqpoint{5.348901in}{3.075482in}}%
\pgfpathlineto{\pgfqpoint{5.352928in}{3.075575in}}%
\pgfpathlineto{\pgfqpoint{5.354942in}{3.080752in}}%
\pgfpathlineto{\pgfqpoint{5.356955in}{3.080798in}}%
\pgfpathlineto{\pgfqpoint{5.362995in}{3.079866in}}%
\pgfpathlineto{\pgfqpoint{5.365009in}{3.081265in}}%
\pgfpathlineto{\pgfqpoint{5.367022in}{3.078234in}}%
\pgfpathlineto{\pgfqpoint{5.369036in}{3.079213in}}%
\pgfpathlineto{\pgfqpoint{5.371049in}{3.074223in}}%
\pgfpathlineto{\pgfqpoint{5.377089in}{3.078793in}}%
\pgfpathlineto{\pgfqpoint{5.379103in}{3.077208in}}%
\pgfpathlineto{\pgfqpoint{5.381116in}{3.078420in}}%
\pgfpathlineto{\pgfqpoint{5.383130in}{3.082384in}}%
\pgfpathlineto{\pgfqpoint{5.385143in}{3.074876in}}%
\pgfpathlineto{\pgfqpoint{5.391184in}{3.078840in}}%
\pgfpathlineto{\pgfqpoint{5.393197in}{3.086068in}}%
\pgfpathlineto{\pgfqpoint{5.395210in}{3.090778in}}%
\pgfpathlineto{\pgfqpoint{5.397224in}{3.099265in}}%
\pgfpathlineto{\pgfqpoint{5.399237in}{3.099218in}}%
\pgfpathlineto{\pgfqpoint{5.409305in}{3.105047in}}%
\pgfpathlineto{\pgfqpoint{5.411318in}{3.104488in}}%
\pgfpathlineto{\pgfqpoint{5.413331in}{3.105980in}}%
\pgfpathlineto{\pgfqpoint{5.423399in}{3.106586in}}%
\pgfpathlineto{\pgfqpoint{5.425412in}{3.107379in}}%
\pgfpathlineto{\pgfqpoint{5.427426in}{3.106586in}}%
\pgfpathlineto{\pgfqpoint{5.433466in}{3.107192in}}%
\pgfpathlineto{\pgfqpoint{5.435479in}{3.116286in}}%
\pgfpathlineto{\pgfqpoint{5.441520in}{3.115353in}}%
\pgfpathlineto{\pgfqpoint{5.447560in}{3.114747in}}%
\pgfpathlineto{\pgfqpoint{5.449574in}{3.115866in}}%
\pgfpathlineto{\pgfqpoint{5.453600in}{3.112881in}}%
\pgfpathlineto{\pgfqpoint{5.455614in}{3.116192in}}%
\pgfpathlineto{\pgfqpoint{5.461654in}{3.116938in}}%
\pgfpathlineto{\pgfqpoint{5.463668in}{3.114560in}}%
\pgfpathlineto{\pgfqpoint{5.465681in}{3.110503in}}%
\pgfpathlineto{\pgfqpoint{5.467695in}{3.110317in}}%
\pgfpathlineto{\pgfqpoint{5.469708in}{3.111949in}}%
\pgfpathlineto{\pgfqpoint{5.479775in}{3.108172in}}%
\pgfpathlineto{\pgfqpoint{5.481789in}{3.109944in}}%
\pgfpathlineto{\pgfqpoint{5.483802in}{3.107565in}}%
\pgfpathlineto{\pgfqpoint{5.489842in}{3.103648in}}%
\pgfpathlineto{\pgfqpoint{5.491856in}{3.095114in}}%
\pgfpathlineto{\pgfqpoint{5.493869in}{3.099311in}}%
\pgfpathlineto{\pgfqpoint{5.495883in}{3.096793in}}%
\pgfpathlineto{\pgfqpoint{5.497896in}{3.096793in}}%
\pgfpathlineto{\pgfqpoint{5.503937in}{3.093342in}}%
\pgfpathlineto{\pgfqpoint{5.505950in}{3.094695in}}%
\pgfpathlineto{\pgfqpoint{5.507964in}{3.091524in}}%
\pgfpathlineto{\pgfqpoint{5.509977in}{3.090918in}}%
\pgfpathlineto{\pgfqpoint{5.511990in}{3.092969in}}%
\pgfpathlineto{\pgfqpoint{5.518031in}{3.096793in}}%
\pgfpathlineto{\pgfqpoint{5.520044in}{3.094835in}}%
\pgfpathlineto{\pgfqpoint{5.524071in}{3.093156in}}%
\pgfpathlineto{\pgfqpoint{5.526085in}{3.094182in}}%
\pgfpathlineto{\pgfqpoint{5.534138in}{3.096047in}}%
\pgfpathlineto{\pgfqpoint{5.538165in}{3.094835in}}%
\pgfpathlineto{\pgfqpoint{5.540179in}{3.089472in}}%
\pgfpathlineto{\pgfqpoint{5.546219in}{3.093482in}}%
\pgfpathlineto{\pgfqpoint{5.548232in}{3.086767in}}%
\pgfpathlineto{\pgfqpoint{5.550246in}{3.087840in}}%
\pgfpathlineto{\pgfqpoint{5.552259in}{3.091197in}}%
\pgfpathlineto{\pgfqpoint{5.554273in}{3.089565in}}%
\pgfpathlineto{\pgfqpoint{5.560313in}{3.087000in}}%
\pgfpathlineto{\pgfqpoint{5.562327in}{3.088259in}}%
\pgfpathlineto{\pgfqpoint{5.564340in}{3.092410in}}%
\pgfpathlineto{\pgfqpoint{5.566353in}{3.094788in}}%
\pgfpathlineto{\pgfqpoint{5.568367in}{3.091990in}}%
\pgfpathlineto{\pgfqpoint{5.574407in}{3.087513in}}%
\pgfpathlineto{\pgfqpoint{5.576421in}{3.093762in}}%
\pgfpathlineto{\pgfqpoint{5.578434in}{3.094508in}}%
\pgfpathlineto{\pgfqpoint{5.580448in}{3.085275in}}%
\pgfpathlineto{\pgfqpoint{5.582461in}{3.089006in}}%
\pgfpathlineto{\pgfqpoint{5.588501in}{3.091990in}}%
\pgfpathlineto{\pgfqpoint{5.590515in}{3.092037in}}%
\pgfpathlineto{\pgfqpoint{5.592528in}{3.093576in}}%
\pgfpathlineto{\pgfqpoint{5.594542in}{3.091710in}}%
\pgfpathlineto{\pgfqpoint{5.596555in}{3.093855in}}%
\pgfpathlineto{\pgfqpoint{5.602596in}{3.096280in}}%
\pgfpathlineto{\pgfqpoint{5.604609in}{3.086907in}}%
\pgfpathlineto{\pgfqpoint{5.608636in}{3.089565in}}%
\pgfpathlineto{\pgfqpoint{5.610649in}{3.086534in}}%
\pgfpathlineto{\pgfqpoint{5.616690in}{3.090591in}}%
\pgfpathlineto{\pgfqpoint{5.618703in}{3.077208in}}%
\pgfpathlineto{\pgfqpoint{5.620717in}{3.073664in}}%
\pgfpathlineto{\pgfqpoint{5.622730in}{3.074876in}}%
\pgfpathlineto{\pgfqpoint{5.624743in}{3.068674in}}%
\pgfpathlineto{\pgfqpoint{5.630784in}{3.069420in}}%
\pgfpathlineto{\pgfqpoint{5.632797in}{3.070959in}}%
\pgfpathlineto{\pgfqpoint{5.634811in}{3.073524in}}%
\pgfpathlineto{\pgfqpoint{5.636824in}{3.078467in}}%
\pgfpathlineto{\pgfqpoint{5.638838in}{3.076881in}}%
\pgfpathlineto{\pgfqpoint{5.644878in}{3.079726in}}%
\pgfpathlineto{\pgfqpoint{5.648905in}{3.074829in}}%
\pgfpathlineto{\pgfqpoint{5.652932in}{3.075949in}}%
\pgfpathlineto{\pgfqpoint{5.660985in}{3.084342in}}%
\pgfpathlineto{\pgfqpoint{5.662999in}{3.098472in}}%
\pgfpathlineto{\pgfqpoint{5.665012in}{3.095114in}}%
\pgfpathlineto{\pgfqpoint{5.667026in}{3.090498in}}%
\pgfpathlineto{\pgfqpoint{5.673066in}{3.082384in}}%
\pgfpathlineto{\pgfqpoint{5.675080in}{3.081171in}}%
\pgfpathlineto{\pgfqpoint{5.677093in}{3.081358in}}%
\pgfpathlineto{\pgfqpoint{5.679107in}{3.082244in}}%
\pgfpathlineto{\pgfqpoint{5.681120in}{3.080472in}}%
\pgfpathlineto{\pgfqpoint{5.687160in}{3.078886in}}%
\pgfpathlineto{\pgfqpoint{5.689174in}{3.069047in}}%
\pgfpathlineto{\pgfqpoint{5.691187in}{3.070446in}}%
\pgfpathlineto{\pgfqpoint{5.695214in}{3.075109in}}%
\pgfpathlineto{\pgfqpoint{5.701254in}{3.070726in}}%
\pgfpathlineto{\pgfqpoint{5.703268in}{3.067881in}}%
\pgfpathlineto{\pgfqpoint{5.705281in}{3.062752in}}%
\pgfpathlineto{\pgfqpoint{5.707295in}{3.063078in}}%
\pgfpathlineto{\pgfqpoint{5.709308in}{3.065643in}}%
\pgfpathlineto{\pgfqpoint{5.717362in}{3.066062in}}%
\pgfpathlineto{\pgfqpoint{5.719375in}{3.061866in}}%
\pgfpathlineto{\pgfqpoint{5.721389in}{3.061399in}}%
\pgfpathlineto{\pgfqpoint{5.723402in}{3.066948in}}%
\pgfpathlineto{\pgfqpoint{5.729443in}{3.080472in}}%
\pgfpathlineto{\pgfqpoint{5.731456in}{3.082804in}}%
\pgfpathlineto{\pgfqpoint{5.733470in}{3.078840in}}%
\pgfpathlineto{\pgfqpoint{5.735483in}{3.082804in}}%
\pgfpathlineto{\pgfqpoint{5.737496in}{3.082757in}}%
\pgfpathlineto{\pgfqpoint{5.743537in}{3.083363in}}%
\pgfpathlineto{\pgfqpoint{5.747564in}{3.080239in}}%
\pgfpathlineto{\pgfqpoint{5.749577in}{3.080798in}}%
\pgfpathlineto{\pgfqpoint{5.751591in}{3.083083in}}%
\pgfpathlineto{\pgfqpoint{5.759644in}{3.082850in}}%
\pgfpathlineto{\pgfqpoint{5.761658in}{3.079353in}}%
\pgfpathlineto{\pgfqpoint{5.763671in}{3.081031in}}%
\pgfpathlineto{\pgfqpoint{5.765685in}{3.079819in}}%
\pgfpathlineto{\pgfqpoint{5.773739in}{3.082570in}}%
\pgfpathlineto{\pgfqpoint{5.775752in}{3.081731in}}%
\pgfpathlineto{\pgfqpoint{5.777765in}{3.087000in}}%
\pgfpathlineto{\pgfqpoint{5.779779in}{3.084576in}}%
\pgfpathlineto{\pgfqpoint{5.787833in}{3.083969in}}%
\pgfpathlineto{\pgfqpoint{5.789846in}{3.077721in}}%
\pgfpathlineto{\pgfqpoint{5.793873in}{3.077161in}}%
\pgfpathlineto{\pgfqpoint{5.801927in}{3.078327in}}%
\pgfpathlineto{\pgfqpoint{5.803940in}{3.077581in}}%
\pgfpathlineto{\pgfqpoint{5.805954in}{3.075389in}}%
\pgfpathlineto{\pgfqpoint{5.814007in}{3.074130in}}%
\pgfpathlineto{\pgfqpoint{5.816021in}{3.064757in}}%
\pgfpathlineto{\pgfqpoint{5.818034in}{3.069280in}}%
\pgfpathlineto{\pgfqpoint{5.820048in}{3.065083in}}%
\pgfpathlineto{\pgfqpoint{5.822061in}{3.071845in}}%
\pgfpathlineto{\pgfqpoint{5.828102in}{3.070726in}}%
\pgfpathlineto{\pgfqpoint{5.830115in}{3.071239in}}%
\pgfpathlineto{\pgfqpoint{5.832129in}{3.071145in}}%
\pgfpathlineto{\pgfqpoint{5.834142in}{3.072638in}}%
\pgfpathlineto{\pgfqpoint{5.836155in}{3.072964in}}%
\pgfpathlineto{\pgfqpoint{5.842196in}{3.071892in}}%
\pgfpathlineto{\pgfqpoint{5.844209in}{3.072265in}}%
\pgfpathlineto{\pgfqpoint{5.846223in}{3.071892in}}%
\pgfpathlineto{\pgfqpoint{5.848236in}{3.074876in}}%
\pgfpathlineto{\pgfqpoint{5.850250in}{3.079959in}}%
\pgfpathlineto{\pgfqpoint{5.856290in}{3.082757in}}%
\pgfpathlineto{\pgfqpoint{5.858303in}{3.084855in}}%
\pgfpathlineto{\pgfqpoint{5.864344in}{3.095767in}}%
\pgfpathlineto{\pgfqpoint{5.872397in}{3.099265in}}%
\pgfpathlineto{\pgfqpoint{5.874411in}{3.098612in}}%
\pgfpathlineto{\pgfqpoint{5.878438in}{3.116192in}}%
\pgfpathlineto{\pgfqpoint{5.886492in}{3.113907in}}%
\pgfpathlineto{\pgfqpoint{5.888505in}{3.121135in}}%
\pgfpathlineto{\pgfqpoint{5.890518in}{3.120156in}}%
\pgfpathlineto{\pgfqpoint{5.892532in}{3.120856in}}%
\pgfpathlineto{\pgfqpoint{5.898572in}{3.120482in}}%
\pgfpathlineto{\pgfqpoint{5.900586in}{3.121042in}}%
\pgfpathlineto{\pgfqpoint{5.902599in}{3.122208in}}%
\pgfpathlineto{\pgfqpoint{5.904613in}{3.130322in}}%
\pgfpathlineto{\pgfqpoint{5.906626in}{3.131488in}}%
\pgfpathlineto{\pgfqpoint{5.912666in}{3.133540in}}%
\pgfpathlineto{\pgfqpoint{5.914680in}{3.135172in}}%
\pgfpathlineto{\pgfqpoint{5.916693in}{3.143566in}}%
\pgfpathlineto{\pgfqpoint{5.920720in}{3.139602in}}%
\pgfpathlineto{\pgfqpoint{5.926761in}{3.139648in}}%
\pgfpathlineto{\pgfqpoint{5.930787in}{3.131721in}}%
\pgfpathlineto{\pgfqpoint{5.934814in}{3.128270in}}%
\pgfpathlineto{\pgfqpoint{5.940855in}{3.129669in}}%
\pgfpathlineto{\pgfqpoint{5.942868in}{3.129063in}}%
\pgfpathlineto{\pgfqpoint{5.944882in}{3.125799in}}%
\pgfpathlineto{\pgfqpoint{5.946895in}{3.124679in}}%
\pgfpathlineto{\pgfqpoint{5.948908in}{3.124213in}}%
\pgfpathlineto{\pgfqpoint{5.958976in}{3.125286in}}%
\pgfpathlineto{\pgfqpoint{5.960989in}{3.126405in}}%
\pgfpathlineto{\pgfqpoint{5.963003in}{3.125799in}}%
\pgfpathlineto{\pgfqpoint{5.971056in}{3.122721in}}%
\pgfpathlineto{\pgfqpoint{5.973070in}{3.127944in}}%
\pgfpathlineto{\pgfqpoint{5.975083in}{3.126125in}}%
\pgfpathlineto{\pgfqpoint{5.983137in}{3.129343in}}%
\pgfpathlineto{\pgfqpoint{5.985151in}{3.112182in}}%
\pgfpathlineto{\pgfqpoint{5.987164in}{3.110223in}}%
\pgfpathlineto{\pgfqpoint{5.989177in}{3.112415in}}%
\pgfpathlineto{\pgfqpoint{5.991191in}{3.111949in}}%
\pgfpathlineto{\pgfqpoint{5.997231in}{3.116892in}}%
\pgfpathlineto{\pgfqpoint{5.999245in}{3.119270in}}%
\pgfpathlineto{\pgfqpoint{6.001258in}{3.119596in}}%
\pgfpathlineto{\pgfqpoint{6.003272in}{3.120622in}}%
\pgfpathlineto{\pgfqpoint{6.005285in}{3.119457in}}%
\pgfpathlineto{\pgfqpoint{6.011325in}{3.118850in}}%
\pgfpathlineto{\pgfqpoint{6.013339in}{3.120436in}}%
\pgfpathlineto{\pgfqpoint{6.015352in}{3.118850in}}%
\pgfpathlineto{\pgfqpoint{6.017366in}{3.121555in}}%
\pgfpathlineto{\pgfqpoint{6.019379in}{3.119596in}}%
\pgfpathlineto{\pgfqpoint{6.027433in}{3.118291in}}%
\pgfpathlineto{\pgfqpoint{6.029446in}{3.116565in}}%
\pgfpathlineto{\pgfqpoint{6.033473in}{3.120203in}}%
\pgfpathlineto{\pgfqpoint{6.041527in}{3.138343in}}%
\pgfpathlineto{\pgfqpoint{6.043540in}{3.133493in}}%
\pgfpathlineto{\pgfqpoint{6.045554in}{3.134845in}}%
\pgfpathlineto{\pgfqpoint{6.047567in}{3.134939in}}%
\pgfpathlineto{\pgfqpoint{6.053608in}{3.136104in}}%
\pgfpathlineto{\pgfqpoint{6.055621in}{3.137224in}}%
\pgfpathlineto{\pgfqpoint{6.057635in}{3.137177in}}%
\pgfpathlineto{\pgfqpoint{6.059648in}{3.141001in}}%
\pgfpathlineto{\pgfqpoint{6.061661in}{3.138296in}}%
\pgfpathlineto{\pgfqpoint{6.069715in}{3.139135in}}%
\pgfpathlineto{\pgfqpoint{6.071729in}{3.144172in}}%
\pgfpathlineto{\pgfqpoint{6.073742in}{3.146550in}}%
\pgfpathlineto{\pgfqpoint{6.075756in}{3.152286in}}%
\pgfpathlineto{\pgfqpoint{6.081796in}{3.153545in}}%
\pgfpathlineto{\pgfqpoint{6.083809in}{3.155597in}}%
\pgfpathlineto{\pgfqpoint{6.087836in}{3.154384in}}%
\pgfpathlineto{\pgfqpoint{6.089850in}{3.158721in}}%
\pgfpathlineto{\pgfqpoint{6.097904in}{3.160866in}}%
\pgfpathlineto{\pgfqpoint{6.099917in}{3.164177in}}%
\pgfpathlineto{\pgfqpoint{6.101930in}{3.165389in}}%
\pgfpathlineto{\pgfqpoint{6.103944in}{3.171172in}}%
\pgfpathlineto{\pgfqpoint{6.109984in}{3.169959in}}%
\pgfpathlineto{\pgfqpoint{6.111998in}{3.170612in}}%
\pgfpathlineto{\pgfqpoint{6.114011in}{3.173643in}}%
\pgfpathlineto{\pgfqpoint{6.116025in}{3.178680in}}%
\pgfpathlineto{\pgfqpoint{6.118038in}{3.180358in}}%
\pgfpathlineto{\pgfqpoint{6.124078in}{3.179985in}}%
\pgfpathlineto{\pgfqpoint{6.130119in}{3.163617in}}%
\pgfpathlineto{\pgfqpoint{6.132132in}{3.162079in}}%
\pgfpathlineto{\pgfqpoint{6.138172in}{3.164783in}}%
\pgfpathlineto{\pgfqpoint{6.142199in}{3.168048in}}%
\pgfpathlineto{\pgfqpoint{6.144213in}{3.163058in}}%
\pgfpathlineto{\pgfqpoint{6.146226in}{3.163151in}}%
\pgfpathlineto{\pgfqpoint{6.154280in}{3.157369in}}%
\pgfpathlineto{\pgfqpoint{6.156294in}{3.161799in}}%
\pgfpathlineto{\pgfqpoint{6.158307in}{3.160167in}}%
\pgfpathlineto{\pgfqpoint{6.160320in}{3.163431in}}%
\pgfpathlineto{\pgfqpoint{6.166361in}{3.161426in}}%
\pgfpathlineto{\pgfqpoint{6.168374in}{3.171685in}}%
\pgfpathlineto{\pgfqpoint{6.170388in}{3.174996in}}%
\pgfpathlineto{\pgfqpoint{6.172401in}{3.181011in}}%
\pgfpathlineto{\pgfqpoint{6.174415in}{3.175415in}}%
\pgfpathlineto{\pgfqpoint{6.180455in}{3.165250in}}%
\pgfpathlineto{\pgfqpoint{6.184482in}{3.156156in}}%
\pgfpathlineto{\pgfqpoint{6.186495in}{3.155597in}}%
\pgfpathlineto{\pgfqpoint{6.188509in}{3.160120in}}%
\pgfpathlineto{\pgfqpoint{6.194549in}{3.163944in}}%
\pgfpathlineto{\pgfqpoint{6.196562in}{3.163058in}}%
\pgfpathlineto{\pgfqpoint{6.198576in}{3.161472in}}%
\pgfpathlineto{\pgfqpoint{6.200589in}{3.166742in}}%
\pgfpathlineto{\pgfqpoint{6.202603in}{3.165996in}}%
\pgfpathlineto{\pgfqpoint{6.208643in}{3.164597in}}%
\pgfpathlineto{\pgfqpoint{6.210657in}{3.161985in}}%
\pgfpathlineto{\pgfqpoint{6.212670in}{3.166229in}}%
\pgfpathlineto{\pgfqpoint{6.214683in}{3.165623in}}%
\pgfpathlineto{\pgfqpoint{6.216697in}{3.165669in}}%
\pgfpathlineto{\pgfqpoint{6.222737in}{3.167348in}}%
\pgfpathlineto{\pgfqpoint{6.224751in}{3.166882in}}%
\pgfpathlineto{\pgfqpoint{6.226764in}{3.170379in}}%
\pgfpathlineto{\pgfqpoint{6.228778in}{3.165389in}}%
\pgfpathlineto{\pgfqpoint{6.230791in}{3.163571in}}%
\pgfpathlineto{\pgfqpoint{6.236831in}{3.167208in}}%
\pgfpathlineto{\pgfqpoint{6.238845in}{3.172711in}}%
\pgfpathlineto{\pgfqpoint{6.240858in}{3.164317in}}%
\pgfpathlineto{\pgfqpoint{6.242872in}{3.164737in}}%
\pgfpathlineto{\pgfqpoint{6.244885in}{3.163058in}}%
\pgfpathlineto{\pgfqpoint{6.250926in}{3.163338in}}%
\pgfpathlineto{\pgfqpoint{6.252939in}{3.165436in}}%
\pgfpathlineto{\pgfqpoint{6.254952in}{3.160353in}}%
\pgfpathlineto{\pgfqpoint{6.256966in}{3.166136in}}%
\pgfpathlineto{\pgfqpoint{6.258979in}{3.160167in}}%
\pgfpathlineto{\pgfqpoint{6.267033in}{3.155130in}}%
\pgfpathlineto{\pgfqpoint{6.269047in}{3.158581in}}%
\pgfpathlineto{\pgfqpoint{6.271060in}{3.165343in}}%
\pgfpathlineto{\pgfqpoint{6.273073in}{3.159933in}}%
\pgfpathlineto{\pgfqpoint{6.279114in}{3.169866in}}%
\pgfpathlineto{\pgfqpoint{6.281127in}{3.167301in}}%
\pgfpathlineto{\pgfqpoint{6.283141in}{3.166509in}}%
\pgfpathlineto{\pgfqpoint{6.285154in}{3.174203in}}%
\pgfpathlineto{\pgfqpoint{6.293208in}{3.179566in}}%
\pgfpathlineto{\pgfqpoint{6.295221in}{3.178820in}}%
\pgfpathlineto{\pgfqpoint{6.299248in}{3.163384in}}%
\pgfpathlineto{\pgfqpoint{6.301262in}{3.161799in}}%
\pgfpathlineto{\pgfqpoint{6.307302in}{3.160820in}}%
\pgfpathlineto{\pgfqpoint{6.309316in}{3.159794in}}%
\pgfpathlineto{\pgfqpoint{6.311329in}{3.154478in}}%
\pgfpathlineto{\pgfqpoint{6.313342in}{3.153218in}}%
\pgfpathlineto{\pgfqpoint{6.315356in}{3.155643in}}%
\pgfpathlineto{\pgfqpoint{6.321396in}{3.161006in}}%
\pgfpathlineto{\pgfqpoint{6.325423in}{3.168467in}}%
\pgfpathlineto{\pgfqpoint{6.327437in}{3.169773in}}%
\pgfpathlineto{\pgfqpoint{6.329450in}{3.169913in}}%
\pgfpathlineto{\pgfqpoint{6.335490in}{3.170939in}}%
\pgfpathlineto{\pgfqpoint{6.337504in}{3.172944in}}%
\pgfpathlineto{\pgfqpoint{6.339517in}{3.185208in}}%
\pgfpathlineto{\pgfqpoint{6.341531in}{3.186001in}}%
\pgfpathlineto{\pgfqpoint{6.343544in}{3.184229in}}%
\pgfpathlineto{\pgfqpoint{6.349584in}{3.182830in}}%
\pgfpathlineto{\pgfqpoint{6.351598in}{3.203628in}}%
\pgfpathlineto{\pgfqpoint{6.353611in}{3.203162in}}%
\pgfpathlineto{\pgfqpoint{6.355625in}{3.209177in}}%
\pgfpathlineto{\pgfqpoint{6.363679in}{3.216219in}}%
\pgfpathlineto{\pgfqpoint{6.365692in}{3.207405in}}%
\pgfpathlineto{\pgfqpoint{6.367705in}{3.210623in}}%
\pgfpathlineto{\pgfqpoint{6.369719in}{3.208151in}}%
\pgfpathlineto{\pgfqpoint{6.371732in}{3.208058in}}%
\pgfpathlineto{\pgfqpoint{6.379786in}{3.197473in}}%
\pgfpathlineto{\pgfqpoint{6.381800in}{3.200037in}}%
\pgfpathlineto{\pgfqpoint{6.383813in}{3.199804in}}%
\pgfpathlineto{\pgfqpoint{6.385827in}{3.200457in}}%
\pgfpathlineto{\pgfqpoint{6.393880in}{3.199105in}}%
\pgfpathlineto{\pgfqpoint{6.395894in}{3.206006in}}%
\pgfpathlineto{\pgfqpoint{6.399921in}{3.198172in}}%
\pgfpathlineto{\pgfqpoint{6.405961in}{3.199058in}}%
\pgfpathlineto{\pgfqpoint{6.412001in}{3.195094in}}%
\pgfpathlineto{\pgfqpoint{6.414015in}{3.191224in}}%
\pgfpathlineto{\pgfqpoint{6.420055in}{3.190897in}}%
\pgfpathlineto{\pgfqpoint{6.422069in}{3.192716in}}%
\pgfpathlineto{\pgfqpoint{6.424082in}{3.188053in}}%
\pgfpathlineto{\pgfqpoint{6.434149in}{3.195421in}}%
\pgfpathlineto{\pgfqpoint{6.436163in}{3.204001in}}%
\pgfpathlineto{\pgfqpoint{6.438176in}{3.203068in}}%
\pgfpathlineto{\pgfqpoint{6.440190in}{3.200923in}}%
\pgfpathlineto{\pgfqpoint{6.442203in}{3.203861in}}%
\pgfpathlineto{\pgfqpoint{6.448243in}{3.199478in}}%
\pgfpathlineto{\pgfqpoint{6.450257in}{3.202462in}}%
\pgfpathlineto{\pgfqpoint{6.452270in}{3.208664in}}%
\pgfpathlineto{\pgfqpoint{6.454284in}{3.203954in}}%
\pgfpathlineto{\pgfqpoint{6.456297in}{3.206566in}}%
\pgfpathlineto{\pgfqpoint{6.462337in}{3.209037in}}%
\pgfpathlineto{\pgfqpoint{6.464351in}{3.215566in}}%
\pgfpathlineto{\pgfqpoint{6.466364in}{3.216872in}}%
\pgfpathlineto{\pgfqpoint{6.468378in}{3.211322in}}%
\pgfpathlineto{\pgfqpoint{6.470391in}{3.214960in}}%
\pgfpathlineto{\pgfqpoint{6.476432in}{3.211975in}}%
\pgfpathlineto{\pgfqpoint{6.478445in}{3.211882in}}%
\pgfpathlineto{\pgfqpoint{6.480459in}{3.209131in}}%
\pgfpathlineto{\pgfqpoint{6.482472in}{3.208664in}}%
\pgfpathlineto{\pgfqpoint{6.484485in}{3.204467in}}%
\pgfpathlineto{\pgfqpoint{6.492539in}{3.204328in}}%
\pgfpathlineto{\pgfqpoint{6.494553in}{3.206473in}}%
\pgfpathlineto{\pgfqpoint{6.496566in}{3.206426in}}%
\pgfpathlineto{\pgfqpoint{6.498580in}{3.202695in}}%
\pgfpathlineto{\pgfqpoint{6.498580in}{3.202695in}}%
\pgfusepath{stroke}%
\end{pgfscope}%
\begin{pgfscope}%
\pgfsetrectcap%
\pgfsetmiterjoin%
\pgfsetlinewidth{0.803000pt}%
\definecolor{currentstroke}{rgb}{1.000000,1.000000,1.000000}%
\pgfsetstrokecolor{currentstroke}%
\pgfsetdash{}{0pt}%
\pgfpathmoveto{\pgfqpoint{1.875000in}{2.792941in}}%
\pgfpathlineto{\pgfqpoint{1.875000in}{3.237059in}}%
\pgfusepath{stroke}%
\end{pgfscope}%
\begin{pgfscope}%
\pgfsetrectcap%
\pgfsetmiterjoin%
\pgfsetlinewidth{0.803000pt}%
\definecolor{currentstroke}{rgb}{1.000000,1.000000,1.000000}%
\pgfsetstrokecolor{currentstroke}%
\pgfsetdash{}{0pt}%
\pgfpathmoveto{\pgfqpoint{6.718750in}{2.792941in}}%
\pgfpathlineto{\pgfqpoint{6.718750in}{3.237059in}}%
\pgfusepath{stroke}%
\end{pgfscope}%
\begin{pgfscope}%
\pgfsetrectcap%
\pgfsetmiterjoin%
\pgfsetlinewidth{0.803000pt}%
\definecolor{currentstroke}{rgb}{1.000000,1.000000,1.000000}%
\pgfsetstrokecolor{currentstroke}%
\pgfsetdash{}{0pt}%
\pgfpathmoveto{\pgfqpoint{1.875000in}{2.792941in}}%
\pgfpathlineto{\pgfqpoint{6.718750in}{2.792941in}}%
\pgfusepath{stroke}%
\end{pgfscope}%
\begin{pgfscope}%
\pgfsetrectcap%
\pgfsetmiterjoin%
\pgfsetlinewidth{0.803000pt}%
\definecolor{currentstroke}{rgb}{1.000000,1.000000,1.000000}%
\pgfsetstrokecolor{currentstroke}%
\pgfsetdash{}{0pt}%
\pgfpathmoveto{\pgfqpoint{1.875000in}{3.237059in}}%
\pgfpathlineto{\pgfqpoint{6.718750in}{3.237059in}}%
\pgfusepath{stroke}%
\end{pgfscope}%
\begin{pgfscope}%
\definecolor{textcolor}{rgb}{0.150000,0.150000,0.150000}%
\pgfsetstrokecolor{textcolor}%
\pgfsetfillcolor{textcolor}%
\pgftext[x=4.296875in,y=3.320392in,,base]{\color{textcolor}\rmfamily\fontsize{16.800000}{20.160000}\selectfont JNJ}%
\end{pgfscope}%
\begin{pgfscope}%
\pgfsetbuttcap%
\pgfsetmiterjoin%
\definecolor{currentfill}{rgb}{0.917647,0.917647,0.949020}%
\pgfsetfillcolor{currentfill}%
\pgfsetlinewidth{0.000000pt}%
\definecolor{currentstroke}{rgb}{0.000000,0.000000,0.000000}%
\pgfsetstrokecolor{currentstroke}%
\pgfsetstrokeopacity{0.000000}%
\pgfsetdash{}{0pt}%
\pgfpathmoveto{\pgfqpoint{8.656250in}{2.792941in}}%
\pgfpathlineto{\pgfqpoint{13.500000in}{2.792941in}}%
\pgfpathlineto{\pgfqpoint{13.500000in}{3.237059in}}%
\pgfpathlineto{\pgfqpoint{8.656250in}{3.237059in}}%
\pgfpathclose%
\pgfusepath{fill}%
\end{pgfscope}%
\begin{pgfscope}%
\pgfpathrectangle{\pgfqpoint{8.656250in}{2.792941in}}{\pgfqpoint{4.843750in}{0.444118in}}%
\pgfusepath{clip}%
\pgfsetroundcap%
\pgfsetroundjoin%
\pgfsetlinewidth{0.803000pt}%
\definecolor{currentstroke}{rgb}{1.000000,1.000000,1.000000}%
\pgfsetstrokecolor{currentstroke}%
\pgfsetdash{}{0pt}%
\pgfpathmoveto{\pgfqpoint{8.872394in}{2.792941in}}%
\pgfpathlineto{\pgfqpoint{8.872394in}{3.237059in}}%
\pgfusepath{stroke}%
\end{pgfscope}%
\begin{pgfscope}%
\definecolor{textcolor}{rgb}{0.150000,0.150000,0.150000}%
\pgfsetstrokecolor{textcolor}%
\pgfsetfillcolor{textcolor}%
\pgftext[x=8.872394in,y=2.695719in,,top]{\color{textcolor}\rmfamily\fontsize{14.000000}{16.800000}\selectfont 2012}%
\end{pgfscope}%
\begin{pgfscope}%
\pgfpathrectangle{\pgfqpoint{8.656250in}{2.792941in}}{\pgfqpoint{4.843750in}{0.444118in}}%
\pgfusepath{clip}%
\pgfsetroundcap%
\pgfsetroundjoin%
\pgfsetlinewidth{0.803000pt}%
\definecolor{currentstroke}{rgb}{1.000000,1.000000,1.000000}%
\pgfsetstrokecolor{currentstroke}%
\pgfsetdash{}{0pt}%
\pgfpathmoveto{\pgfqpoint{9.609315in}{2.792941in}}%
\pgfpathlineto{\pgfqpoint{9.609315in}{3.237059in}}%
\pgfusepath{stroke}%
\end{pgfscope}%
\begin{pgfscope}%
\definecolor{textcolor}{rgb}{0.150000,0.150000,0.150000}%
\pgfsetstrokecolor{textcolor}%
\pgfsetfillcolor{textcolor}%
\pgftext[x=9.609315in,y=2.695719in,,top]{\color{textcolor}\rmfamily\fontsize{14.000000}{16.800000}\selectfont 2013}%
\end{pgfscope}%
\begin{pgfscope}%
\pgfpathrectangle{\pgfqpoint{8.656250in}{2.792941in}}{\pgfqpoint{4.843750in}{0.444118in}}%
\pgfusepath{clip}%
\pgfsetroundcap%
\pgfsetroundjoin%
\pgfsetlinewidth{0.803000pt}%
\definecolor{currentstroke}{rgb}{1.000000,1.000000,1.000000}%
\pgfsetstrokecolor{currentstroke}%
\pgfsetdash{}{0pt}%
\pgfpathmoveto{\pgfqpoint{10.344223in}{2.792941in}}%
\pgfpathlineto{\pgfqpoint{10.344223in}{3.237059in}}%
\pgfusepath{stroke}%
\end{pgfscope}%
\begin{pgfscope}%
\definecolor{textcolor}{rgb}{0.150000,0.150000,0.150000}%
\pgfsetstrokecolor{textcolor}%
\pgfsetfillcolor{textcolor}%
\pgftext[x=10.344223in,y=2.695719in,,top]{\color{textcolor}\rmfamily\fontsize{14.000000}{16.800000}\selectfont 2014}%
\end{pgfscope}%
\begin{pgfscope}%
\pgfpathrectangle{\pgfqpoint{8.656250in}{2.792941in}}{\pgfqpoint{4.843750in}{0.444118in}}%
\pgfusepath{clip}%
\pgfsetroundcap%
\pgfsetroundjoin%
\pgfsetlinewidth{0.803000pt}%
\definecolor{currentstroke}{rgb}{1.000000,1.000000,1.000000}%
\pgfsetstrokecolor{currentstroke}%
\pgfsetdash{}{0pt}%
\pgfpathmoveto{\pgfqpoint{11.079132in}{2.792941in}}%
\pgfpathlineto{\pgfqpoint{11.079132in}{3.237059in}}%
\pgfusepath{stroke}%
\end{pgfscope}%
\begin{pgfscope}%
\definecolor{textcolor}{rgb}{0.150000,0.150000,0.150000}%
\pgfsetstrokecolor{textcolor}%
\pgfsetfillcolor{textcolor}%
\pgftext[x=11.079132in,y=2.695719in,,top]{\color{textcolor}\rmfamily\fontsize{14.000000}{16.800000}\selectfont 2015}%
\end{pgfscope}%
\begin{pgfscope}%
\pgfpathrectangle{\pgfqpoint{8.656250in}{2.792941in}}{\pgfqpoint{4.843750in}{0.444118in}}%
\pgfusepath{clip}%
\pgfsetroundcap%
\pgfsetroundjoin%
\pgfsetlinewidth{0.803000pt}%
\definecolor{currentstroke}{rgb}{1.000000,1.000000,1.000000}%
\pgfsetstrokecolor{currentstroke}%
\pgfsetdash{}{0pt}%
\pgfpathmoveto{\pgfqpoint{11.814040in}{2.792941in}}%
\pgfpathlineto{\pgfqpoint{11.814040in}{3.237059in}}%
\pgfusepath{stroke}%
\end{pgfscope}%
\begin{pgfscope}%
\definecolor{textcolor}{rgb}{0.150000,0.150000,0.150000}%
\pgfsetstrokecolor{textcolor}%
\pgfsetfillcolor{textcolor}%
\pgftext[x=11.814040in,y=2.695719in,,top]{\color{textcolor}\rmfamily\fontsize{14.000000}{16.800000}\selectfont 2016}%
\end{pgfscope}%
\begin{pgfscope}%
\pgfpathrectangle{\pgfqpoint{8.656250in}{2.792941in}}{\pgfqpoint{4.843750in}{0.444118in}}%
\pgfusepath{clip}%
\pgfsetroundcap%
\pgfsetroundjoin%
\pgfsetlinewidth{0.803000pt}%
\definecolor{currentstroke}{rgb}{1.000000,1.000000,1.000000}%
\pgfsetstrokecolor{currentstroke}%
\pgfsetdash{}{0pt}%
\pgfpathmoveto{\pgfqpoint{12.550962in}{2.792941in}}%
\pgfpathlineto{\pgfqpoint{12.550962in}{3.237059in}}%
\pgfusepath{stroke}%
\end{pgfscope}%
\begin{pgfscope}%
\definecolor{textcolor}{rgb}{0.150000,0.150000,0.150000}%
\pgfsetstrokecolor{textcolor}%
\pgfsetfillcolor{textcolor}%
\pgftext[x=12.550962in,y=2.695719in,,top]{\color{textcolor}\rmfamily\fontsize{14.000000}{16.800000}\selectfont 2017}%
\end{pgfscope}%
\begin{pgfscope}%
\pgfpathrectangle{\pgfqpoint{8.656250in}{2.792941in}}{\pgfqpoint{4.843750in}{0.444118in}}%
\pgfusepath{clip}%
\pgfsetroundcap%
\pgfsetroundjoin%
\pgfsetlinewidth{0.803000pt}%
\definecolor{currentstroke}{rgb}{1.000000,1.000000,1.000000}%
\pgfsetstrokecolor{currentstroke}%
\pgfsetdash{}{0pt}%
\pgfpathmoveto{\pgfqpoint{13.285870in}{2.792941in}}%
\pgfpathlineto{\pgfqpoint{13.285870in}{3.237059in}}%
\pgfusepath{stroke}%
\end{pgfscope}%
\begin{pgfscope}%
\definecolor{textcolor}{rgb}{0.150000,0.150000,0.150000}%
\pgfsetstrokecolor{textcolor}%
\pgfsetfillcolor{textcolor}%
\pgftext[x=13.285870in,y=2.695719in,,top]{\color{textcolor}\rmfamily\fontsize{14.000000}{16.800000}\selectfont 2018}%
\end{pgfscope}%
\begin{pgfscope}%
\pgfpathrectangle{\pgfqpoint{8.656250in}{2.792941in}}{\pgfqpoint{4.843750in}{0.444118in}}%
\pgfusepath{clip}%
\pgfsetroundcap%
\pgfsetroundjoin%
\pgfsetlinewidth{0.803000pt}%
\definecolor{currentstroke}{rgb}{1.000000,1.000000,1.000000}%
\pgfsetstrokecolor{currentstroke}%
\pgfsetdash{}{0pt}%
\pgfpathmoveto{\pgfqpoint{8.656250in}{2.838169in}}%
\pgfpathlineto{\pgfqpoint{13.500000in}{2.838169in}}%
\pgfusepath{stroke}%
\end{pgfscope}%
\begin{pgfscope}%
\definecolor{textcolor}{rgb}{0.150000,0.150000,0.150000}%
\pgfsetstrokecolor{textcolor}%
\pgfsetfillcolor{textcolor}%
\pgftext[x=8.311605in,y=2.764303in,left,base]{\color{textcolor}\rmfamily\fontsize{14.000000}{16.800000}\selectfont 50}%
\end{pgfscope}%
\begin{pgfscope}%
\pgfpathrectangle{\pgfqpoint{8.656250in}{2.792941in}}{\pgfqpoint{4.843750in}{0.444118in}}%
\pgfusepath{clip}%
\pgfsetroundcap%
\pgfsetroundjoin%
\pgfsetlinewidth{0.803000pt}%
\definecolor{currentstroke}{rgb}{1.000000,1.000000,1.000000}%
\pgfsetstrokecolor{currentstroke}%
\pgfsetdash{}{0pt}%
\pgfpathmoveto{\pgfqpoint{8.656250in}{3.079874in}}%
\pgfpathlineto{\pgfqpoint{13.500000in}{3.079874in}}%
\pgfusepath{stroke}%
\end{pgfscope}%
\begin{pgfscope}%
\definecolor{textcolor}{rgb}{0.150000,0.150000,0.150000}%
\pgfsetstrokecolor{textcolor}%
\pgfsetfillcolor{textcolor}%
\pgftext[x=8.311605in,y=3.006007in,left,base]{\color{textcolor}\rmfamily\fontsize{14.000000}{16.800000}\selectfont 75}%
\end{pgfscope}%
\begin{pgfscope}%
\pgfpathrectangle{\pgfqpoint{8.656250in}{2.792941in}}{\pgfqpoint{4.843750in}{0.444118in}}%
\pgfusepath{clip}%
\pgfsetroundcap%
\pgfsetroundjoin%
\pgfsetlinewidth{1.505625pt}%
\definecolor{currentstroke}{rgb}{0.121569,0.466667,0.705882}%
\pgfsetstrokecolor{currentstroke}%
\pgfsetdash{}{0pt}%
\pgfpathmoveto{\pgfqpoint{8.876420in}{2.863113in}}%
\pgfpathlineto{\pgfqpoint{8.878434in}{2.862919in}}%
\pgfpathlineto{\pgfqpoint{8.880447in}{2.860792in}}%
\pgfpathlineto{\pgfqpoint{8.882461in}{2.859536in}}%
\pgfpathlineto{\pgfqpoint{8.888501in}{2.861663in}}%
\pgfpathlineto{\pgfqpoint{8.890515in}{2.859342in}}%
\pgfpathlineto{\pgfqpoint{8.892528in}{2.854411in}}%
\pgfpathlineto{\pgfqpoint{8.894541in}{2.855378in}}%
\pgfpathlineto{\pgfqpoint{8.896555in}{2.855378in}}%
\pgfpathlineto{\pgfqpoint{8.904609in}{2.858859in}}%
\pgfpathlineto{\pgfqpoint{8.906622in}{2.861083in}}%
\pgfpathlineto{\pgfqpoint{8.908636in}{2.861469in}}%
\pgfpathlineto{\pgfqpoint{8.910649in}{2.862629in}}%
\pgfpathlineto{\pgfqpoint{8.918703in}{2.849384in}}%
\pgfpathlineto{\pgfqpoint{8.920716in}{2.853058in}}%
\pgfpathlineto{\pgfqpoint{8.922730in}{2.851608in}}%
\pgfpathlineto{\pgfqpoint{8.924743in}{2.847837in}}%
\pgfpathlineto{\pgfqpoint{8.930784in}{2.839426in}}%
\pgfpathlineto{\pgfqpoint{8.932797in}{2.838169in}}%
\pgfpathlineto{\pgfqpoint{8.936824in}{2.840296in}}%
\pgfpathlineto{\pgfqpoint{8.938837in}{2.836042in}}%
\pgfpathlineto{\pgfqpoint{8.946891in}{2.843196in}}%
\pgfpathlineto{\pgfqpoint{8.948905in}{2.842713in}}%
\pgfpathlineto{\pgfqpoint{8.950918in}{2.845807in}}%
\pgfpathlineto{\pgfqpoint{8.952931in}{2.844550in}}%
\pgfpathlineto{\pgfqpoint{8.958972in}{2.847257in}}%
\pgfpathlineto{\pgfqpoint{8.960985in}{2.849191in}}%
\pgfpathlineto{\pgfqpoint{8.962999in}{2.849771in}}%
\pgfpathlineto{\pgfqpoint{8.965012in}{2.854702in}}%
\pgfpathlineto{\pgfqpoint{8.967026in}{2.852478in}}%
\pgfpathlineto{\pgfqpoint{8.975079in}{2.848707in}}%
\pgfpathlineto{\pgfqpoint{8.977093in}{2.848901in}}%
\pgfpathlineto{\pgfqpoint{8.979106in}{2.864080in}}%
\pgfpathlineto{\pgfqpoint{8.981120in}{2.866303in}}%
\pgfpathlineto{\pgfqpoint{8.987160in}{2.866207in}}%
\pgfpathlineto{\pgfqpoint{8.989173in}{2.871524in}}%
\pgfpathlineto{\pgfqpoint{8.991187in}{2.873264in}}%
\pgfpathlineto{\pgfqpoint{8.993200in}{2.865917in}}%
\pgfpathlineto{\pgfqpoint{8.995214in}{2.866013in}}%
\pgfpathlineto{\pgfqpoint{9.001254in}{2.868140in}}%
\pgfpathlineto{\pgfqpoint{9.003268in}{2.867270in}}%
\pgfpathlineto{\pgfqpoint{9.005281in}{2.865337in}}%
\pgfpathlineto{\pgfqpoint{9.007295in}{2.867754in}}%
\pgfpathlineto{\pgfqpoint{9.009308in}{2.867947in}}%
\pgfpathlineto{\pgfqpoint{9.017362in}{2.875391in}}%
\pgfpathlineto{\pgfqpoint{9.019375in}{2.875005in}}%
\pgfpathlineto{\pgfqpoint{9.021389in}{2.873748in}}%
\pgfpathlineto{\pgfqpoint{9.023402in}{2.870461in}}%
\pgfpathlineto{\pgfqpoint{9.033469in}{2.870074in}}%
\pgfpathlineto{\pgfqpoint{9.035483in}{2.872491in}}%
\pgfpathlineto{\pgfqpoint{9.037496in}{2.871814in}}%
\pgfpathlineto{\pgfqpoint{9.043537in}{2.872008in}}%
\pgfpathlineto{\pgfqpoint{9.045550in}{2.869687in}}%
\pgfpathlineto{\pgfqpoint{9.047563in}{2.869977in}}%
\pgfpathlineto{\pgfqpoint{9.049577in}{2.868624in}}%
\pgfpathlineto{\pgfqpoint{9.051590in}{2.870171in}}%
\pgfpathlineto{\pgfqpoint{9.057631in}{2.872781in}}%
\pgfpathlineto{\pgfqpoint{9.059644in}{2.869204in}}%
\pgfpathlineto{\pgfqpoint{9.061658in}{2.870557in}}%
\pgfpathlineto{\pgfqpoint{9.063671in}{2.870847in}}%
\pgfpathlineto{\pgfqpoint{9.071725in}{2.867077in}}%
\pgfpathlineto{\pgfqpoint{9.073738in}{2.863500in}}%
\pgfpathlineto{\pgfqpoint{9.075752in}{2.864176in}}%
\pgfpathlineto{\pgfqpoint{9.079779in}{2.859439in}}%
\pgfpathlineto{\pgfqpoint{9.087832in}{2.868624in}}%
\pgfpathlineto{\pgfqpoint{9.091859in}{2.865240in}}%
\pgfpathlineto{\pgfqpoint{9.093873in}{2.872394in}}%
\pgfpathlineto{\pgfqpoint{9.099913in}{2.865820in}}%
\pgfpathlineto{\pgfqpoint{9.103940in}{2.872008in}}%
\pgfpathlineto{\pgfqpoint{9.105953in}{2.871814in}}%
\pgfpathlineto{\pgfqpoint{9.107967in}{2.853058in}}%
\pgfpathlineto{\pgfqpoint{9.114007in}{2.846870in}}%
\pgfpathlineto{\pgfqpoint{9.116021in}{2.846387in}}%
\pgfpathlineto{\pgfqpoint{9.118034in}{2.849481in}}%
\pgfpathlineto{\pgfqpoint{9.120048in}{2.853638in}}%
\pgfpathlineto{\pgfqpoint{9.122061in}{2.851801in}}%
\pgfpathlineto{\pgfqpoint{9.128101in}{2.851608in}}%
\pgfpathlineto{\pgfqpoint{9.130115in}{2.850931in}}%
\pgfpathlineto{\pgfqpoint{9.132128in}{2.847160in}}%
\pgfpathlineto{\pgfqpoint{9.134142in}{2.850738in}}%
\pgfpathlineto{\pgfqpoint{9.136155in}{2.847160in}}%
\pgfpathlineto{\pgfqpoint{9.142195in}{2.846387in}}%
\pgfpathlineto{\pgfqpoint{9.144209in}{2.847547in}}%
\pgfpathlineto{\pgfqpoint{9.146222in}{2.851898in}}%
\pgfpathlineto{\pgfqpoint{9.150249in}{2.846000in}}%
\pgfpathlineto{\pgfqpoint{9.156290in}{2.844937in}}%
\pgfpathlineto{\pgfqpoint{9.158303in}{2.843100in}}%
\pgfpathlineto{\pgfqpoint{9.160317in}{2.837202in}}%
\pgfpathlineto{\pgfqpoint{9.162330in}{2.838652in}}%
\pgfpathlineto{\pgfqpoint{9.164343in}{2.837976in}}%
\pgfpathlineto{\pgfqpoint{9.172397in}{2.841553in}}%
\pgfpathlineto{\pgfqpoint{9.174411in}{2.836719in}}%
\pgfpathlineto{\pgfqpoint{9.176424in}{2.836429in}}%
\pgfpathlineto{\pgfqpoint{9.178438in}{2.830724in}}%
\pgfpathlineto{\pgfqpoint{9.184478in}{2.829564in}}%
\pgfpathlineto{\pgfqpoint{9.186491in}{2.827824in}}%
\pgfpathlineto{\pgfqpoint{9.188505in}{2.832658in}}%
\pgfpathlineto{\pgfqpoint{9.190518in}{2.840103in}}%
\pgfpathlineto{\pgfqpoint{9.192532in}{2.840006in}}%
\pgfpathlineto{\pgfqpoint{9.198572in}{2.838362in}}%
\pgfpathlineto{\pgfqpoint{9.200585in}{2.840103in}}%
\pgfpathlineto{\pgfqpoint{9.202599in}{2.838652in}}%
\pgfpathlineto{\pgfqpoint{9.204612in}{2.843293in}}%
\pgfpathlineto{\pgfqpoint{9.206626in}{2.840973in}}%
\pgfpathlineto{\pgfqpoint{9.212666in}{2.836429in}}%
\pgfpathlineto{\pgfqpoint{9.214680in}{2.835849in}}%
\pgfpathlineto{\pgfqpoint{9.216693in}{2.821733in}}%
\pgfpathlineto{\pgfqpoint{9.218706in}{2.816802in}}%
\pgfpathlineto{\pgfqpoint{9.220720in}{2.817382in}}%
\pgfpathlineto{\pgfqpoint{9.226760in}{2.813418in}}%
\pgfpathlineto{\pgfqpoint{9.228774in}{2.813128in}}%
\pgfpathlineto{\pgfqpoint{9.230787in}{2.818543in}}%
\pgfpathlineto{\pgfqpoint{9.232801in}{2.820960in}}%
\pgfpathlineto{\pgfqpoint{9.234814in}{2.828404in}}%
\pgfpathlineto{\pgfqpoint{9.240854in}{2.827921in}}%
\pgfpathlineto{\pgfqpoint{9.242868in}{2.829274in}}%
\pgfpathlineto{\pgfqpoint{9.248908in}{2.828597in}}%
\pgfpathlineto{\pgfqpoint{9.254949in}{2.830724in}}%
\pgfpathlineto{\pgfqpoint{9.256962in}{2.832078in}}%
\pgfpathlineto{\pgfqpoint{9.258975in}{2.829564in}}%
\pgfpathlineto{\pgfqpoint{9.260989in}{2.847354in}}%
\pgfpathlineto{\pgfqpoint{9.263002in}{2.858085in}}%
\pgfpathlineto{\pgfqpoint{9.269043in}{2.855958in}}%
\pgfpathlineto{\pgfqpoint{9.271056in}{2.860116in}}%
\pgfpathlineto{\pgfqpoint{9.273070in}{2.860309in}}%
\pgfpathlineto{\pgfqpoint{9.275083in}{2.861179in}}%
\pgfpathlineto{\pgfqpoint{9.277096in}{2.859632in}}%
\pgfpathlineto{\pgfqpoint{9.283137in}{2.857022in}}%
\pgfpathlineto{\pgfqpoint{9.285150in}{2.854121in}}%
\pgfpathlineto{\pgfqpoint{9.287164in}{2.854121in}}%
\pgfpathlineto{\pgfqpoint{9.291191in}{2.862436in}}%
\pgfpathlineto{\pgfqpoint{9.297231in}{2.862533in}}%
\pgfpathlineto{\pgfqpoint{9.303271in}{2.850157in}}%
\pgfpathlineto{\pgfqpoint{9.305285in}{2.865627in}}%
\pgfpathlineto{\pgfqpoint{9.311325in}{2.868044in}}%
\pgfpathlineto{\pgfqpoint{9.315352in}{2.875198in}}%
\pgfpathlineto{\pgfqpoint{9.319379in}{2.875585in}}%
\pgfpathlineto{\pgfqpoint{9.325419in}{2.873264in}}%
\pgfpathlineto{\pgfqpoint{9.327433in}{2.875295in}}%
\pgfpathlineto{\pgfqpoint{9.329446in}{2.874521in}}%
\pgfpathlineto{\pgfqpoint{9.331460in}{2.877325in}}%
\pgfpathlineto{\pgfqpoint{9.333473in}{2.877325in}}%
\pgfpathlineto{\pgfqpoint{9.339513in}{2.875391in}}%
\pgfpathlineto{\pgfqpoint{9.343540in}{2.876165in}}%
\pgfpathlineto{\pgfqpoint{9.345554in}{2.874908in}}%
\pgfpathlineto{\pgfqpoint{9.347567in}{2.877518in}}%
\pgfpathlineto{\pgfqpoint{9.353607in}{2.878195in}}%
\pgfpathlineto{\pgfqpoint{9.357634in}{2.876358in}}%
\pgfpathlineto{\pgfqpoint{9.359648in}{2.876455in}}%
\pgfpathlineto{\pgfqpoint{9.361661in}{2.878872in}}%
\pgfpathlineto{\pgfqpoint{9.369715in}{2.880516in}}%
\pgfpathlineto{\pgfqpoint{9.371728in}{2.879645in}}%
\pgfpathlineto{\pgfqpoint{9.373742in}{2.886993in}}%
\pgfpathlineto{\pgfqpoint{9.375755in}{2.889217in}}%
\pgfpathlineto{\pgfqpoint{9.381796in}{2.889120in}}%
\pgfpathlineto{\pgfqpoint{9.385823in}{2.885930in}}%
\pgfpathlineto{\pgfqpoint{9.387836in}{2.892214in}}%
\pgfpathlineto{\pgfqpoint{9.389849in}{2.894244in}}%
\pgfpathlineto{\pgfqpoint{9.399917in}{2.895018in}}%
\pgfpathlineto{\pgfqpoint{9.401930in}{2.897338in}}%
\pgfpathlineto{\pgfqpoint{9.403944in}{2.896275in}}%
\pgfpathlineto{\pgfqpoint{9.409984in}{2.898885in}}%
\pgfpathlineto{\pgfqpoint{9.411997in}{2.897532in}}%
\pgfpathlineto{\pgfqpoint{9.414011in}{2.895308in}}%
\pgfpathlineto{\pgfqpoint{9.418038in}{2.895791in}}%
\pgfpathlineto{\pgfqpoint{9.424078in}{2.896468in}}%
\pgfpathlineto{\pgfqpoint{9.426092in}{2.891344in}}%
\pgfpathlineto{\pgfqpoint{9.428105in}{2.894244in}}%
\pgfpathlineto{\pgfqpoint{9.432132in}{2.897918in}}%
\pgfpathlineto{\pgfqpoint{9.438172in}{2.893761in}}%
\pgfpathlineto{\pgfqpoint{9.440186in}{2.890570in}}%
\pgfpathlineto{\pgfqpoint{9.442199in}{2.886220in}}%
\pgfpathlineto{\pgfqpoint{9.446226in}{2.884673in}}%
\pgfpathlineto{\pgfqpoint{9.454280in}{2.892988in}}%
\pgfpathlineto{\pgfqpoint{9.456293in}{2.901109in}}%
\pgfpathlineto{\pgfqpoint{9.458307in}{2.901109in}}%
\pgfpathlineto{\pgfqpoint{9.460320in}{2.893954in}}%
\pgfpathlineto{\pgfqpoint{9.466360in}{2.893471in}}%
\pgfpathlineto{\pgfqpoint{9.468374in}{2.885156in}}%
\pgfpathlineto{\pgfqpoint{9.470387in}{2.890184in}}%
\pgfpathlineto{\pgfqpoint{9.472401in}{2.905750in}}%
\pgfpathlineto{\pgfqpoint{9.474414in}{2.900819in}}%
\pgfpathlineto{\pgfqpoint{9.488508in}{2.898885in}}%
\pgfpathlineto{\pgfqpoint{9.494549in}{2.894824in}}%
\pgfpathlineto{\pgfqpoint{9.496562in}{2.896855in}}%
\pgfpathlineto{\pgfqpoint{9.498576in}{2.889990in}}%
\pgfpathlineto{\pgfqpoint{9.500589in}{2.880902in}}%
\pgfpathlineto{\pgfqpoint{9.502603in}{2.881772in}}%
\pgfpathlineto{\pgfqpoint{9.508643in}{2.882256in}}%
\pgfpathlineto{\pgfqpoint{9.510656in}{2.880516in}}%
\pgfpathlineto{\pgfqpoint{9.512670in}{2.877905in}}%
\pgfpathlineto{\pgfqpoint{9.514683in}{2.876262in}}%
\pgfpathlineto{\pgfqpoint{9.516697in}{2.880226in}}%
\pgfpathlineto{\pgfqpoint{9.524750in}{2.891827in}}%
\pgfpathlineto{\pgfqpoint{9.526764in}{2.893181in}}%
\pgfpathlineto{\pgfqpoint{9.530791in}{2.901979in}}%
\pgfpathlineto{\pgfqpoint{9.536831in}{2.901109in}}%
\pgfpathlineto{\pgfqpoint{9.538845in}{2.897338in}}%
\pgfpathlineto{\pgfqpoint{9.540858in}{2.900819in}}%
\pgfpathlineto{\pgfqpoint{9.542871in}{2.901302in}}%
\pgfpathlineto{\pgfqpoint{9.544885in}{2.903913in}}%
\pgfpathlineto{\pgfqpoint{9.550925in}{2.901979in}}%
\pgfpathlineto{\pgfqpoint{9.552939in}{2.899852in}}%
\pgfpathlineto{\pgfqpoint{9.554952in}{2.900625in}}%
\pgfpathlineto{\pgfqpoint{9.556966in}{2.904879in}}%
\pgfpathlineto{\pgfqpoint{9.558979in}{2.907490in}}%
\pgfpathlineto{\pgfqpoint{9.565019in}{2.907006in}}%
\pgfpathlineto{\pgfqpoint{9.567033in}{2.910390in}}%
\pgfpathlineto{\pgfqpoint{9.569046in}{2.911260in}}%
\pgfpathlineto{\pgfqpoint{9.571060in}{2.906233in}}%
\pgfpathlineto{\pgfqpoint{9.573073in}{2.904686in}}%
\pgfpathlineto{\pgfqpoint{9.581127in}{2.904976in}}%
\pgfpathlineto{\pgfqpoint{9.583140in}{2.900045in}}%
\pgfpathlineto{\pgfqpoint{9.585154in}{2.903816in}}%
\pgfpathlineto{\pgfqpoint{9.587167in}{2.895211in}}%
\pgfpathlineto{\pgfqpoint{9.593208in}{2.893568in}}%
\pgfpathlineto{\pgfqpoint{9.597235in}{2.889507in}}%
\pgfpathlineto{\pgfqpoint{9.599248in}{2.889314in}}%
\pgfpathlineto{\pgfqpoint{9.601261in}{2.882836in}}%
\pgfpathlineto{\pgfqpoint{9.607302in}{2.888637in}}%
\pgfpathlineto{\pgfqpoint{9.611329in}{2.900432in}}%
\pgfpathlineto{\pgfqpoint{9.613342in}{2.896951in}}%
\pgfpathlineto{\pgfqpoint{9.615356in}{2.898112in}}%
\pgfpathlineto{\pgfqpoint{9.623409in}{2.893568in}}%
\pgfpathlineto{\pgfqpoint{9.627436in}{2.899465in}}%
\pgfpathlineto{\pgfqpoint{9.629450in}{2.899078in}}%
\pgfpathlineto{\pgfqpoint{9.635490in}{2.902366in}}%
\pgfpathlineto{\pgfqpoint{9.637504in}{2.904299in}}%
\pgfpathlineto{\pgfqpoint{9.639517in}{2.904493in}}%
\pgfpathlineto{\pgfqpoint{9.643544in}{2.909230in}}%
\pgfpathlineto{\pgfqpoint{9.651598in}{2.909327in}}%
\pgfpathlineto{\pgfqpoint{9.653611in}{2.915128in}}%
\pgfpathlineto{\pgfqpoint{9.655625in}{2.913001in}}%
\pgfpathlineto{\pgfqpoint{9.657638in}{2.935431in}}%
\pgfpathlineto{\pgfqpoint{9.663678in}{2.939588in}}%
\pgfpathlineto{\pgfqpoint{9.665692in}{2.949353in}}%
\pgfpathlineto{\pgfqpoint{9.669719in}{2.950610in}}%
\pgfpathlineto{\pgfqpoint{9.671732in}{2.956604in}}%
\pgfpathlineto{\pgfqpoint{9.677772in}{2.951287in}}%
\pgfpathlineto{\pgfqpoint{9.681799in}{2.958441in}}%
\pgfpathlineto{\pgfqpoint{9.683813in}{2.958441in}}%
\pgfpathlineto{\pgfqpoint{9.685826in}{2.955251in}}%
\pgfpathlineto{\pgfqpoint{9.691867in}{2.955734in}}%
\pgfpathlineto{\pgfqpoint{9.693880in}{2.957088in}}%
\pgfpathlineto{\pgfqpoint{9.695893in}{2.961728in}}%
\pgfpathlineto{\pgfqpoint{9.697907in}{2.963469in}}%
\pgfpathlineto{\pgfqpoint{9.699920in}{2.961535in}}%
\pgfpathlineto{\pgfqpoint{9.707974in}{2.968206in}}%
\pgfpathlineto{\pgfqpoint{9.709988in}{2.965789in}}%
\pgfpathlineto{\pgfqpoint{9.714015in}{2.965112in}}%
\pgfpathlineto{\pgfqpoint{9.720055in}{2.956604in}}%
\pgfpathlineto{\pgfqpoint{9.722068in}{2.957861in}}%
\pgfpathlineto{\pgfqpoint{9.724082in}{2.963179in}}%
\pgfpathlineto{\pgfqpoint{9.726095in}{2.958731in}}%
\pgfpathlineto{\pgfqpoint{9.728109in}{2.961148in}}%
\pgfpathlineto{\pgfqpoint{9.734149in}{2.962695in}}%
\pgfpathlineto{\pgfqpoint{9.736162in}{2.965596in}}%
\pgfpathlineto{\pgfqpoint{9.738176in}{2.966756in}}%
\pgfpathlineto{\pgfqpoint{9.740189in}{2.964435in}}%
\pgfpathlineto{\pgfqpoint{9.742203in}{2.966659in}}%
\pgfpathlineto{\pgfqpoint{9.748243in}{2.968013in}}%
\pgfpathlineto{\pgfqpoint{9.750257in}{2.966562in}}%
\pgfpathlineto{\pgfqpoint{9.752270in}{2.963565in}}%
\pgfpathlineto{\pgfqpoint{9.754283in}{2.968303in}}%
\pgfpathlineto{\pgfqpoint{9.756297in}{2.959988in}}%
\pgfpathlineto{\pgfqpoint{9.762337in}{2.958538in}}%
\pgfpathlineto{\pgfqpoint{9.764351in}{2.966079in}}%
\pgfpathlineto{\pgfqpoint{9.766364in}{2.969753in}}%
\pgfpathlineto{\pgfqpoint{9.768378in}{2.966852in}}%
\pgfpathlineto{\pgfqpoint{9.770391in}{2.967336in}}%
\pgfpathlineto{\pgfqpoint{9.776431in}{2.962695in}}%
\pgfpathlineto{\pgfqpoint{9.778445in}{2.968399in}}%
\pgfpathlineto{\pgfqpoint{9.780458in}{2.965692in}}%
\pgfpathlineto{\pgfqpoint{9.782472in}{2.965692in}}%
\pgfpathlineto{\pgfqpoint{9.790525in}{2.970720in}}%
\pgfpathlineto{\pgfqpoint{9.792539in}{2.980775in}}%
\pgfpathlineto{\pgfqpoint{9.794552in}{2.974104in}}%
\pgfpathlineto{\pgfqpoint{9.796566in}{2.977391in}}%
\pgfpathlineto{\pgfqpoint{9.798579in}{2.974974in}}%
\pgfpathlineto{\pgfqpoint{9.804620in}{2.979421in}}%
\pgfpathlineto{\pgfqpoint{9.806633in}{2.975167in}}%
\pgfpathlineto{\pgfqpoint{9.808647in}{2.982998in}}%
\pgfpathlineto{\pgfqpoint{9.812673in}{2.989573in}}%
\pgfpathlineto{\pgfqpoint{9.818714in}{2.986189in}}%
\pgfpathlineto{\pgfqpoint{9.820727in}{2.989766in}}%
\pgfpathlineto{\pgfqpoint{9.822741in}{2.981548in}}%
\pgfpathlineto{\pgfqpoint{9.824754in}{2.987929in}}%
\pgfpathlineto{\pgfqpoint{9.826768in}{3.000304in}}%
\pgfpathlineto{\pgfqpoint{9.832808in}{3.000111in}}%
\pgfpathlineto{\pgfqpoint{9.834821in}{3.009102in}}%
\pgfpathlineto{\pgfqpoint{9.836835in}{2.970623in}}%
\pgfpathlineto{\pgfqpoint{9.838848in}{2.966369in}}%
\pgfpathlineto{\pgfqpoint{9.840862in}{2.970526in}}%
\pgfpathlineto{\pgfqpoint{9.846902in}{2.975070in}}%
\pgfpathlineto{\pgfqpoint{9.848915in}{2.967819in}}%
\pgfpathlineto{\pgfqpoint{9.850929in}{2.969560in}}%
\pgfpathlineto{\pgfqpoint{9.852942in}{2.975747in}}%
\pgfpathlineto{\pgfqpoint{9.854956in}{2.979228in}}%
\pgfpathlineto{\pgfqpoint{9.860996in}{2.975650in}}%
\pgfpathlineto{\pgfqpoint{9.863010in}{2.977197in}}%
\pgfpathlineto{\pgfqpoint{9.865023in}{2.981161in}}%
\pgfpathlineto{\pgfqpoint{9.867036in}{2.979324in}}%
\pgfpathlineto{\pgfqpoint{9.869050in}{2.983772in}}%
\pgfpathlineto{\pgfqpoint{9.875090in}{2.982418in}}%
\pgfpathlineto{\pgfqpoint{9.877104in}{2.989379in}}%
\pgfpathlineto{\pgfqpoint{9.879117in}{2.999047in}}%
\pgfpathlineto{\pgfqpoint{9.881131in}{2.995277in}}%
\pgfpathlineto{\pgfqpoint{9.883144in}{2.993827in}}%
\pgfpathlineto{\pgfqpoint{9.891198in}{2.984062in}}%
\pgfpathlineto{\pgfqpoint{9.893211in}{2.984255in}}%
\pgfpathlineto{\pgfqpoint{9.895225in}{2.983288in}}%
\pgfpathlineto{\pgfqpoint{9.897238in}{3.008619in}}%
\pgfpathlineto{\pgfqpoint{9.905292in}{3.000498in}}%
\pgfpathlineto{\pgfqpoint{9.907305in}{2.984835in}}%
\pgfpathlineto{\pgfqpoint{9.909319in}{2.986382in}}%
\pgfpathlineto{\pgfqpoint{9.911332in}{2.967723in}}%
\pgfpathlineto{\pgfqpoint{9.917373in}{2.974974in}}%
\pgfpathlineto{\pgfqpoint{9.919386in}{2.972653in}}%
\pgfpathlineto{\pgfqpoint{9.921400in}{2.966949in}}%
\pgfpathlineto{\pgfqpoint{9.923413in}{2.968206in}}%
\pgfpathlineto{\pgfqpoint{9.925426in}{2.975650in}}%
\pgfpathlineto{\pgfqpoint{9.933480in}{2.978648in}}%
\pgfpathlineto{\pgfqpoint{9.935494in}{2.974490in}}%
\pgfpathlineto{\pgfqpoint{9.937507in}{2.981065in}}%
\pgfpathlineto{\pgfqpoint{9.939521in}{2.977874in}}%
\pgfpathlineto{\pgfqpoint{9.945561in}{2.985319in}}%
\pgfpathlineto{\pgfqpoint{9.947574in}{2.985995in}}%
\pgfpathlineto{\pgfqpoint{9.949588in}{2.974587in}}%
\pgfpathlineto{\pgfqpoint{9.951601in}{2.955734in}}%
\pgfpathlineto{\pgfqpoint{9.953615in}{2.973137in}}%
\pgfpathlineto{\pgfqpoint{9.959655in}{2.966369in}}%
\pgfpathlineto{\pgfqpoint{9.961669in}{2.967239in}}%
\pgfpathlineto{\pgfqpoint{9.963682in}{2.972847in}}%
\pgfpathlineto{\pgfqpoint{9.965695in}{2.975070in}}%
\pgfpathlineto{\pgfqpoint{9.967709in}{2.969560in}}%
\pgfpathlineto{\pgfqpoint{9.975763in}{2.981161in}}%
\pgfpathlineto{\pgfqpoint{9.977776in}{2.982225in}}%
\pgfpathlineto{\pgfqpoint{9.981803in}{2.980388in}}%
\pgfpathlineto{\pgfqpoint{9.987843in}{2.983772in}}%
\pgfpathlineto{\pgfqpoint{9.989857in}{2.990249in}}%
\pgfpathlineto{\pgfqpoint{9.991870in}{2.992086in}}%
\pgfpathlineto{\pgfqpoint{9.995897in}{3.006009in}}%
\pgfpathlineto{\pgfqpoint{10.001937in}{3.005719in}}%
\pgfpathlineto{\pgfqpoint{10.003951in}{3.001368in}}%
\pgfpathlineto{\pgfqpoint{10.005964in}{2.999144in}}%
\pgfpathlineto{\pgfqpoint{10.007978in}{3.000788in}}%
\pgfpathlineto{\pgfqpoint{10.009991in}{3.009489in}}%
\pgfpathlineto{\pgfqpoint{10.016032in}{3.008232in}}%
\pgfpathlineto{\pgfqpoint{10.018045in}{3.006105in}}%
\pgfpathlineto{\pgfqpoint{10.020058in}{3.000594in}}%
\pgfpathlineto{\pgfqpoint{10.022072in}{3.001755in}}%
\pgfpathlineto{\pgfqpoint{10.024085in}{3.001561in}}%
\pgfpathlineto{\pgfqpoint{10.030126in}{2.999144in}}%
\pgfpathlineto{\pgfqpoint{10.032139in}{3.001948in}}%
\pgfpathlineto{\pgfqpoint{10.034153in}{3.000788in}}%
\pgfpathlineto{\pgfqpoint{10.036166in}{3.011616in}}%
\pgfpathlineto{\pgfqpoint{10.038180in}{3.008812in}}%
\pgfpathlineto{\pgfqpoint{10.044220in}{3.009682in}}%
\pgfpathlineto{\pgfqpoint{10.048247in}{3.014227in}}%
\pgfpathlineto{\pgfqpoint{10.050260in}{3.015870in}}%
\pgfpathlineto{\pgfqpoint{10.052274in}{3.011616in}}%
\pgfpathlineto{\pgfqpoint{10.060327in}{3.011809in}}%
\pgfpathlineto{\pgfqpoint{10.062341in}{3.008522in}}%
\pgfpathlineto{\pgfqpoint{10.066368in}{2.997597in}}%
\pgfpathlineto{\pgfqpoint{10.076435in}{2.993440in}}%
\pgfpathlineto{\pgfqpoint{10.078448in}{2.996534in}}%
\pgfpathlineto{\pgfqpoint{10.080462in}{2.998467in}}%
\pgfpathlineto{\pgfqpoint{10.088516in}{2.982128in}}%
\pgfpathlineto{\pgfqpoint{10.090529in}{2.973040in}}%
\pgfpathlineto{\pgfqpoint{10.094556in}{2.981451in}}%
\pgfpathlineto{\pgfqpoint{10.102610in}{2.980291in}}%
\pgfpathlineto{\pgfqpoint{10.104623in}{2.978261in}}%
\pgfpathlineto{\pgfqpoint{10.106637in}{2.975360in}}%
\pgfpathlineto{\pgfqpoint{10.108650in}{2.975457in}}%
\pgfpathlineto{\pgfqpoint{10.114691in}{2.983578in}}%
\pgfpathlineto{\pgfqpoint{10.116704in}{2.981935in}}%
\pgfpathlineto{\pgfqpoint{10.118717in}{2.984545in}}%
\pgfpathlineto{\pgfqpoint{10.120731in}{2.984449in}}%
\pgfpathlineto{\pgfqpoint{10.122744in}{2.990733in}}%
\pgfpathlineto{\pgfqpoint{10.128785in}{2.999724in}}%
\pgfpathlineto{\pgfqpoint{10.130798in}{2.997017in}}%
\pgfpathlineto{\pgfqpoint{10.132812in}{3.000788in}}%
\pgfpathlineto{\pgfqpoint{10.134825in}{2.999338in}}%
\pgfpathlineto{\pgfqpoint{10.136838in}{2.993537in}}%
\pgfpathlineto{\pgfqpoint{10.142879in}{2.992666in}}%
\pgfpathlineto{\pgfqpoint{10.144892in}{2.987349in}}%
\pgfpathlineto{\pgfqpoint{10.146906in}{2.980098in}}%
\pgfpathlineto{\pgfqpoint{10.148919in}{2.982708in}}%
\pgfpathlineto{\pgfqpoint{10.150933in}{2.975941in}}%
\pgfpathlineto{\pgfqpoint{10.156973in}{2.962985in}}%
\pgfpathlineto{\pgfqpoint{10.158986in}{2.967529in}}%
\pgfpathlineto{\pgfqpoint{10.161000in}{2.965692in}}%
\pgfpathlineto{\pgfqpoint{10.163013in}{2.964919in}}%
\pgfpathlineto{\pgfqpoint{10.165027in}{2.966369in}}%
\pgfpathlineto{\pgfqpoint{10.171067in}{2.963372in}}%
\pgfpathlineto{\pgfqpoint{10.179121in}{2.986189in}}%
\pgfpathlineto{\pgfqpoint{10.185161in}{2.988316in}}%
\pgfpathlineto{\pgfqpoint{10.187175in}{2.979131in}}%
\pgfpathlineto{\pgfqpoint{10.191201in}{2.998757in}}%
\pgfpathlineto{\pgfqpoint{10.193215in}{2.998661in}}%
\pgfpathlineto{\pgfqpoint{10.199255in}{2.995084in}}%
\pgfpathlineto{\pgfqpoint{10.201269in}{3.006492in}}%
\pgfpathlineto{\pgfqpoint{10.203282in}{3.010843in}}%
\pgfpathlineto{\pgfqpoint{10.205296in}{3.008426in}}%
\pgfpathlineto{\pgfqpoint{10.207309in}{3.003495in}}%
\pgfpathlineto{\pgfqpoint{10.213349in}{3.014033in}}%
\pgfpathlineto{\pgfqpoint{10.215363in}{3.023411in}}%
\pgfpathlineto{\pgfqpoint{10.219390in}{3.009489in}}%
\pgfpathlineto{\pgfqpoint{10.221403in}{3.012776in}}%
\pgfpathlineto{\pgfqpoint{10.229457in}{3.015000in}}%
\pgfpathlineto{\pgfqpoint{10.231470in}{3.026215in}}%
\pgfpathlineto{\pgfqpoint{10.233484in}{3.022251in}}%
\pgfpathlineto{\pgfqpoint{10.235497in}{3.023798in}}%
\pgfpathlineto{\pgfqpoint{10.241538in}{3.021768in}}%
\pgfpathlineto{\pgfqpoint{10.243551in}{3.026118in}}%
\pgfpathlineto{\pgfqpoint{10.249591in}{3.042651in}}%
\pgfpathlineto{\pgfqpoint{10.255632in}{3.040524in}}%
\pgfpathlineto{\pgfqpoint{10.257645in}{3.038590in}}%
\pgfpathlineto{\pgfqpoint{10.259659in}{3.041491in}}%
\pgfpathlineto{\pgfqpoint{10.261672in}{3.041297in}}%
\pgfpathlineto{\pgfqpoint{10.263686in}{3.043618in}}%
\pgfpathlineto{\pgfqpoint{10.269726in}{3.047292in}}%
\pgfpathlineto{\pgfqpoint{10.271739in}{3.041104in}}%
\pgfpathlineto{\pgfqpoint{10.273753in}{3.038107in}}%
\pgfpathlineto{\pgfqpoint{10.277780in}{3.037624in}}%
\pgfpathlineto{\pgfqpoint{10.283820in}{3.030566in}}%
\pgfpathlineto{\pgfqpoint{10.285834in}{3.034530in}}%
\pgfpathlineto{\pgfqpoint{10.287847in}{3.030566in}}%
\pgfpathlineto{\pgfqpoint{10.289860in}{3.025248in}}%
\pgfpathlineto{\pgfqpoint{10.291874in}{3.040137in}}%
\pgfpathlineto{\pgfqpoint{10.297914in}{3.042168in}}%
\pgfpathlineto{\pgfqpoint{10.299928in}{3.033080in}}%
\pgfpathlineto{\pgfqpoint{10.301941in}{3.036077in}}%
\pgfpathlineto{\pgfqpoint{10.303955in}{3.022058in}}%
\pgfpathlineto{\pgfqpoint{10.305968in}{3.022638in}}%
\pgfpathlineto{\pgfqpoint{10.312008in}{3.017127in}}%
\pgfpathlineto{\pgfqpoint{10.314022in}{3.010843in}}%
\pgfpathlineto{\pgfqpoint{10.316035in}{3.022735in}}%
\pgfpathlineto{\pgfqpoint{10.318049in}{3.018867in}}%
\pgfpathlineto{\pgfqpoint{10.320062in}{3.018384in}}%
\pgfpathlineto{\pgfqpoint{10.326102in}{3.014033in}}%
\pgfpathlineto{\pgfqpoint{10.328116in}{3.014033in}}%
\pgfpathlineto{\pgfqpoint{10.334156in}{3.019737in}}%
\pgfpathlineto{\pgfqpoint{10.340197in}{3.019641in}}%
\pgfpathlineto{\pgfqpoint{10.342210in}{3.014903in}}%
\pgfpathlineto{\pgfqpoint{10.346237in}{3.007846in}}%
\pgfpathlineto{\pgfqpoint{10.348250in}{3.007072in}}%
\pgfpathlineto{\pgfqpoint{10.354291in}{3.008619in}}%
\pgfpathlineto{\pgfqpoint{10.356304in}{3.015000in}}%
\pgfpathlineto{\pgfqpoint{10.358318in}{3.005428in}}%
\pgfpathlineto{\pgfqpoint{10.360331in}{3.006879in}}%
\pgfpathlineto{\pgfqpoint{10.368385in}{3.003495in}}%
\pgfpathlineto{\pgfqpoint{10.370398in}{3.010456in}}%
\pgfpathlineto{\pgfqpoint{10.372412in}{3.009876in}}%
\pgfpathlineto{\pgfqpoint{10.374425in}{3.007942in}}%
\pgfpathlineto{\pgfqpoint{10.376439in}{3.002431in}}%
\pgfpathlineto{\pgfqpoint{10.384492in}{3.004945in}}%
\pgfpathlineto{\pgfqpoint{10.386506in}{3.002045in}}%
\pgfpathlineto{\pgfqpoint{10.388519in}{2.993923in}}%
\pgfpathlineto{\pgfqpoint{10.390533in}{3.001658in}}%
\pgfpathlineto{\pgfqpoint{10.396573in}{2.995857in}}%
\pgfpathlineto{\pgfqpoint{10.398587in}{3.001078in}}%
\pgfpathlineto{\pgfqpoint{10.400600in}{2.989089in}}%
\pgfpathlineto{\pgfqpoint{10.402613in}{2.982805in}}%
\pgfpathlineto{\pgfqpoint{10.404627in}{2.980775in}}%
\pgfpathlineto{\pgfqpoint{10.410667in}{2.973233in}}%
\pgfpathlineto{\pgfqpoint{10.418721in}{2.986382in}}%
\pgfpathlineto{\pgfqpoint{10.424761in}{2.992280in}}%
\pgfpathlineto{\pgfqpoint{10.426775in}{2.998854in}}%
\pgfpathlineto{\pgfqpoint{10.428788in}{2.987832in}}%
\pgfpathlineto{\pgfqpoint{10.430802in}{2.990346in}}%
\pgfpathlineto{\pgfqpoint{10.432815in}{3.003495in}}%
\pgfpathlineto{\pgfqpoint{10.440869in}{2.991796in}}%
\pgfpathlineto{\pgfqpoint{10.442882in}{2.993150in}}%
\pgfpathlineto{\pgfqpoint{10.444896in}{2.991313in}}%
\pgfpathlineto{\pgfqpoint{10.446909in}{2.991796in}}%
\pgfpathlineto{\pgfqpoint{10.452950in}{2.990830in}}%
\pgfpathlineto{\pgfqpoint{10.454963in}{2.992957in}}%
\pgfpathlineto{\pgfqpoint{10.456977in}{2.990830in}}%
\pgfpathlineto{\pgfqpoint{10.458990in}{2.993537in}}%
\pgfpathlineto{\pgfqpoint{10.461003in}{2.997404in}}%
\pgfpathlineto{\pgfqpoint{10.467044in}{2.987736in}}%
\pgfpathlineto{\pgfqpoint{10.469057in}{2.995664in}}%
\pgfpathlineto{\pgfqpoint{10.471071in}{2.990539in}}%
\pgfpathlineto{\pgfqpoint{10.475098in}{2.995084in}}%
\pgfpathlineto{\pgfqpoint{10.481138in}{2.996340in}}%
\pgfpathlineto{\pgfqpoint{10.485165in}{3.002141in}}%
\pgfpathlineto{\pgfqpoint{10.487178in}{3.001755in}}%
\pgfpathlineto{\pgfqpoint{10.489192in}{3.000014in}}%
\pgfpathlineto{\pgfqpoint{10.495232in}{3.007072in}}%
\pgfpathlineto{\pgfqpoint{10.497245in}{3.006492in}}%
\pgfpathlineto{\pgfqpoint{10.499259in}{2.998371in}}%
\pgfpathlineto{\pgfqpoint{10.503286in}{2.991023in}}%
\pgfpathlineto{\pgfqpoint{10.511340in}{3.006782in}}%
\pgfpathlineto{\pgfqpoint{10.513353in}{3.004268in}}%
\pgfpathlineto{\pgfqpoint{10.517380in}{3.006395in}}%
\pgfpathlineto{\pgfqpoint{10.523420in}{3.013260in}}%
\pgfpathlineto{\pgfqpoint{10.527447in}{3.009392in}}%
\pgfpathlineto{\pgfqpoint{10.529461in}{3.009199in}}%
\pgfpathlineto{\pgfqpoint{10.531474in}{3.006492in}}%
\pgfpathlineto{\pgfqpoint{10.537514in}{3.012390in}}%
\pgfpathlineto{\pgfqpoint{10.539528in}{3.019351in}}%
\pgfpathlineto{\pgfqpoint{10.541541in}{3.020511in}}%
\pgfpathlineto{\pgfqpoint{10.545568in}{3.014517in}}%
\pgfpathlineto{\pgfqpoint{10.553622in}{3.015193in}}%
\pgfpathlineto{\pgfqpoint{10.555635in}{3.021864in}}%
\pgfpathlineto{\pgfqpoint{10.557649in}{3.022735in}}%
\pgfpathlineto{\pgfqpoint{10.565703in}{3.021091in}}%
\pgfpathlineto{\pgfqpoint{10.569730in}{3.016547in}}%
\pgfpathlineto{\pgfqpoint{10.571743in}{3.023025in}}%
\pgfpathlineto{\pgfqpoint{10.573756in}{3.025152in}}%
\pgfpathlineto{\pgfqpoint{10.579797in}{3.037817in}}%
\pgfpathlineto{\pgfqpoint{10.581810in}{3.033660in}}%
\pgfpathlineto{\pgfqpoint{10.583824in}{3.034530in}}%
\pgfpathlineto{\pgfqpoint{10.585837in}{3.032886in}}%
\pgfpathlineto{\pgfqpoint{10.587851in}{3.029406in}}%
\pgfpathlineto{\pgfqpoint{10.593891in}{3.027569in}}%
\pgfpathlineto{\pgfqpoint{10.595904in}{3.022831in}}%
\pgfpathlineto{\pgfqpoint{10.597918in}{3.030759in}}%
\pgfpathlineto{\pgfqpoint{10.599931in}{3.031339in}}%
\pgfpathlineto{\pgfqpoint{10.601945in}{3.033273in}}%
\pgfpathlineto{\pgfqpoint{10.609999in}{3.026795in}}%
\pgfpathlineto{\pgfqpoint{10.612012in}{3.023218in}}%
\pgfpathlineto{\pgfqpoint{10.614025in}{3.017900in}}%
\pgfpathlineto{\pgfqpoint{10.616039in}{3.016257in}}%
\pgfpathlineto{\pgfqpoint{10.622079in}{3.012970in}}%
\pgfpathlineto{\pgfqpoint{10.626106in}{3.017610in}}%
\pgfpathlineto{\pgfqpoint{10.628120in}{3.018964in}}%
\pgfpathlineto{\pgfqpoint{10.630133in}{3.017804in}}%
\pgfpathlineto{\pgfqpoint{10.638187in}{3.014227in}}%
\pgfpathlineto{\pgfqpoint{10.640200in}{3.014420in}}%
\pgfpathlineto{\pgfqpoint{10.644227in}{3.020027in}}%
\pgfpathlineto{\pgfqpoint{10.650267in}{3.016547in}}%
\pgfpathlineto{\pgfqpoint{10.652281in}{3.012970in}}%
\pgfpathlineto{\pgfqpoint{10.654294in}{3.012390in}}%
\pgfpathlineto{\pgfqpoint{10.656308in}{3.014517in}}%
\pgfpathlineto{\pgfqpoint{10.658321in}{3.013840in}}%
\pgfpathlineto{\pgfqpoint{10.666375in}{3.014807in}}%
\pgfpathlineto{\pgfqpoint{10.668388in}{3.014033in}}%
\pgfpathlineto{\pgfqpoint{10.670402in}{3.011616in}}%
\pgfpathlineto{\pgfqpoint{10.672415in}{3.010649in}}%
\pgfpathlineto{\pgfqpoint{10.678456in}{3.011036in}}%
\pgfpathlineto{\pgfqpoint{10.680469in}{3.010069in}}%
\pgfpathlineto{\pgfqpoint{10.682483in}{3.011809in}}%
\pgfpathlineto{\pgfqpoint{10.684496in}{3.015580in}}%
\pgfpathlineto{\pgfqpoint{10.686510in}{3.012970in}}%
\pgfpathlineto{\pgfqpoint{10.692550in}{3.009586in}}%
\pgfpathlineto{\pgfqpoint{10.694563in}{3.005428in}}%
\pgfpathlineto{\pgfqpoint{10.696577in}{3.007942in}}%
\pgfpathlineto{\pgfqpoint{10.698590in}{3.002238in}}%
\pgfpathlineto{\pgfqpoint{10.700604in}{3.005525in}}%
\pgfpathlineto{\pgfqpoint{10.706644in}{3.001948in}}%
\pgfpathlineto{\pgfqpoint{10.708657in}{3.007652in}}%
\pgfpathlineto{\pgfqpoint{10.710671in}{3.009973in}}%
\pgfpathlineto{\pgfqpoint{10.712684in}{3.013356in}}%
\pgfpathlineto{\pgfqpoint{10.720738in}{3.015097in}}%
\pgfpathlineto{\pgfqpoint{10.722752in}{3.018191in}}%
\pgfpathlineto{\pgfqpoint{10.724765in}{3.027279in}}%
\pgfpathlineto{\pgfqpoint{10.726778in}{3.026795in}}%
\pgfpathlineto{\pgfqpoint{10.728792in}{3.023121in}}%
\pgfpathlineto{\pgfqpoint{10.734832in}{3.024475in}}%
\pgfpathlineto{\pgfqpoint{10.736846in}{3.023991in}}%
\pgfpathlineto{\pgfqpoint{10.738859in}{3.026602in}}%
\pgfpathlineto{\pgfqpoint{10.740873in}{3.022154in}}%
\pgfpathlineto{\pgfqpoint{10.742886in}{3.023411in}}%
\pgfpathlineto{\pgfqpoint{10.748926in}{3.021188in}}%
\pgfpathlineto{\pgfqpoint{10.752953in}{3.018771in}}%
\pgfpathlineto{\pgfqpoint{10.754967in}{3.020994in}}%
\pgfpathlineto{\pgfqpoint{10.756980in}{3.015193in}}%
\pgfpathlineto{\pgfqpoint{10.763021in}{3.012680in}}%
\pgfpathlineto{\pgfqpoint{10.769061in}{2.996630in}}%
\pgfpathlineto{\pgfqpoint{10.771074in}{3.015967in}}%
\pgfpathlineto{\pgfqpoint{10.777115in}{3.012390in}}%
\pgfpathlineto{\pgfqpoint{10.779128in}{3.013936in}}%
\pgfpathlineto{\pgfqpoint{10.781142in}{3.027859in}}%
\pgfpathlineto{\pgfqpoint{10.783155in}{3.020027in}}%
\pgfpathlineto{\pgfqpoint{10.785168in}{3.026699in}}%
\pgfpathlineto{\pgfqpoint{10.791209in}{3.031146in}}%
\pgfpathlineto{\pgfqpoint{10.793222in}{3.030662in}}%
\pgfpathlineto{\pgfqpoint{10.795236in}{3.031146in}}%
\pgfpathlineto{\pgfqpoint{10.797249in}{3.035013in}}%
\pgfpathlineto{\pgfqpoint{10.799263in}{3.033563in}}%
\pgfpathlineto{\pgfqpoint{10.809330in}{3.042168in}}%
\pgfpathlineto{\pgfqpoint{10.811343in}{3.046035in}}%
\pgfpathlineto{\pgfqpoint{10.813357in}{3.047002in}}%
\pgfpathlineto{\pgfqpoint{10.819397in}{3.048259in}}%
\pgfpathlineto{\pgfqpoint{10.821410in}{3.046905in}}%
\pgfpathlineto{\pgfqpoint{10.823424in}{3.046325in}}%
\pgfpathlineto{\pgfqpoint{10.825437in}{3.044005in}}%
\pgfpathlineto{\pgfqpoint{10.827451in}{3.044681in}}%
\pgfpathlineto{\pgfqpoint{10.837518in}{3.042941in}}%
\pgfpathlineto{\pgfqpoint{10.839532in}{3.049515in}}%
\pgfpathlineto{\pgfqpoint{10.841545in}{3.050096in}}%
\pgfpathlineto{\pgfqpoint{10.847585in}{3.046422in}}%
\pgfpathlineto{\pgfqpoint{10.849599in}{3.043618in}}%
\pgfpathlineto{\pgfqpoint{10.851612in}{3.049032in}}%
\pgfpathlineto{\pgfqpoint{10.853626in}{3.047775in}}%
\pgfpathlineto{\pgfqpoint{10.855639in}{3.045842in}}%
\pgfpathlineto{\pgfqpoint{10.863693in}{3.052706in}}%
\pgfpathlineto{\pgfqpoint{10.867720in}{3.053576in}}%
\pgfpathlineto{\pgfqpoint{10.869733in}{3.055896in}}%
\pgfpathlineto{\pgfqpoint{10.875774in}{3.058797in}}%
\pgfpathlineto{\pgfqpoint{10.877787in}{3.055703in}}%
\pgfpathlineto{\pgfqpoint{10.879800in}{3.062277in}}%
\pgfpathlineto{\pgfqpoint{10.881814in}{3.054736in}}%
\pgfpathlineto{\pgfqpoint{10.883827in}{3.056863in}}%
\pgfpathlineto{\pgfqpoint{10.889868in}{3.055703in}}%
\pgfpathlineto{\pgfqpoint{10.893895in}{3.044875in}}%
\pgfpathlineto{\pgfqpoint{10.895908in}{3.044101in}}%
\pgfpathlineto{\pgfqpoint{10.897921in}{3.050289in}}%
\pgfpathlineto{\pgfqpoint{10.903962in}{3.048452in}}%
\pgfpathlineto{\pgfqpoint{10.905975in}{3.045068in}}%
\pgfpathlineto{\pgfqpoint{10.907989in}{3.053479in}}%
\pgfpathlineto{\pgfqpoint{10.910002in}{3.049225in}}%
\pgfpathlineto{\pgfqpoint{10.912016in}{3.057733in}}%
\pgfpathlineto{\pgfqpoint{10.918056in}{3.046808in}}%
\pgfpathlineto{\pgfqpoint{10.920069in}{3.048259in}}%
\pgfpathlineto{\pgfqpoint{10.924096in}{3.037430in}}%
\pgfpathlineto{\pgfqpoint{10.926110in}{3.045938in}}%
\pgfpathlineto{\pgfqpoint{10.932150in}{3.053479in}}%
\pgfpathlineto{\pgfqpoint{10.934164in}{3.057057in}}%
\pgfpathlineto{\pgfqpoint{10.936177in}{3.059280in}}%
\pgfpathlineto{\pgfqpoint{10.938190in}{3.050966in}}%
\pgfpathlineto{\pgfqpoint{10.940204in}{3.067112in}}%
\pgfpathlineto{\pgfqpoint{10.946244in}{3.073686in}}%
\pgfpathlineto{\pgfqpoint{10.948258in}{3.078037in}}%
\pgfpathlineto{\pgfqpoint{10.950271in}{3.078617in}}%
\pgfpathlineto{\pgfqpoint{10.954298in}{3.084708in}}%
\pgfpathlineto{\pgfqpoint{10.960338in}{3.085674in}}%
\pgfpathlineto{\pgfqpoint{10.962352in}{3.096213in}}%
\pgfpathlineto{\pgfqpoint{10.964365in}{3.099210in}}%
\pgfpathlineto{\pgfqpoint{10.966379in}{3.098340in}}%
\pgfpathlineto{\pgfqpoint{10.968392in}{3.100273in}}%
\pgfpathlineto{\pgfqpoint{10.974432in}{3.102980in}}%
\pgfpathlineto{\pgfqpoint{10.976446in}{3.104817in}}%
\pgfpathlineto{\pgfqpoint{10.978459in}{3.103174in}}%
\pgfpathlineto{\pgfqpoint{10.980473in}{3.095826in}}%
\pgfpathlineto{\pgfqpoint{10.982486in}{3.091765in}}%
\pgfpathlineto{\pgfqpoint{10.988527in}{3.089542in}}%
\pgfpathlineto{\pgfqpoint{10.990540in}{3.090412in}}%
\pgfpathlineto{\pgfqpoint{10.992553in}{3.096986in}}%
\pgfpathlineto{\pgfqpoint{10.994567in}{3.094763in}}%
\pgfpathlineto{\pgfqpoint{10.996580in}{3.095826in}}%
\pgfpathlineto{\pgfqpoint{11.002621in}{3.091765in}}%
\pgfpathlineto{\pgfqpoint{11.004634in}{3.097566in}}%
\pgfpathlineto{\pgfqpoint{11.006648in}{3.098243in}}%
\pgfpathlineto{\pgfqpoint{11.010675in}{3.111198in}}%
\pgfpathlineto{\pgfqpoint{11.016715in}{3.108201in}}%
\pgfpathlineto{\pgfqpoint{11.018728in}{3.116516in}}%
\pgfpathlineto{\pgfqpoint{11.020742in}{3.107525in}}%
\pgfpathlineto{\pgfqpoint{11.022755in}{3.112455in}}%
\pgfpathlineto{\pgfqpoint{11.024769in}{3.110715in}}%
\pgfpathlineto{\pgfqpoint{11.030809in}{3.113906in}}%
\pgfpathlineto{\pgfqpoint{11.032822in}{3.113519in}}%
\pgfpathlineto{\pgfqpoint{11.034836in}{3.107525in}}%
\pgfpathlineto{\pgfqpoint{11.036849in}{3.111005in}}%
\pgfpathlineto{\pgfqpoint{11.038863in}{3.103754in}}%
\pgfpathlineto{\pgfqpoint{11.044903in}{3.100853in}}%
\pgfpathlineto{\pgfqpoint{11.046917in}{3.102207in}}%
\pgfpathlineto{\pgfqpoint{11.050943in}{3.124250in}}%
\pgfpathlineto{\pgfqpoint{11.052957in}{3.124734in}}%
\pgfpathlineto{\pgfqpoint{11.058997in}{3.129278in}}%
\pgfpathlineto{\pgfqpoint{11.061011in}{3.135079in}}%
\pgfpathlineto{\pgfqpoint{11.063024in}{3.133822in}}%
\pgfpathlineto{\pgfqpoint{11.067051in}{3.136529in}}%
\pgfpathlineto{\pgfqpoint{11.075105in}{3.127634in}}%
\pgfpathlineto{\pgfqpoint{11.077118in}{3.116709in}}%
\pgfpathlineto{\pgfqpoint{11.081145in}{3.111198in}}%
\pgfpathlineto{\pgfqpoint{11.087186in}{3.107621in}}%
\pgfpathlineto{\pgfqpoint{11.089199in}{3.104237in}}%
\pgfpathlineto{\pgfqpoint{11.091212in}{3.108105in}}%
\pgfpathlineto{\pgfqpoint{11.093226in}{3.116806in}}%
\pgfpathlineto{\pgfqpoint{11.095239in}{3.109652in}}%
\pgfpathlineto{\pgfqpoint{11.101280in}{3.106848in}}%
\pgfpathlineto{\pgfqpoint{11.103293in}{3.110038in}}%
\pgfpathlineto{\pgfqpoint{11.105307in}{3.107525in}}%
\pgfpathlineto{\pgfqpoint{11.107320in}{3.106364in}}%
\pgfpathlineto{\pgfqpoint{11.109333in}{3.118063in}}%
\pgfpathlineto{\pgfqpoint{11.117387in}{3.117483in}}%
\pgfpathlineto{\pgfqpoint{11.119401in}{3.119030in}}%
\pgfpathlineto{\pgfqpoint{11.121414in}{3.126571in}}%
\pgfpathlineto{\pgfqpoint{11.123428in}{3.113615in}}%
\pgfpathlineto{\pgfqpoint{11.129468in}{3.109361in}}%
\pgfpathlineto{\pgfqpoint{11.131481in}{3.083354in}}%
\pgfpathlineto{\pgfqpoint{11.133495in}{3.072042in}}%
\pgfpathlineto{\pgfqpoint{11.135508in}{3.076393in}}%
\pgfpathlineto{\pgfqpoint{11.137522in}{3.064791in}}%
\pgfpathlineto{\pgfqpoint{11.143562in}{3.071849in}}%
\pgfpathlineto{\pgfqpoint{11.145575in}{3.078810in}}%
\pgfpathlineto{\pgfqpoint{11.147589in}{3.077456in}}%
\pgfpathlineto{\pgfqpoint{11.149602in}{3.085094in}}%
\pgfpathlineto{\pgfqpoint{11.151616in}{3.075910in}}%
\pgfpathlineto{\pgfqpoint{11.157656in}{3.071172in}}%
\pgfpathlineto{\pgfqpoint{11.163697in}{3.079487in}}%
\pgfpathlineto{\pgfqpoint{11.165710in}{3.078423in}}%
\pgfpathlineto{\pgfqpoint{11.173764in}{3.074943in}}%
\pgfpathlineto{\pgfqpoint{11.175777in}{3.081420in}}%
\pgfpathlineto{\pgfqpoint{11.177791in}{3.072526in}}%
\pgfpathlineto{\pgfqpoint{11.179804in}{3.069722in}}%
\pgfpathlineto{\pgfqpoint{11.187858in}{3.075039in}}%
\pgfpathlineto{\pgfqpoint{11.189871in}{3.074653in}}%
\pgfpathlineto{\pgfqpoint{11.191885in}{3.072236in}}%
\pgfpathlineto{\pgfqpoint{11.193898in}{3.071849in}}%
\pgfpathlineto{\pgfqpoint{11.199939in}{3.074266in}}%
\pgfpathlineto{\pgfqpoint{11.201952in}{3.072139in}}%
\pgfpathlineto{\pgfqpoint{11.203965in}{3.065275in}}%
\pgfpathlineto{\pgfqpoint{11.205979in}{3.067692in}}%
\pgfpathlineto{\pgfqpoint{11.207992in}{3.051062in}}%
\pgfpathlineto{\pgfqpoint{11.214033in}{3.054736in}}%
\pgfpathlineto{\pgfqpoint{11.216046in}{3.041684in}}%
\pgfpathlineto{\pgfqpoint{11.218060in}{3.040427in}}%
\pgfpathlineto{\pgfqpoint{11.220073in}{3.046325in}}%
\pgfpathlineto{\pgfqpoint{11.222086in}{3.044101in}}%
\pgfpathlineto{\pgfqpoint{11.228127in}{3.058700in}}%
\pgfpathlineto{\pgfqpoint{11.230140in}{3.052706in}}%
\pgfpathlineto{\pgfqpoint{11.232154in}{3.060247in}}%
\pgfpathlineto{\pgfqpoint{11.234167in}{3.057153in}}%
\pgfpathlineto{\pgfqpoint{11.236181in}{3.068562in}}%
\pgfpathlineto{\pgfqpoint{11.242221in}{3.069625in}}%
\pgfpathlineto{\pgfqpoint{11.248261in}{3.046808in}}%
\pgfpathlineto{\pgfqpoint{11.250275in}{3.048162in}}%
\pgfpathlineto{\pgfqpoint{11.256315in}{3.051546in}}%
\pgfpathlineto{\pgfqpoint{11.258329in}{3.044971in}}%
\pgfpathlineto{\pgfqpoint{11.260342in}{3.048259in}}%
\pgfpathlineto{\pgfqpoint{11.262355in}{3.049129in}}%
\pgfpathlineto{\pgfqpoint{11.270409in}{3.054253in}}%
\pgfpathlineto{\pgfqpoint{11.272423in}{3.048839in}}%
\pgfpathlineto{\pgfqpoint{11.274436in}{3.052029in}}%
\pgfpathlineto{\pgfqpoint{11.276450in}{3.052996in}}%
\pgfpathlineto{\pgfqpoint{11.278463in}{3.056863in}}%
\pgfpathlineto{\pgfqpoint{11.284503in}{3.057540in}}%
\pgfpathlineto{\pgfqpoint{11.286517in}{3.058990in}}%
\pgfpathlineto{\pgfqpoint{11.288530in}{3.058217in}}%
\pgfpathlineto{\pgfqpoint{11.290544in}{3.058120in}}%
\pgfpathlineto{\pgfqpoint{11.292557in}{3.049999in}}%
\pgfpathlineto{\pgfqpoint{11.298597in}{3.052899in}}%
\pgfpathlineto{\pgfqpoint{11.300611in}{3.054640in}}%
\pgfpathlineto{\pgfqpoint{11.302624in}{3.054736in}}%
\pgfpathlineto{\pgfqpoint{11.304638in}{3.042168in}}%
\pgfpathlineto{\pgfqpoint{11.306651in}{3.042554in}}%
\pgfpathlineto{\pgfqpoint{11.314705in}{3.037624in}}%
\pgfpathlineto{\pgfqpoint{11.316718in}{3.032789in}}%
\pgfpathlineto{\pgfqpoint{11.318732in}{3.029889in}}%
\pgfpathlineto{\pgfqpoint{11.320745in}{3.036560in}}%
\pgfpathlineto{\pgfqpoint{11.326786in}{3.037043in}}%
\pgfpathlineto{\pgfqpoint{11.328799in}{3.034723in}}%
\pgfpathlineto{\pgfqpoint{11.330813in}{3.037527in}}%
\pgfpathlineto{\pgfqpoint{11.332826in}{3.035787in}}%
\pgfpathlineto{\pgfqpoint{11.334840in}{3.042264in}}%
\pgfpathlineto{\pgfqpoint{11.342893in}{3.033660in}}%
\pgfpathlineto{\pgfqpoint{11.344907in}{3.031533in}}%
\pgfpathlineto{\pgfqpoint{11.346920in}{3.038977in}}%
\pgfpathlineto{\pgfqpoint{11.348934in}{3.043038in}}%
\pgfpathlineto{\pgfqpoint{11.354974in}{3.040331in}}%
\pgfpathlineto{\pgfqpoint{11.356987in}{3.041104in}}%
\pgfpathlineto{\pgfqpoint{11.359001in}{3.038204in}}%
\pgfpathlineto{\pgfqpoint{11.361014in}{3.037624in}}%
\pgfpathlineto{\pgfqpoint{11.363028in}{3.033660in}}%
\pgfpathlineto{\pgfqpoint{11.371082in}{3.026795in}}%
\pgfpathlineto{\pgfqpoint{11.373095in}{3.028922in}}%
\pgfpathlineto{\pgfqpoint{11.375108in}{3.028439in}}%
\pgfpathlineto{\pgfqpoint{11.377122in}{3.020414in}}%
\pgfpathlineto{\pgfqpoint{11.383162in}{3.024281in}}%
\pgfpathlineto{\pgfqpoint{11.385176in}{3.021671in}}%
\pgfpathlineto{\pgfqpoint{11.387189in}{3.021864in}}%
\pgfpathlineto{\pgfqpoint{11.389203in}{3.018384in}}%
\pgfpathlineto{\pgfqpoint{11.391216in}{3.012293in}}%
\pgfpathlineto{\pgfqpoint{11.397256in}{3.014613in}}%
\pgfpathlineto{\pgfqpoint{11.399270in}{3.024765in}}%
\pgfpathlineto{\pgfqpoint{11.401283in}{3.030179in}}%
\pgfpathlineto{\pgfqpoint{11.403297in}{3.029116in}}%
\pgfpathlineto{\pgfqpoint{11.405310in}{3.024475in}}%
\pgfpathlineto{\pgfqpoint{11.411351in}{3.018094in}}%
\pgfpathlineto{\pgfqpoint{11.417391in}{3.041104in}}%
\pgfpathlineto{\pgfqpoint{11.419404in}{3.038687in}}%
\pgfpathlineto{\pgfqpoint{11.425445in}{3.037914in}}%
\pgfpathlineto{\pgfqpoint{11.427458in}{3.032306in}}%
\pgfpathlineto{\pgfqpoint{11.429472in}{3.029889in}}%
\pgfpathlineto{\pgfqpoint{11.433498in}{3.028535in}}%
\pgfpathlineto{\pgfqpoint{11.439539in}{3.019834in}}%
\pgfpathlineto{\pgfqpoint{11.441552in}{3.019157in}}%
\pgfpathlineto{\pgfqpoint{11.443566in}{3.031726in}}%
\pgfpathlineto{\pgfqpoint{11.445579in}{3.033466in}}%
\pgfpathlineto{\pgfqpoint{11.453633in}{3.034530in}}%
\pgfpathlineto{\pgfqpoint{11.455646in}{3.048742in}}%
\pgfpathlineto{\pgfqpoint{11.457660in}{3.042458in}}%
\pgfpathlineto{\pgfqpoint{11.459673in}{3.039654in}}%
\pgfpathlineto{\pgfqpoint{11.469740in}{3.051449in}}%
\pgfpathlineto{\pgfqpoint{11.473767in}{3.053673in}}%
\pgfpathlineto{\pgfqpoint{11.475781in}{3.053093in}}%
\pgfpathlineto{\pgfqpoint{11.481821in}{3.052706in}}%
\pgfpathlineto{\pgfqpoint{11.483835in}{3.047969in}}%
\pgfpathlineto{\pgfqpoint{11.487862in}{3.045648in}}%
\pgfpathlineto{\pgfqpoint{11.489875in}{3.042168in}}%
\pgfpathlineto{\pgfqpoint{11.495915in}{3.039364in}}%
\pgfpathlineto{\pgfqpoint{11.497929in}{3.041588in}}%
\pgfpathlineto{\pgfqpoint{11.499942in}{3.044971in}}%
\pgfpathlineto{\pgfqpoint{11.501956in}{3.017320in}}%
\pgfpathlineto{\pgfqpoint{11.503969in}{3.011423in}}%
\pgfpathlineto{\pgfqpoint{11.510009in}{3.008812in}}%
\pgfpathlineto{\pgfqpoint{11.512023in}{3.004655in}}%
\pgfpathlineto{\pgfqpoint{11.514036in}{3.003398in}}%
\pgfpathlineto{\pgfqpoint{11.516050in}{3.003205in}}%
\pgfpathlineto{\pgfqpoint{11.518063in}{3.000981in}}%
\pgfpathlineto{\pgfqpoint{11.524104in}{3.008619in}}%
\pgfpathlineto{\pgfqpoint{11.526117in}{3.007265in}}%
\pgfpathlineto{\pgfqpoint{11.528130in}{3.008716in}}%
\pgfpathlineto{\pgfqpoint{11.530144in}{3.003495in}}%
\pgfpathlineto{\pgfqpoint{11.532157in}{3.002141in}}%
\pgfpathlineto{\pgfqpoint{11.538198in}{3.001368in}}%
\pgfpathlineto{\pgfqpoint{11.540211in}{2.997984in}}%
\pgfpathlineto{\pgfqpoint{11.542225in}{2.989283in}}%
\pgfpathlineto{\pgfqpoint{11.544238in}{2.987542in}}%
\pgfpathlineto{\pgfqpoint{11.546251in}{2.969753in}}%
\pgfpathlineto{\pgfqpoint{11.554305in}{2.940555in}}%
\pgfpathlineto{\pgfqpoint{11.556319in}{2.961728in}}%
\pgfpathlineto{\pgfqpoint{11.558332in}{2.966756in}}%
\pgfpathlineto{\pgfqpoint{11.560346in}{2.964435in}}%
\pgfpathlineto{\pgfqpoint{11.566386in}{2.959795in}}%
\pgfpathlineto{\pgfqpoint{11.568399in}{2.944616in}}%
\pgfpathlineto{\pgfqpoint{11.570413in}{2.952447in}}%
\pgfpathlineto{\pgfqpoint{11.572426in}{2.953414in}}%
\pgfpathlineto{\pgfqpoint{11.574440in}{2.943455in}}%
\pgfpathlineto{\pgfqpoint{11.582494in}{2.953897in}}%
\pgfpathlineto{\pgfqpoint{11.584507in}{2.941038in}}%
\pgfpathlineto{\pgfqpoint{11.586520in}{2.939685in}}%
\pgfpathlineto{\pgfqpoint{11.588534in}{2.940555in}}%
\pgfpathlineto{\pgfqpoint{11.594574in}{2.937461in}}%
\pgfpathlineto{\pgfqpoint{11.596588in}{2.949353in}}%
\pgfpathlineto{\pgfqpoint{11.598601in}{2.954864in}}%
\pgfpathlineto{\pgfqpoint{11.600615in}{2.956121in}}%
\pgfpathlineto{\pgfqpoint{11.602628in}{2.953510in}}%
\pgfpathlineto{\pgfqpoint{11.608668in}{2.959601in}}%
\pgfpathlineto{\pgfqpoint{11.610682in}{2.955637in}}%
\pgfpathlineto{\pgfqpoint{11.612695in}{2.956314in}}%
\pgfpathlineto{\pgfqpoint{11.616722in}{2.976907in}}%
\pgfpathlineto{\pgfqpoint{11.622762in}{2.969173in}}%
\pgfpathlineto{\pgfqpoint{11.624776in}{2.973523in}}%
\pgfpathlineto{\pgfqpoint{11.626789in}{2.970623in}}%
\pgfpathlineto{\pgfqpoint{11.628803in}{2.970720in}}%
\pgfpathlineto{\pgfqpoint{11.630816in}{2.974780in}}%
\pgfpathlineto{\pgfqpoint{11.640884in}{2.985899in}}%
\pgfpathlineto{\pgfqpoint{11.642897in}{2.991700in}}%
\pgfpathlineto{\pgfqpoint{11.644910in}{2.992376in}}%
\pgfpathlineto{\pgfqpoint{11.650951in}{2.991120in}}%
\pgfpathlineto{\pgfqpoint{11.652964in}{2.989186in}}%
\pgfpathlineto{\pgfqpoint{11.656991in}{2.990636in}}%
\pgfpathlineto{\pgfqpoint{11.659005in}{2.995954in}}%
\pgfpathlineto{\pgfqpoint{11.665045in}{2.998177in}}%
\pgfpathlineto{\pgfqpoint{11.667058in}{2.991990in}}%
\pgfpathlineto{\pgfqpoint{11.669072in}{2.990443in}}%
\pgfpathlineto{\pgfqpoint{11.671085in}{3.001368in}}%
\pgfpathlineto{\pgfqpoint{11.673099in}{3.020124in}}%
\pgfpathlineto{\pgfqpoint{11.679139in}{3.024088in}}%
\pgfpathlineto{\pgfqpoint{11.681152in}{3.022445in}}%
\pgfpathlineto{\pgfqpoint{11.683166in}{3.015677in}}%
\pgfpathlineto{\pgfqpoint{11.685179in}{3.020027in}}%
\pgfpathlineto{\pgfqpoint{11.687193in}{3.014517in}}%
\pgfpathlineto{\pgfqpoint{11.693233in}{3.016450in}}%
\pgfpathlineto{\pgfqpoint{11.695247in}{3.020318in}}%
\pgfpathlineto{\pgfqpoint{11.697260in}{3.020414in}}%
\pgfpathlineto{\pgfqpoint{11.701287in}{3.007555in}}%
\pgfpathlineto{\pgfqpoint{11.707327in}{3.006105in}}%
\pgfpathlineto{\pgfqpoint{11.709341in}{3.009296in}}%
\pgfpathlineto{\pgfqpoint{11.711354in}{3.010939in}}%
\pgfpathlineto{\pgfqpoint{11.713368in}{2.999724in}}%
\pgfpathlineto{\pgfqpoint{11.715381in}{2.993633in}}%
\pgfpathlineto{\pgfqpoint{11.721421in}{3.005235in}}%
\pgfpathlineto{\pgfqpoint{11.723435in}{3.003205in}}%
\pgfpathlineto{\pgfqpoint{11.725448in}{3.010359in}}%
\pgfpathlineto{\pgfqpoint{11.727462in}{3.013163in}}%
\pgfpathlineto{\pgfqpoint{11.729475in}{3.009682in}}%
\pgfpathlineto{\pgfqpoint{11.735516in}{3.011036in}}%
\pgfpathlineto{\pgfqpoint{11.737529in}{3.015193in}}%
\pgfpathlineto{\pgfqpoint{11.739542in}{3.010359in}}%
\pgfpathlineto{\pgfqpoint{11.743569in}{3.008716in}}%
\pgfpathlineto{\pgfqpoint{11.749610in}{3.001271in}}%
\pgfpathlineto{\pgfqpoint{11.751623in}{3.010746in}}%
\pgfpathlineto{\pgfqpoint{11.753637in}{3.009489in}}%
\pgfpathlineto{\pgfqpoint{11.755650in}{3.009006in}}%
\pgfpathlineto{\pgfqpoint{11.757663in}{3.027085in}}%
\pgfpathlineto{\pgfqpoint{11.763704in}{3.031726in}}%
\pgfpathlineto{\pgfqpoint{11.765717in}{3.026408in}}%
\pgfpathlineto{\pgfqpoint{11.767731in}{3.025925in}}%
\pgfpathlineto{\pgfqpoint{11.769744in}{3.026699in}}%
\pgfpathlineto{\pgfqpoint{11.771758in}{3.026602in}}%
\pgfpathlineto{\pgfqpoint{11.777798in}{3.031049in}}%
\pgfpathlineto{\pgfqpoint{11.781825in}{3.054350in}}%
\pgfpathlineto{\pgfqpoint{11.783838in}{3.048355in}}%
\pgfpathlineto{\pgfqpoint{11.785852in}{3.029696in}}%
\pgfpathlineto{\pgfqpoint{11.791892in}{3.036753in}}%
\pgfpathlineto{\pgfqpoint{11.793905in}{3.042361in}}%
\pgfpathlineto{\pgfqpoint{11.795919in}{3.045165in}}%
\pgfpathlineto{\pgfqpoint{11.797932in}{3.044005in}}%
\pgfpathlineto{\pgfqpoint{11.805986in}{3.045165in}}%
\pgfpathlineto{\pgfqpoint{11.808000in}{3.048935in}}%
\pgfpathlineto{\pgfqpoint{11.810013in}{3.046422in}}%
\pgfpathlineto{\pgfqpoint{11.812027in}{3.040717in}}%
\pgfpathlineto{\pgfqpoint{11.820080in}{3.031726in}}%
\pgfpathlineto{\pgfqpoint{11.822094in}{3.033853in}}%
\pgfpathlineto{\pgfqpoint{11.826121in}{3.021478in}}%
\pgfpathlineto{\pgfqpoint{11.828134in}{3.011036in}}%
\pgfpathlineto{\pgfqpoint{11.834174in}{3.017030in}}%
\pgfpathlineto{\pgfqpoint{11.836188in}{3.015677in}}%
\pgfpathlineto{\pgfqpoint{11.838201in}{3.009973in}}%
\pgfpathlineto{\pgfqpoint{11.840215in}{3.012583in}}%
\pgfpathlineto{\pgfqpoint{11.842228in}{3.002431in}}%
\pgfpathlineto{\pgfqpoint{11.850282in}{3.017610in}}%
\pgfpathlineto{\pgfqpoint{11.852295in}{3.015483in}}%
\pgfpathlineto{\pgfqpoint{11.856322in}{3.028826in}}%
\pgfpathlineto{\pgfqpoint{11.862363in}{3.024378in}}%
\pgfpathlineto{\pgfqpoint{11.864376in}{3.041491in}}%
\pgfpathlineto{\pgfqpoint{11.866390in}{3.041394in}}%
\pgfpathlineto{\pgfqpoint{11.868403in}{3.050289in}}%
\pgfpathlineto{\pgfqpoint{11.870416in}{3.066531in}}%
\pgfpathlineto{\pgfqpoint{11.876457in}{3.061601in}}%
\pgfpathlineto{\pgfqpoint{11.878470in}{3.053769in}}%
\pgfpathlineto{\pgfqpoint{11.880484in}{3.061407in}}%
\pgfpathlineto{\pgfqpoint{11.882497in}{3.057927in}}%
\pgfpathlineto{\pgfqpoint{11.890551in}{3.074653in}}%
\pgfpathlineto{\pgfqpoint{11.892564in}{3.074846in}}%
\pgfpathlineto{\pgfqpoint{11.894578in}{3.065951in}}%
\pgfpathlineto{\pgfqpoint{11.896591in}{3.050966in}}%
\pgfpathlineto{\pgfqpoint{11.898605in}{3.060440in}}%
\pgfpathlineto{\pgfqpoint{11.906659in}{3.064598in}}%
\pgfpathlineto{\pgfqpoint{11.908672in}{3.073202in}}%
\pgfpathlineto{\pgfqpoint{11.910685in}{3.069045in}}%
\pgfpathlineto{\pgfqpoint{11.912699in}{3.067402in}}%
\pgfpathlineto{\pgfqpoint{11.918739in}{3.070399in}}%
\pgfpathlineto{\pgfqpoint{11.922766in}{3.065468in}}%
\pgfpathlineto{\pgfqpoint{11.924780in}{3.072236in}}%
\pgfpathlineto{\pgfqpoint{11.926793in}{3.061407in}}%
\pgfpathlineto{\pgfqpoint{11.932833in}{3.054350in}}%
\pgfpathlineto{\pgfqpoint{11.934847in}{3.062567in}}%
\pgfpathlineto{\pgfqpoint{11.936860in}{3.074073in}}%
\pgfpathlineto{\pgfqpoint{11.938874in}{3.076586in}}%
\pgfpathlineto{\pgfqpoint{11.940887in}{3.082291in}}%
\pgfpathlineto{\pgfqpoint{11.946927in}{3.078810in}}%
\pgfpathlineto{\pgfqpoint{11.948941in}{3.078520in}}%
\pgfpathlineto{\pgfqpoint{11.950954in}{3.077650in}}%
\pgfpathlineto{\pgfqpoint{11.954981in}{3.067112in}}%
\pgfpathlineto{\pgfqpoint{11.961022in}{3.061987in}}%
\pgfpathlineto{\pgfqpoint{11.963035in}{3.063244in}}%
\pgfpathlineto{\pgfqpoint{11.965049in}{3.063534in}}%
\pgfpathlineto{\pgfqpoint{11.967062in}{3.075813in}}%
\pgfpathlineto{\pgfqpoint{11.969075in}{3.079293in}}%
\pgfpathlineto{\pgfqpoint{11.975116in}{3.080744in}}%
\pgfpathlineto{\pgfqpoint{11.977129in}{3.075716in}}%
\pgfpathlineto{\pgfqpoint{11.981156in}{3.076973in}}%
\pgfpathlineto{\pgfqpoint{11.989210in}{3.074653in}}%
\pgfpathlineto{\pgfqpoint{11.991223in}{3.076200in}}%
\pgfpathlineto{\pgfqpoint{11.993237in}{3.075233in}}%
\pgfpathlineto{\pgfqpoint{11.995250in}{3.071946in}}%
\pgfpathlineto{\pgfqpoint{11.997264in}{3.082581in}}%
\pgfpathlineto{\pgfqpoint{12.005317in}{3.079390in}}%
\pgfpathlineto{\pgfqpoint{12.007331in}{3.084998in}}%
\pgfpathlineto{\pgfqpoint{12.009344in}{3.080067in}}%
\pgfpathlineto{\pgfqpoint{12.011358in}{3.079680in}}%
\pgfpathlineto{\pgfqpoint{12.017398in}{3.075620in}}%
\pgfpathlineto{\pgfqpoint{12.019412in}{3.076490in}}%
\pgfpathlineto{\pgfqpoint{12.021425in}{3.073299in}}%
\pgfpathlineto{\pgfqpoint{12.031492in}{3.082387in}}%
\pgfpathlineto{\pgfqpoint{12.033506in}{3.086351in}}%
\pgfpathlineto{\pgfqpoint{12.035519in}{3.071172in}}%
\pgfpathlineto{\pgfqpoint{12.037533in}{3.064598in}}%
\pgfpathlineto{\pgfqpoint{12.045586in}{3.069915in}}%
\pgfpathlineto{\pgfqpoint{12.047600in}{3.053576in}}%
\pgfpathlineto{\pgfqpoint{12.049613in}{3.056573in}}%
\pgfpathlineto{\pgfqpoint{12.051627in}{3.055413in}}%
\pgfpathlineto{\pgfqpoint{12.053640in}{3.058604in}}%
\pgfpathlineto{\pgfqpoint{12.059681in}{3.066048in}}%
\pgfpathlineto{\pgfqpoint{12.061694in}{3.067208in}}%
\pgfpathlineto{\pgfqpoint{12.063707in}{3.071559in}}%
\pgfpathlineto{\pgfqpoint{12.065721in}{3.068948in}}%
\pgfpathlineto{\pgfqpoint{12.067734in}{3.076296in}}%
\pgfpathlineto{\pgfqpoint{12.073775in}{3.076200in}}%
\pgfpathlineto{\pgfqpoint{12.075788in}{3.079293in}}%
\pgfpathlineto{\pgfqpoint{12.077802in}{3.076393in}}%
\pgfpathlineto{\pgfqpoint{12.079815in}{3.078713in}}%
\pgfpathlineto{\pgfqpoint{12.081828in}{3.068368in}}%
\pgfpathlineto{\pgfqpoint{12.087869in}{3.071849in}}%
\pgfpathlineto{\pgfqpoint{12.091896in}{3.056186in}}%
\pgfpathlineto{\pgfqpoint{12.093909in}{3.059184in}}%
\pgfpathlineto{\pgfqpoint{12.095923in}{3.057733in}}%
\pgfpathlineto{\pgfqpoint{12.101963in}{3.059280in}}%
\pgfpathlineto{\pgfqpoint{12.103976in}{3.066048in}}%
\pgfpathlineto{\pgfqpoint{12.105990in}{3.070495in}}%
\pgfpathlineto{\pgfqpoint{12.108003in}{3.068272in}}%
\pgfpathlineto{\pgfqpoint{12.110017in}{3.070109in}}%
\pgfpathlineto{\pgfqpoint{12.118070in}{3.066628in}}%
\pgfpathlineto{\pgfqpoint{12.120084in}{3.073299in}}%
\pgfpathlineto{\pgfqpoint{12.122097in}{3.074653in}}%
\pgfpathlineto{\pgfqpoint{12.124111in}{3.079197in}}%
\pgfpathlineto{\pgfqpoint{12.130151in}{3.081904in}}%
\pgfpathlineto{\pgfqpoint{12.132165in}{3.077940in}}%
\pgfpathlineto{\pgfqpoint{12.134178in}{3.080840in}}%
\pgfpathlineto{\pgfqpoint{12.136192in}{3.085384in}}%
\pgfpathlineto{\pgfqpoint{12.138205in}{3.085674in}}%
\pgfpathlineto{\pgfqpoint{12.144245in}{3.080164in}}%
\pgfpathlineto{\pgfqpoint{12.146259in}{3.086931in}}%
\pgfpathlineto{\pgfqpoint{12.148272in}{3.083451in}}%
\pgfpathlineto{\pgfqpoint{12.150286in}{3.087511in}}%
\pgfpathlineto{\pgfqpoint{12.152299in}{3.084998in}}%
\pgfpathlineto{\pgfqpoint{12.158339in}{3.084224in}}%
\pgfpathlineto{\pgfqpoint{12.160353in}{3.087511in}}%
\pgfpathlineto{\pgfqpoint{12.162366in}{3.088865in}}%
\pgfpathlineto{\pgfqpoint{12.164380in}{3.094569in}}%
\pgfpathlineto{\pgfqpoint{12.166393in}{3.077360in}}%
\pgfpathlineto{\pgfqpoint{12.172434in}{3.068368in}}%
\pgfpathlineto{\pgfqpoint{12.178474in}{3.098533in}}%
\pgfpathlineto{\pgfqpoint{12.180487in}{3.099500in}}%
\pgfpathlineto{\pgfqpoint{12.188541in}{3.105301in}}%
\pgfpathlineto{\pgfqpoint{12.190555in}{3.101724in}}%
\pgfpathlineto{\pgfqpoint{12.192568in}{3.099983in}}%
\pgfpathlineto{\pgfqpoint{12.194581in}{3.108201in}}%
\pgfpathlineto{\pgfqpoint{12.202635in}{3.108008in}}%
\pgfpathlineto{\pgfqpoint{12.204649in}{3.109265in}}%
\pgfpathlineto{\pgfqpoint{12.206662in}{3.109071in}}%
\pgfpathlineto{\pgfqpoint{12.208676in}{3.110328in}}%
\pgfpathlineto{\pgfqpoint{12.214716in}{3.109652in}}%
\pgfpathlineto{\pgfqpoint{12.216729in}{3.111779in}}%
\pgfpathlineto{\pgfqpoint{12.218743in}{3.110135in}}%
\pgfpathlineto{\pgfqpoint{12.220756in}{3.109652in}}%
\pgfpathlineto{\pgfqpoint{12.222770in}{3.113712in}}%
\pgfpathlineto{\pgfqpoint{12.228810in}{3.114389in}}%
\pgfpathlineto{\pgfqpoint{12.230824in}{3.109748in}}%
\pgfpathlineto{\pgfqpoint{12.232837in}{3.102497in}}%
\pgfpathlineto{\pgfqpoint{12.234850in}{3.105591in}}%
\pgfpathlineto{\pgfqpoint{12.236864in}{3.112552in}}%
\pgfpathlineto{\pgfqpoint{12.244918in}{3.122897in}}%
\pgfpathlineto{\pgfqpoint{12.246931in}{3.115936in}}%
\pgfpathlineto{\pgfqpoint{12.248945in}{3.116613in}}%
\pgfpathlineto{\pgfqpoint{12.250958in}{3.114196in}}%
\pgfpathlineto{\pgfqpoint{12.256998in}{3.114002in}}%
\pgfpathlineto{\pgfqpoint{12.259012in}{3.116033in}}%
\pgfpathlineto{\pgfqpoint{12.265052in}{3.125411in}}%
\pgfpathlineto{\pgfqpoint{12.271092in}{3.125217in}}%
\pgfpathlineto{\pgfqpoint{12.273106in}{3.121253in}}%
\pgfpathlineto{\pgfqpoint{12.277133in}{3.128891in}}%
\pgfpathlineto{\pgfqpoint{12.285187in}{3.123670in}}%
\pgfpathlineto{\pgfqpoint{12.287200in}{3.128601in}}%
\pgfpathlineto{\pgfqpoint{12.289214in}{3.127731in}}%
\pgfpathlineto{\pgfqpoint{12.291227in}{3.132952in}}%
\pgfpathlineto{\pgfqpoint{12.293240in}{3.130148in}}%
\pgfpathlineto{\pgfqpoint{12.299281in}{3.136529in}}%
\pgfpathlineto{\pgfqpoint{12.301294in}{3.129761in}}%
\pgfpathlineto{\pgfqpoint{12.303308in}{3.127731in}}%
\pgfpathlineto{\pgfqpoint{12.305321in}{3.136626in}}%
\pgfpathlineto{\pgfqpoint{12.307335in}{3.135659in}}%
\pgfpathlineto{\pgfqpoint{12.315388in}{3.139526in}}%
\pgfpathlineto{\pgfqpoint{12.317402in}{3.133435in}}%
\pgfpathlineto{\pgfqpoint{12.319415in}{3.131888in}}%
\pgfpathlineto{\pgfqpoint{12.321429in}{3.118256in}}%
\pgfpathlineto{\pgfqpoint{12.327469in}{3.136046in}}%
\pgfpathlineto{\pgfqpoint{12.329482in}{3.125507in}}%
\pgfpathlineto{\pgfqpoint{12.331496in}{3.125121in}}%
\pgfpathlineto{\pgfqpoint{12.333509in}{3.134402in}}%
\pgfpathlineto{\pgfqpoint{12.335523in}{3.134305in}}%
\pgfpathlineto{\pgfqpoint{12.341563in}{3.137109in}}%
\pgfpathlineto{\pgfqpoint{12.343577in}{3.139043in}}%
\pgfpathlineto{\pgfqpoint{12.345590in}{3.132082in}}%
\pgfpathlineto{\pgfqpoint{12.347603in}{3.142620in}}%
\pgfpathlineto{\pgfqpoint{12.349617in}{3.131792in}}%
\pgfpathlineto{\pgfqpoint{12.355657in}{3.132565in}}%
\pgfpathlineto{\pgfqpoint{12.357671in}{3.137012in}}%
\pgfpathlineto{\pgfqpoint{12.359684in}{3.146777in}}%
\pgfpathlineto{\pgfqpoint{12.361698in}{3.135949in}}%
\pgfpathlineto{\pgfqpoint{12.363711in}{3.149388in}}%
\pgfpathlineto{\pgfqpoint{12.371765in}{3.137012in}}%
\pgfpathlineto{\pgfqpoint{12.375792in}{3.144650in}}%
\pgfpathlineto{\pgfqpoint{12.377805in}{3.151611in}}%
\pgfpathlineto{\pgfqpoint{12.383846in}{3.143297in}}%
\pgfpathlineto{\pgfqpoint{12.385859in}{3.138656in}}%
\pgfpathlineto{\pgfqpoint{12.387872in}{3.138946in}}%
\pgfpathlineto{\pgfqpoint{12.389886in}{3.135949in}}%
\pgfpathlineto{\pgfqpoint{12.391899in}{3.137689in}}%
\pgfpathlineto{\pgfqpoint{12.397940in}{3.132372in}}%
\pgfpathlineto{\pgfqpoint{12.399953in}{3.128988in}}%
\pgfpathlineto{\pgfqpoint{12.401967in}{3.117966in}}%
\pgfpathlineto{\pgfqpoint{12.405993in}{3.107138in}}%
\pgfpathlineto{\pgfqpoint{12.412034in}{3.105107in}}%
\pgfpathlineto{\pgfqpoint{12.414047in}{3.130728in}}%
\pgfpathlineto{\pgfqpoint{12.416061in}{3.134499in}}%
\pgfpathlineto{\pgfqpoint{12.418074in}{3.127248in}}%
\pgfpathlineto{\pgfqpoint{12.420088in}{3.129568in}}%
\pgfpathlineto{\pgfqpoint{12.426128in}{3.129181in}}%
\pgfpathlineto{\pgfqpoint{12.428141in}{3.129665in}}%
\pgfpathlineto{\pgfqpoint{12.432168in}{3.127344in}}%
\pgfpathlineto{\pgfqpoint{12.434182in}{3.113809in}}%
\pgfpathlineto{\pgfqpoint{12.440222in}{3.127054in}}%
\pgfpathlineto{\pgfqpoint{12.442235in}{3.135079in}}%
\pgfpathlineto{\pgfqpoint{12.444249in}{3.121447in}}%
\pgfpathlineto{\pgfqpoint{12.446262in}{3.094956in}}%
\pgfpathlineto{\pgfqpoint{12.448276in}{3.100467in}}%
\pgfpathlineto{\pgfqpoint{12.454316in}{3.095246in}}%
\pgfpathlineto{\pgfqpoint{12.456330in}{3.100757in}}%
\pgfpathlineto{\pgfqpoint{12.458343in}{3.096986in}}%
\pgfpathlineto{\pgfqpoint{12.460357in}{3.095923in}}%
\pgfpathlineto{\pgfqpoint{12.462370in}{3.086351in}}%
\pgfpathlineto{\pgfqpoint{12.468410in}{3.092055in}}%
\pgfpathlineto{\pgfqpoint{12.470424in}{3.093119in}}%
\pgfpathlineto{\pgfqpoint{12.472437in}{3.092442in}}%
\pgfpathlineto{\pgfqpoint{12.476464in}{3.099403in}}%
\pgfpathlineto{\pgfqpoint{12.484518in}{3.094279in}}%
\pgfpathlineto{\pgfqpoint{12.486531in}{3.090412in}}%
\pgfpathlineto{\pgfqpoint{12.488545in}{3.085094in}}%
\pgfpathlineto{\pgfqpoint{12.490558in}{3.089928in}}%
\pgfpathlineto{\pgfqpoint{12.496599in}{3.095149in}}%
\pgfpathlineto{\pgfqpoint{12.498612in}{3.094472in}}%
\pgfpathlineto{\pgfqpoint{12.500625in}{3.105784in}}%
\pgfpathlineto{\pgfqpoint{12.502639in}{3.099693in}}%
\pgfpathlineto{\pgfqpoint{12.504652in}{3.107525in}}%
\pgfpathlineto{\pgfqpoint{12.510693in}{3.114292in}}%
\pgfpathlineto{\pgfqpoint{12.512706in}{3.114679in}}%
\pgfpathlineto{\pgfqpoint{12.514720in}{3.107525in}}%
\pgfpathlineto{\pgfqpoint{12.516733in}{3.110232in}}%
\pgfpathlineto{\pgfqpoint{12.524787in}{3.110522in}}%
\pgfpathlineto{\pgfqpoint{12.526800in}{3.109265in}}%
\pgfpathlineto{\pgfqpoint{12.528814in}{3.106654in}}%
\pgfpathlineto{\pgfqpoint{12.530827in}{3.108395in}}%
\pgfpathlineto{\pgfqpoint{12.532841in}{3.112745in}}%
\pgfpathlineto{\pgfqpoint{12.540894in}{3.109555in}}%
\pgfpathlineto{\pgfqpoint{12.542908in}{3.104817in}}%
\pgfpathlineto{\pgfqpoint{12.544921in}{3.107331in}}%
\pgfpathlineto{\pgfqpoint{12.546935in}{3.104914in}}%
\pgfpathlineto{\pgfqpoint{12.554989in}{3.105978in}}%
\pgfpathlineto{\pgfqpoint{12.557002in}{3.108685in}}%
\pgfpathlineto{\pgfqpoint{12.559015in}{3.113615in}}%
\pgfpathlineto{\pgfqpoint{12.561029in}{3.113422in}}%
\pgfpathlineto{\pgfqpoint{12.567069in}{3.107718in}}%
\pgfpathlineto{\pgfqpoint{12.569083in}{3.099597in}}%
\pgfpathlineto{\pgfqpoint{12.571096in}{3.101917in}}%
\pgfpathlineto{\pgfqpoint{12.575123in}{3.104237in}}%
\pgfpathlineto{\pgfqpoint{12.585190in}{3.118450in}}%
\pgfpathlineto{\pgfqpoint{12.587204in}{3.116419in}}%
\pgfpathlineto{\pgfqpoint{12.589217in}{3.141170in}}%
\pgfpathlineto{\pgfqpoint{12.595257in}{3.136722in}}%
\pgfpathlineto{\pgfqpoint{12.597271in}{3.144844in}}%
\pgfpathlineto{\pgfqpoint{12.601298in}{3.133532in}}%
\pgfpathlineto{\pgfqpoint{12.603311in}{3.134595in}}%
\pgfpathlineto{\pgfqpoint{12.609352in}{3.134885in}}%
\pgfpathlineto{\pgfqpoint{12.611365in}{3.142523in}}%
\pgfpathlineto{\pgfqpoint{12.613379in}{3.140106in}}%
\pgfpathlineto{\pgfqpoint{12.615392in}{3.143974in}}%
\pgfpathlineto{\pgfqpoint{12.617405in}{3.140783in}}%
\pgfpathlineto{\pgfqpoint{12.623446in}{3.140686in}}%
\pgfpathlineto{\pgfqpoint{12.625459in}{3.146197in}}%
\pgfpathlineto{\pgfqpoint{12.629486in}{3.152095in}}%
\pgfpathlineto{\pgfqpoint{12.631500in}{3.145811in}}%
\pgfpathlineto{\pgfqpoint{12.637540in}{3.148904in}}%
\pgfpathlineto{\pgfqpoint{12.639553in}{3.144844in}}%
\pgfpathlineto{\pgfqpoint{12.641567in}{3.174138in}}%
\pgfpathlineto{\pgfqpoint{12.643580in}{3.171141in}}%
\pgfpathlineto{\pgfqpoint{12.645594in}{3.173848in}}%
\pgfpathlineto{\pgfqpoint{12.653647in}{3.179069in}}%
\pgfpathlineto{\pgfqpoint{12.659688in}{3.173558in}}%
\pgfpathlineto{\pgfqpoint{12.665728in}{3.172108in}}%
\pgfpathlineto{\pgfqpoint{12.667742in}{3.173752in}}%
\pgfpathlineto{\pgfqpoint{12.669755in}{3.178972in}}%
\pgfpathlineto{\pgfqpoint{12.671768in}{3.172301in}}%
\pgfpathlineto{\pgfqpoint{12.673782in}{3.168627in}}%
\pgfpathlineto{\pgfqpoint{12.681836in}{3.166694in}}%
\pgfpathlineto{\pgfqpoint{12.683849in}{3.165340in}}%
\pgfpathlineto{\pgfqpoint{12.685863in}{3.167177in}}%
\pgfpathlineto{\pgfqpoint{12.687876in}{3.173752in}}%
\pgfpathlineto{\pgfqpoint{12.693916in}{3.175879in}}%
\pgfpathlineto{\pgfqpoint{12.695930in}{3.173075in}}%
\pgfpathlineto{\pgfqpoint{12.697943in}{3.176652in}}%
\pgfpathlineto{\pgfqpoint{12.699957in}{3.177039in}}%
\pgfpathlineto{\pgfqpoint{12.701970in}{3.173075in}}%
\pgfpathlineto{\pgfqpoint{12.708011in}{3.175008in}}%
\pgfpathlineto{\pgfqpoint{12.710024in}{3.174815in}}%
\pgfpathlineto{\pgfqpoint{12.716064in}{3.169208in}}%
\pgfpathlineto{\pgfqpoint{12.722105in}{3.168531in}}%
\pgfpathlineto{\pgfqpoint{12.724118in}{3.170948in}}%
\pgfpathlineto{\pgfqpoint{12.726132in}{3.169498in}}%
\pgfpathlineto{\pgfqpoint{12.730158in}{3.162730in}}%
\pgfpathlineto{\pgfqpoint{12.736199in}{3.161183in}}%
\pgfpathlineto{\pgfqpoint{12.738212in}{3.163310in}}%
\pgfpathlineto{\pgfqpoint{12.740226in}{3.163793in}}%
\pgfpathlineto{\pgfqpoint{12.742239in}{3.158669in}}%
\pgfpathlineto{\pgfqpoint{12.744253in}{3.157122in}}%
\pgfpathlineto{\pgfqpoint{12.750293in}{3.159539in}}%
\pgfpathlineto{\pgfqpoint{12.752306in}{3.162246in}}%
\pgfpathlineto{\pgfqpoint{12.754320in}{3.166887in}}%
\pgfpathlineto{\pgfqpoint{12.756333in}{3.164373in}}%
\pgfpathlineto{\pgfqpoint{12.764387in}{3.167564in}}%
\pgfpathlineto{\pgfqpoint{12.766401in}{3.171238in}}%
\pgfpathlineto{\pgfqpoint{12.768414in}{3.166694in}}%
\pgfpathlineto{\pgfqpoint{12.770427in}{3.164180in}}%
\pgfpathlineto{\pgfqpoint{12.772441in}{3.157799in}}%
\pgfpathlineto{\pgfqpoint{12.778481in}{3.166210in}}%
\pgfpathlineto{\pgfqpoint{12.780495in}{3.170271in}}%
\pgfpathlineto{\pgfqpoint{12.782508in}{3.149774in}}%
\pgfpathlineto{\pgfqpoint{12.784522in}{3.149388in}}%
\pgfpathlineto{\pgfqpoint{12.786535in}{3.146101in}}%
\pgfpathlineto{\pgfqpoint{12.792575in}{3.143877in}}%
\pgfpathlineto{\pgfqpoint{12.794589in}{3.136046in}}%
\pgfpathlineto{\pgfqpoint{12.796602in}{3.137883in}}%
\pgfpathlineto{\pgfqpoint{12.806669in}{3.139043in}}%
\pgfpathlineto{\pgfqpoint{12.808683in}{3.137689in}}%
\pgfpathlineto{\pgfqpoint{12.810696in}{3.138463in}}%
\pgfpathlineto{\pgfqpoint{12.812710in}{3.135562in}}%
\pgfpathlineto{\pgfqpoint{12.814723in}{3.135756in}}%
\pgfpathlineto{\pgfqpoint{12.820764in}{3.137012in}}%
\pgfpathlineto{\pgfqpoint{12.822777in}{3.136239in}}%
\pgfpathlineto{\pgfqpoint{12.824790in}{3.136432in}}%
\pgfpathlineto{\pgfqpoint{12.826804in}{3.132855in}}%
\pgfpathlineto{\pgfqpoint{12.828817in}{3.136239in}}%
\pgfpathlineto{\pgfqpoint{12.834858in}{3.135949in}}%
\pgfpathlineto{\pgfqpoint{12.836871in}{3.134789in}}%
\pgfpathlineto{\pgfqpoint{12.842911in}{3.145327in}}%
\pgfpathlineto{\pgfqpoint{12.850965in}{3.146681in}}%
\pgfpathlineto{\pgfqpoint{12.852979in}{3.152965in}}%
\pgfpathlineto{\pgfqpoint{12.854992in}{3.153352in}}%
\pgfpathlineto{\pgfqpoint{12.857006in}{3.157509in}}%
\pgfpathlineto{\pgfqpoint{12.865059in}{3.159443in}}%
\pgfpathlineto{\pgfqpoint{12.867073in}{3.159153in}}%
\pgfpathlineto{\pgfqpoint{12.869086in}{3.150838in}}%
\pgfpathlineto{\pgfqpoint{12.871100in}{3.153642in}}%
\pgfpathlineto{\pgfqpoint{12.877140in}{3.154319in}}%
\pgfpathlineto{\pgfqpoint{12.879154in}{3.152675in}}%
\pgfpathlineto{\pgfqpoint{12.881167in}{3.156155in}}%
\pgfpathlineto{\pgfqpoint{12.883180in}{3.164663in}}%
\pgfpathlineto{\pgfqpoint{12.885194in}{3.167177in}}%
\pgfpathlineto{\pgfqpoint{12.891234in}{3.169014in}}%
\pgfpathlineto{\pgfqpoint{12.895261in}{3.164857in}}%
\pgfpathlineto{\pgfqpoint{12.897275in}{3.161473in}}%
\pgfpathlineto{\pgfqpoint{12.899288in}{3.165050in}}%
\pgfpathlineto{\pgfqpoint{12.905328in}{3.164470in}}%
\pgfpathlineto{\pgfqpoint{12.907342in}{3.157702in}}%
\pgfpathlineto{\pgfqpoint{12.909355in}{3.155479in}}%
\pgfpathlineto{\pgfqpoint{12.911369in}{3.143007in}}%
\pgfpathlineto{\pgfqpoint{12.913382in}{3.144457in}}%
\pgfpathlineto{\pgfqpoint{12.919422in}{3.149774in}}%
\pgfpathlineto{\pgfqpoint{12.923449in}{3.149194in}}%
\pgfpathlineto{\pgfqpoint{12.925463in}{3.146487in}}%
\pgfpathlineto{\pgfqpoint{12.927476in}{3.149001in}}%
\pgfpathlineto{\pgfqpoint{12.933517in}{3.143877in}}%
\pgfpathlineto{\pgfqpoint{12.935530in}{3.140976in}}%
\pgfpathlineto{\pgfqpoint{12.937544in}{3.142717in}}%
\pgfpathlineto{\pgfqpoint{12.939557in}{3.140396in}}%
\pgfpathlineto{\pgfqpoint{12.941570in}{3.143974in}}%
\pgfpathlineto{\pgfqpoint{12.947611in}{3.148034in}}%
\pgfpathlineto{\pgfqpoint{12.949624in}{3.157992in}}%
\pgfpathlineto{\pgfqpoint{12.951638in}{3.161570in}}%
\pgfpathlineto{\pgfqpoint{12.953651in}{3.163890in}}%
\pgfpathlineto{\pgfqpoint{12.955665in}{3.163987in}}%
\pgfpathlineto{\pgfqpoint{12.961705in}{3.160023in}}%
\pgfpathlineto{\pgfqpoint{12.963718in}{3.168821in}}%
\pgfpathlineto{\pgfqpoint{12.965732in}{3.170271in}}%
\pgfpathlineto{\pgfqpoint{12.967745in}{3.182840in}}%
\pgfpathlineto{\pgfqpoint{12.969759in}{3.178586in}}%
\pgfpathlineto{\pgfqpoint{12.977812in}{3.186707in}}%
\pgfpathlineto{\pgfqpoint{12.979826in}{3.186127in}}%
\pgfpathlineto{\pgfqpoint{12.983853in}{3.182743in}}%
\pgfpathlineto{\pgfqpoint{12.989893in}{3.189801in}}%
\pgfpathlineto{\pgfqpoint{12.991907in}{3.191058in}}%
\pgfpathlineto{\pgfqpoint{12.993920in}{3.194248in}}%
\pgfpathlineto{\pgfqpoint{12.995933in}{3.192605in}}%
\pgfpathlineto{\pgfqpoint{12.997947in}{3.188931in}}%
\pgfpathlineto{\pgfqpoint{13.003987in}{3.191734in}}%
\pgfpathlineto{\pgfqpoint{13.006001in}{3.196762in}}%
\pgfpathlineto{\pgfqpoint{13.008014in}{3.198986in}}%
\pgfpathlineto{\pgfqpoint{13.010028in}{3.195602in}}%
\pgfpathlineto{\pgfqpoint{13.012041in}{3.199276in}}%
\pgfpathlineto{\pgfqpoint{13.018081in}{3.202756in}}%
\pgfpathlineto{\pgfqpoint{13.020095in}{3.202176in}}%
\pgfpathlineto{\pgfqpoint{13.022108in}{3.198986in}}%
\pgfpathlineto{\pgfqpoint{13.024122in}{3.197439in}}%
\pgfpathlineto{\pgfqpoint{13.026135in}{3.199566in}}%
\pgfpathlineto{\pgfqpoint{13.032176in}{3.199276in}}%
\pgfpathlineto{\pgfqpoint{13.034189in}{3.197825in}}%
\pgfpathlineto{\pgfqpoint{13.036202in}{3.193765in}}%
\pgfpathlineto{\pgfqpoint{13.038216in}{3.197439in}}%
\pgfpathlineto{\pgfqpoint{13.040229in}{3.199759in}}%
\pgfpathlineto{\pgfqpoint{13.050297in}{3.201499in}}%
\pgfpathlineto{\pgfqpoint{13.052310in}{3.203820in}}%
\pgfpathlineto{\pgfqpoint{13.054323in}{3.202563in}}%
\pgfpathlineto{\pgfqpoint{13.060364in}{3.213101in}}%
\pgfpathlineto{\pgfqpoint{13.062377in}{3.208750in}}%
\pgfpathlineto{\pgfqpoint{13.066404in}{3.209137in}}%
\pgfpathlineto{\pgfqpoint{13.068418in}{3.206527in}}%
\pgfpathlineto{\pgfqpoint{13.074458in}{3.205463in}}%
\pgfpathlineto{\pgfqpoint{13.076471in}{3.214745in}}%
\pgfpathlineto{\pgfqpoint{13.078485in}{3.216872in}}%
\pgfpathlineto{\pgfqpoint{13.080498in}{3.200822in}}%
\pgfpathlineto{\pgfqpoint{13.082512in}{3.197149in}}%
\pgfpathlineto{\pgfqpoint{13.088552in}{3.201499in}}%
\pgfpathlineto{\pgfqpoint{13.090566in}{3.200919in}}%
\pgfpathlineto{\pgfqpoint{13.092579in}{3.184580in}}%
\pgfpathlineto{\pgfqpoint{13.094592in}{3.184773in}}%
\pgfpathlineto{\pgfqpoint{13.096606in}{3.185643in}}%
\pgfpathlineto{\pgfqpoint{13.104660in}{3.195988in}}%
\pgfpathlineto{\pgfqpoint{13.106673in}{3.198792in}}%
\pgfpathlineto{\pgfqpoint{13.108687in}{3.195215in}}%
\pgfpathlineto{\pgfqpoint{13.110700in}{3.197922in}}%
\pgfpathlineto{\pgfqpoint{13.116740in}{3.195988in}}%
\pgfpathlineto{\pgfqpoint{13.118754in}{3.191444in}}%
\pgfpathlineto{\pgfqpoint{13.120767in}{3.189994in}}%
\pgfpathlineto{\pgfqpoint{13.122781in}{3.196278in}}%
\pgfpathlineto{\pgfqpoint{13.124794in}{3.204400in}}%
\pgfpathlineto{\pgfqpoint{13.130834in}{3.205367in}}%
\pgfpathlineto{\pgfqpoint{13.132848in}{3.202273in}}%
\pgfpathlineto{\pgfqpoint{13.134861in}{3.201983in}}%
\pgfpathlineto{\pgfqpoint{13.136875in}{3.197439in}}%
\pgfpathlineto{\pgfqpoint{13.138888in}{3.166694in}}%
\pgfpathlineto{\pgfqpoint{13.146942in}{3.154995in}}%
\pgfpathlineto{\pgfqpoint{13.148955in}{3.153932in}}%
\pgfpathlineto{\pgfqpoint{13.150969in}{3.159829in}}%
\pgfpathlineto{\pgfqpoint{13.152982in}{3.155575in}}%
\pgfpathlineto{\pgfqpoint{13.159023in}{3.148518in}}%
\pgfpathlineto{\pgfqpoint{13.161036in}{3.149194in}}%
\pgfpathlineto{\pgfqpoint{13.163050in}{3.154222in}}%
\pgfpathlineto{\pgfqpoint{13.165063in}{3.150741in}}%
\pgfpathlineto{\pgfqpoint{13.167077in}{3.151321in}}%
\pgfpathlineto{\pgfqpoint{13.173117in}{3.146487in}}%
\pgfpathlineto{\pgfqpoint{13.175130in}{3.154995in}}%
\pgfpathlineto{\pgfqpoint{13.177144in}{3.160603in}}%
\pgfpathlineto{\pgfqpoint{13.179157in}{3.162440in}}%
\pgfpathlineto{\pgfqpoint{13.181171in}{3.165920in}}%
\pgfpathlineto{\pgfqpoint{13.187211in}{3.173655in}}%
\pgfpathlineto{\pgfqpoint{13.189224in}{3.172398in}}%
\pgfpathlineto{\pgfqpoint{13.191238in}{3.166500in}}%
\pgfpathlineto{\pgfqpoint{13.193251in}{3.175975in}}%
\pgfpathlineto{\pgfqpoint{13.195265in}{3.168337in}}%
\pgfpathlineto{\pgfqpoint{13.201305in}{3.166887in}}%
\pgfpathlineto{\pgfqpoint{13.203319in}{3.171044in}}%
\pgfpathlineto{\pgfqpoint{13.205332in}{3.167467in}}%
\pgfpathlineto{\pgfqpoint{13.209359in}{3.168531in}}%
\pgfpathlineto{\pgfqpoint{13.215399in}{3.173268in}}%
\pgfpathlineto{\pgfqpoint{13.217413in}{3.177329in}}%
\pgfpathlineto{\pgfqpoint{13.219426in}{3.177135in}}%
\pgfpathlineto{\pgfqpoint{13.221440in}{3.182743in}}%
\pgfpathlineto{\pgfqpoint{13.229493in}{3.195795in}}%
\pgfpathlineto{\pgfqpoint{13.231507in}{3.195698in}}%
\pgfpathlineto{\pgfqpoint{13.233520in}{3.194345in}}%
\pgfpathlineto{\pgfqpoint{13.235534in}{3.183710in}}%
\pgfpathlineto{\pgfqpoint{13.237547in}{3.186224in}}%
\pgfpathlineto{\pgfqpoint{13.243587in}{3.184967in}}%
\pgfpathlineto{\pgfqpoint{13.245601in}{3.181486in}}%
\pgfpathlineto{\pgfqpoint{13.247614in}{3.190961in}}%
\pgfpathlineto{\pgfqpoint{13.249628in}{3.192024in}}%
\pgfpathlineto{\pgfqpoint{13.251641in}{3.200242in}}%
\pgfpathlineto{\pgfqpoint{13.257682in}{3.200146in}}%
\pgfpathlineto{\pgfqpoint{13.261709in}{3.196859in}}%
\pgfpathlineto{\pgfqpoint{13.263722in}{3.198212in}}%
\pgfpathlineto{\pgfqpoint{13.265735in}{3.202466in}}%
\pgfpathlineto{\pgfqpoint{13.273789in}{3.205657in}}%
\pgfpathlineto{\pgfqpoint{13.275803in}{3.202176in}}%
\pgfpathlineto{\pgfqpoint{13.277816in}{3.201886in}}%
\pgfpathlineto{\pgfqpoint{13.279830in}{3.200146in}}%
\pgfpathlineto{\pgfqpoint{13.279830in}{3.200146in}}%
\pgfusepath{stroke}%
\end{pgfscope}%
\begin{pgfscope}%
\pgfsetrectcap%
\pgfsetmiterjoin%
\pgfsetlinewidth{0.803000pt}%
\definecolor{currentstroke}{rgb}{1.000000,1.000000,1.000000}%
\pgfsetstrokecolor{currentstroke}%
\pgfsetdash{}{0pt}%
\pgfpathmoveto{\pgfqpoint{8.656250in}{2.792941in}}%
\pgfpathlineto{\pgfqpoint{8.656250in}{3.237059in}}%
\pgfusepath{stroke}%
\end{pgfscope}%
\begin{pgfscope}%
\pgfsetrectcap%
\pgfsetmiterjoin%
\pgfsetlinewidth{0.803000pt}%
\definecolor{currentstroke}{rgb}{1.000000,1.000000,1.000000}%
\pgfsetstrokecolor{currentstroke}%
\pgfsetdash{}{0pt}%
\pgfpathmoveto{\pgfqpoint{13.500000in}{2.792941in}}%
\pgfpathlineto{\pgfqpoint{13.500000in}{3.237059in}}%
\pgfusepath{stroke}%
\end{pgfscope}%
\begin{pgfscope}%
\pgfsetrectcap%
\pgfsetmiterjoin%
\pgfsetlinewidth{0.803000pt}%
\definecolor{currentstroke}{rgb}{1.000000,1.000000,1.000000}%
\pgfsetstrokecolor{currentstroke}%
\pgfsetdash{}{0pt}%
\pgfpathmoveto{\pgfqpoint{8.656250in}{2.792941in}}%
\pgfpathlineto{\pgfqpoint{13.500000in}{2.792941in}}%
\pgfusepath{stroke}%
\end{pgfscope}%
\begin{pgfscope}%
\pgfsetrectcap%
\pgfsetmiterjoin%
\pgfsetlinewidth{0.803000pt}%
\definecolor{currentstroke}{rgb}{1.000000,1.000000,1.000000}%
\pgfsetstrokecolor{currentstroke}%
\pgfsetdash{}{0pt}%
\pgfpathmoveto{\pgfqpoint{8.656250in}{3.237059in}}%
\pgfpathlineto{\pgfqpoint{13.500000in}{3.237059in}}%
\pgfusepath{stroke}%
\end{pgfscope}%
\begin{pgfscope}%
\definecolor{textcolor}{rgb}{0.150000,0.150000,0.150000}%
\pgfsetstrokecolor{textcolor}%
\pgfsetfillcolor{textcolor}%
\pgftext[x=11.078125in,y=3.320392in,,base]{\color{textcolor}\rmfamily\fontsize{16.800000}{20.160000}\selectfont PG}%
\end{pgfscope}%
\begin{pgfscope}%
\pgfsetbuttcap%
\pgfsetmiterjoin%
\definecolor{currentfill}{rgb}{0.917647,0.917647,0.949020}%
\pgfsetfillcolor{currentfill}%
\pgfsetlinewidth{0.000000pt}%
\definecolor{currentstroke}{rgb}{0.000000,0.000000,0.000000}%
\pgfsetstrokecolor{currentstroke}%
\pgfsetstrokeopacity{0.000000}%
\pgfsetdash{}{0pt}%
\pgfpathmoveto{\pgfqpoint{1.875000in}{1.771471in}}%
\pgfpathlineto{\pgfqpoint{6.718750in}{1.771471in}}%
\pgfpathlineto{\pgfqpoint{6.718750in}{2.215588in}}%
\pgfpathlineto{\pgfqpoint{1.875000in}{2.215588in}}%
\pgfpathclose%
\pgfusepath{fill}%
\end{pgfscope}%
\begin{pgfscope}%
\pgfpathrectangle{\pgfqpoint{1.875000in}{1.771471in}}{\pgfqpoint{4.843750in}{0.444118in}}%
\pgfusepath{clip}%
\pgfsetroundcap%
\pgfsetroundjoin%
\pgfsetlinewidth{0.803000pt}%
\definecolor{currentstroke}{rgb}{1.000000,1.000000,1.000000}%
\pgfsetstrokecolor{currentstroke}%
\pgfsetdash{}{0pt}%
\pgfpathmoveto{\pgfqpoint{2.091144in}{1.771471in}}%
\pgfpathlineto{\pgfqpoint{2.091144in}{2.215588in}}%
\pgfusepath{stroke}%
\end{pgfscope}%
\begin{pgfscope}%
\definecolor{textcolor}{rgb}{0.150000,0.150000,0.150000}%
\pgfsetstrokecolor{textcolor}%
\pgfsetfillcolor{textcolor}%
\pgftext[x=2.091144in,y=1.674248in,,top]{\color{textcolor}\rmfamily\fontsize{14.000000}{16.800000}\selectfont 2012}%
\end{pgfscope}%
\begin{pgfscope}%
\pgfpathrectangle{\pgfqpoint{1.875000in}{1.771471in}}{\pgfqpoint{4.843750in}{0.444118in}}%
\pgfusepath{clip}%
\pgfsetroundcap%
\pgfsetroundjoin%
\pgfsetlinewidth{0.803000pt}%
\definecolor{currentstroke}{rgb}{1.000000,1.000000,1.000000}%
\pgfsetstrokecolor{currentstroke}%
\pgfsetdash{}{0pt}%
\pgfpathmoveto{\pgfqpoint{2.828065in}{1.771471in}}%
\pgfpathlineto{\pgfqpoint{2.828065in}{2.215588in}}%
\pgfusepath{stroke}%
\end{pgfscope}%
\begin{pgfscope}%
\definecolor{textcolor}{rgb}{0.150000,0.150000,0.150000}%
\pgfsetstrokecolor{textcolor}%
\pgfsetfillcolor{textcolor}%
\pgftext[x=2.828065in,y=1.674248in,,top]{\color{textcolor}\rmfamily\fontsize{14.000000}{16.800000}\selectfont 2013}%
\end{pgfscope}%
\begin{pgfscope}%
\pgfpathrectangle{\pgfqpoint{1.875000in}{1.771471in}}{\pgfqpoint{4.843750in}{0.444118in}}%
\pgfusepath{clip}%
\pgfsetroundcap%
\pgfsetroundjoin%
\pgfsetlinewidth{0.803000pt}%
\definecolor{currentstroke}{rgb}{1.000000,1.000000,1.000000}%
\pgfsetstrokecolor{currentstroke}%
\pgfsetdash{}{0pt}%
\pgfpathmoveto{\pgfqpoint{3.562973in}{1.771471in}}%
\pgfpathlineto{\pgfqpoint{3.562973in}{2.215588in}}%
\pgfusepath{stroke}%
\end{pgfscope}%
\begin{pgfscope}%
\definecolor{textcolor}{rgb}{0.150000,0.150000,0.150000}%
\pgfsetstrokecolor{textcolor}%
\pgfsetfillcolor{textcolor}%
\pgftext[x=3.562973in,y=1.674248in,,top]{\color{textcolor}\rmfamily\fontsize{14.000000}{16.800000}\selectfont 2014}%
\end{pgfscope}%
\begin{pgfscope}%
\pgfpathrectangle{\pgfqpoint{1.875000in}{1.771471in}}{\pgfqpoint{4.843750in}{0.444118in}}%
\pgfusepath{clip}%
\pgfsetroundcap%
\pgfsetroundjoin%
\pgfsetlinewidth{0.803000pt}%
\definecolor{currentstroke}{rgb}{1.000000,1.000000,1.000000}%
\pgfsetstrokecolor{currentstroke}%
\pgfsetdash{}{0pt}%
\pgfpathmoveto{\pgfqpoint{4.297882in}{1.771471in}}%
\pgfpathlineto{\pgfqpoint{4.297882in}{2.215588in}}%
\pgfusepath{stroke}%
\end{pgfscope}%
\begin{pgfscope}%
\definecolor{textcolor}{rgb}{0.150000,0.150000,0.150000}%
\pgfsetstrokecolor{textcolor}%
\pgfsetfillcolor{textcolor}%
\pgftext[x=4.297882in,y=1.674248in,,top]{\color{textcolor}\rmfamily\fontsize{14.000000}{16.800000}\selectfont 2015}%
\end{pgfscope}%
\begin{pgfscope}%
\pgfpathrectangle{\pgfqpoint{1.875000in}{1.771471in}}{\pgfqpoint{4.843750in}{0.444118in}}%
\pgfusepath{clip}%
\pgfsetroundcap%
\pgfsetroundjoin%
\pgfsetlinewidth{0.803000pt}%
\definecolor{currentstroke}{rgb}{1.000000,1.000000,1.000000}%
\pgfsetstrokecolor{currentstroke}%
\pgfsetdash{}{0pt}%
\pgfpathmoveto{\pgfqpoint{5.032790in}{1.771471in}}%
\pgfpathlineto{\pgfqpoint{5.032790in}{2.215588in}}%
\pgfusepath{stroke}%
\end{pgfscope}%
\begin{pgfscope}%
\definecolor{textcolor}{rgb}{0.150000,0.150000,0.150000}%
\pgfsetstrokecolor{textcolor}%
\pgfsetfillcolor{textcolor}%
\pgftext[x=5.032790in,y=1.674248in,,top]{\color{textcolor}\rmfamily\fontsize{14.000000}{16.800000}\selectfont 2016}%
\end{pgfscope}%
\begin{pgfscope}%
\pgfpathrectangle{\pgfqpoint{1.875000in}{1.771471in}}{\pgfqpoint{4.843750in}{0.444118in}}%
\pgfusepath{clip}%
\pgfsetroundcap%
\pgfsetroundjoin%
\pgfsetlinewidth{0.803000pt}%
\definecolor{currentstroke}{rgb}{1.000000,1.000000,1.000000}%
\pgfsetstrokecolor{currentstroke}%
\pgfsetdash{}{0pt}%
\pgfpathmoveto{\pgfqpoint{5.769712in}{1.771471in}}%
\pgfpathlineto{\pgfqpoint{5.769712in}{2.215588in}}%
\pgfusepath{stroke}%
\end{pgfscope}%
\begin{pgfscope}%
\definecolor{textcolor}{rgb}{0.150000,0.150000,0.150000}%
\pgfsetstrokecolor{textcolor}%
\pgfsetfillcolor{textcolor}%
\pgftext[x=5.769712in,y=1.674248in,,top]{\color{textcolor}\rmfamily\fontsize{14.000000}{16.800000}\selectfont 2017}%
\end{pgfscope}%
\begin{pgfscope}%
\pgfpathrectangle{\pgfqpoint{1.875000in}{1.771471in}}{\pgfqpoint{4.843750in}{0.444118in}}%
\pgfusepath{clip}%
\pgfsetroundcap%
\pgfsetroundjoin%
\pgfsetlinewidth{0.803000pt}%
\definecolor{currentstroke}{rgb}{1.000000,1.000000,1.000000}%
\pgfsetstrokecolor{currentstroke}%
\pgfsetdash{}{0pt}%
\pgfpathmoveto{\pgfqpoint{6.504620in}{1.771471in}}%
\pgfpathlineto{\pgfqpoint{6.504620in}{2.215588in}}%
\pgfusepath{stroke}%
\end{pgfscope}%
\begin{pgfscope}%
\definecolor{textcolor}{rgb}{0.150000,0.150000,0.150000}%
\pgfsetstrokecolor{textcolor}%
\pgfsetfillcolor{textcolor}%
\pgftext[x=6.504620in,y=1.674248in,,top]{\color{textcolor}\rmfamily\fontsize{14.000000}{16.800000}\selectfont 2018}%
\end{pgfscope}%
\begin{pgfscope}%
\pgfpathrectangle{\pgfqpoint{1.875000in}{1.771471in}}{\pgfqpoint{4.843750in}{0.444118in}}%
\pgfusepath{clip}%
\pgfsetroundcap%
\pgfsetroundjoin%
\pgfsetlinewidth{0.803000pt}%
\definecolor{currentstroke}{rgb}{1.000000,1.000000,1.000000}%
\pgfsetstrokecolor{currentstroke}%
\pgfsetdash{}{0pt}%
\pgfpathmoveto{\pgfqpoint{1.875000in}{1.885784in}}%
\pgfpathlineto{\pgfqpoint{6.718750in}{1.885784in}}%
\pgfusepath{stroke}%
\end{pgfscope}%
\begin{pgfscope}%
\definecolor{textcolor}{rgb}{0.150000,0.150000,0.150000}%
\pgfsetstrokecolor{textcolor}%
\pgfsetfillcolor{textcolor}%
\pgftext[x=1.530355in,y=1.811918in,left,base]{\color{textcolor}\rmfamily\fontsize{14.000000}{16.800000}\selectfont 75}%
\end{pgfscope}%
\begin{pgfscope}%
\pgfpathrectangle{\pgfqpoint{1.875000in}{1.771471in}}{\pgfqpoint{4.843750in}{0.444118in}}%
\pgfusepath{clip}%
\pgfsetroundcap%
\pgfsetroundjoin%
\pgfsetlinewidth{0.803000pt}%
\definecolor{currentstroke}{rgb}{1.000000,1.000000,1.000000}%
\pgfsetstrokecolor{currentstroke}%
\pgfsetdash{}{0pt}%
\pgfpathmoveto{\pgfqpoint{1.875000in}{2.044140in}}%
\pgfpathlineto{\pgfqpoint{6.718750in}{2.044140in}}%
\pgfusepath{stroke}%
\end{pgfscope}%
\begin{pgfscope}%
\definecolor{textcolor}{rgb}{0.150000,0.150000,0.150000}%
\pgfsetstrokecolor{textcolor}%
\pgfsetfillcolor{textcolor}%
\pgftext[x=1.406643in,y=1.970274in,left,base]{\color{textcolor}\rmfamily\fontsize{14.000000}{16.800000}\selectfont 100}%
\end{pgfscope}%
\begin{pgfscope}%
\pgfpathrectangle{\pgfqpoint{1.875000in}{1.771471in}}{\pgfqpoint{4.843750in}{0.444118in}}%
\pgfusepath{clip}%
\pgfsetroundcap%
\pgfsetroundjoin%
\pgfsetlinewidth{0.803000pt}%
\definecolor{currentstroke}{rgb}{1.000000,1.000000,1.000000}%
\pgfsetstrokecolor{currentstroke}%
\pgfsetdash{}{0pt}%
\pgfpathmoveto{\pgfqpoint{1.875000in}{2.202495in}}%
\pgfpathlineto{\pgfqpoint{6.718750in}{2.202495in}}%
\pgfusepath{stroke}%
\end{pgfscope}%
\begin{pgfscope}%
\definecolor{textcolor}{rgb}{0.150000,0.150000,0.150000}%
\pgfsetstrokecolor{textcolor}%
\pgfsetfillcolor{textcolor}%
\pgftext[x=1.406643in,y=2.128629in,left,base]{\color{textcolor}\rmfamily\fontsize{14.000000}{16.800000}\selectfont 125}%
\end{pgfscope}%
\begin{pgfscope}%
\pgfpathrectangle{\pgfqpoint{1.875000in}{1.771471in}}{\pgfqpoint{4.843750in}{0.444118in}}%
\pgfusepath{clip}%
\pgfsetroundcap%
\pgfsetroundjoin%
\pgfsetlinewidth{1.505625pt}%
\definecolor{currentstroke}{rgb}{0.121569,0.466667,0.705882}%
\pgfsetstrokecolor{currentstroke}%
\pgfsetdash{}{0pt}%
\pgfpathmoveto{\pgfqpoint{2.095170in}{1.807177in}}%
\pgfpathlineto{\pgfqpoint{2.097184in}{1.809204in}}%
\pgfpathlineto{\pgfqpoint{2.099197in}{1.805403in}}%
\pgfpathlineto{\pgfqpoint{2.101211in}{1.803123in}}%
\pgfpathlineto{\pgfqpoint{2.107251in}{1.804200in}}%
\pgfpathlineto{\pgfqpoint{2.109265in}{1.814398in}}%
\pgfpathlineto{\pgfqpoint{2.113291in}{1.820859in}}%
\pgfpathlineto{\pgfqpoint{2.115305in}{1.814714in}}%
\pgfpathlineto{\pgfqpoint{2.123359in}{1.819782in}}%
\pgfpathlineto{\pgfqpoint{2.125372in}{1.822759in}}%
\pgfpathlineto{\pgfqpoint{2.129399in}{1.817945in}}%
\pgfpathlineto{\pgfqpoint{2.135439in}{1.818832in}}%
\pgfpathlineto{\pgfqpoint{2.137453in}{1.823709in}}%
\pgfpathlineto{\pgfqpoint{2.141480in}{1.821745in}}%
\pgfpathlineto{\pgfqpoint{2.143493in}{1.822885in}}%
\pgfpathlineto{\pgfqpoint{2.149534in}{1.822822in}}%
\pgfpathlineto{\pgfqpoint{2.151547in}{1.826749in}}%
\pgfpathlineto{\pgfqpoint{2.153560in}{1.836694in}}%
\pgfpathlineto{\pgfqpoint{2.155574in}{1.835617in}}%
\pgfpathlineto{\pgfqpoint{2.157587in}{1.841065in}}%
\pgfpathlineto{\pgfqpoint{2.163628in}{1.838531in}}%
\pgfpathlineto{\pgfqpoint{2.165641in}{1.837011in}}%
\pgfpathlineto{\pgfqpoint{2.167655in}{1.844739in}}%
\pgfpathlineto{\pgfqpoint{2.169668in}{1.855570in}}%
\pgfpathlineto{\pgfqpoint{2.171681in}{1.854113in}}%
\pgfpathlineto{\pgfqpoint{2.177722in}{1.861398in}}%
\pgfpathlineto{\pgfqpoint{2.179735in}{1.860131in}}%
\pgfpathlineto{\pgfqpoint{2.181749in}{1.851833in}}%
\pgfpathlineto{\pgfqpoint{2.185776in}{1.858104in}}%
\pgfpathlineto{\pgfqpoint{2.195843in}{1.859054in}}%
\pgfpathlineto{\pgfqpoint{2.197856in}{1.856584in}}%
\pgfpathlineto{\pgfqpoint{2.199870in}{1.859117in}}%
\pgfpathlineto{\pgfqpoint{2.207923in}{1.856457in}}%
\pgfpathlineto{\pgfqpoint{2.213964in}{1.862158in}}%
\pgfpathlineto{\pgfqpoint{2.220004in}{1.855507in}}%
\pgfpathlineto{\pgfqpoint{2.222018in}{1.845372in}}%
\pgfpathlineto{\pgfqpoint{2.226045in}{1.856774in}}%
\pgfpathlineto{\pgfqpoint{2.228058in}{1.856900in}}%
\pgfpathlineto{\pgfqpoint{2.234098in}{1.859244in}}%
\pgfpathlineto{\pgfqpoint{2.236112in}{1.872989in}}%
\pgfpathlineto{\pgfqpoint{2.238125in}{1.874319in}}%
\pgfpathlineto{\pgfqpoint{2.240139in}{1.874699in}}%
\pgfpathlineto{\pgfqpoint{2.242152in}{1.867225in}}%
\pgfpathlineto{\pgfqpoint{2.248192in}{1.862601in}}%
\pgfpathlineto{\pgfqpoint{2.250206in}{1.855317in}}%
\pgfpathlineto{\pgfqpoint{2.256246in}{1.847526in}}%
\pgfpathlineto{\pgfqpoint{2.262287in}{1.856647in}}%
\pgfpathlineto{\pgfqpoint{2.264300in}{1.854557in}}%
\pgfpathlineto{\pgfqpoint{2.266313in}{1.846069in}}%
\pgfpathlineto{\pgfqpoint{2.270340in}{1.853606in}}%
\pgfpathlineto{\pgfqpoint{2.276381in}{1.852530in}}%
\pgfpathlineto{\pgfqpoint{2.278394in}{1.849679in}}%
\pgfpathlineto{\pgfqpoint{2.282421in}{1.846449in}}%
\pgfpathlineto{\pgfqpoint{2.290475in}{1.839164in}}%
\pgfpathlineto{\pgfqpoint{2.292488in}{1.829790in}}%
\pgfpathlineto{\pgfqpoint{2.294502in}{1.835934in}}%
\pgfpathlineto{\pgfqpoint{2.296515in}{1.844168in}}%
\pgfpathlineto{\pgfqpoint{2.298529in}{1.836884in}}%
\pgfpathlineto{\pgfqpoint{2.304569in}{1.837264in}}%
\pgfpathlineto{\pgfqpoint{2.306582in}{1.844549in}}%
\pgfpathlineto{\pgfqpoint{2.308596in}{1.844358in}}%
\pgfpathlineto{\pgfqpoint{2.310609in}{1.840051in}}%
\pgfpathlineto{\pgfqpoint{2.312623in}{1.843282in}}%
\pgfpathlineto{\pgfqpoint{2.318663in}{1.836567in}}%
\pgfpathlineto{\pgfqpoint{2.320677in}{1.837137in}}%
\pgfpathlineto{\pgfqpoint{2.322690in}{1.836947in}}%
\pgfpathlineto{\pgfqpoint{2.324703in}{1.844612in}}%
\pgfpathlineto{\pgfqpoint{2.326717in}{1.848602in}}%
\pgfpathlineto{\pgfqpoint{2.334771in}{1.845562in}}%
\pgfpathlineto{\pgfqpoint{2.336784in}{1.846195in}}%
\pgfpathlineto{\pgfqpoint{2.338798in}{1.842775in}}%
\pgfpathlineto{\pgfqpoint{2.340811in}{1.835174in}}%
\pgfpathlineto{\pgfqpoint{2.348865in}{1.831627in}}%
\pgfpathlineto{\pgfqpoint{2.350878in}{1.821935in}}%
\pgfpathlineto{\pgfqpoint{2.352892in}{1.823456in}}%
\pgfpathlineto{\pgfqpoint{2.354905in}{1.822885in}}%
\pgfpathlineto{\pgfqpoint{2.360945in}{1.816298in}}%
\pgfpathlineto{\pgfqpoint{2.362959in}{1.817501in}}%
\pgfpathlineto{\pgfqpoint{2.364972in}{1.812181in}}%
\pgfpathlineto{\pgfqpoint{2.366986in}{1.804770in}}%
\pgfpathlineto{\pgfqpoint{2.368999in}{1.799702in}}%
\pgfpathlineto{\pgfqpoint{2.375040in}{1.806797in}}%
\pgfpathlineto{\pgfqpoint{2.379067in}{1.808697in}}%
\pgfpathlineto{\pgfqpoint{2.383093in}{1.803123in}}%
\pgfpathlineto{\pgfqpoint{2.391147in}{1.814271in}}%
\pgfpathlineto{\pgfqpoint{2.393161in}{1.806987in}}%
\pgfpathlineto{\pgfqpoint{2.395174in}{1.809014in}}%
\pgfpathlineto{\pgfqpoint{2.397188in}{1.797739in}}%
\pgfpathlineto{\pgfqpoint{2.403228in}{1.794698in}}%
\pgfpathlineto{\pgfqpoint{2.405241in}{1.791658in}}%
\pgfpathlineto{\pgfqpoint{2.407255in}{1.806480in}}%
\pgfpathlineto{\pgfqpoint{2.409268in}{1.815918in}}%
\pgfpathlineto{\pgfqpoint{2.411282in}{1.816488in}}%
\pgfpathlineto{\pgfqpoint{2.419335in}{1.810280in}}%
\pgfpathlineto{\pgfqpoint{2.421349in}{1.805910in}}%
\pgfpathlineto{\pgfqpoint{2.423362in}{1.809140in}}%
\pgfpathlineto{\pgfqpoint{2.425376in}{1.811040in}}%
\pgfpathlineto{\pgfqpoint{2.431416in}{1.813764in}}%
\pgfpathlineto{\pgfqpoint{2.433430in}{1.821302in}}%
\pgfpathlineto{\pgfqpoint{2.435443in}{1.818642in}}%
\pgfpathlineto{\pgfqpoint{2.437456in}{1.812814in}}%
\pgfpathlineto{\pgfqpoint{2.439470in}{1.814841in}}%
\pgfpathlineto{\pgfqpoint{2.445510in}{1.807050in}}%
\pgfpathlineto{\pgfqpoint{2.447524in}{1.806416in}}%
\pgfpathlineto{\pgfqpoint{2.449537in}{1.809267in}}%
\pgfpathlineto{\pgfqpoint{2.451551in}{1.800842in}}%
\pgfpathlineto{\pgfqpoint{2.453564in}{1.816615in}}%
\pgfpathlineto{\pgfqpoint{2.459604in}{1.814018in}}%
\pgfpathlineto{\pgfqpoint{2.461618in}{1.817818in}}%
\pgfpathlineto{\pgfqpoint{2.465645in}{1.815854in}}%
\pgfpathlineto{\pgfqpoint{2.467658in}{1.808887in}}%
\pgfpathlineto{\pgfqpoint{2.473699in}{1.810154in}}%
\pgfpathlineto{\pgfqpoint{2.475712in}{1.809204in}}%
\pgfpathlineto{\pgfqpoint{2.477725in}{1.800399in}}%
\pgfpathlineto{\pgfqpoint{2.479739in}{1.796788in}}%
\pgfpathlineto{\pgfqpoint{2.481752in}{1.806163in}}%
\pgfpathlineto{\pgfqpoint{2.487793in}{1.804136in}}%
\pgfpathlineto{\pgfqpoint{2.489806in}{1.806226in}}%
\pgfpathlineto{\pgfqpoint{2.493833in}{1.818198in}}%
\pgfpathlineto{\pgfqpoint{2.495846in}{1.809647in}}%
\pgfpathlineto{\pgfqpoint{2.501887in}{1.804516in}}%
\pgfpathlineto{\pgfqpoint{2.503900in}{1.797295in}}%
\pgfpathlineto{\pgfqpoint{2.505914in}{1.800906in}}%
\pgfpathlineto{\pgfqpoint{2.507927in}{1.802616in}}%
\pgfpathlineto{\pgfqpoint{2.509941in}{1.809900in}}%
\pgfpathlineto{\pgfqpoint{2.515981in}{1.813638in}}%
\pgfpathlineto{\pgfqpoint{2.517994in}{1.810787in}}%
\pgfpathlineto{\pgfqpoint{2.520008in}{1.812624in}}%
\pgfpathlineto{\pgfqpoint{2.522021in}{1.810787in}}%
\pgfpathlineto{\pgfqpoint{2.524035in}{1.823519in}}%
\pgfpathlineto{\pgfqpoint{2.530075in}{1.822632in}}%
\pgfpathlineto{\pgfqpoint{2.532089in}{1.830550in}}%
\pgfpathlineto{\pgfqpoint{2.536115in}{1.824659in}}%
\pgfpathlineto{\pgfqpoint{2.538129in}{1.829283in}}%
\pgfpathlineto{\pgfqpoint{2.544169in}{1.827066in}}%
\pgfpathlineto{\pgfqpoint{2.546183in}{1.828713in}}%
\pgfpathlineto{\pgfqpoint{2.550210in}{1.836947in}}%
\pgfpathlineto{\pgfqpoint{2.552223in}{1.845625in}}%
\pgfpathlineto{\pgfqpoint{2.558263in}{1.844042in}}%
\pgfpathlineto{\pgfqpoint{2.560277in}{1.839671in}}%
\pgfpathlineto{\pgfqpoint{2.562290in}{1.841761in}}%
\pgfpathlineto{\pgfqpoint{2.564304in}{1.839291in}}%
\pgfpathlineto{\pgfqpoint{2.566317in}{1.844042in}}%
\pgfpathlineto{\pgfqpoint{2.574371in}{1.847082in}}%
\pgfpathlineto{\pgfqpoint{2.576384in}{1.844549in}}%
\pgfpathlineto{\pgfqpoint{2.578398in}{1.838531in}}%
\pgfpathlineto{\pgfqpoint{2.580411in}{1.842838in}}%
\pgfpathlineto{\pgfqpoint{2.590478in}{1.832957in}}%
\pgfpathlineto{\pgfqpoint{2.592492in}{1.840368in}}%
\pgfpathlineto{\pgfqpoint{2.594505in}{1.840431in}}%
\pgfpathlineto{\pgfqpoint{2.600546in}{1.836124in}}%
\pgfpathlineto{\pgfqpoint{2.604573in}{1.837264in}}%
\pgfpathlineto{\pgfqpoint{2.608599in}{1.856900in}}%
\pgfpathlineto{\pgfqpoint{2.614640in}{1.855253in}}%
\pgfpathlineto{\pgfqpoint{2.616653in}{1.851643in}}%
\pgfpathlineto{\pgfqpoint{2.618667in}{1.852910in}}%
\pgfpathlineto{\pgfqpoint{2.620680in}{1.848602in}}%
\pgfpathlineto{\pgfqpoint{2.622694in}{1.847652in}}%
\pgfpathlineto{\pgfqpoint{2.628734in}{1.843788in}}%
\pgfpathlineto{\pgfqpoint{2.630747in}{1.836947in}}%
\pgfpathlineto{\pgfqpoint{2.634774in}{1.834667in}}%
\pgfpathlineto{\pgfqpoint{2.636788in}{1.834350in}}%
\pgfpathlineto{\pgfqpoint{2.644842in}{1.834920in}}%
\pgfpathlineto{\pgfqpoint{2.646855in}{1.835807in}}%
\pgfpathlineto{\pgfqpoint{2.656922in}{1.834794in}}%
\pgfpathlineto{\pgfqpoint{2.660949in}{1.822062in}}%
\pgfpathlineto{\pgfqpoint{2.662963in}{1.822252in}}%
\pgfpathlineto{\pgfqpoint{2.664976in}{1.821745in}}%
\pgfpathlineto{\pgfqpoint{2.671016in}{1.822759in}}%
\pgfpathlineto{\pgfqpoint{2.673030in}{1.830613in}}%
\pgfpathlineto{\pgfqpoint{2.677057in}{1.839481in}}%
\pgfpathlineto{\pgfqpoint{2.679070in}{1.832767in}}%
\pgfpathlineto{\pgfqpoint{2.685110in}{1.831880in}}%
\pgfpathlineto{\pgfqpoint{2.687124in}{1.827763in}}%
\pgfpathlineto{\pgfqpoint{2.689137in}{1.832260in}}%
\pgfpathlineto{\pgfqpoint{2.691151in}{1.828966in}}%
\pgfpathlineto{\pgfqpoint{2.693164in}{1.833907in}}%
\pgfpathlineto{\pgfqpoint{2.703232in}{1.833654in}}%
\pgfpathlineto{\pgfqpoint{2.705245in}{1.838531in}}%
\pgfpathlineto{\pgfqpoint{2.707258in}{1.833147in}}%
\pgfpathlineto{\pgfqpoint{2.713299in}{1.832260in}}%
\pgfpathlineto{\pgfqpoint{2.715312in}{1.843472in}}%
\pgfpathlineto{\pgfqpoint{2.717326in}{1.831057in}}%
\pgfpathlineto{\pgfqpoint{2.719339in}{1.822822in}}%
\pgfpathlineto{\pgfqpoint{2.721353in}{1.821112in}}%
\pgfpathlineto{\pgfqpoint{2.727393in}{1.827129in}}%
\pgfpathlineto{\pgfqpoint{2.729406in}{1.827383in}}%
\pgfpathlineto{\pgfqpoint{2.731420in}{1.817501in}}%
\pgfpathlineto{\pgfqpoint{2.733433in}{1.818515in}}%
\pgfpathlineto{\pgfqpoint{2.741487in}{1.828016in}}%
\pgfpathlineto{\pgfqpoint{2.743500in}{1.828586in}}%
\pgfpathlineto{\pgfqpoint{2.749541in}{1.839101in}}%
\pgfpathlineto{\pgfqpoint{2.755581in}{1.839481in}}%
\pgfpathlineto{\pgfqpoint{2.757595in}{1.840241in}}%
\pgfpathlineto{\pgfqpoint{2.759608in}{1.845562in}}%
\pgfpathlineto{\pgfqpoint{2.761621in}{1.845372in}}%
\pgfpathlineto{\pgfqpoint{2.763635in}{1.847272in}}%
\pgfpathlineto{\pgfqpoint{2.769675in}{1.845562in}}%
\pgfpathlineto{\pgfqpoint{2.771689in}{1.847399in}}%
\pgfpathlineto{\pgfqpoint{2.773702in}{1.847906in}}%
\pgfpathlineto{\pgfqpoint{2.775716in}{1.851389in}}%
\pgfpathlineto{\pgfqpoint{2.777729in}{1.852023in}}%
\pgfpathlineto{\pgfqpoint{2.783769in}{1.852150in}}%
\pgfpathlineto{\pgfqpoint{2.785783in}{1.853163in}}%
\pgfpathlineto{\pgfqpoint{2.787796in}{1.852086in}}%
\pgfpathlineto{\pgfqpoint{2.789810in}{1.848666in}}%
\pgfpathlineto{\pgfqpoint{2.791823in}{1.846512in}}%
\pgfpathlineto{\pgfqpoint{2.797864in}{1.846575in}}%
\pgfpathlineto{\pgfqpoint{2.799877in}{1.858801in}}%
\pgfpathlineto{\pgfqpoint{2.801890in}{1.863171in}}%
\pgfpathlineto{\pgfqpoint{2.803904in}{1.863931in}}%
\pgfpathlineto{\pgfqpoint{2.805917in}{1.860511in}}%
\pgfpathlineto{\pgfqpoint{2.815985in}{1.858167in}}%
\pgfpathlineto{\pgfqpoint{2.817998in}{1.857914in}}%
\pgfpathlineto{\pgfqpoint{2.820011in}{1.851073in}}%
\pgfpathlineto{\pgfqpoint{2.826052in}{1.857597in}}%
\pgfpathlineto{\pgfqpoint{2.830079in}{1.868429in}}%
\pgfpathlineto{\pgfqpoint{2.832092in}{1.870139in}}%
\pgfpathlineto{\pgfqpoint{2.834106in}{1.873813in}}%
\pgfpathlineto{\pgfqpoint{2.840146in}{1.871532in}}%
\pgfpathlineto{\pgfqpoint{2.842159in}{1.866022in}}%
\pgfpathlineto{\pgfqpoint{2.844173in}{1.871469in}}%
\pgfpathlineto{\pgfqpoint{2.846186in}{1.873749in}}%
\pgfpathlineto{\pgfqpoint{2.856254in}{1.879133in}}%
\pgfpathlineto{\pgfqpoint{2.858267in}{1.876980in}}%
\pgfpathlineto{\pgfqpoint{2.860280in}{1.881540in}}%
\pgfpathlineto{\pgfqpoint{2.862294in}{1.884517in}}%
\pgfpathlineto{\pgfqpoint{2.870348in}{1.887368in}}%
\pgfpathlineto{\pgfqpoint{2.872361in}{1.890598in}}%
\pgfpathlineto{\pgfqpoint{2.876388in}{1.899973in}}%
\pgfpathlineto{\pgfqpoint{2.884442in}{1.899846in}}%
\pgfpathlineto{\pgfqpoint{2.886455in}{1.896236in}}%
\pgfpathlineto{\pgfqpoint{2.888469in}{1.887875in}}%
\pgfpathlineto{\pgfqpoint{2.890482in}{1.900290in}}%
\pgfpathlineto{\pgfqpoint{2.896522in}{1.898643in}}%
\pgfpathlineto{\pgfqpoint{2.898536in}{1.897123in}}%
\pgfpathlineto{\pgfqpoint{2.900549in}{1.897693in}}%
\pgfpathlineto{\pgfqpoint{2.902563in}{1.900860in}}%
\pgfpathlineto{\pgfqpoint{2.904576in}{1.901620in}}%
\pgfpathlineto{\pgfqpoint{2.910617in}{1.899023in}}%
\pgfpathlineto{\pgfqpoint{2.912630in}{1.901050in}}%
\pgfpathlineto{\pgfqpoint{2.914643in}{1.901303in}}%
\pgfpathlineto{\pgfqpoint{2.916657in}{1.902380in}}%
\pgfpathlineto{\pgfqpoint{2.918670in}{1.908334in}}%
\pgfpathlineto{\pgfqpoint{2.926724in}{1.909664in}}%
\pgfpathlineto{\pgfqpoint{2.930751in}{1.900860in}}%
\pgfpathlineto{\pgfqpoint{2.932765in}{1.906751in}}%
\pgfpathlineto{\pgfqpoint{2.938805in}{1.895159in}}%
\pgfpathlineto{\pgfqpoint{2.940818in}{1.899656in}}%
\pgfpathlineto{\pgfqpoint{2.942832in}{1.907194in}}%
\pgfpathlineto{\pgfqpoint{2.944845in}{1.907067in}}%
\pgfpathlineto{\pgfqpoint{2.946859in}{1.904787in}}%
\pgfpathlineto{\pgfqpoint{2.952899in}{1.899276in}}%
\pgfpathlineto{\pgfqpoint{2.954912in}{1.909664in}}%
\pgfpathlineto{\pgfqpoint{2.956926in}{1.909918in}}%
\pgfpathlineto{\pgfqpoint{2.958939in}{1.913465in}}%
\pgfpathlineto{\pgfqpoint{2.960953in}{1.915428in}}%
\pgfpathlineto{\pgfqpoint{2.969007in}{1.921256in}}%
\pgfpathlineto{\pgfqpoint{2.971020in}{1.920939in}}%
\pgfpathlineto{\pgfqpoint{2.973033in}{1.922966in}}%
\pgfpathlineto{\pgfqpoint{2.981087in}{1.919229in}}%
\pgfpathlineto{\pgfqpoint{2.985114in}{1.922966in}}%
\pgfpathlineto{\pgfqpoint{2.987128in}{1.917392in}}%
\pgfpathlineto{\pgfqpoint{2.989141in}{1.923726in}}%
\pgfpathlineto{\pgfqpoint{2.997195in}{1.918596in}}%
\pgfpathlineto{\pgfqpoint{2.999208in}{1.918406in}}%
\pgfpathlineto{\pgfqpoint{3.001222in}{1.922903in}}%
\pgfpathlineto{\pgfqpoint{3.009275in}{1.920052in}}%
\pgfpathlineto{\pgfqpoint{3.011289in}{1.920432in}}%
\pgfpathlineto{\pgfqpoint{3.013302in}{1.921636in}}%
\pgfpathlineto{\pgfqpoint{3.015316in}{1.921383in}}%
\pgfpathlineto{\pgfqpoint{3.017329in}{1.918532in}}%
\pgfpathlineto{\pgfqpoint{3.023370in}{1.924423in}}%
\pgfpathlineto{\pgfqpoint{3.029410in}{1.935951in}}%
\pgfpathlineto{\pgfqpoint{3.031423in}{1.935255in}}%
\pgfpathlineto{\pgfqpoint{3.037464in}{1.923346in}}%
\pgfpathlineto{\pgfqpoint{3.039477in}{1.928794in}}%
\pgfpathlineto{\pgfqpoint{3.043504in}{1.912578in}}%
\pgfpathlineto{\pgfqpoint{3.045518in}{1.921699in}}%
\pgfpathlineto{\pgfqpoint{3.051558in}{1.923980in}}%
\pgfpathlineto{\pgfqpoint{3.055585in}{1.915048in}}%
\pgfpathlineto{\pgfqpoint{3.057598in}{1.915618in}}%
\pgfpathlineto{\pgfqpoint{3.059612in}{1.910361in}}%
\pgfpathlineto{\pgfqpoint{3.065652in}{1.912958in}}%
\pgfpathlineto{\pgfqpoint{3.069679in}{1.909854in}}%
\pgfpathlineto{\pgfqpoint{3.071692in}{1.913655in}}%
\pgfpathlineto{\pgfqpoint{3.073706in}{1.921129in}}%
\pgfpathlineto{\pgfqpoint{3.079746in}{1.923220in}}%
\pgfpathlineto{\pgfqpoint{3.085786in}{1.930187in}}%
\pgfpathlineto{\pgfqpoint{3.087800in}{1.932848in}}%
\pgfpathlineto{\pgfqpoint{3.093840in}{1.931201in}}%
\pgfpathlineto{\pgfqpoint{3.095854in}{1.936585in}}%
\pgfpathlineto{\pgfqpoint{3.097867in}{1.938865in}}%
\pgfpathlineto{\pgfqpoint{3.099881in}{1.935318in}}%
\pgfpathlineto{\pgfqpoint{3.101894in}{1.947353in}}%
\pgfpathlineto{\pgfqpoint{3.107934in}{1.946720in}}%
\pgfpathlineto{\pgfqpoint{3.109948in}{1.948430in}}%
\pgfpathlineto{\pgfqpoint{3.111961in}{1.941272in}}%
\pgfpathlineto{\pgfqpoint{3.113975in}{1.937091in}}%
\pgfpathlineto{\pgfqpoint{3.115988in}{1.934621in}}%
\pgfpathlineto{\pgfqpoint{3.124042in}{1.939752in}}%
\pgfpathlineto{\pgfqpoint{3.126055in}{1.935065in}}%
\pgfpathlineto{\pgfqpoint{3.128069in}{1.939689in}}%
\pgfpathlineto{\pgfqpoint{3.130082in}{1.933861in}}%
\pgfpathlineto{\pgfqpoint{3.136123in}{1.935698in}}%
\pgfpathlineto{\pgfqpoint{3.138136in}{1.930441in}}%
\pgfpathlineto{\pgfqpoint{3.140150in}{1.923410in}}%
\pgfpathlineto{\pgfqpoint{3.142163in}{1.922016in}}%
\pgfpathlineto{\pgfqpoint{3.144176in}{1.931644in}}%
\pgfpathlineto{\pgfqpoint{3.150217in}{1.929680in}}%
\pgfpathlineto{\pgfqpoint{3.152230in}{1.927147in}}%
\pgfpathlineto{\pgfqpoint{3.154244in}{1.921636in}}%
\pgfpathlineto{\pgfqpoint{3.156257in}{1.930504in}}%
\pgfpathlineto{\pgfqpoint{3.158271in}{1.928984in}}%
\pgfpathlineto{\pgfqpoint{3.164311in}{1.934304in}}%
\pgfpathlineto{\pgfqpoint{3.166324in}{1.940829in}}%
\pgfpathlineto{\pgfqpoint{3.170351in}{1.919799in}}%
\pgfpathlineto{\pgfqpoint{3.172365in}{1.918849in}}%
\pgfpathlineto{\pgfqpoint{3.178405in}{1.915048in}}%
\pgfpathlineto{\pgfqpoint{3.180419in}{1.916632in}}%
\pgfpathlineto{\pgfqpoint{3.182432in}{1.923283in}}%
\pgfpathlineto{\pgfqpoint{3.184445in}{1.926197in}}%
\pgfpathlineto{\pgfqpoint{3.186459in}{1.923030in}}%
\pgfpathlineto{\pgfqpoint{3.192499in}{1.932974in}}%
\pgfpathlineto{\pgfqpoint{3.194513in}{1.927780in}}%
\pgfpathlineto{\pgfqpoint{3.200553in}{1.942856in}}%
\pgfpathlineto{\pgfqpoint{3.206593in}{1.945326in}}%
\pgfpathlineto{\pgfqpoint{3.208607in}{1.951027in}}%
\pgfpathlineto{\pgfqpoint{3.210620in}{1.949760in}}%
\pgfpathlineto{\pgfqpoint{3.212634in}{1.960275in}}%
\pgfpathlineto{\pgfqpoint{3.220687in}{1.963315in}}%
\pgfpathlineto{\pgfqpoint{3.222701in}{1.961922in}}%
\pgfpathlineto{\pgfqpoint{3.224714in}{1.966736in}}%
\pgfpathlineto{\pgfqpoint{3.226728in}{1.969333in}}%
\pgfpathlineto{\pgfqpoint{3.228741in}{1.975604in}}%
\pgfpathlineto{\pgfqpoint{3.234782in}{1.973577in}}%
\pgfpathlineto{\pgfqpoint{3.236795in}{1.990172in}}%
\pgfpathlineto{\pgfqpoint{3.240822in}{1.988399in}}%
\pgfpathlineto{\pgfqpoint{3.242835in}{1.989349in}}%
\pgfpathlineto{\pgfqpoint{3.248876in}{1.989919in}}%
\pgfpathlineto{\pgfqpoint{3.250889in}{1.992643in}}%
\pgfpathlineto{\pgfqpoint{3.252903in}{1.992643in}}%
\pgfpathlineto{\pgfqpoint{3.254916in}{2.001574in}}%
\pgfpathlineto{\pgfqpoint{3.256930in}{2.004804in}}%
\pgfpathlineto{\pgfqpoint{3.262970in}{1.998597in}}%
\pgfpathlineto{\pgfqpoint{3.264983in}{1.990616in}}%
\pgfpathlineto{\pgfqpoint{3.266997in}{1.995176in}}%
\pgfpathlineto{\pgfqpoint{3.269010in}{1.996443in}}%
\pgfpathlineto{\pgfqpoint{3.271024in}{1.992959in}}%
\pgfpathlineto{\pgfqpoint{3.277064in}{1.992516in}}%
\pgfpathlineto{\pgfqpoint{3.279077in}{1.999420in}}%
\pgfpathlineto{\pgfqpoint{3.281091in}{1.992769in}}%
\pgfpathlineto{\pgfqpoint{3.283104in}{1.981304in}}%
\pgfpathlineto{\pgfqpoint{3.285118in}{1.981811in}}%
\pgfpathlineto{\pgfqpoint{3.293172in}{1.978011in}}%
\pgfpathlineto{\pgfqpoint{3.295185in}{1.974147in}}%
\pgfpathlineto{\pgfqpoint{3.297198in}{1.981114in}}%
\pgfpathlineto{\pgfqpoint{3.305252in}{1.977504in}}%
\pgfpathlineto{\pgfqpoint{3.307266in}{1.964392in}}%
\pgfpathlineto{\pgfqpoint{3.309279in}{1.964645in}}%
\pgfpathlineto{\pgfqpoint{3.311293in}{1.967306in}}%
\pgfpathlineto{\pgfqpoint{3.313306in}{1.965279in}}%
\pgfpathlineto{\pgfqpoint{3.323373in}{1.983331in}}%
\pgfpathlineto{\pgfqpoint{3.325387in}{1.985105in}}%
\pgfpathlineto{\pgfqpoint{3.327400in}{1.982635in}}%
\pgfpathlineto{\pgfqpoint{3.333441in}{1.989539in}}%
\pgfpathlineto{\pgfqpoint{3.337467in}{2.008098in}}%
\pgfpathlineto{\pgfqpoint{3.339481in}{2.008098in}}%
\pgfpathlineto{\pgfqpoint{3.341494in}{2.011202in}}%
\pgfpathlineto{\pgfqpoint{3.347535in}{2.018486in}}%
\pgfpathlineto{\pgfqpoint{3.351562in}{2.025707in}}%
\pgfpathlineto{\pgfqpoint{3.353575in}{2.031218in}}%
\pgfpathlineto{\pgfqpoint{3.355588in}{2.017789in}}%
\pgfpathlineto{\pgfqpoint{3.361629in}{2.016903in}}%
\pgfpathlineto{\pgfqpoint{3.363642in}{2.020133in}}%
\pgfpathlineto{\pgfqpoint{3.365656in}{2.016016in}}%
\pgfpathlineto{\pgfqpoint{3.367669in}{2.018233in}}%
\pgfpathlineto{\pgfqpoint{3.369683in}{2.016586in}}%
\pgfpathlineto{\pgfqpoint{3.377736in}{2.005628in}}%
\pgfpathlineto{\pgfqpoint{3.379750in}{1.992326in}}%
\pgfpathlineto{\pgfqpoint{3.381763in}{1.985168in}}%
\pgfpathlineto{\pgfqpoint{3.383777in}{1.988399in}}%
\pgfpathlineto{\pgfqpoint{3.389817in}{1.987069in}}%
\pgfpathlineto{\pgfqpoint{3.391830in}{1.980038in}}%
\pgfpathlineto{\pgfqpoint{3.393844in}{1.980481in}}%
\pgfpathlineto{\pgfqpoint{3.395857in}{1.997520in}}%
\pgfpathlineto{\pgfqpoint{3.397871in}{2.003601in}}%
\pgfpathlineto{\pgfqpoint{3.403911in}{2.002967in}}%
\pgfpathlineto{\pgfqpoint{3.405925in}{1.996887in}}%
\pgfpathlineto{\pgfqpoint{3.407938in}{2.000370in}}%
\pgfpathlineto{\pgfqpoint{3.409951in}{2.009175in}}%
\pgfpathlineto{\pgfqpoint{3.411965in}{2.007591in}}%
\pgfpathlineto{\pgfqpoint{3.418005in}{2.006958in}}%
\pgfpathlineto{\pgfqpoint{3.420019in}{1.998723in}}%
\pgfpathlineto{\pgfqpoint{3.422032in}{2.000117in}}%
\pgfpathlineto{\pgfqpoint{3.424046in}{2.003854in}}%
\pgfpathlineto{\pgfqpoint{3.426059in}{2.006388in}}%
\pgfpathlineto{\pgfqpoint{3.432099in}{1.997900in}}%
\pgfpathlineto{\pgfqpoint{3.434113in}{2.000307in}}%
\pgfpathlineto{\pgfqpoint{3.436126in}{1.997710in}}%
\pgfpathlineto{\pgfqpoint{3.438140in}{1.999357in}}%
\pgfpathlineto{\pgfqpoint{3.440153in}{2.006325in}}%
\pgfpathlineto{\pgfqpoint{3.446194in}{2.008922in}}%
\pgfpathlineto{\pgfqpoint{3.448207in}{2.007275in}}%
\pgfpathlineto{\pgfqpoint{3.450220in}{2.012785in}}%
\pgfpathlineto{\pgfqpoint{3.452234in}{2.004994in}}%
\pgfpathlineto{\pgfqpoint{3.454247in}{2.012089in}}%
\pgfpathlineto{\pgfqpoint{3.460288in}{2.009492in}}%
\pgfpathlineto{\pgfqpoint{3.462301in}{2.006008in}}%
\pgfpathlineto{\pgfqpoint{3.464315in}{2.009365in}}%
\pgfpathlineto{\pgfqpoint{3.466328in}{2.016206in}}%
\pgfpathlineto{\pgfqpoint{3.468341in}{2.015636in}}%
\pgfpathlineto{\pgfqpoint{3.474382in}{2.018676in}}%
\pgfpathlineto{\pgfqpoint{3.476395in}{2.018486in}}%
\pgfpathlineto{\pgfqpoint{3.478409in}{2.016966in}}%
\pgfpathlineto{\pgfqpoint{3.480422in}{2.022223in}}%
\pgfpathlineto{\pgfqpoint{3.482436in}{2.024630in}}%
\pgfpathlineto{\pgfqpoint{3.488476in}{2.025264in}}%
\pgfpathlineto{\pgfqpoint{3.492503in}{2.031915in}}%
\pgfpathlineto{\pgfqpoint{3.496530in}{2.028304in}}%
\pgfpathlineto{\pgfqpoint{3.502570in}{2.025264in}}%
\pgfpathlineto{\pgfqpoint{3.506597in}{2.017599in}}%
\pgfpathlineto{\pgfqpoint{3.508610in}{2.018233in}}%
\pgfpathlineto{\pgfqpoint{3.510624in}{2.029761in}}%
\pgfpathlineto{\pgfqpoint{3.516664in}{2.030078in}}%
\pgfpathlineto{\pgfqpoint{3.518678in}{2.028938in}}%
\pgfpathlineto{\pgfqpoint{3.520691in}{2.016016in}}%
\pgfpathlineto{\pgfqpoint{3.524718in}{2.008732in}}%
\pgfpathlineto{\pgfqpoint{3.530758in}{2.015319in}}%
\pgfpathlineto{\pgfqpoint{3.532772in}{2.010252in}}%
\pgfpathlineto{\pgfqpoint{3.534785in}{2.022477in}}%
\pgfpathlineto{\pgfqpoint{3.536799in}{2.020703in}}%
\pgfpathlineto{\pgfqpoint{3.538812in}{2.027164in}}%
\pgfpathlineto{\pgfqpoint{3.544852in}{2.028051in}}%
\pgfpathlineto{\pgfqpoint{3.550893in}{2.038502in}}%
\pgfpathlineto{\pgfqpoint{3.552906in}{2.039136in}}%
\pgfpathlineto{\pgfqpoint{3.558947in}{2.038692in}}%
\pgfpathlineto{\pgfqpoint{3.560960in}{2.044647in}}%
\pgfpathlineto{\pgfqpoint{3.564987in}{2.037362in}}%
\pgfpathlineto{\pgfqpoint{3.567000in}{2.039643in}}%
\pgfpathlineto{\pgfqpoint{3.573041in}{2.039072in}}%
\pgfpathlineto{\pgfqpoint{3.575054in}{2.043063in}}%
\pgfpathlineto{\pgfqpoint{3.581095in}{2.044837in}}%
\pgfpathlineto{\pgfqpoint{3.587135in}{2.039136in}}%
\pgfpathlineto{\pgfqpoint{3.589148in}{2.038186in}}%
\pgfpathlineto{\pgfqpoint{3.591162in}{2.046167in}}%
\pgfpathlineto{\pgfqpoint{3.593175in}{2.046990in}}%
\pgfpathlineto{\pgfqpoint{3.595189in}{2.046990in}}%
\pgfpathlineto{\pgfqpoint{3.603242in}{2.051298in}}%
\pgfpathlineto{\pgfqpoint{3.605256in}{2.057568in}}%
\pgfpathlineto{\pgfqpoint{3.607269in}{2.050474in}}%
\pgfpathlineto{\pgfqpoint{3.609283in}{2.033562in}}%
\pgfpathlineto{\pgfqpoint{3.615323in}{2.044647in}}%
\pgfpathlineto{\pgfqpoint{3.617337in}{2.045027in}}%
\pgfpathlineto{\pgfqpoint{3.619350in}{2.041733in}}%
\pgfpathlineto{\pgfqpoint{3.621363in}{2.049587in}}%
\pgfpathlineto{\pgfqpoint{3.623377in}{2.045913in}}%
\pgfpathlineto{\pgfqpoint{3.633444in}{2.011835in}}%
\pgfpathlineto{\pgfqpoint{3.637471in}{2.027544in}}%
\pgfpathlineto{\pgfqpoint{3.643511in}{2.032422in}}%
\pgfpathlineto{\pgfqpoint{3.645525in}{2.040023in}}%
\pgfpathlineto{\pgfqpoint{3.649552in}{2.045217in}}%
\pgfpathlineto{\pgfqpoint{3.651565in}{2.048384in}}%
\pgfpathlineto{\pgfqpoint{3.659619in}{2.047687in}}%
\pgfpathlineto{\pgfqpoint{3.661632in}{2.049334in}}%
\pgfpathlineto{\pgfqpoint{3.663646in}{2.055035in}}%
\pgfpathlineto{\pgfqpoint{3.671700in}{2.063079in}}%
\pgfpathlineto{\pgfqpoint{3.673713in}{2.059722in}}%
\pgfpathlineto{\pgfqpoint{3.675727in}{2.060926in}}%
\pgfpathlineto{\pgfqpoint{3.679753in}{2.066056in}}%
\pgfpathlineto{\pgfqpoint{3.685794in}{2.064599in}}%
\pgfpathlineto{\pgfqpoint{3.687807in}{2.069223in}}%
\pgfpathlineto{\pgfqpoint{3.689821in}{2.068210in}}%
\pgfpathlineto{\pgfqpoint{3.691834in}{2.070173in}}%
\pgfpathlineto{\pgfqpoint{3.693848in}{2.073277in}}%
\pgfpathlineto{\pgfqpoint{3.699888in}{2.070237in}}%
\pgfpathlineto{\pgfqpoint{3.701901in}{2.058329in}}%
\pgfpathlineto{\pgfqpoint{3.703915in}{2.059279in}}%
\pgfpathlineto{\pgfqpoint{3.705928in}{2.042936in}}%
\pgfpathlineto{\pgfqpoint{3.707942in}{2.041289in}}%
\pgfpathlineto{\pgfqpoint{3.713982in}{2.051424in}}%
\pgfpathlineto{\pgfqpoint{3.715995in}{2.052944in}}%
\pgfpathlineto{\pgfqpoint{3.718009in}{2.049207in}}%
\pgfpathlineto{\pgfqpoint{3.720022in}{2.047814in}}%
\pgfpathlineto{\pgfqpoint{3.722036in}{2.052311in}}%
\pgfpathlineto{\pgfqpoint{3.728076in}{2.047750in}}%
\pgfpathlineto{\pgfqpoint{3.730090in}{2.055858in}}%
\pgfpathlineto{\pgfqpoint{3.734116in}{2.048004in}}%
\pgfpathlineto{\pgfqpoint{3.736130in}{2.053641in}}%
\pgfpathlineto{\pgfqpoint{3.742170in}{2.065043in}}%
\pgfpathlineto{\pgfqpoint{3.744184in}{2.071060in}}%
\pgfpathlineto{\pgfqpoint{3.746197in}{2.082018in}}%
\pgfpathlineto{\pgfqpoint{3.748211in}{2.081448in}}%
\pgfpathlineto{\pgfqpoint{3.750224in}{2.072517in}}%
\pgfpathlineto{\pgfqpoint{3.756264in}{2.061306in}}%
\pgfpathlineto{\pgfqpoint{3.758278in}{2.059025in}}%
\pgfpathlineto{\pgfqpoint{3.760291in}{2.066120in}}%
\pgfpathlineto{\pgfqpoint{3.762305in}{2.052248in}}%
\pgfpathlineto{\pgfqpoint{3.764318in}{2.048764in}}%
\pgfpathlineto{\pgfqpoint{3.770359in}{2.054401in}}%
\pgfpathlineto{\pgfqpoint{3.772372in}{2.059405in}}%
\pgfpathlineto{\pgfqpoint{3.774385in}{2.071947in}}%
\pgfpathlineto{\pgfqpoint{3.776399in}{2.074734in}}%
\pgfpathlineto{\pgfqpoint{3.784453in}{2.073214in}}%
\pgfpathlineto{\pgfqpoint{3.786466in}{2.078218in}}%
\pgfpathlineto{\pgfqpoint{3.788480in}{2.080752in}}%
\pgfpathlineto{\pgfqpoint{3.790493in}{2.077014in}}%
\pgfpathlineto{\pgfqpoint{3.792506in}{2.067070in}}%
\pgfpathlineto{\pgfqpoint{3.798547in}{2.069857in}}%
\pgfpathlineto{\pgfqpoint{3.800560in}{2.069033in}}%
\pgfpathlineto{\pgfqpoint{3.802574in}{2.073404in}}%
\pgfpathlineto{\pgfqpoint{3.804587in}{2.064789in}}%
\pgfpathlineto{\pgfqpoint{3.806601in}{2.063206in}}%
\pgfpathlineto{\pgfqpoint{3.812641in}{2.064916in}}%
\pgfpathlineto{\pgfqpoint{3.814654in}{2.060355in}}%
\pgfpathlineto{\pgfqpoint{3.816668in}{2.065423in}}%
\pgfpathlineto{\pgfqpoint{3.818681in}{2.066056in}}%
\pgfpathlineto{\pgfqpoint{3.820695in}{2.065866in}}%
\pgfpathlineto{\pgfqpoint{3.826735in}{2.075051in}}%
\pgfpathlineto{\pgfqpoint{3.828749in}{2.075684in}}%
\pgfpathlineto{\pgfqpoint{3.830762in}{2.070997in}}%
\pgfpathlineto{\pgfqpoint{3.832775in}{2.061876in}}%
\pgfpathlineto{\pgfqpoint{3.834789in}{2.055605in}}%
\pgfpathlineto{\pgfqpoint{3.840829in}{2.058075in}}%
\pgfpathlineto{\pgfqpoint{3.842843in}{2.047244in}}%
\pgfpathlineto{\pgfqpoint{3.844856in}{2.057062in}}%
\pgfpathlineto{\pgfqpoint{3.846870in}{2.058138in}}%
\pgfpathlineto{\pgfqpoint{3.848883in}{2.060989in}}%
\pgfpathlineto{\pgfqpoint{3.858950in}{2.063269in}}%
\pgfpathlineto{\pgfqpoint{3.860964in}{2.065550in}}%
\pgfpathlineto{\pgfqpoint{3.862977in}{2.064789in}}%
\pgfpathlineto{\pgfqpoint{3.871031in}{2.073784in}}%
\pgfpathlineto{\pgfqpoint{3.873044in}{2.069983in}}%
\pgfpathlineto{\pgfqpoint{3.875058in}{2.075811in}}%
\pgfpathlineto{\pgfqpoint{3.877071in}{2.079865in}}%
\pgfpathlineto{\pgfqpoint{3.883112in}{2.086579in}}%
\pgfpathlineto{\pgfqpoint{3.887138in}{2.076508in}}%
\pgfpathlineto{\pgfqpoint{3.889152in}{2.068337in}}%
\pgfpathlineto{\pgfqpoint{3.891165in}{2.068020in}}%
\pgfpathlineto{\pgfqpoint{3.897206in}{2.068463in}}%
\pgfpathlineto{\pgfqpoint{3.901233in}{2.070490in}}%
\pgfpathlineto{\pgfqpoint{3.905260in}{2.074797in}}%
\pgfpathlineto{\pgfqpoint{3.911300in}{2.069920in}}%
\pgfpathlineto{\pgfqpoint{3.913313in}{2.061749in}}%
\pgfpathlineto{\pgfqpoint{3.915327in}{2.064219in}}%
\pgfpathlineto{\pgfqpoint{3.917340in}{2.062192in}}%
\pgfpathlineto{\pgfqpoint{3.919354in}{2.066816in}}%
\pgfpathlineto{\pgfqpoint{3.925394in}{2.060482in}}%
\pgfpathlineto{\pgfqpoint{3.927407in}{2.063206in}}%
\pgfpathlineto{\pgfqpoint{3.929421in}{2.058709in}}%
\pgfpathlineto{\pgfqpoint{3.931434in}{2.060926in}}%
\pgfpathlineto{\pgfqpoint{3.939488in}{2.058519in}}%
\pgfpathlineto{\pgfqpoint{3.941502in}{2.052691in}}%
\pgfpathlineto{\pgfqpoint{3.945528in}{2.049524in}}%
\pgfpathlineto{\pgfqpoint{3.947542in}{2.053071in}}%
\pgfpathlineto{\pgfqpoint{3.953582in}{2.057252in}}%
\pgfpathlineto{\pgfqpoint{3.955596in}{2.057062in}}%
\pgfpathlineto{\pgfqpoint{3.957609in}{2.054338in}}%
\pgfpathlineto{\pgfqpoint{3.959623in}{2.045407in}}%
\pgfpathlineto{\pgfqpoint{3.961636in}{2.049904in}}%
\pgfpathlineto{\pgfqpoint{3.967676in}{2.046547in}}%
\pgfpathlineto{\pgfqpoint{3.969690in}{2.034639in}}%
\pgfpathlineto{\pgfqpoint{3.971703in}{2.028304in}}%
\pgfpathlineto{\pgfqpoint{3.973717in}{2.023870in}}%
\pgfpathlineto{\pgfqpoint{3.975730in}{2.023490in}}%
\pgfpathlineto{\pgfqpoint{3.981771in}{2.024124in}}%
\pgfpathlineto{\pgfqpoint{3.983784in}{2.014749in}}%
\pgfpathlineto{\pgfqpoint{3.985797in}{2.009428in}}%
\pgfpathlineto{\pgfqpoint{3.987811in}{2.002524in}}%
\pgfpathlineto{\pgfqpoint{3.989824in}{2.000244in}}%
\pgfpathlineto{\pgfqpoint{3.995865in}{2.001954in}}%
\pgfpathlineto{\pgfqpoint{3.997878in}{2.001764in}}%
\pgfpathlineto{\pgfqpoint{3.999892in}{1.994860in}}%
\pgfpathlineto{\pgfqpoint{4.001905in}{1.997267in}}%
\pgfpathlineto{\pgfqpoint{4.003918in}{2.007085in}}%
\pgfpathlineto{\pgfqpoint{4.009959in}{2.005691in}}%
\pgfpathlineto{\pgfqpoint{4.011972in}{2.001130in}}%
\pgfpathlineto{\pgfqpoint{4.013986in}{2.008161in}}%
\pgfpathlineto{\pgfqpoint{4.015999in}{2.009365in}}%
\pgfpathlineto{\pgfqpoint{4.018013in}{2.008542in}}%
\pgfpathlineto{\pgfqpoint{4.024053in}{2.022477in}}%
\pgfpathlineto{\pgfqpoint{4.026066in}{2.025137in}}%
\pgfpathlineto{\pgfqpoint{4.028080in}{2.031661in}}%
\pgfpathlineto{\pgfqpoint{4.030093in}{2.033625in}}%
\pgfpathlineto{\pgfqpoint{4.032107in}{2.029825in}}%
\pgfpathlineto{\pgfqpoint{4.038147in}{2.032295in}}%
\pgfpathlineto{\pgfqpoint{4.040160in}{2.031471in}}%
\pgfpathlineto{\pgfqpoint{4.042174in}{2.028178in}}%
\pgfpathlineto{\pgfqpoint{4.044187in}{2.028241in}}%
\pgfpathlineto{\pgfqpoint{4.046201in}{2.021843in}}%
\pgfpathlineto{\pgfqpoint{4.056268in}{2.028748in}}%
\pgfpathlineto{\pgfqpoint{4.058282in}{2.028874in}}%
\pgfpathlineto{\pgfqpoint{4.060295in}{2.025961in}}%
\pgfpathlineto{\pgfqpoint{4.066335in}{2.025391in}}%
\pgfpathlineto{\pgfqpoint{4.068349in}{2.025771in}}%
\pgfpathlineto{\pgfqpoint{4.070362in}{2.024820in}}%
\pgfpathlineto{\pgfqpoint{4.072376in}{2.024884in}}%
\pgfpathlineto{\pgfqpoint{4.074389in}{2.023934in}}%
\pgfpathlineto{\pgfqpoint{4.080429in}{2.023807in}}%
\pgfpathlineto{\pgfqpoint{4.082443in}{2.025010in}}%
\pgfpathlineto{\pgfqpoint{4.084456in}{2.022160in}}%
\pgfpathlineto{\pgfqpoint{4.086470in}{2.024947in}}%
\pgfpathlineto{\pgfqpoint{4.088483in}{2.024504in}}%
\pgfpathlineto{\pgfqpoint{4.094524in}{2.013292in}}%
\pgfpathlineto{\pgfqpoint{4.096537in}{2.007401in}}%
\pgfpathlineto{\pgfqpoint{4.098550in}{2.011075in}}%
\pgfpathlineto{\pgfqpoint{4.100564in}{2.002144in}}%
\pgfpathlineto{\pgfqpoint{4.102577in}{2.006388in}}%
\pgfpathlineto{\pgfqpoint{4.108618in}{2.005438in}}%
\pgfpathlineto{\pgfqpoint{4.110631in}{2.008415in}}%
\pgfpathlineto{\pgfqpoint{4.112645in}{1.998407in}}%
\pgfpathlineto{\pgfqpoint{4.114658in}{1.995050in}}%
\pgfpathlineto{\pgfqpoint{4.116671in}{2.001701in}}%
\pgfpathlineto{\pgfqpoint{4.122712in}{2.000560in}}%
\pgfpathlineto{\pgfqpoint{4.124725in}{1.984725in}}%
\pgfpathlineto{\pgfqpoint{4.126739in}{1.991566in}}%
\pgfpathlineto{\pgfqpoint{4.128752in}{1.976364in}}%
\pgfpathlineto{\pgfqpoint{4.130766in}{1.976364in}}%
\pgfpathlineto{\pgfqpoint{4.136806in}{1.972817in}}%
\pgfpathlineto{\pgfqpoint{4.138819in}{1.977377in}}%
\pgfpathlineto{\pgfqpoint{4.140833in}{1.971993in}}%
\pgfpathlineto{\pgfqpoint{4.142846in}{1.972373in}}%
\pgfpathlineto{\pgfqpoint{4.144860in}{1.985358in}}%
\pgfpathlineto{\pgfqpoint{4.150900in}{1.985105in}}%
\pgfpathlineto{\pgfqpoint{4.152914in}{1.987892in}}%
\pgfpathlineto{\pgfqpoint{4.154927in}{1.983395in}}%
\pgfpathlineto{\pgfqpoint{4.156940in}{1.994733in}}%
\pgfpathlineto{\pgfqpoint{4.158954in}{1.998343in}}%
\pgfpathlineto{\pgfqpoint{4.164994in}{2.000434in}}%
\pgfpathlineto{\pgfqpoint{4.167008in}{2.012215in}}%
\pgfpathlineto{\pgfqpoint{4.169021in}{2.009808in}}%
\pgfpathlineto{\pgfqpoint{4.171035in}{2.012595in}}%
\pgfpathlineto{\pgfqpoint{4.173048in}{2.016333in}}%
\pgfpathlineto{\pgfqpoint{4.179088in}{2.012342in}}%
\pgfpathlineto{\pgfqpoint{4.181102in}{2.015636in}}%
\pgfpathlineto{\pgfqpoint{4.183115in}{2.021590in}}%
\pgfpathlineto{\pgfqpoint{4.187142in}{2.028114in}}%
\pgfpathlineto{\pgfqpoint{4.193182in}{2.027671in}}%
\pgfpathlineto{\pgfqpoint{4.195196in}{2.023427in}}%
\pgfpathlineto{\pgfqpoint{4.197209in}{2.026151in}}%
\pgfpathlineto{\pgfqpoint{4.199223in}{2.026151in}}%
\pgfpathlineto{\pgfqpoint{4.201236in}{2.022223in}}%
\pgfpathlineto{\pgfqpoint{4.207277in}{2.021717in}}%
\pgfpathlineto{\pgfqpoint{4.209290in}{2.029888in}}%
\pgfpathlineto{\pgfqpoint{4.211303in}{2.029064in}}%
\pgfpathlineto{\pgfqpoint{4.213317in}{2.030015in}}%
\pgfpathlineto{\pgfqpoint{4.215330in}{2.038439in}}%
\pgfpathlineto{\pgfqpoint{4.221371in}{2.029825in}}%
\pgfpathlineto{\pgfqpoint{4.223384in}{2.046420in}}%
\pgfpathlineto{\pgfqpoint{4.225398in}{2.037616in}}%
\pgfpathlineto{\pgfqpoint{4.229425in}{2.037172in}}%
\pgfpathlineto{\pgfqpoint{4.235465in}{2.034955in}}%
\pgfpathlineto{\pgfqpoint{4.237478in}{2.034892in}}%
\pgfpathlineto{\pgfqpoint{4.239492in}{2.042366in}}%
\pgfpathlineto{\pgfqpoint{4.241505in}{2.043570in}}%
\pgfpathlineto{\pgfqpoint{4.243519in}{2.044077in}}%
\pgfpathlineto{\pgfqpoint{4.249559in}{2.054085in}}%
\pgfpathlineto{\pgfqpoint{4.251572in}{2.065106in}}%
\pgfpathlineto{\pgfqpoint{4.253586in}{2.056618in}}%
\pgfpathlineto{\pgfqpoint{4.255599in}{2.059722in}}%
\pgfpathlineto{\pgfqpoint{4.257613in}{2.048954in}}%
\pgfpathlineto{\pgfqpoint{4.263653in}{2.047560in}}%
\pgfpathlineto{\pgfqpoint{4.265667in}{2.055098in}}%
\pgfpathlineto{\pgfqpoint{4.267680in}{2.058899in}}%
\pgfpathlineto{\pgfqpoint{4.269693in}{2.076318in}}%
\pgfpathlineto{\pgfqpoint{4.271707in}{2.068527in}}%
\pgfpathlineto{\pgfqpoint{4.277747in}{2.077331in}}%
\pgfpathlineto{\pgfqpoint{4.279761in}{2.077775in}}%
\pgfpathlineto{\pgfqpoint{4.281774in}{2.076064in}}%
\pgfpathlineto{\pgfqpoint{4.285801in}{2.077901in}}%
\pgfpathlineto{\pgfqpoint{4.291841in}{2.075748in}}%
\pgfpathlineto{\pgfqpoint{4.293855in}{2.072010in}}%
\pgfpathlineto{\pgfqpoint{4.295868in}{2.065169in}}%
\pgfpathlineto{\pgfqpoint{4.299895in}{2.065423in}}%
\pgfpathlineto{\pgfqpoint{4.305936in}{2.054465in}}%
\pgfpathlineto{\pgfqpoint{4.307949in}{2.045343in}}%
\pgfpathlineto{\pgfqpoint{4.309962in}{2.052248in}}%
\pgfpathlineto{\pgfqpoint{4.311976in}{2.063206in}}%
\pgfpathlineto{\pgfqpoint{4.313989in}{2.059595in}}%
\pgfpathlineto{\pgfqpoint{4.320030in}{2.062066in}}%
\pgfpathlineto{\pgfqpoint{4.322043in}{2.061432in}}%
\pgfpathlineto{\pgfqpoint{4.324057in}{2.056745in}}%
\pgfpathlineto{\pgfqpoint{4.326070in}{2.056745in}}%
\pgfpathlineto{\pgfqpoint{4.328083in}{2.071694in}}%
\pgfpathlineto{\pgfqpoint{4.336137in}{2.079548in}}%
\pgfpathlineto{\pgfqpoint{4.340164in}{2.096207in}}%
\pgfpathlineto{\pgfqpoint{4.342178in}{2.093483in}}%
\pgfpathlineto{\pgfqpoint{4.348218in}{2.086516in}}%
\pgfpathlineto{\pgfqpoint{4.350231in}{2.088859in}}%
\pgfpathlineto{\pgfqpoint{4.352245in}{2.076064in}}%
\pgfpathlineto{\pgfqpoint{4.354258in}{2.073404in}}%
\pgfpathlineto{\pgfqpoint{4.356272in}{2.063903in}}%
\pgfpathlineto{\pgfqpoint{4.362312in}{2.074164in}}%
\pgfpathlineto{\pgfqpoint{4.364325in}{2.087276in}}%
\pgfpathlineto{\pgfqpoint{4.366339in}{2.081068in}}%
\pgfpathlineto{\pgfqpoint{4.368352in}{2.094244in}}%
\pgfpathlineto{\pgfqpoint{4.370366in}{2.092533in}}%
\pgfpathlineto{\pgfqpoint{4.376406in}{2.089746in}}%
\pgfpathlineto{\pgfqpoint{4.378420in}{2.090190in}}%
\pgfpathlineto{\pgfqpoint{4.380433in}{2.089810in}}%
\pgfpathlineto{\pgfqpoint{4.382447in}{2.094940in}}%
\pgfpathlineto{\pgfqpoint{4.384460in}{2.104442in}}%
\pgfpathlineto{\pgfqpoint{4.392514in}{2.104885in}}%
\pgfpathlineto{\pgfqpoint{4.394527in}{2.108686in}}%
\pgfpathlineto{\pgfqpoint{4.396541in}{2.113943in}}%
\pgfpathlineto{\pgfqpoint{4.398554in}{2.120847in}}%
\pgfpathlineto{\pgfqpoint{4.404594in}{2.118567in}}%
\pgfpathlineto{\pgfqpoint{4.406608in}{2.119010in}}%
\pgfpathlineto{\pgfqpoint{4.408621in}{2.116603in}}%
\pgfpathlineto{\pgfqpoint{4.412648in}{2.108242in}}%
\pgfpathlineto{\pgfqpoint{4.418689in}{2.115527in}}%
\pgfpathlineto{\pgfqpoint{4.420702in}{2.107039in}}%
\pgfpathlineto{\pgfqpoint{4.422715in}{2.103238in}}%
\pgfpathlineto{\pgfqpoint{4.424729in}{2.101845in}}%
\pgfpathlineto{\pgfqpoint{4.426742in}{2.094054in}}%
\pgfpathlineto{\pgfqpoint{4.432783in}{2.107292in}}%
\pgfpathlineto{\pgfqpoint{4.434796in}{2.082399in}}%
\pgfpathlineto{\pgfqpoint{4.436810in}{2.087783in}}%
\pgfpathlineto{\pgfqpoint{4.438823in}{2.104378in}}%
\pgfpathlineto{\pgfqpoint{4.440836in}{2.090063in}}%
\pgfpathlineto{\pgfqpoint{4.446877in}{2.097727in}}%
\pgfpathlineto{\pgfqpoint{4.448890in}{2.096524in}}%
\pgfpathlineto{\pgfqpoint{4.450904in}{2.099058in}}%
\pgfpathlineto{\pgfqpoint{4.452917in}{2.093737in}}%
\pgfpathlineto{\pgfqpoint{4.454931in}{2.094244in}}%
\pgfpathlineto{\pgfqpoint{4.460971in}{2.089746in}}%
\pgfpathlineto{\pgfqpoint{4.462984in}{2.091140in}}%
\pgfpathlineto{\pgfqpoint{4.464998in}{2.077078in}}%
\pgfpathlineto{\pgfqpoint{4.467011in}{2.074671in}}%
\pgfpathlineto{\pgfqpoint{4.469025in}{2.079548in}}%
\pgfpathlineto{\pgfqpoint{4.475065in}{2.090633in}}%
\pgfpathlineto{\pgfqpoint{4.479092in}{2.073974in}}%
\pgfpathlineto{\pgfqpoint{4.481105in}{2.080878in}}%
\pgfpathlineto{\pgfqpoint{4.489159in}{2.085059in}}%
\pgfpathlineto{\pgfqpoint{4.491173in}{2.082969in}}%
\pgfpathlineto{\pgfqpoint{4.493186in}{2.084932in}}%
\pgfpathlineto{\pgfqpoint{4.495200in}{2.085186in}}%
\pgfpathlineto{\pgfqpoint{4.497213in}{2.088479in}}%
\pgfpathlineto{\pgfqpoint{4.503253in}{2.082462in}}%
\pgfpathlineto{\pgfqpoint{4.507280in}{2.085122in}}%
\pgfpathlineto{\pgfqpoint{4.509294in}{2.082905in}}%
\pgfpathlineto{\pgfqpoint{4.511307in}{2.069350in}}%
\pgfpathlineto{\pgfqpoint{4.519361in}{2.079865in}}%
\pgfpathlineto{\pgfqpoint{4.521374in}{2.079928in}}%
\pgfpathlineto{\pgfqpoint{4.523388in}{2.081638in}}%
\pgfpathlineto{\pgfqpoint{4.525401in}{2.075304in}}%
\pgfpathlineto{\pgfqpoint{4.531442in}{2.072897in}}%
\pgfpathlineto{\pgfqpoint{4.533455in}{2.074861in}}%
\pgfpathlineto{\pgfqpoint{4.535468in}{2.070870in}}%
\pgfpathlineto{\pgfqpoint{4.537482in}{2.061559in}}%
\pgfpathlineto{\pgfqpoint{4.539495in}{2.071187in}}%
\pgfpathlineto{\pgfqpoint{4.545536in}{2.077014in}}%
\pgfpathlineto{\pgfqpoint{4.547549in}{2.069097in}}%
\pgfpathlineto{\pgfqpoint{4.549563in}{2.069160in}}%
\pgfpathlineto{\pgfqpoint{4.551576in}{2.074671in}}%
\pgfpathlineto{\pgfqpoint{4.553590in}{2.088289in}}%
\pgfpathlineto{\pgfqpoint{4.561643in}{2.082209in}}%
\pgfpathlineto{\pgfqpoint{4.563657in}{2.085946in}}%
\pgfpathlineto{\pgfqpoint{4.565670in}{2.096144in}}%
\pgfpathlineto{\pgfqpoint{4.567684in}{2.092407in}}%
\pgfpathlineto{\pgfqpoint{4.573724in}{2.092470in}}%
\pgfpathlineto{\pgfqpoint{4.575737in}{2.095700in}}%
\pgfpathlineto{\pgfqpoint{4.577751in}{2.094560in}}%
\pgfpathlineto{\pgfqpoint{4.579764in}{2.096017in}}%
\pgfpathlineto{\pgfqpoint{4.581778in}{2.093103in}}%
\pgfpathlineto{\pgfqpoint{4.589832in}{2.083032in}}%
\pgfpathlineto{\pgfqpoint{4.591845in}{2.087466in}}%
\pgfpathlineto{\pgfqpoint{4.593858in}{2.087783in}}%
\pgfpathlineto{\pgfqpoint{4.595872in}{2.084806in}}%
\pgfpathlineto{\pgfqpoint{4.601912in}{2.083792in}}%
\pgfpathlineto{\pgfqpoint{4.603926in}{2.085756in}}%
\pgfpathlineto{\pgfqpoint{4.605939in}{2.092470in}}%
\pgfpathlineto{\pgfqpoint{4.607953in}{2.084616in}}%
\pgfpathlineto{\pgfqpoint{4.609966in}{2.083792in}}%
\pgfpathlineto{\pgfqpoint{4.616006in}{2.079105in}}%
\pgfpathlineto{\pgfqpoint{4.618020in}{2.079802in}}%
\pgfpathlineto{\pgfqpoint{4.620033in}{2.086706in}}%
\pgfpathlineto{\pgfqpoint{4.622047in}{2.091393in}}%
\pgfpathlineto{\pgfqpoint{4.624060in}{2.087276in}}%
\pgfpathlineto{\pgfqpoint{4.630101in}{2.070047in}}%
\pgfpathlineto{\pgfqpoint{4.632114in}{2.071947in}}%
\pgfpathlineto{\pgfqpoint{4.634127in}{2.072707in}}%
\pgfpathlineto{\pgfqpoint{4.636141in}{2.077458in}}%
\pgfpathlineto{\pgfqpoint{4.638154in}{2.071694in}}%
\pgfpathlineto{\pgfqpoint{4.644195in}{2.073911in}}%
\pgfpathlineto{\pgfqpoint{4.646208in}{2.073594in}}%
\pgfpathlineto{\pgfqpoint{4.648222in}{2.066373in}}%
\pgfpathlineto{\pgfqpoint{4.650235in}{2.062509in}}%
\pgfpathlineto{\pgfqpoint{4.652248in}{2.063776in}}%
\pgfpathlineto{\pgfqpoint{4.660302in}{2.048891in}}%
\pgfpathlineto{\pgfqpoint{4.662316in}{2.048257in}}%
\pgfpathlineto{\pgfqpoint{4.664329in}{2.041036in}}%
\pgfpathlineto{\pgfqpoint{4.672383in}{2.039833in}}%
\pgfpathlineto{\pgfqpoint{4.674396in}{2.044077in}}%
\pgfpathlineto{\pgfqpoint{4.676410in}{2.035905in}}%
\pgfpathlineto{\pgfqpoint{4.678423in}{2.037299in}}%
\pgfpathlineto{\pgfqpoint{4.680437in}{2.044583in}}%
\pgfpathlineto{\pgfqpoint{4.686477in}{2.052628in}}%
\pgfpathlineto{\pgfqpoint{4.688490in}{2.052248in}}%
\pgfpathlineto{\pgfqpoint{4.690504in}{2.050727in}}%
\pgfpathlineto{\pgfqpoint{4.692517in}{2.050791in}}%
\pgfpathlineto{\pgfqpoint{4.694531in}{2.047814in}}%
\pgfpathlineto{\pgfqpoint{4.700571in}{2.046293in}}%
\pgfpathlineto{\pgfqpoint{4.702585in}{2.001574in}}%
\pgfpathlineto{\pgfqpoint{4.704598in}{1.994923in}}%
\pgfpathlineto{\pgfqpoint{4.706612in}{1.992579in}}%
\pgfpathlineto{\pgfqpoint{4.708625in}{1.982064in}}%
\pgfpathlineto{\pgfqpoint{4.714665in}{1.979531in}}%
\pgfpathlineto{\pgfqpoint{4.716679in}{1.980101in}}%
\pgfpathlineto{\pgfqpoint{4.718692in}{1.982318in}}%
\pgfpathlineto{\pgfqpoint{4.720706in}{1.990236in}}%
\pgfpathlineto{\pgfqpoint{4.722719in}{1.987765in}}%
\pgfpathlineto{\pgfqpoint{4.732786in}{1.979531in}}%
\pgfpathlineto{\pgfqpoint{4.734800in}{1.980164in}}%
\pgfpathlineto{\pgfqpoint{4.736813in}{1.976174in}}%
\pgfpathlineto{\pgfqpoint{4.742854in}{1.983395in}}%
\pgfpathlineto{\pgfqpoint{4.744867in}{1.978011in}}%
\pgfpathlineto{\pgfqpoint{4.746880in}{1.982191in}}%
\pgfpathlineto{\pgfqpoint{4.748894in}{1.980418in}}%
\pgfpathlineto{\pgfqpoint{4.750907in}{1.982254in}}%
\pgfpathlineto{\pgfqpoint{4.756948in}{1.986055in}}%
\pgfpathlineto{\pgfqpoint{4.758961in}{1.986372in}}%
\pgfpathlineto{\pgfqpoint{4.760975in}{1.979404in}}%
\pgfpathlineto{\pgfqpoint{4.765001in}{1.949317in}}%
\pgfpathlineto{\pgfqpoint{4.771042in}{1.936901in}}%
\pgfpathlineto{\pgfqpoint{4.773055in}{1.924930in}}%
\pgfpathlineto{\pgfqpoint{4.775069in}{1.940892in}}%
\pgfpathlineto{\pgfqpoint{4.777082in}{1.950773in}}%
\pgfpathlineto{\pgfqpoint{4.779096in}{1.950647in}}%
\pgfpathlineto{\pgfqpoint{4.785136in}{1.941209in}}%
\pgfpathlineto{\pgfqpoint{4.787149in}{1.930251in}}%
\pgfpathlineto{\pgfqpoint{4.789163in}{1.939055in}}%
\pgfpathlineto{\pgfqpoint{4.791176in}{1.942919in}}%
\pgfpathlineto{\pgfqpoint{4.793190in}{1.935825in}}%
\pgfpathlineto{\pgfqpoint{4.801244in}{1.948176in}}%
\pgfpathlineto{\pgfqpoint{4.803257in}{1.942602in}}%
\pgfpathlineto{\pgfqpoint{4.805270in}{1.939942in}}%
\pgfpathlineto{\pgfqpoint{4.807284in}{1.945326in}}%
\pgfpathlineto{\pgfqpoint{4.813324in}{1.942476in}}%
\pgfpathlineto{\pgfqpoint{4.815338in}{1.947353in}}%
\pgfpathlineto{\pgfqpoint{4.817351in}{1.954701in}}%
\pgfpathlineto{\pgfqpoint{4.819365in}{1.951027in}}%
\pgfpathlineto{\pgfqpoint{4.821378in}{1.938042in}}%
\pgfpathlineto{\pgfqpoint{4.827418in}{1.940829in}}%
\pgfpathlineto{\pgfqpoint{4.829432in}{1.921446in}}%
\pgfpathlineto{\pgfqpoint{4.831445in}{1.914288in}}%
\pgfpathlineto{\pgfqpoint{4.833459in}{1.913465in}}%
\pgfpathlineto{\pgfqpoint{4.835472in}{1.916252in}}%
\pgfpathlineto{\pgfqpoint{4.841512in}{1.913528in}}%
\pgfpathlineto{\pgfqpoint{4.845539in}{1.926007in}}%
\pgfpathlineto{\pgfqpoint{4.847553in}{1.922396in}}%
\pgfpathlineto{\pgfqpoint{4.849566in}{1.930504in}}%
\pgfpathlineto{\pgfqpoint{4.855607in}{1.944946in}}%
\pgfpathlineto{\pgfqpoint{4.857620in}{1.946656in}}%
\pgfpathlineto{\pgfqpoint{4.863660in}{1.962935in}}%
\pgfpathlineto{\pgfqpoint{4.869701in}{1.963315in}}%
\pgfpathlineto{\pgfqpoint{4.871714in}{1.956664in}}%
\pgfpathlineto{\pgfqpoint{4.873728in}{1.944439in}}%
\pgfpathlineto{\pgfqpoint{4.875741in}{1.950267in}}%
\pgfpathlineto{\pgfqpoint{4.877755in}{1.949253in}}%
\pgfpathlineto{\pgfqpoint{4.883795in}{1.943742in}}%
\pgfpathlineto{\pgfqpoint{4.885808in}{1.964392in}}%
\pgfpathlineto{\pgfqpoint{4.889835in}{1.988525in}}%
\pgfpathlineto{\pgfqpoint{4.891849in}{1.993339in}}%
\pgfpathlineto{\pgfqpoint{4.897889in}{1.991122in}}%
\pgfpathlineto{\pgfqpoint{4.899902in}{1.983268in}}%
\pgfpathlineto{\pgfqpoint{4.901916in}{1.985802in}}%
\pgfpathlineto{\pgfqpoint{4.903929in}{1.984281in}}%
\pgfpathlineto{\pgfqpoint{4.905943in}{1.980544in}}%
\pgfpathlineto{\pgfqpoint{4.911983in}{1.986245in}}%
\pgfpathlineto{\pgfqpoint{4.913997in}{1.989792in}}%
\pgfpathlineto{\pgfqpoint{4.916010in}{1.991692in}}%
\pgfpathlineto{\pgfqpoint{4.918023in}{1.994416in}}%
\pgfpathlineto{\pgfqpoint{4.920037in}{1.994416in}}%
\pgfpathlineto{\pgfqpoint{4.930104in}{1.983395in}}%
\pgfpathlineto{\pgfqpoint{4.932118in}{1.988969in}}%
\pgfpathlineto{\pgfqpoint{4.934131in}{1.973007in}}%
\pgfpathlineto{\pgfqpoint{4.940171in}{1.980861in}}%
\pgfpathlineto{\pgfqpoint{4.942185in}{1.979214in}}%
\pgfpathlineto{\pgfqpoint{4.944198in}{1.980164in}}%
\pgfpathlineto{\pgfqpoint{4.946212in}{1.983585in}}%
\pgfpathlineto{\pgfqpoint{4.948225in}{1.983141in}}%
\pgfpathlineto{\pgfqpoint{4.954266in}{1.982381in}}%
\pgfpathlineto{\pgfqpoint{4.956279in}{1.978137in}}%
\pgfpathlineto{\pgfqpoint{4.958292in}{1.977631in}}%
\pgfpathlineto{\pgfqpoint{4.968360in}{1.970473in}}%
\pgfpathlineto{\pgfqpoint{4.970373in}{1.974273in}}%
\pgfpathlineto{\pgfqpoint{4.972387in}{1.965849in}}%
\pgfpathlineto{\pgfqpoint{4.974400in}{1.962618in}}%
\pgfpathlineto{\pgfqpoint{4.976413in}{1.968636in}}%
\pgfpathlineto{\pgfqpoint{4.982454in}{1.969649in}}%
\pgfpathlineto{\pgfqpoint{4.984467in}{1.959895in}}%
\pgfpathlineto{\pgfqpoint{4.986481in}{1.959451in}}%
\pgfpathlineto{\pgfqpoint{4.988494in}{1.957804in}}%
\pgfpathlineto{\pgfqpoint{4.990508in}{1.954447in}}%
\pgfpathlineto{\pgfqpoint{4.996548in}{1.952864in}}%
\pgfpathlineto{\pgfqpoint{4.998561in}{1.954257in}}%
\pgfpathlineto{\pgfqpoint{5.000575in}{1.965152in}}%
\pgfpathlineto{\pgfqpoint{5.004602in}{1.948873in}}%
\pgfpathlineto{\pgfqpoint{5.010642in}{1.956284in}}%
\pgfpathlineto{\pgfqpoint{5.014669in}{1.971930in}}%
\pgfpathlineto{\pgfqpoint{5.016682in}{1.971930in}}%
\pgfpathlineto{\pgfqpoint{5.024736in}{1.970726in}}%
\pgfpathlineto{\pgfqpoint{5.026750in}{1.977631in}}%
\pgfpathlineto{\pgfqpoint{5.028763in}{1.975414in}}%
\pgfpathlineto{\pgfqpoint{5.030777in}{1.970600in}}%
\pgfpathlineto{\pgfqpoint{5.038830in}{1.967686in}}%
\pgfpathlineto{\pgfqpoint{5.040844in}{1.968573in}}%
\pgfpathlineto{\pgfqpoint{5.042857in}{1.953434in}}%
\pgfpathlineto{\pgfqpoint{5.046884in}{1.937535in}}%
\pgfpathlineto{\pgfqpoint{5.054938in}{1.938168in}}%
\pgfpathlineto{\pgfqpoint{5.056951in}{1.928414in}}%
\pgfpathlineto{\pgfqpoint{5.058965in}{1.929490in}}%
\pgfpathlineto{\pgfqpoint{5.060978in}{1.909791in}}%
\pgfpathlineto{\pgfqpoint{5.069032in}{1.907511in}}%
\pgfpathlineto{\pgfqpoint{5.071045in}{1.905104in}}%
\pgfpathlineto{\pgfqpoint{5.075072in}{1.913972in}}%
\pgfpathlineto{\pgfqpoint{5.081113in}{1.905484in}}%
\pgfpathlineto{\pgfqpoint{5.083126in}{1.909854in}}%
\pgfpathlineto{\pgfqpoint{5.085140in}{1.910804in}}%
\pgfpathlineto{\pgfqpoint{5.087153in}{1.914542in}}%
\pgfpathlineto{\pgfqpoint{5.089166in}{1.921763in}}%
\pgfpathlineto{\pgfqpoint{5.095207in}{1.921003in}}%
\pgfpathlineto{\pgfqpoint{5.097220in}{1.908524in}}%
\pgfpathlineto{\pgfqpoint{5.099234in}{1.911691in}}%
\pgfpathlineto{\pgfqpoint{5.101247in}{1.924296in}}%
\pgfpathlineto{\pgfqpoint{5.103261in}{1.922649in}}%
\pgfpathlineto{\pgfqpoint{5.109301in}{1.916442in}}%
\pgfpathlineto{\pgfqpoint{5.111314in}{1.919166in}}%
\pgfpathlineto{\pgfqpoint{5.113328in}{1.917519in}}%
\pgfpathlineto{\pgfqpoint{5.115341in}{1.904090in}}%
\pgfpathlineto{\pgfqpoint{5.117355in}{1.911628in}}%
\pgfpathlineto{\pgfqpoint{5.125409in}{1.914732in}}%
\pgfpathlineto{\pgfqpoint{5.127422in}{1.928097in}}%
\pgfpathlineto{\pgfqpoint{5.129435in}{1.929490in}}%
\pgfpathlineto{\pgfqpoint{5.131449in}{1.928730in}}%
\pgfpathlineto{\pgfqpoint{5.137489in}{1.953054in}}%
\pgfpathlineto{\pgfqpoint{5.139503in}{1.948556in}}%
\pgfpathlineto{\pgfqpoint{5.141516in}{1.960338in}}%
\pgfpathlineto{\pgfqpoint{5.143530in}{1.986498in}}%
\pgfpathlineto{\pgfqpoint{5.151583in}{1.978011in}}%
\pgfpathlineto{\pgfqpoint{5.153597in}{1.968763in}}%
\pgfpathlineto{\pgfqpoint{5.157624in}{1.975033in}}%
\pgfpathlineto{\pgfqpoint{5.159637in}{1.980228in}}%
\pgfpathlineto{\pgfqpoint{5.167691in}{1.979657in}}%
\pgfpathlineto{\pgfqpoint{5.171718in}{1.975414in}}%
\pgfpathlineto{\pgfqpoint{5.173731in}{1.978771in}}%
\pgfpathlineto{\pgfqpoint{5.179772in}{1.979214in}}%
\pgfpathlineto{\pgfqpoint{5.181785in}{1.976047in}}%
\pgfpathlineto{\pgfqpoint{5.185812in}{1.991059in}}%
\pgfpathlineto{\pgfqpoint{5.187825in}{1.992263in}}%
\pgfpathlineto{\pgfqpoint{5.193866in}{1.992959in}}%
\pgfpathlineto{\pgfqpoint{5.195879in}{1.989982in}}%
\pgfpathlineto{\pgfqpoint{5.197893in}{1.992643in}}%
\pgfpathlineto{\pgfqpoint{5.207960in}{1.991439in}}%
\pgfpathlineto{\pgfqpoint{5.209973in}{1.998533in}}%
\pgfpathlineto{\pgfqpoint{5.211987in}{1.999294in}}%
\pgfpathlineto{\pgfqpoint{5.216014in}{1.997647in}}%
\pgfpathlineto{\pgfqpoint{5.222054in}{1.999420in}}%
\pgfpathlineto{\pgfqpoint{5.224067in}{1.997013in}}%
\pgfpathlineto{\pgfqpoint{5.230108in}{2.005628in}}%
\pgfpathlineto{\pgfqpoint{5.236148in}{2.010442in}}%
\pgfpathlineto{\pgfqpoint{5.238162in}{2.015953in}}%
\pgfpathlineto{\pgfqpoint{5.240175in}{2.024947in}}%
\pgfpathlineto{\pgfqpoint{5.242188in}{2.025327in}}%
\pgfpathlineto{\pgfqpoint{5.244202in}{2.024694in}}%
\pgfpathlineto{\pgfqpoint{5.250242in}{2.027988in}}%
\pgfpathlineto{\pgfqpoint{5.252256in}{2.027227in}}%
\pgfpathlineto{\pgfqpoint{5.254269in}{2.030331in}}%
\pgfpathlineto{\pgfqpoint{5.256283in}{2.029761in}}%
\pgfpathlineto{\pgfqpoint{5.258296in}{2.031345in}}%
\pgfpathlineto{\pgfqpoint{5.264336in}{2.028368in}}%
\pgfpathlineto{\pgfqpoint{5.266350in}{2.026024in}}%
\pgfpathlineto{\pgfqpoint{5.268363in}{2.032422in}}%
\pgfpathlineto{\pgfqpoint{5.270377in}{2.022667in}}%
\pgfpathlineto{\pgfqpoint{5.272390in}{2.023490in}}%
\pgfpathlineto{\pgfqpoint{5.278431in}{2.023490in}}%
\pgfpathlineto{\pgfqpoint{5.280444in}{2.010378in}}%
\pgfpathlineto{\pgfqpoint{5.282457in}{2.001764in}}%
\pgfpathlineto{\pgfqpoint{5.284471in}{1.999547in}}%
\pgfpathlineto{\pgfqpoint{5.286484in}{2.004424in}}%
\pgfpathlineto{\pgfqpoint{5.292525in}{1.998407in}}%
\pgfpathlineto{\pgfqpoint{5.294538in}{2.010505in}}%
\pgfpathlineto{\pgfqpoint{5.296552in}{2.007275in}}%
\pgfpathlineto{\pgfqpoint{5.298565in}{2.006451in}}%
\pgfpathlineto{\pgfqpoint{5.300578in}{1.999420in}}%
\pgfpathlineto{\pgfqpoint{5.306619in}{2.008542in}}%
\pgfpathlineto{\pgfqpoint{5.308632in}{1.998217in}}%
\pgfpathlineto{\pgfqpoint{5.310646in}{1.997583in}}%
\pgfpathlineto{\pgfqpoint{5.312659in}{1.992959in}}%
\pgfpathlineto{\pgfqpoint{5.314673in}{1.996443in}}%
\pgfpathlineto{\pgfqpoint{5.320713in}{1.995240in}}%
\pgfpathlineto{\pgfqpoint{5.322726in}{2.001194in}}%
\pgfpathlineto{\pgfqpoint{5.324740in}{2.003664in}}%
\pgfpathlineto{\pgfqpoint{5.326753in}{2.004171in}}%
\pgfpathlineto{\pgfqpoint{5.328767in}{2.006261in}}%
\pgfpathlineto{\pgfqpoint{5.336820in}{2.005184in}}%
\pgfpathlineto{\pgfqpoint{5.338834in}{2.003791in}}%
\pgfpathlineto{\pgfqpoint{5.340847in}{2.006451in}}%
\pgfpathlineto{\pgfqpoint{5.342861in}{2.004298in}}%
\pgfpathlineto{\pgfqpoint{5.348901in}{2.009808in}}%
\pgfpathlineto{\pgfqpoint{5.350915in}{2.010315in}}%
\pgfpathlineto{\pgfqpoint{5.352928in}{2.013989in}}%
\pgfpathlineto{\pgfqpoint{5.354942in}{2.015446in}}%
\pgfpathlineto{\pgfqpoint{5.356955in}{2.013546in}}%
\pgfpathlineto{\pgfqpoint{5.362995in}{2.008922in}}%
\pgfpathlineto{\pgfqpoint{5.365009in}{2.008795in}}%
\pgfpathlineto{\pgfqpoint{5.367022in}{2.004931in}}%
\pgfpathlineto{\pgfqpoint{5.369036in}{2.008351in}}%
\pgfpathlineto{\pgfqpoint{5.371049in}{2.008858in}}%
\pgfpathlineto{\pgfqpoint{5.377089in}{2.011772in}}%
\pgfpathlineto{\pgfqpoint{5.381116in}{2.009745in}}%
\pgfpathlineto{\pgfqpoint{5.383130in}{2.015509in}}%
\pgfpathlineto{\pgfqpoint{5.385143in}{1.995176in}}%
\pgfpathlineto{\pgfqpoint{5.391184in}{1.985232in}}%
\pgfpathlineto{\pgfqpoint{5.397224in}{2.016839in}}%
\pgfpathlineto{\pgfqpoint{5.399237in}{2.017916in}}%
\pgfpathlineto{\pgfqpoint{5.407291in}{2.005438in}}%
\pgfpathlineto{\pgfqpoint{5.411318in}{2.013419in}}%
\pgfpathlineto{\pgfqpoint{5.413331in}{2.023364in}}%
\pgfpathlineto{\pgfqpoint{5.419372in}{2.025137in}}%
\pgfpathlineto{\pgfqpoint{5.423399in}{2.031851in}}%
\pgfpathlineto{\pgfqpoint{5.425412in}{2.032105in}}%
\pgfpathlineto{\pgfqpoint{5.427426in}{2.034258in}}%
\pgfpathlineto{\pgfqpoint{5.433466in}{2.034195in}}%
\pgfpathlineto{\pgfqpoint{5.435479in}{2.034955in}}%
\pgfpathlineto{\pgfqpoint{5.437493in}{2.037426in}}%
\pgfpathlineto{\pgfqpoint{5.439506in}{2.036349in}}%
\pgfpathlineto{\pgfqpoint{5.441520in}{2.032041in}}%
\pgfpathlineto{\pgfqpoint{5.447560in}{2.029254in}}%
\pgfpathlineto{\pgfqpoint{5.449574in}{2.048384in}}%
\pgfpathlineto{\pgfqpoint{5.453600in}{2.046547in}}%
\pgfpathlineto{\pgfqpoint{5.455614in}{2.046990in}}%
\pgfpathlineto{\pgfqpoint{5.461654in}{2.042746in}}%
\pgfpathlineto{\pgfqpoint{5.463668in}{2.038819in}}%
\pgfpathlineto{\pgfqpoint{5.467695in}{2.039326in}}%
\pgfpathlineto{\pgfqpoint{5.469708in}{2.047497in}}%
\pgfpathlineto{\pgfqpoint{5.475748in}{2.047814in}}%
\pgfpathlineto{\pgfqpoint{5.477762in}{2.051044in}}%
\pgfpathlineto{\pgfqpoint{5.479775in}{2.049714in}}%
\pgfpathlineto{\pgfqpoint{5.481789in}{2.055985in}}%
\pgfpathlineto{\pgfqpoint{5.483802in}{2.054148in}}%
\pgfpathlineto{\pgfqpoint{5.489842in}{2.059025in}}%
\pgfpathlineto{\pgfqpoint{5.491856in}{2.055921in}}%
\pgfpathlineto{\pgfqpoint{5.495883in}{2.060799in}}%
\pgfpathlineto{\pgfqpoint{5.503937in}{2.055921in}}%
\pgfpathlineto{\pgfqpoint{5.505950in}{2.052944in}}%
\pgfpathlineto{\pgfqpoint{5.507964in}{2.052691in}}%
\pgfpathlineto{\pgfqpoint{5.509977in}{2.051234in}}%
\pgfpathlineto{\pgfqpoint{5.511990in}{2.048827in}}%
\pgfpathlineto{\pgfqpoint{5.518031in}{2.052754in}}%
\pgfpathlineto{\pgfqpoint{5.520044in}{2.049081in}}%
\pgfpathlineto{\pgfqpoint{5.522058in}{2.043570in}}%
\pgfpathlineto{\pgfqpoint{5.526085in}{2.046547in}}%
\pgfpathlineto{\pgfqpoint{5.536152in}{2.040149in}}%
\pgfpathlineto{\pgfqpoint{5.538165in}{2.039516in}}%
\pgfpathlineto{\pgfqpoint{5.540179in}{2.021273in}}%
\pgfpathlineto{\pgfqpoint{5.546219in}{2.029318in}}%
\pgfpathlineto{\pgfqpoint{5.548232in}{2.019120in}}%
\pgfpathlineto{\pgfqpoint{5.550246in}{2.015192in}}%
\pgfpathlineto{\pgfqpoint{5.552259in}{2.021463in}}%
\pgfpathlineto{\pgfqpoint{5.554273in}{2.005944in}}%
\pgfpathlineto{\pgfqpoint{5.560313in}{2.007971in}}%
\pgfpathlineto{\pgfqpoint{5.562327in}{2.006768in}}%
\pgfpathlineto{\pgfqpoint{5.564340in}{2.017029in}}%
\pgfpathlineto{\pgfqpoint{5.566353in}{2.023174in}}%
\pgfpathlineto{\pgfqpoint{5.568367in}{2.020577in}}%
\pgfpathlineto{\pgfqpoint{5.574407in}{2.018613in}}%
\pgfpathlineto{\pgfqpoint{5.576421in}{2.019310in}}%
\pgfpathlineto{\pgfqpoint{5.578434in}{2.019310in}}%
\pgfpathlineto{\pgfqpoint{5.580448in}{2.011645in}}%
\pgfpathlineto{\pgfqpoint{5.582461in}{2.014876in}}%
\pgfpathlineto{\pgfqpoint{5.588501in}{2.019816in}}%
\pgfpathlineto{\pgfqpoint{5.590515in}{2.013862in}}%
\pgfpathlineto{\pgfqpoint{5.592528in}{2.018740in}}%
\pgfpathlineto{\pgfqpoint{5.594542in}{2.017726in}}%
\pgfpathlineto{\pgfqpoint{5.596555in}{2.008795in}}%
\pgfpathlineto{\pgfqpoint{5.602596in}{2.005248in}}%
\pgfpathlineto{\pgfqpoint{5.604609in}{1.998027in}}%
\pgfpathlineto{\pgfqpoint{5.606622in}{1.998977in}}%
\pgfpathlineto{\pgfqpoint{5.608636in}{2.004488in}}%
\pgfpathlineto{\pgfqpoint{5.610649in}{2.006325in}}%
\pgfpathlineto{\pgfqpoint{5.616690in}{2.003601in}}%
\pgfpathlineto{\pgfqpoint{5.618703in}{2.005311in}}%
\pgfpathlineto{\pgfqpoint{5.620717in}{2.003918in}}%
\pgfpathlineto{\pgfqpoint{5.624743in}{1.997457in}}%
\pgfpathlineto{\pgfqpoint{5.630784in}{2.002524in}}%
\pgfpathlineto{\pgfqpoint{5.632797in}{2.013419in}}%
\pgfpathlineto{\pgfqpoint{5.634811in}{2.011329in}}%
\pgfpathlineto{\pgfqpoint{5.636824in}{2.005754in}}%
\pgfpathlineto{\pgfqpoint{5.638838in}{2.016269in}}%
\pgfpathlineto{\pgfqpoint{5.644878in}{2.018423in}}%
\pgfpathlineto{\pgfqpoint{5.646891in}{2.017283in}}%
\pgfpathlineto{\pgfqpoint{5.650918in}{2.011582in}}%
\pgfpathlineto{\pgfqpoint{5.652932in}{2.013229in}}%
\pgfpathlineto{\pgfqpoint{5.658972in}{2.023490in}}%
\pgfpathlineto{\pgfqpoint{5.660985in}{2.025391in}}%
\pgfpathlineto{\pgfqpoint{5.662999in}{2.033942in}}%
\pgfpathlineto{\pgfqpoint{5.665012in}{2.055351in}}%
\pgfpathlineto{\pgfqpoint{5.667026in}{2.058012in}}%
\pgfpathlineto{\pgfqpoint{5.673066in}{2.052058in}}%
\pgfpathlineto{\pgfqpoint{5.675080in}{2.051044in}}%
\pgfpathlineto{\pgfqpoint{5.677093in}{2.050664in}}%
\pgfpathlineto{\pgfqpoint{5.681120in}{2.047877in}}%
\pgfpathlineto{\pgfqpoint{5.689174in}{2.050537in}}%
\pgfpathlineto{\pgfqpoint{5.691187in}{2.057568in}}%
\pgfpathlineto{\pgfqpoint{5.695214in}{2.061559in}}%
\pgfpathlineto{\pgfqpoint{5.701254in}{2.059595in}}%
\pgfpathlineto{\pgfqpoint{5.703268in}{2.061876in}}%
\pgfpathlineto{\pgfqpoint{5.705281in}{2.055225in}}%
\pgfpathlineto{\pgfqpoint{5.707295in}{2.053768in}}%
\pgfpathlineto{\pgfqpoint{5.709308in}{2.058202in}}%
\pgfpathlineto{\pgfqpoint{5.715349in}{2.053388in}}%
\pgfpathlineto{\pgfqpoint{5.717362in}{2.053451in}}%
\pgfpathlineto{\pgfqpoint{5.719375in}{2.066943in}}%
\pgfpathlineto{\pgfqpoint{5.721389in}{2.059469in}}%
\pgfpathlineto{\pgfqpoint{5.723402in}{2.067576in}}%
\pgfpathlineto{\pgfqpoint{5.729443in}{2.071187in}}%
\pgfpathlineto{\pgfqpoint{5.731456in}{2.070680in}}%
\pgfpathlineto{\pgfqpoint{5.735483in}{2.057632in}}%
\pgfpathlineto{\pgfqpoint{5.737496in}{2.059975in}}%
\pgfpathlineto{\pgfqpoint{5.743537in}{2.073721in}}%
\pgfpathlineto{\pgfqpoint{5.747564in}{2.071757in}}%
\pgfpathlineto{\pgfqpoint{5.749577in}{2.071567in}}%
\pgfpathlineto{\pgfqpoint{5.751591in}{2.072707in}}%
\pgfpathlineto{\pgfqpoint{5.759644in}{2.074734in}}%
\pgfpathlineto{\pgfqpoint{5.761658in}{2.069983in}}%
\pgfpathlineto{\pgfqpoint{5.763671in}{2.072137in}}%
\pgfpathlineto{\pgfqpoint{5.765685in}{2.066563in}}%
\pgfpathlineto{\pgfqpoint{5.773739in}{2.073784in}}%
\pgfpathlineto{\pgfqpoint{5.775752in}{2.074227in}}%
\pgfpathlineto{\pgfqpoint{5.777765in}{2.076951in}}%
\pgfpathlineto{\pgfqpoint{5.779779in}{2.084109in}}%
\pgfpathlineto{\pgfqpoint{5.787833in}{2.076381in}}%
\pgfpathlineto{\pgfqpoint{5.791860in}{2.073721in}}%
\pgfpathlineto{\pgfqpoint{5.793873in}{2.070173in}}%
\pgfpathlineto{\pgfqpoint{5.801927in}{2.068653in}}%
\pgfpathlineto{\pgfqpoint{5.805954in}{2.073214in}}%
\pgfpathlineto{\pgfqpoint{5.807967in}{2.073594in}}%
\pgfpathlineto{\pgfqpoint{5.814007in}{2.070870in}}%
\pgfpathlineto{\pgfqpoint{5.816021in}{2.078471in}}%
\pgfpathlineto{\pgfqpoint{5.822061in}{2.067070in}}%
\pgfpathlineto{\pgfqpoint{5.828102in}{2.063903in}}%
\pgfpathlineto{\pgfqpoint{5.830115in}{2.066880in}}%
\pgfpathlineto{\pgfqpoint{5.832129in}{2.057948in}}%
\pgfpathlineto{\pgfqpoint{5.834142in}{2.059025in}}%
\pgfpathlineto{\pgfqpoint{5.836155in}{2.066753in}}%
\pgfpathlineto{\pgfqpoint{5.842196in}{2.072897in}}%
\pgfpathlineto{\pgfqpoint{5.844209in}{2.076318in}}%
\pgfpathlineto{\pgfqpoint{5.846223in}{2.071250in}}%
\pgfpathlineto{\pgfqpoint{5.848236in}{2.069413in}}%
\pgfpathlineto{\pgfqpoint{5.850250in}{2.075114in}}%
\pgfpathlineto{\pgfqpoint{5.856290in}{2.080942in}}%
\pgfpathlineto{\pgfqpoint{5.858303in}{2.078218in}}%
\pgfpathlineto{\pgfqpoint{5.860317in}{2.084235in}}%
\pgfpathlineto{\pgfqpoint{5.864344in}{2.085692in}}%
\pgfpathlineto{\pgfqpoint{5.874411in}{2.088796in}}%
\pgfpathlineto{\pgfqpoint{5.876424in}{2.085312in}}%
\pgfpathlineto{\pgfqpoint{5.878438in}{2.087593in}}%
\pgfpathlineto{\pgfqpoint{5.884478in}{2.089683in}}%
\pgfpathlineto{\pgfqpoint{5.886492in}{2.088099in}}%
\pgfpathlineto{\pgfqpoint{5.888505in}{2.094940in}}%
\pgfpathlineto{\pgfqpoint{5.890518in}{2.088923in}}%
\pgfpathlineto{\pgfqpoint{5.892532in}{2.086833in}}%
\pgfpathlineto{\pgfqpoint{5.898572in}{2.082779in}}%
\pgfpathlineto{\pgfqpoint{5.900586in}{2.086452in}}%
\pgfpathlineto{\pgfqpoint{5.902599in}{2.083285in}}%
\pgfpathlineto{\pgfqpoint{5.906626in}{2.085629in}}%
\pgfpathlineto{\pgfqpoint{5.912666in}{2.086642in}}%
\pgfpathlineto{\pgfqpoint{5.914680in}{2.083792in}}%
\pgfpathlineto{\pgfqpoint{5.916693in}{2.091330in}}%
\pgfpathlineto{\pgfqpoint{5.918707in}{2.086833in}}%
\pgfpathlineto{\pgfqpoint{5.920720in}{2.093547in}}%
\pgfpathlineto{\pgfqpoint{5.926761in}{2.093863in}}%
\pgfpathlineto{\pgfqpoint{5.928774in}{2.085756in}}%
\pgfpathlineto{\pgfqpoint{5.930787in}{2.084362in}}%
\pgfpathlineto{\pgfqpoint{5.934814in}{2.083602in}}%
\pgfpathlineto{\pgfqpoint{5.940855in}{2.083792in}}%
\pgfpathlineto{\pgfqpoint{5.942868in}{2.089240in}}%
\pgfpathlineto{\pgfqpoint{5.944882in}{2.085186in}}%
\pgfpathlineto{\pgfqpoint{5.946895in}{2.087529in}}%
\pgfpathlineto{\pgfqpoint{5.948908in}{2.086072in}}%
\pgfpathlineto{\pgfqpoint{5.954949in}{2.084362in}}%
\pgfpathlineto{\pgfqpoint{5.956962in}{2.090760in}}%
\pgfpathlineto{\pgfqpoint{5.958976in}{2.087783in}}%
\pgfpathlineto{\pgfqpoint{5.960989in}{2.087276in}}%
\pgfpathlineto{\pgfqpoint{5.963003in}{2.090760in}}%
\pgfpathlineto{\pgfqpoint{5.969043in}{2.090126in}}%
\pgfpathlineto{\pgfqpoint{5.971056in}{2.091647in}}%
\pgfpathlineto{\pgfqpoint{5.973070in}{2.086262in}}%
\pgfpathlineto{\pgfqpoint{5.975083in}{2.085312in}}%
\pgfpathlineto{\pgfqpoint{5.985151in}{2.094180in}}%
\pgfpathlineto{\pgfqpoint{5.987164in}{2.090316in}}%
\pgfpathlineto{\pgfqpoint{5.991191in}{2.102795in}}%
\pgfpathlineto{\pgfqpoint{5.999245in}{2.114133in}}%
\pgfpathlineto{\pgfqpoint{6.001258in}{2.122114in}}%
\pgfpathlineto{\pgfqpoint{6.003272in}{2.125598in}}%
\pgfpathlineto{\pgfqpoint{6.005285in}{2.126865in}}%
\pgfpathlineto{\pgfqpoint{6.011325in}{2.126421in}}%
\pgfpathlineto{\pgfqpoint{6.013339in}{2.128828in}}%
\pgfpathlineto{\pgfqpoint{6.015352in}{2.134783in}}%
\pgfpathlineto{\pgfqpoint{6.017366in}{2.138963in}}%
\pgfpathlineto{\pgfqpoint{6.019379in}{2.141053in}}%
\pgfpathlineto{\pgfqpoint{6.025419in}{2.139470in}}%
\pgfpathlineto{\pgfqpoint{6.027433in}{2.141750in}}%
\pgfpathlineto{\pgfqpoint{6.029446in}{2.139090in}}%
\pgfpathlineto{\pgfqpoint{6.031460in}{2.140673in}}%
\pgfpathlineto{\pgfqpoint{6.033473in}{2.137950in}}%
\pgfpathlineto{\pgfqpoint{6.039514in}{2.138456in}}%
\pgfpathlineto{\pgfqpoint{6.041527in}{2.141370in}}%
\pgfpathlineto{\pgfqpoint{6.043540in}{2.134339in}}%
\pgfpathlineto{\pgfqpoint{6.045554in}{2.133072in}}%
\pgfpathlineto{\pgfqpoint{6.047567in}{2.143904in}}%
\pgfpathlineto{\pgfqpoint{6.053608in}{2.146944in}}%
\pgfpathlineto{\pgfqpoint{6.055621in}{2.149415in}}%
\pgfpathlineto{\pgfqpoint{6.057635in}{2.149478in}}%
\pgfpathlineto{\pgfqpoint{6.059648in}{2.150491in}}%
\pgfpathlineto{\pgfqpoint{6.061661in}{2.148084in}}%
\pgfpathlineto{\pgfqpoint{6.069715in}{2.144601in}}%
\pgfpathlineto{\pgfqpoint{6.071729in}{2.144664in}}%
\pgfpathlineto{\pgfqpoint{6.073742in}{2.147704in}}%
\pgfpathlineto{\pgfqpoint{6.075756in}{2.149731in}}%
\pgfpathlineto{\pgfqpoint{6.081796in}{2.142827in}}%
\pgfpathlineto{\pgfqpoint{6.083809in}{2.137633in}}%
\pgfpathlineto{\pgfqpoint{6.085823in}{2.135543in}}%
\pgfpathlineto{\pgfqpoint{6.087836in}{2.136556in}}%
\pgfpathlineto{\pgfqpoint{6.089850in}{2.140737in}}%
\pgfpathlineto{\pgfqpoint{6.095890in}{2.136176in}}%
\pgfpathlineto{\pgfqpoint{6.097904in}{2.137253in}}%
\pgfpathlineto{\pgfqpoint{6.099917in}{2.136936in}}%
\pgfpathlineto{\pgfqpoint{6.101930in}{2.141433in}}%
\pgfpathlineto{\pgfqpoint{6.103944in}{2.139470in}}%
\pgfpathlineto{\pgfqpoint{6.109984in}{2.148084in}}%
\pgfpathlineto{\pgfqpoint{6.111998in}{2.146057in}}%
\pgfpathlineto{\pgfqpoint{6.114011in}{2.147514in}}%
\pgfpathlineto{\pgfqpoint{6.116025in}{2.150048in}}%
\pgfpathlineto{\pgfqpoint{6.118038in}{2.150301in}}%
\pgfpathlineto{\pgfqpoint{6.124078in}{2.148211in}}%
\pgfpathlineto{\pgfqpoint{6.126092in}{2.146501in}}%
\pgfpathlineto{\pgfqpoint{6.128105in}{2.152012in}}%
\pgfpathlineto{\pgfqpoint{6.130119in}{2.146881in}}%
\pgfpathlineto{\pgfqpoint{6.132132in}{2.149668in}}%
\pgfpathlineto{\pgfqpoint{6.138172in}{2.149478in}}%
\pgfpathlineto{\pgfqpoint{6.142199in}{2.153785in}}%
\pgfpathlineto{\pgfqpoint{6.144213in}{2.148338in}}%
\pgfpathlineto{\pgfqpoint{6.146226in}{2.152708in}}%
\pgfpathlineto{\pgfqpoint{6.152267in}{2.155432in}}%
\pgfpathlineto{\pgfqpoint{6.154280in}{2.158346in}}%
\pgfpathlineto{\pgfqpoint{6.156294in}{2.159359in}}%
\pgfpathlineto{\pgfqpoint{6.158307in}{2.155495in}}%
\pgfpathlineto{\pgfqpoint{6.160320in}{2.157522in}}%
\pgfpathlineto{\pgfqpoint{6.166361in}{2.155685in}}%
\pgfpathlineto{\pgfqpoint{6.168374in}{2.152835in}}%
\pgfpathlineto{\pgfqpoint{6.170388in}{2.155559in}}%
\pgfpathlineto{\pgfqpoint{6.172401in}{2.151695in}}%
\pgfpathlineto{\pgfqpoint{6.174415in}{2.158029in}}%
\pgfpathlineto{\pgfqpoint{6.180455in}{2.155812in}}%
\pgfpathlineto{\pgfqpoint{6.182468in}{2.139407in}}%
\pgfpathlineto{\pgfqpoint{6.186495in}{2.128955in}}%
\pgfpathlineto{\pgfqpoint{6.188509in}{2.130032in}}%
\pgfpathlineto{\pgfqpoint{6.194549in}{2.128258in}}%
\pgfpathlineto{\pgfqpoint{6.196562in}{2.130095in}}%
\pgfpathlineto{\pgfqpoint{6.198576in}{2.139026in}}%
\pgfpathlineto{\pgfqpoint{6.200589in}{2.143524in}}%
\pgfpathlineto{\pgfqpoint{6.202603in}{2.145931in}}%
\pgfpathlineto{\pgfqpoint{6.208643in}{2.127942in}}%
\pgfpathlineto{\pgfqpoint{6.210657in}{2.125915in}}%
\pgfpathlineto{\pgfqpoint{6.212670in}{2.119960in}}%
\pgfpathlineto{\pgfqpoint{6.214683in}{2.117300in}}%
\pgfpathlineto{\pgfqpoint{6.216697in}{2.118060in}}%
\pgfpathlineto{\pgfqpoint{6.222737in}{2.119644in}}%
\pgfpathlineto{\pgfqpoint{6.224751in}{2.112423in}}%
\pgfpathlineto{\pgfqpoint{6.226764in}{2.129145in}}%
\pgfpathlineto{\pgfqpoint{6.228778in}{2.117553in}}%
\pgfpathlineto{\pgfqpoint{6.230791in}{2.113816in}}%
\pgfpathlineto{\pgfqpoint{6.236831in}{2.112549in}}%
\pgfpathlineto{\pgfqpoint{6.238845in}{2.115083in}}%
\pgfpathlineto{\pgfqpoint{6.240858in}{2.123254in}}%
\pgfpathlineto{\pgfqpoint{6.242872in}{2.112676in}}%
\pgfpathlineto{\pgfqpoint{6.244885in}{2.111283in}}%
\pgfpathlineto{\pgfqpoint{6.250926in}{2.112866in}}%
\pgfpathlineto{\pgfqpoint{6.252939in}{2.133389in}}%
\pgfpathlineto{\pgfqpoint{6.254952in}{2.138900in}}%
\pgfpathlineto{\pgfqpoint{6.256966in}{2.139597in}}%
\pgfpathlineto{\pgfqpoint{6.267033in}{2.087783in}}%
\pgfpathlineto{\pgfqpoint{6.269047in}{2.078028in}}%
\pgfpathlineto{\pgfqpoint{6.271060in}{2.080308in}}%
\pgfpathlineto{\pgfqpoint{6.273073in}{2.077711in}}%
\pgfpathlineto{\pgfqpoint{6.279114in}{2.078218in}}%
\pgfpathlineto{\pgfqpoint{6.281127in}{2.079548in}}%
\pgfpathlineto{\pgfqpoint{6.283141in}{2.082145in}}%
\pgfpathlineto{\pgfqpoint{6.285154in}{2.099564in}}%
\pgfpathlineto{\pgfqpoint{6.287168in}{2.099184in}}%
\pgfpathlineto{\pgfqpoint{6.293208in}{2.097284in}}%
\pgfpathlineto{\pgfqpoint{6.295221in}{2.102668in}}%
\pgfpathlineto{\pgfqpoint{6.301262in}{2.110903in}}%
\pgfpathlineto{\pgfqpoint{6.307302in}{2.106532in}}%
\pgfpathlineto{\pgfqpoint{6.309316in}{2.109129in}}%
\pgfpathlineto{\pgfqpoint{6.311329in}{2.124141in}}%
\pgfpathlineto{\pgfqpoint{6.313342in}{2.115970in}}%
\pgfpathlineto{\pgfqpoint{6.315356in}{2.117427in}}%
\pgfpathlineto{\pgfqpoint{6.321396in}{2.126548in}}%
\pgfpathlineto{\pgfqpoint{6.323410in}{2.127372in}}%
\pgfpathlineto{\pgfqpoint{6.325423in}{2.126865in}}%
\pgfpathlineto{\pgfqpoint{6.327437in}{2.130159in}}%
\pgfpathlineto{\pgfqpoint{6.329450in}{2.130539in}}%
\pgfpathlineto{\pgfqpoint{6.335490in}{2.132946in}}%
\pgfpathlineto{\pgfqpoint{6.339517in}{2.127625in}}%
\pgfpathlineto{\pgfqpoint{6.341531in}{2.134149in}}%
\pgfpathlineto{\pgfqpoint{6.343544in}{2.133516in}}%
\pgfpathlineto{\pgfqpoint{6.349584in}{2.135289in}}%
\pgfpathlineto{\pgfqpoint{6.351598in}{2.137380in}}%
\pgfpathlineto{\pgfqpoint{6.353611in}{2.136303in}}%
\pgfpathlineto{\pgfqpoint{6.355625in}{2.138203in}}%
\pgfpathlineto{\pgfqpoint{6.357638in}{2.146944in}}%
\pgfpathlineto{\pgfqpoint{6.363679in}{2.146754in}}%
\pgfpathlineto{\pgfqpoint{6.365692in}{2.139723in}}%
\pgfpathlineto{\pgfqpoint{6.367705in}{2.135163in}}%
\pgfpathlineto{\pgfqpoint{6.369719in}{2.140863in}}%
\pgfpathlineto{\pgfqpoint{6.371732in}{2.135669in}}%
\pgfpathlineto{\pgfqpoint{6.377773in}{2.140293in}}%
\pgfpathlineto{\pgfqpoint{6.379786in}{2.139850in}}%
\pgfpathlineto{\pgfqpoint{6.381800in}{2.142067in}}%
\pgfpathlineto{\pgfqpoint{6.383813in}{2.150175in}}%
\pgfpathlineto{\pgfqpoint{6.385827in}{2.147831in}}%
\pgfpathlineto{\pgfqpoint{6.391867in}{2.143650in}}%
\pgfpathlineto{\pgfqpoint{6.393880in}{2.145867in}}%
\pgfpathlineto{\pgfqpoint{6.395894in}{2.143017in}}%
\pgfpathlineto{\pgfqpoint{6.397907in}{2.131362in}}%
\pgfpathlineto{\pgfqpoint{6.399921in}{2.129779in}}%
\pgfpathlineto{\pgfqpoint{6.405961in}{2.123444in}}%
\pgfpathlineto{\pgfqpoint{6.407974in}{2.134022in}}%
\pgfpathlineto{\pgfqpoint{6.409988in}{2.126485in}}%
\pgfpathlineto{\pgfqpoint{6.412001in}{2.132756in}}%
\pgfpathlineto{\pgfqpoint{6.414015in}{2.124458in}}%
\pgfpathlineto{\pgfqpoint{6.420055in}{2.123508in}}%
\pgfpathlineto{\pgfqpoint{6.422069in}{2.127562in}}%
\pgfpathlineto{\pgfqpoint{6.424082in}{2.125661in}}%
\pgfpathlineto{\pgfqpoint{6.434149in}{2.128068in}}%
\pgfpathlineto{\pgfqpoint{6.438176in}{2.134403in}}%
\pgfpathlineto{\pgfqpoint{6.440190in}{2.154545in}}%
\pgfpathlineto{\pgfqpoint{6.442203in}{2.146438in}}%
\pgfpathlineto{\pgfqpoint{6.448243in}{2.145931in}}%
\pgfpathlineto{\pgfqpoint{6.450257in}{2.147451in}}%
\pgfpathlineto{\pgfqpoint{6.452270in}{2.153025in}}%
\pgfpathlineto{\pgfqpoint{6.454284in}{2.160373in}}%
\pgfpathlineto{\pgfqpoint{6.456297in}{2.162907in}}%
\pgfpathlineto{\pgfqpoint{6.464351in}{2.167024in}}%
\pgfpathlineto{\pgfqpoint{6.466364in}{2.172028in}}%
\pgfpathlineto{\pgfqpoint{6.468378in}{2.168734in}}%
\pgfpathlineto{\pgfqpoint{6.470391in}{2.183493in}}%
\pgfpathlineto{\pgfqpoint{6.476432in}{2.186787in}}%
\pgfpathlineto{\pgfqpoint{6.478445in}{2.187230in}}%
\pgfpathlineto{\pgfqpoint{6.482472in}{2.190460in}}%
\pgfpathlineto{\pgfqpoint{6.484485in}{2.189954in}}%
\pgfpathlineto{\pgfqpoint{6.492539in}{2.189384in}}%
\pgfpathlineto{\pgfqpoint{6.496566in}{2.195401in}}%
\pgfpathlineto{\pgfqpoint{6.498580in}{2.192044in}}%
\pgfpathlineto{\pgfqpoint{6.498580in}{2.192044in}}%
\pgfusepath{stroke}%
\end{pgfscope}%
\begin{pgfscope}%
\pgfsetrectcap%
\pgfsetmiterjoin%
\pgfsetlinewidth{0.803000pt}%
\definecolor{currentstroke}{rgb}{1.000000,1.000000,1.000000}%
\pgfsetstrokecolor{currentstroke}%
\pgfsetdash{}{0pt}%
\pgfpathmoveto{\pgfqpoint{1.875000in}{1.771471in}}%
\pgfpathlineto{\pgfqpoint{1.875000in}{2.215588in}}%
\pgfusepath{stroke}%
\end{pgfscope}%
\begin{pgfscope}%
\pgfsetrectcap%
\pgfsetmiterjoin%
\pgfsetlinewidth{0.803000pt}%
\definecolor{currentstroke}{rgb}{1.000000,1.000000,1.000000}%
\pgfsetstrokecolor{currentstroke}%
\pgfsetdash{}{0pt}%
\pgfpathmoveto{\pgfqpoint{6.718750in}{1.771471in}}%
\pgfpathlineto{\pgfqpoint{6.718750in}{2.215588in}}%
\pgfusepath{stroke}%
\end{pgfscope}%
\begin{pgfscope}%
\pgfsetrectcap%
\pgfsetmiterjoin%
\pgfsetlinewidth{0.803000pt}%
\definecolor{currentstroke}{rgb}{1.000000,1.000000,1.000000}%
\pgfsetstrokecolor{currentstroke}%
\pgfsetdash{}{0pt}%
\pgfpathmoveto{\pgfqpoint{1.875000in}{1.771471in}}%
\pgfpathlineto{\pgfqpoint{6.718750in}{1.771471in}}%
\pgfusepath{stroke}%
\end{pgfscope}%
\begin{pgfscope}%
\pgfsetrectcap%
\pgfsetmiterjoin%
\pgfsetlinewidth{0.803000pt}%
\definecolor{currentstroke}{rgb}{1.000000,1.000000,1.000000}%
\pgfsetstrokecolor{currentstroke}%
\pgfsetdash{}{0pt}%
\pgfpathmoveto{\pgfqpoint{1.875000in}{2.215588in}}%
\pgfpathlineto{\pgfqpoint{6.718750in}{2.215588in}}%
\pgfusepath{stroke}%
\end{pgfscope}%
\begin{pgfscope}%
\definecolor{textcolor}{rgb}{0.150000,0.150000,0.150000}%
\pgfsetstrokecolor{textcolor}%
\pgfsetfillcolor{textcolor}%
\pgftext[x=4.296875in,y=2.298922in,,base]{\color{textcolor}\rmfamily\fontsize{16.800000}{20.160000}\selectfont UTX}%
\end{pgfscope}%
\begin{pgfscope}%
\pgfsetbuttcap%
\pgfsetmiterjoin%
\definecolor{currentfill}{rgb}{0.917647,0.917647,0.949020}%
\pgfsetfillcolor{currentfill}%
\pgfsetlinewidth{0.000000pt}%
\definecolor{currentstroke}{rgb}{0.000000,0.000000,0.000000}%
\pgfsetstrokecolor{currentstroke}%
\pgfsetstrokeopacity{0.000000}%
\pgfsetdash{}{0pt}%
\pgfpathmoveto{\pgfqpoint{8.656250in}{1.771471in}}%
\pgfpathlineto{\pgfqpoint{13.500000in}{1.771471in}}%
\pgfpathlineto{\pgfqpoint{13.500000in}{2.215588in}}%
\pgfpathlineto{\pgfqpoint{8.656250in}{2.215588in}}%
\pgfpathclose%
\pgfusepath{fill}%
\end{pgfscope}%
\begin{pgfscope}%
\pgfpathrectangle{\pgfqpoint{8.656250in}{1.771471in}}{\pgfqpoint{4.843750in}{0.444118in}}%
\pgfusepath{clip}%
\pgfsetroundcap%
\pgfsetroundjoin%
\pgfsetlinewidth{0.803000pt}%
\definecolor{currentstroke}{rgb}{1.000000,1.000000,1.000000}%
\pgfsetstrokecolor{currentstroke}%
\pgfsetdash{}{0pt}%
\pgfpathmoveto{\pgfqpoint{8.872394in}{1.771471in}}%
\pgfpathlineto{\pgfqpoint{8.872394in}{2.215588in}}%
\pgfusepath{stroke}%
\end{pgfscope}%
\begin{pgfscope}%
\definecolor{textcolor}{rgb}{0.150000,0.150000,0.150000}%
\pgfsetstrokecolor{textcolor}%
\pgfsetfillcolor{textcolor}%
\pgftext[x=8.872394in,y=1.674248in,,top]{\color{textcolor}\rmfamily\fontsize{14.000000}{16.800000}\selectfont 2012}%
\end{pgfscope}%
\begin{pgfscope}%
\pgfpathrectangle{\pgfqpoint{8.656250in}{1.771471in}}{\pgfqpoint{4.843750in}{0.444118in}}%
\pgfusepath{clip}%
\pgfsetroundcap%
\pgfsetroundjoin%
\pgfsetlinewidth{0.803000pt}%
\definecolor{currentstroke}{rgb}{1.000000,1.000000,1.000000}%
\pgfsetstrokecolor{currentstroke}%
\pgfsetdash{}{0pt}%
\pgfpathmoveto{\pgfqpoint{9.609315in}{1.771471in}}%
\pgfpathlineto{\pgfqpoint{9.609315in}{2.215588in}}%
\pgfusepath{stroke}%
\end{pgfscope}%
\begin{pgfscope}%
\definecolor{textcolor}{rgb}{0.150000,0.150000,0.150000}%
\pgfsetstrokecolor{textcolor}%
\pgfsetfillcolor{textcolor}%
\pgftext[x=9.609315in,y=1.674248in,,top]{\color{textcolor}\rmfamily\fontsize{14.000000}{16.800000}\selectfont 2013}%
\end{pgfscope}%
\begin{pgfscope}%
\pgfpathrectangle{\pgfqpoint{8.656250in}{1.771471in}}{\pgfqpoint{4.843750in}{0.444118in}}%
\pgfusepath{clip}%
\pgfsetroundcap%
\pgfsetroundjoin%
\pgfsetlinewidth{0.803000pt}%
\definecolor{currentstroke}{rgb}{1.000000,1.000000,1.000000}%
\pgfsetstrokecolor{currentstroke}%
\pgfsetdash{}{0pt}%
\pgfpathmoveto{\pgfqpoint{10.344223in}{1.771471in}}%
\pgfpathlineto{\pgfqpoint{10.344223in}{2.215588in}}%
\pgfusepath{stroke}%
\end{pgfscope}%
\begin{pgfscope}%
\definecolor{textcolor}{rgb}{0.150000,0.150000,0.150000}%
\pgfsetstrokecolor{textcolor}%
\pgfsetfillcolor{textcolor}%
\pgftext[x=10.344223in,y=1.674248in,,top]{\color{textcolor}\rmfamily\fontsize{14.000000}{16.800000}\selectfont 2014}%
\end{pgfscope}%
\begin{pgfscope}%
\pgfpathrectangle{\pgfqpoint{8.656250in}{1.771471in}}{\pgfqpoint{4.843750in}{0.444118in}}%
\pgfusepath{clip}%
\pgfsetroundcap%
\pgfsetroundjoin%
\pgfsetlinewidth{0.803000pt}%
\definecolor{currentstroke}{rgb}{1.000000,1.000000,1.000000}%
\pgfsetstrokecolor{currentstroke}%
\pgfsetdash{}{0pt}%
\pgfpathmoveto{\pgfqpoint{11.079132in}{1.771471in}}%
\pgfpathlineto{\pgfqpoint{11.079132in}{2.215588in}}%
\pgfusepath{stroke}%
\end{pgfscope}%
\begin{pgfscope}%
\definecolor{textcolor}{rgb}{0.150000,0.150000,0.150000}%
\pgfsetstrokecolor{textcolor}%
\pgfsetfillcolor{textcolor}%
\pgftext[x=11.079132in,y=1.674248in,,top]{\color{textcolor}\rmfamily\fontsize{14.000000}{16.800000}\selectfont 2015}%
\end{pgfscope}%
\begin{pgfscope}%
\pgfpathrectangle{\pgfqpoint{8.656250in}{1.771471in}}{\pgfqpoint{4.843750in}{0.444118in}}%
\pgfusepath{clip}%
\pgfsetroundcap%
\pgfsetroundjoin%
\pgfsetlinewidth{0.803000pt}%
\definecolor{currentstroke}{rgb}{1.000000,1.000000,1.000000}%
\pgfsetstrokecolor{currentstroke}%
\pgfsetdash{}{0pt}%
\pgfpathmoveto{\pgfqpoint{11.814040in}{1.771471in}}%
\pgfpathlineto{\pgfqpoint{11.814040in}{2.215588in}}%
\pgfusepath{stroke}%
\end{pgfscope}%
\begin{pgfscope}%
\definecolor{textcolor}{rgb}{0.150000,0.150000,0.150000}%
\pgfsetstrokecolor{textcolor}%
\pgfsetfillcolor{textcolor}%
\pgftext[x=11.814040in,y=1.674248in,,top]{\color{textcolor}\rmfamily\fontsize{14.000000}{16.800000}\selectfont 2016}%
\end{pgfscope}%
\begin{pgfscope}%
\pgfpathrectangle{\pgfqpoint{8.656250in}{1.771471in}}{\pgfqpoint{4.843750in}{0.444118in}}%
\pgfusepath{clip}%
\pgfsetroundcap%
\pgfsetroundjoin%
\pgfsetlinewidth{0.803000pt}%
\definecolor{currentstroke}{rgb}{1.000000,1.000000,1.000000}%
\pgfsetstrokecolor{currentstroke}%
\pgfsetdash{}{0pt}%
\pgfpathmoveto{\pgfqpoint{12.550962in}{1.771471in}}%
\pgfpathlineto{\pgfqpoint{12.550962in}{2.215588in}}%
\pgfusepath{stroke}%
\end{pgfscope}%
\begin{pgfscope}%
\definecolor{textcolor}{rgb}{0.150000,0.150000,0.150000}%
\pgfsetstrokecolor{textcolor}%
\pgfsetfillcolor{textcolor}%
\pgftext[x=12.550962in,y=1.674248in,,top]{\color{textcolor}\rmfamily\fontsize{14.000000}{16.800000}\selectfont 2017}%
\end{pgfscope}%
\begin{pgfscope}%
\pgfpathrectangle{\pgfqpoint{8.656250in}{1.771471in}}{\pgfqpoint{4.843750in}{0.444118in}}%
\pgfusepath{clip}%
\pgfsetroundcap%
\pgfsetroundjoin%
\pgfsetlinewidth{0.803000pt}%
\definecolor{currentstroke}{rgb}{1.000000,1.000000,1.000000}%
\pgfsetstrokecolor{currentstroke}%
\pgfsetdash{}{0pt}%
\pgfpathmoveto{\pgfqpoint{13.285870in}{1.771471in}}%
\pgfpathlineto{\pgfqpoint{13.285870in}{2.215588in}}%
\pgfusepath{stroke}%
\end{pgfscope}%
\begin{pgfscope}%
\definecolor{textcolor}{rgb}{0.150000,0.150000,0.150000}%
\pgfsetstrokecolor{textcolor}%
\pgfsetfillcolor{textcolor}%
\pgftext[x=13.285870in,y=1.674248in,,top]{\color{textcolor}\rmfamily\fontsize{14.000000}{16.800000}\selectfont 2018}%
\end{pgfscope}%
\begin{pgfscope}%
\pgfpathrectangle{\pgfqpoint{8.656250in}{1.771471in}}{\pgfqpoint{4.843750in}{0.444118in}}%
\pgfusepath{clip}%
\pgfsetroundcap%
\pgfsetroundjoin%
\pgfsetlinewidth{0.803000pt}%
\definecolor{currentstroke}{rgb}{1.000000,1.000000,1.000000}%
\pgfsetstrokecolor{currentstroke}%
\pgfsetdash{}{0pt}%
\pgfpathmoveto{\pgfqpoint{8.656250in}{1.853082in}}%
\pgfpathlineto{\pgfqpoint{13.500000in}{1.853082in}}%
\pgfusepath{stroke}%
\end{pgfscope}%
\begin{pgfscope}%
\definecolor{textcolor}{rgb}{0.150000,0.150000,0.150000}%
\pgfsetstrokecolor{textcolor}%
\pgfsetfillcolor{textcolor}%
\pgftext[x=8.311605in,y=1.779216in,left,base]{\color{textcolor}\rmfamily\fontsize{14.000000}{16.800000}\selectfont 30}%
\end{pgfscope}%
\begin{pgfscope}%
\pgfpathrectangle{\pgfqpoint{8.656250in}{1.771471in}}{\pgfqpoint{4.843750in}{0.444118in}}%
\pgfusepath{clip}%
\pgfsetroundcap%
\pgfsetroundjoin%
\pgfsetlinewidth{0.803000pt}%
\definecolor{currentstroke}{rgb}{1.000000,1.000000,1.000000}%
\pgfsetstrokecolor{currentstroke}%
\pgfsetdash{}{0pt}%
\pgfpathmoveto{\pgfqpoint{8.656250in}{2.029081in}}%
\pgfpathlineto{\pgfqpoint{13.500000in}{2.029081in}}%
\pgfusepath{stroke}%
\end{pgfscope}%
\begin{pgfscope}%
\definecolor{textcolor}{rgb}{0.150000,0.150000,0.150000}%
\pgfsetstrokecolor{textcolor}%
\pgfsetfillcolor{textcolor}%
\pgftext[x=8.311605in,y=1.955215in,left,base]{\color{textcolor}\rmfamily\fontsize{14.000000}{16.800000}\selectfont 40}%
\end{pgfscope}%
\begin{pgfscope}%
\pgfpathrectangle{\pgfqpoint{8.656250in}{1.771471in}}{\pgfqpoint{4.843750in}{0.444118in}}%
\pgfusepath{clip}%
\pgfsetroundcap%
\pgfsetroundjoin%
\pgfsetlinewidth{0.803000pt}%
\definecolor{currentstroke}{rgb}{1.000000,1.000000,1.000000}%
\pgfsetstrokecolor{currentstroke}%
\pgfsetdash{}{0pt}%
\pgfpathmoveto{\pgfqpoint{8.656250in}{2.205081in}}%
\pgfpathlineto{\pgfqpoint{13.500000in}{2.205081in}}%
\pgfusepath{stroke}%
\end{pgfscope}%
\begin{pgfscope}%
\definecolor{textcolor}{rgb}{0.150000,0.150000,0.150000}%
\pgfsetstrokecolor{textcolor}%
\pgfsetfillcolor{textcolor}%
\pgftext[x=8.311605in,y=2.131215in,left,base]{\color{textcolor}\rmfamily\fontsize{14.000000}{16.800000}\selectfont 50}%
\end{pgfscope}%
\begin{pgfscope}%
\pgfpathrectangle{\pgfqpoint{8.656250in}{1.771471in}}{\pgfqpoint{4.843750in}{0.444118in}}%
\pgfusepath{clip}%
\pgfsetroundcap%
\pgfsetroundjoin%
\pgfsetlinewidth{1.505625pt}%
\definecolor{currentstroke}{rgb}{0.121569,0.466667,0.705882}%
\pgfsetstrokecolor{currentstroke}%
\pgfsetdash{}{0pt}%
\pgfpathmoveto{\pgfqpoint{8.876420in}{1.816826in}}%
\pgfpathlineto{\pgfqpoint{8.878434in}{1.810490in}}%
\pgfpathlineto{\pgfqpoint{8.880447in}{1.807146in}}%
\pgfpathlineto{\pgfqpoint{8.882461in}{1.805738in}}%
\pgfpathlineto{\pgfqpoint{8.888501in}{1.806266in}}%
\pgfpathlineto{\pgfqpoint{8.890515in}{1.808730in}}%
\pgfpathlineto{\pgfqpoint{8.892528in}{1.812954in}}%
\pgfpathlineto{\pgfqpoint{8.908636in}{1.814186in}}%
\pgfpathlineto{\pgfqpoint{8.910649in}{1.813834in}}%
\pgfpathlineto{\pgfqpoint{8.916689in}{1.806618in}}%
\pgfpathlineto{\pgfqpoint{8.918703in}{1.799050in}}%
\pgfpathlineto{\pgfqpoint{8.920716in}{1.797642in}}%
\pgfpathlineto{\pgfqpoint{8.922730in}{1.793242in}}%
\pgfpathlineto{\pgfqpoint{8.924743in}{1.791658in}}%
\pgfpathlineto{\pgfqpoint{8.930784in}{1.796762in}}%
\pgfpathlineto{\pgfqpoint{8.932797in}{1.797290in}}%
\pgfpathlineto{\pgfqpoint{8.934810in}{1.799050in}}%
\pgfpathlineto{\pgfqpoint{8.936824in}{1.796058in}}%
\pgfpathlineto{\pgfqpoint{8.938837in}{1.799578in}}%
\pgfpathlineto{\pgfqpoint{8.944878in}{1.803274in}}%
\pgfpathlineto{\pgfqpoint{8.946891in}{1.800634in}}%
\pgfpathlineto{\pgfqpoint{8.950918in}{1.800634in}}%
\pgfpathlineto{\pgfqpoint{8.952931in}{1.797642in}}%
\pgfpathlineto{\pgfqpoint{8.958972in}{1.803274in}}%
\pgfpathlineto{\pgfqpoint{8.960985in}{1.802042in}}%
\pgfpathlineto{\pgfqpoint{8.962999in}{1.799402in}}%
\pgfpathlineto{\pgfqpoint{8.965012in}{1.802218in}}%
\pgfpathlineto{\pgfqpoint{8.967026in}{1.807322in}}%
\pgfpathlineto{\pgfqpoint{8.975079in}{1.807674in}}%
\pgfpathlineto{\pgfqpoint{8.977093in}{1.804154in}}%
\pgfpathlineto{\pgfqpoint{8.979106in}{1.803274in}}%
\pgfpathlineto{\pgfqpoint{8.991187in}{1.802922in}}%
\pgfpathlineto{\pgfqpoint{8.995214in}{1.809962in}}%
\pgfpathlineto{\pgfqpoint{9.001254in}{1.814186in}}%
\pgfpathlineto{\pgfqpoint{9.003268in}{1.810314in}}%
\pgfpathlineto{\pgfqpoint{9.005281in}{1.812602in}}%
\pgfpathlineto{\pgfqpoint{9.007295in}{1.816826in}}%
\pgfpathlineto{\pgfqpoint{9.009308in}{1.815418in}}%
\pgfpathlineto{\pgfqpoint{9.015348in}{1.818234in}}%
\pgfpathlineto{\pgfqpoint{9.017362in}{1.820170in}}%
\pgfpathlineto{\pgfqpoint{9.019375in}{1.819994in}}%
\pgfpathlineto{\pgfqpoint{9.023402in}{1.821226in}}%
\pgfpathlineto{\pgfqpoint{9.031456in}{1.822106in}}%
\pgfpathlineto{\pgfqpoint{9.033469in}{1.823866in}}%
\pgfpathlineto{\pgfqpoint{9.035483in}{1.822458in}}%
\pgfpathlineto{\pgfqpoint{9.037496in}{1.819466in}}%
\pgfpathlineto{\pgfqpoint{9.043537in}{1.818234in}}%
\pgfpathlineto{\pgfqpoint{9.045550in}{1.809962in}}%
\pgfpathlineto{\pgfqpoint{9.047563in}{1.804858in}}%
\pgfpathlineto{\pgfqpoint{9.049577in}{1.802570in}}%
\pgfpathlineto{\pgfqpoint{9.051590in}{1.804506in}}%
\pgfpathlineto{\pgfqpoint{9.057631in}{1.808202in}}%
\pgfpathlineto{\pgfqpoint{9.061658in}{1.806442in}}%
\pgfpathlineto{\pgfqpoint{9.063671in}{1.803626in}}%
\pgfpathlineto{\pgfqpoint{9.071725in}{1.800986in}}%
\pgfpathlineto{\pgfqpoint{9.073738in}{1.792714in}}%
\pgfpathlineto{\pgfqpoint{9.075752in}{1.799930in}}%
\pgfpathlineto{\pgfqpoint{9.077765in}{1.802218in}}%
\pgfpathlineto{\pgfqpoint{9.079779in}{1.798522in}}%
\pgfpathlineto{\pgfqpoint{9.085819in}{1.800634in}}%
\pgfpathlineto{\pgfqpoint{9.087832in}{1.804506in}}%
\pgfpathlineto{\pgfqpoint{9.089846in}{1.803626in}}%
\pgfpathlineto{\pgfqpoint{9.093873in}{1.817178in}}%
\pgfpathlineto{\pgfqpoint{9.099913in}{1.815066in}}%
\pgfpathlineto{\pgfqpoint{9.101927in}{1.827034in}}%
\pgfpathlineto{\pgfqpoint{9.103940in}{1.826682in}}%
\pgfpathlineto{\pgfqpoint{9.105953in}{1.835130in}}%
\pgfpathlineto{\pgfqpoint{9.107967in}{1.836186in}}%
\pgfpathlineto{\pgfqpoint{9.114007in}{1.838122in}}%
\pgfpathlineto{\pgfqpoint{9.116021in}{1.840410in}}%
\pgfpathlineto{\pgfqpoint{9.120048in}{1.841466in}}%
\pgfpathlineto{\pgfqpoint{9.122061in}{1.836538in}}%
\pgfpathlineto{\pgfqpoint{9.128101in}{1.840058in}}%
\pgfpathlineto{\pgfqpoint{9.130115in}{1.840234in}}%
\pgfpathlineto{\pgfqpoint{9.132128in}{1.836538in}}%
\pgfpathlineto{\pgfqpoint{9.134142in}{1.840234in}}%
\pgfpathlineto{\pgfqpoint{9.136155in}{1.847978in}}%
\pgfpathlineto{\pgfqpoint{9.142195in}{1.844634in}}%
\pgfpathlineto{\pgfqpoint{9.144209in}{1.846570in}}%
\pgfpathlineto{\pgfqpoint{9.146222in}{1.844458in}}%
\pgfpathlineto{\pgfqpoint{9.148236in}{1.850794in}}%
\pgfpathlineto{\pgfqpoint{9.150249in}{1.852730in}}%
\pgfpathlineto{\pgfqpoint{9.156290in}{1.850266in}}%
\pgfpathlineto{\pgfqpoint{9.158303in}{1.850970in}}%
\pgfpathlineto{\pgfqpoint{9.160317in}{1.849562in}}%
\pgfpathlineto{\pgfqpoint{9.162330in}{1.850970in}}%
\pgfpathlineto{\pgfqpoint{9.164343in}{1.851674in}}%
\pgfpathlineto{\pgfqpoint{9.172397in}{1.855546in}}%
\pgfpathlineto{\pgfqpoint{9.174411in}{1.851322in}}%
\pgfpathlineto{\pgfqpoint{9.176424in}{1.854138in}}%
\pgfpathlineto{\pgfqpoint{9.178438in}{1.846394in}}%
\pgfpathlineto{\pgfqpoint{9.184478in}{1.850266in}}%
\pgfpathlineto{\pgfqpoint{9.186491in}{1.848506in}}%
\pgfpathlineto{\pgfqpoint{9.188505in}{1.855722in}}%
\pgfpathlineto{\pgfqpoint{9.190518in}{1.854138in}}%
\pgfpathlineto{\pgfqpoint{9.192532in}{1.864346in}}%
\pgfpathlineto{\pgfqpoint{9.198572in}{1.865754in}}%
\pgfpathlineto{\pgfqpoint{9.200585in}{1.870682in}}%
\pgfpathlineto{\pgfqpoint{9.202599in}{1.871210in}}%
\pgfpathlineto{\pgfqpoint{9.204612in}{1.880890in}}%
\pgfpathlineto{\pgfqpoint{9.206626in}{1.878426in}}%
\pgfpathlineto{\pgfqpoint{9.212666in}{1.881770in}}%
\pgfpathlineto{\pgfqpoint{9.214680in}{1.880714in}}%
\pgfpathlineto{\pgfqpoint{9.216693in}{1.875258in}}%
\pgfpathlineto{\pgfqpoint{9.218706in}{1.875610in}}%
\pgfpathlineto{\pgfqpoint{9.220720in}{1.883530in}}%
\pgfpathlineto{\pgfqpoint{9.226760in}{1.879658in}}%
\pgfpathlineto{\pgfqpoint{9.228774in}{1.882122in}}%
\pgfpathlineto{\pgfqpoint{9.230787in}{1.881418in}}%
\pgfpathlineto{\pgfqpoint{9.232801in}{1.883706in}}%
\pgfpathlineto{\pgfqpoint{9.234814in}{1.889690in}}%
\pgfpathlineto{\pgfqpoint{9.240854in}{1.896026in}}%
\pgfpathlineto{\pgfqpoint{9.242868in}{1.896202in}}%
\pgfpathlineto{\pgfqpoint{9.246895in}{1.895498in}}%
\pgfpathlineto{\pgfqpoint{9.248908in}{1.895850in}}%
\pgfpathlineto{\pgfqpoint{9.254949in}{1.899898in}}%
\pgfpathlineto{\pgfqpoint{9.256962in}{1.899370in}}%
\pgfpathlineto{\pgfqpoint{9.258975in}{1.902010in}}%
\pgfpathlineto{\pgfqpoint{9.260989in}{1.899018in}}%
\pgfpathlineto{\pgfqpoint{9.263002in}{1.906058in}}%
\pgfpathlineto{\pgfqpoint{9.269043in}{1.906762in}}%
\pgfpathlineto{\pgfqpoint{9.271056in}{1.911690in}}%
\pgfpathlineto{\pgfqpoint{9.273070in}{1.914682in}}%
\pgfpathlineto{\pgfqpoint{9.275083in}{1.897434in}}%
\pgfpathlineto{\pgfqpoint{9.283137in}{1.894970in}}%
\pgfpathlineto{\pgfqpoint{9.285150in}{1.887402in}}%
\pgfpathlineto{\pgfqpoint{9.287164in}{1.886874in}}%
\pgfpathlineto{\pgfqpoint{9.289177in}{1.896378in}}%
\pgfpathlineto{\pgfqpoint{9.291191in}{1.902010in}}%
\pgfpathlineto{\pgfqpoint{9.297231in}{1.902714in}}%
\pgfpathlineto{\pgfqpoint{9.299244in}{1.905002in}}%
\pgfpathlineto{\pgfqpoint{9.301258in}{1.906058in}}%
\pgfpathlineto{\pgfqpoint{9.303271in}{1.898314in}}%
\pgfpathlineto{\pgfqpoint{9.305285in}{1.896378in}}%
\pgfpathlineto{\pgfqpoint{9.311325in}{1.899370in}}%
\pgfpathlineto{\pgfqpoint{9.315352in}{1.892682in}}%
\pgfpathlineto{\pgfqpoint{9.317365in}{1.894794in}}%
\pgfpathlineto{\pgfqpoint{9.319379in}{1.898138in}}%
\pgfpathlineto{\pgfqpoint{9.325419in}{1.893914in}}%
\pgfpathlineto{\pgfqpoint{9.327433in}{1.894618in}}%
\pgfpathlineto{\pgfqpoint{9.329446in}{1.892858in}}%
\pgfpathlineto{\pgfqpoint{9.333473in}{1.891274in}}%
\pgfpathlineto{\pgfqpoint{9.339513in}{1.886522in}}%
\pgfpathlineto{\pgfqpoint{9.341527in}{1.876138in}}%
\pgfpathlineto{\pgfqpoint{9.343540in}{1.872618in}}%
\pgfpathlineto{\pgfqpoint{9.345554in}{1.867866in}}%
\pgfpathlineto{\pgfqpoint{9.347567in}{1.879834in}}%
\pgfpathlineto{\pgfqpoint{9.355621in}{1.873322in}}%
\pgfpathlineto{\pgfqpoint{9.357634in}{1.878778in}}%
\pgfpathlineto{\pgfqpoint{9.359648in}{1.874554in}}%
\pgfpathlineto{\pgfqpoint{9.371728in}{1.887930in}}%
\pgfpathlineto{\pgfqpoint{9.373742in}{1.892330in}}%
\pgfpathlineto{\pgfqpoint{9.375755in}{1.886874in}}%
\pgfpathlineto{\pgfqpoint{9.381796in}{1.891274in}}%
\pgfpathlineto{\pgfqpoint{9.383809in}{1.893562in}}%
\pgfpathlineto{\pgfqpoint{9.387836in}{1.910810in}}%
\pgfpathlineto{\pgfqpoint{9.389849in}{1.897258in}}%
\pgfpathlineto{\pgfqpoint{9.395890in}{1.897962in}}%
\pgfpathlineto{\pgfqpoint{9.401930in}{1.909578in}}%
\pgfpathlineto{\pgfqpoint{9.403944in}{1.911514in}}%
\pgfpathlineto{\pgfqpoint{9.409984in}{1.912042in}}%
\pgfpathlineto{\pgfqpoint{9.414011in}{1.910810in}}%
\pgfpathlineto{\pgfqpoint{9.416024in}{1.913098in}}%
\pgfpathlineto{\pgfqpoint{9.418038in}{1.910634in}}%
\pgfpathlineto{\pgfqpoint{9.426092in}{1.914330in}}%
\pgfpathlineto{\pgfqpoint{9.428105in}{1.919786in}}%
\pgfpathlineto{\pgfqpoint{9.430118in}{1.932282in}}%
\pgfpathlineto{\pgfqpoint{9.432132in}{1.936330in}}%
\pgfpathlineto{\pgfqpoint{9.438172in}{1.929994in}}%
\pgfpathlineto{\pgfqpoint{9.440186in}{1.924010in}}%
\pgfpathlineto{\pgfqpoint{9.442199in}{1.919786in}}%
\pgfpathlineto{\pgfqpoint{9.446226in}{1.904650in}}%
\pgfpathlineto{\pgfqpoint{9.452266in}{1.903066in}}%
\pgfpathlineto{\pgfqpoint{9.454280in}{1.897610in}}%
\pgfpathlineto{\pgfqpoint{9.456293in}{1.906058in}}%
\pgfpathlineto{\pgfqpoint{9.458307in}{1.919786in}}%
\pgfpathlineto{\pgfqpoint{9.460320in}{1.911690in}}%
\pgfpathlineto{\pgfqpoint{9.466360in}{1.906586in}}%
\pgfpathlineto{\pgfqpoint{9.468374in}{1.897610in}}%
\pgfpathlineto{\pgfqpoint{9.470387in}{1.899722in}}%
\pgfpathlineto{\pgfqpoint{9.472401in}{1.899898in}}%
\pgfpathlineto{\pgfqpoint{9.474414in}{1.906058in}}%
\pgfpathlineto{\pgfqpoint{9.484482in}{1.905002in}}%
\pgfpathlineto{\pgfqpoint{9.486495in}{1.911514in}}%
\pgfpathlineto{\pgfqpoint{9.488508in}{1.903418in}}%
\pgfpathlineto{\pgfqpoint{9.494549in}{1.899194in}}%
\pgfpathlineto{\pgfqpoint{9.496562in}{1.900602in}}%
\pgfpathlineto{\pgfqpoint{9.498576in}{1.886170in}}%
\pgfpathlineto{\pgfqpoint{9.500589in}{1.878602in}}%
\pgfpathlineto{\pgfqpoint{9.502603in}{1.878954in}}%
\pgfpathlineto{\pgfqpoint{9.510656in}{1.877722in}}%
\pgfpathlineto{\pgfqpoint{9.512670in}{1.873850in}}%
\pgfpathlineto{\pgfqpoint{9.514683in}{1.866810in}}%
\pgfpathlineto{\pgfqpoint{9.516697in}{1.862938in}}%
\pgfpathlineto{\pgfqpoint{9.522737in}{1.881242in}}%
\pgfpathlineto{\pgfqpoint{9.524750in}{1.881242in}}%
\pgfpathlineto{\pgfqpoint{9.526764in}{1.885818in}}%
\pgfpathlineto{\pgfqpoint{9.530791in}{1.893562in}}%
\pgfpathlineto{\pgfqpoint{9.536831in}{1.887578in}}%
\pgfpathlineto{\pgfqpoint{9.538845in}{1.883354in}}%
\pgfpathlineto{\pgfqpoint{9.542871in}{1.896730in}}%
\pgfpathlineto{\pgfqpoint{9.544885in}{1.898138in}}%
\pgfpathlineto{\pgfqpoint{9.550925in}{1.897962in}}%
\pgfpathlineto{\pgfqpoint{9.552939in}{1.892330in}}%
\pgfpathlineto{\pgfqpoint{9.556966in}{1.902538in}}%
\pgfpathlineto{\pgfqpoint{9.558979in}{1.902010in}}%
\pgfpathlineto{\pgfqpoint{9.565019in}{1.897082in}}%
\pgfpathlineto{\pgfqpoint{9.569046in}{1.906938in}}%
\pgfpathlineto{\pgfqpoint{9.573073in}{1.899370in}}%
\pgfpathlineto{\pgfqpoint{9.579114in}{1.898138in}}%
\pgfpathlineto{\pgfqpoint{9.581127in}{1.894970in}}%
\pgfpathlineto{\pgfqpoint{9.583140in}{1.888634in}}%
\pgfpathlineto{\pgfqpoint{9.585154in}{1.894266in}}%
\pgfpathlineto{\pgfqpoint{9.587167in}{1.891098in}}%
\pgfpathlineto{\pgfqpoint{9.593208in}{1.890746in}}%
\pgfpathlineto{\pgfqpoint{9.597235in}{1.889514in}}%
\pgfpathlineto{\pgfqpoint{9.599248in}{1.889866in}}%
\pgfpathlineto{\pgfqpoint{9.601261in}{1.882298in}}%
\pgfpathlineto{\pgfqpoint{9.607302in}{1.887226in}}%
\pgfpathlineto{\pgfqpoint{9.611329in}{1.900074in}}%
\pgfpathlineto{\pgfqpoint{9.613342in}{1.897434in}}%
\pgfpathlineto{\pgfqpoint{9.615356in}{1.900602in}}%
\pgfpathlineto{\pgfqpoint{9.621396in}{1.905530in}}%
\pgfpathlineto{\pgfqpoint{9.623409in}{1.891450in}}%
\pgfpathlineto{\pgfqpoint{9.625423in}{1.890218in}}%
\pgfpathlineto{\pgfqpoint{9.627436in}{1.897962in}}%
\pgfpathlineto{\pgfqpoint{9.629450in}{1.894090in}}%
\pgfpathlineto{\pgfqpoint{9.635490in}{1.884762in}}%
\pgfpathlineto{\pgfqpoint{9.639517in}{1.870506in}}%
\pgfpathlineto{\pgfqpoint{9.641530in}{1.878778in}}%
\pgfpathlineto{\pgfqpoint{9.643544in}{1.884058in}}%
\pgfpathlineto{\pgfqpoint{9.651598in}{1.889338in}}%
\pgfpathlineto{\pgfqpoint{9.655625in}{1.884762in}}%
\pgfpathlineto{\pgfqpoint{9.657638in}{1.885818in}}%
\pgfpathlineto{\pgfqpoint{9.663678in}{1.887226in}}%
\pgfpathlineto{\pgfqpoint{9.665692in}{1.896730in}}%
\pgfpathlineto{\pgfqpoint{9.667705in}{1.898138in}}%
\pgfpathlineto{\pgfqpoint{9.669719in}{1.898138in}}%
\pgfpathlineto{\pgfqpoint{9.671732in}{1.910634in}}%
\pgfpathlineto{\pgfqpoint{9.677772in}{1.910106in}}%
\pgfpathlineto{\pgfqpoint{9.679786in}{1.910634in}}%
\pgfpathlineto{\pgfqpoint{9.681799in}{1.913098in}}%
\pgfpathlineto{\pgfqpoint{9.683813in}{1.909402in}}%
\pgfpathlineto{\pgfqpoint{9.685826in}{1.907994in}}%
\pgfpathlineto{\pgfqpoint{9.691867in}{1.907466in}}%
\pgfpathlineto{\pgfqpoint{9.695893in}{1.910106in}}%
\pgfpathlineto{\pgfqpoint{9.697907in}{1.907642in}}%
\pgfpathlineto{\pgfqpoint{9.699920in}{1.908522in}}%
\pgfpathlineto{\pgfqpoint{9.707974in}{1.909930in}}%
\pgfpathlineto{\pgfqpoint{9.709988in}{1.915386in}}%
\pgfpathlineto{\pgfqpoint{9.712001in}{1.918026in}}%
\pgfpathlineto{\pgfqpoint{9.714015in}{1.921722in}}%
\pgfpathlineto{\pgfqpoint{9.720055in}{1.925946in}}%
\pgfpathlineto{\pgfqpoint{9.722068in}{1.931226in}}%
\pgfpathlineto{\pgfqpoint{9.726095in}{1.936506in}}%
\pgfpathlineto{\pgfqpoint{9.728109in}{1.939146in}}%
\pgfpathlineto{\pgfqpoint{9.734149in}{1.944249in}}%
\pgfpathlineto{\pgfqpoint{9.736162in}{1.951817in}}%
\pgfpathlineto{\pgfqpoint{9.738176in}{1.946361in}}%
\pgfpathlineto{\pgfqpoint{9.740189in}{1.949001in}}%
\pgfpathlineto{\pgfqpoint{9.742203in}{1.955337in}}%
\pgfpathlineto{\pgfqpoint{9.748243in}{1.953401in}}%
\pgfpathlineto{\pgfqpoint{9.750257in}{1.960617in}}%
\pgfpathlineto{\pgfqpoint{9.752270in}{1.955161in}}%
\pgfpathlineto{\pgfqpoint{9.754283in}{1.962201in}}%
\pgfpathlineto{\pgfqpoint{9.756297in}{1.956217in}}%
\pgfpathlineto{\pgfqpoint{9.764351in}{1.968361in}}%
\pgfpathlineto{\pgfqpoint{9.766364in}{1.963785in}}%
\pgfpathlineto{\pgfqpoint{9.770391in}{1.969241in}}%
\pgfpathlineto{\pgfqpoint{9.776431in}{1.971177in}}%
\pgfpathlineto{\pgfqpoint{9.778445in}{1.975401in}}%
\pgfpathlineto{\pgfqpoint{9.780458in}{1.968185in}}%
\pgfpathlineto{\pgfqpoint{9.782472in}{1.971001in}}%
\pgfpathlineto{\pgfqpoint{9.790525in}{1.971881in}}%
\pgfpathlineto{\pgfqpoint{9.792539in}{1.975577in}}%
\pgfpathlineto{\pgfqpoint{9.794552in}{1.968889in}}%
\pgfpathlineto{\pgfqpoint{9.798579in}{1.976457in}}%
\pgfpathlineto{\pgfqpoint{9.804620in}{1.981561in}}%
\pgfpathlineto{\pgfqpoint{9.806633in}{1.980505in}}%
\pgfpathlineto{\pgfqpoint{9.808647in}{1.987193in}}%
\pgfpathlineto{\pgfqpoint{9.810660in}{1.995817in}}%
\pgfpathlineto{\pgfqpoint{9.812673in}{2.000569in}}%
\pgfpathlineto{\pgfqpoint{9.818714in}{1.997577in}}%
\pgfpathlineto{\pgfqpoint{9.820727in}{1.995113in}}%
\pgfpathlineto{\pgfqpoint{9.822741in}{1.982969in}}%
\pgfpathlineto{\pgfqpoint{9.826768in}{2.018873in}}%
\pgfpathlineto{\pgfqpoint{9.832808in}{2.018169in}}%
\pgfpathlineto{\pgfqpoint{9.834821in}{2.019929in}}%
\pgfpathlineto{\pgfqpoint{9.836835in}{2.012889in}}%
\pgfpathlineto{\pgfqpoint{9.838848in}{2.031897in}}%
\pgfpathlineto{\pgfqpoint{9.840862in}{2.037353in}}%
\pgfpathlineto{\pgfqpoint{9.846902in}{2.035065in}}%
\pgfpathlineto{\pgfqpoint{9.848915in}{2.041049in}}%
\pgfpathlineto{\pgfqpoint{9.850929in}{2.020985in}}%
\pgfpathlineto{\pgfqpoint{9.854956in}{2.024681in}}%
\pgfpathlineto{\pgfqpoint{9.860996in}{2.016233in}}%
\pgfpathlineto{\pgfqpoint{9.863010in}{2.027849in}}%
\pgfpathlineto{\pgfqpoint{9.865023in}{2.030313in}}%
\pgfpathlineto{\pgfqpoint{9.867036in}{2.025033in}}%
\pgfpathlineto{\pgfqpoint{9.869050in}{2.027497in}}%
\pgfpathlineto{\pgfqpoint{9.875090in}{2.022921in}}%
\pgfpathlineto{\pgfqpoint{9.877104in}{2.031193in}}%
\pgfpathlineto{\pgfqpoint{9.879117in}{2.036825in}}%
\pgfpathlineto{\pgfqpoint{9.881131in}{2.031545in}}%
\pgfpathlineto{\pgfqpoint{9.883144in}{2.033481in}}%
\pgfpathlineto{\pgfqpoint{9.889184in}{2.025385in}}%
\pgfpathlineto{\pgfqpoint{9.893211in}{2.008665in}}%
\pgfpathlineto{\pgfqpoint{9.895225in}{2.014121in}}%
\pgfpathlineto{\pgfqpoint{9.897238in}{2.007609in}}%
\pgfpathlineto{\pgfqpoint{9.905292in}{2.000041in}}%
\pgfpathlineto{\pgfqpoint{9.907305in}{1.983321in}}%
\pgfpathlineto{\pgfqpoint{9.911332in}{1.968889in}}%
\pgfpathlineto{\pgfqpoint{9.917373in}{1.971353in}}%
\pgfpathlineto{\pgfqpoint{9.919386in}{1.973641in}}%
\pgfpathlineto{\pgfqpoint{9.921400in}{1.966425in}}%
\pgfpathlineto{\pgfqpoint{9.923413in}{1.988601in}}%
\pgfpathlineto{\pgfqpoint{9.925426in}{1.992297in}}%
\pgfpathlineto{\pgfqpoint{9.931467in}{1.996169in}}%
\pgfpathlineto{\pgfqpoint{9.935494in}{1.987721in}}%
\pgfpathlineto{\pgfqpoint{9.937507in}{1.997577in}}%
\pgfpathlineto{\pgfqpoint{9.939521in}{2.003209in}}%
\pgfpathlineto{\pgfqpoint{9.945561in}{1.998457in}}%
\pgfpathlineto{\pgfqpoint{9.947574in}{2.009721in}}%
\pgfpathlineto{\pgfqpoint{9.949588in}{1.989657in}}%
\pgfpathlineto{\pgfqpoint{9.951601in}{1.975225in}}%
\pgfpathlineto{\pgfqpoint{9.953615in}{1.982617in}}%
\pgfpathlineto{\pgfqpoint{9.959655in}{1.977337in}}%
\pgfpathlineto{\pgfqpoint{9.961669in}{1.994937in}}%
\pgfpathlineto{\pgfqpoint{9.963682in}{1.997753in}}%
\pgfpathlineto{\pgfqpoint{9.965695in}{2.002329in}}%
\pgfpathlineto{\pgfqpoint{9.967709in}{1.993529in}}%
\pgfpathlineto{\pgfqpoint{9.973749in}{1.993881in}}%
\pgfpathlineto{\pgfqpoint{9.975763in}{1.997401in}}%
\pgfpathlineto{\pgfqpoint{9.977776in}{2.002505in}}%
\pgfpathlineto{\pgfqpoint{9.987843in}{2.011481in}}%
\pgfpathlineto{\pgfqpoint{9.989857in}{2.008665in}}%
\pgfpathlineto{\pgfqpoint{9.991870in}{2.003033in}}%
\pgfpathlineto{\pgfqpoint{9.993884in}{2.012009in}}%
\pgfpathlineto{\pgfqpoint{9.995897in}{2.001273in}}%
\pgfpathlineto{\pgfqpoint{10.001937in}{1.995289in}}%
\pgfpathlineto{\pgfqpoint{10.003951in}{1.999513in}}%
\pgfpathlineto{\pgfqpoint{10.005964in}{2.005673in}}%
\pgfpathlineto{\pgfqpoint{10.007978in}{1.995465in}}%
\pgfpathlineto{\pgfqpoint{10.009991in}{1.995113in}}%
\pgfpathlineto{\pgfqpoint{10.018045in}{2.000569in}}%
\pgfpathlineto{\pgfqpoint{10.020058in}{2.000921in}}%
\pgfpathlineto{\pgfqpoint{10.024085in}{2.009545in}}%
\pgfpathlineto{\pgfqpoint{10.030126in}{2.015705in}}%
\pgfpathlineto{\pgfqpoint{10.034153in}{1.988777in}}%
\pgfpathlineto{\pgfqpoint{10.036166in}{1.995993in}}%
\pgfpathlineto{\pgfqpoint{10.038180in}{1.999161in}}%
\pgfpathlineto{\pgfqpoint{10.044220in}{1.998633in}}%
\pgfpathlineto{\pgfqpoint{10.048247in}{1.994937in}}%
\pgfpathlineto{\pgfqpoint{10.052274in}{1.986665in}}%
\pgfpathlineto{\pgfqpoint{10.058314in}{1.991065in}}%
\pgfpathlineto{\pgfqpoint{10.060327in}{1.984377in}}%
\pgfpathlineto{\pgfqpoint{10.064354in}{1.976281in}}%
\pgfpathlineto{\pgfqpoint{10.066368in}{1.965017in}}%
\pgfpathlineto{\pgfqpoint{10.072408in}{1.962201in}}%
\pgfpathlineto{\pgfqpoint{10.074422in}{1.967833in}}%
\pgfpathlineto{\pgfqpoint{10.076435in}{1.959209in}}%
\pgfpathlineto{\pgfqpoint{10.078448in}{1.955865in}}%
\pgfpathlineto{\pgfqpoint{10.080462in}{1.963785in}}%
\pgfpathlineto{\pgfqpoint{10.086502in}{1.954809in}}%
\pgfpathlineto{\pgfqpoint{10.088516in}{1.954809in}}%
\pgfpathlineto{\pgfqpoint{10.090529in}{1.949705in}}%
\pgfpathlineto{\pgfqpoint{10.092543in}{1.966601in}}%
\pgfpathlineto{\pgfqpoint{10.094556in}{1.960617in}}%
\pgfpathlineto{\pgfqpoint{10.102610in}{1.942313in}}%
\pgfpathlineto{\pgfqpoint{10.104623in}{1.952521in}}%
\pgfpathlineto{\pgfqpoint{10.106637in}{1.950761in}}%
\pgfpathlineto{\pgfqpoint{10.108650in}{1.946713in}}%
\pgfpathlineto{\pgfqpoint{10.114691in}{1.940906in}}%
\pgfpathlineto{\pgfqpoint{10.116704in}{1.948473in}}%
\pgfpathlineto{\pgfqpoint{10.118717in}{1.949177in}}%
\pgfpathlineto{\pgfqpoint{10.120731in}{1.960265in}}%
\pgfpathlineto{\pgfqpoint{10.122744in}{1.965721in}}%
\pgfpathlineto{\pgfqpoint{10.128785in}{1.972937in}}%
\pgfpathlineto{\pgfqpoint{10.130798in}{1.976633in}}%
\pgfpathlineto{\pgfqpoint{10.132812in}{1.978569in}}%
\pgfpathlineto{\pgfqpoint{10.134825in}{1.975753in}}%
\pgfpathlineto{\pgfqpoint{10.136838in}{1.966073in}}%
\pgfpathlineto{\pgfqpoint{10.142879in}{1.968713in}}%
\pgfpathlineto{\pgfqpoint{10.144892in}{1.959209in}}%
\pgfpathlineto{\pgfqpoint{10.146906in}{1.954809in}}%
\pgfpathlineto{\pgfqpoint{10.148919in}{1.964489in}}%
\pgfpathlineto{\pgfqpoint{10.150933in}{1.955513in}}%
\pgfpathlineto{\pgfqpoint{10.156973in}{1.951113in}}%
\pgfpathlineto{\pgfqpoint{10.158986in}{1.955337in}}%
\pgfpathlineto{\pgfqpoint{10.161000in}{1.952697in}}%
\pgfpathlineto{\pgfqpoint{10.163013in}{1.955689in}}%
\pgfpathlineto{\pgfqpoint{10.165027in}{1.956921in}}%
\pgfpathlineto{\pgfqpoint{10.171067in}{1.959385in}}%
\pgfpathlineto{\pgfqpoint{10.173080in}{1.949881in}}%
\pgfpathlineto{\pgfqpoint{10.175094in}{1.951817in}}%
\pgfpathlineto{\pgfqpoint{10.177107in}{1.960793in}}%
\pgfpathlineto{\pgfqpoint{10.179121in}{1.963961in}}%
\pgfpathlineto{\pgfqpoint{10.185161in}{1.960089in}}%
\pgfpathlineto{\pgfqpoint{10.187175in}{1.953401in}}%
\pgfpathlineto{\pgfqpoint{10.189188in}{1.966073in}}%
\pgfpathlineto{\pgfqpoint{10.191201in}{1.988425in}}%
\pgfpathlineto{\pgfqpoint{10.193215in}{2.003561in}}%
\pgfpathlineto{\pgfqpoint{10.199255in}{2.011305in}}%
\pgfpathlineto{\pgfqpoint{10.201269in}{2.018873in}}%
\pgfpathlineto{\pgfqpoint{10.205296in}{2.009017in}}%
\pgfpathlineto{\pgfqpoint{10.207309in}{2.013065in}}%
\pgfpathlineto{\pgfqpoint{10.213349in}{2.011129in}}%
\pgfpathlineto{\pgfqpoint{10.215363in}{2.018169in}}%
\pgfpathlineto{\pgfqpoint{10.217376in}{2.010601in}}%
\pgfpathlineto{\pgfqpoint{10.221403in}{2.010073in}}%
\pgfpathlineto{\pgfqpoint{10.227444in}{2.017993in}}%
\pgfpathlineto{\pgfqpoint{10.229457in}{2.004793in}}%
\pgfpathlineto{\pgfqpoint{10.231470in}{2.011657in}}%
\pgfpathlineto{\pgfqpoint{10.233484in}{2.005673in}}%
\pgfpathlineto{\pgfqpoint{10.235497in}{2.006201in}}%
\pgfpathlineto{\pgfqpoint{10.241538in}{2.002857in}}%
\pgfpathlineto{\pgfqpoint{10.243551in}{2.005497in}}%
\pgfpathlineto{\pgfqpoint{10.245565in}{2.003209in}}%
\pgfpathlineto{\pgfqpoint{10.247578in}{2.007081in}}%
\pgfpathlineto{\pgfqpoint{10.249591in}{2.007609in}}%
\pgfpathlineto{\pgfqpoint{10.255632in}{2.013769in}}%
\pgfpathlineto{\pgfqpoint{10.257645in}{2.013945in}}%
\pgfpathlineto{\pgfqpoint{10.259659in}{2.008841in}}%
\pgfpathlineto{\pgfqpoint{10.261672in}{2.008489in}}%
\pgfpathlineto{\pgfqpoint{10.263686in}{2.006377in}}%
\pgfpathlineto{\pgfqpoint{10.269726in}{2.003561in}}%
\pgfpathlineto{\pgfqpoint{10.271739in}{2.004089in}}%
\pgfpathlineto{\pgfqpoint{10.273753in}{2.002505in}}%
\pgfpathlineto{\pgfqpoint{10.277780in}{1.998281in}}%
\pgfpathlineto{\pgfqpoint{10.283820in}{1.993353in}}%
\pgfpathlineto{\pgfqpoint{10.285834in}{1.997929in}}%
\pgfpathlineto{\pgfqpoint{10.287847in}{1.994937in}}%
\pgfpathlineto{\pgfqpoint{10.289860in}{1.988601in}}%
\pgfpathlineto{\pgfqpoint{10.291874in}{1.996345in}}%
\pgfpathlineto{\pgfqpoint{10.297914in}{1.997577in}}%
\pgfpathlineto{\pgfqpoint{10.303955in}{1.978041in}}%
\pgfpathlineto{\pgfqpoint{10.305968in}{1.974169in}}%
\pgfpathlineto{\pgfqpoint{10.312008in}{1.979801in}}%
\pgfpathlineto{\pgfqpoint{10.314022in}{1.970297in}}%
\pgfpathlineto{\pgfqpoint{10.316035in}{1.982617in}}%
\pgfpathlineto{\pgfqpoint{10.318049in}{1.982089in}}%
\pgfpathlineto{\pgfqpoint{10.320062in}{1.977337in}}%
\pgfpathlineto{\pgfqpoint{10.326102in}{1.984201in}}%
\pgfpathlineto{\pgfqpoint{10.328116in}{1.989481in}}%
\pgfpathlineto{\pgfqpoint{10.332143in}{1.992297in}}%
\pgfpathlineto{\pgfqpoint{10.342210in}{1.991769in}}%
\pgfpathlineto{\pgfqpoint{10.346237in}{1.989833in}}%
\pgfpathlineto{\pgfqpoint{10.348250in}{1.981913in}}%
\pgfpathlineto{\pgfqpoint{10.354291in}{1.985609in}}%
\pgfpathlineto{\pgfqpoint{10.356304in}{1.993881in}}%
\pgfpathlineto{\pgfqpoint{10.358318in}{1.990185in}}%
\pgfpathlineto{\pgfqpoint{10.360331in}{1.976457in}}%
\pgfpathlineto{\pgfqpoint{10.362345in}{1.979977in}}%
\pgfpathlineto{\pgfqpoint{10.368385in}{1.970121in}}%
\pgfpathlineto{\pgfqpoint{10.370398in}{1.970649in}}%
\pgfpathlineto{\pgfqpoint{10.372412in}{1.987017in}}%
\pgfpathlineto{\pgfqpoint{10.374425in}{1.990537in}}%
\pgfpathlineto{\pgfqpoint{10.376439in}{1.988073in}}%
\pgfpathlineto{\pgfqpoint{10.384492in}{1.979273in}}%
\pgfpathlineto{\pgfqpoint{10.386506in}{1.974169in}}%
\pgfpathlineto{\pgfqpoint{10.388519in}{1.981385in}}%
\pgfpathlineto{\pgfqpoint{10.390533in}{1.978217in}}%
\pgfpathlineto{\pgfqpoint{10.396573in}{1.979097in}}%
\pgfpathlineto{\pgfqpoint{10.398587in}{1.974521in}}%
\pgfpathlineto{\pgfqpoint{10.400600in}{1.979097in}}%
\pgfpathlineto{\pgfqpoint{10.402613in}{1.978217in}}%
\pgfpathlineto{\pgfqpoint{10.404627in}{1.983673in}}%
\pgfpathlineto{\pgfqpoint{10.410667in}{1.961497in}}%
\pgfpathlineto{\pgfqpoint{10.412681in}{1.967129in}}%
\pgfpathlineto{\pgfqpoint{10.414694in}{1.965369in}}%
\pgfpathlineto{\pgfqpoint{10.416708in}{1.965193in}}%
\pgfpathlineto{\pgfqpoint{10.418721in}{1.966953in}}%
\pgfpathlineto{\pgfqpoint{10.424761in}{1.968361in}}%
\pgfpathlineto{\pgfqpoint{10.426775in}{1.972585in}}%
\pgfpathlineto{\pgfqpoint{10.428788in}{1.974697in}}%
\pgfpathlineto{\pgfqpoint{10.430802in}{1.973817in}}%
\pgfpathlineto{\pgfqpoint{10.432815in}{1.962905in}}%
\pgfpathlineto{\pgfqpoint{10.440869in}{1.955689in}}%
\pgfpathlineto{\pgfqpoint{10.442882in}{1.963257in}}%
\pgfpathlineto{\pgfqpoint{10.444896in}{1.984905in}}%
\pgfpathlineto{\pgfqpoint{10.446909in}{1.973289in}}%
\pgfpathlineto{\pgfqpoint{10.452950in}{1.959033in}}%
\pgfpathlineto{\pgfqpoint{10.456977in}{1.960793in}}%
\pgfpathlineto{\pgfqpoint{10.458990in}{1.976457in}}%
\pgfpathlineto{\pgfqpoint{10.461003in}{1.977513in}}%
\pgfpathlineto{\pgfqpoint{10.467044in}{1.973817in}}%
\pgfpathlineto{\pgfqpoint{10.469057in}{1.981913in}}%
\pgfpathlineto{\pgfqpoint{10.471071in}{1.974873in}}%
\pgfpathlineto{\pgfqpoint{10.473084in}{1.975929in}}%
\pgfpathlineto{\pgfqpoint{10.475098in}{1.971705in}}%
\pgfpathlineto{\pgfqpoint{10.481138in}{1.970121in}}%
\pgfpathlineto{\pgfqpoint{10.487178in}{1.956393in}}%
\pgfpathlineto{\pgfqpoint{10.489192in}{1.957097in}}%
\pgfpathlineto{\pgfqpoint{10.495232in}{1.960089in}}%
\pgfpathlineto{\pgfqpoint{10.497245in}{1.965545in}}%
\pgfpathlineto{\pgfqpoint{10.499259in}{1.960793in}}%
\pgfpathlineto{\pgfqpoint{10.501272in}{1.972585in}}%
\pgfpathlineto{\pgfqpoint{10.503286in}{1.968361in}}%
\pgfpathlineto{\pgfqpoint{10.509326in}{1.969769in}}%
\pgfpathlineto{\pgfqpoint{10.511340in}{1.972585in}}%
\pgfpathlineto{\pgfqpoint{10.513353in}{1.969769in}}%
\pgfpathlineto{\pgfqpoint{10.515366in}{1.979097in}}%
\pgfpathlineto{\pgfqpoint{10.517380in}{1.975401in}}%
\pgfpathlineto{\pgfqpoint{10.523420in}{1.977513in}}%
\pgfpathlineto{\pgfqpoint{10.525434in}{1.979977in}}%
\pgfpathlineto{\pgfqpoint{10.527447in}{1.980857in}}%
\pgfpathlineto{\pgfqpoint{10.529461in}{1.984905in}}%
\pgfpathlineto{\pgfqpoint{10.531474in}{1.983849in}}%
\pgfpathlineto{\pgfqpoint{10.537514in}{1.984905in}}%
\pgfpathlineto{\pgfqpoint{10.539528in}{1.993529in}}%
\pgfpathlineto{\pgfqpoint{10.541541in}{1.990361in}}%
\pgfpathlineto{\pgfqpoint{10.545568in}{1.977865in}}%
\pgfpathlineto{\pgfqpoint{10.551609in}{1.980505in}}%
\pgfpathlineto{\pgfqpoint{10.553622in}{1.975753in}}%
\pgfpathlineto{\pgfqpoint{10.555635in}{1.978217in}}%
\pgfpathlineto{\pgfqpoint{10.557649in}{1.985081in}}%
\pgfpathlineto{\pgfqpoint{10.565703in}{1.990361in}}%
\pgfpathlineto{\pgfqpoint{10.567716in}{1.989481in}}%
\pgfpathlineto{\pgfqpoint{10.569730in}{1.982793in}}%
\pgfpathlineto{\pgfqpoint{10.571743in}{1.966777in}}%
\pgfpathlineto{\pgfqpoint{10.573756in}{1.962025in}}%
\pgfpathlineto{\pgfqpoint{10.581810in}{1.973641in}}%
\pgfpathlineto{\pgfqpoint{10.583824in}{1.973113in}}%
\pgfpathlineto{\pgfqpoint{10.585837in}{1.979801in}}%
\pgfpathlineto{\pgfqpoint{10.587851in}{1.978393in}}%
\pgfpathlineto{\pgfqpoint{10.595904in}{1.983321in}}%
\pgfpathlineto{\pgfqpoint{10.597918in}{1.992121in}}%
\pgfpathlineto{\pgfqpoint{10.599931in}{1.997401in}}%
\pgfpathlineto{\pgfqpoint{10.601945in}{1.997401in}}%
\pgfpathlineto{\pgfqpoint{10.607985in}{1.992825in}}%
\pgfpathlineto{\pgfqpoint{10.609999in}{1.988601in}}%
\pgfpathlineto{\pgfqpoint{10.612012in}{1.990889in}}%
\pgfpathlineto{\pgfqpoint{10.614025in}{1.990185in}}%
\pgfpathlineto{\pgfqpoint{10.616039in}{2.005497in}}%
\pgfpathlineto{\pgfqpoint{10.622079in}{2.006553in}}%
\pgfpathlineto{\pgfqpoint{10.624093in}{1.999689in}}%
\pgfpathlineto{\pgfqpoint{10.626106in}{2.006905in}}%
\pgfpathlineto{\pgfqpoint{10.630133in}{2.014825in}}%
\pgfpathlineto{\pgfqpoint{10.638187in}{2.013065in}}%
\pgfpathlineto{\pgfqpoint{10.640200in}{2.014825in}}%
\pgfpathlineto{\pgfqpoint{10.642214in}{2.014473in}}%
\pgfpathlineto{\pgfqpoint{10.644227in}{2.017817in}}%
\pgfpathlineto{\pgfqpoint{10.650267in}{2.019049in}}%
\pgfpathlineto{\pgfqpoint{10.652281in}{2.008489in}}%
\pgfpathlineto{\pgfqpoint{10.654294in}{2.006553in}}%
\pgfpathlineto{\pgfqpoint{10.658321in}{2.010425in}}%
\pgfpathlineto{\pgfqpoint{10.664362in}{2.012361in}}%
\pgfpathlineto{\pgfqpoint{10.666375in}{2.011833in}}%
\pgfpathlineto{\pgfqpoint{10.668388in}{2.009721in}}%
\pgfpathlineto{\pgfqpoint{10.670402in}{2.005145in}}%
\pgfpathlineto{\pgfqpoint{10.672415in}{2.007081in}}%
\pgfpathlineto{\pgfqpoint{10.678456in}{2.008841in}}%
\pgfpathlineto{\pgfqpoint{10.680469in}{2.007433in}}%
\pgfpathlineto{\pgfqpoint{10.682483in}{2.010425in}}%
\pgfpathlineto{\pgfqpoint{10.684496in}{2.011129in}}%
\pgfpathlineto{\pgfqpoint{10.686510in}{2.009897in}}%
\pgfpathlineto{\pgfqpoint{10.692550in}{2.014649in}}%
\pgfpathlineto{\pgfqpoint{10.694563in}{2.008489in}}%
\pgfpathlineto{\pgfqpoint{10.696577in}{2.010249in}}%
\pgfpathlineto{\pgfqpoint{10.698590in}{2.007433in}}%
\pgfpathlineto{\pgfqpoint{10.700604in}{2.009017in}}%
\pgfpathlineto{\pgfqpoint{10.706644in}{2.003561in}}%
\pgfpathlineto{\pgfqpoint{10.708657in}{2.009721in}}%
\pgfpathlineto{\pgfqpoint{10.710671in}{2.013769in}}%
\pgfpathlineto{\pgfqpoint{10.712684in}{2.014473in}}%
\pgfpathlineto{\pgfqpoint{10.720738in}{2.015001in}}%
\pgfpathlineto{\pgfqpoint{10.722752in}{2.008489in}}%
\pgfpathlineto{\pgfqpoint{10.724765in}{2.010425in}}%
\pgfpathlineto{\pgfqpoint{10.728792in}{2.030313in}}%
\pgfpathlineto{\pgfqpoint{10.734832in}{2.033305in}}%
\pgfpathlineto{\pgfqpoint{10.736846in}{2.035769in}}%
\pgfpathlineto{\pgfqpoint{10.738859in}{2.039289in}}%
\pgfpathlineto{\pgfqpoint{10.740873in}{2.030313in}}%
\pgfpathlineto{\pgfqpoint{10.742886in}{2.036473in}}%
\pgfpathlineto{\pgfqpoint{10.748926in}{2.035769in}}%
\pgfpathlineto{\pgfqpoint{10.750940in}{2.039641in}}%
\pgfpathlineto{\pgfqpoint{10.752953in}{2.038585in}}%
\pgfpathlineto{\pgfqpoint{10.754967in}{2.040521in}}%
\pgfpathlineto{\pgfqpoint{10.756980in}{2.043865in}}%
\pgfpathlineto{\pgfqpoint{10.763021in}{2.048089in}}%
\pgfpathlineto{\pgfqpoint{10.765034in}{2.053545in}}%
\pgfpathlineto{\pgfqpoint{10.767047in}{2.050553in}}%
\pgfpathlineto{\pgfqpoint{10.769061in}{2.031721in}}%
\pgfpathlineto{\pgfqpoint{10.771074in}{2.023449in}}%
\pgfpathlineto{\pgfqpoint{10.777115in}{2.028905in}}%
\pgfpathlineto{\pgfqpoint{10.783155in}{2.006905in}}%
\pgfpathlineto{\pgfqpoint{10.785168in}{2.007609in}}%
\pgfpathlineto{\pgfqpoint{10.791209in}{2.007257in}}%
\pgfpathlineto{\pgfqpoint{10.793222in}{2.009545in}}%
\pgfpathlineto{\pgfqpoint{10.797249in}{2.011833in}}%
\pgfpathlineto{\pgfqpoint{10.799263in}{2.009017in}}%
\pgfpathlineto{\pgfqpoint{10.805303in}{2.008841in}}%
\pgfpathlineto{\pgfqpoint{10.807316in}{2.007609in}}%
\pgfpathlineto{\pgfqpoint{10.809330in}{2.009369in}}%
\pgfpathlineto{\pgfqpoint{10.811343in}{2.010073in}}%
\pgfpathlineto{\pgfqpoint{10.813357in}{2.006729in}}%
\pgfpathlineto{\pgfqpoint{10.819397in}{2.013945in}}%
\pgfpathlineto{\pgfqpoint{10.821410in}{2.015353in}}%
\pgfpathlineto{\pgfqpoint{10.823424in}{2.017817in}}%
\pgfpathlineto{\pgfqpoint{10.825437in}{2.017641in}}%
\pgfpathlineto{\pgfqpoint{10.827451in}{2.023273in}}%
\pgfpathlineto{\pgfqpoint{10.835505in}{2.022569in}}%
\pgfpathlineto{\pgfqpoint{10.837518in}{2.024153in}}%
\pgfpathlineto{\pgfqpoint{10.839532in}{2.021865in}}%
\pgfpathlineto{\pgfqpoint{10.841545in}{2.025033in}}%
\pgfpathlineto{\pgfqpoint{10.847585in}{2.019225in}}%
\pgfpathlineto{\pgfqpoint{10.849599in}{2.010425in}}%
\pgfpathlineto{\pgfqpoint{10.851612in}{2.008313in}}%
\pgfpathlineto{\pgfqpoint{10.853626in}{2.012009in}}%
\pgfpathlineto{\pgfqpoint{10.855639in}{2.003385in}}%
\pgfpathlineto{\pgfqpoint{10.861679in}{2.005673in}}%
\pgfpathlineto{\pgfqpoint{10.863693in}{2.011305in}}%
\pgfpathlineto{\pgfqpoint{10.865706in}{2.014825in}}%
\pgfpathlineto{\pgfqpoint{10.867720in}{2.021513in}}%
\pgfpathlineto{\pgfqpoint{10.869733in}{2.030841in}}%
\pgfpathlineto{\pgfqpoint{10.875774in}{2.028377in}}%
\pgfpathlineto{\pgfqpoint{10.877787in}{2.024681in}}%
\pgfpathlineto{\pgfqpoint{10.879800in}{2.026793in}}%
\pgfpathlineto{\pgfqpoint{10.881814in}{2.020457in}}%
\pgfpathlineto{\pgfqpoint{10.883827in}{2.022569in}}%
\pgfpathlineto{\pgfqpoint{10.889868in}{2.022393in}}%
\pgfpathlineto{\pgfqpoint{10.891881in}{2.025737in}}%
\pgfpathlineto{\pgfqpoint{10.893895in}{2.017817in}}%
\pgfpathlineto{\pgfqpoint{10.895908in}{2.015881in}}%
\pgfpathlineto{\pgfqpoint{10.897921in}{2.021865in}}%
\pgfpathlineto{\pgfqpoint{10.903962in}{2.026969in}}%
\pgfpathlineto{\pgfqpoint{10.905975in}{2.021337in}}%
\pgfpathlineto{\pgfqpoint{10.907989in}{2.031897in}}%
\pgfpathlineto{\pgfqpoint{10.910002in}{2.018521in}}%
\pgfpathlineto{\pgfqpoint{10.912016in}{2.018697in}}%
\pgfpathlineto{\pgfqpoint{10.920069in}{2.008489in}}%
\pgfpathlineto{\pgfqpoint{10.924096in}{2.000745in}}%
\pgfpathlineto{\pgfqpoint{10.926110in}{2.006377in}}%
\pgfpathlineto{\pgfqpoint{10.932150in}{2.012185in}}%
\pgfpathlineto{\pgfqpoint{10.934164in}{2.015177in}}%
\pgfpathlineto{\pgfqpoint{10.936177in}{2.009721in}}%
\pgfpathlineto{\pgfqpoint{10.938190in}{2.008489in}}%
\pgfpathlineto{\pgfqpoint{10.940204in}{2.016233in}}%
\pgfpathlineto{\pgfqpoint{10.946244in}{2.025561in}}%
\pgfpathlineto{\pgfqpoint{10.948258in}{2.033129in}}%
\pgfpathlineto{\pgfqpoint{10.950271in}{2.031369in}}%
\pgfpathlineto{\pgfqpoint{10.952285in}{2.032249in}}%
\pgfpathlineto{\pgfqpoint{10.954298in}{2.037177in}}%
\pgfpathlineto{\pgfqpoint{10.960338in}{2.039289in}}%
\pgfpathlineto{\pgfqpoint{10.962352in}{2.038409in}}%
\pgfpathlineto{\pgfqpoint{10.964365in}{2.038585in}}%
\pgfpathlineto{\pgfqpoint{10.966379in}{2.037705in}}%
\pgfpathlineto{\pgfqpoint{10.968392in}{2.045977in}}%
\pgfpathlineto{\pgfqpoint{10.974432in}{2.043865in}}%
\pgfpathlineto{\pgfqpoint{10.976446in}{2.042281in}}%
\pgfpathlineto{\pgfqpoint{10.978459in}{2.045273in}}%
\pgfpathlineto{\pgfqpoint{10.982486in}{2.054953in}}%
\pgfpathlineto{\pgfqpoint{10.988527in}{2.053545in}}%
\pgfpathlineto{\pgfqpoint{10.990540in}{2.051081in}}%
\pgfpathlineto{\pgfqpoint{10.992553in}{2.040697in}}%
\pgfpathlineto{\pgfqpoint{10.994567in}{2.036473in}}%
\pgfpathlineto{\pgfqpoint{10.996580in}{2.036649in}}%
\pgfpathlineto{\pgfqpoint{11.004634in}{2.024329in}}%
\pgfpathlineto{\pgfqpoint{11.006648in}{2.034361in}}%
\pgfpathlineto{\pgfqpoint{11.010675in}{2.042105in}}%
\pgfpathlineto{\pgfqpoint{11.016715in}{2.034185in}}%
\pgfpathlineto{\pgfqpoint{11.018728in}{2.021161in}}%
\pgfpathlineto{\pgfqpoint{11.020742in}{2.016585in}}%
\pgfpathlineto{\pgfqpoint{11.022755in}{2.016409in}}%
\pgfpathlineto{\pgfqpoint{11.024769in}{2.013945in}}%
\pgfpathlineto{\pgfqpoint{11.030809in}{2.018169in}}%
\pgfpathlineto{\pgfqpoint{11.032822in}{1.990009in}}%
\pgfpathlineto{\pgfqpoint{11.034836in}{1.979625in}}%
\pgfpathlineto{\pgfqpoint{11.036849in}{1.982089in}}%
\pgfpathlineto{\pgfqpoint{11.038863in}{1.971001in}}%
\pgfpathlineto{\pgfqpoint{11.044903in}{1.968889in}}%
\pgfpathlineto{\pgfqpoint{11.046917in}{1.970297in}}%
\pgfpathlineto{\pgfqpoint{11.048930in}{1.983321in}}%
\pgfpathlineto{\pgfqpoint{11.050943in}{1.991945in}}%
\pgfpathlineto{\pgfqpoint{11.052957in}{1.991417in}}%
\pgfpathlineto{\pgfqpoint{11.061011in}{2.000745in}}%
\pgfpathlineto{\pgfqpoint{11.063024in}{2.000745in}}%
\pgfpathlineto{\pgfqpoint{11.067051in}{2.003385in}}%
\pgfpathlineto{\pgfqpoint{11.073091in}{1.998985in}}%
\pgfpathlineto{\pgfqpoint{11.075105in}{1.995817in}}%
\pgfpathlineto{\pgfqpoint{11.077118in}{1.988073in}}%
\pgfpathlineto{\pgfqpoint{11.081145in}{1.990713in}}%
\pgfpathlineto{\pgfqpoint{11.087186in}{1.985081in}}%
\pgfpathlineto{\pgfqpoint{11.089199in}{1.991769in}}%
\pgfpathlineto{\pgfqpoint{11.091212in}{1.987545in}}%
\pgfpathlineto{\pgfqpoint{11.093226in}{2.001625in}}%
\pgfpathlineto{\pgfqpoint{11.095239in}{1.995641in}}%
\pgfpathlineto{\pgfqpoint{11.103293in}{2.001625in}}%
\pgfpathlineto{\pgfqpoint{11.105307in}{1.998633in}}%
\pgfpathlineto{\pgfqpoint{11.107320in}{2.000569in}}%
\pgfpathlineto{\pgfqpoint{11.109333in}{2.013241in}}%
\pgfpathlineto{\pgfqpoint{11.119401in}{2.016937in}}%
\pgfpathlineto{\pgfqpoint{11.121414in}{2.010601in}}%
\pgfpathlineto{\pgfqpoint{11.123428in}{2.001273in}}%
\pgfpathlineto{\pgfqpoint{11.129468in}{1.998457in}}%
\pgfpathlineto{\pgfqpoint{11.131481in}{1.989833in}}%
\pgfpathlineto{\pgfqpoint{11.133495in}{1.985433in}}%
\pgfpathlineto{\pgfqpoint{11.135508in}{1.986313in}}%
\pgfpathlineto{\pgfqpoint{11.137522in}{1.980505in}}%
\pgfpathlineto{\pgfqpoint{11.143562in}{1.998809in}}%
\pgfpathlineto{\pgfqpoint{11.145575in}{2.010953in}}%
\pgfpathlineto{\pgfqpoint{11.147589in}{2.010601in}}%
\pgfpathlineto{\pgfqpoint{11.149602in}{2.011481in}}%
\pgfpathlineto{\pgfqpoint{11.151616in}{2.032425in}}%
\pgfpathlineto{\pgfqpoint{11.157656in}{2.029081in}}%
\pgfpathlineto{\pgfqpoint{11.161683in}{2.039465in}}%
\pgfpathlineto{\pgfqpoint{11.165710in}{2.032249in}}%
\pgfpathlineto{\pgfqpoint{11.173764in}{2.030313in}}%
\pgfpathlineto{\pgfqpoint{11.175777in}{2.026969in}}%
\pgfpathlineto{\pgfqpoint{11.177791in}{2.026441in}}%
\pgfpathlineto{\pgfqpoint{11.179804in}{2.027321in}}%
\pgfpathlineto{\pgfqpoint{11.185844in}{2.024505in}}%
\pgfpathlineto{\pgfqpoint{11.187858in}{2.030841in}}%
\pgfpathlineto{\pgfqpoint{11.189871in}{2.030665in}}%
\pgfpathlineto{\pgfqpoint{11.191885in}{2.033129in}}%
\pgfpathlineto{\pgfqpoint{11.193898in}{2.034185in}}%
\pgfpathlineto{\pgfqpoint{11.199939in}{2.034537in}}%
\pgfpathlineto{\pgfqpoint{11.201952in}{2.035593in}}%
\pgfpathlineto{\pgfqpoint{11.203965in}{2.028729in}}%
\pgfpathlineto{\pgfqpoint{11.205979in}{2.026617in}}%
\pgfpathlineto{\pgfqpoint{11.207992in}{2.017641in}}%
\pgfpathlineto{\pgfqpoint{11.214033in}{2.016761in}}%
\pgfpathlineto{\pgfqpoint{11.216046in}{2.006377in}}%
\pgfpathlineto{\pgfqpoint{11.218060in}{2.008841in}}%
\pgfpathlineto{\pgfqpoint{11.220073in}{2.023977in}}%
\pgfpathlineto{\pgfqpoint{11.222086in}{2.025385in}}%
\pgfpathlineto{\pgfqpoint{11.228127in}{2.031721in}}%
\pgfpathlineto{\pgfqpoint{11.230140in}{2.026969in}}%
\pgfpathlineto{\pgfqpoint{11.232154in}{2.035593in}}%
\pgfpathlineto{\pgfqpoint{11.234167in}{2.032073in}}%
\pgfpathlineto{\pgfqpoint{11.236181in}{2.035769in}}%
\pgfpathlineto{\pgfqpoint{11.242221in}{2.037001in}}%
\pgfpathlineto{\pgfqpoint{11.244234in}{2.033657in}}%
\pgfpathlineto{\pgfqpoint{11.246248in}{2.023977in}}%
\pgfpathlineto{\pgfqpoint{11.248261in}{2.019401in}}%
\pgfpathlineto{\pgfqpoint{11.250275in}{2.021513in}}%
\pgfpathlineto{\pgfqpoint{11.256315in}{2.029433in}}%
\pgfpathlineto{\pgfqpoint{11.258329in}{2.022393in}}%
\pgfpathlineto{\pgfqpoint{11.260342in}{2.026617in}}%
\pgfpathlineto{\pgfqpoint{11.262355in}{2.034537in}}%
\pgfpathlineto{\pgfqpoint{11.270409in}{2.036825in}}%
\pgfpathlineto{\pgfqpoint{11.272423in}{2.031721in}}%
\pgfpathlineto{\pgfqpoint{11.274436in}{2.037529in}}%
\pgfpathlineto{\pgfqpoint{11.276450in}{2.035769in}}%
\pgfpathlineto{\pgfqpoint{11.278463in}{2.038937in}}%
\pgfpathlineto{\pgfqpoint{11.284503in}{2.036297in}}%
\pgfpathlineto{\pgfqpoint{11.286517in}{2.038233in}}%
\pgfpathlineto{\pgfqpoint{11.288530in}{2.041401in}}%
\pgfpathlineto{\pgfqpoint{11.290544in}{2.039641in}}%
\pgfpathlineto{\pgfqpoint{11.292557in}{2.034185in}}%
\pgfpathlineto{\pgfqpoint{11.298597in}{2.041225in}}%
\pgfpathlineto{\pgfqpoint{11.300611in}{2.038233in}}%
\pgfpathlineto{\pgfqpoint{11.304638in}{2.050905in}}%
\pgfpathlineto{\pgfqpoint{11.306651in}{2.050553in}}%
\pgfpathlineto{\pgfqpoint{11.312692in}{2.051433in}}%
\pgfpathlineto{\pgfqpoint{11.314705in}{2.058121in}}%
\pgfpathlineto{\pgfqpoint{11.318732in}{2.056537in}}%
\pgfpathlineto{\pgfqpoint{11.320745in}{2.056185in}}%
\pgfpathlineto{\pgfqpoint{11.326786in}{2.057769in}}%
\pgfpathlineto{\pgfqpoint{11.330813in}{2.044921in}}%
\pgfpathlineto{\pgfqpoint{11.332826in}{2.046329in}}%
\pgfpathlineto{\pgfqpoint{11.334840in}{2.052313in}}%
\pgfpathlineto{\pgfqpoint{11.340880in}{2.047385in}}%
\pgfpathlineto{\pgfqpoint{11.342893in}{2.044745in}}%
\pgfpathlineto{\pgfqpoint{11.344907in}{2.046329in}}%
\pgfpathlineto{\pgfqpoint{11.346920in}{2.049849in}}%
\pgfpathlineto{\pgfqpoint{11.348934in}{2.047209in}}%
\pgfpathlineto{\pgfqpoint{11.356987in}{2.043689in}}%
\pgfpathlineto{\pgfqpoint{11.359001in}{2.045625in}}%
\pgfpathlineto{\pgfqpoint{11.361014in}{2.048793in}}%
\pgfpathlineto{\pgfqpoint{11.363028in}{2.044569in}}%
\pgfpathlineto{\pgfqpoint{11.371082in}{2.041753in}}%
\pgfpathlineto{\pgfqpoint{11.373095in}{2.044041in}}%
\pgfpathlineto{\pgfqpoint{11.375108in}{2.043513in}}%
\pgfpathlineto{\pgfqpoint{11.377122in}{2.042105in}}%
\pgfpathlineto{\pgfqpoint{11.387189in}{2.036825in}}%
\pgfpathlineto{\pgfqpoint{11.391216in}{2.010073in}}%
\pgfpathlineto{\pgfqpoint{11.397256in}{2.013065in}}%
\pgfpathlineto{\pgfqpoint{11.399270in}{2.011657in}}%
\pgfpathlineto{\pgfqpoint{11.401283in}{2.013593in}}%
\pgfpathlineto{\pgfqpoint{11.403297in}{2.017641in}}%
\pgfpathlineto{\pgfqpoint{11.405310in}{2.010249in}}%
\pgfpathlineto{\pgfqpoint{11.411351in}{2.006729in}}%
\pgfpathlineto{\pgfqpoint{11.413364in}{2.012713in}}%
\pgfpathlineto{\pgfqpoint{11.415377in}{2.010601in}}%
\pgfpathlineto{\pgfqpoint{11.417391in}{2.017817in}}%
\pgfpathlineto{\pgfqpoint{11.419404in}{2.013417in}}%
\pgfpathlineto{\pgfqpoint{11.425445in}{2.014297in}}%
\pgfpathlineto{\pgfqpoint{11.427458in}{2.017817in}}%
\pgfpathlineto{\pgfqpoint{11.429472in}{2.010953in}}%
\pgfpathlineto{\pgfqpoint{11.433498in}{2.015705in}}%
\pgfpathlineto{\pgfqpoint{11.439539in}{2.003385in}}%
\pgfpathlineto{\pgfqpoint{11.441552in}{2.001097in}}%
\pgfpathlineto{\pgfqpoint{11.443566in}{2.006729in}}%
\pgfpathlineto{\pgfqpoint{11.445579in}{2.009545in}}%
\pgfpathlineto{\pgfqpoint{11.453633in}{2.007257in}}%
\pgfpathlineto{\pgfqpoint{11.455646in}{2.010249in}}%
\pgfpathlineto{\pgfqpoint{11.457660in}{2.008489in}}%
\pgfpathlineto{\pgfqpoint{11.459673in}{2.003913in}}%
\pgfpathlineto{\pgfqpoint{11.461687in}{2.014649in}}%
\pgfpathlineto{\pgfqpoint{11.467727in}{2.017641in}}%
\pgfpathlineto{\pgfqpoint{11.469740in}{2.020809in}}%
\pgfpathlineto{\pgfqpoint{11.471754in}{2.019577in}}%
\pgfpathlineto{\pgfqpoint{11.473767in}{2.026969in}}%
\pgfpathlineto{\pgfqpoint{11.475781in}{2.023449in}}%
\pgfpathlineto{\pgfqpoint{11.481821in}{2.030841in}}%
\pgfpathlineto{\pgfqpoint{11.483835in}{2.014297in}}%
\pgfpathlineto{\pgfqpoint{11.485848in}{2.006729in}}%
\pgfpathlineto{\pgfqpoint{11.487862in}{2.005145in}}%
\pgfpathlineto{\pgfqpoint{11.489875in}{2.000569in}}%
\pgfpathlineto{\pgfqpoint{11.495915in}{1.997577in}}%
\pgfpathlineto{\pgfqpoint{11.497929in}{1.998457in}}%
\pgfpathlineto{\pgfqpoint{11.499942in}{2.008313in}}%
\pgfpathlineto{\pgfqpoint{11.503969in}{2.011657in}}%
\pgfpathlineto{\pgfqpoint{11.510009in}{2.014297in}}%
\pgfpathlineto{\pgfqpoint{11.512023in}{2.009897in}}%
\pgfpathlineto{\pgfqpoint{11.516050in}{2.009369in}}%
\pgfpathlineto{\pgfqpoint{11.518063in}{2.005321in}}%
\pgfpathlineto{\pgfqpoint{11.528130in}{2.027145in}}%
\pgfpathlineto{\pgfqpoint{11.532157in}{2.021865in}}%
\pgfpathlineto{\pgfqpoint{11.538198in}{2.022393in}}%
\pgfpathlineto{\pgfqpoint{11.542225in}{2.021161in}}%
\pgfpathlineto{\pgfqpoint{11.544238in}{2.012889in}}%
\pgfpathlineto{\pgfqpoint{11.546251in}{2.001449in}}%
\pgfpathlineto{\pgfqpoint{11.552292in}{1.981561in}}%
\pgfpathlineto{\pgfqpoint{11.554305in}{1.963433in}}%
\pgfpathlineto{\pgfqpoint{11.556319in}{1.987721in}}%
\pgfpathlineto{\pgfqpoint{11.558332in}{2.002857in}}%
\pgfpathlineto{\pgfqpoint{11.560346in}{2.001097in}}%
\pgfpathlineto{\pgfqpoint{11.566386in}{2.000217in}}%
\pgfpathlineto{\pgfqpoint{11.568399in}{1.983849in}}%
\pgfpathlineto{\pgfqpoint{11.572426in}{1.995993in}}%
\pgfpathlineto{\pgfqpoint{11.574440in}{1.982793in}}%
\pgfpathlineto{\pgfqpoint{11.582494in}{1.998281in}}%
\pgfpathlineto{\pgfqpoint{11.584507in}{1.991065in}}%
\pgfpathlineto{\pgfqpoint{11.586520in}{1.992121in}}%
\pgfpathlineto{\pgfqpoint{11.588534in}{1.996169in}}%
\pgfpathlineto{\pgfqpoint{11.594574in}{1.994937in}}%
\pgfpathlineto{\pgfqpoint{11.596588in}{2.005497in}}%
\pgfpathlineto{\pgfqpoint{11.598601in}{2.002857in}}%
\pgfpathlineto{\pgfqpoint{11.600615in}{1.988777in}}%
\pgfpathlineto{\pgfqpoint{11.602628in}{1.979097in}}%
\pgfpathlineto{\pgfqpoint{11.608668in}{1.982441in}}%
\pgfpathlineto{\pgfqpoint{11.612695in}{1.970473in}}%
\pgfpathlineto{\pgfqpoint{11.616722in}{1.973993in}}%
\pgfpathlineto{\pgfqpoint{11.624776in}{1.963785in}}%
\pgfpathlineto{\pgfqpoint{11.626789in}{1.963433in}}%
\pgfpathlineto{\pgfqpoint{11.628803in}{1.955513in}}%
\pgfpathlineto{\pgfqpoint{11.630816in}{1.953753in}}%
\pgfpathlineto{\pgfqpoint{11.636857in}{1.970649in}}%
\pgfpathlineto{\pgfqpoint{11.638870in}{1.971529in}}%
\pgfpathlineto{\pgfqpoint{11.642897in}{1.982441in}}%
\pgfpathlineto{\pgfqpoint{11.644910in}{1.981385in}}%
\pgfpathlineto{\pgfqpoint{11.652964in}{1.984377in}}%
\pgfpathlineto{\pgfqpoint{11.654978in}{1.978921in}}%
\pgfpathlineto{\pgfqpoint{11.656991in}{1.989129in}}%
\pgfpathlineto{\pgfqpoint{11.659005in}{1.989481in}}%
\pgfpathlineto{\pgfqpoint{11.665045in}{1.989481in}}%
\pgfpathlineto{\pgfqpoint{11.667058in}{1.997577in}}%
\pgfpathlineto{\pgfqpoint{11.669072in}{1.991945in}}%
\pgfpathlineto{\pgfqpoint{11.671085in}{2.007257in}}%
\pgfpathlineto{\pgfqpoint{11.673099in}{2.011129in}}%
\pgfpathlineto{\pgfqpoint{11.679139in}{2.014121in}}%
\pgfpathlineto{\pgfqpoint{11.681152in}{2.011305in}}%
\pgfpathlineto{\pgfqpoint{11.683166in}{2.015881in}}%
\pgfpathlineto{\pgfqpoint{11.685179in}{2.014825in}}%
\pgfpathlineto{\pgfqpoint{11.687193in}{2.021865in}}%
\pgfpathlineto{\pgfqpoint{11.693233in}{2.020457in}}%
\pgfpathlineto{\pgfqpoint{11.697260in}{2.010953in}}%
\pgfpathlineto{\pgfqpoint{11.699273in}{2.011833in}}%
\pgfpathlineto{\pgfqpoint{11.701287in}{2.005497in}}%
\pgfpathlineto{\pgfqpoint{11.709341in}{1.995465in}}%
\pgfpathlineto{\pgfqpoint{11.711354in}{1.998633in}}%
\pgfpathlineto{\pgfqpoint{11.715381in}{1.982441in}}%
\pgfpathlineto{\pgfqpoint{11.721421in}{1.994585in}}%
\pgfpathlineto{\pgfqpoint{11.723435in}{1.995113in}}%
\pgfpathlineto{\pgfqpoint{11.725448in}{1.999513in}}%
\pgfpathlineto{\pgfqpoint{11.727462in}{2.005321in}}%
\pgfpathlineto{\pgfqpoint{11.729475in}{1.999689in}}%
\pgfpathlineto{\pgfqpoint{11.735516in}{1.993881in}}%
\pgfpathlineto{\pgfqpoint{11.737529in}{1.996697in}}%
\pgfpathlineto{\pgfqpoint{11.739542in}{1.992825in}}%
\pgfpathlineto{\pgfqpoint{11.743569in}{1.997401in}}%
\pgfpathlineto{\pgfqpoint{11.749610in}{2.000569in}}%
\pgfpathlineto{\pgfqpoint{11.751623in}{2.002505in}}%
\pgfpathlineto{\pgfqpoint{11.753637in}{1.992825in}}%
\pgfpathlineto{\pgfqpoint{11.755650in}{1.987369in}}%
\pgfpathlineto{\pgfqpoint{11.757663in}{2.004441in}}%
\pgfpathlineto{\pgfqpoint{11.763704in}{2.009721in}}%
\pgfpathlineto{\pgfqpoint{11.767731in}{1.999337in}}%
\pgfpathlineto{\pgfqpoint{11.769744in}{1.998633in}}%
\pgfpathlineto{\pgfqpoint{11.771758in}{1.991241in}}%
\pgfpathlineto{\pgfqpoint{11.777798in}{2.000569in}}%
\pgfpathlineto{\pgfqpoint{11.779811in}{2.002153in}}%
\pgfpathlineto{\pgfqpoint{11.781825in}{2.016409in}}%
\pgfpathlineto{\pgfqpoint{11.785852in}{2.002329in}}%
\pgfpathlineto{\pgfqpoint{11.791892in}{2.007257in}}%
\pgfpathlineto{\pgfqpoint{11.793905in}{2.013945in}}%
\pgfpathlineto{\pgfqpoint{11.795919in}{2.022921in}}%
\pgfpathlineto{\pgfqpoint{11.797932in}{2.019401in}}%
\pgfpathlineto{\pgfqpoint{11.805986in}{2.019929in}}%
\pgfpathlineto{\pgfqpoint{11.808000in}{2.026793in}}%
\pgfpathlineto{\pgfqpoint{11.812027in}{2.012009in}}%
\pgfpathlineto{\pgfqpoint{11.820080in}{2.006905in}}%
\pgfpathlineto{\pgfqpoint{11.822094in}{2.016233in}}%
\pgfpathlineto{\pgfqpoint{11.824107in}{2.010073in}}%
\pgfpathlineto{\pgfqpoint{11.826121in}{2.006201in}}%
\pgfpathlineto{\pgfqpoint{11.828134in}{1.999689in}}%
\pgfpathlineto{\pgfqpoint{11.834174in}{2.003561in}}%
\pgfpathlineto{\pgfqpoint{11.836188in}{2.001097in}}%
\pgfpathlineto{\pgfqpoint{11.838201in}{1.989305in}}%
\pgfpathlineto{\pgfqpoint{11.840215in}{2.000217in}}%
\pgfpathlineto{\pgfqpoint{11.842228in}{1.993529in}}%
\pgfpathlineto{\pgfqpoint{11.850282in}{2.000217in}}%
\pgfpathlineto{\pgfqpoint{11.852295in}{1.993529in}}%
\pgfpathlineto{\pgfqpoint{11.856322in}{2.032953in}}%
\pgfpathlineto{\pgfqpoint{11.862363in}{2.032777in}}%
\pgfpathlineto{\pgfqpoint{11.864376in}{2.051081in}}%
\pgfpathlineto{\pgfqpoint{11.866390in}{2.062873in}}%
\pgfpathlineto{\pgfqpoint{11.868403in}{2.062521in}}%
\pgfpathlineto{\pgfqpoint{11.870416in}{2.076953in}}%
\pgfpathlineto{\pgfqpoint{11.876457in}{2.088921in}}%
\pgfpathlineto{\pgfqpoint{11.878470in}{2.076073in}}%
\pgfpathlineto{\pgfqpoint{11.880484in}{2.086809in}}%
\pgfpathlineto{\pgfqpoint{11.882497in}{2.083817in}}%
\pgfpathlineto{\pgfqpoint{11.884511in}{2.092089in}}%
\pgfpathlineto{\pgfqpoint{11.890551in}{2.088569in}}%
\pgfpathlineto{\pgfqpoint{11.892564in}{2.079593in}}%
\pgfpathlineto{\pgfqpoint{11.894578in}{2.077129in}}%
\pgfpathlineto{\pgfqpoint{11.896591in}{2.068153in}}%
\pgfpathlineto{\pgfqpoint{11.898605in}{2.079065in}}%
\pgfpathlineto{\pgfqpoint{11.906659in}{2.081001in}}%
\pgfpathlineto{\pgfqpoint{11.908672in}{2.082233in}}%
\pgfpathlineto{\pgfqpoint{11.910685in}{2.091561in}}%
\pgfpathlineto{\pgfqpoint{11.912699in}{2.090329in}}%
\pgfpathlineto{\pgfqpoint{11.918739in}{2.093497in}}%
\pgfpathlineto{\pgfqpoint{11.920753in}{2.086809in}}%
\pgfpathlineto{\pgfqpoint{11.922766in}{2.089801in}}%
\pgfpathlineto{\pgfqpoint{11.924780in}{2.094025in}}%
\pgfpathlineto{\pgfqpoint{11.926793in}{2.092793in}}%
\pgfpathlineto{\pgfqpoint{11.932833in}{2.088393in}}%
\pgfpathlineto{\pgfqpoint{11.936860in}{2.109337in}}%
\pgfpathlineto{\pgfqpoint{11.938874in}{2.105817in}}%
\pgfpathlineto{\pgfqpoint{11.940887in}{2.104585in}}%
\pgfpathlineto{\pgfqpoint{11.946927in}{2.110745in}}%
\pgfpathlineto{\pgfqpoint{11.948941in}{2.114441in}}%
\pgfpathlineto{\pgfqpoint{11.950954in}{2.112681in}}%
\pgfpathlineto{\pgfqpoint{11.952968in}{2.112329in}}%
\pgfpathlineto{\pgfqpoint{11.954981in}{2.115497in}}%
\pgfpathlineto{\pgfqpoint{11.961022in}{2.115673in}}%
\pgfpathlineto{\pgfqpoint{11.963035in}{2.117609in}}%
\pgfpathlineto{\pgfqpoint{11.967062in}{2.132041in}}%
\pgfpathlineto{\pgfqpoint{11.969075in}{2.126233in}}%
\pgfpathlineto{\pgfqpoint{11.975116in}{2.129225in}}%
\pgfpathlineto{\pgfqpoint{11.979143in}{2.121129in}}%
\pgfpathlineto{\pgfqpoint{11.981156in}{2.130985in}}%
\pgfpathlineto{\pgfqpoint{11.989210in}{2.128521in}}%
\pgfpathlineto{\pgfqpoint{11.991223in}{2.138377in}}%
\pgfpathlineto{\pgfqpoint{11.993237in}{2.138201in}}%
\pgfpathlineto{\pgfqpoint{11.995250in}{2.138729in}}%
\pgfpathlineto{\pgfqpoint{11.997264in}{2.137673in}}%
\pgfpathlineto{\pgfqpoint{12.003304in}{2.143833in}}%
\pgfpathlineto{\pgfqpoint{12.005317in}{2.138905in}}%
\pgfpathlineto{\pgfqpoint{12.007331in}{2.138905in}}%
\pgfpathlineto{\pgfqpoint{12.009344in}{2.115849in}}%
\pgfpathlineto{\pgfqpoint{12.011358in}{2.118489in}}%
\pgfpathlineto{\pgfqpoint{12.017398in}{2.109865in}}%
\pgfpathlineto{\pgfqpoint{12.019412in}{2.114969in}}%
\pgfpathlineto{\pgfqpoint{12.021425in}{2.104937in}}%
\pgfpathlineto{\pgfqpoint{12.023438in}{2.105993in}}%
\pgfpathlineto{\pgfqpoint{12.025452in}{2.105817in}}%
\pgfpathlineto{\pgfqpoint{12.031492in}{2.111625in}}%
\pgfpathlineto{\pgfqpoint{12.033506in}{2.116905in}}%
\pgfpathlineto{\pgfqpoint{12.035519in}{2.111977in}}%
\pgfpathlineto{\pgfqpoint{12.037533in}{2.085753in}}%
\pgfpathlineto{\pgfqpoint{12.039546in}{2.093673in}}%
\pgfpathlineto{\pgfqpoint{12.045586in}{2.096841in}}%
\pgfpathlineto{\pgfqpoint{12.047600in}{2.092089in}}%
\pgfpathlineto{\pgfqpoint{12.049613in}{2.111097in}}%
\pgfpathlineto{\pgfqpoint{12.051627in}{2.100889in}}%
\pgfpathlineto{\pgfqpoint{12.053640in}{2.099657in}}%
\pgfpathlineto{\pgfqpoint{12.059681in}{2.105465in}}%
\pgfpathlineto{\pgfqpoint{12.061694in}{2.095609in}}%
\pgfpathlineto{\pgfqpoint{12.063707in}{2.098073in}}%
\pgfpathlineto{\pgfqpoint{12.065721in}{2.098073in}}%
\pgfpathlineto{\pgfqpoint{12.067734in}{2.102297in}}%
\pgfpathlineto{\pgfqpoint{12.073775in}{2.101769in}}%
\pgfpathlineto{\pgfqpoint{12.075788in}{2.108809in}}%
\pgfpathlineto{\pgfqpoint{12.077802in}{2.102825in}}%
\pgfpathlineto{\pgfqpoint{12.079815in}{2.107753in}}%
\pgfpathlineto{\pgfqpoint{12.081828in}{2.099657in}}%
\pgfpathlineto{\pgfqpoint{12.087869in}{2.104057in}}%
\pgfpathlineto{\pgfqpoint{12.089882in}{2.098953in}}%
\pgfpathlineto{\pgfqpoint{12.091896in}{2.091209in}}%
\pgfpathlineto{\pgfqpoint{12.093909in}{2.079769in}}%
\pgfpathlineto{\pgfqpoint{12.095923in}{2.080121in}}%
\pgfpathlineto{\pgfqpoint{12.101963in}{2.072201in}}%
\pgfpathlineto{\pgfqpoint{12.103976in}{2.078889in}}%
\pgfpathlineto{\pgfqpoint{12.108003in}{2.087865in}}%
\pgfpathlineto{\pgfqpoint{12.110017in}{2.094729in}}%
\pgfpathlineto{\pgfqpoint{12.118070in}{2.098953in}}%
\pgfpathlineto{\pgfqpoint{12.120084in}{2.091913in}}%
\pgfpathlineto{\pgfqpoint{12.122097in}{2.097017in}}%
\pgfpathlineto{\pgfqpoint{12.124111in}{2.099305in}}%
\pgfpathlineto{\pgfqpoint{12.130151in}{2.096137in}}%
\pgfpathlineto{\pgfqpoint{12.132165in}{2.111977in}}%
\pgfpathlineto{\pgfqpoint{12.134178in}{2.108457in}}%
\pgfpathlineto{\pgfqpoint{12.136192in}{2.114969in}}%
\pgfpathlineto{\pgfqpoint{12.138205in}{2.125881in}}%
\pgfpathlineto{\pgfqpoint{12.144245in}{2.124473in}}%
\pgfpathlineto{\pgfqpoint{12.146259in}{2.130809in}}%
\pgfpathlineto{\pgfqpoint{12.148272in}{2.128521in}}%
\pgfpathlineto{\pgfqpoint{12.150286in}{2.138025in}}%
\pgfpathlineto{\pgfqpoint{12.152299in}{2.142777in}}%
\pgfpathlineto{\pgfqpoint{12.158339in}{2.142601in}}%
\pgfpathlineto{\pgfqpoint{12.160353in}{2.147705in}}%
\pgfpathlineto{\pgfqpoint{12.162366in}{2.146649in}}%
\pgfpathlineto{\pgfqpoint{12.164380in}{2.156329in}}%
\pgfpathlineto{\pgfqpoint{12.166393in}{2.152633in}}%
\pgfpathlineto{\pgfqpoint{12.174447in}{2.158617in}}%
\pgfpathlineto{\pgfqpoint{12.176460in}{2.162313in}}%
\pgfpathlineto{\pgfqpoint{12.178474in}{2.174105in}}%
\pgfpathlineto{\pgfqpoint{12.180487in}{2.180089in}}%
\pgfpathlineto{\pgfqpoint{12.188541in}{2.184665in}}%
\pgfpathlineto{\pgfqpoint{12.190555in}{2.189241in}}%
\pgfpathlineto{\pgfqpoint{12.192568in}{2.175689in}}%
\pgfpathlineto{\pgfqpoint{12.194581in}{2.183609in}}%
\pgfpathlineto{\pgfqpoint{12.200622in}{2.184137in}}%
\pgfpathlineto{\pgfqpoint{12.202635in}{2.177097in}}%
\pgfpathlineto{\pgfqpoint{12.204649in}{2.185193in}}%
\pgfpathlineto{\pgfqpoint{12.206662in}{2.182729in}}%
\pgfpathlineto{\pgfqpoint{12.208676in}{2.182729in}}%
\pgfpathlineto{\pgfqpoint{12.214716in}{2.184137in}}%
\pgfpathlineto{\pgfqpoint{12.216729in}{2.180617in}}%
\pgfpathlineto{\pgfqpoint{12.218743in}{2.179385in}}%
\pgfpathlineto{\pgfqpoint{12.220756in}{2.175513in}}%
\pgfpathlineto{\pgfqpoint{12.222770in}{2.186777in}}%
\pgfpathlineto{\pgfqpoint{12.228810in}{2.183257in}}%
\pgfpathlineto{\pgfqpoint{12.230824in}{2.166889in}}%
\pgfpathlineto{\pgfqpoint{12.232837in}{2.174809in}}%
\pgfpathlineto{\pgfqpoint{12.234850in}{2.167593in}}%
\pgfpathlineto{\pgfqpoint{12.236864in}{2.176041in}}%
\pgfpathlineto{\pgfqpoint{12.242904in}{2.162137in}}%
\pgfpathlineto{\pgfqpoint{12.244918in}{2.154393in}}%
\pgfpathlineto{\pgfqpoint{12.246931in}{2.152985in}}%
\pgfpathlineto{\pgfqpoint{12.248945in}{2.153337in}}%
\pgfpathlineto{\pgfqpoint{12.250958in}{2.148937in}}%
\pgfpathlineto{\pgfqpoint{12.256998in}{2.148233in}}%
\pgfpathlineto{\pgfqpoint{12.259012in}{2.149289in}}%
\pgfpathlineto{\pgfqpoint{12.261025in}{2.151577in}}%
\pgfpathlineto{\pgfqpoint{12.263039in}{2.152281in}}%
\pgfpathlineto{\pgfqpoint{12.265052in}{2.149113in}}%
\pgfpathlineto{\pgfqpoint{12.271092in}{2.148409in}}%
\pgfpathlineto{\pgfqpoint{12.273106in}{2.135385in}}%
\pgfpathlineto{\pgfqpoint{12.275119in}{2.141721in}}%
\pgfpathlineto{\pgfqpoint{12.277133in}{2.137145in}}%
\pgfpathlineto{\pgfqpoint{12.279146in}{2.130633in}}%
\pgfpathlineto{\pgfqpoint{12.285187in}{2.132217in}}%
\pgfpathlineto{\pgfqpoint{12.287200in}{2.133449in}}%
\pgfpathlineto{\pgfqpoint{12.289214in}{2.131865in}}%
\pgfpathlineto{\pgfqpoint{12.291227in}{2.135209in}}%
\pgfpathlineto{\pgfqpoint{12.293240in}{2.124825in}}%
\pgfpathlineto{\pgfqpoint{12.299281in}{2.131337in}}%
\pgfpathlineto{\pgfqpoint{12.301294in}{2.127817in}}%
\pgfpathlineto{\pgfqpoint{12.303308in}{2.128873in}}%
\pgfpathlineto{\pgfqpoint{12.305321in}{2.132393in}}%
\pgfpathlineto{\pgfqpoint{12.307335in}{2.137321in}}%
\pgfpathlineto{\pgfqpoint{12.317402in}{2.149993in}}%
\pgfpathlineto{\pgfqpoint{12.319415in}{2.148233in}}%
\pgfpathlineto{\pgfqpoint{12.321429in}{2.120953in}}%
\pgfpathlineto{\pgfqpoint{12.327469in}{2.132569in}}%
\pgfpathlineto{\pgfqpoint{12.329482in}{2.115321in}}%
\pgfpathlineto{\pgfqpoint{12.331496in}{2.115849in}}%
\pgfpathlineto{\pgfqpoint{12.333509in}{2.123417in}}%
\pgfpathlineto{\pgfqpoint{12.335523in}{2.121833in}}%
\pgfpathlineto{\pgfqpoint{12.341563in}{2.111449in}}%
\pgfpathlineto{\pgfqpoint{12.343577in}{2.112505in}}%
\pgfpathlineto{\pgfqpoint{12.347603in}{2.129049in}}%
\pgfpathlineto{\pgfqpoint{12.349617in}{2.132393in}}%
\pgfpathlineto{\pgfqpoint{12.355657in}{2.126057in}}%
\pgfpathlineto{\pgfqpoint{12.357671in}{2.131337in}}%
\pgfpathlineto{\pgfqpoint{12.359684in}{2.124649in}}%
\pgfpathlineto{\pgfqpoint{12.361698in}{2.125529in}}%
\pgfpathlineto{\pgfqpoint{12.363711in}{2.123417in}}%
\pgfpathlineto{\pgfqpoint{12.369751in}{2.121833in}}%
\pgfpathlineto{\pgfqpoint{12.371765in}{2.112329in}}%
\pgfpathlineto{\pgfqpoint{12.373778in}{2.105993in}}%
\pgfpathlineto{\pgfqpoint{12.375792in}{2.105817in}}%
\pgfpathlineto{\pgfqpoint{12.377805in}{2.100537in}}%
\pgfpathlineto{\pgfqpoint{12.383846in}{2.104761in}}%
\pgfpathlineto{\pgfqpoint{12.385859in}{2.100185in}}%
\pgfpathlineto{\pgfqpoint{12.387872in}{2.106521in}}%
\pgfpathlineto{\pgfqpoint{12.391899in}{2.106169in}}%
\pgfpathlineto{\pgfqpoint{12.397940in}{2.108457in}}%
\pgfpathlineto{\pgfqpoint{12.399953in}{2.105993in}}%
\pgfpathlineto{\pgfqpoint{12.401967in}{2.107753in}}%
\pgfpathlineto{\pgfqpoint{12.405993in}{2.073785in}}%
\pgfpathlineto{\pgfqpoint{12.412034in}{2.073961in}}%
\pgfpathlineto{\pgfqpoint{12.414047in}{2.068153in}}%
\pgfpathlineto{\pgfqpoint{12.416061in}{2.064985in}}%
\pgfpathlineto{\pgfqpoint{12.418074in}{2.079065in}}%
\pgfpathlineto{\pgfqpoint{12.420088in}{2.073961in}}%
\pgfpathlineto{\pgfqpoint{12.426128in}{2.072201in}}%
\pgfpathlineto{\pgfqpoint{12.428141in}{2.065513in}}%
\pgfpathlineto{\pgfqpoint{12.430155in}{2.054249in}}%
\pgfpathlineto{\pgfqpoint{12.432168in}{2.053193in}}%
\pgfpathlineto{\pgfqpoint{12.434182in}{2.056361in}}%
\pgfpathlineto{\pgfqpoint{12.440222in}{2.062345in}}%
\pgfpathlineto{\pgfqpoint{12.444249in}{2.068505in}}%
\pgfpathlineto{\pgfqpoint{12.446262in}{2.050377in}}%
\pgfpathlineto{\pgfqpoint{12.448276in}{2.050377in}}%
\pgfpathlineto{\pgfqpoint{12.454316in}{2.042457in}}%
\pgfpathlineto{\pgfqpoint{12.456330in}{2.060937in}}%
\pgfpathlineto{\pgfqpoint{12.458343in}{2.069561in}}%
\pgfpathlineto{\pgfqpoint{12.460357in}{2.068153in}}%
\pgfpathlineto{\pgfqpoint{12.462370in}{2.071849in}}%
\pgfpathlineto{\pgfqpoint{12.468410in}{2.075545in}}%
\pgfpathlineto{\pgfqpoint{12.470424in}{2.093849in}}%
\pgfpathlineto{\pgfqpoint{12.472437in}{2.105289in}}%
\pgfpathlineto{\pgfqpoint{12.476464in}{2.112153in}}%
\pgfpathlineto{\pgfqpoint{12.482504in}{2.119193in}}%
\pgfpathlineto{\pgfqpoint{12.484518in}{2.116729in}}%
\pgfpathlineto{\pgfqpoint{12.486531in}{2.100185in}}%
\pgfpathlineto{\pgfqpoint{12.488545in}{2.099833in}}%
\pgfpathlineto{\pgfqpoint{12.490558in}{2.098777in}}%
\pgfpathlineto{\pgfqpoint{12.496599in}{2.097897in}}%
\pgfpathlineto{\pgfqpoint{12.498612in}{2.107401in}}%
\pgfpathlineto{\pgfqpoint{12.500625in}{2.123241in}}%
\pgfpathlineto{\pgfqpoint{12.502639in}{2.119369in}}%
\pgfpathlineto{\pgfqpoint{12.504652in}{2.125001in}}%
\pgfpathlineto{\pgfqpoint{12.510693in}{2.129049in}}%
\pgfpathlineto{\pgfqpoint{12.512706in}{2.138377in}}%
\pgfpathlineto{\pgfqpoint{12.514720in}{2.127113in}}%
\pgfpathlineto{\pgfqpoint{12.516733in}{2.129929in}}%
\pgfpathlineto{\pgfqpoint{12.518746in}{2.136969in}}%
\pgfpathlineto{\pgfqpoint{12.526800in}{2.150169in}}%
\pgfpathlineto{\pgfqpoint{12.528814in}{2.147881in}}%
\pgfpathlineto{\pgfqpoint{12.530827in}{2.158441in}}%
\pgfpathlineto{\pgfqpoint{12.532841in}{2.158969in}}%
\pgfpathlineto{\pgfqpoint{12.540894in}{2.158265in}}%
\pgfpathlineto{\pgfqpoint{12.542908in}{2.155273in}}%
\pgfpathlineto{\pgfqpoint{12.544921in}{2.159849in}}%
\pgfpathlineto{\pgfqpoint{12.546935in}{2.154217in}}%
\pgfpathlineto{\pgfqpoint{12.554989in}{2.172873in}}%
\pgfpathlineto{\pgfqpoint{12.557002in}{2.171993in}}%
\pgfpathlineto{\pgfqpoint{12.559015in}{2.173929in}}%
\pgfpathlineto{\pgfqpoint{12.561029in}{2.161257in}}%
\pgfpathlineto{\pgfqpoint{12.567069in}{2.152105in}}%
\pgfpathlineto{\pgfqpoint{12.569083in}{2.153337in}}%
\pgfpathlineto{\pgfqpoint{12.571096in}{2.148761in}}%
\pgfpathlineto{\pgfqpoint{12.573110in}{2.152105in}}%
\pgfpathlineto{\pgfqpoint{12.575123in}{2.150169in}}%
\pgfpathlineto{\pgfqpoint{12.583177in}{2.153161in}}%
\pgfpathlineto{\pgfqpoint{12.585190in}{2.145417in}}%
\pgfpathlineto{\pgfqpoint{12.587204in}{2.147177in}}%
\pgfpathlineto{\pgfqpoint{12.589217in}{2.152809in}}%
\pgfpathlineto{\pgfqpoint{12.595257in}{2.147881in}}%
\pgfpathlineto{\pgfqpoint{12.597271in}{2.111977in}}%
\pgfpathlineto{\pgfqpoint{12.599284in}{2.106521in}}%
\pgfpathlineto{\pgfqpoint{12.601298in}{2.096313in}}%
\pgfpathlineto{\pgfqpoint{12.603311in}{2.103705in}}%
\pgfpathlineto{\pgfqpoint{12.609352in}{2.100185in}}%
\pgfpathlineto{\pgfqpoint{12.611365in}{2.094553in}}%
\pgfpathlineto{\pgfqpoint{12.613379in}{2.084873in}}%
\pgfpathlineto{\pgfqpoint{12.615392in}{2.083113in}}%
\pgfpathlineto{\pgfqpoint{12.617405in}{2.087865in}}%
\pgfpathlineto{\pgfqpoint{12.623446in}{2.079065in}}%
\pgfpathlineto{\pgfqpoint{12.625459in}{2.079241in}}%
\pgfpathlineto{\pgfqpoint{12.627473in}{2.084521in}}%
\pgfpathlineto{\pgfqpoint{12.629486in}{2.091385in}}%
\pgfpathlineto{\pgfqpoint{12.631500in}{2.094025in}}%
\pgfpathlineto{\pgfqpoint{12.637540in}{2.087337in}}%
\pgfpathlineto{\pgfqpoint{12.639553in}{2.082937in}}%
\pgfpathlineto{\pgfqpoint{12.641567in}{2.079945in}}%
\pgfpathlineto{\pgfqpoint{12.643580in}{2.085929in}}%
\pgfpathlineto{\pgfqpoint{12.645594in}{2.097369in}}%
\pgfpathlineto{\pgfqpoint{12.653647in}{2.101065in}}%
\pgfpathlineto{\pgfqpoint{12.655661in}{2.104937in}}%
\pgfpathlineto{\pgfqpoint{12.657674in}{2.114969in}}%
\pgfpathlineto{\pgfqpoint{12.659688in}{2.119545in}}%
\pgfpathlineto{\pgfqpoint{12.665728in}{2.109161in}}%
\pgfpathlineto{\pgfqpoint{12.667742in}{2.104233in}}%
\pgfpathlineto{\pgfqpoint{12.671768in}{2.109689in}}%
\pgfpathlineto{\pgfqpoint{12.673782in}{2.111449in}}%
\pgfpathlineto{\pgfqpoint{12.679822in}{2.110569in}}%
\pgfpathlineto{\pgfqpoint{12.681836in}{2.101241in}}%
\pgfpathlineto{\pgfqpoint{12.683849in}{2.096841in}}%
\pgfpathlineto{\pgfqpoint{12.685863in}{2.098777in}}%
\pgfpathlineto{\pgfqpoint{12.687876in}{2.099833in}}%
\pgfpathlineto{\pgfqpoint{12.693916in}{2.101769in}}%
\pgfpathlineto{\pgfqpoint{12.695930in}{2.100009in}}%
\pgfpathlineto{\pgfqpoint{12.697943in}{2.112329in}}%
\pgfpathlineto{\pgfqpoint{12.699957in}{2.110745in}}%
\pgfpathlineto{\pgfqpoint{12.701970in}{2.116201in}}%
\pgfpathlineto{\pgfqpoint{12.708011in}{2.113913in}}%
\pgfpathlineto{\pgfqpoint{12.710024in}{2.112505in}}%
\pgfpathlineto{\pgfqpoint{12.712037in}{2.105465in}}%
\pgfpathlineto{\pgfqpoint{12.714051in}{2.104409in}}%
\pgfpathlineto{\pgfqpoint{12.716064in}{2.105113in}}%
\pgfpathlineto{\pgfqpoint{12.722105in}{2.096489in}}%
\pgfpathlineto{\pgfqpoint{12.724118in}{2.099129in}}%
\pgfpathlineto{\pgfqpoint{12.726132in}{2.096489in}}%
\pgfpathlineto{\pgfqpoint{12.728145in}{2.095257in}}%
\pgfpathlineto{\pgfqpoint{12.730158in}{2.090505in}}%
\pgfpathlineto{\pgfqpoint{12.738212in}{2.099305in}}%
\pgfpathlineto{\pgfqpoint{12.740226in}{2.094729in}}%
\pgfpathlineto{\pgfqpoint{12.742239in}{2.094553in}}%
\pgfpathlineto{\pgfqpoint{12.744253in}{2.098073in}}%
\pgfpathlineto{\pgfqpoint{12.750293in}{2.096313in}}%
\pgfpathlineto{\pgfqpoint{12.752306in}{2.098777in}}%
\pgfpathlineto{\pgfqpoint{12.754320in}{2.102297in}}%
\pgfpathlineto{\pgfqpoint{12.756333in}{2.097545in}}%
\pgfpathlineto{\pgfqpoint{12.764387in}{2.100537in}}%
\pgfpathlineto{\pgfqpoint{12.766401in}{2.107049in}}%
\pgfpathlineto{\pgfqpoint{12.768414in}{2.102649in}}%
\pgfpathlineto{\pgfqpoint{12.770427in}{2.094201in}}%
\pgfpathlineto{\pgfqpoint{12.772441in}{2.075721in}}%
\pgfpathlineto{\pgfqpoint{12.778481in}{2.072553in}}%
\pgfpathlineto{\pgfqpoint{12.780495in}{2.066921in}}%
\pgfpathlineto{\pgfqpoint{12.782508in}{2.077481in}}%
\pgfpathlineto{\pgfqpoint{12.786535in}{2.054425in}}%
\pgfpathlineto{\pgfqpoint{12.792575in}{2.053897in}}%
\pgfpathlineto{\pgfqpoint{12.794589in}{2.054425in}}%
\pgfpathlineto{\pgfqpoint{12.796602in}{2.058297in}}%
\pgfpathlineto{\pgfqpoint{12.798616in}{2.053897in}}%
\pgfpathlineto{\pgfqpoint{12.800629in}{2.066921in}}%
\pgfpathlineto{\pgfqpoint{12.806669in}{2.065865in}}%
\pgfpathlineto{\pgfqpoint{12.808683in}{2.062521in}}%
\pgfpathlineto{\pgfqpoint{12.810696in}{2.061993in}}%
\pgfpathlineto{\pgfqpoint{12.812710in}{2.056185in}}%
\pgfpathlineto{\pgfqpoint{12.814723in}{2.053369in}}%
\pgfpathlineto{\pgfqpoint{12.820764in}{2.045977in}}%
\pgfpathlineto{\pgfqpoint{12.822777in}{2.044921in}}%
\pgfpathlineto{\pgfqpoint{12.824790in}{2.031721in}}%
\pgfpathlineto{\pgfqpoint{12.826804in}{2.040697in}}%
\pgfpathlineto{\pgfqpoint{12.828817in}{2.046681in}}%
\pgfpathlineto{\pgfqpoint{12.836871in}{2.047561in}}%
\pgfpathlineto{\pgfqpoint{12.838885in}{2.040697in}}%
\pgfpathlineto{\pgfqpoint{12.840898in}{2.044921in}}%
\pgfpathlineto{\pgfqpoint{12.842911in}{2.045097in}}%
\pgfpathlineto{\pgfqpoint{12.850965in}{2.059001in}}%
\pgfpathlineto{\pgfqpoint{12.852979in}{2.066041in}}%
\pgfpathlineto{\pgfqpoint{12.854992in}{2.063929in}}%
\pgfpathlineto{\pgfqpoint{12.857006in}{2.062873in}}%
\pgfpathlineto{\pgfqpoint{12.863046in}{2.061817in}}%
\pgfpathlineto{\pgfqpoint{12.867073in}{2.063753in}}%
\pgfpathlineto{\pgfqpoint{12.869086in}{2.058825in}}%
\pgfpathlineto{\pgfqpoint{12.871100in}{2.067273in}}%
\pgfpathlineto{\pgfqpoint{12.877140in}{2.074841in}}%
\pgfpathlineto{\pgfqpoint{12.879154in}{2.063225in}}%
\pgfpathlineto{\pgfqpoint{12.881167in}{2.066921in}}%
\pgfpathlineto{\pgfqpoint{12.883180in}{2.066041in}}%
\pgfpathlineto{\pgfqpoint{12.891234in}{2.064985in}}%
\pgfpathlineto{\pgfqpoint{12.895261in}{2.046505in}}%
\pgfpathlineto{\pgfqpoint{12.897275in}{2.046681in}}%
\pgfpathlineto{\pgfqpoint{12.899288in}{2.046153in}}%
\pgfpathlineto{\pgfqpoint{12.905328in}{2.051961in}}%
\pgfpathlineto{\pgfqpoint{12.907342in}{2.037529in}}%
\pgfpathlineto{\pgfqpoint{12.909355in}{2.037529in}}%
\pgfpathlineto{\pgfqpoint{12.911369in}{2.030665in}}%
\pgfpathlineto{\pgfqpoint{12.913382in}{2.034537in}}%
\pgfpathlineto{\pgfqpoint{12.919422in}{2.039993in}}%
\pgfpathlineto{\pgfqpoint{12.923449in}{2.034361in}}%
\pgfpathlineto{\pgfqpoint{12.925463in}{2.025561in}}%
\pgfpathlineto{\pgfqpoint{12.927476in}{2.024857in}}%
\pgfpathlineto{\pgfqpoint{12.933517in}{2.020457in}}%
\pgfpathlineto{\pgfqpoint{12.935530in}{2.015353in}}%
\pgfpathlineto{\pgfqpoint{12.939557in}{2.025033in}}%
\pgfpathlineto{\pgfqpoint{12.941570in}{2.026265in}}%
\pgfpathlineto{\pgfqpoint{12.947611in}{2.027849in}}%
\pgfpathlineto{\pgfqpoint{12.949624in}{2.022921in}}%
\pgfpathlineto{\pgfqpoint{12.951638in}{2.024329in}}%
\pgfpathlineto{\pgfqpoint{12.953651in}{2.036825in}}%
\pgfpathlineto{\pgfqpoint{12.955665in}{2.037001in}}%
\pgfpathlineto{\pgfqpoint{12.961705in}{2.028553in}}%
\pgfpathlineto{\pgfqpoint{12.963718in}{2.032953in}}%
\pgfpathlineto{\pgfqpoint{12.965732in}{2.039641in}}%
\pgfpathlineto{\pgfqpoint{12.967745in}{2.094553in}}%
\pgfpathlineto{\pgfqpoint{12.969759in}{2.096665in}}%
\pgfpathlineto{\pgfqpoint{12.975799in}{2.104057in}}%
\pgfpathlineto{\pgfqpoint{12.977812in}{2.111977in}}%
\pgfpathlineto{\pgfqpoint{12.979826in}{2.101065in}}%
\pgfpathlineto{\pgfqpoint{12.983853in}{2.112329in}}%
\pgfpathlineto{\pgfqpoint{12.989893in}{2.111449in}}%
\pgfpathlineto{\pgfqpoint{12.991907in}{2.107225in}}%
\pgfpathlineto{\pgfqpoint{12.993920in}{2.101241in}}%
\pgfpathlineto{\pgfqpoint{12.995933in}{2.097897in}}%
\pgfpathlineto{\pgfqpoint{12.997947in}{2.098601in}}%
\pgfpathlineto{\pgfqpoint{13.003987in}{2.110217in}}%
\pgfpathlineto{\pgfqpoint{13.006001in}{2.105289in}}%
\pgfpathlineto{\pgfqpoint{13.008014in}{2.104233in}}%
\pgfpathlineto{\pgfqpoint{13.010028in}{2.095961in}}%
\pgfpathlineto{\pgfqpoint{13.012041in}{2.092617in}}%
\pgfpathlineto{\pgfqpoint{13.018081in}{2.099833in}}%
\pgfpathlineto{\pgfqpoint{13.020095in}{2.103353in}}%
\pgfpathlineto{\pgfqpoint{13.022108in}{2.102121in}}%
\pgfpathlineto{\pgfqpoint{13.024122in}{2.102649in}}%
\pgfpathlineto{\pgfqpoint{13.026135in}{2.108633in}}%
\pgfpathlineto{\pgfqpoint{13.032176in}{2.107401in}}%
\pgfpathlineto{\pgfqpoint{13.034189in}{2.105817in}}%
\pgfpathlineto{\pgfqpoint{13.036202in}{2.099657in}}%
\pgfpathlineto{\pgfqpoint{13.038216in}{2.097193in}}%
\pgfpathlineto{\pgfqpoint{13.040229in}{2.096313in}}%
\pgfpathlineto{\pgfqpoint{13.048283in}{2.087337in}}%
\pgfpathlineto{\pgfqpoint{13.050297in}{2.080121in}}%
\pgfpathlineto{\pgfqpoint{13.052310in}{2.068681in}}%
\pgfpathlineto{\pgfqpoint{13.054323in}{2.067273in}}%
\pgfpathlineto{\pgfqpoint{13.060364in}{2.070265in}}%
\pgfpathlineto{\pgfqpoint{13.064391in}{2.085577in}}%
\pgfpathlineto{\pgfqpoint{13.066404in}{2.084521in}}%
\pgfpathlineto{\pgfqpoint{13.068418in}{2.095433in}}%
\pgfpathlineto{\pgfqpoint{13.074458in}{2.099129in}}%
\pgfpathlineto{\pgfqpoint{13.076471in}{2.119193in}}%
\pgfpathlineto{\pgfqpoint{13.078485in}{2.121481in}}%
\pgfpathlineto{\pgfqpoint{13.080498in}{2.112857in}}%
\pgfpathlineto{\pgfqpoint{13.082512in}{2.128169in}}%
\pgfpathlineto{\pgfqpoint{13.088552in}{2.128169in}}%
\pgfpathlineto{\pgfqpoint{13.090566in}{2.122009in}}%
\pgfpathlineto{\pgfqpoint{13.092579in}{2.122009in}}%
\pgfpathlineto{\pgfqpoint{13.094592in}{2.120425in}}%
\pgfpathlineto{\pgfqpoint{13.096606in}{2.121657in}}%
\pgfpathlineto{\pgfqpoint{13.102646in}{2.119545in}}%
\pgfpathlineto{\pgfqpoint{13.104660in}{2.127465in}}%
\pgfpathlineto{\pgfqpoint{13.106673in}{2.128169in}}%
\pgfpathlineto{\pgfqpoint{13.108687in}{2.126057in}}%
\pgfpathlineto{\pgfqpoint{13.110700in}{2.120073in}}%
\pgfpathlineto{\pgfqpoint{13.118754in}{2.125529in}}%
\pgfpathlineto{\pgfqpoint{13.120767in}{2.120953in}}%
\pgfpathlineto{\pgfqpoint{13.124794in}{2.104585in}}%
\pgfpathlineto{\pgfqpoint{13.130834in}{2.108457in}}%
\pgfpathlineto{\pgfqpoint{13.134861in}{2.117433in}}%
\pgfpathlineto{\pgfqpoint{13.136875in}{2.126585in}}%
\pgfpathlineto{\pgfqpoint{13.138888in}{2.131865in}}%
\pgfpathlineto{\pgfqpoint{13.144929in}{2.123065in}}%
\pgfpathlineto{\pgfqpoint{13.146942in}{2.122185in}}%
\pgfpathlineto{\pgfqpoint{13.148955in}{2.117257in}}%
\pgfpathlineto{\pgfqpoint{13.150969in}{2.121481in}}%
\pgfpathlineto{\pgfqpoint{13.152982in}{2.121129in}}%
\pgfpathlineto{\pgfqpoint{13.159023in}{2.104233in}}%
\pgfpathlineto{\pgfqpoint{13.161036in}{2.104761in}}%
\pgfpathlineto{\pgfqpoint{13.163050in}{2.104233in}}%
\pgfpathlineto{\pgfqpoint{13.165063in}{2.098073in}}%
\pgfpathlineto{\pgfqpoint{13.167077in}{2.097545in}}%
\pgfpathlineto{\pgfqpoint{13.173117in}{2.066745in}}%
\pgfpathlineto{\pgfqpoint{13.175130in}{2.067273in}}%
\pgfpathlineto{\pgfqpoint{13.177144in}{2.065513in}}%
\pgfpathlineto{\pgfqpoint{13.179157in}{2.059177in}}%
\pgfpathlineto{\pgfqpoint{13.181171in}{2.056185in}}%
\pgfpathlineto{\pgfqpoint{13.187211in}{2.053897in}}%
\pgfpathlineto{\pgfqpoint{13.189224in}{2.045273in}}%
\pgfpathlineto{\pgfqpoint{13.191238in}{2.043513in}}%
\pgfpathlineto{\pgfqpoint{13.195265in}{2.064809in}}%
\pgfpathlineto{\pgfqpoint{13.201305in}{2.077657in}}%
\pgfpathlineto{\pgfqpoint{13.203319in}{2.077305in}}%
\pgfpathlineto{\pgfqpoint{13.205332in}{2.092265in}}%
\pgfpathlineto{\pgfqpoint{13.209359in}{2.090857in}}%
\pgfpathlineto{\pgfqpoint{13.215399in}{2.101417in}}%
\pgfpathlineto{\pgfqpoint{13.221440in}{2.154041in}}%
\pgfpathlineto{\pgfqpoint{13.223453in}{2.159849in}}%
\pgfpathlineto{\pgfqpoint{13.229493in}{2.167593in}}%
\pgfpathlineto{\pgfqpoint{13.231507in}{2.154393in}}%
\pgfpathlineto{\pgfqpoint{13.235534in}{2.146297in}}%
\pgfpathlineto{\pgfqpoint{13.237547in}{2.157209in}}%
\pgfpathlineto{\pgfqpoint{13.243587in}{2.169529in}}%
\pgfpathlineto{\pgfqpoint{13.245601in}{2.191529in}}%
\pgfpathlineto{\pgfqpoint{13.247614in}{2.186601in}}%
\pgfpathlineto{\pgfqpoint{13.249628in}{2.177625in}}%
\pgfpathlineto{\pgfqpoint{13.251641in}{2.182905in}}%
\pgfpathlineto{\pgfqpoint{13.257682in}{2.192409in}}%
\pgfpathlineto{\pgfqpoint{13.259695in}{2.185545in}}%
\pgfpathlineto{\pgfqpoint{13.261709in}{2.184665in}}%
\pgfpathlineto{\pgfqpoint{13.265735in}{2.191529in}}%
\pgfpathlineto{\pgfqpoint{13.273789in}{2.191881in}}%
\pgfpathlineto{\pgfqpoint{13.275803in}{2.192937in}}%
\pgfpathlineto{\pgfqpoint{13.277816in}{2.195401in}}%
\pgfpathlineto{\pgfqpoint{13.279830in}{2.187305in}}%
\pgfpathlineto{\pgfqpoint{13.279830in}{2.187305in}}%
\pgfusepath{stroke}%
\end{pgfscope}%
\begin{pgfscope}%
\pgfsetrectcap%
\pgfsetmiterjoin%
\pgfsetlinewidth{0.803000pt}%
\definecolor{currentstroke}{rgb}{1.000000,1.000000,1.000000}%
\pgfsetstrokecolor{currentstroke}%
\pgfsetdash{}{0pt}%
\pgfpathmoveto{\pgfqpoint{8.656250in}{1.771471in}}%
\pgfpathlineto{\pgfqpoint{8.656250in}{2.215588in}}%
\pgfusepath{stroke}%
\end{pgfscope}%
\begin{pgfscope}%
\pgfsetrectcap%
\pgfsetmiterjoin%
\pgfsetlinewidth{0.803000pt}%
\definecolor{currentstroke}{rgb}{1.000000,1.000000,1.000000}%
\pgfsetstrokecolor{currentstroke}%
\pgfsetdash{}{0pt}%
\pgfpathmoveto{\pgfqpoint{13.500000in}{1.771471in}}%
\pgfpathlineto{\pgfqpoint{13.500000in}{2.215588in}}%
\pgfusepath{stroke}%
\end{pgfscope}%
\begin{pgfscope}%
\pgfsetrectcap%
\pgfsetmiterjoin%
\pgfsetlinewidth{0.803000pt}%
\definecolor{currentstroke}{rgb}{1.000000,1.000000,1.000000}%
\pgfsetstrokecolor{currentstroke}%
\pgfsetdash{}{0pt}%
\pgfpathmoveto{\pgfqpoint{8.656250in}{1.771471in}}%
\pgfpathlineto{\pgfqpoint{13.500000in}{1.771471in}}%
\pgfusepath{stroke}%
\end{pgfscope}%
\begin{pgfscope}%
\pgfsetrectcap%
\pgfsetmiterjoin%
\pgfsetlinewidth{0.803000pt}%
\definecolor{currentstroke}{rgb}{1.000000,1.000000,1.000000}%
\pgfsetstrokecolor{currentstroke}%
\pgfsetdash{}{0pt}%
\pgfpathmoveto{\pgfqpoint{8.656250in}{2.215588in}}%
\pgfpathlineto{\pgfqpoint{13.500000in}{2.215588in}}%
\pgfusepath{stroke}%
\end{pgfscope}%
\begin{pgfscope}%
\definecolor{textcolor}{rgb}{0.150000,0.150000,0.150000}%
\pgfsetstrokecolor{textcolor}%
\pgfsetfillcolor{textcolor}%
\pgftext[x=11.078125in,y=2.298922in,,base]{\color{textcolor}\rmfamily\fontsize{16.800000}{20.160000}\selectfont VZ}%
\end{pgfscope}%
\begin{pgfscope}%
\pgfsetbuttcap%
\pgfsetmiterjoin%
\definecolor{currentfill}{rgb}{0.917647,0.917647,0.949020}%
\pgfsetfillcolor{currentfill}%
\pgfsetlinewidth{0.000000pt}%
\definecolor{currentstroke}{rgb}{0.000000,0.000000,0.000000}%
\pgfsetstrokecolor{currentstroke}%
\pgfsetstrokeopacity{0.000000}%
\pgfsetdash{}{0pt}%
\pgfpathmoveto{\pgfqpoint{1.875000in}{0.750000in}}%
\pgfpathlineto{\pgfqpoint{6.718750in}{0.750000in}}%
\pgfpathlineto{\pgfqpoint{6.718750in}{1.194118in}}%
\pgfpathlineto{\pgfqpoint{1.875000in}{1.194118in}}%
\pgfpathclose%
\pgfusepath{fill}%
\end{pgfscope}%
\begin{pgfscope}%
\pgfpathrectangle{\pgfqpoint{1.875000in}{0.750000in}}{\pgfqpoint{4.843750in}{0.444118in}}%
\pgfusepath{clip}%
\pgfsetroundcap%
\pgfsetroundjoin%
\pgfsetlinewidth{0.803000pt}%
\definecolor{currentstroke}{rgb}{1.000000,1.000000,1.000000}%
\pgfsetstrokecolor{currentstroke}%
\pgfsetdash{}{0pt}%
\pgfpathmoveto{\pgfqpoint{2.091144in}{0.750000in}}%
\pgfpathlineto{\pgfqpoint{2.091144in}{1.194118in}}%
\pgfusepath{stroke}%
\end{pgfscope}%
\begin{pgfscope}%
\definecolor{textcolor}{rgb}{0.150000,0.150000,0.150000}%
\pgfsetstrokecolor{textcolor}%
\pgfsetfillcolor{textcolor}%
\pgftext[x=2.091144in,y=0.652778in,,top]{\color{textcolor}\rmfamily\fontsize{14.000000}{16.800000}\selectfont 2012}%
\end{pgfscope}%
\begin{pgfscope}%
\pgfpathrectangle{\pgfqpoint{1.875000in}{0.750000in}}{\pgfqpoint{4.843750in}{0.444118in}}%
\pgfusepath{clip}%
\pgfsetroundcap%
\pgfsetroundjoin%
\pgfsetlinewidth{0.803000pt}%
\definecolor{currentstroke}{rgb}{1.000000,1.000000,1.000000}%
\pgfsetstrokecolor{currentstroke}%
\pgfsetdash{}{0pt}%
\pgfpathmoveto{\pgfqpoint{2.828065in}{0.750000in}}%
\pgfpathlineto{\pgfqpoint{2.828065in}{1.194118in}}%
\pgfusepath{stroke}%
\end{pgfscope}%
\begin{pgfscope}%
\definecolor{textcolor}{rgb}{0.150000,0.150000,0.150000}%
\pgfsetstrokecolor{textcolor}%
\pgfsetfillcolor{textcolor}%
\pgftext[x=2.828065in,y=0.652778in,,top]{\color{textcolor}\rmfamily\fontsize{14.000000}{16.800000}\selectfont 2013}%
\end{pgfscope}%
\begin{pgfscope}%
\pgfpathrectangle{\pgfqpoint{1.875000in}{0.750000in}}{\pgfqpoint{4.843750in}{0.444118in}}%
\pgfusepath{clip}%
\pgfsetroundcap%
\pgfsetroundjoin%
\pgfsetlinewidth{0.803000pt}%
\definecolor{currentstroke}{rgb}{1.000000,1.000000,1.000000}%
\pgfsetstrokecolor{currentstroke}%
\pgfsetdash{}{0pt}%
\pgfpathmoveto{\pgfqpoint{3.562973in}{0.750000in}}%
\pgfpathlineto{\pgfqpoint{3.562973in}{1.194118in}}%
\pgfusepath{stroke}%
\end{pgfscope}%
\begin{pgfscope}%
\definecolor{textcolor}{rgb}{0.150000,0.150000,0.150000}%
\pgfsetstrokecolor{textcolor}%
\pgfsetfillcolor{textcolor}%
\pgftext[x=3.562973in,y=0.652778in,,top]{\color{textcolor}\rmfamily\fontsize{14.000000}{16.800000}\selectfont 2014}%
\end{pgfscope}%
\begin{pgfscope}%
\pgfpathrectangle{\pgfqpoint{1.875000in}{0.750000in}}{\pgfqpoint{4.843750in}{0.444118in}}%
\pgfusepath{clip}%
\pgfsetroundcap%
\pgfsetroundjoin%
\pgfsetlinewidth{0.803000pt}%
\definecolor{currentstroke}{rgb}{1.000000,1.000000,1.000000}%
\pgfsetstrokecolor{currentstroke}%
\pgfsetdash{}{0pt}%
\pgfpathmoveto{\pgfqpoint{4.297882in}{0.750000in}}%
\pgfpathlineto{\pgfqpoint{4.297882in}{1.194118in}}%
\pgfusepath{stroke}%
\end{pgfscope}%
\begin{pgfscope}%
\definecolor{textcolor}{rgb}{0.150000,0.150000,0.150000}%
\pgfsetstrokecolor{textcolor}%
\pgfsetfillcolor{textcolor}%
\pgftext[x=4.297882in,y=0.652778in,,top]{\color{textcolor}\rmfamily\fontsize{14.000000}{16.800000}\selectfont 2015}%
\end{pgfscope}%
\begin{pgfscope}%
\pgfpathrectangle{\pgfqpoint{1.875000in}{0.750000in}}{\pgfqpoint{4.843750in}{0.444118in}}%
\pgfusepath{clip}%
\pgfsetroundcap%
\pgfsetroundjoin%
\pgfsetlinewidth{0.803000pt}%
\definecolor{currentstroke}{rgb}{1.000000,1.000000,1.000000}%
\pgfsetstrokecolor{currentstroke}%
\pgfsetdash{}{0pt}%
\pgfpathmoveto{\pgfqpoint{5.032790in}{0.750000in}}%
\pgfpathlineto{\pgfqpoint{5.032790in}{1.194118in}}%
\pgfusepath{stroke}%
\end{pgfscope}%
\begin{pgfscope}%
\definecolor{textcolor}{rgb}{0.150000,0.150000,0.150000}%
\pgfsetstrokecolor{textcolor}%
\pgfsetfillcolor{textcolor}%
\pgftext[x=5.032790in,y=0.652778in,,top]{\color{textcolor}\rmfamily\fontsize{14.000000}{16.800000}\selectfont 2016}%
\end{pgfscope}%
\begin{pgfscope}%
\pgfpathrectangle{\pgfqpoint{1.875000in}{0.750000in}}{\pgfqpoint{4.843750in}{0.444118in}}%
\pgfusepath{clip}%
\pgfsetroundcap%
\pgfsetroundjoin%
\pgfsetlinewidth{0.803000pt}%
\definecolor{currentstroke}{rgb}{1.000000,1.000000,1.000000}%
\pgfsetstrokecolor{currentstroke}%
\pgfsetdash{}{0pt}%
\pgfpathmoveto{\pgfqpoint{5.769712in}{0.750000in}}%
\pgfpathlineto{\pgfqpoint{5.769712in}{1.194118in}}%
\pgfusepath{stroke}%
\end{pgfscope}%
\begin{pgfscope}%
\definecolor{textcolor}{rgb}{0.150000,0.150000,0.150000}%
\pgfsetstrokecolor{textcolor}%
\pgfsetfillcolor{textcolor}%
\pgftext[x=5.769712in,y=0.652778in,,top]{\color{textcolor}\rmfamily\fontsize{14.000000}{16.800000}\selectfont 2017}%
\end{pgfscope}%
\begin{pgfscope}%
\pgfpathrectangle{\pgfqpoint{1.875000in}{0.750000in}}{\pgfqpoint{4.843750in}{0.444118in}}%
\pgfusepath{clip}%
\pgfsetroundcap%
\pgfsetroundjoin%
\pgfsetlinewidth{0.803000pt}%
\definecolor{currentstroke}{rgb}{1.000000,1.000000,1.000000}%
\pgfsetstrokecolor{currentstroke}%
\pgfsetdash{}{0pt}%
\pgfpathmoveto{\pgfqpoint{6.504620in}{0.750000in}}%
\pgfpathlineto{\pgfqpoint{6.504620in}{1.194118in}}%
\pgfusepath{stroke}%
\end{pgfscope}%
\begin{pgfscope}%
\definecolor{textcolor}{rgb}{0.150000,0.150000,0.150000}%
\pgfsetstrokecolor{textcolor}%
\pgfsetfillcolor{textcolor}%
\pgftext[x=6.504620in,y=0.652778in,,top]{\color{textcolor}\rmfamily\fontsize{14.000000}{16.800000}\selectfont 2018}%
\end{pgfscope}%
\begin{pgfscope}%
\pgfpathrectangle{\pgfqpoint{1.875000in}{0.750000in}}{\pgfqpoint{4.843750in}{0.444118in}}%
\pgfusepath{clip}%
\pgfsetroundcap%
\pgfsetroundjoin%
\pgfsetlinewidth{0.803000pt}%
\definecolor{currentstroke}{rgb}{1.000000,1.000000,1.000000}%
\pgfsetstrokecolor{currentstroke}%
\pgfsetdash{}{0pt}%
\pgfpathmoveto{\pgfqpoint{1.875000in}{0.894973in}}%
\pgfpathlineto{\pgfqpoint{6.718750in}{0.894973in}}%
\pgfusepath{stroke}%
\end{pgfscope}%
\begin{pgfscope}%
\definecolor{textcolor}{rgb}{0.150000,0.150000,0.150000}%
\pgfsetstrokecolor{textcolor}%
\pgfsetfillcolor{textcolor}%
\pgftext[x=1.530355in,y=0.821106in,left,base]{\color{textcolor}\rmfamily\fontsize{14.000000}{16.800000}\selectfont 50}%
\end{pgfscope}%
\begin{pgfscope}%
\pgfpathrectangle{\pgfqpoint{1.875000in}{0.750000in}}{\pgfqpoint{4.843750in}{0.444118in}}%
\pgfusepath{clip}%
\pgfsetroundcap%
\pgfsetroundjoin%
\pgfsetlinewidth{0.803000pt}%
\definecolor{currentstroke}{rgb}{1.000000,1.000000,1.000000}%
\pgfsetstrokecolor{currentstroke}%
\pgfsetdash{}{0pt}%
\pgfpathmoveto{\pgfqpoint{1.875000in}{1.115597in}}%
\pgfpathlineto{\pgfqpoint{6.718750in}{1.115597in}}%
\pgfusepath{stroke}%
\end{pgfscope}%
\begin{pgfscope}%
\definecolor{textcolor}{rgb}{0.150000,0.150000,0.150000}%
\pgfsetstrokecolor{textcolor}%
\pgfsetfillcolor{textcolor}%
\pgftext[x=1.406643in,y=1.041731in,left,base]{\color{textcolor}\rmfamily\fontsize{14.000000}{16.800000}\selectfont 100}%
\end{pgfscope}%
\begin{pgfscope}%
\pgfpathrectangle{\pgfqpoint{1.875000in}{0.750000in}}{\pgfqpoint{4.843750in}{0.444118in}}%
\pgfusepath{clip}%
\pgfsetroundcap%
\pgfsetroundjoin%
\pgfsetlinewidth{1.505625pt}%
\definecolor{currentstroke}{rgb}{0.121569,0.466667,0.705882}%
\pgfsetstrokecolor{currentstroke}%
\pgfsetdash{}{0pt}%
\pgfpathmoveto{\pgfqpoint{2.095170in}{0.773938in}}%
\pgfpathlineto{\pgfqpoint{2.097184in}{0.772129in}}%
\pgfpathlineto{\pgfqpoint{2.099197in}{0.772879in}}%
\pgfpathlineto{\pgfqpoint{2.101211in}{0.771732in}}%
\pgfpathlineto{\pgfqpoint{2.111278in}{0.770187in}}%
\pgfpathlineto{\pgfqpoint{2.113291in}{0.772349in}}%
\pgfpathlineto{\pgfqpoint{2.115305in}{0.771732in}}%
\pgfpathlineto{\pgfqpoint{2.123359in}{0.773497in}}%
\pgfpathlineto{\pgfqpoint{2.125372in}{0.774688in}}%
\pgfpathlineto{\pgfqpoint{2.129399in}{0.771599in}}%
\pgfpathlineto{\pgfqpoint{2.135439in}{0.770628in}}%
\pgfpathlineto{\pgfqpoint{2.137453in}{0.771996in}}%
\pgfpathlineto{\pgfqpoint{2.139466in}{0.771555in}}%
\pgfpathlineto{\pgfqpoint{2.143493in}{0.772040in}}%
\pgfpathlineto{\pgfqpoint{2.149534in}{0.770937in}}%
\pgfpathlineto{\pgfqpoint{2.151547in}{0.771643in}}%
\pgfpathlineto{\pgfqpoint{2.153560in}{0.773452in}}%
\pgfpathlineto{\pgfqpoint{2.155574in}{0.776894in}}%
\pgfpathlineto{\pgfqpoint{2.157587in}{0.777821in}}%
\pgfpathlineto{\pgfqpoint{2.165641in}{0.777777in}}%
\pgfpathlineto{\pgfqpoint{2.167655in}{0.779100in}}%
\pgfpathlineto{\pgfqpoint{2.169668in}{0.783028in}}%
\pgfpathlineto{\pgfqpoint{2.171681in}{0.784484in}}%
\pgfpathlineto{\pgfqpoint{2.177722in}{0.783336in}}%
\pgfpathlineto{\pgfqpoint{2.179735in}{0.785763in}}%
\pgfpathlineto{\pgfqpoint{2.181749in}{0.786469in}}%
\pgfpathlineto{\pgfqpoint{2.183762in}{0.785234in}}%
\pgfpathlineto{\pgfqpoint{2.185776in}{0.786381in}}%
\pgfpathlineto{\pgfqpoint{2.193829in}{0.785631in}}%
\pgfpathlineto{\pgfqpoint{2.195843in}{0.787705in}}%
\pgfpathlineto{\pgfqpoint{2.197856in}{0.787837in}}%
\pgfpathlineto{\pgfqpoint{2.199870in}{0.788852in}}%
\pgfpathlineto{\pgfqpoint{2.205910in}{0.788190in}}%
\pgfpathlineto{\pgfqpoint{2.207923in}{0.790220in}}%
\pgfpathlineto{\pgfqpoint{2.209937in}{0.787705in}}%
\pgfpathlineto{\pgfqpoint{2.211950in}{0.788499in}}%
\pgfpathlineto{\pgfqpoint{2.213964in}{0.787528in}}%
\pgfpathlineto{\pgfqpoint{2.220004in}{0.787617in}}%
\pgfpathlineto{\pgfqpoint{2.222018in}{0.786249in}}%
\pgfpathlineto{\pgfqpoint{2.224031in}{0.786999in}}%
\pgfpathlineto{\pgfqpoint{2.226045in}{0.789293in}}%
\pgfpathlineto{\pgfqpoint{2.228058in}{0.788499in}}%
\pgfpathlineto{\pgfqpoint{2.234098in}{0.787881in}}%
\pgfpathlineto{\pgfqpoint{2.236112in}{0.788587in}}%
\pgfpathlineto{\pgfqpoint{2.238125in}{0.788102in}}%
\pgfpathlineto{\pgfqpoint{2.242152in}{0.788014in}}%
\pgfpathlineto{\pgfqpoint{2.248192in}{0.790132in}}%
\pgfpathlineto{\pgfqpoint{2.250206in}{0.787881in}}%
\pgfpathlineto{\pgfqpoint{2.254233in}{0.788631in}}%
\pgfpathlineto{\pgfqpoint{2.256246in}{0.790088in}}%
\pgfpathlineto{\pgfqpoint{2.262287in}{0.791323in}}%
\pgfpathlineto{\pgfqpoint{2.268327in}{0.790308in}}%
\pgfpathlineto{\pgfqpoint{2.270340in}{0.789293in}}%
\pgfpathlineto{\pgfqpoint{2.276381in}{0.790264in}}%
\pgfpathlineto{\pgfqpoint{2.278394in}{0.791632in}}%
\pgfpathlineto{\pgfqpoint{2.280408in}{0.790264in}}%
\pgfpathlineto{\pgfqpoint{2.282421in}{0.792250in}}%
\pgfpathlineto{\pgfqpoint{2.290475in}{0.790794in}}%
\pgfpathlineto{\pgfqpoint{2.292488in}{0.788102in}}%
\pgfpathlineto{\pgfqpoint{2.294502in}{0.788720in}}%
\pgfpathlineto{\pgfqpoint{2.296515in}{0.792250in}}%
\pgfpathlineto{\pgfqpoint{2.298529in}{0.794324in}}%
\pgfpathlineto{\pgfqpoint{2.304569in}{0.792073in}}%
\pgfpathlineto{\pgfqpoint{2.306582in}{0.793309in}}%
\pgfpathlineto{\pgfqpoint{2.312623in}{0.792250in}}%
\pgfpathlineto{\pgfqpoint{2.318663in}{0.789426in}}%
\pgfpathlineto{\pgfqpoint{2.320677in}{0.790220in}}%
\pgfpathlineto{\pgfqpoint{2.322690in}{0.793000in}}%
\pgfpathlineto{\pgfqpoint{2.324703in}{0.794368in}}%
\pgfpathlineto{\pgfqpoint{2.326717in}{0.794677in}}%
\pgfpathlineto{\pgfqpoint{2.336784in}{0.793397in}}%
\pgfpathlineto{\pgfqpoint{2.338798in}{0.787749in}}%
\pgfpathlineto{\pgfqpoint{2.340811in}{0.789117in}}%
\pgfpathlineto{\pgfqpoint{2.346851in}{0.789823in}}%
\pgfpathlineto{\pgfqpoint{2.350878in}{0.788764in}}%
\pgfpathlineto{\pgfqpoint{2.352892in}{0.789470in}}%
\pgfpathlineto{\pgfqpoint{2.362959in}{0.787970in}}%
\pgfpathlineto{\pgfqpoint{2.364972in}{0.789735in}}%
\pgfpathlineto{\pgfqpoint{2.368999in}{0.784925in}}%
\pgfpathlineto{\pgfqpoint{2.375040in}{0.788455in}}%
\pgfpathlineto{\pgfqpoint{2.377053in}{0.790573in}}%
\pgfpathlineto{\pgfqpoint{2.381080in}{0.791941in}}%
\pgfpathlineto{\pgfqpoint{2.383093in}{0.791544in}}%
\pgfpathlineto{\pgfqpoint{2.391147in}{0.792426in}}%
\pgfpathlineto{\pgfqpoint{2.397188in}{0.784528in}}%
\pgfpathlineto{\pgfqpoint{2.403228in}{0.786425in}}%
\pgfpathlineto{\pgfqpoint{2.405241in}{0.786072in}}%
\pgfpathlineto{\pgfqpoint{2.407255in}{0.788543in}}%
\pgfpathlineto{\pgfqpoint{2.409268in}{0.789029in}}%
\pgfpathlineto{\pgfqpoint{2.417322in}{0.788587in}}%
\pgfpathlineto{\pgfqpoint{2.419335in}{0.789382in}}%
\pgfpathlineto{\pgfqpoint{2.421349in}{0.787131in}}%
\pgfpathlineto{\pgfqpoint{2.425376in}{0.790573in}}%
\pgfpathlineto{\pgfqpoint{2.431416in}{0.792250in}}%
\pgfpathlineto{\pgfqpoint{2.435443in}{0.794677in}}%
\pgfpathlineto{\pgfqpoint{2.437456in}{0.791632in}}%
\pgfpathlineto{\pgfqpoint{2.439470in}{0.797015in}}%
\pgfpathlineto{\pgfqpoint{2.445510in}{0.793397in}}%
\pgfpathlineto{\pgfqpoint{2.447524in}{0.795250in}}%
\pgfpathlineto{\pgfqpoint{2.449537in}{0.795515in}}%
\pgfpathlineto{\pgfqpoint{2.451551in}{0.793662in}}%
\pgfpathlineto{\pgfqpoint{2.453564in}{0.795736in}}%
\pgfpathlineto{\pgfqpoint{2.459604in}{0.798648in}}%
\pgfpathlineto{\pgfqpoint{2.461618in}{0.798383in}}%
\pgfpathlineto{\pgfqpoint{2.465645in}{0.799045in}}%
\pgfpathlineto{\pgfqpoint{2.467658in}{0.797324in}}%
\pgfpathlineto{\pgfqpoint{2.473699in}{0.795736in}}%
\pgfpathlineto{\pgfqpoint{2.477725in}{0.792867in}}%
\pgfpathlineto{\pgfqpoint{2.481752in}{0.796177in}}%
\pgfpathlineto{\pgfqpoint{2.489806in}{0.800236in}}%
\pgfpathlineto{\pgfqpoint{2.491820in}{0.799839in}}%
\pgfpathlineto{\pgfqpoint{2.493833in}{0.797456in}}%
\pgfpathlineto{\pgfqpoint{2.495846in}{0.797809in}}%
\pgfpathlineto{\pgfqpoint{2.501887in}{0.796530in}}%
\pgfpathlineto{\pgfqpoint{2.503900in}{0.794765in}}%
\pgfpathlineto{\pgfqpoint{2.505914in}{0.794324in}}%
\pgfpathlineto{\pgfqpoint{2.507927in}{0.798780in}}%
\pgfpathlineto{\pgfqpoint{2.509941in}{0.801119in}}%
\pgfpathlineto{\pgfqpoint{2.515981in}{0.802796in}}%
\pgfpathlineto{\pgfqpoint{2.520008in}{0.799486in}}%
\pgfpathlineto{\pgfqpoint{2.524035in}{0.802928in}}%
\pgfpathlineto{\pgfqpoint{2.530075in}{0.802972in}}%
\pgfpathlineto{\pgfqpoint{2.532089in}{0.802354in}}%
\pgfpathlineto{\pgfqpoint{2.534102in}{0.803104in}}%
\pgfpathlineto{\pgfqpoint{2.536115in}{0.800501in}}%
\pgfpathlineto{\pgfqpoint{2.538129in}{0.801075in}}%
\pgfpathlineto{\pgfqpoint{2.544169in}{0.800325in}}%
\pgfpathlineto{\pgfqpoint{2.546183in}{0.801869in}}%
\pgfpathlineto{\pgfqpoint{2.548196in}{0.801913in}}%
\pgfpathlineto{\pgfqpoint{2.550210in}{0.802531in}}%
\pgfpathlineto{\pgfqpoint{2.552223in}{0.801560in}}%
\pgfpathlineto{\pgfqpoint{2.564304in}{0.800016in}}%
\pgfpathlineto{\pgfqpoint{2.566317in}{0.799530in}}%
\pgfpathlineto{\pgfqpoint{2.572357in}{0.800545in}}%
\pgfpathlineto{\pgfqpoint{2.574371in}{0.800280in}}%
\pgfpathlineto{\pgfqpoint{2.576384in}{0.800898in}}%
\pgfpathlineto{\pgfqpoint{2.578398in}{0.799574in}}%
\pgfpathlineto{\pgfqpoint{2.580411in}{0.801119in}}%
\pgfpathlineto{\pgfqpoint{2.588465in}{0.801384in}}%
\pgfpathlineto{\pgfqpoint{2.590478in}{0.800457in}}%
\pgfpathlineto{\pgfqpoint{2.592492in}{0.802354in}}%
\pgfpathlineto{\pgfqpoint{2.594505in}{0.802531in}}%
\pgfpathlineto{\pgfqpoint{2.600546in}{0.801516in}}%
\pgfpathlineto{\pgfqpoint{2.604573in}{0.806237in}}%
\pgfpathlineto{\pgfqpoint{2.606586in}{0.807782in}}%
\pgfpathlineto{\pgfqpoint{2.608599in}{0.807032in}}%
\pgfpathlineto{\pgfqpoint{2.616653in}{0.806458in}}%
\pgfpathlineto{\pgfqpoint{2.618667in}{0.807561in}}%
\pgfpathlineto{\pgfqpoint{2.620680in}{0.807385in}}%
\pgfpathlineto{\pgfqpoint{2.622694in}{0.807782in}}%
\pgfpathlineto{\pgfqpoint{2.628734in}{0.806546in}}%
\pgfpathlineto{\pgfqpoint{2.630747in}{0.807340in}}%
\pgfpathlineto{\pgfqpoint{2.632761in}{0.805311in}}%
\pgfpathlineto{\pgfqpoint{2.634774in}{0.806723in}}%
\pgfpathlineto{\pgfqpoint{2.636788in}{0.807076in}}%
\pgfpathlineto{\pgfqpoint{2.642828in}{0.809502in}}%
\pgfpathlineto{\pgfqpoint{2.644842in}{0.808752in}}%
\pgfpathlineto{\pgfqpoint{2.646855in}{0.811267in}}%
\pgfpathlineto{\pgfqpoint{2.650882in}{0.812988in}}%
\pgfpathlineto{\pgfqpoint{2.656922in}{0.811444in}}%
\pgfpathlineto{\pgfqpoint{2.658936in}{0.809502in}}%
\pgfpathlineto{\pgfqpoint{2.660949in}{0.810164in}}%
\pgfpathlineto{\pgfqpoint{2.662963in}{0.811797in}}%
\pgfpathlineto{\pgfqpoint{2.664976in}{0.811841in}}%
\pgfpathlineto{\pgfqpoint{2.671016in}{0.812635in}}%
\pgfpathlineto{\pgfqpoint{2.675043in}{0.815239in}}%
\pgfpathlineto{\pgfqpoint{2.677057in}{0.814577in}}%
\pgfpathlineto{\pgfqpoint{2.679070in}{0.812679in}}%
\pgfpathlineto{\pgfqpoint{2.685110in}{0.811753in}}%
\pgfpathlineto{\pgfqpoint{2.687124in}{0.809414in}}%
\pgfpathlineto{\pgfqpoint{2.689137in}{0.809238in}}%
\pgfpathlineto{\pgfqpoint{2.693164in}{0.811047in}}%
\pgfpathlineto{\pgfqpoint{2.703232in}{0.811488in}}%
\pgfpathlineto{\pgfqpoint{2.705245in}{0.816562in}}%
\pgfpathlineto{\pgfqpoint{2.713299in}{0.814224in}}%
\pgfpathlineto{\pgfqpoint{2.715312in}{0.816518in}}%
\pgfpathlineto{\pgfqpoint{2.717326in}{0.815239in}}%
\pgfpathlineto{\pgfqpoint{2.719339in}{0.814753in}}%
\pgfpathlineto{\pgfqpoint{2.721353in}{0.815636in}}%
\pgfpathlineto{\pgfqpoint{2.727393in}{0.815945in}}%
\pgfpathlineto{\pgfqpoint{2.733433in}{0.814003in}}%
\pgfpathlineto{\pgfqpoint{2.735447in}{0.816739in}}%
\pgfpathlineto{\pgfqpoint{2.743500in}{0.820357in}}%
\pgfpathlineto{\pgfqpoint{2.745514in}{0.820666in}}%
\pgfpathlineto{\pgfqpoint{2.749541in}{0.822122in}}%
\pgfpathlineto{\pgfqpoint{2.757595in}{0.820798in}}%
\pgfpathlineto{\pgfqpoint{2.761621in}{0.822166in}}%
\pgfpathlineto{\pgfqpoint{2.763635in}{0.823711in}}%
\pgfpathlineto{\pgfqpoint{2.769675in}{0.822652in}}%
\pgfpathlineto{\pgfqpoint{2.771689in}{0.821593in}}%
\pgfpathlineto{\pgfqpoint{2.777729in}{0.822563in}}%
\pgfpathlineto{\pgfqpoint{2.785783in}{0.822916in}}%
\pgfpathlineto{\pgfqpoint{2.789810in}{0.821196in}}%
\pgfpathlineto{\pgfqpoint{2.791823in}{0.820798in}}%
\pgfpathlineto{\pgfqpoint{2.797864in}{0.823137in}}%
\pgfpathlineto{\pgfqpoint{2.799877in}{0.824726in}}%
\pgfpathlineto{\pgfqpoint{2.801890in}{0.823137in}}%
\pgfpathlineto{\pgfqpoint{2.803904in}{0.826446in}}%
\pgfpathlineto{\pgfqpoint{2.805917in}{0.824770in}}%
\pgfpathlineto{\pgfqpoint{2.811958in}{0.825167in}}%
\pgfpathlineto{\pgfqpoint{2.820011in}{0.822652in}}%
\pgfpathlineto{\pgfqpoint{2.826052in}{0.825564in}}%
\pgfpathlineto{\pgfqpoint{2.830079in}{0.829359in}}%
\pgfpathlineto{\pgfqpoint{2.832092in}{0.829491in}}%
\pgfpathlineto{\pgfqpoint{2.834106in}{0.830727in}}%
\pgfpathlineto{\pgfqpoint{2.840146in}{0.831874in}}%
\pgfpathlineto{\pgfqpoint{2.842159in}{0.833330in}}%
\pgfpathlineto{\pgfqpoint{2.844173in}{0.835757in}}%
\pgfpathlineto{\pgfqpoint{2.846186in}{0.834477in}}%
\pgfpathlineto{\pgfqpoint{2.848200in}{0.835139in}}%
\pgfpathlineto{\pgfqpoint{2.860280in}{0.833859in}}%
\pgfpathlineto{\pgfqpoint{2.862294in}{0.832227in}}%
\pgfpathlineto{\pgfqpoint{2.876388in}{0.833815in}}%
\pgfpathlineto{\pgfqpoint{2.882428in}{0.830329in}}%
\pgfpathlineto{\pgfqpoint{2.884442in}{0.830638in}}%
\pgfpathlineto{\pgfqpoint{2.886455in}{0.829006in}}%
\pgfpathlineto{\pgfqpoint{2.888469in}{0.831874in}}%
\pgfpathlineto{\pgfqpoint{2.890482in}{0.832536in}}%
\pgfpathlineto{\pgfqpoint{2.896522in}{0.830638in}}%
\pgfpathlineto{\pgfqpoint{2.898536in}{0.833506in}}%
\pgfpathlineto{\pgfqpoint{2.900549in}{0.834786in}}%
\pgfpathlineto{\pgfqpoint{2.902563in}{0.831035in}}%
\pgfpathlineto{\pgfqpoint{2.904576in}{0.831786in}}%
\pgfpathlineto{\pgfqpoint{2.910617in}{0.830329in}}%
\pgfpathlineto{\pgfqpoint{2.912630in}{0.830771in}}%
\pgfpathlineto{\pgfqpoint{2.914643in}{0.830065in}}%
\pgfpathlineto{\pgfqpoint{2.916657in}{0.831300in}}%
\pgfpathlineto{\pgfqpoint{2.918670in}{0.833286in}}%
\pgfpathlineto{\pgfqpoint{2.926724in}{0.832933in}}%
\pgfpathlineto{\pgfqpoint{2.928738in}{0.830727in}}%
\pgfpathlineto{\pgfqpoint{2.932765in}{0.834742in}}%
\pgfpathlineto{\pgfqpoint{2.938805in}{0.831168in}}%
\pgfpathlineto{\pgfqpoint{2.942832in}{0.835183in}}%
\pgfpathlineto{\pgfqpoint{2.944845in}{0.833948in}}%
\pgfpathlineto{\pgfqpoint{2.946859in}{0.833374in}}%
\pgfpathlineto{\pgfqpoint{2.954912in}{0.835624in}}%
\pgfpathlineto{\pgfqpoint{2.956926in}{0.836375in}}%
\pgfpathlineto{\pgfqpoint{2.958939in}{0.835845in}}%
\pgfpathlineto{\pgfqpoint{2.966993in}{0.836507in}}%
\pgfpathlineto{\pgfqpoint{2.969007in}{0.835139in}}%
\pgfpathlineto{\pgfqpoint{2.971020in}{0.834698in}}%
\pgfpathlineto{\pgfqpoint{2.973033in}{0.835977in}}%
\pgfpathlineto{\pgfqpoint{2.975047in}{0.833859in}}%
\pgfpathlineto{\pgfqpoint{2.981087in}{0.833374in}}%
\pgfpathlineto{\pgfqpoint{2.983101in}{0.831300in}}%
\pgfpathlineto{\pgfqpoint{2.985114in}{0.834654in}}%
\pgfpathlineto{\pgfqpoint{2.987128in}{0.833109in}}%
\pgfpathlineto{\pgfqpoint{2.989141in}{0.835404in}}%
\pgfpathlineto{\pgfqpoint{2.995181in}{0.839331in}}%
\pgfpathlineto{\pgfqpoint{2.997195in}{0.842817in}}%
\pgfpathlineto{\pgfqpoint{3.001222in}{0.845244in}}%
\pgfpathlineto{\pgfqpoint{3.009275in}{0.842729in}}%
\pgfpathlineto{\pgfqpoint{3.011289in}{0.843302in}}%
\pgfpathlineto{\pgfqpoint{3.013302in}{0.840125in}}%
\pgfpathlineto{\pgfqpoint{3.015316in}{0.841846in}}%
\pgfpathlineto{\pgfqpoint{3.017329in}{0.840522in}}%
\pgfpathlineto{\pgfqpoint{3.023370in}{0.841670in}}%
\pgfpathlineto{\pgfqpoint{3.025383in}{0.840125in}}%
\pgfpathlineto{\pgfqpoint{3.027397in}{0.842376in}}%
\pgfpathlineto{\pgfqpoint{3.029410in}{0.842993in}}%
\pgfpathlineto{\pgfqpoint{3.031423in}{0.841096in}}%
\pgfpathlineto{\pgfqpoint{3.037464in}{0.836639in}}%
\pgfpathlineto{\pgfqpoint{3.039477in}{0.840081in}}%
\pgfpathlineto{\pgfqpoint{3.041491in}{0.837566in}}%
\pgfpathlineto{\pgfqpoint{3.043504in}{0.836728in}}%
\pgfpathlineto{\pgfqpoint{3.045518in}{0.839287in}}%
\pgfpathlineto{\pgfqpoint{3.051558in}{0.838890in}}%
\pgfpathlineto{\pgfqpoint{3.055585in}{0.842420in}}%
\pgfpathlineto{\pgfqpoint{3.057598in}{0.844141in}}%
\pgfpathlineto{\pgfqpoint{3.059612in}{0.842596in}}%
\pgfpathlineto{\pgfqpoint{3.067665in}{0.843832in}}%
\pgfpathlineto{\pgfqpoint{3.069679in}{0.841361in}}%
\pgfpathlineto{\pgfqpoint{3.071692in}{0.850803in}}%
\pgfpathlineto{\pgfqpoint{3.073706in}{0.854995in}}%
\pgfpathlineto{\pgfqpoint{3.079746in}{0.854333in}}%
\pgfpathlineto{\pgfqpoint{3.081760in}{0.855216in}}%
\pgfpathlineto{\pgfqpoint{3.087800in}{0.854201in}}%
\pgfpathlineto{\pgfqpoint{3.093840in}{0.854378in}}%
\pgfpathlineto{\pgfqpoint{3.095854in}{0.855790in}}%
\pgfpathlineto{\pgfqpoint{3.097867in}{0.858569in}}%
\pgfpathlineto{\pgfqpoint{3.099881in}{0.856584in}}%
\pgfpathlineto{\pgfqpoint{3.101894in}{0.861393in}}%
\pgfpathlineto{\pgfqpoint{3.107934in}{0.858393in}}%
\pgfpathlineto{\pgfqpoint{3.109948in}{0.858349in}}%
\pgfpathlineto{\pgfqpoint{3.113975in}{0.854863in}}%
\pgfpathlineto{\pgfqpoint{3.115988in}{0.857246in}}%
\pgfpathlineto{\pgfqpoint{3.124042in}{0.856804in}}%
\pgfpathlineto{\pgfqpoint{3.126055in}{0.854907in}}%
\pgfpathlineto{\pgfqpoint{3.128069in}{0.857819in}}%
\pgfpathlineto{\pgfqpoint{3.130082in}{0.854907in}}%
\pgfpathlineto{\pgfqpoint{3.136123in}{0.856981in}}%
\pgfpathlineto{\pgfqpoint{3.138136in}{0.856981in}}%
\pgfpathlineto{\pgfqpoint{3.140150in}{0.853980in}}%
\pgfpathlineto{\pgfqpoint{3.142163in}{0.856143in}}%
\pgfpathlineto{\pgfqpoint{3.144176in}{0.856716in}}%
\pgfpathlineto{\pgfqpoint{3.150217in}{0.859143in}}%
\pgfpathlineto{\pgfqpoint{3.152230in}{0.856451in}}%
\pgfpathlineto{\pgfqpoint{3.154244in}{0.855922in}}%
\pgfpathlineto{\pgfqpoint{3.156257in}{0.859055in}}%
\pgfpathlineto{\pgfqpoint{3.158271in}{0.857731in}}%
\pgfpathlineto{\pgfqpoint{3.164311in}{0.859143in}}%
\pgfpathlineto{\pgfqpoint{3.166324in}{0.860864in}}%
\pgfpathlineto{\pgfqpoint{3.168338in}{0.859452in}}%
\pgfpathlineto{\pgfqpoint{3.170351in}{0.855216in}}%
\pgfpathlineto{\pgfqpoint{3.172365in}{0.856275in}}%
\pgfpathlineto{\pgfqpoint{3.178405in}{0.855039in}}%
\pgfpathlineto{\pgfqpoint{3.184445in}{0.861085in}}%
\pgfpathlineto{\pgfqpoint{3.186459in}{0.859584in}}%
\pgfpathlineto{\pgfqpoint{3.192499in}{0.862055in}}%
\pgfpathlineto{\pgfqpoint{3.194513in}{0.862099in}}%
\pgfpathlineto{\pgfqpoint{3.200553in}{0.867703in}}%
\pgfpathlineto{\pgfqpoint{3.210620in}{0.863556in}}%
\pgfpathlineto{\pgfqpoint{3.212634in}{0.866865in}}%
\pgfpathlineto{\pgfqpoint{3.214647in}{0.867659in}}%
\pgfpathlineto{\pgfqpoint{3.220687in}{0.867174in}}%
\pgfpathlineto{\pgfqpoint{3.222701in}{0.866291in}}%
\pgfpathlineto{\pgfqpoint{3.224714in}{0.866556in}}%
\pgfpathlineto{\pgfqpoint{3.226728in}{0.867924in}}%
\pgfpathlineto{\pgfqpoint{3.228741in}{0.866821in}}%
\pgfpathlineto{\pgfqpoint{3.234782in}{0.868145in}}%
\pgfpathlineto{\pgfqpoint{3.238808in}{0.863644in}}%
\pgfpathlineto{\pgfqpoint{3.240822in}{0.871586in}}%
\pgfpathlineto{\pgfqpoint{3.242835in}{0.870174in}}%
\pgfpathlineto{\pgfqpoint{3.250889in}{0.868365in}}%
\pgfpathlineto{\pgfqpoint{3.252903in}{0.853760in}}%
\pgfpathlineto{\pgfqpoint{3.254916in}{0.855922in}}%
\pgfpathlineto{\pgfqpoint{3.256930in}{0.860820in}}%
\pgfpathlineto{\pgfqpoint{3.262970in}{0.861393in}}%
\pgfpathlineto{\pgfqpoint{3.266997in}{0.858172in}}%
\pgfpathlineto{\pgfqpoint{3.271024in}{0.856584in}}%
\pgfpathlineto{\pgfqpoint{3.281091in}{0.856451in}}%
\pgfpathlineto{\pgfqpoint{3.283104in}{0.851951in}}%
\pgfpathlineto{\pgfqpoint{3.285118in}{0.851112in}}%
\pgfpathlineto{\pgfqpoint{3.291158in}{0.853010in}}%
\pgfpathlineto{\pgfqpoint{3.293172in}{0.851201in}}%
\pgfpathlineto{\pgfqpoint{3.295185in}{0.856496in}}%
\pgfpathlineto{\pgfqpoint{3.299212in}{0.857202in}}%
\pgfpathlineto{\pgfqpoint{3.307266in}{0.852171in}}%
\pgfpathlineto{\pgfqpoint{3.309279in}{0.853230in}}%
\pgfpathlineto{\pgfqpoint{3.311293in}{0.853407in}}%
\pgfpathlineto{\pgfqpoint{3.313306in}{0.852436in}}%
\pgfpathlineto{\pgfqpoint{3.321360in}{0.855084in}}%
\pgfpathlineto{\pgfqpoint{3.323373in}{0.854245in}}%
\pgfpathlineto{\pgfqpoint{3.327400in}{0.854731in}}%
\pgfpathlineto{\pgfqpoint{3.333441in}{0.856672in}}%
\pgfpathlineto{\pgfqpoint{3.335454in}{0.862805in}}%
\pgfpathlineto{\pgfqpoint{3.337467in}{0.864570in}}%
\pgfpathlineto{\pgfqpoint{3.339481in}{0.863291in}}%
\pgfpathlineto{\pgfqpoint{3.341494in}{0.867306in}}%
\pgfpathlineto{\pgfqpoint{3.347535in}{0.867703in}}%
\pgfpathlineto{\pgfqpoint{3.351562in}{0.872072in}}%
\pgfpathlineto{\pgfqpoint{3.353575in}{0.873175in}}%
\pgfpathlineto{\pgfqpoint{3.355588in}{0.877367in}}%
\pgfpathlineto{\pgfqpoint{3.361629in}{0.874719in}}%
\pgfpathlineto{\pgfqpoint{3.363642in}{0.871763in}}%
\pgfpathlineto{\pgfqpoint{3.365656in}{0.869954in}}%
\pgfpathlineto{\pgfqpoint{3.367669in}{0.871983in}}%
\pgfpathlineto{\pgfqpoint{3.375723in}{0.869468in}}%
\pgfpathlineto{\pgfqpoint{3.377736in}{0.871630in}}%
\pgfpathlineto{\pgfqpoint{3.379750in}{0.870218in}}%
\pgfpathlineto{\pgfqpoint{3.381763in}{0.866953in}}%
\pgfpathlineto{\pgfqpoint{3.383777in}{0.868851in}}%
\pgfpathlineto{\pgfqpoint{3.389817in}{0.864615in}}%
\pgfpathlineto{\pgfqpoint{3.391830in}{0.860732in}}%
\pgfpathlineto{\pgfqpoint{3.393844in}{0.862055in}}%
\pgfpathlineto{\pgfqpoint{3.395857in}{0.867350in}}%
\pgfpathlineto{\pgfqpoint{3.397871in}{0.870571in}}%
\pgfpathlineto{\pgfqpoint{3.403911in}{0.871851in}}%
\pgfpathlineto{\pgfqpoint{3.405925in}{0.869733in}}%
\pgfpathlineto{\pgfqpoint{3.407938in}{0.874101in}}%
\pgfpathlineto{\pgfqpoint{3.411965in}{0.878999in}}%
\pgfpathlineto{\pgfqpoint{3.420019in}{0.878514in}}%
\pgfpathlineto{\pgfqpoint{3.422032in}{0.877411in}}%
\pgfpathlineto{\pgfqpoint{3.424046in}{0.881514in}}%
\pgfpathlineto{\pgfqpoint{3.436126in}{0.882441in}}%
\pgfpathlineto{\pgfqpoint{3.438140in}{0.875160in}}%
\pgfpathlineto{\pgfqpoint{3.440153in}{0.877720in}}%
\pgfpathlineto{\pgfqpoint{3.446194in}{0.874896in}}%
\pgfpathlineto{\pgfqpoint{3.448207in}{0.875866in}}%
\pgfpathlineto{\pgfqpoint{3.450220in}{0.877543in}}%
\pgfpathlineto{\pgfqpoint{3.452234in}{0.874587in}}%
\pgfpathlineto{\pgfqpoint{3.454247in}{0.876793in}}%
\pgfpathlineto{\pgfqpoint{3.460288in}{0.877808in}}%
\pgfpathlineto{\pgfqpoint{3.462301in}{0.876793in}}%
\pgfpathlineto{\pgfqpoint{3.464315in}{0.879970in}}%
\pgfpathlineto{\pgfqpoint{3.466328in}{0.880367in}}%
\pgfpathlineto{\pgfqpoint{3.468341in}{0.882265in}}%
\pgfpathlineto{\pgfqpoint{3.474382in}{0.880102in}}%
\pgfpathlineto{\pgfqpoint{3.476395in}{0.877543in}}%
\pgfpathlineto{\pgfqpoint{3.478409in}{0.878293in}}%
\pgfpathlineto{\pgfqpoint{3.480422in}{0.881867in}}%
\pgfpathlineto{\pgfqpoint{3.482436in}{0.882397in}}%
\pgfpathlineto{\pgfqpoint{3.488476in}{0.882220in}}%
\pgfpathlineto{\pgfqpoint{3.490489in}{0.883853in}}%
\pgfpathlineto{\pgfqpoint{3.492503in}{0.884338in}}%
\pgfpathlineto{\pgfqpoint{3.496530in}{0.883765in}}%
\pgfpathlineto{\pgfqpoint{3.502570in}{0.885486in}}%
\pgfpathlineto{\pgfqpoint{3.504584in}{0.882000in}}%
\pgfpathlineto{\pgfqpoint{3.506597in}{0.883059in}}%
\pgfpathlineto{\pgfqpoint{3.508610in}{0.881956in}}%
\pgfpathlineto{\pgfqpoint{3.510624in}{0.882132in}}%
\pgfpathlineto{\pgfqpoint{3.516664in}{0.881867in}}%
\pgfpathlineto{\pgfqpoint{3.518678in}{0.879617in}}%
\pgfpathlineto{\pgfqpoint{3.520691in}{0.886059in}}%
\pgfpathlineto{\pgfqpoint{3.522705in}{0.883765in}}%
\pgfpathlineto{\pgfqpoint{3.524718in}{0.887780in}}%
\pgfpathlineto{\pgfqpoint{3.530758in}{0.888177in}}%
\pgfpathlineto{\pgfqpoint{3.532772in}{0.893869in}}%
\pgfpathlineto{\pgfqpoint{3.534785in}{0.896032in}}%
\pgfpathlineto{\pgfqpoint{3.536799in}{0.896782in}}%
\pgfpathlineto{\pgfqpoint{3.538812in}{0.896649in}}%
\pgfpathlineto{\pgfqpoint{3.550893in}{0.901018in}}%
\pgfpathlineto{\pgfqpoint{3.552906in}{0.900444in}}%
\pgfpathlineto{\pgfqpoint{3.558947in}{0.901724in}}%
\pgfpathlineto{\pgfqpoint{3.560960in}{0.903577in}}%
\pgfpathlineto{\pgfqpoint{3.564987in}{0.901856in}}%
\pgfpathlineto{\pgfqpoint{3.567000in}{0.901988in}}%
\pgfpathlineto{\pgfqpoint{3.573041in}{0.900621in}}%
\pgfpathlineto{\pgfqpoint{3.575054in}{0.902341in}}%
\pgfpathlineto{\pgfqpoint{3.577068in}{0.903092in}}%
\pgfpathlineto{\pgfqpoint{3.581095in}{0.901988in}}%
\pgfpathlineto{\pgfqpoint{3.587135in}{0.899694in}}%
\pgfpathlineto{\pgfqpoint{3.589148in}{0.903533in}}%
\pgfpathlineto{\pgfqpoint{3.591162in}{0.904680in}}%
\pgfpathlineto{\pgfqpoint{3.593175in}{0.902606in}}%
\pgfpathlineto{\pgfqpoint{3.595189in}{0.913329in}}%
\pgfpathlineto{\pgfqpoint{3.603242in}{0.913064in}}%
\pgfpathlineto{\pgfqpoint{3.605256in}{0.914167in}}%
\pgfpathlineto{\pgfqpoint{3.607269in}{0.909313in}}%
\pgfpathlineto{\pgfqpoint{3.609283in}{0.902077in}}%
\pgfpathlineto{\pgfqpoint{3.615323in}{0.896914in}}%
\pgfpathlineto{\pgfqpoint{3.617337in}{0.901812in}}%
\pgfpathlineto{\pgfqpoint{3.619350in}{0.897841in}}%
\pgfpathlineto{\pgfqpoint{3.621363in}{0.901724in}}%
\pgfpathlineto{\pgfqpoint{3.623377in}{0.896120in}}%
\pgfpathlineto{\pgfqpoint{3.629417in}{0.894090in}}%
\pgfpathlineto{\pgfqpoint{3.633444in}{0.896296in}}%
\pgfpathlineto{\pgfqpoint{3.637471in}{0.902650in}}%
\pgfpathlineto{\pgfqpoint{3.643511in}{0.901371in}}%
\pgfpathlineto{\pgfqpoint{3.645525in}{0.903224in}}%
\pgfpathlineto{\pgfqpoint{3.647538in}{0.906754in}}%
\pgfpathlineto{\pgfqpoint{3.649552in}{0.906666in}}%
\pgfpathlineto{\pgfqpoint{3.651565in}{0.908651in}}%
\pgfpathlineto{\pgfqpoint{3.659619in}{0.908740in}}%
\pgfpathlineto{\pgfqpoint{3.661632in}{0.906489in}}%
\pgfpathlineto{\pgfqpoint{3.665659in}{0.905916in}}%
\pgfpathlineto{\pgfqpoint{3.673713in}{0.909799in}}%
\pgfpathlineto{\pgfqpoint{3.675727in}{0.908784in}}%
\pgfpathlineto{\pgfqpoint{3.679753in}{0.908607in}}%
\pgfpathlineto{\pgfqpoint{3.685794in}{0.903930in}}%
\pgfpathlineto{\pgfqpoint{3.687807in}{0.908166in}}%
\pgfpathlineto{\pgfqpoint{3.689821in}{0.905342in}}%
\pgfpathlineto{\pgfqpoint{3.691834in}{0.906445in}}%
\pgfpathlineto{\pgfqpoint{3.693848in}{0.908210in}}%
\pgfpathlineto{\pgfqpoint{3.699888in}{0.908166in}}%
\pgfpathlineto{\pgfqpoint{3.701901in}{0.909887in}}%
\pgfpathlineto{\pgfqpoint{3.703915in}{0.908784in}}%
\pgfpathlineto{\pgfqpoint{3.705928in}{0.903268in}}%
\pgfpathlineto{\pgfqpoint{3.707942in}{0.903224in}}%
\pgfpathlineto{\pgfqpoint{3.713982in}{0.906357in}}%
\pgfpathlineto{\pgfqpoint{3.715995in}{0.909093in}}%
\pgfpathlineto{\pgfqpoint{3.720022in}{0.904327in}}%
\pgfpathlineto{\pgfqpoint{3.722036in}{0.905960in}}%
\pgfpathlineto{\pgfqpoint{3.728076in}{0.903224in}}%
\pgfpathlineto{\pgfqpoint{3.732103in}{0.897973in}}%
\pgfpathlineto{\pgfqpoint{3.734116in}{0.898061in}}%
\pgfpathlineto{\pgfqpoint{3.736130in}{0.894267in}}%
\pgfpathlineto{\pgfqpoint{3.742170in}{0.898150in}}%
\pgfpathlineto{\pgfqpoint{3.744184in}{0.896958in}}%
\pgfpathlineto{\pgfqpoint{3.748211in}{0.897267in}}%
\pgfpathlineto{\pgfqpoint{3.750224in}{0.889678in}}%
\pgfpathlineto{\pgfqpoint{3.758278in}{0.884383in}}%
\pgfpathlineto{\pgfqpoint{3.760291in}{0.889545in}}%
\pgfpathlineto{\pgfqpoint{3.764318in}{0.878205in}}%
\pgfpathlineto{\pgfqpoint{3.770359in}{0.882750in}}%
\pgfpathlineto{\pgfqpoint{3.772372in}{0.885927in}}%
\pgfpathlineto{\pgfqpoint{3.774385in}{0.891398in}}%
\pgfpathlineto{\pgfqpoint{3.776399in}{0.889942in}}%
\pgfpathlineto{\pgfqpoint{3.784453in}{0.891178in}}%
\pgfpathlineto{\pgfqpoint{3.786466in}{0.892060in}}%
\pgfpathlineto{\pgfqpoint{3.788480in}{0.890869in}}%
\pgfpathlineto{\pgfqpoint{3.790493in}{0.891443in}}%
\pgfpathlineto{\pgfqpoint{3.792506in}{0.880588in}}%
\pgfpathlineto{\pgfqpoint{3.802574in}{0.884427in}}%
\pgfpathlineto{\pgfqpoint{3.804587in}{0.888045in}}%
\pgfpathlineto{\pgfqpoint{3.806601in}{0.886280in}}%
\pgfpathlineto{\pgfqpoint{3.812641in}{0.889104in}}%
\pgfpathlineto{\pgfqpoint{3.814654in}{0.887251in}}%
\pgfpathlineto{\pgfqpoint{3.816668in}{0.890737in}}%
\pgfpathlineto{\pgfqpoint{3.818681in}{0.893031in}}%
\pgfpathlineto{\pgfqpoint{3.820695in}{0.892899in}}%
\pgfpathlineto{\pgfqpoint{3.828749in}{0.894222in}}%
\pgfpathlineto{\pgfqpoint{3.830762in}{0.893605in}}%
\pgfpathlineto{\pgfqpoint{3.832775in}{0.891090in}}%
\pgfpathlineto{\pgfqpoint{3.834789in}{0.893516in}}%
\pgfpathlineto{\pgfqpoint{3.840829in}{0.894090in}}%
\pgfpathlineto{\pgfqpoint{3.842843in}{0.891575in}}%
\pgfpathlineto{\pgfqpoint{3.844856in}{0.893825in}}%
\pgfpathlineto{\pgfqpoint{3.846870in}{0.893075in}}%
\pgfpathlineto{\pgfqpoint{3.848883in}{0.895899in}}%
\pgfpathlineto{\pgfqpoint{3.856937in}{0.898414in}}%
\pgfpathlineto{\pgfqpoint{3.858950in}{0.897708in}}%
\pgfpathlineto{\pgfqpoint{3.860964in}{0.898591in}}%
\pgfpathlineto{\pgfqpoint{3.862977in}{0.898767in}}%
\pgfpathlineto{\pgfqpoint{3.869017in}{0.897399in}}%
\pgfpathlineto{\pgfqpoint{3.871031in}{0.895105in}}%
\pgfpathlineto{\pgfqpoint{3.873044in}{0.895237in}}%
\pgfpathlineto{\pgfqpoint{3.877071in}{0.896870in}}%
\pgfpathlineto{\pgfqpoint{3.883112in}{0.896429in}}%
\pgfpathlineto{\pgfqpoint{3.885125in}{0.898194in}}%
\pgfpathlineto{\pgfqpoint{3.887138in}{0.896517in}}%
\pgfpathlineto{\pgfqpoint{3.891165in}{0.895061in}}%
\pgfpathlineto{\pgfqpoint{3.897206in}{0.894002in}}%
\pgfpathlineto{\pgfqpoint{3.901233in}{0.895061in}}%
\pgfpathlineto{\pgfqpoint{3.905260in}{0.893208in}}%
\pgfpathlineto{\pgfqpoint{3.911300in}{0.893163in}}%
\pgfpathlineto{\pgfqpoint{3.913313in}{0.891443in}}%
\pgfpathlineto{\pgfqpoint{3.915327in}{0.892766in}}%
\pgfpathlineto{\pgfqpoint{3.919354in}{0.892987in}}%
\pgfpathlineto{\pgfqpoint{3.925394in}{0.894487in}}%
\pgfpathlineto{\pgfqpoint{3.927407in}{0.898194in}}%
\pgfpathlineto{\pgfqpoint{3.929421in}{0.898811in}}%
\pgfpathlineto{\pgfqpoint{3.931434in}{0.900532in}}%
\pgfpathlineto{\pgfqpoint{3.939488in}{0.900753in}}%
\pgfpathlineto{\pgfqpoint{3.941502in}{0.899209in}}%
\pgfpathlineto{\pgfqpoint{3.943515in}{0.900179in}}%
\pgfpathlineto{\pgfqpoint{3.945528in}{0.899517in}}%
\pgfpathlineto{\pgfqpoint{3.953582in}{0.905254in}}%
\pgfpathlineto{\pgfqpoint{3.957609in}{0.906975in}}%
\pgfpathlineto{\pgfqpoint{3.959623in}{0.901944in}}%
\pgfpathlineto{\pgfqpoint{3.961636in}{0.904371in}}%
\pgfpathlineto{\pgfqpoint{3.967676in}{0.903356in}}%
\pgfpathlineto{\pgfqpoint{3.969690in}{0.905518in}}%
\pgfpathlineto{\pgfqpoint{3.971703in}{0.905430in}}%
\pgfpathlineto{\pgfqpoint{3.973717in}{0.907063in}}%
\pgfpathlineto{\pgfqpoint{3.975730in}{0.898723in}}%
\pgfpathlineto{\pgfqpoint{3.981771in}{0.898150in}}%
\pgfpathlineto{\pgfqpoint{3.983784in}{0.897399in}}%
\pgfpathlineto{\pgfqpoint{3.985797in}{0.897973in}}%
\pgfpathlineto{\pgfqpoint{3.987811in}{0.894796in}}%
\pgfpathlineto{\pgfqpoint{3.989824in}{0.895634in}}%
\pgfpathlineto{\pgfqpoint{3.995865in}{0.895987in}}%
\pgfpathlineto{\pgfqpoint{3.997878in}{0.894311in}}%
\pgfpathlineto{\pgfqpoint{3.999892in}{0.894443in}}%
\pgfpathlineto{\pgfqpoint{4.001905in}{0.892722in}}%
\pgfpathlineto{\pgfqpoint{4.003918in}{0.894222in}}%
\pgfpathlineto{\pgfqpoint{4.011972in}{0.894046in}}%
\pgfpathlineto{\pgfqpoint{4.013986in}{0.896914in}}%
\pgfpathlineto{\pgfqpoint{4.015999in}{0.898061in}}%
\pgfpathlineto{\pgfqpoint{4.018013in}{0.895634in}}%
\pgfpathlineto{\pgfqpoint{4.024053in}{0.899959in}}%
\pgfpathlineto{\pgfqpoint{4.028080in}{0.901944in}}%
\pgfpathlineto{\pgfqpoint{4.030093in}{0.901459in}}%
\pgfpathlineto{\pgfqpoint{4.032107in}{0.901812in}}%
\pgfpathlineto{\pgfqpoint{4.038147in}{0.901768in}}%
\pgfpathlineto{\pgfqpoint{4.042174in}{0.902915in}}%
\pgfpathlineto{\pgfqpoint{4.046201in}{0.898061in}}%
\pgfpathlineto{\pgfqpoint{4.056268in}{0.900576in}}%
\pgfpathlineto{\pgfqpoint{4.060295in}{0.899826in}}%
\pgfpathlineto{\pgfqpoint{4.066335in}{0.901503in}}%
\pgfpathlineto{\pgfqpoint{4.068349in}{0.899915in}}%
\pgfpathlineto{\pgfqpoint{4.070362in}{0.902650in}}%
\pgfpathlineto{\pgfqpoint{4.072376in}{0.900621in}}%
\pgfpathlineto{\pgfqpoint{4.074389in}{0.899650in}}%
\pgfpathlineto{\pgfqpoint{4.080429in}{0.900312in}}%
\pgfpathlineto{\pgfqpoint{4.082443in}{0.903092in}}%
\pgfpathlineto{\pgfqpoint{4.084456in}{0.901238in}}%
\pgfpathlineto{\pgfqpoint{4.086470in}{0.902209in}}%
\pgfpathlineto{\pgfqpoint{4.088483in}{0.901988in}}%
\pgfpathlineto{\pgfqpoint{4.094524in}{0.899517in}}%
\pgfpathlineto{\pgfqpoint{4.096537in}{0.898017in}}%
\pgfpathlineto{\pgfqpoint{4.098550in}{0.900047in}}%
\pgfpathlineto{\pgfqpoint{4.100564in}{0.896164in}}%
\pgfpathlineto{\pgfqpoint{4.102577in}{0.897444in}}%
\pgfpathlineto{\pgfqpoint{4.108618in}{0.896385in}}%
\pgfpathlineto{\pgfqpoint{4.110631in}{0.898944in}}%
\pgfpathlineto{\pgfqpoint{4.112645in}{0.895414in}}%
\pgfpathlineto{\pgfqpoint{4.114658in}{0.895105in}}%
\pgfpathlineto{\pgfqpoint{4.116671in}{0.897532in}}%
\pgfpathlineto{\pgfqpoint{4.122712in}{0.897311in}}%
\pgfpathlineto{\pgfqpoint{4.124725in}{0.893384in}}%
\pgfpathlineto{\pgfqpoint{4.126739in}{0.897885in}}%
\pgfpathlineto{\pgfqpoint{4.128752in}{0.892987in}}%
\pgfpathlineto{\pgfqpoint{4.130766in}{0.890119in}}%
\pgfpathlineto{\pgfqpoint{4.136806in}{0.889413in}}%
\pgfpathlineto{\pgfqpoint{4.138819in}{0.887736in}}%
\pgfpathlineto{\pgfqpoint{4.140833in}{0.885133in}}%
\pgfpathlineto{\pgfqpoint{4.144860in}{0.891222in}}%
\pgfpathlineto{\pgfqpoint{4.150900in}{0.893163in}}%
\pgfpathlineto{\pgfqpoint{4.152914in}{0.898900in}}%
\pgfpathlineto{\pgfqpoint{4.154927in}{0.896385in}}%
\pgfpathlineto{\pgfqpoint{4.156940in}{0.899915in}}%
\pgfpathlineto{\pgfqpoint{4.158954in}{0.899076in}}%
\pgfpathlineto{\pgfqpoint{4.164994in}{0.898988in}}%
\pgfpathlineto{\pgfqpoint{4.167008in}{0.902474in}}%
\pgfpathlineto{\pgfqpoint{4.169021in}{0.900312in}}%
\pgfpathlineto{\pgfqpoint{4.171035in}{0.923477in}}%
\pgfpathlineto{\pgfqpoint{4.173048in}{0.928507in}}%
\pgfpathlineto{\pgfqpoint{4.179088in}{0.928596in}}%
\pgfpathlineto{\pgfqpoint{4.181102in}{0.930140in}}%
\pgfpathlineto{\pgfqpoint{4.183115in}{0.937024in}}%
\pgfpathlineto{\pgfqpoint{4.185129in}{0.937597in}}%
\pgfpathlineto{\pgfqpoint{4.187142in}{0.940068in}}%
\pgfpathlineto{\pgfqpoint{4.195196in}{0.937244in}}%
\pgfpathlineto{\pgfqpoint{4.197209in}{0.941613in}}%
\pgfpathlineto{\pgfqpoint{4.199223in}{0.940554in}}%
\pgfpathlineto{\pgfqpoint{4.201236in}{0.938347in}}%
\pgfpathlineto{\pgfqpoint{4.209290in}{0.939274in}}%
\pgfpathlineto{\pgfqpoint{4.211303in}{0.939406in}}%
\pgfpathlineto{\pgfqpoint{4.215330in}{0.944039in}}%
\pgfpathlineto{\pgfqpoint{4.221371in}{0.944481in}}%
\pgfpathlineto{\pgfqpoint{4.223384in}{0.947261in}}%
\pgfpathlineto{\pgfqpoint{4.225398in}{0.947261in}}%
\pgfpathlineto{\pgfqpoint{4.229425in}{0.948231in}}%
\pgfpathlineto{\pgfqpoint{4.235465in}{0.948187in}}%
\pgfpathlineto{\pgfqpoint{4.239492in}{0.951850in}}%
\pgfpathlineto{\pgfqpoint{4.241505in}{0.951364in}}%
\pgfpathlineto{\pgfqpoint{4.243519in}{0.953703in}}%
\pgfpathlineto{\pgfqpoint{4.249559in}{0.953438in}}%
\pgfpathlineto{\pgfqpoint{4.251572in}{0.954541in}}%
\pgfpathlineto{\pgfqpoint{4.253586in}{0.951982in}}%
\pgfpathlineto{\pgfqpoint{4.255599in}{0.953482in}}%
\pgfpathlineto{\pgfqpoint{4.257613in}{0.946775in}}%
\pgfpathlineto{\pgfqpoint{4.263653in}{0.946687in}}%
\pgfpathlineto{\pgfqpoint{4.265667in}{0.943201in}}%
\pgfpathlineto{\pgfqpoint{4.269693in}{0.954585in}}%
\pgfpathlineto{\pgfqpoint{4.271707in}{0.951938in}}%
\pgfpathlineto{\pgfqpoint{4.279761in}{0.955733in}}%
\pgfpathlineto{\pgfqpoint{4.281774in}{0.958248in}}%
\pgfpathlineto{\pgfqpoint{4.293855in}{0.954982in}}%
\pgfpathlineto{\pgfqpoint{4.295868in}{0.952511in}}%
\pgfpathlineto{\pgfqpoint{4.299895in}{0.955512in}}%
\pgfpathlineto{\pgfqpoint{4.307949in}{0.947525in}}%
\pgfpathlineto{\pgfqpoint{4.311976in}{0.954894in}}%
\pgfpathlineto{\pgfqpoint{4.313989in}{0.950746in}}%
\pgfpathlineto{\pgfqpoint{4.320030in}{0.950173in}}%
\pgfpathlineto{\pgfqpoint{4.322043in}{0.951011in}}%
\pgfpathlineto{\pgfqpoint{4.324057in}{0.945451in}}%
\pgfpathlineto{\pgfqpoint{4.326070in}{0.942892in}}%
\pgfpathlineto{\pgfqpoint{4.328083in}{0.944834in}}%
\pgfpathlineto{\pgfqpoint{4.342178in}{0.948364in}}%
\pgfpathlineto{\pgfqpoint{4.348218in}{0.946510in}}%
\pgfpathlineto{\pgfqpoint{4.352245in}{0.935700in}}%
\pgfpathlineto{\pgfqpoint{4.354258in}{0.937421in}}%
\pgfpathlineto{\pgfqpoint{4.356272in}{0.944790in}}%
\pgfpathlineto{\pgfqpoint{4.362312in}{0.945231in}}%
\pgfpathlineto{\pgfqpoint{4.366339in}{0.955380in}}%
\pgfpathlineto{\pgfqpoint{4.368352in}{0.962704in}}%
\pgfpathlineto{\pgfqpoint{4.370366in}{0.958027in}}%
\pgfpathlineto{\pgfqpoint{4.378420in}{0.954982in}}%
\pgfpathlineto{\pgfqpoint{4.380433in}{0.958601in}}%
\pgfpathlineto{\pgfqpoint{4.382447in}{0.963852in}}%
\pgfpathlineto{\pgfqpoint{4.384460in}{0.962484in}}%
\pgfpathlineto{\pgfqpoint{4.392514in}{0.963807in}}%
\pgfpathlineto{\pgfqpoint{4.394527in}{0.961954in}}%
\pgfpathlineto{\pgfqpoint{4.396541in}{0.961910in}}%
\pgfpathlineto{\pgfqpoint{4.398554in}{0.966058in}}%
\pgfpathlineto{\pgfqpoint{4.404594in}{0.966102in}}%
\pgfpathlineto{\pgfqpoint{4.406608in}{0.965528in}}%
\pgfpathlineto{\pgfqpoint{4.410635in}{0.966896in}}%
\pgfpathlineto{\pgfqpoint{4.412648in}{0.964293in}}%
\pgfpathlineto{\pgfqpoint{4.418689in}{0.971750in}}%
\pgfpathlineto{\pgfqpoint{4.420702in}{0.968529in}}%
\pgfpathlineto{\pgfqpoint{4.422715in}{0.966896in}}%
\pgfpathlineto{\pgfqpoint{4.424729in}{0.967293in}}%
\pgfpathlineto{\pgfqpoint{4.426742in}{0.962175in}}%
\pgfpathlineto{\pgfqpoint{4.432783in}{0.964381in}}%
\pgfpathlineto{\pgfqpoint{4.434796in}{0.957762in}}%
\pgfpathlineto{\pgfqpoint{4.436810in}{0.957277in}}%
\pgfpathlineto{\pgfqpoint{4.438823in}{0.962440in}}%
\pgfpathlineto{\pgfqpoint{4.440836in}{0.957542in}}%
\pgfpathlineto{\pgfqpoint{4.446877in}{0.961822in}}%
\pgfpathlineto{\pgfqpoint{4.448890in}{0.957012in}}%
\pgfpathlineto{\pgfqpoint{4.450904in}{0.960366in}}%
\pgfpathlineto{\pgfqpoint{4.452917in}{0.959924in}}%
\pgfpathlineto{\pgfqpoint{4.454931in}{0.962484in}}%
\pgfpathlineto{\pgfqpoint{4.462984in}{0.961160in}}%
\pgfpathlineto{\pgfqpoint{4.464998in}{0.955291in}}%
\pgfpathlineto{\pgfqpoint{4.469025in}{0.954497in}}%
\pgfpathlineto{\pgfqpoint{4.475065in}{0.955027in}}%
\pgfpathlineto{\pgfqpoint{4.479092in}{0.952953in}}%
\pgfpathlineto{\pgfqpoint{4.481105in}{0.953438in}}%
\pgfpathlineto{\pgfqpoint{4.489159in}{0.953041in}}%
\pgfpathlineto{\pgfqpoint{4.493186in}{0.958601in}}%
\pgfpathlineto{\pgfqpoint{4.497213in}{0.957895in}}%
\pgfpathlineto{\pgfqpoint{4.503253in}{0.954806in}}%
\pgfpathlineto{\pgfqpoint{4.505267in}{0.954453in}}%
\pgfpathlineto{\pgfqpoint{4.507280in}{0.955115in}}%
\pgfpathlineto{\pgfqpoint{4.509294in}{0.954982in}}%
\pgfpathlineto{\pgfqpoint{4.511307in}{0.950129in}}%
\pgfpathlineto{\pgfqpoint{4.517347in}{0.951011in}}%
\pgfpathlineto{\pgfqpoint{4.519361in}{0.953703in}}%
\pgfpathlineto{\pgfqpoint{4.521374in}{0.965043in}}%
\pgfpathlineto{\pgfqpoint{4.525401in}{0.962793in}}%
\pgfpathlineto{\pgfqpoint{4.531442in}{0.961160in}}%
\pgfpathlineto{\pgfqpoint{4.533455in}{0.959748in}}%
\pgfpathlineto{\pgfqpoint{4.535468in}{0.962175in}}%
\pgfpathlineto{\pgfqpoint{4.537482in}{0.956659in}}%
\pgfpathlineto{\pgfqpoint{4.539495in}{0.955468in}}%
\pgfpathlineto{\pgfqpoint{4.545536in}{0.954674in}}%
\pgfpathlineto{\pgfqpoint{4.547549in}{0.956306in}}%
\pgfpathlineto{\pgfqpoint{4.549563in}{0.955071in}}%
\pgfpathlineto{\pgfqpoint{4.551576in}{0.958954in}}%
\pgfpathlineto{\pgfqpoint{4.553590in}{0.971309in}}%
\pgfpathlineto{\pgfqpoint{4.559630in}{0.969367in}}%
\pgfpathlineto{\pgfqpoint{4.561643in}{0.967955in}}%
\pgfpathlineto{\pgfqpoint{4.563657in}{0.968308in}}%
\pgfpathlineto{\pgfqpoint{4.565670in}{0.974089in}}%
\pgfpathlineto{\pgfqpoint{4.567684in}{0.972235in}}%
\pgfpathlineto{\pgfqpoint{4.575737in}{0.974750in}}%
\pgfpathlineto{\pgfqpoint{4.579764in}{0.971397in}}%
\pgfpathlineto{\pgfqpoint{4.581778in}{0.972456in}}%
\pgfpathlineto{\pgfqpoint{4.589832in}{0.967823in}}%
\pgfpathlineto{\pgfqpoint{4.591845in}{0.971882in}}%
\pgfpathlineto{\pgfqpoint{4.593858in}{0.972191in}}%
\pgfpathlineto{\pgfqpoint{4.595872in}{0.968441in}}%
\pgfpathlineto{\pgfqpoint{4.601912in}{0.970294in}}%
\pgfpathlineto{\pgfqpoint{4.605939in}{0.969588in}}%
\pgfpathlineto{\pgfqpoint{4.607953in}{0.966411in}}%
\pgfpathlineto{\pgfqpoint{4.609966in}{0.967117in}}%
\pgfpathlineto{\pgfqpoint{4.616006in}{0.964160in}}%
\pgfpathlineto{\pgfqpoint{4.618020in}{0.965308in}}%
\pgfpathlineto{\pgfqpoint{4.620033in}{0.972500in}}%
\pgfpathlineto{\pgfqpoint{4.622047in}{0.972544in}}%
\pgfpathlineto{\pgfqpoint{4.624060in}{0.971220in}}%
\pgfpathlineto{\pgfqpoint{4.630101in}{0.967955in}}%
\pgfpathlineto{\pgfqpoint{4.632114in}{0.969853in}}%
\pgfpathlineto{\pgfqpoint{4.634127in}{0.968749in}}%
\pgfpathlineto{\pgfqpoint{4.636141in}{0.971926in}}%
\pgfpathlineto{\pgfqpoint{4.638154in}{0.968485in}}%
\pgfpathlineto{\pgfqpoint{4.644195in}{0.970206in}}%
\pgfpathlineto{\pgfqpoint{4.646208in}{0.971618in}}%
\pgfpathlineto{\pgfqpoint{4.648222in}{0.969191in}}%
\pgfpathlineto{\pgfqpoint{4.650235in}{0.968176in}}%
\pgfpathlineto{\pgfqpoint{4.652248in}{0.968749in}}%
\pgfpathlineto{\pgfqpoint{4.658289in}{0.960057in}}%
\pgfpathlineto{\pgfqpoint{4.660302in}{0.961866in}}%
\pgfpathlineto{\pgfqpoint{4.662316in}{0.964866in}}%
\pgfpathlineto{\pgfqpoint{4.664329in}{0.966543in}}%
\pgfpathlineto{\pgfqpoint{4.672383in}{0.966014in}}%
\pgfpathlineto{\pgfqpoint{4.674396in}{0.964513in}}%
\pgfpathlineto{\pgfqpoint{4.676410in}{0.960101in}}%
\pgfpathlineto{\pgfqpoint{4.678423in}{0.961469in}}%
\pgfpathlineto{\pgfqpoint{4.680437in}{0.967337in}}%
\pgfpathlineto{\pgfqpoint{4.686477in}{0.972015in}}%
\pgfpathlineto{\pgfqpoint{4.688490in}{0.974530in}}%
\pgfpathlineto{\pgfqpoint{4.690504in}{0.974177in}}%
\pgfpathlineto{\pgfqpoint{4.692517in}{0.976515in}}%
\pgfpathlineto{\pgfqpoint{4.694531in}{0.977839in}}%
\pgfpathlineto{\pgfqpoint{4.700571in}{0.985649in}}%
\pgfpathlineto{\pgfqpoint{4.702585in}{0.982737in}}%
\pgfpathlineto{\pgfqpoint{4.704598in}{0.982516in}}%
\pgfpathlineto{\pgfqpoint{4.706612in}{0.981590in}}%
\pgfpathlineto{\pgfqpoint{4.708625in}{0.994651in}}%
\pgfpathlineto{\pgfqpoint{4.714665in}{0.990591in}}%
\pgfpathlineto{\pgfqpoint{4.716679in}{0.994386in}}%
\pgfpathlineto{\pgfqpoint{4.718692in}{0.999813in}}%
\pgfpathlineto{\pgfqpoint{4.720706in}{1.001402in}}%
\pgfpathlineto{\pgfqpoint{4.722719in}{0.996945in}}%
\pgfpathlineto{\pgfqpoint{4.728759in}{0.998799in}}%
\pgfpathlineto{\pgfqpoint{4.730773in}{0.996636in}}%
\pgfpathlineto{\pgfqpoint{4.732786in}{0.995401in}}%
\pgfpathlineto{\pgfqpoint{4.734800in}{0.989400in}}%
\pgfpathlineto{\pgfqpoint{4.736813in}{0.992092in}}%
\pgfpathlineto{\pgfqpoint{4.742854in}{0.992842in}}%
\pgfpathlineto{\pgfqpoint{4.744867in}{0.988473in}}%
\pgfpathlineto{\pgfqpoint{4.750907in}{0.992665in}}%
\pgfpathlineto{\pgfqpoint{4.758961in}{0.993724in}}%
\pgfpathlineto{\pgfqpoint{4.760975in}{0.993459in}}%
\pgfpathlineto{\pgfqpoint{4.762988in}{0.991518in}}%
\pgfpathlineto{\pgfqpoint{4.765001in}{0.979692in}}%
\pgfpathlineto{\pgfqpoint{4.771042in}{0.967558in}}%
\pgfpathlineto{\pgfqpoint{4.773055in}{0.961557in}}%
\pgfpathlineto{\pgfqpoint{4.775069in}{0.977530in}}%
\pgfpathlineto{\pgfqpoint{4.777082in}{0.984855in}}%
\pgfpathlineto{\pgfqpoint{4.779096in}{0.985120in}}%
\pgfpathlineto{\pgfqpoint{4.785136in}{0.980134in}}%
\pgfpathlineto{\pgfqpoint{4.787149in}{0.970117in}}%
\pgfpathlineto{\pgfqpoint{4.791176in}{0.976295in}}%
\pgfpathlineto{\pgfqpoint{4.793190in}{0.970956in}}%
\pgfpathlineto{\pgfqpoint{4.801244in}{0.976868in}}%
\pgfpathlineto{\pgfqpoint{4.803257in}{0.972853in}}%
\pgfpathlineto{\pgfqpoint{4.807284in}{0.977839in}}%
\pgfpathlineto{\pgfqpoint{4.813324in}{0.974618in}}%
\pgfpathlineto{\pgfqpoint{4.817351in}{0.978810in}}%
\pgfpathlineto{\pgfqpoint{4.819365in}{0.978678in}}%
\pgfpathlineto{\pgfqpoint{4.821378in}{0.973691in}}%
\pgfpathlineto{\pgfqpoint{4.827418in}{0.977927in}}%
\pgfpathlineto{\pgfqpoint{4.829432in}{0.975633in}}%
\pgfpathlineto{\pgfqpoint{4.831445in}{0.978633in}}%
\pgfpathlineto{\pgfqpoint{4.833459in}{0.975589in}}%
\pgfpathlineto{\pgfqpoint{4.835472in}{0.977530in}}%
\pgfpathlineto{\pgfqpoint{4.841512in}{0.962572in}}%
\pgfpathlineto{\pgfqpoint{4.845539in}{0.973118in}}%
\pgfpathlineto{\pgfqpoint{4.847553in}{0.974530in}}%
\pgfpathlineto{\pgfqpoint{4.849566in}{0.977442in}}%
\pgfpathlineto{\pgfqpoint{4.855607in}{0.984149in}}%
\pgfpathlineto{\pgfqpoint{4.857620in}{0.983487in}}%
\pgfpathlineto{\pgfqpoint{4.859634in}{0.988385in}}%
\pgfpathlineto{\pgfqpoint{4.861647in}{0.991297in}}%
\pgfpathlineto{\pgfqpoint{4.863660in}{0.991650in}}%
\pgfpathlineto{\pgfqpoint{4.869701in}{0.995975in}}%
\pgfpathlineto{\pgfqpoint{4.871714in}{0.996019in}}%
\pgfpathlineto{\pgfqpoint{4.873728in}{0.992577in}}%
\pgfpathlineto{\pgfqpoint{4.875741in}{0.997475in}}%
\pgfpathlineto{\pgfqpoint{4.877755in}{1.000299in}}%
\pgfpathlineto{\pgfqpoint{4.883795in}{1.004535in}}%
\pgfpathlineto{\pgfqpoint{4.887822in}{0.998004in}}%
\pgfpathlineto{\pgfqpoint{4.889835in}{1.002108in}}%
\pgfpathlineto{\pgfqpoint{4.891849in}{1.004888in}}%
\pgfpathlineto{\pgfqpoint{4.897889in}{1.009653in}}%
\pgfpathlineto{\pgfqpoint{4.899902in}{1.006829in}}%
\pgfpathlineto{\pgfqpoint{4.901916in}{1.012610in}}%
\pgfpathlineto{\pgfqpoint{4.903929in}{1.011065in}}%
\pgfpathlineto{\pgfqpoint{4.905943in}{1.007094in}}%
\pgfpathlineto{\pgfqpoint{4.911983in}{0.996945in}}%
\pgfpathlineto{\pgfqpoint{4.913997in}{1.008462in}}%
\pgfpathlineto{\pgfqpoint{4.916010in}{1.010403in}}%
\pgfpathlineto{\pgfqpoint{4.918023in}{1.014286in}}%
\pgfpathlineto{\pgfqpoint{4.920037in}{1.012124in}}%
\pgfpathlineto{\pgfqpoint{4.926077in}{1.009036in}}%
\pgfpathlineto{\pgfqpoint{4.928091in}{1.015831in}}%
\pgfpathlineto{\pgfqpoint{4.930104in}{1.014551in}}%
\pgfpathlineto{\pgfqpoint{4.932118in}{1.010801in}}%
\pgfpathlineto{\pgfqpoint{4.934131in}{1.009962in}}%
\pgfpathlineto{\pgfqpoint{4.940171in}{1.013360in}}%
\pgfpathlineto{\pgfqpoint{4.942185in}{1.013051in}}%
\pgfpathlineto{\pgfqpoint{4.944198in}{1.020067in}}%
\pgfpathlineto{\pgfqpoint{4.946212in}{1.018743in}}%
\pgfpathlineto{\pgfqpoint{4.948225in}{1.018875in}}%
\pgfpathlineto{\pgfqpoint{4.954266in}{1.018567in}}%
\pgfpathlineto{\pgfqpoint{4.958292in}{1.016184in}}%
\pgfpathlineto{\pgfqpoint{4.962319in}{1.017375in}}%
\pgfpathlineto{\pgfqpoint{4.968360in}{1.013845in}}%
\pgfpathlineto{\pgfqpoint{4.970373in}{1.017684in}}%
\pgfpathlineto{\pgfqpoint{4.974400in}{1.010712in}}%
\pgfpathlineto{\pgfqpoint{4.976413in}{1.019802in}}%
\pgfpathlineto{\pgfqpoint{4.982454in}{1.016140in}}%
\pgfpathlineto{\pgfqpoint{4.984467in}{1.014066in}}%
\pgfpathlineto{\pgfqpoint{4.986481in}{1.008991in}}%
\pgfpathlineto{\pgfqpoint{4.988494in}{1.010006in}}%
\pgfpathlineto{\pgfqpoint{4.990508in}{1.001358in}}%
\pgfpathlineto{\pgfqpoint{4.996548in}{1.004623in}}%
\pgfpathlineto{\pgfqpoint{4.998561in}{1.012168in}}%
\pgfpathlineto{\pgfqpoint{5.000575in}{1.017463in}}%
\pgfpathlineto{\pgfqpoint{5.002588in}{1.012433in}}%
\pgfpathlineto{\pgfqpoint{5.004602in}{1.002284in}}%
\pgfpathlineto{\pgfqpoint{5.012655in}{1.006829in}}%
\pgfpathlineto{\pgfqpoint{5.014669in}{1.011904in}}%
\pgfpathlineto{\pgfqpoint{5.016682in}{1.010624in}}%
\pgfpathlineto{\pgfqpoint{5.024736in}{1.011771in}}%
\pgfpathlineto{\pgfqpoint{5.026750in}{1.014684in}}%
\pgfpathlineto{\pgfqpoint{5.030777in}{1.007535in}}%
\pgfpathlineto{\pgfqpoint{5.038830in}{0.999593in}}%
\pgfpathlineto{\pgfqpoint{5.040844in}{1.002064in}}%
\pgfpathlineto{\pgfqpoint{5.042857in}{0.997740in}}%
\pgfpathlineto{\pgfqpoint{5.044871in}{0.991386in}}%
\pgfpathlineto{\pgfqpoint{5.046884in}{0.987503in}}%
\pgfpathlineto{\pgfqpoint{5.052924in}{0.991959in}}%
\pgfpathlineto{\pgfqpoint{5.054938in}{0.995577in}}%
\pgfpathlineto{\pgfqpoint{5.056951in}{0.988429in}}%
\pgfpathlineto{\pgfqpoint{5.058965in}{0.991430in}}%
\pgfpathlineto{\pgfqpoint{5.060978in}{0.982958in}}%
\pgfpathlineto{\pgfqpoint{5.069032in}{0.981104in}}%
\pgfpathlineto{\pgfqpoint{5.071045in}{0.978016in}}%
\pgfpathlineto{\pgfqpoint{5.075072in}{0.986708in}}%
\pgfpathlineto{\pgfqpoint{5.081113in}{0.982605in}}%
\pgfpathlineto{\pgfqpoint{5.083126in}{0.983178in}}%
\pgfpathlineto{\pgfqpoint{5.085140in}{0.979031in}}%
\pgfpathlineto{\pgfqpoint{5.087153in}{0.972235in}}%
\pgfpathlineto{\pgfqpoint{5.089166in}{0.994386in}}%
\pgfpathlineto{\pgfqpoint{5.095207in}{0.993945in}}%
\pgfpathlineto{\pgfqpoint{5.097220in}{0.989709in}}%
\pgfpathlineto{\pgfqpoint{5.099234in}{0.993945in}}%
\pgfpathlineto{\pgfqpoint{5.101247in}{0.990944in}}%
\pgfpathlineto{\pgfqpoint{5.103261in}{0.981722in}}%
\pgfpathlineto{\pgfqpoint{5.109301in}{0.965528in}}%
\pgfpathlineto{\pgfqpoint{5.111314in}{0.967955in}}%
\pgfpathlineto{\pgfqpoint{5.113328in}{0.976383in}}%
\pgfpathlineto{\pgfqpoint{5.115341in}{0.969147in}}%
\pgfpathlineto{\pgfqpoint{5.117355in}{0.977530in}}%
\pgfpathlineto{\pgfqpoint{5.125409in}{0.980487in}}%
\pgfpathlineto{\pgfqpoint{5.127422in}{0.985076in}}%
\pgfpathlineto{\pgfqpoint{5.129435in}{0.981678in}}%
\pgfpathlineto{\pgfqpoint{5.131449in}{0.982914in}}%
\pgfpathlineto{\pgfqpoint{5.137489in}{0.989532in}}%
\pgfpathlineto{\pgfqpoint{5.139503in}{0.985605in}}%
\pgfpathlineto{\pgfqpoint{5.141516in}{0.984326in}}%
\pgfpathlineto{\pgfqpoint{5.143530in}{0.990415in}}%
\pgfpathlineto{\pgfqpoint{5.145543in}{0.988120in}}%
\pgfpathlineto{\pgfqpoint{5.151583in}{0.986620in}}%
\pgfpathlineto{\pgfqpoint{5.153597in}{0.996328in}}%
\pgfpathlineto{\pgfqpoint{5.157624in}{0.993195in}}%
\pgfpathlineto{\pgfqpoint{5.159637in}{0.993151in}}%
\pgfpathlineto{\pgfqpoint{5.165677in}{0.984679in}}%
\pgfpathlineto{\pgfqpoint{5.167691in}{0.978986in}}%
\pgfpathlineto{\pgfqpoint{5.169704in}{0.979251in}}%
\pgfpathlineto{\pgfqpoint{5.171718in}{0.977266in}}%
\pgfpathlineto{\pgfqpoint{5.173731in}{0.983355in}}%
\pgfpathlineto{\pgfqpoint{5.179772in}{0.982737in}}%
\pgfpathlineto{\pgfqpoint{5.183799in}{0.986532in}}%
\pgfpathlineto{\pgfqpoint{5.187825in}{0.992621in}}%
\pgfpathlineto{\pgfqpoint{5.193866in}{0.992577in}}%
\pgfpathlineto{\pgfqpoint{5.195879in}{0.989091in}}%
\pgfpathlineto{\pgfqpoint{5.197893in}{0.993195in}}%
\pgfpathlineto{\pgfqpoint{5.199906in}{0.994165in}}%
\pgfpathlineto{\pgfqpoint{5.207960in}{0.993945in}}%
\pgfpathlineto{\pgfqpoint{5.211987in}{1.005594in}}%
\pgfpathlineto{\pgfqpoint{5.214000in}{1.004270in}}%
\pgfpathlineto{\pgfqpoint{5.216014in}{1.009080in}}%
\pgfpathlineto{\pgfqpoint{5.222054in}{1.010095in}}%
\pgfpathlineto{\pgfqpoint{5.224067in}{1.006388in}}%
\pgfpathlineto{\pgfqpoint{5.226081in}{1.011727in}}%
\pgfpathlineto{\pgfqpoint{5.228094in}{1.009036in}}%
\pgfpathlineto{\pgfqpoint{5.230108in}{1.010977in}}%
\pgfpathlineto{\pgfqpoint{5.236148in}{1.010050in}}%
\pgfpathlineto{\pgfqpoint{5.238162in}{1.013095in}}%
\pgfpathlineto{\pgfqpoint{5.240175in}{1.018214in}}%
\pgfpathlineto{\pgfqpoint{5.242188in}{1.020905in}}%
\pgfpathlineto{\pgfqpoint{5.244202in}{1.019802in}}%
\pgfpathlineto{\pgfqpoint{5.250242in}{1.025715in}}%
\pgfpathlineto{\pgfqpoint{5.252256in}{1.022714in}}%
\pgfpathlineto{\pgfqpoint{5.254269in}{1.024435in}}%
\pgfpathlineto{\pgfqpoint{5.256283in}{1.022891in}}%
\pgfpathlineto{\pgfqpoint{5.258296in}{1.015610in}}%
\pgfpathlineto{\pgfqpoint{5.264336in}{1.011462in}}%
\pgfpathlineto{\pgfqpoint{5.268363in}{1.014110in}}%
\pgfpathlineto{\pgfqpoint{5.270377in}{1.009477in}}%
\pgfpathlineto{\pgfqpoint{5.272390in}{1.007579in}}%
\pgfpathlineto{\pgfqpoint{5.278431in}{1.012830in}}%
\pgfpathlineto{\pgfqpoint{5.280444in}{1.007359in}}%
\pgfpathlineto{\pgfqpoint{5.282457in}{1.006829in}}%
\pgfpathlineto{\pgfqpoint{5.286484in}{1.009609in}}%
\pgfpathlineto{\pgfqpoint{5.292525in}{1.011771in}}%
\pgfpathlineto{\pgfqpoint{5.294538in}{1.016007in}}%
\pgfpathlineto{\pgfqpoint{5.296552in}{1.008418in}}%
\pgfpathlineto{\pgfqpoint{5.298565in}{1.010977in}}%
\pgfpathlineto{\pgfqpoint{5.300578in}{1.006388in}}%
\pgfpathlineto{\pgfqpoint{5.306619in}{1.010580in}}%
\pgfpathlineto{\pgfqpoint{5.308632in}{1.006212in}}%
\pgfpathlineto{\pgfqpoint{5.310646in}{1.008991in}}%
\pgfpathlineto{\pgfqpoint{5.312659in}{1.006609in}}%
\pgfpathlineto{\pgfqpoint{5.314673in}{1.010006in}}%
\pgfpathlineto{\pgfqpoint{5.320713in}{1.008021in}}%
\pgfpathlineto{\pgfqpoint{5.322726in}{1.017419in}}%
\pgfpathlineto{\pgfqpoint{5.324740in}{1.016007in}}%
\pgfpathlineto{\pgfqpoint{5.326753in}{1.015743in}}%
\pgfpathlineto{\pgfqpoint{5.328767in}{1.018611in}}%
\pgfpathlineto{\pgfqpoint{5.336820in}{1.015478in}}%
\pgfpathlineto{\pgfqpoint{5.338834in}{1.016713in}}%
\pgfpathlineto{\pgfqpoint{5.340847in}{1.019846in}}%
\pgfpathlineto{\pgfqpoint{5.342861in}{1.019802in}}%
\pgfpathlineto{\pgfqpoint{5.350915in}{1.022670in}}%
\pgfpathlineto{\pgfqpoint{5.352928in}{1.027215in}}%
\pgfpathlineto{\pgfqpoint{5.354942in}{1.025538in}}%
\pgfpathlineto{\pgfqpoint{5.356955in}{1.020861in}}%
\pgfpathlineto{\pgfqpoint{5.362995in}{1.013183in}}%
\pgfpathlineto{\pgfqpoint{5.365009in}{1.013889in}}%
\pgfpathlineto{\pgfqpoint{5.367022in}{1.012168in}}%
\pgfpathlineto{\pgfqpoint{5.369036in}{1.012963in}}%
\pgfpathlineto{\pgfqpoint{5.371049in}{1.007050in}}%
\pgfpathlineto{\pgfqpoint{5.379103in}{1.008550in}}%
\pgfpathlineto{\pgfqpoint{5.381116in}{1.005064in}}%
\pgfpathlineto{\pgfqpoint{5.383130in}{1.012433in}}%
\pgfpathlineto{\pgfqpoint{5.385143in}{0.998666in}}%
\pgfpathlineto{\pgfqpoint{5.391184in}{0.991297in}}%
\pgfpathlineto{\pgfqpoint{5.395210in}{1.005991in}}%
\pgfpathlineto{\pgfqpoint{5.397224in}{0.994871in}}%
\pgfpathlineto{\pgfqpoint{5.399237in}{0.996239in}}%
\pgfpathlineto{\pgfqpoint{5.407291in}{0.997034in}}%
\pgfpathlineto{\pgfqpoint{5.409305in}{0.994386in}}%
\pgfpathlineto{\pgfqpoint{5.411318in}{0.996372in}}%
\pgfpathlineto{\pgfqpoint{5.413331in}{1.004623in}}%
\pgfpathlineto{\pgfqpoint{5.419372in}{1.005020in}}%
\pgfpathlineto{\pgfqpoint{5.421385in}{1.009168in}}%
\pgfpathlineto{\pgfqpoint{5.423399in}{1.009080in}}%
\pgfpathlineto{\pgfqpoint{5.425412in}{1.012036in}}%
\pgfpathlineto{\pgfqpoint{5.427426in}{1.012742in}}%
\pgfpathlineto{\pgfqpoint{5.433466in}{1.012786in}}%
\pgfpathlineto{\pgfqpoint{5.435479in}{1.014551in}}%
\pgfpathlineto{\pgfqpoint{5.437493in}{1.017331in}}%
\pgfpathlineto{\pgfqpoint{5.439506in}{1.014860in}}%
\pgfpathlineto{\pgfqpoint{5.441520in}{1.019670in}}%
\pgfpathlineto{\pgfqpoint{5.449574in}{1.013536in}}%
\pgfpathlineto{\pgfqpoint{5.451587in}{1.013713in}}%
\pgfpathlineto{\pgfqpoint{5.453600in}{1.016581in}}%
\pgfpathlineto{\pgfqpoint{5.455614in}{1.011639in}}%
\pgfpathlineto{\pgfqpoint{5.463668in}{1.012654in}}%
\pgfpathlineto{\pgfqpoint{5.465681in}{1.014507in}}%
\pgfpathlineto{\pgfqpoint{5.469708in}{1.020685in}}%
\pgfpathlineto{\pgfqpoint{5.475748in}{1.019846in}}%
\pgfpathlineto{\pgfqpoint{5.477762in}{1.020332in}}%
\pgfpathlineto{\pgfqpoint{5.479775in}{1.018699in}}%
\pgfpathlineto{\pgfqpoint{5.481789in}{1.020596in}}%
\pgfpathlineto{\pgfqpoint{5.483802in}{1.020155in}}%
\pgfpathlineto{\pgfqpoint{5.489842in}{1.023994in}}%
\pgfpathlineto{\pgfqpoint{5.491856in}{1.023597in}}%
\pgfpathlineto{\pgfqpoint{5.493869in}{1.024303in}}%
\pgfpathlineto{\pgfqpoint{5.495883in}{1.021832in}}%
\pgfpathlineto{\pgfqpoint{5.503937in}{1.025274in}}%
\pgfpathlineto{\pgfqpoint{5.505950in}{1.024126in}}%
\pgfpathlineto{\pgfqpoint{5.507964in}{1.021920in}}%
\pgfpathlineto{\pgfqpoint{5.509977in}{1.022052in}}%
\pgfpathlineto{\pgfqpoint{5.511990in}{1.023156in}}%
\pgfpathlineto{\pgfqpoint{5.518031in}{1.024435in}}%
\pgfpathlineto{\pgfqpoint{5.520044in}{1.025759in}}%
\pgfpathlineto{\pgfqpoint{5.522058in}{1.024568in}}%
\pgfpathlineto{\pgfqpoint{5.524071in}{1.026288in}}%
\pgfpathlineto{\pgfqpoint{5.526085in}{1.029333in}}%
\pgfpathlineto{\pgfqpoint{5.534138in}{1.031628in}}%
\pgfpathlineto{\pgfqpoint{5.536152in}{1.034716in}}%
\pgfpathlineto{\pgfqpoint{5.538165in}{1.033525in}}%
\pgfpathlineto{\pgfqpoint{5.540179in}{1.026421in}}%
\pgfpathlineto{\pgfqpoint{5.546219in}{1.033525in}}%
\pgfpathlineto{\pgfqpoint{5.548232in}{1.028848in}}%
\pgfpathlineto{\pgfqpoint{5.550246in}{1.027127in}}%
\pgfpathlineto{\pgfqpoint{5.552259in}{1.029377in}}%
\pgfpathlineto{\pgfqpoint{5.554273in}{1.029642in}}%
\pgfpathlineto{\pgfqpoint{5.560313in}{1.031539in}}%
\pgfpathlineto{\pgfqpoint{5.562327in}{1.031407in}}%
\pgfpathlineto{\pgfqpoint{5.564340in}{1.034628in}}%
\pgfpathlineto{\pgfqpoint{5.566353in}{1.035202in}}%
\pgfpathlineto{\pgfqpoint{5.568367in}{1.031672in}}%
\pgfpathlineto{\pgfqpoint{5.574407in}{1.028495in}}%
\pgfpathlineto{\pgfqpoint{5.576421in}{1.030039in}}%
\pgfpathlineto{\pgfqpoint{5.578434in}{1.033525in}}%
\pgfpathlineto{\pgfqpoint{5.580448in}{1.028980in}}%
\pgfpathlineto{\pgfqpoint{5.582461in}{1.032378in}}%
\pgfpathlineto{\pgfqpoint{5.588501in}{1.033040in}}%
\pgfpathlineto{\pgfqpoint{5.590515in}{1.032510in}}%
\pgfpathlineto{\pgfqpoint{5.592528in}{1.035158in}}%
\pgfpathlineto{\pgfqpoint{5.594542in}{1.035202in}}%
\pgfpathlineto{\pgfqpoint{5.596555in}{1.033128in}}%
\pgfpathlineto{\pgfqpoint{5.602596in}{1.034187in}}%
\pgfpathlineto{\pgfqpoint{5.604609in}{1.029510in}}%
\pgfpathlineto{\pgfqpoint{5.606622in}{1.030436in}}%
\pgfpathlineto{\pgfqpoint{5.608636in}{1.028759in}}%
\pgfpathlineto{\pgfqpoint{5.610649in}{1.031275in}}%
\pgfpathlineto{\pgfqpoint{5.616690in}{1.029995in}}%
\pgfpathlineto{\pgfqpoint{5.618703in}{1.027524in}}%
\pgfpathlineto{\pgfqpoint{5.620717in}{1.032863in}}%
\pgfpathlineto{\pgfqpoint{5.622730in}{1.031495in}}%
\pgfpathlineto{\pgfqpoint{5.624743in}{1.030833in}}%
\pgfpathlineto{\pgfqpoint{5.630784in}{1.034407in}}%
\pgfpathlineto{\pgfqpoint{5.632797in}{1.029465in}}%
\pgfpathlineto{\pgfqpoint{5.634811in}{1.028274in}}%
\pgfpathlineto{\pgfqpoint{5.636824in}{1.028980in}}%
\pgfpathlineto{\pgfqpoint{5.638838in}{1.030304in}}%
\pgfpathlineto{\pgfqpoint{5.644878in}{1.031539in}}%
\pgfpathlineto{\pgfqpoint{5.648905in}{1.023200in}}%
\pgfpathlineto{\pgfqpoint{5.650918in}{1.023509in}}%
\pgfpathlineto{\pgfqpoint{5.652932in}{1.022229in}}%
\pgfpathlineto{\pgfqpoint{5.658972in}{1.031539in}}%
\pgfpathlineto{\pgfqpoint{5.662999in}{1.034716in}}%
\pgfpathlineto{\pgfqpoint{5.665012in}{1.028759in}}%
\pgfpathlineto{\pgfqpoint{5.667026in}{1.028804in}}%
\pgfpathlineto{\pgfqpoint{5.673066in}{1.013669in}}%
\pgfpathlineto{\pgfqpoint{5.675080in}{1.014507in}}%
\pgfpathlineto{\pgfqpoint{5.677093in}{1.021744in}}%
\pgfpathlineto{\pgfqpoint{5.679107in}{1.026068in}}%
\pgfpathlineto{\pgfqpoint{5.681120in}{1.024965in}}%
\pgfpathlineto{\pgfqpoint{5.687160in}{1.028715in}}%
\pgfpathlineto{\pgfqpoint{5.689174in}{1.021082in}}%
\pgfpathlineto{\pgfqpoint{5.691187in}{1.019537in}}%
\pgfpathlineto{\pgfqpoint{5.695214in}{1.021964in}}%
\pgfpathlineto{\pgfqpoint{5.701254in}{1.017419in}}%
\pgfpathlineto{\pgfqpoint{5.703268in}{1.017728in}}%
\pgfpathlineto{\pgfqpoint{5.707295in}{1.001578in}}%
\pgfpathlineto{\pgfqpoint{5.709308in}{1.002858in}}%
\pgfpathlineto{\pgfqpoint{5.715349in}{1.009742in}}%
\pgfpathlineto{\pgfqpoint{5.717362in}{1.008859in}}%
\pgfpathlineto{\pgfqpoint{5.719375in}{1.018434in}}%
\pgfpathlineto{\pgfqpoint{5.723402in}{1.017684in}}%
\pgfpathlineto{\pgfqpoint{5.729443in}{1.014904in}}%
\pgfpathlineto{\pgfqpoint{5.731456in}{1.017949in}}%
\pgfpathlineto{\pgfqpoint{5.733470in}{1.017640in}}%
\pgfpathlineto{\pgfqpoint{5.735483in}{1.019228in}}%
\pgfpathlineto{\pgfqpoint{5.737496in}{1.014242in}}%
\pgfpathlineto{\pgfqpoint{5.743537in}{1.013183in}}%
\pgfpathlineto{\pgfqpoint{5.745550in}{1.014286in}}%
\pgfpathlineto{\pgfqpoint{5.749577in}{1.012301in}}%
\pgfpathlineto{\pgfqpoint{5.751591in}{1.013360in}}%
\pgfpathlineto{\pgfqpoint{5.763671in}{1.014154in}}%
\pgfpathlineto{\pgfqpoint{5.765685in}{1.012830in}}%
\pgfpathlineto{\pgfqpoint{5.773739in}{1.019228in}}%
\pgfpathlineto{\pgfqpoint{5.775752in}{1.022052in}}%
\pgfpathlineto{\pgfqpoint{5.779779in}{1.031010in}}%
\pgfpathlineto{\pgfqpoint{5.785819in}{1.028980in}}%
\pgfpathlineto{\pgfqpoint{5.787833in}{1.027083in}}%
\pgfpathlineto{\pgfqpoint{5.789846in}{1.029201in}}%
\pgfpathlineto{\pgfqpoint{5.791860in}{1.027347in}}%
\pgfpathlineto{\pgfqpoint{5.793873in}{1.026465in}}%
\pgfpathlineto{\pgfqpoint{5.801927in}{1.026862in}}%
\pgfpathlineto{\pgfqpoint{5.803940in}{1.028318in}}%
\pgfpathlineto{\pgfqpoint{5.814007in}{1.030745in}}%
\pgfpathlineto{\pgfqpoint{5.816021in}{1.035422in}}%
\pgfpathlineto{\pgfqpoint{5.818034in}{1.038335in}}%
\pgfpathlineto{\pgfqpoint{5.820048in}{1.035466in}}%
\pgfpathlineto{\pgfqpoint{5.822061in}{1.037761in}}%
\pgfpathlineto{\pgfqpoint{5.828102in}{1.037452in}}%
\pgfpathlineto{\pgfqpoint{5.830115in}{1.033172in}}%
\pgfpathlineto{\pgfqpoint{5.834142in}{1.031363in}}%
\pgfpathlineto{\pgfqpoint{5.836155in}{1.047777in}}%
\pgfpathlineto{\pgfqpoint{5.844209in}{1.046498in}}%
\pgfpathlineto{\pgfqpoint{5.846223in}{1.043497in}}%
\pgfpathlineto{\pgfqpoint{5.850250in}{1.046983in}}%
\pgfpathlineto{\pgfqpoint{5.856290in}{1.049366in}}%
\pgfpathlineto{\pgfqpoint{5.858303in}{1.051131in}}%
\pgfpathlineto{\pgfqpoint{5.860317in}{1.054837in}}%
\pgfpathlineto{\pgfqpoint{5.862330in}{1.054264in}}%
\pgfpathlineto{\pgfqpoint{5.864344in}{1.054484in}}%
\pgfpathlineto{\pgfqpoint{5.872397in}{1.056470in}}%
\pgfpathlineto{\pgfqpoint{5.874411in}{1.055984in}}%
\pgfpathlineto{\pgfqpoint{5.878438in}{1.058720in}}%
\pgfpathlineto{\pgfqpoint{5.886492in}{1.056558in}}%
\pgfpathlineto{\pgfqpoint{5.888505in}{1.061147in}}%
\pgfpathlineto{\pgfqpoint{5.890518in}{1.059161in}}%
\pgfpathlineto{\pgfqpoint{5.892532in}{1.060265in}}%
\pgfpathlineto{\pgfqpoint{5.904613in}{1.061677in}}%
\pgfpathlineto{\pgfqpoint{5.906626in}{1.064368in}}%
\pgfpathlineto{\pgfqpoint{5.912666in}{1.066001in}}%
\pgfpathlineto{\pgfqpoint{5.914680in}{1.063574in}}%
\pgfpathlineto{\pgfqpoint{5.916693in}{1.065162in}}%
\pgfpathlineto{\pgfqpoint{5.920720in}{1.066574in}}%
\pgfpathlineto{\pgfqpoint{5.926761in}{1.061985in}}%
\pgfpathlineto{\pgfqpoint{5.928774in}{1.057088in}}%
\pgfpathlineto{\pgfqpoint{5.932801in}{1.060573in}}%
\pgfpathlineto{\pgfqpoint{5.934814in}{1.062030in}}%
\pgfpathlineto{\pgfqpoint{5.940855in}{1.060971in}}%
\pgfpathlineto{\pgfqpoint{5.944882in}{1.061985in}}%
\pgfpathlineto{\pgfqpoint{5.948908in}{1.060618in}}%
\pgfpathlineto{\pgfqpoint{5.954949in}{1.062736in}}%
\pgfpathlineto{\pgfqpoint{5.956962in}{1.060220in}}%
\pgfpathlineto{\pgfqpoint{5.958976in}{1.061324in}}%
\pgfpathlineto{\pgfqpoint{5.960989in}{1.061588in}}%
\pgfpathlineto{\pgfqpoint{5.963003in}{1.060044in}}%
\pgfpathlineto{\pgfqpoint{5.971056in}{1.060529in}}%
\pgfpathlineto{\pgfqpoint{5.973070in}{1.059779in}}%
\pgfpathlineto{\pgfqpoint{5.975083in}{1.060618in}}%
\pgfpathlineto{\pgfqpoint{5.983137in}{1.064721in}}%
\pgfpathlineto{\pgfqpoint{5.987164in}{1.064456in}}%
\pgfpathlineto{\pgfqpoint{5.989177in}{1.070546in}}%
\pgfpathlineto{\pgfqpoint{5.991191in}{1.070546in}}%
\pgfpathlineto{\pgfqpoint{5.999245in}{1.074693in}}%
\pgfpathlineto{\pgfqpoint{6.005285in}{1.070855in}}%
\pgfpathlineto{\pgfqpoint{6.011325in}{1.070987in}}%
\pgfpathlineto{\pgfqpoint{6.013339in}{1.076591in}}%
\pgfpathlineto{\pgfqpoint{6.015352in}{1.076150in}}%
\pgfpathlineto{\pgfqpoint{6.017366in}{1.076900in}}%
\pgfpathlineto{\pgfqpoint{6.019379in}{1.074605in}}%
\pgfpathlineto{\pgfqpoint{6.025419in}{1.073855in}}%
\pgfpathlineto{\pgfqpoint{6.027433in}{1.074208in}}%
\pgfpathlineto{\pgfqpoint{6.029446in}{1.075311in}}%
\pgfpathlineto{\pgfqpoint{6.031460in}{1.074649in}}%
\pgfpathlineto{\pgfqpoint{6.033473in}{1.077385in}}%
\pgfpathlineto{\pgfqpoint{6.039514in}{1.079591in}}%
\pgfpathlineto{\pgfqpoint{6.041527in}{1.079238in}}%
\pgfpathlineto{\pgfqpoint{6.043540in}{1.073899in}}%
\pgfpathlineto{\pgfqpoint{6.045554in}{1.073679in}}%
\pgfpathlineto{\pgfqpoint{6.047567in}{1.077032in}}%
\pgfpathlineto{\pgfqpoint{6.053608in}{1.080650in}}%
\pgfpathlineto{\pgfqpoint{6.055621in}{1.083033in}}%
\pgfpathlineto{\pgfqpoint{6.057635in}{1.087181in}}%
\pgfpathlineto{\pgfqpoint{6.059648in}{1.088152in}}%
\pgfpathlineto{\pgfqpoint{6.061661in}{1.086563in}}%
\pgfpathlineto{\pgfqpoint{6.069715in}{1.086784in}}%
\pgfpathlineto{\pgfqpoint{6.071729in}{1.088990in}}%
\pgfpathlineto{\pgfqpoint{6.073742in}{1.089740in}}%
\pgfpathlineto{\pgfqpoint{6.075756in}{1.093005in}}%
\pgfpathlineto{\pgfqpoint{6.081796in}{1.094726in}}%
\pgfpathlineto{\pgfqpoint{6.083809in}{1.091417in}}%
\pgfpathlineto{\pgfqpoint{6.085823in}{1.092741in}}%
\pgfpathlineto{\pgfqpoint{6.087836in}{1.092741in}}%
\pgfpathlineto{\pgfqpoint{6.089850in}{1.086078in}}%
\pgfpathlineto{\pgfqpoint{6.095890in}{1.081445in}}%
\pgfpathlineto{\pgfqpoint{6.097904in}{1.088328in}}%
\pgfpathlineto{\pgfqpoint{6.099917in}{1.089387in}}%
\pgfpathlineto{\pgfqpoint{6.101930in}{1.084401in}}%
\pgfpathlineto{\pgfqpoint{6.103944in}{1.084401in}}%
\pgfpathlineto{\pgfqpoint{6.109984in}{1.087093in}}%
\pgfpathlineto{\pgfqpoint{6.111998in}{1.085283in}}%
\pgfpathlineto{\pgfqpoint{6.114011in}{1.085989in}}%
\pgfpathlineto{\pgfqpoint{6.116025in}{1.083430in}}%
\pgfpathlineto{\pgfqpoint{6.118038in}{1.090534in}}%
\pgfpathlineto{\pgfqpoint{6.124078in}{1.088946in}}%
\pgfpathlineto{\pgfqpoint{6.126092in}{1.087578in}}%
\pgfpathlineto{\pgfqpoint{6.128105in}{1.093358in}}%
\pgfpathlineto{\pgfqpoint{6.130119in}{1.085460in}}%
\pgfpathlineto{\pgfqpoint{6.132132in}{1.082680in}}%
\pgfpathlineto{\pgfqpoint{6.138172in}{1.080827in}}%
\pgfpathlineto{\pgfqpoint{6.142199in}{1.083695in}}%
\pgfpathlineto{\pgfqpoint{6.144213in}{1.080386in}}%
\pgfpathlineto{\pgfqpoint{6.146226in}{1.083298in}}%
\pgfpathlineto{\pgfqpoint{6.154280in}{1.089784in}}%
\pgfpathlineto{\pgfqpoint{6.156294in}{1.093138in}}%
\pgfpathlineto{\pgfqpoint{6.158307in}{1.092079in}}%
\pgfpathlineto{\pgfqpoint{6.160320in}{1.096403in}}%
\pgfpathlineto{\pgfqpoint{6.166361in}{1.095962in}}%
\pgfpathlineto{\pgfqpoint{6.170388in}{1.102139in}}%
\pgfpathlineto{\pgfqpoint{6.172401in}{1.101521in}}%
\pgfpathlineto{\pgfqpoint{6.174415in}{1.108008in}}%
\pgfpathlineto{\pgfqpoint{6.180455in}{1.111361in}}%
\pgfpathlineto{\pgfqpoint{6.182468in}{1.109729in}}%
\pgfpathlineto{\pgfqpoint{6.184482in}{1.113479in}}%
\pgfpathlineto{\pgfqpoint{6.186495in}{1.107875in}}%
\pgfpathlineto{\pgfqpoint{6.188509in}{1.106066in}}%
\pgfpathlineto{\pgfqpoint{6.194549in}{1.107831in}}%
\pgfpathlineto{\pgfqpoint{6.196562in}{1.113568in}}%
\pgfpathlineto{\pgfqpoint{6.198576in}{1.115333in}}%
\pgfpathlineto{\pgfqpoint{6.200589in}{1.112332in}}%
\pgfpathlineto{\pgfqpoint{6.202603in}{1.113656in}}%
\pgfpathlineto{\pgfqpoint{6.208643in}{1.116259in}}%
\pgfpathlineto{\pgfqpoint{6.212670in}{1.114009in}}%
\pgfpathlineto{\pgfqpoint{6.214683in}{1.107567in}}%
\pgfpathlineto{\pgfqpoint{6.224751in}{1.121201in}}%
\pgfpathlineto{\pgfqpoint{6.226764in}{1.124952in}}%
\pgfpathlineto{\pgfqpoint{6.228778in}{1.119877in}}%
\pgfpathlineto{\pgfqpoint{6.230791in}{1.121422in}}%
\pgfpathlineto{\pgfqpoint{6.236831in}{1.124511in}}%
\pgfpathlineto{\pgfqpoint{6.238845in}{1.128217in}}%
\pgfpathlineto{\pgfqpoint{6.240858in}{1.124334in}}%
\pgfpathlineto{\pgfqpoint{6.244885in}{1.125084in}}%
\pgfpathlineto{\pgfqpoint{6.250926in}{1.126937in}}%
\pgfpathlineto{\pgfqpoint{6.254952in}{1.126717in}}%
\pgfpathlineto{\pgfqpoint{6.256966in}{1.125834in}}%
\pgfpathlineto{\pgfqpoint{6.258979in}{1.127467in}}%
\pgfpathlineto{\pgfqpoint{6.267033in}{1.123584in}}%
\pgfpathlineto{\pgfqpoint{6.269047in}{1.124334in}}%
\pgfpathlineto{\pgfqpoint{6.271060in}{1.130335in}}%
\pgfpathlineto{\pgfqpoint{6.273073in}{1.129806in}}%
\pgfpathlineto{\pgfqpoint{6.279114in}{1.137307in}}%
\pgfpathlineto{\pgfqpoint{6.281127in}{1.137572in}}%
\pgfpathlineto{\pgfqpoint{6.283141in}{1.135895in}}%
\pgfpathlineto{\pgfqpoint{6.285154in}{1.136998in}}%
\pgfpathlineto{\pgfqpoint{6.287168in}{1.133600in}}%
\pgfpathlineto{\pgfqpoint{6.293208in}{1.131526in}}%
\pgfpathlineto{\pgfqpoint{6.295221in}{1.134042in}}%
\pgfpathlineto{\pgfqpoint{6.297235in}{1.132144in}}%
\pgfpathlineto{\pgfqpoint{6.301262in}{1.134703in}}%
\pgfpathlineto{\pgfqpoint{6.307302in}{1.123628in}}%
\pgfpathlineto{\pgfqpoint{6.309316in}{1.123275in}}%
\pgfpathlineto{\pgfqpoint{6.313342in}{1.130423in}}%
\pgfpathlineto{\pgfqpoint{6.315356in}{1.133336in}}%
\pgfpathlineto{\pgfqpoint{6.323410in}{1.134836in}}%
\pgfpathlineto{\pgfqpoint{6.325423in}{1.133644in}}%
\pgfpathlineto{\pgfqpoint{6.327437in}{1.137704in}}%
\pgfpathlineto{\pgfqpoint{6.329450in}{1.139822in}}%
\pgfpathlineto{\pgfqpoint{6.335490in}{1.141057in}}%
\pgfpathlineto{\pgfqpoint{6.337504in}{1.142337in}}%
\pgfpathlineto{\pgfqpoint{6.339517in}{1.147279in}}%
\pgfpathlineto{\pgfqpoint{6.341531in}{1.145823in}}%
\pgfpathlineto{\pgfqpoint{6.343544in}{1.148250in}}%
\pgfpathlineto{\pgfqpoint{6.349584in}{1.146661in}}%
\pgfpathlineto{\pgfqpoint{6.351598in}{1.143352in}}%
\pgfpathlineto{\pgfqpoint{6.353611in}{1.144499in}}%
\pgfpathlineto{\pgfqpoint{6.355625in}{1.141102in}}%
\pgfpathlineto{\pgfqpoint{6.357638in}{1.143396in}}%
\pgfpathlineto{\pgfqpoint{6.363679in}{1.143308in}}%
\pgfpathlineto{\pgfqpoint{6.367705in}{1.151868in}}%
\pgfpathlineto{\pgfqpoint{6.369719in}{1.153192in}}%
\pgfpathlineto{\pgfqpoint{6.371732in}{1.152795in}}%
\pgfpathlineto{\pgfqpoint{6.377773in}{1.154251in}}%
\pgfpathlineto{\pgfqpoint{6.379786in}{1.153986in}}%
\pgfpathlineto{\pgfqpoint{6.381800in}{1.158751in}}%
\pgfpathlineto{\pgfqpoint{6.383813in}{1.158354in}}%
\pgfpathlineto{\pgfqpoint{6.385827in}{1.160031in}}%
\pgfpathlineto{\pgfqpoint{6.393880in}{1.163208in}}%
\pgfpathlineto{\pgfqpoint{6.395894in}{1.164841in}}%
\pgfpathlineto{\pgfqpoint{6.399921in}{1.162281in}}%
\pgfpathlineto{\pgfqpoint{6.405961in}{1.160208in}}%
\pgfpathlineto{\pgfqpoint{6.407974in}{1.162723in}}%
\pgfpathlineto{\pgfqpoint{6.409988in}{1.155177in}}%
\pgfpathlineto{\pgfqpoint{6.412001in}{1.159369in}}%
\pgfpathlineto{\pgfqpoint{6.414015in}{1.154163in}}%
\pgfpathlineto{\pgfqpoint{6.420055in}{1.154692in}}%
\pgfpathlineto{\pgfqpoint{6.422069in}{1.161267in}}%
\pgfpathlineto{\pgfqpoint{6.424082in}{1.158531in}}%
\pgfpathlineto{\pgfqpoint{6.428109in}{1.163517in}}%
\pgfpathlineto{\pgfqpoint{6.434149in}{1.165326in}}%
\pgfpathlineto{\pgfqpoint{6.436163in}{1.169606in}}%
\pgfpathlineto{\pgfqpoint{6.438176in}{1.154339in}}%
\pgfpathlineto{\pgfqpoint{6.440190in}{1.166253in}}%
\pgfpathlineto{\pgfqpoint{6.442203in}{1.158134in}}%
\pgfpathlineto{\pgfqpoint{6.448243in}{1.143705in}}%
\pgfpathlineto{\pgfqpoint{6.452270in}{1.153810in}}%
\pgfpathlineto{\pgfqpoint{6.456297in}{1.166297in}}%
\pgfpathlineto{\pgfqpoint{6.462337in}{1.165282in}}%
\pgfpathlineto{\pgfqpoint{6.464351in}{1.170047in}}%
\pgfpathlineto{\pgfqpoint{6.466364in}{1.169386in}}%
\pgfpathlineto{\pgfqpoint{6.468378in}{1.167709in}}%
\pgfpathlineto{\pgfqpoint{6.470391in}{1.171636in}}%
\pgfpathlineto{\pgfqpoint{6.476432in}{1.170445in}}%
\pgfpathlineto{\pgfqpoint{6.478445in}{1.164267in}}%
\pgfpathlineto{\pgfqpoint{6.480459in}{1.164179in}}%
\pgfpathlineto{\pgfqpoint{6.484485in}{1.166694in}}%
\pgfpathlineto{\pgfqpoint{6.492539in}{1.167974in}}%
\pgfpathlineto{\pgfqpoint{6.494553in}{1.172474in}}%
\pgfpathlineto{\pgfqpoint{6.496566in}{1.173930in}}%
\pgfpathlineto{\pgfqpoint{6.498580in}{1.172474in}}%
\pgfpathlineto{\pgfqpoint{6.498580in}{1.172474in}}%
\pgfusepath{stroke}%
\end{pgfscope}%
\begin{pgfscope}%
\pgfsetrectcap%
\pgfsetmiterjoin%
\pgfsetlinewidth{0.803000pt}%
\definecolor{currentstroke}{rgb}{1.000000,1.000000,1.000000}%
\pgfsetstrokecolor{currentstroke}%
\pgfsetdash{}{0pt}%
\pgfpathmoveto{\pgfqpoint{1.875000in}{0.750000in}}%
\pgfpathlineto{\pgfqpoint{1.875000in}{1.194118in}}%
\pgfusepath{stroke}%
\end{pgfscope}%
\begin{pgfscope}%
\pgfsetrectcap%
\pgfsetmiterjoin%
\pgfsetlinewidth{0.803000pt}%
\definecolor{currentstroke}{rgb}{1.000000,1.000000,1.000000}%
\pgfsetstrokecolor{currentstroke}%
\pgfsetdash{}{0pt}%
\pgfpathmoveto{\pgfqpoint{6.718750in}{0.750000in}}%
\pgfpathlineto{\pgfqpoint{6.718750in}{1.194118in}}%
\pgfusepath{stroke}%
\end{pgfscope}%
\begin{pgfscope}%
\pgfsetrectcap%
\pgfsetmiterjoin%
\pgfsetlinewidth{0.803000pt}%
\definecolor{currentstroke}{rgb}{1.000000,1.000000,1.000000}%
\pgfsetstrokecolor{currentstroke}%
\pgfsetdash{}{0pt}%
\pgfpathmoveto{\pgfqpoint{1.875000in}{0.750000in}}%
\pgfpathlineto{\pgfqpoint{6.718750in}{0.750000in}}%
\pgfusepath{stroke}%
\end{pgfscope}%
\begin{pgfscope}%
\pgfsetrectcap%
\pgfsetmiterjoin%
\pgfsetlinewidth{0.803000pt}%
\definecolor{currentstroke}{rgb}{1.000000,1.000000,1.000000}%
\pgfsetstrokecolor{currentstroke}%
\pgfsetdash{}{0pt}%
\pgfpathmoveto{\pgfqpoint{1.875000in}{1.194118in}}%
\pgfpathlineto{\pgfqpoint{6.718750in}{1.194118in}}%
\pgfusepath{stroke}%
\end{pgfscope}%
\begin{pgfscope}%
\definecolor{textcolor}{rgb}{0.150000,0.150000,0.150000}%
\pgfsetstrokecolor{textcolor}%
\pgfsetfillcolor{textcolor}%
\pgftext[x=4.296875in,y=1.277451in,,base]{\color{textcolor}\rmfamily\fontsize{16.800000}{20.160000}\selectfont V}%
\end{pgfscope}%
\begin{pgfscope}%
\pgfsetbuttcap%
\pgfsetmiterjoin%
\definecolor{currentfill}{rgb}{0.917647,0.917647,0.949020}%
\pgfsetfillcolor{currentfill}%
\pgfsetlinewidth{0.000000pt}%
\definecolor{currentstroke}{rgb}{0.000000,0.000000,0.000000}%
\pgfsetstrokecolor{currentstroke}%
\pgfsetstrokeopacity{0.000000}%
\pgfsetdash{}{0pt}%
\pgfpathmoveto{\pgfqpoint{8.656250in}{0.750000in}}%
\pgfpathlineto{\pgfqpoint{13.500000in}{0.750000in}}%
\pgfpathlineto{\pgfqpoint{13.500000in}{1.194118in}}%
\pgfpathlineto{\pgfqpoint{8.656250in}{1.194118in}}%
\pgfpathclose%
\pgfusepath{fill}%
\end{pgfscope}%
\begin{pgfscope}%
\pgfpathrectangle{\pgfqpoint{8.656250in}{0.750000in}}{\pgfqpoint{4.843750in}{0.444118in}}%
\pgfusepath{clip}%
\pgfsetroundcap%
\pgfsetroundjoin%
\pgfsetlinewidth{0.803000pt}%
\definecolor{currentstroke}{rgb}{1.000000,1.000000,1.000000}%
\pgfsetstrokecolor{currentstroke}%
\pgfsetdash{}{0pt}%
\pgfpathmoveto{\pgfqpoint{8.872394in}{0.750000in}}%
\pgfpathlineto{\pgfqpoint{8.872394in}{1.194118in}}%
\pgfusepath{stroke}%
\end{pgfscope}%
\begin{pgfscope}%
\definecolor{textcolor}{rgb}{0.150000,0.150000,0.150000}%
\pgfsetstrokecolor{textcolor}%
\pgfsetfillcolor{textcolor}%
\pgftext[x=8.872394in,y=0.652778in,,top]{\color{textcolor}\rmfamily\fontsize{14.000000}{16.800000}\selectfont 2012}%
\end{pgfscope}%
\begin{pgfscope}%
\pgfpathrectangle{\pgfqpoint{8.656250in}{0.750000in}}{\pgfqpoint{4.843750in}{0.444118in}}%
\pgfusepath{clip}%
\pgfsetroundcap%
\pgfsetroundjoin%
\pgfsetlinewidth{0.803000pt}%
\definecolor{currentstroke}{rgb}{1.000000,1.000000,1.000000}%
\pgfsetstrokecolor{currentstroke}%
\pgfsetdash{}{0pt}%
\pgfpathmoveto{\pgfqpoint{9.609315in}{0.750000in}}%
\pgfpathlineto{\pgfqpoint{9.609315in}{1.194118in}}%
\pgfusepath{stroke}%
\end{pgfscope}%
\begin{pgfscope}%
\definecolor{textcolor}{rgb}{0.150000,0.150000,0.150000}%
\pgfsetstrokecolor{textcolor}%
\pgfsetfillcolor{textcolor}%
\pgftext[x=9.609315in,y=0.652778in,,top]{\color{textcolor}\rmfamily\fontsize{14.000000}{16.800000}\selectfont 2013}%
\end{pgfscope}%
\begin{pgfscope}%
\pgfpathrectangle{\pgfqpoint{8.656250in}{0.750000in}}{\pgfqpoint{4.843750in}{0.444118in}}%
\pgfusepath{clip}%
\pgfsetroundcap%
\pgfsetroundjoin%
\pgfsetlinewidth{0.803000pt}%
\definecolor{currentstroke}{rgb}{1.000000,1.000000,1.000000}%
\pgfsetstrokecolor{currentstroke}%
\pgfsetdash{}{0pt}%
\pgfpathmoveto{\pgfqpoint{10.344223in}{0.750000in}}%
\pgfpathlineto{\pgfqpoint{10.344223in}{1.194118in}}%
\pgfusepath{stroke}%
\end{pgfscope}%
\begin{pgfscope}%
\definecolor{textcolor}{rgb}{0.150000,0.150000,0.150000}%
\pgfsetstrokecolor{textcolor}%
\pgfsetfillcolor{textcolor}%
\pgftext[x=10.344223in,y=0.652778in,,top]{\color{textcolor}\rmfamily\fontsize{14.000000}{16.800000}\selectfont 2014}%
\end{pgfscope}%
\begin{pgfscope}%
\pgfpathrectangle{\pgfqpoint{8.656250in}{0.750000in}}{\pgfqpoint{4.843750in}{0.444118in}}%
\pgfusepath{clip}%
\pgfsetroundcap%
\pgfsetroundjoin%
\pgfsetlinewidth{0.803000pt}%
\definecolor{currentstroke}{rgb}{1.000000,1.000000,1.000000}%
\pgfsetstrokecolor{currentstroke}%
\pgfsetdash{}{0pt}%
\pgfpathmoveto{\pgfqpoint{11.079132in}{0.750000in}}%
\pgfpathlineto{\pgfqpoint{11.079132in}{1.194118in}}%
\pgfusepath{stroke}%
\end{pgfscope}%
\begin{pgfscope}%
\definecolor{textcolor}{rgb}{0.150000,0.150000,0.150000}%
\pgfsetstrokecolor{textcolor}%
\pgfsetfillcolor{textcolor}%
\pgftext[x=11.079132in,y=0.652778in,,top]{\color{textcolor}\rmfamily\fontsize{14.000000}{16.800000}\selectfont 2015}%
\end{pgfscope}%
\begin{pgfscope}%
\pgfpathrectangle{\pgfqpoint{8.656250in}{0.750000in}}{\pgfqpoint{4.843750in}{0.444118in}}%
\pgfusepath{clip}%
\pgfsetroundcap%
\pgfsetroundjoin%
\pgfsetlinewidth{0.803000pt}%
\definecolor{currentstroke}{rgb}{1.000000,1.000000,1.000000}%
\pgfsetstrokecolor{currentstroke}%
\pgfsetdash{}{0pt}%
\pgfpathmoveto{\pgfqpoint{11.814040in}{0.750000in}}%
\pgfpathlineto{\pgfqpoint{11.814040in}{1.194118in}}%
\pgfusepath{stroke}%
\end{pgfscope}%
\begin{pgfscope}%
\definecolor{textcolor}{rgb}{0.150000,0.150000,0.150000}%
\pgfsetstrokecolor{textcolor}%
\pgfsetfillcolor{textcolor}%
\pgftext[x=11.814040in,y=0.652778in,,top]{\color{textcolor}\rmfamily\fontsize{14.000000}{16.800000}\selectfont 2016}%
\end{pgfscope}%
\begin{pgfscope}%
\pgfpathrectangle{\pgfqpoint{8.656250in}{0.750000in}}{\pgfqpoint{4.843750in}{0.444118in}}%
\pgfusepath{clip}%
\pgfsetroundcap%
\pgfsetroundjoin%
\pgfsetlinewidth{0.803000pt}%
\definecolor{currentstroke}{rgb}{1.000000,1.000000,1.000000}%
\pgfsetstrokecolor{currentstroke}%
\pgfsetdash{}{0pt}%
\pgfpathmoveto{\pgfqpoint{12.550962in}{0.750000in}}%
\pgfpathlineto{\pgfqpoint{12.550962in}{1.194118in}}%
\pgfusepath{stroke}%
\end{pgfscope}%
\begin{pgfscope}%
\definecolor{textcolor}{rgb}{0.150000,0.150000,0.150000}%
\pgfsetstrokecolor{textcolor}%
\pgfsetfillcolor{textcolor}%
\pgftext[x=12.550962in,y=0.652778in,,top]{\color{textcolor}\rmfamily\fontsize{14.000000}{16.800000}\selectfont 2017}%
\end{pgfscope}%
\begin{pgfscope}%
\pgfpathrectangle{\pgfqpoint{8.656250in}{0.750000in}}{\pgfqpoint{4.843750in}{0.444118in}}%
\pgfusepath{clip}%
\pgfsetroundcap%
\pgfsetroundjoin%
\pgfsetlinewidth{0.803000pt}%
\definecolor{currentstroke}{rgb}{1.000000,1.000000,1.000000}%
\pgfsetstrokecolor{currentstroke}%
\pgfsetdash{}{0pt}%
\pgfpathmoveto{\pgfqpoint{13.285870in}{0.750000in}}%
\pgfpathlineto{\pgfqpoint{13.285870in}{1.194118in}}%
\pgfusepath{stroke}%
\end{pgfscope}%
\begin{pgfscope}%
\definecolor{textcolor}{rgb}{0.150000,0.150000,0.150000}%
\pgfsetstrokecolor{textcolor}%
\pgfsetfillcolor{textcolor}%
\pgftext[x=13.285870in,y=0.652778in,,top]{\color{textcolor}\rmfamily\fontsize{14.000000}{16.800000}\selectfont 2018}%
\end{pgfscope}%
\begin{pgfscope}%
\pgfpathrectangle{\pgfqpoint{8.656250in}{0.750000in}}{\pgfqpoint{4.843750in}{0.444118in}}%
\pgfusepath{clip}%
\pgfsetroundcap%
\pgfsetroundjoin%
\pgfsetlinewidth{0.803000pt}%
\definecolor{currentstroke}{rgb}{1.000000,1.000000,1.000000}%
\pgfsetstrokecolor{currentstroke}%
\pgfsetdash{}{0pt}%
\pgfpathmoveto{\pgfqpoint{8.656250in}{0.848118in}}%
\pgfpathlineto{\pgfqpoint{13.500000in}{0.848118in}}%
\pgfusepath{stroke}%
\end{pgfscope}%
\begin{pgfscope}%
\definecolor{textcolor}{rgb}{0.150000,0.150000,0.150000}%
\pgfsetstrokecolor{textcolor}%
\pgfsetfillcolor{textcolor}%
\pgftext[x=8.311605in,y=0.774252in,left,base]{\color{textcolor}\rmfamily\fontsize{14.000000}{16.800000}\selectfont 50}%
\end{pgfscope}%
\begin{pgfscope}%
\pgfpathrectangle{\pgfqpoint{8.656250in}{0.750000in}}{\pgfqpoint{4.843750in}{0.444118in}}%
\pgfusepath{clip}%
\pgfsetroundcap%
\pgfsetroundjoin%
\pgfsetlinewidth{0.803000pt}%
\definecolor{currentstroke}{rgb}{1.000000,1.000000,1.000000}%
\pgfsetstrokecolor{currentstroke}%
\pgfsetdash{}{0pt}%
\pgfpathmoveto{\pgfqpoint{8.656250in}{1.099672in}}%
\pgfpathlineto{\pgfqpoint{13.500000in}{1.099672in}}%
\pgfusepath{stroke}%
\end{pgfscope}%
\begin{pgfscope}%
\definecolor{textcolor}{rgb}{0.150000,0.150000,0.150000}%
\pgfsetstrokecolor{textcolor}%
\pgfsetfillcolor{textcolor}%
\pgftext[x=8.187893in,y=1.025806in,left,base]{\color{textcolor}\rmfamily\fontsize{14.000000}{16.800000}\selectfont 100}%
\end{pgfscope}%
\begin{pgfscope}%
\pgfpathrectangle{\pgfqpoint{8.656250in}{0.750000in}}{\pgfqpoint{4.843750in}{0.444118in}}%
\pgfusepath{clip}%
\pgfsetroundcap%
\pgfsetroundjoin%
\pgfsetlinewidth{1.505625pt}%
\definecolor{currentstroke}{rgb}{0.121569,0.466667,0.705882}%
\pgfsetstrokecolor{currentstroke}%
\pgfsetdash{}{0pt}%
\pgfpathmoveto{\pgfqpoint{8.876420in}{0.770187in}}%
\pgfpathlineto{\pgfqpoint{8.882461in}{0.777432in}}%
\pgfpathlineto{\pgfqpoint{8.890515in}{0.776174in}}%
\pgfpathlineto{\pgfqpoint{8.892528in}{0.771948in}}%
\pgfpathlineto{\pgfqpoint{8.894541in}{0.772049in}}%
\pgfpathlineto{\pgfqpoint{8.896555in}{0.770590in}}%
\pgfpathlineto{\pgfqpoint{8.904609in}{0.770942in}}%
\pgfpathlineto{\pgfqpoint{8.908636in}{0.775269in}}%
\pgfpathlineto{\pgfqpoint{8.910649in}{0.774715in}}%
\pgfpathlineto{\pgfqpoint{8.918703in}{0.774413in}}%
\pgfpathlineto{\pgfqpoint{8.920716in}{0.775822in}}%
\pgfpathlineto{\pgfqpoint{8.924743in}{0.774413in}}%
\pgfpathlineto{\pgfqpoint{8.932797in}{0.772854in}}%
\pgfpathlineto{\pgfqpoint{8.934810in}{0.774765in}}%
\pgfpathlineto{\pgfqpoint{8.936824in}{0.772904in}}%
\pgfpathlineto{\pgfqpoint{8.938837in}{0.777834in}}%
\pgfpathlineto{\pgfqpoint{8.944878in}{0.779897in}}%
\pgfpathlineto{\pgfqpoint{8.946891in}{0.782262in}}%
\pgfpathlineto{\pgfqpoint{8.950918in}{0.784777in}}%
\pgfpathlineto{\pgfqpoint{8.952931in}{0.784375in}}%
\pgfpathlineto{\pgfqpoint{8.958972in}{0.785934in}}%
\pgfpathlineto{\pgfqpoint{8.960985in}{0.785079in}}%
\pgfpathlineto{\pgfqpoint{8.962999in}{0.783469in}}%
\pgfpathlineto{\pgfqpoint{8.967026in}{0.785733in}}%
\pgfpathlineto{\pgfqpoint{8.975079in}{0.784928in}}%
\pgfpathlineto{\pgfqpoint{8.977093in}{0.783570in}}%
\pgfpathlineto{\pgfqpoint{8.979106in}{0.784526in}}%
\pgfpathlineto{\pgfqpoint{8.981120in}{0.783771in}}%
\pgfpathlineto{\pgfqpoint{8.987160in}{0.785280in}}%
\pgfpathlineto{\pgfqpoint{8.989173in}{0.786588in}}%
\pgfpathlineto{\pgfqpoint{8.991187in}{0.786840in}}%
\pgfpathlineto{\pgfqpoint{8.993200in}{0.788651in}}%
\pgfpathlineto{\pgfqpoint{8.995214in}{0.788500in}}%
\pgfpathlineto{\pgfqpoint{9.001254in}{0.790060in}}%
\pgfpathlineto{\pgfqpoint{9.003268in}{0.786890in}}%
\pgfpathlineto{\pgfqpoint{9.005281in}{0.785733in}}%
\pgfpathlineto{\pgfqpoint{9.009308in}{0.787997in}}%
\pgfpathlineto{\pgfqpoint{9.015348in}{0.788450in}}%
\pgfpathlineto{\pgfqpoint{9.017362in}{0.795997in}}%
\pgfpathlineto{\pgfqpoint{9.019375in}{0.793582in}}%
\pgfpathlineto{\pgfqpoint{9.021389in}{0.793531in}}%
\pgfpathlineto{\pgfqpoint{9.023402in}{0.792274in}}%
\pgfpathlineto{\pgfqpoint{9.029442in}{0.793431in}}%
\pgfpathlineto{\pgfqpoint{9.031456in}{0.792525in}}%
\pgfpathlineto{\pgfqpoint{9.035483in}{0.792726in}}%
\pgfpathlineto{\pgfqpoint{9.037496in}{0.794387in}}%
\pgfpathlineto{\pgfqpoint{9.043537in}{0.797657in}}%
\pgfpathlineto{\pgfqpoint{9.045550in}{0.796651in}}%
\pgfpathlineto{\pgfqpoint{9.049577in}{0.791469in}}%
\pgfpathlineto{\pgfqpoint{9.051590in}{0.794940in}}%
\pgfpathlineto{\pgfqpoint{9.057631in}{0.795242in}}%
\pgfpathlineto{\pgfqpoint{9.059644in}{0.792726in}}%
\pgfpathlineto{\pgfqpoint{9.061658in}{0.791116in}}%
\pgfpathlineto{\pgfqpoint{9.063671in}{0.791770in}}%
\pgfpathlineto{\pgfqpoint{9.071725in}{0.787393in}}%
\pgfpathlineto{\pgfqpoint{9.073738in}{0.782312in}}%
\pgfpathlineto{\pgfqpoint{9.075752in}{0.784023in}}%
\pgfpathlineto{\pgfqpoint{9.077765in}{0.787544in}}%
\pgfpathlineto{\pgfqpoint{9.079779in}{0.786186in}}%
\pgfpathlineto{\pgfqpoint{9.085819in}{0.785331in}}%
\pgfpathlineto{\pgfqpoint{9.087832in}{0.789959in}}%
\pgfpathlineto{\pgfqpoint{9.089846in}{0.789104in}}%
\pgfpathlineto{\pgfqpoint{9.091859in}{0.787242in}}%
\pgfpathlineto{\pgfqpoint{9.093873in}{0.788450in}}%
\pgfpathlineto{\pgfqpoint{9.099913in}{0.786941in}}%
\pgfpathlineto{\pgfqpoint{9.101927in}{0.787695in}}%
\pgfpathlineto{\pgfqpoint{9.105953in}{0.793028in}}%
\pgfpathlineto{\pgfqpoint{9.107967in}{0.793028in}}%
\pgfpathlineto{\pgfqpoint{9.114007in}{0.791921in}}%
\pgfpathlineto{\pgfqpoint{9.116021in}{0.794990in}}%
\pgfpathlineto{\pgfqpoint{9.118034in}{0.793884in}}%
\pgfpathlineto{\pgfqpoint{9.120048in}{0.795091in}}%
\pgfpathlineto{\pgfqpoint{9.122061in}{0.791116in}}%
\pgfpathlineto{\pgfqpoint{9.128101in}{0.795141in}}%
\pgfpathlineto{\pgfqpoint{9.130115in}{0.797305in}}%
\pgfpathlineto{\pgfqpoint{9.132128in}{0.800575in}}%
\pgfpathlineto{\pgfqpoint{9.136155in}{0.803040in}}%
\pgfpathlineto{\pgfqpoint{9.144209in}{0.800525in}}%
\pgfpathlineto{\pgfqpoint{9.146222in}{0.800826in}}%
\pgfpathlineto{\pgfqpoint{9.150249in}{0.795091in}}%
\pgfpathlineto{\pgfqpoint{9.156290in}{0.797707in}}%
\pgfpathlineto{\pgfqpoint{9.158303in}{0.797707in}}%
\pgfpathlineto{\pgfqpoint{9.160317in}{0.796902in}}%
\pgfpathlineto{\pgfqpoint{9.162330in}{0.797959in}}%
\pgfpathlineto{\pgfqpoint{9.164343in}{0.798210in}}%
\pgfpathlineto{\pgfqpoint{9.172397in}{0.802638in}}%
\pgfpathlineto{\pgfqpoint{9.174411in}{0.801380in}}%
\pgfpathlineto{\pgfqpoint{9.176424in}{0.803694in}}%
\pgfpathlineto{\pgfqpoint{9.178438in}{0.797757in}}%
\pgfpathlineto{\pgfqpoint{9.184478in}{0.797808in}}%
\pgfpathlineto{\pgfqpoint{9.186491in}{0.799770in}}%
\pgfpathlineto{\pgfqpoint{9.188505in}{0.802889in}}%
\pgfpathlineto{\pgfqpoint{9.190518in}{0.803342in}}%
\pgfpathlineto{\pgfqpoint{9.192532in}{0.806109in}}%
\pgfpathlineto{\pgfqpoint{9.198572in}{0.804097in}}%
\pgfpathlineto{\pgfqpoint{9.200585in}{0.806713in}}%
\pgfpathlineto{\pgfqpoint{9.202599in}{0.806059in}}%
\pgfpathlineto{\pgfqpoint{9.204612in}{0.810385in}}%
\pgfpathlineto{\pgfqpoint{9.206626in}{0.809933in}}%
\pgfpathlineto{\pgfqpoint{9.212666in}{0.809983in}}%
\pgfpathlineto{\pgfqpoint{9.214680in}{0.811844in}}%
\pgfpathlineto{\pgfqpoint{9.216693in}{0.812851in}}%
\pgfpathlineto{\pgfqpoint{9.218706in}{0.811341in}}%
\pgfpathlineto{\pgfqpoint{9.220720in}{0.811693in}}%
\pgfpathlineto{\pgfqpoint{9.226760in}{0.808172in}}%
\pgfpathlineto{\pgfqpoint{9.230787in}{0.813505in}}%
\pgfpathlineto{\pgfqpoint{9.232801in}{0.813253in}}%
\pgfpathlineto{\pgfqpoint{9.234814in}{0.816322in}}%
\pgfpathlineto{\pgfqpoint{9.240854in}{0.817328in}}%
\pgfpathlineto{\pgfqpoint{9.242868in}{0.816775in}}%
\pgfpathlineto{\pgfqpoint{9.246895in}{0.814561in}}%
\pgfpathlineto{\pgfqpoint{9.248908in}{0.814259in}}%
\pgfpathlineto{\pgfqpoint{9.254949in}{0.814058in}}%
\pgfpathlineto{\pgfqpoint{9.256962in}{0.811190in}}%
\pgfpathlineto{\pgfqpoint{9.258975in}{0.810788in}}%
\pgfpathlineto{\pgfqpoint{9.260989in}{0.811392in}}%
\pgfpathlineto{\pgfqpoint{9.263002in}{0.814964in}}%
\pgfpathlineto{\pgfqpoint{9.269043in}{0.813454in}}%
\pgfpathlineto{\pgfqpoint{9.271056in}{0.820196in}}%
\pgfpathlineto{\pgfqpoint{9.273070in}{0.820196in}}%
\pgfpathlineto{\pgfqpoint{9.277096in}{0.816775in}}%
\pgfpathlineto{\pgfqpoint{9.283137in}{0.814008in}}%
\pgfpathlineto{\pgfqpoint{9.287164in}{0.815416in}}%
\pgfpathlineto{\pgfqpoint{9.289177in}{0.821806in}}%
\pgfpathlineto{\pgfqpoint{9.291191in}{0.822862in}}%
\pgfpathlineto{\pgfqpoint{9.297231in}{0.822259in}}%
\pgfpathlineto{\pgfqpoint{9.299244in}{0.819240in}}%
\pgfpathlineto{\pgfqpoint{9.301258in}{0.817781in}}%
\pgfpathlineto{\pgfqpoint{9.303271in}{0.818536in}}%
\pgfpathlineto{\pgfqpoint{9.305285in}{0.822108in}}%
\pgfpathlineto{\pgfqpoint{9.311325in}{0.821554in}}%
\pgfpathlineto{\pgfqpoint{9.313339in}{0.822259in}}%
\pgfpathlineto{\pgfqpoint{9.315352in}{0.825378in}}%
\pgfpathlineto{\pgfqpoint{9.317365in}{0.822963in}}%
\pgfpathlineto{\pgfqpoint{9.319379in}{0.821554in}}%
\pgfpathlineto{\pgfqpoint{9.325419in}{0.822510in}}%
\pgfpathlineto{\pgfqpoint{9.327433in}{0.821756in}}%
\pgfpathlineto{\pgfqpoint{9.329446in}{0.822661in}}%
\pgfpathlineto{\pgfqpoint{9.333473in}{0.825227in}}%
\pgfpathlineto{\pgfqpoint{9.339513in}{0.825177in}}%
\pgfpathlineto{\pgfqpoint{9.341527in}{0.821504in}}%
\pgfpathlineto{\pgfqpoint{9.343540in}{0.821605in}}%
\pgfpathlineto{\pgfqpoint{9.345554in}{0.819391in}}%
\pgfpathlineto{\pgfqpoint{9.347567in}{0.821152in}}%
\pgfpathlineto{\pgfqpoint{9.355621in}{0.821454in}}%
\pgfpathlineto{\pgfqpoint{9.357634in}{0.823315in}}%
\pgfpathlineto{\pgfqpoint{9.359648in}{0.820498in}}%
\pgfpathlineto{\pgfqpoint{9.369715in}{0.821605in}}%
\pgfpathlineto{\pgfqpoint{9.373742in}{0.831566in}}%
\pgfpathlineto{\pgfqpoint{9.375755in}{0.831013in}}%
\pgfpathlineto{\pgfqpoint{9.383809in}{0.830208in}}%
\pgfpathlineto{\pgfqpoint{9.385823in}{0.830862in}}%
\pgfpathlineto{\pgfqpoint{9.387836in}{0.834937in}}%
\pgfpathlineto{\pgfqpoint{9.389849in}{0.833780in}}%
\pgfpathlineto{\pgfqpoint{9.395890in}{0.832874in}}%
\pgfpathlineto{\pgfqpoint{9.397903in}{0.831767in}}%
\pgfpathlineto{\pgfqpoint{9.399917in}{0.835390in}}%
\pgfpathlineto{\pgfqpoint{9.403944in}{0.835541in}}%
\pgfpathlineto{\pgfqpoint{9.409984in}{0.836396in}}%
\pgfpathlineto{\pgfqpoint{9.411997in}{0.834635in}}%
\pgfpathlineto{\pgfqpoint{9.414011in}{0.831918in}}%
\pgfpathlineto{\pgfqpoint{9.416024in}{0.834836in}}%
\pgfpathlineto{\pgfqpoint{9.418038in}{0.833478in}}%
\pgfpathlineto{\pgfqpoint{9.424078in}{0.832522in}}%
\pgfpathlineto{\pgfqpoint{9.426092in}{0.830560in}}%
\pgfpathlineto{\pgfqpoint{9.428105in}{0.834333in}}%
\pgfpathlineto{\pgfqpoint{9.430118in}{0.835038in}}%
\pgfpathlineto{\pgfqpoint{9.432132in}{0.836597in}}%
\pgfpathlineto{\pgfqpoint{9.438172in}{0.833679in}}%
\pgfpathlineto{\pgfqpoint{9.440186in}{0.830007in}}%
\pgfpathlineto{\pgfqpoint{9.442199in}{0.828648in}}%
\pgfpathlineto{\pgfqpoint{9.444213in}{0.824674in}}%
\pgfpathlineto{\pgfqpoint{9.446226in}{0.825831in}}%
\pgfpathlineto{\pgfqpoint{9.452266in}{0.826736in}}%
\pgfpathlineto{\pgfqpoint{9.454280in}{0.828749in}}%
\pgfpathlineto{\pgfqpoint{9.456293in}{0.833528in}}%
\pgfpathlineto{\pgfqpoint{9.458307in}{0.834132in}}%
\pgfpathlineto{\pgfqpoint{9.460320in}{0.831767in}}%
\pgfpathlineto{\pgfqpoint{9.466360in}{0.831264in}}%
\pgfpathlineto{\pgfqpoint{9.468374in}{0.826585in}}%
\pgfpathlineto{\pgfqpoint{9.470387in}{0.826082in}}%
\pgfpathlineto{\pgfqpoint{9.472401in}{0.824321in}}%
\pgfpathlineto{\pgfqpoint{9.484482in}{0.819139in}}%
\pgfpathlineto{\pgfqpoint{9.486495in}{0.822158in}}%
\pgfpathlineto{\pgfqpoint{9.488508in}{0.822510in}}%
\pgfpathlineto{\pgfqpoint{9.496562in}{0.825277in}}%
\pgfpathlineto{\pgfqpoint{9.498576in}{0.823517in}}%
\pgfpathlineto{\pgfqpoint{9.500589in}{0.823315in}}%
\pgfpathlineto{\pgfqpoint{9.502603in}{0.809832in}}%
\pgfpathlineto{\pgfqpoint{9.508643in}{0.811593in}}%
\pgfpathlineto{\pgfqpoint{9.510656in}{0.813907in}}%
\pgfpathlineto{\pgfqpoint{9.512670in}{0.810335in}}%
\pgfpathlineto{\pgfqpoint{9.514683in}{0.811693in}}%
\pgfpathlineto{\pgfqpoint{9.516697in}{0.811442in}}%
\pgfpathlineto{\pgfqpoint{9.522737in}{0.813656in}}%
\pgfpathlineto{\pgfqpoint{9.524750in}{0.816071in}}%
\pgfpathlineto{\pgfqpoint{9.530791in}{0.819794in}}%
\pgfpathlineto{\pgfqpoint{9.536831in}{0.818737in}}%
\pgfpathlineto{\pgfqpoint{9.538845in}{0.816775in}}%
\pgfpathlineto{\pgfqpoint{9.542871in}{0.821856in}}%
\pgfpathlineto{\pgfqpoint{9.544885in}{0.821605in}}%
\pgfpathlineto{\pgfqpoint{9.552939in}{0.819944in}}%
\pgfpathlineto{\pgfqpoint{9.556966in}{0.822309in}}%
\pgfpathlineto{\pgfqpoint{9.558979in}{0.823114in}}%
\pgfpathlineto{\pgfqpoint{9.565019in}{0.823416in}}%
\pgfpathlineto{\pgfqpoint{9.569046in}{0.824976in}}%
\pgfpathlineto{\pgfqpoint{9.573073in}{0.820498in}}%
\pgfpathlineto{\pgfqpoint{9.579114in}{0.823315in}}%
\pgfpathlineto{\pgfqpoint{9.581127in}{0.827642in}}%
\pgfpathlineto{\pgfqpoint{9.583140in}{0.826334in}}%
\pgfpathlineto{\pgfqpoint{9.585154in}{0.830912in}}%
\pgfpathlineto{\pgfqpoint{9.587167in}{0.826636in}}%
\pgfpathlineto{\pgfqpoint{9.597235in}{0.825931in}}%
\pgfpathlineto{\pgfqpoint{9.601261in}{0.822712in}}%
\pgfpathlineto{\pgfqpoint{9.607302in}{0.825630in}}%
\pgfpathlineto{\pgfqpoint{9.611329in}{0.831667in}}%
\pgfpathlineto{\pgfqpoint{9.613342in}{0.832170in}}%
\pgfpathlineto{\pgfqpoint{9.615356in}{0.836698in}}%
\pgfpathlineto{\pgfqpoint{9.621396in}{0.831063in}}%
\pgfpathlineto{\pgfqpoint{9.623409in}{0.830107in}}%
\pgfpathlineto{\pgfqpoint{9.627436in}{0.830258in}}%
\pgfpathlineto{\pgfqpoint{9.629450in}{0.829302in}}%
\pgfpathlineto{\pgfqpoint{9.635490in}{0.829353in}}%
\pgfpathlineto{\pgfqpoint{9.639517in}{0.833679in}}%
\pgfpathlineto{\pgfqpoint{9.641530in}{0.837704in}}%
\pgfpathlineto{\pgfqpoint{9.643544in}{0.837402in}}%
\pgfpathlineto{\pgfqpoint{9.651598in}{0.839163in}}%
\pgfpathlineto{\pgfqpoint{9.653611in}{0.844798in}}%
\pgfpathlineto{\pgfqpoint{9.655625in}{0.844798in}}%
\pgfpathlineto{\pgfqpoint{9.657638in}{0.846760in}}%
\pgfpathlineto{\pgfqpoint{9.663678in}{0.846659in}}%
\pgfpathlineto{\pgfqpoint{9.667705in}{0.844043in}}%
\pgfpathlineto{\pgfqpoint{9.669719in}{0.844496in}}%
\pgfpathlineto{\pgfqpoint{9.671732in}{0.847716in}}%
\pgfpathlineto{\pgfqpoint{9.677772in}{0.844546in}}%
\pgfpathlineto{\pgfqpoint{9.679786in}{0.846358in}}%
\pgfpathlineto{\pgfqpoint{9.681799in}{0.847414in}}%
\pgfpathlineto{\pgfqpoint{9.683813in}{0.846659in}}%
\pgfpathlineto{\pgfqpoint{9.685826in}{0.848068in}}%
\pgfpathlineto{\pgfqpoint{9.691867in}{0.848471in}}%
\pgfpathlineto{\pgfqpoint{9.693880in}{0.849376in}}%
\pgfpathlineto{\pgfqpoint{9.697907in}{0.849074in}}%
\pgfpathlineto{\pgfqpoint{9.699920in}{0.852445in}}%
\pgfpathlineto{\pgfqpoint{9.707974in}{0.852999in}}%
\pgfpathlineto{\pgfqpoint{9.709988in}{0.847766in}}%
\pgfpathlineto{\pgfqpoint{9.712001in}{0.845804in}}%
\pgfpathlineto{\pgfqpoint{9.714015in}{0.846156in}}%
\pgfpathlineto{\pgfqpoint{9.720055in}{0.843138in}}%
\pgfpathlineto{\pgfqpoint{9.722068in}{0.844546in}}%
\pgfpathlineto{\pgfqpoint{9.724082in}{0.847213in}}%
\pgfpathlineto{\pgfqpoint{9.726095in}{0.847716in}}%
\pgfpathlineto{\pgfqpoint{9.728109in}{0.851137in}}%
\pgfpathlineto{\pgfqpoint{9.734149in}{0.853300in}}%
\pgfpathlineto{\pgfqpoint{9.736162in}{0.856420in}}%
\pgfpathlineto{\pgfqpoint{9.740189in}{0.855715in}}%
\pgfpathlineto{\pgfqpoint{9.742203in}{0.860646in}}%
\pgfpathlineto{\pgfqpoint{9.748243in}{0.861853in}}%
\pgfpathlineto{\pgfqpoint{9.750257in}{0.859338in}}%
\pgfpathlineto{\pgfqpoint{9.752270in}{0.860394in}}%
\pgfpathlineto{\pgfqpoint{9.754283in}{0.862256in}}%
\pgfpathlineto{\pgfqpoint{9.756297in}{0.861501in}}%
\pgfpathlineto{\pgfqpoint{9.762337in}{0.858030in}}%
\pgfpathlineto{\pgfqpoint{9.764351in}{0.855665in}}%
\pgfpathlineto{\pgfqpoint{9.766364in}{0.858533in}}%
\pgfpathlineto{\pgfqpoint{9.768378in}{0.855665in}}%
\pgfpathlineto{\pgfqpoint{9.770391in}{0.857828in}}%
\pgfpathlineto{\pgfqpoint{9.776431in}{0.855212in}}%
\pgfpathlineto{\pgfqpoint{9.778445in}{0.857124in}}%
\pgfpathlineto{\pgfqpoint{9.780458in}{0.856369in}}%
\pgfpathlineto{\pgfqpoint{9.782472in}{0.857929in}}%
\pgfpathlineto{\pgfqpoint{9.790525in}{0.857426in}}%
\pgfpathlineto{\pgfqpoint{9.792539in}{0.860948in}}%
\pgfpathlineto{\pgfqpoint{9.794552in}{0.859992in}}%
\pgfpathlineto{\pgfqpoint{9.796566in}{0.861551in}}%
\pgfpathlineto{\pgfqpoint{9.798579in}{0.862054in}}%
\pgfpathlineto{\pgfqpoint{9.806633in}{0.868696in}}%
\pgfpathlineto{\pgfqpoint{9.808647in}{0.873123in}}%
\pgfpathlineto{\pgfqpoint{9.810660in}{0.875186in}}%
\pgfpathlineto{\pgfqpoint{9.812673in}{0.875186in}}%
\pgfpathlineto{\pgfqpoint{9.818714in}{0.867488in}}%
\pgfpathlineto{\pgfqpoint{9.820727in}{0.876091in}}%
\pgfpathlineto{\pgfqpoint{9.822741in}{0.875739in}}%
\pgfpathlineto{\pgfqpoint{9.824754in}{0.872569in}}%
\pgfpathlineto{\pgfqpoint{9.826768in}{0.879814in}}%
\pgfpathlineto{\pgfqpoint{9.832808in}{0.881877in}}%
\pgfpathlineto{\pgfqpoint{9.834821in}{0.884543in}}%
\pgfpathlineto{\pgfqpoint{9.836835in}{0.881575in}}%
\pgfpathlineto{\pgfqpoint{9.838848in}{0.881827in}}%
\pgfpathlineto{\pgfqpoint{9.840862in}{0.881223in}}%
\pgfpathlineto{\pgfqpoint{9.846902in}{0.886455in}}%
\pgfpathlineto{\pgfqpoint{9.848915in}{0.885701in}}%
\pgfpathlineto{\pgfqpoint{9.850929in}{0.887411in}}%
\pgfpathlineto{\pgfqpoint{9.852942in}{0.890480in}}%
\pgfpathlineto{\pgfqpoint{9.854956in}{0.894706in}}%
\pgfpathlineto{\pgfqpoint{9.860996in}{0.895914in}}%
\pgfpathlineto{\pgfqpoint{9.863010in}{0.900542in}}%
\pgfpathlineto{\pgfqpoint{9.865023in}{0.900190in}}%
\pgfpathlineto{\pgfqpoint{9.869050in}{0.905774in}}%
\pgfpathlineto{\pgfqpoint{9.877104in}{0.906982in}}%
\pgfpathlineto{\pgfqpoint{9.879117in}{0.907938in}}%
\pgfpathlineto{\pgfqpoint{9.881131in}{0.902404in}}%
\pgfpathlineto{\pgfqpoint{9.883144in}{0.902907in}}%
\pgfpathlineto{\pgfqpoint{9.889184in}{0.900794in}}%
\pgfpathlineto{\pgfqpoint{9.895225in}{0.896719in}}%
\pgfpathlineto{\pgfqpoint{9.897238in}{0.897876in}}%
\pgfpathlineto{\pgfqpoint{9.905292in}{0.903410in}}%
\pgfpathlineto{\pgfqpoint{9.907305in}{0.901448in}}%
\pgfpathlineto{\pgfqpoint{9.911332in}{0.886807in}}%
\pgfpathlineto{\pgfqpoint{9.917373in}{0.890128in}}%
\pgfpathlineto{\pgfqpoint{9.919386in}{0.892643in}}%
\pgfpathlineto{\pgfqpoint{9.921400in}{0.887009in}}%
\pgfpathlineto{\pgfqpoint{9.923413in}{0.887059in}}%
\pgfpathlineto{\pgfqpoint{9.925426in}{0.894958in}}%
\pgfpathlineto{\pgfqpoint{9.931467in}{0.890279in}}%
\pgfpathlineto{\pgfqpoint{9.933480in}{0.890128in}}%
\pgfpathlineto{\pgfqpoint{9.935494in}{0.886405in}}%
\pgfpathlineto{\pgfqpoint{9.937507in}{0.892392in}}%
\pgfpathlineto{\pgfqpoint{9.939521in}{0.890128in}}%
\pgfpathlineto{\pgfqpoint{9.945561in}{0.893297in}}%
\pgfpathlineto{\pgfqpoint{9.947574in}{0.896970in}}%
\pgfpathlineto{\pgfqpoint{9.949588in}{0.892543in}}%
\pgfpathlineto{\pgfqpoint{9.951601in}{0.881726in}}%
\pgfpathlineto{\pgfqpoint{9.953615in}{0.885197in}}%
\pgfpathlineto{\pgfqpoint{9.959655in}{0.883839in}}%
\pgfpathlineto{\pgfqpoint{9.961669in}{0.884443in}}%
\pgfpathlineto{\pgfqpoint{9.965695in}{0.889725in}}%
\pgfpathlineto{\pgfqpoint{9.967709in}{0.887109in}}%
\pgfpathlineto{\pgfqpoint{9.973749in}{0.890732in}}%
\pgfpathlineto{\pgfqpoint{9.975763in}{0.887612in}}%
\pgfpathlineto{\pgfqpoint{9.977776in}{0.889222in}}%
\pgfpathlineto{\pgfqpoint{9.981803in}{0.890228in}}%
\pgfpathlineto{\pgfqpoint{9.989857in}{0.895360in}}%
\pgfpathlineto{\pgfqpoint{9.991870in}{0.895209in}}%
\pgfpathlineto{\pgfqpoint{9.993884in}{0.902907in}}%
\pgfpathlineto{\pgfqpoint{9.995897in}{0.904768in}}%
\pgfpathlineto{\pgfqpoint{10.001937in}{0.899938in}}%
\pgfpathlineto{\pgfqpoint{10.003951in}{0.895763in}}%
\pgfpathlineto{\pgfqpoint{10.007978in}{0.899385in}}%
\pgfpathlineto{\pgfqpoint{10.009991in}{0.896366in}}%
\pgfpathlineto{\pgfqpoint{10.016032in}{0.892895in}}%
\pgfpathlineto{\pgfqpoint{10.018045in}{0.893096in}}%
\pgfpathlineto{\pgfqpoint{10.020058in}{0.894002in}}%
\pgfpathlineto{\pgfqpoint{10.022072in}{0.893549in}}%
\pgfpathlineto{\pgfqpoint{10.024085in}{0.895561in}}%
\pgfpathlineto{\pgfqpoint{10.030126in}{0.893851in}}%
\pgfpathlineto{\pgfqpoint{10.032139in}{0.891889in}}%
\pgfpathlineto{\pgfqpoint{10.034153in}{0.894002in}}%
\pgfpathlineto{\pgfqpoint{10.036166in}{0.897272in}}%
\pgfpathlineto{\pgfqpoint{10.038180in}{0.902605in}}%
\pgfpathlineto{\pgfqpoint{10.044220in}{0.900341in}}%
\pgfpathlineto{\pgfqpoint{10.046233in}{0.905070in}}%
\pgfpathlineto{\pgfqpoint{10.048247in}{0.899838in}}%
\pgfpathlineto{\pgfqpoint{10.050260in}{0.899083in}}%
\pgfpathlineto{\pgfqpoint{10.052274in}{0.894404in}}%
\pgfpathlineto{\pgfqpoint{10.058314in}{0.890732in}}%
\pgfpathlineto{\pgfqpoint{10.062341in}{0.890832in}}%
\pgfpathlineto{\pgfqpoint{10.064354in}{0.883587in}}%
\pgfpathlineto{\pgfqpoint{10.066368in}{0.882632in}}%
\pgfpathlineto{\pgfqpoint{10.072408in}{0.881072in}}%
\pgfpathlineto{\pgfqpoint{10.074422in}{0.881273in}}%
\pgfpathlineto{\pgfqpoint{10.076435in}{0.877852in}}%
\pgfpathlineto{\pgfqpoint{10.078448in}{0.880166in}}%
\pgfpathlineto{\pgfqpoint{10.080462in}{0.880569in}}%
\pgfpathlineto{\pgfqpoint{10.086502in}{0.878858in}}%
\pgfpathlineto{\pgfqpoint{10.088516in}{0.875789in}}%
\pgfpathlineto{\pgfqpoint{10.090529in}{0.876343in}}%
\pgfpathlineto{\pgfqpoint{10.092543in}{0.877600in}}%
\pgfpathlineto{\pgfqpoint{10.094556in}{0.876443in}}%
\pgfpathlineto{\pgfqpoint{10.102610in}{0.876896in}}%
\pgfpathlineto{\pgfqpoint{10.108650in}{0.879009in}}%
\pgfpathlineto{\pgfqpoint{10.114691in}{0.879965in}}%
\pgfpathlineto{\pgfqpoint{10.122744in}{0.903410in}}%
\pgfpathlineto{\pgfqpoint{10.132812in}{0.905322in}}%
\pgfpathlineto{\pgfqpoint{10.134825in}{0.898932in}}%
\pgfpathlineto{\pgfqpoint{10.136838in}{0.895662in}}%
\pgfpathlineto{\pgfqpoint{10.142879in}{0.894505in}}%
\pgfpathlineto{\pgfqpoint{10.144892in}{0.892492in}}%
\pgfpathlineto{\pgfqpoint{10.146906in}{0.893096in}}%
\pgfpathlineto{\pgfqpoint{10.148919in}{0.896719in}}%
\pgfpathlineto{\pgfqpoint{10.150933in}{0.896517in}}%
\pgfpathlineto{\pgfqpoint{10.156973in}{0.893297in}}%
\pgfpathlineto{\pgfqpoint{10.158986in}{0.894857in}}%
\pgfpathlineto{\pgfqpoint{10.161000in}{0.895109in}}%
\pgfpathlineto{\pgfqpoint{10.163013in}{0.891134in}}%
\pgfpathlineto{\pgfqpoint{10.165027in}{0.897020in}}%
\pgfpathlineto{\pgfqpoint{10.171067in}{0.893750in}}%
\pgfpathlineto{\pgfqpoint{10.175094in}{0.889172in}}%
\pgfpathlineto{\pgfqpoint{10.177107in}{0.898329in}}%
\pgfpathlineto{\pgfqpoint{10.179121in}{0.901196in}}%
\pgfpathlineto{\pgfqpoint{10.185161in}{0.904064in}}%
\pgfpathlineto{\pgfqpoint{10.187175in}{0.902253in}}%
\pgfpathlineto{\pgfqpoint{10.189188in}{0.901850in}}%
\pgfpathlineto{\pgfqpoint{10.191201in}{0.902152in}}%
\pgfpathlineto{\pgfqpoint{10.193215in}{0.905523in}}%
\pgfpathlineto{\pgfqpoint{10.199255in}{0.907636in}}%
\pgfpathlineto{\pgfqpoint{10.201269in}{0.914025in}}%
\pgfpathlineto{\pgfqpoint{10.203282in}{0.910001in}}%
\pgfpathlineto{\pgfqpoint{10.205296in}{0.914277in}}%
\pgfpathlineto{\pgfqpoint{10.207309in}{0.915233in}}%
\pgfpathlineto{\pgfqpoint{10.215363in}{0.913673in}}%
\pgfpathlineto{\pgfqpoint{10.217376in}{0.911611in}}%
\pgfpathlineto{\pgfqpoint{10.219390in}{0.912164in}}%
\pgfpathlineto{\pgfqpoint{10.221403in}{0.914076in}}%
\pgfpathlineto{\pgfqpoint{10.229457in}{0.913371in}}%
\pgfpathlineto{\pgfqpoint{10.231470in}{0.914025in}}%
\pgfpathlineto{\pgfqpoint{10.233484in}{0.905523in}}%
\pgfpathlineto{\pgfqpoint{10.235497in}{0.912114in}}%
\pgfpathlineto{\pgfqpoint{10.241538in}{0.911007in}}%
\pgfpathlineto{\pgfqpoint{10.243551in}{0.908391in}}%
\pgfpathlineto{\pgfqpoint{10.247578in}{0.918755in}}%
\pgfpathlineto{\pgfqpoint{10.249591in}{0.918654in}}%
\pgfpathlineto{\pgfqpoint{10.255632in}{0.916340in}}%
\pgfpathlineto{\pgfqpoint{10.257645in}{0.914579in}}%
\pgfpathlineto{\pgfqpoint{10.259659in}{0.915082in}}%
\pgfpathlineto{\pgfqpoint{10.261672in}{0.918352in}}%
\pgfpathlineto{\pgfqpoint{10.263686in}{0.919560in}}%
\pgfpathlineto{\pgfqpoint{10.269726in}{0.917447in}}%
\pgfpathlineto{\pgfqpoint{10.271739in}{0.924088in}}%
\pgfpathlineto{\pgfqpoint{10.273753in}{0.922176in}}%
\pgfpathlineto{\pgfqpoint{10.277780in}{0.921119in}}%
\pgfpathlineto{\pgfqpoint{10.283820in}{0.922830in}}%
\pgfpathlineto{\pgfqpoint{10.285834in}{0.918201in}}%
\pgfpathlineto{\pgfqpoint{10.287847in}{0.918503in}}%
\pgfpathlineto{\pgfqpoint{10.289860in}{0.919711in}}%
\pgfpathlineto{\pgfqpoint{10.291874in}{0.925345in}}%
\pgfpathlineto{\pgfqpoint{10.297914in}{0.923735in}}%
\pgfpathlineto{\pgfqpoint{10.299928in}{0.925848in}}%
\pgfpathlineto{\pgfqpoint{10.301941in}{0.921019in}}%
\pgfpathlineto{\pgfqpoint{10.305968in}{0.920868in}}%
\pgfpathlineto{\pgfqpoint{10.314022in}{0.925698in}}%
\pgfpathlineto{\pgfqpoint{10.316035in}{0.932842in}}%
\pgfpathlineto{\pgfqpoint{10.318049in}{0.936464in}}%
\pgfpathlineto{\pgfqpoint{10.320062in}{0.933798in}}%
\pgfpathlineto{\pgfqpoint{10.326102in}{0.937873in}}%
\pgfpathlineto{\pgfqpoint{10.328116in}{0.940539in}}%
\pgfpathlineto{\pgfqpoint{10.332143in}{0.944162in}}%
\pgfpathlineto{\pgfqpoint{10.334156in}{0.942904in}}%
\pgfpathlineto{\pgfqpoint{10.340197in}{0.951658in}}%
\pgfpathlineto{\pgfqpoint{10.342210in}{0.952412in}}%
\pgfpathlineto{\pgfqpoint{10.348250in}{0.951054in}}%
\pgfpathlineto{\pgfqpoint{10.354291in}{0.949746in}}%
\pgfpathlineto{\pgfqpoint{10.356304in}{0.952161in}}%
\pgfpathlineto{\pgfqpoint{10.358318in}{0.946929in}}%
\pgfpathlineto{\pgfqpoint{10.360331in}{0.945419in}}%
\pgfpathlineto{\pgfqpoint{10.362345in}{0.947734in}}%
\pgfpathlineto{\pgfqpoint{10.368385in}{0.937873in}}%
\pgfpathlineto{\pgfqpoint{10.370398in}{0.943357in}}%
\pgfpathlineto{\pgfqpoint{10.376439in}{0.941143in}}%
\pgfpathlineto{\pgfqpoint{10.384492in}{0.942199in}}%
\pgfpathlineto{\pgfqpoint{10.386506in}{0.947331in}}%
\pgfpathlineto{\pgfqpoint{10.388519in}{0.944916in}}%
\pgfpathlineto{\pgfqpoint{10.390533in}{0.935307in}}%
\pgfpathlineto{\pgfqpoint{10.396573in}{0.933093in}}%
\pgfpathlineto{\pgfqpoint{10.398587in}{0.936011in}}%
\pgfpathlineto{\pgfqpoint{10.400600in}{0.928817in}}%
\pgfpathlineto{\pgfqpoint{10.402613in}{0.937621in}}%
\pgfpathlineto{\pgfqpoint{10.404627in}{0.934753in}}%
\pgfpathlineto{\pgfqpoint{10.410667in}{0.922578in}}%
\pgfpathlineto{\pgfqpoint{10.412681in}{0.927509in}}%
\pgfpathlineto{\pgfqpoint{10.414694in}{0.930829in}}%
\pgfpathlineto{\pgfqpoint{10.416708in}{0.948539in}}%
\pgfpathlineto{\pgfqpoint{10.418721in}{0.949042in}}%
\pgfpathlineto{\pgfqpoint{10.424761in}{0.955481in}}%
\pgfpathlineto{\pgfqpoint{10.426775in}{0.958903in}}%
\pgfpathlineto{\pgfqpoint{10.428788in}{0.959456in}}%
\pgfpathlineto{\pgfqpoint{10.430802in}{0.959406in}}%
\pgfpathlineto{\pgfqpoint{10.432815in}{0.965594in}}%
\pgfpathlineto{\pgfqpoint{10.440869in}{0.967254in}}%
\pgfpathlineto{\pgfqpoint{10.442882in}{0.963934in}}%
\pgfpathlineto{\pgfqpoint{10.444896in}{0.965443in}}%
\pgfpathlineto{\pgfqpoint{10.446909in}{0.969820in}}%
\pgfpathlineto{\pgfqpoint{10.452950in}{0.972587in}}%
\pgfpathlineto{\pgfqpoint{10.454963in}{0.970172in}}%
\pgfpathlineto{\pgfqpoint{10.456977in}{0.969568in}}%
\pgfpathlineto{\pgfqpoint{10.461003in}{0.972990in}}%
\pgfpathlineto{\pgfqpoint{10.467044in}{0.966701in}}%
\pgfpathlineto{\pgfqpoint{10.469057in}{0.977165in}}%
\pgfpathlineto{\pgfqpoint{10.471071in}{0.981693in}}%
\pgfpathlineto{\pgfqpoint{10.473084in}{0.984762in}}%
\pgfpathlineto{\pgfqpoint{10.475098in}{0.979480in}}%
\pgfpathlineto{\pgfqpoint{10.481138in}{0.978373in}}%
\pgfpathlineto{\pgfqpoint{10.483151in}{0.974096in}}%
\pgfpathlineto{\pgfqpoint{10.485165in}{0.975606in}}%
\pgfpathlineto{\pgfqpoint{10.487178in}{0.968864in}}%
\pgfpathlineto{\pgfqpoint{10.489192in}{0.969518in}}%
\pgfpathlineto{\pgfqpoint{10.497245in}{0.978473in}}%
\pgfpathlineto{\pgfqpoint{10.499259in}{0.971631in}}%
\pgfpathlineto{\pgfqpoint{10.501272in}{0.972990in}}%
\pgfpathlineto{\pgfqpoint{10.503286in}{0.970826in}}%
\pgfpathlineto{\pgfqpoint{10.509326in}{0.966801in}}%
\pgfpathlineto{\pgfqpoint{10.511340in}{0.967103in}}%
\pgfpathlineto{\pgfqpoint{10.513353in}{0.962777in}}%
\pgfpathlineto{\pgfqpoint{10.515366in}{0.962122in}}%
\pgfpathlineto{\pgfqpoint{10.517380in}{0.964487in}}%
\pgfpathlineto{\pgfqpoint{10.523420in}{0.969518in}}%
\pgfpathlineto{\pgfqpoint{10.525434in}{0.976511in}}%
\pgfpathlineto{\pgfqpoint{10.529461in}{0.977065in}}%
\pgfpathlineto{\pgfqpoint{10.531474in}{0.971178in}}%
\pgfpathlineto{\pgfqpoint{10.537514in}{0.965141in}}%
\pgfpathlineto{\pgfqpoint{10.539528in}{0.967204in}}%
\pgfpathlineto{\pgfqpoint{10.541541in}{0.971380in}}%
\pgfpathlineto{\pgfqpoint{10.543555in}{0.957594in}}%
\pgfpathlineto{\pgfqpoint{10.545568in}{0.955280in}}%
\pgfpathlineto{\pgfqpoint{10.551609in}{0.958098in}}%
\pgfpathlineto{\pgfqpoint{10.553622in}{0.958299in}}%
\pgfpathlineto{\pgfqpoint{10.557649in}{0.969166in}}%
\pgfpathlineto{\pgfqpoint{10.565703in}{0.965040in}}%
\pgfpathlineto{\pgfqpoint{10.567716in}{0.966650in}}%
\pgfpathlineto{\pgfqpoint{10.571743in}{0.967304in}}%
\pgfpathlineto{\pgfqpoint{10.573756in}{0.960965in}}%
\pgfpathlineto{\pgfqpoint{10.579797in}{0.958852in}}%
\pgfpathlineto{\pgfqpoint{10.583824in}{0.966147in}}%
\pgfpathlineto{\pgfqpoint{10.585837in}{0.967154in}}%
\pgfpathlineto{\pgfqpoint{10.587851in}{0.970625in}}%
\pgfpathlineto{\pgfqpoint{10.593891in}{0.974901in}}%
\pgfpathlineto{\pgfqpoint{10.595904in}{0.973996in}}%
\pgfpathlineto{\pgfqpoint{10.597918in}{0.970524in}}%
\pgfpathlineto{\pgfqpoint{10.599931in}{0.976662in}}%
\pgfpathlineto{\pgfqpoint{10.601945in}{0.978272in}}%
\pgfpathlineto{\pgfqpoint{10.607985in}{0.980486in}}%
\pgfpathlineto{\pgfqpoint{10.609999in}{0.978876in}}%
\pgfpathlineto{\pgfqpoint{10.612012in}{0.973493in}}%
\pgfpathlineto{\pgfqpoint{10.614025in}{0.969921in}}%
\pgfpathlineto{\pgfqpoint{10.622079in}{0.974096in}}%
\pgfpathlineto{\pgfqpoint{10.624093in}{0.974298in}}%
\pgfpathlineto{\pgfqpoint{10.626106in}{0.979379in}}%
\pgfpathlineto{\pgfqpoint{10.628120in}{0.980134in}}%
\pgfpathlineto{\pgfqpoint{10.630133in}{0.984662in}}%
\pgfpathlineto{\pgfqpoint{10.638187in}{0.986624in}}%
\pgfpathlineto{\pgfqpoint{10.640200in}{0.985970in}}%
\pgfpathlineto{\pgfqpoint{10.642214in}{0.987982in}}%
\pgfpathlineto{\pgfqpoint{10.644227in}{0.987882in}}%
\pgfpathlineto{\pgfqpoint{10.650267in}{0.989089in}}%
\pgfpathlineto{\pgfqpoint{10.652281in}{0.987278in}}%
\pgfpathlineto{\pgfqpoint{10.654294in}{0.988938in}}%
\pgfpathlineto{\pgfqpoint{10.656308in}{0.991454in}}%
\pgfpathlineto{\pgfqpoint{10.658321in}{0.990649in}}%
\pgfpathlineto{\pgfqpoint{10.664362in}{0.994724in}}%
\pgfpathlineto{\pgfqpoint{10.666375in}{0.991303in}}%
\pgfpathlineto{\pgfqpoint{10.668388in}{0.989290in}}%
\pgfpathlineto{\pgfqpoint{10.670402in}{0.982247in}}%
\pgfpathlineto{\pgfqpoint{10.672415in}{0.982247in}}%
\pgfpathlineto{\pgfqpoint{10.678456in}{0.984561in}}%
\pgfpathlineto{\pgfqpoint{10.680469in}{0.983907in}}%
\pgfpathlineto{\pgfqpoint{10.682483in}{0.985869in}}%
\pgfpathlineto{\pgfqpoint{10.684496in}{0.986775in}}%
\pgfpathlineto{\pgfqpoint{10.686510in}{0.982347in}}%
\pgfpathlineto{\pgfqpoint{10.694563in}{0.981693in}}%
\pgfpathlineto{\pgfqpoint{10.696577in}{0.987378in}}%
\pgfpathlineto{\pgfqpoint{10.698590in}{0.989944in}}%
\pgfpathlineto{\pgfqpoint{10.700604in}{0.993869in}}%
\pgfpathlineto{\pgfqpoint{10.706644in}{0.995931in}}%
\pgfpathlineto{\pgfqpoint{10.708657in}{0.999352in}}%
\pgfpathlineto{\pgfqpoint{10.710671in}{0.999201in}}%
\pgfpathlineto{\pgfqpoint{10.712684in}{1.001063in}}%
\pgfpathlineto{\pgfqpoint{10.720738in}{0.999906in}}%
\pgfpathlineto{\pgfqpoint{10.722752in}{0.996485in}}%
\pgfpathlineto{\pgfqpoint{10.724765in}{1.002824in}}%
\pgfpathlineto{\pgfqpoint{10.726778in}{1.001164in}}%
\pgfpathlineto{\pgfqpoint{10.728792in}{1.001314in}}%
\pgfpathlineto{\pgfqpoint{10.734832in}{1.000711in}}%
\pgfpathlineto{\pgfqpoint{10.740873in}{0.992611in}}%
\pgfpathlineto{\pgfqpoint{10.742886in}{0.996283in}}%
\pgfpathlineto{\pgfqpoint{10.748926in}{0.995931in}}%
\pgfpathlineto{\pgfqpoint{10.750940in}{0.998145in}}%
\pgfpathlineto{\pgfqpoint{10.752953in}{0.997340in}}%
\pgfpathlineto{\pgfqpoint{10.754967in}{1.000862in}}%
\pgfpathlineto{\pgfqpoint{10.756980in}{0.998195in}}%
\pgfpathlineto{\pgfqpoint{10.763021in}{1.002522in}}%
\pgfpathlineto{\pgfqpoint{10.765034in}{0.998095in}}%
\pgfpathlineto{\pgfqpoint{10.767047in}{1.002774in}}%
\pgfpathlineto{\pgfqpoint{10.769061in}{0.996585in}}%
\pgfpathlineto{\pgfqpoint{10.771074in}{0.994271in}}%
\pgfpathlineto{\pgfqpoint{10.777115in}{1.002924in}}%
\pgfpathlineto{\pgfqpoint{10.779128in}{1.000660in}}%
\pgfpathlineto{\pgfqpoint{10.781142in}{0.999906in}}%
\pgfpathlineto{\pgfqpoint{10.783155in}{0.994875in}}%
\pgfpathlineto{\pgfqpoint{10.785168in}{1.001113in}}%
\pgfpathlineto{\pgfqpoint{10.791209in}{1.004132in}}%
\pgfpathlineto{\pgfqpoint{10.793222in}{1.002774in}}%
\pgfpathlineto{\pgfqpoint{10.795236in}{1.004585in}}%
\pgfpathlineto{\pgfqpoint{10.797249in}{1.009314in}}%
\pgfpathlineto{\pgfqpoint{10.799263in}{1.012433in}}%
\pgfpathlineto{\pgfqpoint{10.807316in}{1.016206in}}%
\pgfpathlineto{\pgfqpoint{10.809330in}{1.015049in}}%
\pgfpathlineto{\pgfqpoint{10.811343in}{1.017565in}}%
\pgfpathlineto{\pgfqpoint{10.813357in}{1.018068in}}%
\pgfpathlineto{\pgfqpoint{10.819397in}{1.017565in}}%
\pgfpathlineto{\pgfqpoint{10.821410in}{1.015854in}}%
\pgfpathlineto{\pgfqpoint{10.823424in}{1.017515in}}%
\pgfpathlineto{\pgfqpoint{10.825437in}{1.016860in}}%
\pgfpathlineto{\pgfqpoint{10.827451in}{1.015200in}}%
\pgfpathlineto{\pgfqpoint{10.837518in}{1.020131in}}%
\pgfpathlineto{\pgfqpoint{10.839532in}{1.016408in}}%
\pgfpathlineto{\pgfqpoint{10.841545in}{1.020131in}}%
\pgfpathlineto{\pgfqpoint{10.847585in}{1.018370in}}%
\pgfpathlineto{\pgfqpoint{10.849599in}{1.013892in}}%
\pgfpathlineto{\pgfqpoint{10.851612in}{1.013540in}}%
\pgfpathlineto{\pgfqpoint{10.853626in}{1.015653in}}%
\pgfpathlineto{\pgfqpoint{10.855639in}{1.014244in}}%
\pgfpathlineto{\pgfqpoint{10.865706in}{1.017313in}}%
\pgfpathlineto{\pgfqpoint{10.867720in}{1.017364in}}%
\pgfpathlineto{\pgfqpoint{10.869733in}{1.018068in}}%
\pgfpathlineto{\pgfqpoint{10.875774in}{1.012483in}}%
\pgfpathlineto{\pgfqpoint{10.877787in}{1.007905in}}%
\pgfpathlineto{\pgfqpoint{10.879800in}{1.013238in}}%
\pgfpathlineto{\pgfqpoint{10.881814in}{1.006798in}}%
\pgfpathlineto{\pgfqpoint{10.883827in}{1.009918in}}%
\pgfpathlineto{\pgfqpoint{10.889868in}{1.010320in}}%
\pgfpathlineto{\pgfqpoint{10.891881in}{1.011276in}}%
\pgfpathlineto{\pgfqpoint{10.893895in}{1.004082in}}%
\pgfpathlineto{\pgfqpoint{10.895908in}{1.000811in}}%
\pgfpathlineto{\pgfqpoint{10.897921in}{1.008559in}}%
\pgfpathlineto{\pgfqpoint{10.903962in}{1.009062in}}%
\pgfpathlineto{\pgfqpoint{10.905975in}{1.002321in}}%
\pgfpathlineto{\pgfqpoint{10.907989in}{1.006949in}}%
\pgfpathlineto{\pgfqpoint{10.910002in}{0.995780in}}%
\pgfpathlineto{\pgfqpoint{10.912016in}{0.998396in}}%
\pgfpathlineto{\pgfqpoint{10.918056in}{0.987429in}}%
\pgfpathlineto{\pgfqpoint{10.920069in}{0.988485in}}%
\pgfpathlineto{\pgfqpoint{10.922083in}{0.978876in}}%
\pgfpathlineto{\pgfqpoint{10.924096in}{0.977316in}}%
\pgfpathlineto{\pgfqpoint{10.926110in}{0.987026in}}%
\pgfpathlineto{\pgfqpoint{10.932150in}{0.994925in}}%
\pgfpathlineto{\pgfqpoint{10.934164in}{1.004333in}}%
\pgfpathlineto{\pgfqpoint{10.936177in}{1.002270in}}%
\pgfpathlineto{\pgfqpoint{10.940204in}{1.009314in}}%
\pgfpathlineto{\pgfqpoint{10.946244in}{1.008559in}}%
\pgfpathlineto{\pgfqpoint{10.948258in}{1.015452in}}%
\pgfpathlineto{\pgfqpoint{10.950271in}{1.013590in}}%
\pgfpathlineto{\pgfqpoint{10.952285in}{1.016810in}}%
\pgfpathlineto{\pgfqpoint{10.954298in}{1.022193in}}%
\pgfpathlineto{\pgfqpoint{10.960338in}{1.023753in}}%
\pgfpathlineto{\pgfqpoint{10.962352in}{1.017414in}}%
\pgfpathlineto{\pgfqpoint{10.964365in}{1.020433in}}%
\pgfpathlineto{\pgfqpoint{10.966379in}{1.025111in}}%
\pgfpathlineto{\pgfqpoint{10.968392in}{1.015754in}}%
\pgfpathlineto{\pgfqpoint{10.974432in}{1.014747in}}%
\pgfpathlineto{\pgfqpoint{10.976446in}{1.015703in}}%
\pgfpathlineto{\pgfqpoint{10.978459in}{1.015301in}}%
\pgfpathlineto{\pgfqpoint{10.980473in}{1.018068in}}%
\pgfpathlineto{\pgfqpoint{10.982486in}{1.019527in}}%
\pgfpathlineto{\pgfqpoint{10.990540in}{1.017062in}}%
\pgfpathlineto{\pgfqpoint{10.992553in}{1.014949in}}%
\pgfpathlineto{\pgfqpoint{10.994567in}{1.010672in}}%
\pgfpathlineto{\pgfqpoint{10.996580in}{1.010924in}}%
\pgfpathlineto{\pgfqpoint{11.002621in}{1.018470in}}%
\pgfpathlineto{\pgfqpoint{11.004634in}{1.023451in}}%
\pgfpathlineto{\pgfqpoint{11.010675in}{1.027476in}}%
\pgfpathlineto{\pgfqpoint{11.016715in}{1.028331in}}%
\pgfpathlineto{\pgfqpoint{11.018728in}{1.031954in}}%
\pgfpathlineto{\pgfqpoint{11.020742in}{1.030243in}}%
\pgfpathlineto{\pgfqpoint{11.022755in}{1.030847in}}%
\pgfpathlineto{\pgfqpoint{11.024769in}{1.033312in}}%
\pgfpathlineto{\pgfqpoint{11.030809in}{1.033463in}}%
\pgfpathlineto{\pgfqpoint{11.032822in}{1.029488in}}%
\pgfpathlineto{\pgfqpoint{11.034836in}{1.023351in}}%
\pgfpathlineto{\pgfqpoint{11.036849in}{1.029338in}}%
\pgfpathlineto{\pgfqpoint{11.038863in}{1.028130in}}%
\pgfpathlineto{\pgfqpoint{11.044903in}{1.025363in}}%
\pgfpathlineto{\pgfqpoint{11.046917in}{1.021892in}}%
\pgfpathlineto{\pgfqpoint{11.050943in}{1.033413in}}%
\pgfpathlineto{\pgfqpoint{11.052957in}{1.034721in}}%
\pgfpathlineto{\pgfqpoint{11.061011in}{1.043223in}}%
\pgfpathlineto{\pgfqpoint{11.063024in}{1.041966in}}%
\pgfpathlineto{\pgfqpoint{11.067051in}{1.044833in}}%
\pgfpathlineto{\pgfqpoint{11.073091in}{1.047047in}}%
\pgfpathlineto{\pgfqpoint{11.075105in}{1.043274in}}%
\pgfpathlineto{\pgfqpoint{11.077118in}{1.040859in}}%
\pgfpathlineto{\pgfqpoint{11.081145in}{1.038796in}}%
\pgfpathlineto{\pgfqpoint{11.089199in}{1.030042in}}%
\pgfpathlineto{\pgfqpoint{11.093226in}{1.038997in}}%
\pgfpathlineto{\pgfqpoint{11.095239in}{1.041161in}}%
\pgfpathlineto{\pgfqpoint{11.101280in}{1.042167in}}%
\pgfpathlineto{\pgfqpoint{11.103293in}{1.045588in}}%
\pgfpathlineto{\pgfqpoint{11.105307in}{1.041060in}}%
\pgfpathlineto{\pgfqpoint{11.107320in}{1.041613in}}%
\pgfpathlineto{\pgfqpoint{11.109333in}{1.045538in}}%
\pgfpathlineto{\pgfqpoint{11.117387in}{1.043475in}}%
\pgfpathlineto{\pgfqpoint{11.119401in}{1.040758in}}%
\pgfpathlineto{\pgfqpoint{11.121414in}{1.045387in}}%
\pgfpathlineto{\pgfqpoint{11.123428in}{1.043374in}}%
\pgfpathlineto{\pgfqpoint{11.129468in}{1.044531in}}%
\pgfpathlineto{\pgfqpoint{11.131481in}{1.039852in}}%
\pgfpathlineto{\pgfqpoint{11.133495in}{1.033715in}}%
\pgfpathlineto{\pgfqpoint{11.135508in}{1.036280in}}%
\pgfpathlineto{\pgfqpoint{11.137522in}{1.025615in}}%
\pgfpathlineto{\pgfqpoint{11.143562in}{1.030193in}}%
\pgfpathlineto{\pgfqpoint{11.145575in}{1.040456in}}%
\pgfpathlineto{\pgfqpoint{11.147589in}{1.074315in}}%
\pgfpathlineto{\pgfqpoint{11.149602in}{1.080705in}}%
\pgfpathlineto{\pgfqpoint{11.151616in}{1.077787in}}%
\pgfpathlineto{\pgfqpoint{11.157656in}{1.076428in}}%
\pgfpathlineto{\pgfqpoint{11.159670in}{1.077334in}}%
\pgfpathlineto{\pgfqpoint{11.161683in}{1.077082in}}%
\pgfpathlineto{\pgfqpoint{11.163697in}{1.085182in}}%
\pgfpathlineto{\pgfqpoint{11.165710in}{1.087950in}}%
\pgfpathlineto{\pgfqpoint{11.173764in}{1.087698in}}%
\pgfpathlineto{\pgfqpoint{11.175777in}{1.086490in}}%
\pgfpathlineto{\pgfqpoint{11.177791in}{1.086641in}}%
\pgfpathlineto{\pgfqpoint{11.179804in}{1.089761in}}%
\pgfpathlineto{\pgfqpoint{11.185844in}{1.091823in}}%
\pgfpathlineto{\pgfqpoint{11.187858in}{1.090314in}}%
\pgfpathlineto{\pgfqpoint{11.189871in}{1.094540in}}%
\pgfpathlineto{\pgfqpoint{11.191885in}{1.089811in}}%
\pgfpathlineto{\pgfqpoint{11.193898in}{1.087547in}}%
\pgfpathlineto{\pgfqpoint{11.201952in}{1.098213in}}%
\pgfpathlineto{\pgfqpoint{11.203965in}{1.094540in}}%
\pgfpathlineto{\pgfqpoint{11.205979in}{1.092025in}}%
\pgfpathlineto{\pgfqpoint{11.207992in}{1.086289in}}%
\pgfpathlineto{\pgfqpoint{11.214033in}{1.093081in}}%
\pgfpathlineto{\pgfqpoint{11.216046in}{1.082868in}}%
\pgfpathlineto{\pgfqpoint{11.218060in}{1.081912in}}%
\pgfpathlineto{\pgfqpoint{11.220073in}{1.102087in}}%
\pgfpathlineto{\pgfqpoint{11.222086in}{1.098666in}}%
\pgfpathlineto{\pgfqpoint{11.228127in}{1.103043in}}%
\pgfpathlineto{\pgfqpoint{11.230140in}{1.101131in}}%
\pgfpathlineto{\pgfqpoint{11.232154in}{1.105860in}}%
\pgfpathlineto{\pgfqpoint{11.234167in}{1.103043in}}%
\pgfpathlineto{\pgfqpoint{11.236181in}{1.108023in}}%
\pgfpathlineto{\pgfqpoint{11.242221in}{1.107068in}}%
\pgfpathlineto{\pgfqpoint{11.244234in}{1.101835in}}%
\pgfpathlineto{\pgfqpoint{11.246248in}{1.091874in}}%
\pgfpathlineto{\pgfqpoint{11.256315in}{1.097156in}}%
\pgfpathlineto{\pgfqpoint{11.258329in}{1.091320in}}%
\pgfpathlineto{\pgfqpoint{11.262355in}{1.096603in}}%
\pgfpathlineto{\pgfqpoint{11.272423in}{1.093886in}}%
\pgfpathlineto{\pgfqpoint{11.274436in}{1.098263in}}%
\pgfpathlineto{\pgfqpoint{11.276450in}{1.100225in}}%
\pgfpathlineto{\pgfqpoint{11.278463in}{1.101081in}}%
\pgfpathlineto{\pgfqpoint{11.284503in}{1.098968in}}%
\pgfpathlineto{\pgfqpoint{11.286517in}{1.099672in}}%
\pgfpathlineto{\pgfqpoint{11.288530in}{1.101181in}}%
\pgfpathlineto{\pgfqpoint{11.290544in}{1.106464in}}%
\pgfpathlineto{\pgfqpoint{11.292557in}{1.099823in}}%
\pgfpathlineto{\pgfqpoint{11.298597in}{1.107068in}}%
\pgfpathlineto{\pgfqpoint{11.300611in}{1.104502in}}%
\pgfpathlineto{\pgfqpoint{11.302624in}{1.105709in}}%
\pgfpathlineto{\pgfqpoint{11.304638in}{1.110740in}}%
\pgfpathlineto{\pgfqpoint{11.306651in}{1.113256in}}%
\pgfpathlineto{\pgfqpoint{11.312692in}{1.116224in}}%
\pgfpathlineto{\pgfqpoint{11.314705in}{1.115067in}}%
\pgfpathlineto{\pgfqpoint{11.316718in}{1.114564in}}%
\pgfpathlineto{\pgfqpoint{11.318732in}{1.109432in}}%
\pgfpathlineto{\pgfqpoint{11.320745in}{1.117884in}}%
\pgfpathlineto{\pgfqpoint{11.326786in}{1.120299in}}%
\pgfpathlineto{\pgfqpoint{11.328799in}{1.119293in}}%
\pgfpathlineto{\pgfqpoint{11.330813in}{1.114111in}}%
\pgfpathlineto{\pgfqpoint{11.332826in}{1.111948in}}%
\pgfpathlineto{\pgfqpoint{11.334840in}{1.115973in}}%
\pgfpathlineto{\pgfqpoint{11.340880in}{1.108828in}}%
\pgfpathlineto{\pgfqpoint{11.342893in}{1.111847in}}%
\pgfpathlineto{\pgfqpoint{11.344907in}{1.111646in}}%
\pgfpathlineto{\pgfqpoint{11.346920in}{1.115117in}}%
\pgfpathlineto{\pgfqpoint{11.348934in}{1.116878in}}%
\pgfpathlineto{\pgfqpoint{11.354974in}{1.117029in}}%
\pgfpathlineto{\pgfqpoint{11.356987in}{1.118086in}}%
\pgfpathlineto{\pgfqpoint{11.359001in}{1.116375in}}%
\pgfpathlineto{\pgfqpoint{11.361014in}{1.117281in}}%
\pgfpathlineto{\pgfqpoint{11.363028in}{1.116677in}}%
\pgfpathlineto{\pgfqpoint{11.371082in}{1.112803in}}%
\pgfpathlineto{\pgfqpoint{11.373095in}{1.117180in}}%
\pgfpathlineto{\pgfqpoint{11.375108in}{1.117935in}}%
\pgfpathlineto{\pgfqpoint{11.377122in}{1.117180in}}%
\pgfpathlineto{\pgfqpoint{11.383162in}{1.119997in}}%
\pgfpathlineto{\pgfqpoint{11.385176in}{1.118991in}}%
\pgfpathlineto{\pgfqpoint{11.387189in}{1.120953in}}%
\pgfpathlineto{\pgfqpoint{11.389203in}{1.116878in}}%
\pgfpathlineto{\pgfqpoint{11.391216in}{1.116878in}}%
\pgfpathlineto{\pgfqpoint{11.397256in}{1.112099in}}%
\pgfpathlineto{\pgfqpoint{11.399270in}{1.108476in}}%
\pgfpathlineto{\pgfqpoint{11.401283in}{1.115469in}}%
\pgfpathlineto{\pgfqpoint{11.403297in}{1.118387in}}%
\pgfpathlineto{\pgfqpoint{11.405310in}{1.115218in}}%
\pgfpathlineto{\pgfqpoint{11.411351in}{1.116274in}}%
\pgfpathlineto{\pgfqpoint{11.413364in}{1.120450in}}%
\pgfpathlineto{\pgfqpoint{11.415377in}{1.122463in}}%
\pgfpathlineto{\pgfqpoint{11.417391in}{1.130613in}}%
\pgfpathlineto{\pgfqpoint{11.419404in}{1.127796in}}%
\pgfpathlineto{\pgfqpoint{11.425445in}{1.132122in}}%
\pgfpathlineto{\pgfqpoint{11.427458in}{1.136248in}}%
\pgfpathlineto{\pgfqpoint{11.429472in}{1.133229in}}%
\pgfpathlineto{\pgfqpoint{11.433498in}{1.138965in}}%
\pgfpathlineto{\pgfqpoint{11.439539in}{1.129858in}}%
\pgfpathlineto{\pgfqpoint{11.441552in}{1.134990in}}%
\pgfpathlineto{\pgfqpoint{11.443566in}{1.142788in}}%
\pgfpathlineto{\pgfqpoint{11.445579in}{1.142034in}}%
\pgfpathlineto{\pgfqpoint{11.453633in}{1.145505in}}%
\pgfpathlineto{\pgfqpoint{11.455646in}{1.152146in}}%
\pgfpathlineto{\pgfqpoint{11.457660in}{1.143090in}}%
\pgfpathlineto{\pgfqpoint{11.459673in}{1.145052in}}%
\pgfpathlineto{\pgfqpoint{11.461687in}{1.149027in}}%
\pgfpathlineto{\pgfqpoint{11.467727in}{1.156674in}}%
\pgfpathlineto{\pgfqpoint{11.469740in}{1.155718in}}%
\pgfpathlineto{\pgfqpoint{11.471754in}{1.157831in}}%
\pgfpathlineto{\pgfqpoint{11.473767in}{1.161504in}}%
\pgfpathlineto{\pgfqpoint{11.475781in}{1.160498in}}%
\pgfpathlineto{\pgfqpoint{11.481821in}{1.163919in}}%
\pgfpathlineto{\pgfqpoint{11.483835in}{1.162661in}}%
\pgfpathlineto{\pgfqpoint{11.485848in}{1.162711in}}%
\pgfpathlineto{\pgfqpoint{11.487862in}{1.160196in}}%
\pgfpathlineto{\pgfqpoint{11.489875in}{1.160749in}}%
\pgfpathlineto{\pgfqpoint{11.495915in}{1.157630in}}%
\pgfpathlineto{\pgfqpoint{11.497929in}{1.158586in}}%
\pgfpathlineto{\pgfqpoint{11.499942in}{1.165176in}}%
\pgfpathlineto{\pgfqpoint{11.501956in}{1.166032in}}%
\pgfpathlineto{\pgfqpoint{11.503969in}{1.165931in}}%
\pgfpathlineto{\pgfqpoint{11.510009in}{1.171214in}}%
\pgfpathlineto{\pgfqpoint{11.512023in}{1.173930in}}%
\pgfpathlineto{\pgfqpoint{11.514036in}{1.121004in}}%
\pgfpathlineto{\pgfqpoint{11.516050in}{1.111596in}}%
\pgfpathlineto{\pgfqpoint{11.518063in}{1.115369in}}%
\pgfpathlineto{\pgfqpoint{11.524104in}{1.123217in}}%
\pgfpathlineto{\pgfqpoint{11.526117in}{1.108979in}}%
\pgfpathlineto{\pgfqpoint{11.528130in}{1.104200in}}%
\pgfpathlineto{\pgfqpoint{11.530144in}{1.106715in}}%
\pgfpathlineto{\pgfqpoint{11.532157in}{1.105005in}}%
\pgfpathlineto{\pgfqpoint{11.538198in}{1.113960in}}%
\pgfpathlineto{\pgfqpoint{11.540211in}{1.103948in}}%
\pgfpathlineto{\pgfqpoint{11.542225in}{1.101634in}}%
\pgfpathlineto{\pgfqpoint{11.544238in}{1.071095in}}%
\pgfpathlineto{\pgfqpoint{11.552292in}{1.049009in}}%
\pgfpathlineto{\pgfqpoint{11.554305in}{1.051525in}}%
\pgfpathlineto{\pgfqpoint{11.558332in}{1.081308in}}%
\pgfpathlineto{\pgfqpoint{11.560346in}{1.082768in}}%
\pgfpathlineto{\pgfqpoint{11.566386in}{1.079950in}}%
\pgfpathlineto{\pgfqpoint{11.568399in}{1.068681in}}%
\pgfpathlineto{\pgfqpoint{11.570413in}{1.080000in}}%
\pgfpathlineto{\pgfqpoint{11.572426in}{1.080453in}}%
\pgfpathlineto{\pgfqpoint{11.574440in}{1.075623in}}%
\pgfpathlineto{\pgfqpoint{11.582494in}{1.090063in}}%
\pgfpathlineto{\pgfqpoint{11.584507in}{1.080101in}}%
\pgfpathlineto{\pgfqpoint{11.586520in}{1.083371in}}%
\pgfpathlineto{\pgfqpoint{11.588534in}{1.092276in}}%
\pgfpathlineto{\pgfqpoint{11.594574in}{1.089157in}}%
\pgfpathlineto{\pgfqpoint{11.596588in}{1.087295in}}%
\pgfpathlineto{\pgfqpoint{11.598601in}{1.089811in}}%
\pgfpathlineto{\pgfqpoint{11.600615in}{1.090968in}}%
\pgfpathlineto{\pgfqpoint{11.602628in}{1.084478in}}%
\pgfpathlineto{\pgfqpoint{11.608668in}{1.087195in}}%
\pgfpathlineto{\pgfqpoint{11.614709in}{1.073963in}}%
\pgfpathlineto{\pgfqpoint{11.616722in}{1.072454in}}%
\pgfpathlineto{\pgfqpoint{11.622762in}{1.063851in}}%
\pgfpathlineto{\pgfqpoint{11.624776in}{1.068278in}}%
\pgfpathlineto{\pgfqpoint{11.626789in}{1.081459in}}%
\pgfpathlineto{\pgfqpoint{11.630816in}{1.085233in}}%
\pgfpathlineto{\pgfqpoint{11.636857in}{1.089308in}}%
\pgfpathlineto{\pgfqpoint{11.638870in}{1.088905in}}%
\pgfpathlineto{\pgfqpoint{11.640884in}{1.087094in}}%
\pgfpathlineto{\pgfqpoint{11.644910in}{1.097408in}}%
\pgfpathlineto{\pgfqpoint{11.652964in}{1.102288in}}%
\pgfpathlineto{\pgfqpoint{11.654978in}{1.098213in}}%
\pgfpathlineto{\pgfqpoint{11.656991in}{1.108476in}}%
\pgfpathlineto{\pgfqpoint{11.659005in}{1.110137in}}%
\pgfpathlineto{\pgfqpoint{11.667058in}{1.117683in}}%
\pgfpathlineto{\pgfqpoint{11.669072in}{1.118891in}}%
\pgfpathlineto{\pgfqpoint{11.671085in}{1.133883in}}%
\pgfpathlineto{\pgfqpoint{11.673099in}{1.133129in}}%
\pgfpathlineto{\pgfqpoint{11.681152in}{1.136348in}}%
\pgfpathlineto{\pgfqpoint{11.685179in}{1.142386in}}%
\pgfpathlineto{\pgfqpoint{11.687193in}{1.136197in}}%
\pgfpathlineto{\pgfqpoint{11.695247in}{1.144750in}}%
\pgfpathlineto{\pgfqpoint{11.697260in}{1.133883in}}%
\pgfpathlineto{\pgfqpoint{11.699273in}{1.132676in}}%
\pgfpathlineto{\pgfqpoint{11.701287in}{1.145354in}}%
\pgfpathlineto{\pgfqpoint{11.707327in}{1.148926in}}%
\pgfpathlineto{\pgfqpoint{11.709341in}{1.153655in}}%
\pgfpathlineto{\pgfqpoint{11.711354in}{1.149379in}}%
\pgfpathlineto{\pgfqpoint{11.713368in}{1.147920in}}%
\pgfpathlineto{\pgfqpoint{11.715381in}{1.141430in}}%
\pgfpathlineto{\pgfqpoint{11.723435in}{1.147568in}}%
\pgfpathlineto{\pgfqpoint{11.725448in}{1.157076in}}%
\pgfpathlineto{\pgfqpoint{11.727462in}{1.159793in}}%
\pgfpathlineto{\pgfqpoint{11.729475in}{1.166233in}}%
\pgfpathlineto{\pgfqpoint{11.735516in}{1.163164in}}%
\pgfpathlineto{\pgfqpoint{11.737529in}{1.156171in}}%
\pgfpathlineto{\pgfqpoint{11.739542in}{1.159592in}}%
\pgfpathlineto{\pgfqpoint{11.743569in}{1.142788in}}%
\pgfpathlineto{\pgfqpoint{11.749610in}{1.134940in}}%
\pgfpathlineto{\pgfqpoint{11.751623in}{1.144046in}}%
\pgfpathlineto{\pgfqpoint{11.753637in}{1.137455in}}%
\pgfpathlineto{\pgfqpoint{11.755650in}{1.127443in}}%
\pgfpathlineto{\pgfqpoint{11.757663in}{1.138562in}}%
\pgfpathlineto{\pgfqpoint{11.763704in}{1.136650in}}%
\pgfpathlineto{\pgfqpoint{11.767731in}{1.125431in}}%
\pgfpathlineto{\pgfqpoint{11.769744in}{1.125431in}}%
\pgfpathlineto{\pgfqpoint{11.771758in}{1.112451in}}%
\pgfpathlineto{\pgfqpoint{11.777798in}{1.118689in}}%
\pgfpathlineto{\pgfqpoint{11.779811in}{1.132122in}}%
\pgfpathlineto{\pgfqpoint{11.781825in}{1.139920in}}%
\pgfpathlineto{\pgfqpoint{11.783838in}{1.131418in}}%
\pgfpathlineto{\pgfqpoint{11.785852in}{1.110941in}}%
\pgfpathlineto{\pgfqpoint{11.791892in}{1.105508in}}%
\pgfpathlineto{\pgfqpoint{11.793905in}{1.106263in}}%
\pgfpathlineto{\pgfqpoint{11.795919in}{1.100628in}}%
\pgfpathlineto{\pgfqpoint{11.797932in}{1.102037in}}%
\pgfpathlineto{\pgfqpoint{11.805986in}{1.108678in}}%
\pgfpathlineto{\pgfqpoint{11.808000in}{1.107873in}}%
\pgfpathlineto{\pgfqpoint{11.810013in}{1.104351in}}%
\pgfpathlineto{\pgfqpoint{11.812027in}{1.098314in}}%
\pgfpathlineto{\pgfqpoint{11.820080in}{1.088302in}}%
\pgfpathlineto{\pgfqpoint{11.822094in}{1.078340in}}%
\pgfpathlineto{\pgfqpoint{11.824107in}{1.075774in}}%
\pgfpathlineto{\pgfqpoint{11.826121in}{1.071699in}}%
\pgfpathlineto{\pgfqpoint{11.828134in}{1.070492in}}%
\pgfpathlineto{\pgfqpoint{11.834174in}{1.073661in}}%
\pgfpathlineto{\pgfqpoint{11.836188in}{1.081057in}}%
\pgfpathlineto{\pgfqpoint{11.838201in}{1.066819in}}%
\pgfpathlineto{\pgfqpoint{11.840215in}{1.069787in}}%
\pgfpathlineto{\pgfqpoint{11.842228in}{1.044934in}}%
\pgfpathlineto{\pgfqpoint{11.850282in}{1.045286in}}%
\pgfpathlineto{\pgfqpoint{11.852295in}{1.038444in}}%
\pgfpathlineto{\pgfqpoint{11.854309in}{1.045487in}}%
\pgfpathlineto{\pgfqpoint{11.856322in}{1.059272in}}%
\pgfpathlineto{\pgfqpoint{11.862363in}{1.051575in}}%
\pgfpathlineto{\pgfqpoint{11.864376in}{1.056254in}}%
\pgfpathlineto{\pgfqpoint{11.866390in}{1.046946in}}%
\pgfpathlineto{\pgfqpoint{11.868403in}{1.043173in}}%
\pgfpathlineto{\pgfqpoint{11.870416in}{1.054090in}}%
\pgfpathlineto{\pgfqpoint{11.876457in}{1.050921in}}%
\pgfpathlineto{\pgfqpoint{11.878470in}{1.041211in}}%
\pgfpathlineto{\pgfqpoint{11.880484in}{1.050871in}}%
\pgfpathlineto{\pgfqpoint{11.882497in}{1.052229in}}%
\pgfpathlineto{\pgfqpoint{11.884511in}{1.044934in}}%
\pgfpathlineto{\pgfqpoint{11.890551in}{1.036431in}}%
\pgfpathlineto{\pgfqpoint{11.892564in}{1.037387in}}%
\pgfpathlineto{\pgfqpoint{11.894578in}{1.020835in}}%
\pgfpathlineto{\pgfqpoint{11.896591in}{1.027778in}}%
\pgfpathlineto{\pgfqpoint{11.898605in}{1.031803in}}%
\pgfpathlineto{\pgfqpoint{11.906659in}{1.040205in}}%
\pgfpathlineto{\pgfqpoint{11.908672in}{1.052581in}}%
\pgfpathlineto{\pgfqpoint{11.910685in}{1.051021in}}%
\pgfpathlineto{\pgfqpoint{11.912699in}{1.050216in}}%
\pgfpathlineto{\pgfqpoint{11.918739in}{1.056707in}}%
\pgfpathlineto{\pgfqpoint{11.920753in}{1.051977in}}%
\pgfpathlineto{\pgfqpoint{11.922766in}{1.052229in}}%
\pgfpathlineto{\pgfqpoint{11.924780in}{1.053285in}}%
\pgfpathlineto{\pgfqpoint{11.926793in}{1.051675in}}%
\pgfpathlineto{\pgfqpoint{11.932833in}{1.052682in}}%
\pgfpathlineto{\pgfqpoint{11.934847in}{1.062844in}}%
\pgfpathlineto{\pgfqpoint{11.936860in}{1.059725in}}%
\pgfpathlineto{\pgfqpoint{11.938874in}{1.068429in}}%
\pgfpathlineto{\pgfqpoint{11.940887in}{1.066819in}}%
\pgfpathlineto{\pgfqpoint{11.946927in}{1.071146in}}%
\pgfpathlineto{\pgfqpoint{11.948941in}{1.063649in}}%
\pgfpathlineto{\pgfqpoint{11.950954in}{1.062895in}}%
\pgfpathlineto{\pgfqpoint{11.952968in}{1.059926in}}%
\pgfpathlineto{\pgfqpoint{11.954981in}{1.064203in}}%
\pgfpathlineto{\pgfqpoint{11.961022in}{1.068379in}}%
\pgfpathlineto{\pgfqpoint{11.963035in}{1.065662in}}%
\pgfpathlineto{\pgfqpoint{11.965049in}{1.066618in}}%
\pgfpathlineto{\pgfqpoint{11.967062in}{1.072152in}}%
\pgfpathlineto{\pgfqpoint{11.969075in}{1.070240in}}%
\pgfpathlineto{\pgfqpoint{11.975116in}{1.066718in}}%
\pgfpathlineto{\pgfqpoint{11.979143in}{1.058920in}}%
\pgfpathlineto{\pgfqpoint{11.981156in}{1.060782in}}%
\pgfpathlineto{\pgfqpoint{11.989210in}{1.064958in}}%
\pgfpathlineto{\pgfqpoint{11.991223in}{1.065259in}}%
\pgfpathlineto{\pgfqpoint{11.993237in}{1.068882in}}%
\pgfpathlineto{\pgfqpoint{11.995250in}{1.070743in}}%
\pgfpathlineto{\pgfqpoint{11.997264in}{1.069636in}}%
\pgfpathlineto{\pgfqpoint{12.003304in}{1.067775in}}%
\pgfpathlineto{\pgfqpoint{12.005317in}{1.059725in}}%
\pgfpathlineto{\pgfqpoint{12.007331in}{1.062039in}}%
\pgfpathlineto{\pgfqpoint{12.009344in}{1.055751in}}%
\pgfpathlineto{\pgfqpoint{12.011358in}{1.056958in}}%
\pgfpathlineto{\pgfqpoint{12.017398in}{1.056254in}}%
\pgfpathlineto{\pgfqpoint{12.019412in}{1.061385in}}%
\pgfpathlineto{\pgfqpoint{12.021425in}{1.071599in}}%
\pgfpathlineto{\pgfqpoint{12.023438in}{1.067523in}}%
\pgfpathlineto{\pgfqpoint{12.025452in}{1.067322in}}%
\pgfpathlineto{\pgfqpoint{12.035519in}{1.089660in}}%
\pgfpathlineto{\pgfqpoint{12.037533in}{1.087950in}}%
\pgfpathlineto{\pgfqpoint{12.039546in}{1.092075in}}%
\pgfpathlineto{\pgfqpoint{12.047600in}{1.097408in}}%
\pgfpathlineto{\pgfqpoint{12.049613in}{1.099269in}}%
\pgfpathlineto{\pgfqpoint{12.051627in}{1.093282in}}%
\pgfpathlineto{\pgfqpoint{12.053640in}{1.089610in}}%
\pgfpathlineto{\pgfqpoint{12.059681in}{1.094892in}}%
\pgfpathlineto{\pgfqpoint{12.061694in}{1.092125in}}%
\pgfpathlineto{\pgfqpoint{12.063707in}{1.091572in}}%
\pgfpathlineto{\pgfqpoint{12.065721in}{1.097609in}}%
\pgfpathlineto{\pgfqpoint{12.067734in}{1.100527in}}%
\pgfpathlineto{\pgfqpoint{12.073775in}{1.099571in}}%
\pgfpathlineto{\pgfqpoint{12.075788in}{1.105558in}}%
\pgfpathlineto{\pgfqpoint{12.077802in}{1.084981in}}%
\pgfpathlineto{\pgfqpoint{12.079815in}{1.082214in}}%
\pgfpathlineto{\pgfqpoint{12.081828in}{1.076529in}}%
\pgfpathlineto{\pgfqpoint{12.087869in}{1.075774in}}%
\pgfpathlineto{\pgfqpoint{12.089882in}{1.073762in}}%
\pgfpathlineto{\pgfqpoint{12.091896in}{1.069284in}}%
\pgfpathlineto{\pgfqpoint{12.093909in}{1.066467in}}%
\pgfpathlineto{\pgfqpoint{12.095923in}{1.073007in}}%
\pgfpathlineto{\pgfqpoint{12.101963in}{1.070140in}}%
\pgfpathlineto{\pgfqpoint{12.105990in}{1.073410in}}%
\pgfpathlineto{\pgfqpoint{12.108003in}{1.073158in}}%
\pgfpathlineto{\pgfqpoint{12.110017in}{1.075422in}}%
\pgfpathlineto{\pgfqpoint{12.118070in}{1.070341in}}%
\pgfpathlineto{\pgfqpoint{12.120084in}{1.066970in}}%
\pgfpathlineto{\pgfqpoint{12.122097in}{1.067926in}}%
\pgfpathlineto{\pgfqpoint{12.130151in}{1.068228in}}%
\pgfpathlineto{\pgfqpoint{12.134178in}{1.064706in}}%
\pgfpathlineto{\pgfqpoint{12.136192in}{1.063750in}}%
\pgfpathlineto{\pgfqpoint{12.138205in}{1.061335in}}%
\pgfpathlineto{\pgfqpoint{12.144245in}{1.062442in}}%
\pgfpathlineto{\pgfqpoint{12.146259in}{1.066417in}}%
\pgfpathlineto{\pgfqpoint{12.148272in}{1.065813in}}%
\pgfpathlineto{\pgfqpoint{12.150286in}{1.066316in}}%
\pgfpathlineto{\pgfqpoint{12.152299in}{1.069284in}}%
\pgfpathlineto{\pgfqpoint{12.158339in}{1.072001in}}%
\pgfpathlineto{\pgfqpoint{12.160353in}{1.068429in}}%
\pgfpathlineto{\pgfqpoint{12.162366in}{1.068278in}}%
\pgfpathlineto{\pgfqpoint{12.164380in}{1.069385in}}%
\pgfpathlineto{\pgfqpoint{12.166393in}{1.053638in}}%
\pgfpathlineto{\pgfqpoint{12.172434in}{1.047248in}}%
\pgfpathlineto{\pgfqpoint{12.174447in}{1.055197in}}%
\pgfpathlineto{\pgfqpoint{12.178474in}{1.063649in}}%
\pgfpathlineto{\pgfqpoint{12.180487in}{1.064656in}}%
\pgfpathlineto{\pgfqpoint{12.188541in}{1.062895in}}%
\pgfpathlineto{\pgfqpoint{12.192568in}{1.069888in}}%
\pgfpathlineto{\pgfqpoint{12.194581in}{1.075724in}}%
\pgfpathlineto{\pgfqpoint{12.202635in}{1.078491in}}%
\pgfpathlineto{\pgfqpoint{12.204649in}{1.076931in}}%
\pgfpathlineto{\pgfqpoint{12.206662in}{1.077384in}}%
\pgfpathlineto{\pgfqpoint{12.208676in}{1.076579in}}%
\pgfpathlineto{\pgfqpoint{12.214716in}{1.078240in}}%
\pgfpathlineto{\pgfqpoint{12.216729in}{1.074969in}}%
\pgfpathlineto{\pgfqpoint{12.218743in}{1.068982in}}%
\pgfpathlineto{\pgfqpoint{12.222770in}{1.066517in}}%
\pgfpathlineto{\pgfqpoint{12.228810in}{1.064958in}}%
\pgfpathlineto{\pgfqpoint{12.230824in}{1.061587in}}%
\pgfpathlineto{\pgfqpoint{12.234850in}{1.057864in}}%
\pgfpathlineto{\pgfqpoint{12.236864in}{1.058065in}}%
\pgfpathlineto{\pgfqpoint{12.242904in}{1.056103in}}%
\pgfpathlineto{\pgfqpoint{12.244918in}{1.053537in}}%
\pgfpathlineto{\pgfqpoint{12.246931in}{1.058719in}}%
\pgfpathlineto{\pgfqpoint{12.248945in}{1.054241in}}%
\pgfpathlineto{\pgfqpoint{12.250958in}{1.057461in}}%
\pgfpathlineto{\pgfqpoint{12.256998in}{1.057109in}}%
\pgfpathlineto{\pgfqpoint{12.261025in}{1.067222in}}%
\pgfpathlineto{\pgfqpoint{12.263039in}{1.066819in}}%
\pgfpathlineto{\pgfqpoint{12.265052in}{1.062341in}}%
\pgfpathlineto{\pgfqpoint{12.271092in}{1.063599in}}%
\pgfpathlineto{\pgfqpoint{12.273106in}{1.062543in}}%
\pgfpathlineto{\pgfqpoint{12.275119in}{1.062492in}}%
\pgfpathlineto{\pgfqpoint{12.285187in}{1.057662in}}%
\pgfpathlineto{\pgfqpoint{12.287200in}{1.058166in}}%
\pgfpathlineto{\pgfqpoint{12.291227in}{1.056103in}}%
\pgfpathlineto{\pgfqpoint{12.293240in}{1.054493in}}%
\pgfpathlineto{\pgfqpoint{12.301294in}{1.052833in}}%
\pgfpathlineto{\pgfqpoint{12.303308in}{1.050871in}}%
\pgfpathlineto{\pgfqpoint{12.305321in}{1.049915in}}%
\pgfpathlineto{\pgfqpoint{12.307335in}{1.050669in}}%
\pgfpathlineto{\pgfqpoint{12.317402in}{1.047298in}}%
\pgfpathlineto{\pgfqpoint{12.319415in}{1.049110in}}%
\pgfpathlineto{\pgfqpoint{12.321429in}{1.041060in}}%
\pgfpathlineto{\pgfqpoint{12.327469in}{1.046946in}}%
\pgfpathlineto{\pgfqpoint{12.329482in}{1.042418in}}%
\pgfpathlineto{\pgfqpoint{12.331496in}{1.040305in}}%
\pgfpathlineto{\pgfqpoint{12.333509in}{1.041462in}}%
\pgfpathlineto{\pgfqpoint{12.335523in}{1.041764in}}%
\pgfpathlineto{\pgfqpoint{12.341563in}{1.042066in}}%
\pgfpathlineto{\pgfqpoint{12.343577in}{1.043626in}}%
\pgfpathlineto{\pgfqpoint{12.345590in}{1.040909in}}%
\pgfpathlineto{\pgfqpoint{12.347603in}{1.045839in}}%
\pgfpathlineto{\pgfqpoint{12.349617in}{1.045185in}}%
\pgfpathlineto{\pgfqpoint{12.355657in}{1.038846in}}%
\pgfpathlineto{\pgfqpoint{12.357671in}{1.037689in}}%
\pgfpathlineto{\pgfqpoint{12.359684in}{1.040003in}}%
\pgfpathlineto{\pgfqpoint{12.361698in}{1.038092in}}%
\pgfpathlineto{\pgfqpoint{12.363711in}{1.043173in}}%
\pgfpathlineto{\pgfqpoint{12.369751in}{1.041412in}}%
\pgfpathlineto{\pgfqpoint{12.371765in}{1.041915in}}%
\pgfpathlineto{\pgfqpoint{12.373778in}{1.041211in}}%
\pgfpathlineto{\pgfqpoint{12.375792in}{1.043022in}}%
\pgfpathlineto{\pgfqpoint{12.377805in}{1.041412in}}%
\pgfpathlineto{\pgfqpoint{12.383846in}{1.041412in}}%
\pgfpathlineto{\pgfqpoint{12.387872in}{1.036230in}}%
\pgfpathlineto{\pgfqpoint{12.389886in}{1.034821in}}%
\pgfpathlineto{\pgfqpoint{12.391899in}{1.035677in}}%
\pgfpathlineto{\pgfqpoint{12.397940in}{1.033413in}}%
\pgfpathlineto{\pgfqpoint{12.399953in}{1.035073in}}%
\pgfpathlineto{\pgfqpoint{12.401967in}{1.038695in}}%
\pgfpathlineto{\pgfqpoint{12.403980in}{1.039198in}}%
\pgfpathlineto{\pgfqpoint{12.405993in}{1.044028in}}%
\pgfpathlineto{\pgfqpoint{12.412034in}{1.045638in}}%
\pgfpathlineto{\pgfqpoint{12.414047in}{1.042670in}}%
\pgfpathlineto{\pgfqpoint{12.418074in}{1.048757in}}%
\pgfpathlineto{\pgfqpoint{12.420088in}{1.047953in}}%
\pgfpathlineto{\pgfqpoint{12.430155in}{1.038645in}}%
\pgfpathlineto{\pgfqpoint{12.432168in}{1.045638in}}%
\pgfpathlineto{\pgfqpoint{12.434182in}{1.041211in}}%
\pgfpathlineto{\pgfqpoint{12.440222in}{1.050720in}}%
\pgfpathlineto{\pgfqpoint{12.442235in}{1.050518in}}%
\pgfpathlineto{\pgfqpoint{12.446262in}{1.053285in}}%
\pgfpathlineto{\pgfqpoint{12.448276in}{1.066366in}}%
\pgfpathlineto{\pgfqpoint{12.454316in}{1.067523in}}%
\pgfpathlineto{\pgfqpoint{12.456330in}{1.066467in}}%
\pgfpathlineto{\pgfqpoint{12.458343in}{1.073309in}}%
\pgfpathlineto{\pgfqpoint{12.460357in}{1.074517in}}%
\pgfpathlineto{\pgfqpoint{12.462370in}{1.069083in}}%
\pgfpathlineto{\pgfqpoint{12.468410in}{1.066115in}}%
\pgfpathlineto{\pgfqpoint{12.470424in}{1.066517in}}%
\pgfpathlineto{\pgfqpoint{12.472437in}{1.069184in}}%
\pgfpathlineto{\pgfqpoint{12.476464in}{1.071850in}}%
\pgfpathlineto{\pgfqpoint{12.482504in}{1.072554in}}%
\pgfpathlineto{\pgfqpoint{12.484518in}{1.075925in}}%
\pgfpathlineto{\pgfqpoint{12.486531in}{1.073309in}}%
\pgfpathlineto{\pgfqpoint{12.488545in}{1.072454in}}%
\pgfpathlineto{\pgfqpoint{12.490558in}{1.070341in}}%
\pgfpathlineto{\pgfqpoint{12.496599in}{1.077334in}}%
\pgfpathlineto{\pgfqpoint{12.498612in}{1.080705in}}%
\pgfpathlineto{\pgfqpoint{12.500625in}{1.087094in}}%
\pgfpathlineto{\pgfqpoint{12.502639in}{1.097609in}}%
\pgfpathlineto{\pgfqpoint{12.504652in}{1.104804in}}%
\pgfpathlineto{\pgfqpoint{12.512706in}{1.099873in}}%
\pgfpathlineto{\pgfqpoint{12.516733in}{1.102489in}}%
\pgfpathlineto{\pgfqpoint{12.518746in}{1.100175in}}%
\pgfpathlineto{\pgfqpoint{12.524787in}{1.106917in}}%
\pgfpathlineto{\pgfqpoint{12.528814in}{1.108174in}}%
\pgfpathlineto{\pgfqpoint{12.530827in}{1.107520in}}%
\pgfpathlineto{\pgfqpoint{12.532841in}{1.106212in}}%
\pgfpathlineto{\pgfqpoint{12.540894in}{1.106313in}}%
\pgfpathlineto{\pgfqpoint{12.542908in}{1.102087in}}%
\pgfpathlineto{\pgfqpoint{12.544921in}{1.103345in}}%
\pgfpathlineto{\pgfqpoint{12.546935in}{1.101684in}}%
\pgfpathlineto{\pgfqpoint{12.554989in}{1.110690in}}%
\pgfpathlineto{\pgfqpoint{12.557002in}{1.117281in}}%
\pgfpathlineto{\pgfqpoint{12.559015in}{1.117029in}}%
\pgfpathlineto{\pgfqpoint{12.561029in}{1.124777in}}%
\pgfpathlineto{\pgfqpoint{12.567069in}{1.121758in}}%
\pgfpathlineto{\pgfqpoint{12.569083in}{1.121859in}}%
\pgfpathlineto{\pgfqpoint{12.571096in}{1.126991in}}%
\pgfpathlineto{\pgfqpoint{12.573110in}{1.117733in}}%
\pgfpathlineto{\pgfqpoint{12.575123in}{1.120299in}}%
\pgfpathlineto{\pgfqpoint{12.583177in}{1.119846in}}%
\pgfpathlineto{\pgfqpoint{12.585190in}{1.120802in}}%
\pgfpathlineto{\pgfqpoint{12.587204in}{1.116526in}}%
\pgfpathlineto{\pgfqpoint{12.589217in}{1.118387in}}%
\pgfpathlineto{\pgfqpoint{12.595257in}{1.115721in}}%
\pgfpathlineto{\pgfqpoint{12.597271in}{1.119545in}}%
\pgfpathlineto{\pgfqpoint{12.601298in}{1.120299in}}%
\pgfpathlineto{\pgfqpoint{12.603311in}{1.126286in}}%
\pgfpathlineto{\pgfqpoint{12.609352in}{1.134235in}}%
\pgfpathlineto{\pgfqpoint{12.611365in}{1.132877in}}%
\pgfpathlineto{\pgfqpoint{12.613379in}{1.135996in}}%
\pgfpathlineto{\pgfqpoint{12.615392in}{1.132676in}}%
\pgfpathlineto{\pgfqpoint{12.617405in}{1.131166in}}%
\pgfpathlineto{\pgfqpoint{12.623446in}{1.127645in}}%
\pgfpathlineto{\pgfqpoint{12.625459in}{1.124878in}}%
\pgfpathlineto{\pgfqpoint{12.627473in}{1.124878in}}%
\pgfpathlineto{\pgfqpoint{12.629486in}{1.127242in}}%
\pgfpathlineto{\pgfqpoint{12.631500in}{1.126135in}}%
\pgfpathlineto{\pgfqpoint{12.637540in}{1.127997in}}%
\pgfpathlineto{\pgfqpoint{12.639553in}{1.131317in}}%
\pgfpathlineto{\pgfqpoint{12.641567in}{1.130563in}}%
\pgfpathlineto{\pgfqpoint{12.643580in}{1.133129in}}%
\pgfpathlineto{\pgfqpoint{12.645594in}{1.130009in}}%
\pgfpathlineto{\pgfqpoint{12.653647in}{1.129758in}}%
\pgfpathlineto{\pgfqpoint{12.655661in}{1.130261in}}%
\pgfpathlineto{\pgfqpoint{12.657674in}{1.128399in}}%
\pgfpathlineto{\pgfqpoint{12.659688in}{1.131267in}}%
\pgfpathlineto{\pgfqpoint{12.667742in}{1.130160in}}%
\pgfpathlineto{\pgfqpoint{12.669755in}{1.134738in}}%
\pgfpathlineto{\pgfqpoint{12.671768in}{1.132575in}}%
\pgfpathlineto{\pgfqpoint{12.673782in}{1.135694in}}%
\pgfpathlineto{\pgfqpoint{12.679822in}{1.132927in}}%
\pgfpathlineto{\pgfqpoint{12.681836in}{1.133883in}}%
\pgfpathlineto{\pgfqpoint{12.683849in}{1.133783in}}%
\pgfpathlineto{\pgfqpoint{12.685863in}{1.134688in}}%
\pgfpathlineto{\pgfqpoint{12.687876in}{1.134185in}}%
\pgfpathlineto{\pgfqpoint{12.693916in}{1.137053in}}%
\pgfpathlineto{\pgfqpoint{12.695930in}{1.140876in}}%
\pgfpathlineto{\pgfqpoint{12.697943in}{1.138763in}}%
\pgfpathlineto{\pgfqpoint{12.699957in}{1.138009in}}%
\pgfpathlineto{\pgfqpoint{12.701970in}{1.138210in}}%
\pgfpathlineto{\pgfqpoint{12.708011in}{1.142838in}}%
\pgfpathlineto{\pgfqpoint{12.710024in}{1.138210in}}%
\pgfpathlineto{\pgfqpoint{12.712037in}{1.139770in}}%
\pgfpathlineto{\pgfqpoint{12.714051in}{1.140574in}}%
\pgfpathlineto{\pgfqpoint{12.716064in}{1.140071in}}%
\pgfpathlineto{\pgfqpoint{12.722105in}{1.141229in}}%
\pgfpathlineto{\pgfqpoint{12.724118in}{1.144499in}}%
\pgfpathlineto{\pgfqpoint{12.726132in}{1.141983in}}%
\pgfpathlineto{\pgfqpoint{12.728145in}{1.145102in}}%
\pgfpathlineto{\pgfqpoint{12.730158in}{1.146159in}}%
\pgfpathlineto{\pgfqpoint{12.742239in}{1.144499in}}%
\pgfpathlineto{\pgfqpoint{12.744253in}{1.142184in}}%
\pgfpathlineto{\pgfqpoint{12.750293in}{1.141480in}}%
\pgfpathlineto{\pgfqpoint{12.752306in}{1.144599in}}%
\pgfpathlineto{\pgfqpoint{12.754320in}{1.144448in}}%
\pgfpathlineto{\pgfqpoint{12.764387in}{1.148020in}}%
\pgfpathlineto{\pgfqpoint{12.766401in}{1.150033in}}%
\pgfpathlineto{\pgfqpoint{12.768414in}{1.147769in}}%
\pgfpathlineto{\pgfqpoint{12.770427in}{1.152901in}}%
\pgfpathlineto{\pgfqpoint{12.772441in}{1.151240in}}%
\pgfpathlineto{\pgfqpoint{12.778481in}{1.147618in}}%
\pgfpathlineto{\pgfqpoint{12.780495in}{1.154812in}}%
\pgfpathlineto{\pgfqpoint{12.784522in}{1.158032in}}%
\pgfpathlineto{\pgfqpoint{12.792575in}{1.152649in}}%
\pgfpathlineto{\pgfqpoint{12.794589in}{1.150888in}}%
\pgfpathlineto{\pgfqpoint{12.796602in}{1.137556in}}%
\pgfpathlineto{\pgfqpoint{12.798616in}{1.135392in}}%
\pgfpathlineto{\pgfqpoint{12.800629in}{1.139367in}}%
\pgfpathlineto{\pgfqpoint{12.806669in}{1.136600in}}%
\pgfpathlineto{\pgfqpoint{12.808683in}{1.139719in}}%
\pgfpathlineto{\pgfqpoint{12.810696in}{1.128047in}}%
\pgfpathlineto{\pgfqpoint{12.812710in}{1.127645in}}%
\pgfpathlineto{\pgfqpoint{12.814723in}{1.128198in}}%
\pgfpathlineto{\pgfqpoint{12.820764in}{1.125481in}}%
\pgfpathlineto{\pgfqpoint{12.822777in}{1.119897in}}%
\pgfpathlineto{\pgfqpoint{12.824790in}{1.111797in}}%
\pgfpathlineto{\pgfqpoint{12.826804in}{1.113608in}}%
\pgfpathlineto{\pgfqpoint{12.828817in}{1.117683in}}%
\pgfpathlineto{\pgfqpoint{12.834858in}{1.118237in}}%
\pgfpathlineto{\pgfqpoint{12.836871in}{1.115268in}}%
\pgfpathlineto{\pgfqpoint{12.838885in}{1.118589in}}%
\pgfpathlineto{\pgfqpoint{12.840898in}{1.116476in}}%
\pgfpathlineto{\pgfqpoint{12.842911in}{1.122010in}}%
\pgfpathlineto{\pgfqpoint{12.850965in}{1.121658in}}%
\pgfpathlineto{\pgfqpoint{12.852979in}{1.119696in}}%
\pgfpathlineto{\pgfqpoint{12.854992in}{1.121004in}}%
\pgfpathlineto{\pgfqpoint{12.857006in}{1.116023in}}%
\pgfpathlineto{\pgfqpoint{12.863046in}{1.112853in}}%
\pgfpathlineto{\pgfqpoint{12.865059in}{1.107873in}}%
\pgfpathlineto{\pgfqpoint{12.867073in}{1.109935in}}%
\pgfpathlineto{\pgfqpoint{12.869086in}{1.102187in}}%
\pgfpathlineto{\pgfqpoint{12.871100in}{1.108476in}}%
\pgfpathlineto{\pgfqpoint{12.877140in}{1.115369in}}%
\pgfpathlineto{\pgfqpoint{12.881167in}{1.110992in}}%
\pgfpathlineto{\pgfqpoint{12.883180in}{1.110237in}}%
\pgfpathlineto{\pgfqpoint{12.885194in}{1.107923in}}%
\pgfpathlineto{\pgfqpoint{12.891234in}{1.107269in}}%
\pgfpathlineto{\pgfqpoint{12.893248in}{1.100326in}}%
\pgfpathlineto{\pgfqpoint{12.895261in}{1.104502in}}%
\pgfpathlineto{\pgfqpoint{12.897275in}{1.101684in}}%
\pgfpathlineto{\pgfqpoint{12.899288in}{1.102389in}}%
\pgfpathlineto{\pgfqpoint{12.905328in}{1.108225in}}%
\pgfpathlineto{\pgfqpoint{12.907342in}{1.107470in}}%
\pgfpathlineto{\pgfqpoint{12.909355in}{1.114463in}}%
\pgfpathlineto{\pgfqpoint{12.911369in}{1.108929in}}%
\pgfpathlineto{\pgfqpoint{12.913382in}{1.111545in}}%
\pgfpathlineto{\pgfqpoint{12.919422in}{1.117381in}}%
\pgfpathlineto{\pgfqpoint{12.923449in}{1.108527in}}%
\pgfpathlineto{\pgfqpoint{12.925463in}{1.101181in}}%
\pgfpathlineto{\pgfqpoint{12.927476in}{1.101030in}}%
\pgfpathlineto{\pgfqpoint{12.933517in}{1.102389in}}%
\pgfpathlineto{\pgfqpoint{12.935530in}{1.103596in}}%
\pgfpathlineto{\pgfqpoint{12.937544in}{1.106212in}}%
\pgfpathlineto{\pgfqpoint{12.939557in}{1.105810in}}%
\pgfpathlineto{\pgfqpoint{12.941570in}{1.109684in}}%
\pgfpathlineto{\pgfqpoint{12.947611in}{1.108225in}}%
\pgfpathlineto{\pgfqpoint{12.951638in}{1.118136in}}%
\pgfpathlineto{\pgfqpoint{12.953651in}{1.120802in}}%
\pgfpathlineto{\pgfqpoint{12.955665in}{1.119444in}}%
\pgfpathlineto{\pgfqpoint{12.961705in}{1.119042in}}%
\pgfpathlineto{\pgfqpoint{12.963718in}{1.116174in}}%
\pgfpathlineto{\pgfqpoint{12.965732in}{1.118740in}}%
\pgfpathlineto{\pgfqpoint{12.967745in}{1.133682in}}%
\pgfpathlineto{\pgfqpoint{12.975799in}{1.133330in}}%
\pgfpathlineto{\pgfqpoint{12.977812in}{1.136650in}}%
\pgfpathlineto{\pgfqpoint{12.979826in}{1.127192in}}%
\pgfpathlineto{\pgfqpoint{12.981839in}{1.129355in}}%
\pgfpathlineto{\pgfqpoint{12.983853in}{1.122412in}}%
\pgfpathlineto{\pgfqpoint{12.989893in}{1.115872in}}%
\pgfpathlineto{\pgfqpoint{12.991907in}{1.118941in}}%
\pgfpathlineto{\pgfqpoint{12.993920in}{1.098666in}}%
\pgfpathlineto{\pgfqpoint{12.995933in}{1.091421in}}%
\pgfpathlineto{\pgfqpoint{12.997947in}{1.094540in}}%
\pgfpathlineto{\pgfqpoint{13.003987in}{1.091672in}}%
\pgfpathlineto{\pgfqpoint{13.006001in}{1.092226in}}%
\pgfpathlineto{\pgfqpoint{13.008014in}{1.095597in}}%
\pgfpathlineto{\pgfqpoint{13.012041in}{1.088251in}}%
\pgfpathlineto{\pgfqpoint{13.018081in}{1.090566in}}%
\pgfpathlineto{\pgfqpoint{13.020095in}{1.098364in}}%
\pgfpathlineto{\pgfqpoint{13.022108in}{1.092176in}}%
\pgfpathlineto{\pgfqpoint{13.024122in}{1.092276in}}%
\pgfpathlineto{\pgfqpoint{13.026135in}{1.096603in}}%
\pgfpathlineto{\pgfqpoint{13.034189in}{1.097408in}}%
\pgfpathlineto{\pgfqpoint{13.036202in}{1.098867in}}%
\pgfpathlineto{\pgfqpoint{13.038216in}{1.090717in}}%
\pgfpathlineto{\pgfqpoint{13.040229in}{1.092176in}}%
\pgfpathlineto{\pgfqpoint{13.048283in}{1.092679in}}%
\pgfpathlineto{\pgfqpoint{13.050297in}{1.092176in}}%
\pgfpathlineto{\pgfqpoint{13.052310in}{1.070492in}}%
\pgfpathlineto{\pgfqpoint{13.060364in}{1.070643in}}%
\pgfpathlineto{\pgfqpoint{13.064391in}{1.079095in}}%
\pgfpathlineto{\pgfqpoint{13.066404in}{1.074567in}}%
\pgfpathlineto{\pgfqpoint{13.068418in}{1.077636in}}%
\pgfpathlineto{\pgfqpoint{13.074458in}{1.075573in}}%
\pgfpathlineto{\pgfqpoint{13.076471in}{1.077183in}}%
\pgfpathlineto{\pgfqpoint{13.078485in}{1.081007in}}%
\pgfpathlineto{\pgfqpoint{13.082512in}{1.077988in}}%
\pgfpathlineto{\pgfqpoint{13.088552in}{1.082768in}}%
\pgfpathlineto{\pgfqpoint{13.090566in}{1.078139in}}%
\pgfpathlineto{\pgfqpoint{13.092579in}{1.081107in}}%
\pgfpathlineto{\pgfqpoint{13.094592in}{1.075322in}}%
\pgfpathlineto{\pgfqpoint{13.096606in}{1.077837in}}%
\pgfpathlineto{\pgfqpoint{13.102646in}{1.084176in}}%
\pgfpathlineto{\pgfqpoint{13.104660in}{1.088704in}}%
\pgfpathlineto{\pgfqpoint{13.106673in}{1.087547in}}%
\pgfpathlineto{\pgfqpoint{13.108687in}{1.085384in}}%
\pgfpathlineto{\pgfqpoint{13.110700in}{1.085182in}}%
\pgfpathlineto{\pgfqpoint{13.116740in}{1.082768in}}%
\pgfpathlineto{\pgfqpoint{13.118754in}{1.082768in}}%
\pgfpathlineto{\pgfqpoint{13.120767in}{1.077787in}}%
\pgfpathlineto{\pgfqpoint{13.122781in}{1.069838in}}%
\pgfpathlineto{\pgfqpoint{13.124794in}{1.072051in}}%
\pgfpathlineto{\pgfqpoint{13.132848in}{1.076831in}}%
\pgfpathlineto{\pgfqpoint{13.134861in}{1.076277in}}%
\pgfpathlineto{\pgfqpoint{13.136875in}{1.080000in}}%
\pgfpathlineto{\pgfqpoint{13.138888in}{1.081912in}}%
\pgfpathlineto{\pgfqpoint{13.144929in}{1.078491in}}%
\pgfpathlineto{\pgfqpoint{13.148955in}{1.074114in}}%
\pgfpathlineto{\pgfqpoint{13.150969in}{1.077787in}}%
\pgfpathlineto{\pgfqpoint{13.152982in}{1.076579in}}%
\pgfpathlineto{\pgfqpoint{13.159023in}{1.075271in}}%
\pgfpathlineto{\pgfqpoint{13.161036in}{1.074164in}}%
\pgfpathlineto{\pgfqpoint{13.163050in}{1.080101in}}%
\pgfpathlineto{\pgfqpoint{13.165063in}{1.076781in}}%
\pgfpathlineto{\pgfqpoint{13.167077in}{1.078189in}}%
\pgfpathlineto{\pgfqpoint{13.173117in}{1.087950in}}%
\pgfpathlineto{\pgfqpoint{13.175130in}{1.092679in}}%
\pgfpathlineto{\pgfqpoint{13.177144in}{1.090616in}}%
\pgfpathlineto{\pgfqpoint{13.179157in}{1.097911in}}%
\pgfpathlineto{\pgfqpoint{13.181171in}{1.108174in}}%
\pgfpathlineto{\pgfqpoint{13.187211in}{1.107973in}}%
\pgfpathlineto{\pgfqpoint{13.189224in}{1.100326in}}%
\pgfpathlineto{\pgfqpoint{13.191238in}{1.102841in}}%
\pgfpathlineto{\pgfqpoint{13.195265in}{1.101634in}}%
\pgfpathlineto{\pgfqpoint{13.201305in}{1.098263in}}%
\pgfpathlineto{\pgfqpoint{13.203319in}{1.099471in}}%
\pgfpathlineto{\pgfqpoint{13.205332in}{1.098213in}}%
\pgfpathlineto{\pgfqpoint{13.209359in}{1.097710in}}%
\pgfpathlineto{\pgfqpoint{13.215399in}{1.098464in}}%
\pgfpathlineto{\pgfqpoint{13.217413in}{1.101483in}}%
\pgfpathlineto{\pgfqpoint{13.219426in}{1.110438in}}%
\pgfpathlineto{\pgfqpoint{13.221440in}{1.108376in}}%
\pgfpathlineto{\pgfqpoint{13.223453in}{1.110489in}}%
\pgfpathlineto{\pgfqpoint{13.229493in}{1.134738in}}%
\pgfpathlineto{\pgfqpoint{13.231507in}{1.120098in}}%
\pgfpathlineto{\pgfqpoint{13.233520in}{1.111495in}}%
\pgfpathlineto{\pgfqpoint{13.237547in}{1.109583in}}%
\pgfpathlineto{\pgfqpoint{13.243587in}{1.122412in}}%
\pgfpathlineto{\pgfqpoint{13.245601in}{1.125330in}}%
\pgfpathlineto{\pgfqpoint{13.247614in}{1.126236in}}%
\pgfpathlineto{\pgfqpoint{13.249628in}{1.140776in}}%
\pgfpathlineto{\pgfqpoint{13.251641in}{1.144247in}}%
\pgfpathlineto{\pgfqpoint{13.257682in}{1.143040in}}%
\pgfpathlineto{\pgfqpoint{13.259695in}{1.146914in}}%
\pgfpathlineto{\pgfqpoint{13.261709in}{1.136449in}}%
\pgfpathlineto{\pgfqpoint{13.263722in}{1.135896in}}%
\pgfpathlineto{\pgfqpoint{13.265735in}{1.131468in}}%
\pgfpathlineto{\pgfqpoint{13.273789in}{1.128751in}}%
\pgfpathlineto{\pgfqpoint{13.275803in}{1.126387in}}%
\pgfpathlineto{\pgfqpoint{13.277816in}{1.126991in}}%
\pgfpathlineto{\pgfqpoint{13.279830in}{1.125733in}}%
\pgfpathlineto{\pgfqpoint{13.279830in}{1.125733in}}%
\pgfusepath{stroke}%
\end{pgfscope}%
\begin{pgfscope}%
\pgfsetrectcap%
\pgfsetmiterjoin%
\pgfsetlinewidth{0.803000pt}%
\definecolor{currentstroke}{rgb}{1.000000,1.000000,1.000000}%
\pgfsetstrokecolor{currentstroke}%
\pgfsetdash{}{0pt}%
\pgfpathmoveto{\pgfqpoint{8.656250in}{0.750000in}}%
\pgfpathlineto{\pgfqpoint{8.656250in}{1.194118in}}%
\pgfusepath{stroke}%
\end{pgfscope}%
\begin{pgfscope}%
\pgfsetrectcap%
\pgfsetmiterjoin%
\pgfsetlinewidth{0.803000pt}%
\definecolor{currentstroke}{rgb}{1.000000,1.000000,1.000000}%
\pgfsetstrokecolor{currentstroke}%
\pgfsetdash{}{0pt}%
\pgfpathmoveto{\pgfqpoint{13.500000in}{0.750000in}}%
\pgfpathlineto{\pgfqpoint{13.500000in}{1.194118in}}%
\pgfusepath{stroke}%
\end{pgfscope}%
\begin{pgfscope}%
\pgfsetrectcap%
\pgfsetmiterjoin%
\pgfsetlinewidth{0.803000pt}%
\definecolor{currentstroke}{rgb}{1.000000,1.000000,1.000000}%
\pgfsetstrokecolor{currentstroke}%
\pgfsetdash{}{0pt}%
\pgfpathmoveto{\pgfqpoint{8.656250in}{0.750000in}}%
\pgfpathlineto{\pgfqpoint{13.500000in}{0.750000in}}%
\pgfusepath{stroke}%
\end{pgfscope}%
\begin{pgfscope}%
\pgfsetrectcap%
\pgfsetmiterjoin%
\pgfsetlinewidth{0.803000pt}%
\definecolor{currentstroke}{rgb}{1.000000,1.000000,1.000000}%
\pgfsetstrokecolor{currentstroke}%
\pgfsetdash{}{0pt}%
\pgfpathmoveto{\pgfqpoint{8.656250in}{1.194118in}}%
\pgfpathlineto{\pgfqpoint{13.500000in}{1.194118in}}%
\pgfusepath{stroke}%
\end{pgfscope}%
\begin{pgfscope}%
\definecolor{textcolor}{rgb}{0.150000,0.150000,0.150000}%
\pgfsetstrokecolor{textcolor}%
\pgfsetfillcolor{textcolor}%
\pgftext[x=11.078125in,y=1.277451in,,base]{\color{textcolor}\rmfamily\fontsize{16.800000}{20.160000}\selectfont DIS}%
\end{pgfscope}%
\end{pgfpicture}%
\makeatother%
\endgroup%

    \end{adjustbox}  
    \caption{Daily Stock Prices for all Stocks in the Data Set}
    \label{fig:adj_close_all}
\end{figure}{}







We see that the time series exhibits strong autocorrelation through the first lag. We therefore take first differences. 

The data then looks like this: 



Figure3: FD of Log closing prices. Entweder alle oder einige zum Beispiel
fig.savefig('/srv/shared/StatistischesPraktikumReport/Figures/all_log_adjclose_fd_and_qq.png')
Eventuell auch mit QQ-Plot

Visually the data looks like white noise. Also the mean of those values is close to zero

Table1: means of the values

Looking at the QQ-plot (explanation) we see that the distribution has fat tails: some of the values are more extreme than we would have expected if the differenced log-values were randomly distributed. 


\subsection{Predictions With Random Walks}
A random walk follows (and is the same as AR(1)??? MA(1))
XXXXXXXX
Predictions for period t + 1 are therefore exactly the value at time t. --> We do that on the FD of log-values scale but we could also do it on the real scale

Confidence Intervals are computed as
XXXXXXXX

Fitting: 
Our Code fits an RW model, prediicts the next value, compares it to the real value for the MSE and then adds the true value to the time series used for predicting the next value. (we could have done this simpler, by just taking the values of the previous period as prediction for the current one. 

Figure: Plot Real vs. Predicted values. (Real values? or FD of log-Values?
Table: MSE

\subsection{Predictions With AR(x) Models}
AR(x) models follow
XXXXXX

Die Frage ist so ein bisschen, ob wir versuchen sollen, die ganze Zeitreihe mit einem einzelnen Fit zu beschreiben, oder ob wir jede Periode ein neues Modell fitten sollten. 
Table: MSE
Figure: Plot Real vs. Predicted values. (Real values? or FD of log-Values?

\subsection{Predictions With ARMA Models}
ARMA() follows
XXXXXXX


Auto-ARMA


\subsection{Predictions With GARCH Models}
GARCH follows 
XXXXXXX

Idea behind GARCH

predictions
fitting
confidence intervals
MSE

Plot: Squared time series



\subsection{Summary of Prediction with Time Series Models}
Table: MSE all
Figure: All predicted values?


\section{Hybrid Prediction}

\subsection{ARMAX Predictions}
ARMAX works like this: 
XXXXXX

Predictions
Confidence Intervals


Predictors can be 
- Using weather forecasts --> ARMAX
- Number of Tweets?
- Sentiments from Machine Learning Algorithm
- Predictions made by the Algorithm



\subsection{Weighted Average of Predictions}
a) of different time series models
b) of time series and ML models
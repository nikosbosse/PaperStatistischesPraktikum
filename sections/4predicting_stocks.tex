\chapter{Predicting Stocks}\label{ch:predictions}


\section{Predictions Using Machine Learning}

\section{Predictions Using Time Series}
\subsection{Idea/Process and Evaluation}
irgendwas in der Richtung: wir benutzen Time Series Modelle, machen Predictions und gucken uns am Ende dann den Mean Squared Error an. Sinnvollerweise immer die ersten 10 Perioden verwerfen, um den MSE vergleichbar zu machen zwischen allen Gruppen, auch denen, bei denen die ersten paar Perioden nicht definiert sind. 	
In-Sample vs. Out of Sample Prediction?
Do we need a Training, Validation and Test Period? Probably yes, since the respective next value was also used for validation, not only testing (we choose the model that minimizes MSE)

\subsection{Data Preparation}
% Überlegung: Data Preparation Steps hier oder bei Data?

We can clearly see that the Data exhibits a trend. As Figure \ref{fig:cum_sd_all} shows the variance of the time series is not constant over time. 

\begin{figure}[h]
    \centering
    \begin{adjustbox}{width=.9\textwidth,center}
    %% Creator: Matplotlib, PGF backend
%%
%% To include the figure in your LaTeX document, write
%%   \input{<filename>.pgf}
%%
%% Make sure the required packages are loaded in your preamble
%%   \usepackage{pgf}
%%
%% Figures using additional raster images can only be included by \input if
%% they are in the same directory as the main LaTeX file. For loading figures
%% from other directories you can use the `import` package
%%   \usepackage{import}
%% and then include the figures with
%%   \import{<path to file>}{<filename>.pgf}
%%
%% Matplotlib used the following preamble
%%   \usepackage{fontspec}
%%   \setmainfont{DejaVuSerif.ttf}[Path=/opt/tljh/user/lib/python3.6/site-packages/matplotlib/mpl-data/fonts/ttf/]
%%   \setsansfont{DejaVuSans.ttf}[Path=/opt/tljh/user/lib/python3.6/site-packages/matplotlib/mpl-data/fonts/ttf/]
%%   \setmonofont{DejaVuSansMono.ttf}[Path=/opt/tljh/user/lib/python3.6/site-packages/matplotlib/mpl-data/fonts/ttf/]
%%
\begingroup%
\makeatletter%
\begin{pgfpicture}%
\pgfpathrectangle{\pgfpointorigin}{\pgfqpoint{6.863921in}{3.474064in}}%
\pgfusepath{use as bounding box, clip}%
\begin{pgfscope}%
\pgfsetbuttcap%
\pgfsetmiterjoin%
\definecolor{currentfill}{rgb}{1.000000,1.000000,1.000000}%
\pgfsetfillcolor{currentfill}%
\pgfsetlinewidth{0.000000pt}%
\definecolor{currentstroke}{rgb}{1.000000,1.000000,1.000000}%
\pgfsetstrokecolor{currentstroke}%
\pgfsetdash{}{0pt}%
\pgfpathmoveto{\pgfqpoint{0.000000in}{0.000000in}}%
\pgfpathlineto{\pgfqpoint{6.863921in}{0.000000in}}%
\pgfpathlineto{\pgfqpoint{6.863921in}{3.474064in}}%
\pgfpathlineto{\pgfqpoint{0.000000in}{3.474064in}}%
\pgfpathclose%
\pgfusepath{fill}%
\end{pgfscope}%
\begin{pgfscope}%
\pgfsetbuttcap%
\pgfsetmiterjoin%
\definecolor{currentfill}{rgb}{0.917647,0.917647,0.949020}%
\pgfsetfillcolor{currentfill}%
\pgfsetlinewidth{0.000000pt}%
\definecolor{currentstroke}{rgb}{0.000000,0.000000,0.000000}%
\pgfsetstrokecolor{currentstroke}%
\pgfsetstrokeopacity{0.000000}%
\pgfsetdash{}{0pt}%
\pgfpathmoveto{\pgfqpoint{0.563921in}{0.521603in}}%
\pgfpathlineto{\pgfqpoint{6.763921in}{0.521603in}}%
\pgfpathlineto{\pgfqpoint{6.763921in}{3.164103in}}%
\pgfpathlineto{\pgfqpoint{0.563921in}{3.164103in}}%
\pgfpathclose%
\pgfusepath{fill}%
\end{pgfscope}%
\begin{pgfscope}%
\pgfpathrectangle{\pgfqpoint{0.563921in}{0.521603in}}{\pgfqpoint{6.200000in}{2.642500in}}%
\pgfusepath{clip}%
\pgfsetroundcap%
\pgfsetroundjoin%
\pgfsetlinewidth{0.803000pt}%
\definecolor{currentstroke}{rgb}{1.000000,1.000000,1.000000}%
\pgfsetstrokecolor{currentstroke}%
\pgfsetdash{}{0pt}%
\pgfpathmoveto{\pgfqpoint{0.840585in}{0.521603in}}%
\pgfpathlineto{\pgfqpoint{0.840585in}{3.164103in}}%
\pgfusepath{stroke}%
\end{pgfscope}%
\begin{pgfscope}%
\definecolor{textcolor}{rgb}{0.150000,0.150000,0.150000}%
\pgfsetstrokecolor{textcolor}%
\pgfsetfillcolor{textcolor}%
\pgftext[x=0.840585in,y=0.424381in,,top]{\color{textcolor}\rmfamily\fontsize{10.000000}{12.000000}\selectfont 2012}%
\end{pgfscope}%
\begin{pgfscope}%
\pgfpathrectangle{\pgfqpoint{0.563921in}{0.521603in}}{\pgfqpoint{6.200000in}{2.642500in}}%
\pgfusepath{clip}%
\pgfsetroundcap%
\pgfsetroundjoin%
\pgfsetlinewidth{0.803000pt}%
\definecolor{currentstroke}{rgb}{1.000000,1.000000,1.000000}%
\pgfsetstrokecolor{currentstroke}%
\pgfsetdash{}{0pt}%
\pgfpathmoveto{\pgfqpoint{1.783845in}{0.521603in}}%
\pgfpathlineto{\pgfqpoint{1.783845in}{3.164103in}}%
\pgfusepath{stroke}%
\end{pgfscope}%
\begin{pgfscope}%
\definecolor{textcolor}{rgb}{0.150000,0.150000,0.150000}%
\pgfsetstrokecolor{textcolor}%
\pgfsetfillcolor{textcolor}%
\pgftext[x=1.783845in,y=0.424381in,,top]{\color{textcolor}\rmfamily\fontsize{10.000000}{12.000000}\selectfont 2013}%
\end{pgfscope}%
\begin{pgfscope}%
\pgfpathrectangle{\pgfqpoint{0.563921in}{0.521603in}}{\pgfqpoint{6.200000in}{2.642500in}}%
\pgfusepath{clip}%
\pgfsetroundcap%
\pgfsetroundjoin%
\pgfsetlinewidth{0.803000pt}%
\definecolor{currentstroke}{rgb}{1.000000,1.000000,1.000000}%
\pgfsetstrokecolor{currentstroke}%
\pgfsetdash{}{0pt}%
\pgfpathmoveto{\pgfqpoint{2.724527in}{0.521603in}}%
\pgfpathlineto{\pgfqpoint{2.724527in}{3.164103in}}%
\pgfusepath{stroke}%
\end{pgfscope}%
\begin{pgfscope}%
\definecolor{textcolor}{rgb}{0.150000,0.150000,0.150000}%
\pgfsetstrokecolor{textcolor}%
\pgfsetfillcolor{textcolor}%
\pgftext[x=2.724527in,y=0.424381in,,top]{\color{textcolor}\rmfamily\fontsize{10.000000}{12.000000}\selectfont 2014}%
\end{pgfscope}%
\begin{pgfscope}%
\pgfpathrectangle{\pgfqpoint{0.563921in}{0.521603in}}{\pgfqpoint{6.200000in}{2.642500in}}%
\pgfusepath{clip}%
\pgfsetroundcap%
\pgfsetroundjoin%
\pgfsetlinewidth{0.803000pt}%
\definecolor{currentstroke}{rgb}{1.000000,1.000000,1.000000}%
\pgfsetstrokecolor{currentstroke}%
\pgfsetdash{}{0pt}%
\pgfpathmoveto{\pgfqpoint{3.665210in}{0.521603in}}%
\pgfpathlineto{\pgfqpoint{3.665210in}{3.164103in}}%
\pgfusepath{stroke}%
\end{pgfscope}%
\begin{pgfscope}%
\definecolor{textcolor}{rgb}{0.150000,0.150000,0.150000}%
\pgfsetstrokecolor{textcolor}%
\pgfsetfillcolor{textcolor}%
\pgftext[x=3.665210in,y=0.424381in,,top]{\color{textcolor}\rmfamily\fontsize{10.000000}{12.000000}\selectfont 2015}%
\end{pgfscope}%
\begin{pgfscope}%
\pgfpathrectangle{\pgfqpoint{0.563921in}{0.521603in}}{\pgfqpoint{6.200000in}{2.642500in}}%
\pgfusepath{clip}%
\pgfsetroundcap%
\pgfsetroundjoin%
\pgfsetlinewidth{0.803000pt}%
\definecolor{currentstroke}{rgb}{1.000000,1.000000,1.000000}%
\pgfsetstrokecolor{currentstroke}%
\pgfsetdash{}{0pt}%
\pgfpathmoveto{\pgfqpoint{4.605892in}{0.521603in}}%
\pgfpathlineto{\pgfqpoint{4.605892in}{3.164103in}}%
\pgfusepath{stroke}%
\end{pgfscope}%
\begin{pgfscope}%
\definecolor{textcolor}{rgb}{0.150000,0.150000,0.150000}%
\pgfsetstrokecolor{textcolor}%
\pgfsetfillcolor{textcolor}%
\pgftext[x=4.605892in,y=0.424381in,,top]{\color{textcolor}\rmfamily\fontsize{10.000000}{12.000000}\selectfont 2016}%
\end{pgfscope}%
\begin{pgfscope}%
\pgfpathrectangle{\pgfqpoint{0.563921in}{0.521603in}}{\pgfqpoint{6.200000in}{2.642500in}}%
\pgfusepath{clip}%
\pgfsetroundcap%
\pgfsetroundjoin%
\pgfsetlinewidth{0.803000pt}%
\definecolor{currentstroke}{rgb}{1.000000,1.000000,1.000000}%
\pgfsetstrokecolor{currentstroke}%
\pgfsetdash{}{0pt}%
\pgfpathmoveto{\pgfqpoint{5.549152in}{0.521603in}}%
\pgfpathlineto{\pgfqpoint{5.549152in}{3.164103in}}%
\pgfusepath{stroke}%
\end{pgfscope}%
\begin{pgfscope}%
\definecolor{textcolor}{rgb}{0.150000,0.150000,0.150000}%
\pgfsetstrokecolor{textcolor}%
\pgfsetfillcolor{textcolor}%
\pgftext[x=5.549152in,y=0.424381in,,top]{\color{textcolor}\rmfamily\fontsize{10.000000}{12.000000}\selectfont 2017}%
\end{pgfscope}%
\begin{pgfscope}%
\pgfpathrectangle{\pgfqpoint{0.563921in}{0.521603in}}{\pgfqpoint{6.200000in}{2.642500in}}%
\pgfusepath{clip}%
\pgfsetroundcap%
\pgfsetroundjoin%
\pgfsetlinewidth{0.803000pt}%
\definecolor{currentstroke}{rgb}{1.000000,1.000000,1.000000}%
\pgfsetstrokecolor{currentstroke}%
\pgfsetdash{}{0pt}%
\pgfpathmoveto{\pgfqpoint{6.489835in}{0.521603in}}%
\pgfpathlineto{\pgfqpoint{6.489835in}{3.164103in}}%
\pgfusepath{stroke}%
\end{pgfscope}%
\begin{pgfscope}%
\definecolor{textcolor}{rgb}{0.150000,0.150000,0.150000}%
\pgfsetstrokecolor{textcolor}%
\pgfsetfillcolor{textcolor}%
\pgftext[x=6.489835in,y=0.424381in,,top]{\color{textcolor}\rmfamily\fontsize{10.000000}{12.000000}\selectfont 2018}%
\end{pgfscope}%
\begin{pgfscope}%
\definecolor{textcolor}{rgb}{0.150000,0.150000,0.150000}%
\pgfsetstrokecolor{textcolor}%
\pgfsetfillcolor{textcolor}%
\pgftext[x=3.663921in,y=0.234413in,,top]{\color{textcolor}\rmfamily\fontsize{10.000000}{12.000000}\selectfont Time t}%
\end{pgfscope}%
\begin{pgfscope}%
\pgfpathrectangle{\pgfqpoint{0.563921in}{0.521603in}}{\pgfqpoint{6.200000in}{2.642500in}}%
\pgfusepath{clip}%
\pgfsetroundcap%
\pgfsetroundjoin%
\pgfsetlinewidth{0.803000pt}%
\definecolor{currentstroke}{rgb}{1.000000,1.000000,1.000000}%
\pgfsetstrokecolor{currentstroke}%
\pgfsetdash{}{0pt}%
\pgfpathmoveto{\pgfqpoint{0.563921in}{0.641717in}}%
\pgfpathlineto{\pgfqpoint{6.763921in}{0.641717in}}%
\pgfusepath{stroke}%
\end{pgfscope}%
\begin{pgfscope}%
\definecolor{textcolor}{rgb}{0.150000,0.150000,0.150000}%
\pgfsetstrokecolor{textcolor}%
\pgfsetfillcolor{textcolor}%
\pgftext[x=0.378334in,y=0.588955in,left,base]{\color{textcolor}\rmfamily\fontsize{10.000000}{12.000000}\selectfont 0}%
\end{pgfscope}%
\begin{pgfscope}%
\pgfpathrectangle{\pgfqpoint{0.563921in}{0.521603in}}{\pgfqpoint{6.200000in}{2.642500in}}%
\pgfusepath{clip}%
\pgfsetroundcap%
\pgfsetroundjoin%
\pgfsetlinewidth{0.803000pt}%
\definecolor{currentstroke}{rgb}{1.000000,1.000000,1.000000}%
\pgfsetstrokecolor{currentstroke}%
\pgfsetdash{}{0pt}%
\pgfpathmoveto{\pgfqpoint{0.563921in}{0.939662in}}%
\pgfpathlineto{\pgfqpoint{6.763921in}{0.939662in}}%
\pgfusepath{stroke}%
\end{pgfscope}%
\begin{pgfscope}%
\definecolor{textcolor}{rgb}{0.150000,0.150000,0.150000}%
\pgfsetstrokecolor{textcolor}%
\pgfsetfillcolor{textcolor}%
\pgftext[x=0.378334in,y=0.886901in,left,base]{\color{textcolor}\rmfamily\fontsize{10.000000}{12.000000}\selectfont 5}%
\end{pgfscope}%
\begin{pgfscope}%
\pgfpathrectangle{\pgfqpoint{0.563921in}{0.521603in}}{\pgfqpoint{6.200000in}{2.642500in}}%
\pgfusepath{clip}%
\pgfsetroundcap%
\pgfsetroundjoin%
\pgfsetlinewidth{0.803000pt}%
\definecolor{currentstroke}{rgb}{1.000000,1.000000,1.000000}%
\pgfsetstrokecolor{currentstroke}%
\pgfsetdash{}{0pt}%
\pgfpathmoveto{\pgfqpoint{0.563921in}{1.237607in}}%
\pgfpathlineto{\pgfqpoint{6.763921in}{1.237607in}}%
\pgfusepath{stroke}%
\end{pgfscope}%
\begin{pgfscope}%
\definecolor{textcolor}{rgb}{0.150000,0.150000,0.150000}%
\pgfsetstrokecolor{textcolor}%
\pgfsetfillcolor{textcolor}%
\pgftext[x=0.289968in,y=1.184846in,left,base]{\color{textcolor}\rmfamily\fontsize{10.000000}{12.000000}\selectfont 10}%
\end{pgfscope}%
\begin{pgfscope}%
\pgfpathrectangle{\pgfqpoint{0.563921in}{0.521603in}}{\pgfqpoint{6.200000in}{2.642500in}}%
\pgfusepath{clip}%
\pgfsetroundcap%
\pgfsetroundjoin%
\pgfsetlinewidth{0.803000pt}%
\definecolor{currentstroke}{rgb}{1.000000,1.000000,1.000000}%
\pgfsetstrokecolor{currentstroke}%
\pgfsetdash{}{0pt}%
\pgfpathmoveto{\pgfqpoint{0.563921in}{1.535553in}}%
\pgfpathlineto{\pgfqpoint{6.763921in}{1.535553in}}%
\pgfusepath{stroke}%
\end{pgfscope}%
\begin{pgfscope}%
\definecolor{textcolor}{rgb}{0.150000,0.150000,0.150000}%
\pgfsetstrokecolor{textcolor}%
\pgfsetfillcolor{textcolor}%
\pgftext[x=0.289968in,y=1.482791in,left,base]{\color{textcolor}\rmfamily\fontsize{10.000000}{12.000000}\selectfont 15}%
\end{pgfscope}%
\begin{pgfscope}%
\pgfpathrectangle{\pgfqpoint{0.563921in}{0.521603in}}{\pgfqpoint{6.200000in}{2.642500in}}%
\pgfusepath{clip}%
\pgfsetroundcap%
\pgfsetroundjoin%
\pgfsetlinewidth{0.803000pt}%
\definecolor{currentstroke}{rgb}{1.000000,1.000000,1.000000}%
\pgfsetstrokecolor{currentstroke}%
\pgfsetdash{}{0pt}%
\pgfpathmoveto{\pgfqpoint{0.563921in}{1.833498in}}%
\pgfpathlineto{\pgfqpoint{6.763921in}{1.833498in}}%
\pgfusepath{stroke}%
\end{pgfscope}%
\begin{pgfscope}%
\definecolor{textcolor}{rgb}{0.150000,0.150000,0.150000}%
\pgfsetstrokecolor{textcolor}%
\pgfsetfillcolor{textcolor}%
\pgftext[x=0.289968in,y=1.780736in,left,base]{\color{textcolor}\rmfamily\fontsize{10.000000}{12.000000}\selectfont 20}%
\end{pgfscope}%
\begin{pgfscope}%
\pgfpathrectangle{\pgfqpoint{0.563921in}{0.521603in}}{\pgfqpoint{6.200000in}{2.642500in}}%
\pgfusepath{clip}%
\pgfsetroundcap%
\pgfsetroundjoin%
\pgfsetlinewidth{0.803000pt}%
\definecolor{currentstroke}{rgb}{1.000000,1.000000,1.000000}%
\pgfsetstrokecolor{currentstroke}%
\pgfsetdash{}{0pt}%
\pgfpathmoveto{\pgfqpoint{0.563921in}{2.131443in}}%
\pgfpathlineto{\pgfqpoint{6.763921in}{2.131443in}}%
\pgfusepath{stroke}%
\end{pgfscope}%
\begin{pgfscope}%
\definecolor{textcolor}{rgb}{0.150000,0.150000,0.150000}%
\pgfsetstrokecolor{textcolor}%
\pgfsetfillcolor{textcolor}%
\pgftext[x=0.289968in,y=2.078682in,left,base]{\color{textcolor}\rmfamily\fontsize{10.000000}{12.000000}\selectfont 25}%
\end{pgfscope}%
\begin{pgfscope}%
\pgfpathrectangle{\pgfqpoint{0.563921in}{0.521603in}}{\pgfqpoint{6.200000in}{2.642500in}}%
\pgfusepath{clip}%
\pgfsetroundcap%
\pgfsetroundjoin%
\pgfsetlinewidth{0.803000pt}%
\definecolor{currentstroke}{rgb}{1.000000,1.000000,1.000000}%
\pgfsetstrokecolor{currentstroke}%
\pgfsetdash{}{0pt}%
\pgfpathmoveto{\pgfqpoint{0.563921in}{2.429388in}}%
\pgfpathlineto{\pgfqpoint{6.763921in}{2.429388in}}%
\pgfusepath{stroke}%
\end{pgfscope}%
\begin{pgfscope}%
\definecolor{textcolor}{rgb}{0.150000,0.150000,0.150000}%
\pgfsetstrokecolor{textcolor}%
\pgfsetfillcolor{textcolor}%
\pgftext[x=0.289968in,y=2.376627in,left,base]{\color{textcolor}\rmfamily\fontsize{10.000000}{12.000000}\selectfont 30}%
\end{pgfscope}%
\begin{pgfscope}%
\pgfpathrectangle{\pgfqpoint{0.563921in}{0.521603in}}{\pgfqpoint{6.200000in}{2.642500in}}%
\pgfusepath{clip}%
\pgfsetroundcap%
\pgfsetroundjoin%
\pgfsetlinewidth{0.803000pt}%
\definecolor{currentstroke}{rgb}{1.000000,1.000000,1.000000}%
\pgfsetstrokecolor{currentstroke}%
\pgfsetdash{}{0pt}%
\pgfpathmoveto{\pgfqpoint{0.563921in}{2.727334in}}%
\pgfpathlineto{\pgfqpoint{6.763921in}{2.727334in}}%
\pgfusepath{stroke}%
\end{pgfscope}%
\begin{pgfscope}%
\definecolor{textcolor}{rgb}{0.150000,0.150000,0.150000}%
\pgfsetstrokecolor{textcolor}%
\pgfsetfillcolor{textcolor}%
\pgftext[x=0.289968in,y=2.674572in,left,base]{\color{textcolor}\rmfamily\fontsize{10.000000}{12.000000}\selectfont 35}%
\end{pgfscope}%
\begin{pgfscope}%
\pgfpathrectangle{\pgfqpoint{0.563921in}{0.521603in}}{\pgfqpoint{6.200000in}{2.642500in}}%
\pgfusepath{clip}%
\pgfsetroundcap%
\pgfsetroundjoin%
\pgfsetlinewidth{0.803000pt}%
\definecolor{currentstroke}{rgb}{1.000000,1.000000,1.000000}%
\pgfsetstrokecolor{currentstroke}%
\pgfsetdash{}{0pt}%
\pgfpathmoveto{\pgfqpoint{0.563921in}{3.025279in}}%
\pgfpathlineto{\pgfqpoint{6.763921in}{3.025279in}}%
\pgfusepath{stroke}%
\end{pgfscope}%
\begin{pgfscope}%
\definecolor{textcolor}{rgb}{0.150000,0.150000,0.150000}%
\pgfsetstrokecolor{textcolor}%
\pgfsetfillcolor{textcolor}%
\pgftext[x=0.289968in,y=2.972517in,left,base]{\color{textcolor}\rmfamily\fontsize{10.000000}{12.000000}\selectfont 40}%
\end{pgfscope}%
\begin{pgfscope}%
\definecolor{textcolor}{rgb}{0.150000,0.150000,0.150000}%
\pgfsetstrokecolor{textcolor}%
\pgfsetfillcolor{textcolor}%
\pgftext[x=0.234413in,y=1.842853in,,bottom,rotate=90.000000]{\color{textcolor}\rmfamily\fontsize{10.000000}{12.000000}\selectfont Standard Deviation}%
\end{pgfscope}%
\begin{pgfscope}%
\pgfpathrectangle{\pgfqpoint{0.563921in}{0.521603in}}{\pgfqpoint{6.200000in}{2.642500in}}%
\pgfusepath{clip}%
\pgfsetroundcap%
\pgfsetroundjoin%
\pgfsetlinewidth{1.505625pt}%
\definecolor{currentstroke}{rgb}{0.121569,0.466667,0.705882}%
\pgfsetstrokecolor{currentstroke}%
\pgfsetdash{}{0pt}%
\pgfpathmoveto{\pgfqpoint{0.845739in}{0.641717in}}%
\pgfpathlineto{\pgfqpoint{0.848317in}{0.658700in}}%
\pgfpathlineto{\pgfqpoint{0.850894in}{0.655601in}}%
\pgfpathlineto{\pgfqpoint{0.853471in}{0.657178in}}%
\pgfpathlineto{\pgfqpoint{0.861203in}{0.655876in}}%
\pgfpathlineto{\pgfqpoint{0.863780in}{0.658219in}}%
\pgfpathlineto{\pgfqpoint{0.866357in}{0.657038in}}%
\pgfpathlineto{\pgfqpoint{0.868934in}{0.657822in}}%
\pgfpathlineto{\pgfqpoint{0.871512in}{0.657520in}}%
\pgfpathlineto{\pgfqpoint{0.881820in}{0.657714in}}%
\pgfpathlineto{\pgfqpoint{0.884398in}{0.664412in}}%
\pgfpathlineto{\pgfqpoint{0.886975in}{0.674516in}}%
\pgfpathlineto{\pgfqpoint{0.889552in}{0.678938in}}%
\pgfpathlineto{\pgfqpoint{0.897284in}{0.681434in}}%
\pgfpathlineto{\pgfqpoint{0.899861in}{0.684693in}}%
\pgfpathlineto{\pgfqpoint{0.902438in}{0.689764in}}%
\pgfpathlineto{\pgfqpoint{0.905015in}{0.699797in}}%
\pgfpathlineto{\pgfqpoint{0.907593in}{0.705818in}}%
\pgfpathlineto{\pgfqpoint{0.920479in}{0.712517in}}%
\pgfpathlineto{\pgfqpoint{0.925633in}{0.716874in}}%
\pgfpathlineto{\pgfqpoint{0.933365in}{0.718223in}}%
\pgfpathlineto{\pgfqpoint{0.938519in}{0.721714in}}%
\pgfpathlineto{\pgfqpoint{0.941096in}{0.723123in}}%
\pgfpathlineto{\pgfqpoint{0.943674in}{0.722600in}}%
\pgfpathlineto{\pgfqpoint{0.959137in}{0.724998in}}%
\pgfpathlineto{\pgfqpoint{0.961714in}{0.725513in}}%
\pgfpathlineto{\pgfqpoint{0.974600in}{0.726569in}}%
\pgfpathlineto{\pgfqpoint{0.979754in}{0.728341in}}%
\pgfpathlineto{\pgfqpoint{0.992641in}{0.728982in}}%
\pgfpathlineto{\pgfqpoint{1.008104in}{0.727125in}}%
\pgfpathlineto{\pgfqpoint{1.015835in}{0.724771in}}%
\pgfpathlineto{\pgfqpoint{1.023567in}{0.724558in}}%
\pgfpathlineto{\pgfqpoint{1.028722in}{0.727060in}}%
\pgfpathlineto{\pgfqpoint{1.031299in}{0.730154in}}%
\pgfpathlineto{\pgfqpoint{1.033876in}{0.732095in}}%
\pgfpathlineto{\pgfqpoint{1.049339in}{0.736017in}}%
\pgfpathlineto{\pgfqpoint{1.051916in}{0.736007in}}%
\pgfpathlineto{\pgfqpoint{1.080266in}{0.738117in}}%
\pgfpathlineto{\pgfqpoint{1.085420in}{0.736871in}}%
\pgfpathlineto{\pgfqpoint{1.106038in}{0.735102in}}%
\pgfpathlineto{\pgfqpoint{1.116347in}{0.733921in}}%
\pgfpathlineto{\pgfqpoint{1.124078in}{0.732187in}}%
\pgfpathlineto{\pgfqpoint{1.136964in}{0.731630in}}%
\pgfpathlineto{\pgfqpoint{1.142119in}{0.732548in}}%
\pgfpathlineto{\pgfqpoint{1.152428in}{0.733541in}}%
\pgfpathlineto{\pgfqpoint{1.157582in}{0.734353in}}%
\pgfpathlineto{\pgfqpoint{1.170468in}{0.733340in}}%
\pgfpathlineto{\pgfqpoint{1.178200in}{0.731841in}}%
\pgfpathlineto{\pgfqpoint{1.193663in}{0.731199in}}%
\pgfpathlineto{\pgfqpoint{1.196240in}{0.731943in}}%
\pgfpathlineto{\pgfqpoint{1.214281in}{0.731766in}}%
\pgfpathlineto{\pgfqpoint{1.229744in}{0.731411in}}%
\pgfpathlineto{\pgfqpoint{1.232321in}{0.732498in}}%
\pgfpathlineto{\pgfqpoint{1.240053in}{0.733725in}}%
\pgfpathlineto{\pgfqpoint{1.242630in}{0.734979in}}%
\pgfpathlineto{\pgfqpoint{1.258093in}{0.733729in}}%
\pgfpathlineto{\pgfqpoint{1.268402in}{0.732423in}}%
\pgfpathlineto{\pgfqpoint{1.283866in}{0.731618in}}%
\pgfpathlineto{\pgfqpoint{1.286443in}{0.731290in}}%
\pgfpathlineto{\pgfqpoint{1.304483in}{0.730991in}}%
\pgfpathlineto{\pgfqpoint{1.312215in}{0.731656in}}%
\pgfpathlineto{\pgfqpoint{1.314792in}{0.732580in}}%
\pgfpathlineto{\pgfqpoint{1.322524in}{0.733766in}}%
\pgfpathlineto{\pgfqpoint{1.335410in}{0.734017in}}%
\pgfpathlineto{\pgfqpoint{1.340564in}{0.733473in}}%
\pgfpathlineto{\pgfqpoint{1.350873in}{0.733751in}}%
\pgfpathlineto{\pgfqpoint{1.356027in}{0.736834in}}%
\pgfpathlineto{\pgfqpoint{1.358605in}{0.737646in}}%
\pgfpathlineto{\pgfqpoint{1.371491in}{0.738087in}}%
\pgfpathlineto{\pgfqpoint{1.374068in}{0.739209in}}%
\pgfpathlineto{\pgfqpoint{1.376645in}{0.741173in}}%
\pgfpathlineto{\pgfqpoint{1.384377in}{0.742688in}}%
\pgfpathlineto{\pgfqpoint{1.389531in}{0.745424in}}%
\pgfpathlineto{\pgfqpoint{1.392108in}{0.745986in}}%
\pgfpathlineto{\pgfqpoint{1.394686in}{0.747591in}}%
\pgfpathlineto{\pgfqpoint{1.402417in}{0.748914in}}%
\pgfpathlineto{\pgfqpoint{1.412726in}{0.754756in}}%
\pgfpathlineto{\pgfqpoint{1.420458in}{0.756498in}}%
\pgfpathlineto{\pgfqpoint{1.425612in}{0.759785in}}%
\pgfpathlineto{\pgfqpoint{1.430767in}{0.765275in}}%
\pgfpathlineto{\pgfqpoint{1.438498in}{0.767742in}}%
\pgfpathlineto{\pgfqpoint{1.448807in}{0.774412in}}%
\pgfpathlineto{\pgfqpoint{1.456539in}{0.775953in}}%
\pgfpathlineto{\pgfqpoint{1.466848in}{0.780839in}}%
\pgfpathlineto{\pgfqpoint{1.477156in}{0.781629in}}%
\pgfpathlineto{\pgfqpoint{1.479734in}{0.782426in}}%
\pgfpathlineto{\pgfqpoint{1.484888in}{0.785367in}}%
\pgfpathlineto{\pgfqpoint{1.497774in}{0.786314in}}%
\pgfpathlineto{\pgfqpoint{1.500351in}{0.787103in}}%
\pgfpathlineto{\pgfqpoint{1.502929in}{0.788951in}}%
\pgfpathlineto{\pgfqpoint{1.510660in}{0.790605in}}%
\pgfpathlineto{\pgfqpoint{1.520969in}{0.795969in}}%
\pgfpathlineto{\pgfqpoint{1.531278in}{0.798212in}}%
\pgfpathlineto{\pgfqpoint{1.539010in}{0.800370in}}%
\pgfpathlineto{\pgfqpoint{1.546741in}{0.801352in}}%
\pgfpathlineto{\pgfqpoint{1.554473in}{0.805040in}}%
\pgfpathlineto{\pgfqpoint{1.557050in}{0.806751in}}%
\pgfpathlineto{\pgfqpoint{1.575091in}{0.811491in}}%
\pgfpathlineto{\pgfqpoint{1.582822in}{0.811984in}}%
\pgfpathlineto{\pgfqpoint{1.593131in}{0.816090in}}%
\pgfpathlineto{\pgfqpoint{1.603440in}{0.815990in}}%
\pgfpathlineto{\pgfqpoint{1.611172in}{0.814736in}}%
\pgfpathlineto{\pgfqpoint{1.626635in}{0.813970in}}%
\pgfpathlineto{\pgfqpoint{1.629212in}{0.813594in}}%
\pgfpathlineto{\pgfqpoint{1.644675in}{0.812425in}}%
\pgfpathlineto{\pgfqpoint{1.647253in}{0.812050in}}%
\pgfpathlineto{\pgfqpoint{1.660139in}{0.810978in}}%
\pgfpathlineto{\pgfqpoint{1.665293in}{0.810218in}}%
\pgfpathlineto{\pgfqpoint{1.716837in}{0.808676in}}%
\pgfpathlineto{\pgfqpoint{1.729723in}{0.809852in}}%
\pgfpathlineto{\pgfqpoint{1.737455in}{0.811119in}}%
\pgfpathlineto{\pgfqpoint{1.745187in}{0.811641in}}%
\pgfpathlineto{\pgfqpoint{1.755495in}{0.814283in}}%
\pgfpathlineto{\pgfqpoint{1.781268in}{0.816010in}}%
\pgfpathlineto{\pgfqpoint{1.788999in}{0.817960in}}%
\pgfpathlineto{\pgfqpoint{1.791576in}{0.819160in}}%
\pgfpathlineto{\pgfqpoint{1.799308in}{0.820379in}}%
\pgfpathlineto{\pgfqpoint{1.804462in}{0.823129in}}%
\pgfpathlineto{\pgfqpoint{1.809617in}{0.826288in}}%
\pgfpathlineto{\pgfqpoint{1.817349in}{0.828031in}}%
\pgfpathlineto{\pgfqpoint{1.825080in}{0.833800in}}%
\pgfpathlineto{\pgfqpoint{1.827657in}{0.836154in}}%
\pgfpathlineto{\pgfqpoint{1.837966in}{0.838751in}}%
\pgfpathlineto{\pgfqpoint{1.845698in}{0.847037in}}%
\pgfpathlineto{\pgfqpoint{1.853430in}{0.850028in}}%
\pgfpathlineto{\pgfqpoint{1.858584in}{0.856477in}}%
\pgfpathlineto{\pgfqpoint{1.863738in}{0.862285in}}%
\pgfpathlineto{\pgfqpoint{1.871470in}{0.864917in}}%
\pgfpathlineto{\pgfqpoint{1.879202in}{0.874458in}}%
\pgfpathlineto{\pgfqpoint{1.881779in}{0.877728in}}%
\pgfpathlineto{\pgfqpoint{1.889510in}{0.880885in}}%
\pgfpathlineto{\pgfqpoint{1.899819in}{0.894540in}}%
\pgfpathlineto{\pgfqpoint{1.910128in}{0.898382in}}%
\pgfpathlineto{\pgfqpoint{1.917860in}{0.907689in}}%
\pgfpathlineto{\pgfqpoint{1.925591in}{0.909998in}}%
\pgfpathlineto{\pgfqpoint{1.930746in}{0.915511in}}%
\pgfpathlineto{\pgfqpoint{1.935900in}{0.921600in}}%
\pgfpathlineto{\pgfqpoint{1.943632in}{0.924257in}}%
\pgfpathlineto{\pgfqpoint{1.953941in}{0.937004in}}%
\pgfpathlineto{\pgfqpoint{1.961672in}{0.940440in}}%
\pgfpathlineto{\pgfqpoint{1.971981in}{0.953094in}}%
\pgfpathlineto{\pgfqpoint{1.979713in}{0.955958in}}%
\pgfpathlineto{\pgfqpoint{1.990022in}{0.967013in}}%
\pgfpathlineto{\pgfqpoint{1.997753in}{0.969468in}}%
\pgfpathlineto{\pgfqpoint{2.005485in}{0.977360in}}%
\pgfpathlineto{\pgfqpoint{2.015794in}{0.979781in}}%
\pgfpathlineto{\pgfqpoint{2.023526in}{0.987149in}}%
\pgfpathlineto{\pgfqpoint{2.026103in}{0.989415in}}%
\pgfpathlineto{\pgfqpoint{2.033834in}{0.991626in}}%
\pgfpathlineto{\pgfqpoint{2.038989in}{0.996708in}}%
\pgfpathlineto{\pgfqpoint{2.044143in}{1.002423in}}%
\pgfpathlineto{\pgfqpoint{2.051875in}{1.004399in}}%
\pgfpathlineto{\pgfqpoint{2.059606in}{1.009888in}}%
\pgfpathlineto{\pgfqpoint{2.062184in}{1.011699in}}%
\pgfpathlineto{\pgfqpoint{2.069915in}{1.013506in}}%
\pgfpathlineto{\pgfqpoint{2.075070in}{1.018226in}}%
\pgfpathlineto{\pgfqpoint{2.080224in}{1.020745in}}%
\pgfpathlineto{\pgfqpoint{2.087956in}{1.021834in}}%
\pgfpathlineto{\pgfqpoint{2.095687in}{1.026012in}}%
\pgfpathlineto{\pgfqpoint{2.098265in}{1.028180in}}%
\pgfpathlineto{\pgfqpoint{2.105996in}{1.030307in}}%
\pgfpathlineto{\pgfqpoint{2.113728in}{1.037323in}}%
\pgfpathlineto{\pgfqpoint{2.116305in}{1.040177in}}%
\pgfpathlineto{\pgfqpoint{2.124037in}{1.042983in}}%
\pgfpathlineto{\pgfqpoint{2.134346in}{1.054545in}}%
\pgfpathlineto{\pgfqpoint{2.142077in}{1.057500in}}%
\pgfpathlineto{\pgfqpoint{2.149809in}{1.065793in}}%
\pgfpathlineto{\pgfqpoint{2.152386in}{1.068257in}}%
\pgfpathlineto{\pgfqpoint{2.162695in}{1.071116in}}%
\pgfpathlineto{\pgfqpoint{2.170427in}{1.078742in}}%
\pgfpathlineto{\pgfqpoint{2.178158in}{1.081084in}}%
\pgfpathlineto{\pgfqpoint{2.183313in}{1.084874in}}%
\pgfpathlineto{\pgfqpoint{2.185890in}{1.086540in}}%
\pgfpathlineto{\pgfqpoint{2.188467in}{1.088887in}}%
\pgfpathlineto{\pgfqpoint{2.196199in}{1.091101in}}%
\pgfpathlineto{\pgfqpoint{2.206508in}{1.099103in}}%
\pgfpathlineto{\pgfqpoint{2.214239in}{1.101466in}}%
\pgfpathlineto{\pgfqpoint{2.219394in}{1.106293in}}%
\pgfpathlineto{\pgfqpoint{2.224548in}{1.109252in}}%
\pgfpathlineto{\pgfqpoint{2.232280in}{1.110295in}}%
\pgfpathlineto{\pgfqpoint{2.237434in}{1.113014in}}%
\pgfpathlineto{\pgfqpoint{2.242589in}{1.116052in}}%
\pgfpathlineto{\pgfqpoint{2.250320in}{1.117414in}}%
\pgfpathlineto{\pgfqpoint{2.260629in}{1.121831in}}%
\pgfpathlineto{\pgfqpoint{2.268361in}{1.123822in}}%
\pgfpathlineto{\pgfqpoint{2.278670in}{1.133675in}}%
\pgfpathlineto{\pgfqpoint{2.286401in}{1.136247in}}%
\pgfpathlineto{\pgfqpoint{2.296710in}{1.146498in}}%
\pgfpathlineto{\pgfqpoint{2.304442in}{1.149278in}}%
\pgfpathlineto{\pgfqpoint{2.314751in}{1.160362in}}%
\pgfpathlineto{\pgfqpoint{2.322482in}{1.163013in}}%
\pgfpathlineto{\pgfqpoint{2.332791in}{1.174477in}}%
\pgfpathlineto{\pgfqpoint{2.340523in}{1.177277in}}%
\pgfpathlineto{\pgfqpoint{2.350832in}{1.188218in}}%
\pgfpathlineto{\pgfqpoint{2.358563in}{1.190929in}}%
\pgfpathlineto{\pgfqpoint{2.363718in}{1.196094in}}%
\pgfpathlineto{\pgfqpoint{2.368872in}{1.199933in}}%
\pgfpathlineto{\pgfqpoint{2.376604in}{1.201747in}}%
\pgfpathlineto{\pgfqpoint{2.386912in}{1.208141in}}%
\pgfpathlineto{\pgfqpoint{2.394644in}{1.209683in}}%
\pgfpathlineto{\pgfqpoint{2.404953in}{1.214649in}}%
\pgfpathlineto{\pgfqpoint{2.415262in}{1.215858in}}%
\pgfpathlineto{\pgfqpoint{2.422993in}{1.220383in}}%
\pgfpathlineto{\pgfqpoint{2.430725in}{1.222281in}}%
\pgfpathlineto{\pgfqpoint{2.441034in}{1.231063in}}%
\pgfpathlineto{\pgfqpoint{2.448766in}{1.233412in}}%
\pgfpathlineto{\pgfqpoint{2.459074in}{1.243965in}}%
\pgfpathlineto{\pgfqpoint{2.466806in}{1.246609in}}%
\pgfpathlineto{\pgfqpoint{2.477115in}{1.256172in}}%
\pgfpathlineto{\pgfqpoint{2.484847in}{1.258227in}}%
\pgfpathlineto{\pgfqpoint{2.495155in}{1.265972in}}%
\pgfpathlineto{\pgfqpoint{2.502887in}{1.267732in}}%
\pgfpathlineto{\pgfqpoint{2.510619in}{1.272465in}}%
\pgfpathlineto{\pgfqpoint{2.513196in}{1.274580in}}%
\pgfpathlineto{\pgfqpoint{2.520928in}{1.276824in}}%
\pgfpathlineto{\pgfqpoint{2.526082in}{1.280761in}}%
\pgfpathlineto{\pgfqpoint{2.531236in}{1.285663in}}%
\pgfpathlineto{\pgfqpoint{2.538968in}{1.288189in}}%
\pgfpathlineto{\pgfqpoint{2.549277in}{1.298399in}}%
\pgfpathlineto{\pgfqpoint{2.557009in}{1.301151in}}%
\pgfpathlineto{\pgfqpoint{2.567317in}{1.312301in}}%
\pgfpathlineto{\pgfqpoint{2.575049in}{1.315202in}}%
\pgfpathlineto{\pgfqpoint{2.585358in}{1.326996in}}%
\pgfpathlineto{\pgfqpoint{2.593089in}{1.330097in}}%
\pgfpathlineto{\pgfqpoint{2.603398in}{1.343264in}}%
\pgfpathlineto{\pgfqpoint{2.611130in}{1.346687in}}%
\pgfpathlineto{\pgfqpoint{2.621439in}{1.360419in}}%
\pgfpathlineto{\pgfqpoint{2.629170in}{1.364046in}}%
\pgfpathlineto{\pgfqpoint{2.634325in}{1.371986in}}%
\pgfpathlineto{\pgfqpoint{2.639479in}{1.376051in}}%
\pgfpathlineto{\pgfqpoint{2.647211in}{1.378627in}}%
\pgfpathlineto{\pgfqpoint{2.657520in}{1.388132in}}%
\pgfpathlineto{\pgfqpoint{2.665251in}{1.390745in}}%
\pgfpathlineto{\pgfqpoint{2.672983in}{1.397367in}}%
\pgfpathlineto{\pgfqpoint{2.675560in}{1.399410in}}%
\pgfpathlineto{\pgfqpoint{2.683292in}{1.401679in}}%
\pgfpathlineto{\pgfqpoint{2.685869in}{1.404740in}}%
\pgfpathlineto{\pgfqpoint{2.693601in}{1.417294in}}%
\pgfpathlineto{\pgfqpoint{2.701332in}{1.421488in}}%
\pgfpathlineto{\pgfqpoint{2.703910in}{1.425671in}}%
\pgfpathlineto{\pgfqpoint{2.709064in}{1.430137in}}%
\pgfpathlineto{\pgfqpoint{2.711641in}{1.434823in}}%
\pgfpathlineto{\pgfqpoint{2.719373in}{1.439459in}}%
\pgfpathlineto{\pgfqpoint{2.721950in}{1.444253in}}%
\pgfpathlineto{\pgfqpoint{2.727105in}{1.448421in}}%
\pgfpathlineto{\pgfqpoint{2.729682in}{1.452614in}}%
\pgfpathlineto{\pgfqpoint{2.737413in}{1.456544in}}%
\pgfpathlineto{\pgfqpoint{2.745145in}{1.467506in}}%
\pgfpathlineto{\pgfqpoint{2.747722in}{1.470895in}}%
\pgfpathlineto{\pgfqpoint{2.755454in}{1.473906in}}%
\pgfpathlineto{\pgfqpoint{2.765763in}{1.488379in}}%
\pgfpathlineto{\pgfqpoint{2.776072in}{1.491689in}}%
\pgfpathlineto{\pgfqpoint{2.781226in}{1.497587in}}%
\pgfpathlineto{\pgfqpoint{2.783803in}{1.499431in}}%
\pgfpathlineto{\pgfqpoint{2.791535in}{1.501034in}}%
\pgfpathlineto{\pgfqpoint{2.801844in}{1.507354in}}%
\pgfpathlineto{\pgfqpoint{2.809575in}{1.508102in}}%
\pgfpathlineto{\pgfqpoint{2.817307in}{1.511937in}}%
\pgfpathlineto{\pgfqpoint{2.819884in}{1.513618in}}%
\pgfpathlineto{\pgfqpoint{2.827616in}{1.515174in}}%
\pgfpathlineto{\pgfqpoint{2.837925in}{1.522484in}}%
\pgfpathlineto{\pgfqpoint{2.848234in}{1.524451in}}%
\pgfpathlineto{\pgfqpoint{2.855965in}{1.529924in}}%
\pgfpathlineto{\pgfqpoint{2.863697in}{1.531877in}}%
\pgfpathlineto{\pgfqpoint{2.874006in}{1.540551in}}%
\pgfpathlineto{\pgfqpoint{2.881737in}{1.542394in}}%
\pgfpathlineto{\pgfqpoint{2.892046in}{1.550578in}}%
\pgfpathlineto{\pgfqpoint{2.899778in}{1.552544in}}%
\pgfpathlineto{\pgfqpoint{2.907509in}{1.557525in}}%
\pgfpathlineto{\pgfqpoint{2.910087in}{1.558814in}}%
\pgfpathlineto{\pgfqpoint{2.917818in}{1.560471in}}%
\pgfpathlineto{\pgfqpoint{2.928127in}{1.566996in}}%
\pgfpathlineto{\pgfqpoint{2.935859in}{1.568589in}}%
\pgfpathlineto{\pgfqpoint{2.946168in}{1.575510in}}%
\pgfpathlineto{\pgfqpoint{2.953899in}{1.577545in}}%
\pgfpathlineto{\pgfqpoint{2.964208in}{1.585789in}}%
\pgfpathlineto{\pgfqpoint{2.971940in}{1.587511in}}%
\pgfpathlineto{\pgfqpoint{2.982249in}{1.594143in}}%
\pgfpathlineto{\pgfqpoint{2.989980in}{1.595586in}}%
\pgfpathlineto{\pgfqpoint{2.995135in}{1.599136in}}%
\pgfpathlineto{\pgfqpoint{2.997712in}{1.601257in}}%
\pgfpathlineto{\pgfqpoint{3.008021in}{1.603509in}}%
\pgfpathlineto{\pgfqpoint{3.015752in}{1.609736in}}%
\pgfpathlineto{\pgfqpoint{3.018330in}{1.611559in}}%
\pgfpathlineto{\pgfqpoint{3.026061in}{1.613469in}}%
\pgfpathlineto{\pgfqpoint{3.031216in}{1.617589in}}%
\pgfpathlineto{\pgfqpoint{3.036370in}{1.622345in}}%
\pgfpathlineto{\pgfqpoint{3.044102in}{1.624712in}}%
\pgfpathlineto{\pgfqpoint{3.051833in}{1.631571in}}%
\pgfpathlineto{\pgfqpoint{3.054411in}{1.633966in}}%
\pgfpathlineto{\pgfqpoint{3.062142in}{1.636595in}}%
\pgfpathlineto{\pgfqpoint{3.072451in}{1.645965in}}%
\pgfpathlineto{\pgfqpoint{3.080183in}{1.648270in}}%
\pgfpathlineto{\pgfqpoint{3.090491in}{1.656975in}}%
\pgfpathlineto{\pgfqpoint{3.100800in}{1.659273in}}%
\pgfpathlineto{\pgfqpoint{3.108532in}{1.666372in}}%
\pgfpathlineto{\pgfqpoint{3.116264in}{1.668722in}}%
\pgfpathlineto{\pgfqpoint{3.126572in}{1.678602in}}%
\pgfpathlineto{\pgfqpoint{3.134304in}{1.681342in}}%
\pgfpathlineto{\pgfqpoint{3.142036in}{1.688802in}}%
\pgfpathlineto{\pgfqpoint{3.144613in}{1.691108in}}%
\pgfpathlineto{\pgfqpoint{3.152345in}{1.693385in}}%
\pgfpathlineto{\pgfqpoint{3.162653in}{1.702981in}}%
\pgfpathlineto{\pgfqpoint{3.170385in}{1.705268in}}%
\pgfpathlineto{\pgfqpoint{3.180694in}{1.713851in}}%
\pgfpathlineto{\pgfqpoint{3.188426in}{1.715898in}}%
\pgfpathlineto{\pgfqpoint{3.196157in}{1.722774in}}%
\pgfpathlineto{\pgfqpoint{3.206466in}{1.724998in}}%
\pgfpathlineto{\pgfqpoint{3.216775in}{1.733346in}}%
\pgfpathlineto{\pgfqpoint{3.224507in}{1.735481in}}%
\pgfpathlineto{\pgfqpoint{3.234815in}{1.743768in}}%
\pgfpathlineto{\pgfqpoint{3.242547in}{1.745701in}}%
\pgfpathlineto{\pgfqpoint{3.252856in}{1.753671in}}%
\pgfpathlineto{\pgfqpoint{3.260588in}{1.755693in}}%
\pgfpathlineto{\pgfqpoint{3.268319in}{1.760537in}}%
\pgfpathlineto{\pgfqpoint{3.270896in}{1.761756in}}%
\pgfpathlineto{\pgfqpoint{3.278628in}{1.763044in}}%
\pgfpathlineto{\pgfqpoint{3.288937in}{1.767627in}}%
\pgfpathlineto{\pgfqpoint{3.296668in}{1.768837in}}%
\pgfpathlineto{\pgfqpoint{3.306977in}{1.774142in}}%
\pgfpathlineto{\pgfqpoint{3.314709in}{1.775748in}}%
\pgfpathlineto{\pgfqpoint{3.325018in}{1.782505in}}%
\pgfpathlineto{\pgfqpoint{3.332749in}{1.784231in}}%
\pgfpathlineto{\pgfqpoint{3.343058in}{1.790674in}}%
\pgfpathlineto{\pgfqpoint{3.353367in}{1.792260in}}%
\pgfpathlineto{\pgfqpoint{3.361099in}{1.796801in}}%
\pgfpathlineto{\pgfqpoint{3.368830in}{1.798420in}}%
\pgfpathlineto{\pgfqpoint{3.379139in}{1.804474in}}%
\pgfpathlineto{\pgfqpoint{3.386871in}{1.805971in}}%
\pgfpathlineto{\pgfqpoint{3.397180in}{1.812586in}}%
\pgfpathlineto{\pgfqpoint{3.404911in}{1.814173in}}%
\pgfpathlineto{\pgfqpoint{3.415220in}{1.819312in}}%
\pgfpathlineto{\pgfqpoint{3.422952in}{1.820408in}}%
\pgfpathlineto{\pgfqpoint{3.428106in}{1.822182in}}%
\pgfpathlineto{\pgfqpoint{3.433261in}{1.823707in}}%
\pgfpathlineto{\pgfqpoint{3.443570in}{1.825122in}}%
\pgfpathlineto{\pgfqpoint{3.448724in}{1.826720in}}%
\pgfpathlineto{\pgfqpoint{3.479651in}{1.829627in}}%
\pgfpathlineto{\pgfqpoint{3.482228in}{1.830257in}}%
\pgfpathlineto{\pgfqpoint{3.487382in}{1.833254in}}%
\pgfpathlineto{\pgfqpoint{3.495114in}{1.835079in}}%
\pgfpathlineto{\pgfqpoint{3.505423in}{1.843526in}}%
\pgfpathlineto{\pgfqpoint{3.513154in}{1.845774in}}%
\pgfpathlineto{\pgfqpoint{3.523463in}{1.856025in}}%
\pgfpathlineto{\pgfqpoint{3.531195in}{1.858804in}}%
\pgfpathlineto{\pgfqpoint{3.541504in}{1.869885in}}%
\pgfpathlineto{\pgfqpoint{3.549235in}{1.872699in}}%
\pgfpathlineto{\pgfqpoint{3.559544in}{1.884664in}}%
\pgfpathlineto{\pgfqpoint{3.567276in}{1.887662in}}%
\pgfpathlineto{\pgfqpoint{3.572430in}{1.893076in}}%
\pgfpathlineto{\pgfqpoint{3.577585in}{1.896032in}}%
\pgfpathlineto{\pgfqpoint{3.585316in}{1.898676in}}%
\pgfpathlineto{\pgfqpoint{3.595625in}{1.911225in}}%
\pgfpathlineto{\pgfqpoint{3.603357in}{1.914156in}}%
\pgfpathlineto{\pgfqpoint{3.611088in}{1.922138in}}%
\pgfpathlineto{\pgfqpoint{3.613666in}{1.924433in}}%
\pgfpathlineto{\pgfqpoint{3.621397in}{1.926673in}}%
\pgfpathlineto{\pgfqpoint{3.626552in}{1.931922in}}%
\pgfpathlineto{\pgfqpoint{3.631706in}{1.938739in}}%
\pgfpathlineto{\pgfqpoint{3.639438in}{1.942404in}}%
\pgfpathlineto{\pgfqpoint{3.644592in}{1.949523in}}%
\pgfpathlineto{\pgfqpoint{3.649747in}{1.952934in}}%
\pgfpathlineto{\pgfqpoint{3.657478in}{1.956386in}}%
\pgfpathlineto{\pgfqpoint{3.662633in}{1.962703in}}%
\pgfpathlineto{\pgfqpoint{3.667787in}{1.965671in}}%
\pgfpathlineto{\pgfqpoint{3.675519in}{1.968093in}}%
\pgfpathlineto{\pgfqpoint{3.680673in}{1.972577in}}%
\pgfpathlineto{\pgfqpoint{3.685828in}{1.977910in}}%
\pgfpathlineto{\pgfqpoint{3.693559in}{1.980286in}}%
\pgfpathlineto{\pgfqpoint{3.703868in}{1.989496in}}%
\pgfpathlineto{\pgfqpoint{3.714177in}{1.991939in}}%
\pgfpathlineto{\pgfqpoint{3.721909in}{2.000056in}}%
\pgfpathlineto{\pgfqpoint{3.729640in}{2.002728in}}%
\pgfpathlineto{\pgfqpoint{3.739949in}{2.013086in}}%
\pgfpathlineto{\pgfqpoint{3.747681in}{2.015678in}}%
\pgfpathlineto{\pgfqpoint{3.757990in}{2.026615in}}%
\pgfpathlineto{\pgfqpoint{3.765721in}{2.029161in}}%
\pgfpathlineto{\pgfqpoint{3.776030in}{2.039926in}}%
\pgfpathlineto{\pgfqpoint{3.786339in}{2.042789in}}%
\pgfpathlineto{\pgfqpoint{3.794070in}{2.051498in}}%
\pgfpathlineto{\pgfqpoint{3.801802in}{2.054533in}}%
\pgfpathlineto{\pgfqpoint{3.812111in}{2.066582in}}%
\pgfpathlineto{\pgfqpoint{3.819843in}{2.069735in}}%
\pgfpathlineto{\pgfqpoint{3.827574in}{2.077884in}}%
\pgfpathlineto{\pgfqpoint{3.830151in}{2.080133in}}%
\pgfpathlineto{\pgfqpoint{3.837883in}{2.082620in}}%
\pgfpathlineto{\pgfqpoint{3.848192in}{2.090630in}}%
\pgfpathlineto{\pgfqpoint{3.855924in}{2.093010in}}%
\pgfpathlineto{\pgfqpoint{3.866232in}{2.102112in}}%
\pgfpathlineto{\pgfqpoint{3.873964in}{2.104395in}}%
\pgfpathlineto{\pgfqpoint{3.879118in}{2.108528in}}%
\pgfpathlineto{\pgfqpoint{3.884273in}{2.112252in}}%
\pgfpathlineto{\pgfqpoint{3.892005in}{2.114447in}}%
\pgfpathlineto{\pgfqpoint{3.899736in}{2.120020in}}%
\pgfpathlineto{\pgfqpoint{3.910045in}{2.122087in}}%
\pgfpathlineto{\pgfqpoint{3.920354in}{2.130723in}}%
\pgfpathlineto{\pgfqpoint{3.928086in}{2.132754in}}%
\pgfpathlineto{\pgfqpoint{3.938394in}{2.140376in}}%
\pgfpathlineto{\pgfqpoint{3.946126in}{2.142186in}}%
\pgfpathlineto{\pgfqpoint{3.953858in}{2.146995in}}%
\pgfpathlineto{\pgfqpoint{3.956435in}{2.148184in}}%
\pgfpathlineto{\pgfqpoint{3.964167in}{2.149268in}}%
\pgfpathlineto{\pgfqpoint{3.974475in}{2.153286in}}%
\pgfpathlineto{\pgfqpoint{3.982207in}{2.154429in}}%
\pgfpathlineto{\pgfqpoint{3.992516in}{2.158796in}}%
\pgfpathlineto{\pgfqpoint{4.000247in}{2.159981in}}%
\pgfpathlineto{\pgfqpoint{4.007979in}{2.163958in}}%
\pgfpathlineto{\pgfqpoint{4.010556in}{2.165443in}}%
\pgfpathlineto{\pgfqpoint{4.018288in}{2.166873in}}%
\pgfpathlineto{\pgfqpoint{4.028597in}{2.172459in}}%
\pgfpathlineto{\pgfqpoint{4.038906in}{2.173619in}}%
\pgfpathlineto{\pgfqpoint{4.046637in}{2.177261in}}%
\pgfpathlineto{\pgfqpoint{4.054369in}{2.178330in}}%
\pgfpathlineto{\pgfqpoint{4.064678in}{2.182422in}}%
\pgfpathlineto{\pgfqpoint{4.072409in}{2.183241in}}%
\pgfpathlineto{\pgfqpoint{4.082718in}{2.187121in}}%
\pgfpathlineto{\pgfqpoint{4.093027in}{2.188618in}}%
\pgfpathlineto{\pgfqpoint{4.100759in}{2.191422in}}%
\pgfpathlineto{\pgfqpoint{4.108490in}{2.192446in}}%
\pgfpathlineto{\pgfqpoint{4.116222in}{2.194981in}}%
\pgfpathlineto{\pgfqpoint{4.118799in}{2.195753in}}%
\pgfpathlineto{\pgfqpoint{4.129108in}{2.196805in}}%
\pgfpathlineto{\pgfqpoint{4.134263in}{2.198054in}}%
\pgfpathlineto{\pgfqpoint{4.147149in}{2.199244in}}%
\pgfpathlineto{\pgfqpoint{4.170343in}{2.203350in}}%
\pgfpathlineto{\pgfqpoint{4.172921in}{2.204016in}}%
\pgfpathlineto{\pgfqpoint{4.183230in}{2.205294in}}%
\pgfpathlineto{\pgfqpoint{4.188384in}{2.205975in}}%
\pgfpathlineto{\pgfqpoint{4.245083in}{2.208071in}}%
\pgfpathlineto{\pgfqpoint{4.260546in}{2.207880in}}%
\pgfpathlineto{\pgfqpoint{4.281164in}{2.206115in}}%
\pgfpathlineto{\pgfqpoint{4.291472in}{2.205406in}}%
\pgfpathlineto{\pgfqpoint{4.299204in}{2.204415in}}%
\pgfpathlineto{\pgfqpoint{4.312090in}{2.203866in}}%
\pgfpathlineto{\pgfqpoint{4.317245in}{2.203206in}}%
\pgfpathlineto{\pgfqpoint{4.332708in}{2.202238in}}%
\pgfpathlineto{\pgfqpoint{4.335285in}{2.201844in}}%
\pgfpathlineto{\pgfqpoint{4.345594in}{2.201009in}}%
\pgfpathlineto{\pgfqpoint{4.353326in}{2.199665in}}%
\pgfpathlineto{\pgfqpoint{4.368789in}{2.198263in}}%
\pgfpathlineto{\pgfqpoint{4.371366in}{2.198028in}}%
\pgfpathlineto{\pgfqpoint{4.399715in}{2.198265in}}%
\pgfpathlineto{\pgfqpoint{4.420333in}{2.198571in}}%
\pgfpathlineto{\pgfqpoint{4.425488in}{2.199567in}}%
\pgfpathlineto{\pgfqpoint{4.435796in}{2.200653in}}%
\pgfpathlineto{\pgfqpoint{4.443528in}{2.202470in}}%
\pgfpathlineto{\pgfqpoint{4.453837in}{2.203916in}}%
\pgfpathlineto{\pgfqpoint{4.461569in}{2.205983in}}%
\pgfpathlineto{\pgfqpoint{4.471877in}{2.207112in}}%
\pgfpathlineto{\pgfqpoint{4.479609in}{2.208666in}}%
\pgfpathlineto{\pgfqpoint{4.489918in}{2.209767in}}%
\pgfpathlineto{\pgfqpoint{4.497649in}{2.211751in}}%
\pgfpathlineto{\pgfqpoint{4.507958in}{2.213045in}}%
\pgfpathlineto{\pgfqpoint{4.510536in}{2.213642in}}%
\pgfpathlineto{\pgfqpoint{4.549194in}{2.218805in}}%
\pgfpathlineto{\pgfqpoint{4.559503in}{2.219716in}}%
\pgfpathlineto{\pgfqpoint{4.585275in}{2.219751in}}%
\pgfpathlineto{\pgfqpoint{4.618778in}{2.219838in}}%
\pgfpathlineto{\pgfqpoint{4.623933in}{2.219046in}}%
\pgfpathlineto{\pgfqpoint{4.634242in}{2.218248in}}%
\pgfpathlineto{\pgfqpoint{4.641973in}{2.216932in}}%
\pgfpathlineto{\pgfqpoint{4.654859in}{2.215929in}}%
\pgfpathlineto{\pgfqpoint{4.660014in}{2.214991in}}%
\pgfpathlineto{\pgfqpoint{4.678054in}{2.214024in}}%
\pgfpathlineto{\pgfqpoint{4.696095in}{2.214471in}}%
\pgfpathlineto{\pgfqpoint{4.711558in}{2.215300in}}%
\pgfpathlineto{\pgfqpoint{4.714135in}{2.215597in}}%
\pgfpathlineto{\pgfqpoint{4.727021in}{2.216453in}}%
\pgfpathlineto{\pgfqpoint{4.732176in}{2.217367in}}%
\pgfpathlineto{\pgfqpoint{4.745062in}{2.218798in}}%
\pgfpathlineto{\pgfqpoint{4.750216in}{2.219952in}}%
\pgfpathlineto{\pgfqpoint{4.760525in}{2.221038in}}%
\pgfpathlineto{\pgfqpoint{4.768257in}{2.222925in}}%
\pgfpathlineto{\pgfqpoint{4.778566in}{2.224272in}}%
\pgfpathlineto{\pgfqpoint{4.786297in}{2.226311in}}%
\pgfpathlineto{\pgfqpoint{4.796606in}{2.227865in}}%
\pgfpathlineto{\pgfqpoint{4.804338in}{2.230584in}}%
\pgfpathlineto{\pgfqpoint{4.812069in}{2.231545in}}%
\pgfpathlineto{\pgfqpoint{4.819801in}{2.234326in}}%
\pgfpathlineto{\pgfqpoint{4.830110in}{2.235378in}}%
\pgfpathlineto{\pgfqpoint{4.840419in}{2.239544in}}%
\pgfpathlineto{\pgfqpoint{4.848150in}{2.240576in}}%
\pgfpathlineto{\pgfqpoint{4.858459in}{2.244706in}}%
\pgfpathlineto{\pgfqpoint{4.866191in}{2.245723in}}%
\pgfpathlineto{\pgfqpoint{4.876500in}{2.250301in}}%
\pgfpathlineto{\pgfqpoint{4.884231in}{2.251524in}}%
\pgfpathlineto{\pgfqpoint{4.894540in}{2.256162in}}%
\pgfpathlineto{\pgfqpoint{4.904849in}{2.258178in}}%
\pgfpathlineto{\pgfqpoint{4.912581in}{2.261179in}}%
\pgfpathlineto{\pgfqpoint{4.920312in}{2.262242in}}%
\pgfpathlineto{\pgfqpoint{4.930621in}{2.266317in}}%
\pgfpathlineto{\pgfqpoint{4.938353in}{2.267366in}}%
\pgfpathlineto{\pgfqpoint{4.948662in}{2.271846in}}%
\pgfpathlineto{\pgfqpoint{4.956393in}{2.272944in}}%
\pgfpathlineto{\pgfqpoint{4.966702in}{2.276512in}}%
\pgfpathlineto{\pgfqpoint{4.974434in}{2.277406in}}%
\pgfpathlineto{\pgfqpoint{4.984743in}{2.281842in}}%
\pgfpathlineto{\pgfqpoint{4.995051in}{2.282874in}}%
\pgfpathlineto{\pgfqpoint{5.002783in}{2.285949in}}%
\pgfpathlineto{\pgfqpoint{5.010515in}{2.287095in}}%
\pgfpathlineto{\pgfqpoint{5.020824in}{2.291678in}}%
\pgfpathlineto{\pgfqpoint{5.028555in}{2.292570in}}%
\pgfpathlineto{\pgfqpoint{5.038864in}{2.296518in}}%
\pgfpathlineto{\pgfqpoint{5.046596in}{2.297696in}}%
\pgfpathlineto{\pgfqpoint{5.056905in}{2.302424in}}%
\pgfpathlineto{\pgfqpoint{5.064636in}{2.303268in}}%
\pgfpathlineto{\pgfqpoint{5.072368in}{2.306882in}}%
\pgfpathlineto{\pgfqpoint{5.074945in}{2.308352in}}%
\pgfpathlineto{\pgfqpoint{5.085254in}{2.309822in}}%
\pgfpathlineto{\pgfqpoint{5.092986in}{2.314208in}}%
\pgfpathlineto{\pgfqpoint{5.100717in}{2.315850in}}%
\pgfpathlineto{\pgfqpoint{5.111026in}{2.323092in}}%
\pgfpathlineto{\pgfqpoint{5.118758in}{2.324963in}}%
\pgfpathlineto{\pgfqpoint{5.129067in}{2.332251in}}%
\pgfpathlineto{\pgfqpoint{5.136798in}{2.333942in}}%
\pgfpathlineto{\pgfqpoint{5.147107in}{2.340088in}}%
\pgfpathlineto{\pgfqpoint{5.154839in}{2.341624in}}%
\pgfpathlineto{\pgfqpoint{5.165148in}{2.347682in}}%
\pgfpathlineto{\pgfqpoint{5.172879in}{2.349202in}}%
\pgfpathlineto{\pgfqpoint{5.183188in}{2.355553in}}%
\pgfpathlineto{\pgfqpoint{5.190920in}{2.357199in}}%
\pgfpathlineto{\pgfqpoint{5.201228in}{2.363619in}}%
\pgfpathlineto{\pgfqpoint{5.208960in}{2.365191in}}%
\pgfpathlineto{\pgfqpoint{5.219269in}{2.371560in}}%
\pgfpathlineto{\pgfqpoint{5.227001in}{2.373214in}}%
\pgfpathlineto{\pgfqpoint{5.237309in}{2.379588in}}%
\pgfpathlineto{\pgfqpoint{5.247618in}{2.381196in}}%
\pgfpathlineto{\pgfqpoint{5.255350in}{2.385498in}}%
\pgfpathlineto{\pgfqpoint{5.263082in}{2.386883in}}%
\pgfpathlineto{\pgfqpoint{5.273390in}{2.391654in}}%
\pgfpathlineto{\pgfqpoint{5.281122in}{2.392899in}}%
\pgfpathlineto{\pgfqpoint{5.291431in}{2.398402in}}%
\pgfpathlineto{\pgfqpoint{5.299163in}{2.399557in}}%
\pgfpathlineto{\pgfqpoint{5.309471in}{2.404243in}}%
\pgfpathlineto{\pgfqpoint{5.319780in}{2.406120in}}%
\pgfpathlineto{\pgfqpoint{5.327512in}{2.408572in}}%
\pgfpathlineto{\pgfqpoint{5.337821in}{2.410005in}}%
\pgfpathlineto{\pgfqpoint{5.345552in}{2.412049in}}%
\pgfpathlineto{\pgfqpoint{5.355861in}{2.413401in}}%
\pgfpathlineto{\pgfqpoint{5.363593in}{2.415339in}}%
\pgfpathlineto{\pgfqpoint{5.379056in}{2.417327in}}%
\pgfpathlineto{\pgfqpoint{5.381633in}{2.417717in}}%
\pgfpathlineto{\pgfqpoint{5.394519in}{2.418757in}}%
\pgfpathlineto{\pgfqpoint{5.399674in}{2.419614in}}%
\pgfpathlineto{\pgfqpoint{5.409983in}{2.420906in}}%
\pgfpathlineto{\pgfqpoint{5.417714in}{2.423375in}}%
\pgfpathlineto{\pgfqpoint{5.425446in}{2.424164in}}%
\pgfpathlineto{\pgfqpoint{5.435755in}{2.427520in}}%
\pgfpathlineto{\pgfqpoint{5.446064in}{2.429041in}}%
\pgfpathlineto{\pgfqpoint{5.448641in}{2.429830in}}%
\pgfpathlineto{\pgfqpoint{5.469259in}{2.433838in}}%
\pgfpathlineto{\pgfqpoint{5.471836in}{2.434614in}}%
\pgfpathlineto{\pgfqpoint{5.482145in}{2.436063in}}%
\pgfpathlineto{\pgfqpoint{5.489876in}{2.439230in}}%
\pgfpathlineto{\pgfqpoint{5.497608in}{2.440479in}}%
\pgfpathlineto{\pgfqpoint{5.507917in}{2.444725in}}%
\pgfpathlineto{\pgfqpoint{5.515648in}{2.445839in}}%
\pgfpathlineto{\pgfqpoint{5.525957in}{2.450409in}}%
\pgfpathlineto{\pgfqpoint{5.536266in}{2.451550in}}%
\pgfpathlineto{\pgfqpoint{5.543998in}{2.454818in}}%
\pgfpathlineto{\pgfqpoint{5.554307in}{2.455876in}}%
\pgfpathlineto{\pgfqpoint{5.562038in}{2.459028in}}%
\pgfpathlineto{\pgfqpoint{5.569770in}{2.460012in}}%
\pgfpathlineto{\pgfqpoint{5.580079in}{2.463917in}}%
\pgfpathlineto{\pgfqpoint{5.590388in}{2.464877in}}%
\pgfpathlineto{\pgfqpoint{5.598119in}{2.467992in}}%
\pgfpathlineto{\pgfqpoint{5.608428in}{2.469869in}}%
\pgfpathlineto{\pgfqpoint{5.616160in}{2.472600in}}%
\pgfpathlineto{\pgfqpoint{5.626469in}{2.474157in}}%
\pgfpathlineto{\pgfqpoint{5.634200in}{2.476402in}}%
\pgfpathlineto{\pgfqpoint{5.644509in}{2.477959in}}%
\pgfpathlineto{\pgfqpoint{5.652241in}{2.480778in}}%
\pgfpathlineto{\pgfqpoint{5.659972in}{2.481907in}}%
\pgfpathlineto{\pgfqpoint{5.670281in}{2.487019in}}%
\pgfpathlineto{\pgfqpoint{5.680590in}{2.488372in}}%
\pgfpathlineto{\pgfqpoint{5.688322in}{2.493194in}}%
\pgfpathlineto{\pgfqpoint{5.696053in}{2.494788in}}%
\pgfpathlineto{\pgfqpoint{5.703785in}{2.499954in}}%
\pgfpathlineto{\pgfqpoint{5.706362in}{2.501708in}}%
\pgfpathlineto{\pgfqpoint{5.714094in}{2.503420in}}%
\pgfpathlineto{\pgfqpoint{5.724403in}{2.510518in}}%
\pgfpathlineto{\pgfqpoint{5.732134in}{2.512401in}}%
\pgfpathlineto{\pgfqpoint{5.742443in}{2.519694in}}%
\pgfpathlineto{\pgfqpoint{5.750175in}{2.521656in}}%
\pgfpathlineto{\pgfqpoint{5.760484in}{2.529084in}}%
\pgfpathlineto{\pgfqpoint{5.768215in}{2.530794in}}%
\pgfpathlineto{\pgfqpoint{5.778524in}{2.537759in}}%
\pgfpathlineto{\pgfqpoint{5.786256in}{2.539453in}}%
\pgfpathlineto{\pgfqpoint{5.796565in}{2.545956in}}%
\pgfpathlineto{\pgfqpoint{5.804296in}{2.547535in}}%
\pgfpathlineto{\pgfqpoint{5.812028in}{2.552177in}}%
\pgfpathlineto{\pgfqpoint{5.822337in}{2.553777in}}%
\pgfpathlineto{\pgfqpoint{5.832646in}{2.560202in}}%
\pgfpathlineto{\pgfqpoint{5.840377in}{2.562066in}}%
\pgfpathlineto{\pgfqpoint{5.850686in}{2.569848in}}%
\pgfpathlineto{\pgfqpoint{5.858418in}{2.571714in}}%
\pgfpathlineto{\pgfqpoint{5.868727in}{2.580218in}}%
\pgfpathlineto{\pgfqpoint{5.876458in}{2.582351in}}%
\pgfpathlineto{\pgfqpoint{5.886767in}{2.590242in}}%
\pgfpathlineto{\pgfqpoint{5.894499in}{2.592245in}}%
\pgfpathlineto{\pgfqpoint{5.904807in}{2.599909in}}%
\pgfpathlineto{\pgfqpoint{5.912539in}{2.602035in}}%
\pgfpathlineto{\pgfqpoint{5.922848in}{2.610561in}}%
\pgfpathlineto{\pgfqpoint{5.933157in}{2.612969in}}%
\pgfpathlineto{\pgfqpoint{5.940888in}{2.620847in}}%
\pgfpathlineto{\pgfqpoint{5.948620in}{2.623541in}}%
\pgfpathlineto{\pgfqpoint{5.958929in}{2.634051in}}%
\pgfpathlineto{\pgfqpoint{5.966661in}{2.636788in}}%
\pgfpathlineto{\pgfqpoint{5.976969in}{2.648895in}}%
\pgfpathlineto{\pgfqpoint{5.984701in}{2.652116in}}%
\pgfpathlineto{\pgfqpoint{5.995010in}{2.664644in}}%
\pgfpathlineto{\pgfqpoint{6.002742in}{2.667717in}}%
\pgfpathlineto{\pgfqpoint{6.013050in}{2.678608in}}%
\pgfpathlineto{\pgfqpoint{6.020782in}{2.681362in}}%
\pgfpathlineto{\pgfqpoint{6.025936in}{2.684097in}}%
\pgfpathlineto{\pgfqpoint{6.031091in}{2.689363in}}%
\pgfpathlineto{\pgfqpoint{6.038823in}{2.692126in}}%
\pgfpathlineto{\pgfqpoint{6.049131in}{2.703213in}}%
\pgfpathlineto{\pgfqpoint{6.056863in}{2.706017in}}%
\pgfpathlineto{\pgfqpoint{6.067172in}{2.717132in}}%
\pgfpathlineto{\pgfqpoint{6.074904in}{2.719727in}}%
\pgfpathlineto{\pgfqpoint{6.085212in}{2.726683in}}%
\pgfpathlineto{\pgfqpoint{6.092944in}{2.728524in}}%
\pgfpathlineto{\pgfqpoint{6.100676in}{2.734980in}}%
\pgfpathlineto{\pgfqpoint{6.103253in}{2.737292in}}%
\pgfpathlineto{\pgfqpoint{6.110984in}{2.739576in}}%
\pgfpathlineto{\pgfqpoint{6.121293in}{2.748245in}}%
\pgfpathlineto{\pgfqpoint{6.129025in}{2.750475in}}%
\pgfpathlineto{\pgfqpoint{6.139334in}{2.758874in}}%
\pgfpathlineto{\pgfqpoint{6.147065in}{2.760839in}}%
\pgfpathlineto{\pgfqpoint{6.157374in}{2.768442in}}%
\pgfpathlineto{\pgfqpoint{6.165106in}{2.770297in}}%
\pgfpathlineto{\pgfqpoint{6.175415in}{2.777985in}}%
\pgfpathlineto{\pgfqpoint{6.185724in}{2.779692in}}%
\pgfpathlineto{\pgfqpoint{6.193455in}{2.785453in}}%
\pgfpathlineto{\pgfqpoint{6.201187in}{2.787781in}}%
\pgfpathlineto{\pgfqpoint{6.211496in}{2.797457in}}%
\pgfpathlineto{\pgfqpoint{6.219227in}{2.800078in}}%
\pgfpathlineto{\pgfqpoint{6.226959in}{2.807319in}}%
\pgfpathlineto{\pgfqpoint{6.229536in}{2.809651in}}%
\pgfpathlineto{\pgfqpoint{6.237268in}{2.811971in}}%
\pgfpathlineto{\pgfqpoint{6.247577in}{2.820921in}}%
\pgfpathlineto{\pgfqpoint{6.255308in}{2.823354in}}%
\pgfpathlineto{\pgfqpoint{6.265617in}{2.834041in}}%
\pgfpathlineto{\pgfqpoint{6.273349in}{2.836740in}}%
\pgfpathlineto{\pgfqpoint{6.283658in}{2.847574in}}%
\pgfpathlineto{\pgfqpoint{6.291389in}{2.850386in}}%
\pgfpathlineto{\pgfqpoint{6.301698in}{2.861655in}}%
\pgfpathlineto{\pgfqpoint{6.309430in}{2.864645in}}%
\pgfpathlineto{\pgfqpoint{6.319739in}{2.881428in}}%
\pgfpathlineto{\pgfqpoint{6.327470in}{2.885189in}}%
\pgfpathlineto{\pgfqpoint{6.337779in}{2.900127in}}%
\pgfpathlineto{\pgfqpoint{6.345511in}{2.903735in}}%
\pgfpathlineto{\pgfqpoint{6.355820in}{2.917508in}}%
\pgfpathlineto{\pgfqpoint{6.363551in}{2.920847in}}%
\pgfpathlineto{\pgfqpoint{6.373860in}{2.934293in}}%
\pgfpathlineto{\pgfqpoint{6.381592in}{2.937850in}}%
\pgfpathlineto{\pgfqpoint{6.386746in}{2.945269in}}%
\pgfpathlineto{\pgfqpoint{6.391901in}{2.948872in}}%
\pgfpathlineto{\pgfqpoint{6.399632in}{2.952702in}}%
\pgfpathlineto{\pgfqpoint{6.407364in}{2.965501in}}%
\pgfpathlineto{\pgfqpoint{6.409941in}{2.969942in}}%
\pgfpathlineto{\pgfqpoint{6.417673in}{2.974175in}}%
\pgfpathlineto{\pgfqpoint{6.427982in}{2.990750in}}%
\pgfpathlineto{\pgfqpoint{6.435713in}{2.994889in}}%
\pgfpathlineto{\pgfqpoint{6.446022in}{3.010745in}}%
\pgfpathlineto{\pgfqpoint{6.453754in}{3.014708in}}%
\pgfpathlineto{\pgfqpoint{6.464063in}{3.029576in}}%
\pgfpathlineto{\pgfqpoint{6.474371in}{3.033182in}}%
\pgfpathlineto{\pgfqpoint{6.482103in}{3.043990in}}%
\pgfpathlineto{\pgfqpoint{6.482103in}{3.043990in}}%
\pgfusepath{stroke}%
\end{pgfscope}%
\begin{pgfscope}%
\pgfpathrectangle{\pgfqpoint{0.563921in}{0.521603in}}{\pgfqpoint{6.200000in}{2.642500in}}%
\pgfusepath{clip}%
\pgfsetroundcap%
\pgfsetroundjoin%
\pgfsetlinewidth{1.505625pt}%
\definecolor{currentstroke}{rgb}{1.000000,0.498039,0.054902}%
\pgfsetstrokecolor{currentstroke}%
\pgfsetdash{}{0pt}%
\pgfpathmoveto{\pgfqpoint{0.845739in}{0.641717in}}%
\pgfpathlineto{\pgfqpoint{0.848317in}{0.642611in}}%
\pgfpathlineto{\pgfqpoint{0.850894in}{0.656202in}}%
\pgfpathlineto{\pgfqpoint{0.853471in}{0.654748in}}%
\pgfpathlineto{\pgfqpoint{0.861203in}{0.653373in}}%
\pgfpathlineto{\pgfqpoint{0.863780in}{0.653781in}}%
\pgfpathlineto{\pgfqpoint{0.866357in}{0.656485in}}%
\pgfpathlineto{\pgfqpoint{0.868934in}{0.666224in}}%
\pgfpathlineto{\pgfqpoint{0.871512in}{0.671444in}}%
\pgfpathlineto{\pgfqpoint{0.881820in}{0.678235in}}%
\pgfpathlineto{\pgfqpoint{0.886975in}{0.691486in}}%
\pgfpathlineto{\pgfqpoint{0.897284in}{0.689184in}}%
\pgfpathlineto{\pgfqpoint{0.899861in}{0.687584in}}%
\pgfpathlineto{\pgfqpoint{0.902438in}{0.687569in}}%
\pgfpathlineto{\pgfqpoint{0.907593in}{0.686010in}}%
\pgfpathlineto{\pgfqpoint{0.917901in}{0.684751in}}%
\pgfpathlineto{\pgfqpoint{0.920479in}{0.685833in}}%
\pgfpathlineto{\pgfqpoint{0.923056in}{0.688687in}}%
\pgfpathlineto{\pgfqpoint{0.925633in}{0.696231in}}%
\pgfpathlineto{\pgfqpoint{0.933365in}{0.699730in}}%
\pgfpathlineto{\pgfqpoint{0.935942in}{0.703718in}}%
\pgfpathlineto{\pgfqpoint{0.938519in}{0.705206in}}%
\pgfpathlineto{\pgfqpoint{0.941096in}{0.708489in}}%
\pgfpathlineto{\pgfqpoint{0.943674in}{0.709648in}}%
\pgfpathlineto{\pgfqpoint{0.956560in}{0.712167in}}%
\pgfpathlineto{\pgfqpoint{0.961714in}{0.717672in}}%
\pgfpathlineto{\pgfqpoint{0.972023in}{0.719708in}}%
\pgfpathlineto{\pgfqpoint{0.987486in}{0.728898in}}%
\pgfpathlineto{\pgfqpoint{0.990063in}{0.731394in}}%
\pgfpathlineto{\pgfqpoint{0.992641in}{0.731887in}}%
\pgfpathlineto{\pgfqpoint{0.995218in}{0.733458in}}%
\pgfpathlineto{\pgfqpoint{0.997795in}{0.733848in}}%
\pgfpathlineto{\pgfqpoint{1.005527in}{0.734120in}}%
\pgfpathlineto{\pgfqpoint{1.010681in}{0.732687in}}%
\pgfpathlineto{\pgfqpoint{1.023567in}{0.733091in}}%
\pgfpathlineto{\pgfqpoint{1.026144in}{0.735010in}}%
\pgfpathlineto{\pgfqpoint{1.033876in}{0.753113in}}%
\pgfpathlineto{\pgfqpoint{1.041608in}{0.759563in}}%
\pgfpathlineto{\pgfqpoint{1.049339in}{0.773502in}}%
\pgfpathlineto{\pgfqpoint{1.051916in}{0.777479in}}%
\pgfpathlineto{\pgfqpoint{1.059648in}{0.783911in}}%
\pgfpathlineto{\pgfqpoint{1.067380in}{0.797977in}}%
\pgfpathlineto{\pgfqpoint{1.069957in}{0.800878in}}%
\pgfpathlineto{\pgfqpoint{1.077689in}{0.803742in}}%
\pgfpathlineto{\pgfqpoint{1.080266in}{0.807217in}}%
\pgfpathlineto{\pgfqpoint{1.082843in}{0.808969in}}%
\pgfpathlineto{\pgfqpoint{1.085420in}{0.811509in}}%
\pgfpathlineto{\pgfqpoint{1.100883in}{0.813612in}}%
\pgfpathlineto{\pgfqpoint{1.103461in}{0.815481in}}%
\pgfpathlineto{\pgfqpoint{1.106038in}{0.816387in}}%
\pgfpathlineto{\pgfqpoint{1.113770in}{0.817757in}}%
\pgfpathlineto{\pgfqpoint{1.118924in}{0.820740in}}%
\pgfpathlineto{\pgfqpoint{1.124078in}{0.822137in}}%
\pgfpathlineto{\pgfqpoint{1.134387in}{0.823216in}}%
\pgfpathlineto{\pgfqpoint{1.139542in}{0.827415in}}%
\pgfpathlineto{\pgfqpoint{1.142119in}{0.830322in}}%
\pgfpathlineto{\pgfqpoint{1.149851in}{0.833073in}}%
\pgfpathlineto{\pgfqpoint{1.157582in}{0.842727in}}%
\pgfpathlineto{\pgfqpoint{1.160159in}{0.844597in}}%
\pgfpathlineto{\pgfqpoint{1.167891in}{0.846340in}}%
\pgfpathlineto{\pgfqpoint{1.173045in}{0.848740in}}%
\pgfpathlineto{\pgfqpoint{1.178200in}{0.850586in}}%
\pgfpathlineto{\pgfqpoint{1.188509in}{0.850427in}}%
\pgfpathlineto{\pgfqpoint{1.193663in}{0.848928in}}%
\pgfpathlineto{\pgfqpoint{1.196240in}{0.847902in}}%
\pgfpathlineto{\pgfqpoint{1.206549in}{0.846248in}}%
\pgfpathlineto{\pgfqpoint{1.214281in}{0.843643in}}%
\pgfpathlineto{\pgfqpoint{1.224590in}{0.842915in}}%
\pgfpathlineto{\pgfqpoint{1.232321in}{0.840266in}}%
\pgfpathlineto{\pgfqpoint{1.240053in}{0.839414in}}%
\pgfpathlineto{\pgfqpoint{1.250362in}{0.835978in}}%
\pgfpathlineto{\pgfqpoint{1.260671in}{0.834472in}}%
\pgfpathlineto{\pgfqpoint{1.268402in}{0.832132in}}%
\pgfpathlineto{\pgfqpoint{1.281288in}{0.830581in}}%
\pgfpathlineto{\pgfqpoint{1.286443in}{0.829422in}}%
\pgfpathlineto{\pgfqpoint{1.296752in}{0.828092in}}%
\pgfpathlineto{\pgfqpoint{1.304483in}{0.827044in}}%
\pgfpathlineto{\pgfqpoint{1.314792in}{0.828009in}}%
\pgfpathlineto{\pgfqpoint{1.322524in}{0.828678in}}%
\pgfpathlineto{\pgfqpoint{1.335410in}{0.828664in}}%
\pgfpathlineto{\pgfqpoint{1.340564in}{0.828039in}}%
\pgfpathlineto{\pgfqpoint{1.353450in}{0.828188in}}%
\pgfpathlineto{\pgfqpoint{1.358605in}{0.826964in}}%
\pgfpathlineto{\pgfqpoint{1.368914in}{0.825694in}}%
\pgfpathlineto{\pgfqpoint{1.374068in}{0.824873in}}%
\pgfpathlineto{\pgfqpoint{1.386954in}{0.824569in}}%
\pgfpathlineto{\pgfqpoint{1.394686in}{0.823297in}}%
\pgfpathlineto{\pgfqpoint{1.407572in}{0.822411in}}%
\pgfpathlineto{\pgfqpoint{1.412726in}{0.821339in}}%
\pgfpathlineto{\pgfqpoint{1.423035in}{0.820266in}}%
\pgfpathlineto{\pgfqpoint{1.430767in}{0.819196in}}%
\pgfpathlineto{\pgfqpoint{1.441076in}{0.818317in}}%
\pgfpathlineto{\pgfqpoint{1.448807in}{0.817120in}}%
\pgfpathlineto{\pgfqpoint{1.461693in}{0.816224in}}%
\pgfpathlineto{\pgfqpoint{1.466848in}{0.815780in}}%
\pgfpathlineto{\pgfqpoint{1.479734in}{0.815426in}}%
\pgfpathlineto{\pgfqpoint{1.484888in}{0.814871in}}%
\pgfpathlineto{\pgfqpoint{1.513237in}{0.814350in}}%
\pgfpathlineto{\pgfqpoint{1.528701in}{0.813704in}}%
\pgfpathlineto{\pgfqpoint{1.539010in}{0.812094in}}%
\pgfpathlineto{\pgfqpoint{1.600863in}{0.809671in}}%
\pgfpathlineto{\pgfqpoint{1.611172in}{0.808059in}}%
\pgfpathlineto{\pgfqpoint{1.629212in}{0.806990in}}%
\pgfpathlineto{\pgfqpoint{1.642098in}{0.805949in}}%
\pgfpathlineto{\pgfqpoint{1.647253in}{0.805189in}}%
\pgfpathlineto{\pgfqpoint{1.673025in}{0.803397in}}%
\pgfpathlineto{\pgfqpoint{1.683333in}{0.802356in}}%
\pgfpathlineto{\pgfqpoint{1.696220in}{0.801352in}}%
\pgfpathlineto{\pgfqpoint{1.701374in}{0.800663in}}%
\pgfpathlineto{\pgfqpoint{1.714260in}{0.799669in}}%
\pgfpathlineto{\pgfqpoint{1.719414in}{0.799043in}}%
\pgfpathlineto{\pgfqpoint{1.747764in}{0.797524in}}%
\pgfpathlineto{\pgfqpoint{1.755495in}{0.796876in}}%
\pgfpathlineto{\pgfqpoint{1.791576in}{0.796254in}}%
\pgfpathlineto{\pgfqpoint{1.801885in}{0.797123in}}%
\pgfpathlineto{\pgfqpoint{1.827657in}{0.801912in}}%
\pgfpathlineto{\pgfqpoint{1.871470in}{0.803088in}}%
\pgfpathlineto{\pgfqpoint{1.879202in}{0.804928in}}%
\pgfpathlineto{\pgfqpoint{1.881779in}{0.805741in}}%
\pgfpathlineto{\pgfqpoint{1.889510in}{0.806607in}}%
\pgfpathlineto{\pgfqpoint{1.899819in}{0.810036in}}%
\pgfpathlineto{\pgfqpoint{1.912705in}{0.811650in}}%
\pgfpathlineto{\pgfqpoint{1.917860in}{0.813183in}}%
\pgfpathlineto{\pgfqpoint{1.928169in}{0.814577in}}%
\pgfpathlineto{\pgfqpoint{1.935900in}{0.816944in}}%
\pgfpathlineto{\pgfqpoint{1.943632in}{0.817897in}}%
\pgfpathlineto{\pgfqpoint{1.953941in}{0.824121in}}%
\pgfpathlineto{\pgfqpoint{1.961672in}{0.826045in}}%
\pgfpathlineto{\pgfqpoint{1.971981in}{0.833330in}}%
\pgfpathlineto{\pgfqpoint{1.979713in}{0.835160in}}%
\pgfpathlineto{\pgfqpoint{1.987445in}{0.840063in}}%
\pgfpathlineto{\pgfqpoint{1.990022in}{0.841932in}}%
\pgfpathlineto{\pgfqpoint{1.997753in}{0.843708in}}%
\pgfpathlineto{\pgfqpoint{2.005485in}{0.850358in}}%
\pgfpathlineto{\pgfqpoint{2.015794in}{0.852468in}}%
\pgfpathlineto{\pgfqpoint{2.020948in}{0.856366in}}%
\pgfpathlineto{\pgfqpoint{2.026103in}{0.859407in}}%
\pgfpathlineto{\pgfqpoint{2.033834in}{0.860712in}}%
\pgfpathlineto{\pgfqpoint{2.044143in}{0.865554in}}%
\pgfpathlineto{\pgfqpoint{2.051875in}{0.866252in}}%
\pgfpathlineto{\pgfqpoint{2.059606in}{0.868710in}}%
\pgfpathlineto{\pgfqpoint{2.062184in}{0.870428in}}%
\pgfpathlineto{\pgfqpoint{2.069915in}{0.871890in}}%
\pgfpathlineto{\pgfqpoint{2.080224in}{0.879006in}}%
\pgfpathlineto{\pgfqpoint{2.087956in}{0.880669in}}%
\pgfpathlineto{\pgfqpoint{2.095687in}{0.886719in}}%
\pgfpathlineto{\pgfqpoint{2.098265in}{0.889308in}}%
\pgfpathlineto{\pgfqpoint{2.105996in}{0.891765in}}%
\pgfpathlineto{\pgfqpoint{2.116305in}{0.901319in}}%
\pgfpathlineto{\pgfqpoint{2.124037in}{0.903405in}}%
\pgfpathlineto{\pgfqpoint{2.131768in}{0.912450in}}%
\pgfpathlineto{\pgfqpoint{2.134346in}{0.915863in}}%
\pgfpathlineto{\pgfqpoint{2.142077in}{0.919721in}}%
\pgfpathlineto{\pgfqpoint{2.149809in}{0.931238in}}%
\pgfpathlineto{\pgfqpoint{2.152386in}{0.935143in}}%
\pgfpathlineto{\pgfqpoint{2.162695in}{0.939395in}}%
\pgfpathlineto{\pgfqpoint{2.170427in}{0.951129in}}%
\pgfpathlineto{\pgfqpoint{2.178158in}{0.955126in}}%
\pgfpathlineto{\pgfqpoint{2.188467in}{0.969904in}}%
\pgfpathlineto{\pgfqpoint{2.196199in}{0.974301in}}%
\pgfpathlineto{\pgfqpoint{2.201353in}{0.980462in}}%
\pgfpathlineto{\pgfqpoint{2.203930in}{0.983273in}}%
\pgfpathlineto{\pgfqpoint{2.206508in}{0.985217in}}%
\pgfpathlineto{\pgfqpoint{2.214239in}{0.987424in}}%
\pgfpathlineto{\pgfqpoint{2.219394in}{0.992263in}}%
\pgfpathlineto{\pgfqpoint{2.224548in}{0.995917in}}%
\pgfpathlineto{\pgfqpoint{2.232280in}{0.997359in}}%
\pgfpathlineto{\pgfqpoint{2.237434in}{1.001117in}}%
\pgfpathlineto{\pgfqpoint{2.242589in}{1.005654in}}%
\pgfpathlineto{\pgfqpoint{2.250320in}{1.008103in}}%
\pgfpathlineto{\pgfqpoint{2.255475in}{1.012320in}}%
\pgfpathlineto{\pgfqpoint{2.260629in}{1.014965in}}%
\pgfpathlineto{\pgfqpoint{2.268361in}{1.017829in}}%
\pgfpathlineto{\pgfqpoint{2.273515in}{1.023354in}}%
\pgfpathlineto{\pgfqpoint{2.278670in}{1.029154in}}%
\pgfpathlineto{\pgfqpoint{2.286401in}{1.032049in}}%
\pgfpathlineto{\pgfqpoint{2.291556in}{1.037464in}}%
\pgfpathlineto{\pgfqpoint{2.296710in}{1.040502in}}%
\pgfpathlineto{\pgfqpoint{2.304442in}{1.042177in}}%
\pgfpathlineto{\pgfqpoint{2.312173in}{1.047451in}}%
\pgfpathlineto{\pgfqpoint{2.314751in}{1.049196in}}%
\pgfpathlineto{\pgfqpoint{2.322482in}{1.050833in}}%
\pgfpathlineto{\pgfqpoint{2.332791in}{1.057165in}}%
\pgfpathlineto{\pgfqpoint{2.340523in}{1.058921in}}%
\pgfpathlineto{\pgfqpoint{2.350832in}{1.065390in}}%
\pgfpathlineto{\pgfqpoint{2.358563in}{1.066900in}}%
\pgfpathlineto{\pgfqpoint{2.366295in}{1.071434in}}%
\pgfpathlineto{\pgfqpoint{2.368872in}{1.072772in}}%
\pgfpathlineto{\pgfqpoint{2.376604in}{1.073889in}}%
\pgfpathlineto{\pgfqpoint{2.386912in}{1.077723in}}%
\pgfpathlineto{\pgfqpoint{2.399799in}{1.079676in}}%
\pgfpathlineto{\pgfqpoint{2.404953in}{1.080751in}}%
\pgfpathlineto{\pgfqpoint{2.415262in}{1.081343in}}%
\pgfpathlineto{\pgfqpoint{2.422993in}{1.083654in}}%
\pgfpathlineto{\pgfqpoint{2.430725in}{1.084448in}}%
\pgfpathlineto{\pgfqpoint{2.441034in}{1.088715in}}%
\pgfpathlineto{\pgfqpoint{2.448766in}{1.089859in}}%
\pgfpathlineto{\pgfqpoint{2.459074in}{1.095924in}}%
\pgfpathlineto{\pgfqpoint{2.466806in}{1.097168in}}%
\pgfpathlineto{\pgfqpoint{2.477115in}{1.101643in}}%
\pgfpathlineto{\pgfqpoint{2.484847in}{1.102605in}}%
\pgfpathlineto{\pgfqpoint{2.490001in}{1.104436in}}%
\pgfpathlineto{\pgfqpoint{2.495155in}{1.105832in}}%
\pgfpathlineto{\pgfqpoint{2.508041in}{1.106971in}}%
\pgfpathlineto{\pgfqpoint{2.513196in}{1.108611in}}%
\pgfpathlineto{\pgfqpoint{2.523505in}{1.110460in}}%
\pgfpathlineto{\pgfqpoint{2.526082in}{1.111499in}}%
\pgfpathlineto{\pgfqpoint{2.531236in}{1.115445in}}%
\pgfpathlineto{\pgfqpoint{2.538968in}{1.117383in}}%
\pgfpathlineto{\pgfqpoint{2.549277in}{1.125765in}}%
\pgfpathlineto{\pgfqpoint{2.557009in}{1.128190in}}%
\pgfpathlineto{\pgfqpoint{2.564740in}{1.135348in}}%
\pgfpathlineto{\pgfqpoint{2.567317in}{1.137475in}}%
\pgfpathlineto{\pgfqpoint{2.575049in}{1.139514in}}%
\pgfpathlineto{\pgfqpoint{2.585358in}{1.147149in}}%
\pgfpathlineto{\pgfqpoint{2.593089in}{1.148926in}}%
\pgfpathlineto{\pgfqpoint{2.603398in}{1.156181in}}%
\pgfpathlineto{\pgfqpoint{2.611130in}{1.158065in}}%
\pgfpathlineto{\pgfqpoint{2.621439in}{1.166129in}}%
\pgfpathlineto{\pgfqpoint{2.629170in}{1.168444in}}%
\pgfpathlineto{\pgfqpoint{2.634325in}{1.173366in}}%
\pgfpathlineto{\pgfqpoint{2.639479in}{1.175899in}}%
\pgfpathlineto{\pgfqpoint{2.647211in}{1.178264in}}%
\pgfpathlineto{\pgfqpoint{2.657520in}{1.187019in}}%
\pgfpathlineto{\pgfqpoint{2.665251in}{1.189347in}}%
\pgfpathlineto{\pgfqpoint{2.670406in}{1.193354in}}%
\pgfpathlineto{\pgfqpoint{2.675560in}{1.196750in}}%
\pgfpathlineto{\pgfqpoint{2.683292in}{1.198648in}}%
\pgfpathlineto{\pgfqpoint{2.688446in}{1.202627in}}%
\pgfpathlineto{\pgfqpoint{2.693601in}{1.207434in}}%
\pgfpathlineto{\pgfqpoint{2.701332in}{1.210125in}}%
\pgfpathlineto{\pgfqpoint{2.703910in}{1.212879in}}%
\pgfpathlineto{\pgfqpoint{2.709064in}{1.215680in}}%
\pgfpathlineto{\pgfqpoint{2.711641in}{1.218490in}}%
\pgfpathlineto{\pgfqpoint{2.719373in}{1.221370in}}%
\pgfpathlineto{\pgfqpoint{2.721950in}{1.224515in}}%
\pgfpathlineto{\pgfqpoint{2.727105in}{1.227277in}}%
\pgfpathlineto{\pgfqpoint{2.729682in}{1.230078in}}%
\pgfpathlineto{\pgfqpoint{2.737413in}{1.232830in}}%
\pgfpathlineto{\pgfqpoint{2.747722in}{1.243008in}}%
\pgfpathlineto{\pgfqpoint{2.755454in}{1.245003in}}%
\pgfpathlineto{\pgfqpoint{2.763185in}{1.251330in}}%
\pgfpathlineto{\pgfqpoint{2.765763in}{1.254167in}}%
\pgfpathlineto{\pgfqpoint{2.776072in}{1.256872in}}%
\pgfpathlineto{\pgfqpoint{2.781226in}{1.261971in}}%
\pgfpathlineto{\pgfqpoint{2.783803in}{1.263748in}}%
\pgfpathlineto{\pgfqpoint{2.791535in}{1.265254in}}%
\pgfpathlineto{\pgfqpoint{2.801844in}{1.271296in}}%
\pgfpathlineto{\pgfqpoint{2.809575in}{1.272204in}}%
\pgfpathlineto{\pgfqpoint{2.817307in}{1.275700in}}%
\pgfpathlineto{\pgfqpoint{2.819884in}{1.277310in}}%
\pgfpathlineto{\pgfqpoint{2.827616in}{1.279175in}}%
\pgfpathlineto{\pgfqpoint{2.837925in}{1.286861in}}%
\pgfpathlineto{\pgfqpoint{2.848234in}{1.288756in}}%
\pgfpathlineto{\pgfqpoint{2.855965in}{1.294223in}}%
\pgfpathlineto{\pgfqpoint{2.863697in}{1.296220in}}%
\pgfpathlineto{\pgfqpoint{2.874006in}{1.304305in}}%
\pgfpathlineto{\pgfqpoint{2.881737in}{1.306199in}}%
\pgfpathlineto{\pgfqpoint{2.892046in}{1.316154in}}%
\pgfpathlineto{\pgfqpoint{2.899778in}{1.318717in}}%
\pgfpathlineto{\pgfqpoint{2.907509in}{1.325308in}}%
\pgfpathlineto{\pgfqpoint{2.910087in}{1.327029in}}%
\pgfpathlineto{\pgfqpoint{2.917818in}{1.328946in}}%
\pgfpathlineto{\pgfqpoint{2.928127in}{1.336499in}}%
\pgfpathlineto{\pgfqpoint{2.935859in}{1.338264in}}%
\pgfpathlineto{\pgfqpoint{2.946168in}{1.344562in}}%
\pgfpathlineto{\pgfqpoint{2.953899in}{1.346056in}}%
\pgfpathlineto{\pgfqpoint{2.964208in}{1.352290in}}%
\pgfpathlineto{\pgfqpoint{2.971940in}{1.353184in}}%
\pgfpathlineto{\pgfqpoint{2.982249in}{1.356515in}}%
\pgfpathlineto{\pgfqpoint{2.992557in}{1.357964in}}%
\pgfpathlineto{\pgfqpoint{2.997712in}{1.359682in}}%
\pgfpathlineto{\pgfqpoint{3.008021in}{1.360505in}}%
\pgfpathlineto{\pgfqpoint{3.018330in}{1.363996in}}%
\pgfpathlineto{\pgfqpoint{3.026061in}{1.364831in}}%
\pgfpathlineto{\pgfqpoint{3.036370in}{1.368189in}}%
\pgfpathlineto{\pgfqpoint{3.046679in}{1.369688in}}%
\pgfpathlineto{\pgfqpoint{3.054411in}{1.372603in}}%
\pgfpathlineto{\pgfqpoint{3.062142in}{1.373733in}}%
\pgfpathlineto{\pgfqpoint{3.069874in}{1.376489in}}%
\pgfpathlineto{\pgfqpoint{3.072451in}{1.377263in}}%
\pgfpathlineto{\pgfqpoint{3.082760in}{1.378817in}}%
\pgfpathlineto{\pgfqpoint{3.090491in}{1.381360in}}%
\pgfpathlineto{\pgfqpoint{3.100800in}{1.382660in}}%
\pgfpathlineto{\pgfqpoint{3.108532in}{1.386433in}}%
\pgfpathlineto{\pgfqpoint{3.116264in}{1.387762in}}%
\pgfpathlineto{\pgfqpoint{3.123995in}{1.391781in}}%
\pgfpathlineto{\pgfqpoint{3.126572in}{1.393565in}}%
\pgfpathlineto{\pgfqpoint{3.134304in}{1.395449in}}%
\pgfpathlineto{\pgfqpoint{3.144613in}{1.402395in}}%
\pgfpathlineto{\pgfqpoint{3.152345in}{1.403982in}}%
\pgfpathlineto{\pgfqpoint{3.162653in}{1.410742in}}%
\pgfpathlineto{\pgfqpoint{3.170385in}{1.412441in}}%
\pgfpathlineto{\pgfqpoint{3.180694in}{1.418462in}}%
\pgfpathlineto{\pgfqpoint{3.188426in}{1.419978in}}%
\pgfpathlineto{\pgfqpoint{3.196157in}{1.424733in}}%
\pgfpathlineto{\pgfqpoint{3.206466in}{1.426276in}}%
\pgfpathlineto{\pgfqpoint{3.216775in}{1.431896in}}%
\pgfpathlineto{\pgfqpoint{3.224507in}{1.433276in}}%
\pgfpathlineto{\pgfqpoint{3.234815in}{1.438258in}}%
\pgfpathlineto{\pgfqpoint{3.242547in}{1.439332in}}%
\pgfpathlineto{\pgfqpoint{3.252856in}{1.443560in}}%
\pgfpathlineto{\pgfqpoint{3.260588in}{1.444446in}}%
\pgfpathlineto{\pgfqpoint{3.268319in}{1.446448in}}%
\pgfpathlineto{\pgfqpoint{3.288937in}{1.447886in}}%
\pgfpathlineto{\pgfqpoint{3.301823in}{1.448717in}}%
\pgfpathlineto{\pgfqpoint{3.306977in}{1.449212in}}%
\pgfpathlineto{\pgfqpoint{3.317286in}{1.449837in}}%
\pgfpathlineto{\pgfqpoint{3.325018in}{1.451118in}}%
\pgfpathlineto{\pgfqpoint{3.335327in}{1.452066in}}%
\pgfpathlineto{\pgfqpoint{3.343058in}{1.453416in}}%
\pgfpathlineto{\pgfqpoint{3.355944in}{1.454439in}}%
\pgfpathlineto{\pgfqpoint{3.361099in}{1.455373in}}%
\pgfpathlineto{\pgfqpoint{3.376562in}{1.456681in}}%
\pgfpathlineto{\pgfqpoint{3.415220in}{1.460057in}}%
\pgfpathlineto{\pgfqpoint{3.459033in}{1.460988in}}%
\pgfpathlineto{\pgfqpoint{3.482228in}{1.459799in}}%
\pgfpathlineto{\pgfqpoint{3.500268in}{1.460450in}}%
\pgfpathlineto{\pgfqpoint{3.518309in}{1.462770in}}%
\pgfpathlineto{\pgfqpoint{3.523463in}{1.463924in}}%
\pgfpathlineto{\pgfqpoint{3.533772in}{1.465090in}}%
\pgfpathlineto{\pgfqpoint{3.541504in}{1.466494in}}%
\pgfpathlineto{\pgfqpoint{3.554390in}{1.467648in}}%
\pgfpathlineto{\pgfqpoint{3.559544in}{1.468419in}}%
\pgfpathlineto{\pgfqpoint{3.569853in}{1.469315in}}%
\pgfpathlineto{\pgfqpoint{3.577585in}{1.470343in}}%
\pgfpathlineto{\pgfqpoint{3.587893in}{1.471485in}}%
\pgfpathlineto{\pgfqpoint{3.613666in}{1.475762in}}%
\pgfpathlineto{\pgfqpoint{3.626552in}{1.476471in}}%
\pgfpathlineto{\pgfqpoint{3.631706in}{1.477591in}}%
\pgfpathlineto{\pgfqpoint{3.642015in}{1.478836in}}%
\pgfpathlineto{\pgfqpoint{3.644592in}{1.479489in}}%
\pgfpathlineto{\pgfqpoint{3.667787in}{1.482520in}}%
\pgfpathlineto{\pgfqpoint{3.683250in}{1.483566in}}%
\pgfpathlineto{\pgfqpoint{3.685828in}{1.483842in}}%
\pgfpathlineto{\pgfqpoint{3.716754in}{1.484083in}}%
\pgfpathlineto{\pgfqpoint{3.809534in}{1.476952in}}%
\pgfpathlineto{\pgfqpoint{3.812111in}{1.476625in}}%
\pgfpathlineto{\pgfqpoint{3.824997in}{1.475638in}}%
\pgfpathlineto{\pgfqpoint{3.830151in}{1.474916in}}%
\pgfpathlineto{\pgfqpoint{3.840460in}{1.474130in}}%
\pgfpathlineto{\pgfqpoint{3.848192in}{1.473035in}}%
\pgfpathlineto{\pgfqpoint{3.861078in}{1.472047in}}%
\pgfpathlineto{\pgfqpoint{3.866232in}{1.471447in}}%
\pgfpathlineto{\pgfqpoint{3.879118in}{1.470438in}}%
\pgfpathlineto{\pgfqpoint{3.884273in}{1.469566in}}%
\pgfpathlineto{\pgfqpoint{3.897159in}{1.468290in}}%
\pgfpathlineto{\pgfqpoint{3.899736in}{1.467909in}}%
\pgfpathlineto{\pgfqpoint{3.912622in}{1.467103in}}%
\pgfpathlineto{\pgfqpoint{3.920354in}{1.465912in}}%
\pgfpathlineto{\pgfqpoint{3.933240in}{1.464759in}}%
\pgfpathlineto{\pgfqpoint{3.956435in}{1.461841in}}%
\pgfpathlineto{\pgfqpoint{3.966744in}{1.460972in}}%
\pgfpathlineto{\pgfqpoint{3.974475in}{1.459669in}}%
\pgfpathlineto{\pgfqpoint{3.984784in}{1.458835in}}%
\pgfpathlineto{\pgfqpoint{3.992516in}{1.457616in}}%
\pgfpathlineto{\pgfqpoint{4.005402in}{1.456500in}}%
\pgfpathlineto{\pgfqpoint{4.010556in}{1.455821in}}%
\pgfpathlineto{\pgfqpoint{4.023442in}{1.454844in}}%
\pgfpathlineto{\pgfqpoint{4.028597in}{1.454206in}}%
\pgfpathlineto{\pgfqpoint{4.044060in}{1.453157in}}%
\pgfpathlineto{\pgfqpoint{4.046637in}{1.452795in}}%
\pgfpathlineto{\pgfqpoint{4.059523in}{1.451742in}}%
\pgfpathlineto{\pgfqpoint{4.064678in}{1.451000in}}%
\pgfpathlineto{\pgfqpoint{4.077564in}{1.449895in}}%
\pgfpathlineto{\pgfqpoint{4.082718in}{1.449192in}}%
\pgfpathlineto{\pgfqpoint{4.095604in}{1.448115in}}%
\pgfpathlineto{\pgfqpoint{4.100759in}{1.447453in}}%
\pgfpathlineto{\pgfqpoint{4.113645in}{1.446529in}}%
\pgfpathlineto{\pgfqpoint{4.118799in}{1.445830in}}%
\pgfpathlineto{\pgfqpoint{4.131685in}{1.444646in}}%
\pgfpathlineto{\pgfqpoint{4.134263in}{1.444261in}}%
\pgfpathlineto{\pgfqpoint{4.147149in}{1.443461in}}%
\pgfpathlineto{\pgfqpoint{4.154880in}{1.442206in}}%
\pgfpathlineto{\pgfqpoint{4.167766in}{1.441097in}}%
\pgfpathlineto{\pgfqpoint{4.172921in}{1.440381in}}%
\pgfpathlineto{\pgfqpoint{4.185807in}{1.439312in}}%
\pgfpathlineto{\pgfqpoint{4.190961in}{1.438485in}}%
\pgfpathlineto{\pgfqpoint{4.201270in}{1.437621in}}%
\pgfpathlineto{\pgfqpoint{4.209002in}{1.436365in}}%
\pgfpathlineto{\pgfqpoint{4.219311in}{1.435527in}}%
\pgfpathlineto{\pgfqpoint{4.227042in}{1.434348in}}%
\pgfpathlineto{\pgfqpoint{4.239928in}{1.433508in}}%
\pgfpathlineto{\pgfqpoint{4.245083in}{1.432923in}}%
\pgfpathlineto{\pgfqpoint{4.257969in}{1.432071in}}%
\pgfpathlineto{\pgfqpoint{4.263123in}{1.431323in}}%
\pgfpathlineto{\pgfqpoint{4.273432in}{1.430471in}}%
\pgfpathlineto{\pgfqpoint{4.281164in}{1.429269in}}%
\pgfpathlineto{\pgfqpoint{4.291472in}{1.428451in}}%
\pgfpathlineto{\pgfqpoint{4.299204in}{1.427201in}}%
\pgfpathlineto{\pgfqpoint{4.312090in}{1.426380in}}%
\pgfpathlineto{\pgfqpoint{4.317245in}{1.425564in}}%
\pgfpathlineto{\pgfqpoint{4.330131in}{1.424385in}}%
\pgfpathlineto{\pgfqpoint{4.335285in}{1.423606in}}%
\pgfpathlineto{\pgfqpoint{4.348171in}{1.422420in}}%
\pgfpathlineto{\pgfqpoint{4.353326in}{1.421606in}}%
\pgfpathlineto{\pgfqpoint{4.366212in}{1.420368in}}%
\pgfpathlineto{\pgfqpoint{4.371366in}{1.419550in}}%
\pgfpathlineto{\pgfqpoint{4.384252in}{1.418439in}}%
\pgfpathlineto{\pgfqpoint{4.389407in}{1.417713in}}%
\pgfpathlineto{\pgfqpoint{4.402293in}{1.416598in}}%
\pgfpathlineto{\pgfqpoint{4.407447in}{1.415866in}}%
\pgfpathlineto{\pgfqpoint{4.420333in}{1.414764in}}%
\pgfpathlineto{\pgfqpoint{4.425488in}{1.413966in}}%
\pgfpathlineto{\pgfqpoint{4.438374in}{1.412777in}}%
\pgfpathlineto{\pgfqpoint{4.443528in}{1.411983in}}%
\pgfpathlineto{\pgfqpoint{4.456414in}{1.410799in}}%
\pgfpathlineto{\pgfqpoint{4.461569in}{1.410014in}}%
\pgfpathlineto{\pgfqpoint{4.474455in}{1.408830in}}%
\pgfpathlineto{\pgfqpoint{4.479609in}{1.408047in}}%
\pgfpathlineto{\pgfqpoint{4.492495in}{1.406877in}}%
\pgfpathlineto{\pgfqpoint{4.497649in}{1.406096in}}%
\pgfpathlineto{\pgfqpoint{4.510536in}{1.404932in}}%
\pgfpathlineto{\pgfqpoint{4.544039in}{1.401889in}}%
\pgfpathlineto{\pgfqpoint{4.551771in}{1.400799in}}%
\pgfpathlineto{\pgfqpoint{4.564657in}{1.399705in}}%
\pgfpathlineto{\pgfqpoint{4.569811in}{1.399011in}}%
\pgfpathlineto{\pgfqpoint{4.582697in}{1.397956in}}%
\pgfpathlineto{\pgfqpoint{4.585275in}{1.397591in}}%
\pgfpathlineto{\pgfqpoint{4.600738in}{1.396499in}}%
\pgfpathlineto{\pgfqpoint{4.603315in}{1.396143in}}%
\pgfpathlineto{\pgfqpoint{4.621356in}{1.395050in}}%
\pgfpathlineto{\pgfqpoint{4.641973in}{1.393769in}}%
\pgfpathlineto{\pgfqpoint{4.667746in}{1.393686in}}%
\pgfpathlineto{\pgfqpoint{4.711558in}{1.398090in}}%
\pgfpathlineto{\pgfqpoint{4.714135in}{1.398474in}}%
\pgfpathlineto{\pgfqpoint{4.739907in}{1.399776in}}%
\pgfpathlineto{\pgfqpoint{4.760525in}{1.400782in}}%
\pgfpathlineto{\pgfqpoint{4.778566in}{1.400737in}}%
\pgfpathlineto{\pgfqpoint{4.837842in}{1.399130in}}%
\pgfpathlineto{\pgfqpoint{4.858459in}{1.398502in}}%
\pgfpathlineto{\pgfqpoint{4.873922in}{1.397854in}}%
\pgfpathlineto{\pgfqpoint{4.966702in}{1.391064in}}%
\pgfpathlineto{\pgfqpoint{4.979588in}{1.390278in}}%
\pgfpathlineto{\pgfqpoint{4.984743in}{1.389718in}}%
\pgfpathlineto{\pgfqpoint{5.000206in}{1.388841in}}%
\pgfpathlineto{\pgfqpoint{5.020824in}{1.387134in}}%
\pgfpathlineto{\pgfqpoint{5.051750in}{1.385649in}}%
\pgfpathlineto{\pgfqpoint{5.056905in}{1.385313in}}%
\pgfpathlineto{\pgfqpoint{5.085254in}{1.384793in}}%
\pgfpathlineto{\pgfqpoint{5.118758in}{1.383006in}}%
\pgfpathlineto{\pgfqpoint{5.147107in}{1.380700in}}%
\pgfpathlineto{\pgfqpoint{5.162570in}{1.379725in}}%
\pgfpathlineto{\pgfqpoint{5.175456in}{1.378888in}}%
\pgfpathlineto{\pgfqpoint{5.201228in}{1.376694in}}%
\pgfpathlineto{\pgfqpoint{5.216692in}{1.375612in}}%
\pgfpathlineto{\pgfqpoint{5.237309in}{1.374005in}}%
\pgfpathlineto{\pgfqpoint{5.252773in}{1.373164in}}%
\pgfpathlineto{\pgfqpoint{5.265659in}{1.372375in}}%
\pgfpathlineto{\pgfqpoint{5.291431in}{1.370474in}}%
\pgfpathlineto{\pgfqpoint{5.306894in}{1.369539in}}%
\pgfpathlineto{\pgfqpoint{5.324935in}{1.368389in}}%
\pgfpathlineto{\pgfqpoint{5.345552in}{1.367421in}}%
\pgfpathlineto{\pgfqpoint{5.361016in}{1.366706in}}%
\pgfpathlineto{\pgfqpoint{5.363593in}{1.366415in}}%
\pgfpathlineto{\pgfqpoint{5.376479in}{1.365559in}}%
\pgfpathlineto{\pgfqpoint{5.381633in}{1.364993in}}%
\pgfpathlineto{\pgfqpoint{5.397097in}{1.363907in}}%
\pgfpathlineto{\pgfqpoint{5.417714in}{1.362193in}}%
\pgfpathlineto{\pgfqpoint{5.430600in}{1.361353in}}%
\pgfpathlineto{\pgfqpoint{5.435755in}{1.360781in}}%
\pgfpathlineto{\pgfqpoint{5.461527in}{1.359393in}}%
\pgfpathlineto{\pgfqpoint{5.471836in}{1.358278in}}%
\pgfpathlineto{\pgfqpoint{5.487299in}{1.357243in}}%
\pgfpathlineto{\pgfqpoint{5.507917in}{1.355778in}}%
\pgfpathlineto{\pgfqpoint{5.523380in}{1.354842in}}%
\pgfpathlineto{\pgfqpoint{5.525957in}{1.354610in}}%
\pgfpathlineto{\pgfqpoint{5.541421in}{1.353886in}}%
\pgfpathlineto{\pgfqpoint{5.543998in}{1.353639in}}%
\pgfpathlineto{\pgfqpoint{5.569770in}{1.352589in}}%
\pgfpathlineto{\pgfqpoint{5.580079in}{1.351867in}}%
\pgfpathlineto{\pgfqpoint{5.605851in}{1.350946in}}%
\pgfpathlineto{\pgfqpoint{5.641932in}{1.349244in}}%
\pgfpathlineto{\pgfqpoint{5.665127in}{1.348376in}}%
\pgfpathlineto{\pgfqpoint{5.696053in}{1.347810in}}%
\pgfpathlineto{\pgfqpoint{5.809451in}{1.344053in}}%
\pgfpathlineto{\pgfqpoint{5.812028in}{1.343859in}}%
\pgfpathlineto{\pgfqpoint{5.866149in}{1.342422in}}%
\pgfpathlineto{\pgfqpoint{5.922848in}{1.340057in}}%
\pgfpathlineto{\pgfqpoint{6.010473in}{1.339147in}}%
\pgfpathlineto{\pgfqpoint{6.219227in}{1.347803in}}%
\pgfpathlineto{\pgfqpoint{6.229536in}{1.349030in}}%
\pgfpathlineto{\pgfqpoint{6.242422in}{1.349979in}}%
\pgfpathlineto{\pgfqpoint{6.247577in}{1.350793in}}%
\pgfpathlineto{\pgfqpoint{6.257886in}{1.351691in}}%
\pgfpathlineto{\pgfqpoint{6.265617in}{1.353123in}}%
\pgfpathlineto{\pgfqpoint{6.275926in}{1.354160in}}%
\pgfpathlineto{\pgfqpoint{6.283658in}{1.355776in}}%
\pgfpathlineto{\pgfqpoint{6.293967in}{1.356797in}}%
\pgfpathlineto{\pgfqpoint{6.301698in}{1.358354in}}%
\pgfpathlineto{\pgfqpoint{6.312007in}{1.359539in}}%
\pgfpathlineto{\pgfqpoint{6.319739in}{1.361731in}}%
\pgfpathlineto{\pgfqpoint{6.330048in}{1.363219in}}%
\pgfpathlineto{\pgfqpoint{6.337779in}{1.365613in}}%
\pgfpathlineto{\pgfqpoint{6.348088in}{1.367155in}}%
\pgfpathlineto{\pgfqpoint{6.355820in}{1.369054in}}%
\pgfpathlineto{\pgfqpoint{6.366129in}{1.370261in}}%
\pgfpathlineto{\pgfqpoint{6.373860in}{1.371982in}}%
\pgfpathlineto{\pgfqpoint{6.384169in}{1.373216in}}%
\pgfpathlineto{\pgfqpoint{6.386746in}{1.373803in}}%
\pgfpathlineto{\pgfqpoint{6.402209in}{1.375608in}}%
\pgfpathlineto{\pgfqpoint{6.409941in}{1.378155in}}%
\pgfpathlineto{\pgfqpoint{6.417673in}{1.379096in}}%
\pgfpathlineto{\pgfqpoint{6.427982in}{1.382795in}}%
\pgfpathlineto{\pgfqpoint{6.435713in}{1.383751in}}%
\pgfpathlineto{\pgfqpoint{6.446022in}{1.387274in}}%
\pgfpathlineto{\pgfqpoint{6.453754in}{1.388268in}}%
\pgfpathlineto{\pgfqpoint{6.464063in}{1.391882in}}%
\pgfpathlineto{\pgfqpoint{6.474371in}{1.392764in}}%
\pgfpathlineto{\pgfqpoint{6.482103in}{1.395586in}}%
\pgfpathlineto{\pgfqpoint{6.482103in}{1.395586in}}%
\pgfusepath{stroke}%
\end{pgfscope}%
\begin{pgfscope}%
\pgfpathrectangle{\pgfqpoint{0.563921in}{0.521603in}}{\pgfqpoint{6.200000in}{2.642500in}}%
\pgfusepath{clip}%
\pgfsetroundcap%
\pgfsetroundjoin%
\pgfsetlinewidth{1.505625pt}%
\definecolor{currentstroke}{rgb}{0.172549,0.627451,0.172549}%
\pgfsetstrokecolor{currentstroke}%
\pgfsetdash{}{0pt}%
\pgfpathmoveto{\pgfqpoint{0.845739in}{0.641717in}}%
\pgfpathlineto{\pgfqpoint{0.848317in}{0.646186in}}%
\pgfpathlineto{\pgfqpoint{0.850894in}{0.645797in}}%
\pgfpathlineto{\pgfqpoint{0.853471in}{0.646472in}}%
\pgfpathlineto{\pgfqpoint{0.861203in}{0.649087in}}%
\pgfpathlineto{\pgfqpoint{0.863780in}{0.648748in}}%
\pgfpathlineto{\pgfqpoint{0.868934in}{0.650005in}}%
\pgfpathlineto{\pgfqpoint{0.881820in}{0.649413in}}%
\pgfpathlineto{\pgfqpoint{0.889552in}{0.651848in}}%
\pgfpathlineto{\pgfqpoint{0.935942in}{0.650998in}}%
\pgfpathlineto{\pgfqpoint{0.941096in}{0.651392in}}%
\pgfpathlineto{\pgfqpoint{0.953982in}{0.651004in}}%
\pgfpathlineto{\pgfqpoint{0.961714in}{0.651138in}}%
\pgfpathlineto{\pgfqpoint{0.974600in}{0.652163in}}%
\pgfpathlineto{\pgfqpoint{0.979754in}{0.653092in}}%
\pgfpathlineto{\pgfqpoint{1.023567in}{0.653098in}}%
\pgfpathlineto{\pgfqpoint{1.028722in}{0.655233in}}%
\pgfpathlineto{\pgfqpoint{1.033876in}{0.659190in}}%
\pgfpathlineto{\pgfqpoint{1.044185in}{0.661788in}}%
\pgfpathlineto{\pgfqpoint{1.049339in}{0.663127in}}%
\pgfpathlineto{\pgfqpoint{1.082843in}{0.666623in}}%
\pgfpathlineto{\pgfqpoint{1.136964in}{0.664768in}}%
\pgfpathlineto{\pgfqpoint{1.209126in}{0.663750in}}%
\pgfpathlineto{\pgfqpoint{1.232321in}{0.663311in}}%
\pgfpathlineto{\pgfqpoint{1.250362in}{0.663638in}}%
\pgfpathlineto{\pgfqpoint{1.276134in}{0.663776in}}%
\pgfpathlineto{\pgfqpoint{1.286443in}{0.664486in}}%
\pgfpathlineto{\pgfqpoint{1.299329in}{0.665051in}}%
\pgfpathlineto{\pgfqpoint{1.301906in}{0.665475in}}%
\pgfpathlineto{\pgfqpoint{1.304483in}{0.666532in}}%
\pgfpathlineto{\pgfqpoint{1.335410in}{0.668448in}}%
\pgfpathlineto{\pgfqpoint{1.353450in}{0.668482in}}%
\pgfpathlineto{\pgfqpoint{1.371491in}{0.668981in}}%
\pgfpathlineto{\pgfqpoint{1.394686in}{0.672655in}}%
\pgfpathlineto{\pgfqpoint{1.404995in}{0.673929in}}%
\pgfpathlineto{\pgfqpoint{1.412726in}{0.675643in}}%
\pgfpathlineto{\pgfqpoint{1.425612in}{0.676966in}}%
\pgfpathlineto{\pgfqpoint{1.430767in}{0.677831in}}%
\pgfpathlineto{\pgfqpoint{1.459116in}{0.679641in}}%
\pgfpathlineto{\pgfqpoint{1.466848in}{0.680174in}}%
\pgfpathlineto{\pgfqpoint{1.482311in}{0.680832in}}%
\pgfpathlineto{\pgfqpoint{1.484888in}{0.681434in}}%
\pgfpathlineto{\pgfqpoint{1.495197in}{0.682499in}}%
\pgfpathlineto{\pgfqpoint{1.502929in}{0.684836in}}%
\pgfpathlineto{\pgfqpoint{1.510660in}{0.685580in}}%
\pgfpathlineto{\pgfqpoint{1.520969in}{0.689437in}}%
\pgfpathlineto{\pgfqpoint{1.531278in}{0.691146in}}%
\pgfpathlineto{\pgfqpoint{1.536432in}{0.692843in}}%
\pgfpathlineto{\pgfqpoint{1.539010in}{0.693830in}}%
\pgfpathlineto{\pgfqpoint{1.546741in}{0.694837in}}%
\pgfpathlineto{\pgfqpoint{1.557050in}{0.698835in}}%
\pgfpathlineto{\pgfqpoint{1.567359in}{0.700412in}}%
\pgfpathlineto{\pgfqpoint{1.575091in}{0.702106in}}%
\pgfpathlineto{\pgfqpoint{1.585399in}{0.703308in}}%
\pgfpathlineto{\pgfqpoint{1.593131in}{0.704932in}}%
\pgfpathlineto{\pgfqpoint{1.629212in}{0.705200in}}%
\pgfpathlineto{\pgfqpoint{1.660139in}{0.704924in}}%
\pgfpathlineto{\pgfqpoint{1.678179in}{0.704367in}}%
\pgfpathlineto{\pgfqpoint{1.737455in}{0.704397in}}%
\pgfpathlineto{\pgfqpoint{1.763227in}{0.704437in}}%
\pgfpathlineto{\pgfqpoint{1.825080in}{0.704058in}}%
\pgfpathlineto{\pgfqpoint{1.827657in}{0.704245in}}%
\pgfpathlineto{\pgfqpoint{1.845698in}{0.704984in}}%
\pgfpathlineto{\pgfqpoint{1.861161in}{0.705998in}}%
\pgfpathlineto{\pgfqpoint{1.863738in}{0.706314in}}%
\pgfpathlineto{\pgfqpoint{1.879202in}{0.707301in}}%
\pgfpathlineto{\pgfqpoint{1.915283in}{0.711182in}}%
\pgfpathlineto{\pgfqpoint{1.917860in}{0.711714in}}%
\pgfpathlineto{\pgfqpoint{1.928169in}{0.712424in}}%
\pgfpathlineto{\pgfqpoint{1.935900in}{0.713752in}}%
\pgfpathlineto{\pgfqpoint{1.946209in}{0.714707in}}%
\pgfpathlineto{\pgfqpoint{1.953941in}{0.716353in}}%
\pgfpathlineto{\pgfqpoint{1.966827in}{0.717693in}}%
\pgfpathlineto{\pgfqpoint{1.971981in}{0.718577in}}%
\pgfpathlineto{\pgfqpoint{1.984867in}{0.719632in}}%
\pgfpathlineto{\pgfqpoint{1.990022in}{0.720294in}}%
\pgfpathlineto{\pgfqpoint{2.018371in}{0.721869in}}%
\pgfpathlineto{\pgfqpoint{2.051875in}{0.723973in}}%
\pgfpathlineto{\pgfqpoint{2.062184in}{0.724327in}}%
\pgfpathlineto{\pgfqpoint{2.105996in}{0.724332in}}%
\pgfpathlineto{\pgfqpoint{2.144655in}{0.726071in}}%
\pgfpathlineto{\pgfqpoint{2.152386in}{0.726835in}}%
\pgfpathlineto{\pgfqpoint{2.178158in}{0.727882in}}%
\pgfpathlineto{\pgfqpoint{2.196199in}{0.728895in}}%
\pgfpathlineto{\pgfqpoint{2.224548in}{0.730713in}}%
\pgfpathlineto{\pgfqpoint{2.255475in}{0.731533in}}%
\pgfpathlineto{\pgfqpoint{2.286401in}{0.732725in}}%
\pgfpathlineto{\pgfqpoint{2.304442in}{0.733999in}}%
\pgfpathlineto{\pgfqpoint{2.314751in}{0.735461in}}%
\pgfpathlineto{\pgfqpoint{2.327637in}{0.736348in}}%
\pgfpathlineto{\pgfqpoint{2.332791in}{0.737013in}}%
\pgfpathlineto{\pgfqpoint{2.348254in}{0.738042in}}%
\pgfpathlineto{\pgfqpoint{2.366295in}{0.739034in}}%
\pgfpathlineto{\pgfqpoint{2.386912in}{0.739747in}}%
\pgfpathlineto{\pgfqpoint{2.451343in}{0.740844in}}%
\pgfpathlineto{\pgfqpoint{2.459074in}{0.741503in}}%
\pgfpathlineto{\pgfqpoint{2.487424in}{0.742676in}}%
\pgfpathlineto{\pgfqpoint{2.502887in}{0.743254in}}%
\pgfpathlineto{\pgfqpoint{2.528659in}{0.744438in}}%
\pgfpathlineto{\pgfqpoint{2.531236in}{0.744831in}}%
\pgfpathlineto{\pgfqpoint{2.544122in}{0.746253in}}%
\pgfpathlineto{\pgfqpoint{2.549277in}{0.747145in}}%
\pgfpathlineto{\pgfqpoint{2.559586in}{0.748123in}}%
\pgfpathlineto{\pgfqpoint{2.567317in}{0.749680in}}%
\pgfpathlineto{\pgfqpoint{2.577626in}{0.750714in}}%
\pgfpathlineto{\pgfqpoint{2.585358in}{0.752523in}}%
\pgfpathlineto{\pgfqpoint{2.595667in}{0.753775in}}%
\pgfpathlineto{\pgfqpoint{2.603398in}{0.755644in}}%
\pgfpathlineto{\pgfqpoint{2.613707in}{0.756848in}}%
\pgfpathlineto{\pgfqpoint{2.621439in}{0.758495in}}%
\pgfpathlineto{\pgfqpoint{2.631748in}{0.759451in}}%
\pgfpathlineto{\pgfqpoint{2.639479in}{0.760379in}}%
\pgfpathlineto{\pgfqpoint{2.652365in}{0.761649in}}%
\pgfpathlineto{\pgfqpoint{2.675560in}{0.764698in}}%
\pgfpathlineto{\pgfqpoint{2.685869in}{0.765595in}}%
\pgfpathlineto{\pgfqpoint{2.693601in}{0.767215in}}%
\pgfpathlineto{\pgfqpoint{2.703910in}{0.768354in}}%
\pgfpathlineto{\pgfqpoint{2.765763in}{0.775984in}}%
\pgfpathlineto{\pgfqpoint{2.899778in}{0.779611in}}%
\pgfpathlineto{\pgfqpoint{2.910087in}{0.780062in}}%
\pgfpathlineto{\pgfqpoint{2.943590in}{0.780970in}}%
\pgfpathlineto{\pgfqpoint{2.977094in}{0.782232in}}%
\pgfpathlineto{\pgfqpoint{3.031216in}{0.784609in}}%
\pgfpathlineto{\pgfqpoint{3.046679in}{0.785340in}}%
\pgfpathlineto{\pgfqpoint{3.072451in}{0.786811in}}%
\pgfpathlineto{\pgfqpoint{3.105955in}{0.788065in}}%
\pgfpathlineto{\pgfqpoint{3.126572in}{0.789137in}}%
\pgfpathlineto{\pgfqpoint{3.154922in}{0.790552in}}%
\pgfpathlineto{\pgfqpoint{3.180694in}{0.791810in}}%
\pgfpathlineto{\pgfqpoint{3.214198in}{0.792841in}}%
\pgfpathlineto{\pgfqpoint{3.242547in}{0.793739in}}%
\pgfpathlineto{\pgfqpoint{3.270896in}{0.794033in}}%
\pgfpathlineto{\pgfqpoint{3.322441in}{0.794535in}}%
\pgfpathlineto{\pgfqpoint{3.361099in}{0.795046in}}%
\pgfpathlineto{\pgfqpoint{3.482228in}{0.795299in}}%
\pgfpathlineto{\pgfqpoint{3.536349in}{0.795893in}}%
\pgfpathlineto{\pgfqpoint{3.595625in}{0.797221in}}%
\pgfpathlineto{\pgfqpoint{3.685828in}{0.796790in}}%
\pgfpathlineto{\pgfqpoint{3.755412in}{0.795583in}}%
\pgfpathlineto{\pgfqpoint{3.794070in}{0.795314in}}%
\pgfpathlineto{\pgfqpoint{3.920354in}{0.795760in}}%
\pgfpathlineto{\pgfqpoint{3.969321in}{0.797467in}}%
\pgfpathlineto{\pgfqpoint{4.010556in}{0.798906in}}%
\pgfpathlineto{\pgfqpoint{4.041483in}{0.799939in}}%
\pgfpathlineto{\pgfqpoint{4.064678in}{0.800817in}}%
\pgfpathlineto{\pgfqpoint{4.095604in}{0.801754in}}%
\pgfpathlineto{\pgfqpoint{4.118799in}{0.802628in}}%
\pgfpathlineto{\pgfqpoint{4.281164in}{0.803621in}}%
\pgfpathlineto{\pgfqpoint{4.389407in}{0.803431in}}%
\pgfpathlineto{\pgfqpoint{4.407447in}{0.804344in}}%
\pgfpathlineto{\pgfqpoint{4.422910in}{0.805427in}}%
\pgfpathlineto{\pgfqpoint{4.425488in}{0.805744in}}%
\pgfpathlineto{\pgfqpoint{4.438374in}{0.806666in}}%
\pgfpathlineto{\pgfqpoint{4.443528in}{0.807197in}}%
\pgfpathlineto{\pgfqpoint{4.456414in}{0.808106in}}%
\pgfpathlineto{\pgfqpoint{4.461569in}{0.808762in}}%
\pgfpathlineto{\pgfqpoint{4.471877in}{0.809446in}}%
\pgfpathlineto{\pgfqpoint{4.479609in}{0.810609in}}%
\pgfpathlineto{\pgfqpoint{4.492495in}{0.811758in}}%
\pgfpathlineto{\pgfqpoint{4.497649in}{0.812528in}}%
\pgfpathlineto{\pgfqpoint{4.510536in}{0.813687in}}%
\pgfpathlineto{\pgfqpoint{4.546617in}{0.816709in}}%
\pgfpathlineto{\pgfqpoint{4.551771in}{0.817409in}}%
\pgfpathlineto{\pgfqpoint{4.564657in}{0.818466in}}%
\pgfpathlineto{\pgfqpoint{4.569811in}{0.819185in}}%
\pgfpathlineto{\pgfqpoint{4.582697in}{0.820318in}}%
\pgfpathlineto{\pgfqpoint{4.585275in}{0.820716in}}%
\pgfpathlineto{\pgfqpoint{4.598161in}{0.821566in}}%
\pgfpathlineto{\pgfqpoint{4.603315in}{0.822407in}}%
\pgfpathlineto{\pgfqpoint{4.618778in}{0.823463in}}%
\pgfpathlineto{\pgfqpoint{4.639396in}{0.824410in}}%
\pgfpathlineto{\pgfqpoint{4.660014in}{0.825054in}}%
\pgfpathlineto{\pgfqpoint{4.688363in}{0.825929in}}%
\pgfpathlineto{\pgfqpoint{4.714135in}{0.826884in}}%
\pgfpathlineto{\pgfqpoint{4.739907in}{0.827765in}}%
\pgfpathlineto{\pgfqpoint{4.814647in}{0.833403in}}%
\pgfpathlineto{\pgfqpoint{4.819801in}{0.834103in}}%
\pgfpathlineto{\pgfqpoint{4.832687in}{0.834880in}}%
\pgfpathlineto{\pgfqpoint{4.840419in}{0.836151in}}%
\pgfpathlineto{\pgfqpoint{4.853305in}{0.837135in}}%
\pgfpathlineto{\pgfqpoint{4.876500in}{0.839235in}}%
\pgfpathlineto{\pgfqpoint{4.889386in}{0.840189in}}%
\pgfpathlineto{\pgfqpoint{4.894540in}{0.840767in}}%
\pgfpathlineto{\pgfqpoint{4.907426in}{0.841612in}}%
\pgfpathlineto{\pgfqpoint{4.912581in}{0.842165in}}%
\pgfpathlineto{\pgfqpoint{4.940930in}{0.843712in}}%
\pgfpathlineto{\pgfqpoint{4.956393in}{0.844478in}}%
\pgfpathlineto{\pgfqpoint{4.984743in}{0.845964in}}%
\pgfpathlineto{\pgfqpoint{5.010515in}{0.846887in}}%
\pgfpathlineto{\pgfqpoint{5.056905in}{0.849903in}}%
\pgfpathlineto{\pgfqpoint{5.069791in}{0.850423in}}%
\pgfpathlineto{\pgfqpoint{5.074945in}{0.851029in}}%
\pgfpathlineto{\pgfqpoint{5.090408in}{0.851979in}}%
\pgfpathlineto{\pgfqpoint{5.092986in}{0.852344in}}%
\pgfpathlineto{\pgfqpoint{5.105872in}{0.853452in}}%
\pgfpathlineto{\pgfqpoint{5.111026in}{0.854280in}}%
\pgfpathlineto{\pgfqpoint{5.123912in}{0.855540in}}%
\pgfpathlineto{\pgfqpoint{5.129067in}{0.856256in}}%
\pgfpathlineto{\pgfqpoint{5.144530in}{0.857322in}}%
\pgfpathlineto{\pgfqpoint{5.165148in}{0.858745in}}%
\pgfpathlineto{\pgfqpoint{5.180611in}{0.859713in}}%
\pgfpathlineto{\pgfqpoint{5.201228in}{0.861124in}}%
\pgfpathlineto{\pgfqpoint{5.216692in}{0.862029in}}%
\pgfpathlineto{\pgfqpoint{5.237309in}{0.863373in}}%
\pgfpathlineto{\pgfqpoint{5.268236in}{0.864470in}}%
\pgfpathlineto{\pgfqpoint{5.309471in}{0.865841in}}%
\pgfpathlineto{\pgfqpoint{5.415137in}{0.867698in}}%
\pgfpathlineto{\pgfqpoint{5.435755in}{0.868651in}}%
\pgfpathlineto{\pgfqpoint{5.464104in}{0.869789in}}%
\pgfpathlineto{\pgfqpoint{5.562038in}{0.875350in}}%
\pgfpathlineto{\pgfqpoint{5.592965in}{0.876672in}}%
\pgfpathlineto{\pgfqpoint{5.605851in}{0.877053in}}%
\pgfpathlineto{\pgfqpoint{5.649663in}{0.878107in}}%
\pgfpathlineto{\pgfqpoint{5.760484in}{0.880796in}}%
\pgfpathlineto{\pgfqpoint{5.804296in}{0.881646in}}%
\pgfpathlineto{\pgfqpoint{5.832646in}{0.882234in}}%
\pgfpathlineto{\pgfqpoint{5.933157in}{0.882468in}}%
\pgfpathlineto{\pgfqpoint{5.976969in}{0.882394in}}%
\pgfpathlineto{\pgfqpoint{6.031091in}{0.882105in}}%
\pgfpathlineto{\pgfqpoint{6.100676in}{0.881034in}}%
\pgfpathlineto{\pgfqpoint{6.211496in}{0.878718in}}%
\pgfpathlineto{\pgfqpoint{6.263040in}{0.877660in}}%
\pgfpathlineto{\pgfqpoint{6.319739in}{0.876455in}}%
\pgfpathlineto{\pgfqpoint{6.404787in}{0.876523in}}%
\pgfpathlineto{\pgfqpoint{6.482103in}{0.877577in}}%
\pgfpathlineto{\pgfqpoint{6.482103in}{0.877577in}}%
\pgfusepath{stroke}%
\end{pgfscope}%
\begin{pgfscope}%
\pgfpathrectangle{\pgfqpoint{0.563921in}{0.521603in}}{\pgfqpoint{6.200000in}{2.642500in}}%
\pgfusepath{clip}%
\pgfsetroundcap%
\pgfsetroundjoin%
\pgfsetlinewidth{1.505625pt}%
\definecolor{currentstroke}{rgb}{0.839216,0.152941,0.156863}%
\pgfsetstrokecolor{currentstroke}%
\pgfsetdash{}{0pt}%
\pgfpathmoveto{\pgfqpoint{0.845739in}{0.641717in}}%
\pgfpathlineto{\pgfqpoint{0.848317in}{0.655422in}}%
\pgfpathlineto{\pgfqpoint{0.850894in}{0.658811in}}%
\pgfpathlineto{\pgfqpoint{0.853471in}{0.657284in}}%
\pgfpathlineto{\pgfqpoint{0.863780in}{0.658051in}}%
\pgfpathlineto{\pgfqpoint{0.866357in}{0.659514in}}%
\pgfpathlineto{\pgfqpoint{0.868934in}{0.659704in}}%
\pgfpathlineto{\pgfqpoint{0.871512in}{0.658998in}}%
\pgfpathlineto{\pgfqpoint{0.881820in}{0.658654in}}%
\pgfpathlineto{\pgfqpoint{0.884398in}{0.657905in}}%
\pgfpathlineto{\pgfqpoint{0.886975in}{0.657753in}}%
\pgfpathlineto{\pgfqpoint{0.889552in}{0.661881in}}%
\pgfpathlineto{\pgfqpoint{0.897284in}{0.666631in}}%
\pgfpathlineto{\pgfqpoint{0.899861in}{0.670761in}}%
\pgfpathlineto{\pgfqpoint{0.902438in}{0.673459in}}%
\pgfpathlineto{\pgfqpoint{0.907593in}{0.675500in}}%
\pgfpathlineto{\pgfqpoint{0.915324in}{0.676056in}}%
\pgfpathlineto{\pgfqpoint{0.923056in}{0.675251in}}%
\pgfpathlineto{\pgfqpoint{0.925633in}{0.675946in}}%
\pgfpathlineto{\pgfqpoint{0.938519in}{0.677104in}}%
\pgfpathlineto{\pgfqpoint{0.943674in}{0.677541in}}%
\pgfpathlineto{\pgfqpoint{0.959137in}{0.677399in}}%
\pgfpathlineto{\pgfqpoint{0.961714in}{0.678468in}}%
\pgfpathlineto{\pgfqpoint{0.974600in}{0.678711in}}%
\pgfpathlineto{\pgfqpoint{0.979754in}{0.678169in}}%
\pgfpathlineto{\pgfqpoint{1.008104in}{0.677418in}}%
\pgfpathlineto{\pgfqpoint{1.023567in}{0.677060in}}%
\pgfpathlineto{\pgfqpoint{1.033876in}{0.679369in}}%
\pgfpathlineto{\pgfqpoint{1.044185in}{0.680509in}}%
\pgfpathlineto{\pgfqpoint{1.069957in}{0.685428in}}%
\pgfpathlineto{\pgfqpoint{1.085420in}{0.687203in}}%
\pgfpathlineto{\pgfqpoint{1.103461in}{0.687789in}}%
\pgfpathlineto{\pgfqpoint{1.121501in}{0.688944in}}%
\pgfpathlineto{\pgfqpoint{1.136964in}{0.688271in}}%
\pgfpathlineto{\pgfqpoint{1.152428in}{0.689859in}}%
\pgfpathlineto{\pgfqpoint{1.157582in}{0.691364in}}%
\pgfpathlineto{\pgfqpoint{1.173045in}{0.690795in}}%
\pgfpathlineto{\pgfqpoint{1.178200in}{0.690419in}}%
\pgfpathlineto{\pgfqpoint{1.214281in}{0.689728in}}%
\pgfpathlineto{\pgfqpoint{1.232321in}{0.689910in}}%
\pgfpathlineto{\pgfqpoint{1.247785in}{0.690351in}}%
\pgfpathlineto{\pgfqpoint{1.268402in}{0.689318in}}%
\pgfpathlineto{\pgfqpoint{1.301906in}{0.688366in}}%
\pgfpathlineto{\pgfqpoint{1.314792in}{0.687818in}}%
\pgfpathlineto{\pgfqpoint{1.335410in}{0.687670in}}%
\pgfpathlineto{\pgfqpoint{1.340564in}{0.688441in}}%
\pgfpathlineto{\pgfqpoint{1.376645in}{0.689271in}}%
\pgfpathlineto{\pgfqpoint{1.404995in}{0.688564in}}%
\pgfpathlineto{\pgfqpoint{1.430767in}{0.687470in}}%
\pgfpathlineto{\pgfqpoint{1.448807in}{0.687446in}}%
\pgfpathlineto{\pgfqpoint{1.461693in}{0.688065in}}%
\pgfpathlineto{\pgfqpoint{1.466848in}{0.688736in}}%
\pgfpathlineto{\pgfqpoint{1.482311in}{0.689542in}}%
\pgfpathlineto{\pgfqpoint{1.495197in}{0.692087in}}%
\pgfpathlineto{\pgfqpoint{1.502929in}{0.694862in}}%
\pgfpathlineto{\pgfqpoint{1.513237in}{0.696493in}}%
\pgfpathlineto{\pgfqpoint{1.520969in}{0.699066in}}%
\pgfpathlineto{\pgfqpoint{1.528701in}{0.700098in}}%
\pgfpathlineto{\pgfqpoint{1.536432in}{0.703049in}}%
\pgfpathlineto{\pgfqpoint{1.539010in}{0.704018in}}%
\pgfpathlineto{\pgfqpoint{1.549318in}{0.705679in}}%
\pgfpathlineto{\pgfqpoint{1.557050in}{0.708376in}}%
\pgfpathlineto{\pgfqpoint{1.564782in}{0.709241in}}%
\pgfpathlineto{\pgfqpoint{1.575091in}{0.714406in}}%
\pgfpathlineto{\pgfqpoint{1.585399in}{0.716305in}}%
\pgfpathlineto{\pgfqpoint{1.593131in}{0.719722in}}%
\pgfpathlineto{\pgfqpoint{1.600863in}{0.720851in}}%
\pgfpathlineto{\pgfqpoint{1.611172in}{0.724529in}}%
\pgfpathlineto{\pgfqpoint{1.626635in}{0.725936in}}%
\pgfpathlineto{\pgfqpoint{1.629212in}{0.726548in}}%
\pgfpathlineto{\pgfqpoint{1.639521in}{0.727769in}}%
\pgfpathlineto{\pgfqpoint{1.647253in}{0.730846in}}%
\pgfpathlineto{\pgfqpoint{1.654984in}{0.731845in}}%
\pgfpathlineto{\pgfqpoint{1.665293in}{0.736961in}}%
\pgfpathlineto{\pgfqpoint{1.673025in}{0.738062in}}%
\pgfpathlineto{\pgfqpoint{1.678179in}{0.741079in}}%
\pgfpathlineto{\pgfqpoint{1.683333in}{0.742352in}}%
\pgfpathlineto{\pgfqpoint{1.691065in}{0.743493in}}%
\pgfpathlineto{\pgfqpoint{1.701374in}{0.747944in}}%
\pgfpathlineto{\pgfqpoint{1.711683in}{0.750003in}}%
\pgfpathlineto{\pgfqpoint{1.719414in}{0.752482in}}%
\pgfpathlineto{\pgfqpoint{1.732301in}{0.754283in}}%
\pgfpathlineto{\pgfqpoint{1.737455in}{0.755369in}}%
\pgfpathlineto{\pgfqpoint{1.752918in}{0.756856in}}%
\pgfpathlineto{\pgfqpoint{1.755495in}{0.757249in}}%
\pgfpathlineto{\pgfqpoint{1.770959in}{0.758537in}}%
\pgfpathlineto{\pgfqpoint{1.773536in}{0.759073in}}%
\pgfpathlineto{\pgfqpoint{1.809617in}{0.760598in}}%
\pgfpathlineto{\pgfqpoint{1.840543in}{0.760987in}}%
\pgfpathlineto{\pgfqpoint{1.853430in}{0.761586in}}%
\pgfpathlineto{\pgfqpoint{1.889510in}{0.762742in}}%
\pgfpathlineto{\pgfqpoint{1.964250in}{0.763954in}}%
\pgfpathlineto{\pgfqpoint{1.990022in}{0.763881in}}%
\pgfpathlineto{\pgfqpoint{2.038989in}{0.763439in}}%
\pgfpathlineto{\pgfqpoint{2.129191in}{0.759755in}}%
\pgfpathlineto{\pgfqpoint{2.152386in}{0.758658in}}%
\pgfpathlineto{\pgfqpoint{2.196199in}{0.757522in}}%
\pgfpathlineto{\pgfqpoint{2.242589in}{0.755885in}}%
\pgfpathlineto{\pgfqpoint{2.276092in}{0.754717in}}%
\pgfpathlineto{\pgfqpoint{2.296710in}{0.753888in}}%
\pgfpathlineto{\pgfqpoint{2.325059in}{0.752935in}}%
\pgfpathlineto{\pgfqpoint{2.350832in}{0.751880in}}%
\pgfpathlineto{\pgfqpoint{2.397221in}{0.750569in}}%
\pgfpathlineto{\pgfqpoint{2.441034in}{0.749209in}}%
\pgfpathlineto{\pgfqpoint{2.474538in}{0.748160in}}%
\pgfpathlineto{\pgfqpoint{2.513196in}{0.746893in}}%
\pgfpathlineto{\pgfqpoint{2.564740in}{0.745574in}}%
\pgfpathlineto{\pgfqpoint{2.603398in}{0.744809in}}%
\pgfpathlineto{\pgfqpoint{2.649788in}{0.744043in}}%
\pgfpathlineto{\pgfqpoint{2.675560in}{0.743639in}}%
\pgfpathlineto{\pgfqpoint{2.719373in}{0.743821in}}%
\pgfpathlineto{\pgfqpoint{2.742568in}{0.744180in}}%
\pgfpathlineto{\pgfqpoint{2.763185in}{0.744817in}}%
\pgfpathlineto{\pgfqpoint{2.783803in}{0.744890in}}%
\pgfpathlineto{\pgfqpoint{2.863697in}{0.743896in}}%
\pgfpathlineto{\pgfqpoint{2.935859in}{0.743433in}}%
\pgfpathlineto{\pgfqpoint{2.974517in}{0.744494in}}%
\pgfpathlineto{\pgfqpoint{2.982249in}{0.744992in}}%
\pgfpathlineto{\pgfqpoint{3.008021in}{0.746023in}}%
\pgfpathlineto{\pgfqpoint{3.018330in}{0.746691in}}%
\pgfpathlineto{\pgfqpoint{3.046679in}{0.747597in}}%
\pgfpathlineto{\pgfqpoint{3.072451in}{0.748582in}}%
\pgfpathlineto{\pgfqpoint{3.103378in}{0.749462in}}%
\pgfpathlineto{\pgfqpoint{3.142036in}{0.753001in}}%
\pgfpathlineto{\pgfqpoint{3.144613in}{0.753771in}}%
\pgfpathlineto{\pgfqpoint{3.154922in}{0.755336in}}%
\pgfpathlineto{\pgfqpoint{3.162653in}{0.757673in}}%
\pgfpathlineto{\pgfqpoint{3.170385in}{0.758471in}}%
\pgfpathlineto{\pgfqpoint{3.180694in}{0.762111in}}%
\pgfpathlineto{\pgfqpoint{3.188426in}{0.763022in}}%
\pgfpathlineto{\pgfqpoint{3.196157in}{0.765784in}}%
\pgfpathlineto{\pgfqpoint{3.206466in}{0.766677in}}%
\pgfpathlineto{\pgfqpoint{3.216775in}{0.770152in}}%
\pgfpathlineto{\pgfqpoint{3.224507in}{0.771104in}}%
\pgfpathlineto{\pgfqpoint{3.227084in}{0.772099in}}%
\pgfpathlineto{\pgfqpoint{3.232238in}{0.775524in}}%
\pgfpathlineto{\pgfqpoint{3.234815in}{0.777031in}}%
\pgfpathlineto{\pgfqpoint{3.242547in}{0.778627in}}%
\pgfpathlineto{\pgfqpoint{3.252856in}{0.785215in}}%
\pgfpathlineto{\pgfqpoint{3.260588in}{0.786721in}}%
\pgfpathlineto{\pgfqpoint{3.270896in}{0.792300in}}%
\pgfpathlineto{\pgfqpoint{3.278628in}{0.793643in}}%
\pgfpathlineto{\pgfqpoint{3.288937in}{0.797703in}}%
\pgfpathlineto{\pgfqpoint{3.296668in}{0.798752in}}%
\pgfpathlineto{\pgfqpoint{3.304400in}{0.802358in}}%
\pgfpathlineto{\pgfqpoint{3.306977in}{0.803641in}}%
\pgfpathlineto{\pgfqpoint{3.314709in}{0.804970in}}%
\pgfpathlineto{\pgfqpoint{3.325018in}{0.810444in}}%
\pgfpathlineto{\pgfqpoint{3.332749in}{0.811783in}}%
\pgfpathlineto{\pgfqpoint{3.343058in}{0.816937in}}%
\pgfpathlineto{\pgfqpoint{3.353367in}{0.818127in}}%
\pgfpathlineto{\pgfqpoint{3.361099in}{0.821797in}}%
\pgfpathlineto{\pgfqpoint{3.368830in}{0.823111in}}%
\pgfpathlineto{\pgfqpoint{3.379139in}{0.827781in}}%
\pgfpathlineto{\pgfqpoint{3.386871in}{0.828832in}}%
\pgfpathlineto{\pgfqpoint{3.397180in}{0.833300in}}%
\pgfpathlineto{\pgfqpoint{3.404911in}{0.834325in}}%
\pgfpathlineto{\pgfqpoint{3.415220in}{0.838049in}}%
\pgfpathlineto{\pgfqpoint{3.422952in}{0.839060in}}%
\pgfpathlineto{\pgfqpoint{3.430684in}{0.841556in}}%
\pgfpathlineto{\pgfqpoint{3.433261in}{0.842354in}}%
\pgfpathlineto{\pgfqpoint{3.443570in}{0.843837in}}%
\pgfpathlineto{\pgfqpoint{3.451301in}{0.845754in}}%
\pgfpathlineto{\pgfqpoint{3.464187in}{0.846842in}}%
\pgfpathlineto{\pgfqpoint{3.484805in}{0.849175in}}%
\pgfpathlineto{\pgfqpoint{3.487382in}{0.849739in}}%
\pgfpathlineto{\pgfqpoint{3.497691in}{0.850949in}}%
\pgfpathlineto{\pgfqpoint{3.505423in}{0.852750in}}%
\pgfpathlineto{\pgfqpoint{3.515732in}{0.854228in}}%
\pgfpathlineto{\pgfqpoint{3.523463in}{0.856146in}}%
\pgfpathlineto{\pgfqpoint{3.533772in}{0.857255in}}%
\pgfpathlineto{\pgfqpoint{3.541504in}{0.859065in}}%
\pgfpathlineto{\pgfqpoint{3.551813in}{0.860515in}}%
\pgfpathlineto{\pgfqpoint{3.577585in}{0.867446in}}%
\pgfpathlineto{\pgfqpoint{3.585316in}{0.868606in}}%
\pgfpathlineto{\pgfqpoint{3.595625in}{0.873425in}}%
\pgfpathlineto{\pgfqpoint{3.603357in}{0.874527in}}%
\pgfpathlineto{\pgfqpoint{3.613666in}{0.878339in}}%
\pgfpathlineto{\pgfqpoint{3.623974in}{0.879900in}}%
\pgfpathlineto{\pgfqpoint{3.631706in}{0.882608in}}%
\pgfpathlineto{\pgfqpoint{3.639438in}{0.883614in}}%
\pgfpathlineto{\pgfqpoint{3.644592in}{0.885684in}}%
\pgfpathlineto{\pgfqpoint{3.649747in}{0.886724in}}%
\pgfpathlineto{\pgfqpoint{3.660055in}{0.888564in}}%
\pgfpathlineto{\pgfqpoint{3.662633in}{0.889352in}}%
\pgfpathlineto{\pgfqpoint{3.683250in}{0.893011in}}%
\pgfpathlineto{\pgfqpoint{3.685828in}{0.893837in}}%
\pgfpathlineto{\pgfqpoint{3.696136in}{0.895400in}}%
\pgfpathlineto{\pgfqpoint{3.703868in}{0.897584in}}%
\pgfpathlineto{\pgfqpoint{3.714177in}{0.898263in}}%
\pgfpathlineto{\pgfqpoint{3.721909in}{0.900508in}}%
\pgfpathlineto{\pgfqpoint{3.752835in}{0.903448in}}%
\pgfpathlineto{\pgfqpoint{3.757990in}{0.904111in}}%
\pgfpathlineto{\pgfqpoint{3.770876in}{0.905006in}}%
\pgfpathlineto{\pgfqpoint{3.776030in}{0.905792in}}%
\pgfpathlineto{\pgfqpoint{3.791493in}{0.907007in}}%
\pgfpathlineto{\pgfqpoint{3.794070in}{0.907405in}}%
\pgfpathlineto{\pgfqpoint{3.806957in}{0.908453in}}%
\pgfpathlineto{\pgfqpoint{3.812111in}{0.909006in}}%
\pgfpathlineto{\pgfqpoint{3.824997in}{0.910033in}}%
\pgfpathlineto{\pgfqpoint{3.830151in}{0.910565in}}%
\pgfpathlineto{\pgfqpoint{3.879118in}{0.911233in}}%
\pgfpathlineto{\pgfqpoint{3.899736in}{0.911471in}}%
\pgfpathlineto{\pgfqpoint{3.933240in}{0.912065in}}%
\pgfpathlineto{\pgfqpoint{3.953858in}{0.912950in}}%
\pgfpathlineto{\pgfqpoint{4.026020in}{0.916312in}}%
\pgfpathlineto{\pgfqpoint{4.028597in}{0.916529in}}%
\pgfpathlineto{\pgfqpoint{4.044060in}{0.917219in}}%
\pgfpathlineto{\pgfqpoint{4.046637in}{0.917531in}}%
\pgfpathlineto{\pgfqpoint{4.100759in}{0.918848in}}%
\pgfpathlineto{\pgfqpoint{4.147149in}{0.918919in}}%
\pgfpathlineto{\pgfqpoint{4.317245in}{0.914412in}}%
\pgfpathlineto{\pgfqpoint{4.366212in}{0.913422in}}%
\pgfpathlineto{\pgfqpoint{4.384252in}{0.913535in}}%
\pgfpathlineto{\pgfqpoint{4.451260in}{0.916828in}}%
\pgfpathlineto{\pgfqpoint{4.461569in}{0.917749in}}%
\pgfpathlineto{\pgfqpoint{4.495072in}{0.918942in}}%
\pgfpathlineto{\pgfqpoint{4.497649in}{0.919218in}}%
\pgfpathlineto{\pgfqpoint{4.523422in}{0.920498in}}%
\pgfpathlineto{\pgfqpoint{4.533730in}{0.921580in}}%
\pgfpathlineto{\pgfqpoint{4.549194in}{0.922670in}}%
\pgfpathlineto{\pgfqpoint{4.569811in}{0.924176in}}%
\pgfpathlineto{\pgfqpoint{4.582697in}{0.924903in}}%
\pgfpathlineto{\pgfqpoint{4.585275in}{0.925172in}}%
\pgfpathlineto{\pgfqpoint{4.600738in}{0.926006in}}%
\pgfpathlineto{\pgfqpoint{4.603315in}{0.926222in}}%
\pgfpathlineto{\pgfqpoint{4.657437in}{0.926606in}}%
\pgfpathlineto{\pgfqpoint{4.750216in}{0.924619in}}%
\pgfpathlineto{\pgfqpoint{4.804338in}{0.924550in}}%
\pgfpathlineto{\pgfqpoint{4.930621in}{0.925049in}}%
\pgfpathlineto{\pgfqpoint{5.028555in}{0.924786in}}%
\pgfpathlineto{\pgfqpoint{5.108449in}{0.926598in}}%
\pgfpathlineto{\pgfqpoint{5.129067in}{0.927950in}}%
\pgfpathlineto{\pgfqpoint{5.144530in}{0.928780in}}%
\pgfpathlineto{\pgfqpoint{5.165148in}{0.929980in}}%
\pgfpathlineto{\pgfqpoint{5.190920in}{0.931246in}}%
\pgfpathlineto{\pgfqpoint{5.219269in}{0.933364in}}%
\pgfpathlineto{\pgfqpoint{5.232155in}{0.934177in}}%
\pgfpathlineto{\pgfqpoint{5.237309in}{0.934769in}}%
\pgfpathlineto{\pgfqpoint{5.252773in}{0.935760in}}%
\pgfpathlineto{\pgfqpoint{5.322357in}{0.942590in}}%
\pgfpathlineto{\pgfqpoint{5.327512in}{0.943469in}}%
\pgfpathlineto{\pgfqpoint{5.340398in}{0.944599in}}%
\pgfpathlineto{\pgfqpoint{5.345552in}{0.945294in}}%
\pgfpathlineto{\pgfqpoint{5.515648in}{0.953451in}}%
\pgfpathlineto{\pgfqpoint{5.525957in}{0.954638in}}%
\pgfpathlineto{\pgfqpoint{5.554307in}{0.955942in}}%
\pgfpathlineto{\pgfqpoint{5.580079in}{0.957939in}}%
\pgfpathlineto{\pgfqpoint{5.595542in}{0.958692in}}%
\pgfpathlineto{\pgfqpoint{5.616160in}{0.960513in}}%
\pgfpathlineto{\pgfqpoint{5.641932in}{0.961981in}}%
\pgfpathlineto{\pgfqpoint{5.652241in}{0.962752in}}%
\pgfpathlineto{\pgfqpoint{5.667704in}{0.963552in}}%
\pgfpathlineto{\pgfqpoint{5.670281in}{0.963781in}}%
\pgfpathlineto{\pgfqpoint{5.696053in}{0.964865in}}%
\pgfpathlineto{\pgfqpoint{5.716671in}{0.965935in}}%
\pgfpathlineto{\pgfqpoint{5.742443in}{0.967070in}}%
\pgfpathlineto{\pgfqpoint{5.773370in}{0.968136in}}%
\pgfpathlineto{\pgfqpoint{5.812028in}{0.969914in}}%
\pgfpathlineto{\pgfqpoint{5.832646in}{0.970701in}}%
\pgfpathlineto{\pgfqpoint{5.848109in}{0.971601in}}%
\pgfpathlineto{\pgfqpoint{5.886767in}{0.973691in}}%
\pgfpathlineto{\pgfqpoint{5.917694in}{0.974792in}}%
\pgfpathlineto{\pgfqpoint{5.935734in}{0.975473in}}%
\pgfpathlineto{\pgfqpoint{5.958929in}{0.976634in}}%
\pgfpathlineto{\pgfqpoint{6.113562in}{0.979861in}}%
\pgfpathlineto{\pgfqpoint{6.134179in}{0.980768in}}%
\pgfpathlineto{\pgfqpoint{6.165106in}{0.981359in}}%
\pgfpathlineto{\pgfqpoint{6.224382in}{0.983424in}}%
\pgfpathlineto{\pgfqpoint{6.247577in}{0.984953in}}%
\pgfpathlineto{\pgfqpoint{6.260463in}{0.985972in}}%
\pgfpathlineto{\pgfqpoint{6.265617in}{0.986694in}}%
\pgfpathlineto{\pgfqpoint{6.278503in}{0.987772in}}%
\pgfpathlineto{\pgfqpoint{6.283658in}{0.988456in}}%
\pgfpathlineto{\pgfqpoint{6.296544in}{0.989589in}}%
\pgfpathlineto{\pgfqpoint{6.301698in}{0.990390in}}%
\pgfpathlineto{\pgfqpoint{6.312007in}{0.991287in}}%
\pgfpathlineto{\pgfqpoint{6.330048in}{0.994639in}}%
\pgfpathlineto{\pgfqpoint{6.337779in}{0.997696in}}%
\pgfpathlineto{\pgfqpoint{6.345511in}{0.998734in}}%
\pgfpathlineto{\pgfqpoint{6.355820in}{1.002660in}}%
\pgfpathlineto{\pgfqpoint{6.363551in}{1.003560in}}%
\pgfpathlineto{\pgfqpoint{6.373860in}{1.006962in}}%
\pgfpathlineto{\pgfqpoint{6.384169in}{1.008505in}}%
\pgfpathlineto{\pgfqpoint{6.386746in}{1.009257in}}%
\pgfpathlineto{\pgfqpoint{6.404787in}{1.012159in}}%
\pgfpathlineto{\pgfqpoint{6.409941in}{1.013646in}}%
\pgfpathlineto{\pgfqpoint{6.420250in}{1.014962in}}%
\pgfpathlineto{\pgfqpoint{6.427982in}{1.016723in}}%
\pgfpathlineto{\pgfqpoint{6.438290in}{1.017921in}}%
\pgfpathlineto{\pgfqpoint{6.446022in}{1.019763in}}%
\pgfpathlineto{\pgfqpoint{6.453754in}{1.020623in}}%
\pgfpathlineto{\pgfqpoint{6.464063in}{1.024346in}}%
\pgfpathlineto{\pgfqpoint{6.474371in}{1.025164in}}%
\pgfpathlineto{\pgfqpoint{6.482103in}{1.027621in}}%
\pgfpathlineto{\pgfqpoint{6.482103in}{1.027621in}}%
\pgfusepath{stroke}%
\end{pgfscope}%
\begin{pgfscope}%
\pgfpathrectangle{\pgfqpoint{0.563921in}{0.521603in}}{\pgfqpoint{6.200000in}{2.642500in}}%
\pgfusepath{clip}%
\pgfsetroundcap%
\pgfsetroundjoin%
\pgfsetlinewidth{1.505625pt}%
\definecolor{currentstroke}{rgb}{0.580392,0.403922,0.741176}%
\pgfsetstrokecolor{currentstroke}%
\pgfsetdash{}{0pt}%
\pgfpathmoveto{\pgfqpoint{0.845739in}{0.641717in}}%
\pgfpathlineto{\pgfqpoint{0.848317in}{0.651251in}}%
\pgfpathlineto{\pgfqpoint{0.850894in}{0.651656in}}%
\pgfpathlineto{\pgfqpoint{0.853471in}{0.659581in}}%
\pgfpathlineto{\pgfqpoint{0.861203in}{0.660025in}}%
\pgfpathlineto{\pgfqpoint{0.866357in}{0.657500in}}%
\pgfpathlineto{\pgfqpoint{0.871512in}{0.655647in}}%
\pgfpathlineto{\pgfqpoint{0.884398in}{0.654476in}}%
\pgfpathlineto{\pgfqpoint{0.889552in}{0.653479in}}%
\pgfpathlineto{\pgfqpoint{0.902438in}{0.652975in}}%
\pgfpathlineto{\pgfqpoint{0.905015in}{0.653947in}}%
\pgfpathlineto{\pgfqpoint{0.915324in}{0.654728in}}%
\pgfpathlineto{\pgfqpoint{0.917901in}{0.655999in}}%
\pgfpathlineto{\pgfqpoint{0.925633in}{0.656042in}}%
\pgfpathlineto{\pgfqpoint{0.941096in}{0.655691in}}%
\pgfpathlineto{\pgfqpoint{0.943674in}{0.656902in}}%
\pgfpathlineto{\pgfqpoint{0.953982in}{0.658355in}}%
\pgfpathlineto{\pgfqpoint{0.956560in}{0.658885in}}%
\pgfpathlineto{\pgfqpoint{1.023567in}{0.657710in}}%
\pgfpathlineto{\pgfqpoint{1.028722in}{0.658256in}}%
\pgfpathlineto{\pgfqpoint{1.062225in}{0.658674in}}%
\pgfpathlineto{\pgfqpoint{1.080266in}{0.663740in}}%
\pgfpathlineto{\pgfqpoint{1.085420in}{0.663870in}}%
\pgfpathlineto{\pgfqpoint{1.100883in}{0.664058in}}%
\pgfpathlineto{\pgfqpoint{1.103461in}{0.664220in}}%
\pgfpathlineto{\pgfqpoint{1.106038in}{0.665201in}}%
\pgfpathlineto{\pgfqpoint{1.116347in}{0.665526in}}%
\pgfpathlineto{\pgfqpoint{1.124078in}{0.668825in}}%
\pgfpathlineto{\pgfqpoint{1.136964in}{0.669767in}}%
\pgfpathlineto{\pgfqpoint{1.149851in}{0.669355in}}%
\pgfpathlineto{\pgfqpoint{1.191086in}{0.669482in}}%
\pgfpathlineto{\pgfqpoint{1.196240in}{0.670465in}}%
\pgfpathlineto{\pgfqpoint{1.206549in}{0.671243in}}%
\pgfpathlineto{\pgfqpoint{1.214281in}{0.672654in}}%
\pgfpathlineto{\pgfqpoint{1.224590in}{0.673197in}}%
\pgfpathlineto{\pgfqpoint{1.232321in}{0.676148in}}%
\pgfpathlineto{\pgfqpoint{1.265825in}{0.679121in}}%
\pgfpathlineto{\pgfqpoint{1.268402in}{0.680240in}}%
\pgfpathlineto{\pgfqpoint{1.276134in}{0.681634in}}%
\pgfpathlineto{\pgfqpoint{1.283866in}{0.686715in}}%
\pgfpathlineto{\pgfqpoint{1.286443in}{0.688085in}}%
\pgfpathlineto{\pgfqpoint{1.294174in}{0.689227in}}%
\pgfpathlineto{\pgfqpoint{1.299329in}{0.691601in}}%
\pgfpathlineto{\pgfqpoint{1.304483in}{0.695012in}}%
\pgfpathlineto{\pgfqpoint{1.312215in}{0.697443in}}%
\pgfpathlineto{\pgfqpoint{1.314792in}{0.699760in}}%
\pgfpathlineto{\pgfqpoint{1.319947in}{0.701605in}}%
\pgfpathlineto{\pgfqpoint{1.322524in}{0.703195in}}%
\pgfpathlineto{\pgfqpoint{1.330255in}{0.704831in}}%
\pgfpathlineto{\pgfqpoint{1.340564in}{0.711491in}}%
\pgfpathlineto{\pgfqpoint{1.348296in}{0.713328in}}%
\pgfpathlineto{\pgfqpoint{1.358605in}{0.722348in}}%
\pgfpathlineto{\pgfqpoint{1.371491in}{0.724542in}}%
\pgfpathlineto{\pgfqpoint{1.376645in}{0.728018in}}%
\pgfpathlineto{\pgfqpoint{1.384377in}{0.729890in}}%
\pgfpathlineto{\pgfqpoint{1.392108in}{0.734019in}}%
\pgfpathlineto{\pgfqpoint{1.394686in}{0.735337in}}%
\pgfpathlineto{\pgfqpoint{1.404995in}{0.737086in}}%
\pgfpathlineto{\pgfqpoint{1.412726in}{0.739235in}}%
\pgfpathlineto{\pgfqpoint{1.423035in}{0.740654in}}%
\pgfpathlineto{\pgfqpoint{1.430767in}{0.741957in}}%
\pgfpathlineto{\pgfqpoint{1.446230in}{0.742842in}}%
\pgfpathlineto{\pgfqpoint{1.448807in}{0.743235in}}%
\pgfpathlineto{\pgfqpoint{1.464270in}{0.744332in}}%
\pgfpathlineto{\pgfqpoint{1.466848in}{0.744593in}}%
\pgfpathlineto{\pgfqpoint{1.482311in}{0.745363in}}%
\pgfpathlineto{\pgfqpoint{1.484888in}{0.745765in}}%
\pgfpathlineto{\pgfqpoint{1.495197in}{0.746804in}}%
\pgfpathlineto{\pgfqpoint{1.567359in}{0.759547in}}%
\pgfpathlineto{\pgfqpoint{1.585399in}{0.760932in}}%
\pgfpathlineto{\pgfqpoint{1.587977in}{0.762311in}}%
\pgfpathlineto{\pgfqpoint{1.593131in}{0.766396in}}%
\pgfpathlineto{\pgfqpoint{1.603440in}{0.769183in}}%
\pgfpathlineto{\pgfqpoint{1.611172in}{0.772402in}}%
\pgfpathlineto{\pgfqpoint{1.624058in}{0.773356in}}%
\pgfpathlineto{\pgfqpoint{1.629212in}{0.775556in}}%
\pgfpathlineto{\pgfqpoint{1.636944in}{0.776414in}}%
\pgfpathlineto{\pgfqpoint{1.642098in}{0.777969in}}%
\pgfpathlineto{\pgfqpoint{1.647253in}{0.778749in}}%
\pgfpathlineto{\pgfqpoint{1.678179in}{0.780611in}}%
\pgfpathlineto{\pgfqpoint{1.716837in}{0.784437in}}%
\pgfpathlineto{\pgfqpoint{1.719414in}{0.785070in}}%
\pgfpathlineto{\pgfqpoint{1.727146in}{0.785745in}}%
\pgfpathlineto{\pgfqpoint{1.737455in}{0.788684in}}%
\pgfpathlineto{\pgfqpoint{1.747764in}{0.790095in}}%
\pgfpathlineto{\pgfqpoint{1.755495in}{0.791679in}}%
\pgfpathlineto{\pgfqpoint{1.781268in}{0.793225in}}%
\pgfpathlineto{\pgfqpoint{1.788999in}{0.794276in}}%
\pgfpathlineto{\pgfqpoint{1.791576in}{0.795043in}}%
\pgfpathlineto{\pgfqpoint{1.801885in}{0.796425in}}%
\pgfpathlineto{\pgfqpoint{1.809617in}{0.799117in}}%
\pgfpathlineto{\pgfqpoint{1.817349in}{0.800154in}}%
\pgfpathlineto{\pgfqpoint{1.827657in}{0.804391in}}%
\pgfpathlineto{\pgfqpoint{1.837966in}{0.805353in}}%
\pgfpathlineto{\pgfqpoint{1.845698in}{0.808817in}}%
\pgfpathlineto{\pgfqpoint{1.853430in}{0.810043in}}%
\pgfpathlineto{\pgfqpoint{1.861161in}{0.814184in}}%
\pgfpathlineto{\pgfqpoint{1.863738in}{0.815515in}}%
\pgfpathlineto{\pgfqpoint{1.871470in}{0.816785in}}%
\pgfpathlineto{\pgfqpoint{1.881779in}{0.823270in}}%
\pgfpathlineto{\pgfqpoint{1.889510in}{0.824906in}}%
\pgfpathlineto{\pgfqpoint{1.899819in}{0.831805in}}%
\pgfpathlineto{\pgfqpoint{1.910128in}{0.833917in}}%
\pgfpathlineto{\pgfqpoint{1.917860in}{0.839720in}}%
\pgfpathlineto{\pgfqpoint{1.925591in}{0.841297in}}%
\pgfpathlineto{\pgfqpoint{1.935900in}{0.848315in}}%
\pgfpathlineto{\pgfqpoint{1.943632in}{0.850383in}}%
\pgfpathlineto{\pgfqpoint{1.953941in}{0.859105in}}%
\pgfpathlineto{\pgfqpoint{1.961672in}{0.861460in}}%
\pgfpathlineto{\pgfqpoint{1.971981in}{0.871033in}}%
\pgfpathlineto{\pgfqpoint{1.979713in}{0.873262in}}%
\pgfpathlineto{\pgfqpoint{1.990022in}{0.882430in}}%
\pgfpathlineto{\pgfqpoint{1.997753in}{0.884761in}}%
\pgfpathlineto{\pgfqpoint{2.005485in}{0.893428in}}%
\pgfpathlineto{\pgfqpoint{2.015794in}{0.896500in}}%
\pgfpathlineto{\pgfqpoint{2.023526in}{0.905886in}}%
\pgfpathlineto{\pgfqpoint{2.026103in}{0.908703in}}%
\pgfpathlineto{\pgfqpoint{2.033834in}{0.911065in}}%
\pgfpathlineto{\pgfqpoint{2.044143in}{0.921655in}}%
\pgfpathlineto{\pgfqpoint{2.051875in}{0.923976in}}%
\pgfpathlineto{\pgfqpoint{2.062184in}{0.936011in}}%
\pgfpathlineto{\pgfqpoint{2.069915in}{0.939302in}}%
\pgfpathlineto{\pgfqpoint{2.077647in}{0.948967in}}%
\pgfpathlineto{\pgfqpoint{2.080224in}{0.952078in}}%
\pgfpathlineto{\pgfqpoint{2.087956in}{0.955311in}}%
\pgfpathlineto{\pgfqpoint{2.098265in}{0.966780in}}%
\pgfpathlineto{\pgfqpoint{2.105996in}{0.969335in}}%
\pgfpathlineto{\pgfqpoint{2.116305in}{0.980200in}}%
\pgfpathlineto{\pgfqpoint{2.124037in}{0.982916in}}%
\pgfpathlineto{\pgfqpoint{2.134346in}{0.995721in}}%
\pgfpathlineto{\pgfqpoint{2.142077in}{0.998936in}}%
\pgfpathlineto{\pgfqpoint{2.149809in}{1.008511in}}%
\pgfpathlineto{\pgfqpoint{2.152386in}{1.011266in}}%
\pgfpathlineto{\pgfqpoint{2.162695in}{1.014252in}}%
\pgfpathlineto{\pgfqpoint{2.170427in}{1.020536in}}%
\pgfpathlineto{\pgfqpoint{2.178158in}{1.022378in}}%
\pgfpathlineto{\pgfqpoint{2.188467in}{1.028962in}}%
\pgfpathlineto{\pgfqpoint{2.196199in}{1.030779in}}%
\pgfpathlineto{\pgfqpoint{2.206508in}{1.037111in}}%
\pgfpathlineto{\pgfqpoint{2.214239in}{1.038921in}}%
\pgfpathlineto{\pgfqpoint{2.219394in}{1.042451in}}%
\pgfpathlineto{\pgfqpoint{2.224548in}{1.044488in}}%
\pgfpathlineto{\pgfqpoint{2.232280in}{1.045895in}}%
\pgfpathlineto{\pgfqpoint{2.237434in}{1.049484in}}%
\pgfpathlineto{\pgfqpoint{2.242589in}{1.053007in}}%
\pgfpathlineto{\pgfqpoint{2.250320in}{1.054818in}}%
\pgfpathlineto{\pgfqpoint{2.255475in}{1.058378in}}%
\pgfpathlineto{\pgfqpoint{2.260629in}{1.060448in}}%
\pgfpathlineto{\pgfqpoint{2.268361in}{1.062697in}}%
\pgfpathlineto{\pgfqpoint{2.278670in}{1.072320in}}%
\pgfpathlineto{\pgfqpoint{2.286401in}{1.074912in}}%
\pgfpathlineto{\pgfqpoint{2.296710in}{1.085262in}}%
\pgfpathlineto{\pgfqpoint{2.304442in}{1.088212in}}%
\pgfpathlineto{\pgfqpoint{2.314751in}{1.099834in}}%
\pgfpathlineto{\pgfqpoint{2.322482in}{1.102821in}}%
\pgfpathlineto{\pgfqpoint{2.332791in}{1.114878in}}%
\pgfpathlineto{\pgfqpoint{2.340523in}{1.117789in}}%
\pgfpathlineto{\pgfqpoint{2.348254in}{1.126061in}}%
\pgfpathlineto{\pgfqpoint{2.350832in}{1.128375in}}%
\pgfpathlineto{\pgfqpoint{2.358563in}{1.130557in}}%
\pgfpathlineto{\pgfqpoint{2.363718in}{1.134756in}}%
\pgfpathlineto{\pgfqpoint{2.368872in}{1.137657in}}%
\pgfpathlineto{\pgfqpoint{2.376604in}{1.139303in}}%
\pgfpathlineto{\pgfqpoint{2.384335in}{1.143092in}}%
\pgfpathlineto{\pgfqpoint{2.386912in}{1.144348in}}%
\pgfpathlineto{\pgfqpoint{2.397221in}{1.146179in}}%
\pgfpathlineto{\pgfqpoint{2.404953in}{1.148624in}}%
\pgfpathlineto{\pgfqpoint{2.415262in}{1.149400in}}%
\pgfpathlineto{\pgfqpoint{2.422993in}{1.151997in}}%
\pgfpathlineto{\pgfqpoint{2.430725in}{1.152937in}}%
\pgfpathlineto{\pgfqpoint{2.441034in}{1.157557in}}%
\pgfpathlineto{\pgfqpoint{2.448766in}{1.158712in}}%
\pgfpathlineto{\pgfqpoint{2.459074in}{1.163694in}}%
\pgfpathlineto{\pgfqpoint{2.469383in}{1.165684in}}%
\pgfpathlineto{\pgfqpoint{2.477115in}{1.167660in}}%
\pgfpathlineto{\pgfqpoint{2.487424in}{1.168972in}}%
\pgfpathlineto{\pgfqpoint{2.495155in}{1.170860in}}%
\pgfpathlineto{\pgfqpoint{2.508041in}{1.172195in}}%
\pgfpathlineto{\pgfqpoint{2.513196in}{1.173891in}}%
\pgfpathlineto{\pgfqpoint{2.520928in}{1.174934in}}%
\pgfpathlineto{\pgfqpoint{2.528659in}{1.178678in}}%
\pgfpathlineto{\pgfqpoint{2.531236in}{1.180014in}}%
\pgfpathlineto{\pgfqpoint{2.538968in}{1.181247in}}%
\pgfpathlineto{\pgfqpoint{2.549277in}{1.186823in}}%
\pgfpathlineto{\pgfqpoint{2.557009in}{1.188206in}}%
\pgfpathlineto{\pgfqpoint{2.567317in}{1.194030in}}%
\pgfpathlineto{\pgfqpoint{2.575049in}{1.195450in}}%
\pgfpathlineto{\pgfqpoint{2.585358in}{1.201041in}}%
\pgfpathlineto{\pgfqpoint{2.593089in}{1.202623in}}%
\pgfpathlineto{\pgfqpoint{2.603398in}{1.208368in}}%
\pgfpathlineto{\pgfqpoint{2.611130in}{1.209856in}}%
\pgfpathlineto{\pgfqpoint{2.621439in}{1.216409in}}%
\pgfpathlineto{\pgfqpoint{2.629170in}{1.218206in}}%
\pgfpathlineto{\pgfqpoint{2.634325in}{1.221478in}}%
\pgfpathlineto{\pgfqpoint{2.639479in}{1.223009in}}%
\pgfpathlineto{\pgfqpoint{2.647211in}{1.224448in}}%
\pgfpathlineto{\pgfqpoint{2.657520in}{1.229633in}}%
\pgfpathlineto{\pgfqpoint{2.665251in}{1.231016in}}%
\pgfpathlineto{\pgfqpoint{2.670406in}{1.233436in}}%
\pgfpathlineto{\pgfqpoint{2.675560in}{1.234997in}}%
\pgfpathlineto{\pgfqpoint{2.685869in}{1.236446in}}%
\pgfpathlineto{\pgfqpoint{2.693601in}{1.239125in}}%
\pgfpathlineto{\pgfqpoint{2.703910in}{1.240796in}}%
\pgfpathlineto{\pgfqpoint{2.739991in}{1.247400in}}%
\pgfpathlineto{\pgfqpoint{2.747722in}{1.250796in}}%
\pgfpathlineto{\pgfqpoint{2.755454in}{1.251895in}}%
\pgfpathlineto{\pgfqpoint{2.765763in}{1.256363in}}%
\pgfpathlineto{\pgfqpoint{2.776072in}{1.257321in}}%
\pgfpathlineto{\pgfqpoint{2.781226in}{1.259052in}}%
\pgfpathlineto{\pgfqpoint{2.783803in}{1.259486in}}%
\pgfpathlineto{\pgfqpoint{2.827616in}{1.261711in}}%
\pgfpathlineto{\pgfqpoint{2.837925in}{1.264267in}}%
\pgfpathlineto{\pgfqpoint{2.850811in}{1.265291in}}%
\pgfpathlineto{\pgfqpoint{2.855965in}{1.266388in}}%
\pgfpathlineto{\pgfqpoint{2.866274in}{1.267324in}}%
\pgfpathlineto{\pgfqpoint{2.874006in}{1.268850in}}%
\pgfpathlineto{\pgfqpoint{2.881737in}{1.269349in}}%
\pgfpathlineto{\pgfqpoint{2.892046in}{1.272068in}}%
\pgfpathlineto{\pgfqpoint{2.902355in}{1.273501in}}%
\pgfpathlineto{\pgfqpoint{2.910087in}{1.275457in}}%
\pgfpathlineto{\pgfqpoint{2.920395in}{1.276943in}}%
\pgfpathlineto{\pgfqpoint{2.928127in}{1.279380in}}%
\pgfpathlineto{\pgfqpoint{2.935859in}{1.280266in}}%
\pgfpathlineto{\pgfqpoint{2.946168in}{1.284986in}}%
\pgfpathlineto{\pgfqpoint{2.953899in}{1.286287in}}%
\pgfpathlineto{\pgfqpoint{2.964208in}{1.291344in}}%
\pgfpathlineto{\pgfqpoint{2.971940in}{1.292524in}}%
\pgfpathlineto{\pgfqpoint{2.982249in}{1.296958in}}%
\pgfpathlineto{\pgfqpoint{2.989980in}{1.297965in}}%
\pgfpathlineto{\pgfqpoint{2.997712in}{1.301762in}}%
\pgfpathlineto{\pgfqpoint{3.008021in}{1.303170in}}%
\pgfpathlineto{\pgfqpoint{3.018330in}{1.308677in}}%
\pgfpathlineto{\pgfqpoint{3.026061in}{1.310235in}}%
\pgfpathlineto{\pgfqpoint{3.036370in}{1.315790in}}%
\pgfpathlineto{\pgfqpoint{3.044102in}{1.317056in}}%
\pgfpathlineto{\pgfqpoint{3.054411in}{1.322281in}}%
\pgfpathlineto{\pgfqpoint{3.062142in}{1.323563in}}%
\pgfpathlineto{\pgfqpoint{3.072451in}{1.328727in}}%
\pgfpathlineto{\pgfqpoint{3.080183in}{1.329962in}}%
\pgfpathlineto{\pgfqpoint{3.090491in}{1.335118in}}%
\pgfpathlineto{\pgfqpoint{3.100800in}{1.336425in}}%
\pgfpathlineto{\pgfqpoint{3.108532in}{1.340282in}}%
\pgfpathlineto{\pgfqpoint{3.116264in}{1.341750in}}%
\pgfpathlineto{\pgfqpoint{3.126572in}{1.347941in}}%
\pgfpathlineto{\pgfqpoint{3.134304in}{1.349504in}}%
\pgfpathlineto{\pgfqpoint{3.142036in}{1.354153in}}%
\pgfpathlineto{\pgfqpoint{3.144613in}{1.355546in}}%
\pgfpathlineto{\pgfqpoint{3.152345in}{1.356913in}}%
\pgfpathlineto{\pgfqpoint{3.160076in}{1.361126in}}%
\pgfpathlineto{\pgfqpoint{3.162653in}{1.362891in}}%
\pgfpathlineto{\pgfqpoint{3.170385in}{1.364552in}}%
\pgfpathlineto{\pgfqpoint{3.180694in}{1.371361in}}%
\pgfpathlineto{\pgfqpoint{3.188426in}{1.372919in}}%
\pgfpathlineto{\pgfqpoint{3.196157in}{1.378029in}}%
\pgfpathlineto{\pgfqpoint{3.206466in}{1.379824in}}%
\pgfpathlineto{\pgfqpoint{3.216775in}{1.386312in}}%
\pgfpathlineto{\pgfqpoint{3.224507in}{1.387852in}}%
\pgfpathlineto{\pgfqpoint{3.232238in}{1.390917in}}%
\pgfpathlineto{\pgfqpoint{3.234815in}{1.391895in}}%
\pgfpathlineto{\pgfqpoint{3.242547in}{1.392794in}}%
\pgfpathlineto{\pgfqpoint{3.252856in}{1.396813in}}%
\pgfpathlineto{\pgfqpoint{3.260588in}{1.397776in}}%
\pgfpathlineto{\pgfqpoint{3.270896in}{1.401028in}}%
\pgfpathlineto{\pgfqpoint{3.281205in}{1.402346in}}%
\pgfpathlineto{\pgfqpoint{3.288937in}{1.404481in}}%
\pgfpathlineto{\pgfqpoint{3.299246in}{1.405945in}}%
\pgfpathlineto{\pgfqpoint{3.306977in}{1.408359in}}%
\pgfpathlineto{\pgfqpoint{3.314709in}{1.409280in}}%
\pgfpathlineto{\pgfqpoint{3.325018in}{1.413304in}}%
\pgfpathlineto{\pgfqpoint{3.332749in}{1.414337in}}%
\pgfpathlineto{\pgfqpoint{3.343058in}{1.418432in}}%
\pgfpathlineto{\pgfqpoint{3.353367in}{1.419436in}}%
\pgfpathlineto{\pgfqpoint{3.361099in}{1.422645in}}%
\pgfpathlineto{\pgfqpoint{3.368830in}{1.423701in}}%
\pgfpathlineto{\pgfqpoint{3.379139in}{1.428060in}}%
\pgfpathlineto{\pgfqpoint{3.386871in}{1.429156in}}%
\pgfpathlineto{\pgfqpoint{3.397180in}{1.434595in}}%
\pgfpathlineto{\pgfqpoint{3.404911in}{1.436067in}}%
\pgfpathlineto{\pgfqpoint{3.415220in}{1.441664in}}%
\pgfpathlineto{\pgfqpoint{3.422952in}{1.442891in}}%
\pgfpathlineto{\pgfqpoint{3.428106in}{1.445034in}}%
\pgfpathlineto{\pgfqpoint{3.433261in}{1.446899in}}%
\pgfpathlineto{\pgfqpoint{3.443570in}{1.448532in}}%
\pgfpathlineto{\pgfqpoint{3.448724in}{1.450111in}}%
\pgfpathlineto{\pgfqpoint{3.451301in}{1.450637in}}%
\pgfpathlineto{\pgfqpoint{3.477073in}{1.451977in}}%
\pgfpathlineto{\pgfqpoint{3.484805in}{1.453516in}}%
\pgfpathlineto{\pgfqpoint{3.487382in}{1.454200in}}%
\pgfpathlineto{\pgfqpoint{3.495114in}{1.454982in}}%
\pgfpathlineto{\pgfqpoint{3.502845in}{1.457896in}}%
\pgfpathlineto{\pgfqpoint{3.505423in}{1.459095in}}%
\pgfpathlineto{\pgfqpoint{3.513154in}{1.460244in}}%
\pgfpathlineto{\pgfqpoint{3.523463in}{1.465356in}}%
\pgfpathlineto{\pgfqpoint{3.531195in}{1.466628in}}%
\pgfpathlineto{\pgfqpoint{3.541504in}{1.471561in}}%
\pgfpathlineto{\pgfqpoint{3.549235in}{1.472715in}}%
\pgfpathlineto{\pgfqpoint{3.559544in}{1.477378in}}%
\pgfpathlineto{\pgfqpoint{3.567276in}{1.478398in}}%
\pgfpathlineto{\pgfqpoint{3.572430in}{1.480430in}}%
\pgfpathlineto{\pgfqpoint{3.577585in}{1.481591in}}%
\pgfpathlineto{\pgfqpoint{3.585316in}{1.482715in}}%
\pgfpathlineto{\pgfqpoint{3.595625in}{1.487146in}}%
\pgfpathlineto{\pgfqpoint{3.603357in}{1.488286in}}%
\pgfpathlineto{\pgfqpoint{3.611088in}{1.491114in}}%
\pgfpathlineto{\pgfqpoint{3.613666in}{1.491759in}}%
\pgfpathlineto{\pgfqpoint{3.623974in}{1.492820in}}%
\pgfpathlineto{\pgfqpoint{3.644592in}{1.497095in}}%
\pgfpathlineto{\pgfqpoint{3.680673in}{1.501935in}}%
\pgfpathlineto{\pgfqpoint{3.685828in}{1.503290in}}%
\pgfpathlineto{\pgfqpoint{3.696136in}{1.504423in}}%
\pgfpathlineto{\pgfqpoint{3.703868in}{1.505763in}}%
\pgfpathlineto{\pgfqpoint{3.716754in}{1.506299in}}%
\pgfpathlineto{\pgfqpoint{3.721909in}{1.507064in}}%
\pgfpathlineto{\pgfqpoint{3.737372in}{1.508228in}}%
\pgfpathlineto{\pgfqpoint{3.776030in}{1.509988in}}%
\pgfpathlineto{\pgfqpoint{3.804379in}{1.510863in}}%
\pgfpathlineto{\pgfqpoint{3.837883in}{1.513345in}}%
\pgfpathlineto{\pgfqpoint{3.873964in}{1.514907in}}%
\pgfpathlineto{\pgfqpoint{3.884273in}{1.515479in}}%
\pgfpathlineto{\pgfqpoint{4.026020in}{1.519635in}}%
\pgfpathlineto{\pgfqpoint{4.041483in}{1.520071in}}%
\pgfpathlineto{\pgfqpoint{4.056946in}{1.520415in}}%
\pgfpathlineto{\pgfqpoint{4.074987in}{1.520333in}}%
\pgfpathlineto{\pgfqpoint{4.219311in}{1.519921in}}%
\pgfpathlineto{\pgfqpoint{4.239928in}{1.519722in}}%
\pgfpathlineto{\pgfqpoint{4.288895in}{1.517689in}}%
\pgfpathlineto{\pgfqpoint{4.299204in}{1.516410in}}%
\pgfpathlineto{\pgfqpoint{4.314667in}{1.515516in}}%
\pgfpathlineto{\pgfqpoint{4.317245in}{1.515216in}}%
\pgfpathlineto{\pgfqpoint{4.332708in}{1.514223in}}%
\pgfpathlineto{\pgfqpoint{4.353326in}{1.512391in}}%
\pgfpathlineto{\pgfqpoint{4.368789in}{1.511177in}}%
\pgfpathlineto{\pgfqpoint{4.381675in}{1.510406in}}%
\pgfpathlineto{\pgfqpoint{4.402293in}{1.509146in}}%
\pgfpathlineto{\pgfqpoint{4.422910in}{1.508651in}}%
\pgfpathlineto{\pgfqpoint{4.492495in}{1.510251in}}%
\pgfpathlineto{\pgfqpoint{4.515690in}{1.511305in}}%
\pgfpathlineto{\pgfqpoint{4.533730in}{1.512031in}}%
\pgfpathlineto{\pgfqpoint{4.562080in}{1.513324in}}%
\pgfpathlineto{\pgfqpoint{4.569811in}{1.514051in}}%
\pgfpathlineto{\pgfqpoint{4.585275in}{1.514793in}}%
\pgfpathlineto{\pgfqpoint{4.636819in}{1.515272in}}%
\pgfpathlineto{\pgfqpoint{4.678054in}{1.514767in}}%
\pgfpathlineto{\pgfqpoint{4.742485in}{1.517349in}}%
\pgfpathlineto{\pgfqpoint{4.819801in}{1.526835in}}%
\pgfpathlineto{\pgfqpoint{4.832687in}{1.527892in}}%
\pgfpathlineto{\pgfqpoint{4.840419in}{1.529478in}}%
\pgfpathlineto{\pgfqpoint{4.850728in}{1.530516in}}%
\pgfpathlineto{\pgfqpoint{4.858459in}{1.532165in}}%
\pgfpathlineto{\pgfqpoint{4.868768in}{1.533252in}}%
\pgfpathlineto{\pgfqpoint{4.876500in}{1.535018in}}%
\pgfpathlineto{\pgfqpoint{4.884231in}{1.535677in}}%
\pgfpathlineto{\pgfqpoint{4.894540in}{1.539071in}}%
\pgfpathlineto{\pgfqpoint{4.904849in}{1.540726in}}%
\pgfpathlineto{\pgfqpoint{4.912581in}{1.542979in}}%
\pgfpathlineto{\pgfqpoint{4.922890in}{1.544506in}}%
\pgfpathlineto{\pgfqpoint{4.930621in}{1.546740in}}%
\pgfpathlineto{\pgfqpoint{4.938353in}{1.547571in}}%
\pgfpathlineto{\pgfqpoint{4.948662in}{1.551016in}}%
\pgfpathlineto{\pgfqpoint{4.958971in}{1.552698in}}%
\pgfpathlineto{\pgfqpoint{4.966702in}{1.554974in}}%
\pgfpathlineto{\pgfqpoint{4.977011in}{1.556459in}}%
\pgfpathlineto{\pgfqpoint{4.984743in}{1.558827in}}%
\pgfpathlineto{\pgfqpoint{4.995051in}{1.559573in}}%
\pgfpathlineto{\pgfqpoint{5.002783in}{1.562123in}}%
\pgfpathlineto{\pgfqpoint{5.010515in}{1.563120in}}%
\pgfpathlineto{\pgfqpoint{5.020824in}{1.567277in}}%
\pgfpathlineto{\pgfqpoint{5.028555in}{1.568340in}}%
\pgfpathlineto{\pgfqpoint{5.038864in}{1.572396in}}%
\pgfpathlineto{\pgfqpoint{5.046596in}{1.573408in}}%
\pgfpathlineto{\pgfqpoint{5.056905in}{1.577356in}}%
\pgfpathlineto{\pgfqpoint{5.064636in}{1.578342in}}%
\pgfpathlineto{\pgfqpoint{5.072368in}{1.582131in}}%
\pgfpathlineto{\pgfqpoint{5.074945in}{1.583545in}}%
\pgfpathlineto{\pgfqpoint{5.085254in}{1.585051in}}%
\pgfpathlineto{\pgfqpoint{5.092986in}{1.589644in}}%
\pgfpathlineto{\pgfqpoint{5.100717in}{1.591186in}}%
\pgfpathlineto{\pgfqpoint{5.111026in}{1.597304in}}%
\pgfpathlineto{\pgfqpoint{5.118758in}{1.598824in}}%
\pgfpathlineto{\pgfqpoint{5.129067in}{1.605662in}}%
\pgfpathlineto{\pgfqpoint{5.136798in}{1.607321in}}%
\pgfpathlineto{\pgfqpoint{5.147107in}{1.613867in}}%
\pgfpathlineto{\pgfqpoint{5.154839in}{1.615533in}}%
\pgfpathlineto{\pgfqpoint{5.165148in}{1.621617in}}%
\pgfpathlineto{\pgfqpoint{5.172879in}{1.623060in}}%
\pgfpathlineto{\pgfqpoint{5.183188in}{1.628676in}}%
\pgfpathlineto{\pgfqpoint{5.190920in}{1.629952in}}%
\pgfpathlineto{\pgfqpoint{5.201228in}{1.634426in}}%
\pgfpathlineto{\pgfqpoint{5.208960in}{1.635454in}}%
\pgfpathlineto{\pgfqpoint{5.219269in}{1.639452in}}%
\pgfpathlineto{\pgfqpoint{5.227001in}{1.640525in}}%
\pgfpathlineto{\pgfqpoint{5.237309in}{1.644545in}}%
\pgfpathlineto{\pgfqpoint{5.247618in}{1.645575in}}%
\pgfpathlineto{\pgfqpoint{5.255350in}{1.648472in}}%
\pgfpathlineto{\pgfqpoint{5.265659in}{1.650259in}}%
\pgfpathlineto{\pgfqpoint{5.273390in}{1.652871in}}%
\pgfpathlineto{\pgfqpoint{5.281122in}{1.653686in}}%
\pgfpathlineto{\pgfqpoint{5.291431in}{1.657266in}}%
\pgfpathlineto{\pgfqpoint{5.299163in}{1.658069in}}%
\pgfpathlineto{\pgfqpoint{5.309471in}{1.661472in}}%
\pgfpathlineto{\pgfqpoint{5.317203in}{1.662336in}}%
\pgfpathlineto{\pgfqpoint{5.327512in}{1.665809in}}%
\pgfpathlineto{\pgfqpoint{5.337821in}{1.667479in}}%
\pgfpathlineto{\pgfqpoint{5.345552in}{1.669763in}}%
\pgfpathlineto{\pgfqpoint{5.358438in}{1.671632in}}%
\pgfpathlineto{\pgfqpoint{5.363593in}{1.672580in}}%
\pgfpathlineto{\pgfqpoint{5.373902in}{1.673469in}}%
\pgfpathlineto{\pgfqpoint{5.381633in}{1.675066in}}%
\pgfpathlineto{\pgfqpoint{5.391942in}{1.676181in}}%
\pgfpathlineto{\pgfqpoint{5.399674in}{1.677698in}}%
\pgfpathlineto{\pgfqpoint{5.409983in}{1.678953in}}%
\pgfpathlineto{\pgfqpoint{5.417714in}{1.681385in}}%
\pgfpathlineto{\pgfqpoint{5.428023in}{1.682551in}}%
\pgfpathlineto{\pgfqpoint{5.435755in}{1.684257in}}%
\pgfpathlineto{\pgfqpoint{5.461527in}{1.686403in}}%
\pgfpathlineto{\pgfqpoint{5.471836in}{1.687629in}}%
\pgfpathlineto{\pgfqpoint{5.487299in}{1.688767in}}%
\pgfpathlineto{\pgfqpoint{5.523380in}{1.693802in}}%
\pgfpathlineto{\pgfqpoint{5.525957in}{1.694336in}}%
\pgfpathlineto{\pgfqpoint{5.538843in}{1.695336in}}%
\pgfpathlineto{\pgfqpoint{5.543998in}{1.696306in}}%
\pgfpathlineto{\pgfqpoint{5.556884in}{1.697318in}}%
\pgfpathlineto{\pgfqpoint{5.562038in}{1.698430in}}%
\pgfpathlineto{\pgfqpoint{5.572347in}{1.699484in}}%
\pgfpathlineto{\pgfqpoint{5.580079in}{1.700744in}}%
\pgfpathlineto{\pgfqpoint{5.592965in}{1.701589in}}%
\pgfpathlineto{\pgfqpoint{5.598119in}{1.702352in}}%
\pgfpathlineto{\pgfqpoint{5.613582in}{1.703483in}}%
\pgfpathlineto{\pgfqpoint{5.616160in}{1.703807in}}%
\pgfpathlineto{\pgfqpoint{5.629046in}{1.704737in}}%
\pgfpathlineto{\pgfqpoint{5.634200in}{1.705395in}}%
\pgfpathlineto{\pgfqpoint{5.647086in}{1.706339in}}%
\pgfpathlineto{\pgfqpoint{5.685744in}{1.712017in}}%
\pgfpathlineto{\pgfqpoint{5.688322in}{1.712990in}}%
\pgfpathlineto{\pgfqpoint{5.696053in}{1.713932in}}%
\pgfpathlineto{\pgfqpoint{5.706362in}{1.717964in}}%
\pgfpathlineto{\pgfqpoint{5.714094in}{1.718987in}}%
\pgfpathlineto{\pgfqpoint{5.724403in}{1.723462in}}%
\pgfpathlineto{\pgfqpoint{5.732134in}{1.724705in}}%
\pgfpathlineto{\pgfqpoint{5.742443in}{1.730127in}}%
\pgfpathlineto{\pgfqpoint{5.750175in}{1.731461in}}%
\pgfpathlineto{\pgfqpoint{5.760484in}{1.736107in}}%
\pgfpathlineto{\pgfqpoint{5.768215in}{1.737220in}}%
\pgfpathlineto{\pgfqpoint{5.778524in}{1.741349in}}%
\pgfpathlineto{\pgfqpoint{5.786256in}{1.742349in}}%
\pgfpathlineto{\pgfqpoint{5.796565in}{1.746357in}}%
\pgfpathlineto{\pgfqpoint{5.804296in}{1.747308in}}%
\pgfpathlineto{\pgfqpoint{5.812028in}{1.750256in}}%
\pgfpathlineto{\pgfqpoint{5.824914in}{1.752040in}}%
\pgfpathlineto{\pgfqpoint{5.832646in}{1.754222in}}%
\pgfpathlineto{\pgfqpoint{5.840377in}{1.755029in}}%
\pgfpathlineto{\pgfqpoint{5.850686in}{1.758408in}}%
\pgfpathlineto{\pgfqpoint{5.858418in}{1.759228in}}%
\pgfpathlineto{\pgfqpoint{5.868727in}{1.762557in}}%
\pgfpathlineto{\pgfqpoint{5.879035in}{1.764143in}}%
\pgfpathlineto{\pgfqpoint{5.886767in}{1.766494in}}%
\pgfpathlineto{\pgfqpoint{5.894499in}{1.767550in}}%
\pgfpathlineto{\pgfqpoint{5.904807in}{1.771764in}}%
\pgfpathlineto{\pgfqpoint{5.912539in}{1.772818in}}%
\pgfpathlineto{\pgfqpoint{5.922848in}{1.777152in}}%
\pgfpathlineto{\pgfqpoint{5.933157in}{1.778236in}}%
\pgfpathlineto{\pgfqpoint{5.940888in}{1.781924in}}%
\pgfpathlineto{\pgfqpoint{5.948620in}{1.783252in}}%
\pgfpathlineto{\pgfqpoint{5.958929in}{1.788693in}}%
\pgfpathlineto{\pgfqpoint{5.966661in}{1.790115in}}%
\pgfpathlineto{\pgfqpoint{5.976969in}{1.796164in}}%
\pgfpathlineto{\pgfqpoint{5.984701in}{1.797750in}}%
\pgfpathlineto{\pgfqpoint{5.995010in}{1.804501in}}%
\pgfpathlineto{\pgfqpoint{6.002742in}{1.806254in}}%
\pgfpathlineto{\pgfqpoint{6.013050in}{1.812158in}}%
\pgfpathlineto{\pgfqpoint{6.025936in}{1.815038in}}%
\pgfpathlineto{\pgfqpoint{6.031091in}{1.817767in}}%
\pgfpathlineto{\pgfqpoint{6.038823in}{1.819041in}}%
\pgfpathlineto{\pgfqpoint{6.049131in}{1.824223in}}%
\pgfpathlineto{\pgfqpoint{6.056863in}{1.825518in}}%
\pgfpathlineto{\pgfqpoint{6.067172in}{1.831729in}}%
\pgfpathlineto{\pgfqpoint{6.074904in}{1.833063in}}%
\pgfpathlineto{\pgfqpoint{6.085212in}{1.837820in}}%
\pgfpathlineto{\pgfqpoint{6.092944in}{1.839103in}}%
\pgfpathlineto{\pgfqpoint{6.103253in}{1.844203in}}%
\pgfpathlineto{\pgfqpoint{6.110984in}{1.845469in}}%
\pgfpathlineto{\pgfqpoint{6.121293in}{1.850499in}}%
\pgfpathlineto{\pgfqpoint{6.129025in}{1.851787in}}%
\pgfpathlineto{\pgfqpoint{6.139334in}{1.856828in}}%
\pgfpathlineto{\pgfqpoint{6.147065in}{1.858086in}}%
\pgfpathlineto{\pgfqpoint{6.157374in}{1.862996in}}%
\pgfpathlineto{\pgfqpoint{6.165106in}{1.864160in}}%
\pgfpathlineto{\pgfqpoint{6.175415in}{1.868745in}}%
\pgfpathlineto{\pgfqpoint{6.185724in}{1.869747in}}%
\pgfpathlineto{\pgfqpoint{6.193455in}{1.873034in}}%
\pgfpathlineto{\pgfqpoint{6.201187in}{1.874267in}}%
\pgfpathlineto{\pgfqpoint{6.211496in}{1.879218in}}%
\pgfpathlineto{\pgfqpoint{6.219227in}{1.880597in}}%
\pgfpathlineto{\pgfqpoint{6.226959in}{1.884234in}}%
\pgfpathlineto{\pgfqpoint{6.229536in}{1.885285in}}%
\pgfpathlineto{\pgfqpoint{6.237268in}{1.886315in}}%
\pgfpathlineto{\pgfqpoint{6.247577in}{1.890073in}}%
\pgfpathlineto{\pgfqpoint{6.255308in}{1.891085in}}%
\pgfpathlineto{\pgfqpoint{6.265617in}{1.895568in}}%
\pgfpathlineto{\pgfqpoint{6.273349in}{1.896722in}}%
\pgfpathlineto{\pgfqpoint{6.283658in}{1.902060in}}%
\pgfpathlineto{\pgfqpoint{6.291389in}{1.903392in}}%
\pgfpathlineto{\pgfqpoint{6.301698in}{1.910439in}}%
\pgfpathlineto{\pgfqpoint{6.309430in}{1.912370in}}%
\pgfpathlineto{\pgfqpoint{6.319739in}{1.919402in}}%
\pgfpathlineto{\pgfqpoint{6.327470in}{1.920980in}}%
\pgfpathlineto{\pgfqpoint{6.337779in}{1.927173in}}%
\pgfpathlineto{\pgfqpoint{6.345511in}{1.928698in}}%
\pgfpathlineto{\pgfqpoint{6.355820in}{1.934892in}}%
\pgfpathlineto{\pgfqpoint{6.363551in}{1.936383in}}%
\pgfpathlineto{\pgfqpoint{6.373860in}{1.942005in}}%
\pgfpathlineto{\pgfqpoint{6.381592in}{1.943323in}}%
\pgfpathlineto{\pgfqpoint{6.386746in}{1.945925in}}%
\pgfpathlineto{\pgfqpoint{6.391901in}{1.947232in}}%
\pgfpathlineto{\pgfqpoint{6.399632in}{1.948607in}}%
\pgfpathlineto{\pgfqpoint{6.409941in}{1.954577in}}%
\pgfpathlineto{\pgfqpoint{6.417673in}{1.955993in}}%
\pgfpathlineto{\pgfqpoint{6.427982in}{1.962015in}}%
\pgfpathlineto{\pgfqpoint{6.435713in}{1.963569in}}%
\pgfpathlineto{\pgfqpoint{6.446022in}{1.970135in}}%
\pgfpathlineto{\pgfqpoint{6.453754in}{1.971709in}}%
\pgfpathlineto{\pgfqpoint{6.464063in}{1.977696in}}%
\pgfpathlineto{\pgfqpoint{6.474371in}{1.979103in}}%
\pgfpathlineto{\pgfqpoint{6.482103in}{1.983336in}}%
\pgfpathlineto{\pgfqpoint{6.482103in}{1.983336in}}%
\pgfusepath{stroke}%
\end{pgfscope}%
\begin{pgfscope}%
\pgfpathrectangle{\pgfqpoint{0.563921in}{0.521603in}}{\pgfqpoint{6.200000in}{2.642500in}}%
\pgfusepath{clip}%
\pgfsetroundcap%
\pgfsetroundjoin%
\pgfsetlinewidth{1.505625pt}%
\definecolor{currentstroke}{rgb}{0.549020,0.337255,0.294118}%
\pgfsetstrokecolor{currentstroke}%
\pgfsetdash{}{0pt}%
\pgfpathmoveto{\pgfqpoint{0.845739in}{0.641717in}}%
\pgfpathlineto{\pgfqpoint{0.848317in}{0.642313in}}%
\pgfpathlineto{\pgfqpoint{0.850894in}{0.648196in}}%
\pgfpathlineto{\pgfqpoint{0.853471in}{0.650933in}}%
\pgfpathlineto{\pgfqpoint{0.861203in}{0.649962in}}%
\pgfpathlineto{\pgfqpoint{0.863780in}{0.650863in}}%
\pgfpathlineto{\pgfqpoint{0.866357in}{0.658682in}}%
\pgfpathlineto{\pgfqpoint{0.868934in}{0.660441in}}%
\pgfpathlineto{\pgfqpoint{0.871512in}{0.661209in}}%
\pgfpathlineto{\pgfqpoint{0.889552in}{0.659387in}}%
\pgfpathlineto{\pgfqpoint{0.897284in}{0.661694in}}%
\pgfpathlineto{\pgfqpoint{0.899861in}{0.666279in}}%
\pgfpathlineto{\pgfqpoint{0.905015in}{0.667984in}}%
\pgfpathlineto{\pgfqpoint{0.915324in}{0.679311in}}%
\pgfpathlineto{\pgfqpoint{0.917901in}{0.685767in}}%
\pgfpathlineto{\pgfqpoint{0.920479in}{0.689581in}}%
\pgfpathlineto{\pgfqpoint{0.923056in}{0.692003in}}%
\pgfpathlineto{\pgfqpoint{0.925633in}{0.695852in}}%
\pgfpathlineto{\pgfqpoint{0.938519in}{0.696841in}}%
\pgfpathlineto{\pgfqpoint{0.943674in}{0.695946in}}%
\pgfpathlineto{\pgfqpoint{0.951405in}{0.695219in}}%
\pgfpathlineto{\pgfqpoint{0.961714in}{0.692096in}}%
\pgfpathlineto{\pgfqpoint{0.974600in}{0.690777in}}%
\pgfpathlineto{\pgfqpoint{0.979754in}{0.693258in}}%
\pgfpathlineto{\pgfqpoint{0.987486in}{0.694469in}}%
\pgfpathlineto{\pgfqpoint{0.992641in}{0.699748in}}%
\pgfpathlineto{\pgfqpoint{0.997795in}{0.700708in}}%
\pgfpathlineto{\pgfqpoint{1.015835in}{0.702981in}}%
\pgfpathlineto{\pgfqpoint{1.023567in}{0.704499in}}%
\pgfpathlineto{\pgfqpoint{1.028722in}{0.707548in}}%
\pgfpathlineto{\pgfqpoint{1.031299in}{0.708533in}}%
\pgfpathlineto{\pgfqpoint{1.033876in}{0.708876in}}%
\pgfpathlineto{\pgfqpoint{1.046762in}{0.709498in}}%
\pgfpathlineto{\pgfqpoint{1.051916in}{0.710242in}}%
\pgfpathlineto{\pgfqpoint{1.067380in}{0.710350in}}%
\pgfpathlineto{\pgfqpoint{1.098306in}{0.709607in}}%
\pgfpathlineto{\pgfqpoint{1.106038in}{0.708231in}}%
\pgfpathlineto{\pgfqpoint{1.149851in}{0.706863in}}%
\pgfpathlineto{\pgfqpoint{1.155005in}{0.707244in}}%
\pgfpathlineto{\pgfqpoint{1.160159in}{0.706895in}}%
\pgfpathlineto{\pgfqpoint{1.178200in}{0.706962in}}%
\pgfpathlineto{\pgfqpoint{1.206549in}{0.707929in}}%
\pgfpathlineto{\pgfqpoint{1.214281in}{0.710530in}}%
\pgfpathlineto{\pgfqpoint{1.224590in}{0.710941in}}%
\pgfpathlineto{\pgfqpoint{1.229744in}{0.712656in}}%
\pgfpathlineto{\pgfqpoint{1.232321in}{0.714141in}}%
\pgfpathlineto{\pgfqpoint{1.240053in}{0.715688in}}%
\pgfpathlineto{\pgfqpoint{1.242630in}{0.717372in}}%
\pgfpathlineto{\pgfqpoint{1.245207in}{0.718370in}}%
\pgfpathlineto{\pgfqpoint{1.263248in}{0.719781in}}%
\pgfpathlineto{\pgfqpoint{1.268402in}{0.719898in}}%
\pgfpathlineto{\pgfqpoint{1.278711in}{0.720784in}}%
\pgfpathlineto{\pgfqpoint{1.283866in}{0.725189in}}%
\pgfpathlineto{\pgfqpoint{1.286443in}{0.727435in}}%
\pgfpathlineto{\pgfqpoint{1.294174in}{0.730094in}}%
\pgfpathlineto{\pgfqpoint{1.296752in}{0.732628in}}%
\pgfpathlineto{\pgfqpoint{1.301906in}{0.735681in}}%
\pgfpathlineto{\pgfqpoint{1.304483in}{0.736290in}}%
\pgfpathlineto{\pgfqpoint{1.319947in}{0.737867in}}%
\pgfpathlineto{\pgfqpoint{1.322524in}{0.738356in}}%
\pgfpathlineto{\pgfqpoint{1.340564in}{0.738626in}}%
\pgfpathlineto{\pgfqpoint{1.366336in}{0.736989in}}%
\pgfpathlineto{\pgfqpoint{1.376645in}{0.735867in}}%
\pgfpathlineto{\pgfqpoint{1.389531in}{0.735099in}}%
\pgfpathlineto{\pgfqpoint{1.394686in}{0.734717in}}%
\pgfpathlineto{\pgfqpoint{1.407572in}{0.735268in}}%
\pgfpathlineto{\pgfqpoint{1.412726in}{0.735991in}}%
\pgfpathlineto{\pgfqpoint{1.425612in}{0.736796in}}%
\pgfpathlineto{\pgfqpoint{1.430767in}{0.737602in}}%
\pgfpathlineto{\pgfqpoint{1.446230in}{0.738654in}}%
\pgfpathlineto{\pgfqpoint{1.448807in}{0.738996in}}%
\pgfpathlineto{\pgfqpoint{1.464270in}{0.740152in}}%
\pgfpathlineto{\pgfqpoint{1.466848in}{0.740505in}}%
\pgfpathlineto{\pgfqpoint{1.479734in}{0.741305in}}%
\pgfpathlineto{\pgfqpoint{1.484888in}{0.743040in}}%
\pgfpathlineto{\pgfqpoint{1.495197in}{0.744663in}}%
\pgfpathlineto{\pgfqpoint{1.500351in}{0.746316in}}%
\pgfpathlineto{\pgfqpoint{1.502929in}{0.747456in}}%
\pgfpathlineto{\pgfqpoint{1.510660in}{0.748606in}}%
\pgfpathlineto{\pgfqpoint{1.520969in}{0.753074in}}%
\pgfpathlineto{\pgfqpoint{1.528701in}{0.754317in}}%
\pgfpathlineto{\pgfqpoint{1.536432in}{0.757221in}}%
\pgfpathlineto{\pgfqpoint{1.539010in}{0.758109in}}%
\pgfpathlineto{\pgfqpoint{1.549318in}{0.759584in}}%
\pgfpathlineto{\pgfqpoint{1.557050in}{0.761984in}}%
\pgfpathlineto{\pgfqpoint{1.585399in}{0.764509in}}%
\pgfpathlineto{\pgfqpoint{1.593131in}{0.766838in}}%
\pgfpathlineto{\pgfqpoint{1.606017in}{0.767683in}}%
\pgfpathlineto{\pgfqpoint{1.611172in}{0.769542in}}%
\pgfpathlineto{\pgfqpoint{1.626635in}{0.770857in}}%
\pgfpathlineto{\pgfqpoint{1.629212in}{0.771465in}}%
\pgfpathlineto{\pgfqpoint{1.654984in}{0.772303in}}%
\pgfpathlineto{\pgfqpoint{1.673025in}{0.771857in}}%
\pgfpathlineto{\pgfqpoint{1.693642in}{0.773940in}}%
\pgfpathlineto{\pgfqpoint{1.701374in}{0.775665in}}%
\pgfpathlineto{\pgfqpoint{1.714260in}{0.777103in}}%
\pgfpathlineto{\pgfqpoint{1.719414in}{0.778482in}}%
\pgfpathlineto{\pgfqpoint{1.727146in}{0.779183in}}%
\pgfpathlineto{\pgfqpoint{1.737455in}{0.782043in}}%
\pgfpathlineto{\pgfqpoint{1.750341in}{0.783379in}}%
\pgfpathlineto{\pgfqpoint{1.755495in}{0.783955in}}%
\pgfpathlineto{\pgfqpoint{1.809617in}{0.785051in}}%
\pgfpathlineto{\pgfqpoint{1.822503in}{0.786063in}}%
\pgfpathlineto{\pgfqpoint{1.827657in}{0.787021in}}%
\pgfpathlineto{\pgfqpoint{1.837966in}{0.787529in}}%
\pgfpathlineto{\pgfqpoint{1.843121in}{0.788895in}}%
\pgfpathlineto{\pgfqpoint{1.845698in}{0.790672in}}%
\pgfpathlineto{\pgfqpoint{1.853430in}{0.792663in}}%
\pgfpathlineto{\pgfqpoint{1.863738in}{0.803210in}}%
\pgfpathlineto{\pgfqpoint{1.871470in}{0.805618in}}%
\pgfpathlineto{\pgfqpoint{1.881779in}{0.815972in}}%
\pgfpathlineto{\pgfqpoint{1.889510in}{0.818306in}}%
\pgfpathlineto{\pgfqpoint{1.899819in}{0.828291in}}%
\pgfpathlineto{\pgfqpoint{1.910128in}{0.831073in}}%
\pgfpathlineto{\pgfqpoint{1.917860in}{0.838450in}}%
\pgfpathlineto{\pgfqpoint{1.925591in}{0.840277in}}%
\pgfpathlineto{\pgfqpoint{1.935900in}{0.847912in}}%
\pgfpathlineto{\pgfqpoint{1.943632in}{0.849836in}}%
\pgfpathlineto{\pgfqpoint{1.953941in}{0.857733in}}%
\pgfpathlineto{\pgfqpoint{1.961672in}{0.859703in}}%
\pgfpathlineto{\pgfqpoint{1.971981in}{0.866492in}}%
\pgfpathlineto{\pgfqpoint{1.979713in}{0.867821in}}%
\pgfpathlineto{\pgfqpoint{1.990022in}{0.874433in}}%
\pgfpathlineto{\pgfqpoint{1.997753in}{0.875787in}}%
\pgfpathlineto{\pgfqpoint{2.005485in}{0.880179in}}%
\pgfpathlineto{\pgfqpoint{2.015794in}{0.881763in}}%
\pgfpathlineto{\pgfqpoint{2.023526in}{0.887221in}}%
\pgfpathlineto{\pgfqpoint{2.026103in}{0.888852in}}%
\pgfpathlineto{\pgfqpoint{2.033834in}{0.890648in}}%
\pgfpathlineto{\pgfqpoint{2.038989in}{0.894106in}}%
\pgfpathlineto{\pgfqpoint{2.044143in}{0.898223in}}%
\pgfpathlineto{\pgfqpoint{2.051875in}{0.900133in}}%
\pgfpathlineto{\pgfqpoint{2.062184in}{0.908078in}}%
\pgfpathlineto{\pgfqpoint{2.069915in}{0.910428in}}%
\pgfpathlineto{\pgfqpoint{2.072493in}{0.913176in}}%
\pgfpathlineto{\pgfqpoint{2.080224in}{0.916068in}}%
\pgfpathlineto{\pgfqpoint{2.090533in}{0.918066in}}%
\pgfpathlineto{\pgfqpoint{2.098265in}{0.921276in}}%
\pgfpathlineto{\pgfqpoint{2.105996in}{0.922333in}}%
\pgfpathlineto{\pgfqpoint{2.116305in}{0.927018in}}%
\pgfpathlineto{\pgfqpoint{2.124037in}{0.928208in}}%
\pgfpathlineto{\pgfqpoint{2.134346in}{0.934495in}}%
\pgfpathlineto{\pgfqpoint{2.142077in}{0.935703in}}%
\pgfpathlineto{\pgfqpoint{2.149809in}{0.938950in}}%
\pgfpathlineto{\pgfqpoint{2.152386in}{0.940913in}}%
\pgfpathlineto{\pgfqpoint{2.162695in}{0.942517in}}%
\pgfpathlineto{\pgfqpoint{2.170427in}{0.945108in}}%
\pgfpathlineto{\pgfqpoint{2.180735in}{0.946409in}}%
\pgfpathlineto{\pgfqpoint{2.224548in}{0.954400in}}%
\pgfpathlineto{\pgfqpoint{2.234857in}{0.955075in}}%
\pgfpathlineto{\pgfqpoint{2.242589in}{0.956429in}}%
\pgfpathlineto{\pgfqpoint{2.252897in}{0.957632in}}%
\pgfpathlineto{\pgfqpoint{2.255475in}{0.958290in}}%
\pgfpathlineto{\pgfqpoint{2.270938in}{0.960409in}}%
\pgfpathlineto{\pgfqpoint{2.276092in}{0.962383in}}%
\pgfpathlineto{\pgfqpoint{2.278670in}{0.963652in}}%
\pgfpathlineto{\pgfqpoint{2.286401in}{0.964893in}}%
\pgfpathlineto{\pgfqpoint{2.296710in}{0.969304in}}%
\pgfpathlineto{\pgfqpoint{2.304442in}{0.970530in}}%
\pgfpathlineto{\pgfqpoint{2.314751in}{0.974577in}}%
\pgfpathlineto{\pgfqpoint{2.322482in}{0.975457in}}%
\pgfpathlineto{\pgfqpoint{2.332791in}{0.979585in}}%
\pgfpathlineto{\pgfqpoint{2.340523in}{0.980687in}}%
\pgfpathlineto{\pgfqpoint{2.350832in}{0.985376in}}%
\pgfpathlineto{\pgfqpoint{2.358563in}{0.986443in}}%
\pgfpathlineto{\pgfqpoint{2.366295in}{0.989239in}}%
\pgfpathlineto{\pgfqpoint{2.368872in}{0.989891in}}%
\pgfpathlineto{\pgfqpoint{2.379181in}{0.991037in}}%
\pgfpathlineto{\pgfqpoint{2.386912in}{0.992784in}}%
\pgfpathlineto{\pgfqpoint{2.438457in}{0.995570in}}%
\pgfpathlineto{\pgfqpoint{2.441034in}{0.995948in}}%
\pgfpathlineto{\pgfqpoint{2.451343in}{0.997003in}}%
\pgfpathlineto{\pgfqpoint{2.459074in}{0.998501in}}%
\pgfpathlineto{\pgfqpoint{2.487424in}{0.999381in}}%
\pgfpathlineto{\pgfqpoint{2.510619in}{0.999154in}}%
\pgfpathlineto{\pgfqpoint{2.544122in}{1.002373in}}%
\pgfpathlineto{\pgfqpoint{2.549277in}{1.003444in}}%
\pgfpathlineto{\pgfqpoint{2.557009in}{1.004140in}}%
\pgfpathlineto{\pgfqpoint{2.564740in}{1.006332in}}%
\pgfpathlineto{\pgfqpoint{2.567317in}{1.006965in}}%
\pgfpathlineto{\pgfqpoint{2.577626in}{1.008285in}}%
\pgfpathlineto{\pgfqpoint{2.585358in}{1.010834in}}%
\pgfpathlineto{\pgfqpoint{2.593089in}{1.011609in}}%
\pgfpathlineto{\pgfqpoint{2.600821in}{1.014608in}}%
\pgfpathlineto{\pgfqpoint{2.603398in}{1.015846in}}%
\pgfpathlineto{\pgfqpoint{2.611130in}{1.017012in}}%
\pgfpathlineto{\pgfqpoint{2.621439in}{1.021603in}}%
\pgfpathlineto{\pgfqpoint{2.629170in}{1.022869in}}%
\pgfpathlineto{\pgfqpoint{2.634325in}{1.024967in}}%
\pgfpathlineto{\pgfqpoint{2.639479in}{1.025948in}}%
\pgfpathlineto{\pgfqpoint{2.647211in}{1.026755in}}%
\pgfpathlineto{\pgfqpoint{2.657520in}{1.030078in}}%
\pgfpathlineto{\pgfqpoint{2.667829in}{1.031900in}}%
\pgfpathlineto{\pgfqpoint{2.672983in}{1.033313in}}%
\pgfpathlineto{\pgfqpoint{2.675560in}{1.033873in}}%
\pgfpathlineto{\pgfqpoint{2.685869in}{1.034656in}}%
\pgfpathlineto{\pgfqpoint{2.693601in}{1.036112in}}%
\pgfpathlineto{\pgfqpoint{2.709064in}{1.037262in}}%
\pgfpathlineto{\pgfqpoint{2.711641in}{1.037714in}}%
\pgfpathlineto{\pgfqpoint{2.783803in}{1.041486in}}%
\pgfpathlineto{\pgfqpoint{2.809575in}{1.041081in}}%
\pgfpathlineto{\pgfqpoint{2.819884in}{1.040537in}}%
\pgfpathlineto{\pgfqpoint{2.943590in}{1.040644in}}%
\pgfpathlineto{\pgfqpoint{2.992557in}{1.042783in}}%
\pgfpathlineto{\pgfqpoint{2.997712in}{1.043322in}}%
\pgfpathlineto{\pgfqpoint{3.015752in}{1.044224in}}%
\pgfpathlineto{\pgfqpoint{3.036370in}{1.046560in}}%
\pgfpathlineto{\pgfqpoint{3.049256in}{1.047444in}}%
\pgfpathlineto{\pgfqpoint{3.054411in}{1.048167in}}%
\pgfpathlineto{\pgfqpoint{3.144613in}{1.051024in}}%
\pgfpathlineto{\pgfqpoint{3.216775in}{1.051802in}}%
\pgfpathlineto{\pgfqpoint{3.304400in}{1.054044in}}%
\pgfpathlineto{\pgfqpoint{3.392025in}{1.061992in}}%
\pgfpathlineto{\pgfqpoint{3.397180in}{1.062837in}}%
\pgfpathlineto{\pgfqpoint{3.407489in}{1.063738in}}%
\pgfpathlineto{\pgfqpoint{3.415220in}{1.065090in}}%
\pgfpathlineto{\pgfqpoint{3.430684in}{1.066350in}}%
\pgfpathlineto{\pgfqpoint{3.433261in}{1.066677in}}%
\pgfpathlineto{\pgfqpoint{3.446147in}{1.067593in}}%
\pgfpathlineto{\pgfqpoint{3.461610in}{1.068852in}}%
\pgfpathlineto{\pgfqpoint{3.495114in}{1.071992in}}%
\pgfpathlineto{\pgfqpoint{3.505423in}{1.074766in}}%
\pgfpathlineto{\pgfqpoint{3.513154in}{1.075521in}}%
\pgfpathlineto{\pgfqpoint{3.523463in}{1.079330in}}%
\pgfpathlineto{\pgfqpoint{3.531195in}{1.080339in}}%
\pgfpathlineto{\pgfqpoint{3.541504in}{1.084018in}}%
\pgfpathlineto{\pgfqpoint{3.551813in}{1.085517in}}%
\pgfpathlineto{\pgfqpoint{3.559544in}{1.088000in}}%
\pgfpathlineto{\pgfqpoint{3.567276in}{1.088749in}}%
\pgfpathlineto{\pgfqpoint{3.572430in}{1.090423in}}%
\pgfpathlineto{\pgfqpoint{3.577585in}{1.091474in}}%
\pgfpathlineto{\pgfqpoint{3.585316in}{1.092465in}}%
\pgfpathlineto{\pgfqpoint{3.595625in}{1.096592in}}%
\pgfpathlineto{\pgfqpoint{3.603357in}{1.097638in}}%
\pgfpathlineto{\pgfqpoint{3.613666in}{1.101403in}}%
\pgfpathlineto{\pgfqpoint{3.623974in}{1.103000in}}%
\pgfpathlineto{\pgfqpoint{3.631706in}{1.106302in}}%
\pgfpathlineto{\pgfqpoint{3.639438in}{1.107534in}}%
\pgfpathlineto{\pgfqpoint{3.644592in}{1.110157in}}%
\pgfpathlineto{\pgfqpoint{3.649747in}{1.111490in}}%
\pgfpathlineto{\pgfqpoint{3.657478in}{1.112686in}}%
\pgfpathlineto{\pgfqpoint{3.662633in}{1.114790in}}%
\pgfpathlineto{\pgfqpoint{3.680673in}{1.117973in}}%
\pgfpathlineto{\pgfqpoint{3.685828in}{1.119694in}}%
\pgfpathlineto{\pgfqpoint{3.696136in}{1.121241in}}%
\pgfpathlineto{\pgfqpoint{3.703868in}{1.123618in}}%
\pgfpathlineto{\pgfqpoint{3.714177in}{1.124502in}}%
\pgfpathlineto{\pgfqpoint{3.721909in}{1.127220in}}%
\pgfpathlineto{\pgfqpoint{3.776030in}{1.131952in}}%
\pgfpathlineto{\pgfqpoint{3.794070in}{1.132949in}}%
\pgfpathlineto{\pgfqpoint{3.809534in}{1.133880in}}%
\pgfpathlineto{\pgfqpoint{3.827574in}{1.134814in}}%
\pgfpathlineto{\pgfqpoint{3.969321in}{1.134957in}}%
\pgfpathlineto{\pgfqpoint{3.992516in}{1.134174in}}%
\pgfpathlineto{\pgfqpoint{4.038906in}{1.132943in}}%
\pgfpathlineto{\pgfqpoint{4.093027in}{1.130250in}}%
\pgfpathlineto{\pgfqpoint{4.134263in}{1.128581in}}%
\pgfpathlineto{\pgfqpoint{4.206424in}{1.127276in}}%
\pgfpathlineto{\pgfqpoint{4.227042in}{1.125868in}}%
\pgfpathlineto{\pgfqpoint{4.242505in}{1.124950in}}%
\pgfpathlineto{\pgfqpoint{4.263123in}{1.123442in}}%
\pgfpathlineto{\pgfqpoint{4.278586in}{1.122511in}}%
\pgfpathlineto{\pgfqpoint{4.291472in}{1.121797in}}%
\pgfpathlineto{\pgfqpoint{4.299204in}{1.121121in}}%
\pgfpathlineto{\pgfqpoint{4.324976in}{1.120100in}}%
\pgfpathlineto{\pgfqpoint{4.353326in}{1.117959in}}%
\pgfpathlineto{\pgfqpoint{4.368789in}{1.116955in}}%
\pgfpathlineto{\pgfqpoint{4.389407in}{1.115474in}}%
\pgfpathlineto{\pgfqpoint{4.404870in}{1.114507in}}%
\pgfpathlineto{\pgfqpoint{4.425488in}{1.113174in}}%
\pgfpathlineto{\pgfqpoint{4.451260in}{1.112136in}}%
\pgfpathlineto{\pgfqpoint{4.510536in}{1.108772in}}%
\pgfpathlineto{\pgfqpoint{4.585275in}{1.106314in}}%
\pgfpathlineto{\pgfqpoint{4.621356in}{1.105567in}}%
\pgfpathlineto{\pgfqpoint{4.641973in}{1.104443in}}%
\pgfpathlineto{\pgfqpoint{4.706404in}{1.103931in}}%
\pgfpathlineto{\pgfqpoint{4.750216in}{1.104639in}}%
\pgfpathlineto{\pgfqpoint{4.768257in}{1.105062in}}%
\pgfpathlineto{\pgfqpoint{4.837842in}{1.106830in}}%
\pgfpathlineto{\pgfqpoint{4.858459in}{1.107672in}}%
\pgfpathlineto{\pgfqpoint{5.031132in}{1.110366in}}%
\pgfpathlineto{\pgfqpoint{5.108449in}{1.114368in}}%
\pgfpathlineto{\pgfqpoint{5.111026in}{1.114671in}}%
\pgfpathlineto{\pgfqpoint{5.123912in}{1.115572in}}%
\pgfpathlineto{\pgfqpoint{5.139375in}{1.116798in}}%
\pgfpathlineto{\pgfqpoint{5.196074in}{1.122200in}}%
\pgfpathlineto{\pgfqpoint{5.201228in}{1.123034in}}%
\pgfpathlineto{\pgfqpoint{5.211537in}{1.123820in}}%
\pgfpathlineto{\pgfqpoint{5.219269in}{1.125096in}}%
\pgfpathlineto{\pgfqpoint{5.232155in}{1.126384in}}%
\pgfpathlineto{\pgfqpoint{5.237309in}{1.127319in}}%
\pgfpathlineto{\pgfqpoint{5.250196in}{1.128252in}}%
\pgfpathlineto{\pgfqpoint{5.255350in}{1.128973in}}%
\pgfpathlineto{\pgfqpoint{5.268236in}{1.130137in}}%
\pgfpathlineto{\pgfqpoint{5.273390in}{1.131000in}}%
\pgfpathlineto{\pgfqpoint{5.283699in}{1.131923in}}%
\pgfpathlineto{\pgfqpoint{5.291431in}{1.133221in}}%
\pgfpathlineto{\pgfqpoint{5.301740in}{1.134061in}}%
\pgfpathlineto{\pgfqpoint{5.309471in}{1.135565in}}%
\pgfpathlineto{\pgfqpoint{5.319780in}{1.136447in}}%
\pgfpathlineto{\pgfqpoint{5.327512in}{1.137964in}}%
\pgfpathlineto{\pgfqpoint{5.340398in}{1.139303in}}%
\pgfpathlineto{\pgfqpoint{5.345552in}{1.140122in}}%
\pgfpathlineto{\pgfqpoint{5.373902in}{1.141950in}}%
\pgfpathlineto{\pgfqpoint{5.381633in}{1.142969in}}%
\pgfpathlineto{\pgfqpoint{5.394519in}{1.143940in}}%
\pgfpathlineto{\pgfqpoint{5.399674in}{1.144449in}}%
\pgfpathlineto{\pgfqpoint{5.453795in}{1.146195in}}%
\pgfpathlineto{\pgfqpoint{5.497608in}{1.147111in}}%
\pgfpathlineto{\pgfqpoint{5.515648in}{1.147899in}}%
\pgfpathlineto{\pgfqpoint{5.543998in}{1.148978in}}%
\pgfpathlineto{\pgfqpoint{5.574924in}{1.149856in}}%
\pgfpathlineto{\pgfqpoint{5.605851in}{1.151271in}}%
\pgfpathlineto{\pgfqpoint{5.616160in}{1.152581in}}%
\pgfpathlineto{\pgfqpoint{5.629046in}{1.153576in}}%
\pgfpathlineto{\pgfqpoint{5.634200in}{1.154282in}}%
\pgfpathlineto{\pgfqpoint{5.644509in}{1.154998in}}%
\pgfpathlineto{\pgfqpoint{5.652241in}{1.156196in}}%
\pgfpathlineto{\pgfqpoint{5.662550in}{1.156950in}}%
\pgfpathlineto{\pgfqpoint{5.670281in}{1.158759in}}%
\pgfpathlineto{\pgfqpoint{5.683167in}{1.160046in}}%
\pgfpathlineto{\pgfqpoint{5.688322in}{1.161242in}}%
\pgfpathlineto{\pgfqpoint{5.698630in}{1.162408in}}%
\pgfpathlineto{\pgfqpoint{5.706362in}{1.164149in}}%
\pgfpathlineto{\pgfqpoint{5.716671in}{1.165181in}}%
\pgfpathlineto{\pgfqpoint{5.724403in}{1.166758in}}%
\pgfpathlineto{\pgfqpoint{5.734711in}{1.167900in}}%
\pgfpathlineto{\pgfqpoint{5.742443in}{1.169624in}}%
\pgfpathlineto{\pgfqpoint{5.752752in}{1.170747in}}%
\pgfpathlineto{\pgfqpoint{5.760484in}{1.172310in}}%
\pgfpathlineto{\pgfqpoint{5.770792in}{1.173317in}}%
\pgfpathlineto{\pgfqpoint{5.778524in}{1.174715in}}%
\pgfpathlineto{\pgfqpoint{5.788833in}{1.175573in}}%
\pgfpathlineto{\pgfqpoint{5.796565in}{1.176787in}}%
\pgfpathlineto{\pgfqpoint{5.806873in}{1.177604in}}%
\pgfpathlineto{\pgfqpoint{5.812028in}{1.178491in}}%
\pgfpathlineto{\pgfqpoint{5.824914in}{1.179433in}}%
\pgfpathlineto{\pgfqpoint{5.832646in}{1.180671in}}%
\pgfpathlineto{\pgfqpoint{5.845532in}{1.181877in}}%
\pgfpathlineto{\pgfqpoint{5.850686in}{1.182453in}}%
\pgfpathlineto{\pgfqpoint{5.876458in}{1.183786in}}%
\pgfpathlineto{\pgfqpoint{5.904807in}{1.185560in}}%
\pgfpathlineto{\pgfqpoint{5.920271in}{1.186364in}}%
\pgfpathlineto{\pgfqpoint{5.922848in}{1.186612in}}%
\pgfpathlineto{\pgfqpoint{5.938311in}{1.187467in}}%
\pgfpathlineto{\pgfqpoint{5.940888in}{1.187796in}}%
\pgfpathlineto{\pgfqpoint{5.953775in}{1.188811in}}%
\pgfpathlineto{\pgfqpoint{5.976969in}{1.191027in}}%
\pgfpathlineto{\pgfqpoint{5.989855in}{1.192171in}}%
\pgfpathlineto{\pgfqpoint{5.995010in}{1.192871in}}%
\pgfpathlineto{\pgfqpoint{6.028514in}{1.194952in}}%
\pgfpathlineto{\pgfqpoint{6.038823in}{1.195394in}}%
\pgfpathlineto{\pgfqpoint{6.080058in}{1.198683in}}%
\pgfpathlineto{\pgfqpoint{6.085212in}{1.199582in}}%
\pgfpathlineto{\pgfqpoint{6.095521in}{1.200549in}}%
\pgfpathlineto{\pgfqpoint{6.103253in}{1.201959in}}%
\pgfpathlineto{\pgfqpoint{6.113562in}{1.202987in}}%
\pgfpathlineto{\pgfqpoint{6.121293in}{1.204548in}}%
\pgfpathlineto{\pgfqpoint{6.131602in}{1.205617in}}%
\pgfpathlineto{\pgfqpoint{6.139334in}{1.207296in}}%
\pgfpathlineto{\pgfqpoint{6.149643in}{1.208480in}}%
\pgfpathlineto{\pgfqpoint{6.157374in}{1.210140in}}%
\pgfpathlineto{\pgfqpoint{6.167683in}{1.211233in}}%
\pgfpathlineto{\pgfqpoint{6.175415in}{1.212817in}}%
\pgfpathlineto{\pgfqpoint{6.188301in}{1.213936in}}%
\pgfpathlineto{\pgfqpoint{6.193455in}{1.215074in}}%
\pgfpathlineto{\pgfqpoint{6.203764in}{1.216335in}}%
\pgfpathlineto{\pgfqpoint{6.211496in}{1.218137in}}%
\pgfpathlineto{\pgfqpoint{6.221805in}{1.219358in}}%
\pgfpathlineto{\pgfqpoint{6.229536in}{1.221043in}}%
\pgfpathlineto{\pgfqpoint{6.239845in}{1.222090in}}%
\pgfpathlineto{\pgfqpoint{6.247577in}{1.223271in}}%
\pgfpathlineto{\pgfqpoint{6.257886in}{1.224192in}}%
\pgfpathlineto{\pgfqpoint{6.265617in}{1.225625in}}%
\pgfpathlineto{\pgfqpoint{6.278503in}{1.226930in}}%
\pgfpathlineto{\pgfqpoint{6.283658in}{1.227913in}}%
\pgfpathlineto{\pgfqpoint{6.293967in}{1.228941in}}%
\pgfpathlineto{\pgfqpoint{6.301698in}{1.230140in}}%
\pgfpathlineto{\pgfqpoint{6.330048in}{1.231306in}}%
\pgfpathlineto{\pgfqpoint{6.422827in}{1.237576in}}%
\pgfpathlineto{\pgfqpoint{6.427982in}{1.238228in}}%
\pgfpathlineto{\pgfqpoint{6.440868in}{1.239214in}}%
\pgfpathlineto{\pgfqpoint{6.446022in}{1.240007in}}%
\pgfpathlineto{\pgfqpoint{6.458908in}{1.241232in}}%
\pgfpathlineto{\pgfqpoint{6.464063in}{1.242068in}}%
\pgfpathlineto{\pgfqpoint{6.476949in}{1.242949in}}%
\pgfpathlineto{\pgfqpoint{6.482103in}{1.243782in}}%
\pgfpathlineto{\pgfqpoint{6.482103in}{1.243782in}}%
\pgfusepath{stroke}%
\end{pgfscope}%
\begin{pgfscope}%
\pgfpathrectangle{\pgfqpoint{0.563921in}{0.521603in}}{\pgfqpoint{6.200000in}{2.642500in}}%
\pgfusepath{clip}%
\pgfsetroundcap%
\pgfsetroundjoin%
\pgfsetlinewidth{1.505625pt}%
\definecolor{currentstroke}{rgb}{0.890196,0.466667,0.760784}%
\pgfsetstrokecolor{currentstroke}%
\pgfsetdash{}{0pt}%
\pgfpathmoveto{\pgfqpoint{0.845739in}{0.641717in}}%
\pgfpathlineto{\pgfqpoint{0.848317in}{0.651251in}}%
\pgfpathlineto{\pgfqpoint{0.850894in}{0.656324in}}%
\pgfpathlineto{\pgfqpoint{0.853471in}{0.662793in}}%
\pgfpathlineto{\pgfqpoint{0.861203in}{0.662053in}}%
\pgfpathlineto{\pgfqpoint{0.866357in}{0.689930in}}%
\pgfpathlineto{\pgfqpoint{0.868934in}{0.700406in}}%
\pgfpathlineto{\pgfqpoint{0.871512in}{0.698570in}}%
\pgfpathlineto{\pgfqpoint{0.881820in}{0.701328in}}%
\pgfpathlineto{\pgfqpoint{0.884398in}{0.705959in}}%
\pgfpathlineto{\pgfqpoint{0.886975in}{0.706574in}}%
\pgfpathlineto{\pgfqpoint{0.889552in}{0.705084in}}%
\pgfpathlineto{\pgfqpoint{0.897284in}{0.704029in}}%
\pgfpathlineto{\pgfqpoint{0.899861in}{0.706023in}}%
\pgfpathlineto{\pgfqpoint{0.902438in}{0.706786in}}%
\pgfpathlineto{\pgfqpoint{0.907593in}{0.706564in}}%
\pgfpathlineto{\pgfqpoint{0.915324in}{0.706432in}}%
\pgfpathlineto{\pgfqpoint{0.917901in}{0.708281in}}%
\pgfpathlineto{\pgfqpoint{0.920479in}{0.717828in}}%
\pgfpathlineto{\pgfqpoint{0.923056in}{0.723834in}}%
\pgfpathlineto{\pgfqpoint{0.925633in}{0.732820in}}%
\pgfpathlineto{\pgfqpoint{0.933365in}{0.737724in}}%
\pgfpathlineto{\pgfqpoint{0.935942in}{0.740638in}}%
\pgfpathlineto{\pgfqpoint{0.938519in}{0.747744in}}%
\pgfpathlineto{\pgfqpoint{0.941096in}{0.761658in}}%
\pgfpathlineto{\pgfqpoint{0.943674in}{0.771253in}}%
\pgfpathlineto{\pgfqpoint{0.951405in}{0.784017in}}%
\pgfpathlineto{\pgfqpoint{0.953982in}{0.793284in}}%
\pgfpathlineto{\pgfqpoint{0.959137in}{0.800632in}}%
\pgfpathlineto{\pgfqpoint{0.961714in}{0.805359in}}%
\pgfpathlineto{\pgfqpoint{0.972023in}{0.809715in}}%
\pgfpathlineto{\pgfqpoint{0.974600in}{0.813361in}}%
\pgfpathlineto{\pgfqpoint{0.979754in}{0.818161in}}%
\pgfpathlineto{\pgfqpoint{0.990063in}{0.820896in}}%
\pgfpathlineto{\pgfqpoint{0.995218in}{0.824039in}}%
\pgfpathlineto{\pgfqpoint{0.997795in}{0.826024in}}%
\pgfpathlineto{\pgfqpoint{1.005527in}{0.826036in}}%
\pgfpathlineto{\pgfqpoint{1.008104in}{0.824393in}}%
\pgfpathlineto{\pgfqpoint{1.010681in}{0.823601in}}%
\pgfpathlineto{\pgfqpoint{1.023567in}{0.824220in}}%
\pgfpathlineto{\pgfqpoint{1.031299in}{0.835928in}}%
\pgfpathlineto{\pgfqpoint{1.033876in}{0.837224in}}%
\pgfpathlineto{\pgfqpoint{1.041608in}{0.837409in}}%
\pgfpathlineto{\pgfqpoint{1.046762in}{0.835169in}}%
\pgfpathlineto{\pgfqpoint{1.051916in}{0.832110in}}%
\pgfpathlineto{\pgfqpoint{1.059648in}{0.831345in}}%
\pgfpathlineto{\pgfqpoint{1.064802in}{0.828831in}}%
\pgfpathlineto{\pgfqpoint{1.069957in}{0.826515in}}%
\pgfpathlineto{\pgfqpoint{1.077689in}{0.825416in}}%
\pgfpathlineto{\pgfqpoint{1.085420in}{0.821520in}}%
\pgfpathlineto{\pgfqpoint{1.098306in}{0.819531in}}%
\pgfpathlineto{\pgfqpoint{1.106038in}{0.816021in}}%
\pgfpathlineto{\pgfqpoint{1.113770in}{0.814905in}}%
\pgfpathlineto{\pgfqpoint{1.124078in}{0.810335in}}%
\pgfpathlineto{\pgfqpoint{1.131810in}{0.809356in}}%
\pgfpathlineto{\pgfqpoint{1.142119in}{0.805502in}}%
\pgfpathlineto{\pgfqpoint{1.149851in}{0.804560in}}%
\pgfpathlineto{\pgfqpoint{1.160159in}{0.800976in}}%
\pgfpathlineto{\pgfqpoint{1.173045in}{0.800244in}}%
\pgfpathlineto{\pgfqpoint{1.178200in}{0.800628in}}%
\pgfpathlineto{\pgfqpoint{1.188509in}{0.802479in}}%
\pgfpathlineto{\pgfqpoint{1.191086in}{0.804031in}}%
\pgfpathlineto{\pgfqpoint{1.193663in}{0.806827in}}%
\pgfpathlineto{\pgfqpoint{1.196240in}{0.810494in}}%
\pgfpathlineto{\pgfqpoint{1.203972in}{0.812549in}}%
\pgfpathlineto{\pgfqpoint{1.214281in}{0.819915in}}%
\pgfpathlineto{\pgfqpoint{1.224590in}{0.820430in}}%
\pgfpathlineto{\pgfqpoint{1.229744in}{0.822916in}}%
\pgfpathlineto{\pgfqpoint{1.232321in}{0.825668in}}%
\pgfpathlineto{\pgfqpoint{1.240053in}{0.828814in}}%
\pgfpathlineto{\pgfqpoint{1.242630in}{0.832347in}}%
\pgfpathlineto{\pgfqpoint{1.245207in}{0.833351in}}%
\pgfpathlineto{\pgfqpoint{1.258093in}{0.833539in}}%
\pgfpathlineto{\pgfqpoint{1.268402in}{0.835591in}}%
\pgfpathlineto{\pgfqpoint{1.278711in}{0.835093in}}%
\pgfpathlineto{\pgfqpoint{1.283866in}{0.834793in}}%
\pgfpathlineto{\pgfqpoint{1.286443in}{0.834676in}}%
\pgfpathlineto{\pgfqpoint{1.299329in}{0.836063in}}%
\pgfpathlineto{\pgfqpoint{1.301906in}{0.837172in}}%
\pgfpathlineto{\pgfqpoint{1.304483in}{0.836863in}}%
\pgfpathlineto{\pgfqpoint{1.332833in}{0.836505in}}%
\pgfpathlineto{\pgfqpoint{1.340564in}{0.839063in}}%
\pgfpathlineto{\pgfqpoint{1.353450in}{0.839588in}}%
\pgfpathlineto{\pgfqpoint{1.358605in}{0.839104in}}%
\pgfpathlineto{\pgfqpoint{1.366336in}{0.839461in}}%
\pgfpathlineto{\pgfqpoint{1.371491in}{0.841072in}}%
\pgfpathlineto{\pgfqpoint{1.376645in}{0.841404in}}%
\pgfpathlineto{\pgfqpoint{1.392108in}{0.840488in}}%
\pgfpathlineto{\pgfqpoint{1.394686in}{0.839863in}}%
\pgfpathlineto{\pgfqpoint{1.404995in}{0.838609in}}%
\pgfpathlineto{\pgfqpoint{1.412726in}{0.836712in}}%
\pgfpathlineto{\pgfqpoint{1.423035in}{0.835463in}}%
\pgfpathlineto{\pgfqpoint{1.430767in}{0.834146in}}%
\pgfpathlineto{\pgfqpoint{1.441076in}{0.833429in}}%
\pgfpathlineto{\pgfqpoint{1.448807in}{0.832339in}}%
\pgfpathlineto{\pgfqpoint{1.461693in}{0.831773in}}%
\pgfpathlineto{\pgfqpoint{1.466848in}{0.830988in}}%
\pgfpathlineto{\pgfqpoint{1.479734in}{0.829923in}}%
\pgfpathlineto{\pgfqpoint{1.484888in}{0.829132in}}%
\pgfpathlineto{\pgfqpoint{1.495197in}{0.828172in}}%
\pgfpathlineto{\pgfqpoint{1.500351in}{0.827545in}}%
\pgfpathlineto{\pgfqpoint{1.502929in}{0.827971in}}%
\pgfpathlineto{\pgfqpoint{1.520969in}{0.828225in}}%
\pgfpathlineto{\pgfqpoint{1.531278in}{0.827471in}}%
\pgfpathlineto{\pgfqpoint{1.539010in}{0.826041in}}%
\pgfpathlineto{\pgfqpoint{1.549318in}{0.825103in}}%
\pgfpathlineto{\pgfqpoint{1.557050in}{0.823750in}}%
\pgfpathlineto{\pgfqpoint{1.569936in}{0.822476in}}%
\pgfpathlineto{\pgfqpoint{1.575091in}{0.821767in}}%
\pgfpathlineto{\pgfqpoint{1.585399in}{0.820950in}}%
\pgfpathlineto{\pgfqpoint{1.593131in}{0.819738in}}%
\pgfpathlineto{\pgfqpoint{1.603440in}{0.818882in}}%
\pgfpathlineto{\pgfqpoint{1.611172in}{0.817616in}}%
\pgfpathlineto{\pgfqpoint{1.626635in}{0.816851in}}%
\pgfpathlineto{\pgfqpoint{1.629212in}{0.816440in}}%
\pgfpathlineto{\pgfqpoint{1.644675in}{0.815071in}}%
\pgfpathlineto{\pgfqpoint{1.662716in}{0.813640in}}%
\pgfpathlineto{\pgfqpoint{1.683333in}{0.811933in}}%
\pgfpathlineto{\pgfqpoint{1.719414in}{0.810946in}}%
\pgfpathlineto{\pgfqpoint{1.747764in}{0.811425in}}%
\pgfpathlineto{\pgfqpoint{1.755495in}{0.813294in}}%
\pgfpathlineto{\pgfqpoint{1.781268in}{0.814574in}}%
\pgfpathlineto{\pgfqpoint{1.786422in}{0.815501in}}%
\pgfpathlineto{\pgfqpoint{1.791576in}{0.817816in}}%
\pgfpathlineto{\pgfqpoint{1.801885in}{0.819570in}}%
\pgfpathlineto{\pgfqpoint{1.809617in}{0.823000in}}%
\pgfpathlineto{\pgfqpoint{1.817349in}{0.824466in}}%
\pgfpathlineto{\pgfqpoint{1.827657in}{0.830730in}}%
\pgfpathlineto{\pgfqpoint{1.837966in}{0.832777in}}%
\pgfpathlineto{\pgfqpoint{1.843121in}{0.837752in}}%
\pgfpathlineto{\pgfqpoint{1.845698in}{0.840839in}}%
\pgfpathlineto{\pgfqpoint{1.853430in}{0.843839in}}%
\pgfpathlineto{\pgfqpoint{1.858584in}{0.849184in}}%
\pgfpathlineto{\pgfqpoint{1.861161in}{0.850853in}}%
\pgfpathlineto{\pgfqpoint{1.863738in}{0.853572in}}%
\pgfpathlineto{\pgfqpoint{1.871470in}{0.856060in}}%
\pgfpathlineto{\pgfqpoint{1.881779in}{0.865618in}}%
\pgfpathlineto{\pgfqpoint{1.889510in}{0.867826in}}%
\pgfpathlineto{\pgfqpoint{1.899819in}{0.877551in}}%
\pgfpathlineto{\pgfqpoint{1.910128in}{0.880386in}}%
\pgfpathlineto{\pgfqpoint{1.917860in}{0.887089in}}%
\pgfpathlineto{\pgfqpoint{1.925591in}{0.888545in}}%
\pgfpathlineto{\pgfqpoint{1.930746in}{0.892563in}}%
\pgfpathlineto{\pgfqpoint{1.935900in}{0.896763in}}%
\pgfpathlineto{\pgfqpoint{1.943632in}{0.898310in}}%
\pgfpathlineto{\pgfqpoint{1.953941in}{0.907870in}}%
\pgfpathlineto{\pgfqpoint{1.961672in}{0.910763in}}%
\pgfpathlineto{\pgfqpoint{1.971981in}{0.922217in}}%
\pgfpathlineto{\pgfqpoint{1.979713in}{0.924705in}}%
\pgfpathlineto{\pgfqpoint{1.990022in}{0.934733in}}%
\pgfpathlineto{\pgfqpoint{1.997753in}{0.936959in}}%
\pgfpathlineto{\pgfqpoint{2.005485in}{0.943475in}}%
\pgfpathlineto{\pgfqpoint{2.015794in}{0.945560in}}%
\pgfpathlineto{\pgfqpoint{2.026103in}{0.953591in}}%
\pgfpathlineto{\pgfqpoint{2.033834in}{0.955760in}}%
\pgfpathlineto{\pgfqpoint{2.041566in}{0.963649in}}%
\pgfpathlineto{\pgfqpoint{2.044143in}{0.966377in}}%
\pgfpathlineto{\pgfqpoint{2.051875in}{0.968237in}}%
\pgfpathlineto{\pgfqpoint{2.057029in}{0.971975in}}%
\pgfpathlineto{\pgfqpoint{2.062184in}{0.974745in}}%
\pgfpathlineto{\pgfqpoint{2.069915in}{0.976476in}}%
\pgfpathlineto{\pgfqpoint{2.077647in}{0.980297in}}%
\pgfpathlineto{\pgfqpoint{2.080224in}{0.981199in}}%
\pgfpathlineto{\pgfqpoint{2.087956in}{0.982209in}}%
\pgfpathlineto{\pgfqpoint{2.098265in}{0.986287in}}%
\pgfpathlineto{\pgfqpoint{2.105996in}{0.987726in}}%
\pgfpathlineto{\pgfqpoint{2.113728in}{0.992655in}}%
\pgfpathlineto{\pgfqpoint{2.116305in}{0.994522in}}%
\pgfpathlineto{\pgfqpoint{2.124037in}{0.996260in}}%
\pgfpathlineto{\pgfqpoint{2.134346in}{1.004856in}}%
\pgfpathlineto{\pgfqpoint{2.142077in}{1.007345in}}%
\pgfpathlineto{\pgfqpoint{2.147232in}{1.011955in}}%
\pgfpathlineto{\pgfqpoint{2.152386in}{1.015351in}}%
\pgfpathlineto{\pgfqpoint{2.162695in}{1.017220in}}%
\pgfpathlineto{\pgfqpoint{2.170427in}{1.022063in}}%
\pgfpathlineto{\pgfqpoint{2.178158in}{1.023593in}}%
\pgfpathlineto{\pgfqpoint{2.183313in}{1.025741in}}%
\pgfpathlineto{\pgfqpoint{2.188467in}{1.027816in}}%
\pgfpathlineto{\pgfqpoint{2.196199in}{1.028949in}}%
\pgfpathlineto{\pgfqpoint{2.206508in}{1.032855in}}%
\pgfpathlineto{\pgfqpoint{2.214239in}{1.034112in}}%
\pgfpathlineto{\pgfqpoint{2.219394in}{1.036680in}}%
\pgfpathlineto{\pgfqpoint{2.224548in}{1.037826in}}%
\pgfpathlineto{\pgfqpoint{2.234857in}{1.038689in}}%
\pgfpathlineto{\pgfqpoint{2.242589in}{1.040818in}}%
\pgfpathlineto{\pgfqpoint{2.252897in}{1.042671in}}%
\pgfpathlineto{\pgfqpoint{2.260629in}{1.045117in}}%
\pgfpathlineto{\pgfqpoint{2.268361in}{1.046651in}}%
\pgfpathlineto{\pgfqpoint{2.276092in}{1.052371in}}%
\pgfpathlineto{\pgfqpoint{2.278670in}{1.054602in}}%
\pgfpathlineto{\pgfqpoint{2.286401in}{1.056922in}}%
\pgfpathlineto{\pgfqpoint{2.294133in}{1.064099in}}%
\pgfpathlineto{\pgfqpoint{2.296710in}{1.066963in}}%
\pgfpathlineto{\pgfqpoint{2.304442in}{1.069656in}}%
\pgfpathlineto{\pgfqpoint{2.314751in}{1.083739in}}%
\pgfpathlineto{\pgfqpoint{2.322482in}{1.087132in}}%
\pgfpathlineto{\pgfqpoint{2.330214in}{1.098017in}}%
\pgfpathlineto{\pgfqpoint{2.332791in}{1.102101in}}%
\pgfpathlineto{\pgfqpoint{2.340523in}{1.105706in}}%
\pgfpathlineto{\pgfqpoint{2.350832in}{1.118353in}}%
\pgfpathlineto{\pgfqpoint{2.358563in}{1.121306in}}%
\pgfpathlineto{\pgfqpoint{2.363718in}{1.127469in}}%
\pgfpathlineto{\pgfqpoint{2.368872in}{1.131890in}}%
\pgfpathlineto{\pgfqpoint{2.376604in}{1.133941in}}%
\pgfpathlineto{\pgfqpoint{2.386912in}{1.141666in}}%
\pgfpathlineto{\pgfqpoint{2.394644in}{1.143485in}}%
\pgfpathlineto{\pgfqpoint{2.404953in}{1.148491in}}%
\pgfpathlineto{\pgfqpoint{2.415262in}{1.150283in}}%
\pgfpathlineto{\pgfqpoint{2.422993in}{1.156067in}}%
\pgfpathlineto{\pgfqpoint{2.430725in}{1.158207in}}%
\pgfpathlineto{\pgfqpoint{2.438457in}{1.166745in}}%
\pgfpathlineto{\pgfqpoint{2.441034in}{1.169813in}}%
\pgfpathlineto{\pgfqpoint{2.448766in}{1.173229in}}%
\pgfpathlineto{\pgfqpoint{2.459074in}{1.187621in}}%
\pgfpathlineto{\pgfqpoint{2.466806in}{1.190674in}}%
\pgfpathlineto{\pgfqpoint{2.477115in}{1.202591in}}%
\pgfpathlineto{\pgfqpoint{2.484847in}{1.204988in}}%
\pgfpathlineto{\pgfqpoint{2.490001in}{1.208893in}}%
\pgfpathlineto{\pgfqpoint{2.495155in}{1.211703in}}%
\pgfpathlineto{\pgfqpoint{2.502887in}{1.213091in}}%
\pgfpathlineto{\pgfqpoint{2.508041in}{1.215320in}}%
\pgfpathlineto{\pgfqpoint{2.513196in}{1.219026in}}%
\pgfpathlineto{\pgfqpoint{2.520928in}{1.220939in}}%
\pgfpathlineto{\pgfqpoint{2.528659in}{1.226436in}}%
\pgfpathlineto{\pgfqpoint{2.531236in}{1.228439in}}%
\pgfpathlineto{\pgfqpoint{2.538968in}{1.230388in}}%
\pgfpathlineto{\pgfqpoint{2.549277in}{1.237191in}}%
\pgfpathlineto{\pgfqpoint{2.557009in}{1.238670in}}%
\pgfpathlineto{\pgfqpoint{2.567317in}{1.244843in}}%
\pgfpathlineto{\pgfqpoint{2.575049in}{1.246638in}}%
\pgfpathlineto{\pgfqpoint{2.585358in}{1.253652in}}%
\pgfpathlineto{\pgfqpoint{2.593089in}{1.255356in}}%
\pgfpathlineto{\pgfqpoint{2.600821in}{1.260459in}}%
\pgfpathlineto{\pgfqpoint{2.603398in}{1.262309in}}%
\pgfpathlineto{\pgfqpoint{2.611130in}{1.264253in}}%
\pgfpathlineto{\pgfqpoint{2.621439in}{1.272083in}}%
\pgfpathlineto{\pgfqpoint{2.629170in}{1.274161in}}%
\pgfpathlineto{\pgfqpoint{2.634325in}{1.278644in}}%
\pgfpathlineto{\pgfqpoint{2.639479in}{1.280759in}}%
\pgfpathlineto{\pgfqpoint{2.647211in}{1.282729in}}%
\pgfpathlineto{\pgfqpoint{2.657520in}{1.289866in}}%
\pgfpathlineto{\pgfqpoint{2.665251in}{1.291901in}}%
\pgfpathlineto{\pgfqpoint{2.670406in}{1.295342in}}%
\pgfpathlineto{\pgfqpoint{2.675560in}{1.297891in}}%
\pgfpathlineto{\pgfqpoint{2.683292in}{1.299292in}}%
\pgfpathlineto{\pgfqpoint{2.688446in}{1.302121in}}%
\pgfpathlineto{\pgfqpoint{2.693601in}{1.305390in}}%
\pgfpathlineto{\pgfqpoint{2.701332in}{1.307139in}}%
\pgfpathlineto{\pgfqpoint{2.703910in}{1.309000in}}%
\pgfpathlineto{\pgfqpoint{2.709064in}{1.311091in}}%
\pgfpathlineto{\pgfqpoint{2.711641in}{1.313180in}}%
\pgfpathlineto{\pgfqpoint{2.719373in}{1.315226in}}%
\pgfpathlineto{\pgfqpoint{2.721950in}{1.317475in}}%
\pgfpathlineto{\pgfqpoint{2.727105in}{1.319421in}}%
\pgfpathlineto{\pgfqpoint{2.729682in}{1.321428in}}%
\pgfpathlineto{\pgfqpoint{2.737413in}{1.323389in}}%
\pgfpathlineto{\pgfqpoint{2.747722in}{1.331714in}}%
\pgfpathlineto{\pgfqpoint{2.755454in}{1.333558in}}%
\pgfpathlineto{\pgfqpoint{2.763185in}{1.339459in}}%
\pgfpathlineto{\pgfqpoint{2.765763in}{1.341493in}}%
\pgfpathlineto{\pgfqpoint{2.776072in}{1.343661in}}%
\pgfpathlineto{\pgfqpoint{2.781226in}{1.348124in}}%
\pgfpathlineto{\pgfqpoint{2.783803in}{1.349605in}}%
\pgfpathlineto{\pgfqpoint{2.791535in}{1.351438in}}%
\pgfpathlineto{\pgfqpoint{2.801844in}{1.358689in}}%
\pgfpathlineto{\pgfqpoint{2.812153in}{1.360701in}}%
\pgfpathlineto{\pgfqpoint{2.817307in}{1.362383in}}%
\pgfpathlineto{\pgfqpoint{2.819884in}{1.363523in}}%
\pgfpathlineto{\pgfqpoint{2.827616in}{1.364793in}}%
\pgfpathlineto{\pgfqpoint{2.837925in}{1.371156in}}%
\pgfpathlineto{\pgfqpoint{2.848234in}{1.372820in}}%
\pgfpathlineto{\pgfqpoint{2.855965in}{1.378298in}}%
\pgfpathlineto{\pgfqpoint{2.863697in}{1.380401in}}%
\pgfpathlineto{\pgfqpoint{2.874006in}{1.388512in}}%
\pgfpathlineto{\pgfqpoint{2.881737in}{1.390550in}}%
\pgfpathlineto{\pgfqpoint{2.892046in}{1.399246in}}%
\pgfpathlineto{\pgfqpoint{2.899778in}{1.401359in}}%
\pgfpathlineto{\pgfqpoint{2.907509in}{1.405965in}}%
\pgfpathlineto{\pgfqpoint{2.910087in}{1.407107in}}%
\pgfpathlineto{\pgfqpoint{2.917818in}{1.408527in}}%
\pgfpathlineto{\pgfqpoint{2.928127in}{1.413957in}}%
\pgfpathlineto{\pgfqpoint{2.935859in}{1.415197in}}%
\pgfpathlineto{\pgfqpoint{2.946168in}{1.420561in}}%
\pgfpathlineto{\pgfqpoint{2.953899in}{1.422237in}}%
\pgfpathlineto{\pgfqpoint{2.964208in}{1.430243in}}%
\pgfpathlineto{\pgfqpoint{2.971940in}{1.431714in}}%
\pgfpathlineto{\pgfqpoint{2.982249in}{1.436926in}}%
\pgfpathlineto{\pgfqpoint{2.989980in}{1.438132in}}%
\pgfpathlineto{\pgfqpoint{2.995135in}{1.441145in}}%
\pgfpathlineto{\pgfqpoint{2.997712in}{1.442893in}}%
\pgfpathlineto{\pgfqpoint{3.008021in}{1.444576in}}%
\pgfpathlineto{\pgfqpoint{3.018330in}{1.451451in}}%
\pgfpathlineto{\pgfqpoint{3.026061in}{1.452950in}}%
\pgfpathlineto{\pgfqpoint{3.036370in}{1.458551in}}%
\pgfpathlineto{\pgfqpoint{3.044102in}{1.459841in}}%
\pgfpathlineto{\pgfqpoint{3.054411in}{1.464821in}}%
\pgfpathlineto{\pgfqpoint{3.062142in}{1.466323in}}%
\pgfpathlineto{\pgfqpoint{3.069874in}{1.470297in}}%
\pgfpathlineto{\pgfqpoint{3.072451in}{1.471238in}}%
\pgfpathlineto{\pgfqpoint{3.082760in}{1.472957in}}%
\pgfpathlineto{\pgfqpoint{3.090491in}{1.475894in}}%
\pgfpathlineto{\pgfqpoint{3.100800in}{1.476950in}}%
\pgfpathlineto{\pgfqpoint{3.108532in}{1.480190in}}%
\pgfpathlineto{\pgfqpoint{3.116264in}{1.481423in}}%
\pgfpathlineto{\pgfqpoint{3.126572in}{1.486588in}}%
\pgfpathlineto{\pgfqpoint{3.134304in}{1.488161in}}%
\pgfpathlineto{\pgfqpoint{3.142036in}{1.491904in}}%
\pgfpathlineto{\pgfqpoint{3.144613in}{1.492943in}}%
\pgfpathlineto{\pgfqpoint{3.152345in}{1.493983in}}%
\pgfpathlineto{\pgfqpoint{3.162653in}{1.498358in}}%
\pgfpathlineto{\pgfqpoint{3.172962in}{1.500206in}}%
\pgfpathlineto{\pgfqpoint{3.180694in}{1.502809in}}%
\pgfpathlineto{\pgfqpoint{3.188426in}{1.503573in}}%
\pgfpathlineto{\pgfqpoint{3.196157in}{1.505849in}}%
\pgfpathlineto{\pgfqpoint{3.209043in}{1.507104in}}%
\pgfpathlineto{\pgfqpoint{3.216775in}{1.508681in}}%
\pgfpathlineto{\pgfqpoint{3.227084in}{1.509935in}}%
\pgfpathlineto{\pgfqpoint{3.234815in}{1.511343in}}%
\pgfpathlineto{\pgfqpoint{3.252856in}{1.512009in}}%
\pgfpathlineto{\pgfqpoint{3.265742in}{1.511700in}}%
\pgfpathlineto{\pgfqpoint{3.288937in}{1.509646in}}%
\pgfpathlineto{\pgfqpoint{3.304400in}{1.508696in}}%
\pgfpathlineto{\pgfqpoint{3.306977in}{1.508485in}}%
\pgfpathlineto{\pgfqpoint{3.335327in}{1.508914in}}%
\pgfpathlineto{\pgfqpoint{3.376562in}{1.508884in}}%
\pgfpathlineto{\pgfqpoint{3.410066in}{1.507926in}}%
\pgfpathlineto{\pgfqpoint{3.415220in}{1.507352in}}%
\pgfpathlineto{\pgfqpoint{3.428106in}{1.506485in}}%
\pgfpathlineto{\pgfqpoint{3.433261in}{1.505797in}}%
\pgfpathlineto{\pgfqpoint{3.443570in}{1.505018in}}%
\pgfpathlineto{\pgfqpoint{3.469342in}{1.501085in}}%
\pgfpathlineto{\pgfqpoint{3.479651in}{1.500214in}}%
\pgfpathlineto{\pgfqpoint{3.487382in}{1.499048in}}%
\pgfpathlineto{\pgfqpoint{3.551813in}{1.496923in}}%
\pgfpathlineto{\pgfqpoint{3.572430in}{1.497259in}}%
\pgfpathlineto{\pgfqpoint{3.603357in}{1.498169in}}%
\pgfpathlineto{\pgfqpoint{3.613666in}{1.499595in}}%
\pgfpathlineto{\pgfqpoint{3.626552in}{1.500448in}}%
\pgfpathlineto{\pgfqpoint{3.631706in}{1.501615in}}%
\pgfpathlineto{\pgfqpoint{3.642015in}{1.502946in}}%
\pgfpathlineto{\pgfqpoint{3.644592in}{1.503575in}}%
\pgfpathlineto{\pgfqpoint{3.667787in}{1.506232in}}%
\pgfpathlineto{\pgfqpoint{3.683250in}{1.507175in}}%
\pgfpathlineto{\pgfqpoint{3.685828in}{1.507487in}}%
\pgfpathlineto{\pgfqpoint{3.701291in}{1.508687in}}%
\pgfpathlineto{\pgfqpoint{3.703868in}{1.509177in}}%
\pgfpathlineto{\pgfqpoint{3.714177in}{1.509798in}}%
\pgfpathlineto{\pgfqpoint{3.721909in}{1.512355in}}%
\pgfpathlineto{\pgfqpoint{3.732217in}{1.513840in}}%
\pgfpathlineto{\pgfqpoint{3.739949in}{1.515170in}}%
\pgfpathlineto{\pgfqpoint{3.750258in}{1.516360in}}%
\pgfpathlineto{\pgfqpoint{3.757990in}{1.518562in}}%
\pgfpathlineto{\pgfqpoint{3.768298in}{1.520022in}}%
\pgfpathlineto{\pgfqpoint{3.776030in}{1.522531in}}%
\pgfpathlineto{\pgfqpoint{3.786339in}{1.523516in}}%
\pgfpathlineto{\pgfqpoint{3.794070in}{1.527014in}}%
\pgfpathlineto{\pgfqpoint{3.801802in}{1.528247in}}%
\pgfpathlineto{\pgfqpoint{3.812111in}{1.532720in}}%
\pgfpathlineto{\pgfqpoint{3.819843in}{1.533845in}}%
\pgfpathlineto{\pgfqpoint{3.830151in}{1.537185in}}%
\pgfpathlineto{\pgfqpoint{3.843038in}{1.539158in}}%
\pgfpathlineto{\pgfqpoint{3.848192in}{1.540605in}}%
\pgfpathlineto{\pgfqpoint{3.858501in}{1.542020in}}%
\pgfpathlineto{\pgfqpoint{3.866232in}{1.544029in}}%
\pgfpathlineto{\pgfqpoint{3.876541in}{1.545165in}}%
\pgfpathlineto{\pgfqpoint{3.899736in}{1.547851in}}%
\pgfpathlineto{\pgfqpoint{3.912622in}{1.548714in}}%
\pgfpathlineto{\pgfqpoint{3.920354in}{1.550082in}}%
\pgfpathlineto{\pgfqpoint{3.930663in}{1.550886in}}%
\pgfpathlineto{\pgfqpoint{3.938394in}{1.551893in}}%
\pgfpathlineto{\pgfqpoint{3.951280in}{1.552861in}}%
\pgfpathlineto{\pgfqpoint{3.956435in}{1.553476in}}%
\pgfpathlineto{\pgfqpoint{3.984784in}{1.554849in}}%
\pgfpathlineto{\pgfqpoint{4.002825in}{1.556344in}}%
\pgfpathlineto{\pgfqpoint{4.028597in}{1.560129in}}%
\pgfpathlineto{\pgfqpoint{4.041483in}{1.560805in}}%
\pgfpathlineto{\pgfqpoint{4.046637in}{1.561498in}}%
\pgfpathlineto{\pgfqpoint{4.059523in}{1.562561in}}%
\pgfpathlineto{\pgfqpoint{4.064678in}{1.563166in}}%
\pgfpathlineto{\pgfqpoint{4.077564in}{1.563962in}}%
\pgfpathlineto{\pgfqpoint{4.082718in}{1.564682in}}%
\pgfpathlineto{\pgfqpoint{4.108490in}{1.565501in}}%
\pgfpathlineto{\pgfqpoint{4.118799in}{1.565717in}}%
\pgfpathlineto{\pgfqpoint{4.144571in}{1.564931in}}%
\pgfpathlineto{\pgfqpoint{4.172921in}{1.563503in}}%
\pgfpathlineto{\pgfqpoint{4.183230in}{1.562895in}}%
\pgfpathlineto{\pgfqpoint{4.190961in}{1.561452in}}%
\pgfpathlineto{\pgfqpoint{4.201270in}{1.560449in}}%
\pgfpathlineto{\pgfqpoint{4.209002in}{1.558992in}}%
\pgfpathlineto{\pgfqpoint{4.219311in}{1.558002in}}%
\pgfpathlineto{\pgfqpoint{4.227042in}{1.556511in}}%
\pgfpathlineto{\pgfqpoint{4.237351in}{1.555527in}}%
\pgfpathlineto{\pgfqpoint{4.245083in}{1.554061in}}%
\pgfpathlineto{\pgfqpoint{4.255392in}{1.553105in}}%
\pgfpathlineto{\pgfqpoint{4.263123in}{1.551634in}}%
\pgfpathlineto{\pgfqpoint{4.273432in}{1.550773in}}%
\pgfpathlineto{\pgfqpoint{4.281164in}{1.549345in}}%
\pgfpathlineto{\pgfqpoint{4.291472in}{1.548456in}}%
\pgfpathlineto{\pgfqpoint{4.299204in}{1.547094in}}%
\pgfpathlineto{\pgfqpoint{4.312090in}{1.546160in}}%
\pgfpathlineto{\pgfqpoint{4.317245in}{1.545240in}}%
\pgfpathlineto{\pgfqpoint{4.327553in}{1.544313in}}%
\pgfpathlineto{\pgfqpoint{4.335285in}{1.542918in}}%
\pgfpathlineto{\pgfqpoint{4.348171in}{1.541740in}}%
\pgfpathlineto{\pgfqpoint{4.353326in}{1.541044in}}%
\pgfpathlineto{\pgfqpoint{4.366212in}{1.539932in}}%
\pgfpathlineto{\pgfqpoint{4.371366in}{1.539134in}}%
\pgfpathlineto{\pgfqpoint{4.381675in}{1.538222in}}%
\pgfpathlineto{\pgfqpoint{4.389407in}{1.536816in}}%
\pgfpathlineto{\pgfqpoint{4.399715in}{1.535877in}}%
\pgfpathlineto{\pgfqpoint{4.407447in}{1.534514in}}%
\pgfpathlineto{\pgfqpoint{4.417756in}{1.533602in}}%
\pgfpathlineto{\pgfqpoint{4.425488in}{1.532288in}}%
\pgfpathlineto{\pgfqpoint{4.435796in}{1.531415in}}%
\pgfpathlineto{\pgfqpoint{4.443528in}{1.530086in}}%
\pgfpathlineto{\pgfqpoint{4.453837in}{1.529222in}}%
\pgfpathlineto{\pgfqpoint{4.461569in}{1.527975in}}%
\pgfpathlineto{\pgfqpoint{4.471877in}{1.527111in}}%
\pgfpathlineto{\pgfqpoint{4.479609in}{1.525797in}}%
\pgfpathlineto{\pgfqpoint{4.489918in}{1.524914in}}%
\pgfpathlineto{\pgfqpoint{4.497649in}{1.523606in}}%
\pgfpathlineto{\pgfqpoint{4.507958in}{1.522731in}}%
\pgfpathlineto{\pgfqpoint{4.515690in}{1.521847in}}%
\pgfpathlineto{\pgfqpoint{4.525999in}{1.520959in}}%
\pgfpathlineto{\pgfqpoint{4.533730in}{1.519624in}}%
\pgfpathlineto{\pgfqpoint{4.544039in}{1.518740in}}%
\pgfpathlineto{\pgfqpoint{4.551771in}{1.517424in}}%
\pgfpathlineto{\pgfqpoint{4.562080in}{1.516557in}}%
\pgfpathlineto{\pgfqpoint{4.569811in}{1.515257in}}%
\pgfpathlineto{\pgfqpoint{4.580120in}{1.514387in}}%
\pgfpathlineto{\pgfqpoint{4.585275in}{1.513520in}}%
\pgfpathlineto{\pgfqpoint{4.598161in}{1.512661in}}%
\pgfpathlineto{\pgfqpoint{4.603315in}{1.511802in}}%
\pgfpathlineto{\pgfqpoint{4.616201in}{1.510939in}}%
\pgfpathlineto{\pgfqpoint{4.623933in}{1.509711in}}%
\pgfpathlineto{\pgfqpoint{4.636819in}{1.508566in}}%
\pgfpathlineto{\pgfqpoint{4.641973in}{1.507928in}}%
\pgfpathlineto{\pgfqpoint{4.657437in}{1.507149in}}%
\pgfpathlineto{\pgfqpoint{4.660014in}{1.506853in}}%
\pgfpathlineto{\pgfqpoint{4.672900in}{1.506052in}}%
\pgfpathlineto{\pgfqpoint{4.685786in}{1.505097in}}%
\pgfpathlineto{\pgfqpoint{4.714135in}{1.502426in}}%
\pgfpathlineto{\pgfqpoint{4.727021in}{1.501779in}}%
\pgfpathlineto{\pgfqpoint{4.732176in}{1.501070in}}%
\pgfpathlineto{\pgfqpoint{4.745062in}{1.499849in}}%
\pgfpathlineto{\pgfqpoint{4.750216in}{1.499073in}}%
\pgfpathlineto{\pgfqpoint{4.763102in}{1.497858in}}%
\pgfpathlineto{\pgfqpoint{4.768257in}{1.497061in}}%
\pgfpathlineto{\pgfqpoint{4.781143in}{1.495876in}}%
\pgfpathlineto{\pgfqpoint{4.786297in}{1.495082in}}%
\pgfpathlineto{\pgfqpoint{4.799183in}{1.493907in}}%
\pgfpathlineto{\pgfqpoint{4.804338in}{1.493178in}}%
\pgfpathlineto{\pgfqpoint{4.817224in}{1.492089in}}%
\pgfpathlineto{\pgfqpoint{4.819801in}{1.491728in}}%
\pgfpathlineto{\pgfqpoint{4.835264in}{1.490686in}}%
\pgfpathlineto{\pgfqpoint{4.840419in}{1.490002in}}%
\pgfpathlineto{\pgfqpoint{4.853305in}{1.488988in}}%
\pgfpathlineto{\pgfqpoint{4.858459in}{1.488354in}}%
\pgfpathlineto{\pgfqpoint{4.886809in}{1.486752in}}%
\pgfpathlineto{\pgfqpoint{4.912581in}{1.485179in}}%
\pgfpathlineto{\pgfqpoint{4.925467in}{1.484343in}}%
\pgfpathlineto{\pgfqpoint{4.930621in}{1.483701in}}%
\pgfpathlineto{\pgfqpoint{4.943507in}{1.482781in}}%
\pgfpathlineto{\pgfqpoint{4.948662in}{1.482150in}}%
\pgfpathlineto{\pgfqpoint{4.961548in}{1.481191in}}%
\pgfpathlineto{\pgfqpoint{4.966702in}{1.480510in}}%
\pgfpathlineto{\pgfqpoint{4.979588in}{1.479542in}}%
\pgfpathlineto{\pgfqpoint{4.984743in}{1.478933in}}%
\pgfpathlineto{\pgfqpoint{5.000206in}{1.478021in}}%
\pgfpathlineto{\pgfqpoint{5.002783in}{1.477714in}}%
\pgfpathlineto{\pgfqpoint{5.018246in}{1.476622in}}%
\pgfpathlineto{\pgfqpoint{5.038864in}{1.474903in}}%
\pgfpathlineto{\pgfqpoint{5.054327in}{1.473813in}}%
\pgfpathlineto{\pgfqpoint{5.056905in}{1.473482in}}%
\pgfpathlineto{\pgfqpoint{5.072368in}{1.472252in}}%
\pgfpathlineto{\pgfqpoint{5.074945in}{1.472009in}}%
\pgfpathlineto{\pgfqpoint{5.100717in}{1.470748in}}%
\pgfpathlineto{\pgfqpoint{5.121335in}{1.469772in}}%
\pgfpathlineto{\pgfqpoint{5.147107in}{1.468886in}}%
\pgfpathlineto{\pgfqpoint{5.263082in}{1.466918in}}%
\pgfpathlineto{\pgfqpoint{5.291431in}{1.464637in}}%
\pgfpathlineto{\pgfqpoint{5.306894in}{1.463656in}}%
\pgfpathlineto{\pgfqpoint{5.327512in}{1.462152in}}%
\pgfpathlineto{\pgfqpoint{5.340398in}{1.461251in}}%
\pgfpathlineto{\pgfqpoint{5.345552in}{1.460679in}}%
\pgfpathlineto{\pgfqpoint{5.358438in}{1.459811in}}%
\pgfpathlineto{\pgfqpoint{5.363593in}{1.459208in}}%
\pgfpathlineto{\pgfqpoint{5.379056in}{1.458112in}}%
\pgfpathlineto{\pgfqpoint{5.399674in}{1.456622in}}%
\pgfpathlineto{\pgfqpoint{5.453795in}{1.455629in}}%
\pgfpathlineto{\pgfqpoint{5.520803in}{1.456162in}}%
\pgfpathlineto{\pgfqpoint{5.543998in}{1.456692in}}%
\pgfpathlineto{\pgfqpoint{5.572347in}{1.457499in}}%
\pgfpathlineto{\pgfqpoint{5.598119in}{1.458098in}}%
\pgfpathlineto{\pgfqpoint{5.662550in}{1.459207in}}%
\pgfpathlineto{\pgfqpoint{5.688322in}{1.460532in}}%
\pgfpathlineto{\pgfqpoint{5.703785in}{1.461414in}}%
\pgfpathlineto{\pgfqpoint{5.724403in}{1.462399in}}%
\pgfpathlineto{\pgfqpoint{5.742443in}{1.463353in}}%
\pgfpathlineto{\pgfqpoint{5.788833in}{1.465328in}}%
\pgfpathlineto{\pgfqpoint{5.812028in}{1.466531in}}%
\pgfpathlineto{\pgfqpoint{5.830068in}{1.467371in}}%
\pgfpathlineto{\pgfqpoint{5.863572in}{1.471764in}}%
\pgfpathlineto{\pgfqpoint{5.868727in}{1.473151in}}%
\pgfpathlineto{\pgfqpoint{5.879035in}{1.474541in}}%
\pgfpathlineto{\pgfqpoint{5.886767in}{1.476551in}}%
\pgfpathlineto{\pgfqpoint{5.897076in}{1.477893in}}%
\pgfpathlineto{\pgfqpoint{5.904807in}{1.479784in}}%
\pgfpathlineto{\pgfqpoint{5.915116in}{1.481305in}}%
\pgfpathlineto{\pgfqpoint{5.922848in}{1.483605in}}%
\pgfpathlineto{\pgfqpoint{5.935734in}{1.485003in}}%
\pgfpathlineto{\pgfqpoint{5.940888in}{1.486491in}}%
\pgfpathlineto{\pgfqpoint{5.951197in}{1.487757in}}%
\pgfpathlineto{\pgfqpoint{5.958929in}{1.489542in}}%
\pgfpathlineto{\pgfqpoint{5.969238in}{1.490698in}}%
\pgfpathlineto{\pgfqpoint{5.976969in}{1.492503in}}%
\pgfpathlineto{\pgfqpoint{5.987278in}{1.493879in}}%
\pgfpathlineto{\pgfqpoint{5.995010in}{1.496003in}}%
\pgfpathlineto{\pgfqpoint{6.005319in}{1.497356in}}%
\pgfpathlineto{\pgfqpoint{6.013050in}{1.499443in}}%
\pgfpathlineto{\pgfqpoint{6.025936in}{1.500871in}}%
\pgfpathlineto{\pgfqpoint{6.031091in}{1.502260in}}%
\pgfpathlineto{\pgfqpoint{6.041400in}{1.503794in}}%
\pgfpathlineto{\pgfqpoint{6.049131in}{1.506092in}}%
\pgfpathlineto{\pgfqpoint{6.059440in}{1.507527in}}%
\pgfpathlineto{\pgfqpoint{6.067172in}{1.509689in}}%
\pgfpathlineto{\pgfqpoint{6.080058in}{1.511410in}}%
\pgfpathlineto{\pgfqpoint{6.085212in}{1.512246in}}%
\pgfpathlineto{\pgfqpoint{6.095521in}{1.513068in}}%
\pgfpathlineto{\pgfqpoint{6.103253in}{1.514724in}}%
\pgfpathlineto{\pgfqpoint{6.116139in}{1.515791in}}%
\pgfpathlineto{\pgfqpoint{6.129025in}{1.516664in}}%
\pgfpathlineto{\pgfqpoint{6.157374in}{1.519064in}}%
\pgfpathlineto{\pgfqpoint{6.167683in}{1.519712in}}%
\pgfpathlineto{\pgfqpoint{6.175415in}{1.521035in}}%
\pgfpathlineto{\pgfqpoint{6.226959in}{1.521635in}}%
\pgfpathlineto{\pgfqpoint{6.291389in}{1.526856in}}%
\pgfpathlineto{\pgfqpoint{6.301698in}{1.528591in}}%
\pgfpathlineto{\pgfqpoint{6.314584in}{1.529900in}}%
\pgfpathlineto{\pgfqpoint{6.319739in}{1.530717in}}%
\pgfpathlineto{\pgfqpoint{6.330048in}{1.531562in}}%
\pgfpathlineto{\pgfqpoint{6.337779in}{1.533018in}}%
\pgfpathlineto{\pgfqpoint{6.348088in}{1.533938in}}%
\pgfpathlineto{\pgfqpoint{6.355820in}{1.535015in}}%
\pgfpathlineto{\pgfqpoint{6.368706in}{1.535892in}}%
\pgfpathlineto{\pgfqpoint{6.373860in}{1.536482in}}%
\pgfpathlineto{\pgfqpoint{6.399632in}{1.537829in}}%
\pgfpathlineto{\pgfqpoint{6.425404in}{1.541454in}}%
\pgfpathlineto{\pgfqpoint{6.427982in}{1.542064in}}%
\pgfpathlineto{\pgfqpoint{6.438290in}{1.543356in}}%
\pgfpathlineto{\pgfqpoint{6.446022in}{1.545560in}}%
\pgfpathlineto{\pgfqpoint{6.453754in}{1.546431in}}%
\pgfpathlineto{\pgfqpoint{6.464063in}{1.549994in}}%
\pgfpathlineto{\pgfqpoint{6.474371in}{1.550880in}}%
\pgfpathlineto{\pgfqpoint{6.482103in}{1.553657in}}%
\pgfpathlineto{\pgfqpoint{6.482103in}{1.553657in}}%
\pgfusepath{stroke}%
\end{pgfscope}%
\begin{pgfscope}%
\pgfpathrectangle{\pgfqpoint{0.563921in}{0.521603in}}{\pgfqpoint{6.200000in}{2.642500in}}%
\pgfusepath{clip}%
\pgfsetroundcap%
\pgfsetroundjoin%
\pgfsetlinewidth{1.505625pt}%
\definecolor{currentstroke}{rgb}{0.498039,0.498039,0.498039}%
\pgfsetstrokecolor{currentstroke}%
\pgfsetdash{}{0pt}%
\pgfpathmoveto{\pgfqpoint{0.845739in}{0.641717in}}%
\pgfpathlineto{\pgfqpoint{0.848317in}{0.652443in}}%
\pgfpathlineto{\pgfqpoint{0.850894in}{0.655308in}}%
\pgfpathlineto{\pgfqpoint{0.853471in}{0.656194in}}%
\pgfpathlineto{\pgfqpoint{0.861203in}{0.655643in}}%
\pgfpathlineto{\pgfqpoint{0.863780in}{0.654449in}}%
\pgfpathlineto{\pgfqpoint{0.889552in}{0.653292in}}%
\pgfpathlineto{\pgfqpoint{0.897284in}{0.653691in}}%
\pgfpathlineto{\pgfqpoint{0.907593in}{0.666884in}}%
\pgfpathlineto{\pgfqpoint{0.925633in}{0.668074in}}%
\pgfpathlineto{\pgfqpoint{0.956560in}{0.665628in}}%
\pgfpathlineto{\pgfqpoint{0.961714in}{0.664992in}}%
\pgfpathlineto{\pgfqpoint{0.977177in}{0.664086in}}%
\pgfpathlineto{\pgfqpoint{0.979754in}{0.663791in}}%
\pgfpathlineto{\pgfqpoint{1.010681in}{0.663094in}}%
\pgfpathlineto{\pgfqpoint{1.015835in}{0.663967in}}%
\pgfpathlineto{\pgfqpoint{1.023567in}{0.664577in}}%
\pgfpathlineto{\pgfqpoint{1.033876in}{0.667417in}}%
\pgfpathlineto{\pgfqpoint{1.044185in}{0.668762in}}%
\pgfpathlineto{\pgfqpoint{1.049339in}{0.670046in}}%
\pgfpathlineto{\pgfqpoint{1.059648in}{0.670345in}}%
\pgfpathlineto{\pgfqpoint{1.069957in}{0.669614in}}%
\pgfpathlineto{\pgfqpoint{1.098306in}{0.669303in}}%
\pgfpathlineto{\pgfqpoint{1.134387in}{0.669593in}}%
\pgfpathlineto{\pgfqpoint{1.136964in}{0.670324in}}%
\pgfpathlineto{\pgfqpoint{1.142119in}{0.673480in}}%
\pgfpathlineto{\pgfqpoint{1.149851in}{0.675124in}}%
\pgfpathlineto{\pgfqpoint{1.157582in}{0.680097in}}%
\pgfpathlineto{\pgfqpoint{1.160159in}{0.681052in}}%
\pgfpathlineto{\pgfqpoint{1.167891in}{0.682254in}}%
\pgfpathlineto{\pgfqpoint{1.178200in}{0.686717in}}%
\pgfpathlineto{\pgfqpoint{1.185931in}{0.687900in}}%
\pgfpathlineto{\pgfqpoint{1.196240in}{0.693154in}}%
\pgfpathlineto{\pgfqpoint{1.203972in}{0.694367in}}%
\pgfpathlineto{\pgfqpoint{1.214281in}{0.698635in}}%
\pgfpathlineto{\pgfqpoint{1.224590in}{0.699883in}}%
\pgfpathlineto{\pgfqpoint{1.232321in}{0.702264in}}%
\pgfpathlineto{\pgfqpoint{1.242630in}{0.703483in}}%
\pgfpathlineto{\pgfqpoint{1.250362in}{0.706600in}}%
\pgfpathlineto{\pgfqpoint{1.258093in}{0.708022in}}%
\pgfpathlineto{\pgfqpoint{1.265825in}{0.713642in}}%
\pgfpathlineto{\pgfqpoint{1.268402in}{0.715585in}}%
\pgfpathlineto{\pgfqpoint{1.276134in}{0.717663in}}%
\pgfpathlineto{\pgfqpoint{1.281288in}{0.720927in}}%
\pgfpathlineto{\pgfqpoint{1.286443in}{0.724051in}}%
\pgfpathlineto{\pgfqpoint{1.294174in}{0.725484in}}%
\pgfpathlineto{\pgfqpoint{1.304483in}{0.731590in}}%
\pgfpathlineto{\pgfqpoint{1.312215in}{0.733665in}}%
\pgfpathlineto{\pgfqpoint{1.314792in}{0.735638in}}%
\pgfpathlineto{\pgfqpoint{1.319947in}{0.737460in}}%
\pgfpathlineto{\pgfqpoint{1.322524in}{0.739209in}}%
\pgfpathlineto{\pgfqpoint{1.330255in}{0.741119in}}%
\pgfpathlineto{\pgfqpoint{1.340564in}{0.748244in}}%
\pgfpathlineto{\pgfqpoint{1.348296in}{0.750105in}}%
\pgfpathlineto{\pgfqpoint{1.353450in}{0.754305in}}%
\pgfpathlineto{\pgfqpoint{1.358605in}{0.756463in}}%
\pgfpathlineto{\pgfqpoint{1.371491in}{0.758497in}}%
\pgfpathlineto{\pgfqpoint{1.376645in}{0.760458in}}%
\pgfpathlineto{\pgfqpoint{1.384377in}{0.761527in}}%
\pgfpathlineto{\pgfqpoint{1.392108in}{0.764541in}}%
\pgfpathlineto{\pgfqpoint{1.394686in}{0.765203in}}%
\pgfpathlineto{\pgfqpoint{1.404995in}{0.766557in}}%
\pgfpathlineto{\pgfqpoint{1.412726in}{0.768131in}}%
\pgfpathlineto{\pgfqpoint{1.425612in}{0.769391in}}%
\pgfpathlineto{\pgfqpoint{1.430767in}{0.770022in}}%
\pgfpathlineto{\pgfqpoint{1.443653in}{0.769969in}}%
\pgfpathlineto{\pgfqpoint{1.448807in}{0.769746in}}%
\pgfpathlineto{\pgfqpoint{1.482311in}{0.769734in}}%
\pgfpathlineto{\pgfqpoint{1.497774in}{0.770697in}}%
\pgfpathlineto{\pgfqpoint{1.502929in}{0.771684in}}%
\pgfpathlineto{\pgfqpoint{1.513237in}{0.772373in}}%
\pgfpathlineto{\pgfqpoint{1.520969in}{0.774062in}}%
\pgfpathlineto{\pgfqpoint{1.531278in}{0.775230in}}%
\pgfpathlineto{\pgfqpoint{1.539010in}{0.776837in}}%
\pgfpathlineto{\pgfqpoint{1.549318in}{0.777961in}}%
\pgfpathlineto{\pgfqpoint{1.554473in}{0.779756in}}%
\pgfpathlineto{\pgfqpoint{1.557050in}{0.780954in}}%
\pgfpathlineto{\pgfqpoint{1.567359in}{0.782625in}}%
\pgfpathlineto{\pgfqpoint{1.572513in}{0.783579in}}%
\pgfpathlineto{\pgfqpoint{1.582822in}{0.783915in}}%
\pgfpathlineto{\pgfqpoint{1.587977in}{0.784150in}}%
\pgfpathlineto{\pgfqpoint{1.593131in}{0.784959in}}%
\pgfpathlineto{\pgfqpoint{1.642098in}{0.785673in}}%
\pgfpathlineto{\pgfqpoint{1.665293in}{0.783779in}}%
\pgfpathlineto{\pgfqpoint{1.729723in}{0.782377in}}%
\pgfpathlineto{\pgfqpoint{1.745187in}{0.782402in}}%
\pgfpathlineto{\pgfqpoint{1.863738in}{0.778461in}}%
\pgfpathlineto{\pgfqpoint{1.917860in}{0.779533in}}%
\pgfpathlineto{\pgfqpoint{1.928169in}{0.780169in}}%
\pgfpathlineto{\pgfqpoint{1.951364in}{0.784097in}}%
\pgfpathlineto{\pgfqpoint{1.953941in}{0.784877in}}%
\pgfpathlineto{\pgfqpoint{1.961672in}{0.785595in}}%
\pgfpathlineto{\pgfqpoint{1.971981in}{0.788789in}}%
\pgfpathlineto{\pgfqpoint{1.979713in}{0.789715in}}%
\pgfpathlineto{\pgfqpoint{1.990022in}{0.793345in}}%
\pgfpathlineto{\pgfqpoint{1.997753in}{0.794306in}}%
\pgfpathlineto{\pgfqpoint{2.005485in}{0.797099in}}%
\pgfpathlineto{\pgfqpoint{2.015794in}{0.797999in}}%
\pgfpathlineto{\pgfqpoint{2.026103in}{0.801569in}}%
\pgfpathlineto{\pgfqpoint{2.033834in}{0.802604in}}%
\pgfpathlineto{\pgfqpoint{2.041566in}{0.806045in}}%
\pgfpathlineto{\pgfqpoint{2.044143in}{0.807465in}}%
\pgfpathlineto{\pgfqpoint{2.051875in}{0.808776in}}%
\pgfpathlineto{\pgfqpoint{2.062184in}{0.814009in}}%
\pgfpathlineto{\pgfqpoint{2.069915in}{0.815723in}}%
\pgfpathlineto{\pgfqpoint{2.080224in}{0.823040in}}%
\pgfpathlineto{\pgfqpoint{2.087956in}{0.825026in}}%
\pgfpathlineto{\pgfqpoint{2.093110in}{0.828656in}}%
\pgfpathlineto{\pgfqpoint{2.098265in}{0.831729in}}%
\pgfpathlineto{\pgfqpoint{2.105996in}{0.833035in}}%
\pgfpathlineto{\pgfqpoint{2.116305in}{0.839084in}}%
\pgfpathlineto{\pgfqpoint{2.124037in}{0.840407in}}%
\pgfpathlineto{\pgfqpoint{2.134346in}{0.846402in}}%
\pgfpathlineto{\pgfqpoint{2.142077in}{0.847647in}}%
\pgfpathlineto{\pgfqpoint{2.152386in}{0.851302in}}%
\pgfpathlineto{\pgfqpoint{2.185890in}{0.853830in}}%
\pgfpathlineto{\pgfqpoint{2.188467in}{0.854307in}}%
\pgfpathlineto{\pgfqpoint{2.201353in}{0.855668in}}%
\pgfpathlineto{\pgfqpoint{2.206508in}{0.856799in}}%
\pgfpathlineto{\pgfqpoint{2.216816in}{0.858014in}}%
\pgfpathlineto{\pgfqpoint{2.224548in}{0.858845in}}%
\pgfpathlineto{\pgfqpoint{2.234857in}{0.859474in}}%
\pgfpathlineto{\pgfqpoint{2.242589in}{0.860841in}}%
\pgfpathlineto{\pgfqpoint{2.252897in}{0.861652in}}%
\pgfpathlineto{\pgfqpoint{2.260629in}{0.862688in}}%
\pgfpathlineto{\pgfqpoint{2.270938in}{0.863861in}}%
\pgfpathlineto{\pgfqpoint{2.278670in}{0.865350in}}%
\pgfpathlineto{\pgfqpoint{2.288978in}{0.866083in}}%
\pgfpathlineto{\pgfqpoint{2.296710in}{0.867195in}}%
\pgfpathlineto{\pgfqpoint{2.309596in}{0.868309in}}%
\pgfpathlineto{\pgfqpoint{2.314751in}{0.869213in}}%
\pgfpathlineto{\pgfqpoint{2.366295in}{0.872525in}}%
\pgfpathlineto{\pgfqpoint{2.430725in}{0.870682in}}%
\pgfpathlineto{\pgfqpoint{2.441034in}{0.870210in}}%
\pgfpathlineto{\pgfqpoint{2.477115in}{0.869738in}}%
\pgfpathlineto{\pgfqpoint{2.520928in}{0.868374in}}%
\pgfpathlineto{\pgfqpoint{2.528659in}{0.868271in}}%
\pgfpathlineto{\pgfqpoint{2.567317in}{0.871773in}}%
\pgfpathlineto{\pgfqpoint{2.600821in}{0.873855in}}%
\pgfpathlineto{\pgfqpoint{2.621439in}{0.875250in}}%
\pgfpathlineto{\pgfqpoint{2.683292in}{0.876126in}}%
\pgfpathlineto{\pgfqpoint{2.701332in}{0.875902in}}%
\pgfpathlineto{\pgfqpoint{2.755454in}{0.875847in}}%
\pgfpathlineto{\pgfqpoint{2.765763in}{0.875741in}}%
\pgfpathlineto{\pgfqpoint{2.809575in}{0.875018in}}%
\pgfpathlineto{\pgfqpoint{2.837925in}{0.873980in}}%
\pgfpathlineto{\pgfqpoint{2.889469in}{0.872631in}}%
\pgfpathlineto{\pgfqpoint{2.946168in}{0.870717in}}%
\pgfpathlineto{\pgfqpoint{3.031216in}{0.869317in}}%
\pgfpathlineto{\pgfqpoint{3.054411in}{0.869083in}}%
\pgfpathlineto{\pgfqpoint{3.100800in}{0.869578in}}%
\pgfpathlineto{\pgfqpoint{3.126572in}{0.870397in}}%
\pgfpathlineto{\pgfqpoint{3.214198in}{0.872007in}}%
\pgfpathlineto{\pgfqpoint{3.234815in}{0.873328in}}%
\pgfpathlineto{\pgfqpoint{3.250279in}{0.874302in}}%
\pgfpathlineto{\pgfqpoint{3.268319in}{0.875760in}}%
\pgfpathlineto{\pgfqpoint{3.288937in}{0.876202in}}%
\pgfpathlineto{\pgfqpoint{3.353367in}{0.876795in}}%
\pgfpathlineto{\pgfqpoint{3.376562in}{0.877181in}}%
\pgfpathlineto{\pgfqpoint{3.397180in}{0.877400in}}%
\pgfpathlineto{\pgfqpoint{3.531195in}{0.879886in}}%
\pgfpathlineto{\pgfqpoint{3.559544in}{0.881514in}}%
\pgfpathlineto{\pgfqpoint{3.613666in}{0.881494in}}%
\pgfpathlineto{\pgfqpoint{3.770876in}{0.879107in}}%
\pgfpathlineto{\pgfqpoint{3.794070in}{0.879409in}}%
\pgfpathlineto{\pgfqpoint{3.899736in}{0.880273in}}%
\pgfpathlineto{\pgfqpoint{3.951280in}{0.881154in}}%
\pgfpathlineto{\pgfqpoint{3.992516in}{0.882782in}}%
\pgfpathlineto{\pgfqpoint{4.093027in}{0.883607in}}%
\pgfpathlineto{\pgfqpoint{4.154880in}{0.882571in}}%
\pgfpathlineto{\pgfqpoint{4.216733in}{0.881719in}}%
\pgfpathlineto{\pgfqpoint{4.263123in}{0.881085in}}%
\pgfpathlineto{\pgfqpoint{4.314667in}{0.879743in}}%
\pgfpathlineto{\pgfqpoint{4.407447in}{0.876813in}}%
\pgfpathlineto{\pgfqpoint{4.492495in}{0.875158in}}%
\pgfpathlineto{\pgfqpoint{4.569811in}{0.873472in}}%
\pgfpathlineto{\pgfqpoint{4.685786in}{0.872971in}}%
\pgfpathlineto{\pgfqpoint{4.714135in}{0.874893in}}%
\pgfpathlineto{\pgfqpoint{4.729599in}{0.875571in}}%
\pgfpathlineto{\pgfqpoint{4.750216in}{0.877105in}}%
\pgfpathlineto{\pgfqpoint{4.763102in}{0.877996in}}%
\pgfpathlineto{\pgfqpoint{4.768257in}{0.878655in}}%
\pgfpathlineto{\pgfqpoint{4.781143in}{0.879774in}}%
\pgfpathlineto{\pgfqpoint{4.786297in}{0.880526in}}%
\pgfpathlineto{\pgfqpoint{4.796606in}{0.881305in}}%
\pgfpathlineto{\pgfqpoint{4.804338in}{0.882693in}}%
\pgfpathlineto{\pgfqpoint{4.817224in}{0.884001in}}%
\pgfpathlineto{\pgfqpoint{4.819801in}{0.884471in}}%
\pgfpathlineto{\pgfqpoint{4.832687in}{0.885439in}}%
\pgfpathlineto{\pgfqpoint{4.840419in}{0.886971in}}%
\pgfpathlineto{\pgfqpoint{4.850728in}{0.888023in}}%
\pgfpathlineto{\pgfqpoint{4.858459in}{0.889228in}}%
\pgfpathlineto{\pgfqpoint{4.871345in}{0.890133in}}%
\pgfpathlineto{\pgfqpoint{4.891963in}{0.891785in}}%
\pgfpathlineto{\pgfqpoint{4.912581in}{0.893136in}}%
\pgfpathlineto{\pgfqpoint{4.938353in}{0.894477in}}%
\pgfpathlineto{\pgfqpoint{4.948662in}{0.895443in}}%
\pgfpathlineto{\pgfqpoint{5.000206in}{0.897463in}}%
\pgfpathlineto{\pgfqpoint{5.049173in}{0.901709in}}%
\pgfpathlineto{\pgfqpoint{5.056905in}{0.903157in}}%
\pgfpathlineto{\pgfqpoint{5.067213in}{0.904192in}}%
\pgfpathlineto{\pgfqpoint{5.074945in}{0.906030in}}%
\pgfpathlineto{\pgfqpoint{5.087831in}{0.907462in}}%
\pgfpathlineto{\pgfqpoint{5.092986in}{0.908762in}}%
\pgfpathlineto{\pgfqpoint{5.103294in}{0.910060in}}%
\pgfpathlineto{\pgfqpoint{5.111026in}{0.912043in}}%
\pgfpathlineto{\pgfqpoint{5.121335in}{0.913328in}}%
\pgfpathlineto{\pgfqpoint{5.129067in}{0.915185in}}%
\pgfpathlineto{\pgfqpoint{5.139375in}{0.916328in}}%
\pgfpathlineto{\pgfqpoint{5.147107in}{0.917967in}}%
\pgfpathlineto{\pgfqpoint{5.159993in}{0.919258in}}%
\pgfpathlineto{\pgfqpoint{5.165148in}{0.920040in}}%
\pgfpathlineto{\pgfqpoint{5.178034in}{0.921173in}}%
\pgfpathlineto{\pgfqpoint{5.183188in}{0.921930in}}%
\pgfpathlineto{\pgfqpoint{5.198651in}{0.923193in}}%
\pgfpathlineto{\pgfqpoint{5.219269in}{0.924745in}}%
\pgfpathlineto{\pgfqpoint{5.234732in}{0.925727in}}%
\pgfpathlineto{\pgfqpoint{5.237309in}{0.926007in}}%
\pgfpathlineto{\pgfqpoint{5.252773in}{0.927015in}}%
\pgfpathlineto{\pgfqpoint{5.273390in}{0.928181in}}%
\pgfpathlineto{\pgfqpoint{5.288854in}{0.928888in}}%
\pgfpathlineto{\pgfqpoint{5.304317in}{0.929758in}}%
\pgfpathlineto{\pgfqpoint{5.319780in}{0.930468in}}%
\pgfpathlineto{\pgfqpoint{5.358438in}{0.931653in}}%
\pgfpathlineto{\pgfqpoint{5.389365in}{0.931663in}}%
\pgfpathlineto{\pgfqpoint{5.453795in}{0.931432in}}%
\pgfpathlineto{\pgfqpoint{5.497608in}{0.932770in}}%
\pgfpathlineto{\pgfqpoint{5.525957in}{0.935011in}}%
\pgfpathlineto{\pgfqpoint{5.541421in}{0.935937in}}%
\pgfpathlineto{\pgfqpoint{5.543998in}{0.936224in}}%
\pgfpathlineto{\pgfqpoint{5.556884in}{0.936986in}}%
\pgfpathlineto{\pgfqpoint{5.562038in}{0.937690in}}%
\pgfpathlineto{\pgfqpoint{5.577501in}{0.938750in}}%
\pgfpathlineto{\pgfqpoint{5.580079in}{0.939005in}}%
\pgfpathlineto{\pgfqpoint{5.598119in}{0.940005in}}%
\pgfpathlineto{\pgfqpoint{5.634200in}{0.940603in}}%
\pgfpathlineto{\pgfqpoint{5.696053in}{0.941056in}}%
\pgfpathlineto{\pgfqpoint{5.778524in}{0.942324in}}%
\pgfpathlineto{\pgfqpoint{5.868727in}{0.942140in}}%
\pgfpathlineto{\pgfqpoint{5.912539in}{0.941333in}}%
\pgfpathlineto{\pgfqpoint{5.953775in}{0.940618in}}%
\pgfpathlineto{\pgfqpoint{6.085212in}{0.937878in}}%
\pgfpathlineto{\pgfqpoint{6.245000in}{0.939179in}}%
\pgfpathlineto{\pgfqpoint{6.291389in}{0.940050in}}%
\pgfpathlineto{\pgfqpoint{6.348088in}{0.940686in}}%
\pgfpathlineto{\pgfqpoint{6.399632in}{0.940051in}}%
\pgfpathlineto{\pgfqpoint{6.422827in}{0.941173in}}%
\pgfpathlineto{\pgfqpoint{6.456331in}{0.943534in}}%
\pgfpathlineto{\pgfqpoint{6.464063in}{0.944445in}}%
\pgfpathlineto{\pgfqpoint{6.479526in}{0.945416in}}%
\pgfpathlineto{\pgfqpoint{6.482103in}{0.945709in}}%
\pgfpathlineto{\pgfqpoint{6.482103in}{0.945709in}}%
\pgfusepath{stroke}%
\end{pgfscope}%
\begin{pgfscope}%
\pgfpathrectangle{\pgfqpoint{0.563921in}{0.521603in}}{\pgfqpoint{6.200000in}{2.642500in}}%
\pgfusepath{clip}%
\pgfsetroundcap%
\pgfsetroundjoin%
\pgfsetlinewidth{1.505625pt}%
\definecolor{currentstroke}{rgb}{0.737255,0.741176,0.133333}%
\pgfsetstrokecolor{currentstroke}%
\pgfsetdash{}{0pt}%
\pgfpathmoveto{\pgfqpoint{0.845739in}{0.641717in}}%
\pgfpathlineto{\pgfqpoint{0.848317in}{0.653933in}}%
\pgfpathlineto{\pgfqpoint{0.850894in}{0.651739in}}%
\pgfpathlineto{\pgfqpoint{0.853471in}{0.653065in}}%
\pgfpathlineto{\pgfqpoint{0.863780in}{0.656339in}}%
\pgfpathlineto{\pgfqpoint{0.866357in}{0.657935in}}%
\pgfpathlineto{\pgfqpoint{0.871512in}{0.656214in}}%
\pgfpathlineto{\pgfqpoint{0.881820in}{0.656987in}}%
\pgfpathlineto{\pgfqpoint{0.884398in}{0.659578in}}%
\pgfpathlineto{\pgfqpoint{0.889552in}{0.658794in}}%
\pgfpathlineto{\pgfqpoint{0.897284in}{0.659164in}}%
\pgfpathlineto{\pgfqpoint{0.907593in}{0.657250in}}%
\pgfpathlineto{\pgfqpoint{0.920479in}{0.657087in}}%
\pgfpathlineto{\pgfqpoint{0.925633in}{0.666625in}}%
\pgfpathlineto{\pgfqpoint{0.933365in}{0.670374in}}%
\pgfpathlineto{\pgfqpoint{0.935942in}{0.672749in}}%
\pgfpathlineto{\pgfqpoint{0.938519in}{0.676044in}}%
\pgfpathlineto{\pgfqpoint{0.943674in}{0.691156in}}%
\pgfpathlineto{\pgfqpoint{0.951405in}{0.695499in}}%
\pgfpathlineto{\pgfqpoint{0.956560in}{0.706877in}}%
\pgfpathlineto{\pgfqpoint{0.961714in}{0.713774in}}%
\pgfpathlineto{\pgfqpoint{0.972023in}{0.716230in}}%
\pgfpathlineto{\pgfqpoint{0.979754in}{0.726012in}}%
\pgfpathlineto{\pgfqpoint{0.987486in}{0.728351in}}%
\pgfpathlineto{\pgfqpoint{0.990063in}{0.731584in}}%
\pgfpathlineto{\pgfqpoint{0.997795in}{0.735445in}}%
\pgfpathlineto{\pgfqpoint{1.010681in}{0.736817in}}%
\pgfpathlineto{\pgfqpoint{1.015835in}{0.738670in}}%
\pgfpathlineto{\pgfqpoint{1.026144in}{0.739602in}}%
\pgfpathlineto{\pgfqpoint{1.041608in}{0.741138in}}%
\pgfpathlineto{\pgfqpoint{1.049339in}{0.741326in}}%
\pgfpathlineto{\pgfqpoint{1.051916in}{0.741812in}}%
\pgfpathlineto{\pgfqpoint{1.085420in}{0.745347in}}%
\pgfpathlineto{\pgfqpoint{1.098306in}{0.745152in}}%
\pgfpathlineto{\pgfqpoint{1.100883in}{0.744887in}}%
\pgfpathlineto{\pgfqpoint{1.118924in}{0.747420in}}%
\pgfpathlineto{\pgfqpoint{1.124078in}{0.747832in}}%
\pgfpathlineto{\pgfqpoint{1.136964in}{0.747585in}}%
\pgfpathlineto{\pgfqpoint{1.142119in}{0.748648in}}%
\pgfpathlineto{\pgfqpoint{1.157582in}{0.749064in}}%
\pgfpathlineto{\pgfqpoint{1.160159in}{0.748650in}}%
\pgfpathlineto{\pgfqpoint{1.170468in}{0.747928in}}%
\pgfpathlineto{\pgfqpoint{1.178200in}{0.746731in}}%
\pgfpathlineto{\pgfqpoint{1.191086in}{0.745452in}}%
\pgfpathlineto{\pgfqpoint{1.196240in}{0.744416in}}%
\pgfpathlineto{\pgfqpoint{1.240053in}{0.741590in}}%
\pgfpathlineto{\pgfqpoint{1.250362in}{0.740051in}}%
\pgfpathlineto{\pgfqpoint{1.263248in}{0.738963in}}%
\pgfpathlineto{\pgfqpoint{1.268402in}{0.738400in}}%
\pgfpathlineto{\pgfqpoint{1.283866in}{0.738495in}}%
\pgfpathlineto{\pgfqpoint{1.286443in}{0.739059in}}%
\pgfpathlineto{\pgfqpoint{1.301906in}{0.739627in}}%
\pgfpathlineto{\pgfqpoint{1.319947in}{0.741993in}}%
\pgfpathlineto{\pgfqpoint{1.322524in}{0.742409in}}%
\pgfpathlineto{\pgfqpoint{1.348296in}{0.743319in}}%
\pgfpathlineto{\pgfqpoint{1.356027in}{0.745027in}}%
\pgfpathlineto{\pgfqpoint{1.358605in}{0.745347in}}%
\pgfpathlineto{\pgfqpoint{1.374068in}{0.745779in}}%
\pgfpathlineto{\pgfqpoint{1.376645in}{0.746511in}}%
\pgfpathlineto{\pgfqpoint{1.389531in}{0.748565in}}%
\pgfpathlineto{\pgfqpoint{1.394686in}{0.750083in}}%
\pgfpathlineto{\pgfqpoint{1.404995in}{0.751653in}}%
\pgfpathlineto{\pgfqpoint{1.410149in}{0.752868in}}%
\pgfpathlineto{\pgfqpoint{1.412726in}{0.753334in}}%
\pgfpathlineto{\pgfqpoint{1.423035in}{0.754221in}}%
\pgfpathlineto{\pgfqpoint{1.430767in}{0.755761in}}%
\pgfpathlineto{\pgfqpoint{1.459116in}{0.757492in}}%
\pgfpathlineto{\pgfqpoint{1.466848in}{0.758128in}}%
\pgfpathlineto{\pgfqpoint{1.482311in}{0.758944in}}%
\pgfpathlineto{\pgfqpoint{1.484888in}{0.759313in}}%
\pgfpathlineto{\pgfqpoint{1.495197in}{0.760112in}}%
\pgfpathlineto{\pgfqpoint{1.502929in}{0.762717in}}%
\pgfpathlineto{\pgfqpoint{1.513237in}{0.764189in}}%
\pgfpathlineto{\pgfqpoint{1.520969in}{0.766611in}}%
\pgfpathlineto{\pgfqpoint{1.531278in}{0.767940in}}%
\pgfpathlineto{\pgfqpoint{1.549318in}{0.771256in}}%
\pgfpathlineto{\pgfqpoint{1.557050in}{0.774829in}}%
\pgfpathlineto{\pgfqpoint{1.567359in}{0.776559in}}%
\pgfpathlineto{\pgfqpoint{1.575091in}{0.779255in}}%
\pgfpathlineto{\pgfqpoint{1.582822in}{0.780275in}}%
\pgfpathlineto{\pgfqpoint{1.593131in}{0.784821in}}%
\pgfpathlineto{\pgfqpoint{1.606017in}{0.786517in}}%
\pgfpathlineto{\pgfqpoint{1.611172in}{0.787670in}}%
\pgfpathlineto{\pgfqpoint{1.624058in}{0.788308in}}%
\pgfpathlineto{\pgfqpoint{1.629212in}{0.790676in}}%
\pgfpathlineto{\pgfqpoint{1.636944in}{0.791549in}}%
\pgfpathlineto{\pgfqpoint{1.644675in}{0.794474in}}%
\pgfpathlineto{\pgfqpoint{1.647253in}{0.795409in}}%
\pgfpathlineto{\pgfqpoint{1.657561in}{0.797183in}}%
\pgfpathlineto{\pgfqpoint{1.675602in}{0.802070in}}%
\pgfpathlineto{\pgfqpoint{1.678179in}{0.803401in}}%
\pgfpathlineto{\pgfqpoint{1.683333in}{0.804886in}}%
\pgfpathlineto{\pgfqpoint{1.691065in}{0.806213in}}%
\pgfpathlineto{\pgfqpoint{1.701374in}{0.811544in}}%
\pgfpathlineto{\pgfqpoint{1.709106in}{0.812866in}}%
\pgfpathlineto{\pgfqpoint{1.719414in}{0.817589in}}%
\pgfpathlineto{\pgfqpoint{1.727146in}{0.818748in}}%
\pgfpathlineto{\pgfqpoint{1.734878in}{0.821836in}}%
\pgfpathlineto{\pgfqpoint{1.737455in}{0.822689in}}%
\pgfpathlineto{\pgfqpoint{1.745187in}{0.823766in}}%
\pgfpathlineto{\pgfqpoint{1.755495in}{0.828519in}}%
\pgfpathlineto{\pgfqpoint{1.768381in}{0.830652in}}%
\pgfpathlineto{\pgfqpoint{1.773536in}{0.832395in}}%
\pgfpathlineto{\pgfqpoint{1.781268in}{0.833491in}}%
\pgfpathlineto{\pgfqpoint{1.786422in}{0.834982in}}%
\pgfpathlineto{\pgfqpoint{1.791576in}{0.838028in}}%
\pgfpathlineto{\pgfqpoint{1.799308in}{0.839702in}}%
\pgfpathlineto{\pgfqpoint{1.809617in}{0.847322in}}%
\pgfpathlineto{\pgfqpoint{1.817349in}{0.849095in}}%
\pgfpathlineto{\pgfqpoint{1.827657in}{0.855340in}}%
\pgfpathlineto{\pgfqpoint{1.837966in}{0.856736in}}%
\pgfpathlineto{\pgfqpoint{1.845698in}{0.860869in}}%
\pgfpathlineto{\pgfqpoint{1.853430in}{0.861876in}}%
\pgfpathlineto{\pgfqpoint{1.863738in}{0.865943in}}%
\pgfpathlineto{\pgfqpoint{1.871470in}{0.866869in}}%
\pgfpathlineto{\pgfqpoint{1.881779in}{0.871137in}}%
\pgfpathlineto{\pgfqpoint{1.892088in}{0.872748in}}%
\pgfpathlineto{\pgfqpoint{1.899819in}{0.875289in}}%
\pgfpathlineto{\pgfqpoint{1.912705in}{0.876939in}}%
\pgfpathlineto{\pgfqpoint{1.917860in}{0.878837in}}%
\pgfpathlineto{\pgfqpoint{1.925591in}{0.879547in}}%
\pgfpathlineto{\pgfqpoint{1.935900in}{0.883099in}}%
\pgfpathlineto{\pgfqpoint{1.943632in}{0.884016in}}%
\pgfpathlineto{\pgfqpoint{1.953941in}{0.887863in}}%
\pgfpathlineto{\pgfqpoint{1.964250in}{0.889653in}}%
\pgfpathlineto{\pgfqpoint{1.971981in}{0.891982in}}%
\pgfpathlineto{\pgfqpoint{1.982290in}{0.893105in}}%
\pgfpathlineto{\pgfqpoint{1.990022in}{0.895126in}}%
\pgfpathlineto{\pgfqpoint{1.997753in}{0.896148in}}%
\pgfpathlineto{\pgfqpoint{2.005485in}{0.900288in}}%
\pgfpathlineto{\pgfqpoint{2.015794in}{0.901499in}}%
\pgfpathlineto{\pgfqpoint{2.023526in}{0.904759in}}%
\pgfpathlineto{\pgfqpoint{2.026103in}{0.905705in}}%
\pgfpathlineto{\pgfqpoint{2.036412in}{0.907603in}}%
\pgfpathlineto{\pgfqpoint{2.044143in}{0.910604in}}%
\pgfpathlineto{\pgfqpoint{2.051875in}{0.911178in}}%
\pgfpathlineto{\pgfqpoint{2.062184in}{0.913833in}}%
\pgfpathlineto{\pgfqpoint{2.069915in}{0.914496in}}%
\pgfpathlineto{\pgfqpoint{2.080224in}{0.918021in}}%
\pgfpathlineto{\pgfqpoint{2.087956in}{0.918913in}}%
\pgfpathlineto{\pgfqpoint{2.095687in}{0.921988in}}%
\pgfpathlineto{\pgfqpoint{2.098265in}{0.923757in}}%
\pgfpathlineto{\pgfqpoint{2.105996in}{0.925433in}}%
\pgfpathlineto{\pgfqpoint{2.116305in}{0.931892in}}%
\pgfpathlineto{\pgfqpoint{2.124037in}{0.933419in}}%
\pgfpathlineto{\pgfqpoint{2.134346in}{0.940512in}}%
\pgfpathlineto{\pgfqpoint{2.142077in}{0.942226in}}%
\pgfpathlineto{\pgfqpoint{2.149809in}{0.946771in}}%
\pgfpathlineto{\pgfqpoint{2.152386in}{0.948273in}}%
\pgfpathlineto{\pgfqpoint{2.162695in}{0.949713in}}%
\pgfpathlineto{\pgfqpoint{2.170427in}{0.953676in}}%
\pgfpathlineto{\pgfqpoint{2.178158in}{0.955034in}}%
\pgfpathlineto{\pgfqpoint{2.188467in}{0.959941in}}%
\pgfpathlineto{\pgfqpoint{2.196199in}{0.961351in}}%
\pgfpathlineto{\pgfqpoint{2.206508in}{0.966224in}}%
\pgfpathlineto{\pgfqpoint{2.214239in}{0.967524in}}%
\pgfpathlineto{\pgfqpoint{2.221971in}{0.971174in}}%
\pgfpathlineto{\pgfqpoint{2.224548in}{0.972195in}}%
\pgfpathlineto{\pgfqpoint{2.232280in}{0.973120in}}%
\pgfpathlineto{\pgfqpoint{2.242589in}{0.977792in}}%
\pgfpathlineto{\pgfqpoint{2.250320in}{0.979099in}}%
\pgfpathlineto{\pgfqpoint{2.255475in}{0.981785in}}%
\pgfpathlineto{\pgfqpoint{2.260629in}{0.983441in}}%
\pgfpathlineto{\pgfqpoint{2.268361in}{0.984869in}}%
\pgfpathlineto{\pgfqpoint{2.278670in}{0.990512in}}%
\pgfpathlineto{\pgfqpoint{2.286401in}{0.991981in}}%
\pgfpathlineto{\pgfqpoint{2.296710in}{0.997548in}}%
\pgfpathlineto{\pgfqpoint{2.304442in}{0.998971in}}%
\pgfpathlineto{\pgfqpoint{2.314751in}{1.004358in}}%
\pgfpathlineto{\pgfqpoint{2.322482in}{1.005725in}}%
\pgfpathlineto{\pgfqpoint{2.325059in}{1.007035in}}%
\pgfpathlineto{\pgfqpoint{2.366295in}{1.013948in}}%
\pgfpathlineto{\pgfqpoint{2.404953in}{1.017602in}}%
\pgfpathlineto{\pgfqpoint{2.420416in}{1.018583in}}%
\pgfpathlineto{\pgfqpoint{2.422993in}{1.018908in}}%
\pgfpathlineto{\pgfqpoint{2.433302in}{1.019995in}}%
\pgfpathlineto{\pgfqpoint{2.441034in}{1.022342in}}%
\pgfpathlineto{\pgfqpoint{2.448766in}{1.023241in}}%
\pgfpathlineto{\pgfqpoint{2.456497in}{1.026545in}}%
\pgfpathlineto{\pgfqpoint{2.459074in}{1.027954in}}%
\pgfpathlineto{\pgfqpoint{2.466806in}{1.029181in}}%
\pgfpathlineto{\pgfqpoint{2.477115in}{1.033142in}}%
\pgfpathlineto{\pgfqpoint{2.484847in}{1.034001in}}%
\pgfpathlineto{\pgfqpoint{2.495155in}{1.037317in}}%
\pgfpathlineto{\pgfqpoint{2.508041in}{1.038716in}}%
\pgfpathlineto{\pgfqpoint{2.513196in}{1.040212in}}%
\pgfpathlineto{\pgfqpoint{2.523505in}{1.041848in}}%
\pgfpathlineto{\pgfqpoint{2.531236in}{1.045142in}}%
\pgfpathlineto{\pgfqpoint{2.538968in}{1.046318in}}%
\pgfpathlineto{\pgfqpoint{2.549277in}{1.051140in}}%
\pgfpathlineto{\pgfqpoint{2.557009in}{1.052413in}}%
\pgfpathlineto{\pgfqpoint{2.564740in}{1.055893in}}%
\pgfpathlineto{\pgfqpoint{2.567317in}{1.056886in}}%
\pgfpathlineto{\pgfqpoint{2.575049in}{1.057724in}}%
\pgfpathlineto{\pgfqpoint{2.585358in}{1.061223in}}%
\pgfpathlineto{\pgfqpoint{2.595667in}{1.063014in}}%
\pgfpathlineto{\pgfqpoint{2.603398in}{1.066155in}}%
\pgfpathlineto{\pgfqpoint{2.613707in}{1.067978in}}%
\pgfpathlineto{\pgfqpoint{2.621439in}{1.070931in}}%
\pgfpathlineto{\pgfqpoint{2.629170in}{1.071958in}}%
\pgfpathlineto{\pgfqpoint{2.634325in}{1.074166in}}%
\pgfpathlineto{\pgfqpoint{2.652365in}{1.078317in}}%
\pgfpathlineto{\pgfqpoint{2.657520in}{1.080165in}}%
\pgfpathlineto{\pgfqpoint{2.667829in}{1.081846in}}%
\pgfpathlineto{\pgfqpoint{2.675560in}{1.085030in}}%
\pgfpathlineto{\pgfqpoint{2.683292in}{1.086181in}}%
\pgfpathlineto{\pgfqpoint{2.693601in}{1.092281in}}%
\pgfpathlineto{\pgfqpoint{2.701332in}{1.093925in}}%
\pgfpathlineto{\pgfqpoint{2.703910in}{1.095592in}}%
\pgfpathlineto{\pgfqpoint{2.709064in}{1.097325in}}%
\pgfpathlineto{\pgfqpoint{2.711641in}{1.099003in}}%
\pgfpathlineto{\pgfqpoint{2.719373in}{1.100734in}}%
\pgfpathlineto{\pgfqpoint{2.721950in}{1.102553in}}%
\pgfpathlineto{\pgfqpoint{2.727105in}{1.104247in}}%
\pgfpathlineto{\pgfqpoint{2.729682in}{1.105928in}}%
\pgfpathlineto{\pgfqpoint{2.737413in}{1.107510in}}%
\pgfpathlineto{\pgfqpoint{2.747722in}{1.114073in}}%
\pgfpathlineto{\pgfqpoint{2.755454in}{1.115507in}}%
\pgfpathlineto{\pgfqpoint{2.765763in}{1.122478in}}%
\pgfpathlineto{\pgfqpoint{2.776072in}{1.124572in}}%
\pgfpathlineto{\pgfqpoint{2.781226in}{1.128526in}}%
\pgfpathlineto{\pgfqpoint{2.783803in}{1.129929in}}%
\pgfpathlineto{\pgfqpoint{2.791535in}{1.131059in}}%
\pgfpathlineto{\pgfqpoint{2.801844in}{1.135927in}}%
\pgfpathlineto{\pgfqpoint{2.809575in}{1.136862in}}%
\pgfpathlineto{\pgfqpoint{2.819884in}{1.141300in}}%
\pgfpathlineto{\pgfqpoint{2.827616in}{1.142511in}}%
\pgfpathlineto{\pgfqpoint{2.837925in}{1.148185in}}%
\pgfpathlineto{\pgfqpoint{2.848234in}{1.149684in}}%
\pgfpathlineto{\pgfqpoint{2.855965in}{1.153694in}}%
\pgfpathlineto{\pgfqpoint{2.863697in}{1.155135in}}%
\pgfpathlineto{\pgfqpoint{2.874006in}{1.160768in}}%
\pgfpathlineto{\pgfqpoint{2.881737in}{1.161896in}}%
\pgfpathlineto{\pgfqpoint{2.892046in}{1.166852in}}%
\pgfpathlineto{\pgfqpoint{2.899778in}{1.168109in}}%
\pgfpathlineto{\pgfqpoint{2.907509in}{1.171688in}}%
\pgfpathlineto{\pgfqpoint{2.910087in}{1.172673in}}%
\pgfpathlineto{\pgfqpoint{2.917818in}{1.173783in}}%
\pgfpathlineto{\pgfqpoint{2.928127in}{1.178124in}}%
\pgfpathlineto{\pgfqpoint{2.938436in}{1.179856in}}%
\pgfpathlineto{\pgfqpoint{2.946168in}{1.181786in}}%
\pgfpathlineto{\pgfqpoint{2.956476in}{1.183094in}}%
\pgfpathlineto{\pgfqpoint{2.964208in}{1.184702in}}%
\pgfpathlineto{\pgfqpoint{2.992557in}{1.186024in}}%
\pgfpathlineto{\pgfqpoint{2.997712in}{1.186747in}}%
\pgfpathlineto{\pgfqpoint{3.010598in}{1.187511in}}%
\pgfpathlineto{\pgfqpoint{3.018330in}{1.188289in}}%
\pgfpathlineto{\pgfqpoint{3.033793in}{1.188966in}}%
\pgfpathlineto{\pgfqpoint{3.069874in}{1.192177in}}%
\pgfpathlineto{\pgfqpoint{3.072451in}{1.192543in}}%
\pgfpathlineto{\pgfqpoint{3.085337in}{1.193587in}}%
\pgfpathlineto{\pgfqpoint{3.090491in}{1.194342in}}%
\pgfpathlineto{\pgfqpoint{3.103378in}{1.195300in}}%
\pgfpathlineto{\pgfqpoint{3.108532in}{1.196277in}}%
\pgfpathlineto{\pgfqpoint{3.121418in}{1.197449in}}%
\pgfpathlineto{\pgfqpoint{3.126572in}{1.198241in}}%
\pgfpathlineto{\pgfqpoint{3.136881in}{1.199067in}}%
\pgfpathlineto{\pgfqpoint{3.144613in}{1.200132in}}%
\pgfpathlineto{\pgfqpoint{3.157499in}{1.201061in}}%
\pgfpathlineto{\pgfqpoint{3.162653in}{1.201611in}}%
\pgfpathlineto{\pgfqpoint{3.178117in}{1.202550in}}%
\pgfpathlineto{\pgfqpoint{3.227084in}{1.207492in}}%
\pgfpathlineto{\pgfqpoint{3.234815in}{1.209044in}}%
\pgfpathlineto{\pgfqpoint{3.245124in}{1.210053in}}%
\pgfpathlineto{\pgfqpoint{3.252856in}{1.211475in}}%
\pgfpathlineto{\pgfqpoint{3.268319in}{1.212551in}}%
\pgfpathlineto{\pgfqpoint{3.283782in}{1.213374in}}%
\pgfpathlineto{\pgfqpoint{3.322441in}{1.216036in}}%
\pgfpathlineto{\pgfqpoint{3.325018in}{1.216377in}}%
\pgfpathlineto{\pgfqpoint{3.337904in}{1.217427in}}%
\pgfpathlineto{\pgfqpoint{3.343058in}{1.217940in}}%
\pgfpathlineto{\pgfqpoint{3.358522in}{1.218774in}}%
\pgfpathlineto{\pgfqpoint{3.379139in}{1.220421in}}%
\pgfpathlineto{\pgfqpoint{3.392025in}{1.221269in}}%
\pgfpathlineto{\pgfqpoint{3.397180in}{1.221847in}}%
\pgfpathlineto{\pgfqpoint{3.430684in}{1.223309in}}%
\pgfpathlineto{\pgfqpoint{3.448724in}{1.223868in}}%
\pgfpathlineto{\pgfqpoint{3.484805in}{1.224118in}}%
\pgfpathlineto{\pgfqpoint{3.502845in}{1.225677in}}%
\pgfpathlineto{\pgfqpoint{3.505423in}{1.226664in}}%
\pgfpathlineto{\pgfqpoint{3.513154in}{1.227646in}}%
\pgfpathlineto{\pgfqpoint{3.520886in}{1.231226in}}%
\pgfpathlineto{\pgfqpoint{3.523463in}{1.232593in}}%
\pgfpathlineto{\pgfqpoint{3.531195in}{1.233857in}}%
\pgfpathlineto{\pgfqpoint{3.541504in}{1.239067in}}%
\pgfpathlineto{\pgfqpoint{3.549235in}{1.240338in}}%
\pgfpathlineto{\pgfqpoint{3.559544in}{1.245565in}}%
\pgfpathlineto{\pgfqpoint{3.567276in}{1.246975in}}%
\pgfpathlineto{\pgfqpoint{3.572430in}{1.249976in}}%
\pgfpathlineto{\pgfqpoint{3.577585in}{1.251496in}}%
\pgfpathlineto{\pgfqpoint{3.585316in}{1.253000in}}%
\pgfpathlineto{\pgfqpoint{3.595625in}{1.259456in}}%
\pgfpathlineto{\pgfqpoint{3.603357in}{1.261101in}}%
\pgfpathlineto{\pgfqpoint{3.613666in}{1.267274in}}%
\pgfpathlineto{\pgfqpoint{3.621397in}{1.268593in}}%
\pgfpathlineto{\pgfqpoint{3.626552in}{1.271167in}}%
\pgfpathlineto{\pgfqpoint{3.631706in}{1.274223in}}%
\pgfpathlineto{\pgfqpoint{3.639438in}{1.275784in}}%
\pgfpathlineto{\pgfqpoint{3.644592in}{1.279052in}}%
\pgfpathlineto{\pgfqpoint{3.649747in}{1.280672in}}%
\pgfpathlineto{\pgfqpoint{3.657478in}{1.282231in}}%
\pgfpathlineto{\pgfqpoint{3.662633in}{1.285142in}}%
\pgfpathlineto{\pgfqpoint{3.667787in}{1.286646in}}%
\pgfpathlineto{\pgfqpoint{3.675519in}{1.287909in}}%
\pgfpathlineto{\pgfqpoint{3.685828in}{1.293118in}}%
\pgfpathlineto{\pgfqpoint{3.693559in}{1.294361in}}%
\pgfpathlineto{\pgfqpoint{3.701291in}{1.297658in}}%
\pgfpathlineto{\pgfqpoint{3.703868in}{1.298684in}}%
\pgfpathlineto{\pgfqpoint{3.714177in}{1.299768in}}%
\pgfpathlineto{\pgfqpoint{3.721909in}{1.303060in}}%
\pgfpathlineto{\pgfqpoint{3.732217in}{1.304946in}}%
\pgfpathlineto{\pgfqpoint{3.739949in}{1.307341in}}%
\pgfpathlineto{\pgfqpoint{3.747681in}{1.308304in}}%
\pgfpathlineto{\pgfqpoint{3.755412in}{1.312222in}}%
\pgfpathlineto{\pgfqpoint{3.757990in}{1.313574in}}%
\pgfpathlineto{\pgfqpoint{3.765721in}{1.314842in}}%
\pgfpathlineto{\pgfqpoint{3.773453in}{1.318925in}}%
\pgfpathlineto{\pgfqpoint{3.776030in}{1.320380in}}%
\pgfpathlineto{\pgfqpoint{3.786339in}{1.321871in}}%
\pgfpathlineto{\pgfqpoint{3.794070in}{1.326222in}}%
\pgfpathlineto{\pgfqpoint{3.801802in}{1.327749in}}%
\pgfpathlineto{\pgfqpoint{3.812111in}{1.333680in}}%
\pgfpathlineto{\pgfqpoint{3.819843in}{1.335353in}}%
\pgfpathlineto{\pgfqpoint{3.830151in}{1.341127in}}%
\pgfpathlineto{\pgfqpoint{3.837883in}{1.342478in}}%
\pgfpathlineto{\pgfqpoint{3.848192in}{1.347047in}}%
\pgfpathlineto{\pgfqpoint{3.855924in}{1.348266in}}%
\pgfpathlineto{\pgfqpoint{3.866232in}{1.352813in}}%
\pgfpathlineto{\pgfqpoint{3.873964in}{1.353964in}}%
\pgfpathlineto{\pgfqpoint{3.881696in}{1.356998in}}%
\pgfpathlineto{\pgfqpoint{3.884273in}{1.357918in}}%
\pgfpathlineto{\pgfqpoint{3.892005in}{1.358846in}}%
\pgfpathlineto{\pgfqpoint{3.899736in}{1.361454in}}%
\pgfpathlineto{\pgfqpoint{3.910045in}{1.362299in}}%
\pgfpathlineto{\pgfqpoint{3.920354in}{1.366140in}}%
\pgfpathlineto{\pgfqpoint{3.930663in}{1.367849in}}%
\pgfpathlineto{\pgfqpoint{3.938394in}{1.370267in}}%
\pgfpathlineto{\pgfqpoint{3.948703in}{1.371793in}}%
\pgfpathlineto{\pgfqpoint{3.956435in}{1.375017in}}%
\pgfpathlineto{\pgfqpoint{3.964167in}{1.375996in}}%
\pgfpathlineto{\pgfqpoint{3.974475in}{1.379547in}}%
\pgfpathlineto{\pgfqpoint{3.984784in}{1.381118in}}%
\pgfpathlineto{\pgfqpoint{3.992516in}{1.383971in}}%
\pgfpathlineto{\pgfqpoint{4.000247in}{1.385124in}}%
\pgfpathlineto{\pgfqpoint{4.010556in}{1.389812in}}%
\pgfpathlineto{\pgfqpoint{4.018288in}{1.391068in}}%
\pgfpathlineto{\pgfqpoint{4.028597in}{1.395862in}}%
\pgfpathlineto{\pgfqpoint{4.038906in}{1.396888in}}%
\pgfpathlineto{\pgfqpoint{4.046637in}{1.400186in}}%
\pgfpathlineto{\pgfqpoint{4.054369in}{1.401253in}}%
\pgfpathlineto{\pgfqpoint{4.064678in}{1.405222in}}%
\pgfpathlineto{\pgfqpoint{4.072409in}{1.406083in}}%
\pgfpathlineto{\pgfqpoint{4.082718in}{1.410157in}}%
\pgfpathlineto{\pgfqpoint{4.090450in}{1.411089in}}%
\pgfpathlineto{\pgfqpoint{4.100759in}{1.414950in}}%
\pgfpathlineto{\pgfqpoint{4.108490in}{1.415911in}}%
\pgfpathlineto{\pgfqpoint{4.118799in}{1.419607in}}%
\pgfpathlineto{\pgfqpoint{4.129108in}{1.420982in}}%
\pgfpathlineto{\pgfqpoint{4.134263in}{1.422576in}}%
\pgfpathlineto{\pgfqpoint{4.147149in}{1.424126in}}%
\pgfpathlineto{\pgfqpoint{4.154880in}{1.426244in}}%
\pgfpathlineto{\pgfqpoint{4.162612in}{1.427171in}}%
\pgfpathlineto{\pgfqpoint{4.172921in}{1.431220in}}%
\pgfpathlineto{\pgfqpoint{4.180652in}{1.432497in}}%
\pgfpathlineto{\pgfqpoint{4.190961in}{1.437500in}}%
\pgfpathlineto{\pgfqpoint{4.198693in}{1.438882in}}%
\pgfpathlineto{\pgfqpoint{4.209002in}{1.445252in}}%
\pgfpathlineto{\pgfqpoint{4.216733in}{1.446837in}}%
\pgfpathlineto{\pgfqpoint{4.227042in}{1.452405in}}%
\pgfpathlineto{\pgfqpoint{4.234774in}{1.453756in}}%
\pgfpathlineto{\pgfqpoint{4.245083in}{1.458807in}}%
\pgfpathlineto{\pgfqpoint{4.252814in}{1.460134in}}%
\pgfpathlineto{\pgfqpoint{4.263123in}{1.464922in}}%
\pgfpathlineto{\pgfqpoint{4.273432in}{1.466017in}}%
\pgfpathlineto{\pgfqpoint{4.281164in}{1.468915in}}%
\pgfpathlineto{\pgfqpoint{4.291472in}{1.470452in}}%
\pgfpathlineto{\pgfqpoint{4.299204in}{1.472590in}}%
\pgfpathlineto{\pgfqpoint{4.312090in}{1.474061in}}%
\pgfpathlineto{\pgfqpoint{4.317245in}{1.475598in}}%
\pgfpathlineto{\pgfqpoint{4.327553in}{1.477069in}}%
\pgfpathlineto{\pgfqpoint{4.335285in}{1.479337in}}%
\pgfpathlineto{\pgfqpoint{4.345594in}{1.480811in}}%
\pgfpathlineto{\pgfqpoint{4.353326in}{1.483020in}}%
\pgfpathlineto{\pgfqpoint{4.363634in}{1.483953in}}%
\pgfpathlineto{\pgfqpoint{4.371366in}{1.485945in}}%
\pgfpathlineto{\pgfqpoint{4.381675in}{1.487662in}}%
\pgfpathlineto{\pgfqpoint{4.389407in}{1.490691in}}%
\pgfpathlineto{\pgfqpoint{4.397138in}{1.491832in}}%
\pgfpathlineto{\pgfqpoint{4.407447in}{1.496394in}}%
\pgfpathlineto{\pgfqpoint{4.415179in}{1.497732in}}%
\pgfpathlineto{\pgfqpoint{4.425488in}{1.502684in}}%
\pgfpathlineto{\pgfqpoint{4.433219in}{1.504126in}}%
\pgfpathlineto{\pgfqpoint{4.443528in}{1.509776in}}%
\pgfpathlineto{\pgfqpoint{4.451260in}{1.510830in}}%
\pgfpathlineto{\pgfqpoint{4.461569in}{1.516521in}}%
\pgfpathlineto{\pgfqpoint{4.469300in}{1.517857in}}%
\pgfpathlineto{\pgfqpoint{4.479609in}{1.523534in}}%
\pgfpathlineto{\pgfqpoint{4.487341in}{1.524946in}}%
\pgfpathlineto{\pgfqpoint{4.497649in}{1.530996in}}%
\pgfpathlineto{\pgfqpoint{4.505381in}{1.532509in}}%
\pgfpathlineto{\pgfqpoint{4.510536in}{1.535400in}}%
\pgfpathlineto{\pgfqpoint{4.515690in}{1.536851in}}%
\pgfpathlineto{\pgfqpoint{4.523422in}{1.538195in}}%
\pgfpathlineto{\pgfqpoint{4.533730in}{1.543678in}}%
\pgfpathlineto{\pgfqpoint{4.541462in}{1.545043in}}%
\pgfpathlineto{\pgfqpoint{4.551771in}{1.549638in}}%
\pgfpathlineto{\pgfqpoint{4.559503in}{1.550668in}}%
\pgfpathlineto{\pgfqpoint{4.569811in}{1.555381in}}%
\pgfpathlineto{\pgfqpoint{4.577543in}{1.556410in}}%
\pgfpathlineto{\pgfqpoint{4.585275in}{1.559745in}}%
\pgfpathlineto{\pgfqpoint{4.595584in}{1.560893in}}%
\pgfpathlineto{\pgfqpoint{4.603315in}{1.564245in}}%
\pgfpathlineto{\pgfqpoint{4.613624in}{1.565077in}}%
\pgfpathlineto{\pgfqpoint{4.621356in}{1.567382in}}%
\pgfpathlineto{\pgfqpoint{4.623933in}{1.567941in}}%
\pgfpathlineto{\pgfqpoint{4.634242in}{1.569304in}}%
\pgfpathlineto{\pgfqpoint{4.641973in}{1.570946in}}%
\pgfpathlineto{\pgfqpoint{4.654859in}{1.571721in}}%
\pgfpathlineto{\pgfqpoint{4.660014in}{1.572658in}}%
\pgfpathlineto{\pgfqpoint{4.672900in}{1.573894in}}%
\pgfpathlineto{\pgfqpoint{4.685786in}{1.575428in}}%
\pgfpathlineto{\pgfqpoint{4.696095in}{1.577566in}}%
\pgfpathlineto{\pgfqpoint{4.711558in}{1.578316in}}%
\pgfpathlineto{\pgfqpoint{4.714135in}{1.578624in}}%
\pgfpathlineto{\pgfqpoint{4.727021in}{1.579416in}}%
\pgfpathlineto{\pgfqpoint{4.732176in}{1.580179in}}%
\pgfpathlineto{\pgfqpoint{4.745062in}{1.581532in}}%
\pgfpathlineto{\pgfqpoint{4.750216in}{1.582521in}}%
\pgfpathlineto{\pgfqpoint{4.760525in}{1.583586in}}%
\pgfpathlineto{\pgfqpoint{4.768257in}{1.585283in}}%
\pgfpathlineto{\pgfqpoint{4.783720in}{1.586523in}}%
\pgfpathlineto{\pgfqpoint{4.786297in}{1.586883in}}%
\pgfpathlineto{\pgfqpoint{4.799183in}{1.588014in}}%
\pgfpathlineto{\pgfqpoint{4.804338in}{1.588992in}}%
\pgfpathlineto{\pgfqpoint{4.814647in}{1.589944in}}%
\pgfpathlineto{\pgfqpoint{4.819801in}{1.590989in}}%
\pgfpathlineto{\pgfqpoint{4.832687in}{1.592138in}}%
\pgfpathlineto{\pgfqpoint{4.840419in}{1.594406in}}%
\pgfpathlineto{\pgfqpoint{4.850728in}{1.595977in}}%
\pgfpathlineto{\pgfqpoint{4.858459in}{1.598447in}}%
\pgfpathlineto{\pgfqpoint{4.866191in}{1.599250in}}%
\pgfpathlineto{\pgfqpoint{4.876500in}{1.603104in}}%
\pgfpathlineto{\pgfqpoint{4.884231in}{1.604229in}}%
\pgfpathlineto{\pgfqpoint{4.894540in}{1.608278in}}%
\pgfpathlineto{\pgfqpoint{4.904849in}{1.609874in}}%
\pgfpathlineto{\pgfqpoint{4.912581in}{1.612121in}}%
\pgfpathlineto{\pgfqpoint{4.922890in}{1.613586in}}%
\pgfpathlineto{\pgfqpoint{4.930621in}{1.615638in}}%
\pgfpathlineto{\pgfqpoint{4.938353in}{1.616386in}}%
\pgfpathlineto{\pgfqpoint{4.948662in}{1.619236in}}%
\pgfpathlineto{\pgfqpoint{4.958971in}{1.620556in}}%
\pgfpathlineto{\pgfqpoint{4.966702in}{1.622517in}}%
\pgfpathlineto{\pgfqpoint{4.974434in}{1.623153in}}%
\pgfpathlineto{\pgfqpoint{4.984743in}{1.626363in}}%
\pgfpathlineto{\pgfqpoint{4.995051in}{1.627126in}}%
\pgfpathlineto{\pgfqpoint{5.002783in}{1.629588in}}%
\pgfpathlineto{\pgfqpoint{5.010515in}{1.630475in}}%
\pgfpathlineto{\pgfqpoint{5.020824in}{1.634115in}}%
\pgfpathlineto{\pgfqpoint{5.031132in}{1.635483in}}%
\pgfpathlineto{\pgfqpoint{5.038864in}{1.637349in}}%
\pgfpathlineto{\pgfqpoint{5.049173in}{1.638496in}}%
\pgfpathlineto{\pgfqpoint{5.056905in}{1.640031in}}%
\pgfpathlineto{\pgfqpoint{5.067213in}{1.640700in}}%
\pgfpathlineto{\pgfqpoint{5.074945in}{1.641872in}}%
\pgfpathlineto{\pgfqpoint{5.090408in}{1.642870in}}%
\pgfpathlineto{\pgfqpoint{5.092986in}{1.643341in}}%
\pgfpathlineto{\pgfqpoint{5.103294in}{1.644360in}}%
\pgfpathlineto{\pgfqpoint{5.111026in}{1.646089in}}%
\pgfpathlineto{\pgfqpoint{5.121335in}{1.647312in}}%
\pgfpathlineto{\pgfqpoint{5.129067in}{1.649322in}}%
\pgfpathlineto{\pgfqpoint{5.139375in}{1.650542in}}%
\pgfpathlineto{\pgfqpoint{5.147107in}{1.652320in}}%
\pgfpathlineto{\pgfqpoint{5.157416in}{1.653440in}}%
\pgfpathlineto{\pgfqpoint{5.165148in}{1.655371in}}%
\pgfpathlineto{\pgfqpoint{5.175456in}{1.656730in}}%
\pgfpathlineto{\pgfqpoint{5.183188in}{1.658726in}}%
\pgfpathlineto{\pgfqpoint{5.193497in}{1.660188in}}%
\pgfpathlineto{\pgfqpoint{5.201228in}{1.662306in}}%
\pgfpathlineto{\pgfqpoint{5.211537in}{1.663765in}}%
\pgfpathlineto{\pgfqpoint{5.219269in}{1.665795in}}%
\pgfpathlineto{\pgfqpoint{5.229578in}{1.667232in}}%
\pgfpathlineto{\pgfqpoint{5.237309in}{1.669453in}}%
\pgfpathlineto{\pgfqpoint{5.247618in}{1.670279in}}%
\pgfpathlineto{\pgfqpoint{5.255350in}{1.672730in}}%
\pgfpathlineto{\pgfqpoint{5.265659in}{1.674328in}}%
\pgfpathlineto{\pgfqpoint{5.273390in}{1.676555in}}%
\pgfpathlineto{\pgfqpoint{5.283699in}{1.678124in}}%
\pgfpathlineto{\pgfqpoint{5.291431in}{1.680583in}}%
\pgfpathlineto{\pgfqpoint{5.301740in}{1.682030in}}%
\pgfpathlineto{\pgfqpoint{5.309471in}{1.684305in}}%
\pgfpathlineto{\pgfqpoint{5.319780in}{1.685845in}}%
\pgfpathlineto{\pgfqpoint{5.327512in}{1.688223in}}%
\pgfpathlineto{\pgfqpoint{5.337821in}{1.689691in}}%
\pgfpathlineto{\pgfqpoint{5.345552in}{1.691772in}}%
\pgfpathlineto{\pgfqpoint{5.355861in}{1.693093in}}%
\pgfpathlineto{\pgfqpoint{5.363593in}{1.695207in}}%
\pgfpathlineto{\pgfqpoint{5.373902in}{1.696608in}}%
\pgfpathlineto{\pgfqpoint{5.381633in}{1.698535in}}%
\pgfpathlineto{\pgfqpoint{5.391942in}{1.699820in}}%
\pgfpathlineto{\pgfqpoint{5.399674in}{1.701387in}}%
\pgfpathlineto{\pgfqpoint{5.409983in}{1.702735in}}%
\pgfpathlineto{\pgfqpoint{5.417714in}{1.704653in}}%
\pgfpathlineto{\pgfqpoint{5.428023in}{1.705379in}}%
\pgfpathlineto{\pgfqpoint{5.435755in}{1.706927in}}%
\pgfpathlineto{\pgfqpoint{5.448641in}{1.708399in}}%
\pgfpathlineto{\pgfqpoint{5.471836in}{1.710283in}}%
\pgfpathlineto{\pgfqpoint{5.484722in}{1.711212in}}%
\pgfpathlineto{\pgfqpoint{5.489876in}{1.711985in}}%
\pgfpathlineto{\pgfqpoint{5.502762in}{1.713077in}}%
\pgfpathlineto{\pgfqpoint{5.507917in}{1.713793in}}%
\pgfpathlineto{\pgfqpoint{5.520803in}{1.714716in}}%
\pgfpathlineto{\pgfqpoint{5.525957in}{1.715301in}}%
\pgfpathlineto{\pgfqpoint{5.541421in}{1.716221in}}%
\pgfpathlineto{\pgfqpoint{5.543998in}{1.716506in}}%
\pgfpathlineto{\pgfqpoint{5.556884in}{1.717295in}}%
\pgfpathlineto{\pgfqpoint{5.562038in}{1.718320in}}%
\pgfpathlineto{\pgfqpoint{5.572347in}{1.719316in}}%
\pgfpathlineto{\pgfqpoint{5.580079in}{1.720776in}}%
\pgfpathlineto{\pgfqpoint{5.592965in}{1.721734in}}%
\pgfpathlineto{\pgfqpoint{5.598119in}{1.722731in}}%
\pgfpathlineto{\pgfqpoint{5.608428in}{1.723843in}}%
\pgfpathlineto{\pgfqpoint{5.616160in}{1.725691in}}%
\pgfpathlineto{\pgfqpoint{5.626469in}{1.726848in}}%
\pgfpathlineto{\pgfqpoint{5.667704in}{1.735675in}}%
\pgfpathlineto{\pgfqpoint{5.670281in}{1.736538in}}%
\pgfpathlineto{\pgfqpoint{5.680590in}{1.737433in}}%
\pgfpathlineto{\pgfqpoint{5.688322in}{1.740147in}}%
\pgfpathlineto{\pgfqpoint{5.696053in}{1.741032in}}%
\pgfpathlineto{\pgfqpoint{5.706362in}{1.744712in}}%
\pgfpathlineto{\pgfqpoint{5.714094in}{1.745652in}}%
\pgfpathlineto{\pgfqpoint{5.724403in}{1.749457in}}%
\pgfpathlineto{\pgfqpoint{5.732134in}{1.750470in}}%
\pgfpathlineto{\pgfqpoint{5.742443in}{1.754419in}}%
\pgfpathlineto{\pgfqpoint{5.752752in}{1.756158in}}%
\pgfpathlineto{\pgfqpoint{5.760484in}{1.758792in}}%
\pgfpathlineto{\pgfqpoint{5.768215in}{1.759669in}}%
\pgfpathlineto{\pgfqpoint{5.778524in}{1.763167in}}%
\pgfpathlineto{\pgfqpoint{5.788833in}{1.764895in}}%
\pgfpathlineto{\pgfqpoint{5.796565in}{1.767433in}}%
\pgfpathlineto{\pgfqpoint{5.806873in}{1.769090in}}%
\pgfpathlineto{\pgfqpoint{5.812028in}{1.770720in}}%
\pgfpathlineto{\pgfqpoint{5.822337in}{1.771609in}}%
\pgfpathlineto{\pgfqpoint{5.832646in}{1.775325in}}%
\pgfpathlineto{\pgfqpoint{5.840377in}{1.776355in}}%
\pgfpathlineto{\pgfqpoint{5.850686in}{1.780376in}}%
\pgfpathlineto{\pgfqpoint{5.858418in}{1.781335in}}%
\pgfpathlineto{\pgfqpoint{5.868727in}{1.785501in}}%
\pgfpathlineto{\pgfqpoint{5.876458in}{1.786491in}}%
\pgfpathlineto{\pgfqpoint{5.886767in}{1.790516in}}%
\pgfpathlineto{\pgfqpoint{5.894499in}{1.791588in}}%
\pgfpathlineto{\pgfqpoint{5.904807in}{1.795566in}}%
\pgfpathlineto{\pgfqpoint{5.912539in}{1.796636in}}%
\pgfpathlineto{\pgfqpoint{5.922848in}{1.801286in}}%
\pgfpathlineto{\pgfqpoint{5.933157in}{1.802446in}}%
\pgfpathlineto{\pgfqpoint{5.940888in}{1.806115in}}%
\pgfpathlineto{\pgfqpoint{5.948620in}{1.807407in}}%
\pgfpathlineto{\pgfqpoint{5.958929in}{1.812213in}}%
\pgfpathlineto{\pgfqpoint{5.966661in}{1.813228in}}%
\pgfpathlineto{\pgfqpoint{5.976969in}{1.817621in}}%
\pgfpathlineto{\pgfqpoint{5.984701in}{1.818715in}}%
\pgfpathlineto{\pgfqpoint{5.995010in}{1.822989in}}%
\pgfpathlineto{\pgfqpoint{6.002742in}{1.824094in}}%
\pgfpathlineto{\pgfqpoint{6.013050in}{1.828346in}}%
\pgfpathlineto{\pgfqpoint{6.025936in}{1.830264in}}%
\pgfpathlineto{\pgfqpoint{6.031091in}{1.832152in}}%
\pgfpathlineto{\pgfqpoint{6.038823in}{1.833205in}}%
\pgfpathlineto{\pgfqpoint{6.049131in}{1.837699in}}%
\pgfpathlineto{\pgfqpoint{6.056863in}{1.838867in}}%
\pgfpathlineto{\pgfqpoint{6.067172in}{1.844003in}}%
\pgfpathlineto{\pgfqpoint{6.074904in}{1.845441in}}%
\pgfpathlineto{\pgfqpoint{6.085212in}{1.850971in}}%
\pgfpathlineto{\pgfqpoint{6.092944in}{1.852309in}}%
\pgfpathlineto{\pgfqpoint{6.103253in}{1.858064in}}%
\pgfpathlineto{\pgfqpoint{6.110984in}{1.859538in}}%
\pgfpathlineto{\pgfqpoint{6.121293in}{1.865016in}}%
\pgfpathlineto{\pgfqpoint{6.129025in}{1.866491in}}%
\pgfpathlineto{\pgfqpoint{6.139334in}{1.872643in}}%
\pgfpathlineto{\pgfqpoint{6.147065in}{1.874218in}}%
\pgfpathlineto{\pgfqpoint{6.157374in}{1.880530in}}%
\pgfpathlineto{\pgfqpoint{6.165106in}{1.882119in}}%
\pgfpathlineto{\pgfqpoint{6.175415in}{1.888388in}}%
\pgfpathlineto{\pgfqpoint{6.185724in}{1.889875in}}%
\pgfpathlineto{\pgfqpoint{6.193455in}{1.894570in}}%
\pgfpathlineto{\pgfqpoint{6.201187in}{1.896309in}}%
\pgfpathlineto{\pgfqpoint{6.211496in}{1.903081in}}%
\pgfpathlineto{\pgfqpoint{6.219227in}{1.904663in}}%
\pgfpathlineto{\pgfqpoint{6.229536in}{1.911093in}}%
\pgfpathlineto{\pgfqpoint{6.237268in}{1.912486in}}%
\pgfpathlineto{\pgfqpoint{6.247577in}{1.918386in}}%
\pgfpathlineto{\pgfqpoint{6.255308in}{1.919953in}}%
\pgfpathlineto{\pgfqpoint{6.265617in}{1.926340in}}%
\pgfpathlineto{\pgfqpoint{6.273349in}{1.928010in}}%
\pgfpathlineto{\pgfqpoint{6.283658in}{1.935015in}}%
\pgfpathlineto{\pgfqpoint{6.291389in}{1.936762in}}%
\pgfpathlineto{\pgfqpoint{6.301698in}{1.943383in}}%
\pgfpathlineto{\pgfqpoint{6.309430in}{1.945025in}}%
\pgfpathlineto{\pgfqpoint{6.319739in}{1.952162in}}%
\pgfpathlineto{\pgfqpoint{6.327470in}{1.953988in}}%
\pgfpathlineto{\pgfqpoint{6.337779in}{1.961505in}}%
\pgfpathlineto{\pgfqpoint{6.345511in}{1.963461in}}%
\pgfpathlineto{\pgfqpoint{6.355820in}{1.971281in}}%
\pgfpathlineto{\pgfqpoint{6.363551in}{1.973147in}}%
\pgfpathlineto{\pgfqpoint{6.373860in}{1.980338in}}%
\pgfpathlineto{\pgfqpoint{6.381592in}{1.982053in}}%
\pgfpathlineto{\pgfqpoint{6.386746in}{1.985668in}}%
\pgfpathlineto{\pgfqpoint{6.391901in}{1.987537in}}%
\pgfpathlineto{\pgfqpoint{6.399632in}{1.989436in}}%
\pgfpathlineto{\pgfqpoint{6.409941in}{1.996690in}}%
\pgfpathlineto{\pgfqpoint{6.417673in}{1.998128in}}%
\pgfpathlineto{\pgfqpoint{6.427982in}{2.004868in}}%
\pgfpathlineto{\pgfqpoint{6.435713in}{2.006688in}}%
\pgfpathlineto{\pgfqpoint{6.446022in}{2.014244in}}%
\pgfpathlineto{\pgfqpoint{6.453754in}{2.016128in}}%
\pgfpathlineto{\pgfqpoint{6.464063in}{2.023161in}}%
\pgfpathlineto{\pgfqpoint{6.474371in}{2.024956in}}%
\pgfpathlineto{\pgfqpoint{6.482103in}{2.030593in}}%
\pgfpathlineto{\pgfqpoint{6.482103in}{2.030593in}}%
\pgfusepath{stroke}%
\end{pgfscope}%
\begin{pgfscope}%
\pgfpathrectangle{\pgfqpoint{0.563921in}{0.521603in}}{\pgfqpoint{6.200000in}{2.642500in}}%
\pgfusepath{clip}%
\pgfsetroundcap%
\pgfsetroundjoin%
\pgfsetlinewidth{1.505625pt}%
\definecolor{currentstroke}{rgb}{0.090196,0.745098,0.811765}%
\pgfsetstrokecolor{currentstroke}%
\pgfsetdash{}{0pt}%
\pgfpathmoveto{\pgfqpoint{0.845739in}{0.641717in}}%
\pgfpathlineto{\pgfqpoint{0.850894in}{0.667793in}}%
\pgfpathlineto{\pgfqpoint{0.853471in}{0.674543in}}%
\pgfpathlineto{\pgfqpoint{0.861203in}{0.673799in}}%
\pgfpathlineto{\pgfqpoint{0.863780in}{0.671929in}}%
\pgfpathlineto{\pgfqpoint{0.866357in}{0.672040in}}%
\pgfpathlineto{\pgfqpoint{0.868934in}{0.671504in}}%
\pgfpathlineto{\pgfqpoint{0.871512in}{0.672660in}}%
\pgfpathlineto{\pgfqpoint{0.881820in}{0.672656in}}%
\pgfpathlineto{\pgfqpoint{0.884398in}{0.671216in}}%
\pgfpathlineto{\pgfqpoint{0.889552in}{0.669705in}}%
\pgfpathlineto{\pgfqpoint{0.902438in}{0.667795in}}%
\pgfpathlineto{\pgfqpoint{0.907593in}{0.666479in}}%
\pgfpathlineto{\pgfqpoint{0.920479in}{0.664980in}}%
\pgfpathlineto{\pgfqpoint{0.923056in}{0.664584in}}%
\pgfpathlineto{\pgfqpoint{0.925633in}{0.666039in}}%
\pgfpathlineto{\pgfqpoint{0.933365in}{0.669259in}}%
\pgfpathlineto{\pgfqpoint{0.941096in}{0.685017in}}%
\pgfpathlineto{\pgfqpoint{0.943674in}{0.688665in}}%
\pgfpathlineto{\pgfqpoint{0.951405in}{0.692931in}}%
\pgfpathlineto{\pgfqpoint{0.953982in}{0.695645in}}%
\pgfpathlineto{\pgfqpoint{0.959137in}{0.698610in}}%
\pgfpathlineto{\pgfqpoint{0.961714in}{0.700611in}}%
\pgfpathlineto{\pgfqpoint{0.987486in}{0.704062in}}%
\pgfpathlineto{\pgfqpoint{0.995218in}{0.708346in}}%
\pgfpathlineto{\pgfqpoint{0.997795in}{0.709880in}}%
\pgfpathlineto{\pgfqpoint{1.008104in}{0.712454in}}%
\pgfpathlineto{\pgfqpoint{1.013258in}{0.712944in}}%
\pgfpathlineto{\pgfqpoint{1.015835in}{0.713601in}}%
\pgfpathlineto{\pgfqpoint{1.023567in}{0.714318in}}%
\pgfpathlineto{\pgfqpoint{1.026144in}{0.718382in}}%
\pgfpathlineto{\pgfqpoint{1.031299in}{0.722755in}}%
\pgfpathlineto{\pgfqpoint{1.033876in}{0.724084in}}%
\pgfpathlineto{\pgfqpoint{1.041608in}{0.725688in}}%
\pgfpathlineto{\pgfqpoint{1.051916in}{0.730161in}}%
\pgfpathlineto{\pgfqpoint{1.059648in}{0.732594in}}%
\pgfpathlineto{\pgfqpoint{1.062225in}{0.734409in}}%
\pgfpathlineto{\pgfqpoint{1.064802in}{0.735153in}}%
\pgfpathlineto{\pgfqpoint{1.067380in}{0.735286in}}%
\pgfpathlineto{\pgfqpoint{1.069957in}{0.736226in}}%
\pgfpathlineto{\pgfqpoint{1.082843in}{0.737304in}}%
\pgfpathlineto{\pgfqpoint{1.085420in}{0.737305in}}%
\pgfpathlineto{\pgfqpoint{1.098306in}{0.736055in}}%
\pgfpathlineto{\pgfqpoint{1.106038in}{0.734247in}}%
\pgfpathlineto{\pgfqpoint{1.121501in}{0.732512in}}%
\pgfpathlineto{\pgfqpoint{1.124078in}{0.732104in}}%
\pgfpathlineto{\pgfqpoint{1.142119in}{0.731241in}}%
\pgfpathlineto{\pgfqpoint{1.152428in}{0.731676in}}%
\pgfpathlineto{\pgfqpoint{1.170468in}{0.733274in}}%
\pgfpathlineto{\pgfqpoint{1.178200in}{0.738292in}}%
\pgfpathlineto{\pgfqpoint{1.185931in}{0.739608in}}%
\pgfpathlineto{\pgfqpoint{1.191086in}{0.741755in}}%
\pgfpathlineto{\pgfqpoint{1.196240in}{0.742150in}}%
\pgfpathlineto{\pgfqpoint{1.211704in}{0.743409in}}%
\pgfpathlineto{\pgfqpoint{1.227167in}{0.745674in}}%
\pgfpathlineto{\pgfqpoint{1.229744in}{0.746879in}}%
\pgfpathlineto{\pgfqpoint{1.242630in}{0.747587in}}%
\pgfpathlineto{\pgfqpoint{1.250362in}{0.750791in}}%
\pgfpathlineto{\pgfqpoint{1.258093in}{0.751733in}}%
\pgfpathlineto{\pgfqpoint{1.265825in}{0.756335in}}%
\pgfpathlineto{\pgfqpoint{1.268402in}{0.758151in}}%
\pgfpathlineto{\pgfqpoint{1.276134in}{0.759879in}}%
\pgfpathlineto{\pgfqpoint{1.286443in}{0.767480in}}%
\pgfpathlineto{\pgfqpoint{1.294174in}{0.768458in}}%
\pgfpathlineto{\pgfqpoint{1.299329in}{0.771688in}}%
\pgfpathlineto{\pgfqpoint{1.304483in}{0.775601in}}%
\pgfpathlineto{\pgfqpoint{1.312215in}{0.777923in}}%
\pgfpathlineto{\pgfqpoint{1.314792in}{0.780012in}}%
\pgfpathlineto{\pgfqpoint{1.319947in}{0.781581in}}%
\pgfpathlineto{\pgfqpoint{1.322524in}{0.783024in}}%
\pgfpathlineto{\pgfqpoint{1.332833in}{0.785210in}}%
\pgfpathlineto{\pgfqpoint{1.337987in}{0.786754in}}%
\pgfpathlineto{\pgfqpoint{1.340564in}{0.788045in}}%
\pgfpathlineto{\pgfqpoint{1.348296in}{0.789050in}}%
\pgfpathlineto{\pgfqpoint{1.356027in}{0.794737in}}%
\pgfpathlineto{\pgfqpoint{1.358605in}{0.795997in}}%
\pgfpathlineto{\pgfqpoint{1.366336in}{0.796814in}}%
\pgfpathlineto{\pgfqpoint{1.371491in}{0.798634in}}%
\pgfpathlineto{\pgfqpoint{1.376645in}{0.802468in}}%
\pgfpathlineto{\pgfqpoint{1.384377in}{0.804250in}}%
\pgfpathlineto{\pgfqpoint{1.394686in}{0.809064in}}%
\pgfpathlineto{\pgfqpoint{1.402417in}{0.810443in}}%
\pgfpathlineto{\pgfqpoint{1.412726in}{0.816276in}}%
\pgfpathlineto{\pgfqpoint{1.420458in}{0.817519in}}%
\pgfpathlineto{\pgfqpoint{1.428189in}{0.821128in}}%
\pgfpathlineto{\pgfqpoint{1.430767in}{0.822549in}}%
\pgfpathlineto{\pgfqpoint{1.441076in}{0.824765in}}%
\pgfpathlineto{\pgfqpoint{1.448807in}{0.826894in}}%
\pgfpathlineto{\pgfqpoint{1.459116in}{0.828334in}}%
\pgfpathlineto{\pgfqpoint{1.464270in}{0.829773in}}%
\pgfpathlineto{\pgfqpoint{1.466848in}{0.830332in}}%
\pgfpathlineto{\pgfqpoint{1.477156in}{0.830955in}}%
\pgfpathlineto{\pgfqpoint{1.482311in}{0.833869in}}%
\pgfpathlineto{\pgfqpoint{1.484888in}{0.835481in}}%
\pgfpathlineto{\pgfqpoint{1.492620in}{0.836907in}}%
\pgfpathlineto{\pgfqpoint{1.500351in}{0.841670in}}%
\pgfpathlineto{\pgfqpoint{1.502929in}{0.843372in}}%
\pgfpathlineto{\pgfqpoint{1.510660in}{0.844900in}}%
\pgfpathlineto{\pgfqpoint{1.515815in}{0.847991in}}%
\pgfpathlineto{\pgfqpoint{1.520969in}{0.851309in}}%
\pgfpathlineto{\pgfqpoint{1.528701in}{0.853016in}}%
\pgfpathlineto{\pgfqpoint{1.536432in}{0.856910in}}%
\pgfpathlineto{\pgfqpoint{1.539010in}{0.858089in}}%
\pgfpathlineto{\pgfqpoint{1.549318in}{0.859941in}}%
\pgfpathlineto{\pgfqpoint{1.557050in}{0.863657in}}%
\pgfpathlineto{\pgfqpoint{1.587977in}{0.867741in}}%
\pgfpathlineto{\pgfqpoint{1.593131in}{0.869228in}}%
\pgfpathlineto{\pgfqpoint{1.611172in}{0.870185in}}%
\pgfpathlineto{\pgfqpoint{1.647253in}{0.869299in}}%
\pgfpathlineto{\pgfqpoint{1.657561in}{0.868422in}}%
\pgfpathlineto{\pgfqpoint{1.665293in}{0.867030in}}%
\pgfpathlineto{\pgfqpoint{1.737455in}{0.863496in}}%
\pgfpathlineto{\pgfqpoint{1.750341in}{0.863580in}}%
\pgfpathlineto{\pgfqpoint{1.755495in}{0.863936in}}%
\pgfpathlineto{\pgfqpoint{1.788999in}{0.864388in}}%
\pgfpathlineto{\pgfqpoint{1.791576in}{0.865011in}}%
\pgfpathlineto{\pgfqpoint{1.819926in}{0.866262in}}%
\pgfpathlineto{\pgfqpoint{1.840543in}{0.869543in}}%
\pgfpathlineto{\pgfqpoint{1.845698in}{0.871825in}}%
\pgfpathlineto{\pgfqpoint{1.853430in}{0.873000in}}%
\pgfpathlineto{\pgfqpoint{1.863738in}{0.877025in}}%
\pgfpathlineto{\pgfqpoint{1.871470in}{0.877916in}}%
\pgfpathlineto{\pgfqpoint{1.881779in}{0.882066in}}%
\pgfpathlineto{\pgfqpoint{1.889510in}{0.883152in}}%
\pgfpathlineto{\pgfqpoint{1.899819in}{0.887777in}}%
\pgfpathlineto{\pgfqpoint{1.910128in}{0.889107in}}%
\pgfpathlineto{\pgfqpoint{1.917860in}{0.891487in}}%
\pgfpathlineto{\pgfqpoint{1.928169in}{0.892637in}}%
\pgfpathlineto{\pgfqpoint{1.935900in}{0.895225in}}%
\pgfpathlineto{\pgfqpoint{1.943632in}{0.896387in}}%
\pgfpathlineto{\pgfqpoint{1.953941in}{0.902009in}}%
\pgfpathlineto{\pgfqpoint{1.961672in}{0.903732in}}%
\pgfpathlineto{\pgfqpoint{1.971981in}{0.909943in}}%
\pgfpathlineto{\pgfqpoint{1.979713in}{0.911192in}}%
\pgfpathlineto{\pgfqpoint{1.990022in}{0.915629in}}%
\pgfpathlineto{\pgfqpoint{1.997753in}{0.916566in}}%
\pgfpathlineto{\pgfqpoint{2.005485in}{0.919657in}}%
\pgfpathlineto{\pgfqpoint{2.015794in}{0.920667in}}%
\pgfpathlineto{\pgfqpoint{2.026103in}{0.925537in}}%
\pgfpathlineto{\pgfqpoint{2.033834in}{0.927152in}}%
\pgfpathlineto{\pgfqpoint{2.038989in}{0.930894in}}%
\pgfpathlineto{\pgfqpoint{2.044143in}{0.935202in}}%
\pgfpathlineto{\pgfqpoint{2.051875in}{0.936666in}}%
\pgfpathlineto{\pgfqpoint{2.062184in}{0.944933in}}%
\pgfpathlineto{\pgfqpoint{2.069915in}{0.947394in}}%
\pgfpathlineto{\pgfqpoint{2.077647in}{0.954678in}}%
\pgfpathlineto{\pgfqpoint{2.080224in}{0.956883in}}%
\pgfpathlineto{\pgfqpoint{2.087956in}{0.959498in}}%
\pgfpathlineto{\pgfqpoint{2.095687in}{0.967397in}}%
\pgfpathlineto{\pgfqpoint{2.098265in}{0.970545in}}%
\pgfpathlineto{\pgfqpoint{2.105996in}{0.973737in}}%
\pgfpathlineto{\pgfqpoint{2.116305in}{0.988295in}}%
\pgfpathlineto{\pgfqpoint{2.124037in}{0.992106in}}%
\pgfpathlineto{\pgfqpoint{2.131768in}{1.002880in}}%
\pgfpathlineto{\pgfqpoint{2.134346in}{1.006046in}}%
\pgfpathlineto{\pgfqpoint{2.142077in}{1.008961in}}%
\pgfpathlineto{\pgfqpoint{2.152386in}{1.019167in}}%
\pgfpathlineto{\pgfqpoint{2.162695in}{1.022031in}}%
\pgfpathlineto{\pgfqpoint{2.167849in}{1.026705in}}%
\pgfpathlineto{\pgfqpoint{2.170427in}{1.028205in}}%
\pgfpathlineto{\pgfqpoint{2.178158in}{1.029898in}}%
\pgfpathlineto{\pgfqpoint{2.183313in}{1.033164in}}%
\pgfpathlineto{\pgfqpoint{2.188467in}{1.036481in}}%
\pgfpathlineto{\pgfqpoint{2.196199in}{1.038042in}}%
\pgfpathlineto{\pgfqpoint{2.206508in}{1.043912in}}%
\pgfpathlineto{\pgfqpoint{2.214239in}{1.045538in}}%
\pgfpathlineto{\pgfqpoint{2.219394in}{1.048899in}}%
\pgfpathlineto{\pgfqpoint{2.224548in}{1.050837in}}%
\pgfpathlineto{\pgfqpoint{2.232280in}{1.051801in}}%
\pgfpathlineto{\pgfqpoint{2.242589in}{1.056213in}}%
\pgfpathlineto{\pgfqpoint{2.250320in}{1.057468in}}%
\pgfpathlineto{\pgfqpoint{2.255475in}{1.059664in}}%
\pgfpathlineto{\pgfqpoint{2.270938in}{1.063637in}}%
\pgfpathlineto{\pgfqpoint{2.276092in}{1.066864in}}%
\pgfpathlineto{\pgfqpoint{2.278670in}{1.068790in}}%
\pgfpathlineto{\pgfqpoint{2.286401in}{1.070383in}}%
\pgfpathlineto{\pgfqpoint{2.296710in}{1.075886in}}%
\pgfpathlineto{\pgfqpoint{2.304442in}{1.076976in}}%
\pgfpathlineto{\pgfqpoint{2.314751in}{1.081409in}}%
\pgfpathlineto{\pgfqpoint{2.322482in}{1.082465in}}%
\pgfpathlineto{\pgfqpoint{2.330214in}{1.085625in}}%
\pgfpathlineto{\pgfqpoint{2.332791in}{1.087082in}}%
\pgfpathlineto{\pgfqpoint{2.340523in}{1.088392in}}%
\pgfpathlineto{\pgfqpoint{2.345677in}{1.091186in}}%
\pgfpathlineto{\pgfqpoint{2.350832in}{1.093305in}}%
\pgfpathlineto{\pgfqpoint{2.361140in}{1.094805in}}%
\pgfpathlineto{\pgfqpoint{2.366295in}{1.095960in}}%
\pgfpathlineto{\pgfqpoint{2.368872in}{1.096340in}}%
\pgfpathlineto{\pgfqpoint{2.384335in}{1.097430in}}%
\pgfpathlineto{\pgfqpoint{2.386912in}{1.097705in}}%
\pgfpathlineto{\pgfqpoint{2.430725in}{1.099255in}}%
\pgfpathlineto{\pgfqpoint{2.435880in}{1.100248in}}%
\pgfpathlineto{\pgfqpoint{2.441034in}{1.102265in}}%
\pgfpathlineto{\pgfqpoint{2.448766in}{1.103437in}}%
\pgfpathlineto{\pgfqpoint{2.456497in}{1.106642in}}%
\pgfpathlineto{\pgfqpoint{2.459074in}{1.107353in}}%
\pgfpathlineto{\pgfqpoint{2.469383in}{1.108571in}}%
\pgfpathlineto{\pgfqpoint{2.477115in}{1.110566in}}%
\pgfpathlineto{\pgfqpoint{2.487424in}{1.111741in}}%
\pgfpathlineto{\pgfqpoint{2.495155in}{1.113484in}}%
\pgfpathlineto{\pgfqpoint{2.508041in}{1.114808in}}%
\pgfpathlineto{\pgfqpoint{2.513196in}{1.116301in}}%
\pgfpathlineto{\pgfqpoint{2.523505in}{1.118035in}}%
\pgfpathlineto{\pgfqpoint{2.531236in}{1.120554in}}%
\pgfpathlineto{\pgfqpoint{2.538968in}{1.121562in}}%
\pgfpathlineto{\pgfqpoint{2.549277in}{1.126507in}}%
\pgfpathlineto{\pgfqpoint{2.557009in}{1.127731in}}%
\pgfpathlineto{\pgfqpoint{2.567317in}{1.132266in}}%
\pgfpathlineto{\pgfqpoint{2.575049in}{1.133376in}}%
\pgfpathlineto{\pgfqpoint{2.585358in}{1.137358in}}%
\pgfpathlineto{\pgfqpoint{2.595667in}{1.139138in}}%
\pgfpathlineto{\pgfqpoint{2.603398in}{1.142674in}}%
\pgfpathlineto{\pgfqpoint{2.611130in}{1.143787in}}%
\pgfpathlineto{\pgfqpoint{2.621439in}{1.148202in}}%
\pgfpathlineto{\pgfqpoint{2.629170in}{1.149294in}}%
\pgfpathlineto{\pgfqpoint{2.634325in}{1.151941in}}%
\pgfpathlineto{\pgfqpoint{2.652365in}{1.156522in}}%
\pgfpathlineto{\pgfqpoint{2.657520in}{1.158935in}}%
\pgfpathlineto{\pgfqpoint{2.665251in}{1.160171in}}%
\pgfpathlineto{\pgfqpoint{2.672983in}{1.163643in}}%
\pgfpathlineto{\pgfqpoint{2.675560in}{1.164698in}}%
\pgfpathlineto{\pgfqpoint{2.683292in}{1.165916in}}%
\pgfpathlineto{\pgfqpoint{2.688446in}{1.168691in}}%
\pgfpathlineto{\pgfqpoint{2.693601in}{1.171935in}}%
\pgfpathlineto{\pgfqpoint{2.701332in}{1.173658in}}%
\pgfpathlineto{\pgfqpoint{2.703910in}{1.175491in}}%
\pgfpathlineto{\pgfqpoint{2.709064in}{1.177484in}}%
\pgfpathlineto{\pgfqpoint{2.711641in}{1.179387in}}%
\pgfpathlineto{\pgfqpoint{2.719373in}{1.181724in}}%
\pgfpathlineto{\pgfqpoint{2.721950in}{1.184071in}}%
\pgfpathlineto{\pgfqpoint{2.727105in}{1.186355in}}%
\pgfpathlineto{\pgfqpoint{2.729682in}{1.188570in}}%
\pgfpathlineto{\pgfqpoint{2.737413in}{1.190687in}}%
\pgfpathlineto{\pgfqpoint{2.745145in}{1.196649in}}%
\pgfpathlineto{\pgfqpoint{2.747722in}{1.198564in}}%
\pgfpathlineto{\pgfqpoint{2.755454in}{1.199997in}}%
\pgfpathlineto{\pgfqpoint{2.765763in}{1.206344in}}%
\pgfpathlineto{\pgfqpoint{2.776072in}{1.207879in}}%
\pgfpathlineto{\pgfqpoint{2.781226in}{1.211247in}}%
\pgfpathlineto{\pgfqpoint{2.783803in}{1.212435in}}%
\pgfpathlineto{\pgfqpoint{2.791535in}{1.213521in}}%
\pgfpathlineto{\pgfqpoint{2.801844in}{1.217940in}}%
\pgfpathlineto{\pgfqpoint{2.809575in}{1.218589in}}%
\pgfpathlineto{\pgfqpoint{2.814730in}{1.220318in}}%
\pgfpathlineto{\pgfqpoint{2.819884in}{1.223578in}}%
\pgfpathlineto{\pgfqpoint{2.827616in}{1.225486in}}%
\pgfpathlineto{\pgfqpoint{2.837925in}{1.233906in}}%
\pgfpathlineto{\pgfqpoint{2.848234in}{1.236263in}}%
\pgfpathlineto{\pgfqpoint{2.855965in}{1.243039in}}%
\pgfpathlineto{\pgfqpoint{2.863697in}{1.245552in}}%
\pgfpathlineto{\pgfqpoint{2.874006in}{1.254995in}}%
\pgfpathlineto{\pgfqpoint{2.881737in}{1.257077in}}%
\pgfpathlineto{\pgfqpoint{2.892046in}{1.267903in}}%
\pgfpathlineto{\pgfqpoint{2.899778in}{1.270404in}}%
\pgfpathlineto{\pgfqpoint{2.910087in}{1.278919in}}%
\pgfpathlineto{\pgfqpoint{2.917818in}{1.281154in}}%
\pgfpathlineto{\pgfqpoint{2.925550in}{1.287533in}}%
\pgfpathlineto{\pgfqpoint{2.928127in}{1.289451in}}%
\pgfpathlineto{\pgfqpoint{2.935859in}{1.291175in}}%
\pgfpathlineto{\pgfqpoint{2.946168in}{1.297452in}}%
\pgfpathlineto{\pgfqpoint{2.953899in}{1.299199in}}%
\pgfpathlineto{\pgfqpoint{2.964208in}{1.306998in}}%
\pgfpathlineto{\pgfqpoint{2.971940in}{1.308473in}}%
\pgfpathlineto{\pgfqpoint{2.979671in}{1.312851in}}%
\pgfpathlineto{\pgfqpoint{2.982249in}{1.313906in}}%
\pgfpathlineto{\pgfqpoint{2.989980in}{1.315048in}}%
\pgfpathlineto{\pgfqpoint{2.997712in}{1.319041in}}%
\pgfpathlineto{\pgfqpoint{3.008021in}{1.320382in}}%
\pgfpathlineto{\pgfqpoint{3.018330in}{1.325676in}}%
\pgfpathlineto{\pgfqpoint{3.026061in}{1.326732in}}%
\pgfpathlineto{\pgfqpoint{3.036370in}{1.331942in}}%
\pgfpathlineto{\pgfqpoint{3.044102in}{1.333515in}}%
\pgfpathlineto{\pgfqpoint{3.054411in}{1.339646in}}%
\pgfpathlineto{\pgfqpoint{3.062142in}{1.341355in}}%
\pgfpathlineto{\pgfqpoint{3.069874in}{1.345661in}}%
\pgfpathlineto{\pgfqpoint{3.072451in}{1.346955in}}%
\pgfpathlineto{\pgfqpoint{3.080183in}{1.348345in}}%
\pgfpathlineto{\pgfqpoint{3.087914in}{1.352855in}}%
\pgfpathlineto{\pgfqpoint{3.090491in}{1.354582in}}%
\pgfpathlineto{\pgfqpoint{3.100800in}{1.356367in}}%
\pgfpathlineto{\pgfqpoint{3.108532in}{1.361691in}}%
\pgfpathlineto{\pgfqpoint{3.116264in}{1.363501in}}%
\pgfpathlineto{\pgfqpoint{3.126572in}{1.370634in}}%
\pgfpathlineto{\pgfqpoint{3.134304in}{1.372573in}}%
\pgfpathlineto{\pgfqpoint{3.142036in}{1.377468in}}%
\pgfpathlineto{\pgfqpoint{3.144613in}{1.378872in}}%
\pgfpathlineto{\pgfqpoint{3.152345in}{1.380345in}}%
\pgfpathlineto{\pgfqpoint{3.162653in}{1.386116in}}%
\pgfpathlineto{\pgfqpoint{3.170385in}{1.387432in}}%
\pgfpathlineto{\pgfqpoint{3.178117in}{1.391748in}}%
\pgfpathlineto{\pgfqpoint{3.180694in}{1.393426in}}%
\pgfpathlineto{\pgfqpoint{3.188426in}{1.395164in}}%
\pgfpathlineto{\pgfqpoint{3.196157in}{1.400718in}}%
\pgfpathlineto{\pgfqpoint{3.206466in}{1.402535in}}%
\pgfpathlineto{\pgfqpoint{3.216775in}{1.409708in}}%
\pgfpathlineto{\pgfqpoint{3.224507in}{1.411467in}}%
\pgfpathlineto{\pgfqpoint{3.234815in}{1.417564in}}%
\pgfpathlineto{\pgfqpoint{3.242547in}{1.419076in}}%
\pgfpathlineto{\pgfqpoint{3.252856in}{1.425357in}}%
\pgfpathlineto{\pgfqpoint{3.260588in}{1.427026in}}%
\pgfpathlineto{\pgfqpoint{3.270896in}{1.432925in}}%
\pgfpathlineto{\pgfqpoint{3.278628in}{1.434534in}}%
\pgfpathlineto{\pgfqpoint{3.288937in}{1.440320in}}%
\pgfpathlineto{\pgfqpoint{3.296668in}{1.441900in}}%
\pgfpathlineto{\pgfqpoint{3.306977in}{1.448510in}}%
\pgfpathlineto{\pgfqpoint{3.314709in}{1.450418in}}%
\pgfpathlineto{\pgfqpoint{3.325018in}{1.458027in}}%
\pgfpathlineto{\pgfqpoint{3.332749in}{1.459915in}}%
\pgfpathlineto{\pgfqpoint{3.343058in}{1.467127in}}%
\pgfpathlineto{\pgfqpoint{3.353367in}{1.469000in}}%
\pgfpathlineto{\pgfqpoint{3.361099in}{1.474451in}}%
\pgfpathlineto{\pgfqpoint{3.368830in}{1.476217in}}%
\pgfpathlineto{\pgfqpoint{3.379139in}{1.482585in}}%
\pgfpathlineto{\pgfqpoint{3.386871in}{1.484201in}}%
\pgfpathlineto{\pgfqpoint{3.397180in}{1.490693in}}%
\pgfpathlineto{\pgfqpoint{3.404911in}{1.492121in}}%
\pgfpathlineto{\pgfqpoint{3.415220in}{1.497346in}}%
\pgfpathlineto{\pgfqpoint{3.422952in}{1.498649in}}%
\pgfpathlineto{\pgfqpoint{3.430684in}{1.502066in}}%
\pgfpathlineto{\pgfqpoint{3.433261in}{1.503274in}}%
\pgfpathlineto{\pgfqpoint{3.440992in}{1.504487in}}%
\pgfpathlineto{\pgfqpoint{3.451301in}{1.508329in}}%
\pgfpathlineto{\pgfqpoint{3.461610in}{1.509545in}}%
\pgfpathlineto{\pgfqpoint{3.479651in}{1.512609in}}%
\pgfpathlineto{\pgfqpoint{3.487382in}{1.515670in}}%
\pgfpathlineto{\pgfqpoint{3.495114in}{1.516742in}}%
\pgfpathlineto{\pgfqpoint{3.505423in}{1.521896in}}%
\pgfpathlineto{\pgfqpoint{3.513154in}{1.523356in}}%
\pgfpathlineto{\pgfqpoint{3.523463in}{1.528601in}}%
\pgfpathlineto{\pgfqpoint{3.531195in}{1.529744in}}%
\pgfpathlineto{\pgfqpoint{3.541504in}{1.534489in}}%
\pgfpathlineto{\pgfqpoint{3.549235in}{1.535665in}}%
\pgfpathlineto{\pgfqpoint{3.559544in}{1.539815in}}%
\pgfpathlineto{\pgfqpoint{3.567276in}{1.540967in}}%
\pgfpathlineto{\pgfqpoint{3.572430in}{1.543555in}}%
\pgfpathlineto{\pgfqpoint{3.577585in}{1.544930in}}%
\pgfpathlineto{\pgfqpoint{3.585316in}{1.546320in}}%
\pgfpathlineto{\pgfqpoint{3.595625in}{1.552141in}}%
\pgfpathlineto{\pgfqpoint{3.603357in}{1.553622in}}%
\pgfpathlineto{\pgfqpoint{3.613666in}{1.558759in}}%
\pgfpathlineto{\pgfqpoint{3.621397in}{1.559955in}}%
\pgfpathlineto{\pgfqpoint{3.629129in}{1.563681in}}%
\pgfpathlineto{\pgfqpoint{3.631706in}{1.565100in}}%
\pgfpathlineto{\pgfqpoint{3.639438in}{1.566687in}}%
\pgfpathlineto{\pgfqpoint{3.644592in}{1.569925in}}%
\pgfpathlineto{\pgfqpoint{3.649747in}{1.571591in}}%
\pgfpathlineto{\pgfqpoint{3.657478in}{1.573312in}}%
\pgfpathlineto{\pgfqpoint{3.662633in}{1.576414in}}%
\pgfpathlineto{\pgfqpoint{3.667787in}{1.577852in}}%
\pgfpathlineto{\pgfqpoint{3.675519in}{1.579099in}}%
\pgfpathlineto{\pgfqpoint{3.685828in}{1.584412in}}%
\pgfpathlineto{\pgfqpoint{3.693559in}{1.585880in}}%
\pgfpathlineto{\pgfqpoint{3.703868in}{1.591784in}}%
\pgfpathlineto{\pgfqpoint{3.714177in}{1.593231in}}%
\pgfpathlineto{\pgfqpoint{3.721909in}{1.597482in}}%
\pgfpathlineto{\pgfqpoint{3.729640in}{1.598914in}}%
\pgfpathlineto{\pgfqpoint{3.737372in}{1.602508in}}%
\pgfpathlineto{\pgfqpoint{3.739949in}{1.603412in}}%
\pgfpathlineto{\pgfqpoint{3.747681in}{1.604420in}}%
\pgfpathlineto{\pgfqpoint{3.750258in}{1.605683in}}%
\pgfpathlineto{\pgfqpoint{3.757990in}{1.612719in}}%
\pgfpathlineto{\pgfqpoint{3.765721in}{1.614986in}}%
\pgfpathlineto{\pgfqpoint{3.776030in}{1.624593in}}%
\pgfpathlineto{\pgfqpoint{3.786339in}{1.627138in}}%
\pgfpathlineto{\pgfqpoint{3.794070in}{1.634640in}}%
\pgfpathlineto{\pgfqpoint{3.801802in}{1.637240in}}%
\pgfpathlineto{\pgfqpoint{3.812111in}{1.647263in}}%
\pgfpathlineto{\pgfqpoint{3.819843in}{1.649901in}}%
\pgfpathlineto{\pgfqpoint{3.827574in}{1.657574in}}%
\pgfpathlineto{\pgfqpoint{3.830151in}{1.659812in}}%
\pgfpathlineto{\pgfqpoint{3.837883in}{1.662249in}}%
\pgfpathlineto{\pgfqpoint{3.845615in}{1.669073in}}%
\pgfpathlineto{\pgfqpoint{3.848192in}{1.671615in}}%
\pgfpathlineto{\pgfqpoint{3.855924in}{1.674282in}}%
\pgfpathlineto{\pgfqpoint{3.866232in}{1.684929in}}%
\pgfpathlineto{\pgfqpoint{3.873964in}{1.687621in}}%
\pgfpathlineto{\pgfqpoint{3.881696in}{1.694475in}}%
\pgfpathlineto{\pgfqpoint{3.884273in}{1.696679in}}%
\pgfpathlineto{\pgfqpoint{3.892005in}{1.698959in}}%
\pgfpathlineto{\pgfqpoint{3.899736in}{1.705404in}}%
\pgfpathlineto{\pgfqpoint{3.910045in}{1.707545in}}%
\pgfpathlineto{\pgfqpoint{3.920354in}{1.716372in}}%
\pgfpathlineto{\pgfqpoint{3.928086in}{1.718554in}}%
\pgfpathlineto{\pgfqpoint{3.938394in}{1.727455in}}%
\pgfpathlineto{\pgfqpoint{3.946126in}{1.729796in}}%
\pgfpathlineto{\pgfqpoint{3.956435in}{1.739169in}}%
\pgfpathlineto{\pgfqpoint{3.964167in}{1.741702in}}%
\pgfpathlineto{\pgfqpoint{3.974475in}{1.751401in}}%
\pgfpathlineto{\pgfqpoint{3.982207in}{1.753966in}}%
\pgfpathlineto{\pgfqpoint{3.992516in}{1.763430in}}%
\pgfpathlineto{\pgfqpoint{4.000247in}{1.765562in}}%
\pgfpathlineto{\pgfqpoint{4.010556in}{1.774511in}}%
\pgfpathlineto{\pgfqpoint{4.018288in}{1.776799in}}%
\pgfpathlineto{\pgfqpoint{4.028597in}{1.785791in}}%
\pgfpathlineto{\pgfqpoint{4.038906in}{1.787876in}}%
\pgfpathlineto{\pgfqpoint{4.046637in}{1.794431in}}%
\pgfpathlineto{\pgfqpoint{4.054369in}{1.796657in}}%
\pgfpathlineto{\pgfqpoint{4.064678in}{1.805226in}}%
\pgfpathlineto{\pgfqpoint{4.072409in}{1.807156in}}%
\pgfpathlineto{\pgfqpoint{4.082718in}{1.814990in}}%
\pgfpathlineto{\pgfqpoint{4.090450in}{1.816963in}}%
\pgfpathlineto{\pgfqpoint{4.100759in}{1.825709in}}%
\pgfpathlineto{\pgfqpoint{4.108490in}{1.828046in}}%
\pgfpathlineto{\pgfqpoint{4.118799in}{1.837695in}}%
\pgfpathlineto{\pgfqpoint{4.126531in}{1.839885in}}%
\pgfpathlineto{\pgfqpoint{4.134263in}{1.847211in}}%
\pgfpathlineto{\pgfqpoint{4.144571in}{1.849778in}}%
\pgfpathlineto{\pgfqpoint{4.154880in}{1.860076in}}%
\pgfpathlineto{\pgfqpoint{4.162612in}{1.862879in}}%
\pgfpathlineto{\pgfqpoint{4.172921in}{1.874150in}}%
\pgfpathlineto{\pgfqpoint{4.180652in}{1.877066in}}%
\pgfpathlineto{\pgfqpoint{4.190961in}{1.888245in}}%
\pgfpathlineto{\pgfqpoint{4.198693in}{1.890875in}}%
\pgfpathlineto{\pgfqpoint{4.209002in}{1.901933in}}%
\pgfpathlineto{\pgfqpoint{4.216733in}{1.904862in}}%
\pgfpathlineto{\pgfqpoint{4.219311in}{1.907850in}}%
\pgfpathlineto{\pgfqpoint{4.227042in}{1.912136in}}%
\pgfpathlineto{\pgfqpoint{4.234774in}{1.913712in}}%
\pgfpathlineto{\pgfqpoint{4.245083in}{1.918404in}}%
\pgfpathlineto{\pgfqpoint{4.252814in}{1.919724in}}%
\pgfpathlineto{\pgfqpoint{4.257969in}{1.921864in}}%
\pgfpathlineto{\pgfqpoint{4.263123in}{1.922708in}}%
\pgfpathlineto{\pgfqpoint{4.276009in}{1.923383in}}%
\pgfpathlineto{\pgfqpoint{4.281164in}{1.924660in}}%
\pgfpathlineto{\pgfqpoint{4.291472in}{1.925656in}}%
\pgfpathlineto{\pgfqpoint{4.299204in}{1.927339in}}%
\pgfpathlineto{\pgfqpoint{4.312090in}{1.928665in}}%
\pgfpathlineto{\pgfqpoint{4.317245in}{1.930071in}}%
\pgfpathlineto{\pgfqpoint{4.327553in}{1.931469in}}%
\pgfpathlineto{\pgfqpoint{4.335285in}{1.933541in}}%
\pgfpathlineto{\pgfqpoint{4.345594in}{1.934778in}}%
\pgfpathlineto{\pgfqpoint{4.353326in}{1.936099in}}%
\pgfpathlineto{\pgfqpoint{4.363634in}{1.936686in}}%
\pgfpathlineto{\pgfqpoint{4.371366in}{1.938374in}}%
\pgfpathlineto{\pgfqpoint{4.381675in}{1.939673in}}%
\pgfpathlineto{\pgfqpoint{4.389407in}{1.941765in}}%
\pgfpathlineto{\pgfqpoint{4.397138in}{1.942610in}}%
\pgfpathlineto{\pgfqpoint{4.407447in}{1.946209in}}%
\pgfpathlineto{\pgfqpoint{4.415179in}{1.947310in}}%
\pgfpathlineto{\pgfqpoint{4.422910in}{1.951037in}}%
\pgfpathlineto{\pgfqpoint{4.425488in}{1.952465in}}%
\pgfpathlineto{\pgfqpoint{4.433219in}{1.953929in}}%
\pgfpathlineto{\pgfqpoint{4.443528in}{1.959982in}}%
\pgfpathlineto{\pgfqpoint{4.451260in}{1.961558in}}%
\pgfpathlineto{\pgfqpoint{4.461569in}{1.967488in}}%
\pgfpathlineto{\pgfqpoint{4.469300in}{1.969162in}}%
\pgfpathlineto{\pgfqpoint{4.479609in}{1.975684in}}%
\pgfpathlineto{\pgfqpoint{4.487341in}{1.977255in}}%
\pgfpathlineto{\pgfqpoint{4.495072in}{1.982464in}}%
\pgfpathlineto{\pgfqpoint{4.497649in}{1.984442in}}%
\pgfpathlineto{\pgfqpoint{4.505381in}{1.986336in}}%
\pgfpathlineto{\pgfqpoint{4.510536in}{1.989846in}}%
\pgfpathlineto{\pgfqpoint{4.515690in}{1.991255in}}%
\pgfpathlineto{\pgfqpoint{4.523422in}{1.992495in}}%
\pgfpathlineto{\pgfqpoint{4.533730in}{1.997540in}}%
\pgfpathlineto{\pgfqpoint{4.541462in}{1.998776in}}%
\pgfpathlineto{\pgfqpoint{4.551771in}{2.002639in}}%
\pgfpathlineto{\pgfqpoint{4.559503in}{2.003502in}}%
\pgfpathlineto{\pgfqpoint{4.569811in}{2.007649in}}%
\pgfpathlineto{\pgfqpoint{4.580120in}{2.008880in}}%
\pgfpathlineto{\pgfqpoint{4.585275in}{2.009949in}}%
\pgfpathlineto{\pgfqpoint{4.598161in}{2.011226in}}%
\pgfpathlineto{\pgfqpoint{4.603315in}{2.012265in}}%
\pgfpathlineto{\pgfqpoint{4.641973in}{2.013265in}}%
\pgfpathlineto{\pgfqpoint{4.670323in}{2.012251in}}%
\pgfpathlineto{\pgfqpoint{4.696095in}{2.010746in}}%
\pgfpathlineto{\pgfqpoint{4.708981in}{2.009740in}}%
\pgfpathlineto{\pgfqpoint{4.714135in}{2.009036in}}%
\pgfpathlineto{\pgfqpoint{4.747639in}{2.007713in}}%
\pgfpathlineto{\pgfqpoint{4.763102in}{2.007265in}}%
\pgfpathlineto{\pgfqpoint{4.814647in}{2.007001in}}%
\pgfpathlineto{\pgfqpoint{4.832687in}{2.006718in}}%
\pgfpathlineto{\pgfqpoint{4.858459in}{2.006238in}}%
\pgfpathlineto{\pgfqpoint{4.886809in}{2.006260in}}%
\pgfpathlineto{\pgfqpoint{4.930621in}{2.009726in}}%
\pgfpathlineto{\pgfqpoint{4.964125in}{2.010628in}}%
\pgfpathlineto{\pgfqpoint{5.018246in}{2.009907in}}%
\pgfpathlineto{\pgfqpoint{5.038864in}{2.009294in}}%
\pgfpathlineto{\pgfqpoint{5.056905in}{2.008823in}}%
\pgfpathlineto{\pgfqpoint{5.105872in}{2.007663in}}%
\pgfpathlineto{\pgfqpoint{5.136798in}{2.007226in}}%
\pgfpathlineto{\pgfqpoint{5.183188in}{2.004748in}}%
\pgfpathlineto{\pgfqpoint{5.208960in}{2.003753in}}%
\pgfpathlineto{\pgfqpoint{5.237309in}{2.001674in}}%
\pgfpathlineto{\pgfqpoint{5.252773in}{2.000848in}}%
\pgfpathlineto{\pgfqpoint{5.255350in}{2.000525in}}%
\pgfpathlineto{\pgfqpoint{5.268236in}{1.999598in}}%
\pgfpathlineto{\pgfqpoint{5.273390in}{1.998958in}}%
\pgfpathlineto{\pgfqpoint{5.286276in}{1.998010in}}%
\pgfpathlineto{\pgfqpoint{5.309471in}{1.995758in}}%
\pgfpathlineto{\pgfqpoint{5.322357in}{1.994797in}}%
\pgfpathlineto{\pgfqpoint{5.327512in}{1.994167in}}%
\pgfpathlineto{\pgfqpoint{5.340398in}{1.993153in}}%
\pgfpathlineto{\pgfqpoint{5.345552in}{1.992437in}}%
\pgfpathlineto{\pgfqpoint{5.358438in}{1.991372in}}%
\pgfpathlineto{\pgfqpoint{5.363593in}{1.990735in}}%
\pgfpathlineto{\pgfqpoint{5.376479in}{1.989844in}}%
\pgfpathlineto{\pgfqpoint{5.399674in}{1.987711in}}%
\pgfpathlineto{\pgfqpoint{5.487299in}{1.985234in}}%
\pgfpathlineto{\pgfqpoint{5.500185in}{1.985812in}}%
\pgfpathlineto{\pgfqpoint{5.543998in}{1.988333in}}%
\pgfpathlineto{\pgfqpoint{5.559461in}{1.989290in}}%
\pgfpathlineto{\pgfqpoint{5.562038in}{1.989724in}}%
\pgfpathlineto{\pgfqpoint{5.572347in}{1.990511in}}%
\pgfpathlineto{\pgfqpoint{5.580079in}{1.991673in}}%
\pgfpathlineto{\pgfqpoint{5.595542in}{1.992718in}}%
\pgfpathlineto{\pgfqpoint{5.598119in}{1.993055in}}%
\pgfpathlineto{\pgfqpoint{5.611005in}{1.994056in}}%
\pgfpathlineto{\pgfqpoint{5.652241in}{1.999365in}}%
\pgfpathlineto{\pgfqpoint{5.662550in}{2.000239in}}%
\pgfpathlineto{\pgfqpoint{5.670281in}{2.001593in}}%
\pgfpathlineto{\pgfqpoint{5.683167in}{2.002453in}}%
\pgfpathlineto{\pgfqpoint{5.688322in}{2.003300in}}%
\pgfpathlineto{\pgfqpoint{5.698630in}{2.004153in}}%
\pgfpathlineto{\pgfqpoint{5.706362in}{2.005560in}}%
\pgfpathlineto{\pgfqpoint{5.716671in}{2.006462in}}%
\pgfpathlineto{\pgfqpoint{5.724403in}{2.007829in}}%
\pgfpathlineto{\pgfqpoint{5.734711in}{2.008847in}}%
\pgfpathlineto{\pgfqpoint{5.742443in}{2.010331in}}%
\pgfpathlineto{\pgfqpoint{5.752752in}{2.011364in}}%
\pgfpathlineto{\pgfqpoint{5.760484in}{2.012879in}}%
\pgfpathlineto{\pgfqpoint{5.770792in}{2.013946in}}%
\pgfpathlineto{\pgfqpoint{5.778524in}{2.015586in}}%
\pgfpathlineto{\pgfqpoint{5.788833in}{2.016677in}}%
\pgfpathlineto{\pgfqpoint{5.796565in}{2.018247in}}%
\pgfpathlineto{\pgfqpoint{5.806873in}{2.019267in}}%
\pgfpathlineto{\pgfqpoint{5.812028in}{2.020321in}}%
\pgfpathlineto{\pgfqpoint{5.824914in}{2.021472in}}%
\pgfpathlineto{\pgfqpoint{5.832646in}{2.023241in}}%
\pgfpathlineto{\pgfqpoint{5.842954in}{2.024422in}}%
\pgfpathlineto{\pgfqpoint{5.850686in}{2.026406in}}%
\pgfpathlineto{\pgfqpoint{5.860995in}{2.027572in}}%
\pgfpathlineto{\pgfqpoint{5.868727in}{2.028773in}}%
\pgfpathlineto{\pgfqpoint{5.881613in}{2.029866in}}%
\pgfpathlineto{\pgfqpoint{5.886767in}{2.030430in}}%
\pgfpathlineto{\pgfqpoint{5.935734in}{2.032496in}}%
\pgfpathlineto{\pgfqpoint{5.948620in}{2.032932in}}%
\pgfpathlineto{\pgfqpoint{6.020782in}{2.034081in}}%
\pgfpathlineto{\pgfqpoint{6.080058in}{2.035079in}}%
\pgfpathlineto{\pgfqpoint{6.085212in}{2.035627in}}%
\pgfpathlineto{\pgfqpoint{6.129025in}{2.036618in}}%
\pgfpathlineto{\pgfqpoint{6.167683in}{2.035542in}}%
\pgfpathlineto{\pgfqpoint{6.201187in}{2.034251in}}%
\pgfpathlineto{\pgfqpoint{6.221805in}{2.032928in}}%
\pgfpathlineto{\pgfqpoint{6.247577in}{2.031294in}}%
\pgfpathlineto{\pgfqpoint{6.275926in}{2.030126in}}%
\pgfpathlineto{\pgfqpoint{6.301698in}{2.028339in}}%
\pgfpathlineto{\pgfqpoint{6.317161in}{2.027452in}}%
\pgfpathlineto{\pgfqpoint{6.337779in}{2.026123in}}%
\pgfpathlineto{\pgfqpoint{6.409941in}{2.024894in}}%
\pgfpathlineto{\pgfqpoint{6.456331in}{2.026731in}}%
\pgfpathlineto{\pgfqpoint{6.464063in}{2.027334in}}%
\pgfpathlineto{\pgfqpoint{6.482103in}{2.027842in}}%
\pgfpathlineto{\pgfqpoint{6.482103in}{2.027842in}}%
\pgfusepath{stroke}%
\end{pgfscope}%
\begin{pgfscope}%
\pgfsetrectcap%
\pgfsetmiterjoin%
\pgfsetlinewidth{0.803000pt}%
\definecolor{currentstroke}{rgb}{1.000000,1.000000,1.000000}%
\pgfsetstrokecolor{currentstroke}%
\pgfsetdash{}{0pt}%
\pgfpathmoveto{\pgfqpoint{0.563921in}{0.521603in}}%
\pgfpathlineto{\pgfqpoint{0.563921in}{3.164103in}}%
\pgfusepath{stroke}%
\end{pgfscope}%
\begin{pgfscope}%
\pgfsetrectcap%
\pgfsetmiterjoin%
\pgfsetlinewidth{0.803000pt}%
\definecolor{currentstroke}{rgb}{1.000000,1.000000,1.000000}%
\pgfsetstrokecolor{currentstroke}%
\pgfsetdash{}{0pt}%
\pgfpathmoveto{\pgfqpoint{6.763921in}{0.521603in}}%
\pgfpathlineto{\pgfqpoint{6.763921in}{3.164103in}}%
\pgfusepath{stroke}%
\end{pgfscope}%
\begin{pgfscope}%
\pgfsetrectcap%
\pgfsetmiterjoin%
\pgfsetlinewidth{0.803000pt}%
\definecolor{currentstroke}{rgb}{1.000000,1.000000,1.000000}%
\pgfsetstrokecolor{currentstroke}%
\pgfsetdash{}{0pt}%
\pgfpathmoveto{\pgfqpoint{0.563921in}{0.521603in}}%
\pgfpathlineto{\pgfqpoint{6.763921in}{0.521603in}}%
\pgfusepath{stroke}%
\end{pgfscope}%
\begin{pgfscope}%
\pgfsetrectcap%
\pgfsetmiterjoin%
\pgfsetlinewidth{0.803000pt}%
\definecolor{currentstroke}{rgb}{1.000000,1.000000,1.000000}%
\pgfsetstrokecolor{currentstroke}%
\pgfsetdash{}{0pt}%
\pgfpathmoveto{\pgfqpoint{0.563921in}{3.164103in}}%
\pgfpathlineto{\pgfqpoint{6.763921in}{3.164103in}}%
\pgfusepath{stroke}%
\end{pgfscope}%
\begin{pgfscope}%
\definecolor{textcolor}{rgb}{0.150000,0.150000,0.150000}%
\pgfsetstrokecolor{textcolor}%
\pgfsetfillcolor{textcolor}%
\pgftext[x=3.663921in,y=3.247437in,,base]{\color{textcolor}\rmfamily\fontsize{12.000000}{14.400000}\selectfont 'Cumulative' Standard Deviation of Stock Prices}%
\end{pgfscope}%
\begin{pgfscope}%
\pgfsetbuttcap%
\pgfsetmiterjoin%
\definecolor{currentfill}{rgb}{0.917647,0.917647,0.949020}%
\pgfsetfillcolor{currentfill}%
\pgfsetfillopacity{0.800000}%
\pgfsetlinewidth{1.003750pt}%
\definecolor{currentstroke}{rgb}{0.800000,0.800000,0.800000}%
\pgfsetstrokecolor{currentstroke}%
\pgfsetstrokeopacity{0.800000}%
\pgfsetdash{}{0pt}%
\pgfpathmoveto{\pgfqpoint{0.661143in}{1.014420in}}%
\pgfpathlineto{\pgfqpoint{1.532224in}{1.014420in}}%
\pgfpathquadraticcurveto{\pgfqpoint{1.560001in}{1.014420in}}{\pgfqpoint{1.560001in}{1.042198in}}%
\pgfpathlineto{\pgfqpoint{1.560001in}{3.066881in}}%
\pgfpathquadraticcurveto{\pgfqpoint{1.560001in}{3.094659in}}{\pgfqpoint{1.532224in}{3.094659in}}%
\pgfpathlineto{\pgfqpoint{0.661143in}{3.094659in}}%
\pgfpathquadraticcurveto{\pgfqpoint{0.633366in}{3.094659in}}{\pgfqpoint{0.633366in}{3.066881in}}%
\pgfpathlineto{\pgfqpoint{0.633366in}{1.042198in}}%
\pgfpathquadraticcurveto{\pgfqpoint{0.633366in}{1.014420in}}{\pgfqpoint{0.661143in}{1.014420in}}%
\pgfpathclose%
\pgfusepath{stroke,fill}%
\end{pgfscope}%
\begin{pgfscope}%
\pgfsetroundcap%
\pgfsetroundjoin%
\pgfsetlinewidth{1.505625pt}%
\definecolor{currentstroke}{rgb}{0.121569,0.466667,0.705882}%
\pgfsetstrokecolor{currentstroke}%
\pgfsetdash{}{0pt}%
\pgfpathmoveto{\pgfqpoint{0.688921in}{2.982191in}}%
\pgfpathlineto{\pgfqpoint{0.966699in}{2.982191in}}%
\pgfusepath{stroke}%
\end{pgfscope}%
\begin{pgfscope}%
\definecolor{textcolor}{rgb}{0.150000,0.150000,0.150000}%
\pgfsetstrokecolor{textcolor}%
\pgfsetfillcolor{textcolor}%
\pgftext[x=1.077810in,y=2.933580in,left,base]{\color{textcolor}\rmfamily\fontsize{10.000000}{12.000000}\selectfont MMM}%
\end{pgfscope}%
\begin{pgfscope}%
\pgfsetroundcap%
\pgfsetroundjoin%
\pgfsetlinewidth{1.505625pt}%
\definecolor{currentstroke}{rgb}{1.000000,0.498039,0.054902}%
\pgfsetstrokecolor{currentstroke}%
\pgfsetdash{}{0pt}%
\pgfpathmoveto{\pgfqpoint{0.688921in}{2.778334in}}%
\pgfpathlineto{\pgfqpoint{0.966699in}{2.778334in}}%
\pgfusepath{stroke}%
\end{pgfscope}%
\begin{pgfscope}%
\definecolor{textcolor}{rgb}{0.150000,0.150000,0.150000}%
\pgfsetstrokecolor{textcolor}%
\pgfsetfillcolor{textcolor}%
\pgftext[x=1.077810in,y=2.729723in,left,base]{\color{textcolor}\rmfamily\fontsize{10.000000}{12.000000}\selectfont AXP}%
\end{pgfscope}%
\begin{pgfscope}%
\pgfsetroundcap%
\pgfsetroundjoin%
\pgfsetlinewidth{1.505625pt}%
\definecolor{currentstroke}{rgb}{0.172549,0.627451,0.172549}%
\pgfsetstrokecolor{currentstroke}%
\pgfsetdash{}{0pt}%
\pgfpathmoveto{\pgfqpoint{0.688921in}{2.574477in}}%
\pgfpathlineto{\pgfqpoint{0.966699in}{2.574477in}}%
\pgfusepath{stroke}%
\end{pgfscope}%
\begin{pgfscope}%
\definecolor{textcolor}{rgb}{0.150000,0.150000,0.150000}%
\pgfsetstrokecolor{textcolor}%
\pgfsetfillcolor{textcolor}%
\pgftext[x=1.077810in,y=2.525866in,left,base]{\color{textcolor}\rmfamily\fontsize{10.000000}{12.000000}\selectfont GE}%
\end{pgfscope}%
\begin{pgfscope}%
\pgfsetroundcap%
\pgfsetroundjoin%
\pgfsetlinewidth{1.505625pt}%
\definecolor{currentstroke}{rgb}{0.839216,0.152941,0.156863}%
\pgfsetstrokecolor{currentstroke}%
\pgfsetdash{}{0pt}%
\pgfpathmoveto{\pgfqpoint{0.688921in}{2.370620in}}%
\pgfpathlineto{\pgfqpoint{0.966699in}{2.370620in}}%
\pgfusepath{stroke}%
\end{pgfscope}%
\begin{pgfscope}%
\definecolor{textcolor}{rgb}{0.150000,0.150000,0.150000}%
\pgfsetstrokecolor{textcolor}%
\pgfsetfillcolor{textcolor}%
\pgftext[x=1.077810in,y=2.322009in,left,base]{\color{textcolor}\rmfamily\fontsize{10.000000}{12.000000}\selectfont INTC}%
\end{pgfscope}%
\begin{pgfscope}%
\pgfsetroundcap%
\pgfsetroundjoin%
\pgfsetlinewidth{1.505625pt}%
\definecolor{currentstroke}{rgb}{0.580392,0.403922,0.741176}%
\pgfsetstrokecolor{currentstroke}%
\pgfsetdash{}{0pt}%
\pgfpathmoveto{\pgfqpoint{0.688921in}{2.166762in}}%
\pgfpathlineto{\pgfqpoint{0.966699in}{2.166762in}}%
\pgfusepath{stroke}%
\end{pgfscope}%
\begin{pgfscope}%
\definecolor{textcolor}{rgb}{0.150000,0.150000,0.150000}%
\pgfsetstrokecolor{textcolor}%
\pgfsetfillcolor{textcolor}%
\pgftext[x=1.077810in,y=2.118151in,left,base]{\color{textcolor}\rmfamily\fontsize{10.000000}{12.000000}\selectfont JNJ}%
\end{pgfscope}%
\begin{pgfscope}%
\pgfsetroundcap%
\pgfsetroundjoin%
\pgfsetlinewidth{1.505625pt}%
\definecolor{currentstroke}{rgb}{0.549020,0.337255,0.294118}%
\pgfsetstrokecolor{currentstroke}%
\pgfsetdash{}{0pt}%
\pgfpathmoveto{\pgfqpoint{0.688921in}{1.962905in}}%
\pgfpathlineto{\pgfqpoint{0.966699in}{1.962905in}}%
\pgfusepath{stroke}%
\end{pgfscope}%
\begin{pgfscope}%
\definecolor{textcolor}{rgb}{0.150000,0.150000,0.150000}%
\pgfsetstrokecolor{textcolor}%
\pgfsetfillcolor{textcolor}%
\pgftext[x=1.077810in,y=1.914294in,left,base]{\color{textcolor}\rmfamily\fontsize{10.000000}{12.000000}\selectfont PG}%
\end{pgfscope}%
\begin{pgfscope}%
\pgfsetroundcap%
\pgfsetroundjoin%
\pgfsetlinewidth{1.505625pt}%
\definecolor{currentstroke}{rgb}{0.890196,0.466667,0.760784}%
\pgfsetstrokecolor{currentstroke}%
\pgfsetdash{}{0pt}%
\pgfpathmoveto{\pgfqpoint{0.688921in}{1.759048in}}%
\pgfpathlineto{\pgfqpoint{0.966699in}{1.759048in}}%
\pgfusepath{stroke}%
\end{pgfscope}%
\begin{pgfscope}%
\definecolor{textcolor}{rgb}{0.150000,0.150000,0.150000}%
\pgfsetstrokecolor{textcolor}%
\pgfsetfillcolor{textcolor}%
\pgftext[x=1.077810in,y=1.710437in,left,base]{\color{textcolor}\rmfamily\fontsize{10.000000}{12.000000}\selectfont UTX}%
\end{pgfscope}%
\begin{pgfscope}%
\pgfsetroundcap%
\pgfsetroundjoin%
\pgfsetlinewidth{1.505625pt}%
\definecolor{currentstroke}{rgb}{0.498039,0.498039,0.498039}%
\pgfsetstrokecolor{currentstroke}%
\pgfsetdash{}{0pt}%
\pgfpathmoveto{\pgfqpoint{0.688921in}{1.555191in}}%
\pgfpathlineto{\pgfqpoint{0.966699in}{1.555191in}}%
\pgfusepath{stroke}%
\end{pgfscope}%
\begin{pgfscope}%
\definecolor{textcolor}{rgb}{0.150000,0.150000,0.150000}%
\pgfsetstrokecolor{textcolor}%
\pgfsetfillcolor{textcolor}%
\pgftext[x=1.077810in,y=1.506580in,left,base]{\color{textcolor}\rmfamily\fontsize{10.000000}{12.000000}\selectfont VZ}%
\end{pgfscope}%
\begin{pgfscope}%
\pgfsetroundcap%
\pgfsetroundjoin%
\pgfsetlinewidth{1.505625pt}%
\definecolor{currentstroke}{rgb}{0.737255,0.741176,0.133333}%
\pgfsetstrokecolor{currentstroke}%
\pgfsetdash{}{0pt}%
\pgfpathmoveto{\pgfqpoint{0.688921in}{1.351334in}}%
\pgfpathlineto{\pgfqpoint{0.966699in}{1.351334in}}%
\pgfusepath{stroke}%
\end{pgfscope}%
\begin{pgfscope}%
\definecolor{textcolor}{rgb}{0.150000,0.150000,0.150000}%
\pgfsetstrokecolor{textcolor}%
\pgfsetfillcolor{textcolor}%
\pgftext[x=1.077810in,y=1.302722in,left,base]{\color{textcolor}\rmfamily\fontsize{10.000000}{12.000000}\selectfont V}%
\end{pgfscope}%
\begin{pgfscope}%
\pgfsetroundcap%
\pgfsetroundjoin%
\pgfsetlinewidth{1.505625pt}%
\definecolor{currentstroke}{rgb}{0.090196,0.745098,0.811765}%
\pgfsetstrokecolor{currentstroke}%
\pgfsetdash{}{0pt}%
\pgfpathmoveto{\pgfqpoint{0.688921in}{1.147476in}}%
\pgfpathlineto{\pgfqpoint{0.966699in}{1.147476in}}%
\pgfusepath{stroke}%
\end{pgfscope}%
\begin{pgfscope}%
\definecolor{textcolor}{rgb}{0.150000,0.150000,0.150000}%
\pgfsetstrokecolor{textcolor}%
\pgfsetfillcolor{textcolor}%
\pgftext[x=1.077810in,y=1.098865in,left,base]{\color{textcolor}\rmfamily\fontsize{10.000000}{12.000000}\selectfont DIS}%
\end{pgfscope}%
\end{pgfpicture}%
\makeatother%
\endgroup%

    \end{adjustbox}  
    \caption{Standard deviation for the time series of stock prices. The value of the graph at point t is calculated as the standard deviation of all recorded values of the respective stocks up to that point t.}
    \label{fig:cum_sd_all}
\end{figure}{}

We take the log of the data in order to stabilize the variance and convert the exponential trend to a linear trend. 

ALTERNATIVE: calculate returns and plot variance of returns. 

We look at autocorrelation and partial autocorrelation. 
\begin{figure}[h]
    \centering
    \begin{adjustbox}{width=.9\textwidth,center}
    %% Creator: Matplotlib, PGF backend
%%
%% To include the figure in your LaTeX document, write
%%   \input{<filename>.pgf}
%%
%% Make sure the required packages are loaded in your preamble
%%   \usepackage{pgf}
%%
%% Figures using additional raster images can only be included by \input if
%% they are in the same directory as the main LaTeX file. For loading figures
%% from other directories you can use the `import` package
%%   \usepackage{import}
%% and then include the figures with
%%   \import{<path to file>}{<filename>.pgf}
%%
%% Matplotlib used the following preamble
%%   \usepackage{fontspec}
%%   \setmainfont{DejaVuSerif.ttf}[Path=/opt/tljh/user/lib/python3.6/site-packages/matplotlib/mpl-data/fonts/ttf/]
%%   \setsansfont{DejaVuSans.ttf}[Path=/opt/tljh/user/lib/python3.6/site-packages/matplotlib/mpl-data/fonts/ttf/]
%%   \setmonofont{DejaVuSansMono.ttf}[Path=/opt/tljh/user/lib/python3.6/site-packages/matplotlib/mpl-data/fonts/ttf/]
%%
\begingroup%
\makeatletter%
\begin{pgfpicture}%
\pgfpathrectangle{\pgfpointorigin}{\pgfqpoint{17.000000in}{20.000000in}}%
\pgfusepath{use as bounding box, clip}%
\begin{pgfscope}%
\pgfsetbuttcap%
\pgfsetmiterjoin%
\definecolor{currentfill}{rgb}{1.000000,1.000000,1.000000}%
\pgfsetfillcolor{currentfill}%
\pgfsetlinewidth{0.000000pt}%
\definecolor{currentstroke}{rgb}{1.000000,1.000000,1.000000}%
\pgfsetstrokecolor{currentstroke}%
\pgfsetdash{}{0pt}%
\pgfpathmoveto{\pgfqpoint{0.000000in}{0.000000in}}%
\pgfpathlineto{\pgfqpoint{17.000000in}{0.000000in}}%
\pgfpathlineto{\pgfqpoint{17.000000in}{20.000000in}}%
\pgfpathlineto{\pgfqpoint{0.000000in}{20.000000in}}%
\pgfpathclose%
\pgfusepath{fill}%
\end{pgfscope}%
\begin{pgfscope}%
\pgfsetbuttcap%
\pgfsetmiterjoin%
\definecolor{currentfill}{rgb}{0.917647,0.917647,0.949020}%
\pgfsetfillcolor{currentfill}%
\pgfsetlinewidth{0.000000pt}%
\definecolor{currentstroke}{rgb}{0.000000,0.000000,0.000000}%
\pgfsetstrokecolor{currentstroke}%
\pgfsetstrokeopacity{0.000000}%
\pgfsetdash{}{0pt}%
\pgfpathmoveto{\pgfqpoint{2.125000in}{16.722093in}}%
\pgfpathlineto{\pgfqpoint{7.614583in}{16.722093in}}%
\pgfpathlineto{\pgfqpoint{7.614583in}{17.600000in}}%
\pgfpathlineto{\pgfqpoint{2.125000in}{17.600000in}}%
\pgfpathclose%
\pgfusepath{fill}%
\end{pgfscope}%
\begin{pgfscope}%
\pgfpathrectangle{\pgfqpoint{2.125000in}{16.722093in}}{\pgfqpoint{5.489583in}{0.877907in}}%
\pgfusepath{clip}%
\pgfsetroundcap%
\pgfsetroundjoin%
\pgfsetlinewidth{0.803000pt}%
\definecolor{currentstroke}{rgb}{1.000000,1.000000,1.000000}%
\pgfsetstrokecolor{currentstroke}%
\pgfsetdash{}{0pt}%
\pgfpathmoveto{\pgfqpoint{2.374527in}{16.722093in}}%
\pgfpathlineto{\pgfqpoint{2.374527in}{17.600000in}}%
\pgfusepath{stroke}%
\end{pgfscope}%
\begin{pgfscope}%
\definecolor{textcolor}{rgb}{0.150000,0.150000,0.150000}%
\pgfsetstrokecolor{textcolor}%
\pgfsetfillcolor{textcolor}%
\pgftext[x=2.374527in,y=16.624871in,,top]{\color{textcolor}\rmfamily\fontsize{14.000000}{16.800000}\selectfont 0}%
\end{pgfscope}%
\begin{pgfscope}%
\pgfpathrectangle{\pgfqpoint{2.125000in}{16.722093in}}{\pgfqpoint{5.489583in}{0.877907in}}%
\pgfusepath{clip}%
\pgfsetroundcap%
\pgfsetroundjoin%
\pgfsetlinewidth{0.803000pt}%
\definecolor{currentstroke}{rgb}{1.000000,1.000000,1.000000}%
\pgfsetstrokecolor{currentstroke}%
\pgfsetdash{}{0pt}%
\pgfpathmoveto{\pgfqpoint{2.990641in}{16.722093in}}%
\pgfpathlineto{\pgfqpoint{2.990641in}{17.600000in}}%
\pgfusepath{stroke}%
\end{pgfscope}%
\begin{pgfscope}%
\definecolor{textcolor}{rgb}{0.150000,0.150000,0.150000}%
\pgfsetstrokecolor{textcolor}%
\pgfsetfillcolor{textcolor}%
\pgftext[x=2.990641in,y=16.624871in,,top]{\color{textcolor}\rmfamily\fontsize{14.000000}{16.800000}\selectfont 5}%
\end{pgfscope}%
\begin{pgfscope}%
\pgfpathrectangle{\pgfqpoint{2.125000in}{16.722093in}}{\pgfqpoint{5.489583in}{0.877907in}}%
\pgfusepath{clip}%
\pgfsetroundcap%
\pgfsetroundjoin%
\pgfsetlinewidth{0.803000pt}%
\definecolor{currentstroke}{rgb}{1.000000,1.000000,1.000000}%
\pgfsetstrokecolor{currentstroke}%
\pgfsetdash{}{0pt}%
\pgfpathmoveto{\pgfqpoint{3.606756in}{16.722093in}}%
\pgfpathlineto{\pgfqpoint{3.606756in}{17.600000in}}%
\pgfusepath{stroke}%
\end{pgfscope}%
\begin{pgfscope}%
\definecolor{textcolor}{rgb}{0.150000,0.150000,0.150000}%
\pgfsetstrokecolor{textcolor}%
\pgfsetfillcolor{textcolor}%
\pgftext[x=3.606756in,y=16.624871in,,top]{\color{textcolor}\rmfamily\fontsize{14.000000}{16.800000}\selectfont 10}%
\end{pgfscope}%
\begin{pgfscope}%
\pgfpathrectangle{\pgfqpoint{2.125000in}{16.722093in}}{\pgfqpoint{5.489583in}{0.877907in}}%
\pgfusepath{clip}%
\pgfsetroundcap%
\pgfsetroundjoin%
\pgfsetlinewidth{0.803000pt}%
\definecolor{currentstroke}{rgb}{1.000000,1.000000,1.000000}%
\pgfsetstrokecolor{currentstroke}%
\pgfsetdash{}{0pt}%
\pgfpathmoveto{\pgfqpoint{4.222871in}{16.722093in}}%
\pgfpathlineto{\pgfqpoint{4.222871in}{17.600000in}}%
\pgfusepath{stroke}%
\end{pgfscope}%
\begin{pgfscope}%
\definecolor{textcolor}{rgb}{0.150000,0.150000,0.150000}%
\pgfsetstrokecolor{textcolor}%
\pgfsetfillcolor{textcolor}%
\pgftext[x=4.222871in,y=16.624871in,,top]{\color{textcolor}\rmfamily\fontsize{14.000000}{16.800000}\selectfont 15}%
\end{pgfscope}%
\begin{pgfscope}%
\pgfpathrectangle{\pgfqpoint{2.125000in}{16.722093in}}{\pgfqpoint{5.489583in}{0.877907in}}%
\pgfusepath{clip}%
\pgfsetroundcap%
\pgfsetroundjoin%
\pgfsetlinewidth{0.803000pt}%
\definecolor{currentstroke}{rgb}{1.000000,1.000000,1.000000}%
\pgfsetstrokecolor{currentstroke}%
\pgfsetdash{}{0pt}%
\pgfpathmoveto{\pgfqpoint{4.838986in}{16.722093in}}%
\pgfpathlineto{\pgfqpoint{4.838986in}{17.600000in}}%
\pgfusepath{stroke}%
\end{pgfscope}%
\begin{pgfscope}%
\definecolor{textcolor}{rgb}{0.150000,0.150000,0.150000}%
\pgfsetstrokecolor{textcolor}%
\pgfsetfillcolor{textcolor}%
\pgftext[x=4.838986in,y=16.624871in,,top]{\color{textcolor}\rmfamily\fontsize{14.000000}{16.800000}\selectfont 20}%
\end{pgfscope}%
\begin{pgfscope}%
\pgfpathrectangle{\pgfqpoint{2.125000in}{16.722093in}}{\pgfqpoint{5.489583in}{0.877907in}}%
\pgfusepath{clip}%
\pgfsetroundcap%
\pgfsetroundjoin%
\pgfsetlinewidth{0.803000pt}%
\definecolor{currentstroke}{rgb}{1.000000,1.000000,1.000000}%
\pgfsetstrokecolor{currentstroke}%
\pgfsetdash{}{0pt}%
\pgfpathmoveto{\pgfqpoint{5.455101in}{16.722093in}}%
\pgfpathlineto{\pgfqpoint{5.455101in}{17.600000in}}%
\pgfusepath{stroke}%
\end{pgfscope}%
\begin{pgfscope}%
\definecolor{textcolor}{rgb}{0.150000,0.150000,0.150000}%
\pgfsetstrokecolor{textcolor}%
\pgfsetfillcolor{textcolor}%
\pgftext[x=5.455101in,y=16.624871in,,top]{\color{textcolor}\rmfamily\fontsize{14.000000}{16.800000}\selectfont 25}%
\end{pgfscope}%
\begin{pgfscope}%
\pgfpathrectangle{\pgfqpoint{2.125000in}{16.722093in}}{\pgfqpoint{5.489583in}{0.877907in}}%
\pgfusepath{clip}%
\pgfsetroundcap%
\pgfsetroundjoin%
\pgfsetlinewidth{0.803000pt}%
\definecolor{currentstroke}{rgb}{1.000000,1.000000,1.000000}%
\pgfsetstrokecolor{currentstroke}%
\pgfsetdash{}{0pt}%
\pgfpathmoveto{\pgfqpoint{6.071216in}{16.722093in}}%
\pgfpathlineto{\pgfqpoint{6.071216in}{17.600000in}}%
\pgfusepath{stroke}%
\end{pgfscope}%
\begin{pgfscope}%
\definecolor{textcolor}{rgb}{0.150000,0.150000,0.150000}%
\pgfsetstrokecolor{textcolor}%
\pgfsetfillcolor{textcolor}%
\pgftext[x=6.071216in,y=16.624871in,,top]{\color{textcolor}\rmfamily\fontsize{14.000000}{16.800000}\selectfont 30}%
\end{pgfscope}%
\begin{pgfscope}%
\pgfpathrectangle{\pgfqpoint{2.125000in}{16.722093in}}{\pgfqpoint{5.489583in}{0.877907in}}%
\pgfusepath{clip}%
\pgfsetroundcap%
\pgfsetroundjoin%
\pgfsetlinewidth{0.803000pt}%
\definecolor{currentstroke}{rgb}{1.000000,1.000000,1.000000}%
\pgfsetstrokecolor{currentstroke}%
\pgfsetdash{}{0pt}%
\pgfpathmoveto{\pgfqpoint{6.687330in}{16.722093in}}%
\pgfpathlineto{\pgfqpoint{6.687330in}{17.600000in}}%
\pgfusepath{stroke}%
\end{pgfscope}%
\begin{pgfscope}%
\definecolor{textcolor}{rgb}{0.150000,0.150000,0.150000}%
\pgfsetstrokecolor{textcolor}%
\pgfsetfillcolor{textcolor}%
\pgftext[x=6.687330in,y=16.624871in,,top]{\color{textcolor}\rmfamily\fontsize{14.000000}{16.800000}\selectfont 35}%
\end{pgfscope}%
\begin{pgfscope}%
\pgfpathrectangle{\pgfqpoint{2.125000in}{16.722093in}}{\pgfqpoint{5.489583in}{0.877907in}}%
\pgfusepath{clip}%
\pgfsetroundcap%
\pgfsetroundjoin%
\pgfsetlinewidth{0.803000pt}%
\definecolor{currentstroke}{rgb}{1.000000,1.000000,1.000000}%
\pgfsetstrokecolor{currentstroke}%
\pgfsetdash{}{0pt}%
\pgfpathmoveto{\pgfqpoint{7.303445in}{16.722093in}}%
\pgfpathlineto{\pgfqpoint{7.303445in}{17.600000in}}%
\pgfusepath{stroke}%
\end{pgfscope}%
\begin{pgfscope}%
\definecolor{textcolor}{rgb}{0.150000,0.150000,0.150000}%
\pgfsetstrokecolor{textcolor}%
\pgfsetfillcolor{textcolor}%
\pgftext[x=7.303445in,y=16.624871in,,top]{\color{textcolor}\rmfamily\fontsize{14.000000}{16.800000}\selectfont 40}%
\end{pgfscope}%
\begin{pgfscope}%
\pgfpathrectangle{\pgfqpoint{2.125000in}{16.722093in}}{\pgfqpoint{5.489583in}{0.877907in}}%
\pgfusepath{clip}%
\pgfsetroundcap%
\pgfsetroundjoin%
\pgfsetlinewidth{0.803000pt}%
\definecolor{currentstroke}{rgb}{1.000000,1.000000,1.000000}%
\pgfsetstrokecolor{currentstroke}%
\pgfsetdash{}{0pt}%
\pgfpathmoveto{\pgfqpoint{2.125000in}{17.000124in}}%
\pgfpathlineto{\pgfqpoint{7.614583in}{17.000124in}}%
\pgfusepath{stroke}%
\end{pgfscope}%
\begin{pgfscope}%
\definecolor{textcolor}{rgb}{0.150000,0.150000,0.150000}%
\pgfsetstrokecolor{textcolor}%
\pgfsetfillcolor{textcolor}%
\pgftext[x=1.904066in,y=16.926258in,left,base]{\color{textcolor}\rmfamily\fontsize{14.000000}{16.800000}\selectfont 0}%
\end{pgfscope}%
\begin{pgfscope}%
\pgfpathrectangle{\pgfqpoint{2.125000in}{16.722093in}}{\pgfqpoint{5.489583in}{0.877907in}}%
\pgfusepath{clip}%
\pgfsetroundcap%
\pgfsetroundjoin%
\pgfsetlinewidth{0.803000pt}%
\definecolor{currentstroke}{rgb}{1.000000,1.000000,1.000000}%
\pgfsetstrokecolor{currentstroke}%
\pgfsetdash{}{0pt}%
\pgfpathmoveto{\pgfqpoint{2.125000in}{17.560095in}}%
\pgfpathlineto{\pgfqpoint{7.614583in}{17.560095in}}%
\pgfusepath{stroke}%
\end{pgfscope}%
\begin{pgfscope}%
\definecolor{textcolor}{rgb}{0.150000,0.150000,0.150000}%
\pgfsetstrokecolor{textcolor}%
\pgfsetfillcolor{textcolor}%
\pgftext[x=1.904066in,y=17.486229in,left,base]{\color{textcolor}\rmfamily\fontsize{14.000000}{16.800000}\selectfont 1}%
\end{pgfscope}%
\begin{pgfscope}%
\pgfpathrectangle{\pgfqpoint{2.125000in}{16.722093in}}{\pgfqpoint{5.489583in}{0.877907in}}%
\pgfusepath{clip}%
\pgfsetbuttcap%
\pgfsetroundjoin%
\definecolor{currentfill}{rgb}{0.121569,0.466667,0.705882}%
\pgfsetfillcolor{currentfill}%
\pgfsetfillopacity{0.250000}%
\pgfsetlinewidth{1.003750pt}%
\definecolor{currentstroke}{rgb}{1.000000,1.000000,1.000000}%
\pgfsetstrokecolor{currentstroke}%
\pgfsetstrokeopacity{0.250000}%
\pgfsetdash{}{0pt}%
\pgfpathmoveto{\pgfqpoint{2.436138in}{17.028377in}}%
\pgfpathlineto{\pgfqpoint{2.436138in}{16.971870in}}%
\pgfpathlineto{\pgfqpoint{2.620972in}{16.951276in}}%
\pgfpathlineto{\pgfqpoint{2.744195in}{16.937151in}}%
\pgfpathlineto{\pgfqpoint{2.867418in}{16.925718in}}%
\pgfpathlineto{\pgfqpoint{2.990641in}{16.915872in}}%
\pgfpathlineto{\pgfqpoint{3.113864in}{16.907108in}}%
\pgfpathlineto{\pgfqpoint{3.237087in}{16.899142in}}%
\pgfpathlineto{\pgfqpoint{3.360310in}{16.891800in}}%
\pgfpathlineto{\pgfqpoint{3.483533in}{16.884962in}}%
\pgfpathlineto{\pgfqpoint{3.606756in}{16.878543in}}%
\pgfpathlineto{\pgfqpoint{3.729979in}{16.872481in}}%
\pgfpathlineto{\pgfqpoint{3.853202in}{16.866724in}}%
\pgfpathlineto{\pgfqpoint{3.976425in}{16.861236in}}%
\pgfpathlineto{\pgfqpoint{4.099648in}{16.855984in}}%
\pgfpathlineto{\pgfqpoint{4.222871in}{16.850945in}}%
\pgfpathlineto{\pgfqpoint{4.346094in}{16.846096in}}%
\pgfpathlineto{\pgfqpoint{4.469317in}{16.841422in}}%
\pgfpathlineto{\pgfqpoint{4.592540in}{16.836905in}}%
\pgfpathlineto{\pgfqpoint{4.715763in}{16.832534in}}%
\pgfpathlineto{\pgfqpoint{4.838986in}{16.828296in}}%
\pgfpathlineto{\pgfqpoint{4.962209in}{16.824184in}}%
\pgfpathlineto{\pgfqpoint{5.085432in}{16.820188in}}%
\pgfpathlineto{\pgfqpoint{5.208655in}{16.816301in}}%
\pgfpathlineto{\pgfqpoint{5.331878in}{16.812515in}}%
\pgfpathlineto{\pgfqpoint{5.455101in}{16.808823in}}%
\pgfpathlineto{\pgfqpoint{5.578324in}{16.805221in}}%
\pgfpathlineto{\pgfqpoint{5.701547in}{16.801703in}}%
\pgfpathlineto{\pgfqpoint{5.824770in}{16.798264in}}%
\pgfpathlineto{\pgfqpoint{5.947993in}{16.794901in}}%
\pgfpathlineto{\pgfqpoint{6.071216in}{16.791610in}}%
\pgfpathlineto{\pgfqpoint{6.194439in}{16.788387in}}%
\pgfpathlineto{\pgfqpoint{6.317662in}{16.785229in}}%
\pgfpathlineto{\pgfqpoint{6.440885in}{16.782133in}}%
\pgfpathlineto{\pgfqpoint{6.564108in}{16.779096in}}%
\pgfpathlineto{\pgfqpoint{6.687330in}{16.776116in}}%
\pgfpathlineto{\pgfqpoint{6.810553in}{16.773191in}}%
\pgfpathlineto{\pgfqpoint{6.933776in}{16.770318in}}%
\pgfpathlineto{\pgfqpoint{7.056999in}{16.767496in}}%
\pgfpathlineto{\pgfqpoint{7.180222in}{16.764724in}}%
\pgfpathlineto{\pgfqpoint{7.365057in}{16.761998in}}%
\pgfpathlineto{\pgfqpoint{7.365057in}{17.238250in}}%
\pgfpathlineto{\pgfqpoint{7.365057in}{17.238250in}}%
\pgfpathlineto{\pgfqpoint{7.180222in}{17.235524in}}%
\pgfpathlineto{\pgfqpoint{7.056999in}{17.232751in}}%
\pgfpathlineto{\pgfqpoint{6.933776in}{17.229929in}}%
\pgfpathlineto{\pgfqpoint{6.810553in}{17.227057in}}%
\pgfpathlineto{\pgfqpoint{6.687330in}{17.224131in}}%
\pgfpathlineto{\pgfqpoint{6.564108in}{17.221151in}}%
\pgfpathlineto{\pgfqpoint{6.440885in}{17.218114in}}%
\pgfpathlineto{\pgfqpoint{6.317662in}{17.215018in}}%
\pgfpathlineto{\pgfqpoint{6.194439in}{17.211860in}}%
\pgfpathlineto{\pgfqpoint{6.071216in}{17.208637in}}%
\pgfpathlineto{\pgfqpoint{5.947993in}{17.205346in}}%
\pgfpathlineto{\pgfqpoint{5.824770in}{17.201983in}}%
\pgfpathlineto{\pgfqpoint{5.701547in}{17.198545in}}%
\pgfpathlineto{\pgfqpoint{5.578324in}{17.195026in}}%
\pgfpathlineto{\pgfqpoint{5.455101in}{17.191424in}}%
\pgfpathlineto{\pgfqpoint{5.331878in}{17.187733in}}%
\pgfpathlineto{\pgfqpoint{5.208655in}{17.183947in}}%
\pgfpathlineto{\pgfqpoint{5.085432in}{17.180059in}}%
\pgfpathlineto{\pgfqpoint{4.962209in}{17.176064in}}%
\pgfpathlineto{\pgfqpoint{4.838986in}{17.171951in}}%
\pgfpathlineto{\pgfqpoint{4.715763in}{17.167714in}}%
\pgfpathlineto{\pgfqpoint{4.592540in}{17.163342in}}%
\pgfpathlineto{\pgfqpoint{4.469317in}{17.158826in}}%
\pgfpathlineto{\pgfqpoint{4.346094in}{17.154151in}}%
\pgfpathlineto{\pgfqpoint{4.222871in}{17.149303in}}%
\pgfpathlineto{\pgfqpoint{4.099648in}{17.144263in}}%
\pgfpathlineto{\pgfqpoint{3.976425in}{17.139012in}}%
\pgfpathlineto{\pgfqpoint{3.853202in}{17.133523in}}%
\pgfpathlineto{\pgfqpoint{3.729979in}{17.127767in}}%
\pgfpathlineto{\pgfqpoint{3.606756in}{17.121704in}}%
\pgfpathlineto{\pgfqpoint{3.483533in}{17.115286in}}%
\pgfpathlineto{\pgfqpoint{3.360310in}{17.108448in}}%
\pgfpathlineto{\pgfqpoint{3.237087in}{17.101105in}}%
\pgfpathlineto{\pgfqpoint{3.113864in}{17.093140in}}%
\pgfpathlineto{\pgfqpoint{2.990641in}{17.084375in}}%
\pgfpathlineto{\pgfqpoint{2.867418in}{17.074530in}}%
\pgfpathlineto{\pgfqpoint{2.744195in}{17.063096in}}%
\pgfpathlineto{\pgfqpoint{2.620972in}{17.048972in}}%
\pgfpathlineto{\pgfqpoint{2.436138in}{17.028377in}}%
\pgfpathclose%
\pgfusepath{stroke,fill}%
\end{pgfscope}%
\begin{pgfscope}%
\pgfpathrectangle{\pgfqpoint{2.125000in}{16.722093in}}{\pgfqpoint{5.489583in}{0.877907in}}%
\pgfusepath{clip}%
\pgfsetbuttcap%
\pgfsetroundjoin%
\pgfsetlinewidth{1.505625pt}%
\definecolor{currentstroke}{rgb}{0.000000,0.000000,0.000000}%
\pgfsetstrokecolor{currentstroke}%
\pgfsetdash{}{0pt}%
\pgfpathmoveto{\pgfqpoint{2.374527in}{17.000124in}}%
\pgfpathlineto{\pgfqpoint{2.374527in}{17.560095in}}%
\pgfusepath{stroke}%
\end{pgfscope}%
\begin{pgfscope}%
\pgfpathrectangle{\pgfqpoint{2.125000in}{16.722093in}}{\pgfqpoint{5.489583in}{0.877907in}}%
\pgfusepath{clip}%
\pgfsetbuttcap%
\pgfsetroundjoin%
\pgfsetlinewidth{1.505625pt}%
\definecolor{currentstroke}{rgb}{0.000000,0.000000,0.000000}%
\pgfsetstrokecolor{currentstroke}%
\pgfsetdash{}{0pt}%
\pgfpathmoveto{\pgfqpoint{2.497749in}{17.000124in}}%
\pgfpathlineto{\pgfqpoint{2.497749in}{17.558578in}}%
\pgfusepath{stroke}%
\end{pgfscope}%
\begin{pgfscope}%
\pgfpathrectangle{\pgfqpoint{2.125000in}{16.722093in}}{\pgfqpoint{5.489583in}{0.877907in}}%
\pgfusepath{clip}%
\pgfsetbuttcap%
\pgfsetroundjoin%
\pgfsetlinewidth{1.505625pt}%
\definecolor{currentstroke}{rgb}{0.000000,0.000000,0.000000}%
\pgfsetstrokecolor{currentstroke}%
\pgfsetdash{}{0pt}%
\pgfpathmoveto{\pgfqpoint{2.620972in}{17.000124in}}%
\pgfpathlineto{\pgfqpoint{2.620972in}{17.557087in}}%
\pgfusepath{stroke}%
\end{pgfscope}%
\begin{pgfscope}%
\pgfpathrectangle{\pgfqpoint{2.125000in}{16.722093in}}{\pgfqpoint{5.489583in}{0.877907in}}%
\pgfusepath{clip}%
\pgfsetbuttcap%
\pgfsetroundjoin%
\pgfsetlinewidth{1.505625pt}%
\definecolor{currentstroke}{rgb}{0.000000,0.000000,0.000000}%
\pgfsetstrokecolor{currentstroke}%
\pgfsetdash{}{0pt}%
\pgfpathmoveto{\pgfqpoint{2.744195in}{17.000124in}}%
\pgfpathlineto{\pgfqpoint{2.744195in}{17.555564in}}%
\pgfusepath{stroke}%
\end{pgfscope}%
\begin{pgfscope}%
\pgfpathrectangle{\pgfqpoint{2.125000in}{16.722093in}}{\pgfqpoint{5.489583in}{0.877907in}}%
\pgfusepath{clip}%
\pgfsetbuttcap%
\pgfsetroundjoin%
\pgfsetlinewidth{1.505625pt}%
\definecolor{currentstroke}{rgb}{0.000000,0.000000,0.000000}%
\pgfsetstrokecolor{currentstroke}%
\pgfsetdash{}{0pt}%
\pgfpathmoveto{\pgfqpoint{2.867418in}{17.000124in}}%
\pgfpathlineto{\pgfqpoint{2.867418in}{17.554022in}}%
\pgfusepath{stroke}%
\end{pgfscope}%
\begin{pgfscope}%
\pgfpathrectangle{\pgfqpoint{2.125000in}{16.722093in}}{\pgfqpoint{5.489583in}{0.877907in}}%
\pgfusepath{clip}%
\pgfsetbuttcap%
\pgfsetroundjoin%
\pgfsetlinewidth{1.505625pt}%
\definecolor{currentstroke}{rgb}{0.000000,0.000000,0.000000}%
\pgfsetstrokecolor{currentstroke}%
\pgfsetdash{}{0pt}%
\pgfpathmoveto{\pgfqpoint{2.990641in}{17.000124in}}%
\pgfpathlineto{\pgfqpoint{2.990641in}{17.552526in}}%
\pgfusepath{stroke}%
\end{pgfscope}%
\begin{pgfscope}%
\pgfpathrectangle{\pgfqpoint{2.125000in}{16.722093in}}{\pgfqpoint{5.489583in}{0.877907in}}%
\pgfusepath{clip}%
\pgfsetbuttcap%
\pgfsetroundjoin%
\pgfsetlinewidth{1.505625pt}%
\definecolor{currentstroke}{rgb}{0.000000,0.000000,0.000000}%
\pgfsetstrokecolor{currentstroke}%
\pgfsetdash{}{0pt}%
\pgfpathmoveto{\pgfqpoint{3.113864in}{17.000124in}}%
\pgfpathlineto{\pgfqpoint{3.113864in}{17.551044in}}%
\pgfusepath{stroke}%
\end{pgfscope}%
\begin{pgfscope}%
\pgfpathrectangle{\pgfqpoint{2.125000in}{16.722093in}}{\pgfqpoint{5.489583in}{0.877907in}}%
\pgfusepath{clip}%
\pgfsetbuttcap%
\pgfsetroundjoin%
\pgfsetlinewidth{1.505625pt}%
\definecolor{currentstroke}{rgb}{0.000000,0.000000,0.000000}%
\pgfsetstrokecolor{currentstroke}%
\pgfsetdash{}{0pt}%
\pgfpathmoveto{\pgfqpoint{3.237087in}{17.000124in}}%
\pgfpathlineto{\pgfqpoint{3.237087in}{17.549523in}}%
\pgfusepath{stroke}%
\end{pgfscope}%
\begin{pgfscope}%
\pgfpathrectangle{\pgfqpoint{2.125000in}{16.722093in}}{\pgfqpoint{5.489583in}{0.877907in}}%
\pgfusepath{clip}%
\pgfsetbuttcap%
\pgfsetroundjoin%
\pgfsetlinewidth{1.505625pt}%
\definecolor{currentstroke}{rgb}{0.000000,0.000000,0.000000}%
\pgfsetstrokecolor{currentstroke}%
\pgfsetdash{}{0pt}%
\pgfpathmoveto{\pgfqpoint{3.360310in}{17.000124in}}%
\pgfpathlineto{\pgfqpoint{3.360310in}{17.547995in}}%
\pgfusepath{stroke}%
\end{pgfscope}%
\begin{pgfscope}%
\pgfpathrectangle{\pgfqpoint{2.125000in}{16.722093in}}{\pgfqpoint{5.489583in}{0.877907in}}%
\pgfusepath{clip}%
\pgfsetbuttcap%
\pgfsetroundjoin%
\pgfsetlinewidth{1.505625pt}%
\definecolor{currentstroke}{rgb}{0.000000,0.000000,0.000000}%
\pgfsetstrokecolor{currentstroke}%
\pgfsetdash{}{0pt}%
\pgfpathmoveto{\pgfqpoint{3.483533in}{17.000124in}}%
\pgfpathlineto{\pgfqpoint{3.483533in}{17.546426in}}%
\pgfusepath{stroke}%
\end{pgfscope}%
\begin{pgfscope}%
\pgfpathrectangle{\pgfqpoint{2.125000in}{16.722093in}}{\pgfqpoint{5.489583in}{0.877907in}}%
\pgfusepath{clip}%
\pgfsetbuttcap%
\pgfsetroundjoin%
\pgfsetlinewidth{1.505625pt}%
\definecolor{currentstroke}{rgb}{0.000000,0.000000,0.000000}%
\pgfsetstrokecolor{currentstroke}%
\pgfsetdash{}{0pt}%
\pgfpathmoveto{\pgfqpoint{3.606756in}{17.000124in}}%
\pgfpathlineto{\pgfqpoint{3.606756in}{17.544892in}}%
\pgfusepath{stroke}%
\end{pgfscope}%
\begin{pgfscope}%
\pgfpathrectangle{\pgfqpoint{2.125000in}{16.722093in}}{\pgfqpoint{5.489583in}{0.877907in}}%
\pgfusepath{clip}%
\pgfsetbuttcap%
\pgfsetroundjoin%
\pgfsetlinewidth{1.505625pt}%
\definecolor{currentstroke}{rgb}{0.000000,0.000000,0.000000}%
\pgfsetstrokecolor{currentstroke}%
\pgfsetdash{}{0pt}%
\pgfpathmoveto{\pgfqpoint{3.729979in}{17.000124in}}%
\pgfpathlineto{\pgfqpoint{3.729979in}{17.543391in}}%
\pgfusepath{stroke}%
\end{pgfscope}%
\begin{pgfscope}%
\pgfpathrectangle{\pgfqpoint{2.125000in}{16.722093in}}{\pgfqpoint{5.489583in}{0.877907in}}%
\pgfusepath{clip}%
\pgfsetbuttcap%
\pgfsetroundjoin%
\pgfsetlinewidth{1.505625pt}%
\definecolor{currentstroke}{rgb}{0.000000,0.000000,0.000000}%
\pgfsetstrokecolor{currentstroke}%
\pgfsetdash{}{0pt}%
\pgfpathmoveto{\pgfqpoint{3.853202in}{17.000124in}}%
\pgfpathlineto{\pgfqpoint{3.853202in}{17.541902in}}%
\pgfusepath{stroke}%
\end{pgfscope}%
\begin{pgfscope}%
\pgfpathrectangle{\pgfqpoint{2.125000in}{16.722093in}}{\pgfqpoint{5.489583in}{0.877907in}}%
\pgfusepath{clip}%
\pgfsetbuttcap%
\pgfsetroundjoin%
\pgfsetlinewidth{1.505625pt}%
\definecolor{currentstroke}{rgb}{0.000000,0.000000,0.000000}%
\pgfsetstrokecolor{currentstroke}%
\pgfsetdash{}{0pt}%
\pgfpathmoveto{\pgfqpoint{3.976425in}{17.000124in}}%
\pgfpathlineto{\pgfqpoint{3.976425in}{17.540435in}}%
\pgfusepath{stroke}%
\end{pgfscope}%
\begin{pgfscope}%
\pgfpathrectangle{\pgfqpoint{2.125000in}{16.722093in}}{\pgfqpoint{5.489583in}{0.877907in}}%
\pgfusepath{clip}%
\pgfsetbuttcap%
\pgfsetroundjoin%
\pgfsetlinewidth{1.505625pt}%
\definecolor{currentstroke}{rgb}{0.000000,0.000000,0.000000}%
\pgfsetstrokecolor{currentstroke}%
\pgfsetdash{}{0pt}%
\pgfpathmoveto{\pgfqpoint{4.099648in}{17.000124in}}%
\pgfpathlineto{\pgfqpoint{4.099648in}{17.538939in}}%
\pgfusepath{stroke}%
\end{pgfscope}%
\begin{pgfscope}%
\pgfpathrectangle{\pgfqpoint{2.125000in}{16.722093in}}{\pgfqpoint{5.489583in}{0.877907in}}%
\pgfusepath{clip}%
\pgfsetbuttcap%
\pgfsetroundjoin%
\pgfsetlinewidth{1.505625pt}%
\definecolor{currentstroke}{rgb}{0.000000,0.000000,0.000000}%
\pgfsetstrokecolor{currentstroke}%
\pgfsetdash{}{0pt}%
\pgfpathmoveto{\pgfqpoint{4.222871in}{17.000124in}}%
\pgfpathlineto{\pgfqpoint{4.222871in}{17.537452in}}%
\pgfusepath{stroke}%
\end{pgfscope}%
\begin{pgfscope}%
\pgfpathrectangle{\pgfqpoint{2.125000in}{16.722093in}}{\pgfqpoint{5.489583in}{0.877907in}}%
\pgfusepath{clip}%
\pgfsetbuttcap%
\pgfsetroundjoin%
\pgfsetlinewidth{1.505625pt}%
\definecolor{currentstroke}{rgb}{0.000000,0.000000,0.000000}%
\pgfsetstrokecolor{currentstroke}%
\pgfsetdash{}{0pt}%
\pgfpathmoveto{\pgfqpoint{4.346094in}{17.000124in}}%
\pgfpathlineto{\pgfqpoint{4.346094in}{17.535968in}}%
\pgfusepath{stroke}%
\end{pgfscope}%
\begin{pgfscope}%
\pgfpathrectangle{\pgfqpoint{2.125000in}{16.722093in}}{\pgfqpoint{5.489583in}{0.877907in}}%
\pgfusepath{clip}%
\pgfsetbuttcap%
\pgfsetroundjoin%
\pgfsetlinewidth{1.505625pt}%
\definecolor{currentstroke}{rgb}{0.000000,0.000000,0.000000}%
\pgfsetstrokecolor{currentstroke}%
\pgfsetdash{}{0pt}%
\pgfpathmoveto{\pgfqpoint{4.469317in}{17.000124in}}%
\pgfpathlineto{\pgfqpoint{4.469317in}{17.534527in}}%
\pgfusepath{stroke}%
\end{pgfscope}%
\begin{pgfscope}%
\pgfpathrectangle{\pgfqpoint{2.125000in}{16.722093in}}{\pgfqpoint{5.489583in}{0.877907in}}%
\pgfusepath{clip}%
\pgfsetbuttcap%
\pgfsetroundjoin%
\pgfsetlinewidth{1.505625pt}%
\definecolor{currentstroke}{rgb}{0.000000,0.000000,0.000000}%
\pgfsetstrokecolor{currentstroke}%
\pgfsetdash{}{0pt}%
\pgfpathmoveto{\pgfqpoint{4.592540in}{17.000124in}}%
\pgfpathlineto{\pgfqpoint{4.592540in}{17.533072in}}%
\pgfusepath{stroke}%
\end{pgfscope}%
\begin{pgfscope}%
\pgfpathrectangle{\pgfqpoint{2.125000in}{16.722093in}}{\pgfqpoint{5.489583in}{0.877907in}}%
\pgfusepath{clip}%
\pgfsetbuttcap%
\pgfsetroundjoin%
\pgfsetlinewidth{1.505625pt}%
\definecolor{currentstroke}{rgb}{0.000000,0.000000,0.000000}%
\pgfsetstrokecolor{currentstroke}%
\pgfsetdash{}{0pt}%
\pgfpathmoveto{\pgfqpoint{4.715763in}{17.000124in}}%
\pgfpathlineto{\pgfqpoint{4.715763in}{17.531613in}}%
\pgfusepath{stroke}%
\end{pgfscope}%
\begin{pgfscope}%
\pgfpathrectangle{\pgfqpoint{2.125000in}{16.722093in}}{\pgfqpoint{5.489583in}{0.877907in}}%
\pgfusepath{clip}%
\pgfsetbuttcap%
\pgfsetroundjoin%
\pgfsetlinewidth{1.505625pt}%
\definecolor{currentstroke}{rgb}{0.000000,0.000000,0.000000}%
\pgfsetstrokecolor{currentstroke}%
\pgfsetdash{}{0pt}%
\pgfpathmoveto{\pgfqpoint{4.838986in}{17.000124in}}%
\pgfpathlineto{\pgfqpoint{4.838986in}{17.530118in}}%
\pgfusepath{stroke}%
\end{pgfscope}%
\begin{pgfscope}%
\pgfpathrectangle{\pgfqpoint{2.125000in}{16.722093in}}{\pgfqpoint{5.489583in}{0.877907in}}%
\pgfusepath{clip}%
\pgfsetbuttcap%
\pgfsetroundjoin%
\pgfsetlinewidth{1.505625pt}%
\definecolor{currentstroke}{rgb}{0.000000,0.000000,0.000000}%
\pgfsetstrokecolor{currentstroke}%
\pgfsetdash{}{0pt}%
\pgfpathmoveto{\pgfqpoint{4.962209in}{17.000124in}}%
\pgfpathlineto{\pgfqpoint{4.962209in}{17.528611in}}%
\pgfusepath{stroke}%
\end{pgfscope}%
\begin{pgfscope}%
\pgfpathrectangle{\pgfqpoint{2.125000in}{16.722093in}}{\pgfqpoint{5.489583in}{0.877907in}}%
\pgfusepath{clip}%
\pgfsetbuttcap%
\pgfsetroundjoin%
\pgfsetlinewidth{1.505625pt}%
\definecolor{currentstroke}{rgb}{0.000000,0.000000,0.000000}%
\pgfsetstrokecolor{currentstroke}%
\pgfsetdash{}{0pt}%
\pgfpathmoveto{\pgfqpoint{5.085432in}{17.000124in}}%
\pgfpathlineto{\pgfqpoint{5.085432in}{17.527124in}}%
\pgfusepath{stroke}%
\end{pgfscope}%
\begin{pgfscope}%
\pgfpathrectangle{\pgfqpoint{2.125000in}{16.722093in}}{\pgfqpoint{5.489583in}{0.877907in}}%
\pgfusepath{clip}%
\pgfsetbuttcap%
\pgfsetroundjoin%
\pgfsetlinewidth{1.505625pt}%
\definecolor{currentstroke}{rgb}{0.000000,0.000000,0.000000}%
\pgfsetstrokecolor{currentstroke}%
\pgfsetdash{}{0pt}%
\pgfpathmoveto{\pgfqpoint{5.208655in}{17.000124in}}%
\pgfpathlineto{\pgfqpoint{5.208655in}{17.525681in}}%
\pgfusepath{stroke}%
\end{pgfscope}%
\begin{pgfscope}%
\pgfpathrectangle{\pgfqpoint{2.125000in}{16.722093in}}{\pgfqpoint{5.489583in}{0.877907in}}%
\pgfusepath{clip}%
\pgfsetbuttcap%
\pgfsetroundjoin%
\pgfsetlinewidth{1.505625pt}%
\definecolor{currentstroke}{rgb}{0.000000,0.000000,0.000000}%
\pgfsetstrokecolor{currentstroke}%
\pgfsetdash{}{0pt}%
\pgfpathmoveto{\pgfqpoint{5.331878in}{17.000124in}}%
\pgfpathlineto{\pgfqpoint{5.331878in}{17.524252in}}%
\pgfusepath{stroke}%
\end{pgfscope}%
\begin{pgfscope}%
\pgfpathrectangle{\pgfqpoint{2.125000in}{16.722093in}}{\pgfqpoint{5.489583in}{0.877907in}}%
\pgfusepath{clip}%
\pgfsetbuttcap%
\pgfsetroundjoin%
\pgfsetlinewidth{1.505625pt}%
\definecolor{currentstroke}{rgb}{0.000000,0.000000,0.000000}%
\pgfsetstrokecolor{currentstroke}%
\pgfsetdash{}{0pt}%
\pgfpathmoveto{\pgfqpoint{5.455101in}{17.000124in}}%
\pgfpathlineto{\pgfqpoint{5.455101in}{17.522852in}}%
\pgfusepath{stroke}%
\end{pgfscope}%
\begin{pgfscope}%
\pgfpathrectangle{\pgfqpoint{2.125000in}{16.722093in}}{\pgfqpoint{5.489583in}{0.877907in}}%
\pgfusepath{clip}%
\pgfsetbuttcap%
\pgfsetroundjoin%
\pgfsetlinewidth{1.505625pt}%
\definecolor{currentstroke}{rgb}{0.000000,0.000000,0.000000}%
\pgfsetstrokecolor{currentstroke}%
\pgfsetdash{}{0pt}%
\pgfpathmoveto{\pgfqpoint{5.578324in}{17.000124in}}%
\pgfpathlineto{\pgfqpoint{5.578324in}{17.521471in}}%
\pgfusepath{stroke}%
\end{pgfscope}%
\begin{pgfscope}%
\pgfpathrectangle{\pgfqpoint{2.125000in}{16.722093in}}{\pgfqpoint{5.489583in}{0.877907in}}%
\pgfusepath{clip}%
\pgfsetbuttcap%
\pgfsetroundjoin%
\pgfsetlinewidth{1.505625pt}%
\definecolor{currentstroke}{rgb}{0.000000,0.000000,0.000000}%
\pgfsetstrokecolor{currentstroke}%
\pgfsetdash{}{0pt}%
\pgfpathmoveto{\pgfqpoint{5.701547in}{17.000124in}}%
\pgfpathlineto{\pgfqpoint{5.701547in}{17.520063in}}%
\pgfusepath{stroke}%
\end{pgfscope}%
\begin{pgfscope}%
\pgfpathrectangle{\pgfqpoint{2.125000in}{16.722093in}}{\pgfqpoint{5.489583in}{0.877907in}}%
\pgfusepath{clip}%
\pgfsetbuttcap%
\pgfsetroundjoin%
\pgfsetlinewidth{1.505625pt}%
\definecolor{currentstroke}{rgb}{0.000000,0.000000,0.000000}%
\pgfsetstrokecolor{currentstroke}%
\pgfsetdash{}{0pt}%
\pgfpathmoveto{\pgfqpoint{5.824770in}{17.000124in}}%
\pgfpathlineto{\pgfqpoint{5.824770in}{17.518648in}}%
\pgfusepath{stroke}%
\end{pgfscope}%
\begin{pgfscope}%
\pgfpathrectangle{\pgfqpoint{2.125000in}{16.722093in}}{\pgfqpoint{5.489583in}{0.877907in}}%
\pgfusepath{clip}%
\pgfsetbuttcap%
\pgfsetroundjoin%
\pgfsetlinewidth{1.505625pt}%
\definecolor{currentstroke}{rgb}{0.000000,0.000000,0.000000}%
\pgfsetstrokecolor{currentstroke}%
\pgfsetdash{}{0pt}%
\pgfpathmoveto{\pgfqpoint{5.947993in}{17.000124in}}%
\pgfpathlineto{\pgfqpoint{5.947993in}{17.517275in}}%
\pgfusepath{stroke}%
\end{pgfscope}%
\begin{pgfscope}%
\pgfpathrectangle{\pgfqpoint{2.125000in}{16.722093in}}{\pgfqpoint{5.489583in}{0.877907in}}%
\pgfusepath{clip}%
\pgfsetbuttcap%
\pgfsetroundjoin%
\pgfsetlinewidth{1.505625pt}%
\definecolor{currentstroke}{rgb}{0.000000,0.000000,0.000000}%
\pgfsetstrokecolor{currentstroke}%
\pgfsetdash{}{0pt}%
\pgfpathmoveto{\pgfqpoint{6.071216in}{17.000124in}}%
\pgfpathlineto{\pgfqpoint{6.071216in}{17.515893in}}%
\pgfusepath{stroke}%
\end{pgfscope}%
\begin{pgfscope}%
\pgfpathrectangle{\pgfqpoint{2.125000in}{16.722093in}}{\pgfqpoint{5.489583in}{0.877907in}}%
\pgfusepath{clip}%
\pgfsetbuttcap%
\pgfsetroundjoin%
\pgfsetlinewidth{1.505625pt}%
\definecolor{currentstroke}{rgb}{0.000000,0.000000,0.000000}%
\pgfsetstrokecolor{currentstroke}%
\pgfsetdash{}{0pt}%
\pgfpathmoveto{\pgfqpoint{6.194439in}{17.000124in}}%
\pgfpathlineto{\pgfqpoint{6.194439in}{17.514547in}}%
\pgfusepath{stroke}%
\end{pgfscope}%
\begin{pgfscope}%
\pgfpathrectangle{\pgfqpoint{2.125000in}{16.722093in}}{\pgfqpoint{5.489583in}{0.877907in}}%
\pgfusepath{clip}%
\pgfsetbuttcap%
\pgfsetroundjoin%
\pgfsetlinewidth{1.505625pt}%
\definecolor{currentstroke}{rgb}{0.000000,0.000000,0.000000}%
\pgfsetstrokecolor{currentstroke}%
\pgfsetdash{}{0pt}%
\pgfpathmoveto{\pgfqpoint{6.317662in}{17.000124in}}%
\pgfpathlineto{\pgfqpoint{6.317662in}{17.513191in}}%
\pgfusepath{stroke}%
\end{pgfscope}%
\begin{pgfscope}%
\pgfpathrectangle{\pgfqpoint{2.125000in}{16.722093in}}{\pgfqpoint{5.489583in}{0.877907in}}%
\pgfusepath{clip}%
\pgfsetbuttcap%
\pgfsetroundjoin%
\pgfsetlinewidth{1.505625pt}%
\definecolor{currentstroke}{rgb}{0.000000,0.000000,0.000000}%
\pgfsetstrokecolor{currentstroke}%
\pgfsetdash{}{0pt}%
\pgfpathmoveto{\pgfqpoint{6.440885in}{17.000124in}}%
\pgfpathlineto{\pgfqpoint{6.440885in}{17.511844in}}%
\pgfusepath{stroke}%
\end{pgfscope}%
\begin{pgfscope}%
\pgfpathrectangle{\pgfqpoint{2.125000in}{16.722093in}}{\pgfqpoint{5.489583in}{0.877907in}}%
\pgfusepath{clip}%
\pgfsetbuttcap%
\pgfsetroundjoin%
\pgfsetlinewidth{1.505625pt}%
\definecolor{currentstroke}{rgb}{0.000000,0.000000,0.000000}%
\pgfsetstrokecolor{currentstroke}%
\pgfsetdash{}{0pt}%
\pgfpathmoveto{\pgfqpoint{6.564108in}{17.000124in}}%
\pgfpathlineto{\pgfqpoint{6.564108in}{17.510514in}}%
\pgfusepath{stroke}%
\end{pgfscope}%
\begin{pgfscope}%
\pgfpathrectangle{\pgfqpoint{2.125000in}{16.722093in}}{\pgfqpoint{5.489583in}{0.877907in}}%
\pgfusepath{clip}%
\pgfsetbuttcap%
\pgfsetroundjoin%
\pgfsetlinewidth{1.505625pt}%
\definecolor{currentstroke}{rgb}{0.000000,0.000000,0.000000}%
\pgfsetstrokecolor{currentstroke}%
\pgfsetdash{}{0pt}%
\pgfpathmoveto{\pgfqpoint{6.687330in}{17.000124in}}%
\pgfpathlineto{\pgfqpoint{6.687330in}{17.509144in}}%
\pgfusepath{stroke}%
\end{pgfscope}%
\begin{pgfscope}%
\pgfpathrectangle{\pgfqpoint{2.125000in}{16.722093in}}{\pgfqpoint{5.489583in}{0.877907in}}%
\pgfusepath{clip}%
\pgfsetbuttcap%
\pgfsetroundjoin%
\pgfsetlinewidth{1.505625pt}%
\definecolor{currentstroke}{rgb}{0.000000,0.000000,0.000000}%
\pgfsetstrokecolor{currentstroke}%
\pgfsetdash{}{0pt}%
\pgfpathmoveto{\pgfqpoint{6.810553in}{17.000124in}}%
\pgfpathlineto{\pgfqpoint{6.810553in}{17.507760in}}%
\pgfusepath{stroke}%
\end{pgfscope}%
\begin{pgfscope}%
\pgfpathrectangle{\pgfqpoint{2.125000in}{16.722093in}}{\pgfqpoint{5.489583in}{0.877907in}}%
\pgfusepath{clip}%
\pgfsetbuttcap%
\pgfsetroundjoin%
\pgfsetlinewidth{1.505625pt}%
\definecolor{currentstroke}{rgb}{0.000000,0.000000,0.000000}%
\pgfsetstrokecolor{currentstroke}%
\pgfsetdash{}{0pt}%
\pgfpathmoveto{\pgfqpoint{6.933776in}{17.000124in}}%
\pgfpathlineto{\pgfqpoint{6.933776in}{17.506375in}}%
\pgfusepath{stroke}%
\end{pgfscope}%
\begin{pgfscope}%
\pgfpathrectangle{\pgfqpoint{2.125000in}{16.722093in}}{\pgfqpoint{5.489583in}{0.877907in}}%
\pgfusepath{clip}%
\pgfsetbuttcap%
\pgfsetroundjoin%
\pgfsetlinewidth{1.505625pt}%
\definecolor{currentstroke}{rgb}{0.000000,0.000000,0.000000}%
\pgfsetstrokecolor{currentstroke}%
\pgfsetdash{}{0pt}%
\pgfpathmoveto{\pgfqpoint{7.056999in}{17.000124in}}%
\pgfpathlineto{\pgfqpoint{7.056999in}{17.505002in}}%
\pgfusepath{stroke}%
\end{pgfscope}%
\begin{pgfscope}%
\pgfpathrectangle{\pgfqpoint{2.125000in}{16.722093in}}{\pgfqpoint{5.489583in}{0.877907in}}%
\pgfusepath{clip}%
\pgfsetbuttcap%
\pgfsetroundjoin%
\pgfsetlinewidth{1.505625pt}%
\definecolor{currentstroke}{rgb}{0.000000,0.000000,0.000000}%
\pgfsetstrokecolor{currentstroke}%
\pgfsetdash{}{0pt}%
\pgfpathmoveto{\pgfqpoint{7.180222in}{17.000124in}}%
\pgfpathlineto{\pgfqpoint{7.180222in}{17.503608in}}%
\pgfusepath{stroke}%
\end{pgfscope}%
\begin{pgfscope}%
\pgfpathrectangle{\pgfqpoint{2.125000in}{16.722093in}}{\pgfqpoint{5.489583in}{0.877907in}}%
\pgfusepath{clip}%
\pgfsetbuttcap%
\pgfsetroundjoin%
\pgfsetlinewidth{1.505625pt}%
\definecolor{currentstroke}{rgb}{0.000000,0.000000,0.000000}%
\pgfsetstrokecolor{currentstroke}%
\pgfsetdash{}{0pt}%
\pgfpathmoveto{\pgfqpoint{7.303445in}{17.000124in}}%
\pgfpathlineto{\pgfqpoint{7.303445in}{17.502202in}}%
\pgfusepath{stroke}%
\end{pgfscope}%
\begin{pgfscope}%
\pgfpathrectangle{\pgfqpoint{2.125000in}{16.722093in}}{\pgfqpoint{5.489583in}{0.877907in}}%
\pgfusepath{clip}%
\pgfsetroundcap%
\pgfsetroundjoin%
\pgfsetlinewidth{1.505625pt}%
\definecolor{currentstroke}{rgb}{0.121569,0.466667,0.705882}%
\pgfsetstrokecolor{currentstroke}%
\pgfsetdash{}{0pt}%
\pgfpathmoveto{\pgfqpoint{2.125000in}{17.000124in}}%
\pgfpathlineto{\pgfqpoint{7.614583in}{17.000124in}}%
\pgfusepath{stroke}%
\end{pgfscope}%
\begin{pgfscope}%
\pgfpathrectangle{\pgfqpoint{2.125000in}{16.722093in}}{\pgfqpoint{5.489583in}{0.877907in}}%
\pgfusepath{clip}%
\pgfsetbuttcap%
\pgfsetroundjoin%
\definecolor{currentfill}{rgb}{0.121569,0.466667,0.705882}%
\pgfsetfillcolor{currentfill}%
\pgfsetlinewidth{1.003750pt}%
\definecolor{currentstroke}{rgb}{0.121569,0.466667,0.705882}%
\pgfsetstrokecolor{currentstroke}%
\pgfsetdash{}{0pt}%
\pgfsys@defobject{currentmarker}{\pgfqpoint{-0.034722in}{-0.034722in}}{\pgfqpoint{0.034722in}{0.034722in}}{%
\pgfpathmoveto{\pgfqpoint{0.000000in}{-0.034722in}}%
\pgfpathcurveto{\pgfqpoint{0.009208in}{-0.034722in}}{\pgfqpoint{0.018041in}{-0.031064in}}{\pgfqpoint{0.024552in}{-0.024552in}}%
\pgfpathcurveto{\pgfqpoint{0.031064in}{-0.018041in}}{\pgfqpoint{0.034722in}{-0.009208in}}{\pgfqpoint{0.034722in}{0.000000in}}%
\pgfpathcurveto{\pgfqpoint{0.034722in}{0.009208in}}{\pgfqpoint{0.031064in}{0.018041in}}{\pgfqpoint{0.024552in}{0.024552in}}%
\pgfpathcurveto{\pgfqpoint{0.018041in}{0.031064in}}{\pgfqpoint{0.009208in}{0.034722in}}{\pgfqpoint{0.000000in}{0.034722in}}%
\pgfpathcurveto{\pgfqpoint{-0.009208in}{0.034722in}}{\pgfqpoint{-0.018041in}{0.031064in}}{\pgfqpoint{-0.024552in}{0.024552in}}%
\pgfpathcurveto{\pgfqpoint{-0.031064in}{0.018041in}}{\pgfqpoint{-0.034722in}{0.009208in}}{\pgfqpoint{-0.034722in}{0.000000in}}%
\pgfpathcurveto{\pgfqpoint{-0.034722in}{-0.009208in}}{\pgfqpoint{-0.031064in}{-0.018041in}}{\pgfqpoint{-0.024552in}{-0.024552in}}%
\pgfpathcurveto{\pgfqpoint{-0.018041in}{-0.031064in}}{\pgfqpoint{-0.009208in}{-0.034722in}}{\pgfqpoint{0.000000in}{-0.034722in}}%
\pgfpathclose%
\pgfusepath{stroke,fill}%
}%
\begin{pgfscope}%
\pgfsys@transformshift{2.374527in}{17.560095in}%
\pgfsys@useobject{currentmarker}{}%
\end{pgfscope}%
\begin{pgfscope}%
\pgfsys@transformshift{2.497749in}{17.558578in}%
\pgfsys@useobject{currentmarker}{}%
\end{pgfscope}%
\begin{pgfscope}%
\pgfsys@transformshift{2.620972in}{17.557087in}%
\pgfsys@useobject{currentmarker}{}%
\end{pgfscope}%
\begin{pgfscope}%
\pgfsys@transformshift{2.744195in}{17.555564in}%
\pgfsys@useobject{currentmarker}{}%
\end{pgfscope}%
\begin{pgfscope}%
\pgfsys@transformshift{2.867418in}{17.554022in}%
\pgfsys@useobject{currentmarker}{}%
\end{pgfscope}%
\begin{pgfscope}%
\pgfsys@transformshift{2.990641in}{17.552526in}%
\pgfsys@useobject{currentmarker}{}%
\end{pgfscope}%
\begin{pgfscope}%
\pgfsys@transformshift{3.113864in}{17.551044in}%
\pgfsys@useobject{currentmarker}{}%
\end{pgfscope}%
\begin{pgfscope}%
\pgfsys@transformshift{3.237087in}{17.549523in}%
\pgfsys@useobject{currentmarker}{}%
\end{pgfscope}%
\begin{pgfscope}%
\pgfsys@transformshift{3.360310in}{17.547995in}%
\pgfsys@useobject{currentmarker}{}%
\end{pgfscope}%
\begin{pgfscope}%
\pgfsys@transformshift{3.483533in}{17.546426in}%
\pgfsys@useobject{currentmarker}{}%
\end{pgfscope}%
\begin{pgfscope}%
\pgfsys@transformshift{3.606756in}{17.544892in}%
\pgfsys@useobject{currentmarker}{}%
\end{pgfscope}%
\begin{pgfscope}%
\pgfsys@transformshift{3.729979in}{17.543391in}%
\pgfsys@useobject{currentmarker}{}%
\end{pgfscope}%
\begin{pgfscope}%
\pgfsys@transformshift{3.853202in}{17.541902in}%
\pgfsys@useobject{currentmarker}{}%
\end{pgfscope}%
\begin{pgfscope}%
\pgfsys@transformshift{3.976425in}{17.540435in}%
\pgfsys@useobject{currentmarker}{}%
\end{pgfscope}%
\begin{pgfscope}%
\pgfsys@transformshift{4.099648in}{17.538939in}%
\pgfsys@useobject{currentmarker}{}%
\end{pgfscope}%
\begin{pgfscope}%
\pgfsys@transformshift{4.222871in}{17.537452in}%
\pgfsys@useobject{currentmarker}{}%
\end{pgfscope}%
\begin{pgfscope}%
\pgfsys@transformshift{4.346094in}{17.535968in}%
\pgfsys@useobject{currentmarker}{}%
\end{pgfscope}%
\begin{pgfscope}%
\pgfsys@transformshift{4.469317in}{17.534527in}%
\pgfsys@useobject{currentmarker}{}%
\end{pgfscope}%
\begin{pgfscope}%
\pgfsys@transformshift{4.592540in}{17.533072in}%
\pgfsys@useobject{currentmarker}{}%
\end{pgfscope}%
\begin{pgfscope}%
\pgfsys@transformshift{4.715763in}{17.531613in}%
\pgfsys@useobject{currentmarker}{}%
\end{pgfscope}%
\begin{pgfscope}%
\pgfsys@transformshift{4.838986in}{17.530118in}%
\pgfsys@useobject{currentmarker}{}%
\end{pgfscope}%
\begin{pgfscope}%
\pgfsys@transformshift{4.962209in}{17.528611in}%
\pgfsys@useobject{currentmarker}{}%
\end{pgfscope}%
\begin{pgfscope}%
\pgfsys@transformshift{5.085432in}{17.527124in}%
\pgfsys@useobject{currentmarker}{}%
\end{pgfscope}%
\begin{pgfscope}%
\pgfsys@transformshift{5.208655in}{17.525681in}%
\pgfsys@useobject{currentmarker}{}%
\end{pgfscope}%
\begin{pgfscope}%
\pgfsys@transformshift{5.331878in}{17.524252in}%
\pgfsys@useobject{currentmarker}{}%
\end{pgfscope}%
\begin{pgfscope}%
\pgfsys@transformshift{5.455101in}{17.522852in}%
\pgfsys@useobject{currentmarker}{}%
\end{pgfscope}%
\begin{pgfscope}%
\pgfsys@transformshift{5.578324in}{17.521471in}%
\pgfsys@useobject{currentmarker}{}%
\end{pgfscope}%
\begin{pgfscope}%
\pgfsys@transformshift{5.701547in}{17.520063in}%
\pgfsys@useobject{currentmarker}{}%
\end{pgfscope}%
\begin{pgfscope}%
\pgfsys@transformshift{5.824770in}{17.518648in}%
\pgfsys@useobject{currentmarker}{}%
\end{pgfscope}%
\begin{pgfscope}%
\pgfsys@transformshift{5.947993in}{17.517275in}%
\pgfsys@useobject{currentmarker}{}%
\end{pgfscope}%
\begin{pgfscope}%
\pgfsys@transformshift{6.071216in}{17.515893in}%
\pgfsys@useobject{currentmarker}{}%
\end{pgfscope}%
\begin{pgfscope}%
\pgfsys@transformshift{6.194439in}{17.514547in}%
\pgfsys@useobject{currentmarker}{}%
\end{pgfscope}%
\begin{pgfscope}%
\pgfsys@transformshift{6.317662in}{17.513191in}%
\pgfsys@useobject{currentmarker}{}%
\end{pgfscope}%
\begin{pgfscope}%
\pgfsys@transformshift{6.440885in}{17.511844in}%
\pgfsys@useobject{currentmarker}{}%
\end{pgfscope}%
\begin{pgfscope}%
\pgfsys@transformshift{6.564108in}{17.510514in}%
\pgfsys@useobject{currentmarker}{}%
\end{pgfscope}%
\begin{pgfscope}%
\pgfsys@transformshift{6.687330in}{17.509144in}%
\pgfsys@useobject{currentmarker}{}%
\end{pgfscope}%
\begin{pgfscope}%
\pgfsys@transformshift{6.810553in}{17.507760in}%
\pgfsys@useobject{currentmarker}{}%
\end{pgfscope}%
\begin{pgfscope}%
\pgfsys@transformshift{6.933776in}{17.506375in}%
\pgfsys@useobject{currentmarker}{}%
\end{pgfscope}%
\begin{pgfscope}%
\pgfsys@transformshift{7.056999in}{17.505002in}%
\pgfsys@useobject{currentmarker}{}%
\end{pgfscope}%
\begin{pgfscope}%
\pgfsys@transformshift{7.180222in}{17.503608in}%
\pgfsys@useobject{currentmarker}{}%
\end{pgfscope}%
\begin{pgfscope}%
\pgfsys@transformshift{7.303445in}{17.502202in}%
\pgfsys@useobject{currentmarker}{}%
\end{pgfscope}%
\end{pgfscope}%
\begin{pgfscope}%
\pgfsetrectcap%
\pgfsetmiterjoin%
\pgfsetlinewidth{0.803000pt}%
\definecolor{currentstroke}{rgb}{1.000000,1.000000,1.000000}%
\pgfsetstrokecolor{currentstroke}%
\pgfsetdash{}{0pt}%
\pgfpathmoveto{\pgfqpoint{2.125000in}{16.722093in}}%
\pgfpathlineto{\pgfqpoint{2.125000in}{17.600000in}}%
\pgfusepath{stroke}%
\end{pgfscope}%
\begin{pgfscope}%
\pgfsetrectcap%
\pgfsetmiterjoin%
\pgfsetlinewidth{0.803000pt}%
\definecolor{currentstroke}{rgb}{1.000000,1.000000,1.000000}%
\pgfsetstrokecolor{currentstroke}%
\pgfsetdash{}{0pt}%
\pgfpathmoveto{\pgfqpoint{7.614583in}{16.722093in}}%
\pgfpathlineto{\pgfqpoint{7.614583in}{17.600000in}}%
\pgfusepath{stroke}%
\end{pgfscope}%
\begin{pgfscope}%
\pgfsetrectcap%
\pgfsetmiterjoin%
\pgfsetlinewidth{0.803000pt}%
\definecolor{currentstroke}{rgb}{1.000000,1.000000,1.000000}%
\pgfsetstrokecolor{currentstroke}%
\pgfsetdash{}{0pt}%
\pgfpathmoveto{\pgfqpoint{2.125000in}{16.722093in}}%
\pgfpathlineto{\pgfqpoint{7.614583in}{16.722093in}}%
\pgfusepath{stroke}%
\end{pgfscope}%
\begin{pgfscope}%
\pgfsetrectcap%
\pgfsetmiterjoin%
\pgfsetlinewidth{0.803000pt}%
\definecolor{currentstroke}{rgb}{1.000000,1.000000,1.000000}%
\pgfsetstrokecolor{currentstroke}%
\pgfsetdash{}{0pt}%
\pgfpathmoveto{\pgfqpoint{2.125000in}{17.600000in}}%
\pgfpathlineto{\pgfqpoint{7.614583in}{17.600000in}}%
\pgfusepath{stroke}%
\end{pgfscope}%
\begin{pgfscope}%
\definecolor{textcolor}{rgb}{0.150000,0.150000,0.150000}%
\pgfsetstrokecolor{textcolor}%
\pgfsetfillcolor{textcolor}%
\pgftext[x=4.869792in,y=17.683333in,,base]{\color{textcolor}\rmfamily\fontsize{16.800000}{20.160000}\selectfont Autocorrelation}%
\end{pgfscope}%
\begin{pgfscope}%
\pgfsetbuttcap%
\pgfsetmiterjoin%
\definecolor{currentfill}{rgb}{0.917647,0.917647,0.949020}%
\pgfsetfillcolor{currentfill}%
\pgfsetlinewidth{0.000000pt}%
\definecolor{currentstroke}{rgb}{0.000000,0.000000,0.000000}%
\pgfsetstrokecolor{currentstroke}%
\pgfsetstrokeopacity{0.000000}%
\pgfsetdash{}{0pt}%
\pgfpathmoveto{\pgfqpoint{9.810417in}{16.722093in}}%
\pgfpathlineto{\pgfqpoint{15.300000in}{16.722093in}}%
\pgfpathlineto{\pgfqpoint{15.300000in}{17.600000in}}%
\pgfpathlineto{\pgfqpoint{9.810417in}{17.600000in}}%
\pgfpathclose%
\pgfusepath{fill}%
\end{pgfscope}%
\begin{pgfscope}%
\pgfpathrectangle{\pgfqpoint{9.810417in}{16.722093in}}{\pgfqpoint{5.489583in}{0.877907in}}%
\pgfusepath{clip}%
\pgfsetroundcap%
\pgfsetroundjoin%
\pgfsetlinewidth{0.803000pt}%
\definecolor{currentstroke}{rgb}{1.000000,1.000000,1.000000}%
\pgfsetstrokecolor{currentstroke}%
\pgfsetdash{}{0pt}%
\pgfpathmoveto{\pgfqpoint{10.059943in}{16.722093in}}%
\pgfpathlineto{\pgfqpoint{10.059943in}{17.600000in}}%
\pgfusepath{stroke}%
\end{pgfscope}%
\begin{pgfscope}%
\definecolor{textcolor}{rgb}{0.150000,0.150000,0.150000}%
\pgfsetstrokecolor{textcolor}%
\pgfsetfillcolor{textcolor}%
\pgftext[x=10.059943in,y=16.624871in,,top]{\color{textcolor}\rmfamily\fontsize{14.000000}{16.800000}\selectfont 0}%
\end{pgfscope}%
\begin{pgfscope}%
\pgfpathrectangle{\pgfqpoint{9.810417in}{16.722093in}}{\pgfqpoint{5.489583in}{0.877907in}}%
\pgfusepath{clip}%
\pgfsetroundcap%
\pgfsetroundjoin%
\pgfsetlinewidth{0.803000pt}%
\definecolor{currentstroke}{rgb}{1.000000,1.000000,1.000000}%
\pgfsetstrokecolor{currentstroke}%
\pgfsetdash{}{0pt}%
\pgfpathmoveto{\pgfqpoint{10.676058in}{16.722093in}}%
\pgfpathlineto{\pgfqpoint{10.676058in}{17.600000in}}%
\pgfusepath{stroke}%
\end{pgfscope}%
\begin{pgfscope}%
\definecolor{textcolor}{rgb}{0.150000,0.150000,0.150000}%
\pgfsetstrokecolor{textcolor}%
\pgfsetfillcolor{textcolor}%
\pgftext[x=10.676058in,y=16.624871in,,top]{\color{textcolor}\rmfamily\fontsize{14.000000}{16.800000}\selectfont 5}%
\end{pgfscope}%
\begin{pgfscope}%
\pgfpathrectangle{\pgfqpoint{9.810417in}{16.722093in}}{\pgfqpoint{5.489583in}{0.877907in}}%
\pgfusepath{clip}%
\pgfsetroundcap%
\pgfsetroundjoin%
\pgfsetlinewidth{0.803000pt}%
\definecolor{currentstroke}{rgb}{1.000000,1.000000,1.000000}%
\pgfsetstrokecolor{currentstroke}%
\pgfsetdash{}{0pt}%
\pgfpathmoveto{\pgfqpoint{11.292173in}{16.722093in}}%
\pgfpathlineto{\pgfqpoint{11.292173in}{17.600000in}}%
\pgfusepath{stroke}%
\end{pgfscope}%
\begin{pgfscope}%
\definecolor{textcolor}{rgb}{0.150000,0.150000,0.150000}%
\pgfsetstrokecolor{textcolor}%
\pgfsetfillcolor{textcolor}%
\pgftext[x=11.292173in,y=16.624871in,,top]{\color{textcolor}\rmfamily\fontsize{14.000000}{16.800000}\selectfont 10}%
\end{pgfscope}%
\begin{pgfscope}%
\pgfpathrectangle{\pgfqpoint{9.810417in}{16.722093in}}{\pgfqpoint{5.489583in}{0.877907in}}%
\pgfusepath{clip}%
\pgfsetroundcap%
\pgfsetroundjoin%
\pgfsetlinewidth{0.803000pt}%
\definecolor{currentstroke}{rgb}{1.000000,1.000000,1.000000}%
\pgfsetstrokecolor{currentstroke}%
\pgfsetdash{}{0pt}%
\pgfpathmoveto{\pgfqpoint{11.908288in}{16.722093in}}%
\pgfpathlineto{\pgfqpoint{11.908288in}{17.600000in}}%
\pgfusepath{stroke}%
\end{pgfscope}%
\begin{pgfscope}%
\definecolor{textcolor}{rgb}{0.150000,0.150000,0.150000}%
\pgfsetstrokecolor{textcolor}%
\pgfsetfillcolor{textcolor}%
\pgftext[x=11.908288in,y=16.624871in,,top]{\color{textcolor}\rmfamily\fontsize{14.000000}{16.800000}\selectfont 15}%
\end{pgfscope}%
\begin{pgfscope}%
\pgfpathrectangle{\pgfqpoint{9.810417in}{16.722093in}}{\pgfqpoint{5.489583in}{0.877907in}}%
\pgfusepath{clip}%
\pgfsetroundcap%
\pgfsetroundjoin%
\pgfsetlinewidth{0.803000pt}%
\definecolor{currentstroke}{rgb}{1.000000,1.000000,1.000000}%
\pgfsetstrokecolor{currentstroke}%
\pgfsetdash{}{0pt}%
\pgfpathmoveto{\pgfqpoint{12.524403in}{16.722093in}}%
\pgfpathlineto{\pgfqpoint{12.524403in}{17.600000in}}%
\pgfusepath{stroke}%
\end{pgfscope}%
\begin{pgfscope}%
\definecolor{textcolor}{rgb}{0.150000,0.150000,0.150000}%
\pgfsetstrokecolor{textcolor}%
\pgfsetfillcolor{textcolor}%
\pgftext[x=12.524403in,y=16.624871in,,top]{\color{textcolor}\rmfamily\fontsize{14.000000}{16.800000}\selectfont 20}%
\end{pgfscope}%
\begin{pgfscope}%
\pgfpathrectangle{\pgfqpoint{9.810417in}{16.722093in}}{\pgfqpoint{5.489583in}{0.877907in}}%
\pgfusepath{clip}%
\pgfsetroundcap%
\pgfsetroundjoin%
\pgfsetlinewidth{0.803000pt}%
\definecolor{currentstroke}{rgb}{1.000000,1.000000,1.000000}%
\pgfsetstrokecolor{currentstroke}%
\pgfsetdash{}{0pt}%
\pgfpathmoveto{\pgfqpoint{13.140517in}{16.722093in}}%
\pgfpathlineto{\pgfqpoint{13.140517in}{17.600000in}}%
\pgfusepath{stroke}%
\end{pgfscope}%
\begin{pgfscope}%
\definecolor{textcolor}{rgb}{0.150000,0.150000,0.150000}%
\pgfsetstrokecolor{textcolor}%
\pgfsetfillcolor{textcolor}%
\pgftext[x=13.140517in,y=16.624871in,,top]{\color{textcolor}\rmfamily\fontsize{14.000000}{16.800000}\selectfont 25}%
\end{pgfscope}%
\begin{pgfscope}%
\pgfpathrectangle{\pgfqpoint{9.810417in}{16.722093in}}{\pgfqpoint{5.489583in}{0.877907in}}%
\pgfusepath{clip}%
\pgfsetroundcap%
\pgfsetroundjoin%
\pgfsetlinewidth{0.803000pt}%
\definecolor{currentstroke}{rgb}{1.000000,1.000000,1.000000}%
\pgfsetstrokecolor{currentstroke}%
\pgfsetdash{}{0pt}%
\pgfpathmoveto{\pgfqpoint{13.756632in}{16.722093in}}%
\pgfpathlineto{\pgfqpoint{13.756632in}{17.600000in}}%
\pgfusepath{stroke}%
\end{pgfscope}%
\begin{pgfscope}%
\definecolor{textcolor}{rgb}{0.150000,0.150000,0.150000}%
\pgfsetstrokecolor{textcolor}%
\pgfsetfillcolor{textcolor}%
\pgftext[x=13.756632in,y=16.624871in,,top]{\color{textcolor}\rmfamily\fontsize{14.000000}{16.800000}\selectfont 30}%
\end{pgfscope}%
\begin{pgfscope}%
\pgfpathrectangle{\pgfqpoint{9.810417in}{16.722093in}}{\pgfqpoint{5.489583in}{0.877907in}}%
\pgfusepath{clip}%
\pgfsetroundcap%
\pgfsetroundjoin%
\pgfsetlinewidth{0.803000pt}%
\definecolor{currentstroke}{rgb}{1.000000,1.000000,1.000000}%
\pgfsetstrokecolor{currentstroke}%
\pgfsetdash{}{0pt}%
\pgfpathmoveto{\pgfqpoint{14.372747in}{16.722093in}}%
\pgfpathlineto{\pgfqpoint{14.372747in}{17.600000in}}%
\pgfusepath{stroke}%
\end{pgfscope}%
\begin{pgfscope}%
\definecolor{textcolor}{rgb}{0.150000,0.150000,0.150000}%
\pgfsetstrokecolor{textcolor}%
\pgfsetfillcolor{textcolor}%
\pgftext[x=14.372747in,y=16.624871in,,top]{\color{textcolor}\rmfamily\fontsize{14.000000}{16.800000}\selectfont 35}%
\end{pgfscope}%
\begin{pgfscope}%
\pgfpathrectangle{\pgfqpoint{9.810417in}{16.722093in}}{\pgfqpoint{5.489583in}{0.877907in}}%
\pgfusepath{clip}%
\pgfsetroundcap%
\pgfsetroundjoin%
\pgfsetlinewidth{0.803000pt}%
\definecolor{currentstroke}{rgb}{1.000000,1.000000,1.000000}%
\pgfsetstrokecolor{currentstroke}%
\pgfsetdash{}{0pt}%
\pgfpathmoveto{\pgfqpoint{14.988862in}{16.722093in}}%
\pgfpathlineto{\pgfqpoint{14.988862in}{17.600000in}}%
\pgfusepath{stroke}%
\end{pgfscope}%
\begin{pgfscope}%
\definecolor{textcolor}{rgb}{0.150000,0.150000,0.150000}%
\pgfsetstrokecolor{textcolor}%
\pgfsetfillcolor{textcolor}%
\pgftext[x=14.988862in,y=16.624871in,,top]{\color{textcolor}\rmfamily\fontsize{14.000000}{16.800000}\selectfont 40}%
\end{pgfscope}%
\begin{pgfscope}%
\pgfpathrectangle{\pgfqpoint{9.810417in}{16.722093in}}{\pgfqpoint{5.489583in}{0.877907in}}%
\pgfusepath{clip}%
\pgfsetroundcap%
\pgfsetroundjoin%
\pgfsetlinewidth{0.803000pt}%
\definecolor{currentstroke}{rgb}{1.000000,1.000000,1.000000}%
\pgfsetstrokecolor{currentstroke}%
\pgfsetdash{}{0pt}%
\pgfpathmoveto{\pgfqpoint{9.810417in}{16.800332in}}%
\pgfpathlineto{\pgfqpoint{15.300000in}{16.800332in}}%
\pgfusepath{stroke}%
\end{pgfscope}%
\begin{pgfscope}%
\definecolor{textcolor}{rgb}{0.150000,0.150000,0.150000}%
\pgfsetstrokecolor{textcolor}%
\pgfsetfillcolor{textcolor}%
\pgftext[x=9.589483in,y=16.726466in,left,base]{\color{textcolor}\rmfamily\fontsize{14.000000}{16.800000}\selectfont 0}%
\end{pgfscope}%
\begin{pgfscope}%
\pgfpathrectangle{\pgfqpoint{9.810417in}{16.722093in}}{\pgfqpoint{5.489583in}{0.877907in}}%
\pgfusepath{clip}%
\pgfsetroundcap%
\pgfsetroundjoin%
\pgfsetlinewidth{0.803000pt}%
\definecolor{currentstroke}{rgb}{1.000000,1.000000,1.000000}%
\pgfsetstrokecolor{currentstroke}%
\pgfsetdash{}{0pt}%
\pgfpathmoveto{\pgfqpoint{9.810417in}{17.560095in}}%
\pgfpathlineto{\pgfqpoint{15.300000in}{17.560095in}}%
\pgfusepath{stroke}%
\end{pgfscope}%
\begin{pgfscope}%
\definecolor{textcolor}{rgb}{0.150000,0.150000,0.150000}%
\pgfsetstrokecolor{textcolor}%
\pgfsetfillcolor{textcolor}%
\pgftext[x=9.589483in,y=17.486229in,left,base]{\color{textcolor}\rmfamily\fontsize{14.000000}{16.800000}\selectfont 1}%
\end{pgfscope}%
\begin{pgfscope}%
\pgfpathrectangle{\pgfqpoint{9.810417in}{16.722093in}}{\pgfqpoint{5.489583in}{0.877907in}}%
\pgfusepath{clip}%
\pgfsetbuttcap%
\pgfsetroundjoin%
\definecolor{currentfill}{rgb}{0.121569,0.466667,0.705882}%
\pgfsetfillcolor{currentfill}%
\pgfsetfillopacity{0.250000}%
\pgfsetlinewidth{1.003750pt}%
\definecolor{currentstroke}{rgb}{1.000000,1.000000,1.000000}%
\pgfsetstrokecolor{currentstroke}%
\pgfsetstrokeopacity{0.250000}%
\pgfsetdash{}{0pt}%
\pgfpathmoveto{\pgfqpoint{10.121555in}{16.838665in}}%
\pgfpathlineto{\pgfqpoint{10.121555in}{16.761998in}}%
\pgfpathlineto{\pgfqpoint{10.306389in}{16.761998in}}%
\pgfpathlineto{\pgfqpoint{10.429612in}{16.761998in}}%
\pgfpathlineto{\pgfqpoint{10.552835in}{16.761998in}}%
\pgfpathlineto{\pgfqpoint{10.676058in}{16.761998in}}%
\pgfpathlineto{\pgfqpoint{10.799281in}{16.761998in}}%
\pgfpathlineto{\pgfqpoint{10.922504in}{16.761998in}}%
\pgfpathlineto{\pgfqpoint{11.045727in}{16.761998in}}%
\pgfpathlineto{\pgfqpoint{11.168950in}{16.761998in}}%
\pgfpathlineto{\pgfqpoint{11.292173in}{16.761998in}}%
\pgfpathlineto{\pgfqpoint{11.415396in}{16.761998in}}%
\pgfpathlineto{\pgfqpoint{11.538619in}{16.761998in}}%
\pgfpathlineto{\pgfqpoint{11.661842in}{16.761998in}}%
\pgfpathlineto{\pgfqpoint{11.785065in}{16.761998in}}%
\pgfpathlineto{\pgfqpoint{11.908288in}{16.761998in}}%
\pgfpathlineto{\pgfqpoint{12.031511in}{16.761998in}}%
\pgfpathlineto{\pgfqpoint{12.154734in}{16.761998in}}%
\pgfpathlineto{\pgfqpoint{12.277957in}{16.761998in}}%
\pgfpathlineto{\pgfqpoint{12.401180in}{16.761998in}}%
\pgfpathlineto{\pgfqpoint{12.524403in}{16.761998in}}%
\pgfpathlineto{\pgfqpoint{12.647626in}{16.761998in}}%
\pgfpathlineto{\pgfqpoint{12.770849in}{16.761998in}}%
\pgfpathlineto{\pgfqpoint{12.894072in}{16.761998in}}%
\pgfpathlineto{\pgfqpoint{13.017294in}{16.761998in}}%
\pgfpathlineto{\pgfqpoint{13.140517in}{16.761998in}}%
\pgfpathlineto{\pgfqpoint{13.263740in}{16.761998in}}%
\pgfpathlineto{\pgfqpoint{13.386963in}{16.761998in}}%
\pgfpathlineto{\pgfqpoint{13.510186in}{16.761998in}}%
\pgfpathlineto{\pgfqpoint{13.633409in}{16.761998in}}%
\pgfpathlineto{\pgfqpoint{13.756632in}{16.761998in}}%
\pgfpathlineto{\pgfqpoint{13.879855in}{16.761998in}}%
\pgfpathlineto{\pgfqpoint{14.003078in}{16.761998in}}%
\pgfpathlineto{\pgfqpoint{14.126301in}{16.761998in}}%
\pgfpathlineto{\pgfqpoint{14.249524in}{16.761998in}}%
\pgfpathlineto{\pgfqpoint{14.372747in}{16.761998in}}%
\pgfpathlineto{\pgfqpoint{14.495970in}{16.761998in}}%
\pgfpathlineto{\pgfqpoint{14.619193in}{16.761998in}}%
\pgfpathlineto{\pgfqpoint{14.742416in}{16.761998in}}%
\pgfpathlineto{\pgfqpoint{14.865639in}{16.761998in}}%
\pgfpathlineto{\pgfqpoint{15.050473in}{16.761998in}}%
\pgfpathlineto{\pgfqpoint{15.050473in}{16.838665in}}%
\pgfpathlineto{\pgfqpoint{15.050473in}{16.838665in}}%
\pgfpathlineto{\pgfqpoint{14.865639in}{16.838665in}}%
\pgfpathlineto{\pgfqpoint{14.742416in}{16.838665in}}%
\pgfpathlineto{\pgfqpoint{14.619193in}{16.838665in}}%
\pgfpathlineto{\pgfqpoint{14.495970in}{16.838665in}}%
\pgfpathlineto{\pgfqpoint{14.372747in}{16.838665in}}%
\pgfpathlineto{\pgfqpoint{14.249524in}{16.838665in}}%
\pgfpathlineto{\pgfqpoint{14.126301in}{16.838665in}}%
\pgfpathlineto{\pgfqpoint{14.003078in}{16.838665in}}%
\pgfpathlineto{\pgfqpoint{13.879855in}{16.838665in}}%
\pgfpathlineto{\pgfqpoint{13.756632in}{16.838665in}}%
\pgfpathlineto{\pgfqpoint{13.633409in}{16.838665in}}%
\pgfpathlineto{\pgfqpoint{13.510186in}{16.838665in}}%
\pgfpathlineto{\pgfqpoint{13.386963in}{16.838665in}}%
\pgfpathlineto{\pgfqpoint{13.263740in}{16.838665in}}%
\pgfpathlineto{\pgfqpoint{13.140517in}{16.838665in}}%
\pgfpathlineto{\pgfqpoint{13.017294in}{16.838665in}}%
\pgfpathlineto{\pgfqpoint{12.894072in}{16.838665in}}%
\pgfpathlineto{\pgfqpoint{12.770849in}{16.838665in}}%
\pgfpathlineto{\pgfqpoint{12.647626in}{16.838665in}}%
\pgfpathlineto{\pgfqpoint{12.524403in}{16.838665in}}%
\pgfpathlineto{\pgfqpoint{12.401180in}{16.838665in}}%
\pgfpathlineto{\pgfqpoint{12.277957in}{16.838665in}}%
\pgfpathlineto{\pgfqpoint{12.154734in}{16.838665in}}%
\pgfpathlineto{\pgfqpoint{12.031511in}{16.838665in}}%
\pgfpathlineto{\pgfqpoint{11.908288in}{16.838665in}}%
\pgfpathlineto{\pgfqpoint{11.785065in}{16.838665in}}%
\pgfpathlineto{\pgfqpoint{11.661842in}{16.838665in}}%
\pgfpathlineto{\pgfqpoint{11.538619in}{16.838665in}}%
\pgfpathlineto{\pgfqpoint{11.415396in}{16.838665in}}%
\pgfpathlineto{\pgfqpoint{11.292173in}{16.838665in}}%
\pgfpathlineto{\pgfqpoint{11.168950in}{16.838665in}}%
\pgfpathlineto{\pgfqpoint{11.045727in}{16.838665in}}%
\pgfpathlineto{\pgfqpoint{10.922504in}{16.838665in}}%
\pgfpathlineto{\pgfqpoint{10.799281in}{16.838665in}}%
\pgfpathlineto{\pgfqpoint{10.676058in}{16.838665in}}%
\pgfpathlineto{\pgfqpoint{10.552835in}{16.838665in}}%
\pgfpathlineto{\pgfqpoint{10.429612in}{16.838665in}}%
\pgfpathlineto{\pgfqpoint{10.306389in}{16.838665in}}%
\pgfpathlineto{\pgfqpoint{10.121555in}{16.838665in}}%
\pgfpathclose%
\pgfusepath{stroke,fill}%
\end{pgfscope}%
\begin{pgfscope}%
\pgfpathrectangle{\pgfqpoint{9.810417in}{16.722093in}}{\pgfqpoint{5.489583in}{0.877907in}}%
\pgfusepath{clip}%
\pgfsetbuttcap%
\pgfsetroundjoin%
\pgfsetlinewidth{1.505625pt}%
\definecolor{currentstroke}{rgb}{0.000000,0.000000,0.000000}%
\pgfsetstrokecolor{currentstroke}%
\pgfsetdash{}{0pt}%
\pgfpathmoveto{\pgfqpoint{10.059943in}{16.800332in}}%
\pgfpathlineto{\pgfqpoint{10.059943in}{17.560095in}}%
\pgfusepath{stroke}%
\end{pgfscope}%
\begin{pgfscope}%
\pgfpathrectangle{\pgfqpoint{9.810417in}{16.722093in}}{\pgfqpoint{5.489583in}{0.877907in}}%
\pgfusepath{clip}%
\pgfsetbuttcap%
\pgfsetroundjoin%
\pgfsetlinewidth{1.505625pt}%
\definecolor{currentstroke}{rgb}{0.000000,0.000000,0.000000}%
\pgfsetstrokecolor{currentstroke}%
\pgfsetdash{}{0pt}%
\pgfpathmoveto{\pgfqpoint{10.183166in}{16.800332in}}%
\pgfpathlineto{\pgfqpoint{10.183166in}{17.558539in}}%
\pgfusepath{stroke}%
\end{pgfscope}%
\begin{pgfscope}%
\pgfpathrectangle{\pgfqpoint{9.810417in}{16.722093in}}{\pgfqpoint{5.489583in}{0.877907in}}%
\pgfusepath{clip}%
\pgfsetbuttcap%
\pgfsetroundjoin%
\pgfsetlinewidth{1.505625pt}%
\definecolor{currentstroke}{rgb}{0.000000,0.000000,0.000000}%
\pgfsetstrokecolor{currentstroke}%
\pgfsetdash{}{0pt}%
\pgfpathmoveto{\pgfqpoint{10.306389in}{16.800332in}}%
\pgfpathlineto{\pgfqpoint{10.306389in}{16.807931in}}%
\pgfusepath{stroke}%
\end{pgfscope}%
\begin{pgfscope}%
\pgfpathrectangle{\pgfqpoint{9.810417in}{16.722093in}}{\pgfqpoint{5.489583in}{0.877907in}}%
\pgfusepath{clip}%
\pgfsetbuttcap%
\pgfsetroundjoin%
\pgfsetlinewidth{1.505625pt}%
\definecolor{currentstroke}{rgb}{0.000000,0.000000,0.000000}%
\pgfsetstrokecolor{currentstroke}%
\pgfsetdash{}{0pt}%
\pgfpathmoveto{\pgfqpoint{10.429612in}{16.800332in}}%
\pgfpathlineto{\pgfqpoint{10.429612in}{16.788182in}}%
\pgfusepath{stroke}%
\end{pgfscope}%
\begin{pgfscope}%
\pgfpathrectangle{\pgfqpoint{9.810417in}{16.722093in}}{\pgfqpoint{5.489583in}{0.877907in}}%
\pgfusepath{clip}%
\pgfsetbuttcap%
\pgfsetroundjoin%
\pgfsetlinewidth{1.505625pt}%
\definecolor{currentstroke}{rgb}{0.000000,0.000000,0.000000}%
\pgfsetstrokecolor{currentstroke}%
\pgfsetdash{}{0pt}%
\pgfpathmoveto{\pgfqpoint{10.552835in}{16.800332in}}%
\pgfpathlineto{\pgfqpoint{10.552835in}{16.792765in}}%
\pgfusepath{stroke}%
\end{pgfscope}%
\begin{pgfscope}%
\pgfpathrectangle{\pgfqpoint{9.810417in}{16.722093in}}{\pgfqpoint{5.489583in}{0.877907in}}%
\pgfusepath{clip}%
\pgfsetbuttcap%
\pgfsetroundjoin%
\pgfsetlinewidth{1.505625pt}%
\definecolor{currentstroke}{rgb}{0.000000,0.000000,0.000000}%
\pgfsetstrokecolor{currentstroke}%
\pgfsetdash{}{0pt}%
\pgfpathmoveto{\pgfqpoint{10.676058in}{16.800332in}}%
\pgfpathlineto{\pgfqpoint{10.676058in}{16.814088in}}%
\pgfusepath{stroke}%
\end{pgfscope}%
\begin{pgfscope}%
\pgfpathrectangle{\pgfqpoint{9.810417in}{16.722093in}}{\pgfqpoint{5.489583in}{0.877907in}}%
\pgfusepath{clip}%
\pgfsetbuttcap%
\pgfsetroundjoin%
\pgfsetlinewidth{1.505625pt}%
\definecolor{currentstroke}{rgb}{0.000000,0.000000,0.000000}%
\pgfsetstrokecolor{currentstroke}%
\pgfsetdash{}{0pt}%
\pgfpathmoveto{\pgfqpoint{10.799281in}{16.800332in}}%
\pgfpathlineto{\pgfqpoint{10.799281in}{16.804443in}}%
\pgfusepath{stroke}%
\end{pgfscope}%
\begin{pgfscope}%
\pgfpathrectangle{\pgfqpoint{9.810417in}{16.722093in}}{\pgfqpoint{5.489583in}{0.877907in}}%
\pgfusepath{clip}%
\pgfsetbuttcap%
\pgfsetroundjoin%
\pgfsetlinewidth{1.505625pt}%
\definecolor{currentstroke}{rgb}{0.000000,0.000000,0.000000}%
\pgfsetstrokecolor{currentstroke}%
\pgfsetdash{}{0pt}%
\pgfpathmoveto{\pgfqpoint{10.922504in}{16.800332in}}%
\pgfpathlineto{\pgfqpoint{10.922504in}{16.785665in}}%
\pgfusepath{stroke}%
\end{pgfscope}%
\begin{pgfscope}%
\pgfpathrectangle{\pgfqpoint{9.810417in}{16.722093in}}{\pgfqpoint{5.489583in}{0.877907in}}%
\pgfusepath{clip}%
\pgfsetbuttcap%
\pgfsetroundjoin%
\pgfsetlinewidth{1.505625pt}%
\definecolor{currentstroke}{rgb}{0.000000,0.000000,0.000000}%
\pgfsetstrokecolor{currentstroke}%
\pgfsetdash{}{0pt}%
\pgfpathmoveto{\pgfqpoint{11.045727in}{16.800332in}}%
\pgfpathlineto{\pgfqpoint{11.045727in}{16.796228in}}%
\pgfusepath{stroke}%
\end{pgfscope}%
\begin{pgfscope}%
\pgfpathrectangle{\pgfqpoint{9.810417in}{16.722093in}}{\pgfqpoint{5.489583in}{0.877907in}}%
\pgfusepath{clip}%
\pgfsetbuttcap%
\pgfsetroundjoin%
\pgfsetlinewidth{1.505625pt}%
\definecolor{currentstroke}{rgb}{0.000000,0.000000,0.000000}%
\pgfsetstrokecolor{currentstroke}%
\pgfsetdash{}{0pt}%
\pgfpathmoveto{\pgfqpoint{11.168950in}{16.800332in}}%
\pgfpathlineto{\pgfqpoint{11.168950in}{16.785658in}}%
\pgfusepath{stroke}%
\end{pgfscope}%
\begin{pgfscope}%
\pgfpathrectangle{\pgfqpoint{9.810417in}{16.722093in}}{\pgfqpoint{5.489583in}{0.877907in}}%
\pgfusepath{clip}%
\pgfsetbuttcap%
\pgfsetroundjoin%
\pgfsetlinewidth{1.505625pt}%
\definecolor{currentstroke}{rgb}{0.000000,0.000000,0.000000}%
\pgfsetstrokecolor{currentstroke}%
\pgfsetdash{}{0pt}%
\pgfpathmoveto{\pgfqpoint{11.292173in}{16.800332in}}%
\pgfpathlineto{\pgfqpoint{11.292173in}{16.810830in}}%
\pgfusepath{stroke}%
\end{pgfscope}%
\begin{pgfscope}%
\pgfpathrectangle{\pgfqpoint{9.810417in}{16.722093in}}{\pgfqpoint{5.489583in}{0.877907in}}%
\pgfusepath{clip}%
\pgfsetbuttcap%
\pgfsetroundjoin%
\pgfsetlinewidth{1.505625pt}%
\definecolor{currentstroke}{rgb}{0.000000,0.000000,0.000000}%
\pgfsetstrokecolor{currentstroke}%
\pgfsetdash{}{0pt}%
\pgfpathmoveto{\pgfqpoint{11.415396in}{16.800332in}}%
\pgfpathlineto{\pgfqpoint{11.415396in}{16.810565in}}%
\pgfusepath{stroke}%
\end{pgfscope}%
\begin{pgfscope}%
\pgfpathrectangle{\pgfqpoint{9.810417in}{16.722093in}}{\pgfqpoint{5.489583in}{0.877907in}}%
\pgfusepath{clip}%
\pgfsetbuttcap%
\pgfsetroundjoin%
\pgfsetlinewidth{1.505625pt}%
\definecolor{currentstroke}{rgb}{0.000000,0.000000,0.000000}%
\pgfsetstrokecolor{currentstroke}%
\pgfsetdash{}{0pt}%
\pgfpathmoveto{\pgfqpoint{11.538619in}{16.800332in}}%
\pgfpathlineto{\pgfqpoint{11.538619in}{16.802471in}}%
\pgfusepath{stroke}%
\end{pgfscope}%
\begin{pgfscope}%
\pgfpathrectangle{\pgfqpoint{9.810417in}{16.722093in}}{\pgfqpoint{5.489583in}{0.877907in}}%
\pgfusepath{clip}%
\pgfsetbuttcap%
\pgfsetroundjoin%
\pgfsetlinewidth{1.505625pt}%
\definecolor{currentstroke}{rgb}{0.000000,0.000000,0.000000}%
\pgfsetstrokecolor{currentstroke}%
\pgfsetdash{}{0pt}%
\pgfpathmoveto{\pgfqpoint{11.661842in}{16.800332in}}%
\pgfpathlineto{\pgfqpoint{11.661842in}{16.806130in}}%
\pgfusepath{stroke}%
\end{pgfscope}%
\begin{pgfscope}%
\pgfpathrectangle{\pgfqpoint{9.810417in}{16.722093in}}{\pgfqpoint{5.489583in}{0.877907in}}%
\pgfusepath{clip}%
\pgfsetbuttcap%
\pgfsetroundjoin%
\pgfsetlinewidth{1.505625pt}%
\definecolor{currentstroke}{rgb}{0.000000,0.000000,0.000000}%
\pgfsetstrokecolor{currentstroke}%
\pgfsetdash{}{0pt}%
\pgfpathmoveto{\pgfqpoint{11.785065in}{16.800332in}}%
\pgfpathlineto{\pgfqpoint{11.785065in}{16.789038in}}%
\pgfusepath{stroke}%
\end{pgfscope}%
\begin{pgfscope}%
\pgfpathrectangle{\pgfqpoint{9.810417in}{16.722093in}}{\pgfqpoint{5.489583in}{0.877907in}}%
\pgfusepath{clip}%
\pgfsetbuttcap%
\pgfsetroundjoin%
\pgfsetlinewidth{1.505625pt}%
\definecolor{currentstroke}{rgb}{0.000000,0.000000,0.000000}%
\pgfsetstrokecolor{currentstroke}%
\pgfsetdash{}{0pt}%
\pgfpathmoveto{\pgfqpoint{11.908288in}{16.800332in}}%
\pgfpathlineto{\pgfqpoint{11.908288in}{16.803164in}}%
\pgfusepath{stroke}%
\end{pgfscope}%
\begin{pgfscope}%
\pgfpathrectangle{\pgfqpoint{9.810417in}{16.722093in}}{\pgfqpoint{5.489583in}{0.877907in}}%
\pgfusepath{clip}%
\pgfsetbuttcap%
\pgfsetroundjoin%
\pgfsetlinewidth{1.505625pt}%
\definecolor{currentstroke}{rgb}{0.000000,0.000000,0.000000}%
\pgfsetstrokecolor{currentstroke}%
\pgfsetdash{}{0pt}%
\pgfpathmoveto{\pgfqpoint{12.031511in}{16.800332in}}%
\pgfpathlineto{\pgfqpoint{12.031511in}{16.799989in}}%
\pgfusepath{stroke}%
\end{pgfscope}%
\begin{pgfscope}%
\pgfpathrectangle{\pgfqpoint{9.810417in}{16.722093in}}{\pgfqpoint{5.489583in}{0.877907in}}%
\pgfusepath{clip}%
\pgfsetbuttcap%
\pgfsetroundjoin%
\pgfsetlinewidth{1.505625pt}%
\definecolor{currentstroke}{rgb}{0.000000,0.000000,0.000000}%
\pgfsetstrokecolor{currentstroke}%
\pgfsetdash{}{0pt}%
\pgfpathmoveto{\pgfqpoint{12.154734in}{16.800332in}}%
\pgfpathlineto{\pgfqpoint{12.154734in}{16.813690in}}%
\pgfusepath{stroke}%
\end{pgfscope}%
\begin{pgfscope}%
\pgfpathrectangle{\pgfqpoint{9.810417in}{16.722093in}}{\pgfqpoint{5.489583in}{0.877907in}}%
\pgfusepath{clip}%
\pgfsetbuttcap%
\pgfsetroundjoin%
\pgfsetlinewidth{1.505625pt}%
\definecolor{currentstroke}{rgb}{0.000000,0.000000,0.000000}%
\pgfsetstrokecolor{currentstroke}%
\pgfsetdash{}{0pt}%
\pgfpathmoveto{\pgfqpoint{12.277957in}{16.800332in}}%
\pgfpathlineto{\pgfqpoint{12.277957in}{16.793479in}}%
\pgfusepath{stroke}%
\end{pgfscope}%
\begin{pgfscope}%
\pgfpathrectangle{\pgfqpoint{9.810417in}{16.722093in}}{\pgfqpoint{5.489583in}{0.877907in}}%
\pgfusepath{clip}%
\pgfsetbuttcap%
\pgfsetroundjoin%
\pgfsetlinewidth{1.505625pt}%
\definecolor{currentstroke}{rgb}{0.000000,0.000000,0.000000}%
\pgfsetstrokecolor{currentstroke}%
\pgfsetdash{}{0pt}%
\pgfpathmoveto{\pgfqpoint{12.401180in}{16.800332in}}%
\pgfpathlineto{\pgfqpoint{12.401180in}{16.796786in}}%
\pgfusepath{stroke}%
\end{pgfscope}%
\begin{pgfscope}%
\pgfpathrectangle{\pgfqpoint{9.810417in}{16.722093in}}{\pgfqpoint{5.489583in}{0.877907in}}%
\pgfusepath{clip}%
\pgfsetbuttcap%
\pgfsetroundjoin%
\pgfsetlinewidth{1.505625pt}%
\definecolor{currentstroke}{rgb}{0.000000,0.000000,0.000000}%
\pgfsetstrokecolor{currentstroke}%
\pgfsetdash{}{0pt}%
\pgfpathmoveto{\pgfqpoint{12.524403in}{16.800332in}}%
\pgfpathlineto{\pgfqpoint{12.524403in}{16.787424in}}%
\pgfusepath{stroke}%
\end{pgfscope}%
\begin{pgfscope}%
\pgfpathrectangle{\pgfqpoint{9.810417in}{16.722093in}}{\pgfqpoint{5.489583in}{0.877907in}}%
\pgfusepath{clip}%
\pgfsetbuttcap%
\pgfsetroundjoin%
\pgfsetlinewidth{1.505625pt}%
\definecolor{currentstroke}{rgb}{0.000000,0.000000,0.000000}%
\pgfsetstrokecolor{currentstroke}%
\pgfsetdash{}{0pt}%
\pgfpathmoveto{\pgfqpoint{12.647626in}{16.800332in}}%
\pgfpathlineto{\pgfqpoint{12.647626in}{16.795431in}}%
\pgfusepath{stroke}%
\end{pgfscope}%
\begin{pgfscope}%
\pgfpathrectangle{\pgfqpoint{9.810417in}{16.722093in}}{\pgfqpoint{5.489583in}{0.877907in}}%
\pgfusepath{clip}%
\pgfsetbuttcap%
\pgfsetroundjoin%
\pgfsetlinewidth{1.505625pt}%
\definecolor{currentstroke}{rgb}{0.000000,0.000000,0.000000}%
\pgfsetstrokecolor{currentstroke}%
\pgfsetdash{}{0pt}%
\pgfpathmoveto{\pgfqpoint{12.770849in}{16.800332in}}%
\pgfpathlineto{\pgfqpoint{12.770849in}{16.806691in}}%
\pgfusepath{stroke}%
\end{pgfscope}%
\begin{pgfscope}%
\pgfpathrectangle{\pgfqpoint{9.810417in}{16.722093in}}{\pgfqpoint{5.489583in}{0.877907in}}%
\pgfusepath{clip}%
\pgfsetbuttcap%
\pgfsetroundjoin%
\pgfsetlinewidth{1.505625pt}%
\definecolor{currentstroke}{rgb}{0.000000,0.000000,0.000000}%
\pgfsetstrokecolor{currentstroke}%
\pgfsetdash{}{0pt}%
\pgfpathmoveto{\pgfqpoint{12.894072in}{16.800332in}}%
\pgfpathlineto{\pgfqpoint{12.894072in}{16.813541in}}%
\pgfusepath{stroke}%
\end{pgfscope}%
\begin{pgfscope}%
\pgfpathrectangle{\pgfqpoint{9.810417in}{16.722093in}}{\pgfqpoint{5.489583in}{0.877907in}}%
\pgfusepath{clip}%
\pgfsetbuttcap%
\pgfsetroundjoin%
\pgfsetlinewidth{1.505625pt}%
\definecolor{currentstroke}{rgb}{0.000000,0.000000,0.000000}%
\pgfsetstrokecolor{currentstroke}%
\pgfsetdash{}{0pt}%
\pgfpathmoveto{\pgfqpoint{13.017294in}{16.800332in}}%
\pgfpathlineto{\pgfqpoint{13.017294in}{16.803383in}}%
\pgfusepath{stroke}%
\end{pgfscope}%
\begin{pgfscope}%
\pgfpathrectangle{\pgfqpoint{9.810417in}{16.722093in}}{\pgfqpoint{5.489583in}{0.877907in}}%
\pgfusepath{clip}%
\pgfsetbuttcap%
\pgfsetroundjoin%
\pgfsetlinewidth{1.505625pt}%
\definecolor{currentstroke}{rgb}{0.000000,0.000000,0.000000}%
\pgfsetstrokecolor{currentstroke}%
\pgfsetdash{}{0pt}%
\pgfpathmoveto{\pgfqpoint{13.140517in}{16.800332in}}%
\pgfpathlineto{\pgfqpoint{13.140517in}{16.807568in}}%
\pgfusepath{stroke}%
\end{pgfscope}%
\begin{pgfscope}%
\pgfpathrectangle{\pgfqpoint{9.810417in}{16.722093in}}{\pgfqpoint{5.489583in}{0.877907in}}%
\pgfusepath{clip}%
\pgfsetbuttcap%
\pgfsetroundjoin%
\pgfsetlinewidth{1.505625pt}%
\definecolor{currentstroke}{rgb}{0.000000,0.000000,0.000000}%
\pgfsetstrokecolor{currentstroke}%
\pgfsetdash{}{0pt}%
\pgfpathmoveto{\pgfqpoint{13.263740in}{16.800332in}}%
\pgfpathlineto{\pgfqpoint{13.263740in}{16.806899in}}%
\pgfusepath{stroke}%
\end{pgfscope}%
\begin{pgfscope}%
\pgfpathrectangle{\pgfqpoint{9.810417in}{16.722093in}}{\pgfqpoint{5.489583in}{0.877907in}}%
\pgfusepath{clip}%
\pgfsetbuttcap%
\pgfsetroundjoin%
\pgfsetlinewidth{1.505625pt}%
\definecolor{currentstroke}{rgb}{0.000000,0.000000,0.000000}%
\pgfsetstrokecolor{currentstroke}%
\pgfsetdash{}{0pt}%
\pgfpathmoveto{\pgfqpoint{13.386963in}{16.800332in}}%
\pgfpathlineto{\pgfqpoint{13.386963in}{16.790691in}}%
\pgfusepath{stroke}%
\end{pgfscope}%
\begin{pgfscope}%
\pgfpathrectangle{\pgfqpoint{9.810417in}{16.722093in}}{\pgfqpoint{5.489583in}{0.877907in}}%
\pgfusepath{clip}%
\pgfsetbuttcap%
\pgfsetroundjoin%
\pgfsetlinewidth{1.505625pt}%
\definecolor{currentstroke}{rgb}{0.000000,0.000000,0.000000}%
\pgfsetstrokecolor{currentstroke}%
\pgfsetdash{}{0pt}%
\pgfpathmoveto{\pgfqpoint{13.510186in}{16.800332in}}%
\pgfpathlineto{\pgfqpoint{13.510186in}{16.796733in}}%
\pgfusepath{stroke}%
\end{pgfscope}%
\begin{pgfscope}%
\pgfpathrectangle{\pgfqpoint{9.810417in}{16.722093in}}{\pgfqpoint{5.489583in}{0.877907in}}%
\pgfusepath{clip}%
\pgfsetbuttcap%
\pgfsetroundjoin%
\pgfsetlinewidth{1.505625pt}%
\definecolor{currentstroke}{rgb}{0.000000,0.000000,0.000000}%
\pgfsetstrokecolor{currentstroke}%
\pgfsetdash{}{0pt}%
\pgfpathmoveto{\pgfqpoint{13.633409in}{16.800332in}}%
\pgfpathlineto{\pgfqpoint{13.633409in}{16.813213in}}%
\pgfusepath{stroke}%
\end{pgfscope}%
\begin{pgfscope}%
\pgfpathrectangle{\pgfqpoint{9.810417in}{16.722093in}}{\pgfqpoint{5.489583in}{0.877907in}}%
\pgfusepath{clip}%
\pgfsetbuttcap%
\pgfsetroundjoin%
\pgfsetlinewidth{1.505625pt}%
\definecolor{currentstroke}{rgb}{0.000000,0.000000,0.000000}%
\pgfsetstrokecolor{currentstroke}%
\pgfsetdash{}{0pt}%
\pgfpathmoveto{\pgfqpoint{13.756632in}{16.800332in}}%
\pgfpathlineto{\pgfqpoint{13.756632in}{16.795599in}}%
\pgfusepath{stroke}%
\end{pgfscope}%
\begin{pgfscope}%
\pgfpathrectangle{\pgfqpoint{9.810417in}{16.722093in}}{\pgfqpoint{5.489583in}{0.877907in}}%
\pgfusepath{clip}%
\pgfsetbuttcap%
\pgfsetroundjoin%
\pgfsetlinewidth{1.505625pt}%
\definecolor{currentstroke}{rgb}{0.000000,0.000000,0.000000}%
\pgfsetstrokecolor{currentstroke}%
\pgfsetdash{}{0pt}%
\pgfpathmoveto{\pgfqpoint{13.879855in}{16.800332in}}%
\pgfpathlineto{\pgfqpoint{13.879855in}{16.809830in}}%
\pgfusepath{stroke}%
\end{pgfscope}%
\begin{pgfscope}%
\pgfpathrectangle{\pgfqpoint{9.810417in}{16.722093in}}{\pgfqpoint{5.489583in}{0.877907in}}%
\pgfusepath{clip}%
\pgfsetbuttcap%
\pgfsetroundjoin%
\pgfsetlinewidth{1.505625pt}%
\definecolor{currentstroke}{rgb}{0.000000,0.000000,0.000000}%
\pgfsetstrokecolor{currentstroke}%
\pgfsetdash{}{0pt}%
\pgfpathmoveto{\pgfqpoint{14.003078in}{16.800332in}}%
\pgfpathlineto{\pgfqpoint{14.003078in}{16.795792in}}%
\pgfusepath{stroke}%
\end{pgfscope}%
\begin{pgfscope}%
\pgfpathrectangle{\pgfqpoint{9.810417in}{16.722093in}}{\pgfqpoint{5.489583in}{0.877907in}}%
\pgfusepath{clip}%
\pgfsetbuttcap%
\pgfsetroundjoin%
\pgfsetlinewidth{1.505625pt}%
\definecolor{currentstroke}{rgb}{0.000000,0.000000,0.000000}%
\pgfsetstrokecolor{currentstroke}%
\pgfsetdash{}{0pt}%
\pgfpathmoveto{\pgfqpoint{14.126301in}{16.800332in}}%
\pgfpathlineto{\pgfqpoint{14.126301in}{16.803019in}}%
\pgfusepath{stroke}%
\end{pgfscope}%
\begin{pgfscope}%
\pgfpathrectangle{\pgfqpoint{9.810417in}{16.722093in}}{\pgfqpoint{5.489583in}{0.877907in}}%
\pgfusepath{clip}%
\pgfsetbuttcap%
\pgfsetroundjoin%
\pgfsetlinewidth{1.505625pt}%
\definecolor{currentstroke}{rgb}{0.000000,0.000000,0.000000}%
\pgfsetstrokecolor{currentstroke}%
\pgfsetdash{}{0pt}%
\pgfpathmoveto{\pgfqpoint{14.249524in}{16.800332in}}%
\pgfpathlineto{\pgfqpoint{14.249524in}{16.805453in}}%
\pgfusepath{stroke}%
\end{pgfscope}%
\begin{pgfscope}%
\pgfpathrectangle{\pgfqpoint{9.810417in}{16.722093in}}{\pgfqpoint{5.489583in}{0.877907in}}%
\pgfusepath{clip}%
\pgfsetbuttcap%
\pgfsetroundjoin%
\pgfsetlinewidth{1.505625pt}%
\definecolor{currentstroke}{rgb}{0.000000,0.000000,0.000000}%
\pgfsetstrokecolor{currentstroke}%
\pgfsetdash{}{0pt}%
\pgfpathmoveto{\pgfqpoint{14.372747in}{16.800332in}}%
\pgfpathlineto{\pgfqpoint{14.372747in}{16.786284in}}%
\pgfusepath{stroke}%
\end{pgfscope}%
\begin{pgfscope}%
\pgfpathrectangle{\pgfqpoint{9.810417in}{16.722093in}}{\pgfqpoint{5.489583in}{0.877907in}}%
\pgfusepath{clip}%
\pgfsetbuttcap%
\pgfsetroundjoin%
\pgfsetlinewidth{1.505625pt}%
\definecolor{currentstroke}{rgb}{0.000000,0.000000,0.000000}%
\pgfsetstrokecolor{currentstroke}%
\pgfsetdash{}{0pt}%
\pgfpathmoveto{\pgfqpoint{14.495970in}{16.800332in}}%
\pgfpathlineto{\pgfqpoint{14.495970in}{16.793159in}}%
\pgfusepath{stroke}%
\end{pgfscope}%
\begin{pgfscope}%
\pgfpathrectangle{\pgfqpoint{9.810417in}{16.722093in}}{\pgfqpoint{5.489583in}{0.877907in}}%
\pgfusepath{clip}%
\pgfsetbuttcap%
\pgfsetroundjoin%
\pgfsetlinewidth{1.505625pt}%
\definecolor{currentstroke}{rgb}{0.000000,0.000000,0.000000}%
\pgfsetstrokecolor{currentstroke}%
\pgfsetdash{}{0pt}%
\pgfpathmoveto{\pgfqpoint{14.619193in}{16.800332in}}%
\pgfpathlineto{\pgfqpoint{14.619193in}{16.798215in}}%
\pgfusepath{stroke}%
\end{pgfscope}%
\begin{pgfscope}%
\pgfpathrectangle{\pgfqpoint{9.810417in}{16.722093in}}{\pgfqpoint{5.489583in}{0.877907in}}%
\pgfusepath{clip}%
\pgfsetbuttcap%
\pgfsetroundjoin%
\pgfsetlinewidth{1.505625pt}%
\definecolor{currentstroke}{rgb}{0.000000,0.000000,0.000000}%
\pgfsetstrokecolor{currentstroke}%
\pgfsetdash{}{0pt}%
\pgfpathmoveto{\pgfqpoint{14.742416in}{16.800332in}}%
\pgfpathlineto{\pgfqpoint{14.742416in}{16.804587in}}%
\pgfusepath{stroke}%
\end{pgfscope}%
\begin{pgfscope}%
\pgfpathrectangle{\pgfqpoint{9.810417in}{16.722093in}}{\pgfqpoint{5.489583in}{0.877907in}}%
\pgfusepath{clip}%
\pgfsetbuttcap%
\pgfsetroundjoin%
\pgfsetlinewidth{1.505625pt}%
\definecolor{currentstroke}{rgb}{0.000000,0.000000,0.000000}%
\pgfsetstrokecolor{currentstroke}%
\pgfsetdash{}{0pt}%
\pgfpathmoveto{\pgfqpoint{14.865639in}{16.800332in}}%
\pgfpathlineto{\pgfqpoint{14.865639in}{16.791916in}}%
\pgfusepath{stroke}%
\end{pgfscope}%
\begin{pgfscope}%
\pgfpathrectangle{\pgfqpoint{9.810417in}{16.722093in}}{\pgfqpoint{5.489583in}{0.877907in}}%
\pgfusepath{clip}%
\pgfsetbuttcap%
\pgfsetroundjoin%
\pgfsetlinewidth{1.505625pt}%
\definecolor{currentstroke}{rgb}{0.000000,0.000000,0.000000}%
\pgfsetstrokecolor{currentstroke}%
\pgfsetdash{}{0pt}%
\pgfpathmoveto{\pgfqpoint{14.988862in}{16.800332in}}%
\pgfpathlineto{\pgfqpoint{14.988862in}{16.794590in}}%
\pgfusepath{stroke}%
\end{pgfscope}%
\begin{pgfscope}%
\pgfpathrectangle{\pgfqpoint{9.810417in}{16.722093in}}{\pgfqpoint{5.489583in}{0.877907in}}%
\pgfusepath{clip}%
\pgfsetroundcap%
\pgfsetroundjoin%
\pgfsetlinewidth{1.505625pt}%
\definecolor{currentstroke}{rgb}{0.121569,0.466667,0.705882}%
\pgfsetstrokecolor{currentstroke}%
\pgfsetdash{}{0pt}%
\pgfpathmoveto{\pgfqpoint{9.810417in}{16.800332in}}%
\pgfpathlineto{\pgfqpoint{15.300000in}{16.800332in}}%
\pgfusepath{stroke}%
\end{pgfscope}%
\begin{pgfscope}%
\pgfpathrectangle{\pgfqpoint{9.810417in}{16.722093in}}{\pgfqpoint{5.489583in}{0.877907in}}%
\pgfusepath{clip}%
\pgfsetbuttcap%
\pgfsetroundjoin%
\definecolor{currentfill}{rgb}{0.121569,0.466667,0.705882}%
\pgfsetfillcolor{currentfill}%
\pgfsetlinewidth{1.003750pt}%
\definecolor{currentstroke}{rgb}{0.121569,0.466667,0.705882}%
\pgfsetstrokecolor{currentstroke}%
\pgfsetdash{}{0pt}%
\pgfsys@defobject{currentmarker}{\pgfqpoint{-0.034722in}{-0.034722in}}{\pgfqpoint{0.034722in}{0.034722in}}{%
\pgfpathmoveto{\pgfqpoint{0.000000in}{-0.034722in}}%
\pgfpathcurveto{\pgfqpoint{0.009208in}{-0.034722in}}{\pgfqpoint{0.018041in}{-0.031064in}}{\pgfqpoint{0.024552in}{-0.024552in}}%
\pgfpathcurveto{\pgfqpoint{0.031064in}{-0.018041in}}{\pgfqpoint{0.034722in}{-0.009208in}}{\pgfqpoint{0.034722in}{0.000000in}}%
\pgfpathcurveto{\pgfqpoint{0.034722in}{0.009208in}}{\pgfqpoint{0.031064in}{0.018041in}}{\pgfqpoint{0.024552in}{0.024552in}}%
\pgfpathcurveto{\pgfqpoint{0.018041in}{0.031064in}}{\pgfqpoint{0.009208in}{0.034722in}}{\pgfqpoint{0.000000in}{0.034722in}}%
\pgfpathcurveto{\pgfqpoint{-0.009208in}{0.034722in}}{\pgfqpoint{-0.018041in}{0.031064in}}{\pgfqpoint{-0.024552in}{0.024552in}}%
\pgfpathcurveto{\pgfqpoint{-0.031064in}{0.018041in}}{\pgfqpoint{-0.034722in}{0.009208in}}{\pgfqpoint{-0.034722in}{0.000000in}}%
\pgfpathcurveto{\pgfqpoint{-0.034722in}{-0.009208in}}{\pgfqpoint{-0.031064in}{-0.018041in}}{\pgfqpoint{-0.024552in}{-0.024552in}}%
\pgfpathcurveto{\pgfqpoint{-0.018041in}{-0.031064in}}{\pgfqpoint{-0.009208in}{-0.034722in}}{\pgfqpoint{0.000000in}{-0.034722in}}%
\pgfpathclose%
\pgfusepath{stroke,fill}%
}%
\begin{pgfscope}%
\pgfsys@transformshift{10.059943in}{17.560095in}%
\pgfsys@useobject{currentmarker}{}%
\end{pgfscope}%
\begin{pgfscope}%
\pgfsys@transformshift{10.183166in}{17.558539in}%
\pgfsys@useobject{currentmarker}{}%
\end{pgfscope}%
\begin{pgfscope}%
\pgfsys@transformshift{10.306389in}{16.807931in}%
\pgfsys@useobject{currentmarker}{}%
\end{pgfscope}%
\begin{pgfscope}%
\pgfsys@transformshift{10.429612in}{16.788182in}%
\pgfsys@useobject{currentmarker}{}%
\end{pgfscope}%
\begin{pgfscope}%
\pgfsys@transformshift{10.552835in}{16.792765in}%
\pgfsys@useobject{currentmarker}{}%
\end{pgfscope}%
\begin{pgfscope}%
\pgfsys@transformshift{10.676058in}{16.814088in}%
\pgfsys@useobject{currentmarker}{}%
\end{pgfscope}%
\begin{pgfscope}%
\pgfsys@transformshift{10.799281in}{16.804443in}%
\pgfsys@useobject{currentmarker}{}%
\end{pgfscope}%
\begin{pgfscope}%
\pgfsys@transformshift{10.922504in}{16.785665in}%
\pgfsys@useobject{currentmarker}{}%
\end{pgfscope}%
\begin{pgfscope}%
\pgfsys@transformshift{11.045727in}{16.796228in}%
\pgfsys@useobject{currentmarker}{}%
\end{pgfscope}%
\begin{pgfscope}%
\pgfsys@transformshift{11.168950in}{16.785658in}%
\pgfsys@useobject{currentmarker}{}%
\end{pgfscope}%
\begin{pgfscope}%
\pgfsys@transformshift{11.292173in}{16.810830in}%
\pgfsys@useobject{currentmarker}{}%
\end{pgfscope}%
\begin{pgfscope}%
\pgfsys@transformshift{11.415396in}{16.810565in}%
\pgfsys@useobject{currentmarker}{}%
\end{pgfscope}%
\begin{pgfscope}%
\pgfsys@transformshift{11.538619in}{16.802471in}%
\pgfsys@useobject{currentmarker}{}%
\end{pgfscope}%
\begin{pgfscope}%
\pgfsys@transformshift{11.661842in}{16.806130in}%
\pgfsys@useobject{currentmarker}{}%
\end{pgfscope}%
\begin{pgfscope}%
\pgfsys@transformshift{11.785065in}{16.789038in}%
\pgfsys@useobject{currentmarker}{}%
\end{pgfscope}%
\begin{pgfscope}%
\pgfsys@transformshift{11.908288in}{16.803164in}%
\pgfsys@useobject{currentmarker}{}%
\end{pgfscope}%
\begin{pgfscope}%
\pgfsys@transformshift{12.031511in}{16.799989in}%
\pgfsys@useobject{currentmarker}{}%
\end{pgfscope}%
\begin{pgfscope}%
\pgfsys@transformshift{12.154734in}{16.813690in}%
\pgfsys@useobject{currentmarker}{}%
\end{pgfscope}%
\begin{pgfscope}%
\pgfsys@transformshift{12.277957in}{16.793479in}%
\pgfsys@useobject{currentmarker}{}%
\end{pgfscope}%
\begin{pgfscope}%
\pgfsys@transformshift{12.401180in}{16.796786in}%
\pgfsys@useobject{currentmarker}{}%
\end{pgfscope}%
\begin{pgfscope}%
\pgfsys@transformshift{12.524403in}{16.787424in}%
\pgfsys@useobject{currentmarker}{}%
\end{pgfscope}%
\begin{pgfscope}%
\pgfsys@transformshift{12.647626in}{16.795431in}%
\pgfsys@useobject{currentmarker}{}%
\end{pgfscope}%
\begin{pgfscope}%
\pgfsys@transformshift{12.770849in}{16.806691in}%
\pgfsys@useobject{currentmarker}{}%
\end{pgfscope}%
\begin{pgfscope}%
\pgfsys@transformshift{12.894072in}{16.813541in}%
\pgfsys@useobject{currentmarker}{}%
\end{pgfscope}%
\begin{pgfscope}%
\pgfsys@transformshift{13.017294in}{16.803383in}%
\pgfsys@useobject{currentmarker}{}%
\end{pgfscope}%
\begin{pgfscope}%
\pgfsys@transformshift{13.140517in}{16.807568in}%
\pgfsys@useobject{currentmarker}{}%
\end{pgfscope}%
\begin{pgfscope}%
\pgfsys@transformshift{13.263740in}{16.806899in}%
\pgfsys@useobject{currentmarker}{}%
\end{pgfscope}%
\begin{pgfscope}%
\pgfsys@transformshift{13.386963in}{16.790691in}%
\pgfsys@useobject{currentmarker}{}%
\end{pgfscope}%
\begin{pgfscope}%
\pgfsys@transformshift{13.510186in}{16.796733in}%
\pgfsys@useobject{currentmarker}{}%
\end{pgfscope}%
\begin{pgfscope}%
\pgfsys@transformshift{13.633409in}{16.813213in}%
\pgfsys@useobject{currentmarker}{}%
\end{pgfscope}%
\begin{pgfscope}%
\pgfsys@transformshift{13.756632in}{16.795599in}%
\pgfsys@useobject{currentmarker}{}%
\end{pgfscope}%
\begin{pgfscope}%
\pgfsys@transformshift{13.879855in}{16.809830in}%
\pgfsys@useobject{currentmarker}{}%
\end{pgfscope}%
\begin{pgfscope}%
\pgfsys@transformshift{14.003078in}{16.795792in}%
\pgfsys@useobject{currentmarker}{}%
\end{pgfscope}%
\begin{pgfscope}%
\pgfsys@transformshift{14.126301in}{16.803019in}%
\pgfsys@useobject{currentmarker}{}%
\end{pgfscope}%
\begin{pgfscope}%
\pgfsys@transformshift{14.249524in}{16.805453in}%
\pgfsys@useobject{currentmarker}{}%
\end{pgfscope}%
\begin{pgfscope}%
\pgfsys@transformshift{14.372747in}{16.786284in}%
\pgfsys@useobject{currentmarker}{}%
\end{pgfscope}%
\begin{pgfscope}%
\pgfsys@transformshift{14.495970in}{16.793159in}%
\pgfsys@useobject{currentmarker}{}%
\end{pgfscope}%
\begin{pgfscope}%
\pgfsys@transformshift{14.619193in}{16.798215in}%
\pgfsys@useobject{currentmarker}{}%
\end{pgfscope}%
\begin{pgfscope}%
\pgfsys@transformshift{14.742416in}{16.804587in}%
\pgfsys@useobject{currentmarker}{}%
\end{pgfscope}%
\begin{pgfscope}%
\pgfsys@transformshift{14.865639in}{16.791916in}%
\pgfsys@useobject{currentmarker}{}%
\end{pgfscope}%
\begin{pgfscope}%
\pgfsys@transformshift{14.988862in}{16.794590in}%
\pgfsys@useobject{currentmarker}{}%
\end{pgfscope}%
\end{pgfscope}%
\begin{pgfscope}%
\pgfsetrectcap%
\pgfsetmiterjoin%
\pgfsetlinewidth{0.803000pt}%
\definecolor{currentstroke}{rgb}{1.000000,1.000000,1.000000}%
\pgfsetstrokecolor{currentstroke}%
\pgfsetdash{}{0pt}%
\pgfpathmoveto{\pgfqpoint{9.810417in}{16.722093in}}%
\pgfpathlineto{\pgfqpoint{9.810417in}{17.600000in}}%
\pgfusepath{stroke}%
\end{pgfscope}%
\begin{pgfscope}%
\pgfsetrectcap%
\pgfsetmiterjoin%
\pgfsetlinewidth{0.803000pt}%
\definecolor{currentstroke}{rgb}{1.000000,1.000000,1.000000}%
\pgfsetstrokecolor{currentstroke}%
\pgfsetdash{}{0pt}%
\pgfpathmoveto{\pgfqpoint{15.300000in}{16.722093in}}%
\pgfpathlineto{\pgfqpoint{15.300000in}{17.600000in}}%
\pgfusepath{stroke}%
\end{pgfscope}%
\begin{pgfscope}%
\pgfsetrectcap%
\pgfsetmiterjoin%
\pgfsetlinewidth{0.803000pt}%
\definecolor{currentstroke}{rgb}{1.000000,1.000000,1.000000}%
\pgfsetstrokecolor{currentstroke}%
\pgfsetdash{}{0pt}%
\pgfpathmoveto{\pgfqpoint{9.810417in}{16.722093in}}%
\pgfpathlineto{\pgfqpoint{15.300000in}{16.722093in}}%
\pgfusepath{stroke}%
\end{pgfscope}%
\begin{pgfscope}%
\pgfsetrectcap%
\pgfsetmiterjoin%
\pgfsetlinewidth{0.803000pt}%
\definecolor{currentstroke}{rgb}{1.000000,1.000000,1.000000}%
\pgfsetstrokecolor{currentstroke}%
\pgfsetdash{}{0pt}%
\pgfpathmoveto{\pgfqpoint{9.810417in}{17.600000in}}%
\pgfpathlineto{\pgfqpoint{15.300000in}{17.600000in}}%
\pgfusepath{stroke}%
\end{pgfscope}%
\begin{pgfscope}%
\definecolor{textcolor}{rgb}{0.150000,0.150000,0.150000}%
\pgfsetstrokecolor{textcolor}%
\pgfsetfillcolor{textcolor}%
\pgftext[x=12.555208in,y=17.683333in,,base]{\color{textcolor}\rmfamily\fontsize{16.800000}{20.160000}\selectfont Partial Autocorrelation}%
\end{pgfscope}%
\begin{pgfscope}%
\pgfsetbuttcap%
\pgfsetmiterjoin%
\definecolor{currentfill}{rgb}{0.917647,0.917647,0.949020}%
\pgfsetfillcolor{currentfill}%
\pgfsetlinewidth{0.000000pt}%
\definecolor{currentstroke}{rgb}{0.000000,0.000000,0.000000}%
\pgfsetstrokecolor{currentstroke}%
\pgfsetstrokeopacity{0.000000}%
\pgfsetdash{}{0pt}%
\pgfpathmoveto{\pgfqpoint{2.125000in}{15.141860in}}%
\pgfpathlineto{\pgfqpoint{7.614583in}{15.141860in}}%
\pgfpathlineto{\pgfqpoint{7.614583in}{16.019767in}}%
\pgfpathlineto{\pgfqpoint{2.125000in}{16.019767in}}%
\pgfpathclose%
\pgfusepath{fill}%
\end{pgfscope}%
\begin{pgfscope}%
\pgfpathrectangle{\pgfqpoint{2.125000in}{15.141860in}}{\pgfqpoint{5.489583in}{0.877907in}}%
\pgfusepath{clip}%
\pgfsetroundcap%
\pgfsetroundjoin%
\pgfsetlinewidth{0.803000pt}%
\definecolor{currentstroke}{rgb}{1.000000,1.000000,1.000000}%
\pgfsetstrokecolor{currentstroke}%
\pgfsetdash{}{0pt}%
\pgfpathmoveto{\pgfqpoint{2.374527in}{15.141860in}}%
\pgfpathlineto{\pgfqpoint{2.374527in}{16.019767in}}%
\pgfusepath{stroke}%
\end{pgfscope}%
\begin{pgfscope}%
\definecolor{textcolor}{rgb}{0.150000,0.150000,0.150000}%
\pgfsetstrokecolor{textcolor}%
\pgfsetfillcolor{textcolor}%
\pgftext[x=2.374527in,y=15.044638in,,top]{\color{textcolor}\rmfamily\fontsize{14.000000}{16.800000}\selectfont 0}%
\end{pgfscope}%
\begin{pgfscope}%
\pgfpathrectangle{\pgfqpoint{2.125000in}{15.141860in}}{\pgfqpoint{5.489583in}{0.877907in}}%
\pgfusepath{clip}%
\pgfsetroundcap%
\pgfsetroundjoin%
\pgfsetlinewidth{0.803000pt}%
\definecolor{currentstroke}{rgb}{1.000000,1.000000,1.000000}%
\pgfsetstrokecolor{currentstroke}%
\pgfsetdash{}{0pt}%
\pgfpathmoveto{\pgfqpoint{2.990641in}{15.141860in}}%
\pgfpathlineto{\pgfqpoint{2.990641in}{16.019767in}}%
\pgfusepath{stroke}%
\end{pgfscope}%
\begin{pgfscope}%
\definecolor{textcolor}{rgb}{0.150000,0.150000,0.150000}%
\pgfsetstrokecolor{textcolor}%
\pgfsetfillcolor{textcolor}%
\pgftext[x=2.990641in,y=15.044638in,,top]{\color{textcolor}\rmfamily\fontsize{14.000000}{16.800000}\selectfont 5}%
\end{pgfscope}%
\begin{pgfscope}%
\pgfpathrectangle{\pgfqpoint{2.125000in}{15.141860in}}{\pgfqpoint{5.489583in}{0.877907in}}%
\pgfusepath{clip}%
\pgfsetroundcap%
\pgfsetroundjoin%
\pgfsetlinewidth{0.803000pt}%
\definecolor{currentstroke}{rgb}{1.000000,1.000000,1.000000}%
\pgfsetstrokecolor{currentstroke}%
\pgfsetdash{}{0pt}%
\pgfpathmoveto{\pgfqpoint{3.606756in}{15.141860in}}%
\pgfpathlineto{\pgfqpoint{3.606756in}{16.019767in}}%
\pgfusepath{stroke}%
\end{pgfscope}%
\begin{pgfscope}%
\definecolor{textcolor}{rgb}{0.150000,0.150000,0.150000}%
\pgfsetstrokecolor{textcolor}%
\pgfsetfillcolor{textcolor}%
\pgftext[x=3.606756in,y=15.044638in,,top]{\color{textcolor}\rmfamily\fontsize{14.000000}{16.800000}\selectfont 10}%
\end{pgfscope}%
\begin{pgfscope}%
\pgfpathrectangle{\pgfqpoint{2.125000in}{15.141860in}}{\pgfqpoint{5.489583in}{0.877907in}}%
\pgfusepath{clip}%
\pgfsetroundcap%
\pgfsetroundjoin%
\pgfsetlinewidth{0.803000pt}%
\definecolor{currentstroke}{rgb}{1.000000,1.000000,1.000000}%
\pgfsetstrokecolor{currentstroke}%
\pgfsetdash{}{0pt}%
\pgfpathmoveto{\pgfqpoint{4.222871in}{15.141860in}}%
\pgfpathlineto{\pgfqpoint{4.222871in}{16.019767in}}%
\pgfusepath{stroke}%
\end{pgfscope}%
\begin{pgfscope}%
\definecolor{textcolor}{rgb}{0.150000,0.150000,0.150000}%
\pgfsetstrokecolor{textcolor}%
\pgfsetfillcolor{textcolor}%
\pgftext[x=4.222871in,y=15.044638in,,top]{\color{textcolor}\rmfamily\fontsize{14.000000}{16.800000}\selectfont 15}%
\end{pgfscope}%
\begin{pgfscope}%
\pgfpathrectangle{\pgfqpoint{2.125000in}{15.141860in}}{\pgfqpoint{5.489583in}{0.877907in}}%
\pgfusepath{clip}%
\pgfsetroundcap%
\pgfsetroundjoin%
\pgfsetlinewidth{0.803000pt}%
\definecolor{currentstroke}{rgb}{1.000000,1.000000,1.000000}%
\pgfsetstrokecolor{currentstroke}%
\pgfsetdash{}{0pt}%
\pgfpathmoveto{\pgfqpoint{4.838986in}{15.141860in}}%
\pgfpathlineto{\pgfqpoint{4.838986in}{16.019767in}}%
\pgfusepath{stroke}%
\end{pgfscope}%
\begin{pgfscope}%
\definecolor{textcolor}{rgb}{0.150000,0.150000,0.150000}%
\pgfsetstrokecolor{textcolor}%
\pgfsetfillcolor{textcolor}%
\pgftext[x=4.838986in,y=15.044638in,,top]{\color{textcolor}\rmfamily\fontsize{14.000000}{16.800000}\selectfont 20}%
\end{pgfscope}%
\begin{pgfscope}%
\pgfpathrectangle{\pgfqpoint{2.125000in}{15.141860in}}{\pgfqpoint{5.489583in}{0.877907in}}%
\pgfusepath{clip}%
\pgfsetroundcap%
\pgfsetroundjoin%
\pgfsetlinewidth{0.803000pt}%
\definecolor{currentstroke}{rgb}{1.000000,1.000000,1.000000}%
\pgfsetstrokecolor{currentstroke}%
\pgfsetdash{}{0pt}%
\pgfpathmoveto{\pgfqpoint{5.455101in}{15.141860in}}%
\pgfpathlineto{\pgfqpoint{5.455101in}{16.019767in}}%
\pgfusepath{stroke}%
\end{pgfscope}%
\begin{pgfscope}%
\definecolor{textcolor}{rgb}{0.150000,0.150000,0.150000}%
\pgfsetstrokecolor{textcolor}%
\pgfsetfillcolor{textcolor}%
\pgftext[x=5.455101in,y=15.044638in,,top]{\color{textcolor}\rmfamily\fontsize{14.000000}{16.800000}\selectfont 25}%
\end{pgfscope}%
\begin{pgfscope}%
\pgfpathrectangle{\pgfqpoint{2.125000in}{15.141860in}}{\pgfqpoint{5.489583in}{0.877907in}}%
\pgfusepath{clip}%
\pgfsetroundcap%
\pgfsetroundjoin%
\pgfsetlinewidth{0.803000pt}%
\definecolor{currentstroke}{rgb}{1.000000,1.000000,1.000000}%
\pgfsetstrokecolor{currentstroke}%
\pgfsetdash{}{0pt}%
\pgfpathmoveto{\pgfqpoint{6.071216in}{15.141860in}}%
\pgfpathlineto{\pgfqpoint{6.071216in}{16.019767in}}%
\pgfusepath{stroke}%
\end{pgfscope}%
\begin{pgfscope}%
\definecolor{textcolor}{rgb}{0.150000,0.150000,0.150000}%
\pgfsetstrokecolor{textcolor}%
\pgfsetfillcolor{textcolor}%
\pgftext[x=6.071216in,y=15.044638in,,top]{\color{textcolor}\rmfamily\fontsize{14.000000}{16.800000}\selectfont 30}%
\end{pgfscope}%
\begin{pgfscope}%
\pgfpathrectangle{\pgfqpoint{2.125000in}{15.141860in}}{\pgfqpoint{5.489583in}{0.877907in}}%
\pgfusepath{clip}%
\pgfsetroundcap%
\pgfsetroundjoin%
\pgfsetlinewidth{0.803000pt}%
\definecolor{currentstroke}{rgb}{1.000000,1.000000,1.000000}%
\pgfsetstrokecolor{currentstroke}%
\pgfsetdash{}{0pt}%
\pgfpathmoveto{\pgfqpoint{6.687330in}{15.141860in}}%
\pgfpathlineto{\pgfqpoint{6.687330in}{16.019767in}}%
\pgfusepath{stroke}%
\end{pgfscope}%
\begin{pgfscope}%
\definecolor{textcolor}{rgb}{0.150000,0.150000,0.150000}%
\pgfsetstrokecolor{textcolor}%
\pgfsetfillcolor{textcolor}%
\pgftext[x=6.687330in,y=15.044638in,,top]{\color{textcolor}\rmfamily\fontsize{14.000000}{16.800000}\selectfont 35}%
\end{pgfscope}%
\begin{pgfscope}%
\pgfpathrectangle{\pgfqpoint{2.125000in}{15.141860in}}{\pgfqpoint{5.489583in}{0.877907in}}%
\pgfusepath{clip}%
\pgfsetroundcap%
\pgfsetroundjoin%
\pgfsetlinewidth{0.803000pt}%
\definecolor{currentstroke}{rgb}{1.000000,1.000000,1.000000}%
\pgfsetstrokecolor{currentstroke}%
\pgfsetdash{}{0pt}%
\pgfpathmoveto{\pgfqpoint{7.303445in}{15.141860in}}%
\pgfpathlineto{\pgfqpoint{7.303445in}{16.019767in}}%
\pgfusepath{stroke}%
\end{pgfscope}%
\begin{pgfscope}%
\definecolor{textcolor}{rgb}{0.150000,0.150000,0.150000}%
\pgfsetstrokecolor{textcolor}%
\pgfsetfillcolor{textcolor}%
\pgftext[x=7.303445in,y=15.044638in,,top]{\color{textcolor}\rmfamily\fontsize{14.000000}{16.800000}\selectfont 40}%
\end{pgfscope}%
\begin{pgfscope}%
\pgfpathrectangle{\pgfqpoint{2.125000in}{15.141860in}}{\pgfqpoint{5.489583in}{0.877907in}}%
\pgfusepath{clip}%
\pgfsetroundcap%
\pgfsetroundjoin%
\pgfsetlinewidth{0.803000pt}%
\definecolor{currentstroke}{rgb}{1.000000,1.000000,1.000000}%
\pgfsetstrokecolor{currentstroke}%
\pgfsetdash{}{0pt}%
\pgfpathmoveto{\pgfqpoint{2.125000in}{15.412639in}}%
\pgfpathlineto{\pgfqpoint{7.614583in}{15.412639in}}%
\pgfusepath{stroke}%
\end{pgfscope}%
\begin{pgfscope}%
\definecolor{textcolor}{rgb}{0.150000,0.150000,0.150000}%
\pgfsetstrokecolor{textcolor}%
\pgfsetfillcolor{textcolor}%
\pgftext[x=1.904066in,y=15.338773in,left,base]{\color{textcolor}\rmfamily\fontsize{14.000000}{16.800000}\selectfont 0}%
\end{pgfscope}%
\begin{pgfscope}%
\pgfpathrectangle{\pgfqpoint{2.125000in}{15.141860in}}{\pgfqpoint{5.489583in}{0.877907in}}%
\pgfusepath{clip}%
\pgfsetroundcap%
\pgfsetroundjoin%
\pgfsetlinewidth{0.803000pt}%
\definecolor{currentstroke}{rgb}{1.000000,1.000000,1.000000}%
\pgfsetstrokecolor{currentstroke}%
\pgfsetdash{}{0pt}%
\pgfpathmoveto{\pgfqpoint{2.125000in}{15.979863in}}%
\pgfpathlineto{\pgfqpoint{7.614583in}{15.979863in}}%
\pgfusepath{stroke}%
\end{pgfscope}%
\begin{pgfscope}%
\definecolor{textcolor}{rgb}{0.150000,0.150000,0.150000}%
\pgfsetstrokecolor{textcolor}%
\pgfsetfillcolor{textcolor}%
\pgftext[x=1.904066in,y=15.905996in,left,base]{\color{textcolor}\rmfamily\fontsize{14.000000}{16.800000}\selectfont 1}%
\end{pgfscope}%
\begin{pgfscope}%
\pgfpathrectangle{\pgfqpoint{2.125000in}{15.141860in}}{\pgfqpoint{5.489583in}{0.877907in}}%
\pgfusepath{clip}%
\pgfsetbuttcap%
\pgfsetroundjoin%
\definecolor{currentfill}{rgb}{0.121569,0.466667,0.705882}%
\pgfsetfillcolor{currentfill}%
\pgfsetfillopacity{0.250000}%
\pgfsetlinewidth{1.003750pt}%
\definecolor{currentstroke}{rgb}{1.000000,1.000000,1.000000}%
\pgfsetstrokecolor{currentstroke}%
\pgfsetstrokeopacity{0.250000}%
\pgfsetdash{}{0pt}%
\pgfpathmoveto{\pgfqpoint{2.436138in}{15.441259in}}%
\pgfpathlineto{\pgfqpoint{2.436138in}{15.384020in}}%
\pgfpathlineto{\pgfqpoint{2.620972in}{15.363240in}}%
\pgfpathlineto{\pgfqpoint{2.744195in}{15.349046in}}%
\pgfpathlineto{\pgfqpoint{2.867418in}{15.337597in}}%
\pgfpathlineto{\pgfqpoint{2.990641in}{15.327775in}}%
\pgfpathlineto{\pgfqpoint{3.113864in}{15.319064in}}%
\pgfpathlineto{\pgfqpoint{3.237087in}{15.311174in}}%
\pgfpathlineto{\pgfqpoint{3.360310in}{15.303926in}}%
\pgfpathlineto{\pgfqpoint{3.483533in}{15.297196in}}%
\pgfpathlineto{\pgfqpoint{3.606756in}{15.290901in}}%
\pgfpathlineto{\pgfqpoint{3.729979in}{15.284972in}}%
\pgfpathlineto{\pgfqpoint{3.853202in}{15.279358in}}%
\pgfpathlineto{\pgfqpoint{3.976425in}{15.274018in}}%
\pgfpathlineto{\pgfqpoint{4.099648in}{15.268926in}}%
\pgfpathlineto{\pgfqpoint{4.222871in}{15.264057in}}%
\pgfpathlineto{\pgfqpoint{4.346094in}{15.259392in}}%
\pgfpathlineto{\pgfqpoint{4.469317in}{15.254909in}}%
\pgfpathlineto{\pgfqpoint{4.592540in}{15.250595in}}%
\pgfpathlineto{\pgfqpoint{4.715763in}{15.246437in}}%
\pgfpathlineto{\pgfqpoint{4.838986in}{15.242425in}}%
\pgfpathlineto{\pgfqpoint{4.962209in}{15.238545in}}%
\pgfpathlineto{\pgfqpoint{5.085432in}{15.234788in}}%
\pgfpathlineto{\pgfqpoint{5.208655in}{15.231147in}}%
\pgfpathlineto{\pgfqpoint{5.331878in}{15.227612in}}%
\pgfpathlineto{\pgfqpoint{5.455101in}{15.224176in}}%
\pgfpathlineto{\pgfqpoint{5.578324in}{15.220833in}}%
\pgfpathlineto{\pgfqpoint{5.701547in}{15.217578in}}%
\pgfpathlineto{\pgfqpoint{5.824770in}{15.214408in}}%
\pgfpathlineto{\pgfqpoint{5.947993in}{15.211319in}}%
\pgfpathlineto{\pgfqpoint{6.071216in}{15.208307in}}%
\pgfpathlineto{\pgfqpoint{6.194439in}{15.205370in}}%
\pgfpathlineto{\pgfqpoint{6.317662in}{15.202503in}}%
\pgfpathlineto{\pgfqpoint{6.440885in}{15.199702in}}%
\pgfpathlineto{\pgfqpoint{6.564108in}{15.196964in}}%
\pgfpathlineto{\pgfqpoint{6.687330in}{15.194288in}}%
\pgfpathlineto{\pgfqpoint{6.810553in}{15.191672in}}%
\pgfpathlineto{\pgfqpoint{6.933776in}{15.189115in}}%
\pgfpathlineto{\pgfqpoint{7.056999in}{15.186614in}}%
\pgfpathlineto{\pgfqpoint{7.180222in}{15.184165in}}%
\pgfpathlineto{\pgfqpoint{7.365057in}{15.181765in}}%
\pgfpathlineto{\pgfqpoint{7.365057in}{15.643514in}}%
\pgfpathlineto{\pgfqpoint{7.365057in}{15.643514in}}%
\pgfpathlineto{\pgfqpoint{7.180222in}{15.641114in}}%
\pgfpathlineto{\pgfqpoint{7.056999in}{15.638665in}}%
\pgfpathlineto{\pgfqpoint{6.933776in}{15.636164in}}%
\pgfpathlineto{\pgfqpoint{6.810553in}{15.633607in}}%
\pgfpathlineto{\pgfqpoint{6.687330in}{15.630991in}}%
\pgfpathlineto{\pgfqpoint{6.564108in}{15.628315in}}%
\pgfpathlineto{\pgfqpoint{6.440885in}{15.625577in}}%
\pgfpathlineto{\pgfqpoint{6.317662in}{15.622776in}}%
\pgfpathlineto{\pgfqpoint{6.194439in}{15.619909in}}%
\pgfpathlineto{\pgfqpoint{6.071216in}{15.616972in}}%
\pgfpathlineto{\pgfqpoint{5.947993in}{15.613960in}}%
\pgfpathlineto{\pgfqpoint{5.824770in}{15.610871in}}%
\pgfpathlineto{\pgfqpoint{5.701547in}{15.607700in}}%
\pgfpathlineto{\pgfqpoint{5.578324in}{15.604446in}}%
\pgfpathlineto{\pgfqpoint{5.455101in}{15.601103in}}%
\pgfpathlineto{\pgfqpoint{5.331878in}{15.597667in}}%
\pgfpathlineto{\pgfqpoint{5.208655in}{15.594132in}}%
\pgfpathlineto{\pgfqpoint{5.085432in}{15.590490in}}%
\pgfpathlineto{\pgfqpoint{4.962209in}{15.586734in}}%
\pgfpathlineto{\pgfqpoint{4.838986in}{15.582854in}}%
\pgfpathlineto{\pgfqpoint{4.715763in}{15.578842in}}%
\pgfpathlineto{\pgfqpoint{4.592540in}{15.574684in}}%
\pgfpathlineto{\pgfqpoint{4.469317in}{15.570369in}}%
\pgfpathlineto{\pgfqpoint{4.346094in}{15.565887in}}%
\pgfpathlineto{\pgfqpoint{4.222871in}{15.561222in}}%
\pgfpathlineto{\pgfqpoint{4.099648in}{15.556353in}}%
\pgfpathlineto{\pgfqpoint{3.976425in}{15.551261in}}%
\pgfpathlineto{\pgfqpoint{3.853202in}{15.545921in}}%
\pgfpathlineto{\pgfqpoint{3.729979in}{15.540307in}}%
\pgfpathlineto{\pgfqpoint{3.606756in}{15.534378in}}%
\pgfpathlineto{\pgfqpoint{3.483533in}{15.528082in}}%
\pgfpathlineto{\pgfqpoint{3.360310in}{15.521353in}}%
\pgfpathlineto{\pgfqpoint{3.237087in}{15.514104in}}%
\pgfpathlineto{\pgfqpoint{3.113864in}{15.506214in}}%
\pgfpathlineto{\pgfqpoint{2.990641in}{15.497504in}}%
\pgfpathlineto{\pgfqpoint{2.867418in}{15.487682in}}%
\pgfpathlineto{\pgfqpoint{2.744195in}{15.476233in}}%
\pgfpathlineto{\pgfqpoint{2.620972in}{15.462039in}}%
\pgfpathlineto{\pgfqpoint{2.436138in}{15.441259in}}%
\pgfpathclose%
\pgfusepath{stroke,fill}%
\end{pgfscope}%
\begin{pgfscope}%
\pgfpathrectangle{\pgfqpoint{2.125000in}{15.141860in}}{\pgfqpoint{5.489583in}{0.877907in}}%
\pgfusepath{clip}%
\pgfsetbuttcap%
\pgfsetroundjoin%
\pgfsetlinewidth{1.505625pt}%
\definecolor{currentstroke}{rgb}{0.000000,0.000000,0.000000}%
\pgfsetstrokecolor{currentstroke}%
\pgfsetdash{}{0pt}%
\pgfpathmoveto{\pgfqpoint{2.374527in}{15.412639in}}%
\pgfpathlineto{\pgfqpoint{2.374527in}{15.979863in}}%
\pgfusepath{stroke}%
\end{pgfscope}%
\begin{pgfscope}%
\pgfpathrectangle{\pgfqpoint{2.125000in}{15.141860in}}{\pgfqpoint{5.489583in}{0.877907in}}%
\pgfusepath{clip}%
\pgfsetbuttcap%
\pgfsetroundjoin%
\pgfsetlinewidth{1.505625pt}%
\definecolor{currentstroke}{rgb}{0.000000,0.000000,0.000000}%
\pgfsetstrokecolor{currentstroke}%
\pgfsetdash{}{0pt}%
\pgfpathmoveto{\pgfqpoint{2.497749in}{15.412639in}}%
\pgfpathlineto{\pgfqpoint{2.497749in}{15.976931in}}%
\pgfusepath{stroke}%
\end{pgfscope}%
\begin{pgfscope}%
\pgfpathrectangle{\pgfqpoint{2.125000in}{15.141860in}}{\pgfqpoint{5.489583in}{0.877907in}}%
\pgfusepath{clip}%
\pgfsetbuttcap%
\pgfsetroundjoin%
\pgfsetlinewidth{1.505625pt}%
\definecolor{currentstroke}{rgb}{0.000000,0.000000,0.000000}%
\pgfsetstrokecolor{currentstroke}%
\pgfsetdash{}{0pt}%
\pgfpathmoveto{\pgfqpoint{2.620972in}{15.412639in}}%
\pgfpathlineto{\pgfqpoint{2.620972in}{15.973898in}}%
\pgfusepath{stroke}%
\end{pgfscope}%
\begin{pgfscope}%
\pgfpathrectangle{\pgfqpoint{2.125000in}{15.141860in}}{\pgfqpoint{5.489583in}{0.877907in}}%
\pgfusepath{clip}%
\pgfsetbuttcap%
\pgfsetroundjoin%
\pgfsetlinewidth{1.505625pt}%
\definecolor{currentstroke}{rgb}{0.000000,0.000000,0.000000}%
\pgfsetstrokecolor{currentstroke}%
\pgfsetdash{}{0pt}%
\pgfpathmoveto{\pgfqpoint{2.744195in}{15.412639in}}%
\pgfpathlineto{\pgfqpoint{2.744195in}{15.970976in}}%
\pgfusepath{stroke}%
\end{pgfscope}%
\begin{pgfscope}%
\pgfpathrectangle{\pgfqpoint{2.125000in}{15.141860in}}{\pgfqpoint{5.489583in}{0.877907in}}%
\pgfusepath{clip}%
\pgfsetbuttcap%
\pgfsetroundjoin%
\pgfsetlinewidth{1.505625pt}%
\definecolor{currentstroke}{rgb}{0.000000,0.000000,0.000000}%
\pgfsetstrokecolor{currentstroke}%
\pgfsetdash{}{0pt}%
\pgfpathmoveto{\pgfqpoint{2.867418in}{15.412639in}}%
\pgfpathlineto{\pgfqpoint{2.867418in}{15.968054in}}%
\pgfusepath{stroke}%
\end{pgfscope}%
\begin{pgfscope}%
\pgfpathrectangle{\pgfqpoint{2.125000in}{15.141860in}}{\pgfqpoint{5.489583in}{0.877907in}}%
\pgfusepath{clip}%
\pgfsetbuttcap%
\pgfsetroundjoin%
\pgfsetlinewidth{1.505625pt}%
\definecolor{currentstroke}{rgb}{0.000000,0.000000,0.000000}%
\pgfsetstrokecolor{currentstroke}%
\pgfsetdash{}{0pt}%
\pgfpathmoveto{\pgfqpoint{2.990641in}{15.412639in}}%
\pgfpathlineto{\pgfqpoint{2.990641in}{15.965165in}}%
\pgfusepath{stroke}%
\end{pgfscope}%
\begin{pgfscope}%
\pgfpathrectangle{\pgfqpoint{2.125000in}{15.141860in}}{\pgfqpoint{5.489583in}{0.877907in}}%
\pgfusepath{clip}%
\pgfsetbuttcap%
\pgfsetroundjoin%
\pgfsetlinewidth{1.505625pt}%
\definecolor{currentstroke}{rgb}{0.000000,0.000000,0.000000}%
\pgfsetstrokecolor{currentstroke}%
\pgfsetdash{}{0pt}%
\pgfpathmoveto{\pgfqpoint{3.113864in}{15.412639in}}%
\pgfpathlineto{\pgfqpoint{3.113864in}{15.962414in}}%
\pgfusepath{stroke}%
\end{pgfscope}%
\begin{pgfscope}%
\pgfpathrectangle{\pgfqpoint{2.125000in}{15.141860in}}{\pgfqpoint{5.489583in}{0.877907in}}%
\pgfusepath{clip}%
\pgfsetbuttcap%
\pgfsetroundjoin%
\pgfsetlinewidth{1.505625pt}%
\definecolor{currentstroke}{rgb}{0.000000,0.000000,0.000000}%
\pgfsetstrokecolor{currentstroke}%
\pgfsetdash{}{0pt}%
\pgfpathmoveto{\pgfqpoint{3.237087in}{15.412639in}}%
\pgfpathlineto{\pgfqpoint{3.237087in}{15.959647in}}%
\pgfusepath{stroke}%
\end{pgfscope}%
\begin{pgfscope}%
\pgfpathrectangle{\pgfqpoint{2.125000in}{15.141860in}}{\pgfqpoint{5.489583in}{0.877907in}}%
\pgfusepath{clip}%
\pgfsetbuttcap%
\pgfsetroundjoin%
\pgfsetlinewidth{1.505625pt}%
\definecolor{currentstroke}{rgb}{0.000000,0.000000,0.000000}%
\pgfsetstrokecolor{currentstroke}%
\pgfsetdash{}{0pt}%
\pgfpathmoveto{\pgfqpoint{3.360310in}{15.412639in}}%
\pgfpathlineto{\pgfqpoint{3.360310in}{15.956958in}}%
\pgfusepath{stroke}%
\end{pgfscope}%
\begin{pgfscope}%
\pgfpathrectangle{\pgfqpoint{2.125000in}{15.141860in}}{\pgfqpoint{5.489583in}{0.877907in}}%
\pgfusepath{clip}%
\pgfsetbuttcap%
\pgfsetroundjoin%
\pgfsetlinewidth{1.505625pt}%
\definecolor{currentstroke}{rgb}{0.000000,0.000000,0.000000}%
\pgfsetstrokecolor{currentstroke}%
\pgfsetdash{}{0pt}%
\pgfpathmoveto{\pgfqpoint{3.483533in}{15.412639in}}%
\pgfpathlineto{\pgfqpoint{3.483533in}{15.954179in}}%
\pgfusepath{stroke}%
\end{pgfscope}%
\begin{pgfscope}%
\pgfpathrectangle{\pgfqpoint{2.125000in}{15.141860in}}{\pgfqpoint{5.489583in}{0.877907in}}%
\pgfusepath{clip}%
\pgfsetbuttcap%
\pgfsetroundjoin%
\pgfsetlinewidth{1.505625pt}%
\definecolor{currentstroke}{rgb}{0.000000,0.000000,0.000000}%
\pgfsetstrokecolor{currentstroke}%
\pgfsetdash{}{0pt}%
\pgfpathmoveto{\pgfqpoint{3.606756in}{15.412639in}}%
\pgfpathlineto{\pgfqpoint{3.606756in}{15.951552in}}%
\pgfusepath{stroke}%
\end{pgfscope}%
\begin{pgfscope}%
\pgfpathrectangle{\pgfqpoint{2.125000in}{15.141860in}}{\pgfqpoint{5.489583in}{0.877907in}}%
\pgfusepath{clip}%
\pgfsetbuttcap%
\pgfsetroundjoin%
\pgfsetlinewidth{1.505625pt}%
\definecolor{currentstroke}{rgb}{0.000000,0.000000,0.000000}%
\pgfsetstrokecolor{currentstroke}%
\pgfsetdash{}{0pt}%
\pgfpathmoveto{\pgfqpoint{3.729979in}{15.412639in}}%
\pgfpathlineto{\pgfqpoint{3.729979in}{15.949071in}}%
\pgfusepath{stroke}%
\end{pgfscope}%
\begin{pgfscope}%
\pgfpathrectangle{\pgfqpoint{2.125000in}{15.141860in}}{\pgfqpoint{5.489583in}{0.877907in}}%
\pgfusepath{clip}%
\pgfsetbuttcap%
\pgfsetroundjoin%
\pgfsetlinewidth{1.505625pt}%
\definecolor{currentstroke}{rgb}{0.000000,0.000000,0.000000}%
\pgfsetstrokecolor{currentstroke}%
\pgfsetdash{}{0pt}%
\pgfpathmoveto{\pgfqpoint{3.853202in}{15.412639in}}%
\pgfpathlineto{\pgfqpoint{3.853202in}{15.946640in}}%
\pgfusepath{stroke}%
\end{pgfscope}%
\begin{pgfscope}%
\pgfpathrectangle{\pgfqpoint{2.125000in}{15.141860in}}{\pgfqpoint{5.489583in}{0.877907in}}%
\pgfusepath{clip}%
\pgfsetbuttcap%
\pgfsetroundjoin%
\pgfsetlinewidth{1.505625pt}%
\definecolor{currentstroke}{rgb}{0.000000,0.000000,0.000000}%
\pgfsetstrokecolor{currentstroke}%
\pgfsetdash{}{0pt}%
\pgfpathmoveto{\pgfqpoint{3.976425in}{15.412639in}}%
\pgfpathlineto{\pgfqpoint{3.976425in}{15.944043in}}%
\pgfusepath{stroke}%
\end{pgfscope}%
\begin{pgfscope}%
\pgfpathrectangle{\pgfqpoint{2.125000in}{15.141860in}}{\pgfqpoint{5.489583in}{0.877907in}}%
\pgfusepath{clip}%
\pgfsetbuttcap%
\pgfsetroundjoin%
\pgfsetlinewidth{1.505625pt}%
\definecolor{currentstroke}{rgb}{0.000000,0.000000,0.000000}%
\pgfsetstrokecolor{currentstroke}%
\pgfsetdash{}{0pt}%
\pgfpathmoveto{\pgfqpoint{4.099648in}{15.412639in}}%
\pgfpathlineto{\pgfqpoint{4.099648in}{15.941321in}}%
\pgfusepath{stroke}%
\end{pgfscope}%
\begin{pgfscope}%
\pgfpathrectangle{\pgfqpoint{2.125000in}{15.141860in}}{\pgfqpoint{5.489583in}{0.877907in}}%
\pgfusepath{clip}%
\pgfsetbuttcap%
\pgfsetroundjoin%
\pgfsetlinewidth{1.505625pt}%
\definecolor{currentstroke}{rgb}{0.000000,0.000000,0.000000}%
\pgfsetstrokecolor{currentstroke}%
\pgfsetdash{}{0pt}%
\pgfpathmoveto{\pgfqpoint{4.222871in}{15.412639in}}%
\pgfpathlineto{\pgfqpoint{4.222871in}{15.938536in}}%
\pgfusepath{stroke}%
\end{pgfscope}%
\begin{pgfscope}%
\pgfpathrectangle{\pgfqpoint{2.125000in}{15.141860in}}{\pgfqpoint{5.489583in}{0.877907in}}%
\pgfusepath{clip}%
\pgfsetbuttcap%
\pgfsetroundjoin%
\pgfsetlinewidth{1.505625pt}%
\definecolor{currentstroke}{rgb}{0.000000,0.000000,0.000000}%
\pgfsetstrokecolor{currentstroke}%
\pgfsetdash{}{0pt}%
\pgfpathmoveto{\pgfqpoint{4.346094in}{15.412639in}}%
\pgfpathlineto{\pgfqpoint{4.346094in}{15.935889in}}%
\pgfusepath{stroke}%
\end{pgfscope}%
\begin{pgfscope}%
\pgfpathrectangle{\pgfqpoint{2.125000in}{15.141860in}}{\pgfqpoint{5.489583in}{0.877907in}}%
\pgfusepath{clip}%
\pgfsetbuttcap%
\pgfsetroundjoin%
\pgfsetlinewidth{1.505625pt}%
\definecolor{currentstroke}{rgb}{0.000000,0.000000,0.000000}%
\pgfsetstrokecolor{currentstroke}%
\pgfsetdash{}{0pt}%
\pgfpathmoveto{\pgfqpoint{4.469317in}{15.412639in}}%
\pgfpathlineto{\pgfqpoint{4.469317in}{15.933211in}}%
\pgfusepath{stroke}%
\end{pgfscope}%
\begin{pgfscope}%
\pgfpathrectangle{\pgfqpoint{2.125000in}{15.141860in}}{\pgfqpoint{5.489583in}{0.877907in}}%
\pgfusepath{clip}%
\pgfsetbuttcap%
\pgfsetroundjoin%
\pgfsetlinewidth{1.505625pt}%
\definecolor{currentstroke}{rgb}{0.000000,0.000000,0.000000}%
\pgfsetstrokecolor{currentstroke}%
\pgfsetdash{}{0pt}%
\pgfpathmoveto{\pgfqpoint{4.592540in}{15.412639in}}%
\pgfpathlineto{\pgfqpoint{4.592540in}{15.930379in}}%
\pgfusepath{stroke}%
\end{pgfscope}%
\begin{pgfscope}%
\pgfpathrectangle{\pgfqpoint{2.125000in}{15.141860in}}{\pgfqpoint{5.489583in}{0.877907in}}%
\pgfusepath{clip}%
\pgfsetbuttcap%
\pgfsetroundjoin%
\pgfsetlinewidth{1.505625pt}%
\definecolor{currentstroke}{rgb}{0.000000,0.000000,0.000000}%
\pgfsetstrokecolor{currentstroke}%
\pgfsetdash{}{0pt}%
\pgfpathmoveto{\pgfqpoint{4.715763in}{15.412639in}}%
\pgfpathlineto{\pgfqpoint{4.715763in}{15.927532in}}%
\pgfusepath{stroke}%
\end{pgfscope}%
\begin{pgfscope}%
\pgfpathrectangle{\pgfqpoint{2.125000in}{15.141860in}}{\pgfqpoint{5.489583in}{0.877907in}}%
\pgfusepath{clip}%
\pgfsetbuttcap%
\pgfsetroundjoin%
\pgfsetlinewidth{1.505625pt}%
\definecolor{currentstroke}{rgb}{0.000000,0.000000,0.000000}%
\pgfsetstrokecolor{currentstroke}%
\pgfsetdash{}{0pt}%
\pgfpathmoveto{\pgfqpoint{4.838986in}{15.412639in}}%
\pgfpathlineto{\pgfqpoint{4.838986in}{15.924879in}}%
\pgfusepath{stroke}%
\end{pgfscope}%
\begin{pgfscope}%
\pgfpathrectangle{\pgfqpoint{2.125000in}{15.141860in}}{\pgfqpoint{5.489583in}{0.877907in}}%
\pgfusepath{clip}%
\pgfsetbuttcap%
\pgfsetroundjoin%
\pgfsetlinewidth{1.505625pt}%
\definecolor{currentstroke}{rgb}{0.000000,0.000000,0.000000}%
\pgfsetstrokecolor{currentstroke}%
\pgfsetdash{}{0pt}%
\pgfpathmoveto{\pgfqpoint{4.962209in}{15.412639in}}%
\pgfpathlineto{\pgfqpoint{4.962209in}{15.922206in}}%
\pgfusepath{stroke}%
\end{pgfscope}%
\begin{pgfscope}%
\pgfpathrectangle{\pgfqpoint{2.125000in}{15.141860in}}{\pgfqpoint{5.489583in}{0.877907in}}%
\pgfusepath{clip}%
\pgfsetbuttcap%
\pgfsetroundjoin%
\pgfsetlinewidth{1.505625pt}%
\definecolor{currentstroke}{rgb}{0.000000,0.000000,0.000000}%
\pgfsetstrokecolor{currentstroke}%
\pgfsetdash{}{0pt}%
\pgfpathmoveto{\pgfqpoint{5.085432in}{15.412639in}}%
\pgfpathlineto{\pgfqpoint{5.085432in}{15.919575in}}%
\pgfusepath{stroke}%
\end{pgfscope}%
\begin{pgfscope}%
\pgfpathrectangle{\pgfqpoint{2.125000in}{15.141860in}}{\pgfqpoint{5.489583in}{0.877907in}}%
\pgfusepath{clip}%
\pgfsetbuttcap%
\pgfsetroundjoin%
\pgfsetlinewidth{1.505625pt}%
\definecolor{currentstroke}{rgb}{0.000000,0.000000,0.000000}%
\pgfsetstrokecolor{currentstroke}%
\pgfsetdash{}{0pt}%
\pgfpathmoveto{\pgfqpoint{5.208655in}{15.412639in}}%
\pgfpathlineto{\pgfqpoint{5.208655in}{15.917107in}}%
\pgfusepath{stroke}%
\end{pgfscope}%
\begin{pgfscope}%
\pgfpathrectangle{\pgfqpoint{2.125000in}{15.141860in}}{\pgfqpoint{5.489583in}{0.877907in}}%
\pgfusepath{clip}%
\pgfsetbuttcap%
\pgfsetroundjoin%
\pgfsetlinewidth{1.505625pt}%
\definecolor{currentstroke}{rgb}{0.000000,0.000000,0.000000}%
\pgfsetstrokecolor{currentstroke}%
\pgfsetdash{}{0pt}%
\pgfpathmoveto{\pgfqpoint{5.331878in}{15.412639in}}%
\pgfpathlineto{\pgfqpoint{5.331878in}{15.914701in}}%
\pgfusepath{stroke}%
\end{pgfscope}%
\begin{pgfscope}%
\pgfpathrectangle{\pgfqpoint{2.125000in}{15.141860in}}{\pgfqpoint{5.489583in}{0.877907in}}%
\pgfusepath{clip}%
\pgfsetbuttcap%
\pgfsetroundjoin%
\pgfsetlinewidth{1.505625pt}%
\definecolor{currentstroke}{rgb}{0.000000,0.000000,0.000000}%
\pgfsetstrokecolor{currentstroke}%
\pgfsetdash{}{0pt}%
\pgfpathmoveto{\pgfqpoint{5.455101in}{15.412639in}}%
\pgfpathlineto{\pgfqpoint{5.455101in}{15.912328in}}%
\pgfusepath{stroke}%
\end{pgfscope}%
\begin{pgfscope}%
\pgfpathrectangle{\pgfqpoint{2.125000in}{15.141860in}}{\pgfqpoint{5.489583in}{0.877907in}}%
\pgfusepath{clip}%
\pgfsetbuttcap%
\pgfsetroundjoin%
\pgfsetlinewidth{1.505625pt}%
\definecolor{currentstroke}{rgb}{0.000000,0.000000,0.000000}%
\pgfsetstrokecolor{currentstroke}%
\pgfsetdash{}{0pt}%
\pgfpathmoveto{\pgfqpoint{5.578324in}{15.412639in}}%
\pgfpathlineto{\pgfqpoint{5.578324in}{15.909927in}}%
\pgfusepath{stroke}%
\end{pgfscope}%
\begin{pgfscope}%
\pgfpathrectangle{\pgfqpoint{2.125000in}{15.141860in}}{\pgfqpoint{5.489583in}{0.877907in}}%
\pgfusepath{clip}%
\pgfsetbuttcap%
\pgfsetroundjoin%
\pgfsetlinewidth{1.505625pt}%
\definecolor{currentstroke}{rgb}{0.000000,0.000000,0.000000}%
\pgfsetstrokecolor{currentstroke}%
\pgfsetdash{}{0pt}%
\pgfpathmoveto{\pgfqpoint{5.701547in}{15.412639in}}%
\pgfpathlineto{\pgfqpoint{5.701547in}{15.907513in}}%
\pgfusepath{stroke}%
\end{pgfscope}%
\begin{pgfscope}%
\pgfpathrectangle{\pgfqpoint{2.125000in}{15.141860in}}{\pgfqpoint{5.489583in}{0.877907in}}%
\pgfusepath{clip}%
\pgfsetbuttcap%
\pgfsetroundjoin%
\pgfsetlinewidth{1.505625pt}%
\definecolor{currentstroke}{rgb}{0.000000,0.000000,0.000000}%
\pgfsetstrokecolor{currentstroke}%
\pgfsetdash{}{0pt}%
\pgfpathmoveto{\pgfqpoint{5.824770in}{15.412639in}}%
\pgfpathlineto{\pgfqpoint{5.824770in}{15.905015in}}%
\pgfusepath{stroke}%
\end{pgfscope}%
\begin{pgfscope}%
\pgfpathrectangle{\pgfqpoint{2.125000in}{15.141860in}}{\pgfqpoint{5.489583in}{0.877907in}}%
\pgfusepath{clip}%
\pgfsetbuttcap%
\pgfsetroundjoin%
\pgfsetlinewidth{1.505625pt}%
\definecolor{currentstroke}{rgb}{0.000000,0.000000,0.000000}%
\pgfsetstrokecolor{currentstroke}%
\pgfsetdash{}{0pt}%
\pgfpathmoveto{\pgfqpoint{5.947993in}{15.412639in}}%
\pgfpathlineto{\pgfqpoint{5.947993in}{15.902482in}}%
\pgfusepath{stroke}%
\end{pgfscope}%
\begin{pgfscope}%
\pgfpathrectangle{\pgfqpoint{2.125000in}{15.141860in}}{\pgfqpoint{5.489583in}{0.877907in}}%
\pgfusepath{clip}%
\pgfsetbuttcap%
\pgfsetroundjoin%
\pgfsetlinewidth{1.505625pt}%
\definecolor{currentstroke}{rgb}{0.000000,0.000000,0.000000}%
\pgfsetstrokecolor{currentstroke}%
\pgfsetdash{}{0pt}%
\pgfpathmoveto{\pgfqpoint{6.071216in}{15.412639in}}%
\pgfpathlineto{\pgfqpoint{6.071216in}{15.899967in}}%
\pgfusepath{stroke}%
\end{pgfscope}%
\begin{pgfscope}%
\pgfpathrectangle{\pgfqpoint{2.125000in}{15.141860in}}{\pgfqpoint{5.489583in}{0.877907in}}%
\pgfusepath{clip}%
\pgfsetbuttcap%
\pgfsetroundjoin%
\pgfsetlinewidth{1.505625pt}%
\definecolor{currentstroke}{rgb}{0.000000,0.000000,0.000000}%
\pgfsetstrokecolor{currentstroke}%
\pgfsetdash{}{0pt}%
\pgfpathmoveto{\pgfqpoint{6.194439in}{15.412639in}}%
\pgfpathlineto{\pgfqpoint{6.194439in}{15.897447in}}%
\pgfusepath{stroke}%
\end{pgfscope}%
\begin{pgfscope}%
\pgfpathrectangle{\pgfqpoint{2.125000in}{15.141860in}}{\pgfqpoint{5.489583in}{0.877907in}}%
\pgfusepath{clip}%
\pgfsetbuttcap%
\pgfsetroundjoin%
\pgfsetlinewidth{1.505625pt}%
\definecolor{currentstroke}{rgb}{0.000000,0.000000,0.000000}%
\pgfsetstrokecolor{currentstroke}%
\pgfsetdash{}{0pt}%
\pgfpathmoveto{\pgfqpoint{6.317662in}{15.412639in}}%
\pgfpathlineto{\pgfqpoint{6.317662in}{15.895061in}}%
\pgfusepath{stroke}%
\end{pgfscope}%
\begin{pgfscope}%
\pgfpathrectangle{\pgfqpoint{2.125000in}{15.141860in}}{\pgfqpoint{5.489583in}{0.877907in}}%
\pgfusepath{clip}%
\pgfsetbuttcap%
\pgfsetroundjoin%
\pgfsetlinewidth{1.505625pt}%
\definecolor{currentstroke}{rgb}{0.000000,0.000000,0.000000}%
\pgfsetstrokecolor{currentstroke}%
\pgfsetdash{}{0pt}%
\pgfpathmoveto{\pgfqpoint{6.440885in}{15.412639in}}%
\pgfpathlineto{\pgfqpoint{6.440885in}{15.892727in}}%
\pgfusepath{stroke}%
\end{pgfscope}%
\begin{pgfscope}%
\pgfpathrectangle{\pgfqpoint{2.125000in}{15.141860in}}{\pgfqpoint{5.489583in}{0.877907in}}%
\pgfusepath{clip}%
\pgfsetbuttcap%
\pgfsetroundjoin%
\pgfsetlinewidth{1.505625pt}%
\definecolor{currentstroke}{rgb}{0.000000,0.000000,0.000000}%
\pgfsetstrokecolor{currentstroke}%
\pgfsetdash{}{0pt}%
\pgfpathmoveto{\pgfqpoint{6.564108in}{15.412639in}}%
\pgfpathlineto{\pgfqpoint{6.564108in}{15.890261in}}%
\pgfusepath{stroke}%
\end{pgfscope}%
\begin{pgfscope}%
\pgfpathrectangle{\pgfqpoint{2.125000in}{15.141860in}}{\pgfqpoint{5.489583in}{0.877907in}}%
\pgfusepath{clip}%
\pgfsetbuttcap%
\pgfsetroundjoin%
\pgfsetlinewidth{1.505625pt}%
\definecolor{currentstroke}{rgb}{0.000000,0.000000,0.000000}%
\pgfsetstrokecolor{currentstroke}%
\pgfsetdash{}{0pt}%
\pgfpathmoveto{\pgfqpoint{6.687330in}{15.412639in}}%
\pgfpathlineto{\pgfqpoint{6.687330in}{15.887721in}}%
\pgfusepath{stroke}%
\end{pgfscope}%
\begin{pgfscope}%
\pgfpathrectangle{\pgfqpoint{2.125000in}{15.141860in}}{\pgfqpoint{5.489583in}{0.877907in}}%
\pgfusepath{clip}%
\pgfsetbuttcap%
\pgfsetroundjoin%
\pgfsetlinewidth{1.505625pt}%
\definecolor{currentstroke}{rgb}{0.000000,0.000000,0.000000}%
\pgfsetstrokecolor{currentstroke}%
\pgfsetdash{}{0pt}%
\pgfpathmoveto{\pgfqpoint{6.810553in}{15.412639in}}%
\pgfpathlineto{\pgfqpoint{6.810553in}{15.885132in}}%
\pgfusepath{stroke}%
\end{pgfscope}%
\begin{pgfscope}%
\pgfpathrectangle{\pgfqpoint{2.125000in}{15.141860in}}{\pgfqpoint{5.489583in}{0.877907in}}%
\pgfusepath{clip}%
\pgfsetbuttcap%
\pgfsetroundjoin%
\pgfsetlinewidth{1.505625pt}%
\definecolor{currentstroke}{rgb}{0.000000,0.000000,0.000000}%
\pgfsetstrokecolor{currentstroke}%
\pgfsetdash{}{0pt}%
\pgfpathmoveto{\pgfqpoint{6.933776in}{15.412639in}}%
\pgfpathlineto{\pgfqpoint{6.933776in}{15.882594in}}%
\pgfusepath{stroke}%
\end{pgfscope}%
\begin{pgfscope}%
\pgfpathrectangle{\pgfqpoint{2.125000in}{15.141860in}}{\pgfqpoint{5.489583in}{0.877907in}}%
\pgfusepath{clip}%
\pgfsetbuttcap%
\pgfsetroundjoin%
\pgfsetlinewidth{1.505625pt}%
\definecolor{currentstroke}{rgb}{0.000000,0.000000,0.000000}%
\pgfsetstrokecolor{currentstroke}%
\pgfsetdash{}{0pt}%
\pgfpathmoveto{\pgfqpoint{7.056999in}{15.412639in}}%
\pgfpathlineto{\pgfqpoint{7.056999in}{15.880217in}}%
\pgfusepath{stroke}%
\end{pgfscope}%
\begin{pgfscope}%
\pgfpathrectangle{\pgfqpoint{2.125000in}{15.141860in}}{\pgfqpoint{5.489583in}{0.877907in}}%
\pgfusepath{clip}%
\pgfsetbuttcap%
\pgfsetroundjoin%
\pgfsetlinewidth{1.505625pt}%
\definecolor{currentstroke}{rgb}{0.000000,0.000000,0.000000}%
\pgfsetstrokecolor{currentstroke}%
\pgfsetdash{}{0pt}%
\pgfpathmoveto{\pgfqpoint{7.180222in}{15.412639in}}%
\pgfpathlineto{\pgfqpoint{7.180222in}{15.877894in}}%
\pgfusepath{stroke}%
\end{pgfscope}%
\begin{pgfscope}%
\pgfpathrectangle{\pgfqpoint{2.125000in}{15.141860in}}{\pgfqpoint{5.489583in}{0.877907in}}%
\pgfusepath{clip}%
\pgfsetbuttcap%
\pgfsetroundjoin%
\pgfsetlinewidth{1.505625pt}%
\definecolor{currentstroke}{rgb}{0.000000,0.000000,0.000000}%
\pgfsetstrokecolor{currentstroke}%
\pgfsetdash{}{0pt}%
\pgfpathmoveto{\pgfqpoint{7.303445in}{15.412639in}}%
\pgfpathlineto{\pgfqpoint{7.303445in}{15.875623in}}%
\pgfusepath{stroke}%
\end{pgfscope}%
\begin{pgfscope}%
\pgfpathrectangle{\pgfqpoint{2.125000in}{15.141860in}}{\pgfqpoint{5.489583in}{0.877907in}}%
\pgfusepath{clip}%
\pgfsetroundcap%
\pgfsetroundjoin%
\pgfsetlinewidth{1.505625pt}%
\definecolor{currentstroke}{rgb}{0.121569,0.466667,0.705882}%
\pgfsetstrokecolor{currentstroke}%
\pgfsetdash{}{0pt}%
\pgfpathmoveto{\pgfqpoint{2.125000in}{15.412639in}}%
\pgfpathlineto{\pgfqpoint{7.614583in}{15.412639in}}%
\pgfusepath{stroke}%
\end{pgfscope}%
\begin{pgfscope}%
\pgfpathrectangle{\pgfqpoint{2.125000in}{15.141860in}}{\pgfqpoint{5.489583in}{0.877907in}}%
\pgfusepath{clip}%
\pgfsetbuttcap%
\pgfsetroundjoin%
\definecolor{currentfill}{rgb}{0.121569,0.466667,0.705882}%
\pgfsetfillcolor{currentfill}%
\pgfsetlinewidth{1.003750pt}%
\definecolor{currentstroke}{rgb}{0.121569,0.466667,0.705882}%
\pgfsetstrokecolor{currentstroke}%
\pgfsetdash{}{0pt}%
\pgfsys@defobject{currentmarker}{\pgfqpoint{-0.034722in}{-0.034722in}}{\pgfqpoint{0.034722in}{0.034722in}}{%
\pgfpathmoveto{\pgfqpoint{0.000000in}{-0.034722in}}%
\pgfpathcurveto{\pgfqpoint{0.009208in}{-0.034722in}}{\pgfqpoint{0.018041in}{-0.031064in}}{\pgfqpoint{0.024552in}{-0.024552in}}%
\pgfpathcurveto{\pgfqpoint{0.031064in}{-0.018041in}}{\pgfqpoint{0.034722in}{-0.009208in}}{\pgfqpoint{0.034722in}{0.000000in}}%
\pgfpathcurveto{\pgfqpoint{0.034722in}{0.009208in}}{\pgfqpoint{0.031064in}{0.018041in}}{\pgfqpoint{0.024552in}{0.024552in}}%
\pgfpathcurveto{\pgfqpoint{0.018041in}{0.031064in}}{\pgfqpoint{0.009208in}{0.034722in}}{\pgfqpoint{0.000000in}{0.034722in}}%
\pgfpathcurveto{\pgfqpoint{-0.009208in}{0.034722in}}{\pgfqpoint{-0.018041in}{0.031064in}}{\pgfqpoint{-0.024552in}{0.024552in}}%
\pgfpathcurveto{\pgfqpoint{-0.031064in}{0.018041in}}{\pgfqpoint{-0.034722in}{0.009208in}}{\pgfqpoint{-0.034722in}{0.000000in}}%
\pgfpathcurveto{\pgfqpoint{-0.034722in}{-0.009208in}}{\pgfqpoint{-0.031064in}{-0.018041in}}{\pgfqpoint{-0.024552in}{-0.024552in}}%
\pgfpathcurveto{\pgfqpoint{-0.018041in}{-0.031064in}}{\pgfqpoint{-0.009208in}{-0.034722in}}{\pgfqpoint{0.000000in}{-0.034722in}}%
\pgfpathclose%
\pgfusepath{stroke,fill}%
}%
\begin{pgfscope}%
\pgfsys@transformshift{2.374527in}{15.979863in}%
\pgfsys@useobject{currentmarker}{}%
\end{pgfscope}%
\begin{pgfscope}%
\pgfsys@transformshift{2.497749in}{15.976931in}%
\pgfsys@useobject{currentmarker}{}%
\end{pgfscope}%
\begin{pgfscope}%
\pgfsys@transformshift{2.620972in}{15.973898in}%
\pgfsys@useobject{currentmarker}{}%
\end{pgfscope}%
\begin{pgfscope}%
\pgfsys@transformshift{2.744195in}{15.970976in}%
\pgfsys@useobject{currentmarker}{}%
\end{pgfscope}%
\begin{pgfscope}%
\pgfsys@transformshift{2.867418in}{15.968054in}%
\pgfsys@useobject{currentmarker}{}%
\end{pgfscope}%
\begin{pgfscope}%
\pgfsys@transformshift{2.990641in}{15.965165in}%
\pgfsys@useobject{currentmarker}{}%
\end{pgfscope}%
\begin{pgfscope}%
\pgfsys@transformshift{3.113864in}{15.962414in}%
\pgfsys@useobject{currentmarker}{}%
\end{pgfscope}%
\begin{pgfscope}%
\pgfsys@transformshift{3.237087in}{15.959647in}%
\pgfsys@useobject{currentmarker}{}%
\end{pgfscope}%
\begin{pgfscope}%
\pgfsys@transformshift{3.360310in}{15.956958in}%
\pgfsys@useobject{currentmarker}{}%
\end{pgfscope}%
\begin{pgfscope}%
\pgfsys@transformshift{3.483533in}{15.954179in}%
\pgfsys@useobject{currentmarker}{}%
\end{pgfscope}%
\begin{pgfscope}%
\pgfsys@transformshift{3.606756in}{15.951552in}%
\pgfsys@useobject{currentmarker}{}%
\end{pgfscope}%
\begin{pgfscope}%
\pgfsys@transformshift{3.729979in}{15.949071in}%
\pgfsys@useobject{currentmarker}{}%
\end{pgfscope}%
\begin{pgfscope}%
\pgfsys@transformshift{3.853202in}{15.946640in}%
\pgfsys@useobject{currentmarker}{}%
\end{pgfscope}%
\begin{pgfscope}%
\pgfsys@transformshift{3.976425in}{15.944043in}%
\pgfsys@useobject{currentmarker}{}%
\end{pgfscope}%
\begin{pgfscope}%
\pgfsys@transformshift{4.099648in}{15.941321in}%
\pgfsys@useobject{currentmarker}{}%
\end{pgfscope}%
\begin{pgfscope}%
\pgfsys@transformshift{4.222871in}{15.938536in}%
\pgfsys@useobject{currentmarker}{}%
\end{pgfscope}%
\begin{pgfscope}%
\pgfsys@transformshift{4.346094in}{15.935889in}%
\pgfsys@useobject{currentmarker}{}%
\end{pgfscope}%
\begin{pgfscope}%
\pgfsys@transformshift{4.469317in}{15.933211in}%
\pgfsys@useobject{currentmarker}{}%
\end{pgfscope}%
\begin{pgfscope}%
\pgfsys@transformshift{4.592540in}{15.930379in}%
\pgfsys@useobject{currentmarker}{}%
\end{pgfscope}%
\begin{pgfscope}%
\pgfsys@transformshift{4.715763in}{15.927532in}%
\pgfsys@useobject{currentmarker}{}%
\end{pgfscope}%
\begin{pgfscope}%
\pgfsys@transformshift{4.838986in}{15.924879in}%
\pgfsys@useobject{currentmarker}{}%
\end{pgfscope}%
\begin{pgfscope}%
\pgfsys@transformshift{4.962209in}{15.922206in}%
\pgfsys@useobject{currentmarker}{}%
\end{pgfscope}%
\begin{pgfscope}%
\pgfsys@transformshift{5.085432in}{15.919575in}%
\pgfsys@useobject{currentmarker}{}%
\end{pgfscope}%
\begin{pgfscope}%
\pgfsys@transformshift{5.208655in}{15.917107in}%
\pgfsys@useobject{currentmarker}{}%
\end{pgfscope}%
\begin{pgfscope}%
\pgfsys@transformshift{5.331878in}{15.914701in}%
\pgfsys@useobject{currentmarker}{}%
\end{pgfscope}%
\begin{pgfscope}%
\pgfsys@transformshift{5.455101in}{15.912328in}%
\pgfsys@useobject{currentmarker}{}%
\end{pgfscope}%
\begin{pgfscope}%
\pgfsys@transformshift{5.578324in}{15.909927in}%
\pgfsys@useobject{currentmarker}{}%
\end{pgfscope}%
\begin{pgfscope}%
\pgfsys@transformshift{5.701547in}{15.907513in}%
\pgfsys@useobject{currentmarker}{}%
\end{pgfscope}%
\begin{pgfscope}%
\pgfsys@transformshift{5.824770in}{15.905015in}%
\pgfsys@useobject{currentmarker}{}%
\end{pgfscope}%
\begin{pgfscope}%
\pgfsys@transformshift{5.947993in}{15.902482in}%
\pgfsys@useobject{currentmarker}{}%
\end{pgfscope}%
\begin{pgfscope}%
\pgfsys@transformshift{6.071216in}{15.899967in}%
\pgfsys@useobject{currentmarker}{}%
\end{pgfscope}%
\begin{pgfscope}%
\pgfsys@transformshift{6.194439in}{15.897447in}%
\pgfsys@useobject{currentmarker}{}%
\end{pgfscope}%
\begin{pgfscope}%
\pgfsys@transformshift{6.317662in}{15.895061in}%
\pgfsys@useobject{currentmarker}{}%
\end{pgfscope}%
\begin{pgfscope}%
\pgfsys@transformshift{6.440885in}{15.892727in}%
\pgfsys@useobject{currentmarker}{}%
\end{pgfscope}%
\begin{pgfscope}%
\pgfsys@transformshift{6.564108in}{15.890261in}%
\pgfsys@useobject{currentmarker}{}%
\end{pgfscope}%
\begin{pgfscope}%
\pgfsys@transformshift{6.687330in}{15.887721in}%
\pgfsys@useobject{currentmarker}{}%
\end{pgfscope}%
\begin{pgfscope}%
\pgfsys@transformshift{6.810553in}{15.885132in}%
\pgfsys@useobject{currentmarker}{}%
\end{pgfscope}%
\begin{pgfscope}%
\pgfsys@transformshift{6.933776in}{15.882594in}%
\pgfsys@useobject{currentmarker}{}%
\end{pgfscope}%
\begin{pgfscope}%
\pgfsys@transformshift{7.056999in}{15.880217in}%
\pgfsys@useobject{currentmarker}{}%
\end{pgfscope}%
\begin{pgfscope}%
\pgfsys@transformshift{7.180222in}{15.877894in}%
\pgfsys@useobject{currentmarker}{}%
\end{pgfscope}%
\begin{pgfscope}%
\pgfsys@transformshift{7.303445in}{15.875623in}%
\pgfsys@useobject{currentmarker}{}%
\end{pgfscope}%
\end{pgfscope}%
\begin{pgfscope}%
\pgfsetrectcap%
\pgfsetmiterjoin%
\pgfsetlinewidth{0.803000pt}%
\definecolor{currentstroke}{rgb}{1.000000,1.000000,1.000000}%
\pgfsetstrokecolor{currentstroke}%
\pgfsetdash{}{0pt}%
\pgfpathmoveto{\pgfqpoint{2.125000in}{15.141860in}}%
\pgfpathlineto{\pgfqpoint{2.125000in}{16.019767in}}%
\pgfusepath{stroke}%
\end{pgfscope}%
\begin{pgfscope}%
\pgfsetrectcap%
\pgfsetmiterjoin%
\pgfsetlinewidth{0.803000pt}%
\definecolor{currentstroke}{rgb}{1.000000,1.000000,1.000000}%
\pgfsetstrokecolor{currentstroke}%
\pgfsetdash{}{0pt}%
\pgfpathmoveto{\pgfqpoint{7.614583in}{15.141860in}}%
\pgfpathlineto{\pgfqpoint{7.614583in}{16.019767in}}%
\pgfusepath{stroke}%
\end{pgfscope}%
\begin{pgfscope}%
\pgfsetrectcap%
\pgfsetmiterjoin%
\pgfsetlinewidth{0.803000pt}%
\definecolor{currentstroke}{rgb}{1.000000,1.000000,1.000000}%
\pgfsetstrokecolor{currentstroke}%
\pgfsetdash{}{0pt}%
\pgfpathmoveto{\pgfqpoint{2.125000in}{15.141860in}}%
\pgfpathlineto{\pgfqpoint{7.614583in}{15.141860in}}%
\pgfusepath{stroke}%
\end{pgfscope}%
\begin{pgfscope}%
\pgfsetrectcap%
\pgfsetmiterjoin%
\pgfsetlinewidth{0.803000pt}%
\definecolor{currentstroke}{rgb}{1.000000,1.000000,1.000000}%
\pgfsetstrokecolor{currentstroke}%
\pgfsetdash{}{0pt}%
\pgfpathmoveto{\pgfqpoint{2.125000in}{16.019767in}}%
\pgfpathlineto{\pgfqpoint{7.614583in}{16.019767in}}%
\pgfusepath{stroke}%
\end{pgfscope}%
\begin{pgfscope}%
\definecolor{textcolor}{rgb}{0.150000,0.150000,0.150000}%
\pgfsetstrokecolor{textcolor}%
\pgfsetfillcolor{textcolor}%
\pgftext[x=4.869792in,y=16.103101in,,base]{\color{textcolor}\rmfamily\fontsize{16.800000}{20.160000}\selectfont Autocorrelation}%
\end{pgfscope}%
\begin{pgfscope}%
\pgfsetbuttcap%
\pgfsetmiterjoin%
\definecolor{currentfill}{rgb}{0.917647,0.917647,0.949020}%
\pgfsetfillcolor{currentfill}%
\pgfsetlinewidth{0.000000pt}%
\definecolor{currentstroke}{rgb}{0.000000,0.000000,0.000000}%
\pgfsetstrokecolor{currentstroke}%
\pgfsetstrokeopacity{0.000000}%
\pgfsetdash{}{0pt}%
\pgfpathmoveto{\pgfqpoint{9.810417in}{15.141860in}}%
\pgfpathlineto{\pgfqpoint{15.300000in}{15.141860in}}%
\pgfpathlineto{\pgfqpoint{15.300000in}{16.019767in}}%
\pgfpathlineto{\pgfqpoint{9.810417in}{16.019767in}}%
\pgfpathclose%
\pgfusepath{fill}%
\end{pgfscope}%
\begin{pgfscope}%
\pgfpathrectangle{\pgfqpoint{9.810417in}{15.141860in}}{\pgfqpoint{5.489583in}{0.877907in}}%
\pgfusepath{clip}%
\pgfsetroundcap%
\pgfsetroundjoin%
\pgfsetlinewidth{0.803000pt}%
\definecolor{currentstroke}{rgb}{1.000000,1.000000,1.000000}%
\pgfsetstrokecolor{currentstroke}%
\pgfsetdash{}{0pt}%
\pgfpathmoveto{\pgfqpoint{10.059943in}{15.141860in}}%
\pgfpathlineto{\pgfqpoint{10.059943in}{16.019767in}}%
\pgfusepath{stroke}%
\end{pgfscope}%
\begin{pgfscope}%
\definecolor{textcolor}{rgb}{0.150000,0.150000,0.150000}%
\pgfsetstrokecolor{textcolor}%
\pgfsetfillcolor{textcolor}%
\pgftext[x=10.059943in,y=15.044638in,,top]{\color{textcolor}\rmfamily\fontsize{14.000000}{16.800000}\selectfont 0}%
\end{pgfscope}%
\begin{pgfscope}%
\pgfpathrectangle{\pgfqpoint{9.810417in}{15.141860in}}{\pgfqpoint{5.489583in}{0.877907in}}%
\pgfusepath{clip}%
\pgfsetroundcap%
\pgfsetroundjoin%
\pgfsetlinewidth{0.803000pt}%
\definecolor{currentstroke}{rgb}{1.000000,1.000000,1.000000}%
\pgfsetstrokecolor{currentstroke}%
\pgfsetdash{}{0pt}%
\pgfpathmoveto{\pgfqpoint{10.676058in}{15.141860in}}%
\pgfpathlineto{\pgfqpoint{10.676058in}{16.019767in}}%
\pgfusepath{stroke}%
\end{pgfscope}%
\begin{pgfscope}%
\definecolor{textcolor}{rgb}{0.150000,0.150000,0.150000}%
\pgfsetstrokecolor{textcolor}%
\pgfsetfillcolor{textcolor}%
\pgftext[x=10.676058in,y=15.044638in,,top]{\color{textcolor}\rmfamily\fontsize{14.000000}{16.800000}\selectfont 5}%
\end{pgfscope}%
\begin{pgfscope}%
\pgfpathrectangle{\pgfqpoint{9.810417in}{15.141860in}}{\pgfqpoint{5.489583in}{0.877907in}}%
\pgfusepath{clip}%
\pgfsetroundcap%
\pgfsetroundjoin%
\pgfsetlinewidth{0.803000pt}%
\definecolor{currentstroke}{rgb}{1.000000,1.000000,1.000000}%
\pgfsetstrokecolor{currentstroke}%
\pgfsetdash{}{0pt}%
\pgfpathmoveto{\pgfqpoint{11.292173in}{15.141860in}}%
\pgfpathlineto{\pgfqpoint{11.292173in}{16.019767in}}%
\pgfusepath{stroke}%
\end{pgfscope}%
\begin{pgfscope}%
\definecolor{textcolor}{rgb}{0.150000,0.150000,0.150000}%
\pgfsetstrokecolor{textcolor}%
\pgfsetfillcolor{textcolor}%
\pgftext[x=11.292173in,y=15.044638in,,top]{\color{textcolor}\rmfamily\fontsize{14.000000}{16.800000}\selectfont 10}%
\end{pgfscope}%
\begin{pgfscope}%
\pgfpathrectangle{\pgfqpoint{9.810417in}{15.141860in}}{\pgfqpoint{5.489583in}{0.877907in}}%
\pgfusepath{clip}%
\pgfsetroundcap%
\pgfsetroundjoin%
\pgfsetlinewidth{0.803000pt}%
\definecolor{currentstroke}{rgb}{1.000000,1.000000,1.000000}%
\pgfsetstrokecolor{currentstroke}%
\pgfsetdash{}{0pt}%
\pgfpathmoveto{\pgfqpoint{11.908288in}{15.141860in}}%
\pgfpathlineto{\pgfqpoint{11.908288in}{16.019767in}}%
\pgfusepath{stroke}%
\end{pgfscope}%
\begin{pgfscope}%
\definecolor{textcolor}{rgb}{0.150000,0.150000,0.150000}%
\pgfsetstrokecolor{textcolor}%
\pgfsetfillcolor{textcolor}%
\pgftext[x=11.908288in,y=15.044638in,,top]{\color{textcolor}\rmfamily\fontsize{14.000000}{16.800000}\selectfont 15}%
\end{pgfscope}%
\begin{pgfscope}%
\pgfpathrectangle{\pgfqpoint{9.810417in}{15.141860in}}{\pgfqpoint{5.489583in}{0.877907in}}%
\pgfusepath{clip}%
\pgfsetroundcap%
\pgfsetroundjoin%
\pgfsetlinewidth{0.803000pt}%
\definecolor{currentstroke}{rgb}{1.000000,1.000000,1.000000}%
\pgfsetstrokecolor{currentstroke}%
\pgfsetdash{}{0pt}%
\pgfpathmoveto{\pgfqpoint{12.524403in}{15.141860in}}%
\pgfpathlineto{\pgfqpoint{12.524403in}{16.019767in}}%
\pgfusepath{stroke}%
\end{pgfscope}%
\begin{pgfscope}%
\definecolor{textcolor}{rgb}{0.150000,0.150000,0.150000}%
\pgfsetstrokecolor{textcolor}%
\pgfsetfillcolor{textcolor}%
\pgftext[x=12.524403in,y=15.044638in,,top]{\color{textcolor}\rmfamily\fontsize{14.000000}{16.800000}\selectfont 20}%
\end{pgfscope}%
\begin{pgfscope}%
\pgfpathrectangle{\pgfqpoint{9.810417in}{15.141860in}}{\pgfqpoint{5.489583in}{0.877907in}}%
\pgfusepath{clip}%
\pgfsetroundcap%
\pgfsetroundjoin%
\pgfsetlinewidth{0.803000pt}%
\definecolor{currentstroke}{rgb}{1.000000,1.000000,1.000000}%
\pgfsetstrokecolor{currentstroke}%
\pgfsetdash{}{0pt}%
\pgfpathmoveto{\pgfqpoint{13.140517in}{15.141860in}}%
\pgfpathlineto{\pgfqpoint{13.140517in}{16.019767in}}%
\pgfusepath{stroke}%
\end{pgfscope}%
\begin{pgfscope}%
\definecolor{textcolor}{rgb}{0.150000,0.150000,0.150000}%
\pgfsetstrokecolor{textcolor}%
\pgfsetfillcolor{textcolor}%
\pgftext[x=13.140517in,y=15.044638in,,top]{\color{textcolor}\rmfamily\fontsize{14.000000}{16.800000}\selectfont 25}%
\end{pgfscope}%
\begin{pgfscope}%
\pgfpathrectangle{\pgfqpoint{9.810417in}{15.141860in}}{\pgfqpoint{5.489583in}{0.877907in}}%
\pgfusepath{clip}%
\pgfsetroundcap%
\pgfsetroundjoin%
\pgfsetlinewidth{0.803000pt}%
\definecolor{currentstroke}{rgb}{1.000000,1.000000,1.000000}%
\pgfsetstrokecolor{currentstroke}%
\pgfsetdash{}{0pt}%
\pgfpathmoveto{\pgfqpoint{13.756632in}{15.141860in}}%
\pgfpathlineto{\pgfqpoint{13.756632in}{16.019767in}}%
\pgfusepath{stroke}%
\end{pgfscope}%
\begin{pgfscope}%
\definecolor{textcolor}{rgb}{0.150000,0.150000,0.150000}%
\pgfsetstrokecolor{textcolor}%
\pgfsetfillcolor{textcolor}%
\pgftext[x=13.756632in,y=15.044638in,,top]{\color{textcolor}\rmfamily\fontsize{14.000000}{16.800000}\selectfont 30}%
\end{pgfscope}%
\begin{pgfscope}%
\pgfpathrectangle{\pgfqpoint{9.810417in}{15.141860in}}{\pgfqpoint{5.489583in}{0.877907in}}%
\pgfusepath{clip}%
\pgfsetroundcap%
\pgfsetroundjoin%
\pgfsetlinewidth{0.803000pt}%
\definecolor{currentstroke}{rgb}{1.000000,1.000000,1.000000}%
\pgfsetstrokecolor{currentstroke}%
\pgfsetdash{}{0pt}%
\pgfpathmoveto{\pgfqpoint{14.372747in}{15.141860in}}%
\pgfpathlineto{\pgfqpoint{14.372747in}{16.019767in}}%
\pgfusepath{stroke}%
\end{pgfscope}%
\begin{pgfscope}%
\definecolor{textcolor}{rgb}{0.150000,0.150000,0.150000}%
\pgfsetstrokecolor{textcolor}%
\pgfsetfillcolor{textcolor}%
\pgftext[x=14.372747in,y=15.044638in,,top]{\color{textcolor}\rmfamily\fontsize{14.000000}{16.800000}\selectfont 35}%
\end{pgfscope}%
\begin{pgfscope}%
\pgfpathrectangle{\pgfqpoint{9.810417in}{15.141860in}}{\pgfqpoint{5.489583in}{0.877907in}}%
\pgfusepath{clip}%
\pgfsetroundcap%
\pgfsetroundjoin%
\pgfsetlinewidth{0.803000pt}%
\definecolor{currentstroke}{rgb}{1.000000,1.000000,1.000000}%
\pgfsetstrokecolor{currentstroke}%
\pgfsetdash{}{0pt}%
\pgfpathmoveto{\pgfqpoint{14.988862in}{15.141860in}}%
\pgfpathlineto{\pgfqpoint{14.988862in}{16.019767in}}%
\pgfusepath{stroke}%
\end{pgfscope}%
\begin{pgfscope}%
\definecolor{textcolor}{rgb}{0.150000,0.150000,0.150000}%
\pgfsetstrokecolor{textcolor}%
\pgfsetfillcolor{textcolor}%
\pgftext[x=14.988862in,y=15.044638in,,top]{\color{textcolor}\rmfamily\fontsize{14.000000}{16.800000}\selectfont 40}%
\end{pgfscope}%
\begin{pgfscope}%
\pgfpathrectangle{\pgfqpoint{9.810417in}{15.141860in}}{\pgfqpoint{5.489583in}{0.877907in}}%
\pgfusepath{clip}%
\pgfsetroundcap%
\pgfsetroundjoin%
\pgfsetlinewidth{0.803000pt}%
\definecolor{currentstroke}{rgb}{1.000000,1.000000,1.000000}%
\pgfsetstrokecolor{currentstroke}%
\pgfsetdash{}{0pt}%
\pgfpathmoveto{\pgfqpoint{9.810417in}{15.220099in}}%
\pgfpathlineto{\pgfqpoint{15.300000in}{15.220099in}}%
\pgfusepath{stroke}%
\end{pgfscope}%
\begin{pgfscope}%
\definecolor{textcolor}{rgb}{0.150000,0.150000,0.150000}%
\pgfsetstrokecolor{textcolor}%
\pgfsetfillcolor{textcolor}%
\pgftext[x=9.589483in,y=15.146233in,left,base]{\color{textcolor}\rmfamily\fontsize{14.000000}{16.800000}\selectfont 0}%
\end{pgfscope}%
\begin{pgfscope}%
\pgfpathrectangle{\pgfqpoint{9.810417in}{15.141860in}}{\pgfqpoint{5.489583in}{0.877907in}}%
\pgfusepath{clip}%
\pgfsetroundcap%
\pgfsetroundjoin%
\pgfsetlinewidth{0.803000pt}%
\definecolor{currentstroke}{rgb}{1.000000,1.000000,1.000000}%
\pgfsetstrokecolor{currentstroke}%
\pgfsetdash{}{0pt}%
\pgfpathmoveto{\pgfqpoint{9.810417in}{15.979863in}}%
\pgfpathlineto{\pgfqpoint{15.300000in}{15.979863in}}%
\pgfusepath{stroke}%
\end{pgfscope}%
\begin{pgfscope}%
\definecolor{textcolor}{rgb}{0.150000,0.150000,0.150000}%
\pgfsetstrokecolor{textcolor}%
\pgfsetfillcolor{textcolor}%
\pgftext[x=9.589483in,y=15.905996in,left,base]{\color{textcolor}\rmfamily\fontsize{14.000000}{16.800000}\selectfont 1}%
\end{pgfscope}%
\begin{pgfscope}%
\pgfpathrectangle{\pgfqpoint{9.810417in}{15.141860in}}{\pgfqpoint{5.489583in}{0.877907in}}%
\pgfusepath{clip}%
\pgfsetbuttcap%
\pgfsetroundjoin%
\definecolor{currentfill}{rgb}{0.121569,0.466667,0.705882}%
\pgfsetfillcolor{currentfill}%
\pgfsetfillopacity{0.250000}%
\pgfsetlinewidth{1.003750pt}%
\definecolor{currentstroke}{rgb}{1.000000,1.000000,1.000000}%
\pgfsetstrokecolor{currentstroke}%
\pgfsetstrokeopacity{0.250000}%
\pgfsetdash{}{0pt}%
\pgfpathmoveto{\pgfqpoint{10.121555in}{15.258433in}}%
\pgfpathlineto{\pgfqpoint{10.121555in}{15.181765in}}%
\pgfpathlineto{\pgfqpoint{10.306389in}{15.181765in}}%
\pgfpathlineto{\pgfqpoint{10.429612in}{15.181765in}}%
\pgfpathlineto{\pgfqpoint{10.552835in}{15.181765in}}%
\pgfpathlineto{\pgfqpoint{10.676058in}{15.181765in}}%
\pgfpathlineto{\pgfqpoint{10.799281in}{15.181765in}}%
\pgfpathlineto{\pgfqpoint{10.922504in}{15.181765in}}%
\pgfpathlineto{\pgfqpoint{11.045727in}{15.181765in}}%
\pgfpathlineto{\pgfqpoint{11.168950in}{15.181765in}}%
\pgfpathlineto{\pgfqpoint{11.292173in}{15.181765in}}%
\pgfpathlineto{\pgfqpoint{11.415396in}{15.181765in}}%
\pgfpathlineto{\pgfqpoint{11.538619in}{15.181765in}}%
\pgfpathlineto{\pgfqpoint{11.661842in}{15.181765in}}%
\pgfpathlineto{\pgfqpoint{11.785065in}{15.181765in}}%
\pgfpathlineto{\pgfqpoint{11.908288in}{15.181765in}}%
\pgfpathlineto{\pgfqpoint{12.031511in}{15.181765in}}%
\pgfpathlineto{\pgfqpoint{12.154734in}{15.181765in}}%
\pgfpathlineto{\pgfqpoint{12.277957in}{15.181765in}}%
\pgfpathlineto{\pgfqpoint{12.401180in}{15.181765in}}%
\pgfpathlineto{\pgfqpoint{12.524403in}{15.181765in}}%
\pgfpathlineto{\pgfqpoint{12.647626in}{15.181765in}}%
\pgfpathlineto{\pgfqpoint{12.770849in}{15.181765in}}%
\pgfpathlineto{\pgfqpoint{12.894072in}{15.181765in}}%
\pgfpathlineto{\pgfqpoint{13.017294in}{15.181765in}}%
\pgfpathlineto{\pgfqpoint{13.140517in}{15.181765in}}%
\pgfpathlineto{\pgfqpoint{13.263740in}{15.181765in}}%
\pgfpathlineto{\pgfqpoint{13.386963in}{15.181765in}}%
\pgfpathlineto{\pgfqpoint{13.510186in}{15.181765in}}%
\pgfpathlineto{\pgfqpoint{13.633409in}{15.181765in}}%
\pgfpathlineto{\pgfqpoint{13.756632in}{15.181765in}}%
\pgfpathlineto{\pgfqpoint{13.879855in}{15.181765in}}%
\pgfpathlineto{\pgfqpoint{14.003078in}{15.181765in}}%
\pgfpathlineto{\pgfqpoint{14.126301in}{15.181765in}}%
\pgfpathlineto{\pgfqpoint{14.249524in}{15.181765in}}%
\pgfpathlineto{\pgfqpoint{14.372747in}{15.181765in}}%
\pgfpathlineto{\pgfqpoint{14.495970in}{15.181765in}}%
\pgfpathlineto{\pgfqpoint{14.619193in}{15.181765in}}%
\pgfpathlineto{\pgfqpoint{14.742416in}{15.181765in}}%
\pgfpathlineto{\pgfqpoint{14.865639in}{15.181765in}}%
\pgfpathlineto{\pgfqpoint{15.050473in}{15.181765in}}%
\pgfpathlineto{\pgfqpoint{15.050473in}{15.258433in}}%
\pgfpathlineto{\pgfqpoint{15.050473in}{15.258433in}}%
\pgfpathlineto{\pgfqpoint{14.865639in}{15.258433in}}%
\pgfpathlineto{\pgfqpoint{14.742416in}{15.258433in}}%
\pgfpathlineto{\pgfqpoint{14.619193in}{15.258433in}}%
\pgfpathlineto{\pgfqpoint{14.495970in}{15.258433in}}%
\pgfpathlineto{\pgfqpoint{14.372747in}{15.258433in}}%
\pgfpathlineto{\pgfqpoint{14.249524in}{15.258433in}}%
\pgfpathlineto{\pgfqpoint{14.126301in}{15.258433in}}%
\pgfpathlineto{\pgfqpoint{14.003078in}{15.258433in}}%
\pgfpathlineto{\pgfqpoint{13.879855in}{15.258433in}}%
\pgfpathlineto{\pgfqpoint{13.756632in}{15.258433in}}%
\pgfpathlineto{\pgfqpoint{13.633409in}{15.258433in}}%
\pgfpathlineto{\pgfqpoint{13.510186in}{15.258433in}}%
\pgfpathlineto{\pgfqpoint{13.386963in}{15.258433in}}%
\pgfpathlineto{\pgfqpoint{13.263740in}{15.258433in}}%
\pgfpathlineto{\pgfqpoint{13.140517in}{15.258433in}}%
\pgfpathlineto{\pgfqpoint{13.017294in}{15.258433in}}%
\pgfpathlineto{\pgfqpoint{12.894072in}{15.258433in}}%
\pgfpathlineto{\pgfqpoint{12.770849in}{15.258433in}}%
\pgfpathlineto{\pgfqpoint{12.647626in}{15.258433in}}%
\pgfpathlineto{\pgfqpoint{12.524403in}{15.258433in}}%
\pgfpathlineto{\pgfqpoint{12.401180in}{15.258433in}}%
\pgfpathlineto{\pgfqpoint{12.277957in}{15.258433in}}%
\pgfpathlineto{\pgfqpoint{12.154734in}{15.258433in}}%
\pgfpathlineto{\pgfqpoint{12.031511in}{15.258433in}}%
\pgfpathlineto{\pgfqpoint{11.908288in}{15.258433in}}%
\pgfpathlineto{\pgfqpoint{11.785065in}{15.258433in}}%
\pgfpathlineto{\pgfqpoint{11.661842in}{15.258433in}}%
\pgfpathlineto{\pgfqpoint{11.538619in}{15.258433in}}%
\pgfpathlineto{\pgfqpoint{11.415396in}{15.258433in}}%
\pgfpathlineto{\pgfqpoint{11.292173in}{15.258433in}}%
\pgfpathlineto{\pgfqpoint{11.168950in}{15.258433in}}%
\pgfpathlineto{\pgfqpoint{11.045727in}{15.258433in}}%
\pgfpathlineto{\pgfqpoint{10.922504in}{15.258433in}}%
\pgfpathlineto{\pgfqpoint{10.799281in}{15.258433in}}%
\pgfpathlineto{\pgfqpoint{10.676058in}{15.258433in}}%
\pgfpathlineto{\pgfqpoint{10.552835in}{15.258433in}}%
\pgfpathlineto{\pgfqpoint{10.429612in}{15.258433in}}%
\pgfpathlineto{\pgfqpoint{10.306389in}{15.258433in}}%
\pgfpathlineto{\pgfqpoint{10.121555in}{15.258433in}}%
\pgfpathclose%
\pgfusepath{stroke,fill}%
\end{pgfscope}%
\begin{pgfscope}%
\pgfpathrectangle{\pgfqpoint{9.810417in}{15.141860in}}{\pgfqpoint{5.489583in}{0.877907in}}%
\pgfusepath{clip}%
\pgfsetbuttcap%
\pgfsetroundjoin%
\pgfsetlinewidth{1.505625pt}%
\definecolor{currentstroke}{rgb}{0.000000,0.000000,0.000000}%
\pgfsetstrokecolor{currentstroke}%
\pgfsetdash{}{0pt}%
\pgfpathmoveto{\pgfqpoint{10.059943in}{15.220099in}}%
\pgfpathlineto{\pgfqpoint{10.059943in}{15.979863in}}%
\pgfusepath{stroke}%
\end{pgfscope}%
\begin{pgfscope}%
\pgfpathrectangle{\pgfqpoint{9.810417in}{15.141860in}}{\pgfqpoint{5.489583in}{0.877907in}}%
\pgfusepath{clip}%
\pgfsetbuttcap%
\pgfsetroundjoin%
\pgfsetlinewidth{1.505625pt}%
\definecolor{currentstroke}{rgb}{0.000000,0.000000,0.000000}%
\pgfsetstrokecolor{currentstroke}%
\pgfsetdash{}{0pt}%
\pgfpathmoveto{\pgfqpoint{10.183166in}{15.220099in}}%
\pgfpathlineto{\pgfqpoint{10.183166in}{15.976437in}}%
\pgfusepath{stroke}%
\end{pgfscope}%
\begin{pgfscope}%
\pgfpathrectangle{\pgfqpoint{9.810417in}{15.141860in}}{\pgfqpoint{5.489583in}{0.877907in}}%
\pgfusepath{clip}%
\pgfsetbuttcap%
\pgfsetroundjoin%
\pgfsetlinewidth{1.505625pt}%
\definecolor{currentstroke}{rgb}{0.000000,0.000000,0.000000}%
\pgfsetstrokecolor{currentstroke}%
\pgfsetdash{}{0pt}%
\pgfpathmoveto{\pgfqpoint{10.306389in}{15.220099in}}%
\pgfpathlineto{\pgfqpoint{10.306389in}{15.202667in}}%
\pgfusepath{stroke}%
\end{pgfscope}%
\begin{pgfscope}%
\pgfpathrectangle{\pgfqpoint{9.810417in}{15.141860in}}{\pgfqpoint{5.489583in}{0.877907in}}%
\pgfusepath{clip}%
\pgfsetbuttcap%
\pgfsetroundjoin%
\pgfsetlinewidth{1.505625pt}%
\definecolor{currentstroke}{rgb}{0.000000,0.000000,0.000000}%
\pgfsetstrokecolor{currentstroke}%
\pgfsetdash{}{0pt}%
\pgfpathmoveto{\pgfqpoint{10.429612in}{15.220099in}}%
\pgfpathlineto{\pgfqpoint{10.429612in}{15.234885in}}%
\pgfusepath{stroke}%
\end{pgfscope}%
\begin{pgfscope}%
\pgfpathrectangle{\pgfqpoint{9.810417in}{15.141860in}}{\pgfqpoint{5.489583in}{0.877907in}}%
\pgfusepath{clip}%
\pgfsetbuttcap%
\pgfsetroundjoin%
\pgfsetlinewidth{1.505625pt}%
\definecolor{currentstroke}{rgb}{0.000000,0.000000,0.000000}%
\pgfsetstrokecolor{currentstroke}%
\pgfsetdash{}{0pt}%
\pgfpathmoveto{\pgfqpoint{10.552835in}{15.220099in}}%
\pgfpathlineto{\pgfqpoint{10.552835in}{15.216929in}}%
\pgfusepath{stroke}%
\end{pgfscope}%
\begin{pgfscope}%
\pgfpathrectangle{\pgfqpoint{9.810417in}{15.141860in}}{\pgfqpoint{5.489583in}{0.877907in}}%
\pgfusepath{clip}%
\pgfsetbuttcap%
\pgfsetroundjoin%
\pgfsetlinewidth{1.505625pt}%
\definecolor{currentstroke}{rgb}{0.000000,0.000000,0.000000}%
\pgfsetstrokecolor{currentstroke}%
\pgfsetdash{}{0pt}%
\pgfpathmoveto{\pgfqpoint{10.676058in}{15.220099in}}%
\pgfpathlineto{\pgfqpoint{10.676058in}{15.223376in}}%
\pgfusepath{stroke}%
\end{pgfscope}%
\begin{pgfscope}%
\pgfpathrectangle{\pgfqpoint{9.810417in}{15.141860in}}{\pgfqpoint{5.489583in}{0.877907in}}%
\pgfusepath{clip}%
\pgfsetbuttcap%
\pgfsetroundjoin%
\pgfsetlinewidth{1.505625pt}%
\definecolor{currentstroke}{rgb}{0.000000,0.000000,0.000000}%
\pgfsetstrokecolor{currentstroke}%
\pgfsetdash{}{0pt}%
\pgfpathmoveto{\pgfqpoint{10.799281in}{15.220099in}}%
\pgfpathlineto{\pgfqpoint{10.799281in}{15.238183in}}%
\pgfusepath{stroke}%
\end{pgfscope}%
\begin{pgfscope}%
\pgfpathrectangle{\pgfqpoint{9.810417in}{15.141860in}}{\pgfqpoint{5.489583in}{0.877907in}}%
\pgfusepath{clip}%
\pgfsetbuttcap%
\pgfsetroundjoin%
\pgfsetlinewidth{1.505625pt}%
\definecolor{currentstroke}{rgb}{0.000000,0.000000,0.000000}%
\pgfsetstrokecolor{currentstroke}%
\pgfsetdash{}{0pt}%
\pgfpathmoveto{\pgfqpoint{10.922504in}{15.220099in}}%
\pgfpathlineto{\pgfqpoint{10.922504in}{15.215006in}}%
\pgfusepath{stroke}%
\end{pgfscope}%
\begin{pgfscope}%
\pgfpathrectangle{\pgfqpoint{9.810417in}{15.141860in}}{\pgfqpoint{5.489583in}{0.877907in}}%
\pgfusepath{clip}%
\pgfsetbuttcap%
\pgfsetroundjoin%
\pgfsetlinewidth{1.505625pt}%
\definecolor{currentstroke}{rgb}{0.000000,0.000000,0.000000}%
\pgfsetstrokecolor{currentstroke}%
\pgfsetdash{}{0pt}%
\pgfpathmoveto{\pgfqpoint{11.045727in}{15.220099in}}%
\pgfpathlineto{\pgfqpoint{11.045727in}{15.230690in}}%
\pgfusepath{stroke}%
\end{pgfscope}%
\begin{pgfscope}%
\pgfpathrectangle{\pgfqpoint{9.810417in}{15.141860in}}{\pgfqpoint{5.489583in}{0.877907in}}%
\pgfusepath{clip}%
\pgfsetbuttcap%
\pgfsetroundjoin%
\pgfsetlinewidth{1.505625pt}%
\definecolor{currentstroke}{rgb}{0.000000,0.000000,0.000000}%
\pgfsetstrokecolor{currentstroke}%
\pgfsetdash{}{0pt}%
\pgfpathmoveto{\pgfqpoint{11.168950in}{15.220099in}}%
\pgfpathlineto{\pgfqpoint{11.168950in}{15.203849in}}%
\pgfusepath{stroke}%
\end{pgfscope}%
\begin{pgfscope}%
\pgfpathrectangle{\pgfqpoint{9.810417in}{15.141860in}}{\pgfqpoint{5.489583in}{0.877907in}}%
\pgfusepath{clip}%
\pgfsetbuttcap%
\pgfsetroundjoin%
\pgfsetlinewidth{1.505625pt}%
\definecolor{currentstroke}{rgb}{0.000000,0.000000,0.000000}%
\pgfsetstrokecolor{currentstroke}%
\pgfsetdash{}{0pt}%
\pgfpathmoveto{\pgfqpoint{11.292173in}{15.220099in}}%
\pgfpathlineto{\pgfqpoint{11.292173in}{15.242116in}}%
\pgfusepath{stroke}%
\end{pgfscope}%
\begin{pgfscope}%
\pgfpathrectangle{\pgfqpoint{9.810417in}{15.141860in}}{\pgfqpoint{5.489583in}{0.877907in}}%
\pgfusepath{clip}%
\pgfsetbuttcap%
\pgfsetroundjoin%
\pgfsetlinewidth{1.505625pt}%
\definecolor{currentstroke}{rgb}{0.000000,0.000000,0.000000}%
\pgfsetstrokecolor{currentstroke}%
\pgfsetdash{}{0pt}%
\pgfpathmoveto{\pgfqpoint{11.415396in}{15.220099in}}%
\pgfpathlineto{\pgfqpoint{11.415396in}{15.238844in}}%
\pgfusepath{stroke}%
\end{pgfscope}%
\begin{pgfscope}%
\pgfpathrectangle{\pgfqpoint{9.810417in}{15.141860in}}{\pgfqpoint{5.489583in}{0.877907in}}%
\pgfusepath{clip}%
\pgfsetbuttcap%
\pgfsetroundjoin%
\pgfsetlinewidth{1.505625pt}%
\definecolor{currentstroke}{rgb}{0.000000,0.000000,0.000000}%
\pgfsetstrokecolor{currentstroke}%
\pgfsetdash{}{0pt}%
\pgfpathmoveto{\pgfqpoint{11.538619in}{15.220099in}}%
\pgfpathlineto{\pgfqpoint{11.538619in}{15.225891in}}%
\pgfusepath{stroke}%
\end{pgfscope}%
\begin{pgfscope}%
\pgfpathrectangle{\pgfqpoint{9.810417in}{15.141860in}}{\pgfqpoint{5.489583in}{0.877907in}}%
\pgfusepath{clip}%
\pgfsetbuttcap%
\pgfsetroundjoin%
\pgfsetlinewidth{1.505625pt}%
\definecolor{currentstroke}{rgb}{0.000000,0.000000,0.000000}%
\pgfsetstrokecolor{currentstroke}%
\pgfsetdash{}{0pt}%
\pgfpathmoveto{\pgfqpoint{11.661842in}{15.220099in}}%
\pgfpathlineto{\pgfqpoint{11.661842in}{15.194351in}}%
\pgfusepath{stroke}%
\end{pgfscope}%
\begin{pgfscope}%
\pgfpathrectangle{\pgfqpoint{9.810417in}{15.141860in}}{\pgfqpoint{5.489583in}{0.877907in}}%
\pgfusepath{clip}%
\pgfsetbuttcap%
\pgfsetroundjoin%
\pgfsetlinewidth{1.505625pt}%
\definecolor{currentstroke}{rgb}{0.000000,0.000000,0.000000}%
\pgfsetstrokecolor{currentstroke}%
\pgfsetdash{}{0pt}%
\pgfpathmoveto{\pgfqpoint{11.785065in}{15.220099in}}%
\pgfpathlineto{\pgfqpoint{11.785065in}{15.199830in}}%
\pgfusepath{stroke}%
\end{pgfscope}%
\begin{pgfscope}%
\pgfpathrectangle{\pgfqpoint{9.810417in}{15.141860in}}{\pgfqpoint{5.489583in}{0.877907in}}%
\pgfusepath{clip}%
\pgfsetbuttcap%
\pgfsetroundjoin%
\pgfsetlinewidth{1.505625pt}%
\definecolor{currentstroke}{rgb}{0.000000,0.000000,0.000000}%
\pgfsetstrokecolor{currentstroke}%
\pgfsetdash{}{0pt}%
\pgfpathmoveto{\pgfqpoint{11.908288in}{15.220099in}}%
\pgfpathlineto{\pgfqpoint{11.908288in}{15.209845in}}%
\pgfusepath{stroke}%
\end{pgfscope}%
\begin{pgfscope}%
\pgfpathrectangle{\pgfqpoint{9.810417in}{15.141860in}}{\pgfqpoint{5.489583in}{0.877907in}}%
\pgfusepath{clip}%
\pgfsetbuttcap%
\pgfsetroundjoin%
\pgfsetlinewidth{1.505625pt}%
\definecolor{currentstroke}{rgb}{0.000000,0.000000,0.000000}%
\pgfsetstrokecolor{currentstroke}%
\pgfsetdash{}{0pt}%
\pgfpathmoveto{\pgfqpoint{12.031511in}{15.220099in}}%
\pgfpathlineto{\pgfqpoint{12.031511in}{15.239118in}}%
\pgfusepath{stroke}%
\end{pgfscope}%
\begin{pgfscope}%
\pgfpathrectangle{\pgfqpoint{9.810417in}{15.141860in}}{\pgfqpoint{5.489583in}{0.877907in}}%
\pgfusepath{clip}%
\pgfsetbuttcap%
\pgfsetroundjoin%
\pgfsetlinewidth{1.505625pt}%
\definecolor{currentstroke}{rgb}{0.000000,0.000000,0.000000}%
\pgfsetstrokecolor{currentstroke}%
\pgfsetdash{}{0pt}%
\pgfpathmoveto{\pgfqpoint{12.154734in}{15.220099in}}%
\pgfpathlineto{\pgfqpoint{12.154734in}{15.212988in}}%
\pgfusepath{stroke}%
\end{pgfscope}%
\begin{pgfscope}%
\pgfpathrectangle{\pgfqpoint{9.810417in}{15.141860in}}{\pgfqpoint{5.489583in}{0.877907in}}%
\pgfusepath{clip}%
\pgfsetbuttcap%
\pgfsetroundjoin%
\pgfsetlinewidth{1.505625pt}%
\definecolor{currentstroke}{rgb}{0.000000,0.000000,0.000000}%
\pgfsetstrokecolor{currentstroke}%
\pgfsetdash{}{0pt}%
\pgfpathmoveto{\pgfqpoint{12.277957in}{15.220099in}}%
\pgfpathlineto{\pgfqpoint{12.277957in}{15.194128in}}%
\pgfusepath{stroke}%
\end{pgfscope}%
\begin{pgfscope}%
\pgfpathrectangle{\pgfqpoint{9.810417in}{15.141860in}}{\pgfqpoint{5.489583in}{0.877907in}}%
\pgfusepath{clip}%
\pgfsetbuttcap%
\pgfsetroundjoin%
\pgfsetlinewidth{1.505625pt}%
\definecolor{currentstroke}{rgb}{0.000000,0.000000,0.000000}%
\pgfsetstrokecolor{currentstroke}%
\pgfsetdash{}{0pt}%
\pgfpathmoveto{\pgfqpoint{12.401180in}{15.220099in}}%
\pgfpathlineto{\pgfqpoint{12.401180in}{15.215222in}}%
\pgfusepath{stroke}%
\end{pgfscope}%
\begin{pgfscope}%
\pgfpathrectangle{\pgfqpoint{9.810417in}{15.141860in}}{\pgfqpoint{5.489583in}{0.877907in}}%
\pgfusepath{clip}%
\pgfsetbuttcap%
\pgfsetroundjoin%
\pgfsetlinewidth{1.505625pt}%
\definecolor{currentstroke}{rgb}{0.000000,0.000000,0.000000}%
\pgfsetstrokecolor{currentstroke}%
\pgfsetdash{}{0pt}%
\pgfpathmoveto{\pgfqpoint{12.524403in}{15.220099in}}%
\pgfpathlineto{\pgfqpoint{12.524403in}{15.246563in}}%
\pgfusepath{stroke}%
\end{pgfscope}%
\begin{pgfscope}%
\pgfpathrectangle{\pgfqpoint{9.810417in}{15.141860in}}{\pgfqpoint{5.489583in}{0.877907in}}%
\pgfusepath{clip}%
\pgfsetbuttcap%
\pgfsetroundjoin%
\pgfsetlinewidth{1.505625pt}%
\definecolor{currentstroke}{rgb}{0.000000,0.000000,0.000000}%
\pgfsetstrokecolor{currentstroke}%
\pgfsetdash{}{0pt}%
\pgfpathmoveto{\pgfqpoint{12.647626in}{15.220099in}}%
\pgfpathlineto{\pgfqpoint{12.647626in}{15.216437in}}%
\pgfusepath{stroke}%
\end{pgfscope}%
\begin{pgfscope}%
\pgfpathrectangle{\pgfqpoint{9.810417in}{15.141860in}}{\pgfqpoint{5.489583in}{0.877907in}}%
\pgfusepath{clip}%
\pgfsetbuttcap%
\pgfsetroundjoin%
\pgfsetlinewidth{1.505625pt}%
\definecolor{currentstroke}{rgb}{0.000000,0.000000,0.000000}%
\pgfsetstrokecolor{currentstroke}%
\pgfsetdash{}{0pt}%
\pgfpathmoveto{\pgfqpoint{12.770849in}{15.220099in}}%
\pgfpathlineto{\pgfqpoint{12.770849in}{15.224132in}}%
\pgfusepath{stroke}%
\end{pgfscope}%
\begin{pgfscope}%
\pgfpathrectangle{\pgfqpoint{9.810417in}{15.141860in}}{\pgfqpoint{5.489583in}{0.877907in}}%
\pgfusepath{clip}%
\pgfsetbuttcap%
\pgfsetroundjoin%
\pgfsetlinewidth{1.505625pt}%
\definecolor{currentstroke}{rgb}{0.000000,0.000000,0.000000}%
\pgfsetstrokecolor{currentstroke}%
\pgfsetdash{}{0pt}%
\pgfpathmoveto{\pgfqpoint{12.894072in}{15.220099in}}%
\pgfpathlineto{\pgfqpoint{12.894072in}{15.239570in}}%
\pgfusepath{stroke}%
\end{pgfscope}%
\begin{pgfscope}%
\pgfpathrectangle{\pgfqpoint{9.810417in}{15.141860in}}{\pgfqpoint{5.489583in}{0.877907in}}%
\pgfusepath{clip}%
\pgfsetbuttcap%
\pgfsetroundjoin%
\pgfsetlinewidth{1.505625pt}%
\definecolor{currentstroke}{rgb}{0.000000,0.000000,0.000000}%
\pgfsetstrokecolor{currentstroke}%
\pgfsetdash{}{0pt}%
\pgfpathmoveto{\pgfqpoint{13.017294in}{15.220099in}}%
\pgfpathlineto{\pgfqpoint{13.017294in}{15.224306in}}%
\pgfusepath{stroke}%
\end{pgfscope}%
\begin{pgfscope}%
\pgfpathrectangle{\pgfqpoint{9.810417in}{15.141860in}}{\pgfqpoint{5.489583in}{0.877907in}}%
\pgfusepath{clip}%
\pgfsetbuttcap%
\pgfsetroundjoin%
\pgfsetlinewidth{1.505625pt}%
\definecolor{currentstroke}{rgb}{0.000000,0.000000,0.000000}%
\pgfsetstrokecolor{currentstroke}%
\pgfsetdash{}{0pt}%
\pgfpathmoveto{\pgfqpoint{13.140517in}{15.220099in}}%
\pgfpathlineto{\pgfqpoint{13.140517in}{15.226160in}}%
\pgfusepath{stroke}%
\end{pgfscope}%
\begin{pgfscope}%
\pgfpathrectangle{\pgfqpoint{9.810417in}{15.141860in}}{\pgfqpoint{5.489583in}{0.877907in}}%
\pgfusepath{clip}%
\pgfsetbuttcap%
\pgfsetroundjoin%
\pgfsetlinewidth{1.505625pt}%
\definecolor{currentstroke}{rgb}{0.000000,0.000000,0.000000}%
\pgfsetstrokecolor{currentstroke}%
\pgfsetdash{}{0pt}%
\pgfpathmoveto{\pgfqpoint{13.263740in}{15.220099in}}%
\pgfpathlineto{\pgfqpoint{13.263740in}{15.215944in}}%
\pgfusepath{stroke}%
\end{pgfscope}%
\begin{pgfscope}%
\pgfpathrectangle{\pgfqpoint{9.810417in}{15.141860in}}{\pgfqpoint{5.489583in}{0.877907in}}%
\pgfusepath{clip}%
\pgfsetbuttcap%
\pgfsetroundjoin%
\pgfsetlinewidth{1.505625pt}%
\definecolor{currentstroke}{rgb}{0.000000,0.000000,0.000000}%
\pgfsetstrokecolor{currentstroke}%
\pgfsetdash{}{0pt}%
\pgfpathmoveto{\pgfqpoint{13.386963in}{15.220099in}}%
\pgfpathlineto{\pgfqpoint{13.386963in}{15.218044in}}%
\pgfusepath{stroke}%
\end{pgfscope}%
\begin{pgfscope}%
\pgfpathrectangle{\pgfqpoint{9.810417in}{15.141860in}}{\pgfqpoint{5.489583in}{0.877907in}}%
\pgfusepath{clip}%
\pgfsetbuttcap%
\pgfsetroundjoin%
\pgfsetlinewidth{1.505625pt}%
\definecolor{currentstroke}{rgb}{0.000000,0.000000,0.000000}%
\pgfsetstrokecolor{currentstroke}%
\pgfsetdash{}{0pt}%
\pgfpathmoveto{\pgfqpoint{13.510186in}{15.220099in}}%
\pgfpathlineto{\pgfqpoint{13.510186in}{15.202795in}}%
\pgfusepath{stroke}%
\end{pgfscope}%
\begin{pgfscope}%
\pgfpathrectangle{\pgfqpoint{9.810417in}{15.141860in}}{\pgfqpoint{5.489583in}{0.877907in}}%
\pgfusepath{clip}%
\pgfsetbuttcap%
\pgfsetroundjoin%
\pgfsetlinewidth{1.505625pt}%
\definecolor{currentstroke}{rgb}{0.000000,0.000000,0.000000}%
\pgfsetstrokecolor{currentstroke}%
\pgfsetdash{}{0pt}%
\pgfpathmoveto{\pgfqpoint{13.633409in}{15.220099in}}%
\pgfpathlineto{\pgfqpoint{13.633409in}{15.214413in}}%
\pgfusepath{stroke}%
\end{pgfscope}%
\begin{pgfscope}%
\pgfpathrectangle{\pgfqpoint{9.810417in}{15.141860in}}{\pgfqpoint{5.489583in}{0.877907in}}%
\pgfusepath{clip}%
\pgfsetbuttcap%
\pgfsetroundjoin%
\pgfsetlinewidth{1.505625pt}%
\definecolor{currentstroke}{rgb}{0.000000,0.000000,0.000000}%
\pgfsetstrokecolor{currentstroke}%
\pgfsetdash{}{0pt}%
\pgfpathmoveto{\pgfqpoint{13.756632in}{15.220099in}}%
\pgfpathlineto{\pgfqpoint{13.756632in}{15.223549in}}%
\pgfusepath{stroke}%
\end{pgfscope}%
\begin{pgfscope}%
\pgfpathrectangle{\pgfqpoint{9.810417in}{15.141860in}}{\pgfqpoint{5.489583in}{0.877907in}}%
\pgfusepath{clip}%
\pgfsetbuttcap%
\pgfsetroundjoin%
\pgfsetlinewidth{1.505625pt}%
\definecolor{currentstroke}{rgb}{0.000000,0.000000,0.000000}%
\pgfsetstrokecolor{currentstroke}%
\pgfsetdash{}{0pt}%
\pgfpathmoveto{\pgfqpoint{13.879855in}{15.220099in}}%
\pgfpathlineto{\pgfqpoint{13.879855in}{15.218953in}}%
\pgfusepath{stroke}%
\end{pgfscope}%
\begin{pgfscope}%
\pgfpathrectangle{\pgfqpoint{9.810417in}{15.141860in}}{\pgfqpoint{5.489583in}{0.877907in}}%
\pgfusepath{clip}%
\pgfsetbuttcap%
\pgfsetroundjoin%
\pgfsetlinewidth{1.505625pt}%
\definecolor{currentstroke}{rgb}{0.000000,0.000000,0.000000}%
\pgfsetstrokecolor{currentstroke}%
\pgfsetdash{}{0pt}%
\pgfpathmoveto{\pgfqpoint{14.003078in}{15.220099in}}%
\pgfpathlineto{\pgfqpoint{14.003078in}{15.238695in}}%
\pgfusepath{stroke}%
\end{pgfscope}%
\begin{pgfscope}%
\pgfpathrectangle{\pgfqpoint{9.810417in}{15.141860in}}{\pgfqpoint{5.489583in}{0.877907in}}%
\pgfusepath{clip}%
\pgfsetbuttcap%
\pgfsetroundjoin%
\pgfsetlinewidth{1.505625pt}%
\definecolor{currentstroke}{rgb}{0.000000,0.000000,0.000000}%
\pgfsetstrokecolor{currentstroke}%
\pgfsetdash{}{0pt}%
\pgfpathmoveto{\pgfqpoint{14.126301in}{15.220099in}}%
\pgfpathlineto{\pgfqpoint{14.126301in}{15.223453in}}%
\pgfusepath{stroke}%
\end{pgfscope}%
\begin{pgfscope}%
\pgfpathrectangle{\pgfqpoint{9.810417in}{15.141860in}}{\pgfqpoint{5.489583in}{0.877907in}}%
\pgfusepath{clip}%
\pgfsetbuttcap%
\pgfsetroundjoin%
\pgfsetlinewidth{1.505625pt}%
\definecolor{currentstroke}{rgb}{0.000000,0.000000,0.000000}%
\pgfsetstrokecolor{currentstroke}%
\pgfsetdash{}{0pt}%
\pgfpathmoveto{\pgfqpoint{14.249524in}{15.220099in}}%
\pgfpathlineto{\pgfqpoint{14.249524in}{15.198821in}}%
\pgfusepath{stroke}%
\end{pgfscope}%
\begin{pgfscope}%
\pgfpathrectangle{\pgfqpoint{9.810417in}{15.141860in}}{\pgfqpoint{5.489583in}{0.877907in}}%
\pgfusepath{clip}%
\pgfsetbuttcap%
\pgfsetroundjoin%
\pgfsetlinewidth{1.505625pt}%
\definecolor{currentstroke}{rgb}{0.000000,0.000000,0.000000}%
\pgfsetstrokecolor{currentstroke}%
\pgfsetdash{}{0pt}%
\pgfpathmoveto{\pgfqpoint{14.372747in}{15.220099in}}%
\pgfpathlineto{\pgfqpoint{14.372747in}{15.208401in}}%
\pgfusepath{stroke}%
\end{pgfscope}%
\begin{pgfscope}%
\pgfpathrectangle{\pgfqpoint{9.810417in}{15.141860in}}{\pgfqpoint{5.489583in}{0.877907in}}%
\pgfusepath{clip}%
\pgfsetbuttcap%
\pgfsetroundjoin%
\pgfsetlinewidth{1.505625pt}%
\definecolor{currentstroke}{rgb}{0.000000,0.000000,0.000000}%
\pgfsetstrokecolor{currentstroke}%
\pgfsetdash{}{0pt}%
\pgfpathmoveto{\pgfqpoint{14.495970in}{15.220099in}}%
\pgfpathlineto{\pgfqpoint{14.495970in}{15.208474in}}%
\pgfusepath{stroke}%
\end{pgfscope}%
\begin{pgfscope}%
\pgfpathrectangle{\pgfqpoint{9.810417in}{15.141860in}}{\pgfqpoint{5.489583in}{0.877907in}}%
\pgfusepath{clip}%
\pgfsetbuttcap%
\pgfsetroundjoin%
\pgfsetlinewidth{1.505625pt}%
\definecolor{currentstroke}{rgb}{0.000000,0.000000,0.000000}%
\pgfsetstrokecolor{currentstroke}%
\pgfsetdash{}{0pt}%
\pgfpathmoveto{\pgfqpoint{14.619193in}{15.220099in}}%
\pgfpathlineto{\pgfqpoint{14.619193in}{15.223827in}}%
\pgfusepath{stroke}%
\end{pgfscope}%
\begin{pgfscope}%
\pgfpathrectangle{\pgfqpoint{9.810417in}{15.141860in}}{\pgfqpoint{5.489583in}{0.877907in}}%
\pgfusepath{clip}%
\pgfsetbuttcap%
\pgfsetroundjoin%
\pgfsetlinewidth{1.505625pt}%
\definecolor{currentstroke}{rgb}{0.000000,0.000000,0.000000}%
\pgfsetstrokecolor{currentstroke}%
\pgfsetdash{}{0pt}%
\pgfpathmoveto{\pgfqpoint{14.742416in}{15.220099in}}%
\pgfpathlineto{\pgfqpoint{14.742416in}{15.241676in}}%
\pgfusepath{stroke}%
\end{pgfscope}%
\begin{pgfscope}%
\pgfpathrectangle{\pgfqpoint{9.810417in}{15.141860in}}{\pgfqpoint{5.489583in}{0.877907in}}%
\pgfusepath{clip}%
\pgfsetbuttcap%
\pgfsetroundjoin%
\pgfsetlinewidth{1.505625pt}%
\definecolor{currentstroke}{rgb}{0.000000,0.000000,0.000000}%
\pgfsetstrokecolor{currentstroke}%
\pgfsetdash{}{0pt}%
\pgfpathmoveto{\pgfqpoint{14.865639in}{15.220099in}}%
\pgfpathlineto{\pgfqpoint{14.865639in}{15.225734in}}%
\pgfusepath{stroke}%
\end{pgfscope}%
\begin{pgfscope}%
\pgfpathrectangle{\pgfqpoint{9.810417in}{15.141860in}}{\pgfqpoint{5.489583in}{0.877907in}}%
\pgfusepath{clip}%
\pgfsetbuttcap%
\pgfsetroundjoin%
\pgfsetlinewidth{1.505625pt}%
\definecolor{currentstroke}{rgb}{0.000000,0.000000,0.000000}%
\pgfsetstrokecolor{currentstroke}%
\pgfsetdash{}{0pt}%
\pgfpathmoveto{\pgfqpoint{14.988862in}{15.220099in}}%
\pgfpathlineto{\pgfqpoint{14.988862in}{15.225291in}}%
\pgfusepath{stroke}%
\end{pgfscope}%
\begin{pgfscope}%
\pgfpathrectangle{\pgfqpoint{9.810417in}{15.141860in}}{\pgfqpoint{5.489583in}{0.877907in}}%
\pgfusepath{clip}%
\pgfsetroundcap%
\pgfsetroundjoin%
\pgfsetlinewidth{1.505625pt}%
\definecolor{currentstroke}{rgb}{0.121569,0.466667,0.705882}%
\pgfsetstrokecolor{currentstroke}%
\pgfsetdash{}{0pt}%
\pgfpathmoveto{\pgfqpoint{9.810417in}{15.220099in}}%
\pgfpathlineto{\pgfqpoint{15.300000in}{15.220099in}}%
\pgfusepath{stroke}%
\end{pgfscope}%
\begin{pgfscope}%
\pgfpathrectangle{\pgfqpoint{9.810417in}{15.141860in}}{\pgfqpoint{5.489583in}{0.877907in}}%
\pgfusepath{clip}%
\pgfsetbuttcap%
\pgfsetroundjoin%
\definecolor{currentfill}{rgb}{0.121569,0.466667,0.705882}%
\pgfsetfillcolor{currentfill}%
\pgfsetlinewidth{1.003750pt}%
\definecolor{currentstroke}{rgb}{0.121569,0.466667,0.705882}%
\pgfsetstrokecolor{currentstroke}%
\pgfsetdash{}{0pt}%
\pgfsys@defobject{currentmarker}{\pgfqpoint{-0.034722in}{-0.034722in}}{\pgfqpoint{0.034722in}{0.034722in}}{%
\pgfpathmoveto{\pgfqpoint{0.000000in}{-0.034722in}}%
\pgfpathcurveto{\pgfqpoint{0.009208in}{-0.034722in}}{\pgfqpoint{0.018041in}{-0.031064in}}{\pgfqpoint{0.024552in}{-0.024552in}}%
\pgfpathcurveto{\pgfqpoint{0.031064in}{-0.018041in}}{\pgfqpoint{0.034722in}{-0.009208in}}{\pgfqpoint{0.034722in}{0.000000in}}%
\pgfpathcurveto{\pgfqpoint{0.034722in}{0.009208in}}{\pgfqpoint{0.031064in}{0.018041in}}{\pgfqpoint{0.024552in}{0.024552in}}%
\pgfpathcurveto{\pgfqpoint{0.018041in}{0.031064in}}{\pgfqpoint{0.009208in}{0.034722in}}{\pgfqpoint{0.000000in}{0.034722in}}%
\pgfpathcurveto{\pgfqpoint{-0.009208in}{0.034722in}}{\pgfqpoint{-0.018041in}{0.031064in}}{\pgfqpoint{-0.024552in}{0.024552in}}%
\pgfpathcurveto{\pgfqpoint{-0.031064in}{0.018041in}}{\pgfqpoint{-0.034722in}{0.009208in}}{\pgfqpoint{-0.034722in}{0.000000in}}%
\pgfpathcurveto{\pgfqpoint{-0.034722in}{-0.009208in}}{\pgfqpoint{-0.031064in}{-0.018041in}}{\pgfqpoint{-0.024552in}{-0.024552in}}%
\pgfpathcurveto{\pgfqpoint{-0.018041in}{-0.031064in}}{\pgfqpoint{-0.009208in}{-0.034722in}}{\pgfqpoint{0.000000in}{-0.034722in}}%
\pgfpathclose%
\pgfusepath{stroke,fill}%
}%
\begin{pgfscope}%
\pgfsys@transformshift{10.059943in}{15.979863in}%
\pgfsys@useobject{currentmarker}{}%
\end{pgfscope}%
\begin{pgfscope}%
\pgfsys@transformshift{10.183166in}{15.976437in}%
\pgfsys@useobject{currentmarker}{}%
\end{pgfscope}%
\begin{pgfscope}%
\pgfsys@transformshift{10.306389in}{15.202667in}%
\pgfsys@useobject{currentmarker}{}%
\end{pgfscope}%
\begin{pgfscope}%
\pgfsys@transformshift{10.429612in}{15.234885in}%
\pgfsys@useobject{currentmarker}{}%
\end{pgfscope}%
\begin{pgfscope}%
\pgfsys@transformshift{10.552835in}{15.216929in}%
\pgfsys@useobject{currentmarker}{}%
\end{pgfscope}%
\begin{pgfscope}%
\pgfsys@transformshift{10.676058in}{15.223376in}%
\pgfsys@useobject{currentmarker}{}%
\end{pgfscope}%
\begin{pgfscope}%
\pgfsys@transformshift{10.799281in}{15.238183in}%
\pgfsys@useobject{currentmarker}{}%
\end{pgfscope}%
\begin{pgfscope}%
\pgfsys@transformshift{10.922504in}{15.215006in}%
\pgfsys@useobject{currentmarker}{}%
\end{pgfscope}%
\begin{pgfscope}%
\pgfsys@transformshift{11.045727in}{15.230690in}%
\pgfsys@useobject{currentmarker}{}%
\end{pgfscope}%
\begin{pgfscope}%
\pgfsys@transformshift{11.168950in}{15.203849in}%
\pgfsys@useobject{currentmarker}{}%
\end{pgfscope}%
\begin{pgfscope}%
\pgfsys@transformshift{11.292173in}{15.242116in}%
\pgfsys@useobject{currentmarker}{}%
\end{pgfscope}%
\begin{pgfscope}%
\pgfsys@transformshift{11.415396in}{15.238844in}%
\pgfsys@useobject{currentmarker}{}%
\end{pgfscope}%
\begin{pgfscope}%
\pgfsys@transformshift{11.538619in}{15.225891in}%
\pgfsys@useobject{currentmarker}{}%
\end{pgfscope}%
\begin{pgfscope}%
\pgfsys@transformshift{11.661842in}{15.194351in}%
\pgfsys@useobject{currentmarker}{}%
\end{pgfscope}%
\begin{pgfscope}%
\pgfsys@transformshift{11.785065in}{15.199830in}%
\pgfsys@useobject{currentmarker}{}%
\end{pgfscope}%
\begin{pgfscope}%
\pgfsys@transformshift{11.908288in}{15.209845in}%
\pgfsys@useobject{currentmarker}{}%
\end{pgfscope}%
\begin{pgfscope}%
\pgfsys@transformshift{12.031511in}{15.239118in}%
\pgfsys@useobject{currentmarker}{}%
\end{pgfscope}%
\begin{pgfscope}%
\pgfsys@transformshift{12.154734in}{15.212988in}%
\pgfsys@useobject{currentmarker}{}%
\end{pgfscope}%
\begin{pgfscope}%
\pgfsys@transformshift{12.277957in}{15.194128in}%
\pgfsys@useobject{currentmarker}{}%
\end{pgfscope}%
\begin{pgfscope}%
\pgfsys@transformshift{12.401180in}{15.215222in}%
\pgfsys@useobject{currentmarker}{}%
\end{pgfscope}%
\begin{pgfscope}%
\pgfsys@transformshift{12.524403in}{15.246563in}%
\pgfsys@useobject{currentmarker}{}%
\end{pgfscope}%
\begin{pgfscope}%
\pgfsys@transformshift{12.647626in}{15.216437in}%
\pgfsys@useobject{currentmarker}{}%
\end{pgfscope}%
\begin{pgfscope}%
\pgfsys@transformshift{12.770849in}{15.224132in}%
\pgfsys@useobject{currentmarker}{}%
\end{pgfscope}%
\begin{pgfscope}%
\pgfsys@transformshift{12.894072in}{15.239570in}%
\pgfsys@useobject{currentmarker}{}%
\end{pgfscope}%
\begin{pgfscope}%
\pgfsys@transformshift{13.017294in}{15.224306in}%
\pgfsys@useobject{currentmarker}{}%
\end{pgfscope}%
\begin{pgfscope}%
\pgfsys@transformshift{13.140517in}{15.226160in}%
\pgfsys@useobject{currentmarker}{}%
\end{pgfscope}%
\begin{pgfscope}%
\pgfsys@transformshift{13.263740in}{15.215944in}%
\pgfsys@useobject{currentmarker}{}%
\end{pgfscope}%
\begin{pgfscope}%
\pgfsys@transformshift{13.386963in}{15.218044in}%
\pgfsys@useobject{currentmarker}{}%
\end{pgfscope}%
\begin{pgfscope}%
\pgfsys@transformshift{13.510186in}{15.202795in}%
\pgfsys@useobject{currentmarker}{}%
\end{pgfscope}%
\begin{pgfscope}%
\pgfsys@transformshift{13.633409in}{15.214413in}%
\pgfsys@useobject{currentmarker}{}%
\end{pgfscope}%
\begin{pgfscope}%
\pgfsys@transformshift{13.756632in}{15.223549in}%
\pgfsys@useobject{currentmarker}{}%
\end{pgfscope}%
\begin{pgfscope}%
\pgfsys@transformshift{13.879855in}{15.218953in}%
\pgfsys@useobject{currentmarker}{}%
\end{pgfscope}%
\begin{pgfscope}%
\pgfsys@transformshift{14.003078in}{15.238695in}%
\pgfsys@useobject{currentmarker}{}%
\end{pgfscope}%
\begin{pgfscope}%
\pgfsys@transformshift{14.126301in}{15.223453in}%
\pgfsys@useobject{currentmarker}{}%
\end{pgfscope}%
\begin{pgfscope}%
\pgfsys@transformshift{14.249524in}{15.198821in}%
\pgfsys@useobject{currentmarker}{}%
\end{pgfscope}%
\begin{pgfscope}%
\pgfsys@transformshift{14.372747in}{15.208401in}%
\pgfsys@useobject{currentmarker}{}%
\end{pgfscope}%
\begin{pgfscope}%
\pgfsys@transformshift{14.495970in}{15.208474in}%
\pgfsys@useobject{currentmarker}{}%
\end{pgfscope}%
\begin{pgfscope}%
\pgfsys@transformshift{14.619193in}{15.223827in}%
\pgfsys@useobject{currentmarker}{}%
\end{pgfscope}%
\begin{pgfscope}%
\pgfsys@transformshift{14.742416in}{15.241676in}%
\pgfsys@useobject{currentmarker}{}%
\end{pgfscope}%
\begin{pgfscope}%
\pgfsys@transformshift{14.865639in}{15.225734in}%
\pgfsys@useobject{currentmarker}{}%
\end{pgfscope}%
\begin{pgfscope}%
\pgfsys@transformshift{14.988862in}{15.225291in}%
\pgfsys@useobject{currentmarker}{}%
\end{pgfscope}%
\end{pgfscope}%
\begin{pgfscope}%
\pgfsetrectcap%
\pgfsetmiterjoin%
\pgfsetlinewidth{0.803000pt}%
\definecolor{currentstroke}{rgb}{1.000000,1.000000,1.000000}%
\pgfsetstrokecolor{currentstroke}%
\pgfsetdash{}{0pt}%
\pgfpathmoveto{\pgfqpoint{9.810417in}{15.141860in}}%
\pgfpathlineto{\pgfqpoint{9.810417in}{16.019767in}}%
\pgfusepath{stroke}%
\end{pgfscope}%
\begin{pgfscope}%
\pgfsetrectcap%
\pgfsetmiterjoin%
\pgfsetlinewidth{0.803000pt}%
\definecolor{currentstroke}{rgb}{1.000000,1.000000,1.000000}%
\pgfsetstrokecolor{currentstroke}%
\pgfsetdash{}{0pt}%
\pgfpathmoveto{\pgfqpoint{15.300000in}{15.141860in}}%
\pgfpathlineto{\pgfqpoint{15.300000in}{16.019767in}}%
\pgfusepath{stroke}%
\end{pgfscope}%
\begin{pgfscope}%
\pgfsetrectcap%
\pgfsetmiterjoin%
\pgfsetlinewidth{0.803000pt}%
\definecolor{currentstroke}{rgb}{1.000000,1.000000,1.000000}%
\pgfsetstrokecolor{currentstroke}%
\pgfsetdash{}{0pt}%
\pgfpathmoveto{\pgfqpoint{9.810417in}{15.141860in}}%
\pgfpathlineto{\pgfqpoint{15.300000in}{15.141860in}}%
\pgfusepath{stroke}%
\end{pgfscope}%
\begin{pgfscope}%
\pgfsetrectcap%
\pgfsetmiterjoin%
\pgfsetlinewidth{0.803000pt}%
\definecolor{currentstroke}{rgb}{1.000000,1.000000,1.000000}%
\pgfsetstrokecolor{currentstroke}%
\pgfsetdash{}{0pt}%
\pgfpathmoveto{\pgfqpoint{9.810417in}{16.019767in}}%
\pgfpathlineto{\pgfqpoint{15.300000in}{16.019767in}}%
\pgfusepath{stroke}%
\end{pgfscope}%
\begin{pgfscope}%
\definecolor{textcolor}{rgb}{0.150000,0.150000,0.150000}%
\pgfsetstrokecolor{textcolor}%
\pgfsetfillcolor{textcolor}%
\pgftext[x=12.555208in,y=16.103101in,,base]{\color{textcolor}\rmfamily\fontsize{16.800000}{20.160000}\selectfont Partial Autocorrelation}%
\end{pgfscope}%
\begin{pgfscope}%
\pgfsetbuttcap%
\pgfsetmiterjoin%
\definecolor{currentfill}{rgb}{0.917647,0.917647,0.949020}%
\pgfsetfillcolor{currentfill}%
\pgfsetlinewidth{0.000000pt}%
\definecolor{currentstroke}{rgb}{0.000000,0.000000,0.000000}%
\pgfsetstrokecolor{currentstroke}%
\pgfsetstrokeopacity{0.000000}%
\pgfsetdash{}{0pt}%
\pgfpathmoveto{\pgfqpoint{2.125000in}{13.561628in}}%
\pgfpathlineto{\pgfqpoint{7.614583in}{13.561628in}}%
\pgfpathlineto{\pgfqpoint{7.614583in}{14.439535in}}%
\pgfpathlineto{\pgfqpoint{2.125000in}{14.439535in}}%
\pgfpathclose%
\pgfusepath{fill}%
\end{pgfscope}%
\begin{pgfscope}%
\pgfpathrectangle{\pgfqpoint{2.125000in}{13.561628in}}{\pgfqpoint{5.489583in}{0.877907in}}%
\pgfusepath{clip}%
\pgfsetroundcap%
\pgfsetroundjoin%
\pgfsetlinewidth{0.803000pt}%
\definecolor{currentstroke}{rgb}{1.000000,1.000000,1.000000}%
\pgfsetstrokecolor{currentstroke}%
\pgfsetdash{}{0pt}%
\pgfpathmoveto{\pgfqpoint{2.374527in}{13.561628in}}%
\pgfpathlineto{\pgfqpoint{2.374527in}{14.439535in}}%
\pgfusepath{stroke}%
\end{pgfscope}%
\begin{pgfscope}%
\definecolor{textcolor}{rgb}{0.150000,0.150000,0.150000}%
\pgfsetstrokecolor{textcolor}%
\pgfsetfillcolor{textcolor}%
\pgftext[x=2.374527in,y=13.464406in,,top]{\color{textcolor}\rmfamily\fontsize{14.000000}{16.800000}\selectfont 0}%
\end{pgfscope}%
\begin{pgfscope}%
\pgfpathrectangle{\pgfqpoint{2.125000in}{13.561628in}}{\pgfqpoint{5.489583in}{0.877907in}}%
\pgfusepath{clip}%
\pgfsetroundcap%
\pgfsetroundjoin%
\pgfsetlinewidth{0.803000pt}%
\definecolor{currentstroke}{rgb}{1.000000,1.000000,1.000000}%
\pgfsetstrokecolor{currentstroke}%
\pgfsetdash{}{0pt}%
\pgfpathmoveto{\pgfqpoint{2.990641in}{13.561628in}}%
\pgfpathlineto{\pgfqpoint{2.990641in}{14.439535in}}%
\pgfusepath{stroke}%
\end{pgfscope}%
\begin{pgfscope}%
\definecolor{textcolor}{rgb}{0.150000,0.150000,0.150000}%
\pgfsetstrokecolor{textcolor}%
\pgfsetfillcolor{textcolor}%
\pgftext[x=2.990641in,y=13.464406in,,top]{\color{textcolor}\rmfamily\fontsize{14.000000}{16.800000}\selectfont 5}%
\end{pgfscope}%
\begin{pgfscope}%
\pgfpathrectangle{\pgfqpoint{2.125000in}{13.561628in}}{\pgfqpoint{5.489583in}{0.877907in}}%
\pgfusepath{clip}%
\pgfsetroundcap%
\pgfsetroundjoin%
\pgfsetlinewidth{0.803000pt}%
\definecolor{currentstroke}{rgb}{1.000000,1.000000,1.000000}%
\pgfsetstrokecolor{currentstroke}%
\pgfsetdash{}{0pt}%
\pgfpathmoveto{\pgfqpoint{3.606756in}{13.561628in}}%
\pgfpathlineto{\pgfqpoint{3.606756in}{14.439535in}}%
\pgfusepath{stroke}%
\end{pgfscope}%
\begin{pgfscope}%
\definecolor{textcolor}{rgb}{0.150000,0.150000,0.150000}%
\pgfsetstrokecolor{textcolor}%
\pgfsetfillcolor{textcolor}%
\pgftext[x=3.606756in,y=13.464406in,,top]{\color{textcolor}\rmfamily\fontsize{14.000000}{16.800000}\selectfont 10}%
\end{pgfscope}%
\begin{pgfscope}%
\pgfpathrectangle{\pgfqpoint{2.125000in}{13.561628in}}{\pgfqpoint{5.489583in}{0.877907in}}%
\pgfusepath{clip}%
\pgfsetroundcap%
\pgfsetroundjoin%
\pgfsetlinewidth{0.803000pt}%
\definecolor{currentstroke}{rgb}{1.000000,1.000000,1.000000}%
\pgfsetstrokecolor{currentstroke}%
\pgfsetdash{}{0pt}%
\pgfpathmoveto{\pgfqpoint{4.222871in}{13.561628in}}%
\pgfpathlineto{\pgfqpoint{4.222871in}{14.439535in}}%
\pgfusepath{stroke}%
\end{pgfscope}%
\begin{pgfscope}%
\definecolor{textcolor}{rgb}{0.150000,0.150000,0.150000}%
\pgfsetstrokecolor{textcolor}%
\pgfsetfillcolor{textcolor}%
\pgftext[x=4.222871in,y=13.464406in,,top]{\color{textcolor}\rmfamily\fontsize{14.000000}{16.800000}\selectfont 15}%
\end{pgfscope}%
\begin{pgfscope}%
\pgfpathrectangle{\pgfqpoint{2.125000in}{13.561628in}}{\pgfqpoint{5.489583in}{0.877907in}}%
\pgfusepath{clip}%
\pgfsetroundcap%
\pgfsetroundjoin%
\pgfsetlinewidth{0.803000pt}%
\definecolor{currentstroke}{rgb}{1.000000,1.000000,1.000000}%
\pgfsetstrokecolor{currentstroke}%
\pgfsetdash{}{0pt}%
\pgfpathmoveto{\pgfqpoint{4.838986in}{13.561628in}}%
\pgfpathlineto{\pgfqpoint{4.838986in}{14.439535in}}%
\pgfusepath{stroke}%
\end{pgfscope}%
\begin{pgfscope}%
\definecolor{textcolor}{rgb}{0.150000,0.150000,0.150000}%
\pgfsetstrokecolor{textcolor}%
\pgfsetfillcolor{textcolor}%
\pgftext[x=4.838986in,y=13.464406in,,top]{\color{textcolor}\rmfamily\fontsize{14.000000}{16.800000}\selectfont 20}%
\end{pgfscope}%
\begin{pgfscope}%
\pgfpathrectangle{\pgfqpoint{2.125000in}{13.561628in}}{\pgfqpoint{5.489583in}{0.877907in}}%
\pgfusepath{clip}%
\pgfsetroundcap%
\pgfsetroundjoin%
\pgfsetlinewidth{0.803000pt}%
\definecolor{currentstroke}{rgb}{1.000000,1.000000,1.000000}%
\pgfsetstrokecolor{currentstroke}%
\pgfsetdash{}{0pt}%
\pgfpathmoveto{\pgfqpoint{5.455101in}{13.561628in}}%
\pgfpathlineto{\pgfqpoint{5.455101in}{14.439535in}}%
\pgfusepath{stroke}%
\end{pgfscope}%
\begin{pgfscope}%
\definecolor{textcolor}{rgb}{0.150000,0.150000,0.150000}%
\pgfsetstrokecolor{textcolor}%
\pgfsetfillcolor{textcolor}%
\pgftext[x=5.455101in,y=13.464406in,,top]{\color{textcolor}\rmfamily\fontsize{14.000000}{16.800000}\selectfont 25}%
\end{pgfscope}%
\begin{pgfscope}%
\pgfpathrectangle{\pgfqpoint{2.125000in}{13.561628in}}{\pgfqpoint{5.489583in}{0.877907in}}%
\pgfusepath{clip}%
\pgfsetroundcap%
\pgfsetroundjoin%
\pgfsetlinewidth{0.803000pt}%
\definecolor{currentstroke}{rgb}{1.000000,1.000000,1.000000}%
\pgfsetstrokecolor{currentstroke}%
\pgfsetdash{}{0pt}%
\pgfpathmoveto{\pgfqpoint{6.071216in}{13.561628in}}%
\pgfpathlineto{\pgfqpoint{6.071216in}{14.439535in}}%
\pgfusepath{stroke}%
\end{pgfscope}%
\begin{pgfscope}%
\definecolor{textcolor}{rgb}{0.150000,0.150000,0.150000}%
\pgfsetstrokecolor{textcolor}%
\pgfsetfillcolor{textcolor}%
\pgftext[x=6.071216in,y=13.464406in,,top]{\color{textcolor}\rmfamily\fontsize{14.000000}{16.800000}\selectfont 30}%
\end{pgfscope}%
\begin{pgfscope}%
\pgfpathrectangle{\pgfqpoint{2.125000in}{13.561628in}}{\pgfqpoint{5.489583in}{0.877907in}}%
\pgfusepath{clip}%
\pgfsetroundcap%
\pgfsetroundjoin%
\pgfsetlinewidth{0.803000pt}%
\definecolor{currentstroke}{rgb}{1.000000,1.000000,1.000000}%
\pgfsetstrokecolor{currentstroke}%
\pgfsetdash{}{0pt}%
\pgfpathmoveto{\pgfqpoint{6.687330in}{13.561628in}}%
\pgfpathlineto{\pgfqpoint{6.687330in}{14.439535in}}%
\pgfusepath{stroke}%
\end{pgfscope}%
\begin{pgfscope}%
\definecolor{textcolor}{rgb}{0.150000,0.150000,0.150000}%
\pgfsetstrokecolor{textcolor}%
\pgfsetfillcolor{textcolor}%
\pgftext[x=6.687330in,y=13.464406in,,top]{\color{textcolor}\rmfamily\fontsize{14.000000}{16.800000}\selectfont 35}%
\end{pgfscope}%
\begin{pgfscope}%
\pgfpathrectangle{\pgfqpoint{2.125000in}{13.561628in}}{\pgfqpoint{5.489583in}{0.877907in}}%
\pgfusepath{clip}%
\pgfsetroundcap%
\pgfsetroundjoin%
\pgfsetlinewidth{0.803000pt}%
\definecolor{currentstroke}{rgb}{1.000000,1.000000,1.000000}%
\pgfsetstrokecolor{currentstroke}%
\pgfsetdash{}{0pt}%
\pgfpathmoveto{\pgfqpoint{7.303445in}{13.561628in}}%
\pgfpathlineto{\pgfqpoint{7.303445in}{14.439535in}}%
\pgfusepath{stroke}%
\end{pgfscope}%
\begin{pgfscope}%
\definecolor{textcolor}{rgb}{0.150000,0.150000,0.150000}%
\pgfsetstrokecolor{textcolor}%
\pgfsetfillcolor{textcolor}%
\pgftext[x=7.303445in,y=13.464406in,,top]{\color{textcolor}\rmfamily\fontsize{14.000000}{16.800000}\selectfont 40}%
\end{pgfscope}%
\begin{pgfscope}%
\pgfpathrectangle{\pgfqpoint{2.125000in}{13.561628in}}{\pgfqpoint{5.489583in}{0.877907in}}%
\pgfusepath{clip}%
\pgfsetroundcap%
\pgfsetroundjoin%
\pgfsetlinewidth{0.803000pt}%
\definecolor{currentstroke}{rgb}{1.000000,1.000000,1.000000}%
\pgfsetstrokecolor{currentstroke}%
\pgfsetdash{}{0pt}%
\pgfpathmoveto{\pgfqpoint{2.125000in}{13.835489in}}%
\pgfpathlineto{\pgfqpoint{7.614583in}{13.835489in}}%
\pgfusepath{stroke}%
\end{pgfscope}%
\begin{pgfscope}%
\definecolor{textcolor}{rgb}{0.150000,0.150000,0.150000}%
\pgfsetstrokecolor{textcolor}%
\pgfsetfillcolor{textcolor}%
\pgftext[x=1.904066in,y=13.761623in,left,base]{\color{textcolor}\rmfamily\fontsize{14.000000}{16.800000}\selectfont 0}%
\end{pgfscope}%
\begin{pgfscope}%
\pgfpathrectangle{\pgfqpoint{2.125000in}{13.561628in}}{\pgfqpoint{5.489583in}{0.877907in}}%
\pgfusepath{clip}%
\pgfsetroundcap%
\pgfsetroundjoin%
\pgfsetlinewidth{0.803000pt}%
\definecolor{currentstroke}{rgb}{1.000000,1.000000,1.000000}%
\pgfsetstrokecolor{currentstroke}%
\pgfsetdash{}{0pt}%
\pgfpathmoveto{\pgfqpoint{2.125000in}{14.399630in}}%
\pgfpathlineto{\pgfqpoint{7.614583in}{14.399630in}}%
\pgfusepath{stroke}%
\end{pgfscope}%
\begin{pgfscope}%
\definecolor{textcolor}{rgb}{0.150000,0.150000,0.150000}%
\pgfsetstrokecolor{textcolor}%
\pgfsetfillcolor{textcolor}%
\pgftext[x=1.904066in,y=14.325764in,left,base]{\color{textcolor}\rmfamily\fontsize{14.000000}{16.800000}\selectfont 1}%
\end{pgfscope}%
\begin{pgfscope}%
\pgfpathrectangle{\pgfqpoint{2.125000in}{13.561628in}}{\pgfqpoint{5.489583in}{0.877907in}}%
\pgfusepath{clip}%
\pgfsetbuttcap%
\pgfsetroundjoin%
\definecolor{currentfill}{rgb}{0.121569,0.466667,0.705882}%
\pgfsetfillcolor{currentfill}%
\pgfsetfillopacity{0.250000}%
\pgfsetlinewidth{1.003750pt}%
\definecolor{currentstroke}{rgb}{1.000000,1.000000,1.000000}%
\pgfsetstrokecolor{currentstroke}%
\pgfsetstrokeopacity{0.250000}%
\pgfsetdash{}{0pt}%
\pgfpathmoveto{\pgfqpoint{2.436138in}{13.863953in}}%
\pgfpathlineto{\pgfqpoint{2.436138in}{13.807025in}}%
\pgfpathlineto{\pgfqpoint{2.620972in}{13.786321in}}%
\pgfpathlineto{\pgfqpoint{2.744195in}{13.772152in}}%
\pgfpathlineto{\pgfqpoint{2.867418in}{13.760703in}}%
\pgfpathlineto{\pgfqpoint{2.990641in}{13.750862in}}%
\pgfpathlineto{\pgfqpoint{3.113864in}{13.742117in}}%
\pgfpathlineto{\pgfqpoint{3.237087in}{13.734184in}}%
\pgfpathlineto{\pgfqpoint{3.360310in}{13.726883in}}%
\pgfpathlineto{\pgfqpoint{3.483533in}{13.720091in}}%
\pgfpathlineto{\pgfqpoint{3.606756in}{13.713725in}}%
\pgfpathlineto{\pgfqpoint{3.729979in}{13.707718in}}%
\pgfpathlineto{\pgfqpoint{3.853202in}{13.702024in}}%
\pgfpathlineto{\pgfqpoint{3.976425in}{13.696604in}}%
\pgfpathlineto{\pgfqpoint{4.099648in}{13.691427in}}%
\pgfpathlineto{\pgfqpoint{4.222871in}{13.686471in}}%
\pgfpathlineto{\pgfqpoint{4.346094in}{13.681716in}}%
\pgfpathlineto{\pgfqpoint{4.469317in}{13.677141in}}%
\pgfpathlineto{\pgfqpoint{4.592540in}{13.672732in}}%
\pgfpathlineto{\pgfqpoint{4.715763in}{13.668477in}}%
\pgfpathlineto{\pgfqpoint{4.838986in}{13.664363in}}%
\pgfpathlineto{\pgfqpoint{4.962209in}{13.660381in}}%
\pgfpathlineto{\pgfqpoint{5.085432in}{13.656520in}}%
\pgfpathlineto{\pgfqpoint{5.208655in}{13.652772in}}%
\pgfpathlineto{\pgfqpoint{5.331878in}{13.649130in}}%
\pgfpathlineto{\pgfqpoint{5.455101in}{13.645587in}}%
\pgfpathlineto{\pgfqpoint{5.578324in}{13.642140in}}%
\pgfpathlineto{\pgfqpoint{5.701547in}{13.638782in}}%
\pgfpathlineto{\pgfqpoint{5.824770in}{13.635509in}}%
\pgfpathlineto{\pgfqpoint{5.947993in}{13.632317in}}%
\pgfpathlineto{\pgfqpoint{6.071216in}{13.629203in}}%
\pgfpathlineto{\pgfqpoint{6.194439in}{13.626163in}}%
\pgfpathlineto{\pgfqpoint{6.317662in}{13.623195in}}%
\pgfpathlineto{\pgfqpoint{6.440885in}{13.620293in}}%
\pgfpathlineto{\pgfqpoint{6.564108in}{13.617452in}}%
\pgfpathlineto{\pgfqpoint{6.687330in}{13.614667in}}%
\pgfpathlineto{\pgfqpoint{6.810553in}{13.611937in}}%
\pgfpathlineto{\pgfqpoint{6.933776in}{13.609260in}}%
\pgfpathlineto{\pgfqpoint{7.056999in}{13.606635in}}%
\pgfpathlineto{\pgfqpoint{7.180222in}{13.604060in}}%
\pgfpathlineto{\pgfqpoint{7.365057in}{13.601533in}}%
\pgfpathlineto{\pgfqpoint{7.365057in}{14.069445in}}%
\pgfpathlineto{\pgfqpoint{7.365057in}{14.069445in}}%
\pgfpathlineto{\pgfqpoint{7.180222in}{14.066918in}}%
\pgfpathlineto{\pgfqpoint{7.056999in}{14.064343in}}%
\pgfpathlineto{\pgfqpoint{6.933776in}{14.061718in}}%
\pgfpathlineto{\pgfqpoint{6.810553in}{14.059041in}}%
\pgfpathlineto{\pgfqpoint{6.687330in}{14.056311in}}%
\pgfpathlineto{\pgfqpoint{6.564108in}{14.053526in}}%
\pgfpathlineto{\pgfqpoint{6.440885in}{14.050685in}}%
\pgfpathlineto{\pgfqpoint{6.317662in}{14.047783in}}%
\pgfpathlineto{\pgfqpoint{6.194439in}{14.044815in}}%
\pgfpathlineto{\pgfqpoint{6.071216in}{14.041775in}}%
\pgfpathlineto{\pgfqpoint{5.947993in}{14.038661in}}%
\pgfpathlineto{\pgfqpoint{5.824770in}{14.035469in}}%
\pgfpathlineto{\pgfqpoint{5.701547in}{14.032197in}}%
\pgfpathlineto{\pgfqpoint{5.578324in}{14.028838in}}%
\pgfpathlineto{\pgfqpoint{5.455101in}{14.025391in}}%
\pgfpathlineto{\pgfqpoint{5.331878in}{14.021849in}}%
\pgfpathlineto{\pgfqpoint{5.208655in}{14.018206in}}%
\pgfpathlineto{\pgfqpoint{5.085432in}{14.014458in}}%
\pgfpathlineto{\pgfqpoint{4.962209in}{14.010597in}}%
\pgfpathlineto{\pgfqpoint{4.838986in}{14.006615in}}%
\pgfpathlineto{\pgfqpoint{4.715763in}{14.002501in}}%
\pgfpathlineto{\pgfqpoint{4.592540in}{13.998246in}}%
\pgfpathlineto{\pgfqpoint{4.469317in}{13.993837in}}%
\pgfpathlineto{\pgfqpoint{4.346094in}{13.989262in}}%
\pgfpathlineto{\pgfqpoint{4.222871in}{13.984507in}}%
\pgfpathlineto{\pgfqpoint{4.099648in}{13.979551in}}%
\pgfpathlineto{\pgfqpoint{3.976425in}{13.974374in}}%
\pgfpathlineto{\pgfqpoint{3.853202in}{13.968954in}}%
\pgfpathlineto{\pgfqpoint{3.729979in}{13.963260in}}%
\pgfpathlineto{\pgfqpoint{3.606756in}{13.957253in}}%
\pgfpathlineto{\pgfqpoint{3.483533in}{13.950887in}}%
\pgfpathlineto{\pgfqpoint{3.360310in}{13.944095in}}%
\pgfpathlineto{\pgfqpoint{3.237087in}{13.936794in}}%
\pgfpathlineto{\pgfqpoint{3.113864in}{13.928861in}}%
\pgfpathlineto{\pgfqpoint{2.990641in}{13.920116in}}%
\pgfpathlineto{\pgfqpoint{2.867418in}{13.910275in}}%
\pgfpathlineto{\pgfqpoint{2.744195in}{13.898826in}}%
\pgfpathlineto{\pgfqpoint{2.620972in}{13.884658in}}%
\pgfpathlineto{\pgfqpoint{2.436138in}{13.863953in}}%
\pgfpathclose%
\pgfusepath{stroke,fill}%
\end{pgfscope}%
\begin{pgfscope}%
\pgfpathrectangle{\pgfqpoint{2.125000in}{13.561628in}}{\pgfqpoint{5.489583in}{0.877907in}}%
\pgfusepath{clip}%
\pgfsetbuttcap%
\pgfsetroundjoin%
\pgfsetlinewidth{1.505625pt}%
\definecolor{currentstroke}{rgb}{0.000000,0.000000,0.000000}%
\pgfsetstrokecolor{currentstroke}%
\pgfsetdash{}{0pt}%
\pgfpathmoveto{\pgfqpoint{2.374527in}{13.835489in}}%
\pgfpathlineto{\pgfqpoint{2.374527in}{14.399630in}}%
\pgfusepath{stroke}%
\end{pgfscope}%
\begin{pgfscope}%
\pgfpathrectangle{\pgfqpoint{2.125000in}{13.561628in}}{\pgfqpoint{5.489583in}{0.877907in}}%
\pgfusepath{clip}%
\pgfsetbuttcap%
\pgfsetroundjoin%
\pgfsetlinewidth{1.505625pt}%
\definecolor{currentstroke}{rgb}{0.000000,0.000000,0.000000}%
\pgfsetstrokecolor{currentstroke}%
\pgfsetdash{}{0pt}%
\pgfpathmoveto{\pgfqpoint{2.497749in}{13.835489in}}%
\pgfpathlineto{\pgfqpoint{2.497749in}{14.397362in}}%
\pgfusepath{stroke}%
\end{pgfscope}%
\begin{pgfscope}%
\pgfpathrectangle{\pgfqpoint{2.125000in}{13.561628in}}{\pgfqpoint{5.489583in}{0.877907in}}%
\pgfusepath{clip}%
\pgfsetbuttcap%
\pgfsetroundjoin%
\pgfsetlinewidth{1.505625pt}%
\definecolor{currentstroke}{rgb}{0.000000,0.000000,0.000000}%
\pgfsetstrokecolor{currentstroke}%
\pgfsetdash{}{0pt}%
\pgfpathmoveto{\pgfqpoint{2.620972in}{13.835489in}}%
\pgfpathlineto{\pgfqpoint{2.620972in}{14.395030in}}%
\pgfusepath{stroke}%
\end{pgfscope}%
\begin{pgfscope}%
\pgfpathrectangle{\pgfqpoint{2.125000in}{13.561628in}}{\pgfqpoint{5.489583in}{0.877907in}}%
\pgfusepath{clip}%
\pgfsetbuttcap%
\pgfsetroundjoin%
\pgfsetlinewidth{1.505625pt}%
\definecolor{currentstroke}{rgb}{0.000000,0.000000,0.000000}%
\pgfsetstrokecolor{currentstroke}%
\pgfsetdash{}{0pt}%
\pgfpathmoveto{\pgfqpoint{2.744195in}{13.835489in}}%
\pgfpathlineto{\pgfqpoint{2.744195in}{14.392807in}}%
\pgfusepath{stroke}%
\end{pgfscope}%
\begin{pgfscope}%
\pgfpathrectangle{\pgfqpoint{2.125000in}{13.561628in}}{\pgfqpoint{5.489583in}{0.877907in}}%
\pgfusepath{clip}%
\pgfsetbuttcap%
\pgfsetroundjoin%
\pgfsetlinewidth{1.505625pt}%
\definecolor{currentstroke}{rgb}{0.000000,0.000000,0.000000}%
\pgfsetstrokecolor{currentstroke}%
\pgfsetdash{}{0pt}%
\pgfpathmoveto{\pgfqpoint{2.867418in}{13.835489in}}%
\pgfpathlineto{\pgfqpoint{2.867418in}{14.390575in}}%
\pgfusepath{stroke}%
\end{pgfscope}%
\begin{pgfscope}%
\pgfpathrectangle{\pgfqpoint{2.125000in}{13.561628in}}{\pgfqpoint{5.489583in}{0.877907in}}%
\pgfusepath{clip}%
\pgfsetbuttcap%
\pgfsetroundjoin%
\pgfsetlinewidth{1.505625pt}%
\definecolor{currentstroke}{rgb}{0.000000,0.000000,0.000000}%
\pgfsetstrokecolor{currentstroke}%
\pgfsetdash{}{0pt}%
\pgfpathmoveto{\pgfqpoint{2.990641in}{13.835489in}}%
\pgfpathlineto{\pgfqpoint{2.990641in}{14.388411in}}%
\pgfusepath{stroke}%
\end{pgfscope}%
\begin{pgfscope}%
\pgfpathrectangle{\pgfqpoint{2.125000in}{13.561628in}}{\pgfqpoint{5.489583in}{0.877907in}}%
\pgfusepath{clip}%
\pgfsetbuttcap%
\pgfsetroundjoin%
\pgfsetlinewidth{1.505625pt}%
\definecolor{currentstroke}{rgb}{0.000000,0.000000,0.000000}%
\pgfsetstrokecolor{currentstroke}%
\pgfsetdash{}{0pt}%
\pgfpathmoveto{\pgfqpoint{3.113864in}{13.835489in}}%
\pgfpathlineto{\pgfqpoint{3.113864in}{14.386256in}}%
\pgfusepath{stroke}%
\end{pgfscope}%
\begin{pgfscope}%
\pgfpathrectangle{\pgfqpoint{2.125000in}{13.561628in}}{\pgfqpoint{5.489583in}{0.877907in}}%
\pgfusepath{clip}%
\pgfsetbuttcap%
\pgfsetroundjoin%
\pgfsetlinewidth{1.505625pt}%
\definecolor{currentstroke}{rgb}{0.000000,0.000000,0.000000}%
\pgfsetstrokecolor{currentstroke}%
\pgfsetdash{}{0pt}%
\pgfpathmoveto{\pgfqpoint{3.237087in}{13.835489in}}%
\pgfpathlineto{\pgfqpoint{3.237087in}{14.384126in}}%
\pgfusepath{stroke}%
\end{pgfscope}%
\begin{pgfscope}%
\pgfpathrectangle{\pgfqpoint{2.125000in}{13.561628in}}{\pgfqpoint{5.489583in}{0.877907in}}%
\pgfusepath{clip}%
\pgfsetbuttcap%
\pgfsetroundjoin%
\pgfsetlinewidth{1.505625pt}%
\definecolor{currentstroke}{rgb}{0.000000,0.000000,0.000000}%
\pgfsetstrokecolor{currentstroke}%
\pgfsetdash{}{0pt}%
\pgfpathmoveto{\pgfqpoint{3.360310in}{13.835489in}}%
\pgfpathlineto{\pgfqpoint{3.360310in}{14.382124in}}%
\pgfusepath{stroke}%
\end{pgfscope}%
\begin{pgfscope}%
\pgfpathrectangle{\pgfqpoint{2.125000in}{13.561628in}}{\pgfqpoint{5.489583in}{0.877907in}}%
\pgfusepath{clip}%
\pgfsetbuttcap%
\pgfsetroundjoin%
\pgfsetlinewidth{1.505625pt}%
\definecolor{currentstroke}{rgb}{0.000000,0.000000,0.000000}%
\pgfsetstrokecolor{currentstroke}%
\pgfsetdash{}{0pt}%
\pgfpathmoveto{\pgfqpoint{3.483533in}{13.835489in}}%
\pgfpathlineto{\pgfqpoint{3.483533in}{14.380057in}}%
\pgfusepath{stroke}%
\end{pgfscope}%
\begin{pgfscope}%
\pgfpathrectangle{\pgfqpoint{2.125000in}{13.561628in}}{\pgfqpoint{5.489583in}{0.877907in}}%
\pgfusepath{clip}%
\pgfsetbuttcap%
\pgfsetroundjoin%
\pgfsetlinewidth{1.505625pt}%
\definecolor{currentstroke}{rgb}{0.000000,0.000000,0.000000}%
\pgfsetstrokecolor{currentstroke}%
\pgfsetdash{}{0pt}%
\pgfpathmoveto{\pgfqpoint{3.606756in}{13.835489in}}%
\pgfpathlineto{\pgfqpoint{3.606756in}{14.378057in}}%
\pgfusepath{stroke}%
\end{pgfscope}%
\begin{pgfscope}%
\pgfpathrectangle{\pgfqpoint{2.125000in}{13.561628in}}{\pgfqpoint{5.489583in}{0.877907in}}%
\pgfusepath{clip}%
\pgfsetbuttcap%
\pgfsetroundjoin%
\pgfsetlinewidth{1.505625pt}%
\definecolor{currentstroke}{rgb}{0.000000,0.000000,0.000000}%
\pgfsetstrokecolor{currentstroke}%
\pgfsetdash{}{0pt}%
\pgfpathmoveto{\pgfqpoint{3.729979in}{13.835489in}}%
\pgfpathlineto{\pgfqpoint{3.729979in}{14.376005in}}%
\pgfusepath{stroke}%
\end{pgfscope}%
\begin{pgfscope}%
\pgfpathrectangle{\pgfqpoint{2.125000in}{13.561628in}}{\pgfqpoint{5.489583in}{0.877907in}}%
\pgfusepath{clip}%
\pgfsetbuttcap%
\pgfsetroundjoin%
\pgfsetlinewidth{1.505625pt}%
\definecolor{currentstroke}{rgb}{0.000000,0.000000,0.000000}%
\pgfsetstrokecolor{currentstroke}%
\pgfsetdash{}{0pt}%
\pgfpathmoveto{\pgfqpoint{3.853202in}{13.835489in}}%
\pgfpathlineto{\pgfqpoint{3.853202in}{14.373965in}}%
\pgfusepath{stroke}%
\end{pgfscope}%
\begin{pgfscope}%
\pgfpathrectangle{\pgfqpoint{2.125000in}{13.561628in}}{\pgfqpoint{5.489583in}{0.877907in}}%
\pgfusepath{clip}%
\pgfsetbuttcap%
\pgfsetroundjoin%
\pgfsetlinewidth{1.505625pt}%
\definecolor{currentstroke}{rgb}{0.000000,0.000000,0.000000}%
\pgfsetstrokecolor{currentstroke}%
\pgfsetdash{}{0pt}%
\pgfpathmoveto{\pgfqpoint{3.976425in}{13.835489in}}%
\pgfpathlineto{\pgfqpoint{3.976425in}{14.371866in}}%
\pgfusepath{stroke}%
\end{pgfscope}%
\begin{pgfscope}%
\pgfpathrectangle{\pgfqpoint{2.125000in}{13.561628in}}{\pgfqpoint{5.489583in}{0.877907in}}%
\pgfusepath{clip}%
\pgfsetbuttcap%
\pgfsetroundjoin%
\pgfsetlinewidth{1.505625pt}%
\definecolor{currentstroke}{rgb}{0.000000,0.000000,0.000000}%
\pgfsetstrokecolor{currentstroke}%
\pgfsetdash{}{0pt}%
\pgfpathmoveto{\pgfqpoint{4.099648in}{13.835489in}}%
\pgfpathlineto{\pgfqpoint{4.099648in}{14.369584in}}%
\pgfusepath{stroke}%
\end{pgfscope}%
\begin{pgfscope}%
\pgfpathrectangle{\pgfqpoint{2.125000in}{13.561628in}}{\pgfqpoint{5.489583in}{0.877907in}}%
\pgfusepath{clip}%
\pgfsetbuttcap%
\pgfsetroundjoin%
\pgfsetlinewidth{1.505625pt}%
\definecolor{currentstroke}{rgb}{0.000000,0.000000,0.000000}%
\pgfsetstrokecolor{currentstroke}%
\pgfsetdash{}{0pt}%
\pgfpathmoveto{\pgfqpoint{4.222871in}{13.835489in}}%
\pgfpathlineto{\pgfqpoint{4.222871in}{14.367296in}}%
\pgfusepath{stroke}%
\end{pgfscope}%
\begin{pgfscope}%
\pgfpathrectangle{\pgfqpoint{2.125000in}{13.561628in}}{\pgfqpoint{5.489583in}{0.877907in}}%
\pgfusepath{clip}%
\pgfsetbuttcap%
\pgfsetroundjoin%
\pgfsetlinewidth{1.505625pt}%
\definecolor{currentstroke}{rgb}{0.000000,0.000000,0.000000}%
\pgfsetstrokecolor{currentstroke}%
\pgfsetdash{}{0pt}%
\pgfpathmoveto{\pgfqpoint{4.346094in}{13.835489in}}%
\pgfpathlineto{\pgfqpoint{4.346094in}{14.365092in}}%
\pgfusepath{stroke}%
\end{pgfscope}%
\begin{pgfscope}%
\pgfpathrectangle{\pgfqpoint{2.125000in}{13.561628in}}{\pgfqpoint{5.489583in}{0.877907in}}%
\pgfusepath{clip}%
\pgfsetbuttcap%
\pgfsetroundjoin%
\pgfsetlinewidth{1.505625pt}%
\definecolor{currentstroke}{rgb}{0.000000,0.000000,0.000000}%
\pgfsetstrokecolor{currentstroke}%
\pgfsetdash{}{0pt}%
\pgfpathmoveto{\pgfqpoint{4.469317in}{13.835489in}}%
\pgfpathlineto{\pgfqpoint{4.469317in}{14.362772in}}%
\pgfusepath{stroke}%
\end{pgfscope}%
\begin{pgfscope}%
\pgfpathrectangle{\pgfqpoint{2.125000in}{13.561628in}}{\pgfqpoint{5.489583in}{0.877907in}}%
\pgfusepath{clip}%
\pgfsetbuttcap%
\pgfsetroundjoin%
\pgfsetlinewidth{1.505625pt}%
\definecolor{currentstroke}{rgb}{0.000000,0.000000,0.000000}%
\pgfsetstrokecolor{currentstroke}%
\pgfsetdash{}{0pt}%
\pgfpathmoveto{\pgfqpoint{4.592540in}{13.835489in}}%
\pgfpathlineto{\pgfqpoint{4.592540in}{14.360499in}}%
\pgfusepath{stroke}%
\end{pgfscope}%
\begin{pgfscope}%
\pgfpathrectangle{\pgfqpoint{2.125000in}{13.561628in}}{\pgfqpoint{5.489583in}{0.877907in}}%
\pgfusepath{clip}%
\pgfsetbuttcap%
\pgfsetroundjoin%
\pgfsetlinewidth{1.505625pt}%
\definecolor{currentstroke}{rgb}{0.000000,0.000000,0.000000}%
\pgfsetstrokecolor{currentstroke}%
\pgfsetdash{}{0pt}%
\pgfpathmoveto{\pgfqpoint{4.715763in}{13.835489in}}%
\pgfpathlineto{\pgfqpoint{4.715763in}{14.358170in}}%
\pgfusepath{stroke}%
\end{pgfscope}%
\begin{pgfscope}%
\pgfpathrectangle{\pgfqpoint{2.125000in}{13.561628in}}{\pgfqpoint{5.489583in}{0.877907in}}%
\pgfusepath{clip}%
\pgfsetbuttcap%
\pgfsetroundjoin%
\pgfsetlinewidth{1.505625pt}%
\definecolor{currentstroke}{rgb}{0.000000,0.000000,0.000000}%
\pgfsetstrokecolor{currentstroke}%
\pgfsetdash{}{0pt}%
\pgfpathmoveto{\pgfqpoint{4.838986in}{13.835489in}}%
\pgfpathlineto{\pgfqpoint{4.838986in}{14.355879in}}%
\pgfusepath{stroke}%
\end{pgfscope}%
\begin{pgfscope}%
\pgfpathrectangle{\pgfqpoint{2.125000in}{13.561628in}}{\pgfqpoint{5.489583in}{0.877907in}}%
\pgfusepath{clip}%
\pgfsetbuttcap%
\pgfsetroundjoin%
\pgfsetlinewidth{1.505625pt}%
\definecolor{currentstroke}{rgb}{0.000000,0.000000,0.000000}%
\pgfsetstrokecolor{currentstroke}%
\pgfsetdash{}{0pt}%
\pgfpathmoveto{\pgfqpoint{4.962209in}{13.835489in}}%
\pgfpathlineto{\pgfqpoint{4.962209in}{14.353680in}}%
\pgfusepath{stroke}%
\end{pgfscope}%
\begin{pgfscope}%
\pgfpathrectangle{\pgfqpoint{2.125000in}{13.561628in}}{\pgfqpoint{5.489583in}{0.877907in}}%
\pgfusepath{clip}%
\pgfsetbuttcap%
\pgfsetroundjoin%
\pgfsetlinewidth{1.505625pt}%
\definecolor{currentstroke}{rgb}{0.000000,0.000000,0.000000}%
\pgfsetstrokecolor{currentstroke}%
\pgfsetdash{}{0pt}%
\pgfpathmoveto{\pgfqpoint{5.085432in}{13.835489in}}%
\pgfpathlineto{\pgfqpoint{5.085432in}{14.351460in}}%
\pgfusepath{stroke}%
\end{pgfscope}%
\begin{pgfscope}%
\pgfpathrectangle{\pgfqpoint{2.125000in}{13.561628in}}{\pgfqpoint{5.489583in}{0.877907in}}%
\pgfusepath{clip}%
\pgfsetbuttcap%
\pgfsetroundjoin%
\pgfsetlinewidth{1.505625pt}%
\definecolor{currentstroke}{rgb}{0.000000,0.000000,0.000000}%
\pgfsetstrokecolor{currentstroke}%
\pgfsetdash{}{0pt}%
\pgfpathmoveto{\pgfqpoint{5.208655in}{13.835489in}}%
\pgfpathlineto{\pgfqpoint{5.208655in}{14.349347in}}%
\pgfusepath{stroke}%
\end{pgfscope}%
\begin{pgfscope}%
\pgfpathrectangle{\pgfqpoint{2.125000in}{13.561628in}}{\pgfqpoint{5.489583in}{0.877907in}}%
\pgfusepath{clip}%
\pgfsetbuttcap%
\pgfsetroundjoin%
\pgfsetlinewidth{1.505625pt}%
\definecolor{currentstroke}{rgb}{0.000000,0.000000,0.000000}%
\pgfsetstrokecolor{currentstroke}%
\pgfsetdash{}{0pt}%
\pgfpathmoveto{\pgfqpoint{5.331878in}{13.835489in}}%
\pgfpathlineto{\pgfqpoint{5.331878in}{14.347147in}}%
\pgfusepath{stroke}%
\end{pgfscope}%
\begin{pgfscope}%
\pgfpathrectangle{\pgfqpoint{2.125000in}{13.561628in}}{\pgfqpoint{5.489583in}{0.877907in}}%
\pgfusepath{clip}%
\pgfsetbuttcap%
\pgfsetroundjoin%
\pgfsetlinewidth{1.505625pt}%
\definecolor{currentstroke}{rgb}{0.000000,0.000000,0.000000}%
\pgfsetstrokecolor{currentstroke}%
\pgfsetdash{}{0pt}%
\pgfpathmoveto{\pgfqpoint{5.455101in}{13.835489in}}%
\pgfpathlineto{\pgfqpoint{5.455101in}{14.344897in}}%
\pgfusepath{stroke}%
\end{pgfscope}%
\begin{pgfscope}%
\pgfpathrectangle{\pgfqpoint{2.125000in}{13.561628in}}{\pgfqpoint{5.489583in}{0.877907in}}%
\pgfusepath{clip}%
\pgfsetbuttcap%
\pgfsetroundjoin%
\pgfsetlinewidth{1.505625pt}%
\definecolor{currentstroke}{rgb}{0.000000,0.000000,0.000000}%
\pgfsetstrokecolor{currentstroke}%
\pgfsetdash{}{0pt}%
\pgfpathmoveto{\pgfqpoint{5.578324in}{13.835489in}}%
\pgfpathlineto{\pgfqpoint{5.578324in}{14.342708in}}%
\pgfusepath{stroke}%
\end{pgfscope}%
\begin{pgfscope}%
\pgfpathrectangle{\pgfqpoint{2.125000in}{13.561628in}}{\pgfqpoint{5.489583in}{0.877907in}}%
\pgfusepath{clip}%
\pgfsetbuttcap%
\pgfsetroundjoin%
\pgfsetlinewidth{1.505625pt}%
\definecolor{currentstroke}{rgb}{0.000000,0.000000,0.000000}%
\pgfsetstrokecolor{currentstroke}%
\pgfsetdash{}{0pt}%
\pgfpathmoveto{\pgfqpoint{5.701547in}{13.835489in}}%
\pgfpathlineto{\pgfqpoint{5.701547in}{14.340452in}}%
\pgfusepath{stroke}%
\end{pgfscope}%
\begin{pgfscope}%
\pgfpathrectangle{\pgfqpoint{2.125000in}{13.561628in}}{\pgfqpoint{5.489583in}{0.877907in}}%
\pgfusepath{clip}%
\pgfsetbuttcap%
\pgfsetroundjoin%
\pgfsetlinewidth{1.505625pt}%
\definecolor{currentstroke}{rgb}{0.000000,0.000000,0.000000}%
\pgfsetstrokecolor{currentstroke}%
\pgfsetdash{}{0pt}%
\pgfpathmoveto{\pgfqpoint{5.824770in}{13.835489in}}%
\pgfpathlineto{\pgfqpoint{5.824770in}{14.338206in}}%
\pgfusepath{stroke}%
\end{pgfscope}%
\begin{pgfscope}%
\pgfpathrectangle{\pgfqpoint{2.125000in}{13.561628in}}{\pgfqpoint{5.489583in}{0.877907in}}%
\pgfusepath{clip}%
\pgfsetbuttcap%
\pgfsetroundjoin%
\pgfsetlinewidth{1.505625pt}%
\definecolor{currentstroke}{rgb}{0.000000,0.000000,0.000000}%
\pgfsetstrokecolor{currentstroke}%
\pgfsetdash{}{0pt}%
\pgfpathmoveto{\pgfqpoint{5.947993in}{13.835489in}}%
\pgfpathlineto{\pgfqpoint{5.947993in}{14.335944in}}%
\pgfusepath{stroke}%
\end{pgfscope}%
\begin{pgfscope}%
\pgfpathrectangle{\pgfqpoint{2.125000in}{13.561628in}}{\pgfqpoint{5.489583in}{0.877907in}}%
\pgfusepath{clip}%
\pgfsetbuttcap%
\pgfsetroundjoin%
\pgfsetlinewidth{1.505625pt}%
\definecolor{currentstroke}{rgb}{0.000000,0.000000,0.000000}%
\pgfsetstrokecolor{currentstroke}%
\pgfsetdash{}{0pt}%
\pgfpathmoveto{\pgfqpoint{6.071216in}{13.835489in}}%
\pgfpathlineto{\pgfqpoint{6.071216in}{14.333612in}}%
\pgfusepath{stroke}%
\end{pgfscope}%
\begin{pgfscope}%
\pgfpathrectangle{\pgfqpoint{2.125000in}{13.561628in}}{\pgfqpoint{5.489583in}{0.877907in}}%
\pgfusepath{clip}%
\pgfsetbuttcap%
\pgfsetroundjoin%
\pgfsetlinewidth{1.505625pt}%
\definecolor{currentstroke}{rgb}{0.000000,0.000000,0.000000}%
\pgfsetstrokecolor{currentstroke}%
\pgfsetdash{}{0pt}%
\pgfpathmoveto{\pgfqpoint{6.194439in}{13.835489in}}%
\pgfpathlineto{\pgfqpoint{6.194439in}{14.331311in}}%
\pgfusepath{stroke}%
\end{pgfscope}%
\begin{pgfscope}%
\pgfpathrectangle{\pgfqpoint{2.125000in}{13.561628in}}{\pgfqpoint{5.489583in}{0.877907in}}%
\pgfusepath{clip}%
\pgfsetbuttcap%
\pgfsetroundjoin%
\pgfsetlinewidth{1.505625pt}%
\definecolor{currentstroke}{rgb}{0.000000,0.000000,0.000000}%
\pgfsetstrokecolor{currentstroke}%
\pgfsetdash{}{0pt}%
\pgfpathmoveto{\pgfqpoint{6.317662in}{13.835489in}}%
\pgfpathlineto{\pgfqpoint{6.317662in}{14.329089in}}%
\pgfusepath{stroke}%
\end{pgfscope}%
\begin{pgfscope}%
\pgfpathrectangle{\pgfqpoint{2.125000in}{13.561628in}}{\pgfqpoint{5.489583in}{0.877907in}}%
\pgfusepath{clip}%
\pgfsetbuttcap%
\pgfsetroundjoin%
\pgfsetlinewidth{1.505625pt}%
\definecolor{currentstroke}{rgb}{0.000000,0.000000,0.000000}%
\pgfsetstrokecolor{currentstroke}%
\pgfsetdash{}{0pt}%
\pgfpathmoveto{\pgfqpoint{6.440885in}{13.835489in}}%
\pgfpathlineto{\pgfqpoint{6.440885in}{14.327145in}}%
\pgfusepath{stroke}%
\end{pgfscope}%
\begin{pgfscope}%
\pgfpathrectangle{\pgfqpoint{2.125000in}{13.561628in}}{\pgfqpoint{5.489583in}{0.877907in}}%
\pgfusepath{clip}%
\pgfsetbuttcap%
\pgfsetroundjoin%
\pgfsetlinewidth{1.505625pt}%
\definecolor{currentstroke}{rgb}{0.000000,0.000000,0.000000}%
\pgfsetstrokecolor{currentstroke}%
\pgfsetdash{}{0pt}%
\pgfpathmoveto{\pgfqpoint{6.564108in}{13.835489in}}%
\pgfpathlineto{\pgfqpoint{6.564108in}{14.325463in}}%
\pgfusepath{stroke}%
\end{pgfscope}%
\begin{pgfscope}%
\pgfpathrectangle{\pgfqpoint{2.125000in}{13.561628in}}{\pgfqpoint{5.489583in}{0.877907in}}%
\pgfusepath{clip}%
\pgfsetbuttcap%
\pgfsetroundjoin%
\pgfsetlinewidth{1.505625pt}%
\definecolor{currentstroke}{rgb}{0.000000,0.000000,0.000000}%
\pgfsetstrokecolor{currentstroke}%
\pgfsetdash{}{0pt}%
\pgfpathmoveto{\pgfqpoint{6.687330in}{13.835489in}}%
\pgfpathlineto{\pgfqpoint{6.687330in}{14.323599in}}%
\pgfusepath{stroke}%
\end{pgfscope}%
\begin{pgfscope}%
\pgfpathrectangle{\pgfqpoint{2.125000in}{13.561628in}}{\pgfqpoint{5.489583in}{0.877907in}}%
\pgfusepath{clip}%
\pgfsetbuttcap%
\pgfsetroundjoin%
\pgfsetlinewidth{1.505625pt}%
\definecolor{currentstroke}{rgb}{0.000000,0.000000,0.000000}%
\pgfsetstrokecolor{currentstroke}%
\pgfsetdash{}{0pt}%
\pgfpathmoveto{\pgfqpoint{6.810553in}{13.835489in}}%
\pgfpathlineto{\pgfqpoint{6.810553in}{14.321759in}}%
\pgfusepath{stroke}%
\end{pgfscope}%
\begin{pgfscope}%
\pgfpathrectangle{\pgfqpoint{2.125000in}{13.561628in}}{\pgfqpoint{5.489583in}{0.877907in}}%
\pgfusepath{clip}%
\pgfsetbuttcap%
\pgfsetroundjoin%
\pgfsetlinewidth{1.505625pt}%
\definecolor{currentstroke}{rgb}{0.000000,0.000000,0.000000}%
\pgfsetstrokecolor{currentstroke}%
\pgfsetdash{}{0pt}%
\pgfpathmoveto{\pgfqpoint{6.933776in}{13.835489in}}%
\pgfpathlineto{\pgfqpoint{6.933776in}{14.319873in}}%
\pgfusepath{stroke}%
\end{pgfscope}%
\begin{pgfscope}%
\pgfpathrectangle{\pgfqpoint{2.125000in}{13.561628in}}{\pgfqpoint{5.489583in}{0.877907in}}%
\pgfusepath{clip}%
\pgfsetbuttcap%
\pgfsetroundjoin%
\pgfsetlinewidth{1.505625pt}%
\definecolor{currentstroke}{rgb}{0.000000,0.000000,0.000000}%
\pgfsetstrokecolor{currentstroke}%
\pgfsetdash{}{0pt}%
\pgfpathmoveto{\pgfqpoint{7.056999in}{13.835489in}}%
\pgfpathlineto{\pgfqpoint{7.056999in}{14.317976in}}%
\pgfusepath{stroke}%
\end{pgfscope}%
\begin{pgfscope}%
\pgfpathrectangle{\pgfqpoint{2.125000in}{13.561628in}}{\pgfqpoint{5.489583in}{0.877907in}}%
\pgfusepath{clip}%
\pgfsetbuttcap%
\pgfsetroundjoin%
\pgfsetlinewidth{1.505625pt}%
\definecolor{currentstroke}{rgb}{0.000000,0.000000,0.000000}%
\pgfsetstrokecolor{currentstroke}%
\pgfsetdash{}{0pt}%
\pgfpathmoveto{\pgfqpoint{7.180222in}{13.835489in}}%
\pgfpathlineto{\pgfqpoint{7.180222in}{14.316161in}}%
\pgfusepath{stroke}%
\end{pgfscope}%
\begin{pgfscope}%
\pgfpathrectangle{\pgfqpoint{2.125000in}{13.561628in}}{\pgfqpoint{5.489583in}{0.877907in}}%
\pgfusepath{clip}%
\pgfsetbuttcap%
\pgfsetroundjoin%
\pgfsetlinewidth{1.505625pt}%
\definecolor{currentstroke}{rgb}{0.000000,0.000000,0.000000}%
\pgfsetstrokecolor{currentstroke}%
\pgfsetdash{}{0pt}%
\pgfpathmoveto{\pgfqpoint{7.303445in}{13.835489in}}%
\pgfpathlineto{\pgfqpoint{7.303445in}{14.314258in}}%
\pgfusepath{stroke}%
\end{pgfscope}%
\begin{pgfscope}%
\pgfpathrectangle{\pgfqpoint{2.125000in}{13.561628in}}{\pgfqpoint{5.489583in}{0.877907in}}%
\pgfusepath{clip}%
\pgfsetroundcap%
\pgfsetroundjoin%
\pgfsetlinewidth{1.505625pt}%
\definecolor{currentstroke}{rgb}{0.121569,0.466667,0.705882}%
\pgfsetstrokecolor{currentstroke}%
\pgfsetdash{}{0pt}%
\pgfpathmoveto{\pgfqpoint{2.125000in}{13.835489in}}%
\pgfpathlineto{\pgfqpoint{7.614583in}{13.835489in}}%
\pgfusepath{stroke}%
\end{pgfscope}%
\begin{pgfscope}%
\pgfpathrectangle{\pgfqpoint{2.125000in}{13.561628in}}{\pgfqpoint{5.489583in}{0.877907in}}%
\pgfusepath{clip}%
\pgfsetbuttcap%
\pgfsetroundjoin%
\definecolor{currentfill}{rgb}{0.121569,0.466667,0.705882}%
\pgfsetfillcolor{currentfill}%
\pgfsetlinewidth{1.003750pt}%
\definecolor{currentstroke}{rgb}{0.121569,0.466667,0.705882}%
\pgfsetstrokecolor{currentstroke}%
\pgfsetdash{}{0pt}%
\pgfsys@defobject{currentmarker}{\pgfqpoint{-0.034722in}{-0.034722in}}{\pgfqpoint{0.034722in}{0.034722in}}{%
\pgfpathmoveto{\pgfqpoint{0.000000in}{-0.034722in}}%
\pgfpathcurveto{\pgfqpoint{0.009208in}{-0.034722in}}{\pgfqpoint{0.018041in}{-0.031064in}}{\pgfqpoint{0.024552in}{-0.024552in}}%
\pgfpathcurveto{\pgfqpoint{0.031064in}{-0.018041in}}{\pgfqpoint{0.034722in}{-0.009208in}}{\pgfqpoint{0.034722in}{0.000000in}}%
\pgfpathcurveto{\pgfqpoint{0.034722in}{0.009208in}}{\pgfqpoint{0.031064in}{0.018041in}}{\pgfqpoint{0.024552in}{0.024552in}}%
\pgfpathcurveto{\pgfqpoint{0.018041in}{0.031064in}}{\pgfqpoint{0.009208in}{0.034722in}}{\pgfqpoint{0.000000in}{0.034722in}}%
\pgfpathcurveto{\pgfqpoint{-0.009208in}{0.034722in}}{\pgfqpoint{-0.018041in}{0.031064in}}{\pgfqpoint{-0.024552in}{0.024552in}}%
\pgfpathcurveto{\pgfqpoint{-0.031064in}{0.018041in}}{\pgfqpoint{-0.034722in}{0.009208in}}{\pgfqpoint{-0.034722in}{0.000000in}}%
\pgfpathcurveto{\pgfqpoint{-0.034722in}{-0.009208in}}{\pgfqpoint{-0.031064in}{-0.018041in}}{\pgfqpoint{-0.024552in}{-0.024552in}}%
\pgfpathcurveto{\pgfqpoint{-0.018041in}{-0.031064in}}{\pgfqpoint{-0.009208in}{-0.034722in}}{\pgfqpoint{0.000000in}{-0.034722in}}%
\pgfpathclose%
\pgfusepath{stroke,fill}%
}%
\begin{pgfscope}%
\pgfsys@transformshift{2.374527in}{14.399630in}%
\pgfsys@useobject{currentmarker}{}%
\end{pgfscope}%
\begin{pgfscope}%
\pgfsys@transformshift{2.497749in}{14.397362in}%
\pgfsys@useobject{currentmarker}{}%
\end{pgfscope}%
\begin{pgfscope}%
\pgfsys@transformshift{2.620972in}{14.395030in}%
\pgfsys@useobject{currentmarker}{}%
\end{pgfscope}%
\begin{pgfscope}%
\pgfsys@transformshift{2.744195in}{14.392807in}%
\pgfsys@useobject{currentmarker}{}%
\end{pgfscope}%
\begin{pgfscope}%
\pgfsys@transformshift{2.867418in}{14.390575in}%
\pgfsys@useobject{currentmarker}{}%
\end{pgfscope}%
\begin{pgfscope}%
\pgfsys@transformshift{2.990641in}{14.388411in}%
\pgfsys@useobject{currentmarker}{}%
\end{pgfscope}%
\begin{pgfscope}%
\pgfsys@transformshift{3.113864in}{14.386256in}%
\pgfsys@useobject{currentmarker}{}%
\end{pgfscope}%
\begin{pgfscope}%
\pgfsys@transformshift{3.237087in}{14.384126in}%
\pgfsys@useobject{currentmarker}{}%
\end{pgfscope}%
\begin{pgfscope}%
\pgfsys@transformshift{3.360310in}{14.382124in}%
\pgfsys@useobject{currentmarker}{}%
\end{pgfscope}%
\begin{pgfscope}%
\pgfsys@transformshift{3.483533in}{14.380057in}%
\pgfsys@useobject{currentmarker}{}%
\end{pgfscope}%
\begin{pgfscope}%
\pgfsys@transformshift{3.606756in}{14.378057in}%
\pgfsys@useobject{currentmarker}{}%
\end{pgfscope}%
\begin{pgfscope}%
\pgfsys@transformshift{3.729979in}{14.376005in}%
\pgfsys@useobject{currentmarker}{}%
\end{pgfscope}%
\begin{pgfscope}%
\pgfsys@transformshift{3.853202in}{14.373965in}%
\pgfsys@useobject{currentmarker}{}%
\end{pgfscope}%
\begin{pgfscope}%
\pgfsys@transformshift{3.976425in}{14.371866in}%
\pgfsys@useobject{currentmarker}{}%
\end{pgfscope}%
\begin{pgfscope}%
\pgfsys@transformshift{4.099648in}{14.369584in}%
\pgfsys@useobject{currentmarker}{}%
\end{pgfscope}%
\begin{pgfscope}%
\pgfsys@transformshift{4.222871in}{14.367296in}%
\pgfsys@useobject{currentmarker}{}%
\end{pgfscope}%
\begin{pgfscope}%
\pgfsys@transformshift{4.346094in}{14.365092in}%
\pgfsys@useobject{currentmarker}{}%
\end{pgfscope}%
\begin{pgfscope}%
\pgfsys@transformshift{4.469317in}{14.362772in}%
\pgfsys@useobject{currentmarker}{}%
\end{pgfscope}%
\begin{pgfscope}%
\pgfsys@transformshift{4.592540in}{14.360499in}%
\pgfsys@useobject{currentmarker}{}%
\end{pgfscope}%
\begin{pgfscope}%
\pgfsys@transformshift{4.715763in}{14.358170in}%
\pgfsys@useobject{currentmarker}{}%
\end{pgfscope}%
\begin{pgfscope}%
\pgfsys@transformshift{4.838986in}{14.355879in}%
\pgfsys@useobject{currentmarker}{}%
\end{pgfscope}%
\begin{pgfscope}%
\pgfsys@transformshift{4.962209in}{14.353680in}%
\pgfsys@useobject{currentmarker}{}%
\end{pgfscope}%
\begin{pgfscope}%
\pgfsys@transformshift{5.085432in}{14.351460in}%
\pgfsys@useobject{currentmarker}{}%
\end{pgfscope}%
\begin{pgfscope}%
\pgfsys@transformshift{5.208655in}{14.349347in}%
\pgfsys@useobject{currentmarker}{}%
\end{pgfscope}%
\begin{pgfscope}%
\pgfsys@transformshift{5.331878in}{14.347147in}%
\pgfsys@useobject{currentmarker}{}%
\end{pgfscope}%
\begin{pgfscope}%
\pgfsys@transformshift{5.455101in}{14.344897in}%
\pgfsys@useobject{currentmarker}{}%
\end{pgfscope}%
\begin{pgfscope}%
\pgfsys@transformshift{5.578324in}{14.342708in}%
\pgfsys@useobject{currentmarker}{}%
\end{pgfscope}%
\begin{pgfscope}%
\pgfsys@transformshift{5.701547in}{14.340452in}%
\pgfsys@useobject{currentmarker}{}%
\end{pgfscope}%
\begin{pgfscope}%
\pgfsys@transformshift{5.824770in}{14.338206in}%
\pgfsys@useobject{currentmarker}{}%
\end{pgfscope}%
\begin{pgfscope}%
\pgfsys@transformshift{5.947993in}{14.335944in}%
\pgfsys@useobject{currentmarker}{}%
\end{pgfscope}%
\begin{pgfscope}%
\pgfsys@transformshift{6.071216in}{14.333612in}%
\pgfsys@useobject{currentmarker}{}%
\end{pgfscope}%
\begin{pgfscope}%
\pgfsys@transformshift{6.194439in}{14.331311in}%
\pgfsys@useobject{currentmarker}{}%
\end{pgfscope}%
\begin{pgfscope}%
\pgfsys@transformshift{6.317662in}{14.329089in}%
\pgfsys@useobject{currentmarker}{}%
\end{pgfscope}%
\begin{pgfscope}%
\pgfsys@transformshift{6.440885in}{14.327145in}%
\pgfsys@useobject{currentmarker}{}%
\end{pgfscope}%
\begin{pgfscope}%
\pgfsys@transformshift{6.564108in}{14.325463in}%
\pgfsys@useobject{currentmarker}{}%
\end{pgfscope}%
\begin{pgfscope}%
\pgfsys@transformshift{6.687330in}{14.323599in}%
\pgfsys@useobject{currentmarker}{}%
\end{pgfscope}%
\begin{pgfscope}%
\pgfsys@transformshift{6.810553in}{14.321759in}%
\pgfsys@useobject{currentmarker}{}%
\end{pgfscope}%
\begin{pgfscope}%
\pgfsys@transformshift{6.933776in}{14.319873in}%
\pgfsys@useobject{currentmarker}{}%
\end{pgfscope}%
\begin{pgfscope}%
\pgfsys@transformshift{7.056999in}{14.317976in}%
\pgfsys@useobject{currentmarker}{}%
\end{pgfscope}%
\begin{pgfscope}%
\pgfsys@transformshift{7.180222in}{14.316161in}%
\pgfsys@useobject{currentmarker}{}%
\end{pgfscope}%
\begin{pgfscope}%
\pgfsys@transformshift{7.303445in}{14.314258in}%
\pgfsys@useobject{currentmarker}{}%
\end{pgfscope}%
\end{pgfscope}%
\begin{pgfscope}%
\pgfsetrectcap%
\pgfsetmiterjoin%
\pgfsetlinewidth{0.803000pt}%
\definecolor{currentstroke}{rgb}{1.000000,1.000000,1.000000}%
\pgfsetstrokecolor{currentstroke}%
\pgfsetdash{}{0pt}%
\pgfpathmoveto{\pgfqpoint{2.125000in}{13.561628in}}%
\pgfpathlineto{\pgfqpoint{2.125000in}{14.439535in}}%
\pgfusepath{stroke}%
\end{pgfscope}%
\begin{pgfscope}%
\pgfsetrectcap%
\pgfsetmiterjoin%
\pgfsetlinewidth{0.803000pt}%
\definecolor{currentstroke}{rgb}{1.000000,1.000000,1.000000}%
\pgfsetstrokecolor{currentstroke}%
\pgfsetdash{}{0pt}%
\pgfpathmoveto{\pgfqpoint{7.614583in}{13.561628in}}%
\pgfpathlineto{\pgfqpoint{7.614583in}{14.439535in}}%
\pgfusepath{stroke}%
\end{pgfscope}%
\begin{pgfscope}%
\pgfsetrectcap%
\pgfsetmiterjoin%
\pgfsetlinewidth{0.803000pt}%
\definecolor{currentstroke}{rgb}{1.000000,1.000000,1.000000}%
\pgfsetstrokecolor{currentstroke}%
\pgfsetdash{}{0pt}%
\pgfpathmoveto{\pgfqpoint{2.125000in}{13.561628in}}%
\pgfpathlineto{\pgfqpoint{7.614583in}{13.561628in}}%
\pgfusepath{stroke}%
\end{pgfscope}%
\begin{pgfscope}%
\pgfsetrectcap%
\pgfsetmiterjoin%
\pgfsetlinewidth{0.803000pt}%
\definecolor{currentstroke}{rgb}{1.000000,1.000000,1.000000}%
\pgfsetstrokecolor{currentstroke}%
\pgfsetdash{}{0pt}%
\pgfpathmoveto{\pgfqpoint{2.125000in}{14.439535in}}%
\pgfpathlineto{\pgfqpoint{7.614583in}{14.439535in}}%
\pgfusepath{stroke}%
\end{pgfscope}%
\begin{pgfscope}%
\definecolor{textcolor}{rgb}{0.150000,0.150000,0.150000}%
\pgfsetstrokecolor{textcolor}%
\pgfsetfillcolor{textcolor}%
\pgftext[x=4.869792in,y=14.522868in,,base]{\color{textcolor}\rmfamily\fontsize{16.800000}{20.160000}\selectfont Autocorrelation}%
\end{pgfscope}%
\begin{pgfscope}%
\pgfsetbuttcap%
\pgfsetmiterjoin%
\definecolor{currentfill}{rgb}{0.917647,0.917647,0.949020}%
\pgfsetfillcolor{currentfill}%
\pgfsetlinewidth{0.000000pt}%
\definecolor{currentstroke}{rgb}{0.000000,0.000000,0.000000}%
\pgfsetstrokecolor{currentstroke}%
\pgfsetstrokeopacity{0.000000}%
\pgfsetdash{}{0pt}%
\pgfpathmoveto{\pgfqpoint{9.810417in}{13.561628in}}%
\pgfpathlineto{\pgfqpoint{15.300000in}{13.561628in}}%
\pgfpathlineto{\pgfqpoint{15.300000in}{14.439535in}}%
\pgfpathlineto{\pgfqpoint{9.810417in}{14.439535in}}%
\pgfpathclose%
\pgfusepath{fill}%
\end{pgfscope}%
\begin{pgfscope}%
\pgfpathrectangle{\pgfqpoint{9.810417in}{13.561628in}}{\pgfqpoint{5.489583in}{0.877907in}}%
\pgfusepath{clip}%
\pgfsetroundcap%
\pgfsetroundjoin%
\pgfsetlinewidth{0.803000pt}%
\definecolor{currentstroke}{rgb}{1.000000,1.000000,1.000000}%
\pgfsetstrokecolor{currentstroke}%
\pgfsetdash{}{0pt}%
\pgfpathmoveto{\pgfqpoint{10.059943in}{13.561628in}}%
\pgfpathlineto{\pgfqpoint{10.059943in}{14.439535in}}%
\pgfusepath{stroke}%
\end{pgfscope}%
\begin{pgfscope}%
\definecolor{textcolor}{rgb}{0.150000,0.150000,0.150000}%
\pgfsetstrokecolor{textcolor}%
\pgfsetfillcolor{textcolor}%
\pgftext[x=10.059943in,y=13.464406in,,top]{\color{textcolor}\rmfamily\fontsize{14.000000}{16.800000}\selectfont 0}%
\end{pgfscope}%
\begin{pgfscope}%
\pgfpathrectangle{\pgfqpoint{9.810417in}{13.561628in}}{\pgfqpoint{5.489583in}{0.877907in}}%
\pgfusepath{clip}%
\pgfsetroundcap%
\pgfsetroundjoin%
\pgfsetlinewidth{0.803000pt}%
\definecolor{currentstroke}{rgb}{1.000000,1.000000,1.000000}%
\pgfsetstrokecolor{currentstroke}%
\pgfsetdash{}{0pt}%
\pgfpathmoveto{\pgfqpoint{10.676058in}{13.561628in}}%
\pgfpathlineto{\pgfqpoint{10.676058in}{14.439535in}}%
\pgfusepath{stroke}%
\end{pgfscope}%
\begin{pgfscope}%
\definecolor{textcolor}{rgb}{0.150000,0.150000,0.150000}%
\pgfsetstrokecolor{textcolor}%
\pgfsetfillcolor{textcolor}%
\pgftext[x=10.676058in,y=13.464406in,,top]{\color{textcolor}\rmfamily\fontsize{14.000000}{16.800000}\selectfont 5}%
\end{pgfscope}%
\begin{pgfscope}%
\pgfpathrectangle{\pgfqpoint{9.810417in}{13.561628in}}{\pgfqpoint{5.489583in}{0.877907in}}%
\pgfusepath{clip}%
\pgfsetroundcap%
\pgfsetroundjoin%
\pgfsetlinewidth{0.803000pt}%
\definecolor{currentstroke}{rgb}{1.000000,1.000000,1.000000}%
\pgfsetstrokecolor{currentstroke}%
\pgfsetdash{}{0pt}%
\pgfpathmoveto{\pgfqpoint{11.292173in}{13.561628in}}%
\pgfpathlineto{\pgfqpoint{11.292173in}{14.439535in}}%
\pgfusepath{stroke}%
\end{pgfscope}%
\begin{pgfscope}%
\definecolor{textcolor}{rgb}{0.150000,0.150000,0.150000}%
\pgfsetstrokecolor{textcolor}%
\pgfsetfillcolor{textcolor}%
\pgftext[x=11.292173in,y=13.464406in,,top]{\color{textcolor}\rmfamily\fontsize{14.000000}{16.800000}\selectfont 10}%
\end{pgfscope}%
\begin{pgfscope}%
\pgfpathrectangle{\pgfqpoint{9.810417in}{13.561628in}}{\pgfqpoint{5.489583in}{0.877907in}}%
\pgfusepath{clip}%
\pgfsetroundcap%
\pgfsetroundjoin%
\pgfsetlinewidth{0.803000pt}%
\definecolor{currentstroke}{rgb}{1.000000,1.000000,1.000000}%
\pgfsetstrokecolor{currentstroke}%
\pgfsetdash{}{0pt}%
\pgfpathmoveto{\pgfqpoint{11.908288in}{13.561628in}}%
\pgfpathlineto{\pgfqpoint{11.908288in}{14.439535in}}%
\pgfusepath{stroke}%
\end{pgfscope}%
\begin{pgfscope}%
\definecolor{textcolor}{rgb}{0.150000,0.150000,0.150000}%
\pgfsetstrokecolor{textcolor}%
\pgfsetfillcolor{textcolor}%
\pgftext[x=11.908288in,y=13.464406in,,top]{\color{textcolor}\rmfamily\fontsize{14.000000}{16.800000}\selectfont 15}%
\end{pgfscope}%
\begin{pgfscope}%
\pgfpathrectangle{\pgfqpoint{9.810417in}{13.561628in}}{\pgfqpoint{5.489583in}{0.877907in}}%
\pgfusepath{clip}%
\pgfsetroundcap%
\pgfsetroundjoin%
\pgfsetlinewidth{0.803000pt}%
\definecolor{currentstroke}{rgb}{1.000000,1.000000,1.000000}%
\pgfsetstrokecolor{currentstroke}%
\pgfsetdash{}{0pt}%
\pgfpathmoveto{\pgfqpoint{12.524403in}{13.561628in}}%
\pgfpathlineto{\pgfqpoint{12.524403in}{14.439535in}}%
\pgfusepath{stroke}%
\end{pgfscope}%
\begin{pgfscope}%
\definecolor{textcolor}{rgb}{0.150000,0.150000,0.150000}%
\pgfsetstrokecolor{textcolor}%
\pgfsetfillcolor{textcolor}%
\pgftext[x=12.524403in,y=13.464406in,,top]{\color{textcolor}\rmfamily\fontsize{14.000000}{16.800000}\selectfont 20}%
\end{pgfscope}%
\begin{pgfscope}%
\pgfpathrectangle{\pgfqpoint{9.810417in}{13.561628in}}{\pgfqpoint{5.489583in}{0.877907in}}%
\pgfusepath{clip}%
\pgfsetroundcap%
\pgfsetroundjoin%
\pgfsetlinewidth{0.803000pt}%
\definecolor{currentstroke}{rgb}{1.000000,1.000000,1.000000}%
\pgfsetstrokecolor{currentstroke}%
\pgfsetdash{}{0pt}%
\pgfpathmoveto{\pgfqpoint{13.140517in}{13.561628in}}%
\pgfpathlineto{\pgfqpoint{13.140517in}{14.439535in}}%
\pgfusepath{stroke}%
\end{pgfscope}%
\begin{pgfscope}%
\definecolor{textcolor}{rgb}{0.150000,0.150000,0.150000}%
\pgfsetstrokecolor{textcolor}%
\pgfsetfillcolor{textcolor}%
\pgftext[x=13.140517in,y=13.464406in,,top]{\color{textcolor}\rmfamily\fontsize{14.000000}{16.800000}\selectfont 25}%
\end{pgfscope}%
\begin{pgfscope}%
\pgfpathrectangle{\pgfqpoint{9.810417in}{13.561628in}}{\pgfqpoint{5.489583in}{0.877907in}}%
\pgfusepath{clip}%
\pgfsetroundcap%
\pgfsetroundjoin%
\pgfsetlinewidth{0.803000pt}%
\definecolor{currentstroke}{rgb}{1.000000,1.000000,1.000000}%
\pgfsetstrokecolor{currentstroke}%
\pgfsetdash{}{0pt}%
\pgfpathmoveto{\pgfqpoint{13.756632in}{13.561628in}}%
\pgfpathlineto{\pgfqpoint{13.756632in}{14.439535in}}%
\pgfusepath{stroke}%
\end{pgfscope}%
\begin{pgfscope}%
\definecolor{textcolor}{rgb}{0.150000,0.150000,0.150000}%
\pgfsetstrokecolor{textcolor}%
\pgfsetfillcolor{textcolor}%
\pgftext[x=13.756632in,y=13.464406in,,top]{\color{textcolor}\rmfamily\fontsize{14.000000}{16.800000}\selectfont 30}%
\end{pgfscope}%
\begin{pgfscope}%
\pgfpathrectangle{\pgfqpoint{9.810417in}{13.561628in}}{\pgfqpoint{5.489583in}{0.877907in}}%
\pgfusepath{clip}%
\pgfsetroundcap%
\pgfsetroundjoin%
\pgfsetlinewidth{0.803000pt}%
\definecolor{currentstroke}{rgb}{1.000000,1.000000,1.000000}%
\pgfsetstrokecolor{currentstroke}%
\pgfsetdash{}{0pt}%
\pgfpathmoveto{\pgfqpoint{14.372747in}{13.561628in}}%
\pgfpathlineto{\pgfqpoint{14.372747in}{14.439535in}}%
\pgfusepath{stroke}%
\end{pgfscope}%
\begin{pgfscope}%
\definecolor{textcolor}{rgb}{0.150000,0.150000,0.150000}%
\pgfsetstrokecolor{textcolor}%
\pgfsetfillcolor{textcolor}%
\pgftext[x=14.372747in,y=13.464406in,,top]{\color{textcolor}\rmfamily\fontsize{14.000000}{16.800000}\selectfont 35}%
\end{pgfscope}%
\begin{pgfscope}%
\pgfpathrectangle{\pgfqpoint{9.810417in}{13.561628in}}{\pgfqpoint{5.489583in}{0.877907in}}%
\pgfusepath{clip}%
\pgfsetroundcap%
\pgfsetroundjoin%
\pgfsetlinewidth{0.803000pt}%
\definecolor{currentstroke}{rgb}{1.000000,1.000000,1.000000}%
\pgfsetstrokecolor{currentstroke}%
\pgfsetdash{}{0pt}%
\pgfpathmoveto{\pgfqpoint{14.988862in}{13.561628in}}%
\pgfpathlineto{\pgfqpoint{14.988862in}{14.439535in}}%
\pgfusepath{stroke}%
\end{pgfscope}%
\begin{pgfscope}%
\definecolor{textcolor}{rgb}{0.150000,0.150000,0.150000}%
\pgfsetstrokecolor{textcolor}%
\pgfsetfillcolor{textcolor}%
\pgftext[x=14.988862in,y=13.464406in,,top]{\color{textcolor}\rmfamily\fontsize{14.000000}{16.800000}\selectfont 40}%
\end{pgfscope}%
\begin{pgfscope}%
\pgfpathrectangle{\pgfqpoint{9.810417in}{13.561628in}}{\pgfqpoint{5.489583in}{0.877907in}}%
\pgfusepath{clip}%
\pgfsetroundcap%
\pgfsetroundjoin%
\pgfsetlinewidth{0.803000pt}%
\definecolor{currentstroke}{rgb}{1.000000,1.000000,1.000000}%
\pgfsetstrokecolor{currentstroke}%
\pgfsetdash{}{0pt}%
\pgfpathmoveto{\pgfqpoint{9.810417in}{13.639867in}}%
\pgfpathlineto{\pgfqpoint{15.300000in}{13.639867in}}%
\pgfusepath{stroke}%
\end{pgfscope}%
\begin{pgfscope}%
\definecolor{textcolor}{rgb}{0.150000,0.150000,0.150000}%
\pgfsetstrokecolor{textcolor}%
\pgfsetfillcolor{textcolor}%
\pgftext[x=9.589483in,y=13.566000in,left,base]{\color{textcolor}\rmfamily\fontsize{14.000000}{16.800000}\selectfont 0}%
\end{pgfscope}%
\begin{pgfscope}%
\pgfpathrectangle{\pgfqpoint{9.810417in}{13.561628in}}{\pgfqpoint{5.489583in}{0.877907in}}%
\pgfusepath{clip}%
\pgfsetroundcap%
\pgfsetroundjoin%
\pgfsetlinewidth{0.803000pt}%
\definecolor{currentstroke}{rgb}{1.000000,1.000000,1.000000}%
\pgfsetstrokecolor{currentstroke}%
\pgfsetdash{}{0pt}%
\pgfpathmoveto{\pgfqpoint{9.810417in}{14.399630in}}%
\pgfpathlineto{\pgfqpoint{15.300000in}{14.399630in}}%
\pgfusepath{stroke}%
\end{pgfscope}%
\begin{pgfscope}%
\definecolor{textcolor}{rgb}{0.150000,0.150000,0.150000}%
\pgfsetstrokecolor{textcolor}%
\pgfsetfillcolor{textcolor}%
\pgftext[x=9.589483in,y=14.325764in,left,base]{\color{textcolor}\rmfamily\fontsize{14.000000}{16.800000}\selectfont 1}%
\end{pgfscope}%
\begin{pgfscope}%
\pgfpathrectangle{\pgfqpoint{9.810417in}{13.561628in}}{\pgfqpoint{5.489583in}{0.877907in}}%
\pgfusepath{clip}%
\pgfsetbuttcap%
\pgfsetroundjoin%
\definecolor{currentfill}{rgb}{0.121569,0.466667,0.705882}%
\pgfsetfillcolor{currentfill}%
\pgfsetfillopacity{0.250000}%
\pgfsetlinewidth{1.003750pt}%
\definecolor{currentstroke}{rgb}{1.000000,1.000000,1.000000}%
\pgfsetstrokecolor{currentstroke}%
\pgfsetstrokeopacity{0.250000}%
\pgfsetdash{}{0pt}%
\pgfpathmoveto{\pgfqpoint{10.121555in}{13.678200in}}%
\pgfpathlineto{\pgfqpoint{10.121555in}{13.601533in}}%
\pgfpathlineto{\pgfqpoint{10.306389in}{13.601533in}}%
\pgfpathlineto{\pgfqpoint{10.429612in}{13.601533in}}%
\pgfpathlineto{\pgfqpoint{10.552835in}{13.601533in}}%
\pgfpathlineto{\pgfqpoint{10.676058in}{13.601533in}}%
\pgfpathlineto{\pgfqpoint{10.799281in}{13.601533in}}%
\pgfpathlineto{\pgfqpoint{10.922504in}{13.601533in}}%
\pgfpathlineto{\pgfqpoint{11.045727in}{13.601533in}}%
\pgfpathlineto{\pgfqpoint{11.168950in}{13.601533in}}%
\pgfpathlineto{\pgfqpoint{11.292173in}{13.601533in}}%
\pgfpathlineto{\pgfqpoint{11.415396in}{13.601533in}}%
\pgfpathlineto{\pgfqpoint{11.538619in}{13.601533in}}%
\pgfpathlineto{\pgfqpoint{11.661842in}{13.601533in}}%
\pgfpathlineto{\pgfqpoint{11.785065in}{13.601533in}}%
\pgfpathlineto{\pgfqpoint{11.908288in}{13.601533in}}%
\pgfpathlineto{\pgfqpoint{12.031511in}{13.601533in}}%
\pgfpathlineto{\pgfqpoint{12.154734in}{13.601533in}}%
\pgfpathlineto{\pgfqpoint{12.277957in}{13.601533in}}%
\pgfpathlineto{\pgfqpoint{12.401180in}{13.601533in}}%
\pgfpathlineto{\pgfqpoint{12.524403in}{13.601533in}}%
\pgfpathlineto{\pgfqpoint{12.647626in}{13.601533in}}%
\pgfpathlineto{\pgfqpoint{12.770849in}{13.601533in}}%
\pgfpathlineto{\pgfqpoint{12.894072in}{13.601533in}}%
\pgfpathlineto{\pgfqpoint{13.017294in}{13.601533in}}%
\pgfpathlineto{\pgfqpoint{13.140517in}{13.601533in}}%
\pgfpathlineto{\pgfqpoint{13.263740in}{13.601533in}}%
\pgfpathlineto{\pgfqpoint{13.386963in}{13.601533in}}%
\pgfpathlineto{\pgfqpoint{13.510186in}{13.601533in}}%
\pgfpathlineto{\pgfqpoint{13.633409in}{13.601533in}}%
\pgfpathlineto{\pgfqpoint{13.756632in}{13.601533in}}%
\pgfpathlineto{\pgfqpoint{13.879855in}{13.601533in}}%
\pgfpathlineto{\pgfqpoint{14.003078in}{13.601533in}}%
\pgfpathlineto{\pgfqpoint{14.126301in}{13.601533in}}%
\pgfpathlineto{\pgfqpoint{14.249524in}{13.601533in}}%
\pgfpathlineto{\pgfqpoint{14.372747in}{13.601533in}}%
\pgfpathlineto{\pgfqpoint{14.495970in}{13.601533in}}%
\pgfpathlineto{\pgfqpoint{14.619193in}{13.601533in}}%
\pgfpathlineto{\pgfqpoint{14.742416in}{13.601533in}}%
\pgfpathlineto{\pgfqpoint{14.865639in}{13.601533in}}%
\pgfpathlineto{\pgfqpoint{15.050473in}{13.601533in}}%
\pgfpathlineto{\pgfqpoint{15.050473in}{13.678200in}}%
\pgfpathlineto{\pgfqpoint{15.050473in}{13.678200in}}%
\pgfpathlineto{\pgfqpoint{14.865639in}{13.678200in}}%
\pgfpathlineto{\pgfqpoint{14.742416in}{13.678200in}}%
\pgfpathlineto{\pgfqpoint{14.619193in}{13.678200in}}%
\pgfpathlineto{\pgfqpoint{14.495970in}{13.678200in}}%
\pgfpathlineto{\pgfqpoint{14.372747in}{13.678200in}}%
\pgfpathlineto{\pgfqpoint{14.249524in}{13.678200in}}%
\pgfpathlineto{\pgfqpoint{14.126301in}{13.678200in}}%
\pgfpathlineto{\pgfqpoint{14.003078in}{13.678200in}}%
\pgfpathlineto{\pgfqpoint{13.879855in}{13.678200in}}%
\pgfpathlineto{\pgfqpoint{13.756632in}{13.678200in}}%
\pgfpathlineto{\pgfqpoint{13.633409in}{13.678200in}}%
\pgfpathlineto{\pgfqpoint{13.510186in}{13.678200in}}%
\pgfpathlineto{\pgfqpoint{13.386963in}{13.678200in}}%
\pgfpathlineto{\pgfqpoint{13.263740in}{13.678200in}}%
\pgfpathlineto{\pgfqpoint{13.140517in}{13.678200in}}%
\pgfpathlineto{\pgfqpoint{13.017294in}{13.678200in}}%
\pgfpathlineto{\pgfqpoint{12.894072in}{13.678200in}}%
\pgfpathlineto{\pgfqpoint{12.770849in}{13.678200in}}%
\pgfpathlineto{\pgfqpoint{12.647626in}{13.678200in}}%
\pgfpathlineto{\pgfqpoint{12.524403in}{13.678200in}}%
\pgfpathlineto{\pgfqpoint{12.401180in}{13.678200in}}%
\pgfpathlineto{\pgfqpoint{12.277957in}{13.678200in}}%
\pgfpathlineto{\pgfqpoint{12.154734in}{13.678200in}}%
\pgfpathlineto{\pgfqpoint{12.031511in}{13.678200in}}%
\pgfpathlineto{\pgfqpoint{11.908288in}{13.678200in}}%
\pgfpathlineto{\pgfqpoint{11.785065in}{13.678200in}}%
\pgfpathlineto{\pgfqpoint{11.661842in}{13.678200in}}%
\pgfpathlineto{\pgfqpoint{11.538619in}{13.678200in}}%
\pgfpathlineto{\pgfqpoint{11.415396in}{13.678200in}}%
\pgfpathlineto{\pgfqpoint{11.292173in}{13.678200in}}%
\pgfpathlineto{\pgfqpoint{11.168950in}{13.678200in}}%
\pgfpathlineto{\pgfqpoint{11.045727in}{13.678200in}}%
\pgfpathlineto{\pgfqpoint{10.922504in}{13.678200in}}%
\pgfpathlineto{\pgfqpoint{10.799281in}{13.678200in}}%
\pgfpathlineto{\pgfqpoint{10.676058in}{13.678200in}}%
\pgfpathlineto{\pgfqpoint{10.552835in}{13.678200in}}%
\pgfpathlineto{\pgfqpoint{10.429612in}{13.678200in}}%
\pgfpathlineto{\pgfqpoint{10.306389in}{13.678200in}}%
\pgfpathlineto{\pgfqpoint{10.121555in}{13.678200in}}%
\pgfpathclose%
\pgfusepath{stroke,fill}%
\end{pgfscope}%
\begin{pgfscope}%
\pgfpathrectangle{\pgfqpoint{9.810417in}{13.561628in}}{\pgfqpoint{5.489583in}{0.877907in}}%
\pgfusepath{clip}%
\pgfsetbuttcap%
\pgfsetroundjoin%
\pgfsetlinewidth{1.505625pt}%
\definecolor{currentstroke}{rgb}{0.000000,0.000000,0.000000}%
\pgfsetstrokecolor{currentstroke}%
\pgfsetdash{}{0pt}%
\pgfpathmoveto{\pgfqpoint{10.059943in}{13.639867in}}%
\pgfpathlineto{\pgfqpoint{10.059943in}{14.399630in}}%
\pgfusepath{stroke}%
\end{pgfscope}%
\begin{pgfscope}%
\pgfpathrectangle{\pgfqpoint{9.810417in}{13.561628in}}{\pgfqpoint{5.489583in}{0.877907in}}%
\pgfusepath{clip}%
\pgfsetbuttcap%
\pgfsetroundjoin%
\pgfsetlinewidth{1.505625pt}%
\definecolor{currentstroke}{rgb}{0.000000,0.000000,0.000000}%
\pgfsetstrokecolor{currentstroke}%
\pgfsetdash{}{0pt}%
\pgfpathmoveto{\pgfqpoint{10.183166in}{13.639867in}}%
\pgfpathlineto{\pgfqpoint{10.183166in}{14.397078in}}%
\pgfusepath{stroke}%
\end{pgfscope}%
\begin{pgfscope}%
\pgfpathrectangle{\pgfqpoint{9.810417in}{13.561628in}}{\pgfqpoint{5.489583in}{0.877907in}}%
\pgfusepath{clip}%
\pgfsetbuttcap%
\pgfsetroundjoin%
\pgfsetlinewidth{1.505625pt}%
\definecolor{currentstroke}{rgb}{0.000000,0.000000,0.000000}%
\pgfsetstrokecolor{currentstroke}%
\pgfsetdash{}{0pt}%
\pgfpathmoveto{\pgfqpoint{10.306389in}{13.639867in}}%
\pgfpathlineto{\pgfqpoint{10.306389in}{13.625105in}}%
\pgfusepath{stroke}%
\end{pgfscope}%
\begin{pgfscope}%
\pgfpathrectangle{\pgfqpoint{9.810417in}{13.561628in}}{\pgfqpoint{5.489583in}{0.877907in}}%
\pgfusepath{clip}%
\pgfsetbuttcap%
\pgfsetroundjoin%
\pgfsetlinewidth{1.505625pt}%
\definecolor{currentstroke}{rgb}{0.000000,0.000000,0.000000}%
\pgfsetstrokecolor{currentstroke}%
\pgfsetdash{}{0pt}%
\pgfpathmoveto{\pgfqpoint{10.429612in}{13.639867in}}%
\pgfpathlineto{\pgfqpoint{10.429612in}{13.660246in}}%
\pgfusepath{stroke}%
\end{pgfscope}%
\begin{pgfscope}%
\pgfpathrectangle{\pgfqpoint{9.810417in}{13.561628in}}{\pgfqpoint{5.489583in}{0.877907in}}%
\pgfusepath{clip}%
\pgfsetbuttcap%
\pgfsetroundjoin%
\pgfsetlinewidth{1.505625pt}%
\definecolor{currentstroke}{rgb}{0.000000,0.000000,0.000000}%
\pgfsetstrokecolor{currentstroke}%
\pgfsetdash{}{0pt}%
\pgfpathmoveto{\pgfqpoint{10.552835in}{13.639867in}}%
\pgfpathlineto{\pgfqpoint{10.552835in}{13.635487in}}%
\pgfusepath{stroke}%
\end{pgfscope}%
\begin{pgfscope}%
\pgfpathrectangle{\pgfqpoint{9.810417in}{13.561628in}}{\pgfqpoint{5.489583in}{0.877907in}}%
\pgfusepath{clip}%
\pgfsetbuttcap%
\pgfsetroundjoin%
\pgfsetlinewidth{1.505625pt}%
\definecolor{currentstroke}{rgb}{0.000000,0.000000,0.000000}%
\pgfsetstrokecolor{currentstroke}%
\pgfsetdash{}{0pt}%
\pgfpathmoveto{\pgfqpoint{10.676058in}{13.639867in}}%
\pgfpathlineto{\pgfqpoint{10.676058in}{13.652440in}}%
\pgfusepath{stroke}%
\end{pgfscope}%
\begin{pgfscope}%
\pgfpathrectangle{\pgfqpoint{9.810417in}{13.561628in}}{\pgfqpoint{5.489583in}{0.877907in}}%
\pgfusepath{clip}%
\pgfsetbuttcap%
\pgfsetroundjoin%
\pgfsetlinewidth{1.505625pt}%
\definecolor{currentstroke}{rgb}{0.000000,0.000000,0.000000}%
\pgfsetstrokecolor{currentstroke}%
\pgfsetdash{}{0pt}%
\pgfpathmoveto{\pgfqpoint{10.799281in}{13.639867in}}%
\pgfpathlineto{\pgfqpoint{10.799281in}{13.639411in}}%
\pgfusepath{stroke}%
\end{pgfscope}%
\begin{pgfscope}%
\pgfpathrectangle{\pgfqpoint{9.810417in}{13.561628in}}{\pgfqpoint{5.489583in}{0.877907in}}%
\pgfusepath{clip}%
\pgfsetbuttcap%
\pgfsetroundjoin%
\pgfsetlinewidth{1.505625pt}%
\definecolor{currentstroke}{rgb}{0.000000,0.000000,0.000000}%
\pgfsetstrokecolor{currentstroke}%
\pgfsetdash{}{0pt}%
\pgfpathmoveto{\pgfqpoint{10.922504in}{13.639867in}}%
\pgfpathlineto{\pgfqpoint{10.922504in}{13.644059in}}%
\pgfusepath{stroke}%
\end{pgfscope}%
\begin{pgfscope}%
\pgfpathrectangle{\pgfqpoint{9.810417in}{13.561628in}}{\pgfqpoint{5.489583in}{0.877907in}}%
\pgfusepath{clip}%
\pgfsetbuttcap%
\pgfsetroundjoin%
\pgfsetlinewidth{1.505625pt}%
\definecolor{currentstroke}{rgb}{0.000000,0.000000,0.000000}%
\pgfsetstrokecolor{currentstroke}%
\pgfsetdash{}{0pt}%
\pgfpathmoveto{\pgfqpoint{11.045727in}{13.639867in}}%
\pgfpathlineto{\pgfqpoint{11.045727in}{13.663785in}}%
\pgfusepath{stroke}%
\end{pgfscope}%
\begin{pgfscope}%
\pgfpathrectangle{\pgfqpoint{9.810417in}{13.561628in}}{\pgfqpoint{5.489583in}{0.877907in}}%
\pgfusepath{clip}%
\pgfsetbuttcap%
\pgfsetroundjoin%
\pgfsetlinewidth{1.505625pt}%
\definecolor{currentstroke}{rgb}{0.000000,0.000000,0.000000}%
\pgfsetstrokecolor{currentstroke}%
\pgfsetdash{}{0pt}%
\pgfpathmoveto{\pgfqpoint{11.168950in}{13.639867in}}%
\pgfpathlineto{\pgfqpoint{11.168950in}{13.624725in}}%
\pgfusepath{stroke}%
\end{pgfscope}%
\begin{pgfscope}%
\pgfpathrectangle{\pgfqpoint{9.810417in}{13.561628in}}{\pgfqpoint{5.489583in}{0.877907in}}%
\pgfusepath{clip}%
\pgfsetbuttcap%
\pgfsetroundjoin%
\pgfsetlinewidth{1.505625pt}%
\definecolor{currentstroke}{rgb}{0.000000,0.000000,0.000000}%
\pgfsetstrokecolor{currentstroke}%
\pgfsetdash{}{0pt}%
\pgfpathmoveto{\pgfqpoint{11.292173in}{13.639867in}}%
\pgfpathlineto{\pgfqpoint{11.292173in}{13.653613in}}%
\pgfusepath{stroke}%
\end{pgfscope}%
\begin{pgfscope}%
\pgfpathrectangle{\pgfqpoint{9.810417in}{13.561628in}}{\pgfqpoint{5.489583in}{0.877907in}}%
\pgfusepath{clip}%
\pgfsetbuttcap%
\pgfsetroundjoin%
\pgfsetlinewidth{1.505625pt}%
\definecolor{currentstroke}{rgb}{0.000000,0.000000,0.000000}%
\pgfsetstrokecolor{currentstroke}%
\pgfsetdash{}{0pt}%
\pgfpathmoveto{\pgfqpoint{11.415396in}{13.639867in}}%
\pgfpathlineto{\pgfqpoint{11.415396in}{13.626753in}}%
\pgfusepath{stroke}%
\end{pgfscope}%
\begin{pgfscope}%
\pgfpathrectangle{\pgfqpoint{9.810417in}{13.561628in}}{\pgfqpoint{5.489583in}{0.877907in}}%
\pgfusepath{clip}%
\pgfsetbuttcap%
\pgfsetroundjoin%
\pgfsetlinewidth{1.505625pt}%
\definecolor{currentstroke}{rgb}{0.000000,0.000000,0.000000}%
\pgfsetstrokecolor{currentstroke}%
\pgfsetdash{}{0pt}%
\pgfpathmoveto{\pgfqpoint{11.538619in}{13.639867in}}%
\pgfpathlineto{\pgfqpoint{11.538619in}{13.642618in}}%
\pgfusepath{stroke}%
\end{pgfscope}%
\begin{pgfscope}%
\pgfpathrectangle{\pgfqpoint{9.810417in}{13.561628in}}{\pgfqpoint{5.489583in}{0.877907in}}%
\pgfusepath{clip}%
\pgfsetbuttcap%
\pgfsetroundjoin%
\pgfsetlinewidth{1.505625pt}%
\definecolor{currentstroke}{rgb}{0.000000,0.000000,0.000000}%
\pgfsetstrokecolor{currentstroke}%
\pgfsetdash{}{0pt}%
\pgfpathmoveto{\pgfqpoint{11.661842in}{13.639867in}}%
\pgfpathlineto{\pgfqpoint{11.661842in}{13.625213in}}%
\pgfusepath{stroke}%
\end{pgfscope}%
\begin{pgfscope}%
\pgfpathrectangle{\pgfqpoint{9.810417in}{13.561628in}}{\pgfqpoint{5.489583in}{0.877907in}}%
\pgfusepath{clip}%
\pgfsetbuttcap%
\pgfsetroundjoin%
\pgfsetlinewidth{1.505625pt}%
\definecolor{currentstroke}{rgb}{0.000000,0.000000,0.000000}%
\pgfsetstrokecolor{currentstroke}%
\pgfsetdash{}{0pt}%
\pgfpathmoveto{\pgfqpoint{11.785065in}{13.639867in}}%
\pgfpathlineto{\pgfqpoint{11.785065in}{13.602345in}}%
\pgfusepath{stroke}%
\end{pgfscope}%
\begin{pgfscope}%
\pgfpathrectangle{\pgfqpoint{9.810417in}{13.561628in}}{\pgfqpoint{5.489583in}{0.877907in}}%
\pgfusepath{clip}%
\pgfsetbuttcap%
\pgfsetroundjoin%
\pgfsetlinewidth{1.505625pt}%
\definecolor{currentstroke}{rgb}{0.000000,0.000000,0.000000}%
\pgfsetstrokecolor{currentstroke}%
\pgfsetdash{}{0pt}%
\pgfpathmoveto{\pgfqpoint{11.908288in}{13.639867in}}%
\pgfpathlineto{\pgfqpoint{11.908288in}{13.637704in}}%
\pgfusepath{stroke}%
\end{pgfscope}%
\begin{pgfscope}%
\pgfpathrectangle{\pgfqpoint{9.810417in}{13.561628in}}{\pgfqpoint{5.489583in}{0.877907in}}%
\pgfusepath{clip}%
\pgfsetbuttcap%
\pgfsetroundjoin%
\pgfsetlinewidth{1.505625pt}%
\definecolor{currentstroke}{rgb}{0.000000,0.000000,0.000000}%
\pgfsetstrokecolor{currentstroke}%
\pgfsetdash{}{0pt}%
\pgfpathmoveto{\pgfqpoint{12.031511in}{13.639867in}}%
\pgfpathlineto{\pgfqpoint{12.031511in}{13.652483in}}%
\pgfusepath{stroke}%
\end{pgfscope}%
\begin{pgfscope}%
\pgfpathrectangle{\pgfqpoint{9.810417in}{13.561628in}}{\pgfqpoint{5.489583in}{0.877907in}}%
\pgfusepath{clip}%
\pgfsetbuttcap%
\pgfsetroundjoin%
\pgfsetlinewidth{1.505625pt}%
\definecolor{currentstroke}{rgb}{0.000000,0.000000,0.000000}%
\pgfsetstrokecolor{currentstroke}%
\pgfsetdash{}{0pt}%
\pgfpathmoveto{\pgfqpoint{12.154734in}{13.639867in}}%
\pgfpathlineto{\pgfqpoint{12.154734in}{13.614715in}}%
\pgfusepath{stroke}%
\end{pgfscope}%
\begin{pgfscope}%
\pgfpathrectangle{\pgfqpoint{9.810417in}{13.561628in}}{\pgfqpoint{5.489583in}{0.877907in}}%
\pgfusepath{clip}%
\pgfsetbuttcap%
\pgfsetroundjoin%
\pgfsetlinewidth{1.505625pt}%
\definecolor{currentstroke}{rgb}{0.000000,0.000000,0.000000}%
\pgfsetstrokecolor{currentstroke}%
\pgfsetdash{}{0pt}%
\pgfpathmoveto{\pgfqpoint{12.277957in}{13.639867in}}%
\pgfpathlineto{\pgfqpoint{12.277957in}{13.646557in}}%
\pgfusepath{stroke}%
\end{pgfscope}%
\begin{pgfscope}%
\pgfpathrectangle{\pgfqpoint{9.810417in}{13.561628in}}{\pgfqpoint{5.489583in}{0.877907in}}%
\pgfusepath{clip}%
\pgfsetbuttcap%
\pgfsetroundjoin%
\pgfsetlinewidth{1.505625pt}%
\definecolor{currentstroke}{rgb}{0.000000,0.000000,0.000000}%
\pgfsetstrokecolor{currentstroke}%
\pgfsetdash{}{0pt}%
\pgfpathmoveto{\pgfqpoint{12.401180in}{13.639867in}}%
\pgfpathlineto{\pgfqpoint{12.401180in}{13.625305in}}%
\pgfusepath{stroke}%
\end{pgfscope}%
\begin{pgfscope}%
\pgfpathrectangle{\pgfqpoint{9.810417in}{13.561628in}}{\pgfqpoint{5.489583in}{0.877907in}}%
\pgfusepath{clip}%
\pgfsetbuttcap%
\pgfsetroundjoin%
\pgfsetlinewidth{1.505625pt}%
\definecolor{currentstroke}{rgb}{0.000000,0.000000,0.000000}%
\pgfsetstrokecolor{currentstroke}%
\pgfsetdash{}{0pt}%
\pgfpathmoveto{\pgfqpoint{12.524403in}{13.639867in}}%
\pgfpathlineto{\pgfqpoint{12.524403in}{13.645622in}}%
\pgfusepath{stroke}%
\end{pgfscope}%
\begin{pgfscope}%
\pgfpathrectangle{\pgfqpoint{9.810417in}{13.561628in}}{\pgfqpoint{5.489583in}{0.877907in}}%
\pgfusepath{clip}%
\pgfsetbuttcap%
\pgfsetroundjoin%
\pgfsetlinewidth{1.505625pt}%
\definecolor{currentstroke}{rgb}{0.000000,0.000000,0.000000}%
\pgfsetstrokecolor{currentstroke}%
\pgfsetdash{}{0pt}%
\pgfpathmoveto{\pgfqpoint{12.647626in}{13.639867in}}%
\pgfpathlineto{\pgfqpoint{12.647626in}{13.653251in}}%
\pgfusepath{stroke}%
\end{pgfscope}%
\begin{pgfscope}%
\pgfpathrectangle{\pgfqpoint{9.810417in}{13.561628in}}{\pgfqpoint{5.489583in}{0.877907in}}%
\pgfusepath{clip}%
\pgfsetbuttcap%
\pgfsetroundjoin%
\pgfsetlinewidth{1.505625pt}%
\definecolor{currentstroke}{rgb}{0.000000,0.000000,0.000000}%
\pgfsetstrokecolor{currentstroke}%
\pgfsetdash{}{0pt}%
\pgfpathmoveto{\pgfqpoint{12.770849in}{13.639867in}}%
\pgfpathlineto{\pgfqpoint{12.770849in}{13.634705in}}%
\pgfusepath{stroke}%
\end{pgfscope}%
\begin{pgfscope}%
\pgfpathrectangle{\pgfqpoint{9.810417in}{13.561628in}}{\pgfqpoint{5.489583in}{0.877907in}}%
\pgfusepath{clip}%
\pgfsetbuttcap%
\pgfsetroundjoin%
\pgfsetlinewidth{1.505625pt}%
\definecolor{currentstroke}{rgb}{0.000000,0.000000,0.000000}%
\pgfsetstrokecolor{currentstroke}%
\pgfsetdash{}{0pt}%
\pgfpathmoveto{\pgfqpoint{12.894072in}{13.639867in}}%
\pgfpathlineto{\pgfqpoint{12.894072in}{13.660459in}}%
\pgfusepath{stroke}%
\end{pgfscope}%
\begin{pgfscope}%
\pgfpathrectangle{\pgfqpoint{9.810417in}{13.561628in}}{\pgfqpoint{5.489583in}{0.877907in}}%
\pgfusepath{clip}%
\pgfsetbuttcap%
\pgfsetroundjoin%
\pgfsetlinewidth{1.505625pt}%
\definecolor{currentstroke}{rgb}{0.000000,0.000000,0.000000}%
\pgfsetstrokecolor{currentstroke}%
\pgfsetdash{}{0pt}%
\pgfpathmoveto{\pgfqpoint{13.017294in}{13.639867in}}%
\pgfpathlineto{\pgfqpoint{13.017294in}{13.618121in}}%
\pgfusepath{stroke}%
\end{pgfscope}%
\begin{pgfscope}%
\pgfpathrectangle{\pgfqpoint{9.810417in}{13.561628in}}{\pgfqpoint{5.489583in}{0.877907in}}%
\pgfusepath{clip}%
\pgfsetbuttcap%
\pgfsetroundjoin%
\pgfsetlinewidth{1.505625pt}%
\definecolor{currentstroke}{rgb}{0.000000,0.000000,0.000000}%
\pgfsetstrokecolor{currentstroke}%
\pgfsetdash{}{0pt}%
\pgfpathmoveto{\pgfqpoint{13.140517in}{13.639867in}}%
\pgfpathlineto{\pgfqpoint{13.140517in}{13.632358in}}%
\pgfusepath{stroke}%
\end{pgfscope}%
\begin{pgfscope}%
\pgfpathrectangle{\pgfqpoint{9.810417in}{13.561628in}}{\pgfqpoint{5.489583in}{0.877907in}}%
\pgfusepath{clip}%
\pgfsetbuttcap%
\pgfsetroundjoin%
\pgfsetlinewidth{1.505625pt}%
\definecolor{currentstroke}{rgb}{0.000000,0.000000,0.000000}%
\pgfsetstrokecolor{currentstroke}%
\pgfsetdash{}{0pt}%
\pgfpathmoveto{\pgfqpoint{13.263740in}{13.639867in}}%
\pgfpathlineto{\pgfqpoint{13.263740in}{13.648504in}}%
\pgfusepath{stroke}%
\end{pgfscope}%
\begin{pgfscope}%
\pgfpathrectangle{\pgfqpoint{9.810417in}{13.561628in}}{\pgfqpoint{5.489583in}{0.877907in}}%
\pgfusepath{clip}%
\pgfsetbuttcap%
\pgfsetroundjoin%
\pgfsetlinewidth{1.505625pt}%
\definecolor{currentstroke}{rgb}{0.000000,0.000000,0.000000}%
\pgfsetstrokecolor{currentstroke}%
\pgfsetdash{}{0pt}%
\pgfpathmoveto{\pgfqpoint{13.386963in}{13.639867in}}%
\pgfpathlineto{\pgfqpoint{13.386963in}{13.628033in}}%
\pgfusepath{stroke}%
\end{pgfscope}%
\begin{pgfscope}%
\pgfpathrectangle{\pgfqpoint{9.810417in}{13.561628in}}{\pgfqpoint{5.489583in}{0.877907in}}%
\pgfusepath{clip}%
\pgfsetbuttcap%
\pgfsetroundjoin%
\pgfsetlinewidth{1.505625pt}%
\definecolor{currentstroke}{rgb}{0.000000,0.000000,0.000000}%
\pgfsetstrokecolor{currentstroke}%
\pgfsetdash{}{0pt}%
\pgfpathmoveto{\pgfqpoint{13.510186in}{13.639867in}}%
\pgfpathlineto{\pgfqpoint{13.510186in}{13.640320in}}%
\pgfusepath{stroke}%
\end{pgfscope}%
\begin{pgfscope}%
\pgfpathrectangle{\pgfqpoint{9.810417in}{13.561628in}}{\pgfqpoint{5.489583in}{0.877907in}}%
\pgfusepath{clip}%
\pgfsetbuttcap%
\pgfsetroundjoin%
\pgfsetlinewidth{1.505625pt}%
\definecolor{currentstroke}{rgb}{0.000000,0.000000,0.000000}%
\pgfsetstrokecolor{currentstroke}%
\pgfsetdash{}{0pt}%
\pgfpathmoveto{\pgfqpoint{13.633409in}{13.639867in}}%
\pgfpathlineto{\pgfqpoint{13.633409in}{13.632564in}}%
\pgfusepath{stroke}%
\end{pgfscope}%
\begin{pgfscope}%
\pgfpathrectangle{\pgfqpoint{9.810417in}{13.561628in}}{\pgfqpoint{5.489583in}{0.877907in}}%
\pgfusepath{clip}%
\pgfsetbuttcap%
\pgfsetroundjoin%
\pgfsetlinewidth{1.505625pt}%
\definecolor{currentstroke}{rgb}{0.000000,0.000000,0.000000}%
\pgfsetstrokecolor{currentstroke}%
\pgfsetdash{}{0pt}%
\pgfpathmoveto{\pgfqpoint{13.756632in}{13.639867in}}%
\pgfpathlineto{\pgfqpoint{13.756632in}{13.627775in}}%
\pgfusepath{stroke}%
\end{pgfscope}%
\begin{pgfscope}%
\pgfpathrectangle{\pgfqpoint{9.810417in}{13.561628in}}{\pgfqpoint{5.489583in}{0.877907in}}%
\pgfusepath{clip}%
\pgfsetbuttcap%
\pgfsetroundjoin%
\pgfsetlinewidth{1.505625pt}%
\definecolor{currentstroke}{rgb}{0.000000,0.000000,0.000000}%
\pgfsetstrokecolor{currentstroke}%
\pgfsetdash{}{0pt}%
\pgfpathmoveto{\pgfqpoint{13.879855in}{13.639867in}}%
\pgfpathlineto{\pgfqpoint{13.879855in}{13.641695in}}%
\pgfusepath{stroke}%
\end{pgfscope}%
\begin{pgfscope}%
\pgfpathrectangle{\pgfqpoint{9.810417in}{13.561628in}}{\pgfqpoint{5.489583in}{0.877907in}}%
\pgfusepath{clip}%
\pgfsetbuttcap%
\pgfsetroundjoin%
\pgfsetlinewidth{1.505625pt}%
\definecolor{currentstroke}{rgb}{0.000000,0.000000,0.000000}%
\pgfsetstrokecolor{currentstroke}%
\pgfsetdash{}{0pt}%
\pgfpathmoveto{\pgfqpoint{14.003078in}{13.639867in}}%
\pgfpathlineto{\pgfqpoint{14.003078in}{13.654167in}}%
\pgfusepath{stroke}%
\end{pgfscope}%
\begin{pgfscope}%
\pgfpathrectangle{\pgfqpoint{9.810417in}{13.561628in}}{\pgfqpoint{5.489583in}{0.877907in}}%
\pgfusepath{clip}%
\pgfsetbuttcap%
\pgfsetroundjoin%
\pgfsetlinewidth{1.505625pt}%
\definecolor{currentstroke}{rgb}{0.000000,0.000000,0.000000}%
\pgfsetstrokecolor{currentstroke}%
\pgfsetdash{}{0pt}%
\pgfpathmoveto{\pgfqpoint{14.126301in}{13.639867in}}%
\pgfpathlineto{\pgfqpoint{14.126301in}{13.694911in}}%
\pgfusepath{stroke}%
\end{pgfscope}%
\begin{pgfscope}%
\pgfpathrectangle{\pgfqpoint{9.810417in}{13.561628in}}{\pgfqpoint{5.489583in}{0.877907in}}%
\pgfusepath{clip}%
\pgfsetbuttcap%
\pgfsetroundjoin%
\pgfsetlinewidth{1.505625pt}%
\definecolor{currentstroke}{rgb}{0.000000,0.000000,0.000000}%
\pgfsetstrokecolor{currentstroke}%
\pgfsetdash{}{0pt}%
\pgfpathmoveto{\pgfqpoint{14.249524in}{13.639867in}}%
\pgfpathlineto{\pgfqpoint{14.249524in}{13.687956in}}%
\pgfusepath{stroke}%
\end{pgfscope}%
\begin{pgfscope}%
\pgfpathrectangle{\pgfqpoint{9.810417in}{13.561628in}}{\pgfqpoint{5.489583in}{0.877907in}}%
\pgfusepath{clip}%
\pgfsetbuttcap%
\pgfsetroundjoin%
\pgfsetlinewidth{1.505625pt}%
\definecolor{currentstroke}{rgb}{0.000000,0.000000,0.000000}%
\pgfsetstrokecolor{currentstroke}%
\pgfsetdash{}{0pt}%
\pgfpathmoveto{\pgfqpoint{14.372747in}{13.639867in}}%
\pgfpathlineto{\pgfqpoint{14.372747in}{13.603316in}}%
\pgfusepath{stroke}%
\end{pgfscope}%
\begin{pgfscope}%
\pgfpathrectangle{\pgfqpoint{9.810417in}{13.561628in}}{\pgfqpoint{5.489583in}{0.877907in}}%
\pgfusepath{clip}%
\pgfsetbuttcap%
\pgfsetroundjoin%
\pgfsetlinewidth{1.505625pt}%
\definecolor{currentstroke}{rgb}{0.000000,0.000000,0.000000}%
\pgfsetstrokecolor{currentstroke}%
\pgfsetdash{}{0pt}%
\pgfpathmoveto{\pgfqpoint{14.495970in}{13.639867in}}%
\pgfpathlineto{\pgfqpoint{14.495970in}{13.645697in}}%
\pgfusepath{stroke}%
\end{pgfscope}%
\begin{pgfscope}%
\pgfpathrectangle{\pgfqpoint{9.810417in}{13.561628in}}{\pgfqpoint{5.489583in}{0.877907in}}%
\pgfusepath{clip}%
\pgfsetbuttcap%
\pgfsetroundjoin%
\pgfsetlinewidth{1.505625pt}%
\definecolor{currentstroke}{rgb}{0.000000,0.000000,0.000000}%
\pgfsetstrokecolor{currentstroke}%
\pgfsetdash{}{0pt}%
\pgfpathmoveto{\pgfqpoint{14.619193in}{13.639867in}}%
\pgfpathlineto{\pgfqpoint{14.619193in}{13.628356in}}%
\pgfusepath{stroke}%
\end{pgfscope}%
\begin{pgfscope}%
\pgfpathrectangle{\pgfqpoint{9.810417in}{13.561628in}}{\pgfqpoint{5.489583in}{0.877907in}}%
\pgfusepath{clip}%
\pgfsetbuttcap%
\pgfsetroundjoin%
\pgfsetlinewidth{1.505625pt}%
\definecolor{currentstroke}{rgb}{0.000000,0.000000,0.000000}%
\pgfsetstrokecolor{currentstroke}%
\pgfsetdash{}{0pt}%
\pgfpathmoveto{\pgfqpoint{14.742416in}{13.639867in}}%
\pgfpathlineto{\pgfqpoint{14.742416in}{13.641580in}}%
\pgfusepath{stroke}%
\end{pgfscope}%
\begin{pgfscope}%
\pgfpathrectangle{\pgfqpoint{9.810417in}{13.561628in}}{\pgfqpoint{5.489583in}{0.877907in}}%
\pgfusepath{clip}%
\pgfsetbuttcap%
\pgfsetroundjoin%
\pgfsetlinewidth{1.505625pt}%
\definecolor{currentstroke}{rgb}{0.000000,0.000000,0.000000}%
\pgfsetstrokecolor{currentstroke}%
\pgfsetdash{}{0pt}%
\pgfpathmoveto{\pgfqpoint{14.865639in}{13.639867in}}%
\pgfpathlineto{\pgfqpoint{14.865639in}{13.652100in}}%
\pgfusepath{stroke}%
\end{pgfscope}%
\begin{pgfscope}%
\pgfpathrectangle{\pgfqpoint{9.810417in}{13.561628in}}{\pgfqpoint{5.489583in}{0.877907in}}%
\pgfusepath{clip}%
\pgfsetbuttcap%
\pgfsetroundjoin%
\pgfsetlinewidth{1.505625pt}%
\definecolor{currentstroke}{rgb}{0.000000,0.000000,0.000000}%
\pgfsetstrokecolor{currentstroke}%
\pgfsetdash{}{0pt}%
\pgfpathmoveto{\pgfqpoint{14.988862in}{13.639867in}}%
\pgfpathlineto{\pgfqpoint{14.988862in}{13.625585in}}%
\pgfusepath{stroke}%
\end{pgfscope}%
\begin{pgfscope}%
\pgfpathrectangle{\pgfqpoint{9.810417in}{13.561628in}}{\pgfqpoint{5.489583in}{0.877907in}}%
\pgfusepath{clip}%
\pgfsetroundcap%
\pgfsetroundjoin%
\pgfsetlinewidth{1.505625pt}%
\definecolor{currentstroke}{rgb}{0.121569,0.466667,0.705882}%
\pgfsetstrokecolor{currentstroke}%
\pgfsetdash{}{0pt}%
\pgfpathmoveto{\pgfqpoint{9.810417in}{13.639867in}}%
\pgfpathlineto{\pgfqpoint{15.300000in}{13.639867in}}%
\pgfusepath{stroke}%
\end{pgfscope}%
\begin{pgfscope}%
\pgfpathrectangle{\pgfqpoint{9.810417in}{13.561628in}}{\pgfqpoint{5.489583in}{0.877907in}}%
\pgfusepath{clip}%
\pgfsetbuttcap%
\pgfsetroundjoin%
\definecolor{currentfill}{rgb}{0.121569,0.466667,0.705882}%
\pgfsetfillcolor{currentfill}%
\pgfsetlinewidth{1.003750pt}%
\definecolor{currentstroke}{rgb}{0.121569,0.466667,0.705882}%
\pgfsetstrokecolor{currentstroke}%
\pgfsetdash{}{0pt}%
\pgfsys@defobject{currentmarker}{\pgfqpoint{-0.034722in}{-0.034722in}}{\pgfqpoint{0.034722in}{0.034722in}}{%
\pgfpathmoveto{\pgfqpoint{0.000000in}{-0.034722in}}%
\pgfpathcurveto{\pgfqpoint{0.009208in}{-0.034722in}}{\pgfqpoint{0.018041in}{-0.031064in}}{\pgfqpoint{0.024552in}{-0.024552in}}%
\pgfpathcurveto{\pgfqpoint{0.031064in}{-0.018041in}}{\pgfqpoint{0.034722in}{-0.009208in}}{\pgfqpoint{0.034722in}{0.000000in}}%
\pgfpathcurveto{\pgfqpoint{0.034722in}{0.009208in}}{\pgfqpoint{0.031064in}{0.018041in}}{\pgfqpoint{0.024552in}{0.024552in}}%
\pgfpathcurveto{\pgfqpoint{0.018041in}{0.031064in}}{\pgfqpoint{0.009208in}{0.034722in}}{\pgfqpoint{0.000000in}{0.034722in}}%
\pgfpathcurveto{\pgfqpoint{-0.009208in}{0.034722in}}{\pgfqpoint{-0.018041in}{0.031064in}}{\pgfqpoint{-0.024552in}{0.024552in}}%
\pgfpathcurveto{\pgfqpoint{-0.031064in}{0.018041in}}{\pgfqpoint{-0.034722in}{0.009208in}}{\pgfqpoint{-0.034722in}{0.000000in}}%
\pgfpathcurveto{\pgfqpoint{-0.034722in}{-0.009208in}}{\pgfqpoint{-0.031064in}{-0.018041in}}{\pgfqpoint{-0.024552in}{-0.024552in}}%
\pgfpathcurveto{\pgfqpoint{-0.018041in}{-0.031064in}}{\pgfqpoint{-0.009208in}{-0.034722in}}{\pgfqpoint{0.000000in}{-0.034722in}}%
\pgfpathclose%
\pgfusepath{stroke,fill}%
}%
\begin{pgfscope}%
\pgfsys@transformshift{10.059943in}{14.399630in}%
\pgfsys@useobject{currentmarker}{}%
\end{pgfscope}%
\begin{pgfscope}%
\pgfsys@transformshift{10.183166in}{14.397078in}%
\pgfsys@useobject{currentmarker}{}%
\end{pgfscope}%
\begin{pgfscope}%
\pgfsys@transformshift{10.306389in}{13.625105in}%
\pgfsys@useobject{currentmarker}{}%
\end{pgfscope}%
\begin{pgfscope}%
\pgfsys@transformshift{10.429612in}{13.660246in}%
\pgfsys@useobject{currentmarker}{}%
\end{pgfscope}%
\begin{pgfscope}%
\pgfsys@transformshift{10.552835in}{13.635487in}%
\pgfsys@useobject{currentmarker}{}%
\end{pgfscope}%
\begin{pgfscope}%
\pgfsys@transformshift{10.676058in}{13.652440in}%
\pgfsys@useobject{currentmarker}{}%
\end{pgfscope}%
\begin{pgfscope}%
\pgfsys@transformshift{10.799281in}{13.639411in}%
\pgfsys@useobject{currentmarker}{}%
\end{pgfscope}%
\begin{pgfscope}%
\pgfsys@transformshift{10.922504in}{13.644059in}%
\pgfsys@useobject{currentmarker}{}%
\end{pgfscope}%
\begin{pgfscope}%
\pgfsys@transformshift{11.045727in}{13.663785in}%
\pgfsys@useobject{currentmarker}{}%
\end{pgfscope}%
\begin{pgfscope}%
\pgfsys@transformshift{11.168950in}{13.624725in}%
\pgfsys@useobject{currentmarker}{}%
\end{pgfscope}%
\begin{pgfscope}%
\pgfsys@transformshift{11.292173in}{13.653613in}%
\pgfsys@useobject{currentmarker}{}%
\end{pgfscope}%
\begin{pgfscope}%
\pgfsys@transformshift{11.415396in}{13.626753in}%
\pgfsys@useobject{currentmarker}{}%
\end{pgfscope}%
\begin{pgfscope}%
\pgfsys@transformshift{11.538619in}{13.642618in}%
\pgfsys@useobject{currentmarker}{}%
\end{pgfscope}%
\begin{pgfscope}%
\pgfsys@transformshift{11.661842in}{13.625213in}%
\pgfsys@useobject{currentmarker}{}%
\end{pgfscope}%
\begin{pgfscope}%
\pgfsys@transformshift{11.785065in}{13.602345in}%
\pgfsys@useobject{currentmarker}{}%
\end{pgfscope}%
\begin{pgfscope}%
\pgfsys@transformshift{11.908288in}{13.637704in}%
\pgfsys@useobject{currentmarker}{}%
\end{pgfscope}%
\begin{pgfscope}%
\pgfsys@transformshift{12.031511in}{13.652483in}%
\pgfsys@useobject{currentmarker}{}%
\end{pgfscope}%
\begin{pgfscope}%
\pgfsys@transformshift{12.154734in}{13.614715in}%
\pgfsys@useobject{currentmarker}{}%
\end{pgfscope}%
\begin{pgfscope}%
\pgfsys@transformshift{12.277957in}{13.646557in}%
\pgfsys@useobject{currentmarker}{}%
\end{pgfscope}%
\begin{pgfscope}%
\pgfsys@transformshift{12.401180in}{13.625305in}%
\pgfsys@useobject{currentmarker}{}%
\end{pgfscope}%
\begin{pgfscope}%
\pgfsys@transformshift{12.524403in}{13.645622in}%
\pgfsys@useobject{currentmarker}{}%
\end{pgfscope}%
\begin{pgfscope}%
\pgfsys@transformshift{12.647626in}{13.653251in}%
\pgfsys@useobject{currentmarker}{}%
\end{pgfscope}%
\begin{pgfscope}%
\pgfsys@transformshift{12.770849in}{13.634705in}%
\pgfsys@useobject{currentmarker}{}%
\end{pgfscope}%
\begin{pgfscope}%
\pgfsys@transformshift{12.894072in}{13.660459in}%
\pgfsys@useobject{currentmarker}{}%
\end{pgfscope}%
\begin{pgfscope}%
\pgfsys@transformshift{13.017294in}{13.618121in}%
\pgfsys@useobject{currentmarker}{}%
\end{pgfscope}%
\begin{pgfscope}%
\pgfsys@transformshift{13.140517in}{13.632358in}%
\pgfsys@useobject{currentmarker}{}%
\end{pgfscope}%
\begin{pgfscope}%
\pgfsys@transformshift{13.263740in}{13.648504in}%
\pgfsys@useobject{currentmarker}{}%
\end{pgfscope}%
\begin{pgfscope}%
\pgfsys@transformshift{13.386963in}{13.628033in}%
\pgfsys@useobject{currentmarker}{}%
\end{pgfscope}%
\begin{pgfscope}%
\pgfsys@transformshift{13.510186in}{13.640320in}%
\pgfsys@useobject{currentmarker}{}%
\end{pgfscope}%
\begin{pgfscope}%
\pgfsys@transformshift{13.633409in}{13.632564in}%
\pgfsys@useobject{currentmarker}{}%
\end{pgfscope}%
\begin{pgfscope}%
\pgfsys@transformshift{13.756632in}{13.627775in}%
\pgfsys@useobject{currentmarker}{}%
\end{pgfscope}%
\begin{pgfscope}%
\pgfsys@transformshift{13.879855in}{13.641695in}%
\pgfsys@useobject{currentmarker}{}%
\end{pgfscope}%
\begin{pgfscope}%
\pgfsys@transformshift{14.003078in}{13.654167in}%
\pgfsys@useobject{currentmarker}{}%
\end{pgfscope}%
\begin{pgfscope}%
\pgfsys@transformshift{14.126301in}{13.694911in}%
\pgfsys@useobject{currentmarker}{}%
\end{pgfscope}%
\begin{pgfscope}%
\pgfsys@transformshift{14.249524in}{13.687956in}%
\pgfsys@useobject{currentmarker}{}%
\end{pgfscope}%
\begin{pgfscope}%
\pgfsys@transformshift{14.372747in}{13.603316in}%
\pgfsys@useobject{currentmarker}{}%
\end{pgfscope}%
\begin{pgfscope}%
\pgfsys@transformshift{14.495970in}{13.645697in}%
\pgfsys@useobject{currentmarker}{}%
\end{pgfscope}%
\begin{pgfscope}%
\pgfsys@transformshift{14.619193in}{13.628356in}%
\pgfsys@useobject{currentmarker}{}%
\end{pgfscope}%
\begin{pgfscope}%
\pgfsys@transformshift{14.742416in}{13.641580in}%
\pgfsys@useobject{currentmarker}{}%
\end{pgfscope}%
\begin{pgfscope}%
\pgfsys@transformshift{14.865639in}{13.652100in}%
\pgfsys@useobject{currentmarker}{}%
\end{pgfscope}%
\begin{pgfscope}%
\pgfsys@transformshift{14.988862in}{13.625585in}%
\pgfsys@useobject{currentmarker}{}%
\end{pgfscope}%
\end{pgfscope}%
\begin{pgfscope}%
\pgfsetrectcap%
\pgfsetmiterjoin%
\pgfsetlinewidth{0.803000pt}%
\definecolor{currentstroke}{rgb}{1.000000,1.000000,1.000000}%
\pgfsetstrokecolor{currentstroke}%
\pgfsetdash{}{0pt}%
\pgfpathmoveto{\pgfqpoint{9.810417in}{13.561628in}}%
\pgfpathlineto{\pgfqpoint{9.810417in}{14.439535in}}%
\pgfusepath{stroke}%
\end{pgfscope}%
\begin{pgfscope}%
\pgfsetrectcap%
\pgfsetmiterjoin%
\pgfsetlinewidth{0.803000pt}%
\definecolor{currentstroke}{rgb}{1.000000,1.000000,1.000000}%
\pgfsetstrokecolor{currentstroke}%
\pgfsetdash{}{0pt}%
\pgfpathmoveto{\pgfqpoint{15.300000in}{13.561628in}}%
\pgfpathlineto{\pgfqpoint{15.300000in}{14.439535in}}%
\pgfusepath{stroke}%
\end{pgfscope}%
\begin{pgfscope}%
\pgfsetrectcap%
\pgfsetmiterjoin%
\pgfsetlinewidth{0.803000pt}%
\definecolor{currentstroke}{rgb}{1.000000,1.000000,1.000000}%
\pgfsetstrokecolor{currentstroke}%
\pgfsetdash{}{0pt}%
\pgfpathmoveto{\pgfqpoint{9.810417in}{13.561628in}}%
\pgfpathlineto{\pgfqpoint{15.300000in}{13.561628in}}%
\pgfusepath{stroke}%
\end{pgfscope}%
\begin{pgfscope}%
\pgfsetrectcap%
\pgfsetmiterjoin%
\pgfsetlinewidth{0.803000pt}%
\definecolor{currentstroke}{rgb}{1.000000,1.000000,1.000000}%
\pgfsetstrokecolor{currentstroke}%
\pgfsetdash{}{0pt}%
\pgfpathmoveto{\pgfqpoint{9.810417in}{14.439535in}}%
\pgfpathlineto{\pgfqpoint{15.300000in}{14.439535in}}%
\pgfusepath{stroke}%
\end{pgfscope}%
\begin{pgfscope}%
\definecolor{textcolor}{rgb}{0.150000,0.150000,0.150000}%
\pgfsetstrokecolor{textcolor}%
\pgfsetfillcolor{textcolor}%
\pgftext[x=12.555208in,y=14.522868in,,base]{\color{textcolor}\rmfamily\fontsize{16.800000}{20.160000}\selectfont Partial Autocorrelation}%
\end{pgfscope}%
\begin{pgfscope}%
\pgfsetbuttcap%
\pgfsetmiterjoin%
\definecolor{currentfill}{rgb}{0.917647,0.917647,0.949020}%
\pgfsetfillcolor{currentfill}%
\pgfsetlinewidth{0.000000pt}%
\definecolor{currentstroke}{rgb}{0.000000,0.000000,0.000000}%
\pgfsetstrokecolor{currentstroke}%
\pgfsetstrokeopacity{0.000000}%
\pgfsetdash{}{0pt}%
\pgfpathmoveto{\pgfqpoint{2.125000in}{11.981395in}}%
\pgfpathlineto{\pgfqpoint{7.614583in}{11.981395in}}%
\pgfpathlineto{\pgfqpoint{7.614583in}{12.859302in}}%
\pgfpathlineto{\pgfqpoint{2.125000in}{12.859302in}}%
\pgfpathclose%
\pgfusepath{fill}%
\end{pgfscope}%
\begin{pgfscope}%
\pgfpathrectangle{\pgfqpoint{2.125000in}{11.981395in}}{\pgfqpoint{5.489583in}{0.877907in}}%
\pgfusepath{clip}%
\pgfsetroundcap%
\pgfsetroundjoin%
\pgfsetlinewidth{0.803000pt}%
\definecolor{currentstroke}{rgb}{1.000000,1.000000,1.000000}%
\pgfsetstrokecolor{currentstroke}%
\pgfsetdash{}{0pt}%
\pgfpathmoveto{\pgfqpoint{2.374527in}{11.981395in}}%
\pgfpathlineto{\pgfqpoint{2.374527in}{12.859302in}}%
\pgfusepath{stroke}%
\end{pgfscope}%
\begin{pgfscope}%
\definecolor{textcolor}{rgb}{0.150000,0.150000,0.150000}%
\pgfsetstrokecolor{textcolor}%
\pgfsetfillcolor{textcolor}%
\pgftext[x=2.374527in,y=11.884173in,,top]{\color{textcolor}\rmfamily\fontsize{14.000000}{16.800000}\selectfont 0}%
\end{pgfscope}%
\begin{pgfscope}%
\pgfpathrectangle{\pgfqpoint{2.125000in}{11.981395in}}{\pgfqpoint{5.489583in}{0.877907in}}%
\pgfusepath{clip}%
\pgfsetroundcap%
\pgfsetroundjoin%
\pgfsetlinewidth{0.803000pt}%
\definecolor{currentstroke}{rgb}{1.000000,1.000000,1.000000}%
\pgfsetstrokecolor{currentstroke}%
\pgfsetdash{}{0pt}%
\pgfpathmoveto{\pgfqpoint{2.990641in}{11.981395in}}%
\pgfpathlineto{\pgfqpoint{2.990641in}{12.859302in}}%
\pgfusepath{stroke}%
\end{pgfscope}%
\begin{pgfscope}%
\definecolor{textcolor}{rgb}{0.150000,0.150000,0.150000}%
\pgfsetstrokecolor{textcolor}%
\pgfsetfillcolor{textcolor}%
\pgftext[x=2.990641in,y=11.884173in,,top]{\color{textcolor}\rmfamily\fontsize{14.000000}{16.800000}\selectfont 5}%
\end{pgfscope}%
\begin{pgfscope}%
\pgfpathrectangle{\pgfqpoint{2.125000in}{11.981395in}}{\pgfqpoint{5.489583in}{0.877907in}}%
\pgfusepath{clip}%
\pgfsetroundcap%
\pgfsetroundjoin%
\pgfsetlinewidth{0.803000pt}%
\definecolor{currentstroke}{rgb}{1.000000,1.000000,1.000000}%
\pgfsetstrokecolor{currentstroke}%
\pgfsetdash{}{0pt}%
\pgfpathmoveto{\pgfqpoint{3.606756in}{11.981395in}}%
\pgfpathlineto{\pgfqpoint{3.606756in}{12.859302in}}%
\pgfusepath{stroke}%
\end{pgfscope}%
\begin{pgfscope}%
\definecolor{textcolor}{rgb}{0.150000,0.150000,0.150000}%
\pgfsetstrokecolor{textcolor}%
\pgfsetfillcolor{textcolor}%
\pgftext[x=3.606756in,y=11.884173in,,top]{\color{textcolor}\rmfamily\fontsize{14.000000}{16.800000}\selectfont 10}%
\end{pgfscope}%
\begin{pgfscope}%
\pgfpathrectangle{\pgfqpoint{2.125000in}{11.981395in}}{\pgfqpoint{5.489583in}{0.877907in}}%
\pgfusepath{clip}%
\pgfsetroundcap%
\pgfsetroundjoin%
\pgfsetlinewidth{0.803000pt}%
\definecolor{currentstroke}{rgb}{1.000000,1.000000,1.000000}%
\pgfsetstrokecolor{currentstroke}%
\pgfsetdash{}{0pt}%
\pgfpathmoveto{\pgfqpoint{4.222871in}{11.981395in}}%
\pgfpathlineto{\pgfqpoint{4.222871in}{12.859302in}}%
\pgfusepath{stroke}%
\end{pgfscope}%
\begin{pgfscope}%
\definecolor{textcolor}{rgb}{0.150000,0.150000,0.150000}%
\pgfsetstrokecolor{textcolor}%
\pgfsetfillcolor{textcolor}%
\pgftext[x=4.222871in,y=11.884173in,,top]{\color{textcolor}\rmfamily\fontsize{14.000000}{16.800000}\selectfont 15}%
\end{pgfscope}%
\begin{pgfscope}%
\pgfpathrectangle{\pgfqpoint{2.125000in}{11.981395in}}{\pgfqpoint{5.489583in}{0.877907in}}%
\pgfusepath{clip}%
\pgfsetroundcap%
\pgfsetroundjoin%
\pgfsetlinewidth{0.803000pt}%
\definecolor{currentstroke}{rgb}{1.000000,1.000000,1.000000}%
\pgfsetstrokecolor{currentstroke}%
\pgfsetdash{}{0pt}%
\pgfpathmoveto{\pgfqpoint{4.838986in}{11.981395in}}%
\pgfpathlineto{\pgfqpoint{4.838986in}{12.859302in}}%
\pgfusepath{stroke}%
\end{pgfscope}%
\begin{pgfscope}%
\definecolor{textcolor}{rgb}{0.150000,0.150000,0.150000}%
\pgfsetstrokecolor{textcolor}%
\pgfsetfillcolor{textcolor}%
\pgftext[x=4.838986in,y=11.884173in,,top]{\color{textcolor}\rmfamily\fontsize{14.000000}{16.800000}\selectfont 20}%
\end{pgfscope}%
\begin{pgfscope}%
\pgfpathrectangle{\pgfqpoint{2.125000in}{11.981395in}}{\pgfqpoint{5.489583in}{0.877907in}}%
\pgfusepath{clip}%
\pgfsetroundcap%
\pgfsetroundjoin%
\pgfsetlinewidth{0.803000pt}%
\definecolor{currentstroke}{rgb}{1.000000,1.000000,1.000000}%
\pgfsetstrokecolor{currentstroke}%
\pgfsetdash{}{0pt}%
\pgfpathmoveto{\pgfqpoint{5.455101in}{11.981395in}}%
\pgfpathlineto{\pgfqpoint{5.455101in}{12.859302in}}%
\pgfusepath{stroke}%
\end{pgfscope}%
\begin{pgfscope}%
\definecolor{textcolor}{rgb}{0.150000,0.150000,0.150000}%
\pgfsetstrokecolor{textcolor}%
\pgfsetfillcolor{textcolor}%
\pgftext[x=5.455101in,y=11.884173in,,top]{\color{textcolor}\rmfamily\fontsize{14.000000}{16.800000}\selectfont 25}%
\end{pgfscope}%
\begin{pgfscope}%
\pgfpathrectangle{\pgfqpoint{2.125000in}{11.981395in}}{\pgfqpoint{5.489583in}{0.877907in}}%
\pgfusepath{clip}%
\pgfsetroundcap%
\pgfsetroundjoin%
\pgfsetlinewidth{0.803000pt}%
\definecolor{currentstroke}{rgb}{1.000000,1.000000,1.000000}%
\pgfsetstrokecolor{currentstroke}%
\pgfsetdash{}{0pt}%
\pgfpathmoveto{\pgfqpoint{6.071216in}{11.981395in}}%
\pgfpathlineto{\pgfqpoint{6.071216in}{12.859302in}}%
\pgfusepath{stroke}%
\end{pgfscope}%
\begin{pgfscope}%
\definecolor{textcolor}{rgb}{0.150000,0.150000,0.150000}%
\pgfsetstrokecolor{textcolor}%
\pgfsetfillcolor{textcolor}%
\pgftext[x=6.071216in,y=11.884173in,,top]{\color{textcolor}\rmfamily\fontsize{14.000000}{16.800000}\selectfont 30}%
\end{pgfscope}%
\begin{pgfscope}%
\pgfpathrectangle{\pgfqpoint{2.125000in}{11.981395in}}{\pgfqpoint{5.489583in}{0.877907in}}%
\pgfusepath{clip}%
\pgfsetroundcap%
\pgfsetroundjoin%
\pgfsetlinewidth{0.803000pt}%
\definecolor{currentstroke}{rgb}{1.000000,1.000000,1.000000}%
\pgfsetstrokecolor{currentstroke}%
\pgfsetdash{}{0pt}%
\pgfpathmoveto{\pgfqpoint{6.687330in}{11.981395in}}%
\pgfpathlineto{\pgfqpoint{6.687330in}{12.859302in}}%
\pgfusepath{stroke}%
\end{pgfscope}%
\begin{pgfscope}%
\definecolor{textcolor}{rgb}{0.150000,0.150000,0.150000}%
\pgfsetstrokecolor{textcolor}%
\pgfsetfillcolor{textcolor}%
\pgftext[x=6.687330in,y=11.884173in,,top]{\color{textcolor}\rmfamily\fontsize{14.000000}{16.800000}\selectfont 35}%
\end{pgfscope}%
\begin{pgfscope}%
\pgfpathrectangle{\pgfqpoint{2.125000in}{11.981395in}}{\pgfqpoint{5.489583in}{0.877907in}}%
\pgfusepath{clip}%
\pgfsetroundcap%
\pgfsetroundjoin%
\pgfsetlinewidth{0.803000pt}%
\definecolor{currentstroke}{rgb}{1.000000,1.000000,1.000000}%
\pgfsetstrokecolor{currentstroke}%
\pgfsetdash{}{0pt}%
\pgfpathmoveto{\pgfqpoint{7.303445in}{11.981395in}}%
\pgfpathlineto{\pgfqpoint{7.303445in}{12.859302in}}%
\pgfusepath{stroke}%
\end{pgfscope}%
\begin{pgfscope}%
\definecolor{textcolor}{rgb}{0.150000,0.150000,0.150000}%
\pgfsetstrokecolor{textcolor}%
\pgfsetfillcolor{textcolor}%
\pgftext[x=7.303445in,y=11.884173in,,top]{\color{textcolor}\rmfamily\fontsize{14.000000}{16.800000}\selectfont 40}%
\end{pgfscope}%
\begin{pgfscope}%
\pgfpathrectangle{\pgfqpoint{2.125000in}{11.981395in}}{\pgfqpoint{5.489583in}{0.877907in}}%
\pgfusepath{clip}%
\pgfsetroundcap%
\pgfsetroundjoin%
\pgfsetlinewidth{0.803000pt}%
\definecolor{currentstroke}{rgb}{1.000000,1.000000,1.000000}%
\pgfsetstrokecolor{currentstroke}%
\pgfsetdash{}{0pt}%
\pgfpathmoveto{\pgfqpoint{2.125000in}{12.257086in}}%
\pgfpathlineto{\pgfqpoint{7.614583in}{12.257086in}}%
\pgfusepath{stroke}%
\end{pgfscope}%
\begin{pgfscope}%
\definecolor{textcolor}{rgb}{0.150000,0.150000,0.150000}%
\pgfsetstrokecolor{textcolor}%
\pgfsetfillcolor{textcolor}%
\pgftext[x=1.904066in,y=12.183220in,left,base]{\color{textcolor}\rmfamily\fontsize{14.000000}{16.800000}\selectfont 0}%
\end{pgfscope}%
\begin{pgfscope}%
\pgfpathrectangle{\pgfqpoint{2.125000in}{11.981395in}}{\pgfqpoint{5.489583in}{0.877907in}}%
\pgfusepath{clip}%
\pgfsetroundcap%
\pgfsetroundjoin%
\pgfsetlinewidth{0.803000pt}%
\definecolor{currentstroke}{rgb}{1.000000,1.000000,1.000000}%
\pgfsetstrokecolor{currentstroke}%
\pgfsetdash{}{0pt}%
\pgfpathmoveto{\pgfqpoint{2.125000in}{12.819397in}}%
\pgfpathlineto{\pgfqpoint{7.614583in}{12.819397in}}%
\pgfusepath{stroke}%
\end{pgfscope}%
\begin{pgfscope}%
\definecolor{textcolor}{rgb}{0.150000,0.150000,0.150000}%
\pgfsetstrokecolor{textcolor}%
\pgfsetfillcolor{textcolor}%
\pgftext[x=1.904066in,y=12.745531in,left,base]{\color{textcolor}\rmfamily\fontsize{14.000000}{16.800000}\selectfont 1}%
\end{pgfscope}%
\begin{pgfscope}%
\pgfpathrectangle{\pgfqpoint{2.125000in}{11.981395in}}{\pgfqpoint{5.489583in}{0.877907in}}%
\pgfusepath{clip}%
\pgfsetbuttcap%
\pgfsetroundjoin%
\definecolor{currentfill}{rgb}{0.121569,0.466667,0.705882}%
\pgfsetfillcolor{currentfill}%
\pgfsetfillopacity{0.250000}%
\pgfsetlinewidth{1.003750pt}%
\definecolor{currentstroke}{rgb}{1.000000,1.000000,1.000000}%
\pgfsetstrokecolor{currentstroke}%
\pgfsetstrokeopacity{0.250000}%
\pgfsetdash{}{0pt}%
\pgfpathmoveto{\pgfqpoint{2.436138in}{12.285457in}}%
\pgfpathlineto{\pgfqpoint{2.436138in}{12.228715in}}%
\pgfpathlineto{\pgfqpoint{2.620972in}{12.208063in}}%
\pgfpathlineto{\pgfqpoint{2.744195in}{12.193916in}}%
\pgfpathlineto{\pgfqpoint{2.867418in}{12.182476in}}%
\pgfpathlineto{\pgfqpoint{2.990641in}{12.172636in}}%
\pgfpathlineto{\pgfqpoint{3.113864in}{12.163888in}}%
\pgfpathlineto{\pgfqpoint{3.237087in}{12.155948in}}%
\pgfpathlineto{\pgfqpoint{3.360310in}{12.148639in}}%
\pgfpathlineto{\pgfqpoint{3.483533in}{12.141840in}}%
\pgfpathlineto{\pgfqpoint{3.606756in}{12.135467in}}%
\pgfpathlineto{\pgfqpoint{3.729979in}{12.129451in}}%
\pgfpathlineto{\pgfqpoint{3.853202in}{12.123743in}}%
\pgfpathlineto{\pgfqpoint{3.976425in}{12.118306in}}%
\pgfpathlineto{\pgfqpoint{4.099648in}{12.113108in}}%
\pgfpathlineto{\pgfqpoint{4.222871in}{12.108123in}}%
\pgfpathlineto{\pgfqpoint{4.346094in}{12.103329in}}%
\pgfpathlineto{\pgfqpoint{4.469317in}{12.098707in}}%
\pgfpathlineto{\pgfqpoint{4.592540in}{12.094240in}}%
\pgfpathlineto{\pgfqpoint{4.715763in}{12.089919in}}%
\pgfpathlineto{\pgfqpoint{4.838986in}{12.085734in}}%
\pgfpathlineto{\pgfqpoint{4.962209in}{12.081678in}}%
\pgfpathlineto{\pgfqpoint{5.085432in}{12.077741in}}%
\pgfpathlineto{\pgfqpoint{5.208655in}{12.073915in}}%
\pgfpathlineto{\pgfqpoint{5.331878in}{12.070194in}}%
\pgfpathlineto{\pgfqpoint{5.455101in}{12.066574in}}%
\pgfpathlineto{\pgfqpoint{5.578324in}{12.063046in}}%
\pgfpathlineto{\pgfqpoint{5.701547in}{12.059608in}}%
\pgfpathlineto{\pgfqpoint{5.824770in}{12.056252in}}%
\pgfpathlineto{\pgfqpoint{5.947993in}{12.052976in}}%
\pgfpathlineto{\pgfqpoint{6.071216in}{12.049774in}}%
\pgfpathlineto{\pgfqpoint{6.194439in}{12.046645in}}%
\pgfpathlineto{\pgfqpoint{6.317662in}{12.043585in}}%
\pgfpathlineto{\pgfqpoint{6.440885in}{12.040592in}}%
\pgfpathlineto{\pgfqpoint{6.564108in}{12.037662in}}%
\pgfpathlineto{\pgfqpoint{6.687330in}{12.034792in}}%
\pgfpathlineto{\pgfqpoint{6.810553in}{12.031981in}}%
\pgfpathlineto{\pgfqpoint{6.933776in}{12.029229in}}%
\pgfpathlineto{\pgfqpoint{7.056999in}{12.026533in}}%
\pgfpathlineto{\pgfqpoint{7.180222in}{12.023891in}}%
\pgfpathlineto{\pgfqpoint{7.365057in}{12.021300in}}%
\pgfpathlineto{\pgfqpoint{7.365057in}{12.492872in}}%
\pgfpathlineto{\pgfqpoint{7.365057in}{12.492872in}}%
\pgfpathlineto{\pgfqpoint{7.180222in}{12.490281in}}%
\pgfpathlineto{\pgfqpoint{7.056999in}{12.487639in}}%
\pgfpathlineto{\pgfqpoint{6.933776in}{12.484943in}}%
\pgfpathlineto{\pgfqpoint{6.810553in}{12.482190in}}%
\pgfpathlineto{\pgfqpoint{6.687330in}{12.479380in}}%
\pgfpathlineto{\pgfqpoint{6.564108in}{12.476510in}}%
\pgfpathlineto{\pgfqpoint{6.440885in}{12.473580in}}%
\pgfpathlineto{\pgfqpoint{6.317662in}{12.470586in}}%
\pgfpathlineto{\pgfqpoint{6.194439in}{12.467527in}}%
\pgfpathlineto{\pgfqpoint{6.071216in}{12.464398in}}%
\pgfpathlineto{\pgfqpoint{5.947993in}{12.461196in}}%
\pgfpathlineto{\pgfqpoint{5.824770in}{12.457919in}}%
\pgfpathlineto{\pgfqpoint{5.701547in}{12.454564in}}%
\pgfpathlineto{\pgfqpoint{5.578324in}{12.451125in}}%
\pgfpathlineto{\pgfqpoint{5.455101in}{12.447598in}}%
\pgfpathlineto{\pgfqpoint{5.331878in}{12.443977in}}%
\pgfpathlineto{\pgfqpoint{5.208655in}{12.440257in}}%
\pgfpathlineto{\pgfqpoint{5.085432in}{12.436431in}}%
\pgfpathlineto{\pgfqpoint{4.962209in}{12.432494in}}%
\pgfpathlineto{\pgfqpoint{4.838986in}{12.428438in}}%
\pgfpathlineto{\pgfqpoint{4.715763in}{12.424253in}}%
\pgfpathlineto{\pgfqpoint{4.592540in}{12.419932in}}%
\pgfpathlineto{\pgfqpoint{4.469317in}{12.415465in}}%
\pgfpathlineto{\pgfqpoint{4.346094in}{12.410843in}}%
\pgfpathlineto{\pgfqpoint{4.222871in}{12.406049in}}%
\pgfpathlineto{\pgfqpoint{4.099648in}{12.401063in}}%
\pgfpathlineto{\pgfqpoint{3.976425in}{12.395865in}}%
\pgfpathlineto{\pgfqpoint{3.853202in}{12.390428in}}%
\pgfpathlineto{\pgfqpoint{3.729979in}{12.384721in}}%
\pgfpathlineto{\pgfqpoint{3.606756in}{12.378705in}}%
\pgfpathlineto{\pgfqpoint{3.483533in}{12.372332in}}%
\pgfpathlineto{\pgfqpoint{3.360310in}{12.365533in}}%
\pgfpathlineto{\pgfqpoint{3.237087in}{12.358224in}}%
\pgfpathlineto{\pgfqpoint{3.113864in}{12.350284in}}%
\pgfpathlineto{\pgfqpoint{2.990641in}{12.341536in}}%
\pgfpathlineto{\pgfqpoint{2.867418in}{12.331696in}}%
\pgfpathlineto{\pgfqpoint{2.744195in}{12.320256in}}%
\pgfpathlineto{\pgfqpoint{2.620972in}{12.306109in}}%
\pgfpathlineto{\pgfqpoint{2.436138in}{12.285457in}}%
\pgfpathclose%
\pgfusepath{stroke,fill}%
\end{pgfscope}%
\begin{pgfscope}%
\pgfpathrectangle{\pgfqpoint{2.125000in}{11.981395in}}{\pgfqpoint{5.489583in}{0.877907in}}%
\pgfusepath{clip}%
\pgfsetbuttcap%
\pgfsetroundjoin%
\pgfsetlinewidth{1.505625pt}%
\definecolor{currentstroke}{rgb}{0.000000,0.000000,0.000000}%
\pgfsetstrokecolor{currentstroke}%
\pgfsetdash{}{0pt}%
\pgfpathmoveto{\pgfqpoint{2.374527in}{12.257086in}}%
\pgfpathlineto{\pgfqpoint{2.374527in}{12.819397in}}%
\pgfusepath{stroke}%
\end{pgfscope}%
\begin{pgfscope}%
\pgfpathrectangle{\pgfqpoint{2.125000in}{11.981395in}}{\pgfqpoint{5.489583in}{0.877907in}}%
\pgfusepath{clip}%
\pgfsetbuttcap%
\pgfsetroundjoin%
\pgfsetlinewidth{1.505625pt}%
\definecolor{currentstroke}{rgb}{0.000000,0.000000,0.000000}%
\pgfsetstrokecolor{currentstroke}%
\pgfsetdash{}{0pt}%
\pgfpathmoveto{\pgfqpoint{2.497749in}{12.257086in}}%
\pgfpathlineto{\pgfqpoint{2.497749in}{12.817378in}}%
\pgfusepath{stroke}%
\end{pgfscope}%
\begin{pgfscope}%
\pgfpathrectangle{\pgfqpoint{2.125000in}{11.981395in}}{\pgfqpoint{5.489583in}{0.877907in}}%
\pgfusepath{clip}%
\pgfsetbuttcap%
\pgfsetroundjoin%
\pgfsetlinewidth{1.505625pt}%
\definecolor{currentstroke}{rgb}{0.000000,0.000000,0.000000}%
\pgfsetstrokecolor{currentstroke}%
\pgfsetdash{}{0pt}%
\pgfpathmoveto{\pgfqpoint{2.620972in}{12.257086in}}%
\pgfpathlineto{\pgfqpoint{2.620972in}{12.815418in}}%
\pgfusepath{stroke}%
\end{pgfscope}%
\begin{pgfscope}%
\pgfpathrectangle{\pgfqpoint{2.125000in}{11.981395in}}{\pgfqpoint{5.489583in}{0.877907in}}%
\pgfusepath{clip}%
\pgfsetbuttcap%
\pgfsetroundjoin%
\pgfsetlinewidth{1.505625pt}%
\definecolor{currentstroke}{rgb}{0.000000,0.000000,0.000000}%
\pgfsetstrokecolor{currentstroke}%
\pgfsetdash{}{0pt}%
\pgfpathmoveto{\pgfqpoint{2.744195in}{12.257086in}}%
\pgfpathlineto{\pgfqpoint{2.744195in}{12.813480in}}%
\pgfusepath{stroke}%
\end{pgfscope}%
\begin{pgfscope}%
\pgfpathrectangle{\pgfqpoint{2.125000in}{11.981395in}}{\pgfqpoint{5.489583in}{0.877907in}}%
\pgfusepath{clip}%
\pgfsetbuttcap%
\pgfsetroundjoin%
\pgfsetlinewidth{1.505625pt}%
\definecolor{currentstroke}{rgb}{0.000000,0.000000,0.000000}%
\pgfsetstrokecolor{currentstroke}%
\pgfsetdash{}{0pt}%
\pgfpathmoveto{\pgfqpoint{2.867418in}{12.257086in}}%
\pgfpathlineto{\pgfqpoint{2.867418in}{12.811532in}}%
\pgfusepath{stroke}%
\end{pgfscope}%
\begin{pgfscope}%
\pgfpathrectangle{\pgfqpoint{2.125000in}{11.981395in}}{\pgfqpoint{5.489583in}{0.877907in}}%
\pgfusepath{clip}%
\pgfsetbuttcap%
\pgfsetroundjoin%
\pgfsetlinewidth{1.505625pt}%
\definecolor{currentstroke}{rgb}{0.000000,0.000000,0.000000}%
\pgfsetstrokecolor{currentstroke}%
\pgfsetdash{}{0pt}%
\pgfpathmoveto{\pgfqpoint{2.990641in}{12.257086in}}%
\pgfpathlineto{\pgfqpoint{2.990641in}{12.809570in}}%
\pgfusepath{stroke}%
\end{pgfscope}%
\begin{pgfscope}%
\pgfpathrectangle{\pgfqpoint{2.125000in}{11.981395in}}{\pgfqpoint{5.489583in}{0.877907in}}%
\pgfusepath{clip}%
\pgfsetbuttcap%
\pgfsetroundjoin%
\pgfsetlinewidth{1.505625pt}%
\definecolor{currentstroke}{rgb}{0.000000,0.000000,0.000000}%
\pgfsetstrokecolor{currentstroke}%
\pgfsetdash{}{0pt}%
\pgfpathmoveto{\pgfqpoint{3.113864in}{12.257086in}}%
\pgfpathlineto{\pgfqpoint{3.113864in}{12.807608in}}%
\pgfusepath{stroke}%
\end{pgfscope}%
\begin{pgfscope}%
\pgfpathrectangle{\pgfqpoint{2.125000in}{11.981395in}}{\pgfqpoint{5.489583in}{0.877907in}}%
\pgfusepath{clip}%
\pgfsetbuttcap%
\pgfsetroundjoin%
\pgfsetlinewidth{1.505625pt}%
\definecolor{currentstroke}{rgb}{0.000000,0.000000,0.000000}%
\pgfsetstrokecolor{currentstroke}%
\pgfsetdash{}{0pt}%
\pgfpathmoveto{\pgfqpoint{3.237087in}{12.257086in}}%
\pgfpathlineto{\pgfqpoint{3.237087in}{12.805615in}}%
\pgfusepath{stroke}%
\end{pgfscope}%
\begin{pgfscope}%
\pgfpathrectangle{\pgfqpoint{2.125000in}{11.981395in}}{\pgfqpoint{5.489583in}{0.877907in}}%
\pgfusepath{clip}%
\pgfsetbuttcap%
\pgfsetroundjoin%
\pgfsetlinewidth{1.505625pt}%
\definecolor{currentstroke}{rgb}{0.000000,0.000000,0.000000}%
\pgfsetstrokecolor{currentstroke}%
\pgfsetdash{}{0pt}%
\pgfpathmoveto{\pgfqpoint{3.360310in}{12.257086in}}%
\pgfpathlineto{\pgfqpoint{3.360310in}{12.803613in}}%
\pgfusepath{stroke}%
\end{pgfscope}%
\begin{pgfscope}%
\pgfpathrectangle{\pgfqpoint{2.125000in}{11.981395in}}{\pgfqpoint{5.489583in}{0.877907in}}%
\pgfusepath{clip}%
\pgfsetbuttcap%
\pgfsetroundjoin%
\pgfsetlinewidth{1.505625pt}%
\definecolor{currentstroke}{rgb}{0.000000,0.000000,0.000000}%
\pgfsetstrokecolor{currentstroke}%
\pgfsetdash{}{0pt}%
\pgfpathmoveto{\pgfqpoint{3.483533in}{12.257086in}}%
\pgfpathlineto{\pgfqpoint{3.483533in}{12.801601in}}%
\pgfusepath{stroke}%
\end{pgfscope}%
\begin{pgfscope}%
\pgfpathrectangle{\pgfqpoint{2.125000in}{11.981395in}}{\pgfqpoint{5.489583in}{0.877907in}}%
\pgfusepath{clip}%
\pgfsetbuttcap%
\pgfsetroundjoin%
\pgfsetlinewidth{1.505625pt}%
\definecolor{currentstroke}{rgb}{0.000000,0.000000,0.000000}%
\pgfsetstrokecolor{currentstroke}%
\pgfsetdash{}{0pt}%
\pgfpathmoveto{\pgfqpoint{3.606756in}{12.257086in}}%
\pgfpathlineto{\pgfqpoint{3.606756in}{12.799768in}}%
\pgfusepath{stroke}%
\end{pgfscope}%
\begin{pgfscope}%
\pgfpathrectangle{\pgfqpoint{2.125000in}{11.981395in}}{\pgfqpoint{5.489583in}{0.877907in}}%
\pgfusepath{clip}%
\pgfsetbuttcap%
\pgfsetroundjoin%
\pgfsetlinewidth{1.505625pt}%
\definecolor{currentstroke}{rgb}{0.000000,0.000000,0.000000}%
\pgfsetstrokecolor{currentstroke}%
\pgfsetdash{}{0pt}%
\pgfpathmoveto{\pgfqpoint{3.729979in}{12.257086in}}%
\pgfpathlineto{\pgfqpoint{3.729979in}{12.797987in}}%
\pgfusepath{stroke}%
\end{pgfscope}%
\begin{pgfscope}%
\pgfpathrectangle{\pgfqpoint{2.125000in}{11.981395in}}{\pgfqpoint{5.489583in}{0.877907in}}%
\pgfusepath{clip}%
\pgfsetbuttcap%
\pgfsetroundjoin%
\pgfsetlinewidth{1.505625pt}%
\definecolor{currentstroke}{rgb}{0.000000,0.000000,0.000000}%
\pgfsetstrokecolor{currentstroke}%
\pgfsetdash{}{0pt}%
\pgfpathmoveto{\pgfqpoint{3.853202in}{12.257086in}}%
\pgfpathlineto{\pgfqpoint{3.853202in}{12.796160in}}%
\pgfusepath{stroke}%
\end{pgfscope}%
\begin{pgfscope}%
\pgfpathrectangle{\pgfqpoint{2.125000in}{11.981395in}}{\pgfqpoint{5.489583in}{0.877907in}}%
\pgfusepath{clip}%
\pgfsetbuttcap%
\pgfsetroundjoin%
\pgfsetlinewidth{1.505625pt}%
\definecolor{currentstroke}{rgb}{0.000000,0.000000,0.000000}%
\pgfsetstrokecolor{currentstroke}%
\pgfsetdash{}{0pt}%
\pgfpathmoveto{\pgfqpoint{3.976425in}{12.257086in}}%
\pgfpathlineto{\pgfqpoint{3.976425in}{12.794368in}}%
\pgfusepath{stroke}%
\end{pgfscope}%
\begin{pgfscope}%
\pgfpathrectangle{\pgfqpoint{2.125000in}{11.981395in}}{\pgfqpoint{5.489583in}{0.877907in}}%
\pgfusepath{clip}%
\pgfsetbuttcap%
\pgfsetroundjoin%
\pgfsetlinewidth{1.505625pt}%
\definecolor{currentstroke}{rgb}{0.000000,0.000000,0.000000}%
\pgfsetstrokecolor{currentstroke}%
\pgfsetdash{}{0pt}%
\pgfpathmoveto{\pgfqpoint{4.099648in}{12.257086in}}%
\pgfpathlineto{\pgfqpoint{4.099648in}{12.792655in}}%
\pgfusepath{stroke}%
\end{pgfscope}%
\begin{pgfscope}%
\pgfpathrectangle{\pgfqpoint{2.125000in}{11.981395in}}{\pgfqpoint{5.489583in}{0.877907in}}%
\pgfusepath{clip}%
\pgfsetbuttcap%
\pgfsetroundjoin%
\pgfsetlinewidth{1.505625pt}%
\definecolor{currentstroke}{rgb}{0.000000,0.000000,0.000000}%
\pgfsetstrokecolor{currentstroke}%
\pgfsetdash{}{0pt}%
\pgfpathmoveto{\pgfqpoint{4.222871in}{12.257086in}}%
\pgfpathlineto{\pgfqpoint{4.222871in}{12.790969in}}%
\pgfusepath{stroke}%
\end{pgfscope}%
\begin{pgfscope}%
\pgfpathrectangle{\pgfqpoint{2.125000in}{11.981395in}}{\pgfqpoint{5.489583in}{0.877907in}}%
\pgfusepath{clip}%
\pgfsetbuttcap%
\pgfsetroundjoin%
\pgfsetlinewidth{1.505625pt}%
\definecolor{currentstroke}{rgb}{0.000000,0.000000,0.000000}%
\pgfsetstrokecolor{currentstroke}%
\pgfsetdash{}{0pt}%
\pgfpathmoveto{\pgfqpoint{4.346094in}{12.257086in}}%
\pgfpathlineto{\pgfqpoint{4.346094in}{12.789432in}}%
\pgfusepath{stroke}%
\end{pgfscope}%
\begin{pgfscope}%
\pgfpathrectangle{\pgfqpoint{2.125000in}{11.981395in}}{\pgfqpoint{5.489583in}{0.877907in}}%
\pgfusepath{clip}%
\pgfsetbuttcap%
\pgfsetroundjoin%
\pgfsetlinewidth{1.505625pt}%
\definecolor{currentstroke}{rgb}{0.000000,0.000000,0.000000}%
\pgfsetstrokecolor{currentstroke}%
\pgfsetdash{}{0pt}%
\pgfpathmoveto{\pgfqpoint{4.469317in}{12.257086in}}%
\pgfpathlineto{\pgfqpoint{4.469317in}{12.787934in}}%
\pgfusepath{stroke}%
\end{pgfscope}%
\begin{pgfscope}%
\pgfpathrectangle{\pgfqpoint{2.125000in}{11.981395in}}{\pgfqpoint{5.489583in}{0.877907in}}%
\pgfusepath{clip}%
\pgfsetbuttcap%
\pgfsetroundjoin%
\pgfsetlinewidth{1.505625pt}%
\definecolor{currentstroke}{rgb}{0.000000,0.000000,0.000000}%
\pgfsetstrokecolor{currentstroke}%
\pgfsetdash{}{0pt}%
\pgfpathmoveto{\pgfqpoint{4.592540in}{12.257086in}}%
\pgfpathlineto{\pgfqpoint{4.592540in}{12.786344in}}%
\pgfusepath{stroke}%
\end{pgfscope}%
\begin{pgfscope}%
\pgfpathrectangle{\pgfqpoint{2.125000in}{11.981395in}}{\pgfqpoint{5.489583in}{0.877907in}}%
\pgfusepath{clip}%
\pgfsetbuttcap%
\pgfsetroundjoin%
\pgfsetlinewidth{1.505625pt}%
\definecolor{currentstroke}{rgb}{0.000000,0.000000,0.000000}%
\pgfsetstrokecolor{currentstroke}%
\pgfsetdash{}{0pt}%
\pgfpathmoveto{\pgfqpoint{4.715763in}{12.257086in}}%
\pgfpathlineto{\pgfqpoint{4.715763in}{12.784548in}}%
\pgfusepath{stroke}%
\end{pgfscope}%
\begin{pgfscope}%
\pgfpathrectangle{\pgfqpoint{2.125000in}{11.981395in}}{\pgfqpoint{5.489583in}{0.877907in}}%
\pgfusepath{clip}%
\pgfsetbuttcap%
\pgfsetroundjoin%
\pgfsetlinewidth{1.505625pt}%
\definecolor{currentstroke}{rgb}{0.000000,0.000000,0.000000}%
\pgfsetstrokecolor{currentstroke}%
\pgfsetdash{}{0pt}%
\pgfpathmoveto{\pgfqpoint{4.838986in}{12.257086in}}%
\pgfpathlineto{\pgfqpoint{4.838986in}{12.782705in}}%
\pgfusepath{stroke}%
\end{pgfscope}%
\begin{pgfscope}%
\pgfpathrectangle{\pgfqpoint{2.125000in}{11.981395in}}{\pgfqpoint{5.489583in}{0.877907in}}%
\pgfusepath{clip}%
\pgfsetbuttcap%
\pgfsetroundjoin%
\pgfsetlinewidth{1.505625pt}%
\definecolor{currentstroke}{rgb}{0.000000,0.000000,0.000000}%
\pgfsetstrokecolor{currentstroke}%
\pgfsetdash{}{0pt}%
\pgfpathmoveto{\pgfqpoint{4.962209in}{12.257086in}}%
\pgfpathlineto{\pgfqpoint{4.962209in}{12.780851in}}%
\pgfusepath{stroke}%
\end{pgfscope}%
\begin{pgfscope}%
\pgfpathrectangle{\pgfqpoint{2.125000in}{11.981395in}}{\pgfqpoint{5.489583in}{0.877907in}}%
\pgfusepath{clip}%
\pgfsetbuttcap%
\pgfsetroundjoin%
\pgfsetlinewidth{1.505625pt}%
\definecolor{currentstroke}{rgb}{0.000000,0.000000,0.000000}%
\pgfsetstrokecolor{currentstroke}%
\pgfsetdash{}{0pt}%
\pgfpathmoveto{\pgfqpoint{5.085432in}{12.257086in}}%
\pgfpathlineto{\pgfqpoint{5.085432in}{12.779013in}}%
\pgfusepath{stroke}%
\end{pgfscope}%
\begin{pgfscope}%
\pgfpathrectangle{\pgfqpoint{2.125000in}{11.981395in}}{\pgfqpoint{5.489583in}{0.877907in}}%
\pgfusepath{clip}%
\pgfsetbuttcap%
\pgfsetroundjoin%
\pgfsetlinewidth{1.505625pt}%
\definecolor{currentstroke}{rgb}{0.000000,0.000000,0.000000}%
\pgfsetstrokecolor{currentstroke}%
\pgfsetdash{}{0pt}%
\pgfpathmoveto{\pgfqpoint{5.208655in}{12.257086in}}%
\pgfpathlineto{\pgfqpoint{5.208655in}{12.777090in}}%
\pgfusepath{stroke}%
\end{pgfscope}%
\begin{pgfscope}%
\pgfpathrectangle{\pgfqpoint{2.125000in}{11.981395in}}{\pgfqpoint{5.489583in}{0.877907in}}%
\pgfusepath{clip}%
\pgfsetbuttcap%
\pgfsetroundjoin%
\pgfsetlinewidth{1.505625pt}%
\definecolor{currentstroke}{rgb}{0.000000,0.000000,0.000000}%
\pgfsetstrokecolor{currentstroke}%
\pgfsetdash{}{0pt}%
\pgfpathmoveto{\pgfqpoint{5.331878in}{12.257086in}}%
\pgfpathlineto{\pgfqpoint{5.331878in}{12.775157in}}%
\pgfusepath{stroke}%
\end{pgfscope}%
\begin{pgfscope}%
\pgfpathrectangle{\pgfqpoint{2.125000in}{11.981395in}}{\pgfqpoint{5.489583in}{0.877907in}}%
\pgfusepath{clip}%
\pgfsetbuttcap%
\pgfsetroundjoin%
\pgfsetlinewidth{1.505625pt}%
\definecolor{currentstroke}{rgb}{0.000000,0.000000,0.000000}%
\pgfsetstrokecolor{currentstroke}%
\pgfsetdash{}{0pt}%
\pgfpathmoveto{\pgfqpoint{5.455101in}{12.257086in}}%
\pgfpathlineto{\pgfqpoint{5.455101in}{12.773228in}}%
\pgfusepath{stroke}%
\end{pgfscope}%
\begin{pgfscope}%
\pgfpathrectangle{\pgfqpoint{2.125000in}{11.981395in}}{\pgfqpoint{5.489583in}{0.877907in}}%
\pgfusepath{clip}%
\pgfsetbuttcap%
\pgfsetroundjoin%
\pgfsetlinewidth{1.505625pt}%
\definecolor{currentstroke}{rgb}{0.000000,0.000000,0.000000}%
\pgfsetstrokecolor{currentstroke}%
\pgfsetdash{}{0pt}%
\pgfpathmoveto{\pgfqpoint{5.578324in}{12.257086in}}%
\pgfpathlineto{\pgfqpoint{5.578324in}{12.771310in}}%
\pgfusepath{stroke}%
\end{pgfscope}%
\begin{pgfscope}%
\pgfpathrectangle{\pgfqpoint{2.125000in}{11.981395in}}{\pgfqpoint{5.489583in}{0.877907in}}%
\pgfusepath{clip}%
\pgfsetbuttcap%
\pgfsetroundjoin%
\pgfsetlinewidth{1.505625pt}%
\definecolor{currentstroke}{rgb}{0.000000,0.000000,0.000000}%
\pgfsetstrokecolor{currentstroke}%
\pgfsetdash{}{0pt}%
\pgfpathmoveto{\pgfqpoint{5.701547in}{12.257086in}}%
\pgfpathlineto{\pgfqpoint{5.701547in}{12.769439in}}%
\pgfusepath{stroke}%
\end{pgfscope}%
\begin{pgfscope}%
\pgfpathrectangle{\pgfqpoint{2.125000in}{11.981395in}}{\pgfqpoint{5.489583in}{0.877907in}}%
\pgfusepath{clip}%
\pgfsetbuttcap%
\pgfsetroundjoin%
\pgfsetlinewidth{1.505625pt}%
\definecolor{currentstroke}{rgb}{0.000000,0.000000,0.000000}%
\pgfsetstrokecolor{currentstroke}%
\pgfsetdash{}{0pt}%
\pgfpathmoveto{\pgfqpoint{5.824770in}{12.257086in}}%
\pgfpathlineto{\pgfqpoint{5.824770in}{12.767589in}}%
\pgfusepath{stroke}%
\end{pgfscope}%
\begin{pgfscope}%
\pgfpathrectangle{\pgfqpoint{2.125000in}{11.981395in}}{\pgfqpoint{5.489583in}{0.877907in}}%
\pgfusepath{clip}%
\pgfsetbuttcap%
\pgfsetroundjoin%
\pgfsetlinewidth{1.505625pt}%
\definecolor{currentstroke}{rgb}{0.000000,0.000000,0.000000}%
\pgfsetstrokecolor{currentstroke}%
\pgfsetdash{}{0pt}%
\pgfpathmoveto{\pgfqpoint{5.947993in}{12.257086in}}%
\pgfpathlineto{\pgfqpoint{5.947993in}{12.765726in}}%
\pgfusepath{stroke}%
\end{pgfscope}%
\begin{pgfscope}%
\pgfpathrectangle{\pgfqpoint{2.125000in}{11.981395in}}{\pgfqpoint{5.489583in}{0.877907in}}%
\pgfusepath{clip}%
\pgfsetbuttcap%
\pgfsetroundjoin%
\pgfsetlinewidth{1.505625pt}%
\definecolor{currentstroke}{rgb}{0.000000,0.000000,0.000000}%
\pgfsetstrokecolor{currentstroke}%
\pgfsetdash{}{0pt}%
\pgfpathmoveto{\pgfqpoint{6.071216in}{12.257086in}}%
\pgfpathlineto{\pgfqpoint{6.071216in}{12.763788in}}%
\pgfusepath{stroke}%
\end{pgfscope}%
\begin{pgfscope}%
\pgfpathrectangle{\pgfqpoint{2.125000in}{11.981395in}}{\pgfqpoint{5.489583in}{0.877907in}}%
\pgfusepath{clip}%
\pgfsetbuttcap%
\pgfsetroundjoin%
\pgfsetlinewidth{1.505625pt}%
\definecolor{currentstroke}{rgb}{0.000000,0.000000,0.000000}%
\pgfsetstrokecolor{currentstroke}%
\pgfsetdash{}{0pt}%
\pgfpathmoveto{\pgfqpoint{6.194439in}{12.257086in}}%
\pgfpathlineto{\pgfqpoint{6.194439in}{12.761813in}}%
\pgfusepath{stroke}%
\end{pgfscope}%
\begin{pgfscope}%
\pgfpathrectangle{\pgfqpoint{2.125000in}{11.981395in}}{\pgfqpoint{5.489583in}{0.877907in}}%
\pgfusepath{clip}%
\pgfsetbuttcap%
\pgfsetroundjoin%
\pgfsetlinewidth{1.505625pt}%
\definecolor{currentstroke}{rgb}{0.000000,0.000000,0.000000}%
\pgfsetstrokecolor{currentstroke}%
\pgfsetdash{}{0pt}%
\pgfpathmoveto{\pgfqpoint{6.317662in}{12.257086in}}%
\pgfpathlineto{\pgfqpoint{6.317662in}{12.759876in}}%
\pgfusepath{stroke}%
\end{pgfscope}%
\begin{pgfscope}%
\pgfpathrectangle{\pgfqpoint{2.125000in}{11.981395in}}{\pgfqpoint{5.489583in}{0.877907in}}%
\pgfusepath{clip}%
\pgfsetbuttcap%
\pgfsetroundjoin%
\pgfsetlinewidth{1.505625pt}%
\definecolor{currentstroke}{rgb}{0.000000,0.000000,0.000000}%
\pgfsetstrokecolor{currentstroke}%
\pgfsetdash{}{0pt}%
\pgfpathmoveto{\pgfqpoint{6.440885in}{12.257086in}}%
\pgfpathlineto{\pgfqpoint{6.440885in}{12.757979in}}%
\pgfusepath{stroke}%
\end{pgfscope}%
\begin{pgfscope}%
\pgfpathrectangle{\pgfqpoint{2.125000in}{11.981395in}}{\pgfqpoint{5.489583in}{0.877907in}}%
\pgfusepath{clip}%
\pgfsetbuttcap%
\pgfsetroundjoin%
\pgfsetlinewidth{1.505625pt}%
\definecolor{currentstroke}{rgb}{0.000000,0.000000,0.000000}%
\pgfsetstrokecolor{currentstroke}%
\pgfsetdash{}{0pt}%
\pgfpathmoveto{\pgfqpoint{6.564108in}{12.257086in}}%
\pgfpathlineto{\pgfqpoint{6.564108in}{12.756069in}}%
\pgfusepath{stroke}%
\end{pgfscope}%
\begin{pgfscope}%
\pgfpathrectangle{\pgfqpoint{2.125000in}{11.981395in}}{\pgfqpoint{5.489583in}{0.877907in}}%
\pgfusepath{clip}%
\pgfsetbuttcap%
\pgfsetroundjoin%
\pgfsetlinewidth{1.505625pt}%
\definecolor{currentstroke}{rgb}{0.000000,0.000000,0.000000}%
\pgfsetstrokecolor{currentstroke}%
\pgfsetdash{}{0pt}%
\pgfpathmoveto{\pgfqpoint{6.687330in}{12.257086in}}%
\pgfpathlineto{\pgfqpoint{6.687330in}{12.754041in}}%
\pgfusepath{stroke}%
\end{pgfscope}%
\begin{pgfscope}%
\pgfpathrectangle{\pgfqpoint{2.125000in}{11.981395in}}{\pgfqpoint{5.489583in}{0.877907in}}%
\pgfusepath{clip}%
\pgfsetbuttcap%
\pgfsetroundjoin%
\pgfsetlinewidth{1.505625pt}%
\definecolor{currentstroke}{rgb}{0.000000,0.000000,0.000000}%
\pgfsetstrokecolor{currentstroke}%
\pgfsetdash{}{0pt}%
\pgfpathmoveto{\pgfqpoint{6.810553in}{12.257086in}}%
\pgfpathlineto{\pgfqpoint{6.810553in}{12.751921in}}%
\pgfusepath{stroke}%
\end{pgfscope}%
\begin{pgfscope}%
\pgfpathrectangle{\pgfqpoint{2.125000in}{11.981395in}}{\pgfqpoint{5.489583in}{0.877907in}}%
\pgfusepath{clip}%
\pgfsetbuttcap%
\pgfsetroundjoin%
\pgfsetlinewidth{1.505625pt}%
\definecolor{currentstroke}{rgb}{0.000000,0.000000,0.000000}%
\pgfsetstrokecolor{currentstroke}%
\pgfsetdash{}{0pt}%
\pgfpathmoveto{\pgfqpoint{6.933776in}{12.257086in}}%
\pgfpathlineto{\pgfqpoint{6.933776in}{12.749796in}}%
\pgfusepath{stroke}%
\end{pgfscope}%
\begin{pgfscope}%
\pgfpathrectangle{\pgfqpoint{2.125000in}{11.981395in}}{\pgfqpoint{5.489583in}{0.877907in}}%
\pgfusepath{clip}%
\pgfsetbuttcap%
\pgfsetroundjoin%
\pgfsetlinewidth{1.505625pt}%
\definecolor{currentstroke}{rgb}{0.000000,0.000000,0.000000}%
\pgfsetstrokecolor{currentstroke}%
\pgfsetdash{}{0pt}%
\pgfpathmoveto{\pgfqpoint{7.056999in}{12.257086in}}%
\pgfpathlineto{\pgfqpoint{7.056999in}{12.747644in}}%
\pgfusepath{stroke}%
\end{pgfscope}%
\begin{pgfscope}%
\pgfpathrectangle{\pgfqpoint{2.125000in}{11.981395in}}{\pgfqpoint{5.489583in}{0.877907in}}%
\pgfusepath{clip}%
\pgfsetbuttcap%
\pgfsetroundjoin%
\pgfsetlinewidth{1.505625pt}%
\definecolor{currentstroke}{rgb}{0.000000,0.000000,0.000000}%
\pgfsetstrokecolor{currentstroke}%
\pgfsetdash{}{0pt}%
\pgfpathmoveto{\pgfqpoint{7.180222in}{12.257086in}}%
\pgfpathlineto{\pgfqpoint{7.180222in}{12.745580in}}%
\pgfusepath{stroke}%
\end{pgfscope}%
\begin{pgfscope}%
\pgfpathrectangle{\pgfqpoint{2.125000in}{11.981395in}}{\pgfqpoint{5.489583in}{0.877907in}}%
\pgfusepath{clip}%
\pgfsetbuttcap%
\pgfsetroundjoin%
\pgfsetlinewidth{1.505625pt}%
\definecolor{currentstroke}{rgb}{0.000000,0.000000,0.000000}%
\pgfsetstrokecolor{currentstroke}%
\pgfsetdash{}{0pt}%
\pgfpathmoveto{\pgfqpoint{7.303445in}{12.257086in}}%
\pgfpathlineto{\pgfqpoint{7.303445in}{12.743478in}}%
\pgfusepath{stroke}%
\end{pgfscope}%
\begin{pgfscope}%
\pgfpathrectangle{\pgfqpoint{2.125000in}{11.981395in}}{\pgfqpoint{5.489583in}{0.877907in}}%
\pgfusepath{clip}%
\pgfsetroundcap%
\pgfsetroundjoin%
\pgfsetlinewidth{1.505625pt}%
\definecolor{currentstroke}{rgb}{0.121569,0.466667,0.705882}%
\pgfsetstrokecolor{currentstroke}%
\pgfsetdash{}{0pt}%
\pgfpathmoveto{\pgfqpoint{2.125000in}{12.257086in}}%
\pgfpathlineto{\pgfqpoint{7.614583in}{12.257086in}}%
\pgfusepath{stroke}%
\end{pgfscope}%
\begin{pgfscope}%
\pgfpathrectangle{\pgfqpoint{2.125000in}{11.981395in}}{\pgfqpoint{5.489583in}{0.877907in}}%
\pgfusepath{clip}%
\pgfsetbuttcap%
\pgfsetroundjoin%
\definecolor{currentfill}{rgb}{0.121569,0.466667,0.705882}%
\pgfsetfillcolor{currentfill}%
\pgfsetlinewidth{1.003750pt}%
\definecolor{currentstroke}{rgb}{0.121569,0.466667,0.705882}%
\pgfsetstrokecolor{currentstroke}%
\pgfsetdash{}{0pt}%
\pgfsys@defobject{currentmarker}{\pgfqpoint{-0.034722in}{-0.034722in}}{\pgfqpoint{0.034722in}{0.034722in}}{%
\pgfpathmoveto{\pgfqpoint{0.000000in}{-0.034722in}}%
\pgfpathcurveto{\pgfqpoint{0.009208in}{-0.034722in}}{\pgfqpoint{0.018041in}{-0.031064in}}{\pgfqpoint{0.024552in}{-0.024552in}}%
\pgfpathcurveto{\pgfqpoint{0.031064in}{-0.018041in}}{\pgfqpoint{0.034722in}{-0.009208in}}{\pgfqpoint{0.034722in}{0.000000in}}%
\pgfpathcurveto{\pgfqpoint{0.034722in}{0.009208in}}{\pgfqpoint{0.031064in}{0.018041in}}{\pgfqpoint{0.024552in}{0.024552in}}%
\pgfpathcurveto{\pgfqpoint{0.018041in}{0.031064in}}{\pgfqpoint{0.009208in}{0.034722in}}{\pgfqpoint{0.000000in}{0.034722in}}%
\pgfpathcurveto{\pgfqpoint{-0.009208in}{0.034722in}}{\pgfqpoint{-0.018041in}{0.031064in}}{\pgfqpoint{-0.024552in}{0.024552in}}%
\pgfpathcurveto{\pgfqpoint{-0.031064in}{0.018041in}}{\pgfqpoint{-0.034722in}{0.009208in}}{\pgfqpoint{-0.034722in}{0.000000in}}%
\pgfpathcurveto{\pgfqpoint{-0.034722in}{-0.009208in}}{\pgfqpoint{-0.031064in}{-0.018041in}}{\pgfqpoint{-0.024552in}{-0.024552in}}%
\pgfpathcurveto{\pgfqpoint{-0.018041in}{-0.031064in}}{\pgfqpoint{-0.009208in}{-0.034722in}}{\pgfqpoint{0.000000in}{-0.034722in}}%
\pgfpathclose%
\pgfusepath{stroke,fill}%
}%
\begin{pgfscope}%
\pgfsys@transformshift{2.374527in}{12.819397in}%
\pgfsys@useobject{currentmarker}{}%
\end{pgfscope}%
\begin{pgfscope}%
\pgfsys@transformshift{2.497749in}{12.817378in}%
\pgfsys@useobject{currentmarker}{}%
\end{pgfscope}%
\begin{pgfscope}%
\pgfsys@transformshift{2.620972in}{12.815418in}%
\pgfsys@useobject{currentmarker}{}%
\end{pgfscope}%
\begin{pgfscope}%
\pgfsys@transformshift{2.744195in}{12.813480in}%
\pgfsys@useobject{currentmarker}{}%
\end{pgfscope}%
\begin{pgfscope}%
\pgfsys@transformshift{2.867418in}{12.811532in}%
\pgfsys@useobject{currentmarker}{}%
\end{pgfscope}%
\begin{pgfscope}%
\pgfsys@transformshift{2.990641in}{12.809570in}%
\pgfsys@useobject{currentmarker}{}%
\end{pgfscope}%
\begin{pgfscope}%
\pgfsys@transformshift{3.113864in}{12.807608in}%
\pgfsys@useobject{currentmarker}{}%
\end{pgfscope}%
\begin{pgfscope}%
\pgfsys@transformshift{3.237087in}{12.805615in}%
\pgfsys@useobject{currentmarker}{}%
\end{pgfscope}%
\begin{pgfscope}%
\pgfsys@transformshift{3.360310in}{12.803613in}%
\pgfsys@useobject{currentmarker}{}%
\end{pgfscope}%
\begin{pgfscope}%
\pgfsys@transformshift{3.483533in}{12.801601in}%
\pgfsys@useobject{currentmarker}{}%
\end{pgfscope}%
\begin{pgfscope}%
\pgfsys@transformshift{3.606756in}{12.799768in}%
\pgfsys@useobject{currentmarker}{}%
\end{pgfscope}%
\begin{pgfscope}%
\pgfsys@transformshift{3.729979in}{12.797987in}%
\pgfsys@useobject{currentmarker}{}%
\end{pgfscope}%
\begin{pgfscope}%
\pgfsys@transformshift{3.853202in}{12.796160in}%
\pgfsys@useobject{currentmarker}{}%
\end{pgfscope}%
\begin{pgfscope}%
\pgfsys@transformshift{3.976425in}{12.794368in}%
\pgfsys@useobject{currentmarker}{}%
\end{pgfscope}%
\begin{pgfscope}%
\pgfsys@transformshift{4.099648in}{12.792655in}%
\pgfsys@useobject{currentmarker}{}%
\end{pgfscope}%
\begin{pgfscope}%
\pgfsys@transformshift{4.222871in}{12.790969in}%
\pgfsys@useobject{currentmarker}{}%
\end{pgfscope}%
\begin{pgfscope}%
\pgfsys@transformshift{4.346094in}{12.789432in}%
\pgfsys@useobject{currentmarker}{}%
\end{pgfscope}%
\begin{pgfscope}%
\pgfsys@transformshift{4.469317in}{12.787934in}%
\pgfsys@useobject{currentmarker}{}%
\end{pgfscope}%
\begin{pgfscope}%
\pgfsys@transformshift{4.592540in}{12.786344in}%
\pgfsys@useobject{currentmarker}{}%
\end{pgfscope}%
\begin{pgfscope}%
\pgfsys@transformshift{4.715763in}{12.784548in}%
\pgfsys@useobject{currentmarker}{}%
\end{pgfscope}%
\begin{pgfscope}%
\pgfsys@transformshift{4.838986in}{12.782705in}%
\pgfsys@useobject{currentmarker}{}%
\end{pgfscope}%
\begin{pgfscope}%
\pgfsys@transformshift{4.962209in}{12.780851in}%
\pgfsys@useobject{currentmarker}{}%
\end{pgfscope}%
\begin{pgfscope}%
\pgfsys@transformshift{5.085432in}{12.779013in}%
\pgfsys@useobject{currentmarker}{}%
\end{pgfscope}%
\begin{pgfscope}%
\pgfsys@transformshift{5.208655in}{12.777090in}%
\pgfsys@useobject{currentmarker}{}%
\end{pgfscope}%
\begin{pgfscope}%
\pgfsys@transformshift{5.331878in}{12.775157in}%
\pgfsys@useobject{currentmarker}{}%
\end{pgfscope}%
\begin{pgfscope}%
\pgfsys@transformshift{5.455101in}{12.773228in}%
\pgfsys@useobject{currentmarker}{}%
\end{pgfscope}%
\begin{pgfscope}%
\pgfsys@transformshift{5.578324in}{12.771310in}%
\pgfsys@useobject{currentmarker}{}%
\end{pgfscope}%
\begin{pgfscope}%
\pgfsys@transformshift{5.701547in}{12.769439in}%
\pgfsys@useobject{currentmarker}{}%
\end{pgfscope}%
\begin{pgfscope}%
\pgfsys@transformshift{5.824770in}{12.767589in}%
\pgfsys@useobject{currentmarker}{}%
\end{pgfscope}%
\begin{pgfscope}%
\pgfsys@transformshift{5.947993in}{12.765726in}%
\pgfsys@useobject{currentmarker}{}%
\end{pgfscope}%
\begin{pgfscope}%
\pgfsys@transformshift{6.071216in}{12.763788in}%
\pgfsys@useobject{currentmarker}{}%
\end{pgfscope}%
\begin{pgfscope}%
\pgfsys@transformshift{6.194439in}{12.761813in}%
\pgfsys@useobject{currentmarker}{}%
\end{pgfscope}%
\begin{pgfscope}%
\pgfsys@transformshift{6.317662in}{12.759876in}%
\pgfsys@useobject{currentmarker}{}%
\end{pgfscope}%
\begin{pgfscope}%
\pgfsys@transformshift{6.440885in}{12.757979in}%
\pgfsys@useobject{currentmarker}{}%
\end{pgfscope}%
\begin{pgfscope}%
\pgfsys@transformshift{6.564108in}{12.756069in}%
\pgfsys@useobject{currentmarker}{}%
\end{pgfscope}%
\begin{pgfscope}%
\pgfsys@transformshift{6.687330in}{12.754041in}%
\pgfsys@useobject{currentmarker}{}%
\end{pgfscope}%
\begin{pgfscope}%
\pgfsys@transformshift{6.810553in}{12.751921in}%
\pgfsys@useobject{currentmarker}{}%
\end{pgfscope}%
\begin{pgfscope}%
\pgfsys@transformshift{6.933776in}{12.749796in}%
\pgfsys@useobject{currentmarker}{}%
\end{pgfscope}%
\begin{pgfscope}%
\pgfsys@transformshift{7.056999in}{12.747644in}%
\pgfsys@useobject{currentmarker}{}%
\end{pgfscope}%
\begin{pgfscope}%
\pgfsys@transformshift{7.180222in}{12.745580in}%
\pgfsys@useobject{currentmarker}{}%
\end{pgfscope}%
\begin{pgfscope}%
\pgfsys@transformshift{7.303445in}{12.743478in}%
\pgfsys@useobject{currentmarker}{}%
\end{pgfscope}%
\end{pgfscope}%
\begin{pgfscope}%
\pgfsetrectcap%
\pgfsetmiterjoin%
\pgfsetlinewidth{0.803000pt}%
\definecolor{currentstroke}{rgb}{1.000000,1.000000,1.000000}%
\pgfsetstrokecolor{currentstroke}%
\pgfsetdash{}{0pt}%
\pgfpathmoveto{\pgfqpoint{2.125000in}{11.981395in}}%
\pgfpathlineto{\pgfqpoint{2.125000in}{12.859302in}}%
\pgfusepath{stroke}%
\end{pgfscope}%
\begin{pgfscope}%
\pgfsetrectcap%
\pgfsetmiterjoin%
\pgfsetlinewidth{0.803000pt}%
\definecolor{currentstroke}{rgb}{1.000000,1.000000,1.000000}%
\pgfsetstrokecolor{currentstroke}%
\pgfsetdash{}{0pt}%
\pgfpathmoveto{\pgfqpoint{7.614583in}{11.981395in}}%
\pgfpathlineto{\pgfqpoint{7.614583in}{12.859302in}}%
\pgfusepath{stroke}%
\end{pgfscope}%
\begin{pgfscope}%
\pgfsetrectcap%
\pgfsetmiterjoin%
\pgfsetlinewidth{0.803000pt}%
\definecolor{currentstroke}{rgb}{1.000000,1.000000,1.000000}%
\pgfsetstrokecolor{currentstroke}%
\pgfsetdash{}{0pt}%
\pgfpathmoveto{\pgfqpoint{2.125000in}{11.981395in}}%
\pgfpathlineto{\pgfqpoint{7.614583in}{11.981395in}}%
\pgfusepath{stroke}%
\end{pgfscope}%
\begin{pgfscope}%
\pgfsetrectcap%
\pgfsetmiterjoin%
\pgfsetlinewidth{0.803000pt}%
\definecolor{currentstroke}{rgb}{1.000000,1.000000,1.000000}%
\pgfsetstrokecolor{currentstroke}%
\pgfsetdash{}{0pt}%
\pgfpathmoveto{\pgfqpoint{2.125000in}{12.859302in}}%
\pgfpathlineto{\pgfqpoint{7.614583in}{12.859302in}}%
\pgfusepath{stroke}%
\end{pgfscope}%
\begin{pgfscope}%
\definecolor{textcolor}{rgb}{0.150000,0.150000,0.150000}%
\pgfsetstrokecolor{textcolor}%
\pgfsetfillcolor{textcolor}%
\pgftext[x=4.869792in,y=12.942636in,,base]{\color{textcolor}\rmfamily\fontsize{16.800000}{20.160000}\selectfont Autocorrelation}%
\end{pgfscope}%
\begin{pgfscope}%
\pgfsetbuttcap%
\pgfsetmiterjoin%
\definecolor{currentfill}{rgb}{0.917647,0.917647,0.949020}%
\pgfsetfillcolor{currentfill}%
\pgfsetlinewidth{0.000000pt}%
\definecolor{currentstroke}{rgb}{0.000000,0.000000,0.000000}%
\pgfsetstrokecolor{currentstroke}%
\pgfsetstrokeopacity{0.000000}%
\pgfsetdash{}{0pt}%
\pgfpathmoveto{\pgfqpoint{9.810417in}{11.981395in}}%
\pgfpathlineto{\pgfqpoint{15.300000in}{11.981395in}}%
\pgfpathlineto{\pgfqpoint{15.300000in}{12.859302in}}%
\pgfpathlineto{\pgfqpoint{9.810417in}{12.859302in}}%
\pgfpathclose%
\pgfusepath{fill}%
\end{pgfscope}%
\begin{pgfscope}%
\pgfpathrectangle{\pgfqpoint{9.810417in}{11.981395in}}{\pgfqpoint{5.489583in}{0.877907in}}%
\pgfusepath{clip}%
\pgfsetroundcap%
\pgfsetroundjoin%
\pgfsetlinewidth{0.803000pt}%
\definecolor{currentstroke}{rgb}{1.000000,1.000000,1.000000}%
\pgfsetstrokecolor{currentstroke}%
\pgfsetdash{}{0pt}%
\pgfpathmoveto{\pgfqpoint{10.059943in}{11.981395in}}%
\pgfpathlineto{\pgfqpoint{10.059943in}{12.859302in}}%
\pgfusepath{stroke}%
\end{pgfscope}%
\begin{pgfscope}%
\definecolor{textcolor}{rgb}{0.150000,0.150000,0.150000}%
\pgfsetstrokecolor{textcolor}%
\pgfsetfillcolor{textcolor}%
\pgftext[x=10.059943in,y=11.884173in,,top]{\color{textcolor}\rmfamily\fontsize{14.000000}{16.800000}\selectfont 0}%
\end{pgfscope}%
\begin{pgfscope}%
\pgfpathrectangle{\pgfqpoint{9.810417in}{11.981395in}}{\pgfqpoint{5.489583in}{0.877907in}}%
\pgfusepath{clip}%
\pgfsetroundcap%
\pgfsetroundjoin%
\pgfsetlinewidth{0.803000pt}%
\definecolor{currentstroke}{rgb}{1.000000,1.000000,1.000000}%
\pgfsetstrokecolor{currentstroke}%
\pgfsetdash{}{0pt}%
\pgfpathmoveto{\pgfqpoint{10.676058in}{11.981395in}}%
\pgfpathlineto{\pgfqpoint{10.676058in}{12.859302in}}%
\pgfusepath{stroke}%
\end{pgfscope}%
\begin{pgfscope}%
\definecolor{textcolor}{rgb}{0.150000,0.150000,0.150000}%
\pgfsetstrokecolor{textcolor}%
\pgfsetfillcolor{textcolor}%
\pgftext[x=10.676058in,y=11.884173in,,top]{\color{textcolor}\rmfamily\fontsize{14.000000}{16.800000}\selectfont 5}%
\end{pgfscope}%
\begin{pgfscope}%
\pgfpathrectangle{\pgfqpoint{9.810417in}{11.981395in}}{\pgfqpoint{5.489583in}{0.877907in}}%
\pgfusepath{clip}%
\pgfsetroundcap%
\pgfsetroundjoin%
\pgfsetlinewidth{0.803000pt}%
\definecolor{currentstroke}{rgb}{1.000000,1.000000,1.000000}%
\pgfsetstrokecolor{currentstroke}%
\pgfsetdash{}{0pt}%
\pgfpathmoveto{\pgfqpoint{11.292173in}{11.981395in}}%
\pgfpathlineto{\pgfqpoint{11.292173in}{12.859302in}}%
\pgfusepath{stroke}%
\end{pgfscope}%
\begin{pgfscope}%
\definecolor{textcolor}{rgb}{0.150000,0.150000,0.150000}%
\pgfsetstrokecolor{textcolor}%
\pgfsetfillcolor{textcolor}%
\pgftext[x=11.292173in,y=11.884173in,,top]{\color{textcolor}\rmfamily\fontsize{14.000000}{16.800000}\selectfont 10}%
\end{pgfscope}%
\begin{pgfscope}%
\pgfpathrectangle{\pgfqpoint{9.810417in}{11.981395in}}{\pgfqpoint{5.489583in}{0.877907in}}%
\pgfusepath{clip}%
\pgfsetroundcap%
\pgfsetroundjoin%
\pgfsetlinewidth{0.803000pt}%
\definecolor{currentstroke}{rgb}{1.000000,1.000000,1.000000}%
\pgfsetstrokecolor{currentstroke}%
\pgfsetdash{}{0pt}%
\pgfpathmoveto{\pgfqpoint{11.908288in}{11.981395in}}%
\pgfpathlineto{\pgfqpoint{11.908288in}{12.859302in}}%
\pgfusepath{stroke}%
\end{pgfscope}%
\begin{pgfscope}%
\definecolor{textcolor}{rgb}{0.150000,0.150000,0.150000}%
\pgfsetstrokecolor{textcolor}%
\pgfsetfillcolor{textcolor}%
\pgftext[x=11.908288in,y=11.884173in,,top]{\color{textcolor}\rmfamily\fontsize{14.000000}{16.800000}\selectfont 15}%
\end{pgfscope}%
\begin{pgfscope}%
\pgfpathrectangle{\pgfqpoint{9.810417in}{11.981395in}}{\pgfqpoint{5.489583in}{0.877907in}}%
\pgfusepath{clip}%
\pgfsetroundcap%
\pgfsetroundjoin%
\pgfsetlinewidth{0.803000pt}%
\definecolor{currentstroke}{rgb}{1.000000,1.000000,1.000000}%
\pgfsetstrokecolor{currentstroke}%
\pgfsetdash{}{0pt}%
\pgfpathmoveto{\pgfqpoint{12.524403in}{11.981395in}}%
\pgfpathlineto{\pgfqpoint{12.524403in}{12.859302in}}%
\pgfusepath{stroke}%
\end{pgfscope}%
\begin{pgfscope}%
\definecolor{textcolor}{rgb}{0.150000,0.150000,0.150000}%
\pgfsetstrokecolor{textcolor}%
\pgfsetfillcolor{textcolor}%
\pgftext[x=12.524403in,y=11.884173in,,top]{\color{textcolor}\rmfamily\fontsize{14.000000}{16.800000}\selectfont 20}%
\end{pgfscope}%
\begin{pgfscope}%
\pgfpathrectangle{\pgfqpoint{9.810417in}{11.981395in}}{\pgfqpoint{5.489583in}{0.877907in}}%
\pgfusepath{clip}%
\pgfsetroundcap%
\pgfsetroundjoin%
\pgfsetlinewidth{0.803000pt}%
\definecolor{currentstroke}{rgb}{1.000000,1.000000,1.000000}%
\pgfsetstrokecolor{currentstroke}%
\pgfsetdash{}{0pt}%
\pgfpathmoveto{\pgfqpoint{13.140517in}{11.981395in}}%
\pgfpathlineto{\pgfqpoint{13.140517in}{12.859302in}}%
\pgfusepath{stroke}%
\end{pgfscope}%
\begin{pgfscope}%
\definecolor{textcolor}{rgb}{0.150000,0.150000,0.150000}%
\pgfsetstrokecolor{textcolor}%
\pgfsetfillcolor{textcolor}%
\pgftext[x=13.140517in,y=11.884173in,,top]{\color{textcolor}\rmfamily\fontsize{14.000000}{16.800000}\selectfont 25}%
\end{pgfscope}%
\begin{pgfscope}%
\pgfpathrectangle{\pgfqpoint{9.810417in}{11.981395in}}{\pgfqpoint{5.489583in}{0.877907in}}%
\pgfusepath{clip}%
\pgfsetroundcap%
\pgfsetroundjoin%
\pgfsetlinewidth{0.803000pt}%
\definecolor{currentstroke}{rgb}{1.000000,1.000000,1.000000}%
\pgfsetstrokecolor{currentstroke}%
\pgfsetdash{}{0pt}%
\pgfpathmoveto{\pgfqpoint{13.756632in}{11.981395in}}%
\pgfpathlineto{\pgfqpoint{13.756632in}{12.859302in}}%
\pgfusepath{stroke}%
\end{pgfscope}%
\begin{pgfscope}%
\definecolor{textcolor}{rgb}{0.150000,0.150000,0.150000}%
\pgfsetstrokecolor{textcolor}%
\pgfsetfillcolor{textcolor}%
\pgftext[x=13.756632in,y=11.884173in,,top]{\color{textcolor}\rmfamily\fontsize{14.000000}{16.800000}\selectfont 30}%
\end{pgfscope}%
\begin{pgfscope}%
\pgfpathrectangle{\pgfqpoint{9.810417in}{11.981395in}}{\pgfqpoint{5.489583in}{0.877907in}}%
\pgfusepath{clip}%
\pgfsetroundcap%
\pgfsetroundjoin%
\pgfsetlinewidth{0.803000pt}%
\definecolor{currentstroke}{rgb}{1.000000,1.000000,1.000000}%
\pgfsetstrokecolor{currentstroke}%
\pgfsetdash{}{0pt}%
\pgfpathmoveto{\pgfqpoint{14.372747in}{11.981395in}}%
\pgfpathlineto{\pgfqpoint{14.372747in}{12.859302in}}%
\pgfusepath{stroke}%
\end{pgfscope}%
\begin{pgfscope}%
\definecolor{textcolor}{rgb}{0.150000,0.150000,0.150000}%
\pgfsetstrokecolor{textcolor}%
\pgfsetfillcolor{textcolor}%
\pgftext[x=14.372747in,y=11.884173in,,top]{\color{textcolor}\rmfamily\fontsize{14.000000}{16.800000}\selectfont 35}%
\end{pgfscope}%
\begin{pgfscope}%
\pgfpathrectangle{\pgfqpoint{9.810417in}{11.981395in}}{\pgfqpoint{5.489583in}{0.877907in}}%
\pgfusepath{clip}%
\pgfsetroundcap%
\pgfsetroundjoin%
\pgfsetlinewidth{0.803000pt}%
\definecolor{currentstroke}{rgb}{1.000000,1.000000,1.000000}%
\pgfsetstrokecolor{currentstroke}%
\pgfsetdash{}{0pt}%
\pgfpathmoveto{\pgfqpoint{14.988862in}{11.981395in}}%
\pgfpathlineto{\pgfqpoint{14.988862in}{12.859302in}}%
\pgfusepath{stroke}%
\end{pgfscope}%
\begin{pgfscope}%
\definecolor{textcolor}{rgb}{0.150000,0.150000,0.150000}%
\pgfsetstrokecolor{textcolor}%
\pgfsetfillcolor{textcolor}%
\pgftext[x=14.988862in,y=11.884173in,,top]{\color{textcolor}\rmfamily\fontsize{14.000000}{16.800000}\selectfont 40}%
\end{pgfscope}%
\begin{pgfscope}%
\pgfpathrectangle{\pgfqpoint{9.810417in}{11.981395in}}{\pgfqpoint{5.489583in}{0.877907in}}%
\pgfusepath{clip}%
\pgfsetroundcap%
\pgfsetroundjoin%
\pgfsetlinewidth{0.803000pt}%
\definecolor{currentstroke}{rgb}{1.000000,1.000000,1.000000}%
\pgfsetstrokecolor{currentstroke}%
\pgfsetdash{}{0pt}%
\pgfpathmoveto{\pgfqpoint{9.810417in}{12.069362in}}%
\pgfpathlineto{\pgfqpoint{15.300000in}{12.069362in}}%
\pgfusepath{stroke}%
\end{pgfscope}%
\begin{pgfscope}%
\definecolor{textcolor}{rgb}{0.150000,0.150000,0.150000}%
\pgfsetstrokecolor{textcolor}%
\pgfsetfillcolor{textcolor}%
\pgftext[x=9.589483in,y=11.995495in,left,base]{\color{textcolor}\rmfamily\fontsize{14.000000}{16.800000}\selectfont 0}%
\end{pgfscope}%
\begin{pgfscope}%
\pgfpathrectangle{\pgfqpoint{9.810417in}{11.981395in}}{\pgfqpoint{5.489583in}{0.877907in}}%
\pgfusepath{clip}%
\pgfsetroundcap%
\pgfsetroundjoin%
\pgfsetlinewidth{0.803000pt}%
\definecolor{currentstroke}{rgb}{1.000000,1.000000,1.000000}%
\pgfsetstrokecolor{currentstroke}%
\pgfsetdash{}{0pt}%
\pgfpathmoveto{\pgfqpoint{9.810417in}{12.819397in}}%
\pgfpathlineto{\pgfqpoint{15.300000in}{12.819397in}}%
\pgfusepath{stroke}%
\end{pgfscope}%
\begin{pgfscope}%
\definecolor{textcolor}{rgb}{0.150000,0.150000,0.150000}%
\pgfsetstrokecolor{textcolor}%
\pgfsetfillcolor{textcolor}%
\pgftext[x=9.589483in,y=12.745531in,left,base]{\color{textcolor}\rmfamily\fontsize{14.000000}{16.800000}\selectfont 1}%
\end{pgfscope}%
\begin{pgfscope}%
\pgfpathrectangle{\pgfqpoint{9.810417in}{11.981395in}}{\pgfqpoint{5.489583in}{0.877907in}}%
\pgfusepath{clip}%
\pgfsetbuttcap%
\pgfsetroundjoin%
\definecolor{currentfill}{rgb}{0.121569,0.466667,0.705882}%
\pgfsetfillcolor{currentfill}%
\pgfsetfillopacity{0.250000}%
\pgfsetlinewidth{1.003750pt}%
\definecolor{currentstroke}{rgb}{1.000000,1.000000,1.000000}%
\pgfsetstrokecolor{currentstroke}%
\pgfsetstrokeopacity{0.250000}%
\pgfsetdash{}{0pt}%
\pgfpathmoveto{\pgfqpoint{10.121555in}{12.107205in}}%
\pgfpathlineto{\pgfqpoint{10.121555in}{12.031519in}}%
\pgfpathlineto{\pgfqpoint{10.306389in}{12.031519in}}%
\pgfpathlineto{\pgfqpoint{10.429612in}{12.031519in}}%
\pgfpathlineto{\pgfqpoint{10.552835in}{12.031519in}}%
\pgfpathlineto{\pgfqpoint{10.676058in}{12.031519in}}%
\pgfpathlineto{\pgfqpoint{10.799281in}{12.031519in}}%
\pgfpathlineto{\pgfqpoint{10.922504in}{12.031519in}}%
\pgfpathlineto{\pgfqpoint{11.045727in}{12.031519in}}%
\pgfpathlineto{\pgfqpoint{11.168950in}{12.031519in}}%
\pgfpathlineto{\pgfqpoint{11.292173in}{12.031519in}}%
\pgfpathlineto{\pgfqpoint{11.415396in}{12.031519in}}%
\pgfpathlineto{\pgfqpoint{11.538619in}{12.031519in}}%
\pgfpathlineto{\pgfqpoint{11.661842in}{12.031519in}}%
\pgfpathlineto{\pgfqpoint{11.785065in}{12.031519in}}%
\pgfpathlineto{\pgfqpoint{11.908288in}{12.031519in}}%
\pgfpathlineto{\pgfqpoint{12.031511in}{12.031519in}}%
\pgfpathlineto{\pgfqpoint{12.154734in}{12.031519in}}%
\pgfpathlineto{\pgfqpoint{12.277957in}{12.031519in}}%
\pgfpathlineto{\pgfqpoint{12.401180in}{12.031519in}}%
\pgfpathlineto{\pgfqpoint{12.524403in}{12.031519in}}%
\pgfpathlineto{\pgfqpoint{12.647626in}{12.031519in}}%
\pgfpathlineto{\pgfqpoint{12.770849in}{12.031519in}}%
\pgfpathlineto{\pgfqpoint{12.894072in}{12.031519in}}%
\pgfpathlineto{\pgfqpoint{13.017294in}{12.031519in}}%
\pgfpathlineto{\pgfqpoint{13.140517in}{12.031519in}}%
\pgfpathlineto{\pgfqpoint{13.263740in}{12.031519in}}%
\pgfpathlineto{\pgfqpoint{13.386963in}{12.031519in}}%
\pgfpathlineto{\pgfqpoint{13.510186in}{12.031519in}}%
\pgfpathlineto{\pgfqpoint{13.633409in}{12.031519in}}%
\pgfpathlineto{\pgfqpoint{13.756632in}{12.031519in}}%
\pgfpathlineto{\pgfqpoint{13.879855in}{12.031519in}}%
\pgfpathlineto{\pgfqpoint{14.003078in}{12.031519in}}%
\pgfpathlineto{\pgfqpoint{14.126301in}{12.031519in}}%
\pgfpathlineto{\pgfqpoint{14.249524in}{12.031519in}}%
\pgfpathlineto{\pgfqpoint{14.372747in}{12.031519in}}%
\pgfpathlineto{\pgfqpoint{14.495970in}{12.031519in}}%
\pgfpathlineto{\pgfqpoint{14.619193in}{12.031519in}}%
\pgfpathlineto{\pgfqpoint{14.742416in}{12.031519in}}%
\pgfpathlineto{\pgfqpoint{14.865639in}{12.031519in}}%
\pgfpathlineto{\pgfqpoint{15.050473in}{12.031519in}}%
\pgfpathlineto{\pgfqpoint{15.050473in}{12.107205in}}%
\pgfpathlineto{\pgfqpoint{15.050473in}{12.107205in}}%
\pgfpathlineto{\pgfqpoint{14.865639in}{12.107205in}}%
\pgfpathlineto{\pgfqpoint{14.742416in}{12.107205in}}%
\pgfpathlineto{\pgfqpoint{14.619193in}{12.107205in}}%
\pgfpathlineto{\pgfqpoint{14.495970in}{12.107205in}}%
\pgfpathlineto{\pgfqpoint{14.372747in}{12.107205in}}%
\pgfpathlineto{\pgfqpoint{14.249524in}{12.107205in}}%
\pgfpathlineto{\pgfqpoint{14.126301in}{12.107205in}}%
\pgfpathlineto{\pgfqpoint{14.003078in}{12.107205in}}%
\pgfpathlineto{\pgfqpoint{13.879855in}{12.107205in}}%
\pgfpathlineto{\pgfqpoint{13.756632in}{12.107205in}}%
\pgfpathlineto{\pgfqpoint{13.633409in}{12.107205in}}%
\pgfpathlineto{\pgfqpoint{13.510186in}{12.107205in}}%
\pgfpathlineto{\pgfqpoint{13.386963in}{12.107205in}}%
\pgfpathlineto{\pgfqpoint{13.263740in}{12.107205in}}%
\pgfpathlineto{\pgfqpoint{13.140517in}{12.107205in}}%
\pgfpathlineto{\pgfqpoint{13.017294in}{12.107205in}}%
\pgfpathlineto{\pgfqpoint{12.894072in}{12.107205in}}%
\pgfpathlineto{\pgfqpoint{12.770849in}{12.107205in}}%
\pgfpathlineto{\pgfqpoint{12.647626in}{12.107205in}}%
\pgfpathlineto{\pgfqpoint{12.524403in}{12.107205in}}%
\pgfpathlineto{\pgfqpoint{12.401180in}{12.107205in}}%
\pgfpathlineto{\pgfqpoint{12.277957in}{12.107205in}}%
\pgfpathlineto{\pgfqpoint{12.154734in}{12.107205in}}%
\pgfpathlineto{\pgfqpoint{12.031511in}{12.107205in}}%
\pgfpathlineto{\pgfqpoint{11.908288in}{12.107205in}}%
\pgfpathlineto{\pgfqpoint{11.785065in}{12.107205in}}%
\pgfpathlineto{\pgfqpoint{11.661842in}{12.107205in}}%
\pgfpathlineto{\pgfqpoint{11.538619in}{12.107205in}}%
\pgfpathlineto{\pgfqpoint{11.415396in}{12.107205in}}%
\pgfpathlineto{\pgfqpoint{11.292173in}{12.107205in}}%
\pgfpathlineto{\pgfqpoint{11.168950in}{12.107205in}}%
\pgfpathlineto{\pgfqpoint{11.045727in}{12.107205in}}%
\pgfpathlineto{\pgfqpoint{10.922504in}{12.107205in}}%
\pgfpathlineto{\pgfqpoint{10.799281in}{12.107205in}}%
\pgfpathlineto{\pgfqpoint{10.676058in}{12.107205in}}%
\pgfpathlineto{\pgfqpoint{10.552835in}{12.107205in}}%
\pgfpathlineto{\pgfqpoint{10.429612in}{12.107205in}}%
\pgfpathlineto{\pgfqpoint{10.306389in}{12.107205in}}%
\pgfpathlineto{\pgfqpoint{10.121555in}{12.107205in}}%
\pgfpathclose%
\pgfusepath{stroke,fill}%
\end{pgfscope}%
\begin{pgfscope}%
\pgfpathrectangle{\pgfqpoint{9.810417in}{11.981395in}}{\pgfqpoint{5.489583in}{0.877907in}}%
\pgfusepath{clip}%
\pgfsetbuttcap%
\pgfsetroundjoin%
\pgfsetlinewidth{1.505625pt}%
\definecolor{currentstroke}{rgb}{0.000000,0.000000,0.000000}%
\pgfsetstrokecolor{currentstroke}%
\pgfsetdash{}{0pt}%
\pgfpathmoveto{\pgfqpoint{10.059943in}{12.069362in}}%
\pgfpathlineto{\pgfqpoint{10.059943in}{12.819397in}}%
\pgfusepath{stroke}%
\end{pgfscope}%
\begin{pgfscope}%
\pgfpathrectangle{\pgfqpoint{9.810417in}{11.981395in}}{\pgfqpoint{5.489583in}{0.877907in}}%
\pgfusepath{clip}%
\pgfsetbuttcap%
\pgfsetroundjoin%
\pgfsetlinewidth{1.505625pt}%
\definecolor{currentstroke}{rgb}{0.000000,0.000000,0.000000}%
\pgfsetstrokecolor{currentstroke}%
\pgfsetdash{}{0pt}%
\pgfpathmoveto{\pgfqpoint{10.183166in}{12.069362in}}%
\pgfpathlineto{\pgfqpoint{10.183166in}{12.817200in}}%
\pgfusepath{stroke}%
\end{pgfscope}%
\begin{pgfscope}%
\pgfpathrectangle{\pgfqpoint{9.810417in}{11.981395in}}{\pgfqpoint{5.489583in}{0.877907in}}%
\pgfusepath{clip}%
\pgfsetbuttcap%
\pgfsetroundjoin%
\pgfsetlinewidth{1.505625pt}%
\definecolor{currentstroke}{rgb}{0.000000,0.000000,0.000000}%
\pgfsetstrokecolor{currentstroke}%
\pgfsetdash{}{0pt}%
\pgfpathmoveto{\pgfqpoint{10.306389in}{12.069362in}}%
\pgfpathlineto{\pgfqpoint{10.306389in}{12.081146in}}%
\pgfusepath{stroke}%
\end{pgfscope}%
\begin{pgfscope}%
\pgfpathrectangle{\pgfqpoint{9.810417in}{11.981395in}}{\pgfqpoint{5.489583in}{0.877907in}}%
\pgfusepath{clip}%
\pgfsetbuttcap%
\pgfsetroundjoin%
\pgfsetlinewidth{1.505625pt}%
\definecolor{currentstroke}{rgb}{0.000000,0.000000,0.000000}%
\pgfsetstrokecolor{currentstroke}%
\pgfsetdash{}{0pt}%
\pgfpathmoveto{\pgfqpoint{10.429612in}{12.069362in}}%
\pgfpathlineto{\pgfqpoint{10.429612in}{12.073290in}}%
\pgfusepath{stroke}%
\end{pgfscope}%
\begin{pgfscope}%
\pgfpathrectangle{\pgfqpoint{9.810417in}{11.981395in}}{\pgfqpoint{5.489583in}{0.877907in}}%
\pgfusepath{clip}%
\pgfsetbuttcap%
\pgfsetroundjoin%
\pgfsetlinewidth{1.505625pt}%
\definecolor{currentstroke}{rgb}{0.000000,0.000000,0.000000}%
\pgfsetstrokecolor{currentstroke}%
\pgfsetdash{}{0pt}%
\pgfpathmoveto{\pgfqpoint{10.552835in}{12.069362in}}%
\pgfpathlineto{\pgfqpoint{10.552835in}{12.065606in}}%
\pgfusepath{stroke}%
\end{pgfscope}%
\begin{pgfscope}%
\pgfpathrectangle{\pgfqpoint{9.810417in}{11.981395in}}{\pgfqpoint{5.489583in}{0.877907in}}%
\pgfusepath{clip}%
\pgfsetbuttcap%
\pgfsetroundjoin%
\pgfsetlinewidth{1.505625pt}%
\definecolor{currentstroke}{rgb}{0.000000,0.000000,0.000000}%
\pgfsetstrokecolor{currentstroke}%
\pgfsetdash{}{0pt}%
\pgfpathmoveto{\pgfqpoint{10.676058in}{12.069362in}}%
\pgfpathlineto{\pgfqpoint{10.676058in}{12.064613in}}%
\pgfusepath{stroke}%
\end{pgfscope}%
\begin{pgfscope}%
\pgfpathrectangle{\pgfqpoint{9.810417in}{11.981395in}}{\pgfqpoint{5.489583in}{0.877907in}}%
\pgfusepath{clip}%
\pgfsetbuttcap%
\pgfsetroundjoin%
\pgfsetlinewidth{1.505625pt}%
\definecolor{currentstroke}{rgb}{0.000000,0.000000,0.000000}%
\pgfsetstrokecolor{currentstroke}%
\pgfsetdash{}{0pt}%
\pgfpathmoveto{\pgfqpoint{10.799281in}{12.069362in}}%
\pgfpathlineto{\pgfqpoint{10.799281in}{12.067475in}}%
\pgfusepath{stroke}%
\end{pgfscope}%
\begin{pgfscope}%
\pgfpathrectangle{\pgfqpoint{9.810417in}{11.981395in}}{\pgfqpoint{5.489583in}{0.877907in}}%
\pgfusepath{clip}%
\pgfsetbuttcap%
\pgfsetroundjoin%
\pgfsetlinewidth{1.505625pt}%
\definecolor{currentstroke}{rgb}{0.000000,0.000000,0.000000}%
\pgfsetstrokecolor{currentstroke}%
\pgfsetdash{}{0pt}%
\pgfpathmoveto{\pgfqpoint{10.922504in}{12.069362in}}%
\pgfpathlineto{\pgfqpoint{10.922504in}{12.060884in}}%
\pgfusepath{stroke}%
\end{pgfscope}%
\begin{pgfscope}%
\pgfpathrectangle{\pgfqpoint{9.810417in}{11.981395in}}{\pgfqpoint{5.489583in}{0.877907in}}%
\pgfusepath{clip}%
\pgfsetbuttcap%
\pgfsetroundjoin%
\pgfsetlinewidth{1.505625pt}%
\definecolor{currentstroke}{rgb}{0.000000,0.000000,0.000000}%
\pgfsetstrokecolor{currentstroke}%
\pgfsetdash{}{0pt}%
\pgfpathmoveto{\pgfqpoint{11.045727in}{12.069362in}}%
\pgfpathlineto{\pgfqpoint{11.045727in}{12.065368in}}%
\pgfusepath{stroke}%
\end{pgfscope}%
\begin{pgfscope}%
\pgfpathrectangle{\pgfqpoint{9.810417in}{11.981395in}}{\pgfqpoint{5.489583in}{0.877907in}}%
\pgfusepath{clip}%
\pgfsetbuttcap%
\pgfsetroundjoin%
\pgfsetlinewidth{1.505625pt}%
\definecolor{currentstroke}{rgb}{0.000000,0.000000,0.000000}%
\pgfsetstrokecolor{currentstroke}%
\pgfsetdash{}{0pt}%
\pgfpathmoveto{\pgfqpoint{11.168950in}{12.069362in}}%
\pgfpathlineto{\pgfqpoint{11.168950in}{12.065133in}}%
\pgfusepath{stroke}%
\end{pgfscope}%
\begin{pgfscope}%
\pgfpathrectangle{\pgfqpoint{9.810417in}{11.981395in}}{\pgfqpoint{5.489583in}{0.877907in}}%
\pgfusepath{clip}%
\pgfsetbuttcap%
\pgfsetroundjoin%
\pgfsetlinewidth{1.505625pt}%
\definecolor{currentstroke}{rgb}{0.000000,0.000000,0.000000}%
\pgfsetstrokecolor{currentstroke}%
\pgfsetdash{}{0pt}%
\pgfpathmoveto{\pgfqpoint{11.292173in}{12.069362in}}%
\pgfpathlineto{\pgfqpoint{11.292173in}{12.108848in}}%
\pgfusepath{stroke}%
\end{pgfscope}%
\begin{pgfscope}%
\pgfpathrectangle{\pgfqpoint{9.810417in}{11.981395in}}{\pgfqpoint{5.489583in}{0.877907in}}%
\pgfusepath{clip}%
\pgfsetbuttcap%
\pgfsetroundjoin%
\pgfsetlinewidth{1.505625pt}%
\definecolor{currentstroke}{rgb}{0.000000,0.000000,0.000000}%
\pgfsetstrokecolor{currentstroke}%
\pgfsetdash{}{0pt}%
\pgfpathmoveto{\pgfqpoint{11.415396in}{12.069362in}}%
\pgfpathlineto{\pgfqpoint{11.415396in}{12.081383in}}%
\pgfusepath{stroke}%
\end{pgfscope}%
\begin{pgfscope}%
\pgfpathrectangle{\pgfqpoint{9.810417in}{11.981395in}}{\pgfqpoint{5.489583in}{0.877907in}}%
\pgfusepath{clip}%
\pgfsetbuttcap%
\pgfsetroundjoin%
\pgfsetlinewidth{1.505625pt}%
\definecolor{currentstroke}{rgb}{0.000000,0.000000,0.000000}%
\pgfsetstrokecolor{currentstroke}%
\pgfsetdash{}{0pt}%
\pgfpathmoveto{\pgfqpoint{11.538619in}{12.069362in}}%
\pgfpathlineto{\pgfqpoint{11.538619in}{12.058489in}}%
\pgfusepath{stroke}%
\end{pgfscope}%
\begin{pgfscope}%
\pgfpathrectangle{\pgfqpoint{9.810417in}{11.981395in}}{\pgfqpoint{5.489583in}{0.877907in}}%
\pgfusepath{clip}%
\pgfsetbuttcap%
\pgfsetroundjoin%
\pgfsetlinewidth{1.505625pt}%
\definecolor{currentstroke}{rgb}{0.000000,0.000000,0.000000}%
\pgfsetstrokecolor{currentstroke}%
\pgfsetdash{}{0pt}%
\pgfpathmoveto{\pgfqpoint{11.661842in}{12.069362in}}%
\pgfpathlineto{\pgfqpoint{11.661842in}{12.075366in}}%
\pgfusepath{stroke}%
\end{pgfscope}%
\begin{pgfscope}%
\pgfpathrectangle{\pgfqpoint{9.810417in}{11.981395in}}{\pgfqpoint{5.489583in}{0.877907in}}%
\pgfusepath{clip}%
\pgfsetbuttcap%
\pgfsetroundjoin%
\pgfsetlinewidth{1.505625pt}%
\definecolor{currentstroke}{rgb}{0.000000,0.000000,0.000000}%
\pgfsetstrokecolor{currentstroke}%
\pgfsetdash{}{0pt}%
\pgfpathmoveto{\pgfqpoint{11.785065in}{12.069362in}}%
\pgfpathlineto{\pgfqpoint{11.785065in}{12.085929in}}%
\pgfusepath{stroke}%
\end{pgfscope}%
\begin{pgfscope}%
\pgfpathrectangle{\pgfqpoint{9.810417in}{11.981395in}}{\pgfqpoint{5.489583in}{0.877907in}}%
\pgfusepath{clip}%
\pgfsetbuttcap%
\pgfsetroundjoin%
\pgfsetlinewidth{1.505625pt}%
\definecolor{currentstroke}{rgb}{0.000000,0.000000,0.000000}%
\pgfsetstrokecolor{currentstroke}%
\pgfsetdash{}{0pt}%
\pgfpathmoveto{\pgfqpoint{11.908288in}{12.069362in}}%
\pgfpathlineto{\pgfqpoint{11.908288in}{12.074803in}}%
\pgfusepath{stroke}%
\end{pgfscope}%
\begin{pgfscope}%
\pgfpathrectangle{\pgfqpoint{9.810417in}{11.981395in}}{\pgfqpoint{5.489583in}{0.877907in}}%
\pgfusepath{clip}%
\pgfsetbuttcap%
\pgfsetroundjoin%
\pgfsetlinewidth{1.505625pt}%
\definecolor{currentstroke}{rgb}{0.000000,0.000000,0.000000}%
\pgfsetstrokecolor{currentstroke}%
\pgfsetdash{}{0pt}%
\pgfpathmoveto{\pgfqpoint{12.031511in}{12.069362in}}%
\pgfpathlineto{\pgfqpoint{12.031511in}{12.102232in}}%
\pgfusepath{stroke}%
\end{pgfscope}%
\begin{pgfscope}%
\pgfpathrectangle{\pgfqpoint{9.810417in}{11.981395in}}{\pgfqpoint{5.489583in}{0.877907in}}%
\pgfusepath{clip}%
\pgfsetbuttcap%
\pgfsetroundjoin%
\pgfsetlinewidth{1.505625pt}%
\definecolor{currentstroke}{rgb}{0.000000,0.000000,0.000000}%
\pgfsetstrokecolor{currentstroke}%
\pgfsetdash{}{0pt}%
\pgfpathmoveto{\pgfqpoint{12.154734in}{12.069362in}}%
\pgfpathlineto{\pgfqpoint{12.154734in}{12.077707in}}%
\pgfusepath{stroke}%
\end{pgfscope}%
\begin{pgfscope}%
\pgfpathrectangle{\pgfqpoint{9.810417in}{11.981395in}}{\pgfqpoint{5.489583in}{0.877907in}}%
\pgfusepath{clip}%
\pgfsetbuttcap%
\pgfsetroundjoin%
\pgfsetlinewidth{1.505625pt}%
\definecolor{currentstroke}{rgb}{0.000000,0.000000,0.000000}%
\pgfsetstrokecolor{currentstroke}%
\pgfsetdash{}{0pt}%
\pgfpathmoveto{\pgfqpoint{12.277957in}{12.069362in}}%
\pgfpathlineto{\pgfqpoint{12.277957in}{12.047248in}}%
\pgfusepath{stroke}%
\end{pgfscope}%
\begin{pgfscope}%
\pgfpathrectangle{\pgfqpoint{9.810417in}{11.981395in}}{\pgfqpoint{5.489583in}{0.877907in}}%
\pgfusepath{clip}%
\pgfsetbuttcap%
\pgfsetroundjoin%
\pgfsetlinewidth{1.505625pt}%
\definecolor{currentstroke}{rgb}{0.000000,0.000000,0.000000}%
\pgfsetstrokecolor{currentstroke}%
\pgfsetdash{}{0pt}%
\pgfpathmoveto{\pgfqpoint{12.401180in}{12.069362in}}%
\pgfpathlineto{\pgfqpoint{12.401180in}{12.021300in}}%
\pgfusepath{stroke}%
\end{pgfscope}%
\begin{pgfscope}%
\pgfpathrectangle{\pgfqpoint{9.810417in}{11.981395in}}{\pgfqpoint{5.489583in}{0.877907in}}%
\pgfusepath{clip}%
\pgfsetbuttcap%
\pgfsetroundjoin%
\pgfsetlinewidth{1.505625pt}%
\definecolor{currentstroke}{rgb}{0.000000,0.000000,0.000000}%
\pgfsetstrokecolor{currentstroke}%
\pgfsetdash{}{0pt}%
\pgfpathmoveto{\pgfqpoint{12.524403in}{12.069362in}}%
\pgfpathlineto{\pgfqpoint{12.524403in}{12.055468in}}%
\pgfusepath{stroke}%
\end{pgfscope}%
\begin{pgfscope}%
\pgfpathrectangle{\pgfqpoint{9.810417in}{11.981395in}}{\pgfqpoint{5.489583in}{0.877907in}}%
\pgfusepath{clip}%
\pgfsetbuttcap%
\pgfsetroundjoin%
\pgfsetlinewidth{1.505625pt}%
\definecolor{currentstroke}{rgb}{0.000000,0.000000,0.000000}%
\pgfsetstrokecolor{currentstroke}%
\pgfsetdash{}{0pt}%
\pgfpathmoveto{\pgfqpoint{12.647626in}{12.069362in}}%
\pgfpathlineto{\pgfqpoint{12.647626in}{12.063044in}}%
\pgfusepath{stroke}%
\end{pgfscope}%
\begin{pgfscope}%
\pgfpathrectangle{\pgfqpoint{9.810417in}{11.981395in}}{\pgfqpoint{5.489583in}{0.877907in}}%
\pgfusepath{clip}%
\pgfsetbuttcap%
\pgfsetroundjoin%
\pgfsetlinewidth{1.505625pt}%
\definecolor{currentstroke}{rgb}{0.000000,0.000000,0.000000}%
\pgfsetstrokecolor{currentstroke}%
\pgfsetdash{}{0pt}%
\pgfpathmoveto{\pgfqpoint{12.770849in}{12.069362in}}%
\pgfpathlineto{\pgfqpoint{12.770849in}{12.071533in}}%
\pgfusepath{stroke}%
\end{pgfscope}%
\begin{pgfscope}%
\pgfpathrectangle{\pgfqpoint{9.810417in}{11.981395in}}{\pgfqpoint{5.489583in}{0.877907in}}%
\pgfusepath{clip}%
\pgfsetbuttcap%
\pgfsetroundjoin%
\pgfsetlinewidth{1.505625pt}%
\definecolor{currentstroke}{rgb}{0.000000,0.000000,0.000000}%
\pgfsetstrokecolor{currentstroke}%
\pgfsetdash{}{0pt}%
\pgfpathmoveto{\pgfqpoint{12.894072in}{12.069362in}}%
\pgfpathlineto{\pgfqpoint{12.894072in}{12.050384in}}%
\pgfusepath{stroke}%
\end{pgfscope}%
\begin{pgfscope}%
\pgfpathrectangle{\pgfqpoint{9.810417in}{11.981395in}}{\pgfqpoint{5.489583in}{0.877907in}}%
\pgfusepath{clip}%
\pgfsetbuttcap%
\pgfsetroundjoin%
\pgfsetlinewidth{1.505625pt}%
\definecolor{currentstroke}{rgb}{0.000000,0.000000,0.000000}%
\pgfsetstrokecolor{currentstroke}%
\pgfsetdash{}{0pt}%
\pgfpathmoveto{\pgfqpoint{13.017294in}{12.069362in}}%
\pgfpathlineto{\pgfqpoint{13.017294in}{12.066209in}}%
\pgfusepath{stroke}%
\end{pgfscope}%
\begin{pgfscope}%
\pgfpathrectangle{\pgfqpoint{9.810417in}{11.981395in}}{\pgfqpoint{5.489583in}{0.877907in}}%
\pgfusepath{clip}%
\pgfsetbuttcap%
\pgfsetroundjoin%
\pgfsetlinewidth{1.505625pt}%
\definecolor{currentstroke}{rgb}{0.000000,0.000000,0.000000}%
\pgfsetstrokecolor{currentstroke}%
\pgfsetdash{}{0pt}%
\pgfpathmoveto{\pgfqpoint{13.140517in}{12.069362in}}%
\pgfpathlineto{\pgfqpoint{13.140517in}{12.072928in}}%
\pgfusepath{stroke}%
\end{pgfscope}%
\begin{pgfscope}%
\pgfpathrectangle{\pgfqpoint{9.810417in}{11.981395in}}{\pgfqpoint{5.489583in}{0.877907in}}%
\pgfusepath{clip}%
\pgfsetbuttcap%
\pgfsetroundjoin%
\pgfsetlinewidth{1.505625pt}%
\definecolor{currentstroke}{rgb}{0.000000,0.000000,0.000000}%
\pgfsetstrokecolor{currentstroke}%
\pgfsetdash{}{0pt}%
\pgfpathmoveto{\pgfqpoint{13.263740in}{12.069362in}}%
\pgfpathlineto{\pgfqpoint{13.263740in}{12.073318in}}%
\pgfusepath{stroke}%
\end{pgfscope}%
\begin{pgfscope}%
\pgfpathrectangle{\pgfqpoint{9.810417in}{11.981395in}}{\pgfqpoint{5.489583in}{0.877907in}}%
\pgfusepath{clip}%
\pgfsetbuttcap%
\pgfsetroundjoin%
\pgfsetlinewidth{1.505625pt}%
\definecolor{currentstroke}{rgb}{0.000000,0.000000,0.000000}%
\pgfsetstrokecolor{currentstroke}%
\pgfsetdash{}{0pt}%
\pgfpathmoveto{\pgfqpoint{13.386963in}{12.069362in}}%
\pgfpathlineto{\pgfqpoint{13.386963in}{12.077699in}}%
\pgfusepath{stroke}%
\end{pgfscope}%
\begin{pgfscope}%
\pgfpathrectangle{\pgfqpoint{9.810417in}{11.981395in}}{\pgfqpoint{5.489583in}{0.877907in}}%
\pgfusepath{clip}%
\pgfsetbuttcap%
\pgfsetroundjoin%
\pgfsetlinewidth{1.505625pt}%
\definecolor{currentstroke}{rgb}{0.000000,0.000000,0.000000}%
\pgfsetstrokecolor{currentstroke}%
\pgfsetdash{}{0pt}%
\pgfpathmoveto{\pgfqpoint{13.510186in}{12.069362in}}%
\pgfpathlineto{\pgfqpoint{13.510186in}{12.068278in}}%
\pgfusepath{stroke}%
\end{pgfscope}%
\begin{pgfscope}%
\pgfpathrectangle{\pgfqpoint{9.810417in}{11.981395in}}{\pgfqpoint{5.489583in}{0.877907in}}%
\pgfusepath{clip}%
\pgfsetbuttcap%
\pgfsetroundjoin%
\pgfsetlinewidth{1.505625pt}%
\definecolor{currentstroke}{rgb}{0.000000,0.000000,0.000000}%
\pgfsetstrokecolor{currentstroke}%
\pgfsetdash{}{0pt}%
\pgfpathmoveto{\pgfqpoint{13.633409in}{12.069362in}}%
\pgfpathlineto{\pgfqpoint{13.633409in}{12.065080in}}%
\pgfusepath{stroke}%
\end{pgfscope}%
\begin{pgfscope}%
\pgfpathrectangle{\pgfqpoint{9.810417in}{11.981395in}}{\pgfqpoint{5.489583in}{0.877907in}}%
\pgfusepath{clip}%
\pgfsetbuttcap%
\pgfsetroundjoin%
\pgfsetlinewidth{1.505625pt}%
\definecolor{currentstroke}{rgb}{0.000000,0.000000,0.000000}%
\pgfsetstrokecolor{currentstroke}%
\pgfsetdash{}{0pt}%
\pgfpathmoveto{\pgfqpoint{13.756632in}{12.069362in}}%
\pgfpathlineto{\pgfqpoint{13.756632in}{12.051108in}}%
\pgfusepath{stroke}%
\end{pgfscope}%
\begin{pgfscope}%
\pgfpathrectangle{\pgfqpoint{9.810417in}{11.981395in}}{\pgfqpoint{5.489583in}{0.877907in}}%
\pgfusepath{clip}%
\pgfsetbuttcap%
\pgfsetroundjoin%
\pgfsetlinewidth{1.505625pt}%
\definecolor{currentstroke}{rgb}{0.000000,0.000000,0.000000}%
\pgfsetstrokecolor{currentstroke}%
\pgfsetdash{}{0pt}%
\pgfpathmoveto{\pgfqpoint{13.879855in}{12.069362in}}%
\pgfpathlineto{\pgfqpoint{13.879855in}{12.058356in}}%
\pgfusepath{stroke}%
\end{pgfscope}%
\begin{pgfscope}%
\pgfpathrectangle{\pgfqpoint{9.810417in}{11.981395in}}{\pgfqpoint{5.489583in}{0.877907in}}%
\pgfusepath{clip}%
\pgfsetbuttcap%
\pgfsetroundjoin%
\pgfsetlinewidth{1.505625pt}%
\definecolor{currentstroke}{rgb}{0.000000,0.000000,0.000000}%
\pgfsetstrokecolor{currentstroke}%
\pgfsetdash{}{0pt}%
\pgfpathmoveto{\pgfqpoint{14.003078in}{12.069362in}}%
\pgfpathlineto{\pgfqpoint{14.003078in}{12.072160in}}%
\pgfusepath{stroke}%
\end{pgfscope}%
\begin{pgfscope}%
\pgfpathrectangle{\pgfqpoint{9.810417in}{11.981395in}}{\pgfqpoint{5.489583in}{0.877907in}}%
\pgfusepath{clip}%
\pgfsetbuttcap%
\pgfsetroundjoin%
\pgfsetlinewidth{1.505625pt}%
\definecolor{currentstroke}{rgb}{0.000000,0.000000,0.000000}%
\pgfsetstrokecolor{currentstroke}%
\pgfsetdash{}{0pt}%
\pgfpathmoveto{\pgfqpoint{14.126301in}{12.069362in}}%
\pgfpathlineto{\pgfqpoint{14.126301in}{12.072771in}}%
\pgfusepath{stroke}%
\end{pgfscope}%
\begin{pgfscope}%
\pgfpathrectangle{\pgfqpoint{9.810417in}{11.981395in}}{\pgfqpoint{5.489583in}{0.877907in}}%
\pgfusepath{clip}%
\pgfsetbuttcap%
\pgfsetroundjoin%
\pgfsetlinewidth{1.505625pt}%
\definecolor{currentstroke}{rgb}{0.000000,0.000000,0.000000}%
\pgfsetstrokecolor{currentstroke}%
\pgfsetdash{}{0pt}%
\pgfpathmoveto{\pgfqpoint{14.249524in}{12.069362in}}%
\pgfpathlineto{\pgfqpoint{14.249524in}{12.060944in}}%
\pgfusepath{stroke}%
\end{pgfscope}%
\begin{pgfscope}%
\pgfpathrectangle{\pgfqpoint{9.810417in}{11.981395in}}{\pgfqpoint{5.489583in}{0.877907in}}%
\pgfusepath{clip}%
\pgfsetbuttcap%
\pgfsetroundjoin%
\pgfsetlinewidth{1.505625pt}%
\definecolor{currentstroke}{rgb}{0.000000,0.000000,0.000000}%
\pgfsetstrokecolor{currentstroke}%
\pgfsetdash{}{0pt}%
\pgfpathmoveto{\pgfqpoint{14.372747in}{12.069362in}}%
\pgfpathlineto{\pgfqpoint{14.372747in}{12.038156in}}%
\pgfusepath{stroke}%
\end{pgfscope}%
\begin{pgfscope}%
\pgfpathrectangle{\pgfqpoint{9.810417in}{11.981395in}}{\pgfqpoint{5.489583in}{0.877907in}}%
\pgfusepath{clip}%
\pgfsetbuttcap%
\pgfsetroundjoin%
\pgfsetlinewidth{1.505625pt}%
\definecolor{currentstroke}{rgb}{0.000000,0.000000,0.000000}%
\pgfsetstrokecolor{currentstroke}%
\pgfsetdash{}{0pt}%
\pgfpathmoveto{\pgfqpoint{14.495970in}{12.069362in}}%
\pgfpathlineto{\pgfqpoint{14.495970in}{12.046751in}}%
\pgfusepath{stroke}%
\end{pgfscope}%
\begin{pgfscope}%
\pgfpathrectangle{\pgfqpoint{9.810417in}{11.981395in}}{\pgfqpoint{5.489583in}{0.877907in}}%
\pgfusepath{clip}%
\pgfsetbuttcap%
\pgfsetroundjoin%
\pgfsetlinewidth{1.505625pt}%
\definecolor{currentstroke}{rgb}{0.000000,0.000000,0.000000}%
\pgfsetstrokecolor{currentstroke}%
\pgfsetdash{}{0pt}%
\pgfpathmoveto{\pgfqpoint{14.619193in}{12.069362in}}%
\pgfpathlineto{\pgfqpoint{14.619193in}{12.069568in}}%
\pgfusepath{stroke}%
\end{pgfscope}%
\begin{pgfscope}%
\pgfpathrectangle{\pgfqpoint{9.810417in}{11.981395in}}{\pgfqpoint{5.489583in}{0.877907in}}%
\pgfusepath{clip}%
\pgfsetbuttcap%
\pgfsetroundjoin%
\pgfsetlinewidth{1.505625pt}%
\definecolor{currentstroke}{rgb}{0.000000,0.000000,0.000000}%
\pgfsetstrokecolor{currentstroke}%
\pgfsetdash{}{0pt}%
\pgfpathmoveto{\pgfqpoint{14.742416in}{12.069362in}}%
\pgfpathlineto{\pgfqpoint{14.742416in}{12.060900in}}%
\pgfusepath{stroke}%
\end{pgfscope}%
\begin{pgfscope}%
\pgfpathrectangle{\pgfqpoint{9.810417in}{11.981395in}}{\pgfqpoint{5.489583in}{0.877907in}}%
\pgfusepath{clip}%
\pgfsetbuttcap%
\pgfsetroundjoin%
\pgfsetlinewidth{1.505625pt}%
\definecolor{currentstroke}{rgb}{0.000000,0.000000,0.000000}%
\pgfsetstrokecolor{currentstroke}%
\pgfsetdash{}{0pt}%
\pgfpathmoveto{\pgfqpoint{14.865639in}{12.069362in}}%
\pgfpathlineto{\pgfqpoint{14.865639in}{12.087149in}}%
\pgfusepath{stroke}%
\end{pgfscope}%
\begin{pgfscope}%
\pgfpathrectangle{\pgfqpoint{9.810417in}{11.981395in}}{\pgfqpoint{5.489583in}{0.877907in}}%
\pgfusepath{clip}%
\pgfsetbuttcap%
\pgfsetroundjoin%
\pgfsetlinewidth{1.505625pt}%
\definecolor{currentstroke}{rgb}{0.000000,0.000000,0.000000}%
\pgfsetstrokecolor{currentstroke}%
\pgfsetdash{}{0pt}%
\pgfpathmoveto{\pgfqpoint{14.988862in}{12.069362in}}%
\pgfpathlineto{\pgfqpoint{14.988862in}{12.059867in}}%
\pgfusepath{stroke}%
\end{pgfscope}%
\begin{pgfscope}%
\pgfpathrectangle{\pgfqpoint{9.810417in}{11.981395in}}{\pgfqpoint{5.489583in}{0.877907in}}%
\pgfusepath{clip}%
\pgfsetroundcap%
\pgfsetroundjoin%
\pgfsetlinewidth{1.505625pt}%
\definecolor{currentstroke}{rgb}{0.121569,0.466667,0.705882}%
\pgfsetstrokecolor{currentstroke}%
\pgfsetdash{}{0pt}%
\pgfpathmoveto{\pgfqpoint{9.810417in}{12.069362in}}%
\pgfpathlineto{\pgfqpoint{15.300000in}{12.069362in}}%
\pgfusepath{stroke}%
\end{pgfscope}%
\begin{pgfscope}%
\pgfpathrectangle{\pgfqpoint{9.810417in}{11.981395in}}{\pgfqpoint{5.489583in}{0.877907in}}%
\pgfusepath{clip}%
\pgfsetbuttcap%
\pgfsetroundjoin%
\definecolor{currentfill}{rgb}{0.121569,0.466667,0.705882}%
\pgfsetfillcolor{currentfill}%
\pgfsetlinewidth{1.003750pt}%
\definecolor{currentstroke}{rgb}{0.121569,0.466667,0.705882}%
\pgfsetstrokecolor{currentstroke}%
\pgfsetdash{}{0pt}%
\pgfsys@defobject{currentmarker}{\pgfqpoint{-0.034722in}{-0.034722in}}{\pgfqpoint{0.034722in}{0.034722in}}{%
\pgfpathmoveto{\pgfqpoint{0.000000in}{-0.034722in}}%
\pgfpathcurveto{\pgfqpoint{0.009208in}{-0.034722in}}{\pgfqpoint{0.018041in}{-0.031064in}}{\pgfqpoint{0.024552in}{-0.024552in}}%
\pgfpathcurveto{\pgfqpoint{0.031064in}{-0.018041in}}{\pgfqpoint{0.034722in}{-0.009208in}}{\pgfqpoint{0.034722in}{0.000000in}}%
\pgfpathcurveto{\pgfqpoint{0.034722in}{0.009208in}}{\pgfqpoint{0.031064in}{0.018041in}}{\pgfqpoint{0.024552in}{0.024552in}}%
\pgfpathcurveto{\pgfqpoint{0.018041in}{0.031064in}}{\pgfqpoint{0.009208in}{0.034722in}}{\pgfqpoint{0.000000in}{0.034722in}}%
\pgfpathcurveto{\pgfqpoint{-0.009208in}{0.034722in}}{\pgfqpoint{-0.018041in}{0.031064in}}{\pgfqpoint{-0.024552in}{0.024552in}}%
\pgfpathcurveto{\pgfqpoint{-0.031064in}{0.018041in}}{\pgfqpoint{-0.034722in}{0.009208in}}{\pgfqpoint{-0.034722in}{0.000000in}}%
\pgfpathcurveto{\pgfqpoint{-0.034722in}{-0.009208in}}{\pgfqpoint{-0.031064in}{-0.018041in}}{\pgfqpoint{-0.024552in}{-0.024552in}}%
\pgfpathcurveto{\pgfqpoint{-0.018041in}{-0.031064in}}{\pgfqpoint{-0.009208in}{-0.034722in}}{\pgfqpoint{0.000000in}{-0.034722in}}%
\pgfpathclose%
\pgfusepath{stroke,fill}%
}%
\begin{pgfscope}%
\pgfsys@transformshift{10.059943in}{12.819397in}%
\pgfsys@useobject{currentmarker}{}%
\end{pgfscope}%
\begin{pgfscope}%
\pgfsys@transformshift{10.183166in}{12.817200in}%
\pgfsys@useobject{currentmarker}{}%
\end{pgfscope}%
\begin{pgfscope}%
\pgfsys@transformshift{10.306389in}{12.081146in}%
\pgfsys@useobject{currentmarker}{}%
\end{pgfscope}%
\begin{pgfscope}%
\pgfsys@transformshift{10.429612in}{12.073290in}%
\pgfsys@useobject{currentmarker}{}%
\end{pgfscope}%
\begin{pgfscope}%
\pgfsys@transformshift{10.552835in}{12.065606in}%
\pgfsys@useobject{currentmarker}{}%
\end{pgfscope}%
\begin{pgfscope}%
\pgfsys@transformshift{10.676058in}{12.064613in}%
\pgfsys@useobject{currentmarker}{}%
\end{pgfscope}%
\begin{pgfscope}%
\pgfsys@transformshift{10.799281in}{12.067475in}%
\pgfsys@useobject{currentmarker}{}%
\end{pgfscope}%
\begin{pgfscope}%
\pgfsys@transformshift{10.922504in}{12.060884in}%
\pgfsys@useobject{currentmarker}{}%
\end{pgfscope}%
\begin{pgfscope}%
\pgfsys@transformshift{11.045727in}{12.065368in}%
\pgfsys@useobject{currentmarker}{}%
\end{pgfscope}%
\begin{pgfscope}%
\pgfsys@transformshift{11.168950in}{12.065133in}%
\pgfsys@useobject{currentmarker}{}%
\end{pgfscope}%
\begin{pgfscope}%
\pgfsys@transformshift{11.292173in}{12.108848in}%
\pgfsys@useobject{currentmarker}{}%
\end{pgfscope}%
\begin{pgfscope}%
\pgfsys@transformshift{11.415396in}{12.081383in}%
\pgfsys@useobject{currentmarker}{}%
\end{pgfscope}%
\begin{pgfscope}%
\pgfsys@transformshift{11.538619in}{12.058489in}%
\pgfsys@useobject{currentmarker}{}%
\end{pgfscope}%
\begin{pgfscope}%
\pgfsys@transformshift{11.661842in}{12.075366in}%
\pgfsys@useobject{currentmarker}{}%
\end{pgfscope}%
\begin{pgfscope}%
\pgfsys@transformshift{11.785065in}{12.085929in}%
\pgfsys@useobject{currentmarker}{}%
\end{pgfscope}%
\begin{pgfscope}%
\pgfsys@transformshift{11.908288in}{12.074803in}%
\pgfsys@useobject{currentmarker}{}%
\end{pgfscope}%
\begin{pgfscope}%
\pgfsys@transformshift{12.031511in}{12.102232in}%
\pgfsys@useobject{currentmarker}{}%
\end{pgfscope}%
\begin{pgfscope}%
\pgfsys@transformshift{12.154734in}{12.077707in}%
\pgfsys@useobject{currentmarker}{}%
\end{pgfscope}%
\begin{pgfscope}%
\pgfsys@transformshift{12.277957in}{12.047248in}%
\pgfsys@useobject{currentmarker}{}%
\end{pgfscope}%
\begin{pgfscope}%
\pgfsys@transformshift{12.401180in}{12.021300in}%
\pgfsys@useobject{currentmarker}{}%
\end{pgfscope}%
\begin{pgfscope}%
\pgfsys@transformshift{12.524403in}{12.055468in}%
\pgfsys@useobject{currentmarker}{}%
\end{pgfscope}%
\begin{pgfscope}%
\pgfsys@transformshift{12.647626in}{12.063044in}%
\pgfsys@useobject{currentmarker}{}%
\end{pgfscope}%
\begin{pgfscope}%
\pgfsys@transformshift{12.770849in}{12.071533in}%
\pgfsys@useobject{currentmarker}{}%
\end{pgfscope}%
\begin{pgfscope}%
\pgfsys@transformshift{12.894072in}{12.050384in}%
\pgfsys@useobject{currentmarker}{}%
\end{pgfscope}%
\begin{pgfscope}%
\pgfsys@transformshift{13.017294in}{12.066209in}%
\pgfsys@useobject{currentmarker}{}%
\end{pgfscope}%
\begin{pgfscope}%
\pgfsys@transformshift{13.140517in}{12.072928in}%
\pgfsys@useobject{currentmarker}{}%
\end{pgfscope}%
\begin{pgfscope}%
\pgfsys@transformshift{13.263740in}{12.073318in}%
\pgfsys@useobject{currentmarker}{}%
\end{pgfscope}%
\begin{pgfscope}%
\pgfsys@transformshift{13.386963in}{12.077699in}%
\pgfsys@useobject{currentmarker}{}%
\end{pgfscope}%
\begin{pgfscope}%
\pgfsys@transformshift{13.510186in}{12.068278in}%
\pgfsys@useobject{currentmarker}{}%
\end{pgfscope}%
\begin{pgfscope}%
\pgfsys@transformshift{13.633409in}{12.065080in}%
\pgfsys@useobject{currentmarker}{}%
\end{pgfscope}%
\begin{pgfscope}%
\pgfsys@transformshift{13.756632in}{12.051108in}%
\pgfsys@useobject{currentmarker}{}%
\end{pgfscope}%
\begin{pgfscope}%
\pgfsys@transformshift{13.879855in}{12.058356in}%
\pgfsys@useobject{currentmarker}{}%
\end{pgfscope}%
\begin{pgfscope}%
\pgfsys@transformshift{14.003078in}{12.072160in}%
\pgfsys@useobject{currentmarker}{}%
\end{pgfscope}%
\begin{pgfscope}%
\pgfsys@transformshift{14.126301in}{12.072771in}%
\pgfsys@useobject{currentmarker}{}%
\end{pgfscope}%
\begin{pgfscope}%
\pgfsys@transformshift{14.249524in}{12.060944in}%
\pgfsys@useobject{currentmarker}{}%
\end{pgfscope}%
\begin{pgfscope}%
\pgfsys@transformshift{14.372747in}{12.038156in}%
\pgfsys@useobject{currentmarker}{}%
\end{pgfscope}%
\begin{pgfscope}%
\pgfsys@transformshift{14.495970in}{12.046751in}%
\pgfsys@useobject{currentmarker}{}%
\end{pgfscope}%
\begin{pgfscope}%
\pgfsys@transformshift{14.619193in}{12.069568in}%
\pgfsys@useobject{currentmarker}{}%
\end{pgfscope}%
\begin{pgfscope}%
\pgfsys@transformshift{14.742416in}{12.060900in}%
\pgfsys@useobject{currentmarker}{}%
\end{pgfscope}%
\begin{pgfscope}%
\pgfsys@transformshift{14.865639in}{12.087149in}%
\pgfsys@useobject{currentmarker}{}%
\end{pgfscope}%
\begin{pgfscope}%
\pgfsys@transformshift{14.988862in}{12.059867in}%
\pgfsys@useobject{currentmarker}{}%
\end{pgfscope}%
\end{pgfscope}%
\begin{pgfscope}%
\pgfsetrectcap%
\pgfsetmiterjoin%
\pgfsetlinewidth{0.803000pt}%
\definecolor{currentstroke}{rgb}{1.000000,1.000000,1.000000}%
\pgfsetstrokecolor{currentstroke}%
\pgfsetdash{}{0pt}%
\pgfpathmoveto{\pgfqpoint{9.810417in}{11.981395in}}%
\pgfpathlineto{\pgfqpoint{9.810417in}{12.859302in}}%
\pgfusepath{stroke}%
\end{pgfscope}%
\begin{pgfscope}%
\pgfsetrectcap%
\pgfsetmiterjoin%
\pgfsetlinewidth{0.803000pt}%
\definecolor{currentstroke}{rgb}{1.000000,1.000000,1.000000}%
\pgfsetstrokecolor{currentstroke}%
\pgfsetdash{}{0pt}%
\pgfpathmoveto{\pgfqpoint{15.300000in}{11.981395in}}%
\pgfpathlineto{\pgfqpoint{15.300000in}{12.859302in}}%
\pgfusepath{stroke}%
\end{pgfscope}%
\begin{pgfscope}%
\pgfsetrectcap%
\pgfsetmiterjoin%
\pgfsetlinewidth{0.803000pt}%
\definecolor{currentstroke}{rgb}{1.000000,1.000000,1.000000}%
\pgfsetstrokecolor{currentstroke}%
\pgfsetdash{}{0pt}%
\pgfpathmoveto{\pgfqpoint{9.810417in}{11.981395in}}%
\pgfpathlineto{\pgfqpoint{15.300000in}{11.981395in}}%
\pgfusepath{stroke}%
\end{pgfscope}%
\begin{pgfscope}%
\pgfsetrectcap%
\pgfsetmiterjoin%
\pgfsetlinewidth{0.803000pt}%
\definecolor{currentstroke}{rgb}{1.000000,1.000000,1.000000}%
\pgfsetstrokecolor{currentstroke}%
\pgfsetdash{}{0pt}%
\pgfpathmoveto{\pgfqpoint{9.810417in}{12.859302in}}%
\pgfpathlineto{\pgfqpoint{15.300000in}{12.859302in}}%
\pgfusepath{stroke}%
\end{pgfscope}%
\begin{pgfscope}%
\definecolor{textcolor}{rgb}{0.150000,0.150000,0.150000}%
\pgfsetstrokecolor{textcolor}%
\pgfsetfillcolor{textcolor}%
\pgftext[x=12.555208in,y=12.942636in,,base]{\color{textcolor}\rmfamily\fontsize{16.800000}{20.160000}\selectfont Partial Autocorrelation}%
\end{pgfscope}%
\begin{pgfscope}%
\pgfsetbuttcap%
\pgfsetmiterjoin%
\definecolor{currentfill}{rgb}{0.917647,0.917647,0.949020}%
\pgfsetfillcolor{currentfill}%
\pgfsetlinewidth{0.000000pt}%
\definecolor{currentstroke}{rgb}{0.000000,0.000000,0.000000}%
\pgfsetstrokecolor{currentstroke}%
\pgfsetstrokeopacity{0.000000}%
\pgfsetdash{}{0pt}%
\pgfpathmoveto{\pgfqpoint{2.125000in}{10.401163in}}%
\pgfpathlineto{\pgfqpoint{7.614583in}{10.401163in}}%
\pgfpathlineto{\pgfqpoint{7.614583in}{11.279070in}}%
\pgfpathlineto{\pgfqpoint{2.125000in}{11.279070in}}%
\pgfpathclose%
\pgfusepath{fill}%
\end{pgfscope}%
\begin{pgfscope}%
\pgfpathrectangle{\pgfqpoint{2.125000in}{10.401163in}}{\pgfqpoint{5.489583in}{0.877907in}}%
\pgfusepath{clip}%
\pgfsetroundcap%
\pgfsetroundjoin%
\pgfsetlinewidth{0.803000pt}%
\definecolor{currentstroke}{rgb}{1.000000,1.000000,1.000000}%
\pgfsetstrokecolor{currentstroke}%
\pgfsetdash{}{0pt}%
\pgfpathmoveto{\pgfqpoint{2.374527in}{10.401163in}}%
\pgfpathlineto{\pgfqpoint{2.374527in}{11.279070in}}%
\pgfusepath{stroke}%
\end{pgfscope}%
\begin{pgfscope}%
\definecolor{textcolor}{rgb}{0.150000,0.150000,0.150000}%
\pgfsetstrokecolor{textcolor}%
\pgfsetfillcolor{textcolor}%
\pgftext[x=2.374527in,y=10.303941in,,top]{\color{textcolor}\rmfamily\fontsize{14.000000}{16.800000}\selectfont 0}%
\end{pgfscope}%
\begin{pgfscope}%
\pgfpathrectangle{\pgfqpoint{2.125000in}{10.401163in}}{\pgfqpoint{5.489583in}{0.877907in}}%
\pgfusepath{clip}%
\pgfsetroundcap%
\pgfsetroundjoin%
\pgfsetlinewidth{0.803000pt}%
\definecolor{currentstroke}{rgb}{1.000000,1.000000,1.000000}%
\pgfsetstrokecolor{currentstroke}%
\pgfsetdash{}{0pt}%
\pgfpathmoveto{\pgfqpoint{2.990641in}{10.401163in}}%
\pgfpathlineto{\pgfqpoint{2.990641in}{11.279070in}}%
\pgfusepath{stroke}%
\end{pgfscope}%
\begin{pgfscope}%
\definecolor{textcolor}{rgb}{0.150000,0.150000,0.150000}%
\pgfsetstrokecolor{textcolor}%
\pgfsetfillcolor{textcolor}%
\pgftext[x=2.990641in,y=10.303941in,,top]{\color{textcolor}\rmfamily\fontsize{14.000000}{16.800000}\selectfont 5}%
\end{pgfscope}%
\begin{pgfscope}%
\pgfpathrectangle{\pgfqpoint{2.125000in}{10.401163in}}{\pgfqpoint{5.489583in}{0.877907in}}%
\pgfusepath{clip}%
\pgfsetroundcap%
\pgfsetroundjoin%
\pgfsetlinewidth{0.803000pt}%
\definecolor{currentstroke}{rgb}{1.000000,1.000000,1.000000}%
\pgfsetstrokecolor{currentstroke}%
\pgfsetdash{}{0pt}%
\pgfpathmoveto{\pgfqpoint{3.606756in}{10.401163in}}%
\pgfpathlineto{\pgfqpoint{3.606756in}{11.279070in}}%
\pgfusepath{stroke}%
\end{pgfscope}%
\begin{pgfscope}%
\definecolor{textcolor}{rgb}{0.150000,0.150000,0.150000}%
\pgfsetstrokecolor{textcolor}%
\pgfsetfillcolor{textcolor}%
\pgftext[x=3.606756in,y=10.303941in,,top]{\color{textcolor}\rmfamily\fontsize{14.000000}{16.800000}\selectfont 10}%
\end{pgfscope}%
\begin{pgfscope}%
\pgfpathrectangle{\pgfqpoint{2.125000in}{10.401163in}}{\pgfqpoint{5.489583in}{0.877907in}}%
\pgfusepath{clip}%
\pgfsetroundcap%
\pgfsetroundjoin%
\pgfsetlinewidth{0.803000pt}%
\definecolor{currentstroke}{rgb}{1.000000,1.000000,1.000000}%
\pgfsetstrokecolor{currentstroke}%
\pgfsetdash{}{0pt}%
\pgfpathmoveto{\pgfqpoint{4.222871in}{10.401163in}}%
\pgfpathlineto{\pgfqpoint{4.222871in}{11.279070in}}%
\pgfusepath{stroke}%
\end{pgfscope}%
\begin{pgfscope}%
\definecolor{textcolor}{rgb}{0.150000,0.150000,0.150000}%
\pgfsetstrokecolor{textcolor}%
\pgfsetfillcolor{textcolor}%
\pgftext[x=4.222871in,y=10.303941in,,top]{\color{textcolor}\rmfamily\fontsize{14.000000}{16.800000}\selectfont 15}%
\end{pgfscope}%
\begin{pgfscope}%
\pgfpathrectangle{\pgfqpoint{2.125000in}{10.401163in}}{\pgfqpoint{5.489583in}{0.877907in}}%
\pgfusepath{clip}%
\pgfsetroundcap%
\pgfsetroundjoin%
\pgfsetlinewidth{0.803000pt}%
\definecolor{currentstroke}{rgb}{1.000000,1.000000,1.000000}%
\pgfsetstrokecolor{currentstroke}%
\pgfsetdash{}{0pt}%
\pgfpathmoveto{\pgfqpoint{4.838986in}{10.401163in}}%
\pgfpathlineto{\pgfqpoint{4.838986in}{11.279070in}}%
\pgfusepath{stroke}%
\end{pgfscope}%
\begin{pgfscope}%
\definecolor{textcolor}{rgb}{0.150000,0.150000,0.150000}%
\pgfsetstrokecolor{textcolor}%
\pgfsetfillcolor{textcolor}%
\pgftext[x=4.838986in,y=10.303941in,,top]{\color{textcolor}\rmfamily\fontsize{14.000000}{16.800000}\selectfont 20}%
\end{pgfscope}%
\begin{pgfscope}%
\pgfpathrectangle{\pgfqpoint{2.125000in}{10.401163in}}{\pgfqpoint{5.489583in}{0.877907in}}%
\pgfusepath{clip}%
\pgfsetroundcap%
\pgfsetroundjoin%
\pgfsetlinewidth{0.803000pt}%
\definecolor{currentstroke}{rgb}{1.000000,1.000000,1.000000}%
\pgfsetstrokecolor{currentstroke}%
\pgfsetdash{}{0pt}%
\pgfpathmoveto{\pgfqpoint{5.455101in}{10.401163in}}%
\pgfpathlineto{\pgfqpoint{5.455101in}{11.279070in}}%
\pgfusepath{stroke}%
\end{pgfscope}%
\begin{pgfscope}%
\definecolor{textcolor}{rgb}{0.150000,0.150000,0.150000}%
\pgfsetstrokecolor{textcolor}%
\pgfsetfillcolor{textcolor}%
\pgftext[x=5.455101in,y=10.303941in,,top]{\color{textcolor}\rmfamily\fontsize{14.000000}{16.800000}\selectfont 25}%
\end{pgfscope}%
\begin{pgfscope}%
\pgfpathrectangle{\pgfqpoint{2.125000in}{10.401163in}}{\pgfqpoint{5.489583in}{0.877907in}}%
\pgfusepath{clip}%
\pgfsetroundcap%
\pgfsetroundjoin%
\pgfsetlinewidth{0.803000pt}%
\definecolor{currentstroke}{rgb}{1.000000,1.000000,1.000000}%
\pgfsetstrokecolor{currentstroke}%
\pgfsetdash{}{0pt}%
\pgfpathmoveto{\pgfqpoint{6.071216in}{10.401163in}}%
\pgfpathlineto{\pgfqpoint{6.071216in}{11.279070in}}%
\pgfusepath{stroke}%
\end{pgfscope}%
\begin{pgfscope}%
\definecolor{textcolor}{rgb}{0.150000,0.150000,0.150000}%
\pgfsetstrokecolor{textcolor}%
\pgfsetfillcolor{textcolor}%
\pgftext[x=6.071216in,y=10.303941in,,top]{\color{textcolor}\rmfamily\fontsize{14.000000}{16.800000}\selectfont 30}%
\end{pgfscope}%
\begin{pgfscope}%
\pgfpathrectangle{\pgfqpoint{2.125000in}{10.401163in}}{\pgfqpoint{5.489583in}{0.877907in}}%
\pgfusepath{clip}%
\pgfsetroundcap%
\pgfsetroundjoin%
\pgfsetlinewidth{0.803000pt}%
\definecolor{currentstroke}{rgb}{1.000000,1.000000,1.000000}%
\pgfsetstrokecolor{currentstroke}%
\pgfsetdash{}{0pt}%
\pgfpathmoveto{\pgfqpoint{6.687330in}{10.401163in}}%
\pgfpathlineto{\pgfqpoint{6.687330in}{11.279070in}}%
\pgfusepath{stroke}%
\end{pgfscope}%
\begin{pgfscope}%
\definecolor{textcolor}{rgb}{0.150000,0.150000,0.150000}%
\pgfsetstrokecolor{textcolor}%
\pgfsetfillcolor{textcolor}%
\pgftext[x=6.687330in,y=10.303941in,,top]{\color{textcolor}\rmfamily\fontsize{14.000000}{16.800000}\selectfont 35}%
\end{pgfscope}%
\begin{pgfscope}%
\pgfpathrectangle{\pgfqpoint{2.125000in}{10.401163in}}{\pgfqpoint{5.489583in}{0.877907in}}%
\pgfusepath{clip}%
\pgfsetroundcap%
\pgfsetroundjoin%
\pgfsetlinewidth{0.803000pt}%
\definecolor{currentstroke}{rgb}{1.000000,1.000000,1.000000}%
\pgfsetstrokecolor{currentstroke}%
\pgfsetdash{}{0pt}%
\pgfpathmoveto{\pgfqpoint{7.303445in}{10.401163in}}%
\pgfpathlineto{\pgfqpoint{7.303445in}{11.279070in}}%
\pgfusepath{stroke}%
\end{pgfscope}%
\begin{pgfscope}%
\definecolor{textcolor}{rgb}{0.150000,0.150000,0.150000}%
\pgfsetstrokecolor{textcolor}%
\pgfsetfillcolor{textcolor}%
\pgftext[x=7.303445in,y=10.303941in,,top]{\color{textcolor}\rmfamily\fontsize{14.000000}{16.800000}\selectfont 40}%
\end{pgfscope}%
\begin{pgfscope}%
\pgfpathrectangle{\pgfqpoint{2.125000in}{10.401163in}}{\pgfqpoint{5.489583in}{0.877907in}}%
\pgfusepath{clip}%
\pgfsetroundcap%
\pgfsetroundjoin%
\pgfsetlinewidth{0.803000pt}%
\definecolor{currentstroke}{rgb}{1.000000,1.000000,1.000000}%
\pgfsetstrokecolor{currentstroke}%
\pgfsetdash{}{0pt}%
\pgfpathmoveto{\pgfqpoint{2.125000in}{10.679420in}}%
\pgfpathlineto{\pgfqpoint{7.614583in}{10.679420in}}%
\pgfusepath{stroke}%
\end{pgfscope}%
\begin{pgfscope}%
\definecolor{textcolor}{rgb}{0.150000,0.150000,0.150000}%
\pgfsetstrokecolor{textcolor}%
\pgfsetfillcolor{textcolor}%
\pgftext[x=1.904066in,y=10.605554in,left,base]{\color{textcolor}\rmfamily\fontsize{14.000000}{16.800000}\selectfont 0}%
\end{pgfscope}%
\begin{pgfscope}%
\pgfpathrectangle{\pgfqpoint{2.125000in}{10.401163in}}{\pgfqpoint{5.489583in}{0.877907in}}%
\pgfusepath{clip}%
\pgfsetroundcap%
\pgfsetroundjoin%
\pgfsetlinewidth{0.803000pt}%
\definecolor{currentstroke}{rgb}{1.000000,1.000000,1.000000}%
\pgfsetstrokecolor{currentstroke}%
\pgfsetdash{}{0pt}%
\pgfpathmoveto{\pgfqpoint{2.125000in}{11.239165in}}%
\pgfpathlineto{\pgfqpoint{7.614583in}{11.239165in}}%
\pgfusepath{stroke}%
\end{pgfscope}%
\begin{pgfscope}%
\definecolor{textcolor}{rgb}{0.150000,0.150000,0.150000}%
\pgfsetstrokecolor{textcolor}%
\pgfsetfillcolor{textcolor}%
\pgftext[x=1.904066in,y=11.165299in,left,base]{\color{textcolor}\rmfamily\fontsize{14.000000}{16.800000}\selectfont 1}%
\end{pgfscope}%
\begin{pgfscope}%
\pgfpathrectangle{\pgfqpoint{2.125000in}{10.401163in}}{\pgfqpoint{5.489583in}{0.877907in}}%
\pgfusepath{clip}%
\pgfsetbuttcap%
\pgfsetroundjoin%
\definecolor{currentfill}{rgb}{0.121569,0.466667,0.705882}%
\pgfsetfillcolor{currentfill}%
\pgfsetfillopacity{0.250000}%
\pgfsetlinewidth{1.003750pt}%
\definecolor{currentstroke}{rgb}{1.000000,1.000000,1.000000}%
\pgfsetstrokecolor{currentstroke}%
\pgfsetstrokeopacity{0.250000}%
\pgfsetdash{}{0pt}%
\pgfpathmoveto{\pgfqpoint{2.436138in}{10.707662in}}%
\pgfpathlineto{\pgfqpoint{2.436138in}{10.651178in}}%
\pgfpathlineto{\pgfqpoint{2.620972in}{10.630586in}}%
\pgfpathlineto{\pgfqpoint{2.744195in}{10.616461in}}%
\pgfpathlineto{\pgfqpoint{2.867418in}{10.605024in}}%
\pgfpathlineto{\pgfqpoint{2.990641in}{10.595174in}}%
\pgfpathlineto{\pgfqpoint{3.113864in}{10.586404in}}%
\pgfpathlineto{\pgfqpoint{3.237087in}{10.578434in}}%
\pgfpathlineto{\pgfqpoint{3.360310in}{10.571087in}}%
\pgfpathlineto{\pgfqpoint{3.483533in}{10.564244in}}%
\pgfpathlineto{\pgfqpoint{3.606756in}{10.557819in}}%
\pgfpathlineto{\pgfqpoint{3.729979in}{10.551750in}}%
\pgfpathlineto{\pgfqpoint{3.853202in}{10.545987in}}%
\pgfpathlineto{\pgfqpoint{3.976425in}{10.540492in}}%
\pgfpathlineto{\pgfqpoint{4.099648in}{10.535235in}}%
\pgfpathlineto{\pgfqpoint{4.222871in}{10.530190in}}%
\pgfpathlineto{\pgfqpoint{4.346094in}{10.525336in}}%
\pgfpathlineto{\pgfqpoint{4.469317in}{10.520655in}}%
\pgfpathlineto{\pgfqpoint{4.592540in}{10.516132in}}%
\pgfpathlineto{\pgfqpoint{4.715763in}{10.511753in}}%
\pgfpathlineto{\pgfqpoint{4.838986in}{10.507508in}}%
\pgfpathlineto{\pgfqpoint{4.962209in}{10.503386in}}%
\pgfpathlineto{\pgfqpoint{5.085432in}{10.499379in}}%
\pgfpathlineto{\pgfqpoint{5.208655in}{10.495481in}}%
\pgfpathlineto{\pgfqpoint{5.331878in}{10.491684in}}%
\pgfpathlineto{\pgfqpoint{5.455101in}{10.487982in}}%
\pgfpathlineto{\pgfqpoint{5.578324in}{10.484369in}}%
\pgfpathlineto{\pgfqpoint{5.701547in}{10.480840in}}%
\pgfpathlineto{\pgfqpoint{5.824770in}{10.477391in}}%
\pgfpathlineto{\pgfqpoint{5.947993in}{10.474019in}}%
\pgfpathlineto{\pgfqpoint{6.071216in}{10.470718in}}%
\pgfpathlineto{\pgfqpoint{6.194439in}{10.467487in}}%
\pgfpathlineto{\pgfqpoint{6.317662in}{10.464322in}}%
\pgfpathlineto{\pgfqpoint{6.440885in}{10.461220in}}%
\pgfpathlineto{\pgfqpoint{6.564108in}{10.458179in}}%
\pgfpathlineto{\pgfqpoint{6.687330in}{10.455195in}}%
\pgfpathlineto{\pgfqpoint{6.810553in}{10.452266in}}%
\pgfpathlineto{\pgfqpoint{6.933776in}{10.449392in}}%
\pgfpathlineto{\pgfqpoint{7.056999in}{10.446569in}}%
\pgfpathlineto{\pgfqpoint{7.180222in}{10.443795in}}%
\pgfpathlineto{\pgfqpoint{7.365057in}{10.441068in}}%
\pgfpathlineto{\pgfqpoint{7.365057in}{10.917773in}}%
\pgfpathlineto{\pgfqpoint{7.365057in}{10.917773in}}%
\pgfpathlineto{\pgfqpoint{7.180222in}{10.915046in}}%
\pgfpathlineto{\pgfqpoint{7.056999in}{10.912272in}}%
\pgfpathlineto{\pgfqpoint{6.933776in}{10.909449in}}%
\pgfpathlineto{\pgfqpoint{6.810553in}{10.906574in}}%
\pgfpathlineto{\pgfqpoint{6.687330in}{10.903646in}}%
\pgfpathlineto{\pgfqpoint{6.564108in}{10.900662in}}%
\pgfpathlineto{\pgfqpoint{6.440885in}{10.897621in}}%
\pgfpathlineto{\pgfqpoint{6.317662in}{10.894519in}}%
\pgfpathlineto{\pgfqpoint{6.194439in}{10.891354in}}%
\pgfpathlineto{\pgfqpoint{6.071216in}{10.888122in}}%
\pgfpathlineto{\pgfqpoint{5.947993in}{10.884822in}}%
\pgfpathlineto{\pgfqpoint{5.824770in}{10.881449in}}%
\pgfpathlineto{\pgfqpoint{5.701547in}{10.878001in}}%
\pgfpathlineto{\pgfqpoint{5.578324in}{10.874472in}}%
\pgfpathlineto{\pgfqpoint{5.455101in}{10.870859in}}%
\pgfpathlineto{\pgfqpoint{5.331878in}{10.867157in}}%
\pgfpathlineto{\pgfqpoint{5.208655in}{10.863360in}}%
\pgfpathlineto{\pgfqpoint{5.085432in}{10.859461in}}%
\pgfpathlineto{\pgfqpoint{4.962209in}{10.855455in}}%
\pgfpathlineto{\pgfqpoint{4.838986in}{10.851333in}}%
\pgfpathlineto{\pgfqpoint{4.715763in}{10.847088in}}%
\pgfpathlineto{\pgfqpoint{4.592540in}{10.842709in}}%
\pgfpathlineto{\pgfqpoint{4.469317in}{10.838185in}}%
\pgfpathlineto{\pgfqpoint{4.346094in}{10.833504in}}%
\pgfpathlineto{\pgfqpoint{4.222871in}{10.828650in}}%
\pgfpathlineto{\pgfqpoint{4.099648in}{10.823606in}}%
\pgfpathlineto{\pgfqpoint{3.976425in}{10.818349in}}%
\pgfpathlineto{\pgfqpoint{3.853202in}{10.812854in}}%
\pgfpathlineto{\pgfqpoint{3.729979in}{10.807091in}}%
\pgfpathlineto{\pgfqpoint{3.606756in}{10.801022in}}%
\pgfpathlineto{\pgfqpoint{3.483533in}{10.794597in}}%
\pgfpathlineto{\pgfqpoint{3.360310in}{10.787753in}}%
\pgfpathlineto{\pgfqpoint{3.237087in}{10.780406in}}%
\pgfpathlineto{\pgfqpoint{3.113864in}{10.772436in}}%
\pgfpathlineto{\pgfqpoint{2.990641in}{10.763667in}}%
\pgfpathlineto{\pgfqpoint{2.867418in}{10.753817in}}%
\pgfpathlineto{\pgfqpoint{2.744195in}{10.742380in}}%
\pgfpathlineto{\pgfqpoint{2.620972in}{10.728254in}}%
\pgfpathlineto{\pgfqpoint{2.436138in}{10.707662in}}%
\pgfpathclose%
\pgfusepath{stroke,fill}%
\end{pgfscope}%
\begin{pgfscope}%
\pgfpathrectangle{\pgfqpoint{2.125000in}{10.401163in}}{\pgfqpoint{5.489583in}{0.877907in}}%
\pgfusepath{clip}%
\pgfsetbuttcap%
\pgfsetroundjoin%
\pgfsetlinewidth{1.505625pt}%
\definecolor{currentstroke}{rgb}{0.000000,0.000000,0.000000}%
\pgfsetstrokecolor{currentstroke}%
\pgfsetdash{}{0pt}%
\pgfpathmoveto{\pgfqpoint{2.374527in}{10.679420in}}%
\pgfpathlineto{\pgfqpoint{2.374527in}{11.239165in}}%
\pgfusepath{stroke}%
\end{pgfscope}%
\begin{pgfscope}%
\pgfpathrectangle{\pgfqpoint{2.125000in}{10.401163in}}{\pgfqpoint{5.489583in}{0.877907in}}%
\pgfusepath{clip}%
\pgfsetbuttcap%
\pgfsetroundjoin%
\pgfsetlinewidth{1.505625pt}%
\definecolor{currentstroke}{rgb}{0.000000,0.000000,0.000000}%
\pgfsetstrokecolor{currentstroke}%
\pgfsetdash{}{0pt}%
\pgfpathmoveto{\pgfqpoint{2.497749in}{10.679420in}}%
\pgfpathlineto{\pgfqpoint{2.497749in}{11.237753in}}%
\pgfusepath{stroke}%
\end{pgfscope}%
\begin{pgfscope}%
\pgfpathrectangle{\pgfqpoint{2.125000in}{10.401163in}}{\pgfqpoint{5.489583in}{0.877907in}}%
\pgfusepath{clip}%
\pgfsetbuttcap%
\pgfsetroundjoin%
\pgfsetlinewidth{1.505625pt}%
\definecolor{currentstroke}{rgb}{0.000000,0.000000,0.000000}%
\pgfsetstrokecolor{currentstroke}%
\pgfsetdash{}{0pt}%
\pgfpathmoveto{\pgfqpoint{2.620972in}{10.679420in}}%
\pgfpathlineto{\pgfqpoint{2.620972in}{11.236331in}}%
\pgfusepath{stroke}%
\end{pgfscope}%
\begin{pgfscope}%
\pgfpathrectangle{\pgfqpoint{2.125000in}{10.401163in}}{\pgfqpoint{5.489583in}{0.877907in}}%
\pgfusepath{clip}%
\pgfsetbuttcap%
\pgfsetroundjoin%
\pgfsetlinewidth{1.505625pt}%
\definecolor{currentstroke}{rgb}{0.000000,0.000000,0.000000}%
\pgfsetstrokecolor{currentstroke}%
\pgfsetdash{}{0pt}%
\pgfpathmoveto{\pgfqpoint{2.744195in}{10.679420in}}%
\pgfpathlineto{\pgfqpoint{2.744195in}{11.234897in}}%
\pgfusepath{stroke}%
\end{pgfscope}%
\begin{pgfscope}%
\pgfpathrectangle{\pgfqpoint{2.125000in}{10.401163in}}{\pgfqpoint{5.489583in}{0.877907in}}%
\pgfusepath{clip}%
\pgfsetbuttcap%
\pgfsetroundjoin%
\pgfsetlinewidth{1.505625pt}%
\definecolor{currentstroke}{rgb}{0.000000,0.000000,0.000000}%
\pgfsetstrokecolor{currentstroke}%
\pgfsetdash{}{0pt}%
\pgfpathmoveto{\pgfqpoint{2.867418in}{10.679420in}}%
\pgfpathlineto{\pgfqpoint{2.867418in}{11.233421in}}%
\pgfusepath{stroke}%
\end{pgfscope}%
\begin{pgfscope}%
\pgfpathrectangle{\pgfqpoint{2.125000in}{10.401163in}}{\pgfqpoint{5.489583in}{0.877907in}}%
\pgfusepath{clip}%
\pgfsetbuttcap%
\pgfsetroundjoin%
\pgfsetlinewidth{1.505625pt}%
\definecolor{currentstroke}{rgb}{0.000000,0.000000,0.000000}%
\pgfsetstrokecolor{currentstroke}%
\pgfsetdash{}{0pt}%
\pgfpathmoveto{\pgfqpoint{2.990641in}{10.679420in}}%
\pgfpathlineto{\pgfqpoint{2.990641in}{11.231974in}}%
\pgfusepath{stroke}%
\end{pgfscope}%
\begin{pgfscope}%
\pgfpathrectangle{\pgfqpoint{2.125000in}{10.401163in}}{\pgfqpoint{5.489583in}{0.877907in}}%
\pgfusepath{clip}%
\pgfsetbuttcap%
\pgfsetroundjoin%
\pgfsetlinewidth{1.505625pt}%
\definecolor{currentstroke}{rgb}{0.000000,0.000000,0.000000}%
\pgfsetstrokecolor{currentstroke}%
\pgfsetdash{}{0pt}%
\pgfpathmoveto{\pgfqpoint{3.113864in}{10.679420in}}%
\pgfpathlineto{\pgfqpoint{3.113864in}{11.230498in}}%
\pgfusepath{stroke}%
\end{pgfscope}%
\begin{pgfscope}%
\pgfpathrectangle{\pgfqpoint{2.125000in}{10.401163in}}{\pgfqpoint{5.489583in}{0.877907in}}%
\pgfusepath{clip}%
\pgfsetbuttcap%
\pgfsetroundjoin%
\pgfsetlinewidth{1.505625pt}%
\definecolor{currentstroke}{rgb}{0.000000,0.000000,0.000000}%
\pgfsetstrokecolor{currentstroke}%
\pgfsetdash{}{0pt}%
\pgfpathmoveto{\pgfqpoint{3.237087in}{10.679420in}}%
\pgfpathlineto{\pgfqpoint{3.237087in}{11.229019in}}%
\pgfusepath{stroke}%
\end{pgfscope}%
\begin{pgfscope}%
\pgfpathrectangle{\pgfqpoint{2.125000in}{10.401163in}}{\pgfqpoint{5.489583in}{0.877907in}}%
\pgfusepath{clip}%
\pgfsetbuttcap%
\pgfsetroundjoin%
\pgfsetlinewidth{1.505625pt}%
\definecolor{currentstroke}{rgb}{0.000000,0.000000,0.000000}%
\pgfsetstrokecolor{currentstroke}%
\pgfsetdash{}{0pt}%
\pgfpathmoveto{\pgfqpoint{3.360310in}{10.679420in}}%
\pgfpathlineto{\pgfqpoint{3.360310in}{11.227536in}}%
\pgfusepath{stroke}%
\end{pgfscope}%
\begin{pgfscope}%
\pgfpathrectangle{\pgfqpoint{2.125000in}{10.401163in}}{\pgfqpoint{5.489583in}{0.877907in}}%
\pgfusepath{clip}%
\pgfsetbuttcap%
\pgfsetroundjoin%
\pgfsetlinewidth{1.505625pt}%
\definecolor{currentstroke}{rgb}{0.000000,0.000000,0.000000}%
\pgfsetstrokecolor{currentstroke}%
\pgfsetdash{}{0pt}%
\pgfpathmoveto{\pgfqpoint{3.483533in}{10.679420in}}%
\pgfpathlineto{\pgfqpoint{3.483533in}{11.226034in}}%
\pgfusepath{stroke}%
\end{pgfscope}%
\begin{pgfscope}%
\pgfpathrectangle{\pgfqpoint{2.125000in}{10.401163in}}{\pgfqpoint{5.489583in}{0.877907in}}%
\pgfusepath{clip}%
\pgfsetbuttcap%
\pgfsetroundjoin%
\pgfsetlinewidth{1.505625pt}%
\definecolor{currentstroke}{rgb}{0.000000,0.000000,0.000000}%
\pgfsetstrokecolor{currentstroke}%
\pgfsetdash{}{0pt}%
\pgfpathmoveto{\pgfqpoint{3.606756in}{10.679420in}}%
\pgfpathlineto{\pgfqpoint{3.606756in}{11.224535in}}%
\pgfusepath{stroke}%
\end{pgfscope}%
\begin{pgfscope}%
\pgfpathrectangle{\pgfqpoint{2.125000in}{10.401163in}}{\pgfqpoint{5.489583in}{0.877907in}}%
\pgfusepath{clip}%
\pgfsetbuttcap%
\pgfsetroundjoin%
\pgfsetlinewidth{1.505625pt}%
\definecolor{currentstroke}{rgb}{0.000000,0.000000,0.000000}%
\pgfsetstrokecolor{currentstroke}%
\pgfsetdash{}{0pt}%
\pgfpathmoveto{\pgfqpoint{3.729979in}{10.679420in}}%
\pgfpathlineto{\pgfqpoint{3.729979in}{11.223062in}}%
\pgfusepath{stroke}%
\end{pgfscope}%
\begin{pgfscope}%
\pgfpathrectangle{\pgfqpoint{2.125000in}{10.401163in}}{\pgfqpoint{5.489583in}{0.877907in}}%
\pgfusepath{clip}%
\pgfsetbuttcap%
\pgfsetroundjoin%
\pgfsetlinewidth{1.505625pt}%
\definecolor{currentstroke}{rgb}{0.000000,0.000000,0.000000}%
\pgfsetstrokecolor{currentstroke}%
\pgfsetdash{}{0pt}%
\pgfpathmoveto{\pgfqpoint{3.853202in}{10.679420in}}%
\pgfpathlineto{\pgfqpoint{3.853202in}{11.221579in}}%
\pgfusepath{stroke}%
\end{pgfscope}%
\begin{pgfscope}%
\pgfpathrectangle{\pgfqpoint{2.125000in}{10.401163in}}{\pgfqpoint{5.489583in}{0.877907in}}%
\pgfusepath{clip}%
\pgfsetbuttcap%
\pgfsetroundjoin%
\pgfsetlinewidth{1.505625pt}%
\definecolor{currentstroke}{rgb}{0.000000,0.000000,0.000000}%
\pgfsetstrokecolor{currentstroke}%
\pgfsetdash{}{0pt}%
\pgfpathmoveto{\pgfqpoint{3.976425in}{10.679420in}}%
\pgfpathlineto{\pgfqpoint{3.976425in}{11.220089in}}%
\pgfusepath{stroke}%
\end{pgfscope}%
\begin{pgfscope}%
\pgfpathrectangle{\pgfqpoint{2.125000in}{10.401163in}}{\pgfqpoint{5.489583in}{0.877907in}}%
\pgfusepath{clip}%
\pgfsetbuttcap%
\pgfsetroundjoin%
\pgfsetlinewidth{1.505625pt}%
\definecolor{currentstroke}{rgb}{0.000000,0.000000,0.000000}%
\pgfsetstrokecolor{currentstroke}%
\pgfsetdash{}{0pt}%
\pgfpathmoveto{\pgfqpoint{4.099648in}{10.679420in}}%
\pgfpathlineto{\pgfqpoint{4.099648in}{11.218607in}}%
\pgfusepath{stroke}%
\end{pgfscope}%
\begin{pgfscope}%
\pgfpathrectangle{\pgfqpoint{2.125000in}{10.401163in}}{\pgfqpoint{5.489583in}{0.877907in}}%
\pgfusepath{clip}%
\pgfsetbuttcap%
\pgfsetroundjoin%
\pgfsetlinewidth{1.505625pt}%
\definecolor{currentstroke}{rgb}{0.000000,0.000000,0.000000}%
\pgfsetstrokecolor{currentstroke}%
\pgfsetdash{}{0pt}%
\pgfpathmoveto{\pgfqpoint{4.222871in}{10.679420in}}%
\pgfpathlineto{\pgfqpoint{4.222871in}{11.217158in}}%
\pgfusepath{stroke}%
\end{pgfscope}%
\begin{pgfscope}%
\pgfpathrectangle{\pgfqpoint{2.125000in}{10.401163in}}{\pgfqpoint{5.489583in}{0.877907in}}%
\pgfusepath{clip}%
\pgfsetbuttcap%
\pgfsetroundjoin%
\pgfsetlinewidth{1.505625pt}%
\definecolor{currentstroke}{rgb}{0.000000,0.000000,0.000000}%
\pgfsetstrokecolor{currentstroke}%
\pgfsetdash{}{0pt}%
\pgfpathmoveto{\pgfqpoint{4.346094in}{10.679420in}}%
\pgfpathlineto{\pgfqpoint{4.346094in}{11.215743in}}%
\pgfusepath{stroke}%
\end{pgfscope}%
\begin{pgfscope}%
\pgfpathrectangle{\pgfqpoint{2.125000in}{10.401163in}}{\pgfqpoint{5.489583in}{0.877907in}}%
\pgfusepath{clip}%
\pgfsetbuttcap%
\pgfsetroundjoin%
\pgfsetlinewidth{1.505625pt}%
\definecolor{currentstroke}{rgb}{0.000000,0.000000,0.000000}%
\pgfsetstrokecolor{currentstroke}%
\pgfsetdash{}{0pt}%
\pgfpathmoveto{\pgfqpoint{4.469317in}{10.679420in}}%
\pgfpathlineto{\pgfqpoint{4.469317in}{11.214345in}}%
\pgfusepath{stroke}%
\end{pgfscope}%
\begin{pgfscope}%
\pgfpathrectangle{\pgfqpoint{2.125000in}{10.401163in}}{\pgfqpoint{5.489583in}{0.877907in}}%
\pgfusepath{clip}%
\pgfsetbuttcap%
\pgfsetroundjoin%
\pgfsetlinewidth{1.505625pt}%
\definecolor{currentstroke}{rgb}{0.000000,0.000000,0.000000}%
\pgfsetstrokecolor{currentstroke}%
\pgfsetdash{}{0pt}%
\pgfpathmoveto{\pgfqpoint{4.592540in}{10.679420in}}%
\pgfpathlineto{\pgfqpoint{4.592540in}{11.212926in}}%
\pgfusepath{stroke}%
\end{pgfscope}%
\begin{pgfscope}%
\pgfpathrectangle{\pgfqpoint{2.125000in}{10.401163in}}{\pgfqpoint{5.489583in}{0.877907in}}%
\pgfusepath{clip}%
\pgfsetbuttcap%
\pgfsetroundjoin%
\pgfsetlinewidth{1.505625pt}%
\definecolor{currentstroke}{rgb}{0.000000,0.000000,0.000000}%
\pgfsetstrokecolor{currentstroke}%
\pgfsetdash{}{0pt}%
\pgfpathmoveto{\pgfqpoint{4.715763in}{10.679420in}}%
\pgfpathlineto{\pgfqpoint{4.715763in}{11.211537in}}%
\pgfusepath{stroke}%
\end{pgfscope}%
\begin{pgfscope}%
\pgfpathrectangle{\pgfqpoint{2.125000in}{10.401163in}}{\pgfqpoint{5.489583in}{0.877907in}}%
\pgfusepath{clip}%
\pgfsetbuttcap%
\pgfsetroundjoin%
\pgfsetlinewidth{1.505625pt}%
\definecolor{currentstroke}{rgb}{0.000000,0.000000,0.000000}%
\pgfsetstrokecolor{currentstroke}%
\pgfsetdash{}{0pt}%
\pgfpathmoveto{\pgfqpoint{4.838986in}{10.679420in}}%
\pgfpathlineto{\pgfqpoint{4.838986in}{11.210151in}}%
\pgfusepath{stroke}%
\end{pgfscope}%
\begin{pgfscope}%
\pgfpathrectangle{\pgfqpoint{2.125000in}{10.401163in}}{\pgfqpoint{5.489583in}{0.877907in}}%
\pgfusepath{clip}%
\pgfsetbuttcap%
\pgfsetroundjoin%
\pgfsetlinewidth{1.505625pt}%
\definecolor{currentstroke}{rgb}{0.000000,0.000000,0.000000}%
\pgfsetstrokecolor{currentstroke}%
\pgfsetdash{}{0pt}%
\pgfpathmoveto{\pgfqpoint{4.962209in}{10.679420in}}%
\pgfpathlineto{\pgfqpoint{4.962209in}{11.208759in}}%
\pgfusepath{stroke}%
\end{pgfscope}%
\begin{pgfscope}%
\pgfpathrectangle{\pgfqpoint{2.125000in}{10.401163in}}{\pgfqpoint{5.489583in}{0.877907in}}%
\pgfusepath{clip}%
\pgfsetbuttcap%
\pgfsetroundjoin%
\pgfsetlinewidth{1.505625pt}%
\definecolor{currentstroke}{rgb}{0.000000,0.000000,0.000000}%
\pgfsetstrokecolor{currentstroke}%
\pgfsetdash{}{0pt}%
\pgfpathmoveto{\pgfqpoint{5.085432in}{10.679420in}}%
\pgfpathlineto{\pgfqpoint{5.085432in}{11.207344in}}%
\pgfusepath{stroke}%
\end{pgfscope}%
\begin{pgfscope}%
\pgfpathrectangle{\pgfqpoint{2.125000in}{10.401163in}}{\pgfqpoint{5.489583in}{0.877907in}}%
\pgfusepath{clip}%
\pgfsetbuttcap%
\pgfsetroundjoin%
\pgfsetlinewidth{1.505625pt}%
\definecolor{currentstroke}{rgb}{0.000000,0.000000,0.000000}%
\pgfsetstrokecolor{currentstroke}%
\pgfsetdash{}{0pt}%
\pgfpathmoveto{\pgfqpoint{5.208655in}{10.679420in}}%
\pgfpathlineto{\pgfqpoint{5.208655in}{11.205919in}}%
\pgfusepath{stroke}%
\end{pgfscope}%
\begin{pgfscope}%
\pgfpathrectangle{\pgfqpoint{2.125000in}{10.401163in}}{\pgfqpoint{5.489583in}{0.877907in}}%
\pgfusepath{clip}%
\pgfsetbuttcap%
\pgfsetroundjoin%
\pgfsetlinewidth{1.505625pt}%
\definecolor{currentstroke}{rgb}{0.000000,0.000000,0.000000}%
\pgfsetstrokecolor{currentstroke}%
\pgfsetdash{}{0pt}%
\pgfpathmoveto{\pgfqpoint{5.331878in}{10.679420in}}%
\pgfpathlineto{\pgfqpoint{5.331878in}{11.204502in}}%
\pgfusepath{stroke}%
\end{pgfscope}%
\begin{pgfscope}%
\pgfpathrectangle{\pgfqpoint{2.125000in}{10.401163in}}{\pgfqpoint{5.489583in}{0.877907in}}%
\pgfusepath{clip}%
\pgfsetbuttcap%
\pgfsetroundjoin%
\pgfsetlinewidth{1.505625pt}%
\definecolor{currentstroke}{rgb}{0.000000,0.000000,0.000000}%
\pgfsetstrokecolor{currentstroke}%
\pgfsetdash{}{0pt}%
\pgfpathmoveto{\pgfqpoint{5.455101in}{10.679420in}}%
\pgfpathlineto{\pgfqpoint{5.455101in}{11.203108in}}%
\pgfusepath{stroke}%
\end{pgfscope}%
\begin{pgfscope}%
\pgfpathrectangle{\pgfqpoint{2.125000in}{10.401163in}}{\pgfqpoint{5.489583in}{0.877907in}}%
\pgfusepath{clip}%
\pgfsetbuttcap%
\pgfsetroundjoin%
\pgfsetlinewidth{1.505625pt}%
\definecolor{currentstroke}{rgb}{0.000000,0.000000,0.000000}%
\pgfsetstrokecolor{currentstroke}%
\pgfsetdash{}{0pt}%
\pgfpathmoveto{\pgfqpoint{5.578324in}{10.679420in}}%
\pgfpathlineto{\pgfqpoint{5.578324in}{11.201736in}}%
\pgfusepath{stroke}%
\end{pgfscope}%
\begin{pgfscope}%
\pgfpathrectangle{\pgfqpoint{2.125000in}{10.401163in}}{\pgfqpoint{5.489583in}{0.877907in}}%
\pgfusepath{clip}%
\pgfsetbuttcap%
\pgfsetroundjoin%
\pgfsetlinewidth{1.505625pt}%
\definecolor{currentstroke}{rgb}{0.000000,0.000000,0.000000}%
\pgfsetstrokecolor{currentstroke}%
\pgfsetdash{}{0pt}%
\pgfpathmoveto{\pgfqpoint{5.701547in}{10.679420in}}%
\pgfpathlineto{\pgfqpoint{5.701547in}{11.200343in}}%
\pgfusepath{stroke}%
\end{pgfscope}%
\begin{pgfscope}%
\pgfpathrectangle{\pgfqpoint{2.125000in}{10.401163in}}{\pgfqpoint{5.489583in}{0.877907in}}%
\pgfusepath{clip}%
\pgfsetbuttcap%
\pgfsetroundjoin%
\pgfsetlinewidth{1.505625pt}%
\definecolor{currentstroke}{rgb}{0.000000,0.000000,0.000000}%
\pgfsetstrokecolor{currentstroke}%
\pgfsetdash{}{0pt}%
\pgfpathmoveto{\pgfqpoint{5.824770in}{10.679420in}}%
\pgfpathlineto{\pgfqpoint{5.824770in}{11.198933in}}%
\pgfusepath{stroke}%
\end{pgfscope}%
\begin{pgfscope}%
\pgfpathrectangle{\pgfqpoint{2.125000in}{10.401163in}}{\pgfqpoint{5.489583in}{0.877907in}}%
\pgfusepath{clip}%
\pgfsetbuttcap%
\pgfsetroundjoin%
\pgfsetlinewidth{1.505625pt}%
\definecolor{currentstroke}{rgb}{0.000000,0.000000,0.000000}%
\pgfsetstrokecolor{currentstroke}%
\pgfsetdash{}{0pt}%
\pgfpathmoveto{\pgfqpoint{5.947993in}{10.679420in}}%
\pgfpathlineto{\pgfqpoint{5.947993in}{11.197525in}}%
\pgfusepath{stroke}%
\end{pgfscope}%
\begin{pgfscope}%
\pgfpathrectangle{\pgfqpoint{2.125000in}{10.401163in}}{\pgfqpoint{5.489583in}{0.877907in}}%
\pgfusepath{clip}%
\pgfsetbuttcap%
\pgfsetroundjoin%
\pgfsetlinewidth{1.505625pt}%
\definecolor{currentstroke}{rgb}{0.000000,0.000000,0.000000}%
\pgfsetstrokecolor{currentstroke}%
\pgfsetdash{}{0pt}%
\pgfpathmoveto{\pgfqpoint{6.071216in}{10.679420in}}%
\pgfpathlineto{\pgfqpoint{6.071216in}{11.196102in}}%
\pgfusepath{stroke}%
\end{pgfscope}%
\begin{pgfscope}%
\pgfpathrectangle{\pgfqpoint{2.125000in}{10.401163in}}{\pgfqpoint{5.489583in}{0.877907in}}%
\pgfusepath{clip}%
\pgfsetbuttcap%
\pgfsetroundjoin%
\pgfsetlinewidth{1.505625pt}%
\definecolor{currentstroke}{rgb}{0.000000,0.000000,0.000000}%
\pgfsetstrokecolor{currentstroke}%
\pgfsetdash{}{0pt}%
\pgfpathmoveto{\pgfqpoint{6.194439in}{10.679420in}}%
\pgfpathlineto{\pgfqpoint{6.194439in}{11.194646in}}%
\pgfusepath{stroke}%
\end{pgfscope}%
\begin{pgfscope}%
\pgfpathrectangle{\pgfqpoint{2.125000in}{10.401163in}}{\pgfqpoint{5.489583in}{0.877907in}}%
\pgfusepath{clip}%
\pgfsetbuttcap%
\pgfsetroundjoin%
\pgfsetlinewidth{1.505625pt}%
\definecolor{currentstroke}{rgb}{0.000000,0.000000,0.000000}%
\pgfsetstrokecolor{currentstroke}%
\pgfsetdash{}{0pt}%
\pgfpathmoveto{\pgfqpoint{6.317662in}{10.679420in}}%
\pgfpathlineto{\pgfqpoint{6.317662in}{11.193211in}}%
\pgfusepath{stroke}%
\end{pgfscope}%
\begin{pgfscope}%
\pgfpathrectangle{\pgfqpoint{2.125000in}{10.401163in}}{\pgfqpoint{5.489583in}{0.877907in}}%
\pgfusepath{clip}%
\pgfsetbuttcap%
\pgfsetroundjoin%
\pgfsetlinewidth{1.505625pt}%
\definecolor{currentstroke}{rgb}{0.000000,0.000000,0.000000}%
\pgfsetstrokecolor{currentstroke}%
\pgfsetdash{}{0pt}%
\pgfpathmoveto{\pgfqpoint{6.440885in}{10.679420in}}%
\pgfpathlineto{\pgfqpoint{6.440885in}{11.191786in}}%
\pgfusepath{stroke}%
\end{pgfscope}%
\begin{pgfscope}%
\pgfpathrectangle{\pgfqpoint{2.125000in}{10.401163in}}{\pgfqpoint{5.489583in}{0.877907in}}%
\pgfusepath{clip}%
\pgfsetbuttcap%
\pgfsetroundjoin%
\pgfsetlinewidth{1.505625pt}%
\definecolor{currentstroke}{rgb}{0.000000,0.000000,0.000000}%
\pgfsetstrokecolor{currentstroke}%
\pgfsetdash{}{0pt}%
\pgfpathmoveto{\pgfqpoint{6.564108in}{10.679420in}}%
\pgfpathlineto{\pgfqpoint{6.564108in}{11.190375in}}%
\pgfusepath{stroke}%
\end{pgfscope}%
\begin{pgfscope}%
\pgfpathrectangle{\pgfqpoint{2.125000in}{10.401163in}}{\pgfqpoint{5.489583in}{0.877907in}}%
\pgfusepath{clip}%
\pgfsetbuttcap%
\pgfsetroundjoin%
\pgfsetlinewidth{1.505625pt}%
\definecolor{currentstroke}{rgb}{0.000000,0.000000,0.000000}%
\pgfsetstrokecolor{currentstroke}%
\pgfsetdash{}{0pt}%
\pgfpathmoveto{\pgfqpoint{6.687330in}{10.679420in}}%
\pgfpathlineto{\pgfqpoint{6.687330in}{11.188930in}}%
\pgfusepath{stroke}%
\end{pgfscope}%
\begin{pgfscope}%
\pgfpathrectangle{\pgfqpoint{2.125000in}{10.401163in}}{\pgfqpoint{5.489583in}{0.877907in}}%
\pgfusepath{clip}%
\pgfsetbuttcap%
\pgfsetroundjoin%
\pgfsetlinewidth{1.505625pt}%
\definecolor{currentstroke}{rgb}{0.000000,0.000000,0.000000}%
\pgfsetstrokecolor{currentstroke}%
\pgfsetdash{}{0pt}%
\pgfpathmoveto{\pgfqpoint{6.810553in}{10.679420in}}%
\pgfpathlineto{\pgfqpoint{6.810553in}{11.187480in}}%
\pgfusepath{stroke}%
\end{pgfscope}%
\begin{pgfscope}%
\pgfpathrectangle{\pgfqpoint{2.125000in}{10.401163in}}{\pgfqpoint{5.489583in}{0.877907in}}%
\pgfusepath{clip}%
\pgfsetbuttcap%
\pgfsetroundjoin%
\pgfsetlinewidth{1.505625pt}%
\definecolor{currentstroke}{rgb}{0.000000,0.000000,0.000000}%
\pgfsetstrokecolor{currentstroke}%
\pgfsetdash{}{0pt}%
\pgfpathmoveto{\pgfqpoint{6.933776in}{10.679420in}}%
\pgfpathlineto{\pgfqpoint{6.933776in}{11.186056in}}%
\pgfusepath{stroke}%
\end{pgfscope}%
\begin{pgfscope}%
\pgfpathrectangle{\pgfqpoint{2.125000in}{10.401163in}}{\pgfqpoint{5.489583in}{0.877907in}}%
\pgfusepath{clip}%
\pgfsetbuttcap%
\pgfsetroundjoin%
\pgfsetlinewidth{1.505625pt}%
\definecolor{currentstroke}{rgb}{0.000000,0.000000,0.000000}%
\pgfsetstrokecolor{currentstroke}%
\pgfsetdash{}{0pt}%
\pgfpathmoveto{\pgfqpoint{7.056999in}{10.679420in}}%
\pgfpathlineto{\pgfqpoint{7.056999in}{11.184643in}}%
\pgfusepath{stroke}%
\end{pgfscope}%
\begin{pgfscope}%
\pgfpathrectangle{\pgfqpoint{2.125000in}{10.401163in}}{\pgfqpoint{5.489583in}{0.877907in}}%
\pgfusepath{clip}%
\pgfsetbuttcap%
\pgfsetroundjoin%
\pgfsetlinewidth{1.505625pt}%
\definecolor{currentstroke}{rgb}{0.000000,0.000000,0.000000}%
\pgfsetstrokecolor{currentstroke}%
\pgfsetdash{}{0pt}%
\pgfpathmoveto{\pgfqpoint{7.180222in}{10.679420in}}%
\pgfpathlineto{\pgfqpoint{7.180222in}{11.183260in}}%
\pgfusepath{stroke}%
\end{pgfscope}%
\begin{pgfscope}%
\pgfpathrectangle{\pgfqpoint{2.125000in}{10.401163in}}{\pgfqpoint{5.489583in}{0.877907in}}%
\pgfusepath{clip}%
\pgfsetbuttcap%
\pgfsetroundjoin%
\pgfsetlinewidth{1.505625pt}%
\definecolor{currentstroke}{rgb}{0.000000,0.000000,0.000000}%
\pgfsetstrokecolor{currentstroke}%
\pgfsetdash{}{0pt}%
\pgfpathmoveto{\pgfqpoint{7.303445in}{10.679420in}}%
\pgfpathlineto{\pgfqpoint{7.303445in}{11.181878in}}%
\pgfusepath{stroke}%
\end{pgfscope}%
\begin{pgfscope}%
\pgfpathrectangle{\pgfqpoint{2.125000in}{10.401163in}}{\pgfqpoint{5.489583in}{0.877907in}}%
\pgfusepath{clip}%
\pgfsetroundcap%
\pgfsetroundjoin%
\pgfsetlinewidth{1.505625pt}%
\definecolor{currentstroke}{rgb}{0.121569,0.466667,0.705882}%
\pgfsetstrokecolor{currentstroke}%
\pgfsetdash{}{0pt}%
\pgfpathmoveto{\pgfqpoint{2.125000in}{10.679420in}}%
\pgfpathlineto{\pgfqpoint{7.614583in}{10.679420in}}%
\pgfusepath{stroke}%
\end{pgfscope}%
\begin{pgfscope}%
\pgfpathrectangle{\pgfqpoint{2.125000in}{10.401163in}}{\pgfqpoint{5.489583in}{0.877907in}}%
\pgfusepath{clip}%
\pgfsetbuttcap%
\pgfsetroundjoin%
\definecolor{currentfill}{rgb}{0.121569,0.466667,0.705882}%
\pgfsetfillcolor{currentfill}%
\pgfsetlinewidth{1.003750pt}%
\definecolor{currentstroke}{rgb}{0.121569,0.466667,0.705882}%
\pgfsetstrokecolor{currentstroke}%
\pgfsetdash{}{0pt}%
\pgfsys@defobject{currentmarker}{\pgfqpoint{-0.034722in}{-0.034722in}}{\pgfqpoint{0.034722in}{0.034722in}}{%
\pgfpathmoveto{\pgfqpoint{0.000000in}{-0.034722in}}%
\pgfpathcurveto{\pgfqpoint{0.009208in}{-0.034722in}}{\pgfqpoint{0.018041in}{-0.031064in}}{\pgfqpoint{0.024552in}{-0.024552in}}%
\pgfpathcurveto{\pgfqpoint{0.031064in}{-0.018041in}}{\pgfqpoint{0.034722in}{-0.009208in}}{\pgfqpoint{0.034722in}{0.000000in}}%
\pgfpathcurveto{\pgfqpoint{0.034722in}{0.009208in}}{\pgfqpoint{0.031064in}{0.018041in}}{\pgfqpoint{0.024552in}{0.024552in}}%
\pgfpathcurveto{\pgfqpoint{0.018041in}{0.031064in}}{\pgfqpoint{0.009208in}{0.034722in}}{\pgfqpoint{0.000000in}{0.034722in}}%
\pgfpathcurveto{\pgfqpoint{-0.009208in}{0.034722in}}{\pgfqpoint{-0.018041in}{0.031064in}}{\pgfqpoint{-0.024552in}{0.024552in}}%
\pgfpathcurveto{\pgfqpoint{-0.031064in}{0.018041in}}{\pgfqpoint{-0.034722in}{0.009208in}}{\pgfqpoint{-0.034722in}{0.000000in}}%
\pgfpathcurveto{\pgfqpoint{-0.034722in}{-0.009208in}}{\pgfqpoint{-0.031064in}{-0.018041in}}{\pgfqpoint{-0.024552in}{-0.024552in}}%
\pgfpathcurveto{\pgfqpoint{-0.018041in}{-0.031064in}}{\pgfqpoint{-0.009208in}{-0.034722in}}{\pgfqpoint{0.000000in}{-0.034722in}}%
\pgfpathclose%
\pgfusepath{stroke,fill}%
}%
\begin{pgfscope}%
\pgfsys@transformshift{2.374527in}{11.239165in}%
\pgfsys@useobject{currentmarker}{}%
\end{pgfscope}%
\begin{pgfscope}%
\pgfsys@transformshift{2.497749in}{11.237753in}%
\pgfsys@useobject{currentmarker}{}%
\end{pgfscope}%
\begin{pgfscope}%
\pgfsys@transformshift{2.620972in}{11.236331in}%
\pgfsys@useobject{currentmarker}{}%
\end{pgfscope}%
\begin{pgfscope}%
\pgfsys@transformshift{2.744195in}{11.234897in}%
\pgfsys@useobject{currentmarker}{}%
\end{pgfscope}%
\begin{pgfscope}%
\pgfsys@transformshift{2.867418in}{11.233421in}%
\pgfsys@useobject{currentmarker}{}%
\end{pgfscope}%
\begin{pgfscope}%
\pgfsys@transformshift{2.990641in}{11.231974in}%
\pgfsys@useobject{currentmarker}{}%
\end{pgfscope}%
\begin{pgfscope}%
\pgfsys@transformshift{3.113864in}{11.230498in}%
\pgfsys@useobject{currentmarker}{}%
\end{pgfscope}%
\begin{pgfscope}%
\pgfsys@transformshift{3.237087in}{11.229019in}%
\pgfsys@useobject{currentmarker}{}%
\end{pgfscope}%
\begin{pgfscope}%
\pgfsys@transformshift{3.360310in}{11.227536in}%
\pgfsys@useobject{currentmarker}{}%
\end{pgfscope}%
\begin{pgfscope}%
\pgfsys@transformshift{3.483533in}{11.226034in}%
\pgfsys@useobject{currentmarker}{}%
\end{pgfscope}%
\begin{pgfscope}%
\pgfsys@transformshift{3.606756in}{11.224535in}%
\pgfsys@useobject{currentmarker}{}%
\end{pgfscope}%
\begin{pgfscope}%
\pgfsys@transformshift{3.729979in}{11.223062in}%
\pgfsys@useobject{currentmarker}{}%
\end{pgfscope}%
\begin{pgfscope}%
\pgfsys@transformshift{3.853202in}{11.221579in}%
\pgfsys@useobject{currentmarker}{}%
\end{pgfscope}%
\begin{pgfscope}%
\pgfsys@transformshift{3.976425in}{11.220089in}%
\pgfsys@useobject{currentmarker}{}%
\end{pgfscope}%
\begin{pgfscope}%
\pgfsys@transformshift{4.099648in}{11.218607in}%
\pgfsys@useobject{currentmarker}{}%
\end{pgfscope}%
\begin{pgfscope}%
\pgfsys@transformshift{4.222871in}{11.217158in}%
\pgfsys@useobject{currentmarker}{}%
\end{pgfscope}%
\begin{pgfscope}%
\pgfsys@transformshift{4.346094in}{11.215743in}%
\pgfsys@useobject{currentmarker}{}%
\end{pgfscope}%
\begin{pgfscope}%
\pgfsys@transformshift{4.469317in}{11.214345in}%
\pgfsys@useobject{currentmarker}{}%
\end{pgfscope}%
\begin{pgfscope}%
\pgfsys@transformshift{4.592540in}{11.212926in}%
\pgfsys@useobject{currentmarker}{}%
\end{pgfscope}%
\begin{pgfscope}%
\pgfsys@transformshift{4.715763in}{11.211537in}%
\pgfsys@useobject{currentmarker}{}%
\end{pgfscope}%
\begin{pgfscope}%
\pgfsys@transformshift{4.838986in}{11.210151in}%
\pgfsys@useobject{currentmarker}{}%
\end{pgfscope}%
\begin{pgfscope}%
\pgfsys@transformshift{4.962209in}{11.208759in}%
\pgfsys@useobject{currentmarker}{}%
\end{pgfscope}%
\begin{pgfscope}%
\pgfsys@transformshift{5.085432in}{11.207344in}%
\pgfsys@useobject{currentmarker}{}%
\end{pgfscope}%
\begin{pgfscope}%
\pgfsys@transformshift{5.208655in}{11.205919in}%
\pgfsys@useobject{currentmarker}{}%
\end{pgfscope}%
\begin{pgfscope}%
\pgfsys@transformshift{5.331878in}{11.204502in}%
\pgfsys@useobject{currentmarker}{}%
\end{pgfscope}%
\begin{pgfscope}%
\pgfsys@transformshift{5.455101in}{11.203108in}%
\pgfsys@useobject{currentmarker}{}%
\end{pgfscope}%
\begin{pgfscope}%
\pgfsys@transformshift{5.578324in}{11.201736in}%
\pgfsys@useobject{currentmarker}{}%
\end{pgfscope}%
\begin{pgfscope}%
\pgfsys@transformshift{5.701547in}{11.200343in}%
\pgfsys@useobject{currentmarker}{}%
\end{pgfscope}%
\begin{pgfscope}%
\pgfsys@transformshift{5.824770in}{11.198933in}%
\pgfsys@useobject{currentmarker}{}%
\end{pgfscope}%
\begin{pgfscope}%
\pgfsys@transformshift{5.947993in}{11.197525in}%
\pgfsys@useobject{currentmarker}{}%
\end{pgfscope}%
\begin{pgfscope}%
\pgfsys@transformshift{6.071216in}{11.196102in}%
\pgfsys@useobject{currentmarker}{}%
\end{pgfscope}%
\begin{pgfscope}%
\pgfsys@transformshift{6.194439in}{11.194646in}%
\pgfsys@useobject{currentmarker}{}%
\end{pgfscope}%
\begin{pgfscope}%
\pgfsys@transformshift{6.317662in}{11.193211in}%
\pgfsys@useobject{currentmarker}{}%
\end{pgfscope}%
\begin{pgfscope}%
\pgfsys@transformshift{6.440885in}{11.191786in}%
\pgfsys@useobject{currentmarker}{}%
\end{pgfscope}%
\begin{pgfscope}%
\pgfsys@transformshift{6.564108in}{11.190375in}%
\pgfsys@useobject{currentmarker}{}%
\end{pgfscope}%
\begin{pgfscope}%
\pgfsys@transformshift{6.687330in}{11.188930in}%
\pgfsys@useobject{currentmarker}{}%
\end{pgfscope}%
\begin{pgfscope}%
\pgfsys@transformshift{6.810553in}{11.187480in}%
\pgfsys@useobject{currentmarker}{}%
\end{pgfscope}%
\begin{pgfscope}%
\pgfsys@transformshift{6.933776in}{11.186056in}%
\pgfsys@useobject{currentmarker}{}%
\end{pgfscope}%
\begin{pgfscope}%
\pgfsys@transformshift{7.056999in}{11.184643in}%
\pgfsys@useobject{currentmarker}{}%
\end{pgfscope}%
\begin{pgfscope}%
\pgfsys@transformshift{7.180222in}{11.183260in}%
\pgfsys@useobject{currentmarker}{}%
\end{pgfscope}%
\begin{pgfscope}%
\pgfsys@transformshift{7.303445in}{11.181878in}%
\pgfsys@useobject{currentmarker}{}%
\end{pgfscope}%
\end{pgfscope}%
\begin{pgfscope}%
\pgfsetrectcap%
\pgfsetmiterjoin%
\pgfsetlinewidth{0.803000pt}%
\definecolor{currentstroke}{rgb}{1.000000,1.000000,1.000000}%
\pgfsetstrokecolor{currentstroke}%
\pgfsetdash{}{0pt}%
\pgfpathmoveto{\pgfqpoint{2.125000in}{10.401163in}}%
\pgfpathlineto{\pgfqpoint{2.125000in}{11.279070in}}%
\pgfusepath{stroke}%
\end{pgfscope}%
\begin{pgfscope}%
\pgfsetrectcap%
\pgfsetmiterjoin%
\pgfsetlinewidth{0.803000pt}%
\definecolor{currentstroke}{rgb}{1.000000,1.000000,1.000000}%
\pgfsetstrokecolor{currentstroke}%
\pgfsetdash{}{0pt}%
\pgfpathmoveto{\pgfqpoint{7.614583in}{10.401163in}}%
\pgfpathlineto{\pgfqpoint{7.614583in}{11.279070in}}%
\pgfusepath{stroke}%
\end{pgfscope}%
\begin{pgfscope}%
\pgfsetrectcap%
\pgfsetmiterjoin%
\pgfsetlinewidth{0.803000pt}%
\definecolor{currentstroke}{rgb}{1.000000,1.000000,1.000000}%
\pgfsetstrokecolor{currentstroke}%
\pgfsetdash{}{0pt}%
\pgfpathmoveto{\pgfqpoint{2.125000in}{10.401163in}}%
\pgfpathlineto{\pgfqpoint{7.614583in}{10.401163in}}%
\pgfusepath{stroke}%
\end{pgfscope}%
\begin{pgfscope}%
\pgfsetrectcap%
\pgfsetmiterjoin%
\pgfsetlinewidth{0.803000pt}%
\definecolor{currentstroke}{rgb}{1.000000,1.000000,1.000000}%
\pgfsetstrokecolor{currentstroke}%
\pgfsetdash{}{0pt}%
\pgfpathmoveto{\pgfqpoint{2.125000in}{11.279070in}}%
\pgfpathlineto{\pgfqpoint{7.614583in}{11.279070in}}%
\pgfusepath{stroke}%
\end{pgfscope}%
\begin{pgfscope}%
\definecolor{textcolor}{rgb}{0.150000,0.150000,0.150000}%
\pgfsetstrokecolor{textcolor}%
\pgfsetfillcolor{textcolor}%
\pgftext[x=4.869792in,y=11.362403in,,base]{\color{textcolor}\rmfamily\fontsize{16.800000}{20.160000}\selectfont Autocorrelation}%
\end{pgfscope}%
\begin{pgfscope}%
\pgfsetbuttcap%
\pgfsetmiterjoin%
\definecolor{currentfill}{rgb}{0.917647,0.917647,0.949020}%
\pgfsetfillcolor{currentfill}%
\pgfsetlinewidth{0.000000pt}%
\definecolor{currentstroke}{rgb}{0.000000,0.000000,0.000000}%
\pgfsetstrokecolor{currentstroke}%
\pgfsetstrokeopacity{0.000000}%
\pgfsetdash{}{0pt}%
\pgfpathmoveto{\pgfqpoint{9.810417in}{10.401163in}}%
\pgfpathlineto{\pgfqpoint{15.300000in}{10.401163in}}%
\pgfpathlineto{\pgfqpoint{15.300000in}{11.279070in}}%
\pgfpathlineto{\pgfqpoint{9.810417in}{11.279070in}}%
\pgfpathclose%
\pgfusepath{fill}%
\end{pgfscope}%
\begin{pgfscope}%
\pgfpathrectangle{\pgfqpoint{9.810417in}{10.401163in}}{\pgfqpoint{5.489583in}{0.877907in}}%
\pgfusepath{clip}%
\pgfsetroundcap%
\pgfsetroundjoin%
\pgfsetlinewidth{0.803000pt}%
\definecolor{currentstroke}{rgb}{1.000000,1.000000,1.000000}%
\pgfsetstrokecolor{currentstroke}%
\pgfsetdash{}{0pt}%
\pgfpathmoveto{\pgfqpoint{10.059943in}{10.401163in}}%
\pgfpathlineto{\pgfqpoint{10.059943in}{11.279070in}}%
\pgfusepath{stroke}%
\end{pgfscope}%
\begin{pgfscope}%
\definecolor{textcolor}{rgb}{0.150000,0.150000,0.150000}%
\pgfsetstrokecolor{textcolor}%
\pgfsetfillcolor{textcolor}%
\pgftext[x=10.059943in,y=10.303941in,,top]{\color{textcolor}\rmfamily\fontsize{14.000000}{16.800000}\selectfont 0}%
\end{pgfscope}%
\begin{pgfscope}%
\pgfpathrectangle{\pgfqpoint{9.810417in}{10.401163in}}{\pgfqpoint{5.489583in}{0.877907in}}%
\pgfusepath{clip}%
\pgfsetroundcap%
\pgfsetroundjoin%
\pgfsetlinewidth{0.803000pt}%
\definecolor{currentstroke}{rgb}{1.000000,1.000000,1.000000}%
\pgfsetstrokecolor{currentstroke}%
\pgfsetdash{}{0pt}%
\pgfpathmoveto{\pgfqpoint{10.676058in}{10.401163in}}%
\pgfpathlineto{\pgfqpoint{10.676058in}{11.279070in}}%
\pgfusepath{stroke}%
\end{pgfscope}%
\begin{pgfscope}%
\definecolor{textcolor}{rgb}{0.150000,0.150000,0.150000}%
\pgfsetstrokecolor{textcolor}%
\pgfsetfillcolor{textcolor}%
\pgftext[x=10.676058in,y=10.303941in,,top]{\color{textcolor}\rmfamily\fontsize{14.000000}{16.800000}\selectfont 5}%
\end{pgfscope}%
\begin{pgfscope}%
\pgfpathrectangle{\pgfqpoint{9.810417in}{10.401163in}}{\pgfqpoint{5.489583in}{0.877907in}}%
\pgfusepath{clip}%
\pgfsetroundcap%
\pgfsetroundjoin%
\pgfsetlinewidth{0.803000pt}%
\definecolor{currentstroke}{rgb}{1.000000,1.000000,1.000000}%
\pgfsetstrokecolor{currentstroke}%
\pgfsetdash{}{0pt}%
\pgfpathmoveto{\pgfqpoint{11.292173in}{10.401163in}}%
\pgfpathlineto{\pgfqpoint{11.292173in}{11.279070in}}%
\pgfusepath{stroke}%
\end{pgfscope}%
\begin{pgfscope}%
\definecolor{textcolor}{rgb}{0.150000,0.150000,0.150000}%
\pgfsetstrokecolor{textcolor}%
\pgfsetfillcolor{textcolor}%
\pgftext[x=11.292173in,y=10.303941in,,top]{\color{textcolor}\rmfamily\fontsize{14.000000}{16.800000}\selectfont 10}%
\end{pgfscope}%
\begin{pgfscope}%
\pgfpathrectangle{\pgfqpoint{9.810417in}{10.401163in}}{\pgfqpoint{5.489583in}{0.877907in}}%
\pgfusepath{clip}%
\pgfsetroundcap%
\pgfsetroundjoin%
\pgfsetlinewidth{0.803000pt}%
\definecolor{currentstroke}{rgb}{1.000000,1.000000,1.000000}%
\pgfsetstrokecolor{currentstroke}%
\pgfsetdash{}{0pt}%
\pgfpathmoveto{\pgfqpoint{11.908288in}{10.401163in}}%
\pgfpathlineto{\pgfqpoint{11.908288in}{11.279070in}}%
\pgfusepath{stroke}%
\end{pgfscope}%
\begin{pgfscope}%
\definecolor{textcolor}{rgb}{0.150000,0.150000,0.150000}%
\pgfsetstrokecolor{textcolor}%
\pgfsetfillcolor{textcolor}%
\pgftext[x=11.908288in,y=10.303941in,,top]{\color{textcolor}\rmfamily\fontsize{14.000000}{16.800000}\selectfont 15}%
\end{pgfscope}%
\begin{pgfscope}%
\pgfpathrectangle{\pgfqpoint{9.810417in}{10.401163in}}{\pgfqpoint{5.489583in}{0.877907in}}%
\pgfusepath{clip}%
\pgfsetroundcap%
\pgfsetroundjoin%
\pgfsetlinewidth{0.803000pt}%
\definecolor{currentstroke}{rgb}{1.000000,1.000000,1.000000}%
\pgfsetstrokecolor{currentstroke}%
\pgfsetdash{}{0pt}%
\pgfpathmoveto{\pgfqpoint{12.524403in}{10.401163in}}%
\pgfpathlineto{\pgfqpoint{12.524403in}{11.279070in}}%
\pgfusepath{stroke}%
\end{pgfscope}%
\begin{pgfscope}%
\definecolor{textcolor}{rgb}{0.150000,0.150000,0.150000}%
\pgfsetstrokecolor{textcolor}%
\pgfsetfillcolor{textcolor}%
\pgftext[x=12.524403in,y=10.303941in,,top]{\color{textcolor}\rmfamily\fontsize{14.000000}{16.800000}\selectfont 20}%
\end{pgfscope}%
\begin{pgfscope}%
\pgfpathrectangle{\pgfqpoint{9.810417in}{10.401163in}}{\pgfqpoint{5.489583in}{0.877907in}}%
\pgfusepath{clip}%
\pgfsetroundcap%
\pgfsetroundjoin%
\pgfsetlinewidth{0.803000pt}%
\definecolor{currentstroke}{rgb}{1.000000,1.000000,1.000000}%
\pgfsetstrokecolor{currentstroke}%
\pgfsetdash{}{0pt}%
\pgfpathmoveto{\pgfqpoint{13.140517in}{10.401163in}}%
\pgfpathlineto{\pgfqpoint{13.140517in}{11.279070in}}%
\pgfusepath{stroke}%
\end{pgfscope}%
\begin{pgfscope}%
\definecolor{textcolor}{rgb}{0.150000,0.150000,0.150000}%
\pgfsetstrokecolor{textcolor}%
\pgfsetfillcolor{textcolor}%
\pgftext[x=13.140517in,y=10.303941in,,top]{\color{textcolor}\rmfamily\fontsize{14.000000}{16.800000}\selectfont 25}%
\end{pgfscope}%
\begin{pgfscope}%
\pgfpathrectangle{\pgfqpoint{9.810417in}{10.401163in}}{\pgfqpoint{5.489583in}{0.877907in}}%
\pgfusepath{clip}%
\pgfsetroundcap%
\pgfsetroundjoin%
\pgfsetlinewidth{0.803000pt}%
\definecolor{currentstroke}{rgb}{1.000000,1.000000,1.000000}%
\pgfsetstrokecolor{currentstroke}%
\pgfsetdash{}{0pt}%
\pgfpathmoveto{\pgfqpoint{13.756632in}{10.401163in}}%
\pgfpathlineto{\pgfqpoint{13.756632in}{11.279070in}}%
\pgfusepath{stroke}%
\end{pgfscope}%
\begin{pgfscope}%
\definecolor{textcolor}{rgb}{0.150000,0.150000,0.150000}%
\pgfsetstrokecolor{textcolor}%
\pgfsetfillcolor{textcolor}%
\pgftext[x=13.756632in,y=10.303941in,,top]{\color{textcolor}\rmfamily\fontsize{14.000000}{16.800000}\selectfont 30}%
\end{pgfscope}%
\begin{pgfscope}%
\pgfpathrectangle{\pgfqpoint{9.810417in}{10.401163in}}{\pgfqpoint{5.489583in}{0.877907in}}%
\pgfusepath{clip}%
\pgfsetroundcap%
\pgfsetroundjoin%
\pgfsetlinewidth{0.803000pt}%
\definecolor{currentstroke}{rgb}{1.000000,1.000000,1.000000}%
\pgfsetstrokecolor{currentstroke}%
\pgfsetdash{}{0pt}%
\pgfpathmoveto{\pgfqpoint{14.372747in}{10.401163in}}%
\pgfpathlineto{\pgfqpoint{14.372747in}{11.279070in}}%
\pgfusepath{stroke}%
\end{pgfscope}%
\begin{pgfscope}%
\definecolor{textcolor}{rgb}{0.150000,0.150000,0.150000}%
\pgfsetstrokecolor{textcolor}%
\pgfsetfillcolor{textcolor}%
\pgftext[x=14.372747in,y=10.303941in,,top]{\color{textcolor}\rmfamily\fontsize{14.000000}{16.800000}\selectfont 35}%
\end{pgfscope}%
\begin{pgfscope}%
\pgfpathrectangle{\pgfqpoint{9.810417in}{10.401163in}}{\pgfqpoint{5.489583in}{0.877907in}}%
\pgfusepath{clip}%
\pgfsetroundcap%
\pgfsetroundjoin%
\pgfsetlinewidth{0.803000pt}%
\definecolor{currentstroke}{rgb}{1.000000,1.000000,1.000000}%
\pgfsetstrokecolor{currentstroke}%
\pgfsetdash{}{0pt}%
\pgfpathmoveto{\pgfqpoint{14.988862in}{10.401163in}}%
\pgfpathlineto{\pgfqpoint{14.988862in}{11.279070in}}%
\pgfusepath{stroke}%
\end{pgfscope}%
\begin{pgfscope}%
\definecolor{textcolor}{rgb}{0.150000,0.150000,0.150000}%
\pgfsetstrokecolor{textcolor}%
\pgfsetfillcolor{textcolor}%
\pgftext[x=14.988862in,y=10.303941in,,top]{\color{textcolor}\rmfamily\fontsize{14.000000}{16.800000}\selectfont 40}%
\end{pgfscope}%
\begin{pgfscope}%
\pgfpathrectangle{\pgfqpoint{9.810417in}{10.401163in}}{\pgfqpoint{5.489583in}{0.877907in}}%
\pgfusepath{clip}%
\pgfsetroundcap%
\pgfsetroundjoin%
\pgfsetlinewidth{0.803000pt}%
\definecolor{currentstroke}{rgb}{1.000000,1.000000,1.000000}%
\pgfsetstrokecolor{currentstroke}%
\pgfsetdash{}{0pt}%
\pgfpathmoveto{\pgfqpoint{9.810417in}{10.479401in}}%
\pgfpathlineto{\pgfqpoint{15.300000in}{10.479401in}}%
\pgfusepath{stroke}%
\end{pgfscope}%
\begin{pgfscope}%
\definecolor{textcolor}{rgb}{0.150000,0.150000,0.150000}%
\pgfsetstrokecolor{textcolor}%
\pgfsetfillcolor{textcolor}%
\pgftext[x=9.589483in,y=10.405535in,left,base]{\color{textcolor}\rmfamily\fontsize{14.000000}{16.800000}\selectfont 0}%
\end{pgfscope}%
\begin{pgfscope}%
\pgfpathrectangle{\pgfqpoint{9.810417in}{10.401163in}}{\pgfqpoint{5.489583in}{0.877907in}}%
\pgfusepath{clip}%
\pgfsetroundcap%
\pgfsetroundjoin%
\pgfsetlinewidth{0.803000pt}%
\definecolor{currentstroke}{rgb}{1.000000,1.000000,1.000000}%
\pgfsetstrokecolor{currentstroke}%
\pgfsetdash{}{0pt}%
\pgfpathmoveto{\pgfqpoint{9.810417in}{11.239165in}}%
\pgfpathlineto{\pgfqpoint{15.300000in}{11.239165in}}%
\pgfusepath{stroke}%
\end{pgfscope}%
\begin{pgfscope}%
\definecolor{textcolor}{rgb}{0.150000,0.150000,0.150000}%
\pgfsetstrokecolor{textcolor}%
\pgfsetfillcolor{textcolor}%
\pgftext[x=9.589483in,y=11.165299in,left,base]{\color{textcolor}\rmfamily\fontsize{14.000000}{16.800000}\selectfont 1}%
\end{pgfscope}%
\begin{pgfscope}%
\pgfpathrectangle{\pgfqpoint{9.810417in}{10.401163in}}{\pgfqpoint{5.489583in}{0.877907in}}%
\pgfusepath{clip}%
\pgfsetbuttcap%
\pgfsetroundjoin%
\definecolor{currentfill}{rgb}{0.121569,0.466667,0.705882}%
\pgfsetfillcolor{currentfill}%
\pgfsetfillopacity{0.250000}%
\pgfsetlinewidth{1.003750pt}%
\definecolor{currentstroke}{rgb}{1.000000,1.000000,1.000000}%
\pgfsetstrokecolor{currentstroke}%
\pgfsetstrokeopacity{0.250000}%
\pgfsetdash{}{0pt}%
\pgfpathmoveto{\pgfqpoint{10.121555in}{10.517735in}}%
\pgfpathlineto{\pgfqpoint{10.121555in}{10.441068in}}%
\pgfpathlineto{\pgfqpoint{10.306389in}{10.441068in}}%
\pgfpathlineto{\pgfqpoint{10.429612in}{10.441068in}}%
\pgfpathlineto{\pgfqpoint{10.552835in}{10.441068in}}%
\pgfpathlineto{\pgfqpoint{10.676058in}{10.441068in}}%
\pgfpathlineto{\pgfqpoint{10.799281in}{10.441068in}}%
\pgfpathlineto{\pgfqpoint{10.922504in}{10.441068in}}%
\pgfpathlineto{\pgfqpoint{11.045727in}{10.441068in}}%
\pgfpathlineto{\pgfqpoint{11.168950in}{10.441068in}}%
\pgfpathlineto{\pgfqpoint{11.292173in}{10.441068in}}%
\pgfpathlineto{\pgfqpoint{11.415396in}{10.441068in}}%
\pgfpathlineto{\pgfqpoint{11.538619in}{10.441068in}}%
\pgfpathlineto{\pgfqpoint{11.661842in}{10.441068in}}%
\pgfpathlineto{\pgfqpoint{11.785065in}{10.441068in}}%
\pgfpathlineto{\pgfqpoint{11.908288in}{10.441068in}}%
\pgfpathlineto{\pgfqpoint{12.031511in}{10.441068in}}%
\pgfpathlineto{\pgfqpoint{12.154734in}{10.441068in}}%
\pgfpathlineto{\pgfqpoint{12.277957in}{10.441068in}}%
\pgfpathlineto{\pgfqpoint{12.401180in}{10.441068in}}%
\pgfpathlineto{\pgfqpoint{12.524403in}{10.441068in}}%
\pgfpathlineto{\pgfqpoint{12.647626in}{10.441068in}}%
\pgfpathlineto{\pgfqpoint{12.770849in}{10.441068in}}%
\pgfpathlineto{\pgfqpoint{12.894072in}{10.441068in}}%
\pgfpathlineto{\pgfqpoint{13.017294in}{10.441068in}}%
\pgfpathlineto{\pgfqpoint{13.140517in}{10.441068in}}%
\pgfpathlineto{\pgfqpoint{13.263740in}{10.441068in}}%
\pgfpathlineto{\pgfqpoint{13.386963in}{10.441068in}}%
\pgfpathlineto{\pgfqpoint{13.510186in}{10.441068in}}%
\pgfpathlineto{\pgfqpoint{13.633409in}{10.441068in}}%
\pgfpathlineto{\pgfqpoint{13.756632in}{10.441068in}}%
\pgfpathlineto{\pgfqpoint{13.879855in}{10.441068in}}%
\pgfpathlineto{\pgfqpoint{14.003078in}{10.441068in}}%
\pgfpathlineto{\pgfqpoint{14.126301in}{10.441068in}}%
\pgfpathlineto{\pgfqpoint{14.249524in}{10.441068in}}%
\pgfpathlineto{\pgfqpoint{14.372747in}{10.441068in}}%
\pgfpathlineto{\pgfqpoint{14.495970in}{10.441068in}}%
\pgfpathlineto{\pgfqpoint{14.619193in}{10.441068in}}%
\pgfpathlineto{\pgfqpoint{14.742416in}{10.441068in}}%
\pgfpathlineto{\pgfqpoint{14.865639in}{10.441068in}}%
\pgfpathlineto{\pgfqpoint{15.050473in}{10.441068in}}%
\pgfpathlineto{\pgfqpoint{15.050473in}{10.517735in}}%
\pgfpathlineto{\pgfqpoint{15.050473in}{10.517735in}}%
\pgfpathlineto{\pgfqpoint{14.865639in}{10.517735in}}%
\pgfpathlineto{\pgfqpoint{14.742416in}{10.517735in}}%
\pgfpathlineto{\pgfqpoint{14.619193in}{10.517735in}}%
\pgfpathlineto{\pgfqpoint{14.495970in}{10.517735in}}%
\pgfpathlineto{\pgfqpoint{14.372747in}{10.517735in}}%
\pgfpathlineto{\pgfqpoint{14.249524in}{10.517735in}}%
\pgfpathlineto{\pgfqpoint{14.126301in}{10.517735in}}%
\pgfpathlineto{\pgfqpoint{14.003078in}{10.517735in}}%
\pgfpathlineto{\pgfqpoint{13.879855in}{10.517735in}}%
\pgfpathlineto{\pgfqpoint{13.756632in}{10.517735in}}%
\pgfpathlineto{\pgfqpoint{13.633409in}{10.517735in}}%
\pgfpathlineto{\pgfqpoint{13.510186in}{10.517735in}}%
\pgfpathlineto{\pgfqpoint{13.386963in}{10.517735in}}%
\pgfpathlineto{\pgfqpoint{13.263740in}{10.517735in}}%
\pgfpathlineto{\pgfqpoint{13.140517in}{10.517735in}}%
\pgfpathlineto{\pgfqpoint{13.017294in}{10.517735in}}%
\pgfpathlineto{\pgfqpoint{12.894072in}{10.517735in}}%
\pgfpathlineto{\pgfqpoint{12.770849in}{10.517735in}}%
\pgfpathlineto{\pgfqpoint{12.647626in}{10.517735in}}%
\pgfpathlineto{\pgfqpoint{12.524403in}{10.517735in}}%
\pgfpathlineto{\pgfqpoint{12.401180in}{10.517735in}}%
\pgfpathlineto{\pgfqpoint{12.277957in}{10.517735in}}%
\pgfpathlineto{\pgfqpoint{12.154734in}{10.517735in}}%
\pgfpathlineto{\pgfqpoint{12.031511in}{10.517735in}}%
\pgfpathlineto{\pgfqpoint{11.908288in}{10.517735in}}%
\pgfpathlineto{\pgfqpoint{11.785065in}{10.517735in}}%
\pgfpathlineto{\pgfqpoint{11.661842in}{10.517735in}}%
\pgfpathlineto{\pgfqpoint{11.538619in}{10.517735in}}%
\pgfpathlineto{\pgfqpoint{11.415396in}{10.517735in}}%
\pgfpathlineto{\pgfqpoint{11.292173in}{10.517735in}}%
\pgfpathlineto{\pgfqpoint{11.168950in}{10.517735in}}%
\pgfpathlineto{\pgfqpoint{11.045727in}{10.517735in}}%
\pgfpathlineto{\pgfqpoint{10.922504in}{10.517735in}}%
\pgfpathlineto{\pgfqpoint{10.799281in}{10.517735in}}%
\pgfpathlineto{\pgfqpoint{10.676058in}{10.517735in}}%
\pgfpathlineto{\pgfqpoint{10.552835in}{10.517735in}}%
\pgfpathlineto{\pgfqpoint{10.429612in}{10.517735in}}%
\pgfpathlineto{\pgfqpoint{10.306389in}{10.517735in}}%
\pgfpathlineto{\pgfqpoint{10.121555in}{10.517735in}}%
\pgfpathclose%
\pgfusepath{stroke,fill}%
\end{pgfscope}%
\begin{pgfscope}%
\pgfpathrectangle{\pgfqpoint{9.810417in}{10.401163in}}{\pgfqpoint{5.489583in}{0.877907in}}%
\pgfusepath{clip}%
\pgfsetbuttcap%
\pgfsetroundjoin%
\pgfsetlinewidth{1.505625pt}%
\definecolor{currentstroke}{rgb}{0.000000,0.000000,0.000000}%
\pgfsetstrokecolor{currentstroke}%
\pgfsetdash{}{0pt}%
\pgfpathmoveto{\pgfqpoint{10.059943in}{10.479401in}}%
\pgfpathlineto{\pgfqpoint{10.059943in}{11.239165in}}%
\pgfusepath{stroke}%
\end{pgfscope}%
\begin{pgfscope}%
\pgfpathrectangle{\pgfqpoint{9.810417in}{10.401163in}}{\pgfqpoint{5.489583in}{0.877907in}}%
\pgfusepath{clip}%
\pgfsetbuttcap%
\pgfsetroundjoin%
\pgfsetlinewidth{1.505625pt}%
\definecolor{currentstroke}{rgb}{0.000000,0.000000,0.000000}%
\pgfsetstrokecolor{currentstroke}%
\pgfsetdash{}{0pt}%
\pgfpathmoveto{\pgfqpoint{10.183166in}{10.479401in}}%
\pgfpathlineto{\pgfqpoint{10.183166in}{11.237752in}}%
\pgfusepath{stroke}%
\end{pgfscope}%
\begin{pgfscope}%
\pgfpathrectangle{\pgfqpoint{9.810417in}{10.401163in}}{\pgfqpoint{5.489583in}{0.877907in}}%
\pgfusepath{clip}%
\pgfsetbuttcap%
\pgfsetroundjoin%
\pgfsetlinewidth{1.505625pt}%
\definecolor{currentstroke}{rgb}{0.000000,0.000000,0.000000}%
\pgfsetstrokecolor{currentstroke}%
\pgfsetdash{}{0pt}%
\pgfpathmoveto{\pgfqpoint{10.306389in}{10.479401in}}%
\pgfpathlineto{\pgfqpoint{10.306389in}{10.474090in}}%
\pgfusepath{stroke}%
\end{pgfscope}%
\begin{pgfscope}%
\pgfpathrectangle{\pgfqpoint{9.810417in}{10.401163in}}{\pgfqpoint{5.489583in}{0.877907in}}%
\pgfusepath{clip}%
\pgfsetbuttcap%
\pgfsetroundjoin%
\pgfsetlinewidth{1.505625pt}%
\definecolor{currentstroke}{rgb}{0.000000,0.000000,0.000000}%
\pgfsetstrokecolor{currentstroke}%
\pgfsetdash{}{0pt}%
\pgfpathmoveto{\pgfqpoint{10.429612in}{10.479401in}}%
\pgfpathlineto{\pgfqpoint{10.429612in}{10.474101in}}%
\pgfusepath{stroke}%
\end{pgfscope}%
\begin{pgfscope}%
\pgfpathrectangle{\pgfqpoint{9.810417in}{10.401163in}}{\pgfqpoint{5.489583in}{0.877907in}}%
\pgfusepath{clip}%
\pgfsetbuttcap%
\pgfsetroundjoin%
\pgfsetlinewidth{1.505625pt}%
\definecolor{currentstroke}{rgb}{0.000000,0.000000,0.000000}%
\pgfsetstrokecolor{currentstroke}%
\pgfsetdash{}{0pt}%
\pgfpathmoveto{\pgfqpoint{10.552835in}{10.479401in}}%
\pgfpathlineto{\pgfqpoint{10.552835in}{10.462792in}}%
\pgfusepath{stroke}%
\end{pgfscope}%
\begin{pgfscope}%
\pgfpathrectangle{\pgfqpoint{9.810417in}{10.401163in}}{\pgfqpoint{5.489583in}{0.877907in}}%
\pgfusepath{clip}%
\pgfsetbuttcap%
\pgfsetroundjoin%
\pgfsetlinewidth{1.505625pt}%
\definecolor{currentstroke}{rgb}{0.000000,0.000000,0.000000}%
\pgfsetstrokecolor{currentstroke}%
\pgfsetdash{}{0pt}%
\pgfpathmoveto{\pgfqpoint{10.676058in}{10.479401in}}%
\pgfpathlineto{\pgfqpoint{10.676058in}{10.488750in}}%
\pgfusepath{stroke}%
\end{pgfscope}%
\begin{pgfscope}%
\pgfpathrectangle{\pgfqpoint{9.810417in}{10.401163in}}{\pgfqpoint{5.489583in}{0.877907in}}%
\pgfusepath{clip}%
\pgfsetbuttcap%
\pgfsetroundjoin%
\pgfsetlinewidth{1.505625pt}%
\definecolor{currentstroke}{rgb}{0.000000,0.000000,0.000000}%
\pgfsetstrokecolor{currentstroke}%
\pgfsetdash{}{0pt}%
\pgfpathmoveto{\pgfqpoint{10.799281in}{10.479401in}}%
\pgfpathlineto{\pgfqpoint{10.799281in}{10.467843in}}%
\pgfusepath{stroke}%
\end{pgfscope}%
\begin{pgfscope}%
\pgfpathrectangle{\pgfqpoint{9.810417in}{10.401163in}}{\pgfqpoint{5.489583in}{0.877907in}}%
\pgfusepath{clip}%
\pgfsetbuttcap%
\pgfsetroundjoin%
\pgfsetlinewidth{1.505625pt}%
\definecolor{currentstroke}{rgb}{0.000000,0.000000,0.000000}%
\pgfsetstrokecolor{currentstroke}%
\pgfsetdash{}{0pt}%
\pgfpathmoveto{\pgfqpoint{10.922504in}{10.479401in}}%
\pgfpathlineto{\pgfqpoint{10.922504in}{10.477460in}}%
\pgfusepath{stroke}%
\end{pgfscope}%
\begin{pgfscope}%
\pgfpathrectangle{\pgfqpoint{9.810417in}{10.401163in}}{\pgfqpoint{5.489583in}{0.877907in}}%
\pgfusepath{clip}%
\pgfsetbuttcap%
\pgfsetroundjoin%
\pgfsetlinewidth{1.505625pt}%
\definecolor{currentstroke}{rgb}{0.000000,0.000000,0.000000}%
\pgfsetstrokecolor{currentstroke}%
\pgfsetdash{}{0pt}%
\pgfpathmoveto{\pgfqpoint{11.045727in}{10.479401in}}%
\pgfpathlineto{\pgfqpoint{11.045727in}{10.475896in}}%
\pgfusepath{stroke}%
\end{pgfscope}%
\begin{pgfscope}%
\pgfpathrectangle{\pgfqpoint{9.810417in}{10.401163in}}{\pgfqpoint{5.489583in}{0.877907in}}%
\pgfusepath{clip}%
\pgfsetbuttcap%
\pgfsetroundjoin%
\pgfsetlinewidth{1.505625pt}%
\definecolor{currentstroke}{rgb}{0.000000,0.000000,0.000000}%
\pgfsetstrokecolor{currentstroke}%
\pgfsetdash{}{0pt}%
\pgfpathmoveto{\pgfqpoint{11.168950in}{10.479401in}}%
\pgfpathlineto{\pgfqpoint{11.168950in}{10.471942in}}%
\pgfusepath{stroke}%
\end{pgfscope}%
\begin{pgfscope}%
\pgfpathrectangle{\pgfqpoint{9.810417in}{10.401163in}}{\pgfqpoint{5.489583in}{0.877907in}}%
\pgfusepath{clip}%
\pgfsetbuttcap%
\pgfsetroundjoin%
\pgfsetlinewidth{1.505625pt}%
\definecolor{currentstroke}{rgb}{0.000000,0.000000,0.000000}%
\pgfsetstrokecolor{currentstroke}%
\pgfsetdash{}{0pt}%
\pgfpathmoveto{\pgfqpoint{11.292173in}{10.479401in}}%
\pgfpathlineto{\pgfqpoint{11.292173in}{10.479226in}}%
\pgfusepath{stroke}%
\end{pgfscope}%
\begin{pgfscope}%
\pgfpathrectangle{\pgfqpoint{9.810417in}{10.401163in}}{\pgfqpoint{5.489583in}{0.877907in}}%
\pgfusepath{clip}%
\pgfsetbuttcap%
\pgfsetroundjoin%
\pgfsetlinewidth{1.505625pt}%
\definecolor{currentstroke}{rgb}{0.000000,0.000000,0.000000}%
\pgfsetstrokecolor{currentstroke}%
\pgfsetdash{}{0pt}%
\pgfpathmoveto{\pgfqpoint{11.415396in}{10.479401in}}%
\pgfpathlineto{\pgfqpoint{11.415396in}{10.487917in}}%
\pgfusepath{stroke}%
\end{pgfscope}%
\begin{pgfscope}%
\pgfpathrectangle{\pgfqpoint{9.810417in}{10.401163in}}{\pgfqpoint{5.489583in}{0.877907in}}%
\pgfusepath{clip}%
\pgfsetbuttcap%
\pgfsetroundjoin%
\pgfsetlinewidth{1.505625pt}%
\definecolor{currentstroke}{rgb}{0.000000,0.000000,0.000000}%
\pgfsetstrokecolor{currentstroke}%
\pgfsetdash{}{0pt}%
\pgfpathmoveto{\pgfqpoint{11.538619in}{10.479401in}}%
\pgfpathlineto{\pgfqpoint{11.538619in}{10.474265in}}%
\pgfusepath{stroke}%
\end{pgfscope}%
\begin{pgfscope}%
\pgfpathrectangle{\pgfqpoint{9.810417in}{10.401163in}}{\pgfqpoint{5.489583in}{0.877907in}}%
\pgfusepath{clip}%
\pgfsetbuttcap%
\pgfsetroundjoin%
\pgfsetlinewidth{1.505625pt}%
\definecolor{currentstroke}{rgb}{0.000000,0.000000,0.000000}%
\pgfsetstrokecolor{currentstroke}%
\pgfsetdash{}{0pt}%
\pgfpathmoveto{\pgfqpoint{11.661842in}{10.479401in}}%
\pgfpathlineto{\pgfqpoint{11.661842in}{10.475457in}}%
\pgfusepath{stroke}%
\end{pgfscope}%
\begin{pgfscope}%
\pgfpathrectangle{\pgfqpoint{9.810417in}{10.401163in}}{\pgfqpoint{5.489583in}{0.877907in}}%
\pgfusepath{clip}%
\pgfsetbuttcap%
\pgfsetroundjoin%
\pgfsetlinewidth{1.505625pt}%
\definecolor{currentstroke}{rgb}{0.000000,0.000000,0.000000}%
\pgfsetstrokecolor{currentstroke}%
\pgfsetdash{}{0pt}%
\pgfpathmoveto{\pgfqpoint{11.785065in}{10.479401in}}%
\pgfpathlineto{\pgfqpoint{11.785065in}{10.480883in}}%
\pgfusepath{stroke}%
\end{pgfscope}%
\begin{pgfscope}%
\pgfpathrectangle{\pgfqpoint{9.810417in}{10.401163in}}{\pgfqpoint{5.489583in}{0.877907in}}%
\pgfusepath{clip}%
\pgfsetbuttcap%
\pgfsetroundjoin%
\pgfsetlinewidth{1.505625pt}%
\definecolor{currentstroke}{rgb}{0.000000,0.000000,0.000000}%
\pgfsetstrokecolor{currentstroke}%
\pgfsetdash{}{0pt}%
\pgfpathmoveto{\pgfqpoint{11.908288in}{10.479401in}}%
\pgfpathlineto{\pgfqpoint{11.908288in}{10.491011in}}%
\pgfusepath{stroke}%
\end{pgfscope}%
\begin{pgfscope}%
\pgfpathrectangle{\pgfqpoint{9.810417in}{10.401163in}}{\pgfqpoint{5.489583in}{0.877907in}}%
\pgfusepath{clip}%
\pgfsetbuttcap%
\pgfsetroundjoin%
\pgfsetlinewidth{1.505625pt}%
\definecolor{currentstroke}{rgb}{0.000000,0.000000,0.000000}%
\pgfsetstrokecolor{currentstroke}%
\pgfsetdash{}{0pt}%
\pgfpathmoveto{\pgfqpoint{12.031511in}{10.479401in}}%
\pgfpathlineto{\pgfqpoint{12.031511in}{10.490082in}}%
\pgfusepath{stroke}%
\end{pgfscope}%
\begin{pgfscope}%
\pgfpathrectangle{\pgfqpoint{9.810417in}{10.401163in}}{\pgfqpoint{5.489583in}{0.877907in}}%
\pgfusepath{clip}%
\pgfsetbuttcap%
\pgfsetroundjoin%
\pgfsetlinewidth{1.505625pt}%
\definecolor{currentstroke}{rgb}{0.000000,0.000000,0.000000}%
\pgfsetstrokecolor{currentstroke}%
\pgfsetdash{}{0pt}%
\pgfpathmoveto{\pgfqpoint{12.154734in}{10.479401in}}%
\pgfpathlineto{\pgfqpoint{12.154734in}{10.483925in}}%
\pgfusepath{stroke}%
\end{pgfscope}%
\begin{pgfscope}%
\pgfpathrectangle{\pgfqpoint{9.810417in}{10.401163in}}{\pgfqpoint{5.489583in}{0.877907in}}%
\pgfusepath{clip}%
\pgfsetbuttcap%
\pgfsetroundjoin%
\pgfsetlinewidth{1.505625pt}%
\definecolor{currentstroke}{rgb}{0.000000,0.000000,0.000000}%
\pgfsetstrokecolor{currentstroke}%
\pgfsetdash{}{0pt}%
\pgfpathmoveto{\pgfqpoint{12.277957in}{10.479401in}}%
\pgfpathlineto{\pgfqpoint{12.277957in}{10.470171in}}%
\pgfusepath{stroke}%
\end{pgfscope}%
\begin{pgfscope}%
\pgfpathrectangle{\pgfqpoint{9.810417in}{10.401163in}}{\pgfqpoint{5.489583in}{0.877907in}}%
\pgfusepath{clip}%
\pgfsetbuttcap%
\pgfsetroundjoin%
\pgfsetlinewidth{1.505625pt}%
\definecolor{currentstroke}{rgb}{0.000000,0.000000,0.000000}%
\pgfsetstrokecolor{currentstroke}%
\pgfsetdash{}{0pt}%
\pgfpathmoveto{\pgfqpoint{12.401180in}{10.479401in}}%
\pgfpathlineto{\pgfqpoint{12.401180in}{10.488543in}}%
\pgfusepath{stroke}%
\end{pgfscope}%
\begin{pgfscope}%
\pgfpathrectangle{\pgfqpoint{9.810417in}{10.401163in}}{\pgfqpoint{5.489583in}{0.877907in}}%
\pgfusepath{clip}%
\pgfsetbuttcap%
\pgfsetroundjoin%
\pgfsetlinewidth{1.505625pt}%
\definecolor{currentstroke}{rgb}{0.000000,0.000000,0.000000}%
\pgfsetstrokecolor{currentstroke}%
\pgfsetdash{}{0pt}%
\pgfpathmoveto{\pgfqpoint{12.524403in}{10.479401in}}%
\pgfpathlineto{\pgfqpoint{12.524403in}{10.479145in}}%
\pgfusepath{stroke}%
\end{pgfscope}%
\begin{pgfscope}%
\pgfpathrectangle{\pgfqpoint{9.810417in}{10.401163in}}{\pgfqpoint{5.489583in}{0.877907in}}%
\pgfusepath{clip}%
\pgfsetbuttcap%
\pgfsetroundjoin%
\pgfsetlinewidth{1.505625pt}%
\definecolor{currentstroke}{rgb}{0.000000,0.000000,0.000000}%
\pgfsetstrokecolor{currentstroke}%
\pgfsetdash{}{0pt}%
\pgfpathmoveto{\pgfqpoint{12.647626in}{10.479401in}}%
\pgfpathlineto{\pgfqpoint{12.647626in}{10.476369in}}%
\pgfusepath{stroke}%
\end{pgfscope}%
\begin{pgfscope}%
\pgfpathrectangle{\pgfqpoint{9.810417in}{10.401163in}}{\pgfqpoint{5.489583in}{0.877907in}}%
\pgfusepath{clip}%
\pgfsetbuttcap%
\pgfsetroundjoin%
\pgfsetlinewidth{1.505625pt}%
\definecolor{currentstroke}{rgb}{0.000000,0.000000,0.000000}%
\pgfsetstrokecolor{currentstroke}%
\pgfsetdash{}{0pt}%
\pgfpathmoveto{\pgfqpoint{12.770849in}{10.479401in}}%
\pgfpathlineto{\pgfqpoint{12.770849in}{10.468494in}}%
\pgfusepath{stroke}%
\end{pgfscope}%
\begin{pgfscope}%
\pgfpathrectangle{\pgfqpoint{9.810417in}{10.401163in}}{\pgfqpoint{5.489583in}{0.877907in}}%
\pgfusepath{clip}%
\pgfsetbuttcap%
\pgfsetroundjoin%
\pgfsetlinewidth{1.505625pt}%
\definecolor{currentstroke}{rgb}{0.000000,0.000000,0.000000}%
\pgfsetstrokecolor{currentstroke}%
\pgfsetdash{}{0pt}%
\pgfpathmoveto{\pgfqpoint{12.894072in}{10.479401in}}%
\pgfpathlineto{\pgfqpoint{12.894072in}{10.474579in}}%
\pgfusepath{stroke}%
\end{pgfscope}%
\begin{pgfscope}%
\pgfpathrectangle{\pgfqpoint{9.810417in}{10.401163in}}{\pgfqpoint{5.489583in}{0.877907in}}%
\pgfusepath{clip}%
\pgfsetbuttcap%
\pgfsetroundjoin%
\pgfsetlinewidth{1.505625pt}%
\definecolor{currentstroke}{rgb}{0.000000,0.000000,0.000000}%
\pgfsetstrokecolor{currentstroke}%
\pgfsetdash{}{0pt}%
\pgfpathmoveto{\pgfqpoint{13.017294in}{10.479401in}}%
\pgfpathlineto{\pgfqpoint{13.017294in}{10.480993in}}%
\pgfusepath{stroke}%
\end{pgfscope}%
\begin{pgfscope}%
\pgfpathrectangle{\pgfqpoint{9.810417in}{10.401163in}}{\pgfqpoint{5.489583in}{0.877907in}}%
\pgfusepath{clip}%
\pgfsetbuttcap%
\pgfsetroundjoin%
\pgfsetlinewidth{1.505625pt}%
\definecolor{currentstroke}{rgb}{0.000000,0.000000,0.000000}%
\pgfsetstrokecolor{currentstroke}%
\pgfsetdash{}{0pt}%
\pgfpathmoveto{\pgfqpoint{13.140517in}{10.479401in}}%
\pgfpathlineto{\pgfqpoint{13.140517in}{10.486893in}}%
\pgfusepath{stroke}%
\end{pgfscope}%
\begin{pgfscope}%
\pgfpathrectangle{\pgfqpoint{9.810417in}{10.401163in}}{\pgfqpoint{5.489583in}{0.877907in}}%
\pgfusepath{clip}%
\pgfsetbuttcap%
\pgfsetroundjoin%
\pgfsetlinewidth{1.505625pt}%
\definecolor{currentstroke}{rgb}{0.000000,0.000000,0.000000}%
\pgfsetstrokecolor{currentstroke}%
\pgfsetdash{}{0pt}%
\pgfpathmoveto{\pgfqpoint{13.263740in}{10.479401in}}%
\pgfpathlineto{\pgfqpoint{13.263740in}{10.486707in}}%
\pgfusepath{stroke}%
\end{pgfscope}%
\begin{pgfscope}%
\pgfpathrectangle{\pgfqpoint{9.810417in}{10.401163in}}{\pgfqpoint{5.489583in}{0.877907in}}%
\pgfusepath{clip}%
\pgfsetbuttcap%
\pgfsetroundjoin%
\pgfsetlinewidth{1.505625pt}%
\definecolor{currentstroke}{rgb}{0.000000,0.000000,0.000000}%
\pgfsetstrokecolor{currentstroke}%
\pgfsetdash{}{0pt}%
\pgfpathmoveto{\pgfqpoint{13.386963in}{10.479401in}}%
\pgfpathlineto{\pgfqpoint{13.386963in}{10.470054in}}%
\pgfusepath{stroke}%
\end{pgfscope}%
\begin{pgfscope}%
\pgfpathrectangle{\pgfqpoint{9.810417in}{10.401163in}}{\pgfqpoint{5.489583in}{0.877907in}}%
\pgfusepath{clip}%
\pgfsetbuttcap%
\pgfsetroundjoin%
\pgfsetlinewidth{1.505625pt}%
\definecolor{currentstroke}{rgb}{0.000000,0.000000,0.000000}%
\pgfsetstrokecolor{currentstroke}%
\pgfsetdash{}{0pt}%
\pgfpathmoveto{\pgfqpoint{13.510186in}{10.479401in}}%
\pgfpathlineto{\pgfqpoint{13.510186in}{10.471396in}}%
\pgfusepath{stroke}%
\end{pgfscope}%
\begin{pgfscope}%
\pgfpathrectangle{\pgfqpoint{9.810417in}{10.401163in}}{\pgfqpoint{5.489583in}{0.877907in}}%
\pgfusepath{clip}%
\pgfsetbuttcap%
\pgfsetroundjoin%
\pgfsetlinewidth{1.505625pt}%
\definecolor{currentstroke}{rgb}{0.000000,0.000000,0.000000}%
\pgfsetstrokecolor{currentstroke}%
\pgfsetdash{}{0pt}%
\pgfpathmoveto{\pgfqpoint{13.633409in}{10.479401in}}%
\pgfpathlineto{\pgfqpoint{13.633409in}{10.479059in}}%
\pgfusepath{stroke}%
\end{pgfscope}%
\begin{pgfscope}%
\pgfpathrectangle{\pgfqpoint{9.810417in}{10.401163in}}{\pgfqpoint{5.489583in}{0.877907in}}%
\pgfusepath{clip}%
\pgfsetbuttcap%
\pgfsetroundjoin%
\pgfsetlinewidth{1.505625pt}%
\definecolor{currentstroke}{rgb}{0.000000,0.000000,0.000000}%
\pgfsetstrokecolor{currentstroke}%
\pgfsetdash{}{0pt}%
\pgfpathmoveto{\pgfqpoint{13.756632in}{10.479401in}}%
\pgfpathlineto{\pgfqpoint{13.756632in}{10.474079in}}%
\pgfusepath{stroke}%
\end{pgfscope}%
\begin{pgfscope}%
\pgfpathrectangle{\pgfqpoint{9.810417in}{10.401163in}}{\pgfqpoint{5.489583in}{0.877907in}}%
\pgfusepath{clip}%
\pgfsetbuttcap%
\pgfsetroundjoin%
\pgfsetlinewidth{1.505625pt}%
\definecolor{currentstroke}{rgb}{0.000000,0.000000,0.000000}%
\pgfsetstrokecolor{currentstroke}%
\pgfsetdash{}{0pt}%
\pgfpathmoveto{\pgfqpoint{13.879855in}{10.479401in}}%
\pgfpathlineto{\pgfqpoint{13.879855in}{10.465764in}}%
\pgfusepath{stroke}%
\end{pgfscope}%
\begin{pgfscope}%
\pgfpathrectangle{\pgfqpoint{9.810417in}{10.401163in}}{\pgfqpoint{5.489583in}{0.877907in}}%
\pgfusepath{clip}%
\pgfsetbuttcap%
\pgfsetroundjoin%
\pgfsetlinewidth{1.505625pt}%
\definecolor{currentstroke}{rgb}{0.000000,0.000000,0.000000}%
\pgfsetstrokecolor{currentstroke}%
\pgfsetdash{}{0pt}%
\pgfpathmoveto{\pgfqpoint{14.003078in}{10.479401in}}%
\pgfpathlineto{\pgfqpoint{14.003078in}{10.485554in}}%
\pgfusepath{stroke}%
\end{pgfscope}%
\begin{pgfscope}%
\pgfpathrectangle{\pgfqpoint{9.810417in}{10.401163in}}{\pgfqpoint{5.489583in}{0.877907in}}%
\pgfusepath{clip}%
\pgfsetbuttcap%
\pgfsetroundjoin%
\pgfsetlinewidth{1.505625pt}%
\definecolor{currentstroke}{rgb}{0.000000,0.000000,0.000000}%
\pgfsetstrokecolor{currentstroke}%
\pgfsetdash{}{0pt}%
\pgfpathmoveto{\pgfqpoint{14.126301in}{10.479401in}}%
\pgfpathlineto{\pgfqpoint{14.126301in}{10.482125in}}%
\pgfusepath{stroke}%
\end{pgfscope}%
\begin{pgfscope}%
\pgfpathrectangle{\pgfqpoint{9.810417in}{10.401163in}}{\pgfqpoint{5.489583in}{0.877907in}}%
\pgfusepath{clip}%
\pgfsetbuttcap%
\pgfsetroundjoin%
\pgfsetlinewidth{1.505625pt}%
\definecolor{currentstroke}{rgb}{0.000000,0.000000,0.000000}%
\pgfsetstrokecolor{currentstroke}%
\pgfsetdash{}{0pt}%
\pgfpathmoveto{\pgfqpoint{14.249524in}{10.479401in}}%
\pgfpathlineto{\pgfqpoint{14.249524in}{10.484175in}}%
\pgfusepath{stroke}%
\end{pgfscope}%
\begin{pgfscope}%
\pgfpathrectangle{\pgfqpoint{9.810417in}{10.401163in}}{\pgfqpoint{5.489583in}{0.877907in}}%
\pgfusepath{clip}%
\pgfsetbuttcap%
\pgfsetroundjoin%
\pgfsetlinewidth{1.505625pt}%
\definecolor{currentstroke}{rgb}{0.000000,0.000000,0.000000}%
\pgfsetstrokecolor{currentstroke}%
\pgfsetdash{}{0pt}%
\pgfpathmoveto{\pgfqpoint{14.372747in}{10.479401in}}%
\pgfpathlineto{\pgfqpoint{14.372747in}{10.465506in}}%
\pgfusepath{stroke}%
\end{pgfscope}%
\begin{pgfscope}%
\pgfpathrectangle{\pgfqpoint{9.810417in}{10.401163in}}{\pgfqpoint{5.489583in}{0.877907in}}%
\pgfusepath{clip}%
\pgfsetbuttcap%
\pgfsetroundjoin%
\pgfsetlinewidth{1.505625pt}%
\definecolor{currentstroke}{rgb}{0.000000,0.000000,0.000000}%
\pgfsetstrokecolor{currentstroke}%
\pgfsetdash{}{0pt}%
\pgfpathmoveto{\pgfqpoint{14.495970in}{10.479401in}}%
\pgfpathlineto{\pgfqpoint{14.495970in}{10.476061in}}%
\pgfusepath{stroke}%
\end{pgfscope}%
\begin{pgfscope}%
\pgfpathrectangle{\pgfqpoint{9.810417in}{10.401163in}}{\pgfqpoint{5.489583in}{0.877907in}}%
\pgfusepath{clip}%
\pgfsetbuttcap%
\pgfsetroundjoin%
\pgfsetlinewidth{1.505625pt}%
\definecolor{currentstroke}{rgb}{0.000000,0.000000,0.000000}%
\pgfsetstrokecolor{currentstroke}%
\pgfsetdash{}{0pt}%
\pgfpathmoveto{\pgfqpoint{14.619193in}{10.479401in}}%
\pgfpathlineto{\pgfqpoint{14.619193in}{10.487083in}}%
\pgfusepath{stroke}%
\end{pgfscope}%
\begin{pgfscope}%
\pgfpathrectangle{\pgfqpoint{9.810417in}{10.401163in}}{\pgfqpoint{5.489583in}{0.877907in}}%
\pgfusepath{clip}%
\pgfsetbuttcap%
\pgfsetroundjoin%
\pgfsetlinewidth{1.505625pt}%
\definecolor{currentstroke}{rgb}{0.000000,0.000000,0.000000}%
\pgfsetstrokecolor{currentstroke}%
\pgfsetdash{}{0pt}%
\pgfpathmoveto{\pgfqpoint{14.742416in}{10.479401in}}%
\pgfpathlineto{\pgfqpoint{14.742416in}{10.482670in}}%
\pgfusepath{stroke}%
\end{pgfscope}%
\begin{pgfscope}%
\pgfpathrectangle{\pgfqpoint{9.810417in}{10.401163in}}{\pgfqpoint{5.489583in}{0.877907in}}%
\pgfusepath{clip}%
\pgfsetbuttcap%
\pgfsetroundjoin%
\pgfsetlinewidth{1.505625pt}%
\definecolor{currentstroke}{rgb}{0.000000,0.000000,0.000000}%
\pgfsetstrokecolor{currentstroke}%
\pgfsetdash{}{0pt}%
\pgfpathmoveto{\pgfqpoint{14.865639in}{10.479401in}}%
\pgfpathlineto{\pgfqpoint{14.865639in}{10.489412in}}%
\pgfusepath{stroke}%
\end{pgfscope}%
\begin{pgfscope}%
\pgfpathrectangle{\pgfqpoint{9.810417in}{10.401163in}}{\pgfqpoint{5.489583in}{0.877907in}}%
\pgfusepath{clip}%
\pgfsetbuttcap%
\pgfsetroundjoin%
\pgfsetlinewidth{1.505625pt}%
\definecolor{currentstroke}{rgb}{0.000000,0.000000,0.000000}%
\pgfsetstrokecolor{currentstroke}%
\pgfsetdash{}{0pt}%
\pgfpathmoveto{\pgfqpoint{14.988862in}{10.479401in}}%
\pgfpathlineto{\pgfqpoint{14.988862in}{10.478444in}}%
\pgfusepath{stroke}%
\end{pgfscope}%
\begin{pgfscope}%
\pgfpathrectangle{\pgfqpoint{9.810417in}{10.401163in}}{\pgfqpoint{5.489583in}{0.877907in}}%
\pgfusepath{clip}%
\pgfsetroundcap%
\pgfsetroundjoin%
\pgfsetlinewidth{1.505625pt}%
\definecolor{currentstroke}{rgb}{0.121569,0.466667,0.705882}%
\pgfsetstrokecolor{currentstroke}%
\pgfsetdash{}{0pt}%
\pgfpathmoveto{\pgfqpoint{9.810417in}{10.479401in}}%
\pgfpathlineto{\pgfqpoint{15.300000in}{10.479401in}}%
\pgfusepath{stroke}%
\end{pgfscope}%
\begin{pgfscope}%
\pgfpathrectangle{\pgfqpoint{9.810417in}{10.401163in}}{\pgfqpoint{5.489583in}{0.877907in}}%
\pgfusepath{clip}%
\pgfsetbuttcap%
\pgfsetroundjoin%
\definecolor{currentfill}{rgb}{0.121569,0.466667,0.705882}%
\pgfsetfillcolor{currentfill}%
\pgfsetlinewidth{1.003750pt}%
\definecolor{currentstroke}{rgb}{0.121569,0.466667,0.705882}%
\pgfsetstrokecolor{currentstroke}%
\pgfsetdash{}{0pt}%
\pgfsys@defobject{currentmarker}{\pgfqpoint{-0.034722in}{-0.034722in}}{\pgfqpoint{0.034722in}{0.034722in}}{%
\pgfpathmoveto{\pgfqpoint{0.000000in}{-0.034722in}}%
\pgfpathcurveto{\pgfqpoint{0.009208in}{-0.034722in}}{\pgfqpoint{0.018041in}{-0.031064in}}{\pgfqpoint{0.024552in}{-0.024552in}}%
\pgfpathcurveto{\pgfqpoint{0.031064in}{-0.018041in}}{\pgfqpoint{0.034722in}{-0.009208in}}{\pgfqpoint{0.034722in}{0.000000in}}%
\pgfpathcurveto{\pgfqpoint{0.034722in}{0.009208in}}{\pgfqpoint{0.031064in}{0.018041in}}{\pgfqpoint{0.024552in}{0.024552in}}%
\pgfpathcurveto{\pgfqpoint{0.018041in}{0.031064in}}{\pgfqpoint{0.009208in}{0.034722in}}{\pgfqpoint{0.000000in}{0.034722in}}%
\pgfpathcurveto{\pgfqpoint{-0.009208in}{0.034722in}}{\pgfqpoint{-0.018041in}{0.031064in}}{\pgfqpoint{-0.024552in}{0.024552in}}%
\pgfpathcurveto{\pgfqpoint{-0.031064in}{0.018041in}}{\pgfqpoint{-0.034722in}{0.009208in}}{\pgfqpoint{-0.034722in}{0.000000in}}%
\pgfpathcurveto{\pgfqpoint{-0.034722in}{-0.009208in}}{\pgfqpoint{-0.031064in}{-0.018041in}}{\pgfqpoint{-0.024552in}{-0.024552in}}%
\pgfpathcurveto{\pgfqpoint{-0.018041in}{-0.031064in}}{\pgfqpoint{-0.009208in}{-0.034722in}}{\pgfqpoint{0.000000in}{-0.034722in}}%
\pgfpathclose%
\pgfusepath{stroke,fill}%
}%
\begin{pgfscope}%
\pgfsys@transformshift{10.059943in}{11.239165in}%
\pgfsys@useobject{currentmarker}{}%
\end{pgfscope}%
\begin{pgfscope}%
\pgfsys@transformshift{10.183166in}{11.237752in}%
\pgfsys@useobject{currentmarker}{}%
\end{pgfscope}%
\begin{pgfscope}%
\pgfsys@transformshift{10.306389in}{10.474090in}%
\pgfsys@useobject{currentmarker}{}%
\end{pgfscope}%
\begin{pgfscope}%
\pgfsys@transformshift{10.429612in}{10.474101in}%
\pgfsys@useobject{currentmarker}{}%
\end{pgfscope}%
\begin{pgfscope}%
\pgfsys@transformshift{10.552835in}{10.462792in}%
\pgfsys@useobject{currentmarker}{}%
\end{pgfscope}%
\begin{pgfscope}%
\pgfsys@transformshift{10.676058in}{10.488750in}%
\pgfsys@useobject{currentmarker}{}%
\end{pgfscope}%
\begin{pgfscope}%
\pgfsys@transformshift{10.799281in}{10.467843in}%
\pgfsys@useobject{currentmarker}{}%
\end{pgfscope}%
\begin{pgfscope}%
\pgfsys@transformshift{10.922504in}{10.477460in}%
\pgfsys@useobject{currentmarker}{}%
\end{pgfscope}%
\begin{pgfscope}%
\pgfsys@transformshift{11.045727in}{10.475896in}%
\pgfsys@useobject{currentmarker}{}%
\end{pgfscope}%
\begin{pgfscope}%
\pgfsys@transformshift{11.168950in}{10.471942in}%
\pgfsys@useobject{currentmarker}{}%
\end{pgfscope}%
\begin{pgfscope}%
\pgfsys@transformshift{11.292173in}{10.479226in}%
\pgfsys@useobject{currentmarker}{}%
\end{pgfscope}%
\begin{pgfscope}%
\pgfsys@transformshift{11.415396in}{10.487917in}%
\pgfsys@useobject{currentmarker}{}%
\end{pgfscope}%
\begin{pgfscope}%
\pgfsys@transformshift{11.538619in}{10.474265in}%
\pgfsys@useobject{currentmarker}{}%
\end{pgfscope}%
\begin{pgfscope}%
\pgfsys@transformshift{11.661842in}{10.475457in}%
\pgfsys@useobject{currentmarker}{}%
\end{pgfscope}%
\begin{pgfscope}%
\pgfsys@transformshift{11.785065in}{10.480883in}%
\pgfsys@useobject{currentmarker}{}%
\end{pgfscope}%
\begin{pgfscope}%
\pgfsys@transformshift{11.908288in}{10.491011in}%
\pgfsys@useobject{currentmarker}{}%
\end{pgfscope}%
\begin{pgfscope}%
\pgfsys@transformshift{12.031511in}{10.490082in}%
\pgfsys@useobject{currentmarker}{}%
\end{pgfscope}%
\begin{pgfscope}%
\pgfsys@transformshift{12.154734in}{10.483925in}%
\pgfsys@useobject{currentmarker}{}%
\end{pgfscope}%
\begin{pgfscope}%
\pgfsys@transformshift{12.277957in}{10.470171in}%
\pgfsys@useobject{currentmarker}{}%
\end{pgfscope}%
\begin{pgfscope}%
\pgfsys@transformshift{12.401180in}{10.488543in}%
\pgfsys@useobject{currentmarker}{}%
\end{pgfscope}%
\begin{pgfscope}%
\pgfsys@transformshift{12.524403in}{10.479145in}%
\pgfsys@useobject{currentmarker}{}%
\end{pgfscope}%
\begin{pgfscope}%
\pgfsys@transformshift{12.647626in}{10.476369in}%
\pgfsys@useobject{currentmarker}{}%
\end{pgfscope}%
\begin{pgfscope}%
\pgfsys@transformshift{12.770849in}{10.468494in}%
\pgfsys@useobject{currentmarker}{}%
\end{pgfscope}%
\begin{pgfscope}%
\pgfsys@transformshift{12.894072in}{10.474579in}%
\pgfsys@useobject{currentmarker}{}%
\end{pgfscope}%
\begin{pgfscope}%
\pgfsys@transformshift{13.017294in}{10.480993in}%
\pgfsys@useobject{currentmarker}{}%
\end{pgfscope}%
\begin{pgfscope}%
\pgfsys@transformshift{13.140517in}{10.486893in}%
\pgfsys@useobject{currentmarker}{}%
\end{pgfscope}%
\begin{pgfscope}%
\pgfsys@transformshift{13.263740in}{10.486707in}%
\pgfsys@useobject{currentmarker}{}%
\end{pgfscope}%
\begin{pgfscope}%
\pgfsys@transformshift{13.386963in}{10.470054in}%
\pgfsys@useobject{currentmarker}{}%
\end{pgfscope}%
\begin{pgfscope}%
\pgfsys@transformshift{13.510186in}{10.471396in}%
\pgfsys@useobject{currentmarker}{}%
\end{pgfscope}%
\begin{pgfscope}%
\pgfsys@transformshift{13.633409in}{10.479059in}%
\pgfsys@useobject{currentmarker}{}%
\end{pgfscope}%
\begin{pgfscope}%
\pgfsys@transformshift{13.756632in}{10.474079in}%
\pgfsys@useobject{currentmarker}{}%
\end{pgfscope}%
\begin{pgfscope}%
\pgfsys@transformshift{13.879855in}{10.465764in}%
\pgfsys@useobject{currentmarker}{}%
\end{pgfscope}%
\begin{pgfscope}%
\pgfsys@transformshift{14.003078in}{10.485554in}%
\pgfsys@useobject{currentmarker}{}%
\end{pgfscope}%
\begin{pgfscope}%
\pgfsys@transformshift{14.126301in}{10.482125in}%
\pgfsys@useobject{currentmarker}{}%
\end{pgfscope}%
\begin{pgfscope}%
\pgfsys@transformshift{14.249524in}{10.484175in}%
\pgfsys@useobject{currentmarker}{}%
\end{pgfscope}%
\begin{pgfscope}%
\pgfsys@transformshift{14.372747in}{10.465506in}%
\pgfsys@useobject{currentmarker}{}%
\end{pgfscope}%
\begin{pgfscope}%
\pgfsys@transformshift{14.495970in}{10.476061in}%
\pgfsys@useobject{currentmarker}{}%
\end{pgfscope}%
\begin{pgfscope}%
\pgfsys@transformshift{14.619193in}{10.487083in}%
\pgfsys@useobject{currentmarker}{}%
\end{pgfscope}%
\begin{pgfscope}%
\pgfsys@transformshift{14.742416in}{10.482670in}%
\pgfsys@useobject{currentmarker}{}%
\end{pgfscope}%
\begin{pgfscope}%
\pgfsys@transformshift{14.865639in}{10.489412in}%
\pgfsys@useobject{currentmarker}{}%
\end{pgfscope}%
\begin{pgfscope}%
\pgfsys@transformshift{14.988862in}{10.478444in}%
\pgfsys@useobject{currentmarker}{}%
\end{pgfscope}%
\end{pgfscope}%
\begin{pgfscope}%
\pgfsetrectcap%
\pgfsetmiterjoin%
\pgfsetlinewidth{0.803000pt}%
\definecolor{currentstroke}{rgb}{1.000000,1.000000,1.000000}%
\pgfsetstrokecolor{currentstroke}%
\pgfsetdash{}{0pt}%
\pgfpathmoveto{\pgfqpoint{9.810417in}{10.401163in}}%
\pgfpathlineto{\pgfqpoint{9.810417in}{11.279070in}}%
\pgfusepath{stroke}%
\end{pgfscope}%
\begin{pgfscope}%
\pgfsetrectcap%
\pgfsetmiterjoin%
\pgfsetlinewidth{0.803000pt}%
\definecolor{currentstroke}{rgb}{1.000000,1.000000,1.000000}%
\pgfsetstrokecolor{currentstroke}%
\pgfsetdash{}{0pt}%
\pgfpathmoveto{\pgfqpoint{15.300000in}{10.401163in}}%
\pgfpathlineto{\pgfqpoint{15.300000in}{11.279070in}}%
\pgfusepath{stroke}%
\end{pgfscope}%
\begin{pgfscope}%
\pgfsetrectcap%
\pgfsetmiterjoin%
\pgfsetlinewidth{0.803000pt}%
\definecolor{currentstroke}{rgb}{1.000000,1.000000,1.000000}%
\pgfsetstrokecolor{currentstroke}%
\pgfsetdash{}{0pt}%
\pgfpathmoveto{\pgfqpoint{9.810417in}{10.401163in}}%
\pgfpathlineto{\pgfqpoint{15.300000in}{10.401163in}}%
\pgfusepath{stroke}%
\end{pgfscope}%
\begin{pgfscope}%
\pgfsetrectcap%
\pgfsetmiterjoin%
\pgfsetlinewidth{0.803000pt}%
\definecolor{currentstroke}{rgb}{1.000000,1.000000,1.000000}%
\pgfsetstrokecolor{currentstroke}%
\pgfsetdash{}{0pt}%
\pgfpathmoveto{\pgfqpoint{9.810417in}{11.279070in}}%
\pgfpathlineto{\pgfqpoint{15.300000in}{11.279070in}}%
\pgfusepath{stroke}%
\end{pgfscope}%
\begin{pgfscope}%
\definecolor{textcolor}{rgb}{0.150000,0.150000,0.150000}%
\pgfsetstrokecolor{textcolor}%
\pgfsetfillcolor{textcolor}%
\pgftext[x=12.555208in,y=11.362403in,,base]{\color{textcolor}\rmfamily\fontsize{16.800000}{20.160000}\selectfont Partial Autocorrelation}%
\end{pgfscope}%
\begin{pgfscope}%
\pgfsetbuttcap%
\pgfsetmiterjoin%
\definecolor{currentfill}{rgb}{0.917647,0.917647,0.949020}%
\pgfsetfillcolor{currentfill}%
\pgfsetlinewidth{0.000000pt}%
\definecolor{currentstroke}{rgb}{0.000000,0.000000,0.000000}%
\pgfsetstrokecolor{currentstroke}%
\pgfsetstrokeopacity{0.000000}%
\pgfsetdash{}{0pt}%
\pgfpathmoveto{\pgfqpoint{2.125000in}{8.820930in}}%
\pgfpathlineto{\pgfqpoint{7.614583in}{8.820930in}}%
\pgfpathlineto{\pgfqpoint{7.614583in}{9.698837in}}%
\pgfpathlineto{\pgfqpoint{2.125000in}{9.698837in}}%
\pgfpathclose%
\pgfusepath{fill}%
\end{pgfscope}%
\begin{pgfscope}%
\pgfpathrectangle{\pgfqpoint{2.125000in}{8.820930in}}{\pgfqpoint{5.489583in}{0.877907in}}%
\pgfusepath{clip}%
\pgfsetroundcap%
\pgfsetroundjoin%
\pgfsetlinewidth{0.803000pt}%
\definecolor{currentstroke}{rgb}{1.000000,1.000000,1.000000}%
\pgfsetstrokecolor{currentstroke}%
\pgfsetdash{}{0pt}%
\pgfpathmoveto{\pgfqpoint{2.374527in}{8.820930in}}%
\pgfpathlineto{\pgfqpoint{2.374527in}{9.698837in}}%
\pgfusepath{stroke}%
\end{pgfscope}%
\begin{pgfscope}%
\definecolor{textcolor}{rgb}{0.150000,0.150000,0.150000}%
\pgfsetstrokecolor{textcolor}%
\pgfsetfillcolor{textcolor}%
\pgftext[x=2.374527in,y=8.723708in,,top]{\color{textcolor}\rmfamily\fontsize{14.000000}{16.800000}\selectfont 0}%
\end{pgfscope}%
\begin{pgfscope}%
\pgfpathrectangle{\pgfqpoint{2.125000in}{8.820930in}}{\pgfqpoint{5.489583in}{0.877907in}}%
\pgfusepath{clip}%
\pgfsetroundcap%
\pgfsetroundjoin%
\pgfsetlinewidth{0.803000pt}%
\definecolor{currentstroke}{rgb}{1.000000,1.000000,1.000000}%
\pgfsetstrokecolor{currentstroke}%
\pgfsetdash{}{0pt}%
\pgfpathmoveto{\pgfqpoint{2.990641in}{8.820930in}}%
\pgfpathlineto{\pgfqpoint{2.990641in}{9.698837in}}%
\pgfusepath{stroke}%
\end{pgfscope}%
\begin{pgfscope}%
\definecolor{textcolor}{rgb}{0.150000,0.150000,0.150000}%
\pgfsetstrokecolor{textcolor}%
\pgfsetfillcolor{textcolor}%
\pgftext[x=2.990641in,y=8.723708in,,top]{\color{textcolor}\rmfamily\fontsize{14.000000}{16.800000}\selectfont 5}%
\end{pgfscope}%
\begin{pgfscope}%
\pgfpathrectangle{\pgfqpoint{2.125000in}{8.820930in}}{\pgfqpoint{5.489583in}{0.877907in}}%
\pgfusepath{clip}%
\pgfsetroundcap%
\pgfsetroundjoin%
\pgfsetlinewidth{0.803000pt}%
\definecolor{currentstroke}{rgb}{1.000000,1.000000,1.000000}%
\pgfsetstrokecolor{currentstroke}%
\pgfsetdash{}{0pt}%
\pgfpathmoveto{\pgfqpoint{3.606756in}{8.820930in}}%
\pgfpathlineto{\pgfqpoint{3.606756in}{9.698837in}}%
\pgfusepath{stroke}%
\end{pgfscope}%
\begin{pgfscope}%
\definecolor{textcolor}{rgb}{0.150000,0.150000,0.150000}%
\pgfsetstrokecolor{textcolor}%
\pgfsetfillcolor{textcolor}%
\pgftext[x=3.606756in,y=8.723708in,,top]{\color{textcolor}\rmfamily\fontsize{14.000000}{16.800000}\selectfont 10}%
\end{pgfscope}%
\begin{pgfscope}%
\pgfpathrectangle{\pgfqpoint{2.125000in}{8.820930in}}{\pgfqpoint{5.489583in}{0.877907in}}%
\pgfusepath{clip}%
\pgfsetroundcap%
\pgfsetroundjoin%
\pgfsetlinewidth{0.803000pt}%
\definecolor{currentstroke}{rgb}{1.000000,1.000000,1.000000}%
\pgfsetstrokecolor{currentstroke}%
\pgfsetdash{}{0pt}%
\pgfpathmoveto{\pgfqpoint{4.222871in}{8.820930in}}%
\pgfpathlineto{\pgfqpoint{4.222871in}{9.698837in}}%
\pgfusepath{stroke}%
\end{pgfscope}%
\begin{pgfscope}%
\definecolor{textcolor}{rgb}{0.150000,0.150000,0.150000}%
\pgfsetstrokecolor{textcolor}%
\pgfsetfillcolor{textcolor}%
\pgftext[x=4.222871in,y=8.723708in,,top]{\color{textcolor}\rmfamily\fontsize{14.000000}{16.800000}\selectfont 15}%
\end{pgfscope}%
\begin{pgfscope}%
\pgfpathrectangle{\pgfqpoint{2.125000in}{8.820930in}}{\pgfqpoint{5.489583in}{0.877907in}}%
\pgfusepath{clip}%
\pgfsetroundcap%
\pgfsetroundjoin%
\pgfsetlinewidth{0.803000pt}%
\definecolor{currentstroke}{rgb}{1.000000,1.000000,1.000000}%
\pgfsetstrokecolor{currentstroke}%
\pgfsetdash{}{0pt}%
\pgfpathmoveto{\pgfqpoint{4.838986in}{8.820930in}}%
\pgfpathlineto{\pgfqpoint{4.838986in}{9.698837in}}%
\pgfusepath{stroke}%
\end{pgfscope}%
\begin{pgfscope}%
\definecolor{textcolor}{rgb}{0.150000,0.150000,0.150000}%
\pgfsetstrokecolor{textcolor}%
\pgfsetfillcolor{textcolor}%
\pgftext[x=4.838986in,y=8.723708in,,top]{\color{textcolor}\rmfamily\fontsize{14.000000}{16.800000}\selectfont 20}%
\end{pgfscope}%
\begin{pgfscope}%
\pgfpathrectangle{\pgfqpoint{2.125000in}{8.820930in}}{\pgfqpoint{5.489583in}{0.877907in}}%
\pgfusepath{clip}%
\pgfsetroundcap%
\pgfsetroundjoin%
\pgfsetlinewidth{0.803000pt}%
\definecolor{currentstroke}{rgb}{1.000000,1.000000,1.000000}%
\pgfsetstrokecolor{currentstroke}%
\pgfsetdash{}{0pt}%
\pgfpathmoveto{\pgfqpoint{5.455101in}{8.820930in}}%
\pgfpathlineto{\pgfqpoint{5.455101in}{9.698837in}}%
\pgfusepath{stroke}%
\end{pgfscope}%
\begin{pgfscope}%
\definecolor{textcolor}{rgb}{0.150000,0.150000,0.150000}%
\pgfsetstrokecolor{textcolor}%
\pgfsetfillcolor{textcolor}%
\pgftext[x=5.455101in,y=8.723708in,,top]{\color{textcolor}\rmfamily\fontsize{14.000000}{16.800000}\selectfont 25}%
\end{pgfscope}%
\begin{pgfscope}%
\pgfpathrectangle{\pgfqpoint{2.125000in}{8.820930in}}{\pgfqpoint{5.489583in}{0.877907in}}%
\pgfusepath{clip}%
\pgfsetroundcap%
\pgfsetroundjoin%
\pgfsetlinewidth{0.803000pt}%
\definecolor{currentstroke}{rgb}{1.000000,1.000000,1.000000}%
\pgfsetstrokecolor{currentstroke}%
\pgfsetdash{}{0pt}%
\pgfpathmoveto{\pgfqpoint{6.071216in}{8.820930in}}%
\pgfpathlineto{\pgfqpoint{6.071216in}{9.698837in}}%
\pgfusepath{stroke}%
\end{pgfscope}%
\begin{pgfscope}%
\definecolor{textcolor}{rgb}{0.150000,0.150000,0.150000}%
\pgfsetstrokecolor{textcolor}%
\pgfsetfillcolor{textcolor}%
\pgftext[x=6.071216in,y=8.723708in,,top]{\color{textcolor}\rmfamily\fontsize{14.000000}{16.800000}\selectfont 30}%
\end{pgfscope}%
\begin{pgfscope}%
\pgfpathrectangle{\pgfqpoint{2.125000in}{8.820930in}}{\pgfqpoint{5.489583in}{0.877907in}}%
\pgfusepath{clip}%
\pgfsetroundcap%
\pgfsetroundjoin%
\pgfsetlinewidth{0.803000pt}%
\definecolor{currentstroke}{rgb}{1.000000,1.000000,1.000000}%
\pgfsetstrokecolor{currentstroke}%
\pgfsetdash{}{0pt}%
\pgfpathmoveto{\pgfqpoint{6.687330in}{8.820930in}}%
\pgfpathlineto{\pgfqpoint{6.687330in}{9.698837in}}%
\pgfusepath{stroke}%
\end{pgfscope}%
\begin{pgfscope}%
\definecolor{textcolor}{rgb}{0.150000,0.150000,0.150000}%
\pgfsetstrokecolor{textcolor}%
\pgfsetfillcolor{textcolor}%
\pgftext[x=6.687330in,y=8.723708in,,top]{\color{textcolor}\rmfamily\fontsize{14.000000}{16.800000}\selectfont 35}%
\end{pgfscope}%
\begin{pgfscope}%
\pgfpathrectangle{\pgfqpoint{2.125000in}{8.820930in}}{\pgfqpoint{5.489583in}{0.877907in}}%
\pgfusepath{clip}%
\pgfsetroundcap%
\pgfsetroundjoin%
\pgfsetlinewidth{0.803000pt}%
\definecolor{currentstroke}{rgb}{1.000000,1.000000,1.000000}%
\pgfsetstrokecolor{currentstroke}%
\pgfsetdash{}{0pt}%
\pgfpathmoveto{\pgfqpoint{7.303445in}{8.820930in}}%
\pgfpathlineto{\pgfqpoint{7.303445in}{9.698837in}}%
\pgfusepath{stroke}%
\end{pgfscope}%
\begin{pgfscope}%
\definecolor{textcolor}{rgb}{0.150000,0.150000,0.150000}%
\pgfsetstrokecolor{textcolor}%
\pgfsetfillcolor{textcolor}%
\pgftext[x=7.303445in,y=8.723708in,,top]{\color{textcolor}\rmfamily\fontsize{14.000000}{16.800000}\selectfont 40}%
\end{pgfscope}%
\begin{pgfscope}%
\pgfpathrectangle{\pgfqpoint{2.125000in}{8.820930in}}{\pgfqpoint{5.489583in}{0.877907in}}%
\pgfusepath{clip}%
\pgfsetroundcap%
\pgfsetroundjoin%
\pgfsetlinewidth{0.803000pt}%
\definecolor{currentstroke}{rgb}{1.000000,1.000000,1.000000}%
\pgfsetstrokecolor{currentstroke}%
\pgfsetdash{}{0pt}%
\pgfpathmoveto{\pgfqpoint{2.125000in}{9.096129in}}%
\pgfpathlineto{\pgfqpoint{7.614583in}{9.096129in}}%
\pgfusepath{stroke}%
\end{pgfscope}%
\begin{pgfscope}%
\definecolor{textcolor}{rgb}{0.150000,0.150000,0.150000}%
\pgfsetstrokecolor{textcolor}%
\pgfsetfillcolor{textcolor}%
\pgftext[x=1.904066in,y=9.022263in,left,base]{\color{textcolor}\rmfamily\fontsize{14.000000}{16.800000}\selectfont 0}%
\end{pgfscope}%
\begin{pgfscope}%
\pgfpathrectangle{\pgfqpoint{2.125000in}{8.820930in}}{\pgfqpoint{5.489583in}{0.877907in}}%
\pgfusepath{clip}%
\pgfsetroundcap%
\pgfsetroundjoin%
\pgfsetlinewidth{0.803000pt}%
\definecolor{currentstroke}{rgb}{1.000000,1.000000,1.000000}%
\pgfsetstrokecolor{currentstroke}%
\pgfsetdash{}{0pt}%
\pgfpathmoveto{\pgfqpoint{2.125000in}{9.658932in}}%
\pgfpathlineto{\pgfqpoint{7.614583in}{9.658932in}}%
\pgfusepath{stroke}%
\end{pgfscope}%
\begin{pgfscope}%
\definecolor{textcolor}{rgb}{0.150000,0.150000,0.150000}%
\pgfsetstrokecolor{textcolor}%
\pgfsetfillcolor{textcolor}%
\pgftext[x=1.904066in,y=9.585066in,left,base]{\color{textcolor}\rmfamily\fontsize{14.000000}{16.800000}\selectfont 1}%
\end{pgfscope}%
\begin{pgfscope}%
\pgfpathrectangle{\pgfqpoint{2.125000in}{8.820930in}}{\pgfqpoint{5.489583in}{0.877907in}}%
\pgfusepath{clip}%
\pgfsetbuttcap%
\pgfsetroundjoin%
\definecolor{currentfill}{rgb}{0.121569,0.466667,0.705882}%
\pgfsetfillcolor{currentfill}%
\pgfsetfillopacity{0.250000}%
\pgfsetlinewidth{1.003750pt}%
\definecolor{currentstroke}{rgb}{1.000000,1.000000,1.000000}%
\pgfsetstrokecolor{currentstroke}%
\pgfsetstrokeopacity{0.250000}%
\pgfsetdash{}{0pt}%
\pgfpathmoveto{\pgfqpoint{2.436138in}{9.124525in}}%
\pgfpathlineto{\pgfqpoint{2.436138in}{9.067732in}}%
\pgfpathlineto{\pgfqpoint{2.620972in}{9.047062in}}%
\pgfpathlineto{\pgfqpoint{2.744195in}{9.032904in}}%
\pgfpathlineto{\pgfqpoint{2.867418in}{9.021455in}}%
\pgfpathlineto{\pgfqpoint{2.990641in}{9.011608in}}%
\pgfpathlineto{\pgfqpoint{3.113864in}{9.002850in}}%
\pgfpathlineto{\pgfqpoint{3.237087in}{8.994898in}}%
\pgfpathlineto{\pgfqpoint{3.360310in}{8.987579in}}%
\pgfpathlineto{\pgfqpoint{3.483533in}{8.980771in}}%
\pgfpathlineto{\pgfqpoint{3.606756in}{8.974389in}}%
\pgfpathlineto{\pgfqpoint{3.729979in}{8.968367in}}%
\pgfpathlineto{\pgfqpoint{3.853202in}{8.962655in}}%
\pgfpathlineto{\pgfqpoint{3.976425in}{8.957214in}}%
\pgfpathlineto{\pgfqpoint{4.099648in}{8.952012in}}%
\pgfpathlineto{\pgfqpoint{4.222871in}{8.947026in}}%
\pgfpathlineto{\pgfqpoint{4.346094in}{8.942235in}}%
\pgfpathlineto{\pgfqpoint{4.469317in}{8.937620in}}%
\pgfpathlineto{\pgfqpoint{4.592540in}{8.933167in}}%
\pgfpathlineto{\pgfqpoint{4.715763in}{8.928867in}}%
\pgfpathlineto{\pgfqpoint{4.838986in}{8.924707in}}%
\pgfpathlineto{\pgfqpoint{4.962209in}{8.920678in}}%
\pgfpathlineto{\pgfqpoint{5.085432in}{8.916769in}}%
\pgfpathlineto{\pgfqpoint{5.208655in}{8.912974in}}%
\pgfpathlineto{\pgfqpoint{5.331878in}{8.909285in}}%
\pgfpathlineto{\pgfqpoint{5.455101in}{8.905696in}}%
\pgfpathlineto{\pgfqpoint{5.578324in}{8.902202in}}%
\pgfpathlineto{\pgfqpoint{5.701547in}{8.898795in}}%
\pgfpathlineto{\pgfqpoint{5.824770in}{8.895473in}}%
\pgfpathlineto{\pgfqpoint{5.947993in}{8.892229in}}%
\pgfpathlineto{\pgfqpoint{6.071216in}{8.889062in}}%
\pgfpathlineto{\pgfqpoint{6.194439in}{8.885967in}}%
\pgfpathlineto{\pgfqpoint{6.317662in}{8.882942in}}%
\pgfpathlineto{\pgfqpoint{6.440885in}{8.879983in}}%
\pgfpathlineto{\pgfqpoint{6.564108in}{8.877086in}}%
\pgfpathlineto{\pgfqpoint{6.687330in}{8.874248in}}%
\pgfpathlineto{\pgfqpoint{6.810553in}{8.871466in}}%
\pgfpathlineto{\pgfqpoint{6.933776in}{8.868736in}}%
\pgfpathlineto{\pgfqpoint{7.056999in}{8.866056in}}%
\pgfpathlineto{\pgfqpoint{7.180222in}{8.863423in}}%
\pgfpathlineto{\pgfqpoint{7.365057in}{8.860835in}}%
\pgfpathlineto{\pgfqpoint{7.365057in}{9.331422in}}%
\pgfpathlineto{\pgfqpoint{7.365057in}{9.331422in}}%
\pgfpathlineto{\pgfqpoint{7.180222in}{9.328834in}}%
\pgfpathlineto{\pgfqpoint{7.056999in}{9.326201in}}%
\pgfpathlineto{\pgfqpoint{6.933776in}{9.323521in}}%
\pgfpathlineto{\pgfqpoint{6.810553in}{9.320791in}}%
\pgfpathlineto{\pgfqpoint{6.687330in}{9.318010in}}%
\pgfpathlineto{\pgfqpoint{6.564108in}{9.315172in}}%
\pgfpathlineto{\pgfqpoint{6.440885in}{9.312275in}}%
\pgfpathlineto{\pgfqpoint{6.317662in}{9.309315in}}%
\pgfpathlineto{\pgfqpoint{6.194439in}{9.306290in}}%
\pgfpathlineto{\pgfqpoint{6.071216in}{9.303196in}}%
\pgfpathlineto{\pgfqpoint{5.947993in}{9.300028in}}%
\pgfpathlineto{\pgfqpoint{5.824770in}{9.296784in}}%
\pgfpathlineto{\pgfqpoint{5.701547in}{9.293462in}}%
\pgfpathlineto{\pgfqpoint{5.578324in}{9.290056in}}%
\pgfpathlineto{\pgfqpoint{5.455101in}{9.286561in}}%
\pgfpathlineto{\pgfqpoint{5.331878in}{9.282972in}}%
\pgfpathlineto{\pgfqpoint{5.208655in}{9.279283in}}%
\pgfpathlineto{\pgfqpoint{5.085432in}{9.275488in}}%
\pgfpathlineto{\pgfqpoint{4.962209in}{9.271580in}}%
\pgfpathlineto{\pgfqpoint{4.838986in}{9.267550in}}%
\pgfpathlineto{\pgfqpoint{4.715763in}{9.263391in}}%
\pgfpathlineto{\pgfqpoint{4.592540in}{9.259090in}}%
\pgfpathlineto{\pgfqpoint{4.469317in}{9.254638in}}%
\pgfpathlineto{\pgfqpoint{4.346094in}{9.250022in}}%
\pgfpathlineto{\pgfqpoint{4.222871in}{9.245231in}}%
\pgfpathlineto{\pgfqpoint{4.099648in}{9.240245in}}%
\pgfpathlineto{\pgfqpoint{3.976425in}{9.235043in}}%
\pgfpathlineto{\pgfqpoint{3.853202in}{9.229602in}}%
\pgfpathlineto{\pgfqpoint{3.729979in}{9.223890in}}%
\pgfpathlineto{\pgfqpoint{3.606756in}{9.217868in}}%
\pgfpathlineto{\pgfqpoint{3.483533in}{9.211486in}}%
\pgfpathlineto{\pgfqpoint{3.360310in}{9.204678in}}%
\pgfpathlineto{\pgfqpoint{3.237087in}{9.197359in}}%
\pgfpathlineto{\pgfqpoint{3.113864in}{9.189408in}}%
\pgfpathlineto{\pgfqpoint{2.990641in}{9.180649in}}%
\pgfpathlineto{\pgfqpoint{2.867418in}{9.170802in}}%
\pgfpathlineto{\pgfqpoint{2.744195in}{9.159353in}}%
\pgfpathlineto{\pgfqpoint{2.620972in}{9.145195in}}%
\pgfpathlineto{\pgfqpoint{2.436138in}{9.124525in}}%
\pgfpathclose%
\pgfusepath{stroke,fill}%
\end{pgfscope}%
\begin{pgfscope}%
\pgfpathrectangle{\pgfqpoint{2.125000in}{8.820930in}}{\pgfqpoint{5.489583in}{0.877907in}}%
\pgfusepath{clip}%
\pgfsetbuttcap%
\pgfsetroundjoin%
\pgfsetlinewidth{1.505625pt}%
\definecolor{currentstroke}{rgb}{0.000000,0.000000,0.000000}%
\pgfsetstrokecolor{currentstroke}%
\pgfsetdash{}{0pt}%
\pgfpathmoveto{\pgfqpoint{2.374527in}{9.096129in}}%
\pgfpathlineto{\pgfqpoint{2.374527in}{9.658932in}}%
\pgfusepath{stroke}%
\end{pgfscope}%
\begin{pgfscope}%
\pgfpathrectangle{\pgfqpoint{2.125000in}{8.820930in}}{\pgfqpoint{5.489583in}{0.877907in}}%
\pgfusepath{clip}%
\pgfsetbuttcap%
\pgfsetroundjoin%
\pgfsetlinewidth{1.505625pt}%
\definecolor{currentstroke}{rgb}{0.000000,0.000000,0.000000}%
\pgfsetstrokecolor{currentstroke}%
\pgfsetdash{}{0pt}%
\pgfpathmoveto{\pgfqpoint{2.497749in}{9.096129in}}%
\pgfpathlineto{\pgfqpoint{2.497749in}{9.656919in}}%
\pgfusepath{stroke}%
\end{pgfscope}%
\begin{pgfscope}%
\pgfpathrectangle{\pgfqpoint{2.125000in}{8.820930in}}{\pgfqpoint{5.489583in}{0.877907in}}%
\pgfusepath{clip}%
\pgfsetbuttcap%
\pgfsetroundjoin%
\pgfsetlinewidth{1.505625pt}%
\definecolor{currentstroke}{rgb}{0.000000,0.000000,0.000000}%
\pgfsetstrokecolor{currentstroke}%
\pgfsetdash{}{0pt}%
\pgfpathmoveto{\pgfqpoint{2.620972in}{9.096129in}}%
\pgfpathlineto{\pgfqpoint{2.620972in}{9.654932in}}%
\pgfusepath{stroke}%
\end{pgfscope}%
\begin{pgfscope}%
\pgfpathrectangle{\pgfqpoint{2.125000in}{8.820930in}}{\pgfqpoint{5.489583in}{0.877907in}}%
\pgfusepath{clip}%
\pgfsetbuttcap%
\pgfsetroundjoin%
\pgfsetlinewidth{1.505625pt}%
\definecolor{currentstroke}{rgb}{0.000000,0.000000,0.000000}%
\pgfsetstrokecolor{currentstroke}%
\pgfsetdash{}{0pt}%
\pgfpathmoveto{\pgfqpoint{2.744195in}{9.096129in}}%
\pgfpathlineto{\pgfqpoint{2.744195in}{9.652972in}}%
\pgfusepath{stroke}%
\end{pgfscope}%
\begin{pgfscope}%
\pgfpathrectangle{\pgfqpoint{2.125000in}{8.820930in}}{\pgfqpoint{5.489583in}{0.877907in}}%
\pgfusepath{clip}%
\pgfsetbuttcap%
\pgfsetroundjoin%
\pgfsetlinewidth{1.505625pt}%
\definecolor{currentstroke}{rgb}{0.000000,0.000000,0.000000}%
\pgfsetstrokecolor{currentstroke}%
\pgfsetdash{}{0pt}%
\pgfpathmoveto{\pgfqpoint{2.867418in}{9.096129in}}%
\pgfpathlineto{\pgfqpoint{2.867418in}{9.651012in}}%
\pgfusepath{stroke}%
\end{pgfscope}%
\begin{pgfscope}%
\pgfpathrectangle{\pgfqpoint{2.125000in}{8.820930in}}{\pgfqpoint{5.489583in}{0.877907in}}%
\pgfusepath{clip}%
\pgfsetbuttcap%
\pgfsetroundjoin%
\pgfsetlinewidth{1.505625pt}%
\definecolor{currentstroke}{rgb}{0.000000,0.000000,0.000000}%
\pgfsetstrokecolor{currentstroke}%
\pgfsetdash{}{0pt}%
\pgfpathmoveto{\pgfqpoint{2.990641in}{9.096129in}}%
\pgfpathlineto{\pgfqpoint{2.990641in}{9.649181in}}%
\pgfusepath{stroke}%
\end{pgfscope}%
\begin{pgfscope}%
\pgfpathrectangle{\pgfqpoint{2.125000in}{8.820930in}}{\pgfqpoint{5.489583in}{0.877907in}}%
\pgfusepath{clip}%
\pgfsetbuttcap%
\pgfsetroundjoin%
\pgfsetlinewidth{1.505625pt}%
\definecolor{currentstroke}{rgb}{0.000000,0.000000,0.000000}%
\pgfsetstrokecolor{currentstroke}%
\pgfsetdash{}{0pt}%
\pgfpathmoveto{\pgfqpoint{3.113864in}{9.096129in}}%
\pgfpathlineto{\pgfqpoint{3.113864in}{9.647275in}}%
\pgfusepath{stroke}%
\end{pgfscope}%
\begin{pgfscope}%
\pgfpathrectangle{\pgfqpoint{2.125000in}{8.820930in}}{\pgfqpoint{5.489583in}{0.877907in}}%
\pgfusepath{clip}%
\pgfsetbuttcap%
\pgfsetroundjoin%
\pgfsetlinewidth{1.505625pt}%
\definecolor{currentstroke}{rgb}{0.000000,0.000000,0.000000}%
\pgfsetstrokecolor{currentstroke}%
\pgfsetdash{}{0pt}%
\pgfpathmoveto{\pgfqpoint{3.237087in}{9.096129in}}%
\pgfpathlineto{\pgfqpoint{3.237087in}{9.645299in}}%
\pgfusepath{stroke}%
\end{pgfscope}%
\begin{pgfscope}%
\pgfpathrectangle{\pgfqpoint{2.125000in}{8.820930in}}{\pgfqpoint{5.489583in}{0.877907in}}%
\pgfusepath{clip}%
\pgfsetbuttcap%
\pgfsetroundjoin%
\pgfsetlinewidth{1.505625pt}%
\definecolor{currentstroke}{rgb}{0.000000,0.000000,0.000000}%
\pgfsetstrokecolor{currentstroke}%
\pgfsetdash{}{0pt}%
\pgfpathmoveto{\pgfqpoint{3.360310in}{9.096129in}}%
\pgfpathlineto{\pgfqpoint{3.360310in}{9.643302in}}%
\pgfusepath{stroke}%
\end{pgfscope}%
\begin{pgfscope}%
\pgfpathrectangle{\pgfqpoint{2.125000in}{8.820930in}}{\pgfqpoint{5.489583in}{0.877907in}}%
\pgfusepath{clip}%
\pgfsetbuttcap%
\pgfsetroundjoin%
\pgfsetlinewidth{1.505625pt}%
\definecolor{currentstroke}{rgb}{0.000000,0.000000,0.000000}%
\pgfsetstrokecolor{currentstroke}%
\pgfsetdash{}{0pt}%
\pgfpathmoveto{\pgfqpoint{3.483533in}{9.096129in}}%
\pgfpathlineto{\pgfqpoint{3.483533in}{9.641283in}}%
\pgfusepath{stroke}%
\end{pgfscope}%
\begin{pgfscope}%
\pgfpathrectangle{\pgfqpoint{2.125000in}{8.820930in}}{\pgfqpoint{5.489583in}{0.877907in}}%
\pgfusepath{clip}%
\pgfsetbuttcap%
\pgfsetroundjoin%
\pgfsetlinewidth{1.505625pt}%
\definecolor{currentstroke}{rgb}{0.000000,0.000000,0.000000}%
\pgfsetstrokecolor{currentstroke}%
\pgfsetdash{}{0pt}%
\pgfpathmoveto{\pgfqpoint{3.606756in}{9.096129in}}%
\pgfpathlineto{\pgfqpoint{3.606756in}{9.639340in}}%
\pgfusepath{stroke}%
\end{pgfscope}%
\begin{pgfscope}%
\pgfpathrectangle{\pgfqpoint{2.125000in}{8.820930in}}{\pgfqpoint{5.489583in}{0.877907in}}%
\pgfusepath{clip}%
\pgfsetbuttcap%
\pgfsetroundjoin%
\pgfsetlinewidth{1.505625pt}%
\definecolor{currentstroke}{rgb}{0.000000,0.000000,0.000000}%
\pgfsetstrokecolor{currentstroke}%
\pgfsetdash{}{0pt}%
\pgfpathmoveto{\pgfqpoint{3.729979in}{9.096129in}}%
\pgfpathlineto{\pgfqpoint{3.729979in}{9.637500in}}%
\pgfusepath{stroke}%
\end{pgfscope}%
\begin{pgfscope}%
\pgfpathrectangle{\pgfqpoint{2.125000in}{8.820930in}}{\pgfqpoint{5.489583in}{0.877907in}}%
\pgfusepath{clip}%
\pgfsetbuttcap%
\pgfsetroundjoin%
\pgfsetlinewidth{1.505625pt}%
\definecolor{currentstroke}{rgb}{0.000000,0.000000,0.000000}%
\pgfsetstrokecolor{currentstroke}%
\pgfsetdash{}{0pt}%
\pgfpathmoveto{\pgfqpoint{3.853202in}{9.096129in}}%
\pgfpathlineto{\pgfqpoint{3.853202in}{9.635664in}}%
\pgfusepath{stroke}%
\end{pgfscope}%
\begin{pgfscope}%
\pgfpathrectangle{\pgfqpoint{2.125000in}{8.820930in}}{\pgfqpoint{5.489583in}{0.877907in}}%
\pgfusepath{clip}%
\pgfsetbuttcap%
\pgfsetroundjoin%
\pgfsetlinewidth{1.505625pt}%
\definecolor{currentstroke}{rgb}{0.000000,0.000000,0.000000}%
\pgfsetstrokecolor{currentstroke}%
\pgfsetdash{}{0pt}%
\pgfpathmoveto{\pgfqpoint{3.976425in}{9.096129in}}%
\pgfpathlineto{\pgfqpoint{3.976425in}{9.633884in}}%
\pgfusepath{stroke}%
\end{pgfscope}%
\begin{pgfscope}%
\pgfpathrectangle{\pgfqpoint{2.125000in}{8.820930in}}{\pgfqpoint{5.489583in}{0.877907in}}%
\pgfusepath{clip}%
\pgfsetbuttcap%
\pgfsetroundjoin%
\pgfsetlinewidth{1.505625pt}%
\definecolor{currentstroke}{rgb}{0.000000,0.000000,0.000000}%
\pgfsetstrokecolor{currentstroke}%
\pgfsetdash{}{0pt}%
\pgfpathmoveto{\pgfqpoint{4.099648in}{9.096129in}}%
\pgfpathlineto{\pgfqpoint{4.099648in}{9.632011in}}%
\pgfusepath{stroke}%
\end{pgfscope}%
\begin{pgfscope}%
\pgfpathrectangle{\pgfqpoint{2.125000in}{8.820930in}}{\pgfqpoint{5.489583in}{0.877907in}}%
\pgfusepath{clip}%
\pgfsetbuttcap%
\pgfsetroundjoin%
\pgfsetlinewidth{1.505625pt}%
\definecolor{currentstroke}{rgb}{0.000000,0.000000,0.000000}%
\pgfsetstrokecolor{currentstroke}%
\pgfsetdash{}{0pt}%
\pgfpathmoveto{\pgfqpoint{4.222871in}{9.096129in}}%
\pgfpathlineto{\pgfqpoint{4.222871in}{9.630098in}}%
\pgfusepath{stroke}%
\end{pgfscope}%
\begin{pgfscope}%
\pgfpathrectangle{\pgfqpoint{2.125000in}{8.820930in}}{\pgfqpoint{5.489583in}{0.877907in}}%
\pgfusepath{clip}%
\pgfsetbuttcap%
\pgfsetroundjoin%
\pgfsetlinewidth{1.505625pt}%
\definecolor{currentstroke}{rgb}{0.000000,0.000000,0.000000}%
\pgfsetstrokecolor{currentstroke}%
\pgfsetdash{}{0pt}%
\pgfpathmoveto{\pgfqpoint{4.346094in}{9.096129in}}%
\pgfpathlineto{\pgfqpoint{4.346094in}{9.628279in}}%
\pgfusepath{stroke}%
\end{pgfscope}%
\begin{pgfscope}%
\pgfpathrectangle{\pgfqpoint{2.125000in}{8.820930in}}{\pgfqpoint{5.489583in}{0.877907in}}%
\pgfusepath{clip}%
\pgfsetbuttcap%
\pgfsetroundjoin%
\pgfsetlinewidth{1.505625pt}%
\definecolor{currentstroke}{rgb}{0.000000,0.000000,0.000000}%
\pgfsetstrokecolor{currentstroke}%
\pgfsetdash{}{0pt}%
\pgfpathmoveto{\pgfqpoint{4.469317in}{9.096129in}}%
\pgfpathlineto{\pgfqpoint{4.469317in}{9.626323in}}%
\pgfusepath{stroke}%
\end{pgfscope}%
\begin{pgfscope}%
\pgfpathrectangle{\pgfqpoint{2.125000in}{8.820930in}}{\pgfqpoint{5.489583in}{0.877907in}}%
\pgfusepath{clip}%
\pgfsetbuttcap%
\pgfsetroundjoin%
\pgfsetlinewidth{1.505625pt}%
\definecolor{currentstroke}{rgb}{0.000000,0.000000,0.000000}%
\pgfsetstrokecolor{currentstroke}%
\pgfsetdash{}{0pt}%
\pgfpathmoveto{\pgfqpoint{4.592540in}{9.096129in}}%
\pgfpathlineto{\pgfqpoint{4.592540in}{9.624287in}}%
\pgfusepath{stroke}%
\end{pgfscope}%
\begin{pgfscope}%
\pgfpathrectangle{\pgfqpoint{2.125000in}{8.820930in}}{\pgfqpoint{5.489583in}{0.877907in}}%
\pgfusepath{clip}%
\pgfsetbuttcap%
\pgfsetroundjoin%
\pgfsetlinewidth{1.505625pt}%
\definecolor{currentstroke}{rgb}{0.000000,0.000000,0.000000}%
\pgfsetstrokecolor{currentstroke}%
\pgfsetdash{}{0pt}%
\pgfpathmoveto{\pgfqpoint{4.715763in}{9.096129in}}%
\pgfpathlineto{\pgfqpoint{4.715763in}{9.622164in}}%
\pgfusepath{stroke}%
\end{pgfscope}%
\begin{pgfscope}%
\pgfpathrectangle{\pgfqpoint{2.125000in}{8.820930in}}{\pgfqpoint{5.489583in}{0.877907in}}%
\pgfusepath{clip}%
\pgfsetbuttcap%
\pgfsetroundjoin%
\pgfsetlinewidth{1.505625pt}%
\definecolor{currentstroke}{rgb}{0.000000,0.000000,0.000000}%
\pgfsetstrokecolor{currentstroke}%
\pgfsetdash{}{0pt}%
\pgfpathmoveto{\pgfqpoint{4.838986in}{9.096129in}}%
\pgfpathlineto{\pgfqpoint{4.838986in}{9.620059in}}%
\pgfusepath{stroke}%
\end{pgfscope}%
\begin{pgfscope}%
\pgfpathrectangle{\pgfqpoint{2.125000in}{8.820930in}}{\pgfqpoint{5.489583in}{0.877907in}}%
\pgfusepath{clip}%
\pgfsetbuttcap%
\pgfsetroundjoin%
\pgfsetlinewidth{1.505625pt}%
\definecolor{currentstroke}{rgb}{0.000000,0.000000,0.000000}%
\pgfsetstrokecolor{currentstroke}%
\pgfsetdash{}{0pt}%
\pgfpathmoveto{\pgfqpoint{4.962209in}{9.096129in}}%
\pgfpathlineto{\pgfqpoint{4.962209in}{9.618016in}}%
\pgfusepath{stroke}%
\end{pgfscope}%
\begin{pgfscope}%
\pgfpathrectangle{\pgfqpoint{2.125000in}{8.820930in}}{\pgfqpoint{5.489583in}{0.877907in}}%
\pgfusepath{clip}%
\pgfsetbuttcap%
\pgfsetroundjoin%
\pgfsetlinewidth{1.505625pt}%
\definecolor{currentstroke}{rgb}{0.000000,0.000000,0.000000}%
\pgfsetstrokecolor{currentstroke}%
\pgfsetdash{}{0pt}%
\pgfpathmoveto{\pgfqpoint{5.085432in}{9.096129in}}%
\pgfpathlineto{\pgfqpoint{5.085432in}{9.615977in}}%
\pgfusepath{stroke}%
\end{pgfscope}%
\begin{pgfscope}%
\pgfpathrectangle{\pgfqpoint{2.125000in}{8.820930in}}{\pgfqpoint{5.489583in}{0.877907in}}%
\pgfusepath{clip}%
\pgfsetbuttcap%
\pgfsetroundjoin%
\pgfsetlinewidth{1.505625pt}%
\definecolor{currentstroke}{rgb}{0.000000,0.000000,0.000000}%
\pgfsetstrokecolor{currentstroke}%
\pgfsetdash{}{0pt}%
\pgfpathmoveto{\pgfqpoint{5.208655in}{9.096129in}}%
\pgfpathlineto{\pgfqpoint{5.208655in}{9.613882in}}%
\pgfusepath{stroke}%
\end{pgfscope}%
\begin{pgfscope}%
\pgfpathrectangle{\pgfqpoint{2.125000in}{8.820930in}}{\pgfqpoint{5.489583in}{0.877907in}}%
\pgfusepath{clip}%
\pgfsetbuttcap%
\pgfsetroundjoin%
\pgfsetlinewidth{1.505625pt}%
\definecolor{currentstroke}{rgb}{0.000000,0.000000,0.000000}%
\pgfsetstrokecolor{currentstroke}%
\pgfsetdash{}{0pt}%
\pgfpathmoveto{\pgfqpoint{5.331878in}{9.096129in}}%
\pgfpathlineto{\pgfqpoint{5.331878in}{9.611825in}}%
\pgfusepath{stroke}%
\end{pgfscope}%
\begin{pgfscope}%
\pgfpathrectangle{\pgfqpoint{2.125000in}{8.820930in}}{\pgfqpoint{5.489583in}{0.877907in}}%
\pgfusepath{clip}%
\pgfsetbuttcap%
\pgfsetroundjoin%
\pgfsetlinewidth{1.505625pt}%
\definecolor{currentstroke}{rgb}{0.000000,0.000000,0.000000}%
\pgfsetstrokecolor{currentstroke}%
\pgfsetdash{}{0pt}%
\pgfpathmoveto{\pgfqpoint{5.455101in}{9.096129in}}%
\pgfpathlineto{\pgfqpoint{5.455101in}{9.609737in}}%
\pgfusepath{stroke}%
\end{pgfscope}%
\begin{pgfscope}%
\pgfpathrectangle{\pgfqpoint{2.125000in}{8.820930in}}{\pgfqpoint{5.489583in}{0.877907in}}%
\pgfusepath{clip}%
\pgfsetbuttcap%
\pgfsetroundjoin%
\pgfsetlinewidth{1.505625pt}%
\definecolor{currentstroke}{rgb}{0.000000,0.000000,0.000000}%
\pgfsetstrokecolor{currentstroke}%
\pgfsetdash{}{0pt}%
\pgfpathmoveto{\pgfqpoint{5.578324in}{9.096129in}}%
\pgfpathlineto{\pgfqpoint{5.578324in}{9.607754in}}%
\pgfusepath{stroke}%
\end{pgfscope}%
\begin{pgfscope}%
\pgfpathrectangle{\pgfqpoint{2.125000in}{8.820930in}}{\pgfqpoint{5.489583in}{0.877907in}}%
\pgfusepath{clip}%
\pgfsetbuttcap%
\pgfsetroundjoin%
\pgfsetlinewidth{1.505625pt}%
\definecolor{currentstroke}{rgb}{0.000000,0.000000,0.000000}%
\pgfsetstrokecolor{currentstroke}%
\pgfsetdash{}{0pt}%
\pgfpathmoveto{\pgfqpoint{5.701547in}{9.096129in}}%
\pgfpathlineto{\pgfqpoint{5.701547in}{9.605766in}}%
\pgfusepath{stroke}%
\end{pgfscope}%
\begin{pgfscope}%
\pgfpathrectangle{\pgfqpoint{2.125000in}{8.820930in}}{\pgfqpoint{5.489583in}{0.877907in}}%
\pgfusepath{clip}%
\pgfsetbuttcap%
\pgfsetroundjoin%
\pgfsetlinewidth{1.505625pt}%
\definecolor{currentstroke}{rgb}{0.000000,0.000000,0.000000}%
\pgfsetstrokecolor{currentstroke}%
\pgfsetdash{}{0pt}%
\pgfpathmoveto{\pgfqpoint{5.824770in}{9.096129in}}%
\pgfpathlineto{\pgfqpoint{5.824770in}{9.603797in}}%
\pgfusepath{stroke}%
\end{pgfscope}%
\begin{pgfscope}%
\pgfpathrectangle{\pgfqpoint{2.125000in}{8.820930in}}{\pgfqpoint{5.489583in}{0.877907in}}%
\pgfusepath{clip}%
\pgfsetbuttcap%
\pgfsetroundjoin%
\pgfsetlinewidth{1.505625pt}%
\definecolor{currentstroke}{rgb}{0.000000,0.000000,0.000000}%
\pgfsetstrokecolor{currentstroke}%
\pgfsetdash{}{0pt}%
\pgfpathmoveto{\pgfqpoint{5.947993in}{9.096129in}}%
\pgfpathlineto{\pgfqpoint{5.947993in}{9.601791in}}%
\pgfusepath{stroke}%
\end{pgfscope}%
\begin{pgfscope}%
\pgfpathrectangle{\pgfqpoint{2.125000in}{8.820930in}}{\pgfqpoint{5.489583in}{0.877907in}}%
\pgfusepath{clip}%
\pgfsetbuttcap%
\pgfsetroundjoin%
\pgfsetlinewidth{1.505625pt}%
\definecolor{currentstroke}{rgb}{0.000000,0.000000,0.000000}%
\pgfsetstrokecolor{currentstroke}%
\pgfsetdash{}{0pt}%
\pgfpathmoveto{\pgfqpoint{6.071216in}{9.096129in}}%
\pgfpathlineto{\pgfqpoint{6.071216in}{9.599677in}}%
\pgfusepath{stroke}%
\end{pgfscope}%
\begin{pgfscope}%
\pgfpathrectangle{\pgfqpoint{2.125000in}{8.820930in}}{\pgfqpoint{5.489583in}{0.877907in}}%
\pgfusepath{clip}%
\pgfsetbuttcap%
\pgfsetroundjoin%
\pgfsetlinewidth{1.505625pt}%
\definecolor{currentstroke}{rgb}{0.000000,0.000000,0.000000}%
\pgfsetstrokecolor{currentstroke}%
\pgfsetdash{}{0pt}%
\pgfpathmoveto{\pgfqpoint{6.194439in}{9.096129in}}%
\pgfpathlineto{\pgfqpoint{6.194439in}{9.597661in}}%
\pgfusepath{stroke}%
\end{pgfscope}%
\begin{pgfscope}%
\pgfpathrectangle{\pgfqpoint{2.125000in}{8.820930in}}{\pgfqpoint{5.489583in}{0.877907in}}%
\pgfusepath{clip}%
\pgfsetbuttcap%
\pgfsetroundjoin%
\pgfsetlinewidth{1.505625pt}%
\definecolor{currentstroke}{rgb}{0.000000,0.000000,0.000000}%
\pgfsetstrokecolor{currentstroke}%
\pgfsetdash{}{0pt}%
\pgfpathmoveto{\pgfqpoint{6.317662in}{9.096129in}}%
\pgfpathlineto{\pgfqpoint{6.317662in}{9.595690in}}%
\pgfusepath{stroke}%
\end{pgfscope}%
\begin{pgfscope}%
\pgfpathrectangle{\pgfqpoint{2.125000in}{8.820930in}}{\pgfqpoint{5.489583in}{0.877907in}}%
\pgfusepath{clip}%
\pgfsetbuttcap%
\pgfsetroundjoin%
\pgfsetlinewidth{1.505625pt}%
\definecolor{currentstroke}{rgb}{0.000000,0.000000,0.000000}%
\pgfsetstrokecolor{currentstroke}%
\pgfsetdash{}{0pt}%
\pgfpathmoveto{\pgfqpoint{6.440885in}{9.096129in}}%
\pgfpathlineto{\pgfqpoint{6.440885in}{9.593759in}}%
\pgfusepath{stroke}%
\end{pgfscope}%
\begin{pgfscope}%
\pgfpathrectangle{\pgfqpoint{2.125000in}{8.820930in}}{\pgfqpoint{5.489583in}{0.877907in}}%
\pgfusepath{clip}%
\pgfsetbuttcap%
\pgfsetroundjoin%
\pgfsetlinewidth{1.505625pt}%
\definecolor{currentstroke}{rgb}{0.000000,0.000000,0.000000}%
\pgfsetstrokecolor{currentstroke}%
\pgfsetdash{}{0pt}%
\pgfpathmoveto{\pgfqpoint{6.564108in}{9.096129in}}%
\pgfpathlineto{\pgfqpoint{6.564108in}{9.591879in}}%
\pgfusepath{stroke}%
\end{pgfscope}%
\begin{pgfscope}%
\pgfpathrectangle{\pgfqpoint{2.125000in}{8.820930in}}{\pgfqpoint{5.489583in}{0.877907in}}%
\pgfusepath{clip}%
\pgfsetbuttcap%
\pgfsetroundjoin%
\pgfsetlinewidth{1.505625pt}%
\definecolor{currentstroke}{rgb}{0.000000,0.000000,0.000000}%
\pgfsetstrokecolor{currentstroke}%
\pgfsetdash{}{0pt}%
\pgfpathmoveto{\pgfqpoint{6.687330in}{9.096129in}}%
\pgfpathlineto{\pgfqpoint{6.687330in}{9.590064in}}%
\pgfusepath{stroke}%
\end{pgfscope}%
\begin{pgfscope}%
\pgfpathrectangle{\pgfqpoint{2.125000in}{8.820930in}}{\pgfqpoint{5.489583in}{0.877907in}}%
\pgfusepath{clip}%
\pgfsetbuttcap%
\pgfsetroundjoin%
\pgfsetlinewidth{1.505625pt}%
\definecolor{currentstroke}{rgb}{0.000000,0.000000,0.000000}%
\pgfsetstrokecolor{currentstroke}%
\pgfsetdash{}{0pt}%
\pgfpathmoveto{\pgfqpoint{6.810553in}{9.096129in}}%
\pgfpathlineto{\pgfqpoint{6.810553in}{9.588455in}}%
\pgfusepath{stroke}%
\end{pgfscope}%
\begin{pgfscope}%
\pgfpathrectangle{\pgfqpoint{2.125000in}{8.820930in}}{\pgfqpoint{5.489583in}{0.877907in}}%
\pgfusepath{clip}%
\pgfsetbuttcap%
\pgfsetroundjoin%
\pgfsetlinewidth{1.505625pt}%
\definecolor{currentstroke}{rgb}{0.000000,0.000000,0.000000}%
\pgfsetstrokecolor{currentstroke}%
\pgfsetdash{}{0pt}%
\pgfpathmoveto{\pgfqpoint{6.933776in}{9.096129in}}%
\pgfpathlineto{\pgfqpoint{6.933776in}{9.586843in}}%
\pgfusepath{stroke}%
\end{pgfscope}%
\begin{pgfscope}%
\pgfpathrectangle{\pgfqpoint{2.125000in}{8.820930in}}{\pgfqpoint{5.489583in}{0.877907in}}%
\pgfusepath{clip}%
\pgfsetbuttcap%
\pgfsetroundjoin%
\pgfsetlinewidth{1.505625pt}%
\definecolor{currentstroke}{rgb}{0.000000,0.000000,0.000000}%
\pgfsetstrokecolor{currentstroke}%
\pgfsetdash{}{0pt}%
\pgfpathmoveto{\pgfqpoint{7.056999in}{9.096129in}}%
\pgfpathlineto{\pgfqpoint{7.056999in}{9.585317in}}%
\pgfusepath{stroke}%
\end{pgfscope}%
\begin{pgfscope}%
\pgfpathrectangle{\pgfqpoint{2.125000in}{8.820930in}}{\pgfqpoint{5.489583in}{0.877907in}}%
\pgfusepath{clip}%
\pgfsetbuttcap%
\pgfsetroundjoin%
\pgfsetlinewidth{1.505625pt}%
\definecolor{currentstroke}{rgb}{0.000000,0.000000,0.000000}%
\pgfsetstrokecolor{currentstroke}%
\pgfsetdash{}{0pt}%
\pgfpathmoveto{\pgfqpoint{7.180222in}{9.096129in}}%
\pgfpathlineto{\pgfqpoint{7.180222in}{9.583876in}}%
\pgfusepath{stroke}%
\end{pgfscope}%
\begin{pgfscope}%
\pgfpathrectangle{\pgfqpoint{2.125000in}{8.820930in}}{\pgfqpoint{5.489583in}{0.877907in}}%
\pgfusepath{clip}%
\pgfsetbuttcap%
\pgfsetroundjoin%
\pgfsetlinewidth{1.505625pt}%
\definecolor{currentstroke}{rgb}{0.000000,0.000000,0.000000}%
\pgfsetstrokecolor{currentstroke}%
\pgfsetdash{}{0pt}%
\pgfpathmoveto{\pgfqpoint{7.303445in}{9.096129in}}%
\pgfpathlineto{\pgfqpoint{7.303445in}{9.582497in}}%
\pgfusepath{stroke}%
\end{pgfscope}%
\begin{pgfscope}%
\pgfpathrectangle{\pgfqpoint{2.125000in}{8.820930in}}{\pgfqpoint{5.489583in}{0.877907in}}%
\pgfusepath{clip}%
\pgfsetroundcap%
\pgfsetroundjoin%
\pgfsetlinewidth{1.505625pt}%
\definecolor{currentstroke}{rgb}{0.121569,0.466667,0.705882}%
\pgfsetstrokecolor{currentstroke}%
\pgfsetdash{}{0pt}%
\pgfpathmoveto{\pgfqpoint{2.125000in}{9.096129in}}%
\pgfpathlineto{\pgfqpoint{7.614583in}{9.096129in}}%
\pgfusepath{stroke}%
\end{pgfscope}%
\begin{pgfscope}%
\pgfpathrectangle{\pgfqpoint{2.125000in}{8.820930in}}{\pgfqpoint{5.489583in}{0.877907in}}%
\pgfusepath{clip}%
\pgfsetbuttcap%
\pgfsetroundjoin%
\definecolor{currentfill}{rgb}{0.121569,0.466667,0.705882}%
\pgfsetfillcolor{currentfill}%
\pgfsetlinewidth{1.003750pt}%
\definecolor{currentstroke}{rgb}{0.121569,0.466667,0.705882}%
\pgfsetstrokecolor{currentstroke}%
\pgfsetdash{}{0pt}%
\pgfsys@defobject{currentmarker}{\pgfqpoint{-0.034722in}{-0.034722in}}{\pgfqpoint{0.034722in}{0.034722in}}{%
\pgfpathmoveto{\pgfqpoint{0.000000in}{-0.034722in}}%
\pgfpathcurveto{\pgfqpoint{0.009208in}{-0.034722in}}{\pgfqpoint{0.018041in}{-0.031064in}}{\pgfqpoint{0.024552in}{-0.024552in}}%
\pgfpathcurveto{\pgfqpoint{0.031064in}{-0.018041in}}{\pgfqpoint{0.034722in}{-0.009208in}}{\pgfqpoint{0.034722in}{0.000000in}}%
\pgfpathcurveto{\pgfqpoint{0.034722in}{0.009208in}}{\pgfqpoint{0.031064in}{0.018041in}}{\pgfqpoint{0.024552in}{0.024552in}}%
\pgfpathcurveto{\pgfqpoint{0.018041in}{0.031064in}}{\pgfqpoint{0.009208in}{0.034722in}}{\pgfqpoint{0.000000in}{0.034722in}}%
\pgfpathcurveto{\pgfqpoint{-0.009208in}{0.034722in}}{\pgfqpoint{-0.018041in}{0.031064in}}{\pgfqpoint{-0.024552in}{0.024552in}}%
\pgfpathcurveto{\pgfqpoint{-0.031064in}{0.018041in}}{\pgfqpoint{-0.034722in}{0.009208in}}{\pgfqpoint{-0.034722in}{0.000000in}}%
\pgfpathcurveto{\pgfqpoint{-0.034722in}{-0.009208in}}{\pgfqpoint{-0.031064in}{-0.018041in}}{\pgfqpoint{-0.024552in}{-0.024552in}}%
\pgfpathcurveto{\pgfqpoint{-0.018041in}{-0.031064in}}{\pgfqpoint{-0.009208in}{-0.034722in}}{\pgfqpoint{0.000000in}{-0.034722in}}%
\pgfpathclose%
\pgfusepath{stroke,fill}%
}%
\begin{pgfscope}%
\pgfsys@transformshift{2.374527in}{9.658932in}%
\pgfsys@useobject{currentmarker}{}%
\end{pgfscope}%
\begin{pgfscope}%
\pgfsys@transformshift{2.497749in}{9.656919in}%
\pgfsys@useobject{currentmarker}{}%
\end{pgfscope}%
\begin{pgfscope}%
\pgfsys@transformshift{2.620972in}{9.654932in}%
\pgfsys@useobject{currentmarker}{}%
\end{pgfscope}%
\begin{pgfscope}%
\pgfsys@transformshift{2.744195in}{9.652972in}%
\pgfsys@useobject{currentmarker}{}%
\end{pgfscope}%
\begin{pgfscope}%
\pgfsys@transformshift{2.867418in}{9.651012in}%
\pgfsys@useobject{currentmarker}{}%
\end{pgfscope}%
\begin{pgfscope}%
\pgfsys@transformshift{2.990641in}{9.649181in}%
\pgfsys@useobject{currentmarker}{}%
\end{pgfscope}%
\begin{pgfscope}%
\pgfsys@transformshift{3.113864in}{9.647275in}%
\pgfsys@useobject{currentmarker}{}%
\end{pgfscope}%
\begin{pgfscope}%
\pgfsys@transformshift{3.237087in}{9.645299in}%
\pgfsys@useobject{currentmarker}{}%
\end{pgfscope}%
\begin{pgfscope}%
\pgfsys@transformshift{3.360310in}{9.643302in}%
\pgfsys@useobject{currentmarker}{}%
\end{pgfscope}%
\begin{pgfscope}%
\pgfsys@transformshift{3.483533in}{9.641283in}%
\pgfsys@useobject{currentmarker}{}%
\end{pgfscope}%
\begin{pgfscope}%
\pgfsys@transformshift{3.606756in}{9.639340in}%
\pgfsys@useobject{currentmarker}{}%
\end{pgfscope}%
\begin{pgfscope}%
\pgfsys@transformshift{3.729979in}{9.637500in}%
\pgfsys@useobject{currentmarker}{}%
\end{pgfscope}%
\begin{pgfscope}%
\pgfsys@transformshift{3.853202in}{9.635664in}%
\pgfsys@useobject{currentmarker}{}%
\end{pgfscope}%
\begin{pgfscope}%
\pgfsys@transformshift{3.976425in}{9.633884in}%
\pgfsys@useobject{currentmarker}{}%
\end{pgfscope}%
\begin{pgfscope}%
\pgfsys@transformshift{4.099648in}{9.632011in}%
\pgfsys@useobject{currentmarker}{}%
\end{pgfscope}%
\begin{pgfscope}%
\pgfsys@transformshift{4.222871in}{9.630098in}%
\pgfsys@useobject{currentmarker}{}%
\end{pgfscope}%
\begin{pgfscope}%
\pgfsys@transformshift{4.346094in}{9.628279in}%
\pgfsys@useobject{currentmarker}{}%
\end{pgfscope}%
\begin{pgfscope}%
\pgfsys@transformshift{4.469317in}{9.626323in}%
\pgfsys@useobject{currentmarker}{}%
\end{pgfscope}%
\begin{pgfscope}%
\pgfsys@transformshift{4.592540in}{9.624287in}%
\pgfsys@useobject{currentmarker}{}%
\end{pgfscope}%
\begin{pgfscope}%
\pgfsys@transformshift{4.715763in}{9.622164in}%
\pgfsys@useobject{currentmarker}{}%
\end{pgfscope}%
\begin{pgfscope}%
\pgfsys@transformshift{4.838986in}{9.620059in}%
\pgfsys@useobject{currentmarker}{}%
\end{pgfscope}%
\begin{pgfscope}%
\pgfsys@transformshift{4.962209in}{9.618016in}%
\pgfsys@useobject{currentmarker}{}%
\end{pgfscope}%
\begin{pgfscope}%
\pgfsys@transformshift{5.085432in}{9.615977in}%
\pgfsys@useobject{currentmarker}{}%
\end{pgfscope}%
\begin{pgfscope}%
\pgfsys@transformshift{5.208655in}{9.613882in}%
\pgfsys@useobject{currentmarker}{}%
\end{pgfscope}%
\begin{pgfscope}%
\pgfsys@transformshift{5.331878in}{9.611825in}%
\pgfsys@useobject{currentmarker}{}%
\end{pgfscope}%
\begin{pgfscope}%
\pgfsys@transformshift{5.455101in}{9.609737in}%
\pgfsys@useobject{currentmarker}{}%
\end{pgfscope}%
\begin{pgfscope}%
\pgfsys@transformshift{5.578324in}{9.607754in}%
\pgfsys@useobject{currentmarker}{}%
\end{pgfscope}%
\begin{pgfscope}%
\pgfsys@transformshift{5.701547in}{9.605766in}%
\pgfsys@useobject{currentmarker}{}%
\end{pgfscope}%
\begin{pgfscope}%
\pgfsys@transformshift{5.824770in}{9.603797in}%
\pgfsys@useobject{currentmarker}{}%
\end{pgfscope}%
\begin{pgfscope}%
\pgfsys@transformshift{5.947993in}{9.601791in}%
\pgfsys@useobject{currentmarker}{}%
\end{pgfscope}%
\begin{pgfscope}%
\pgfsys@transformshift{6.071216in}{9.599677in}%
\pgfsys@useobject{currentmarker}{}%
\end{pgfscope}%
\begin{pgfscope}%
\pgfsys@transformshift{6.194439in}{9.597661in}%
\pgfsys@useobject{currentmarker}{}%
\end{pgfscope}%
\begin{pgfscope}%
\pgfsys@transformshift{6.317662in}{9.595690in}%
\pgfsys@useobject{currentmarker}{}%
\end{pgfscope}%
\begin{pgfscope}%
\pgfsys@transformshift{6.440885in}{9.593759in}%
\pgfsys@useobject{currentmarker}{}%
\end{pgfscope}%
\begin{pgfscope}%
\pgfsys@transformshift{6.564108in}{9.591879in}%
\pgfsys@useobject{currentmarker}{}%
\end{pgfscope}%
\begin{pgfscope}%
\pgfsys@transformshift{6.687330in}{9.590064in}%
\pgfsys@useobject{currentmarker}{}%
\end{pgfscope}%
\begin{pgfscope}%
\pgfsys@transformshift{6.810553in}{9.588455in}%
\pgfsys@useobject{currentmarker}{}%
\end{pgfscope}%
\begin{pgfscope}%
\pgfsys@transformshift{6.933776in}{9.586843in}%
\pgfsys@useobject{currentmarker}{}%
\end{pgfscope}%
\begin{pgfscope}%
\pgfsys@transformshift{7.056999in}{9.585317in}%
\pgfsys@useobject{currentmarker}{}%
\end{pgfscope}%
\begin{pgfscope}%
\pgfsys@transformshift{7.180222in}{9.583876in}%
\pgfsys@useobject{currentmarker}{}%
\end{pgfscope}%
\begin{pgfscope}%
\pgfsys@transformshift{7.303445in}{9.582497in}%
\pgfsys@useobject{currentmarker}{}%
\end{pgfscope}%
\end{pgfscope}%
\begin{pgfscope}%
\pgfsetrectcap%
\pgfsetmiterjoin%
\pgfsetlinewidth{0.803000pt}%
\definecolor{currentstroke}{rgb}{1.000000,1.000000,1.000000}%
\pgfsetstrokecolor{currentstroke}%
\pgfsetdash{}{0pt}%
\pgfpathmoveto{\pgfqpoint{2.125000in}{8.820930in}}%
\pgfpathlineto{\pgfqpoint{2.125000in}{9.698837in}}%
\pgfusepath{stroke}%
\end{pgfscope}%
\begin{pgfscope}%
\pgfsetrectcap%
\pgfsetmiterjoin%
\pgfsetlinewidth{0.803000pt}%
\definecolor{currentstroke}{rgb}{1.000000,1.000000,1.000000}%
\pgfsetstrokecolor{currentstroke}%
\pgfsetdash{}{0pt}%
\pgfpathmoveto{\pgfqpoint{7.614583in}{8.820930in}}%
\pgfpathlineto{\pgfqpoint{7.614583in}{9.698837in}}%
\pgfusepath{stroke}%
\end{pgfscope}%
\begin{pgfscope}%
\pgfsetrectcap%
\pgfsetmiterjoin%
\pgfsetlinewidth{0.803000pt}%
\definecolor{currentstroke}{rgb}{1.000000,1.000000,1.000000}%
\pgfsetstrokecolor{currentstroke}%
\pgfsetdash{}{0pt}%
\pgfpathmoveto{\pgfqpoint{2.125000in}{8.820930in}}%
\pgfpathlineto{\pgfqpoint{7.614583in}{8.820930in}}%
\pgfusepath{stroke}%
\end{pgfscope}%
\begin{pgfscope}%
\pgfsetrectcap%
\pgfsetmiterjoin%
\pgfsetlinewidth{0.803000pt}%
\definecolor{currentstroke}{rgb}{1.000000,1.000000,1.000000}%
\pgfsetstrokecolor{currentstroke}%
\pgfsetdash{}{0pt}%
\pgfpathmoveto{\pgfqpoint{2.125000in}{9.698837in}}%
\pgfpathlineto{\pgfqpoint{7.614583in}{9.698837in}}%
\pgfusepath{stroke}%
\end{pgfscope}%
\begin{pgfscope}%
\definecolor{textcolor}{rgb}{0.150000,0.150000,0.150000}%
\pgfsetstrokecolor{textcolor}%
\pgfsetfillcolor{textcolor}%
\pgftext[x=4.869792in,y=9.782171in,,base]{\color{textcolor}\rmfamily\fontsize{16.800000}{20.160000}\selectfont Autocorrelation}%
\end{pgfscope}%
\begin{pgfscope}%
\pgfsetbuttcap%
\pgfsetmiterjoin%
\definecolor{currentfill}{rgb}{0.917647,0.917647,0.949020}%
\pgfsetfillcolor{currentfill}%
\pgfsetlinewidth{0.000000pt}%
\definecolor{currentstroke}{rgb}{0.000000,0.000000,0.000000}%
\pgfsetstrokecolor{currentstroke}%
\pgfsetstrokeopacity{0.000000}%
\pgfsetdash{}{0pt}%
\pgfpathmoveto{\pgfqpoint{9.810417in}{8.820930in}}%
\pgfpathlineto{\pgfqpoint{15.300000in}{8.820930in}}%
\pgfpathlineto{\pgfqpoint{15.300000in}{9.698837in}}%
\pgfpathlineto{\pgfqpoint{9.810417in}{9.698837in}}%
\pgfpathclose%
\pgfusepath{fill}%
\end{pgfscope}%
\begin{pgfscope}%
\pgfpathrectangle{\pgfqpoint{9.810417in}{8.820930in}}{\pgfqpoint{5.489583in}{0.877907in}}%
\pgfusepath{clip}%
\pgfsetroundcap%
\pgfsetroundjoin%
\pgfsetlinewidth{0.803000pt}%
\definecolor{currentstroke}{rgb}{1.000000,1.000000,1.000000}%
\pgfsetstrokecolor{currentstroke}%
\pgfsetdash{}{0pt}%
\pgfpathmoveto{\pgfqpoint{10.059943in}{8.820930in}}%
\pgfpathlineto{\pgfqpoint{10.059943in}{9.698837in}}%
\pgfusepath{stroke}%
\end{pgfscope}%
\begin{pgfscope}%
\definecolor{textcolor}{rgb}{0.150000,0.150000,0.150000}%
\pgfsetstrokecolor{textcolor}%
\pgfsetfillcolor{textcolor}%
\pgftext[x=10.059943in,y=8.723708in,,top]{\color{textcolor}\rmfamily\fontsize{14.000000}{16.800000}\selectfont 0}%
\end{pgfscope}%
\begin{pgfscope}%
\pgfpathrectangle{\pgfqpoint{9.810417in}{8.820930in}}{\pgfqpoint{5.489583in}{0.877907in}}%
\pgfusepath{clip}%
\pgfsetroundcap%
\pgfsetroundjoin%
\pgfsetlinewidth{0.803000pt}%
\definecolor{currentstroke}{rgb}{1.000000,1.000000,1.000000}%
\pgfsetstrokecolor{currentstroke}%
\pgfsetdash{}{0pt}%
\pgfpathmoveto{\pgfqpoint{10.676058in}{8.820930in}}%
\pgfpathlineto{\pgfqpoint{10.676058in}{9.698837in}}%
\pgfusepath{stroke}%
\end{pgfscope}%
\begin{pgfscope}%
\definecolor{textcolor}{rgb}{0.150000,0.150000,0.150000}%
\pgfsetstrokecolor{textcolor}%
\pgfsetfillcolor{textcolor}%
\pgftext[x=10.676058in,y=8.723708in,,top]{\color{textcolor}\rmfamily\fontsize{14.000000}{16.800000}\selectfont 5}%
\end{pgfscope}%
\begin{pgfscope}%
\pgfpathrectangle{\pgfqpoint{9.810417in}{8.820930in}}{\pgfqpoint{5.489583in}{0.877907in}}%
\pgfusepath{clip}%
\pgfsetroundcap%
\pgfsetroundjoin%
\pgfsetlinewidth{0.803000pt}%
\definecolor{currentstroke}{rgb}{1.000000,1.000000,1.000000}%
\pgfsetstrokecolor{currentstroke}%
\pgfsetdash{}{0pt}%
\pgfpathmoveto{\pgfqpoint{11.292173in}{8.820930in}}%
\pgfpathlineto{\pgfqpoint{11.292173in}{9.698837in}}%
\pgfusepath{stroke}%
\end{pgfscope}%
\begin{pgfscope}%
\definecolor{textcolor}{rgb}{0.150000,0.150000,0.150000}%
\pgfsetstrokecolor{textcolor}%
\pgfsetfillcolor{textcolor}%
\pgftext[x=11.292173in,y=8.723708in,,top]{\color{textcolor}\rmfamily\fontsize{14.000000}{16.800000}\selectfont 10}%
\end{pgfscope}%
\begin{pgfscope}%
\pgfpathrectangle{\pgfqpoint{9.810417in}{8.820930in}}{\pgfqpoint{5.489583in}{0.877907in}}%
\pgfusepath{clip}%
\pgfsetroundcap%
\pgfsetroundjoin%
\pgfsetlinewidth{0.803000pt}%
\definecolor{currentstroke}{rgb}{1.000000,1.000000,1.000000}%
\pgfsetstrokecolor{currentstroke}%
\pgfsetdash{}{0pt}%
\pgfpathmoveto{\pgfqpoint{11.908288in}{8.820930in}}%
\pgfpathlineto{\pgfqpoint{11.908288in}{9.698837in}}%
\pgfusepath{stroke}%
\end{pgfscope}%
\begin{pgfscope}%
\definecolor{textcolor}{rgb}{0.150000,0.150000,0.150000}%
\pgfsetstrokecolor{textcolor}%
\pgfsetfillcolor{textcolor}%
\pgftext[x=11.908288in,y=8.723708in,,top]{\color{textcolor}\rmfamily\fontsize{14.000000}{16.800000}\selectfont 15}%
\end{pgfscope}%
\begin{pgfscope}%
\pgfpathrectangle{\pgfqpoint{9.810417in}{8.820930in}}{\pgfqpoint{5.489583in}{0.877907in}}%
\pgfusepath{clip}%
\pgfsetroundcap%
\pgfsetroundjoin%
\pgfsetlinewidth{0.803000pt}%
\definecolor{currentstroke}{rgb}{1.000000,1.000000,1.000000}%
\pgfsetstrokecolor{currentstroke}%
\pgfsetdash{}{0pt}%
\pgfpathmoveto{\pgfqpoint{12.524403in}{8.820930in}}%
\pgfpathlineto{\pgfqpoint{12.524403in}{9.698837in}}%
\pgfusepath{stroke}%
\end{pgfscope}%
\begin{pgfscope}%
\definecolor{textcolor}{rgb}{0.150000,0.150000,0.150000}%
\pgfsetstrokecolor{textcolor}%
\pgfsetfillcolor{textcolor}%
\pgftext[x=12.524403in,y=8.723708in,,top]{\color{textcolor}\rmfamily\fontsize{14.000000}{16.800000}\selectfont 20}%
\end{pgfscope}%
\begin{pgfscope}%
\pgfpathrectangle{\pgfqpoint{9.810417in}{8.820930in}}{\pgfqpoint{5.489583in}{0.877907in}}%
\pgfusepath{clip}%
\pgfsetroundcap%
\pgfsetroundjoin%
\pgfsetlinewidth{0.803000pt}%
\definecolor{currentstroke}{rgb}{1.000000,1.000000,1.000000}%
\pgfsetstrokecolor{currentstroke}%
\pgfsetdash{}{0pt}%
\pgfpathmoveto{\pgfqpoint{13.140517in}{8.820930in}}%
\pgfpathlineto{\pgfqpoint{13.140517in}{9.698837in}}%
\pgfusepath{stroke}%
\end{pgfscope}%
\begin{pgfscope}%
\definecolor{textcolor}{rgb}{0.150000,0.150000,0.150000}%
\pgfsetstrokecolor{textcolor}%
\pgfsetfillcolor{textcolor}%
\pgftext[x=13.140517in,y=8.723708in,,top]{\color{textcolor}\rmfamily\fontsize{14.000000}{16.800000}\selectfont 25}%
\end{pgfscope}%
\begin{pgfscope}%
\pgfpathrectangle{\pgfqpoint{9.810417in}{8.820930in}}{\pgfqpoint{5.489583in}{0.877907in}}%
\pgfusepath{clip}%
\pgfsetroundcap%
\pgfsetroundjoin%
\pgfsetlinewidth{0.803000pt}%
\definecolor{currentstroke}{rgb}{1.000000,1.000000,1.000000}%
\pgfsetstrokecolor{currentstroke}%
\pgfsetdash{}{0pt}%
\pgfpathmoveto{\pgfqpoint{13.756632in}{8.820930in}}%
\pgfpathlineto{\pgfqpoint{13.756632in}{9.698837in}}%
\pgfusepath{stroke}%
\end{pgfscope}%
\begin{pgfscope}%
\definecolor{textcolor}{rgb}{0.150000,0.150000,0.150000}%
\pgfsetstrokecolor{textcolor}%
\pgfsetfillcolor{textcolor}%
\pgftext[x=13.756632in,y=8.723708in,,top]{\color{textcolor}\rmfamily\fontsize{14.000000}{16.800000}\selectfont 30}%
\end{pgfscope}%
\begin{pgfscope}%
\pgfpathrectangle{\pgfqpoint{9.810417in}{8.820930in}}{\pgfqpoint{5.489583in}{0.877907in}}%
\pgfusepath{clip}%
\pgfsetroundcap%
\pgfsetroundjoin%
\pgfsetlinewidth{0.803000pt}%
\definecolor{currentstroke}{rgb}{1.000000,1.000000,1.000000}%
\pgfsetstrokecolor{currentstroke}%
\pgfsetdash{}{0pt}%
\pgfpathmoveto{\pgfqpoint{14.372747in}{8.820930in}}%
\pgfpathlineto{\pgfqpoint{14.372747in}{9.698837in}}%
\pgfusepath{stroke}%
\end{pgfscope}%
\begin{pgfscope}%
\definecolor{textcolor}{rgb}{0.150000,0.150000,0.150000}%
\pgfsetstrokecolor{textcolor}%
\pgfsetfillcolor{textcolor}%
\pgftext[x=14.372747in,y=8.723708in,,top]{\color{textcolor}\rmfamily\fontsize{14.000000}{16.800000}\selectfont 35}%
\end{pgfscope}%
\begin{pgfscope}%
\pgfpathrectangle{\pgfqpoint{9.810417in}{8.820930in}}{\pgfqpoint{5.489583in}{0.877907in}}%
\pgfusepath{clip}%
\pgfsetroundcap%
\pgfsetroundjoin%
\pgfsetlinewidth{0.803000pt}%
\definecolor{currentstroke}{rgb}{1.000000,1.000000,1.000000}%
\pgfsetstrokecolor{currentstroke}%
\pgfsetdash{}{0pt}%
\pgfpathmoveto{\pgfqpoint{14.988862in}{8.820930in}}%
\pgfpathlineto{\pgfqpoint{14.988862in}{9.698837in}}%
\pgfusepath{stroke}%
\end{pgfscope}%
\begin{pgfscope}%
\definecolor{textcolor}{rgb}{0.150000,0.150000,0.150000}%
\pgfsetstrokecolor{textcolor}%
\pgfsetfillcolor{textcolor}%
\pgftext[x=14.988862in,y=8.723708in,,top]{\color{textcolor}\rmfamily\fontsize{14.000000}{16.800000}\selectfont 40}%
\end{pgfscope}%
\begin{pgfscope}%
\pgfpathrectangle{\pgfqpoint{9.810417in}{8.820930in}}{\pgfqpoint{5.489583in}{0.877907in}}%
\pgfusepath{clip}%
\pgfsetroundcap%
\pgfsetroundjoin%
\pgfsetlinewidth{0.803000pt}%
\definecolor{currentstroke}{rgb}{1.000000,1.000000,1.000000}%
\pgfsetstrokecolor{currentstroke}%
\pgfsetdash{}{0pt}%
\pgfpathmoveto{\pgfqpoint{9.810417in}{8.899169in}}%
\pgfpathlineto{\pgfqpoint{15.300000in}{8.899169in}}%
\pgfusepath{stroke}%
\end{pgfscope}%
\begin{pgfscope}%
\definecolor{textcolor}{rgb}{0.150000,0.150000,0.150000}%
\pgfsetstrokecolor{textcolor}%
\pgfsetfillcolor{textcolor}%
\pgftext[x=9.589483in,y=8.825303in,left,base]{\color{textcolor}\rmfamily\fontsize{14.000000}{16.800000}\selectfont 0}%
\end{pgfscope}%
\begin{pgfscope}%
\pgfpathrectangle{\pgfqpoint{9.810417in}{8.820930in}}{\pgfqpoint{5.489583in}{0.877907in}}%
\pgfusepath{clip}%
\pgfsetroundcap%
\pgfsetroundjoin%
\pgfsetlinewidth{0.803000pt}%
\definecolor{currentstroke}{rgb}{1.000000,1.000000,1.000000}%
\pgfsetstrokecolor{currentstroke}%
\pgfsetdash{}{0pt}%
\pgfpathmoveto{\pgfqpoint{9.810417in}{9.658932in}}%
\pgfpathlineto{\pgfqpoint{15.300000in}{9.658932in}}%
\pgfusepath{stroke}%
\end{pgfscope}%
\begin{pgfscope}%
\definecolor{textcolor}{rgb}{0.150000,0.150000,0.150000}%
\pgfsetstrokecolor{textcolor}%
\pgfsetfillcolor{textcolor}%
\pgftext[x=9.589483in,y=9.585066in,left,base]{\color{textcolor}\rmfamily\fontsize{14.000000}{16.800000}\selectfont 1}%
\end{pgfscope}%
\begin{pgfscope}%
\pgfpathrectangle{\pgfqpoint{9.810417in}{8.820930in}}{\pgfqpoint{5.489583in}{0.877907in}}%
\pgfusepath{clip}%
\pgfsetbuttcap%
\pgfsetroundjoin%
\definecolor{currentfill}{rgb}{0.121569,0.466667,0.705882}%
\pgfsetfillcolor{currentfill}%
\pgfsetfillopacity{0.250000}%
\pgfsetlinewidth{1.003750pt}%
\definecolor{currentstroke}{rgb}{1.000000,1.000000,1.000000}%
\pgfsetstrokecolor{currentstroke}%
\pgfsetstrokeopacity{0.250000}%
\pgfsetdash{}{0pt}%
\pgfpathmoveto{\pgfqpoint{10.121555in}{8.937503in}}%
\pgfpathlineto{\pgfqpoint{10.121555in}{8.860835in}}%
\pgfpathlineto{\pgfqpoint{10.306389in}{8.860835in}}%
\pgfpathlineto{\pgfqpoint{10.429612in}{8.860835in}}%
\pgfpathlineto{\pgfqpoint{10.552835in}{8.860835in}}%
\pgfpathlineto{\pgfqpoint{10.676058in}{8.860835in}}%
\pgfpathlineto{\pgfqpoint{10.799281in}{8.860835in}}%
\pgfpathlineto{\pgfqpoint{10.922504in}{8.860835in}}%
\pgfpathlineto{\pgfqpoint{11.045727in}{8.860835in}}%
\pgfpathlineto{\pgfqpoint{11.168950in}{8.860835in}}%
\pgfpathlineto{\pgfqpoint{11.292173in}{8.860835in}}%
\pgfpathlineto{\pgfqpoint{11.415396in}{8.860835in}}%
\pgfpathlineto{\pgfqpoint{11.538619in}{8.860835in}}%
\pgfpathlineto{\pgfqpoint{11.661842in}{8.860835in}}%
\pgfpathlineto{\pgfqpoint{11.785065in}{8.860835in}}%
\pgfpathlineto{\pgfqpoint{11.908288in}{8.860835in}}%
\pgfpathlineto{\pgfqpoint{12.031511in}{8.860835in}}%
\pgfpathlineto{\pgfqpoint{12.154734in}{8.860835in}}%
\pgfpathlineto{\pgfqpoint{12.277957in}{8.860835in}}%
\pgfpathlineto{\pgfqpoint{12.401180in}{8.860835in}}%
\pgfpathlineto{\pgfqpoint{12.524403in}{8.860835in}}%
\pgfpathlineto{\pgfqpoint{12.647626in}{8.860835in}}%
\pgfpathlineto{\pgfqpoint{12.770849in}{8.860835in}}%
\pgfpathlineto{\pgfqpoint{12.894072in}{8.860835in}}%
\pgfpathlineto{\pgfqpoint{13.017294in}{8.860835in}}%
\pgfpathlineto{\pgfqpoint{13.140517in}{8.860835in}}%
\pgfpathlineto{\pgfqpoint{13.263740in}{8.860835in}}%
\pgfpathlineto{\pgfqpoint{13.386963in}{8.860835in}}%
\pgfpathlineto{\pgfqpoint{13.510186in}{8.860835in}}%
\pgfpathlineto{\pgfqpoint{13.633409in}{8.860835in}}%
\pgfpathlineto{\pgfqpoint{13.756632in}{8.860835in}}%
\pgfpathlineto{\pgfqpoint{13.879855in}{8.860835in}}%
\pgfpathlineto{\pgfqpoint{14.003078in}{8.860835in}}%
\pgfpathlineto{\pgfqpoint{14.126301in}{8.860835in}}%
\pgfpathlineto{\pgfqpoint{14.249524in}{8.860835in}}%
\pgfpathlineto{\pgfqpoint{14.372747in}{8.860835in}}%
\pgfpathlineto{\pgfqpoint{14.495970in}{8.860835in}}%
\pgfpathlineto{\pgfqpoint{14.619193in}{8.860835in}}%
\pgfpathlineto{\pgfqpoint{14.742416in}{8.860835in}}%
\pgfpathlineto{\pgfqpoint{14.865639in}{8.860835in}}%
\pgfpathlineto{\pgfqpoint{15.050473in}{8.860835in}}%
\pgfpathlineto{\pgfqpoint{15.050473in}{8.937503in}}%
\pgfpathlineto{\pgfqpoint{15.050473in}{8.937503in}}%
\pgfpathlineto{\pgfqpoint{14.865639in}{8.937503in}}%
\pgfpathlineto{\pgfqpoint{14.742416in}{8.937503in}}%
\pgfpathlineto{\pgfqpoint{14.619193in}{8.937503in}}%
\pgfpathlineto{\pgfqpoint{14.495970in}{8.937503in}}%
\pgfpathlineto{\pgfqpoint{14.372747in}{8.937503in}}%
\pgfpathlineto{\pgfqpoint{14.249524in}{8.937503in}}%
\pgfpathlineto{\pgfqpoint{14.126301in}{8.937503in}}%
\pgfpathlineto{\pgfqpoint{14.003078in}{8.937503in}}%
\pgfpathlineto{\pgfqpoint{13.879855in}{8.937503in}}%
\pgfpathlineto{\pgfqpoint{13.756632in}{8.937503in}}%
\pgfpathlineto{\pgfqpoint{13.633409in}{8.937503in}}%
\pgfpathlineto{\pgfqpoint{13.510186in}{8.937503in}}%
\pgfpathlineto{\pgfqpoint{13.386963in}{8.937503in}}%
\pgfpathlineto{\pgfqpoint{13.263740in}{8.937503in}}%
\pgfpathlineto{\pgfqpoint{13.140517in}{8.937503in}}%
\pgfpathlineto{\pgfqpoint{13.017294in}{8.937503in}}%
\pgfpathlineto{\pgfqpoint{12.894072in}{8.937503in}}%
\pgfpathlineto{\pgfqpoint{12.770849in}{8.937503in}}%
\pgfpathlineto{\pgfqpoint{12.647626in}{8.937503in}}%
\pgfpathlineto{\pgfqpoint{12.524403in}{8.937503in}}%
\pgfpathlineto{\pgfqpoint{12.401180in}{8.937503in}}%
\pgfpathlineto{\pgfqpoint{12.277957in}{8.937503in}}%
\pgfpathlineto{\pgfqpoint{12.154734in}{8.937503in}}%
\pgfpathlineto{\pgfqpoint{12.031511in}{8.937503in}}%
\pgfpathlineto{\pgfqpoint{11.908288in}{8.937503in}}%
\pgfpathlineto{\pgfqpoint{11.785065in}{8.937503in}}%
\pgfpathlineto{\pgfqpoint{11.661842in}{8.937503in}}%
\pgfpathlineto{\pgfqpoint{11.538619in}{8.937503in}}%
\pgfpathlineto{\pgfqpoint{11.415396in}{8.937503in}}%
\pgfpathlineto{\pgfqpoint{11.292173in}{8.937503in}}%
\pgfpathlineto{\pgfqpoint{11.168950in}{8.937503in}}%
\pgfpathlineto{\pgfqpoint{11.045727in}{8.937503in}}%
\pgfpathlineto{\pgfqpoint{10.922504in}{8.937503in}}%
\pgfpathlineto{\pgfqpoint{10.799281in}{8.937503in}}%
\pgfpathlineto{\pgfqpoint{10.676058in}{8.937503in}}%
\pgfpathlineto{\pgfqpoint{10.552835in}{8.937503in}}%
\pgfpathlineto{\pgfqpoint{10.429612in}{8.937503in}}%
\pgfpathlineto{\pgfqpoint{10.306389in}{8.937503in}}%
\pgfpathlineto{\pgfqpoint{10.121555in}{8.937503in}}%
\pgfpathclose%
\pgfusepath{stroke,fill}%
\end{pgfscope}%
\begin{pgfscope}%
\pgfpathrectangle{\pgfqpoint{9.810417in}{8.820930in}}{\pgfqpoint{5.489583in}{0.877907in}}%
\pgfusepath{clip}%
\pgfsetbuttcap%
\pgfsetroundjoin%
\pgfsetlinewidth{1.505625pt}%
\definecolor{currentstroke}{rgb}{0.000000,0.000000,0.000000}%
\pgfsetstrokecolor{currentstroke}%
\pgfsetdash{}{0pt}%
\pgfpathmoveto{\pgfqpoint{10.059943in}{8.899169in}}%
\pgfpathlineto{\pgfqpoint{10.059943in}{9.658932in}}%
\pgfusepath{stroke}%
\end{pgfscope}%
\begin{pgfscope}%
\pgfpathrectangle{\pgfqpoint{9.810417in}{8.820930in}}{\pgfqpoint{5.489583in}{0.877907in}}%
\pgfusepath{clip}%
\pgfsetbuttcap%
\pgfsetroundjoin%
\pgfsetlinewidth{1.505625pt}%
\definecolor{currentstroke}{rgb}{0.000000,0.000000,0.000000}%
\pgfsetstrokecolor{currentstroke}%
\pgfsetdash{}{0pt}%
\pgfpathmoveto{\pgfqpoint{10.183166in}{8.899169in}}%
\pgfpathlineto{\pgfqpoint{10.183166in}{9.656717in}}%
\pgfusepath{stroke}%
\end{pgfscope}%
\begin{pgfscope}%
\pgfpathrectangle{\pgfqpoint{9.810417in}{8.820930in}}{\pgfqpoint{5.489583in}{0.877907in}}%
\pgfusepath{clip}%
\pgfsetbuttcap%
\pgfsetroundjoin%
\pgfsetlinewidth{1.505625pt}%
\definecolor{currentstroke}{rgb}{0.000000,0.000000,0.000000}%
\pgfsetstrokecolor{currentstroke}%
\pgfsetdash{}{0pt}%
\pgfpathmoveto{\pgfqpoint{10.306389in}{8.899169in}}%
\pgfpathlineto{\pgfqpoint{10.306389in}{8.903534in}}%
\pgfusepath{stroke}%
\end{pgfscope}%
\begin{pgfscope}%
\pgfpathrectangle{\pgfqpoint{9.810417in}{8.820930in}}{\pgfqpoint{5.489583in}{0.877907in}}%
\pgfusepath{clip}%
\pgfsetbuttcap%
\pgfsetroundjoin%
\pgfsetlinewidth{1.505625pt}%
\definecolor{currentstroke}{rgb}{0.000000,0.000000,0.000000}%
\pgfsetstrokecolor{currentstroke}%
\pgfsetdash{}{0pt}%
\pgfpathmoveto{\pgfqpoint{10.429612in}{8.899169in}}%
\pgfpathlineto{\pgfqpoint{10.429612in}{8.904148in}}%
\pgfusepath{stroke}%
\end{pgfscope}%
\begin{pgfscope}%
\pgfpathrectangle{\pgfqpoint{9.810417in}{8.820930in}}{\pgfqpoint{5.489583in}{0.877907in}}%
\pgfusepath{clip}%
\pgfsetbuttcap%
\pgfsetroundjoin%
\pgfsetlinewidth{1.505625pt}%
\definecolor{currentstroke}{rgb}{0.000000,0.000000,0.000000}%
\pgfsetstrokecolor{currentstroke}%
\pgfsetdash{}{0pt}%
\pgfpathmoveto{\pgfqpoint{10.552835in}{8.899169in}}%
\pgfpathlineto{\pgfqpoint{10.552835in}{8.897394in}}%
\pgfusepath{stroke}%
\end{pgfscope}%
\begin{pgfscope}%
\pgfpathrectangle{\pgfqpoint{9.810417in}{8.820930in}}{\pgfqpoint{5.489583in}{0.877907in}}%
\pgfusepath{clip}%
\pgfsetbuttcap%
\pgfsetroundjoin%
\pgfsetlinewidth{1.505625pt}%
\definecolor{currentstroke}{rgb}{0.000000,0.000000,0.000000}%
\pgfsetstrokecolor{currentstroke}%
\pgfsetdash{}{0pt}%
\pgfpathmoveto{\pgfqpoint{10.676058in}{8.899169in}}%
\pgfpathlineto{\pgfqpoint{10.676058in}{8.927893in}}%
\pgfusepath{stroke}%
\end{pgfscope}%
\begin{pgfscope}%
\pgfpathrectangle{\pgfqpoint{9.810417in}{8.820930in}}{\pgfqpoint{5.489583in}{0.877907in}}%
\pgfusepath{clip}%
\pgfsetbuttcap%
\pgfsetroundjoin%
\pgfsetlinewidth{1.505625pt}%
\definecolor{currentstroke}{rgb}{0.000000,0.000000,0.000000}%
\pgfsetstrokecolor{currentstroke}%
\pgfsetdash{}{0pt}%
\pgfpathmoveto{\pgfqpoint{10.799281in}{8.899169in}}%
\pgfpathlineto{\pgfqpoint{10.799281in}{8.880456in}}%
\pgfusepath{stroke}%
\end{pgfscope}%
\begin{pgfscope}%
\pgfpathrectangle{\pgfqpoint{9.810417in}{8.820930in}}{\pgfqpoint{5.489583in}{0.877907in}}%
\pgfusepath{clip}%
\pgfsetbuttcap%
\pgfsetroundjoin%
\pgfsetlinewidth{1.505625pt}%
\definecolor{currentstroke}{rgb}{0.000000,0.000000,0.000000}%
\pgfsetstrokecolor{currentstroke}%
\pgfsetdash{}{0pt}%
\pgfpathmoveto{\pgfqpoint{10.922504in}{8.899169in}}%
\pgfpathlineto{\pgfqpoint{10.922504in}{8.881465in}}%
\pgfusepath{stroke}%
\end{pgfscope}%
\begin{pgfscope}%
\pgfpathrectangle{\pgfqpoint{9.810417in}{8.820930in}}{\pgfqpoint{5.489583in}{0.877907in}}%
\pgfusepath{clip}%
\pgfsetbuttcap%
\pgfsetroundjoin%
\pgfsetlinewidth{1.505625pt}%
\definecolor{currentstroke}{rgb}{0.000000,0.000000,0.000000}%
\pgfsetstrokecolor{currentstroke}%
\pgfsetdash{}{0pt}%
\pgfpathmoveto{\pgfqpoint{11.045727in}{8.899169in}}%
\pgfpathlineto{\pgfqpoint{11.045727in}{8.892266in}}%
\pgfusepath{stroke}%
\end{pgfscope}%
\begin{pgfscope}%
\pgfpathrectangle{\pgfqpoint{9.810417in}{8.820930in}}{\pgfqpoint{5.489583in}{0.877907in}}%
\pgfusepath{clip}%
\pgfsetbuttcap%
\pgfsetroundjoin%
\pgfsetlinewidth{1.505625pt}%
\definecolor{currentstroke}{rgb}{0.000000,0.000000,0.000000}%
\pgfsetstrokecolor{currentstroke}%
\pgfsetdash{}{0pt}%
\pgfpathmoveto{\pgfqpoint{11.168950in}{8.899169in}}%
\pgfpathlineto{\pgfqpoint{11.168950in}{8.893383in}}%
\pgfusepath{stroke}%
\end{pgfscope}%
\begin{pgfscope}%
\pgfpathrectangle{\pgfqpoint{9.810417in}{8.820930in}}{\pgfqpoint{5.489583in}{0.877907in}}%
\pgfusepath{clip}%
\pgfsetbuttcap%
\pgfsetroundjoin%
\pgfsetlinewidth{1.505625pt}%
\definecolor{currentstroke}{rgb}{0.000000,0.000000,0.000000}%
\pgfsetstrokecolor{currentstroke}%
\pgfsetdash{}{0pt}%
\pgfpathmoveto{\pgfqpoint{11.292173in}{8.899169in}}%
\pgfpathlineto{\pgfqpoint{11.292173in}{8.913614in}}%
\pgfusepath{stroke}%
\end{pgfscope}%
\begin{pgfscope}%
\pgfpathrectangle{\pgfqpoint{9.810417in}{8.820930in}}{\pgfqpoint{5.489583in}{0.877907in}}%
\pgfusepath{clip}%
\pgfsetbuttcap%
\pgfsetroundjoin%
\pgfsetlinewidth{1.505625pt}%
\definecolor{currentstroke}{rgb}{0.000000,0.000000,0.000000}%
\pgfsetstrokecolor{currentstroke}%
\pgfsetdash{}{0pt}%
\pgfpathmoveto{\pgfqpoint{11.415396in}{8.899169in}}%
\pgfpathlineto{\pgfqpoint{11.415396in}{8.920884in}}%
\pgfusepath{stroke}%
\end{pgfscope}%
\begin{pgfscope}%
\pgfpathrectangle{\pgfqpoint{9.810417in}{8.820930in}}{\pgfqpoint{5.489583in}{0.877907in}}%
\pgfusepath{clip}%
\pgfsetbuttcap%
\pgfsetroundjoin%
\pgfsetlinewidth{1.505625pt}%
\definecolor{currentstroke}{rgb}{0.000000,0.000000,0.000000}%
\pgfsetstrokecolor{currentstroke}%
\pgfsetdash{}{0pt}%
\pgfpathmoveto{\pgfqpoint{11.538619in}{8.899169in}}%
\pgfpathlineto{\pgfqpoint{11.538619in}{8.899769in}}%
\pgfusepath{stroke}%
\end{pgfscope}%
\begin{pgfscope}%
\pgfpathrectangle{\pgfqpoint{9.810417in}{8.820930in}}{\pgfqpoint{5.489583in}{0.877907in}}%
\pgfusepath{clip}%
\pgfsetbuttcap%
\pgfsetroundjoin%
\pgfsetlinewidth{1.505625pt}%
\definecolor{currentstroke}{rgb}{0.000000,0.000000,0.000000}%
\pgfsetstrokecolor{currentstroke}%
\pgfsetdash{}{0pt}%
\pgfpathmoveto{\pgfqpoint{11.661842in}{8.899169in}}%
\pgfpathlineto{\pgfqpoint{11.661842in}{8.911209in}}%
\pgfusepath{stroke}%
\end{pgfscope}%
\begin{pgfscope}%
\pgfpathrectangle{\pgfqpoint{9.810417in}{8.820930in}}{\pgfqpoint{5.489583in}{0.877907in}}%
\pgfusepath{clip}%
\pgfsetbuttcap%
\pgfsetroundjoin%
\pgfsetlinewidth{1.505625pt}%
\definecolor{currentstroke}{rgb}{0.000000,0.000000,0.000000}%
\pgfsetstrokecolor{currentstroke}%
\pgfsetdash{}{0pt}%
\pgfpathmoveto{\pgfqpoint{11.785065in}{8.899169in}}%
\pgfpathlineto{\pgfqpoint{11.785065in}{8.877984in}}%
\pgfusepath{stroke}%
\end{pgfscope}%
\begin{pgfscope}%
\pgfpathrectangle{\pgfqpoint{9.810417in}{8.820930in}}{\pgfqpoint{5.489583in}{0.877907in}}%
\pgfusepath{clip}%
\pgfsetbuttcap%
\pgfsetroundjoin%
\pgfsetlinewidth{1.505625pt}%
\definecolor{currentstroke}{rgb}{0.000000,0.000000,0.000000}%
\pgfsetstrokecolor{currentstroke}%
\pgfsetdash{}{0pt}%
\pgfpathmoveto{\pgfqpoint{11.908288in}{8.899169in}}%
\pgfpathlineto{\pgfqpoint{11.908288in}{8.889088in}}%
\pgfusepath{stroke}%
\end{pgfscope}%
\begin{pgfscope}%
\pgfpathrectangle{\pgfqpoint{9.810417in}{8.820930in}}{\pgfqpoint{5.489583in}{0.877907in}}%
\pgfusepath{clip}%
\pgfsetbuttcap%
\pgfsetroundjoin%
\pgfsetlinewidth{1.505625pt}%
\definecolor{currentstroke}{rgb}{0.000000,0.000000,0.000000}%
\pgfsetstrokecolor{currentstroke}%
\pgfsetdash{}{0pt}%
\pgfpathmoveto{\pgfqpoint{12.031511in}{8.899169in}}%
\pgfpathlineto{\pgfqpoint{12.031511in}{8.917679in}}%
\pgfusepath{stroke}%
\end{pgfscope}%
\begin{pgfscope}%
\pgfpathrectangle{\pgfqpoint{9.810417in}{8.820930in}}{\pgfqpoint{5.489583in}{0.877907in}}%
\pgfusepath{clip}%
\pgfsetbuttcap%
\pgfsetroundjoin%
\pgfsetlinewidth{1.505625pt}%
\definecolor{currentstroke}{rgb}{0.000000,0.000000,0.000000}%
\pgfsetstrokecolor{currentstroke}%
\pgfsetdash{}{0pt}%
\pgfpathmoveto{\pgfqpoint{12.154734in}{8.899169in}}%
\pgfpathlineto{\pgfqpoint{12.154734in}{8.865466in}}%
\pgfusepath{stroke}%
\end{pgfscope}%
\begin{pgfscope}%
\pgfpathrectangle{\pgfqpoint{9.810417in}{8.820930in}}{\pgfqpoint{5.489583in}{0.877907in}}%
\pgfusepath{clip}%
\pgfsetbuttcap%
\pgfsetroundjoin%
\pgfsetlinewidth{1.505625pt}%
\definecolor{currentstroke}{rgb}{0.000000,0.000000,0.000000}%
\pgfsetstrokecolor{currentstroke}%
\pgfsetdash{}{0pt}%
\pgfpathmoveto{\pgfqpoint{12.277957in}{8.899169in}}%
\pgfpathlineto{\pgfqpoint{12.277957in}{8.875761in}}%
\pgfusepath{stroke}%
\end{pgfscope}%
\begin{pgfscope}%
\pgfpathrectangle{\pgfqpoint{9.810417in}{8.820930in}}{\pgfqpoint{5.489583in}{0.877907in}}%
\pgfusepath{clip}%
\pgfsetbuttcap%
\pgfsetroundjoin%
\pgfsetlinewidth{1.505625pt}%
\definecolor{currentstroke}{rgb}{0.000000,0.000000,0.000000}%
\pgfsetstrokecolor{currentstroke}%
\pgfsetdash{}{0pt}%
\pgfpathmoveto{\pgfqpoint{12.401180in}{8.899169in}}%
\pgfpathlineto{\pgfqpoint{12.401180in}{8.875551in}}%
\pgfusepath{stroke}%
\end{pgfscope}%
\begin{pgfscope}%
\pgfpathrectangle{\pgfqpoint{9.810417in}{8.820930in}}{\pgfqpoint{5.489583in}{0.877907in}}%
\pgfusepath{clip}%
\pgfsetbuttcap%
\pgfsetroundjoin%
\pgfsetlinewidth{1.505625pt}%
\definecolor{currentstroke}{rgb}{0.000000,0.000000,0.000000}%
\pgfsetstrokecolor{currentstroke}%
\pgfsetdash{}{0pt}%
\pgfpathmoveto{\pgfqpoint{12.524403in}{8.899169in}}%
\pgfpathlineto{\pgfqpoint{12.524403in}{8.904916in}}%
\pgfusepath{stroke}%
\end{pgfscope}%
\begin{pgfscope}%
\pgfpathrectangle{\pgfqpoint{9.810417in}{8.820930in}}{\pgfqpoint{5.489583in}{0.877907in}}%
\pgfusepath{clip}%
\pgfsetbuttcap%
\pgfsetroundjoin%
\pgfsetlinewidth{1.505625pt}%
\definecolor{currentstroke}{rgb}{0.000000,0.000000,0.000000}%
\pgfsetstrokecolor{currentstroke}%
\pgfsetdash{}{0pt}%
\pgfpathmoveto{\pgfqpoint{12.647626in}{8.899169in}}%
\pgfpathlineto{\pgfqpoint{12.647626in}{8.909892in}}%
\pgfusepath{stroke}%
\end{pgfscope}%
\begin{pgfscope}%
\pgfpathrectangle{\pgfqpoint{9.810417in}{8.820930in}}{\pgfqpoint{5.489583in}{0.877907in}}%
\pgfusepath{clip}%
\pgfsetbuttcap%
\pgfsetroundjoin%
\pgfsetlinewidth{1.505625pt}%
\definecolor{currentstroke}{rgb}{0.000000,0.000000,0.000000}%
\pgfsetstrokecolor{currentstroke}%
\pgfsetdash{}{0pt}%
\pgfpathmoveto{\pgfqpoint{12.770849in}{8.899169in}}%
\pgfpathlineto{\pgfqpoint{12.770849in}{8.898822in}}%
\pgfusepath{stroke}%
\end{pgfscope}%
\begin{pgfscope}%
\pgfpathrectangle{\pgfqpoint{9.810417in}{8.820930in}}{\pgfqpoint{5.489583in}{0.877907in}}%
\pgfusepath{clip}%
\pgfsetbuttcap%
\pgfsetroundjoin%
\pgfsetlinewidth{1.505625pt}%
\definecolor{currentstroke}{rgb}{0.000000,0.000000,0.000000}%
\pgfsetstrokecolor{currentstroke}%
\pgfsetdash{}{0pt}%
\pgfpathmoveto{\pgfqpoint{12.894072in}{8.899169in}}%
\pgfpathlineto{\pgfqpoint{12.894072in}{8.884663in}}%
\pgfusepath{stroke}%
\end{pgfscope}%
\begin{pgfscope}%
\pgfpathrectangle{\pgfqpoint{9.810417in}{8.820930in}}{\pgfqpoint{5.489583in}{0.877907in}}%
\pgfusepath{clip}%
\pgfsetbuttcap%
\pgfsetroundjoin%
\pgfsetlinewidth{1.505625pt}%
\definecolor{currentstroke}{rgb}{0.000000,0.000000,0.000000}%
\pgfsetstrokecolor{currentstroke}%
\pgfsetdash{}{0pt}%
\pgfpathmoveto{\pgfqpoint{13.017294in}{8.899169in}}%
\pgfpathlineto{\pgfqpoint{13.017294in}{8.907107in}}%
\pgfusepath{stroke}%
\end{pgfscope}%
\begin{pgfscope}%
\pgfpathrectangle{\pgfqpoint{9.810417in}{8.820930in}}{\pgfqpoint{5.489583in}{0.877907in}}%
\pgfusepath{clip}%
\pgfsetbuttcap%
\pgfsetroundjoin%
\pgfsetlinewidth{1.505625pt}%
\definecolor{currentstroke}{rgb}{0.000000,0.000000,0.000000}%
\pgfsetstrokecolor{currentstroke}%
\pgfsetdash{}{0pt}%
\pgfpathmoveto{\pgfqpoint{13.140517in}{8.899169in}}%
\pgfpathlineto{\pgfqpoint{13.140517in}{8.892804in}}%
\pgfusepath{stroke}%
\end{pgfscope}%
\begin{pgfscope}%
\pgfpathrectangle{\pgfqpoint{9.810417in}{8.820930in}}{\pgfqpoint{5.489583in}{0.877907in}}%
\pgfusepath{clip}%
\pgfsetbuttcap%
\pgfsetroundjoin%
\pgfsetlinewidth{1.505625pt}%
\definecolor{currentstroke}{rgb}{0.000000,0.000000,0.000000}%
\pgfsetstrokecolor{currentstroke}%
\pgfsetdash{}{0pt}%
\pgfpathmoveto{\pgfqpoint{13.263740in}{8.899169in}}%
\pgfpathlineto{\pgfqpoint{13.263740in}{8.921495in}}%
\pgfusepath{stroke}%
\end{pgfscope}%
\begin{pgfscope}%
\pgfpathrectangle{\pgfqpoint{9.810417in}{8.820930in}}{\pgfqpoint{5.489583in}{0.877907in}}%
\pgfusepath{clip}%
\pgfsetbuttcap%
\pgfsetroundjoin%
\pgfsetlinewidth{1.505625pt}%
\definecolor{currentstroke}{rgb}{0.000000,0.000000,0.000000}%
\pgfsetstrokecolor{currentstroke}%
\pgfsetdash{}{0pt}%
\pgfpathmoveto{\pgfqpoint{13.386963in}{8.899169in}}%
\pgfpathlineto{\pgfqpoint{13.386963in}{8.892523in}}%
\pgfusepath{stroke}%
\end{pgfscope}%
\begin{pgfscope}%
\pgfpathrectangle{\pgfqpoint{9.810417in}{8.820930in}}{\pgfqpoint{5.489583in}{0.877907in}}%
\pgfusepath{clip}%
\pgfsetbuttcap%
\pgfsetroundjoin%
\pgfsetlinewidth{1.505625pt}%
\definecolor{currentstroke}{rgb}{0.000000,0.000000,0.000000}%
\pgfsetstrokecolor{currentstroke}%
\pgfsetdash{}{0pt}%
\pgfpathmoveto{\pgfqpoint{13.510186in}{8.899169in}}%
\pgfpathlineto{\pgfqpoint{13.510186in}{8.902054in}}%
\pgfusepath{stroke}%
\end{pgfscope}%
\begin{pgfscope}%
\pgfpathrectangle{\pgfqpoint{9.810417in}{8.820930in}}{\pgfqpoint{5.489583in}{0.877907in}}%
\pgfusepath{clip}%
\pgfsetbuttcap%
\pgfsetroundjoin%
\pgfsetlinewidth{1.505625pt}%
\definecolor{currentstroke}{rgb}{0.000000,0.000000,0.000000}%
\pgfsetstrokecolor{currentstroke}%
\pgfsetdash{}{0pt}%
\pgfpathmoveto{\pgfqpoint{13.633409in}{8.899169in}}%
\pgfpathlineto{\pgfqpoint{13.633409in}{8.884742in}}%
\pgfusepath{stroke}%
\end{pgfscope}%
\begin{pgfscope}%
\pgfpathrectangle{\pgfqpoint{9.810417in}{8.820930in}}{\pgfqpoint{5.489583in}{0.877907in}}%
\pgfusepath{clip}%
\pgfsetbuttcap%
\pgfsetroundjoin%
\pgfsetlinewidth{1.505625pt}%
\definecolor{currentstroke}{rgb}{0.000000,0.000000,0.000000}%
\pgfsetstrokecolor{currentstroke}%
\pgfsetdash{}{0pt}%
\pgfpathmoveto{\pgfqpoint{13.756632in}{8.899169in}}%
\pgfpathlineto{\pgfqpoint{13.756632in}{8.874818in}}%
\pgfusepath{stroke}%
\end{pgfscope}%
\begin{pgfscope}%
\pgfpathrectangle{\pgfqpoint{9.810417in}{8.820930in}}{\pgfqpoint{5.489583in}{0.877907in}}%
\pgfusepath{clip}%
\pgfsetbuttcap%
\pgfsetroundjoin%
\pgfsetlinewidth{1.505625pt}%
\definecolor{currentstroke}{rgb}{0.000000,0.000000,0.000000}%
\pgfsetstrokecolor{currentstroke}%
\pgfsetdash{}{0pt}%
\pgfpathmoveto{\pgfqpoint{13.879855in}{8.899169in}}%
\pgfpathlineto{\pgfqpoint{13.879855in}{8.921886in}}%
\pgfusepath{stroke}%
\end{pgfscope}%
\begin{pgfscope}%
\pgfpathrectangle{\pgfqpoint{9.810417in}{8.820930in}}{\pgfqpoint{5.489583in}{0.877907in}}%
\pgfusepath{clip}%
\pgfsetbuttcap%
\pgfsetroundjoin%
\pgfsetlinewidth{1.505625pt}%
\definecolor{currentstroke}{rgb}{0.000000,0.000000,0.000000}%
\pgfsetstrokecolor{currentstroke}%
\pgfsetdash{}{0pt}%
\pgfpathmoveto{\pgfqpoint{14.003078in}{8.899169in}}%
\pgfpathlineto{\pgfqpoint{14.003078in}{8.907410in}}%
\pgfusepath{stroke}%
\end{pgfscope}%
\begin{pgfscope}%
\pgfpathrectangle{\pgfqpoint{9.810417in}{8.820930in}}{\pgfqpoint{5.489583in}{0.877907in}}%
\pgfusepath{clip}%
\pgfsetbuttcap%
\pgfsetroundjoin%
\pgfsetlinewidth{1.505625pt}%
\definecolor{currentstroke}{rgb}{0.000000,0.000000,0.000000}%
\pgfsetstrokecolor{currentstroke}%
\pgfsetdash{}{0pt}%
\pgfpathmoveto{\pgfqpoint{14.126301in}{8.899169in}}%
\pgfpathlineto{\pgfqpoint{14.126301in}{8.906983in}}%
\pgfusepath{stroke}%
\end{pgfscope}%
\begin{pgfscope}%
\pgfpathrectangle{\pgfqpoint{9.810417in}{8.820930in}}{\pgfqpoint{5.489583in}{0.877907in}}%
\pgfusepath{clip}%
\pgfsetbuttcap%
\pgfsetroundjoin%
\pgfsetlinewidth{1.505625pt}%
\definecolor{currentstroke}{rgb}{0.000000,0.000000,0.000000}%
\pgfsetstrokecolor{currentstroke}%
\pgfsetdash{}{0pt}%
\pgfpathmoveto{\pgfqpoint{14.249524in}{8.899169in}}%
\pgfpathlineto{\pgfqpoint{14.249524in}{8.907925in}}%
\pgfusepath{stroke}%
\end{pgfscope}%
\begin{pgfscope}%
\pgfpathrectangle{\pgfqpoint{9.810417in}{8.820930in}}{\pgfqpoint{5.489583in}{0.877907in}}%
\pgfusepath{clip}%
\pgfsetbuttcap%
\pgfsetroundjoin%
\pgfsetlinewidth{1.505625pt}%
\definecolor{currentstroke}{rgb}{0.000000,0.000000,0.000000}%
\pgfsetstrokecolor{currentstroke}%
\pgfsetdash{}{0pt}%
\pgfpathmoveto{\pgfqpoint{14.372747in}{8.899169in}}%
\pgfpathlineto{\pgfqpoint{14.372747in}{8.919555in}}%
\pgfusepath{stroke}%
\end{pgfscope}%
\begin{pgfscope}%
\pgfpathrectangle{\pgfqpoint{9.810417in}{8.820930in}}{\pgfqpoint{5.489583in}{0.877907in}}%
\pgfusepath{clip}%
\pgfsetbuttcap%
\pgfsetroundjoin%
\pgfsetlinewidth{1.505625pt}%
\definecolor{currentstroke}{rgb}{0.000000,0.000000,0.000000}%
\pgfsetstrokecolor{currentstroke}%
\pgfsetdash{}{0pt}%
\pgfpathmoveto{\pgfqpoint{14.495970in}{8.899169in}}%
\pgfpathlineto{\pgfqpoint{14.495970in}{8.952129in}}%
\pgfusepath{stroke}%
\end{pgfscope}%
\begin{pgfscope}%
\pgfpathrectangle{\pgfqpoint{9.810417in}{8.820930in}}{\pgfqpoint{5.489583in}{0.877907in}}%
\pgfusepath{clip}%
\pgfsetbuttcap%
\pgfsetroundjoin%
\pgfsetlinewidth{1.505625pt}%
\definecolor{currentstroke}{rgb}{0.000000,0.000000,0.000000}%
\pgfsetstrokecolor{currentstroke}%
\pgfsetdash{}{0pt}%
\pgfpathmoveto{\pgfqpoint{14.619193in}{8.899169in}}%
\pgfpathlineto{\pgfqpoint{14.619193in}{8.897277in}}%
\pgfusepath{stroke}%
\end{pgfscope}%
\begin{pgfscope}%
\pgfpathrectangle{\pgfqpoint{9.810417in}{8.820930in}}{\pgfqpoint{5.489583in}{0.877907in}}%
\pgfusepath{clip}%
\pgfsetbuttcap%
\pgfsetroundjoin%
\pgfsetlinewidth{1.505625pt}%
\definecolor{currentstroke}{rgb}{0.000000,0.000000,0.000000}%
\pgfsetstrokecolor{currentstroke}%
\pgfsetdash{}{0pt}%
\pgfpathmoveto{\pgfqpoint{14.742416in}{8.899169in}}%
\pgfpathlineto{\pgfqpoint{14.742416in}{8.917881in}}%
\pgfusepath{stroke}%
\end{pgfscope}%
\begin{pgfscope}%
\pgfpathrectangle{\pgfqpoint{9.810417in}{8.820930in}}{\pgfqpoint{5.489583in}{0.877907in}}%
\pgfusepath{clip}%
\pgfsetbuttcap%
\pgfsetroundjoin%
\pgfsetlinewidth{1.505625pt}%
\definecolor{currentstroke}{rgb}{0.000000,0.000000,0.000000}%
\pgfsetstrokecolor{currentstroke}%
\pgfsetdash{}{0pt}%
\pgfpathmoveto{\pgfqpoint{14.865639in}{8.899169in}}%
\pgfpathlineto{\pgfqpoint{14.865639in}{8.916638in}}%
\pgfusepath{stroke}%
\end{pgfscope}%
\begin{pgfscope}%
\pgfpathrectangle{\pgfqpoint{9.810417in}{8.820930in}}{\pgfqpoint{5.489583in}{0.877907in}}%
\pgfusepath{clip}%
\pgfsetbuttcap%
\pgfsetroundjoin%
\pgfsetlinewidth{1.505625pt}%
\definecolor{currentstroke}{rgb}{0.000000,0.000000,0.000000}%
\pgfsetstrokecolor{currentstroke}%
\pgfsetdash{}{0pt}%
\pgfpathmoveto{\pgfqpoint{14.988862in}{8.899169in}}%
\pgfpathlineto{\pgfqpoint{14.988862in}{8.914839in}}%
\pgfusepath{stroke}%
\end{pgfscope}%
\begin{pgfscope}%
\pgfpathrectangle{\pgfqpoint{9.810417in}{8.820930in}}{\pgfqpoint{5.489583in}{0.877907in}}%
\pgfusepath{clip}%
\pgfsetroundcap%
\pgfsetroundjoin%
\pgfsetlinewidth{1.505625pt}%
\definecolor{currentstroke}{rgb}{0.121569,0.466667,0.705882}%
\pgfsetstrokecolor{currentstroke}%
\pgfsetdash{}{0pt}%
\pgfpathmoveto{\pgfqpoint{9.810417in}{8.899169in}}%
\pgfpathlineto{\pgfqpoint{15.300000in}{8.899169in}}%
\pgfusepath{stroke}%
\end{pgfscope}%
\begin{pgfscope}%
\pgfpathrectangle{\pgfqpoint{9.810417in}{8.820930in}}{\pgfqpoint{5.489583in}{0.877907in}}%
\pgfusepath{clip}%
\pgfsetbuttcap%
\pgfsetroundjoin%
\definecolor{currentfill}{rgb}{0.121569,0.466667,0.705882}%
\pgfsetfillcolor{currentfill}%
\pgfsetlinewidth{1.003750pt}%
\definecolor{currentstroke}{rgb}{0.121569,0.466667,0.705882}%
\pgfsetstrokecolor{currentstroke}%
\pgfsetdash{}{0pt}%
\pgfsys@defobject{currentmarker}{\pgfqpoint{-0.034722in}{-0.034722in}}{\pgfqpoint{0.034722in}{0.034722in}}{%
\pgfpathmoveto{\pgfqpoint{0.000000in}{-0.034722in}}%
\pgfpathcurveto{\pgfqpoint{0.009208in}{-0.034722in}}{\pgfqpoint{0.018041in}{-0.031064in}}{\pgfqpoint{0.024552in}{-0.024552in}}%
\pgfpathcurveto{\pgfqpoint{0.031064in}{-0.018041in}}{\pgfqpoint{0.034722in}{-0.009208in}}{\pgfqpoint{0.034722in}{0.000000in}}%
\pgfpathcurveto{\pgfqpoint{0.034722in}{0.009208in}}{\pgfqpoint{0.031064in}{0.018041in}}{\pgfqpoint{0.024552in}{0.024552in}}%
\pgfpathcurveto{\pgfqpoint{0.018041in}{0.031064in}}{\pgfqpoint{0.009208in}{0.034722in}}{\pgfqpoint{0.000000in}{0.034722in}}%
\pgfpathcurveto{\pgfqpoint{-0.009208in}{0.034722in}}{\pgfqpoint{-0.018041in}{0.031064in}}{\pgfqpoint{-0.024552in}{0.024552in}}%
\pgfpathcurveto{\pgfqpoint{-0.031064in}{0.018041in}}{\pgfqpoint{-0.034722in}{0.009208in}}{\pgfqpoint{-0.034722in}{0.000000in}}%
\pgfpathcurveto{\pgfqpoint{-0.034722in}{-0.009208in}}{\pgfqpoint{-0.031064in}{-0.018041in}}{\pgfqpoint{-0.024552in}{-0.024552in}}%
\pgfpathcurveto{\pgfqpoint{-0.018041in}{-0.031064in}}{\pgfqpoint{-0.009208in}{-0.034722in}}{\pgfqpoint{0.000000in}{-0.034722in}}%
\pgfpathclose%
\pgfusepath{stroke,fill}%
}%
\begin{pgfscope}%
\pgfsys@transformshift{10.059943in}{9.658932in}%
\pgfsys@useobject{currentmarker}{}%
\end{pgfscope}%
\begin{pgfscope}%
\pgfsys@transformshift{10.183166in}{9.656717in}%
\pgfsys@useobject{currentmarker}{}%
\end{pgfscope}%
\begin{pgfscope}%
\pgfsys@transformshift{10.306389in}{8.903534in}%
\pgfsys@useobject{currentmarker}{}%
\end{pgfscope}%
\begin{pgfscope}%
\pgfsys@transformshift{10.429612in}{8.904148in}%
\pgfsys@useobject{currentmarker}{}%
\end{pgfscope}%
\begin{pgfscope}%
\pgfsys@transformshift{10.552835in}{8.897394in}%
\pgfsys@useobject{currentmarker}{}%
\end{pgfscope}%
\begin{pgfscope}%
\pgfsys@transformshift{10.676058in}{8.927893in}%
\pgfsys@useobject{currentmarker}{}%
\end{pgfscope}%
\begin{pgfscope}%
\pgfsys@transformshift{10.799281in}{8.880456in}%
\pgfsys@useobject{currentmarker}{}%
\end{pgfscope}%
\begin{pgfscope}%
\pgfsys@transformshift{10.922504in}{8.881465in}%
\pgfsys@useobject{currentmarker}{}%
\end{pgfscope}%
\begin{pgfscope}%
\pgfsys@transformshift{11.045727in}{8.892266in}%
\pgfsys@useobject{currentmarker}{}%
\end{pgfscope}%
\begin{pgfscope}%
\pgfsys@transformshift{11.168950in}{8.893383in}%
\pgfsys@useobject{currentmarker}{}%
\end{pgfscope}%
\begin{pgfscope}%
\pgfsys@transformshift{11.292173in}{8.913614in}%
\pgfsys@useobject{currentmarker}{}%
\end{pgfscope}%
\begin{pgfscope}%
\pgfsys@transformshift{11.415396in}{8.920884in}%
\pgfsys@useobject{currentmarker}{}%
\end{pgfscope}%
\begin{pgfscope}%
\pgfsys@transformshift{11.538619in}{8.899769in}%
\pgfsys@useobject{currentmarker}{}%
\end{pgfscope}%
\begin{pgfscope}%
\pgfsys@transformshift{11.661842in}{8.911209in}%
\pgfsys@useobject{currentmarker}{}%
\end{pgfscope}%
\begin{pgfscope}%
\pgfsys@transformshift{11.785065in}{8.877984in}%
\pgfsys@useobject{currentmarker}{}%
\end{pgfscope}%
\begin{pgfscope}%
\pgfsys@transformshift{11.908288in}{8.889088in}%
\pgfsys@useobject{currentmarker}{}%
\end{pgfscope}%
\begin{pgfscope}%
\pgfsys@transformshift{12.031511in}{8.917679in}%
\pgfsys@useobject{currentmarker}{}%
\end{pgfscope}%
\begin{pgfscope}%
\pgfsys@transformshift{12.154734in}{8.865466in}%
\pgfsys@useobject{currentmarker}{}%
\end{pgfscope}%
\begin{pgfscope}%
\pgfsys@transformshift{12.277957in}{8.875761in}%
\pgfsys@useobject{currentmarker}{}%
\end{pgfscope}%
\begin{pgfscope}%
\pgfsys@transformshift{12.401180in}{8.875551in}%
\pgfsys@useobject{currentmarker}{}%
\end{pgfscope}%
\begin{pgfscope}%
\pgfsys@transformshift{12.524403in}{8.904916in}%
\pgfsys@useobject{currentmarker}{}%
\end{pgfscope}%
\begin{pgfscope}%
\pgfsys@transformshift{12.647626in}{8.909892in}%
\pgfsys@useobject{currentmarker}{}%
\end{pgfscope}%
\begin{pgfscope}%
\pgfsys@transformshift{12.770849in}{8.898822in}%
\pgfsys@useobject{currentmarker}{}%
\end{pgfscope}%
\begin{pgfscope}%
\pgfsys@transformshift{12.894072in}{8.884663in}%
\pgfsys@useobject{currentmarker}{}%
\end{pgfscope}%
\begin{pgfscope}%
\pgfsys@transformshift{13.017294in}{8.907107in}%
\pgfsys@useobject{currentmarker}{}%
\end{pgfscope}%
\begin{pgfscope}%
\pgfsys@transformshift{13.140517in}{8.892804in}%
\pgfsys@useobject{currentmarker}{}%
\end{pgfscope}%
\begin{pgfscope}%
\pgfsys@transformshift{13.263740in}{8.921495in}%
\pgfsys@useobject{currentmarker}{}%
\end{pgfscope}%
\begin{pgfscope}%
\pgfsys@transformshift{13.386963in}{8.892523in}%
\pgfsys@useobject{currentmarker}{}%
\end{pgfscope}%
\begin{pgfscope}%
\pgfsys@transformshift{13.510186in}{8.902054in}%
\pgfsys@useobject{currentmarker}{}%
\end{pgfscope}%
\begin{pgfscope}%
\pgfsys@transformshift{13.633409in}{8.884742in}%
\pgfsys@useobject{currentmarker}{}%
\end{pgfscope}%
\begin{pgfscope}%
\pgfsys@transformshift{13.756632in}{8.874818in}%
\pgfsys@useobject{currentmarker}{}%
\end{pgfscope}%
\begin{pgfscope}%
\pgfsys@transformshift{13.879855in}{8.921886in}%
\pgfsys@useobject{currentmarker}{}%
\end{pgfscope}%
\begin{pgfscope}%
\pgfsys@transformshift{14.003078in}{8.907410in}%
\pgfsys@useobject{currentmarker}{}%
\end{pgfscope}%
\begin{pgfscope}%
\pgfsys@transformshift{14.126301in}{8.906983in}%
\pgfsys@useobject{currentmarker}{}%
\end{pgfscope}%
\begin{pgfscope}%
\pgfsys@transformshift{14.249524in}{8.907925in}%
\pgfsys@useobject{currentmarker}{}%
\end{pgfscope}%
\begin{pgfscope}%
\pgfsys@transformshift{14.372747in}{8.919555in}%
\pgfsys@useobject{currentmarker}{}%
\end{pgfscope}%
\begin{pgfscope}%
\pgfsys@transformshift{14.495970in}{8.952129in}%
\pgfsys@useobject{currentmarker}{}%
\end{pgfscope}%
\begin{pgfscope}%
\pgfsys@transformshift{14.619193in}{8.897277in}%
\pgfsys@useobject{currentmarker}{}%
\end{pgfscope}%
\begin{pgfscope}%
\pgfsys@transformshift{14.742416in}{8.917881in}%
\pgfsys@useobject{currentmarker}{}%
\end{pgfscope}%
\begin{pgfscope}%
\pgfsys@transformshift{14.865639in}{8.916638in}%
\pgfsys@useobject{currentmarker}{}%
\end{pgfscope}%
\begin{pgfscope}%
\pgfsys@transformshift{14.988862in}{8.914839in}%
\pgfsys@useobject{currentmarker}{}%
\end{pgfscope}%
\end{pgfscope}%
\begin{pgfscope}%
\pgfsetrectcap%
\pgfsetmiterjoin%
\pgfsetlinewidth{0.803000pt}%
\definecolor{currentstroke}{rgb}{1.000000,1.000000,1.000000}%
\pgfsetstrokecolor{currentstroke}%
\pgfsetdash{}{0pt}%
\pgfpathmoveto{\pgfqpoint{9.810417in}{8.820930in}}%
\pgfpathlineto{\pgfqpoint{9.810417in}{9.698837in}}%
\pgfusepath{stroke}%
\end{pgfscope}%
\begin{pgfscope}%
\pgfsetrectcap%
\pgfsetmiterjoin%
\pgfsetlinewidth{0.803000pt}%
\definecolor{currentstroke}{rgb}{1.000000,1.000000,1.000000}%
\pgfsetstrokecolor{currentstroke}%
\pgfsetdash{}{0pt}%
\pgfpathmoveto{\pgfqpoint{15.300000in}{8.820930in}}%
\pgfpathlineto{\pgfqpoint{15.300000in}{9.698837in}}%
\pgfusepath{stroke}%
\end{pgfscope}%
\begin{pgfscope}%
\pgfsetrectcap%
\pgfsetmiterjoin%
\pgfsetlinewidth{0.803000pt}%
\definecolor{currentstroke}{rgb}{1.000000,1.000000,1.000000}%
\pgfsetstrokecolor{currentstroke}%
\pgfsetdash{}{0pt}%
\pgfpathmoveto{\pgfqpoint{9.810417in}{8.820930in}}%
\pgfpathlineto{\pgfqpoint{15.300000in}{8.820930in}}%
\pgfusepath{stroke}%
\end{pgfscope}%
\begin{pgfscope}%
\pgfsetrectcap%
\pgfsetmiterjoin%
\pgfsetlinewidth{0.803000pt}%
\definecolor{currentstroke}{rgb}{1.000000,1.000000,1.000000}%
\pgfsetstrokecolor{currentstroke}%
\pgfsetdash{}{0pt}%
\pgfpathmoveto{\pgfqpoint{9.810417in}{9.698837in}}%
\pgfpathlineto{\pgfqpoint{15.300000in}{9.698837in}}%
\pgfusepath{stroke}%
\end{pgfscope}%
\begin{pgfscope}%
\definecolor{textcolor}{rgb}{0.150000,0.150000,0.150000}%
\pgfsetstrokecolor{textcolor}%
\pgfsetfillcolor{textcolor}%
\pgftext[x=12.555208in,y=9.782171in,,base]{\color{textcolor}\rmfamily\fontsize{16.800000}{20.160000}\selectfont Partial Autocorrelation}%
\end{pgfscope}%
\begin{pgfscope}%
\pgfsetbuttcap%
\pgfsetmiterjoin%
\definecolor{currentfill}{rgb}{0.917647,0.917647,0.949020}%
\pgfsetfillcolor{currentfill}%
\pgfsetlinewidth{0.000000pt}%
\definecolor{currentstroke}{rgb}{0.000000,0.000000,0.000000}%
\pgfsetstrokecolor{currentstroke}%
\pgfsetstrokeopacity{0.000000}%
\pgfsetdash{}{0pt}%
\pgfpathmoveto{\pgfqpoint{2.125000in}{7.240698in}}%
\pgfpathlineto{\pgfqpoint{7.614583in}{7.240698in}}%
\pgfpathlineto{\pgfqpoint{7.614583in}{8.118605in}}%
\pgfpathlineto{\pgfqpoint{2.125000in}{8.118605in}}%
\pgfpathclose%
\pgfusepath{fill}%
\end{pgfscope}%
\begin{pgfscope}%
\pgfpathrectangle{\pgfqpoint{2.125000in}{7.240698in}}{\pgfqpoint{5.489583in}{0.877907in}}%
\pgfusepath{clip}%
\pgfsetroundcap%
\pgfsetroundjoin%
\pgfsetlinewidth{0.803000pt}%
\definecolor{currentstroke}{rgb}{1.000000,1.000000,1.000000}%
\pgfsetstrokecolor{currentstroke}%
\pgfsetdash{}{0pt}%
\pgfpathmoveto{\pgfqpoint{2.374527in}{7.240698in}}%
\pgfpathlineto{\pgfqpoint{2.374527in}{8.118605in}}%
\pgfusepath{stroke}%
\end{pgfscope}%
\begin{pgfscope}%
\definecolor{textcolor}{rgb}{0.150000,0.150000,0.150000}%
\pgfsetstrokecolor{textcolor}%
\pgfsetfillcolor{textcolor}%
\pgftext[x=2.374527in,y=7.143475in,,top]{\color{textcolor}\rmfamily\fontsize{14.000000}{16.800000}\selectfont 0}%
\end{pgfscope}%
\begin{pgfscope}%
\pgfpathrectangle{\pgfqpoint{2.125000in}{7.240698in}}{\pgfqpoint{5.489583in}{0.877907in}}%
\pgfusepath{clip}%
\pgfsetroundcap%
\pgfsetroundjoin%
\pgfsetlinewidth{0.803000pt}%
\definecolor{currentstroke}{rgb}{1.000000,1.000000,1.000000}%
\pgfsetstrokecolor{currentstroke}%
\pgfsetdash{}{0pt}%
\pgfpathmoveto{\pgfqpoint{2.990641in}{7.240698in}}%
\pgfpathlineto{\pgfqpoint{2.990641in}{8.118605in}}%
\pgfusepath{stroke}%
\end{pgfscope}%
\begin{pgfscope}%
\definecolor{textcolor}{rgb}{0.150000,0.150000,0.150000}%
\pgfsetstrokecolor{textcolor}%
\pgfsetfillcolor{textcolor}%
\pgftext[x=2.990641in,y=7.143475in,,top]{\color{textcolor}\rmfamily\fontsize{14.000000}{16.800000}\selectfont 5}%
\end{pgfscope}%
\begin{pgfscope}%
\pgfpathrectangle{\pgfqpoint{2.125000in}{7.240698in}}{\pgfqpoint{5.489583in}{0.877907in}}%
\pgfusepath{clip}%
\pgfsetroundcap%
\pgfsetroundjoin%
\pgfsetlinewidth{0.803000pt}%
\definecolor{currentstroke}{rgb}{1.000000,1.000000,1.000000}%
\pgfsetstrokecolor{currentstroke}%
\pgfsetdash{}{0pt}%
\pgfpathmoveto{\pgfqpoint{3.606756in}{7.240698in}}%
\pgfpathlineto{\pgfqpoint{3.606756in}{8.118605in}}%
\pgfusepath{stroke}%
\end{pgfscope}%
\begin{pgfscope}%
\definecolor{textcolor}{rgb}{0.150000,0.150000,0.150000}%
\pgfsetstrokecolor{textcolor}%
\pgfsetfillcolor{textcolor}%
\pgftext[x=3.606756in,y=7.143475in,,top]{\color{textcolor}\rmfamily\fontsize{14.000000}{16.800000}\selectfont 10}%
\end{pgfscope}%
\begin{pgfscope}%
\pgfpathrectangle{\pgfqpoint{2.125000in}{7.240698in}}{\pgfqpoint{5.489583in}{0.877907in}}%
\pgfusepath{clip}%
\pgfsetroundcap%
\pgfsetroundjoin%
\pgfsetlinewidth{0.803000pt}%
\definecolor{currentstroke}{rgb}{1.000000,1.000000,1.000000}%
\pgfsetstrokecolor{currentstroke}%
\pgfsetdash{}{0pt}%
\pgfpathmoveto{\pgfqpoint{4.222871in}{7.240698in}}%
\pgfpathlineto{\pgfqpoint{4.222871in}{8.118605in}}%
\pgfusepath{stroke}%
\end{pgfscope}%
\begin{pgfscope}%
\definecolor{textcolor}{rgb}{0.150000,0.150000,0.150000}%
\pgfsetstrokecolor{textcolor}%
\pgfsetfillcolor{textcolor}%
\pgftext[x=4.222871in,y=7.143475in,,top]{\color{textcolor}\rmfamily\fontsize{14.000000}{16.800000}\selectfont 15}%
\end{pgfscope}%
\begin{pgfscope}%
\pgfpathrectangle{\pgfqpoint{2.125000in}{7.240698in}}{\pgfqpoint{5.489583in}{0.877907in}}%
\pgfusepath{clip}%
\pgfsetroundcap%
\pgfsetroundjoin%
\pgfsetlinewidth{0.803000pt}%
\definecolor{currentstroke}{rgb}{1.000000,1.000000,1.000000}%
\pgfsetstrokecolor{currentstroke}%
\pgfsetdash{}{0pt}%
\pgfpathmoveto{\pgfqpoint{4.838986in}{7.240698in}}%
\pgfpathlineto{\pgfqpoint{4.838986in}{8.118605in}}%
\pgfusepath{stroke}%
\end{pgfscope}%
\begin{pgfscope}%
\definecolor{textcolor}{rgb}{0.150000,0.150000,0.150000}%
\pgfsetstrokecolor{textcolor}%
\pgfsetfillcolor{textcolor}%
\pgftext[x=4.838986in,y=7.143475in,,top]{\color{textcolor}\rmfamily\fontsize{14.000000}{16.800000}\selectfont 20}%
\end{pgfscope}%
\begin{pgfscope}%
\pgfpathrectangle{\pgfqpoint{2.125000in}{7.240698in}}{\pgfqpoint{5.489583in}{0.877907in}}%
\pgfusepath{clip}%
\pgfsetroundcap%
\pgfsetroundjoin%
\pgfsetlinewidth{0.803000pt}%
\definecolor{currentstroke}{rgb}{1.000000,1.000000,1.000000}%
\pgfsetstrokecolor{currentstroke}%
\pgfsetdash{}{0pt}%
\pgfpathmoveto{\pgfqpoint{5.455101in}{7.240698in}}%
\pgfpathlineto{\pgfqpoint{5.455101in}{8.118605in}}%
\pgfusepath{stroke}%
\end{pgfscope}%
\begin{pgfscope}%
\definecolor{textcolor}{rgb}{0.150000,0.150000,0.150000}%
\pgfsetstrokecolor{textcolor}%
\pgfsetfillcolor{textcolor}%
\pgftext[x=5.455101in,y=7.143475in,,top]{\color{textcolor}\rmfamily\fontsize{14.000000}{16.800000}\selectfont 25}%
\end{pgfscope}%
\begin{pgfscope}%
\pgfpathrectangle{\pgfqpoint{2.125000in}{7.240698in}}{\pgfqpoint{5.489583in}{0.877907in}}%
\pgfusepath{clip}%
\pgfsetroundcap%
\pgfsetroundjoin%
\pgfsetlinewidth{0.803000pt}%
\definecolor{currentstroke}{rgb}{1.000000,1.000000,1.000000}%
\pgfsetstrokecolor{currentstroke}%
\pgfsetdash{}{0pt}%
\pgfpathmoveto{\pgfqpoint{6.071216in}{7.240698in}}%
\pgfpathlineto{\pgfqpoint{6.071216in}{8.118605in}}%
\pgfusepath{stroke}%
\end{pgfscope}%
\begin{pgfscope}%
\definecolor{textcolor}{rgb}{0.150000,0.150000,0.150000}%
\pgfsetstrokecolor{textcolor}%
\pgfsetfillcolor{textcolor}%
\pgftext[x=6.071216in,y=7.143475in,,top]{\color{textcolor}\rmfamily\fontsize{14.000000}{16.800000}\selectfont 30}%
\end{pgfscope}%
\begin{pgfscope}%
\pgfpathrectangle{\pgfqpoint{2.125000in}{7.240698in}}{\pgfqpoint{5.489583in}{0.877907in}}%
\pgfusepath{clip}%
\pgfsetroundcap%
\pgfsetroundjoin%
\pgfsetlinewidth{0.803000pt}%
\definecolor{currentstroke}{rgb}{1.000000,1.000000,1.000000}%
\pgfsetstrokecolor{currentstroke}%
\pgfsetdash{}{0pt}%
\pgfpathmoveto{\pgfqpoint{6.687330in}{7.240698in}}%
\pgfpathlineto{\pgfqpoint{6.687330in}{8.118605in}}%
\pgfusepath{stroke}%
\end{pgfscope}%
\begin{pgfscope}%
\definecolor{textcolor}{rgb}{0.150000,0.150000,0.150000}%
\pgfsetstrokecolor{textcolor}%
\pgfsetfillcolor{textcolor}%
\pgftext[x=6.687330in,y=7.143475in,,top]{\color{textcolor}\rmfamily\fontsize{14.000000}{16.800000}\selectfont 35}%
\end{pgfscope}%
\begin{pgfscope}%
\pgfpathrectangle{\pgfqpoint{2.125000in}{7.240698in}}{\pgfqpoint{5.489583in}{0.877907in}}%
\pgfusepath{clip}%
\pgfsetroundcap%
\pgfsetroundjoin%
\pgfsetlinewidth{0.803000pt}%
\definecolor{currentstroke}{rgb}{1.000000,1.000000,1.000000}%
\pgfsetstrokecolor{currentstroke}%
\pgfsetdash{}{0pt}%
\pgfpathmoveto{\pgfqpoint{7.303445in}{7.240698in}}%
\pgfpathlineto{\pgfqpoint{7.303445in}{8.118605in}}%
\pgfusepath{stroke}%
\end{pgfscope}%
\begin{pgfscope}%
\definecolor{textcolor}{rgb}{0.150000,0.150000,0.150000}%
\pgfsetstrokecolor{textcolor}%
\pgfsetfillcolor{textcolor}%
\pgftext[x=7.303445in,y=7.143475in,,top]{\color{textcolor}\rmfamily\fontsize{14.000000}{16.800000}\selectfont 40}%
\end{pgfscope}%
\begin{pgfscope}%
\pgfpathrectangle{\pgfqpoint{2.125000in}{7.240698in}}{\pgfqpoint{5.489583in}{0.877907in}}%
\pgfusepath{clip}%
\pgfsetroundcap%
\pgfsetroundjoin%
\pgfsetlinewidth{0.803000pt}%
\definecolor{currentstroke}{rgb}{1.000000,1.000000,1.000000}%
\pgfsetstrokecolor{currentstroke}%
\pgfsetdash{}{0pt}%
\pgfpathmoveto{\pgfqpoint{2.125000in}{7.514385in}}%
\pgfpathlineto{\pgfqpoint{7.614583in}{7.514385in}}%
\pgfusepath{stroke}%
\end{pgfscope}%
\begin{pgfscope}%
\definecolor{textcolor}{rgb}{0.150000,0.150000,0.150000}%
\pgfsetstrokecolor{textcolor}%
\pgfsetfillcolor{textcolor}%
\pgftext[x=1.904066in,y=7.440518in,left,base]{\color{textcolor}\rmfamily\fontsize{14.000000}{16.800000}\selectfont 0}%
\end{pgfscope}%
\begin{pgfscope}%
\pgfpathrectangle{\pgfqpoint{2.125000in}{7.240698in}}{\pgfqpoint{5.489583in}{0.877907in}}%
\pgfusepath{clip}%
\pgfsetroundcap%
\pgfsetroundjoin%
\pgfsetlinewidth{0.803000pt}%
\definecolor{currentstroke}{rgb}{1.000000,1.000000,1.000000}%
\pgfsetstrokecolor{currentstroke}%
\pgfsetdash{}{0pt}%
\pgfpathmoveto{\pgfqpoint{2.125000in}{8.078700in}}%
\pgfpathlineto{\pgfqpoint{7.614583in}{8.078700in}}%
\pgfusepath{stroke}%
\end{pgfscope}%
\begin{pgfscope}%
\definecolor{textcolor}{rgb}{0.150000,0.150000,0.150000}%
\pgfsetstrokecolor{textcolor}%
\pgfsetfillcolor{textcolor}%
\pgftext[x=1.904066in,y=8.004834in,left,base]{\color{textcolor}\rmfamily\fontsize{14.000000}{16.800000}\selectfont 1}%
\end{pgfscope}%
\begin{pgfscope}%
\pgfpathrectangle{\pgfqpoint{2.125000in}{7.240698in}}{\pgfqpoint{5.489583in}{0.877907in}}%
\pgfusepath{clip}%
\pgfsetbuttcap%
\pgfsetroundjoin%
\definecolor{currentfill}{rgb}{0.121569,0.466667,0.705882}%
\pgfsetfillcolor{currentfill}%
\pgfsetfillopacity{0.250000}%
\pgfsetlinewidth{1.003750pt}%
\definecolor{currentstroke}{rgb}{1.000000,1.000000,1.000000}%
\pgfsetstrokecolor{currentstroke}%
\pgfsetstrokeopacity{0.250000}%
\pgfsetdash{}{0pt}%
\pgfpathmoveto{\pgfqpoint{2.436138in}{7.542857in}}%
\pgfpathlineto{\pgfqpoint{2.436138in}{7.485912in}}%
\pgfpathlineto{\pgfqpoint{2.620972in}{7.465211in}}%
\pgfpathlineto{\pgfqpoint{2.744195in}{7.451052in}}%
\pgfpathlineto{\pgfqpoint{2.867418in}{7.439618in}}%
\pgfpathlineto{\pgfqpoint{2.990641in}{7.429800in}}%
\pgfpathlineto{\pgfqpoint{3.113864in}{7.421083in}}%
\pgfpathlineto{\pgfqpoint{3.237087in}{7.413177in}}%
\pgfpathlineto{\pgfqpoint{3.360310in}{7.405902in}}%
\pgfpathlineto{\pgfqpoint{3.483533in}{7.399140in}}%
\pgfpathlineto{\pgfqpoint{3.606756in}{7.392806in}}%
\pgfpathlineto{\pgfqpoint{3.729979in}{7.386838in}}%
\pgfpathlineto{\pgfqpoint{3.853202in}{7.381183in}}%
\pgfpathlineto{\pgfqpoint{3.976425in}{7.375803in}}%
\pgfpathlineto{\pgfqpoint{4.099648in}{7.370670in}}%
\pgfpathlineto{\pgfqpoint{4.222871in}{7.365755in}}%
\pgfpathlineto{\pgfqpoint{4.346094in}{7.361035in}}%
\pgfpathlineto{\pgfqpoint{4.469317in}{7.356492in}}%
\pgfpathlineto{\pgfqpoint{4.592540in}{7.352111in}}%
\pgfpathlineto{\pgfqpoint{4.715763in}{7.347878in}}%
\pgfpathlineto{\pgfqpoint{4.838986in}{7.343783in}}%
\pgfpathlineto{\pgfqpoint{4.962209in}{7.339816in}}%
\pgfpathlineto{\pgfqpoint{5.085432in}{7.335968in}}%
\pgfpathlineto{\pgfqpoint{5.208655in}{7.332231in}}%
\pgfpathlineto{\pgfqpoint{5.331878in}{7.328596in}}%
\pgfpathlineto{\pgfqpoint{5.455101in}{7.325059in}}%
\pgfpathlineto{\pgfqpoint{5.578324in}{7.321613in}}%
\pgfpathlineto{\pgfqpoint{5.701547in}{7.318253in}}%
\pgfpathlineto{\pgfqpoint{5.824770in}{7.314971in}}%
\pgfpathlineto{\pgfqpoint{5.947993in}{7.311765in}}%
\pgfpathlineto{\pgfqpoint{6.071216in}{7.308630in}}%
\pgfpathlineto{\pgfqpoint{6.194439in}{7.305564in}}%
\pgfpathlineto{\pgfqpoint{6.317662in}{7.302565in}}%
\pgfpathlineto{\pgfqpoint{6.440885in}{7.299627in}}%
\pgfpathlineto{\pgfqpoint{6.564108in}{7.296749in}}%
\pgfpathlineto{\pgfqpoint{6.687330in}{7.293925in}}%
\pgfpathlineto{\pgfqpoint{6.810553in}{7.291156in}}%
\pgfpathlineto{\pgfqpoint{6.933776in}{7.288440in}}%
\pgfpathlineto{\pgfqpoint{7.056999in}{7.285778in}}%
\pgfpathlineto{\pgfqpoint{7.180222in}{7.283166in}}%
\pgfpathlineto{\pgfqpoint{7.365057in}{7.280603in}}%
\pgfpathlineto{\pgfqpoint{7.365057in}{7.748166in}}%
\pgfpathlineto{\pgfqpoint{7.365057in}{7.748166in}}%
\pgfpathlineto{\pgfqpoint{7.180222in}{7.745603in}}%
\pgfpathlineto{\pgfqpoint{7.056999in}{7.742991in}}%
\pgfpathlineto{\pgfqpoint{6.933776in}{7.740329in}}%
\pgfpathlineto{\pgfqpoint{6.810553in}{7.737613in}}%
\pgfpathlineto{\pgfqpoint{6.687330in}{7.734844in}}%
\pgfpathlineto{\pgfqpoint{6.564108in}{7.732020in}}%
\pgfpathlineto{\pgfqpoint{6.440885in}{7.729142in}}%
\pgfpathlineto{\pgfqpoint{6.317662in}{7.726204in}}%
\pgfpathlineto{\pgfqpoint{6.194439in}{7.723205in}}%
\pgfpathlineto{\pgfqpoint{6.071216in}{7.720139in}}%
\pgfpathlineto{\pgfqpoint{5.947993in}{7.717004in}}%
\pgfpathlineto{\pgfqpoint{5.824770in}{7.713798in}}%
\pgfpathlineto{\pgfqpoint{5.701547in}{7.710516in}}%
\pgfpathlineto{\pgfqpoint{5.578324in}{7.707156in}}%
\pgfpathlineto{\pgfqpoint{5.455101in}{7.703710in}}%
\pgfpathlineto{\pgfqpoint{5.331878in}{7.700173in}}%
\pgfpathlineto{\pgfqpoint{5.208655in}{7.696538in}}%
\pgfpathlineto{\pgfqpoint{5.085432in}{7.692801in}}%
\pgfpathlineto{\pgfqpoint{4.962209in}{7.688953in}}%
\pgfpathlineto{\pgfqpoint{4.838986in}{7.684986in}}%
\pgfpathlineto{\pgfqpoint{4.715763in}{7.680891in}}%
\pgfpathlineto{\pgfqpoint{4.592540in}{7.676658in}}%
\pgfpathlineto{\pgfqpoint{4.469317in}{7.672277in}}%
\pgfpathlineto{\pgfqpoint{4.346094in}{7.667734in}}%
\pgfpathlineto{\pgfqpoint{4.222871in}{7.663014in}}%
\pgfpathlineto{\pgfqpoint{4.099648in}{7.658099in}}%
\pgfpathlineto{\pgfqpoint{3.976425in}{7.652966in}}%
\pgfpathlineto{\pgfqpoint{3.853202in}{7.647586in}}%
\pgfpathlineto{\pgfqpoint{3.729979in}{7.641931in}}%
\pgfpathlineto{\pgfqpoint{3.606756in}{7.635963in}}%
\pgfpathlineto{\pgfqpoint{3.483533in}{7.629629in}}%
\pgfpathlineto{\pgfqpoint{3.360310in}{7.622867in}}%
\pgfpathlineto{\pgfqpoint{3.237087in}{7.615592in}}%
\pgfpathlineto{\pgfqpoint{3.113864in}{7.607686in}}%
\pgfpathlineto{\pgfqpoint{2.990641in}{7.598969in}}%
\pgfpathlineto{\pgfqpoint{2.867418in}{7.589151in}}%
\pgfpathlineto{\pgfqpoint{2.744195in}{7.577717in}}%
\pgfpathlineto{\pgfqpoint{2.620972in}{7.563558in}}%
\pgfpathlineto{\pgfqpoint{2.436138in}{7.542857in}}%
\pgfpathclose%
\pgfusepath{stroke,fill}%
\end{pgfscope}%
\begin{pgfscope}%
\pgfpathrectangle{\pgfqpoint{2.125000in}{7.240698in}}{\pgfqpoint{5.489583in}{0.877907in}}%
\pgfusepath{clip}%
\pgfsetbuttcap%
\pgfsetroundjoin%
\pgfsetlinewidth{1.505625pt}%
\definecolor{currentstroke}{rgb}{0.000000,0.000000,0.000000}%
\pgfsetstrokecolor{currentstroke}%
\pgfsetdash{}{0pt}%
\pgfpathmoveto{\pgfqpoint{2.374527in}{7.514385in}}%
\pgfpathlineto{\pgfqpoint{2.374527in}{8.078700in}}%
\pgfusepath{stroke}%
\end{pgfscope}%
\begin{pgfscope}%
\pgfpathrectangle{\pgfqpoint{2.125000in}{7.240698in}}{\pgfqpoint{5.489583in}{0.877907in}}%
\pgfusepath{clip}%
\pgfsetbuttcap%
\pgfsetroundjoin%
\pgfsetlinewidth{1.505625pt}%
\definecolor{currentstroke}{rgb}{0.000000,0.000000,0.000000}%
\pgfsetstrokecolor{currentstroke}%
\pgfsetdash{}{0pt}%
\pgfpathmoveto{\pgfqpoint{2.497749in}{7.514385in}}%
\pgfpathlineto{\pgfqpoint{2.497749in}{8.076247in}}%
\pgfusepath{stroke}%
\end{pgfscope}%
\begin{pgfscope}%
\pgfpathrectangle{\pgfqpoint{2.125000in}{7.240698in}}{\pgfqpoint{5.489583in}{0.877907in}}%
\pgfusepath{clip}%
\pgfsetbuttcap%
\pgfsetroundjoin%
\pgfsetlinewidth{1.505625pt}%
\definecolor{currentstroke}{rgb}{0.000000,0.000000,0.000000}%
\pgfsetstrokecolor{currentstroke}%
\pgfsetdash{}{0pt}%
\pgfpathmoveto{\pgfqpoint{2.620972in}{7.514385in}}%
\pgfpathlineto{\pgfqpoint{2.620972in}{8.073754in}}%
\pgfusepath{stroke}%
\end{pgfscope}%
\begin{pgfscope}%
\pgfpathrectangle{\pgfqpoint{2.125000in}{7.240698in}}{\pgfqpoint{5.489583in}{0.877907in}}%
\pgfusepath{clip}%
\pgfsetbuttcap%
\pgfsetroundjoin%
\pgfsetlinewidth{1.505625pt}%
\definecolor{currentstroke}{rgb}{0.000000,0.000000,0.000000}%
\pgfsetstrokecolor{currentstroke}%
\pgfsetdash{}{0pt}%
\pgfpathmoveto{\pgfqpoint{2.744195in}{7.514385in}}%
\pgfpathlineto{\pgfqpoint{2.744195in}{8.071266in}}%
\pgfusepath{stroke}%
\end{pgfscope}%
\begin{pgfscope}%
\pgfpathrectangle{\pgfqpoint{2.125000in}{7.240698in}}{\pgfqpoint{5.489583in}{0.877907in}}%
\pgfusepath{clip}%
\pgfsetbuttcap%
\pgfsetroundjoin%
\pgfsetlinewidth{1.505625pt}%
\definecolor{currentstroke}{rgb}{0.000000,0.000000,0.000000}%
\pgfsetstrokecolor{currentstroke}%
\pgfsetdash{}{0pt}%
\pgfpathmoveto{\pgfqpoint{2.867418in}{7.514385in}}%
\pgfpathlineto{\pgfqpoint{2.867418in}{8.068713in}}%
\pgfusepath{stroke}%
\end{pgfscope}%
\begin{pgfscope}%
\pgfpathrectangle{\pgfqpoint{2.125000in}{7.240698in}}{\pgfqpoint{5.489583in}{0.877907in}}%
\pgfusepath{clip}%
\pgfsetbuttcap%
\pgfsetroundjoin%
\pgfsetlinewidth{1.505625pt}%
\definecolor{currentstroke}{rgb}{0.000000,0.000000,0.000000}%
\pgfsetstrokecolor{currentstroke}%
\pgfsetdash{}{0pt}%
\pgfpathmoveto{\pgfqpoint{2.990641in}{7.514385in}}%
\pgfpathlineto{\pgfqpoint{2.990641in}{8.066264in}}%
\pgfusepath{stroke}%
\end{pgfscope}%
\begin{pgfscope}%
\pgfpathrectangle{\pgfqpoint{2.125000in}{7.240698in}}{\pgfqpoint{5.489583in}{0.877907in}}%
\pgfusepath{clip}%
\pgfsetbuttcap%
\pgfsetroundjoin%
\pgfsetlinewidth{1.505625pt}%
\definecolor{currentstroke}{rgb}{0.000000,0.000000,0.000000}%
\pgfsetstrokecolor{currentstroke}%
\pgfsetdash{}{0pt}%
\pgfpathmoveto{\pgfqpoint{3.113864in}{7.514385in}}%
\pgfpathlineto{\pgfqpoint{3.113864in}{8.063966in}}%
\pgfusepath{stroke}%
\end{pgfscope}%
\begin{pgfscope}%
\pgfpathrectangle{\pgfqpoint{2.125000in}{7.240698in}}{\pgfqpoint{5.489583in}{0.877907in}}%
\pgfusepath{clip}%
\pgfsetbuttcap%
\pgfsetroundjoin%
\pgfsetlinewidth{1.505625pt}%
\definecolor{currentstroke}{rgb}{0.000000,0.000000,0.000000}%
\pgfsetstrokecolor{currentstroke}%
\pgfsetdash{}{0pt}%
\pgfpathmoveto{\pgfqpoint{3.237087in}{7.514385in}}%
\pgfpathlineto{\pgfqpoint{3.237087in}{8.061742in}}%
\pgfusepath{stroke}%
\end{pgfscope}%
\begin{pgfscope}%
\pgfpathrectangle{\pgfqpoint{2.125000in}{7.240698in}}{\pgfqpoint{5.489583in}{0.877907in}}%
\pgfusepath{clip}%
\pgfsetbuttcap%
\pgfsetroundjoin%
\pgfsetlinewidth{1.505625pt}%
\definecolor{currentstroke}{rgb}{0.000000,0.000000,0.000000}%
\pgfsetstrokecolor{currentstroke}%
\pgfsetdash{}{0pt}%
\pgfpathmoveto{\pgfqpoint{3.360310in}{7.514385in}}%
\pgfpathlineto{\pgfqpoint{3.360310in}{8.059523in}}%
\pgfusepath{stroke}%
\end{pgfscope}%
\begin{pgfscope}%
\pgfpathrectangle{\pgfqpoint{2.125000in}{7.240698in}}{\pgfqpoint{5.489583in}{0.877907in}}%
\pgfusepath{clip}%
\pgfsetbuttcap%
\pgfsetroundjoin%
\pgfsetlinewidth{1.505625pt}%
\definecolor{currentstroke}{rgb}{0.000000,0.000000,0.000000}%
\pgfsetstrokecolor{currentstroke}%
\pgfsetdash{}{0pt}%
\pgfpathmoveto{\pgfqpoint{3.483533in}{7.514385in}}%
\pgfpathlineto{\pgfqpoint{3.483533in}{8.057148in}}%
\pgfusepath{stroke}%
\end{pgfscope}%
\begin{pgfscope}%
\pgfpathrectangle{\pgfqpoint{2.125000in}{7.240698in}}{\pgfqpoint{5.489583in}{0.877907in}}%
\pgfusepath{clip}%
\pgfsetbuttcap%
\pgfsetroundjoin%
\pgfsetlinewidth{1.505625pt}%
\definecolor{currentstroke}{rgb}{0.000000,0.000000,0.000000}%
\pgfsetstrokecolor{currentstroke}%
\pgfsetdash{}{0pt}%
\pgfpathmoveto{\pgfqpoint{3.606756in}{7.514385in}}%
\pgfpathlineto{\pgfqpoint{3.606756in}{8.054778in}}%
\pgfusepath{stroke}%
\end{pgfscope}%
\begin{pgfscope}%
\pgfpathrectangle{\pgfqpoint{2.125000in}{7.240698in}}{\pgfqpoint{5.489583in}{0.877907in}}%
\pgfusepath{clip}%
\pgfsetbuttcap%
\pgfsetroundjoin%
\pgfsetlinewidth{1.505625pt}%
\definecolor{currentstroke}{rgb}{0.000000,0.000000,0.000000}%
\pgfsetstrokecolor{currentstroke}%
\pgfsetdash{}{0pt}%
\pgfpathmoveto{\pgfqpoint{3.729979in}{7.514385in}}%
\pgfpathlineto{\pgfqpoint{3.729979in}{8.052557in}}%
\pgfusepath{stroke}%
\end{pgfscope}%
\begin{pgfscope}%
\pgfpathrectangle{\pgfqpoint{2.125000in}{7.240698in}}{\pgfqpoint{5.489583in}{0.877907in}}%
\pgfusepath{clip}%
\pgfsetbuttcap%
\pgfsetroundjoin%
\pgfsetlinewidth{1.505625pt}%
\definecolor{currentstroke}{rgb}{0.000000,0.000000,0.000000}%
\pgfsetstrokecolor{currentstroke}%
\pgfsetdash{}{0pt}%
\pgfpathmoveto{\pgfqpoint{3.853202in}{7.514385in}}%
\pgfpathlineto{\pgfqpoint{3.853202in}{8.050251in}}%
\pgfusepath{stroke}%
\end{pgfscope}%
\begin{pgfscope}%
\pgfpathrectangle{\pgfqpoint{2.125000in}{7.240698in}}{\pgfqpoint{5.489583in}{0.877907in}}%
\pgfusepath{clip}%
\pgfsetbuttcap%
\pgfsetroundjoin%
\pgfsetlinewidth{1.505625pt}%
\definecolor{currentstroke}{rgb}{0.000000,0.000000,0.000000}%
\pgfsetstrokecolor{currentstroke}%
\pgfsetdash{}{0pt}%
\pgfpathmoveto{\pgfqpoint{3.976425in}{7.514385in}}%
\pgfpathlineto{\pgfqpoint{3.976425in}{8.047897in}}%
\pgfusepath{stroke}%
\end{pgfscope}%
\begin{pgfscope}%
\pgfpathrectangle{\pgfqpoint{2.125000in}{7.240698in}}{\pgfqpoint{5.489583in}{0.877907in}}%
\pgfusepath{clip}%
\pgfsetbuttcap%
\pgfsetroundjoin%
\pgfsetlinewidth{1.505625pt}%
\definecolor{currentstroke}{rgb}{0.000000,0.000000,0.000000}%
\pgfsetstrokecolor{currentstroke}%
\pgfsetdash{}{0pt}%
\pgfpathmoveto{\pgfqpoint{4.099648in}{7.514385in}}%
\pgfpathlineto{\pgfqpoint{4.099648in}{8.045634in}}%
\pgfusepath{stroke}%
\end{pgfscope}%
\begin{pgfscope}%
\pgfpathrectangle{\pgfqpoint{2.125000in}{7.240698in}}{\pgfqpoint{5.489583in}{0.877907in}}%
\pgfusepath{clip}%
\pgfsetbuttcap%
\pgfsetroundjoin%
\pgfsetlinewidth{1.505625pt}%
\definecolor{currentstroke}{rgb}{0.000000,0.000000,0.000000}%
\pgfsetstrokecolor{currentstroke}%
\pgfsetdash{}{0pt}%
\pgfpathmoveto{\pgfqpoint{4.222871in}{7.514385in}}%
\pgfpathlineto{\pgfqpoint{4.222871in}{8.043489in}}%
\pgfusepath{stroke}%
\end{pgfscope}%
\begin{pgfscope}%
\pgfpathrectangle{\pgfqpoint{2.125000in}{7.240698in}}{\pgfqpoint{5.489583in}{0.877907in}}%
\pgfusepath{clip}%
\pgfsetbuttcap%
\pgfsetroundjoin%
\pgfsetlinewidth{1.505625pt}%
\definecolor{currentstroke}{rgb}{0.000000,0.000000,0.000000}%
\pgfsetstrokecolor{currentstroke}%
\pgfsetdash{}{0pt}%
\pgfpathmoveto{\pgfqpoint{4.346094in}{7.514385in}}%
\pgfpathlineto{\pgfqpoint{4.346094in}{8.041350in}}%
\pgfusepath{stroke}%
\end{pgfscope}%
\begin{pgfscope}%
\pgfpathrectangle{\pgfqpoint{2.125000in}{7.240698in}}{\pgfqpoint{5.489583in}{0.877907in}}%
\pgfusepath{clip}%
\pgfsetbuttcap%
\pgfsetroundjoin%
\pgfsetlinewidth{1.505625pt}%
\definecolor{currentstroke}{rgb}{0.000000,0.000000,0.000000}%
\pgfsetstrokecolor{currentstroke}%
\pgfsetdash{}{0pt}%
\pgfpathmoveto{\pgfqpoint{4.469317in}{7.514385in}}%
\pgfpathlineto{\pgfqpoint{4.469317in}{8.039282in}}%
\pgfusepath{stroke}%
\end{pgfscope}%
\begin{pgfscope}%
\pgfpathrectangle{\pgfqpoint{2.125000in}{7.240698in}}{\pgfqpoint{5.489583in}{0.877907in}}%
\pgfusepath{clip}%
\pgfsetbuttcap%
\pgfsetroundjoin%
\pgfsetlinewidth{1.505625pt}%
\definecolor{currentstroke}{rgb}{0.000000,0.000000,0.000000}%
\pgfsetstrokecolor{currentstroke}%
\pgfsetdash{}{0pt}%
\pgfpathmoveto{\pgfqpoint{4.592540in}{7.514385in}}%
\pgfpathlineto{\pgfqpoint{4.592540in}{8.037195in}}%
\pgfusepath{stroke}%
\end{pgfscope}%
\begin{pgfscope}%
\pgfpathrectangle{\pgfqpoint{2.125000in}{7.240698in}}{\pgfqpoint{5.489583in}{0.877907in}}%
\pgfusepath{clip}%
\pgfsetbuttcap%
\pgfsetroundjoin%
\pgfsetlinewidth{1.505625pt}%
\definecolor{currentstroke}{rgb}{0.000000,0.000000,0.000000}%
\pgfsetstrokecolor{currentstroke}%
\pgfsetdash{}{0pt}%
\pgfpathmoveto{\pgfqpoint{4.715763in}{7.514385in}}%
\pgfpathlineto{\pgfqpoint{4.715763in}{8.035070in}}%
\pgfusepath{stroke}%
\end{pgfscope}%
\begin{pgfscope}%
\pgfpathrectangle{\pgfqpoint{2.125000in}{7.240698in}}{\pgfqpoint{5.489583in}{0.877907in}}%
\pgfusepath{clip}%
\pgfsetbuttcap%
\pgfsetroundjoin%
\pgfsetlinewidth{1.505625pt}%
\definecolor{currentstroke}{rgb}{0.000000,0.000000,0.000000}%
\pgfsetstrokecolor{currentstroke}%
\pgfsetdash{}{0pt}%
\pgfpathmoveto{\pgfqpoint{4.838986in}{7.514385in}}%
\pgfpathlineto{\pgfqpoint{4.838986in}{8.033007in}}%
\pgfusepath{stroke}%
\end{pgfscope}%
\begin{pgfscope}%
\pgfpathrectangle{\pgfqpoint{2.125000in}{7.240698in}}{\pgfqpoint{5.489583in}{0.877907in}}%
\pgfusepath{clip}%
\pgfsetbuttcap%
\pgfsetroundjoin%
\pgfsetlinewidth{1.505625pt}%
\definecolor{currentstroke}{rgb}{0.000000,0.000000,0.000000}%
\pgfsetstrokecolor{currentstroke}%
\pgfsetdash{}{0pt}%
\pgfpathmoveto{\pgfqpoint{4.962209in}{7.514385in}}%
\pgfpathlineto{\pgfqpoint{4.962209in}{8.030903in}}%
\pgfusepath{stroke}%
\end{pgfscope}%
\begin{pgfscope}%
\pgfpathrectangle{\pgfqpoint{2.125000in}{7.240698in}}{\pgfqpoint{5.489583in}{0.877907in}}%
\pgfusepath{clip}%
\pgfsetbuttcap%
\pgfsetroundjoin%
\pgfsetlinewidth{1.505625pt}%
\definecolor{currentstroke}{rgb}{0.000000,0.000000,0.000000}%
\pgfsetstrokecolor{currentstroke}%
\pgfsetdash{}{0pt}%
\pgfpathmoveto{\pgfqpoint{5.085432in}{7.514385in}}%
\pgfpathlineto{\pgfqpoint{5.085432in}{8.028828in}}%
\pgfusepath{stroke}%
\end{pgfscope}%
\begin{pgfscope}%
\pgfpathrectangle{\pgfqpoint{2.125000in}{7.240698in}}{\pgfqpoint{5.489583in}{0.877907in}}%
\pgfusepath{clip}%
\pgfsetbuttcap%
\pgfsetroundjoin%
\pgfsetlinewidth{1.505625pt}%
\definecolor{currentstroke}{rgb}{0.000000,0.000000,0.000000}%
\pgfsetstrokecolor{currentstroke}%
\pgfsetdash{}{0pt}%
\pgfpathmoveto{\pgfqpoint{5.208655in}{7.514385in}}%
\pgfpathlineto{\pgfqpoint{5.208655in}{8.026882in}}%
\pgfusepath{stroke}%
\end{pgfscope}%
\begin{pgfscope}%
\pgfpathrectangle{\pgfqpoint{2.125000in}{7.240698in}}{\pgfqpoint{5.489583in}{0.877907in}}%
\pgfusepath{clip}%
\pgfsetbuttcap%
\pgfsetroundjoin%
\pgfsetlinewidth{1.505625pt}%
\definecolor{currentstroke}{rgb}{0.000000,0.000000,0.000000}%
\pgfsetstrokecolor{currentstroke}%
\pgfsetdash{}{0pt}%
\pgfpathmoveto{\pgfqpoint{5.331878in}{7.514385in}}%
\pgfpathlineto{\pgfqpoint{5.331878in}{8.024897in}}%
\pgfusepath{stroke}%
\end{pgfscope}%
\begin{pgfscope}%
\pgfpathrectangle{\pgfqpoint{2.125000in}{7.240698in}}{\pgfqpoint{5.489583in}{0.877907in}}%
\pgfusepath{clip}%
\pgfsetbuttcap%
\pgfsetroundjoin%
\pgfsetlinewidth{1.505625pt}%
\definecolor{currentstroke}{rgb}{0.000000,0.000000,0.000000}%
\pgfsetstrokecolor{currentstroke}%
\pgfsetdash{}{0pt}%
\pgfpathmoveto{\pgfqpoint{5.455101in}{7.514385in}}%
\pgfpathlineto{\pgfqpoint{5.455101in}{8.022899in}}%
\pgfusepath{stroke}%
\end{pgfscope}%
\begin{pgfscope}%
\pgfpathrectangle{\pgfqpoint{2.125000in}{7.240698in}}{\pgfqpoint{5.489583in}{0.877907in}}%
\pgfusepath{clip}%
\pgfsetbuttcap%
\pgfsetroundjoin%
\pgfsetlinewidth{1.505625pt}%
\definecolor{currentstroke}{rgb}{0.000000,0.000000,0.000000}%
\pgfsetstrokecolor{currentstroke}%
\pgfsetdash{}{0pt}%
\pgfpathmoveto{\pgfqpoint{5.578324in}{7.514385in}}%
\pgfpathlineto{\pgfqpoint{5.578324in}{8.021028in}}%
\pgfusepath{stroke}%
\end{pgfscope}%
\begin{pgfscope}%
\pgfpathrectangle{\pgfqpoint{2.125000in}{7.240698in}}{\pgfqpoint{5.489583in}{0.877907in}}%
\pgfusepath{clip}%
\pgfsetbuttcap%
\pgfsetroundjoin%
\pgfsetlinewidth{1.505625pt}%
\definecolor{currentstroke}{rgb}{0.000000,0.000000,0.000000}%
\pgfsetstrokecolor{currentstroke}%
\pgfsetdash{}{0pt}%
\pgfpathmoveto{\pgfqpoint{5.701547in}{7.514385in}}%
\pgfpathlineto{\pgfqpoint{5.701547in}{8.019284in}}%
\pgfusepath{stroke}%
\end{pgfscope}%
\begin{pgfscope}%
\pgfpathrectangle{\pgfqpoint{2.125000in}{7.240698in}}{\pgfqpoint{5.489583in}{0.877907in}}%
\pgfusepath{clip}%
\pgfsetbuttcap%
\pgfsetroundjoin%
\pgfsetlinewidth{1.505625pt}%
\definecolor{currentstroke}{rgb}{0.000000,0.000000,0.000000}%
\pgfsetstrokecolor{currentstroke}%
\pgfsetdash{}{0pt}%
\pgfpathmoveto{\pgfqpoint{5.824770in}{7.514385in}}%
\pgfpathlineto{\pgfqpoint{5.824770in}{8.017557in}}%
\pgfusepath{stroke}%
\end{pgfscope}%
\begin{pgfscope}%
\pgfpathrectangle{\pgfqpoint{2.125000in}{7.240698in}}{\pgfqpoint{5.489583in}{0.877907in}}%
\pgfusepath{clip}%
\pgfsetbuttcap%
\pgfsetroundjoin%
\pgfsetlinewidth{1.505625pt}%
\definecolor{currentstroke}{rgb}{0.000000,0.000000,0.000000}%
\pgfsetstrokecolor{currentstroke}%
\pgfsetdash{}{0pt}%
\pgfpathmoveto{\pgfqpoint{5.947993in}{7.514385in}}%
\pgfpathlineto{\pgfqpoint{5.947993in}{8.015860in}}%
\pgfusepath{stroke}%
\end{pgfscope}%
\begin{pgfscope}%
\pgfpathrectangle{\pgfqpoint{2.125000in}{7.240698in}}{\pgfqpoint{5.489583in}{0.877907in}}%
\pgfusepath{clip}%
\pgfsetbuttcap%
\pgfsetroundjoin%
\pgfsetlinewidth{1.505625pt}%
\definecolor{currentstroke}{rgb}{0.000000,0.000000,0.000000}%
\pgfsetstrokecolor{currentstroke}%
\pgfsetdash{}{0pt}%
\pgfpathmoveto{\pgfqpoint{6.071216in}{7.514385in}}%
\pgfpathlineto{\pgfqpoint{6.071216in}{8.014035in}}%
\pgfusepath{stroke}%
\end{pgfscope}%
\begin{pgfscope}%
\pgfpathrectangle{\pgfqpoint{2.125000in}{7.240698in}}{\pgfqpoint{5.489583in}{0.877907in}}%
\pgfusepath{clip}%
\pgfsetbuttcap%
\pgfsetroundjoin%
\pgfsetlinewidth{1.505625pt}%
\definecolor{currentstroke}{rgb}{0.000000,0.000000,0.000000}%
\pgfsetstrokecolor{currentstroke}%
\pgfsetdash{}{0pt}%
\pgfpathmoveto{\pgfqpoint{6.194439in}{7.514385in}}%
\pgfpathlineto{\pgfqpoint{6.194439in}{8.012172in}}%
\pgfusepath{stroke}%
\end{pgfscope}%
\begin{pgfscope}%
\pgfpathrectangle{\pgfqpoint{2.125000in}{7.240698in}}{\pgfqpoint{5.489583in}{0.877907in}}%
\pgfusepath{clip}%
\pgfsetbuttcap%
\pgfsetroundjoin%
\pgfsetlinewidth{1.505625pt}%
\definecolor{currentstroke}{rgb}{0.000000,0.000000,0.000000}%
\pgfsetstrokecolor{currentstroke}%
\pgfsetdash{}{0pt}%
\pgfpathmoveto{\pgfqpoint{6.317662in}{7.514385in}}%
\pgfpathlineto{\pgfqpoint{6.317662in}{8.010449in}}%
\pgfusepath{stroke}%
\end{pgfscope}%
\begin{pgfscope}%
\pgfpathrectangle{\pgfqpoint{2.125000in}{7.240698in}}{\pgfqpoint{5.489583in}{0.877907in}}%
\pgfusepath{clip}%
\pgfsetbuttcap%
\pgfsetroundjoin%
\pgfsetlinewidth{1.505625pt}%
\definecolor{currentstroke}{rgb}{0.000000,0.000000,0.000000}%
\pgfsetstrokecolor{currentstroke}%
\pgfsetdash{}{0pt}%
\pgfpathmoveto{\pgfqpoint{6.440885in}{7.514385in}}%
\pgfpathlineto{\pgfqpoint{6.440885in}{8.008842in}}%
\pgfusepath{stroke}%
\end{pgfscope}%
\begin{pgfscope}%
\pgfpathrectangle{\pgfqpoint{2.125000in}{7.240698in}}{\pgfqpoint{5.489583in}{0.877907in}}%
\pgfusepath{clip}%
\pgfsetbuttcap%
\pgfsetroundjoin%
\pgfsetlinewidth{1.505625pt}%
\definecolor{currentstroke}{rgb}{0.000000,0.000000,0.000000}%
\pgfsetstrokecolor{currentstroke}%
\pgfsetdash{}{0pt}%
\pgfpathmoveto{\pgfqpoint{6.564108in}{7.514385in}}%
\pgfpathlineto{\pgfqpoint{6.564108in}{8.007270in}}%
\pgfusepath{stroke}%
\end{pgfscope}%
\begin{pgfscope}%
\pgfpathrectangle{\pgfqpoint{2.125000in}{7.240698in}}{\pgfqpoint{5.489583in}{0.877907in}}%
\pgfusepath{clip}%
\pgfsetbuttcap%
\pgfsetroundjoin%
\pgfsetlinewidth{1.505625pt}%
\definecolor{currentstroke}{rgb}{0.000000,0.000000,0.000000}%
\pgfsetstrokecolor{currentstroke}%
\pgfsetdash{}{0pt}%
\pgfpathmoveto{\pgfqpoint{6.687330in}{7.514385in}}%
\pgfpathlineto{\pgfqpoint{6.687330in}{8.005645in}}%
\pgfusepath{stroke}%
\end{pgfscope}%
\begin{pgfscope}%
\pgfpathrectangle{\pgfqpoint{2.125000in}{7.240698in}}{\pgfqpoint{5.489583in}{0.877907in}}%
\pgfusepath{clip}%
\pgfsetbuttcap%
\pgfsetroundjoin%
\pgfsetlinewidth{1.505625pt}%
\definecolor{currentstroke}{rgb}{0.000000,0.000000,0.000000}%
\pgfsetstrokecolor{currentstroke}%
\pgfsetdash{}{0pt}%
\pgfpathmoveto{\pgfqpoint{6.810553in}{7.514385in}}%
\pgfpathlineto{\pgfqpoint{6.810553in}{8.003833in}}%
\pgfusepath{stroke}%
\end{pgfscope}%
\begin{pgfscope}%
\pgfpathrectangle{\pgfqpoint{2.125000in}{7.240698in}}{\pgfqpoint{5.489583in}{0.877907in}}%
\pgfusepath{clip}%
\pgfsetbuttcap%
\pgfsetroundjoin%
\pgfsetlinewidth{1.505625pt}%
\definecolor{currentstroke}{rgb}{0.000000,0.000000,0.000000}%
\pgfsetstrokecolor{currentstroke}%
\pgfsetdash{}{0pt}%
\pgfpathmoveto{\pgfqpoint{6.933776in}{7.514385in}}%
\pgfpathlineto{\pgfqpoint{6.933776in}{8.001976in}}%
\pgfusepath{stroke}%
\end{pgfscope}%
\begin{pgfscope}%
\pgfpathrectangle{\pgfqpoint{2.125000in}{7.240698in}}{\pgfqpoint{5.489583in}{0.877907in}}%
\pgfusepath{clip}%
\pgfsetbuttcap%
\pgfsetroundjoin%
\pgfsetlinewidth{1.505625pt}%
\definecolor{currentstroke}{rgb}{0.000000,0.000000,0.000000}%
\pgfsetstrokecolor{currentstroke}%
\pgfsetdash{}{0pt}%
\pgfpathmoveto{\pgfqpoint{7.056999in}{7.514385in}}%
\pgfpathlineto{\pgfqpoint{7.056999in}{8.000079in}}%
\pgfusepath{stroke}%
\end{pgfscope}%
\begin{pgfscope}%
\pgfpathrectangle{\pgfqpoint{2.125000in}{7.240698in}}{\pgfqpoint{5.489583in}{0.877907in}}%
\pgfusepath{clip}%
\pgfsetbuttcap%
\pgfsetroundjoin%
\pgfsetlinewidth{1.505625pt}%
\definecolor{currentstroke}{rgb}{0.000000,0.000000,0.000000}%
\pgfsetstrokecolor{currentstroke}%
\pgfsetdash{}{0pt}%
\pgfpathmoveto{\pgfqpoint{7.180222in}{7.514385in}}%
\pgfpathlineto{\pgfqpoint{7.180222in}{7.998204in}}%
\pgfusepath{stroke}%
\end{pgfscope}%
\begin{pgfscope}%
\pgfpathrectangle{\pgfqpoint{2.125000in}{7.240698in}}{\pgfqpoint{5.489583in}{0.877907in}}%
\pgfusepath{clip}%
\pgfsetbuttcap%
\pgfsetroundjoin%
\pgfsetlinewidth{1.505625pt}%
\definecolor{currentstroke}{rgb}{0.000000,0.000000,0.000000}%
\pgfsetstrokecolor{currentstroke}%
\pgfsetdash{}{0pt}%
\pgfpathmoveto{\pgfqpoint{7.303445in}{7.514385in}}%
\pgfpathlineto{\pgfqpoint{7.303445in}{7.996364in}}%
\pgfusepath{stroke}%
\end{pgfscope}%
\begin{pgfscope}%
\pgfpathrectangle{\pgfqpoint{2.125000in}{7.240698in}}{\pgfqpoint{5.489583in}{0.877907in}}%
\pgfusepath{clip}%
\pgfsetroundcap%
\pgfsetroundjoin%
\pgfsetlinewidth{1.505625pt}%
\definecolor{currentstroke}{rgb}{0.121569,0.466667,0.705882}%
\pgfsetstrokecolor{currentstroke}%
\pgfsetdash{}{0pt}%
\pgfpathmoveto{\pgfqpoint{2.125000in}{7.514385in}}%
\pgfpathlineto{\pgfqpoint{7.614583in}{7.514385in}}%
\pgfusepath{stroke}%
\end{pgfscope}%
\begin{pgfscope}%
\pgfpathrectangle{\pgfqpoint{2.125000in}{7.240698in}}{\pgfqpoint{5.489583in}{0.877907in}}%
\pgfusepath{clip}%
\pgfsetbuttcap%
\pgfsetroundjoin%
\definecolor{currentfill}{rgb}{0.121569,0.466667,0.705882}%
\pgfsetfillcolor{currentfill}%
\pgfsetlinewidth{1.003750pt}%
\definecolor{currentstroke}{rgb}{0.121569,0.466667,0.705882}%
\pgfsetstrokecolor{currentstroke}%
\pgfsetdash{}{0pt}%
\pgfsys@defobject{currentmarker}{\pgfqpoint{-0.034722in}{-0.034722in}}{\pgfqpoint{0.034722in}{0.034722in}}{%
\pgfpathmoveto{\pgfqpoint{0.000000in}{-0.034722in}}%
\pgfpathcurveto{\pgfqpoint{0.009208in}{-0.034722in}}{\pgfqpoint{0.018041in}{-0.031064in}}{\pgfqpoint{0.024552in}{-0.024552in}}%
\pgfpathcurveto{\pgfqpoint{0.031064in}{-0.018041in}}{\pgfqpoint{0.034722in}{-0.009208in}}{\pgfqpoint{0.034722in}{0.000000in}}%
\pgfpathcurveto{\pgfqpoint{0.034722in}{0.009208in}}{\pgfqpoint{0.031064in}{0.018041in}}{\pgfqpoint{0.024552in}{0.024552in}}%
\pgfpathcurveto{\pgfqpoint{0.018041in}{0.031064in}}{\pgfqpoint{0.009208in}{0.034722in}}{\pgfqpoint{0.000000in}{0.034722in}}%
\pgfpathcurveto{\pgfqpoint{-0.009208in}{0.034722in}}{\pgfqpoint{-0.018041in}{0.031064in}}{\pgfqpoint{-0.024552in}{0.024552in}}%
\pgfpathcurveto{\pgfqpoint{-0.031064in}{0.018041in}}{\pgfqpoint{-0.034722in}{0.009208in}}{\pgfqpoint{-0.034722in}{0.000000in}}%
\pgfpathcurveto{\pgfqpoint{-0.034722in}{-0.009208in}}{\pgfqpoint{-0.031064in}{-0.018041in}}{\pgfqpoint{-0.024552in}{-0.024552in}}%
\pgfpathcurveto{\pgfqpoint{-0.018041in}{-0.031064in}}{\pgfqpoint{-0.009208in}{-0.034722in}}{\pgfqpoint{0.000000in}{-0.034722in}}%
\pgfpathclose%
\pgfusepath{stroke,fill}%
}%
\begin{pgfscope}%
\pgfsys@transformshift{2.374527in}{8.078700in}%
\pgfsys@useobject{currentmarker}{}%
\end{pgfscope}%
\begin{pgfscope}%
\pgfsys@transformshift{2.497749in}{8.076247in}%
\pgfsys@useobject{currentmarker}{}%
\end{pgfscope}%
\begin{pgfscope}%
\pgfsys@transformshift{2.620972in}{8.073754in}%
\pgfsys@useobject{currentmarker}{}%
\end{pgfscope}%
\begin{pgfscope}%
\pgfsys@transformshift{2.744195in}{8.071266in}%
\pgfsys@useobject{currentmarker}{}%
\end{pgfscope}%
\begin{pgfscope}%
\pgfsys@transformshift{2.867418in}{8.068713in}%
\pgfsys@useobject{currentmarker}{}%
\end{pgfscope}%
\begin{pgfscope}%
\pgfsys@transformshift{2.990641in}{8.066264in}%
\pgfsys@useobject{currentmarker}{}%
\end{pgfscope}%
\begin{pgfscope}%
\pgfsys@transformshift{3.113864in}{8.063966in}%
\pgfsys@useobject{currentmarker}{}%
\end{pgfscope}%
\begin{pgfscope}%
\pgfsys@transformshift{3.237087in}{8.061742in}%
\pgfsys@useobject{currentmarker}{}%
\end{pgfscope}%
\begin{pgfscope}%
\pgfsys@transformshift{3.360310in}{8.059523in}%
\pgfsys@useobject{currentmarker}{}%
\end{pgfscope}%
\begin{pgfscope}%
\pgfsys@transformshift{3.483533in}{8.057148in}%
\pgfsys@useobject{currentmarker}{}%
\end{pgfscope}%
\begin{pgfscope}%
\pgfsys@transformshift{3.606756in}{8.054778in}%
\pgfsys@useobject{currentmarker}{}%
\end{pgfscope}%
\begin{pgfscope}%
\pgfsys@transformshift{3.729979in}{8.052557in}%
\pgfsys@useobject{currentmarker}{}%
\end{pgfscope}%
\begin{pgfscope}%
\pgfsys@transformshift{3.853202in}{8.050251in}%
\pgfsys@useobject{currentmarker}{}%
\end{pgfscope}%
\begin{pgfscope}%
\pgfsys@transformshift{3.976425in}{8.047897in}%
\pgfsys@useobject{currentmarker}{}%
\end{pgfscope}%
\begin{pgfscope}%
\pgfsys@transformshift{4.099648in}{8.045634in}%
\pgfsys@useobject{currentmarker}{}%
\end{pgfscope}%
\begin{pgfscope}%
\pgfsys@transformshift{4.222871in}{8.043489in}%
\pgfsys@useobject{currentmarker}{}%
\end{pgfscope}%
\begin{pgfscope}%
\pgfsys@transformshift{4.346094in}{8.041350in}%
\pgfsys@useobject{currentmarker}{}%
\end{pgfscope}%
\begin{pgfscope}%
\pgfsys@transformshift{4.469317in}{8.039282in}%
\pgfsys@useobject{currentmarker}{}%
\end{pgfscope}%
\begin{pgfscope}%
\pgfsys@transformshift{4.592540in}{8.037195in}%
\pgfsys@useobject{currentmarker}{}%
\end{pgfscope}%
\begin{pgfscope}%
\pgfsys@transformshift{4.715763in}{8.035070in}%
\pgfsys@useobject{currentmarker}{}%
\end{pgfscope}%
\begin{pgfscope}%
\pgfsys@transformshift{4.838986in}{8.033007in}%
\pgfsys@useobject{currentmarker}{}%
\end{pgfscope}%
\begin{pgfscope}%
\pgfsys@transformshift{4.962209in}{8.030903in}%
\pgfsys@useobject{currentmarker}{}%
\end{pgfscope}%
\begin{pgfscope}%
\pgfsys@transformshift{5.085432in}{8.028828in}%
\pgfsys@useobject{currentmarker}{}%
\end{pgfscope}%
\begin{pgfscope}%
\pgfsys@transformshift{5.208655in}{8.026882in}%
\pgfsys@useobject{currentmarker}{}%
\end{pgfscope}%
\begin{pgfscope}%
\pgfsys@transformshift{5.331878in}{8.024897in}%
\pgfsys@useobject{currentmarker}{}%
\end{pgfscope}%
\begin{pgfscope}%
\pgfsys@transformshift{5.455101in}{8.022899in}%
\pgfsys@useobject{currentmarker}{}%
\end{pgfscope}%
\begin{pgfscope}%
\pgfsys@transformshift{5.578324in}{8.021028in}%
\pgfsys@useobject{currentmarker}{}%
\end{pgfscope}%
\begin{pgfscope}%
\pgfsys@transformshift{5.701547in}{8.019284in}%
\pgfsys@useobject{currentmarker}{}%
\end{pgfscope}%
\begin{pgfscope}%
\pgfsys@transformshift{5.824770in}{8.017557in}%
\pgfsys@useobject{currentmarker}{}%
\end{pgfscope}%
\begin{pgfscope}%
\pgfsys@transformshift{5.947993in}{8.015860in}%
\pgfsys@useobject{currentmarker}{}%
\end{pgfscope}%
\begin{pgfscope}%
\pgfsys@transformshift{6.071216in}{8.014035in}%
\pgfsys@useobject{currentmarker}{}%
\end{pgfscope}%
\begin{pgfscope}%
\pgfsys@transformshift{6.194439in}{8.012172in}%
\pgfsys@useobject{currentmarker}{}%
\end{pgfscope}%
\begin{pgfscope}%
\pgfsys@transformshift{6.317662in}{8.010449in}%
\pgfsys@useobject{currentmarker}{}%
\end{pgfscope}%
\begin{pgfscope}%
\pgfsys@transformshift{6.440885in}{8.008842in}%
\pgfsys@useobject{currentmarker}{}%
\end{pgfscope}%
\begin{pgfscope}%
\pgfsys@transformshift{6.564108in}{8.007270in}%
\pgfsys@useobject{currentmarker}{}%
\end{pgfscope}%
\begin{pgfscope}%
\pgfsys@transformshift{6.687330in}{8.005645in}%
\pgfsys@useobject{currentmarker}{}%
\end{pgfscope}%
\begin{pgfscope}%
\pgfsys@transformshift{6.810553in}{8.003833in}%
\pgfsys@useobject{currentmarker}{}%
\end{pgfscope}%
\begin{pgfscope}%
\pgfsys@transformshift{6.933776in}{8.001976in}%
\pgfsys@useobject{currentmarker}{}%
\end{pgfscope}%
\begin{pgfscope}%
\pgfsys@transformshift{7.056999in}{8.000079in}%
\pgfsys@useobject{currentmarker}{}%
\end{pgfscope}%
\begin{pgfscope}%
\pgfsys@transformshift{7.180222in}{7.998204in}%
\pgfsys@useobject{currentmarker}{}%
\end{pgfscope}%
\begin{pgfscope}%
\pgfsys@transformshift{7.303445in}{7.996364in}%
\pgfsys@useobject{currentmarker}{}%
\end{pgfscope}%
\end{pgfscope}%
\begin{pgfscope}%
\pgfsetrectcap%
\pgfsetmiterjoin%
\pgfsetlinewidth{0.803000pt}%
\definecolor{currentstroke}{rgb}{1.000000,1.000000,1.000000}%
\pgfsetstrokecolor{currentstroke}%
\pgfsetdash{}{0pt}%
\pgfpathmoveto{\pgfqpoint{2.125000in}{7.240698in}}%
\pgfpathlineto{\pgfqpoint{2.125000in}{8.118605in}}%
\pgfusepath{stroke}%
\end{pgfscope}%
\begin{pgfscope}%
\pgfsetrectcap%
\pgfsetmiterjoin%
\pgfsetlinewidth{0.803000pt}%
\definecolor{currentstroke}{rgb}{1.000000,1.000000,1.000000}%
\pgfsetstrokecolor{currentstroke}%
\pgfsetdash{}{0pt}%
\pgfpathmoveto{\pgfqpoint{7.614583in}{7.240698in}}%
\pgfpathlineto{\pgfqpoint{7.614583in}{8.118605in}}%
\pgfusepath{stroke}%
\end{pgfscope}%
\begin{pgfscope}%
\pgfsetrectcap%
\pgfsetmiterjoin%
\pgfsetlinewidth{0.803000pt}%
\definecolor{currentstroke}{rgb}{1.000000,1.000000,1.000000}%
\pgfsetstrokecolor{currentstroke}%
\pgfsetdash{}{0pt}%
\pgfpathmoveto{\pgfqpoint{2.125000in}{7.240698in}}%
\pgfpathlineto{\pgfqpoint{7.614583in}{7.240698in}}%
\pgfusepath{stroke}%
\end{pgfscope}%
\begin{pgfscope}%
\pgfsetrectcap%
\pgfsetmiterjoin%
\pgfsetlinewidth{0.803000pt}%
\definecolor{currentstroke}{rgb}{1.000000,1.000000,1.000000}%
\pgfsetstrokecolor{currentstroke}%
\pgfsetdash{}{0pt}%
\pgfpathmoveto{\pgfqpoint{2.125000in}{8.118605in}}%
\pgfpathlineto{\pgfqpoint{7.614583in}{8.118605in}}%
\pgfusepath{stroke}%
\end{pgfscope}%
\begin{pgfscope}%
\definecolor{textcolor}{rgb}{0.150000,0.150000,0.150000}%
\pgfsetstrokecolor{textcolor}%
\pgfsetfillcolor{textcolor}%
\pgftext[x=4.869792in,y=8.201938in,,base]{\color{textcolor}\rmfamily\fontsize{16.800000}{20.160000}\selectfont Autocorrelation}%
\end{pgfscope}%
\begin{pgfscope}%
\pgfsetbuttcap%
\pgfsetmiterjoin%
\definecolor{currentfill}{rgb}{0.917647,0.917647,0.949020}%
\pgfsetfillcolor{currentfill}%
\pgfsetlinewidth{0.000000pt}%
\definecolor{currentstroke}{rgb}{0.000000,0.000000,0.000000}%
\pgfsetstrokecolor{currentstroke}%
\pgfsetstrokeopacity{0.000000}%
\pgfsetdash{}{0pt}%
\pgfpathmoveto{\pgfqpoint{9.810417in}{7.240698in}}%
\pgfpathlineto{\pgfqpoint{15.300000in}{7.240698in}}%
\pgfpathlineto{\pgfqpoint{15.300000in}{8.118605in}}%
\pgfpathlineto{\pgfqpoint{9.810417in}{8.118605in}}%
\pgfpathclose%
\pgfusepath{fill}%
\end{pgfscope}%
\begin{pgfscope}%
\pgfpathrectangle{\pgfqpoint{9.810417in}{7.240698in}}{\pgfqpoint{5.489583in}{0.877907in}}%
\pgfusepath{clip}%
\pgfsetroundcap%
\pgfsetroundjoin%
\pgfsetlinewidth{0.803000pt}%
\definecolor{currentstroke}{rgb}{1.000000,1.000000,1.000000}%
\pgfsetstrokecolor{currentstroke}%
\pgfsetdash{}{0pt}%
\pgfpathmoveto{\pgfqpoint{10.059943in}{7.240698in}}%
\pgfpathlineto{\pgfqpoint{10.059943in}{8.118605in}}%
\pgfusepath{stroke}%
\end{pgfscope}%
\begin{pgfscope}%
\definecolor{textcolor}{rgb}{0.150000,0.150000,0.150000}%
\pgfsetstrokecolor{textcolor}%
\pgfsetfillcolor{textcolor}%
\pgftext[x=10.059943in,y=7.143475in,,top]{\color{textcolor}\rmfamily\fontsize{14.000000}{16.800000}\selectfont 0}%
\end{pgfscope}%
\begin{pgfscope}%
\pgfpathrectangle{\pgfqpoint{9.810417in}{7.240698in}}{\pgfqpoint{5.489583in}{0.877907in}}%
\pgfusepath{clip}%
\pgfsetroundcap%
\pgfsetroundjoin%
\pgfsetlinewidth{0.803000pt}%
\definecolor{currentstroke}{rgb}{1.000000,1.000000,1.000000}%
\pgfsetstrokecolor{currentstroke}%
\pgfsetdash{}{0pt}%
\pgfpathmoveto{\pgfqpoint{10.676058in}{7.240698in}}%
\pgfpathlineto{\pgfqpoint{10.676058in}{8.118605in}}%
\pgfusepath{stroke}%
\end{pgfscope}%
\begin{pgfscope}%
\definecolor{textcolor}{rgb}{0.150000,0.150000,0.150000}%
\pgfsetstrokecolor{textcolor}%
\pgfsetfillcolor{textcolor}%
\pgftext[x=10.676058in,y=7.143475in,,top]{\color{textcolor}\rmfamily\fontsize{14.000000}{16.800000}\selectfont 5}%
\end{pgfscope}%
\begin{pgfscope}%
\pgfpathrectangle{\pgfqpoint{9.810417in}{7.240698in}}{\pgfqpoint{5.489583in}{0.877907in}}%
\pgfusepath{clip}%
\pgfsetroundcap%
\pgfsetroundjoin%
\pgfsetlinewidth{0.803000pt}%
\definecolor{currentstroke}{rgb}{1.000000,1.000000,1.000000}%
\pgfsetstrokecolor{currentstroke}%
\pgfsetdash{}{0pt}%
\pgfpathmoveto{\pgfqpoint{11.292173in}{7.240698in}}%
\pgfpathlineto{\pgfqpoint{11.292173in}{8.118605in}}%
\pgfusepath{stroke}%
\end{pgfscope}%
\begin{pgfscope}%
\definecolor{textcolor}{rgb}{0.150000,0.150000,0.150000}%
\pgfsetstrokecolor{textcolor}%
\pgfsetfillcolor{textcolor}%
\pgftext[x=11.292173in,y=7.143475in,,top]{\color{textcolor}\rmfamily\fontsize{14.000000}{16.800000}\selectfont 10}%
\end{pgfscope}%
\begin{pgfscope}%
\pgfpathrectangle{\pgfqpoint{9.810417in}{7.240698in}}{\pgfqpoint{5.489583in}{0.877907in}}%
\pgfusepath{clip}%
\pgfsetroundcap%
\pgfsetroundjoin%
\pgfsetlinewidth{0.803000pt}%
\definecolor{currentstroke}{rgb}{1.000000,1.000000,1.000000}%
\pgfsetstrokecolor{currentstroke}%
\pgfsetdash{}{0pt}%
\pgfpathmoveto{\pgfqpoint{11.908288in}{7.240698in}}%
\pgfpathlineto{\pgfqpoint{11.908288in}{8.118605in}}%
\pgfusepath{stroke}%
\end{pgfscope}%
\begin{pgfscope}%
\definecolor{textcolor}{rgb}{0.150000,0.150000,0.150000}%
\pgfsetstrokecolor{textcolor}%
\pgfsetfillcolor{textcolor}%
\pgftext[x=11.908288in,y=7.143475in,,top]{\color{textcolor}\rmfamily\fontsize{14.000000}{16.800000}\selectfont 15}%
\end{pgfscope}%
\begin{pgfscope}%
\pgfpathrectangle{\pgfqpoint{9.810417in}{7.240698in}}{\pgfqpoint{5.489583in}{0.877907in}}%
\pgfusepath{clip}%
\pgfsetroundcap%
\pgfsetroundjoin%
\pgfsetlinewidth{0.803000pt}%
\definecolor{currentstroke}{rgb}{1.000000,1.000000,1.000000}%
\pgfsetstrokecolor{currentstroke}%
\pgfsetdash{}{0pt}%
\pgfpathmoveto{\pgfqpoint{12.524403in}{7.240698in}}%
\pgfpathlineto{\pgfqpoint{12.524403in}{8.118605in}}%
\pgfusepath{stroke}%
\end{pgfscope}%
\begin{pgfscope}%
\definecolor{textcolor}{rgb}{0.150000,0.150000,0.150000}%
\pgfsetstrokecolor{textcolor}%
\pgfsetfillcolor{textcolor}%
\pgftext[x=12.524403in,y=7.143475in,,top]{\color{textcolor}\rmfamily\fontsize{14.000000}{16.800000}\selectfont 20}%
\end{pgfscope}%
\begin{pgfscope}%
\pgfpathrectangle{\pgfqpoint{9.810417in}{7.240698in}}{\pgfqpoint{5.489583in}{0.877907in}}%
\pgfusepath{clip}%
\pgfsetroundcap%
\pgfsetroundjoin%
\pgfsetlinewidth{0.803000pt}%
\definecolor{currentstroke}{rgb}{1.000000,1.000000,1.000000}%
\pgfsetstrokecolor{currentstroke}%
\pgfsetdash{}{0pt}%
\pgfpathmoveto{\pgfqpoint{13.140517in}{7.240698in}}%
\pgfpathlineto{\pgfqpoint{13.140517in}{8.118605in}}%
\pgfusepath{stroke}%
\end{pgfscope}%
\begin{pgfscope}%
\definecolor{textcolor}{rgb}{0.150000,0.150000,0.150000}%
\pgfsetstrokecolor{textcolor}%
\pgfsetfillcolor{textcolor}%
\pgftext[x=13.140517in,y=7.143475in,,top]{\color{textcolor}\rmfamily\fontsize{14.000000}{16.800000}\selectfont 25}%
\end{pgfscope}%
\begin{pgfscope}%
\pgfpathrectangle{\pgfqpoint{9.810417in}{7.240698in}}{\pgfqpoint{5.489583in}{0.877907in}}%
\pgfusepath{clip}%
\pgfsetroundcap%
\pgfsetroundjoin%
\pgfsetlinewidth{0.803000pt}%
\definecolor{currentstroke}{rgb}{1.000000,1.000000,1.000000}%
\pgfsetstrokecolor{currentstroke}%
\pgfsetdash{}{0pt}%
\pgfpathmoveto{\pgfqpoint{13.756632in}{7.240698in}}%
\pgfpathlineto{\pgfqpoint{13.756632in}{8.118605in}}%
\pgfusepath{stroke}%
\end{pgfscope}%
\begin{pgfscope}%
\definecolor{textcolor}{rgb}{0.150000,0.150000,0.150000}%
\pgfsetstrokecolor{textcolor}%
\pgfsetfillcolor{textcolor}%
\pgftext[x=13.756632in,y=7.143475in,,top]{\color{textcolor}\rmfamily\fontsize{14.000000}{16.800000}\selectfont 30}%
\end{pgfscope}%
\begin{pgfscope}%
\pgfpathrectangle{\pgfqpoint{9.810417in}{7.240698in}}{\pgfqpoint{5.489583in}{0.877907in}}%
\pgfusepath{clip}%
\pgfsetroundcap%
\pgfsetroundjoin%
\pgfsetlinewidth{0.803000pt}%
\definecolor{currentstroke}{rgb}{1.000000,1.000000,1.000000}%
\pgfsetstrokecolor{currentstroke}%
\pgfsetdash{}{0pt}%
\pgfpathmoveto{\pgfqpoint{14.372747in}{7.240698in}}%
\pgfpathlineto{\pgfqpoint{14.372747in}{8.118605in}}%
\pgfusepath{stroke}%
\end{pgfscope}%
\begin{pgfscope}%
\definecolor{textcolor}{rgb}{0.150000,0.150000,0.150000}%
\pgfsetstrokecolor{textcolor}%
\pgfsetfillcolor{textcolor}%
\pgftext[x=14.372747in,y=7.143475in,,top]{\color{textcolor}\rmfamily\fontsize{14.000000}{16.800000}\selectfont 35}%
\end{pgfscope}%
\begin{pgfscope}%
\pgfpathrectangle{\pgfqpoint{9.810417in}{7.240698in}}{\pgfqpoint{5.489583in}{0.877907in}}%
\pgfusepath{clip}%
\pgfsetroundcap%
\pgfsetroundjoin%
\pgfsetlinewidth{0.803000pt}%
\definecolor{currentstroke}{rgb}{1.000000,1.000000,1.000000}%
\pgfsetstrokecolor{currentstroke}%
\pgfsetdash{}{0pt}%
\pgfpathmoveto{\pgfqpoint{14.988862in}{7.240698in}}%
\pgfpathlineto{\pgfqpoint{14.988862in}{8.118605in}}%
\pgfusepath{stroke}%
\end{pgfscope}%
\begin{pgfscope}%
\definecolor{textcolor}{rgb}{0.150000,0.150000,0.150000}%
\pgfsetstrokecolor{textcolor}%
\pgfsetfillcolor{textcolor}%
\pgftext[x=14.988862in,y=7.143475in,,top]{\color{textcolor}\rmfamily\fontsize{14.000000}{16.800000}\selectfont 40}%
\end{pgfscope}%
\begin{pgfscope}%
\pgfpathrectangle{\pgfqpoint{9.810417in}{7.240698in}}{\pgfqpoint{5.489583in}{0.877907in}}%
\pgfusepath{clip}%
\pgfsetroundcap%
\pgfsetroundjoin%
\pgfsetlinewidth{0.803000pt}%
\definecolor{currentstroke}{rgb}{1.000000,1.000000,1.000000}%
\pgfsetstrokecolor{currentstroke}%
\pgfsetdash{}{0pt}%
\pgfpathmoveto{\pgfqpoint{9.810417in}{7.318936in}}%
\pgfpathlineto{\pgfqpoint{15.300000in}{7.318936in}}%
\pgfusepath{stroke}%
\end{pgfscope}%
\begin{pgfscope}%
\definecolor{textcolor}{rgb}{0.150000,0.150000,0.150000}%
\pgfsetstrokecolor{textcolor}%
\pgfsetfillcolor{textcolor}%
\pgftext[x=9.589483in,y=7.245070in,left,base]{\color{textcolor}\rmfamily\fontsize{14.000000}{16.800000}\selectfont 0}%
\end{pgfscope}%
\begin{pgfscope}%
\pgfpathrectangle{\pgfqpoint{9.810417in}{7.240698in}}{\pgfqpoint{5.489583in}{0.877907in}}%
\pgfusepath{clip}%
\pgfsetroundcap%
\pgfsetroundjoin%
\pgfsetlinewidth{0.803000pt}%
\definecolor{currentstroke}{rgb}{1.000000,1.000000,1.000000}%
\pgfsetstrokecolor{currentstroke}%
\pgfsetdash{}{0pt}%
\pgfpathmoveto{\pgfqpoint{9.810417in}{8.078700in}}%
\pgfpathlineto{\pgfqpoint{15.300000in}{8.078700in}}%
\pgfusepath{stroke}%
\end{pgfscope}%
\begin{pgfscope}%
\definecolor{textcolor}{rgb}{0.150000,0.150000,0.150000}%
\pgfsetstrokecolor{textcolor}%
\pgfsetfillcolor{textcolor}%
\pgftext[x=9.589483in,y=8.004834in,left,base]{\color{textcolor}\rmfamily\fontsize{14.000000}{16.800000}\selectfont 1}%
\end{pgfscope}%
\begin{pgfscope}%
\pgfpathrectangle{\pgfqpoint{9.810417in}{7.240698in}}{\pgfqpoint{5.489583in}{0.877907in}}%
\pgfusepath{clip}%
\pgfsetbuttcap%
\pgfsetroundjoin%
\definecolor{currentfill}{rgb}{0.121569,0.466667,0.705882}%
\pgfsetfillcolor{currentfill}%
\pgfsetfillopacity{0.250000}%
\pgfsetlinewidth{1.003750pt}%
\definecolor{currentstroke}{rgb}{1.000000,1.000000,1.000000}%
\pgfsetstrokecolor{currentstroke}%
\pgfsetstrokeopacity{0.250000}%
\pgfsetdash{}{0pt}%
\pgfpathmoveto{\pgfqpoint{10.121555in}{7.357270in}}%
\pgfpathlineto{\pgfqpoint{10.121555in}{7.280603in}}%
\pgfpathlineto{\pgfqpoint{10.306389in}{7.280603in}}%
\pgfpathlineto{\pgfqpoint{10.429612in}{7.280603in}}%
\pgfpathlineto{\pgfqpoint{10.552835in}{7.280603in}}%
\pgfpathlineto{\pgfqpoint{10.676058in}{7.280603in}}%
\pgfpathlineto{\pgfqpoint{10.799281in}{7.280603in}}%
\pgfpathlineto{\pgfqpoint{10.922504in}{7.280603in}}%
\pgfpathlineto{\pgfqpoint{11.045727in}{7.280603in}}%
\pgfpathlineto{\pgfqpoint{11.168950in}{7.280603in}}%
\pgfpathlineto{\pgfqpoint{11.292173in}{7.280603in}}%
\pgfpathlineto{\pgfqpoint{11.415396in}{7.280603in}}%
\pgfpathlineto{\pgfqpoint{11.538619in}{7.280603in}}%
\pgfpathlineto{\pgfqpoint{11.661842in}{7.280603in}}%
\pgfpathlineto{\pgfqpoint{11.785065in}{7.280603in}}%
\pgfpathlineto{\pgfqpoint{11.908288in}{7.280603in}}%
\pgfpathlineto{\pgfqpoint{12.031511in}{7.280603in}}%
\pgfpathlineto{\pgfqpoint{12.154734in}{7.280603in}}%
\pgfpathlineto{\pgfqpoint{12.277957in}{7.280603in}}%
\pgfpathlineto{\pgfqpoint{12.401180in}{7.280603in}}%
\pgfpathlineto{\pgfqpoint{12.524403in}{7.280603in}}%
\pgfpathlineto{\pgfqpoint{12.647626in}{7.280603in}}%
\pgfpathlineto{\pgfqpoint{12.770849in}{7.280603in}}%
\pgfpathlineto{\pgfqpoint{12.894072in}{7.280603in}}%
\pgfpathlineto{\pgfqpoint{13.017294in}{7.280603in}}%
\pgfpathlineto{\pgfqpoint{13.140517in}{7.280603in}}%
\pgfpathlineto{\pgfqpoint{13.263740in}{7.280603in}}%
\pgfpathlineto{\pgfqpoint{13.386963in}{7.280603in}}%
\pgfpathlineto{\pgfqpoint{13.510186in}{7.280603in}}%
\pgfpathlineto{\pgfqpoint{13.633409in}{7.280603in}}%
\pgfpathlineto{\pgfqpoint{13.756632in}{7.280603in}}%
\pgfpathlineto{\pgfqpoint{13.879855in}{7.280603in}}%
\pgfpathlineto{\pgfqpoint{14.003078in}{7.280603in}}%
\pgfpathlineto{\pgfqpoint{14.126301in}{7.280603in}}%
\pgfpathlineto{\pgfqpoint{14.249524in}{7.280603in}}%
\pgfpathlineto{\pgfqpoint{14.372747in}{7.280603in}}%
\pgfpathlineto{\pgfqpoint{14.495970in}{7.280603in}}%
\pgfpathlineto{\pgfqpoint{14.619193in}{7.280603in}}%
\pgfpathlineto{\pgfqpoint{14.742416in}{7.280603in}}%
\pgfpathlineto{\pgfqpoint{14.865639in}{7.280603in}}%
\pgfpathlineto{\pgfqpoint{15.050473in}{7.280603in}}%
\pgfpathlineto{\pgfqpoint{15.050473in}{7.357270in}}%
\pgfpathlineto{\pgfqpoint{15.050473in}{7.357270in}}%
\pgfpathlineto{\pgfqpoint{14.865639in}{7.357270in}}%
\pgfpathlineto{\pgfqpoint{14.742416in}{7.357270in}}%
\pgfpathlineto{\pgfqpoint{14.619193in}{7.357270in}}%
\pgfpathlineto{\pgfqpoint{14.495970in}{7.357270in}}%
\pgfpathlineto{\pgfqpoint{14.372747in}{7.357270in}}%
\pgfpathlineto{\pgfqpoint{14.249524in}{7.357270in}}%
\pgfpathlineto{\pgfqpoint{14.126301in}{7.357270in}}%
\pgfpathlineto{\pgfqpoint{14.003078in}{7.357270in}}%
\pgfpathlineto{\pgfqpoint{13.879855in}{7.357270in}}%
\pgfpathlineto{\pgfqpoint{13.756632in}{7.357270in}}%
\pgfpathlineto{\pgfqpoint{13.633409in}{7.357270in}}%
\pgfpathlineto{\pgfqpoint{13.510186in}{7.357270in}}%
\pgfpathlineto{\pgfqpoint{13.386963in}{7.357270in}}%
\pgfpathlineto{\pgfqpoint{13.263740in}{7.357270in}}%
\pgfpathlineto{\pgfqpoint{13.140517in}{7.357270in}}%
\pgfpathlineto{\pgfqpoint{13.017294in}{7.357270in}}%
\pgfpathlineto{\pgfqpoint{12.894072in}{7.357270in}}%
\pgfpathlineto{\pgfqpoint{12.770849in}{7.357270in}}%
\pgfpathlineto{\pgfqpoint{12.647626in}{7.357270in}}%
\pgfpathlineto{\pgfqpoint{12.524403in}{7.357270in}}%
\pgfpathlineto{\pgfqpoint{12.401180in}{7.357270in}}%
\pgfpathlineto{\pgfqpoint{12.277957in}{7.357270in}}%
\pgfpathlineto{\pgfqpoint{12.154734in}{7.357270in}}%
\pgfpathlineto{\pgfqpoint{12.031511in}{7.357270in}}%
\pgfpathlineto{\pgfqpoint{11.908288in}{7.357270in}}%
\pgfpathlineto{\pgfqpoint{11.785065in}{7.357270in}}%
\pgfpathlineto{\pgfqpoint{11.661842in}{7.357270in}}%
\pgfpathlineto{\pgfqpoint{11.538619in}{7.357270in}}%
\pgfpathlineto{\pgfqpoint{11.415396in}{7.357270in}}%
\pgfpathlineto{\pgfqpoint{11.292173in}{7.357270in}}%
\pgfpathlineto{\pgfqpoint{11.168950in}{7.357270in}}%
\pgfpathlineto{\pgfqpoint{11.045727in}{7.357270in}}%
\pgfpathlineto{\pgfqpoint{10.922504in}{7.357270in}}%
\pgfpathlineto{\pgfqpoint{10.799281in}{7.357270in}}%
\pgfpathlineto{\pgfqpoint{10.676058in}{7.357270in}}%
\pgfpathlineto{\pgfqpoint{10.552835in}{7.357270in}}%
\pgfpathlineto{\pgfqpoint{10.429612in}{7.357270in}}%
\pgfpathlineto{\pgfqpoint{10.306389in}{7.357270in}}%
\pgfpathlineto{\pgfqpoint{10.121555in}{7.357270in}}%
\pgfpathclose%
\pgfusepath{stroke,fill}%
\end{pgfscope}%
\begin{pgfscope}%
\pgfpathrectangle{\pgfqpoint{9.810417in}{7.240698in}}{\pgfqpoint{5.489583in}{0.877907in}}%
\pgfusepath{clip}%
\pgfsetbuttcap%
\pgfsetroundjoin%
\pgfsetlinewidth{1.505625pt}%
\definecolor{currentstroke}{rgb}{0.000000,0.000000,0.000000}%
\pgfsetstrokecolor{currentstroke}%
\pgfsetdash{}{0pt}%
\pgfpathmoveto{\pgfqpoint{10.059943in}{7.318936in}}%
\pgfpathlineto{\pgfqpoint{10.059943in}{8.078700in}}%
\pgfusepath{stroke}%
\end{pgfscope}%
\begin{pgfscope}%
\pgfpathrectangle{\pgfqpoint{9.810417in}{7.240698in}}{\pgfqpoint{5.489583in}{0.877907in}}%
\pgfusepath{clip}%
\pgfsetbuttcap%
\pgfsetroundjoin%
\pgfsetlinewidth{1.505625pt}%
\definecolor{currentstroke}{rgb}{0.000000,0.000000,0.000000}%
\pgfsetstrokecolor{currentstroke}%
\pgfsetdash{}{0pt}%
\pgfpathmoveto{\pgfqpoint{10.183166in}{7.318936in}}%
\pgfpathlineto{\pgfqpoint{10.183166in}{8.075900in}}%
\pgfusepath{stroke}%
\end{pgfscope}%
\begin{pgfscope}%
\pgfpathrectangle{\pgfqpoint{9.810417in}{7.240698in}}{\pgfqpoint{5.489583in}{0.877907in}}%
\pgfusepath{clip}%
\pgfsetbuttcap%
\pgfsetroundjoin%
\pgfsetlinewidth{1.505625pt}%
\definecolor{currentstroke}{rgb}{0.000000,0.000000,0.000000}%
\pgfsetstrokecolor{currentstroke}%
\pgfsetdash{}{0pt}%
\pgfpathmoveto{\pgfqpoint{10.306389in}{7.318936in}}%
\pgfpathlineto{\pgfqpoint{10.306389in}{7.309470in}}%
\pgfusepath{stroke}%
\end{pgfscope}%
\begin{pgfscope}%
\pgfpathrectangle{\pgfqpoint{9.810417in}{7.240698in}}{\pgfqpoint{5.489583in}{0.877907in}}%
\pgfusepath{clip}%
\pgfsetbuttcap%
\pgfsetroundjoin%
\pgfsetlinewidth{1.505625pt}%
\definecolor{currentstroke}{rgb}{0.000000,0.000000,0.000000}%
\pgfsetstrokecolor{currentstroke}%
\pgfsetdash{}{0pt}%
\pgfpathmoveto{\pgfqpoint{10.429612in}{7.318936in}}%
\pgfpathlineto{\pgfqpoint{10.429612in}{7.318066in}}%
\pgfusepath{stroke}%
\end{pgfscope}%
\begin{pgfscope}%
\pgfpathrectangle{\pgfqpoint{9.810417in}{7.240698in}}{\pgfqpoint{5.489583in}{0.877907in}}%
\pgfusepath{clip}%
\pgfsetbuttcap%
\pgfsetroundjoin%
\pgfsetlinewidth{1.505625pt}%
\definecolor{currentstroke}{rgb}{0.000000,0.000000,0.000000}%
\pgfsetstrokecolor{currentstroke}%
\pgfsetdash{}{0pt}%
\pgfpathmoveto{\pgfqpoint{10.552835in}{7.318936in}}%
\pgfpathlineto{\pgfqpoint{10.552835in}{7.305082in}}%
\pgfusepath{stroke}%
\end{pgfscope}%
\begin{pgfscope}%
\pgfpathrectangle{\pgfqpoint{9.810417in}{7.240698in}}{\pgfqpoint{5.489583in}{0.877907in}}%
\pgfusepath{clip}%
\pgfsetbuttcap%
\pgfsetroundjoin%
\pgfsetlinewidth{1.505625pt}%
\definecolor{currentstroke}{rgb}{0.000000,0.000000,0.000000}%
\pgfsetstrokecolor{currentstroke}%
\pgfsetdash{}{0pt}%
\pgfpathmoveto{\pgfqpoint{10.676058in}{7.318936in}}%
\pgfpathlineto{\pgfqpoint{10.676058in}{7.336369in}}%
\pgfusepath{stroke}%
\end{pgfscope}%
\begin{pgfscope}%
\pgfpathrectangle{\pgfqpoint{9.810417in}{7.240698in}}{\pgfqpoint{5.489583in}{0.877907in}}%
\pgfusepath{clip}%
\pgfsetbuttcap%
\pgfsetroundjoin%
\pgfsetlinewidth{1.505625pt}%
\definecolor{currentstroke}{rgb}{0.000000,0.000000,0.000000}%
\pgfsetstrokecolor{currentstroke}%
\pgfsetdash{}{0pt}%
\pgfpathmoveto{\pgfqpoint{10.799281in}{7.318936in}}%
\pgfpathlineto{\pgfqpoint{10.799281in}{7.344185in}}%
\pgfusepath{stroke}%
\end{pgfscope}%
\begin{pgfscope}%
\pgfpathrectangle{\pgfqpoint{9.810417in}{7.240698in}}{\pgfqpoint{5.489583in}{0.877907in}}%
\pgfusepath{clip}%
\pgfsetbuttcap%
\pgfsetroundjoin%
\pgfsetlinewidth{1.505625pt}%
\definecolor{currentstroke}{rgb}{0.000000,0.000000,0.000000}%
\pgfsetstrokecolor{currentstroke}%
\pgfsetdash{}{0pt}%
\pgfpathmoveto{\pgfqpoint{10.922504in}{7.318936in}}%
\pgfpathlineto{\pgfqpoint{10.922504in}{7.330536in}}%
\pgfusepath{stroke}%
\end{pgfscope}%
\begin{pgfscope}%
\pgfpathrectangle{\pgfqpoint{9.810417in}{7.240698in}}{\pgfqpoint{5.489583in}{0.877907in}}%
\pgfusepath{clip}%
\pgfsetbuttcap%
\pgfsetroundjoin%
\pgfsetlinewidth{1.505625pt}%
\definecolor{currentstroke}{rgb}{0.000000,0.000000,0.000000}%
\pgfsetstrokecolor{currentstroke}%
\pgfsetdash{}{0pt}%
\pgfpathmoveto{\pgfqpoint{11.045727in}{7.318936in}}%
\pgfpathlineto{\pgfqpoint{11.045727in}{7.317247in}}%
\pgfusepath{stroke}%
\end{pgfscope}%
\begin{pgfscope}%
\pgfpathrectangle{\pgfqpoint{9.810417in}{7.240698in}}{\pgfqpoint{5.489583in}{0.877907in}}%
\pgfusepath{clip}%
\pgfsetbuttcap%
\pgfsetroundjoin%
\pgfsetlinewidth{1.505625pt}%
\definecolor{currentstroke}{rgb}{0.000000,0.000000,0.000000}%
\pgfsetstrokecolor{currentstroke}%
\pgfsetdash{}{0pt}%
\pgfpathmoveto{\pgfqpoint{11.168950in}{7.318936in}}%
\pgfpathlineto{\pgfqpoint{11.168950in}{7.287925in}}%
\pgfusepath{stroke}%
\end{pgfscope}%
\begin{pgfscope}%
\pgfpathrectangle{\pgfqpoint{9.810417in}{7.240698in}}{\pgfqpoint{5.489583in}{0.877907in}}%
\pgfusepath{clip}%
\pgfsetbuttcap%
\pgfsetroundjoin%
\pgfsetlinewidth{1.505625pt}%
\definecolor{currentstroke}{rgb}{0.000000,0.000000,0.000000}%
\pgfsetstrokecolor{currentstroke}%
\pgfsetdash{}{0pt}%
\pgfpathmoveto{\pgfqpoint{11.292173in}{7.318936in}}%
\pgfpathlineto{\pgfqpoint{11.292173in}{7.319602in}}%
\pgfusepath{stroke}%
\end{pgfscope}%
\begin{pgfscope}%
\pgfpathrectangle{\pgfqpoint{9.810417in}{7.240698in}}{\pgfqpoint{5.489583in}{0.877907in}}%
\pgfusepath{clip}%
\pgfsetbuttcap%
\pgfsetroundjoin%
\pgfsetlinewidth{1.505625pt}%
\definecolor{currentstroke}{rgb}{0.000000,0.000000,0.000000}%
\pgfsetstrokecolor{currentstroke}%
\pgfsetdash{}{0pt}%
\pgfpathmoveto{\pgfqpoint{11.415396in}{7.318936in}}%
\pgfpathlineto{\pgfqpoint{11.415396in}{7.345977in}}%
\pgfusepath{stroke}%
\end{pgfscope}%
\begin{pgfscope}%
\pgfpathrectangle{\pgfqpoint{9.810417in}{7.240698in}}{\pgfqpoint{5.489583in}{0.877907in}}%
\pgfusepath{clip}%
\pgfsetbuttcap%
\pgfsetroundjoin%
\pgfsetlinewidth{1.505625pt}%
\definecolor{currentstroke}{rgb}{0.000000,0.000000,0.000000}%
\pgfsetstrokecolor{currentstroke}%
\pgfsetdash{}{0pt}%
\pgfpathmoveto{\pgfqpoint{11.538619in}{7.318936in}}%
\pgfpathlineto{\pgfqpoint{11.538619in}{7.303012in}}%
\pgfusepath{stroke}%
\end{pgfscope}%
\begin{pgfscope}%
\pgfpathrectangle{\pgfqpoint{9.810417in}{7.240698in}}{\pgfqpoint{5.489583in}{0.877907in}}%
\pgfusepath{clip}%
\pgfsetbuttcap%
\pgfsetroundjoin%
\pgfsetlinewidth{1.505625pt}%
\definecolor{currentstroke}{rgb}{0.000000,0.000000,0.000000}%
\pgfsetstrokecolor{currentstroke}%
\pgfsetdash{}{0pt}%
\pgfpathmoveto{\pgfqpoint{11.661842in}{7.318936in}}%
\pgfpathlineto{\pgfqpoint{11.661842in}{7.307171in}}%
\pgfusepath{stroke}%
\end{pgfscope}%
\begin{pgfscope}%
\pgfpathrectangle{\pgfqpoint{9.810417in}{7.240698in}}{\pgfqpoint{5.489583in}{0.877907in}}%
\pgfusepath{clip}%
\pgfsetbuttcap%
\pgfsetroundjoin%
\pgfsetlinewidth{1.505625pt}%
\definecolor{currentstroke}{rgb}{0.000000,0.000000,0.000000}%
\pgfsetstrokecolor{currentstroke}%
\pgfsetdash{}{0pt}%
\pgfpathmoveto{\pgfqpoint{11.785065in}{7.318936in}}%
\pgfpathlineto{\pgfqpoint{11.785065in}{7.331168in}}%
\pgfusepath{stroke}%
\end{pgfscope}%
\begin{pgfscope}%
\pgfpathrectangle{\pgfqpoint{9.810417in}{7.240698in}}{\pgfqpoint{5.489583in}{0.877907in}}%
\pgfusepath{clip}%
\pgfsetbuttcap%
\pgfsetroundjoin%
\pgfsetlinewidth{1.505625pt}%
\definecolor{currentstroke}{rgb}{0.000000,0.000000,0.000000}%
\pgfsetstrokecolor{currentstroke}%
\pgfsetdash{}{0pt}%
\pgfpathmoveto{\pgfqpoint{11.908288in}{7.318936in}}%
\pgfpathlineto{\pgfqpoint{11.908288in}{7.339705in}}%
\pgfusepath{stroke}%
\end{pgfscope}%
\begin{pgfscope}%
\pgfpathrectangle{\pgfqpoint{9.810417in}{7.240698in}}{\pgfqpoint{5.489583in}{0.877907in}}%
\pgfusepath{clip}%
\pgfsetbuttcap%
\pgfsetroundjoin%
\pgfsetlinewidth{1.505625pt}%
\definecolor{currentstroke}{rgb}{0.000000,0.000000,0.000000}%
\pgfsetstrokecolor{currentstroke}%
\pgfsetdash{}{0pt}%
\pgfpathmoveto{\pgfqpoint{12.031511in}{7.318936in}}%
\pgfpathlineto{\pgfqpoint{12.031511in}{7.319448in}}%
\pgfusepath{stroke}%
\end{pgfscope}%
\begin{pgfscope}%
\pgfpathrectangle{\pgfqpoint{9.810417in}{7.240698in}}{\pgfqpoint{5.489583in}{0.877907in}}%
\pgfusepath{clip}%
\pgfsetbuttcap%
\pgfsetroundjoin%
\pgfsetlinewidth{1.505625pt}%
\definecolor{currentstroke}{rgb}{0.000000,0.000000,0.000000}%
\pgfsetstrokecolor{currentstroke}%
\pgfsetdash{}{0pt}%
\pgfpathmoveto{\pgfqpoint{12.154734in}{7.318936in}}%
\pgfpathlineto{\pgfqpoint{12.154734in}{7.330533in}}%
\pgfusepath{stroke}%
\end{pgfscope}%
\begin{pgfscope}%
\pgfpathrectangle{\pgfqpoint{9.810417in}{7.240698in}}{\pgfqpoint{5.489583in}{0.877907in}}%
\pgfusepath{clip}%
\pgfsetbuttcap%
\pgfsetroundjoin%
\pgfsetlinewidth{1.505625pt}%
\definecolor{currentstroke}{rgb}{0.000000,0.000000,0.000000}%
\pgfsetstrokecolor{currentstroke}%
\pgfsetdash{}{0pt}%
\pgfpathmoveto{\pgfqpoint{12.277957in}{7.318936in}}%
\pgfpathlineto{\pgfqpoint{12.277957in}{7.311897in}}%
\pgfusepath{stroke}%
\end{pgfscope}%
\begin{pgfscope}%
\pgfpathrectangle{\pgfqpoint{9.810417in}{7.240698in}}{\pgfqpoint{5.489583in}{0.877907in}}%
\pgfusepath{clip}%
\pgfsetbuttcap%
\pgfsetroundjoin%
\pgfsetlinewidth{1.505625pt}%
\definecolor{currentstroke}{rgb}{0.000000,0.000000,0.000000}%
\pgfsetstrokecolor{currentstroke}%
\pgfsetdash{}{0pt}%
\pgfpathmoveto{\pgfqpoint{12.401180in}{7.318936in}}%
\pgfpathlineto{\pgfqpoint{12.401180in}{7.310123in}}%
\pgfusepath{stroke}%
\end{pgfscope}%
\begin{pgfscope}%
\pgfpathrectangle{\pgfqpoint{9.810417in}{7.240698in}}{\pgfqpoint{5.489583in}{0.877907in}}%
\pgfusepath{clip}%
\pgfsetbuttcap%
\pgfsetroundjoin%
\pgfsetlinewidth{1.505625pt}%
\definecolor{currentstroke}{rgb}{0.000000,0.000000,0.000000}%
\pgfsetstrokecolor{currentstroke}%
\pgfsetdash{}{0pt}%
\pgfpathmoveto{\pgfqpoint{12.524403in}{7.318936in}}%
\pgfpathlineto{\pgfqpoint{12.524403in}{7.331737in}}%
\pgfusepath{stroke}%
\end{pgfscope}%
\begin{pgfscope}%
\pgfpathrectangle{\pgfqpoint{9.810417in}{7.240698in}}{\pgfqpoint{5.489583in}{0.877907in}}%
\pgfusepath{clip}%
\pgfsetbuttcap%
\pgfsetroundjoin%
\pgfsetlinewidth{1.505625pt}%
\definecolor{currentstroke}{rgb}{0.000000,0.000000,0.000000}%
\pgfsetstrokecolor{currentstroke}%
\pgfsetdash{}{0pt}%
\pgfpathmoveto{\pgfqpoint{12.647626in}{7.318936in}}%
\pgfpathlineto{\pgfqpoint{12.647626in}{7.312940in}}%
\pgfusepath{stroke}%
\end{pgfscope}%
\begin{pgfscope}%
\pgfpathrectangle{\pgfqpoint{9.810417in}{7.240698in}}{\pgfqpoint{5.489583in}{0.877907in}}%
\pgfusepath{clip}%
\pgfsetbuttcap%
\pgfsetroundjoin%
\pgfsetlinewidth{1.505625pt}%
\definecolor{currentstroke}{rgb}{0.000000,0.000000,0.000000}%
\pgfsetstrokecolor{currentstroke}%
\pgfsetdash{}{0pt}%
\pgfpathmoveto{\pgfqpoint{12.770849in}{7.318936in}}%
\pgfpathlineto{\pgfqpoint{12.770849in}{7.321245in}}%
\pgfusepath{stroke}%
\end{pgfscope}%
\begin{pgfscope}%
\pgfpathrectangle{\pgfqpoint{9.810417in}{7.240698in}}{\pgfqpoint{5.489583in}{0.877907in}}%
\pgfusepath{clip}%
\pgfsetbuttcap%
\pgfsetroundjoin%
\pgfsetlinewidth{1.505625pt}%
\definecolor{currentstroke}{rgb}{0.000000,0.000000,0.000000}%
\pgfsetstrokecolor{currentstroke}%
\pgfsetdash{}{0pt}%
\pgfpathmoveto{\pgfqpoint{12.894072in}{7.318936in}}%
\pgfpathlineto{\pgfqpoint{12.894072in}{7.338549in}}%
\pgfusepath{stroke}%
\end{pgfscope}%
\begin{pgfscope}%
\pgfpathrectangle{\pgfqpoint{9.810417in}{7.240698in}}{\pgfqpoint{5.489583in}{0.877907in}}%
\pgfusepath{clip}%
\pgfsetbuttcap%
\pgfsetroundjoin%
\pgfsetlinewidth{1.505625pt}%
\definecolor{currentstroke}{rgb}{0.000000,0.000000,0.000000}%
\pgfsetstrokecolor{currentstroke}%
\pgfsetdash{}{0pt}%
\pgfpathmoveto{\pgfqpoint{13.017294in}{7.318936in}}%
\pgfpathlineto{\pgfqpoint{13.017294in}{7.311332in}}%
\pgfusepath{stroke}%
\end{pgfscope}%
\begin{pgfscope}%
\pgfpathrectangle{\pgfqpoint{9.810417in}{7.240698in}}{\pgfqpoint{5.489583in}{0.877907in}}%
\pgfusepath{clip}%
\pgfsetbuttcap%
\pgfsetroundjoin%
\pgfsetlinewidth{1.505625pt}%
\definecolor{currentstroke}{rgb}{0.000000,0.000000,0.000000}%
\pgfsetstrokecolor{currentstroke}%
\pgfsetdash{}{0pt}%
\pgfpathmoveto{\pgfqpoint{13.140517in}{7.318936in}}%
\pgfpathlineto{\pgfqpoint{13.140517in}{7.315351in}}%
\pgfusepath{stroke}%
\end{pgfscope}%
\begin{pgfscope}%
\pgfpathrectangle{\pgfqpoint{9.810417in}{7.240698in}}{\pgfqpoint{5.489583in}{0.877907in}}%
\pgfusepath{clip}%
\pgfsetbuttcap%
\pgfsetroundjoin%
\pgfsetlinewidth{1.505625pt}%
\definecolor{currentstroke}{rgb}{0.000000,0.000000,0.000000}%
\pgfsetstrokecolor{currentstroke}%
\pgfsetdash{}{0pt}%
\pgfpathmoveto{\pgfqpoint{13.263740in}{7.318936in}}%
\pgfpathlineto{\pgfqpoint{13.263740in}{7.339863in}}%
\pgfusepath{stroke}%
\end{pgfscope}%
\begin{pgfscope}%
\pgfpathrectangle{\pgfqpoint{9.810417in}{7.240698in}}{\pgfqpoint{5.489583in}{0.877907in}}%
\pgfusepath{clip}%
\pgfsetbuttcap%
\pgfsetroundjoin%
\pgfsetlinewidth{1.505625pt}%
\definecolor{currentstroke}{rgb}{0.000000,0.000000,0.000000}%
\pgfsetstrokecolor{currentstroke}%
\pgfsetdash{}{0pt}%
\pgfpathmoveto{\pgfqpoint{13.386963in}{7.318936in}}%
\pgfpathlineto{\pgfqpoint{13.386963in}{7.343497in}}%
\pgfusepath{stroke}%
\end{pgfscope}%
\begin{pgfscope}%
\pgfpathrectangle{\pgfqpoint{9.810417in}{7.240698in}}{\pgfqpoint{5.489583in}{0.877907in}}%
\pgfusepath{clip}%
\pgfsetbuttcap%
\pgfsetroundjoin%
\pgfsetlinewidth{1.505625pt}%
\definecolor{currentstroke}{rgb}{0.000000,0.000000,0.000000}%
\pgfsetstrokecolor{currentstroke}%
\pgfsetdash{}{0pt}%
\pgfpathmoveto{\pgfqpoint{13.510186in}{7.318936in}}%
\pgfpathlineto{\pgfqpoint{13.510186in}{7.320560in}}%
\pgfusepath{stroke}%
\end{pgfscope}%
\begin{pgfscope}%
\pgfpathrectangle{\pgfqpoint{9.810417in}{7.240698in}}{\pgfqpoint{5.489583in}{0.877907in}}%
\pgfusepath{clip}%
\pgfsetbuttcap%
\pgfsetroundjoin%
\pgfsetlinewidth{1.505625pt}%
\definecolor{currentstroke}{rgb}{0.000000,0.000000,0.000000}%
\pgfsetstrokecolor{currentstroke}%
\pgfsetdash{}{0pt}%
\pgfpathmoveto{\pgfqpoint{13.633409in}{7.318936in}}%
\pgfpathlineto{\pgfqpoint{13.633409in}{7.322852in}}%
\pgfusepath{stroke}%
\end{pgfscope}%
\begin{pgfscope}%
\pgfpathrectangle{\pgfqpoint{9.810417in}{7.240698in}}{\pgfqpoint{5.489583in}{0.877907in}}%
\pgfusepath{clip}%
\pgfsetbuttcap%
\pgfsetroundjoin%
\pgfsetlinewidth{1.505625pt}%
\definecolor{currentstroke}{rgb}{0.000000,0.000000,0.000000}%
\pgfsetstrokecolor{currentstroke}%
\pgfsetdash{}{0pt}%
\pgfpathmoveto{\pgfqpoint{13.756632in}{7.318936in}}%
\pgfpathlineto{\pgfqpoint{13.756632in}{7.294712in}}%
\pgfusepath{stroke}%
\end{pgfscope}%
\begin{pgfscope}%
\pgfpathrectangle{\pgfqpoint{9.810417in}{7.240698in}}{\pgfqpoint{5.489583in}{0.877907in}}%
\pgfusepath{clip}%
\pgfsetbuttcap%
\pgfsetroundjoin%
\pgfsetlinewidth{1.505625pt}%
\definecolor{currentstroke}{rgb}{0.000000,0.000000,0.000000}%
\pgfsetstrokecolor{currentstroke}%
\pgfsetdash{}{0pt}%
\pgfpathmoveto{\pgfqpoint{13.879855in}{7.318936in}}%
\pgfpathlineto{\pgfqpoint{13.879855in}{7.311488in}}%
\pgfusepath{stroke}%
\end{pgfscope}%
\begin{pgfscope}%
\pgfpathrectangle{\pgfqpoint{9.810417in}{7.240698in}}{\pgfqpoint{5.489583in}{0.877907in}}%
\pgfusepath{clip}%
\pgfsetbuttcap%
\pgfsetroundjoin%
\pgfsetlinewidth{1.505625pt}%
\definecolor{currentstroke}{rgb}{0.000000,0.000000,0.000000}%
\pgfsetstrokecolor{currentstroke}%
\pgfsetdash{}{0pt}%
\pgfpathmoveto{\pgfqpoint{14.003078in}{7.318936in}}%
\pgfpathlineto{\pgfqpoint{14.003078in}{7.347219in}}%
\pgfusepath{stroke}%
\end{pgfscope}%
\begin{pgfscope}%
\pgfpathrectangle{\pgfqpoint{9.810417in}{7.240698in}}{\pgfqpoint{5.489583in}{0.877907in}}%
\pgfusepath{clip}%
\pgfsetbuttcap%
\pgfsetroundjoin%
\pgfsetlinewidth{1.505625pt}%
\definecolor{currentstroke}{rgb}{0.000000,0.000000,0.000000}%
\pgfsetstrokecolor{currentstroke}%
\pgfsetdash{}{0pt}%
\pgfpathmoveto{\pgfqpoint{14.126301in}{7.318936in}}%
\pgfpathlineto{\pgfqpoint{14.126301in}{7.343185in}}%
\pgfusepath{stroke}%
\end{pgfscope}%
\begin{pgfscope}%
\pgfpathrectangle{\pgfqpoint{9.810417in}{7.240698in}}{\pgfqpoint{5.489583in}{0.877907in}}%
\pgfusepath{clip}%
\pgfsetbuttcap%
\pgfsetroundjoin%
\pgfsetlinewidth{1.505625pt}%
\definecolor{currentstroke}{rgb}{0.000000,0.000000,0.000000}%
\pgfsetstrokecolor{currentstroke}%
\pgfsetdash{}{0pt}%
\pgfpathmoveto{\pgfqpoint{14.249524in}{7.318936in}}%
\pgfpathlineto{\pgfqpoint{14.249524in}{7.320056in}}%
\pgfusepath{stroke}%
\end{pgfscope}%
\begin{pgfscope}%
\pgfpathrectangle{\pgfqpoint{9.810417in}{7.240698in}}{\pgfqpoint{5.489583in}{0.877907in}}%
\pgfusepath{clip}%
\pgfsetbuttcap%
\pgfsetroundjoin%
\pgfsetlinewidth{1.505625pt}%
\definecolor{currentstroke}{rgb}{0.000000,0.000000,0.000000}%
\pgfsetstrokecolor{currentstroke}%
\pgfsetdash{}{0pt}%
\pgfpathmoveto{\pgfqpoint{14.372747in}{7.318936in}}%
\pgfpathlineto{\pgfqpoint{14.372747in}{7.301917in}}%
\pgfusepath{stroke}%
\end{pgfscope}%
\begin{pgfscope}%
\pgfpathrectangle{\pgfqpoint{9.810417in}{7.240698in}}{\pgfqpoint{5.489583in}{0.877907in}}%
\pgfusepath{clip}%
\pgfsetbuttcap%
\pgfsetroundjoin%
\pgfsetlinewidth{1.505625pt}%
\definecolor{currentstroke}{rgb}{0.000000,0.000000,0.000000}%
\pgfsetstrokecolor{currentstroke}%
\pgfsetdash{}{0pt}%
\pgfpathmoveto{\pgfqpoint{14.495970in}{7.318936in}}%
\pgfpathlineto{\pgfqpoint{14.495970in}{7.284857in}}%
\pgfusepath{stroke}%
\end{pgfscope}%
\begin{pgfscope}%
\pgfpathrectangle{\pgfqpoint{9.810417in}{7.240698in}}{\pgfqpoint{5.489583in}{0.877907in}}%
\pgfusepath{clip}%
\pgfsetbuttcap%
\pgfsetroundjoin%
\pgfsetlinewidth{1.505625pt}%
\definecolor{currentstroke}{rgb}{0.000000,0.000000,0.000000}%
\pgfsetstrokecolor{currentstroke}%
\pgfsetdash{}{0pt}%
\pgfpathmoveto{\pgfqpoint{14.619193in}{7.318936in}}%
\pgfpathlineto{\pgfqpoint{14.619193in}{7.313659in}}%
\pgfusepath{stroke}%
\end{pgfscope}%
\begin{pgfscope}%
\pgfpathrectangle{\pgfqpoint{9.810417in}{7.240698in}}{\pgfqpoint{5.489583in}{0.877907in}}%
\pgfusepath{clip}%
\pgfsetbuttcap%
\pgfsetroundjoin%
\pgfsetlinewidth{1.505625pt}%
\definecolor{currentstroke}{rgb}{0.000000,0.000000,0.000000}%
\pgfsetstrokecolor{currentstroke}%
\pgfsetdash{}{0pt}%
\pgfpathmoveto{\pgfqpoint{14.742416in}{7.318936in}}%
\pgfpathlineto{\pgfqpoint{14.742416in}{7.314069in}}%
\pgfusepath{stroke}%
\end{pgfscope}%
\begin{pgfscope}%
\pgfpathrectangle{\pgfqpoint{9.810417in}{7.240698in}}{\pgfqpoint{5.489583in}{0.877907in}}%
\pgfusepath{clip}%
\pgfsetbuttcap%
\pgfsetroundjoin%
\pgfsetlinewidth{1.505625pt}%
\definecolor{currentstroke}{rgb}{0.000000,0.000000,0.000000}%
\pgfsetstrokecolor{currentstroke}%
\pgfsetdash{}{0pt}%
\pgfpathmoveto{\pgfqpoint{14.865639in}{7.318936in}}%
\pgfpathlineto{\pgfqpoint{14.865639in}{7.325345in}}%
\pgfusepath{stroke}%
\end{pgfscope}%
\begin{pgfscope}%
\pgfpathrectangle{\pgfqpoint{9.810417in}{7.240698in}}{\pgfqpoint{5.489583in}{0.877907in}}%
\pgfusepath{clip}%
\pgfsetbuttcap%
\pgfsetroundjoin%
\pgfsetlinewidth{1.505625pt}%
\definecolor{currentstroke}{rgb}{0.000000,0.000000,0.000000}%
\pgfsetstrokecolor{currentstroke}%
\pgfsetdash{}{0pt}%
\pgfpathmoveto{\pgfqpoint{14.988862in}{7.318936in}}%
\pgfpathlineto{\pgfqpoint{14.988862in}{7.318718in}}%
\pgfusepath{stroke}%
\end{pgfscope}%
\begin{pgfscope}%
\pgfpathrectangle{\pgfqpoint{9.810417in}{7.240698in}}{\pgfqpoint{5.489583in}{0.877907in}}%
\pgfusepath{clip}%
\pgfsetroundcap%
\pgfsetroundjoin%
\pgfsetlinewidth{1.505625pt}%
\definecolor{currentstroke}{rgb}{0.121569,0.466667,0.705882}%
\pgfsetstrokecolor{currentstroke}%
\pgfsetdash{}{0pt}%
\pgfpathmoveto{\pgfqpoint{9.810417in}{7.318936in}}%
\pgfpathlineto{\pgfqpoint{15.300000in}{7.318936in}}%
\pgfusepath{stroke}%
\end{pgfscope}%
\begin{pgfscope}%
\pgfpathrectangle{\pgfqpoint{9.810417in}{7.240698in}}{\pgfqpoint{5.489583in}{0.877907in}}%
\pgfusepath{clip}%
\pgfsetbuttcap%
\pgfsetroundjoin%
\definecolor{currentfill}{rgb}{0.121569,0.466667,0.705882}%
\pgfsetfillcolor{currentfill}%
\pgfsetlinewidth{1.003750pt}%
\definecolor{currentstroke}{rgb}{0.121569,0.466667,0.705882}%
\pgfsetstrokecolor{currentstroke}%
\pgfsetdash{}{0pt}%
\pgfsys@defobject{currentmarker}{\pgfqpoint{-0.034722in}{-0.034722in}}{\pgfqpoint{0.034722in}{0.034722in}}{%
\pgfpathmoveto{\pgfqpoint{0.000000in}{-0.034722in}}%
\pgfpathcurveto{\pgfqpoint{0.009208in}{-0.034722in}}{\pgfqpoint{0.018041in}{-0.031064in}}{\pgfqpoint{0.024552in}{-0.024552in}}%
\pgfpathcurveto{\pgfqpoint{0.031064in}{-0.018041in}}{\pgfqpoint{0.034722in}{-0.009208in}}{\pgfqpoint{0.034722in}{0.000000in}}%
\pgfpathcurveto{\pgfqpoint{0.034722in}{0.009208in}}{\pgfqpoint{0.031064in}{0.018041in}}{\pgfqpoint{0.024552in}{0.024552in}}%
\pgfpathcurveto{\pgfqpoint{0.018041in}{0.031064in}}{\pgfqpoint{0.009208in}{0.034722in}}{\pgfqpoint{0.000000in}{0.034722in}}%
\pgfpathcurveto{\pgfqpoint{-0.009208in}{0.034722in}}{\pgfqpoint{-0.018041in}{0.031064in}}{\pgfqpoint{-0.024552in}{0.024552in}}%
\pgfpathcurveto{\pgfqpoint{-0.031064in}{0.018041in}}{\pgfqpoint{-0.034722in}{0.009208in}}{\pgfqpoint{-0.034722in}{0.000000in}}%
\pgfpathcurveto{\pgfqpoint{-0.034722in}{-0.009208in}}{\pgfqpoint{-0.031064in}{-0.018041in}}{\pgfqpoint{-0.024552in}{-0.024552in}}%
\pgfpathcurveto{\pgfqpoint{-0.018041in}{-0.031064in}}{\pgfqpoint{-0.009208in}{-0.034722in}}{\pgfqpoint{0.000000in}{-0.034722in}}%
\pgfpathclose%
\pgfusepath{stroke,fill}%
}%
\begin{pgfscope}%
\pgfsys@transformshift{10.059943in}{8.078700in}%
\pgfsys@useobject{currentmarker}{}%
\end{pgfscope}%
\begin{pgfscope}%
\pgfsys@transformshift{10.183166in}{8.075900in}%
\pgfsys@useobject{currentmarker}{}%
\end{pgfscope}%
\begin{pgfscope}%
\pgfsys@transformshift{10.306389in}{7.309470in}%
\pgfsys@useobject{currentmarker}{}%
\end{pgfscope}%
\begin{pgfscope}%
\pgfsys@transformshift{10.429612in}{7.318066in}%
\pgfsys@useobject{currentmarker}{}%
\end{pgfscope}%
\begin{pgfscope}%
\pgfsys@transformshift{10.552835in}{7.305082in}%
\pgfsys@useobject{currentmarker}{}%
\end{pgfscope}%
\begin{pgfscope}%
\pgfsys@transformshift{10.676058in}{7.336369in}%
\pgfsys@useobject{currentmarker}{}%
\end{pgfscope}%
\begin{pgfscope}%
\pgfsys@transformshift{10.799281in}{7.344185in}%
\pgfsys@useobject{currentmarker}{}%
\end{pgfscope}%
\begin{pgfscope}%
\pgfsys@transformshift{10.922504in}{7.330536in}%
\pgfsys@useobject{currentmarker}{}%
\end{pgfscope}%
\begin{pgfscope}%
\pgfsys@transformshift{11.045727in}{7.317247in}%
\pgfsys@useobject{currentmarker}{}%
\end{pgfscope}%
\begin{pgfscope}%
\pgfsys@transformshift{11.168950in}{7.287925in}%
\pgfsys@useobject{currentmarker}{}%
\end{pgfscope}%
\begin{pgfscope}%
\pgfsys@transformshift{11.292173in}{7.319602in}%
\pgfsys@useobject{currentmarker}{}%
\end{pgfscope}%
\begin{pgfscope}%
\pgfsys@transformshift{11.415396in}{7.345977in}%
\pgfsys@useobject{currentmarker}{}%
\end{pgfscope}%
\begin{pgfscope}%
\pgfsys@transformshift{11.538619in}{7.303012in}%
\pgfsys@useobject{currentmarker}{}%
\end{pgfscope}%
\begin{pgfscope}%
\pgfsys@transformshift{11.661842in}{7.307171in}%
\pgfsys@useobject{currentmarker}{}%
\end{pgfscope}%
\begin{pgfscope}%
\pgfsys@transformshift{11.785065in}{7.331168in}%
\pgfsys@useobject{currentmarker}{}%
\end{pgfscope}%
\begin{pgfscope}%
\pgfsys@transformshift{11.908288in}{7.339705in}%
\pgfsys@useobject{currentmarker}{}%
\end{pgfscope}%
\begin{pgfscope}%
\pgfsys@transformshift{12.031511in}{7.319448in}%
\pgfsys@useobject{currentmarker}{}%
\end{pgfscope}%
\begin{pgfscope}%
\pgfsys@transformshift{12.154734in}{7.330533in}%
\pgfsys@useobject{currentmarker}{}%
\end{pgfscope}%
\begin{pgfscope}%
\pgfsys@transformshift{12.277957in}{7.311897in}%
\pgfsys@useobject{currentmarker}{}%
\end{pgfscope}%
\begin{pgfscope}%
\pgfsys@transformshift{12.401180in}{7.310123in}%
\pgfsys@useobject{currentmarker}{}%
\end{pgfscope}%
\begin{pgfscope}%
\pgfsys@transformshift{12.524403in}{7.331737in}%
\pgfsys@useobject{currentmarker}{}%
\end{pgfscope}%
\begin{pgfscope}%
\pgfsys@transformshift{12.647626in}{7.312940in}%
\pgfsys@useobject{currentmarker}{}%
\end{pgfscope}%
\begin{pgfscope}%
\pgfsys@transformshift{12.770849in}{7.321245in}%
\pgfsys@useobject{currentmarker}{}%
\end{pgfscope}%
\begin{pgfscope}%
\pgfsys@transformshift{12.894072in}{7.338549in}%
\pgfsys@useobject{currentmarker}{}%
\end{pgfscope}%
\begin{pgfscope}%
\pgfsys@transformshift{13.017294in}{7.311332in}%
\pgfsys@useobject{currentmarker}{}%
\end{pgfscope}%
\begin{pgfscope}%
\pgfsys@transformshift{13.140517in}{7.315351in}%
\pgfsys@useobject{currentmarker}{}%
\end{pgfscope}%
\begin{pgfscope}%
\pgfsys@transformshift{13.263740in}{7.339863in}%
\pgfsys@useobject{currentmarker}{}%
\end{pgfscope}%
\begin{pgfscope}%
\pgfsys@transformshift{13.386963in}{7.343497in}%
\pgfsys@useobject{currentmarker}{}%
\end{pgfscope}%
\begin{pgfscope}%
\pgfsys@transformshift{13.510186in}{7.320560in}%
\pgfsys@useobject{currentmarker}{}%
\end{pgfscope}%
\begin{pgfscope}%
\pgfsys@transformshift{13.633409in}{7.322852in}%
\pgfsys@useobject{currentmarker}{}%
\end{pgfscope}%
\begin{pgfscope}%
\pgfsys@transformshift{13.756632in}{7.294712in}%
\pgfsys@useobject{currentmarker}{}%
\end{pgfscope}%
\begin{pgfscope}%
\pgfsys@transformshift{13.879855in}{7.311488in}%
\pgfsys@useobject{currentmarker}{}%
\end{pgfscope}%
\begin{pgfscope}%
\pgfsys@transformshift{14.003078in}{7.347219in}%
\pgfsys@useobject{currentmarker}{}%
\end{pgfscope}%
\begin{pgfscope}%
\pgfsys@transformshift{14.126301in}{7.343185in}%
\pgfsys@useobject{currentmarker}{}%
\end{pgfscope}%
\begin{pgfscope}%
\pgfsys@transformshift{14.249524in}{7.320056in}%
\pgfsys@useobject{currentmarker}{}%
\end{pgfscope}%
\begin{pgfscope}%
\pgfsys@transformshift{14.372747in}{7.301917in}%
\pgfsys@useobject{currentmarker}{}%
\end{pgfscope}%
\begin{pgfscope}%
\pgfsys@transformshift{14.495970in}{7.284857in}%
\pgfsys@useobject{currentmarker}{}%
\end{pgfscope}%
\begin{pgfscope}%
\pgfsys@transformshift{14.619193in}{7.313659in}%
\pgfsys@useobject{currentmarker}{}%
\end{pgfscope}%
\begin{pgfscope}%
\pgfsys@transformshift{14.742416in}{7.314069in}%
\pgfsys@useobject{currentmarker}{}%
\end{pgfscope}%
\begin{pgfscope}%
\pgfsys@transformshift{14.865639in}{7.325345in}%
\pgfsys@useobject{currentmarker}{}%
\end{pgfscope}%
\begin{pgfscope}%
\pgfsys@transformshift{14.988862in}{7.318718in}%
\pgfsys@useobject{currentmarker}{}%
\end{pgfscope}%
\end{pgfscope}%
\begin{pgfscope}%
\pgfsetrectcap%
\pgfsetmiterjoin%
\pgfsetlinewidth{0.803000pt}%
\definecolor{currentstroke}{rgb}{1.000000,1.000000,1.000000}%
\pgfsetstrokecolor{currentstroke}%
\pgfsetdash{}{0pt}%
\pgfpathmoveto{\pgfqpoint{9.810417in}{7.240698in}}%
\pgfpathlineto{\pgfqpoint{9.810417in}{8.118605in}}%
\pgfusepath{stroke}%
\end{pgfscope}%
\begin{pgfscope}%
\pgfsetrectcap%
\pgfsetmiterjoin%
\pgfsetlinewidth{0.803000pt}%
\definecolor{currentstroke}{rgb}{1.000000,1.000000,1.000000}%
\pgfsetstrokecolor{currentstroke}%
\pgfsetdash{}{0pt}%
\pgfpathmoveto{\pgfqpoint{15.300000in}{7.240698in}}%
\pgfpathlineto{\pgfqpoint{15.300000in}{8.118605in}}%
\pgfusepath{stroke}%
\end{pgfscope}%
\begin{pgfscope}%
\pgfsetrectcap%
\pgfsetmiterjoin%
\pgfsetlinewidth{0.803000pt}%
\definecolor{currentstroke}{rgb}{1.000000,1.000000,1.000000}%
\pgfsetstrokecolor{currentstroke}%
\pgfsetdash{}{0pt}%
\pgfpathmoveto{\pgfqpoint{9.810417in}{7.240698in}}%
\pgfpathlineto{\pgfqpoint{15.300000in}{7.240698in}}%
\pgfusepath{stroke}%
\end{pgfscope}%
\begin{pgfscope}%
\pgfsetrectcap%
\pgfsetmiterjoin%
\pgfsetlinewidth{0.803000pt}%
\definecolor{currentstroke}{rgb}{1.000000,1.000000,1.000000}%
\pgfsetstrokecolor{currentstroke}%
\pgfsetdash{}{0pt}%
\pgfpathmoveto{\pgfqpoint{9.810417in}{8.118605in}}%
\pgfpathlineto{\pgfqpoint{15.300000in}{8.118605in}}%
\pgfusepath{stroke}%
\end{pgfscope}%
\begin{pgfscope}%
\definecolor{textcolor}{rgb}{0.150000,0.150000,0.150000}%
\pgfsetstrokecolor{textcolor}%
\pgfsetfillcolor{textcolor}%
\pgftext[x=12.555208in,y=8.201938in,,base]{\color{textcolor}\rmfamily\fontsize{16.800000}{20.160000}\selectfont Partial Autocorrelation}%
\end{pgfscope}%
\begin{pgfscope}%
\pgfsetbuttcap%
\pgfsetmiterjoin%
\definecolor{currentfill}{rgb}{0.917647,0.917647,0.949020}%
\pgfsetfillcolor{currentfill}%
\pgfsetlinewidth{0.000000pt}%
\definecolor{currentstroke}{rgb}{0.000000,0.000000,0.000000}%
\pgfsetstrokecolor{currentstroke}%
\pgfsetstrokeopacity{0.000000}%
\pgfsetdash{}{0pt}%
\pgfpathmoveto{\pgfqpoint{2.125000in}{5.660465in}}%
\pgfpathlineto{\pgfqpoint{7.614583in}{5.660465in}}%
\pgfpathlineto{\pgfqpoint{7.614583in}{6.538372in}}%
\pgfpathlineto{\pgfqpoint{2.125000in}{6.538372in}}%
\pgfpathclose%
\pgfusepath{fill}%
\end{pgfscope}%
\begin{pgfscope}%
\pgfpathrectangle{\pgfqpoint{2.125000in}{5.660465in}}{\pgfqpoint{5.489583in}{0.877907in}}%
\pgfusepath{clip}%
\pgfsetroundcap%
\pgfsetroundjoin%
\pgfsetlinewidth{0.803000pt}%
\definecolor{currentstroke}{rgb}{1.000000,1.000000,1.000000}%
\pgfsetstrokecolor{currentstroke}%
\pgfsetdash{}{0pt}%
\pgfpathmoveto{\pgfqpoint{2.374527in}{5.660465in}}%
\pgfpathlineto{\pgfqpoint{2.374527in}{6.538372in}}%
\pgfusepath{stroke}%
\end{pgfscope}%
\begin{pgfscope}%
\definecolor{textcolor}{rgb}{0.150000,0.150000,0.150000}%
\pgfsetstrokecolor{textcolor}%
\pgfsetfillcolor{textcolor}%
\pgftext[x=2.374527in,y=5.563243in,,top]{\color{textcolor}\rmfamily\fontsize{14.000000}{16.800000}\selectfont 0}%
\end{pgfscope}%
\begin{pgfscope}%
\pgfpathrectangle{\pgfqpoint{2.125000in}{5.660465in}}{\pgfqpoint{5.489583in}{0.877907in}}%
\pgfusepath{clip}%
\pgfsetroundcap%
\pgfsetroundjoin%
\pgfsetlinewidth{0.803000pt}%
\definecolor{currentstroke}{rgb}{1.000000,1.000000,1.000000}%
\pgfsetstrokecolor{currentstroke}%
\pgfsetdash{}{0pt}%
\pgfpathmoveto{\pgfqpoint{2.990641in}{5.660465in}}%
\pgfpathlineto{\pgfqpoint{2.990641in}{6.538372in}}%
\pgfusepath{stroke}%
\end{pgfscope}%
\begin{pgfscope}%
\definecolor{textcolor}{rgb}{0.150000,0.150000,0.150000}%
\pgfsetstrokecolor{textcolor}%
\pgfsetfillcolor{textcolor}%
\pgftext[x=2.990641in,y=5.563243in,,top]{\color{textcolor}\rmfamily\fontsize{14.000000}{16.800000}\selectfont 5}%
\end{pgfscope}%
\begin{pgfscope}%
\pgfpathrectangle{\pgfqpoint{2.125000in}{5.660465in}}{\pgfqpoint{5.489583in}{0.877907in}}%
\pgfusepath{clip}%
\pgfsetroundcap%
\pgfsetroundjoin%
\pgfsetlinewidth{0.803000pt}%
\definecolor{currentstroke}{rgb}{1.000000,1.000000,1.000000}%
\pgfsetstrokecolor{currentstroke}%
\pgfsetdash{}{0pt}%
\pgfpathmoveto{\pgfqpoint{3.606756in}{5.660465in}}%
\pgfpathlineto{\pgfqpoint{3.606756in}{6.538372in}}%
\pgfusepath{stroke}%
\end{pgfscope}%
\begin{pgfscope}%
\definecolor{textcolor}{rgb}{0.150000,0.150000,0.150000}%
\pgfsetstrokecolor{textcolor}%
\pgfsetfillcolor{textcolor}%
\pgftext[x=3.606756in,y=5.563243in,,top]{\color{textcolor}\rmfamily\fontsize{14.000000}{16.800000}\selectfont 10}%
\end{pgfscope}%
\begin{pgfscope}%
\pgfpathrectangle{\pgfqpoint{2.125000in}{5.660465in}}{\pgfqpoint{5.489583in}{0.877907in}}%
\pgfusepath{clip}%
\pgfsetroundcap%
\pgfsetroundjoin%
\pgfsetlinewidth{0.803000pt}%
\definecolor{currentstroke}{rgb}{1.000000,1.000000,1.000000}%
\pgfsetstrokecolor{currentstroke}%
\pgfsetdash{}{0pt}%
\pgfpathmoveto{\pgfqpoint{4.222871in}{5.660465in}}%
\pgfpathlineto{\pgfqpoint{4.222871in}{6.538372in}}%
\pgfusepath{stroke}%
\end{pgfscope}%
\begin{pgfscope}%
\definecolor{textcolor}{rgb}{0.150000,0.150000,0.150000}%
\pgfsetstrokecolor{textcolor}%
\pgfsetfillcolor{textcolor}%
\pgftext[x=4.222871in,y=5.563243in,,top]{\color{textcolor}\rmfamily\fontsize{14.000000}{16.800000}\selectfont 15}%
\end{pgfscope}%
\begin{pgfscope}%
\pgfpathrectangle{\pgfqpoint{2.125000in}{5.660465in}}{\pgfqpoint{5.489583in}{0.877907in}}%
\pgfusepath{clip}%
\pgfsetroundcap%
\pgfsetroundjoin%
\pgfsetlinewidth{0.803000pt}%
\definecolor{currentstroke}{rgb}{1.000000,1.000000,1.000000}%
\pgfsetstrokecolor{currentstroke}%
\pgfsetdash{}{0pt}%
\pgfpathmoveto{\pgfqpoint{4.838986in}{5.660465in}}%
\pgfpathlineto{\pgfqpoint{4.838986in}{6.538372in}}%
\pgfusepath{stroke}%
\end{pgfscope}%
\begin{pgfscope}%
\definecolor{textcolor}{rgb}{0.150000,0.150000,0.150000}%
\pgfsetstrokecolor{textcolor}%
\pgfsetfillcolor{textcolor}%
\pgftext[x=4.838986in,y=5.563243in,,top]{\color{textcolor}\rmfamily\fontsize{14.000000}{16.800000}\selectfont 20}%
\end{pgfscope}%
\begin{pgfscope}%
\pgfpathrectangle{\pgfqpoint{2.125000in}{5.660465in}}{\pgfqpoint{5.489583in}{0.877907in}}%
\pgfusepath{clip}%
\pgfsetroundcap%
\pgfsetroundjoin%
\pgfsetlinewidth{0.803000pt}%
\definecolor{currentstroke}{rgb}{1.000000,1.000000,1.000000}%
\pgfsetstrokecolor{currentstroke}%
\pgfsetdash{}{0pt}%
\pgfpathmoveto{\pgfqpoint{5.455101in}{5.660465in}}%
\pgfpathlineto{\pgfqpoint{5.455101in}{6.538372in}}%
\pgfusepath{stroke}%
\end{pgfscope}%
\begin{pgfscope}%
\definecolor{textcolor}{rgb}{0.150000,0.150000,0.150000}%
\pgfsetstrokecolor{textcolor}%
\pgfsetfillcolor{textcolor}%
\pgftext[x=5.455101in,y=5.563243in,,top]{\color{textcolor}\rmfamily\fontsize{14.000000}{16.800000}\selectfont 25}%
\end{pgfscope}%
\begin{pgfscope}%
\pgfpathrectangle{\pgfqpoint{2.125000in}{5.660465in}}{\pgfqpoint{5.489583in}{0.877907in}}%
\pgfusepath{clip}%
\pgfsetroundcap%
\pgfsetroundjoin%
\pgfsetlinewidth{0.803000pt}%
\definecolor{currentstroke}{rgb}{1.000000,1.000000,1.000000}%
\pgfsetstrokecolor{currentstroke}%
\pgfsetdash{}{0pt}%
\pgfpathmoveto{\pgfqpoint{6.071216in}{5.660465in}}%
\pgfpathlineto{\pgfqpoint{6.071216in}{6.538372in}}%
\pgfusepath{stroke}%
\end{pgfscope}%
\begin{pgfscope}%
\definecolor{textcolor}{rgb}{0.150000,0.150000,0.150000}%
\pgfsetstrokecolor{textcolor}%
\pgfsetfillcolor{textcolor}%
\pgftext[x=6.071216in,y=5.563243in,,top]{\color{textcolor}\rmfamily\fontsize{14.000000}{16.800000}\selectfont 30}%
\end{pgfscope}%
\begin{pgfscope}%
\pgfpathrectangle{\pgfqpoint{2.125000in}{5.660465in}}{\pgfqpoint{5.489583in}{0.877907in}}%
\pgfusepath{clip}%
\pgfsetroundcap%
\pgfsetroundjoin%
\pgfsetlinewidth{0.803000pt}%
\definecolor{currentstroke}{rgb}{1.000000,1.000000,1.000000}%
\pgfsetstrokecolor{currentstroke}%
\pgfsetdash{}{0pt}%
\pgfpathmoveto{\pgfqpoint{6.687330in}{5.660465in}}%
\pgfpathlineto{\pgfqpoint{6.687330in}{6.538372in}}%
\pgfusepath{stroke}%
\end{pgfscope}%
\begin{pgfscope}%
\definecolor{textcolor}{rgb}{0.150000,0.150000,0.150000}%
\pgfsetstrokecolor{textcolor}%
\pgfsetfillcolor{textcolor}%
\pgftext[x=6.687330in,y=5.563243in,,top]{\color{textcolor}\rmfamily\fontsize{14.000000}{16.800000}\selectfont 35}%
\end{pgfscope}%
\begin{pgfscope}%
\pgfpathrectangle{\pgfqpoint{2.125000in}{5.660465in}}{\pgfqpoint{5.489583in}{0.877907in}}%
\pgfusepath{clip}%
\pgfsetroundcap%
\pgfsetroundjoin%
\pgfsetlinewidth{0.803000pt}%
\definecolor{currentstroke}{rgb}{1.000000,1.000000,1.000000}%
\pgfsetstrokecolor{currentstroke}%
\pgfsetdash{}{0pt}%
\pgfpathmoveto{\pgfqpoint{7.303445in}{5.660465in}}%
\pgfpathlineto{\pgfqpoint{7.303445in}{6.538372in}}%
\pgfusepath{stroke}%
\end{pgfscope}%
\begin{pgfscope}%
\definecolor{textcolor}{rgb}{0.150000,0.150000,0.150000}%
\pgfsetstrokecolor{textcolor}%
\pgfsetfillcolor{textcolor}%
\pgftext[x=7.303445in,y=5.563243in,,top]{\color{textcolor}\rmfamily\fontsize{14.000000}{16.800000}\selectfont 40}%
\end{pgfscope}%
\begin{pgfscope}%
\pgfpathrectangle{\pgfqpoint{2.125000in}{5.660465in}}{\pgfqpoint{5.489583in}{0.877907in}}%
\pgfusepath{clip}%
\pgfsetroundcap%
\pgfsetroundjoin%
\pgfsetlinewidth{0.803000pt}%
\definecolor{currentstroke}{rgb}{1.000000,1.000000,1.000000}%
\pgfsetstrokecolor{currentstroke}%
\pgfsetdash{}{0pt}%
\pgfpathmoveto{\pgfqpoint{2.125000in}{5.928493in}}%
\pgfpathlineto{\pgfqpoint{7.614583in}{5.928493in}}%
\pgfusepath{stroke}%
\end{pgfscope}%
\begin{pgfscope}%
\definecolor{textcolor}{rgb}{0.150000,0.150000,0.150000}%
\pgfsetstrokecolor{textcolor}%
\pgfsetfillcolor{textcolor}%
\pgftext[x=1.904066in,y=5.854627in,left,base]{\color{textcolor}\rmfamily\fontsize{14.000000}{16.800000}\selectfont 0}%
\end{pgfscope}%
\begin{pgfscope}%
\pgfpathrectangle{\pgfqpoint{2.125000in}{5.660465in}}{\pgfqpoint{5.489583in}{0.877907in}}%
\pgfusepath{clip}%
\pgfsetroundcap%
\pgfsetroundjoin%
\pgfsetlinewidth{0.803000pt}%
\definecolor{currentstroke}{rgb}{1.000000,1.000000,1.000000}%
\pgfsetstrokecolor{currentstroke}%
\pgfsetdash{}{0pt}%
\pgfpathmoveto{\pgfqpoint{2.125000in}{6.498467in}}%
\pgfpathlineto{\pgfqpoint{7.614583in}{6.498467in}}%
\pgfusepath{stroke}%
\end{pgfscope}%
\begin{pgfscope}%
\definecolor{textcolor}{rgb}{0.150000,0.150000,0.150000}%
\pgfsetstrokecolor{textcolor}%
\pgfsetfillcolor{textcolor}%
\pgftext[x=1.904066in,y=6.424601in,left,base]{\color{textcolor}\rmfamily\fontsize{14.000000}{16.800000}\selectfont 1}%
\end{pgfscope}%
\begin{pgfscope}%
\pgfpathrectangle{\pgfqpoint{2.125000in}{5.660465in}}{\pgfqpoint{5.489583in}{0.877907in}}%
\pgfusepath{clip}%
\pgfsetbuttcap%
\pgfsetroundjoin%
\definecolor{currentfill}{rgb}{0.121569,0.466667,0.705882}%
\pgfsetfillcolor{currentfill}%
\pgfsetfillopacity{0.250000}%
\pgfsetlinewidth{1.003750pt}%
\definecolor{currentstroke}{rgb}{1.000000,1.000000,1.000000}%
\pgfsetstrokecolor{currentstroke}%
\pgfsetstrokeopacity{0.250000}%
\pgfsetdash{}{0pt}%
\pgfpathmoveto{\pgfqpoint{2.436138in}{5.957251in}}%
\pgfpathlineto{\pgfqpoint{2.436138in}{5.899735in}}%
\pgfpathlineto{\pgfqpoint{2.620972in}{5.878864in}}%
\pgfpathlineto{\pgfqpoint{2.744195in}{5.864620in}}%
\pgfpathlineto{\pgfqpoint{2.867418in}{5.853147in}}%
\pgfpathlineto{\pgfqpoint{2.990641in}{5.843319in}}%
\pgfpathlineto{\pgfqpoint{3.113864in}{5.834615in}}%
\pgfpathlineto{\pgfqpoint{3.237087in}{5.826747in}}%
\pgfpathlineto{\pgfqpoint{3.360310in}{5.819535in}}%
\pgfpathlineto{\pgfqpoint{3.483533in}{5.812856in}}%
\pgfpathlineto{\pgfqpoint{3.606756in}{5.806622in}}%
\pgfpathlineto{\pgfqpoint{3.729979in}{5.800761in}}%
\pgfpathlineto{\pgfqpoint{3.853202in}{5.795223in}}%
\pgfpathlineto{\pgfqpoint{3.976425in}{5.789972in}}%
\pgfpathlineto{\pgfqpoint{4.099648in}{5.784973in}}%
\pgfpathlineto{\pgfqpoint{4.222871in}{5.780204in}}%
\pgfpathlineto{\pgfqpoint{4.346094in}{5.775643in}}%
\pgfpathlineto{\pgfqpoint{4.469317in}{5.771269in}}%
\pgfpathlineto{\pgfqpoint{4.592540in}{5.767067in}}%
\pgfpathlineto{\pgfqpoint{4.715763in}{5.763026in}}%
\pgfpathlineto{\pgfqpoint{4.838986in}{5.759133in}}%
\pgfpathlineto{\pgfqpoint{4.962209in}{5.755375in}}%
\pgfpathlineto{\pgfqpoint{5.085432in}{5.751745in}}%
\pgfpathlineto{\pgfqpoint{5.208655in}{5.748231in}}%
\pgfpathlineto{\pgfqpoint{5.331878in}{5.744823in}}%
\pgfpathlineto{\pgfqpoint{5.455101in}{5.741514in}}%
\pgfpathlineto{\pgfqpoint{5.578324in}{5.738298in}}%
\pgfpathlineto{\pgfqpoint{5.701547in}{5.735166in}}%
\pgfpathlineto{\pgfqpoint{5.824770in}{5.732114in}}%
\pgfpathlineto{\pgfqpoint{5.947993in}{5.729137in}}%
\pgfpathlineto{\pgfqpoint{6.071216in}{5.726232in}}%
\pgfpathlineto{\pgfqpoint{6.194439in}{5.723394in}}%
\pgfpathlineto{\pgfqpoint{6.317662in}{5.720619in}}%
\pgfpathlineto{\pgfqpoint{6.440885in}{5.717904in}}%
\pgfpathlineto{\pgfqpoint{6.564108in}{5.715245in}}%
\pgfpathlineto{\pgfqpoint{6.687330in}{5.712639in}}%
\pgfpathlineto{\pgfqpoint{6.810553in}{5.710086in}}%
\pgfpathlineto{\pgfqpoint{6.933776in}{5.707584in}}%
\pgfpathlineto{\pgfqpoint{7.056999in}{5.705132in}}%
\pgfpathlineto{\pgfqpoint{7.180222in}{5.702727in}}%
\pgfpathlineto{\pgfqpoint{7.365057in}{5.700370in}}%
\pgfpathlineto{\pgfqpoint{7.365057in}{6.156617in}}%
\pgfpathlineto{\pgfqpoint{7.365057in}{6.156617in}}%
\pgfpathlineto{\pgfqpoint{7.180222in}{6.154260in}}%
\pgfpathlineto{\pgfqpoint{7.056999in}{6.151855in}}%
\pgfpathlineto{\pgfqpoint{6.933776in}{6.149403in}}%
\pgfpathlineto{\pgfqpoint{6.810553in}{6.146901in}}%
\pgfpathlineto{\pgfqpoint{6.687330in}{6.144348in}}%
\pgfpathlineto{\pgfqpoint{6.564108in}{6.141742in}}%
\pgfpathlineto{\pgfqpoint{6.440885in}{6.139083in}}%
\pgfpathlineto{\pgfqpoint{6.317662in}{6.136367in}}%
\pgfpathlineto{\pgfqpoint{6.194439in}{6.133593in}}%
\pgfpathlineto{\pgfqpoint{6.071216in}{6.130755in}}%
\pgfpathlineto{\pgfqpoint{5.947993in}{6.127850in}}%
\pgfpathlineto{\pgfqpoint{5.824770in}{6.124873in}}%
\pgfpathlineto{\pgfqpoint{5.701547in}{6.121820in}}%
\pgfpathlineto{\pgfqpoint{5.578324in}{6.118689in}}%
\pgfpathlineto{\pgfqpoint{5.455101in}{6.115473in}}%
\pgfpathlineto{\pgfqpoint{5.331878in}{6.112164in}}%
\pgfpathlineto{\pgfqpoint{5.208655in}{6.108756in}}%
\pgfpathlineto{\pgfqpoint{5.085432in}{6.105242in}}%
\pgfpathlineto{\pgfqpoint{4.962209in}{6.101611in}}%
\pgfpathlineto{\pgfqpoint{4.838986in}{6.097854in}}%
\pgfpathlineto{\pgfqpoint{4.715763in}{6.093961in}}%
\pgfpathlineto{\pgfqpoint{4.592540in}{6.089919in}}%
\pgfpathlineto{\pgfqpoint{4.469317in}{6.085718in}}%
\pgfpathlineto{\pgfqpoint{4.346094in}{6.081344in}}%
\pgfpathlineto{\pgfqpoint{4.222871in}{6.076783in}}%
\pgfpathlineto{\pgfqpoint{4.099648in}{6.072014in}}%
\pgfpathlineto{\pgfqpoint{3.976425in}{6.067015in}}%
\pgfpathlineto{\pgfqpoint{3.853202in}{6.061763in}}%
\pgfpathlineto{\pgfqpoint{3.729979in}{6.056226in}}%
\pgfpathlineto{\pgfqpoint{3.606756in}{6.050365in}}%
\pgfpathlineto{\pgfqpoint{3.483533in}{6.044131in}}%
\pgfpathlineto{\pgfqpoint{3.360310in}{6.037452in}}%
\pgfpathlineto{\pgfqpoint{3.237087in}{6.030240in}}%
\pgfpathlineto{\pgfqpoint{3.113864in}{6.022372in}}%
\pgfpathlineto{\pgfqpoint{2.990641in}{6.013667in}}%
\pgfpathlineto{\pgfqpoint{2.867418in}{6.003840in}}%
\pgfpathlineto{\pgfqpoint{2.744195in}{5.992367in}}%
\pgfpathlineto{\pgfqpoint{2.620972in}{5.978122in}}%
\pgfpathlineto{\pgfqpoint{2.436138in}{5.957251in}}%
\pgfpathclose%
\pgfusepath{stroke,fill}%
\end{pgfscope}%
\begin{pgfscope}%
\pgfpathrectangle{\pgfqpoint{2.125000in}{5.660465in}}{\pgfqpoint{5.489583in}{0.877907in}}%
\pgfusepath{clip}%
\pgfsetbuttcap%
\pgfsetroundjoin%
\pgfsetlinewidth{1.505625pt}%
\definecolor{currentstroke}{rgb}{0.000000,0.000000,0.000000}%
\pgfsetstrokecolor{currentstroke}%
\pgfsetdash{}{0pt}%
\pgfpathmoveto{\pgfqpoint{2.374527in}{5.928493in}}%
\pgfpathlineto{\pgfqpoint{2.374527in}{6.498467in}}%
\pgfusepath{stroke}%
\end{pgfscope}%
\begin{pgfscope}%
\pgfpathrectangle{\pgfqpoint{2.125000in}{5.660465in}}{\pgfqpoint{5.489583in}{0.877907in}}%
\pgfusepath{clip}%
\pgfsetbuttcap%
\pgfsetroundjoin%
\pgfsetlinewidth{1.505625pt}%
\definecolor{currentstroke}{rgb}{0.000000,0.000000,0.000000}%
\pgfsetstrokecolor{currentstroke}%
\pgfsetdash{}{0pt}%
\pgfpathmoveto{\pgfqpoint{2.497749in}{5.928493in}}%
\pgfpathlineto{\pgfqpoint{2.497749in}{6.495353in}}%
\pgfusepath{stroke}%
\end{pgfscope}%
\begin{pgfscope}%
\pgfpathrectangle{\pgfqpoint{2.125000in}{5.660465in}}{\pgfqpoint{5.489583in}{0.877907in}}%
\pgfusepath{clip}%
\pgfsetbuttcap%
\pgfsetroundjoin%
\pgfsetlinewidth{1.505625pt}%
\definecolor{currentstroke}{rgb}{0.000000,0.000000,0.000000}%
\pgfsetstrokecolor{currentstroke}%
\pgfsetdash{}{0pt}%
\pgfpathmoveto{\pgfqpoint{2.620972in}{5.928493in}}%
\pgfpathlineto{\pgfqpoint{2.620972in}{6.492011in}}%
\pgfusepath{stroke}%
\end{pgfscope}%
\begin{pgfscope}%
\pgfpathrectangle{\pgfqpoint{2.125000in}{5.660465in}}{\pgfqpoint{5.489583in}{0.877907in}}%
\pgfusepath{clip}%
\pgfsetbuttcap%
\pgfsetroundjoin%
\pgfsetlinewidth{1.505625pt}%
\definecolor{currentstroke}{rgb}{0.000000,0.000000,0.000000}%
\pgfsetstrokecolor{currentstroke}%
\pgfsetdash{}{0pt}%
\pgfpathmoveto{\pgfqpoint{2.744195in}{5.928493in}}%
\pgfpathlineto{\pgfqpoint{2.744195in}{6.488593in}}%
\pgfusepath{stroke}%
\end{pgfscope}%
\begin{pgfscope}%
\pgfpathrectangle{\pgfqpoint{2.125000in}{5.660465in}}{\pgfqpoint{5.489583in}{0.877907in}}%
\pgfusepath{clip}%
\pgfsetbuttcap%
\pgfsetroundjoin%
\pgfsetlinewidth{1.505625pt}%
\definecolor{currentstroke}{rgb}{0.000000,0.000000,0.000000}%
\pgfsetstrokecolor{currentstroke}%
\pgfsetdash{}{0pt}%
\pgfpathmoveto{\pgfqpoint{2.867418in}{5.928493in}}%
\pgfpathlineto{\pgfqpoint{2.867418in}{6.485133in}}%
\pgfusepath{stroke}%
\end{pgfscope}%
\begin{pgfscope}%
\pgfpathrectangle{\pgfqpoint{2.125000in}{5.660465in}}{\pgfqpoint{5.489583in}{0.877907in}}%
\pgfusepath{clip}%
\pgfsetbuttcap%
\pgfsetroundjoin%
\pgfsetlinewidth{1.505625pt}%
\definecolor{currentstroke}{rgb}{0.000000,0.000000,0.000000}%
\pgfsetstrokecolor{currentstroke}%
\pgfsetdash{}{0pt}%
\pgfpathmoveto{\pgfqpoint{2.990641in}{5.928493in}}%
\pgfpathlineto{\pgfqpoint{2.990641in}{6.481784in}}%
\pgfusepath{stroke}%
\end{pgfscope}%
\begin{pgfscope}%
\pgfpathrectangle{\pgfqpoint{2.125000in}{5.660465in}}{\pgfqpoint{5.489583in}{0.877907in}}%
\pgfusepath{clip}%
\pgfsetbuttcap%
\pgfsetroundjoin%
\pgfsetlinewidth{1.505625pt}%
\definecolor{currentstroke}{rgb}{0.000000,0.000000,0.000000}%
\pgfsetstrokecolor{currentstroke}%
\pgfsetdash{}{0pt}%
\pgfpathmoveto{\pgfqpoint{3.113864in}{5.928493in}}%
\pgfpathlineto{\pgfqpoint{3.113864in}{6.478315in}}%
\pgfusepath{stroke}%
\end{pgfscope}%
\begin{pgfscope}%
\pgfpathrectangle{\pgfqpoint{2.125000in}{5.660465in}}{\pgfqpoint{5.489583in}{0.877907in}}%
\pgfusepath{clip}%
\pgfsetbuttcap%
\pgfsetroundjoin%
\pgfsetlinewidth{1.505625pt}%
\definecolor{currentstroke}{rgb}{0.000000,0.000000,0.000000}%
\pgfsetstrokecolor{currentstroke}%
\pgfsetdash{}{0pt}%
\pgfpathmoveto{\pgfqpoint{3.237087in}{5.928493in}}%
\pgfpathlineto{\pgfqpoint{3.237087in}{6.474795in}}%
\pgfusepath{stroke}%
\end{pgfscope}%
\begin{pgfscope}%
\pgfpathrectangle{\pgfqpoint{2.125000in}{5.660465in}}{\pgfqpoint{5.489583in}{0.877907in}}%
\pgfusepath{clip}%
\pgfsetbuttcap%
\pgfsetroundjoin%
\pgfsetlinewidth{1.505625pt}%
\definecolor{currentstroke}{rgb}{0.000000,0.000000,0.000000}%
\pgfsetstrokecolor{currentstroke}%
\pgfsetdash{}{0pt}%
\pgfpathmoveto{\pgfqpoint{3.360310in}{5.928493in}}%
\pgfpathlineto{\pgfqpoint{3.360310in}{6.471287in}}%
\pgfusepath{stroke}%
\end{pgfscope}%
\begin{pgfscope}%
\pgfpathrectangle{\pgfqpoint{2.125000in}{5.660465in}}{\pgfqpoint{5.489583in}{0.877907in}}%
\pgfusepath{clip}%
\pgfsetbuttcap%
\pgfsetroundjoin%
\pgfsetlinewidth{1.505625pt}%
\definecolor{currentstroke}{rgb}{0.000000,0.000000,0.000000}%
\pgfsetstrokecolor{currentstroke}%
\pgfsetdash{}{0pt}%
\pgfpathmoveto{\pgfqpoint{3.483533in}{5.928493in}}%
\pgfpathlineto{\pgfqpoint{3.483533in}{6.467787in}}%
\pgfusepath{stroke}%
\end{pgfscope}%
\begin{pgfscope}%
\pgfpathrectangle{\pgfqpoint{2.125000in}{5.660465in}}{\pgfqpoint{5.489583in}{0.877907in}}%
\pgfusepath{clip}%
\pgfsetbuttcap%
\pgfsetroundjoin%
\pgfsetlinewidth{1.505625pt}%
\definecolor{currentstroke}{rgb}{0.000000,0.000000,0.000000}%
\pgfsetstrokecolor{currentstroke}%
\pgfsetdash{}{0pt}%
\pgfpathmoveto{\pgfqpoint{3.606756in}{5.928493in}}%
\pgfpathlineto{\pgfqpoint{3.606756in}{6.464512in}}%
\pgfusepath{stroke}%
\end{pgfscope}%
\begin{pgfscope}%
\pgfpathrectangle{\pgfqpoint{2.125000in}{5.660465in}}{\pgfqpoint{5.489583in}{0.877907in}}%
\pgfusepath{clip}%
\pgfsetbuttcap%
\pgfsetroundjoin%
\pgfsetlinewidth{1.505625pt}%
\definecolor{currentstroke}{rgb}{0.000000,0.000000,0.000000}%
\pgfsetstrokecolor{currentstroke}%
\pgfsetdash{}{0pt}%
\pgfpathmoveto{\pgfqpoint{3.729979in}{5.928493in}}%
\pgfpathlineto{\pgfqpoint{3.729979in}{6.461297in}}%
\pgfusepath{stroke}%
\end{pgfscope}%
\begin{pgfscope}%
\pgfpathrectangle{\pgfqpoint{2.125000in}{5.660465in}}{\pgfqpoint{5.489583in}{0.877907in}}%
\pgfusepath{clip}%
\pgfsetbuttcap%
\pgfsetroundjoin%
\pgfsetlinewidth{1.505625pt}%
\definecolor{currentstroke}{rgb}{0.000000,0.000000,0.000000}%
\pgfsetstrokecolor{currentstroke}%
\pgfsetdash{}{0pt}%
\pgfpathmoveto{\pgfqpoint{3.853202in}{5.928493in}}%
\pgfpathlineto{\pgfqpoint{3.853202in}{6.457978in}}%
\pgfusepath{stroke}%
\end{pgfscope}%
\begin{pgfscope}%
\pgfpathrectangle{\pgfqpoint{2.125000in}{5.660465in}}{\pgfqpoint{5.489583in}{0.877907in}}%
\pgfusepath{clip}%
\pgfsetbuttcap%
\pgfsetroundjoin%
\pgfsetlinewidth{1.505625pt}%
\definecolor{currentstroke}{rgb}{0.000000,0.000000,0.000000}%
\pgfsetstrokecolor{currentstroke}%
\pgfsetdash{}{0pt}%
\pgfpathmoveto{\pgfqpoint{3.976425in}{5.928493in}}%
\pgfpathlineto{\pgfqpoint{3.976425in}{6.454706in}}%
\pgfusepath{stroke}%
\end{pgfscope}%
\begin{pgfscope}%
\pgfpathrectangle{\pgfqpoint{2.125000in}{5.660465in}}{\pgfqpoint{5.489583in}{0.877907in}}%
\pgfusepath{clip}%
\pgfsetbuttcap%
\pgfsetroundjoin%
\pgfsetlinewidth{1.505625pt}%
\definecolor{currentstroke}{rgb}{0.000000,0.000000,0.000000}%
\pgfsetstrokecolor{currentstroke}%
\pgfsetdash{}{0pt}%
\pgfpathmoveto{\pgfqpoint{4.099648in}{5.928493in}}%
\pgfpathlineto{\pgfqpoint{4.099648in}{6.451288in}}%
\pgfusepath{stroke}%
\end{pgfscope}%
\begin{pgfscope}%
\pgfpathrectangle{\pgfqpoint{2.125000in}{5.660465in}}{\pgfqpoint{5.489583in}{0.877907in}}%
\pgfusepath{clip}%
\pgfsetbuttcap%
\pgfsetroundjoin%
\pgfsetlinewidth{1.505625pt}%
\definecolor{currentstroke}{rgb}{0.000000,0.000000,0.000000}%
\pgfsetstrokecolor{currentstroke}%
\pgfsetdash{}{0pt}%
\pgfpathmoveto{\pgfqpoint{4.222871in}{5.928493in}}%
\pgfpathlineto{\pgfqpoint{4.222871in}{6.447915in}}%
\pgfusepath{stroke}%
\end{pgfscope}%
\begin{pgfscope}%
\pgfpathrectangle{\pgfqpoint{2.125000in}{5.660465in}}{\pgfqpoint{5.489583in}{0.877907in}}%
\pgfusepath{clip}%
\pgfsetbuttcap%
\pgfsetroundjoin%
\pgfsetlinewidth{1.505625pt}%
\definecolor{currentstroke}{rgb}{0.000000,0.000000,0.000000}%
\pgfsetstrokecolor{currentstroke}%
\pgfsetdash{}{0pt}%
\pgfpathmoveto{\pgfqpoint{4.346094in}{5.928493in}}%
\pgfpathlineto{\pgfqpoint{4.346094in}{6.444638in}}%
\pgfusepath{stroke}%
\end{pgfscope}%
\begin{pgfscope}%
\pgfpathrectangle{\pgfqpoint{2.125000in}{5.660465in}}{\pgfqpoint{5.489583in}{0.877907in}}%
\pgfusepath{clip}%
\pgfsetbuttcap%
\pgfsetroundjoin%
\pgfsetlinewidth{1.505625pt}%
\definecolor{currentstroke}{rgb}{0.000000,0.000000,0.000000}%
\pgfsetstrokecolor{currentstroke}%
\pgfsetdash{}{0pt}%
\pgfpathmoveto{\pgfqpoint{4.469317in}{5.928493in}}%
\pgfpathlineto{\pgfqpoint{4.469317in}{6.441261in}}%
\pgfusepath{stroke}%
\end{pgfscope}%
\begin{pgfscope}%
\pgfpathrectangle{\pgfqpoint{2.125000in}{5.660465in}}{\pgfqpoint{5.489583in}{0.877907in}}%
\pgfusepath{clip}%
\pgfsetbuttcap%
\pgfsetroundjoin%
\pgfsetlinewidth{1.505625pt}%
\definecolor{currentstroke}{rgb}{0.000000,0.000000,0.000000}%
\pgfsetstrokecolor{currentstroke}%
\pgfsetdash{}{0pt}%
\pgfpathmoveto{\pgfqpoint{4.592540in}{5.928493in}}%
\pgfpathlineto{\pgfqpoint{4.592540in}{6.437869in}}%
\pgfusepath{stroke}%
\end{pgfscope}%
\begin{pgfscope}%
\pgfpathrectangle{\pgfqpoint{2.125000in}{5.660465in}}{\pgfqpoint{5.489583in}{0.877907in}}%
\pgfusepath{clip}%
\pgfsetbuttcap%
\pgfsetroundjoin%
\pgfsetlinewidth{1.505625pt}%
\definecolor{currentstroke}{rgb}{0.000000,0.000000,0.000000}%
\pgfsetstrokecolor{currentstroke}%
\pgfsetdash{}{0pt}%
\pgfpathmoveto{\pgfqpoint{4.715763in}{5.928493in}}%
\pgfpathlineto{\pgfqpoint{4.715763in}{6.434492in}}%
\pgfusepath{stroke}%
\end{pgfscope}%
\begin{pgfscope}%
\pgfpathrectangle{\pgfqpoint{2.125000in}{5.660465in}}{\pgfqpoint{5.489583in}{0.877907in}}%
\pgfusepath{clip}%
\pgfsetbuttcap%
\pgfsetroundjoin%
\pgfsetlinewidth{1.505625pt}%
\definecolor{currentstroke}{rgb}{0.000000,0.000000,0.000000}%
\pgfsetstrokecolor{currentstroke}%
\pgfsetdash{}{0pt}%
\pgfpathmoveto{\pgfqpoint{4.838986in}{5.928493in}}%
\pgfpathlineto{\pgfqpoint{4.838986in}{6.431243in}}%
\pgfusepath{stroke}%
\end{pgfscope}%
\begin{pgfscope}%
\pgfpathrectangle{\pgfqpoint{2.125000in}{5.660465in}}{\pgfqpoint{5.489583in}{0.877907in}}%
\pgfusepath{clip}%
\pgfsetbuttcap%
\pgfsetroundjoin%
\pgfsetlinewidth{1.505625pt}%
\definecolor{currentstroke}{rgb}{0.000000,0.000000,0.000000}%
\pgfsetstrokecolor{currentstroke}%
\pgfsetdash{}{0pt}%
\pgfpathmoveto{\pgfqpoint{4.962209in}{5.928493in}}%
\pgfpathlineto{\pgfqpoint{4.962209in}{6.427989in}}%
\pgfusepath{stroke}%
\end{pgfscope}%
\begin{pgfscope}%
\pgfpathrectangle{\pgfqpoint{2.125000in}{5.660465in}}{\pgfqpoint{5.489583in}{0.877907in}}%
\pgfusepath{clip}%
\pgfsetbuttcap%
\pgfsetroundjoin%
\pgfsetlinewidth{1.505625pt}%
\definecolor{currentstroke}{rgb}{0.000000,0.000000,0.000000}%
\pgfsetstrokecolor{currentstroke}%
\pgfsetdash{}{0pt}%
\pgfpathmoveto{\pgfqpoint{5.085432in}{5.928493in}}%
\pgfpathlineto{\pgfqpoint{5.085432in}{6.424890in}}%
\pgfusepath{stroke}%
\end{pgfscope}%
\begin{pgfscope}%
\pgfpathrectangle{\pgfqpoint{2.125000in}{5.660465in}}{\pgfqpoint{5.489583in}{0.877907in}}%
\pgfusepath{clip}%
\pgfsetbuttcap%
\pgfsetroundjoin%
\pgfsetlinewidth{1.505625pt}%
\definecolor{currentstroke}{rgb}{0.000000,0.000000,0.000000}%
\pgfsetstrokecolor{currentstroke}%
\pgfsetdash{}{0pt}%
\pgfpathmoveto{\pgfqpoint{5.208655in}{5.928493in}}%
\pgfpathlineto{\pgfqpoint{5.208655in}{6.422039in}}%
\pgfusepath{stroke}%
\end{pgfscope}%
\begin{pgfscope}%
\pgfpathrectangle{\pgfqpoint{2.125000in}{5.660465in}}{\pgfqpoint{5.489583in}{0.877907in}}%
\pgfusepath{clip}%
\pgfsetbuttcap%
\pgfsetroundjoin%
\pgfsetlinewidth{1.505625pt}%
\definecolor{currentstroke}{rgb}{0.000000,0.000000,0.000000}%
\pgfsetstrokecolor{currentstroke}%
\pgfsetdash{}{0pt}%
\pgfpathmoveto{\pgfqpoint{5.331878in}{5.928493in}}%
\pgfpathlineto{\pgfqpoint{5.331878in}{6.419286in}}%
\pgfusepath{stroke}%
\end{pgfscope}%
\begin{pgfscope}%
\pgfpathrectangle{\pgfqpoint{2.125000in}{5.660465in}}{\pgfqpoint{5.489583in}{0.877907in}}%
\pgfusepath{clip}%
\pgfsetbuttcap%
\pgfsetroundjoin%
\pgfsetlinewidth{1.505625pt}%
\definecolor{currentstroke}{rgb}{0.000000,0.000000,0.000000}%
\pgfsetstrokecolor{currentstroke}%
\pgfsetdash{}{0pt}%
\pgfpathmoveto{\pgfqpoint{5.455101in}{5.928493in}}%
\pgfpathlineto{\pgfqpoint{5.455101in}{6.416623in}}%
\pgfusepath{stroke}%
\end{pgfscope}%
\begin{pgfscope}%
\pgfpathrectangle{\pgfqpoint{2.125000in}{5.660465in}}{\pgfqpoint{5.489583in}{0.877907in}}%
\pgfusepath{clip}%
\pgfsetbuttcap%
\pgfsetroundjoin%
\pgfsetlinewidth{1.505625pt}%
\definecolor{currentstroke}{rgb}{0.000000,0.000000,0.000000}%
\pgfsetstrokecolor{currentstroke}%
\pgfsetdash{}{0pt}%
\pgfpathmoveto{\pgfqpoint{5.578324in}{5.928493in}}%
\pgfpathlineto{\pgfqpoint{5.578324in}{6.414160in}}%
\pgfusepath{stroke}%
\end{pgfscope}%
\begin{pgfscope}%
\pgfpathrectangle{\pgfqpoint{2.125000in}{5.660465in}}{\pgfqpoint{5.489583in}{0.877907in}}%
\pgfusepath{clip}%
\pgfsetbuttcap%
\pgfsetroundjoin%
\pgfsetlinewidth{1.505625pt}%
\definecolor{currentstroke}{rgb}{0.000000,0.000000,0.000000}%
\pgfsetstrokecolor{currentstroke}%
\pgfsetdash{}{0pt}%
\pgfpathmoveto{\pgfqpoint{5.701547in}{5.928493in}}%
\pgfpathlineto{\pgfqpoint{5.701547in}{6.411886in}}%
\pgfusepath{stroke}%
\end{pgfscope}%
\begin{pgfscope}%
\pgfpathrectangle{\pgfqpoint{2.125000in}{5.660465in}}{\pgfqpoint{5.489583in}{0.877907in}}%
\pgfusepath{clip}%
\pgfsetbuttcap%
\pgfsetroundjoin%
\pgfsetlinewidth{1.505625pt}%
\definecolor{currentstroke}{rgb}{0.000000,0.000000,0.000000}%
\pgfsetstrokecolor{currentstroke}%
\pgfsetdash{}{0pt}%
\pgfpathmoveto{\pgfqpoint{5.824770in}{5.928493in}}%
\pgfpathlineto{\pgfqpoint{5.824770in}{6.409473in}}%
\pgfusepath{stroke}%
\end{pgfscope}%
\begin{pgfscope}%
\pgfpathrectangle{\pgfqpoint{2.125000in}{5.660465in}}{\pgfqpoint{5.489583in}{0.877907in}}%
\pgfusepath{clip}%
\pgfsetbuttcap%
\pgfsetroundjoin%
\pgfsetlinewidth{1.505625pt}%
\definecolor{currentstroke}{rgb}{0.000000,0.000000,0.000000}%
\pgfsetstrokecolor{currentstroke}%
\pgfsetdash{}{0pt}%
\pgfpathmoveto{\pgfqpoint{5.947993in}{5.928493in}}%
\pgfpathlineto{\pgfqpoint{5.947993in}{6.407214in}}%
\pgfusepath{stroke}%
\end{pgfscope}%
\begin{pgfscope}%
\pgfpathrectangle{\pgfqpoint{2.125000in}{5.660465in}}{\pgfqpoint{5.489583in}{0.877907in}}%
\pgfusepath{clip}%
\pgfsetbuttcap%
\pgfsetroundjoin%
\pgfsetlinewidth{1.505625pt}%
\definecolor{currentstroke}{rgb}{0.000000,0.000000,0.000000}%
\pgfsetstrokecolor{currentstroke}%
\pgfsetdash{}{0pt}%
\pgfpathmoveto{\pgfqpoint{6.071216in}{5.928493in}}%
\pgfpathlineto{\pgfqpoint{6.071216in}{6.405011in}}%
\pgfusepath{stroke}%
\end{pgfscope}%
\begin{pgfscope}%
\pgfpathrectangle{\pgfqpoint{2.125000in}{5.660465in}}{\pgfqpoint{5.489583in}{0.877907in}}%
\pgfusepath{clip}%
\pgfsetbuttcap%
\pgfsetroundjoin%
\pgfsetlinewidth{1.505625pt}%
\definecolor{currentstroke}{rgb}{0.000000,0.000000,0.000000}%
\pgfsetstrokecolor{currentstroke}%
\pgfsetdash{}{0pt}%
\pgfpathmoveto{\pgfqpoint{6.194439in}{5.928493in}}%
\pgfpathlineto{\pgfqpoint{6.194439in}{6.402876in}}%
\pgfusepath{stroke}%
\end{pgfscope}%
\begin{pgfscope}%
\pgfpathrectangle{\pgfqpoint{2.125000in}{5.660465in}}{\pgfqpoint{5.489583in}{0.877907in}}%
\pgfusepath{clip}%
\pgfsetbuttcap%
\pgfsetroundjoin%
\pgfsetlinewidth{1.505625pt}%
\definecolor{currentstroke}{rgb}{0.000000,0.000000,0.000000}%
\pgfsetstrokecolor{currentstroke}%
\pgfsetdash{}{0pt}%
\pgfpathmoveto{\pgfqpoint{6.317662in}{5.928493in}}%
\pgfpathlineto{\pgfqpoint{6.317662in}{6.400928in}}%
\pgfusepath{stroke}%
\end{pgfscope}%
\begin{pgfscope}%
\pgfpathrectangle{\pgfqpoint{2.125000in}{5.660465in}}{\pgfqpoint{5.489583in}{0.877907in}}%
\pgfusepath{clip}%
\pgfsetbuttcap%
\pgfsetroundjoin%
\pgfsetlinewidth{1.505625pt}%
\definecolor{currentstroke}{rgb}{0.000000,0.000000,0.000000}%
\pgfsetstrokecolor{currentstroke}%
\pgfsetdash{}{0pt}%
\pgfpathmoveto{\pgfqpoint{6.440885in}{5.928493in}}%
\pgfpathlineto{\pgfqpoint{6.440885in}{6.398997in}}%
\pgfusepath{stroke}%
\end{pgfscope}%
\begin{pgfscope}%
\pgfpathrectangle{\pgfqpoint{2.125000in}{5.660465in}}{\pgfqpoint{5.489583in}{0.877907in}}%
\pgfusepath{clip}%
\pgfsetbuttcap%
\pgfsetroundjoin%
\pgfsetlinewidth{1.505625pt}%
\definecolor{currentstroke}{rgb}{0.000000,0.000000,0.000000}%
\pgfsetstrokecolor{currentstroke}%
\pgfsetdash{}{0pt}%
\pgfpathmoveto{\pgfqpoint{6.564108in}{5.928493in}}%
\pgfpathlineto{\pgfqpoint{6.564108in}{6.397107in}}%
\pgfusepath{stroke}%
\end{pgfscope}%
\begin{pgfscope}%
\pgfpathrectangle{\pgfqpoint{2.125000in}{5.660465in}}{\pgfqpoint{5.489583in}{0.877907in}}%
\pgfusepath{clip}%
\pgfsetbuttcap%
\pgfsetroundjoin%
\pgfsetlinewidth{1.505625pt}%
\definecolor{currentstroke}{rgb}{0.000000,0.000000,0.000000}%
\pgfsetstrokecolor{currentstroke}%
\pgfsetdash{}{0pt}%
\pgfpathmoveto{\pgfqpoint{6.687330in}{5.928493in}}%
\pgfpathlineto{\pgfqpoint{6.687330in}{6.395153in}}%
\pgfusepath{stroke}%
\end{pgfscope}%
\begin{pgfscope}%
\pgfpathrectangle{\pgfqpoint{2.125000in}{5.660465in}}{\pgfqpoint{5.489583in}{0.877907in}}%
\pgfusepath{clip}%
\pgfsetbuttcap%
\pgfsetroundjoin%
\pgfsetlinewidth{1.505625pt}%
\definecolor{currentstroke}{rgb}{0.000000,0.000000,0.000000}%
\pgfsetstrokecolor{currentstroke}%
\pgfsetdash{}{0pt}%
\pgfpathmoveto{\pgfqpoint{6.810553in}{5.928493in}}%
\pgfpathlineto{\pgfqpoint{6.810553in}{6.393100in}}%
\pgfusepath{stroke}%
\end{pgfscope}%
\begin{pgfscope}%
\pgfpathrectangle{\pgfqpoint{2.125000in}{5.660465in}}{\pgfqpoint{5.489583in}{0.877907in}}%
\pgfusepath{clip}%
\pgfsetbuttcap%
\pgfsetroundjoin%
\pgfsetlinewidth{1.505625pt}%
\definecolor{currentstroke}{rgb}{0.000000,0.000000,0.000000}%
\pgfsetstrokecolor{currentstroke}%
\pgfsetdash{}{0pt}%
\pgfpathmoveto{\pgfqpoint{6.933776in}{5.928493in}}%
\pgfpathlineto{\pgfqpoint{6.933776in}{6.391090in}}%
\pgfusepath{stroke}%
\end{pgfscope}%
\begin{pgfscope}%
\pgfpathrectangle{\pgfqpoint{2.125000in}{5.660465in}}{\pgfqpoint{5.489583in}{0.877907in}}%
\pgfusepath{clip}%
\pgfsetbuttcap%
\pgfsetroundjoin%
\pgfsetlinewidth{1.505625pt}%
\definecolor{currentstroke}{rgb}{0.000000,0.000000,0.000000}%
\pgfsetstrokecolor{currentstroke}%
\pgfsetdash{}{0pt}%
\pgfpathmoveto{\pgfqpoint{7.056999in}{5.928493in}}%
\pgfpathlineto{\pgfqpoint{7.056999in}{6.389099in}}%
\pgfusepath{stroke}%
\end{pgfscope}%
\begin{pgfscope}%
\pgfpathrectangle{\pgfqpoint{2.125000in}{5.660465in}}{\pgfqpoint{5.489583in}{0.877907in}}%
\pgfusepath{clip}%
\pgfsetbuttcap%
\pgfsetroundjoin%
\pgfsetlinewidth{1.505625pt}%
\definecolor{currentstroke}{rgb}{0.000000,0.000000,0.000000}%
\pgfsetstrokecolor{currentstroke}%
\pgfsetdash{}{0pt}%
\pgfpathmoveto{\pgfqpoint{7.180222in}{5.928493in}}%
\pgfpathlineto{\pgfqpoint{7.180222in}{6.386862in}}%
\pgfusepath{stroke}%
\end{pgfscope}%
\begin{pgfscope}%
\pgfpathrectangle{\pgfqpoint{2.125000in}{5.660465in}}{\pgfqpoint{5.489583in}{0.877907in}}%
\pgfusepath{clip}%
\pgfsetbuttcap%
\pgfsetroundjoin%
\pgfsetlinewidth{1.505625pt}%
\definecolor{currentstroke}{rgb}{0.000000,0.000000,0.000000}%
\pgfsetstrokecolor{currentstroke}%
\pgfsetdash{}{0pt}%
\pgfpathmoveto{\pgfqpoint{7.303445in}{5.928493in}}%
\pgfpathlineto{\pgfqpoint{7.303445in}{6.384645in}}%
\pgfusepath{stroke}%
\end{pgfscope}%
\begin{pgfscope}%
\pgfpathrectangle{\pgfqpoint{2.125000in}{5.660465in}}{\pgfqpoint{5.489583in}{0.877907in}}%
\pgfusepath{clip}%
\pgfsetroundcap%
\pgfsetroundjoin%
\pgfsetlinewidth{1.505625pt}%
\definecolor{currentstroke}{rgb}{0.121569,0.466667,0.705882}%
\pgfsetstrokecolor{currentstroke}%
\pgfsetdash{}{0pt}%
\pgfpathmoveto{\pgfqpoint{2.125000in}{5.928493in}}%
\pgfpathlineto{\pgfqpoint{7.614583in}{5.928493in}}%
\pgfusepath{stroke}%
\end{pgfscope}%
\begin{pgfscope}%
\pgfpathrectangle{\pgfqpoint{2.125000in}{5.660465in}}{\pgfqpoint{5.489583in}{0.877907in}}%
\pgfusepath{clip}%
\pgfsetbuttcap%
\pgfsetroundjoin%
\definecolor{currentfill}{rgb}{0.121569,0.466667,0.705882}%
\pgfsetfillcolor{currentfill}%
\pgfsetlinewidth{1.003750pt}%
\definecolor{currentstroke}{rgb}{0.121569,0.466667,0.705882}%
\pgfsetstrokecolor{currentstroke}%
\pgfsetdash{}{0pt}%
\pgfsys@defobject{currentmarker}{\pgfqpoint{-0.034722in}{-0.034722in}}{\pgfqpoint{0.034722in}{0.034722in}}{%
\pgfpathmoveto{\pgfqpoint{0.000000in}{-0.034722in}}%
\pgfpathcurveto{\pgfqpoint{0.009208in}{-0.034722in}}{\pgfqpoint{0.018041in}{-0.031064in}}{\pgfqpoint{0.024552in}{-0.024552in}}%
\pgfpathcurveto{\pgfqpoint{0.031064in}{-0.018041in}}{\pgfqpoint{0.034722in}{-0.009208in}}{\pgfqpoint{0.034722in}{0.000000in}}%
\pgfpathcurveto{\pgfqpoint{0.034722in}{0.009208in}}{\pgfqpoint{0.031064in}{0.018041in}}{\pgfqpoint{0.024552in}{0.024552in}}%
\pgfpathcurveto{\pgfqpoint{0.018041in}{0.031064in}}{\pgfqpoint{0.009208in}{0.034722in}}{\pgfqpoint{0.000000in}{0.034722in}}%
\pgfpathcurveto{\pgfqpoint{-0.009208in}{0.034722in}}{\pgfqpoint{-0.018041in}{0.031064in}}{\pgfqpoint{-0.024552in}{0.024552in}}%
\pgfpathcurveto{\pgfqpoint{-0.031064in}{0.018041in}}{\pgfqpoint{-0.034722in}{0.009208in}}{\pgfqpoint{-0.034722in}{0.000000in}}%
\pgfpathcurveto{\pgfqpoint{-0.034722in}{-0.009208in}}{\pgfqpoint{-0.031064in}{-0.018041in}}{\pgfqpoint{-0.024552in}{-0.024552in}}%
\pgfpathcurveto{\pgfqpoint{-0.018041in}{-0.031064in}}{\pgfqpoint{-0.009208in}{-0.034722in}}{\pgfqpoint{0.000000in}{-0.034722in}}%
\pgfpathclose%
\pgfusepath{stroke,fill}%
}%
\begin{pgfscope}%
\pgfsys@transformshift{2.374527in}{6.498467in}%
\pgfsys@useobject{currentmarker}{}%
\end{pgfscope}%
\begin{pgfscope}%
\pgfsys@transformshift{2.497749in}{6.495353in}%
\pgfsys@useobject{currentmarker}{}%
\end{pgfscope}%
\begin{pgfscope}%
\pgfsys@transformshift{2.620972in}{6.492011in}%
\pgfsys@useobject{currentmarker}{}%
\end{pgfscope}%
\begin{pgfscope}%
\pgfsys@transformshift{2.744195in}{6.488593in}%
\pgfsys@useobject{currentmarker}{}%
\end{pgfscope}%
\begin{pgfscope}%
\pgfsys@transformshift{2.867418in}{6.485133in}%
\pgfsys@useobject{currentmarker}{}%
\end{pgfscope}%
\begin{pgfscope}%
\pgfsys@transformshift{2.990641in}{6.481784in}%
\pgfsys@useobject{currentmarker}{}%
\end{pgfscope}%
\begin{pgfscope}%
\pgfsys@transformshift{3.113864in}{6.478315in}%
\pgfsys@useobject{currentmarker}{}%
\end{pgfscope}%
\begin{pgfscope}%
\pgfsys@transformshift{3.237087in}{6.474795in}%
\pgfsys@useobject{currentmarker}{}%
\end{pgfscope}%
\begin{pgfscope}%
\pgfsys@transformshift{3.360310in}{6.471287in}%
\pgfsys@useobject{currentmarker}{}%
\end{pgfscope}%
\begin{pgfscope}%
\pgfsys@transformshift{3.483533in}{6.467787in}%
\pgfsys@useobject{currentmarker}{}%
\end{pgfscope}%
\begin{pgfscope}%
\pgfsys@transformshift{3.606756in}{6.464512in}%
\pgfsys@useobject{currentmarker}{}%
\end{pgfscope}%
\begin{pgfscope}%
\pgfsys@transformshift{3.729979in}{6.461297in}%
\pgfsys@useobject{currentmarker}{}%
\end{pgfscope}%
\begin{pgfscope}%
\pgfsys@transformshift{3.853202in}{6.457978in}%
\pgfsys@useobject{currentmarker}{}%
\end{pgfscope}%
\begin{pgfscope}%
\pgfsys@transformshift{3.976425in}{6.454706in}%
\pgfsys@useobject{currentmarker}{}%
\end{pgfscope}%
\begin{pgfscope}%
\pgfsys@transformshift{4.099648in}{6.451288in}%
\pgfsys@useobject{currentmarker}{}%
\end{pgfscope}%
\begin{pgfscope}%
\pgfsys@transformshift{4.222871in}{6.447915in}%
\pgfsys@useobject{currentmarker}{}%
\end{pgfscope}%
\begin{pgfscope}%
\pgfsys@transformshift{4.346094in}{6.444638in}%
\pgfsys@useobject{currentmarker}{}%
\end{pgfscope}%
\begin{pgfscope}%
\pgfsys@transformshift{4.469317in}{6.441261in}%
\pgfsys@useobject{currentmarker}{}%
\end{pgfscope}%
\begin{pgfscope}%
\pgfsys@transformshift{4.592540in}{6.437869in}%
\pgfsys@useobject{currentmarker}{}%
\end{pgfscope}%
\begin{pgfscope}%
\pgfsys@transformshift{4.715763in}{6.434492in}%
\pgfsys@useobject{currentmarker}{}%
\end{pgfscope}%
\begin{pgfscope}%
\pgfsys@transformshift{4.838986in}{6.431243in}%
\pgfsys@useobject{currentmarker}{}%
\end{pgfscope}%
\begin{pgfscope}%
\pgfsys@transformshift{4.962209in}{6.427989in}%
\pgfsys@useobject{currentmarker}{}%
\end{pgfscope}%
\begin{pgfscope}%
\pgfsys@transformshift{5.085432in}{6.424890in}%
\pgfsys@useobject{currentmarker}{}%
\end{pgfscope}%
\begin{pgfscope}%
\pgfsys@transformshift{5.208655in}{6.422039in}%
\pgfsys@useobject{currentmarker}{}%
\end{pgfscope}%
\begin{pgfscope}%
\pgfsys@transformshift{5.331878in}{6.419286in}%
\pgfsys@useobject{currentmarker}{}%
\end{pgfscope}%
\begin{pgfscope}%
\pgfsys@transformshift{5.455101in}{6.416623in}%
\pgfsys@useobject{currentmarker}{}%
\end{pgfscope}%
\begin{pgfscope}%
\pgfsys@transformshift{5.578324in}{6.414160in}%
\pgfsys@useobject{currentmarker}{}%
\end{pgfscope}%
\begin{pgfscope}%
\pgfsys@transformshift{5.701547in}{6.411886in}%
\pgfsys@useobject{currentmarker}{}%
\end{pgfscope}%
\begin{pgfscope}%
\pgfsys@transformshift{5.824770in}{6.409473in}%
\pgfsys@useobject{currentmarker}{}%
\end{pgfscope}%
\begin{pgfscope}%
\pgfsys@transformshift{5.947993in}{6.407214in}%
\pgfsys@useobject{currentmarker}{}%
\end{pgfscope}%
\begin{pgfscope}%
\pgfsys@transformshift{6.071216in}{6.405011in}%
\pgfsys@useobject{currentmarker}{}%
\end{pgfscope}%
\begin{pgfscope}%
\pgfsys@transformshift{6.194439in}{6.402876in}%
\pgfsys@useobject{currentmarker}{}%
\end{pgfscope}%
\begin{pgfscope}%
\pgfsys@transformshift{6.317662in}{6.400928in}%
\pgfsys@useobject{currentmarker}{}%
\end{pgfscope}%
\begin{pgfscope}%
\pgfsys@transformshift{6.440885in}{6.398997in}%
\pgfsys@useobject{currentmarker}{}%
\end{pgfscope}%
\begin{pgfscope}%
\pgfsys@transformshift{6.564108in}{6.397107in}%
\pgfsys@useobject{currentmarker}{}%
\end{pgfscope}%
\begin{pgfscope}%
\pgfsys@transformshift{6.687330in}{6.395153in}%
\pgfsys@useobject{currentmarker}{}%
\end{pgfscope}%
\begin{pgfscope}%
\pgfsys@transformshift{6.810553in}{6.393100in}%
\pgfsys@useobject{currentmarker}{}%
\end{pgfscope}%
\begin{pgfscope}%
\pgfsys@transformshift{6.933776in}{6.391090in}%
\pgfsys@useobject{currentmarker}{}%
\end{pgfscope}%
\begin{pgfscope}%
\pgfsys@transformshift{7.056999in}{6.389099in}%
\pgfsys@useobject{currentmarker}{}%
\end{pgfscope}%
\begin{pgfscope}%
\pgfsys@transformshift{7.180222in}{6.386862in}%
\pgfsys@useobject{currentmarker}{}%
\end{pgfscope}%
\begin{pgfscope}%
\pgfsys@transformshift{7.303445in}{6.384645in}%
\pgfsys@useobject{currentmarker}{}%
\end{pgfscope}%
\end{pgfscope}%
\begin{pgfscope}%
\pgfsetrectcap%
\pgfsetmiterjoin%
\pgfsetlinewidth{0.803000pt}%
\definecolor{currentstroke}{rgb}{1.000000,1.000000,1.000000}%
\pgfsetstrokecolor{currentstroke}%
\pgfsetdash{}{0pt}%
\pgfpathmoveto{\pgfqpoint{2.125000in}{5.660465in}}%
\pgfpathlineto{\pgfqpoint{2.125000in}{6.538372in}}%
\pgfusepath{stroke}%
\end{pgfscope}%
\begin{pgfscope}%
\pgfsetrectcap%
\pgfsetmiterjoin%
\pgfsetlinewidth{0.803000pt}%
\definecolor{currentstroke}{rgb}{1.000000,1.000000,1.000000}%
\pgfsetstrokecolor{currentstroke}%
\pgfsetdash{}{0pt}%
\pgfpathmoveto{\pgfqpoint{7.614583in}{5.660465in}}%
\pgfpathlineto{\pgfqpoint{7.614583in}{6.538372in}}%
\pgfusepath{stroke}%
\end{pgfscope}%
\begin{pgfscope}%
\pgfsetrectcap%
\pgfsetmiterjoin%
\pgfsetlinewidth{0.803000pt}%
\definecolor{currentstroke}{rgb}{1.000000,1.000000,1.000000}%
\pgfsetstrokecolor{currentstroke}%
\pgfsetdash{}{0pt}%
\pgfpathmoveto{\pgfqpoint{2.125000in}{5.660465in}}%
\pgfpathlineto{\pgfqpoint{7.614583in}{5.660465in}}%
\pgfusepath{stroke}%
\end{pgfscope}%
\begin{pgfscope}%
\pgfsetrectcap%
\pgfsetmiterjoin%
\pgfsetlinewidth{0.803000pt}%
\definecolor{currentstroke}{rgb}{1.000000,1.000000,1.000000}%
\pgfsetstrokecolor{currentstroke}%
\pgfsetdash{}{0pt}%
\pgfpathmoveto{\pgfqpoint{2.125000in}{6.538372in}}%
\pgfpathlineto{\pgfqpoint{7.614583in}{6.538372in}}%
\pgfusepath{stroke}%
\end{pgfscope}%
\begin{pgfscope}%
\definecolor{textcolor}{rgb}{0.150000,0.150000,0.150000}%
\pgfsetstrokecolor{textcolor}%
\pgfsetfillcolor{textcolor}%
\pgftext[x=4.869792in,y=6.621705in,,base]{\color{textcolor}\rmfamily\fontsize{16.800000}{20.160000}\selectfont Autocorrelation}%
\end{pgfscope}%
\begin{pgfscope}%
\pgfsetbuttcap%
\pgfsetmiterjoin%
\definecolor{currentfill}{rgb}{0.917647,0.917647,0.949020}%
\pgfsetfillcolor{currentfill}%
\pgfsetlinewidth{0.000000pt}%
\definecolor{currentstroke}{rgb}{0.000000,0.000000,0.000000}%
\pgfsetstrokecolor{currentstroke}%
\pgfsetstrokeopacity{0.000000}%
\pgfsetdash{}{0pt}%
\pgfpathmoveto{\pgfqpoint{9.810417in}{5.660465in}}%
\pgfpathlineto{\pgfqpoint{15.300000in}{5.660465in}}%
\pgfpathlineto{\pgfqpoint{15.300000in}{6.538372in}}%
\pgfpathlineto{\pgfqpoint{9.810417in}{6.538372in}}%
\pgfpathclose%
\pgfusepath{fill}%
\end{pgfscope}%
\begin{pgfscope}%
\pgfpathrectangle{\pgfqpoint{9.810417in}{5.660465in}}{\pgfqpoint{5.489583in}{0.877907in}}%
\pgfusepath{clip}%
\pgfsetroundcap%
\pgfsetroundjoin%
\pgfsetlinewidth{0.803000pt}%
\definecolor{currentstroke}{rgb}{1.000000,1.000000,1.000000}%
\pgfsetstrokecolor{currentstroke}%
\pgfsetdash{}{0pt}%
\pgfpathmoveto{\pgfqpoint{10.059943in}{5.660465in}}%
\pgfpathlineto{\pgfqpoint{10.059943in}{6.538372in}}%
\pgfusepath{stroke}%
\end{pgfscope}%
\begin{pgfscope}%
\definecolor{textcolor}{rgb}{0.150000,0.150000,0.150000}%
\pgfsetstrokecolor{textcolor}%
\pgfsetfillcolor{textcolor}%
\pgftext[x=10.059943in,y=5.563243in,,top]{\color{textcolor}\rmfamily\fontsize{14.000000}{16.800000}\selectfont 0}%
\end{pgfscope}%
\begin{pgfscope}%
\pgfpathrectangle{\pgfqpoint{9.810417in}{5.660465in}}{\pgfqpoint{5.489583in}{0.877907in}}%
\pgfusepath{clip}%
\pgfsetroundcap%
\pgfsetroundjoin%
\pgfsetlinewidth{0.803000pt}%
\definecolor{currentstroke}{rgb}{1.000000,1.000000,1.000000}%
\pgfsetstrokecolor{currentstroke}%
\pgfsetdash{}{0pt}%
\pgfpathmoveto{\pgfqpoint{10.676058in}{5.660465in}}%
\pgfpathlineto{\pgfqpoint{10.676058in}{6.538372in}}%
\pgfusepath{stroke}%
\end{pgfscope}%
\begin{pgfscope}%
\definecolor{textcolor}{rgb}{0.150000,0.150000,0.150000}%
\pgfsetstrokecolor{textcolor}%
\pgfsetfillcolor{textcolor}%
\pgftext[x=10.676058in,y=5.563243in,,top]{\color{textcolor}\rmfamily\fontsize{14.000000}{16.800000}\selectfont 5}%
\end{pgfscope}%
\begin{pgfscope}%
\pgfpathrectangle{\pgfqpoint{9.810417in}{5.660465in}}{\pgfqpoint{5.489583in}{0.877907in}}%
\pgfusepath{clip}%
\pgfsetroundcap%
\pgfsetroundjoin%
\pgfsetlinewidth{0.803000pt}%
\definecolor{currentstroke}{rgb}{1.000000,1.000000,1.000000}%
\pgfsetstrokecolor{currentstroke}%
\pgfsetdash{}{0pt}%
\pgfpathmoveto{\pgfqpoint{11.292173in}{5.660465in}}%
\pgfpathlineto{\pgfqpoint{11.292173in}{6.538372in}}%
\pgfusepath{stroke}%
\end{pgfscope}%
\begin{pgfscope}%
\definecolor{textcolor}{rgb}{0.150000,0.150000,0.150000}%
\pgfsetstrokecolor{textcolor}%
\pgfsetfillcolor{textcolor}%
\pgftext[x=11.292173in,y=5.563243in,,top]{\color{textcolor}\rmfamily\fontsize{14.000000}{16.800000}\selectfont 10}%
\end{pgfscope}%
\begin{pgfscope}%
\pgfpathrectangle{\pgfqpoint{9.810417in}{5.660465in}}{\pgfqpoint{5.489583in}{0.877907in}}%
\pgfusepath{clip}%
\pgfsetroundcap%
\pgfsetroundjoin%
\pgfsetlinewidth{0.803000pt}%
\definecolor{currentstroke}{rgb}{1.000000,1.000000,1.000000}%
\pgfsetstrokecolor{currentstroke}%
\pgfsetdash{}{0pt}%
\pgfpathmoveto{\pgfqpoint{11.908288in}{5.660465in}}%
\pgfpathlineto{\pgfqpoint{11.908288in}{6.538372in}}%
\pgfusepath{stroke}%
\end{pgfscope}%
\begin{pgfscope}%
\definecolor{textcolor}{rgb}{0.150000,0.150000,0.150000}%
\pgfsetstrokecolor{textcolor}%
\pgfsetfillcolor{textcolor}%
\pgftext[x=11.908288in,y=5.563243in,,top]{\color{textcolor}\rmfamily\fontsize{14.000000}{16.800000}\selectfont 15}%
\end{pgfscope}%
\begin{pgfscope}%
\pgfpathrectangle{\pgfqpoint{9.810417in}{5.660465in}}{\pgfqpoint{5.489583in}{0.877907in}}%
\pgfusepath{clip}%
\pgfsetroundcap%
\pgfsetroundjoin%
\pgfsetlinewidth{0.803000pt}%
\definecolor{currentstroke}{rgb}{1.000000,1.000000,1.000000}%
\pgfsetstrokecolor{currentstroke}%
\pgfsetdash{}{0pt}%
\pgfpathmoveto{\pgfqpoint{12.524403in}{5.660465in}}%
\pgfpathlineto{\pgfqpoint{12.524403in}{6.538372in}}%
\pgfusepath{stroke}%
\end{pgfscope}%
\begin{pgfscope}%
\definecolor{textcolor}{rgb}{0.150000,0.150000,0.150000}%
\pgfsetstrokecolor{textcolor}%
\pgfsetfillcolor{textcolor}%
\pgftext[x=12.524403in,y=5.563243in,,top]{\color{textcolor}\rmfamily\fontsize{14.000000}{16.800000}\selectfont 20}%
\end{pgfscope}%
\begin{pgfscope}%
\pgfpathrectangle{\pgfqpoint{9.810417in}{5.660465in}}{\pgfqpoint{5.489583in}{0.877907in}}%
\pgfusepath{clip}%
\pgfsetroundcap%
\pgfsetroundjoin%
\pgfsetlinewidth{0.803000pt}%
\definecolor{currentstroke}{rgb}{1.000000,1.000000,1.000000}%
\pgfsetstrokecolor{currentstroke}%
\pgfsetdash{}{0pt}%
\pgfpathmoveto{\pgfqpoint{13.140517in}{5.660465in}}%
\pgfpathlineto{\pgfqpoint{13.140517in}{6.538372in}}%
\pgfusepath{stroke}%
\end{pgfscope}%
\begin{pgfscope}%
\definecolor{textcolor}{rgb}{0.150000,0.150000,0.150000}%
\pgfsetstrokecolor{textcolor}%
\pgfsetfillcolor{textcolor}%
\pgftext[x=13.140517in,y=5.563243in,,top]{\color{textcolor}\rmfamily\fontsize{14.000000}{16.800000}\selectfont 25}%
\end{pgfscope}%
\begin{pgfscope}%
\pgfpathrectangle{\pgfqpoint{9.810417in}{5.660465in}}{\pgfqpoint{5.489583in}{0.877907in}}%
\pgfusepath{clip}%
\pgfsetroundcap%
\pgfsetroundjoin%
\pgfsetlinewidth{0.803000pt}%
\definecolor{currentstroke}{rgb}{1.000000,1.000000,1.000000}%
\pgfsetstrokecolor{currentstroke}%
\pgfsetdash{}{0pt}%
\pgfpathmoveto{\pgfqpoint{13.756632in}{5.660465in}}%
\pgfpathlineto{\pgfqpoint{13.756632in}{6.538372in}}%
\pgfusepath{stroke}%
\end{pgfscope}%
\begin{pgfscope}%
\definecolor{textcolor}{rgb}{0.150000,0.150000,0.150000}%
\pgfsetstrokecolor{textcolor}%
\pgfsetfillcolor{textcolor}%
\pgftext[x=13.756632in,y=5.563243in,,top]{\color{textcolor}\rmfamily\fontsize{14.000000}{16.800000}\selectfont 30}%
\end{pgfscope}%
\begin{pgfscope}%
\pgfpathrectangle{\pgfqpoint{9.810417in}{5.660465in}}{\pgfqpoint{5.489583in}{0.877907in}}%
\pgfusepath{clip}%
\pgfsetroundcap%
\pgfsetroundjoin%
\pgfsetlinewidth{0.803000pt}%
\definecolor{currentstroke}{rgb}{1.000000,1.000000,1.000000}%
\pgfsetstrokecolor{currentstroke}%
\pgfsetdash{}{0pt}%
\pgfpathmoveto{\pgfqpoint{14.372747in}{5.660465in}}%
\pgfpathlineto{\pgfqpoint{14.372747in}{6.538372in}}%
\pgfusepath{stroke}%
\end{pgfscope}%
\begin{pgfscope}%
\definecolor{textcolor}{rgb}{0.150000,0.150000,0.150000}%
\pgfsetstrokecolor{textcolor}%
\pgfsetfillcolor{textcolor}%
\pgftext[x=14.372747in,y=5.563243in,,top]{\color{textcolor}\rmfamily\fontsize{14.000000}{16.800000}\selectfont 35}%
\end{pgfscope}%
\begin{pgfscope}%
\pgfpathrectangle{\pgfqpoint{9.810417in}{5.660465in}}{\pgfqpoint{5.489583in}{0.877907in}}%
\pgfusepath{clip}%
\pgfsetroundcap%
\pgfsetroundjoin%
\pgfsetlinewidth{0.803000pt}%
\definecolor{currentstroke}{rgb}{1.000000,1.000000,1.000000}%
\pgfsetstrokecolor{currentstroke}%
\pgfsetdash{}{0pt}%
\pgfpathmoveto{\pgfqpoint{14.988862in}{5.660465in}}%
\pgfpathlineto{\pgfqpoint{14.988862in}{6.538372in}}%
\pgfusepath{stroke}%
\end{pgfscope}%
\begin{pgfscope}%
\definecolor{textcolor}{rgb}{0.150000,0.150000,0.150000}%
\pgfsetstrokecolor{textcolor}%
\pgfsetfillcolor{textcolor}%
\pgftext[x=14.988862in,y=5.563243in,,top]{\color{textcolor}\rmfamily\fontsize{14.000000}{16.800000}\selectfont 40}%
\end{pgfscope}%
\begin{pgfscope}%
\pgfpathrectangle{\pgfqpoint{9.810417in}{5.660465in}}{\pgfqpoint{5.489583in}{0.877907in}}%
\pgfusepath{clip}%
\pgfsetroundcap%
\pgfsetroundjoin%
\pgfsetlinewidth{0.803000pt}%
\definecolor{currentstroke}{rgb}{1.000000,1.000000,1.000000}%
\pgfsetstrokecolor{currentstroke}%
\pgfsetdash{}{0pt}%
\pgfpathmoveto{\pgfqpoint{9.810417in}{5.738704in}}%
\pgfpathlineto{\pgfqpoint{15.300000in}{5.738704in}}%
\pgfusepath{stroke}%
\end{pgfscope}%
\begin{pgfscope}%
\definecolor{textcolor}{rgb}{0.150000,0.150000,0.150000}%
\pgfsetstrokecolor{textcolor}%
\pgfsetfillcolor{textcolor}%
\pgftext[x=9.589483in,y=5.664838in,left,base]{\color{textcolor}\rmfamily\fontsize{14.000000}{16.800000}\selectfont 0}%
\end{pgfscope}%
\begin{pgfscope}%
\pgfpathrectangle{\pgfqpoint{9.810417in}{5.660465in}}{\pgfqpoint{5.489583in}{0.877907in}}%
\pgfusepath{clip}%
\pgfsetroundcap%
\pgfsetroundjoin%
\pgfsetlinewidth{0.803000pt}%
\definecolor{currentstroke}{rgb}{1.000000,1.000000,1.000000}%
\pgfsetstrokecolor{currentstroke}%
\pgfsetdash{}{0pt}%
\pgfpathmoveto{\pgfqpoint{9.810417in}{6.498467in}}%
\pgfpathlineto{\pgfqpoint{15.300000in}{6.498467in}}%
\pgfusepath{stroke}%
\end{pgfscope}%
\begin{pgfscope}%
\definecolor{textcolor}{rgb}{0.150000,0.150000,0.150000}%
\pgfsetstrokecolor{textcolor}%
\pgfsetfillcolor{textcolor}%
\pgftext[x=9.589483in,y=6.424601in,left,base]{\color{textcolor}\rmfamily\fontsize{14.000000}{16.800000}\selectfont 1}%
\end{pgfscope}%
\begin{pgfscope}%
\pgfpathrectangle{\pgfqpoint{9.810417in}{5.660465in}}{\pgfqpoint{5.489583in}{0.877907in}}%
\pgfusepath{clip}%
\pgfsetbuttcap%
\pgfsetroundjoin%
\definecolor{currentfill}{rgb}{0.121569,0.466667,0.705882}%
\pgfsetfillcolor{currentfill}%
\pgfsetfillopacity{0.250000}%
\pgfsetlinewidth{1.003750pt}%
\definecolor{currentstroke}{rgb}{1.000000,1.000000,1.000000}%
\pgfsetstrokecolor{currentstroke}%
\pgfsetstrokeopacity{0.250000}%
\pgfsetdash{}{0pt}%
\pgfpathmoveto{\pgfqpoint{10.121555in}{5.777038in}}%
\pgfpathlineto{\pgfqpoint{10.121555in}{5.700370in}}%
\pgfpathlineto{\pgfqpoint{10.306389in}{5.700370in}}%
\pgfpathlineto{\pgfqpoint{10.429612in}{5.700370in}}%
\pgfpathlineto{\pgfqpoint{10.552835in}{5.700370in}}%
\pgfpathlineto{\pgfqpoint{10.676058in}{5.700370in}}%
\pgfpathlineto{\pgfqpoint{10.799281in}{5.700370in}}%
\pgfpathlineto{\pgfqpoint{10.922504in}{5.700370in}}%
\pgfpathlineto{\pgfqpoint{11.045727in}{5.700370in}}%
\pgfpathlineto{\pgfqpoint{11.168950in}{5.700370in}}%
\pgfpathlineto{\pgfqpoint{11.292173in}{5.700370in}}%
\pgfpathlineto{\pgfqpoint{11.415396in}{5.700370in}}%
\pgfpathlineto{\pgfqpoint{11.538619in}{5.700370in}}%
\pgfpathlineto{\pgfqpoint{11.661842in}{5.700370in}}%
\pgfpathlineto{\pgfqpoint{11.785065in}{5.700370in}}%
\pgfpathlineto{\pgfqpoint{11.908288in}{5.700370in}}%
\pgfpathlineto{\pgfqpoint{12.031511in}{5.700370in}}%
\pgfpathlineto{\pgfqpoint{12.154734in}{5.700370in}}%
\pgfpathlineto{\pgfqpoint{12.277957in}{5.700370in}}%
\pgfpathlineto{\pgfqpoint{12.401180in}{5.700370in}}%
\pgfpathlineto{\pgfqpoint{12.524403in}{5.700370in}}%
\pgfpathlineto{\pgfqpoint{12.647626in}{5.700370in}}%
\pgfpathlineto{\pgfqpoint{12.770849in}{5.700370in}}%
\pgfpathlineto{\pgfqpoint{12.894072in}{5.700370in}}%
\pgfpathlineto{\pgfqpoint{13.017294in}{5.700370in}}%
\pgfpathlineto{\pgfqpoint{13.140517in}{5.700370in}}%
\pgfpathlineto{\pgfqpoint{13.263740in}{5.700370in}}%
\pgfpathlineto{\pgfqpoint{13.386963in}{5.700370in}}%
\pgfpathlineto{\pgfqpoint{13.510186in}{5.700370in}}%
\pgfpathlineto{\pgfqpoint{13.633409in}{5.700370in}}%
\pgfpathlineto{\pgfqpoint{13.756632in}{5.700370in}}%
\pgfpathlineto{\pgfqpoint{13.879855in}{5.700370in}}%
\pgfpathlineto{\pgfqpoint{14.003078in}{5.700370in}}%
\pgfpathlineto{\pgfqpoint{14.126301in}{5.700370in}}%
\pgfpathlineto{\pgfqpoint{14.249524in}{5.700370in}}%
\pgfpathlineto{\pgfqpoint{14.372747in}{5.700370in}}%
\pgfpathlineto{\pgfqpoint{14.495970in}{5.700370in}}%
\pgfpathlineto{\pgfqpoint{14.619193in}{5.700370in}}%
\pgfpathlineto{\pgfqpoint{14.742416in}{5.700370in}}%
\pgfpathlineto{\pgfqpoint{14.865639in}{5.700370in}}%
\pgfpathlineto{\pgfqpoint{15.050473in}{5.700370in}}%
\pgfpathlineto{\pgfqpoint{15.050473in}{5.777038in}}%
\pgfpathlineto{\pgfqpoint{15.050473in}{5.777038in}}%
\pgfpathlineto{\pgfqpoint{14.865639in}{5.777038in}}%
\pgfpathlineto{\pgfqpoint{14.742416in}{5.777038in}}%
\pgfpathlineto{\pgfqpoint{14.619193in}{5.777038in}}%
\pgfpathlineto{\pgfqpoint{14.495970in}{5.777038in}}%
\pgfpathlineto{\pgfqpoint{14.372747in}{5.777038in}}%
\pgfpathlineto{\pgfqpoint{14.249524in}{5.777038in}}%
\pgfpathlineto{\pgfqpoint{14.126301in}{5.777038in}}%
\pgfpathlineto{\pgfqpoint{14.003078in}{5.777038in}}%
\pgfpathlineto{\pgfqpoint{13.879855in}{5.777038in}}%
\pgfpathlineto{\pgfqpoint{13.756632in}{5.777038in}}%
\pgfpathlineto{\pgfqpoint{13.633409in}{5.777038in}}%
\pgfpathlineto{\pgfqpoint{13.510186in}{5.777038in}}%
\pgfpathlineto{\pgfqpoint{13.386963in}{5.777038in}}%
\pgfpathlineto{\pgfqpoint{13.263740in}{5.777038in}}%
\pgfpathlineto{\pgfqpoint{13.140517in}{5.777038in}}%
\pgfpathlineto{\pgfqpoint{13.017294in}{5.777038in}}%
\pgfpathlineto{\pgfqpoint{12.894072in}{5.777038in}}%
\pgfpathlineto{\pgfqpoint{12.770849in}{5.777038in}}%
\pgfpathlineto{\pgfqpoint{12.647626in}{5.777038in}}%
\pgfpathlineto{\pgfqpoint{12.524403in}{5.777038in}}%
\pgfpathlineto{\pgfqpoint{12.401180in}{5.777038in}}%
\pgfpathlineto{\pgfqpoint{12.277957in}{5.777038in}}%
\pgfpathlineto{\pgfqpoint{12.154734in}{5.777038in}}%
\pgfpathlineto{\pgfqpoint{12.031511in}{5.777038in}}%
\pgfpathlineto{\pgfqpoint{11.908288in}{5.777038in}}%
\pgfpathlineto{\pgfqpoint{11.785065in}{5.777038in}}%
\pgfpathlineto{\pgfqpoint{11.661842in}{5.777038in}}%
\pgfpathlineto{\pgfqpoint{11.538619in}{5.777038in}}%
\pgfpathlineto{\pgfqpoint{11.415396in}{5.777038in}}%
\pgfpathlineto{\pgfqpoint{11.292173in}{5.777038in}}%
\pgfpathlineto{\pgfqpoint{11.168950in}{5.777038in}}%
\pgfpathlineto{\pgfqpoint{11.045727in}{5.777038in}}%
\pgfpathlineto{\pgfqpoint{10.922504in}{5.777038in}}%
\pgfpathlineto{\pgfqpoint{10.799281in}{5.777038in}}%
\pgfpathlineto{\pgfqpoint{10.676058in}{5.777038in}}%
\pgfpathlineto{\pgfqpoint{10.552835in}{5.777038in}}%
\pgfpathlineto{\pgfqpoint{10.429612in}{5.777038in}}%
\pgfpathlineto{\pgfqpoint{10.306389in}{5.777038in}}%
\pgfpathlineto{\pgfqpoint{10.121555in}{5.777038in}}%
\pgfpathclose%
\pgfusepath{stroke,fill}%
\end{pgfscope}%
\begin{pgfscope}%
\pgfpathrectangle{\pgfqpoint{9.810417in}{5.660465in}}{\pgfqpoint{5.489583in}{0.877907in}}%
\pgfusepath{clip}%
\pgfsetbuttcap%
\pgfsetroundjoin%
\pgfsetlinewidth{1.505625pt}%
\definecolor{currentstroke}{rgb}{0.000000,0.000000,0.000000}%
\pgfsetstrokecolor{currentstroke}%
\pgfsetdash{}{0pt}%
\pgfpathmoveto{\pgfqpoint{10.059943in}{5.738704in}}%
\pgfpathlineto{\pgfqpoint{10.059943in}{6.498467in}}%
\pgfusepath{stroke}%
\end{pgfscope}%
\begin{pgfscope}%
\pgfpathrectangle{\pgfqpoint{9.810417in}{5.660465in}}{\pgfqpoint{5.489583in}{0.877907in}}%
\pgfusepath{clip}%
\pgfsetbuttcap%
\pgfsetroundjoin%
\pgfsetlinewidth{1.505625pt}%
\definecolor{currentstroke}{rgb}{0.000000,0.000000,0.000000}%
\pgfsetstrokecolor{currentstroke}%
\pgfsetdash{}{0pt}%
\pgfpathmoveto{\pgfqpoint{10.183166in}{5.738704in}}%
\pgfpathlineto{\pgfqpoint{10.183166in}{6.494818in}}%
\pgfusepath{stroke}%
\end{pgfscope}%
\begin{pgfscope}%
\pgfpathrectangle{\pgfqpoint{9.810417in}{5.660465in}}{\pgfqpoint{5.489583in}{0.877907in}}%
\pgfusepath{clip}%
\pgfsetbuttcap%
\pgfsetroundjoin%
\pgfsetlinewidth{1.505625pt}%
\definecolor{currentstroke}{rgb}{0.000000,0.000000,0.000000}%
\pgfsetstrokecolor{currentstroke}%
\pgfsetdash{}{0pt}%
\pgfpathmoveto{\pgfqpoint{10.306389in}{5.738704in}}%
\pgfpathlineto{\pgfqpoint{10.306389in}{5.704530in}}%
\pgfusepath{stroke}%
\end{pgfscope}%
\begin{pgfscope}%
\pgfpathrectangle{\pgfqpoint{9.810417in}{5.660465in}}{\pgfqpoint{5.489583in}{0.877907in}}%
\pgfusepath{clip}%
\pgfsetbuttcap%
\pgfsetroundjoin%
\pgfsetlinewidth{1.505625pt}%
\definecolor{currentstroke}{rgb}{0.000000,0.000000,0.000000}%
\pgfsetstrokecolor{currentstroke}%
\pgfsetdash{}{0pt}%
\pgfpathmoveto{\pgfqpoint{10.429612in}{5.738704in}}%
\pgfpathlineto{\pgfqpoint{10.429612in}{5.726947in}}%
\pgfusepath{stroke}%
\end{pgfscope}%
\begin{pgfscope}%
\pgfpathrectangle{\pgfqpoint{9.810417in}{5.660465in}}{\pgfqpoint{5.489583in}{0.877907in}}%
\pgfusepath{clip}%
\pgfsetbuttcap%
\pgfsetroundjoin%
\pgfsetlinewidth{1.505625pt}%
\definecolor{currentstroke}{rgb}{0.000000,0.000000,0.000000}%
\pgfsetstrokecolor{currentstroke}%
\pgfsetdash{}{0pt}%
\pgfpathmoveto{\pgfqpoint{10.552835in}{5.738704in}}%
\pgfpathlineto{\pgfqpoint{10.552835in}{5.731162in}}%
\pgfusepath{stroke}%
\end{pgfscope}%
\begin{pgfscope}%
\pgfpathrectangle{\pgfqpoint{9.810417in}{5.660465in}}{\pgfqpoint{5.489583in}{0.877907in}}%
\pgfusepath{clip}%
\pgfsetbuttcap%
\pgfsetroundjoin%
\pgfsetlinewidth{1.505625pt}%
\definecolor{currentstroke}{rgb}{0.000000,0.000000,0.000000}%
\pgfsetstrokecolor{currentstroke}%
\pgfsetdash{}{0pt}%
\pgfpathmoveto{\pgfqpoint{10.676058in}{5.738704in}}%
\pgfpathlineto{\pgfqpoint{10.676058in}{5.752214in}}%
\pgfusepath{stroke}%
\end{pgfscope}%
\begin{pgfscope}%
\pgfpathrectangle{\pgfqpoint{9.810417in}{5.660465in}}{\pgfqpoint{5.489583in}{0.877907in}}%
\pgfusepath{clip}%
\pgfsetbuttcap%
\pgfsetroundjoin%
\pgfsetlinewidth{1.505625pt}%
\definecolor{currentstroke}{rgb}{0.000000,0.000000,0.000000}%
\pgfsetstrokecolor{currentstroke}%
\pgfsetdash{}{0pt}%
\pgfpathmoveto{\pgfqpoint{10.799281in}{5.738704in}}%
\pgfpathlineto{\pgfqpoint{10.799281in}{5.718116in}}%
\pgfusepath{stroke}%
\end{pgfscope}%
\begin{pgfscope}%
\pgfpathrectangle{\pgfqpoint{9.810417in}{5.660465in}}{\pgfqpoint{5.489583in}{0.877907in}}%
\pgfusepath{clip}%
\pgfsetbuttcap%
\pgfsetroundjoin%
\pgfsetlinewidth{1.505625pt}%
\definecolor{currentstroke}{rgb}{0.000000,0.000000,0.000000}%
\pgfsetstrokecolor{currentstroke}%
\pgfsetdash{}{0pt}%
\pgfpathmoveto{\pgfqpoint{10.922504in}{5.738704in}}%
\pgfpathlineto{\pgfqpoint{10.922504in}{5.730071in}}%
\pgfusepath{stroke}%
\end{pgfscope}%
\begin{pgfscope}%
\pgfpathrectangle{\pgfqpoint{9.810417in}{5.660465in}}{\pgfqpoint{5.489583in}{0.877907in}}%
\pgfusepath{clip}%
\pgfsetbuttcap%
\pgfsetroundjoin%
\pgfsetlinewidth{1.505625pt}%
\definecolor{currentstroke}{rgb}{0.000000,0.000000,0.000000}%
\pgfsetstrokecolor{currentstroke}%
\pgfsetdash{}{0pt}%
\pgfpathmoveto{\pgfqpoint{11.045727in}{5.738704in}}%
\pgfpathlineto{\pgfqpoint{11.045727in}{5.738515in}}%
\pgfusepath{stroke}%
\end{pgfscope}%
\begin{pgfscope}%
\pgfpathrectangle{\pgfqpoint{9.810417in}{5.660465in}}{\pgfqpoint{5.489583in}{0.877907in}}%
\pgfusepath{clip}%
\pgfsetbuttcap%
\pgfsetroundjoin%
\pgfsetlinewidth{1.505625pt}%
\definecolor{currentstroke}{rgb}{0.000000,0.000000,0.000000}%
\pgfsetstrokecolor{currentstroke}%
\pgfsetdash{}{0pt}%
\pgfpathmoveto{\pgfqpoint{11.168950in}{5.738704in}}%
\pgfpathlineto{\pgfqpoint{11.168950in}{5.737618in}}%
\pgfusepath{stroke}%
\end{pgfscope}%
\begin{pgfscope}%
\pgfpathrectangle{\pgfqpoint{9.810417in}{5.660465in}}{\pgfqpoint{5.489583in}{0.877907in}}%
\pgfusepath{clip}%
\pgfsetbuttcap%
\pgfsetroundjoin%
\pgfsetlinewidth{1.505625pt}%
\definecolor{currentstroke}{rgb}{0.000000,0.000000,0.000000}%
\pgfsetstrokecolor{currentstroke}%
\pgfsetdash{}{0pt}%
\pgfpathmoveto{\pgfqpoint{11.292173in}{5.738704in}}%
\pgfpathlineto{\pgfqpoint{11.292173in}{5.766994in}}%
\pgfusepath{stroke}%
\end{pgfscope}%
\begin{pgfscope}%
\pgfpathrectangle{\pgfqpoint{9.810417in}{5.660465in}}{\pgfqpoint{5.489583in}{0.877907in}}%
\pgfusepath{clip}%
\pgfsetbuttcap%
\pgfsetroundjoin%
\pgfsetlinewidth{1.505625pt}%
\definecolor{currentstroke}{rgb}{0.000000,0.000000,0.000000}%
\pgfsetstrokecolor{currentstroke}%
\pgfsetdash{}{0pt}%
\pgfpathmoveto{\pgfqpoint{11.415396in}{5.738704in}}%
\pgfpathlineto{\pgfqpoint{11.415396in}{5.742105in}}%
\pgfusepath{stroke}%
\end{pgfscope}%
\begin{pgfscope}%
\pgfpathrectangle{\pgfqpoint{9.810417in}{5.660465in}}{\pgfqpoint{5.489583in}{0.877907in}}%
\pgfusepath{clip}%
\pgfsetbuttcap%
\pgfsetroundjoin%
\pgfsetlinewidth{1.505625pt}%
\definecolor{currentstroke}{rgb}{0.000000,0.000000,0.000000}%
\pgfsetstrokecolor{currentstroke}%
\pgfsetdash{}{0pt}%
\pgfpathmoveto{\pgfqpoint{11.538619in}{5.738704in}}%
\pgfpathlineto{\pgfqpoint{11.538619in}{5.720343in}}%
\pgfusepath{stroke}%
\end{pgfscope}%
\begin{pgfscope}%
\pgfpathrectangle{\pgfqpoint{9.810417in}{5.660465in}}{\pgfqpoint{5.489583in}{0.877907in}}%
\pgfusepath{clip}%
\pgfsetbuttcap%
\pgfsetroundjoin%
\pgfsetlinewidth{1.505625pt}%
\definecolor{currentstroke}{rgb}{0.000000,0.000000,0.000000}%
\pgfsetstrokecolor{currentstroke}%
\pgfsetdash{}{0pt}%
\pgfpathmoveto{\pgfqpoint{11.661842in}{5.738704in}}%
\pgfpathlineto{\pgfqpoint{11.661842in}{5.743483in}}%
\pgfusepath{stroke}%
\end{pgfscope}%
\begin{pgfscope}%
\pgfpathrectangle{\pgfqpoint{9.810417in}{5.660465in}}{\pgfqpoint{5.489583in}{0.877907in}}%
\pgfusepath{clip}%
\pgfsetbuttcap%
\pgfsetroundjoin%
\pgfsetlinewidth{1.505625pt}%
\definecolor{currentstroke}{rgb}{0.000000,0.000000,0.000000}%
\pgfsetstrokecolor{currentstroke}%
\pgfsetdash{}{0pt}%
\pgfpathmoveto{\pgfqpoint{11.785065in}{5.738704in}}%
\pgfpathlineto{\pgfqpoint{11.785065in}{5.716537in}}%
\pgfusepath{stroke}%
\end{pgfscope}%
\begin{pgfscope}%
\pgfpathrectangle{\pgfqpoint{9.810417in}{5.660465in}}{\pgfqpoint{5.489583in}{0.877907in}}%
\pgfusepath{clip}%
\pgfsetbuttcap%
\pgfsetroundjoin%
\pgfsetlinewidth{1.505625pt}%
\definecolor{currentstroke}{rgb}{0.000000,0.000000,0.000000}%
\pgfsetstrokecolor{currentstroke}%
\pgfsetdash{}{0pt}%
\pgfpathmoveto{\pgfqpoint{11.908288in}{5.738704in}}%
\pgfpathlineto{\pgfqpoint{11.908288in}{5.742957in}}%
\pgfusepath{stroke}%
\end{pgfscope}%
\begin{pgfscope}%
\pgfpathrectangle{\pgfqpoint{9.810417in}{5.660465in}}{\pgfqpoint{5.489583in}{0.877907in}}%
\pgfusepath{clip}%
\pgfsetbuttcap%
\pgfsetroundjoin%
\pgfsetlinewidth{1.505625pt}%
\definecolor{currentstroke}{rgb}{0.000000,0.000000,0.000000}%
\pgfsetstrokecolor{currentstroke}%
\pgfsetdash{}{0pt}%
\pgfpathmoveto{\pgfqpoint{12.031511in}{5.738704in}}%
\pgfpathlineto{\pgfqpoint{12.031511in}{5.748041in}}%
\pgfusepath{stroke}%
\end{pgfscope}%
\begin{pgfscope}%
\pgfpathrectangle{\pgfqpoint{9.810417in}{5.660465in}}{\pgfqpoint{5.489583in}{0.877907in}}%
\pgfusepath{clip}%
\pgfsetbuttcap%
\pgfsetroundjoin%
\pgfsetlinewidth{1.505625pt}%
\definecolor{currentstroke}{rgb}{0.000000,0.000000,0.000000}%
\pgfsetstrokecolor{currentstroke}%
\pgfsetdash{}{0pt}%
\pgfpathmoveto{\pgfqpoint{12.154734in}{5.738704in}}%
\pgfpathlineto{\pgfqpoint{12.154734in}{5.722072in}}%
\pgfusepath{stroke}%
\end{pgfscope}%
\begin{pgfscope}%
\pgfpathrectangle{\pgfqpoint{9.810417in}{5.660465in}}{\pgfqpoint{5.489583in}{0.877907in}}%
\pgfusepath{clip}%
\pgfsetbuttcap%
\pgfsetroundjoin%
\pgfsetlinewidth{1.505625pt}%
\definecolor{currentstroke}{rgb}{0.000000,0.000000,0.000000}%
\pgfsetstrokecolor{currentstroke}%
\pgfsetdash{}{0pt}%
\pgfpathmoveto{\pgfqpoint{12.277957in}{5.738704in}}%
\pgfpathlineto{\pgfqpoint{12.277957in}{5.733855in}}%
\pgfusepath{stroke}%
\end{pgfscope}%
\begin{pgfscope}%
\pgfpathrectangle{\pgfqpoint{9.810417in}{5.660465in}}{\pgfqpoint{5.489583in}{0.877907in}}%
\pgfusepath{clip}%
\pgfsetbuttcap%
\pgfsetroundjoin%
\pgfsetlinewidth{1.505625pt}%
\definecolor{currentstroke}{rgb}{0.000000,0.000000,0.000000}%
\pgfsetstrokecolor{currentstroke}%
\pgfsetdash{}{0pt}%
\pgfpathmoveto{\pgfqpoint{12.401180in}{5.738704in}}%
\pgfpathlineto{\pgfqpoint{12.401180in}{5.740818in}}%
\pgfusepath{stroke}%
\end{pgfscope}%
\begin{pgfscope}%
\pgfpathrectangle{\pgfqpoint{9.810417in}{5.660465in}}{\pgfqpoint{5.489583in}{0.877907in}}%
\pgfusepath{clip}%
\pgfsetbuttcap%
\pgfsetroundjoin%
\pgfsetlinewidth{1.505625pt}%
\definecolor{currentstroke}{rgb}{0.000000,0.000000,0.000000}%
\pgfsetstrokecolor{currentstroke}%
\pgfsetdash{}{0pt}%
\pgfpathmoveto{\pgfqpoint{12.524403in}{5.738704in}}%
\pgfpathlineto{\pgfqpoint{12.524403in}{5.755199in}}%
\pgfusepath{stroke}%
\end{pgfscope}%
\begin{pgfscope}%
\pgfpathrectangle{\pgfqpoint{9.810417in}{5.660465in}}{\pgfqpoint{5.489583in}{0.877907in}}%
\pgfusepath{clip}%
\pgfsetbuttcap%
\pgfsetroundjoin%
\pgfsetlinewidth{1.505625pt}%
\definecolor{currentstroke}{rgb}{0.000000,0.000000,0.000000}%
\pgfsetstrokecolor{currentstroke}%
\pgfsetdash{}{0pt}%
\pgfpathmoveto{\pgfqpoint{12.647626in}{5.738704in}}%
\pgfpathlineto{\pgfqpoint{12.647626in}{5.731481in}}%
\pgfusepath{stroke}%
\end{pgfscope}%
\begin{pgfscope}%
\pgfpathrectangle{\pgfqpoint{9.810417in}{5.660465in}}{\pgfqpoint{5.489583in}{0.877907in}}%
\pgfusepath{clip}%
\pgfsetbuttcap%
\pgfsetroundjoin%
\pgfsetlinewidth{1.505625pt}%
\definecolor{currentstroke}{rgb}{0.000000,0.000000,0.000000}%
\pgfsetstrokecolor{currentstroke}%
\pgfsetdash{}{0pt}%
\pgfpathmoveto{\pgfqpoint{12.770849in}{5.738704in}}%
\pgfpathlineto{\pgfqpoint{12.770849in}{5.758217in}}%
\pgfusepath{stroke}%
\end{pgfscope}%
\begin{pgfscope}%
\pgfpathrectangle{\pgfqpoint{9.810417in}{5.660465in}}{\pgfqpoint{5.489583in}{0.877907in}}%
\pgfusepath{clip}%
\pgfsetbuttcap%
\pgfsetroundjoin%
\pgfsetlinewidth{1.505625pt}%
\definecolor{currentstroke}{rgb}{0.000000,0.000000,0.000000}%
\pgfsetstrokecolor{currentstroke}%
\pgfsetdash{}{0pt}%
\pgfpathmoveto{\pgfqpoint{12.894072in}{5.738704in}}%
\pgfpathlineto{\pgfqpoint{12.894072in}{5.768832in}}%
\pgfusepath{stroke}%
\end{pgfscope}%
\begin{pgfscope}%
\pgfpathrectangle{\pgfqpoint{9.810417in}{5.660465in}}{\pgfqpoint{5.489583in}{0.877907in}}%
\pgfusepath{clip}%
\pgfsetbuttcap%
\pgfsetroundjoin%
\pgfsetlinewidth{1.505625pt}%
\definecolor{currentstroke}{rgb}{0.000000,0.000000,0.000000}%
\pgfsetstrokecolor{currentstroke}%
\pgfsetdash{}{0pt}%
\pgfpathmoveto{\pgfqpoint{13.017294in}{5.738704in}}%
\pgfpathlineto{\pgfqpoint{13.017294in}{5.747465in}}%
\pgfusepath{stroke}%
\end{pgfscope}%
\begin{pgfscope}%
\pgfpathrectangle{\pgfqpoint{9.810417in}{5.660465in}}{\pgfqpoint{5.489583in}{0.877907in}}%
\pgfusepath{clip}%
\pgfsetbuttcap%
\pgfsetroundjoin%
\pgfsetlinewidth{1.505625pt}%
\definecolor{currentstroke}{rgb}{0.000000,0.000000,0.000000}%
\pgfsetstrokecolor{currentstroke}%
\pgfsetdash{}{0pt}%
\pgfpathmoveto{\pgfqpoint{13.140517in}{5.738704in}}%
\pgfpathlineto{\pgfqpoint{13.140517in}{5.748047in}}%
\pgfusepath{stroke}%
\end{pgfscope}%
\begin{pgfscope}%
\pgfpathrectangle{\pgfqpoint{9.810417in}{5.660465in}}{\pgfqpoint{5.489583in}{0.877907in}}%
\pgfusepath{clip}%
\pgfsetbuttcap%
\pgfsetroundjoin%
\pgfsetlinewidth{1.505625pt}%
\definecolor{currentstroke}{rgb}{0.000000,0.000000,0.000000}%
\pgfsetstrokecolor{currentstroke}%
\pgfsetdash{}{0pt}%
\pgfpathmoveto{\pgfqpoint{13.263740in}{5.738704in}}%
\pgfpathlineto{\pgfqpoint{13.263740in}{5.762313in}}%
\pgfusepath{stroke}%
\end{pgfscope}%
\begin{pgfscope}%
\pgfpathrectangle{\pgfqpoint{9.810417in}{5.660465in}}{\pgfqpoint{5.489583in}{0.877907in}}%
\pgfusepath{clip}%
\pgfsetbuttcap%
\pgfsetroundjoin%
\pgfsetlinewidth{1.505625pt}%
\definecolor{currentstroke}{rgb}{0.000000,0.000000,0.000000}%
\pgfsetstrokecolor{currentstroke}%
\pgfsetdash{}{0pt}%
\pgfpathmoveto{\pgfqpoint{13.386963in}{5.738704in}}%
\pgfpathlineto{\pgfqpoint{13.386963in}{5.761192in}}%
\pgfusepath{stroke}%
\end{pgfscope}%
\begin{pgfscope}%
\pgfpathrectangle{\pgfqpoint{9.810417in}{5.660465in}}{\pgfqpoint{5.489583in}{0.877907in}}%
\pgfusepath{clip}%
\pgfsetbuttcap%
\pgfsetroundjoin%
\pgfsetlinewidth{1.505625pt}%
\definecolor{currentstroke}{rgb}{0.000000,0.000000,0.000000}%
\pgfsetstrokecolor{currentstroke}%
\pgfsetdash{}{0pt}%
\pgfpathmoveto{\pgfqpoint{13.510186in}{5.738704in}}%
\pgfpathlineto{\pgfqpoint{13.510186in}{5.713273in}}%
\pgfusepath{stroke}%
\end{pgfscope}%
\begin{pgfscope}%
\pgfpathrectangle{\pgfqpoint{9.810417in}{5.660465in}}{\pgfqpoint{5.489583in}{0.877907in}}%
\pgfusepath{clip}%
\pgfsetbuttcap%
\pgfsetroundjoin%
\pgfsetlinewidth{1.505625pt}%
\definecolor{currentstroke}{rgb}{0.000000,0.000000,0.000000}%
\pgfsetstrokecolor{currentstroke}%
\pgfsetdash{}{0pt}%
\pgfpathmoveto{\pgfqpoint{13.633409in}{5.738704in}}%
\pgfpathlineto{\pgfqpoint{13.633409in}{5.759513in}}%
\pgfusepath{stroke}%
\end{pgfscope}%
\begin{pgfscope}%
\pgfpathrectangle{\pgfqpoint{9.810417in}{5.660465in}}{\pgfqpoint{5.489583in}{0.877907in}}%
\pgfusepath{clip}%
\pgfsetbuttcap%
\pgfsetroundjoin%
\pgfsetlinewidth{1.505625pt}%
\definecolor{currentstroke}{rgb}{0.000000,0.000000,0.000000}%
\pgfsetstrokecolor{currentstroke}%
\pgfsetdash{}{0pt}%
\pgfpathmoveto{\pgfqpoint{13.756632in}{5.738704in}}%
\pgfpathlineto{\pgfqpoint{13.756632in}{5.744728in}}%
\pgfusepath{stroke}%
\end{pgfscope}%
\begin{pgfscope}%
\pgfpathrectangle{\pgfqpoint{9.810417in}{5.660465in}}{\pgfqpoint{5.489583in}{0.877907in}}%
\pgfusepath{clip}%
\pgfsetbuttcap%
\pgfsetroundjoin%
\pgfsetlinewidth{1.505625pt}%
\definecolor{currentstroke}{rgb}{0.000000,0.000000,0.000000}%
\pgfsetstrokecolor{currentstroke}%
\pgfsetdash{}{0pt}%
\pgfpathmoveto{\pgfqpoint{13.879855in}{5.738704in}}%
\pgfpathlineto{\pgfqpoint{13.879855in}{5.746406in}}%
\pgfusepath{stroke}%
\end{pgfscope}%
\begin{pgfscope}%
\pgfpathrectangle{\pgfqpoint{9.810417in}{5.660465in}}{\pgfqpoint{5.489583in}{0.877907in}}%
\pgfusepath{clip}%
\pgfsetbuttcap%
\pgfsetroundjoin%
\pgfsetlinewidth{1.505625pt}%
\definecolor{currentstroke}{rgb}{0.000000,0.000000,0.000000}%
\pgfsetstrokecolor{currentstroke}%
\pgfsetdash{}{0pt}%
\pgfpathmoveto{\pgfqpoint{14.003078in}{5.738704in}}%
\pgfpathlineto{\pgfqpoint{14.003078in}{5.762792in}}%
\pgfusepath{stroke}%
\end{pgfscope}%
\begin{pgfscope}%
\pgfpathrectangle{\pgfqpoint{9.810417in}{5.660465in}}{\pgfqpoint{5.489583in}{0.877907in}}%
\pgfusepath{clip}%
\pgfsetbuttcap%
\pgfsetroundjoin%
\pgfsetlinewidth{1.505625pt}%
\definecolor{currentstroke}{rgb}{0.000000,0.000000,0.000000}%
\pgfsetstrokecolor{currentstroke}%
\pgfsetdash{}{0pt}%
\pgfpathmoveto{\pgfqpoint{14.126301in}{5.738704in}}%
\pgfpathlineto{\pgfqpoint{14.126301in}{5.738581in}}%
\pgfusepath{stroke}%
\end{pgfscope}%
\begin{pgfscope}%
\pgfpathrectangle{\pgfqpoint{9.810417in}{5.660465in}}{\pgfqpoint{5.489583in}{0.877907in}}%
\pgfusepath{clip}%
\pgfsetbuttcap%
\pgfsetroundjoin%
\pgfsetlinewidth{1.505625pt}%
\definecolor{currentstroke}{rgb}{0.000000,0.000000,0.000000}%
\pgfsetstrokecolor{currentstroke}%
\pgfsetdash{}{0pt}%
\pgfpathmoveto{\pgfqpoint{14.249524in}{5.738704in}}%
\pgfpathlineto{\pgfqpoint{14.249524in}{5.742813in}}%
\pgfusepath{stroke}%
\end{pgfscope}%
\begin{pgfscope}%
\pgfpathrectangle{\pgfqpoint{9.810417in}{5.660465in}}{\pgfqpoint{5.489583in}{0.877907in}}%
\pgfusepath{clip}%
\pgfsetbuttcap%
\pgfsetroundjoin%
\pgfsetlinewidth{1.505625pt}%
\definecolor{currentstroke}{rgb}{0.000000,0.000000,0.000000}%
\pgfsetstrokecolor{currentstroke}%
\pgfsetdash{}{0pt}%
\pgfpathmoveto{\pgfqpoint{14.372747in}{5.738704in}}%
\pgfpathlineto{\pgfqpoint{14.372747in}{5.728299in}}%
\pgfusepath{stroke}%
\end{pgfscope}%
\begin{pgfscope}%
\pgfpathrectangle{\pgfqpoint{9.810417in}{5.660465in}}{\pgfqpoint{5.489583in}{0.877907in}}%
\pgfusepath{clip}%
\pgfsetbuttcap%
\pgfsetroundjoin%
\pgfsetlinewidth{1.505625pt}%
\definecolor{currentstroke}{rgb}{0.000000,0.000000,0.000000}%
\pgfsetstrokecolor{currentstroke}%
\pgfsetdash{}{0pt}%
\pgfpathmoveto{\pgfqpoint{14.495970in}{5.738704in}}%
\pgfpathlineto{\pgfqpoint{14.495970in}{5.724068in}}%
\pgfusepath{stroke}%
\end{pgfscope}%
\begin{pgfscope}%
\pgfpathrectangle{\pgfqpoint{9.810417in}{5.660465in}}{\pgfqpoint{5.489583in}{0.877907in}}%
\pgfusepath{clip}%
\pgfsetbuttcap%
\pgfsetroundjoin%
\pgfsetlinewidth{1.505625pt}%
\definecolor{currentstroke}{rgb}{0.000000,0.000000,0.000000}%
\pgfsetstrokecolor{currentstroke}%
\pgfsetdash{}{0pt}%
\pgfpathmoveto{\pgfqpoint{14.619193in}{5.738704in}}%
\pgfpathlineto{\pgfqpoint{14.619193in}{5.741766in}}%
\pgfusepath{stroke}%
\end{pgfscope}%
\begin{pgfscope}%
\pgfpathrectangle{\pgfqpoint{9.810417in}{5.660465in}}{\pgfqpoint{5.489583in}{0.877907in}}%
\pgfusepath{clip}%
\pgfsetbuttcap%
\pgfsetroundjoin%
\pgfsetlinewidth{1.505625pt}%
\definecolor{currentstroke}{rgb}{0.000000,0.000000,0.000000}%
\pgfsetstrokecolor{currentstroke}%
\pgfsetdash{}{0pt}%
\pgfpathmoveto{\pgfqpoint{14.742416in}{5.738704in}}%
\pgfpathlineto{\pgfqpoint{14.742416in}{5.740102in}}%
\pgfusepath{stroke}%
\end{pgfscope}%
\begin{pgfscope}%
\pgfpathrectangle{\pgfqpoint{9.810417in}{5.660465in}}{\pgfqpoint{5.489583in}{0.877907in}}%
\pgfusepath{clip}%
\pgfsetbuttcap%
\pgfsetroundjoin%
\pgfsetlinewidth{1.505625pt}%
\definecolor{currentstroke}{rgb}{0.000000,0.000000,0.000000}%
\pgfsetstrokecolor{currentstroke}%
\pgfsetdash{}{0pt}%
\pgfpathmoveto{\pgfqpoint{14.865639in}{5.738704in}}%
\pgfpathlineto{\pgfqpoint{14.865639in}{5.701217in}}%
\pgfusepath{stroke}%
\end{pgfscope}%
\begin{pgfscope}%
\pgfpathrectangle{\pgfqpoint{9.810417in}{5.660465in}}{\pgfqpoint{5.489583in}{0.877907in}}%
\pgfusepath{clip}%
\pgfsetbuttcap%
\pgfsetroundjoin%
\pgfsetlinewidth{1.505625pt}%
\definecolor{currentstroke}{rgb}{0.000000,0.000000,0.000000}%
\pgfsetstrokecolor{currentstroke}%
\pgfsetdash{}{0pt}%
\pgfpathmoveto{\pgfqpoint{14.988862in}{5.738704in}}%
\pgfpathlineto{\pgfqpoint{14.988862in}{5.740465in}}%
\pgfusepath{stroke}%
\end{pgfscope}%
\begin{pgfscope}%
\pgfpathrectangle{\pgfqpoint{9.810417in}{5.660465in}}{\pgfqpoint{5.489583in}{0.877907in}}%
\pgfusepath{clip}%
\pgfsetroundcap%
\pgfsetroundjoin%
\pgfsetlinewidth{1.505625pt}%
\definecolor{currentstroke}{rgb}{0.121569,0.466667,0.705882}%
\pgfsetstrokecolor{currentstroke}%
\pgfsetdash{}{0pt}%
\pgfpathmoveto{\pgfqpoint{9.810417in}{5.738704in}}%
\pgfpathlineto{\pgfqpoint{15.300000in}{5.738704in}}%
\pgfusepath{stroke}%
\end{pgfscope}%
\begin{pgfscope}%
\pgfpathrectangle{\pgfqpoint{9.810417in}{5.660465in}}{\pgfqpoint{5.489583in}{0.877907in}}%
\pgfusepath{clip}%
\pgfsetbuttcap%
\pgfsetroundjoin%
\definecolor{currentfill}{rgb}{0.121569,0.466667,0.705882}%
\pgfsetfillcolor{currentfill}%
\pgfsetlinewidth{1.003750pt}%
\definecolor{currentstroke}{rgb}{0.121569,0.466667,0.705882}%
\pgfsetstrokecolor{currentstroke}%
\pgfsetdash{}{0pt}%
\pgfsys@defobject{currentmarker}{\pgfqpoint{-0.034722in}{-0.034722in}}{\pgfqpoint{0.034722in}{0.034722in}}{%
\pgfpathmoveto{\pgfqpoint{0.000000in}{-0.034722in}}%
\pgfpathcurveto{\pgfqpoint{0.009208in}{-0.034722in}}{\pgfqpoint{0.018041in}{-0.031064in}}{\pgfqpoint{0.024552in}{-0.024552in}}%
\pgfpathcurveto{\pgfqpoint{0.031064in}{-0.018041in}}{\pgfqpoint{0.034722in}{-0.009208in}}{\pgfqpoint{0.034722in}{0.000000in}}%
\pgfpathcurveto{\pgfqpoint{0.034722in}{0.009208in}}{\pgfqpoint{0.031064in}{0.018041in}}{\pgfqpoint{0.024552in}{0.024552in}}%
\pgfpathcurveto{\pgfqpoint{0.018041in}{0.031064in}}{\pgfqpoint{0.009208in}{0.034722in}}{\pgfqpoint{0.000000in}{0.034722in}}%
\pgfpathcurveto{\pgfqpoint{-0.009208in}{0.034722in}}{\pgfqpoint{-0.018041in}{0.031064in}}{\pgfqpoint{-0.024552in}{0.024552in}}%
\pgfpathcurveto{\pgfqpoint{-0.031064in}{0.018041in}}{\pgfqpoint{-0.034722in}{0.009208in}}{\pgfqpoint{-0.034722in}{0.000000in}}%
\pgfpathcurveto{\pgfqpoint{-0.034722in}{-0.009208in}}{\pgfqpoint{-0.031064in}{-0.018041in}}{\pgfqpoint{-0.024552in}{-0.024552in}}%
\pgfpathcurveto{\pgfqpoint{-0.018041in}{-0.031064in}}{\pgfqpoint{-0.009208in}{-0.034722in}}{\pgfqpoint{0.000000in}{-0.034722in}}%
\pgfpathclose%
\pgfusepath{stroke,fill}%
}%
\begin{pgfscope}%
\pgfsys@transformshift{10.059943in}{6.498467in}%
\pgfsys@useobject{currentmarker}{}%
\end{pgfscope}%
\begin{pgfscope}%
\pgfsys@transformshift{10.183166in}{6.494818in}%
\pgfsys@useobject{currentmarker}{}%
\end{pgfscope}%
\begin{pgfscope}%
\pgfsys@transformshift{10.306389in}{5.704530in}%
\pgfsys@useobject{currentmarker}{}%
\end{pgfscope}%
\begin{pgfscope}%
\pgfsys@transformshift{10.429612in}{5.726947in}%
\pgfsys@useobject{currentmarker}{}%
\end{pgfscope}%
\begin{pgfscope}%
\pgfsys@transformshift{10.552835in}{5.731162in}%
\pgfsys@useobject{currentmarker}{}%
\end{pgfscope}%
\begin{pgfscope}%
\pgfsys@transformshift{10.676058in}{5.752214in}%
\pgfsys@useobject{currentmarker}{}%
\end{pgfscope}%
\begin{pgfscope}%
\pgfsys@transformshift{10.799281in}{5.718116in}%
\pgfsys@useobject{currentmarker}{}%
\end{pgfscope}%
\begin{pgfscope}%
\pgfsys@transformshift{10.922504in}{5.730071in}%
\pgfsys@useobject{currentmarker}{}%
\end{pgfscope}%
\begin{pgfscope}%
\pgfsys@transformshift{11.045727in}{5.738515in}%
\pgfsys@useobject{currentmarker}{}%
\end{pgfscope}%
\begin{pgfscope}%
\pgfsys@transformshift{11.168950in}{5.737618in}%
\pgfsys@useobject{currentmarker}{}%
\end{pgfscope}%
\begin{pgfscope}%
\pgfsys@transformshift{11.292173in}{5.766994in}%
\pgfsys@useobject{currentmarker}{}%
\end{pgfscope}%
\begin{pgfscope}%
\pgfsys@transformshift{11.415396in}{5.742105in}%
\pgfsys@useobject{currentmarker}{}%
\end{pgfscope}%
\begin{pgfscope}%
\pgfsys@transformshift{11.538619in}{5.720343in}%
\pgfsys@useobject{currentmarker}{}%
\end{pgfscope}%
\begin{pgfscope}%
\pgfsys@transformshift{11.661842in}{5.743483in}%
\pgfsys@useobject{currentmarker}{}%
\end{pgfscope}%
\begin{pgfscope}%
\pgfsys@transformshift{11.785065in}{5.716537in}%
\pgfsys@useobject{currentmarker}{}%
\end{pgfscope}%
\begin{pgfscope}%
\pgfsys@transformshift{11.908288in}{5.742957in}%
\pgfsys@useobject{currentmarker}{}%
\end{pgfscope}%
\begin{pgfscope}%
\pgfsys@transformshift{12.031511in}{5.748041in}%
\pgfsys@useobject{currentmarker}{}%
\end{pgfscope}%
\begin{pgfscope}%
\pgfsys@transformshift{12.154734in}{5.722072in}%
\pgfsys@useobject{currentmarker}{}%
\end{pgfscope}%
\begin{pgfscope}%
\pgfsys@transformshift{12.277957in}{5.733855in}%
\pgfsys@useobject{currentmarker}{}%
\end{pgfscope}%
\begin{pgfscope}%
\pgfsys@transformshift{12.401180in}{5.740818in}%
\pgfsys@useobject{currentmarker}{}%
\end{pgfscope}%
\begin{pgfscope}%
\pgfsys@transformshift{12.524403in}{5.755199in}%
\pgfsys@useobject{currentmarker}{}%
\end{pgfscope}%
\begin{pgfscope}%
\pgfsys@transformshift{12.647626in}{5.731481in}%
\pgfsys@useobject{currentmarker}{}%
\end{pgfscope}%
\begin{pgfscope}%
\pgfsys@transformshift{12.770849in}{5.758217in}%
\pgfsys@useobject{currentmarker}{}%
\end{pgfscope}%
\begin{pgfscope}%
\pgfsys@transformshift{12.894072in}{5.768832in}%
\pgfsys@useobject{currentmarker}{}%
\end{pgfscope}%
\begin{pgfscope}%
\pgfsys@transformshift{13.017294in}{5.747465in}%
\pgfsys@useobject{currentmarker}{}%
\end{pgfscope}%
\begin{pgfscope}%
\pgfsys@transformshift{13.140517in}{5.748047in}%
\pgfsys@useobject{currentmarker}{}%
\end{pgfscope}%
\begin{pgfscope}%
\pgfsys@transformshift{13.263740in}{5.762313in}%
\pgfsys@useobject{currentmarker}{}%
\end{pgfscope}%
\begin{pgfscope}%
\pgfsys@transformshift{13.386963in}{5.761192in}%
\pgfsys@useobject{currentmarker}{}%
\end{pgfscope}%
\begin{pgfscope}%
\pgfsys@transformshift{13.510186in}{5.713273in}%
\pgfsys@useobject{currentmarker}{}%
\end{pgfscope}%
\begin{pgfscope}%
\pgfsys@transformshift{13.633409in}{5.759513in}%
\pgfsys@useobject{currentmarker}{}%
\end{pgfscope}%
\begin{pgfscope}%
\pgfsys@transformshift{13.756632in}{5.744728in}%
\pgfsys@useobject{currentmarker}{}%
\end{pgfscope}%
\begin{pgfscope}%
\pgfsys@transformshift{13.879855in}{5.746406in}%
\pgfsys@useobject{currentmarker}{}%
\end{pgfscope}%
\begin{pgfscope}%
\pgfsys@transformshift{14.003078in}{5.762792in}%
\pgfsys@useobject{currentmarker}{}%
\end{pgfscope}%
\begin{pgfscope}%
\pgfsys@transformshift{14.126301in}{5.738581in}%
\pgfsys@useobject{currentmarker}{}%
\end{pgfscope}%
\begin{pgfscope}%
\pgfsys@transformshift{14.249524in}{5.742813in}%
\pgfsys@useobject{currentmarker}{}%
\end{pgfscope}%
\begin{pgfscope}%
\pgfsys@transformshift{14.372747in}{5.728299in}%
\pgfsys@useobject{currentmarker}{}%
\end{pgfscope}%
\begin{pgfscope}%
\pgfsys@transformshift{14.495970in}{5.724068in}%
\pgfsys@useobject{currentmarker}{}%
\end{pgfscope}%
\begin{pgfscope}%
\pgfsys@transformshift{14.619193in}{5.741766in}%
\pgfsys@useobject{currentmarker}{}%
\end{pgfscope}%
\begin{pgfscope}%
\pgfsys@transformshift{14.742416in}{5.740102in}%
\pgfsys@useobject{currentmarker}{}%
\end{pgfscope}%
\begin{pgfscope}%
\pgfsys@transformshift{14.865639in}{5.701217in}%
\pgfsys@useobject{currentmarker}{}%
\end{pgfscope}%
\begin{pgfscope}%
\pgfsys@transformshift{14.988862in}{5.740465in}%
\pgfsys@useobject{currentmarker}{}%
\end{pgfscope}%
\end{pgfscope}%
\begin{pgfscope}%
\pgfsetrectcap%
\pgfsetmiterjoin%
\pgfsetlinewidth{0.803000pt}%
\definecolor{currentstroke}{rgb}{1.000000,1.000000,1.000000}%
\pgfsetstrokecolor{currentstroke}%
\pgfsetdash{}{0pt}%
\pgfpathmoveto{\pgfqpoint{9.810417in}{5.660465in}}%
\pgfpathlineto{\pgfqpoint{9.810417in}{6.538372in}}%
\pgfusepath{stroke}%
\end{pgfscope}%
\begin{pgfscope}%
\pgfsetrectcap%
\pgfsetmiterjoin%
\pgfsetlinewidth{0.803000pt}%
\definecolor{currentstroke}{rgb}{1.000000,1.000000,1.000000}%
\pgfsetstrokecolor{currentstroke}%
\pgfsetdash{}{0pt}%
\pgfpathmoveto{\pgfqpoint{15.300000in}{5.660465in}}%
\pgfpathlineto{\pgfqpoint{15.300000in}{6.538372in}}%
\pgfusepath{stroke}%
\end{pgfscope}%
\begin{pgfscope}%
\pgfsetrectcap%
\pgfsetmiterjoin%
\pgfsetlinewidth{0.803000pt}%
\definecolor{currentstroke}{rgb}{1.000000,1.000000,1.000000}%
\pgfsetstrokecolor{currentstroke}%
\pgfsetdash{}{0pt}%
\pgfpathmoveto{\pgfqpoint{9.810417in}{5.660465in}}%
\pgfpathlineto{\pgfqpoint{15.300000in}{5.660465in}}%
\pgfusepath{stroke}%
\end{pgfscope}%
\begin{pgfscope}%
\pgfsetrectcap%
\pgfsetmiterjoin%
\pgfsetlinewidth{0.803000pt}%
\definecolor{currentstroke}{rgb}{1.000000,1.000000,1.000000}%
\pgfsetstrokecolor{currentstroke}%
\pgfsetdash{}{0pt}%
\pgfpathmoveto{\pgfqpoint{9.810417in}{6.538372in}}%
\pgfpathlineto{\pgfqpoint{15.300000in}{6.538372in}}%
\pgfusepath{stroke}%
\end{pgfscope}%
\begin{pgfscope}%
\definecolor{textcolor}{rgb}{0.150000,0.150000,0.150000}%
\pgfsetstrokecolor{textcolor}%
\pgfsetfillcolor{textcolor}%
\pgftext[x=12.555208in,y=6.621705in,,base]{\color{textcolor}\rmfamily\fontsize{16.800000}{20.160000}\selectfont Partial Autocorrelation}%
\end{pgfscope}%
\begin{pgfscope}%
\pgfsetbuttcap%
\pgfsetmiterjoin%
\definecolor{currentfill}{rgb}{0.917647,0.917647,0.949020}%
\pgfsetfillcolor{currentfill}%
\pgfsetlinewidth{0.000000pt}%
\definecolor{currentstroke}{rgb}{0.000000,0.000000,0.000000}%
\pgfsetstrokecolor{currentstroke}%
\pgfsetstrokeopacity{0.000000}%
\pgfsetdash{}{0pt}%
\pgfpathmoveto{\pgfqpoint{2.125000in}{4.080233in}}%
\pgfpathlineto{\pgfqpoint{7.614583in}{4.080233in}}%
\pgfpathlineto{\pgfqpoint{7.614583in}{4.958140in}}%
\pgfpathlineto{\pgfqpoint{2.125000in}{4.958140in}}%
\pgfpathclose%
\pgfusepath{fill}%
\end{pgfscope}%
\begin{pgfscope}%
\pgfpathrectangle{\pgfqpoint{2.125000in}{4.080233in}}{\pgfqpoint{5.489583in}{0.877907in}}%
\pgfusepath{clip}%
\pgfsetroundcap%
\pgfsetroundjoin%
\pgfsetlinewidth{0.803000pt}%
\definecolor{currentstroke}{rgb}{1.000000,1.000000,1.000000}%
\pgfsetstrokecolor{currentstroke}%
\pgfsetdash{}{0pt}%
\pgfpathmoveto{\pgfqpoint{2.374527in}{4.080233in}}%
\pgfpathlineto{\pgfqpoint{2.374527in}{4.958140in}}%
\pgfusepath{stroke}%
\end{pgfscope}%
\begin{pgfscope}%
\definecolor{textcolor}{rgb}{0.150000,0.150000,0.150000}%
\pgfsetstrokecolor{textcolor}%
\pgfsetfillcolor{textcolor}%
\pgftext[x=2.374527in,y=3.983010in,,top]{\color{textcolor}\rmfamily\fontsize{14.000000}{16.800000}\selectfont 0}%
\end{pgfscope}%
\begin{pgfscope}%
\pgfpathrectangle{\pgfqpoint{2.125000in}{4.080233in}}{\pgfqpoint{5.489583in}{0.877907in}}%
\pgfusepath{clip}%
\pgfsetroundcap%
\pgfsetroundjoin%
\pgfsetlinewidth{0.803000pt}%
\definecolor{currentstroke}{rgb}{1.000000,1.000000,1.000000}%
\pgfsetstrokecolor{currentstroke}%
\pgfsetdash{}{0pt}%
\pgfpathmoveto{\pgfqpoint{2.990641in}{4.080233in}}%
\pgfpathlineto{\pgfqpoint{2.990641in}{4.958140in}}%
\pgfusepath{stroke}%
\end{pgfscope}%
\begin{pgfscope}%
\definecolor{textcolor}{rgb}{0.150000,0.150000,0.150000}%
\pgfsetstrokecolor{textcolor}%
\pgfsetfillcolor{textcolor}%
\pgftext[x=2.990641in,y=3.983010in,,top]{\color{textcolor}\rmfamily\fontsize{14.000000}{16.800000}\selectfont 5}%
\end{pgfscope}%
\begin{pgfscope}%
\pgfpathrectangle{\pgfqpoint{2.125000in}{4.080233in}}{\pgfqpoint{5.489583in}{0.877907in}}%
\pgfusepath{clip}%
\pgfsetroundcap%
\pgfsetroundjoin%
\pgfsetlinewidth{0.803000pt}%
\definecolor{currentstroke}{rgb}{1.000000,1.000000,1.000000}%
\pgfsetstrokecolor{currentstroke}%
\pgfsetdash{}{0pt}%
\pgfpathmoveto{\pgfqpoint{3.606756in}{4.080233in}}%
\pgfpathlineto{\pgfqpoint{3.606756in}{4.958140in}}%
\pgfusepath{stroke}%
\end{pgfscope}%
\begin{pgfscope}%
\definecolor{textcolor}{rgb}{0.150000,0.150000,0.150000}%
\pgfsetstrokecolor{textcolor}%
\pgfsetfillcolor{textcolor}%
\pgftext[x=3.606756in,y=3.983010in,,top]{\color{textcolor}\rmfamily\fontsize{14.000000}{16.800000}\selectfont 10}%
\end{pgfscope}%
\begin{pgfscope}%
\pgfpathrectangle{\pgfqpoint{2.125000in}{4.080233in}}{\pgfqpoint{5.489583in}{0.877907in}}%
\pgfusepath{clip}%
\pgfsetroundcap%
\pgfsetroundjoin%
\pgfsetlinewidth{0.803000pt}%
\definecolor{currentstroke}{rgb}{1.000000,1.000000,1.000000}%
\pgfsetstrokecolor{currentstroke}%
\pgfsetdash{}{0pt}%
\pgfpathmoveto{\pgfqpoint{4.222871in}{4.080233in}}%
\pgfpathlineto{\pgfqpoint{4.222871in}{4.958140in}}%
\pgfusepath{stroke}%
\end{pgfscope}%
\begin{pgfscope}%
\definecolor{textcolor}{rgb}{0.150000,0.150000,0.150000}%
\pgfsetstrokecolor{textcolor}%
\pgfsetfillcolor{textcolor}%
\pgftext[x=4.222871in,y=3.983010in,,top]{\color{textcolor}\rmfamily\fontsize{14.000000}{16.800000}\selectfont 15}%
\end{pgfscope}%
\begin{pgfscope}%
\pgfpathrectangle{\pgfqpoint{2.125000in}{4.080233in}}{\pgfqpoint{5.489583in}{0.877907in}}%
\pgfusepath{clip}%
\pgfsetroundcap%
\pgfsetroundjoin%
\pgfsetlinewidth{0.803000pt}%
\definecolor{currentstroke}{rgb}{1.000000,1.000000,1.000000}%
\pgfsetstrokecolor{currentstroke}%
\pgfsetdash{}{0pt}%
\pgfpathmoveto{\pgfqpoint{4.838986in}{4.080233in}}%
\pgfpathlineto{\pgfqpoint{4.838986in}{4.958140in}}%
\pgfusepath{stroke}%
\end{pgfscope}%
\begin{pgfscope}%
\definecolor{textcolor}{rgb}{0.150000,0.150000,0.150000}%
\pgfsetstrokecolor{textcolor}%
\pgfsetfillcolor{textcolor}%
\pgftext[x=4.838986in,y=3.983010in,,top]{\color{textcolor}\rmfamily\fontsize{14.000000}{16.800000}\selectfont 20}%
\end{pgfscope}%
\begin{pgfscope}%
\pgfpathrectangle{\pgfqpoint{2.125000in}{4.080233in}}{\pgfqpoint{5.489583in}{0.877907in}}%
\pgfusepath{clip}%
\pgfsetroundcap%
\pgfsetroundjoin%
\pgfsetlinewidth{0.803000pt}%
\definecolor{currentstroke}{rgb}{1.000000,1.000000,1.000000}%
\pgfsetstrokecolor{currentstroke}%
\pgfsetdash{}{0pt}%
\pgfpathmoveto{\pgfqpoint{5.455101in}{4.080233in}}%
\pgfpathlineto{\pgfqpoint{5.455101in}{4.958140in}}%
\pgfusepath{stroke}%
\end{pgfscope}%
\begin{pgfscope}%
\definecolor{textcolor}{rgb}{0.150000,0.150000,0.150000}%
\pgfsetstrokecolor{textcolor}%
\pgfsetfillcolor{textcolor}%
\pgftext[x=5.455101in,y=3.983010in,,top]{\color{textcolor}\rmfamily\fontsize{14.000000}{16.800000}\selectfont 25}%
\end{pgfscope}%
\begin{pgfscope}%
\pgfpathrectangle{\pgfqpoint{2.125000in}{4.080233in}}{\pgfqpoint{5.489583in}{0.877907in}}%
\pgfusepath{clip}%
\pgfsetroundcap%
\pgfsetroundjoin%
\pgfsetlinewidth{0.803000pt}%
\definecolor{currentstroke}{rgb}{1.000000,1.000000,1.000000}%
\pgfsetstrokecolor{currentstroke}%
\pgfsetdash{}{0pt}%
\pgfpathmoveto{\pgfqpoint{6.071216in}{4.080233in}}%
\pgfpathlineto{\pgfqpoint{6.071216in}{4.958140in}}%
\pgfusepath{stroke}%
\end{pgfscope}%
\begin{pgfscope}%
\definecolor{textcolor}{rgb}{0.150000,0.150000,0.150000}%
\pgfsetstrokecolor{textcolor}%
\pgfsetfillcolor{textcolor}%
\pgftext[x=6.071216in,y=3.983010in,,top]{\color{textcolor}\rmfamily\fontsize{14.000000}{16.800000}\selectfont 30}%
\end{pgfscope}%
\begin{pgfscope}%
\pgfpathrectangle{\pgfqpoint{2.125000in}{4.080233in}}{\pgfqpoint{5.489583in}{0.877907in}}%
\pgfusepath{clip}%
\pgfsetroundcap%
\pgfsetroundjoin%
\pgfsetlinewidth{0.803000pt}%
\definecolor{currentstroke}{rgb}{1.000000,1.000000,1.000000}%
\pgfsetstrokecolor{currentstroke}%
\pgfsetdash{}{0pt}%
\pgfpathmoveto{\pgfqpoint{6.687330in}{4.080233in}}%
\pgfpathlineto{\pgfqpoint{6.687330in}{4.958140in}}%
\pgfusepath{stroke}%
\end{pgfscope}%
\begin{pgfscope}%
\definecolor{textcolor}{rgb}{0.150000,0.150000,0.150000}%
\pgfsetstrokecolor{textcolor}%
\pgfsetfillcolor{textcolor}%
\pgftext[x=6.687330in,y=3.983010in,,top]{\color{textcolor}\rmfamily\fontsize{14.000000}{16.800000}\selectfont 35}%
\end{pgfscope}%
\begin{pgfscope}%
\pgfpathrectangle{\pgfqpoint{2.125000in}{4.080233in}}{\pgfqpoint{5.489583in}{0.877907in}}%
\pgfusepath{clip}%
\pgfsetroundcap%
\pgfsetroundjoin%
\pgfsetlinewidth{0.803000pt}%
\definecolor{currentstroke}{rgb}{1.000000,1.000000,1.000000}%
\pgfsetstrokecolor{currentstroke}%
\pgfsetdash{}{0pt}%
\pgfpathmoveto{\pgfqpoint{7.303445in}{4.080233in}}%
\pgfpathlineto{\pgfqpoint{7.303445in}{4.958140in}}%
\pgfusepath{stroke}%
\end{pgfscope}%
\begin{pgfscope}%
\definecolor{textcolor}{rgb}{0.150000,0.150000,0.150000}%
\pgfsetstrokecolor{textcolor}%
\pgfsetfillcolor{textcolor}%
\pgftext[x=7.303445in,y=3.983010in,,top]{\color{textcolor}\rmfamily\fontsize{14.000000}{16.800000}\selectfont 40}%
\end{pgfscope}%
\begin{pgfscope}%
\pgfpathrectangle{\pgfqpoint{2.125000in}{4.080233in}}{\pgfqpoint{5.489583in}{0.877907in}}%
\pgfusepath{clip}%
\pgfsetroundcap%
\pgfsetroundjoin%
\pgfsetlinewidth{0.803000pt}%
\definecolor{currentstroke}{rgb}{1.000000,1.000000,1.000000}%
\pgfsetstrokecolor{currentstroke}%
\pgfsetdash{}{0pt}%
\pgfpathmoveto{\pgfqpoint{2.125000in}{4.357721in}}%
\pgfpathlineto{\pgfqpoint{7.614583in}{4.357721in}}%
\pgfusepath{stroke}%
\end{pgfscope}%
\begin{pgfscope}%
\definecolor{textcolor}{rgb}{0.150000,0.150000,0.150000}%
\pgfsetstrokecolor{textcolor}%
\pgfsetfillcolor{textcolor}%
\pgftext[x=1.904066in,y=4.283855in,left,base]{\color{textcolor}\rmfamily\fontsize{14.000000}{16.800000}\selectfont 0}%
\end{pgfscope}%
\begin{pgfscope}%
\pgfpathrectangle{\pgfqpoint{2.125000in}{4.080233in}}{\pgfqpoint{5.489583in}{0.877907in}}%
\pgfusepath{clip}%
\pgfsetroundcap%
\pgfsetroundjoin%
\pgfsetlinewidth{0.803000pt}%
\definecolor{currentstroke}{rgb}{1.000000,1.000000,1.000000}%
\pgfsetstrokecolor{currentstroke}%
\pgfsetdash{}{0pt}%
\pgfpathmoveto{\pgfqpoint{2.125000in}{4.918235in}}%
\pgfpathlineto{\pgfqpoint{7.614583in}{4.918235in}}%
\pgfusepath{stroke}%
\end{pgfscope}%
\begin{pgfscope}%
\definecolor{textcolor}{rgb}{0.150000,0.150000,0.150000}%
\pgfsetstrokecolor{textcolor}%
\pgfsetfillcolor{textcolor}%
\pgftext[x=1.904066in,y=4.844369in,left,base]{\color{textcolor}\rmfamily\fontsize{14.000000}{16.800000}\selectfont 1}%
\end{pgfscope}%
\begin{pgfscope}%
\pgfpathrectangle{\pgfqpoint{2.125000in}{4.080233in}}{\pgfqpoint{5.489583in}{0.877907in}}%
\pgfusepath{clip}%
\pgfsetbuttcap%
\pgfsetroundjoin%
\definecolor{currentfill}{rgb}{0.121569,0.466667,0.705882}%
\pgfsetfillcolor{currentfill}%
\pgfsetfillopacity{0.250000}%
\pgfsetlinewidth{1.003750pt}%
\definecolor{currentstroke}{rgb}{1.000000,1.000000,1.000000}%
\pgfsetstrokecolor{currentstroke}%
\pgfsetstrokeopacity{0.250000}%
\pgfsetdash{}{0pt}%
\pgfpathmoveto{\pgfqpoint{2.436138in}{4.386002in}}%
\pgfpathlineto{\pgfqpoint{2.436138in}{4.329441in}}%
\pgfpathlineto{\pgfqpoint{2.620972in}{4.308834in}}%
\pgfpathlineto{\pgfqpoint{2.744195in}{4.294707in}}%
\pgfpathlineto{\pgfqpoint{2.867418in}{4.283273in}}%
\pgfpathlineto{\pgfqpoint{2.990641in}{4.273430in}}%
\pgfpathlineto{\pgfqpoint{3.113864in}{4.264670in}}%
\pgfpathlineto{\pgfqpoint{3.237087in}{4.256711in}}%
\pgfpathlineto{\pgfqpoint{3.360310in}{4.249376in}}%
\pgfpathlineto{\pgfqpoint{3.483533in}{4.242545in}}%
\pgfpathlineto{\pgfqpoint{3.606756in}{4.236134in}}%
\pgfpathlineto{\pgfqpoint{3.729979in}{4.230080in}}%
\pgfpathlineto{\pgfqpoint{3.853202in}{4.224333in}}%
\pgfpathlineto{\pgfqpoint{3.976425in}{4.218856in}}%
\pgfpathlineto{\pgfqpoint{4.099648in}{4.213618in}}%
\pgfpathlineto{\pgfqpoint{4.222871in}{4.208595in}}%
\pgfpathlineto{\pgfqpoint{4.346094in}{4.203765in}}%
\pgfpathlineto{\pgfqpoint{4.469317in}{4.199109in}}%
\pgfpathlineto{\pgfqpoint{4.592540in}{4.194612in}}%
\pgfpathlineto{\pgfqpoint{4.715763in}{4.190261in}}%
\pgfpathlineto{\pgfqpoint{4.838986in}{4.186045in}}%
\pgfpathlineto{\pgfqpoint{4.962209in}{4.181954in}}%
\pgfpathlineto{\pgfqpoint{5.085432in}{4.177980in}}%
\pgfpathlineto{\pgfqpoint{5.208655in}{4.174115in}}%
\pgfpathlineto{\pgfqpoint{5.331878in}{4.170351in}}%
\pgfpathlineto{\pgfqpoint{5.455101in}{4.166682in}}%
\pgfpathlineto{\pgfqpoint{5.578324in}{4.163102in}}%
\pgfpathlineto{\pgfqpoint{5.701547in}{4.159607in}}%
\pgfpathlineto{\pgfqpoint{5.824770in}{4.156191in}}%
\pgfpathlineto{\pgfqpoint{5.947993in}{4.152850in}}%
\pgfpathlineto{\pgfqpoint{6.071216in}{4.149580in}}%
\pgfpathlineto{\pgfqpoint{6.194439in}{4.146377in}}%
\pgfpathlineto{\pgfqpoint{6.317662in}{4.143239in}}%
\pgfpathlineto{\pgfqpoint{6.440885in}{4.140161in}}%
\pgfpathlineto{\pgfqpoint{6.564108in}{4.137143in}}%
\pgfpathlineto{\pgfqpoint{6.687330in}{4.134180in}}%
\pgfpathlineto{\pgfqpoint{6.810553in}{4.131272in}}%
\pgfpathlineto{\pgfqpoint{6.933776in}{4.128416in}}%
\pgfpathlineto{\pgfqpoint{7.056999in}{4.125609in}}%
\pgfpathlineto{\pgfqpoint{7.180222in}{4.122850in}}%
\pgfpathlineto{\pgfqpoint{7.365057in}{4.120137in}}%
\pgfpathlineto{\pgfqpoint{7.365057in}{4.595305in}}%
\pgfpathlineto{\pgfqpoint{7.365057in}{4.595305in}}%
\pgfpathlineto{\pgfqpoint{7.180222in}{4.592592in}}%
\pgfpathlineto{\pgfqpoint{7.056999in}{4.589834in}}%
\pgfpathlineto{\pgfqpoint{6.933776in}{4.587027in}}%
\pgfpathlineto{\pgfqpoint{6.810553in}{4.584170in}}%
\pgfpathlineto{\pgfqpoint{6.687330in}{4.581262in}}%
\pgfpathlineto{\pgfqpoint{6.564108in}{4.578300in}}%
\pgfpathlineto{\pgfqpoint{6.440885in}{4.575281in}}%
\pgfpathlineto{\pgfqpoint{6.317662in}{4.572204in}}%
\pgfpathlineto{\pgfqpoint{6.194439in}{4.569065in}}%
\pgfpathlineto{\pgfqpoint{6.071216in}{4.565862in}}%
\pgfpathlineto{\pgfqpoint{5.947993in}{4.562592in}}%
\pgfpathlineto{\pgfqpoint{5.824770in}{4.559251in}}%
\pgfpathlineto{\pgfqpoint{5.701547in}{4.555835in}}%
\pgfpathlineto{\pgfqpoint{5.578324in}{4.552340in}}%
\pgfpathlineto{\pgfqpoint{5.455101in}{4.548761in}}%
\pgfpathlineto{\pgfqpoint{5.331878in}{4.545092in}}%
\pgfpathlineto{\pgfqpoint{5.208655in}{4.541328in}}%
\pgfpathlineto{\pgfqpoint{5.085432in}{4.537462in}}%
\pgfpathlineto{\pgfqpoint{4.962209in}{4.533488in}}%
\pgfpathlineto{\pgfqpoint{4.838986in}{4.529398in}}%
\pgfpathlineto{\pgfqpoint{4.715763in}{4.525181in}}%
\pgfpathlineto{\pgfqpoint{4.592540in}{4.520830in}}%
\pgfpathlineto{\pgfqpoint{4.469317in}{4.516333in}}%
\pgfpathlineto{\pgfqpoint{4.346094in}{4.511677in}}%
\pgfpathlineto{\pgfqpoint{4.222871in}{4.506847in}}%
\pgfpathlineto{\pgfqpoint{4.099648in}{4.501824in}}%
\pgfpathlineto{\pgfqpoint{3.976425in}{4.496586in}}%
\pgfpathlineto{\pgfqpoint{3.853202in}{4.491109in}}%
\pgfpathlineto{\pgfqpoint{3.729979in}{4.485362in}}%
\pgfpathlineto{\pgfqpoint{3.606756in}{4.479308in}}%
\pgfpathlineto{\pgfqpoint{3.483533in}{4.472898in}}%
\pgfpathlineto{\pgfqpoint{3.360310in}{4.466067in}}%
\pgfpathlineto{\pgfqpoint{3.237087in}{4.458732in}}%
\pgfpathlineto{\pgfqpoint{3.113864in}{4.450772in}}%
\pgfpathlineto{\pgfqpoint{2.990641in}{4.442013in}}%
\pgfpathlineto{\pgfqpoint{2.867418in}{4.432169in}}%
\pgfpathlineto{\pgfqpoint{2.744195in}{4.420736in}}%
\pgfpathlineto{\pgfqpoint{2.620972in}{4.406609in}}%
\pgfpathlineto{\pgfqpoint{2.436138in}{4.386002in}}%
\pgfpathclose%
\pgfusepath{stroke,fill}%
\end{pgfscope}%
\begin{pgfscope}%
\pgfpathrectangle{\pgfqpoint{2.125000in}{4.080233in}}{\pgfqpoint{5.489583in}{0.877907in}}%
\pgfusepath{clip}%
\pgfsetbuttcap%
\pgfsetroundjoin%
\pgfsetlinewidth{1.505625pt}%
\definecolor{currentstroke}{rgb}{0.000000,0.000000,0.000000}%
\pgfsetstrokecolor{currentstroke}%
\pgfsetdash{}{0pt}%
\pgfpathmoveto{\pgfqpoint{2.374527in}{4.357721in}}%
\pgfpathlineto{\pgfqpoint{2.374527in}{4.918235in}}%
\pgfusepath{stroke}%
\end{pgfscope}%
\begin{pgfscope}%
\pgfpathrectangle{\pgfqpoint{2.125000in}{4.080233in}}{\pgfqpoint{5.489583in}{0.877907in}}%
\pgfusepath{clip}%
\pgfsetbuttcap%
\pgfsetroundjoin%
\pgfsetlinewidth{1.505625pt}%
\definecolor{currentstroke}{rgb}{0.000000,0.000000,0.000000}%
\pgfsetstrokecolor{currentstroke}%
\pgfsetdash{}{0pt}%
\pgfpathmoveto{\pgfqpoint{2.497749in}{4.357721in}}%
\pgfpathlineto{\pgfqpoint{2.497749in}{4.916588in}}%
\pgfusepath{stroke}%
\end{pgfscope}%
\begin{pgfscope}%
\pgfpathrectangle{\pgfqpoint{2.125000in}{4.080233in}}{\pgfqpoint{5.489583in}{0.877907in}}%
\pgfusepath{clip}%
\pgfsetbuttcap%
\pgfsetroundjoin%
\pgfsetlinewidth{1.505625pt}%
\definecolor{currentstroke}{rgb}{0.000000,0.000000,0.000000}%
\pgfsetstrokecolor{currentstroke}%
\pgfsetdash{}{0pt}%
\pgfpathmoveto{\pgfqpoint{2.620972in}{4.357721in}}%
\pgfpathlineto{\pgfqpoint{2.620972in}{4.914936in}}%
\pgfusepath{stroke}%
\end{pgfscope}%
\begin{pgfscope}%
\pgfpathrectangle{\pgfqpoint{2.125000in}{4.080233in}}{\pgfqpoint{5.489583in}{0.877907in}}%
\pgfusepath{clip}%
\pgfsetbuttcap%
\pgfsetroundjoin%
\pgfsetlinewidth{1.505625pt}%
\definecolor{currentstroke}{rgb}{0.000000,0.000000,0.000000}%
\pgfsetstrokecolor{currentstroke}%
\pgfsetdash{}{0pt}%
\pgfpathmoveto{\pgfqpoint{2.744195in}{4.357721in}}%
\pgfpathlineto{\pgfqpoint{2.744195in}{4.913320in}}%
\pgfusepath{stroke}%
\end{pgfscope}%
\begin{pgfscope}%
\pgfpathrectangle{\pgfqpoint{2.125000in}{4.080233in}}{\pgfqpoint{5.489583in}{0.877907in}}%
\pgfusepath{clip}%
\pgfsetbuttcap%
\pgfsetroundjoin%
\pgfsetlinewidth{1.505625pt}%
\definecolor{currentstroke}{rgb}{0.000000,0.000000,0.000000}%
\pgfsetstrokecolor{currentstroke}%
\pgfsetdash{}{0pt}%
\pgfpathmoveto{\pgfqpoint{2.867418in}{4.357721in}}%
\pgfpathlineto{\pgfqpoint{2.867418in}{4.911705in}}%
\pgfusepath{stroke}%
\end{pgfscope}%
\begin{pgfscope}%
\pgfpathrectangle{\pgfqpoint{2.125000in}{4.080233in}}{\pgfqpoint{5.489583in}{0.877907in}}%
\pgfusepath{clip}%
\pgfsetbuttcap%
\pgfsetroundjoin%
\pgfsetlinewidth{1.505625pt}%
\definecolor{currentstroke}{rgb}{0.000000,0.000000,0.000000}%
\pgfsetstrokecolor{currentstroke}%
\pgfsetdash{}{0pt}%
\pgfpathmoveto{\pgfqpoint{2.990641in}{4.357721in}}%
\pgfpathlineto{\pgfqpoint{2.990641in}{4.910102in}}%
\pgfusepath{stroke}%
\end{pgfscope}%
\begin{pgfscope}%
\pgfpathrectangle{\pgfqpoint{2.125000in}{4.080233in}}{\pgfqpoint{5.489583in}{0.877907in}}%
\pgfusepath{clip}%
\pgfsetbuttcap%
\pgfsetroundjoin%
\pgfsetlinewidth{1.505625pt}%
\definecolor{currentstroke}{rgb}{0.000000,0.000000,0.000000}%
\pgfsetstrokecolor{currentstroke}%
\pgfsetdash{}{0pt}%
\pgfpathmoveto{\pgfqpoint{3.113864in}{4.357721in}}%
\pgfpathlineto{\pgfqpoint{3.113864in}{4.908515in}}%
\pgfusepath{stroke}%
\end{pgfscope}%
\begin{pgfscope}%
\pgfpathrectangle{\pgfqpoint{2.125000in}{4.080233in}}{\pgfqpoint{5.489583in}{0.877907in}}%
\pgfusepath{clip}%
\pgfsetbuttcap%
\pgfsetroundjoin%
\pgfsetlinewidth{1.505625pt}%
\definecolor{currentstroke}{rgb}{0.000000,0.000000,0.000000}%
\pgfsetstrokecolor{currentstroke}%
\pgfsetdash{}{0pt}%
\pgfpathmoveto{\pgfqpoint{3.237087in}{4.357721in}}%
\pgfpathlineto{\pgfqpoint{3.237087in}{4.906909in}}%
\pgfusepath{stroke}%
\end{pgfscope}%
\begin{pgfscope}%
\pgfpathrectangle{\pgfqpoint{2.125000in}{4.080233in}}{\pgfqpoint{5.489583in}{0.877907in}}%
\pgfusepath{clip}%
\pgfsetbuttcap%
\pgfsetroundjoin%
\pgfsetlinewidth{1.505625pt}%
\definecolor{currentstroke}{rgb}{0.000000,0.000000,0.000000}%
\pgfsetstrokecolor{currentstroke}%
\pgfsetdash{}{0pt}%
\pgfpathmoveto{\pgfqpoint{3.360310in}{4.357721in}}%
\pgfpathlineto{\pgfqpoint{3.360310in}{4.905343in}}%
\pgfusepath{stroke}%
\end{pgfscope}%
\begin{pgfscope}%
\pgfpathrectangle{\pgfqpoint{2.125000in}{4.080233in}}{\pgfqpoint{5.489583in}{0.877907in}}%
\pgfusepath{clip}%
\pgfsetbuttcap%
\pgfsetroundjoin%
\pgfsetlinewidth{1.505625pt}%
\definecolor{currentstroke}{rgb}{0.000000,0.000000,0.000000}%
\pgfsetstrokecolor{currentstroke}%
\pgfsetdash{}{0pt}%
\pgfpathmoveto{\pgfqpoint{3.483533in}{4.357721in}}%
\pgfpathlineto{\pgfqpoint{3.483533in}{4.903709in}}%
\pgfusepath{stroke}%
\end{pgfscope}%
\begin{pgfscope}%
\pgfpathrectangle{\pgfqpoint{2.125000in}{4.080233in}}{\pgfqpoint{5.489583in}{0.877907in}}%
\pgfusepath{clip}%
\pgfsetbuttcap%
\pgfsetroundjoin%
\pgfsetlinewidth{1.505625pt}%
\definecolor{currentstroke}{rgb}{0.000000,0.000000,0.000000}%
\pgfsetstrokecolor{currentstroke}%
\pgfsetdash{}{0pt}%
\pgfpathmoveto{\pgfqpoint{3.606756in}{4.357721in}}%
\pgfpathlineto{\pgfqpoint{3.606756in}{4.902111in}}%
\pgfusepath{stroke}%
\end{pgfscope}%
\begin{pgfscope}%
\pgfpathrectangle{\pgfqpoint{2.125000in}{4.080233in}}{\pgfqpoint{5.489583in}{0.877907in}}%
\pgfusepath{clip}%
\pgfsetbuttcap%
\pgfsetroundjoin%
\pgfsetlinewidth{1.505625pt}%
\definecolor{currentstroke}{rgb}{0.000000,0.000000,0.000000}%
\pgfsetstrokecolor{currentstroke}%
\pgfsetdash{}{0pt}%
\pgfpathmoveto{\pgfqpoint{3.729979in}{4.357721in}}%
\pgfpathlineto{\pgfqpoint{3.729979in}{4.900528in}}%
\pgfusepath{stroke}%
\end{pgfscope}%
\begin{pgfscope}%
\pgfpathrectangle{\pgfqpoint{2.125000in}{4.080233in}}{\pgfqpoint{5.489583in}{0.877907in}}%
\pgfusepath{clip}%
\pgfsetbuttcap%
\pgfsetroundjoin%
\pgfsetlinewidth{1.505625pt}%
\definecolor{currentstroke}{rgb}{0.000000,0.000000,0.000000}%
\pgfsetstrokecolor{currentstroke}%
\pgfsetdash{}{0pt}%
\pgfpathmoveto{\pgfqpoint{3.853202in}{4.357721in}}%
\pgfpathlineto{\pgfqpoint{3.853202in}{4.898906in}}%
\pgfusepath{stroke}%
\end{pgfscope}%
\begin{pgfscope}%
\pgfpathrectangle{\pgfqpoint{2.125000in}{4.080233in}}{\pgfqpoint{5.489583in}{0.877907in}}%
\pgfusepath{clip}%
\pgfsetbuttcap%
\pgfsetroundjoin%
\pgfsetlinewidth{1.505625pt}%
\definecolor{currentstroke}{rgb}{0.000000,0.000000,0.000000}%
\pgfsetstrokecolor{currentstroke}%
\pgfsetdash{}{0pt}%
\pgfpathmoveto{\pgfqpoint{3.976425in}{4.357721in}}%
\pgfpathlineto{\pgfqpoint{3.976425in}{4.897248in}}%
\pgfusepath{stroke}%
\end{pgfscope}%
\begin{pgfscope}%
\pgfpathrectangle{\pgfqpoint{2.125000in}{4.080233in}}{\pgfqpoint{5.489583in}{0.877907in}}%
\pgfusepath{clip}%
\pgfsetbuttcap%
\pgfsetroundjoin%
\pgfsetlinewidth{1.505625pt}%
\definecolor{currentstroke}{rgb}{0.000000,0.000000,0.000000}%
\pgfsetstrokecolor{currentstroke}%
\pgfsetdash{}{0pt}%
\pgfpathmoveto{\pgfqpoint{4.099648in}{4.357721in}}%
\pgfpathlineto{\pgfqpoint{4.099648in}{4.895586in}}%
\pgfusepath{stroke}%
\end{pgfscope}%
\begin{pgfscope}%
\pgfpathrectangle{\pgfqpoint{2.125000in}{4.080233in}}{\pgfqpoint{5.489583in}{0.877907in}}%
\pgfusepath{clip}%
\pgfsetbuttcap%
\pgfsetroundjoin%
\pgfsetlinewidth{1.505625pt}%
\definecolor{currentstroke}{rgb}{0.000000,0.000000,0.000000}%
\pgfsetstrokecolor{currentstroke}%
\pgfsetdash{}{0pt}%
\pgfpathmoveto{\pgfqpoint{4.222871in}{4.357721in}}%
\pgfpathlineto{\pgfqpoint{4.222871in}{4.893948in}}%
\pgfusepath{stroke}%
\end{pgfscope}%
\begin{pgfscope}%
\pgfpathrectangle{\pgfqpoint{2.125000in}{4.080233in}}{\pgfqpoint{5.489583in}{0.877907in}}%
\pgfusepath{clip}%
\pgfsetbuttcap%
\pgfsetroundjoin%
\pgfsetlinewidth{1.505625pt}%
\definecolor{currentstroke}{rgb}{0.000000,0.000000,0.000000}%
\pgfsetstrokecolor{currentstroke}%
\pgfsetdash{}{0pt}%
\pgfpathmoveto{\pgfqpoint{4.346094in}{4.357721in}}%
\pgfpathlineto{\pgfqpoint{4.346094in}{4.892356in}}%
\pgfusepath{stroke}%
\end{pgfscope}%
\begin{pgfscope}%
\pgfpathrectangle{\pgfqpoint{2.125000in}{4.080233in}}{\pgfqpoint{5.489583in}{0.877907in}}%
\pgfusepath{clip}%
\pgfsetbuttcap%
\pgfsetroundjoin%
\pgfsetlinewidth{1.505625pt}%
\definecolor{currentstroke}{rgb}{0.000000,0.000000,0.000000}%
\pgfsetstrokecolor{currentstroke}%
\pgfsetdash{}{0pt}%
\pgfpathmoveto{\pgfqpoint{4.469317in}{4.357721in}}%
\pgfpathlineto{\pgfqpoint{4.469317in}{4.890790in}}%
\pgfusepath{stroke}%
\end{pgfscope}%
\begin{pgfscope}%
\pgfpathrectangle{\pgfqpoint{2.125000in}{4.080233in}}{\pgfqpoint{5.489583in}{0.877907in}}%
\pgfusepath{clip}%
\pgfsetbuttcap%
\pgfsetroundjoin%
\pgfsetlinewidth{1.505625pt}%
\definecolor{currentstroke}{rgb}{0.000000,0.000000,0.000000}%
\pgfsetstrokecolor{currentstroke}%
\pgfsetdash{}{0pt}%
\pgfpathmoveto{\pgfqpoint{4.592540in}{4.357721in}}%
\pgfpathlineto{\pgfqpoint{4.592540in}{4.889232in}}%
\pgfusepath{stroke}%
\end{pgfscope}%
\begin{pgfscope}%
\pgfpathrectangle{\pgfqpoint{2.125000in}{4.080233in}}{\pgfqpoint{5.489583in}{0.877907in}}%
\pgfusepath{clip}%
\pgfsetbuttcap%
\pgfsetroundjoin%
\pgfsetlinewidth{1.505625pt}%
\definecolor{currentstroke}{rgb}{0.000000,0.000000,0.000000}%
\pgfsetstrokecolor{currentstroke}%
\pgfsetdash{}{0pt}%
\pgfpathmoveto{\pgfqpoint{4.715763in}{4.357721in}}%
\pgfpathlineto{\pgfqpoint{4.715763in}{4.887672in}}%
\pgfusepath{stroke}%
\end{pgfscope}%
\begin{pgfscope}%
\pgfpathrectangle{\pgfqpoint{2.125000in}{4.080233in}}{\pgfqpoint{5.489583in}{0.877907in}}%
\pgfusepath{clip}%
\pgfsetbuttcap%
\pgfsetroundjoin%
\pgfsetlinewidth{1.505625pt}%
\definecolor{currentstroke}{rgb}{0.000000,0.000000,0.000000}%
\pgfsetstrokecolor{currentstroke}%
\pgfsetdash{}{0pt}%
\pgfpathmoveto{\pgfqpoint{4.838986in}{4.357721in}}%
\pgfpathlineto{\pgfqpoint{4.838986in}{4.886071in}}%
\pgfusepath{stroke}%
\end{pgfscope}%
\begin{pgfscope}%
\pgfpathrectangle{\pgfqpoint{2.125000in}{4.080233in}}{\pgfqpoint{5.489583in}{0.877907in}}%
\pgfusepath{clip}%
\pgfsetbuttcap%
\pgfsetroundjoin%
\pgfsetlinewidth{1.505625pt}%
\definecolor{currentstroke}{rgb}{0.000000,0.000000,0.000000}%
\pgfsetstrokecolor{currentstroke}%
\pgfsetdash{}{0pt}%
\pgfpathmoveto{\pgfqpoint{4.962209in}{4.357721in}}%
\pgfpathlineto{\pgfqpoint{4.962209in}{4.884462in}}%
\pgfusepath{stroke}%
\end{pgfscope}%
\begin{pgfscope}%
\pgfpathrectangle{\pgfqpoint{2.125000in}{4.080233in}}{\pgfqpoint{5.489583in}{0.877907in}}%
\pgfusepath{clip}%
\pgfsetbuttcap%
\pgfsetroundjoin%
\pgfsetlinewidth{1.505625pt}%
\definecolor{currentstroke}{rgb}{0.000000,0.000000,0.000000}%
\pgfsetstrokecolor{currentstroke}%
\pgfsetdash{}{0pt}%
\pgfpathmoveto{\pgfqpoint{5.085432in}{4.357721in}}%
\pgfpathlineto{\pgfqpoint{5.085432in}{4.882956in}}%
\pgfusepath{stroke}%
\end{pgfscope}%
\begin{pgfscope}%
\pgfpathrectangle{\pgfqpoint{2.125000in}{4.080233in}}{\pgfqpoint{5.489583in}{0.877907in}}%
\pgfusepath{clip}%
\pgfsetbuttcap%
\pgfsetroundjoin%
\pgfsetlinewidth{1.505625pt}%
\definecolor{currentstroke}{rgb}{0.000000,0.000000,0.000000}%
\pgfsetstrokecolor{currentstroke}%
\pgfsetdash{}{0pt}%
\pgfpathmoveto{\pgfqpoint{5.208655in}{4.357721in}}%
\pgfpathlineto{\pgfqpoint{5.208655in}{4.881433in}}%
\pgfusepath{stroke}%
\end{pgfscope}%
\begin{pgfscope}%
\pgfpathrectangle{\pgfqpoint{2.125000in}{4.080233in}}{\pgfqpoint{5.489583in}{0.877907in}}%
\pgfusepath{clip}%
\pgfsetbuttcap%
\pgfsetroundjoin%
\pgfsetlinewidth{1.505625pt}%
\definecolor{currentstroke}{rgb}{0.000000,0.000000,0.000000}%
\pgfsetstrokecolor{currentstroke}%
\pgfsetdash{}{0pt}%
\pgfpathmoveto{\pgfqpoint{5.331878in}{4.357721in}}%
\pgfpathlineto{\pgfqpoint{5.331878in}{4.879925in}}%
\pgfusepath{stroke}%
\end{pgfscope}%
\begin{pgfscope}%
\pgfpathrectangle{\pgfqpoint{2.125000in}{4.080233in}}{\pgfqpoint{5.489583in}{0.877907in}}%
\pgfusepath{clip}%
\pgfsetbuttcap%
\pgfsetroundjoin%
\pgfsetlinewidth{1.505625pt}%
\definecolor{currentstroke}{rgb}{0.000000,0.000000,0.000000}%
\pgfsetstrokecolor{currentstroke}%
\pgfsetdash{}{0pt}%
\pgfpathmoveto{\pgfqpoint{5.455101in}{4.357721in}}%
\pgfpathlineto{\pgfqpoint{5.455101in}{4.878413in}}%
\pgfusepath{stroke}%
\end{pgfscope}%
\begin{pgfscope}%
\pgfpathrectangle{\pgfqpoint{2.125000in}{4.080233in}}{\pgfqpoint{5.489583in}{0.877907in}}%
\pgfusepath{clip}%
\pgfsetbuttcap%
\pgfsetroundjoin%
\pgfsetlinewidth{1.505625pt}%
\definecolor{currentstroke}{rgb}{0.000000,0.000000,0.000000}%
\pgfsetstrokecolor{currentstroke}%
\pgfsetdash{}{0pt}%
\pgfpathmoveto{\pgfqpoint{5.578324in}{4.357721in}}%
\pgfpathlineto{\pgfqpoint{5.578324in}{4.876938in}}%
\pgfusepath{stroke}%
\end{pgfscope}%
\begin{pgfscope}%
\pgfpathrectangle{\pgfqpoint{2.125000in}{4.080233in}}{\pgfqpoint{5.489583in}{0.877907in}}%
\pgfusepath{clip}%
\pgfsetbuttcap%
\pgfsetroundjoin%
\pgfsetlinewidth{1.505625pt}%
\definecolor{currentstroke}{rgb}{0.000000,0.000000,0.000000}%
\pgfsetstrokecolor{currentstroke}%
\pgfsetdash{}{0pt}%
\pgfpathmoveto{\pgfqpoint{5.701547in}{4.357721in}}%
\pgfpathlineto{\pgfqpoint{5.701547in}{4.875525in}}%
\pgfusepath{stroke}%
\end{pgfscope}%
\begin{pgfscope}%
\pgfpathrectangle{\pgfqpoint{2.125000in}{4.080233in}}{\pgfqpoint{5.489583in}{0.877907in}}%
\pgfusepath{clip}%
\pgfsetbuttcap%
\pgfsetroundjoin%
\pgfsetlinewidth{1.505625pt}%
\definecolor{currentstroke}{rgb}{0.000000,0.000000,0.000000}%
\pgfsetstrokecolor{currentstroke}%
\pgfsetdash{}{0pt}%
\pgfpathmoveto{\pgfqpoint{5.824770in}{4.357721in}}%
\pgfpathlineto{\pgfqpoint{5.824770in}{4.874151in}}%
\pgfusepath{stroke}%
\end{pgfscope}%
\begin{pgfscope}%
\pgfpathrectangle{\pgfqpoint{2.125000in}{4.080233in}}{\pgfqpoint{5.489583in}{0.877907in}}%
\pgfusepath{clip}%
\pgfsetbuttcap%
\pgfsetroundjoin%
\pgfsetlinewidth{1.505625pt}%
\definecolor{currentstroke}{rgb}{0.000000,0.000000,0.000000}%
\pgfsetstrokecolor{currentstroke}%
\pgfsetdash{}{0pt}%
\pgfpathmoveto{\pgfqpoint{5.947993in}{4.357721in}}%
\pgfpathlineto{\pgfqpoint{5.947993in}{4.872767in}}%
\pgfusepath{stroke}%
\end{pgfscope}%
\begin{pgfscope}%
\pgfpathrectangle{\pgfqpoint{2.125000in}{4.080233in}}{\pgfqpoint{5.489583in}{0.877907in}}%
\pgfusepath{clip}%
\pgfsetbuttcap%
\pgfsetroundjoin%
\pgfsetlinewidth{1.505625pt}%
\definecolor{currentstroke}{rgb}{0.000000,0.000000,0.000000}%
\pgfsetstrokecolor{currentstroke}%
\pgfsetdash{}{0pt}%
\pgfpathmoveto{\pgfqpoint{6.071216in}{4.357721in}}%
\pgfpathlineto{\pgfqpoint{6.071216in}{4.871404in}}%
\pgfusepath{stroke}%
\end{pgfscope}%
\begin{pgfscope}%
\pgfpathrectangle{\pgfqpoint{2.125000in}{4.080233in}}{\pgfqpoint{5.489583in}{0.877907in}}%
\pgfusepath{clip}%
\pgfsetbuttcap%
\pgfsetroundjoin%
\pgfsetlinewidth{1.505625pt}%
\definecolor{currentstroke}{rgb}{0.000000,0.000000,0.000000}%
\pgfsetstrokecolor{currentstroke}%
\pgfsetdash{}{0pt}%
\pgfpathmoveto{\pgfqpoint{6.194439in}{4.357721in}}%
\pgfpathlineto{\pgfqpoint{6.194439in}{4.870086in}}%
\pgfusepath{stroke}%
\end{pgfscope}%
\begin{pgfscope}%
\pgfpathrectangle{\pgfqpoint{2.125000in}{4.080233in}}{\pgfqpoint{5.489583in}{0.877907in}}%
\pgfusepath{clip}%
\pgfsetbuttcap%
\pgfsetroundjoin%
\pgfsetlinewidth{1.505625pt}%
\definecolor{currentstroke}{rgb}{0.000000,0.000000,0.000000}%
\pgfsetstrokecolor{currentstroke}%
\pgfsetdash{}{0pt}%
\pgfpathmoveto{\pgfqpoint{6.317662in}{4.357721in}}%
\pgfpathlineto{\pgfqpoint{6.317662in}{4.868728in}}%
\pgfusepath{stroke}%
\end{pgfscope}%
\begin{pgfscope}%
\pgfpathrectangle{\pgfqpoint{2.125000in}{4.080233in}}{\pgfqpoint{5.489583in}{0.877907in}}%
\pgfusepath{clip}%
\pgfsetbuttcap%
\pgfsetroundjoin%
\pgfsetlinewidth{1.505625pt}%
\definecolor{currentstroke}{rgb}{0.000000,0.000000,0.000000}%
\pgfsetstrokecolor{currentstroke}%
\pgfsetdash{}{0pt}%
\pgfpathmoveto{\pgfqpoint{6.440885in}{4.357721in}}%
\pgfpathlineto{\pgfqpoint{6.440885in}{4.867418in}}%
\pgfusepath{stroke}%
\end{pgfscope}%
\begin{pgfscope}%
\pgfpathrectangle{\pgfqpoint{2.125000in}{4.080233in}}{\pgfqpoint{5.489583in}{0.877907in}}%
\pgfusepath{clip}%
\pgfsetbuttcap%
\pgfsetroundjoin%
\pgfsetlinewidth{1.505625pt}%
\definecolor{currentstroke}{rgb}{0.000000,0.000000,0.000000}%
\pgfsetstrokecolor{currentstroke}%
\pgfsetdash{}{0pt}%
\pgfpathmoveto{\pgfqpoint{6.564108in}{4.357721in}}%
\pgfpathlineto{\pgfqpoint{6.564108in}{4.866053in}}%
\pgfusepath{stroke}%
\end{pgfscope}%
\begin{pgfscope}%
\pgfpathrectangle{\pgfqpoint{2.125000in}{4.080233in}}{\pgfqpoint{5.489583in}{0.877907in}}%
\pgfusepath{clip}%
\pgfsetbuttcap%
\pgfsetroundjoin%
\pgfsetlinewidth{1.505625pt}%
\definecolor{currentstroke}{rgb}{0.000000,0.000000,0.000000}%
\pgfsetstrokecolor{currentstroke}%
\pgfsetdash{}{0pt}%
\pgfpathmoveto{\pgfqpoint{6.687330in}{4.357721in}}%
\pgfpathlineto{\pgfqpoint{6.687330in}{4.864713in}}%
\pgfusepath{stroke}%
\end{pgfscope}%
\begin{pgfscope}%
\pgfpathrectangle{\pgfqpoint{2.125000in}{4.080233in}}{\pgfqpoint{5.489583in}{0.877907in}}%
\pgfusepath{clip}%
\pgfsetbuttcap%
\pgfsetroundjoin%
\pgfsetlinewidth{1.505625pt}%
\definecolor{currentstroke}{rgb}{0.000000,0.000000,0.000000}%
\pgfsetstrokecolor{currentstroke}%
\pgfsetdash{}{0pt}%
\pgfpathmoveto{\pgfqpoint{6.810553in}{4.357721in}}%
\pgfpathlineto{\pgfqpoint{6.810553in}{4.863378in}}%
\pgfusepath{stroke}%
\end{pgfscope}%
\begin{pgfscope}%
\pgfpathrectangle{\pgfqpoint{2.125000in}{4.080233in}}{\pgfqpoint{5.489583in}{0.877907in}}%
\pgfusepath{clip}%
\pgfsetbuttcap%
\pgfsetroundjoin%
\pgfsetlinewidth{1.505625pt}%
\definecolor{currentstroke}{rgb}{0.000000,0.000000,0.000000}%
\pgfsetstrokecolor{currentstroke}%
\pgfsetdash{}{0pt}%
\pgfpathmoveto{\pgfqpoint{6.933776in}{4.357721in}}%
\pgfpathlineto{\pgfqpoint{6.933776in}{4.862067in}}%
\pgfusepath{stroke}%
\end{pgfscope}%
\begin{pgfscope}%
\pgfpathrectangle{\pgfqpoint{2.125000in}{4.080233in}}{\pgfqpoint{5.489583in}{0.877907in}}%
\pgfusepath{clip}%
\pgfsetbuttcap%
\pgfsetroundjoin%
\pgfsetlinewidth{1.505625pt}%
\definecolor{currentstroke}{rgb}{0.000000,0.000000,0.000000}%
\pgfsetstrokecolor{currentstroke}%
\pgfsetdash{}{0pt}%
\pgfpathmoveto{\pgfqpoint{7.056999in}{4.357721in}}%
\pgfpathlineto{\pgfqpoint{7.056999in}{4.860738in}}%
\pgfusepath{stroke}%
\end{pgfscope}%
\begin{pgfscope}%
\pgfpathrectangle{\pgfqpoint{2.125000in}{4.080233in}}{\pgfqpoint{5.489583in}{0.877907in}}%
\pgfusepath{clip}%
\pgfsetbuttcap%
\pgfsetroundjoin%
\pgfsetlinewidth{1.505625pt}%
\definecolor{currentstroke}{rgb}{0.000000,0.000000,0.000000}%
\pgfsetstrokecolor{currentstroke}%
\pgfsetdash{}{0pt}%
\pgfpathmoveto{\pgfqpoint{7.180222in}{4.357721in}}%
\pgfpathlineto{\pgfqpoint{7.180222in}{4.859445in}}%
\pgfusepath{stroke}%
\end{pgfscope}%
\begin{pgfscope}%
\pgfpathrectangle{\pgfqpoint{2.125000in}{4.080233in}}{\pgfqpoint{5.489583in}{0.877907in}}%
\pgfusepath{clip}%
\pgfsetbuttcap%
\pgfsetroundjoin%
\pgfsetlinewidth{1.505625pt}%
\definecolor{currentstroke}{rgb}{0.000000,0.000000,0.000000}%
\pgfsetstrokecolor{currentstroke}%
\pgfsetdash{}{0pt}%
\pgfpathmoveto{\pgfqpoint{7.303445in}{4.357721in}}%
\pgfpathlineto{\pgfqpoint{7.303445in}{4.858133in}}%
\pgfusepath{stroke}%
\end{pgfscope}%
\begin{pgfscope}%
\pgfpathrectangle{\pgfqpoint{2.125000in}{4.080233in}}{\pgfqpoint{5.489583in}{0.877907in}}%
\pgfusepath{clip}%
\pgfsetroundcap%
\pgfsetroundjoin%
\pgfsetlinewidth{1.505625pt}%
\definecolor{currentstroke}{rgb}{0.121569,0.466667,0.705882}%
\pgfsetstrokecolor{currentstroke}%
\pgfsetdash{}{0pt}%
\pgfpathmoveto{\pgfqpoint{2.125000in}{4.357721in}}%
\pgfpathlineto{\pgfqpoint{7.614583in}{4.357721in}}%
\pgfusepath{stroke}%
\end{pgfscope}%
\begin{pgfscope}%
\pgfpathrectangle{\pgfqpoint{2.125000in}{4.080233in}}{\pgfqpoint{5.489583in}{0.877907in}}%
\pgfusepath{clip}%
\pgfsetbuttcap%
\pgfsetroundjoin%
\definecolor{currentfill}{rgb}{0.121569,0.466667,0.705882}%
\pgfsetfillcolor{currentfill}%
\pgfsetlinewidth{1.003750pt}%
\definecolor{currentstroke}{rgb}{0.121569,0.466667,0.705882}%
\pgfsetstrokecolor{currentstroke}%
\pgfsetdash{}{0pt}%
\pgfsys@defobject{currentmarker}{\pgfqpoint{-0.034722in}{-0.034722in}}{\pgfqpoint{0.034722in}{0.034722in}}{%
\pgfpathmoveto{\pgfqpoint{0.000000in}{-0.034722in}}%
\pgfpathcurveto{\pgfqpoint{0.009208in}{-0.034722in}}{\pgfqpoint{0.018041in}{-0.031064in}}{\pgfqpoint{0.024552in}{-0.024552in}}%
\pgfpathcurveto{\pgfqpoint{0.031064in}{-0.018041in}}{\pgfqpoint{0.034722in}{-0.009208in}}{\pgfqpoint{0.034722in}{0.000000in}}%
\pgfpathcurveto{\pgfqpoint{0.034722in}{0.009208in}}{\pgfqpoint{0.031064in}{0.018041in}}{\pgfqpoint{0.024552in}{0.024552in}}%
\pgfpathcurveto{\pgfqpoint{0.018041in}{0.031064in}}{\pgfqpoint{0.009208in}{0.034722in}}{\pgfqpoint{0.000000in}{0.034722in}}%
\pgfpathcurveto{\pgfqpoint{-0.009208in}{0.034722in}}{\pgfqpoint{-0.018041in}{0.031064in}}{\pgfqpoint{-0.024552in}{0.024552in}}%
\pgfpathcurveto{\pgfqpoint{-0.031064in}{0.018041in}}{\pgfqpoint{-0.034722in}{0.009208in}}{\pgfqpoint{-0.034722in}{0.000000in}}%
\pgfpathcurveto{\pgfqpoint{-0.034722in}{-0.009208in}}{\pgfqpoint{-0.031064in}{-0.018041in}}{\pgfqpoint{-0.024552in}{-0.024552in}}%
\pgfpathcurveto{\pgfqpoint{-0.018041in}{-0.031064in}}{\pgfqpoint{-0.009208in}{-0.034722in}}{\pgfqpoint{0.000000in}{-0.034722in}}%
\pgfpathclose%
\pgfusepath{stroke,fill}%
}%
\begin{pgfscope}%
\pgfsys@transformshift{2.374527in}{4.918235in}%
\pgfsys@useobject{currentmarker}{}%
\end{pgfscope}%
\begin{pgfscope}%
\pgfsys@transformshift{2.497749in}{4.916588in}%
\pgfsys@useobject{currentmarker}{}%
\end{pgfscope}%
\begin{pgfscope}%
\pgfsys@transformshift{2.620972in}{4.914936in}%
\pgfsys@useobject{currentmarker}{}%
\end{pgfscope}%
\begin{pgfscope}%
\pgfsys@transformshift{2.744195in}{4.913320in}%
\pgfsys@useobject{currentmarker}{}%
\end{pgfscope}%
\begin{pgfscope}%
\pgfsys@transformshift{2.867418in}{4.911705in}%
\pgfsys@useobject{currentmarker}{}%
\end{pgfscope}%
\begin{pgfscope}%
\pgfsys@transformshift{2.990641in}{4.910102in}%
\pgfsys@useobject{currentmarker}{}%
\end{pgfscope}%
\begin{pgfscope}%
\pgfsys@transformshift{3.113864in}{4.908515in}%
\pgfsys@useobject{currentmarker}{}%
\end{pgfscope}%
\begin{pgfscope}%
\pgfsys@transformshift{3.237087in}{4.906909in}%
\pgfsys@useobject{currentmarker}{}%
\end{pgfscope}%
\begin{pgfscope}%
\pgfsys@transformshift{3.360310in}{4.905343in}%
\pgfsys@useobject{currentmarker}{}%
\end{pgfscope}%
\begin{pgfscope}%
\pgfsys@transformshift{3.483533in}{4.903709in}%
\pgfsys@useobject{currentmarker}{}%
\end{pgfscope}%
\begin{pgfscope}%
\pgfsys@transformshift{3.606756in}{4.902111in}%
\pgfsys@useobject{currentmarker}{}%
\end{pgfscope}%
\begin{pgfscope}%
\pgfsys@transformshift{3.729979in}{4.900528in}%
\pgfsys@useobject{currentmarker}{}%
\end{pgfscope}%
\begin{pgfscope}%
\pgfsys@transformshift{3.853202in}{4.898906in}%
\pgfsys@useobject{currentmarker}{}%
\end{pgfscope}%
\begin{pgfscope}%
\pgfsys@transformshift{3.976425in}{4.897248in}%
\pgfsys@useobject{currentmarker}{}%
\end{pgfscope}%
\begin{pgfscope}%
\pgfsys@transformshift{4.099648in}{4.895586in}%
\pgfsys@useobject{currentmarker}{}%
\end{pgfscope}%
\begin{pgfscope}%
\pgfsys@transformshift{4.222871in}{4.893948in}%
\pgfsys@useobject{currentmarker}{}%
\end{pgfscope}%
\begin{pgfscope}%
\pgfsys@transformshift{4.346094in}{4.892356in}%
\pgfsys@useobject{currentmarker}{}%
\end{pgfscope}%
\begin{pgfscope}%
\pgfsys@transformshift{4.469317in}{4.890790in}%
\pgfsys@useobject{currentmarker}{}%
\end{pgfscope}%
\begin{pgfscope}%
\pgfsys@transformshift{4.592540in}{4.889232in}%
\pgfsys@useobject{currentmarker}{}%
\end{pgfscope}%
\begin{pgfscope}%
\pgfsys@transformshift{4.715763in}{4.887672in}%
\pgfsys@useobject{currentmarker}{}%
\end{pgfscope}%
\begin{pgfscope}%
\pgfsys@transformshift{4.838986in}{4.886071in}%
\pgfsys@useobject{currentmarker}{}%
\end{pgfscope}%
\begin{pgfscope}%
\pgfsys@transformshift{4.962209in}{4.884462in}%
\pgfsys@useobject{currentmarker}{}%
\end{pgfscope}%
\begin{pgfscope}%
\pgfsys@transformshift{5.085432in}{4.882956in}%
\pgfsys@useobject{currentmarker}{}%
\end{pgfscope}%
\begin{pgfscope}%
\pgfsys@transformshift{5.208655in}{4.881433in}%
\pgfsys@useobject{currentmarker}{}%
\end{pgfscope}%
\begin{pgfscope}%
\pgfsys@transformshift{5.331878in}{4.879925in}%
\pgfsys@useobject{currentmarker}{}%
\end{pgfscope}%
\begin{pgfscope}%
\pgfsys@transformshift{5.455101in}{4.878413in}%
\pgfsys@useobject{currentmarker}{}%
\end{pgfscope}%
\begin{pgfscope}%
\pgfsys@transformshift{5.578324in}{4.876938in}%
\pgfsys@useobject{currentmarker}{}%
\end{pgfscope}%
\begin{pgfscope}%
\pgfsys@transformshift{5.701547in}{4.875525in}%
\pgfsys@useobject{currentmarker}{}%
\end{pgfscope}%
\begin{pgfscope}%
\pgfsys@transformshift{5.824770in}{4.874151in}%
\pgfsys@useobject{currentmarker}{}%
\end{pgfscope}%
\begin{pgfscope}%
\pgfsys@transformshift{5.947993in}{4.872767in}%
\pgfsys@useobject{currentmarker}{}%
\end{pgfscope}%
\begin{pgfscope}%
\pgfsys@transformshift{6.071216in}{4.871404in}%
\pgfsys@useobject{currentmarker}{}%
\end{pgfscope}%
\begin{pgfscope}%
\pgfsys@transformshift{6.194439in}{4.870086in}%
\pgfsys@useobject{currentmarker}{}%
\end{pgfscope}%
\begin{pgfscope}%
\pgfsys@transformshift{6.317662in}{4.868728in}%
\pgfsys@useobject{currentmarker}{}%
\end{pgfscope}%
\begin{pgfscope}%
\pgfsys@transformshift{6.440885in}{4.867418in}%
\pgfsys@useobject{currentmarker}{}%
\end{pgfscope}%
\begin{pgfscope}%
\pgfsys@transformshift{6.564108in}{4.866053in}%
\pgfsys@useobject{currentmarker}{}%
\end{pgfscope}%
\begin{pgfscope}%
\pgfsys@transformshift{6.687330in}{4.864713in}%
\pgfsys@useobject{currentmarker}{}%
\end{pgfscope}%
\begin{pgfscope}%
\pgfsys@transformshift{6.810553in}{4.863378in}%
\pgfsys@useobject{currentmarker}{}%
\end{pgfscope}%
\begin{pgfscope}%
\pgfsys@transformshift{6.933776in}{4.862067in}%
\pgfsys@useobject{currentmarker}{}%
\end{pgfscope}%
\begin{pgfscope}%
\pgfsys@transformshift{7.056999in}{4.860738in}%
\pgfsys@useobject{currentmarker}{}%
\end{pgfscope}%
\begin{pgfscope}%
\pgfsys@transformshift{7.180222in}{4.859445in}%
\pgfsys@useobject{currentmarker}{}%
\end{pgfscope}%
\begin{pgfscope}%
\pgfsys@transformshift{7.303445in}{4.858133in}%
\pgfsys@useobject{currentmarker}{}%
\end{pgfscope}%
\end{pgfscope}%
\begin{pgfscope}%
\pgfsetrectcap%
\pgfsetmiterjoin%
\pgfsetlinewidth{0.803000pt}%
\definecolor{currentstroke}{rgb}{1.000000,1.000000,1.000000}%
\pgfsetstrokecolor{currentstroke}%
\pgfsetdash{}{0pt}%
\pgfpathmoveto{\pgfqpoint{2.125000in}{4.080233in}}%
\pgfpathlineto{\pgfqpoint{2.125000in}{4.958140in}}%
\pgfusepath{stroke}%
\end{pgfscope}%
\begin{pgfscope}%
\pgfsetrectcap%
\pgfsetmiterjoin%
\pgfsetlinewidth{0.803000pt}%
\definecolor{currentstroke}{rgb}{1.000000,1.000000,1.000000}%
\pgfsetstrokecolor{currentstroke}%
\pgfsetdash{}{0pt}%
\pgfpathmoveto{\pgfqpoint{7.614583in}{4.080233in}}%
\pgfpathlineto{\pgfqpoint{7.614583in}{4.958140in}}%
\pgfusepath{stroke}%
\end{pgfscope}%
\begin{pgfscope}%
\pgfsetrectcap%
\pgfsetmiterjoin%
\pgfsetlinewidth{0.803000pt}%
\definecolor{currentstroke}{rgb}{1.000000,1.000000,1.000000}%
\pgfsetstrokecolor{currentstroke}%
\pgfsetdash{}{0pt}%
\pgfpathmoveto{\pgfqpoint{2.125000in}{4.080233in}}%
\pgfpathlineto{\pgfqpoint{7.614583in}{4.080233in}}%
\pgfusepath{stroke}%
\end{pgfscope}%
\begin{pgfscope}%
\pgfsetrectcap%
\pgfsetmiterjoin%
\pgfsetlinewidth{0.803000pt}%
\definecolor{currentstroke}{rgb}{1.000000,1.000000,1.000000}%
\pgfsetstrokecolor{currentstroke}%
\pgfsetdash{}{0pt}%
\pgfpathmoveto{\pgfqpoint{2.125000in}{4.958140in}}%
\pgfpathlineto{\pgfqpoint{7.614583in}{4.958140in}}%
\pgfusepath{stroke}%
\end{pgfscope}%
\begin{pgfscope}%
\definecolor{textcolor}{rgb}{0.150000,0.150000,0.150000}%
\pgfsetstrokecolor{textcolor}%
\pgfsetfillcolor{textcolor}%
\pgftext[x=4.869792in,y=5.041473in,,base]{\color{textcolor}\rmfamily\fontsize{16.800000}{20.160000}\selectfont Autocorrelation}%
\end{pgfscope}%
\begin{pgfscope}%
\pgfsetbuttcap%
\pgfsetmiterjoin%
\definecolor{currentfill}{rgb}{0.917647,0.917647,0.949020}%
\pgfsetfillcolor{currentfill}%
\pgfsetlinewidth{0.000000pt}%
\definecolor{currentstroke}{rgb}{0.000000,0.000000,0.000000}%
\pgfsetstrokecolor{currentstroke}%
\pgfsetstrokeopacity{0.000000}%
\pgfsetdash{}{0pt}%
\pgfpathmoveto{\pgfqpoint{9.810417in}{4.080233in}}%
\pgfpathlineto{\pgfqpoint{15.300000in}{4.080233in}}%
\pgfpathlineto{\pgfqpoint{15.300000in}{4.958140in}}%
\pgfpathlineto{\pgfqpoint{9.810417in}{4.958140in}}%
\pgfpathclose%
\pgfusepath{fill}%
\end{pgfscope}%
\begin{pgfscope}%
\pgfpathrectangle{\pgfqpoint{9.810417in}{4.080233in}}{\pgfqpoint{5.489583in}{0.877907in}}%
\pgfusepath{clip}%
\pgfsetroundcap%
\pgfsetroundjoin%
\pgfsetlinewidth{0.803000pt}%
\definecolor{currentstroke}{rgb}{1.000000,1.000000,1.000000}%
\pgfsetstrokecolor{currentstroke}%
\pgfsetdash{}{0pt}%
\pgfpathmoveto{\pgfqpoint{10.059943in}{4.080233in}}%
\pgfpathlineto{\pgfqpoint{10.059943in}{4.958140in}}%
\pgfusepath{stroke}%
\end{pgfscope}%
\begin{pgfscope}%
\definecolor{textcolor}{rgb}{0.150000,0.150000,0.150000}%
\pgfsetstrokecolor{textcolor}%
\pgfsetfillcolor{textcolor}%
\pgftext[x=10.059943in,y=3.983010in,,top]{\color{textcolor}\rmfamily\fontsize{14.000000}{16.800000}\selectfont 0}%
\end{pgfscope}%
\begin{pgfscope}%
\pgfpathrectangle{\pgfqpoint{9.810417in}{4.080233in}}{\pgfqpoint{5.489583in}{0.877907in}}%
\pgfusepath{clip}%
\pgfsetroundcap%
\pgfsetroundjoin%
\pgfsetlinewidth{0.803000pt}%
\definecolor{currentstroke}{rgb}{1.000000,1.000000,1.000000}%
\pgfsetstrokecolor{currentstroke}%
\pgfsetdash{}{0pt}%
\pgfpathmoveto{\pgfqpoint{10.676058in}{4.080233in}}%
\pgfpathlineto{\pgfqpoint{10.676058in}{4.958140in}}%
\pgfusepath{stroke}%
\end{pgfscope}%
\begin{pgfscope}%
\definecolor{textcolor}{rgb}{0.150000,0.150000,0.150000}%
\pgfsetstrokecolor{textcolor}%
\pgfsetfillcolor{textcolor}%
\pgftext[x=10.676058in,y=3.983010in,,top]{\color{textcolor}\rmfamily\fontsize{14.000000}{16.800000}\selectfont 5}%
\end{pgfscope}%
\begin{pgfscope}%
\pgfpathrectangle{\pgfqpoint{9.810417in}{4.080233in}}{\pgfqpoint{5.489583in}{0.877907in}}%
\pgfusepath{clip}%
\pgfsetroundcap%
\pgfsetroundjoin%
\pgfsetlinewidth{0.803000pt}%
\definecolor{currentstroke}{rgb}{1.000000,1.000000,1.000000}%
\pgfsetstrokecolor{currentstroke}%
\pgfsetdash{}{0pt}%
\pgfpathmoveto{\pgfqpoint{11.292173in}{4.080233in}}%
\pgfpathlineto{\pgfqpoint{11.292173in}{4.958140in}}%
\pgfusepath{stroke}%
\end{pgfscope}%
\begin{pgfscope}%
\definecolor{textcolor}{rgb}{0.150000,0.150000,0.150000}%
\pgfsetstrokecolor{textcolor}%
\pgfsetfillcolor{textcolor}%
\pgftext[x=11.292173in,y=3.983010in,,top]{\color{textcolor}\rmfamily\fontsize{14.000000}{16.800000}\selectfont 10}%
\end{pgfscope}%
\begin{pgfscope}%
\pgfpathrectangle{\pgfqpoint{9.810417in}{4.080233in}}{\pgfqpoint{5.489583in}{0.877907in}}%
\pgfusepath{clip}%
\pgfsetroundcap%
\pgfsetroundjoin%
\pgfsetlinewidth{0.803000pt}%
\definecolor{currentstroke}{rgb}{1.000000,1.000000,1.000000}%
\pgfsetstrokecolor{currentstroke}%
\pgfsetdash{}{0pt}%
\pgfpathmoveto{\pgfqpoint{11.908288in}{4.080233in}}%
\pgfpathlineto{\pgfqpoint{11.908288in}{4.958140in}}%
\pgfusepath{stroke}%
\end{pgfscope}%
\begin{pgfscope}%
\definecolor{textcolor}{rgb}{0.150000,0.150000,0.150000}%
\pgfsetstrokecolor{textcolor}%
\pgfsetfillcolor{textcolor}%
\pgftext[x=11.908288in,y=3.983010in,,top]{\color{textcolor}\rmfamily\fontsize{14.000000}{16.800000}\selectfont 15}%
\end{pgfscope}%
\begin{pgfscope}%
\pgfpathrectangle{\pgfqpoint{9.810417in}{4.080233in}}{\pgfqpoint{5.489583in}{0.877907in}}%
\pgfusepath{clip}%
\pgfsetroundcap%
\pgfsetroundjoin%
\pgfsetlinewidth{0.803000pt}%
\definecolor{currentstroke}{rgb}{1.000000,1.000000,1.000000}%
\pgfsetstrokecolor{currentstroke}%
\pgfsetdash{}{0pt}%
\pgfpathmoveto{\pgfqpoint{12.524403in}{4.080233in}}%
\pgfpathlineto{\pgfqpoint{12.524403in}{4.958140in}}%
\pgfusepath{stroke}%
\end{pgfscope}%
\begin{pgfscope}%
\definecolor{textcolor}{rgb}{0.150000,0.150000,0.150000}%
\pgfsetstrokecolor{textcolor}%
\pgfsetfillcolor{textcolor}%
\pgftext[x=12.524403in,y=3.983010in,,top]{\color{textcolor}\rmfamily\fontsize{14.000000}{16.800000}\selectfont 20}%
\end{pgfscope}%
\begin{pgfscope}%
\pgfpathrectangle{\pgfqpoint{9.810417in}{4.080233in}}{\pgfqpoint{5.489583in}{0.877907in}}%
\pgfusepath{clip}%
\pgfsetroundcap%
\pgfsetroundjoin%
\pgfsetlinewidth{0.803000pt}%
\definecolor{currentstroke}{rgb}{1.000000,1.000000,1.000000}%
\pgfsetstrokecolor{currentstroke}%
\pgfsetdash{}{0pt}%
\pgfpathmoveto{\pgfqpoint{13.140517in}{4.080233in}}%
\pgfpathlineto{\pgfqpoint{13.140517in}{4.958140in}}%
\pgfusepath{stroke}%
\end{pgfscope}%
\begin{pgfscope}%
\definecolor{textcolor}{rgb}{0.150000,0.150000,0.150000}%
\pgfsetstrokecolor{textcolor}%
\pgfsetfillcolor{textcolor}%
\pgftext[x=13.140517in,y=3.983010in,,top]{\color{textcolor}\rmfamily\fontsize{14.000000}{16.800000}\selectfont 25}%
\end{pgfscope}%
\begin{pgfscope}%
\pgfpathrectangle{\pgfqpoint{9.810417in}{4.080233in}}{\pgfqpoint{5.489583in}{0.877907in}}%
\pgfusepath{clip}%
\pgfsetroundcap%
\pgfsetroundjoin%
\pgfsetlinewidth{0.803000pt}%
\definecolor{currentstroke}{rgb}{1.000000,1.000000,1.000000}%
\pgfsetstrokecolor{currentstroke}%
\pgfsetdash{}{0pt}%
\pgfpathmoveto{\pgfqpoint{13.756632in}{4.080233in}}%
\pgfpathlineto{\pgfqpoint{13.756632in}{4.958140in}}%
\pgfusepath{stroke}%
\end{pgfscope}%
\begin{pgfscope}%
\definecolor{textcolor}{rgb}{0.150000,0.150000,0.150000}%
\pgfsetstrokecolor{textcolor}%
\pgfsetfillcolor{textcolor}%
\pgftext[x=13.756632in,y=3.983010in,,top]{\color{textcolor}\rmfamily\fontsize{14.000000}{16.800000}\selectfont 30}%
\end{pgfscope}%
\begin{pgfscope}%
\pgfpathrectangle{\pgfqpoint{9.810417in}{4.080233in}}{\pgfqpoint{5.489583in}{0.877907in}}%
\pgfusepath{clip}%
\pgfsetroundcap%
\pgfsetroundjoin%
\pgfsetlinewidth{0.803000pt}%
\definecolor{currentstroke}{rgb}{1.000000,1.000000,1.000000}%
\pgfsetstrokecolor{currentstroke}%
\pgfsetdash{}{0pt}%
\pgfpathmoveto{\pgfqpoint{14.372747in}{4.080233in}}%
\pgfpathlineto{\pgfqpoint{14.372747in}{4.958140in}}%
\pgfusepath{stroke}%
\end{pgfscope}%
\begin{pgfscope}%
\definecolor{textcolor}{rgb}{0.150000,0.150000,0.150000}%
\pgfsetstrokecolor{textcolor}%
\pgfsetfillcolor{textcolor}%
\pgftext[x=14.372747in,y=3.983010in,,top]{\color{textcolor}\rmfamily\fontsize{14.000000}{16.800000}\selectfont 35}%
\end{pgfscope}%
\begin{pgfscope}%
\pgfpathrectangle{\pgfqpoint{9.810417in}{4.080233in}}{\pgfqpoint{5.489583in}{0.877907in}}%
\pgfusepath{clip}%
\pgfsetroundcap%
\pgfsetroundjoin%
\pgfsetlinewidth{0.803000pt}%
\definecolor{currentstroke}{rgb}{1.000000,1.000000,1.000000}%
\pgfsetstrokecolor{currentstroke}%
\pgfsetdash{}{0pt}%
\pgfpathmoveto{\pgfqpoint{14.988862in}{4.080233in}}%
\pgfpathlineto{\pgfqpoint{14.988862in}{4.958140in}}%
\pgfusepath{stroke}%
\end{pgfscope}%
\begin{pgfscope}%
\definecolor{textcolor}{rgb}{0.150000,0.150000,0.150000}%
\pgfsetstrokecolor{textcolor}%
\pgfsetfillcolor{textcolor}%
\pgftext[x=14.988862in,y=3.983010in,,top]{\color{textcolor}\rmfamily\fontsize{14.000000}{16.800000}\selectfont 40}%
\end{pgfscope}%
\begin{pgfscope}%
\pgfpathrectangle{\pgfqpoint{9.810417in}{4.080233in}}{\pgfqpoint{5.489583in}{0.877907in}}%
\pgfusepath{clip}%
\pgfsetroundcap%
\pgfsetroundjoin%
\pgfsetlinewidth{0.803000pt}%
\definecolor{currentstroke}{rgb}{1.000000,1.000000,1.000000}%
\pgfsetstrokecolor{currentstroke}%
\pgfsetdash{}{0pt}%
\pgfpathmoveto{\pgfqpoint{9.810417in}{4.158471in}}%
\pgfpathlineto{\pgfqpoint{15.300000in}{4.158471in}}%
\pgfusepath{stroke}%
\end{pgfscope}%
\begin{pgfscope}%
\definecolor{textcolor}{rgb}{0.150000,0.150000,0.150000}%
\pgfsetstrokecolor{textcolor}%
\pgfsetfillcolor{textcolor}%
\pgftext[x=9.589483in,y=4.084605in,left,base]{\color{textcolor}\rmfamily\fontsize{14.000000}{16.800000}\selectfont 0}%
\end{pgfscope}%
\begin{pgfscope}%
\pgfpathrectangle{\pgfqpoint{9.810417in}{4.080233in}}{\pgfqpoint{5.489583in}{0.877907in}}%
\pgfusepath{clip}%
\pgfsetroundcap%
\pgfsetroundjoin%
\pgfsetlinewidth{0.803000pt}%
\definecolor{currentstroke}{rgb}{1.000000,1.000000,1.000000}%
\pgfsetstrokecolor{currentstroke}%
\pgfsetdash{}{0pt}%
\pgfpathmoveto{\pgfqpoint{9.810417in}{4.918235in}}%
\pgfpathlineto{\pgfqpoint{15.300000in}{4.918235in}}%
\pgfusepath{stroke}%
\end{pgfscope}%
\begin{pgfscope}%
\definecolor{textcolor}{rgb}{0.150000,0.150000,0.150000}%
\pgfsetstrokecolor{textcolor}%
\pgfsetfillcolor{textcolor}%
\pgftext[x=9.589483in,y=4.844369in,left,base]{\color{textcolor}\rmfamily\fontsize{14.000000}{16.800000}\selectfont 1}%
\end{pgfscope}%
\begin{pgfscope}%
\pgfpathrectangle{\pgfqpoint{9.810417in}{4.080233in}}{\pgfqpoint{5.489583in}{0.877907in}}%
\pgfusepath{clip}%
\pgfsetbuttcap%
\pgfsetroundjoin%
\definecolor{currentfill}{rgb}{0.121569,0.466667,0.705882}%
\pgfsetfillcolor{currentfill}%
\pgfsetfillopacity{0.250000}%
\pgfsetlinewidth{1.003750pt}%
\definecolor{currentstroke}{rgb}{1.000000,1.000000,1.000000}%
\pgfsetstrokecolor{currentstroke}%
\pgfsetstrokeopacity{0.250000}%
\pgfsetdash{}{0pt}%
\pgfpathmoveto{\pgfqpoint{10.121555in}{4.196805in}}%
\pgfpathlineto{\pgfqpoint{10.121555in}{4.120137in}}%
\pgfpathlineto{\pgfqpoint{10.306389in}{4.120137in}}%
\pgfpathlineto{\pgfqpoint{10.429612in}{4.120137in}}%
\pgfpathlineto{\pgfqpoint{10.552835in}{4.120137in}}%
\pgfpathlineto{\pgfqpoint{10.676058in}{4.120137in}}%
\pgfpathlineto{\pgfqpoint{10.799281in}{4.120137in}}%
\pgfpathlineto{\pgfqpoint{10.922504in}{4.120137in}}%
\pgfpathlineto{\pgfqpoint{11.045727in}{4.120137in}}%
\pgfpathlineto{\pgfqpoint{11.168950in}{4.120137in}}%
\pgfpathlineto{\pgfqpoint{11.292173in}{4.120137in}}%
\pgfpathlineto{\pgfqpoint{11.415396in}{4.120137in}}%
\pgfpathlineto{\pgfqpoint{11.538619in}{4.120137in}}%
\pgfpathlineto{\pgfqpoint{11.661842in}{4.120137in}}%
\pgfpathlineto{\pgfqpoint{11.785065in}{4.120137in}}%
\pgfpathlineto{\pgfqpoint{11.908288in}{4.120137in}}%
\pgfpathlineto{\pgfqpoint{12.031511in}{4.120137in}}%
\pgfpathlineto{\pgfqpoint{12.154734in}{4.120137in}}%
\pgfpathlineto{\pgfqpoint{12.277957in}{4.120137in}}%
\pgfpathlineto{\pgfqpoint{12.401180in}{4.120137in}}%
\pgfpathlineto{\pgfqpoint{12.524403in}{4.120137in}}%
\pgfpathlineto{\pgfqpoint{12.647626in}{4.120137in}}%
\pgfpathlineto{\pgfqpoint{12.770849in}{4.120137in}}%
\pgfpathlineto{\pgfqpoint{12.894072in}{4.120137in}}%
\pgfpathlineto{\pgfqpoint{13.017294in}{4.120137in}}%
\pgfpathlineto{\pgfqpoint{13.140517in}{4.120137in}}%
\pgfpathlineto{\pgfqpoint{13.263740in}{4.120137in}}%
\pgfpathlineto{\pgfqpoint{13.386963in}{4.120137in}}%
\pgfpathlineto{\pgfqpoint{13.510186in}{4.120137in}}%
\pgfpathlineto{\pgfqpoint{13.633409in}{4.120137in}}%
\pgfpathlineto{\pgfqpoint{13.756632in}{4.120137in}}%
\pgfpathlineto{\pgfqpoint{13.879855in}{4.120137in}}%
\pgfpathlineto{\pgfqpoint{14.003078in}{4.120137in}}%
\pgfpathlineto{\pgfqpoint{14.126301in}{4.120137in}}%
\pgfpathlineto{\pgfqpoint{14.249524in}{4.120137in}}%
\pgfpathlineto{\pgfqpoint{14.372747in}{4.120137in}}%
\pgfpathlineto{\pgfqpoint{14.495970in}{4.120137in}}%
\pgfpathlineto{\pgfqpoint{14.619193in}{4.120137in}}%
\pgfpathlineto{\pgfqpoint{14.742416in}{4.120137in}}%
\pgfpathlineto{\pgfqpoint{14.865639in}{4.120137in}}%
\pgfpathlineto{\pgfqpoint{15.050473in}{4.120137in}}%
\pgfpathlineto{\pgfqpoint{15.050473in}{4.196805in}}%
\pgfpathlineto{\pgfqpoint{15.050473in}{4.196805in}}%
\pgfpathlineto{\pgfqpoint{14.865639in}{4.196805in}}%
\pgfpathlineto{\pgfqpoint{14.742416in}{4.196805in}}%
\pgfpathlineto{\pgfqpoint{14.619193in}{4.196805in}}%
\pgfpathlineto{\pgfqpoint{14.495970in}{4.196805in}}%
\pgfpathlineto{\pgfqpoint{14.372747in}{4.196805in}}%
\pgfpathlineto{\pgfqpoint{14.249524in}{4.196805in}}%
\pgfpathlineto{\pgfqpoint{14.126301in}{4.196805in}}%
\pgfpathlineto{\pgfqpoint{14.003078in}{4.196805in}}%
\pgfpathlineto{\pgfqpoint{13.879855in}{4.196805in}}%
\pgfpathlineto{\pgfqpoint{13.756632in}{4.196805in}}%
\pgfpathlineto{\pgfqpoint{13.633409in}{4.196805in}}%
\pgfpathlineto{\pgfqpoint{13.510186in}{4.196805in}}%
\pgfpathlineto{\pgfqpoint{13.386963in}{4.196805in}}%
\pgfpathlineto{\pgfqpoint{13.263740in}{4.196805in}}%
\pgfpathlineto{\pgfqpoint{13.140517in}{4.196805in}}%
\pgfpathlineto{\pgfqpoint{13.017294in}{4.196805in}}%
\pgfpathlineto{\pgfqpoint{12.894072in}{4.196805in}}%
\pgfpathlineto{\pgfqpoint{12.770849in}{4.196805in}}%
\pgfpathlineto{\pgfqpoint{12.647626in}{4.196805in}}%
\pgfpathlineto{\pgfqpoint{12.524403in}{4.196805in}}%
\pgfpathlineto{\pgfqpoint{12.401180in}{4.196805in}}%
\pgfpathlineto{\pgfqpoint{12.277957in}{4.196805in}}%
\pgfpathlineto{\pgfqpoint{12.154734in}{4.196805in}}%
\pgfpathlineto{\pgfqpoint{12.031511in}{4.196805in}}%
\pgfpathlineto{\pgfqpoint{11.908288in}{4.196805in}}%
\pgfpathlineto{\pgfqpoint{11.785065in}{4.196805in}}%
\pgfpathlineto{\pgfqpoint{11.661842in}{4.196805in}}%
\pgfpathlineto{\pgfqpoint{11.538619in}{4.196805in}}%
\pgfpathlineto{\pgfqpoint{11.415396in}{4.196805in}}%
\pgfpathlineto{\pgfqpoint{11.292173in}{4.196805in}}%
\pgfpathlineto{\pgfqpoint{11.168950in}{4.196805in}}%
\pgfpathlineto{\pgfqpoint{11.045727in}{4.196805in}}%
\pgfpathlineto{\pgfqpoint{10.922504in}{4.196805in}}%
\pgfpathlineto{\pgfqpoint{10.799281in}{4.196805in}}%
\pgfpathlineto{\pgfqpoint{10.676058in}{4.196805in}}%
\pgfpathlineto{\pgfqpoint{10.552835in}{4.196805in}}%
\pgfpathlineto{\pgfqpoint{10.429612in}{4.196805in}}%
\pgfpathlineto{\pgfqpoint{10.306389in}{4.196805in}}%
\pgfpathlineto{\pgfqpoint{10.121555in}{4.196805in}}%
\pgfpathclose%
\pgfusepath{stroke,fill}%
\end{pgfscope}%
\begin{pgfscope}%
\pgfpathrectangle{\pgfqpoint{9.810417in}{4.080233in}}{\pgfqpoint{5.489583in}{0.877907in}}%
\pgfusepath{clip}%
\pgfsetbuttcap%
\pgfsetroundjoin%
\pgfsetlinewidth{1.505625pt}%
\definecolor{currentstroke}{rgb}{0.000000,0.000000,0.000000}%
\pgfsetstrokecolor{currentstroke}%
\pgfsetdash{}{0pt}%
\pgfpathmoveto{\pgfqpoint{10.059943in}{4.158471in}}%
\pgfpathlineto{\pgfqpoint{10.059943in}{4.918235in}}%
\pgfusepath{stroke}%
\end{pgfscope}%
\begin{pgfscope}%
\pgfpathrectangle{\pgfqpoint{9.810417in}{4.080233in}}{\pgfqpoint{5.489583in}{0.877907in}}%
\pgfusepath{clip}%
\pgfsetbuttcap%
\pgfsetroundjoin%
\pgfsetlinewidth{1.505625pt}%
\definecolor{currentstroke}{rgb}{0.000000,0.000000,0.000000}%
\pgfsetstrokecolor{currentstroke}%
\pgfsetdash{}{0pt}%
\pgfpathmoveto{\pgfqpoint{10.183166in}{4.158471in}}%
\pgfpathlineto{\pgfqpoint{10.183166in}{4.916505in}}%
\pgfusepath{stroke}%
\end{pgfscope}%
\begin{pgfscope}%
\pgfpathrectangle{\pgfqpoint{9.810417in}{4.080233in}}{\pgfqpoint{5.489583in}{0.877907in}}%
\pgfusepath{clip}%
\pgfsetbuttcap%
\pgfsetroundjoin%
\pgfsetlinewidth{1.505625pt}%
\definecolor{currentstroke}{rgb}{0.000000,0.000000,0.000000}%
\pgfsetstrokecolor{currentstroke}%
\pgfsetdash{}{0pt}%
\pgfpathmoveto{\pgfqpoint{10.306389in}{4.158471in}}%
\pgfpathlineto{\pgfqpoint{10.306389in}{4.155629in}}%
\pgfusepath{stroke}%
\end{pgfscope}%
\begin{pgfscope}%
\pgfpathrectangle{\pgfqpoint{9.810417in}{4.080233in}}{\pgfqpoint{5.489583in}{0.877907in}}%
\pgfusepath{clip}%
\pgfsetbuttcap%
\pgfsetroundjoin%
\pgfsetlinewidth{1.505625pt}%
\definecolor{currentstroke}{rgb}{0.000000,0.000000,0.000000}%
\pgfsetstrokecolor{currentstroke}%
\pgfsetdash{}{0pt}%
\pgfpathmoveto{\pgfqpoint{10.429612in}{4.158471in}}%
\pgfpathlineto{\pgfqpoint{10.429612in}{4.167532in}}%
\pgfusepath{stroke}%
\end{pgfscope}%
\begin{pgfscope}%
\pgfpathrectangle{\pgfqpoint{9.810417in}{4.080233in}}{\pgfqpoint{5.489583in}{0.877907in}}%
\pgfusepath{clip}%
\pgfsetbuttcap%
\pgfsetroundjoin%
\pgfsetlinewidth{1.505625pt}%
\definecolor{currentstroke}{rgb}{0.000000,0.000000,0.000000}%
\pgfsetstrokecolor{currentstroke}%
\pgfsetdash{}{0pt}%
\pgfpathmoveto{\pgfqpoint{10.552835in}{4.158471in}}%
\pgfpathlineto{\pgfqpoint{10.552835in}{4.157868in}}%
\pgfusepath{stroke}%
\end{pgfscope}%
\begin{pgfscope}%
\pgfpathrectangle{\pgfqpoint{9.810417in}{4.080233in}}{\pgfqpoint{5.489583in}{0.877907in}}%
\pgfusepath{clip}%
\pgfsetbuttcap%
\pgfsetroundjoin%
\pgfsetlinewidth{1.505625pt}%
\definecolor{currentstroke}{rgb}{0.000000,0.000000,0.000000}%
\pgfsetstrokecolor{currentstroke}%
\pgfsetdash{}{0pt}%
\pgfpathmoveto{\pgfqpoint{10.676058in}{4.158471in}}%
\pgfpathlineto{\pgfqpoint{10.676058in}{4.160529in}}%
\pgfusepath{stroke}%
\end{pgfscope}%
\begin{pgfscope}%
\pgfpathrectangle{\pgfqpoint{9.810417in}{4.080233in}}{\pgfqpoint{5.489583in}{0.877907in}}%
\pgfusepath{clip}%
\pgfsetbuttcap%
\pgfsetroundjoin%
\pgfsetlinewidth{1.505625pt}%
\definecolor{currentstroke}{rgb}{0.000000,0.000000,0.000000}%
\pgfsetstrokecolor{currentstroke}%
\pgfsetdash{}{0pt}%
\pgfpathmoveto{\pgfqpoint{10.799281in}{4.158471in}}%
\pgfpathlineto{\pgfqpoint{10.799281in}{4.161884in}}%
\pgfusepath{stroke}%
\end{pgfscope}%
\begin{pgfscope}%
\pgfpathrectangle{\pgfqpoint{9.810417in}{4.080233in}}{\pgfqpoint{5.489583in}{0.877907in}}%
\pgfusepath{clip}%
\pgfsetbuttcap%
\pgfsetroundjoin%
\pgfsetlinewidth{1.505625pt}%
\definecolor{currentstroke}{rgb}{0.000000,0.000000,0.000000}%
\pgfsetstrokecolor{currentstroke}%
\pgfsetdash{}{0pt}%
\pgfpathmoveto{\pgfqpoint{10.922504in}{4.158471in}}%
\pgfpathlineto{\pgfqpoint{10.922504in}{4.151472in}}%
\pgfusepath{stroke}%
\end{pgfscope}%
\begin{pgfscope}%
\pgfpathrectangle{\pgfqpoint{9.810417in}{4.080233in}}{\pgfqpoint{5.489583in}{0.877907in}}%
\pgfusepath{clip}%
\pgfsetbuttcap%
\pgfsetroundjoin%
\pgfsetlinewidth{1.505625pt}%
\definecolor{currentstroke}{rgb}{0.000000,0.000000,0.000000}%
\pgfsetstrokecolor{currentstroke}%
\pgfsetdash{}{0pt}%
\pgfpathmoveto{\pgfqpoint{11.045727in}{4.158471in}}%
\pgfpathlineto{\pgfqpoint{11.045727in}{4.169515in}}%
\pgfusepath{stroke}%
\end{pgfscope}%
\begin{pgfscope}%
\pgfpathrectangle{\pgfqpoint{9.810417in}{4.080233in}}{\pgfqpoint{5.489583in}{0.877907in}}%
\pgfusepath{clip}%
\pgfsetbuttcap%
\pgfsetroundjoin%
\pgfsetlinewidth{1.505625pt}%
\definecolor{currentstroke}{rgb}{0.000000,0.000000,0.000000}%
\pgfsetstrokecolor{currentstroke}%
\pgfsetdash{}{0pt}%
\pgfpathmoveto{\pgfqpoint{11.168950in}{4.158471in}}%
\pgfpathlineto{\pgfqpoint{11.168950in}{4.136413in}}%
\pgfusepath{stroke}%
\end{pgfscope}%
\begin{pgfscope}%
\pgfpathrectangle{\pgfqpoint{9.810417in}{4.080233in}}{\pgfqpoint{5.489583in}{0.877907in}}%
\pgfusepath{clip}%
\pgfsetbuttcap%
\pgfsetroundjoin%
\pgfsetlinewidth{1.505625pt}%
\definecolor{currentstroke}{rgb}{0.000000,0.000000,0.000000}%
\pgfsetstrokecolor{currentstroke}%
\pgfsetdash{}{0pt}%
\pgfpathmoveto{\pgfqpoint{11.292173in}{4.158471in}}%
\pgfpathlineto{\pgfqpoint{11.292173in}{4.168312in}}%
\pgfusepath{stroke}%
\end{pgfscope}%
\begin{pgfscope}%
\pgfpathrectangle{\pgfqpoint{9.810417in}{4.080233in}}{\pgfqpoint{5.489583in}{0.877907in}}%
\pgfusepath{clip}%
\pgfsetbuttcap%
\pgfsetroundjoin%
\pgfsetlinewidth{1.505625pt}%
\definecolor{currentstroke}{rgb}{0.000000,0.000000,0.000000}%
\pgfsetstrokecolor{currentstroke}%
\pgfsetdash{}{0pt}%
\pgfpathmoveto{\pgfqpoint{11.415396in}{4.158471in}}%
\pgfpathlineto{\pgfqpoint{11.415396in}{4.161032in}}%
\pgfusepath{stroke}%
\end{pgfscope}%
\begin{pgfscope}%
\pgfpathrectangle{\pgfqpoint{9.810417in}{4.080233in}}{\pgfqpoint{5.489583in}{0.877907in}}%
\pgfusepath{clip}%
\pgfsetbuttcap%
\pgfsetroundjoin%
\pgfsetlinewidth{1.505625pt}%
\definecolor{currentstroke}{rgb}{0.000000,0.000000,0.000000}%
\pgfsetstrokecolor{currentstroke}%
\pgfsetdash{}{0pt}%
\pgfpathmoveto{\pgfqpoint{11.538619in}{4.158471in}}%
\pgfpathlineto{\pgfqpoint{11.538619in}{4.145539in}}%
\pgfusepath{stroke}%
\end{pgfscope}%
\begin{pgfscope}%
\pgfpathrectangle{\pgfqpoint{9.810417in}{4.080233in}}{\pgfqpoint{5.489583in}{0.877907in}}%
\pgfusepath{clip}%
\pgfsetbuttcap%
\pgfsetroundjoin%
\pgfsetlinewidth{1.505625pt}%
\definecolor{currentstroke}{rgb}{0.000000,0.000000,0.000000}%
\pgfsetstrokecolor{currentstroke}%
\pgfsetdash{}{0pt}%
\pgfpathmoveto{\pgfqpoint{11.661842in}{4.158471in}}%
\pgfpathlineto{\pgfqpoint{11.661842in}{4.146648in}}%
\pgfusepath{stroke}%
\end{pgfscope}%
\begin{pgfscope}%
\pgfpathrectangle{\pgfqpoint{9.810417in}{4.080233in}}{\pgfqpoint{5.489583in}{0.877907in}}%
\pgfusepath{clip}%
\pgfsetbuttcap%
\pgfsetroundjoin%
\pgfsetlinewidth{1.505625pt}%
\definecolor{currentstroke}{rgb}{0.000000,0.000000,0.000000}%
\pgfsetstrokecolor{currentstroke}%
\pgfsetdash{}{0pt}%
\pgfpathmoveto{\pgfqpoint{11.785065in}{4.158471in}}%
\pgfpathlineto{\pgfqpoint{11.785065in}{4.155147in}}%
\pgfusepath{stroke}%
\end{pgfscope}%
\begin{pgfscope}%
\pgfpathrectangle{\pgfqpoint{9.810417in}{4.080233in}}{\pgfqpoint{5.489583in}{0.877907in}}%
\pgfusepath{clip}%
\pgfsetbuttcap%
\pgfsetroundjoin%
\pgfsetlinewidth{1.505625pt}%
\definecolor{currentstroke}{rgb}{0.000000,0.000000,0.000000}%
\pgfsetstrokecolor{currentstroke}%
\pgfsetdash{}{0pt}%
\pgfpathmoveto{\pgfqpoint{11.908288in}{4.158471in}}%
\pgfpathlineto{\pgfqpoint{11.908288in}{4.164772in}}%
\pgfusepath{stroke}%
\end{pgfscope}%
\begin{pgfscope}%
\pgfpathrectangle{\pgfqpoint{9.810417in}{4.080233in}}{\pgfqpoint{5.489583in}{0.877907in}}%
\pgfusepath{clip}%
\pgfsetbuttcap%
\pgfsetroundjoin%
\pgfsetlinewidth{1.505625pt}%
\definecolor{currentstroke}{rgb}{0.000000,0.000000,0.000000}%
\pgfsetstrokecolor{currentstroke}%
\pgfsetdash{}{0pt}%
\pgfpathmoveto{\pgfqpoint{12.031511in}{4.158471in}}%
\pgfpathlineto{\pgfqpoint{12.031511in}{4.170073in}}%
\pgfusepath{stroke}%
\end{pgfscope}%
\begin{pgfscope}%
\pgfpathrectangle{\pgfqpoint{9.810417in}{4.080233in}}{\pgfqpoint{5.489583in}{0.877907in}}%
\pgfusepath{clip}%
\pgfsetbuttcap%
\pgfsetroundjoin%
\pgfsetlinewidth{1.505625pt}%
\definecolor{currentstroke}{rgb}{0.000000,0.000000,0.000000}%
\pgfsetstrokecolor{currentstroke}%
\pgfsetdash{}{0pt}%
\pgfpathmoveto{\pgfqpoint{12.154734in}{4.158471in}}%
\pgfpathlineto{\pgfqpoint{12.154734in}{4.165321in}}%
\pgfusepath{stroke}%
\end{pgfscope}%
\begin{pgfscope}%
\pgfpathrectangle{\pgfqpoint{9.810417in}{4.080233in}}{\pgfqpoint{5.489583in}{0.877907in}}%
\pgfusepath{clip}%
\pgfsetbuttcap%
\pgfsetroundjoin%
\pgfsetlinewidth{1.505625pt}%
\definecolor{currentstroke}{rgb}{0.000000,0.000000,0.000000}%
\pgfsetstrokecolor{currentstroke}%
\pgfsetdash{}{0pt}%
\pgfpathmoveto{\pgfqpoint{12.277957in}{4.158471in}}%
\pgfpathlineto{\pgfqpoint{12.277957in}{4.159858in}}%
\pgfusepath{stroke}%
\end{pgfscope}%
\begin{pgfscope}%
\pgfpathrectangle{\pgfqpoint{9.810417in}{4.080233in}}{\pgfqpoint{5.489583in}{0.877907in}}%
\pgfusepath{clip}%
\pgfsetbuttcap%
\pgfsetroundjoin%
\pgfsetlinewidth{1.505625pt}%
\definecolor{currentstroke}{rgb}{0.000000,0.000000,0.000000}%
\pgfsetstrokecolor{currentstroke}%
\pgfsetdash{}{0pt}%
\pgfpathmoveto{\pgfqpoint{12.401180in}{4.158471in}}%
\pgfpathlineto{\pgfqpoint{12.401180in}{4.156250in}}%
\pgfusepath{stroke}%
\end{pgfscope}%
\begin{pgfscope}%
\pgfpathrectangle{\pgfqpoint{9.810417in}{4.080233in}}{\pgfqpoint{5.489583in}{0.877907in}}%
\pgfusepath{clip}%
\pgfsetbuttcap%
\pgfsetroundjoin%
\pgfsetlinewidth{1.505625pt}%
\definecolor{currentstroke}{rgb}{0.000000,0.000000,0.000000}%
\pgfsetstrokecolor{currentstroke}%
\pgfsetdash{}{0pt}%
\pgfpathmoveto{\pgfqpoint{12.524403in}{4.158471in}}%
\pgfpathlineto{\pgfqpoint{12.524403in}{4.145457in}}%
\pgfusepath{stroke}%
\end{pgfscope}%
\begin{pgfscope}%
\pgfpathrectangle{\pgfqpoint{9.810417in}{4.080233in}}{\pgfqpoint{5.489583in}{0.877907in}}%
\pgfusepath{clip}%
\pgfsetbuttcap%
\pgfsetroundjoin%
\pgfsetlinewidth{1.505625pt}%
\definecolor{currentstroke}{rgb}{0.000000,0.000000,0.000000}%
\pgfsetstrokecolor{currentstroke}%
\pgfsetdash{}{0pt}%
\pgfpathmoveto{\pgfqpoint{12.647626in}{4.158471in}}%
\pgfpathlineto{\pgfqpoint{12.647626in}{4.155106in}}%
\pgfusepath{stroke}%
\end{pgfscope}%
\begin{pgfscope}%
\pgfpathrectangle{\pgfqpoint{9.810417in}{4.080233in}}{\pgfqpoint{5.489583in}{0.877907in}}%
\pgfusepath{clip}%
\pgfsetbuttcap%
\pgfsetroundjoin%
\pgfsetlinewidth{1.505625pt}%
\definecolor{currentstroke}{rgb}{0.000000,0.000000,0.000000}%
\pgfsetstrokecolor{currentstroke}%
\pgfsetdash{}{0pt}%
\pgfpathmoveto{\pgfqpoint{12.770849in}{4.158471in}}%
\pgfpathlineto{\pgfqpoint{12.770849in}{4.187559in}}%
\pgfusepath{stroke}%
\end{pgfscope}%
\begin{pgfscope}%
\pgfpathrectangle{\pgfqpoint{9.810417in}{4.080233in}}{\pgfqpoint{5.489583in}{0.877907in}}%
\pgfusepath{clip}%
\pgfsetbuttcap%
\pgfsetroundjoin%
\pgfsetlinewidth{1.505625pt}%
\definecolor{currentstroke}{rgb}{0.000000,0.000000,0.000000}%
\pgfsetstrokecolor{currentstroke}%
\pgfsetdash{}{0pt}%
\pgfpathmoveto{\pgfqpoint{12.894072in}{4.158471in}}%
\pgfpathlineto{\pgfqpoint{12.894072in}{4.151936in}}%
\pgfusepath{stroke}%
\end{pgfscope}%
\begin{pgfscope}%
\pgfpathrectangle{\pgfqpoint{9.810417in}{4.080233in}}{\pgfqpoint{5.489583in}{0.877907in}}%
\pgfusepath{clip}%
\pgfsetbuttcap%
\pgfsetroundjoin%
\pgfsetlinewidth{1.505625pt}%
\definecolor{currentstroke}{rgb}{0.000000,0.000000,0.000000}%
\pgfsetstrokecolor{currentstroke}%
\pgfsetdash{}{0pt}%
\pgfpathmoveto{\pgfqpoint{13.017294in}{4.158471in}}%
\pgfpathlineto{\pgfqpoint{13.017294in}{4.162514in}}%
\pgfusepath{stroke}%
\end{pgfscope}%
\begin{pgfscope}%
\pgfpathrectangle{\pgfqpoint{9.810417in}{4.080233in}}{\pgfqpoint{5.489583in}{0.877907in}}%
\pgfusepath{clip}%
\pgfsetbuttcap%
\pgfsetroundjoin%
\pgfsetlinewidth{1.505625pt}%
\definecolor{currentstroke}{rgb}{0.000000,0.000000,0.000000}%
\pgfsetstrokecolor{currentstroke}%
\pgfsetdash{}{0pt}%
\pgfpathmoveto{\pgfqpoint{13.140517in}{4.158471in}}%
\pgfpathlineto{\pgfqpoint{13.140517in}{4.155848in}}%
\pgfusepath{stroke}%
\end{pgfscope}%
\begin{pgfscope}%
\pgfpathrectangle{\pgfqpoint{9.810417in}{4.080233in}}{\pgfqpoint{5.489583in}{0.877907in}}%
\pgfusepath{clip}%
\pgfsetbuttcap%
\pgfsetroundjoin%
\pgfsetlinewidth{1.505625pt}%
\definecolor{currentstroke}{rgb}{0.000000,0.000000,0.000000}%
\pgfsetstrokecolor{currentstroke}%
\pgfsetdash{}{0pt}%
\pgfpathmoveto{\pgfqpoint{13.263740in}{4.158471in}}%
\pgfpathlineto{\pgfqpoint{13.263740in}{4.168743in}}%
\pgfusepath{stroke}%
\end{pgfscope}%
\begin{pgfscope}%
\pgfpathrectangle{\pgfqpoint{9.810417in}{4.080233in}}{\pgfqpoint{5.489583in}{0.877907in}}%
\pgfusepath{clip}%
\pgfsetbuttcap%
\pgfsetroundjoin%
\pgfsetlinewidth{1.505625pt}%
\definecolor{currentstroke}{rgb}{0.000000,0.000000,0.000000}%
\pgfsetstrokecolor{currentstroke}%
\pgfsetdash{}{0pt}%
\pgfpathmoveto{\pgfqpoint{13.386963in}{4.158471in}}%
\pgfpathlineto{\pgfqpoint{13.386963in}{4.175848in}}%
\pgfusepath{stroke}%
\end{pgfscope}%
\begin{pgfscope}%
\pgfpathrectangle{\pgfqpoint{9.810417in}{4.080233in}}{\pgfqpoint{5.489583in}{0.877907in}}%
\pgfusepath{clip}%
\pgfsetbuttcap%
\pgfsetroundjoin%
\pgfsetlinewidth{1.505625pt}%
\definecolor{currentstroke}{rgb}{0.000000,0.000000,0.000000}%
\pgfsetstrokecolor{currentstroke}%
\pgfsetdash{}{0pt}%
\pgfpathmoveto{\pgfqpoint{13.510186in}{4.158471in}}%
\pgfpathlineto{\pgfqpoint{13.510186in}{4.168638in}}%
\pgfusepath{stroke}%
\end{pgfscope}%
\begin{pgfscope}%
\pgfpathrectangle{\pgfqpoint{9.810417in}{4.080233in}}{\pgfqpoint{5.489583in}{0.877907in}}%
\pgfusepath{clip}%
\pgfsetbuttcap%
\pgfsetroundjoin%
\pgfsetlinewidth{1.505625pt}%
\definecolor{currentstroke}{rgb}{0.000000,0.000000,0.000000}%
\pgfsetstrokecolor{currentstroke}%
\pgfsetdash{}{0pt}%
\pgfpathmoveto{\pgfqpoint{13.633409in}{4.158471in}}%
\pgfpathlineto{\pgfqpoint{13.633409in}{4.155242in}}%
\pgfusepath{stroke}%
\end{pgfscope}%
\begin{pgfscope}%
\pgfpathrectangle{\pgfqpoint{9.810417in}{4.080233in}}{\pgfqpoint{5.489583in}{0.877907in}}%
\pgfusepath{clip}%
\pgfsetbuttcap%
\pgfsetroundjoin%
\pgfsetlinewidth{1.505625pt}%
\definecolor{currentstroke}{rgb}{0.000000,0.000000,0.000000}%
\pgfsetstrokecolor{currentstroke}%
\pgfsetdash{}{0pt}%
\pgfpathmoveto{\pgfqpoint{13.756632in}{4.158471in}}%
\pgfpathlineto{\pgfqpoint{13.756632in}{4.161979in}}%
\pgfusepath{stroke}%
\end{pgfscope}%
\begin{pgfscope}%
\pgfpathrectangle{\pgfqpoint{9.810417in}{4.080233in}}{\pgfqpoint{5.489583in}{0.877907in}}%
\pgfusepath{clip}%
\pgfsetbuttcap%
\pgfsetroundjoin%
\pgfsetlinewidth{1.505625pt}%
\definecolor{currentstroke}{rgb}{0.000000,0.000000,0.000000}%
\pgfsetstrokecolor{currentstroke}%
\pgfsetdash{}{0pt}%
\pgfpathmoveto{\pgfqpoint{13.879855in}{4.158471in}}%
\pgfpathlineto{\pgfqpoint{13.879855in}{4.173305in}}%
\pgfusepath{stroke}%
\end{pgfscope}%
\begin{pgfscope}%
\pgfpathrectangle{\pgfqpoint{9.810417in}{4.080233in}}{\pgfqpoint{5.489583in}{0.877907in}}%
\pgfusepath{clip}%
\pgfsetbuttcap%
\pgfsetroundjoin%
\pgfsetlinewidth{1.505625pt}%
\definecolor{currentstroke}{rgb}{0.000000,0.000000,0.000000}%
\pgfsetstrokecolor{currentstroke}%
\pgfsetdash{}{0pt}%
\pgfpathmoveto{\pgfqpoint{14.003078in}{4.158471in}}%
\pgfpathlineto{\pgfqpoint{14.003078in}{4.146073in}}%
\pgfusepath{stroke}%
\end{pgfscope}%
\begin{pgfscope}%
\pgfpathrectangle{\pgfqpoint{9.810417in}{4.080233in}}{\pgfqpoint{5.489583in}{0.877907in}}%
\pgfusepath{clip}%
\pgfsetbuttcap%
\pgfsetroundjoin%
\pgfsetlinewidth{1.505625pt}%
\definecolor{currentstroke}{rgb}{0.000000,0.000000,0.000000}%
\pgfsetstrokecolor{currentstroke}%
\pgfsetdash{}{0pt}%
\pgfpathmoveto{\pgfqpoint{14.126301in}{4.158471in}}%
\pgfpathlineto{\pgfqpoint{14.126301in}{4.171898in}}%
\pgfusepath{stroke}%
\end{pgfscope}%
\begin{pgfscope}%
\pgfpathrectangle{\pgfqpoint{9.810417in}{4.080233in}}{\pgfqpoint{5.489583in}{0.877907in}}%
\pgfusepath{clip}%
\pgfsetbuttcap%
\pgfsetroundjoin%
\pgfsetlinewidth{1.505625pt}%
\definecolor{currentstroke}{rgb}{0.000000,0.000000,0.000000}%
\pgfsetstrokecolor{currentstroke}%
\pgfsetdash{}{0pt}%
\pgfpathmoveto{\pgfqpoint{14.249524in}{4.158471in}}%
\pgfpathlineto{\pgfqpoint{14.249524in}{4.139357in}}%
\pgfusepath{stroke}%
\end{pgfscope}%
\begin{pgfscope}%
\pgfpathrectangle{\pgfqpoint{9.810417in}{4.080233in}}{\pgfqpoint{5.489583in}{0.877907in}}%
\pgfusepath{clip}%
\pgfsetbuttcap%
\pgfsetroundjoin%
\pgfsetlinewidth{1.505625pt}%
\definecolor{currentstroke}{rgb}{0.000000,0.000000,0.000000}%
\pgfsetstrokecolor{currentstroke}%
\pgfsetdash{}{0pt}%
\pgfpathmoveto{\pgfqpoint{14.372747in}{4.158471in}}%
\pgfpathlineto{\pgfqpoint{14.372747in}{4.164160in}}%
\pgfusepath{stroke}%
\end{pgfscope}%
\begin{pgfscope}%
\pgfpathrectangle{\pgfqpoint{9.810417in}{4.080233in}}{\pgfqpoint{5.489583in}{0.877907in}}%
\pgfusepath{clip}%
\pgfsetbuttcap%
\pgfsetroundjoin%
\pgfsetlinewidth{1.505625pt}%
\definecolor{currentstroke}{rgb}{0.000000,0.000000,0.000000}%
\pgfsetstrokecolor{currentstroke}%
\pgfsetdash{}{0pt}%
\pgfpathmoveto{\pgfqpoint{14.495970in}{4.158471in}}%
\pgfpathlineto{\pgfqpoint{14.495970in}{4.158660in}}%
\pgfusepath{stroke}%
\end{pgfscope}%
\begin{pgfscope}%
\pgfpathrectangle{\pgfqpoint{9.810417in}{4.080233in}}{\pgfqpoint{5.489583in}{0.877907in}}%
\pgfusepath{clip}%
\pgfsetbuttcap%
\pgfsetroundjoin%
\pgfsetlinewidth{1.505625pt}%
\definecolor{currentstroke}{rgb}{0.000000,0.000000,0.000000}%
\pgfsetstrokecolor{currentstroke}%
\pgfsetdash{}{0pt}%
\pgfpathmoveto{\pgfqpoint{14.619193in}{4.158471in}}%
\pgfpathlineto{\pgfqpoint{14.619193in}{4.165907in}}%
\pgfusepath{stroke}%
\end{pgfscope}%
\begin{pgfscope}%
\pgfpathrectangle{\pgfqpoint{9.810417in}{4.080233in}}{\pgfqpoint{5.489583in}{0.877907in}}%
\pgfusepath{clip}%
\pgfsetbuttcap%
\pgfsetroundjoin%
\pgfsetlinewidth{1.505625pt}%
\definecolor{currentstroke}{rgb}{0.000000,0.000000,0.000000}%
\pgfsetstrokecolor{currentstroke}%
\pgfsetdash{}{0pt}%
\pgfpathmoveto{\pgfqpoint{14.742416in}{4.158471in}}%
\pgfpathlineto{\pgfqpoint{14.742416in}{4.151998in}}%
\pgfusepath{stroke}%
\end{pgfscope}%
\begin{pgfscope}%
\pgfpathrectangle{\pgfqpoint{9.810417in}{4.080233in}}{\pgfqpoint{5.489583in}{0.877907in}}%
\pgfusepath{clip}%
\pgfsetbuttcap%
\pgfsetroundjoin%
\pgfsetlinewidth{1.505625pt}%
\definecolor{currentstroke}{rgb}{0.000000,0.000000,0.000000}%
\pgfsetstrokecolor{currentstroke}%
\pgfsetdash{}{0pt}%
\pgfpathmoveto{\pgfqpoint{14.865639in}{4.158471in}}%
\pgfpathlineto{\pgfqpoint{14.865639in}{4.166518in}}%
\pgfusepath{stroke}%
\end{pgfscope}%
\begin{pgfscope}%
\pgfpathrectangle{\pgfqpoint{9.810417in}{4.080233in}}{\pgfqpoint{5.489583in}{0.877907in}}%
\pgfusepath{clip}%
\pgfsetbuttcap%
\pgfsetroundjoin%
\pgfsetlinewidth{1.505625pt}%
\definecolor{currentstroke}{rgb}{0.000000,0.000000,0.000000}%
\pgfsetstrokecolor{currentstroke}%
\pgfsetdash{}{0pt}%
\pgfpathmoveto{\pgfqpoint{14.988862in}{4.158471in}}%
\pgfpathlineto{\pgfqpoint{14.988862in}{4.152790in}}%
\pgfusepath{stroke}%
\end{pgfscope}%
\begin{pgfscope}%
\pgfpathrectangle{\pgfqpoint{9.810417in}{4.080233in}}{\pgfqpoint{5.489583in}{0.877907in}}%
\pgfusepath{clip}%
\pgfsetroundcap%
\pgfsetroundjoin%
\pgfsetlinewidth{1.505625pt}%
\definecolor{currentstroke}{rgb}{0.121569,0.466667,0.705882}%
\pgfsetstrokecolor{currentstroke}%
\pgfsetdash{}{0pt}%
\pgfpathmoveto{\pgfqpoint{9.810417in}{4.158471in}}%
\pgfpathlineto{\pgfqpoint{15.300000in}{4.158471in}}%
\pgfusepath{stroke}%
\end{pgfscope}%
\begin{pgfscope}%
\pgfpathrectangle{\pgfqpoint{9.810417in}{4.080233in}}{\pgfqpoint{5.489583in}{0.877907in}}%
\pgfusepath{clip}%
\pgfsetbuttcap%
\pgfsetroundjoin%
\definecolor{currentfill}{rgb}{0.121569,0.466667,0.705882}%
\pgfsetfillcolor{currentfill}%
\pgfsetlinewidth{1.003750pt}%
\definecolor{currentstroke}{rgb}{0.121569,0.466667,0.705882}%
\pgfsetstrokecolor{currentstroke}%
\pgfsetdash{}{0pt}%
\pgfsys@defobject{currentmarker}{\pgfqpoint{-0.034722in}{-0.034722in}}{\pgfqpoint{0.034722in}{0.034722in}}{%
\pgfpathmoveto{\pgfqpoint{0.000000in}{-0.034722in}}%
\pgfpathcurveto{\pgfqpoint{0.009208in}{-0.034722in}}{\pgfqpoint{0.018041in}{-0.031064in}}{\pgfqpoint{0.024552in}{-0.024552in}}%
\pgfpathcurveto{\pgfqpoint{0.031064in}{-0.018041in}}{\pgfqpoint{0.034722in}{-0.009208in}}{\pgfqpoint{0.034722in}{0.000000in}}%
\pgfpathcurveto{\pgfqpoint{0.034722in}{0.009208in}}{\pgfqpoint{0.031064in}{0.018041in}}{\pgfqpoint{0.024552in}{0.024552in}}%
\pgfpathcurveto{\pgfqpoint{0.018041in}{0.031064in}}{\pgfqpoint{0.009208in}{0.034722in}}{\pgfqpoint{0.000000in}{0.034722in}}%
\pgfpathcurveto{\pgfqpoint{-0.009208in}{0.034722in}}{\pgfqpoint{-0.018041in}{0.031064in}}{\pgfqpoint{-0.024552in}{0.024552in}}%
\pgfpathcurveto{\pgfqpoint{-0.031064in}{0.018041in}}{\pgfqpoint{-0.034722in}{0.009208in}}{\pgfqpoint{-0.034722in}{0.000000in}}%
\pgfpathcurveto{\pgfqpoint{-0.034722in}{-0.009208in}}{\pgfqpoint{-0.031064in}{-0.018041in}}{\pgfqpoint{-0.024552in}{-0.024552in}}%
\pgfpathcurveto{\pgfqpoint{-0.018041in}{-0.031064in}}{\pgfqpoint{-0.009208in}{-0.034722in}}{\pgfqpoint{0.000000in}{-0.034722in}}%
\pgfpathclose%
\pgfusepath{stroke,fill}%
}%
\begin{pgfscope}%
\pgfsys@transformshift{10.059943in}{4.918235in}%
\pgfsys@useobject{currentmarker}{}%
\end{pgfscope}%
\begin{pgfscope}%
\pgfsys@transformshift{10.183166in}{4.916505in}%
\pgfsys@useobject{currentmarker}{}%
\end{pgfscope}%
\begin{pgfscope}%
\pgfsys@transformshift{10.306389in}{4.155629in}%
\pgfsys@useobject{currentmarker}{}%
\end{pgfscope}%
\begin{pgfscope}%
\pgfsys@transformshift{10.429612in}{4.167532in}%
\pgfsys@useobject{currentmarker}{}%
\end{pgfscope}%
\begin{pgfscope}%
\pgfsys@transformshift{10.552835in}{4.157868in}%
\pgfsys@useobject{currentmarker}{}%
\end{pgfscope}%
\begin{pgfscope}%
\pgfsys@transformshift{10.676058in}{4.160529in}%
\pgfsys@useobject{currentmarker}{}%
\end{pgfscope}%
\begin{pgfscope}%
\pgfsys@transformshift{10.799281in}{4.161884in}%
\pgfsys@useobject{currentmarker}{}%
\end{pgfscope}%
\begin{pgfscope}%
\pgfsys@transformshift{10.922504in}{4.151472in}%
\pgfsys@useobject{currentmarker}{}%
\end{pgfscope}%
\begin{pgfscope}%
\pgfsys@transformshift{11.045727in}{4.169515in}%
\pgfsys@useobject{currentmarker}{}%
\end{pgfscope}%
\begin{pgfscope}%
\pgfsys@transformshift{11.168950in}{4.136413in}%
\pgfsys@useobject{currentmarker}{}%
\end{pgfscope}%
\begin{pgfscope}%
\pgfsys@transformshift{11.292173in}{4.168312in}%
\pgfsys@useobject{currentmarker}{}%
\end{pgfscope}%
\begin{pgfscope}%
\pgfsys@transformshift{11.415396in}{4.161032in}%
\pgfsys@useobject{currentmarker}{}%
\end{pgfscope}%
\begin{pgfscope}%
\pgfsys@transformshift{11.538619in}{4.145539in}%
\pgfsys@useobject{currentmarker}{}%
\end{pgfscope}%
\begin{pgfscope}%
\pgfsys@transformshift{11.661842in}{4.146648in}%
\pgfsys@useobject{currentmarker}{}%
\end{pgfscope}%
\begin{pgfscope}%
\pgfsys@transformshift{11.785065in}{4.155147in}%
\pgfsys@useobject{currentmarker}{}%
\end{pgfscope}%
\begin{pgfscope}%
\pgfsys@transformshift{11.908288in}{4.164772in}%
\pgfsys@useobject{currentmarker}{}%
\end{pgfscope}%
\begin{pgfscope}%
\pgfsys@transformshift{12.031511in}{4.170073in}%
\pgfsys@useobject{currentmarker}{}%
\end{pgfscope}%
\begin{pgfscope}%
\pgfsys@transformshift{12.154734in}{4.165321in}%
\pgfsys@useobject{currentmarker}{}%
\end{pgfscope}%
\begin{pgfscope}%
\pgfsys@transformshift{12.277957in}{4.159858in}%
\pgfsys@useobject{currentmarker}{}%
\end{pgfscope}%
\begin{pgfscope}%
\pgfsys@transformshift{12.401180in}{4.156250in}%
\pgfsys@useobject{currentmarker}{}%
\end{pgfscope}%
\begin{pgfscope}%
\pgfsys@transformshift{12.524403in}{4.145457in}%
\pgfsys@useobject{currentmarker}{}%
\end{pgfscope}%
\begin{pgfscope}%
\pgfsys@transformshift{12.647626in}{4.155106in}%
\pgfsys@useobject{currentmarker}{}%
\end{pgfscope}%
\begin{pgfscope}%
\pgfsys@transformshift{12.770849in}{4.187559in}%
\pgfsys@useobject{currentmarker}{}%
\end{pgfscope}%
\begin{pgfscope}%
\pgfsys@transformshift{12.894072in}{4.151936in}%
\pgfsys@useobject{currentmarker}{}%
\end{pgfscope}%
\begin{pgfscope}%
\pgfsys@transformshift{13.017294in}{4.162514in}%
\pgfsys@useobject{currentmarker}{}%
\end{pgfscope}%
\begin{pgfscope}%
\pgfsys@transformshift{13.140517in}{4.155848in}%
\pgfsys@useobject{currentmarker}{}%
\end{pgfscope}%
\begin{pgfscope}%
\pgfsys@transformshift{13.263740in}{4.168743in}%
\pgfsys@useobject{currentmarker}{}%
\end{pgfscope}%
\begin{pgfscope}%
\pgfsys@transformshift{13.386963in}{4.175848in}%
\pgfsys@useobject{currentmarker}{}%
\end{pgfscope}%
\begin{pgfscope}%
\pgfsys@transformshift{13.510186in}{4.168638in}%
\pgfsys@useobject{currentmarker}{}%
\end{pgfscope}%
\begin{pgfscope}%
\pgfsys@transformshift{13.633409in}{4.155242in}%
\pgfsys@useobject{currentmarker}{}%
\end{pgfscope}%
\begin{pgfscope}%
\pgfsys@transformshift{13.756632in}{4.161979in}%
\pgfsys@useobject{currentmarker}{}%
\end{pgfscope}%
\begin{pgfscope}%
\pgfsys@transformshift{13.879855in}{4.173305in}%
\pgfsys@useobject{currentmarker}{}%
\end{pgfscope}%
\begin{pgfscope}%
\pgfsys@transformshift{14.003078in}{4.146073in}%
\pgfsys@useobject{currentmarker}{}%
\end{pgfscope}%
\begin{pgfscope}%
\pgfsys@transformshift{14.126301in}{4.171898in}%
\pgfsys@useobject{currentmarker}{}%
\end{pgfscope}%
\begin{pgfscope}%
\pgfsys@transformshift{14.249524in}{4.139357in}%
\pgfsys@useobject{currentmarker}{}%
\end{pgfscope}%
\begin{pgfscope}%
\pgfsys@transformshift{14.372747in}{4.164160in}%
\pgfsys@useobject{currentmarker}{}%
\end{pgfscope}%
\begin{pgfscope}%
\pgfsys@transformshift{14.495970in}{4.158660in}%
\pgfsys@useobject{currentmarker}{}%
\end{pgfscope}%
\begin{pgfscope}%
\pgfsys@transformshift{14.619193in}{4.165907in}%
\pgfsys@useobject{currentmarker}{}%
\end{pgfscope}%
\begin{pgfscope}%
\pgfsys@transformshift{14.742416in}{4.151998in}%
\pgfsys@useobject{currentmarker}{}%
\end{pgfscope}%
\begin{pgfscope}%
\pgfsys@transformshift{14.865639in}{4.166518in}%
\pgfsys@useobject{currentmarker}{}%
\end{pgfscope}%
\begin{pgfscope}%
\pgfsys@transformshift{14.988862in}{4.152790in}%
\pgfsys@useobject{currentmarker}{}%
\end{pgfscope}%
\end{pgfscope}%
\begin{pgfscope}%
\pgfsetrectcap%
\pgfsetmiterjoin%
\pgfsetlinewidth{0.803000pt}%
\definecolor{currentstroke}{rgb}{1.000000,1.000000,1.000000}%
\pgfsetstrokecolor{currentstroke}%
\pgfsetdash{}{0pt}%
\pgfpathmoveto{\pgfqpoint{9.810417in}{4.080233in}}%
\pgfpathlineto{\pgfqpoint{9.810417in}{4.958140in}}%
\pgfusepath{stroke}%
\end{pgfscope}%
\begin{pgfscope}%
\pgfsetrectcap%
\pgfsetmiterjoin%
\pgfsetlinewidth{0.803000pt}%
\definecolor{currentstroke}{rgb}{1.000000,1.000000,1.000000}%
\pgfsetstrokecolor{currentstroke}%
\pgfsetdash{}{0pt}%
\pgfpathmoveto{\pgfqpoint{15.300000in}{4.080233in}}%
\pgfpathlineto{\pgfqpoint{15.300000in}{4.958140in}}%
\pgfusepath{stroke}%
\end{pgfscope}%
\begin{pgfscope}%
\pgfsetrectcap%
\pgfsetmiterjoin%
\pgfsetlinewidth{0.803000pt}%
\definecolor{currentstroke}{rgb}{1.000000,1.000000,1.000000}%
\pgfsetstrokecolor{currentstroke}%
\pgfsetdash{}{0pt}%
\pgfpathmoveto{\pgfqpoint{9.810417in}{4.080233in}}%
\pgfpathlineto{\pgfqpoint{15.300000in}{4.080233in}}%
\pgfusepath{stroke}%
\end{pgfscope}%
\begin{pgfscope}%
\pgfsetrectcap%
\pgfsetmiterjoin%
\pgfsetlinewidth{0.803000pt}%
\definecolor{currentstroke}{rgb}{1.000000,1.000000,1.000000}%
\pgfsetstrokecolor{currentstroke}%
\pgfsetdash{}{0pt}%
\pgfpathmoveto{\pgfqpoint{9.810417in}{4.958140in}}%
\pgfpathlineto{\pgfqpoint{15.300000in}{4.958140in}}%
\pgfusepath{stroke}%
\end{pgfscope}%
\begin{pgfscope}%
\definecolor{textcolor}{rgb}{0.150000,0.150000,0.150000}%
\pgfsetstrokecolor{textcolor}%
\pgfsetfillcolor{textcolor}%
\pgftext[x=12.555208in,y=5.041473in,,base]{\color{textcolor}\rmfamily\fontsize{16.800000}{20.160000}\selectfont Partial Autocorrelation}%
\end{pgfscope}%
\begin{pgfscope}%
\pgfsetbuttcap%
\pgfsetmiterjoin%
\definecolor{currentfill}{rgb}{0.917647,0.917647,0.949020}%
\pgfsetfillcolor{currentfill}%
\pgfsetlinewidth{0.000000pt}%
\definecolor{currentstroke}{rgb}{0.000000,0.000000,0.000000}%
\pgfsetstrokecolor{currentstroke}%
\pgfsetstrokeopacity{0.000000}%
\pgfsetdash{}{0pt}%
\pgfpathmoveto{\pgfqpoint{2.125000in}{2.500000in}}%
\pgfpathlineto{\pgfqpoint{7.614583in}{2.500000in}}%
\pgfpathlineto{\pgfqpoint{7.614583in}{3.377907in}}%
\pgfpathlineto{\pgfqpoint{2.125000in}{3.377907in}}%
\pgfpathclose%
\pgfusepath{fill}%
\end{pgfscope}%
\begin{pgfscope}%
\pgfpathrectangle{\pgfqpoint{2.125000in}{2.500000in}}{\pgfqpoint{5.489583in}{0.877907in}}%
\pgfusepath{clip}%
\pgfsetroundcap%
\pgfsetroundjoin%
\pgfsetlinewidth{0.803000pt}%
\definecolor{currentstroke}{rgb}{1.000000,1.000000,1.000000}%
\pgfsetstrokecolor{currentstroke}%
\pgfsetdash{}{0pt}%
\pgfpathmoveto{\pgfqpoint{2.374527in}{2.500000in}}%
\pgfpathlineto{\pgfqpoint{2.374527in}{3.377907in}}%
\pgfusepath{stroke}%
\end{pgfscope}%
\begin{pgfscope}%
\definecolor{textcolor}{rgb}{0.150000,0.150000,0.150000}%
\pgfsetstrokecolor{textcolor}%
\pgfsetfillcolor{textcolor}%
\pgftext[x=2.374527in,y=2.402778in,,top]{\color{textcolor}\rmfamily\fontsize{14.000000}{16.800000}\selectfont 0}%
\end{pgfscope}%
\begin{pgfscope}%
\pgfpathrectangle{\pgfqpoint{2.125000in}{2.500000in}}{\pgfqpoint{5.489583in}{0.877907in}}%
\pgfusepath{clip}%
\pgfsetroundcap%
\pgfsetroundjoin%
\pgfsetlinewidth{0.803000pt}%
\definecolor{currentstroke}{rgb}{1.000000,1.000000,1.000000}%
\pgfsetstrokecolor{currentstroke}%
\pgfsetdash{}{0pt}%
\pgfpathmoveto{\pgfqpoint{2.990641in}{2.500000in}}%
\pgfpathlineto{\pgfqpoint{2.990641in}{3.377907in}}%
\pgfusepath{stroke}%
\end{pgfscope}%
\begin{pgfscope}%
\definecolor{textcolor}{rgb}{0.150000,0.150000,0.150000}%
\pgfsetstrokecolor{textcolor}%
\pgfsetfillcolor{textcolor}%
\pgftext[x=2.990641in,y=2.402778in,,top]{\color{textcolor}\rmfamily\fontsize{14.000000}{16.800000}\selectfont 5}%
\end{pgfscope}%
\begin{pgfscope}%
\pgfpathrectangle{\pgfqpoint{2.125000in}{2.500000in}}{\pgfqpoint{5.489583in}{0.877907in}}%
\pgfusepath{clip}%
\pgfsetroundcap%
\pgfsetroundjoin%
\pgfsetlinewidth{0.803000pt}%
\definecolor{currentstroke}{rgb}{1.000000,1.000000,1.000000}%
\pgfsetstrokecolor{currentstroke}%
\pgfsetdash{}{0pt}%
\pgfpathmoveto{\pgfqpoint{3.606756in}{2.500000in}}%
\pgfpathlineto{\pgfqpoint{3.606756in}{3.377907in}}%
\pgfusepath{stroke}%
\end{pgfscope}%
\begin{pgfscope}%
\definecolor{textcolor}{rgb}{0.150000,0.150000,0.150000}%
\pgfsetstrokecolor{textcolor}%
\pgfsetfillcolor{textcolor}%
\pgftext[x=3.606756in,y=2.402778in,,top]{\color{textcolor}\rmfamily\fontsize{14.000000}{16.800000}\selectfont 10}%
\end{pgfscope}%
\begin{pgfscope}%
\pgfpathrectangle{\pgfqpoint{2.125000in}{2.500000in}}{\pgfqpoint{5.489583in}{0.877907in}}%
\pgfusepath{clip}%
\pgfsetroundcap%
\pgfsetroundjoin%
\pgfsetlinewidth{0.803000pt}%
\definecolor{currentstroke}{rgb}{1.000000,1.000000,1.000000}%
\pgfsetstrokecolor{currentstroke}%
\pgfsetdash{}{0pt}%
\pgfpathmoveto{\pgfqpoint{4.222871in}{2.500000in}}%
\pgfpathlineto{\pgfqpoint{4.222871in}{3.377907in}}%
\pgfusepath{stroke}%
\end{pgfscope}%
\begin{pgfscope}%
\definecolor{textcolor}{rgb}{0.150000,0.150000,0.150000}%
\pgfsetstrokecolor{textcolor}%
\pgfsetfillcolor{textcolor}%
\pgftext[x=4.222871in,y=2.402778in,,top]{\color{textcolor}\rmfamily\fontsize{14.000000}{16.800000}\selectfont 15}%
\end{pgfscope}%
\begin{pgfscope}%
\pgfpathrectangle{\pgfqpoint{2.125000in}{2.500000in}}{\pgfqpoint{5.489583in}{0.877907in}}%
\pgfusepath{clip}%
\pgfsetroundcap%
\pgfsetroundjoin%
\pgfsetlinewidth{0.803000pt}%
\definecolor{currentstroke}{rgb}{1.000000,1.000000,1.000000}%
\pgfsetstrokecolor{currentstroke}%
\pgfsetdash{}{0pt}%
\pgfpathmoveto{\pgfqpoint{4.838986in}{2.500000in}}%
\pgfpathlineto{\pgfqpoint{4.838986in}{3.377907in}}%
\pgfusepath{stroke}%
\end{pgfscope}%
\begin{pgfscope}%
\definecolor{textcolor}{rgb}{0.150000,0.150000,0.150000}%
\pgfsetstrokecolor{textcolor}%
\pgfsetfillcolor{textcolor}%
\pgftext[x=4.838986in,y=2.402778in,,top]{\color{textcolor}\rmfamily\fontsize{14.000000}{16.800000}\selectfont 20}%
\end{pgfscope}%
\begin{pgfscope}%
\pgfpathrectangle{\pgfqpoint{2.125000in}{2.500000in}}{\pgfqpoint{5.489583in}{0.877907in}}%
\pgfusepath{clip}%
\pgfsetroundcap%
\pgfsetroundjoin%
\pgfsetlinewidth{0.803000pt}%
\definecolor{currentstroke}{rgb}{1.000000,1.000000,1.000000}%
\pgfsetstrokecolor{currentstroke}%
\pgfsetdash{}{0pt}%
\pgfpathmoveto{\pgfqpoint{5.455101in}{2.500000in}}%
\pgfpathlineto{\pgfqpoint{5.455101in}{3.377907in}}%
\pgfusepath{stroke}%
\end{pgfscope}%
\begin{pgfscope}%
\definecolor{textcolor}{rgb}{0.150000,0.150000,0.150000}%
\pgfsetstrokecolor{textcolor}%
\pgfsetfillcolor{textcolor}%
\pgftext[x=5.455101in,y=2.402778in,,top]{\color{textcolor}\rmfamily\fontsize{14.000000}{16.800000}\selectfont 25}%
\end{pgfscope}%
\begin{pgfscope}%
\pgfpathrectangle{\pgfqpoint{2.125000in}{2.500000in}}{\pgfqpoint{5.489583in}{0.877907in}}%
\pgfusepath{clip}%
\pgfsetroundcap%
\pgfsetroundjoin%
\pgfsetlinewidth{0.803000pt}%
\definecolor{currentstroke}{rgb}{1.000000,1.000000,1.000000}%
\pgfsetstrokecolor{currentstroke}%
\pgfsetdash{}{0pt}%
\pgfpathmoveto{\pgfqpoint{6.071216in}{2.500000in}}%
\pgfpathlineto{\pgfqpoint{6.071216in}{3.377907in}}%
\pgfusepath{stroke}%
\end{pgfscope}%
\begin{pgfscope}%
\definecolor{textcolor}{rgb}{0.150000,0.150000,0.150000}%
\pgfsetstrokecolor{textcolor}%
\pgfsetfillcolor{textcolor}%
\pgftext[x=6.071216in,y=2.402778in,,top]{\color{textcolor}\rmfamily\fontsize{14.000000}{16.800000}\selectfont 30}%
\end{pgfscope}%
\begin{pgfscope}%
\pgfpathrectangle{\pgfqpoint{2.125000in}{2.500000in}}{\pgfqpoint{5.489583in}{0.877907in}}%
\pgfusepath{clip}%
\pgfsetroundcap%
\pgfsetroundjoin%
\pgfsetlinewidth{0.803000pt}%
\definecolor{currentstroke}{rgb}{1.000000,1.000000,1.000000}%
\pgfsetstrokecolor{currentstroke}%
\pgfsetdash{}{0pt}%
\pgfpathmoveto{\pgfqpoint{6.687330in}{2.500000in}}%
\pgfpathlineto{\pgfqpoint{6.687330in}{3.377907in}}%
\pgfusepath{stroke}%
\end{pgfscope}%
\begin{pgfscope}%
\definecolor{textcolor}{rgb}{0.150000,0.150000,0.150000}%
\pgfsetstrokecolor{textcolor}%
\pgfsetfillcolor{textcolor}%
\pgftext[x=6.687330in,y=2.402778in,,top]{\color{textcolor}\rmfamily\fontsize{14.000000}{16.800000}\selectfont 35}%
\end{pgfscope}%
\begin{pgfscope}%
\pgfpathrectangle{\pgfqpoint{2.125000in}{2.500000in}}{\pgfqpoint{5.489583in}{0.877907in}}%
\pgfusepath{clip}%
\pgfsetroundcap%
\pgfsetroundjoin%
\pgfsetlinewidth{0.803000pt}%
\definecolor{currentstroke}{rgb}{1.000000,1.000000,1.000000}%
\pgfsetstrokecolor{currentstroke}%
\pgfsetdash{}{0pt}%
\pgfpathmoveto{\pgfqpoint{7.303445in}{2.500000in}}%
\pgfpathlineto{\pgfqpoint{7.303445in}{3.377907in}}%
\pgfusepath{stroke}%
\end{pgfscope}%
\begin{pgfscope}%
\definecolor{textcolor}{rgb}{0.150000,0.150000,0.150000}%
\pgfsetstrokecolor{textcolor}%
\pgfsetfillcolor{textcolor}%
\pgftext[x=7.303445in,y=2.402778in,,top]{\color{textcolor}\rmfamily\fontsize{14.000000}{16.800000}\selectfont 40}%
\end{pgfscope}%
\begin{pgfscope}%
\pgfpathrectangle{\pgfqpoint{2.125000in}{2.500000in}}{\pgfqpoint{5.489583in}{0.877907in}}%
\pgfusepath{clip}%
\pgfsetroundcap%
\pgfsetroundjoin%
\pgfsetlinewidth{0.803000pt}%
\definecolor{currentstroke}{rgb}{1.000000,1.000000,1.000000}%
\pgfsetstrokecolor{currentstroke}%
\pgfsetdash{}{0pt}%
\pgfpathmoveto{\pgfqpoint{2.125000in}{2.778370in}}%
\pgfpathlineto{\pgfqpoint{7.614583in}{2.778370in}}%
\pgfusepath{stroke}%
\end{pgfscope}%
\begin{pgfscope}%
\definecolor{textcolor}{rgb}{0.150000,0.150000,0.150000}%
\pgfsetstrokecolor{textcolor}%
\pgfsetfillcolor{textcolor}%
\pgftext[x=1.904066in,y=2.704504in,left,base]{\color{textcolor}\rmfamily\fontsize{14.000000}{16.800000}\selectfont 0}%
\end{pgfscope}%
\begin{pgfscope}%
\pgfpathrectangle{\pgfqpoint{2.125000in}{2.500000in}}{\pgfqpoint{5.489583in}{0.877907in}}%
\pgfusepath{clip}%
\pgfsetroundcap%
\pgfsetroundjoin%
\pgfsetlinewidth{0.803000pt}%
\definecolor{currentstroke}{rgb}{1.000000,1.000000,1.000000}%
\pgfsetstrokecolor{currentstroke}%
\pgfsetdash{}{0pt}%
\pgfpathmoveto{\pgfqpoint{2.125000in}{3.338002in}}%
\pgfpathlineto{\pgfqpoint{7.614583in}{3.338002in}}%
\pgfusepath{stroke}%
\end{pgfscope}%
\begin{pgfscope}%
\definecolor{textcolor}{rgb}{0.150000,0.150000,0.150000}%
\pgfsetstrokecolor{textcolor}%
\pgfsetfillcolor{textcolor}%
\pgftext[x=1.904066in,y=3.264136in,left,base]{\color{textcolor}\rmfamily\fontsize{14.000000}{16.800000}\selectfont 1}%
\end{pgfscope}%
\begin{pgfscope}%
\pgfpathrectangle{\pgfqpoint{2.125000in}{2.500000in}}{\pgfqpoint{5.489583in}{0.877907in}}%
\pgfusepath{clip}%
\pgfsetbuttcap%
\pgfsetroundjoin%
\definecolor{currentfill}{rgb}{0.121569,0.466667,0.705882}%
\pgfsetfillcolor{currentfill}%
\pgfsetfillopacity{0.250000}%
\pgfsetlinewidth{1.003750pt}%
\definecolor{currentstroke}{rgb}{1.000000,1.000000,1.000000}%
\pgfsetstrokecolor{currentstroke}%
\pgfsetstrokeopacity{0.250000}%
\pgfsetdash{}{0pt}%
\pgfpathmoveto{\pgfqpoint{2.436138in}{2.806607in}}%
\pgfpathlineto{\pgfqpoint{2.436138in}{2.750134in}}%
\pgfpathlineto{\pgfqpoint{2.620972in}{2.729551in}}%
\pgfpathlineto{\pgfqpoint{2.744195in}{2.715434in}}%
\pgfpathlineto{\pgfqpoint{2.867418in}{2.704003in}}%
\pgfpathlineto{\pgfqpoint{2.990641in}{2.694155in}}%
\pgfpathlineto{\pgfqpoint{3.113864in}{2.685387in}}%
\pgfpathlineto{\pgfqpoint{3.237087in}{2.677415in}}%
\pgfpathlineto{\pgfqpoint{3.360310in}{2.670064in}}%
\pgfpathlineto{\pgfqpoint{3.483533in}{2.663218in}}%
\pgfpathlineto{\pgfqpoint{3.606756in}{2.656790in}}%
\pgfpathlineto{\pgfqpoint{3.729979in}{2.650719in}}%
\pgfpathlineto{\pgfqpoint{3.853202in}{2.644955in}}%
\pgfpathlineto{\pgfqpoint{3.976425in}{2.639458in}}%
\pgfpathlineto{\pgfqpoint{4.099648in}{2.634199in}}%
\pgfpathlineto{\pgfqpoint{4.222871in}{2.629152in}}%
\pgfpathlineto{\pgfqpoint{4.346094in}{2.624296in}}%
\pgfpathlineto{\pgfqpoint{4.469317in}{2.619611in}}%
\pgfpathlineto{\pgfqpoint{4.592540in}{2.615085in}}%
\pgfpathlineto{\pgfqpoint{4.715763in}{2.610705in}}%
\pgfpathlineto{\pgfqpoint{4.838986in}{2.606459in}}%
\pgfpathlineto{\pgfqpoint{4.962209in}{2.602339in}}%
\pgfpathlineto{\pgfqpoint{5.085432in}{2.598335in}}%
\pgfpathlineto{\pgfqpoint{5.208655in}{2.594439in}}%
\pgfpathlineto{\pgfqpoint{5.331878in}{2.590643in}}%
\pgfpathlineto{\pgfqpoint{5.455101in}{2.586940in}}%
\pgfpathlineto{\pgfqpoint{5.578324in}{2.583326in}}%
\pgfpathlineto{\pgfqpoint{5.701547in}{2.579795in}}%
\pgfpathlineto{\pgfqpoint{5.824770in}{2.576343in}}%
\pgfpathlineto{\pgfqpoint{5.947993in}{2.572967in}}%
\pgfpathlineto{\pgfqpoint{6.071216in}{2.569661in}}%
\pgfpathlineto{\pgfqpoint{6.194439in}{2.566424in}}%
\pgfpathlineto{\pgfqpoint{6.317662in}{2.563252in}}%
\pgfpathlineto{\pgfqpoint{6.440885in}{2.560142in}}%
\pgfpathlineto{\pgfqpoint{6.564108in}{2.557091in}}%
\pgfpathlineto{\pgfqpoint{6.687330in}{2.554097in}}%
\pgfpathlineto{\pgfqpoint{6.810553in}{2.551158in}}%
\pgfpathlineto{\pgfqpoint{6.933776in}{2.548272in}}%
\pgfpathlineto{\pgfqpoint{7.056999in}{2.545435in}}%
\pgfpathlineto{\pgfqpoint{7.180222in}{2.542647in}}%
\pgfpathlineto{\pgfqpoint{7.365057in}{2.539905in}}%
\pgfpathlineto{\pgfqpoint{7.365057in}{3.016836in}}%
\pgfpathlineto{\pgfqpoint{7.365057in}{3.016836in}}%
\pgfpathlineto{\pgfqpoint{7.180222in}{3.014094in}}%
\pgfpathlineto{\pgfqpoint{7.056999in}{3.011305in}}%
\pgfpathlineto{\pgfqpoint{6.933776in}{3.008469in}}%
\pgfpathlineto{\pgfqpoint{6.810553in}{3.005582in}}%
\pgfpathlineto{\pgfqpoint{6.687330in}{3.002643in}}%
\pgfpathlineto{\pgfqpoint{6.564108in}{2.999650in}}%
\pgfpathlineto{\pgfqpoint{6.440885in}{2.996599in}}%
\pgfpathlineto{\pgfqpoint{6.317662in}{2.993489in}}%
\pgfpathlineto{\pgfqpoint{6.194439in}{2.990317in}}%
\pgfpathlineto{\pgfqpoint{6.071216in}{2.987079in}}%
\pgfpathlineto{\pgfqpoint{5.947993in}{2.983774in}}%
\pgfpathlineto{\pgfqpoint{5.824770in}{2.980398in}}%
\pgfpathlineto{\pgfqpoint{5.701547in}{2.976946in}}%
\pgfpathlineto{\pgfqpoint{5.578324in}{2.973415in}}%
\pgfpathlineto{\pgfqpoint{5.455101in}{2.969801in}}%
\pgfpathlineto{\pgfqpoint{5.331878in}{2.966098in}}%
\pgfpathlineto{\pgfqpoint{5.208655in}{2.962302in}}%
\pgfpathlineto{\pgfqpoint{5.085432in}{2.958406in}}%
\pgfpathlineto{\pgfqpoint{4.962209in}{2.954401in}}%
\pgfpathlineto{\pgfqpoint{4.838986in}{2.950281in}}%
\pgfpathlineto{\pgfqpoint{4.715763in}{2.946036in}}%
\pgfpathlineto{\pgfqpoint{4.592540in}{2.941656in}}%
\pgfpathlineto{\pgfqpoint{4.469317in}{2.937129in}}%
\pgfpathlineto{\pgfqpoint{4.346094in}{2.932445in}}%
\pgfpathlineto{\pgfqpoint{4.222871in}{2.927589in}}%
\pgfpathlineto{\pgfqpoint{4.099648in}{2.922541in}}%
\pgfpathlineto{\pgfqpoint{3.976425in}{2.917282in}}%
\pgfpathlineto{\pgfqpoint{3.853202in}{2.911786in}}%
\pgfpathlineto{\pgfqpoint{3.729979in}{2.906022in}}%
\pgfpathlineto{\pgfqpoint{3.606756in}{2.899950in}}%
\pgfpathlineto{\pgfqpoint{3.483533in}{2.893523in}}%
\pgfpathlineto{\pgfqpoint{3.360310in}{2.886676in}}%
\pgfpathlineto{\pgfqpoint{3.237087in}{2.879326in}}%
\pgfpathlineto{\pgfqpoint{3.113864in}{2.871354in}}%
\pgfpathlineto{\pgfqpoint{2.990641in}{2.862586in}}%
\pgfpathlineto{\pgfqpoint{2.867418in}{2.852738in}}%
\pgfpathlineto{\pgfqpoint{2.744195in}{2.841307in}}%
\pgfpathlineto{\pgfqpoint{2.620972in}{2.827189in}}%
\pgfpathlineto{\pgfqpoint{2.436138in}{2.806607in}}%
\pgfpathclose%
\pgfusepath{stroke,fill}%
\end{pgfscope}%
\begin{pgfscope}%
\pgfpathrectangle{\pgfqpoint{2.125000in}{2.500000in}}{\pgfqpoint{5.489583in}{0.877907in}}%
\pgfusepath{clip}%
\pgfsetbuttcap%
\pgfsetroundjoin%
\pgfsetlinewidth{1.505625pt}%
\definecolor{currentstroke}{rgb}{0.000000,0.000000,0.000000}%
\pgfsetstrokecolor{currentstroke}%
\pgfsetdash{}{0pt}%
\pgfpathmoveto{\pgfqpoint{2.374527in}{2.778370in}}%
\pgfpathlineto{\pgfqpoint{2.374527in}{3.338002in}}%
\pgfusepath{stroke}%
\end{pgfscope}%
\begin{pgfscope}%
\pgfpathrectangle{\pgfqpoint{2.125000in}{2.500000in}}{\pgfqpoint{5.489583in}{0.877907in}}%
\pgfusepath{clip}%
\pgfsetbuttcap%
\pgfsetroundjoin%
\pgfsetlinewidth{1.505625pt}%
\definecolor{currentstroke}{rgb}{0.000000,0.000000,0.000000}%
\pgfsetstrokecolor{currentstroke}%
\pgfsetdash{}{0pt}%
\pgfpathmoveto{\pgfqpoint{2.497749in}{2.778370in}}%
\pgfpathlineto{\pgfqpoint{2.497749in}{3.336501in}}%
\pgfusepath{stroke}%
\end{pgfscope}%
\begin{pgfscope}%
\pgfpathrectangle{\pgfqpoint{2.125000in}{2.500000in}}{\pgfqpoint{5.489583in}{0.877907in}}%
\pgfusepath{clip}%
\pgfsetbuttcap%
\pgfsetroundjoin%
\pgfsetlinewidth{1.505625pt}%
\definecolor{currentstroke}{rgb}{0.000000,0.000000,0.000000}%
\pgfsetstrokecolor{currentstroke}%
\pgfsetdash{}{0pt}%
\pgfpathmoveto{\pgfqpoint{2.620972in}{2.778370in}}%
\pgfpathlineto{\pgfqpoint{2.620972in}{3.335030in}}%
\pgfusepath{stroke}%
\end{pgfscope}%
\begin{pgfscope}%
\pgfpathrectangle{\pgfqpoint{2.125000in}{2.500000in}}{\pgfqpoint{5.489583in}{0.877907in}}%
\pgfusepath{clip}%
\pgfsetbuttcap%
\pgfsetroundjoin%
\pgfsetlinewidth{1.505625pt}%
\definecolor{currentstroke}{rgb}{0.000000,0.000000,0.000000}%
\pgfsetstrokecolor{currentstroke}%
\pgfsetdash{}{0pt}%
\pgfpathmoveto{\pgfqpoint{2.744195in}{2.778370in}}%
\pgfpathlineto{\pgfqpoint{2.744195in}{3.333602in}}%
\pgfusepath{stroke}%
\end{pgfscope}%
\begin{pgfscope}%
\pgfpathrectangle{\pgfqpoint{2.125000in}{2.500000in}}{\pgfqpoint{5.489583in}{0.877907in}}%
\pgfusepath{clip}%
\pgfsetbuttcap%
\pgfsetroundjoin%
\pgfsetlinewidth{1.505625pt}%
\definecolor{currentstroke}{rgb}{0.000000,0.000000,0.000000}%
\pgfsetstrokecolor{currentstroke}%
\pgfsetdash{}{0pt}%
\pgfpathmoveto{\pgfqpoint{2.867418in}{2.778370in}}%
\pgfpathlineto{\pgfqpoint{2.867418in}{3.332197in}}%
\pgfusepath{stroke}%
\end{pgfscope}%
\begin{pgfscope}%
\pgfpathrectangle{\pgfqpoint{2.125000in}{2.500000in}}{\pgfqpoint{5.489583in}{0.877907in}}%
\pgfusepath{clip}%
\pgfsetbuttcap%
\pgfsetroundjoin%
\pgfsetlinewidth{1.505625pt}%
\definecolor{currentstroke}{rgb}{0.000000,0.000000,0.000000}%
\pgfsetstrokecolor{currentstroke}%
\pgfsetdash{}{0pt}%
\pgfpathmoveto{\pgfqpoint{2.990641in}{2.778370in}}%
\pgfpathlineto{\pgfqpoint{2.990641in}{3.330797in}}%
\pgfusepath{stroke}%
\end{pgfscope}%
\begin{pgfscope}%
\pgfpathrectangle{\pgfqpoint{2.125000in}{2.500000in}}{\pgfqpoint{5.489583in}{0.877907in}}%
\pgfusepath{clip}%
\pgfsetbuttcap%
\pgfsetroundjoin%
\pgfsetlinewidth{1.505625pt}%
\definecolor{currentstroke}{rgb}{0.000000,0.000000,0.000000}%
\pgfsetstrokecolor{currentstroke}%
\pgfsetdash{}{0pt}%
\pgfpathmoveto{\pgfqpoint{3.113864in}{2.778370in}}%
\pgfpathlineto{\pgfqpoint{3.113864in}{3.329429in}}%
\pgfusepath{stroke}%
\end{pgfscope}%
\begin{pgfscope}%
\pgfpathrectangle{\pgfqpoint{2.125000in}{2.500000in}}{\pgfqpoint{5.489583in}{0.877907in}}%
\pgfusepath{clip}%
\pgfsetbuttcap%
\pgfsetroundjoin%
\pgfsetlinewidth{1.505625pt}%
\definecolor{currentstroke}{rgb}{0.000000,0.000000,0.000000}%
\pgfsetstrokecolor{currentstroke}%
\pgfsetdash{}{0pt}%
\pgfpathmoveto{\pgfqpoint{3.237087in}{2.778370in}}%
\pgfpathlineto{\pgfqpoint{3.237087in}{3.328006in}}%
\pgfusepath{stroke}%
\end{pgfscope}%
\begin{pgfscope}%
\pgfpathrectangle{\pgfqpoint{2.125000in}{2.500000in}}{\pgfqpoint{5.489583in}{0.877907in}}%
\pgfusepath{clip}%
\pgfsetbuttcap%
\pgfsetroundjoin%
\pgfsetlinewidth{1.505625pt}%
\definecolor{currentstroke}{rgb}{0.000000,0.000000,0.000000}%
\pgfsetstrokecolor{currentstroke}%
\pgfsetdash{}{0pt}%
\pgfpathmoveto{\pgfqpoint{3.360310in}{2.778370in}}%
\pgfpathlineto{\pgfqpoint{3.360310in}{3.326544in}}%
\pgfusepath{stroke}%
\end{pgfscope}%
\begin{pgfscope}%
\pgfpathrectangle{\pgfqpoint{2.125000in}{2.500000in}}{\pgfqpoint{5.489583in}{0.877907in}}%
\pgfusepath{clip}%
\pgfsetbuttcap%
\pgfsetroundjoin%
\pgfsetlinewidth{1.505625pt}%
\definecolor{currentstroke}{rgb}{0.000000,0.000000,0.000000}%
\pgfsetstrokecolor{currentstroke}%
\pgfsetdash{}{0pt}%
\pgfpathmoveto{\pgfqpoint{3.483533in}{2.778370in}}%
\pgfpathlineto{\pgfqpoint{3.483533in}{3.325041in}}%
\pgfusepath{stroke}%
\end{pgfscope}%
\begin{pgfscope}%
\pgfpathrectangle{\pgfqpoint{2.125000in}{2.500000in}}{\pgfqpoint{5.489583in}{0.877907in}}%
\pgfusepath{clip}%
\pgfsetbuttcap%
\pgfsetroundjoin%
\pgfsetlinewidth{1.505625pt}%
\definecolor{currentstroke}{rgb}{0.000000,0.000000,0.000000}%
\pgfsetstrokecolor{currentstroke}%
\pgfsetdash{}{0pt}%
\pgfpathmoveto{\pgfqpoint{3.606756in}{2.778370in}}%
\pgfpathlineto{\pgfqpoint{3.606756in}{3.323537in}}%
\pgfusepath{stroke}%
\end{pgfscope}%
\begin{pgfscope}%
\pgfpathrectangle{\pgfqpoint{2.125000in}{2.500000in}}{\pgfqpoint{5.489583in}{0.877907in}}%
\pgfusepath{clip}%
\pgfsetbuttcap%
\pgfsetroundjoin%
\pgfsetlinewidth{1.505625pt}%
\definecolor{currentstroke}{rgb}{0.000000,0.000000,0.000000}%
\pgfsetstrokecolor{currentstroke}%
\pgfsetdash{}{0pt}%
\pgfpathmoveto{\pgfqpoint{3.729979in}{2.778370in}}%
\pgfpathlineto{\pgfqpoint{3.729979in}{3.322044in}}%
\pgfusepath{stroke}%
\end{pgfscope}%
\begin{pgfscope}%
\pgfpathrectangle{\pgfqpoint{2.125000in}{2.500000in}}{\pgfqpoint{5.489583in}{0.877907in}}%
\pgfusepath{clip}%
\pgfsetbuttcap%
\pgfsetroundjoin%
\pgfsetlinewidth{1.505625pt}%
\definecolor{currentstroke}{rgb}{0.000000,0.000000,0.000000}%
\pgfsetstrokecolor{currentstroke}%
\pgfsetdash{}{0pt}%
\pgfpathmoveto{\pgfqpoint{3.853202in}{2.778370in}}%
\pgfpathlineto{\pgfqpoint{3.853202in}{3.320573in}}%
\pgfusepath{stroke}%
\end{pgfscope}%
\begin{pgfscope}%
\pgfpathrectangle{\pgfqpoint{2.125000in}{2.500000in}}{\pgfqpoint{5.489583in}{0.877907in}}%
\pgfusepath{clip}%
\pgfsetbuttcap%
\pgfsetroundjoin%
\pgfsetlinewidth{1.505625pt}%
\definecolor{currentstroke}{rgb}{0.000000,0.000000,0.000000}%
\pgfsetstrokecolor{currentstroke}%
\pgfsetdash{}{0pt}%
\pgfpathmoveto{\pgfqpoint{3.976425in}{2.778370in}}%
\pgfpathlineto{\pgfqpoint{3.976425in}{3.319111in}}%
\pgfusepath{stroke}%
\end{pgfscope}%
\begin{pgfscope}%
\pgfpathrectangle{\pgfqpoint{2.125000in}{2.500000in}}{\pgfqpoint{5.489583in}{0.877907in}}%
\pgfusepath{clip}%
\pgfsetbuttcap%
\pgfsetroundjoin%
\pgfsetlinewidth{1.505625pt}%
\definecolor{currentstroke}{rgb}{0.000000,0.000000,0.000000}%
\pgfsetstrokecolor{currentstroke}%
\pgfsetdash{}{0pt}%
\pgfpathmoveto{\pgfqpoint{4.099648in}{2.778370in}}%
\pgfpathlineto{\pgfqpoint{4.099648in}{3.317665in}}%
\pgfusepath{stroke}%
\end{pgfscope}%
\begin{pgfscope}%
\pgfpathrectangle{\pgfqpoint{2.125000in}{2.500000in}}{\pgfqpoint{5.489583in}{0.877907in}}%
\pgfusepath{clip}%
\pgfsetbuttcap%
\pgfsetroundjoin%
\pgfsetlinewidth{1.505625pt}%
\definecolor{currentstroke}{rgb}{0.000000,0.000000,0.000000}%
\pgfsetstrokecolor{currentstroke}%
\pgfsetdash{}{0pt}%
\pgfpathmoveto{\pgfqpoint{4.222871in}{2.778370in}}%
\pgfpathlineto{\pgfqpoint{4.222871in}{3.316243in}}%
\pgfusepath{stroke}%
\end{pgfscope}%
\begin{pgfscope}%
\pgfpathrectangle{\pgfqpoint{2.125000in}{2.500000in}}{\pgfqpoint{5.489583in}{0.877907in}}%
\pgfusepath{clip}%
\pgfsetbuttcap%
\pgfsetroundjoin%
\pgfsetlinewidth{1.505625pt}%
\definecolor{currentstroke}{rgb}{0.000000,0.000000,0.000000}%
\pgfsetstrokecolor{currentstroke}%
\pgfsetdash{}{0pt}%
\pgfpathmoveto{\pgfqpoint{4.346094in}{2.778370in}}%
\pgfpathlineto{\pgfqpoint{4.346094in}{3.314848in}}%
\pgfusepath{stroke}%
\end{pgfscope}%
\begin{pgfscope}%
\pgfpathrectangle{\pgfqpoint{2.125000in}{2.500000in}}{\pgfqpoint{5.489583in}{0.877907in}}%
\pgfusepath{clip}%
\pgfsetbuttcap%
\pgfsetroundjoin%
\pgfsetlinewidth{1.505625pt}%
\definecolor{currentstroke}{rgb}{0.000000,0.000000,0.000000}%
\pgfsetstrokecolor{currentstroke}%
\pgfsetdash{}{0pt}%
\pgfpathmoveto{\pgfqpoint{4.469317in}{2.778370in}}%
\pgfpathlineto{\pgfqpoint{4.469317in}{3.313441in}}%
\pgfusepath{stroke}%
\end{pgfscope}%
\begin{pgfscope}%
\pgfpathrectangle{\pgfqpoint{2.125000in}{2.500000in}}{\pgfqpoint{5.489583in}{0.877907in}}%
\pgfusepath{clip}%
\pgfsetbuttcap%
\pgfsetroundjoin%
\pgfsetlinewidth{1.505625pt}%
\definecolor{currentstroke}{rgb}{0.000000,0.000000,0.000000}%
\pgfsetstrokecolor{currentstroke}%
\pgfsetdash{}{0pt}%
\pgfpathmoveto{\pgfqpoint{4.592540in}{2.778370in}}%
\pgfpathlineto{\pgfqpoint{4.592540in}{3.311986in}}%
\pgfusepath{stroke}%
\end{pgfscope}%
\begin{pgfscope}%
\pgfpathrectangle{\pgfqpoint{2.125000in}{2.500000in}}{\pgfqpoint{5.489583in}{0.877907in}}%
\pgfusepath{clip}%
\pgfsetbuttcap%
\pgfsetroundjoin%
\pgfsetlinewidth{1.505625pt}%
\definecolor{currentstroke}{rgb}{0.000000,0.000000,0.000000}%
\pgfsetstrokecolor{currentstroke}%
\pgfsetdash{}{0pt}%
\pgfpathmoveto{\pgfqpoint{4.715763in}{2.778370in}}%
\pgfpathlineto{\pgfqpoint{4.715763in}{3.310478in}}%
\pgfusepath{stroke}%
\end{pgfscope}%
\begin{pgfscope}%
\pgfpathrectangle{\pgfqpoint{2.125000in}{2.500000in}}{\pgfqpoint{5.489583in}{0.877907in}}%
\pgfusepath{clip}%
\pgfsetbuttcap%
\pgfsetroundjoin%
\pgfsetlinewidth{1.505625pt}%
\definecolor{currentstroke}{rgb}{0.000000,0.000000,0.000000}%
\pgfsetstrokecolor{currentstroke}%
\pgfsetdash{}{0pt}%
\pgfpathmoveto{\pgfqpoint{4.838986in}{2.778370in}}%
\pgfpathlineto{\pgfqpoint{4.838986in}{3.308997in}}%
\pgfusepath{stroke}%
\end{pgfscope}%
\begin{pgfscope}%
\pgfpathrectangle{\pgfqpoint{2.125000in}{2.500000in}}{\pgfqpoint{5.489583in}{0.877907in}}%
\pgfusepath{clip}%
\pgfsetbuttcap%
\pgfsetroundjoin%
\pgfsetlinewidth{1.505625pt}%
\definecolor{currentstroke}{rgb}{0.000000,0.000000,0.000000}%
\pgfsetstrokecolor{currentstroke}%
\pgfsetdash{}{0pt}%
\pgfpathmoveto{\pgfqpoint{4.962209in}{2.778370in}}%
\pgfpathlineto{\pgfqpoint{4.962209in}{3.307567in}}%
\pgfusepath{stroke}%
\end{pgfscope}%
\begin{pgfscope}%
\pgfpathrectangle{\pgfqpoint{2.125000in}{2.500000in}}{\pgfqpoint{5.489583in}{0.877907in}}%
\pgfusepath{clip}%
\pgfsetbuttcap%
\pgfsetroundjoin%
\pgfsetlinewidth{1.505625pt}%
\definecolor{currentstroke}{rgb}{0.000000,0.000000,0.000000}%
\pgfsetstrokecolor{currentstroke}%
\pgfsetdash{}{0pt}%
\pgfpathmoveto{\pgfqpoint{5.085432in}{2.778370in}}%
\pgfpathlineto{\pgfqpoint{5.085432in}{3.306120in}}%
\pgfusepath{stroke}%
\end{pgfscope}%
\begin{pgfscope}%
\pgfpathrectangle{\pgfqpoint{2.125000in}{2.500000in}}{\pgfqpoint{5.489583in}{0.877907in}}%
\pgfusepath{clip}%
\pgfsetbuttcap%
\pgfsetroundjoin%
\pgfsetlinewidth{1.505625pt}%
\definecolor{currentstroke}{rgb}{0.000000,0.000000,0.000000}%
\pgfsetstrokecolor{currentstroke}%
\pgfsetdash{}{0pt}%
\pgfpathmoveto{\pgfqpoint{5.208655in}{2.778370in}}%
\pgfpathlineto{\pgfqpoint{5.208655in}{3.304781in}}%
\pgfusepath{stroke}%
\end{pgfscope}%
\begin{pgfscope}%
\pgfpathrectangle{\pgfqpoint{2.125000in}{2.500000in}}{\pgfqpoint{5.489583in}{0.877907in}}%
\pgfusepath{clip}%
\pgfsetbuttcap%
\pgfsetroundjoin%
\pgfsetlinewidth{1.505625pt}%
\definecolor{currentstroke}{rgb}{0.000000,0.000000,0.000000}%
\pgfsetstrokecolor{currentstroke}%
\pgfsetdash{}{0pt}%
\pgfpathmoveto{\pgfqpoint{5.331878in}{2.778370in}}%
\pgfpathlineto{\pgfqpoint{5.331878in}{3.303469in}}%
\pgfusepath{stroke}%
\end{pgfscope}%
\begin{pgfscope}%
\pgfpathrectangle{\pgfqpoint{2.125000in}{2.500000in}}{\pgfqpoint{5.489583in}{0.877907in}}%
\pgfusepath{clip}%
\pgfsetbuttcap%
\pgfsetroundjoin%
\pgfsetlinewidth{1.505625pt}%
\definecolor{currentstroke}{rgb}{0.000000,0.000000,0.000000}%
\pgfsetstrokecolor{currentstroke}%
\pgfsetdash{}{0pt}%
\pgfpathmoveto{\pgfqpoint{5.455101in}{2.778370in}}%
\pgfpathlineto{\pgfqpoint{5.455101in}{3.302157in}}%
\pgfusepath{stroke}%
\end{pgfscope}%
\begin{pgfscope}%
\pgfpathrectangle{\pgfqpoint{2.125000in}{2.500000in}}{\pgfqpoint{5.489583in}{0.877907in}}%
\pgfusepath{clip}%
\pgfsetbuttcap%
\pgfsetroundjoin%
\pgfsetlinewidth{1.505625pt}%
\definecolor{currentstroke}{rgb}{0.000000,0.000000,0.000000}%
\pgfsetstrokecolor{currentstroke}%
\pgfsetdash{}{0pt}%
\pgfpathmoveto{\pgfqpoint{5.578324in}{2.778370in}}%
\pgfpathlineto{\pgfqpoint{5.578324in}{3.300837in}}%
\pgfusepath{stroke}%
\end{pgfscope}%
\begin{pgfscope}%
\pgfpathrectangle{\pgfqpoint{2.125000in}{2.500000in}}{\pgfqpoint{5.489583in}{0.877907in}}%
\pgfusepath{clip}%
\pgfsetbuttcap%
\pgfsetroundjoin%
\pgfsetlinewidth{1.505625pt}%
\definecolor{currentstroke}{rgb}{0.000000,0.000000,0.000000}%
\pgfsetstrokecolor{currentstroke}%
\pgfsetdash{}{0pt}%
\pgfpathmoveto{\pgfqpoint{5.701547in}{2.778370in}}%
\pgfpathlineto{\pgfqpoint{5.701547in}{3.299513in}}%
\pgfusepath{stroke}%
\end{pgfscope}%
\begin{pgfscope}%
\pgfpathrectangle{\pgfqpoint{2.125000in}{2.500000in}}{\pgfqpoint{5.489583in}{0.877907in}}%
\pgfusepath{clip}%
\pgfsetbuttcap%
\pgfsetroundjoin%
\pgfsetlinewidth{1.505625pt}%
\definecolor{currentstroke}{rgb}{0.000000,0.000000,0.000000}%
\pgfsetstrokecolor{currentstroke}%
\pgfsetdash{}{0pt}%
\pgfpathmoveto{\pgfqpoint{5.824770in}{2.778370in}}%
\pgfpathlineto{\pgfqpoint{5.824770in}{3.298173in}}%
\pgfusepath{stroke}%
\end{pgfscope}%
\begin{pgfscope}%
\pgfpathrectangle{\pgfqpoint{2.125000in}{2.500000in}}{\pgfqpoint{5.489583in}{0.877907in}}%
\pgfusepath{clip}%
\pgfsetbuttcap%
\pgfsetroundjoin%
\pgfsetlinewidth{1.505625pt}%
\definecolor{currentstroke}{rgb}{0.000000,0.000000,0.000000}%
\pgfsetstrokecolor{currentstroke}%
\pgfsetdash{}{0pt}%
\pgfpathmoveto{\pgfqpoint{5.947993in}{2.778370in}}%
\pgfpathlineto{\pgfqpoint{5.947993in}{3.296870in}}%
\pgfusepath{stroke}%
\end{pgfscope}%
\begin{pgfscope}%
\pgfpathrectangle{\pgfqpoint{2.125000in}{2.500000in}}{\pgfqpoint{5.489583in}{0.877907in}}%
\pgfusepath{clip}%
\pgfsetbuttcap%
\pgfsetroundjoin%
\pgfsetlinewidth{1.505625pt}%
\definecolor{currentstroke}{rgb}{0.000000,0.000000,0.000000}%
\pgfsetstrokecolor{currentstroke}%
\pgfsetdash{}{0pt}%
\pgfpathmoveto{\pgfqpoint{6.071216in}{2.778370in}}%
\pgfpathlineto{\pgfqpoint{6.071216in}{3.295553in}}%
\pgfusepath{stroke}%
\end{pgfscope}%
\begin{pgfscope}%
\pgfpathrectangle{\pgfqpoint{2.125000in}{2.500000in}}{\pgfqpoint{5.489583in}{0.877907in}}%
\pgfusepath{clip}%
\pgfsetbuttcap%
\pgfsetroundjoin%
\pgfsetlinewidth{1.505625pt}%
\definecolor{currentstroke}{rgb}{0.000000,0.000000,0.000000}%
\pgfsetstrokecolor{currentstroke}%
\pgfsetdash{}{0pt}%
\pgfpathmoveto{\pgfqpoint{6.194439in}{2.778370in}}%
\pgfpathlineto{\pgfqpoint{6.194439in}{3.294204in}}%
\pgfusepath{stroke}%
\end{pgfscope}%
\begin{pgfscope}%
\pgfpathrectangle{\pgfqpoint{2.125000in}{2.500000in}}{\pgfqpoint{5.489583in}{0.877907in}}%
\pgfusepath{clip}%
\pgfsetbuttcap%
\pgfsetroundjoin%
\pgfsetlinewidth{1.505625pt}%
\definecolor{currentstroke}{rgb}{0.000000,0.000000,0.000000}%
\pgfsetstrokecolor{currentstroke}%
\pgfsetdash{}{0pt}%
\pgfpathmoveto{\pgfqpoint{6.317662in}{2.778370in}}%
\pgfpathlineto{\pgfqpoint{6.317662in}{3.292871in}}%
\pgfusepath{stroke}%
\end{pgfscope}%
\begin{pgfscope}%
\pgfpathrectangle{\pgfqpoint{2.125000in}{2.500000in}}{\pgfqpoint{5.489583in}{0.877907in}}%
\pgfusepath{clip}%
\pgfsetbuttcap%
\pgfsetroundjoin%
\pgfsetlinewidth{1.505625pt}%
\definecolor{currentstroke}{rgb}{0.000000,0.000000,0.000000}%
\pgfsetstrokecolor{currentstroke}%
\pgfsetdash{}{0pt}%
\pgfpathmoveto{\pgfqpoint{6.440885in}{2.778370in}}%
\pgfpathlineto{\pgfqpoint{6.440885in}{3.291550in}}%
\pgfusepath{stroke}%
\end{pgfscope}%
\begin{pgfscope}%
\pgfpathrectangle{\pgfqpoint{2.125000in}{2.500000in}}{\pgfqpoint{5.489583in}{0.877907in}}%
\pgfusepath{clip}%
\pgfsetbuttcap%
\pgfsetroundjoin%
\pgfsetlinewidth{1.505625pt}%
\definecolor{currentstroke}{rgb}{0.000000,0.000000,0.000000}%
\pgfsetstrokecolor{currentstroke}%
\pgfsetdash{}{0pt}%
\pgfpathmoveto{\pgfqpoint{6.564108in}{2.778370in}}%
\pgfpathlineto{\pgfqpoint{6.564108in}{3.290213in}}%
\pgfusepath{stroke}%
\end{pgfscope}%
\begin{pgfscope}%
\pgfpathrectangle{\pgfqpoint{2.125000in}{2.500000in}}{\pgfqpoint{5.489583in}{0.877907in}}%
\pgfusepath{clip}%
\pgfsetbuttcap%
\pgfsetroundjoin%
\pgfsetlinewidth{1.505625pt}%
\definecolor{currentstroke}{rgb}{0.000000,0.000000,0.000000}%
\pgfsetstrokecolor{currentstroke}%
\pgfsetdash{}{0pt}%
\pgfpathmoveto{\pgfqpoint{6.687330in}{2.778370in}}%
\pgfpathlineto{\pgfqpoint{6.687330in}{3.288874in}}%
\pgfusepath{stroke}%
\end{pgfscope}%
\begin{pgfscope}%
\pgfpathrectangle{\pgfqpoint{2.125000in}{2.500000in}}{\pgfqpoint{5.489583in}{0.877907in}}%
\pgfusepath{clip}%
\pgfsetbuttcap%
\pgfsetroundjoin%
\pgfsetlinewidth{1.505625pt}%
\definecolor{currentstroke}{rgb}{0.000000,0.000000,0.000000}%
\pgfsetstrokecolor{currentstroke}%
\pgfsetdash{}{0pt}%
\pgfpathmoveto{\pgfqpoint{6.810553in}{2.778370in}}%
\pgfpathlineto{\pgfqpoint{6.810553in}{3.287570in}}%
\pgfusepath{stroke}%
\end{pgfscope}%
\begin{pgfscope}%
\pgfpathrectangle{\pgfqpoint{2.125000in}{2.500000in}}{\pgfqpoint{5.489583in}{0.877907in}}%
\pgfusepath{clip}%
\pgfsetbuttcap%
\pgfsetroundjoin%
\pgfsetlinewidth{1.505625pt}%
\definecolor{currentstroke}{rgb}{0.000000,0.000000,0.000000}%
\pgfsetstrokecolor{currentstroke}%
\pgfsetdash{}{0pt}%
\pgfpathmoveto{\pgfqpoint{6.933776in}{2.778370in}}%
\pgfpathlineto{\pgfqpoint{6.933776in}{3.286257in}}%
\pgfusepath{stroke}%
\end{pgfscope}%
\begin{pgfscope}%
\pgfpathrectangle{\pgfqpoint{2.125000in}{2.500000in}}{\pgfqpoint{5.489583in}{0.877907in}}%
\pgfusepath{clip}%
\pgfsetbuttcap%
\pgfsetroundjoin%
\pgfsetlinewidth{1.505625pt}%
\definecolor{currentstroke}{rgb}{0.000000,0.000000,0.000000}%
\pgfsetstrokecolor{currentstroke}%
\pgfsetdash{}{0pt}%
\pgfpathmoveto{\pgfqpoint{7.056999in}{2.778370in}}%
\pgfpathlineto{\pgfqpoint{7.056999in}{3.284967in}}%
\pgfusepath{stroke}%
\end{pgfscope}%
\begin{pgfscope}%
\pgfpathrectangle{\pgfqpoint{2.125000in}{2.500000in}}{\pgfqpoint{5.489583in}{0.877907in}}%
\pgfusepath{clip}%
\pgfsetbuttcap%
\pgfsetroundjoin%
\pgfsetlinewidth{1.505625pt}%
\definecolor{currentstroke}{rgb}{0.000000,0.000000,0.000000}%
\pgfsetstrokecolor{currentstroke}%
\pgfsetdash{}{0pt}%
\pgfpathmoveto{\pgfqpoint{7.180222in}{2.778370in}}%
\pgfpathlineto{\pgfqpoint{7.180222in}{3.283749in}}%
\pgfusepath{stroke}%
\end{pgfscope}%
\begin{pgfscope}%
\pgfpathrectangle{\pgfqpoint{2.125000in}{2.500000in}}{\pgfqpoint{5.489583in}{0.877907in}}%
\pgfusepath{clip}%
\pgfsetbuttcap%
\pgfsetroundjoin%
\pgfsetlinewidth{1.505625pt}%
\definecolor{currentstroke}{rgb}{0.000000,0.000000,0.000000}%
\pgfsetstrokecolor{currentstroke}%
\pgfsetdash{}{0pt}%
\pgfpathmoveto{\pgfqpoint{7.303445in}{2.778370in}}%
\pgfpathlineto{\pgfqpoint{7.303445in}{3.282568in}}%
\pgfusepath{stroke}%
\end{pgfscope}%
\begin{pgfscope}%
\pgfpathrectangle{\pgfqpoint{2.125000in}{2.500000in}}{\pgfqpoint{5.489583in}{0.877907in}}%
\pgfusepath{clip}%
\pgfsetroundcap%
\pgfsetroundjoin%
\pgfsetlinewidth{1.505625pt}%
\definecolor{currentstroke}{rgb}{0.121569,0.466667,0.705882}%
\pgfsetstrokecolor{currentstroke}%
\pgfsetdash{}{0pt}%
\pgfpathmoveto{\pgfqpoint{2.125000in}{2.778370in}}%
\pgfpathlineto{\pgfqpoint{7.614583in}{2.778370in}}%
\pgfusepath{stroke}%
\end{pgfscope}%
\begin{pgfscope}%
\pgfpathrectangle{\pgfqpoint{2.125000in}{2.500000in}}{\pgfqpoint{5.489583in}{0.877907in}}%
\pgfusepath{clip}%
\pgfsetbuttcap%
\pgfsetroundjoin%
\definecolor{currentfill}{rgb}{0.121569,0.466667,0.705882}%
\pgfsetfillcolor{currentfill}%
\pgfsetlinewidth{1.003750pt}%
\definecolor{currentstroke}{rgb}{0.121569,0.466667,0.705882}%
\pgfsetstrokecolor{currentstroke}%
\pgfsetdash{}{0pt}%
\pgfsys@defobject{currentmarker}{\pgfqpoint{-0.034722in}{-0.034722in}}{\pgfqpoint{0.034722in}{0.034722in}}{%
\pgfpathmoveto{\pgfqpoint{0.000000in}{-0.034722in}}%
\pgfpathcurveto{\pgfqpoint{0.009208in}{-0.034722in}}{\pgfqpoint{0.018041in}{-0.031064in}}{\pgfqpoint{0.024552in}{-0.024552in}}%
\pgfpathcurveto{\pgfqpoint{0.031064in}{-0.018041in}}{\pgfqpoint{0.034722in}{-0.009208in}}{\pgfqpoint{0.034722in}{0.000000in}}%
\pgfpathcurveto{\pgfqpoint{0.034722in}{0.009208in}}{\pgfqpoint{0.031064in}{0.018041in}}{\pgfqpoint{0.024552in}{0.024552in}}%
\pgfpathcurveto{\pgfqpoint{0.018041in}{0.031064in}}{\pgfqpoint{0.009208in}{0.034722in}}{\pgfqpoint{0.000000in}{0.034722in}}%
\pgfpathcurveto{\pgfqpoint{-0.009208in}{0.034722in}}{\pgfqpoint{-0.018041in}{0.031064in}}{\pgfqpoint{-0.024552in}{0.024552in}}%
\pgfpathcurveto{\pgfqpoint{-0.031064in}{0.018041in}}{\pgfqpoint{-0.034722in}{0.009208in}}{\pgfqpoint{-0.034722in}{0.000000in}}%
\pgfpathcurveto{\pgfqpoint{-0.034722in}{-0.009208in}}{\pgfqpoint{-0.031064in}{-0.018041in}}{\pgfqpoint{-0.024552in}{-0.024552in}}%
\pgfpathcurveto{\pgfqpoint{-0.018041in}{-0.031064in}}{\pgfqpoint{-0.009208in}{-0.034722in}}{\pgfqpoint{0.000000in}{-0.034722in}}%
\pgfpathclose%
\pgfusepath{stroke,fill}%
}%
\begin{pgfscope}%
\pgfsys@transformshift{2.374527in}{3.338002in}%
\pgfsys@useobject{currentmarker}{}%
\end{pgfscope}%
\begin{pgfscope}%
\pgfsys@transformshift{2.497749in}{3.336501in}%
\pgfsys@useobject{currentmarker}{}%
\end{pgfscope}%
\begin{pgfscope}%
\pgfsys@transformshift{2.620972in}{3.335030in}%
\pgfsys@useobject{currentmarker}{}%
\end{pgfscope}%
\begin{pgfscope}%
\pgfsys@transformshift{2.744195in}{3.333602in}%
\pgfsys@useobject{currentmarker}{}%
\end{pgfscope}%
\begin{pgfscope}%
\pgfsys@transformshift{2.867418in}{3.332197in}%
\pgfsys@useobject{currentmarker}{}%
\end{pgfscope}%
\begin{pgfscope}%
\pgfsys@transformshift{2.990641in}{3.330797in}%
\pgfsys@useobject{currentmarker}{}%
\end{pgfscope}%
\begin{pgfscope}%
\pgfsys@transformshift{3.113864in}{3.329429in}%
\pgfsys@useobject{currentmarker}{}%
\end{pgfscope}%
\begin{pgfscope}%
\pgfsys@transformshift{3.237087in}{3.328006in}%
\pgfsys@useobject{currentmarker}{}%
\end{pgfscope}%
\begin{pgfscope}%
\pgfsys@transformshift{3.360310in}{3.326544in}%
\pgfsys@useobject{currentmarker}{}%
\end{pgfscope}%
\begin{pgfscope}%
\pgfsys@transformshift{3.483533in}{3.325041in}%
\pgfsys@useobject{currentmarker}{}%
\end{pgfscope}%
\begin{pgfscope}%
\pgfsys@transformshift{3.606756in}{3.323537in}%
\pgfsys@useobject{currentmarker}{}%
\end{pgfscope}%
\begin{pgfscope}%
\pgfsys@transformshift{3.729979in}{3.322044in}%
\pgfsys@useobject{currentmarker}{}%
\end{pgfscope}%
\begin{pgfscope}%
\pgfsys@transformshift{3.853202in}{3.320573in}%
\pgfsys@useobject{currentmarker}{}%
\end{pgfscope}%
\begin{pgfscope}%
\pgfsys@transformshift{3.976425in}{3.319111in}%
\pgfsys@useobject{currentmarker}{}%
\end{pgfscope}%
\begin{pgfscope}%
\pgfsys@transformshift{4.099648in}{3.317665in}%
\pgfsys@useobject{currentmarker}{}%
\end{pgfscope}%
\begin{pgfscope}%
\pgfsys@transformshift{4.222871in}{3.316243in}%
\pgfsys@useobject{currentmarker}{}%
\end{pgfscope}%
\begin{pgfscope}%
\pgfsys@transformshift{4.346094in}{3.314848in}%
\pgfsys@useobject{currentmarker}{}%
\end{pgfscope}%
\begin{pgfscope}%
\pgfsys@transformshift{4.469317in}{3.313441in}%
\pgfsys@useobject{currentmarker}{}%
\end{pgfscope}%
\begin{pgfscope}%
\pgfsys@transformshift{4.592540in}{3.311986in}%
\pgfsys@useobject{currentmarker}{}%
\end{pgfscope}%
\begin{pgfscope}%
\pgfsys@transformshift{4.715763in}{3.310478in}%
\pgfsys@useobject{currentmarker}{}%
\end{pgfscope}%
\begin{pgfscope}%
\pgfsys@transformshift{4.838986in}{3.308997in}%
\pgfsys@useobject{currentmarker}{}%
\end{pgfscope}%
\begin{pgfscope}%
\pgfsys@transformshift{4.962209in}{3.307567in}%
\pgfsys@useobject{currentmarker}{}%
\end{pgfscope}%
\begin{pgfscope}%
\pgfsys@transformshift{5.085432in}{3.306120in}%
\pgfsys@useobject{currentmarker}{}%
\end{pgfscope}%
\begin{pgfscope}%
\pgfsys@transformshift{5.208655in}{3.304781in}%
\pgfsys@useobject{currentmarker}{}%
\end{pgfscope}%
\begin{pgfscope}%
\pgfsys@transformshift{5.331878in}{3.303469in}%
\pgfsys@useobject{currentmarker}{}%
\end{pgfscope}%
\begin{pgfscope}%
\pgfsys@transformshift{5.455101in}{3.302157in}%
\pgfsys@useobject{currentmarker}{}%
\end{pgfscope}%
\begin{pgfscope}%
\pgfsys@transformshift{5.578324in}{3.300837in}%
\pgfsys@useobject{currentmarker}{}%
\end{pgfscope}%
\begin{pgfscope}%
\pgfsys@transformshift{5.701547in}{3.299513in}%
\pgfsys@useobject{currentmarker}{}%
\end{pgfscope}%
\begin{pgfscope}%
\pgfsys@transformshift{5.824770in}{3.298173in}%
\pgfsys@useobject{currentmarker}{}%
\end{pgfscope}%
\begin{pgfscope}%
\pgfsys@transformshift{5.947993in}{3.296870in}%
\pgfsys@useobject{currentmarker}{}%
\end{pgfscope}%
\begin{pgfscope}%
\pgfsys@transformshift{6.071216in}{3.295553in}%
\pgfsys@useobject{currentmarker}{}%
\end{pgfscope}%
\begin{pgfscope}%
\pgfsys@transformshift{6.194439in}{3.294204in}%
\pgfsys@useobject{currentmarker}{}%
\end{pgfscope}%
\begin{pgfscope}%
\pgfsys@transformshift{6.317662in}{3.292871in}%
\pgfsys@useobject{currentmarker}{}%
\end{pgfscope}%
\begin{pgfscope}%
\pgfsys@transformshift{6.440885in}{3.291550in}%
\pgfsys@useobject{currentmarker}{}%
\end{pgfscope}%
\begin{pgfscope}%
\pgfsys@transformshift{6.564108in}{3.290213in}%
\pgfsys@useobject{currentmarker}{}%
\end{pgfscope}%
\begin{pgfscope}%
\pgfsys@transformshift{6.687330in}{3.288874in}%
\pgfsys@useobject{currentmarker}{}%
\end{pgfscope}%
\begin{pgfscope}%
\pgfsys@transformshift{6.810553in}{3.287570in}%
\pgfsys@useobject{currentmarker}{}%
\end{pgfscope}%
\begin{pgfscope}%
\pgfsys@transformshift{6.933776in}{3.286257in}%
\pgfsys@useobject{currentmarker}{}%
\end{pgfscope}%
\begin{pgfscope}%
\pgfsys@transformshift{7.056999in}{3.284967in}%
\pgfsys@useobject{currentmarker}{}%
\end{pgfscope}%
\begin{pgfscope}%
\pgfsys@transformshift{7.180222in}{3.283749in}%
\pgfsys@useobject{currentmarker}{}%
\end{pgfscope}%
\begin{pgfscope}%
\pgfsys@transformshift{7.303445in}{3.282568in}%
\pgfsys@useobject{currentmarker}{}%
\end{pgfscope}%
\end{pgfscope}%
\begin{pgfscope}%
\pgfsetrectcap%
\pgfsetmiterjoin%
\pgfsetlinewidth{0.803000pt}%
\definecolor{currentstroke}{rgb}{1.000000,1.000000,1.000000}%
\pgfsetstrokecolor{currentstroke}%
\pgfsetdash{}{0pt}%
\pgfpathmoveto{\pgfqpoint{2.125000in}{2.500000in}}%
\pgfpathlineto{\pgfqpoint{2.125000in}{3.377907in}}%
\pgfusepath{stroke}%
\end{pgfscope}%
\begin{pgfscope}%
\pgfsetrectcap%
\pgfsetmiterjoin%
\pgfsetlinewidth{0.803000pt}%
\definecolor{currentstroke}{rgb}{1.000000,1.000000,1.000000}%
\pgfsetstrokecolor{currentstroke}%
\pgfsetdash{}{0pt}%
\pgfpathmoveto{\pgfqpoint{7.614583in}{2.500000in}}%
\pgfpathlineto{\pgfqpoint{7.614583in}{3.377907in}}%
\pgfusepath{stroke}%
\end{pgfscope}%
\begin{pgfscope}%
\pgfsetrectcap%
\pgfsetmiterjoin%
\pgfsetlinewidth{0.803000pt}%
\definecolor{currentstroke}{rgb}{1.000000,1.000000,1.000000}%
\pgfsetstrokecolor{currentstroke}%
\pgfsetdash{}{0pt}%
\pgfpathmoveto{\pgfqpoint{2.125000in}{2.500000in}}%
\pgfpathlineto{\pgfqpoint{7.614583in}{2.500000in}}%
\pgfusepath{stroke}%
\end{pgfscope}%
\begin{pgfscope}%
\pgfsetrectcap%
\pgfsetmiterjoin%
\pgfsetlinewidth{0.803000pt}%
\definecolor{currentstroke}{rgb}{1.000000,1.000000,1.000000}%
\pgfsetstrokecolor{currentstroke}%
\pgfsetdash{}{0pt}%
\pgfpathmoveto{\pgfqpoint{2.125000in}{3.377907in}}%
\pgfpathlineto{\pgfqpoint{7.614583in}{3.377907in}}%
\pgfusepath{stroke}%
\end{pgfscope}%
\begin{pgfscope}%
\definecolor{textcolor}{rgb}{0.150000,0.150000,0.150000}%
\pgfsetstrokecolor{textcolor}%
\pgfsetfillcolor{textcolor}%
\pgftext[x=4.869792in,y=3.461240in,,base]{\color{textcolor}\rmfamily\fontsize{16.800000}{20.160000}\selectfont Autocorrelation}%
\end{pgfscope}%
\begin{pgfscope}%
\pgfsetbuttcap%
\pgfsetmiterjoin%
\definecolor{currentfill}{rgb}{0.917647,0.917647,0.949020}%
\pgfsetfillcolor{currentfill}%
\pgfsetlinewidth{0.000000pt}%
\definecolor{currentstroke}{rgb}{0.000000,0.000000,0.000000}%
\pgfsetstrokecolor{currentstroke}%
\pgfsetstrokeopacity{0.000000}%
\pgfsetdash{}{0pt}%
\pgfpathmoveto{\pgfqpoint{9.810417in}{2.500000in}}%
\pgfpathlineto{\pgfqpoint{15.300000in}{2.500000in}}%
\pgfpathlineto{\pgfqpoint{15.300000in}{3.377907in}}%
\pgfpathlineto{\pgfqpoint{9.810417in}{3.377907in}}%
\pgfpathclose%
\pgfusepath{fill}%
\end{pgfscope}%
\begin{pgfscope}%
\pgfpathrectangle{\pgfqpoint{9.810417in}{2.500000in}}{\pgfqpoint{5.489583in}{0.877907in}}%
\pgfusepath{clip}%
\pgfsetroundcap%
\pgfsetroundjoin%
\pgfsetlinewidth{0.803000pt}%
\definecolor{currentstroke}{rgb}{1.000000,1.000000,1.000000}%
\pgfsetstrokecolor{currentstroke}%
\pgfsetdash{}{0pt}%
\pgfpathmoveto{\pgfqpoint{10.059943in}{2.500000in}}%
\pgfpathlineto{\pgfqpoint{10.059943in}{3.377907in}}%
\pgfusepath{stroke}%
\end{pgfscope}%
\begin{pgfscope}%
\definecolor{textcolor}{rgb}{0.150000,0.150000,0.150000}%
\pgfsetstrokecolor{textcolor}%
\pgfsetfillcolor{textcolor}%
\pgftext[x=10.059943in,y=2.402778in,,top]{\color{textcolor}\rmfamily\fontsize{14.000000}{16.800000}\selectfont 0}%
\end{pgfscope}%
\begin{pgfscope}%
\pgfpathrectangle{\pgfqpoint{9.810417in}{2.500000in}}{\pgfqpoint{5.489583in}{0.877907in}}%
\pgfusepath{clip}%
\pgfsetroundcap%
\pgfsetroundjoin%
\pgfsetlinewidth{0.803000pt}%
\definecolor{currentstroke}{rgb}{1.000000,1.000000,1.000000}%
\pgfsetstrokecolor{currentstroke}%
\pgfsetdash{}{0pt}%
\pgfpathmoveto{\pgfqpoint{10.676058in}{2.500000in}}%
\pgfpathlineto{\pgfqpoint{10.676058in}{3.377907in}}%
\pgfusepath{stroke}%
\end{pgfscope}%
\begin{pgfscope}%
\definecolor{textcolor}{rgb}{0.150000,0.150000,0.150000}%
\pgfsetstrokecolor{textcolor}%
\pgfsetfillcolor{textcolor}%
\pgftext[x=10.676058in,y=2.402778in,,top]{\color{textcolor}\rmfamily\fontsize{14.000000}{16.800000}\selectfont 5}%
\end{pgfscope}%
\begin{pgfscope}%
\pgfpathrectangle{\pgfqpoint{9.810417in}{2.500000in}}{\pgfqpoint{5.489583in}{0.877907in}}%
\pgfusepath{clip}%
\pgfsetroundcap%
\pgfsetroundjoin%
\pgfsetlinewidth{0.803000pt}%
\definecolor{currentstroke}{rgb}{1.000000,1.000000,1.000000}%
\pgfsetstrokecolor{currentstroke}%
\pgfsetdash{}{0pt}%
\pgfpathmoveto{\pgfqpoint{11.292173in}{2.500000in}}%
\pgfpathlineto{\pgfqpoint{11.292173in}{3.377907in}}%
\pgfusepath{stroke}%
\end{pgfscope}%
\begin{pgfscope}%
\definecolor{textcolor}{rgb}{0.150000,0.150000,0.150000}%
\pgfsetstrokecolor{textcolor}%
\pgfsetfillcolor{textcolor}%
\pgftext[x=11.292173in,y=2.402778in,,top]{\color{textcolor}\rmfamily\fontsize{14.000000}{16.800000}\selectfont 10}%
\end{pgfscope}%
\begin{pgfscope}%
\pgfpathrectangle{\pgfqpoint{9.810417in}{2.500000in}}{\pgfqpoint{5.489583in}{0.877907in}}%
\pgfusepath{clip}%
\pgfsetroundcap%
\pgfsetroundjoin%
\pgfsetlinewidth{0.803000pt}%
\definecolor{currentstroke}{rgb}{1.000000,1.000000,1.000000}%
\pgfsetstrokecolor{currentstroke}%
\pgfsetdash{}{0pt}%
\pgfpathmoveto{\pgfqpoint{11.908288in}{2.500000in}}%
\pgfpathlineto{\pgfqpoint{11.908288in}{3.377907in}}%
\pgfusepath{stroke}%
\end{pgfscope}%
\begin{pgfscope}%
\definecolor{textcolor}{rgb}{0.150000,0.150000,0.150000}%
\pgfsetstrokecolor{textcolor}%
\pgfsetfillcolor{textcolor}%
\pgftext[x=11.908288in,y=2.402778in,,top]{\color{textcolor}\rmfamily\fontsize{14.000000}{16.800000}\selectfont 15}%
\end{pgfscope}%
\begin{pgfscope}%
\pgfpathrectangle{\pgfqpoint{9.810417in}{2.500000in}}{\pgfqpoint{5.489583in}{0.877907in}}%
\pgfusepath{clip}%
\pgfsetroundcap%
\pgfsetroundjoin%
\pgfsetlinewidth{0.803000pt}%
\definecolor{currentstroke}{rgb}{1.000000,1.000000,1.000000}%
\pgfsetstrokecolor{currentstroke}%
\pgfsetdash{}{0pt}%
\pgfpathmoveto{\pgfqpoint{12.524403in}{2.500000in}}%
\pgfpathlineto{\pgfqpoint{12.524403in}{3.377907in}}%
\pgfusepath{stroke}%
\end{pgfscope}%
\begin{pgfscope}%
\definecolor{textcolor}{rgb}{0.150000,0.150000,0.150000}%
\pgfsetstrokecolor{textcolor}%
\pgfsetfillcolor{textcolor}%
\pgftext[x=12.524403in,y=2.402778in,,top]{\color{textcolor}\rmfamily\fontsize{14.000000}{16.800000}\selectfont 20}%
\end{pgfscope}%
\begin{pgfscope}%
\pgfpathrectangle{\pgfqpoint{9.810417in}{2.500000in}}{\pgfqpoint{5.489583in}{0.877907in}}%
\pgfusepath{clip}%
\pgfsetroundcap%
\pgfsetroundjoin%
\pgfsetlinewidth{0.803000pt}%
\definecolor{currentstroke}{rgb}{1.000000,1.000000,1.000000}%
\pgfsetstrokecolor{currentstroke}%
\pgfsetdash{}{0pt}%
\pgfpathmoveto{\pgfqpoint{13.140517in}{2.500000in}}%
\pgfpathlineto{\pgfqpoint{13.140517in}{3.377907in}}%
\pgfusepath{stroke}%
\end{pgfscope}%
\begin{pgfscope}%
\definecolor{textcolor}{rgb}{0.150000,0.150000,0.150000}%
\pgfsetstrokecolor{textcolor}%
\pgfsetfillcolor{textcolor}%
\pgftext[x=13.140517in,y=2.402778in,,top]{\color{textcolor}\rmfamily\fontsize{14.000000}{16.800000}\selectfont 25}%
\end{pgfscope}%
\begin{pgfscope}%
\pgfpathrectangle{\pgfqpoint{9.810417in}{2.500000in}}{\pgfqpoint{5.489583in}{0.877907in}}%
\pgfusepath{clip}%
\pgfsetroundcap%
\pgfsetroundjoin%
\pgfsetlinewidth{0.803000pt}%
\definecolor{currentstroke}{rgb}{1.000000,1.000000,1.000000}%
\pgfsetstrokecolor{currentstroke}%
\pgfsetdash{}{0pt}%
\pgfpathmoveto{\pgfqpoint{13.756632in}{2.500000in}}%
\pgfpathlineto{\pgfqpoint{13.756632in}{3.377907in}}%
\pgfusepath{stroke}%
\end{pgfscope}%
\begin{pgfscope}%
\definecolor{textcolor}{rgb}{0.150000,0.150000,0.150000}%
\pgfsetstrokecolor{textcolor}%
\pgfsetfillcolor{textcolor}%
\pgftext[x=13.756632in,y=2.402778in,,top]{\color{textcolor}\rmfamily\fontsize{14.000000}{16.800000}\selectfont 30}%
\end{pgfscope}%
\begin{pgfscope}%
\pgfpathrectangle{\pgfqpoint{9.810417in}{2.500000in}}{\pgfqpoint{5.489583in}{0.877907in}}%
\pgfusepath{clip}%
\pgfsetroundcap%
\pgfsetroundjoin%
\pgfsetlinewidth{0.803000pt}%
\definecolor{currentstroke}{rgb}{1.000000,1.000000,1.000000}%
\pgfsetstrokecolor{currentstroke}%
\pgfsetdash{}{0pt}%
\pgfpathmoveto{\pgfqpoint{14.372747in}{2.500000in}}%
\pgfpathlineto{\pgfqpoint{14.372747in}{3.377907in}}%
\pgfusepath{stroke}%
\end{pgfscope}%
\begin{pgfscope}%
\definecolor{textcolor}{rgb}{0.150000,0.150000,0.150000}%
\pgfsetstrokecolor{textcolor}%
\pgfsetfillcolor{textcolor}%
\pgftext[x=14.372747in,y=2.402778in,,top]{\color{textcolor}\rmfamily\fontsize{14.000000}{16.800000}\selectfont 35}%
\end{pgfscope}%
\begin{pgfscope}%
\pgfpathrectangle{\pgfqpoint{9.810417in}{2.500000in}}{\pgfqpoint{5.489583in}{0.877907in}}%
\pgfusepath{clip}%
\pgfsetroundcap%
\pgfsetroundjoin%
\pgfsetlinewidth{0.803000pt}%
\definecolor{currentstroke}{rgb}{1.000000,1.000000,1.000000}%
\pgfsetstrokecolor{currentstroke}%
\pgfsetdash{}{0pt}%
\pgfpathmoveto{\pgfqpoint{14.988862in}{2.500000in}}%
\pgfpathlineto{\pgfqpoint{14.988862in}{3.377907in}}%
\pgfusepath{stroke}%
\end{pgfscope}%
\begin{pgfscope}%
\definecolor{textcolor}{rgb}{0.150000,0.150000,0.150000}%
\pgfsetstrokecolor{textcolor}%
\pgfsetfillcolor{textcolor}%
\pgftext[x=14.988862in,y=2.402778in,,top]{\color{textcolor}\rmfamily\fontsize{14.000000}{16.800000}\selectfont 40}%
\end{pgfscope}%
\begin{pgfscope}%
\pgfpathrectangle{\pgfqpoint{9.810417in}{2.500000in}}{\pgfqpoint{5.489583in}{0.877907in}}%
\pgfusepath{clip}%
\pgfsetroundcap%
\pgfsetroundjoin%
\pgfsetlinewidth{0.803000pt}%
\definecolor{currentstroke}{rgb}{1.000000,1.000000,1.000000}%
\pgfsetstrokecolor{currentstroke}%
\pgfsetdash{}{0pt}%
\pgfpathmoveto{\pgfqpoint{9.810417in}{2.578239in}}%
\pgfpathlineto{\pgfqpoint{15.300000in}{2.578239in}}%
\pgfusepath{stroke}%
\end{pgfscope}%
\begin{pgfscope}%
\definecolor{textcolor}{rgb}{0.150000,0.150000,0.150000}%
\pgfsetstrokecolor{textcolor}%
\pgfsetfillcolor{textcolor}%
\pgftext[x=9.589483in,y=2.504373in,left,base]{\color{textcolor}\rmfamily\fontsize{14.000000}{16.800000}\selectfont 0}%
\end{pgfscope}%
\begin{pgfscope}%
\pgfpathrectangle{\pgfqpoint{9.810417in}{2.500000in}}{\pgfqpoint{5.489583in}{0.877907in}}%
\pgfusepath{clip}%
\pgfsetroundcap%
\pgfsetroundjoin%
\pgfsetlinewidth{0.803000pt}%
\definecolor{currentstroke}{rgb}{1.000000,1.000000,1.000000}%
\pgfsetstrokecolor{currentstroke}%
\pgfsetdash{}{0pt}%
\pgfpathmoveto{\pgfqpoint{9.810417in}{3.338002in}}%
\pgfpathlineto{\pgfqpoint{15.300000in}{3.338002in}}%
\pgfusepath{stroke}%
\end{pgfscope}%
\begin{pgfscope}%
\definecolor{textcolor}{rgb}{0.150000,0.150000,0.150000}%
\pgfsetstrokecolor{textcolor}%
\pgfsetfillcolor{textcolor}%
\pgftext[x=9.589483in,y=3.264136in,left,base]{\color{textcolor}\rmfamily\fontsize{14.000000}{16.800000}\selectfont 1}%
\end{pgfscope}%
\begin{pgfscope}%
\pgfpathrectangle{\pgfqpoint{9.810417in}{2.500000in}}{\pgfqpoint{5.489583in}{0.877907in}}%
\pgfusepath{clip}%
\pgfsetbuttcap%
\pgfsetroundjoin%
\definecolor{currentfill}{rgb}{0.121569,0.466667,0.705882}%
\pgfsetfillcolor{currentfill}%
\pgfsetfillopacity{0.250000}%
\pgfsetlinewidth{1.003750pt}%
\definecolor{currentstroke}{rgb}{1.000000,1.000000,1.000000}%
\pgfsetstrokecolor{currentstroke}%
\pgfsetstrokeopacity{0.250000}%
\pgfsetdash{}{0pt}%
\pgfpathmoveto{\pgfqpoint{10.121555in}{2.616572in}}%
\pgfpathlineto{\pgfqpoint{10.121555in}{2.539905in}}%
\pgfpathlineto{\pgfqpoint{10.306389in}{2.539905in}}%
\pgfpathlineto{\pgfqpoint{10.429612in}{2.539905in}}%
\pgfpathlineto{\pgfqpoint{10.552835in}{2.539905in}}%
\pgfpathlineto{\pgfqpoint{10.676058in}{2.539905in}}%
\pgfpathlineto{\pgfqpoint{10.799281in}{2.539905in}}%
\pgfpathlineto{\pgfqpoint{10.922504in}{2.539905in}}%
\pgfpathlineto{\pgfqpoint{11.045727in}{2.539905in}}%
\pgfpathlineto{\pgfqpoint{11.168950in}{2.539905in}}%
\pgfpathlineto{\pgfqpoint{11.292173in}{2.539905in}}%
\pgfpathlineto{\pgfqpoint{11.415396in}{2.539905in}}%
\pgfpathlineto{\pgfqpoint{11.538619in}{2.539905in}}%
\pgfpathlineto{\pgfqpoint{11.661842in}{2.539905in}}%
\pgfpathlineto{\pgfqpoint{11.785065in}{2.539905in}}%
\pgfpathlineto{\pgfqpoint{11.908288in}{2.539905in}}%
\pgfpathlineto{\pgfqpoint{12.031511in}{2.539905in}}%
\pgfpathlineto{\pgfqpoint{12.154734in}{2.539905in}}%
\pgfpathlineto{\pgfqpoint{12.277957in}{2.539905in}}%
\pgfpathlineto{\pgfqpoint{12.401180in}{2.539905in}}%
\pgfpathlineto{\pgfqpoint{12.524403in}{2.539905in}}%
\pgfpathlineto{\pgfqpoint{12.647626in}{2.539905in}}%
\pgfpathlineto{\pgfqpoint{12.770849in}{2.539905in}}%
\pgfpathlineto{\pgfqpoint{12.894072in}{2.539905in}}%
\pgfpathlineto{\pgfqpoint{13.017294in}{2.539905in}}%
\pgfpathlineto{\pgfqpoint{13.140517in}{2.539905in}}%
\pgfpathlineto{\pgfqpoint{13.263740in}{2.539905in}}%
\pgfpathlineto{\pgfqpoint{13.386963in}{2.539905in}}%
\pgfpathlineto{\pgfqpoint{13.510186in}{2.539905in}}%
\pgfpathlineto{\pgfqpoint{13.633409in}{2.539905in}}%
\pgfpathlineto{\pgfqpoint{13.756632in}{2.539905in}}%
\pgfpathlineto{\pgfqpoint{13.879855in}{2.539905in}}%
\pgfpathlineto{\pgfqpoint{14.003078in}{2.539905in}}%
\pgfpathlineto{\pgfqpoint{14.126301in}{2.539905in}}%
\pgfpathlineto{\pgfqpoint{14.249524in}{2.539905in}}%
\pgfpathlineto{\pgfqpoint{14.372747in}{2.539905in}}%
\pgfpathlineto{\pgfqpoint{14.495970in}{2.539905in}}%
\pgfpathlineto{\pgfqpoint{14.619193in}{2.539905in}}%
\pgfpathlineto{\pgfqpoint{14.742416in}{2.539905in}}%
\pgfpathlineto{\pgfqpoint{14.865639in}{2.539905in}}%
\pgfpathlineto{\pgfqpoint{15.050473in}{2.539905in}}%
\pgfpathlineto{\pgfqpoint{15.050473in}{2.616572in}}%
\pgfpathlineto{\pgfqpoint{15.050473in}{2.616572in}}%
\pgfpathlineto{\pgfqpoint{14.865639in}{2.616572in}}%
\pgfpathlineto{\pgfqpoint{14.742416in}{2.616572in}}%
\pgfpathlineto{\pgfqpoint{14.619193in}{2.616572in}}%
\pgfpathlineto{\pgfqpoint{14.495970in}{2.616572in}}%
\pgfpathlineto{\pgfqpoint{14.372747in}{2.616572in}}%
\pgfpathlineto{\pgfqpoint{14.249524in}{2.616572in}}%
\pgfpathlineto{\pgfqpoint{14.126301in}{2.616572in}}%
\pgfpathlineto{\pgfqpoint{14.003078in}{2.616572in}}%
\pgfpathlineto{\pgfqpoint{13.879855in}{2.616572in}}%
\pgfpathlineto{\pgfqpoint{13.756632in}{2.616572in}}%
\pgfpathlineto{\pgfqpoint{13.633409in}{2.616572in}}%
\pgfpathlineto{\pgfqpoint{13.510186in}{2.616572in}}%
\pgfpathlineto{\pgfqpoint{13.386963in}{2.616572in}}%
\pgfpathlineto{\pgfqpoint{13.263740in}{2.616572in}}%
\pgfpathlineto{\pgfqpoint{13.140517in}{2.616572in}}%
\pgfpathlineto{\pgfqpoint{13.017294in}{2.616572in}}%
\pgfpathlineto{\pgfqpoint{12.894072in}{2.616572in}}%
\pgfpathlineto{\pgfqpoint{12.770849in}{2.616572in}}%
\pgfpathlineto{\pgfqpoint{12.647626in}{2.616572in}}%
\pgfpathlineto{\pgfqpoint{12.524403in}{2.616572in}}%
\pgfpathlineto{\pgfqpoint{12.401180in}{2.616572in}}%
\pgfpathlineto{\pgfqpoint{12.277957in}{2.616572in}}%
\pgfpathlineto{\pgfqpoint{12.154734in}{2.616572in}}%
\pgfpathlineto{\pgfqpoint{12.031511in}{2.616572in}}%
\pgfpathlineto{\pgfqpoint{11.908288in}{2.616572in}}%
\pgfpathlineto{\pgfqpoint{11.785065in}{2.616572in}}%
\pgfpathlineto{\pgfqpoint{11.661842in}{2.616572in}}%
\pgfpathlineto{\pgfqpoint{11.538619in}{2.616572in}}%
\pgfpathlineto{\pgfqpoint{11.415396in}{2.616572in}}%
\pgfpathlineto{\pgfqpoint{11.292173in}{2.616572in}}%
\pgfpathlineto{\pgfqpoint{11.168950in}{2.616572in}}%
\pgfpathlineto{\pgfqpoint{11.045727in}{2.616572in}}%
\pgfpathlineto{\pgfqpoint{10.922504in}{2.616572in}}%
\pgfpathlineto{\pgfqpoint{10.799281in}{2.616572in}}%
\pgfpathlineto{\pgfqpoint{10.676058in}{2.616572in}}%
\pgfpathlineto{\pgfqpoint{10.552835in}{2.616572in}}%
\pgfpathlineto{\pgfqpoint{10.429612in}{2.616572in}}%
\pgfpathlineto{\pgfqpoint{10.306389in}{2.616572in}}%
\pgfpathlineto{\pgfqpoint{10.121555in}{2.616572in}}%
\pgfpathclose%
\pgfusepath{stroke,fill}%
\end{pgfscope}%
\begin{pgfscope}%
\pgfpathrectangle{\pgfqpoint{9.810417in}{2.500000in}}{\pgfqpoint{5.489583in}{0.877907in}}%
\pgfusepath{clip}%
\pgfsetbuttcap%
\pgfsetroundjoin%
\pgfsetlinewidth{1.505625pt}%
\definecolor{currentstroke}{rgb}{0.000000,0.000000,0.000000}%
\pgfsetstrokecolor{currentstroke}%
\pgfsetdash{}{0pt}%
\pgfpathmoveto{\pgfqpoint{10.059943in}{2.578239in}}%
\pgfpathlineto{\pgfqpoint{10.059943in}{3.338002in}}%
\pgfusepath{stroke}%
\end{pgfscope}%
\begin{pgfscope}%
\pgfpathrectangle{\pgfqpoint{9.810417in}{2.500000in}}{\pgfqpoint{5.489583in}{0.877907in}}%
\pgfusepath{clip}%
\pgfsetbuttcap%
\pgfsetroundjoin%
\pgfsetlinewidth{1.505625pt}%
\definecolor{currentstroke}{rgb}{0.000000,0.000000,0.000000}%
\pgfsetstrokecolor{currentstroke}%
\pgfsetdash{}{0pt}%
\pgfpathmoveto{\pgfqpoint{10.183166in}{2.578239in}}%
\pgfpathlineto{\pgfqpoint{10.183166in}{3.336467in}}%
\pgfusepath{stroke}%
\end{pgfscope}%
\begin{pgfscope}%
\pgfpathrectangle{\pgfqpoint{9.810417in}{2.500000in}}{\pgfqpoint{5.489583in}{0.877907in}}%
\pgfusepath{clip}%
\pgfsetbuttcap%
\pgfsetroundjoin%
\pgfsetlinewidth{1.505625pt}%
\definecolor{currentstroke}{rgb}{0.000000,0.000000,0.000000}%
\pgfsetstrokecolor{currentstroke}%
\pgfsetdash{}{0pt}%
\pgfpathmoveto{\pgfqpoint{10.306389in}{2.578239in}}%
\pgfpathlineto{\pgfqpoint{10.306389in}{2.587040in}}%
\pgfusepath{stroke}%
\end{pgfscope}%
\begin{pgfscope}%
\pgfpathrectangle{\pgfqpoint{9.810417in}{2.500000in}}{\pgfqpoint{5.489583in}{0.877907in}}%
\pgfusepath{clip}%
\pgfsetbuttcap%
\pgfsetroundjoin%
\pgfsetlinewidth{1.505625pt}%
\definecolor{currentstroke}{rgb}{0.000000,0.000000,0.000000}%
\pgfsetstrokecolor{currentstroke}%
\pgfsetdash{}{0pt}%
\pgfpathmoveto{\pgfqpoint{10.429612in}{2.578239in}}%
\pgfpathlineto{\pgfqpoint{10.429612in}{2.591726in}}%
\pgfusepath{stroke}%
\end{pgfscope}%
\begin{pgfscope}%
\pgfpathrectangle{\pgfqpoint{9.810417in}{2.500000in}}{\pgfqpoint{5.489583in}{0.877907in}}%
\pgfusepath{clip}%
\pgfsetbuttcap%
\pgfsetroundjoin%
\pgfsetlinewidth{1.505625pt}%
\definecolor{currentstroke}{rgb}{0.000000,0.000000,0.000000}%
\pgfsetstrokecolor{currentstroke}%
\pgfsetdash{}{0pt}%
\pgfpathmoveto{\pgfqpoint{10.552835in}{2.578239in}}%
\pgfpathlineto{\pgfqpoint{10.552835in}{2.585178in}}%
\pgfusepath{stroke}%
\end{pgfscope}%
\begin{pgfscope}%
\pgfpathrectangle{\pgfqpoint{9.810417in}{2.500000in}}{\pgfqpoint{5.489583in}{0.877907in}}%
\pgfusepath{clip}%
\pgfsetbuttcap%
\pgfsetroundjoin%
\pgfsetlinewidth{1.505625pt}%
\definecolor{currentstroke}{rgb}{0.000000,0.000000,0.000000}%
\pgfsetstrokecolor{currentstroke}%
\pgfsetdash{}{0pt}%
\pgfpathmoveto{\pgfqpoint{10.676058in}{2.578239in}}%
\pgfpathlineto{\pgfqpoint{10.676058in}{2.579114in}}%
\pgfusepath{stroke}%
\end{pgfscope}%
\begin{pgfscope}%
\pgfpathrectangle{\pgfqpoint{9.810417in}{2.500000in}}{\pgfqpoint{5.489583in}{0.877907in}}%
\pgfusepath{clip}%
\pgfsetbuttcap%
\pgfsetroundjoin%
\pgfsetlinewidth{1.505625pt}%
\definecolor{currentstroke}{rgb}{0.000000,0.000000,0.000000}%
\pgfsetstrokecolor{currentstroke}%
\pgfsetdash{}{0pt}%
\pgfpathmoveto{\pgfqpoint{10.799281in}{2.578239in}}%
\pgfpathlineto{\pgfqpoint{10.799281in}{2.588180in}}%
\pgfusepath{stroke}%
\end{pgfscope}%
\begin{pgfscope}%
\pgfpathrectangle{\pgfqpoint{9.810417in}{2.500000in}}{\pgfqpoint{5.489583in}{0.877907in}}%
\pgfusepath{clip}%
\pgfsetbuttcap%
\pgfsetroundjoin%
\pgfsetlinewidth{1.505625pt}%
\definecolor{currentstroke}{rgb}{0.000000,0.000000,0.000000}%
\pgfsetstrokecolor{currentstroke}%
\pgfsetdash{}{0pt}%
\pgfpathmoveto{\pgfqpoint{10.922504in}{2.578239in}}%
\pgfpathlineto{\pgfqpoint{10.922504in}{2.559086in}}%
\pgfusepath{stroke}%
\end{pgfscope}%
\begin{pgfscope}%
\pgfpathrectangle{\pgfqpoint{9.810417in}{2.500000in}}{\pgfqpoint{5.489583in}{0.877907in}}%
\pgfusepath{clip}%
\pgfsetbuttcap%
\pgfsetroundjoin%
\pgfsetlinewidth{1.505625pt}%
\definecolor{currentstroke}{rgb}{0.000000,0.000000,0.000000}%
\pgfsetstrokecolor{currentstroke}%
\pgfsetdash{}{0pt}%
\pgfpathmoveto{\pgfqpoint{11.045727in}{2.578239in}}%
\pgfpathlineto{\pgfqpoint{11.045727in}{2.563569in}}%
\pgfusepath{stroke}%
\end{pgfscope}%
\begin{pgfscope}%
\pgfpathrectangle{\pgfqpoint{9.810417in}{2.500000in}}{\pgfqpoint{5.489583in}{0.877907in}}%
\pgfusepath{clip}%
\pgfsetbuttcap%
\pgfsetroundjoin%
\pgfsetlinewidth{1.505625pt}%
\definecolor{currentstroke}{rgb}{0.000000,0.000000,0.000000}%
\pgfsetstrokecolor{currentstroke}%
\pgfsetdash{}{0pt}%
\pgfpathmoveto{\pgfqpoint{11.168950in}{2.578239in}}%
\pgfpathlineto{\pgfqpoint{11.168950in}{2.562190in}}%
\pgfusepath{stroke}%
\end{pgfscope}%
\begin{pgfscope}%
\pgfpathrectangle{\pgfqpoint{9.810417in}{2.500000in}}{\pgfqpoint{5.489583in}{0.877907in}}%
\pgfusepath{clip}%
\pgfsetbuttcap%
\pgfsetroundjoin%
\pgfsetlinewidth{1.505625pt}%
\definecolor{currentstroke}{rgb}{0.000000,0.000000,0.000000}%
\pgfsetstrokecolor{currentstroke}%
\pgfsetdash{}{0pt}%
\pgfpathmoveto{\pgfqpoint{11.292173in}{2.578239in}}%
\pgfpathlineto{\pgfqpoint{11.292173in}{2.575613in}}%
\pgfusepath{stroke}%
\end{pgfscope}%
\begin{pgfscope}%
\pgfpathrectangle{\pgfqpoint{9.810417in}{2.500000in}}{\pgfqpoint{5.489583in}{0.877907in}}%
\pgfusepath{clip}%
\pgfsetbuttcap%
\pgfsetroundjoin%
\pgfsetlinewidth{1.505625pt}%
\definecolor{currentstroke}{rgb}{0.000000,0.000000,0.000000}%
\pgfsetstrokecolor{currentstroke}%
\pgfsetdash{}{0pt}%
\pgfpathmoveto{\pgfqpoint{11.415396in}{2.578239in}}%
\pgfpathlineto{\pgfqpoint{11.415396in}{2.579625in}}%
\pgfusepath{stroke}%
\end{pgfscope}%
\begin{pgfscope}%
\pgfpathrectangle{\pgfqpoint{9.810417in}{2.500000in}}{\pgfqpoint{5.489583in}{0.877907in}}%
\pgfusepath{clip}%
\pgfsetbuttcap%
\pgfsetroundjoin%
\pgfsetlinewidth{1.505625pt}%
\definecolor{currentstroke}{rgb}{0.000000,0.000000,0.000000}%
\pgfsetstrokecolor{currentstroke}%
\pgfsetdash{}{0pt}%
\pgfpathmoveto{\pgfqpoint{11.538619in}{2.578239in}}%
\pgfpathlineto{\pgfqpoint{11.538619in}{2.583756in}}%
\pgfusepath{stroke}%
\end{pgfscope}%
\begin{pgfscope}%
\pgfpathrectangle{\pgfqpoint{9.810417in}{2.500000in}}{\pgfqpoint{5.489583in}{0.877907in}}%
\pgfusepath{clip}%
\pgfsetbuttcap%
\pgfsetroundjoin%
\pgfsetlinewidth{1.505625pt}%
\definecolor{currentstroke}{rgb}{0.000000,0.000000,0.000000}%
\pgfsetstrokecolor{currentstroke}%
\pgfsetdash{}{0pt}%
\pgfpathmoveto{\pgfqpoint{11.661842in}{2.578239in}}%
\pgfpathlineto{\pgfqpoint{11.661842in}{2.580273in}}%
\pgfusepath{stroke}%
\end{pgfscope}%
\begin{pgfscope}%
\pgfpathrectangle{\pgfqpoint{9.810417in}{2.500000in}}{\pgfqpoint{5.489583in}{0.877907in}}%
\pgfusepath{clip}%
\pgfsetbuttcap%
\pgfsetroundjoin%
\pgfsetlinewidth{1.505625pt}%
\definecolor{currentstroke}{rgb}{0.000000,0.000000,0.000000}%
\pgfsetstrokecolor{currentstroke}%
\pgfsetdash{}{0pt}%
\pgfpathmoveto{\pgfqpoint{11.785065in}{2.578239in}}%
\pgfpathlineto{\pgfqpoint{11.785065in}{2.583029in}}%
\pgfusepath{stroke}%
\end{pgfscope}%
\begin{pgfscope}%
\pgfpathrectangle{\pgfqpoint{9.810417in}{2.500000in}}{\pgfqpoint{5.489583in}{0.877907in}}%
\pgfusepath{clip}%
\pgfsetbuttcap%
\pgfsetroundjoin%
\pgfsetlinewidth{1.505625pt}%
\definecolor{currentstroke}{rgb}{0.000000,0.000000,0.000000}%
\pgfsetstrokecolor{currentstroke}%
\pgfsetdash{}{0pt}%
\pgfpathmoveto{\pgfqpoint{11.908288in}{2.578239in}}%
\pgfpathlineto{\pgfqpoint{11.908288in}{2.586527in}}%
\pgfusepath{stroke}%
\end{pgfscope}%
\begin{pgfscope}%
\pgfpathrectangle{\pgfqpoint{9.810417in}{2.500000in}}{\pgfqpoint{5.489583in}{0.877907in}}%
\pgfusepath{clip}%
\pgfsetbuttcap%
\pgfsetroundjoin%
\pgfsetlinewidth{1.505625pt}%
\definecolor{currentstroke}{rgb}{0.000000,0.000000,0.000000}%
\pgfsetstrokecolor{currentstroke}%
\pgfsetdash{}{0pt}%
\pgfpathmoveto{\pgfqpoint{12.031511in}{2.578239in}}%
\pgfpathlineto{\pgfqpoint{12.031511in}{2.587634in}}%
\pgfusepath{stroke}%
\end{pgfscope}%
\begin{pgfscope}%
\pgfpathrectangle{\pgfqpoint{9.810417in}{2.500000in}}{\pgfqpoint{5.489583in}{0.877907in}}%
\pgfusepath{clip}%
\pgfsetbuttcap%
\pgfsetroundjoin%
\pgfsetlinewidth{1.505625pt}%
\definecolor{currentstroke}{rgb}{0.000000,0.000000,0.000000}%
\pgfsetstrokecolor{currentstroke}%
\pgfsetdash{}{0pt}%
\pgfpathmoveto{\pgfqpoint{12.154734in}{2.578239in}}%
\pgfpathlineto{\pgfqpoint{12.154734in}{2.573724in}}%
\pgfusepath{stroke}%
\end{pgfscope}%
\begin{pgfscope}%
\pgfpathrectangle{\pgfqpoint{9.810417in}{2.500000in}}{\pgfqpoint{5.489583in}{0.877907in}}%
\pgfusepath{clip}%
\pgfsetbuttcap%
\pgfsetroundjoin%
\pgfsetlinewidth{1.505625pt}%
\definecolor{currentstroke}{rgb}{0.000000,0.000000,0.000000}%
\pgfsetstrokecolor{currentstroke}%
\pgfsetdash{}{0pt}%
\pgfpathmoveto{\pgfqpoint{12.277957in}{2.578239in}}%
\pgfpathlineto{\pgfqpoint{12.277957in}{2.560912in}}%
\pgfusepath{stroke}%
\end{pgfscope}%
\begin{pgfscope}%
\pgfpathrectangle{\pgfqpoint{9.810417in}{2.500000in}}{\pgfqpoint{5.489583in}{0.877907in}}%
\pgfusepath{clip}%
\pgfsetbuttcap%
\pgfsetroundjoin%
\pgfsetlinewidth{1.505625pt}%
\definecolor{currentstroke}{rgb}{0.000000,0.000000,0.000000}%
\pgfsetstrokecolor{currentstroke}%
\pgfsetdash{}{0pt}%
\pgfpathmoveto{\pgfqpoint{12.401180in}{2.578239in}}%
\pgfpathlineto{\pgfqpoint{12.401180in}{2.557842in}}%
\pgfusepath{stroke}%
\end{pgfscope}%
\begin{pgfscope}%
\pgfpathrectangle{\pgfqpoint{9.810417in}{2.500000in}}{\pgfqpoint{5.489583in}{0.877907in}}%
\pgfusepath{clip}%
\pgfsetbuttcap%
\pgfsetroundjoin%
\pgfsetlinewidth{1.505625pt}%
\definecolor{currentstroke}{rgb}{0.000000,0.000000,0.000000}%
\pgfsetstrokecolor{currentstroke}%
\pgfsetdash{}{0pt}%
\pgfpathmoveto{\pgfqpoint{12.524403in}{2.578239in}}%
\pgfpathlineto{\pgfqpoint{12.524403in}{2.584788in}}%
\pgfusepath{stroke}%
\end{pgfscope}%
\begin{pgfscope}%
\pgfpathrectangle{\pgfqpoint{9.810417in}{2.500000in}}{\pgfqpoint{5.489583in}{0.877907in}}%
\pgfusepath{clip}%
\pgfsetbuttcap%
\pgfsetroundjoin%
\pgfsetlinewidth{1.505625pt}%
\definecolor{currentstroke}{rgb}{0.000000,0.000000,0.000000}%
\pgfsetstrokecolor{currentstroke}%
\pgfsetdash{}{0pt}%
\pgfpathmoveto{\pgfqpoint{12.647626in}{2.578239in}}%
\pgfpathlineto{\pgfqpoint{12.647626in}{2.592913in}}%
\pgfusepath{stroke}%
\end{pgfscope}%
\begin{pgfscope}%
\pgfpathrectangle{\pgfqpoint{9.810417in}{2.500000in}}{\pgfqpoint{5.489583in}{0.877907in}}%
\pgfusepath{clip}%
\pgfsetbuttcap%
\pgfsetroundjoin%
\pgfsetlinewidth{1.505625pt}%
\definecolor{currentstroke}{rgb}{0.000000,0.000000,0.000000}%
\pgfsetstrokecolor{currentstroke}%
\pgfsetdash{}{0pt}%
\pgfpathmoveto{\pgfqpoint{12.770849in}{2.578239in}}%
\pgfpathlineto{\pgfqpoint{12.770849in}{2.570571in}}%
\pgfusepath{stroke}%
\end{pgfscope}%
\begin{pgfscope}%
\pgfpathrectangle{\pgfqpoint{9.810417in}{2.500000in}}{\pgfqpoint{5.489583in}{0.877907in}}%
\pgfusepath{clip}%
\pgfsetbuttcap%
\pgfsetroundjoin%
\pgfsetlinewidth{1.505625pt}%
\definecolor{currentstroke}{rgb}{0.000000,0.000000,0.000000}%
\pgfsetstrokecolor{currentstroke}%
\pgfsetdash{}{0pt}%
\pgfpathmoveto{\pgfqpoint{12.894072in}{2.578239in}}%
\pgfpathlineto{\pgfqpoint{12.894072in}{2.613410in}}%
\pgfusepath{stroke}%
\end{pgfscope}%
\begin{pgfscope}%
\pgfpathrectangle{\pgfqpoint{9.810417in}{2.500000in}}{\pgfqpoint{5.489583in}{0.877907in}}%
\pgfusepath{clip}%
\pgfsetbuttcap%
\pgfsetroundjoin%
\pgfsetlinewidth{1.505625pt}%
\definecolor{currentstroke}{rgb}{0.000000,0.000000,0.000000}%
\pgfsetstrokecolor{currentstroke}%
\pgfsetdash{}{0pt}%
\pgfpathmoveto{\pgfqpoint{13.017294in}{2.578239in}}%
\pgfpathlineto{\pgfqpoint{13.017294in}{2.588035in}}%
\pgfusepath{stroke}%
\end{pgfscope}%
\begin{pgfscope}%
\pgfpathrectangle{\pgfqpoint{9.810417in}{2.500000in}}{\pgfqpoint{5.489583in}{0.877907in}}%
\pgfusepath{clip}%
\pgfsetbuttcap%
\pgfsetroundjoin%
\pgfsetlinewidth{1.505625pt}%
\definecolor{currentstroke}{rgb}{0.000000,0.000000,0.000000}%
\pgfsetstrokecolor{currentstroke}%
\pgfsetdash{}{0pt}%
\pgfpathmoveto{\pgfqpoint{13.140517in}{2.578239in}}%
\pgfpathlineto{\pgfqpoint{13.140517in}{2.580547in}}%
\pgfusepath{stroke}%
\end{pgfscope}%
\begin{pgfscope}%
\pgfpathrectangle{\pgfqpoint{9.810417in}{2.500000in}}{\pgfqpoint{5.489583in}{0.877907in}}%
\pgfusepath{clip}%
\pgfsetbuttcap%
\pgfsetroundjoin%
\pgfsetlinewidth{1.505625pt}%
\definecolor{currentstroke}{rgb}{0.000000,0.000000,0.000000}%
\pgfsetstrokecolor{currentstroke}%
\pgfsetdash{}{0pt}%
\pgfpathmoveto{\pgfqpoint{13.263740in}{2.578239in}}%
\pgfpathlineto{\pgfqpoint{13.263740in}{2.577271in}}%
\pgfusepath{stroke}%
\end{pgfscope}%
\begin{pgfscope}%
\pgfpathrectangle{\pgfqpoint{9.810417in}{2.500000in}}{\pgfqpoint{5.489583in}{0.877907in}}%
\pgfusepath{clip}%
\pgfsetbuttcap%
\pgfsetroundjoin%
\pgfsetlinewidth{1.505625pt}%
\definecolor{currentstroke}{rgb}{0.000000,0.000000,0.000000}%
\pgfsetstrokecolor{currentstroke}%
\pgfsetdash{}{0pt}%
\pgfpathmoveto{\pgfqpoint{13.386963in}{2.578239in}}%
\pgfpathlineto{\pgfqpoint{13.386963in}{2.575820in}}%
\pgfusepath{stroke}%
\end{pgfscope}%
\begin{pgfscope}%
\pgfpathrectangle{\pgfqpoint{9.810417in}{2.500000in}}{\pgfqpoint{5.489583in}{0.877907in}}%
\pgfusepath{clip}%
\pgfsetbuttcap%
\pgfsetroundjoin%
\pgfsetlinewidth{1.505625pt}%
\definecolor{currentstroke}{rgb}{0.000000,0.000000,0.000000}%
\pgfsetstrokecolor{currentstroke}%
\pgfsetdash{}{0pt}%
\pgfpathmoveto{\pgfqpoint{13.510186in}{2.578239in}}%
\pgfpathlineto{\pgfqpoint{13.510186in}{2.571232in}}%
\pgfusepath{stroke}%
\end{pgfscope}%
\begin{pgfscope}%
\pgfpathrectangle{\pgfqpoint{9.810417in}{2.500000in}}{\pgfqpoint{5.489583in}{0.877907in}}%
\pgfusepath{clip}%
\pgfsetbuttcap%
\pgfsetroundjoin%
\pgfsetlinewidth{1.505625pt}%
\definecolor{currentstroke}{rgb}{0.000000,0.000000,0.000000}%
\pgfsetstrokecolor{currentstroke}%
\pgfsetdash{}{0pt}%
\pgfpathmoveto{\pgfqpoint{13.633409in}{2.578239in}}%
\pgfpathlineto{\pgfqpoint{13.633409in}{2.587742in}}%
\pgfusepath{stroke}%
\end{pgfscope}%
\begin{pgfscope}%
\pgfpathrectangle{\pgfqpoint{9.810417in}{2.500000in}}{\pgfqpoint{5.489583in}{0.877907in}}%
\pgfusepath{clip}%
\pgfsetbuttcap%
\pgfsetroundjoin%
\pgfsetlinewidth{1.505625pt}%
\definecolor{currentstroke}{rgb}{0.000000,0.000000,0.000000}%
\pgfsetstrokecolor{currentstroke}%
\pgfsetdash{}{0pt}%
\pgfpathmoveto{\pgfqpoint{13.756632in}{2.578239in}}%
\pgfpathlineto{\pgfqpoint{13.756632in}{2.570066in}}%
\pgfusepath{stroke}%
\end{pgfscope}%
\begin{pgfscope}%
\pgfpathrectangle{\pgfqpoint{9.810417in}{2.500000in}}{\pgfqpoint{5.489583in}{0.877907in}}%
\pgfusepath{clip}%
\pgfsetbuttcap%
\pgfsetroundjoin%
\pgfsetlinewidth{1.505625pt}%
\definecolor{currentstroke}{rgb}{0.000000,0.000000,0.000000}%
\pgfsetstrokecolor{currentstroke}%
\pgfsetdash{}{0pt}%
\pgfpathmoveto{\pgfqpoint{13.879855in}{2.578239in}}%
\pgfpathlineto{\pgfqpoint{13.879855in}{2.564442in}}%
\pgfusepath{stroke}%
\end{pgfscope}%
\begin{pgfscope}%
\pgfpathrectangle{\pgfqpoint{9.810417in}{2.500000in}}{\pgfqpoint{5.489583in}{0.877907in}}%
\pgfusepath{clip}%
\pgfsetbuttcap%
\pgfsetroundjoin%
\pgfsetlinewidth{1.505625pt}%
\definecolor{currentstroke}{rgb}{0.000000,0.000000,0.000000}%
\pgfsetstrokecolor{currentstroke}%
\pgfsetdash{}{0pt}%
\pgfpathmoveto{\pgfqpoint{14.003078in}{2.578239in}}%
\pgfpathlineto{\pgfqpoint{14.003078in}{2.581334in}}%
\pgfusepath{stroke}%
\end{pgfscope}%
\begin{pgfscope}%
\pgfpathrectangle{\pgfqpoint{9.810417in}{2.500000in}}{\pgfqpoint{5.489583in}{0.877907in}}%
\pgfusepath{clip}%
\pgfsetbuttcap%
\pgfsetroundjoin%
\pgfsetlinewidth{1.505625pt}%
\definecolor{currentstroke}{rgb}{0.000000,0.000000,0.000000}%
\pgfsetstrokecolor{currentstroke}%
\pgfsetdash{}{0pt}%
\pgfpathmoveto{\pgfqpoint{14.126301in}{2.578239in}}%
\pgfpathlineto{\pgfqpoint{14.126301in}{2.580231in}}%
\pgfusepath{stroke}%
\end{pgfscope}%
\begin{pgfscope}%
\pgfpathrectangle{\pgfqpoint{9.810417in}{2.500000in}}{\pgfqpoint{5.489583in}{0.877907in}}%
\pgfusepath{clip}%
\pgfsetbuttcap%
\pgfsetroundjoin%
\pgfsetlinewidth{1.505625pt}%
\definecolor{currentstroke}{rgb}{0.000000,0.000000,0.000000}%
\pgfsetstrokecolor{currentstroke}%
\pgfsetdash{}{0pt}%
\pgfpathmoveto{\pgfqpoint{14.249524in}{2.578239in}}%
\pgfpathlineto{\pgfqpoint{14.249524in}{2.573166in}}%
\pgfusepath{stroke}%
\end{pgfscope}%
\begin{pgfscope}%
\pgfpathrectangle{\pgfqpoint{9.810417in}{2.500000in}}{\pgfqpoint{5.489583in}{0.877907in}}%
\pgfusepath{clip}%
\pgfsetbuttcap%
\pgfsetroundjoin%
\pgfsetlinewidth{1.505625pt}%
\definecolor{currentstroke}{rgb}{0.000000,0.000000,0.000000}%
\pgfsetstrokecolor{currentstroke}%
\pgfsetdash{}{0pt}%
\pgfpathmoveto{\pgfqpoint{14.372747in}{2.578239in}}%
\pgfpathlineto{\pgfqpoint{14.372747in}{2.577758in}}%
\pgfusepath{stroke}%
\end{pgfscope}%
\begin{pgfscope}%
\pgfpathrectangle{\pgfqpoint{9.810417in}{2.500000in}}{\pgfqpoint{5.489583in}{0.877907in}}%
\pgfusepath{clip}%
\pgfsetbuttcap%
\pgfsetroundjoin%
\pgfsetlinewidth{1.505625pt}%
\definecolor{currentstroke}{rgb}{0.000000,0.000000,0.000000}%
\pgfsetstrokecolor{currentstroke}%
\pgfsetdash{}{0pt}%
\pgfpathmoveto{\pgfqpoint{14.495970in}{2.578239in}}%
\pgfpathlineto{\pgfqpoint{14.495970in}{2.590639in}}%
\pgfusepath{stroke}%
\end{pgfscope}%
\begin{pgfscope}%
\pgfpathrectangle{\pgfqpoint{9.810417in}{2.500000in}}{\pgfqpoint{5.489583in}{0.877907in}}%
\pgfusepath{clip}%
\pgfsetbuttcap%
\pgfsetroundjoin%
\pgfsetlinewidth{1.505625pt}%
\definecolor{currentstroke}{rgb}{0.000000,0.000000,0.000000}%
\pgfsetstrokecolor{currentstroke}%
\pgfsetdash{}{0pt}%
\pgfpathmoveto{\pgfqpoint{14.619193in}{2.578239in}}%
\pgfpathlineto{\pgfqpoint{14.619193in}{2.575078in}}%
\pgfusepath{stroke}%
\end{pgfscope}%
\begin{pgfscope}%
\pgfpathrectangle{\pgfqpoint{9.810417in}{2.500000in}}{\pgfqpoint{5.489583in}{0.877907in}}%
\pgfusepath{clip}%
\pgfsetbuttcap%
\pgfsetroundjoin%
\pgfsetlinewidth{1.505625pt}%
\definecolor{currentstroke}{rgb}{0.000000,0.000000,0.000000}%
\pgfsetstrokecolor{currentstroke}%
\pgfsetdash{}{0pt}%
\pgfpathmoveto{\pgfqpoint{14.742416in}{2.578239in}}%
\pgfpathlineto{\pgfqpoint{14.742416in}{2.585655in}}%
\pgfusepath{stroke}%
\end{pgfscope}%
\begin{pgfscope}%
\pgfpathrectangle{\pgfqpoint{9.810417in}{2.500000in}}{\pgfqpoint{5.489583in}{0.877907in}}%
\pgfusepath{clip}%
\pgfsetbuttcap%
\pgfsetroundjoin%
\pgfsetlinewidth{1.505625pt}%
\definecolor{currentstroke}{rgb}{0.000000,0.000000,0.000000}%
\pgfsetstrokecolor{currentstroke}%
\pgfsetdash{}{0pt}%
\pgfpathmoveto{\pgfqpoint{14.865639in}{2.578239in}}%
\pgfpathlineto{\pgfqpoint{14.865639in}{2.602179in}}%
\pgfusepath{stroke}%
\end{pgfscope}%
\begin{pgfscope}%
\pgfpathrectangle{\pgfqpoint{9.810417in}{2.500000in}}{\pgfqpoint{5.489583in}{0.877907in}}%
\pgfusepath{clip}%
\pgfsetbuttcap%
\pgfsetroundjoin%
\pgfsetlinewidth{1.505625pt}%
\definecolor{currentstroke}{rgb}{0.000000,0.000000,0.000000}%
\pgfsetstrokecolor{currentstroke}%
\pgfsetdash{}{0pt}%
\pgfpathmoveto{\pgfqpoint{14.988862in}{2.578239in}}%
\pgfpathlineto{\pgfqpoint{14.988862in}{2.589722in}}%
\pgfusepath{stroke}%
\end{pgfscope}%
\begin{pgfscope}%
\pgfpathrectangle{\pgfqpoint{9.810417in}{2.500000in}}{\pgfqpoint{5.489583in}{0.877907in}}%
\pgfusepath{clip}%
\pgfsetroundcap%
\pgfsetroundjoin%
\pgfsetlinewidth{1.505625pt}%
\definecolor{currentstroke}{rgb}{0.121569,0.466667,0.705882}%
\pgfsetstrokecolor{currentstroke}%
\pgfsetdash{}{0pt}%
\pgfpathmoveto{\pgfqpoint{9.810417in}{2.578239in}}%
\pgfpathlineto{\pgfqpoint{15.300000in}{2.578239in}}%
\pgfusepath{stroke}%
\end{pgfscope}%
\begin{pgfscope}%
\pgfpathrectangle{\pgfqpoint{9.810417in}{2.500000in}}{\pgfqpoint{5.489583in}{0.877907in}}%
\pgfusepath{clip}%
\pgfsetbuttcap%
\pgfsetroundjoin%
\definecolor{currentfill}{rgb}{0.121569,0.466667,0.705882}%
\pgfsetfillcolor{currentfill}%
\pgfsetlinewidth{1.003750pt}%
\definecolor{currentstroke}{rgb}{0.121569,0.466667,0.705882}%
\pgfsetstrokecolor{currentstroke}%
\pgfsetdash{}{0pt}%
\pgfsys@defobject{currentmarker}{\pgfqpoint{-0.034722in}{-0.034722in}}{\pgfqpoint{0.034722in}{0.034722in}}{%
\pgfpathmoveto{\pgfqpoint{0.000000in}{-0.034722in}}%
\pgfpathcurveto{\pgfqpoint{0.009208in}{-0.034722in}}{\pgfqpoint{0.018041in}{-0.031064in}}{\pgfqpoint{0.024552in}{-0.024552in}}%
\pgfpathcurveto{\pgfqpoint{0.031064in}{-0.018041in}}{\pgfqpoint{0.034722in}{-0.009208in}}{\pgfqpoint{0.034722in}{0.000000in}}%
\pgfpathcurveto{\pgfqpoint{0.034722in}{0.009208in}}{\pgfqpoint{0.031064in}{0.018041in}}{\pgfqpoint{0.024552in}{0.024552in}}%
\pgfpathcurveto{\pgfqpoint{0.018041in}{0.031064in}}{\pgfqpoint{0.009208in}{0.034722in}}{\pgfqpoint{0.000000in}{0.034722in}}%
\pgfpathcurveto{\pgfqpoint{-0.009208in}{0.034722in}}{\pgfqpoint{-0.018041in}{0.031064in}}{\pgfqpoint{-0.024552in}{0.024552in}}%
\pgfpathcurveto{\pgfqpoint{-0.031064in}{0.018041in}}{\pgfqpoint{-0.034722in}{0.009208in}}{\pgfqpoint{-0.034722in}{0.000000in}}%
\pgfpathcurveto{\pgfqpoint{-0.034722in}{-0.009208in}}{\pgfqpoint{-0.031064in}{-0.018041in}}{\pgfqpoint{-0.024552in}{-0.024552in}}%
\pgfpathcurveto{\pgfqpoint{-0.018041in}{-0.031064in}}{\pgfqpoint{-0.009208in}{-0.034722in}}{\pgfqpoint{0.000000in}{-0.034722in}}%
\pgfpathclose%
\pgfusepath{stroke,fill}%
}%
\begin{pgfscope}%
\pgfsys@transformshift{10.059943in}{3.338002in}%
\pgfsys@useobject{currentmarker}{}%
\end{pgfscope}%
\begin{pgfscope}%
\pgfsys@transformshift{10.183166in}{3.336467in}%
\pgfsys@useobject{currentmarker}{}%
\end{pgfscope}%
\begin{pgfscope}%
\pgfsys@transformshift{10.306389in}{2.587040in}%
\pgfsys@useobject{currentmarker}{}%
\end{pgfscope}%
\begin{pgfscope}%
\pgfsys@transformshift{10.429612in}{2.591726in}%
\pgfsys@useobject{currentmarker}{}%
\end{pgfscope}%
\begin{pgfscope}%
\pgfsys@transformshift{10.552835in}{2.585178in}%
\pgfsys@useobject{currentmarker}{}%
\end{pgfscope}%
\begin{pgfscope}%
\pgfsys@transformshift{10.676058in}{2.579114in}%
\pgfsys@useobject{currentmarker}{}%
\end{pgfscope}%
\begin{pgfscope}%
\pgfsys@transformshift{10.799281in}{2.588180in}%
\pgfsys@useobject{currentmarker}{}%
\end{pgfscope}%
\begin{pgfscope}%
\pgfsys@transformshift{10.922504in}{2.559086in}%
\pgfsys@useobject{currentmarker}{}%
\end{pgfscope}%
\begin{pgfscope}%
\pgfsys@transformshift{11.045727in}{2.563569in}%
\pgfsys@useobject{currentmarker}{}%
\end{pgfscope}%
\begin{pgfscope}%
\pgfsys@transformshift{11.168950in}{2.562190in}%
\pgfsys@useobject{currentmarker}{}%
\end{pgfscope}%
\begin{pgfscope}%
\pgfsys@transformshift{11.292173in}{2.575613in}%
\pgfsys@useobject{currentmarker}{}%
\end{pgfscope}%
\begin{pgfscope}%
\pgfsys@transformshift{11.415396in}{2.579625in}%
\pgfsys@useobject{currentmarker}{}%
\end{pgfscope}%
\begin{pgfscope}%
\pgfsys@transformshift{11.538619in}{2.583756in}%
\pgfsys@useobject{currentmarker}{}%
\end{pgfscope}%
\begin{pgfscope}%
\pgfsys@transformshift{11.661842in}{2.580273in}%
\pgfsys@useobject{currentmarker}{}%
\end{pgfscope}%
\begin{pgfscope}%
\pgfsys@transformshift{11.785065in}{2.583029in}%
\pgfsys@useobject{currentmarker}{}%
\end{pgfscope}%
\begin{pgfscope}%
\pgfsys@transformshift{11.908288in}{2.586527in}%
\pgfsys@useobject{currentmarker}{}%
\end{pgfscope}%
\begin{pgfscope}%
\pgfsys@transformshift{12.031511in}{2.587634in}%
\pgfsys@useobject{currentmarker}{}%
\end{pgfscope}%
\begin{pgfscope}%
\pgfsys@transformshift{12.154734in}{2.573724in}%
\pgfsys@useobject{currentmarker}{}%
\end{pgfscope}%
\begin{pgfscope}%
\pgfsys@transformshift{12.277957in}{2.560912in}%
\pgfsys@useobject{currentmarker}{}%
\end{pgfscope}%
\begin{pgfscope}%
\pgfsys@transformshift{12.401180in}{2.557842in}%
\pgfsys@useobject{currentmarker}{}%
\end{pgfscope}%
\begin{pgfscope}%
\pgfsys@transformshift{12.524403in}{2.584788in}%
\pgfsys@useobject{currentmarker}{}%
\end{pgfscope}%
\begin{pgfscope}%
\pgfsys@transformshift{12.647626in}{2.592913in}%
\pgfsys@useobject{currentmarker}{}%
\end{pgfscope}%
\begin{pgfscope}%
\pgfsys@transformshift{12.770849in}{2.570571in}%
\pgfsys@useobject{currentmarker}{}%
\end{pgfscope}%
\begin{pgfscope}%
\pgfsys@transformshift{12.894072in}{2.613410in}%
\pgfsys@useobject{currentmarker}{}%
\end{pgfscope}%
\begin{pgfscope}%
\pgfsys@transformshift{13.017294in}{2.588035in}%
\pgfsys@useobject{currentmarker}{}%
\end{pgfscope}%
\begin{pgfscope}%
\pgfsys@transformshift{13.140517in}{2.580547in}%
\pgfsys@useobject{currentmarker}{}%
\end{pgfscope}%
\begin{pgfscope}%
\pgfsys@transformshift{13.263740in}{2.577271in}%
\pgfsys@useobject{currentmarker}{}%
\end{pgfscope}%
\begin{pgfscope}%
\pgfsys@transformshift{13.386963in}{2.575820in}%
\pgfsys@useobject{currentmarker}{}%
\end{pgfscope}%
\begin{pgfscope}%
\pgfsys@transformshift{13.510186in}{2.571232in}%
\pgfsys@useobject{currentmarker}{}%
\end{pgfscope}%
\begin{pgfscope}%
\pgfsys@transformshift{13.633409in}{2.587742in}%
\pgfsys@useobject{currentmarker}{}%
\end{pgfscope}%
\begin{pgfscope}%
\pgfsys@transformshift{13.756632in}{2.570066in}%
\pgfsys@useobject{currentmarker}{}%
\end{pgfscope}%
\begin{pgfscope}%
\pgfsys@transformshift{13.879855in}{2.564442in}%
\pgfsys@useobject{currentmarker}{}%
\end{pgfscope}%
\begin{pgfscope}%
\pgfsys@transformshift{14.003078in}{2.581334in}%
\pgfsys@useobject{currentmarker}{}%
\end{pgfscope}%
\begin{pgfscope}%
\pgfsys@transformshift{14.126301in}{2.580231in}%
\pgfsys@useobject{currentmarker}{}%
\end{pgfscope}%
\begin{pgfscope}%
\pgfsys@transformshift{14.249524in}{2.573166in}%
\pgfsys@useobject{currentmarker}{}%
\end{pgfscope}%
\begin{pgfscope}%
\pgfsys@transformshift{14.372747in}{2.577758in}%
\pgfsys@useobject{currentmarker}{}%
\end{pgfscope}%
\begin{pgfscope}%
\pgfsys@transformshift{14.495970in}{2.590639in}%
\pgfsys@useobject{currentmarker}{}%
\end{pgfscope}%
\begin{pgfscope}%
\pgfsys@transformshift{14.619193in}{2.575078in}%
\pgfsys@useobject{currentmarker}{}%
\end{pgfscope}%
\begin{pgfscope}%
\pgfsys@transformshift{14.742416in}{2.585655in}%
\pgfsys@useobject{currentmarker}{}%
\end{pgfscope}%
\begin{pgfscope}%
\pgfsys@transformshift{14.865639in}{2.602179in}%
\pgfsys@useobject{currentmarker}{}%
\end{pgfscope}%
\begin{pgfscope}%
\pgfsys@transformshift{14.988862in}{2.589722in}%
\pgfsys@useobject{currentmarker}{}%
\end{pgfscope}%
\end{pgfscope}%
\begin{pgfscope}%
\pgfsetrectcap%
\pgfsetmiterjoin%
\pgfsetlinewidth{0.803000pt}%
\definecolor{currentstroke}{rgb}{1.000000,1.000000,1.000000}%
\pgfsetstrokecolor{currentstroke}%
\pgfsetdash{}{0pt}%
\pgfpathmoveto{\pgfqpoint{9.810417in}{2.500000in}}%
\pgfpathlineto{\pgfqpoint{9.810417in}{3.377907in}}%
\pgfusepath{stroke}%
\end{pgfscope}%
\begin{pgfscope}%
\pgfsetrectcap%
\pgfsetmiterjoin%
\pgfsetlinewidth{0.803000pt}%
\definecolor{currentstroke}{rgb}{1.000000,1.000000,1.000000}%
\pgfsetstrokecolor{currentstroke}%
\pgfsetdash{}{0pt}%
\pgfpathmoveto{\pgfqpoint{15.300000in}{2.500000in}}%
\pgfpathlineto{\pgfqpoint{15.300000in}{3.377907in}}%
\pgfusepath{stroke}%
\end{pgfscope}%
\begin{pgfscope}%
\pgfsetrectcap%
\pgfsetmiterjoin%
\pgfsetlinewidth{0.803000pt}%
\definecolor{currentstroke}{rgb}{1.000000,1.000000,1.000000}%
\pgfsetstrokecolor{currentstroke}%
\pgfsetdash{}{0pt}%
\pgfpathmoveto{\pgfqpoint{9.810417in}{2.500000in}}%
\pgfpathlineto{\pgfqpoint{15.300000in}{2.500000in}}%
\pgfusepath{stroke}%
\end{pgfscope}%
\begin{pgfscope}%
\pgfsetrectcap%
\pgfsetmiterjoin%
\pgfsetlinewidth{0.803000pt}%
\definecolor{currentstroke}{rgb}{1.000000,1.000000,1.000000}%
\pgfsetstrokecolor{currentstroke}%
\pgfsetdash{}{0pt}%
\pgfpathmoveto{\pgfqpoint{9.810417in}{3.377907in}}%
\pgfpathlineto{\pgfqpoint{15.300000in}{3.377907in}}%
\pgfusepath{stroke}%
\end{pgfscope}%
\begin{pgfscope}%
\definecolor{textcolor}{rgb}{0.150000,0.150000,0.150000}%
\pgfsetstrokecolor{textcolor}%
\pgfsetfillcolor{textcolor}%
\pgftext[x=12.555208in,y=3.461240in,,base]{\color{textcolor}\rmfamily\fontsize{16.800000}{20.160000}\selectfont Partial Autocorrelation}%
\end{pgfscope}%
\end{pgfpicture}%
\makeatother%
\endgroup%

    \end{adjustbox}  
    \caption{Autocorrelation and partial autocorrelation for the log of the adjusted closing prices for all stocks}
    \label{fig:acf_pacf_log_adjclose}
\end{figure}{}






Figure2: Plot of Log closing prices.
%\input{figrues.pgf}

Also ACF and PACF plots
\begin{figure}[h]
    \centering
    \begin{adjustbox}{width=.9\textwidth,center}
    %% Creator: Matplotlib, PGF backend
%%
%% To include the figure in your LaTeX document, write
%%   \input{<filename>.pgf}
%%
%% Make sure the required packages are loaded in your preamble
%%   \usepackage{pgf}
%%
%% Figures using additional raster images can only be included by \input if
%% they are in the same directory as the main LaTeX file. For loading figures
%% from other directories you can use the `import` package
%%   \usepackage{import}
%% and then include the figures with
%%   \import{<path to file>}{<filename>.pgf}
%%
%% Matplotlib used the following preamble
%%   \usepackage{fontspec}
%%   \setmainfont{DejaVuSerif.ttf}[Path=/opt/tljh/user/lib/python3.6/site-packages/matplotlib/mpl-data/fonts/ttf/]
%%   \setsansfont{DejaVuSans.ttf}[Path=/opt/tljh/user/lib/python3.6/site-packages/matplotlib/mpl-data/fonts/ttf/]
%%   \setmonofont{DejaVuSansMono.ttf}[Path=/opt/tljh/user/lib/python3.6/site-packages/matplotlib/mpl-data/fonts/ttf/]
%%
\begingroup%
\makeatletter%
\begin{pgfpicture}%
\pgfpathrectangle{\pgfpointorigin}{\pgfqpoint{17.000000in}{10.000000in}}%
\pgfusepath{use as bounding box, clip}%
\begin{pgfscope}%
\pgfsetbuttcap%
\pgfsetmiterjoin%
\definecolor{currentfill}{rgb}{1.000000,1.000000,1.000000}%
\pgfsetfillcolor{currentfill}%
\pgfsetlinewidth{0.000000pt}%
\definecolor{currentstroke}{rgb}{1.000000,1.000000,1.000000}%
\pgfsetstrokecolor{currentstroke}%
\pgfsetdash{}{0pt}%
\pgfpathmoveto{\pgfqpoint{0.000000in}{0.000000in}}%
\pgfpathlineto{\pgfqpoint{17.000000in}{0.000000in}}%
\pgfpathlineto{\pgfqpoint{17.000000in}{10.000000in}}%
\pgfpathlineto{\pgfqpoint{0.000000in}{10.000000in}}%
\pgfpathclose%
\pgfusepath{fill}%
\end{pgfscope}%
\begin{pgfscope}%
\pgfsetbuttcap%
\pgfsetmiterjoin%
\definecolor{currentfill}{rgb}{0.917647,0.917647,0.949020}%
\pgfsetfillcolor{currentfill}%
\pgfsetlinewidth{0.000000pt}%
\definecolor{currentstroke}{rgb}{0.000000,0.000000,0.000000}%
\pgfsetstrokecolor{currentstroke}%
\pgfsetstrokeopacity{0.000000}%
\pgfsetdash{}{0pt}%
\pgfpathmoveto{\pgfqpoint{2.125000in}{7.879268in}}%
\pgfpathlineto{\pgfqpoint{7.614583in}{7.879268in}}%
\pgfpathlineto{\pgfqpoint{7.614583in}{8.800000in}}%
\pgfpathlineto{\pgfqpoint{2.125000in}{8.800000in}}%
\pgfpathclose%
\pgfusepath{fill}%
\end{pgfscope}%
\begin{pgfscope}%
\pgfpathrectangle{\pgfqpoint{2.125000in}{7.879268in}}{\pgfqpoint{5.489583in}{0.920732in}}%
\pgfusepath{clip}%
\pgfsetroundcap%
\pgfsetroundjoin%
\pgfsetlinewidth{0.803000pt}%
\definecolor{currentstroke}{rgb}{1.000000,1.000000,1.000000}%
\pgfsetstrokecolor{currentstroke}%
\pgfsetdash{}{0pt}%
\pgfpathmoveto{\pgfqpoint{2.369963in}{7.879268in}}%
\pgfpathlineto{\pgfqpoint{2.369963in}{8.800000in}}%
\pgfusepath{stroke}%
\end{pgfscope}%
\begin{pgfscope}%
\definecolor{textcolor}{rgb}{0.150000,0.150000,0.150000}%
\pgfsetstrokecolor{textcolor}%
\pgfsetfillcolor{textcolor}%
\pgftext[x=2.369963in,y=7.782046in,,top]{\color{textcolor}\rmfamily\fontsize{10.000000}{12.000000}\selectfont 2012}%
\end{pgfscope}%
\begin{pgfscope}%
\pgfpathrectangle{\pgfqpoint{2.125000in}{7.879268in}}{\pgfqpoint{5.489583in}{0.920732in}}%
\pgfusepath{clip}%
\pgfsetroundcap%
\pgfsetroundjoin%
\pgfsetlinewidth{0.803000pt}%
\definecolor{currentstroke}{rgb}{1.000000,1.000000,1.000000}%
\pgfsetstrokecolor{currentstroke}%
\pgfsetdash{}{0pt}%
\pgfpathmoveto{\pgfqpoint{3.205141in}{7.879268in}}%
\pgfpathlineto{\pgfqpoint{3.205141in}{8.800000in}}%
\pgfusepath{stroke}%
\end{pgfscope}%
\begin{pgfscope}%
\definecolor{textcolor}{rgb}{0.150000,0.150000,0.150000}%
\pgfsetstrokecolor{textcolor}%
\pgfsetfillcolor{textcolor}%
\pgftext[x=3.205141in,y=7.782046in,,top]{\color{textcolor}\rmfamily\fontsize{10.000000}{12.000000}\selectfont 2013}%
\end{pgfscope}%
\begin{pgfscope}%
\pgfpathrectangle{\pgfqpoint{2.125000in}{7.879268in}}{\pgfqpoint{5.489583in}{0.920732in}}%
\pgfusepath{clip}%
\pgfsetroundcap%
\pgfsetroundjoin%
\pgfsetlinewidth{0.803000pt}%
\definecolor{currentstroke}{rgb}{1.000000,1.000000,1.000000}%
\pgfsetstrokecolor{currentstroke}%
\pgfsetdash{}{0pt}%
\pgfpathmoveto{\pgfqpoint{4.038037in}{7.879268in}}%
\pgfpathlineto{\pgfqpoint{4.038037in}{8.800000in}}%
\pgfusepath{stroke}%
\end{pgfscope}%
\begin{pgfscope}%
\definecolor{textcolor}{rgb}{0.150000,0.150000,0.150000}%
\pgfsetstrokecolor{textcolor}%
\pgfsetfillcolor{textcolor}%
\pgftext[x=4.038037in,y=7.782046in,,top]{\color{textcolor}\rmfamily\fontsize{10.000000}{12.000000}\selectfont 2014}%
\end{pgfscope}%
\begin{pgfscope}%
\pgfpathrectangle{\pgfqpoint{2.125000in}{7.879268in}}{\pgfqpoint{5.489583in}{0.920732in}}%
\pgfusepath{clip}%
\pgfsetroundcap%
\pgfsetroundjoin%
\pgfsetlinewidth{0.803000pt}%
\definecolor{currentstroke}{rgb}{1.000000,1.000000,1.000000}%
\pgfsetstrokecolor{currentstroke}%
\pgfsetdash{}{0pt}%
\pgfpathmoveto{\pgfqpoint{4.870933in}{7.879268in}}%
\pgfpathlineto{\pgfqpoint{4.870933in}{8.800000in}}%
\pgfusepath{stroke}%
\end{pgfscope}%
\begin{pgfscope}%
\definecolor{textcolor}{rgb}{0.150000,0.150000,0.150000}%
\pgfsetstrokecolor{textcolor}%
\pgfsetfillcolor{textcolor}%
\pgftext[x=4.870933in,y=7.782046in,,top]{\color{textcolor}\rmfamily\fontsize{10.000000}{12.000000}\selectfont 2015}%
\end{pgfscope}%
\begin{pgfscope}%
\pgfpathrectangle{\pgfqpoint{2.125000in}{7.879268in}}{\pgfqpoint{5.489583in}{0.920732in}}%
\pgfusepath{clip}%
\pgfsetroundcap%
\pgfsetroundjoin%
\pgfsetlinewidth{0.803000pt}%
\definecolor{currentstroke}{rgb}{1.000000,1.000000,1.000000}%
\pgfsetstrokecolor{currentstroke}%
\pgfsetdash{}{0pt}%
\pgfpathmoveto{\pgfqpoint{5.703829in}{7.879268in}}%
\pgfpathlineto{\pgfqpoint{5.703829in}{8.800000in}}%
\pgfusepath{stroke}%
\end{pgfscope}%
\begin{pgfscope}%
\definecolor{textcolor}{rgb}{0.150000,0.150000,0.150000}%
\pgfsetstrokecolor{textcolor}%
\pgfsetfillcolor{textcolor}%
\pgftext[x=5.703829in,y=7.782046in,,top]{\color{textcolor}\rmfamily\fontsize{10.000000}{12.000000}\selectfont 2016}%
\end{pgfscope}%
\begin{pgfscope}%
\pgfpathrectangle{\pgfqpoint{2.125000in}{7.879268in}}{\pgfqpoint{5.489583in}{0.920732in}}%
\pgfusepath{clip}%
\pgfsetroundcap%
\pgfsetroundjoin%
\pgfsetlinewidth{0.803000pt}%
\definecolor{currentstroke}{rgb}{1.000000,1.000000,1.000000}%
\pgfsetstrokecolor{currentstroke}%
\pgfsetdash{}{0pt}%
\pgfpathmoveto{\pgfqpoint{6.539007in}{7.879268in}}%
\pgfpathlineto{\pgfqpoint{6.539007in}{8.800000in}}%
\pgfusepath{stroke}%
\end{pgfscope}%
\begin{pgfscope}%
\definecolor{textcolor}{rgb}{0.150000,0.150000,0.150000}%
\pgfsetstrokecolor{textcolor}%
\pgfsetfillcolor{textcolor}%
\pgftext[x=6.539007in,y=7.782046in,,top]{\color{textcolor}\rmfamily\fontsize{10.000000}{12.000000}\selectfont 2017}%
\end{pgfscope}%
\begin{pgfscope}%
\pgfpathrectangle{\pgfqpoint{2.125000in}{7.879268in}}{\pgfqpoint{5.489583in}{0.920732in}}%
\pgfusepath{clip}%
\pgfsetroundcap%
\pgfsetroundjoin%
\pgfsetlinewidth{0.803000pt}%
\definecolor{currentstroke}{rgb}{1.000000,1.000000,1.000000}%
\pgfsetstrokecolor{currentstroke}%
\pgfsetdash{}{0pt}%
\pgfpathmoveto{\pgfqpoint{7.371903in}{7.879268in}}%
\pgfpathlineto{\pgfqpoint{7.371903in}{8.800000in}}%
\pgfusepath{stroke}%
\end{pgfscope}%
\begin{pgfscope}%
\definecolor{textcolor}{rgb}{0.150000,0.150000,0.150000}%
\pgfsetstrokecolor{textcolor}%
\pgfsetfillcolor{textcolor}%
\pgftext[x=7.371903in,y=7.782046in,,top]{\color{textcolor}\rmfamily\fontsize{10.000000}{12.000000}\selectfont 2018}%
\end{pgfscope}%
\begin{pgfscope}%
\pgfpathrectangle{\pgfqpoint{2.125000in}{7.879268in}}{\pgfqpoint{5.489583in}{0.920732in}}%
\pgfusepath{clip}%
\pgfsetroundcap%
\pgfsetroundjoin%
\pgfsetlinewidth{0.803000pt}%
\definecolor{currentstroke}{rgb}{1.000000,1.000000,1.000000}%
\pgfsetstrokecolor{currentstroke}%
\pgfsetdash{}{0pt}%
\pgfpathmoveto{\pgfqpoint{2.125000in}{8.079365in}}%
\pgfpathlineto{\pgfqpoint{7.614583in}{8.079365in}}%
\pgfusepath{stroke}%
\end{pgfscope}%
\begin{pgfscope}%
\definecolor{textcolor}{rgb}{0.150000,0.150000,0.150000}%
\pgfsetstrokecolor{textcolor}%
\pgfsetfillcolor{textcolor}%
\pgftext[x=1.762682in,y=8.026603in,left,base]{\color{textcolor}\rmfamily\fontsize{10.000000}{12.000000}\selectfont 100}%
\end{pgfscope}%
\begin{pgfscope}%
\pgfpathrectangle{\pgfqpoint{2.125000in}{7.879268in}}{\pgfqpoint{5.489583in}{0.920732in}}%
\pgfusepath{clip}%
\pgfsetroundcap%
\pgfsetroundjoin%
\pgfsetlinewidth{0.803000pt}%
\definecolor{currentstroke}{rgb}{1.000000,1.000000,1.000000}%
\pgfsetstrokecolor{currentstroke}%
\pgfsetdash{}{0pt}%
\pgfpathmoveto{\pgfqpoint{2.125000in}{8.588848in}}%
\pgfpathlineto{\pgfqpoint{7.614583in}{8.588848in}}%
\pgfusepath{stroke}%
\end{pgfscope}%
\begin{pgfscope}%
\definecolor{textcolor}{rgb}{0.150000,0.150000,0.150000}%
\pgfsetstrokecolor{textcolor}%
\pgfsetfillcolor{textcolor}%
\pgftext[x=1.762682in,y=8.536086in,left,base]{\color{textcolor}\rmfamily\fontsize{10.000000}{12.000000}\selectfont 200}%
\end{pgfscope}%
\begin{pgfscope}%
\pgfpathrectangle{\pgfqpoint{2.125000in}{7.879268in}}{\pgfqpoint{5.489583in}{0.920732in}}%
\pgfusepath{clip}%
\pgfsetroundcap%
\pgfsetroundjoin%
\pgfsetlinewidth{1.505625pt}%
\definecolor{currentstroke}{rgb}{0.121569,0.466667,0.705882}%
\pgfsetstrokecolor{currentstroke}%
\pgfsetdash{}{0pt}%
\pgfpathmoveto{\pgfqpoint{2.374527in}{7.921629in}}%
\pgfpathlineto{\pgfqpoint{2.376808in}{7.924533in}}%
\pgfpathlineto{\pgfqpoint{2.381372in}{7.921120in}}%
\pgfpathlineto{\pgfqpoint{2.388218in}{7.923209in}}%
\pgfpathlineto{\pgfqpoint{2.390500in}{7.925043in}}%
\pgfpathlineto{\pgfqpoint{2.392782in}{7.922801in}}%
\pgfpathlineto{\pgfqpoint{2.395064in}{7.924941in}}%
\pgfpathlineto{\pgfqpoint{2.397346in}{7.922088in}}%
\pgfpathlineto{\pgfqpoint{2.406473in}{7.924737in}}%
\pgfpathlineto{\pgfqpoint{2.411037in}{7.931360in}}%
\pgfpathlineto{\pgfqpoint{2.413319in}{7.930749in}}%
\pgfpathlineto{\pgfqpoint{2.420165in}{7.930545in}}%
\pgfpathlineto{\pgfqpoint{2.422447in}{7.931921in}}%
\pgfpathlineto{\pgfqpoint{2.424728in}{7.934213in}}%
\pgfpathlineto{\pgfqpoint{2.427010in}{7.938850in}}%
\pgfpathlineto{\pgfqpoint{2.429292in}{7.938340in}}%
\pgfpathlineto{\pgfqpoint{2.436138in}{7.937831in}}%
\pgfpathlineto{\pgfqpoint{2.438420in}{7.935181in}}%
\pgfpathlineto{\pgfqpoint{2.440702in}{7.937882in}}%
\pgfpathlineto{\pgfqpoint{2.442984in}{7.938238in}}%
\pgfpathlineto{\pgfqpoint{2.445266in}{7.939512in}}%
\pgfpathlineto{\pgfqpoint{2.452111in}{7.938799in}}%
\pgfpathlineto{\pgfqpoint{2.454393in}{7.940174in}}%
\pgfpathlineto{\pgfqpoint{2.458957in}{7.940735in}}%
\pgfpathlineto{\pgfqpoint{2.461239in}{7.937016in}}%
\pgfpathlineto{\pgfqpoint{2.468085in}{7.940735in}}%
\pgfpathlineto{\pgfqpoint{2.470367in}{7.940582in}}%
\pgfpathlineto{\pgfqpoint{2.472649in}{7.938952in}}%
\pgfpathlineto{\pgfqpoint{2.474930in}{7.941652in}}%
\pgfpathlineto{\pgfqpoint{2.477212in}{7.941244in}}%
\pgfpathlineto{\pgfqpoint{2.486340in}{7.941448in}}%
\pgfpathlineto{\pgfqpoint{2.493186in}{7.943995in}}%
\pgfpathlineto{\pgfqpoint{2.500031in}{7.943435in}}%
\pgfpathlineto{\pgfqpoint{2.504595in}{7.941448in}}%
\pgfpathlineto{\pgfqpoint{2.516005in}{7.939155in}}%
\pgfpathlineto{\pgfqpoint{2.518287in}{7.930189in}}%
\pgfpathlineto{\pgfqpoint{2.520569in}{7.932379in}}%
\pgfpathlineto{\pgfqpoint{2.522850in}{7.937678in}}%
\pgfpathlineto{\pgfqpoint{2.525132in}{7.938035in}}%
\pgfpathlineto{\pgfqpoint{2.531978in}{7.941193in}}%
\pgfpathlineto{\pgfqpoint{2.534260in}{7.946339in}}%
\pgfpathlineto{\pgfqpoint{2.536542in}{7.946849in}}%
\pgfpathlineto{\pgfqpoint{2.538824in}{7.951638in}}%
\pgfpathlineto{\pgfqpoint{2.541106in}{7.949753in}}%
\pgfpathlineto{\pgfqpoint{2.547951in}{7.950517in}}%
\pgfpathlineto{\pgfqpoint{2.554797in}{7.945575in}}%
\pgfpathlineto{\pgfqpoint{2.557079in}{7.945065in}}%
\pgfpathlineto{\pgfqpoint{2.563925in}{7.947868in}}%
\pgfpathlineto{\pgfqpoint{2.566207in}{7.947969in}}%
\pgfpathlineto{\pgfqpoint{2.568489in}{7.945065in}}%
\pgfpathlineto{\pgfqpoint{2.573052in}{7.948275in}}%
\pgfpathlineto{\pgfqpoint{2.579898in}{7.948377in}}%
\pgfpathlineto{\pgfqpoint{2.582180in}{7.946492in}}%
\pgfpathlineto{\pgfqpoint{2.584462in}{7.942467in}}%
\pgfpathlineto{\pgfqpoint{2.586744in}{7.940022in}}%
\pgfpathlineto{\pgfqpoint{2.595871in}{7.936048in}}%
\pgfpathlineto{\pgfqpoint{2.598153in}{7.928609in}}%
\pgfpathlineto{\pgfqpoint{2.600435in}{7.931870in}}%
\pgfpathlineto{\pgfqpoint{2.602717in}{7.938289in}}%
\pgfpathlineto{\pgfqpoint{2.604999in}{7.933347in}}%
\pgfpathlineto{\pgfqpoint{2.611845in}{7.936353in}}%
\pgfpathlineto{\pgfqpoint{2.614127in}{7.940786in}}%
\pgfpathlineto{\pgfqpoint{2.618691in}{7.938035in}}%
\pgfpathlineto{\pgfqpoint{2.620972in}{7.940939in}}%
\pgfpathlineto{\pgfqpoint{2.627818in}{7.939461in}}%
\pgfpathlineto{\pgfqpoint{2.630100in}{7.945218in}}%
\pgfpathlineto{\pgfqpoint{2.632382in}{7.946543in}}%
\pgfpathlineto{\pgfqpoint{2.634664in}{7.948734in}}%
\pgfpathlineto{\pgfqpoint{2.648355in}{7.949600in}}%
\pgfpathlineto{\pgfqpoint{2.650637in}{7.949039in}}%
\pgfpathlineto{\pgfqpoint{2.652919in}{7.945982in}}%
\pgfpathlineto{\pgfqpoint{2.659765in}{7.943180in}}%
\pgfpathlineto{\pgfqpoint{2.664329in}{7.940022in}}%
\pgfpathlineto{\pgfqpoint{2.666611in}{7.939461in}}%
\pgfpathlineto{\pgfqpoint{2.668893in}{7.937729in}}%
\pgfpathlineto{\pgfqpoint{2.675738in}{7.933908in}}%
\pgfpathlineto{\pgfqpoint{2.678020in}{7.933704in}}%
\pgfpathlineto{\pgfqpoint{2.680302in}{7.934417in}}%
\pgfpathlineto{\pgfqpoint{2.684866in}{7.926520in}}%
\pgfpathlineto{\pgfqpoint{2.691712in}{7.930596in}}%
\pgfpathlineto{\pgfqpoint{2.693993in}{7.929170in}}%
\pgfpathlineto{\pgfqpoint{2.696275in}{7.932074in}}%
\pgfpathlineto{\pgfqpoint{2.698557in}{7.932838in}}%
\pgfpathlineto{\pgfqpoint{2.700839in}{7.931972in}}%
\pgfpathlineto{\pgfqpoint{2.709967in}{7.936048in}}%
\pgfpathlineto{\pgfqpoint{2.712249in}{7.930545in}}%
\pgfpathlineto{\pgfqpoint{2.714531in}{7.930392in}}%
\pgfpathlineto{\pgfqpoint{2.716813in}{7.923718in}}%
\pgfpathlineto{\pgfqpoint{2.725940in}{7.922292in}}%
\pgfpathlineto{\pgfqpoint{2.728222in}{7.931309in}}%
\pgfpathlineto{\pgfqpoint{2.730504in}{7.935029in}}%
\pgfpathlineto{\pgfqpoint{2.732786in}{7.937168in}}%
\pgfpathlineto{\pgfqpoint{2.739632in}{7.933653in}}%
\pgfpathlineto{\pgfqpoint{2.741914in}{7.940276in}}%
\pgfpathlineto{\pgfqpoint{2.744195in}{7.937729in}}%
\pgfpathlineto{\pgfqpoint{2.748759in}{7.943333in}}%
\pgfpathlineto{\pgfqpoint{2.755605in}{7.942773in}}%
\pgfpathlineto{\pgfqpoint{2.757887in}{7.944964in}}%
\pgfpathlineto{\pgfqpoint{2.760169in}{7.943741in}}%
\pgfpathlineto{\pgfqpoint{2.762451in}{7.940276in}}%
\pgfpathlineto{\pgfqpoint{2.764733in}{7.940735in}}%
\pgfpathlineto{\pgfqpoint{2.771578in}{7.936557in}}%
\pgfpathlineto{\pgfqpoint{2.773860in}{7.937933in}}%
\pgfpathlineto{\pgfqpoint{2.776142in}{7.942110in}}%
\pgfpathlineto{\pgfqpoint{2.778424in}{7.942110in}}%
\pgfpathlineto{\pgfqpoint{2.780706in}{7.952555in}}%
\pgfpathlineto{\pgfqpoint{2.787552in}{7.951179in}}%
\pgfpathlineto{\pgfqpoint{2.789834in}{7.952962in}}%
\pgfpathlineto{\pgfqpoint{2.794397in}{7.952402in}}%
\pgfpathlineto{\pgfqpoint{2.796679in}{7.949956in}}%
\pgfpathlineto{\pgfqpoint{2.803525in}{7.949804in}}%
\pgfpathlineto{\pgfqpoint{2.805807in}{7.946798in}}%
\pgfpathlineto{\pgfqpoint{2.808089in}{7.945065in}}%
\pgfpathlineto{\pgfqpoint{2.810371in}{7.938952in}}%
\pgfpathlineto{\pgfqpoint{2.812653in}{7.943945in}}%
\pgfpathlineto{\pgfqpoint{2.819498in}{7.946135in}}%
\pgfpathlineto{\pgfqpoint{2.821780in}{7.949905in}}%
\pgfpathlineto{\pgfqpoint{2.824062in}{7.957955in}}%
\pgfpathlineto{\pgfqpoint{2.826344in}{7.957802in}}%
\pgfpathlineto{\pgfqpoint{2.828626in}{7.954236in}}%
\pgfpathlineto{\pgfqpoint{2.835472in}{7.951536in}}%
\pgfpathlineto{\pgfqpoint{2.837754in}{7.946747in}}%
\pgfpathlineto{\pgfqpoint{2.840036in}{7.948937in}}%
\pgfpathlineto{\pgfqpoint{2.842317in}{7.956784in}}%
\pgfpathlineto{\pgfqpoint{2.844599in}{7.961573in}}%
\pgfpathlineto{\pgfqpoint{2.853727in}{7.959535in}}%
\pgfpathlineto{\pgfqpoint{2.856009in}{7.959382in}}%
\pgfpathlineto{\pgfqpoint{2.858291in}{7.954593in}}%
\pgfpathlineto{\pgfqpoint{2.860573in}{7.961522in}}%
\pgfpathlineto{\pgfqpoint{2.867418in}{7.960299in}}%
\pgfpathlineto{\pgfqpoint{2.869700in}{7.961522in}}%
\pgfpathlineto{\pgfqpoint{2.874264in}{7.961063in}}%
\pgfpathlineto{\pgfqpoint{2.876546in}{7.964018in}}%
\pgfpathlineto{\pgfqpoint{2.883392in}{7.964528in}}%
\pgfpathlineto{\pgfqpoint{2.885674in}{7.964069in}}%
\pgfpathlineto{\pgfqpoint{2.887956in}{7.965088in}}%
\pgfpathlineto{\pgfqpoint{2.890237in}{7.970234in}}%
\pgfpathlineto{\pgfqpoint{2.892519in}{7.972374in}}%
\pgfpathlineto{\pgfqpoint{2.899365in}{7.970896in}}%
\pgfpathlineto{\pgfqpoint{2.901647in}{7.967992in}}%
\pgfpathlineto{\pgfqpoint{2.903929in}{7.968247in}}%
\pgfpathlineto{\pgfqpoint{2.906211in}{7.965241in}}%
\pgfpathlineto{\pgfqpoint{2.908493in}{7.968858in}}%
\pgfpathlineto{\pgfqpoint{2.915338in}{7.967839in}}%
\pgfpathlineto{\pgfqpoint{2.917620in}{7.966617in}}%
\pgfpathlineto{\pgfqpoint{2.919902in}{7.967177in}}%
\pgfpathlineto{\pgfqpoint{2.922184in}{7.964273in}}%
\pgfpathlineto{\pgfqpoint{2.924466in}{7.967890in}}%
\pgfpathlineto{\pgfqpoint{2.933594in}{7.963916in}}%
\pgfpathlineto{\pgfqpoint{2.935876in}{7.964222in}}%
\pgfpathlineto{\pgfqpoint{2.938158in}{7.970794in}}%
\pgfpathlineto{\pgfqpoint{2.940439in}{7.968807in}}%
\pgfpathlineto{\pgfqpoint{2.947285in}{7.959586in}}%
\pgfpathlineto{\pgfqpoint{2.949567in}{7.961725in}}%
\pgfpathlineto{\pgfqpoint{2.951849in}{7.960197in}}%
\pgfpathlineto{\pgfqpoint{2.954131in}{7.965547in}}%
\pgfpathlineto{\pgfqpoint{2.956413in}{7.973800in}}%
\pgfpathlineto{\pgfqpoint{2.963258in}{7.972934in}}%
\pgfpathlineto{\pgfqpoint{2.965540in}{7.971457in}}%
\pgfpathlineto{\pgfqpoint{2.967822in}{7.972323in}}%
\pgfpathlineto{\pgfqpoint{2.970104in}{7.972119in}}%
\pgfpathlineto{\pgfqpoint{2.972386in}{7.970489in}}%
\pgfpathlineto{\pgfqpoint{2.979232in}{7.972730in}}%
\pgfpathlineto{\pgfqpoint{2.981514in}{7.968858in}}%
\pgfpathlineto{\pgfqpoint{2.983796in}{7.967839in}}%
\pgfpathlineto{\pgfqpoint{2.986078in}{7.968807in}}%
\pgfpathlineto{\pgfqpoint{2.988359in}{7.967126in}}%
\pgfpathlineto{\pgfqpoint{2.999769in}{7.972934in}}%
\pgfpathlineto{\pgfqpoint{3.004333in}{7.978029in}}%
\pgfpathlineto{\pgfqpoint{3.011179in}{7.979812in}}%
\pgfpathlineto{\pgfqpoint{3.013460in}{7.973647in}}%
\pgfpathlineto{\pgfqpoint{3.015742in}{7.970794in}}%
\pgfpathlineto{\pgfqpoint{3.018024in}{7.968858in}}%
\pgfpathlineto{\pgfqpoint{3.020306in}{7.968553in}}%
\pgfpathlineto{\pgfqpoint{3.027152in}{7.968705in}}%
\pgfpathlineto{\pgfqpoint{3.029434in}{7.974870in}}%
\pgfpathlineto{\pgfqpoint{3.031716in}{7.977367in}}%
\pgfpathlineto{\pgfqpoint{3.033998in}{7.977061in}}%
\pgfpathlineto{\pgfqpoint{3.036280in}{7.969368in}}%
\pgfpathlineto{\pgfqpoint{3.043125in}{7.967585in}}%
\pgfpathlineto{\pgfqpoint{3.045407in}{7.951230in}}%
\pgfpathlineto{\pgfqpoint{3.047689in}{7.949804in}}%
\pgfpathlineto{\pgfqpoint{3.049971in}{7.947307in}}%
\pgfpathlineto{\pgfqpoint{3.052253in}{7.948224in}}%
\pgfpathlineto{\pgfqpoint{3.063662in}{7.946390in}}%
\pgfpathlineto{\pgfqpoint{3.065944in}{7.953472in}}%
\pgfpathlineto{\pgfqpoint{3.068226in}{7.952300in}}%
\pgfpathlineto{\pgfqpoint{3.075072in}{7.955000in}}%
\pgfpathlineto{\pgfqpoint{3.077354in}{7.960401in}}%
\pgfpathlineto{\pgfqpoint{3.079636in}{7.954032in}}%
\pgfpathlineto{\pgfqpoint{3.081918in}{7.950466in}}%
\pgfpathlineto{\pgfqpoint{3.084200in}{7.951587in}}%
\pgfpathlineto{\pgfqpoint{3.091045in}{7.953166in}}%
\pgfpathlineto{\pgfqpoint{3.093327in}{7.952759in}}%
\pgfpathlineto{\pgfqpoint{3.095609in}{7.945167in}}%
\pgfpathlineto{\pgfqpoint{3.097891in}{7.948224in}}%
\pgfpathlineto{\pgfqpoint{3.100173in}{7.950262in}}%
\pgfpathlineto{\pgfqpoint{3.107019in}{7.954847in}}%
\pgfpathlineto{\pgfqpoint{3.109301in}{7.955051in}}%
\pgfpathlineto{\pgfqpoint{3.111582in}{7.954695in}}%
\pgfpathlineto{\pgfqpoint{3.116146in}{7.960503in}}%
\pgfpathlineto{\pgfqpoint{3.122992in}{7.959688in}}%
\pgfpathlineto{\pgfqpoint{3.125274in}{7.960605in}}%
\pgfpathlineto{\pgfqpoint{3.127556in}{7.963305in}}%
\pgfpathlineto{\pgfqpoint{3.129838in}{7.962082in}}%
\pgfpathlineto{\pgfqpoint{3.132120in}{7.963407in}}%
\pgfpathlineto{\pgfqpoint{3.141247in}{7.959789in}}%
\pgfpathlineto{\pgfqpoint{3.143529in}{7.962541in}}%
\pgfpathlineto{\pgfqpoint{3.145811in}{7.963407in}}%
\pgfpathlineto{\pgfqpoint{3.148093in}{7.965801in}}%
\pgfpathlineto{\pgfqpoint{3.154939in}{7.967381in}}%
\pgfpathlineto{\pgfqpoint{3.157221in}{7.975176in}}%
\pgfpathlineto{\pgfqpoint{3.161784in}{7.970081in}}%
\pgfpathlineto{\pgfqpoint{3.164066in}{7.969164in}}%
\pgfpathlineto{\pgfqpoint{3.170912in}{7.972374in}}%
\pgfpathlineto{\pgfqpoint{3.173194in}{7.975940in}}%
\pgfpathlineto{\pgfqpoint{3.175476in}{7.972323in}}%
\pgfpathlineto{\pgfqpoint{3.177758in}{7.977163in}}%
\pgfpathlineto{\pgfqpoint{3.180040in}{7.972679in}}%
\pgfpathlineto{\pgfqpoint{3.186885in}{7.973087in}}%
\pgfpathlineto{\pgfqpoint{3.191449in}{7.972577in}}%
\pgfpathlineto{\pgfqpoint{3.193731in}{7.970692in}}%
\pgfpathlineto{\pgfqpoint{3.196013in}{7.966973in}}%
\pgfpathlineto{\pgfqpoint{3.202859in}{7.971609in}}%
\pgfpathlineto{\pgfqpoint{3.207423in}{7.979965in}}%
\pgfpathlineto{\pgfqpoint{3.209704in}{7.979455in}}%
\pgfpathlineto{\pgfqpoint{3.211986in}{7.982512in}}%
\pgfpathlineto{\pgfqpoint{3.221114in}{7.983073in}}%
\pgfpathlineto{\pgfqpoint{3.223396in}{7.986996in}}%
\pgfpathlineto{\pgfqpoint{3.225678in}{7.989085in}}%
\pgfpathlineto{\pgfqpoint{3.227960in}{7.986435in}}%
\pgfpathlineto{\pgfqpoint{3.239369in}{7.992142in}}%
\pgfpathlineto{\pgfqpoint{3.241651in}{7.994230in}}%
\pgfpathlineto{\pgfqpoint{3.243933in}{7.997084in}}%
\pgfpathlineto{\pgfqpoint{3.257624in}{8.001108in}}%
\pgfpathlineto{\pgfqpoint{3.259906in}{8.005082in}}%
\pgfpathlineto{\pgfqpoint{3.266752in}{8.005337in}}%
\pgfpathlineto{\pgfqpoint{3.269034in}{8.010381in}}%
\pgfpathlineto{\pgfqpoint{3.271316in}{8.006000in}}%
\pgfpathlineto{\pgfqpoint{3.273598in}{8.004930in}}%
\pgfpathlineto{\pgfqpoint{3.275880in}{8.009311in}}%
\pgfpathlineto{\pgfqpoint{3.282725in}{8.005898in}}%
\pgfpathlineto{\pgfqpoint{3.285007in}{8.009005in}}%
\pgfpathlineto{\pgfqpoint{3.287289in}{8.014202in}}%
\pgfpathlineto{\pgfqpoint{3.289571in}{8.012164in}}%
\pgfpathlineto{\pgfqpoint{3.291853in}{8.014049in}}%
\pgfpathlineto{\pgfqpoint{3.298699in}{8.013897in}}%
\pgfpathlineto{\pgfqpoint{3.300981in}{8.017514in}}%
\pgfpathlineto{\pgfqpoint{3.305545in}{8.017310in}}%
\pgfpathlineto{\pgfqpoint{3.307826in}{8.019297in}}%
\pgfpathlineto{\pgfqpoint{3.316954in}{8.023424in}}%
\pgfpathlineto{\pgfqpoint{3.319236in}{8.018940in}}%
\pgfpathlineto{\pgfqpoint{3.321518in}{8.017055in}}%
\pgfpathlineto{\pgfqpoint{3.323800in}{8.020622in}}%
\pgfpathlineto{\pgfqpoint{3.330645in}{8.012827in}}%
\pgfpathlineto{\pgfqpoint{3.332927in}{8.015272in}}%
\pgfpathlineto{\pgfqpoint{3.335209in}{8.020775in}}%
\pgfpathlineto{\pgfqpoint{3.337491in}{8.022609in}}%
\pgfpathlineto{\pgfqpoint{3.339773in}{8.021641in}}%
\pgfpathlineto{\pgfqpoint{3.346619in}{8.019501in}}%
\pgfpathlineto{\pgfqpoint{3.348901in}{8.024596in}}%
\pgfpathlineto{\pgfqpoint{3.351183in}{8.025513in}}%
\pgfpathlineto{\pgfqpoint{3.353465in}{8.024952in}}%
\pgfpathlineto{\pgfqpoint{3.355746in}{8.030047in}}%
\pgfpathlineto{\pgfqpoint{3.362592in}{8.030506in}}%
\pgfpathlineto{\pgfqpoint{3.364874in}{8.027551in}}%
\pgfpathlineto{\pgfqpoint{3.367156in}{8.027347in}}%
\pgfpathlineto{\pgfqpoint{3.369438in}{8.031423in}}%
\pgfpathlineto{\pgfqpoint{3.371720in}{8.033053in}}%
\pgfpathlineto{\pgfqpoint{3.380847in}{8.027754in}}%
\pgfpathlineto{\pgfqpoint{3.383129in}{8.029843in}}%
\pgfpathlineto{\pgfqpoint{3.385411in}{8.026735in}}%
\pgfpathlineto{\pgfqpoint{3.387693in}{8.033155in}}%
\pgfpathlineto{\pgfqpoint{3.394539in}{8.027703in}}%
\pgfpathlineto{\pgfqpoint{3.396821in}{8.031626in}}%
\pgfpathlineto{\pgfqpoint{3.399103in}{8.028264in}}%
\pgfpathlineto{\pgfqpoint{3.401385in}{8.032696in}}%
\pgfpathlineto{\pgfqpoint{3.410512in}{8.029792in}}%
\pgfpathlineto{\pgfqpoint{3.412794in}{8.033613in}}%
\pgfpathlineto{\pgfqpoint{3.415076in}{8.029945in}}%
\pgfpathlineto{\pgfqpoint{3.417358in}{8.030557in}}%
\pgfpathlineto{\pgfqpoint{3.426486in}{8.030251in}}%
\pgfpathlineto{\pgfqpoint{3.428767in}{8.031117in}}%
\pgfpathlineto{\pgfqpoint{3.431049in}{8.038708in}}%
\pgfpathlineto{\pgfqpoint{3.433331in}{8.041103in}}%
\pgfpathlineto{\pgfqpoint{3.435613in}{8.038657in}}%
\pgfpathlineto{\pgfqpoint{3.442459in}{8.030302in}}%
\pgfpathlineto{\pgfqpoint{3.444741in}{8.031983in}}%
\pgfpathlineto{\pgfqpoint{3.449305in}{8.026888in}}%
\pgfpathlineto{\pgfqpoint{3.451587in}{8.030047in}}%
\pgfpathlineto{\pgfqpoint{3.458432in}{8.030455in}}%
\pgfpathlineto{\pgfqpoint{3.460714in}{8.037384in}}%
\pgfpathlineto{\pgfqpoint{3.462996in}{8.039473in}}%
\pgfpathlineto{\pgfqpoint{3.465278in}{8.026481in}}%
\pgfpathlineto{\pgfqpoint{3.467560in}{8.021692in}}%
\pgfpathlineto{\pgfqpoint{3.474406in}{8.021895in}}%
\pgfpathlineto{\pgfqpoint{3.476688in}{8.025716in}}%
\pgfpathlineto{\pgfqpoint{3.478969in}{8.025003in}}%
\pgfpathlineto{\pgfqpoint{3.483533in}{8.039320in}}%
\pgfpathlineto{\pgfqpoint{3.490379in}{8.039320in}}%
\pgfpathlineto{\pgfqpoint{3.494943in}{8.040899in}}%
\pgfpathlineto{\pgfqpoint{3.497225in}{8.048388in}}%
\pgfpathlineto{\pgfqpoint{3.499507in}{8.050834in}}%
\pgfpathlineto{\pgfqpoint{3.508634in}{8.051293in}}%
\pgfpathlineto{\pgfqpoint{3.510916in}{8.055266in}}%
\pgfpathlineto{\pgfqpoint{3.513198in}{8.053381in}}%
\pgfpathlineto{\pgfqpoint{3.515480in}{8.054808in}}%
\pgfpathlineto{\pgfqpoint{3.522326in}{8.056234in}}%
\pgfpathlineto{\pgfqpoint{3.524608in}{8.057508in}}%
\pgfpathlineto{\pgfqpoint{3.526889in}{8.054910in}}%
\pgfpathlineto{\pgfqpoint{3.529171in}{8.053279in}}%
\pgfpathlineto{\pgfqpoint{3.531453in}{8.052668in}}%
\pgfpathlineto{\pgfqpoint{3.540581in}{8.058425in}}%
\pgfpathlineto{\pgfqpoint{3.542863in}{8.056438in}}%
\pgfpathlineto{\pgfqpoint{3.545145in}{8.057712in}}%
\pgfpathlineto{\pgfqpoint{3.547427in}{8.052668in}}%
\pgfpathlineto{\pgfqpoint{3.554272in}{8.054146in}}%
\pgfpathlineto{\pgfqpoint{3.556554in}{8.051547in}}%
\pgfpathlineto{\pgfqpoint{3.558836in}{8.045281in}}%
\pgfpathlineto{\pgfqpoint{3.561118in}{8.045637in}}%
\pgfpathlineto{\pgfqpoint{3.563400in}{8.056336in}}%
\pgfpathlineto{\pgfqpoint{3.570246in}{8.055012in}}%
\pgfpathlineto{\pgfqpoint{3.572528in}{8.052464in}}%
\pgfpathlineto{\pgfqpoint{3.574810in}{8.047217in}}%
\pgfpathlineto{\pgfqpoint{3.577091in}{8.056744in}}%
\pgfpathlineto{\pgfqpoint{3.579373in}{8.055980in}}%
\pgfpathlineto{\pgfqpoint{3.586219in}{8.059852in}}%
\pgfpathlineto{\pgfqpoint{3.588501in}{8.064488in}}%
\pgfpathlineto{\pgfqpoint{3.590783in}{8.058374in}}%
\pgfpathlineto{\pgfqpoint{3.593065in}{8.046147in}}%
\pgfpathlineto{\pgfqpoint{3.595347in}{8.049662in}}%
\pgfpathlineto{\pgfqpoint{3.602192in}{8.040542in}}%
\pgfpathlineto{\pgfqpoint{3.604474in}{8.043752in}}%
\pgfpathlineto{\pgfqpoint{3.606756in}{8.050070in}}%
\pgfpathlineto{\pgfqpoint{3.609038in}{8.052515in}}%
\pgfpathlineto{\pgfqpoint{3.611320in}{8.048643in}}%
\pgfpathlineto{\pgfqpoint{3.618166in}{8.048439in}}%
\pgfpathlineto{\pgfqpoint{3.620448in}{8.045892in}}%
\pgfpathlineto{\pgfqpoint{3.622730in}{8.049051in}}%
\pgfpathlineto{\pgfqpoint{3.627293in}{8.058221in}}%
\pgfpathlineto{\pgfqpoint{3.634139in}{8.060820in}}%
\pgfpathlineto{\pgfqpoint{3.636421in}{8.066118in}}%
\pgfpathlineto{\pgfqpoint{3.638703in}{8.066475in}}%
\pgfpathlineto{\pgfqpoint{3.640985in}{8.071315in}}%
\pgfpathlineto{\pgfqpoint{3.643267in}{8.073964in}}%
\pgfpathlineto{\pgfqpoint{3.650112in}{8.072691in}}%
\pgfpathlineto{\pgfqpoint{3.652394in}{8.070500in}}%
\pgfpathlineto{\pgfqpoint{3.654676in}{8.071774in}}%
\pgfpathlineto{\pgfqpoint{3.659240in}{8.078601in}}%
\pgfpathlineto{\pgfqpoint{3.666086in}{8.078957in}}%
\pgfpathlineto{\pgfqpoint{3.668368in}{8.081046in}}%
\pgfpathlineto{\pgfqpoint{3.670650in}{8.079212in}}%
\pgfpathlineto{\pgfqpoint{3.672932in}{8.080129in}}%
\pgfpathlineto{\pgfqpoint{3.675213in}{8.081709in}}%
\pgfpathlineto{\pgfqpoint{3.682059in}{8.080435in}}%
\pgfpathlineto{\pgfqpoint{3.684341in}{8.081403in}}%
\pgfpathlineto{\pgfqpoint{3.686623in}{8.084001in}}%
\pgfpathlineto{\pgfqpoint{3.688905in}{8.088281in}}%
\pgfpathlineto{\pgfqpoint{3.698032in}{8.086141in}}%
\pgfpathlineto{\pgfqpoint{3.700314in}{8.084307in}}%
\pgfpathlineto{\pgfqpoint{3.702596in}{8.085734in}}%
\pgfpathlineto{\pgfqpoint{3.704878in}{8.089453in}}%
\pgfpathlineto{\pgfqpoint{3.707160in}{8.087873in}}%
\pgfpathlineto{\pgfqpoint{3.714006in}{8.088179in}}%
\pgfpathlineto{\pgfqpoint{3.716288in}{8.089453in}}%
\pgfpathlineto{\pgfqpoint{3.718570in}{8.084816in}}%
\pgfpathlineto{\pgfqpoint{3.720852in}{8.077072in}}%
\pgfpathlineto{\pgfqpoint{3.723133in}{8.077327in}}%
\pgfpathlineto{\pgfqpoint{3.732261in}{8.075187in}}%
\pgfpathlineto{\pgfqpoint{3.734543in}{8.069175in}}%
\pgfpathlineto{\pgfqpoint{3.736825in}{8.074678in}}%
\pgfpathlineto{\pgfqpoint{3.739107in}{8.073506in}}%
\pgfpathlineto{\pgfqpoint{3.745953in}{8.073200in}}%
\pgfpathlineto{\pgfqpoint{3.748234in}{8.066169in}}%
\pgfpathlineto{\pgfqpoint{3.755080in}{8.069889in}}%
\pgfpathlineto{\pgfqpoint{3.764208in}{8.068309in}}%
\pgfpathlineto{\pgfqpoint{3.766490in}{8.074168in}}%
\pgfpathlineto{\pgfqpoint{3.768772in}{8.075595in}}%
\pgfpathlineto{\pgfqpoint{3.771054in}{8.076308in}}%
\pgfpathlineto{\pgfqpoint{3.777899in}{8.083798in}}%
\pgfpathlineto{\pgfqpoint{3.780181in}{8.087568in}}%
\pgfpathlineto{\pgfqpoint{3.782463in}{8.092510in}}%
\pgfpathlineto{\pgfqpoint{3.784745in}{8.090217in}}%
\pgfpathlineto{\pgfqpoint{3.787027in}{8.092000in}}%
\pgfpathlineto{\pgfqpoint{3.793873in}{8.094751in}}%
\pgfpathlineto{\pgfqpoint{3.796154in}{8.097859in}}%
\pgfpathlineto{\pgfqpoint{3.798436in}{8.103820in}}%
\pgfpathlineto{\pgfqpoint{3.800718in}{8.105094in}}%
\pgfpathlineto{\pgfqpoint{3.803000in}{8.098216in}}%
\pgfpathlineto{\pgfqpoint{3.809846in}{8.103056in}}%
\pgfpathlineto{\pgfqpoint{3.812128in}{8.101680in}}%
\pgfpathlineto{\pgfqpoint{3.814410in}{8.099031in}}%
\pgfpathlineto{\pgfqpoint{3.816692in}{8.101069in}}%
\pgfpathlineto{\pgfqpoint{3.818974in}{8.099133in}}%
\pgfpathlineto{\pgfqpoint{3.825819in}{8.095567in}}%
\pgfpathlineto{\pgfqpoint{3.828101in}{8.096484in}}%
\pgfpathlineto{\pgfqpoint{3.830383in}{8.094649in}}%
\pgfpathlineto{\pgfqpoint{3.832665in}{8.091796in}}%
\pgfpathlineto{\pgfqpoint{3.834947in}{8.095516in}}%
\pgfpathlineto{\pgfqpoint{3.841793in}{8.092408in}}%
\pgfpathlineto{\pgfqpoint{3.844075in}{8.085683in}}%
\pgfpathlineto{\pgfqpoint{3.846356in}{8.087466in}}%
\pgfpathlineto{\pgfqpoint{3.850920in}{8.101324in}}%
\pgfpathlineto{\pgfqpoint{3.857766in}{8.104330in}}%
\pgfpathlineto{\pgfqpoint{3.860048in}{8.097350in}}%
\pgfpathlineto{\pgfqpoint{3.862330in}{8.102292in}}%
\pgfpathlineto{\pgfqpoint{3.864612in}{8.109679in}}%
\pgfpathlineto{\pgfqpoint{3.866894in}{8.110647in}}%
\pgfpathlineto{\pgfqpoint{3.873739in}{8.112481in}}%
\pgfpathlineto{\pgfqpoint{3.876021in}{8.114876in}}%
\pgfpathlineto{\pgfqpoint{3.878303in}{8.112227in}}%
\pgfpathlineto{\pgfqpoint{3.880585in}{8.113551in}}%
\pgfpathlineto{\pgfqpoint{3.882867in}{8.117627in}}%
\pgfpathlineto{\pgfqpoint{3.889713in}{8.119818in}}%
\pgfpathlineto{\pgfqpoint{3.891995in}{8.121397in}}%
\pgfpathlineto{\pgfqpoint{3.894276in}{8.119257in}}%
\pgfpathlineto{\pgfqpoint{3.896558in}{8.123894in}}%
\pgfpathlineto{\pgfqpoint{3.898840in}{8.124149in}}%
\pgfpathlineto{\pgfqpoint{3.905686in}{8.125983in}}%
\pgfpathlineto{\pgfqpoint{3.907968in}{8.125117in}}%
\pgfpathlineto{\pgfqpoint{3.910250in}{8.129549in}}%
\pgfpathlineto{\pgfqpoint{3.912532in}{8.126390in}}%
\pgfpathlineto{\pgfqpoint{3.914814in}{8.133319in}}%
\pgfpathlineto{\pgfqpoint{3.921659in}{8.133166in}}%
\pgfpathlineto{\pgfqpoint{3.923941in}{8.134950in}}%
\pgfpathlineto{\pgfqpoint{3.926223in}{8.135969in}}%
\pgfpathlineto{\pgfqpoint{3.928505in}{8.141267in}}%
\pgfpathlineto{\pgfqpoint{3.939915in}{8.142388in}}%
\pgfpathlineto{\pgfqpoint{3.942197in}{8.141369in}}%
\pgfpathlineto{\pgfqpoint{3.944478in}{8.146566in}}%
\pgfpathlineto{\pgfqpoint{3.946760in}{8.149521in}}%
\pgfpathlineto{\pgfqpoint{3.953606in}{8.150438in}}%
\pgfpathlineto{\pgfqpoint{3.955888in}{8.154514in}}%
\pgfpathlineto{\pgfqpoint{3.958170in}{8.160475in}}%
\pgfpathlineto{\pgfqpoint{3.962734in}{8.160526in}}%
\pgfpathlineto{\pgfqpoint{3.969579in}{8.134746in}}%
\pgfpathlineto{\pgfqpoint{3.971861in}{8.129957in}}%
\pgfpathlineto{\pgfqpoint{3.974143in}{8.129345in}}%
\pgfpathlineto{\pgfqpoint{3.976425in}{8.130976in}}%
\pgfpathlineto{\pgfqpoint{3.978707in}{8.138822in}}%
\pgfpathlineto{\pgfqpoint{3.985553in}{8.138669in}}%
\pgfpathlineto{\pgfqpoint{3.990117in}{8.130721in}}%
\pgfpathlineto{\pgfqpoint{3.994680in}{8.129192in}}%
\pgfpathlineto{\pgfqpoint{4.001526in}{8.134644in}}%
\pgfpathlineto{\pgfqpoint{4.006090in}{8.170664in}}%
\pgfpathlineto{\pgfqpoint{4.008372in}{8.173365in}}%
\pgfpathlineto{\pgfqpoint{4.010654in}{8.174740in}}%
\pgfpathlineto{\pgfqpoint{4.017499in}{8.175097in}}%
\pgfpathlineto{\pgfqpoint{4.019781in}{8.175912in}}%
\pgfpathlineto{\pgfqpoint{4.024345in}{8.181669in}}%
\pgfpathlineto{\pgfqpoint{4.026627in}{8.186356in}}%
\pgfpathlineto{\pgfqpoint{4.033473in}{8.186662in}}%
\pgfpathlineto{\pgfqpoint{4.035755in}{8.190330in}}%
\pgfpathlineto{\pgfqpoint{4.040319in}{8.180956in}}%
\pgfpathlineto{\pgfqpoint{4.042600in}{8.182382in}}%
\pgfpathlineto{\pgfqpoint{4.049446in}{8.178765in}}%
\pgfpathlineto{\pgfqpoint{4.051728in}{8.178816in}}%
\pgfpathlineto{\pgfqpoint{4.054010in}{8.174333in}}%
\pgfpathlineto{\pgfqpoint{4.058574in}{8.172346in}}%
\pgfpathlineto{\pgfqpoint{4.065419in}{8.165722in}}%
\pgfpathlineto{\pgfqpoint{4.067701in}{8.177797in}}%
\pgfpathlineto{\pgfqpoint{4.069983in}{8.182331in}}%
\pgfpathlineto{\pgfqpoint{4.072265in}{8.181109in}}%
\pgfpathlineto{\pgfqpoint{4.074547in}{8.177338in}}%
\pgfpathlineto{\pgfqpoint{4.083675in}{8.175912in}}%
\pgfpathlineto{\pgfqpoint{4.085957in}{8.173670in}}%
\pgfpathlineto{\pgfqpoint{4.088239in}{8.165824in}}%
\pgfpathlineto{\pgfqpoint{4.090520in}{8.145954in}}%
\pgfpathlineto{\pgfqpoint{4.097366in}{8.140401in}}%
\pgfpathlineto{\pgfqpoint{4.099648in}{8.144171in}}%
\pgfpathlineto{\pgfqpoint{4.101930in}{8.146107in}}%
\pgfpathlineto{\pgfqpoint{4.104212in}{8.136376in}}%
\pgfpathlineto{\pgfqpoint{4.106494in}{8.136987in}}%
\pgfpathlineto{\pgfqpoint{4.113340in}{8.117984in}}%
\pgfpathlineto{\pgfqpoint{4.115621in}{8.130466in}}%
\pgfpathlineto{\pgfqpoint{4.117903in}{8.133319in}}%
\pgfpathlineto{\pgfqpoint{4.122467in}{8.146464in}}%
\pgfpathlineto{\pgfqpoint{4.129313in}{8.143662in}}%
\pgfpathlineto{\pgfqpoint{4.131595in}{8.149317in}}%
\pgfpathlineto{\pgfqpoint{4.133877in}{8.150744in}}%
\pgfpathlineto{\pgfqpoint{4.136159in}{8.149419in}}%
\pgfpathlineto{\pgfqpoint{4.138441in}{8.158233in}}%
\pgfpathlineto{\pgfqpoint{4.147568in}{8.156806in}}%
\pgfpathlineto{\pgfqpoint{4.149850in}{8.151253in}}%
\pgfpathlineto{\pgfqpoint{4.152132in}{8.155736in}}%
\pgfpathlineto{\pgfqpoint{4.154414in}{8.155787in}}%
\pgfpathlineto{\pgfqpoint{4.161260in}{8.158590in}}%
\pgfpathlineto{\pgfqpoint{4.163541in}{8.161799in}}%
\pgfpathlineto{\pgfqpoint{4.165823in}{8.161494in}}%
\pgfpathlineto{\pgfqpoint{4.168105in}{8.168117in}}%
\pgfpathlineto{\pgfqpoint{4.170387in}{8.169849in}}%
\pgfpathlineto{\pgfqpoint{4.177233in}{8.158640in}}%
\pgfpathlineto{\pgfqpoint{4.179515in}{8.160729in}}%
\pgfpathlineto{\pgfqpoint{4.181797in}{8.165926in}}%
\pgfpathlineto{\pgfqpoint{4.184079in}{8.166945in}}%
\pgfpathlineto{\pgfqpoint{4.186361in}{8.167098in}}%
\pgfpathlineto{\pgfqpoint{4.193206in}{8.164601in}}%
\pgfpathlineto{\pgfqpoint{4.195488in}{8.159965in}}%
\pgfpathlineto{\pgfqpoint{4.197770in}{8.160220in}}%
\pgfpathlineto{\pgfqpoint{4.200052in}{8.152374in}}%
\pgfpathlineto{\pgfqpoint{4.202334in}{8.147992in}}%
\pgfpathlineto{\pgfqpoint{4.209180in}{8.158895in}}%
\pgfpathlineto{\pgfqpoint{4.211462in}{8.160984in}}%
\pgfpathlineto{\pgfqpoint{4.213743in}{8.154259in}}%
\pgfpathlineto{\pgfqpoint{4.216025in}{8.159608in}}%
\pgfpathlineto{\pgfqpoint{4.218307in}{8.162665in}}%
\pgfpathlineto{\pgfqpoint{4.225153in}{8.159558in}}%
\pgfpathlineto{\pgfqpoint{4.227435in}{8.166843in}}%
\pgfpathlineto{\pgfqpoint{4.229717in}{8.162513in}}%
\pgfpathlineto{\pgfqpoint{4.231999in}{8.161443in}}%
\pgfpathlineto{\pgfqpoint{4.234281in}{8.167455in}}%
\pgfpathlineto{\pgfqpoint{4.241126in}{8.173976in}}%
\pgfpathlineto{\pgfqpoint{4.243408in}{8.177848in}}%
\pgfpathlineto{\pgfqpoint{4.245690in}{8.175504in}}%
\pgfpathlineto{\pgfqpoint{4.247972in}{8.176116in}}%
\pgfpathlineto{\pgfqpoint{4.250254in}{8.174893in}}%
\pgfpathlineto{\pgfqpoint{4.257100in}{8.168372in}}%
\pgfpathlineto{\pgfqpoint{4.259382in}{8.170308in}}%
\pgfpathlineto{\pgfqpoint{4.261663in}{8.174791in}}%
\pgfpathlineto{\pgfqpoint{4.266227in}{8.159405in}}%
\pgfpathlineto{\pgfqpoint{4.273073in}{8.162818in}}%
\pgfpathlineto{\pgfqpoint{4.275355in}{8.166996in}}%
\pgfpathlineto{\pgfqpoint{4.277637in}{8.178918in}}%
\pgfpathlineto{\pgfqpoint{4.279919in}{8.183198in}}%
\pgfpathlineto{\pgfqpoint{4.291328in}{8.188343in}}%
\pgfpathlineto{\pgfqpoint{4.293610in}{8.184369in}}%
\pgfpathlineto{\pgfqpoint{4.295892in}{8.178408in}}%
\pgfpathlineto{\pgfqpoint{4.298174in}{8.178001in}}%
\pgfpathlineto{\pgfqpoint{4.307302in}{8.182382in}}%
\pgfpathlineto{\pgfqpoint{4.311865in}{8.196903in}}%
\pgfpathlineto{\pgfqpoint{4.314147in}{8.193846in}}%
\pgfpathlineto{\pgfqpoint{4.320993in}{8.196036in}}%
\pgfpathlineto{\pgfqpoint{4.323275in}{8.190127in}}%
\pgfpathlineto{\pgfqpoint{4.325557in}{8.198380in}}%
\pgfpathlineto{\pgfqpoint{4.327839in}{8.197005in}}%
\pgfpathlineto{\pgfqpoint{4.336966in}{8.205870in}}%
\pgfpathlineto{\pgfqpoint{4.339248in}{8.204137in}}%
\pgfpathlineto{\pgfqpoint{4.343812in}{8.197667in}}%
\pgfpathlineto{\pgfqpoint{4.352940in}{8.200724in}}%
\pgfpathlineto{\pgfqpoint{4.355222in}{8.194304in}}%
\pgfpathlineto{\pgfqpoint{4.357504in}{8.200061in}}%
\pgfpathlineto{\pgfqpoint{4.359785in}{8.198533in}}%
\pgfpathlineto{\pgfqpoint{4.362067in}{8.202252in}}%
\pgfpathlineto{\pgfqpoint{4.373477in}{8.203526in}}%
\pgfpathlineto{\pgfqpoint{4.375759in}{8.207856in}}%
\pgfpathlineto{\pgfqpoint{4.378041in}{8.208570in}}%
\pgfpathlineto{\pgfqpoint{4.384886in}{8.207500in}}%
\pgfpathlineto{\pgfqpoint{4.387168in}{8.210098in}}%
\pgfpathlineto{\pgfqpoint{4.389450in}{8.207245in}}%
\pgfpathlineto{\pgfqpoint{4.391732in}{8.213766in}}%
\pgfpathlineto{\pgfqpoint{4.394014in}{8.217893in}}%
\pgfpathlineto{\pgfqpoint{4.400860in}{8.220950in}}%
\pgfpathlineto{\pgfqpoint{4.403142in}{8.219371in}}%
\pgfpathlineto{\pgfqpoint{4.405424in}{8.216874in}}%
\pgfpathlineto{\pgfqpoint{4.407706in}{8.210862in}}%
\pgfpathlineto{\pgfqpoint{4.409987in}{8.212187in}}%
\pgfpathlineto{\pgfqpoint{4.416833in}{8.211983in}}%
\pgfpathlineto{\pgfqpoint{4.419115in}{8.213665in}}%
\pgfpathlineto{\pgfqpoint{4.421397in}{8.216518in}}%
\pgfpathlineto{\pgfqpoint{4.423679in}{8.217231in}}%
\pgfpathlineto{\pgfqpoint{4.425961in}{8.220237in}}%
\pgfpathlineto{\pgfqpoint{4.432807in}{8.215448in}}%
\pgfpathlineto{\pgfqpoint{4.435088in}{8.210964in}}%
\pgfpathlineto{\pgfqpoint{4.437370in}{8.213461in}}%
\pgfpathlineto{\pgfqpoint{4.441934in}{8.213665in}}%
\pgfpathlineto{\pgfqpoint{4.448780in}{8.211627in}}%
\pgfpathlineto{\pgfqpoint{4.451062in}{8.217537in}}%
\pgfpathlineto{\pgfqpoint{4.453344in}{8.220543in}}%
\pgfpathlineto{\pgfqpoint{4.455626in}{8.221358in}}%
\pgfpathlineto{\pgfqpoint{4.464753in}{8.219116in}}%
\pgfpathlineto{\pgfqpoint{4.467035in}{8.217537in}}%
\pgfpathlineto{\pgfqpoint{4.469317in}{8.218097in}}%
\pgfpathlineto{\pgfqpoint{4.471599in}{8.214582in}}%
\pgfpathlineto{\pgfqpoint{4.473881in}{8.216416in}}%
\pgfpathlineto{\pgfqpoint{4.480727in}{8.219524in}}%
\pgfpathlineto{\pgfqpoint{4.483008in}{8.219778in}}%
\pgfpathlineto{\pgfqpoint{4.485290in}{8.224771in}}%
\pgfpathlineto{\pgfqpoint{4.487572in}{8.213002in}}%
\pgfpathlineto{\pgfqpoint{4.489854in}{8.218759in}}%
\pgfpathlineto{\pgfqpoint{4.496700in}{8.216416in}}%
\pgfpathlineto{\pgfqpoint{4.498982in}{8.220084in}}%
\pgfpathlineto{\pgfqpoint{4.501264in}{8.218097in}}%
\pgfpathlineto{\pgfqpoint{4.503546in}{8.220135in}}%
\pgfpathlineto{\pgfqpoint{4.505828in}{8.220084in}}%
\pgfpathlineto{\pgfqpoint{4.512673in}{8.221918in}}%
\pgfpathlineto{\pgfqpoint{4.514955in}{8.215142in}}%
\pgfpathlineto{\pgfqpoint{4.517237in}{8.213868in}}%
\pgfpathlineto{\pgfqpoint{4.519519in}{8.201131in}}%
\pgfpathlineto{\pgfqpoint{4.521801in}{8.197616in}}%
\pgfpathlineto{\pgfqpoint{4.528647in}{8.200469in}}%
\pgfpathlineto{\pgfqpoint{4.530928in}{8.196138in}}%
\pgfpathlineto{\pgfqpoint{4.533210in}{8.195119in}}%
\pgfpathlineto{\pgfqpoint{4.535492in}{8.193234in}}%
\pgfpathlineto{\pgfqpoint{4.537774in}{8.200928in}}%
\pgfpathlineto{\pgfqpoint{4.544620in}{8.199705in}}%
\pgfpathlineto{\pgfqpoint{4.546902in}{8.201080in}}%
\pgfpathlineto{\pgfqpoint{4.549184in}{8.205207in}}%
\pgfpathlineto{\pgfqpoint{4.551466in}{8.207602in}}%
\pgfpathlineto{\pgfqpoint{4.553748in}{8.205717in}}%
\pgfpathlineto{\pgfqpoint{4.560593in}{8.215601in}}%
\pgfpathlineto{\pgfqpoint{4.562875in}{8.216161in}}%
\pgfpathlineto{\pgfqpoint{4.565157in}{8.221613in}}%
\pgfpathlineto{\pgfqpoint{4.567439in}{8.221154in}}%
\pgfpathlineto{\pgfqpoint{4.569721in}{8.219473in}}%
\pgfpathlineto{\pgfqpoint{4.576567in}{8.222071in}}%
\pgfpathlineto{\pgfqpoint{4.578849in}{8.221613in}}%
\pgfpathlineto{\pgfqpoint{4.581130in}{8.218607in}}%
\pgfpathlineto{\pgfqpoint{4.594822in}{8.219880in}}%
\pgfpathlineto{\pgfqpoint{4.597104in}{8.218097in}}%
\pgfpathlineto{\pgfqpoint{4.599386in}{8.217129in}}%
\pgfpathlineto{\pgfqpoint{4.601668in}{8.219778in}}%
\pgfpathlineto{\pgfqpoint{4.608513in}{8.222886in}}%
\pgfpathlineto{\pgfqpoint{4.610795in}{8.220950in}}%
\pgfpathlineto{\pgfqpoint{4.613077in}{8.221765in}}%
\pgfpathlineto{\pgfqpoint{4.617641in}{8.218607in}}%
\pgfpathlineto{\pgfqpoint{4.624487in}{8.221052in}}%
\pgfpathlineto{\pgfqpoint{4.629050in}{8.225077in}}%
\pgfpathlineto{\pgfqpoint{4.631332in}{8.231700in}}%
\pgfpathlineto{\pgfqpoint{4.633614in}{8.231038in}}%
\pgfpathlineto{\pgfqpoint{4.640460in}{8.226605in}}%
\pgfpathlineto{\pgfqpoint{4.642742in}{8.220543in}}%
\pgfpathlineto{\pgfqpoint{4.645024in}{8.222886in}}%
\pgfpathlineto{\pgfqpoint{4.647306in}{8.212034in}}%
\pgfpathlineto{\pgfqpoint{4.656433in}{8.210761in}}%
\pgfpathlineto{\pgfqpoint{4.658715in}{8.208417in}}%
\pgfpathlineto{\pgfqpoint{4.660997in}{8.197157in}}%
\pgfpathlineto{\pgfqpoint{4.663279in}{8.194865in}}%
\pgfpathlineto{\pgfqpoint{4.665561in}{8.201641in}}%
\pgfpathlineto{\pgfqpoint{4.672407in}{8.202405in}}%
\pgfpathlineto{\pgfqpoint{4.674689in}{8.190432in}}%
\pgfpathlineto{\pgfqpoint{4.676971in}{8.207245in}}%
\pgfpathlineto{\pgfqpoint{4.679252in}{8.194763in}}%
\pgfpathlineto{\pgfqpoint{4.681534in}{8.173059in}}%
\pgfpathlineto{\pgfqpoint{4.688380in}{8.168881in}}%
\pgfpathlineto{\pgfqpoint{4.690662in}{8.174638in}}%
\pgfpathlineto{\pgfqpoint{4.692944in}{8.174842in}}%
\pgfpathlineto{\pgfqpoint{4.695226in}{8.178561in}}%
\pgfpathlineto{\pgfqpoint{4.697508in}{8.189158in}}%
\pgfpathlineto{\pgfqpoint{4.704353in}{8.190076in}}%
\pgfpathlineto{\pgfqpoint{4.706635in}{8.205054in}}%
\pgfpathlineto{\pgfqpoint{4.708917in}{8.196138in}}%
\pgfpathlineto{\pgfqpoint{4.711199in}{8.223650in}}%
\pgfpathlineto{\pgfqpoint{4.713481in}{8.239597in}}%
\pgfpathlineto{\pgfqpoint{4.720327in}{8.243979in}}%
\pgfpathlineto{\pgfqpoint{4.722609in}{8.250704in}}%
\pgfpathlineto{\pgfqpoint{4.724891in}{8.250500in}}%
\pgfpathlineto{\pgfqpoint{4.727172in}{8.255391in}}%
\pgfpathlineto{\pgfqpoint{4.729454in}{8.262932in}}%
\pgfpathlineto{\pgfqpoint{4.736300in}{8.260792in}}%
\pgfpathlineto{\pgfqpoint{4.738582in}{8.268689in}}%
\pgfpathlineto{\pgfqpoint{4.740864in}{8.271593in}}%
\pgfpathlineto{\pgfqpoint{4.745428in}{8.275465in}}%
\pgfpathlineto{\pgfqpoint{4.752273in}{8.280458in}}%
\pgfpathlineto{\pgfqpoint{4.754555in}{8.277961in}}%
\pgfpathlineto{\pgfqpoint{4.761401in}{8.285807in}}%
\pgfpathlineto{\pgfqpoint{4.768247in}{8.285094in}}%
\pgfpathlineto{\pgfqpoint{4.770529in}{8.291666in}}%
\pgfpathlineto{\pgfqpoint{4.772811in}{8.289577in}}%
\pgfpathlineto{\pgfqpoint{4.775093in}{8.291972in}}%
\pgfpathlineto{\pgfqpoint{4.777374in}{8.295589in}}%
\pgfpathlineto{\pgfqpoint{4.784220in}{8.294316in}}%
\pgfpathlineto{\pgfqpoint{4.786502in}{8.286113in}}%
\pgfpathlineto{\pgfqpoint{4.788784in}{8.287234in}}%
\pgfpathlineto{\pgfqpoint{4.793348in}{8.295284in}}%
\pgfpathlineto{\pgfqpoint{4.800194in}{8.286521in}}%
\pgfpathlineto{\pgfqpoint{4.802475in}{8.297576in}}%
\pgfpathlineto{\pgfqpoint{4.804757in}{8.305117in}}%
\pgfpathlineto{\pgfqpoint{4.809321in}{8.305168in}}%
\pgfpathlineto{\pgfqpoint{4.816167in}{8.299105in}}%
\pgfpathlineto{\pgfqpoint{4.818449in}{8.298646in}}%
\pgfpathlineto{\pgfqpoint{4.820731in}{8.286928in}}%
\pgfpathlineto{\pgfqpoint{4.823013in}{8.291004in}}%
\pgfpathlineto{\pgfqpoint{4.825294in}{8.281833in}}%
\pgfpathlineto{\pgfqpoint{4.832140in}{8.280611in}}%
\pgfpathlineto{\pgfqpoint{4.834422in}{8.290596in}}%
\pgfpathlineto{\pgfqpoint{4.836704in}{8.297576in}}%
\pgfpathlineto{\pgfqpoint{4.838986in}{8.318873in}}%
\pgfpathlineto{\pgfqpoint{4.841268in}{8.319688in}}%
\pgfpathlineto{\pgfqpoint{4.848114in}{8.327840in}}%
\pgfpathlineto{\pgfqpoint{4.850395in}{8.326005in}}%
\pgfpathlineto{\pgfqpoint{4.852677in}{8.326413in}}%
\pgfpathlineto{\pgfqpoint{4.857241in}{8.323254in}}%
\pgfpathlineto{\pgfqpoint{4.864087in}{8.325292in}}%
\pgfpathlineto{\pgfqpoint{4.866369in}{8.321318in}}%
\pgfpathlineto{\pgfqpoint{4.868651in}{8.314440in}}%
\pgfpathlineto{\pgfqpoint{4.873215in}{8.313268in}}%
\pgfpathlineto{\pgfqpoint{4.880060in}{8.296506in}}%
\pgfpathlineto{\pgfqpoint{4.882342in}{8.288762in}}%
\pgfpathlineto{\pgfqpoint{4.884624in}{8.293959in}}%
\pgfpathlineto{\pgfqpoint{4.886906in}{8.311332in}}%
\pgfpathlineto{\pgfqpoint{4.889188in}{8.302213in}}%
\pgfpathlineto{\pgfqpoint{4.896034in}{8.298239in}}%
\pgfpathlineto{\pgfqpoint{4.898316in}{8.297678in}}%
\pgfpathlineto{\pgfqpoint{4.900597in}{8.294163in}}%
\pgfpathlineto{\pgfqpoint{4.902879in}{8.293348in}}%
\pgfpathlineto{\pgfqpoint{4.905161in}{8.303945in}}%
\pgfpathlineto{\pgfqpoint{4.914289in}{8.303690in}}%
\pgfpathlineto{\pgfqpoint{4.916571in}{8.306390in}}%
\pgfpathlineto{\pgfqpoint{4.918853in}{8.321573in}}%
\pgfpathlineto{\pgfqpoint{4.921135in}{8.313116in}}%
\pgfpathlineto{\pgfqpoint{4.927980in}{8.314084in}}%
\pgfpathlineto{\pgfqpoint{4.930262in}{8.311332in}}%
\pgfpathlineto{\pgfqpoint{4.932544in}{8.312708in}}%
\pgfpathlineto{\pgfqpoint{4.934826in}{8.322643in}}%
\pgfpathlineto{\pgfqpoint{4.937108in}{8.305320in}}%
\pgfpathlineto{\pgfqpoint{4.943954in}{8.314899in}}%
\pgfpathlineto{\pgfqpoint{4.946236in}{8.321777in}}%
\pgfpathlineto{\pgfqpoint{4.948517in}{8.316682in}}%
\pgfpathlineto{\pgfqpoint{4.950799in}{8.324579in}}%
\pgfpathlineto{\pgfqpoint{4.953081in}{8.322388in}}%
\pgfpathlineto{\pgfqpoint{4.959927in}{8.316835in}}%
\pgfpathlineto{\pgfqpoint{4.962209in}{8.321267in}}%
\pgfpathlineto{\pgfqpoint{4.964491in}{8.319841in}}%
\pgfpathlineto{\pgfqpoint{4.966773in}{8.326311in}}%
\pgfpathlineto{\pgfqpoint{4.969055in}{8.326464in}}%
\pgfpathlineto{\pgfqpoint{4.978182in}{8.330947in}}%
\pgfpathlineto{\pgfqpoint{4.980464in}{8.333138in}}%
\pgfpathlineto{\pgfqpoint{4.982746in}{8.331712in}}%
\pgfpathlineto{\pgfqpoint{4.985028in}{8.336399in}}%
\pgfpathlineto{\pgfqpoint{4.994156in}{8.340984in}}%
\pgfpathlineto{\pgfqpoint{4.996437in}{8.339914in}}%
\pgfpathlineto{\pgfqpoint{4.998719in}{8.343226in}}%
\pgfpathlineto{\pgfqpoint{5.001001in}{8.338844in}}%
\pgfpathlineto{\pgfqpoint{5.007847in}{8.347251in}}%
\pgfpathlineto{\pgfqpoint{5.010129in}{8.337112in}}%
\pgfpathlineto{\pgfqpoint{5.012411in}{8.332017in}}%
\pgfpathlineto{\pgfqpoint{5.014693in}{8.333902in}}%
\pgfpathlineto{\pgfqpoint{5.016975in}{8.319280in}}%
\pgfpathlineto{\pgfqpoint{5.023820in}{8.328451in}}%
\pgfpathlineto{\pgfqpoint{5.026102in}{8.309702in}}%
\pgfpathlineto{\pgfqpoint{5.028384in}{8.307257in}}%
\pgfpathlineto{\pgfqpoint{5.030666in}{8.319790in}}%
\pgfpathlineto{\pgfqpoint{5.032948in}{8.311893in}}%
\pgfpathlineto{\pgfqpoint{5.039794in}{8.327687in}}%
\pgfpathlineto{\pgfqpoint{5.042076in}{8.318669in}}%
\pgfpathlineto{\pgfqpoint{5.044358in}{8.328910in}}%
\pgfpathlineto{\pgfqpoint{5.046639in}{8.325190in}}%
\pgfpathlineto{\pgfqpoint{5.048921in}{8.328960in}}%
\pgfpathlineto{\pgfqpoint{5.055767in}{8.327330in}}%
\pgfpathlineto{\pgfqpoint{5.058049in}{8.328196in}}%
\pgfpathlineto{\pgfqpoint{5.060331in}{8.311689in}}%
\pgfpathlineto{\pgfqpoint{5.062613in}{8.311180in}}%
\pgfpathlineto{\pgfqpoint{5.071740in}{8.326974in}}%
\pgfpathlineto{\pgfqpoint{5.074022in}{8.321981in}}%
\pgfpathlineto{\pgfqpoint{5.076304in}{8.310874in}}%
\pgfpathlineto{\pgfqpoint{5.078586in}{8.312148in}}%
\pgfpathlineto{\pgfqpoint{5.089996in}{8.327687in}}%
\pgfpathlineto{\pgfqpoint{5.092278in}{8.327992in}}%
\pgfpathlineto{\pgfqpoint{5.094559in}{8.330336in}}%
\pgfpathlineto{\pgfqpoint{5.096841in}{8.331610in}}%
\pgfpathlineto{\pgfqpoint{5.103687in}{8.326005in}}%
\pgfpathlineto{\pgfqpoint{5.105969in}{8.326719in}}%
\pgfpathlineto{\pgfqpoint{5.108251in}{8.328757in}}%
\pgfpathlineto{\pgfqpoint{5.110533in}{8.326260in}}%
\pgfpathlineto{\pgfqpoint{5.112815in}{8.307206in}}%
\pgfpathlineto{\pgfqpoint{5.119660in}{8.319943in}}%
\pgfpathlineto{\pgfqpoint{5.121942in}{8.317650in}}%
\pgfpathlineto{\pgfqpoint{5.124224in}{8.320707in}}%
\pgfpathlineto{\pgfqpoint{5.126506in}{8.297831in}}%
\pgfpathlineto{\pgfqpoint{5.128788in}{8.294774in}}%
\pgfpathlineto{\pgfqpoint{5.135634in}{8.290189in}}%
\pgfpathlineto{\pgfqpoint{5.137916in}{8.291768in}}%
\pgfpathlineto{\pgfqpoint{5.140198in}{8.285553in}}%
\pgfpathlineto{\pgfqpoint{5.142480in}{8.282954in}}%
\pgfpathlineto{\pgfqpoint{5.144761in}{8.288813in}}%
\pgfpathlineto{\pgfqpoint{5.151607in}{8.294927in}}%
\pgfpathlineto{\pgfqpoint{5.153889in}{8.289883in}}%
\pgfpathlineto{\pgfqpoint{5.156171in}{8.288609in}}%
\pgfpathlineto{\pgfqpoint{5.158453in}{8.293042in}}%
\pgfpathlineto{\pgfqpoint{5.160735in}{8.302111in}}%
\pgfpathlineto{\pgfqpoint{5.167581in}{8.298952in}}%
\pgfpathlineto{\pgfqpoint{5.169862in}{8.299716in}}%
\pgfpathlineto{\pgfqpoint{5.172144in}{8.305371in}}%
\pgfpathlineto{\pgfqpoint{5.174426in}{8.313727in}}%
\pgfpathlineto{\pgfqpoint{5.176708in}{8.314440in}}%
\pgfpathlineto{\pgfqpoint{5.183554in}{8.312453in}}%
\pgfpathlineto{\pgfqpoint{5.185836in}{8.313625in}}%
\pgfpathlineto{\pgfqpoint{5.188118in}{8.312708in}}%
\pgfpathlineto{\pgfqpoint{5.190400in}{8.313472in}}%
\pgfpathlineto{\pgfqpoint{5.192681in}{8.308479in}}%
\pgfpathlineto{\pgfqpoint{5.201809in}{8.302111in}}%
\pgfpathlineto{\pgfqpoint{5.204091in}{8.308683in}}%
\pgfpathlineto{\pgfqpoint{5.206373in}{8.307970in}}%
\pgfpathlineto{\pgfqpoint{5.208655in}{8.299767in}}%
\pgfpathlineto{\pgfqpoint{5.215501in}{8.299309in}}%
\pgfpathlineto{\pgfqpoint{5.217782in}{8.299767in}}%
\pgfpathlineto{\pgfqpoint{5.220064in}{8.304760in}}%
\pgfpathlineto{\pgfqpoint{5.222346in}{8.296456in}}%
\pgfpathlineto{\pgfqpoint{5.224628in}{8.290596in}}%
\pgfpathlineto{\pgfqpoint{5.231474in}{8.288355in}}%
\pgfpathlineto{\pgfqpoint{5.233756in}{8.290138in}}%
\pgfpathlineto{\pgfqpoint{5.236038in}{8.299614in}}%
\pgfpathlineto{\pgfqpoint{5.238320in}{8.303384in}}%
\pgfpathlineto{\pgfqpoint{5.240602in}{8.295029in}}%
\pgfpathlineto{\pgfqpoint{5.247447in}{8.285043in}}%
\pgfpathlineto{\pgfqpoint{5.252011in}{8.289985in}}%
\pgfpathlineto{\pgfqpoint{5.254293in}{8.302264in}}%
\pgfpathlineto{\pgfqpoint{5.256575in}{8.299207in}}%
\pgfpathlineto{\pgfqpoint{5.265703in}{8.303283in}}%
\pgfpathlineto{\pgfqpoint{5.267984in}{8.292736in}}%
\pgfpathlineto{\pgfqpoint{5.270266in}{8.285298in}}%
\pgfpathlineto{\pgfqpoint{5.272548in}{8.290647in}}%
\pgfpathlineto{\pgfqpoint{5.279394in}{8.276229in}}%
\pgfpathlineto{\pgfqpoint{5.281676in}{8.277859in}}%
\pgfpathlineto{\pgfqpoint{5.283958in}{8.284024in}}%
\pgfpathlineto{\pgfqpoint{5.286240in}{8.282801in}}%
\pgfpathlineto{\pgfqpoint{5.295367in}{8.281426in}}%
\pgfpathlineto{\pgfqpoint{5.297649in}{8.283056in}}%
\pgfpathlineto{\pgfqpoint{5.299931in}{8.270574in}}%
\pgfpathlineto{\pgfqpoint{5.304495in}{8.280814in}}%
\pgfpathlineto{\pgfqpoint{5.311341in}{8.287794in}}%
\pgfpathlineto{\pgfqpoint{5.313623in}{8.289221in}}%
\pgfpathlineto{\pgfqpoint{5.315904in}{8.285654in}}%
\pgfpathlineto{\pgfqpoint{5.318186in}{8.291004in}}%
\pgfpathlineto{\pgfqpoint{5.320468in}{8.288813in}}%
\pgfpathlineto{\pgfqpoint{5.327314in}{8.290800in}}%
\pgfpathlineto{\pgfqpoint{5.329596in}{8.284534in}}%
\pgfpathlineto{\pgfqpoint{5.331878in}{8.282954in}}%
\pgfpathlineto{\pgfqpoint{5.334160in}{8.255850in}}%
\pgfpathlineto{\pgfqpoint{5.343287in}{8.252181in}}%
\pgfpathlineto{\pgfqpoint{5.345569in}{8.263237in}}%
\pgfpathlineto{\pgfqpoint{5.347851in}{8.264766in}}%
\pgfpathlineto{\pgfqpoint{5.350133in}{8.265326in}}%
\pgfpathlineto{\pgfqpoint{5.352415in}{8.264256in}}%
\pgfpathlineto{\pgfqpoint{5.359261in}{8.258652in}}%
\pgfpathlineto{\pgfqpoint{5.361543in}{8.259977in}}%
\pgfpathlineto{\pgfqpoint{5.363825in}{8.262779in}}%
\pgfpathlineto{\pgfqpoint{5.366106in}{8.254729in}}%
\pgfpathlineto{\pgfqpoint{5.368388in}{8.253048in}}%
\pgfpathlineto{\pgfqpoint{5.375234in}{8.263798in}}%
\pgfpathlineto{\pgfqpoint{5.377516in}{8.251163in}}%
\pgfpathlineto{\pgfqpoint{5.379798in}{8.251315in}}%
\pgfpathlineto{\pgfqpoint{5.382080in}{8.246170in}}%
\pgfpathlineto{\pgfqpoint{5.384362in}{8.250245in}}%
\pgfpathlineto{\pgfqpoint{5.391207in}{8.254627in}}%
\pgfpathlineto{\pgfqpoint{5.393489in}{8.250347in}}%
\pgfpathlineto{\pgfqpoint{5.395771in}{8.243826in}}%
\pgfpathlineto{\pgfqpoint{5.398053in}{8.232210in}}%
\pgfpathlineto{\pgfqpoint{5.407181in}{8.210149in}}%
\pgfpathlineto{\pgfqpoint{5.409463in}{8.205819in}}%
\pgfpathlineto{\pgfqpoint{5.411745in}{8.229255in}}%
\pgfpathlineto{\pgfqpoint{5.414026in}{8.234706in}}%
\pgfpathlineto{\pgfqpoint{5.416308in}{8.236184in}}%
\pgfpathlineto{\pgfqpoint{5.423154in}{8.226605in}}%
\pgfpathlineto{\pgfqpoint{5.425436in}{8.209793in}}%
\pgfpathlineto{\pgfqpoint{5.427718in}{8.222428in}}%
\pgfpathlineto{\pgfqpoint{5.430000in}{8.224771in}}%
\pgfpathlineto{\pgfqpoint{5.432282in}{8.215957in}}%
\pgfpathlineto{\pgfqpoint{5.441409in}{8.232566in}}%
\pgfpathlineto{\pgfqpoint{5.443691in}{8.220695in}}%
\pgfpathlineto{\pgfqpoint{5.445973in}{8.220288in}}%
\pgfpathlineto{\pgfqpoint{5.448255in}{8.222530in}}%
\pgfpathlineto{\pgfqpoint{5.455101in}{8.220441in}}%
\pgfpathlineto{\pgfqpoint{5.457383in}{8.233331in}}%
\pgfpathlineto{\pgfqpoint{5.459665in}{8.236133in}}%
\pgfpathlineto{\pgfqpoint{5.461946in}{8.230325in}}%
\pgfpathlineto{\pgfqpoint{5.464228in}{8.214938in}}%
\pgfpathlineto{\pgfqpoint{5.471074in}{8.216874in}}%
\pgfpathlineto{\pgfqpoint{5.473356in}{8.207551in}}%
\pgfpathlineto{\pgfqpoint{5.475638in}{8.205920in}}%
\pgfpathlineto{\pgfqpoint{5.477920in}{8.205513in}}%
\pgfpathlineto{\pgfqpoint{5.480202in}{8.214684in}}%
\pgfpathlineto{\pgfqpoint{5.487047in}{8.209232in}}%
\pgfpathlineto{\pgfqpoint{5.489329in}{8.223854in}}%
\pgfpathlineto{\pgfqpoint{5.491611in}{8.224873in}}%
\pgfpathlineto{\pgfqpoint{5.493893in}{8.220390in}}%
\pgfpathlineto{\pgfqpoint{5.496175in}{8.231496in}}%
\pgfpathlineto{\pgfqpoint{5.503021in}{8.245915in}}%
\pgfpathlineto{\pgfqpoint{5.505303in}{8.243469in}}%
\pgfpathlineto{\pgfqpoint{5.509867in}{8.260486in}}%
\pgfpathlineto{\pgfqpoint{5.512148in}{8.262422in}}%
\pgfpathlineto{\pgfqpoint{5.518994in}{8.263186in}}%
\pgfpathlineto{\pgfqpoint{5.523558in}{8.255136in}}%
\pgfpathlineto{\pgfqpoint{5.525840in}{8.259161in}}%
\pgfpathlineto{\pgfqpoint{5.528122in}{8.256920in}}%
\pgfpathlineto{\pgfqpoint{5.534968in}{8.253659in}}%
\pgfpathlineto{\pgfqpoint{5.537249in}{8.258805in}}%
\pgfpathlineto{\pgfqpoint{5.539531in}{8.262065in}}%
\pgfpathlineto{\pgfqpoint{5.541813in}{8.290647in}}%
\pgfpathlineto{\pgfqpoint{5.544095in}{8.289679in}}%
\pgfpathlineto{\pgfqpoint{5.553223in}{8.294010in}}%
\pgfpathlineto{\pgfqpoint{5.555505in}{8.300226in}}%
\pgfpathlineto{\pgfqpoint{5.560068in}{8.296201in}}%
\pgfpathlineto{\pgfqpoint{5.566914in}{8.308734in}}%
\pgfpathlineto{\pgfqpoint{5.569196in}{8.303334in}}%
\pgfpathlineto{\pgfqpoint{5.576042in}{8.305677in}}%
\pgfpathlineto{\pgfqpoint{5.582888in}{8.297424in}}%
\pgfpathlineto{\pgfqpoint{5.585169in}{8.297984in}}%
\pgfpathlineto{\pgfqpoint{5.587451in}{8.304709in}}%
\pgfpathlineto{\pgfqpoint{5.589733in}{8.292023in}}%
\pgfpathlineto{\pgfqpoint{5.592015in}{8.289017in}}%
\pgfpathlineto{\pgfqpoint{5.598861in}{8.300837in}}%
\pgfpathlineto{\pgfqpoint{5.601143in}{8.295386in}}%
\pgfpathlineto{\pgfqpoint{5.603425in}{8.301805in}}%
\pgfpathlineto{\pgfqpoint{5.605707in}{8.306441in}}%
\pgfpathlineto{\pgfqpoint{5.607989in}{8.309091in}}%
\pgfpathlineto{\pgfqpoint{5.614834in}{8.307715in}}%
\pgfpathlineto{\pgfqpoint{5.617116in}{8.303486in}}%
\pgfpathlineto{\pgfqpoint{5.619398in}{8.302926in}}%
\pgfpathlineto{\pgfqpoint{5.623962in}{8.304607in}}%
\pgfpathlineto{\pgfqpoint{5.630808in}{8.298086in}}%
\pgfpathlineto{\pgfqpoint{5.633090in}{8.299563in}}%
\pgfpathlineto{\pgfqpoint{5.637653in}{8.287845in}}%
\pgfpathlineto{\pgfqpoint{5.639935in}{8.305677in}}%
\pgfpathlineto{\pgfqpoint{5.646781in}{8.303486in}}%
\pgfpathlineto{\pgfqpoint{5.651345in}{8.295589in}}%
\pgfpathlineto{\pgfqpoint{5.653627in}{8.302416in}}%
\pgfpathlineto{\pgfqpoint{5.655909in}{8.289934in}}%
\pgfpathlineto{\pgfqpoint{5.662754in}{8.302926in}}%
\pgfpathlineto{\pgfqpoint{5.665036in}{8.258754in}}%
\pgfpathlineto{\pgfqpoint{5.667318in}{8.267211in}}%
\pgfpathlineto{\pgfqpoint{5.669600in}{8.262116in}}%
\pgfpathlineto{\pgfqpoint{5.671882in}{8.253149in}}%
\pgfpathlineto{\pgfqpoint{5.678728in}{8.255748in}}%
\pgfpathlineto{\pgfqpoint{5.681010in}{8.262677in}}%
\pgfpathlineto{\pgfqpoint{5.683291in}{8.272866in}}%
\pgfpathlineto{\pgfqpoint{5.694701in}{8.273121in}}%
\pgfpathlineto{\pgfqpoint{5.696983in}{8.280916in}}%
\pgfpathlineto{\pgfqpoint{5.699265in}{8.276331in}}%
\pgfpathlineto{\pgfqpoint{5.701547in}{8.270421in}}%
\pgfpathlineto{\pgfqpoint{5.710674in}{8.252691in}}%
\pgfpathlineto{\pgfqpoint{5.712956in}{8.255646in}}%
\pgfpathlineto{\pgfqpoint{5.717520in}{8.225485in}}%
\pgfpathlineto{\pgfqpoint{5.719802in}{8.223243in}}%
\pgfpathlineto{\pgfqpoint{5.726648in}{8.223090in}}%
\pgfpathlineto{\pgfqpoint{5.728930in}{8.224975in}}%
\pgfpathlineto{\pgfqpoint{5.731212in}{8.214989in}}%
\pgfpathlineto{\pgfqpoint{5.733493in}{8.226453in}}%
\pgfpathlineto{\pgfqpoint{5.735775in}{8.214887in}}%
\pgfpathlineto{\pgfqpoint{5.744903in}{8.213461in}}%
\pgfpathlineto{\pgfqpoint{5.747185in}{8.206838in}}%
\pgfpathlineto{\pgfqpoint{5.749467in}{8.210557in}}%
\pgfpathlineto{\pgfqpoint{5.751749in}{8.218708in}}%
\pgfpathlineto{\pgfqpoint{5.758594in}{8.209640in}}%
\pgfpathlineto{\pgfqpoint{5.760876in}{8.243164in}}%
\pgfpathlineto{\pgfqpoint{5.763158in}{8.246781in}}%
\pgfpathlineto{\pgfqpoint{5.765440in}{8.254984in}}%
\pgfpathlineto{\pgfqpoint{5.767722in}{8.272102in}}%
\pgfpathlineto{\pgfqpoint{5.776850in}{8.257582in}}%
\pgfpathlineto{\pgfqpoint{5.779132in}{8.279184in}}%
\pgfpathlineto{\pgfqpoint{5.781413in}{8.283464in}}%
\pgfpathlineto{\pgfqpoint{5.783695in}{8.283617in}}%
\pgfpathlineto{\pgfqpoint{5.790541in}{8.285604in}}%
\pgfpathlineto{\pgfqpoint{5.792823in}{8.289527in}}%
\pgfpathlineto{\pgfqpoint{5.795105in}{8.283973in}}%
\pgfpathlineto{\pgfqpoint{5.797387in}{8.275618in}}%
\pgfpathlineto{\pgfqpoint{5.799669in}{8.291055in}}%
\pgfpathlineto{\pgfqpoint{5.808796in}{8.298392in}}%
\pgfpathlineto{\pgfqpoint{5.811078in}{8.303537in}}%
\pgfpathlineto{\pgfqpoint{5.813360in}{8.304251in}}%
\pgfpathlineto{\pgfqpoint{5.815642in}{8.302722in}}%
\pgfpathlineto{\pgfqpoint{5.822488in}{8.308581in}}%
\pgfpathlineto{\pgfqpoint{5.824770in}{8.301397in}}%
\pgfpathlineto{\pgfqpoint{5.827052in}{8.306339in}}%
\pgfpathlineto{\pgfqpoint{5.829334in}{8.314593in}}%
\pgfpathlineto{\pgfqpoint{5.831615in}{8.311230in}}%
\pgfpathlineto{\pgfqpoint{5.838461in}{8.304658in}}%
\pgfpathlineto{\pgfqpoint{5.840743in}{8.317446in}}%
\pgfpathlineto{\pgfqpoint{5.843025in}{8.316580in}}%
\pgfpathlineto{\pgfqpoint{5.845307in}{8.316427in}}%
\pgfpathlineto{\pgfqpoint{5.847589in}{8.319688in}}%
\pgfpathlineto{\pgfqpoint{5.854434in}{8.322032in}}%
\pgfpathlineto{\pgfqpoint{5.856716in}{8.320299in}}%
\pgfpathlineto{\pgfqpoint{5.861280in}{8.318822in}}%
\pgfpathlineto{\pgfqpoint{5.863562in}{8.328145in}}%
\pgfpathlineto{\pgfqpoint{5.870408in}{8.327840in}}%
\pgfpathlineto{\pgfqpoint{5.877254in}{8.336908in}}%
\pgfpathlineto{\pgfqpoint{5.879535in}{8.344296in}}%
\pgfpathlineto{\pgfqpoint{5.886381in}{8.342003in}}%
\pgfpathlineto{\pgfqpoint{5.888663in}{8.342360in}}%
\pgfpathlineto{\pgfqpoint{5.890945in}{8.339354in}}%
\pgfpathlineto{\pgfqpoint{5.893227in}{8.340220in}}%
\pgfpathlineto{\pgfqpoint{5.902355in}{8.348779in}}%
\pgfpathlineto{\pgfqpoint{5.904636in}{8.339914in}}%
\pgfpathlineto{\pgfqpoint{5.906918in}{8.350970in}}%
\pgfpathlineto{\pgfqpoint{5.909200in}{8.350410in}}%
\pgfpathlineto{\pgfqpoint{5.911482in}{8.354638in}}%
\pgfpathlineto{\pgfqpoint{5.918328in}{8.349340in}}%
\pgfpathlineto{\pgfqpoint{5.920610in}{8.346538in}}%
\pgfpathlineto{\pgfqpoint{5.922892in}{8.351225in}}%
\pgfpathlineto{\pgfqpoint{5.925174in}{8.352906in}}%
\pgfpathlineto{\pgfqpoint{5.927455in}{8.350410in}}%
\pgfpathlineto{\pgfqpoint{5.934301in}{8.350206in}}%
\pgfpathlineto{\pgfqpoint{5.936583in}{8.356982in}}%
\pgfpathlineto{\pgfqpoint{5.938865in}{8.359580in}}%
\pgfpathlineto{\pgfqpoint{5.941147in}{8.357542in}}%
\pgfpathlineto{\pgfqpoint{5.943429in}{8.360447in}}%
\pgfpathlineto{\pgfqpoint{5.950275in}{8.364064in}}%
\pgfpathlineto{\pgfqpoint{5.952556in}{8.364522in}}%
\pgfpathlineto{\pgfqpoint{5.954838in}{8.361211in}}%
\pgfpathlineto{\pgfqpoint{5.957120in}{8.359886in}}%
\pgfpathlineto{\pgfqpoint{5.959402in}{8.359886in}}%
\pgfpathlineto{\pgfqpoint{5.966248in}{8.358612in}}%
\pgfpathlineto{\pgfqpoint{5.968530in}{8.348270in}}%
\pgfpathlineto{\pgfqpoint{5.970812in}{8.355708in}}%
\pgfpathlineto{\pgfqpoint{5.973094in}{8.352142in}}%
\pgfpathlineto{\pgfqpoint{5.975376in}{8.353925in}}%
\pgfpathlineto{\pgfqpoint{5.982221in}{8.358409in}}%
\pgfpathlineto{\pgfqpoint{5.984503in}{8.356676in}}%
\pgfpathlineto{\pgfqpoint{5.986785in}{8.352906in}}%
\pgfpathlineto{\pgfqpoint{5.989067in}{8.355759in}}%
\pgfpathlineto{\pgfqpoint{5.991349in}{8.361007in}}%
\pgfpathlineto{\pgfqpoint{5.998195in}{8.359275in}}%
\pgfpathlineto{\pgfqpoint{6.000477in}{8.367426in}}%
\pgfpathlineto{\pgfqpoint{6.002758in}{8.365083in}}%
\pgfpathlineto{\pgfqpoint{6.005040in}{8.366866in}}%
\pgfpathlineto{\pgfqpoint{6.007322in}{8.358307in}}%
\pgfpathlineto{\pgfqpoint{6.014168in}{8.364064in}}%
\pgfpathlineto{\pgfqpoint{6.016450in}{8.355912in}}%
\pgfpathlineto{\pgfqpoint{6.018732in}{8.356473in}}%
\pgfpathlineto{\pgfqpoint{6.021014in}{8.348423in}}%
\pgfpathlineto{\pgfqpoint{6.023296in}{8.347964in}}%
\pgfpathlineto{\pgfqpoint{6.030141in}{8.353161in}}%
\pgfpathlineto{\pgfqpoint{6.032423in}{8.365032in}}%
\pgfpathlineto{\pgfqpoint{6.034705in}{8.371451in}}%
\pgfpathlineto{\pgfqpoint{6.036987in}{8.366357in}}%
\pgfpathlineto{\pgfqpoint{6.039269in}{8.366255in}}%
\pgfpathlineto{\pgfqpoint{6.048397in}{8.363554in}}%
\pgfpathlineto{\pgfqpoint{6.050678in}{8.365287in}}%
\pgfpathlineto{\pgfqpoint{6.052960in}{8.362230in}}%
\pgfpathlineto{\pgfqpoint{6.055242in}{8.363860in}}%
\pgfpathlineto{\pgfqpoint{6.062088in}{8.371859in}}%
\pgfpathlineto{\pgfqpoint{6.064370in}{8.375782in}}%
\pgfpathlineto{\pgfqpoint{6.066652in}{8.378176in}}%
\pgfpathlineto{\pgfqpoint{6.071216in}{8.364675in}}%
\pgfpathlineto{\pgfqpoint{6.078061in}{8.357950in}}%
\pgfpathlineto{\pgfqpoint{6.080343in}{8.360243in}}%
\pgfpathlineto{\pgfqpoint{6.082625in}{8.361109in}}%
\pgfpathlineto{\pgfqpoint{6.084907in}{8.371146in}}%
\pgfpathlineto{\pgfqpoint{6.087189in}{8.366509in}}%
\pgfpathlineto{\pgfqpoint{6.094035in}{8.377310in}}%
\pgfpathlineto{\pgfqpoint{6.096317in}{8.377973in}}%
\pgfpathlineto{\pgfqpoint{6.098599in}{8.377310in}}%
\pgfpathlineto{\pgfqpoint{6.100880in}{8.390863in}}%
\pgfpathlineto{\pgfqpoint{6.103162in}{8.367325in}}%
\pgfpathlineto{\pgfqpoint{6.110008in}{8.358256in}}%
\pgfpathlineto{\pgfqpoint{6.114572in}{8.378635in}}%
\pgfpathlineto{\pgfqpoint{6.116854in}{8.395652in}}%
\pgfpathlineto{\pgfqpoint{6.119136in}{8.397588in}}%
\pgfpathlineto{\pgfqpoint{6.128263in}{8.398046in}}%
\pgfpathlineto{\pgfqpoint{6.130545in}{8.395805in}}%
\pgfpathlineto{\pgfqpoint{6.132827in}{8.394429in}}%
\pgfpathlineto{\pgfqpoint{6.135109in}{8.405077in}}%
\pgfpathlineto{\pgfqpoint{6.141955in}{8.409255in}}%
\pgfpathlineto{\pgfqpoint{6.144237in}{8.414655in}}%
\pgfpathlineto{\pgfqpoint{6.146519in}{8.414910in}}%
\pgfpathlineto{\pgfqpoint{6.148800in}{8.422909in}}%
\pgfpathlineto{\pgfqpoint{6.151082in}{8.425253in}}%
\pgfpathlineto{\pgfqpoint{6.157928in}{8.423877in}}%
\pgfpathlineto{\pgfqpoint{6.162492in}{8.425355in}}%
\pgfpathlineto{\pgfqpoint{6.164774in}{8.419852in}}%
\pgfpathlineto{\pgfqpoint{6.167056in}{8.420718in}}%
\pgfpathlineto{\pgfqpoint{6.173901in}{8.416897in}}%
\pgfpathlineto{\pgfqpoint{6.176183in}{8.407625in}}%
\pgfpathlineto{\pgfqpoint{6.178465in}{8.410478in}}%
\pgfpathlineto{\pgfqpoint{6.180747in}{8.409102in}}%
\pgfpathlineto{\pgfqpoint{6.183029in}{8.410885in}}%
\pgfpathlineto{\pgfqpoint{6.192157in}{8.410885in}}%
\pgfpathlineto{\pgfqpoint{6.194439in}{8.410987in}}%
\pgfpathlineto{\pgfqpoint{6.196721in}{8.408338in}}%
\pgfpathlineto{\pgfqpoint{6.199002in}{8.411904in}}%
\pgfpathlineto{\pgfqpoint{6.205848in}{8.411955in}}%
\pgfpathlineto{\pgfqpoint{6.208130in}{8.411038in}}%
\pgfpathlineto{\pgfqpoint{6.210412in}{8.413076in}}%
\pgfpathlineto{\pgfqpoint{6.212694in}{8.423368in}}%
\pgfpathlineto{\pgfqpoint{6.214976in}{8.419903in}}%
\pgfpathlineto{\pgfqpoint{6.221821in}{8.421279in}}%
\pgfpathlineto{\pgfqpoint{6.224103in}{8.415114in}}%
\pgfpathlineto{\pgfqpoint{6.226385in}{8.423317in}}%
\pgfpathlineto{\pgfqpoint{6.228667in}{8.420056in}}%
\pgfpathlineto{\pgfqpoint{6.230949in}{8.422094in}}%
\pgfpathlineto{\pgfqpoint{6.237795in}{8.419496in}}%
\pgfpathlineto{\pgfqpoint{6.240077in}{8.422858in}}%
\pgfpathlineto{\pgfqpoint{6.242359in}{8.421126in}}%
\pgfpathlineto{\pgfqpoint{6.244641in}{8.421941in}}%
\pgfpathlineto{\pgfqpoint{6.246922in}{8.421126in}}%
\pgfpathlineto{\pgfqpoint{6.253768in}{8.426323in}}%
\pgfpathlineto{\pgfqpoint{6.256050in}{8.424641in}}%
\pgfpathlineto{\pgfqpoint{6.258332in}{8.420311in}}%
\pgfpathlineto{\pgfqpoint{6.262896in}{8.427749in}}%
\pgfpathlineto{\pgfqpoint{6.272023in}{8.426119in}}%
\pgfpathlineto{\pgfqpoint{6.274305in}{8.422756in}}%
\pgfpathlineto{\pgfqpoint{6.276587in}{8.424387in}}%
\pgfpathlineto{\pgfqpoint{6.278869in}{8.403243in}}%
\pgfpathlineto{\pgfqpoint{6.285715in}{8.414910in}}%
\pgfpathlineto{\pgfqpoint{6.287997in}{8.405281in}}%
\pgfpathlineto{\pgfqpoint{6.290279in}{8.403141in}}%
\pgfpathlineto{\pgfqpoint{6.292561in}{8.407726in}}%
\pgfpathlineto{\pgfqpoint{6.294843in}{8.400492in}}%
\pgfpathlineto{\pgfqpoint{6.301688in}{8.408389in}}%
\pgfpathlineto{\pgfqpoint{6.303970in}{8.412567in}}%
\pgfpathlineto{\pgfqpoint{6.306252in}{8.421992in}}%
\pgfpathlineto{\pgfqpoint{6.308534in}{8.423215in}}%
\pgfpathlineto{\pgfqpoint{6.310816in}{8.411548in}}%
\pgfpathlineto{\pgfqpoint{6.317662in}{8.404721in}}%
\pgfpathlineto{\pgfqpoint{6.319943in}{8.406351in}}%
\pgfpathlineto{\pgfqpoint{6.322225in}{8.412516in}}%
\pgfpathlineto{\pgfqpoint{6.324507in}{8.401969in}}%
\pgfpathlineto{\pgfqpoint{6.326789in}{8.406045in}}%
\pgfpathlineto{\pgfqpoint{6.333635in}{8.400390in}}%
\pgfpathlineto{\pgfqpoint{6.335917in}{8.384443in}}%
\pgfpathlineto{\pgfqpoint{6.338199in}{8.387806in}}%
\pgfpathlineto{\pgfqpoint{6.340481in}{8.384239in}}%
\pgfpathlineto{\pgfqpoint{6.342763in}{8.382813in}}%
\pgfpathlineto{\pgfqpoint{6.349608in}{8.381896in}}%
\pgfpathlineto{\pgfqpoint{6.351890in}{8.374967in}}%
\pgfpathlineto{\pgfqpoint{6.354172in}{8.375171in}}%
\pgfpathlineto{\pgfqpoint{6.358736in}{8.378024in}}%
\pgfpathlineto{\pgfqpoint{6.365582in}{8.377463in}}%
\pgfpathlineto{\pgfqpoint{6.367864in}{8.376240in}}%
\pgfpathlineto{\pgfqpoint{6.372427in}{8.375833in}}%
\pgfpathlineto{\pgfqpoint{6.374709in}{8.374101in}}%
\pgfpathlineto{\pgfqpoint{6.381555in}{8.382507in}}%
\pgfpathlineto{\pgfqpoint{6.383837in}{8.358612in}}%
\pgfpathlineto{\pgfqpoint{6.386119in}{8.359937in}}%
\pgfpathlineto{\pgfqpoint{6.388401in}{8.356371in}}%
\pgfpathlineto{\pgfqpoint{6.390683in}{8.356422in}}%
\pgfpathlineto{\pgfqpoint{6.397528in}{8.354180in}}%
\pgfpathlineto{\pgfqpoint{6.399810in}{8.349187in}}%
\pgfpathlineto{\pgfqpoint{6.404374in}{8.361465in}}%
\pgfpathlineto{\pgfqpoint{6.406656in}{8.359784in}}%
\pgfpathlineto{\pgfqpoint{6.415784in}{8.381386in}}%
\pgfpathlineto{\pgfqpoint{6.418065in}{8.378329in}}%
\pgfpathlineto{\pgfqpoint{6.420347in}{8.396773in}}%
\pgfpathlineto{\pgfqpoint{6.422629in}{8.400594in}}%
\pgfpathlineto{\pgfqpoint{6.429475in}{8.390302in}}%
\pgfpathlineto{\pgfqpoint{6.431757in}{8.396620in}}%
\pgfpathlineto{\pgfqpoint{6.434039in}{8.391270in}}%
\pgfpathlineto{\pgfqpoint{6.436321in}{8.394888in}}%
\pgfpathlineto{\pgfqpoint{6.438603in}{8.395805in}}%
\pgfpathlineto{\pgfqpoint{6.445448in}{8.388927in}}%
\pgfpathlineto{\pgfqpoint{6.450012in}{8.392391in}}%
\pgfpathlineto{\pgfqpoint{6.454576in}{8.398301in}}%
\pgfpathlineto{\pgfqpoint{6.461422in}{8.393614in}}%
\pgfpathlineto{\pgfqpoint{6.463704in}{8.394888in}}%
\pgfpathlineto{\pgfqpoint{6.465986in}{8.389946in}}%
\pgfpathlineto{\pgfqpoint{6.468267in}{8.394225in}}%
\pgfpathlineto{\pgfqpoint{6.470549in}{8.393257in}}%
\pgfpathlineto{\pgfqpoint{6.477395in}{8.389385in}}%
\pgfpathlineto{\pgfqpoint{6.479677in}{8.390251in}}%
\pgfpathlineto{\pgfqpoint{6.481959in}{8.410529in}}%
\pgfpathlineto{\pgfqpoint{6.484241in}{8.409713in}}%
\pgfpathlineto{\pgfqpoint{6.486523in}{8.422196in}}%
\pgfpathlineto{\pgfqpoint{6.493368in}{8.427698in}}%
\pgfpathlineto{\pgfqpoint{6.495650in}{8.423826in}}%
\pgfpathlineto{\pgfqpoint{6.497932in}{8.413178in}}%
\pgfpathlineto{\pgfqpoint{6.500214in}{8.410376in}}%
\pgfpathlineto{\pgfqpoint{6.502496in}{8.417254in}}%
\pgfpathlineto{\pgfqpoint{6.509342in}{8.420616in}}%
\pgfpathlineto{\pgfqpoint{6.511624in}{8.422960in}}%
\pgfpathlineto{\pgfqpoint{6.513906in}{8.421890in}}%
\pgfpathlineto{\pgfqpoint{6.516187in}{8.425609in}}%
\pgfpathlineto{\pgfqpoint{6.518469in}{8.423419in}}%
\pgfpathlineto{\pgfqpoint{6.527597in}{8.424234in}}%
\pgfpathlineto{\pgfqpoint{6.529879in}{8.420260in}}%
\pgfpathlineto{\pgfqpoint{6.532161in}{8.421788in}}%
\pgfpathlineto{\pgfqpoint{6.534443in}{8.422552in}}%
\pgfpathlineto{\pgfqpoint{6.543570in}{8.420107in}}%
\pgfpathlineto{\pgfqpoint{6.545852in}{8.421381in}}%
\pgfpathlineto{\pgfqpoint{6.548134in}{8.418477in}}%
\pgfpathlineto{\pgfqpoint{6.550416in}{8.420973in}}%
\pgfpathlineto{\pgfqpoint{6.557262in}{8.416388in}}%
\pgfpathlineto{\pgfqpoint{6.559544in}{8.413076in}}%
\pgfpathlineto{\pgfqpoint{6.561826in}{8.419343in}}%
\pgfpathlineto{\pgfqpoint{6.564108in}{8.417203in}}%
\pgfpathlineto{\pgfqpoint{6.575517in}{8.416337in}}%
\pgfpathlineto{\pgfqpoint{6.577799in}{8.422196in}}%
\pgfpathlineto{\pgfqpoint{6.580081in}{8.423113in}}%
\pgfpathlineto{\pgfqpoint{6.582363in}{8.422196in}}%
\pgfpathlineto{\pgfqpoint{6.589208in}{8.422298in}}%
\pgfpathlineto{\pgfqpoint{6.591490in}{8.410172in}}%
\pgfpathlineto{\pgfqpoint{6.593772in}{8.413789in}}%
\pgfpathlineto{\pgfqpoint{6.596054in}{8.414197in}}%
\pgfpathlineto{\pgfqpoint{6.598336in}{8.417356in}}%
\pgfpathlineto{\pgfqpoint{6.607464in}{8.404670in}}%
\pgfpathlineto{\pgfqpoint{6.609746in}{8.406351in}}%
\pgfpathlineto{\pgfqpoint{6.612028in}{8.401613in}}%
\pgfpathlineto{\pgfqpoint{6.614309in}{8.405740in}}%
\pgfpathlineto{\pgfqpoint{6.621155in}{8.405994in}}%
\pgfpathlineto{\pgfqpoint{6.623437in}{8.409153in}}%
\pgfpathlineto{\pgfqpoint{6.628001in}{8.420718in}}%
\pgfpathlineto{\pgfqpoint{6.630283in}{8.424641in}}%
\pgfpathlineto{\pgfqpoint{6.639410in}{8.437022in}}%
\pgfpathlineto{\pgfqpoint{6.641692in}{8.443186in}}%
\pgfpathlineto{\pgfqpoint{6.643974in}{8.451389in}}%
\pgfpathlineto{\pgfqpoint{6.646256in}{8.449198in}}%
\pgfpathlineto{\pgfqpoint{6.655384in}{8.451185in}}%
\pgfpathlineto{\pgfqpoint{6.657666in}{8.464024in}}%
\pgfpathlineto{\pgfqpoint{6.659948in}{8.469578in}}%
\pgfpathlineto{\pgfqpoint{6.662230in}{8.470597in}}%
\pgfpathlineto{\pgfqpoint{6.669075in}{8.468202in}}%
\pgfpathlineto{\pgfqpoint{6.671357in}{8.465502in}}%
\pgfpathlineto{\pgfqpoint{6.673639in}{8.482417in}}%
\pgfpathlineto{\pgfqpoint{6.675921in}{8.482519in}}%
\pgfpathlineto{\pgfqpoint{6.678203in}{8.479767in}}%
\pgfpathlineto{\pgfqpoint{6.685049in}{8.477780in}}%
\pgfpathlineto{\pgfqpoint{6.687330in}{8.478697in}}%
\pgfpathlineto{\pgfqpoint{6.691894in}{8.482569in}}%
\pgfpathlineto{\pgfqpoint{6.694176in}{8.488887in}}%
\pgfpathlineto{\pgfqpoint{6.701022in}{8.490365in}}%
\pgfpathlineto{\pgfqpoint{6.703304in}{8.484658in}}%
\pgfpathlineto{\pgfqpoint{6.705586in}{8.488836in}}%
\pgfpathlineto{\pgfqpoint{6.707868in}{8.484556in}}%
\pgfpathlineto{\pgfqpoint{6.710150in}{8.494389in}}%
\pgfpathlineto{\pgfqpoint{6.716995in}{8.497548in}}%
\pgfpathlineto{\pgfqpoint{6.719277in}{8.493320in}}%
\pgfpathlineto{\pgfqpoint{6.721559in}{8.493727in}}%
\pgfpathlineto{\pgfqpoint{6.723841in}{8.493370in}}%
\pgfpathlineto{\pgfqpoint{6.726123in}{8.490314in}}%
\pgfpathlineto{\pgfqpoint{6.732969in}{8.485117in}}%
\pgfpathlineto{\pgfqpoint{6.735251in}{8.487817in}}%
\pgfpathlineto{\pgfqpoint{6.737532in}{8.486340in}}%
\pgfpathlineto{\pgfqpoint{6.739814in}{8.489193in}}%
\pgfpathlineto{\pgfqpoint{6.742096in}{8.489447in}}%
\pgfpathlineto{\pgfqpoint{6.748942in}{8.486543in}}%
\pgfpathlineto{\pgfqpoint{6.751224in}{8.483996in}}%
\pgfpathlineto{\pgfqpoint{6.753506in}{8.484200in}}%
\pgfpathlineto{\pgfqpoint{6.755788in}{8.482519in}}%
\pgfpathlineto{\pgfqpoint{6.758070in}{8.483028in}}%
\pgfpathlineto{\pgfqpoint{6.764915in}{8.481652in}}%
\pgfpathlineto{\pgfqpoint{6.767197in}{8.483385in}}%
\pgfpathlineto{\pgfqpoint{6.769479in}{8.481601in}}%
\pgfpathlineto{\pgfqpoint{6.771761in}{8.476558in}}%
\pgfpathlineto{\pgfqpoint{6.780889in}{8.484811in}}%
\pgfpathlineto{\pgfqpoint{6.783171in}{8.484098in}}%
\pgfpathlineto{\pgfqpoint{6.785452in}{8.482264in}}%
\pgfpathlineto{\pgfqpoint{6.787734in}{8.488632in}}%
\pgfpathlineto{\pgfqpoint{6.790016in}{8.490263in}}%
\pgfpathlineto{\pgfqpoint{6.799144in}{8.507738in}}%
\pgfpathlineto{\pgfqpoint{6.801426in}{8.507076in}}%
\pgfpathlineto{\pgfqpoint{6.803708in}{8.512425in}}%
\pgfpathlineto{\pgfqpoint{6.805990in}{8.511100in}}%
\pgfpathlineto{\pgfqpoint{6.812835in}{8.506159in}}%
\pgfpathlineto{\pgfqpoint{6.815117in}{8.514463in}}%
\pgfpathlineto{\pgfqpoint{6.817399in}{8.519762in}}%
\pgfpathlineto{\pgfqpoint{6.819681in}{8.528780in}}%
\pgfpathlineto{\pgfqpoint{6.821963in}{8.528219in}}%
\pgfpathlineto{\pgfqpoint{6.828809in}{8.524653in}}%
\pgfpathlineto{\pgfqpoint{6.831091in}{8.521341in}}%
\pgfpathlineto{\pgfqpoint{6.833373in}{8.514973in}}%
\pgfpathlineto{\pgfqpoint{6.835654in}{8.515482in}}%
\pgfpathlineto{\pgfqpoint{6.837936in}{8.514667in}}%
\pgfpathlineto{\pgfqpoint{6.847064in}{8.521086in}}%
\pgfpathlineto{\pgfqpoint{6.849346in}{8.511916in}}%
\pgfpathlineto{\pgfqpoint{6.851628in}{8.513852in}}%
\pgfpathlineto{\pgfqpoint{6.853910in}{8.516552in}}%
\pgfpathlineto{\pgfqpoint{6.860755in}{8.529646in}}%
\pgfpathlineto{\pgfqpoint{6.863037in}{8.525875in}}%
\pgfpathlineto{\pgfqpoint{6.865319in}{8.524755in}}%
\pgfpathlineto{\pgfqpoint{6.867601in}{8.534639in}}%
\pgfpathlineto{\pgfqpoint{6.869883in}{8.540090in}}%
\pgfpathlineto{\pgfqpoint{6.879011in}{8.548649in}}%
\pgfpathlineto{\pgfqpoint{6.881293in}{8.558482in}}%
\pgfpathlineto{\pgfqpoint{6.883574in}{8.557922in}}%
\pgfpathlineto{\pgfqpoint{6.885856in}{8.569283in}}%
\pgfpathlineto{\pgfqpoint{6.892702in}{8.566940in}}%
\pgfpathlineto{\pgfqpoint{6.894984in}{8.563017in}}%
\pgfpathlineto{\pgfqpoint{6.897266in}{8.561081in}}%
\pgfpathlineto{\pgfqpoint{6.901830in}{8.570353in}}%
\pgfpathlineto{\pgfqpoint{6.908675in}{8.572595in}}%
\pgfpathlineto{\pgfqpoint{6.910957in}{8.580900in}}%
\pgfpathlineto{\pgfqpoint{6.913239in}{8.585281in}}%
\pgfpathlineto{\pgfqpoint{6.915521in}{8.591446in}}%
\pgfpathlineto{\pgfqpoint{6.917803in}{8.600871in}}%
\pgfpathlineto{\pgfqpoint{6.926931in}{8.601483in}}%
\pgfpathlineto{\pgfqpoint{6.931495in}{8.595929in}}%
\pgfpathlineto{\pgfqpoint{6.933776in}{8.599241in}}%
\pgfpathlineto{\pgfqpoint{6.940622in}{8.597814in}}%
\pgfpathlineto{\pgfqpoint{6.942904in}{8.584160in}}%
\pgfpathlineto{\pgfqpoint{6.945186in}{8.588236in}}%
\pgfpathlineto{\pgfqpoint{6.947468in}{8.574837in}}%
\pgfpathlineto{\pgfqpoint{6.949750in}{8.576467in}}%
\pgfpathlineto{\pgfqpoint{6.956595in}{8.584415in}}%
\pgfpathlineto{\pgfqpoint{6.961159in}{8.584058in}}%
\pgfpathlineto{\pgfqpoint{6.963441in}{8.575652in}}%
\pgfpathlineto{\pgfqpoint{6.965723in}{8.583243in}}%
\pgfpathlineto{\pgfqpoint{6.972569in}{8.587574in}}%
\pgfpathlineto{\pgfqpoint{6.974851in}{8.583549in}}%
\pgfpathlineto{\pgfqpoint{6.977133in}{8.591497in}}%
\pgfpathlineto{\pgfqpoint{6.979415in}{8.590478in}}%
\pgfpathlineto{\pgfqpoint{6.981696in}{8.593789in}}%
\pgfpathlineto{\pgfqpoint{6.988542in}{8.593331in}}%
\pgfpathlineto{\pgfqpoint{6.990824in}{8.591548in}}%
\pgfpathlineto{\pgfqpoint{6.993106in}{8.595369in}}%
\pgfpathlineto{\pgfqpoint{6.995388in}{8.597050in}}%
\pgfpathlineto{\pgfqpoint{6.997670in}{8.590835in}}%
\pgfpathlineto{\pgfqpoint{7.004516in}{8.585230in}}%
\pgfpathlineto{\pgfqpoint{7.006797in}{8.533925in}}%
\pgfpathlineto{\pgfqpoint{7.009079in}{8.532193in}}%
\pgfpathlineto{\pgfqpoint{7.011361in}{8.537135in}}%
\pgfpathlineto{\pgfqpoint{7.013643in}{8.535505in}}%
\pgfpathlineto{\pgfqpoint{7.020489in}{8.542536in}}%
\pgfpathlineto{\pgfqpoint{7.027335in}{8.573716in}}%
\pgfpathlineto{\pgfqpoint{7.029617in}{8.573869in}}%
\pgfpathlineto{\pgfqpoint{7.036462in}{8.572850in}}%
\pgfpathlineto{\pgfqpoint{7.038744in}{8.567959in}}%
\pgfpathlineto{\pgfqpoint{7.041026in}{8.568213in}}%
\pgfpathlineto{\pgfqpoint{7.045590in}{8.565768in}}%
\pgfpathlineto{\pgfqpoint{7.052436in}{8.572493in}}%
\pgfpathlineto{\pgfqpoint{7.054717in}{8.571576in}}%
\pgfpathlineto{\pgfqpoint{7.056999in}{8.575142in}}%
\pgfpathlineto{\pgfqpoint{7.059281in}{8.562711in}}%
\pgfpathlineto{\pgfqpoint{7.061563in}{8.553948in}}%
\pgfpathlineto{\pgfqpoint{7.068409in}{8.558890in}}%
\pgfpathlineto{\pgfqpoint{7.070691in}{8.563271in}}%
\pgfpathlineto{\pgfqpoint{7.072973in}{8.555425in}}%
\pgfpathlineto{\pgfqpoint{7.075255in}{8.552725in}}%
\pgfpathlineto{\pgfqpoint{7.077537in}{8.552776in}}%
\pgfpathlineto{\pgfqpoint{7.084382in}{8.554407in}}%
\pgfpathlineto{\pgfqpoint{7.086664in}{8.556648in}}%
\pgfpathlineto{\pgfqpoint{7.091228in}{8.563424in}}%
\pgfpathlineto{\pgfqpoint{7.093510in}{8.559756in}}%
\pgfpathlineto{\pgfqpoint{7.102638in}{8.547172in}}%
\pgfpathlineto{\pgfqpoint{7.104919in}{8.552420in}}%
\pgfpathlineto{\pgfqpoint{7.107201in}{8.564138in}}%
\pgfpathlineto{\pgfqpoint{7.116329in}{8.588898in}}%
\pgfpathlineto{\pgfqpoint{7.118611in}{8.589306in}}%
\pgfpathlineto{\pgfqpoint{7.120893in}{8.588593in}}%
\pgfpathlineto{\pgfqpoint{7.125457in}{8.607342in}}%
\pgfpathlineto{\pgfqpoint{7.132302in}{8.609329in}}%
\pgfpathlineto{\pgfqpoint{7.134584in}{8.608361in}}%
\pgfpathlineto{\pgfqpoint{7.136866in}{8.593535in}}%
\pgfpathlineto{\pgfqpoint{7.139148in}{8.593178in}}%
\pgfpathlineto{\pgfqpoint{7.141430in}{8.594758in}}%
\pgfpathlineto{\pgfqpoint{7.148276in}{8.594605in}}%
\pgfpathlineto{\pgfqpoint{7.150558in}{8.596082in}}%
\pgfpathlineto{\pgfqpoint{7.152839in}{8.588542in}}%
\pgfpathlineto{\pgfqpoint{7.155121in}{8.588848in}}%
\pgfpathlineto{\pgfqpoint{7.157403in}{8.590580in}}%
\pgfpathlineto{\pgfqpoint{7.164249in}{8.604489in}}%
\pgfpathlineto{\pgfqpoint{7.168813in}{8.622779in}}%
\pgfpathlineto{\pgfqpoint{7.171095in}{8.622015in}}%
\pgfpathlineto{\pgfqpoint{7.173377in}{8.622779in}}%
\pgfpathlineto{\pgfqpoint{7.182504in}{8.623900in}}%
\pgfpathlineto{\pgfqpoint{7.184786in}{8.622728in}}%
\pgfpathlineto{\pgfqpoint{7.187068in}{8.627976in}}%
\pgfpathlineto{\pgfqpoint{7.189350in}{8.628587in}}%
\pgfpathlineto{\pgfqpoint{7.196196in}{8.633478in}}%
\pgfpathlineto{\pgfqpoint{7.198478in}{8.628740in}}%
\pgfpathlineto{\pgfqpoint{7.200760in}{8.631287in}}%
\pgfpathlineto{\pgfqpoint{7.203041in}{8.635975in}}%
\pgfpathlineto{\pgfqpoint{7.205323in}{8.646113in}}%
\pgfpathlineto{\pgfqpoint{7.212169in}{8.647234in}}%
\pgfpathlineto{\pgfqpoint{7.214451in}{8.710920in}}%
\pgfpathlineto{\pgfqpoint{7.216733in}{8.725644in}}%
\pgfpathlineto{\pgfqpoint{7.219015in}{8.702615in}}%
\pgfpathlineto{\pgfqpoint{7.221297in}{8.711378in}}%
\pgfpathlineto{\pgfqpoint{7.228142in}{8.693291in}}%
\pgfpathlineto{\pgfqpoint{7.230424in}{8.689216in}}%
\pgfpathlineto{\pgfqpoint{7.232706in}{8.689165in}}%
\pgfpathlineto{\pgfqpoint{7.234988in}{8.699150in}}%
\pgfpathlineto{\pgfqpoint{7.237270in}{8.699100in}}%
\pgfpathlineto{\pgfqpoint{7.244116in}{8.689827in}}%
\pgfpathlineto{\pgfqpoint{7.248680in}{8.687483in}}%
\pgfpathlineto{\pgfqpoint{7.250961in}{8.680503in}}%
\pgfpathlineto{\pgfqpoint{7.253243in}{8.675918in}}%
\pgfpathlineto{\pgfqpoint{7.260089in}{8.679637in}}%
\pgfpathlineto{\pgfqpoint{7.262371in}{8.685038in}}%
\pgfpathlineto{\pgfqpoint{7.264653in}{8.675663in}}%
\pgfpathlineto{\pgfqpoint{7.266935in}{8.685496in}}%
\pgfpathlineto{\pgfqpoint{7.269217in}{8.685191in}}%
\pgfpathlineto{\pgfqpoint{7.276062in}{8.695533in}}%
\pgfpathlineto{\pgfqpoint{7.278344in}{8.708219in}}%
\pgfpathlineto{\pgfqpoint{7.280626in}{8.701698in}}%
\pgfpathlineto{\pgfqpoint{7.285190in}{8.700679in}}%
\pgfpathlineto{\pgfqpoint{7.292036in}{8.713518in}}%
\pgfpathlineto{\pgfqpoint{7.294318in}{8.721466in}}%
\pgfpathlineto{\pgfqpoint{7.296600in}{8.733082in}}%
\pgfpathlineto{\pgfqpoint{7.298882in}{8.758149in}}%
\pgfpathlineto{\pgfqpoint{7.301163in}{8.748468in}}%
\pgfpathlineto{\pgfqpoint{7.308009in}{8.739196in}}%
\pgfpathlineto{\pgfqpoint{7.310291in}{8.734305in}}%
\pgfpathlineto{\pgfqpoint{7.312573in}{8.736343in}}%
\pgfpathlineto{\pgfqpoint{7.314855in}{8.744291in}}%
\pgfpathlineto{\pgfqpoint{7.317137in}{8.733693in}}%
\pgfpathlineto{\pgfqpoint{7.323982in}{8.739400in}}%
\pgfpathlineto{\pgfqpoint{7.326264in}{8.726102in}}%
\pgfpathlineto{\pgfqpoint{7.328546in}{8.738483in}}%
\pgfpathlineto{\pgfqpoint{7.330828in}{8.733439in}}%
\pgfpathlineto{\pgfqpoint{7.333110in}{8.733031in}}%
\pgfpathlineto{\pgfqpoint{7.339956in}{8.735120in}}%
\pgfpathlineto{\pgfqpoint{7.342238in}{8.734967in}}%
\pgfpathlineto{\pgfqpoint{7.344520in}{8.727783in}}%
\pgfpathlineto{\pgfqpoint{7.346802in}{8.717339in}}%
\pgfpathlineto{\pgfqpoint{7.349083in}{8.717084in}}%
\pgfpathlineto{\pgfqpoint{7.358211in}{8.720600in}}%
\pgfpathlineto{\pgfqpoint{7.360493in}{8.724268in}}%
\pgfpathlineto{\pgfqpoint{7.365057in}{8.720192in}}%
\pgfpathlineto{\pgfqpoint{7.365057in}{8.720192in}}%
\pgfusepath{stroke}%
\end{pgfscope}%
\begin{pgfscope}%
\pgfsetrectcap%
\pgfsetmiterjoin%
\pgfsetlinewidth{0.803000pt}%
\definecolor{currentstroke}{rgb}{1.000000,1.000000,1.000000}%
\pgfsetstrokecolor{currentstroke}%
\pgfsetdash{}{0pt}%
\pgfpathmoveto{\pgfqpoint{2.125000in}{7.879268in}}%
\pgfpathlineto{\pgfqpoint{2.125000in}{8.800000in}}%
\pgfusepath{stroke}%
\end{pgfscope}%
\begin{pgfscope}%
\pgfsetrectcap%
\pgfsetmiterjoin%
\pgfsetlinewidth{0.803000pt}%
\definecolor{currentstroke}{rgb}{1.000000,1.000000,1.000000}%
\pgfsetstrokecolor{currentstroke}%
\pgfsetdash{}{0pt}%
\pgfpathmoveto{\pgfqpoint{7.614583in}{7.879268in}}%
\pgfpathlineto{\pgfqpoint{7.614583in}{8.800000in}}%
\pgfusepath{stroke}%
\end{pgfscope}%
\begin{pgfscope}%
\pgfsetrectcap%
\pgfsetmiterjoin%
\pgfsetlinewidth{0.803000pt}%
\definecolor{currentstroke}{rgb}{1.000000,1.000000,1.000000}%
\pgfsetstrokecolor{currentstroke}%
\pgfsetdash{}{0pt}%
\pgfpathmoveto{\pgfqpoint{2.125000in}{7.879268in}}%
\pgfpathlineto{\pgfqpoint{7.614583in}{7.879268in}}%
\pgfusepath{stroke}%
\end{pgfscope}%
\begin{pgfscope}%
\pgfsetrectcap%
\pgfsetmiterjoin%
\pgfsetlinewidth{0.803000pt}%
\definecolor{currentstroke}{rgb}{1.000000,1.000000,1.000000}%
\pgfsetstrokecolor{currentstroke}%
\pgfsetdash{}{0pt}%
\pgfpathmoveto{\pgfqpoint{2.125000in}{8.800000in}}%
\pgfpathlineto{\pgfqpoint{7.614583in}{8.800000in}}%
\pgfusepath{stroke}%
\end{pgfscope}%
\begin{pgfscope}%
\definecolor{textcolor}{rgb}{0.150000,0.150000,0.150000}%
\pgfsetstrokecolor{textcolor}%
\pgfsetfillcolor{textcolor}%
\pgftext[x=4.869792in,y=8.883333in,,base]{\color{textcolor}\rmfamily\fontsize{12.000000}{14.400000}\selectfont MMM}%
\end{pgfscope}%
\begin{pgfscope}%
\pgfsetbuttcap%
\pgfsetmiterjoin%
\definecolor{currentfill}{rgb}{0.917647,0.917647,0.949020}%
\pgfsetfillcolor{currentfill}%
\pgfsetlinewidth{0.000000pt}%
\definecolor{currentstroke}{rgb}{0.000000,0.000000,0.000000}%
\pgfsetstrokecolor{currentstroke}%
\pgfsetstrokeopacity{0.000000}%
\pgfsetdash{}{0pt}%
\pgfpathmoveto{\pgfqpoint{9.810417in}{7.879268in}}%
\pgfpathlineto{\pgfqpoint{15.300000in}{7.879268in}}%
\pgfpathlineto{\pgfqpoint{15.300000in}{8.800000in}}%
\pgfpathlineto{\pgfqpoint{9.810417in}{8.800000in}}%
\pgfpathclose%
\pgfusepath{fill}%
\end{pgfscope}%
\begin{pgfscope}%
\pgfpathrectangle{\pgfqpoint{9.810417in}{7.879268in}}{\pgfqpoint{5.489583in}{0.920732in}}%
\pgfusepath{clip}%
\pgfsetroundcap%
\pgfsetroundjoin%
\pgfsetlinewidth{0.803000pt}%
\definecolor{currentstroke}{rgb}{1.000000,1.000000,1.000000}%
\pgfsetstrokecolor{currentstroke}%
\pgfsetdash{}{0pt}%
\pgfpathmoveto{\pgfqpoint{10.055379in}{7.879268in}}%
\pgfpathlineto{\pgfqpoint{10.055379in}{8.800000in}}%
\pgfusepath{stroke}%
\end{pgfscope}%
\begin{pgfscope}%
\definecolor{textcolor}{rgb}{0.150000,0.150000,0.150000}%
\pgfsetstrokecolor{textcolor}%
\pgfsetfillcolor{textcolor}%
\pgftext[x=10.055379in,y=7.782046in,,top]{\color{textcolor}\rmfamily\fontsize{10.000000}{12.000000}\selectfont 2012}%
\end{pgfscope}%
\begin{pgfscope}%
\pgfpathrectangle{\pgfqpoint{9.810417in}{7.879268in}}{\pgfqpoint{5.489583in}{0.920732in}}%
\pgfusepath{clip}%
\pgfsetroundcap%
\pgfsetroundjoin%
\pgfsetlinewidth{0.803000pt}%
\definecolor{currentstroke}{rgb}{1.000000,1.000000,1.000000}%
\pgfsetstrokecolor{currentstroke}%
\pgfsetdash{}{0pt}%
\pgfpathmoveto{\pgfqpoint{10.890557in}{7.879268in}}%
\pgfpathlineto{\pgfqpoint{10.890557in}{8.800000in}}%
\pgfusepath{stroke}%
\end{pgfscope}%
\begin{pgfscope}%
\definecolor{textcolor}{rgb}{0.150000,0.150000,0.150000}%
\pgfsetstrokecolor{textcolor}%
\pgfsetfillcolor{textcolor}%
\pgftext[x=10.890557in,y=7.782046in,,top]{\color{textcolor}\rmfamily\fontsize{10.000000}{12.000000}\selectfont 2013}%
\end{pgfscope}%
\begin{pgfscope}%
\pgfpathrectangle{\pgfqpoint{9.810417in}{7.879268in}}{\pgfqpoint{5.489583in}{0.920732in}}%
\pgfusepath{clip}%
\pgfsetroundcap%
\pgfsetroundjoin%
\pgfsetlinewidth{0.803000pt}%
\definecolor{currentstroke}{rgb}{1.000000,1.000000,1.000000}%
\pgfsetstrokecolor{currentstroke}%
\pgfsetdash{}{0pt}%
\pgfpathmoveto{\pgfqpoint{11.723453in}{7.879268in}}%
\pgfpathlineto{\pgfqpoint{11.723453in}{8.800000in}}%
\pgfusepath{stroke}%
\end{pgfscope}%
\begin{pgfscope}%
\definecolor{textcolor}{rgb}{0.150000,0.150000,0.150000}%
\pgfsetstrokecolor{textcolor}%
\pgfsetfillcolor{textcolor}%
\pgftext[x=11.723453in,y=7.782046in,,top]{\color{textcolor}\rmfamily\fontsize{10.000000}{12.000000}\selectfont 2014}%
\end{pgfscope}%
\begin{pgfscope}%
\pgfpathrectangle{\pgfqpoint{9.810417in}{7.879268in}}{\pgfqpoint{5.489583in}{0.920732in}}%
\pgfusepath{clip}%
\pgfsetroundcap%
\pgfsetroundjoin%
\pgfsetlinewidth{0.803000pt}%
\definecolor{currentstroke}{rgb}{1.000000,1.000000,1.000000}%
\pgfsetstrokecolor{currentstroke}%
\pgfsetdash{}{0pt}%
\pgfpathmoveto{\pgfqpoint{12.556349in}{7.879268in}}%
\pgfpathlineto{\pgfqpoint{12.556349in}{8.800000in}}%
\pgfusepath{stroke}%
\end{pgfscope}%
\begin{pgfscope}%
\definecolor{textcolor}{rgb}{0.150000,0.150000,0.150000}%
\pgfsetstrokecolor{textcolor}%
\pgfsetfillcolor{textcolor}%
\pgftext[x=12.556349in,y=7.782046in,,top]{\color{textcolor}\rmfamily\fontsize{10.000000}{12.000000}\selectfont 2015}%
\end{pgfscope}%
\begin{pgfscope}%
\pgfpathrectangle{\pgfqpoint{9.810417in}{7.879268in}}{\pgfqpoint{5.489583in}{0.920732in}}%
\pgfusepath{clip}%
\pgfsetroundcap%
\pgfsetroundjoin%
\pgfsetlinewidth{0.803000pt}%
\definecolor{currentstroke}{rgb}{1.000000,1.000000,1.000000}%
\pgfsetstrokecolor{currentstroke}%
\pgfsetdash{}{0pt}%
\pgfpathmoveto{\pgfqpoint{13.389245in}{7.879268in}}%
\pgfpathlineto{\pgfqpoint{13.389245in}{8.800000in}}%
\pgfusepath{stroke}%
\end{pgfscope}%
\begin{pgfscope}%
\definecolor{textcolor}{rgb}{0.150000,0.150000,0.150000}%
\pgfsetstrokecolor{textcolor}%
\pgfsetfillcolor{textcolor}%
\pgftext[x=13.389245in,y=7.782046in,,top]{\color{textcolor}\rmfamily\fontsize{10.000000}{12.000000}\selectfont 2016}%
\end{pgfscope}%
\begin{pgfscope}%
\pgfpathrectangle{\pgfqpoint{9.810417in}{7.879268in}}{\pgfqpoint{5.489583in}{0.920732in}}%
\pgfusepath{clip}%
\pgfsetroundcap%
\pgfsetroundjoin%
\pgfsetlinewidth{0.803000pt}%
\definecolor{currentstroke}{rgb}{1.000000,1.000000,1.000000}%
\pgfsetstrokecolor{currentstroke}%
\pgfsetdash{}{0pt}%
\pgfpathmoveto{\pgfqpoint{14.224423in}{7.879268in}}%
\pgfpathlineto{\pgfqpoint{14.224423in}{8.800000in}}%
\pgfusepath{stroke}%
\end{pgfscope}%
\begin{pgfscope}%
\definecolor{textcolor}{rgb}{0.150000,0.150000,0.150000}%
\pgfsetstrokecolor{textcolor}%
\pgfsetfillcolor{textcolor}%
\pgftext[x=14.224423in,y=7.782046in,,top]{\color{textcolor}\rmfamily\fontsize{10.000000}{12.000000}\selectfont 2017}%
\end{pgfscope}%
\begin{pgfscope}%
\pgfpathrectangle{\pgfqpoint{9.810417in}{7.879268in}}{\pgfqpoint{5.489583in}{0.920732in}}%
\pgfusepath{clip}%
\pgfsetroundcap%
\pgfsetroundjoin%
\pgfsetlinewidth{0.803000pt}%
\definecolor{currentstroke}{rgb}{1.000000,1.000000,1.000000}%
\pgfsetstrokecolor{currentstroke}%
\pgfsetdash{}{0pt}%
\pgfpathmoveto{\pgfqpoint{15.057319in}{7.879268in}}%
\pgfpathlineto{\pgfqpoint{15.057319in}{8.800000in}}%
\pgfusepath{stroke}%
\end{pgfscope}%
\begin{pgfscope}%
\definecolor{textcolor}{rgb}{0.150000,0.150000,0.150000}%
\pgfsetstrokecolor{textcolor}%
\pgfsetfillcolor{textcolor}%
\pgftext[x=15.057319in,y=7.782046in,,top]{\color{textcolor}\rmfamily\fontsize{10.000000}{12.000000}\selectfont 2018}%
\end{pgfscope}%
\begin{pgfscope}%
\pgfpathrectangle{\pgfqpoint{9.810417in}{7.879268in}}{\pgfqpoint{5.489583in}{0.920732in}}%
\pgfusepath{clip}%
\pgfsetroundcap%
\pgfsetroundjoin%
\pgfsetlinewidth{0.803000pt}%
\definecolor{currentstroke}{rgb}{1.000000,1.000000,1.000000}%
\pgfsetstrokecolor{currentstroke}%
\pgfsetdash{}{0pt}%
\pgfpathmoveto{\pgfqpoint{9.810417in}{8.180577in}}%
\pgfpathlineto{\pgfqpoint{15.300000in}{8.180577in}}%
\pgfusepath{stroke}%
\end{pgfscope}%
\begin{pgfscope}%
\definecolor{textcolor}{rgb}{0.150000,0.150000,0.150000}%
\pgfsetstrokecolor{textcolor}%
\pgfsetfillcolor{textcolor}%
\pgftext[x=9.536464in,y=8.127815in,left,base]{\color{textcolor}\rmfamily\fontsize{10.000000}{12.000000}\selectfont 60}%
\end{pgfscope}%
\begin{pgfscope}%
\pgfpathrectangle{\pgfqpoint{9.810417in}{7.879268in}}{\pgfqpoint{5.489583in}{0.920732in}}%
\pgfusepath{clip}%
\pgfsetroundcap%
\pgfsetroundjoin%
\pgfsetlinewidth{0.803000pt}%
\definecolor{currentstroke}{rgb}{1.000000,1.000000,1.000000}%
\pgfsetstrokecolor{currentstroke}%
\pgfsetdash{}{0pt}%
\pgfpathmoveto{\pgfqpoint{9.810417in}{8.490932in}}%
\pgfpathlineto{\pgfqpoint{15.300000in}{8.490932in}}%
\pgfusepath{stroke}%
\end{pgfscope}%
\begin{pgfscope}%
\definecolor{textcolor}{rgb}{0.150000,0.150000,0.150000}%
\pgfsetstrokecolor{textcolor}%
\pgfsetfillcolor{textcolor}%
\pgftext[x=9.536464in,y=8.438171in,left,base]{\color{textcolor}\rmfamily\fontsize{10.000000}{12.000000}\selectfont 80}%
\end{pgfscope}%
\begin{pgfscope}%
\pgfpathrectangle{\pgfqpoint{9.810417in}{7.879268in}}{\pgfqpoint{5.489583in}{0.920732in}}%
\pgfusepath{clip}%
\pgfsetroundcap%
\pgfsetroundjoin%
\pgfsetlinewidth{1.505625pt}%
\definecolor{currentstroke}{rgb}{0.121569,0.466667,0.705882}%
\pgfsetstrokecolor{currentstroke}%
\pgfsetdash{}{0pt}%
\pgfpathmoveto{\pgfqpoint{10.059943in}{7.921120in}}%
\pgfpathlineto{\pgfqpoint{10.062225in}{7.921585in}}%
\pgfpathlineto{\pgfqpoint{10.064507in}{7.929344in}}%
\pgfpathlineto{\pgfqpoint{10.066789in}{7.921896in}}%
\pgfpathlineto{\pgfqpoint{10.073635in}{7.923603in}}%
\pgfpathlineto{\pgfqpoint{10.078198in}{7.931361in}}%
\pgfpathlineto{\pgfqpoint{10.080480in}{7.941138in}}%
\pgfpathlineto{\pgfqpoint{10.091890in}{7.949052in}}%
\pgfpathlineto{\pgfqpoint{10.096454in}{7.959293in}}%
\pgfpathlineto{\pgfqpoint{10.098736in}{7.946569in}}%
\pgfpathlineto{\pgfqpoint{10.107863in}{7.935241in}}%
\pgfpathlineto{\pgfqpoint{10.110145in}{7.948431in}}%
\pgfpathlineto{\pgfqpoint{10.114709in}{7.943931in}}%
\pgfpathlineto{\pgfqpoint{10.121555in}{7.933844in}}%
\pgfpathlineto{\pgfqpoint{10.123837in}{7.947965in}}%
\pgfpathlineto{\pgfqpoint{10.128400in}{7.962242in}}%
\pgfpathlineto{\pgfqpoint{10.130682in}{7.977449in}}%
\pgfpathlineto{\pgfqpoint{10.137528in}{7.971242in}}%
\pgfpathlineto{\pgfqpoint{10.139810in}{7.975742in}}%
\pgfpathlineto{\pgfqpoint{10.142092in}{7.968914in}}%
\pgfpathlineto{\pgfqpoint{10.144374in}{7.978225in}}%
\pgfpathlineto{\pgfqpoint{10.146656in}{7.971242in}}%
\pgfpathlineto{\pgfqpoint{10.153501in}{7.974811in}}%
\pgfpathlineto{\pgfqpoint{10.155783in}{7.973415in}}%
\pgfpathlineto{\pgfqpoint{10.158065in}{7.967208in}}%
\pgfpathlineto{\pgfqpoint{10.160347in}{7.985984in}}%
\pgfpathlineto{\pgfqpoint{10.174039in}{7.984898in}}%
\pgfpathlineto{\pgfqpoint{10.176320in}{7.983191in}}%
\pgfpathlineto{\pgfqpoint{10.178602in}{7.992502in}}%
\pgfpathlineto{\pgfqpoint{10.185448in}{8.004295in}}%
\pgfpathlineto{\pgfqpoint{10.187730in}{7.998398in}}%
\pgfpathlineto{\pgfqpoint{10.190012in}{7.986294in}}%
\pgfpathlineto{\pgfqpoint{10.192294in}{7.995760in}}%
\pgfpathlineto{\pgfqpoint{10.194576in}{7.987691in}}%
\pgfpathlineto{\pgfqpoint{10.201421in}{7.987381in}}%
\pgfpathlineto{\pgfqpoint{10.203703in}{7.970311in}}%
\pgfpathlineto{\pgfqpoint{10.205985in}{7.977604in}}%
\pgfpathlineto{\pgfqpoint{10.208267in}{7.987070in}}%
\pgfpathlineto{\pgfqpoint{10.210549in}{7.990639in}}%
\pgfpathlineto{\pgfqpoint{10.217395in}{7.984587in}}%
\pgfpathlineto{\pgfqpoint{10.219677in}{8.005226in}}%
\pgfpathlineto{\pgfqpoint{10.221959in}{8.031761in}}%
\pgfpathlineto{\pgfqpoint{10.224240in}{8.039676in}}%
\pgfpathlineto{\pgfqpoint{10.226522in}{8.037348in}}%
\pgfpathlineto{\pgfqpoint{10.233368in}{8.047279in}}%
\pgfpathlineto{\pgfqpoint{10.235650in}{8.042314in}}%
\pgfpathlineto{\pgfqpoint{10.237932in}{8.044176in}}%
\pgfpathlineto{\pgfqpoint{10.240214in}{8.047590in}}%
\pgfpathlineto{\pgfqpoint{10.242496in}{8.046969in}}%
\pgfpathlineto{\pgfqpoint{10.249341in}{8.066676in}}%
\pgfpathlineto{\pgfqpoint{10.251623in}{8.060625in}}%
\pgfpathlineto{\pgfqpoint{10.253905in}{8.072263in}}%
\pgfpathlineto{\pgfqpoint{10.256187in}{8.055969in}}%
\pgfpathlineto{\pgfqpoint{10.258469in}{8.055504in}}%
\pgfpathlineto{\pgfqpoint{10.265315in}{8.057676in}}%
\pgfpathlineto{\pgfqpoint{10.267597in}{8.065745in}}%
\pgfpathlineto{\pgfqpoint{10.269879in}{8.051469in}}%
\pgfpathlineto{\pgfqpoint{10.272161in}{8.061711in}}%
\pgfpathlineto{\pgfqpoint{10.281288in}{8.048521in}}%
\pgfpathlineto{\pgfqpoint{10.283570in}{8.034865in}}%
\pgfpathlineto{\pgfqpoint{10.285852in}{8.045572in}}%
\pgfpathlineto{\pgfqpoint{10.288134in}{8.060780in}}%
\pgfpathlineto{\pgfqpoint{10.290416in}{8.050228in}}%
\pgfpathlineto{\pgfqpoint{10.297261in}{8.057831in}}%
\pgfpathlineto{\pgfqpoint{10.299543in}{8.062797in}}%
\pgfpathlineto{\pgfqpoint{10.301825in}{8.060780in}}%
\pgfpathlineto{\pgfqpoint{10.304107in}{8.054262in}}%
\pgfpathlineto{\pgfqpoint{10.306389in}{8.052555in}}%
\pgfpathlineto{\pgfqpoint{10.313235in}{8.050848in}}%
\pgfpathlineto{\pgfqpoint{10.315517in}{8.055038in}}%
\pgfpathlineto{\pgfqpoint{10.317799in}{8.073039in}}%
\pgfpathlineto{\pgfqpoint{10.322362in}{8.090574in}}%
\pgfpathlineto{\pgfqpoint{10.329208in}{8.091195in}}%
\pgfpathlineto{\pgfqpoint{10.331490in}{8.101747in}}%
\pgfpathlineto{\pgfqpoint{10.333772in}{8.102988in}}%
\pgfpathlineto{\pgfqpoint{10.336054in}{8.100195in}}%
\pgfpathlineto{\pgfqpoint{10.338336in}{8.089643in}}%
\pgfpathlineto{\pgfqpoint{10.345182in}{8.089643in}}%
\pgfpathlineto{\pgfqpoint{10.347463in}{8.086694in}}%
\pgfpathlineto{\pgfqpoint{10.349745in}{8.080642in}}%
\pgfpathlineto{\pgfqpoint{10.352027in}{8.080177in}}%
\pgfpathlineto{\pgfqpoint{10.354309in}{8.083280in}}%
\pgfpathlineto{\pgfqpoint{10.363437in}{8.059383in}}%
\pgfpathlineto{\pgfqpoint{10.365719in}{8.051779in}}%
\pgfpathlineto{\pgfqpoint{10.368001in}{8.027416in}}%
\pgfpathlineto{\pgfqpoint{10.370283in}{8.024003in}}%
\pgfpathlineto{\pgfqpoint{10.377128in}{8.036882in}}%
\pgfpathlineto{\pgfqpoint{10.379410in}{8.037193in}}%
\pgfpathlineto{\pgfqpoint{10.381692in}{8.031917in}}%
\pgfpathlineto{\pgfqpoint{10.383974in}{8.037038in}}%
\pgfpathlineto{\pgfqpoint{10.386256in}{8.029589in}}%
\pgfpathlineto{\pgfqpoint{10.395383in}{8.040141in}}%
\pgfpathlineto{\pgfqpoint{10.397665in}{8.024778in}}%
\pgfpathlineto{\pgfqpoint{10.399947in}{8.029899in}}%
\pgfpathlineto{\pgfqpoint{10.402229in}{7.996381in}}%
\pgfpathlineto{\pgfqpoint{10.409075in}{7.997467in}}%
\pgfpathlineto{\pgfqpoint{10.411357in}{8.004140in}}%
\pgfpathlineto{\pgfqpoint{10.413639in}{8.023692in}}%
\pgfpathlineto{\pgfqpoint{10.415921in}{8.021675in}}%
\pgfpathlineto{\pgfqpoint{10.418203in}{8.030365in}}%
\pgfpathlineto{\pgfqpoint{10.425048in}{8.019813in}}%
\pgfpathlineto{\pgfqpoint{10.427330in}{8.039055in}}%
\pgfpathlineto{\pgfqpoint{10.429612in}{8.019813in}}%
\pgfpathlineto{\pgfqpoint{10.431894in}{8.019037in}}%
\pgfpathlineto{\pgfqpoint{10.434176in}{8.036262in}}%
\pgfpathlineto{\pgfqpoint{10.441022in}{8.030210in}}%
\pgfpathlineto{\pgfqpoint{10.443304in}{8.045417in}}%
\pgfpathlineto{\pgfqpoint{10.445585in}{8.052400in}}%
\pgfpathlineto{\pgfqpoint{10.447867in}{8.036727in}}%
\pgfpathlineto{\pgfqpoint{10.450149in}{8.043400in}}%
\pgfpathlineto{\pgfqpoint{10.456995in}{8.033003in}}%
\pgfpathlineto{\pgfqpoint{10.459277in}{8.033934in}}%
\pgfpathlineto{\pgfqpoint{10.461559in}{8.044641in}}%
\pgfpathlineto{\pgfqpoint{10.463841in}{8.042003in}}%
\pgfpathlineto{\pgfqpoint{10.466123in}{8.063263in}}%
\pgfpathlineto{\pgfqpoint{10.472968in}{8.074901in}}%
\pgfpathlineto{\pgfqpoint{10.475250in}{8.082815in}}%
\pgfpathlineto{\pgfqpoint{10.479814in}{8.080642in}}%
\pgfpathlineto{\pgfqpoint{10.482096in}{8.071797in}}%
\pgfpathlineto{\pgfqpoint{10.488942in}{8.068849in}}%
\pgfpathlineto{\pgfqpoint{10.491224in}{8.068539in}}%
\pgfpathlineto{\pgfqpoint{10.493505in}{8.064194in}}%
\pgfpathlineto{\pgfqpoint{10.495787in}{8.048055in}}%
\pgfpathlineto{\pgfqpoint{10.498069in}{8.062021in}}%
\pgfpathlineto{\pgfqpoint{10.504915in}{8.071953in}}%
\pgfpathlineto{\pgfqpoint{10.507197in}{8.072573in}}%
\pgfpathlineto{\pgfqpoint{10.509479in}{8.067142in}}%
\pgfpathlineto{\pgfqpoint{10.511761in}{8.038279in}}%
\pgfpathlineto{\pgfqpoint{10.514043in}{8.032382in}}%
\pgfpathlineto{\pgfqpoint{10.520888in}{8.031141in}}%
\pgfpathlineto{\pgfqpoint{10.523170in}{8.029744in}}%
\pgfpathlineto{\pgfqpoint{10.525452in}{8.035641in}}%
\pgfpathlineto{\pgfqpoint{10.527734in}{8.059693in}}%
\pgfpathlineto{\pgfqpoint{10.530016in}{8.070401in}}%
\pgfpathlineto{\pgfqpoint{10.536862in}{8.067142in}}%
\pgfpathlineto{\pgfqpoint{10.539144in}{8.058918in}}%
\pgfpathlineto{\pgfqpoint{10.541426in}{8.046193in}}%
\pgfpathlineto{\pgfqpoint{10.543707in}{8.041848in}}%
\pgfpathlineto{\pgfqpoint{10.545989in}{8.057521in}}%
\pgfpathlineto{\pgfqpoint{10.552835in}{8.050383in}}%
\pgfpathlineto{\pgfqpoint{10.555117in}{8.057987in}}%
\pgfpathlineto{\pgfqpoint{10.557399in}{8.061711in}}%
\pgfpathlineto{\pgfqpoint{10.559681in}{8.041538in}}%
\pgfpathlineto{\pgfqpoint{10.561963in}{8.032848in}}%
\pgfpathlineto{\pgfqpoint{10.568808in}{8.036727in}}%
\pgfpathlineto{\pgfqpoint{10.571090in}{8.036106in}}%
\pgfpathlineto{\pgfqpoint{10.575654in}{8.054262in}}%
\pgfpathlineto{\pgfqpoint{10.577936in}{8.057211in}}%
\pgfpathlineto{\pgfqpoint{10.584782in}{8.047590in}}%
\pgfpathlineto{\pgfqpoint{10.587064in}{8.043400in}}%
\pgfpathlineto{\pgfqpoint{10.589346in}{8.046503in}}%
\pgfpathlineto{\pgfqpoint{10.591627in}{8.040917in}}%
\pgfpathlineto{\pgfqpoint{10.593909in}{8.055814in}}%
\pgfpathlineto{\pgfqpoint{10.600755in}{8.054883in}}%
\pgfpathlineto{\pgfqpoint{10.603037in}{8.056900in}}%
\pgfpathlineto{\pgfqpoint{10.605319in}{8.054883in}}%
\pgfpathlineto{\pgfqpoint{10.607601in}{8.051469in}}%
\pgfpathlineto{\pgfqpoint{10.609883in}{8.067297in}}%
\pgfpathlineto{\pgfqpoint{10.619010in}{8.071642in}}%
\pgfpathlineto{\pgfqpoint{10.621292in}{8.051624in}}%
\pgfpathlineto{\pgfqpoint{10.623574in}{8.054728in}}%
\pgfpathlineto{\pgfqpoint{10.625856in}{8.059228in}}%
\pgfpathlineto{\pgfqpoint{10.632702in}{8.056280in}}%
\pgfpathlineto{\pgfqpoint{10.634984in}{8.052245in}}%
\pgfpathlineto{\pgfqpoint{10.637266in}{8.052866in}}%
\pgfpathlineto{\pgfqpoint{10.639548in}{8.077694in}}%
\pgfpathlineto{\pgfqpoint{10.641829in}{8.080798in}}%
\pgfpathlineto{\pgfqpoint{10.648675in}{8.079556in}}%
\pgfpathlineto{\pgfqpoint{10.650957in}{8.072418in}}%
\pgfpathlineto{\pgfqpoint{10.653239in}{8.072418in}}%
\pgfpathlineto{\pgfqpoint{10.655521in}{8.067763in}}%
\pgfpathlineto{\pgfqpoint{10.657803in}{8.061090in}}%
\pgfpathlineto{\pgfqpoint{10.664648in}{8.058452in}}%
\pgfpathlineto{\pgfqpoint{10.666930in}{8.050848in}}%
\pgfpathlineto{\pgfqpoint{10.669212in}{8.036882in}}%
\pgfpathlineto{\pgfqpoint{10.671494in}{8.042934in}}%
\pgfpathlineto{\pgfqpoint{10.680622in}{8.059228in}}%
\pgfpathlineto{\pgfqpoint{10.682904in}{8.051469in}}%
\pgfpathlineto{\pgfqpoint{10.685186in}{8.057521in}}%
\pgfpathlineto{\pgfqpoint{10.687468in}{8.071021in}}%
\pgfpathlineto{\pgfqpoint{10.689749in}{8.073815in}}%
\pgfpathlineto{\pgfqpoint{10.696595in}{8.077384in}}%
\pgfpathlineto{\pgfqpoint{10.698877in}{8.069314in}}%
\pgfpathlineto{\pgfqpoint{10.701159in}{8.065435in}}%
\pgfpathlineto{\pgfqpoint{10.703441in}{8.072573in}}%
\pgfpathlineto{\pgfqpoint{10.705723in}{8.064349in}}%
\pgfpathlineto{\pgfqpoint{10.712569in}{8.060159in}}%
\pgfpathlineto{\pgfqpoint{10.714850in}{8.074746in}}%
\pgfpathlineto{\pgfqpoint{10.717132in}{8.085143in}}%
\pgfpathlineto{\pgfqpoint{10.719414in}{8.060469in}}%
\pgfpathlineto{\pgfqpoint{10.721696in}{8.049917in}}%
\pgfpathlineto{\pgfqpoint{10.728542in}{8.048210in}}%
\pgfpathlineto{\pgfqpoint{10.730824in}{8.028968in}}%
\pgfpathlineto{\pgfqpoint{10.733106in}{8.026951in}}%
\pgfpathlineto{\pgfqpoint{10.735388in}{8.029899in}}%
\pgfpathlineto{\pgfqpoint{10.737670in}{8.034244in}}%
\pgfpathlineto{\pgfqpoint{10.749079in}{8.037348in}}%
\pgfpathlineto{\pgfqpoint{10.751361in}{8.049762in}}%
\pgfpathlineto{\pgfqpoint{10.753643in}{8.047745in}}%
\pgfpathlineto{\pgfqpoint{10.760489in}{8.042779in}}%
\pgfpathlineto{\pgfqpoint{10.762770in}{8.054417in}}%
\pgfpathlineto{\pgfqpoint{10.765052in}{8.031761in}}%
\pgfpathlineto{\pgfqpoint{10.767334in}{8.031606in}}%
\pgfpathlineto{\pgfqpoint{10.769616in}{8.035331in}}%
\pgfpathlineto{\pgfqpoint{10.776462in}{8.030986in}}%
\pgfpathlineto{\pgfqpoint{10.778744in}{8.019968in}}%
\pgfpathlineto{\pgfqpoint{10.781026in}{8.004605in}}%
\pgfpathlineto{\pgfqpoint{10.783308in}{8.004450in}}%
\pgfpathlineto{\pgfqpoint{10.785590in}{8.013761in}}%
\pgfpathlineto{\pgfqpoint{10.792435in}{8.026951in}}%
\pgfpathlineto{\pgfqpoint{10.794717in}{8.035796in}}%
\pgfpathlineto{\pgfqpoint{10.796999in}{8.037503in}}%
\pgfpathlineto{\pgfqpoint{10.801563in}{8.044952in}}%
\pgfpathlineto{\pgfqpoint{10.808409in}{8.033313in}}%
\pgfpathlineto{\pgfqpoint{10.810691in}{8.015778in}}%
\pgfpathlineto{\pgfqpoint{10.812972in}{8.030830in}}%
\pgfpathlineto{\pgfqpoint{10.815254in}{8.036572in}}%
\pgfpathlineto{\pgfqpoint{10.817536in}{8.036262in}}%
\pgfpathlineto{\pgfqpoint{10.824382in}{8.037658in}}%
\pgfpathlineto{\pgfqpoint{10.826664in}{8.035486in}}%
\pgfpathlineto{\pgfqpoint{10.828946in}{8.043400in}}%
\pgfpathlineto{\pgfqpoint{10.831228in}{8.039365in}}%
\pgfpathlineto{\pgfqpoint{10.833510in}{8.046348in}}%
\pgfpathlineto{\pgfqpoint{10.840355in}{8.048365in}}%
\pgfpathlineto{\pgfqpoint{10.842637in}{8.052710in}}%
\pgfpathlineto{\pgfqpoint{10.844919in}{8.061245in}}%
\pgfpathlineto{\pgfqpoint{10.847201in}{8.062021in}}%
\pgfpathlineto{\pgfqpoint{10.849483in}{8.046814in}}%
\pgfpathlineto{\pgfqpoint{10.856329in}{8.055659in}}%
\pgfpathlineto{\pgfqpoint{10.858611in}{8.063418in}}%
\pgfpathlineto{\pgfqpoint{10.860892in}{8.048831in}}%
\pgfpathlineto{\pgfqpoint{10.863174in}{8.057521in}}%
\pgfpathlineto{\pgfqpoint{10.865456in}{8.060935in}}%
\pgfpathlineto{\pgfqpoint{10.872302in}{8.059228in}}%
\pgfpathlineto{\pgfqpoint{10.876866in}{8.053642in}}%
\pgfpathlineto{\pgfqpoint{10.879148in}{8.047434in}}%
\pgfpathlineto{\pgfqpoint{10.881430in}{8.047124in}}%
\pgfpathlineto{\pgfqpoint{10.888275in}{8.058607in}}%
\pgfpathlineto{\pgfqpoint{10.892839in}{8.079401in}}%
\pgfpathlineto{\pgfqpoint{10.895121in}{8.082815in}}%
\pgfpathlineto{\pgfqpoint{10.897403in}{8.091505in}}%
\pgfpathlineto{\pgfqpoint{10.904249in}{8.095074in}}%
\pgfpathlineto{\pgfqpoint{10.906531in}{8.099885in}}%
\pgfpathlineto{\pgfqpoint{10.908813in}{8.100660in}}%
\pgfpathlineto{\pgfqpoint{10.913376in}{8.114471in}}%
\pgfpathlineto{\pgfqpoint{10.920222in}{8.114006in}}%
\pgfpathlineto{\pgfqpoint{10.922504in}{8.107799in}}%
\pgfpathlineto{\pgfqpoint{10.924786in}{8.105781in}}%
\pgfpathlineto{\pgfqpoint{10.927068in}{8.107488in}}%
\pgfpathlineto{\pgfqpoint{10.929350in}{8.093833in}}%
\pgfpathlineto{\pgfqpoint{10.938477in}{8.088557in}}%
\pgfpathlineto{\pgfqpoint{10.940759in}{8.081884in}}%
\pgfpathlineto{\pgfqpoint{10.943041in}{8.087160in}}%
\pgfpathlineto{\pgfqpoint{10.945323in}{8.089953in}}%
\pgfpathlineto{\pgfqpoint{10.952169in}{8.083901in}}%
\pgfpathlineto{\pgfqpoint{10.954451in}{8.089177in}}%
\pgfpathlineto{\pgfqpoint{10.956733in}{8.087470in}}%
\pgfpathlineto{\pgfqpoint{10.959014in}{8.080177in}}%
\pgfpathlineto{\pgfqpoint{10.961296in}{8.095695in}}%
\pgfpathlineto{\pgfqpoint{10.968142in}{8.088712in}}%
\pgfpathlineto{\pgfqpoint{10.970424in}{8.106247in}}%
\pgfpathlineto{\pgfqpoint{10.972706in}{8.104850in}}%
\pgfpathlineto{\pgfqpoint{10.974988in}{8.127196in}}%
\pgfpathlineto{\pgfqpoint{10.977270in}{8.122385in}}%
\pgfpathlineto{\pgfqpoint{10.984115in}{8.125023in}}%
\pgfpathlineto{\pgfqpoint{10.986397in}{8.128127in}}%
\pgfpathlineto{\pgfqpoint{10.988679in}{8.126575in}}%
\pgfpathlineto{\pgfqpoint{10.990961in}{8.129989in}}%
\pgfpathlineto{\pgfqpoint{10.993243in}{8.120834in}}%
\pgfpathlineto{\pgfqpoint{11.002371in}{8.129213in}}%
\pgfpathlineto{\pgfqpoint{11.006935in}{8.118816in}}%
\pgfpathlineto{\pgfqpoint{11.009216in}{8.133248in}}%
\pgfpathlineto{\pgfqpoint{11.016062in}{8.125489in}}%
\pgfpathlineto{\pgfqpoint{11.018344in}{8.124092in}}%
\pgfpathlineto{\pgfqpoint{11.020626in}{8.132317in}}%
\pgfpathlineto{\pgfqpoint{11.022908in}{8.127351in}}%
\pgfpathlineto{\pgfqpoint{11.025190in}{8.130455in}}%
\pgfpathlineto{\pgfqpoint{11.032036in}{8.137748in}}%
\pgfpathlineto{\pgfqpoint{11.034317in}{8.155128in}}%
\pgfpathlineto{\pgfqpoint{11.036599in}{8.162732in}}%
\pgfpathlineto{\pgfqpoint{11.038881in}{8.161335in}}%
\pgfpathlineto{\pgfqpoint{11.041163in}{8.163352in}}%
\pgfpathlineto{\pgfqpoint{11.048009in}{8.174680in}}%
\pgfpathlineto{\pgfqpoint{11.050291in}{8.171732in}}%
\pgfpathlineto{\pgfqpoint{11.052573in}{8.171887in}}%
\pgfpathlineto{\pgfqpoint{11.054855in}{8.172973in}}%
\pgfpathlineto{\pgfqpoint{11.057136in}{8.183060in}}%
\pgfpathlineto{\pgfqpoint{11.063982in}{8.178560in}}%
\pgfpathlineto{\pgfqpoint{11.066264in}{8.169404in}}%
\pgfpathlineto{\pgfqpoint{11.068546in}{8.181663in}}%
\pgfpathlineto{\pgfqpoint{11.070828in}{8.173439in}}%
\pgfpathlineto{\pgfqpoint{11.073110in}{8.184767in}}%
\pgfpathlineto{\pgfqpoint{11.079956in}{8.183215in}}%
\pgfpathlineto{\pgfqpoint{11.082237in}{8.198267in}}%
\pgfpathlineto{\pgfqpoint{11.084519in}{8.197957in}}%
\pgfpathlineto{\pgfqpoint{11.086801in}{8.202302in}}%
\pgfpathlineto{\pgfqpoint{11.095929in}{8.199509in}}%
\pgfpathlineto{\pgfqpoint{11.098211in}{8.204940in}}%
\pgfpathlineto{\pgfqpoint{11.100493in}{8.188025in}}%
\pgfpathlineto{\pgfqpoint{11.102775in}{8.194853in}}%
\pgfpathlineto{\pgfqpoint{11.105057in}{8.174525in}}%
\pgfpathlineto{\pgfqpoint{11.111902in}{8.178870in}}%
\pgfpathlineto{\pgfqpoint{11.114184in}{8.173439in}}%
\pgfpathlineto{\pgfqpoint{11.116466in}{8.176077in}}%
\pgfpathlineto{\pgfqpoint{11.118748in}{8.180887in}}%
\pgfpathlineto{\pgfqpoint{11.121030in}{8.179956in}}%
\pgfpathlineto{\pgfqpoint{11.127876in}{8.157611in}}%
\pgfpathlineto{\pgfqpoint{11.130157in}{8.164594in}}%
\pgfpathlineto{\pgfqpoint{11.132439in}{8.158076in}}%
\pgfpathlineto{\pgfqpoint{11.134721in}{8.170956in}}%
\pgfpathlineto{\pgfqpoint{11.137003in}{8.201991in}}%
\pgfpathlineto{\pgfqpoint{11.143849in}{8.193922in}}%
\pgfpathlineto{\pgfqpoint{11.146131in}{8.205095in}}%
\pgfpathlineto{\pgfqpoint{11.148413in}{8.204629in}}%
\pgfpathlineto{\pgfqpoint{11.150695in}{8.214716in}}%
\pgfpathlineto{\pgfqpoint{11.152977in}{8.209285in}}%
\pgfpathlineto{\pgfqpoint{11.159822in}{8.207578in}}%
\pgfpathlineto{\pgfqpoint{11.162104in}{8.218595in}}%
\pgfpathlineto{\pgfqpoint{11.164386in}{8.216733in}}%
\pgfpathlineto{\pgfqpoint{11.168950in}{8.244355in}}%
\pgfpathlineto{\pgfqpoint{11.175796in}{8.242027in}}%
\pgfpathlineto{\pgfqpoint{11.178078in}{8.244510in}}%
\pgfpathlineto{\pgfqpoint{11.180359in}{8.245596in}}%
\pgfpathlineto{\pgfqpoint{11.191769in}{8.238148in}}%
\pgfpathlineto{\pgfqpoint{11.194051in}{8.262511in}}%
\pgfpathlineto{\pgfqpoint{11.196333in}{8.280511in}}%
\pgfpathlineto{\pgfqpoint{11.198615in}{8.272753in}}%
\pgfpathlineto{\pgfqpoint{11.200897in}{8.288270in}}%
\pgfpathlineto{\pgfqpoint{11.207742in}{8.303478in}}%
\pgfpathlineto{\pgfqpoint{11.210024in}{8.313564in}}%
\pgfpathlineto{\pgfqpoint{11.212306in}{8.304098in}}%
\pgfpathlineto{\pgfqpoint{11.214588in}{8.307668in}}%
\pgfpathlineto{\pgfqpoint{11.216870in}{8.315892in}}%
\pgfpathlineto{\pgfqpoint{11.225998in}{8.328461in}}%
\pgfpathlineto{\pgfqpoint{11.228279in}{8.323806in}}%
\pgfpathlineto{\pgfqpoint{11.230561in}{8.328151in}}%
\pgfpathlineto{\pgfqpoint{11.232843in}{8.322099in}}%
\pgfpathlineto{\pgfqpoint{11.239689in}{8.332806in}}%
\pgfpathlineto{\pgfqpoint{11.241971in}{8.327065in}}%
\pgfpathlineto{\pgfqpoint{11.244253in}{8.308599in}}%
\pgfpathlineto{\pgfqpoint{11.248817in}{8.354997in}}%
\pgfpathlineto{\pgfqpoint{11.255662in}{8.358411in}}%
\pgfpathlineto{\pgfqpoint{11.260226in}{8.307978in}}%
\pgfpathlineto{\pgfqpoint{11.262508in}{8.314961in}}%
\pgfpathlineto{\pgfqpoint{11.264790in}{8.283305in}}%
\pgfpathlineto{\pgfqpoint{11.271636in}{8.295564in}}%
\pgfpathlineto{\pgfqpoint{11.273918in}{8.311702in}}%
\pgfpathlineto{\pgfqpoint{11.276200in}{8.301150in}}%
\pgfpathlineto{\pgfqpoint{11.278481in}{8.282529in}}%
\pgfpathlineto{\pgfqpoint{11.280763in}{8.288115in}}%
\pgfpathlineto{\pgfqpoint{11.287609in}{8.269649in}}%
\pgfpathlineto{\pgfqpoint{11.289891in}{8.286874in}}%
\pgfpathlineto{\pgfqpoint{11.292173in}{8.296340in}}%
\pgfpathlineto{\pgfqpoint{11.294455in}{8.313719in}}%
\pgfpathlineto{\pgfqpoint{11.296737in}{8.308599in}}%
\pgfpathlineto{\pgfqpoint{11.303582in}{8.321013in}}%
\pgfpathlineto{\pgfqpoint{11.305864in}{8.309840in}}%
\pgfpathlineto{\pgfqpoint{11.308146in}{8.309219in}}%
\pgfpathlineto{\pgfqpoint{11.312710in}{8.333893in}}%
\pgfpathlineto{\pgfqpoint{11.319556in}{8.344289in}}%
\pgfpathlineto{\pgfqpoint{11.321838in}{8.352359in}}%
\pgfpathlineto{\pgfqpoint{11.324120in}{8.334203in}}%
\pgfpathlineto{\pgfqpoint{11.326401in}{8.342738in}}%
\pgfpathlineto{\pgfqpoint{11.328683in}{8.362600in}}%
\pgfpathlineto{\pgfqpoint{11.335529in}{8.355773in}}%
\pgfpathlineto{\pgfqpoint{11.337811in}{8.361669in}}%
\pgfpathlineto{\pgfqpoint{11.340093in}{8.340876in}}%
\pgfpathlineto{\pgfqpoint{11.342375in}{8.301150in}}%
\pgfpathlineto{\pgfqpoint{11.344657in}{8.301926in}}%
\pgfpathlineto{\pgfqpoint{11.351502in}{8.311237in}}%
\pgfpathlineto{\pgfqpoint{11.353784in}{8.306892in}}%
\pgfpathlineto{\pgfqpoint{11.356066in}{8.320392in}}%
\pgfpathlineto{\pgfqpoint{11.358348in}{8.326134in}}%
\pgfpathlineto{\pgfqpoint{11.360630in}{8.320082in}}%
\pgfpathlineto{\pgfqpoint{11.367476in}{8.316047in}}%
\pgfpathlineto{\pgfqpoint{11.369758in}{8.317909in}}%
\pgfpathlineto{\pgfqpoint{11.372040in}{8.297736in}}%
\pgfpathlineto{\pgfqpoint{11.374322in}{8.324116in}}%
\pgfpathlineto{\pgfqpoint{11.376603in}{8.325047in}}%
\pgfpathlineto{\pgfqpoint{11.383449in}{8.328461in}}%
\pgfpathlineto{\pgfqpoint{11.385731in}{8.326754in}}%
\pgfpathlineto{\pgfqpoint{11.388013in}{8.317289in}}%
\pgfpathlineto{\pgfqpoint{11.390295in}{8.332186in}}%
\pgfpathlineto{\pgfqpoint{11.392577in}{8.322409in}}%
\pgfpathlineto{\pgfqpoint{11.399423in}{8.322254in}}%
\pgfpathlineto{\pgfqpoint{11.401704in}{8.332651in}}%
\pgfpathlineto{\pgfqpoint{11.403986in}{8.327685in}}%
\pgfpathlineto{\pgfqpoint{11.406268in}{8.313564in}}%
\pgfpathlineto{\pgfqpoint{11.408550in}{8.317599in}}%
\pgfpathlineto{\pgfqpoint{11.415396in}{8.305805in}}%
\pgfpathlineto{\pgfqpoint{11.417678in}{8.304874in}}%
\pgfpathlineto{\pgfqpoint{11.419960in}{8.293081in}}%
\pgfpathlineto{\pgfqpoint{11.422242in}{8.298978in}}%
\pgfpathlineto{\pgfqpoint{11.424523in}{8.296029in}}%
\pgfpathlineto{\pgfqpoint{11.431369in}{8.295564in}}%
\pgfpathlineto{\pgfqpoint{11.433651in}{8.271356in}}%
\pgfpathlineto{\pgfqpoint{11.435933in}{8.272753in}}%
\pgfpathlineto{\pgfqpoint{11.438215in}{8.275701in}}%
\pgfpathlineto{\pgfqpoint{11.440497in}{8.271356in}}%
\pgfpathlineto{\pgfqpoint{11.449624in}{8.278649in}}%
\pgfpathlineto{\pgfqpoint{11.454188in}{8.298202in}}%
\pgfpathlineto{\pgfqpoint{11.456470in}{8.291839in}}%
\pgfpathlineto{\pgfqpoint{11.463316in}{8.296184in}}%
\pgfpathlineto{\pgfqpoint{11.465598in}{8.309530in}}%
\pgfpathlineto{\pgfqpoint{11.467880in}{8.317754in}}%
\pgfpathlineto{\pgfqpoint{11.470162in}{8.319306in}}%
\pgfpathlineto{\pgfqpoint{11.472444in}{8.319461in}}%
\pgfpathlineto{\pgfqpoint{11.479289in}{8.323806in}}%
\pgfpathlineto{\pgfqpoint{11.481571in}{8.342583in}}%
\pgfpathlineto{\pgfqpoint{11.483853in}{8.352824in}}%
\pgfpathlineto{\pgfqpoint{11.486135in}{8.351738in}}%
\pgfpathlineto{\pgfqpoint{11.488417in}{8.348169in}}%
\pgfpathlineto{\pgfqpoint{11.497545in}{8.330479in}}%
\pgfpathlineto{\pgfqpoint{11.499826in}{8.329237in}}%
\pgfpathlineto{\pgfqpoint{11.502108in}{8.334048in}}%
\pgfpathlineto{\pgfqpoint{11.504390in}{8.327841in}}%
\pgfpathlineto{\pgfqpoint{11.511236in}{8.322565in}}%
\pgfpathlineto{\pgfqpoint{11.513518in}{8.328461in}}%
\pgfpathlineto{\pgfqpoint{11.515800in}{8.312478in}}%
\pgfpathlineto{\pgfqpoint{11.518082in}{8.304564in}}%
\pgfpathlineto{\pgfqpoint{11.520364in}{8.308599in}}%
\pgfpathlineto{\pgfqpoint{11.527209in}{8.289046in}}%
\pgfpathlineto{\pgfqpoint{11.529491in}{8.278804in}}%
\pgfpathlineto{\pgfqpoint{11.531773in}{8.278804in}}%
\pgfpathlineto{\pgfqpoint{11.534055in}{8.313564in}}%
\pgfpathlineto{\pgfqpoint{11.536337in}{8.324116in}}%
\pgfpathlineto{\pgfqpoint{11.543183in}{8.333737in}}%
\pgfpathlineto{\pgfqpoint{11.545465in}{8.322099in}}%
\pgfpathlineto{\pgfqpoint{11.547746in}{8.337306in}}%
\pgfpathlineto{\pgfqpoint{11.550028in}{8.393015in}}%
\pgfpathlineto{\pgfqpoint{11.552310in}{8.397205in}}%
\pgfpathlineto{\pgfqpoint{11.559156in}{8.395498in}}%
\pgfpathlineto{\pgfqpoint{11.561438in}{8.402171in}}%
\pgfpathlineto{\pgfqpoint{11.563720in}{8.398757in}}%
\pgfpathlineto{\pgfqpoint{11.566002in}{8.402636in}}%
\pgfpathlineto{\pgfqpoint{11.568284in}{8.426999in}}%
\pgfpathlineto{\pgfqpoint{11.575129in}{8.429017in}}%
\pgfpathlineto{\pgfqpoint{11.577411in}{8.441896in}}%
\pgfpathlineto{\pgfqpoint{11.579693in}{8.433982in}}%
\pgfpathlineto{\pgfqpoint{11.581975in}{8.415361in}}%
\pgfpathlineto{\pgfqpoint{11.584257in}{8.420482in}}%
\pgfpathlineto{\pgfqpoint{11.591103in}{8.417223in}}%
\pgfpathlineto{\pgfqpoint{11.593385in}{8.416757in}}%
\pgfpathlineto{\pgfqpoint{11.595666in}{8.420016in}}%
\pgfpathlineto{\pgfqpoint{11.597948in}{8.402791in}}%
\pgfpathlineto{\pgfqpoint{11.600230in}{8.415051in}}%
\pgfpathlineto{\pgfqpoint{11.607076in}{8.409774in}}%
\pgfpathlineto{\pgfqpoint{11.609358in}{8.404498in}}%
\pgfpathlineto{\pgfqpoint{11.613922in}{8.416757in}}%
\pgfpathlineto{\pgfqpoint{11.616204in}{8.429637in}}%
\pgfpathlineto{\pgfqpoint{11.623049in}{8.423120in}}%
\pgfpathlineto{\pgfqpoint{11.625331in}{8.424827in}}%
\pgfpathlineto{\pgfqpoint{11.627613in}{8.421878in}}%
\pgfpathlineto{\pgfqpoint{11.629895in}{8.446086in}}%
\pgfpathlineto{\pgfqpoint{11.632177in}{8.444689in}}%
\pgfpathlineto{\pgfqpoint{11.639023in}{8.455242in}}%
\pgfpathlineto{\pgfqpoint{11.643587in}{8.469363in}}%
\pgfpathlineto{\pgfqpoint{11.648150in}{8.472466in}}%
\pgfpathlineto{\pgfqpoint{11.654996in}{8.465173in}}%
\pgfpathlineto{\pgfqpoint{11.657278in}{8.455397in}}%
\pgfpathlineto{\pgfqpoint{11.659560in}{8.453845in}}%
\pgfpathlineto{\pgfqpoint{11.661842in}{8.454466in}}%
\pgfpathlineto{\pgfqpoint{11.664124in}{8.474328in}}%
\pgfpathlineto{\pgfqpoint{11.670969in}{8.471690in}}%
\pgfpathlineto{\pgfqpoint{11.673251in}{8.465173in}}%
\pgfpathlineto{\pgfqpoint{11.675533in}{8.445310in}}%
\pgfpathlineto{\pgfqpoint{11.677815in}{8.436775in}}%
\pgfpathlineto{\pgfqpoint{11.680097in}{8.442207in}}%
\pgfpathlineto{\pgfqpoint{11.686943in}{8.454155in}}%
\pgfpathlineto{\pgfqpoint{11.689225in}{8.448259in}}%
\pgfpathlineto{\pgfqpoint{11.691507in}{8.475104in}}%
\pgfpathlineto{\pgfqpoint{11.693788in}{8.481156in}}%
\pgfpathlineto{\pgfqpoint{11.696070in}{8.497605in}}%
\pgfpathlineto{\pgfqpoint{11.702916in}{8.508157in}}%
\pgfpathlineto{\pgfqpoint{11.705198in}{8.513588in}}%
\pgfpathlineto{\pgfqpoint{11.718889in}{8.526468in}}%
\pgfpathlineto{\pgfqpoint{11.721171in}{8.542607in}}%
\pgfpathlineto{\pgfqpoint{11.725735in}{8.524451in}}%
\pgfpathlineto{\pgfqpoint{11.728017in}{8.528641in}}%
\pgfpathlineto{\pgfqpoint{11.734863in}{8.528020in}}%
\pgfpathlineto{\pgfqpoint{11.737145in}{8.523209in}}%
\pgfpathlineto{\pgfqpoint{11.739427in}{8.527089in}}%
\pgfpathlineto{\pgfqpoint{11.741709in}{8.519640in}}%
\pgfpathlineto{\pgfqpoint{11.743990in}{8.514830in}}%
\pgfpathlineto{\pgfqpoint{11.750836in}{8.492639in}}%
\pgfpathlineto{\pgfqpoint{11.753118in}{8.494502in}}%
\pgfpathlineto{\pgfqpoint{11.755400in}{8.510640in}}%
\pgfpathlineto{\pgfqpoint{11.757682in}{8.503812in}}%
\pgfpathlineto{\pgfqpoint{11.759964in}{8.549434in}}%
\pgfpathlineto{\pgfqpoint{11.769091in}{8.544003in}}%
\pgfpathlineto{\pgfqpoint{11.771373in}{8.551762in}}%
\pgfpathlineto{\pgfqpoint{11.775937in}{8.492019in}}%
\pgfpathlineto{\pgfqpoint{11.782783in}{8.474484in}}%
\pgfpathlineto{\pgfqpoint{11.785065in}{8.487519in}}%
\pgfpathlineto{\pgfqpoint{11.787347in}{8.472001in}}%
\pgfpathlineto{\pgfqpoint{11.789629in}{8.487208in}}%
\pgfpathlineto{\pgfqpoint{11.791910in}{8.464397in}}%
\pgfpathlineto{\pgfqpoint{11.798756in}{8.433517in}}%
\pgfpathlineto{\pgfqpoint{11.801038in}{8.449810in}}%
\pgfpathlineto{\pgfqpoint{11.803320in}{8.445776in}}%
\pgfpathlineto{\pgfqpoint{11.805602in}{8.474018in}}%
\pgfpathlineto{\pgfqpoint{11.807884in}{8.492639in}}%
\pgfpathlineto{\pgfqpoint{11.814730in}{8.511881in}}%
\pgfpathlineto{\pgfqpoint{11.817011in}{8.515295in}}%
\pgfpathlineto{\pgfqpoint{11.819293in}{8.521502in}}%
\pgfpathlineto{\pgfqpoint{11.821575in}{8.519796in}}%
\pgfpathlineto{\pgfqpoint{11.823857in}{8.521347in}}%
\pgfpathlineto{\pgfqpoint{11.832985in}{8.521502in}}%
\pgfpathlineto{\pgfqpoint{11.835267in}{8.519020in}}%
\pgfpathlineto{\pgfqpoint{11.837549in}{8.521502in}}%
\pgfpathlineto{\pgfqpoint{11.839831in}{8.517778in}}%
\pgfpathlineto{\pgfqpoint{11.846676in}{8.534382in}}%
\pgfpathlineto{\pgfqpoint{11.848958in}{8.534227in}}%
\pgfpathlineto{\pgfqpoint{11.851240in}{8.531744in}}%
\pgfpathlineto{\pgfqpoint{11.853522in}{8.539658in}}%
\pgfpathlineto{\pgfqpoint{11.855804in}{8.553935in}}%
\pgfpathlineto{\pgfqpoint{11.862650in}{8.535158in}}%
\pgfpathlineto{\pgfqpoint{11.864932in}{8.572866in}}%
\pgfpathlineto{\pgfqpoint{11.867213in}{8.566194in}}%
\pgfpathlineto{\pgfqpoint{11.869495in}{8.585901in}}%
\pgfpathlineto{\pgfqpoint{11.871777in}{8.590712in}}%
\pgfpathlineto{\pgfqpoint{11.878623in}{8.588384in}}%
\pgfpathlineto{\pgfqpoint{11.880905in}{8.581246in}}%
\pgfpathlineto{\pgfqpoint{11.883187in}{8.577056in}}%
\pgfpathlineto{\pgfqpoint{11.885469in}{8.545400in}}%
\pgfpathlineto{\pgfqpoint{11.887751in}{8.537951in}}%
\pgfpathlineto{\pgfqpoint{11.896878in}{8.558280in}}%
\pgfpathlineto{\pgfqpoint{11.899160in}{8.546021in}}%
\pgfpathlineto{\pgfqpoint{11.901442in}{8.559676in}}%
\pgfpathlineto{\pgfqpoint{11.912852in}{8.547572in}}%
\pgfpathlineto{\pgfqpoint{11.915133in}{8.530658in}}%
\pgfpathlineto{\pgfqpoint{11.917415in}{8.535468in}}%
\pgfpathlineto{\pgfqpoint{11.919697in}{8.542141in}}%
\pgfpathlineto{\pgfqpoint{11.926543in}{8.535934in}}%
\pgfpathlineto{\pgfqpoint{11.928825in}{8.552228in}}%
\pgfpathlineto{\pgfqpoint{11.931107in}{8.544624in}}%
\pgfpathlineto{\pgfqpoint{11.933389in}{8.552848in}}%
\pgfpathlineto{\pgfqpoint{11.935671in}{8.526934in}}%
\pgfpathlineto{\pgfqpoint{11.942516in}{8.490157in}}%
\pgfpathlineto{\pgfqpoint{11.944798in}{8.488605in}}%
\pgfpathlineto{\pgfqpoint{11.947080in}{8.520571in}}%
\pgfpathlineto{\pgfqpoint{11.949362in}{8.472311in}}%
\pgfpathlineto{\pgfqpoint{11.951644in}{8.460673in}}%
\pgfpathlineto{\pgfqpoint{11.958490in}{8.474328in}}%
\pgfpathlineto{\pgfqpoint{11.960772in}{8.482087in}}%
\pgfpathlineto{\pgfqpoint{11.963054in}{8.501640in}}%
\pgfpathlineto{\pgfqpoint{11.965335in}{8.484725in}}%
\pgfpathlineto{\pgfqpoint{11.974463in}{8.491088in}}%
\pgfpathlineto{\pgfqpoint{11.976745in}{8.496829in}}%
\pgfpathlineto{\pgfqpoint{11.979027in}{8.497760in}}%
\pgfpathlineto{\pgfqpoint{11.981309in}{8.501795in}}%
\pgfpathlineto{\pgfqpoint{11.983591in}{8.496364in}}%
\pgfpathlineto{\pgfqpoint{11.990436in}{8.496674in}}%
\pgfpathlineto{\pgfqpoint{11.992718in}{8.507226in}}%
\pgfpathlineto{\pgfqpoint{11.995000in}{8.501950in}}%
\pgfpathlineto{\pgfqpoint{11.997282in}{8.493260in}}%
\pgfpathlineto{\pgfqpoint{11.999564in}{8.494812in}}%
\pgfpathlineto{\pgfqpoint{12.006410in}{8.501019in}}%
\pgfpathlineto{\pgfqpoint{12.008692in}{8.484415in}}%
\pgfpathlineto{\pgfqpoint{12.010974in}{8.509864in}}%
\pgfpathlineto{\pgfqpoint{12.013255in}{8.519020in}}%
\pgfpathlineto{\pgfqpoint{12.015537in}{8.522278in}}%
\pgfpathlineto{\pgfqpoint{12.022383in}{8.533917in}}%
\pgfpathlineto{\pgfqpoint{12.026947in}{8.516847in}}%
\pgfpathlineto{\pgfqpoint{12.029229in}{8.504433in}}%
\pgfpathlineto{\pgfqpoint{12.031511in}{8.503036in}}%
\pgfpathlineto{\pgfqpoint{12.038356in}{8.511881in}}%
\pgfpathlineto{\pgfqpoint{12.040638in}{8.497140in}}%
\pgfpathlineto{\pgfqpoint{12.042920in}{8.508157in}}%
\pgfpathlineto{\pgfqpoint{12.045202in}{8.512347in}}%
\pgfpathlineto{\pgfqpoint{12.056612in}{8.558745in}}%
\pgfpathlineto{\pgfqpoint{12.058894in}{8.554090in}}%
\pgfpathlineto{\pgfqpoint{12.063457in}{8.560297in}}%
\pgfpathlineto{\pgfqpoint{12.070303in}{8.565883in}}%
\pgfpathlineto{\pgfqpoint{12.072585in}{8.563556in}}%
\pgfpathlineto{\pgfqpoint{12.074867in}{8.564797in}}%
\pgfpathlineto{\pgfqpoint{12.077149in}{8.578918in}}%
\pgfpathlineto{\pgfqpoint{12.079431in}{8.609178in}}%
\pgfpathlineto{\pgfqpoint{12.086276in}{8.618644in}}%
\pgfpathlineto{\pgfqpoint{12.088558in}{8.614454in}}%
\pgfpathlineto{\pgfqpoint{12.090840in}{8.611661in}}%
\pgfpathlineto{\pgfqpoint{12.093122in}{8.607005in}}%
\pgfpathlineto{\pgfqpoint{12.095404in}{8.608247in}}%
\pgfpathlineto{\pgfqpoint{12.102250in}{8.601419in}}%
\pgfpathlineto{\pgfqpoint{12.104532in}{8.605609in}}%
\pgfpathlineto{\pgfqpoint{12.106814in}{8.618489in}}%
\pgfpathlineto{\pgfqpoint{12.109096in}{8.611506in}}%
\pgfpathlineto{\pgfqpoint{12.111377in}{8.618178in}}%
\pgfpathlineto{\pgfqpoint{12.118223in}{8.617713in}}%
\pgfpathlineto{\pgfqpoint{12.120505in}{8.602350in}}%
\pgfpathlineto{\pgfqpoint{12.122787in}{8.605298in}}%
\pgfpathlineto{\pgfqpoint{12.125069in}{8.600488in}}%
\pgfpathlineto{\pgfqpoint{12.127351in}{8.609488in}}%
\pgfpathlineto{\pgfqpoint{12.134197in}{8.608557in}}%
\pgfpathlineto{\pgfqpoint{12.136478in}{8.615540in}}%
\pgfpathlineto{\pgfqpoint{12.138760in}{8.613523in}}%
\pgfpathlineto{\pgfqpoint{12.141042in}{8.622523in}}%
\pgfpathlineto{\pgfqpoint{12.150170in}{8.616161in}}%
\pgfpathlineto{\pgfqpoint{12.152452in}{8.604367in}}%
\pgfpathlineto{\pgfqpoint{12.154734in}{8.610419in}}%
\pgfpathlineto{\pgfqpoint{12.157016in}{8.606385in}}%
\pgfpathlineto{\pgfqpoint{12.159297in}{8.606540in}}%
\pgfpathlineto{\pgfqpoint{12.166143in}{8.607626in}}%
\pgfpathlineto{\pgfqpoint{12.168425in}{8.606385in}}%
\pgfpathlineto{\pgfqpoint{12.170707in}{8.606540in}}%
\pgfpathlineto{\pgfqpoint{12.172989in}{8.585281in}}%
\pgfpathlineto{\pgfqpoint{12.175271in}{8.593039in}}%
\pgfpathlineto{\pgfqpoint{12.182117in}{8.583729in}}%
\pgfpathlineto{\pgfqpoint{12.184398in}{8.590867in}}%
\pgfpathlineto{\pgfqpoint{12.188962in}{8.587608in}}%
\pgfpathlineto{\pgfqpoint{12.191244in}{8.570073in}}%
\pgfpathlineto{\pgfqpoint{12.198090in}{8.569142in}}%
\pgfpathlineto{\pgfqpoint{12.200372in}{8.566970in}}%
\pgfpathlineto{\pgfqpoint{12.202654in}{8.555486in}}%
\pgfpathlineto{\pgfqpoint{12.204936in}{8.513588in}}%
\pgfpathlineto{\pgfqpoint{12.207218in}{8.491708in}}%
\pgfpathlineto{\pgfqpoint{12.214063in}{8.498226in}}%
\pgfpathlineto{\pgfqpoint{12.216345in}{8.490157in}}%
\pgfpathlineto{\pgfqpoint{12.218627in}{8.490622in}}%
\pgfpathlineto{\pgfqpoint{12.220909in}{8.485191in}}%
\pgfpathlineto{\pgfqpoint{12.223191in}{8.505985in}}%
\pgfpathlineto{\pgfqpoint{12.230037in}{8.498536in}}%
\pgfpathlineto{\pgfqpoint{12.232319in}{8.500398in}}%
\pgfpathlineto{\pgfqpoint{12.234600in}{8.505054in}}%
\pgfpathlineto{\pgfqpoint{12.236882in}{8.503192in}}%
\pgfpathlineto{\pgfqpoint{12.239164in}{8.493570in}}%
\pgfpathlineto{\pgfqpoint{12.246010in}{8.501640in}}%
\pgfpathlineto{\pgfqpoint{12.248292in}{8.515451in}}%
\pgfpathlineto{\pgfqpoint{12.250574in}{8.520727in}}%
\pgfpathlineto{\pgfqpoint{12.252856in}{8.530037in}}%
\pgfpathlineto{\pgfqpoint{12.255138in}{8.526313in}}%
\pgfpathlineto{\pgfqpoint{12.261983in}{8.536555in}}%
\pgfpathlineto{\pgfqpoint{12.264265in}{8.530192in}}%
\pgfpathlineto{\pgfqpoint{12.266547in}{8.531434in}}%
\pgfpathlineto{\pgfqpoint{12.268829in}{8.528330in}}%
\pgfpathlineto{\pgfqpoint{12.271111in}{8.535934in}}%
\pgfpathlineto{\pgfqpoint{12.280239in}{8.538417in}}%
\pgfpathlineto{\pgfqpoint{12.282520in}{8.544469in}}%
\pgfpathlineto{\pgfqpoint{12.284802in}{8.537331in}}%
\pgfpathlineto{\pgfqpoint{12.287084in}{8.536710in}}%
\pgfpathlineto{\pgfqpoint{12.293930in}{8.527089in}}%
\pgfpathlineto{\pgfqpoint{12.296212in}{8.512192in}}%
\pgfpathlineto{\pgfqpoint{12.298494in}{8.519485in}}%
\pgfpathlineto{\pgfqpoint{12.300776in}{8.519640in}}%
\pgfpathlineto{\pgfqpoint{12.303058in}{8.508468in}}%
\pgfpathlineto{\pgfqpoint{12.309903in}{8.504743in}}%
\pgfpathlineto{\pgfqpoint{12.312185in}{8.518864in}}%
\pgfpathlineto{\pgfqpoint{12.314467in}{8.529106in}}%
\pgfpathlineto{\pgfqpoint{12.316749in}{8.543848in}}%
\pgfpathlineto{\pgfqpoint{12.319031in}{8.538106in}}%
\pgfpathlineto{\pgfqpoint{12.325877in}{8.527865in}}%
\pgfpathlineto{\pgfqpoint{12.328159in}{8.518089in}}%
\pgfpathlineto{\pgfqpoint{12.330441in}{8.521192in}}%
\pgfpathlineto{\pgfqpoint{12.332722in}{8.495898in}}%
\pgfpathlineto{\pgfqpoint{12.335004in}{8.519020in}}%
\pgfpathlineto{\pgfqpoint{12.341850in}{8.512968in}}%
\pgfpathlineto{\pgfqpoint{12.344132in}{8.507071in}}%
\pgfpathlineto{\pgfqpoint{12.346414in}{8.485812in}}%
\pgfpathlineto{\pgfqpoint{12.348696in}{8.486743in}}%
\pgfpathlineto{\pgfqpoint{12.350978in}{8.505364in}}%
\pgfpathlineto{\pgfqpoint{12.357823in}{8.503502in}}%
\pgfpathlineto{\pgfqpoint{12.360105in}{8.479294in}}%
\pgfpathlineto{\pgfqpoint{12.362387in}{8.508933in}}%
\pgfpathlineto{\pgfqpoint{12.364669in}{8.487053in}}%
\pgfpathlineto{\pgfqpoint{12.366951in}{8.474018in}}%
\pgfpathlineto{\pgfqpoint{12.373797in}{8.442207in}}%
\pgfpathlineto{\pgfqpoint{12.376079in}{8.441586in}}%
\pgfpathlineto{\pgfqpoint{12.378361in}{8.415516in}}%
\pgfpathlineto{\pgfqpoint{12.380642in}{8.405585in}}%
\pgfpathlineto{\pgfqpoint{12.382924in}{8.439258in}}%
\pgfpathlineto{\pgfqpoint{12.389770in}{8.459897in}}%
\pgfpathlineto{\pgfqpoint{12.392052in}{8.483484in}}%
\pgfpathlineto{\pgfqpoint{12.394334in}{8.459276in}}%
\pgfpathlineto{\pgfqpoint{12.396616in}{8.483018in}}%
\pgfpathlineto{\pgfqpoint{12.398898in}{8.494346in}}%
\pgfpathlineto{\pgfqpoint{12.405743in}{8.497760in}}%
\pgfpathlineto{\pgfqpoint{12.408025in}{8.517623in}}%
\pgfpathlineto{\pgfqpoint{12.412589in}{8.528020in}}%
\pgfpathlineto{\pgfqpoint{12.414871in}{8.545555in}}%
\pgfpathlineto{\pgfqpoint{12.421717in}{8.558435in}}%
\pgfpathlineto{\pgfqpoint{12.423999in}{8.566194in}}%
\pgfpathlineto{\pgfqpoint{12.426281in}{8.581091in}}%
\pgfpathlineto{\pgfqpoint{12.428563in}{8.568987in}}%
\pgfpathlineto{\pgfqpoint{12.430844in}{8.578763in}}%
\pgfpathlineto{\pgfqpoint{12.437690in}{8.580780in}}%
\pgfpathlineto{\pgfqpoint{12.439972in}{8.571315in}}%
\pgfpathlineto{\pgfqpoint{12.442254in}{8.568521in}}%
\pgfpathlineto{\pgfqpoint{12.446818in}{8.555952in}}%
\pgfpathlineto{\pgfqpoint{12.453663in}{8.548038in}}%
\pgfpathlineto{\pgfqpoint{12.455945in}{8.554555in}}%
\pgfpathlineto{\pgfqpoint{12.458227in}{8.553469in}}%
\pgfpathlineto{\pgfqpoint{12.460509in}{8.555176in}}%
\pgfpathlineto{\pgfqpoint{12.462791in}{8.551917in}}%
\pgfpathlineto{\pgfqpoint{12.469637in}{8.561228in}}%
\pgfpathlineto{\pgfqpoint{12.471919in}{8.566349in}}%
\pgfpathlineto{\pgfqpoint{12.474201in}{8.567125in}}%
\pgfpathlineto{\pgfqpoint{12.478764in}{8.581091in}}%
\pgfpathlineto{\pgfqpoint{12.485610in}{8.577056in}}%
\pgfpathlineto{\pgfqpoint{12.487892in}{8.589470in}}%
\pgfpathlineto{\pgfqpoint{12.490174in}{8.564176in}}%
\pgfpathlineto{\pgfqpoint{12.492456in}{8.572556in}}%
\pgfpathlineto{\pgfqpoint{12.494738in}{8.584349in}}%
\pgfpathlineto{\pgfqpoint{12.501584in}{8.597540in}}%
\pgfpathlineto{\pgfqpoint{12.503865in}{8.594591in}}%
\pgfpathlineto{\pgfqpoint{12.506147in}{8.585901in}}%
\pgfpathlineto{\pgfqpoint{12.508429in}{8.591332in}}%
\pgfpathlineto{\pgfqpoint{12.510711in}{8.560607in}}%
\pgfpathlineto{\pgfqpoint{12.517557in}{8.546796in}}%
\pgfpathlineto{\pgfqpoint{12.519839in}{8.520882in}}%
\pgfpathlineto{\pgfqpoint{12.522121in}{8.552538in}}%
\pgfpathlineto{\pgfqpoint{12.524403in}{8.591953in}}%
\pgfpathlineto{\pgfqpoint{12.526684in}{8.588074in}}%
\pgfpathlineto{\pgfqpoint{12.538094in}{8.604988in}}%
\pgfpathlineto{\pgfqpoint{12.542658in}{8.608092in}}%
\pgfpathlineto{\pgfqpoint{12.551785in}{8.607781in}}%
\pgfpathlineto{\pgfqpoint{12.554067in}{8.590091in}}%
\pgfpathlineto{\pgfqpoint{12.558631in}{8.589781in}}%
\pgfpathlineto{\pgfqpoint{12.565477in}{8.554245in}}%
\pgfpathlineto{\pgfqpoint{12.567759in}{8.526468in}}%
\pgfpathlineto{\pgfqpoint{12.570041in}{8.554400in}}%
\pgfpathlineto{\pgfqpoint{12.572323in}{8.572866in}}%
\pgfpathlineto{\pgfqpoint{12.574605in}{8.556107in}}%
\pgfpathlineto{\pgfqpoint{12.583732in}{8.538882in}}%
\pgfpathlineto{\pgfqpoint{12.586014in}{8.507692in}}%
\pgfpathlineto{\pgfqpoint{12.588296in}{8.490467in}}%
\pgfpathlineto{\pgfqpoint{12.590578in}{8.492795in}}%
\pgfpathlineto{\pgfqpoint{12.599706in}{8.510485in}}%
\pgfpathlineto{\pgfqpoint{12.601987in}{8.516382in}}%
\pgfpathlineto{\pgfqpoint{12.604269in}{8.468742in}}%
\pgfpathlineto{\pgfqpoint{12.606551in}{8.465173in}}%
\pgfpathlineto{\pgfqpoint{12.613397in}{8.453224in}}%
\pgfpathlineto{\pgfqpoint{12.615679in}{8.440189in}}%
\pgfpathlineto{\pgfqpoint{12.617961in}{8.430568in}}%
\pgfpathlineto{\pgfqpoint{12.620243in}{8.434448in}}%
\pgfpathlineto{\pgfqpoint{12.622525in}{8.415516in}}%
\pgfpathlineto{\pgfqpoint{12.629370in}{8.436310in}}%
\pgfpathlineto{\pgfqpoint{12.631652in}{8.459431in}}%
\pgfpathlineto{\pgfqpoint{12.633934in}{8.457724in}}%
\pgfpathlineto{\pgfqpoint{12.636216in}{8.473863in}}%
\pgfpathlineto{\pgfqpoint{12.638498in}{8.477898in}}%
\pgfpathlineto{\pgfqpoint{12.645344in}{8.477587in}}%
\pgfpathlineto{\pgfqpoint{12.647626in}{8.490001in}}%
\pgfpathlineto{\pgfqpoint{12.649907in}{8.492329in}}%
\pgfpathlineto{\pgfqpoint{12.652189in}{8.412413in}}%
\pgfpathlineto{\pgfqpoint{12.654471in}{8.377808in}}%
\pgfpathlineto{\pgfqpoint{12.663599in}{8.392239in}}%
\pgfpathlineto{\pgfqpoint{12.665881in}{8.402326in}}%
\pgfpathlineto{\pgfqpoint{12.668163in}{8.382463in}}%
\pgfpathlineto{\pgfqpoint{12.670445in}{8.403102in}}%
\pgfpathlineto{\pgfqpoint{12.677290in}{8.409930in}}%
\pgfpathlineto{\pgfqpoint{12.679572in}{8.417999in}}%
\pgfpathlineto{\pgfqpoint{12.684136in}{8.452448in}}%
\pgfpathlineto{\pgfqpoint{12.686418in}{8.428551in}}%
\pgfpathlineto{\pgfqpoint{12.693264in}{8.434913in}}%
\pgfpathlineto{\pgfqpoint{12.695546in}{8.433206in}}%
\pgfpathlineto{\pgfqpoint{12.697828in}{8.414585in}}%
\pgfpathlineto{\pgfqpoint{12.700109in}{8.422189in}}%
\pgfpathlineto{\pgfqpoint{12.702391in}{8.410085in}}%
\pgfpathlineto{\pgfqpoint{12.709237in}{8.412878in}}%
\pgfpathlineto{\pgfqpoint{12.711519in}{8.392705in}}%
\pgfpathlineto{\pgfqpoint{12.713801in}{8.397671in}}%
\pgfpathlineto{\pgfqpoint{12.716083in}{8.428085in}}%
\pgfpathlineto{\pgfqpoint{12.718365in}{8.414275in}}%
\pgfpathlineto{\pgfqpoint{12.725210in}{8.427154in}}%
\pgfpathlineto{\pgfqpoint{12.727492in}{8.420792in}}%
\pgfpathlineto{\pgfqpoint{12.729774in}{8.432430in}}%
\pgfpathlineto{\pgfqpoint{12.732056in}{8.427775in}}%
\pgfpathlineto{\pgfqpoint{12.734338in}{8.444534in}}%
\pgfpathlineto{\pgfqpoint{12.741184in}{8.437396in}}%
\pgfpathlineto{\pgfqpoint{12.743466in}{8.425292in}}%
\pgfpathlineto{\pgfqpoint{12.745748in}{8.407136in}}%
\pgfpathlineto{\pgfqpoint{12.748029in}{8.383549in}}%
\pgfpathlineto{\pgfqpoint{12.750311in}{8.376256in}}%
\pgfpathlineto{\pgfqpoint{12.757157in}{8.377342in}}%
\pgfpathlineto{\pgfqpoint{12.759439in}{8.382153in}}%
\pgfpathlineto{\pgfqpoint{12.764003in}{8.405119in}}%
\pgfpathlineto{\pgfqpoint{12.773130in}{8.404033in}}%
\pgfpathlineto{\pgfqpoint{12.775412in}{8.385256in}}%
\pgfpathlineto{\pgfqpoint{12.777694in}{8.390377in}}%
\pgfpathlineto{\pgfqpoint{12.782258in}{8.403412in}}%
\pgfpathlineto{\pgfqpoint{12.789104in}{8.398757in}}%
\pgfpathlineto{\pgfqpoint{12.793668in}{8.405740in}}%
\pgfpathlineto{\pgfqpoint{12.795950in}{8.422654in}}%
\pgfpathlineto{\pgfqpoint{12.798231in}{8.370515in}}%
\pgfpathlineto{\pgfqpoint{12.805077in}{8.369273in}}%
\pgfpathlineto{\pgfqpoint{12.807359in}{8.370049in}}%
\pgfpathlineto{\pgfqpoint{12.809641in}{8.386808in}}%
\pgfpathlineto{\pgfqpoint{12.811923in}{8.382618in}}%
\pgfpathlineto{\pgfqpoint{12.821050in}{8.372997in}}%
\pgfpathlineto{\pgfqpoint{12.823332in}{8.372997in}}%
\pgfpathlineto{\pgfqpoint{12.825614in}{8.368187in}}%
\pgfpathlineto{\pgfqpoint{12.830178in}{8.375791in}}%
\pgfpathlineto{\pgfqpoint{12.837024in}{8.384481in}}%
\pgfpathlineto{\pgfqpoint{12.839306in}{8.378118in}}%
\pgfpathlineto{\pgfqpoint{12.841588in}{8.378273in}}%
\pgfpathlineto{\pgfqpoint{12.846151in}{8.394567in}}%
\pgfpathlineto{\pgfqpoint{12.852997in}{8.405119in}}%
\pgfpathlineto{\pgfqpoint{12.855279in}{8.396119in}}%
\pgfpathlineto{\pgfqpoint{12.859843in}{8.420327in}}%
\pgfpathlineto{\pgfqpoint{12.862125in}{8.412568in}}%
\pgfpathlineto{\pgfqpoint{12.868971in}{8.411947in}}%
\pgfpathlineto{\pgfqpoint{12.871252in}{8.429172in}}%
\pgfpathlineto{\pgfqpoint{12.873534in}{8.423275in}}%
\pgfpathlineto{\pgfqpoint{12.875816in}{8.420482in}}%
\pgfpathlineto{\pgfqpoint{12.878098in}{8.427465in}}%
\pgfpathlineto{\pgfqpoint{12.887226in}{8.412102in}}%
\pgfpathlineto{\pgfqpoint{12.889508in}{8.411016in}}%
\pgfpathlineto{\pgfqpoint{12.891790in}{8.410706in}}%
\pgfpathlineto{\pgfqpoint{12.894072in}{8.405430in}}%
\pgfpathlineto{\pgfqpoint{12.900917in}{8.401705in}}%
\pgfpathlineto{\pgfqpoint{12.905481in}{8.418154in}}%
\pgfpathlineto{\pgfqpoint{12.907763in}{8.400153in}}%
\pgfpathlineto{\pgfqpoint{12.910045in}{8.400619in}}%
\pgfpathlineto{\pgfqpoint{12.916891in}{8.392084in}}%
\pgfpathlineto{\pgfqpoint{12.919172in}{8.397360in}}%
\pgfpathlineto{\pgfqpoint{12.921454in}{8.412102in}}%
\pgfpathlineto{\pgfqpoint{12.923736in}{8.413809in}}%
\pgfpathlineto{\pgfqpoint{12.926018in}{8.402636in}}%
\pgfpathlineto{\pgfqpoint{12.932864in}{8.398602in}}%
\pgfpathlineto{\pgfqpoint{12.935146in}{8.400153in}}%
\pgfpathlineto{\pgfqpoint{12.937428in}{8.414119in}}%
\pgfpathlineto{\pgfqpoint{12.939710in}{8.420637in}}%
\pgfpathlineto{\pgfqpoint{12.941992in}{8.412102in}}%
\pgfpathlineto{\pgfqpoint{12.948837in}{8.427465in}}%
\pgfpathlineto{\pgfqpoint{12.951119in}{8.429172in}}%
\pgfpathlineto{\pgfqpoint{12.953401in}{8.420171in}}%
\pgfpathlineto{\pgfqpoint{12.955683in}{8.407602in}}%
\pgfpathlineto{\pgfqpoint{12.957965in}{8.407602in}}%
\pgfpathlineto{\pgfqpoint{12.964811in}{8.377498in}}%
\pgfpathlineto{\pgfqpoint{12.967093in}{8.380601in}}%
\pgfpathlineto{\pgfqpoint{12.969374in}{8.390532in}}%
\pgfpathlineto{\pgfqpoint{12.971656in}{8.387739in}}%
\pgfpathlineto{\pgfqpoint{12.980784in}{8.378739in}}%
\pgfpathlineto{\pgfqpoint{12.983066in}{8.378118in}}%
\pgfpathlineto{\pgfqpoint{12.985348in}{8.354531in}}%
\pgfpathlineto{\pgfqpoint{12.987630in}{8.360428in}}%
\pgfpathlineto{\pgfqpoint{12.989912in}{8.374549in}}%
\pgfpathlineto{\pgfqpoint{12.999039in}{8.398757in}}%
\pgfpathlineto{\pgfqpoint{13.001321in}{8.392550in}}%
\pgfpathlineto{\pgfqpoint{13.005885in}{8.402326in}}%
\pgfpathlineto{\pgfqpoint{13.012731in}{8.403567in}}%
\pgfpathlineto{\pgfqpoint{13.015013in}{8.398447in}}%
\pgfpathlineto{\pgfqpoint{13.017294in}{8.399067in}}%
\pgfpathlineto{\pgfqpoint{13.019576in}{8.370204in}}%
\pgfpathlineto{\pgfqpoint{13.021858in}{8.354066in}}%
\pgfpathlineto{\pgfqpoint{13.028704in}{8.339789in}}%
\pgfpathlineto{\pgfqpoint{13.030986in}{8.342583in}}%
\pgfpathlineto{\pgfqpoint{13.033268in}{8.351117in}}%
\pgfpathlineto{\pgfqpoint{13.035550in}{8.357324in}}%
\pgfpathlineto{\pgfqpoint{13.037832in}{8.356393in}}%
\pgfpathlineto{\pgfqpoint{13.044677in}{8.355307in}}%
\pgfpathlineto{\pgfqpoint{13.046959in}{8.351428in}}%
\pgfpathlineto{\pgfqpoint{13.049241in}{8.349566in}}%
\pgfpathlineto{\pgfqpoint{13.051523in}{8.341031in}}%
\pgfpathlineto{\pgfqpoint{13.053805in}{8.409619in}}%
\pgfpathlineto{\pgfqpoint{13.060651in}{8.432120in}}%
\pgfpathlineto{\pgfqpoint{13.062933in}{8.433206in}}%
\pgfpathlineto{\pgfqpoint{13.065215in}{8.428241in}}%
\pgfpathlineto{\pgfqpoint{13.067496in}{8.424827in}}%
\pgfpathlineto{\pgfqpoint{13.069778in}{8.426999in}}%
\pgfpathlineto{\pgfqpoint{13.076624in}{8.428241in}}%
\pgfpathlineto{\pgfqpoint{13.078906in}{8.431810in}}%
\pgfpathlineto{\pgfqpoint{13.081188in}{8.427620in}}%
\pgfpathlineto{\pgfqpoint{13.085752in}{8.370515in}}%
\pgfpathlineto{\pgfqpoint{13.092597in}{8.335910in}}%
\pgfpathlineto{\pgfqpoint{13.094879in}{8.319461in}}%
\pgfpathlineto{\pgfqpoint{13.097161in}{8.350186in}}%
\pgfpathlineto{\pgfqpoint{13.099443in}{8.368342in}}%
\pgfpathlineto{\pgfqpoint{13.101725in}{8.364928in}}%
\pgfpathlineto{\pgfqpoint{13.108571in}{8.366014in}}%
\pgfpathlineto{\pgfqpoint{13.110853in}{8.326444in}}%
\pgfpathlineto{\pgfqpoint{13.113135in}{8.340100in}}%
\pgfpathlineto{\pgfqpoint{13.115416in}{8.344755in}}%
\pgfpathlineto{\pgfqpoint{13.117698in}{8.327530in}}%
\pgfpathlineto{\pgfqpoint{13.126826in}{8.348169in}}%
\pgfpathlineto{\pgfqpoint{13.129108in}{8.342738in}}%
\pgfpathlineto{\pgfqpoint{13.133672in}{8.348324in}}%
\pgfpathlineto{\pgfqpoint{13.140517in}{8.343048in}}%
\pgfpathlineto{\pgfqpoint{13.142799in}{8.362756in}}%
\pgfpathlineto{\pgfqpoint{13.145081in}{8.374704in}}%
\pgfpathlineto{\pgfqpoint{13.147363in}{8.370980in}}%
\pgfpathlineto{\pgfqpoint{13.149645in}{8.354842in}}%
\pgfpathlineto{\pgfqpoint{13.156491in}{8.366325in}}%
\pgfpathlineto{\pgfqpoint{13.158773in}{8.351272in}}%
\pgfpathlineto{\pgfqpoint{13.161055in}{8.350186in}}%
\pgfpathlineto{\pgfqpoint{13.163337in}{8.336531in}}%
\pgfpathlineto{\pgfqpoint{13.165618in}{8.342427in}}%
\pgfpathlineto{\pgfqpoint{13.172464in}{8.316823in}}%
\pgfpathlineto{\pgfqpoint{13.174746in}{8.313409in}}%
\pgfpathlineto{\pgfqpoint{13.177028in}{8.328306in}}%
\pgfpathlineto{\pgfqpoint{13.179310in}{8.324892in}}%
\pgfpathlineto{\pgfqpoint{13.181592in}{8.332341in}}%
\pgfpathlineto{\pgfqpoint{13.188437in}{8.374394in}}%
\pgfpathlineto{\pgfqpoint{13.190719in}{8.368808in}}%
\pgfpathlineto{\pgfqpoint{13.193001in}{8.377032in}}%
\pgfpathlineto{\pgfqpoint{13.195283in}{8.377032in}}%
\pgfpathlineto{\pgfqpoint{13.197565in}{8.379204in}}%
\pgfpathlineto{\pgfqpoint{13.204411in}{8.378894in}}%
\pgfpathlineto{\pgfqpoint{13.206693in}{8.368497in}}%
\pgfpathlineto{\pgfqpoint{13.208975in}{8.362135in}}%
\pgfpathlineto{\pgfqpoint{13.213538in}{8.377342in}}%
\pgfpathlineto{\pgfqpoint{13.222666in}{8.373618in}}%
\pgfpathlineto{\pgfqpoint{13.224948in}{8.367256in}}%
\pgfpathlineto{\pgfqpoint{13.227230in}{8.308599in}}%
\pgfpathlineto{\pgfqpoint{13.229512in}{8.339169in}}%
\pgfpathlineto{\pgfqpoint{13.238639in}{8.330634in}}%
\pgfpathlineto{\pgfqpoint{13.240921in}{8.337151in}}%
\pgfpathlineto{\pgfqpoint{13.243203in}{8.333737in}}%
\pgfpathlineto{\pgfqpoint{13.245485in}{8.319771in}}%
\pgfpathlineto{\pgfqpoint{13.252331in}{8.329703in}}%
\pgfpathlineto{\pgfqpoint{13.256895in}{8.331565in}}%
\pgfpathlineto{\pgfqpoint{13.259177in}{8.329548in}}%
\pgfpathlineto{\pgfqpoint{13.261459in}{8.334824in}}%
\pgfpathlineto{\pgfqpoint{13.268304in}{8.322099in}}%
\pgfpathlineto{\pgfqpoint{13.270586in}{8.321323in}}%
\pgfpathlineto{\pgfqpoint{13.272868in}{8.314651in}}%
\pgfpathlineto{\pgfqpoint{13.277432in}{8.289667in}}%
\pgfpathlineto{\pgfqpoint{13.284278in}{8.296340in}}%
\pgfpathlineto{\pgfqpoint{13.286559in}{8.288270in}}%
\pgfpathlineto{\pgfqpoint{13.288841in}{8.303478in}}%
\pgfpathlineto{\pgfqpoint{13.291123in}{8.312168in}}%
\pgfpathlineto{\pgfqpoint{13.293405in}{8.307357in}}%
\pgfpathlineto{\pgfqpoint{13.300251in}{8.304874in}}%
\pgfpathlineto{\pgfqpoint{13.302533in}{8.295874in}}%
\pgfpathlineto{\pgfqpoint{13.309379in}{8.299133in}}%
\pgfpathlineto{\pgfqpoint{13.316224in}{8.296029in}}%
\pgfpathlineto{\pgfqpoint{13.318506in}{8.304254in}}%
\pgfpathlineto{\pgfqpoint{13.323070in}{8.278184in}}%
\pgfpathlineto{\pgfqpoint{13.325352in}{8.288115in}}%
\pgfpathlineto{\pgfqpoint{13.332198in}{8.280977in}}%
\pgfpathlineto{\pgfqpoint{13.334480in}{8.270735in}}%
\pgfpathlineto{\pgfqpoint{13.336761in}{8.269959in}}%
\pgfpathlineto{\pgfqpoint{13.339043in}{8.273684in}}%
\pgfpathlineto{\pgfqpoint{13.341325in}{8.255373in}}%
\pgfpathlineto{\pgfqpoint{13.348171in}{8.255062in}}%
\pgfpathlineto{\pgfqpoint{13.350453in}{8.274304in}}%
\pgfpathlineto{\pgfqpoint{13.352735in}{8.282529in}}%
\pgfpathlineto{\pgfqpoint{13.355017in}{8.265770in}}%
\pgfpathlineto{\pgfqpoint{13.357299in}{8.240941in}}%
\pgfpathlineto{\pgfqpoint{13.364144in}{8.248700in}}%
\pgfpathlineto{\pgfqpoint{13.366426in}{8.255217in}}%
\pgfpathlineto{\pgfqpoint{13.368708in}{8.271666in}}%
\pgfpathlineto{\pgfqpoint{13.370990in}{8.274459in}}%
\pgfpathlineto{\pgfqpoint{13.380118in}{8.268563in}}%
\pgfpathlineto{\pgfqpoint{13.382400in}{8.280046in}}%
\pgfpathlineto{\pgfqpoint{13.384681in}{8.274459in}}%
\pgfpathlineto{\pgfqpoint{13.386963in}{8.265459in}}%
\pgfpathlineto{\pgfqpoint{13.396091in}{8.236906in}}%
\pgfpathlineto{\pgfqpoint{13.398373in}{8.221699in}}%
\pgfpathlineto{\pgfqpoint{13.400655in}{8.194698in}}%
\pgfpathlineto{\pgfqpoint{13.402937in}{8.186163in}}%
\pgfpathlineto{\pgfqpoint{13.405219in}{8.183060in}}%
\pgfpathlineto{\pgfqpoint{13.412064in}{8.189267in}}%
\pgfpathlineto{\pgfqpoint{13.414346in}{8.194388in}}%
\pgfpathlineto{\pgfqpoint{13.416628in}{8.171577in}}%
\pgfpathlineto{\pgfqpoint{13.418910in}{8.178094in}}%
\pgfpathlineto{\pgfqpoint{13.421192in}{8.172508in}}%
\pgfpathlineto{\pgfqpoint{13.430320in}{8.168628in}}%
\pgfpathlineto{\pgfqpoint{13.432602in}{8.174215in}}%
\pgfpathlineto{\pgfqpoint{13.434883in}{8.168628in}}%
\pgfpathlineto{\pgfqpoint{13.437165in}{8.057366in}}%
\pgfpathlineto{\pgfqpoint{13.444011in}{8.056745in}}%
\pgfpathlineto{\pgfqpoint{13.446293in}{8.057831in}}%
\pgfpathlineto{\pgfqpoint{13.448575in}{8.049452in}}%
\pgfpathlineto{\pgfqpoint{13.450857in}{8.025399in}}%
\pgfpathlineto{\pgfqpoint{13.453139in}{8.034399in}}%
\pgfpathlineto{\pgfqpoint{13.459984in}{8.052090in}}%
\pgfpathlineto{\pgfqpoint{13.462266in}{8.036882in}}%
\pgfpathlineto{\pgfqpoint{13.464548in}{8.043400in}}%
\pgfpathlineto{\pgfqpoint{13.466830in}{8.047434in}}%
\pgfpathlineto{\pgfqpoint{13.469112in}{8.041538in}}%
\pgfpathlineto{\pgfqpoint{13.475958in}{8.018261in}}%
\pgfpathlineto{\pgfqpoint{13.478240in}{8.021675in}}%
\pgfpathlineto{\pgfqpoint{13.480522in}{8.016709in}}%
\pgfpathlineto{\pgfqpoint{13.482803in}{7.999329in}}%
\pgfpathlineto{\pgfqpoint{13.485085in}{8.022140in}}%
\pgfpathlineto{\pgfqpoint{13.494213in}{8.029744in}}%
\pgfpathlineto{\pgfqpoint{13.498777in}{8.044021in}}%
\pgfpathlineto{\pgfqpoint{13.501059in}{8.052245in}}%
\pgfpathlineto{\pgfqpoint{13.507904in}{8.065745in}}%
\pgfpathlineto{\pgfqpoint{13.512468in}{8.051159in}}%
\pgfpathlineto{\pgfqpoint{13.514750in}{8.062176in}}%
\pgfpathlineto{\pgfqpoint{13.517032in}{8.062021in}}%
\pgfpathlineto{\pgfqpoint{13.523878in}{8.064970in}}%
\pgfpathlineto{\pgfqpoint{13.526160in}{8.082815in}}%
\pgfpathlineto{\pgfqpoint{13.528442in}{8.087625in}}%
\pgfpathlineto{\pgfqpoint{13.530724in}{8.101747in}}%
\pgfpathlineto{\pgfqpoint{13.533005in}{8.104695in}}%
\pgfpathlineto{\pgfqpoint{13.539851in}{8.115092in}}%
\pgfpathlineto{\pgfqpoint{13.542133in}{8.121454in}}%
\pgfpathlineto{\pgfqpoint{13.546697in}{8.111523in}}%
\pgfpathlineto{\pgfqpoint{13.548979in}{8.121920in}}%
\pgfpathlineto{\pgfqpoint{13.555824in}{8.123472in}}%
\pgfpathlineto{\pgfqpoint{13.558106in}{8.118506in}}%
\pgfpathlineto{\pgfqpoint{13.562670in}{8.131075in}}%
\pgfpathlineto{\pgfqpoint{13.564952in}{8.147679in}}%
\pgfpathlineto{\pgfqpoint{13.571798in}{8.147524in}}%
\pgfpathlineto{\pgfqpoint{13.574080in}{8.138834in}}%
\pgfpathlineto{\pgfqpoint{13.576362in}{8.139144in}}%
\pgfpathlineto{\pgfqpoint{13.578644in}{8.136662in}}%
\pgfpathlineto{\pgfqpoint{13.587771in}{8.133868in}}%
\pgfpathlineto{\pgfqpoint{13.590053in}{8.138679in}}%
\pgfpathlineto{\pgfqpoint{13.592335in}{8.134024in}}%
\pgfpathlineto{\pgfqpoint{13.594617in}{8.150317in}}%
\pgfpathlineto{\pgfqpoint{13.596899in}{8.145972in}}%
\pgfpathlineto{\pgfqpoint{13.603745in}{8.138989in}}%
\pgfpathlineto{\pgfqpoint{13.606026in}{8.132627in}}%
\pgfpathlineto{\pgfqpoint{13.608308in}{8.132317in}}%
\pgfpathlineto{\pgfqpoint{13.610590in}{8.117264in}}%
\pgfpathlineto{\pgfqpoint{13.612872in}{8.126730in}}%
\pgfpathlineto{\pgfqpoint{13.619718in}{8.131075in}}%
\pgfpathlineto{\pgfqpoint{13.622000in}{8.144110in}}%
\pgfpathlineto{\pgfqpoint{13.624282in}{8.165990in}}%
\pgfpathlineto{\pgfqpoint{13.626564in}{8.171111in}}%
\pgfpathlineto{\pgfqpoint{13.628846in}{8.165680in}}%
\pgfpathlineto{\pgfqpoint{13.635691in}{8.172353in}}%
\pgfpathlineto{\pgfqpoint{13.637973in}{8.186784in}}%
\pgfpathlineto{\pgfqpoint{13.640255in}{8.208043in}}%
\pgfpathlineto{\pgfqpoint{13.642537in}{8.216733in}}%
\pgfpathlineto{\pgfqpoint{13.644819in}{8.221544in}}%
\pgfpathlineto{\pgfqpoint{13.651665in}{8.218130in}}%
\pgfpathlineto{\pgfqpoint{13.653946in}{8.225578in}}%
\pgfpathlineto{\pgfqpoint{13.656228in}{8.225889in}}%
\pgfpathlineto{\pgfqpoint{13.660792in}{8.214095in}}%
\pgfpathlineto{\pgfqpoint{13.667638in}{8.217820in}}%
\pgfpathlineto{\pgfqpoint{13.669920in}{8.205561in}}%
\pgfpathlineto{\pgfqpoint{13.672202in}{8.196871in}}%
\pgfpathlineto{\pgfqpoint{13.674484in}{8.191905in}}%
\pgfpathlineto{\pgfqpoint{13.676766in}{8.200595in}}%
\pgfpathlineto{\pgfqpoint{13.683611in}{8.192836in}}%
\pgfpathlineto{\pgfqpoint{13.685893in}{8.205405in}}%
\pgfpathlineto{\pgfqpoint{13.692739in}{8.194853in}}%
\pgfpathlineto{\pgfqpoint{13.699585in}{8.194077in}}%
\pgfpathlineto{\pgfqpoint{13.701867in}{8.174835in}}%
\pgfpathlineto{\pgfqpoint{13.704148in}{8.186008in}}%
\pgfpathlineto{\pgfqpoint{13.706430in}{8.174525in}}%
\pgfpathlineto{\pgfqpoint{13.708712in}{8.191905in}}%
\pgfpathlineto{\pgfqpoint{13.715558in}{8.187094in}}%
\pgfpathlineto{\pgfqpoint{13.717840in}{8.205871in}}%
\pgfpathlineto{\pgfqpoint{13.720122in}{8.212388in}}%
\pgfpathlineto{\pgfqpoint{13.722404in}{8.211147in}}%
\pgfpathlineto{\pgfqpoint{13.724686in}{8.215492in}}%
\pgfpathlineto{\pgfqpoint{13.733813in}{8.219061in}}%
\pgfpathlineto{\pgfqpoint{13.736095in}{8.221234in}}%
\pgfpathlineto{\pgfqpoint{13.738377in}{8.228527in}}%
\pgfpathlineto{\pgfqpoint{13.740659in}{8.215026in}}%
\pgfpathlineto{\pgfqpoint{13.747505in}{8.221699in}}%
\pgfpathlineto{\pgfqpoint{13.749787in}{8.220923in}}%
\pgfpathlineto{\pgfqpoint{13.752068in}{8.226354in}}%
\pgfpathlineto{\pgfqpoint{13.754350in}{8.218285in}}%
\pgfpathlineto{\pgfqpoint{13.756632in}{8.207423in}}%
\pgfpathlineto{\pgfqpoint{13.763478in}{8.188181in}}%
\pgfpathlineto{\pgfqpoint{13.765760in}{8.149852in}}%
\pgfpathlineto{\pgfqpoint{13.768042in}{8.154973in}}%
\pgfpathlineto{\pgfqpoint{13.770324in}{8.162576in}}%
\pgfpathlineto{\pgfqpoint{13.772606in}{8.161490in}}%
\pgfpathlineto{\pgfqpoint{13.779451in}{8.168008in}}%
\pgfpathlineto{\pgfqpoint{13.781733in}{8.167852in}}%
\pgfpathlineto{\pgfqpoint{13.784015in}{8.162887in}}%
\pgfpathlineto{\pgfqpoint{13.786297in}{8.181974in}}%
\pgfpathlineto{\pgfqpoint{13.788579in}{8.134955in}}%
\pgfpathlineto{\pgfqpoint{13.795425in}{8.099729in}}%
\pgfpathlineto{\pgfqpoint{13.797707in}{8.103143in}}%
\pgfpathlineto{\pgfqpoint{13.799989in}{8.133093in}}%
\pgfpathlineto{\pgfqpoint{13.802270in}{8.149852in}}%
\pgfpathlineto{\pgfqpoint{13.804552in}{8.148766in}}%
\pgfpathlineto{\pgfqpoint{13.813680in}{8.125954in}}%
\pgfpathlineto{\pgfqpoint{13.818244in}{8.135731in}}%
\pgfpathlineto{\pgfqpoint{13.820526in}{8.160559in}}%
\pgfpathlineto{\pgfqpoint{13.827371in}{8.170490in}}%
\pgfpathlineto{\pgfqpoint{13.829653in}{8.183060in}}%
\pgfpathlineto{\pgfqpoint{13.831935in}{8.184456in}}%
\pgfpathlineto{\pgfqpoint{13.834217in}{8.192060in}}%
\pgfpathlineto{\pgfqpoint{13.836499in}{8.194543in}}%
\pgfpathlineto{\pgfqpoint{13.843345in}{8.197646in}}%
\pgfpathlineto{\pgfqpoint{13.845627in}{8.200129in}}%
\pgfpathlineto{\pgfqpoint{13.847909in}{8.204940in}}%
\pgfpathlineto{\pgfqpoint{13.850190in}{8.189422in}}%
\pgfpathlineto{\pgfqpoint{13.852472in}{8.201991in}}%
\pgfpathlineto{\pgfqpoint{13.861600in}{8.203233in}}%
\pgfpathlineto{\pgfqpoint{13.866164in}{8.209595in}}%
\pgfpathlineto{\pgfqpoint{13.868446in}{8.204629in}}%
\pgfpathlineto{\pgfqpoint{13.875291in}{8.199819in}}%
\pgfpathlineto{\pgfqpoint{13.877573in}{8.190043in}}%
\pgfpathlineto{\pgfqpoint{13.879855in}{8.195319in}}%
\pgfpathlineto{\pgfqpoint{13.882137in}{8.196871in}}%
\pgfpathlineto{\pgfqpoint{13.884419in}{8.220302in}}%
\pgfpathlineto{\pgfqpoint{13.891265in}{8.224958in}}%
\pgfpathlineto{\pgfqpoint{13.893547in}{8.218440in}}%
\pgfpathlineto{\pgfqpoint{13.895829in}{8.208819in}}%
\pgfpathlineto{\pgfqpoint{13.898111in}{8.219682in}}%
\pgfpathlineto{\pgfqpoint{13.900392in}{8.218285in}}%
\pgfpathlineto{\pgfqpoint{13.907238in}{8.222009in}}%
\pgfpathlineto{\pgfqpoint{13.909520in}{8.217044in}}%
\pgfpathlineto{\pgfqpoint{13.911802in}{8.222785in}}%
\pgfpathlineto{\pgfqpoint{13.914084in}{8.222009in}}%
\pgfpathlineto{\pgfqpoint{13.916366in}{8.220458in}}%
\pgfpathlineto{\pgfqpoint{13.923211in}{8.217975in}}%
\pgfpathlineto{\pgfqpoint{13.925493in}{8.222785in}}%
\pgfpathlineto{\pgfqpoint{13.927775in}{8.215026in}}%
\pgfpathlineto{\pgfqpoint{13.932339in}{8.209595in}}%
\pgfpathlineto{\pgfqpoint{13.939185in}{8.220302in}}%
\pgfpathlineto{\pgfqpoint{13.941467in}{8.219371in}}%
\pgfpathlineto{\pgfqpoint{13.943749in}{8.221234in}}%
\pgfpathlineto{\pgfqpoint{13.946031in}{8.210526in}}%
\pgfpathlineto{\pgfqpoint{13.948312in}{8.215492in}}%
\pgfpathlineto{\pgfqpoint{13.957440in}{8.223872in}}%
\pgfpathlineto{\pgfqpoint{13.959722in}{8.230234in}}%
\pgfpathlineto{\pgfqpoint{13.962004in}{8.231010in}}%
\pgfpathlineto{\pgfqpoint{13.964286in}{8.214406in}}%
\pgfpathlineto{\pgfqpoint{13.971132in}{8.226044in}}%
\pgfpathlineto{\pgfqpoint{13.973413in}{8.203078in}}%
\pgfpathlineto{\pgfqpoint{13.975695in}{8.190043in}}%
\pgfpathlineto{\pgfqpoint{13.977977in}{8.195319in}}%
\pgfpathlineto{\pgfqpoint{13.980259in}{8.192836in}}%
\pgfpathlineto{\pgfqpoint{13.987105in}{8.198733in}}%
\pgfpathlineto{\pgfqpoint{13.989387in}{8.193612in}}%
\pgfpathlineto{\pgfqpoint{13.991669in}{8.201991in}}%
\pgfpathlineto{\pgfqpoint{13.993951in}{8.207112in}}%
\pgfpathlineto{\pgfqpoint{13.996233in}{8.195629in}}%
\pgfpathlineto{\pgfqpoint{14.003078in}{8.189267in}}%
\pgfpathlineto{\pgfqpoint{14.005360in}{8.201991in}}%
\pgfpathlineto{\pgfqpoint{14.007642in}{8.201060in}}%
\pgfpathlineto{\pgfqpoint{14.009924in}{8.188491in}}%
\pgfpathlineto{\pgfqpoint{14.012206in}{8.198422in}}%
\pgfpathlineto{\pgfqpoint{14.019052in}{8.195008in}}%
\pgfpathlineto{\pgfqpoint{14.021333in}{8.196560in}}%
\pgfpathlineto{\pgfqpoint{14.023615in}{8.207888in}}%
\pgfpathlineto{\pgfqpoint{14.025897in}{8.171887in}}%
\pgfpathlineto{\pgfqpoint{14.028179in}{8.169249in}}%
\pgfpathlineto{\pgfqpoint{14.035025in}{8.171266in}}%
\pgfpathlineto{\pgfqpoint{14.037307in}{8.155904in}}%
\pgfpathlineto{\pgfqpoint{14.039589in}{8.153266in}}%
\pgfpathlineto{\pgfqpoint{14.044153in}{8.145196in}}%
\pgfpathlineto{\pgfqpoint{14.050998in}{8.141472in}}%
\pgfpathlineto{\pgfqpoint{14.053280in}{8.144265in}}%
\pgfpathlineto{\pgfqpoint{14.055562in}{8.161645in}}%
\pgfpathlineto{\pgfqpoint{14.057844in}{8.244045in}}%
\pgfpathlineto{\pgfqpoint{14.060126in}{8.252579in}}%
\pgfpathlineto{\pgfqpoint{14.066972in}{8.248545in}}%
\pgfpathlineto{\pgfqpoint{14.069254in}{8.243424in}}%
\pgfpathlineto{\pgfqpoint{14.071535in}{8.244355in}}%
\pgfpathlineto{\pgfqpoint{14.073817in}{8.246217in}}%
\pgfpathlineto{\pgfqpoint{14.076099in}{8.239079in}}%
\pgfpathlineto{\pgfqpoint{14.082945in}{8.238613in}}%
\pgfpathlineto{\pgfqpoint{14.085227in}{8.236131in}}%
\pgfpathlineto{\pgfqpoint{14.087509in}{8.224182in}}%
\pgfpathlineto{\pgfqpoint{14.089791in}{8.222475in}}%
\pgfpathlineto{\pgfqpoint{14.092073in}{8.225113in}}%
\pgfpathlineto{\pgfqpoint{14.098918in}{8.247303in}}%
\pgfpathlineto{\pgfqpoint{14.101200in}{8.248234in}}%
\pgfpathlineto{\pgfqpoint{14.105764in}{8.293546in}}%
\pgfpathlineto{\pgfqpoint{14.108046in}{8.299443in}}%
\pgfpathlineto{\pgfqpoint{14.114892in}{8.327996in}}%
\pgfpathlineto{\pgfqpoint{14.117174in}{8.328772in}}%
\pgfpathlineto{\pgfqpoint{14.119455in}{8.316978in}}%
\pgfpathlineto{\pgfqpoint{14.121737in}{8.318375in}}%
\pgfpathlineto{\pgfqpoint{14.124019in}{8.306892in}}%
\pgfpathlineto{\pgfqpoint{14.133147in}{8.317599in}}%
\pgfpathlineto{\pgfqpoint{14.135429in}{8.334824in}}%
\pgfpathlineto{\pgfqpoint{14.139993in}{8.334513in}}%
\pgfpathlineto{\pgfqpoint{14.146838in}{8.323651in}}%
\pgfpathlineto{\pgfqpoint{14.149120in}{8.314185in}}%
\pgfpathlineto{\pgfqpoint{14.153684in}{8.329548in}}%
\pgfpathlineto{\pgfqpoint{14.155966in}{8.319616in}}%
\pgfpathlineto{\pgfqpoint{14.162812in}{8.322099in}}%
\pgfpathlineto{\pgfqpoint{14.165094in}{8.326289in}}%
\pgfpathlineto{\pgfqpoint{14.167376in}{8.355773in}}%
\pgfpathlineto{\pgfqpoint{14.169657in}{8.365083in}}%
\pgfpathlineto{\pgfqpoint{14.171939in}{8.362911in}}%
\pgfpathlineto{\pgfqpoint{14.178785in}{8.345221in}}%
\pgfpathlineto{\pgfqpoint{14.183349in}{8.352514in}}%
\pgfpathlineto{\pgfqpoint{14.185631in}{8.365549in}}%
\pgfpathlineto{\pgfqpoint{14.187913in}{8.366325in}}%
\pgfpathlineto{\pgfqpoint{14.194758in}{8.359652in}}%
\pgfpathlineto{\pgfqpoint{14.197040in}{8.367256in}}%
\pgfpathlineto{\pgfqpoint{14.199322in}{8.371135in}}%
\pgfpathlineto{\pgfqpoint{14.201604in}{8.360118in}}%
\pgfpathlineto{\pgfqpoint{14.203886in}{8.366014in}}%
\pgfpathlineto{\pgfqpoint{14.213014in}{8.366170in}}%
\pgfpathlineto{\pgfqpoint{14.215296in}{8.357014in}}%
\pgfpathlineto{\pgfqpoint{14.217577in}{8.350341in}}%
\pgfpathlineto{\pgfqpoint{14.219859in}{8.352669in}}%
\pgfpathlineto{\pgfqpoint{14.228987in}{8.371601in}}%
\pgfpathlineto{\pgfqpoint{14.231269in}{8.390067in}}%
\pgfpathlineto{\pgfqpoint{14.233551in}{8.375946in}}%
\pgfpathlineto{\pgfqpoint{14.242678in}{8.384015in}}%
\pgfpathlineto{\pgfqpoint{14.244960in}{8.395808in}}%
\pgfpathlineto{\pgfqpoint{14.247242in}{8.399688in}}%
\pgfpathlineto{\pgfqpoint{14.249524in}{8.399222in}}%
\pgfpathlineto{\pgfqpoint{14.251806in}{8.395343in}}%
\pgfpathlineto{\pgfqpoint{14.260934in}{8.395033in}}%
\pgfpathlineto{\pgfqpoint{14.263216in}{8.408378in}}%
\pgfpathlineto{\pgfqpoint{14.265498in}{8.396429in}}%
\pgfpathlineto{\pgfqpoint{14.267779in}{8.389136in}}%
\pgfpathlineto{\pgfqpoint{14.274625in}{8.385722in}}%
\pgfpathlineto{\pgfqpoint{14.276907in}{8.407447in}}%
\pgfpathlineto{\pgfqpoint{14.279189in}{8.399378in}}%
\pgfpathlineto{\pgfqpoint{14.281471in}{8.399998in}}%
\pgfpathlineto{\pgfqpoint{14.283753in}{8.398757in}}%
\pgfpathlineto{\pgfqpoint{14.290599in}{8.405274in}}%
\pgfpathlineto{\pgfqpoint{14.292880in}{8.391774in}}%
\pgfpathlineto{\pgfqpoint{14.295162in}{8.397515in}}%
\pgfpathlineto{\pgfqpoint{14.297444in}{8.393791in}}%
\pgfpathlineto{\pgfqpoint{14.299726in}{8.416602in}}%
\pgfpathlineto{\pgfqpoint{14.308854in}{8.411792in}}%
\pgfpathlineto{\pgfqpoint{14.311136in}{8.413033in}}%
\pgfpathlineto{\pgfqpoint{14.315699in}{8.423275in}}%
\pgfpathlineto{\pgfqpoint{14.322545in}{8.429637in}}%
\pgfpathlineto{\pgfqpoint{14.324827in}{8.437086in}}%
\pgfpathlineto{\pgfqpoint{14.327109in}{8.440034in}}%
\pgfpathlineto{\pgfqpoint{14.329391in}{8.438638in}}%
\pgfpathlineto{\pgfqpoint{14.331673in}{8.441586in}}%
\pgfpathlineto{\pgfqpoint{14.340800in}{8.445465in}}%
\pgfpathlineto{\pgfqpoint{14.343082in}{8.443914in}}%
\pgfpathlineto{\pgfqpoint{14.345364in}{8.446707in}}%
\pgfpathlineto{\pgfqpoint{14.347646in}{8.442362in}}%
\pgfpathlineto{\pgfqpoint{14.354492in}{8.448414in}}%
\pgfpathlineto{\pgfqpoint{14.356774in}{8.446862in}}%
\pgfpathlineto{\pgfqpoint{14.359056in}{8.474639in}}%
\pgfpathlineto{\pgfqpoint{14.361338in}{8.447483in}}%
\pgfpathlineto{\pgfqpoint{14.363620in}{8.444224in}}%
\pgfpathlineto{\pgfqpoint{14.370465in}{8.438482in}}%
\pgfpathlineto{\pgfqpoint{14.372747in}{8.439724in}}%
\pgfpathlineto{\pgfqpoint{14.375029in}{8.431655in}}%
\pgfpathlineto{\pgfqpoint{14.377311in}{8.435534in}}%
\pgfpathlineto{\pgfqpoint{14.379593in}{8.436620in}}%
\pgfpathlineto{\pgfqpoint{14.386439in}{8.434137in}}%
\pgfpathlineto{\pgfqpoint{14.388720in}{8.441276in}}%
\pgfpathlineto{\pgfqpoint{14.391002in}{8.434448in}}%
\pgfpathlineto{\pgfqpoint{14.393284in}{8.442517in}}%
\pgfpathlineto{\pgfqpoint{14.395566in}{8.434758in}}%
\pgfpathlineto{\pgfqpoint{14.402412in}{8.428706in}}%
\pgfpathlineto{\pgfqpoint{14.404694in}{8.408999in}}%
\pgfpathlineto{\pgfqpoint{14.409258in}{8.413654in}}%
\pgfpathlineto{\pgfqpoint{14.411540in}{8.419085in}}%
\pgfpathlineto{\pgfqpoint{14.418385in}{8.410085in}}%
\pgfpathlineto{\pgfqpoint{14.420667in}{8.425603in}}%
\pgfpathlineto{\pgfqpoint{14.422949in}{8.419706in}}%
\pgfpathlineto{\pgfqpoint{14.425231in}{8.434137in}}%
\pgfpathlineto{\pgfqpoint{14.427513in}{8.432586in}}%
\pgfpathlineto{\pgfqpoint{14.434359in}{8.424827in}}%
\pgfpathlineto{\pgfqpoint{14.436641in}{8.419861in}}%
\pgfpathlineto{\pgfqpoint{14.438922in}{8.417223in}}%
\pgfpathlineto{\pgfqpoint{14.441204in}{8.419551in}}%
\pgfpathlineto{\pgfqpoint{14.443486in}{8.417378in}}%
\pgfpathlineto{\pgfqpoint{14.450332in}{8.413188in}}%
\pgfpathlineto{\pgfqpoint{14.452614in}{8.409774in}}%
\pgfpathlineto{\pgfqpoint{14.454896in}{8.400929in}}%
\pgfpathlineto{\pgfqpoint{14.457178in}{8.387739in}}%
\pgfpathlineto{\pgfqpoint{14.466305in}{8.400774in}}%
\pgfpathlineto{\pgfqpoint{14.468587in}{8.387584in}}%
\pgfpathlineto{\pgfqpoint{14.470869in}{8.384015in}}%
\pgfpathlineto{\pgfqpoint{14.473151in}{8.451207in}}%
\pgfpathlineto{\pgfqpoint{14.475433in}{8.444689in}}%
\pgfpathlineto{\pgfqpoint{14.482279in}{8.457569in}}%
\pgfpathlineto{\pgfqpoint{14.484561in}{8.460362in}}%
\pgfpathlineto{\pgfqpoint{14.486842in}{8.458655in}}%
\pgfpathlineto{\pgfqpoint{14.489124in}{8.455862in}}%
\pgfpathlineto{\pgfqpoint{14.491406in}{8.439569in}}%
\pgfpathlineto{\pgfqpoint{14.498252in}{8.439258in}}%
\pgfpathlineto{\pgfqpoint{14.500534in}{8.443914in}}%
\pgfpathlineto{\pgfqpoint{14.502816in}{8.433362in}}%
\pgfpathlineto{\pgfqpoint{14.505098in}{8.425758in}}%
\pgfpathlineto{\pgfqpoint{14.507380in}{8.425603in}}%
\pgfpathlineto{\pgfqpoint{14.514225in}{8.423275in}}%
\pgfpathlineto{\pgfqpoint{14.518789in}{8.430568in}}%
\pgfpathlineto{\pgfqpoint{14.521071in}{8.419551in}}%
\pgfpathlineto{\pgfqpoint{14.523353in}{8.413188in}}%
\pgfpathlineto{\pgfqpoint{14.530199in}{8.425758in}}%
\pgfpathlineto{\pgfqpoint{14.532481in}{8.422809in}}%
\pgfpathlineto{\pgfqpoint{14.534763in}{8.396274in}}%
\pgfpathlineto{\pgfqpoint{14.537044in}{8.396429in}}%
\pgfpathlineto{\pgfqpoint{14.539326in}{8.402791in}}%
\pgfpathlineto{\pgfqpoint{14.546172in}{8.405430in}}%
\pgfpathlineto{\pgfqpoint{14.548454in}{8.408843in}}%
\pgfpathlineto{\pgfqpoint{14.550736in}{8.407602in}}%
\pgfpathlineto{\pgfqpoint{14.553018in}{8.412413in}}%
\pgfpathlineto{\pgfqpoint{14.555300in}{8.412723in}}%
\pgfpathlineto{\pgfqpoint{14.564427in}{8.407136in}}%
\pgfpathlineto{\pgfqpoint{14.566709in}{8.404964in}}%
\pgfpathlineto{\pgfqpoint{14.568991in}{8.424982in}}%
\pgfpathlineto{\pgfqpoint{14.571273in}{8.428241in}}%
\pgfpathlineto{\pgfqpoint{14.578119in}{8.435379in}}%
\pgfpathlineto{\pgfqpoint{14.580401in}{8.433517in}}%
\pgfpathlineto{\pgfqpoint{14.582683in}{8.447948in}}%
\pgfpathlineto{\pgfqpoint{14.584964in}{8.450121in}}%
\pgfpathlineto{\pgfqpoint{14.587246in}{8.455552in}}%
\pgfpathlineto{\pgfqpoint{14.594092in}{8.453379in}}%
\pgfpathlineto{\pgfqpoint{14.596374in}{8.459742in}}%
\pgfpathlineto{\pgfqpoint{14.598656in}{8.463466in}}%
\pgfpathlineto{\pgfqpoint{14.600938in}{8.461294in}}%
\pgfpathlineto{\pgfqpoint{14.603220in}{8.472621in}}%
\pgfpathlineto{\pgfqpoint{14.610065in}{8.479139in}}%
\pgfpathlineto{\pgfqpoint{14.612347in}{8.488605in}}%
\pgfpathlineto{\pgfqpoint{14.614629in}{8.484105in}}%
\pgfpathlineto{\pgfqpoint{14.616911in}{8.484570in}}%
\pgfpathlineto{\pgfqpoint{14.619193in}{8.484260in}}%
\pgfpathlineto{\pgfqpoint{14.626039in}{8.495277in}}%
\pgfpathlineto{\pgfqpoint{14.628321in}{8.497140in}}%
\pgfpathlineto{\pgfqpoint{14.630603in}{8.510485in}}%
\pgfpathlineto{\pgfqpoint{14.632885in}{8.505364in}}%
\pgfpathlineto{\pgfqpoint{14.635166in}{8.514519in}}%
\pgfpathlineto{\pgfqpoint{14.642012in}{8.527399in}}%
\pgfpathlineto{\pgfqpoint{14.646576in}{8.529106in}}%
\pgfpathlineto{\pgfqpoint{14.648858in}{8.511726in}}%
\pgfpathlineto{\pgfqpoint{14.651140in}{8.520727in}}%
\pgfpathlineto{\pgfqpoint{14.657986in}{8.520261in}}%
\pgfpathlineto{\pgfqpoint{14.660267in}{8.518089in}}%
\pgfpathlineto{\pgfqpoint{14.664831in}{8.536400in}}%
\pgfpathlineto{\pgfqpoint{14.667113in}{8.535003in}}%
\pgfpathlineto{\pgfqpoint{14.673959in}{8.534072in}}%
\pgfpathlineto{\pgfqpoint{14.676241in}{8.538572in}}%
\pgfpathlineto{\pgfqpoint{14.678523in}{8.544779in}}%
\pgfpathlineto{\pgfqpoint{14.680805in}{8.536089in}}%
\pgfpathlineto{\pgfqpoint{14.683086in}{8.539658in}}%
\pgfpathlineto{\pgfqpoint{14.689932in}{8.530813in}}%
\pgfpathlineto{\pgfqpoint{14.692214in}{8.536865in}}%
\pgfpathlineto{\pgfqpoint{14.694496in}{8.535313in}}%
\pgfpathlineto{\pgfqpoint{14.696778in}{8.513433in}}%
\pgfpathlineto{\pgfqpoint{14.699060in}{8.527710in}}%
\pgfpathlineto{\pgfqpoint{14.705906in}{8.534227in}}%
\pgfpathlineto{\pgfqpoint{14.708187in}{8.534382in}}%
\pgfpathlineto{\pgfqpoint{14.710469in}{8.535313in}}%
\pgfpathlineto{\pgfqpoint{14.712751in}{8.538882in}}%
\pgfpathlineto{\pgfqpoint{14.715033in}{8.545400in}}%
\pgfpathlineto{\pgfqpoint{14.721879in}{8.543538in}}%
\pgfpathlineto{\pgfqpoint{14.724161in}{8.544934in}}%
\pgfpathlineto{\pgfqpoint{14.726443in}{8.541210in}}%
\pgfpathlineto{\pgfqpoint{14.728725in}{8.524141in}}%
\pgfpathlineto{\pgfqpoint{14.731007in}{8.520106in}}%
\pgfpathlineto{\pgfqpoint{14.737852in}{8.537796in}}%
\pgfpathlineto{\pgfqpoint{14.740134in}{8.557814in}}%
\pgfpathlineto{\pgfqpoint{14.742416in}{8.566814in}}%
\pgfpathlineto{\pgfqpoint{14.744698in}{8.547262in}}%
\pgfpathlineto{\pgfqpoint{14.746980in}{8.536555in}}%
\pgfpathlineto{\pgfqpoint{14.758389in}{8.535003in}}%
\pgfpathlineto{\pgfqpoint{14.762953in}{8.537796in}}%
\pgfpathlineto{\pgfqpoint{14.772081in}{8.537020in}}%
\pgfpathlineto{\pgfqpoint{14.774363in}{8.541210in}}%
\pgfpathlineto{\pgfqpoint{14.776645in}{8.547262in}}%
\pgfpathlineto{\pgfqpoint{14.778927in}{8.547883in}}%
\pgfpathlineto{\pgfqpoint{14.788054in}{8.535468in}}%
\pgfpathlineto{\pgfqpoint{14.790336in}{8.534227in}}%
\pgfpathlineto{\pgfqpoint{14.792618in}{8.521813in}}%
\pgfpathlineto{\pgfqpoint{14.794900in}{8.519485in}}%
\pgfpathlineto{\pgfqpoint{14.801746in}{8.541210in}}%
\pgfpathlineto{\pgfqpoint{14.804028in}{8.554090in}}%
\pgfpathlineto{\pgfqpoint{14.806309in}{8.555486in}}%
\pgfpathlineto{\pgfqpoint{14.808591in}{8.548659in}}%
\pgfpathlineto{\pgfqpoint{14.810873in}{8.560762in}}%
\pgfpathlineto{\pgfqpoint{14.817719in}{8.573953in}}%
\pgfpathlineto{\pgfqpoint{14.820001in}{8.591022in}}%
\pgfpathlineto{\pgfqpoint{14.822283in}{8.582487in}}%
\pgfpathlineto{\pgfqpoint{14.826847in}{8.582022in}}%
\pgfpathlineto{\pgfqpoint{14.833692in}{8.579073in}}%
\pgfpathlineto{\pgfqpoint{14.835974in}{8.586367in}}%
\pgfpathlineto{\pgfqpoint{14.840538in}{8.608247in}}%
\pgfpathlineto{\pgfqpoint{14.842820in}{8.613057in}}%
\pgfpathlineto{\pgfqpoint{14.849666in}{8.614299in}}%
\pgfpathlineto{\pgfqpoint{14.851948in}{8.627644in}}%
\pgfpathlineto{\pgfqpoint{14.854229in}{8.621282in}}%
\pgfpathlineto{\pgfqpoint{14.858793in}{8.634782in}}%
\pgfpathlineto{\pgfqpoint{14.865639in}{8.636955in}}%
\pgfpathlineto{\pgfqpoint{14.867921in}{8.639903in}}%
\pgfpathlineto{\pgfqpoint{14.870203in}{8.640989in}}%
\pgfpathlineto{\pgfqpoint{14.872485in}{8.635713in}}%
\pgfpathlineto{\pgfqpoint{14.874767in}{8.654645in}}%
\pgfpathlineto{\pgfqpoint{14.883894in}{8.636955in}}%
\pgfpathlineto{\pgfqpoint{14.886176in}{8.642851in}}%
\pgfpathlineto{\pgfqpoint{14.888458in}{8.640058in}}%
\pgfpathlineto{\pgfqpoint{14.890740in}{8.643007in}}%
\pgfpathlineto{\pgfqpoint{14.897586in}{8.647352in}}%
\pgfpathlineto{\pgfqpoint{14.899868in}{8.669697in}}%
\pgfpathlineto{\pgfqpoint{14.902150in}{8.664732in}}%
\pgfpathlineto{\pgfqpoint{14.904431in}{8.697474in}}%
\pgfpathlineto{\pgfqpoint{14.906713in}{8.699026in}}%
\pgfpathlineto{\pgfqpoint{14.913559in}{8.688008in}}%
\pgfpathlineto{\pgfqpoint{14.915841in}{8.694836in}}%
\pgfpathlineto{\pgfqpoint{14.918123in}{8.699026in}}%
\pgfpathlineto{\pgfqpoint{14.920405in}{8.701819in}}%
\pgfpathlineto{\pgfqpoint{14.922687in}{8.708647in}}%
\pgfpathlineto{\pgfqpoint{14.929532in}{8.706474in}}%
\pgfpathlineto{\pgfqpoint{14.931814in}{8.692664in}}%
\pgfpathlineto{\pgfqpoint{14.934096in}{8.688939in}}%
\pgfpathlineto{\pgfqpoint{14.936378in}{8.668145in}}%
\pgfpathlineto{\pgfqpoint{14.938660in}{8.664576in}}%
\pgfpathlineto{\pgfqpoint{14.945506in}{8.670318in}}%
\pgfpathlineto{\pgfqpoint{14.947788in}{8.668301in}}%
\pgfpathlineto{\pgfqpoint{14.950070in}{8.660697in}}%
\pgfpathlineto{\pgfqpoint{14.952351in}{8.665197in}}%
\pgfpathlineto{\pgfqpoint{14.954633in}{8.667214in}}%
\pgfpathlineto{\pgfqpoint{14.961479in}{8.671094in}}%
\pgfpathlineto{\pgfqpoint{14.963761in}{8.678077in}}%
\pgfpathlineto{\pgfqpoint{14.966043in}{8.669232in}}%
\pgfpathlineto{\pgfqpoint{14.970607in}{8.663956in}}%
\pgfpathlineto{\pgfqpoint{14.977452in}{8.663800in}}%
\pgfpathlineto{\pgfqpoint{14.979734in}{8.691267in}}%
\pgfpathlineto{\pgfqpoint{14.984298in}{8.728044in}}%
\pgfpathlineto{\pgfqpoint{14.986580in}{8.730372in}}%
\pgfpathlineto{\pgfqpoint{14.993426in}{8.741389in}}%
\pgfpathlineto{\pgfqpoint{14.995708in}{8.743096in}}%
\pgfpathlineto{\pgfqpoint{14.997990in}{8.735648in}}%
\pgfpathlineto{\pgfqpoint{15.000272in}{8.741234in}}%
\pgfpathlineto{\pgfqpoint{15.002553in}{8.740769in}}%
\pgfpathlineto{\pgfqpoint{15.009399in}{8.747596in}}%
\pgfpathlineto{\pgfqpoint{15.011681in}{8.753183in}}%
\pgfpathlineto{\pgfqpoint{15.013963in}{8.729130in}}%
\pgfpathlineto{\pgfqpoint{15.016245in}{8.719509in}}%
\pgfpathlineto{\pgfqpoint{15.018527in}{8.740303in}}%
\pgfpathlineto{\pgfqpoint{15.025373in}{8.757838in}}%
\pgfpathlineto{\pgfqpoint{15.029936in}{8.740148in}}%
\pgfpathlineto{\pgfqpoint{15.032218in}{8.739993in}}%
\pgfpathlineto{\pgfqpoint{15.034500in}{8.743562in}}%
\pgfpathlineto{\pgfqpoint{15.043628in}{8.741079in}}%
\pgfpathlineto{\pgfqpoint{15.048192in}{8.758149in}}%
\pgfpathlineto{\pgfqpoint{15.050473in}{8.752252in}}%
\pgfpathlineto{\pgfqpoint{15.050473in}{8.752252in}}%
\pgfusepath{stroke}%
\end{pgfscope}%
\begin{pgfscope}%
\pgfsetrectcap%
\pgfsetmiterjoin%
\pgfsetlinewidth{0.803000pt}%
\definecolor{currentstroke}{rgb}{1.000000,1.000000,1.000000}%
\pgfsetstrokecolor{currentstroke}%
\pgfsetdash{}{0pt}%
\pgfpathmoveto{\pgfqpoint{9.810417in}{7.879268in}}%
\pgfpathlineto{\pgfqpoint{9.810417in}{8.800000in}}%
\pgfusepath{stroke}%
\end{pgfscope}%
\begin{pgfscope}%
\pgfsetrectcap%
\pgfsetmiterjoin%
\pgfsetlinewidth{0.803000pt}%
\definecolor{currentstroke}{rgb}{1.000000,1.000000,1.000000}%
\pgfsetstrokecolor{currentstroke}%
\pgfsetdash{}{0pt}%
\pgfpathmoveto{\pgfqpoint{15.300000in}{7.879268in}}%
\pgfpathlineto{\pgfqpoint{15.300000in}{8.800000in}}%
\pgfusepath{stroke}%
\end{pgfscope}%
\begin{pgfscope}%
\pgfsetrectcap%
\pgfsetmiterjoin%
\pgfsetlinewidth{0.803000pt}%
\definecolor{currentstroke}{rgb}{1.000000,1.000000,1.000000}%
\pgfsetstrokecolor{currentstroke}%
\pgfsetdash{}{0pt}%
\pgfpathmoveto{\pgfqpoint{9.810417in}{7.879268in}}%
\pgfpathlineto{\pgfqpoint{15.300000in}{7.879268in}}%
\pgfusepath{stroke}%
\end{pgfscope}%
\begin{pgfscope}%
\pgfsetrectcap%
\pgfsetmiterjoin%
\pgfsetlinewidth{0.803000pt}%
\definecolor{currentstroke}{rgb}{1.000000,1.000000,1.000000}%
\pgfsetstrokecolor{currentstroke}%
\pgfsetdash{}{0pt}%
\pgfpathmoveto{\pgfqpoint{9.810417in}{8.800000in}}%
\pgfpathlineto{\pgfqpoint{15.300000in}{8.800000in}}%
\pgfusepath{stroke}%
\end{pgfscope}%
\begin{pgfscope}%
\definecolor{textcolor}{rgb}{0.150000,0.150000,0.150000}%
\pgfsetstrokecolor{textcolor}%
\pgfsetfillcolor{textcolor}%
\pgftext[x=12.555208in,y=8.883333in,,base]{\color{textcolor}\rmfamily\fontsize{12.000000}{14.400000}\selectfont AXP}%
\end{pgfscope}%
\begin{pgfscope}%
\pgfsetbuttcap%
\pgfsetmiterjoin%
\definecolor{currentfill}{rgb}{0.917647,0.917647,0.949020}%
\pgfsetfillcolor{currentfill}%
\pgfsetlinewidth{0.000000pt}%
\definecolor{currentstroke}{rgb}{0.000000,0.000000,0.000000}%
\pgfsetstrokecolor{currentstroke}%
\pgfsetstrokeopacity{0.000000}%
\pgfsetdash{}{0pt}%
\pgfpathmoveto{\pgfqpoint{2.125000in}{6.221951in}}%
\pgfpathlineto{\pgfqpoint{7.614583in}{6.221951in}}%
\pgfpathlineto{\pgfqpoint{7.614583in}{7.142683in}}%
\pgfpathlineto{\pgfqpoint{2.125000in}{7.142683in}}%
\pgfpathclose%
\pgfusepath{fill}%
\end{pgfscope}%
\begin{pgfscope}%
\pgfpathrectangle{\pgfqpoint{2.125000in}{6.221951in}}{\pgfqpoint{5.489583in}{0.920732in}}%
\pgfusepath{clip}%
\pgfsetroundcap%
\pgfsetroundjoin%
\pgfsetlinewidth{0.803000pt}%
\definecolor{currentstroke}{rgb}{1.000000,1.000000,1.000000}%
\pgfsetstrokecolor{currentstroke}%
\pgfsetdash{}{0pt}%
\pgfpathmoveto{\pgfqpoint{2.369963in}{6.221951in}}%
\pgfpathlineto{\pgfqpoint{2.369963in}{7.142683in}}%
\pgfusepath{stroke}%
\end{pgfscope}%
\begin{pgfscope}%
\definecolor{textcolor}{rgb}{0.150000,0.150000,0.150000}%
\pgfsetstrokecolor{textcolor}%
\pgfsetfillcolor{textcolor}%
\pgftext[x=2.369963in,y=6.124729in,,top]{\color{textcolor}\rmfamily\fontsize{10.000000}{12.000000}\selectfont 2012}%
\end{pgfscope}%
\begin{pgfscope}%
\pgfpathrectangle{\pgfqpoint{2.125000in}{6.221951in}}{\pgfqpoint{5.489583in}{0.920732in}}%
\pgfusepath{clip}%
\pgfsetroundcap%
\pgfsetroundjoin%
\pgfsetlinewidth{0.803000pt}%
\definecolor{currentstroke}{rgb}{1.000000,1.000000,1.000000}%
\pgfsetstrokecolor{currentstroke}%
\pgfsetdash{}{0pt}%
\pgfpathmoveto{\pgfqpoint{3.205141in}{6.221951in}}%
\pgfpathlineto{\pgfqpoint{3.205141in}{7.142683in}}%
\pgfusepath{stroke}%
\end{pgfscope}%
\begin{pgfscope}%
\definecolor{textcolor}{rgb}{0.150000,0.150000,0.150000}%
\pgfsetstrokecolor{textcolor}%
\pgfsetfillcolor{textcolor}%
\pgftext[x=3.205141in,y=6.124729in,,top]{\color{textcolor}\rmfamily\fontsize{10.000000}{12.000000}\selectfont 2013}%
\end{pgfscope}%
\begin{pgfscope}%
\pgfpathrectangle{\pgfqpoint{2.125000in}{6.221951in}}{\pgfqpoint{5.489583in}{0.920732in}}%
\pgfusepath{clip}%
\pgfsetroundcap%
\pgfsetroundjoin%
\pgfsetlinewidth{0.803000pt}%
\definecolor{currentstroke}{rgb}{1.000000,1.000000,1.000000}%
\pgfsetstrokecolor{currentstroke}%
\pgfsetdash{}{0pt}%
\pgfpathmoveto{\pgfqpoint{4.038037in}{6.221951in}}%
\pgfpathlineto{\pgfqpoint{4.038037in}{7.142683in}}%
\pgfusepath{stroke}%
\end{pgfscope}%
\begin{pgfscope}%
\definecolor{textcolor}{rgb}{0.150000,0.150000,0.150000}%
\pgfsetstrokecolor{textcolor}%
\pgfsetfillcolor{textcolor}%
\pgftext[x=4.038037in,y=6.124729in,,top]{\color{textcolor}\rmfamily\fontsize{10.000000}{12.000000}\selectfont 2014}%
\end{pgfscope}%
\begin{pgfscope}%
\pgfpathrectangle{\pgfqpoint{2.125000in}{6.221951in}}{\pgfqpoint{5.489583in}{0.920732in}}%
\pgfusepath{clip}%
\pgfsetroundcap%
\pgfsetroundjoin%
\pgfsetlinewidth{0.803000pt}%
\definecolor{currentstroke}{rgb}{1.000000,1.000000,1.000000}%
\pgfsetstrokecolor{currentstroke}%
\pgfsetdash{}{0pt}%
\pgfpathmoveto{\pgfqpoint{4.870933in}{6.221951in}}%
\pgfpathlineto{\pgfqpoint{4.870933in}{7.142683in}}%
\pgfusepath{stroke}%
\end{pgfscope}%
\begin{pgfscope}%
\definecolor{textcolor}{rgb}{0.150000,0.150000,0.150000}%
\pgfsetstrokecolor{textcolor}%
\pgfsetfillcolor{textcolor}%
\pgftext[x=4.870933in,y=6.124729in,,top]{\color{textcolor}\rmfamily\fontsize{10.000000}{12.000000}\selectfont 2015}%
\end{pgfscope}%
\begin{pgfscope}%
\pgfpathrectangle{\pgfqpoint{2.125000in}{6.221951in}}{\pgfqpoint{5.489583in}{0.920732in}}%
\pgfusepath{clip}%
\pgfsetroundcap%
\pgfsetroundjoin%
\pgfsetlinewidth{0.803000pt}%
\definecolor{currentstroke}{rgb}{1.000000,1.000000,1.000000}%
\pgfsetstrokecolor{currentstroke}%
\pgfsetdash{}{0pt}%
\pgfpathmoveto{\pgfqpoint{5.703829in}{6.221951in}}%
\pgfpathlineto{\pgfqpoint{5.703829in}{7.142683in}}%
\pgfusepath{stroke}%
\end{pgfscope}%
\begin{pgfscope}%
\definecolor{textcolor}{rgb}{0.150000,0.150000,0.150000}%
\pgfsetstrokecolor{textcolor}%
\pgfsetfillcolor{textcolor}%
\pgftext[x=5.703829in,y=6.124729in,,top]{\color{textcolor}\rmfamily\fontsize{10.000000}{12.000000}\selectfont 2016}%
\end{pgfscope}%
\begin{pgfscope}%
\pgfpathrectangle{\pgfqpoint{2.125000in}{6.221951in}}{\pgfqpoint{5.489583in}{0.920732in}}%
\pgfusepath{clip}%
\pgfsetroundcap%
\pgfsetroundjoin%
\pgfsetlinewidth{0.803000pt}%
\definecolor{currentstroke}{rgb}{1.000000,1.000000,1.000000}%
\pgfsetstrokecolor{currentstroke}%
\pgfsetdash{}{0pt}%
\pgfpathmoveto{\pgfqpoint{6.539007in}{6.221951in}}%
\pgfpathlineto{\pgfqpoint{6.539007in}{7.142683in}}%
\pgfusepath{stroke}%
\end{pgfscope}%
\begin{pgfscope}%
\definecolor{textcolor}{rgb}{0.150000,0.150000,0.150000}%
\pgfsetstrokecolor{textcolor}%
\pgfsetfillcolor{textcolor}%
\pgftext[x=6.539007in,y=6.124729in,,top]{\color{textcolor}\rmfamily\fontsize{10.000000}{12.000000}\selectfont 2017}%
\end{pgfscope}%
\begin{pgfscope}%
\pgfpathrectangle{\pgfqpoint{2.125000in}{6.221951in}}{\pgfqpoint{5.489583in}{0.920732in}}%
\pgfusepath{clip}%
\pgfsetroundcap%
\pgfsetroundjoin%
\pgfsetlinewidth{0.803000pt}%
\definecolor{currentstroke}{rgb}{1.000000,1.000000,1.000000}%
\pgfsetstrokecolor{currentstroke}%
\pgfsetdash{}{0pt}%
\pgfpathmoveto{\pgfqpoint{7.371903in}{6.221951in}}%
\pgfpathlineto{\pgfqpoint{7.371903in}{7.142683in}}%
\pgfusepath{stroke}%
\end{pgfscope}%
\begin{pgfscope}%
\definecolor{textcolor}{rgb}{0.150000,0.150000,0.150000}%
\pgfsetstrokecolor{textcolor}%
\pgfsetfillcolor{textcolor}%
\pgftext[x=7.371903in,y=6.124729in,,top]{\color{textcolor}\rmfamily\fontsize{10.000000}{12.000000}\selectfont 2018}%
\end{pgfscope}%
\begin{pgfscope}%
\pgfpathrectangle{\pgfqpoint{2.125000in}{6.221951in}}{\pgfqpoint{5.489583in}{0.920732in}}%
\pgfusepath{clip}%
\pgfsetroundcap%
\pgfsetroundjoin%
\pgfsetlinewidth{0.803000pt}%
\definecolor{currentstroke}{rgb}{1.000000,1.000000,1.000000}%
\pgfsetstrokecolor{currentstroke}%
\pgfsetdash{}{0pt}%
\pgfpathmoveto{\pgfqpoint{2.125000in}{6.333231in}}%
\pgfpathlineto{\pgfqpoint{7.614583in}{6.333231in}}%
\pgfusepath{stroke}%
\end{pgfscope}%
\begin{pgfscope}%
\definecolor{textcolor}{rgb}{0.150000,0.150000,0.150000}%
\pgfsetstrokecolor{textcolor}%
\pgfsetfillcolor{textcolor}%
\pgftext[x=1.851047in,y=6.280470in,left,base]{\color{textcolor}\rmfamily\fontsize{10.000000}{12.000000}\selectfont 15}%
\end{pgfscope}%
\begin{pgfscope}%
\pgfpathrectangle{\pgfqpoint{2.125000in}{6.221951in}}{\pgfqpoint{5.489583in}{0.920732in}}%
\pgfusepath{clip}%
\pgfsetroundcap%
\pgfsetroundjoin%
\pgfsetlinewidth{0.803000pt}%
\definecolor{currentstroke}{rgb}{1.000000,1.000000,1.000000}%
\pgfsetstrokecolor{currentstroke}%
\pgfsetdash{}{0pt}%
\pgfpathmoveto{\pgfqpoint{2.125000in}{6.610945in}}%
\pgfpathlineto{\pgfqpoint{7.614583in}{6.610945in}}%
\pgfusepath{stroke}%
\end{pgfscope}%
\begin{pgfscope}%
\definecolor{textcolor}{rgb}{0.150000,0.150000,0.150000}%
\pgfsetstrokecolor{textcolor}%
\pgfsetfillcolor{textcolor}%
\pgftext[x=1.851047in,y=6.558183in,left,base]{\color{textcolor}\rmfamily\fontsize{10.000000}{12.000000}\selectfont 20}%
\end{pgfscope}%
\begin{pgfscope}%
\pgfpathrectangle{\pgfqpoint{2.125000in}{6.221951in}}{\pgfqpoint{5.489583in}{0.920732in}}%
\pgfusepath{clip}%
\pgfsetroundcap%
\pgfsetroundjoin%
\pgfsetlinewidth{0.803000pt}%
\definecolor{currentstroke}{rgb}{1.000000,1.000000,1.000000}%
\pgfsetstrokecolor{currentstroke}%
\pgfsetdash{}{0pt}%
\pgfpathmoveto{\pgfqpoint{2.125000in}{6.888658in}}%
\pgfpathlineto{\pgfqpoint{7.614583in}{6.888658in}}%
\pgfusepath{stroke}%
\end{pgfscope}%
\begin{pgfscope}%
\definecolor{textcolor}{rgb}{0.150000,0.150000,0.150000}%
\pgfsetstrokecolor{textcolor}%
\pgfsetfillcolor{textcolor}%
\pgftext[x=1.851047in,y=6.835897in,left,base]{\color{textcolor}\rmfamily\fontsize{10.000000}{12.000000}\selectfont 25}%
\end{pgfscope}%
\begin{pgfscope}%
\pgfpathrectangle{\pgfqpoint{2.125000in}{6.221951in}}{\pgfqpoint{5.489583in}{0.920732in}}%
\pgfusepath{clip}%
\pgfsetroundcap%
\pgfsetroundjoin%
\pgfsetlinewidth{1.505625pt}%
\definecolor{currentstroke}{rgb}{0.121569,0.466667,0.705882}%
\pgfsetstrokecolor{currentstroke}%
\pgfsetdash{}{0pt}%
\pgfpathmoveto{\pgfqpoint{2.374527in}{6.266024in}}%
\pgfpathlineto{\pgfqpoint{2.376808in}{6.274356in}}%
\pgfpathlineto{\pgfqpoint{2.379090in}{6.273800in}}%
\pgfpathlineto{\pgfqpoint{2.381372in}{6.278244in}}%
\pgfpathlineto{\pgfqpoint{2.388218in}{6.287131in}}%
\pgfpathlineto{\pgfqpoint{2.390500in}{6.281021in}}%
\pgfpathlineto{\pgfqpoint{2.392782in}{6.287686in}}%
\pgfpathlineto{\pgfqpoint{2.395064in}{6.289908in}}%
\pgfpathlineto{\pgfqpoint{2.397346in}{6.286020in}}%
\pgfpathlineto{\pgfqpoint{2.406473in}{6.282132in}}%
\pgfpathlineto{\pgfqpoint{2.408755in}{6.293796in}}%
\pgfpathlineto{\pgfqpoint{2.411037in}{6.298795in}}%
\pgfpathlineto{\pgfqpoint{2.413319in}{6.298795in}}%
\pgfpathlineto{\pgfqpoint{2.420165in}{6.290463in}}%
\pgfpathlineto{\pgfqpoint{2.422447in}{6.286020in}}%
\pgfpathlineto{\pgfqpoint{2.424728in}{6.298239in}}%
\pgfpathlineto{\pgfqpoint{2.427010in}{6.295462in}}%
\pgfpathlineto{\pgfqpoint{2.436138in}{6.288797in}}%
\pgfpathlineto{\pgfqpoint{2.438420in}{6.280465in}}%
\pgfpathlineto{\pgfqpoint{2.440702in}{6.283243in}}%
\pgfpathlineto{\pgfqpoint{2.442984in}{6.282132in}}%
\pgfpathlineto{\pgfqpoint{2.445266in}{6.293796in}}%
\pgfpathlineto{\pgfqpoint{2.452111in}{6.294907in}}%
\pgfpathlineto{\pgfqpoint{2.454393in}{6.300461in}}%
\pgfpathlineto{\pgfqpoint{2.456675in}{6.302683in}}%
\pgfpathlineto{\pgfqpoint{2.458957in}{6.298239in}}%
\pgfpathlineto{\pgfqpoint{2.461239in}{6.287686in}}%
\pgfpathlineto{\pgfqpoint{2.468085in}{6.295462in}}%
\pgfpathlineto{\pgfqpoint{2.470367in}{6.290463in}}%
\pgfpathlineto{\pgfqpoint{2.472649in}{6.282687in}}%
\pgfpathlineto{\pgfqpoint{2.477212in}{6.304349in}}%
\pgfpathlineto{\pgfqpoint{2.486340in}{6.309903in}}%
\pgfpathlineto{\pgfqpoint{2.488622in}{6.308792in}}%
\pgfpathlineto{\pgfqpoint{2.490904in}{6.312680in}}%
\pgfpathlineto{\pgfqpoint{2.493186in}{6.309903in}}%
\pgfpathlineto{\pgfqpoint{2.500031in}{6.302683in}}%
\pgfpathlineto{\pgfqpoint{2.502313in}{6.306571in}}%
\pgfpathlineto{\pgfqpoint{2.504595in}{6.302127in}}%
\pgfpathlineto{\pgfqpoint{2.506877in}{6.304904in}}%
\pgfpathlineto{\pgfqpoint{2.509159in}{6.298795in}}%
\pgfpathlineto{\pgfqpoint{2.516005in}{6.293240in}}%
\pgfpathlineto{\pgfqpoint{2.518287in}{6.275467in}}%
\pgfpathlineto{\pgfqpoint{2.522850in}{6.301016in}}%
\pgfpathlineto{\pgfqpoint{2.525132in}{6.301572in}}%
\pgfpathlineto{\pgfqpoint{2.531978in}{6.305460in}}%
\pgfpathlineto{\pgfqpoint{2.534260in}{6.324900in}}%
\pgfpathlineto{\pgfqpoint{2.536542in}{6.333231in}}%
\pgfpathlineto{\pgfqpoint{2.538824in}{6.348783in}}%
\pgfpathlineto{\pgfqpoint{2.541106in}{6.350449in}}%
\pgfpathlineto{\pgfqpoint{2.547951in}{6.351005in}}%
\pgfpathlineto{\pgfqpoint{2.550233in}{6.344895in}}%
\pgfpathlineto{\pgfqpoint{2.552515in}{6.344895in}}%
\pgfpathlineto{\pgfqpoint{2.554797in}{6.335453in}}%
\pgfpathlineto{\pgfqpoint{2.557079in}{6.332676in}}%
\pgfpathlineto{\pgfqpoint{2.563925in}{6.343784in}}%
\pgfpathlineto{\pgfqpoint{2.566207in}{6.343784in}}%
\pgfpathlineto{\pgfqpoint{2.570771in}{6.339896in}}%
\pgfpathlineto{\pgfqpoint{2.573052in}{6.344895in}}%
\pgfpathlineto{\pgfqpoint{2.579898in}{6.342673in}}%
\pgfpathlineto{\pgfqpoint{2.582180in}{6.340452in}}%
\pgfpathlineto{\pgfqpoint{2.586744in}{6.320456in}}%
\pgfpathlineto{\pgfqpoint{2.595871in}{6.308237in}}%
\pgfpathlineto{\pgfqpoint{2.598153in}{6.288797in}}%
\pgfpathlineto{\pgfqpoint{2.602717in}{6.312680in}}%
\pgfpathlineto{\pgfqpoint{2.604999in}{6.294907in}}%
\pgfpathlineto{\pgfqpoint{2.611845in}{6.295462in}}%
\pgfpathlineto{\pgfqpoint{2.614127in}{6.314347in}}%
\pgfpathlineto{\pgfqpoint{2.616409in}{6.303793in}}%
\pgfpathlineto{\pgfqpoint{2.618691in}{6.305460in}}%
\pgfpathlineto{\pgfqpoint{2.620972in}{6.314902in}}%
\pgfpathlineto{\pgfqpoint{2.627818in}{6.302683in}}%
\pgfpathlineto{\pgfqpoint{2.630100in}{6.322678in}}%
\pgfpathlineto{\pgfqpoint{2.632382in}{6.318790in}}%
\pgfpathlineto{\pgfqpoint{2.636946in}{6.332676in}}%
\pgfpathlineto{\pgfqpoint{2.643792in}{6.324344in}}%
\pgfpathlineto{\pgfqpoint{2.646073in}{6.333231in}}%
\pgfpathlineto{\pgfqpoint{2.648355in}{6.332120in}}%
\pgfpathlineto{\pgfqpoint{2.650637in}{6.325455in}}%
\pgfpathlineto{\pgfqpoint{2.652919in}{6.314347in}}%
\pgfpathlineto{\pgfqpoint{2.659765in}{6.313236in}}%
\pgfpathlineto{\pgfqpoint{2.662047in}{6.310459in}}%
\pgfpathlineto{\pgfqpoint{2.664329in}{6.296017in}}%
\pgfpathlineto{\pgfqpoint{2.666611in}{6.303793in}}%
\pgfpathlineto{\pgfqpoint{2.668893in}{6.300461in}}%
\pgfpathlineto{\pgfqpoint{2.675738in}{6.282687in}}%
\pgfpathlineto{\pgfqpoint{2.678020in}{6.274356in}}%
\pgfpathlineto{\pgfqpoint{2.680302in}{6.299905in}}%
\pgfpathlineto{\pgfqpoint{2.682584in}{6.294907in}}%
\pgfpathlineto{\pgfqpoint{2.684866in}{6.297684in}}%
\pgfpathlineto{\pgfqpoint{2.693993in}{6.307126in}}%
\pgfpathlineto{\pgfqpoint{2.696275in}{6.307126in}}%
\pgfpathlineto{\pgfqpoint{2.698557in}{6.310459in}}%
\pgfpathlineto{\pgfqpoint{2.700839in}{6.308237in}}%
\pgfpathlineto{\pgfqpoint{2.709967in}{6.314347in}}%
\pgfpathlineto{\pgfqpoint{2.712249in}{6.301572in}}%
\pgfpathlineto{\pgfqpoint{2.714531in}{6.303793in}}%
\pgfpathlineto{\pgfqpoint{2.716813in}{6.280465in}}%
\pgfpathlineto{\pgfqpoint{2.723658in}{6.263803in}}%
\pgfpathlineto{\pgfqpoint{2.725940in}{6.267691in}}%
\pgfpathlineto{\pgfqpoint{2.728222in}{6.294907in}}%
\pgfpathlineto{\pgfqpoint{2.730504in}{6.299905in}}%
\pgfpathlineto{\pgfqpoint{2.732786in}{6.308237in}}%
\pgfpathlineto{\pgfqpoint{2.739632in}{6.304349in}}%
\pgfpathlineto{\pgfqpoint{2.741914in}{6.319901in}}%
\pgfpathlineto{\pgfqpoint{2.744195in}{6.315457in}}%
\pgfpathlineto{\pgfqpoint{2.748759in}{6.344340in}}%
\pgfpathlineto{\pgfqpoint{2.755605in}{6.333786in}}%
\pgfpathlineto{\pgfqpoint{2.757887in}{6.344340in}}%
\pgfpathlineto{\pgfqpoint{2.760169in}{6.348228in}}%
\pgfpathlineto{\pgfqpoint{2.762451in}{6.331565in}}%
\pgfpathlineto{\pgfqpoint{2.764733in}{6.343229in}}%
\pgfpathlineto{\pgfqpoint{2.771578in}{6.331009in}}%
\pgfpathlineto{\pgfqpoint{2.776142in}{6.357114in}}%
\pgfpathlineto{\pgfqpoint{2.778424in}{6.359892in}}%
\pgfpathlineto{\pgfqpoint{2.780706in}{6.387108in}}%
\pgfpathlineto{\pgfqpoint{2.787552in}{6.372111in}}%
\pgfpathlineto{\pgfqpoint{2.794397in}{6.365446in}}%
\pgfpathlineto{\pgfqpoint{2.796679in}{6.351560in}}%
\pgfpathlineto{\pgfqpoint{2.803525in}{6.353226in}}%
\pgfpathlineto{\pgfqpoint{2.805807in}{6.335453in}}%
\pgfpathlineto{\pgfqpoint{2.808089in}{6.337674in}}%
\pgfpathlineto{\pgfqpoint{2.810371in}{6.327677in}}%
\pgfpathlineto{\pgfqpoint{2.812653in}{6.341562in}}%
\pgfpathlineto{\pgfqpoint{2.819498in}{6.333786in}}%
\pgfpathlineto{\pgfqpoint{2.824062in}{6.344895in}}%
\pgfpathlineto{\pgfqpoint{2.826344in}{6.342673in}}%
\pgfpathlineto{\pgfqpoint{2.835472in}{6.355448in}}%
\pgfpathlineto{\pgfqpoint{2.837754in}{6.349894in}}%
\pgfpathlineto{\pgfqpoint{2.840036in}{6.351560in}}%
\pgfpathlineto{\pgfqpoint{2.842317in}{6.375444in}}%
\pgfpathlineto{\pgfqpoint{2.844599in}{6.390440in}}%
\pgfpathlineto{\pgfqpoint{2.856009in}{6.382664in}}%
\pgfpathlineto{\pgfqpoint{2.858291in}{6.373777in}}%
\pgfpathlineto{\pgfqpoint{2.860573in}{6.392106in}}%
\pgfpathlineto{\pgfqpoint{2.867418in}{6.392662in}}%
\pgfpathlineto{\pgfqpoint{2.869700in}{6.399327in}}%
\pgfpathlineto{\pgfqpoint{2.871982in}{6.394328in}}%
\pgfpathlineto{\pgfqpoint{2.876546in}{6.398216in}}%
\pgfpathlineto{\pgfqpoint{2.883392in}{6.393773in}}%
\pgfpathlineto{\pgfqpoint{2.885674in}{6.391551in}}%
\pgfpathlineto{\pgfqpoint{2.887956in}{6.392106in}}%
\pgfpathlineto{\pgfqpoint{2.890237in}{6.395994in}}%
\pgfpathlineto{\pgfqpoint{2.892519in}{6.393773in}}%
\pgfpathlineto{\pgfqpoint{2.899365in}{6.390995in}}%
\pgfpathlineto{\pgfqpoint{2.903929in}{6.384886in}}%
\pgfpathlineto{\pgfqpoint{2.906211in}{6.378776in}}%
\pgfpathlineto{\pgfqpoint{2.908493in}{6.385441in}}%
\pgfpathlineto{\pgfqpoint{2.915338in}{6.387663in}}%
\pgfpathlineto{\pgfqpoint{2.917620in}{6.385997in}}%
\pgfpathlineto{\pgfqpoint{2.919902in}{6.386552in}}%
\pgfpathlineto{\pgfqpoint{2.922184in}{6.378776in}}%
\pgfpathlineto{\pgfqpoint{2.924466in}{6.381553in}}%
\pgfpathlineto{\pgfqpoint{2.933594in}{6.373222in}}%
\pgfpathlineto{\pgfqpoint{2.935876in}{6.379332in}}%
\pgfpathlineto{\pgfqpoint{2.938158in}{6.407103in}}%
\pgfpathlineto{\pgfqpoint{2.940439in}{6.419322in}}%
\pgfpathlineto{\pgfqpoint{2.947285in}{6.414323in}}%
\pgfpathlineto{\pgfqpoint{2.949567in}{6.419322in}}%
\pgfpathlineto{\pgfqpoint{2.951849in}{6.432097in}}%
\pgfpathlineto{\pgfqpoint{2.954131in}{6.437651in}}%
\pgfpathlineto{\pgfqpoint{2.956413in}{6.440984in}}%
\pgfpathlineto{\pgfqpoint{2.963258in}{6.438762in}}%
\pgfpathlineto{\pgfqpoint{2.970104in}{6.462090in}}%
\pgfpathlineto{\pgfqpoint{2.972386in}{6.466534in}}%
\pgfpathlineto{\pgfqpoint{2.981514in}{6.457091in}}%
\pgfpathlineto{\pgfqpoint{2.983796in}{6.448204in}}%
\pgfpathlineto{\pgfqpoint{2.986078in}{6.474865in}}%
\pgfpathlineto{\pgfqpoint{2.988359in}{6.474310in}}%
\pgfpathlineto{\pgfqpoint{2.995205in}{6.478198in}}%
\pgfpathlineto{\pgfqpoint{2.997487in}{6.477642in}}%
\pgfpathlineto{\pgfqpoint{2.999769in}{6.482641in}}%
\pgfpathlineto{\pgfqpoint{3.002051in}{6.484307in}}%
\pgfpathlineto{\pgfqpoint{3.004333in}{6.491528in}}%
\pgfpathlineto{\pgfqpoint{3.011179in}{6.483196in}}%
\pgfpathlineto{\pgfqpoint{3.013460in}{6.470422in}}%
\pgfpathlineto{\pgfqpoint{3.015742in}{6.462090in}}%
\pgfpathlineto{\pgfqpoint{3.018024in}{6.465423in}}%
\pgfpathlineto{\pgfqpoint{3.020306in}{6.464312in}}%
\pgfpathlineto{\pgfqpoint{3.027152in}{6.470977in}}%
\pgfpathlineto{\pgfqpoint{3.029434in}{6.470977in}}%
\pgfpathlineto{\pgfqpoint{3.031716in}{6.482641in}}%
\pgfpathlineto{\pgfqpoint{3.033998in}{6.478198in}}%
\pgfpathlineto{\pgfqpoint{3.036280in}{6.444872in}}%
\pgfpathlineto{\pgfqpoint{3.043125in}{6.430986in}}%
\pgfpathlineto{\pgfqpoint{3.045407in}{6.412657in}}%
\pgfpathlineto{\pgfqpoint{3.049971in}{6.412102in}}%
\pgfpathlineto{\pgfqpoint{3.052253in}{6.405437in}}%
\pgfpathlineto{\pgfqpoint{3.063662in}{6.403215in}}%
\pgfpathlineto{\pgfqpoint{3.065944in}{6.415434in}}%
\pgfpathlineto{\pgfqpoint{3.068226in}{6.414323in}}%
\pgfpathlineto{\pgfqpoint{3.075072in}{6.418211in}}%
\pgfpathlineto{\pgfqpoint{3.077354in}{6.425987in}}%
\pgfpathlineto{\pgfqpoint{3.079636in}{6.406547in}}%
\pgfpathlineto{\pgfqpoint{3.081918in}{6.395994in}}%
\pgfpathlineto{\pgfqpoint{3.084200in}{6.400993in}}%
\pgfpathlineto{\pgfqpoint{3.091045in}{6.395994in}}%
\pgfpathlineto{\pgfqpoint{3.093327in}{6.387108in}}%
\pgfpathlineto{\pgfqpoint{3.095609in}{6.358225in}}%
\pgfpathlineto{\pgfqpoint{3.097891in}{6.360447in}}%
\pgfpathlineto{\pgfqpoint{3.100173in}{6.364335in}}%
\pgfpathlineto{\pgfqpoint{3.107019in}{6.385997in}}%
\pgfpathlineto{\pgfqpoint{3.109301in}{6.384330in}}%
\pgfpathlineto{\pgfqpoint{3.111582in}{6.387108in}}%
\pgfpathlineto{\pgfqpoint{3.116146in}{6.402659in}}%
\pgfpathlineto{\pgfqpoint{3.122992in}{6.403215in}}%
\pgfpathlineto{\pgfqpoint{3.125274in}{6.395439in}}%
\pgfpathlineto{\pgfqpoint{3.127556in}{6.407103in}}%
\pgfpathlineto{\pgfqpoint{3.132120in}{6.406547in}}%
\pgfpathlineto{\pgfqpoint{3.138965in}{6.393217in}}%
\pgfpathlineto{\pgfqpoint{3.141247in}{6.394883in}}%
\pgfpathlineto{\pgfqpoint{3.143529in}{6.410435in}}%
\pgfpathlineto{\pgfqpoint{3.145811in}{6.416545in}}%
\pgfpathlineto{\pgfqpoint{3.148093in}{6.420433in}}%
\pgfpathlineto{\pgfqpoint{3.154939in}{6.417656in}}%
\pgfpathlineto{\pgfqpoint{3.157221in}{6.422655in}}%
\pgfpathlineto{\pgfqpoint{3.159502in}{6.434319in}}%
\pgfpathlineto{\pgfqpoint{3.161784in}{6.427654in}}%
\pgfpathlineto{\pgfqpoint{3.164066in}{6.427654in}}%
\pgfpathlineto{\pgfqpoint{3.170912in}{6.440984in}}%
\pgfpathlineto{\pgfqpoint{3.173194in}{6.430431in}}%
\pgfpathlineto{\pgfqpoint{3.175476in}{6.400993in}}%
\pgfpathlineto{\pgfqpoint{3.177758in}{6.410991in}}%
\pgfpathlineto{\pgfqpoint{3.180040in}{6.403770in}}%
\pgfpathlineto{\pgfqpoint{3.191449in}{6.399327in}}%
\pgfpathlineto{\pgfqpoint{3.193731in}{6.395439in}}%
\pgfpathlineto{\pgfqpoint{3.196013in}{6.384886in}}%
\pgfpathlineto{\pgfqpoint{3.207423in}{6.423766in}}%
\pgfpathlineto{\pgfqpoint{3.209704in}{6.413213in}}%
\pgfpathlineto{\pgfqpoint{3.211986in}{6.417656in}}%
\pgfpathlineto{\pgfqpoint{3.218832in}{6.414879in}}%
\pgfpathlineto{\pgfqpoint{3.221114in}{6.404881in}}%
\pgfpathlineto{\pgfqpoint{3.223396in}{6.407103in}}%
\pgfpathlineto{\pgfqpoint{3.225678in}{6.416545in}}%
\pgfpathlineto{\pgfqpoint{3.227960in}{6.414879in}}%
\pgfpathlineto{\pgfqpoint{3.234805in}{6.414323in}}%
\pgfpathlineto{\pgfqpoint{3.237087in}{6.417656in}}%
\pgfpathlineto{\pgfqpoint{3.239369in}{6.414323in}}%
\pgfpathlineto{\pgfqpoint{3.241651in}{6.422099in}}%
\pgfpathlineto{\pgfqpoint{3.243933in}{6.454314in}}%
\pgfpathlineto{\pgfqpoint{3.253061in}{6.452648in}}%
\pgfpathlineto{\pgfqpoint{3.255343in}{6.449871in}}%
\pgfpathlineto{\pgfqpoint{3.257624in}{6.454314in}}%
\pgfpathlineto{\pgfqpoint{3.259906in}{6.464867in}}%
\pgfpathlineto{\pgfqpoint{3.266752in}{6.473754in}}%
\pgfpathlineto{\pgfqpoint{3.269034in}{6.473754in}}%
\pgfpathlineto{\pgfqpoint{3.271316in}{6.462090in}}%
\pgfpathlineto{\pgfqpoint{3.273598in}{6.464312in}}%
\pgfpathlineto{\pgfqpoint{3.275880in}{6.479308in}}%
\pgfpathlineto{\pgfqpoint{3.282725in}{6.465978in}}%
\pgfpathlineto{\pgfqpoint{3.285007in}{6.475976in}}%
\pgfpathlineto{\pgfqpoint{3.287289in}{6.471532in}}%
\pgfpathlineto{\pgfqpoint{3.289571in}{6.473199in}}%
\pgfpathlineto{\pgfqpoint{3.291853in}{6.473754in}}%
\pgfpathlineto{\pgfqpoint{3.298699in}{6.472088in}}%
\pgfpathlineto{\pgfqpoint{3.300981in}{6.477642in}}%
\pgfpathlineto{\pgfqpoint{3.303263in}{6.512634in}}%
\pgfpathlineto{\pgfqpoint{3.305545in}{6.513189in}}%
\pgfpathlineto{\pgfqpoint{3.307826in}{6.508191in}}%
\pgfpathlineto{\pgfqpoint{3.316954in}{6.528186in}}%
\pgfpathlineto{\pgfqpoint{3.319236in}{6.513189in}}%
\pgfpathlineto{\pgfqpoint{3.321518in}{6.514856in}}%
\pgfpathlineto{\pgfqpoint{3.323800in}{6.520965in}}%
\pgfpathlineto{\pgfqpoint{3.330645in}{6.495416in}}%
\pgfpathlineto{\pgfqpoint{3.332927in}{6.505969in}}%
\pgfpathlineto{\pgfqpoint{3.335209in}{6.519855in}}%
\pgfpathlineto{\pgfqpoint{3.337491in}{6.513189in}}%
\pgfpathlineto{\pgfqpoint{3.339773in}{6.512079in}}%
\pgfpathlineto{\pgfqpoint{3.346619in}{6.515411in}}%
\pgfpathlineto{\pgfqpoint{3.348901in}{6.529297in}}%
\pgfpathlineto{\pgfqpoint{3.351183in}{6.533185in}}%
\pgfpathlineto{\pgfqpoint{3.353465in}{6.533185in}}%
\pgfpathlineto{\pgfqpoint{3.355746in}{6.537628in}}%
\pgfpathlineto{\pgfqpoint{3.362592in}{6.530963in}}%
\pgfpathlineto{\pgfqpoint{3.364874in}{6.521521in}}%
\pgfpathlineto{\pgfqpoint{3.367156in}{6.525409in}}%
\pgfpathlineto{\pgfqpoint{3.369438in}{6.533740in}}%
\pgfpathlineto{\pgfqpoint{3.371720in}{6.523187in}}%
\pgfpathlineto{\pgfqpoint{3.378566in}{6.514856in}}%
\pgfpathlineto{\pgfqpoint{3.380847in}{6.517633in}}%
\pgfpathlineto{\pgfqpoint{3.383129in}{6.523743in}}%
\pgfpathlineto{\pgfqpoint{3.385411in}{6.516522in}}%
\pgfpathlineto{\pgfqpoint{3.387693in}{6.519855in}}%
\pgfpathlineto{\pgfqpoint{3.394539in}{6.514300in}}%
\pgfpathlineto{\pgfqpoint{3.396821in}{6.508746in}}%
\pgfpathlineto{\pgfqpoint{3.399103in}{6.508191in}}%
\pgfpathlineto{\pgfqpoint{3.401385in}{6.508746in}}%
\pgfpathlineto{\pgfqpoint{3.410512in}{6.507080in}}%
\pgfpathlineto{\pgfqpoint{3.412794in}{6.518744in}}%
\pgfpathlineto{\pgfqpoint{3.415076in}{6.503747in}}%
\pgfpathlineto{\pgfqpoint{3.417358in}{6.507080in}}%
\pgfpathlineto{\pgfqpoint{3.419640in}{6.500970in}}%
\pgfpathlineto{\pgfqpoint{3.426486in}{6.508746in}}%
\pgfpathlineto{\pgfqpoint{3.428767in}{6.506524in}}%
\pgfpathlineto{\pgfqpoint{3.431049in}{6.529297in}}%
\pgfpathlineto{\pgfqpoint{3.433331in}{6.529297in}}%
\pgfpathlineto{\pgfqpoint{3.435613in}{6.523743in}}%
\pgfpathlineto{\pgfqpoint{3.442459in}{6.495416in}}%
\pgfpathlineto{\pgfqpoint{3.444741in}{6.508191in}}%
\pgfpathlineto{\pgfqpoint{3.447023in}{6.493194in}}%
\pgfpathlineto{\pgfqpoint{3.449305in}{6.489306in}}%
\pgfpathlineto{\pgfqpoint{3.451587in}{6.449315in}}%
\pgfpathlineto{\pgfqpoint{3.458432in}{6.431542in}}%
\pgfpathlineto{\pgfqpoint{3.460714in}{6.438207in}}%
\pgfpathlineto{\pgfqpoint{3.462996in}{6.458202in}}%
\pgfpathlineto{\pgfqpoint{3.465278in}{6.458202in}}%
\pgfpathlineto{\pgfqpoint{3.467560in}{6.469311in}}%
\pgfpathlineto{\pgfqpoint{3.476688in}{6.472643in}}%
\pgfpathlineto{\pgfqpoint{3.478969in}{6.466534in}}%
\pgfpathlineto{\pgfqpoint{3.481251in}{6.474310in}}%
\pgfpathlineto{\pgfqpoint{3.483533in}{6.484863in}}%
\pgfpathlineto{\pgfqpoint{3.490379in}{6.485418in}}%
\pgfpathlineto{\pgfqpoint{3.492661in}{6.489862in}}%
\pgfpathlineto{\pgfqpoint{3.494943in}{6.504303in}}%
\pgfpathlineto{\pgfqpoint{3.497225in}{6.494305in}}%
\pgfpathlineto{\pgfqpoint{3.499507in}{6.499304in}}%
\pgfpathlineto{\pgfqpoint{3.506352in}{6.497082in}}%
\pgfpathlineto{\pgfqpoint{3.508634in}{6.504303in}}%
\pgfpathlineto{\pgfqpoint{3.510916in}{6.514300in}}%
\pgfpathlineto{\pgfqpoint{3.513198in}{6.515411in}}%
\pgfpathlineto{\pgfqpoint{3.515480in}{6.523743in}}%
\pgfpathlineto{\pgfqpoint{3.522326in}{6.528741in}}%
\pgfpathlineto{\pgfqpoint{3.524608in}{6.532629in}}%
\pgfpathlineto{\pgfqpoint{3.526889in}{6.541516in}}%
\pgfpathlineto{\pgfqpoint{3.529171in}{6.532629in}}%
\pgfpathlineto{\pgfqpoint{3.531453in}{6.527075in}}%
\pgfpathlineto{\pgfqpoint{3.540581in}{6.529852in}}%
\pgfpathlineto{\pgfqpoint{3.542863in}{6.531519in}}%
\pgfpathlineto{\pgfqpoint{3.545145in}{6.529852in}}%
\pgfpathlineto{\pgfqpoint{3.547427in}{6.517633in}}%
\pgfpathlineto{\pgfqpoint{3.554272in}{6.531519in}}%
\pgfpathlineto{\pgfqpoint{3.556554in}{6.532629in}}%
\pgfpathlineto{\pgfqpoint{3.558836in}{6.517633in}}%
\pgfpathlineto{\pgfqpoint{3.561118in}{6.520410in}}%
\pgfpathlineto{\pgfqpoint{3.563400in}{6.541516in}}%
\pgfpathlineto{\pgfqpoint{3.570246in}{6.537628in}}%
\pgfpathlineto{\pgfqpoint{3.572528in}{6.529297in}}%
\pgfpathlineto{\pgfqpoint{3.574810in}{6.525409in}}%
\pgfpathlineto{\pgfqpoint{3.577091in}{6.533185in}}%
\pgfpathlineto{\pgfqpoint{3.579373in}{6.526520in}}%
\pgfpathlineto{\pgfqpoint{3.586219in}{6.537628in}}%
\pgfpathlineto{\pgfqpoint{3.588501in}{6.562067in}}%
\pgfpathlineto{\pgfqpoint{3.590783in}{6.546515in}}%
\pgfpathlineto{\pgfqpoint{3.593065in}{6.522632in}}%
\pgfpathlineto{\pgfqpoint{3.595347in}{6.527631in}}%
\pgfpathlineto{\pgfqpoint{3.602192in}{6.508746in}}%
\pgfpathlineto{\pgfqpoint{3.606756in}{6.522632in}}%
\pgfpathlineto{\pgfqpoint{3.609038in}{6.525964in}}%
\pgfpathlineto{\pgfqpoint{3.611320in}{6.520410in}}%
\pgfpathlineto{\pgfqpoint{3.618166in}{6.526520in}}%
\pgfpathlineto{\pgfqpoint{3.620448in}{6.507635in}}%
\pgfpathlineto{\pgfqpoint{3.622730in}{6.507635in}}%
\pgfpathlineto{\pgfqpoint{3.627293in}{6.522076in}}%
\pgfpathlineto{\pgfqpoint{3.634139in}{6.525964in}}%
\pgfpathlineto{\pgfqpoint{3.636421in}{6.539295in}}%
\pgfpathlineto{\pgfqpoint{3.638703in}{6.535407in}}%
\pgfpathlineto{\pgfqpoint{3.640985in}{6.553180in}}%
\pgfpathlineto{\pgfqpoint{3.643267in}{6.545404in}}%
\pgfpathlineto{\pgfqpoint{3.650112in}{6.539295in}}%
\pgfpathlineto{\pgfqpoint{3.652394in}{6.530963in}}%
\pgfpathlineto{\pgfqpoint{3.656958in}{6.539295in}}%
\pgfpathlineto{\pgfqpoint{3.659240in}{6.587617in}}%
\pgfpathlineto{\pgfqpoint{3.666086in}{6.593726in}}%
\pgfpathlineto{\pgfqpoint{3.668368in}{6.587061in}}%
\pgfpathlineto{\pgfqpoint{3.670650in}{6.583173in}}%
\pgfpathlineto{\pgfqpoint{3.672932in}{6.585950in}}%
\pgfpathlineto{\pgfqpoint{3.675213in}{6.584284in}}%
\pgfpathlineto{\pgfqpoint{3.682059in}{6.577064in}}%
\pgfpathlineto{\pgfqpoint{3.684341in}{6.577064in}}%
\pgfpathlineto{\pgfqpoint{3.686623in}{6.572065in}}%
\pgfpathlineto{\pgfqpoint{3.688905in}{6.583173in}}%
\pgfpathlineto{\pgfqpoint{3.691187in}{6.586506in}}%
\pgfpathlineto{\pgfqpoint{3.698032in}{6.578730in}}%
\pgfpathlineto{\pgfqpoint{3.700314in}{6.569288in}}%
\pgfpathlineto{\pgfqpoint{3.702596in}{6.570954in}}%
\pgfpathlineto{\pgfqpoint{3.704878in}{6.570398in}}%
\pgfpathlineto{\pgfqpoint{3.707160in}{6.566510in}}%
\pgfpathlineto{\pgfqpoint{3.714006in}{6.567621in}}%
\pgfpathlineto{\pgfqpoint{3.716288in}{6.564844in}}%
\pgfpathlineto{\pgfqpoint{3.718570in}{6.558735in}}%
\pgfpathlineto{\pgfqpoint{3.723133in}{6.553736in}}%
\pgfpathlineto{\pgfqpoint{3.729979in}{6.549292in}}%
\pgfpathlineto{\pgfqpoint{3.732261in}{6.543183in}}%
\pgfpathlineto{\pgfqpoint{3.734543in}{6.538739in}}%
\pgfpathlineto{\pgfqpoint{3.736825in}{6.545960in}}%
\pgfpathlineto{\pgfqpoint{3.739107in}{6.545960in}}%
\pgfpathlineto{\pgfqpoint{3.745953in}{6.538739in}}%
\pgfpathlineto{\pgfqpoint{3.748234in}{6.519855in}}%
\pgfpathlineto{\pgfqpoint{3.750516in}{6.520410in}}%
\pgfpathlineto{\pgfqpoint{3.752798in}{6.516522in}}%
\pgfpathlineto{\pgfqpoint{3.755080in}{6.518188in}}%
\pgfpathlineto{\pgfqpoint{3.764208in}{6.514300in}}%
\pgfpathlineto{\pgfqpoint{3.766490in}{6.519299in}}%
\pgfpathlineto{\pgfqpoint{3.771054in}{6.518744in}}%
\pgfpathlineto{\pgfqpoint{3.777899in}{6.528741in}}%
\pgfpathlineto{\pgfqpoint{3.780181in}{6.549848in}}%
\pgfpathlineto{\pgfqpoint{3.782463in}{6.559845in}}%
\pgfpathlineto{\pgfqpoint{3.784745in}{6.549292in}}%
\pgfpathlineto{\pgfqpoint{3.787027in}{6.545960in}}%
\pgfpathlineto{\pgfqpoint{3.793873in}{6.562067in}}%
\pgfpathlineto{\pgfqpoint{3.796154in}{6.575397in}}%
\pgfpathlineto{\pgfqpoint{3.798436in}{6.593726in}}%
\pgfpathlineto{\pgfqpoint{3.800718in}{6.584284in}}%
\pgfpathlineto{\pgfqpoint{3.803000in}{6.564289in}}%
\pgfpathlineto{\pgfqpoint{3.809846in}{6.576508in}}%
\pgfpathlineto{\pgfqpoint{3.812128in}{6.578174in}}%
\pgfpathlineto{\pgfqpoint{3.814410in}{6.574286in}}%
\pgfpathlineto{\pgfqpoint{3.816692in}{6.574842in}}%
\pgfpathlineto{\pgfqpoint{3.818974in}{6.565955in}}%
\pgfpathlineto{\pgfqpoint{3.825819in}{6.559290in}}%
\pgfpathlineto{\pgfqpoint{3.828101in}{6.571509in}}%
\pgfpathlineto{\pgfqpoint{3.830383in}{6.578730in}}%
\pgfpathlineto{\pgfqpoint{3.832665in}{6.568177in}}%
\pgfpathlineto{\pgfqpoint{3.834947in}{6.565955in}}%
\pgfpathlineto{\pgfqpoint{3.841793in}{6.561512in}}%
\pgfpathlineto{\pgfqpoint{3.844075in}{6.549292in}}%
\pgfpathlineto{\pgfqpoint{3.846356in}{6.544849in}}%
\pgfpathlineto{\pgfqpoint{3.848638in}{6.574842in}}%
\pgfpathlineto{\pgfqpoint{3.850920in}{6.581507in}}%
\pgfpathlineto{\pgfqpoint{3.857766in}{6.580952in}}%
\pgfpathlineto{\pgfqpoint{3.860048in}{6.572065in}}%
\pgfpathlineto{\pgfqpoint{3.862330in}{6.579841in}}%
\pgfpathlineto{\pgfqpoint{3.864612in}{6.594282in}}%
\pgfpathlineto{\pgfqpoint{3.866894in}{6.632606in}}%
\pgfpathlineto{\pgfqpoint{3.873739in}{6.658711in}}%
\pgfpathlineto{\pgfqpoint{3.876021in}{6.653713in}}%
\pgfpathlineto{\pgfqpoint{3.878303in}{6.639271in}}%
\pgfpathlineto{\pgfqpoint{3.880585in}{6.649825in}}%
\pgfpathlineto{\pgfqpoint{3.882867in}{6.647047in}}%
\pgfpathlineto{\pgfqpoint{3.889713in}{6.656490in}}%
\pgfpathlineto{\pgfqpoint{3.894276in}{6.668709in}}%
\pgfpathlineto{\pgfqpoint{3.896558in}{6.658711in}}%
\pgfpathlineto{\pgfqpoint{3.898840in}{6.676485in}}%
\pgfpathlineto{\pgfqpoint{3.905686in}{6.671486in}}%
\pgfpathlineto{\pgfqpoint{3.907968in}{6.670931in}}%
\pgfpathlineto{\pgfqpoint{3.910250in}{6.692592in}}%
\pgfpathlineto{\pgfqpoint{3.912532in}{6.679262in}}%
\pgfpathlineto{\pgfqpoint{3.914814in}{6.699258in}}%
\pgfpathlineto{\pgfqpoint{3.921659in}{6.697591in}}%
\pgfpathlineto{\pgfqpoint{3.923941in}{6.699258in}}%
\pgfpathlineto{\pgfqpoint{3.926223in}{6.703701in}}%
\pgfpathlineto{\pgfqpoint{3.928505in}{6.696480in}}%
\pgfpathlineto{\pgfqpoint{3.930787in}{6.705923in}}%
\pgfpathlineto{\pgfqpoint{3.937633in}{6.706478in}}%
\pgfpathlineto{\pgfqpoint{3.939915in}{6.698147in}}%
\pgfpathlineto{\pgfqpoint{3.944478in}{6.693148in}}%
\pgfpathlineto{\pgfqpoint{3.946760in}{6.700368in}}%
\pgfpathlineto{\pgfqpoint{3.953606in}{6.684816in}}%
\pgfpathlineto{\pgfqpoint{3.958170in}{6.689260in}}%
\pgfpathlineto{\pgfqpoint{3.962734in}{6.682039in}}%
\pgfpathlineto{\pgfqpoint{3.969579in}{6.682039in}}%
\pgfpathlineto{\pgfqpoint{3.971861in}{6.677596in}}%
\pgfpathlineto{\pgfqpoint{3.974143in}{6.680929in}}%
\pgfpathlineto{\pgfqpoint{3.976425in}{6.672597in}}%
\pgfpathlineto{\pgfqpoint{3.978707in}{6.694259in}}%
\pgfpathlineto{\pgfqpoint{3.985553in}{6.705367in}}%
\pgfpathlineto{\pgfqpoint{3.987835in}{6.703146in}}%
\pgfpathlineto{\pgfqpoint{3.990117in}{6.678151in}}%
\pgfpathlineto{\pgfqpoint{3.992398in}{6.676485in}}%
\pgfpathlineto{\pgfqpoint{3.994680in}{6.689815in}}%
\pgfpathlineto{\pgfqpoint{4.003808in}{6.698147in}}%
\pgfpathlineto{\pgfqpoint{4.006090in}{6.714810in}}%
\pgfpathlineto{\pgfqpoint{4.008372in}{6.720919in}}%
\pgfpathlineto{\pgfqpoint{4.010654in}{6.722586in}}%
\pgfpathlineto{\pgfqpoint{4.017499in}{6.724252in}}%
\pgfpathlineto{\pgfqpoint{4.019781in}{6.733694in}}%
\pgfpathlineto{\pgfqpoint{4.024345in}{6.743692in}}%
\pgfpathlineto{\pgfqpoint{4.026627in}{6.743692in}}%
\pgfpathlineto{\pgfqpoint{4.033473in}{6.746469in}}%
\pgfpathlineto{\pgfqpoint{4.035755in}{6.752579in}}%
\pgfpathlineto{\pgfqpoint{4.040319in}{6.728695in}}%
\pgfpathlineto{\pgfqpoint{4.042600in}{6.728140in}}%
\pgfpathlineto{\pgfqpoint{4.049446in}{6.718142in}}%
\pgfpathlineto{\pgfqpoint{4.051728in}{6.719253in}}%
\pgfpathlineto{\pgfqpoint{4.054010in}{6.715920in}}%
\pgfpathlineto{\pgfqpoint{4.056292in}{6.716476in}}%
\pgfpathlineto{\pgfqpoint{4.058574in}{6.704812in}}%
\pgfpathlineto{\pgfqpoint{4.065419in}{6.694259in}}%
\pgfpathlineto{\pgfqpoint{4.067701in}{6.705367in}}%
\pgfpathlineto{\pgfqpoint{4.069983in}{6.722030in}}%
\pgfpathlineto{\pgfqpoint{4.072265in}{6.715365in}}%
\pgfpathlineto{\pgfqpoint{4.074547in}{6.687594in}}%
\pgfpathlineto{\pgfqpoint{4.083675in}{6.674819in}}%
\pgfpathlineto{\pgfqpoint{4.085957in}{6.661489in}}%
\pgfpathlineto{\pgfqpoint{4.088239in}{6.653713in}}%
\pgfpathlineto{\pgfqpoint{4.090520in}{6.614833in}}%
\pgfpathlineto{\pgfqpoint{4.097366in}{6.620387in}}%
\pgfpathlineto{\pgfqpoint{4.099648in}{6.637605in}}%
\pgfpathlineto{\pgfqpoint{4.101930in}{6.630385in}}%
\pgfpathlineto{\pgfqpoint{4.104212in}{6.639271in}}%
\pgfpathlineto{\pgfqpoint{4.106494in}{6.623164in}}%
\pgfpathlineto{\pgfqpoint{4.113340in}{6.588172in}}%
\pgfpathlineto{\pgfqpoint{4.115621in}{6.598170in}}%
\pgfpathlineto{\pgfqpoint{4.117903in}{6.595948in}}%
\pgfpathlineto{\pgfqpoint{4.120185in}{6.614833in}}%
\pgfpathlineto{\pgfqpoint{4.122467in}{6.625941in}}%
\pgfpathlineto{\pgfqpoint{4.129313in}{6.619276in}}%
\pgfpathlineto{\pgfqpoint{4.131595in}{6.636494in}}%
\pgfpathlineto{\pgfqpoint{4.133877in}{6.634828in}}%
\pgfpathlineto{\pgfqpoint{4.136159in}{6.637050in}}%
\pgfpathlineto{\pgfqpoint{4.138441in}{6.650380in}}%
\pgfpathlineto{\pgfqpoint{4.147568in}{6.646492in}}%
\pgfpathlineto{\pgfqpoint{4.149850in}{6.634828in}}%
\pgfpathlineto{\pgfqpoint{4.152132in}{6.632606in}}%
\pgfpathlineto{\pgfqpoint{4.154414in}{6.624275in}}%
\pgfpathlineto{\pgfqpoint{4.161260in}{6.639827in}}%
\pgfpathlineto{\pgfqpoint{4.163541in}{6.639271in}}%
\pgfpathlineto{\pgfqpoint{4.165823in}{6.640382in}}%
\pgfpathlineto{\pgfqpoint{4.168105in}{6.649269in}}%
\pgfpathlineto{\pgfqpoint{4.170387in}{6.648158in}}%
\pgfpathlineto{\pgfqpoint{4.177233in}{6.632606in}}%
\pgfpathlineto{\pgfqpoint{4.179515in}{6.656490in}}%
\pgfpathlineto{\pgfqpoint{4.184079in}{6.682039in}}%
\pgfpathlineto{\pgfqpoint{4.186361in}{6.678151in}}%
\pgfpathlineto{\pgfqpoint{4.193206in}{6.673708in}}%
\pgfpathlineto{\pgfqpoint{4.197770in}{6.661489in}}%
\pgfpathlineto{\pgfqpoint{4.200052in}{6.642049in}}%
\pgfpathlineto{\pgfqpoint{4.202334in}{6.632051in}}%
\pgfpathlineto{\pgfqpoint{4.209180in}{6.646492in}}%
\pgfpathlineto{\pgfqpoint{4.211462in}{6.656490in}}%
\pgfpathlineto{\pgfqpoint{4.213743in}{6.639827in}}%
\pgfpathlineto{\pgfqpoint{4.216025in}{6.639271in}}%
\pgfpathlineto{\pgfqpoint{4.218307in}{6.644826in}}%
\pgfpathlineto{\pgfqpoint{4.225153in}{6.645381in}}%
\pgfpathlineto{\pgfqpoint{4.227435in}{6.658711in}}%
\pgfpathlineto{\pgfqpoint{4.229717in}{6.654823in}}%
\pgfpathlineto{\pgfqpoint{4.231999in}{6.663710in}}%
\pgfpathlineto{\pgfqpoint{4.234281in}{6.666487in}}%
\pgfpathlineto{\pgfqpoint{4.241126in}{6.667043in}}%
\pgfpathlineto{\pgfqpoint{4.243408in}{6.665932in}}%
\pgfpathlineto{\pgfqpoint{4.247972in}{6.682595in}}%
\pgfpathlineto{\pgfqpoint{4.250254in}{6.673153in}}%
\pgfpathlineto{\pgfqpoint{4.257100in}{6.665377in}}%
\pgfpathlineto{\pgfqpoint{4.259382in}{6.660933in}}%
\pgfpathlineto{\pgfqpoint{4.261663in}{6.669820in}}%
\pgfpathlineto{\pgfqpoint{4.263945in}{6.653157in}}%
\pgfpathlineto{\pgfqpoint{4.266227in}{6.646492in}}%
\pgfpathlineto{\pgfqpoint{4.275355in}{6.663710in}}%
\pgfpathlineto{\pgfqpoint{4.277637in}{6.677596in}}%
\pgfpathlineto{\pgfqpoint{4.279919in}{6.697036in}}%
\pgfpathlineto{\pgfqpoint{4.289046in}{6.698702in}}%
\pgfpathlineto{\pgfqpoint{4.291328in}{6.698147in}}%
\pgfpathlineto{\pgfqpoint{4.293610in}{6.690926in}}%
\pgfpathlineto{\pgfqpoint{4.295892in}{6.692592in}}%
\pgfpathlineto{\pgfqpoint{4.298174in}{6.699258in}}%
\pgfpathlineto{\pgfqpoint{4.305020in}{6.707034in}}%
\pgfpathlineto{\pgfqpoint{4.307302in}{6.706478in}}%
\pgfpathlineto{\pgfqpoint{4.309584in}{6.712032in}}%
\pgfpathlineto{\pgfqpoint{4.311865in}{6.706478in}}%
\pgfpathlineto{\pgfqpoint{4.314147in}{6.702590in}}%
\pgfpathlineto{\pgfqpoint{4.320993in}{6.698147in}}%
\pgfpathlineto{\pgfqpoint{4.323275in}{6.680373in}}%
\pgfpathlineto{\pgfqpoint{4.325557in}{6.695925in}}%
\pgfpathlineto{\pgfqpoint{4.327839in}{6.692037in}}%
\pgfpathlineto{\pgfqpoint{4.330121in}{6.690926in}}%
\pgfpathlineto{\pgfqpoint{4.336966in}{6.710366in}}%
\pgfpathlineto{\pgfqpoint{4.339248in}{6.713699in}}%
\pgfpathlineto{\pgfqpoint{4.343812in}{6.699258in}}%
\pgfpathlineto{\pgfqpoint{4.346094in}{6.702035in}}%
\pgfpathlineto{\pgfqpoint{4.352940in}{6.699258in}}%
\pgfpathlineto{\pgfqpoint{4.355222in}{6.685372in}}%
\pgfpathlineto{\pgfqpoint{4.357504in}{6.694259in}}%
\pgfpathlineto{\pgfqpoint{4.371195in}{6.697591in}}%
\pgfpathlineto{\pgfqpoint{4.375759in}{6.705367in}}%
\pgfpathlineto{\pgfqpoint{4.378041in}{6.707589in}}%
\pgfpathlineto{\pgfqpoint{4.384886in}{6.709255in}}%
\pgfpathlineto{\pgfqpoint{4.387168in}{6.707589in}}%
\pgfpathlineto{\pgfqpoint{4.389450in}{6.697036in}}%
\pgfpathlineto{\pgfqpoint{4.391732in}{6.706478in}}%
\pgfpathlineto{\pgfqpoint{4.394014in}{6.725363in}}%
\pgfpathlineto{\pgfqpoint{4.400860in}{6.737027in}}%
\pgfpathlineto{\pgfqpoint{4.403142in}{6.735360in}}%
\pgfpathlineto{\pgfqpoint{4.405424in}{6.723696in}}%
\pgfpathlineto{\pgfqpoint{4.407706in}{6.715365in}}%
\pgfpathlineto{\pgfqpoint{4.409987in}{6.718698in}}%
\pgfpathlineto{\pgfqpoint{4.416833in}{6.709255in}}%
\pgfpathlineto{\pgfqpoint{4.419115in}{6.711477in}}%
\pgfpathlineto{\pgfqpoint{4.421397in}{6.712032in}}%
\pgfpathlineto{\pgfqpoint{4.423679in}{6.723696in}}%
\pgfpathlineto{\pgfqpoint{4.425961in}{6.725918in}}%
\pgfpathlineto{\pgfqpoint{4.435088in}{6.708144in}}%
\pgfpathlineto{\pgfqpoint{4.439652in}{6.694814in}}%
\pgfpathlineto{\pgfqpoint{4.441934in}{6.701479in}}%
\pgfpathlineto{\pgfqpoint{4.448780in}{6.694259in}}%
\pgfpathlineto{\pgfqpoint{4.451062in}{6.699813in}}%
\pgfpathlineto{\pgfqpoint{4.455626in}{6.720919in}}%
\pgfpathlineto{\pgfqpoint{4.464753in}{6.715920in}}%
\pgfpathlineto{\pgfqpoint{4.467035in}{6.698702in}}%
\pgfpathlineto{\pgfqpoint{4.469317in}{6.696480in}}%
\pgfpathlineto{\pgfqpoint{4.471599in}{6.690926in}}%
\pgfpathlineto{\pgfqpoint{4.473881in}{6.706478in}}%
\pgfpathlineto{\pgfqpoint{4.480727in}{6.711477in}}%
\pgfpathlineto{\pgfqpoint{4.483008in}{6.709255in}}%
\pgfpathlineto{\pgfqpoint{4.485290in}{6.728140in}}%
\pgfpathlineto{\pgfqpoint{4.487572in}{6.709255in}}%
\pgfpathlineto{\pgfqpoint{4.496700in}{6.680929in}}%
\pgfpathlineto{\pgfqpoint{4.498982in}{6.682595in}}%
\pgfpathlineto{\pgfqpoint{4.501264in}{6.677596in}}%
\pgfpathlineto{\pgfqpoint{4.503546in}{6.678707in}}%
\pgfpathlineto{\pgfqpoint{4.505828in}{6.672042in}}%
\pgfpathlineto{\pgfqpoint{4.512673in}{6.663155in}}%
\pgfpathlineto{\pgfqpoint{4.514955in}{6.656490in}}%
\pgfpathlineto{\pgfqpoint{4.517237in}{6.665377in}}%
\pgfpathlineto{\pgfqpoint{4.519519in}{6.643159in}}%
\pgfpathlineto{\pgfqpoint{4.521801in}{6.652046in}}%
\pgfpathlineto{\pgfqpoint{4.528647in}{6.648714in}}%
\pgfpathlineto{\pgfqpoint{4.530928in}{6.637050in}}%
\pgfpathlineto{\pgfqpoint{4.533210in}{6.656490in}}%
\pgfpathlineto{\pgfqpoint{4.535492in}{6.658711in}}%
\pgfpathlineto{\pgfqpoint{4.537774in}{6.666487in}}%
\pgfpathlineto{\pgfqpoint{4.544620in}{6.672042in}}%
\pgfpathlineto{\pgfqpoint{4.546902in}{6.663710in}}%
\pgfpathlineto{\pgfqpoint{4.549184in}{6.673708in}}%
\pgfpathlineto{\pgfqpoint{4.551466in}{6.676485in}}%
\pgfpathlineto{\pgfqpoint{4.553748in}{6.665377in}}%
\pgfpathlineto{\pgfqpoint{4.560593in}{6.684816in}}%
\pgfpathlineto{\pgfqpoint{4.562875in}{6.683706in}}%
\pgfpathlineto{\pgfqpoint{4.565157in}{6.698147in}}%
\pgfpathlineto{\pgfqpoint{4.567439in}{6.701479in}}%
\pgfpathlineto{\pgfqpoint{4.569721in}{6.688704in}}%
\pgfpathlineto{\pgfqpoint{4.576567in}{6.690926in}}%
\pgfpathlineto{\pgfqpoint{4.578849in}{6.682039in}}%
\pgfpathlineto{\pgfqpoint{4.581130in}{6.687594in}}%
\pgfpathlineto{\pgfqpoint{4.583412in}{6.682039in}}%
\pgfpathlineto{\pgfqpoint{4.585694in}{6.680929in}}%
\pgfpathlineto{\pgfqpoint{4.594822in}{6.674819in}}%
\pgfpathlineto{\pgfqpoint{4.597104in}{6.679262in}}%
\pgfpathlineto{\pgfqpoint{4.599386in}{6.679818in}}%
\pgfpathlineto{\pgfqpoint{4.601668in}{6.686483in}}%
\pgfpathlineto{\pgfqpoint{4.608513in}{6.685372in}}%
\pgfpathlineto{\pgfqpoint{4.610795in}{6.677041in}}%
\pgfpathlineto{\pgfqpoint{4.613077in}{6.679262in}}%
\pgfpathlineto{\pgfqpoint{4.615359in}{6.682595in}}%
\pgfpathlineto{\pgfqpoint{4.617641in}{6.675930in}}%
\pgfpathlineto{\pgfqpoint{4.624487in}{6.678151in}}%
\pgfpathlineto{\pgfqpoint{4.626769in}{6.691482in}}%
\pgfpathlineto{\pgfqpoint{4.629050in}{6.693703in}}%
\pgfpathlineto{\pgfqpoint{4.631332in}{6.701479in}}%
\pgfpathlineto{\pgfqpoint{4.633614in}{6.704812in}}%
\pgfpathlineto{\pgfqpoint{4.645024in}{6.688704in}}%
\pgfpathlineto{\pgfqpoint{4.647306in}{6.670931in}}%
\pgfpathlineto{\pgfqpoint{4.649588in}{6.674819in}}%
\pgfpathlineto{\pgfqpoint{4.656433in}{6.664821in}}%
\pgfpathlineto{\pgfqpoint{4.658715in}{6.674263in}}%
\pgfpathlineto{\pgfqpoint{4.660997in}{6.653157in}}%
\pgfpathlineto{\pgfqpoint{4.663279in}{6.651491in}}%
\pgfpathlineto{\pgfqpoint{4.665561in}{6.664266in}}%
\pgfpathlineto{\pgfqpoint{4.672407in}{6.655934in}}%
\pgfpathlineto{\pgfqpoint{4.674689in}{6.637050in}}%
\pgfpathlineto{\pgfqpoint{4.676971in}{6.657045in}}%
\pgfpathlineto{\pgfqpoint{4.681534in}{6.612611in}}%
\pgfpathlineto{\pgfqpoint{4.688380in}{6.597614in}}%
\pgfpathlineto{\pgfqpoint{4.692944in}{6.612611in}}%
\pgfpathlineto{\pgfqpoint{4.695226in}{6.611500in}}%
\pgfpathlineto{\pgfqpoint{4.697508in}{6.637605in}}%
\pgfpathlineto{\pgfqpoint{4.704353in}{6.647047in}}%
\pgfpathlineto{\pgfqpoint{4.706635in}{6.666487in}}%
\pgfpathlineto{\pgfqpoint{4.708917in}{6.654823in}}%
\pgfpathlineto{\pgfqpoint{4.713481in}{6.675374in}}%
\pgfpathlineto{\pgfqpoint{4.720327in}{6.669820in}}%
\pgfpathlineto{\pgfqpoint{4.722609in}{6.685927in}}%
\pgfpathlineto{\pgfqpoint{4.724891in}{6.675930in}}%
\pgfpathlineto{\pgfqpoint{4.727172in}{6.676485in}}%
\pgfpathlineto{\pgfqpoint{4.729454in}{6.683150in}}%
\pgfpathlineto{\pgfqpoint{4.736300in}{6.678151in}}%
\pgfpathlineto{\pgfqpoint{4.738582in}{6.678151in}}%
\pgfpathlineto{\pgfqpoint{4.740864in}{6.683706in}}%
\pgfpathlineto{\pgfqpoint{4.743146in}{6.708144in}}%
\pgfpathlineto{\pgfqpoint{4.745428in}{6.710366in}}%
\pgfpathlineto{\pgfqpoint{4.752273in}{6.713143in}}%
\pgfpathlineto{\pgfqpoint{4.754555in}{6.709255in}}%
\pgfpathlineto{\pgfqpoint{4.756837in}{6.715365in}}%
\pgfpathlineto{\pgfqpoint{4.759119in}{6.710922in}}%
\pgfpathlineto{\pgfqpoint{4.761401in}{6.712588in}}%
\pgfpathlineto{\pgfqpoint{4.768247in}{6.719808in}}%
\pgfpathlineto{\pgfqpoint{4.770529in}{6.738138in}}%
\pgfpathlineto{\pgfqpoint{4.775093in}{6.730362in}}%
\pgfpathlineto{\pgfqpoint{4.777374in}{6.737027in}}%
\pgfpathlineto{\pgfqpoint{4.784220in}{6.737582in}}%
\pgfpathlineto{\pgfqpoint{4.786502in}{6.730917in}}%
\pgfpathlineto{\pgfqpoint{4.788784in}{6.731472in}}%
\pgfpathlineto{\pgfqpoint{4.793348in}{6.714254in}}%
\pgfpathlineto{\pgfqpoint{4.800194in}{6.692592in}}%
\pgfpathlineto{\pgfqpoint{4.802475in}{6.693703in}}%
\pgfpathlineto{\pgfqpoint{4.804757in}{6.709255in}}%
\pgfpathlineto{\pgfqpoint{4.807039in}{6.695925in}}%
\pgfpathlineto{\pgfqpoint{4.809321in}{6.692037in}}%
\pgfpathlineto{\pgfqpoint{4.818449in}{6.672597in}}%
\pgfpathlineto{\pgfqpoint{4.820731in}{6.658156in}}%
\pgfpathlineto{\pgfqpoint{4.823013in}{6.664821in}}%
\pgfpathlineto{\pgfqpoint{4.825294in}{6.640938in}}%
\pgfpathlineto{\pgfqpoint{4.834422in}{6.622609in}}%
\pgfpathlineto{\pgfqpoint{4.836704in}{6.630385in}}%
\pgfpathlineto{\pgfqpoint{4.838986in}{6.663155in}}%
\pgfpathlineto{\pgfqpoint{4.841268in}{6.685372in}}%
\pgfpathlineto{\pgfqpoint{4.848114in}{6.689260in}}%
\pgfpathlineto{\pgfqpoint{4.850395in}{6.697591in}}%
\pgfpathlineto{\pgfqpoint{4.852677in}{6.694814in}}%
\pgfpathlineto{\pgfqpoint{4.864087in}{6.689260in}}%
\pgfpathlineto{\pgfqpoint{4.866369in}{6.683150in}}%
\pgfpathlineto{\pgfqpoint{4.868651in}{6.669265in}}%
\pgfpathlineto{\pgfqpoint{4.873215in}{6.659267in}}%
\pgfpathlineto{\pgfqpoint{4.880060in}{6.638161in}}%
\pgfpathlineto{\pgfqpoint{4.882342in}{6.613722in}}%
\pgfpathlineto{\pgfqpoint{4.884624in}{6.614277in}}%
\pgfpathlineto{\pgfqpoint{4.886906in}{6.627607in}}%
\pgfpathlineto{\pgfqpoint{4.889188in}{6.611500in}}%
\pgfpathlineto{\pgfqpoint{4.896034in}{6.609278in}}%
\pgfpathlineto{\pgfqpoint{4.898316in}{6.603724in}}%
\pgfpathlineto{\pgfqpoint{4.900597in}{6.600392in}}%
\pgfpathlineto{\pgfqpoint{4.902879in}{6.590949in}}%
\pgfpathlineto{\pgfqpoint{4.905161in}{6.591505in}}%
\pgfpathlineto{\pgfqpoint{4.914289in}{6.603169in}}%
\pgfpathlineto{\pgfqpoint{4.918853in}{6.623164in}}%
\pgfpathlineto{\pgfqpoint{4.921135in}{6.632606in}}%
\pgfpathlineto{\pgfqpoint{4.927980in}{6.637605in}}%
\pgfpathlineto{\pgfqpoint{4.930262in}{6.628163in}}%
\pgfpathlineto{\pgfqpoint{4.932544in}{6.603169in}}%
\pgfpathlineto{\pgfqpoint{4.934826in}{6.614277in}}%
\pgfpathlineto{\pgfqpoint{4.937108in}{6.605390in}}%
\pgfpathlineto{\pgfqpoint{4.943954in}{6.619832in}}%
\pgfpathlineto{\pgfqpoint{4.946236in}{6.632051in}}%
\pgfpathlineto{\pgfqpoint{4.948517in}{6.617610in}}%
\pgfpathlineto{\pgfqpoint{4.950799in}{6.633717in}}%
\pgfpathlineto{\pgfqpoint{4.953081in}{6.634273in}}%
\pgfpathlineto{\pgfqpoint{4.959927in}{6.639827in}}%
\pgfpathlineto{\pgfqpoint{4.962209in}{6.643715in}}%
\pgfpathlineto{\pgfqpoint{4.964491in}{6.645937in}}%
\pgfpathlineto{\pgfqpoint{4.966773in}{6.651491in}}%
\pgfpathlineto{\pgfqpoint{4.969055in}{6.663710in}}%
\pgfpathlineto{\pgfqpoint{4.978182in}{6.664266in}}%
\pgfpathlineto{\pgfqpoint{4.980464in}{6.668154in}}%
\pgfpathlineto{\pgfqpoint{4.982746in}{6.667598in}}%
\pgfpathlineto{\pgfqpoint{4.985028in}{6.677041in}}%
\pgfpathlineto{\pgfqpoint{4.991874in}{6.675374in}}%
\pgfpathlineto{\pgfqpoint{4.994156in}{6.685372in}}%
\pgfpathlineto{\pgfqpoint{4.996437in}{6.709811in}}%
\pgfpathlineto{\pgfqpoint{4.998719in}{6.708700in}}%
\pgfpathlineto{\pgfqpoint{5.001001in}{6.713699in}}%
\pgfpathlineto{\pgfqpoint{5.007847in}{6.719253in}}%
\pgfpathlineto{\pgfqpoint{5.012411in}{6.698147in}}%
\pgfpathlineto{\pgfqpoint{5.014693in}{6.705367in}}%
\pgfpathlineto{\pgfqpoint{5.016975in}{6.687038in}}%
\pgfpathlineto{\pgfqpoint{5.023820in}{6.697036in}}%
\pgfpathlineto{\pgfqpoint{5.026102in}{6.675374in}}%
\pgfpathlineto{\pgfqpoint{5.028384in}{6.675930in}}%
\pgfpathlineto{\pgfqpoint{5.030666in}{6.685927in}}%
\pgfpathlineto{\pgfqpoint{5.032948in}{6.669265in}}%
\pgfpathlineto{\pgfqpoint{5.039794in}{6.688149in}}%
\pgfpathlineto{\pgfqpoint{5.042076in}{6.681484in}}%
\pgfpathlineto{\pgfqpoint{5.044358in}{6.697036in}}%
\pgfpathlineto{\pgfqpoint{5.046639in}{6.682595in}}%
\pgfpathlineto{\pgfqpoint{5.048921in}{6.685927in}}%
\pgfpathlineto{\pgfqpoint{5.055767in}{6.689260in}}%
\pgfpathlineto{\pgfqpoint{5.058049in}{6.679818in}}%
\pgfpathlineto{\pgfqpoint{5.060331in}{6.663155in}}%
\pgfpathlineto{\pgfqpoint{5.062613in}{6.658156in}}%
\pgfpathlineto{\pgfqpoint{5.064895in}{6.660933in}}%
\pgfpathlineto{\pgfqpoint{5.071740in}{6.672597in}}%
\pgfpathlineto{\pgfqpoint{5.074022in}{6.658156in}}%
\pgfpathlineto{\pgfqpoint{5.076304in}{6.659822in}}%
\pgfpathlineto{\pgfqpoint{5.078586in}{6.664266in}}%
\pgfpathlineto{\pgfqpoint{5.087714in}{6.675930in}}%
\pgfpathlineto{\pgfqpoint{5.089996in}{6.668154in}}%
\pgfpathlineto{\pgfqpoint{5.092278in}{6.667598in}}%
\pgfpathlineto{\pgfqpoint{5.094559in}{6.701479in}}%
\pgfpathlineto{\pgfqpoint{5.096841in}{6.830894in}}%
\pgfpathlineto{\pgfqpoint{5.103687in}{6.789792in}}%
\pgfpathlineto{\pgfqpoint{5.105969in}{6.794791in}}%
\pgfpathlineto{\pgfqpoint{5.108251in}{6.782016in}}%
\pgfpathlineto{\pgfqpoint{5.110533in}{6.773685in}}%
\pgfpathlineto{\pgfqpoint{5.112815in}{6.772574in}}%
\pgfpathlineto{\pgfqpoint{5.119660in}{6.761465in}}%
\pgfpathlineto{\pgfqpoint{5.121942in}{6.743136in}}%
\pgfpathlineto{\pgfqpoint{5.124224in}{6.756467in}}%
\pgfpathlineto{\pgfqpoint{5.128788in}{6.751468in}}%
\pgfpathlineto{\pgfqpoint{5.135634in}{6.754800in}}%
\pgfpathlineto{\pgfqpoint{5.137916in}{6.766464in}}%
\pgfpathlineto{\pgfqpoint{5.140198in}{6.764798in}}%
\pgfpathlineto{\pgfqpoint{5.142480in}{6.764243in}}%
\pgfpathlineto{\pgfqpoint{5.144761in}{6.775351in}}%
\pgfpathlineto{\pgfqpoint{5.151607in}{6.773129in}}%
\pgfpathlineto{\pgfqpoint{5.153889in}{6.757022in}}%
\pgfpathlineto{\pgfqpoint{5.156171in}{6.752023in}}%
\pgfpathlineto{\pgfqpoint{5.158453in}{6.762576in}}%
\pgfpathlineto{\pgfqpoint{5.160735in}{6.777573in}}%
\pgfpathlineto{\pgfqpoint{5.167581in}{6.757022in}}%
\pgfpathlineto{\pgfqpoint{5.169862in}{6.762021in}}%
\pgfpathlineto{\pgfqpoint{5.174426in}{6.779795in}}%
\pgfpathlineto{\pgfqpoint{5.176708in}{6.773129in}}%
\pgfpathlineto{\pgfqpoint{5.183554in}{6.775351in}}%
\pgfpathlineto{\pgfqpoint{5.185836in}{6.777017in}}%
\pgfpathlineto{\pgfqpoint{5.188118in}{6.790348in}}%
\pgfpathlineto{\pgfqpoint{5.190400in}{6.794236in}}%
\pgfpathlineto{\pgfqpoint{5.192681in}{6.792569in}}%
\pgfpathlineto{\pgfqpoint{5.201809in}{6.784793in}}%
\pgfpathlineto{\pgfqpoint{5.204091in}{6.784793in}}%
\pgfpathlineto{\pgfqpoint{5.206373in}{6.789792in}}%
\pgfpathlineto{\pgfqpoint{5.208655in}{6.773129in}}%
\pgfpathlineto{\pgfqpoint{5.215501in}{6.773685in}}%
\pgfpathlineto{\pgfqpoint{5.217782in}{6.775907in}}%
\pgfpathlineto{\pgfqpoint{5.220064in}{6.785349in}}%
\pgfpathlineto{\pgfqpoint{5.222346in}{6.772574in}}%
\pgfpathlineto{\pgfqpoint{5.224628in}{6.774240in}}%
\pgfpathlineto{\pgfqpoint{5.231474in}{6.772019in}}%
\pgfpathlineto{\pgfqpoint{5.233756in}{6.775907in}}%
\pgfpathlineto{\pgfqpoint{5.236038in}{6.789792in}}%
\pgfpathlineto{\pgfqpoint{5.240602in}{6.778684in}}%
\pgfpathlineto{\pgfqpoint{5.247447in}{6.770352in}}%
\pgfpathlineto{\pgfqpoint{5.249729in}{6.770908in}}%
\pgfpathlineto{\pgfqpoint{5.252011in}{6.773129in}}%
\pgfpathlineto{\pgfqpoint{5.254293in}{6.788681in}}%
\pgfpathlineto{\pgfqpoint{5.256575in}{6.782572in}}%
\pgfpathlineto{\pgfqpoint{5.263421in}{6.790903in}}%
\pgfpathlineto{\pgfqpoint{5.265703in}{6.797013in}}%
\pgfpathlineto{\pgfqpoint{5.270266in}{6.773129in}}%
\pgfpathlineto{\pgfqpoint{5.272548in}{6.775351in}}%
\pgfpathlineto{\pgfqpoint{5.279394in}{6.754245in}}%
\pgfpathlineto{\pgfqpoint{5.281676in}{6.750912in}}%
\pgfpathlineto{\pgfqpoint{5.286240in}{6.760910in}}%
\pgfpathlineto{\pgfqpoint{5.295367in}{6.738693in}}%
\pgfpathlineto{\pgfqpoint{5.297649in}{6.746469in}}%
\pgfpathlineto{\pgfqpoint{5.299931in}{6.719253in}}%
\pgfpathlineto{\pgfqpoint{5.302213in}{6.725363in}}%
\pgfpathlineto{\pgfqpoint{5.304495in}{6.737027in}}%
\pgfpathlineto{\pgfqpoint{5.311341in}{6.746469in}}%
\pgfpathlineto{\pgfqpoint{5.313623in}{6.755356in}}%
\pgfpathlineto{\pgfqpoint{5.315904in}{6.760355in}}%
\pgfpathlineto{\pgfqpoint{5.318186in}{6.773129in}}%
\pgfpathlineto{\pgfqpoint{5.320468in}{6.782572in}}%
\pgfpathlineto{\pgfqpoint{5.327314in}{6.778128in}}%
\pgfpathlineto{\pgfqpoint{5.334160in}{6.736471in}}%
\pgfpathlineto{\pgfqpoint{5.336442in}{6.712588in}}%
\pgfpathlineto{\pgfqpoint{5.343287in}{6.722030in}}%
\pgfpathlineto{\pgfqpoint{5.347851in}{6.736471in}}%
\pgfpathlineto{\pgfqpoint{5.350133in}{6.729806in}}%
\pgfpathlineto{\pgfqpoint{5.352415in}{6.729251in}}%
\pgfpathlineto{\pgfqpoint{5.359261in}{6.718142in}}%
\pgfpathlineto{\pgfqpoint{5.361543in}{6.719808in}}%
\pgfpathlineto{\pgfqpoint{5.363825in}{6.729251in}}%
\pgfpathlineto{\pgfqpoint{5.366106in}{6.725918in}}%
\pgfpathlineto{\pgfqpoint{5.368388in}{6.714254in}}%
\pgfpathlineto{\pgfqpoint{5.375234in}{6.735360in}}%
\pgfpathlineto{\pgfqpoint{5.377516in}{6.710366in}}%
\pgfpathlineto{\pgfqpoint{5.379798in}{6.717587in}}%
\pgfpathlineto{\pgfqpoint{5.382080in}{6.714254in}}%
\pgfpathlineto{\pgfqpoint{5.384362in}{6.728140in}}%
\pgfpathlineto{\pgfqpoint{5.391207in}{6.734250in}}%
\pgfpathlineto{\pgfqpoint{5.393489in}{6.727584in}}%
\pgfpathlineto{\pgfqpoint{5.395771in}{6.711477in}}%
\pgfpathlineto{\pgfqpoint{5.400335in}{6.658156in}}%
\pgfpathlineto{\pgfqpoint{5.407181in}{6.623720in}}%
\pgfpathlineto{\pgfqpoint{5.409463in}{6.595948in}}%
\pgfpathlineto{\pgfqpoint{5.411745in}{6.630385in}}%
\pgfpathlineto{\pgfqpoint{5.414026in}{6.677596in}}%
\pgfpathlineto{\pgfqpoint{5.416308in}{6.684816in}}%
\pgfpathlineto{\pgfqpoint{5.423154in}{6.668709in}}%
\pgfpathlineto{\pgfqpoint{5.425436in}{6.624275in}}%
\pgfpathlineto{\pgfqpoint{5.427718in}{6.657045in}}%
\pgfpathlineto{\pgfqpoint{5.430000in}{6.654268in}}%
\pgfpathlineto{\pgfqpoint{5.432282in}{6.630385in}}%
\pgfpathlineto{\pgfqpoint{5.441409in}{6.675374in}}%
\pgfpathlineto{\pgfqpoint{5.443691in}{6.655934in}}%
\pgfpathlineto{\pgfqpoint{5.445973in}{6.662044in}}%
\pgfpathlineto{\pgfqpoint{5.448255in}{6.674819in}}%
\pgfpathlineto{\pgfqpoint{5.455101in}{6.666487in}}%
\pgfpathlineto{\pgfqpoint{5.459665in}{6.720919in}}%
\pgfpathlineto{\pgfqpoint{5.461946in}{6.704256in}}%
\pgfpathlineto{\pgfqpoint{5.464228in}{6.678151in}}%
\pgfpathlineto{\pgfqpoint{5.471074in}{6.692037in}}%
\pgfpathlineto{\pgfqpoint{5.475638in}{6.694259in}}%
\pgfpathlineto{\pgfqpoint{5.477920in}{6.683706in}}%
\pgfpathlineto{\pgfqpoint{5.480202in}{6.683706in}}%
\pgfpathlineto{\pgfqpoint{5.487047in}{6.654823in}}%
\pgfpathlineto{\pgfqpoint{5.489329in}{6.667043in}}%
\pgfpathlineto{\pgfqpoint{5.491611in}{6.698147in}}%
\pgfpathlineto{\pgfqpoint{5.493893in}{6.696480in}}%
\pgfpathlineto{\pgfqpoint{5.496175in}{6.709811in}}%
\pgfpathlineto{\pgfqpoint{5.507585in}{6.819230in}}%
\pgfpathlineto{\pgfqpoint{5.509867in}{6.831449in}}%
\pgfpathlineto{\pgfqpoint{5.512148in}{6.833671in}}%
\pgfpathlineto{\pgfqpoint{5.518994in}{6.834782in}}%
\pgfpathlineto{\pgfqpoint{5.523558in}{6.811454in}}%
\pgfpathlineto{\pgfqpoint{5.525840in}{6.831449in}}%
\pgfpathlineto{\pgfqpoint{5.528122in}{6.876994in}}%
\pgfpathlineto{\pgfqpoint{5.534968in}{6.877550in}}%
\pgfpathlineto{\pgfqpoint{5.537249in}{6.867552in}}%
\pgfpathlineto{\pgfqpoint{5.539531in}{6.870885in}}%
\pgfpathlineto{\pgfqpoint{5.541813in}{6.905321in}}%
\pgfpathlineto{\pgfqpoint{5.544095in}{6.901989in}}%
\pgfpathlineto{\pgfqpoint{5.550941in}{6.903655in}}%
\pgfpathlineto{\pgfqpoint{5.555505in}{6.896434in}}%
\pgfpathlineto{\pgfqpoint{5.557787in}{6.893657in}}%
\pgfpathlineto{\pgfqpoint{5.560068in}{6.874217in}}%
\pgfpathlineto{\pgfqpoint{5.569196in}{6.905877in}}%
\pgfpathlineto{\pgfqpoint{5.571478in}{6.903655in}}%
\pgfpathlineto{\pgfqpoint{5.573760in}{6.908098in}}%
\pgfpathlineto{\pgfqpoint{5.576042in}{6.921428in}}%
\pgfpathlineto{\pgfqpoint{5.582888in}{6.913653in}}%
\pgfpathlineto{\pgfqpoint{5.585169in}{6.930871in}}%
\pgfpathlineto{\pgfqpoint{5.587451in}{6.956976in}}%
\pgfpathlineto{\pgfqpoint{5.589733in}{6.933092in}}%
\pgfpathlineto{\pgfqpoint{5.592015in}{6.938647in}}%
\pgfpathlineto{\pgfqpoint{5.598861in}{6.942535in}}%
\pgfpathlineto{\pgfqpoint{5.601143in}{6.940313in}}%
\pgfpathlineto{\pgfqpoint{5.603425in}{6.949755in}}%
\pgfpathlineto{\pgfqpoint{5.605707in}{6.938091in}}%
\pgfpathlineto{\pgfqpoint{5.607989in}{6.956420in}}%
\pgfpathlineto{\pgfqpoint{5.614834in}{6.953088in}}%
\pgfpathlineto{\pgfqpoint{5.617116in}{6.956420in}}%
\pgfpathlineto{\pgfqpoint{5.619398in}{6.942535in}}%
\pgfpathlineto{\pgfqpoint{5.623962in}{6.942535in}}%
\pgfpathlineto{\pgfqpoint{5.630808in}{6.922539in}}%
\pgfpathlineto{\pgfqpoint{5.633090in}{6.933092in}}%
\pgfpathlineto{\pgfqpoint{5.635371in}{6.923650in}}%
\pgfpathlineto{\pgfqpoint{5.637653in}{6.926427in}}%
\pgfpathlineto{\pgfqpoint{5.639935in}{6.948644in}}%
\pgfpathlineto{\pgfqpoint{5.646781in}{6.943090in}}%
\pgfpathlineto{\pgfqpoint{5.649063in}{6.934203in}}%
\pgfpathlineto{\pgfqpoint{5.651345in}{6.947534in}}%
\pgfpathlineto{\pgfqpoint{5.653627in}{6.955865in}}%
\pgfpathlineto{\pgfqpoint{5.655909in}{6.937536in}}%
\pgfpathlineto{\pgfqpoint{5.662754in}{6.937536in}}%
\pgfpathlineto{\pgfqpoint{5.665036in}{6.940313in}}%
\pgfpathlineto{\pgfqpoint{5.667318in}{6.971972in}}%
\pgfpathlineto{\pgfqpoint{5.669600in}{6.961975in}}%
\pgfpathlineto{\pgfqpoint{5.671882in}{6.949200in}}%
\pgfpathlineto{\pgfqpoint{5.678728in}{6.955310in}}%
\pgfpathlineto{\pgfqpoint{5.681010in}{6.959198in}}%
\pgfpathlineto{\pgfqpoint{5.683291in}{6.981415in}}%
\pgfpathlineto{\pgfqpoint{5.685573in}{6.975860in}}%
\pgfpathlineto{\pgfqpoint{5.694701in}{6.979193in}}%
\pgfpathlineto{\pgfqpoint{5.696983in}{6.996967in}}%
\pgfpathlineto{\pgfqpoint{5.699265in}{6.986413in}}%
\pgfpathlineto{\pgfqpoint{5.701547in}{6.990857in}}%
\pgfpathlineto{\pgfqpoint{5.710674in}{6.969751in}}%
\pgfpathlineto{\pgfqpoint{5.712956in}{6.971417in}}%
\pgfpathlineto{\pgfqpoint{5.715238in}{6.948089in}}%
\pgfpathlineto{\pgfqpoint{5.717520in}{6.886437in}}%
\pgfpathlineto{\pgfqpoint{5.719802in}{6.861998in}}%
\pgfpathlineto{\pgfqpoint{5.728930in}{6.870885in}}%
\pgfpathlineto{\pgfqpoint{5.731212in}{6.851445in}}%
\pgfpathlineto{\pgfqpoint{5.733493in}{6.890880in}}%
\pgfpathlineto{\pgfqpoint{5.735775in}{6.863664in}}%
\pgfpathlineto{\pgfqpoint{5.744903in}{6.863664in}}%
\pgfpathlineto{\pgfqpoint{5.747185in}{6.840336in}}%
\pgfpathlineto{\pgfqpoint{5.749467in}{6.868663in}}%
\pgfpathlineto{\pgfqpoint{5.751749in}{6.851445in}}%
\pgfpathlineto{\pgfqpoint{5.758594in}{6.842002in}}%
\pgfpathlineto{\pgfqpoint{5.760876in}{6.854777in}}%
\pgfpathlineto{\pgfqpoint{5.763158in}{6.840336in}}%
\pgfpathlineto{\pgfqpoint{5.765440in}{6.850334in}}%
\pgfpathlineto{\pgfqpoint{5.767722in}{6.893102in}}%
\pgfpathlineto{\pgfqpoint{5.774568in}{6.870885in}}%
\pgfpathlineto{\pgfqpoint{5.776850in}{6.851445in}}%
\pgfpathlineto{\pgfqpoint{5.781413in}{6.896434in}}%
\pgfpathlineto{\pgfqpoint{5.783695in}{6.865886in}}%
\pgfpathlineto{\pgfqpoint{5.790541in}{6.848112in}}%
\pgfpathlineto{\pgfqpoint{5.792823in}{6.853666in}}%
\pgfpathlineto{\pgfqpoint{5.795105in}{6.854777in}}%
\pgfpathlineto{\pgfqpoint{5.797387in}{6.813676in}}%
\pgfpathlineto{\pgfqpoint{5.799669in}{6.852556in}}%
\pgfpathlineto{\pgfqpoint{5.808796in}{6.881438in}}%
\pgfpathlineto{\pgfqpoint{5.811078in}{6.904210in}}%
\pgfpathlineto{\pgfqpoint{5.813360in}{6.891991in}}%
\pgfpathlineto{\pgfqpoint{5.815642in}{6.889214in}}%
\pgfpathlineto{\pgfqpoint{5.822488in}{6.907543in}}%
\pgfpathlineto{\pgfqpoint{5.824770in}{6.898656in}}%
\pgfpathlineto{\pgfqpoint{5.827052in}{6.885881in}}%
\pgfpathlineto{\pgfqpoint{5.829334in}{6.910320in}}%
\pgfpathlineto{\pgfqpoint{5.831615in}{6.918651in}}%
\pgfpathlineto{\pgfqpoint{5.838461in}{6.905877in}}%
\pgfpathlineto{\pgfqpoint{5.840743in}{6.941424in}}%
\pgfpathlineto{\pgfqpoint{5.843025in}{6.955865in}}%
\pgfpathlineto{\pgfqpoint{5.845307in}{6.958087in}}%
\pgfpathlineto{\pgfqpoint{5.847589in}{6.969751in}}%
\pgfpathlineto{\pgfqpoint{5.854434in}{6.961419in}}%
\pgfpathlineto{\pgfqpoint{5.856716in}{6.950311in}}%
\pgfpathlineto{\pgfqpoint{5.858998in}{6.949755in}}%
\pgfpathlineto{\pgfqpoint{5.861280in}{6.944756in}}%
\pgfpathlineto{\pgfqpoint{5.863562in}{6.963641in}}%
\pgfpathlineto{\pgfqpoint{5.870408in}{6.960308in}}%
\pgfpathlineto{\pgfqpoint{5.872690in}{6.960864in}}%
\pgfpathlineto{\pgfqpoint{5.874972in}{6.955865in}}%
\pgfpathlineto{\pgfqpoint{5.877254in}{6.993634in}}%
\pgfpathlineto{\pgfqpoint{5.879535in}{6.991968in}}%
\pgfpathlineto{\pgfqpoint{5.886381in}{6.999744in}}%
\pgfpathlineto{\pgfqpoint{5.888663in}{6.998633in}}%
\pgfpathlineto{\pgfqpoint{5.890945in}{6.999188in}}%
\pgfpathlineto{\pgfqpoint{5.893227in}{7.000855in}}%
\pgfpathlineto{\pgfqpoint{5.902355in}{7.019184in}}%
\pgfpathlineto{\pgfqpoint{5.904636in}{7.018628in}}%
\pgfpathlineto{\pgfqpoint{5.906918in}{7.035847in}}%
\pgfpathlineto{\pgfqpoint{5.909200in}{7.033625in}}%
\pgfpathlineto{\pgfqpoint{5.911482in}{7.040290in}}%
\pgfpathlineto{\pgfqpoint{5.920610in}{6.994745in}}%
\pgfpathlineto{\pgfqpoint{5.922892in}{6.990857in}}%
\pgfpathlineto{\pgfqpoint{5.925174in}{6.978082in}}%
\pgfpathlineto{\pgfqpoint{5.927455in}{6.985303in}}%
\pgfpathlineto{\pgfqpoint{5.934301in}{6.981415in}}%
\pgfpathlineto{\pgfqpoint{5.936583in}{6.986413in}}%
\pgfpathlineto{\pgfqpoint{5.938865in}{6.994745in}}%
\pgfpathlineto{\pgfqpoint{5.941147in}{6.996411in}}%
\pgfpathlineto{\pgfqpoint{5.950275in}{6.998633in}}%
\pgfpathlineto{\pgfqpoint{5.952556in}{7.003076in}}%
\pgfpathlineto{\pgfqpoint{5.954838in}{7.003076in}}%
\pgfpathlineto{\pgfqpoint{5.957120in}{6.994745in}}%
\pgfpathlineto{\pgfqpoint{5.959402in}{6.984192in}}%
\pgfpathlineto{\pgfqpoint{5.966248in}{6.980304in}}%
\pgfpathlineto{\pgfqpoint{5.968530in}{6.990857in}}%
\pgfpathlineto{\pgfqpoint{5.970812in}{6.992523in}}%
\pgfpathlineto{\pgfqpoint{5.973094in}{6.990857in}}%
\pgfpathlineto{\pgfqpoint{5.975376in}{6.983636in}}%
\pgfpathlineto{\pgfqpoint{5.982221in}{6.990301in}}%
\pgfpathlineto{\pgfqpoint{5.984503in}{6.978082in}}%
\pgfpathlineto{\pgfqpoint{5.986785in}{6.950866in}}%
\pgfpathlineto{\pgfqpoint{5.989067in}{6.941979in}}%
\pgfpathlineto{\pgfqpoint{5.991349in}{6.953088in}}%
\pgfpathlineto{\pgfqpoint{5.998195in}{6.940868in}}%
\pgfpathlineto{\pgfqpoint{6.000477in}{6.970306in}}%
\pgfpathlineto{\pgfqpoint{6.002758in}{6.963641in}}%
\pgfpathlineto{\pgfqpoint{6.005040in}{6.951977in}}%
\pgfpathlineto{\pgfqpoint{6.007322in}{6.930315in}}%
\pgfpathlineto{\pgfqpoint{6.014168in}{6.945312in}}%
\pgfpathlineto{\pgfqpoint{6.016450in}{6.933648in}}%
\pgfpathlineto{\pgfqpoint{6.018732in}{6.928649in}}%
\pgfpathlineto{\pgfqpoint{6.021014in}{6.916430in}}%
\pgfpathlineto{\pgfqpoint{6.023296in}{6.926427in}}%
\pgfpathlineto{\pgfqpoint{6.030141in}{6.922539in}}%
\pgfpathlineto{\pgfqpoint{6.032423in}{6.940313in}}%
\pgfpathlineto{\pgfqpoint{6.034705in}{6.951977in}}%
\pgfpathlineto{\pgfqpoint{6.036987in}{6.948644in}}%
\pgfpathlineto{\pgfqpoint{6.039269in}{6.953088in}}%
\pgfpathlineto{\pgfqpoint{6.048397in}{6.958642in}}%
\pgfpathlineto{\pgfqpoint{6.050678in}{6.952532in}}%
\pgfpathlineto{\pgfqpoint{6.052960in}{6.949755in}}%
\pgfpathlineto{\pgfqpoint{6.055242in}{6.944756in}}%
\pgfpathlineto{\pgfqpoint{6.062088in}{6.953088in}}%
\pgfpathlineto{\pgfqpoint{6.064370in}{6.954199in}}%
\pgfpathlineto{\pgfqpoint{6.066652in}{6.962530in}}%
\pgfpathlineto{\pgfqpoint{6.068934in}{6.959198in}}%
\pgfpathlineto{\pgfqpoint{6.071216in}{6.949200in}}%
\pgfpathlineto{\pgfqpoint{6.078061in}{6.939202in}}%
\pgfpathlineto{\pgfqpoint{6.080343in}{6.968640in}}%
\pgfpathlineto{\pgfqpoint{6.082625in}{6.975860in}}%
\pgfpathlineto{\pgfqpoint{6.084907in}{6.989746in}}%
\pgfpathlineto{\pgfqpoint{6.087189in}{6.987524in}}%
\pgfpathlineto{\pgfqpoint{6.094035in}{6.998633in}}%
\pgfpathlineto{\pgfqpoint{6.096317in}{7.004187in}}%
\pgfpathlineto{\pgfqpoint{6.098599in}{6.996411in}}%
\pgfpathlineto{\pgfqpoint{6.100880in}{7.016407in}}%
\pgfpathlineto{\pgfqpoint{6.103162in}{6.949755in}}%
\pgfpathlineto{\pgfqpoint{6.110008in}{6.925316in}}%
\pgfpathlineto{\pgfqpoint{6.114572in}{6.985303in}}%
\pgfpathlineto{\pgfqpoint{6.116854in}{7.030292in}}%
\pgfpathlineto{\pgfqpoint{6.119136in}{7.030848in}}%
\pgfpathlineto{\pgfqpoint{6.128263in}{7.028626in}}%
\pgfpathlineto{\pgfqpoint{6.130545in}{7.043067in}}%
\pgfpathlineto{\pgfqpoint{6.132827in}{7.046955in}}%
\pgfpathlineto{\pgfqpoint{6.135109in}{7.065284in}}%
\pgfpathlineto{\pgfqpoint{6.141955in}{7.065840in}}%
\pgfpathlineto{\pgfqpoint{6.144237in}{7.068061in}}%
\pgfpathlineto{\pgfqpoint{6.146519in}{7.073060in}}%
\pgfpathlineto{\pgfqpoint{6.151082in}{7.098610in}}%
\pgfpathlineto{\pgfqpoint{6.157928in}{7.099721in}}%
\pgfpathlineto{\pgfqpoint{6.160210in}{7.100831in}}%
\pgfpathlineto{\pgfqpoint{6.162492in}{7.093611in}}%
\pgfpathlineto{\pgfqpoint{6.164774in}{7.084169in}}%
\pgfpathlineto{\pgfqpoint{6.167056in}{7.058619in}}%
\pgfpathlineto{\pgfqpoint{6.176183in}{7.029737in}}%
\pgfpathlineto{\pgfqpoint{6.178465in}{7.020850in}}%
\pgfpathlineto{\pgfqpoint{6.180747in}{7.019184in}}%
\pgfpathlineto{\pgfqpoint{6.183029in}{7.013629in}}%
\pgfpathlineto{\pgfqpoint{6.189875in}{7.014185in}}%
\pgfpathlineto{\pgfqpoint{6.192157in}{7.009186in}}%
\pgfpathlineto{\pgfqpoint{6.194439in}{7.013074in}}%
\pgfpathlineto{\pgfqpoint{6.196721in}{7.015296in}}%
\pgfpathlineto{\pgfqpoint{6.199002in}{7.020850in}}%
\pgfpathlineto{\pgfqpoint{6.205848in}{7.020295in}}%
\pgfpathlineto{\pgfqpoint{6.208130in}{7.021405in}}%
\pgfpathlineto{\pgfqpoint{6.210412in}{7.020295in}}%
\pgfpathlineto{\pgfqpoint{6.212694in}{7.020850in}}%
\pgfpathlineto{\pgfqpoint{6.214976in}{7.018628in}}%
\pgfpathlineto{\pgfqpoint{6.221821in}{7.018628in}}%
\pgfpathlineto{\pgfqpoint{6.224103in}{7.016407in}}%
\pgfpathlineto{\pgfqpoint{6.226385in}{7.020850in}}%
\pgfpathlineto{\pgfqpoint{6.228667in}{7.028071in}}%
\pgfpathlineto{\pgfqpoint{6.230949in}{7.019184in}}%
\pgfpathlineto{\pgfqpoint{6.237795in}{7.022516in}}%
\pgfpathlineto{\pgfqpoint{6.240077in}{7.018073in}}%
\pgfpathlineto{\pgfqpoint{6.244641in}{7.016962in}}%
\pgfpathlineto{\pgfqpoint{6.246922in}{7.018073in}}%
\pgfpathlineto{\pgfqpoint{6.253768in}{7.024738in}}%
\pgfpathlineto{\pgfqpoint{6.256050in}{7.024738in}}%
\pgfpathlineto{\pgfqpoint{6.258332in}{7.018628in}}%
\pgfpathlineto{\pgfqpoint{6.260614in}{7.016962in}}%
\pgfpathlineto{\pgfqpoint{6.262896in}{7.020850in}}%
\pgfpathlineto{\pgfqpoint{6.272023in}{7.009186in}}%
\pgfpathlineto{\pgfqpoint{6.274305in}{7.009741in}}%
\pgfpathlineto{\pgfqpoint{6.276587in}{7.009186in}}%
\pgfpathlineto{\pgfqpoint{6.278869in}{6.963641in}}%
\pgfpathlineto{\pgfqpoint{6.285715in}{6.981970in}}%
\pgfpathlineto{\pgfqpoint{6.287997in}{6.950866in}}%
\pgfpathlineto{\pgfqpoint{6.290279in}{6.943646in}}%
\pgfpathlineto{\pgfqpoint{6.292561in}{6.957531in}}%
\pgfpathlineto{\pgfqpoint{6.294843in}{6.954199in}}%
\pgfpathlineto{\pgfqpoint{6.301688in}{6.941979in}}%
\pgfpathlineto{\pgfqpoint{6.306252in}{6.962530in}}%
\pgfpathlineto{\pgfqpoint{6.308534in}{6.971972in}}%
\pgfpathlineto{\pgfqpoint{6.310816in}{6.964196in}}%
\pgfpathlineto{\pgfqpoint{6.317662in}{6.946978in}}%
\pgfpathlineto{\pgfqpoint{6.319943in}{6.963641in}}%
\pgfpathlineto{\pgfqpoint{6.322225in}{6.964752in}}%
\pgfpathlineto{\pgfqpoint{6.324507in}{6.946978in}}%
\pgfpathlineto{\pgfqpoint{6.326789in}{6.950866in}}%
\pgfpathlineto{\pgfqpoint{6.333635in}{6.951977in}}%
\pgfpathlineto{\pgfqpoint{6.335917in}{6.945312in}}%
\pgfpathlineto{\pgfqpoint{6.338199in}{6.945312in}}%
\pgfpathlineto{\pgfqpoint{6.342763in}{6.924761in}}%
\pgfpathlineto{\pgfqpoint{6.349608in}{6.913653in}}%
\pgfpathlineto{\pgfqpoint{6.351890in}{6.916985in}}%
\pgfpathlineto{\pgfqpoint{6.354172in}{6.915874in}}%
\pgfpathlineto{\pgfqpoint{6.356454in}{6.909765in}}%
\pgfpathlineto{\pgfqpoint{6.358736in}{6.915319in}}%
\pgfpathlineto{\pgfqpoint{6.365582in}{6.913653in}}%
\pgfpathlineto{\pgfqpoint{6.367864in}{6.919762in}}%
\pgfpathlineto{\pgfqpoint{6.370145in}{6.923650in}}%
\pgfpathlineto{\pgfqpoint{6.372427in}{6.924206in}}%
\pgfpathlineto{\pgfqpoint{6.374709in}{6.919762in}}%
\pgfpathlineto{\pgfqpoint{6.381555in}{6.916985in}}%
\pgfpathlineto{\pgfqpoint{6.383837in}{6.903655in}}%
\pgfpathlineto{\pgfqpoint{6.386119in}{6.914208in}}%
\pgfpathlineto{\pgfqpoint{6.388401in}{6.902544in}}%
\pgfpathlineto{\pgfqpoint{6.390683in}{6.931426in}}%
\pgfpathlineto{\pgfqpoint{6.397528in}{6.925872in}}%
\pgfpathlineto{\pgfqpoint{6.399810in}{6.914763in}}%
\pgfpathlineto{\pgfqpoint{6.402092in}{6.895879in}}%
\pgfpathlineto{\pgfqpoint{6.404374in}{6.885326in}}%
\pgfpathlineto{\pgfqpoint{6.406656in}{6.893102in}}%
\pgfpathlineto{\pgfqpoint{6.413502in}{6.935870in}}%
\pgfpathlineto{\pgfqpoint{6.415784in}{6.941424in}}%
\pgfpathlineto{\pgfqpoint{6.418065in}{6.951422in}}%
\pgfpathlineto{\pgfqpoint{6.420347in}{6.989746in}}%
\pgfpathlineto{\pgfqpoint{6.422629in}{7.004743in}}%
\pgfpathlineto{\pgfqpoint{6.429475in}{6.994745in}}%
\pgfpathlineto{\pgfqpoint{6.431757in}{7.006409in}}%
\pgfpathlineto{\pgfqpoint{6.434039in}{7.005853in}}%
\pgfpathlineto{\pgfqpoint{6.436321in}{7.008631in}}%
\pgfpathlineto{\pgfqpoint{6.438603in}{7.002521in}}%
\pgfpathlineto{\pgfqpoint{6.445448in}{7.012519in}}%
\pgfpathlineto{\pgfqpoint{6.447730in}{7.027515in}}%
\pgfpathlineto{\pgfqpoint{6.450012in}{7.035291in}}%
\pgfpathlineto{\pgfqpoint{6.454576in}{7.040290in}}%
\pgfpathlineto{\pgfqpoint{6.461422in}{7.030848in}}%
\pgfpathlineto{\pgfqpoint{6.463704in}{7.021405in}}%
\pgfpathlineto{\pgfqpoint{6.465986in}{7.006964in}}%
\pgfpathlineto{\pgfqpoint{6.468267in}{7.038068in}}%
\pgfpathlineto{\pgfqpoint{6.470549in}{7.035291in}}%
\pgfpathlineto{\pgfqpoint{6.477395in}{7.024183in}}%
\pgfpathlineto{\pgfqpoint{6.479677in}{7.026960in}}%
\pgfpathlineto{\pgfqpoint{6.481959in}{7.048066in}}%
\pgfpathlineto{\pgfqpoint{6.484241in}{7.044733in}}%
\pgfpathlineto{\pgfqpoint{6.486523in}{7.056953in}}%
\pgfpathlineto{\pgfqpoint{6.493368in}{7.060841in}}%
\pgfpathlineto{\pgfqpoint{6.495650in}{7.054731in}}%
\pgfpathlineto{\pgfqpoint{6.500214in}{7.031403in}}%
\pgfpathlineto{\pgfqpoint{6.502496in}{7.055286in}}%
\pgfpathlineto{\pgfqpoint{6.509342in}{7.063618in}}%
\pgfpathlineto{\pgfqpoint{6.511624in}{7.079725in}}%
\pgfpathlineto{\pgfqpoint{6.513906in}{7.074171in}}%
\pgfpathlineto{\pgfqpoint{6.516187in}{7.070838in}}%
\pgfpathlineto{\pgfqpoint{6.518469in}{7.073616in}}%
\pgfpathlineto{\pgfqpoint{6.527597in}{7.074726in}}%
\pgfpathlineto{\pgfqpoint{6.529879in}{7.064729in}}%
\pgfpathlineto{\pgfqpoint{6.532161in}{7.065284in}}%
\pgfpathlineto{\pgfqpoint{6.534443in}{7.059730in}}%
\pgfpathlineto{\pgfqpoint{6.545852in}{7.064729in}}%
\pgfpathlineto{\pgfqpoint{6.548134in}{7.055842in}}%
\pgfpathlineto{\pgfqpoint{6.550416in}{7.060285in}}%
\pgfpathlineto{\pgfqpoint{6.557262in}{7.053065in}}%
\pgfpathlineto{\pgfqpoint{6.559544in}{7.048621in}}%
\pgfpathlineto{\pgfqpoint{6.561826in}{7.053620in}}%
\pgfpathlineto{\pgfqpoint{6.564108in}{7.049177in}}%
\pgfpathlineto{\pgfqpoint{6.575517in}{7.043622in}}%
\pgfpathlineto{\pgfqpoint{6.577799in}{7.041401in}}%
\pgfpathlineto{\pgfqpoint{6.580081in}{7.040290in}}%
\pgfpathlineto{\pgfqpoint{6.582363in}{7.006964in}}%
\pgfpathlineto{\pgfqpoint{6.589208in}{6.968640in}}%
\pgfpathlineto{\pgfqpoint{6.591490in}{6.980859in}}%
\pgfpathlineto{\pgfqpoint{6.593772in}{6.999188in}}%
\pgfpathlineto{\pgfqpoint{6.596054in}{6.996411in}}%
\pgfpathlineto{\pgfqpoint{6.598336in}{6.981415in}}%
\pgfpathlineto{\pgfqpoint{6.605182in}{6.978637in}}%
\pgfpathlineto{\pgfqpoint{6.607464in}{6.965863in}}%
\pgfpathlineto{\pgfqpoint{6.612028in}{6.964752in}}%
\pgfpathlineto{\pgfqpoint{6.614309in}{6.965863in}}%
\pgfpathlineto{\pgfqpoint{6.621155in}{6.964196in}}%
\pgfpathlineto{\pgfqpoint{6.623437in}{6.959198in}}%
\pgfpathlineto{\pgfqpoint{6.625719in}{6.952532in}}%
\pgfpathlineto{\pgfqpoint{6.630283in}{6.966974in}}%
\pgfpathlineto{\pgfqpoint{6.637129in}{6.982525in}}%
\pgfpathlineto{\pgfqpoint{6.639410in}{6.994745in}}%
\pgfpathlineto{\pgfqpoint{6.641692in}{6.998077in}}%
\pgfpathlineto{\pgfqpoint{6.643974in}{7.003076in}}%
\pgfpathlineto{\pgfqpoint{6.646256in}{6.999188in}}%
\pgfpathlineto{\pgfqpoint{6.655384in}{7.006409in}}%
\pgfpathlineto{\pgfqpoint{6.657666in}{6.996967in}}%
\pgfpathlineto{\pgfqpoint{6.659948in}{6.993634in}}%
\pgfpathlineto{\pgfqpoint{6.662230in}{7.001965in}}%
\pgfpathlineto{\pgfqpoint{6.669075in}{6.989746in}}%
\pgfpathlineto{\pgfqpoint{6.671357in}{6.983081in}}%
\pgfpathlineto{\pgfqpoint{6.673639in}{7.001965in}}%
\pgfpathlineto{\pgfqpoint{6.675921in}{7.001965in}}%
\pgfpathlineto{\pgfqpoint{6.678203in}{6.998633in}}%
\pgfpathlineto{\pgfqpoint{6.685049in}{6.992523in}}%
\pgfpathlineto{\pgfqpoint{6.687330in}{6.985858in}}%
\pgfpathlineto{\pgfqpoint{6.689612in}{6.982525in}}%
\pgfpathlineto{\pgfqpoint{6.691894in}{6.975860in}}%
\pgfpathlineto{\pgfqpoint{6.694176in}{7.006409in}}%
\pgfpathlineto{\pgfqpoint{6.701022in}{6.985858in}}%
\pgfpathlineto{\pgfqpoint{6.703304in}{6.969751in}}%
\pgfpathlineto{\pgfqpoint{6.705586in}{6.980859in}}%
\pgfpathlineto{\pgfqpoint{6.707868in}{6.980304in}}%
\pgfpathlineto{\pgfqpoint{6.710150in}{6.986413in}}%
\pgfpathlineto{\pgfqpoint{6.716995in}{6.979748in}}%
\pgfpathlineto{\pgfqpoint{6.719277in}{6.962530in}}%
\pgfpathlineto{\pgfqpoint{6.721559in}{6.969195in}}%
\pgfpathlineto{\pgfqpoint{6.726123in}{6.978637in}}%
\pgfpathlineto{\pgfqpoint{6.732969in}{6.964752in}}%
\pgfpathlineto{\pgfqpoint{6.735251in}{6.973639in}}%
\pgfpathlineto{\pgfqpoint{6.737532in}{6.976971in}}%
\pgfpathlineto{\pgfqpoint{6.739814in}{6.986413in}}%
\pgfpathlineto{\pgfqpoint{6.742096in}{6.982525in}}%
\pgfpathlineto{\pgfqpoint{6.748942in}{6.986413in}}%
\pgfpathlineto{\pgfqpoint{6.751224in}{6.993634in}}%
\pgfpathlineto{\pgfqpoint{6.753506in}{6.990857in}}%
\pgfpathlineto{\pgfqpoint{6.755788in}{6.989191in}}%
\pgfpathlineto{\pgfqpoint{6.758070in}{6.991968in}}%
\pgfpathlineto{\pgfqpoint{6.764915in}{6.993079in}}%
\pgfpathlineto{\pgfqpoint{6.767197in}{6.994745in}}%
\pgfpathlineto{\pgfqpoint{6.771761in}{6.970862in}}%
\pgfpathlineto{\pgfqpoint{6.780889in}{6.974750in}}%
\pgfpathlineto{\pgfqpoint{6.787734in}{7.005853in}}%
\pgfpathlineto{\pgfqpoint{6.790016in}{6.970306in}}%
\pgfpathlineto{\pgfqpoint{6.796862in}{6.970306in}}%
\pgfpathlineto{\pgfqpoint{6.799144in}{6.965307in}}%
\pgfpathlineto{\pgfqpoint{6.803708in}{6.946978in}}%
\pgfpathlineto{\pgfqpoint{6.805990in}{6.942535in}}%
\pgfpathlineto{\pgfqpoint{6.812835in}{6.939758in}}%
\pgfpathlineto{\pgfqpoint{6.815117in}{6.942535in}}%
\pgfpathlineto{\pgfqpoint{6.817399in}{6.954199in}}%
\pgfpathlineto{\pgfqpoint{6.819681in}{6.953088in}}%
\pgfpathlineto{\pgfqpoint{6.821963in}{6.953643in}}%
\pgfpathlineto{\pgfqpoint{6.828809in}{6.946423in}}%
\pgfpathlineto{\pgfqpoint{6.831091in}{6.939202in}}%
\pgfpathlineto{\pgfqpoint{6.833373in}{6.928094in}}%
\pgfpathlineto{\pgfqpoint{6.835654in}{6.936425in}}%
\pgfpathlineto{\pgfqpoint{6.837936in}{6.906432in}}%
\pgfpathlineto{\pgfqpoint{6.844782in}{6.901989in}}%
\pgfpathlineto{\pgfqpoint{6.847064in}{6.895323in}}%
\pgfpathlineto{\pgfqpoint{6.849346in}{6.863664in}}%
\pgfpathlineto{\pgfqpoint{6.851628in}{6.866997in}}%
\pgfpathlineto{\pgfqpoint{6.853910in}{6.895879in}}%
\pgfpathlineto{\pgfqpoint{6.860755in}{6.901989in}}%
\pgfpathlineto{\pgfqpoint{6.863037in}{6.906987in}}%
\pgfpathlineto{\pgfqpoint{6.867601in}{6.867552in}}%
\pgfpathlineto{\pgfqpoint{6.869883in}{6.865886in}}%
\pgfpathlineto{\pgfqpoint{6.879011in}{6.861442in}}%
\pgfpathlineto{\pgfqpoint{6.881293in}{6.862553in}}%
\pgfpathlineto{\pgfqpoint{6.883574in}{6.879216in}}%
\pgfpathlineto{\pgfqpoint{6.885856in}{6.886992in}}%
\pgfpathlineto{\pgfqpoint{6.892702in}{6.891991in}}%
\pgfpathlineto{\pgfqpoint{6.894984in}{6.889769in}}%
\pgfpathlineto{\pgfqpoint{6.897266in}{6.876994in}}%
\pgfpathlineto{\pgfqpoint{6.899548in}{6.872551in}}%
\pgfpathlineto{\pgfqpoint{6.908675in}{6.939758in}}%
\pgfpathlineto{\pgfqpoint{6.910957in}{6.915319in}}%
\pgfpathlineto{\pgfqpoint{6.913239in}{6.927538in}}%
\pgfpathlineto{\pgfqpoint{6.915521in}{6.951977in}}%
\pgfpathlineto{\pgfqpoint{6.917803in}{6.967529in}}%
\pgfpathlineto{\pgfqpoint{6.924649in}{6.957531in}}%
\pgfpathlineto{\pgfqpoint{6.926931in}{6.923650in}}%
\pgfpathlineto{\pgfqpoint{6.929213in}{6.905877in}}%
\pgfpathlineto{\pgfqpoint{6.931495in}{6.894213in}}%
\pgfpathlineto{\pgfqpoint{6.933776in}{6.895323in}}%
\pgfpathlineto{\pgfqpoint{6.940622in}{6.897545in}}%
\pgfpathlineto{\pgfqpoint{6.942904in}{6.876994in}}%
\pgfpathlineto{\pgfqpoint{6.945186in}{6.870329in}}%
\pgfpathlineto{\pgfqpoint{6.947468in}{6.867552in}}%
\pgfpathlineto{\pgfqpoint{6.949750in}{6.866997in}}%
\pgfpathlineto{\pgfqpoint{6.956595in}{6.889214in}}%
\pgfpathlineto{\pgfqpoint{6.961159in}{6.884215in}}%
\pgfpathlineto{\pgfqpoint{6.963441in}{6.831449in}}%
\pgfpathlineto{\pgfqpoint{6.965723in}{6.823673in}}%
\pgfpathlineto{\pgfqpoint{6.972569in}{6.818119in}}%
\pgfpathlineto{\pgfqpoint{6.974851in}{6.835337in}}%
\pgfpathlineto{\pgfqpoint{6.979415in}{6.855888in}}%
\pgfpathlineto{\pgfqpoint{6.981696in}{6.855333in}}%
\pgfpathlineto{\pgfqpoint{6.988542in}{6.857554in}}%
\pgfpathlineto{\pgfqpoint{6.993106in}{6.863664in}}%
\pgfpathlineto{\pgfqpoint{6.995388in}{6.850889in}}%
\pgfpathlineto{\pgfqpoint{6.997670in}{6.811454in}}%
\pgfpathlineto{\pgfqpoint{7.004516in}{6.787015in}}%
\pgfpathlineto{\pgfqpoint{7.006797in}{6.787571in}}%
\pgfpathlineto{\pgfqpoint{7.009079in}{6.795347in}}%
\pgfpathlineto{\pgfqpoint{7.011361in}{6.805344in}}%
\pgfpathlineto{\pgfqpoint{7.013643in}{6.792014in}}%
\pgfpathlineto{\pgfqpoint{7.020489in}{6.795902in}}%
\pgfpathlineto{\pgfqpoint{7.022771in}{6.787571in}}%
\pgfpathlineto{\pgfqpoint{7.025053in}{6.791459in}}%
\pgfpathlineto{\pgfqpoint{7.027335in}{6.803678in}}%
\pgfpathlineto{\pgfqpoint{7.029617in}{6.804789in}}%
\pgfpathlineto{\pgfqpoint{7.038744in}{6.793680in}}%
\pgfpathlineto{\pgfqpoint{7.041026in}{6.801456in}}%
\pgfpathlineto{\pgfqpoint{7.043308in}{6.780350in}}%
\pgfpathlineto{\pgfqpoint{7.045590in}{6.775351in}}%
\pgfpathlineto{\pgfqpoint{7.052436in}{6.783683in}}%
\pgfpathlineto{\pgfqpoint{7.054717in}{6.772574in}}%
\pgfpathlineto{\pgfqpoint{7.056999in}{6.770352in}}%
\pgfpathlineto{\pgfqpoint{7.059281in}{6.752579in}}%
\pgfpathlineto{\pgfqpoint{7.061563in}{6.742581in}}%
\pgfpathlineto{\pgfqpoint{7.068409in}{6.739248in}}%
\pgfpathlineto{\pgfqpoint{7.070691in}{6.744803in}}%
\pgfpathlineto{\pgfqpoint{7.072973in}{6.734250in}}%
\pgfpathlineto{\pgfqpoint{7.075255in}{6.729806in}}%
\pgfpathlineto{\pgfqpoint{7.077537in}{6.739248in}}%
\pgfpathlineto{\pgfqpoint{7.084382in}{6.738693in}}%
\pgfpathlineto{\pgfqpoint{7.086664in}{6.737027in}}%
\pgfpathlineto{\pgfqpoint{7.088946in}{6.728695in}}%
\pgfpathlineto{\pgfqpoint{7.091228in}{6.742581in}}%
\pgfpathlineto{\pgfqpoint{7.093510in}{6.772574in}}%
\pgfpathlineto{\pgfqpoint{7.102638in}{6.752579in}}%
\pgfpathlineto{\pgfqpoint{7.104919in}{6.761465in}}%
\pgfpathlineto{\pgfqpoint{7.107201in}{6.715920in}}%
\pgfpathlineto{\pgfqpoint{7.109483in}{6.705367in}}%
\pgfpathlineto{\pgfqpoint{7.116329in}{6.700368in}}%
\pgfpathlineto{\pgfqpoint{7.120893in}{6.720364in}}%
\pgfpathlineto{\pgfqpoint{7.123175in}{6.727584in}}%
\pgfpathlineto{\pgfqpoint{7.125457in}{6.723141in}}%
\pgfpathlineto{\pgfqpoint{7.132302in}{6.750357in}}%
\pgfpathlineto{\pgfqpoint{7.134584in}{6.737027in}}%
\pgfpathlineto{\pgfqpoint{7.136866in}{6.743136in}}%
\pgfpathlineto{\pgfqpoint{7.139148in}{6.765353in}}%
\pgfpathlineto{\pgfqpoint{7.141430in}{6.771463in}}%
\pgfpathlineto{\pgfqpoint{7.148276in}{6.783683in}}%
\pgfpathlineto{\pgfqpoint{7.150558in}{6.774240in}}%
\pgfpathlineto{\pgfqpoint{7.152839in}{6.745913in}}%
\pgfpathlineto{\pgfqpoint{7.155121in}{6.739248in}}%
\pgfpathlineto{\pgfqpoint{7.157403in}{6.735916in}}%
\pgfpathlineto{\pgfqpoint{7.164249in}{6.755911in}}%
\pgfpathlineto{\pgfqpoint{7.166531in}{6.767575in}}%
\pgfpathlineto{\pgfqpoint{7.168813in}{6.751468in}}%
\pgfpathlineto{\pgfqpoint{7.171095in}{6.754245in}}%
\pgfpathlineto{\pgfqpoint{7.173377in}{6.746469in}}%
\pgfpathlineto{\pgfqpoint{7.180222in}{6.697591in}}%
\pgfpathlineto{\pgfqpoint{7.182504in}{6.694259in}}%
\pgfpathlineto{\pgfqpoint{7.184786in}{6.679262in}}%
\pgfpathlineto{\pgfqpoint{7.187068in}{6.678151in}}%
\pgfpathlineto{\pgfqpoint{7.189350in}{6.674819in}}%
\pgfpathlineto{\pgfqpoint{7.196196in}{6.694259in}}%
\pgfpathlineto{\pgfqpoint{7.198478in}{6.685372in}}%
\pgfpathlineto{\pgfqpoint{7.200760in}{6.682039in}}%
\pgfpathlineto{\pgfqpoint{7.203041in}{6.705367in}}%
\pgfpathlineto{\pgfqpoint{7.205323in}{6.718142in}}%
\pgfpathlineto{\pgfqpoint{7.214451in}{6.618721in}}%
\pgfpathlineto{\pgfqpoint{7.216733in}{6.599281in}}%
\pgfpathlineto{\pgfqpoint{7.219015in}{6.589838in}}%
\pgfpathlineto{\pgfqpoint{7.221297in}{6.562623in}}%
\pgfpathlineto{\pgfqpoint{7.228142in}{6.543183in}}%
\pgfpathlineto{\pgfqpoint{7.230424in}{6.530408in}}%
\pgfpathlineto{\pgfqpoint{7.232706in}{6.523187in}}%
\pgfpathlineto{\pgfqpoint{7.234988in}{6.519299in}}%
\pgfpathlineto{\pgfqpoint{7.237270in}{6.529297in}}%
\pgfpathlineto{\pgfqpoint{7.244116in}{6.528741in}}%
\pgfpathlineto{\pgfqpoint{7.246398in}{6.533185in}}%
\pgfpathlineto{\pgfqpoint{7.248680in}{6.528741in}}%
\pgfpathlineto{\pgfqpoint{7.250961in}{6.522076in}}%
\pgfpathlineto{\pgfqpoint{7.253243in}{6.547626in}}%
\pgfpathlineto{\pgfqpoint{7.260089in}{6.472088in}}%
\pgfpathlineto{\pgfqpoint{7.262371in}{6.414879in}}%
\pgfpathlineto{\pgfqpoint{7.264653in}{6.433208in}}%
\pgfpathlineto{\pgfqpoint{7.266935in}{6.432653in}}%
\pgfpathlineto{\pgfqpoint{7.269217in}{6.430986in}}%
\pgfpathlineto{\pgfqpoint{7.276062in}{6.419322in}}%
\pgfpathlineto{\pgfqpoint{7.278344in}{6.411546in}}%
\pgfpathlineto{\pgfqpoint{7.280626in}{6.427654in}}%
\pgfpathlineto{\pgfqpoint{7.285190in}{6.429875in}}%
\pgfpathlineto{\pgfqpoint{7.292036in}{6.425987in}}%
\pgfpathlineto{\pgfqpoint{7.294318in}{6.440984in}}%
\pgfpathlineto{\pgfqpoint{7.296600in}{6.444872in}}%
\pgfpathlineto{\pgfqpoint{7.298882in}{6.434874in}}%
\pgfpathlineto{\pgfqpoint{7.301163in}{6.413768in}}%
\pgfpathlineto{\pgfqpoint{7.308009in}{6.417656in}}%
\pgfpathlineto{\pgfqpoint{7.310291in}{6.407658in}}%
\pgfpathlineto{\pgfqpoint{7.314855in}{6.411546in}}%
\pgfpathlineto{\pgfqpoint{7.317137in}{6.411546in}}%
\pgfpathlineto{\pgfqpoint{7.323982in}{6.408769in}}%
\pgfpathlineto{\pgfqpoint{7.326264in}{6.422099in}}%
\pgfpathlineto{\pgfqpoint{7.330828in}{6.408214in}}%
\pgfpathlineto{\pgfqpoint{7.333110in}{6.417101in}}%
\pgfpathlineto{\pgfqpoint{7.339956in}{6.414323in}}%
\pgfpathlineto{\pgfqpoint{7.344520in}{6.398216in}}%
\pgfpathlineto{\pgfqpoint{7.349083in}{6.400993in}}%
\pgfpathlineto{\pgfqpoint{7.358211in}{6.403215in}}%
\pgfpathlineto{\pgfqpoint{7.360493in}{6.400993in}}%
\pgfpathlineto{\pgfqpoint{7.362775in}{6.399882in}}%
\pgfpathlineto{\pgfqpoint{7.365057in}{6.404326in}}%
\pgfpathlineto{\pgfqpoint{7.365057in}{6.404326in}}%
\pgfusepath{stroke}%
\end{pgfscope}%
\begin{pgfscope}%
\pgfsetrectcap%
\pgfsetmiterjoin%
\pgfsetlinewidth{0.803000pt}%
\definecolor{currentstroke}{rgb}{1.000000,1.000000,1.000000}%
\pgfsetstrokecolor{currentstroke}%
\pgfsetdash{}{0pt}%
\pgfpathmoveto{\pgfqpoint{2.125000in}{6.221951in}}%
\pgfpathlineto{\pgfqpoint{2.125000in}{7.142683in}}%
\pgfusepath{stroke}%
\end{pgfscope}%
\begin{pgfscope}%
\pgfsetrectcap%
\pgfsetmiterjoin%
\pgfsetlinewidth{0.803000pt}%
\definecolor{currentstroke}{rgb}{1.000000,1.000000,1.000000}%
\pgfsetstrokecolor{currentstroke}%
\pgfsetdash{}{0pt}%
\pgfpathmoveto{\pgfqpoint{7.614583in}{6.221951in}}%
\pgfpathlineto{\pgfqpoint{7.614583in}{7.142683in}}%
\pgfusepath{stroke}%
\end{pgfscope}%
\begin{pgfscope}%
\pgfsetrectcap%
\pgfsetmiterjoin%
\pgfsetlinewidth{0.803000pt}%
\definecolor{currentstroke}{rgb}{1.000000,1.000000,1.000000}%
\pgfsetstrokecolor{currentstroke}%
\pgfsetdash{}{0pt}%
\pgfpathmoveto{\pgfqpoint{2.125000in}{6.221951in}}%
\pgfpathlineto{\pgfqpoint{7.614583in}{6.221951in}}%
\pgfusepath{stroke}%
\end{pgfscope}%
\begin{pgfscope}%
\pgfsetrectcap%
\pgfsetmiterjoin%
\pgfsetlinewidth{0.803000pt}%
\definecolor{currentstroke}{rgb}{1.000000,1.000000,1.000000}%
\pgfsetstrokecolor{currentstroke}%
\pgfsetdash{}{0pt}%
\pgfpathmoveto{\pgfqpoint{2.125000in}{7.142683in}}%
\pgfpathlineto{\pgfqpoint{7.614583in}{7.142683in}}%
\pgfusepath{stroke}%
\end{pgfscope}%
\begin{pgfscope}%
\definecolor{textcolor}{rgb}{0.150000,0.150000,0.150000}%
\pgfsetstrokecolor{textcolor}%
\pgfsetfillcolor{textcolor}%
\pgftext[x=4.869792in,y=7.226016in,,base]{\color{textcolor}\rmfamily\fontsize{12.000000}{14.400000}\selectfont GE}%
\end{pgfscope}%
\begin{pgfscope}%
\pgfsetbuttcap%
\pgfsetmiterjoin%
\definecolor{currentfill}{rgb}{0.917647,0.917647,0.949020}%
\pgfsetfillcolor{currentfill}%
\pgfsetlinewidth{0.000000pt}%
\definecolor{currentstroke}{rgb}{0.000000,0.000000,0.000000}%
\pgfsetstrokecolor{currentstroke}%
\pgfsetstrokeopacity{0.000000}%
\pgfsetdash{}{0pt}%
\pgfpathmoveto{\pgfqpoint{9.810417in}{6.221951in}}%
\pgfpathlineto{\pgfqpoint{15.300000in}{6.221951in}}%
\pgfpathlineto{\pgfqpoint{15.300000in}{7.142683in}}%
\pgfpathlineto{\pgfqpoint{9.810417in}{7.142683in}}%
\pgfpathclose%
\pgfusepath{fill}%
\end{pgfscope}%
\begin{pgfscope}%
\pgfpathrectangle{\pgfqpoint{9.810417in}{6.221951in}}{\pgfqpoint{5.489583in}{0.920732in}}%
\pgfusepath{clip}%
\pgfsetroundcap%
\pgfsetroundjoin%
\pgfsetlinewidth{0.803000pt}%
\definecolor{currentstroke}{rgb}{1.000000,1.000000,1.000000}%
\pgfsetstrokecolor{currentstroke}%
\pgfsetdash{}{0pt}%
\pgfpathmoveto{\pgfqpoint{10.055379in}{6.221951in}}%
\pgfpathlineto{\pgfqpoint{10.055379in}{7.142683in}}%
\pgfusepath{stroke}%
\end{pgfscope}%
\begin{pgfscope}%
\definecolor{textcolor}{rgb}{0.150000,0.150000,0.150000}%
\pgfsetstrokecolor{textcolor}%
\pgfsetfillcolor{textcolor}%
\pgftext[x=10.055379in,y=6.124729in,,top]{\color{textcolor}\rmfamily\fontsize{10.000000}{12.000000}\selectfont 2012}%
\end{pgfscope}%
\begin{pgfscope}%
\pgfpathrectangle{\pgfqpoint{9.810417in}{6.221951in}}{\pgfqpoint{5.489583in}{0.920732in}}%
\pgfusepath{clip}%
\pgfsetroundcap%
\pgfsetroundjoin%
\pgfsetlinewidth{0.803000pt}%
\definecolor{currentstroke}{rgb}{1.000000,1.000000,1.000000}%
\pgfsetstrokecolor{currentstroke}%
\pgfsetdash{}{0pt}%
\pgfpathmoveto{\pgfqpoint{10.890557in}{6.221951in}}%
\pgfpathlineto{\pgfqpoint{10.890557in}{7.142683in}}%
\pgfusepath{stroke}%
\end{pgfscope}%
\begin{pgfscope}%
\definecolor{textcolor}{rgb}{0.150000,0.150000,0.150000}%
\pgfsetstrokecolor{textcolor}%
\pgfsetfillcolor{textcolor}%
\pgftext[x=10.890557in,y=6.124729in,,top]{\color{textcolor}\rmfamily\fontsize{10.000000}{12.000000}\selectfont 2013}%
\end{pgfscope}%
\begin{pgfscope}%
\pgfpathrectangle{\pgfqpoint{9.810417in}{6.221951in}}{\pgfqpoint{5.489583in}{0.920732in}}%
\pgfusepath{clip}%
\pgfsetroundcap%
\pgfsetroundjoin%
\pgfsetlinewidth{0.803000pt}%
\definecolor{currentstroke}{rgb}{1.000000,1.000000,1.000000}%
\pgfsetstrokecolor{currentstroke}%
\pgfsetdash{}{0pt}%
\pgfpathmoveto{\pgfqpoint{11.723453in}{6.221951in}}%
\pgfpathlineto{\pgfqpoint{11.723453in}{7.142683in}}%
\pgfusepath{stroke}%
\end{pgfscope}%
\begin{pgfscope}%
\definecolor{textcolor}{rgb}{0.150000,0.150000,0.150000}%
\pgfsetstrokecolor{textcolor}%
\pgfsetfillcolor{textcolor}%
\pgftext[x=11.723453in,y=6.124729in,,top]{\color{textcolor}\rmfamily\fontsize{10.000000}{12.000000}\selectfont 2014}%
\end{pgfscope}%
\begin{pgfscope}%
\pgfpathrectangle{\pgfqpoint{9.810417in}{6.221951in}}{\pgfqpoint{5.489583in}{0.920732in}}%
\pgfusepath{clip}%
\pgfsetroundcap%
\pgfsetroundjoin%
\pgfsetlinewidth{0.803000pt}%
\definecolor{currentstroke}{rgb}{1.000000,1.000000,1.000000}%
\pgfsetstrokecolor{currentstroke}%
\pgfsetdash{}{0pt}%
\pgfpathmoveto{\pgfqpoint{12.556349in}{6.221951in}}%
\pgfpathlineto{\pgfqpoint{12.556349in}{7.142683in}}%
\pgfusepath{stroke}%
\end{pgfscope}%
\begin{pgfscope}%
\definecolor{textcolor}{rgb}{0.150000,0.150000,0.150000}%
\pgfsetstrokecolor{textcolor}%
\pgfsetfillcolor{textcolor}%
\pgftext[x=12.556349in,y=6.124729in,,top]{\color{textcolor}\rmfamily\fontsize{10.000000}{12.000000}\selectfont 2015}%
\end{pgfscope}%
\begin{pgfscope}%
\pgfpathrectangle{\pgfqpoint{9.810417in}{6.221951in}}{\pgfqpoint{5.489583in}{0.920732in}}%
\pgfusepath{clip}%
\pgfsetroundcap%
\pgfsetroundjoin%
\pgfsetlinewidth{0.803000pt}%
\definecolor{currentstroke}{rgb}{1.000000,1.000000,1.000000}%
\pgfsetstrokecolor{currentstroke}%
\pgfsetdash{}{0pt}%
\pgfpathmoveto{\pgfqpoint{13.389245in}{6.221951in}}%
\pgfpathlineto{\pgfqpoint{13.389245in}{7.142683in}}%
\pgfusepath{stroke}%
\end{pgfscope}%
\begin{pgfscope}%
\definecolor{textcolor}{rgb}{0.150000,0.150000,0.150000}%
\pgfsetstrokecolor{textcolor}%
\pgfsetfillcolor{textcolor}%
\pgftext[x=13.389245in,y=6.124729in,,top]{\color{textcolor}\rmfamily\fontsize{10.000000}{12.000000}\selectfont 2016}%
\end{pgfscope}%
\begin{pgfscope}%
\pgfpathrectangle{\pgfqpoint{9.810417in}{6.221951in}}{\pgfqpoint{5.489583in}{0.920732in}}%
\pgfusepath{clip}%
\pgfsetroundcap%
\pgfsetroundjoin%
\pgfsetlinewidth{0.803000pt}%
\definecolor{currentstroke}{rgb}{1.000000,1.000000,1.000000}%
\pgfsetstrokecolor{currentstroke}%
\pgfsetdash{}{0pt}%
\pgfpathmoveto{\pgfqpoint{14.224423in}{6.221951in}}%
\pgfpathlineto{\pgfqpoint{14.224423in}{7.142683in}}%
\pgfusepath{stroke}%
\end{pgfscope}%
\begin{pgfscope}%
\definecolor{textcolor}{rgb}{0.150000,0.150000,0.150000}%
\pgfsetstrokecolor{textcolor}%
\pgfsetfillcolor{textcolor}%
\pgftext[x=14.224423in,y=6.124729in,,top]{\color{textcolor}\rmfamily\fontsize{10.000000}{12.000000}\selectfont 2017}%
\end{pgfscope}%
\begin{pgfscope}%
\pgfpathrectangle{\pgfqpoint{9.810417in}{6.221951in}}{\pgfqpoint{5.489583in}{0.920732in}}%
\pgfusepath{clip}%
\pgfsetroundcap%
\pgfsetroundjoin%
\pgfsetlinewidth{0.803000pt}%
\definecolor{currentstroke}{rgb}{1.000000,1.000000,1.000000}%
\pgfsetstrokecolor{currentstroke}%
\pgfsetdash{}{0pt}%
\pgfpathmoveto{\pgfqpoint{15.057319in}{6.221951in}}%
\pgfpathlineto{\pgfqpoint{15.057319in}{7.142683in}}%
\pgfusepath{stroke}%
\end{pgfscope}%
\begin{pgfscope}%
\definecolor{textcolor}{rgb}{0.150000,0.150000,0.150000}%
\pgfsetstrokecolor{textcolor}%
\pgfsetfillcolor{textcolor}%
\pgftext[x=15.057319in,y=6.124729in,,top]{\color{textcolor}\rmfamily\fontsize{10.000000}{12.000000}\selectfont 2018}%
\end{pgfscope}%
\begin{pgfscope}%
\pgfpathrectangle{\pgfqpoint{9.810417in}{6.221951in}}{\pgfqpoint{5.489583in}{0.920732in}}%
\pgfusepath{clip}%
\pgfsetroundcap%
\pgfsetroundjoin%
\pgfsetlinewidth{0.803000pt}%
\definecolor{currentstroke}{rgb}{1.000000,1.000000,1.000000}%
\pgfsetstrokecolor{currentstroke}%
\pgfsetdash{}{0pt}%
\pgfpathmoveto{\pgfqpoint{9.810417in}{6.378744in}}%
\pgfpathlineto{\pgfqpoint{15.300000in}{6.378744in}}%
\pgfusepath{stroke}%
\end{pgfscope}%
\begin{pgfscope}%
\definecolor{textcolor}{rgb}{0.150000,0.150000,0.150000}%
\pgfsetstrokecolor{textcolor}%
\pgfsetfillcolor{textcolor}%
\pgftext[x=9.536464in,y=6.325982in,left,base]{\color{textcolor}\rmfamily\fontsize{10.000000}{12.000000}\selectfont 20}%
\end{pgfscope}%
\begin{pgfscope}%
\pgfpathrectangle{\pgfqpoint{9.810417in}{6.221951in}}{\pgfqpoint{5.489583in}{0.920732in}}%
\pgfusepath{clip}%
\pgfsetroundcap%
\pgfsetroundjoin%
\pgfsetlinewidth{0.803000pt}%
\definecolor{currentstroke}{rgb}{1.000000,1.000000,1.000000}%
\pgfsetstrokecolor{currentstroke}%
\pgfsetdash{}{0pt}%
\pgfpathmoveto{\pgfqpoint{9.810417in}{6.938068in}}%
\pgfpathlineto{\pgfqpoint{15.300000in}{6.938068in}}%
\pgfusepath{stroke}%
\end{pgfscope}%
\begin{pgfscope}%
\definecolor{textcolor}{rgb}{0.150000,0.150000,0.150000}%
\pgfsetstrokecolor{textcolor}%
\pgfsetfillcolor{textcolor}%
\pgftext[x=9.536464in,y=6.885307in,left,base]{\color{textcolor}\rmfamily\fontsize{10.000000}{12.000000}\selectfont 40}%
\end{pgfscope}%
\begin{pgfscope}%
\pgfpathrectangle{\pgfqpoint{9.810417in}{6.221951in}}{\pgfqpoint{5.489583in}{0.920732in}}%
\pgfusepath{clip}%
\pgfsetroundcap%
\pgfsetroundjoin%
\pgfsetlinewidth{1.505625pt}%
\definecolor{currentstroke}{rgb}{0.121569,0.466667,0.705882}%
\pgfsetstrokecolor{currentstroke}%
\pgfsetdash{}{0pt}%
\pgfpathmoveto{\pgfqpoint{10.059943in}{6.363642in}}%
\pgfpathlineto{\pgfqpoint{10.062225in}{6.376507in}}%
\pgfpathlineto{\pgfqpoint{10.064507in}{6.382939in}}%
\pgfpathlineto{\pgfqpoint{10.066789in}{6.379583in}}%
\pgfpathlineto{\pgfqpoint{10.073635in}{6.384337in}}%
\pgfpathlineto{\pgfqpoint{10.075917in}{6.387134in}}%
\pgfpathlineto{\pgfqpoint{10.078198in}{6.391608in}}%
\pgfpathlineto{\pgfqpoint{10.080480in}{6.390490in}}%
\pgfpathlineto{\pgfqpoint{10.082762in}{6.377066in}}%
\pgfpathlineto{\pgfqpoint{10.091890in}{6.374829in}}%
\pgfpathlineto{\pgfqpoint{10.094172in}{6.382659in}}%
\pgfpathlineto{\pgfqpoint{10.096454in}{6.387973in}}%
\pgfpathlineto{\pgfqpoint{10.098736in}{6.404473in}}%
\pgfpathlineto{\pgfqpoint{10.105581in}{6.411744in}}%
\pgfpathlineto{\pgfqpoint{10.107863in}{6.415939in}}%
\pgfpathlineto{\pgfqpoint{10.110145in}{6.415939in}}%
\pgfpathlineto{\pgfqpoint{10.112427in}{6.412863in}}%
\pgfpathlineto{\pgfqpoint{10.114709in}{6.412303in}}%
\pgfpathlineto{\pgfqpoint{10.121555in}{6.412583in}}%
\pgfpathlineto{\pgfqpoint{10.123837in}{6.405312in}}%
\pgfpathlineto{\pgfqpoint{10.126118in}{6.408388in}}%
\pgfpathlineto{\pgfqpoint{10.128400in}{6.406990in}}%
\pgfpathlineto{\pgfqpoint{10.130682in}{6.417337in}}%
\pgfpathlineto{\pgfqpoint{10.137528in}{6.416778in}}%
\pgfpathlineto{\pgfqpoint{10.139810in}{6.415100in}}%
\pgfpathlineto{\pgfqpoint{10.142092in}{6.419854in}}%
\pgfpathlineto{\pgfqpoint{10.144374in}{6.419854in}}%
\pgfpathlineto{\pgfqpoint{10.146656in}{6.416498in}}%
\pgfpathlineto{\pgfqpoint{10.153501in}{6.416498in}}%
\pgfpathlineto{\pgfqpoint{10.155783in}{6.418176in}}%
\pgfpathlineto{\pgfqpoint{10.158065in}{6.413702in}}%
\pgfpathlineto{\pgfqpoint{10.160347in}{6.419295in}}%
\pgfpathlineto{\pgfqpoint{10.162629in}{6.431320in}}%
\pgfpathlineto{\pgfqpoint{10.171757in}{6.426566in}}%
\pgfpathlineto{\pgfqpoint{10.174039in}{6.417058in}}%
\pgfpathlineto{\pgfqpoint{10.176320in}{6.415380in}}%
\pgfpathlineto{\pgfqpoint{10.178602in}{6.416498in}}%
\pgfpathlineto{\pgfqpoint{10.185448in}{6.420693in}}%
\pgfpathlineto{\pgfqpoint{10.187730in}{6.428524in}}%
\pgfpathlineto{\pgfqpoint{10.190012in}{6.420413in}}%
\pgfpathlineto{\pgfqpoint{10.192294in}{6.419854in}}%
\pgfpathlineto{\pgfqpoint{10.194576in}{6.421252in}}%
\pgfpathlineto{\pgfqpoint{10.201421in}{6.412863in}}%
\pgfpathlineto{\pgfqpoint{10.203703in}{6.414261in}}%
\pgfpathlineto{\pgfqpoint{10.205985in}{6.420973in}}%
\pgfpathlineto{\pgfqpoint{10.208267in}{6.419574in}}%
\pgfpathlineto{\pgfqpoint{10.210549in}{6.424608in}}%
\pgfpathlineto{\pgfqpoint{10.217395in}{6.422930in}}%
\pgfpathlineto{\pgfqpoint{10.219677in}{6.434117in}}%
\pgfpathlineto{\pgfqpoint{10.221959in}{6.433278in}}%
\pgfpathlineto{\pgfqpoint{10.224240in}{6.439710in}}%
\pgfpathlineto{\pgfqpoint{10.226522in}{6.439430in}}%
\pgfpathlineto{\pgfqpoint{10.235650in}{6.439710in}}%
\pgfpathlineto{\pgfqpoint{10.237932in}{6.440549in}}%
\pgfpathlineto{\pgfqpoint{10.240214in}{6.443066in}}%
\pgfpathlineto{\pgfqpoint{10.242496in}{6.442786in}}%
\pgfpathlineto{\pgfqpoint{10.249341in}{6.449778in}}%
\pgfpathlineto{\pgfqpoint{10.251623in}{6.449778in}}%
\pgfpathlineto{\pgfqpoint{10.253905in}{6.440829in}}%
\pgfpathlineto{\pgfqpoint{10.256187in}{6.448939in}}%
\pgfpathlineto{\pgfqpoint{10.258469in}{6.448100in}}%
\pgfpathlineto{\pgfqpoint{10.265315in}{6.453973in}}%
\pgfpathlineto{\pgfqpoint{10.267597in}{6.447820in}}%
\pgfpathlineto{\pgfqpoint{10.269879in}{6.443905in}}%
\pgfpathlineto{\pgfqpoint{10.272161in}{6.446981in}}%
\pgfpathlineto{\pgfqpoint{10.281288in}{6.439990in}}%
\pgfpathlineto{\pgfqpoint{10.283570in}{6.433278in}}%
\pgfpathlineto{\pgfqpoint{10.285852in}{6.441947in}}%
\pgfpathlineto{\pgfqpoint{10.288134in}{6.456210in}}%
\pgfpathlineto{\pgfqpoint{10.290416in}{6.447541in}}%
\pgfpathlineto{\pgfqpoint{10.297261in}{6.454532in}}%
\pgfpathlineto{\pgfqpoint{10.299543in}{6.455931in}}%
\pgfpathlineto{\pgfqpoint{10.301825in}{6.444185in}}%
\pgfpathlineto{\pgfqpoint{10.304107in}{6.438592in}}%
\pgfpathlineto{\pgfqpoint{10.306389in}{6.436634in}}%
\pgfpathlineto{\pgfqpoint{10.313235in}{6.433278in}}%
\pgfpathlineto{\pgfqpoint{10.315517in}{6.429922in}}%
\pgfpathlineto{\pgfqpoint{10.317799in}{6.442227in}}%
\pgfpathlineto{\pgfqpoint{10.320081in}{6.450337in}}%
\pgfpathlineto{\pgfqpoint{10.322362in}{6.453973in}}%
\pgfpathlineto{\pgfqpoint{10.329208in}{6.454253in}}%
\pgfpathlineto{\pgfqpoint{10.331490in}{6.466558in}}%
\pgfpathlineto{\pgfqpoint{10.333772in}{6.471871in}}%
\pgfpathlineto{\pgfqpoint{10.336054in}{6.462642in}}%
\pgfpathlineto{\pgfqpoint{10.338336in}{6.447820in}}%
\pgfpathlineto{\pgfqpoint{10.345182in}{6.444464in}}%
\pgfpathlineto{\pgfqpoint{10.347463in}{6.435795in}}%
\pgfpathlineto{\pgfqpoint{10.349745in}{6.431600in}}%
\pgfpathlineto{\pgfqpoint{10.352027in}{6.432998in}}%
\pgfpathlineto{\pgfqpoint{10.354309in}{6.441668in}}%
\pgfpathlineto{\pgfqpoint{10.361155in}{6.427964in}}%
\pgfpathlineto{\pgfqpoint{10.363437in}{6.424888in}}%
\pgfpathlineto{\pgfqpoint{10.368001in}{6.409227in}}%
\pgfpathlineto{\pgfqpoint{10.370283in}{6.406430in}}%
\pgfpathlineto{\pgfqpoint{10.377128in}{6.408388in}}%
\pgfpathlineto{\pgfqpoint{10.379410in}{6.405591in}}%
\pgfpathlineto{\pgfqpoint{10.381692in}{6.392447in}}%
\pgfpathlineto{\pgfqpoint{10.383974in}{6.396922in}}%
\pgfpathlineto{\pgfqpoint{10.386256in}{6.399159in}}%
\pgfpathlineto{\pgfqpoint{10.395383in}{6.406990in}}%
\pgfpathlineto{\pgfqpoint{10.397665in}{6.407829in}}%
\pgfpathlineto{\pgfqpoint{10.399947in}{6.401396in}}%
\pgfpathlineto{\pgfqpoint{10.402229in}{6.385456in}}%
\pgfpathlineto{\pgfqpoint{10.409075in}{6.383218in}}%
\pgfpathlineto{\pgfqpoint{10.411357in}{6.392168in}}%
\pgfpathlineto{\pgfqpoint{10.413639in}{6.406430in}}%
\pgfpathlineto{\pgfqpoint{10.415921in}{6.403634in}}%
\pgfpathlineto{\pgfqpoint{10.418203in}{6.414261in}}%
\pgfpathlineto{\pgfqpoint{10.425048in}{6.404752in}}%
\pgfpathlineto{\pgfqpoint{10.427330in}{6.416778in}}%
\pgfpathlineto{\pgfqpoint{10.429612in}{6.417058in}}%
\pgfpathlineto{\pgfqpoint{10.434176in}{6.435236in}}%
\pgfpathlineto{\pgfqpoint{10.441022in}{6.436914in}}%
\pgfpathlineto{\pgfqpoint{10.443304in}{6.438871in}}%
\pgfpathlineto{\pgfqpoint{10.445585in}{6.441947in}}%
\pgfpathlineto{\pgfqpoint{10.447867in}{6.420973in}}%
\pgfpathlineto{\pgfqpoint{10.450149in}{6.426007in}}%
\pgfpathlineto{\pgfqpoint{10.456995in}{6.406151in}}%
\pgfpathlineto{\pgfqpoint{10.459277in}{6.405032in}}%
\pgfpathlineto{\pgfqpoint{10.461559in}{6.409786in}}%
\pgfpathlineto{\pgfqpoint{10.463841in}{6.401117in}}%
\pgfpathlineto{\pgfqpoint{10.466123in}{6.419574in}}%
\pgfpathlineto{\pgfqpoint{10.472968in}{6.420134in}}%
\pgfpathlineto{\pgfqpoint{10.475250in}{6.424329in}}%
\pgfpathlineto{\pgfqpoint{10.479814in}{6.417337in}}%
\pgfpathlineto{\pgfqpoint{10.482096in}{6.408668in}}%
\pgfpathlineto{\pgfqpoint{10.488942in}{6.408668in}}%
\pgfpathlineto{\pgfqpoint{10.491224in}{6.394964in}}%
\pgfpathlineto{\pgfqpoint{10.493505in}{6.391329in}}%
\pgfpathlineto{\pgfqpoint{10.495787in}{6.376507in}}%
\pgfpathlineto{\pgfqpoint{10.498069in}{6.387973in}}%
\pgfpathlineto{\pgfqpoint{10.504915in}{6.385456in}}%
\pgfpathlineto{\pgfqpoint{10.507197in}{6.391049in}}%
\pgfpathlineto{\pgfqpoint{10.509479in}{6.409786in}}%
\pgfpathlineto{\pgfqpoint{10.511761in}{6.406151in}}%
\pgfpathlineto{\pgfqpoint{10.514043in}{6.394125in}}%
\pgfpathlineto{\pgfqpoint{10.520888in}{6.388252in}}%
\pgfpathlineto{\pgfqpoint{10.523170in}{6.382659in}}%
\pgfpathlineto{\pgfqpoint{10.525452in}{6.385456in}}%
\pgfpathlineto{\pgfqpoint{10.527734in}{6.393566in}}%
\pgfpathlineto{\pgfqpoint{10.530016in}{6.405312in}}%
\pgfpathlineto{\pgfqpoint{10.539144in}{6.398040in}}%
\pgfpathlineto{\pgfqpoint{10.541426in}{6.403354in}}%
\pgfpathlineto{\pgfqpoint{10.543707in}{6.402795in}}%
\pgfpathlineto{\pgfqpoint{10.545989in}{6.415380in}}%
\pgfpathlineto{\pgfqpoint{10.552835in}{6.417058in}}%
\pgfpathlineto{\pgfqpoint{10.555117in}{6.421532in}}%
\pgfpathlineto{\pgfqpoint{10.559681in}{6.426007in}}%
\pgfpathlineto{\pgfqpoint{10.561963in}{6.429922in}}%
\pgfpathlineto{\pgfqpoint{10.568808in}{6.425727in}}%
\pgfpathlineto{\pgfqpoint{10.573372in}{6.416219in}}%
\pgfpathlineto{\pgfqpoint{10.575654in}{6.423490in}}%
\pgfpathlineto{\pgfqpoint{10.577936in}{6.417617in}}%
\pgfpathlineto{\pgfqpoint{10.584782in}{6.415380in}}%
\pgfpathlineto{\pgfqpoint{10.587064in}{6.412583in}}%
\pgfpathlineto{\pgfqpoint{10.589346in}{6.403913in}}%
\pgfpathlineto{\pgfqpoint{10.591627in}{6.388252in}}%
\pgfpathlineto{\pgfqpoint{10.593909in}{6.385176in}}%
\pgfpathlineto{\pgfqpoint{10.600755in}{6.383778in}}%
\pgfpathlineto{\pgfqpoint{10.603037in}{6.387413in}}%
\pgfpathlineto{\pgfqpoint{10.607601in}{6.370634in}}%
\pgfpathlineto{\pgfqpoint{10.609883in}{6.383498in}}%
\pgfpathlineto{\pgfqpoint{10.619010in}{6.374269in}}%
\pgfpathlineto{\pgfqpoint{10.621292in}{6.373430in}}%
\pgfpathlineto{\pgfqpoint{10.623574in}{6.389651in}}%
\pgfpathlineto{\pgfqpoint{10.625856in}{6.368956in}}%
\pgfpathlineto{\pgfqpoint{10.632702in}{6.347701in}}%
\pgfpathlineto{\pgfqpoint{10.634984in}{6.349659in}}%
\pgfpathlineto{\pgfqpoint{10.637266in}{6.346303in}}%
\pgfpathlineto{\pgfqpoint{10.639548in}{6.350218in}}%
\pgfpathlineto{\pgfqpoint{10.641829in}{6.350218in}}%
\pgfpathlineto{\pgfqpoint{10.648675in}{6.348820in}}%
\pgfpathlineto{\pgfqpoint{10.650957in}{6.350218in}}%
\pgfpathlineto{\pgfqpoint{10.653239in}{6.345184in}}%
\pgfpathlineto{\pgfqpoint{10.655521in}{6.346023in}}%
\pgfpathlineto{\pgfqpoint{10.657803in}{6.344905in}}%
\pgfpathlineto{\pgfqpoint{10.664648in}{6.337354in}}%
\pgfpathlineto{\pgfqpoint{10.666930in}{6.331481in}}%
\pgfpathlineto{\pgfqpoint{10.669212in}{6.333998in}}%
\pgfpathlineto{\pgfqpoint{10.671494in}{6.344066in}}%
\pgfpathlineto{\pgfqpoint{10.673776in}{6.334278in}}%
\pgfpathlineto{\pgfqpoint{10.680622in}{6.336515in}}%
\pgfpathlineto{\pgfqpoint{10.682904in}{6.338193in}}%
\pgfpathlineto{\pgfqpoint{10.685186in}{6.331761in}}%
\pgfpathlineto{\pgfqpoint{10.687468in}{6.329803in}}%
\pgfpathlineto{\pgfqpoint{10.689749in}{6.334557in}}%
\pgfpathlineto{\pgfqpoint{10.696595in}{6.330922in}}%
\pgfpathlineto{\pgfqpoint{10.698877in}{6.316938in}}%
\pgfpathlineto{\pgfqpoint{10.701159in}{6.313862in}}%
\pgfpathlineto{\pgfqpoint{10.703441in}{6.311905in}}%
\pgfpathlineto{\pgfqpoint{10.705723in}{6.307430in}}%
\pgfpathlineto{\pgfqpoint{10.712569in}{6.313023in}}%
\pgfpathlineto{\pgfqpoint{10.714850in}{6.327286in}}%
\pgfpathlineto{\pgfqpoint{10.717132in}{6.314422in}}%
\pgfpathlineto{\pgfqpoint{10.719414in}{6.311625in}}%
\pgfpathlineto{\pgfqpoint{10.721696in}{6.302676in}}%
\pgfpathlineto{\pgfqpoint{10.728542in}{6.306871in}}%
\pgfpathlineto{\pgfqpoint{10.730824in}{6.309947in}}%
\pgfpathlineto{\pgfqpoint{10.733106in}{6.306871in}}%
\pgfpathlineto{\pgfqpoint{10.737670in}{6.318057in}}%
\pgfpathlineto{\pgfqpoint{10.749079in}{6.310786in}}%
\pgfpathlineto{\pgfqpoint{10.751361in}{6.325049in}}%
\pgfpathlineto{\pgfqpoint{10.753643in}{6.320574in}}%
\pgfpathlineto{\pgfqpoint{10.760489in}{6.320574in}}%
\pgfpathlineto{\pgfqpoint{10.762770in}{6.318057in}}%
\pgfpathlineto{\pgfqpoint{10.765052in}{6.299320in}}%
\pgfpathlineto{\pgfqpoint{10.767334in}{6.297362in}}%
\pgfpathlineto{\pgfqpoint{10.769616in}{6.296803in}}%
\pgfpathlineto{\pgfqpoint{10.776462in}{6.296243in}}%
\pgfpathlineto{\pgfqpoint{10.778744in}{6.284777in}}%
\pgfpathlineto{\pgfqpoint{10.781026in}{6.277506in}}%
\pgfpathlineto{\pgfqpoint{10.783308in}{6.279184in}}%
\pgfpathlineto{\pgfqpoint{10.785590in}{6.282820in}}%
\pgfpathlineto{\pgfqpoint{10.792435in}{6.284218in}}%
\pgfpathlineto{\pgfqpoint{10.794717in}{6.267159in}}%
\pgfpathlineto{\pgfqpoint{10.796999in}{6.263803in}}%
\pgfpathlineto{\pgfqpoint{10.801563in}{6.271913in}}%
\pgfpathlineto{\pgfqpoint{10.810691in}{6.276947in}}%
\pgfpathlineto{\pgfqpoint{10.812972in}{6.280582in}}%
\pgfpathlineto{\pgfqpoint{10.815254in}{6.267718in}}%
\pgfpathlineto{\pgfqpoint{10.817536in}{6.268557in}}%
\pgfpathlineto{\pgfqpoint{10.824382in}{6.267998in}}%
\pgfpathlineto{\pgfqpoint{10.826664in}{6.277786in}}%
\pgfpathlineto{\pgfqpoint{10.828946in}{6.274989in}}%
\pgfpathlineto{\pgfqpoint{10.831228in}{6.281981in}}%
\pgfpathlineto{\pgfqpoint{10.833510in}{6.281981in}}%
\pgfpathlineto{\pgfqpoint{10.840355in}{6.280303in}}%
\pgfpathlineto{\pgfqpoint{10.842637in}{6.293447in}}%
\pgfpathlineto{\pgfqpoint{10.844919in}{6.293727in}}%
\pgfpathlineto{\pgfqpoint{10.847201in}{6.289811in}}%
\pgfpathlineto{\pgfqpoint{10.849483in}{6.290650in}}%
\pgfpathlineto{\pgfqpoint{10.856329in}{6.291489in}}%
\pgfpathlineto{\pgfqpoint{10.858611in}{6.300438in}}%
\pgfpathlineto{\pgfqpoint{10.860892in}{6.303794in}}%
\pgfpathlineto{\pgfqpoint{10.863174in}{6.302116in}}%
\pgfpathlineto{\pgfqpoint{10.865456in}{6.296243in}}%
\pgfpathlineto{\pgfqpoint{10.872302in}{6.293167in}}%
\pgfpathlineto{\pgfqpoint{10.876866in}{6.293447in}}%
\pgfpathlineto{\pgfqpoint{10.879148in}{6.290091in}}%
\pgfpathlineto{\pgfqpoint{10.881430in}{6.283659in}}%
\pgfpathlineto{\pgfqpoint{10.888275in}{6.292608in}}%
\pgfpathlineto{\pgfqpoint{10.892839in}{6.310227in}}%
\pgfpathlineto{\pgfqpoint{10.895121in}{6.308828in}}%
\pgfpathlineto{\pgfqpoint{10.897403in}{6.305193in}}%
\pgfpathlineto{\pgfqpoint{10.904249in}{6.307150in}}%
\pgfpathlineto{\pgfqpoint{10.906531in}{6.303515in}}%
\pgfpathlineto{\pgfqpoint{10.911094in}{6.319735in}}%
\pgfpathlineto{\pgfqpoint{10.913376in}{6.324210in}}%
\pgfpathlineto{\pgfqpoint{10.920222in}{6.324210in}}%
\pgfpathlineto{\pgfqpoint{10.922504in}{6.321693in}}%
\pgfpathlineto{\pgfqpoint{10.924786in}{6.327006in}}%
\pgfpathlineto{\pgfqpoint{10.927068in}{6.339871in}}%
\pgfpathlineto{\pgfqpoint{10.929350in}{6.307150in}}%
\pgfpathlineto{\pgfqpoint{10.938477in}{6.305193in}}%
\pgfpathlineto{\pgfqpoint{10.940759in}{6.303794in}}%
\pgfpathlineto{\pgfqpoint{10.943041in}{6.300159in}}%
\pgfpathlineto{\pgfqpoint{10.945323in}{6.300438in}}%
\pgfpathlineto{\pgfqpoint{10.952169in}{6.302676in}}%
\pgfpathlineto{\pgfqpoint{10.954451in}{6.307710in}}%
\pgfpathlineto{\pgfqpoint{10.956733in}{6.309947in}}%
\pgfpathlineto{\pgfqpoint{10.959014in}{6.302396in}}%
\pgfpathlineto{\pgfqpoint{10.961296in}{6.309667in}}%
\pgfpathlineto{\pgfqpoint{10.968142in}{6.305193in}}%
\pgfpathlineto{\pgfqpoint{10.970424in}{6.310786in}}%
\pgfpathlineto{\pgfqpoint{10.974988in}{6.302116in}}%
\pgfpathlineto{\pgfqpoint{10.977270in}{6.306591in}}%
\pgfpathlineto{\pgfqpoint{10.984115in}{6.307150in}}%
\pgfpathlineto{\pgfqpoint{10.986397in}{6.311066in}}%
\pgfpathlineto{\pgfqpoint{10.988679in}{6.312464in}}%
\pgfpathlineto{\pgfqpoint{10.990961in}{6.311905in}}%
\pgfpathlineto{\pgfqpoint{10.993243in}{6.309388in}}%
\pgfpathlineto{\pgfqpoint{11.002371in}{6.308549in}}%
\pgfpathlineto{\pgfqpoint{11.004653in}{6.300438in}}%
\pgfpathlineto{\pgfqpoint{11.006935in}{6.289252in}}%
\pgfpathlineto{\pgfqpoint{11.009216in}{6.293167in}}%
\pgfpathlineto{\pgfqpoint{11.016062in}{6.288693in}}%
\pgfpathlineto{\pgfqpoint{11.020626in}{6.304913in}}%
\pgfpathlineto{\pgfqpoint{11.022908in}{6.303794in}}%
\pgfpathlineto{\pgfqpoint{11.025190in}{6.307150in}}%
\pgfpathlineto{\pgfqpoint{11.032036in}{6.312744in}}%
\pgfpathlineto{\pgfqpoint{11.036599in}{6.323930in}}%
\pgfpathlineto{\pgfqpoint{11.038881in}{6.327286in}}%
\pgfpathlineto{\pgfqpoint{11.041163in}{6.320015in}}%
\pgfpathlineto{\pgfqpoint{11.048009in}{6.322532in}}%
\pgfpathlineto{\pgfqpoint{11.050291in}{6.321413in}}%
\pgfpathlineto{\pgfqpoint{11.052573in}{6.321972in}}%
\pgfpathlineto{\pgfqpoint{11.054855in}{6.321693in}}%
\pgfpathlineto{\pgfqpoint{11.057136in}{6.315260in}}%
\pgfpathlineto{\pgfqpoint{11.063982in}{6.312744in}}%
\pgfpathlineto{\pgfqpoint{11.066264in}{6.309947in}}%
\pgfpathlineto{\pgfqpoint{11.068546in}{6.310786in}}%
\pgfpathlineto{\pgfqpoint{11.070828in}{6.307430in}}%
\pgfpathlineto{\pgfqpoint{11.073110in}{6.314142in}}%
\pgfpathlineto{\pgfqpoint{11.079956in}{6.309947in}}%
\pgfpathlineto{\pgfqpoint{11.082237in}{6.324489in}}%
\pgfpathlineto{\pgfqpoint{11.084519in}{6.325888in}}%
\pgfpathlineto{\pgfqpoint{11.086801in}{6.326167in}}%
\pgfpathlineto{\pgfqpoint{11.095929in}{6.316659in}}%
\pgfpathlineto{\pgfqpoint{11.098211in}{6.317218in}}%
\pgfpathlineto{\pgfqpoint{11.100493in}{6.307710in}}%
\pgfpathlineto{\pgfqpoint{11.102775in}{6.309947in}}%
\pgfpathlineto{\pgfqpoint{11.105057in}{6.305193in}}%
\pgfpathlineto{\pgfqpoint{11.111902in}{6.308549in}}%
\pgfpathlineto{\pgfqpoint{11.116466in}{6.335676in}}%
\pgfpathlineto{\pgfqpoint{11.118748in}{6.325888in}}%
\pgfpathlineto{\pgfqpoint{11.121030in}{6.322252in}}%
\pgfpathlineto{\pgfqpoint{11.127876in}{6.315260in}}%
\pgfpathlineto{\pgfqpoint{11.130157in}{6.327845in}}%
\pgfpathlineto{\pgfqpoint{11.132439in}{6.328125in}}%
\pgfpathlineto{\pgfqpoint{11.134721in}{6.335396in}}%
\pgfpathlineto{\pgfqpoint{11.137003in}{6.339871in}}%
\pgfpathlineto{\pgfqpoint{11.143849in}{6.350218in}}%
\pgfpathlineto{\pgfqpoint{11.146131in}{6.361684in}}%
\pgfpathlineto{\pgfqpoint{11.148413in}{6.368396in}}%
\pgfpathlineto{\pgfqpoint{11.150695in}{6.361684in}}%
\pgfpathlineto{\pgfqpoint{11.152977in}{6.362244in}}%
\pgfpathlineto{\pgfqpoint{11.159822in}{6.370634in}}%
\pgfpathlineto{\pgfqpoint{11.162104in}{6.375108in}}%
\pgfpathlineto{\pgfqpoint{11.164386in}{6.375947in}}%
\pgfpathlineto{\pgfqpoint{11.166668in}{6.378744in}}%
\pgfpathlineto{\pgfqpoint{11.168950in}{6.380422in}}%
\pgfpathlineto{\pgfqpoint{11.175796in}{6.379303in}}%
\pgfpathlineto{\pgfqpoint{11.178078in}{6.384896in}}%
\pgfpathlineto{\pgfqpoint{11.182641in}{6.389930in}}%
\pgfpathlineto{\pgfqpoint{11.184923in}{6.393007in}}%
\pgfpathlineto{\pgfqpoint{11.191769in}{6.383218in}}%
\pgfpathlineto{\pgfqpoint{11.194051in}{6.377625in}}%
\pgfpathlineto{\pgfqpoint{11.196333in}{6.386015in}}%
\pgfpathlineto{\pgfqpoint{11.198615in}{6.380142in}}%
\pgfpathlineto{\pgfqpoint{11.200897in}{6.382379in}}%
\pgfpathlineto{\pgfqpoint{11.207742in}{6.383218in}}%
\pgfpathlineto{\pgfqpoint{11.210024in}{6.384896in}}%
\pgfpathlineto{\pgfqpoint{11.212306in}{6.382939in}}%
\pgfpathlineto{\pgfqpoint{11.214588in}{6.382659in}}%
\pgfpathlineto{\pgfqpoint{11.216870in}{6.379583in}}%
\pgfpathlineto{\pgfqpoint{11.225998in}{6.383218in}}%
\pgfpathlineto{\pgfqpoint{11.228279in}{6.387693in}}%
\pgfpathlineto{\pgfqpoint{11.230561in}{6.386295in}}%
\pgfpathlineto{\pgfqpoint{11.232843in}{6.387973in}}%
\pgfpathlineto{\pgfqpoint{11.239689in}{6.410346in}}%
\pgfpathlineto{\pgfqpoint{11.241971in}{6.413142in}}%
\pgfpathlineto{\pgfqpoint{11.244253in}{6.397761in}}%
\pgfpathlineto{\pgfqpoint{11.248817in}{6.395244in}}%
\pgfpathlineto{\pgfqpoint{11.255662in}{6.405032in}}%
\pgfpathlineto{\pgfqpoint{11.260226in}{6.392168in}}%
\pgfpathlineto{\pgfqpoint{11.262508in}{6.404473in}}%
\pgfpathlineto{\pgfqpoint{11.264790in}{6.403074in}}%
\pgfpathlineto{\pgfqpoint{11.271636in}{6.407269in}}%
\pgfpathlineto{\pgfqpoint{11.273918in}{6.415939in}}%
\pgfpathlineto{\pgfqpoint{11.276200in}{6.404752in}}%
\pgfpathlineto{\pgfqpoint{11.278481in}{6.385735in}}%
\pgfpathlineto{\pgfqpoint{11.280763in}{6.386015in}}%
\pgfpathlineto{\pgfqpoint{11.287609in}{6.371473in}}%
\pgfpathlineto{\pgfqpoint{11.289891in}{6.378464in}}%
\pgfpathlineto{\pgfqpoint{11.292173in}{6.381540in}}%
\pgfpathlineto{\pgfqpoint{11.294455in}{6.382659in}}%
\pgfpathlineto{\pgfqpoint{11.296737in}{6.386854in}}%
\pgfpathlineto{\pgfqpoint{11.303582in}{6.378744in}}%
\pgfpathlineto{\pgfqpoint{11.305864in}{6.374829in}}%
\pgfpathlineto{\pgfqpoint{11.308146in}{6.375668in}}%
\pgfpathlineto{\pgfqpoint{11.312710in}{6.382939in}}%
\pgfpathlineto{\pgfqpoint{11.319556in}{6.362523in}}%
\pgfpathlineto{\pgfqpoint{11.321838in}{6.361405in}}%
\pgfpathlineto{\pgfqpoint{11.324120in}{6.363922in}}%
\pgfpathlineto{\pgfqpoint{11.326401in}{6.381261in}}%
\pgfpathlineto{\pgfqpoint{11.328683in}{6.379023in}}%
\pgfpathlineto{\pgfqpoint{11.335529in}{6.380142in}}%
\pgfpathlineto{\pgfqpoint{11.337811in}{6.387134in}}%
\pgfpathlineto{\pgfqpoint{11.340093in}{6.384896in}}%
\pgfpathlineto{\pgfqpoint{11.342375in}{6.363642in}}%
\pgfpathlineto{\pgfqpoint{11.344657in}{6.358888in}}%
\pgfpathlineto{\pgfqpoint{11.351502in}{6.352735in}}%
\pgfpathlineto{\pgfqpoint{11.353784in}{6.352176in}}%
\pgfpathlineto{\pgfqpoint{11.360630in}{6.364201in}}%
\pgfpathlineto{\pgfqpoint{11.367476in}{6.363642in}}%
\pgfpathlineto{\pgfqpoint{11.369758in}{6.366998in}}%
\pgfpathlineto{\pgfqpoint{11.372040in}{6.365879in}}%
\pgfpathlineto{\pgfqpoint{11.374322in}{6.362803in}}%
\pgfpathlineto{\pgfqpoint{11.376603in}{6.363083in}}%
\pgfpathlineto{\pgfqpoint{11.383449in}{6.361405in}}%
\pgfpathlineto{\pgfqpoint{11.388013in}{6.356091in}}%
\pgfpathlineto{\pgfqpoint{11.390295in}{6.350218in}}%
\pgfpathlineto{\pgfqpoint{11.392577in}{6.351617in}}%
\pgfpathlineto{\pgfqpoint{11.399423in}{6.354693in}}%
\pgfpathlineto{\pgfqpoint{11.401704in}{6.351896in}}%
\pgfpathlineto{\pgfqpoint{11.403986in}{6.353015in}}%
\pgfpathlineto{\pgfqpoint{11.406268in}{6.340430in}}%
\pgfpathlineto{\pgfqpoint{11.408550in}{6.337354in}}%
\pgfpathlineto{\pgfqpoint{11.415396in}{6.346303in}}%
\pgfpathlineto{\pgfqpoint{11.417678in}{6.351896in}}%
\pgfpathlineto{\pgfqpoint{11.419960in}{6.343506in}}%
\pgfpathlineto{\pgfqpoint{11.422242in}{6.345744in}}%
\pgfpathlineto{\pgfqpoint{11.424523in}{6.349939in}}%
\pgfpathlineto{\pgfqpoint{11.431369in}{6.346023in}}%
\pgfpathlineto{\pgfqpoint{11.433651in}{6.344066in}}%
\pgfpathlineto{\pgfqpoint{11.435933in}{6.346583in}}%
\pgfpathlineto{\pgfqpoint{11.438215in}{6.340989in}}%
\pgfpathlineto{\pgfqpoint{11.440497in}{6.339032in}}%
\pgfpathlineto{\pgfqpoint{11.449624in}{6.341269in}}%
\pgfpathlineto{\pgfqpoint{11.451906in}{6.354693in}}%
\pgfpathlineto{\pgfqpoint{11.454188in}{6.353854in}}%
\pgfpathlineto{\pgfqpoint{11.465598in}{6.362803in}}%
\pgfpathlineto{\pgfqpoint{11.470162in}{6.354413in}}%
\pgfpathlineto{\pgfqpoint{11.472444in}{6.373710in}}%
\pgfpathlineto{\pgfqpoint{11.479289in}{6.372591in}}%
\pgfpathlineto{\pgfqpoint{11.481571in}{6.380701in}}%
\pgfpathlineto{\pgfqpoint{11.483853in}{6.384617in}}%
\pgfpathlineto{\pgfqpoint{11.486135in}{6.384896in}}%
\pgfpathlineto{\pgfqpoint{11.488417in}{6.381540in}}%
\pgfpathlineto{\pgfqpoint{11.495263in}{6.377905in}}%
\pgfpathlineto{\pgfqpoint{11.497545in}{6.379862in}}%
\pgfpathlineto{\pgfqpoint{11.499826in}{6.379862in}}%
\pgfpathlineto{\pgfqpoint{11.502108in}{6.372871in}}%
\pgfpathlineto{\pgfqpoint{11.504390in}{6.362803in}}%
\pgfpathlineto{\pgfqpoint{11.511236in}{6.361405in}}%
\pgfpathlineto{\pgfqpoint{11.513518in}{6.359167in}}%
\pgfpathlineto{\pgfqpoint{11.515800in}{6.360566in}}%
\pgfpathlineto{\pgfqpoint{11.518082in}{6.353854in}}%
\pgfpathlineto{\pgfqpoint{11.520364in}{6.358888in}}%
\pgfpathlineto{\pgfqpoint{11.527209in}{6.359167in}}%
\pgfpathlineto{\pgfqpoint{11.529491in}{6.351057in}}%
\pgfpathlineto{\pgfqpoint{11.531773in}{6.353574in}}%
\pgfpathlineto{\pgfqpoint{11.534055in}{6.365600in}}%
\pgfpathlineto{\pgfqpoint{11.536337in}{6.369515in}}%
\pgfpathlineto{\pgfqpoint{11.543183in}{6.373990in}}%
\pgfpathlineto{\pgfqpoint{11.545465in}{6.372591in}}%
\pgfpathlineto{\pgfqpoint{11.547746in}{6.379862in}}%
\pgfpathlineto{\pgfqpoint{11.550028in}{6.384896in}}%
\pgfpathlineto{\pgfqpoint{11.552310in}{6.384057in}}%
\pgfpathlineto{\pgfqpoint{11.559156in}{6.390210in}}%
\pgfpathlineto{\pgfqpoint{11.561438in}{6.388532in}}%
\pgfpathlineto{\pgfqpoint{11.563720in}{6.380701in}}%
\pgfpathlineto{\pgfqpoint{11.566002in}{6.381820in}}%
\pgfpathlineto{\pgfqpoint{11.568284in}{6.392727in}}%
\pgfpathlineto{\pgfqpoint{11.575129in}{6.395524in}}%
\pgfpathlineto{\pgfqpoint{11.577411in}{6.399159in}}%
\pgfpathlineto{\pgfqpoint{11.581975in}{6.398040in}}%
\pgfpathlineto{\pgfqpoint{11.584257in}{6.394685in}}%
\pgfpathlineto{\pgfqpoint{11.593385in}{6.393007in}}%
\pgfpathlineto{\pgfqpoint{11.595666in}{6.398320in}}%
\pgfpathlineto{\pgfqpoint{11.597948in}{6.393566in}}%
\pgfpathlineto{\pgfqpoint{11.607076in}{6.396363in}}%
\pgfpathlineto{\pgfqpoint{11.609358in}{6.402515in}}%
\pgfpathlineto{\pgfqpoint{11.611640in}{6.406430in}}%
\pgfpathlineto{\pgfqpoint{11.613922in}{6.401396in}}%
\pgfpathlineto{\pgfqpoint{11.616204in}{6.404752in}}%
\pgfpathlineto{\pgfqpoint{11.623049in}{6.406430in}}%
\pgfpathlineto{\pgfqpoint{11.625331in}{6.408947in}}%
\pgfpathlineto{\pgfqpoint{11.627613in}{6.405591in}}%
\pgfpathlineto{\pgfqpoint{11.629895in}{6.421532in}}%
\pgfpathlineto{\pgfqpoint{11.632177in}{6.389091in}}%
\pgfpathlineto{\pgfqpoint{11.639023in}{6.386295in}}%
\pgfpathlineto{\pgfqpoint{11.641305in}{6.383778in}}%
\pgfpathlineto{\pgfqpoint{11.643587in}{6.389930in}}%
\pgfpathlineto{\pgfqpoint{11.648150in}{6.388532in}}%
\pgfpathlineto{\pgfqpoint{11.654996in}{6.385176in}}%
\pgfpathlineto{\pgfqpoint{11.657278in}{6.381540in}}%
\pgfpathlineto{\pgfqpoint{11.659560in}{6.386015in}}%
\pgfpathlineto{\pgfqpoint{11.664124in}{6.411744in}}%
\pgfpathlineto{\pgfqpoint{11.670969in}{6.414541in}}%
\pgfpathlineto{\pgfqpoint{11.673251in}{6.411744in}}%
\pgfpathlineto{\pgfqpoint{11.675533in}{6.402235in}}%
\pgfpathlineto{\pgfqpoint{11.677815in}{6.403354in}}%
\pgfpathlineto{\pgfqpoint{11.680097in}{6.399159in}}%
\pgfpathlineto{\pgfqpoint{11.686943in}{6.403074in}}%
\pgfpathlineto{\pgfqpoint{11.689225in}{6.408108in}}%
\pgfpathlineto{\pgfqpoint{11.691507in}{6.419574in}}%
\pgfpathlineto{\pgfqpoint{11.693788in}{6.419295in}}%
\pgfpathlineto{\pgfqpoint{11.696070in}{6.417617in}}%
\pgfpathlineto{\pgfqpoint{11.705198in}{6.426286in}}%
\pgfpathlineto{\pgfqpoint{11.709762in}{6.432719in}}%
\pgfpathlineto{\pgfqpoint{11.712044in}{6.430481in}}%
\pgfpathlineto{\pgfqpoint{11.721171in}{6.438871in}}%
\pgfpathlineto{\pgfqpoint{11.725735in}{6.434956in}}%
\pgfpathlineto{\pgfqpoint{11.728017in}{6.434676in}}%
\pgfpathlineto{\pgfqpoint{11.734863in}{6.427125in}}%
\pgfpathlineto{\pgfqpoint{11.737145in}{6.430202in}}%
\pgfpathlineto{\pgfqpoint{11.739427in}{6.426286in}}%
\pgfpathlineto{\pgfqpoint{11.741709in}{6.423490in}}%
\pgfpathlineto{\pgfqpoint{11.743990in}{6.428803in}}%
\pgfpathlineto{\pgfqpoint{11.750836in}{6.427964in}}%
\pgfpathlineto{\pgfqpoint{11.753118in}{6.452015in}}%
\pgfpathlineto{\pgfqpoint{11.755400in}{6.455931in}}%
\pgfpathlineto{\pgfqpoint{11.757682in}{6.452854in}}%
\pgfpathlineto{\pgfqpoint{11.759964in}{6.436354in}}%
\pgfpathlineto{\pgfqpoint{11.769091in}{6.430202in}}%
\pgfpathlineto{\pgfqpoint{11.771373in}{6.423490in}}%
\pgfpathlineto{\pgfqpoint{11.773655in}{6.419295in}}%
\pgfpathlineto{\pgfqpoint{11.775937in}{6.411464in}}%
\pgfpathlineto{\pgfqpoint{11.782783in}{6.409507in}}%
\pgfpathlineto{\pgfqpoint{11.785065in}{6.413702in}}%
\pgfpathlineto{\pgfqpoint{11.787347in}{6.408388in}}%
\pgfpathlineto{\pgfqpoint{11.789629in}{6.409786in}}%
\pgfpathlineto{\pgfqpoint{11.798756in}{6.391049in}}%
\pgfpathlineto{\pgfqpoint{11.801038in}{6.387973in}}%
\pgfpathlineto{\pgfqpoint{11.803320in}{6.386015in}}%
\pgfpathlineto{\pgfqpoint{11.805602in}{6.397481in}}%
\pgfpathlineto{\pgfqpoint{11.807884in}{6.402795in}}%
\pgfpathlineto{\pgfqpoint{11.814730in}{6.404752in}}%
\pgfpathlineto{\pgfqpoint{11.817011in}{6.408947in}}%
\pgfpathlineto{\pgfqpoint{11.819293in}{6.410905in}}%
\pgfpathlineto{\pgfqpoint{11.821575in}{6.414541in}}%
\pgfpathlineto{\pgfqpoint{11.823857in}{6.415939in}}%
\pgfpathlineto{\pgfqpoint{11.832985in}{6.415939in}}%
\pgfpathlineto{\pgfqpoint{11.835267in}{6.409786in}}%
\pgfpathlineto{\pgfqpoint{11.837549in}{6.415380in}}%
\pgfpathlineto{\pgfqpoint{11.839831in}{6.407829in}}%
\pgfpathlineto{\pgfqpoint{11.846676in}{6.412863in}}%
\pgfpathlineto{\pgfqpoint{11.848958in}{6.412583in}}%
\pgfpathlineto{\pgfqpoint{11.851240in}{6.417058in}}%
\pgfpathlineto{\pgfqpoint{11.853522in}{6.415939in}}%
\pgfpathlineto{\pgfqpoint{11.855804in}{6.415939in}}%
\pgfpathlineto{\pgfqpoint{11.862650in}{6.409786in}}%
\pgfpathlineto{\pgfqpoint{11.864932in}{6.412303in}}%
\pgfpathlineto{\pgfqpoint{11.867213in}{6.409786in}}%
\pgfpathlineto{\pgfqpoint{11.869495in}{6.412863in}}%
\pgfpathlineto{\pgfqpoint{11.871777in}{6.413142in}}%
\pgfpathlineto{\pgfqpoint{11.878623in}{6.417897in}}%
\pgfpathlineto{\pgfqpoint{11.880905in}{6.415380in}}%
\pgfpathlineto{\pgfqpoint{11.883187in}{6.415939in}}%
\pgfpathlineto{\pgfqpoint{11.885469in}{6.411464in}}%
\pgfpathlineto{\pgfqpoint{11.887751in}{6.409786in}}%
\pgfpathlineto{\pgfqpoint{11.894596in}{6.414541in}}%
\pgfpathlineto{\pgfqpoint{11.896878in}{6.417337in}}%
\pgfpathlineto{\pgfqpoint{11.899160in}{6.422371in}}%
\pgfpathlineto{\pgfqpoint{11.901442in}{6.432159in}}%
\pgfpathlineto{\pgfqpoint{11.903724in}{6.426007in}}%
\pgfpathlineto{\pgfqpoint{11.910570in}{6.424608in}}%
\pgfpathlineto{\pgfqpoint{11.912852in}{6.432998in}}%
\pgfpathlineto{\pgfqpoint{11.917415in}{6.429363in}}%
\pgfpathlineto{\pgfqpoint{11.919697in}{6.436634in}}%
\pgfpathlineto{\pgfqpoint{11.926543in}{6.441388in}}%
\pgfpathlineto{\pgfqpoint{11.928825in}{6.445583in}}%
\pgfpathlineto{\pgfqpoint{11.931107in}{6.443346in}}%
\pgfpathlineto{\pgfqpoint{11.933389in}{6.455651in}}%
\pgfpathlineto{\pgfqpoint{11.935671in}{6.449778in}}%
\pgfpathlineto{\pgfqpoint{11.942516in}{6.457609in}}%
\pgfpathlineto{\pgfqpoint{11.944798in}{6.467676in}}%
\pgfpathlineto{\pgfqpoint{11.947080in}{6.469354in}}%
\pgfpathlineto{\pgfqpoint{11.949362in}{6.456210in}}%
\pgfpathlineto{\pgfqpoint{11.951644in}{6.450337in}}%
\pgfpathlineto{\pgfqpoint{11.958490in}{6.459287in}}%
\pgfpathlineto{\pgfqpoint{11.963054in}{6.468236in}}%
\pgfpathlineto{\pgfqpoint{11.965335in}{6.471032in}}%
\pgfpathlineto{\pgfqpoint{11.974463in}{6.468795in}}%
\pgfpathlineto{\pgfqpoint{11.979027in}{6.464041in}}%
\pgfpathlineto{\pgfqpoint{11.981309in}{6.464041in}}%
\pgfpathlineto{\pgfqpoint{11.983591in}{6.452015in}}%
\pgfpathlineto{\pgfqpoint{11.990436in}{6.453693in}}%
\pgfpathlineto{\pgfqpoint{11.992718in}{6.457329in}}%
\pgfpathlineto{\pgfqpoint{11.995000in}{6.462363in}}%
\pgfpathlineto{\pgfqpoint{11.997282in}{6.456770in}}%
\pgfpathlineto{\pgfqpoint{11.999564in}{6.455651in}}%
\pgfpathlineto{\pgfqpoint{12.006410in}{6.455371in}}%
\pgfpathlineto{\pgfqpoint{12.008692in}{6.456210in}}%
\pgfpathlineto{\pgfqpoint{12.010974in}{6.460125in}}%
\pgfpathlineto{\pgfqpoint{12.015537in}{6.458448in}}%
\pgfpathlineto{\pgfqpoint{12.022383in}{6.460125in}}%
\pgfpathlineto{\pgfqpoint{12.024665in}{6.462083in}}%
\pgfpathlineto{\pgfqpoint{12.026947in}{6.459287in}}%
\pgfpathlineto{\pgfqpoint{12.029229in}{6.451456in}}%
\pgfpathlineto{\pgfqpoint{12.031511in}{6.446981in}}%
\pgfpathlineto{\pgfqpoint{12.038356in}{6.452295in}}%
\pgfpathlineto{\pgfqpoint{12.040638in}{6.452295in}}%
\pgfpathlineto{\pgfqpoint{12.042920in}{6.456210in}}%
\pgfpathlineto{\pgfqpoint{12.045202in}{6.454812in}}%
\pgfpathlineto{\pgfqpoint{12.047484in}{6.458168in}}%
\pgfpathlineto{\pgfqpoint{12.056612in}{6.468515in}}%
\pgfpathlineto{\pgfqpoint{12.058894in}{6.472710in}}%
\pgfpathlineto{\pgfqpoint{12.061175in}{6.474668in}}%
\pgfpathlineto{\pgfqpoint{12.063457in}{6.483337in}}%
\pgfpathlineto{\pgfqpoint{12.070303in}{6.481939in}}%
\pgfpathlineto{\pgfqpoint{12.072585in}{6.491727in}}%
\pgfpathlineto{\pgfqpoint{12.074867in}{6.490049in}}%
\pgfpathlineto{\pgfqpoint{12.077149in}{6.491727in}}%
\pgfpathlineto{\pgfqpoint{12.079431in}{6.504032in}}%
\pgfpathlineto{\pgfqpoint{12.086276in}{6.497600in}}%
\pgfpathlineto{\pgfqpoint{12.088558in}{6.505710in}}%
\pgfpathlineto{\pgfqpoint{12.090840in}{6.498160in}}%
\pgfpathlineto{\pgfqpoint{12.093122in}{6.498999in}}%
\pgfpathlineto{\pgfqpoint{12.095404in}{6.545422in}}%
\pgfpathlineto{\pgfqpoint{12.102250in}{6.548778in}}%
\pgfpathlineto{\pgfqpoint{12.104532in}{6.547380in}}%
\pgfpathlineto{\pgfqpoint{12.106814in}{6.546821in}}%
\pgfpathlineto{\pgfqpoint{12.109096in}{6.550736in}}%
\pgfpathlineto{\pgfqpoint{12.111377in}{6.553253in}}%
\pgfpathlineto{\pgfqpoint{12.118223in}{6.554092in}}%
\pgfpathlineto{\pgfqpoint{12.120505in}{6.560524in}}%
\pgfpathlineto{\pgfqpoint{12.122787in}{6.569753in}}%
\pgfpathlineto{\pgfqpoint{12.125069in}{6.567516in}}%
\pgfpathlineto{\pgfqpoint{12.127351in}{6.571151in}}%
\pgfpathlineto{\pgfqpoint{12.134197in}{6.570312in}}%
\pgfpathlineto{\pgfqpoint{12.136478in}{6.572270in}}%
\pgfpathlineto{\pgfqpoint{12.138760in}{6.572270in}}%
\pgfpathlineto{\pgfqpoint{12.141042in}{6.576185in}}%
\pgfpathlineto{\pgfqpoint{12.150170in}{6.573389in}}%
\pgfpathlineto{\pgfqpoint{12.152452in}{6.567795in}}%
\pgfpathlineto{\pgfqpoint{12.154734in}{6.570033in}}%
\pgfpathlineto{\pgfqpoint{12.157016in}{6.578982in}}%
\pgfpathlineto{\pgfqpoint{12.159297in}{6.578702in}}%
\pgfpathlineto{\pgfqpoint{12.166143in}{6.584575in}}%
\pgfpathlineto{\pgfqpoint{12.168425in}{6.589889in}}%
\pgfpathlineto{\pgfqpoint{12.170707in}{6.661482in}}%
\pgfpathlineto{\pgfqpoint{12.172989in}{6.638270in}}%
\pgfpathlineto{\pgfqpoint{12.175271in}{6.638270in}}%
\pgfpathlineto{\pgfqpoint{12.182117in}{6.647219in}}%
\pgfpathlineto{\pgfqpoint{12.184398in}{6.664838in}}%
\pgfpathlineto{\pgfqpoint{12.188962in}{6.651694in}}%
\pgfpathlineto{\pgfqpoint{12.191244in}{6.651694in}}%
\pgfpathlineto{\pgfqpoint{12.198090in}{6.651135in}}%
\pgfpathlineto{\pgfqpoint{12.200372in}{6.650296in}}%
\pgfpathlineto{\pgfqpoint{12.202654in}{6.654211in}}%
\pgfpathlineto{\pgfqpoint{12.204936in}{6.643025in}}%
\pgfpathlineto{\pgfqpoint{12.207218in}{6.639389in}}%
\pgfpathlineto{\pgfqpoint{12.214063in}{6.646940in}}%
\pgfpathlineto{\pgfqpoint{12.216345in}{6.622330in}}%
\pgfpathlineto{\pgfqpoint{12.218627in}{6.623169in}}%
\pgfpathlineto{\pgfqpoint{12.220909in}{6.618974in}}%
\pgfpathlineto{\pgfqpoint{12.223191in}{6.617016in}}%
\pgfpathlineto{\pgfqpoint{12.232319in}{6.629880in}}%
\pgfpathlineto{\pgfqpoint{12.234600in}{6.653652in}}%
\pgfpathlineto{\pgfqpoint{12.236882in}{6.649736in}}%
\pgfpathlineto{\pgfqpoint{12.239164in}{6.655330in}}%
\pgfpathlineto{\pgfqpoint{12.246010in}{6.661203in}}%
\pgfpathlineto{\pgfqpoint{12.248292in}{6.659525in}}%
\pgfpathlineto{\pgfqpoint{12.250574in}{6.663440in}}%
\pgfpathlineto{\pgfqpoint{12.252856in}{6.679381in}}%
\pgfpathlineto{\pgfqpoint{12.255138in}{6.674067in}}%
\pgfpathlineto{\pgfqpoint{12.261983in}{6.670991in}}%
\pgfpathlineto{\pgfqpoint{12.266547in}{6.670431in}}%
\pgfpathlineto{\pgfqpoint{12.268829in}{6.667075in}}%
\pgfpathlineto{\pgfqpoint{12.271111in}{6.673787in}}%
\pgfpathlineto{\pgfqpoint{12.280239in}{6.665118in}}%
\pgfpathlineto{\pgfqpoint{12.282520in}{6.665118in}}%
\pgfpathlineto{\pgfqpoint{12.284802in}{6.673508in}}%
\pgfpathlineto{\pgfqpoint{12.287084in}{6.675745in}}%
\pgfpathlineto{\pgfqpoint{12.293930in}{6.683576in}}%
\pgfpathlineto{\pgfqpoint{12.296212in}{6.673508in}}%
\pgfpathlineto{\pgfqpoint{12.298494in}{6.676025in}}%
\pgfpathlineto{\pgfqpoint{12.300776in}{6.676025in}}%
\pgfpathlineto{\pgfqpoint{12.303058in}{6.666236in}}%
\pgfpathlineto{\pgfqpoint{12.309903in}{6.664279in}}%
\pgfpathlineto{\pgfqpoint{12.312185in}{6.673787in}}%
\pgfpathlineto{\pgfqpoint{12.314467in}{6.675186in}}%
\pgfpathlineto{\pgfqpoint{12.316749in}{6.679660in}}%
\pgfpathlineto{\pgfqpoint{12.319031in}{6.671270in}}%
\pgfpathlineto{\pgfqpoint{12.325877in}{6.668474in}}%
\pgfpathlineto{\pgfqpoint{12.328159in}{6.661482in}}%
\pgfpathlineto{\pgfqpoint{12.330441in}{6.669592in}}%
\pgfpathlineto{\pgfqpoint{12.332722in}{6.654491in}}%
\pgfpathlineto{\pgfqpoint{12.335004in}{6.657567in}}%
\pgfpathlineto{\pgfqpoint{12.341850in}{6.673228in}}%
\pgfpathlineto{\pgfqpoint{12.344132in}{6.671270in}}%
\pgfpathlineto{\pgfqpoint{12.346414in}{6.650855in}}%
\pgfpathlineto{\pgfqpoint{12.348696in}{6.639389in}}%
\pgfpathlineto{\pgfqpoint{12.350978in}{6.651974in}}%
\pgfpathlineto{\pgfqpoint{12.357823in}{6.653931in}}%
\pgfpathlineto{\pgfqpoint{12.360105in}{6.638270in}}%
\pgfpathlineto{\pgfqpoint{12.362387in}{6.657847in}}%
\pgfpathlineto{\pgfqpoint{12.364669in}{6.641906in}}%
\pgfpathlineto{\pgfqpoint{12.366951in}{6.599957in}}%
\pgfpathlineto{\pgfqpoint{12.373797in}{6.589329in}}%
\pgfpathlineto{\pgfqpoint{12.376079in}{6.605550in}}%
\pgfpathlineto{\pgfqpoint{12.378361in}{6.584575in}}%
\pgfpathlineto{\pgfqpoint{12.380642in}{6.574228in}}%
\pgfpathlineto{\pgfqpoint{12.382924in}{6.587092in}}%
\pgfpathlineto{\pgfqpoint{12.389770in}{6.591846in}}%
\pgfpathlineto{\pgfqpoint{12.392052in}{6.617296in}}%
\pgfpathlineto{\pgfqpoint{12.394334in}{6.608906in}}%
\pgfpathlineto{\pgfqpoint{12.398898in}{6.630999in}}%
\pgfpathlineto{\pgfqpoint{12.405743in}{6.631558in}}%
\pgfpathlineto{\pgfqpoint{12.408025in}{6.644703in}}%
\pgfpathlineto{\pgfqpoint{12.410307in}{6.649177in}}%
\pgfpathlineto{\pgfqpoint{12.412589in}{6.616457in}}%
\pgfpathlineto{\pgfqpoint{12.414871in}{6.651414in}}%
\pgfpathlineto{\pgfqpoint{12.421717in}{6.658686in}}%
\pgfpathlineto{\pgfqpoint{12.423999in}{6.664279in}}%
\pgfpathlineto{\pgfqpoint{12.426281in}{6.650575in}}%
\pgfpathlineto{\pgfqpoint{12.428563in}{6.652253in}}%
\pgfpathlineto{\pgfqpoint{12.430844in}{6.646380in}}%
\pgfpathlineto{\pgfqpoint{12.437690in}{6.638270in}}%
\pgfpathlineto{\pgfqpoint{12.442254in}{6.641347in}}%
\pgfpathlineto{\pgfqpoint{12.446818in}{6.655330in}}%
\pgfpathlineto{\pgfqpoint{12.453663in}{6.662601in}}%
\pgfpathlineto{\pgfqpoint{12.455945in}{6.674067in}}%
\pgfpathlineto{\pgfqpoint{12.458227in}{6.665118in}}%
\pgfpathlineto{\pgfqpoint{12.460509in}{6.704550in}}%
\pgfpathlineto{\pgfqpoint{12.462791in}{6.695881in}}%
\pgfpathlineto{\pgfqpoint{12.469637in}{6.712101in}}%
\pgfpathlineto{\pgfqpoint{12.471919in}{6.713779in}}%
\pgfpathlineto{\pgfqpoint{12.474201in}{6.728042in}}%
\pgfpathlineto{\pgfqpoint{12.478764in}{6.736711in}}%
\pgfpathlineto{\pgfqpoint{12.485610in}{6.734754in}}%
\pgfpathlineto{\pgfqpoint{12.487892in}{6.745381in}}%
\pgfpathlineto{\pgfqpoint{12.490174in}{6.741186in}}%
\pgfpathlineto{\pgfqpoint{12.492456in}{6.741745in}}%
\pgfpathlineto{\pgfqpoint{12.494738in}{6.747059in}}%
\pgfpathlineto{\pgfqpoint{12.501584in}{6.735313in}}%
\pgfpathlineto{\pgfqpoint{12.503865in}{6.727762in}}%
\pgfpathlineto{\pgfqpoint{12.506147in}{6.716296in}}%
\pgfpathlineto{\pgfqpoint{12.508429in}{6.723008in}}%
\pgfpathlineto{\pgfqpoint{12.510711in}{6.711542in}}%
\pgfpathlineto{\pgfqpoint{12.517557in}{6.703991in}}%
\pgfpathlineto{\pgfqpoint{12.519839in}{6.695042in}}%
\pgfpathlineto{\pgfqpoint{12.524403in}{6.731118in}}%
\pgfpathlineto{\pgfqpoint{12.526684in}{6.714898in}}%
\pgfpathlineto{\pgfqpoint{12.535812in}{6.741186in}}%
\pgfpathlineto{\pgfqpoint{12.538094in}{6.741466in}}%
\pgfpathlineto{\pgfqpoint{12.542658in}{6.743983in}}%
\pgfpathlineto{\pgfqpoint{12.549504in}{6.735033in}}%
\pgfpathlineto{\pgfqpoint{12.554067in}{6.712940in}}%
\pgfpathlineto{\pgfqpoint{12.558631in}{6.714618in}}%
\pgfpathlineto{\pgfqpoint{12.565477in}{6.704550in}}%
\pgfpathlineto{\pgfqpoint{12.567759in}{6.688050in}}%
\pgfpathlineto{\pgfqpoint{12.572323in}{6.722728in}}%
\pgfpathlineto{\pgfqpoint{12.574605in}{6.724686in}}%
\pgfpathlineto{\pgfqpoint{12.581450in}{6.720771in}}%
\pgfpathlineto{\pgfqpoint{12.583732in}{6.718254in}}%
\pgfpathlineto{\pgfqpoint{12.588296in}{6.710423in}}%
\pgfpathlineto{\pgfqpoint{12.590578in}{6.716855in}}%
\pgfpathlineto{\pgfqpoint{12.599706in}{6.708186in}}%
\pgfpathlineto{\pgfqpoint{12.604269in}{6.728321in}}%
\pgfpathlineto{\pgfqpoint{12.606551in}{6.716855in}}%
\pgfpathlineto{\pgfqpoint{12.613397in}{6.701194in}}%
\pgfpathlineto{\pgfqpoint{12.615679in}{6.661203in}}%
\pgfpathlineto{\pgfqpoint{12.617961in}{6.651135in}}%
\pgfpathlineto{\pgfqpoint{12.620243in}{6.661762in}}%
\pgfpathlineto{\pgfqpoint{12.622525in}{6.632957in}}%
\pgfpathlineto{\pgfqpoint{12.629370in}{6.648058in}}%
\pgfpathlineto{\pgfqpoint{12.631652in}{6.649177in}}%
\pgfpathlineto{\pgfqpoint{12.633934in}{6.652813in}}%
\pgfpathlineto{\pgfqpoint{12.636216in}{6.661203in}}%
\pgfpathlineto{\pgfqpoint{12.638498in}{6.645262in}}%
\pgfpathlineto{\pgfqpoint{12.645344in}{6.636033in}}%
\pgfpathlineto{\pgfqpoint{12.647626in}{6.655050in}}%
\pgfpathlineto{\pgfqpoint{12.649907in}{6.651414in}}%
\pgfpathlineto{\pgfqpoint{12.652189in}{6.665957in}}%
\pgfpathlineto{\pgfqpoint{12.654471in}{6.671550in}}%
\pgfpathlineto{\pgfqpoint{12.663599in}{6.681059in}}%
\pgfpathlineto{\pgfqpoint{12.665881in}{6.669313in}}%
\pgfpathlineto{\pgfqpoint{12.668163in}{6.667914in}}%
\pgfpathlineto{\pgfqpoint{12.670445in}{6.672669in}}%
\pgfpathlineto{\pgfqpoint{12.677290in}{6.656728in}}%
\pgfpathlineto{\pgfqpoint{12.679572in}{6.672669in}}%
\pgfpathlineto{\pgfqpoint{12.681854in}{6.661482in}}%
\pgfpathlineto{\pgfqpoint{12.684136in}{6.653931in}}%
\pgfpathlineto{\pgfqpoint{12.686418in}{6.644143in}}%
\pgfpathlineto{\pgfqpoint{12.693264in}{6.663999in}}%
\pgfpathlineto{\pgfqpoint{12.697828in}{6.665677in}}%
\pgfpathlineto{\pgfqpoint{12.700109in}{6.655889in}}%
\pgfpathlineto{\pgfqpoint{12.702391in}{6.642465in}}%
\pgfpathlineto{\pgfqpoint{12.709237in}{6.630999in}}%
\pgfpathlineto{\pgfqpoint{12.711519in}{6.605550in}}%
\pgfpathlineto{\pgfqpoint{12.713801in}{6.621211in}}%
\pgfpathlineto{\pgfqpoint{12.716083in}{6.583177in}}%
\pgfpathlineto{\pgfqpoint{12.718365in}{6.586533in}}%
\pgfpathlineto{\pgfqpoint{12.725210in}{6.584016in}}%
\pgfpathlineto{\pgfqpoint{12.727492in}{6.578143in}}%
\pgfpathlineto{\pgfqpoint{12.729774in}{6.585414in}}%
\pgfpathlineto{\pgfqpoint{12.732056in}{6.581779in}}%
\pgfpathlineto{\pgfqpoint{12.734338in}{6.596041in}}%
\pgfpathlineto{\pgfqpoint{12.741184in}{6.593245in}}%
\pgfpathlineto{\pgfqpoint{12.743466in}{6.582897in}}%
\pgfpathlineto{\pgfqpoint{12.745748in}{6.560804in}}%
\pgfpathlineto{\pgfqpoint{12.748029in}{6.565278in}}%
\pgfpathlineto{\pgfqpoint{12.750311in}{6.613101in}}%
\pgfpathlineto{\pgfqpoint{12.759439in}{6.594923in}}%
\pgfpathlineto{\pgfqpoint{12.761721in}{6.583457in}}%
\pgfpathlineto{\pgfqpoint{12.764003in}{6.583457in}}%
\pgfpathlineto{\pgfqpoint{12.773130in}{6.589329in}}%
\pgfpathlineto{\pgfqpoint{12.775412in}{6.594643in}}%
\pgfpathlineto{\pgfqpoint{12.777694in}{6.596041in}}%
\pgfpathlineto{\pgfqpoint{12.779976in}{6.594084in}}%
\pgfpathlineto{\pgfqpoint{12.782258in}{6.611423in}}%
\pgfpathlineto{\pgfqpoint{12.789104in}{6.606389in}}%
\pgfpathlineto{\pgfqpoint{12.791386in}{6.600516in}}%
\pgfpathlineto{\pgfqpoint{12.793668in}{6.633516in}}%
\pgfpathlineto{\pgfqpoint{12.795950in}{6.634635in}}%
\pgfpathlineto{\pgfqpoint{12.798231in}{6.624567in}}%
\pgfpathlineto{\pgfqpoint{12.805077in}{6.630999in}}%
\pgfpathlineto{\pgfqpoint{12.807359in}{6.623728in}}%
\pgfpathlineto{\pgfqpoint{12.809641in}{6.630440in}}%
\pgfpathlineto{\pgfqpoint{12.814205in}{6.615058in}}%
\pgfpathlineto{\pgfqpoint{12.821050in}{6.625406in}}%
\pgfpathlineto{\pgfqpoint{12.823332in}{6.638270in}}%
\pgfpathlineto{\pgfqpoint{12.825614in}{6.635194in}}%
\pgfpathlineto{\pgfqpoint{12.827896in}{6.626804in}}%
\pgfpathlineto{\pgfqpoint{12.830178in}{6.648338in}}%
\pgfpathlineto{\pgfqpoint{12.837024in}{6.648618in}}%
\pgfpathlineto{\pgfqpoint{12.839306in}{6.634635in}}%
\pgfpathlineto{\pgfqpoint{12.841588in}{6.624287in}}%
\pgfpathlineto{\pgfqpoint{12.843870in}{6.624847in}}%
\pgfpathlineto{\pgfqpoint{12.846151in}{6.638830in}}%
\pgfpathlineto{\pgfqpoint{12.852997in}{6.636033in}}%
\pgfpathlineto{\pgfqpoint{12.855279in}{6.625126in}}%
\pgfpathlineto{\pgfqpoint{12.859843in}{6.643025in}}%
\pgfpathlineto{\pgfqpoint{12.862125in}{6.643584in}}%
\pgfpathlineto{\pgfqpoint{12.868971in}{6.653931in}}%
\pgfpathlineto{\pgfqpoint{12.871252in}{6.647499in}}%
\pgfpathlineto{\pgfqpoint{12.875816in}{6.657567in}}%
\pgfpathlineto{\pgfqpoint{12.878098in}{6.655050in}}%
\pgfpathlineto{\pgfqpoint{12.887226in}{6.646380in}}%
\pgfpathlineto{\pgfqpoint{12.889508in}{6.661482in}}%
\pgfpathlineto{\pgfqpoint{12.891790in}{6.669033in}}%
\pgfpathlineto{\pgfqpoint{12.894072in}{6.680220in}}%
\pgfpathlineto{\pgfqpoint{12.900917in}{6.666516in}}%
\pgfpathlineto{\pgfqpoint{12.905481in}{6.636872in}}%
\pgfpathlineto{\pgfqpoint{12.910045in}{6.614779in}}%
\pgfpathlineto{\pgfqpoint{12.916891in}{6.601355in}}%
\pgfpathlineto{\pgfqpoint{12.919172in}{6.599957in}}%
\pgfpathlineto{\pgfqpoint{12.921454in}{6.614219in}}%
\pgfpathlineto{\pgfqpoint{12.923736in}{6.615058in}}%
\pgfpathlineto{\pgfqpoint{12.926018in}{6.601914in}}%
\pgfpathlineto{\pgfqpoint{12.932864in}{6.603592in}}%
\pgfpathlineto{\pgfqpoint{12.935146in}{6.609745in}}%
\pgfpathlineto{\pgfqpoint{12.937428in}{6.617575in}}%
\pgfpathlineto{\pgfqpoint{12.939710in}{6.628202in}}%
\pgfpathlineto{\pgfqpoint{12.941992in}{6.620652in}}%
\pgfpathlineto{\pgfqpoint{12.948837in}{6.625126in}}%
\pgfpathlineto{\pgfqpoint{12.953401in}{6.616457in}}%
\pgfpathlineto{\pgfqpoint{12.955683in}{6.618414in}}%
\pgfpathlineto{\pgfqpoint{12.957965in}{6.594363in}}%
\pgfpathlineto{\pgfqpoint{12.964811in}{6.578423in}}%
\pgfpathlineto{\pgfqpoint{12.967093in}{6.579262in}}%
\pgfpathlineto{\pgfqpoint{12.969374in}{6.573389in}}%
\pgfpathlineto{\pgfqpoint{12.971656in}{6.582897in}}%
\pgfpathlineto{\pgfqpoint{12.983066in}{6.566397in}}%
\pgfpathlineto{\pgfqpoint{12.985348in}{6.556329in}}%
\pgfpathlineto{\pgfqpoint{12.987630in}{6.542067in}}%
\pgfpathlineto{\pgfqpoint{12.989912in}{6.547939in}}%
\pgfpathlineto{\pgfqpoint{12.996757in}{6.561923in}}%
\pgfpathlineto{\pgfqpoint{12.999039in}{6.559965in}}%
\pgfpathlineto{\pgfqpoint{13.001321in}{6.561084in}}%
\pgfpathlineto{\pgfqpoint{13.003603in}{6.566397in}}%
\pgfpathlineto{\pgfqpoint{13.005885in}{6.555490in}}%
\pgfpathlineto{\pgfqpoint{13.012731in}{6.546261in}}%
\pgfpathlineto{\pgfqpoint{13.015013in}{6.536753in}}%
\pgfpathlineto{\pgfqpoint{13.017294in}{6.533956in}}%
\pgfpathlineto{\pgfqpoint{13.019576in}{6.533956in}}%
\pgfpathlineto{\pgfqpoint{13.021858in}{6.520253in}}%
\pgfpathlineto{\pgfqpoint{13.028704in}{6.527524in}}%
\pgfpathlineto{\pgfqpoint{13.030986in}{6.542905in}}%
\pgfpathlineto{\pgfqpoint{13.033268in}{6.544024in}}%
\pgfpathlineto{\pgfqpoint{13.035550in}{6.541507in}}%
\pgfpathlineto{\pgfqpoint{13.037832in}{6.542626in}}%
\pgfpathlineto{\pgfqpoint{13.044677in}{6.545143in}}%
\pgfpathlineto{\pgfqpoint{13.046959in}{6.547100in}}%
\pgfpathlineto{\pgfqpoint{13.049241in}{6.552973in}}%
\pgfpathlineto{\pgfqpoint{13.053805in}{6.546821in}}%
\pgfpathlineto{\pgfqpoint{13.060651in}{6.565838in}}%
\pgfpathlineto{\pgfqpoint{13.062933in}{6.549058in}}%
\pgfpathlineto{\pgfqpoint{13.065215in}{6.561363in}}%
\pgfpathlineto{\pgfqpoint{13.067496in}{6.546541in}}%
\pgfpathlineto{\pgfqpoint{13.069778in}{6.550456in}}%
\pgfpathlineto{\pgfqpoint{13.076624in}{6.551855in}}%
\pgfpathlineto{\pgfqpoint{13.078906in}{6.547660in}}%
\pgfpathlineto{\pgfqpoint{13.081188in}{6.533677in}}%
\pgfpathlineto{\pgfqpoint{13.085752in}{6.488371in}}%
\pgfpathlineto{\pgfqpoint{13.092597in}{6.480541in}}%
\pgfpathlineto{\pgfqpoint{13.094879in}{6.471032in}}%
\pgfpathlineto{\pgfqpoint{13.097161in}{6.507109in}}%
\pgfpathlineto{\pgfqpoint{13.099443in}{6.517736in}}%
\pgfpathlineto{\pgfqpoint{13.101725in}{6.535355in}}%
\pgfpathlineto{\pgfqpoint{13.108571in}{6.538151in}}%
\pgfpathlineto{\pgfqpoint{13.110853in}{6.520253in}}%
\pgfpathlineto{\pgfqpoint{13.113135in}{6.539829in}}%
\pgfpathlineto{\pgfqpoint{13.115416in}{6.551855in}}%
\pgfpathlineto{\pgfqpoint{13.117698in}{6.537872in}}%
\pgfpathlineto{\pgfqpoint{13.126826in}{6.562482in}}%
\pgfpathlineto{\pgfqpoint{13.129108in}{6.555770in}}%
\pgfpathlineto{\pgfqpoint{13.131390in}{6.556609in}}%
\pgfpathlineto{\pgfqpoint{13.133672in}{6.561643in}}%
\pgfpathlineto{\pgfqpoint{13.140517in}{6.559685in}}%
\pgfpathlineto{\pgfqpoint{13.142799in}{6.568355in}}%
\pgfpathlineto{\pgfqpoint{13.145081in}{6.569194in}}%
\pgfpathlineto{\pgfqpoint{13.147363in}{6.567795in}}%
\pgfpathlineto{\pgfqpoint{13.149645in}{6.550456in}}%
\pgfpathlineto{\pgfqpoint{13.156491in}{6.553812in}}%
\pgfpathlineto{\pgfqpoint{13.158773in}{6.541507in}}%
\pgfpathlineto{\pgfqpoint{13.161055in}{6.543185in}}%
\pgfpathlineto{\pgfqpoint{13.163337in}{6.536753in}}%
\pgfpathlineto{\pgfqpoint{13.165618in}{6.545143in}}%
\pgfpathlineto{\pgfqpoint{13.172464in}{6.543744in}}%
\pgfpathlineto{\pgfqpoint{13.174746in}{6.555770in}}%
\pgfpathlineto{\pgfqpoint{13.177028in}{6.578423in}}%
\pgfpathlineto{\pgfqpoint{13.179310in}{6.575067in}}%
\pgfpathlineto{\pgfqpoint{13.181592in}{6.587931in}}%
\pgfpathlineto{\pgfqpoint{13.188437in}{6.605550in}}%
\pgfpathlineto{\pgfqpoint{13.193001in}{6.633516in}}%
\pgfpathlineto{\pgfqpoint{13.195283in}{6.638550in}}%
\pgfpathlineto{\pgfqpoint{13.197565in}{6.629041in}}%
\pgfpathlineto{\pgfqpoint{13.204411in}{6.630719in}}%
\pgfpathlineto{\pgfqpoint{13.206693in}{6.626524in}}%
\pgfpathlineto{\pgfqpoint{13.208975in}{6.645542in}}%
\pgfpathlineto{\pgfqpoint{13.211257in}{6.644423in}}%
\pgfpathlineto{\pgfqpoint{13.213538in}{6.651694in}}%
\pgfpathlineto{\pgfqpoint{13.220384in}{6.665398in}}%
\pgfpathlineto{\pgfqpoint{13.222666in}{6.661762in}}%
\pgfpathlineto{\pgfqpoint{13.224948in}{6.659804in}}%
\pgfpathlineto{\pgfqpoint{13.227230in}{6.685813in}}%
\pgfpathlineto{\pgfqpoint{13.229512in}{6.698398in}}%
\pgfpathlineto{\pgfqpoint{13.238639in}{6.687491in}}%
\pgfpathlineto{\pgfqpoint{13.240921in}{6.693643in}}%
\pgfpathlineto{\pgfqpoint{13.243203in}{6.676584in}}%
\pgfpathlineto{\pgfqpoint{13.245485in}{6.672389in}}%
\pgfpathlineto{\pgfqpoint{13.252331in}{6.678542in}}%
\pgfpathlineto{\pgfqpoint{13.254613in}{6.683576in}}%
\pgfpathlineto{\pgfqpoint{13.256895in}{6.685533in}}%
\pgfpathlineto{\pgfqpoint{13.261459in}{6.677703in}}%
\pgfpathlineto{\pgfqpoint{13.270586in}{6.661762in}}%
\pgfpathlineto{\pgfqpoint{13.272868in}{6.652813in}}%
\pgfpathlineto{\pgfqpoint{13.275150in}{6.641347in}}%
\pgfpathlineto{\pgfqpoint{13.277432in}{6.633796in}}%
\pgfpathlineto{\pgfqpoint{13.284278in}{6.633516in}}%
\pgfpathlineto{\pgfqpoint{13.288841in}{6.660643in}}%
\pgfpathlineto{\pgfqpoint{13.291123in}{6.689448in}}%
\pgfpathlineto{\pgfqpoint{13.293405in}{6.698677in}}%
\pgfpathlineto{\pgfqpoint{13.300251in}{6.693923in}}%
\pgfpathlineto{\pgfqpoint{13.302533in}{6.690847in}}%
\pgfpathlineto{\pgfqpoint{13.304815in}{6.693364in}}%
\pgfpathlineto{\pgfqpoint{13.309379in}{6.693364in}}%
\pgfpathlineto{\pgfqpoint{13.316224in}{6.701474in}}%
\pgfpathlineto{\pgfqpoint{13.318506in}{6.709584in}}%
\pgfpathlineto{\pgfqpoint{13.320788in}{6.702872in}}%
\pgfpathlineto{\pgfqpoint{13.323070in}{6.682737in}}%
\pgfpathlineto{\pgfqpoint{13.325352in}{6.705669in}}%
\pgfpathlineto{\pgfqpoint{13.332198in}{6.706788in}}%
\pgfpathlineto{\pgfqpoint{13.334480in}{6.700915in}}%
\pgfpathlineto{\pgfqpoint{13.336761in}{6.702313in}}%
\pgfpathlineto{\pgfqpoint{13.339043in}{6.701474in}}%
\pgfpathlineto{\pgfqpoint{13.341325in}{6.688609in}}%
\pgfpathlineto{\pgfqpoint{13.348171in}{6.693643in}}%
\pgfpathlineto{\pgfqpoint{13.350453in}{6.711821in}}%
\pgfpathlineto{\pgfqpoint{13.352735in}{6.714898in}}%
\pgfpathlineto{\pgfqpoint{13.355017in}{6.704830in}}%
\pgfpathlineto{\pgfqpoint{13.357299in}{6.678542in}}%
\pgfpathlineto{\pgfqpoint{13.364144in}{6.687770in}}%
\pgfpathlineto{\pgfqpoint{13.366426in}{6.700355in}}%
\pgfpathlineto{\pgfqpoint{13.368708in}{6.707067in}}%
\pgfpathlineto{\pgfqpoint{13.380118in}{6.705389in}}%
\pgfpathlineto{\pgfqpoint{13.382400in}{6.718254in}}%
\pgfpathlineto{\pgfqpoint{13.386963in}{6.693364in}}%
\pgfpathlineto{\pgfqpoint{13.396091in}{6.681618in}}%
\pgfpathlineto{\pgfqpoint{13.398373in}{6.677423in}}%
\pgfpathlineto{\pgfqpoint{13.400655in}{6.658406in}}%
\pgfpathlineto{\pgfqpoint{13.402937in}{6.627084in}}%
\pgfpathlineto{\pgfqpoint{13.405219in}{6.618694in}}%
\pgfpathlineto{\pgfqpoint{13.412064in}{6.632677in}}%
\pgfpathlineto{\pgfqpoint{13.414346in}{6.648338in}}%
\pgfpathlineto{\pgfqpoint{13.416628in}{6.628762in}}%
\pgfpathlineto{\pgfqpoint{13.418910in}{6.649736in}}%
\pgfpathlineto{\pgfqpoint{13.421192in}{6.574228in}}%
\pgfpathlineto{\pgfqpoint{13.430320in}{6.575346in}}%
\pgfpathlineto{\pgfqpoint{13.432602in}{6.570033in}}%
\pgfpathlineto{\pgfqpoint{13.434883in}{6.571711in}}%
\pgfpathlineto{\pgfqpoint{13.437165in}{6.578702in}}%
\pgfpathlineto{\pgfqpoint{13.444011in}{6.570312in}}%
\pgfpathlineto{\pgfqpoint{13.446293in}{6.578702in}}%
\pgfpathlineto{\pgfqpoint{13.448575in}{6.575626in}}%
\pgfpathlineto{\pgfqpoint{13.450857in}{6.579541in}}%
\pgfpathlineto{\pgfqpoint{13.453139in}{6.606109in}}%
\pgfpathlineto{\pgfqpoint{13.459984in}{6.601075in}}%
\pgfpathlineto{\pgfqpoint{13.462266in}{6.575346in}}%
\pgfpathlineto{\pgfqpoint{13.464548in}{6.570033in}}%
\pgfpathlineto{\pgfqpoint{13.466830in}{6.581219in}}%
\pgfpathlineto{\pgfqpoint{13.469112in}{6.562482in}}%
\pgfpathlineto{\pgfqpoint{13.475958in}{6.556889in}}%
\pgfpathlineto{\pgfqpoint{13.478240in}{6.556609in}}%
\pgfpathlineto{\pgfqpoint{13.480522in}{6.541787in}}%
\pgfpathlineto{\pgfqpoint{13.482803in}{6.541507in}}%
\pgfpathlineto{\pgfqpoint{13.485085in}{6.552134in}}%
\pgfpathlineto{\pgfqpoint{13.494213in}{6.555770in}}%
\pgfpathlineto{\pgfqpoint{13.496495in}{6.573389in}}%
\pgfpathlineto{\pgfqpoint{13.498777in}{6.572270in}}%
\pgfpathlineto{\pgfqpoint{13.501059in}{6.554092in}}%
\pgfpathlineto{\pgfqpoint{13.507904in}{6.570312in}}%
\pgfpathlineto{\pgfqpoint{13.510186in}{6.556329in}}%
\pgfpathlineto{\pgfqpoint{13.514750in}{6.577304in}}%
\pgfpathlineto{\pgfqpoint{13.517032in}{6.582058in}}%
\pgfpathlineto{\pgfqpoint{13.523878in}{6.576465in}}%
\pgfpathlineto{\pgfqpoint{13.526160in}{6.596601in}}%
\pgfpathlineto{\pgfqpoint{13.528442in}{6.600796in}}%
\pgfpathlineto{\pgfqpoint{13.533005in}{6.603033in}}%
\pgfpathlineto{\pgfqpoint{13.539851in}{6.611143in}}%
\pgfpathlineto{\pgfqpoint{13.542133in}{6.601355in}}%
\pgfpathlineto{\pgfqpoint{13.546697in}{6.618974in}}%
\pgfpathlineto{\pgfqpoint{13.548979in}{6.632118in}}%
\pgfpathlineto{\pgfqpoint{13.555824in}{6.623728in}}%
\pgfpathlineto{\pgfqpoint{13.558106in}{6.629321in}}%
\pgfpathlineto{\pgfqpoint{13.560388in}{6.630160in}}%
\pgfpathlineto{\pgfqpoint{13.562670in}{6.637711in}}%
\pgfpathlineto{\pgfqpoint{13.564952in}{6.655609in}}%
\pgfpathlineto{\pgfqpoint{13.571798in}{6.646940in}}%
\pgfpathlineto{\pgfqpoint{13.574080in}{6.646380in}}%
\pgfpathlineto{\pgfqpoint{13.576362in}{6.638270in}}%
\pgfpathlineto{\pgfqpoint{13.578644in}{6.635194in}}%
\pgfpathlineto{\pgfqpoint{13.587771in}{6.635753in}}%
\pgfpathlineto{\pgfqpoint{13.590053in}{6.648058in}}%
\pgfpathlineto{\pgfqpoint{13.592335in}{6.656448in}}%
\pgfpathlineto{\pgfqpoint{13.594617in}{6.647219in}}%
\pgfpathlineto{\pgfqpoint{13.596899in}{6.649736in}}%
\pgfpathlineto{\pgfqpoint{13.603745in}{6.638270in}}%
\pgfpathlineto{\pgfqpoint{13.606026in}{6.635753in}}%
\pgfpathlineto{\pgfqpoint{13.608308in}{6.640228in}}%
\pgfpathlineto{\pgfqpoint{13.610590in}{6.626804in}}%
\pgfpathlineto{\pgfqpoint{13.612872in}{6.628762in}}%
\pgfpathlineto{\pgfqpoint{13.619718in}{6.629880in}}%
\pgfpathlineto{\pgfqpoint{13.622000in}{6.634635in}}%
\pgfpathlineto{\pgfqpoint{13.624282in}{6.641626in}}%
\pgfpathlineto{\pgfqpoint{13.628846in}{6.624287in}}%
\pgfpathlineto{\pgfqpoint{13.635691in}{6.629321in}}%
\pgfpathlineto{\pgfqpoint{13.637973in}{6.627923in}}%
\pgfpathlineto{\pgfqpoint{13.640255in}{6.638270in}}%
\pgfpathlineto{\pgfqpoint{13.642537in}{6.637431in}}%
\pgfpathlineto{\pgfqpoint{13.644819in}{6.629041in}}%
\pgfpathlineto{\pgfqpoint{13.651665in}{6.622609in}}%
\pgfpathlineto{\pgfqpoint{13.653946in}{6.622889in}}%
\pgfpathlineto{\pgfqpoint{13.656228in}{6.631838in}}%
\pgfpathlineto{\pgfqpoint{13.658510in}{6.615338in}}%
\pgfpathlineto{\pgfqpoint{13.660792in}{6.594084in}}%
\pgfpathlineto{\pgfqpoint{13.667638in}{6.602753in}}%
\pgfpathlineto{\pgfqpoint{13.672202in}{6.589889in}}%
\pgfpathlineto{\pgfqpoint{13.674484in}{6.591007in}}%
\pgfpathlineto{\pgfqpoint{13.676766in}{6.594643in}}%
\pgfpathlineto{\pgfqpoint{13.683611in}{6.588490in}}%
\pgfpathlineto{\pgfqpoint{13.685893in}{6.597160in}}%
\pgfpathlineto{\pgfqpoint{13.688175in}{6.595202in}}%
\pgfpathlineto{\pgfqpoint{13.690457in}{6.587372in}}%
\pgfpathlineto{\pgfqpoint{13.699585in}{6.603872in}}%
\pgfpathlineto{\pgfqpoint{13.701867in}{6.593245in}}%
\pgfpathlineto{\pgfqpoint{13.704148in}{6.593524in}}%
\pgfpathlineto{\pgfqpoint{13.706430in}{6.584016in}}%
\pgfpathlineto{\pgfqpoint{13.708712in}{6.597440in}}%
\pgfpathlineto{\pgfqpoint{13.715558in}{6.599677in}}%
\pgfpathlineto{\pgfqpoint{13.717840in}{6.620931in}}%
\pgfpathlineto{\pgfqpoint{13.720122in}{6.629601in}}%
\pgfpathlineto{\pgfqpoint{13.724686in}{6.634075in}}%
\pgfpathlineto{\pgfqpoint{13.733813in}{6.634635in}}%
\pgfpathlineto{\pgfqpoint{13.738377in}{6.639109in}}%
\pgfpathlineto{\pgfqpoint{13.740659in}{6.635474in}}%
\pgfpathlineto{\pgfqpoint{13.747505in}{6.637152in}}%
\pgfpathlineto{\pgfqpoint{13.749787in}{6.642186in}}%
\pgfpathlineto{\pgfqpoint{13.752068in}{6.642465in}}%
\pgfpathlineto{\pgfqpoint{13.754350in}{6.643864in}}%
\pgfpathlineto{\pgfqpoint{13.756632in}{6.646380in}}%
\pgfpathlineto{\pgfqpoint{13.763478in}{6.649736in}}%
\pgfpathlineto{\pgfqpoint{13.765760in}{6.648897in}}%
\pgfpathlineto{\pgfqpoint{13.768042in}{6.635194in}}%
\pgfpathlineto{\pgfqpoint{13.772606in}{6.639109in}}%
\pgfpathlineto{\pgfqpoint{13.781733in}{6.653652in}}%
\pgfpathlineto{\pgfqpoint{13.784015in}{6.652813in}}%
\pgfpathlineto{\pgfqpoint{13.786297in}{6.670711in}}%
\pgfpathlineto{\pgfqpoint{13.788579in}{6.633796in}}%
\pgfpathlineto{\pgfqpoint{13.795425in}{6.612262in}}%
\pgfpathlineto{\pgfqpoint{13.797707in}{6.624287in}}%
\pgfpathlineto{\pgfqpoint{13.802270in}{6.665957in}}%
\pgfpathlineto{\pgfqpoint{13.804552in}{6.664559in}}%
\pgfpathlineto{\pgfqpoint{13.813680in}{6.662881in}}%
\pgfpathlineto{\pgfqpoint{13.818244in}{6.676304in}}%
\pgfpathlineto{\pgfqpoint{13.820526in}{6.696999in}}%
\pgfpathlineto{\pgfqpoint{13.827371in}{6.706788in}}%
\pgfpathlineto{\pgfqpoint{13.829653in}{6.721050in}}%
\pgfpathlineto{\pgfqpoint{13.831935in}{6.723008in}}%
\pgfpathlineto{\pgfqpoint{13.834217in}{6.727762in}}%
\pgfpathlineto{\pgfqpoint{13.836499in}{6.724406in}}%
\pgfpathlineto{\pgfqpoint{13.843345in}{6.724127in}}%
\pgfpathlineto{\pgfqpoint{13.845627in}{6.726644in}}%
\pgfpathlineto{\pgfqpoint{13.847909in}{6.740627in}}%
\pgfpathlineto{\pgfqpoint{13.850190in}{6.703991in}}%
\pgfpathlineto{\pgfqpoint{13.852472in}{6.714059in}}%
\pgfpathlineto{\pgfqpoint{13.859318in}{6.714618in}}%
\pgfpathlineto{\pgfqpoint{13.861600in}{6.724966in}}%
\pgfpathlineto{\pgfqpoint{13.863882in}{6.718254in}}%
\pgfpathlineto{\pgfqpoint{13.866164in}{6.716855in}}%
\pgfpathlineto{\pgfqpoint{13.868446in}{6.719093in}}%
\pgfpathlineto{\pgfqpoint{13.875291in}{6.718813in}}%
\pgfpathlineto{\pgfqpoint{13.877573in}{6.711262in}}%
\pgfpathlineto{\pgfqpoint{13.879855in}{6.710143in}}%
\pgfpathlineto{\pgfqpoint{13.882137in}{6.718533in}}%
\pgfpathlineto{\pgfqpoint{13.884419in}{6.729160in}}%
\pgfpathlineto{\pgfqpoint{13.891265in}{6.730559in}}%
\pgfpathlineto{\pgfqpoint{13.893547in}{6.727483in}}%
\pgfpathlineto{\pgfqpoint{13.895829in}{6.717415in}}%
\pgfpathlineto{\pgfqpoint{13.898111in}{6.721330in}}%
\pgfpathlineto{\pgfqpoint{13.900392in}{6.718254in}}%
\pgfpathlineto{\pgfqpoint{13.907238in}{6.727203in}}%
\pgfpathlineto{\pgfqpoint{13.909520in}{6.735033in}}%
\pgfpathlineto{\pgfqpoint{13.911802in}{6.729999in}}%
\pgfpathlineto{\pgfqpoint{13.914084in}{6.728881in}}%
\pgfpathlineto{\pgfqpoint{13.916366in}{6.735872in}}%
\pgfpathlineto{\pgfqpoint{13.925493in}{6.740067in}}%
\pgfpathlineto{\pgfqpoint{13.927775in}{6.733355in}}%
\pgfpathlineto{\pgfqpoint{13.930057in}{6.731957in}}%
\pgfpathlineto{\pgfqpoint{13.932339in}{6.736432in}}%
\pgfpathlineto{\pgfqpoint{13.939185in}{6.743983in}}%
\pgfpathlineto{\pgfqpoint{13.946031in}{6.756008in}}%
\pgfpathlineto{\pgfqpoint{13.948312in}{6.757686in}}%
\pgfpathlineto{\pgfqpoint{13.957440in}{6.770271in}}%
\pgfpathlineto{\pgfqpoint{13.959722in}{6.767474in}}%
\pgfpathlineto{\pgfqpoint{13.962004in}{6.766915in}}%
\pgfpathlineto{\pgfqpoint{13.964286in}{6.740906in}}%
\pgfpathlineto{\pgfqpoint{13.971132in}{6.757686in}}%
\pgfpathlineto{\pgfqpoint{13.973413in}{6.745381in}}%
\pgfpathlineto{\pgfqpoint{13.975695in}{6.745661in}}%
\pgfpathlineto{\pgfqpoint{13.980259in}{6.799076in}}%
\pgfpathlineto{\pgfqpoint{13.987105in}{6.785652in}}%
\pgfpathlineto{\pgfqpoint{13.989387in}{6.785093in}}%
\pgfpathlineto{\pgfqpoint{13.991669in}{6.793203in}}%
\pgfpathlineto{\pgfqpoint{13.993951in}{6.796000in}}%
\pgfpathlineto{\pgfqpoint{13.996233in}{6.786491in}}%
\pgfpathlineto{\pgfqpoint{14.003078in}{6.772508in}}%
\pgfpathlineto{\pgfqpoint{14.005360in}{6.786212in}}%
\pgfpathlineto{\pgfqpoint{14.007642in}{6.792923in}}%
\pgfpathlineto{\pgfqpoint{14.009924in}{6.789847in}}%
\pgfpathlineto{\pgfqpoint{14.012206in}{6.801034in}}%
\pgfpathlineto{\pgfqpoint{14.019052in}{6.798796in}}%
\pgfpathlineto{\pgfqpoint{14.021333in}{6.795720in}}%
\pgfpathlineto{\pgfqpoint{14.023615in}{6.807186in}}%
\pgfpathlineto{\pgfqpoint{14.025897in}{6.809424in}}%
\pgfpathlineto{\pgfqpoint{14.028179in}{6.810263in}}%
\pgfpathlineto{\pgfqpoint{14.035025in}{6.808025in}}%
\pgfpathlineto{\pgfqpoint{14.037307in}{6.788729in}}%
\pgfpathlineto{\pgfqpoint{14.041871in}{6.780898in}}%
\pgfpathlineto{\pgfqpoint{14.044153in}{6.793203in}}%
\pgfpathlineto{\pgfqpoint{14.050998in}{6.789008in}}%
\pgfpathlineto{\pgfqpoint{14.053280in}{6.801034in}}%
\pgfpathlineto{\pgfqpoint{14.055562in}{6.742864in}}%
\pgfpathlineto{\pgfqpoint{14.057844in}{6.740627in}}%
\pgfpathlineto{\pgfqpoint{14.060126in}{6.733355in}}%
\pgfpathlineto{\pgfqpoint{14.066972in}{6.736432in}}%
\pgfpathlineto{\pgfqpoint{14.073817in}{6.724686in}}%
\pgfpathlineto{\pgfqpoint{14.076099in}{6.722728in}}%
\pgfpathlineto{\pgfqpoint{14.082945in}{6.726084in}}%
\pgfpathlineto{\pgfqpoint{14.085227in}{6.717135in}}%
\pgfpathlineto{\pgfqpoint{14.087509in}{6.719093in}}%
\pgfpathlineto{\pgfqpoint{14.092073in}{6.700076in}}%
\pgfpathlineto{\pgfqpoint{14.098918in}{6.728321in}}%
\pgfpathlineto{\pgfqpoint{14.101200in}{6.729720in}}%
\pgfpathlineto{\pgfqpoint{14.103482in}{6.729999in}}%
\pgfpathlineto{\pgfqpoint{14.105764in}{6.723288in}}%
\pgfpathlineto{\pgfqpoint{14.108046in}{6.726364in}}%
\pgfpathlineto{\pgfqpoint{14.114892in}{6.722728in}}%
\pgfpathlineto{\pgfqpoint{14.117174in}{6.734194in}}%
\pgfpathlineto{\pgfqpoint{14.119455in}{6.732237in}}%
\pgfpathlineto{\pgfqpoint{14.121737in}{6.736991in}}%
\pgfpathlineto{\pgfqpoint{14.124019in}{6.735033in}}%
\pgfpathlineto{\pgfqpoint{14.130865in}{6.735872in}}%
\pgfpathlineto{\pgfqpoint{14.133147in}{6.749016in}}%
\pgfpathlineto{\pgfqpoint{14.135429in}{6.741745in}}%
\pgfpathlineto{\pgfqpoint{14.139993in}{6.747898in}}%
\pgfpathlineto{\pgfqpoint{14.146838in}{6.749855in}}%
\pgfpathlineto{\pgfqpoint{14.149120in}{6.744542in}}%
\pgfpathlineto{\pgfqpoint{14.151402in}{6.728601in}}%
\pgfpathlineto{\pgfqpoint{14.153684in}{6.703991in}}%
\pgfpathlineto{\pgfqpoint{14.155966in}{6.714338in}}%
\pgfpathlineto{\pgfqpoint{14.162812in}{6.720491in}}%
\pgfpathlineto{\pgfqpoint{14.165094in}{6.729160in}}%
\pgfpathlineto{\pgfqpoint{14.167376in}{6.749576in}}%
\pgfpathlineto{\pgfqpoint{14.169657in}{6.754889in}}%
\pgfpathlineto{\pgfqpoint{14.171939in}{6.756288in}}%
\pgfpathlineto{\pgfqpoint{14.178785in}{6.761881in}}%
\pgfpathlineto{\pgfqpoint{14.181067in}{6.783695in}}%
\pgfpathlineto{\pgfqpoint{14.183349in}{6.776983in}}%
\pgfpathlineto{\pgfqpoint{14.185631in}{6.783415in}}%
\pgfpathlineto{\pgfqpoint{14.187913in}{6.770830in}}%
\pgfpathlineto{\pgfqpoint{14.194758in}{6.785932in}}%
\pgfpathlineto{\pgfqpoint{14.197040in}{6.794322in}}%
\pgfpathlineto{\pgfqpoint{14.199322in}{6.788449in}}%
\pgfpathlineto{\pgfqpoint{14.201604in}{6.787051in}}%
\pgfpathlineto{\pgfqpoint{14.203886in}{6.788169in}}%
\pgfpathlineto{\pgfqpoint{14.213014in}{6.790686in}}%
\pgfpathlineto{\pgfqpoint{14.215296in}{6.779220in}}%
\pgfpathlineto{\pgfqpoint{14.217577in}{6.780059in}}%
\pgfpathlineto{\pgfqpoint{14.219859in}{6.769711in}}%
\pgfpathlineto{\pgfqpoint{14.228987in}{6.778381in}}%
\pgfpathlineto{\pgfqpoint{14.231269in}{6.773347in}}%
\pgfpathlineto{\pgfqpoint{14.233551in}{6.771949in}}%
\pgfpathlineto{\pgfqpoint{14.235833in}{6.775305in}}%
\pgfpathlineto{\pgfqpoint{14.242678in}{6.778661in}}%
\pgfpathlineto{\pgfqpoint{14.244960in}{6.776703in}}%
\pgfpathlineto{\pgfqpoint{14.247242in}{6.787610in}}%
\pgfpathlineto{\pgfqpoint{14.249524in}{6.781178in}}%
\pgfpathlineto{\pgfqpoint{14.251806in}{6.783415in}}%
\pgfpathlineto{\pgfqpoint{14.260934in}{6.783695in}}%
\pgfpathlineto{\pgfqpoint{14.263216in}{6.782576in}}%
\pgfpathlineto{\pgfqpoint{14.265498in}{6.777542in}}%
\pgfpathlineto{\pgfqpoint{14.267779in}{6.787330in}}%
\pgfpathlineto{\pgfqpoint{14.274625in}{6.782856in}}%
\pgfpathlineto{\pgfqpoint{14.276907in}{6.805229in}}%
\pgfpathlineto{\pgfqpoint{14.279189in}{6.809703in}}%
\pgfpathlineto{\pgfqpoint{14.281471in}{6.803551in}}%
\pgfpathlineto{\pgfqpoint{14.283753in}{6.814457in}}%
\pgfpathlineto{\pgfqpoint{14.290599in}{6.799915in}}%
\pgfpathlineto{\pgfqpoint{14.292880in}{6.784254in}}%
\pgfpathlineto{\pgfqpoint{14.295162in}{6.776423in}}%
\pgfpathlineto{\pgfqpoint{14.297444in}{6.780618in}}%
\pgfpathlineto{\pgfqpoint{14.299726in}{6.783135in}}%
\pgfpathlineto{\pgfqpoint{14.306572in}{6.776423in}}%
\pgfpathlineto{\pgfqpoint{14.308854in}{6.778661in}}%
\pgfpathlineto{\pgfqpoint{14.311136in}{6.779500in}}%
\pgfpathlineto{\pgfqpoint{14.313418in}{6.755169in}}%
\pgfpathlineto{\pgfqpoint{14.315699in}{6.752093in}}%
\pgfpathlineto{\pgfqpoint{14.324827in}{6.767474in}}%
\pgfpathlineto{\pgfqpoint{14.327109in}{6.770830in}}%
\pgfpathlineto{\pgfqpoint{14.329391in}{6.780339in}}%
\pgfpathlineto{\pgfqpoint{14.331673in}{6.782017in}}%
\pgfpathlineto{\pgfqpoint{14.340800in}{6.783135in}}%
\pgfpathlineto{\pgfqpoint{14.343082in}{6.771110in}}%
\pgfpathlineto{\pgfqpoint{14.345364in}{6.774186in}}%
\pgfpathlineto{\pgfqpoint{14.347646in}{6.783415in}}%
\pgfpathlineto{\pgfqpoint{14.354492in}{6.782856in}}%
\pgfpathlineto{\pgfqpoint{14.359056in}{6.767474in}}%
\pgfpathlineto{\pgfqpoint{14.363620in}{6.766635in}}%
\pgfpathlineto{\pgfqpoint{14.370465in}{6.757966in}}%
\pgfpathlineto{\pgfqpoint{14.372747in}{6.764118in}}%
\pgfpathlineto{\pgfqpoint{14.375029in}{6.759364in}}%
\pgfpathlineto{\pgfqpoint{14.377311in}{6.764678in}}%
\pgfpathlineto{\pgfqpoint{14.379593in}{6.766915in}}%
\pgfpathlineto{\pgfqpoint{14.386439in}{6.747339in}}%
\pgfpathlineto{\pgfqpoint{14.388720in}{6.747898in}}%
\pgfpathlineto{\pgfqpoint{14.391002in}{6.745661in}}%
\pgfpathlineto{\pgfqpoint{14.393284in}{6.746779in}}%
\pgfpathlineto{\pgfqpoint{14.395566in}{6.750135in}}%
\pgfpathlineto{\pgfqpoint{14.402412in}{6.754330in}}%
\pgfpathlineto{\pgfqpoint{14.404694in}{6.743983in}}%
\pgfpathlineto{\pgfqpoint{14.406976in}{6.752652in}}%
\pgfpathlineto{\pgfqpoint{14.411540in}{6.747339in}}%
\pgfpathlineto{\pgfqpoint{14.418385in}{6.753211in}}%
\pgfpathlineto{\pgfqpoint{14.420667in}{6.758805in}}%
\pgfpathlineto{\pgfqpoint{14.422949in}{6.757966in}}%
\pgfpathlineto{\pgfqpoint{14.425231in}{6.762720in}}%
\pgfpathlineto{\pgfqpoint{14.427513in}{6.771110in}}%
\pgfpathlineto{\pgfqpoint{14.434359in}{6.773627in}}%
\pgfpathlineto{\pgfqpoint{14.436641in}{6.776703in}}%
\pgfpathlineto{\pgfqpoint{14.438922in}{6.775305in}}%
\pgfpathlineto{\pgfqpoint{14.441204in}{6.770271in}}%
\pgfpathlineto{\pgfqpoint{14.443486in}{6.770271in}}%
\pgfpathlineto{\pgfqpoint{14.454896in}{6.759644in}}%
\pgfpathlineto{\pgfqpoint{14.457178in}{6.749576in}}%
\pgfpathlineto{\pgfqpoint{14.466305in}{6.755728in}}%
\pgfpathlineto{\pgfqpoint{14.468587in}{6.763279in}}%
\pgfpathlineto{\pgfqpoint{14.470869in}{6.766915in}}%
\pgfpathlineto{\pgfqpoint{14.473151in}{6.774186in}}%
\pgfpathlineto{\pgfqpoint{14.484561in}{6.792364in}}%
\pgfpathlineto{\pgfqpoint{14.486842in}{6.794042in}}%
\pgfpathlineto{\pgfqpoint{14.489124in}{6.807186in}}%
\pgfpathlineto{\pgfqpoint{14.491406in}{6.773347in}}%
\pgfpathlineto{\pgfqpoint{14.498252in}{6.777542in}}%
\pgfpathlineto{\pgfqpoint{14.500534in}{6.794881in}}%
\pgfpathlineto{\pgfqpoint{14.502816in}{6.802432in}}%
\pgfpathlineto{\pgfqpoint{14.505098in}{6.799076in}}%
\pgfpathlineto{\pgfqpoint{14.507380in}{6.798237in}}%
\pgfpathlineto{\pgfqpoint{14.514225in}{6.790686in}}%
\pgfpathlineto{\pgfqpoint{14.516507in}{6.786212in}}%
\pgfpathlineto{\pgfqpoint{14.521071in}{6.768313in}}%
\pgfpathlineto{\pgfqpoint{14.523353in}{6.763839in}}%
\pgfpathlineto{\pgfqpoint{14.530199in}{6.766635in}}%
\pgfpathlineto{\pgfqpoint{14.532481in}{6.771669in}}%
\pgfpathlineto{\pgfqpoint{14.534763in}{6.750974in}}%
\pgfpathlineto{\pgfqpoint{14.539326in}{6.760483in}}%
\pgfpathlineto{\pgfqpoint{14.548454in}{6.772788in}}%
\pgfpathlineto{\pgfqpoint{14.550736in}{6.779779in}}%
\pgfpathlineto{\pgfqpoint{14.553018in}{6.783415in}}%
\pgfpathlineto{\pgfqpoint{14.555300in}{6.783415in}}%
\pgfpathlineto{\pgfqpoint{14.564427in}{6.781178in}}%
\pgfpathlineto{\pgfqpoint{14.566709in}{6.779500in}}%
\pgfpathlineto{\pgfqpoint{14.568991in}{6.779779in}}%
\pgfpathlineto{\pgfqpoint{14.571273in}{6.785093in}}%
\pgfpathlineto{\pgfqpoint{14.578119in}{6.785373in}}%
\pgfpathlineto{\pgfqpoint{14.580401in}{6.780059in}}%
\pgfpathlineto{\pgfqpoint{14.582683in}{6.783415in}}%
\pgfpathlineto{\pgfqpoint{14.584964in}{6.789288in}}%
\pgfpathlineto{\pgfqpoint{14.587246in}{6.768873in}}%
\pgfpathlineto{\pgfqpoint{14.594092in}{6.769152in}}%
\pgfpathlineto{\pgfqpoint{14.596374in}{6.773347in}}%
\pgfpathlineto{\pgfqpoint{14.598656in}{6.763839in}}%
\pgfpathlineto{\pgfqpoint{14.600938in}{6.758245in}}%
\pgfpathlineto{\pgfqpoint{14.603220in}{6.755449in}}%
\pgfpathlineto{\pgfqpoint{14.610065in}{6.763559in}}%
\pgfpathlineto{\pgfqpoint{14.612347in}{6.746220in}}%
\pgfpathlineto{\pgfqpoint{14.616911in}{6.732796in}}%
\pgfpathlineto{\pgfqpoint{14.619193in}{6.728321in}}%
\pgfpathlineto{\pgfqpoint{14.626039in}{6.725245in}}%
\pgfpathlineto{\pgfqpoint{14.628321in}{6.714059in}}%
\pgfpathlineto{\pgfqpoint{14.630603in}{6.728601in}}%
\pgfpathlineto{\pgfqpoint{14.632885in}{6.710982in}}%
\pgfpathlineto{\pgfqpoint{14.635166in}{6.716296in}}%
\pgfpathlineto{\pgfqpoint{14.642012in}{6.709025in}}%
\pgfpathlineto{\pgfqpoint{14.646576in}{6.732237in}}%
\pgfpathlineto{\pgfqpoint{14.648858in}{6.713499in}}%
\pgfpathlineto{\pgfqpoint{14.651140in}{6.720211in}}%
\pgfpathlineto{\pgfqpoint{14.657986in}{6.714059in}}%
\pgfpathlineto{\pgfqpoint{14.660267in}{6.721050in}}%
\pgfpathlineto{\pgfqpoint{14.662549in}{6.729999in}}%
\pgfpathlineto{\pgfqpoint{14.664831in}{6.729720in}}%
\pgfpathlineto{\pgfqpoint{14.667113in}{6.741466in}}%
\pgfpathlineto{\pgfqpoint{14.673959in}{6.735872in}}%
\pgfpathlineto{\pgfqpoint{14.678523in}{6.738110in}}%
\pgfpathlineto{\pgfqpoint{14.680805in}{6.743144in}}%
\pgfpathlineto{\pgfqpoint{14.683086in}{6.742584in}}%
\pgfpathlineto{\pgfqpoint{14.689932in}{6.736711in}}%
\pgfpathlineto{\pgfqpoint{14.692214in}{6.741186in}}%
\pgfpathlineto{\pgfqpoint{14.694496in}{6.743144in}}%
\pgfpathlineto{\pgfqpoint{14.696778in}{6.749016in}}%
\pgfpathlineto{\pgfqpoint{14.699060in}{6.758245in}}%
\pgfpathlineto{\pgfqpoint{14.705906in}{6.762440in}}%
\pgfpathlineto{\pgfqpoint{14.708187in}{6.785652in}}%
\pgfpathlineto{\pgfqpoint{14.710469in}{6.793483in}}%
\pgfpathlineto{\pgfqpoint{14.712751in}{6.796839in}}%
\pgfpathlineto{\pgfqpoint{14.715033in}{6.791805in}}%
\pgfpathlineto{\pgfqpoint{14.721879in}{6.795161in}}%
\pgfpathlineto{\pgfqpoint{14.724161in}{6.794601in}}%
\pgfpathlineto{\pgfqpoint{14.726443in}{6.799356in}}%
\pgfpathlineto{\pgfqpoint{14.728725in}{6.787330in}}%
\pgfpathlineto{\pgfqpoint{14.731007in}{6.780059in}}%
\pgfpathlineto{\pgfqpoint{14.737852in}{6.792644in}}%
\pgfpathlineto{\pgfqpoint{14.740134in}{6.783695in}}%
\pgfpathlineto{\pgfqpoint{14.742416in}{6.778661in}}%
\pgfpathlineto{\pgfqpoint{14.744698in}{6.761322in}}%
\pgfpathlineto{\pgfqpoint{14.746980in}{6.757127in}}%
\pgfpathlineto{\pgfqpoint{14.753826in}{6.754610in}}%
\pgfpathlineto{\pgfqpoint{14.756107in}{6.747618in}}%
\pgfpathlineto{\pgfqpoint{14.758389in}{6.747898in}}%
\pgfpathlineto{\pgfqpoint{14.760671in}{6.749016in}}%
\pgfpathlineto{\pgfqpoint{14.762953in}{6.747898in}}%
\pgfpathlineto{\pgfqpoint{14.769799in}{6.747618in}}%
\pgfpathlineto{\pgfqpoint{14.772081in}{6.749576in}}%
\pgfpathlineto{\pgfqpoint{14.776645in}{6.758805in}}%
\pgfpathlineto{\pgfqpoint{14.778927in}{6.759364in}}%
\pgfpathlineto{\pgfqpoint{14.788054in}{6.757406in}}%
\pgfpathlineto{\pgfqpoint{14.790336in}{6.777262in}}%
\pgfpathlineto{\pgfqpoint{14.792618in}{6.771389in}}%
\pgfpathlineto{\pgfqpoint{14.794900in}{6.761881in}}%
\pgfpathlineto{\pgfqpoint{14.801746in}{6.777542in}}%
\pgfpathlineto{\pgfqpoint{14.806309in}{6.792364in}}%
\pgfpathlineto{\pgfqpoint{14.808591in}{6.796559in}}%
\pgfpathlineto{\pgfqpoint{14.810873in}{6.810542in}}%
\pgfpathlineto{\pgfqpoint{14.817719in}{6.810542in}}%
\pgfpathlineto{\pgfqpoint{14.820001in}{6.816695in}}%
\pgfpathlineto{\pgfqpoint{14.822283in}{6.812220in}}%
\pgfpathlineto{\pgfqpoint{14.824565in}{6.815856in}}%
\pgfpathlineto{\pgfqpoint{14.826847in}{6.815296in}}%
\pgfpathlineto{\pgfqpoint{14.833692in}{6.814737in}}%
\pgfpathlineto{\pgfqpoint{14.835974in}{6.823127in}}%
\pgfpathlineto{\pgfqpoint{14.838256in}{6.824805in}}%
\pgfpathlineto{\pgfqpoint{14.849666in}{6.865076in}}%
\pgfpathlineto{\pgfqpoint{14.851948in}{6.874305in}}%
\pgfpathlineto{\pgfqpoint{14.854229in}{6.873186in}}%
\pgfpathlineto{\pgfqpoint{14.856511in}{6.878220in}}%
\pgfpathlineto{\pgfqpoint{14.858793in}{6.880737in}}%
\pgfpathlineto{\pgfqpoint{14.865639in}{6.887170in}}%
\pgfpathlineto{\pgfqpoint{14.867921in}{6.881297in}}%
\pgfpathlineto{\pgfqpoint{14.870203in}{6.872068in}}%
\pgfpathlineto{\pgfqpoint{14.872485in}{6.868992in}}%
\pgfpathlineto{\pgfqpoint{14.874767in}{6.881856in}}%
\pgfpathlineto{\pgfqpoint{14.883894in}{6.885212in}}%
\pgfpathlineto{\pgfqpoint{14.886176in}{6.897517in}}%
\pgfpathlineto{\pgfqpoint{14.888458in}{6.893322in}}%
\pgfpathlineto{\pgfqpoint{14.890740in}{6.902271in}}%
\pgfpathlineto{\pgfqpoint{14.899868in}{6.916254in}}%
\pgfpathlineto{\pgfqpoint{14.902150in}{6.911780in}}%
\pgfpathlineto{\pgfqpoint{14.904431in}{6.926882in}}%
\pgfpathlineto{\pgfqpoint{14.906713in}{7.008543in}}%
\pgfpathlineto{\pgfqpoint{14.913559in}{7.007704in}}%
\pgfpathlineto{\pgfqpoint{14.918123in}{7.070628in}}%
\pgfpathlineto{\pgfqpoint{14.920405in}{7.080975in}}%
\pgfpathlineto{\pgfqpoint{14.922687in}{7.060560in}}%
\pgfpathlineto{\pgfqpoint{14.929532in}{7.077620in}}%
\pgfpathlineto{\pgfqpoint{14.931814in}{7.079857in}}%
\pgfpathlineto{\pgfqpoint{14.934096in}{7.077620in}}%
\pgfpathlineto{\pgfqpoint{14.936378in}{7.066992in}}%
\pgfpathlineto{\pgfqpoint{14.938660in}{7.047416in}}%
\pgfpathlineto{\pgfqpoint{14.945506in}{7.052170in}}%
\pgfpathlineto{\pgfqpoint{14.947788in}{7.054967in}}%
\pgfpathlineto{\pgfqpoint{14.950070in}{7.044340in}}%
\pgfpathlineto{\pgfqpoint{14.952351in}{7.049374in}}%
\pgfpathlineto{\pgfqpoint{14.954633in}{7.021967in}}%
\pgfpathlineto{\pgfqpoint{14.961479in}{7.021687in}}%
\pgfpathlineto{\pgfqpoint{14.963761in}{7.030357in}}%
\pgfpathlineto{\pgfqpoint{14.966043in}{7.022526in}}%
\pgfpathlineto{\pgfqpoint{14.970607in}{7.025043in}}%
\pgfpathlineto{\pgfqpoint{14.977452in}{7.018051in}}%
\pgfpathlineto{\pgfqpoint{14.979734in}{7.024484in}}%
\pgfpathlineto{\pgfqpoint{14.982016in}{7.003509in}}%
\pgfpathlineto{\pgfqpoint{14.984298in}{7.027560in}}%
\pgfpathlineto{\pgfqpoint{14.986580in}{7.023365in}}%
\pgfpathlineto{\pgfqpoint{14.993426in}{7.018051in}}%
\pgfpathlineto{\pgfqpoint{14.995708in}{6.989806in}}%
\pgfpathlineto{\pgfqpoint{14.997990in}{6.990085in}}%
\pgfpathlineto{\pgfqpoint{15.000272in}{6.980017in}}%
\pgfpathlineto{\pgfqpoint{15.002553in}{6.987289in}}%
\pgfpathlineto{\pgfqpoint{15.009399in}{6.995679in}}%
\pgfpathlineto{\pgfqpoint{15.011681in}{6.986729in}}%
\pgfpathlineto{\pgfqpoint{15.013963in}{6.987009in}}%
\pgfpathlineto{\pgfqpoint{15.016245in}{6.985051in}}%
\pgfpathlineto{\pgfqpoint{15.018527in}{7.020009in}}%
\pgfpathlineto{\pgfqpoint{15.025373in}{7.065874in}}%
\pgfpathlineto{\pgfqpoint{15.027654in}{7.086848in}}%
\pgfpathlineto{\pgfqpoint{15.029936in}{7.100831in}}%
\pgfpathlineto{\pgfqpoint{15.032218in}{7.079297in}}%
\pgfpathlineto{\pgfqpoint{15.034500in}{7.077620in}}%
\pgfpathlineto{\pgfqpoint{15.043628in}{7.060840in}}%
\pgfpathlineto{\pgfqpoint{15.045910in}{7.061679in}}%
\pgfpathlineto{\pgfqpoint{15.048192in}{7.064755in}}%
\pgfpathlineto{\pgfqpoint{15.050473in}{7.063077in}}%
\pgfpathlineto{\pgfqpoint{15.050473in}{7.063077in}}%
\pgfusepath{stroke}%
\end{pgfscope}%
\begin{pgfscope}%
\pgfsetrectcap%
\pgfsetmiterjoin%
\pgfsetlinewidth{0.803000pt}%
\definecolor{currentstroke}{rgb}{1.000000,1.000000,1.000000}%
\pgfsetstrokecolor{currentstroke}%
\pgfsetdash{}{0pt}%
\pgfpathmoveto{\pgfqpoint{9.810417in}{6.221951in}}%
\pgfpathlineto{\pgfqpoint{9.810417in}{7.142683in}}%
\pgfusepath{stroke}%
\end{pgfscope}%
\begin{pgfscope}%
\pgfsetrectcap%
\pgfsetmiterjoin%
\pgfsetlinewidth{0.803000pt}%
\definecolor{currentstroke}{rgb}{1.000000,1.000000,1.000000}%
\pgfsetstrokecolor{currentstroke}%
\pgfsetdash{}{0pt}%
\pgfpathmoveto{\pgfqpoint{15.300000in}{6.221951in}}%
\pgfpathlineto{\pgfqpoint{15.300000in}{7.142683in}}%
\pgfusepath{stroke}%
\end{pgfscope}%
\begin{pgfscope}%
\pgfsetrectcap%
\pgfsetmiterjoin%
\pgfsetlinewidth{0.803000pt}%
\definecolor{currentstroke}{rgb}{1.000000,1.000000,1.000000}%
\pgfsetstrokecolor{currentstroke}%
\pgfsetdash{}{0pt}%
\pgfpathmoveto{\pgfqpoint{9.810417in}{6.221951in}}%
\pgfpathlineto{\pgfqpoint{15.300000in}{6.221951in}}%
\pgfusepath{stroke}%
\end{pgfscope}%
\begin{pgfscope}%
\pgfsetrectcap%
\pgfsetmiterjoin%
\pgfsetlinewidth{0.803000pt}%
\definecolor{currentstroke}{rgb}{1.000000,1.000000,1.000000}%
\pgfsetstrokecolor{currentstroke}%
\pgfsetdash{}{0pt}%
\pgfpathmoveto{\pgfqpoint{9.810417in}{7.142683in}}%
\pgfpathlineto{\pgfqpoint{15.300000in}{7.142683in}}%
\pgfusepath{stroke}%
\end{pgfscope}%
\begin{pgfscope}%
\definecolor{textcolor}{rgb}{0.150000,0.150000,0.150000}%
\pgfsetstrokecolor{textcolor}%
\pgfsetfillcolor{textcolor}%
\pgftext[x=12.555208in,y=7.226016in,,base]{\color{textcolor}\rmfamily\fontsize{12.000000}{14.400000}\selectfont INTC}%
\end{pgfscope}%
\begin{pgfscope}%
\pgfsetbuttcap%
\pgfsetmiterjoin%
\definecolor{currentfill}{rgb}{0.917647,0.917647,0.949020}%
\pgfsetfillcolor{currentfill}%
\pgfsetlinewidth{0.000000pt}%
\definecolor{currentstroke}{rgb}{0.000000,0.000000,0.000000}%
\pgfsetstrokecolor{currentstroke}%
\pgfsetstrokeopacity{0.000000}%
\pgfsetdash{}{0pt}%
\pgfpathmoveto{\pgfqpoint{2.125000in}{4.564634in}}%
\pgfpathlineto{\pgfqpoint{7.614583in}{4.564634in}}%
\pgfpathlineto{\pgfqpoint{7.614583in}{5.485366in}}%
\pgfpathlineto{\pgfqpoint{2.125000in}{5.485366in}}%
\pgfpathclose%
\pgfusepath{fill}%
\end{pgfscope}%
\begin{pgfscope}%
\pgfpathrectangle{\pgfqpoint{2.125000in}{4.564634in}}{\pgfqpoint{5.489583in}{0.920732in}}%
\pgfusepath{clip}%
\pgfsetroundcap%
\pgfsetroundjoin%
\pgfsetlinewidth{0.803000pt}%
\definecolor{currentstroke}{rgb}{1.000000,1.000000,1.000000}%
\pgfsetstrokecolor{currentstroke}%
\pgfsetdash{}{0pt}%
\pgfpathmoveto{\pgfqpoint{2.369963in}{4.564634in}}%
\pgfpathlineto{\pgfqpoint{2.369963in}{5.485366in}}%
\pgfusepath{stroke}%
\end{pgfscope}%
\begin{pgfscope}%
\definecolor{textcolor}{rgb}{0.150000,0.150000,0.150000}%
\pgfsetstrokecolor{textcolor}%
\pgfsetfillcolor{textcolor}%
\pgftext[x=2.369963in,y=4.467412in,,top]{\color{textcolor}\rmfamily\fontsize{10.000000}{12.000000}\selectfont 2012}%
\end{pgfscope}%
\begin{pgfscope}%
\pgfpathrectangle{\pgfqpoint{2.125000in}{4.564634in}}{\pgfqpoint{5.489583in}{0.920732in}}%
\pgfusepath{clip}%
\pgfsetroundcap%
\pgfsetroundjoin%
\pgfsetlinewidth{0.803000pt}%
\definecolor{currentstroke}{rgb}{1.000000,1.000000,1.000000}%
\pgfsetstrokecolor{currentstroke}%
\pgfsetdash{}{0pt}%
\pgfpathmoveto{\pgfqpoint{3.205141in}{4.564634in}}%
\pgfpathlineto{\pgfqpoint{3.205141in}{5.485366in}}%
\pgfusepath{stroke}%
\end{pgfscope}%
\begin{pgfscope}%
\definecolor{textcolor}{rgb}{0.150000,0.150000,0.150000}%
\pgfsetstrokecolor{textcolor}%
\pgfsetfillcolor{textcolor}%
\pgftext[x=3.205141in,y=4.467412in,,top]{\color{textcolor}\rmfamily\fontsize{10.000000}{12.000000}\selectfont 2013}%
\end{pgfscope}%
\begin{pgfscope}%
\pgfpathrectangle{\pgfqpoint{2.125000in}{4.564634in}}{\pgfqpoint{5.489583in}{0.920732in}}%
\pgfusepath{clip}%
\pgfsetroundcap%
\pgfsetroundjoin%
\pgfsetlinewidth{0.803000pt}%
\definecolor{currentstroke}{rgb}{1.000000,1.000000,1.000000}%
\pgfsetstrokecolor{currentstroke}%
\pgfsetdash{}{0pt}%
\pgfpathmoveto{\pgfqpoint{4.038037in}{4.564634in}}%
\pgfpathlineto{\pgfqpoint{4.038037in}{5.485366in}}%
\pgfusepath{stroke}%
\end{pgfscope}%
\begin{pgfscope}%
\definecolor{textcolor}{rgb}{0.150000,0.150000,0.150000}%
\pgfsetstrokecolor{textcolor}%
\pgfsetfillcolor{textcolor}%
\pgftext[x=4.038037in,y=4.467412in,,top]{\color{textcolor}\rmfamily\fontsize{10.000000}{12.000000}\selectfont 2014}%
\end{pgfscope}%
\begin{pgfscope}%
\pgfpathrectangle{\pgfqpoint{2.125000in}{4.564634in}}{\pgfqpoint{5.489583in}{0.920732in}}%
\pgfusepath{clip}%
\pgfsetroundcap%
\pgfsetroundjoin%
\pgfsetlinewidth{0.803000pt}%
\definecolor{currentstroke}{rgb}{1.000000,1.000000,1.000000}%
\pgfsetstrokecolor{currentstroke}%
\pgfsetdash{}{0pt}%
\pgfpathmoveto{\pgfqpoint{4.870933in}{4.564634in}}%
\pgfpathlineto{\pgfqpoint{4.870933in}{5.485366in}}%
\pgfusepath{stroke}%
\end{pgfscope}%
\begin{pgfscope}%
\definecolor{textcolor}{rgb}{0.150000,0.150000,0.150000}%
\pgfsetstrokecolor{textcolor}%
\pgfsetfillcolor{textcolor}%
\pgftext[x=4.870933in,y=4.467412in,,top]{\color{textcolor}\rmfamily\fontsize{10.000000}{12.000000}\selectfont 2015}%
\end{pgfscope}%
\begin{pgfscope}%
\pgfpathrectangle{\pgfqpoint{2.125000in}{4.564634in}}{\pgfqpoint{5.489583in}{0.920732in}}%
\pgfusepath{clip}%
\pgfsetroundcap%
\pgfsetroundjoin%
\pgfsetlinewidth{0.803000pt}%
\definecolor{currentstroke}{rgb}{1.000000,1.000000,1.000000}%
\pgfsetstrokecolor{currentstroke}%
\pgfsetdash{}{0pt}%
\pgfpathmoveto{\pgfqpoint{5.703829in}{4.564634in}}%
\pgfpathlineto{\pgfqpoint{5.703829in}{5.485366in}}%
\pgfusepath{stroke}%
\end{pgfscope}%
\begin{pgfscope}%
\definecolor{textcolor}{rgb}{0.150000,0.150000,0.150000}%
\pgfsetstrokecolor{textcolor}%
\pgfsetfillcolor{textcolor}%
\pgftext[x=5.703829in,y=4.467412in,,top]{\color{textcolor}\rmfamily\fontsize{10.000000}{12.000000}\selectfont 2016}%
\end{pgfscope}%
\begin{pgfscope}%
\pgfpathrectangle{\pgfqpoint{2.125000in}{4.564634in}}{\pgfqpoint{5.489583in}{0.920732in}}%
\pgfusepath{clip}%
\pgfsetroundcap%
\pgfsetroundjoin%
\pgfsetlinewidth{0.803000pt}%
\definecolor{currentstroke}{rgb}{1.000000,1.000000,1.000000}%
\pgfsetstrokecolor{currentstroke}%
\pgfsetdash{}{0pt}%
\pgfpathmoveto{\pgfqpoint{6.539007in}{4.564634in}}%
\pgfpathlineto{\pgfqpoint{6.539007in}{5.485366in}}%
\pgfusepath{stroke}%
\end{pgfscope}%
\begin{pgfscope}%
\definecolor{textcolor}{rgb}{0.150000,0.150000,0.150000}%
\pgfsetstrokecolor{textcolor}%
\pgfsetfillcolor{textcolor}%
\pgftext[x=6.539007in,y=4.467412in,,top]{\color{textcolor}\rmfamily\fontsize{10.000000}{12.000000}\selectfont 2017}%
\end{pgfscope}%
\begin{pgfscope}%
\pgfpathrectangle{\pgfqpoint{2.125000in}{4.564634in}}{\pgfqpoint{5.489583in}{0.920732in}}%
\pgfusepath{clip}%
\pgfsetroundcap%
\pgfsetroundjoin%
\pgfsetlinewidth{0.803000pt}%
\definecolor{currentstroke}{rgb}{1.000000,1.000000,1.000000}%
\pgfsetstrokecolor{currentstroke}%
\pgfsetdash{}{0pt}%
\pgfpathmoveto{\pgfqpoint{7.371903in}{4.564634in}}%
\pgfpathlineto{\pgfqpoint{7.371903in}{5.485366in}}%
\pgfusepath{stroke}%
\end{pgfscope}%
\begin{pgfscope}%
\definecolor{textcolor}{rgb}{0.150000,0.150000,0.150000}%
\pgfsetstrokecolor{textcolor}%
\pgfsetfillcolor{textcolor}%
\pgftext[x=7.371903in,y=4.467412in,,top]{\color{textcolor}\rmfamily\fontsize{10.000000}{12.000000}\selectfont 2018}%
\end{pgfscope}%
\begin{pgfscope}%
\pgfpathrectangle{\pgfqpoint{2.125000in}{4.564634in}}{\pgfqpoint{5.489583in}{0.920732in}}%
\pgfusepath{clip}%
\pgfsetroundcap%
\pgfsetroundjoin%
\pgfsetlinewidth{0.803000pt}%
\definecolor{currentstroke}{rgb}{1.000000,1.000000,1.000000}%
\pgfsetstrokecolor{currentstroke}%
\pgfsetdash{}{0pt}%
\pgfpathmoveto{\pgfqpoint{2.125000in}{4.599912in}}%
\pgfpathlineto{\pgfqpoint{7.614583in}{4.599912in}}%
\pgfusepath{stroke}%
\end{pgfscope}%
\begin{pgfscope}%
\definecolor{textcolor}{rgb}{0.150000,0.150000,0.150000}%
\pgfsetstrokecolor{textcolor}%
\pgfsetfillcolor{textcolor}%
\pgftext[x=1.851047in,y=4.547150in,left,base]{\color{textcolor}\rmfamily\fontsize{10.000000}{12.000000}\selectfont 50}%
\end{pgfscope}%
\begin{pgfscope}%
\pgfpathrectangle{\pgfqpoint{2.125000in}{4.564634in}}{\pgfqpoint{5.489583in}{0.920732in}}%
\pgfusepath{clip}%
\pgfsetroundcap%
\pgfsetroundjoin%
\pgfsetlinewidth{0.803000pt}%
\definecolor{currentstroke}{rgb}{1.000000,1.000000,1.000000}%
\pgfsetstrokecolor{currentstroke}%
\pgfsetdash{}{0pt}%
\pgfpathmoveto{\pgfqpoint{2.125000in}{5.083296in}}%
\pgfpathlineto{\pgfqpoint{7.614583in}{5.083296in}}%
\pgfusepath{stroke}%
\end{pgfscope}%
\begin{pgfscope}%
\definecolor{textcolor}{rgb}{0.150000,0.150000,0.150000}%
\pgfsetstrokecolor{textcolor}%
\pgfsetfillcolor{textcolor}%
\pgftext[x=1.762682in,y=5.030535in,left,base]{\color{textcolor}\rmfamily\fontsize{10.000000}{12.000000}\selectfont 100}%
\end{pgfscope}%
\begin{pgfscope}%
\pgfpathrectangle{\pgfqpoint{2.125000in}{4.564634in}}{\pgfqpoint{5.489583in}{0.920732in}}%
\pgfusepath{clip}%
\pgfsetroundcap%
\pgfsetroundjoin%
\pgfsetlinewidth{1.505625pt}%
\definecolor{currentstroke}{rgb}{0.121569,0.466667,0.705882}%
\pgfsetstrokecolor{currentstroke}%
\pgfsetdash{}{0pt}%
\pgfpathmoveto{\pgfqpoint{2.374527in}{4.629398in}}%
\pgfpathlineto{\pgfqpoint{2.376808in}{4.626304in}}%
\pgfpathlineto{\pgfqpoint{2.379090in}{4.625724in}}%
\pgfpathlineto{\pgfqpoint{2.381372in}{4.621277in}}%
\pgfpathlineto{\pgfqpoint{2.388218in}{4.622051in}}%
\pgfpathlineto{\pgfqpoint{2.390500in}{4.624177in}}%
\pgfpathlineto{\pgfqpoint{2.392782in}{4.623597in}}%
\pgfpathlineto{\pgfqpoint{2.397346in}{4.624564in}}%
\pgfpathlineto{\pgfqpoint{2.406473in}{4.623501in}}%
\pgfpathlineto{\pgfqpoint{2.408755in}{4.624758in}}%
\pgfpathlineto{\pgfqpoint{2.411037in}{4.624081in}}%
\pgfpathlineto{\pgfqpoint{2.413319in}{4.624661in}}%
\pgfpathlineto{\pgfqpoint{2.420165in}{4.622631in}}%
\pgfpathlineto{\pgfqpoint{2.422447in}{4.622631in}}%
\pgfpathlineto{\pgfqpoint{2.424728in}{4.624274in}}%
\pgfpathlineto{\pgfqpoint{2.427010in}{4.628045in}}%
\pgfpathlineto{\pgfqpoint{2.429292in}{4.626981in}}%
\pgfpathlineto{\pgfqpoint{2.436138in}{4.628141in}}%
\pgfpathlineto{\pgfqpoint{2.438420in}{4.629688in}}%
\pgfpathlineto{\pgfqpoint{2.440702in}{4.627948in}}%
\pgfpathlineto{\pgfqpoint{2.442984in}{4.627174in}}%
\pgfpathlineto{\pgfqpoint{2.445266in}{4.627561in}}%
\pgfpathlineto{\pgfqpoint{2.452111in}{4.624081in}}%
\pgfpathlineto{\pgfqpoint{2.454393in}{4.624564in}}%
\pgfpathlineto{\pgfqpoint{2.456675in}{4.624467in}}%
\pgfpathlineto{\pgfqpoint{2.461239in}{4.619440in}}%
\pgfpathlineto{\pgfqpoint{2.468085in}{4.620117in}}%
\pgfpathlineto{\pgfqpoint{2.470367in}{4.619537in}}%
\pgfpathlineto{\pgfqpoint{2.472649in}{4.619827in}}%
\pgfpathlineto{\pgfqpoint{2.474930in}{4.621954in}}%
\pgfpathlineto{\pgfqpoint{2.477212in}{4.622534in}}%
\pgfpathlineto{\pgfqpoint{2.493186in}{4.622824in}}%
\pgfpathlineto{\pgfqpoint{2.500031in}{4.622727in}}%
\pgfpathlineto{\pgfqpoint{2.502313in}{4.628431in}}%
\pgfpathlineto{\pgfqpoint{2.504595in}{4.627658in}}%
\pgfpathlineto{\pgfqpoint{2.506877in}{4.625724in}}%
\pgfpathlineto{\pgfqpoint{2.509159in}{4.625241in}}%
\pgfpathlineto{\pgfqpoint{2.516005in}{4.626304in}}%
\pgfpathlineto{\pgfqpoint{2.518287in}{4.621954in}}%
\pgfpathlineto{\pgfqpoint{2.520569in}{4.621567in}}%
\pgfpathlineto{\pgfqpoint{2.522850in}{4.625918in}}%
\pgfpathlineto{\pgfqpoint{2.525132in}{4.625048in}}%
\pgfpathlineto{\pgfqpoint{2.531978in}{4.627851in}}%
\pgfpathlineto{\pgfqpoint{2.534260in}{4.629688in}}%
\pgfpathlineto{\pgfqpoint{2.536542in}{4.627658in}}%
\pgfpathlineto{\pgfqpoint{2.538824in}{4.627561in}}%
\pgfpathlineto{\pgfqpoint{2.541106in}{4.628045in}}%
\pgfpathlineto{\pgfqpoint{2.547951in}{4.628721in}}%
\pgfpathlineto{\pgfqpoint{2.554797in}{4.622824in}}%
\pgfpathlineto{\pgfqpoint{2.557079in}{4.623501in}}%
\pgfpathlineto{\pgfqpoint{2.568489in}{4.631912in}}%
\pgfpathlineto{\pgfqpoint{2.570771in}{4.631332in}}%
\pgfpathlineto{\pgfqpoint{2.573052in}{4.634619in}}%
\pgfpathlineto{\pgfqpoint{2.579898in}{4.636552in}}%
\pgfpathlineto{\pgfqpoint{2.582180in}{4.633942in}}%
\pgfpathlineto{\pgfqpoint{2.584462in}{4.630075in}}%
\pgfpathlineto{\pgfqpoint{2.586744in}{4.629688in}}%
\pgfpathlineto{\pgfqpoint{2.595871in}{4.626401in}}%
\pgfpathlineto{\pgfqpoint{2.598153in}{4.620794in}}%
\pgfpathlineto{\pgfqpoint{2.600435in}{4.620214in}}%
\pgfpathlineto{\pgfqpoint{2.602717in}{4.620407in}}%
\pgfpathlineto{\pgfqpoint{2.604999in}{4.615573in}}%
\pgfpathlineto{\pgfqpoint{2.611845in}{4.619054in}}%
\pgfpathlineto{\pgfqpoint{2.614127in}{4.620890in}}%
\pgfpathlineto{\pgfqpoint{2.616409in}{4.613350in}}%
\pgfpathlineto{\pgfqpoint{2.618691in}{4.611609in}}%
\pgfpathlineto{\pgfqpoint{2.620972in}{4.616927in}}%
\pgfpathlineto{\pgfqpoint{2.627818in}{4.614220in}}%
\pgfpathlineto{\pgfqpoint{2.630100in}{4.617410in}}%
\pgfpathlineto{\pgfqpoint{2.632382in}{4.622534in}}%
\pgfpathlineto{\pgfqpoint{2.634664in}{4.625048in}}%
\pgfpathlineto{\pgfqpoint{2.648355in}{4.629688in}}%
\pgfpathlineto{\pgfqpoint{2.650637in}{4.629688in}}%
\pgfpathlineto{\pgfqpoint{2.652919in}{4.625048in}}%
\pgfpathlineto{\pgfqpoint{2.659765in}{4.625531in}}%
\pgfpathlineto{\pgfqpoint{2.662047in}{4.626884in}}%
\pgfpathlineto{\pgfqpoint{2.664329in}{4.621374in}}%
\pgfpathlineto{\pgfqpoint{2.666611in}{4.623694in}}%
\pgfpathlineto{\pgfqpoint{2.668893in}{4.621857in}}%
\pgfpathlineto{\pgfqpoint{2.675738in}{4.618667in}}%
\pgfpathlineto{\pgfqpoint{2.678020in}{4.616153in}}%
\pgfpathlineto{\pgfqpoint{2.680302in}{4.616927in}}%
\pgfpathlineto{\pgfqpoint{2.684866in}{4.614123in}}%
\pgfpathlineto{\pgfqpoint{2.693993in}{4.615477in}}%
\pgfpathlineto{\pgfqpoint{2.696275in}{4.613446in}}%
\pgfpathlineto{\pgfqpoint{2.698557in}{4.616927in}}%
\pgfpathlineto{\pgfqpoint{2.700839in}{4.612286in}}%
\pgfpathlineto{\pgfqpoint{2.709967in}{4.612963in}}%
\pgfpathlineto{\pgfqpoint{2.712249in}{4.609869in}}%
\pgfpathlineto{\pgfqpoint{2.714531in}{4.611609in}}%
\pgfpathlineto{\pgfqpoint{2.716813in}{4.606486in}}%
\pgfpathlineto{\pgfqpoint{2.723658in}{4.610933in}}%
\pgfpathlineto{\pgfqpoint{2.725940in}{4.609869in}}%
\pgfpathlineto{\pgfqpoint{2.728222in}{4.614606in}}%
\pgfpathlineto{\pgfqpoint{2.730504in}{4.614606in}}%
\pgfpathlineto{\pgfqpoint{2.732786in}{4.615960in}}%
\pgfpathlineto{\pgfqpoint{2.739632in}{4.609193in}}%
\pgfpathlineto{\pgfqpoint{2.741914in}{4.616830in}}%
\pgfpathlineto{\pgfqpoint{2.744195in}{4.627658in}}%
\pgfpathlineto{\pgfqpoint{2.746477in}{4.635585in}}%
\pgfpathlineto{\pgfqpoint{2.748759in}{4.640032in}}%
\pgfpathlineto{\pgfqpoint{2.755605in}{4.642353in}}%
\pgfpathlineto{\pgfqpoint{2.757887in}{4.645640in}}%
\pgfpathlineto{\pgfqpoint{2.760169in}{4.647863in}}%
\pgfpathlineto{\pgfqpoint{2.762451in}{4.643029in}}%
\pgfpathlineto{\pgfqpoint{2.764733in}{4.644963in}}%
\pgfpathlineto{\pgfqpoint{2.773860in}{4.643319in}}%
\pgfpathlineto{\pgfqpoint{2.776142in}{4.646607in}}%
\pgfpathlineto{\pgfqpoint{2.778424in}{4.647380in}}%
\pgfpathlineto{\pgfqpoint{2.780706in}{4.652310in}}%
\pgfpathlineto{\pgfqpoint{2.787552in}{4.655791in}}%
\pgfpathlineto{\pgfqpoint{2.789834in}{4.656178in}}%
\pgfpathlineto{\pgfqpoint{2.796679in}{4.652987in}}%
\pgfpathlineto{\pgfqpoint{2.808089in}{4.655017in}}%
\pgfpathlineto{\pgfqpoint{2.810371in}{4.653567in}}%
\pgfpathlineto{\pgfqpoint{2.812653in}{4.660625in}}%
\pgfpathlineto{\pgfqpoint{2.819498in}{4.659368in}}%
\pgfpathlineto{\pgfqpoint{2.821780in}{4.663718in}}%
\pgfpathlineto{\pgfqpoint{2.824062in}{4.666715in}}%
\pgfpathlineto{\pgfqpoint{2.826344in}{4.667972in}}%
\pgfpathlineto{\pgfqpoint{2.828626in}{4.660818in}}%
\pgfpathlineto{\pgfqpoint{2.835472in}{4.656661in}}%
\pgfpathlineto{\pgfqpoint{2.837754in}{4.650667in}}%
\pgfpathlineto{\pgfqpoint{2.840036in}{4.652117in}}%
\pgfpathlineto{\pgfqpoint{2.842317in}{4.661688in}}%
\pgfpathlineto{\pgfqpoint{2.844599in}{4.667875in}}%
\pgfpathlineto{\pgfqpoint{2.851445in}{4.667295in}}%
\pgfpathlineto{\pgfqpoint{2.853727in}{4.665459in}}%
\pgfpathlineto{\pgfqpoint{2.856009in}{4.666812in}}%
\pgfpathlineto{\pgfqpoint{2.858291in}{4.659368in}}%
\pgfpathlineto{\pgfqpoint{2.860573in}{4.664685in}}%
\pgfpathlineto{\pgfqpoint{2.867418in}{4.662462in}}%
\pgfpathlineto{\pgfqpoint{2.869700in}{4.658111in}}%
\pgfpathlineto{\pgfqpoint{2.871982in}{4.658594in}}%
\pgfpathlineto{\pgfqpoint{2.874264in}{4.658401in}}%
\pgfpathlineto{\pgfqpoint{2.876546in}{4.660915in}}%
\pgfpathlineto{\pgfqpoint{2.883392in}{4.659465in}}%
\pgfpathlineto{\pgfqpoint{2.885674in}{4.660915in}}%
\pgfpathlineto{\pgfqpoint{2.887956in}{4.658594in}}%
\pgfpathlineto{\pgfqpoint{2.890237in}{4.657434in}}%
\pgfpathlineto{\pgfqpoint{2.892519in}{4.654244in}}%
\pgfpathlineto{\pgfqpoint{2.899365in}{4.653471in}}%
\pgfpathlineto{\pgfqpoint{2.901647in}{4.654051in}}%
\pgfpathlineto{\pgfqpoint{2.906211in}{4.653761in}}%
\pgfpathlineto{\pgfqpoint{2.908493in}{4.657531in}}%
\pgfpathlineto{\pgfqpoint{2.919902in}{4.655694in}}%
\pgfpathlineto{\pgfqpoint{2.922184in}{4.654437in}}%
\pgfpathlineto{\pgfqpoint{2.924466in}{4.656178in}}%
\pgfpathlineto{\pgfqpoint{2.935876in}{4.654921in}}%
\pgfpathlineto{\pgfqpoint{2.938158in}{4.659465in}}%
\pgfpathlineto{\pgfqpoint{2.940439in}{4.659755in}}%
\pgfpathlineto{\pgfqpoint{2.947285in}{4.662171in}}%
\pgfpathlineto{\pgfqpoint{2.949567in}{4.662365in}}%
\pgfpathlineto{\pgfqpoint{2.951849in}{4.661881in}}%
\pgfpathlineto{\pgfqpoint{2.954131in}{4.668649in}}%
\pgfpathlineto{\pgfqpoint{2.956413in}{4.664492in}}%
\pgfpathlineto{\pgfqpoint{2.963258in}{4.662752in}}%
\pgfpathlineto{\pgfqpoint{2.965540in}{4.665168in}}%
\pgfpathlineto{\pgfqpoint{2.967822in}{4.665555in}}%
\pgfpathlineto{\pgfqpoint{2.970104in}{4.667875in}}%
\pgfpathlineto{\pgfqpoint{2.972386in}{4.669229in}}%
\pgfpathlineto{\pgfqpoint{2.979232in}{4.668746in}}%
\pgfpathlineto{\pgfqpoint{2.981514in}{4.671259in}}%
\pgfpathlineto{\pgfqpoint{2.983796in}{4.668746in}}%
\pgfpathlineto{\pgfqpoint{2.986078in}{4.668842in}}%
\pgfpathlineto{\pgfqpoint{2.988359in}{4.667972in}}%
\pgfpathlineto{\pgfqpoint{2.995205in}{4.669809in}}%
\pgfpathlineto{\pgfqpoint{2.997487in}{4.668359in}}%
\pgfpathlineto{\pgfqpoint{2.999769in}{4.668746in}}%
\pgfpathlineto{\pgfqpoint{3.002051in}{4.670872in}}%
\pgfpathlineto{\pgfqpoint{3.004333in}{4.673966in}}%
\pgfpathlineto{\pgfqpoint{3.011179in}{4.672226in}}%
\pgfpathlineto{\pgfqpoint{3.013460in}{4.664008in}}%
\pgfpathlineto{\pgfqpoint{3.018024in}{4.660528in}}%
\pgfpathlineto{\pgfqpoint{3.020306in}{4.660528in}}%
\pgfpathlineto{\pgfqpoint{3.027152in}{4.665555in}}%
\pgfpathlineto{\pgfqpoint{3.029434in}{4.673096in}}%
\pgfpathlineto{\pgfqpoint{3.033998in}{4.696879in}}%
\pgfpathlineto{\pgfqpoint{3.036280in}{4.691658in}}%
\pgfpathlineto{\pgfqpoint{3.043125in}{4.690981in}}%
\pgfpathlineto{\pgfqpoint{3.045407in}{4.683827in}}%
\pgfpathlineto{\pgfqpoint{3.047689in}{4.682667in}}%
\pgfpathlineto{\pgfqpoint{3.049971in}{4.685857in}}%
\pgfpathlineto{\pgfqpoint{3.052253in}{4.683924in}}%
\pgfpathlineto{\pgfqpoint{3.063662in}{4.683247in}}%
\pgfpathlineto{\pgfqpoint{3.065944in}{4.688758in}}%
\pgfpathlineto{\pgfqpoint{3.068226in}{4.683924in}}%
\pgfpathlineto{\pgfqpoint{3.075072in}{4.683054in}}%
\pgfpathlineto{\pgfqpoint{3.077354in}{4.684794in}}%
\pgfpathlineto{\pgfqpoint{3.081918in}{4.673966in}}%
\pgfpathlineto{\pgfqpoint{3.084200in}{4.675706in}}%
\pgfpathlineto{\pgfqpoint{3.091045in}{4.674159in}}%
\pgfpathlineto{\pgfqpoint{3.095609in}{4.670872in}}%
\pgfpathlineto{\pgfqpoint{3.097891in}{4.669326in}}%
\pgfpathlineto{\pgfqpoint{3.100173in}{4.670292in}}%
\pgfpathlineto{\pgfqpoint{3.107019in}{4.670679in}}%
\pgfpathlineto{\pgfqpoint{3.109301in}{4.674063in}}%
\pgfpathlineto{\pgfqpoint{3.111582in}{4.673483in}}%
\pgfpathlineto{\pgfqpoint{3.116146in}{4.678123in}}%
\pgfpathlineto{\pgfqpoint{3.122992in}{4.674353in}}%
\pgfpathlineto{\pgfqpoint{3.125274in}{4.672033in}}%
\pgfpathlineto{\pgfqpoint{3.127556in}{4.675996in}}%
\pgfpathlineto{\pgfqpoint{3.129838in}{4.675416in}}%
\pgfpathlineto{\pgfqpoint{3.132120in}{4.679477in}}%
\pgfpathlineto{\pgfqpoint{3.138965in}{4.678897in}}%
\pgfpathlineto{\pgfqpoint{3.141247in}{4.680540in}}%
\pgfpathlineto{\pgfqpoint{3.145811in}{4.682087in}}%
\pgfpathlineto{\pgfqpoint{3.148093in}{4.685277in}}%
\pgfpathlineto{\pgfqpoint{3.154939in}{4.686534in}}%
\pgfpathlineto{\pgfqpoint{3.157221in}{4.690595in}}%
\pgfpathlineto{\pgfqpoint{3.164066in}{4.687211in}}%
\pgfpathlineto{\pgfqpoint{3.170912in}{4.689241in}}%
\pgfpathlineto{\pgfqpoint{3.173194in}{4.689338in}}%
\pgfpathlineto{\pgfqpoint{3.175476in}{4.686727in}}%
\pgfpathlineto{\pgfqpoint{3.177758in}{4.687694in}}%
\pgfpathlineto{\pgfqpoint{3.180040in}{4.683827in}}%
\pgfpathlineto{\pgfqpoint{3.186885in}{4.681797in}}%
\pgfpathlineto{\pgfqpoint{3.191449in}{4.683054in}}%
\pgfpathlineto{\pgfqpoint{3.193731in}{4.682377in}}%
\pgfpathlineto{\pgfqpoint{3.196013in}{4.677446in}}%
\pgfpathlineto{\pgfqpoint{3.202859in}{4.682474in}}%
\pgfpathlineto{\pgfqpoint{3.207423in}{4.688468in}}%
\pgfpathlineto{\pgfqpoint{3.209704in}{4.687694in}}%
\pgfpathlineto{\pgfqpoint{3.211986in}{4.694172in}}%
\pgfpathlineto{\pgfqpoint{3.221114in}{4.693108in}}%
\pgfpathlineto{\pgfqpoint{3.223396in}{4.695622in}}%
\pgfpathlineto{\pgfqpoint{3.225678in}{4.699392in}}%
\pgfpathlineto{\pgfqpoint{3.227960in}{4.700649in}}%
\pgfpathlineto{\pgfqpoint{3.234805in}{4.702389in}}%
\pgfpathlineto{\pgfqpoint{3.237087in}{4.700842in}}%
\pgfpathlineto{\pgfqpoint{3.239369in}{4.702486in}}%
\pgfpathlineto{\pgfqpoint{3.243933in}{4.707803in}}%
\pgfpathlineto{\pgfqpoint{3.253061in}{4.703356in}}%
\pgfpathlineto{\pgfqpoint{3.255343in}{4.704709in}}%
\pgfpathlineto{\pgfqpoint{3.257624in}{4.706740in}}%
\pgfpathlineto{\pgfqpoint{3.259906in}{4.713314in}}%
\pgfpathlineto{\pgfqpoint{3.266752in}{4.710897in}}%
\pgfpathlineto{\pgfqpoint{3.269034in}{4.717277in}}%
\pgfpathlineto{\pgfqpoint{3.271316in}{4.714667in}}%
\pgfpathlineto{\pgfqpoint{3.273598in}{4.713314in}}%
\pgfpathlineto{\pgfqpoint{3.275880in}{4.715440in}}%
\pgfpathlineto{\pgfqpoint{3.282725in}{4.714860in}}%
\pgfpathlineto{\pgfqpoint{3.285007in}{4.719308in}}%
\pgfpathlineto{\pgfqpoint{3.287289in}{4.725205in}}%
\pgfpathlineto{\pgfqpoint{3.289571in}{4.722498in}}%
\pgfpathlineto{\pgfqpoint{3.291853in}{4.725882in}}%
\pgfpathlineto{\pgfqpoint{3.298699in}{4.725398in}}%
\pgfpathlineto{\pgfqpoint{3.300981in}{4.728492in}}%
\pgfpathlineto{\pgfqpoint{3.303263in}{4.727332in}}%
\pgfpathlineto{\pgfqpoint{3.305545in}{4.728589in}}%
\pgfpathlineto{\pgfqpoint{3.307826in}{4.731392in}}%
\pgfpathlineto{\pgfqpoint{3.316954in}{4.737870in}}%
\pgfpathlineto{\pgfqpoint{3.319236in}{4.735356in}}%
\pgfpathlineto{\pgfqpoint{3.321518in}{4.737193in}}%
\pgfpathlineto{\pgfqpoint{3.323800in}{4.737096in}}%
\pgfpathlineto{\pgfqpoint{3.330645in}{4.731489in}}%
\pgfpathlineto{\pgfqpoint{3.332927in}{4.733036in}}%
\pgfpathlineto{\pgfqpoint{3.335209in}{4.737676in}}%
\pgfpathlineto{\pgfqpoint{3.337491in}{4.735936in}}%
\pgfpathlineto{\pgfqpoint{3.339773in}{4.740770in}}%
\pgfpathlineto{\pgfqpoint{3.346619in}{4.744830in}}%
\pgfpathlineto{\pgfqpoint{3.348901in}{4.748504in}}%
\pgfpathlineto{\pgfqpoint{3.351183in}{4.746377in}}%
\pgfpathlineto{\pgfqpoint{3.355746in}{4.752854in}}%
\pgfpathlineto{\pgfqpoint{3.367156in}{4.755755in}}%
\pgfpathlineto{\pgfqpoint{3.369438in}{4.760299in}}%
\pgfpathlineto{\pgfqpoint{3.371720in}{4.760975in}}%
\pgfpathlineto{\pgfqpoint{3.378566in}{4.757882in}}%
\pgfpathlineto{\pgfqpoint{3.380847in}{4.758268in}}%
\pgfpathlineto{\pgfqpoint{3.383129in}{4.763102in}}%
\pgfpathlineto{\pgfqpoint{3.385411in}{4.759525in}}%
\pgfpathlineto{\pgfqpoint{3.387693in}{4.765422in}}%
\pgfpathlineto{\pgfqpoint{3.394539in}{4.764939in}}%
\pgfpathlineto{\pgfqpoint{3.396821in}{4.774510in}}%
\pgfpathlineto{\pgfqpoint{3.399103in}{4.777894in}}%
\pgfpathlineto{\pgfqpoint{3.401385in}{4.780021in}}%
\pgfpathlineto{\pgfqpoint{3.410512in}{4.783308in}}%
\pgfpathlineto{\pgfqpoint{3.412794in}{4.789495in}}%
\pgfpathlineto{\pgfqpoint{3.415076in}{4.784468in}}%
\pgfpathlineto{\pgfqpoint{3.417358in}{4.787175in}}%
\pgfpathlineto{\pgfqpoint{3.419640in}{4.784178in}}%
\pgfpathlineto{\pgfqpoint{3.426486in}{4.776637in}}%
\pgfpathlineto{\pgfqpoint{3.435613in}{4.789882in}}%
\pgfpathlineto{\pgfqpoint{3.442459in}{4.781471in}}%
\pgfpathlineto{\pgfqpoint{3.444741in}{4.795586in}}%
\pgfpathlineto{\pgfqpoint{3.447023in}{4.799356in}}%
\pgfpathlineto{\pgfqpoint{3.449305in}{4.793459in}}%
\pgfpathlineto{\pgfqpoint{3.451587in}{4.804093in}}%
\pgfpathlineto{\pgfqpoint{3.458432in}{4.806897in}}%
\pgfpathlineto{\pgfqpoint{3.460714in}{4.811924in}}%
\pgfpathlineto{\pgfqpoint{3.462996in}{4.803320in}}%
\pgfpathlineto{\pgfqpoint{3.465278in}{4.810087in}}%
\pgfpathlineto{\pgfqpoint{3.467560in}{4.809217in}}%
\pgfpathlineto{\pgfqpoint{3.474406in}{4.812988in}}%
\pgfpathlineto{\pgfqpoint{3.476688in}{4.810184in}}%
\pgfpathlineto{\pgfqpoint{3.478969in}{4.801966in}}%
\pgfpathlineto{\pgfqpoint{3.481251in}{4.809604in}}%
\pgfpathlineto{\pgfqpoint{3.483533in}{4.814341in}}%
\pgfpathlineto{\pgfqpoint{3.490379in}{4.805640in}}%
\pgfpathlineto{\pgfqpoint{3.492661in}{4.812601in}}%
\pgfpathlineto{\pgfqpoint{3.494943in}{4.812021in}}%
\pgfpathlineto{\pgfqpoint{3.497225in}{4.809507in}}%
\pgfpathlineto{\pgfqpoint{3.499507in}{4.814438in}}%
\pgfpathlineto{\pgfqpoint{3.506352in}{4.815211in}}%
\pgfpathlineto{\pgfqpoint{3.508634in}{4.824299in}}%
\pgfpathlineto{\pgfqpoint{3.510916in}{4.829713in}}%
\pgfpathlineto{\pgfqpoint{3.513198in}{4.828166in}}%
\pgfpathlineto{\pgfqpoint{3.515480in}{4.833386in}}%
\pgfpathlineto{\pgfqpoint{3.522326in}{4.832806in}}%
\pgfpathlineto{\pgfqpoint{3.524608in}{4.837447in}}%
\pgfpathlineto{\pgfqpoint{3.526889in}{4.836480in}}%
\pgfpathlineto{\pgfqpoint{3.529171in}{4.831549in}}%
\pgfpathlineto{\pgfqpoint{3.531453in}{4.828359in}}%
\pgfpathlineto{\pgfqpoint{3.540581in}{4.834836in}}%
\pgfpathlineto{\pgfqpoint{3.542863in}{4.818788in}}%
\pgfpathlineto{\pgfqpoint{3.545145in}{4.821592in}}%
\pgfpathlineto{\pgfqpoint{3.547427in}{4.806704in}}%
\pgfpathlineto{\pgfqpoint{3.554272in}{4.811054in}}%
\pgfpathlineto{\pgfqpoint{3.556554in}{4.806123in}}%
\pgfpathlineto{\pgfqpoint{3.558836in}{4.802643in}}%
\pgfpathlineto{\pgfqpoint{3.561118in}{4.809024in}}%
\pgfpathlineto{\pgfqpoint{3.563400in}{4.812697in}}%
\pgfpathlineto{\pgfqpoint{3.570246in}{4.814631in}}%
\pgfpathlineto{\pgfqpoint{3.572528in}{4.810957in}}%
\pgfpathlineto{\pgfqpoint{3.574810in}{4.803126in}}%
\pgfpathlineto{\pgfqpoint{3.577091in}{4.812697in}}%
\pgfpathlineto{\pgfqpoint{3.579373in}{4.812697in}}%
\pgfpathlineto{\pgfqpoint{3.586219in}{4.818595in}}%
\pgfpathlineto{\pgfqpoint{3.588501in}{4.824589in}}%
\pgfpathlineto{\pgfqpoint{3.590783in}{4.812311in}}%
\pgfpathlineto{\pgfqpoint{3.593065in}{4.793942in}}%
\pgfpathlineto{\pgfqpoint{3.595347in}{4.798679in}}%
\pgfpathlineto{\pgfqpoint{3.602192in}{4.810281in}}%
\pgfpathlineto{\pgfqpoint{3.604474in}{4.816371in}}%
\pgfpathlineto{\pgfqpoint{3.606756in}{4.829809in}}%
\pgfpathlineto{\pgfqpoint{3.609038in}{4.827489in}}%
\pgfpathlineto{\pgfqpoint{3.611320in}{4.820528in}}%
\pgfpathlineto{\pgfqpoint{3.618166in}{4.826812in}}%
\pgfpathlineto{\pgfqpoint{3.620448in}{4.826329in}}%
\pgfpathlineto{\pgfqpoint{3.622730in}{4.828069in}}%
\pgfpathlineto{\pgfqpoint{3.627293in}{4.836963in}}%
\pgfpathlineto{\pgfqpoint{3.636421in}{4.845278in}}%
\pgfpathlineto{\pgfqpoint{3.643267in}{4.854365in}}%
\pgfpathlineto{\pgfqpoint{3.650112in}{4.857749in}}%
\pgfpathlineto{\pgfqpoint{3.652394in}{4.857749in}}%
\pgfpathlineto{\pgfqpoint{3.654676in}{4.855525in}}%
\pgfpathlineto{\pgfqpoint{3.656958in}{4.855912in}}%
\pgfpathlineto{\pgfqpoint{3.659240in}{4.872734in}}%
\pgfpathlineto{\pgfqpoint{3.666086in}{4.873121in}}%
\pgfpathlineto{\pgfqpoint{3.668368in}{4.874281in}}%
\pgfpathlineto{\pgfqpoint{3.670650in}{4.873797in}}%
\pgfpathlineto{\pgfqpoint{3.675213in}{4.877664in}}%
\pgfpathlineto{\pgfqpoint{3.682059in}{4.880758in}}%
\pgfpathlineto{\pgfqpoint{3.684341in}{4.880468in}}%
\pgfpathlineto{\pgfqpoint{3.688905in}{4.885399in}}%
\pgfpathlineto{\pgfqpoint{3.691187in}{4.890426in}}%
\pgfpathlineto{\pgfqpoint{3.698032in}{4.885592in}}%
\pgfpathlineto{\pgfqpoint{3.700314in}{4.885689in}}%
\pgfpathlineto{\pgfqpoint{3.702596in}{4.884528in}}%
\pgfpathlineto{\pgfqpoint{3.704878in}{4.881821in}}%
\pgfpathlineto{\pgfqpoint{3.707160in}{4.873797in}}%
\pgfpathlineto{\pgfqpoint{3.714006in}{4.870897in}}%
\pgfpathlineto{\pgfqpoint{3.716288in}{4.879115in}}%
\pgfpathlineto{\pgfqpoint{3.718570in}{4.860166in}}%
\pgfpathlineto{\pgfqpoint{3.720852in}{4.850788in}}%
\pgfpathlineto{\pgfqpoint{3.723133in}{4.849338in}}%
\pgfpathlineto{\pgfqpoint{3.729979in}{4.858136in}}%
\pgfpathlineto{\pgfqpoint{3.732261in}{4.852238in}}%
\pgfpathlineto{\pgfqpoint{3.734543in}{4.844601in}}%
\pgfpathlineto{\pgfqpoint{3.736825in}{4.840250in}}%
\pgfpathlineto{\pgfqpoint{3.739107in}{4.846921in}}%
\pgfpathlineto{\pgfqpoint{3.745953in}{4.839670in}}%
\pgfpathlineto{\pgfqpoint{3.748234in}{4.828359in}}%
\pgfpathlineto{\pgfqpoint{3.750516in}{4.831356in}}%
\pgfpathlineto{\pgfqpoint{3.752798in}{4.831743in}}%
\pgfpathlineto{\pgfqpoint{3.755080in}{4.830389in}}%
\pgfpathlineto{\pgfqpoint{3.764208in}{4.830486in}}%
\pgfpathlineto{\pgfqpoint{3.766490in}{4.834450in}}%
\pgfpathlineto{\pgfqpoint{3.777899in}{4.839864in}}%
\pgfpathlineto{\pgfqpoint{3.780181in}{4.847888in}}%
\pgfpathlineto{\pgfqpoint{3.782463in}{4.853688in}}%
\pgfpathlineto{\pgfqpoint{3.784745in}{4.851852in}}%
\pgfpathlineto{\pgfqpoint{3.787027in}{4.848178in}}%
\pgfpathlineto{\pgfqpoint{3.793873in}{4.852045in}}%
\pgfpathlineto{\pgfqpoint{3.796154in}{4.852238in}}%
\pgfpathlineto{\pgfqpoint{3.798436in}{4.859296in}}%
\pgfpathlineto{\pgfqpoint{3.800718in}{4.860649in}}%
\pgfpathlineto{\pgfqpoint{3.803000in}{4.857362in}}%
\pgfpathlineto{\pgfqpoint{3.809846in}{4.852528in}}%
\pgfpathlineto{\pgfqpoint{3.812128in}{4.845374in}}%
\pgfpathlineto{\pgfqpoint{3.814410in}{4.835900in}}%
\pgfpathlineto{\pgfqpoint{3.816692in}{4.835803in}}%
\pgfpathlineto{\pgfqpoint{3.818974in}{4.833000in}}%
\pgfpathlineto{\pgfqpoint{3.825819in}{4.832710in}}%
\pgfpathlineto{\pgfqpoint{3.828101in}{4.839090in}}%
\pgfpathlineto{\pgfqpoint{3.830383in}{4.837640in}}%
\pgfpathlineto{\pgfqpoint{3.832665in}{4.831743in}}%
\pgfpathlineto{\pgfqpoint{3.834947in}{4.837833in}}%
\pgfpathlineto{\pgfqpoint{3.841793in}{4.831840in}}%
\pgfpathlineto{\pgfqpoint{3.844075in}{4.823719in}}%
\pgfpathlineto{\pgfqpoint{3.846356in}{4.826619in}}%
\pgfpathlineto{\pgfqpoint{3.850920in}{4.855525in}}%
\pgfpathlineto{\pgfqpoint{3.860048in}{4.859489in}}%
\pgfpathlineto{\pgfqpoint{3.862330in}{4.869157in}}%
\pgfpathlineto{\pgfqpoint{3.864612in}{4.876311in}}%
\pgfpathlineto{\pgfqpoint{3.866894in}{4.873507in}}%
\pgfpathlineto{\pgfqpoint{3.873739in}{4.869930in}}%
\pgfpathlineto{\pgfqpoint{3.876021in}{4.879501in}}%
\pgfpathlineto{\pgfqpoint{3.878303in}{4.877374in}}%
\pgfpathlineto{\pgfqpoint{3.880585in}{4.879405in}}%
\pgfpathlineto{\pgfqpoint{3.882867in}{4.877278in}}%
\pgfpathlineto{\pgfqpoint{3.889713in}{4.879791in}}%
\pgfpathlineto{\pgfqpoint{3.891995in}{4.885979in}}%
\pgfpathlineto{\pgfqpoint{3.894276in}{4.883078in}}%
\pgfpathlineto{\pgfqpoint{3.896558in}{4.881628in}}%
\pgfpathlineto{\pgfqpoint{3.898840in}{4.887912in}}%
\pgfpathlineto{\pgfqpoint{3.905686in}{4.885108in}}%
\pgfpathlineto{\pgfqpoint{3.907968in}{4.883272in}}%
\pgfpathlineto{\pgfqpoint{3.910250in}{4.885108in}}%
\pgfpathlineto{\pgfqpoint{3.912532in}{4.882208in}}%
\pgfpathlineto{\pgfqpoint{3.914814in}{4.893519in}}%
\pgfpathlineto{\pgfqpoint{3.921659in}{4.895453in}}%
\pgfpathlineto{\pgfqpoint{3.923941in}{4.889459in}}%
\pgfpathlineto{\pgfqpoint{3.926223in}{4.887622in}}%
\pgfpathlineto{\pgfqpoint{3.930787in}{4.896323in}}%
\pgfpathlineto{\pgfqpoint{3.937633in}{4.895550in}}%
\pgfpathlineto{\pgfqpoint{3.939915in}{4.900190in}}%
\pgfpathlineto{\pgfqpoint{3.942197in}{4.902607in}}%
\pgfpathlineto{\pgfqpoint{3.944478in}{4.902994in}}%
\pgfpathlineto{\pgfqpoint{3.946760in}{4.908891in}}%
\pgfpathlineto{\pgfqpoint{3.953606in}{4.912081in}}%
\pgfpathlineto{\pgfqpoint{3.955888in}{4.907344in}}%
\pgfpathlineto{\pgfqpoint{3.958170in}{4.906667in}}%
\pgfpathlineto{\pgfqpoint{3.962734in}{4.903960in}}%
\pgfpathlineto{\pgfqpoint{3.969579in}{4.900867in}}%
\pgfpathlineto{\pgfqpoint{3.974143in}{4.895453in}}%
\pgfpathlineto{\pgfqpoint{3.976425in}{4.889942in}}%
\pgfpathlineto{\pgfqpoint{3.978707in}{4.902124in}}%
\pgfpathlineto{\pgfqpoint{3.985553in}{4.902124in}}%
\pgfpathlineto{\pgfqpoint{3.987835in}{4.899707in}}%
\pgfpathlineto{\pgfqpoint{3.990117in}{4.891102in}}%
\pgfpathlineto{\pgfqpoint{3.992398in}{4.874861in}}%
\pgfpathlineto{\pgfqpoint{3.994680in}{4.876408in}}%
\pgfpathlineto{\pgfqpoint{4.001526in}{4.876601in}}%
\pgfpathlineto{\pgfqpoint{4.003808in}{4.870704in}}%
\pgfpathlineto{\pgfqpoint{4.006090in}{4.887139in}}%
\pgfpathlineto{\pgfqpoint{4.008372in}{4.881725in}}%
\pgfpathlineto{\pgfqpoint{4.010654in}{4.882595in}}%
\pgfpathlineto{\pgfqpoint{4.019781in}{4.882402in}}%
\pgfpathlineto{\pgfqpoint{4.024345in}{4.886365in}}%
\pgfpathlineto{\pgfqpoint{4.026627in}{4.884818in}}%
\pgfpathlineto{\pgfqpoint{4.033473in}{4.884335in}}%
\pgfpathlineto{\pgfqpoint{4.035755in}{4.878438in}}%
\pgfpathlineto{\pgfqpoint{4.040319in}{4.873797in}}%
\pgfpathlineto{\pgfqpoint{4.042600in}{4.880565in}}%
\pgfpathlineto{\pgfqpoint{4.049446in}{4.884625in}}%
\pgfpathlineto{\pgfqpoint{4.051728in}{4.900867in}}%
\pgfpathlineto{\pgfqpoint{4.054010in}{4.899803in}}%
\pgfpathlineto{\pgfqpoint{4.056292in}{4.904541in}}%
\pgfpathlineto{\pgfqpoint{4.058574in}{4.904637in}}%
\pgfpathlineto{\pgfqpoint{4.065419in}{4.902704in}}%
\pgfpathlineto{\pgfqpoint{4.067701in}{4.904444in}}%
\pgfpathlineto{\pgfqpoint{4.069983in}{4.905121in}}%
\pgfpathlineto{\pgfqpoint{4.072265in}{4.903864in}}%
\pgfpathlineto{\pgfqpoint{4.074547in}{4.907344in}}%
\pgfpathlineto{\pgfqpoint{4.083675in}{4.898740in}}%
\pgfpathlineto{\pgfqpoint{4.085957in}{4.901157in}}%
\pgfpathlineto{\pgfqpoint{4.088239in}{4.888009in}}%
\pgfpathlineto{\pgfqpoint{4.090520in}{4.870317in}}%
\pgfpathlineto{\pgfqpoint{4.097366in}{4.864710in}}%
\pgfpathlineto{\pgfqpoint{4.099648in}{4.866063in}}%
\pgfpathlineto{\pgfqpoint{4.101930in}{4.856105in}}%
\pgfpathlineto{\pgfqpoint{4.104212in}{4.861036in}}%
\pgfpathlineto{\pgfqpoint{4.106494in}{4.852528in}}%
\pgfpathlineto{\pgfqpoint{4.113340in}{4.838414in}}%
\pgfpathlineto{\pgfqpoint{4.115621in}{4.837060in}}%
\pgfpathlineto{\pgfqpoint{4.117903in}{4.842571in}}%
\pgfpathlineto{\pgfqpoint{4.122467in}{4.865580in}}%
\pgfpathlineto{\pgfqpoint{4.129313in}{4.874087in}}%
\pgfpathlineto{\pgfqpoint{4.131595in}{4.889942in}}%
\pgfpathlineto{\pgfqpoint{4.133877in}{4.885399in}}%
\pgfpathlineto{\pgfqpoint{4.138441in}{4.888202in}}%
\pgfpathlineto{\pgfqpoint{4.147568in}{4.883272in}}%
\pgfpathlineto{\pgfqpoint{4.149850in}{4.878824in}}%
\pgfpathlineto{\pgfqpoint{4.152132in}{4.885012in}}%
\pgfpathlineto{\pgfqpoint{4.154414in}{4.883368in}}%
\pgfpathlineto{\pgfqpoint{4.161260in}{4.879888in}}%
\pgfpathlineto{\pgfqpoint{4.165823in}{4.879888in}}%
\pgfpathlineto{\pgfqpoint{4.168105in}{4.882015in}}%
\pgfpathlineto{\pgfqpoint{4.170387in}{4.888396in}}%
\pgfpathlineto{\pgfqpoint{4.177233in}{4.883658in}}%
\pgfpathlineto{\pgfqpoint{4.179515in}{4.898547in}}%
\pgfpathlineto{\pgfqpoint{4.181797in}{4.892263in}}%
\pgfpathlineto{\pgfqpoint{4.184079in}{4.894776in}}%
\pgfpathlineto{\pgfqpoint{4.186361in}{4.898450in}}%
\pgfpathlineto{\pgfqpoint{4.197770in}{4.900770in}}%
\pgfpathlineto{\pgfqpoint{4.200052in}{4.895743in}}%
\pgfpathlineto{\pgfqpoint{4.202334in}{4.894196in}}%
\pgfpathlineto{\pgfqpoint{4.209180in}{4.903574in}}%
\pgfpathlineto{\pgfqpoint{4.211462in}{4.903574in}}%
\pgfpathlineto{\pgfqpoint{4.213743in}{4.900673in}}%
\pgfpathlineto{\pgfqpoint{4.216025in}{4.905121in}}%
\pgfpathlineto{\pgfqpoint{4.218307in}{4.920299in}}%
\pgfpathlineto{\pgfqpoint{4.225153in}{4.914208in}}%
\pgfpathlineto{\pgfqpoint{4.227435in}{4.932480in}}%
\pgfpathlineto{\pgfqpoint{4.229717in}{4.929677in}}%
\pgfpathlineto{\pgfqpoint{4.241126in}{4.939538in}}%
\pgfpathlineto{\pgfqpoint{4.243408in}{4.937121in}}%
\pgfpathlineto{\pgfqpoint{4.245690in}{4.939538in}}%
\pgfpathlineto{\pgfqpoint{4.247972in}{4.939828in}}%
\pgfpathlineto{\pgfqpoint{4.250254in}{4.941181in}}%
\pgfpathlineto{\pgfqpoint{4.257100in}{4.936734in}}%
\pgfpathlineto{\pgfqpoint{4.259382in}{4.938184in}}%
\pgfpathlineto{\pgfqpoint{4.261663in}{4.945628in}}%
\pgfpathlineto{\pgfqpoint{4.263945in}{4.925423in}}%
\pgfpathlineto{\pgfqpoint{4.266227in}{4.928130in}}%
\pgfpathlineto{\pgfqpoint{4.273073in}{4.930450in}}%
\pgfpathlineto{\pgfqpoint{4.275355in}{4.947658in}}%
\pgfpathlineto{\pgfqpoint{4.277637in}{4.943888in}}%
\pgfpathlineto{\pgfqpoint{4.279919in}{4.945725in}}%
\pgfpathlineto{\pgfqpoint{4.291328in}{4.955876in}}%
\pgfpathlineto{\pgfqpoint{4.293610in}{4.956263in}}%
\pgfpathlineto{\pgfqpoint{4.298174in}{4.952589in}}%
\pgfpathlineto{\pgfqpoint{4.305020in}{4.965640in}}%
\pgfpathlineto{\pgfqpoint{4.307302in}{4.963030in}}%
\pgfpathlineto{\pgfqpoint{4.309584in}{4.965157in}}%
\pgfpathlineto{\pgfqpoint{4.311865in}{4.958873in}}%
\pgfpathlineto{\pgfqpoint{4.314147in}{4.948625in}}%
\pgfpathlineto{\pgfqpoint{4.320993in}{4.954426in}}%
\pgfpathlineto{\pgfqpoint{4.323275in}{4.950269in}}%
\pgfpathlineto{\pgfqpoint{4.325557in}{4.962063in}}%
\pgfpathlineto{\pgfqpoint{4.327839in}{4.958583in}}%
\pgfpathlineto{\pgfqpoint{4.330121in}{4.962063in}}%
\pgfpathlineto{\pgfqpoint{4.336966in}{4.958776in}}%
\pgfpathlineto{\pgfqpoint{4.339248in}{4.962933in}}%
\pgfpathlineto{\pgfqpoint{4.346094in}{4.959260in}}%
\pgfpathlineto{\pgfqpoint{4.352940in}{4.959743in}}%
\pgfpathlineto{\pgfqpoint{4.355222in}{4.956456in}}%
\pgfpathlineto{\pgfqpoint{4.357504in}{4.963900in}}%
\pgfpathlineto{\pgfqpoint{4.359785in}{4.968347in}}%
\pgfpathlineto{\pgfqpoint{4.362067in}{4.968541in}}%
\pgfpathlineto{\pgfqpoint{4.371195in}{4.967091in}}%
\pgfpathlineto{\pgfqpoint{4.373477in}{4.962740in}}%
\pgfpathlineto{\pgfqpoint{4.375759in}{4.966607in}}%
\pgfpathlineto{\pgfqpoint{4.378041in}{4.972601in}}%
\pgfpathlineto{\pgfqpoint{4.389450in}{4.982849in}}%
\pgfpathlineto{\pgfqpoint{4.391732in}{4.987393in}}%
\pgfpathlineto{\pgfqpoint{4.394014in}{4.987103in}}%
\pgfpathlineto{\pgfqpoint{4.400860in}{4.987393in}}%
\pgfpathlineto{\pgfqpoint{4.403142in}{4.994837in}}%
\pgfpathlineto{\pgfqpoint{4.407706in}{4.981592in}}%
\pgfpathlineto{\pgfqpoint{4.409987in}{4.981592in}}%
\pgfpathlineto{\pgfqpoint{4.416833in}{4.980915in}}%
\pgfpathlineto{\pgfqpoint{4.419115in}{4.976565in}}%
\pgfpathlineto{\pgfqpoint{4.423679in}{4.992420in}}%
\pgfpathlineto{\pgfqpoint{4.425961in}{5.004698in}}%
\pgfpathlineto{\pgfqpoint{4.435088in}{4.998994in}}%
\pgfpathlineto{\pgfqpoint{4.437370in}{5.008855in}}%
\pgfpathlineto{\pgfqpoint{4.439652in}{5.007888in}}%
\pgfpathlineto{\pgfqpoint{4.441934in}{5.002378in}}%
\pgfpathlineto{\pgfqpoint{4.448780in}{4.999187in}}%
\pgfpathlineto{\pgfqpoint{4.451062in}{5.009725in}}%
\pgfpathlineto{\pgfqpoint{4.453344in}{5.009628in}}%
\pgfpathlineto{\pgfqpoint{4.455626in}{5.005955in}}%
\pgfpathlineto{\pgfqpoint{4.464753in}{5.014849in}}%
\pgfpathlineto{\pgfqpoint{4.467035in}{5.008468in}}%
\pgfpathlineto{\pgfqpoint{4.469317in}{5.011175in}}%
\pgfpathlineto{\pgfqpoint{4.471599in}{5.009145in}}%
\pgfpathlineto{\pgfqpoint{4.473881in}{5.003248in}}%
\pgfpathlineto{\pgfqpoint{4.480727in}{5.005665in}}%
\pgfpathlineto{\pgfqpoint{4.483008in}{4.987876in}}%
\pgfpathlineto{\pgfqpoint{4.485290in}{4.978982in}}%
\pgfpathlineto{\pgfqpoint{4.487572in}{4.963320in}}%
\pgfpathlineto{\pgfqpoint{4.489854in}{4.975405in}}%
\pgfpathlineto{\pgfqpoint{4.496700in}{4.970958in}}%
\pgfpathlineto{\pgfqpoint{4.498982in}{4.981109in}}%
\pgfpathlineto{\pgfqpoint{4.501264in}{4.978692in}}%
\pgfpathlineto{\pgfqpoint{4.503546in}{4.978788in}}%
\pgfpathlineto{\pgfqpoint{4.505828in}{4.978015in}}%
\pgfpathlineto{\pgfqpoint{4.512673in}{4.978015in}}%
\pgfpathlineto{\pgfqpoint{4.514955in}{4.976758in}}%
\pgfpathlineto{\pgfqpoint{4.517237in}{4.979659in}}%
\pgfpathlineto{\pgfqpoint{4.519519in}{4.961000in}}%
\pgfpathlineto{\pgfqpoint{4.521801in}{4.959356in}}%
\pgfpathlineto{\pgfqpoint{4.528647in}{4.961580in}}%
\pgfpathlineto{\pgfqpoint{4.530928in}{4.958680in}}%
\pgfpathlineto{\pgfqpoint{4.533210in}{4.966220in}}%
\pgfpathlineto{\pgfqpoint{4.535492in}{4.959646in}}%
\pgfpathlineto{\pgfqpoint{4.537774in}{4.969314in}}%
\pgfpathlineto{\pgfqpoint{4.544620in}{4.969991in}}%
\pgfpathlineto{\pgfqpoint{4.546902in}{4.965544in}}%
\pgfpathlineto{\pgfqpoint{4.549184in}{4.974921in}}%
\pgfpathlineto{\pgfqpoint{4.551466in}{4.977242in}}%
\pgfpathlineto{\pgfqpoint{4.553748in}{4.970088in}}%
\pgfpathlineto{\pgfqpoint{4.560593in}{4.983042in}}%
\pgfpathlineto{\pgfqpoint{4.565157in}{4.987296in}}%
\pgfpathlineto{\pgfqpoint{4.567439in}{4.995610in}}%
\pgfpathlineto{\pgfqpoint{4.569721in}{4.992227in}}%
\pgfpathlineto{\pgfqpoint{4.576567in}{4.993387in}}%
\pgfpathlineto{\pgfqpoint{4.578849in}{4.995127in}}%
\pgfpathlineto{\pgfqpoint{4.583412in}{4.990970in}}%
\pgfpathlineto{\pgfqpoint{4.585694in}{4.997640in}}%
\pgfpathlineto{\pgfqpoint{4.594822in}{4.994450in}}%
\pgfpathlineto{\pgfqpoint{4.597104in}{4.997834in}}%
\pgfpathlineto{\pgfqpoint{4.599386in}{4.998607in}}%
\pgfpathlineto{\pgfqpoint{4.601668in}{5.003441in}}%
\pgfpathlineto{\pgfqpoint{4.608513in}{5.000347in}}%
\pgfpathlineto{\pgfqpoint{4.610795in}{4.998220in}}%
\pgfpathlineto{\pgfqpoint{4.613077in}{5.008372in}}%
\pgfpathlineto{\pgfqpoint{4.615359in}{5.004601in}}%
\pgfpathlineto{\pgfqpoint{4.624487in}{5.006051in}}%
\pgfpathlineto{\pgfqpoint{4.626769in}{5.015912in}}%
\pgfpathlineto{\pgfqpoint{4.629050in}{5.018523in}}%
\pgfpathlineto{\pgfqpoint{4.631332in}{5.028384in}}%
\pgfpathlineto{\pgfqpoint{4.633614in}{5.033798in}}%
\pgfpathlineto{\pgfqpoint{4.640460in}{5.032831in}}%
\pgfpathlineto{\pgfqpoint{4.642742in}{5.029350in}}%
\pgfpathlineto{\pgfqpoint{4.645024in}{5.039308in}}%
\pgfpathlineto{\pgfqpoint{4.647306in}{5.026257in}}%
\pgfpathlineto{\pgfqpoint{4.649588in}{5.026257in}}%
\pgfpathlineto{\pgfqpoint{4.656433in}{5.021520in}}%
\pgfpathlineto{\pgfqpoint{4.658715in}{5.021906in}}%
\pgfpathlineto{\pgfqpoint{4.660997in}{5.002474in}}%
\pgfpathlineto{\pgfqpoint{4.663279in}{4.998607in}}%
\pgfpathlineto{\pgfqpoint{4.665561in}{5.009532in}}%
\pgfpathlineto{\pgfqpoint{4.672407in}{5.007211in}}%
\pgfpathlineto{\pgfqpoint{4.674689in}{4.986233in}}%
\pgfpathlineto{\pgfqpoint{4.676971in}{5.007695in}}%
\pgfpathlineto{\pgfqpoint{4.679252in}{4.983622in}}%
\pgfpathlineto{\pgfqpoint{4.681534in}{4.976372in}}%
\pgfpathlineto{\pgfqpoint{4.688380in}{4.958486in}}%
\pgfpathlineto{\pgfqpoint{4.690662in}{4.940504in}}%
\pgfpathlineto{\pgfqpoint{4.692944in}{4.950752in}}%
\pgfpathlineto{\pgfqpoint{4.695226in}{4.938571in}}%
\pgfpathlineto{\pgfqpoint{4.697508in}{4.954909in}}%
\pgfpathlineto{\pgfqpoint{4.704353in}{4.959163in}}%
\pgfpathlineto{\pgfqpoint{4.706635in}{4.969024in}}%
\pgfpathlineto{\pgfqpoint{4.708917in}{4.976275in}}%
\pgfpathlineto{\pgfqpoint{4.711199in}{4.988263in}}%
\pgfpathlineto{\pgfqpoint{4.713481in}{4.992517in}}%
\pgfpathlineto{\pgfqpoint{4.720327in}{5.000541in}}%
\pgfpathlineto{\pgfqpoint{4.724891in}{5.013205in}}%
\pgfpathlineto{\pgfqpoint{4.727172in}{5.025773in}}%
\pgfpathlineto{\pgfqpoint{4.729454in}{5.032057in}}%
\pgfpathlineto{\pgfqpoint{4.736300in}{5.029350in}}%
\pgfpathlineto{\pgfqpoint{4.738582in}{5.039115in}}%
\pgfpathlineto{\pgfqpoint{4.743146in}{5.042499in}}%
\pgfpathlineto{\pgfqpoint{4.745428in}{5.035634in}}%
\pgfpathlineto{\pgfqpoint{4.752273in}{5.040855in}}%
\pgfpathlineto{\pgfqpoint{4.754555in}{5.041628in}}%
\pgfpathlineto{\pgfqpoint{4.756837in}{5.040275in}}%
\pgfpathlineto{\pgfqpoint{4.759119in}{5.042982in}}%
\pgfpathlineto{\pgfqpoint{4.761401in}{5.035248in}}%
\pgfpathlineto{\pgfqpoint{4.768247in}{5.036408in}}%
\pgfpathlineto{\pgfqpoint{4.770529in}{5.040952in}}%
\pgfpathlineto{\pgfqpoint{4.772811in}{5.040372in}}%
\pgfpathlineto{\pgfqpoint{4.775093in}{5.035441in}}%
\pgfpathlineto{\pgfqpoint{4.777374in}{5.038631in}}%
\pgfpathlineto{\pgfqpoint{4.784220in}{5.030317in}}%
\pgfpathlineto{\pgfqpoint{4.786502in}{5.028770in}}%
\pgfpathlineto{\pgfqpoint{4.793348in}{5.042015in}}%
\pgfpathlineto{\pgfqpoint{4.800194in}{5.040082in}}%
\pgfpathlineto{\pgfqpoint{4.802475in}{5.044239in}}%
\pgfpathlineto{\pgfqpoint{4.804757in}{5.037471in}}%
\pgfpathlineto{\pgfqpoint{4.807039in}{5.036118in}}%
\pgfpathlineto{\pgfqpoint{4.809321in}{5.044239in}}%
\pgfpathlineto{\pgfqpoint{4.816167in}{5.044335in}}%
\pgfpathlineto{\pgfqpoint{4.818449in}{5.040275in}}%
\pgfpathlineto{\pgfqpoint{4.820731in}{5.024807in}}%
\pgfpathlineto{\pgfqpoint{4.823013in}{5.028964in}}%
\pgfpathlineto{\pgfqpoint{4.825294in}{5.009338in}}%
\pgfpathlineto{\pgfqpoint{4.832140in}{5.005375in}}%
\pgfpathlineto{\pgfqpoint{4.834422in}{4.995030in}}%
\pgfpathlineto{\pgfqpoint{4.836704in}{5.006245in}}%
\pgfpathlineto{\pgfqpoint{4.838986in}{5.029737in}}%
\pgfpathlineto{\pgfqpoint{4.841268in}{5.018909in}}%
\pgfpathlineto{\pgfqpoint{4.848114in}{5.029060in}}%
\pgfpathlineto{\pgfqpoint{4.850395in}{5.008082in}}%
\pgfpathlineto{\pgfqpoint{4.852677in}{5.010692in}}%
\pgfpathlineto{\pgfqpoint{4.857241in}{5.014752in}}%
\pgfpathlineto{\pgfqpoint{4.866369in}{5.017266in}}%
\pgfpathlineto{\pgfqpoint{4.868651in}{5.010498in}}%
\pgfpathlineto{\pgfqpoint{4.873215in}{5.010112in}}%
\pgfpathlineto{\pgfqpoint{4.880060in}{5.003924in}}%
\pgfpathlineto{\pgfqpoint{4.882342in}{4.999477in}}%
\pgfpathlineto{\pgfqpoint{4.884624in}{5.019006in}}%
\pgfpathlineto{\pgfqpoint{4.886906in}{5.026063in}}%
\pgfpathlineto{\pgfqpoint{4.889188in}{5.013689in}}%
\pgfpathlineto{\pgfqpoint{4.896034in}{5.010595in}}%
\pgfpathlineto{\pgfqpoint{4.898316in}{5.012142in}}%
\pgfpathlineto{\pgfqpoint{4.900597in}{5.005665in}}%
\pgfpathlineto{\pgfqpoint{4.902879in}{4.992807in}}%
\pgfpathlineto{\pgfqpoint{4.905161in}{5.006051in}}%
\pgfpathlineto{\pgfqpoint{4.914289in}{4.982462in}}%
\pgfpathlineto{\pgfqpoint{4.916571in}{4.987683in}}%
\pgfpathlineto{\pgfqpoint{4.918853in}{5.003634in}}%
\pgfpathlineto{\pgfqpoint{4.921135in}{4.990293in}}%
\pgfpathlineto{\pgfqpoint{4.927980in}{4.990776in}}%
\pgfpathlineto{\pgfqpoint{4.930262in}{4.989326in}}%
\pgfpathlineto{\pgfqpoint{4.932544in}{4.984106in}}%
\pgfpathlineto{\pgfqpoint{4.934826in}{4.991840in}}%
\pgfpathlineto{\pgfqpoint{4.937108in}{4.972698in}}%
\pgfpathlineto{\pgfqpoint{4.943954in}{4.978595in}}%
\pgfpathlineto{\pgfqpoint{4.946236in}{4.992517in}}%
\pgfpathlineto{\pgfqpoint{4.948517in}{4.983139in}}%
\pgfpathlineto{\pgfqpoint{4.950799in}{4.992517in}}%
\pgfpathlineto{\pgfqpoint{4.953081in}{4.980915in}}%
\pgfpathlineto{\pgfqpoint{4.959927in}{4.969604in}}%
\pgfpathlineto{\pgfqpoint{4.962209in}{4.974438in}}%
\pgfpathlineto{\pgfqpoint{4.964491in}{4.974728in}}%
\pgfpathlineto{\pgfqpoint{4.966773in}{4.958100in}}%
\pgfpathlineto{\pgfqpoint{4.969055in}{4.968251in}}%
\pgfpathlineto{\pgfqpoint{4.978182in}{4.975211in}}%
\pgfpathlineto{\pgfqpoint{4.980464in}{4.971151in}}%
\pgfpathlineto{\pgfqpoint{4.982746in}{4.977435in}}%
\pgfpathlineto{\pgfqpoint{4.985028in}{4.979659in}}%
\pgfpathlineto{\pgfqpoint{4.991874in}{4.978982in}}%
\pgfpathlineto{\pgfqpoint{4.996437in}{4.987876in}}%
\pgfpathlineto{\pgfqpoint{4.998719in}{5.001604in}}%
\pgfpathlineto{\pgfqpoint{5.001001in}{4.999091in}}%
\pgfpathlineto{\pgfqpoint{5.007847in}{5.005181in}}%
\pgfpathlineto{\pgfqpoint{5.012411in}{4.991646in}}%
\pgfpathlineto{\pgfqpoint{5.014693in}{4.999187in}}%
\pgfpathlineto{\pgfqpoint{5.016975in}{4.978402in}}%
\pgfpathlineto{\pgfqpoint{5.023820in}{4.983139in}}%
\pgfpathlineto{\pgfqpoint{5.028384in}{4.963030in}}%
\pgfpathlineto{\pgfqpoint{5.030666in}{4.975985in}}%
\pgfpathlineto{\pgfqpoint{5.032948in}{4.970668in}}%
\pgfpathlineto{\pgfqpoint{5.039794in}{4.986619in}}%
\pgfpathlineto{\pgfqpoint{5.042076in}{4.976468in}}%
\pgfpathlineto{\pgfqpoint{5.044358in}{4.990003in}}%
\pgfpathlineto{\pgfqpoint{5.046639in}{4.992033in}}%
\pgfpathlineto{\pgfqpoint{5.048921in}{4.998124in}}%
\pgfpathlineto{\pgfqpoint{5.055767in}{5.003151in}}%
\pgfpathlineto{\pgfqpoint{5.058049in}{4.994353in}}%
\pgfpathlineto{\pgfqpoint{5.060331in}{4.980432in}}%
\pgfpathlineto{\pgfqpoint{5.062613in}{4.978692in}}%
\pgfpathlineto{\pgfqpoint{5.064895in}{4.980432in}}%
\pgfpathlineto{\pgfqpoint{5.071740in}{4.990776in}}%
\pgfpathlineto{\pgfqpoint{5.074022in}{4.982656in}}%
\pgfpathlineto{\pgfqpoint{5.076304in}{4.970184in}}%
\pgfpathlineto{\pgfqpoint{5.078586in}{4.974341in}}%
\pgfpathlineto{\pgfqpoint{5.087714in}{4.970184in}}%
\pgfpathlineto{\pgfqpoint{5.089996in}{4.978305in}}%
\pgfpathlineto{\pgfqpoint{5.092278in}{4.978788in}}%
\pgfpathlineto{\pgfqpoint{5.094559in}{4.988940in}}%
\pgfpathlineto{\pgfqpoint{5.096841in}{4.995224in}}%
\pgfpathlineto{\pgfqpoint{5.103687in}{4.982172in}}%
\pgfpathlineto{\pgfqpoint{5.105969in}{4.981979in}}%
\pgfpathlineto{\pgfqpoint{5.108251in}{4.982656in}}%
\pgfpathlineto{\pgfqpoint{5.110533in}{4.975695in}}%
\pgfpathlineto{\pgfqpoint{5.112815in}{4.973858in}}%
\pgfpathlineto{\pgfqpoint{5.119660in}{4.979272in}}%
\pgfpathlineto{\pgfqpoint{5.126506in}{4.981399in}}%
\pgfpathlineto{\pgfqpoint{5.128788in}{4.986716in}}%
\pgfpathlineto{\pgfqpoint{5.135634in}{4.982462in}}%
\pgfpathlineto{\pgfqpoint{5.137916in}{4.983816in}}%
\pgfpathlineto{\pgfqpoint{5.140198in}{4.980819in}}%
\pgfpathlineto{\pgfqpoint{5.142480in}{4.970571in}}%
\pgfpathlineto{\pgfqpoint{5.144761in}{4.978595in}}%
\pgfpathlineto{\pgfqpoint{5.151607in}{4.980432in}}%
\pgfpathlineto{\pgfqpoint{5.153889in}{4.973181in}}%
\pgfpathlineto{\pgfqpoint{5.156171in}{4.970281in}}%
\pgfpathlineto{\pgfqpoint{5.158453in}{4.974535in}}%
\pgfpathlineto{\pgfqpoint{5.160735in}{4.990100in}}%
\pgfpathlineto{\pgfqpoint{5.167581in}{4.986329in}}%
\pgfpathlineto{\pgfqpoint{5.169862in}{4.981495in}}%
\pgfpathlineto{\pgfqpoint{5.172144in}{4.982172in}}%
\pgfpathlineto{\pgfqpoint{5.174426in}{4.993193in}}%
\pgfpathlineto{\pgfqpoint{5.176708in}{4.997254in}}%
\pgfpathlineto{\pgfqpoint{5.183554in}{5.006921in}}%
\pgfpathlineto{\pgfqpoint{5.185836in}{5.011562in}}%
\pgfpathlineto{\pgfqpoint{5.190400in}{5.004698in}}%
\pgfpathlineto{\pgfqpoint{5.192681in}{4.995417in}}%
\pgfpathlineto{\pgfqpoint{5.201809in}{4.990873in}}%
\pgfpathlineto{\pgfqpoint{5.204091in}{4.993580in}}%
\pgfpathlineto{\pgfqpoint{5.206373in}{4.993677in}}%
\pgfpathlineto{\pgfqpoint{5.208655in}{4.984976in}}%
\pgfpathlineto{\pgfqpoint{5.217782in}{4.983526in}}%
\pgfpathlineto{\pgfqpoint{5.220064in}{4.984492in}}%
\pgfpathlineto{\pgfqpoint{5.222346in}{4.977048in}}%
\pgfpathlineto{\pgfqpoint{5.224628in}{4.971538in}}%
\pgfpathlineto{\pgfqpoint{5.231474in}{4.966027in}}%
\pgfpathlineto{\pgfqpoint{5.233756in}{4.968251in}}%
\pgfpathlineto{\pgfqpoint{5.236038in}{4.974051in}}%
\pgfpathlineto{\pgfqpoint{5.238320in}{4.977145in}}%
\pgfpathlineto{\pgfqpoint{5.240602in}{4.969411in}}%
\pgfpathlineto{\pgfqpoint{5.247447in}{4.961967in}}%
\pgfpathlineto{\pgfqpoint{5.249729in}{4.969507in}}%
\pgfpathlineto{\pgfqpoint{5.252011in}{4.972698in}}%
\pgfpathlineto{\pgfqpoint{5.254293in}{4.986813in}}%
\pgfpathlineto{\pgfqpoint{5.256575in}{4.982559in}}%
\pgfpathlineto{\pgfqpoint{5.263421in}{4.984492in}}%
\pgfpathlineto{\pgfqpoint{5.265703in}{4.981785in}}%
\pgfpathlineto{\pgfqpoint{5.267984in}{4.977918in}}%
\pgfpathlineto{\pgfqpoint{5.270266in}{4.976081in}}%
\pgfpathlineto{\pgfqpoint{5.272548in}{4.980625in}}%
\pgfpathlineto{\pgfqpoint{5.279394in}{4.963610in}}%
\pgfpathlineto{\pgfqpoint{5.281676in}{4.961677in}}%
\pgfpathlineto{\pgfqpoint{5.283958in}{4.970474in}}%
\pgfpathlineto{\pgfqpoint{5.286240in}{4.970184in}}%
\pgfpathlineto{\pgfqpoint{5.295367in}{4.968154in}}%
\pgfpathlineto{\pgfqpoint{5.297649in}{4.974341in}}%
\pgfpathlineto{\pgfqpoint{5.299931in}{4.964384in}}%
\pgfpathlineto{\pgfqpoint{5.302213in}{4.969894in}}%
\pgfpathlineto{\pgfqpoint{5.304495in}{4.979659in}}%
\pgfpathlineto{\pgfqpoint{5.311341in}{4.986039in}}%
\pgfpathlineto{\pgfqpoint{5.313623in}{4.981785in}}%
\pgfpathlineto{\pgfqpoint{5.318186in}{4.993387in}}%
\pgfpathlineto{\pgfqpoint{5.320468in}{4.984396in}}%
\pgfpathlineto{\pgfqpoint{5.327314in}{4.986909in}}%
\pgfpathlineto{\pgfqpoint{5.329596in}{4.986716in}}%
\pgfpathlineto{\pgfqpoint{5.331878in}{4.985266in}}%
\pgfpathlineto{\pgfqpoint{5.334160in}{4.985459in}}%
\pgfpathlineto{\pgfqpoint{5.336442in}{4.976372in}}%
\pgfpathlineto{\pgfqpoint{5.343287in}{4.968831in}}%
\pgfpathlineto{\pgfqpoint{5.347851in}{4.981399in}}%
\pgfpathlineto{\pgfqpoint{5.350133in}{4.982365in}}%
\pgfpathlineto{\pgfqpoint{5.352415in}{4.985556in}}%
\pgfpathlineto{\pgfqpoint{5.359261in}{4.983912in}}%
\pgfpathlineto{\pgfqpoint{5.361543in}{4.981979in}}%
\pgfpathlineto{\pgfqpoint{5.363825in}{4.988263in}}%
\pgfpathlineto{\pgfqpoint{5.366106in}{4.975695in}}%
\pgfpathlineto{\pgfqpoint{5.368388in}{4.973761in}}%
\pgfpathlineto{\pgfqpoint{5.375234in}{4.981882in}}%
\pgfpathlineto{\pgfqpoint{5.377516in}{4.975115in}}%
\pgfpathlineto{\pgfqpoint{5.382080in}{4.970571in}}%
\pgfpathlineto{\pgfqpoint{5.391207in}{4.982656in}}%
\pgfpathlineto{\pgfqpoint{5.393489in}{4.978305in}}%
\pgfpathlineto{\pgfqpoint{5.395771in}{4.977725in}}%
\pgfpathlineto{\pgfqpoint{5.398053in}{4.973278in}}%
\pgfpathlineto{\pgfqpoint{5.400335in}{4.951526in}}%
\pgfpathlineto{\pgfqpoint{5.407181in}{4.927646in}}%
\pgfpathlineto{\pgfqpoint{5.409463in}{4.909374in}}%
\pgfpathlineto{\pgfqpoint{5.411745in}{4.947658in}}%
\pgfpathlineto{\pgfqpoint{5.414026in}{4.957326in}}%
\pgfpathlineto{\pgfqpoint{5.416308in}{4.948142in}}%
\pgfpathlineto{\pgfqpoint{5.423154in}{4.937797in}}%
\pgfpathlineto{\pgfqpoint{5.425436in}{4.921266in}}%
\pgfpathlineto{\pgfqpoint{5.427718in}{4.932287in}}%
\pgfpathlineto{\pgfqpoint{5.430000in}{4.926100in}}%
\pgfpathlineto{\pgfqpoint{5.432282in}{4.914402in}}%
\pgfpathlineto{\pgfqpoint{5.441409in}{4.937411in}}%
\pgfpathlineto{\pgfqpoint{5.443691in}{4.922426in}}%
\pgfpathlineto{\pgfqpoint{5.445973in}{4.926776in}}%
\pgfpathlineto{\pgfqpoint{5.448255in}{4.928613in}}%
\pgfpathlineto{\pgfqpoint{5.455101in}{4.931997in}}%
\pgfpathlineto{\pgfqpoint{5.457383in}{4.941471in}}%
\pgfpathlineto{\pgfqpoint{5.461946in}{4.945242in}}%
\pgfpathlineto{\pgfqpoint{5.464228in}{4.932577in}}%
\pgfpathlineto{\pgfqpoint{5.471074in}{4.930353in}}%
\pgfpathlineto{\pgfqpoint{5.473356in}{4.931320in}}%
\pgfpathlineto{\pgfqpoint{5.475638in}{4.929096in}}%
\pgfpathlineto{\pgfqpoint{5.477920in}{4.924649in}}%
\pgfpathlineto{\pgfqpoint{5.480202in}{4.911695in}}%
\pgfpathlineto{\pgfqpoint{5.487047in}{4.914982in}}%
\pgfpathlineto{\pgfqpoint{5.489329in}{4.929580in}}%
\pgfpathlineto{\pgfqpoint{5.491611in}{4.932287in}}%
\pgfpathlineto{\pgfqpoint{5.493893in}{4.930643in}}%
\pgfpathlineto{\pgfqpoint{5.496175in}{4.937314in}}%
\pgfpathlineto{\pgfqpoint{5.503021in}{4.944565in}}%
\pgfpathlineto{\pgfqpoint{5.505303in}{4.932867in}}%
\pgfpathlineto{\pgfqpoint{5.507585in}{4.946498in}}%
\pgfpathlineto{\pgfqpoint{5.509867in}{4.947368in}}%
\pgfpathlineto{\pgfqpoint{5.512148in}{4.949882in}}%
\pgfpathlineto{\pgfqpoint{5.518994in}{4.955296in}}%
\pgfpathlineto{\pgfqpoint{5.521276in}{4.950655in}}%
\pgfpathlineto{\pgfqpoint{5.523558in}{4.942535in}}%
\pgfpathlineto{\pgfqpoint{5.525840in}{4.965447in}}%
\pgfpathlineto{\pgfqpoint{5.528122in}{4.975018in}}%
\pgfpathlineto{\pgfqpoint{5.534968in}{4.972408in}}%
\pgfpathlineto{\pgfqpoint{5.537249in}{4.969314in}}%
\pgfpathlineto{\pgfqpoint{5.539531in}{4.969701in}}%
\pgfpathlineto{\pgfqpoint{5.541813in}{4.986233in}}%
\pgfpathlineto{\pgfqpoint{5.544095in}{4.993097in}}%
\pgfpathlineto{\pgfqpoint{5.550941in}{4.989810in}}%
\pgfpathlineto{\pgfqpoint{5.555505in}{4.994547in}}%
\pgfpathlineto{\pgfqpoint{5.557787in}{5.002378in}}%
\pgfpathlineto{\pgfqpoint{5.560068in}{4.999381in}}%
\pgfpathlineto{\pgfqpoint{5.566914in}{5.009338in}}%
\pgfpathlineto{\pgfqpoint{5.569196in}{5.007501in}}%
\pgfpathlineto{\pgfqpoint{5.571478in}{5.007308in}}%
\pgfpathlineto{\pgfqpoint{5.573760in}{5.010692in}}%
\pgfpathlineto{\pgfqpoint{5.576042in}{5.007115in}}%
\pgfpathlineto{\pgfqpoint{5.582888in}{4.997737in}}%
\pgfpathlineto{\pgfqpoint{5.585169in}{5.003054in}}%
\pgfpathlineto{\pgfqpoint{5.587451in}{5.006631in}}%
\pgfpathlineto{\pgfqpoint{5.589733in}{4.992710in}}%
\pgfpathlineto{\pgfqpoint{5.592015in}{4.989326in}}%
\pgfpathlineto{\pgfqpoint{5.598861in}{5.001024in}}%
\pgfpathlineto{\pgfqpoint{5.601143in}{5.003441in}}%
\pgfpathlineto{\pgfqpoint{5.603425in}{5.013689in}}%
\pgfpathlineto{\pgfqpoint{5.605707in}{5.012142in}}%
\pgfpathlineto{\pgfqpoint{5.607989in}{5.018619in}}%
\pgfpathlineto{\pgfqpoint{5.614834in}{5.021230in}}%
\pgfpathlineto{\pgfqpoint{5.617116in}{5.014752in}}%
\pgfpathlineto{\pgfqpoint{5.619398in}{5.014076in}}%
\pgfpathlineto{\pgfqpoint{5.623962in}{5.017653in}}%
\pgfpathlineto{\pgfqpoint{5.630808in}{5.007695in}}%
\pgfpathlineto{\pgfqpoint{5.633090in}{5.017556in}}%
\pgfpathlineto{\pgfqpoint{5.635371in}{5.014849in}}%
\pgfpathlineto{\pgfqpoint{5.637653in}{5.003731in}}%
\pgfpathlineto{\pgfqpoint{5.639935in}{5.022776in}}%
\pgfpathlineto{\pgfqpoint{5.646781in}{5.026063in}}%
\pgfpathlineto{\pgfqpoint{5.649063in}{5.018136in}}%
\pgfpathlineto{\pgfqpoint{5.651345in}{5.015719in}}%
\pgfpathlineto{\pgfqpoint{5.653627in}{5.020069in}}%
\pgfpathlineto{\pgfqpoint{5.655909in}{5.011562in}}%
\pgfpathlineto{\pgfqpoint{5.662754in}{5.015719in}}%
\pgfpathlineto{\pgfqpoint{5.665036in}{5.033121in}}%
\pgfpathlineto{\pgfqpoint{5.667318in}{5.042982in}}%
\pgfpathlineto{\pgfqpoint{5.671882in}{5.013979in}}%
\pgfpathlineto{\pgfqpoint{5.678728in}{5.010692in}}%
\pgfpathlineto{\pgfqpoint{5.681010in}{5.020650in}}%
\pgfpathlineto{\pgfqpoint{5.683291in}{5.027514in}}%
\pgfpathlineto{\pgfqpoint{5.685573in}{5.029544in}}%
\pgfpathlineto{\pgfqpoint{5.694701in}{5.025193in}}%
\pgfpathlineto{\pgfqpoint{5.696983in}{5.032251in}}%
\pgfpathlineto{\pgfqpoint{5.699265in}{5.030124in}}%
\pgfpathlineto{\pgfqpoint{5.701547in}{5.020746in}}%
\pgfpathlineto{\pgfqpoint{5.710674in}{5.001024in}}%
\pgfpathlineto{\pgfqpoint{5.712956in}{5.004698in}}%
\pgfpathlineto{\pgfqpoint{5.715238in}{5.000251in}}%
\pgfpathlineto{\pgfqpoint{5.719802in}{4.980625in}}%
\pgfpathlineto{\pgfqpoint{5.726648in}{4.975405in}}%
\pgfpathlineto{\pgfqpoint{5.728930in}{4.981302in}}%
\pgfpathlineto{\pgfqpoint{5.731212in}{4.970571in}}%
\pgfpathlineto{\pgfqpoint{5.733493in}{4.987006in}}%
\pgfpathlineto{\pgfqpoint{5.735775in}{4.970378in}}%
\pgfpathlineto{\pgfqpoint{5.744903in}{4.974825in}}%
\pgfpathlineto{\pgfqpoint{5.747185in}{4.959356in}}%
\pgfpathlineto{\pgfqpoint{5.749467in}{4.961097in}}%
\pgfpathlineto{\pgfqpoint{5.751749in}{4.968251in}}%
\pgfpathlineto{\pgfqpoint{5.758594in}{4.965157in}}%
\pgfpathlineto{\pgfqpoint{5.760876in}{5.007211in}}%
\pgfpathlineto{\pgfqpoint{5.763158in}{5.015816in}}%
\pgfpathlineto{\pgfqpoint{5.765440in}{5.016782in}}%
\pgfpathlineto{\pgfqpoint{5.767722in}{5.035924in}}%
\pgfpathlineto{\pgfqpoint{5.774568in}{5.035248in}}%
\pgfpathlineto{\pgfqpoint{5.776850in}{5.026837in}}%
\pgfpathlineto{\pgfqpoint{5.779132in}{5.033218in}}%
\pgfpathlineto{\pgfqpoint{5.781413in}{5.031187in}}%
\pgfpathlineto{\pgfqpoint{5.783695in}{5.001604in}}%
\pgfpathlineto{\pgfqpoint{5.790541in}{5.014462in}}%
\pgfpathlineto{\pgfqpoint{5.792823in}{5.014172in}}%
\pgfpathlineto{\pgfqpoint{5.795105in}{5.012045in}}%
\pgfpathlineto{\pgfqpoint{5.797387in}{5.011755in}}%
\pgfpathlineto{\pgfqpoint{5.808796in}{5.017266in}}%
\pgfpathlineto{\pgfqpoint{5.811078in}{5.018813in}}%
\pgfpathlineto{\pgfqpoint{5.813360in}{5.034184in}}%
\pgfpathlineto{\pgfqpoint{5.815642in}{5.040082in}}%
\pgfpathlineto{\pgfqpoint{5.822488in}{5.045302in}}%
\pgfpathlineto{\pgfqpoint{5.824770in}{5.039405in}}%
\pgfpathlineto{\pgfqpoint{5.827052in}{5.047139in}}%
\pgfpathlineto{\pgfqpoint{5.829334in}{5.059804in}}%
\pgfpathlineto{\pgfqpoint{5.831615in}{5.054486in}}%
\pgfpathlineto{\pgfqpoint{5.838461in}{5.049363in}}%
\pgfpathlineto{\pgfqpoint{5.840743in}{5.067248in}}%
\pgfpathlineto{\pgfqpoint{5.843025in}{5.065604in}}%
\pgfpathlineto{\pgfqpoint{5.845307in}{5.062124in}}%
\pgfpathlineto{\pgfqpoint{5.847589in}{5.060867in}}%
\pgfpathlineto{\pgfqpoint{5.854434in}{5.062994in}}%
\pgfpathlineto{\pgfqpoint{5.856716in}{5.058354in}}%
\pgfpathlineto{\pgfqpoint{5.858998in}{5.063864in}}%
\pgfpathlineto{\pgfqpoint{5.861280in}{5.066474in}}%
\pgfpathlineto{\pgfqpoint{5.863562in}{5.071598in}}%
\pgfpathlineto{\pgfqpoint{5.870408in}{5.071115in}}%
\pgfpathlineto{\pgfqpoint{5.872690in}{5.071985in}}%
\pgfpathlineto{\pgfqpoint{5.874972in}{5.068891in}}%
\pgfpathlineto{\pgfqpoint{5.877254in}{5.062994in}}%
\pgfpathlineto{\pgfqpoint{5.879535in}{5.069665in}}%
\pgfpathlineto{\pgfqpoint{5.886381in}{5.067731in}}%
\pgfpathlineto{\pgfqpoint{5.888663in}{5.068601in}}%
\pgfpathlineto{\pgfqpoint{5.890945in}{5.078559in}}%
\pgfpathlineto{\pgfqpoint{5.893227in}{5.076916in}}%
\pgfpathlineto{\pgfqpoint{5.902355in}{5.076142in}}%
\pgfpathlineto{\pgfqpoint{5.904636in}{5.084263in}}%
\pgfpathlineto{\pgfqpoint{5.906918in}{5.082813in}}%
\pgfpathlineto{\pgfqpoint{5.909200in}{5.075852in}}%
\pgfpathlineto{\pgfqpoint{5.911482in}{5.084650in}}%
\pgfpathlineto{\pgfqpoint{5.918328in}{5.079332in}}%
\pgfpathlineto{\pgfqpoint{5.920610in}{5.082523in}}%
\pgfpathlineto{\pgfqpoint{5.922892in}{5.086680in}}%
\pgfpathlineto{\pgfqpoint{5.927455in}{5.083876in}}%
\pgfpathlineto{\pgfqpoint{5.934301in}{5.082716in}}%
\pgfpathlineto{\pgfqpoint{5.936583in}{5.088323in}}%
\pgfpathlineto{\pgfqpoint{5.938865in}{5.090740in}}%
\pgfpathlineto{\pgfqpoint{5.941147in}{5.090450in}}%
\pgfpathlineto{\pgfqpoint{5.943429in}{5.093447in}}%
\pgfpathlineto{\pgfqpoint{5.950275in}{5.100118in}}%
\pgfpathlineto{\pgfqpoint{5.952556in}{5.115586in}}%
\pgfpathlineto{\pgfqpoint{5.954838in}{5.123610in}}%
\pgfpathlineto{\pgfqpoint{5.957120in}{5.123514in}}%
\pgfpathlineto{\pgfqpoint{5.959402in}{5.121290in}}%
\pgfpathlineto{\pgfqpoint{5.966248in}{5.122934in}}%
\pgfpathlineto{\pgfqpoint{5.968530in}{5.117616in}}%
\pgfpathlineto{\pgfqpoint{5.970812in}{5.116456in}}%
\pgfpathlineto{\pgfqpoint{5.973094in}{5.113846in}}%
\pgfpathlineto{\pgfqpoint{5.975376in}{5.110269in}}%
\pgfpathlineto{\pgfqpoint{5.982221in}{5.116263in}}%
\pgfpathlineto{\pgfqpoint{5.984503in}{5.115683in}}%
\pgfpathlineto{\pgfqpoint{5.986785in}{5.111526in}}%
\pgfpathlineto{\pgfqpoint{5.989067in}{5.117423in}}%
\pgfpathlineto{\pgfqpoint{5.991349in}{5.116166in}}%
\pgfpathlineto{\pgfqpoint{5.998195in}{5.124867in}}%
\pgfpathlineto{\pgfqpoint{6.000477in}{5.133278in}}%
\pgfpathlineto{\pgfqpoint{6.002758in}{5.130571in}}%
\pgfpathlineto{\pgfqpoint{6.005040in}{5.129411in}}%
\pgfpathlineto{\pgfqpoint{6.007322in}{5.123417in}}%
\pgfpathlineto{\pgfqpoint{6.014168in}{5.131248in}}%
\pgfpathlineto{\pgfqpoint{6.016450in}{5.125834in}}%
\pgfpathlineto{\pgfqpoint{6.018732in}{5.123707in}}%
\pgfpathlineto{\pgfqpoint{6.021014in}{5.117133in}}%
\pgfpathlineto{\pgfqpoint{6.023296in}{5.122354in}}%
\pgfpathlineto{\pgfqpoint{6.030141in}{5.118100in}}%
\pgfpathlineto{\pgfqpoint{6.034705in}{5.128734in}}%
\pgfpathlineto{\pgfqpoint{6.036987in}{5.124674in}}%
\pgfpathlineto{\pgfqpoint{6.039269in}{5.126124in}}%
\pgfpathlineto{\pgfqpoint{6.048397in}{5.122837in}}%
\pgfpathlineto{\pgfqpoint{6.050678in}{5.123610in}}%
\pgfpathlineto{\pgfqpoint{6.052960in}{5.138885in}}%
\pgfpathlineto{\pgfqpoint{6.055242in}{5.141302in}}%
\pgfpathlineto{\pgfqpoint{6.062088in}{5.150390in}}%
\pgfpathlineto{\pgfqpoint{6.064370in}{5.150003in}}%
\pgfpathlineto{\pgfqpoint{6.066652in}{5.150583in}}%
\pgfpathlineto{\pgfqpoint{6.068934in}{5.161314in}}%
\pgfpathlineto{\pgfqpoint{6.071216in}{5.161411in}}%
\pgfpathlineto{\pgfqpoint{6.078061in}{5.159478in}}%
\pgfpathlineto{\pgfqpoint{6.080343in}{5.162378in}}%
\pgfpathlineto{\pgfqpoint{6.082625in}{5.156094in}}%
\pgfpathlineto{\pgfqpoint{6.084907in}{5.158124in}}%
\pgfpathlineto{\pgfqpoint{6.087189in}{5.147780in}}%
\pgfpathlineto{\pgfqpoint{6.094035in}{5.157254in}}%
\pgfpathlineto{\pgfqpoint{6.096317in}{5.153967in}}%
\pgfpathlineto{\pgfqpoint{6.098599in}{5.156481in}}%
\pgfpathlineto{\pgfqpoint{6.100880in}{5.164698in}}%
\pgfpathlineto{\pgfqpoint{6.103162in}{5.149133in}}%
\pgfpathlineto{\pgfqpoint{6.110008in}{5.157351in}}%
\pgfpathlineto{\pgfqpoint{6.112290in}{5.172336in}}%
\pgfpathlineto{\pgfqpoint{6.114572in}{5.182100in}}%
\pgfpathlineto{\pgfqpoint{6.116854in}{5.199695in}}%
\pgfpathlineto{\pgfqpoint{6.119136in}{5.199599in}}%
\pgfpathlineto{\pgfqpoint{6.128263in}{5.208396in}}%
\pgfpathlineto{\pgfqpoint{6.130545in}{5.211683in}}%
\pgfpathlineto{\pgfqpoint{6.132827in}{5.210523in}}%
\pgfpathlineto{\pgfqpoint{6.135109in}{5.213617in}}%
\pgfpathlineto{\pgfqpoint{6.146519in}{5.214873in}}%
\pgfpathlineto{\pgfqpoint{6.148800in}{5.216517in}}%
\pgfpathlineto{\pgfqpoint{6.151082in}{5.214873in}}%
\pgfpathlineto{\pgfqpoint{6.157928in}{5.216130in}}%
\pgfpathlineto{\pgfqpoint{6.160210in}{5.234982in}}%
\pgfpathlineto{\pgfqpoint{6.162492in}{5.234016in}}%
\pgfpathlineto{\pgfqpoint{6.164774in}{5.234112in}}%
\pgfpathlineto{\pgfqpoint{6.167056in}{5.233049in}}%
\pgfpathlineto{\pgfqpoint{6.173901in}{5.231792in}}%
\pgfpathlineto{\pgfqpoint{6.176183in}{5.234112in}}%
\pgfpathlineto{\pgfqpoint{6.180747in}{5.227925in}}%
\pgfpathlineto{\pgfqpoint{6.183029in}{5.234789in}}%
\pgfpathlineto{\pgfqpoint{6.189875in}{5.236336in}}%
\pgfpathlineto{\pgfqpoint{6.192157in}{5.231405in}}%
\pgfpathlineto{\pgfqpoint{6.194439in}{5.222994in}}%
\pgfpathlineto{\pgfqpoint{6.196721in}{5.222608in}}%
\pgfpathlineto{\pgfqpoint{6.199002in}{5.225991in}}%
\pgfpathlineto{\pgfqpoint{6.208130in}{5.218741in}}%
\pgfpathlineto{\pgfqpoint{6.210412in}{5.218160in}}%
\pgfpathlineto{\pgfqpoint{6.212694in}{5.221834in}}%
\pgfpathlineto{\pgfqpoint{6.214976in}{5.216904in}}%
\pgfpathlineto{\pgfqpoint{6.221821in}{5.208783in}}%
\pgfpathlineto{\pgfqpoint{6.224103in}{5.191091in}}%
\pgfpathlineto{\pgfqpoint{6.226385in}{5.199792in}}%
\pgfpathlineto{\pgfqpoint{6.228667in}{5.194571in}}%
\pgfpathlineto{\pgfqpoint{6.230949in}{5.194571in}}%
\pgfpathlineto{\pgfqpoint{6.237795in}{5.187417in}}%
\pgfpathlineto{\pgfqpoint{6.240077in}{5.190221in}}%
\pgfpathlineto{\pgfqpoint{6.242359in}{5.183647in}}%
\pgfpathlineto{\pgfqpoint{6.244641in}{5.182390in}}%
\pgfpathlineto{\pgfqpoint{6.246922in}{5.186644in}}%
\pgfpathlineto{\pgfqpoint{6.253768in}{5.194571in}}%
\pgfpathlineto{\pgfqpoint{6.256050in}{5.190511in}}%
\pgfpathlineto{\pgfqpoint{6.258332in}{5.189351in}}%
\pgfpathlineto{\pgfqpoint{6.260614in}{5.187031in}}%
\pgfpathlineto{\pgfqpoint{6.262896in}{5.189157in}}%
\pgfpathlineto{\pgfqpoint{6.272023in}{5.193024in}}%
\pgfpathlineto{\pgfqpoint{6.276587in}{5.190511in}}%
\pgfpathlineto{\pgfqpoint{6.278869in}{5.179393in}}%
\pgfpathlineto{\pgfqpoint{6.285715in}{5.187707in}}%
\pgfpathlineto{\pgfqpoint{6.287997in}{5.173786in}}%
\pgfpathlineto{\pgfqpoint{6.290279in}{5.176009in}}%
\pgfpathlineto{\pgfqpoint{6.292561in}{5.182970in}}%
\pgfpathlineto{\pgfqpoint{6.294843in}{5.179586in}}%
\pgfpathlineto{\pgfqpoint{6.301688in}{5.174269in}}%
\pgfpathlineto{\pgfqpoint{6.303970in}{5.176879in}}%
\pgfpathlineto{\pgfqpoint{6.306252in}{5.185484in}}%
\pgfpathlineto{\pgfqpoint{6.308534in}{5.190414in}}%
\pgfpathlineto{\pgfqpoint{6.310816in}{5.184614in}}%
\pgfpathlineto{\pgfqpoint{6.317662in}{5.175333in}}%
\pgfpathlineto{\pgfqpoint{6.319943in}{5.188287in}}%
\pgfpathlineto{\pgfqpoint{6.322225in}{5.189834in}}%
\pgfpathlineto{\pgfqpoint{6.324507in}{5.170692in}}%
\pgfpathlineto{\pgfqpoint{6.326789in}{5.178426in}}%
\pgfpathlineto{\pgfqpoint{6.333635in}{5.184614in}}%
\pgfpathlineto{\pgfqpoint{6.335917in}{5.184710in}}%
\pgfpathlineto{\pgfqpoint{6.338199in}{5.187901in}}%
\pgfpathlineto{\pgfqpoint{6.340481in}{5.184034in}}%
\pgfpathlineto{\pgfqpoint{6.342763in}{5.188481in}}%
\pgfpathlineto{\pgfqpoint{6.349608in}{5.193508in}}%
\pgfpathlineto{\pgfqpoint{6.351890in}{5.174076in}}%
\pgfpathlineto{\pgfqpoint{6.356454in}{5.179586in}}%
\pgfpathlineto{\pgfqpoint{6.358736in}{5.173302in}}%
\pgfpathlineto{\pgfqpoint{6.365582in}{5.181713in}}%
\pgfpathlineto{\pgfqpoint{6.367864in}{5.153967in}}%
\pgfpathlineto{\pgfqpoint{6.370145in}{5.146620in}}%
\pgfpathlineto{\pgfqpoint{6.372427in}{5.149133in}}%
\pgfpathlineto{\pgfqpoint{6.374709in}{5.136275in}}%
\pgfpathlineto{\pgfqpoint{6.381555in}{5.137822in}}%
\pgfpathlineto{\pgfqpoint{6.383837in}{5.141012in}}%
\pgfpathlineto{\pgfqpoint{6.386119in}{5.146330in}}%
\pgfpathlineto{\pgfqpoint{6.388401in}{5.156577in}}%
\pgfpathlineto{\pgfqpoint{6.390683in}{5.153290in}}%
\pgfpathlineto{\pgfqpoint{6.397528in}{5.159188in}}%
\pgfpathlineto{\pgfqpoint{6.402092in}{5.149036in}}%
\pgfpathlineto{\pgfqpoint{6.404374in}{5.150583in}}%
\pgfpathlineto{\pgfqpoint{6.406656in}{5.151357in}}%
\pgfpathlineto{\pgfqpoint{6.415784in}{5.168759in}}%
\pgfpathlineto{\pgfqpoint{6.418065in}{5.198052in}}%
\pgfpathlineto{\pgfqpoint{6.420347in}{5.191091in}}%
\pgfpathlineto{\pgfqpoint{6.422629in}{5.181520in}}%
\pgfpathlineto{\pgfqpoint{6.429475in}{5.164698in}}%
\pgfpathlineto{\pgfqpoint{6.431757in}{5.162185in}}%
\pgfpathlineto{\pgfqpoint{6.434039in}{5.162571in}}%
\pgfpathlineto{\pgfqpoint{6.436321in}{5.164408in}}%
\pgfpathlineto{\pgfqpoint{6.438603in}{5.160734in}}%
\pgfpathlineto{\pgfqpoint{6.445448in}{5.157447in}}%
\pgfpathlineto{\pgfqpoint{6.447730in}{5.137049in}}%
\pgfpathlineto{\pgfqpoint{6.450012in}{5.139949in}}%
\pgfpathlineto{\pgfqpoint{6.454576in}{5.149617in}}%
\pgfpathlineto{\pgfqpoint{6.461422in}{5.140529in}}%
\pgfpathlineto{\pgfqpoint{6.463704in}{5.134632in}}%
\pgfpathlineto{\pgfqpoint{6.465986in}{5.123997in}}%
\pgfpathlineto{\pgfqpoint{6.468267in}{5.124674in}}%
\pgfpathlineto{\pgfqpoint{6.470549in}{5.129991in}}%
\pgfpathlineto{\pgfqpoint{6.477395in}{5.129798in}}%
\pgfpathlineto{\pgfqpoint{6.479677in}{5.130861in}}%
\pgfpathlineto{\pgfqpoint{6.481959in}{5.122160in}}%
\pgfpathlineto{\pgfqpoint{6.484241in}{5.121194in}}%
\pgfpathlineto{\pgfqpoint{6.486523in}{5.132698in}}%
\pgfpathlineto{\pgfqpoint{6.493368in}{5.160734in}}%
\pgfpathlineto{\pgfqpoint{6.495650in}{5.165568in}}%
\pgfpathlineto{\pgfqpoint{6.497932in}{5.157351in}}%
\pgfpathlineto{\pgfqpoint{6.500214in}{5.165568in}}%
\pgfpathlineto{\pgfqpoint{6.502496in}{5.165472in}}%
\pgfpathlineto{\pgfqpoint{6.509342in}{5.166728in}}%
\pgfpathlineto{\pgfqpoint{6.513906in}{5.160251in}}%
\pgfpathlineto{\pgfqpoint{6.516187in}{5.161411in}}%
\pgfpathlineto{\pgfqpoint{6.518469in}{5.166148in}}%
\pgfpathlineto{\pgfqpoint{6.527597in}{5.165665in}}%
\pgfpathlineto{\pgfqpoint{6.529879in}{5.158414in}}%
\pgfpathlineto{\pgfqpoint{6.532161in}{5.161895in}}%
\pgfpathlineto{\pgfqpoint{6.534443in}{5.159381in}}%
\pgfpathlineto{\pgfqpoint{6.543570in}{5.165085in}}%
\pgfpathlineto{\pgfqpoint{6.545852in}{5.163345in}}%
\pgfpathlineto{\pgfqpoint{6.548134in}{5.174269in}}%
\pgfpathlineto{\pgfqpoint{6.550416in}{5.169242in}}%
\pgfpathlineto{\pgfqpoint{6.557262in}{5.169049in}}%
\pgfpathlineto{\pgfqpoint{6.559544in}{5.167985in}}%
\pgfpathlineto{\pgfqpoint{6.561826in}{5.155030in}}%
\pgfpathlineto{\pgfqpoint{6.564108in}{5.154064in}}%
\pgfpathlineto{\pgfqpoint{6.566389in}{5.153870in}}%
\pgfpathlineto{\pgfqpoint{6.575517in}{5.156287in}}%
\pgfpathlineto{\pgfqpoint{6.577799in}{5.154740in}}%
\pgfpathlineto{\pgfqpoint{6.580081in}{5.150197in}}%
\pgfpathlineto{\pgfqpoint{6.582363in}{5.149810in}}%
\pgfpathlineto{\pgfqpoint{6.589208in}{5.147586in}}%
\pgfpathlineto{\pgfqpoint{6.591490in}{5.128154in}}%
\pgfpathlineto{\pgfqpoint{6.593772in}{5.137532in}}%
\pgfpathlineto{\pgfqpoint{6.596054in}{5.128831in}}%
\pgfpathlineto{\pgfqpoint{6.598336in}{5.142849in}}%
\pgfpathlineto{\pgfqpoint{6.605182in}{5.140529in}}%
\pgfpathlineto{\pgfqpoint{6.607464in}{5.141592in}}%
\pgfpathlineto{\pgfqpoint{6.609746in}{5.141399in}}%
\pgfpathlineto{\pgfqpoint{6.612028in}{5.144493in}}%
\pgfpathlineto{\pgfqpoint{6.614309in}{5.145169in}}%
\pgfpathlineto{\pgfqpoint{6.621155in}{5.142946in}}%
\pgfpathlineto{\pgfqpoint{6.623437in}{5.143719in}}%
\pgfpathlineto{\pgfqpoint{6.625719in}{5.142946in}}%
\pgfpathlineto{\pgfqpoint{6.628001in}{5.149133in}}%
\pgfpathlineto{\pgfqpoint{6.630283in}{5.159671in}}%
\pgfpathlineto{\pgfqpoint{6.637129in}{5.165472in}}%
\pgfpathlineto{\pgfqpoint{6.639410in}{5.169822in}}%
\pgfpathlineto{\pgfqpoint{6.646256in}{5.192444in}}%
\pgfpathlineto{\pgfqpoint{6.655384in}{5.199695in}}%
\pgfpathlineto{\pgfqpoint{6.657666in}{5.198342in}}%
\pgfpathlineto{\pgfqpoint{6.662230in}{5.234789in}}%
\pgfpathlineto{\pgfqpoint{6.669075in}{5.231792in}}%
\pgfpathlineto{\pgfqpoint{6.671357in}{5.230052in}}%
\pgfpathlineto{\pgfqpoint{6.673639in}{5.245037in}}%
\pgfpathlineto{\pgfqpoint{6.675921in}{5.243006in}}%
\pgfpathlineto{\pgfqpoint{6.678203in}{5.244457in}}%
\pgfpathlineto{\pgfqpoint{6.685049in}{5.243683in}}%
\pgfpathlineto{\pgfqpoint{6.687330in}{5.244843in}}%
\pgfpathlineto{\pgfqpoint{6.689612in}{5.247260in}}%
\pgfpathlineto{\pgfqpoint{6.691894in}{5.264082in}}%
\pgfpathlineto{\pgfqpoint{6.694176in}{5.266499in}}%
\pgfpathlineto{\pgfqpoint{6.701022in}{5.270753in}}%
\pgfpathlineto{\pgfqpoint{6.703304in}{5.274136in}}%
\pgfpathlineto{\pgfqpoint{6.705586in}{5.291538in}}%
\pgfpathlineto{\pgfqpoint{6.710150in}{5.283321in}}%
\pgfpathlineto{\pgfqpoint{6.716995in}{5.283417in}}%
\pgfpathlineto{\pgfqpoint{6.721559in}{5.266982in}}%
\pgfpathlineto{\pgfqpoint{6.726123in}{5.259828in}}%
\pgfpathlineto{\pgfqpoint{6.732969in}{5.262729in}}%
\pgfpathlineto{\pgfqpoint{6.735251in}{5.261472in}}%
\pgfpathlineto{\pgfqpoint{6.737532in}{5.254704in}}%
\pgfpathlineto{\pgfqpoint{6.739814in}{5.252384in}}%
\pgfpathlineto{\pgfqpoint{6.742096in}{5.251417in}}%
\pgfpathlineto{\pgfqpoint{6.753506in}{5.253641in}}%
\pgfpathlineto{\pgfqpoint{6.755788in}{5.255961in}}%
\pgfpathlineto{\pgfqpoint{6.758070in}{5.254704in}}%
\pgfpathlineto{\pgfqpoint{6.767197in}{5.248324in}}%
\pgfpathlineto{\pgfqpoint{6.769479in}{5.259152in}}%
\pgfpathlineto{\pgfqpoint{6.771761in}{5.255381in}}%
\pgfpathlineto{\pgfqpoint{6.780889in}{5.262052in}}%
\pgfpathlineto{\pgfqpoint{6.783171in}{5.226475in}}%
\pgfpathlineto{\pgfqpoint{6.785452in}{5.222414in}}%
\pgfpathlineto{\pgfqpoint{6.787734in}{5.226958in}}%
\pgfpathlineto{\pgfqpoint{6.790016in}{5.225991in}}%
\pgfpathlineto{\pgfqpoint{6.796862in}{5.236239in}}%
\pgfpathlineto{\pgfqpoint{6.799144in}{5.241170in}}%
\pgfpathlineto{\pgfqpoint{6.801426in}{5.241846in}}%
\pgfpathlineto{\pgfqpoint{6.803708in}{5.243973in}}%
\pgfpathlineto{\pgfqpoint{6.805990in}{5.241556in}}%
\pgfpathlineto{\pgfqpoint{6.812835in}{5.240300in}}%
\pgfpathlineto{\pgfqpoint{6.815117in}{5.243587in}}%
\pgfpathlineto{\pgfqpoint{6.817399in}{5.240300in}}%
\pgfpathlineto{\pgfqpoint{6.819681in}{5.245907in}}%
\pgfpathlineto{\pgfqpoint{6.821963in}{5.241846in}}%
\pgfpathlineto{\pgfqpoint{6.831091in}{5.239139in}}%
\pgfpathlineto{\pgfqpoint{6.833373in}{5.235562in}}%
\pgfpathlineto{\pgfqpoint{6.837936in}{5.243103in}}%
\pgfpathlineto{\pgfqpoint{6.844782in}{5.273556in}}%
\pgfpathlineto{\pgfqpoint{6.847064in}{5.280710in}}%
\pgfpathlineto{\pgfqpoint{6.849346in}{5.270656in}}%
\pgfpathlineto{\pgfqpoint{6.851628in}{5.273460in}}%
\pgfpathlineto{\pgfqpoint{6.853910in}{5.273653in}}%
\pgfpathlineto{\pgfqpoint{6.860755in}{5.276070in}}%
\pgfpathlineto{\pgfqpoint{6.863037in}{5.278390in}}%
\pgfpathlineto{\pgfqpoint{6.865319in}{5.278294in}}%
\pgfpathlineto{\pgfqpoint{6.867601in}{5.286221in}}%
\pgfpathlineto{\pgfqpoint{6.869883in}{5.280614in}}%
\pgfpathlineto{\pgfqpoint{6.879011in}{5.282354in}}%
\pgfpathlineto{\pgfqpoint{6.881293in}{5.292795in}}%
\pgfpathlineto{\pgfqpoint{6.883574in}{5.297726in}}%
\pgfpathlineto{\pgfqpoint{6.885856in}{5.309617in}}%
\pgfpathlineto{\pgfqpoint{6.892702in}{5.312227in}}%
\pgfpathlineto{\pgfqpoint{6.894984in}{5.316481in}}%
\pgfpathlineto{\pgfqpoint{6.897266in}{5.315804in}}%
\pgfpathlineto{\pgfqpoint{6.899548in}{5.313967in}}%
\pgfpathlineto{\pgfqpoint{6.901830in}{5.322958in}}%
\pgfpathlineto{\pgfqpoint{6.908675in}{5.325569in}}%
\pgfpathlineto{\pgfqpoint{6.910957in}{5.327405in}}%
\pgfpathlineto{\pgfqpoint{6.913239in}{5.334269in}}%
\pgfpathlineto{\pgfqpoint{6.915521in}{5.336783in}}%
\pgfpathlineto{\pgfqpoint{6.917803in}{5.348771in}}%
\pgfpathlineto{\pgfqpoint{6.924649in}{5.346257in}}%
\pgfpathlineto{\pgfqpoint{6.926931in}{5.347611in}}%
\pgfpathlineto{\pgfqpoint{6.929213in}{5.353895in}}%
\pgfpathlineto{\pgfqpoint{6.931495in}{5.364336in}}%
\pgfpathlineto{\pgfqpoint{6.933776in}{5.367816in}}%
\pgfpathlineto{\pgfqpoint{6.940622in}{5.367043in}}%
\pgfpathlineto{\pgfqpoint{6.947468in}{5.333109in}}%
\pgfpathlineto{\pgfqpoint{6.949750in}{5.329919in}}%
\pgfpathlineto{\pgfqpoint{6.956595in}{5.335526in}}%
\pgfpathlineto{\pgfqpoint{6.961159in}{5.342294in}}%
\pgfpathlineto{\pgfqpoint{6.963441in}{5.331949in}}%
\pgfpathlineto{\pgfqpoint{6.965723in}{5.332143in}}%
\pgfpathlineto{\pgfqpoint{6.974851in}{5.320155in}}%
\pgfpathlineto{\pgfqpoint{6.977133in}{5.329339in}}%
\pgfpathlineto{\pgfqpoint{6.979415in}{5.325955in}}%
\pgfpathlineto{\pgfqpoint{6.981696in}{5.332723in}}%
\pgfpathlineto{\pgfqpoint{6.988542in}{5.328566in}}%
\pgfpathlineto{\pgfqpoint{6.990824in}{5.349834in}}%
\pgfpathlineto{\pgfqpoint{6.993106in}{5.356699in}}%
\pgfpathlineto{\pgfqpoint{6.995388in}{5.369170in}}%
\pgfpathlineto{\pgfqpoint{6.997670in}{5.357569in}}%
\pgfpathlineto{\pgfqpoint{7.004516in}{5.336493in}}%
\pgfpathlineto{\pgfqpoint{7.009079in}{5.317641in}}%
\pgfpathlineto{\pgfqpoint{7.011361in}{5.316481in}}%
\pgfpathlineto{\pgfqpoint{7.013643in}{5.325859in}}%
\pgfpathlineto{\pgfqpoint{7.020489in}{5.333786in}}%
\pgfpathlineto{\pgfqpoint{7.022771in}{5.331949in}}%
\pgfpathlineto{\pgfqpoint{7.025053in}{5.328662in}}%
\pgfpathlineto{\pgfqpoint{7.027335in}{5.339587in}}%
\pgfpathlineto{\pgfqpoint{7.029617in}{5.338040in}}%
\pgfpathlineto{\pgfqpoint{7.036462in}{5.335140in}}%
\pgfpathlineto{\pgfqpoint{7.038744in}{5.329726in}}%
\pgfpathlineto{\pgfqpoint{7.041026in}{5.338523in}}%
\pgfpathlineto{\pgfqpoint{7.043308in}{5.337266in}}%
\pgfpathlineto{\pgfqpoint{7.045590in}{5.337363in}}%
\pgfpathlineto{\pgfqpoint{7.052436in}{5.340844in}}%
\pgfpathlineto{\pgfqpoint{7.054717in}{5.339877in}}%
\pgfpathlineto{\pgfqpoint{7.056999in}{5.347128in}}%
\pgfpathlineto{\pgfqpoint{7.059281in}{5.336783in}}%
\pgfpathlineto{\pgfqpoint{7.061563in}{5.333013in}}%
\pgfpathlineto{\pgfqpoint{7.068409in}{5.340553in}}%
\pgfpathlineto{\pgfqpoint{7.070691in}{5.351961in}}%
\pgfpathlineto{\pgfqpoint{7.072973in}{5.334560in}}%
\pgfpathlineto{\pgfqpoint{7.075255in}{5.335430in}}%
\pgfpathlineto{\pgfqpoint{7.077537in}{5.331949in}}%
\pgfpathlineto{\pgfqpoint{7.084382in}{5.332529in}}%
\pgfpathlineto{\pgfqpoint{7.086664in}{5.336880in}}%
\pgfpathlineto{\pgfqpoint{7.088946in}{5.326342in}}%
\pgfpathlineto{\pgfqpoint{7.091228in}{5.338330in}}%
\pgfpathlineto{\pgfqpoint{7.093510in}{5.325955in}}%
\pgfpathlineto{\pgfqpoint{7.102638in}{5.315514in}}%
\pgfpathlineto{\pgfqpoint{7.104919in}{5.322668in}}%
\pgfpathlineto{\pgfqpoint{7.107201in}{5.336686in}}%
\pgfpathlineto{\pgfqpoint{7.109483in}{5.325472in}}%
\pgfpathlineto{\pgfqpoint{7.116329in}{5.346064in}}%
\pgfpathlineto{\pgfqpoint{7.118611in}{5.340747in}}%
\pgfpathlineto{\pgfqpoint{7.120893in}{5.339103in}}%
\pgfpathlineto{\pgfqpoint{7.123175in}{5.355055in}}%
\pgfpathlineto{\pgfqpoint{7.125457in}{5.357569in}}%
\pgfpathlineto{\pgfqpoint{7.132302in}{5.366173in}}%
\pgfpathlineto{\pgfqpoint{7.134584in}{5.364626in}}%
\pgfpathlineto{\pgfqpoint{7.136866in}{5.346161in}}%
\pgfpathlineto{\pgfqpoint{7.139148in}{5.332626in}}%
\pgfpathlineto{\pgfqpoint{7.141430in}{5.329339in}}%
\pgfpathlineto{\pgfqpoint{7.148276in}{5.327309in}}%
\pgfpathlineto{\pgfqpoint{7.150558in}{5.325182in}}%
\pgfpathlineto{\pgfqpoint{7.152839in}{5.314161in}}%
\pgfpathlineto{\pgfqpoint{7.155121in}{5.311550in}}%
\pgfpathlineto{\pgfqpoint{7.157403in}{5.316578in}}%
\pgfpathlineto{\pgfqpoint{7.164249in}{5.327695in}}%
\pgfpathlineto{\pgfqpoint{7.168813in}{5.343164in}}%
\pgfpathlineto{\pgfqpoint{7.171095in}{5.345871in}}%
\pgfpathlineto{\pgfqpoint{7.173377in}{5.346161in}}%
\pgfpathlineto{\pgfqpoint{7.180222in}{5.348288in}}%
\pgfpathlineto{\pgfqpoint{7.182504in}{5.352445in}}%
\pgfpathlineto{\pgfqpoint{7.184786in}{5.377871in}}%
\pgfpathlineto{\pgfqpoint{7.187068in}{5.379514in}}%
\pgfpathlineto{\pgfqpoint{7.189350in}{5.375841in}}%
\pgfpathlineto{\pgfqpoint{7.196196in}{5.372940in}}%
\pgfpathlineto{\pgfqpoint{7.198478in}{5.416058in}}%
\pgfpathlineto{\pgfqpoint{7.200760in}{5.415091in}}%
\pgfpathlineto{\pgfqpoint{7.203041in}{5.427563in}}%
\pgfpathlineto{\pgfqpoint{7.205323in}{5.430946in}}%
\pgfpathlineto{\pgfqpoint{7.212169in}{5.442161in}}%
\pgfpathlineto{\pgfqpoint{7.214451in}{5.423889in}}%
\pgfpathlineto{\pgfqpoint{7.216733in}{5.430560in}}%
\pgfpathlineto{\pgfqpoint{7.219015in}{5.425436in}}%
\pgfpathlineto{\pgfqpoint{7.221297in}{5.425242in}}%
\pgfpathlineto{\pgfqpoint{7.230424in}{5.403297in}}%
\pgfpathlineto{\pgfqpoint{7.232706in}{5.408614in}}%
\pgfpathlineto{\pgfqpoint{7.234988in}{5.408131in}}%
\pgfpathlineto{\pgfqpoint{7.237270in}{5.409484in}}%
\pgfpathlineto{\pgfqpoint{7.244116in}{5.406584in}}%
\pgfpathlineto{\pgfqpoint{7.246398in}{5.406681in}}%
\pgfpathlineto{\pgfqpoint{7.248680in}{5.420989in}}%
\pgfpathlineto{\pgfqpoint{7.253243in}{5.404747in}}%
\pgfpathlineto{\pgfqpoint{7.260089in}{5.406584in}}%
\pgfpathlineto{\pgfqpoint{7.262371in}{5.404070in}}%
\pgfpathlineto{\pgfqpoint{7.264653in}{5.400493in}}%
\pgfpathlineto{\pgfqpoint{7.266935in}{5.398366in}}%
\pgfpathlineto{\pgfqpoint{7.269217in}{5.390342in}}%
\pgfpathlineto{\pgfqpoint{7.276062in}{5.389665in}}%
\pgfpathlineto{\pgfqpoint{7.278344in}{5.393436in}}%
\pgfpathlineto{\pgfqpoint{7.280626in}{5.383768in}}%
\pgfpathlineto{\pgfqpoint{7.285190in}{5.390439in}}%
\pgfpathlineto{\pgfqpoint{7.292036in}{5.399043in}}%
\pgfpathlineto{\pgfqpoint{7.294318in}{5.416832in}}%
\pgfpathlineto{\pgfqpoint{7.296600in}{5.414898in}}%
\pgfpathlineto{\pgfqpoint{7.298882in}{5.410451in}}%
\pgfpathlineto{\pgfqpoint{7.301163in}{5.416542in}}%
\pgfpathlineto{\pgfqpoint{7.308009in}{5.407454in}}%
\pgfpathlineto{\pgfqpoint{7.310291in}{5.413641in}}%
\pgfpathlineto{\pgfqpoint{7.312573in}{5.426499in}}%
\pgfpathlineto{\pgfqpoint{7.314855in}{5.416735in}}%
\pgfpathlineto{\pgfqpoint{7.317137in}{5.422149in}}%
\pgfpathlineto{\pgfqpoint{7.323982in}{5.427273in}}%
\pgfpathlineto{\pgfqpoint{7.326264in}{5.440807in}}%
\pgfpathlineto{\pgfqpoint{7.328546in}{5.443514in}}%
\pgfpathlineto{\pgfqpoint{7.330828in}{5.432010in}}%
\pgfpathlineto{\pgfqpoint{7.333110in}{5.439551in}}%
\pgfpathlineto{\pgfqpoint{7.339956in}{5.433363in}}%
\pgfpathlineto{\pgfqpoint{7.342238in}{5.433170in}}%
\pgfpathlineto{\pgfqpoint{7.344520in}{5.427466in}}%
\pgfpathlineto{\pgfqpoint{7.346802in}{5.426499in}}%
\pgfpathlineto{\pgfqpoint{7.349083in}{5.417798in}}%
\pgfpathlineto{\pgfqpoint{7.358211in}{5.417508in}}%
\pgfpathlineto{\pgfqpoint{7.360493in}{5.421955in}}%
\pgfpathlineto{\pgfqpoint{7.362775in}{5.421859in}}%
\pgfpathlineto{\pgfqpoint{7.365057in}{5.414125in}}%
\pgfpathlineto{\pgfqpoint{7.365057in}{5.414125in}}%
\pgfusepath{stroke}%
\end{pgfscope}%
\begin{pgfscope}%
\pgfsetrectcap%
\pgfsetmiterjoin%
\pgfsetlinewidth{0.803000pt}%
\definecolor{currentstroke}{rgb}{1.000000,1.000000,1.000000}%
\pgfsetstrokecolor{currentstroke}%
\pgfsetdash{}{0pt}%
\pgfpathmoveto{\pgfqpoint{2.125000in}{4.564634in}}%
\pgfpathlineto{\pgfqpoint{2.125000in}{5.485366in}}%
\pgfusepath{stroke}%
\end{pgfscope}%
\begin{pgfscope}%
\pgfsetrectcap%
\pgfsetmiterjoin%
\pgfsetlinewidth{0.803000pt}%
\definecolor{currentstroke}{rgb}{1.000000,1.000000,1.000000}%
\pgfsetstrokecolor{currentstroke}%
\pgfsetdash{}{0pt}%
\pgfpathmoveto{\pgfqpoint{7.614583in}{4.564634in}}%
\pgfpathlineto{\pgfqpoint{7.614583in}{5.485366in}}%
\pgfusepath{stroke}%
\end{pgfscope}%
\begin{pgfscope}%
\pgfsetrectcap%
\pgfsetmiterjoin%
\pgfsetlinewidth{0.803000pt}%
\definecolor{currentstroke}{rgb}{1.000000,1.000000,1.000000}%
\pgfsetstrokecolor{currentstroke}%
\pgfsetdash{}{0pt}%
\pgfpathmoveto{\pgfqpoint{2.125000in}{4.564634in}}%
\pgfpathlineto{\pgfqpoint{7.614583in}{4.564634in}}%
\pgfusepath{stroke}%
\end{pgfscope}%
\begin{pgfscope}%
\pgfsetrectcap%
\pgfsetmiterjoin%
\pgfsetlinewidth{0.803000pt}%
\definecolor{currentstroke}{rgb}{1.000000,1.000000,1.000000}%
\pgfsetstrokecolor{currentstroke}%
\pgfsetdash{}{0pt}%
\pgfpathmoveto{\pgfqpoint{2.125000in}{5.485366in}}%
\pgfpathlineto{\pgfqpoint{7.614583in}{5.485366in}}%
\pgfusepath{stroke}%
\end{pgfscope}%
\begin{pgfscope}%
\definecolor{textcolor}{rgb}{0.150000,0.150000,0.150000}%
\pgfsetstrokecolor{textcolor}%
\pgfsetfillcolor{textcolor}%
\pgftext[x=4.869792in,y=5.568699in,,base]{\color{textcolor}\rmfamily\fontsize{12.000000}{14.400000}\selectfont JNJ}%
\end{pgfscope}%
\begin{pgfscope}%
\pgfsetbuttcap%
\pgfsetmiterjoin%
\definecolor{currentfill}{rgb}{0.917647,0.917647,0.949020}%
\pgfsetfillcolor{currentfill}%
\pgfsetlinewidth{0.000000pt}%
\definecolor{currentstroke}{rgb}{0.000000,0.000000,0.000000}%
\pgfsetstrokecolor{currentstroke}%
\pgfsetstrokeopacity{0.000000}%
\pgfsetdash{}{0pt}%
\pgfpathmoveto{\pgfqpoint{9.810417in}{4.564634in}}%
\pgfpathlineto{\pgfqpoint{15.300000in}{4.564634in}}%
\pgfpathlineto{\pgfqpoint{15.300000in}{5.485366in}}%
\pgfpathlineto{\pgfqpoint{9.810417in}{5.485366in}}%
\pgfpathclose%
\pgfusepath{fill}%
\end{pgfscope}%
\begin{pgfscope}%
\pgfpathrectangle{\pgfqpoint{9.810417in}{4.564634in}}{\pgfqpoint{5.489583in}{0.920732in}}%
\pgfusepath{clip}%
\pgfsetroundcap%
\pgfsetroundjoin%
\pgfsetlinewidth{0.803000pt}%
\definecolor{currentstroke}{rgb}{1.000000,1.000000,1.000000}%
\pgfsetstrokecolor{currentstroke}%
\pgfsetdash{}{0pt}%
\pgfpathmoveto{\pgfqpoint{10.055379in}{4.564634in}}%
\pgfpathlineto{\pgfqpoint{10.055379in}{5.485366in}}%
\pgfusepath{stroke}%
\end{pgfscope}%
\begin{pgfscope}%
\definecolor{textcolor}{rgb}{0.150000,0.150000,0.150000}%
\pgfsetstrokecolor{textcolor}%
\pgfsetfillcolor{textcolor}%
\pgftext[x=10.055379in,y=4.467412in,,top]{\color{textcolor}\rmfamily\fontsize{10.000000}{12.000000}\selectfont 2012}%
\end{pgfscope}%
\begin{pgfscope}%
\pgfpathrectangle{\pgfqpoint{9.810417in}{4.564634in}}{\pgfqpoint{5.489583in}{0.920732in}}%
\pgfusepath{clip}%
\pgfsetroundcap%
\pgfsetroundjoin%
\pgfsetlinewidth{0.803000pt}%
\definecolor{currentstroke}{rgb}{1.000000,1.000000,1.000000}%
\pgfsetstrokecolor{currentstroke}%
\pgfsetdash{}{0pt}%
\pgfpathmoveto{\pgfqpoint{10.890557in}{4.564634in}}%
\pgfpathlineto{\pgfqpoint{10.890557in}{5.485366in}}%
\pgfusepath{stroke}%
\end{pgfscope}%
\begin{pgfscope}%
\definecolor{textcolor}{rgb}{0.150000,0.150000,0.150000}%
\pgfsetstrokecolor{textcolor}%
\pgfsetfillcolor{textcolor}%
\pgftext[x=10.890557in,y=4.467412in,,top]{\color{textcolor}\rmfamily\fontsize{10.000000}{12.000000}\selectfont 2013}%
\end{pgfscope}%
\begin{pgfscope}%
\pgfpathrectangle{\pgfqpoint{9.810417in}{4.564634in}}{\pgfqpoint{5.489583in}{0.920732in}}%
\pgfusepath{clip}%
\pgfsetroundcap%
\pgfsetroundjoin%
\pgfsetlinewidth{0.803000pt}%
\definecolor{currentstroke}{rgb}{1.000000,1.000000,1.000000}%
\pgfsetstrokecolor{currentstroke}%
\pgfsetdash{}{0pt}%
\pgfpathmoveto{\pgfqpoint{11.723453in}{4.564634in}}%
\pgfpathlineto{\pgfqpoint{11.723453in}{5.485366in}}%
\pgfusepath{stroke}%
\end{pgfscope}%
\begin{pgfscope}%
\definecolor{textcolor}{rgb}{0.150000,0.150000,0.150000}%
\pgfsetstrokecolor{textcolor}%
\pgfsetfillcolor{textcolor}%
\pgftext[x=11.723453in,y=4.467412in,,top]{\color{textcolor}\rmfamily\fontsize{10.000000}{12.000000}\selectfont 2014}%
\end{pgfscope}%
\begin{pgfscope}%
\pgfpathrectangle{\pgfqpoint{9.810417in}{4.564634in}}{\pgfqpoint{5.489583in}{0.920732in}}%
\pgfusepath{clip}%
\pgfsetroundcap%
\pgfsetroundjoin%
\pgfsetlinewidth{0.803000pt}%
\definecolor{currentstroke}{rgb}{1.000000,1.000000,1.000000}%
\pgfsetstrokecolor{currentstroke}%
\pgfsetdash{}{0pt}%
\pgfpathmoveto{\pgfqpoint{12.556349in}{4.564634in}}%
\pgfpathlineto{\pgfqpoint{12.556349in}{5.485366in}}%
\pgfusepath{stroke}%
\end{pgfscope}%
\begin{pgfscope}%
\definecolor{textcolor}{rgb}{0.150000,0.150000,0.150000}%
\pgfsetstrokecolor{textcolor}%
\pgfsetfillcolor{textcolor}%
\pgftext[x=12.556349in,y=4.467412in,,top]{\color{textcolor}\rmfamily\fontsize{10.000000}{12.000000}\selectfont 2015}%
\end{pgfscope}%
\begin{pgfscope}%
\pgfpathrectangle{\pgfqpoint{9.810417in}{4.564634in}}{\pgfqpoint{5.489583in}{0.920732in}}%
\pgfusepath{clip}%
\pgfsetroundcap%
\pgfsetroundjoin%
\pgfsetlinewidth{0.803000pt}%
\definecolor{currentstroke}{rgb}{1.000000,1.000000,1.000000}%
\pgfsetstrokecolor{currentstroke}%
\pgfsetdash{}{0pt}%
\pgfpathmoveto{\pgfqpoint{13.389245in}{4.564634in}}%
\pgfpathlineto{\pgfqpoint{13.389245in}{5.485366in}}%
\pgfusepath{stroke}%
\end{pgfscope}%
\begin{pgfscope}%
\definecolor{textcolor}{rgb}{0.150000,0.150000,0.150000}%
\pgfsetstrokecolor{textcolor}%
\pgfsetfillcolor{textcolor}%
\pgftext[x=13.389245in,y=4.467412in,,top]{\color{textcolor}\rmfamily\fontsize{10.000000}{12.000000}\selectfont 2016}%
\end{pgfscope}%
\begin{pgfscope}%
\pgfpathrectangle{\pgfqpoint{9.810417in}{4.564634in}}{\pgfqpoint{5.489583in}{0.920732in}}%
\pgfusepath{clip}%
\pgfsetroundcap%
\pgfsetroundjoin%
\pgfsetlinewidth{0.803000pt}%
\definecolor{currentstroke}{rgb}{1.000000,1.000000,1.000000}%
\pgfsetstrokecolor{currentstroke}%
\pgfsetdash{}{0pt}%
\pgfpathmoveto{\pgfqpoint{14.224423in}{4.564634in}}%
\pgfpathlineto{\pgfqpoint{14.224423in}{5.485366in}}%
\pgfusepath{stroke}%
\end{pgfscope}%
\begin{pgfscope}%
\definecolor{textcolor}{rgb}{0.150000,0.150000,0.150000}%
\pgfsetstrokecolor{textcolor}%
\pgfsetfillcolor{textcolor}%
\pgftext[x=14.224423in,y=4.467412in,,top]{\color{textcolor}\rmfamily\fontsize{10.000000}{12.000000}\selectfont 2017}%
\end{pgfscope}%
\begin{pgfscope}%
\pgfpathrectangle{\pgfqpoint{9.810417in}{4.564634in}}{\pgfqpoint{5.489583in}{0.920732in}}%
\pgfusepath{clip}%
\pgfsetroundcap%
\pgfsetroundjoin%
\pgfsetlinewidth{0.803000pt}%
\definecolor{currentstroke}{rgb}{1.000000,1.000000,1.000000}%
\pgfsetstrokecolor{currentstroke}%
\pgfsetdash{}{0pt}%
\pgfpathmoveto{\pgfqpoint{15.057319in}{4.564634in}}%
\pgfpathlineto{\pgfqpoint{15.057319in}{5.485366in}}%
\pgfusepath{stroke}%
\end{pgfscope}%
\begin{pgfscope}%
\definecolor{textcolor}{rgb}{0.150000,0.150000,0.150000}%
\pgfsetstrokecolor{textcolor}%
\pgfsetfillcolor{textcolor}%
\pgftext[x=15.057319in,y=4.467412in,,top]{\color{textcolor}\rmfamily\fontsize{10.000000}{12.000000}\selectfont 2018}%
\end{pgfscope}%
\begin{pgfscope}%
\pgfpathrectangle{\pgfqpoint{9.810417in}{4.564634in}}{\pgfqpoint{5.489583in}{0.920732in}}%
\pgfusepath{clip}%
\pgfsetroundcap%
\pgfsetroundjoin%
\pgfsetlinewidth{0.803000pt}%
\definecolor{currentstroke}{rgb}{1.000000,1.000000,1.000000}%
\pgfsetstrokecolor{currentstroke}%
\pgfsetdash{}{0pt}%
\pgfpathmoveto{\pgfqpoint{9.810417in}{4.858837in}}%
\pgfpathlineto{\pgfqpoint{15.300000in}{4.858837in}}%
\pgfusepath{stroke}%
\end{pgfscope}%
\begin{pgfscope}%
\definecolor{textcolor}{rgb}{0.150000,0.150000,0.150000}%
\pgfsetstrokecolor{textcolor}%
\pgfsetfillcolor{textcolor}%
\pgftext[x=9.536464in,y=4.806075in,left,base]{\color{textcolor}\rmfamily\fontsize{10.000000}{12.000000}\selectfont 60}%
\end{pgfscope}%
\begin{pgfscope}%
\pgfpathrectangle{\pgfqpoint{9.810417in}{4.564634in}}{\pgfqpoint{5.489583in}{0.920732in}}%
\pgfusepath{clip}%
\pgfsetroundcap%
\pgfsetroundjoin%
\pgfsetlinewidth{0.803000pt}%
\definecolor{currentstroke}{rgb}{1.000000,1.000000,1.000000}%
\pgfsetstrokecolor{currentstroke}%
\pgfsetdash{}{0pt}%
\pgfpathmoveto{\pgfqpoint{9.810417in}{5.259713in}}%
\pgfpathlineto{\pgfqpoint{15.300000in}{5.259713in}}%
\pgfusepath{stroke}%
\end{pgfscope}%
\begin{pgfscope}%
\definecolor{textcolor}{rgb}{0.150000,0.150000,0.150000}%
\pgfsetstrokecolor{textcolor}%
\pgfsetfillcolor{textcolor}%
\pgftext[x=9.536464in,y=5.206951in,left,base]{\color{textcolor}\rmfamily\fontsize{10.000000}{12.000000}\selectfont 80}%
\end{pgfscope}%
\begin{pgfscope}%
\pgfpathrectangle{\pgfqpoint{9.810417in}{4.564634in}}{\pgfqpoint{5.489583in}{0.920732in}}%
\pgfusepath{clip}%
\pgfsetroundcap%
\pgfsetroundjoin%
\pgfsetlinewidth{1.505625pt}%
\definecolor{currentstroke}{rgb}{0.121569,0.466667,0.705882}%
\pgfsetstrokecolor{currentstroke}%
\pgfsetdash{}{0pt}%
\pgfpathmoveto{\pgfqpoint{10.059943in}{4.710112in}}%
\pgfpathlineto{\pgfqpoint{10.062225in}{4.709711in}}%
\pgfpathlineto{\pgfqpoint{10.064507in}{4.705301in}}%
\pgfpathlineto{\pgfqpoint{10.066789in}{4.702696in}}%
\pgfpathlineto{\pgfqpoint{10.073635in}{4.707105in}}%
\pgfpathlineto{\pgfqpoint{10.075917in}{4.702295in}}%
\pgfpathlineto{\pgfqpoint{10.078198in}{4.692073in}}%
\pgfpathlineto{\pgfqpoint{10.080480in}{4.694077in}}%
\pgfpathlineto{\pgfqpoint{10.082762in}{4.694077in}}%
\pgfpathlineto{\pgfqpoint{10.091890in}{4.701293in}}%
\pgfpathlineto{\pgfqpoint{10.094172in}{4.705903in}}%
\pgfpathlineto{\pgfqpoint{10.096454in}{4.706705in}}%
\pgfpathlineto{\pgfqpoint{10.098736in}{4.709110in}}%
\pgfpathlineto{\pgfqpoint{10.107863in}{4.681650in}}%
\pgfpathlineto{\pgfqpoint{10.110145in}{4.689266in}}%
\pgfpathlineto{\pgfqpoint{10.112427in}{4.686260in}}%
\pgfpathlineto{\pgfqpoint{10.114709in}{4.678443in}}%
\pgfpathlineto{\pgfqpoint{10.121555in}{4.661005in}}%
\pgfpathlineto{\pgfqpoint{10.123837in}{4.658399in}}%
\pgfpathlineto{\pgfqpoint{10.128400in}{4.662809in}}%
\pgfpathlineto{\pgfqpoint{10.130682in}{4.653989in}}%
\pgfpathlineto{\pgfqpoint{10.139810in}{4.668822in}}%
\pgfpathlineto{\pgfqpoint{10.142092in}{4.667820in}}%
\pgfpathlineto{\pgfqpoint{10.144374in}{4.674234in}}%
\pgfpathlineto{\pgfqpoint{10.146656in}{4.671628in}}%
\pgfpathlineto{\pgfqpoint{10.153501in}{4.677240in}}%
\pgfpathlineto{\pgfqpoint{10.155783in}{4.681249in}}%
\pgfpathlineto{\pgfqpoint{10.158065in}{4.682452in}}%
\pgfpathlineto{\pgfqpoint{10.160347in}{4.692674in}}%
\pgfpathlineto{\pgfqpoint{10.162629in}{4.688064in}}%
\pgfpathlineto{\pgfqpoint{10.171757in}{4.680247in}}%
\pgfpathlineto{\pgfqpoint{10.174039in}{4.680648in}}%
\pgfpathlineto{\pgfqpoint{10.176320in}{4.712116in}}%
\pgfpathlineto{\pgfqpoint{10.178602in}{4.716726in}}%
\pgfpathlineto{\pgfqpoint{10.185448in}{4.716526in}}%
\pgfpathlineto{\pgfqpoint{10.187730in}{4.727550in}}%
\pgfpathlineto{\pgfqpoint{10.190012in}{4.731158in}}%
\pgfpathlineto{\pgfqpoint{10.192294in}{4.715925in}}%
\pgfpathlineto{\pgfqpoint{10.194576in}{4.716125in}}%
\pgfpathlineto{\pgfqpoint{10.201421in}{4.720535in}}%
\pgfpathlineto{\pgfqpoint{10.203703in}{4.718731in}}%
\pgfpathlineto{\pgfqpoint{10.205985in}{4.714722in}}%
\pgfpathlineto{\pgfqpoint{10.208267in}{4.719733in}}%
\pgfpathlineto{\pgfqpoint{10.210549in}{4.720134in}}%
\pgfpathlineto{\pgfqpoint{10.219677in}{4.735568in}}%
\pgfpathlineto{\pgfqpoint{10.221959in}{4.734766in}}%
\pgfpathlineto{\pgfqpoint{10.224240in}{4.732160in}}%
\pgfpathlineto{\pgfqpoint{10.226522in}{4.725345in}}%
\pgfpathlineto{\pgfqpoint{10.237932in}{4.724544in}}%
\pgfpathlineto{\pgfqpoint{10.240214in}{4.729554in}}%
\pgfpathlineto{\pgfqpoint{10.242496in}{4.728151in}}%
\pgfpathlineto{\pgfqpoint{10.249341in}{4.728552in}}%
\pgfpathlineto{\pgfqpoint{10.251623in}{4.723742in}}%
\pgfpathlineto{\pgfqpoint{10.253905in}{4.724343in}}%
\pgfpathlineto{\pgfqpoint{10.256187in}{4.721537in}}%
\pgfpathlineto{\pgfqpoint{10.258469in}{4.724744in}}%
\pgfpathlineto{\pgfqpoint{10.265315in}{4.730156in}}%
\pgfpathlineto{\pgfqpoint{10.267597in}{4.722740in}}%
\pgfpathlineto{\pgfqpoint{10.269879in}{4.725546in}}%
\pgfpathlineto{\pgfqpoint{10.272161in}{4.726147in}}%
\pgfpathlineto{\pgfqpoint{10.281288in}{4.718330in}}%
\pgfpathlineto{\pgfqpoint{10.283570in}{4.710914in}}%
\pgfpathlineto{\pgfqpoint{10.285852in}{4.712317in}}%
\pgfpathlineto{\pgfqpoint{10.290416in}{4.702495in}}%
\pgfpathlineto{\pgfqpoint{10.299543in}{4.721537in}}%
\pgfpathlineto{\pgfqpoint{10.301825in}{4.717328in}}%
\pgfpathlineto{\pgfqpoint{10.304107in}{4.714522in}}%
\pgfpathlineto{\pgfqpoint{10.306389in}{4.729354in}}%
\pgfpathlineto{\pgfqpoint{10.313235in}{4.715724in}}%
\pgfpathlineto{\pgfqpoint{10.315517in}{4.721337in}}%
\pgfpathlineto{\pgfqpoint{10.317799in}{4.728552in}}%
\pgfpathlineto{\pgfqpoint{10.320081in}{4.728151in}}%
\pgfpathlineto{\pgfqpoint{10.322362in}{4.689266in}}%
\pgfpathlineto{\pgfqpoint{10.329208in}{4.676438in}}%
\pgfpathlineto{\pgfqpoint{10.331490in}{4.675436in}}%
\pgfpathlineto{\pgfqpoint{10.333772in}{4.681850in}}%
\pgfpathlineto{\pgfqpoint{10.336054in}{4.690469in}}%
\pgfpathlineto{\pgfqpoint{10.338336in}{4.686661in}}%
\pgfpathlineto{\pgfqpoint{10.345182in}{4.686260in}}%
\pgfpathlineto{\pgfqpoint{10.347463in}{4.684857in}}%
\pgfpathlineto{\pgfqpoint{10.349745in}{4.677040in}}%
\pgfpathlineto{\pgfqpoint{10.352027in}{4.684456in}}%
\pgfpathlineto{\pgfqpoint{10.354309in}{4.677040in}}%
\pgfpathlineto{\pgfqpoint{10.361155in}{4.675436in}}%
\pgfpathlineto{\pgfqpoint{10.363437in}{4.677841in}}%
\pgfpathlineto{\pgfqpoint{10.365719in}{4.686861in}}%
\pgfpathlineto{\pgfqpoint{10.368001in}{4.681650in}}%
\pgfpathlineto{\pgfqpoint{10.370283in}{4.674634in}}%
\pgfpathlineto{\pgfqpoint{10.377128in}{4.672430in}}%
\pgfpathlineto{\pgfqpoint{10.379410in}{4.668621in}}%
\pgfpathlineto{\pgfqpoint{10.381692in}{4.656395in}}%
\pgfpathlineto{\pgfqpoint{10.383974in}{4.659401in}}%
\pgfpathlineto{\pgfqpoint{10.386256in}{4.657998in}}%
\pgfpathlineto{\pgfqpoint{10.395383in}{4.665414in}}%
\pgfpathlineto{\pgfqpoint{10.397665in}{4.655392in}}%
\pgfpathlineto{\pgfqpoint{10.399947in}{4.654791in}}%
\pgfpathlineto{\pgfqpoint{10.402229in}{4.642965in}}%
\pgfpathlineto{\pgfqpoint{10.409075in}{4.640560in}}%
\pgfpathlineto{\pgfqpoint{10.411357in}{4.636952in}}%
\pgfpathlineto{\pgfqpoint{10.413639in}{4.646974in}}%
\pgfpathlineto{\pgfqpoint{10.415921in}{4.662408in}}%
\pgfpathlineto{\pgfqpoint{10.418203in}{4.662207in}}%
\pgfpathlineto{\pgfqpoint{10.425048in}{4.658800in}}%
\pgfpathlineto{\pgfqpoint{10.427330in}{4.662408in}}%
\pgfpathlineto{\pgfqpoint{10.429612in}{4.659401in}}%
\pgfpathlineto{\pgfqpoint{10.431894in}{4.669022in}}%
\pgfpathlineto{\pgfqpoint{10.434176in}{4.664212in}}%
\pgfpathlineto{\pgfqpoint{10.441022in}{4.654791in}}%
\pgfpathlineto{\pgfqpoint{10.443304in}{4.653589in}}%
\pgfpathlineto{\pgfqpoint{10.445585in}{4.624325in}}%
\pgfpathlineto{\pgfqpoint{10.447867in}{4.614102in}}%
\pgfpathlineto{\pgfqpoint{10.450149in}{4.615305in}}%
\pgfpathlineto{\pgfqpoint{10.456995in}{4.607087in}}%
\pgfpathlineto{\pgfqpoint{10.459277in}{4.606486in}}%
\pgfpathlineto{\pgfqpoint{10.461559in}{4.617710in}}%
\pgfpathlineto{\pgfqpoint{10.463841in}{4.622721in}}%
\pgfpathlineto{\pgfqpoint{10.466123in}{4.638155in}}%
\pgfpathlineto{\pgfqpoint{10.472968in}{4.637153in}}%
\pgfpathlineto{\pgfqpoint{10.475250in}{4.639959in}}%
\pgfpathlineto{\pgfqpoint{10.479814in}{4.639758in}}%
\pgfpathlineto{\pgfqpoint{10.482096in}{4.638556in}}%
\pgfpathlineto{\pgfqpoint{10.488942in}{4.642965in}}%
\pgfpathlineto{\pgfqpoint{10.491224in}{4.645771in}}%
\pgfpathlineto{\pgfqpoint{10.493505in}{4.640560in}}%
\pgfpathlineto{\pgfqpoint{10.495787in}{4.677441in}}%
\pgfpathlineto{\pgfqpoint{10.498069in}{4.699689in}}%
\pgfpathlineto{\pgfqpoint{10.504915in}{4.695280in}}%
\pgfpathlineto{\pgfqpoint{10.507197in}{4.703898in}}%
\pgfpathlineto{\pgfqpoint{10.509479in}{4.704299in}}%
\pgfpathlineto{\pgfqpoint{10.511761in}{4.706103in}}%
\pgfpathlineto{\pgfqpoint{10.514043in}{4.702896in}}%
\pgfpathlineto{\pgfqpoint{10.520888in}{4.697484in}}%
\pgfpathlineto{\pgfqpoint{10.523170in}{4.691471in}}%
\pgfpathlineto{\pgfqpoint{10.525452in}{4.691471in}}%
\pgfpathlineto{\pgfqpoint{10.530016in}{4.708709in}}%
\pgfpathlineto{\pgfqpoint{10.536862in}{4.708909in}}%
\pgfpathlineto{\pgfqpoint{10.543707in}{4.683253in}}%
\pgfpathlineto{\pgfqpoint{10.545989in}{4.715323in}}%
\pgfpathlineto{\pgfqpoint{10.552835in}{4.720334in}}%
\pgfpathlineto{\pgfqpoint{10.557399in}{4.735167in}}%
\pgfpathlineto{\pgfqpoint{10.561963in}{4.735968in}}%
\pgfpathlineto{\pgfqpoint{10.568808in}{4.731158in}}%
\pgfpathlineto{\pgfqpoint{10.571090in}{4.735367in}}%
\pgfpathlineto{\pgfqpoint{10.573372in}{4.733764in}}%
\pgfpathlineto{\pgfqpoint{10.575654in}{4.739576in}}%
\pgfpathlineto{\pgfqpoint{10.577936in}{4.739576in}}%
\pgfpathlineto{\pgfqpoint{10.584782in}{4.735568in}}%
\pgfpathlineto{\pgfqpoint{10.587064in}{4.735968in}}%
\pgfpathlineto{\pgfqpoint{10.589346in}{4.737171in}}%
\pgfpathlineto{\pgfqpoint{10.591627in}{4.734565in}}%
\pgfpathlineto{\pgfqpoint{10.593909in}{4.739977in}}%
\pgfpathlineto{\pgfqpoint{10.600755in}{4.741380in}}%
\pgfpathlineto{\pgfqpoint{10.605319in}{4.737572in}}%
\pgfpathlineto{\pgfqpoint{10.607601in}{4.737772in}}%
\pgfpathlineto{\pgfqpoint{10.609883in}{4.742783in}}%
\pgfpathlineto{\pgfqpoint{10.619010in}{4.746191in}}%
\pgfpathlineto{\pgfqpoint{10.621292in}{4.744387in}}%
\pgfpathlineto{\pgfqpoint{10.623574in}{4.759620in}}%
\pgfpathlineto{\pgfqpoint{10.625856in}{4.764230in}}%
\pgfpathlineto{\pgfqpoint{10.632702in}{4.764030in}}%
\pgfpathlineto{\pgfqpoint{10.634984in}{4.760021in}}%
\pgfpathlineto{\pgfqpoint{10.637266in}{4.757415in}}%
\pgfpathlineto{\pgfqpoint{10.639548in}{4.770444in}}%
\pgfpathlineto{\pgfqpoint{10.641829in}{4.774653in}}%
\pgfpathlineto{\pgfqpoint{10.648675in}{4.776056in}}%
\pgfpathlineto{\pgfqpoint{10.650957in}{4.775455in}}%
\pgfpathlineto{\pgfqpoint{10.653239in}{4.776257in}}%
\pgfpathlineto{\pgfqpoint{10.655521in}{4.781067in}}%
\pgfpathlineto{\pgfqpoint{10.657803in}{4.778862in}}%
\pgfpathlineto{\pgfqpoint{10.664648in}{4.784274in}}%
\pgfpathlineto{\pgfqpoint{10.666930in}{4.781468in}}%
\pgfpathlineto{\pgfqpoint{10.669212in}{4.776858in}}%
\pgfpathlineto{\pgfqpoint{10.671494in}{4.776858in}}%
\pgfpathlineto{\pgfqpoint{10.673776in}{4.777860in}}%
\pgfpathlineto{\pgfqpoint{10.680622in}{4.779263in}}%
\pgfpathlineto{\pgfqpoint{10.682904in}{4.768640in}}%
\pgfpathlineto{\pgfqpoint{10.685186in}{4.774653in}}%
\pgfpathlineto{\pgfqpoint{10.689749in}{4.782270in}}%
\pgfpathlineto{\pgfqpoint{10.696595in}{4.773651in}}%
\pgfpathlineto{\pgfqpoint{10.698877in}{4.767036in}}%
\pgfpathlineto{\pgfqpoint{10.701159in}{4.758017in}}%
\pgfpathlineto{\pgfqpoint{10.703441in}{4.755812in}}%
\pgfpathlineto{\pgfqpoint{10.705723in}{4.754810in}}%
\pgfpathlineto{\pgfqpoint{10.714850in}{4.772047in}}%
\pgfpathlineto{\pgfqpoint{10.717132in}{4.788884in}}%
\pgfpathlineto{\pgfqpoint{10.719414in}{4.788884in}}%
\pgfpathlineto{\pgfqpoint{10.721696in}{4.774052in}}%
\pgfpathlineto{\pgfqpoint{10.728542in}{4.773050in}}%
\pgfpathlineto{\pgfqpoint{10.730824in}{4.755812in}}%
\pgfpathlineto{\pgfqpoint{10.733106in}{4.766235in}}%
\pgfpathlineto{\pgfqpoint{10.735388in}{4.798505in}}%
\pgfpathlineto{\pgfqpoint{10.737670in}{4.788283in}}%
\pgfpathlineto{\pgfqpoint{10.749079in}{4.785076in}}%
\pgfpathlineto{\pgfqpoint{10.751361in}{4.785276in}}%
\pgfpathlineto{\pgfqpoint{10.753643in}{4.784274in}}%
\pgfpathlineto{\pgfqpoint{10.760489in}{4.775856in}}%
\pgfpathlineto{\pgfqpoint{10.762770in}{4.780065in}}%
\pgfpathlineto{\pgfqpoint{10.765052in}{4.765834in}}%
\pgfpathlineto{\pgfqpoint{10.767334in}{4.746993in}}%
\pgfpathlineto{\pgfqpoint{10.769616in}{4.748797in}}%
\pgfpathlineto{\pgfqpoint{10.776462in}{4.749799in}}%
\pgfpathlineto{\pgfqpoint{10.778744in}{4.746191in}}%
\pgfpathlineto{\pgfqpoint{10.781026in}{4.740779in}}%
\pgfpathlineto{\pgfqpoint{10.783308in}{4.737372in}}%
\pgfpathlineto{\pgfqpoint{10.785590in}{4.745590in}}%
\pgfpathlineto{\pgfqpoint{10.794717in}{4.769642in}}%
\pgfpathlineto{\pgfqpoint{10.796999in}{4.772448in}}%
\pgfpathlineto{\pgfqpoint{10.801563in}{4.790688in}}%
\pgfpathlineto{\pgfqpoint{10.808409in}{4.788884in}}%
\pgfpathlineto{\pgfqpoint{10.810691in}{4.781067in}}%
\pgfpathlineto{\pgfqpoint{10.812972in}{4.788283in}}%
\pgfpathlineto{\pgfqpoint{10.815254in}{4.789285in}}%
\pgfpathlineto{\pgfqpoint{10.817536in}{4.794697in}}%
\pgfpathlineto{\pgfqpoint{10.824382in}{4.790688in}}%
\pgfpathlineto{\pgfqpoint{10.826664in}{4.786278in}}%
\pgfpathlineto{\pgfqpoint{10.828946in}{4.787882in}}%
\pgfpathlineto{\pgfqpoint{10.831228in}{4.796701in}}%
\pgfpathlineto{\pgfqpoint{10.833510in}{4.802113in}}%
\pgfpathlineto{\pgfqpoint{10.840355in}{4.801111in}}%
\pgfpathlineto{\pgfqpoint{10.842637in}{4.808126in}}%
\pgfpathlineto{\pgfqpoint{10.844919in}{4.809930in}}%
\pgfpathlineto{\pgfqpoint{10.847201in}{4.799507in}}%
\pgfpathlineto{\pgfqpoint{10.849483in}{4.796300in}}%
\pgfpathlineto{\pgfqpoint{10.856329in}{4.796300in}}%
\pgfpathlineto{\pgfqpoint{10.858611in}{4.796902in}}%
\pgfpathlineto{\pgfqpoint{10.860892in}{4.786679in}}%
\pgfpathlineto{\pgfqpoint{10.863174in}{4.794496in}}%
\pgfpathlineto{\pgfqpoint{10.865456in}{4.776657in}}%
\pgfpathlineto{\pgfqpoint{10.872302in}{4.773250in}}%
\pgfpathlineto{\pgfqpoint{10.876866in}{4.764832in}}%
\pgfpathlineto{\pgfqpoint{10.879148in}{4.764431in}}%
\pgfpathlineto{\pgfqpoint{10.881430in}{4.751001in}}%
\pgfpathlineto{\pgfqpoint{10.888275in}{4.763028in}}%
\pgfpathlineto{\pgfqpoint{10.892839in}{4.787481in}}%
\pgfpathlineto{\pgfqpoint{10.895121in}{4.780265in}}%
\pgfpathlineto{\pgfqpoint{10.897403in}{4.782671in}}%
\pgfpathlineto{\pgfqpoint{10.904249in}{4.774853in}}%
\pgfpathlineto{\pgfqpoint{10.906531in}{4.773250in}}%
\pgfpathlineto{\pgfqpoint{10.911094in}{4.785477in}}%
\pgfpathlineto{\pgfqpoint{10.913376in}{4.784675in}}%
\pgfpathlineto{\pgfqpoint{10.920222in}{4.791490in}}%
\pgfpathlineto{\pgfqpoint{10.922504in}{4.795499in}}%
\pgfpathlineto{\pgfqpoint{10.924786in}{4.795899in}}%
\pgfpathlineto{\pgfqpoint{10.929350in}{4.805721in}}%
\pgfpathlineto{\pgfqpoint{10.938477in}{4.805921in}}%
\pgfpathlineto{\pgfqpoint{10.940759in}{4.817948in}}%
\pgfpathlineto{\pgfqpoint{10.943041in}{4.813538in}}%
\pgfpathlineto{\pgfqpoint{10.945323in}{4.860040in}}%
\pgfpathlineto{\pgfqpoint{10.952169in}{4.868658in}}%
\pgfpathlineto{\pgfqpoint{10.954451in}{4.888903in}}%
\pgfpathlineto{\pgfqpoint{10.959014in}{4.891508in}}%
\pgfpathlineto{\pgfqpoint{10.961296in}{4.903935in}}%
\pgfpathlineto{\pgfqpoint{10.968142in}{4.892911in}}%
\pgfpathlineto{\pgfqpoint{10.972706in}{4.907744in}}%
\pgfpathlineto{\pgfqpoint{10.974988in}{4.907744in}}%
\pgfpathlineto{\pgfqpoint{10.977270in}{4.901129in}}%
\pgfpathlineto{\pgfqpoint{10.984115in}{4.902132in}}%
\pgfpathlineto{\pgfqpoint{10.986397in}{4.904938in}}%
\pgfpathlineto{\pgfqpoint{10.988679in}{4.914559in}}%
\pgfpathlineto{\pgfqpoint{10.990961in}{4.918167in}}%
\pgfpathlineto{\pgfqpoint{10.993243in}{4.914158in}}%
\pgfpathlineto{\pgfqpoint{11.002371in}{4.927988in}}%
\pgfpathlineto{\pgfqpoint{11.004653in}{4.922977in}}%
\pgfpathlineto{\pgfqpoint{11.009216in}{4.921574in}}%
\pgfpathlineto{\pgfqpoint{11.016062in}{4.903935in}}%
\pgfpathlineto{\pgfqpoint{11.018344in}{4.906541in}}%
\pgfpathlineto{\pgfqpoint{11.020626in}{4.917565in}}%
\pgfpathlineto{\pgfqpoint{11.022908in}{4.908345in}}%
\pgfpathlineto{\pgfqpoint{11.025190in}{4.913356in}}%
\pgfpathlineto{\pgfqpoint{11.032036in}{4.916563in}}%
\pgfpathlineto{\pgfqpoint{11.034317in}{4.922576in}}%
\pgfpathlineto{\pgfqpoint{11.036599in}{4.924981in}}%
\pgfpathlineto{\pgfqpoint{11.038881in}{4.920171in}}%
\pgfpathlineto{\pgfqpoint{11.041163in}{4.924781in}}%
\pgfpathlineto{\pgfqpoint{11.048009in}{4.927587in}}%
\pgfpathlineto{\pgfqpoint{11.050291in}{4.924581in}}%
\pgfpathlineto{\pgfqpoint{11.052573in}{4.918367in}}%
\pgfpathlineto{\pgfqpoint{11.054855in}{4.928188in}}%
\pgfpathlineto{\pgfqpoint{11.057136in}{4.910951in}}%
\pgfpathlineto{\pgfqpoint{11.063982in}{4.907944in}}%
\pgfpathlineto{\pgfqpoint{11.066264in}{4.923578in}}%
\pgfpathlineto{\pgfqpoint{11.068546in}{4.931195in}}%
\pgfpathlineto{\pgfqpoint{11.070828in}{4.925182in}}%
\pgfpathlineto{\pgfqpoint{11.073110in}{4.926184in}}%
\pgfpathlineto{\pgfqpoint{11.079956in}{4.916563in}}%
\pgfpathlineto{\pgfqpoint{11.082237in}{4.928389in}}%
\pgfpathlineto{\pgfqpoint{11.084519in}{4.922777in}}%
\pgfpathlineto{\pgfqpoint{11.086801in}{4.922777in}}%
\pgfpathlineto{\pgfqpoint{11.095929in}{4.933199in}}%
\pgfpathlineto{\pgfqpoint{11.098211in}{4.954045in}}%
\pgfpathlineto{\pgfqpoint{11.100493in}{4.940215in}}%
\pgfpathlineto{\pgfqpoint{11.102775in}{4.947030in}}%
\pgfpathlineto{\pgfqpoint{11.105057in}{4.942019in}}%
\pgfpathlineto{\pgfqpoint{11.111902in}{4.951239in}}%
\pgfpathlineto{\pgfqpoint{11.114184in}{4.942420in}}%
\pgfpathlineto{\pgfqpoint{11.116466in}{4.958655in}}%
\pgfpathlineto{\pgfqpoint{11.121030in}{4.972285in}}%
\pgfpathlineto{\pgfqpoint{11.127876in}{4.965269in}}%
\pgfpathlineto{\pgfqpoint{11.130157in}{4.972686in}}%
\pgfpathlineto{\pgfqpoint{11.132439in}{4.955648in}}%
\pgfpathlineto{\pgfqpoint{11.134721in}{4.968877in}}%
\pgfpathlineto{\pgfqpoint{11.137003in}{4.994533in}}%
\pgfpathlineto{\pgfqpoint{11.143849in}{4.994133in}}%
\pgfpathlineto{\pgfqpoint{11.146131in}{5.012773in}}%
\pgfpathlineto{\pgfqpoint{11.148413in}{4.932999in}}%
\pgfpathlineto{\pgfqpoint{11.150695in}{4.924180in}}%
\pgfpathlineto{\pgfqpoint{11.152977in}{4.932799in}}%
\pgfpathlineto{\pgfqpoint{11.159822in}{4.942219in}}%
\pgfpathlineto{\pgfqpoint{11.162104in}{4.927186in}}%
\pgfpathlineto{\pgfqpoint{11.164386in}{4.930794in}}%
\pgfpathlineto{\pgfqpoint{11.166668in}{4.943622in}}%
\pgfpathlineto{\pgfqpoint{11.168950in}{4.950838in}}%
\pgfpathlineto{\pgfqpoint{11.175796in}{4.943422in}}%
\pgfpathlineto{\pgfqpoint{11.178078in}{4.946629in}}%
\pgfpathlineto{\pgfqpoint{11.180359in}{4.954847in}}%
\pgfpathlineto{\pgfqpoint{11.182641in}{4.951038in}}%
\pgfpathlineto{\pgfqpoint{11.184923in}{4.960259in}}%
\pgfpathlineto{\pgfqpoint{11.191769in}{4.957452in}}%
\pgfpathlineto{\pgfqpoint{11.194051in}{4.971884in}}%
\pgfpathlineto{\pgfqpoint{11.196333in}{4.991928in}}%
\pgfpathlineto{\pgfqpoint{11.198615in}{4.984111in}}%
\pgfpathlineto{\pgfqpoint{11.200897in}{4.981104in}}%
\pgfpathlineto{\pgfqpoint{11.210024in}{4.960860in}}%
\pgfpathlineto{\pgfqpoint{11.212306in}{4.961261in}}%
\pgfpathlineto{\pgfqpoint{11.214588in}{4.959256in}}%
\pgfpathlineto{\pgfqpoint{11.216870in}{5.011771in}}%
\pgfpathlineto{\pgfqpoint{11.225998in}{4.994934in}}%
\pgfpathlineto{\pgfqpoint{11.228279in}{4.962463in}}%
\pgfpathlineto{\pgfqpoint{11.230561in}{4.965670in}}%
\pgfpathlineto{\pgfqpoint{11.232843in}{4.926986in}}%
\pgfpathlineto{\pgfqpoint{11.239689in}{4.942019in}}%
\pgfpathlineto{\pgfqpoint{11.241971in}{4.937208in}}%
\pgfpathlineto{\pgfqpoint{11.244253in}{4.925382in}}%
\pgfpathlineto{\pgfqpoint{11.246535in}{4.927988in}}%
\pgfpathlineto{\pgfqpoint{11.248817in}{4.943422in}}%
\pgfpathlineto{\pgfqpoint{11.255662in}{4.948834in}}%
\pgfpathlineto{\pgfqpoint{11.257944in}{4.949635in}}%
\pgfpathlineto{\pgfqpoint{11.260226in}{4.941017in}}%
\pgfpathlineto{\pgfqpoint{11.262508in}{4.954646in}}%
\pgfpathlineto{\pgfqpoint{11.264790in}{4.948032in}}%
\pgfpathlineto{\pgfqpoint{11.271636in}{4.963466in}}%
\pgfpathlineto{\pgfqpoint{11.273918in}{4.964869in}}%
\pgfpathlineto{\pgfqpoint{11.276200in}{4.941217in}}%
\pgfpathlineto{\pgfqpoint{11.278481in}{4.902132in}}%
\pgfpathlineto{\pgfqpoint{11.280763in}{4.938210in}}%
\pgfpathlineto{\pgfqpoint{11.287609in}{4.924180in}}%
\pgfpathlineto{\pgfqpoint{11.289891in}{4.925984in}}%
\pgfpathlineto{\pgfqpoint{11.292173in}{4.937609in}}%
\pgfpathlineto{\pgfqpoint{11.294455in}{4.942219in}}%
\pgfpathlineto{\pgfqpoint{11.296737in}{4.930794in}}%
\pgfpathlineto{\pgfqpoint{11.305864in}{4.954847in}}%
\pgfpathlineto{\pgfqpoint{11.308146in}{4.957052in}}%
\pgfpathlineto{\pgfqpoint{11.312710in}{4.953243in}}%
\pgfpathlineto{\pgfqpoint{11.319556in}{4.960259in}}%
\pgfpathlineto{\pgfqpoint{11.321838in}{4.973688in}}%
\pgfpathlineto{\pgfqpoint{11.324120in}{4.977496in}}%
\pgfpathlineto{\pgfqpoint{11.328683in}{5.006359in}}%
\pgfpathlineto{\pgfqpoint{11.335529in}{5.005758in}}%
\pgfpathlineto{\pgfqpoint{11.337811in}{4.996738in}}%
\pgfpathlineto{\pgfqpoint{11.340093in}{4.992128in}}%
\pgfpathlineto{\pgfqpoint{11.342375in}{4.995536in}}%
\pgfpathlineto{\pgfqpoint{11.344657in}{5.013575in}}%
\pgfpathlineto{\pgfqpoint{11.351502in}{5.010969in}}%
\pgfpathlineto{\pgfqpoint{11.353784in}{5.006560in}}%
\pgfpathlineto{\pgfqpoint{11.356066in}{4.995135in}}%
\pgfpathlineto{\pgfqpoint{11.358348in}{4.997540in}}%
\pgfpathlineto{\pgfqpoint{11.360630in}{4.997139in}}%
\pgfpathlineto{\pgfqpoint{11.367476in}{4.992128in}}%
\pgfpathlineto{\pgfqpoint{11.369758in}{4.997941in}}%
\pgfpathlineto{\pgfqpoint{11.372040in}{4.995536in}}%
\pgfpathlineto{\pgfqpoint{11.374322in}{5.017985in}}%
\pgfpathlineto{\pgfqpoint{11.376603in}{5.012172in}}%
\pgfpathlineto{\pgfqpoint{11.383449in}{5.013976in}}%
\pgfpathlineto{\pgfqpoint{11.385731in}{5.019588in}}%
\pgfpathlineto{\pgfqpoint{11.390295in}{5.026804in}}%
\pgfpathlineto{\pgfqpoint{11.392577in}{5.017985in}}%
\pgfpathlineto{\pgfqpoint{11.399423in}{5.017584in}}%
\pgfpathlineto{\pgfqpoint{11.401704in}{5.018386in}}%
\pgfpathlineto{\pgfqpoint{11.403986in}{5.011571in}}%
\pgfpathlineto{\pgfqpoint{11.406268in}{4.998542in}}%
\pgfpathlineto{\pgfqpoint{11.408550in}{4.988921in}}%
\pgfpathlineto{\pgfqpoint{11.415396in}{4.983710in}}%
\pgfpathlineto{\pgfqpoint{11.417678in}{4.982708in}}%
\pgfpathlineto{\pgfqpoint{11.419960in}{4.980302in}}%
\pgfpathlineto{\pgfqpoint{11.422242in}{4.986716in}}%
\pgfpathlineto{\pgfqpoint{11.424523in}{4.990725in}}%
\pgfpathlineto{\pgfqpoint{11.433651in}{4.956851in}}%
\pgfpathlineto{\pgfqpoint{11.435933in}{4.938010in}}%
\pgfpathlineto{\pgfqpoint{11.440497in}{4.955448in}}%
\pgfpathlineto{\pgfqpoint{11.449624in}{4.953043in}}%
\pgfpathlineto{\pgfqpoint{11.451906in}{4.948834in}}%
\pgfpathlineto{\pgfqpoint{11.454188in}{4.942820in}}%
\pgfpathlineto{\pgfqpoint{11.456470in}{4.943021in}}%
\pgfpathlineto{\pgfqpoint{11.463316in}{4.959858in}}%
\pgfpathlineto{\pgfqpoint{11.465598in}{4.956450in}}%
\pgfpathlineto{\pgfqpoint{11.467880in}{4.961862in}}%
\pgfpathlineto{\pgfqpoint{11.470162in}{4.961662in}}%
\pgfpathlineto{\pgfqpoint{11.472444in}{4.974690in}}%
\pgfpathlineto{\pgfqpoint{11.479289in}{4.993331in}}%
\pgfpathlineto{\pgfqpoint{11.481571in}{4.987719in}}%
\pgfpathlineto{\pgfqpoint{11.483853in}{4.995536in}}%
\pgfpathlineto{\pgfqpoint{11.486135in}{4.992529in}}%
\pgfpathlineto{\pgfqpoint{11.488417in}{4.980503in}}%
\pgfpathlineto{\pgfqpoint{11.495263in}{4.978699in}}%
\pgfpathlineto{\pgfqpoint{11.497545in}{4.967675in}}%
\pgfpathlineto{\pgfqpoint{11.499826in}{4.952642in}}%
\pgfpathlineto{\pgfqpoint{11.502108in}{4.958054in}}%
\pgfpathlineto{\pgfqpoint{11.504390in}{4.944023in}}%
\pgfpathlineto{\pgfqpoint{11.511236in}{4.917164in}}%
\pgfpathlineto{\pgfqpoint{11.513518in}{4.926585in}}%
\pgfpathlineto{\pgfqpoint{11.515800in}{4.922777in}}%
\pgfpathlineto{\pgfqpoint{11.518082in}{4.921173in}}%
\pgfpathlineto{\pgfqpoint{11.520364in}{4.924180in}}%
\pgfpathlineto{\pgfqpoint{11.527209in}{4.917966in}}%
\pgfpathlineto{\pgfqpoint{11.531773in}{4.939814in}}%
\pgfpathlineto{\pgfqpoint{11.534055in}{4.955448in}}%
\pgfpathlineto{\pgfqpoint{11.536337in}{4.965269in}}%
\pgfpathlineto{\pgfqpoint{11.543183in}{4.969679in}}%
\pgfpathlineto{\pgfqpoint{11.545465in}{4.950638in}}%
\pgfpathlineto{\pgfqpoint{11.550028in}{4.991326in}}%
\pgfpathlineto{\pgfqpoint{11.552310in}{4.991126in}}%
\pgfpathlineto{\pgfqpoint{11.559156in}{4.983710in}}%
\pgfpathlineto{\pgfqpoint{11.561438in}{5.007361in}}%
\pgfpathlineto{\pgfqpoint{11.563720in}{5.016381in}}%
\pgfpathlineto{\pgfqpoint{11.566002in}{5.011370in}}%
\pgfpathlineto{\pgfqpoint{11.568284in}{5.001148in}}%
\pgfpathlineto{\pgfqpoint{11.575129in}{5.022996in}}%
\pgfpathlineto{\pgfqpoint{11.577411in}{5.042438in}}%
\pgfpathlineto{\pgfqpoint{11.581975in}{5.013575in}}%
\pgfpathlineto{\pgfqpoint{11.584257in}{5.020390in}}%
\pgfpathlineto{\pgfqpoint{11.593385in}{5.025000in}}%
\pgfpathlineto{\pgfqpoint{11.595666in}{5.048251in}}%
\pgfpathlineto{\pgfqpoint{11.597948in}{5.040033in}}%
\pgfpathlineto{\pgfqpoint{11.600230in}{5.043240in}}%
\pgfpathlineto{\pgfqpoint{11.607076in}{5.039031in}}%
\pgfpathlineto{\pgfqpoint{11.609358in}{5.048050in}}%
\pgfpathlineto{\pgfqpoint{11.616204in}{5.082325in}}%
\pgfpathlineto{\pgfqpoint{11.623049in}{5.077916in}}%
\pgfpathlineto{\pgfqpoint{11.625331in}{5.073907in}}%
\pgfpathlineto{\pgfqpoint{11.627613in}{5.079920in}}%
\pgfpathlineto{\pgfqpoint{11.629895in}{5.079519in}}%
\pgfpathlineto{\pgfqpoint{11.632177in}{5.084330in}}%
\pgfpathlineto{\pgfqpoint{11.639023in}{5.091946in}}%
\pgfpathlineto{\pgfqpoint{11.641305in}{5.079118in}}%
\pgfpathlineto{\pgfqpoint{11.643587in}{5.072905in}}%
\pgfpathlineto{\pgfqpoint{11.648150in}{5.071902in}}%
\pgfpathlineto{\pgfqpoint{11.654996in}{5.057271in}}%
\pgfpathlineto{\pgfqpoint{11.657278in}{5.065488in}}%
\pgfpathlineto{\pgfqpoint{11.659560in}{5.057271in}}%
\pgfpathlineto{\pgfqpoint{11.661842in}{5.046246in}}%
\pgfpathlineto{\pgfqpoint{11.664124in}{5.077114in}}%
\pgfpathlineto{\pgfqpoint{11.670969in}{5.081323in}}%
\pgfpathlineto{\pgfqpoint{11.673251in}{5.062482in}}%
\pgfpathlineto{\pgfqpoint{11.675533in}{5.068695in}}%
\pgfpathlineto{\pgfqpoint{11.677815in}{5.039632in}}%
\pgfpathlineto{\pgfqpoint{11.680097in}{5.040835in}}%
\pgfpathlineto{\pgfqpoint{11.686943in}{5.029410in}}%
\pgfpathlineto{\pgfqpoint{11.689225in}{5.016381in}}%
\pgfpathlineto{\pgfqpoint{11.691507in}{5.041035in}}%
\pgfpathlineto{\pgfqpoint{11.693788in}{5.033018in}}%
\pgfpathlineto{\pgfqpoint{11.696070in}{5.032015in}}%
\pgfpathlineto{\pgfqpoint{11.702916in}{5.022996in}}%
\pgfpathlineto{\pgfqpoint{11.705198in}{5.022996in}}%
\pgfpathlineto{\pgfqpoint{11.712044in}{5.034821in}}%
\pgfpathlineto{\pgfqpoint{11.718889in}{5.034621in}}%
\pgfpathlineto{\pgfqpoint{11.721171in}{5.024800in}}%
\pgfpathlineto{\pgfqpoint{11.725735in}{5.010168in}}%
\pgfpathlineto{\pgfqpoint{11.728017in}{5.008564in}}%
\pgfpathlineto{\pgfqpoint{11.734863in}{5.011771in}}%
\pgfpathlineto{\pgfqpoint{11.737145in}{5.025000in}}%
\pgfpathlineto{\pgfqpoint{11.739427in}{5.005157in}}%
\pgfpathlineto{\pgfqpoint{11.741709in}{5.008163in}}%
\pgfpathlineto{\pgfqpoint{11.743990in}{5.006159in}}%
\pgfpathlineto{\pgfqpoint{11.750836in}{5.001148in}}%
\pgfpathlineto{\pgfqpoint{11.753118in}{5.015579in}}%
\pgfpathlineto{\pgfqpoint{11.755400in}{5.014377in}}%
\pgfpathlineto{\pgfqpoint{11.757682in}{5.010368in}}%
\pgfpathlineto{\pgfqpoint{11.759964in}{4.998943in}}%
\pgfpathlineto{\pgfqpoint{11.769091in}{5.004154in}}%
\pgfpathlineto{\pgfqpoint{11.771373in}{4.998141in}}%
\pgfpathlineto{\pgfqpoint{11.773655in}{4.981305in}}%
\pgfpathlineto{\pgfqpoint{11.775937in}{4.997340in}}%
\pgfpathlineto{\pgfqpoint{11.782783in}{4.985313in}}%
\pgfpathlineto{\pgfqpoint{11.785065in}{4.996137in}}%
\pgfpathlineto{\pgfqpoint{11.787347in}{4.971283in}}%
\pgfpathlineto{\pgfqpoint{11.789629in}{4.958254in}}%
\pgfpathlineto{\pgfqpoint{11.791910in}{4.954045in}}%
\pgfpathlineto{\pgfqpoint{11.798756in}{4.938411in}}%
\pgfpathlineto{\pgfqpoint{11.807884in}{4.965670in}}%
\pgfpathlineto{\pgfqpoint{11.814730in}{4.977897in}}%
\pgfpathlineto{\pgfqpoint{11.817011in}{4.991527in}}%
\pgfpathlineto{\pgfqpoint{11.819293in}{4.968677in}}%
\pgfpathlineto{\pgfqpoint{11.821575in}{4.973888in}}%
\pgfpathlineto{\pgfqpoint{11.823857in}{5.001148in}}%
\pgfpathlineto{\pgfqpoint{11.832985in}{4.976895in}}%
\pgfpathlineto{\pgfqpoint{11.835267in}{4.979701in}}%
\pgfpathlineto{\pgfqpoint{11.837549in}{4.975893in}}%
\pgfpathlineto{\pgfqpoint{11.839831in}{4.976895in}}%
\pgfpathlineto{\pgfqpoint{11.846676in}{4.974891in}}%
\pgfpathlineto{\pgfqpoint{11.848958in}{4.979300in}}%
\pgfpathlineto{\pgfqpoint{11.851240in}{4.974891in}}%
\pgfpathlineto{\pgfqpoint{11.853522in}{4.980503in}}%
\pgfpathlineto{\pgfqpoint{11.855804in}{4.988520in}}%
\pgfpathlineto{\pgfqpoint{11.862650in}{4.968477in}}%
\pgfpathlineto{\pgfqpoint{11.864932in}{4.984912in}}%
\pgfpathlineto{\pgfqpoint{11.867213in}{4.974289in}}%
\pgfpathlineto{\pgfqpoint{11.869495in}{4.978298in}}%
\pgfpathlineto{\pgfqpoint{11.871777in}{4.983710in}}%
\pgfpathlineto{\pgfqpoint{11.878623in}{4.986315in}}%
\pgfpathlineto{\pgfqpoint{11.880905in}{4.993331in}}%
\pgfpathlineto{\pgfqpoint{11.883187in}{4.998342in}}%
\pgfpathlineto{\pgfqpoint{11.885469in}{4.997540in}}%
\pgfpathlineto{\pgfqpoint{11.887751in}{4.993932in}}%
\pgfpathlineto{\pgfqpoint{11.894596in}{5.008564in}}%
\pgfpathlineto{\pgfqpoint{11.896878in}{5.007361in}}%
\pgfpathlineto{\pgfqpoint{11.899160in}{4.990525in}}%
\pgfpathlineto{\pgfqpoint{11.903724in}{4.975291in}}%
\pgfpathlineto{\pgfqpoint{11.912852in}{5.007963in}}%
\pgfpathlineto{\pgfqpoint{11.915133in}{5.002751in}}%
\pgfpathlineto{\pgfqpoint{11.919697in}{5.007161in}}%
\pgfpathlineto{\pgfqpoint{11.926543in}{5.021392in}}%
\pgfpathlineto{\pgfqpoint{11.931107in}{5.013375in}}%
\pgfpathlineto{\pgfqpoint{11.933389in}{5.012974in}}%
\pgfpathlineto{\pgfqpoint{11.935671in}{5.007361in}}%
\pgfpathlineto{\pgfqpoint{11.942516in}{5.019588in}}%
\pgfpathlineto{\pgfqpoint{11.944798in}{5.034020in}}%
\pgfpathlineto{\pgfqpoint{11.947080in}{5.036425in}}%
\pgfpathlineto{\pgfqpoint{11.951644in}{5.023998in}}%
\pgfpathlineto{\pgfqpoint{11.960772in}{5.025401in}}%
\pgfpathlineto{\pgfqpoint{11.963054in}{5.039231in}}%
\pgfpathlineto{\pgfqpoint{11.965335in}{5.041035in}}%
\pgfpathlineto{\pgfqpoint{11.974463in}{5.037628in}}%
\pgfpathlineto{\pgfqpoint{11.979027in}{5.028207in}}%
\pgfpathlineto{\pgfqpoint{11.981309in}{5.041636in}}%
\pgfpathlineto{\pgfqpoint{11.983591in}{5.046046in}}%
\pgfpathlineto{\pgfqpoint{11.990436in}{5.072303in}}%
\pgfpathlineto{\pgfqpoint{11.992718in}{5.063685in}}%
\pgfpathlineto{\pgfqpoint{11.995000in}{5.065488in}}%
\pgfpathlineto{\pgfqpoint{11.997282in}{5.062081in}}%
\pgfpathlineto{\pgfqpoint{11.999564in}{5.054865in}}%
\pgfpathlineto{\pgfqpoint{12.006410in}{5.051057in}}%
\pgfpathlineto{\pgfqpoint{12.008692in}{5.041235in}}%
\pgfpathlineto{\pgfqpoint{12.010974in}{5.057671in}}%
\pgfpathlineto{\pgfqpoint{12.013255in}{5.058874in}}%
\pgfpathlineto{\pgfqpoint{12.015537in}{5.062883in}}%
\pgfpathlineto{\pgfqpoint{12.022383in}{5.051658in}}%
\pgfpathlineto{\pgfqpoint{12.024665in}{5.049453in}}%
\pgfpathlineto{\pgfqpoint{12.026947in}{5.042037in}}%
\pgfpathlineto{\pgfqpoint{12.029229in}{5.031013in}}%
\pgfpathlineto{\pgfqpoint{12.031511in}{5.027606in}}%
\pgfpathlineto{\pgfqpoint{12.038356in}{5.020791in}}%
\pgfpathlineto{\pgfqpoint{12.042920in}{5.030412in}}%
\pgfpathlineto{\pgfqpoint{12.045202in}{5.033218in}}%
\pgfpathlineto{\pgfqpoint{12.047484in}{5.030813in}}%
\pgfpathlineto{\pgfqpoint{12.056612in}{5.023396in}}%
\pgfpathlineto{\pgfqpoint{12.058894in}{5.023797in}}%
\pgfpathlineto{\pgfqpoint{12.061175in}{5.028808in}}%
\pgfpathlineto{\pgfqpoint{12.063457in}{5.035423in}}%
\pgfpathlineto{\pgfqpoint{12.070303in}{5.028207in}}%
\pgfpathlineto{\pgfqpoint{12.072585in}{5.020791in}}%
\pgfpathlineto{\pgfqpoint{12.074867in}{5.019588in}}%
\pgfpathlineto{\pgfqpoint{12.077149in}{5.023998in}}%
\pgfpathlineto{\pgfqpoint{12.079431in}{5.022595in}}%
\pgfpathlineto{\pgfqpoint{12.086276in}{5.023597in}}%
\pgfpathlineto{\pgfqpoint{12.088558in}{5.024599in}}%
\pgfpathlineto{\pgfqpoint{12.090840in}{5.022996in}}%
\pgfpathlineto{\pgfqpoint{12.093122in}{5.017985in}}%
\pgfpathlineto{\pgfqpoint{12.095404in}{5.015980in}}%
\pgfpathlineto{\pgfqpoint{12.102250in}{5.016782in}}%
\pgfpathlineto{\pgfqpoint{12.104532in}{5.014778in}}%
\pgfpathlineto{\pgfqpoint{12.106814in}{5.018386in}}%
\pgfpathlineto{\pgfqpoint{12.109096in}{5.026203in}}%
\pgfpathlineto{\pgfqpoint{12.111377in}{5.020791in}}%
\pgfpathlineto{\pgfqpoint{12.118223in}{5.013775in}}%
\pgfpathlineto{\pgfqpoint{12.120505in}{5.005157in}}%
\pgfpathlineto{\pgfqpoint{12.122787in}{5.010368in}}%
\pgfpathlineto{\pgfqpoint{12.125069in}{4.998542in}}%
\pgfpathlineto{\pgfqpoint{12.127351in}{5.005357in}}%
\pgfpathlineto{\pgfqpoint{12.134197in}{4.997941in}}%
\pgfpathlineto{\pgfqpoint{12.136478in}{5.009767in}}%
\pgfpathlineto{\pgfqpoint{12.138760in}{5.014577in}}%
\pgfpathlineto{\pgfqpoint{12.141042in}{5.021593in}}%
\pgfpathlineto{\pgfqpoint{12.150170in}{5.025200in}}%
\pgfpathlineto{\pgfqpoint{12.152452in}{5.031614in}}%
\pgfpathlineto{\pgfqpoint{12.154734in}{5.050456in}}%
\pgfpathlineto{\pgfqpoint{12.157016in}{5.049453in}}%
\pgfpathlineto{\pgfqpoint{12.159297in}{5.041837in}}%
\pgfpathlineto{\pgfqpoint{12.166143in}{5.044643in}}%
\pgfpathlineto{\pgfqpoint{12.168425in}{5.043641in}}%
\pgfpathlineto{\pgfqpoint{12.170707in}{5.049053in}}%
\pgfpathlineto{\pgfqpoint{12.172989in}{5.039832in}}%
\pgfpathlineto{\pgfqpoint{12.175271in}{5.042438in}}%
\pgfpathlineto{\pgfqpoint{12.182117in}{5.037828in}}%
\pgfpathlineto{\pgfqpoint{12.184398in}{5.034621in}}%
\pgfpathlineto{\pgfqpoint{12.186680in}{5.032817in}}%
\pgfpathlineto{\pgfqpoint{12.188962in}{5.037427in}}%
\pgfpathlineto{\pgfqpoint{12.191244in}{5.025401in}}%
\pgfpathlineto{\pgfqpoint{12.198090in}{5.020189in}}%
\pgfpathlineto{\pgfqpoint{12.204936in}{4.986917in}}%
\pgfpathlineto{\pgfqpoint{12.207218in}{5.027004in}}%
\pgfpathlineto{\pgfqpoint{12.214063in}{5.019588in}}%
\pgfpathlineto{\pgfqpoint{12.216345in}{5.022795in}}%
\pgfpathlineto{\pgfqpoint{12.218627in}{5.051658in}}%
\pgfpathlineto{\pgfqpoint{12.220909in}{5.035423in}}%
\pgfpathlineto{\pgfqpoint{12.223191in}{5.049253in}}%
\pgfpathlineto{\pgfqpoint{12.230037in}{5.058473in}}%
\pgfpathlineto{\pgfqpoint{12.232319in}{5.057471in}}%
\pgfpathlineto{\pgfqpoint{12.234600in}{5.058473in}}%
\pgfpathlineto{\pgfqpoint{12.236882in}{5.066491in}}%
\pgfpathlineto{\pgfqpoint{12.239164in}{5.063484in}}%
\pgfpathlineto{\pgfqpoint{12.248292in}{5.079319in}}%
\pgfpathlineto{\pgfqpoint{12.250574in}{5.081323in}}%
\pgfpathlineto{\pgfqpoint{12.252856in}{5.089341in}}%
\pgfpathlineto{\pgfqpoint{12.255138in}{5.091345in}}%
\pgfpathlineto{\pgfqpoint{12.261983in}{5.093951in}}%
\pgfpathlineto{\pgfqpoint{12.264265in}{5.091145in}}%
\pgfpathlineto{\pgfqpoint{12.266547in}{5.089942in}}%
\pgfpathlineto{\pgfqpoint{12.268829in}{5.085131in}}%
\pgfpathlineto{\pgfqpoint{12.271111in}{5.086534in}}%
\pgfpathlineto{\pgfqpoint{12.280239in}{5.084129in}}%
\pgfpathlineto{\pgfqpoint{12.282520in}{5.082927in}}%
\pgfpathlineto{\pgfqpoint{12.284802in}{5.096556in}}%
\pgfpathlineto{\pgfqpoint{12.287084in}{5.097759in}}%
\pgfpathlineto{\pgfqpoint{12.293930in}{5.090142in}}%
\pgfpathlineto{\pgfqpoint{12.296212in}{5.084330in}}%
\pgfpathlineto{\pgfqpoint{12.298494in}{5.095554in}}%
\pgfpathlineto{\pgfqpoint{12.300776in}{5.092948in}}%
\pgfpathlineto{\pgfqpoint{12.303058in}{5.088940in}}%
\pgfpathlineto{\pgfqpoint{12.312185in}{5.103171in}}%
\pgfpathlineto{\pgfqpoint{12.314467in}{5.104574in}}%
\pgfpathlineto{\pgfqpoint{12.316749in}{5.104975in}}%
\pgfpathlineto{\pgfqpoint{12.319031in}{5.109785in}}%
\pgfpathlineto{\pgfqpoint{12.325877in}{5.115798in}}%
\pgfpathlineto{\pgfqpoint{12.328159in}{5.109384in}}%
\pgfpathlineto{\pgfqpoint{12.330441in}{5.123014in}}%
\pgfpathlineto{\pgfqpoint{12.332722in}{5.107380in}}%
\pgfpathlineto{\pgfqpoint{12.335004in}{5.111790in}}%
\pgfpathlineto{\pgfqpoint{12.341850in}{5.109384in}}%
\pgfpathlineto{\pgfqpoint{12.346414in}{5.086935in}}%
\pgfpathlineto{\pgfqpoint{12.348696in}{5.085332in}}%
\pgfpathlineto{\pgfqpoint{12.350978in}{5.098160in}}%
\pgfpathlineto{\pgfqpoint{12.357823in}{5.094352in}}%
\pgfpathlineto{\pgfqpoint{12.360105in}{5.087336in}}%
\pgfpathlineto{\pgfqpoint{12.362387in}{5.104774in}}%
\pgfpathlineto{\pgfqpoint{12.364669in}{5.095955in}}%
\pgfpathlineto{\pgfqpoint{12.366951in}{5.113594in}}%
\pgfpathlineto{\pgfqpoint{12.373797in}{5.090944in}}%
\pgfpathlineto{\pgfqpoint{12.376079in}{5.093951in}}%
\pgfpathlineto{\pgfqpoint{12.380642in}{5.071502in}}%
\pgfpathlineto{\pgfqpoint{12.382924in}{5.089140in}}%
\pgfpathlineto{\pgfqpoint{12.389770in}{5.104774in}}%
\pgfpathlineto{\pgfqpoint{12.392052in}{5.112191in}}%
\pgfpathlineto{\pgfqpoint{12.394334in}{5.116801in}}%
\pgfpathlineto{\pgfqpoint{12.396616in}{5.099563in}}%
\pgfpathlineto{\pgfqpoint{12.398898in}{5.133036in}}%
\pgfpathlineto{\pgfqpoint{12.405743in}{5.146666in}}%
\pgfpathlineto{\pgfqpoint{12.408025in}{5.155686in}}%
\pgfpathlineto{\pgfqpoint{12.410307in}{5.156888in}}%
\pgfpathlineto{\pgfqpoint{12.414871in}{5.169516in}}%
\pgfpathlineto{\pgfqpoint{12.421717in}{5.171520in}}%
\pgfpathlineto{\pgfqpoint{12.423999in}{5.193368in}}%
\pgfpathlineto{\pgfqpoint{12.426281in}{5.199581in}}%
\pgfpathlineto{\pgfqpoint{12.428563in}{5.197778in}}%
\pgfpathlineto{\pgfqpoint{12.430844in}{5.201786in}}%
\pgfpathlineto{\pgfqpoint{12.437690in}{5.207399in}}%
\pgfpathlineto{\pgfqpoint{12.439972in}{5.211207in}}%
\pgfpathlineto{\pgfqpoint{12.442254in}{5.207799in}}%
\pgfpathlineto{\pgfqpoint{12.444536in}{5.192566in}}%
\pgfpathlineto{\pgfqpoint{12.446818in}{5.184148in}}%
\pgfpathlineto{\pgfqpoint{12.453663in}{5.179538in}}%
\pgfpathlineto{\pgfqpoint{12.455945in}{5.181342in}}%
\pgfpathlineto{\pgfqpoint{12.458227in}{5.194971in}}%
\pgfpathlineto{\pgfqpoint{12.460509in}{5.190361in}}%
\pgfpathlineto{\pgfqpoint{12.462791in}{5.192566in}}%
\pgfpathlineto{\pgfqpoint{12.469637in}{5.184148in}}%
\pgfpathlineto{\pgfqpoint{12.471919in}{5.196174in}}%
\pgfpathlineto{\pgfqpoint{12.474201in}{5.197577in}}%
\pgfpathlineto{\pgfqpoint{12.478764in}{5.224436in}}%
\pgfpathlineto{\pgfqpoint{12.485610in}{5.218222in}}%
\pgfpathlineto{\pgfqpoint{12.487892in}{5.235460in}}%
\pgfpathlineto{\pgfqpoint{12.490174in}{5.216819in}}%
\pgfpathlineto{\pgfqpoint{12.492456in}{5.227041in}}%
\pgfpathlineto{\pgfqpoint{12.494738in}{5.223434in}}%
\pgfpathlineto{\pgfqpoint{12.501584in}{5.230048in}}%
\pgfpathlineto{\pgfqpoint{12.503865in}{5.229246in}}%
\pgfpathlineto{\pgfqpoint{12.506147in}{5.216819in}}%
\pgfpathlineto{\pgfqpoint{12.508429in}{5.224035in}}%
\pgfpathlineto{\pgfqpoint{12.510711in}{5.209002in}}%
\pgfpathlineto{\pgfqpoint{12.517557in}{5.202989in}}%
\pgfpathlineto{\pgfqpoint{12.519839in}{5.205795in}}%
\pgfpathlineto{\pgfqpoint{12.524403in}{5.251495in}}%
\pgfpathlineto{\pgfqpoint{12.526684in}{5.252497in}}%
\pgfpathlineto{\pgfqpoint{12.533530in}{5.261918in}}%
\pgfpathlineto{\pgfqpoint{12.535812in}{5.273944in}}%
\pgfpathlineto{\pgfqpoint{12.538094in}{5.271338in}}%
\pgfpathlineto{\pgfqpoint{12.542658in}{5.276950in}}%
\pgfpathlineto{\pgfqpoint{12.551785in}{5.258510in}}%
\pgfpathlineto{\pgfqpoint{12.554067in}{5.235861in}}%
\pgfpathlineto{\pgfqpoint{12.558631in}{5.224436in}}%
\pgfpathlineto{\pgfqpoint{12.565477in}{5.217020in}}%
\pgfpathlineto{\pgfqpoint{12.567759in}{5.210004in}}%
\pgfpathlineto{\pgfqpoint{12.570041in}{5.218022in}}%
\pgfpathlineto{\pgfqpoint{12.572323in}{5.236061in}}%
\pgfpathlineto{\pgfqpoint{12.574605in}{5.221229in}}%
\pgfpathlineto{\pgfqpoint{12.581450in}{5.215416in}}%
\pgfpathlineto{\pgfqpoint{12.583732in}{5.222030in}}%
\pgfpathlineto{\pgfqpoint{12.586014in}{5.216819in}}%
\pgfpathlineto{\pgfqpoint{12.588296in}{5.214414in}}%
\pgfpathlineto{\pgfqpoint{12.590578in}{5.238667in}}%
\pgfpathlineto{\pgfqpoint{12.599706in}{5.237464in}}%
\pgfpathlineto{\pgfqpoint{12.601987in}{5.240671in}}%
\pgfpathlineto{\pgfqpoint{12.604269in}{5.256305in}}%
\pgfpathlineto{\pgfqpoint{12.606551in}{5.229447in}}%
\pgfpathlineto{\pgfqpoint{12.613397in}{5.220627in}}%
\pgfpathlineto{\pgfqpoint{12.615679in}{5.166710in}}%
\pgfpathlineto{\pgfqpoint{12.617961in}{5.143258in}}%
\pgfpathlineto{\pgfqpoint{12.620243in}{5.152278in}}%
\pgfpathlineto{\pgfqpoint{12.622525in}{5.128226in}}%
\pgfpathlineto{\pgfqpoint{12.629370in}{5.142858in}}%
\pgfpathlineto{\pgfqpoint{12.631652in}{5.157289in}}%
\pgfpathlineto{\pgfqpoint{12.633934in}{5.154483in}}%
\pgfpathlineto{\pgfqpoint{12.636216in}{5.170318in}}%
\pgfpathlineto{\pgfqpoint{12.638498in}{5.151276in}}%
\pgfpathlineto{\pgfqpoint{12.645344in}{5.141454in}}%
\pgfpathlineto{\pgfqpoint{12.647626in}{5.147468in}}%
\pgfpathlineto{\pgfqpoint{12.649907in}{5.151877in}}%
\pgfpathlineto{\pgfqpoint{12.652189in}{5.158692in}}%
\pgfpathlineto{\pgfqpoint{12.654471in}{5.156487in}}%
\pgfpathlineto{\pgfqpoint{12.663599in}{5.149272in}}%
\pgfpathlineto{\pgfqpoint{12.665881in}{5.162701in}}%
\pgfpathlineto{\pgfqpoint{12.668163in}{5.144261in}}%
\pgfpathlineto{\pgfqpoint{12.670445in}{5.138448in}}%
\pgfpathlineto{\pgfqpoint{12.677290in}{5.147468in}}%
\pgfpathlineto{\pgfqpoint{12.679572in}{5.149472in}}%
\pgfpathlineto{\pgfqpoint{12.681854in}{5.148670in}}%
\pgfpathlineto{\pgfqpoint{12.684136in}{5.143659in}}%
\pgfpathlineto{\pgfqpoint{12.686418in}{5.142858in}}%
\pgfpathlineto{\pgfqpoint{12.693264in}{5.147868in}}%
\pgfpathlineto{\pgfqpoint{12.695546in}{5.143459in}}%
\pgfpathlineto{\pgfqpoint{12.697828in}{5.129228in}}%
\pgfpathlineto{\pgfqpoint{12.700109in}{5.134239in}}%
\pgfpathlineto{\pgfqpoint{12.702391in}{5.099763in}}%
\pgfpathlineto{\pgfqpoint{12.709237in}{5.107380in}}%
\pgfpathlineto{\pgfqpoint{12.711519in}{5.080321in}}%
\pgfpathlineto{\pgfqpoint{12.713801in}{5.077715in}}%
\pgfpathlineto{\pgfqpoint{12.716083in}{5.089942in}}%
\pgfpathlineto{\pgfqpoint{12.718365in}{5.085332in}}%
\pgfpathlineto{\pgfqpoint{12.725210in}{5.115598in}}%
\pgfpathlineto{\pgfqpoint{12.727492in}{5.103171in}}%
\pgfpathlineto{\pgfqpoint{12.729774in}{5.118805in}}%
\pgfpathlineto{\pgfqpoint{12.732056in}{5.112391in}}%
\pgfpathlineto{\pgfqpoint{12.734338in}{5.136043in}}%
\pgfpathlineto{\pgfqpoint{12.741184in}{5.138247in}}%
\pgfpathlineto{\pgfqpoint{12.748029in}{5.090944in}}%
\pgfpathlineto{\pgfqpoint{12.750311in}{5.093750in}}%
\pgfpathlineto{\pgfqpoint{12.757157in}{5.100766in}}%
\pgfpathlineto{\pgfqpoint{12.759439in}{5.087136in}}%
\pgfpathlineto{\pgfqpoint{12.761721in}{5.093951in}}%
\pgfpathlineto{\pgfqpoint{12.764003in}{5.095755in}}%
\pgfpathlineto{\pgfqpoint{12.773130in}{5.106378in}}%
\pgfpathlineto{\pgfqpoint{12.775412in}{5.095153in}}%
\pgfpathlineto{\pgfqpoint{12.777694in}{5.101768in}}%
\pgfpathlineto{\pgfqpoint{12.779976in}{5.103772in}}%
\pgfpathlineto{\pgfqpoint{12.782258in}{5.111790in}}%
\pgfpathlineto{\pgfqpoint{12.789104in}{5.113193in}}%
\pgfpathlineto{\pgfqpoint{12.791386in}{5.116199in}}%
\pgfpathlineto{\pgfqpoint{12.793668in}{5.114596in}}%
\pgfpathlineto{\pgfqpoint{12.795950in}{5.114395in}}%
\pgfpathlineto{\pgfqpoint{12.798231in}{5.097559in}}%
\pgfpathlineto{\pgfqpoint{12.805077in}{5.103572in}}%
\pgfpathlineto{\pgfqpoint{12.807359in}{5.107180in}}%
\pgfpathlineto{\pgfqpoint{12.809641in}{5.107380in}}%
\pgfpathlineto{\pgfqpoint{12.811923in}{5.081323in}}%
\pgfpathlineto{\pgfqpoint{12.814205in}{5.082125in}}%
\pgfpathlineto{\pgfqpoint{12.821050in}{5.075109in}}%
\pgfpathlineto{\pgfqpoint{12.823332in}{5.071902in}}%
\pgfpathlineto{\pgfqpoint{12.825614in}{5.061881in}}%
\pgfpathlineto{\pgfqpoint{12.827896in}{5.055867in}}%
\pgfpathlineto{\pgfqpoint{12.830178in}{5.069698in}}%
\pgfpathlineto{\pgfqpoint{12.837024in}{5.070700in}}%
\pgfpathlineto{\pgfqpoint{12.839306in}{5.065889in}}%
\pgfpathlineto{\pgfqpoint{12.841588in}{5.071702in}}%
\pgfpathlineto{\pgfqpoint{12.843870in}{5.068094in}}%
\pgfpathlineto{\pgfqpoint{12.846151in}{5.081523in}}%
\pgfpathlineto{\pgfqpoint{12.852997in}{5.069297in}}%
\pgfpathlineto{\pgfqpoint{12.857561in}{5.059275in}}%
\pgfpathlineto{\pgfqpoint{12.859843in}{5.074709in}}%
\pgfpathlineto{\pgfqpoint{12.862125in}{5.083127in}}%
\pgfpathlineto{\pgfqpoint{12.868971in}{5.077515in}}%
\pgfpathlineto{\pgfqpoint{12.871252in}{5.079118in}}%
\pgfpathlineto{\pgfqpoint{12.873534in}{5.073105in}}%
\pgfpathlineto{\pgfqpoint{12.875816in}{5.071902in}}%
\pgfpathlineto{\pgfqpoint{12.878098in}{5.063685in}}%
\pgfpathlineto{\pgfqpoint{12.887226in}{5.049453in}}%
\pgfpathlineto{\pgfqpoint{12.889508in}{5.053863in}}%
\pgfpathlineto{\pgfqpoint{12.891790in}{5.052861in}}%
\pgfpathlineto{\pgfqpoint{12.894072in}{5.036225in}}%
\pgfpathlineto{\pgfqpoint{12.900917in}{5.044242in}}%
\pgfpathlineto{\pgfqpoint{12.903199in}{5.038830in}}%
\pgfpathlineto{\pgfqpoint{12.905481in}{5.039231in}}%
\pgfpathlineto{\pgfqpoint{12.907763in}{5.032015in}}%
\pgfpathlineto{\pgfqpoint{12.910045in}{5.019388in}}%
\pgfpathlineto{\pgfqpoint{12.916891in}{5.024198in}}%
\pgfpathlineto{\pgfqpoint{12.919172in}{5.045244in}}%
\pgfpathlineto{\pgfqpoint{12.921454in}{5.056469in}}%
\pgfpathlineto{\pgfqpoint{12.923736in}{5.054264in}}%
\pgfpathlineto{\pgfqpoint{12.926018in}{5.044643in}}%
\pgfpathlineto{\pgfqpoint{12.932864in}{5.031414in}}%
\pgfpathlineto{\pgfqpoint{12.939710in}{5.079118in}}%
\pgfpathlineto{\pgfqpoint{12.941992in}{5.074107in}}%
\pgfpathlineto{\pgfqpoint{12.948837in}{5.072504in}}%
\pgfpathlineto{\pgfqpoint{12.951119in}{5.060878in}}%
\pgfpathlineto{\pgfqpoint{12.953401in}{5.055867in}}%
\pgfpathlineto{\pgfqpoint{12.955683in}{5.053863in}}%
\pgfpathlineto{\pgfqpoint{12.957965in}{5.053061in}}%
\pgfpathlineto{\pgfqpoint{12.964811in}{5.035022in}}%
\pgfpathlineto{\pgfqpoint{12.967093in}{5.033619in}}%
\pgfpathlineto{\pgfqpoint{12.969374in}{5.059676in}}%
\pgfpathlineto{\pgfqpoint{12.971656in}{5.063284in}}%
\pgfpathlineto{\pgfqpoint{12.980784in}{5.065488in}}%
\pgfpathlineto{\pgfqpoint{12.983066in}{5.094953in}}%
\pgfpathlineto{\pgfqpoint{12.985348in}{5.081924in}}%
\pgfpathlineto{\pgfqpoint{12.987630in}{5.076112in}}%
\pgfpathlineto{\pgfqpoint{12.996757in}{5.098160in}}%
\pgfpathlineto{\pgfqpoint{13.003603in}{5.105175in}}%
\pgfpathlineto{\pgfqpoint{13.005885in}{5.103973in}}%
\pgfpathlineto{\pgfqpoint{13.012731in}{5.103171in}}%
\pgfpathlineto{\pgfqpoint{13.015013in}{5.093349in}}%
\pgfpathlineto{\pgfqpoint{13.019576in}{5.088539in}}%
\pgfpathlineto{\pgfqpoint{13.021858in}{5.081323in}}%
\pgfpathlineto{\pgfqpoint{13.028704in}{5.075510in}}%
\pgfpathlineto{\pgfqpoint{13.030986in}{5.080120in}}%
\pgfpathlineto{\pgfqpoint{13.033268in}{5.087136in}}%
\pgfpathlineto{\pgfqpoint{13.035550in}{5.029811in}}%
\pgfpathlineto{\pgfqpoint{13.037832in}{5.017584in}}%
\pgfpathlineto{\pgfqpoint{13.044677in}{5.012172in}}%
\pgfpathlineto{\pgfqpoint{13.046959in}{5.003553in}}%
\pgfpathlineto{\pgfqpoint{13.049241in}{5.000947in}}%
\pgfpathlineto{\pgfqpoint{13.051523in}{5.000547in}}%
\pgfpathlineto{\pgfqpoint{13.053805in}{4.995936in}}%
\pgfpathlineto{\pgfqpoint{13.060651in}{5.011771in}}%
\pgfpathlineto{\pgfqpoint{13.062933in}{5.008965in}}%
\pgfpathlineto{\pgfqpoint{13.065215in}{5.011972in}}%
\pgfpathlineto{\pgfqpoint{13.067496in}{5.001148in}}%
\pgfpathlineto{\pgfqpoint{13.069778in}{4.998342in}}%
\pgfpathlineto{\pgfqpoint{13.076624in}{4.996738in}}%
\pgfpathlineto{\pgfqpoint{13.078906in}{4.989723in}}%
\pgfpathlineto{\pgfqpoint{13.081188in}{4.971684in}}%
\pgfpathlineto{\pgfqpoint{13.083470in}{4.968076in}}%
\pgfpathlineto{\pgfqpoint{13.085752in}{4.931195in}}%
\pgfpathlineto{\pgfqpoint{13.094879in}{4.870663in}}%
\pgfpathlineto{\pgfqpoint{13.097161in}{4.914559in}}%
\pgfpathlineto{\pgfqpoint{13.099443in}{4.924981in}}%
\pgfpathlineto{\pgfqpoint{13.101725in}{4.920171in}}%
\pgfpathlineto{\pgfqpoint{13.108571in}{4.910550in}}%
\pgfpathlineto{\pgfqpoint{13.110853in}{4.879081in}}%
\pgfpathlineto{\pgfqpoint{13.113135in}{4.895317in}}%
\pgfpathlineto{\pgfqpoint{13.115416in}{4.897321in}}%
\pgfpathlineto{\pgfqpoint{13.117698in}{4.876676in}}%
\pgfpathlineto{\pgfqpoint{13.126826in}{4.898323in}}%
\pgfpathlineto{\pgfqpoint{13.129108in}{4.871665in}}%
\pgfpathlineto{\pgfqpoint{13.131390in}{4.868859in}}%
\pgfpathlineto{\pgfqpoint{13.133672in}{4.870663in}}%
\pgfpathlineto{\pgfqpoint{13.140517in}{4.864249in}}%
\pgfpathlineto{\pgfqpoint{13.142799in}{4.888903in}}%
\pgfpathlineto{\pgfqpoint{13.145081in}{4.900328in}}%
\pgfpathlineto{\pgfqpoint{13.147363in}{4.902933in}}%
\pgfpathlineto{\pgfqpoint{13.149645in}{4.897521in}}%
\pgfpathlineto{\pgfqpoint{13.156491in}{4.910149in}}%
\pgfpathlineto{\pgfqpoint{13.158773in}{4.901931in}}%
\pgfpathlineto{\pgfqpoint{13.161055in}{4.903334in}}%
\pgfpathlineto{\pgfqpoint{13.165618in}{4.946027in}}%
\pgfpathlineto{\pgfqpoint{13.172464in}{4.929992in}}%
\pgfpathlineto{\pgfqpoint{13.174746in}{4.939012in}}%
\pgfpathlineto{\pgfqpoint{13.177028in}{4.932999in}}%
\pgfpathlineto{\pgfqpoint{13.179310in}{4.933199in}}%
\pgfpathlineto{\pgfqpoint{13.181592in}{4.941618in}}%
\pgfpathlineto{\pgfqpoint{13.188437in}{4.955849in}}%
\pgfpathlineto{\pgfqpoint{13.190719in}{4.959056in}}%
\pgfpathlineto{\pgfqpoint{13.193001in}{4.964668in}}%
\pgfpathlineto{\pgfqpoint{13.195283in}{4.976694in}}%
\pgfpathlineto{\pgfqpoint{13.197565in}{4.978098in}}%
\pgfpathlineto{\pgfqpoint{13.204411in}{4.975492in}}%
\pgfpathlineto{\pgfqpoint{13.206693in}{4.971483in}}%
\pgfpathlineto{\pgfqpoint{13.211257in}{4.974490in}}%
\pgfpathlineto{\pgfqpoint{13.213538in}{4.985514in}}%
\pgfpathlineto{\pgfqpoint{13.220384in}{4.990124in}}%
\pgfpathlineto{\pgfqpoint{13.222666in}{4.977296in}}%
\pgfpathlineto{\pgfqpoint{13.224948in}{4.974089in}}%
\pgfpathlineto{\pgfqpoint{13.227230in}{4.996738in}}%
\pgfpathlineto{\pgfqpoint{13.229512in}{5.035623in}}%
\pgfpathlineto{\pgfqpoint{13.236358in}{5.043841in}}%
\pgfpathlineto{\pgfqpoint{13.238639in}{5.040434in}}%
\pgfpathlineto{\pgfqpoint{13.240921in}{5.026403in}}%
\pgfpathlineto{\pgfqpoint{13.243203in}{5.035423in}}%
\pgfpathlineto{\pgfqpoint{13.245485in}{5.023998in}}%
\pgfpathlineto{\pgfqpoint{13.252331in}{5.028007in}}%
\pgfpathlineto{\pgfqpoint{13.254613in}{5.036024in}}%
\pgfpathlineto{\pgfqpoint{13.256895in}{5.036225in}}%
\pgfpathlineto{\pgfqpoint{13.261459in}{5.009566in}}%
\pgfpathlineto{\pgfqpoint{13.268304in}{5.006560in}}%
\pgfpathlineto{\pgfqpoint{13.270586in}{5.013174in}}%
\pgfpathlineto{\pgfqpoint{13.272868in}{5.016582in}}%
\pgfpathlineto{\pgfqpoint{13.275150in}{4.993331in}}%
\pgfpathlineto{\pgfqpoint{13.277432in}{4.980703in}}%
\pgfpathlineto{\pgfqpoint{13.284278in}{5.004756in}}%
\pgfpathlineto{\pgfqpoint{13.286559in}{5.000547in}}%
\pgfpathlineto{\pgfqpoint{13.288841in}{5.015379in}}%
\pgfpathlineto{\pgfqpoint{13.291123in}{5.021192in}}%
\pgfpathlineto{\pgfqpoint{13.293405in}{5.013976in}}%
\pgfpathlineto{\pgfqpoint{13.300251in}{5.016782in}}%
\pgfpathlineto{\pgfqpoint{13.302533in}{5.025401in}}%
\pgfpathlineto{\pgfqpoint{13.304815in}{5.015379in}}%
\pgfpathlineto{\pgfqpoint{13.309379in}{5.011972in}}%
\pgfpathlineto{\pgfqpoint{13.316224in}{4.996538in}}%
\pgfpathlineto{\pgfqpoint{13.318506in}{5.016181in}}%
\pgfpathlineto{\pgfqpoint{13.320788in}{5.013575in}}%
\pgfpathlineto{\pgfqpoint{13.323070in}{5.012573in}}%
\pgfpathlineto{\pgfqpoint{13.325352in}{5.050055in}}%
\pgfpathlineto{\pgfqpoint{13.332198in}{5.059676in}}%
\pgfpathlineto{\pgfqpoint{13.334480in}{5.048652in}}%
\pgfpathlineto{\pgfqpoint{13.336761in}{5.047649in}}%
\pgfpathlineto{\pgfqpoint{13.339043in}{5.049253in}}%
\pgfpathlineto{\pgfqpoint{13.341325in}{5.049053in}}%
\pgfpathlineto{\pgfqpoint{13.348171in}{5.058273in}}%
\pgfpathlineto{\pgfqpoint{13.352735in}{5.106578in}}%
\pgfpathlineto{\pgfqpoint{13.355017in}{5.094151in}}%
\pgfpathlineto{\pgfqpoint{13.357299in}{5.055467in}}%
\pgfpathlineto{\pgfqpoint{13.364144in}{5.070099in}}%
\pgfpathlineto{\pgfqpoint{13.366426in}{5.081724in}}%
\pgfpathlineto{\pgfqpoint{13.368708in}{5.087537in}}%
\pgfpathlineto{\pgfqpoint{13.370990in}{5.085131in}}%
\pgfpathlineto{\pgfqpoint{13.380118in}{5.087537in}}%
\pgfpathlineto{\pgfqpoint{13.382400in}{5.095354in}}%
\pgfpathlineto{\pgfqpoint{13.384681in}{5.090142in}}%
\pgfpathlineto{\pgfqpoint{13.386963in}{5.078316in}}%
\pgfpathlineto{\pgfqpoint{13.396091in}{5.059676in}}%
\pgfpathlineto{\pgfqpoint{13.398373in}{5.064085in}}%
\pgfpathlineto{\pgfqpoint{13.402937in}{5.038429in}}%
\pgfpathlineto{\pgfqpoint{13.405219in}{5.016782in}}%
\pgfpathlineto{\pgfqpoint{13.412064in}{5.029209in}}%
\pgfpathlineto{\pgfqpoint{13.414346in}{5.026403in}}%
\pgfpathlineto{\pgfqpoint{13.416628in}{5.014577in}}%
\pgfpathlineto{\pgfqpoint{13.418910in}{5.019989in}}%
\pgfpathlineto{\pgfqpoint{13.421192in}{4.998943in}}%
\pgfpathlineto{\pgfqpoint{13.430320in}{5.030412in}}%
\pgfpathlineto{\pgfqpoint{13.432602in}{5.026002in}}%
\pgfpathlineto{\pgfqpoint{13.434883in}{5.042037in}}%
\pgfpathlineto{\pgfqpoint{13.437165in}{5.053663in}}%
\pgfpathlineto{\pgfqpoint{13.444011in}{5.044442in}}%
\pgfpathlineto{\pgfqpoint{13.446293in}{5.079920in}}%
\pgfpathlineto{\pgfqpoint{13.448575in}{5.079720in}}%
\pgfpathlineto{\pgfqpoint{13.450857in}{5.098160in}}%
\pgfpathlineto{\pgfqpoint{13.453139in}{5.131833in}}%
\pgfpathlineto{\pgfqpoint{13.459984in}{5.121611in}}%
\pgfpathlineto{\pgfqpoint{13.462266in}{5.105376in}}%
\pgfpathlineto{\pgfqpoint{13.464548in}{5.121210in}}%
\pgfpathlineto{\pgfqpoint{13.466830in}{5.113994in}}%
\pgfpathlineto{\pgfqpoint{13.475958in}{5.148670in}}%
\pgfpathlineto{\pgfqpoint{13.478240in}{5.149071in}}%
\pgfpathlineto{\pgfqpoint{13.480522in}{5.130631in}}%
\pgfpathlineto{\pgfqpoint{13.482803in}{5.099563in}}%
\pgfpathlineto{\pgfqpoint{13.485085in}{5.119206in}}%
\pgfpathlineto{\pgfqpoint{13.494213in}{5.127825in}}%
\pgfpathlineto{\pgfqpoint{13.496495in}{5.145664in}}%
\pgfpathlineto{\pgfqpoint{13.498777in}{5.137045in}}%
\pgfpathlineto{\pgfqpoint{13.501059in}{5.133637in}}%
\pgfpathlineto{\pgfqpoint{13.507904in}{5.139851in}}%
\pgfpathlineto{\pgfqpoint{13.512468in}{5.129629in}}%
\pgfpathlineto{\pgfqpoint{13.514750in}{5.143659in}}%
\pgfpathlineto{\pgfqpoint{13.517032in}{5.121210in}}%
\pgfpathlineto{\pgfqpoint{13.523878in}{5.106578in}}%
\pgfpathlineto{\pgfqpoint{13.526160in}{5.123615in}}%
\pgfpathlineto{\pgfqpoint{13.528442in}{5.147468in}}%
\pgfpathlineto{\pgfqpoint{13.530724in}{5.152679in}}%
\pgfpathlineto{\pgfqpoint{13.533005in}{5.164505in}}%
\pgfpathlineto{\pgfqpoint{13.539851in}{5.157289in}}%
\pgfpathlineto{\pgfqpoint{13.542133in}{5.156688in}}%
\pgfpathlineto{\pgfqpoint{13.544415in}{5.154884in}}%
\pgfpathlineto{\pgfqpoint{13.548979in}{5.133036in}}%
\pgfpathlineto{\pgfqpoint{13.555824in}{5.122413in}}%
\pgfpathlineto{\pgfqpoint{13.558106in}{5.125019in}}%
\pgfpathlineto{\pgfqpoint{13.560388in}{5.125620in}}%
\pgfpathlineto{\pgfqpoint{13.562670in}{5.151075in}}%
\pgfpathlineto{\pgfqpoint{13.564952in}{5.158291in}}%
\pgfpathlineto{\pgfqpoint{13.571798in}{5.161298in}}%
\pgfpathlineto{\pgfqpoint{13.574080in}{5.150875in}}%
\pgfpathlineto{\pgfqpoint{13.578644in}{5.153481in}}%
\pgfpathlineto{\pgfqpoint{13.587771in}{5.148670in}}%
\pgfpathlineto{\pgfqpoint{13.590053in}{5.151877in}}%
\pgfpathlineto{\pgfqpoint{13.592335in}{5.149873in}}%
\pgfpathlineto{\pgfqpoint{13.594617in}{5.143058in}}%
\pgfpathlineto{\pgfqpoint{13.596899in}{5.165106in}}%
\pgfpathlineto{\pgfqpoint{13.603745in}{5.159293in}}%
\pgfpathlineto{\pgfqpoint{13.606026in}{5.158492in}}%
\pgfpathlineto{\pgfqpoint{13.608308in}{5.170117in}}%
\pgfpathlineto{\pgfqpoint{13.610590in}{5.159895in}}%
\pgfpathlineto{\pgfqpoint{13.612872in}{5.159093in}}%
\pgfpathlineto{\pgfqpoint{13.619718in}{5.150675in}}%
\pgfpathlineto{\pgfqpoint{13.622000in}{5.152479in}}%
\pgfpathlineto{\pgfqpoint{13.624282in}{5.145864in}}%
\pgfpathlineto{\pgfqpoint{13.626564in}{5.149873in}}%
\pgfpathlineto{\pgfqpoint{13.628846in}{5.155084in}}%
\pgfpathlineto{\pgfqpoint{13.635691in}{5.164705in}}%
\pgfpathlineto{\pgfqpoint{13.637973in}{5.172923in}}%
\pgfpathlineto{\pgfqpoint{13.640255in}{5.141454in}}%
\pgfpathlineto{\pgfqpoint{13.642537in}{5.127825in}}%
\pgfpathlineto{\pgfqpoint{13.651665in}{5.138849in}}%
\pgfpathlineto{\pgfqpoint{13.653946in}{5.104975in}}%
\pgfpathlineto{\pgfqpoint{13.656228in}{5.111188in}}%
\pgfpathlineto{\pgfqpoint{13.658510in}{5.108783in}}%
\pgfpathlineto{\pgfqpoint{13.660792in}{5.115398in}}%
\pgfpathlineto{\pgfqpoint{13.667638in}{5.130831in}}%
\pgfpathlineto{\pgfqpoint{13.669920in}{5.133236in}}%
\pgfpathlineto{\pgfqpoint{13.672202in}{5.142256in}}%
\pgfpathlineto{\pgfqpoint{13.674484in}{5.136844in}}%
\pgfpathlineto{\pgfqpoint{13.676766in}{5.152078in}}%
\pgfpathlineto{\pgfqpoint{13.683611in}{5.151877in}}%
\pgfpathlineto{\pgfqpoint{13.685893in}{5.158291in}}%
\pgfpathlineto{\pgfqpoint{13.688175in}{5.152278in}}%
\pgfpathlineto{\pgfqpoint{13.690457in}{5.157089in}}%
\pgfpathlineto{\pgfqpoint{13.692739in}{5.135642in}}%
\pgfpathlineto{\pgfqpoint{13.699585in}{5.142858in}}%
\pgfpathlineto{\pgfqpoint{13.704148in}{5.110387in}}%
\pgfpathlineto{\pgfqpoint{13.706430in}{5.116600in}}%
\pgfpathlineto{\pgfqpoint{13.708712in}{5.113594in}}%
\pgfpathlineto{\pgfqpoint{13.715558in}{5.116801in}}%
\pgfpathlineto{\pgfqpoint{13.717840in}{5.130831in}}%
\pgfpathlineto{\pgfqpoint{13.720122in}{5.140051in}}%
\pgfpathlineto{\pgfqpoint{13.722404in}{5.135441in}}%
\pgfpathlineto{\pgfqpoint{13.724686in}{5.139250in}}%
\pgfpathlineto{\pgfqpoint{13.733813in}{5.132034in}}%
\pgfpathlineto{\pgfqpoint{13.736095in}{5.145864in}}%
\pgfpathlineto{\pgfqpoint{13.738377in}{5.148670in}}%
\pgfpathlineto{\pgfqpoint{13.740659in}{5.158091in}}%
\pgfpathlineto{\pgfqpoint{13.747505in}{5.163703in}}%
\pgfpathlineto{\pgfqpoint{13.749787in}{5.155485in}}%
\pgfpathlineto{\pgfqpoint{13.752068in}{5.161498in}}%
\pgfpathlineto{\pgfqpoint{13.754350in}{5.170919in}}%
\pgfpathlineto{\pgfqpoint{13.756632in}{5.171520in}}%
\pgfpathlineto{\pgfqpoint{13.763478in}{5.160095in}}%
\pgfpathlineto{\pgfqpoint{13.765760in}{5.174126in}}%
\pgfpathlineto{\pgfqpoint{13.768042in}{5.166910in}}%
\pgfpathlineto{\pgfqpoint{13.770324in}{5.175328in}}%
\pgfpathlineto{\pgfqpoint{13.772606in}{5.170117in}}%
\pgfpathlineto{\pgfqpoint{13.779451in}{5.168514in}}%
\pgfpathlineto{\pgfqpoint{13.781733in}{5.175328in}}%
\pgfpathlineto{\pgfqpoint{13.784015in}{5.178135in}}%
\pgfpathlineto{\pgfqpoint{13.786297in}{5.189960in}}%
\pgfpathlineto{\pgfqpoint{13.788579in}{5.154282in}}%
\pgfpathlineto{\pgfqpoint{13.795425in}{5.135642in}}%
\pgfpathlineto{\pgfqpoint{13.802270in}{5.198178in}}%
\pgfpathlineto{\pgfqpoint{13.804552in}{5.200183in}}%
\pgfpathlineto{\pgfqpoint{13.813680in}{5.212209in}}%
\pgfpathlineto{\pgfqpoint{13.815962in}{5.204793in}}%
\pgfpathlineto{\pgfqpoint{13.818244in}{5.201185in}}%
\pgfpathlineto{\pgfqpoint{13.820526in}{5.218222in}}%
\pgfpathlineto{\pgfqpoint{13.829653in}{5.217821in}}%
\pgfpathlineto{\pgfqpoint{13.831935in}{5.220427in}}%
\pgfpathlineto{\pgfqpoint{13.834217in}{5.220026in}}%
\pgfpathlineto{\pgfqpoint{13.836499in}{5.222632in}}%
\pgfpathlineto{\pgfqpoint{13.843345in}{5.221229in}}%
\pgfpathlineto{\pgfqpoint{13.845627in}{5.225638in}}%
\pgfpathlineto{\pgfqpoint{13.847909in}{5.222231in}}%
\pgfpathlineto{\pgfqpoint{13.850190in}{5.221229in}}%
\pgfpathlineto{\pgfqpoint{13.852472in}{5.229647in}}%
\pgfpathlineto{\pgfqpoint{13.859318in}{5.231050in}}%
\pgfpathlineto{\pgfqpoint{13.861600in}{5.221429in}}%
\pgfpathlineto{\pgfqpoint{13.863882in}{5.206396in}}%
\pgfpathlineto{\pgfqpoint{13.866164in}{5.212810in}}%
\pgfpathlineto{\pgfqpoint{13.868446in}{5.227242in}}%
\pgfpathlineto{\pgfqpoint{13.875291in}{5.242275in}}%
\pgfpathlineto{\pgfqpoint{13.877573in}{5.248689in}}%
\pgfpathlineto{\pgfqpoint{13.879855in}{5.234257in}}%
\pgfpathlineto{\pgfqpoint{13.882137in}{5.235660in}}%
\pgfpathlineto{\pgfqpoint{13.884419in}{5.230649in}}%
\pgfpathlineto{\pgfqpoint{13.891265in}{5.230248in}}%
\pgfpathlineto{\pgfqpoint{13.893547in}{5.234458in}}%
\pgfpathlineto{\pgfqpoint{13.895829in}{5.240471in}}%
\pgfpathlineto{\pgfqpoint{13.900392in}{5.253900in}}%
\pgfpathlineto{\pgfqpoint{13.907238in}{5.253499in}}%
\pgfpathlineto{\pgfqpoint{13.909520in}{5.245281in}}%
\pgfpathlineto{\pgfqpoint{13.911802in}{5.252297in}}%
\pgfpathlineto{\pgfqpoint{13.914084in}{5.261116in}}%
\pgfpathlineto{\pgfqpoint{13.916366in}{5.258711in}}%
\pgfpathlineto{\pgfqpoint{13.923211in}{5.250292in}}%
\pgfpathlineto{\pgfqpoint{13.925493in}{5.260515in}}%
\pgfpathlineto{\pgfqpoint{13.927775in}{5.258711in}}%
\pgfpathlineto{\pgfqpoint{13.930057in}{5.269534in}}%
\pgfpathlineto{\pgfqpoint{13.932339in}{5.263722in}}%
\pgfpathlineto{\pgfqpoint{13.939185in}{5.276950in}}%
\pgfpathlineto{\pgfqpoint{13.941467in}{5.262920in}}%
\pgfpathlineto{\pgfqpoint{13.943749in}{5.258711in}}%
\pgfpathlineto{\pgfqpoint{13.946031in}{5.277151in}}%
\pgfpathlineto{\pgfqpoint{13.948312in}{5.275147in}}%
\pgfpathlineto{\pgfqpoint{13.957440in}{5.283164in}}%
\pgfpathlineto{\pgfqpoint{13.959722in}{5.270536in}}%
\pgfpathlineto{\pgfqpoint{13.962004in}{5.267329in}}%
\pgfpathlineto{\pgfqpoint{13.964286in}{5.239068in}}%
\pgfpathlineto{\pgfqpoint{13.971132in}{5.275948in}}%
\pgfpathlineto{\pgfqpoint{13.973413in}{5.254101in}}%
\pgfpathlineto{\pgfqpoint{13.975695in}{5.253299in}}%
\pgfpathlineto{\pgfqpoint{13.977977in}{5.272541in}}%
\pgfpathlineto{\pgfqpoint{13.980259in}{5.272340in}}%
\pgfpathlineto{\pgfqpoint{13.987105in}{5.278153in}}%
\pgfpathlineto{\pgfqpoint{13.989387in}{5.282162in}}%
\pgfpathlineto{\pgfqpoint{13.991669in}{5.267730in}}%
\pgfpathlineto{\pgfqpoint{13.993951in}{5.289578in}}%
\pgfpathlineto{\pgfqpoint{13.996233in}{5.267129in}}%
\pgfpathlineto{\pgfqpoint{14.003078in}{5.268733in}}%
\pgfpathlineto{\pgfqpoint{14.005360in}{5.277953in}}%
\pgfpathlineto{\pgfqpoint{14.007642in}{5.298197in}}%
\pgfpathlineto{\pgfqpoint{14.009924in}{5.275748in}}%
\pgfpathlineto{\pgfqpoint{14.012206in}{5.303609in}}%
\pgfpathlineto{\pgfqpoint{14.021333in}{5.277953in}}%
\pgfpathlineto{\pgfqpoint{14.025897in}{5.293787in}}%
\pgfpathlineto{\pgfqpoint{14.028179in}{5.308219in}}%
\pgfpathlineto{\pgfqpoint{14.035025in}{5.290981in}}%
\pgfpathlineto{\pgfqpoint{14.037307in}{5.281360in}}%
\pgfpathlineto{\pgfqpoint{14.039589in}{5.281961in}}%
\pgfpathlineto{\pgfqpoint{14.041871in}{5.275748in}}%
\pgfpathlineto{\pgfqpoint{14.044153in}{5.279356in}}%
\pgfpathlineto{\pgfqpoint{14.050998in}{5.268332in}}%
\pgfpathlineto{\pgfqpoint{14.053280in}{5.261316in}}%
\pgfpathlineto{\pgfqpoint{14.055562in}{5.238466in}}%
\pgfpathlineto{\pgfqpoint{14.060126in}{5.216017in}}%
\pgfpathlineto{\pgfqpoint{14.066972in}{5.211808in}}%
\pgfpathlineto{\pgfqpoint{14.069254in}{5.264924in}}%
\pgfpathlineto{\pgfqpoint{14.071535in}{5.272741in}}%
\pgfpathlineto{\pgfqpoint{14.073817in}{5.257708in}}%
\pgfpathlineto{\pgfqpoint{14.076099in}{5.262519in}}%
\pgfpathlineto{\pgfqpoint{14.082945in}{5.261717in}}%
\pgfpathlineto{\pgfqpoint{14.085227in}{5.262719in}}%
\pgfpathlineto{\pgfqpoint{14.089791in}{5.257909in}}%
\pgfpathlineto{\pgfqpoint{14.092073in}{5.229848in}}%
\pgfpathlineto{\pgfqpoint{14.098918in}{5.257308in}}%
\pgfpathlineto{\pgfqpoint{14.101200in}{5.273944in}}%
\pgfpathlineto{\pgfqpoint{14.103482in}{5.245682in}}%
\pgfpathlineto{\pgfqpoint{14.105764in}{5.190762in}}%
\pgfpathlineto{\pgfqpoint{14.108046in}{5.202187in}}%
\pgfpathlineto{\pgfqpoint{14.114892in}{5.191363in}}%
\pgfpathlineto{\pgfqpoint{14.117174in}{5.202788in}}%
\pgfpathlineto{\pgfqpoint{14.119455in}{5.194971in}}%
\pgfpathlineto{\pgfqpoint{14.121737in}{5.192767in}}%
\pgfpathlineto{\pgfqpoint{14.124019in}{5.172923in}}%
\pgfpathlineto{\pgfqpoint{14.130865in}{5.184749in}}%
\pgfpathlineto{\pgfqpoint{14.133147in}{5.186954in}}%
\pgfpathlineto{\pgfqpoint{14.135429in}{5.185551in}}%
\pgfpathlineto{\pgfqpoint{14.139993in}{5.199982in}}%
\pgfpathlineto{\pgfqpoint{14.146838in}{5.192767in}}%
\pgfpathlineto{\pgfqpoint{14.149120in}{5.189359in}}%
\pgfpathlineto{\pgfqpoint{14.151402in}{5.181342in}}%
\pgfpathlineto{\pgfqpoint{14.153684in}{5.170318in}}%
\pgfpathlineto{\pgfqpoint{14.155966in}{5.180339in}}%
\pgfpathlineto{\pgfqpoint{14.162812in}{5.191163in}}%
\pgfpathlineto{\pgfqpoint{14.165094in}{5.189760in}}%
\pgfpathlineto{\pgfqpoint{14.167376in}{5.213211in}}%
\pgfpathlineto{\pgfqpoint{14.169657in}{5.200584in}}%
\pgfpathlineto{\pgfqpoint{14.171939in}{5.216819in}}%
\pgfpathlineto{\pgfqpoint{14.178785in}{5.230850in}}%
\pgfpathlineto{\pgfqpoint{14.181067in}{5.231652in}}%
\pgfpathlineto{\pgfqpoint{14.183349in}{5.216819in}}%
\pgfpathlineto{\pgfqpoint{14.185631in}{5.222431in}}%
\pgfpathlineto{\pgfqpoint{14.187913in}{5.222431in}}%
\pgfpathlineto{\pgfqpoint{14.194758in}{5.223033in}}%
\pgfpathlineto{\pgfqpoint{14.197040in}{5.220427in}}%
\pgfpathlineto{\pgfqpoint{14.199322in}{5.215015in}}%
\pgfpathlineto{\pgfqpoint{14.201604in}{5.218623in}}%
\pgfpathlineto{\pgfqpoint{14.203886in}{5.227643in}}%
\pgfpathlineto{\pgfqpoint{14.213014in}{5.221028in}}%
\pgfpathlineto{\pgfqpoint{14.215296in}{5.211207in}}%
\pgfpathlineto{\pgfqpoint{14.217577in}{5.216418in}}%
\pgfpathlineto{\pgfqpoint{14.219859in}{5.211407in}}%
\pgfpathlineto{\pgfqpoint{14.228987in}{5.213612in}}%
\pgfpathlineto{\pgfqpoint{14.231269in}{5.219224in}}%
\pgfpathlineto{\pgfqpoint{14.233551in}{5.229447in}}%
\pgfpathlineto{\pgfqpoint{14.235833in}{5.229046in}}%
\pgfpathlineto{\pgfqpoint{14.242678in}{5.217220in}}%
\pgfpathlineto{\pgfqpoint{14.244960in}{5.200383in}}%
\pgfpathlineto{\pgfqpoint{14.247242in}{5.205194in}}%
\pgfpathlineto{\pgfqpoint{14.249524in}{5.206998in}}%
\pgfpathlineto{\pgfqpoint{14.251806in}{5.210004in}}%
\pgfpathlineto{\pgfqpoint{14.260934in}{5.232253in}}%
\pgfpathlineto{\pgfqpoint{14.263216in}{5.239469in}}%
\pgfpathlineto{\pgfqpoint{14.265498in}{5.235259in}}%
\pgfpathlineto{\pgfqpoint{14.267779in}{5.286572in}}%
\pgfpathlineto{\pgfqpoint{14.274625in}{5.277351in}}%
\pgfpathlineto{\pgfqpoint{14.276907in}{5.294188in}}%
\pgfpathlineto{\pgfqpoint{14.281471in}{5.270737in}}%
\pgfpathlineto{\pgfqpoint{14.283753in}{5.272942in}}%
\pgfpathlineto{\pgfqpoint{14.290599in}{5.273543in}}%
\pgfpathlineto{\pgfqpoint{14.292880in}{5.289378in}}%
\pgfpathlineto{\pgfqpoint{14.295162in}{5.284367in}}%
\pgfpathlineto{\pgfqpoint{14.297444in}{5.292384in}}%
\pgfpathlineto{\pgfqpoint{14.299726in}{5.285770in}}%
\pgfpathlineto{\pgfqpoint{14.306572in}{5.285569in}}%
\pgfpathlineto{\pgfqpoint{14.308854in}{5.296994in}}%
\pgfpathlineto{\pgfqpoint{14.313418in}{5.309221in}}%
\pgfpathlineto{\pgfqpoint{14.315699in}{5.296193in}}%
\pgfpathlineto{\pgfqpoint{14.322545in}{5.302607in}}%
\pgfpathlineto{\pgfqpoint{14.324827in}{5.294188in}}%
\pgfpathlineto{\pgfqpoint{14.327109in}{5.354921in}}%
\pgfpathlineto{\pgfqpoint{14.329391in}{5.348707in}}%
\pgfpathlineto{\pgfqpoint{14.331673in}{5.354320in}}%
\pgfpathlineto{\pgfqpoint{14.340800in}{5.365143in}}%
\pgfpathlineto{\pgfqpoint{14.343082in}{5.360934in}}%
\pgfpathlineto{\pgfqpoint{14.345364in}{5.355121in}}%
\pgfpathlineto{\pgfqpoint{14.347646in}{5.353718in}}%
\pgfpathlineto{\pgfqpoint{14.354492in}{5.350712in}}%
\pgfpathlineto{\pgfqpoint{14.356774in}{5.354119in}}%
\pgfpathlineto{\pgfqpoint{14.359056in}{5.364943in}}%
\pgfpathlineto{\pgfqpoint{14.361338in}{5.351113in}}%
\pgfpathlineto{\pgfqpoint{14.363620in}{5.343496in}}%
\pgfpathlineto{\pgfqpoint{14.372747in}{5.339487in}}%
\pgfpathlineto{\pgfqpoint{14.375029in}{5.336681in}}%
\pgfpathlineto{\pgfqpoint{14.377311in}{5.340489in}}%
\pgfpathlineto{\pgfqpoint{14.379593in}{5.354119in}}%
\pgfpathlineto{\pgfqpoint{14.386439in}{5.358529in}}%
\pgfpathlineto{\pgfqpoint{14.388720in}{5.352716in}}%
\pgfpathlineto{\pgfqpoint{14.391002in}{5.360132in}}%
\pgfpathlineto{\pgfqpoint{14.393284in}{5.360934in}}%
\pgfpathlineto{\pgfqpoint{14.395566in}{5.352716in}}%
\pgfpathlineto{\pgfqpoint{14.402412in}{5.356725in}}%
\pgfpathlineto{\pgfqpoint{14.404694in}{5.356324in}}%
\pgfpathlineto{\pgfqpoint{14.411540in}{5.344699in}}%
\pgfpathlineto{\pgfqpoint{14.418385in}{5.343295in}}%
\pgfpathlineto{\pgfqpoint{14.420667in}{5.348306in}}%
\pgfpathlineto{\pgfqpoint{14.422949in}{5.345300in}}%
\pgfpathlineto{\pgfqpoint{14.427513in}{5.331269in}}%
\pgfpathlineto{\pgfqpoint{14.434359in}{5.328062in}}%
\pgfpathlineto{\pgfqpoint{14.436641in}{5.332472in}}%
\pgfpathlineto{\pgfqpoint{14.438922in}{5.333474in}}%
\pgfpathlineto{\pgfqpoint{14.441204in}{5.322851in}}%
\pgfpathlineto{\pgfqpoint{14.443486in}{5.319644in}}%
\pgfpathlineto{\pgfqpoint{14.450332in}{5.324655in}}%
\pgfpathlineto{\pgfqpoint{14.452614in}{5.330267in}}%
\pgfpathlineto{\pgfqpoint{14.454896in}{5.339888in}}%
\pgfpathlineto{\pgfqpoint{14.457178in}{5.334677in}}%
\pgfpathlineto{\pgfqpoint{14.466305in}{5.341291in}}%
\pgfpathlineto{\pgfqpoint{14.468587in}{5.348908in}}%
\pgfpathlineto{\pgfqpoint{14.470869in}{5.339487in}}%
\pgfpathlineto{\pgfqpoint{14.473151in}{5.334276in}}%
\pgfpathlineto{\pgfqpoint{14.475433in}{5.321047in}}%
\pgfpathlineto{\pgfqpoint{14.482279in}{5.338485in}}%
\pgfpathlineto{\pgfqpoint{14.484561in}{5.346903in}}%
\pgfpathlineto{\pgfqpoint{14.486842in}{5.304410in}}%
\pgfpathlineto{\pgfqpoint{14.489124in}{5.303609in}}%
\pgfpathlineto{\pgfqpoint{14.491406in}{5.296794in}}%
\pgfpathlineto{\pgfqpoint{14.498252in}{5.292184in}}%
\pgfpathlineto{\pgfqpoint{14.500534in}{5.275948in}}%
\pgfpathlineto{\pgfqpoint{14.502816in}{5.279757in}}%
\pgfpathlineto{\pgfqpoint{14.505098in}{5.280157in}}%
\pgfpathlineto{\pgfqpoint{14.507380in}{5.281160in}}%
\pgfpathlineto{\pgfqpoint{14.514225in}{5.282162in}}%
\pgfpathlineto{\pgfqpoint{14.516507in}{5.279356in}}%
\pgfpathlineto{\pgfqpoint{14.518789in}{5.280959in}}%
\pgfpathlineto{\pgfqpoint{14.521071in}{5.274946in}}%
\pgfpathlineto{\pgfqpoint{14.523353in}{5.275347in}}%
\pgfpathlineto{\pgfqpoint{14.530199in}{5.277953in}}%
\pgfpathlineto{\pgfqpoint{14.532481in}{5.276349in}}%
\pgfpathlineto{\pgfqpoint{14.534763in}{5.276750in}}%
\pgfpathlineto{\pgfqpoint{14.537044in}{5.269334in}}%
\pgfpathlineto{\pgfqpoint{14.539326in}{5.276349in}}%
\pgfpathlineto{\pgfqpoint{14.546172in}{5.275748in}}%
\pgfpathlineto{\pgfqpoint{14.548454in}{5.273343in}}%
\pgfpathlineto{\pgfqpoint{14.555300in}{5.295190in}}%
\pgfpathlineto{\pgfqpoint{14.564427in}{5.297996in}}%
\pgfpathlineto{\pgfqpoint{14.566709in}{5.311025in}}%
\pgfpathlineto{\pgfqpoint{14.568991in}{5.311827in}}%
\pgfpathlineto{\pgfqpoint{14.571273in}{5.320446in}}%
\pgfpathlineto{\pgfqpoint{14.580401in}{5.324454in}}%
\pgfpathlineto{\pgfqpoint{14.582683in}{5.323853in}}%
\pgfpathlineto{\pgfqpoint{14.584964in}{5.306615in}}%
\pgfpathlineto{\pgfqpoint{14.587246in}{5.312428in}}%
\pgfpathlineto{\pgfqpoint{14.594092in}{5.313831in}}%
\pgfpathlineto{\pgfqpoint{14.596374in}{5.310424in}}%
\pgfpathlineto{\pgfqpoint{14.598656in}{5.317639in}}%
\pgfpathlineto{\pgfqpoint{14.600938in}{5.335278in}}%
\pgfpathlineto{\pgfqpoint{14.603220in}{5.340489in}}%
\pgfpathlineto{\pgfqpoint{14.610065in}{5.344298in}}%
\pgfpathlineto{\pgfqpoint{14.614629in}{5.335679in}}%
\pgfpathlineto{\pgfqpoint{14.616911in}{5.328663in}}%
\pgfpathlineto{\pgfqpoint{14.619193in}{5.336080in}}%
\pgfpathlineto{\pgfqpoint{14.626039in}{5.334877in}}%
\pgfpathlineto{\pgfqpoint{14.628321in}{5.320846in}}%
\pgfpathlineto{\pgfqpoint{14.630603in}{5.316236in}}%
\pgfpathlineto{\pgfqpoint{14.632885in}{5.290380in}}%
\pgfpathlineto{\pgfqpoint{14.635166in}{5.293386in}}%
\pgfpathlineto{\pgfqpoint{14.642012in}{5.304410in}}%
\pgfpathlineto{\pgfqpoint{14.646576in}{5.303208in}}%
\pgfpathlineto{\pgfqpoint{14.648858in}{5.297596in}}%
\pgfpathlineto{\pgfqpoint{14.651140in}{5.302807in}}%
\pgfpathlineto{\pgfqpoint{14.657986in}{5.292184in}}%
\pgfpathlineto{\pgfqpoint{14.660267in}{5.286171in}}%
\pgfpathlineto{\pgfqpoint{14.662549in}{5.289779in}}%
\pgfpathlineto{\pgfqpoint{14.664831in}{5.284968in}}%
\pgfpathlineto{\pgfqpoint{14.667113in}{5.292384in}}%
\pgfpathlineto{\pgfqpoint{14.673959in}{5.300803in}}%
\pgfpathlineto{\pgfqpoint{14.676241in}{5.321448in}}%
\pgfpathlineto{\pgfqpoint{14.678523in}{5.328864in}}%
\pgfpathlineto{\pgfqpoint{14.680805in}{5.333674in}}%
\pgfpathlineto{\pgfqpoint{14.683086in}{5.333875in}}%
\pgfpathlineto{\pgfqpoint{14.689932in}{5.325657in}}%
\pgfpathlineto{\pgfqpoint{14.692214in}{5.343897in}}%
\pgfpathlineto{\pgfqpoint{14.694496in}{5.346903in}}%
\pgfpathlineto{\pgfqpoint{14.696778in}{5.372960in}}%
\pgfpathlineto{\pgfqpoint{14.699060in}{5.364141in}}%
\pgfpathlineto{\pgfqpoint{14.705906in}{5.375766in}}%
\pgfpathlineto{\pgfqpoint{14.708187in}{5.380978in}}%
\pgfpathlineto{\pgfqpoint{14.710469in}{5.379775in}}%
\pgfpathlineto{\pgfqpoint{14.715033in}{5.372760in}}%
\pgfpathlineto{\pgfqpoint{14.721879in}{5.387392in}}%
\pgfpathlineto{\pgfqpoint{14.724161in}{5.389997in}}%
\pgfpathlineto{\pgfqpoint{14.726443in}{5.396612in}}%
\pgfpathlineto{\pgfqpoint{14.728725in}{5.393204in}}%
\pgfpathlineto{\pgfqpoint{14.731007in}{5.385588in}}%
\pgfpathlineto{\pgfqpoint{14.737852in}{5.391401in}}%
\pgfpathlineto{\pgfqpoint{14.740134in}{5.401823in}}%
\pgfpathlineto{\pgfqpoint{14.742416in}{5.406433in}}%
\pgfpathlineto{\pgfqpoint{14.744698in}{5.399418in}}%
\pgfpathlineto{\pgfqpoint{14.746980in}{5.407035in}}%
\pgfpathlineto{\pgfqpoint{14.753826in}{5.414250in}}%
\pgfpathlineto{\pgfqpoint{14.756107in}{5.413048in}}%
\pgfpathlineto{\pgfqpoint{14.758389in}{5.406433in}}%
\pgfpathlineto{\pgfqpoint{14.760671in}{5.403226in}}%
\pgfpathlineto{\pgfqpoint{14.762953in}{5.407636in}}%
\pgfpathlineto{\pgfqpoint{14.769799in}{5.407035in}}%
\pgfpathlineto{\pgfqpoint{14.772081in}{5.404028in}}%
\pgfpathlineto{\pgfqpoint{14.774363in}{5.395610in}}%
\pgfpathlineto{\pgfqpoint{14.776645in}{5.403226in}}%
\pgfpathlineto{\pgfqpoint{14.778927in}{5.408037in}}%
\pgfpathlineto{\pgfqpoint{14.788054in}{5.411645in}}%
\pgfpathlineto{\pgfqpoint{14.790336in}{5.411645in}}%
\pgfpathlineto{\pgfqpoint{14.792618in}{5.416455in}}%
\pgfpathlineto{\pgfqpoint{14.794900in}{5.413850in}}%
\pgfpathlineto{\pgfqpoint{14.801746in}{5.435697in}}%
\pgfpathlineto{\pgfqpoint{14.804028in}{5.426678in}}%
\pgfpathlineto{\pgfqpoint{14.806309in}{5.427479in}}%
\pgfpathlineto{\pgfqpoint{14.808591in}{5.427479in}}%
\pgfpathlineto{\pgfqpoint{14.810873in}{5.422068in}}%
\pgfpathlineto{\pgfqpoint{14.817719in}{5.419863in}}%
\pgfpathlineto{\pgfqpoint{14.820001in}{5.439105in}}%
\pgfpathlineto{\pgfqpoint{14.822283in}{5.443514in}}%
\pgfpathlineto{\pgfqpoint{14.824565in}{5.410242in}}%
\pgfpathlineto{\pgfqpoint{14.826847in}{5.402625in}}%
\pgfpathlineto{\pgfqpoint{14.833692in}{5.411645in}}%
\pgfpathlineto{\pgfqpoint{14.835974in}{5.410442in}}%
\pgfpathlineto{\pgfqpoint{14.838256in}{5.376568in}}%
\pgfpathlineto{\pgfqpoint{14.840538in}{5.376969in}}%
\pgfpathlineto{\pgfqpoint{14.842820in}{5.378773in}}%
\pgfpathlineto{\pgfqpoint{14.849666in}{5.393605in}}%
\pgfpathlineto{\pgfqpoint{14.854229in}{5.406033in}}%
\pgfpathlineto{\pgfqpoint{14.856511in}{5.398616in}}%
\pgfpathlineto{\pgfqpoint{14.858793in}{5.404229in}}%
\pgfpathlineto{\pgfqpoint{14.865639in}{5.400220in}}%
\pgfpathlineto{\pgfqpoint{14.867921in}{5.390799in}}%
\pgfpathlineto{\pgfqpoint{14.870203in}{5.387793in}}%
\pgfpathlineto{\pgfqpoint{14.872485in}{5.400821in}}%
\pgfpathlineto{\pgfqpoint{14.874767in}{5.417658in}}%
\pgfpathlineto{\pgfqpoint{14.881612in}{5.419662in}}%
\pgfpathlineto{\pgfqpoint{14.883894in}{5.413248in}}%
\pgfpathlineto{\pgfqpoint{14.886176in}{5.412647in}}%
\pgfpathlineto{\pgfqpoint{14.888458in}{5.403226in}}%
\pgfpathlineto{\pgfqpoint{14.890740in}{5.339487in}}%
\pgfpathlineto{\pgfqpoint{14.899868in}{5.315234in}}%
\pgfpathlineto{\pgfqpoint{14.902150in}{5.313029in}}%
\pgfpathlineto{\pgfqpoint{14.904431in}{5.325256in}}%
\pgfpathlineto{\pgfqpoint{14.906713in}{5.316437in}}%
\pgfpathlineto{\pgfqpoint{14.913559in}{5.301805in}}%
\pgfpathlineto{\pgfqpoint{14.915841in}{5.303208in}}%
\pgfpathlineto{\pgfqpoint{14.918123in}{5.313631in}}%
\pgfpathlineto{\pgfqpoint{14.920405in}{5.306415in}}%
\pgfpathlineto{\pgfqpoint{14.922687in}{5.307617in}}%
\pgfpathlineto{\pgfqpoint{14.929532in}{5.297596in}}%
\pgfpathlineto{\pgfqpoint{14.931814in}{5.315234in}}%
\pgfpathlineto{\pgfqpoint{14.934096in}{5.326860in}}%
\pgfpathlineto{\pgfqpoint{14.936378in}{5.330668in}}%
\pgfpathlineto{\pgfqpoint{14.938660in}{5.337884in}}%
\pgfpathlineto{\pgfqpoint{14.945506in}{5.353919in}}%
\pgfpathlineto{\pgfqpoint{14.947788in}{5.351313in}}%
\pgfpathlineto{\pgfqpoint{14.950070in}{5.339086in}}%
\pgfpathlineto{\pgfqpoint{14.952351in}{5.358729in}}%
\pgfpathlineto{\pgfqpoint{14.954633in}{5.342895in}}%
\pgfpathlineto{\pgfqpoint{14.961479in}{5.339888in}}%
\pgfpathlineto{\pgfqpoint{14.963761in}{5.348507in}}%
\pgfpathlineto{\pgfqpoint{14.966043in}{5.341091in}}%
\pgfpathlineto{\pgfqpoint{14.970607in}{5.343295in}}%
\pgfpathlineto{\pgfqpoint{14.977452in}{5.353117in}}%
\pgfpathlineto{\pgfqpoint{14.979734in}{5.361535in}}%
\pgfpathlineto{\pgfqpoint{14.982016in}{5.361134in}}%
\pgfpathlineto{\pgfqpoint{14.984298in}{5.372760in}}%
\pgfpathlineto{\pgfqpoint{14.993426in}{5.399819in}}%
\pgfpathlineto{\pgfqpoint{14.995708in}{5.399619in}}%
\pgfpathlineto{\pgfqpoint{14.997990in}{5.396812in}}%
\pgfpathlineto{\pgfqpoint{15.000272in}{5.374764in}}%
\pgfpathlineto{\pgfqpoint{15.002553in}{5.379976in}}%
\pgfpathlineto{\pgfqpoint{15.009399in}{5.377370in}}%
\pgfpathlineto{\pgfqpoint{15.011681in}{5.370154in}}%
\pgfpathlineto{\pgfqpoint{15.013963in}{5.389797in}}%
\pgfpathlineto{\pgfqpoint{15.016245in}{5.392002in}}%
\pgfpathlineto{\pgfqpoint{15.018527in}{5.409039in}}%
\pgfpathlineto{\pgfqpoint{15.025373in}{5.408839in}}%
\pgfpathlineto{\pgfqpoint{15.029936in}{5.402024in}}%
\pgfpathlineto{\pgfqpoint{15.032218in}{5.404830in}}%
\pgfpathlineto{\pgfqpoint{15.034500in}{5.413649in}}%
\pgfpathlineto{\pgfqpoint{15.043628in}{5.420264in}}%
\pgfpathlineto{\pgfqpoint{15.045910in}{5.413048in}}%
\pgfpathlineto{\pgfqpoint{15.048192in}{5.412447in}}%
\pgfpathlineto{\pgfqpoint{15.050473in}{5.408839in}}%
\pgfpathlineto{\pgfqpoint{15.050473in}{5.408839in}}%
\pgfusepath{stroke}%
\end{pgfscope}%
\begin{pgfscope}%
\pgfsetrectcap%
\pgfsetmiterjoin%
\pgfsetlinewidth{0.803000pt}%
\definecolor{currentstroke}{rgb}{1.000000,1.000000,1.000000}%
\pgfsetstrokecolor{currentstroke}%
\pgfsetdash{}{0pt}%
\pgfpathmoveto{\pgfqpoint{9.810417in}{4.564634in}}%
\pgfpathlineto{\pgfqpoint{9.810417in}{5.485366in}}%
\pgfusepath{stroke}%
\end{pgfscope}%
\begin{pgfscope}%
\pgfsetrectcap%
\pgfsetmiterjoin%
\pgfsetlinewidth{0.803000pt}%
\definecolor{currentstroke}{rgb}{1.000000,1.000000,1.000000}%
\pgfsetstrokecolor{currentstroke}%
\pgfsetdash{}{0pt}%
\pgfpathmoveto{\pgfqpoint{15.300000in}{4.564634in}}%
\pgfpathlineto{\pgfqpoint{15.300000in}{5.485366in}}%
\pgfusepath{stroke}%
\end{pgfscope}%
\begin{pgfscope}%
\pgfsetrectcap%
\pgfsetmiterjoin%
\pgfsetlinewidth{0.803000pt}%
\definecolor{currentstroke}{rgb}{1.000000,1.000000,1.000000}%
\pgfsetstrokecolor{currentstroke}%
\pgfsetdash{}{0pt}%
\pgfpathmoveto{\pgfqpoint{9.810417in}{4.564634in}}%
\pgfpathlineto{\pgfqpoint{15.300000in}{4.564634in}}%
\pgfusepath{stroke}%
\end{pgfscope}%
\begin{pgfscope}%
\pgfsetrectcap%
\pgfsetmiterjoin%
\pgfsetlinewidth{0.803000pt}%
\definecolor{currentstroke}{rgb}{1.000000,1.000000,1.000000}%
\pgfsetstrokecolor{currentstroke}%
\pgfsetdash{}{0pt}%
\pgfpathmoveto{\pgfqpoint{9.810417in}{5.485366in}}%
\pgfpathlineto{\pgfqpoint{15.300000in}{5.485366in}}%
\pgfusepath{stroke}%
\end{pgfscope}%
\begin{pgfscope}%
\definecolor{textcolor}{rgb}{0.150000,0.150000,0.150000}%
\pgfsetstrokecolor{textcolor}%
\pgfsetfillcolor{textcolor}%
\pgftext[x=12.555208in,y=5.568699in,,base]{\color{textcolor}\rmfamily\fontsize{12.000000}{14.400000}\selectfont PG}%
\end{pgfscope}%
\begin{pgfscope}%
\pgfsetbuttcap%
\pgfsetmiterjoin%
\definecolor{currentfill}{rgb}{0.917647,0.917647,0.949020}%
\pgfsetfillcolor{currentfill}%
\pgfsetlinewidth{0.000000pt}%
\definecolor{currentstroke}{rgb}{0.000000,0.000000,0.000000}%
\pgfsetstrokecolor{currentstroke}%
\pgfsetstrokeopacity{0.000000}%
\pgfsetdash{}{0pt}%
\pgfpathmoveto{\pgfqpoint{2.125000in}{2.907317in}}%
\pgfpathlineto{\pgfqpoint{7.614583in}{2.907317in}}%
\pgfpathlineto{\pgfqpoint{7.614583in}{3.828049in}}%
\pgfpathlineto{\pgfqpoint{2.125000in}{3.828049in}}%
\pgfpathclose%
\pgfusepath{fill}%
\end{pgfscope}%
\begin{pgfscope}%
\pgfpathrectangle{\pgfqpoint{2.125000in}{2.907317in}}{\pgfqpoint{5.489583in}{0.920732in}}%
\pgfusepath{clip}%
\pgfsetroundcap%
\pgfsetroundjoin%
\pgfsetlinewidth{0.803000pt}%
\definecolor{currentstroke}{rgb}{1.000000,1.000000,1.000000}%
\pgfsetstrokecolor{currentstroke}%
\pgfsetdash{}{0pt}%
\pgfpathmoveto{\pgfqpoint{2.369963in}{2.907317in}}%
\pgfpathlineto{\pgfqpoint{2.369963in}{3.828049in}}%
\pgfusepath{stroke}%
\end{pgfscope}%
\begin{pgfscope}%
\definecolor{textcolor}{rgb}{0.150000,0.150000,0.150000}%
\pgfsetstrokecolor{textcolor}%
\pgfsetfillcolor{textcolor}%
\pgftext[x=2.369963in,y=2.810095in,,top]{\color{textcolor}\rmfamily\fontsize{10.000000}{12.000000}\selectfont 2012}%
\end{pgfscope}%
\begin{pgfscope}%
\pgfpathrectangle{\pgfqpoint{2.125000in}{2.907317in}}{\pgfqpoint{5.489583in}{0.920732in}}%
\pgfusepath{clip}%
\pgfsetroundcap%
\pgfsetroundjoin%
\pgfsetlinewidth{0.803000pt}%
\definecolor{currentstroke}{rgb}{1.000000,1.000000,1.000000}%
\pgfsetstrokecolor{currentstroke}%
\pgfsetdash{}{0pt}%
\pgfpathmoveto{\pgfqpoint{3.205141in}{2.907317in}}%
\pgfpathlineto{\pgfqpoint{3.205141in}{3.828049in}}%
\pgfusepath{stroke}%
\end{pgfscope}%
\begin{pgfscope}%
\definecolor{textcolor}{rgb}{0.150000,0.150000,0.150000}%
\pgfsetstrokecolor{textcolor}%
\pgfsetfillcolor{textcolor}%
\pgftext[x=3.205141in,y=2.810095in,,top]{\color{textcolor}\rmfamily\fontsize{10.000000}{12.000000}\selectfont 2013}%
\end{pgfscope}%
\begin{pgfscope}%
\pgfpathrectangle{\pgfqpoint{2.125000in}{2.907317in}}{\pgfqpoint{5.489583in}{0.920732in}}%
\pgfusepath{clip}%
\pgfsetroundcap%
\pgfsetroundjoin%
\pgfsetlinewidth{0.803000pt}%
\definecolor{currentstroke}{rgb}{1.000000,1.000000,1.000000}%
\pgfsetstrokecolor{currentstroke}%
\pgfsetdash{}{0pt}%
\pgfpathmoveto{\pgfqpoint{4.038037in}{2.907317in}}%
\pgfpathlineto{\pgfqpoint{4.038037in}{3.828049in}}%
\pgfusepath{stroke}%
\end{pgfscope}%
\begin{pgfscope}%
\definecolor{textcolor}{rgb}{0.150000,0.150000,0.150000}%
\pgfsetstrokecolor{textcolor}%
\pgfsetfillcolor{textcolor}%
\pgftext[x=4.038037in,y=2.810095in,,top]{\color{textcolor}\rmfamily\fontsize{10.000000}{12.000000}\selectfont 2014}%
\end{pgfscope}%
\begin{pgfscope}%
\pgfpathrectangle{\pgfqpoint{2.125000in}{2.907317in}}{\pgfqpoint{5.489583in}{0.920732in}}%
\pgfusepath{clip}%
\pgfsetroundcap%
\pgfsetroundjoin%
\pgfsetlinewidth{0.803000pt}%
\definecolor{currentstroke}{rgb}{1.000000,1.000000,1.000000}%
\pgfsetstrokecolor{currentstroke}%
\pgfsetdash{}{0pt}%
\pgfpathmoveto{\pgfqpoint{4.870933in}{2.907317in}}%
\pgfpathlineto{\pgfqpoint{4.870933in}{3.828049in}}%
\pgfusepath{stroke}%
\end{pgfscope}%
\begin{pgfscope}%
\definecolor{textcolor}{rgb}{0.150000,0.150000,0.150000}%
\pgfsetstrokecolor{textcolor}%
\pgfsetfillcolor{textcolor}%
\pgftext[x=4.870933in,y=2.810095in,,top]{\color{textcolor}\rmfamily\fontsize{10.000000}{12.000000}\selectfont 2015}%
\end{pgfscope}%
\begin{pgfscope}%
\pgfpathrectangle{\pgfqpoint{2.125000in}{2.907317in}}{\pgfqpoint{5.489583in}{0.920732in}}%
\pgfusepath{clip}%
\pgfsetroundcap%
\pgfsetroundjoin%
\pgfsetlinewidth{0.803000pt}%
\definecolor{currentstroke}{rgb}{1.000000,1.000000,1.000000}%
\pgfsetstrokecolor{currentstroke}%
\pgfsetdash{}{0pt}%
\pgfpathmoveto{\pgfqpoint{5.703829in}{2.907317in}}%
\pgfpathlineto{\pgfqpoint{5.703829in}{3.828049in}}%
\pgfusepath{stroke}%
\end{pgfscope}%
\begin{pgfscope}%
\definecolor{textcolor}{rgb}{0.150000,0.150000,0.150000}%
\pgfsetstrokecolor{textcolor}%
\pgfsetfillcolor{textcolor}%
\pgftext[x=5.703829in,y=2.810095in,,top]{\color{textcolor}\rmfamily\fontsize{10.000000}{12.000000}\selectfont 2016}%
\end{pgfscope}%
\begin{pgfscope}%
\pgfpathrectangle{\pgfqpoint{2.125000in}{2.907317in}}{\pgfqpoint{5.489583in}{0.920732in}}%
\pgfusepath{clip}%
\pgfsetroundcap%
\pgfsetroundjoin%
\pgfsetlinewidth{0.803000pt}%
\definecolor{currentstroke}{rgb}{1.000000,1.000000,1.000000}%
\pgfsetstrokecolor{currentstroke}%
\pgfsetdash{}{0pt}%
\pgfpathmoveto{\pgfqpoint{6.539007in}{2.907317in}}%
\pgfpathlineto{\pgfqpoint{6.539007in}{3.828049in}}%
\pgfusepath{stroke}%
\end{pgfscope}%
\begin{pgfscope}%
\definecolor{textcolor}{rgb}{0.150000,0.150000,0.150000}%
\pgfsetstrokecolor{textcolor}%
\pgfsetfillcolor{textcolor}%
\pgftext[x=6.539007in,y=2.810095in,,top]{\color{textcolor}\rmfamily\fontsize{10.000000}{12.000000}\selectfont 2017}%
\end{pgfscope}%
\begin{pgfscope}%
\pgfpathrectangle{\pgfqpoint{2.125000in}{2.907317in}}{\pgfqpoint{5.489583in}{0.920732in}}%
\pgfusepath{clip}%
\pgfsetroundcap%
\pgfsetroundjoin%
\pgfsetlinewidth{0.803000pt}%
\definecolor{currentstroke}{rgb}{1.000000,1.000000,1.000000}%
\pgfsetstrokecolor{currentstroke}%
\pgfsetdash{}{0pt}%
\pgfpathmoveto{\pgfqpoint{7.371903in}{2.907317in}}%
\pgfpathlineto{\pgfqpoint{7.371903in}{3.828049in}}%
\pgfusepath{stroke}%
\end{pgfscope}%
\begin{pgfscope}%
\definecolor{textcolor}{rgb}{0.150000,0.150000,0.150000}%
\pgfsetstrokecolor{textcolor}%
\pgfsetfillcolor{textcolor}%
\pgftext[x=7.371903in,y=2.810095in,,top]{\color{textcolor}\rmfamily\fontsize{10.000000}{12.000000}\selectfont 2018}%
\end{pgfscope}%
\begin{pgfscope}%
\pgfpathrectangle{\pgfqpoint{2.125000in}{2.907317in}}{\pgfqpoint{5.489583in}{0.920732in}}%
\pgfusepath{clip}%
\pgfsetroundcap%
\pgfsetroundjoin%
\pgfsetlinewidth{0.803000pt}%
\definecolor{currentstroke}{rgb}{1.000000,1.000000,1.000000}%
\pgfsetstrokecolor{currentstroke}%
\pgfsetdash{}{0pt}%
\pgfpathmoveto{\pgfqpoint{2.125000in}{3.144309in}}%
\pgfpathlineto{\pgfqpoint{7.614583in}{3.144309in}}%
\pgfusepath{stroke}%
\end{pgfscope}%
\begin{pgfscope}%
\definecolor{textcolor}{rgb}{0.150000,0.150000,0.150000}%
\pgfsetstrokecolor{textcolor}%
\pgfsetfillcolor{textcolor}%
\pgftext[x=1.851047in,y=3.091547in,left,base]{\color{textcolor}\rmfamily\fontsize{10.000000}{12.000000}\selectfont 75}%
\end{pgfscope}%
\begin{pgfscope}%
\pgfpathrectangle{\pgfqpoint{2.125000in}{2.907317in}}{\pgfqpoint{5.489583in}{0.920732in}}%
\pgfusepath{clip}%
\pgfsetroundcap%
\pgfsetroundjoin%
\pgfsetlinewidth{0.803000pt}%
\definecolor{currentstroke}{rgb}{1.000000,1.000000,1.000000}%
\pgfsetstrokecolor{currentstroke}%
\pgfsetdash{}{0pt}%
\pgfpathmoveto{\pgfqpoint{2.125000in}{3.472607in}}%
\pgfpathlineto{\pgfqpoint{7.614583in}{3.472607in}}%
\pgfusepath{stroke}%
\end{pgfscope}%
\begin{pgfscope}%
\definecolor{textcolor}{rgb}{0.150000,0.150000,0.150000}%
\pgfsetstrokecolor{textcolor}%
\pgfsetfillcolor{textcolor}%
\pgftext[x=1.762682in,y=3.419845in,left,base]{\color{textcolor}\rmfamily\fontsize{10.000000}{12.000000}\selectfont 100}%
\end{pgfscope}%
\begin{pgfscope}%
\pgfpathrectangle{\pgfqpoint{2.125000in}{2.907317in}}{\pgfqpoint{5.489583in}{0.920732in}}%
\pgfusepath{clip}%
\pgfsetroundcap%
\pgfsetroundjoin%
\pgfsetlinewidth{0.803000pt}%
\definecolor{currentstroke}{rgb}{1.000000,1.000000,1.000000}%
\pgfsetstrokecolor{currentstroke}%
\pgfsetdash{}{0pt}%
\pgfpathmoveto{\pgfqpoint{2.125000in}{3.800905in}}%
\pgfpathlineto{\pgfqpoint{7.614583in}{3.800905in}}%
\pgfusepath{stroke}%
\end{pgfscope}%
\begin{pgfscope}%
\definecolor{textcolor}{rgb}{0.150000,0.150000,0.150000}%
\pgfsetstrokecolor{textcolor}%
\pgfsetfillcolor{textcolor}%
\pgftext[x=1.762682in,y=3.748144in,left,base]{\color{textcolor}\rmfamily\fontsize{10.000000}{12.000000}\selectfont 125}%
\end{pgfscope}%
\begin{pgfscope}%
\pgfpathrectangle{\pgfqpoint{2.125000in}{2.907317in}}{\pgfqpoint{5.489583in}{0.920732in}}%
\pgfusepath{clip}%
\pgfsetroundcap%
\pgfsetroundjoin%
\pgfsetlinewidth{1.505625pt}%
\definecolor{currentstroke}{rgb}{0.121569,0.466667,0.705882}%
\pgfsetstrokecolor{currentstroke}%
\pgfsetdash{}{0pt}%
\pgfpathmoveto{\pgfqpoint{2.374527in}{2.981342in}}%
\pgfpathlineto{\pgfqpoint{2.376808in}{2.985544in}}%
\pgfpathlineto{\pgfqpoint{2.379090in}{2.977665in}}%
\pgfpathlineto{\pgfqpoint{2.381372in}{2.972937in}}%
\pgfpathlineto{\pgfqpoint{2.388218in}{2.975170in}}%
\pgfpathlineto{\pgfqpoint{2.390500in}{2.996312in}}%
\pgfpathlineto{\pgfqpoint{2.395064in}{3.009707in}}%
\pgfpathlineto{\pgfqpoint{2.397346in}{2.996969in}}%
\pgfpathlineto{\pgfqpoint{2.406473in}{3.007474in}}%
\pgfpathlineto{\pgfqpoint{2.408755in}{3.013646in}}%
\pgfpathlineto{\pgfqpoint{2.413319in}{3.003666in}}%
\pgfpathlineto{\pgfqpoint{2.420165in}{3.005504in}}%
\pgfpathlineto{\pgfqpoint{2.422447in}{3.015616in}}%
\pgfpathlineto{\pgfqpoint{2.424728in}{3.014172in}}%
\pgfpathlineto{\pgfqpoint{2.427010in}{3.011545in}}%
\pgfpathlineto{\pgfqpoint{2.429292in}{3.013909in}}%
\pgfpathlineto{\pgfqpoint{2.436138in}{3.013778in}}%
\pgfpathlineto{\pgfqpoint{2.438420in}{3.021919in}}%
\pgfpathlineto{\pgfqpoint{2.440702in}{3.042536in}}%
\pgfpathlineto{\pgfqpoint{2.442984in}{3.040304in}}%
\pgfpathlineto{\pgfqpoint{2.445266in}{3.051598in}}%
\pgfpathlineto{\pgfqpoint{2.452111in}{3.046345in}}%
\pgfpathlineto{\pgfqpoint{2.454393in}{3.043193in}}%
\pgfpathlineto{\pgfqpoint{2.456675in}{3.059214in}}%
\pgfpathlineto{\pgfqpoint{2.458957in}{3.081670in}}%
\pgfpathlineto{\pgfqpoint{2.461239in}{3.078649in}}%
\pgfpathlineto{\pgfqpoint{2.468085in}{3.093751in}}%
\pgfpathlineto{\pgfqpoint{2.470367in}{3.091125in}}%
\pgfpathlineto{\pgfqpoint{2.472649in}{3.073922in}}%
\pgfpathlineto{\pgfqpoint{2.477212in}{3.086922in}}%
\pgfpathlineto{\pgfqpoint{2.488622in}{3.088892in}}%
\pgfpathlineto{\pgfqpoint{2.490904in}{3.083771in}}%
\pgfpathlineto{\pgfqpoint{2.493186in}{3.089024in}}%
\pgfpathlineto{\pgfqpoint{2.502313in}{3.083508in}}%
\pgfpathlineto{\pgfqpoint{2.504595in}{3.087973in}}%
\pgfpathlineto{\pgfqpoint{2.506877in}{3.091125in}}%
\pgfpathlineto{\pgfqpoint{2.509159in}{3.095327in}}%
\pgfpathlineto{\pgfqpoint{2.516005in}{3.081538in}}%
\pgfpathlineto{\pgfqpoint{2.518287in}{3.060527in}}%
\pgfpathlineto{\pgfqpoint{2.522850in}{3.084165in}}%
\pgfpathlineto{\pgfqpoint{2.525132in}{3.084427in}}%
\pgfpathlineto{\pgfqpoint{2.531978in}{3.089286in}}%
\pgfpathlineto{\pgfqpoint{2.534260in}{3.117782in}}%
\pgfpathlineto{\pgfqpoint{2.536542in}{3.120540in}}%
\pgfpathlineto{\pgfqpoint{2.538824in}{3.121328in}}%
\pgfpathlineto{\pgfqpoint{2.541106in}{3.105832in}}%
\pgfpathlineto{\pgfqpoint{2.547951in}{3.096246in}}%
\pgfpathlineto{\pgfqpoint{2.550233in}{3.081144in}}%
\pgfpathlineto{\pgfqpoint{2.554797in}{3.068012in}}%
\pgfpathlineto{\pgfqpoint{2.557079in}{3.064992in}}%
\pgfpathlineto{\pgfqpoint{2.563925in}{3.083902in}}%
\pgfpathlineto{\pgfqpoint{2.566207in}{3.079569in}}%
\pgfpathlineto{\pgfqpoint{2.568489in}{3.061972in}}%
\pgfpathlineto{\pgfqpoint{2.573052in}{3.077599in}}%
\pgfpathlineto{\pgfqpoint{2.579898in}{3.075366in}}%
\pgfpathlineto{\pgfqpoint{2.582180in}{3.069457in}}%
\pgfpathlineto{\pgfqpoint{2.586744in}{3.062760in}}%
\pgfpathlineto{\pgfqpoint{2.595871in}{3.047658in}}%
\pgfpathlineto{\pgfqpoint{2.598153in}{3.028223in}}%
\pgfpathlineto{\pgfqpoint{2.600435in}{3.040961in}}%
\pgfpathlineto{\pgfqpoint{2.602717in}{3.058032in}}%
\pgfpathlineto{\pgfqpoint{2.604999in}{3.042930in}}%
\pgfpathlineto{\pgfqpoint{2.611845in}{3.043718in}}%
\pgfpathlineto{\pgfqpoint{2.614127in}{3.058820in}}%
\pgfpathlineto{\pgfqpoint{2.616409in}{3.058426in}}%
\pgfpathlineto{\pgfqpoint{2.618691in}{3.049496in}}%
\pgfpathlineto{\pgfqpoint{2.620972in}{3.056194in}}%
\pgfpathlineto{\pgfqpoint{2.627818in}{3.042274in}}%
\pgfpathlineto{\pgfqpoint{2.630100in}{3.043456in}}%
\pgfpathlineto{\pgfqpoint{2.632382in}{3.043062in}}%
\pgfpathlineto{\pgfqpoint{2.634664in}{3.058951in}}%
\pgfpathlineto{\pgfqpoint{2.636946in}{3.067225in}}%
\pgfpathlineto{\pgfqpoint{2.643792in}{3.063285in}}%
\pgfpathlineto{\pgfqpoint{2.646073in}{3.060921in}}%
\pgfpathlineto{\pgfqpoint{2.648355in}{3.062234in}}%
\pgfpathlineto{\pgfqpoint{2.650637in}{3.055143in}}%
\pgfpathlineto{\pgfqpoint{2.652919in}{3.039385in}}%
\pgfpathlineto{\pgfqpoint{2.659765in}{3.033082in}}%
\pgfpathlineto{\pgfqpoint{2.662047in}{3.032031in}}%
\pgfpathlineto{\pgfqpoint{2.664329in}{3.011939in}}%
\pgfpathlineto{\pgfqpoint{2.666611in}{3.015091in}}%
\pgfpathlineto{\pgfqpoint{2.668893in}{3.013909in}}%
\pgfpathlineto{\pgfqpoint{2.675738in}{3.000252in}}%
\pgfpathlineto{\pgfqpoint{2.678020in}{3.002747in}}%
\pgfpathlineto{\pgfqpoint{2.680302in}{2.991716in}}%
\pgfpathlineto{\pgfqpoint{2.682584in}{2.976352in}}%
\pgfpathlineto{\pgfqpoint{2.684866in}{2.965846in}}%
\pgfpathlineto{\pgfqpoint{2.691712in}{2.980554in}}%
\pgfpathlineto{\pgfqpoint{2.693993in}{2.983180in}}%
\pgfpathlineto{\pgfqpoint{2.696275in}{2.984493in}}%
\pgfpathlineto{\pgfqpoint{2.700839in}{2.972937in}}%
\pgfpathlineto{\pgfqpoint{2.709967in}{2.996049in}}%
\pgfpathlineto{\pgfqpoint{2.712249in}{2.980948in}}%
\pgfpathlineto{\pgfqpoint{2.714531in}{2.985150in}}%
\pgfpathlineto{\pgfqpoint{2.716813in}{2.961775in}}%
\pgfpathlineto{\pgfqpoint{2.723658in}{2.955472in}}%
\pgfpathlineto{\pgfqpoint{2.725940in}{2.949169in}}%
\pgfpathlineto{\pgfqpoint{2.728222in}{2.979897in}}%
\pgfpathlineto{\pgfqpoint{2.730504in}{2.999464in}}%
\pgfpathlineto{\pgfqpoint{2.732786in}{3.000646in}}%
\pgfpathlineto{\pgfqpoint{2.741914in}{2.987776in}}%
\pgfpathlineto{\pgfqpoint{2.744195in}{2.978715in}}%
\pgfpathlineto{\pgfqpoint{2.746477in}{2.985413in}}%
\pgfpathlineto{\pgfqpoint{2.748759in}{2.989352in}}%
\pgfpathlineto{\pgfqpoint{2.755605in}{2.994999in}}%
\pgfpathlineto{\pgfqpoint{2.757887in}{3.010626in}}%
\pgfpathlineto{\pgfqpoint{2.760169in}{3.005111in}}%
\pgfpathlineto{\pgfqpoint{2.762451in}{2.993029in}}%
\pgfpathlineto{\pgfqpoint{2.764733in}{2.997231in}}%
\pgfpathlineto{\pgfqpoint{2.771578in}{2.981079in}}%
\pgfpathlineto{\pgfqpoint{2.773860in}{2.979766in}}%
\pgfpathlineto{\pgfqpoint{2.776142in}{2.985675in}}%
\pgfpathlineto{\pgfqpoint{2.778424in}{2.968210in}}%
\pgfpathlineto{\pgfqpoint{2.780706in}{3.000908in}}%
\pgfpathlineto{\pgfqpoint{2.787552in}{2.995524in}}%
\pgfpathlineto{\pgfqpoint{2.789834in}{3.003403in}}%
\pgfpathlineto{\pgfqpoint{2.794397in}{2.999332in}}%
\pgfpathlineto{\pgfqpoint{2.796679in}{2.984887in}}%
\pgfpathlineto{\pgfqpoint{2.803525in}{2.987514in}}%
\pgfpathlineto{\pgfqpoint{2.805807in}{2.985544in}}%
\pgfpathlineto{\pgfqpoint{2.808089in}{2.967291in}}%
\pgfpathlineto{\pgfqpoint{2.810371in}{2.959805in}}%
\pgfpathlineto{\pgfqpoint{2.812653in}{2.979241in}}%
\pgfpathlineto{\pgfqpoint{2.819498in}{2.975038in}}%
\pgfpathlineto{\pgfqpoint{2.821780in}{2.979372in}}%
\pgfpathlineto{\pgfqpoint{2.826344in}{3.004191in}}%
\pgfpathlineto{\pgfqpoint{2.828626in}{2.986463in}}%
\pgfpathlineto{\pgfqpoint{2.835472in}{2.975826in}}%
\pgfpathlineto{\pgfqpoint{2.837754in}{2.960856in}}%
\pgfpathlineto{\pgfqpoint{2.840036in}{2.968341in}}%
\pgfpathlineto{\pgfqpoint{2.842317in}{2.971887in}}%
\pgfpathlineto{\pgfqpoint{2.844599in}{2.986988in}}%
\pgfpathlineto{\pgfqpoint{2.851445in}{2.994736in}}%
\pgfpathlineto{\pgfqpoint{2.853727in}{2.988827in}}%
\pgfpathlineto{\pgfqpoint{2.856009in}{2.992635in}}%
\pgfpathlineto{\pgfqpoint{2.858291in}{2.988827in}}%
\pgfpathlineto{\pgfqpoint{2.860573in}{3.015222in}}%
\pgfpathlineto{\pgfqpoint{2.867418in}{3.013384in}}%
\pgfpathlineto{\pgfqpoint{2.869700in}{3.029799in}}%
\pgfpathlineto{\pgfqpoint{2.871982in}{3.022707in}}%
\pgfpathlineto{\pgfqpoint{2.874264in}{3.017586in}}%
\pgfpathlineto{\pgfqpoint{2.876546in}{3.027172in}}%
\pgfpathlineto{\pgfqpoint{2.883392in}{3.022576in}}%
\pgfpathlineto{\pgfqpoint{2.885674in}{3.025990in}}%
\pgfpathlineto{\pgfqpoint{2.890237in}{3.043062in}}%
\pgfpathlineto{\pgfqpoint{2.892519in}{3.061053in}}%
\pgfpathlineto{\pgfqpoint{2.899365in}{3.057770in}}%
\pgfpathlineto{\pgfqpoint{2.901647in}{3.048708in}}%
\pgfpathlineto{\pgfqpoint{2.903929in}{3.053042in}}%
\pgfpathlineto{\pgfqpoint{2.906211in}{3.047921in}}%
\pgfpathlineto{\pgfqpoint{2.908493in}{3.057770in}}%
\pgfpathlineto{\pgfqpoint{2.915338in}{3.063154in}}%
\pgfpathlineto{\pgfqpoint{2.917620in}{3.064073in}}%
\pgfpathlineto{\pgfqpoint{2.919902in}{3.058820in}}%
\pgfpathlineto{\pgfqpoint{2.922184in}{3.046345in}}%
\pgfpathlineto{\pgfqpoint{2.924466in}{3.055274in}}%
\pgfpathlineto{\pgfqpoint{2.935876in}{3.034789in}}%
\pgfpathlineto{\pgfqpoint{2.938158in}{3.050153in}}%
\pgfpathlineto{\pgfqpoint{2.940439in}{3.050284in}}%
\pgfpathlineto{\pgfqpoint{2.947285in}{3.041355in}}%
\pgfpathlineto{\pgfqpoint{2.949567in}{3.042930in}}%
\pgfpathlineto{\pgfqpoint{2.951849in}{3.043718in}}%
\pgfpathlineto{\pgfqpoint{2.956413in}{3.084427in}}%
\pgfpathlineto{\pgfqpoint{2.963258in}{3.081013in}}%
\pgfpathlineto{\pgfqpoint{2.965540in}{3.073528in}}%
\pgfpathlineto{\pgfqpoint{2.967822in}{3.076154in}}%
\pgfpathlineto{\pgfqpoint{2.970104in}{3.067225in}}%
\pgfpathlineto{\pgfqpoint{2.972386in}{3.065255in}}%
\pgfpathlineto{\pgfqpoint{2.979232in}{3.057244in}}%
\pgfpathlineto{\pgfqpoint{2.981514in}{3.043062in}}%
\pgfpathlineto{\pgfqpoint{2.986078in}{3.038334in}}%
\pgfpathlineto{\pgfqpoint{2.988359in}{3.037678in}}%
\pgfpathlineto{\pgfqpoint{2.997487in}{3.038860in}}%
\pgfpathlineto{\pgfqpoint{2.999769in}{3.040698in}}%
\pgfpathlineto{\pgfqpoint{3.004333in}{3.040173in}}%
\pgfpathlineto{\pgfqpoint{3.011179in}{3.038597in}}%
\pgfpathlineto{\pgfqpoint{3.013460in}{3.026910in}}%
\pgfpathlineto{\pgfqpoint{3.015742in}{3.012202in}}%
\pgfpathlineto{\pgfqpoint{3.018024in}{3.012596in}}%
\pgfpathlineto{\pgfqpoint{3.020306in}{3.011545in}}%
\pgfpathlineto{\pgfqpoint{3.027152in}{3.013646in}}%
\pgfpathlineto{\pgfqpoint{3.029434in}{3.029930in}}%
\pgfpathlineto{\pgfqpoint{3.031716in}{3.040567in}}%
\pgfpathlineto{\pgfqpoint{3.033998in}{3.048315in}}%
\pgfpathlineto{\pgfqpoint{3.036280in}{3.034395in}}%
\pgfpathlineto{\pgfqpoint{3.043125in}{3.032556in}}%
\pgfpathlineto{\pgfqpoint{3.045407in}{3.024020in}}%
\pgfpathlineto{\pgfqpoint{3.047689in}{3.033344in}}%
\pgfpathlineto{\pgfqpoint{3.049971in}{3.026516in}}%
\pgfpathlineto{\pgfqpoint{3.052253in}{3.036758in}}%
\pgfpathlineto{\pgfqpoint{3.063662in}{3.036233in}}%
\pgfpathlineto{\pgfqpoint{3.065944in}{3.046345in}}%
\pgfpathlineto{\pgfqpoint{3.068226in}{3.035183in}}%
\pgfpathlineto{\pgfqpoint{3.075072in}{3.033344in}}%
\pgfpathlineto{\pgfqpoint{3.077354in}{3.056588in}}%
\pgfpathlineto{\pgfqpoint{3.079636in}{3.030849in}}%
\pgfpathlineto{\pgfqpoint{3.081918in}{3.013778in}}%
\pgfpathlineto{\pgfqpoint{3.084200in}{3.010232in}}%
\pgfpathlineto{\pgfqpoint{3.091045in}{3.022707in}}%
\pgfpathlineto{\pgfqpoint{3.093327in}{3.023233in}}%
\pgfpathlineto{\pgfqpoint{3.095609in}{3.002747in}}%
\pgfpathlineto{\pgfqpoint{3.097891in}{3.004848in}}%
\pgfpathlineto{\pgfqpoint{3.107019in}{3.024546in}}%
\pgfpathlineto{\pgfqpoint{3.109301in}{3.025728in}}%
\pgfpathlineto{\pgfqpoint{3.116146in}{3.047527in}}%
\pgfpathlineto{\pgfqpoint{3.122992in}{3.048315in}}%
\pgfpathlineto{\pgfqpoint{3.125274in}{3.049890in}}%
\pgfpathlineto{\pgfqpoint{3.127556in}{3.060921in}}%
\pgfpathlineto{\pgfqpoint{3.129838in}{3.060527in}}%
\pgfpathlineto{\pgfqpoint{3.132120in}{3.064467in}}%
\pgfpathlineto{\pgfqpoint{3.138965in}{3.060921in}}%
\pgfpathlineto{\pgfqpoint{3.141247in}{3.064729in}}%
\pgfpathlineto{\pgfqpoint{3.143529in}{3.065780in}}%
\pgfpathlineto{\pgfqpoint{3.145811in}{3.073003in}}%
\pgfpathlineto{\pgfqpoint{3.148093in}{3.074316in}}%
\pgfpathlineto{\pgfqpoint{3.154939in}{3.074578in}}%
\pgfpathlineto{\pgfqpoint{3.157221in}{3.076679in}}%
\pgfpathlineto{\pgfqpoint{3.159502in}{3.074447in}}%
\pgfpathlineto{\pgfqpoint{3.161784in}{3.067356in}}%
\pgfpathlineto{\pgfqpoint{3.164066in}{3.062891in}}%
\pgfpathlineto{\pgfqpoint{3.170912in}{3.063022in}}%
\pgfpathlineto{\pgfqpoint{3.173194in}{3.088367in}}%
\pgfpathlineto{\pgfqpoint{3.175476in}{3.097428in}}%
\pgfpathlineto{\pgfqpoint{3.177758in}{3.099004in}}%
\pgfpathlineto{\pgfqpoint{3.180040in}{3.091913in}}%
\pgfpathlineto{\pgfqpoint{3.186885in}{3.088498in}}%
\pgfpathlineto{\pgfqpoint{3.193731in}{3.086528in}}%
\pgfpathlineto{\pgfqpoint{3.196013in}{3.072346in}}%
\pgfpathlineto{\pgfqpoint{3.202859in}{3.085872in}}%
\pgfpathlineto{\pgfqpoint{3.207423in}{3.108327in}}%
\pgfpathlineto{\pgfqpoint{3.209704in}{3.111873in}}%
\pgfpathlineto{\pgfqpoint{3.211986in}{3.119490in}}%
\pgfpathlineto{\pgfqpoint{3.218832in}{3.114762in}}%
\pgfpathlineto{\pgfqpoint{3.221114in}{3.103337in}}%
\pgfpathlineto{\pgfqpoint{3.223396in}{3.114631in}}%
\pgfpathlineto{\pgfqpoint{3.225678in}{3.119358in}}%
\pgfpathlineto{\pgfqpoint{3.237087in}{3.130520in}}%
\pgfpathlineto{\pgfqpoint{3.239369in}{3.126056in}}%
\pgfpathlineto{\pgfqpoint{3.241651in}{3.135511in}}%
\pgfpathlineto{\pgfqpoint{3.243933in}{3.141683in}}%
\pgfpathlineto{\pgfqpoint{3.253061in}{3.147592in}}%
\pgfpathlineto{\pgfqpoint{3.255343in}{3.154289in}}%
\pgfpathlineto{\pgfqpoint{3.259906in}{3.173724in}}%
\pgfpathlineto{\pgfqpoint{3.269034in}{3.173462in}}%
\pgfpathlineto{\pgfqpoint{3.271316in}{3.165977in}}%
\pgfpathlineto{\pgfqpoint{3.273598in}{3.148642in}}%
\pgfpathlineto{\pgfqpoint{3.275880in}{3.174381in}}%
\pgfpathlineto{\pgfqpoint{3.282725in}{3.170967in}}%
\pgfpathlineto{\pgfqpoint{3.285007in}{3.167815in}}%
\pgfpathlineto{\pgfqpoint{3.287289in}{3.168997in}}%
\pgfpathlineto{\pgfqpoint{3.289571in}{3.175563in}}%
\pgfpathlineto{\pgfqpoint{3.291853in}{3.177139in}}%
\pgfpathlineto{\pgfqpoint{3.298699in}{3.171755in}}%
\pgfpathlineto{\pgfqpoint{3.300981in}{3.175957in}}%
\pgfpathlineto{\pgfqpoint{3.303263in}{3.176482in}}%
\pgfpathlineto{\pgfqpoint{3.305545in}{3.178715in}}%
\pgfpathlineto{\pgfqpoint{3.307826in}{3.191059in}}%
\pgfpathlineto{\pgfqpoint{3.316954in}{3.193816in}}%
\pgfpathlineto{\pgfqpoint{3.321518in}{3.175563in}}%
\pgfpathlineto{\pgfqpoint{3.323800in}{3.187776in}}%
\pgfpathlineto{\pgfqpoint{3.330645in}{3.163744in}}%
\pgfpathlineto{\pgfqpoint{3.332927in}{3.173068in}}%
\pgfpathlineto{\pgfqpoint{3.335209in}{3.188695in}}%
\pgfpathlineto{\pgfqpoint{3.337491in}{3.188432in}}%
\pgfpathlineto{\pgfqpoint{3.339773in}{3.183705in}}%
\pgfpathlineto{\pgfqpoint{3.346619in}{3.172280in}}%
\pgfpathlineto{\pgfqpoint{3.348901in}{3.193816in}}%
\pgfpathlineto{\pgfqpoint{3.351183in}{3.194342in}}%
\pgfpathlineto{\pgfqpoint{3.353465in}{3.201695in}}%
\pgfpathlineto{\pgfqpoint{3.355746in}{3.205766in}}%
\pgfpathlineto{\pgfqpoint{3.364874in}{3.217848in}}%
\pgfpathlineto{\pgfqpoint{3.367156in}{3.217191in}}%
\pgfpathlineto{\pgfqpoint{3.369438in}{3.221393in}}%
\pgfpathlineto{\pgfqpoint{3.378566in}{3.213645in}}%
\pgfpathlineto{\pgfqpoint{3.383129in}{3.221393in}}%
\pgfpathlineto{\pgfqpoint{3.385411in}{3.209837in}}%
\pgfpathlineto{\pgfqpoint{3.387693in}{3.222969in}}%
\pgfpathlineto{\pgfqpoint{3.394539in}{3.214039in}}%
\pgfpathlineto{\pgfqpoint{3.396821in}{3.212332in}}%
\pgfpathlineto{\pgfqpoint{3.399103in}{3.211938in}}%
\pgfpathlineto{\pgfqpoint{3.401385in}{3.221262in}}%
\pgfpathlineto{\pgfqpoint{3.410512in}{3.215353in}}%
\pgfpathlineto{\pgfqpoint{3.412794in}{3.216141in}}%
\pgfpathlineto{\pgfqpoint{3.415076in}{3.218636in}}%
\pgfpathlineto{\pgfqpoint{3.417358in}{3.218110in}}%
\pgfpathlineto{\pgfqpoint{3.419640in}{3.212201in}}%
\pgfpathlineto{\pgfqpoint{3.426486in}{3.224414in}}%
\pgfpathlineto{\pgfqpoint{3.433331in}{3.248314in}}%
\pgfpathlineto{\pgfqpoint{3.435613in}{3.246869in}}%
\pgfpathlineto{\pgfqpoint{3.442459in}{3.222181in}}%
\pgfpathlineto{\pgfqpoint{3.444741in}{3.233475in}}%
\pgfpathlineto{\pgfqpoint{3.449305in}{3.199857in}}%
\pgfpathlineto{\pgfqpoint{3.451587in}{3.218767in}}%
\pgfpathlineto{\pgfqpoint{3.458432in}{3.223494in}}%
\pgfpathlineto{\pgfqpoint{3.462996in}{3.204978in}}%
\pgfpathlineto{\pgfqpoint{3.465278in}{3.206160in}}%
\pgfpathlineto{\pgfqpoint{3.467560in}{3.195261in}}%
\pgfpathlineto{\pgfqpoint{3.474406in}{3.200645in}}%
\pgfpathlineto{\pgfqpoint{3.476688in}{3.196837in}}%
\pgfpathlineto{\pgfqpoint{3.478969in}{3.194210in}}%
\pgfpathlineto{\pgfqpoint{3.481251in}{3.202089in}}%
\pgfpathlineto{\pgfqpoint{3.483533in}{3.217585in}}%
\pgfpathlineto{\pgfqpoint{3.490379in}{3.221919in}}%
\pgfpathlineto{\pgfqpoint{3.492661in}{3.226252in}}%
\pgfpathlineto{\pgfqpoint{3.497225in}{3.236364in}}%
\pgfpathlineto{\pgfqpoint{3.499507in}{3.241879in}}%
\pgfpathlineto{\pgfqpoint{3.506352in}{3.238465in}}%
\pgfpathlineto{\pgfqpoint{3.508634in}{3.249627in}}%
\pgfpathlineto{\pgfqpoint{3.510916in}{3.254354in}}%
\pgfpathlineto{\pgfqpoint{3.513198in}{3.247001in}}%
\pgfpathlineto{\pgfqpoint{3.515480in}{3.271951in}}%
\pgfpathlineto{\pgfqpoint{3.522326in}{3.270638in}}%
\pgfpathlineto{\pgfqpoint{3.524608in}{3.274184in}}%
\pgfpathlineto{\pgfqpoint{3.526889in}{3.259345in}}%
\pgfpathlineto{\pgfqpoint{3.529171in}{3.250677in}}%
\pgfpathlineto{\pgfqpoint{3.531453in}{3.245556in}}%
\pgfpathlineto{\pgfqpoint{3.540581in}{3.256193in}}%
\pgfpathlineto{\pgfqpoint{3.542863in}{3.246475in}}%
\pgfpathlineto{\pgfqpoint{3.545145in}{3.256062in}}%
\pgfpathlineto{\pgfqpoint{3.547427in}{3.243980in}}%
\pgfpathlineto{\pgfqpoint{3.554272in}{3.247788in}}%
\pgfpathlineto{\pgfqpoint{3.556554in}{3.236889in}}%
\pgfpathlineto{\pgfqpoint{3.558836in}{3.222313in}}%
\pgfpathlineto{\pgfqpoint{3.561118in}{3.219424in}}%
\pgfpathlineto{\pgfqpoint{3.563400in}{3.239384in}}%
\pgfpathlineto{\pgfqpoint{3.570246in}{3.235313in}}%
\pgfpathlineto{\pgfqpoint{3.572528in}{3.230060in}}%
\pgfpathlineto{\pgfqpoint{3.574810in}{3.218636in}}%
\pgfpathlineto{\pgfqpoint{3.577091in}{3.237020in}}%
\pgfpathlineto{\pgfqpoint{3.579373in}{3.233869in}}%
\pgfpathlineto{\pgfqpoint{3.586219in}{3.244899in}}%
\pgfpathlineto{\pgfqpoint{3.588501in}{3.258425in}}%
\pgfpathlineto{\pgfqpoint{3.593065in}{3.214827in}}%
\pgfpathlineto{\pgfqpoint{3.595347in}{3.212858in}}%
\pgfpathlineto{\pgfqpoint{3.602192in}{3.204978in}}%
\pgfpathlineto{\pgfqpoint{3.604474in}{3.208261in}}%
\pgfpathlineto{\pgfqpoint{3.606756in}{3.222050in}}%
\pgfpathlineto{\pgfqpoint{3.609038in}{3.228091in}}%
\pgfpathlineto{\pgfqpoint{3.611320in}{3.221525in}}%
\pgfpathlineto{\pgfqpoint{3.618166in}{3.242142in}}%
\pgfpathlineto{\pgfqpoint{3.620448in}{3.231374in}}%
\pgfpathlineto{\pgfqpoint{3.627293in}{3.262628in}}%
\pgfpathlineto{\pgfqpoint{3.634139in}{3.267749in}}%
\pgfpathlineto{\pgfqpoint{3.636421in}{3.279568in}}%
\pgfpathlineto{\pgfqpoint{3.638703in}{3.276941in}}%
\pgfpathlineto{\pgfqpoint{3.640985in}{3.298740in}}%
\pgfpathlineto{\pgfqpoint{3.650112in}{3.305044in}}%
\pgfpathlineto{\pgfqpoint{3.652394in}{3.302155in}}%
\pgfpathlineto{\pgfqpoint{3.654676in}{3.312135in}}%
\pgfpathlineto{\pgfqpoint{3.656958in}{3.317519in}}%
\pgfpathlineto{\pgfqpoint{3.659240in}{3.330520in}}%
\pgfpathlineto{\pgfqpoint{3.666086in}{3.326317in}}%
\pgfpathlineto{\pgfqpoint{3.668368in}{3.360723in}}%
\pgfpathlineto{\pgfqpoint{3.670650in}{3.359278in}}%
\pgfpathlineto{\pgfqpoint{3.672932in}{3.357046in}}%
\pgfpathlineto{\pgfqpoint{3.675213in}{3.359016in}}%
\pgfpathlineto{\pgfqpoint{3.682059in}{3.360198in}}%
\pgfpathlineto{\pgfqpoint{3.684341in}{3.365844in}}%
\pgfpathlineto{\pgfqpoint{3.686623in}{3.365844in}}%
\pgfpathlineto{\pgfqpoint{3.688905in}{3.384360in}}%
\pgfpathlineto{\pgfqpoint{3.691187in}{3.391058in}}%
\pgfpathlineto{\pgfqpoint{3.698032in}{3.378188in}}%
\pgfpathlineto{\pgfqpoint{3.700314in}{3.361642in}}%
\pgfpathlineto{\pgfqpoint{3.702596in}{3.371097in}}%
\pgfpathlineto{\pgfqpoint{3.704878in}{3.373724in}}%
\pgfpathlineto{\pgfqpoint{3.707160in}{3.366501in}}%
\pgfpathlineto{\pgfqpoint{3.714006in}{3.365582in}}%
\pgfpathlineto{\pgfqpoint{3.716288in}{3.379896in}}%
\pgfpathlineto{\pgfqpoint{3.718570in}{3.366107in}}%
\pgfpathlineto{\pgfqpoint{3.720852in}{3.342338in}}%
\pgfpathlineto{\pgfqpoint{3.723133in}{3.343389in}}%
\pgfpathlineto{\pgfqpoint{3.729979in}{3.338267in}}%
\pgfpathlineto{\pgfqpoint{3.732261in}{3.335510in}}%
\pgfpathlineto{\pgfqpoint{3.734543in}{3.327499in}}%
\pgfpathlineto{\pgfqpoint{3.736825in}{3.341944in}}%
\pgfpathlineto{\pgfqpoint{3.745953in}{3.334459in}}%
\pgfpathlineto{\pgfqpoint{3.748234in}{3.307276in}}%
\pgfpathlineto{\pgfqpoint{3.750516in}{3.307801in}}%
\pgfpathlineto{\pgfqpoint{3.752798in}{3.313317in}}%
\pgfpathlineto{\pgfqpoint{3.755080in}{3.309115in}}%
\pgfpathlineto{\pgfqpoint{3.766490in}{3.346541in}}%
\pgfpathlineto{\pgfqpoint{3.768772in}{3.350217in}}%
\pgfpathlineto{\pgfqpoint{3.771054in}{3.345096in}}%
\pgfpathlineto{\pgfqpoint{3.777899in}{3.359410in}}%
\pgfpathlineto{\pgfqpoint{3.782463in}{3.397886in}}%
\pgfpathlineto{\pgfqpoint{3.784745in}{3.397886in}}%
\pgfpathlineto{\pgfqpoint{3.787027in}{3.404321in}}%
\pgfpathlineto{\pgfqpoint{3.793873in}{3.419423in}}%
\pgfpathlineto{\pgfqpoint{3.798436in}{3.434393in}}%
\pgfpathlineto{\pgfqpoint{3.800718in}{3.445818in}}%
\pgfpathlineto{\pgfqpoint{3.803000in}{3.417978in}}%
\pgfpathlineto{\pgfqpoint{3.809846in}{3.416140in}}%
\pgfpathlineto{\pgfqpoint{3.812128in}{3.422837in}}%
\pgfpathlineto{\pgfqpoint{3.814410in}{3.414301in}}%
\pgfpathlineto{\pgfqpoint{3.816692in}{3.418897in}}%
\pgfpathlineto{\pgfqpoint{3.818974in}{3.415483in}}%
\pgfpathlineto{\pgfqpoint{3.828101in}{3.392765in}}%
\pgfpathlineto{\pgfqpoint{3.830383in}{3.365188in}}%
\pgfpathlineto{\pgfqpoint{3.832665in}{3.350349in}}%
\pgfpathlineto{\pgfqpoint{3.834947in}{3.357046in}}%
\pgfpathlineto{\pgfqpoint{3.841793in}{3.354288in}}%
\pgfpathlineto{\pgfqpoint{3.844075in}{3.339712in}}%
\pgfpathlineto{\pgfqpoint{3.846356in}{3.340631in}}%
\pgfpathlineto{\pgfqpoint{3.848638in}{3.375956in}}%
\pgfpathlineto{\pgfqpoint{3.850920in}{3.388563in}}%
\pgfpathlineto{\pgfqpoint{3.857766in}{3.387249in}}%
\pgfpathlineto{\pgfqpoint{3.860048in}{3.374643in}}%
\pgfpathlineto{\pgfqpoint{3.862330in}{3.381865in}}%
\pgfpathlineto{\pgfqpoint{3.864612in}{3.400119in}}%
\pgfpathlineto{\pgfqpoint{3.866894in}{3.396836in}}%
\pgfpathlineto{\pgfqpoint{3.873739in}{3.395523in}}%
\pgfpathlineto{\pgfqpoint{3.876021in}{3.378451in}}%
\pgfpathlineto{\pgfqpoint{3.878303in}{3.381340in}}%
\pgfpathlineto{\pgfqpoint{3.880585in}{3.389088in}}%
\pgfpathlineto{\pgfqpoint{3.882867in}{3.394341in}}%
\pgfpathlineto{\pgfqpoint{3.889713in}{3.376744in}}%
\pgfpathlineto{\pgfqpoint{3.891995in}{3.381734in}}%
\pgfpathlineto{\pgfqpoint{3.894276in}{3.376350in}}%
\pgfpathlineto{\pgfqpoint{3.896558in}{3.379764in}}%
\pgfpathlineto{\pgfqpoint{3.898840in}{3.394209in}}%
\pgfpathlineto{\pgfqpoint{3.905686in}{3.399594in}}%
\pgfpathlineto{\pgfqpoint{3.907968in}{3.396179in}}%
\pgfpathlineto{\pgfqpoint{3.910250in}{3.407604in}}%
\pgfpathlineto{\pgfqpoint{3.912532in}{3.391452in}}%
\pgfpathlineto{\pgfqpoint{3.914814in}{3.406159in}}%
\pgfpathlineto{\pgfqpoint{3.921659in}{3.400775in}}%
\pgfpathlineto{\pgfqpoint{3.923941in}{3.393553in}}%
\pgfpathlineto{\pgfqpoint{3.926223in}{3.400513in}}%
\pgfpathlineto{\pgfqpoint{3.928505in}{3.414695in}}%
\pgfpathlineto{\pgfqpoint{3.930787in}{3.413513in}}%
\pgfpathlineto{\pgfqpoint{3.937633in}{3.419817in}}%
\pgfpathlineto{\pgfqpoint{3.939915in}{3.419423in}}%
\pgfpathlineto{\pgfqpoint{3.942197in}{3.416271in}}%
\pgfpathlineto{\pgfqpoint{3.944478in}{3.427171in}}%
\pgfpathlineto{\pgfqpoint{3.946760in}{3.432161in}}%
\pgfpathlineto{\pgfqpoint{3.953606in}{3.433474in}}%
\pgfpathlineto{\pgfqpoint{3.958170in}{3.447262in}}%
\pgfpathlineto{\pgfqpoint{3.962734in}{3.439777in}}%
\pgfpathlineto{\pgfqpoint{3.969579in}{3.433474in}}%
\pgfpathlineto{\pgfqpoint{3.974143in}{3.417584in}}%
\pgfpathlineto{\pgfqpoint{3.976425in}{3.418897in}}%
\pgfpathlineto{\pgfqpoint{3.978707in}{3.442798in}}%
\pgfpathlineto{\pgfqpoint{3.985553in}{3.443454in}}%
\pgfpathlineto{\pgfqpoint{3.987835in}{3.441090in}}%
\pgfpathlineto{\pgfqpoint{3.990117in}{3.414301in}}%
\pgfpathlineto{\pgfqpoint{3.992398in}{3.407735in}}%
\pgfpathlineto{\pgfqpoint{3.994680in}{3.399200in}}%
\pgfpathlineto{\pgfqpoint{4.001526in}{3.412857in}}%
\pgfpathlineto{\pgfqpoint{4.003808in}{3.402351in}}%
\pgfpathlineto{\pgfqpoint{4.006090in}{3.427696in}}%
\pgfpathlineto{\pgfqpoint{4.008372in}{3.424019in}}%
\pgfpathlineto{\pgfqpoint{4.010654in}{3.437413in}}%
\pgfpathlineto{\pgfqpoint{4.017499in}{3.439252in}}%
\pgfpathlineto{\pgfqpoint{4.024345in}{3.460920in}}%
\pgfpathlineto{\pgfqpoint{4.026627in}{3.462233in}}%
\pgfpathlineto{\pgfqpoint{4.033473in}{3.461314in}}%
\pgfpathlineto{\pgfqpoint{4.035755in}{3.473658in}}%
\pgfpathlineto{\pgfqpoint{4.040319in}{3.458556in}}%
\pgfpathlineto{\pgfqpoint{4.042600in}{3.463283in}}%
\pgfpathlineto{\pgfqpoint{4.049446in}{3.462101in}}%
\pgfpathlineto{\pgfqpoint{4.051728in}{3.470375in}}%
\pgfpathlineto{\pgfqpoint{4.054010in}{3.471950in}}%
\pgfpathlineto{\pgfqpoint{4.058574in}{3.474052in}}%
\pgfpathlineto{\pgfqpoint{4.065419in}{3.462233in}}%
\pgfpathlineto{\pgfqpoint{4.067701in}{3.460263in}}%
\pgfpathlineto{\pgfqpoint{4.069983in}{3.476809in}}%
\pgfpathlineto{\pgfqpoint{4.072265in}{3.478516in}}%
\pgfpathlineto{\pgfqpoint{4.074547in}{3.478516in}}%
\pgfpathlineto{\pgfqpoint{4.083675in}{3.487446in}}%
\pgfpathlineto{\pgfqpoint{4.085957in}{3.500447in}}%
\pgfpathlineto{\pgfqpoint{4.088239in}{3.485739in}}%
\pgfpathlineto{\pgfqpoint{4.090520in}{3.450677in}}%
\pgfpathlineto{\pgfqpoint{4.097366in}{3.473658in}}%
\pgfpathlineto{\pgfqpoint{4.099648in}{3.474445in}}%
\pgfpathlineto{\pgfqpoint{4.101930in}{3.467617in}}%
\pgfpathlineto{\pgfqpoint{4.104212in}{3.483900in}}%
\pgfpathlineto{\pgfqpoint{4.106494in}{3.476284in}}%
\pgfpathlineto{\pgfqpoint{4.115621in}{3.419423in}}%
\pgfpathlineto{\pgfqpoint{4.117903in}{3.405634in}}%
\pgfpathlineto{\pgfqpoint{4.122467in}{3.438201in}}%
\pgfpathlineto{\pgfqpoint{4.129313in}{3.448313in}}%
\pgfpathlineto{\pgfqpoint{4.131595in}{3.464071in}}%
\pgfpathlineto{\pgfqpoint{4.133877in}{3.468799in}}%
\pgfpathlineto{\pgfqpoint{4.138441in}{3.481405in}}%
\pgfpathlineto{\pgfqpoint{4.147568in}{3.479961in}}%
\pgfpathlineto{\pgfqpoint{4.149850in}{3.483375in}}%
\pgfpathlineto{\pgfqpoint{4.152132in}{3.495194in}}%
\pgfpathlineto{\pgfqpoint{4.154414in}{3.499002in}}%
\pgfpathlineto{\pgfqpoint{4.161260in}{3.511871in}}%
\pgfpathlineto{\pgfqpoint{4.163541in}{3.504912in}}%
\pgfpathlineto{\pgfqpoint{4.165823in}{3.507407in}}%
\pgfpathlineto{\pgfqpoint{4.170387in}{3.518043in}}%
\pgfpathlineto{\pgfqpoint{4.177233in}{3.515023in}}%
\pgfpathlineto{\pgfqpoint{4.179515in}{3.524609in}}%
\pgfpathlineto{\pgfqpoint{4.181797in}{3.522508in}}%
\pgfpathlineto{\pgfqpoint{4.184079in}{3.526579in}}%
\pgfpathlineto{\pgfqpoint{4.186361in}{3.533014in}}%
\pgfpathlineto{\pgfqpoint{4.193206in}{3.526711in}}%
\pgfpathlineto{\pgfqpoint{4.195488in}{3.502023in}}%
\pgfpathlineto{\pgfqpoint{4.197770in}{3.503992in}}%
\pgfpathlineto{\pgfqpoint{4.200052in}{3.470112in}}%
\pgfpathlineto{\pgfqpoint{4.202334in}{3.466698in}}%
\pgfpathlineto{\pgfqpoint{4.209180in}{3.487709in}}%
\pgfpathlineto{\pgfqpoint{4.211462in}{3.490860in}}%
\pgfpathlineto{\pgfqpoint{4.213743in}{3.483113in}}%
\pgfpathlineto{\pgfqpoint{4.216025in}{3.480224in}}%
\pgfpathlineto{\pgfqpoint{4.218307in}{3.489547in}}%
\pgfpathlineto{\pgfqpoint{4.225153in}{3.480092in}}%
\pgfpathlineto{\pgfqpoint{4.227435in}{3.496901in}}%
\pgfpathlineto{\pgfqpoint{4.231999in}{3.480617in}}%
\pgfpathlineto{\pgfqpoint{4.234281in}{3.492305in}}%
\pgfpathlineto{\pgfqpoint{4.241126in}{3.515942in}}%
\pgfpathlineto{\pgfqpoint{4.243408in}{3.528418in}}%
\pgfpathlineto{\pgfqpoint{4.245690in}{3.551136in}}%
\pgfpathlineto{\pgfqpoint{4.247972in}{3.549954in}}%
\pgfpathlineto{\pgfqpoint{4.250254in}{3.531438in}}%
\pgfpathlineto{\pgfqpoint{4.257100in}{3.508195in}}%
\pgfpathlineto{\pgfqpoint{4.259382in}{3.503467in}}%
\pgfpathlineto{\pgfqpoint{4.261663in}{3.518175in}}%
\pgfpathlineto{\pgfqpoint{4.263945in}{3.489416in}}%
\pgfpathlineto{\pgfqpoint{4.266227in}{3.482193in}}%
\pgfpathlineto{\pgfqpoint{4.273073in}{3.493881in}}%
\pgfpathlineto{\pgfqpoint{4.275355in}{3.504255in}}%
\pgfpathlineto{\pgfqpoint{4.277637in}{3.530256in}}%
\pgfpathlineto{\pgfqpoint{4.279919in}{3.536034in}}%
\pgfpathlineto{\pgfqpoint{4.289046in}{3.532883in}}%
\pgfpathlineto{\pgfqpoint{4.291328in}{3.543257in}}%
\pgfpathlineto{\pgfqpoint{4.293610in}{3.548510in}}%
\pgfpathlineto{\pgfqpoint{4.295892in}{3.540762in}}%
\pgfpathlineto{\pgfqpoint{4.298174in}{3.520145in}}%
\pgfpathlineto{\pgfqpoint{4.305020in}{3.525923in}}%
\pgfpathlineto{\pgfqpoint{4.307302in}{3.524215in}}%
\pgfpathlineto{\pgfqpoint{4.309584in}{3.533276in}}%
\pgfpathlineto{\pgfqpoint{4.311865in}{3.515417in}}%
\pgfpathlineto{\pgfqpoint{4.314147in}{3.512134in}}%
\pgfpathlineto{\pgfqpoint{4.320993in}{3.515680in}}%
\pgfpathlineto{\pgfqpoint{4.323275in}{3.506225in}}%
\pgfpathlineto{\pgfqpoint{4.325557in}{3.516730in}}%
\pgfpathlineto{\pgfqpoint{4.327839in}{3.518043in}}%
\pgfpathlineto{\pgfqpoint{4.330121in}{3.517649in}}%
\pgfpathlineto{\pgfqpoint{4.336966in}{3.536691in}}%
\pgfpathlineto{\pgfqpoint{4.339248in}{3.538004in}}%
\pgfpathlineto{\pgfqpoint{4.341530in}{3.528286in}}%
\pgfpathlineto{\pgfqpoint{4.343812in}{3.509376in}}%
\pgfpathlineto{\pgfqpoint{4.346094in}{3.496376in}}%
\pgfpathlineto{\pgfqpoint{4.352940in}{3.501497in}}%
\pgfpathlineto{\pgfqpoint{4.355222in}{3.479042in}}%
\pgfpathlineto{\pgfqpoint{4.357504in}{3.499396in}}%
\pgfpathlineto{\pgfqpoint{4.359785in}{3.501629in}}%
\pgfpathlineto{\pgfqpoint{4.362067in}{3.507538in}}%
\pgfpathlineto{\pgfqpoint{4.373477in}{3.512265in}}%
\pgfpathlineto{\pgfqpoint{4.375759in}{3.516993in}}%
\pgfpathlineto{\pgfqpoint{4.378041in}{3.515417in}}%
\pgfpathlineto{\pgfqpoint{4.387168in}{3.534064in}}%
\pgfpathlineto{\pgfqpoint{4.389450in}{3.526185in}}%
\pgfpathlineto{\pgfqpoint{4.391732in}{3.538267in}}%
\pgfpathlineto{\pgfqpoint{4.394014in}{3.546671in}}%
\pgfpathlineto{\pgfqpoint{4.400860in}{3.560591in}}%
\pgfpathlineto{\pgfqpoint{4.405424in}{3.539711in}}%
\pgfpathlineto{\pgfqpoint{4.407706in}{3.522771in}}%
\pgfpathlineto{\pgfqpoint{4.409987in}{3.522114in}}%
\pgfpathlineto{\pgfqpoint{4.416833in}{3.523034in}}%
\pgfpathlineto{\pgfqpoint{4.419115in}{3.524609in}}%
\pgfpathlineto{\pgfqpoint{4.421397in}{3.527236in}}%
\pgfpathlineto{\pgfqpoint{4.425961in}{3.536166in}}%
\pgfpathlineto{\pgfqpoint{4.432807in}{3.526054in}}%
\pgfpathlineto{\pgfqpoint{4.435088in}{3.509114in}}%
\pgfpathlineto{\pgfqpoint{4.437370in}{3.514235in}}%
\pgfpathlineto{\pgfqpoint{4.439652in}{3.510033in}}%
\pgfpathlineto{\pgfqpoint{4.441934in}{3.519619in}}%
\pgfpathlineto{\pgfqpoint{4.448780in}{3.506487in}}%
\pgfpathlineto{\pgfqpoint{4.451062in}{3.512134in}}%
\pgfpathlineto{\pgfqpoint{4.453344in}{3.502810in}}%
\pgfpathlineto{\pgfqpoint{4.455626in}{3.507407in}}%
\pgfpathlineto{\pgfqpoint{4.464753in}{3.502416in}}%
\pgfpathlineto{\pgfqpoint{4.467035in}{3.490335in}}%
\pgfpathlineto{\pgfqpoint{4.469317in}{3.487709in}}%
\pgfpathlineto{\pgfqpoint{4.471599in}{3.483769in}}%
\pgfpathlineto{\pgfqpoint{4.473881in}{3.491123in}}%
\pgfpathlineto{\pgfqpoint{4.480727in}{3.499790in}}%
\pgfpathlineto{\pgfqpoint{4.483008in}{3.499396in}}%
\pgfpathlineto{\pgfqpoint{4.485290in}{3.493749in}}%
\pgfpathlineto{\pgfqpoint{4.487572in}{3.475233in}}%
\pgfpathlineto{\pgfqpoint{4.489854in}{3.484557in}}%
\pgfpathlineto{\pgfqpoint{4.496700in}{3.477597in}}%
\pgfpathlineto{\pgfqpoint{4.498982in}{3.452909in}}%
\pgfpathlineto{\pgfqpoint{4.501264in}{3.439777in}}%
\pgfpathlineto{\pgfqpoint{4.503546in}{3.430585in}}%
\pgfpathlineto{\pgfqpoint{4.505828in}{3.429797in}}%
\pgfpathlineto{\pgfqpoint{4.512673in}{3.431110in}}%
\pgfpathlineto{\pgfqpoint{4.514955in}{3.411675in}}%
\pgfpathlineto{\pgfqpoint{4.517237in}{3.400644in}}%
\pgfpathlineto{\pgfqpoint{4.519519in}{3.386330in}}%
\pgfpathlineto{\pgfqpoint{4.521801in}{3.381603in}}%
\pgfpathlineto{\pgfqpoint{4.528647in}{3.385148in}}%
\pgfpathlineto{\pgfqpoint{4.530928in}{3.384754in}}%
\pgfpathlineto{\pgfqpoint{4.533210in}{3.370441in}}%
\pgfpathlineto{\pgfqpoint{4.535492in}{3.375431in}}%
\pgfpathlineto{\pgfqpoint{4.537774in}{3.395785in}}%
\pgfpathlineto{\pgfqpoint{4.544620in}{3.392896in}}%
\pgfpathlineto{\pgfqpoint{4.546902in}{3.383441in}}%
\pgfpathlineto{\pgfqpoint{4.549184in}{3.398018in}}%
\pgfpathlineto{\pgfqpoint{4.551466in}{3.400513in}}%
\pgfpathlineto{\pgfqpoint{4.553748in}{3.398806in}}%
\pgfpathlineto{\pgfqpoint{4.560593in}{3.427696in}}%
\pgfpathlineto{\pgfqpoint{4.562875in}{3.433211in}}%
\pgfpathlineto{\pgfqpoint{4.565157in}{3.446737in}}%
\pgfpathlineto{\pgfqpoint{4.567439in}{3.450808in}}%
\pgfpathlineto{\pgfqpoint{4.569721in}{3.442929in}}%
\pgfpathlineto{\pgfqpoint{4.576567in}{3.448050in}}%
\pgfpathlineto{\pgfqpoint{4.578849in}{3.446343in}}%
\pgfpathlineto{\pgfqpoint{4.581130in}{3.439515in}}%
\pgfpathlineto{\pgfqpoint{4.583412in}{3.439646in}}%
\pgfpathlineto{\pgfqpoint{4.585694in}{3.426383in}}%
\pgfpathlineto{\pgfqpoint{4.594822in}{3.438858in}}%
\pgfpathlineto{\pgfqpoint{4.597104in}{3.440696in}}%
\pgfpathlineto{\pgfqpoint{4.599386in}{3.440959in}}%
\pgfpathlineto{\pgfqpoint{4.601668in}{3.434918in}}%
\pgfpathlineto{\pgfqpoint{4.608513in}{3.433737in}}%
\pgfpathlineto{\pgfqpoint{4.610795in}{3.434524in}}%
\pgfpathlineto{\pgfqpoint{4.613077in}{3.432555in}}%
\pgfpathlineto{\pgfqpoint{4.615359in}{3.432686in}}%
\pgfpathlineto{\pgfqpoint{4.617641in}{3.430716in}}%
\pgfpathlineto{\pgfqpoint{4.624487in}{3.430454in}}%
\pgfpathlineto{\pgfqpoint{4.626769in}{3.432949in}}%
\pgfpathlineto{\pgfqpoint{4.629050in}{3.427039in}}%
\pgfpathlineto{\pgfqpoint{4.631332in}{3.432817in}}%
\pgfpathlineto{\pgfqpoint{4.633614in}{3.431898in}}%
\pgfpathlineto{\pgfqpoint{4.640460in}{3.408655in}}%
\pgfpathlineto{\pgfqpoint{4.642742in}{3.396442in}}%
\pgfpathlineto{\pgfqpoint{4.645024in}{3.404058in}}%
\pgfpathlineto{\pgfqpoint{4.647306in}{3.385542in}}%
\pgfpathlineto{\pgfqpoint{4.649588in}{3.394341in}}%
\pgfpathlineto{\pgfqpoint{4.656433in}{3.392371in}}%
\pgfpathlineto{\pgfqpoint{4.658715in}{3.398543in}}%
\pgfpathlineto{\pgfqpoint{4.660997in}{3.377795in}}%
\pgfpathlineto{\pgfqpoint{4.663279in}{3.370835in}}%
\pgfpathlineto{\pgfqpoint{4.665561in}{3.384623in}}%
\pgfpathlineto{\pgfqpoint{4.672407in}{3.382259in}}%
\pgfpathlineto{\pgfqpoint{4.674689in}{3.349430in}}%
\pgfpathlineto{\pgfqpoint{4.676971in}{3.363612in}}%
\pgfpathlineto{\pgfqpoint{4.679252in}{3.332095in}}%
\pgfpathlineto{\pgfqpoint{4.681534in}{3.332095in}}%
\pgfpathlineto{\pgfqpoint{4.688380in}{3.324742in}}%
\pgfpathlineto{\pgfqpoint{4.690662in}{3.334197in}}%
\pgfpathlineto{\pgfqpoint{4.692944in}{3.323034in}}%
\pgfpathlineto{\pgfqpoint{4.695226in}{3.323822in}}%
\pgfpathlineto{\pgfqpoint{4.697508in}{3.350743in}}%
\pgfpathlineto{\pgfqpoint{4.704353in}{3.350217in}}%
\pgfpathlineto{\pgfqpoint{4.706635in}{3.355996in}}%
\pgfpathlineto{\pgfqpoint{4.708917in}{3.346672in}}%
\pgfpathlineto{\pgfqpoint{4.711199in}{3.370178in}}%
\pgfpathlineto{\pgfqpoint{4.713481in}{3.377663in}}%
\pgfpathlineto{\pgfqpoint{4.720327in}{3.381997in}}%
\pgfpathlineto{\pgfqpoint{4.722609in}{3.406422in}}%
\pgfpathlineto{\pgfqpoint{4.724891in}{3.401432in}}%
\pgfpathlineto{\pgfqpoint{4.727172in}{3.407210in}}%
\pgfpathlineto{\pgfqpoint{4.729454in}{3.414958in}}%
\pgfpathlineto{\pgfqpoint{4.736300in}{3.406685in}}%
\pgfpathlineto{\pgfqpoint{4.738582in}{3.413513in}}%
\pgfpathlineto{\pgfqpoint{4.740864in}{3.425857in}}%
\pgfpathlineto{\pgfqpoint{4.745428in}{3.439383in}}%
\pgfpathlineto{\pgfqpoint{4.752273in}{3.438464in}}%
\pgfpathlineto{\pgfqpoint{4.754555in}{3.429666in}}%
\pgfpathlineto{\pgfqpoint{4.756837in}{3.435312in}}%
\pgfpathlineto{\pgfqpoint{4.759119in}{3.435312in}}%
\pgfpathlineto{\pgfqpoint{4.761401in}{3.427171in}}%
\pgfpathlineto{\pgfqpoint{4.768247in}{3.426120in}}%
\pgfpathlineto{\pgfqpoint{4.770529in}{3.443060in}}%
\pgfpathlineto{\pgfqpoint{4.772811in}{3.441353in}}%
\pgfpathlineto{\pgfqpoint{4.775093in}{3.443323in}}%
\pgfpathlineto{\pgfqpoint{4.777374in}{3.460788in}}%
\pgfpathlineto{\pgfqpoint{4.784220in}{3.442929in}}%
\pgfpathlineto{\pgfqpoint{4.786502in}{3.477334in}}%
\pgfpathlineto{\pgfqpoint{4.788784in}{3.459081in}}%
\pgfpathlineto{\pgfqpoint{4.793348in}{3.458162in}}%
\pgfpathlineto{\pgfqpoint{4.800194in}{3.453566in}}%
\pgfpathlineto{\pgfqpoint{4.802475in}{3.453434in}}%
\pgfpathlineto{\pgfqpoint{4.804757in}{3.468930in}}%
\pgfpathlineto{\pgfqpoint{4.807039in}{3.471425in}}%
\pgfpathlineto{\pgfqpoint{4.809321in}{3.472476in}}%
\pgfpathlineto{\pgfqpoint{4.816167in}{3.493224in}}%
\pgfpathlineto{\pgfqpoint{4.818449in}{3.516074in}}%
\pgfpathlineto{\pgfqpoint{4.820731in}{3.498477in}}%
\pgfpathlineto{\pgfqpoint{4.823013in}{3.504912in}}%
\pgfpathlineto{\pgfqpoint{4.825294in}{3.482587in}}%
\pgfpathlineto{\pgfqpoint{4.832140in}{3.479698in}}%
\pgfpathlineto{\pgfqpoint{4.834422in}{3.495325in}}%
\pgfpathlineto{\pgfqpoint{4.836704in}{3.503204in}}%
\pgfpathlineto{\pgfqpoint{4.838986in}{3.539317in}}%
\pgfpathlineto{\pgfqpoint{4.841268in}{3.523165in}}%
\pgfpathlineto{\pgfqpoint{4.848114in}{3.541418in}}%
\pgfpathlineto{\pgfqpoint{4.850395in}{3.542338in}}%
\pgfpathlineto{\pgfqpoint{4.852677in}{3.538792in}}%
\pgfpathlineto{\pgfqpoint{4.857241in}{3.542600in}}%
\pgfpathlineto{\pgfqpoint{4.864087in}{3.538135in}}%
\pgfpathlineto{\pgfqpoint{4.866369in}{3.530387in}}%
\pgfpathlineto{\pgfqpoint{4.868651in}{3.516205in}}%
\pgfpathlineto{\pgfqpoint{4.873215in}{3.516730in}}%
\pgfpathlineto{\pgfqpoint{4.880060in}{3.494012in}}%
\pgfpathlineto{\pgfqpoint{4.882342in}{3.475102in}}%
\pgfpathlineto{\pgfqpoint{4.884624in}{3.489416in}}%
\pgfpathlineto{\pgfqpoint{4.886906in}{3.512134in}}%
\pgfpathlineto{\pgfqpoint{4.889188in}{3.504649in}}%
\pgfpathlineto{\pgfqpoint{4.896034in}{3.509770in}}%
\pgfpathlineto{\pgfqpoint{4.898316in}{3.508457in}}%
\pgfpathlineto{\pgfqpoint{4.900597in}{3.498740in}}%
\pgfpathlineto{\pgfqpoint{4.902879in}{3.498740in}}%
\pgfpathlineto{\pgfqpoint{4.905161in}{3.529731in}}%
\pgfpathlineto{\pgfqpoint{4.914289in}{3.546014in}}%
\pgfpathlineto{\pgfqpoint{4.918853in}{3.580551in}}%
\pgfpathlineto{\pgfqpoint{4.921135in}{3.574905in}}%
\pgfpathlineto{\pgfqpoint{4.927980in}{3.560460in}}%
\pgfpathlineto{\pgfqpoint{4.930262in}{3.565318in}}%
\pgfpathlineto{\pgfqpoint{4.932544in}{3.538792in}}%
\pgfpathlineto{\pgfqpoint{4.934826in}{3.533276in}}%
\pgfpathlineto{\pgfqpoint{4.937108in}{3.513579in}}%
\pgfpathlineto{\pgfqpoint{4.943954in}{3.534852in}}%
\pgfpathlineto{\pgfqpoint{4.946236in}{3.562035in}}%
\pgfpathlineto{\pgfqpoint{4.948517in}{3.549166in}}%
\pgfpathlineto{\pgfqpoint{4.950799in}{3.576481in}}%
\pgfpathlineto{\pgfqpoint{4.953081in}{3.572935in}}%
\pgfpathlineto{\pgfqpoint{4.959927in}{3.567157in}}%
\pgfpathlineto{\pgfqpoint{4.962209in}{3.568076in}}%
\pgfpathlineto{\pgfqpoint{4.964491in}{3.567288in}}%
\pgfpathlineto{\pgfqpoint{4.966773in}{3.577925in}}%
\pgfpathlineto{\pgfqpoint{4.969055in}{3.597623in}}%
\pgfpathlineto{\pgfqpoint{4.978182in}{3.598542in}}%
\pgfpathlineto{\pgfqpoint{4.980464in}{3.606421in}}%
\pgfpathlineto{\pgfqpoint{4.982746in}{3.617321in}}%
\pgfpathlineto{\pgfqpoint{4.985028in}{3.631635in}}%
\pgfpathlineto{\pgfqpoint{4.991874in}{3.626907in}}%
\pgfpathlineto{\pgfqpoint{4.994156in}{3.627826in}}%
\pgfpathlineto{\pgfqpoint{4.996437in}{3.622836in}}%
\pgfpathlineto{\pgfqpoint{5.001001in}{3.605502in}}%
\pgfpathlineto{\pgfqpoint{5.007847in}{3.620604in}}%
\pgfpathlineto{\pgfqpoint{5.010129in}{3.603007in}}%
\pgfpathlineto{\pgfqpoint{5.012411in}{3.595128in}}%
\pgfpathlineto{\pgfqpoint{5.014693in}{3.592239in}}%
\pgfpathlineto{\pgfqpoint{5.016975in}{3.576087in}}%
\pgfpathlineto{\pgfqpoint{5.023820in}{3.603532in}}%
\pgfpathlineto{\pgfqpoint{5.026102in}{3.551924in}}%
\pgfpathlineto{\pgfqpoint{5.028384in}{3.563086in}}%
\pgfpathlineto{\pgfqpoint{5.030666in}{3.597492in}}%
\pgfpathlineto{\pgfqpoint{5.032948in}{3.567813in}}%
\pgfpathlineto{\pgfqpoint{5.039794in}{3.583703in}}%
\pgfpathlineto{\pgfqpoint{5.042076in}{3.581208in}}%
\pgfpathlineto{\pgfqpoint{5.044358in}{3.586461in}}%
\pgfpathlineto{\pgfqpoint{5.046639in}{3.575430in}}%
\pgfpathlineto{\pgfqpoint{5.048921in}{3.576481in}}%
\pgfpathlineto{\pgfqpoint{5.055767in}{3.567157in}}%
\pgfpathlineto{\pgfqpoint{5.058049in}{3.570046in}}%
\pgfpathlineto{\pgfqpoint{5.060331in}{3.540893in}}%
\pgfpathlineto{\pgfqpoint{5.062613in}{3.535903in}}%
\pgfpathlineto{\pgfqpoint{5.064895in}{3.546014in}}%
\pgfpathlineto{\pgfqpoint{5.071740in}{3.568995in}}%
\pgfpathlineto{\pgfqpoint{5.076304in}{3.534458in}}%
\pgfpathlineto{\pgfqpoint{5.078586in}{3.548772in}}%
\pgfpathlineto{\pgfqpoint{5.087714in}{3.557439in}}%
\pgfpathlineto{\pgfqpoint{5.089996in}{3.553106in}}%
\pgfpathlineto{\pgfqpoint{5.092278in}{3.557177in}}%
\pgfpathlineto{\pgfqpoint{5.094559in}{3.557702in}}%
\pgfpathlineto{\pgfqpoint{5.096841in}{3.564530in}}%
\pgfpathlineto{\pgfqpoint{5.103687in}{3.552055in}}%
\pgfpathlineto{\pgfqpoint{5.105969in}{3.555469in}}%
\pgfpathlineto{\pgfqpoint{5.108251in}{3.557571in}}%
\pgfpathlineto{\pgfqpoint{5.110533in}{3.552974in}}%
\pgfpathlineto{\pgfqpoint{5.112815in}{3.524872in}}%
\pgfpathlineto{\pgfqpoint{5.121942in}{3.546671in}}%
\pgfpathlineto{\pgfqpoint{5.124224in}{3.546802in}}%
\pgfpathlineto{\pgfqpoint{5.126506in}{3.550348in}}%
\pgfpathlineto{\pgfqpoint{5.128788in}{3.537216in}}%
\pgfpathlineto{\pgfqpoint{5.135634in}{3.532226in}}%
\pgfpathlineto{\pgfqpoint{5.137916in}{3.536297in}}%
\pgfpathlineto{\pgfqpoint{5.140198in}{3.528024in}}%
\pgfpathlineto{\pgfqpoint{5.142480in}{3.508720in}}%
\pgfpathlineto{\pgfqpoint{5.144761in}{3.528680in}}%
\pgfpathlineto{\pgfqpoint{5.151607in}{3.540762in}}%
\pgfpathlineto{\pgfqpoint{5.153889in}{3.524347in}}%
\pgfpathlineto{\pgfqpoint{5.156171in}{3.524478in}}%
\pgfpathlineto{\pgfqpoint{5.158453in}{3.535903in}}%
\pgfpathlineto{\pgfqpoint{5.160735in}{3.564137in}}%
\pgfpathlineto{\pgfqpoint{5.167581in}{3.555601in}}%
\pgfpathlineto{\pgfqpoint{5.169862in}{3.551530in}}%
\pgfpathlineto{\pgfqpoint{5.172144in}{3.559278in}}%
\pgfpathlineto{\pgfqpoint{5.174426in}{3.580420in}}%
\pgfpathlineto{\pgfqpoint{5.176708in}{3.572672in}}%
\pgfpathlineto{\pgfqpoint{5.183554in}{3.572804in}}%
\pgfpathlineto{\pgfqpoint{5.185836in}{3.579501in}}%
\pgfpathlineto{\pgfqpoint{5.188118in}{3.577137in}}%
\pgfpathlineto{\pgfqpoint{5.190400in}{3.580157in}}%
\pgfpathlineto{\pgfqpoint{5.192681in}{3.574117in}}%
\pgfpathlineto{\pgfqpoint{5.201809in}{3.553237in}}%
\pgfpathlineto{\pgfqpoint{5.204091in}{3.562429in}}%
\pgfpathlineto{\pgfqpoint{5.206373in}{3.563086in}}%
\pgfpathlineto{\pgfqpoint{5.208655in}{3.556914in}}%
\pgfpathlineto{\pgfqpoint{5.215501in}{3.554813in}}%
\pgfpathlineto{\pgfqpoint{5.217782in}{3.558884in}}%
\pgfpathlineto{\pgfqpoint{5.220064in}{3.572804in}}%
\pgfpathlineto{\pgfqpoint{5.222346in}{3.556520in}}%
\pgfpathlineto{\pgfqpoint{5.224628in}{3.554813in}}%
\pgfpathlineto{\pgfqpoint{5.231474in}{3.545095in}}%
\pgfpathlineto{\pgfqpoint{5.233756in}{3.546540in}}%
\pgfpathlineto{\pgfqpoint{5.236038in}{3.560854in}}%
\pgfpathlineto{\pgfqpoint{5.238320in}{3.570571in}}%
\pgfpathlineto{\pgfqpoint{5.240602in}{3.562035in}}%
\pgfpathlineto{\pgfqpoint{5.247447in}{3.526317in}}%
\pgfpathlineto{\pgfqpoint{5.249729in}{3.530256in}}%
\pgfpathlineto{\pgfqpoint{5.252011in}{3.531832in}}%
\pgfpathlineto{\pgfqpoint{5.254293in}{3.541681in}}%
\pgfpathlineto{\pgfqpoint{5.256575in}{3.529731in}}%
\pgfpathlineto{\pgfqpoint{5.263421in}{3.534327in}}%
\pgfpathlineto{\pgfqpoint{5.265703in}{3.533670in}}%
\pgfpathlineto{\pgfqpoint{5.267984in}{3.518700in}}%
\pgfpathlineto{\pgfqpoint{5.270266in}{3.510690in}}%
\pgfpathlineto{\pgfqpoint{5.272548in}{3.513316in}}%
\pgfpathlineto{\pgfqpoint{5.281676in}{3.482456in}}%
\pgfpathlineto{\pgfqpoint{5.283958in}{3.481143in}}%
\pgfpathlineto{\pgfqpoint{5.286240in}{3.466172in}}%
\pgfpathlineto{\pgfqpoint{5.295367in}{3.463677in}}%
\pgfpathlineto{\pgfqpoint{5.297649in}{3.472476in}}%
\pgfpathlineto{\pgfqpoint{5.299931in}{3.455535in}}%
\pgfpathlineto{\pgfqpoint{5.302213in}{3.458425in}}%
\pgfpathlineto{\pgfqpoint{5.304495in}{3.473526in}}%
\pgfpathlineto{\pgfqpoint{5.311341in}{3.490204in}}%
\pgfpathlineto{\pgfqpoint{5.313623in}{3.489416in}}%
\pgfpathlineto{\pgfqpoint{5.315904in}{3.486264in}}%
\pgfpathlineto{\pgfqpoint{5.318186in}{3.486396in}}%
\pgfpathlineto{\pgfqpoint{5.320468in}{3.480224in}}%
\pgfpathlineto{\pgfqpoint{5.327314in}{3.477072in}}%
\pgfpathlineto{\pgfqpoint{5.329596in}{3.384360in}}%
\pgfpathlineto{\pgfqpoint{5.331878in}{3.370572in}}%
\pgfpathlineto{\pgfqpoint{5.334160in}{3.365713in}}%
\pgfpathlineto{\pgfqpoint{5.336442in}{3.343914in}}%
\pgfpathlineto{\pgfqpoint{5.343287in}{3.338661in}}%
\pgfpathlineto{\pgfqpoint{5.345569in}{3.339843in}}%
\pgfpathlineto{\pgfqpoint{5.347851in}{3.344439in}}%
\pgfpathlineto{\pgfqpoint{5.350133in}{3.360854in}}%
\pgfpathlineto{\pgfqpoint{5.352415in}{3.355733in}}%
\pgfpathlineto{\pgfqpoint{5.359261in}{3.345227in}}%
\pgfpathlineto{\pgfqpoint{5.361543in}{3.340500in}}%
\pgfpathlineto{\pgfqpoint{5.363825in}{3.338661in}}%
\pgfpathlineto{\pgfqpoint{5.366106in}{3.339975in}}%
\pgfpathlineto{\pgfqpoint{5.368388in}{3.331701in}}%
\pgfpathlineto{\pgfqpoint{5.375234in}{3.346672in}}%
\pgfpathlineto{\pgfqpoint{5.377516in}{3.335510in}}%
\pgfpathlineto{\pgfqpoint{5.379798in}{3.344177in}}%
\pgfpathlineto{\pgfqpoint{5.382080in}{3.340500in}}%
\pgfpathlineto{\pgfqpoint{5.384362in}{3.344308in}}%
\pgfpathlineto{\pgfqpoint{5.391207in}{3.352187in}}%
\pgfpathlineto{\pgfqpoint{5.393489in}{3.352844in}}%
\pgfpathlineto{\pgfqpoint{5.395771in}{3.338399in}}%
\pgfpathlineto{\pgfqpoint{5.400335in}{3.276022in}}%
\pgfpathlineto{\pgfqpoint{5.407181in}{3.250284in}}%
\pgfpathlineto{\pgfqpoint{5.409463in}{3.225464in}}%
\pgfpathlineto{\pgfqpoint{5.411745in}{3.258557in}}%
\pgfpathlineto{\pgfqpoint{5.414026in}{3.279042in}}%
\pgfpathlineto{\pgfqpoint{5.416308in}{3.278780in}}%
\pgfpathlineto{\pgfqpoint{5.423154in}{3.259213in}}%
\pgfpathlineto{\pgfqpoint{5.425436in}{3.236495in}}%
\pgfpathlineto{\pgfqpoint{5.427718in}{3.254748in}}%
\pgfpathlineto{\pgfqpoint{5.430000in}{3.262759in}}%
\pgfpathlineto{\pgfqpoint{5.432282in}{3.248051in}}%
\pgfpathlineto{\pgfqpoint{5.441409in}{3.273658in}}%
\pgfpathlineto{\pgfqpoint{5.443691in}{3.262102in}}%
\pgfpathlineto{\pgfqpoint{5.445973in}{3.256587in}}%
\pgfpathlineto{\pgfqpoint{5.448255in}{3.267749in}}%
\pgfpathlineto{\pgfqpoint{5.455101in}{3.261840in}}%
\pgfpathlineto{\pgfqpoint{5.457383in}{3.271951in}}%
\pgfpathlineto{\pgfqpoint{5.459665in}{3.287184in}}%
\pgfpathlineto{\pgfqpoint{5.461946in}{3.279568in}}%
\pgfpathlineto{\pgfqpoint{5.464228in}{3.252647in}}%
\pgfpathlineto{\pgfqpoint{5.471074in}{3.258425in}}%
\pgfpathlineto{\pgfqpoint{5.473356in}{3.218242in}}%
\pgfpathlineto{\pgfqpoint{5.475638in}{3.203403in}}%
\pgfpathlineto{\pgfqpoint{5.477920in}{3.201695in}}%
\pgfpathlineto{\pgfqpoint{5.480202in}{3.207473in}}%
\pgfpathlineto{\pgfqpoint{5.487047in}{3.201827in}}%
\pgfpathlineto{\pgfqpoint{5.491611in}{3.227697in}}%
\pgfpathlineto{\pgfqpoint{5.493893in}{3.220211in}}%
\pgfpathlineto{\pgfqpoint{5.496175in}{3.237020in}}%
\pgfpathlineto{\pgfqpoint{5.503021in}{3.266961in}}%
\pgfpathlineto{\pgfqpoint{5.505303in}{3.270507in}}%
\pgfpathlineto{\pgfqpoint{5.512148in}{3.304256in}}%
\pgfpathlineto{\pgfqpoint{5.518994in}{3.305044in}}%
\pgfpathlineto{\pgfqpoint{5.521276in}{3.291255in}}%
\pgfpathlineto{\pgfqpoint{5.523558in}{3.265911in}}%
\pgfpathlineto{\pgfqpoint{5.525840in}{3.277992in}}%
\pgfpathlineto{\pgfqpoint{5.528122in}{3.275891in}}%
\pgfpathlineto{\pgfqpoint{5.534968in}{3.264466in}}%
\pgfpathlineto{\pgfqpoint{5.537249in}{3.307276in}}%
\pgfpathlineto{\pgfqpoint{5.541813in}{3.357309in}}%
\pgfpathlineto{\pgfqpoint{5.544095in}{3.367289in}}%
\pgfpathlineto{\pgfqpoint{5.550941in}{3.362693in}}%
\pgfpathlineto{\pgfqpoint{5.553223in}{3.346409in}}%
\pgfpathlineto{\pgfqpoint{5.555505in}{3.351662in}}%
\pgfpathlineto{\pgfqpoint{5.557787in}{3.348510in}}%
\pgfpathlineto{\pgfqpoint{5.560068in}{3.340762in}}%
\pgfpathlineto{\pgfqpoint{5.566914in}{3.352581in}}%
\pgfpathlineto{\pgfqpoint{5.569196in}{3.359935in}}%
\pgfpathlineto{\pgfqpoint{5.571478in}{3.363875in}}%
\pgfpathlineto{\pgfqpoint{5.573760in}{3.369521in}}%
\pgfpathlineto{\pgfqpoint{5.576042in}{3.369521in}}%
\pgfpathlineto{\pgfqpoint{5.587451in}{3.346672in}}%
\pgfpathlineto{\pgfqpoint{5.589733in}{3.358228in}}%
\pgfpathlineto{\pgfqpoint{5.592015in}{3.325135in}}%
\pgfpathlineto{\pgfqpoint{5.598861in}{3.341419in}}%
\pgfpathlineto{\pgfqpoint{5.601143in}{3.338005in}}%
\pgfpathlineto{\pgfqpoint{5.603425in}{3.339975in}}%
\pgfpathlineto{\pgfqpoint{5.605707in}{3.347066in}}%
\pgfpathlineto{\pgfqpoint{5.607989in}{3.346147in}}%
\pgfpathlineto{\pgfqpoint{5.614834in}{3.344571in}}%
\pgfpathlineto{\pgfqpoint{5.617116in}{3.335772in}}%
\pgfpathlineto{\pgfqpoint{5.619398in}{3.334722in}}%
\pgfpathlineto{\pgfqpoint{5.630808in}{3.319883in}}%
\pgfpathlineto{\pgfqpoint{5.633090in}{3.327762in}}%
\pgfpathlineto{\pgfqpoint{5.635371in}{3.310296in}}%
\pgfpathlineto{\pgfqpoint{5.637653in}{3.303599in}}%
\pgfpathlineto{\pgfqpoint{5.639935in}{3.316074in}}%
\pgfpathlineto{\pgfqpoint{5.646781in}{3.318176in}}%
\pgfpathlineto{\pgfqpoint{5.649063in}{3.297952in}}%
\pgfpathlineto{\pgfqpoint{5.651345in}{3.297033in}}%
\pgfpathlineto{\pgfqpoint{5.653627in}{3.293619in}}%
\pgfpathlineto{\pgfqpoint{5.655909in}{3.286659in}}%
\pgfpathlineto{\pgfqpoint{5.662754in}{3.283376in}}%
\pgfpathlineto{\pgfqpoint{5.665036in}{3.286265in}}%
\pgfpathlineto{\pgfqpoint{5.667318in}{3.308852in}}%
\pgfpathlineto{\pgfqpoint{5.671882in}{3.275103in}}%
\pgfpathlineto{\pgfqpoint{5.678728in}{3.290467in}}%
\pgfpathlineto{\pgfqpoint{5.683291in}{3.322903in}}%
\pgfpathlineto{\pgfqpoint{5.685573in}{3.322903in}}%
\pgfpathlineto{\pgfqpoint{5.694701in}{3.320408in}}%
\pgfpathlineto{\pgfqpoint{5.696983in}{3.334722in}}%
\pgfpathlineto{\pgfqpoint{5.699265in}{3.330126in}}%
\pgfpathlineto{\pgfqpoint{5.701547in}{3.320145in}}%
\pgfpathlineto{\pgfqpoint{5.710674in}{3.314105in}}%
\pgfpathlineto{\pgfqpoint{5.712956in}{3.315943in}}%
\pgfpathlineto{\pgfqpoint{5.715238in}{3.284558in}}%
\pgfpathlineto{\pgfqpoint{5.719802in}{3.251597in}}%
\pgfpathlineto{\pgfqpoint{5.728930in}{3.252910in}}%
\pgfpathlineto{\pgfqpoint{5.731212in}{3.232687in}}%
\pgfpathlineto{\pgfqpoint{5.733493in}{3.234919in}}%
\pgfpathlineto{\pgfqpoint{5.735775in}{3.194079in}}%
\pgfpathlineto{\pgfqpoint{5.744903in}{3.189351in}}%
\pgfpathlineto{\pgfqpoint{5.747185in}{3.184361in}}%
\pgfpathlineto{\pgfqpoint{5.751749in}{3.202746in}}%
\pgfpathlineto{\pgfqpoint{5.758594in}{3.185149in}}%
\pgfpathlineto{\pgfqpoint{5.760876in}{3.194210in}}%
\pgfpathlineto{\pgfqpoint{5.763158in}{3.196180in}}%
\pgfpathlineto{\pgfqpoint{5.765440in}{3.203928in}}%
\pgfpathlineto{\pgfqpoint{5.767722in}{3.218898in}}%
\pgfpathlineto{\pgfqpoint{5.774568in}{3.217322in}}%
\pgfpathlineto{\pgfqpoint{5.776850in}{3.191453in}}%
\pgfpathlineto{\pgfqpoint{5.779132in}{3.198018in}}%
\pgfpathlineto{\pgfqpoint{5.781413in}{3.224151in}}%
\pgfpathlineto{\pgfqpoint{5.783695in}{3.220737in}}%
\pgfpathlineto{\pgfqpoint{5.790541in}{3.207867in}}%
\pgfpathlineto{\pgfqpoint{5.792823in}{3.213514in}}%
\pgfpathlineto{\pgfqpoint{5.795105in}{3.210100in}}%
\pgfpathlineto{\pgfqpoint{5.797387in}{3.182260in}}%
\pgfpathlineto{\pgfqpoint{5.799669in}{3.197887in}}%
\pgfpathlineto{\pgfqpoint{5.808796in}{3.204322in}}%
\pgfpathlineto{\pgfqpoint{5.811078in}{3.232030in}}%
\pgfpathlineto{\pgfqpoint{5.813360in}{3.234919in}}%
\pgfpathlineto{\pgfqpoint{5.815642in}{3.233343in}}%
\pgfpathlineto{\pgfqpoint{5.822488in}{3.283770in}}%
\pgfpathlineto{\pgfqpoint{5.824770in}{3.274446in}}%
\pgfpathlineto{\pgfqpoint{5.827052in}{3.298872in}}%
\pgfpathlineto{\pgfqpoint{5.829334in}{3.353106in}}%
\pgfpathlineto{\pgfqpoint{5.838461in}{3.335510in}}%
\pgfpathlineto{\pgfqpoint{5.840743in}{3.316337in}}%
\pgfpathlineto{\pgfqpoint{5.845307in}{3.329338in}}%
\pgfpathlineto{\pgfqpoint{5.847589in}{3.340106in}}%
\pgfpathlineto{\pgfqpoint{5.856716in}{3.338924in}}%
\pgfpathlineto{\pgfqpoint{5.858998in}{3.335247in}}%
\pgfpathlineto{\pgfqpoint{5.861280in}{3.330126in}}%
\pgfpathlineto{\pgfqpoint{5.863562in}{3.337086in}}%
\pgfpathlineto{\pgfqpoint{5.870408in}{3.338005in}}%
\pgfpathlineto{\pgfqpoint{5.872690in}{3.331439in}}%
\pgfpathlineto{\pgfqpoint{5.877254in}{3.362561in}}%
\pgfpathlineto{\pgfqpoint{5.879535in}{3.365057in}}%
\pgfpathlineto{\pgfqpoint{5.886381in}{3.366501in}}%
\pgfpathlineto{\pgfqpoint{5.888663in}{3.360329in}}%
\pgfpathlineto{\pgfqpoint{5.890945in}{3.365844in}}%
\pgfpathlineto{\pgfqpoint{5.893227in}{3.365188in}}%
\pgfpathlineto{\pgfqpoint{5.902355in}{3.363349in}}%
\pgfpathlineto{\pgfqpoint{5.904636in}{3.378057in}}%
\pgfpathlineto{\pgfqpoint{5.906918in}{3.379633in}}%
\pgfpathlineto{\pgfqpoint{5.911482in}{3.376219in}}%
\pgfpathlineto{\pgfqpoint{5.918328in}{3.379896in}}%
\pgfpathlineto{\pgfqpoint{5.920610in}{3.374905in}}%
\pgfpathlineto{\pgfqpoint{5.927455in}{3.392765in}}%
\pgfpathlineto{\pgfqpoint{5.934301in}{3.402745in}}%
\pgfpathlineto{\pgfqpoint{5.936583in}{3.414170in}}%
\pgfpathlineto{\pgfqpoint{5.938865in}{3.432817in}}%
\pgfpathlineto{\pgfqpoint{5.941147in}{3.433605in}}%
\pgfpathlineto{\pgfqpoint{5.943429in}{3.432292in}}%
\pgfpathlineto{\pgfqpoint{5.950275in}{3.439121in}}%
\pgfpathlineto{\pgfqpoint{5.952556in}{3.437545in}}%
\pgfpathlineto{\pgfqpoint{5.954838in}{3.443979in}}%
\pgfpathlineto{\pgfqpoint{5.957120in}{3.442798in}}%
\pgfpathlineto{\pgfqpoint{5.959402in}{3.446081in}}%
\pgfpathlineto{\pgfqpoint{5.966248in}{3.439909in}}%
\pgfpathlineto{\pgfqpoint{5.968530in}{3.435050in}}%
\pgfpathlineto{\pgfqpoint{5.970812in}{3.448313in}}%
\pgfpathlineto{\pgfqpoint{5.973094in}{3.428090in}}%
\pgfpathlineto{\pgfqpoint{5.975376in}{3.429797in}}%
\pgfpathlineto{\pgfqpoint{5.982221in}{3.429797in}}%
\pgfpathlineto{\pgfqpoint{5.984503in}{3.402614in}}%
\pgfpathlineto{\pgfqpoint{5.986785in}{3.384754in}}%
\pgfpathlineto{\pgfqpoint{5.989067in}{3.380158in}}%
\pgfpathlineto{\pgfqpoint{5.991349in}{3.390270in}}%
\pgfpathlineto{\pgfqpoint{5.998195in}{3.377795in}}%
\pgfpathlineto{\pgfqpoint{6.000477in}{3.402876in}}%
\pgfpathlineto{\pgfqpoint{6.002758in}{3.396179in}}%
\pgfpathlineto{\pgfqpoint{6.005040in}{3.394472in}}%
\pgfpathlineto{\pgfqpoint{6.007322in}{3.379896in}}%
\pgfpathlineto{\pgfqpoint{6.014168in}{3.398806in}}%
\pgfpathlineto{\pgfqpoint{6.016450in}{3.377401in}}%
\pgfpathlineto{\pgfqpoint{6.018732in}{3.376087in}}%
\pgfpathlineto{\pgfqpoint{6.021014in}{3.366501in}}%
\pgfpathlineto{\pgfqpoint{6.023296in}{3.373724in}}%
\pgfpathlineto{\pgfqpoint{6.030141in}{3.371229in}}%
\pgfpathlineto{\pgfqpoint{6.032423in}{3.383573in}}%
\pgfpathlineto{\pgfqpoint{6.034705in}{3.388694in}}%
\pgfpathlineto{\pgfqpoint{6.036987in}{3.389745in}}%
\pgfpathlineto{\pgfqpoint{6.039269in}{3.394078in}}%
\pgfpathlineto{\pgfqpoint{6.048397in}{3.391846in}}%
\pgfpathlineto{\pgfqpoint{6.050678in}{3.388957in}}%
\pgfpathlineto{\pgfqpoint{6.052960in}{3.394472in}}%
\pgfpathlineto{\pgfqpoint{6.055242in}{3.390007in}}%
\pgfpathlineto{\pgfqpoint{6.062088in}{3.401432in}}%
\pgfpathlineto{\pgfqpoint{6.064370in}{3.402483in}}%
\pgfpathlineto{\pgfqpoint{6.066652in}{3.410099in}}%
\pgfpathlineto{\pgfqpoint{6.068934in}{3.413119in}}%
\pgfpathlineto{\pgfqpoint{6.071216in}{3.409180in}}%
\pgfpathlineto{\pgfqpoint{6.078061in}{3.399594in}}%
\pgfpathlineto{\pgfqpoint{6.080343in}{3.399331in}}%
\pgfpathlineto{\pgfqpoint{6.082625in}{3.391320in}}%
\pgfpathlineto{\pgfqpoint{6.084907in}{3.398412in}}%
\pgfpathlineto{\pgfqpoint{6.087189in}{3.399462in}}%
\pgfpathlineto{\pgfqpoint{6.094035in}{3.405503in}}%
\pgfpathlineto{\pgfqpoint{6.096317in}{3.403927in}}%
\pgfpathlineto{\pgfqpoint{6.098599in}{3.401301in}}%
\pgfpathlineto{\pgfqpoint{6.100880in}{3.413251in}}%
\pgfpathlineto{\pgfqpoint{6.103162in}{3.371097in}}%
\pgfpathlineto{\pgfqpoint{6.110008in}{3.350480in}}%
\pgfpathlineto{\pgfqpoint{6.116854in}{3.416008in}}%
\pgfpathlineto{\pgfqpoint{6.119136in}{3.418241in}}%
\pgfpathlineto{\pgfqpoint{6.128263in}{3.392371in}}%
\pgfpathlineto{\pgfqpoint{6.132827in}{3.408917in}}%
\pgfpathlineto{\pgfqpoint{6.135109in}{3.429534in}}%
\pgfpathlineto{\pgfqpoint{6.141955in}{3.433211in}}%
\pgfpathlineto{\pgfqpoint{6.146519in}{3.447131in}}%
\pgfpathlineto{\pgfqpoint{6.148800in}{3.447656in}}%
\pgfpathlineto{\pgfqpoint{6.151082in}{3.452121in}}%
\pgfpathlineto{\pgfqpoint{6.157928in}{3.451990in}}%
\pgfpathlineto{\pgfqpoint{6.160210in}{3.453566in}}%
\pgfpathlineto{\pgfqpoint{6.162492in}{3.458687in}}%
\pgfpathlineto{\pgfqpoint{6.164774in}{3.456455in}}%
\pgfpathlineto{\pgfqpoint{6.167056in}{3.447525in}}%
\pgfpathlineto{\pgfqpoint{6.173901in}{3.441747in}}%
\pgfpathlineto{\pgfqpoint{6.176183in}{3.481405in}}%
\pgfpathlineto{\pgfqpoint{6.178465in}{3.478910in}}%
\pgfpathlineto{\pgfqpoint{6.180747in}{3.477597in}}%
\pgfpathlineto{\pgfqpoint{6.183029in}{3.478516in}}%
\pgfpathlineto{\pgfqpoint{6.189875in}{3.469718in}}%
\pgfpathlineto{\pgfqpoint{6.192157in}{3.461576in}}%
\pgfpathlineto{\pgfqpoint{6.196721in}{3.462627in}}%
\pgfpathlineto{\pgfqpoint{6.199002in}{3.479567in}}%
\pgfpathlineto{\pgfqpoint{6.205848in}{3.480224in}}%
\pgfpathlineto{\pgfqpoint{6.208130in}{3.486921in}}%
\pgfpathlineto{\pgfqpoint{6.210412in}{3.484163in}}%
\pgfpathlineto{\pgfqpoint{6.212694in}{3.497164in}}%
\pgfpathlineto{\pgfqpoint{6.214976in}{3.493355in}}%
\pgfpathlineto{\pgfqpoint{6.221821in}{3.503467in}}%
\pgfpathlineto{\pgfqpoint{6.224103in}{3.497032in}}%
\pgfpathlineto{\pgfqpoint{6.228667in}{3.507144in}}%
\pgfpathlineto{\pgfqpoint{6.230949in}{3.504912in}}%
\pgfpathlineto{\pgfqpoint{6.237795in}{3.497032in}}%
\pgfpathlineto{\pgfqpoint{6.240077in}{3.490860in}}%
\pgfpathlineto{\pgfqpoint{6.242359in}{3.490335in}}%
\pgfpathlineto{\pgfqpoint{6.244641in}{3.487315in}}%
\pgfpathlineto{\pgfqpoint{6.246922in}{3.482325in}}%
\pgfpathlineto{\pgfqpoint{6.253768in}{3.490466in}}%
\pgfpathlineto{\pgfqpoint{6.256050in}{3.482850in}}%
\pgfpathlineto{\pgfqpoint{6.258332in}{3.471425in}}%
\pgfpathlineto{\pgfqpoint{6.262896in}{3.477597in}}%
\pgfpathlineto{\pgfqpoint{6.274305in}{3.464334in}}%
\pgfpathlineto{\pgfqpoint{6.276587in}{3.463021in}}%
\pgfpathlineto{\pgfqpoint{6.278869in}{3.425201in}}%
\pgfpathlineto{\pgfqpoint{6.285715in}{3.441878in}}%
\pgfpathlineto{\pgfqpoint{6.287997in}{3.420736in}}%
\pgfpathlineto{\pgfqpoint{6.290279in}{3.412594in}}%
\pgfpathlineto{\pgfqpoint{6.292561in}{3.425595in}}%
\pgfpathlineto{\pgfqpoint{6.294843in}{3.393421in}}%
\pgfpathlineto{\pgfqpoint{6.301688in}{3.397624in}}%
\pgfpathlineto{\pgfqpoint{6.303970in}{3.395129in}}%
\pgfpathlineto{\pgfqpoint{6.306252in}{3.416402in}}%
\pgfpathlineto{\pgfqpoint{6.308534in}{3.429140in}}%
\pgfpathlineto{\pgfqpoint{6.310816in}{3.423756in}}%
\pgfpathlineto{\pgfqpoint{6.317662in}{3.419685in}}%
\pgfpathlineto{\pgfqpoint{6.319943in}{3.421130in}}%
\pgfpathlineto{\pgfqpoint{6.322225in}{3.421130in}}%
\pgfpathlineto{\pgfqpoint{6.324507in}{3.405240in}}%
\pgfpathlineto{\pgfqpoint{6.326789in}{3.411938in}}%
\pgfpathlineto{\pgfqpoint{6.333635in}{3.422180in}}%
\pgfpathlineto{\pgfqpoint{6.335917in}{3.409836in}}%
\pgfpathlineto{\pgfqpoint{6.338199in}{3.419948in}}%
\pgfpathlineto{\pgfqpoint{6.340481in}{3.417847in}}%
\pgfpathlineto{\pgfqpoint{6.342763in}{3.399331in}}%
\pgfpathlineto{\pgfqpoint{6.349608in}{3.391977in}}%
\pgfpathlineto{\pgfqpoint{6.351890in}{3.377007in}}%
\pgfpathlineto{\pgfqpoint{6.354172in}{3.378976in}}%
\pgfpathlineto{\pgfqpoint{6.356454in}{3.390401in}}%
\pgfpathlineto{\pgfqpoint{6.358736in}{3.394209in}}%
\pgfpathlineto{\pgfqpoint{6.365582in}{3.388563in}}%
\pgfpathlineto{\pgfqpoint{6.367864in}{3.392108in}}%
\pgfpathlineto{\pgfqpoint{6.370145in}{3.389219in}}%
\pgfpathlineto{\pgfqpoint{6.372427in}{3.383441in}}%
\pgfpathlineto{\pgfqpoint{6.374709in}{3.375825in}}%
\pgfpathlineto{\pgfqpoint{6.381555in}{3.386330in}}%
\pgfpathlineto{\pgfqpoint{6.383837in}{3.408917in}}%
\pgfpathlineto{\pgfqpoint{6.386119in}{3.404584in}}%
\pgfpathlineto{\pgfqpoint{6.388401in}{3.393028in}}%
\pgfpathlineto{\pgfqpoint{6.390683in}{3.414827in}}%
\pgfpathlineto{\pgfqpoint{6.397528in}{3.419291in}}%
\pgfpathlineto{\pgfqpoint{6.399810in}{3.416928in}}%
\pgfpathlineto{\pgfqpoint{6.404374in}{3.405109in}}%
\pgfpathlineto{\pgfqpoint{6.406656in}{3.408523in}}%
\pgfpathlineto{\pgfqpoint{6.413502in}{3.429797in}}%
\pgfpathlineto{\pgfqpoint{6.415784in}{3.433737in}}%
\pgfpathlineto{\pgfqpoint{6.418065in}{3.451465in}}%
\pgfpathlineto{\pgfqpoint{6.420347in}{3.495851in}}%
\pgfpathlineto{\pgfqpoint{6.422629in}{3.501366in}}%
\pgfpathlineto{\pgfqpoint{6.429475in}{3.489022in}}%
\pgfpathlineto{\pgfqpoint{6.431757in}{3.486921in}}%
\pgfpathlineto{\pgfqpoint{6.434039in}{3.486133in}}%
\pgfpathlineto{\pgfqpoint{6.438603in}{3.480355in}}%
\pgfpathlineto{\pgfqpoint{6.447730in}{3.485870in}}%
\pgfpathlineto{\pgfqpoint{6.450012in}{3.500447in}}%
\pgfpathlineto{\pgfqpoint{6.454576in}{3.508720in}}%
\pgfpathlineto{\pgfqpoint{6.461422in}{3.504649in}}%
\pgfpathlineto{\pgfqpoint{6.463704in}{3.509376in}}%
\pgfpathlineto{\pgfqpoint{6.465986in}{3.495588in}}%
\pgfpathlineto{\pgfqpoint{6.468267in}{3.492568in}}%
\pgfpathlineto{\pgfqpoint{6.470549in}{3.501760in}}%
\pgfpathlineto{\pgfqpoint{6.477395in}{3.491780in}}%
\pgfpathlineto{\pgfqpoint{6.479677in}{3.491911in}}%
\pgfpathlineto{\pgfqpoint{6.481959in}{3.519882in}}%
\pgfpathlineto{\pgfqpoint{6.484241in}{3.504386in}}%
\pgfpathlineto{\pgfqpoint{6.486523in}{3.521195in}}%
\pgfpathlineto{\pgfqpoint{6.493368in}{3.528680in}}%
\pgfpathlineto{\pgfqpoint{6.495650in}{3.527630in}}%
\pgfpathlineto{\pgfqpoint{6.500214in}{3.500578in}}%
\pgfpathlineto{\pgfqpoint{6.502496in}{3.505437in}}%
\pgfpathlineto{\pgfqpoint{6.509342in}{3.533933in}}%
\pgfpathlineto{\pgfqpoint{6.513906in}{3.529862in}}%
\pgfpathlineto{\pgfqpoint{6.516187in}{3.529468in}}%
\pgfpathlineto{\pgfqpoint{6.518469in}{3.531832in}}%
\pgfpathlineto{\pgfqpoint{6.527597in}{3.536034in}}%
\pgfpathlineto{\pgfqpoint{6.529879in}{3.526185in}}%
\pgfpathlineto{\pgfqpoint{6.532161in}{3.530650in}}%
\pgfpathlineto{\pgfqpoint{6.534443in}{3.519094in}}%
\pgfpathlineto{\pgfqpoint{6.543570in}{3.534064in}}%
\pgfpathlineto{\pgfqpoint{6.545852in}{3.534984in}}%
\pgfpathlineto{\pgfqpoint{6.548134in}{3.540630in}}%
\pgfpathlineto{\pgfqpoint{6.550416in}{3.555469in}}%
\pgfpathlineto{\pgfqpoint{6.557262in}{3.542469in}}%
\pgfpathlineto{\pgfqpoint{6.564108in}{3.533933in}}%
\pgfpathlineto{\pgfqpoint{6.566389in}{3.526579in}}%
\pgfpathlineto{\pgfqpoint{6.575517in}{3.523428in}}%
\pgfpathlineto{\pgfqpoint{6.577799in}{3.529074in}}%
\pgfpathlineto{\pgfqpoint{6.580081in}{3.532883in}}%
\pgfpathlineto{\pgfqpoint{6.582363in}{3.533670in}}%
\pgfpathlineto{\pgfqpoint{6.589208in}{3.528024in}}%
\pgfpathlineto{\pgfqpoint{6.591490in}{3.543782in}}%
\pgfpathlineto{\pgfqpoint{6.598336in}{3.520145in}}%
\pgfpathlineto{\pgfqpoint{6.605182in}{3.513579in}}%
\pgfpathlineto{\pgfqpoint{6.607464in}{3.519751in}}%
\pgfpathlineto{\pgfqpoint{6.609746in}{3.501235in}}%
\pgfpathlineto{\pgfqpoint{6.612028in}{3.503467in}}%
\pgfpathlineto{\pgfqpoint{6.614309in}{3.519488in}}%
\pgfpathlineto{\pgfqpoint{6.621155in}{3.532226in}}%
\pgfpathlineto{\pgfqpoint{6.623437in}{3.539317in}}%
\pgfpathlineto{\pgfqpoint{6.625719in}{3.528812in}}%
\pgfpathlineto{\pgfqpoint{6.628001in}{3.525003in}}%
\pgfpathlineto{\pgfqpoint{6.630283in}{3.536822in}}%
\pgfpathlineto{\pgfqpoint{6.637129in}{3.548903in}}%
\pgfpathlineto{\pgfqpoint{6.639410in}{3.543257in}}%
\pgfpathlineto{\pgfqpoint{6.641692in}{3.555732in}}%
\pgfpathlineto{\pgfqpoint{6.643974in}{3.557833in}}%
\pgfpathlineto{\pgfqpoint{6.646256in}{3.558752in}}%
\pgfpathlineto{\pgfqpoint{6.655384in}{3.563086in}}%
\pgfpathlineto{\pgfqpoint{6.657666in}{3.565187in}}%
\pgfpathlineto{\pgfqpoint{6.659948in}{3.557965in}}%
\pgfpathlineto{\pgfqpoint{6.662230in}{3.562692in}}%
\pgfpathlineto{\pgfqpoint{6.669075in}{3.567026in}}%
\pgfpathlineto{\pgfqpoint{6.671357in}{3.563743in}}%
\pgfpathlineto{\pgfqpoint{6.673639in}{3.577925in}}%
\pgfpathlineto{\pgfqpoint{6.675921in}{3.565450in}}%
\pgfpathlineto{\pgfqpoint{6.678203in}{3.561116in}}%
\pgfpathlineto{\pgfqpoint{6.685049in}{3.552712in}}%
\pgfpathlineto{\pgfqpoint{6.687330in}{3.560328in}}%
\pgfpathlineto{\pgfqpoint{6.689612in}{3.553762in}}%
\pgfpathlineto{\pgfqpoint{6.694176in}{3.558621in}}%
\pgfpathlineto{\pgfqpoint{6.701022in}{3.560722in}}%
\pgfpathlineto{\pgfqpoint{6.703304in}{3.554813in}}%
\pgfpathlineto{\pgfqpoint{6.705586in}{3.570440in}}%
\pgfpathlineto{\pgfqpoint{6.707868in}{3.561116in}}%
\pgfpathlineto{\pgfqpoint{6.710150in}{3.575036in}}%
\pgfpathlineto{\pgfqpoint{6.716995in}{3.575693in}}%
\pgfpathlineto{\pgfqpoint{6.719277in}{3.558884in}}%
\pgfpathlineto{\pgfqpoint{6.721559in}{3.555995in}}%
\pgfpathlineto{\pgfqpoint{6.726123in}{3.554419in}}%
\pgfpathlineto{\pgfqpoint{6.732969in}{3.554813in}}%
\pgfpathlineto{\pgfqpoint{6.735251in}{3.566106in}}%
\pgfpathlineto{\pgfqpoint{6.737532in}{3.557702in}}%
\pgfpathlineto{\pgfqpoint{6.739814in}{3.562561in}}%
\pgfpathlineto{\pgfqpoint{6.742096in}{3.559540in}}%
\pgfpathlineto{\pgfqpoint{6.748942in}{3.555995in}}%
\pgfpathlineto{\pgfqpoint{6.751224in}{3.569258in}}%
\pgfpathlineto{\pgfqpoint{6.753506in}{3.563086in}}%
\pgfpathlineto{\pgfqpoint{6.755788in}{3.562035in}}%
\pgfpathlineto{\pgfqpoint{6.758070in}{3.569258in}}%
\pgfpathlineto{\pgfqpoint{6.764915in}{3.567945in}}%
\pgfpathlineto{\pgfqpoint{6.767197in}{3.571096in}}%
\pgfpathlineto{\pgfqpoint{6.769479in}{3.559934in}}%
\pgfpathlineto{\pgfqpoint{6.771761in}{3.557965in}}%
\pgfpathlineto{\pgfqpoint{6.783171in}{3.576349in}}%
\pgfpathlineto{\pgfqpoint{6.785452in}{3.568339in}}%
\pgfpathlineto{\pgfqpoint{6.790016in}{3.594209in}}%
\pgfpathlineto{\pgfqpoint{6.799144in}{3.617715in}}%
\pgfpathlineto{\pgfqpoint{6.801426in}{3.634261in}}%
\pgfpathlineto{\pgfqpoint{6.803708in}{3.641484in}}%
\pgfpathlineto{\pgfqpoint{6.805990in}{3.644110in}}%
\pgfpathlineto{\pgfqpoint{6.812835in}{3.643191in}}%
\pgfpathlineto{\pgfqpoint{6.815117in}{3.648181in}}%
\pgfpathlineto{\pgfqpoint{6.817399in}{3.660525in}}%
\pgfpathlineto{\pgfqpoint{6.819681in}{3.669192in}}%
\pgfpathlineto{\pgfqpoint{6.821963in}{3.673525in}}%
\pgfpathlineto{\pgfqpoint{6.828809in}{3.670242in}}%
\pgfpathlineto{\pgfqpoint{6.831091in}{3.674970in}}%
\pgfpathlineto{\pgfqpoint{6.833373in}{3.669455in}}%
\pgfpathlineto{\pgfqpoint{6.835654in}{3.672738in}}%
\pgfpathlineto{\pgfqpoint{6.837936in}{3.667091in}}%
\pgfpathlineto{\pgfqpoint{6.844782in}{3.668141in}}%
\pgfpathlineto{\pgfqpoint{6.847064in}{3.674182in}}%
\pgfpathlineto{\pgfqpoint{6.849346in}{3.659606in}}%
\pgfpathlineto{\pgfqpoint{6.851628in}{3.656979in}}%
\pgfpathlineto{\pgfqpoint{6.853910in}{3.679435in}}%
\pgfpathlineto{\pgfqpoint{6.860755in}{3.685738in}}%
\pgfpathlineto{\pgfqpoint{6.863037in}{3.690860in}}%
\pgfpathlineto{\pgfqpoint{6.865319in}{3.690991in}}%
\pgfpathlineto{\pgfqpoint{6.867601in}{3.693092in}}%
\pgfpathlineto{\pgfqpoint{6.869883in}{3.688102in}}%
\pgfpathlineto{\pgfqpoint{6.879011in}{3.680879in}}%
\pgfpathlineto{\pgfqpoint{6.881293in}{3.681011in}}%
\pgfpathlineto{\pgfqpoint{6.883574in}{3.687314in}}%
\pgfpathlineto{\pgfqpoint{6.885856in}{3.691516in}}%
\pgfpathlineto{\pgfqpoint{6.892702in}{3.677202in}}%
\pgfpathlineto{\pgfqpoint{6.894984in}{3.666434in}}%
\pgfpathlineto{\pgfqpoint{6.897266in}{3.662101in}}%
\pgfpathlineto{\pgfqpoint{6.899548in}{3.664202in}}%
\pgfpathlineto{\pgfqpoint{6.901830in}{3.672869in}}%
\pgfpathlineto{\pgfqpoint{6.908675in}{3.663414in}}%
\pgfpathlineto{\pgfqpoint{6.910957in}{3.665646in}}%
\pgfpathlineto{\pgfqpoint{6.913239in}{3.664990in}}%
\pgfpathlineto{\pgfqpoint{6.915521in}{3.674313in}}%
\pgfpathlineto{\pgfqpoint{6.917803in}{3.670242in}}%
\pgfpathlineto{\pgfqpoint{6.924649in}{3.688102in}}%
\pgfpathlineto{\pgfqpoint{6.926931in}{3.683900in}}%
\pgfpathlineto{\pgfqpoint{6.929213in}{3.686920in}}%
\pgfpathlineto{\pgfqpoint{6.931495in}{3.692173in}}%
\pgfpathlineto{\pgfqpoint{6.933776in}{3.692698in}}%
\pgfpathlineto{\pgfqpoint{6.940622in}{3.688365in}}%
\pgfpathlineto{\pgfqpoint{6.942904in}{3.684819in}}%
\pgfpathlineto{\pgfqpoint{6.945186in}{3.696244in}}%
\pgfpathlineto{\pgfqpoint{6.947468in}{3.685607in}}%
\pgfpathlineto{\pgfqpoint{6.949750in}{3.691385in}}%
\pgfpathlineto{\pgfqpoint{6.956595in}{3.690991in}}%
\pgfpathlineto{\pgfqpoint{6.961159in}{3.699921in}}%
\pgfpathlineto{\pgfqpoint{6.963441in}{3.688627in}}%
\pgfpathlineto{\pgfqpoint{6.965723in}{3.697688in}}%
\pgfpathlineto{\pgfqpoint{6.972569in}{3.703335in}}%
\pgfpathlineto{\pgfqpoint{6.974851in}{3.709376in}}%
\pgfpathlineto{\pgfqpoint{6.977133in}{3.711477in}}%
\pgfpathlineto{\pgfqpoint{6.979415in}{3.703466in}}%
\pgfpathlineto{\pgfqpoint{6.981696in}{3.707668in}}%
\pgfpathlineto{\pgfqpoint{6.988542in}{3.703860in}}%
\pgfpathlineto{\pgfqpoint{6.990824in}{3.697951in}}%
\pgfpathlineto{\pgfqpoint{6.993106in}{3.703598in}}%
\pgfpathlineto{\pgfqpoint{6.995388in}{3.695587in}}%
\pgfpathlineto{\pgfqpoint{6.997670in}{3.708719in}}%
\pgfpathlineto{\pgfqpoint{7.004516in}{3.704123in}}%
\pgfpathlineto{\pgfqpoint{7.006797in}{3.670111in}}%
\pgfpathlineto{\pgfqpoint{7.011361in}{3.648443in}}%
\pgfpathlineto{\pgfqpoint{7.013643in}{3.650676in}}%
\pgfpathlineto{\pgfqpoint{7.020489in}{3.646999in}}%
\pgfpathlineto{\pgfqpoint{7.022771in}{3.650807in}}%
\pgfpathlineto{\pgfqpoint{7.025053in}{3.669323in}}%
\pgfpathlineto{\pgfqpoint{7.027335in}{3.678647in}}%
\pgfpathlineto{\pgfqpoint{7.029617in}{3.683637in}}%
\pgfpathlineto{\pgfqpoint{7.036462in}{3.646342in}}%
\pgfpathlineto{\pgfqpoint{7.038744in}{3.642140in}}%
\pgfpathlineto{\pgfqpoint{7.041026in}{3.629796in}}%
\pgfpathlineto{\pgfqpoint{7.043308in}{3.624281in}}%
\pgfpathlineto{\pgfqpoint{7.045590in}{3.625857in}}%
\pgfpathlineto{\pgfqpoint{7.052436in}{3.629140in}}%
\pgfpathlineto{\pgfqpoint{7.054717in}{3.614169in}}%
\pgfpathlineto{\pgfqpoint{7.056999in}{3.648837in}}%
\pgfpathlineto{\pgfqpoint{7.059281in}{3.624806in}}%
\pgfpathlineto{\pgfqpoint{7.061563in}{3.617058in}}%
\pgfpathlineto{\pgfqpoint{7.068409in}{3.614432in}}%
\pgfpathlineto{\pgfqpoint{7.070691in}{3.619685in}}%
\pgfpathlineto{\pgfqpoint{7.072973in}{3.636625in}}%
\pgfpathlineto{\pgfqpoint{7.075255in}{3.614694in}}%
\pgfpathlineto{\pgfqpoint{7.077537in}{3.611805in}}%
\pgfpathlineto{\pgfqpoint{7.084382in}{3.615088in}}%
\pgfpathlineto{\pgfqpoint{7.086664in}{3.657636in}}%
\pgfpathlineto{\pgfqpoint{7.088946in}{3.669061in}}%
\pgfpathlineto{\pgfqpoint{7.091228in}{3.670505in}}%
\pgfpathlineto{\pgfqpoint{7.093510in}{3.647787in}}%
\pgfpathlineto{\pgfqpoint{7.104919in}{3.542863in}}%
\pgfpathlineto{\pgfqpoint{7.107201in}{3.547590in}}%
\pgfpathlineto{\pgfqpoint{7.109483in}{3.542206in}}%
\pgfpathlineto{\pgfqpoint{7.116329in}{3.543257in}}%
\pgfpathlineto{\pgfqpoint{7.118611in}{3.546014in}}%
\pgfpathlineto{\pgfqpoint{7.120893in}{3.551399in}}%
\pgfpathlineto{\pgfqpoint{7.123175in}{3.587511in}}%
\pgfpathlineto{\pgfqpoint{7.125457in}{3.586723in}}%
\pgfpathlineto{\pgfqpoint{7.132302in}{3.582784in}}%
\pgfpathlineto{\pgfqpoint{7.134584in}{3.593946in}}%
\pgfpathlineto{\pgfqpoint{7.141430in}{3.611017in}}%
\pgfpathlineto{\pgfqpoint{7.148276in}{3.601956in}}%
\pgfpathlineto{\pgfqpoint{7.150558in}{3.607341in}}%
\pgfpathlineto{\pgfqpoint{7.152839in}{3.638463in}}%
\pgfpathlineto{\pgfqpoint{7.155121in}{3.621523in}}%
\pgfpathlineto{\pgfqpoint{7.157403in}{3.624543in}}%
\pgfpathlineto{\pgfqpoint{7.164249in}{3.643453in}}%
\pgfpathlineto{\pgfqpoint{7.166531in}{3.645160in}}%
\pgfpathlineto{\pgfqpoint{7.168813in}{3.644110in}}%
\pgfpathlineto{\pgfqpoint{7.171095in}{3.650939in}}%
\pgfpathlineto{\pgfqpoint{7.173377in}{3.651726in}}%
\pgfpathlineto{\pgfqpoint{7.180222in}{3.656717in}}%
\pgfpathlineto{\pgfqpoint{7.182504in}{3.650282in}}%
\pgfpathlineto{\pgfqpoint{7.184786in}{3.645686in}}%
\pgfpathlineto{\pgfqpoint{7.187068in}{3.659212in}}%
\pgfpathlineto{\pgfqpoint{7.189350in}{3.657898in}}%
\pgfpathlineto{\pgfqpoint{7.196196in}{3.661575in}}%
\pgfpathlineto{\pgfqpoint{7.198478in}{3.665909in}}%
\pgfpathlineto{\pgfqpoint{7.200760in}{3.663676in}}%
\pgfpathlineto{\pgfqpoint{7.203041in}{3.667616in}}%
\pgfpathlineto{\pgfqpoint{7.205323in}{3.685738in}}%
\pgfpathlineto{\pgfqpoint{7.212169in}{3.685344in}}%
\pgfpathlineto{\pgfqpoint{7.214451in}{3.670768in}}%
\pgfpathlineto{\pgfqpoint{7.216733in}{3.661313in}}%
\pgfpathlineto{\pgfqpoint{7.219015in}{3.673131in}}%
\pgfpathlineto{\pgfqpoint{7.221297in}{3.662363in}}%
\pgfpathlineto{\pgfqpoint{7.228142in}{3.671950in}}%
\pgfpathlineto{\pgfqpoint{7.230424in}{3.671030in}}%
\pgfpathlineto{\pgfqpoint{7.232706in}{3.675627in}}%
\pgfpathlineto{\pgfqpoint{7.234988in}{3.692435in}}%
\pgfpathlineto{\pgfqpoint{7.237270in}{3.687577in}}%
\pgfpathlineto{\pgfqpoint{7.244116in}{3.678910in}}%
\pgfpathlineto{\pgfqpoint{7.246398in}{3.683506in}}%
\pgfpathlineto{\pgfqpoint{7.248680in}{3.677596in}}%
\pgfpathlineto{\pgfqpoint{7.250961in}{3.653434in}}%
\pgfpathlineto{\pgfqpoint{7.253243in}{3.650151in}}%
\pgfpathlineto{\pgfqpoint{7.260089in}{3.637019in}}%
\pgfpathlineto{\pgfqpoint{7.262371in}{3.658949in}}%
\pgfpathlineto{\pgfqpoint{7.264653in}{3.643322in}}%
\pgfpathlineto{\pgfqpoint{7.266935in}{3.656323in}}%
\pgfpathlineto{\pgfqpoint{7.269217in}{3.639120in}}%
\pgfpathlineto{\pgfqpoint{7.276062in}{3.637150in}}%
\pgfpathlineto{\pgfqpoint{7.278344in}{3.645554in}}%
\pgfpathlineto{\pgfqpoint{7.280626in}{3.641615in}}%
\pgfpathlineto{\pgfqpoint{7.292036in}{3.646605in}}%
\pgfpathlineto{\pgfqpoint{7.296600in}{3.659737in}}%
\pgfpathlineto{\pgfqpoint{7.298882in}{3.701496in}}%
\pgfpathlineto{\pgfqpoint{7.301163in}{3.684688in}}%
\pgfpathlineto{\pgfqpoint{7.308009in}{3.683637in}}%
\pgfpathlineto{\pgfqpoint{7.310291in}{3.686789in}}%
\pgfpathlineto{\pgfqpoint{7.312573in}{3.698345in}}%
\pgfpathlineto{\pgfqpoint{7.314855in}{3.713578in}}%
\pgfpathlineto{\pgfqpoint{7.317137in}{3.718831in}}%
\pgfpathlineto{\pgfqpoint{7.326264in}{3.727366in}}%
\pgfpathlineto{\pgfqpoint{7.328546in}{3.737741in}}%
\pgfpathlineto{\pgfqpoint{7.330828in}{3.730912in}}%
\pgfpathlineto{\pgfqpoint{7.333110in}{3.761509in}}%
\pgfpathlineto{\pgfqpoint{7.339956in}{3.768338in}}%
\pgfpathlineto{\pgfqpoint{7.342238in}{3.769257in}}%
\pgfpathlineto{\pgfqpoint{7.344520in}{3.772015in}}%
\pgfpathlineto{\pgfqpoint{7.346802in}{3.775954in}}%
\pgfpathlineto{\pgfqpoint{7.349083in}{3.774904in}}%
\pgfpathlineto{\pgfqpoint{7.358211in}{3.773722in}}%
\pgfpathlineto{\pgfqpoint{7.362775in}{3.786197in}}%
\pgfpathlineto{\pgfqpoint{7.365057in}{3.779237in}}%
\pgfpathlineto{\pgfqpoint{7.365057in}{3.779237in}}%
\pgfusepath{stroke}%
\end{pgfscope}%
\begin{pgfscope}%
\pgfsetrectcap%
\pgfsetmiterjoin%
\pgfsetlinewidth{0.803000pt}%
\definecolor{currentstroke}{rgb}{1.000000,1.000000,1.000000}%
\pgfsetstrokecolor{currentstroke}%
\pgfsetdash{}{0pt}%
\pgfpathmoveto{\pgfqpoint{2.125000in}{2.907317in}}%
\pgfpathlineto{\pgfqpoint{2.125000in}{3.828049in}}%
\pgfusepath{stroke}%
\end{pgfscope}%
\begin{pgfscope}%
\pgfsetrectcap%
\pgfsetmiterjoin%
\pgfsetlinewidth{0.803000pt}%
\definecolor{currentstroke}{rgb}{1.000000,1.000000,1.000000}%
\pgfsetstrokecolor{currentstroke}%
\pgfsetdash{}{0pt}%
\pgfpathmoveto{\pgfqpoint{7.614583in}{2.907317in}}%
\pgfpathlineto{\pgfqpoint{7.614583in}{3.828049in}}%
\pgfusepath{stroke}%
\end{pgfscope}%
\begin{pgfscope}%
\pgfsetrectcap%
\pgfsetmiterjoin%
\pgfsetlinewidth{0.803000pt}%
\definecolor{currentstroke}{rgb}{1.000000,1.000000,1.000000}%
\pgfsetstrokecolor{currentstroke}%
\pgfsetdash{}{0pt}%
\pgfpathmoveto{\pgfqpoint{2.125000in}{2.907317in}}%
\pgfpathlineto{\pgfqpoint{7.614583in}{2.907317in}}%
\pgfusepath{stroke}%
\end{pgfscope}%
\begin{pgfscope}%
\pgfsetrectcap%
\pgfsetmiterjoin%
\pgfsetlinewidth{0.803000pt}%
\definecolor{currentstroke}{rgb}{1.000000,1.000000,1.000000}%
\pgfsetstrokecolor{currentstroke}%
\pgfsetdash{}{0pt}%
\pgfpathmoveto{\pgfqpoint{2.125000in}{3.828049in}}%
\pgfpathlineto{\pgfqpoint{7.614583in}{3.828049in}}%
\pgfusepath{stroke}%
\end{pgfscope}%
\begin{pgfscope}%
\definecolor{textcolor}{rgb}{0.150000,0.150000,0.150000}%
\pgfsetstrokecolor{textcolor}%
\pgfsetfillcolor{textcolor}%
\pgftext[x=4.869792in,y=3.911382in,,base]{\color{textcolor}\rmfamily\fontsize{12.000000}{14.400000}\selectfont UTX}%
\end{pgfscope}%
\begin{pgfscope}%
\pgfsetbuttcap%
\pgfsetmiterjoin%
\definecolor{currentfill}{rgb}{0.917647,0.917647,0.949020}%
\pgfsetfillcolor{currentfill}%
\pgfsetlinewidth{0.000000pt}%
\definecolor{currentstroke}{rgb}{0.000000,0.000000,0.000000}%
\pgfsetstrokecolor{currentstroke}%
\pgfsetstrokeopacity{0.000000}%
\pgfsetdash{}{0pt}%
\pgfpathmoveto{\pgfqpoint{9.810417in}{2.907317in}}%
\pgfpathlineto{\pgfqpoint{15.300000in}{2.907317in}}%
\pgfpathlineto{\pgfqpoint{15.300000in}{3.828049in}}%
\pgfpathlineto{\pgfqpoint{9.810417in}{3.828049in}}%
\pgfpathclose%
\pgfusepath{fill}%
\end{pgfscope}%
\begin{pgfscope}%
\pgfpathrectangle{\pgfqpoint{9.810417in}{2.907317in}}{\pgfqpoint{5.489583in}{0.920732in}}%
\pgfusepath{clip}%
\pgfsetroundcap%
\pgfsetroundjoin%
\pgfsetlinewidth{0.803000pt}%
\definecolor{currentstroke}{rgb}{1.000000,1.000000,1.000000}%
\pgfsetstrokecolor{currentstroke}%
\pgfsetdash{}{0pt}%
\pgfpathmoveto{\pgfqpoint{10.055379in}{2.907317in}}%
\pgfpathlineto{\pgfqpoint{10.055379in}{3.828049in}}%
\pgfusepath{stroke}%
\end{pgfscope}%
\begin{pgfscope}%
\definecolor{textcolor}{rgb}{0.150000,0.150000,0.150000}%
\pgfsetstrokecolor{textcolor}%
\pgfsetfillcolor{textcolor}%
\pgftext[x=10.055379in,y=2.810095in,,top]{\color{textcolor}\rmfamily\fontsize{10.000000}{12.000000}\selectfont 2012}%
\end{pgfscope}%
\begin{pgfscope}%
\pgfpathrectangle{\pgfqpoint{9.810417in}{2.907317in}}{\pgfqpoint{5.489583in}{0.920732in}}%
\pgfusepath{clip}%
\pgfsetroundcap%
\pgfsetroundjoin%
\pgfsetlinewidth{0.803000pt}%
\definecolor{currentstroke}{rgb}{1.000000,1.000000,1.000000}%
\pgfsetstrokecolor{currentstroke}%
\pgfsetdash{}{0pt}%
\pgfpathmoveto{\pgfqpoint{10.890557in}{2.907317in}}%
\pgfpathlineto{\pgfqpoint{10.890557in}{3.828049in}}%
\pgfusepath{stroke}%
\end{pgfscope}%
\begin{pgfscope}%
\definecolor{textcolor}{rgb}{0.150000,0.150000,0.150000}%
\pgfsetstrokecolor{textcolor}%
\pgfsetfillcolor{textcolor}%
\pgftext[x=10.890557in,y=2.810095in,,top]{\color{textcolor}\rmfamily\fontsize{10.000000}{12.000000}\selectfont 2013}%
\end{pgfscope}%
\begin{pgfscope}%
\pgfpathrectangle{\pgfqpoint{9.810417in}{2.907317in}}{\pgfqpoint{5.489583in}{0.920732in}}%
\pgfusepath{clip}%
\pgfsetroundcap%
\pgfsetroundjoin%
\pgfsetlinewidth{0.803000pt}%
\definecolor{currentstroke}{rgb}{1.000000,1.000000,1.000000}%
\pgfsetstrokecolor{currentstroke}%
\pgfsetdash{}{0pt}%
\pgfpathmoveto{\pgfqpoint{11.723453in}{2.907317in}}%
\pgfpathlineto{\pgfqpoint{11.723453in}{3.828049in}}%
\pgfusepath{stroke}%
\end{pgfscope}%
\begin{pgfscope}%
\definecolor{textcolor}{rgb}{0.150000,0.150000,0.150000}%
\pgfsetstrokecolor{textcolor}%
\pgfsetfillcolor{textcolor}%
\pgftext[x=11.723453in,y=2.810095in,,top]{\color{textcolor}\rmfamily\fontsize{10.000000}{12.000000}\selectfont 2014}%
\end{pgfscope}%
\begin{pgfscope}%
\pgfpathrectangle{\pgfqpoint{9.810417in}{2.907317in}}{\pgfqpoint{5.489583in}{0.920732in}}%
\pgfusepath{clip}%
\pgfsetroundcap%
\pgfsetroundjoin%
\pgfsetlinewidth{0.803000pt}%
\definecolor{currentstroke}{rgb}{1.000000,1.000000,1.000000}%
\pgfsetstrokecolor{currentstroke}%
\pgfsetdash{}{0pt}%
\pgfpathmoveto{\pgfqpoint{12.556349in}{2.907317in}}%
\pgfpathlineto{\pgfqpoint{12.556349in}{3.828049in}}%
\pgfusepath{stroke}%
\end{pgfscope}%
\begin{pgfscope}%
\definecolor{textcolor}{rgb}{0.150000,0.150000,0.150000}%
\pgfsetstrokecolor{textcolor}%
\pgfsetfillcolor{textcolor}%
\pgftext[x=12.556349in,y=2.810095in,,top]{\color{textcolor}\rmfamily\fontsize{10.000000}{12.000000}\selectfont 2015}%
\end{pgfscope}%
\begin{pgfscope}%
\pgfpathrectangle{\pgfqpoint{9.810417in}{2.907317in}}{\pgfqpoint{5.489583in}{0.920732in}}%
\pgfusepath{clip}%
\pgfsetroundcap%
\pgfsetroundjoin%
\pgfsetlinewidth{0.803000pt}%
\definecolor{currentstroke}{rgb}{1.000000,1.000000,1.000000}%
\pgfsetstrokecolor{currentstroke}%
\pgfsetdash{}{0pt}%
\pgfpathmoveto{\pgfqpoint{13.389245in}{2.907317in}}%
\pgfpathlineto{\pgfqpoint{13.389245in}{3.828049in}}%
\pgfusepath{stroke}%
\end{pgfscope}%
\begin{pgfscope}%
\definecolor{textcolor}{rgb}{0.150000,0.150000,0.150000}%
\pgfsetstrokecolor{textcolor}%
\pgfsetfillcolor{textcolor}%
\pgftext[x=13.389245in,y=2.810095in,,top]{\color{textcolor}\rmfamily\fontsize{10.000000}{12.000000}\selectfont 2016}%
\end{pgfscope}%
\begin{pgfscope}%
\pgfpathrectangle{\pgfqpoint{9.810417in}{2.907317in}}{\pgfqpoint{5.489583in}{0.920732in}}%
\pgfusepath{clip}%
\pgfsetroundcap%
\pgfsetroundjoin%
\pgfsetlinewidth{0.803000pt}%
\definecolor{currentstroke}{rgb}{1.000000,1.000000,1.000000}%
\pgfsetstrokecolor{currentstroke}%
\pgfsetdash{}{0pt}%
\pgfpathmoveto{\pgfqpoint{14.224423in}{2.907317in}}%
\pgfpathlineto{\pgfqpoint{14.224423in}{3.828049in}}%
\pgfusepath{stroke}%
\end{pgfscope}%
\begin{pgfscope}%
\definecolor{textcolor}{rgb}{0.150000,0.150000,0.150000}%
\pgfsetstrokecolor{textcolor}%
\pgfsetfillcolor{textcolor}%
\pgftext[x=14.224423in,y=2.810095in,,top]{\color{textcolor}\rmfamily\fontsize{10.000000}{12.000000}\selectfont 2017}%
\end{pgfscope}%
\begin{pgfscope}%
\pgfpathrectangle{\pgfqpoint{9.810417in}{2.907317in}}{\pgfqpoint{5.489583in}{0.920732in}}%
\pgfusepath{clip}%
\pgfsetroundcap%
\pgfsetroundjoin%
\pgfsetlinewidth{0.803000pt}%
\definecolor{currentstroke}{rgb}{1.000000,1.000000,1.000000}%
\pgfsetstrokecolor{currentstroke}%
\pgfsetdash{}{0pt}%
\pgfpathmoveto{\pgfqpoint{15.057319in}{2.907317in}}%
\pgfpathlineto{\pgfqpoint{15.057319in}{3.828049in}}%
\pgfusepath{stroke}%
\end{pgfscope}%
\begin{pgfscope}%
\definecolor{textcolor}{rgb}{0.150000,0.150000,0.150000}%
\pgfsetstrokecolor{textcolor}%
\pgfsetfillcolor{textcolor}%
\pgftext[x=15.057319in,y=2.810095in,,top]{\color{textcolor}\rmfamily\fontsize{10.000000}{12.000000}\selectfont 2018}%
\end{pgfscope}%
\begin{pgfscope}%
\pgfpathrectangle{\pgfqpoint{9.810417in}{2.907317in}}{\pgfqpoint{5.489583in}{0.920732in}}%
\pgfusepath{clip}%
\pgfsetroundcap%
\pgfsetroundjoin%
\pgfsetlinewidth{0.803000pt}%
\definecolor{currentstroke}{rgb}{1.000000,1.000000,1.000000}%
\pgfsetstrokecolor{currentstroke}%
\pgfsetdash{}{0pt}%
\pgfpathmoveto{\pgfqpoint{9.810417in}{3.076511in}}%
\pgfpathlineto{\pgfqpoint{15.300000in}{3.076511in}}%
\pgfusepath{stroke}%
\end{pgfscope}%
\begin{pgfscope}%
\definecolor{textcolor}{rgb}{0.150000,0.150000,0.150000}%
\pgfsetstrokecolor{textcolor}%
\pgfsetfillcolor{textcolor}%
\pgftext[x=9.536464in,y=3.023749in,left,base]{\color{textcolor}\rmfamily\fontsize{10.000000}{12.000000}\selectfont 30}%
\end{pgfscope}%
\begin{pgfscope}%
\pgfpathrectangle{\pgfqpoint{9.810417in}{2.907317in}}{\pgfqpoint{5.489583in}{0.920732in}}%
\pgfusepath{clip}%
\pgfsetroundcap%
\pgfsetroundjoin%
\pgfsetlinewidth{0.803000pt}%
\definecolor{currentstroke}{rgb}{1.000000,1.000000,1.000000}%
\pgfsetstrokecolor{currentstroke}%
\pgfsetdash{}{0pt}%
\pgfpathmoveto{\pgfqpoint{9.810417in}{3.441388in}}%
\pgfpathlineto{\pgfqpoint{15.300000in}{3.441388in}}%
\pgfusepath{stroke}%
\end{pgfscope}%
\begin{pgfscope}%
\definecolor{textcolor}{rgb}{0.150000,0.150000,0.150000}%
\pgfsetstrokecolor{textcolor}%
\pgfsetfillcolor{textcolor}%
\pgftext[x=9.536464in,y=3.388627in,left,base]{\color{textcolor}\rmfamily\fontsize{10.000000}{12.000000}\selectfont 40}%
\end{pgfscope}%
\begin{pgfscope}%
\pgfpathrectangle{\pgfqpoint{9.810417in}{2.907317in}}{\pgfqpoint{5.489583in}{0.920732in}}%
\pgfusepath{clip}%
\pgfsetroundcap%
\pgfsetroundjoin%
\pgfsetlinewidth{0.803000pt}%
\definecolor{currentstroke}{rgb}{1.000000,1.000000,1.000000}%
\pgfsetstrokecolor{currentstroke}%
\pgfsetdash{}{0pt}%
\pgfpathmoveto{\pgfqpoint{9.810417in}{3.806266in}}%
\pgfpathlineto{\pgfqpoint{15.300000in}{3.806266in}}%
\pgfusepath{stroke}%
\end{pgfscope}%
\begin{pgfscope}%
\definecolor{textcolor}{rgb}{0.150000,0.150000,0.150000}%
\pgfsetstrokecolor{textcolor}%
\pgfsetfillcolor{textcolor}%
\pgftext[x=9.536464in,y=3.753504in,left,base]{\color{textcolor}\rmfamily\fontsize{10.000000}{12.000000}\selectfont 50}%
\end{pgfscope}%
\begin{pgfscope}%
\pgfpathrectangle{\pgfqpoint{9.810417in}{2.907317in}}{\pgfqpoint{5.489583in}{0.920732in}}%
\pgfusepath{clip}%
\pgfsetroundcap%
\pgfsetroundjoin%
\pgfsetlinewidth{1.505625pt}%
\definecolor{currentstroke}{rgb}{0.121569,0.466667,0.705882}%
\pgfsetstrokecolor{currentstroke}%
\pgfsetdash{}{0pt}%
\pgfpathmoveto{\pgfqpoint{10.059943in}{3.001346in}}%
\pgfpathlineto{\pgfqpoint{10.062225in}{2.988210in}}%
\pgfpathlineto{\pgfqpoint{10.064507in}{2.981278in}}%
\pgfpathlineto{\pgfqpoint{10.066789in}{2.978359in}}%
\pgfpathlineto{\pgfqpoint{10.073635in}{2.979453in}}%
\pgfpathlineto{\pgfqpoint{10.075917in}{2.984562in}}%
\pgfpathlineto{\pgfqpoint{10.078198in}{2.993319in}}%
\pgfpathlineto{\pgfqpoint{10.082762in}{2.993684in}}%
\pgfpathlineto{\pgfqpoint{10.091890in}{2.996238in}}%
\pgfpathlineto{\pgfqpoint{10.098736in}{2.995143in}}%
\pgfpathlineto{\pgfqpoint{10.105581in}{2.980183in}}%
\pgfpathlineto{\pgfqpoint{10.107863in}{2.964493in}}%
\pgfpathlineto{\pgfqpoint{10.110145in}{2.961574in}}%
\pgfpathlineto{\pgfqpoint{10.112427in}{2.952452in}}%
\pgfpathlineto{\pgfqpoint{10.114709in}{2.949169in}}%
\pgfpathlineto{\pgfqpoint{10.121555in}{2.959750in}}%
\pgfpathlineto{\pgfqpoint{10.123837in}{2.960845in}}%
\pgfpathlineto{\pgfqpoint{10.126118in}{2.964493in}}%
\pgfpathlineto{\pgfqpoint{10.128400in}{2.958290in}}%
\pgfpathlineto{\pgfqpoint{10.130682in}{2.965588in}}%
\pgfpathlineto{\pgfqpoint{10.137528in}{2.973250in}}%
\pgfpathlineto{\pgfqpoint{10.139810in}{2.967777in}}%
\pgfpathlineto{\pgfqpoint{10.144374in}{2.967777in}}%
\pgfpathlineto{\pgfqpoint{10.146656in}{2.961574in}}%
\pgfpathlineto{\pgfqpoint{10.153501in}{2.973250in}}%
\pgfpathlineto{\pgfqpoint{10.155783in}{2.970696in}}%
\pgfpathlineto{\pgfqpoint{10.158065in}{2.965223in}}%
\pgfpathlineto{\pgfqpoint{10.160347in}{2.971061in}}%
\pgfpathlineto{\pgfqpoint{10.162629in}{2.981643in}}%
\pgfpathlineto{\pgfqpoint{10.171757in}{2.982372in}}%
\pgfpathlineto{\pgfqpoint{10.174039in}{2.975075in}}%
\pgfpathlineto{\pgfqpoint{10.176320in}{2.973250in}}%
\pgfpathlineto{\pgfqpoint{10.187730in}{2.973615in}}%
\pgfpathlineto{\pgfqpoint{10.190012in}{2.972521in}}%
\pgfpathlineto{\pgfqpoint{10.192294in}{2.980913in}}%
\pgfpathlineto{\pgfqpoint{10.194576in}{2.987116in}}%
\pgfpathlineto{\pgfqpoint{10.201421in}{2.995873in}}%
\pgfpathlineto{\pgfqpoint{10.203703in}{2.987846in}}%
\pgfpathlineto{\pgfqpoint{10.205985in}{2.992589in}}%
\pgfpathlineto{\pgfqpoint{10.208267in}{3.001346in}}%
\pgfpathlineto{\pgfqpoint{10.210549in}{2.998427in}}%
\pgfpathlineto{\pgfqpoint{10.217395in}{3.004265in}}%
\pgfpathlineto{\pgfqpoint{10.219677in}{3.008279in}}%
\pgfpathlineto{\pgfqpoint{10.221959in}{3.007914in}}%
\pgfpathlineto{\pgfqpoint{10.224240in}{3.009738in}}%
\pgfpathlineto{\pgfqpoint{10.233368in}{3.012657in}}%
\pgfpathlineto{\pgfqpoint{10.235650in}{3.012292in}}%
\pgfpathlineto{\pgfqpoint{10.237932in}{3.015941in}}%
\pgfpathlineto{\pgfqpoint{10.240214in}{3.013022in}}%
\pgfpathlineto{\pgfqpoint{10.242496in}{3.006819in}}%
\pgfpathlineto{\pgfqpoint{10.249341in}{3.004265in}}%
\pgfpathlineto{\pgfqpoint{10.251623in}{2.987116in}}%
\pgfpathlineto{\pgfqpoint{10.253905in}{2.976534in}}%
\pgfpathlineto{\pgfqpoint{10.256187in}{2.971791in}}%
\pgfpathlineto{\pgfqpoint{10.258469in}{2.975805in}}%
\pgfpathlineto{\pgfqpoint{10.265315in}{2.983467in}}%
\pgfpathlineto{\pgfqpoint{10.269879in}{2.979818in}}%
\pgfpathlineto{\pgfqpoint{10.272161in}{2.973980in}}%
\pgfpathlineto{\pgfqpoint{10.281288in}{2.968507in}}%
\pgfpathlineto{\pgfqpoint{10.283570in}{2.951358in}}%
\pgfpathlineto{\pgfqpoint{10.285852in}{2.966318in}}%
\pgfpathlineto{\pgfqpoint{10.288134in}{2.971061in}}%
\pgfpathlineto{\pgfqpoint{10.290416in}{2.963399in}}%
\pgfpathlineto{\pgfqpoint{10.297261in}{2.967777in}}%
\pgfpathlineto{\pgfqpoint{10.299543in}{2.975805in}}%
\pgfpathlineto{\pgfqpoint{10.301825in}{2.973980in}}%
\pgfpathlineto{\pgfqpoint{10.306389in}{3.002076in}}%
\pgfpathlineto{\pgfqpoint{10.313235in}{2.997697in}}%
\pgfpathlineto{\pgfqpoint{10.315517in}{3.022509in}}%
\pgfpathlineto{\pgfqpoint{10.317799in}{3.021779in}}%
\pgfpathlineto{\pgfqpoint{10.320081in}{3.039293in}}%
\pgfpathlineto{\pgfqpoint{10.322362in}{3.041483in}}%
\pgfpathlineto{\pgfqpoint{10.329208in}{3.045496in}}%
\pgfpathlineto{\pgfqpoint{10.331490in}{3.050240in}}%
\pgfpathlineto{\pgfqpoint{10.336054in}{3.052429in}}%
\pgfpathlineto{\pgfqpoint{10.338336in}{3.042212in}}%
\pgfpathlineto{\pgfqpoint{10.345182in}{3.049510in}}%
\pgfpathlineto{\pgfqpoint{10.347463in}{3.049875in}}%
\pgfpathlineto{\pgfqpoint{10.349745in}{3.042212in}}%
\pgfpathlineto{\pgfqpoint{10.352027in}{3.049875in}}%
\pgfpathlineto{\pgfqpoint{10.354309in}{3.065929in}}%
\pgfpathlineto{\pgfqpoint{10.361155in}{3.058997in}}%
\pgfpathlineto{\pgfqpoint{10.363437in}{3.063010in}}%
\pgfpathlineto{\pgfqpoint{10.365719in}{3.058632in}}%
\pgfpathlineto{\pgfqpoint{10.368001in}{3.071767in}}%
\pgfpathlineto{\pgfqpoint{10.370283in}{3.075781in}}%
\pgfpathlineto{\pgfqpoint{10.377128in}{3.070673in}}%
\pgfpathlineto{\pgfqpoint{10.379410in}{3.072132in}}%
\pgfpathlineto{\pgfqpoint{10.381692in}{3.069213in}}%
\pgfpathlineto{\pgfqpoint{10.383974in}{3.072132in}}%
\pgfpathlineto{\pgfqpoint{10.386256in}{3.073592in}}%
\pgfpathlineto{\pgfqpoint{10.395383in}{3.081619in}}%
\pgfpathlineto{\pgfqpoint{10.397665in}{3.072862in}}%
\pgfpathlineto{\pgfqpoint{10.399947in}{3.078700in}}%
\pgfpathlineto{\pgfqpoint{10.402229in}{3.062645in}}%
\pgfpathlineto{\pgfqpoint{10.409075in}{3.070673in}}%
\pgfpathlineto{\pgfqpoint{10.411357in}{3.067024in}}%
\pgfpathlineto{\pgfqpoint{10.413639in}{3.081984in}}%
\pgfpathlineto{\pgfqpoint{10.415921in}{3.078700in}}%
\pgfpathlineto{\pgfqpoint{10.418203in}{3.099863in}}%
\pgfpathlineto{\pgfqpoint{10.425048in}{3.102782in}}%
\pgfpathlineto{\pgfqpoint{10.427330in}{3.112998in}}%
\pgfpathlineto{\pgfqpoint{10.429612in}{3.114093in}}%
\pgfpathlineto{\pgfqpoint{10.431894in}{3.134161in}}%
\pgfpathlineto{\pgfqpoint{10.434176in}{3.129053in}}%
\pgfpathlineto{\pgfqpoint{10.441022in}{3.135986in}}%
\pgfpathlineto{\pgfqpoint{10.443304in}{3.133796in}}%
\pgfpathlineto{\pgfqpoint{10.445585in}{3.122485in}}%
\pgfpathlineto{\pgfqpoint{10.447867in}{3.123215in}}%
\pgfpathlineto{\pgfqpoint{10.450149in}{3.139635in}}%
\pgfpathlineto{\pgfqpoint{10.456995in}{3.131607in}}%
\pgfpathlineto{\pgfqpoint{10.459277in}{3.136716in}}%
\pgfpathlineto{\pgfqpoint{10.461559in}{3.135256in}}%
\pgfpathlineto{\pgfqpoint{10.463841in}{3.139999in}}%
\pgfpathlineto{\pgfqpoint{10.466123in}{3.152405in}}%
\pgfpathlineto{\pgfqpoint{10.472968in}{3.165541in}}%
\pgfpathlineto{\pgfqpoint{10.475250in}{3.165906in}}%
\pgfpathlineto{\pgfqpoint{10.479814in}{3.164446in}}%
\pgfpathlineto{\pgfqpoint{10.482096in}{3.165176in}}%
\pgfpathlineto{\pgfqpoint{10.488942in}{3.173568in}}%
\pgfpathlineto{\pgfqpoint{10.491224in}{3.172474in}}%
\pgfpathlineto{\pgfqpoint{10.493505in}{3.177947in}}%
\pgfpathlineto{\pgfqpoint{10.495787in}{3.171744in}}%
\pgfpathlineto{\pgfqpoint{10.498069in}{3.186339in}}%
\pgfpathlineto{\pgfqpoint{10.504915in}{3.187798in}}%
\pgfpathlineto{\pgfqpoint{10.507197in}{3.198015in}}%
\pgfpathlineto{\pgfqpoint{10.509479in}{3.204218in}}%
\pgfpathlineto{\pgfqpoint{10.511761in}{3.168460in}}%
\pgfpathlineto{\pgfqpoint{10.514043in}{3.167000in}}%
\pgfpathlineto{\pgfqpoint{10.520888in}{3.163352in}}%
\pgfpathlineto{\pgfqpoint{10.523170in}{3.147662in}}%
\pgfpathlineto{\pgfqpoint{10.525452in}{3.146567in}}%
\pgfpathlineto{\pgfqpoint{10.527734in}{3.166271in}}%
\pgfpathlineto{\pgfqpoint{10.530016in}{3.177947in}}%
\pgfpathlineto{\pgfqpoint{10.536862in}{3.179406in}}%
\pgfpathlineto{\pgfqpoint{10.539144in}{3.184150in}}%
\pgfpathlineto{\pgfqpoint{10.541426in}{3.186339in}}%
\pgfpathlineto{\pgfqpoint{10.543707in}{3.170284in}}%
\pgfpathlineto{\pgfqpoint{10.545989in}{3.166271in}}%
\pgfpathlineto{\pgfqpoint{10.552835in}{3.172474in}}%
\pgfpathlineto{\pgfqpoint{10.557399in}{3.158608in}}%
\pgfpathlineto{\pgfqpoint{10.559681in}{3.162987in}}%
\pgfpathlineto{\pgfqpoint{10.561963in}{3.169919in}}%
\pgfpathlineto{\pgfqpoint{10.568808in}{3.161162in}}%
\pgfpathlineto{\pgfqpoint{10.571090in}{3.162622in}}%
\pgfpathlineto{\pgfqpoint{10.573372in}{3.158973in}}%
\pgfpathlineto{\pgfqpoint{10.577936in}{3.155689in}}%
\pgfpathlineto{\pgfqpoint{10.584782in}{3.145837in}}%
\pgfpathlineto{\pgfqpoint{10.587064in}{3.124310in}}%
\pgfpathlineto{\pgfqpoint{10.589346in}{3.117012in}}%
\pgfpathlineto{\pgfqpoint{10.591627in}{3.107160in}}%
\pgfpathlineto{\pgfqpoint{10.593909in}{3.131972in}}%
\pgfpathlineto{\pgfqpoint{10.600755in}{3.121026in}}%
\pgfpathlineto{\pgfqpoint{10.603037in}{3.118472in}}%
\pgfpathlineto{\pgfqpoint{10.605319in}{3.129783in}}%
\pgfpathlineto{\pgfqpoint{10.607601in}{3.121026in}}%
\pgfpathlineto{\pgfqpoint{10.621292in}{3.148756in}}%
\pgfpathlineto{\pgfqpoint{10.623574in}{3.157878in}}%
\pgfpathlineto{\pgfqpoint{10.625856in}{3.146567in}}%
\pgfpathlineto{\pgfqpoint{10.632702in}{3.155689in}}%
\pgfpathlineto{\pgfqpoint{10.634984in}{3.160433in}}%
\pgfpathlineto{\pgfqpoint{10.639548in}{3.196191in}}%
\pgfpathlineto{\pgfqpoint{10.641829in}{3.168095in}}%
\pgfpathlineto{\pgfqpoint{10.648675in}{3.169554in}}%
\pgfpathlineto{\pgfqpoint{10.655521in}{3.193636in}}%
\pgfpathlineto{\pgfqpoint{10.657803in}{3.197650in}}%
\pgfpathlineto{\pgfqpoint{10.664648in}{3.198745in}}%
\pgfpathlineto{\pgfqpoint{10.666930in}{3.196920in}}%
\pgfpathlineto{\pgfqpoint{10.669212in}{3.196191in}}%
\pgfpathlineto{\pgfqpoint{10.671494in}{3.200934in}}%
\pgfpathlineto{\pgfqpoint{10.673776in}{3.195826in}}%
\pgfpathlineto{\pgfqpoint{10.682904in}{3.203488in}}%
\pgfpathlineto{\pgfqpoint{10.685186in}{3.214799in}}%
\pgfpathlineto{\pgfqpoint{10.687468in}{3.240706in}}%
\pgfpathlineto{\pgfqpoint{10.689749in}{3.249098in}}%
\pgfpathlineto{\pgfqpoint{10.696595in}{3.235962in}}%
\pgfpathlineto{\pgfqpoint{10.698877in}{3.223556in}}%
\pgfpathlineto{\pgfqpoint{10.701159in}{3.214799in}}%
\pgfpathlineto{\pgfqpoint{10.705723in}{3.183420in}}%
\pgfpathlineto{\pgfqpoint{10.712569in}{3.180136in}}%
\pgfpathlineto{\pgfqpoint{10.714850in}{3.168825in}}%
\pgfpathlineto{\pgfqpoint{10.717132in}{3.186339in}}%
\pgfpathlineto{\pgfqpoint{10.719414in}{3.214799in}}%
\pgfpathlineto{\pgfqpoint{10.721696in}{3.198015in}}%
\pgfpathlineto{\pgfqpoint{10.728542in}{3.187433in}}%
\pgfpathlineto{\pgfqpoint{10.730824in}{3.168825in}}%
\pgfpathlineto{\pgfqpoint{10.733106in}{3.173203in}}%
\pgfpathlineto{\pgfqpoint{10.735388in}{3.173568in}}%
\pgfpathlineto{\pgfqpoint{10.737670in}{3.186339in}}%
\pgfpathlineto{\pgfqpoint{10.749079in}{3.184150in}}%
\pgfpathlineto{\pgfqpoint{10.751361in}{3.197650in}}%
\pgfpathlineto{\pgfqpoint{10.753643in}{3.180866in}}%
\pgfpathlineto{\pgfqpoint{10.760489in}{3.172109in}}%
\pgfpathlineto{\pgfqpoint{10.762770in}{3.175028in}}%
\pgfpathlineto{\pgfqpoint{10.765052in}{3.145108in}}%
\pgfpathlineto{\pgfqpoint{10.767334in}{3.129418in}}%
\pgfpathlineto{\pgfqpoint{10.769616in}{3.130148in}}%
\pgfpathlineto{\pgfqpoint{10.778744in}{3.127594in}}%
\pgfpathlineto{\pgfqpoint{10.781026in}{3.119566in}}%
\pgfpathlineto{\pgfqpoint{10.783308in}{3.104971in}}%
\pgfpathlineto{\pgfqpoint{10.785590in}{3.096944in}}%
\pgfpathlineto{\pgfqpoint{10.792435in}{3.134891in}}%
\pgfpathlineto{\pgfqpoint{10.794717in}{3.134891in}}%
\pgfpathlineto{\pgfqpoint{10.796999in}{3.144378in}}%
\pgfpathlineto{\pgfqpoint{10.801563in}{3.160433in}}%
\pgfpathlineto{\pgfqpoint{10.808409in}{3.148027in}}%
\pgfpathlineto{\pgfqpoint{10.810691in}{3.139270in}}%
\pgfpathlineto{\pgfqpoint{10.815254in}{3.167000in}}%
\pgfpathlineto{\pgfqpoint{10.817536in}{3.169919in}}%
\pgfpathlineto{\pgfqpoint{10.824382in}{3.169554in}}%
\pgfpathlineto{\pgfqpoint{10.826664in}{3.157878in}}%
\pgfpathlineto{\pgfqpoint{10.831228in}{3.179041in}}%
\pgfpathlineto{\pgfqpoint{10.833510in}{3.177947in}}%
\pgfpathlineto{\pgfqpoint{10.840355in}{3.167730in}}%
\pgfpathlineto{\pgfqpoint{10.844919in}{3.188163in}}%
\pgfpathlineto{\pgfqpoint{10.849483in}{3.172474in}}%
\pgfpathlineto{\pgfqpoint{10.856329in}{3.169919in}}%
\pgfpathlineto{\pgfqpoint{10.858611in}{3.163352in}}%
\pgfpathlineto{\pgfqpoint{10.860892in}{3.150216in}}%
\pgfpathlineto{\pgfqpoint{10.863174in}{3.161892in}}%
\pgfpathlineto{\pgfqpoint{10.865456in}{3.155324in}}%
\pgfpathlineto{\pgfqpoint{10.872302in}{3.154595in}}%
\pgfpathlineto{\pgfqpoint{10.876866in}{3.152040in}}%
\pgfpathlineto{\pgfqpoint{10.879148in}{3.152770in}}%
\pgfpathlineto{\pgfqpoint{10.881430in}{3.137080in}}%
\pgfpathlineto{\pgfqpoint{10.888275in}{3.147297in}}%
\pgfpathlineto{\pgfqpoint{10.892839in}{3.173933in}}%
\pgfpathlineto{\pgfqpoint{10.895121in}{3.168460in}}%
\pgfpathlineto{\pgfqpoint{10.897403in}{3.175028in}}%
\pgfpathlineto{\pgfqpoint{10.904249in}{3.185244in}}%
\pgfpathlineto{\pgfqpoint{10.906531in}{3.156054in}}%
\pgfpathlineto{\pgfqpoint{10.908813in}{3.153500in}}%
\pgfpathlineto{\pgfqpoint{10.911094in}{3.169554in}}%
\pgfpathlineto{\pgfqpoint{10.913376in}{3.161527in}}%
\pgfpathlineto{\pgfqpoint{10.920222in}{3.142189in}}%
\pgfpathlineto{\pgfqpoint{10.924786in}{3.112634in}}%
\pgfpathlineto{\pgfqpoint{10.927068in}{3.129783in}}%
\pgfpathlineto{\pgfqpoint{10.929350in}{3.140729in}}%
\pgfpathlineto{\pgfqpoint{10.938477in}{3.151675in}}%
\pgfpathlineto{\pgfqpoint{10.940759in}{3.147662in}}%
\pgfpathlineto{\pgfqpoint{10.943041in}{3.142189in}}%
\pgfpathlineto{\pgfqpoint{10.945323in}{3.144378in}}%
\pgfpathlineto{\pgfqpoint{10.952169in}{3.147297in}}%
\pgfpathlineto{\pgfqpoint{10.954451in}{3.167000in}}%
\pgfpathlineto{\pgfqpoint{10.956733in}{3.169919in}}%
\pgfpathlineto{\pgfqpoint{10.959014in}{3.169919in}}%
\pgfpathlineto{\pgfqpoint{10.961296in}{3.195826in}}%
\pgfpathlineto{\pgfqpoint{10.968142in}{3.194731in}}%
\pgfpathlineto{\pgfqpoint{10.970424in}{3.195826in}}%
\pgfpathlineto{\pgfqpoint{10.972706in}{3.200934in}}%
\pgfpathlineto{\pgfqpoint{10.974988in}{3.193272in}}%
\pgfpathlineto{\pgfqpoint{10.977270in}{3.190352in}}%
\pgfpathlineto{\pgfqpoint{10.984115in}{3.189258in}}%
\pgfpathlineto{\pgfqpoint{10.986397in}{3.192542in}}%
\pgfpathlineto{\pgfqpoint{10.988679in}{3.194731in}}%
\pgfpathlineto{\pgfqpoint{10.990961in}{3.189623in}}%
\pgfpathlineto{\pgfqpoint{10.993243in}{3.191447in}}%
\pgfpathlineto{\pgfqpoint{11.002371in}{3.194366in}}%
\pgfpathlineto{\pgfqpoint{11.004653in}{3.205677in}}%
\pgfpathlineto{\pgfqpoint{11.006935in}{3.211151in}}%
\pgfpathlineto{\pgfqpoint{11.009216in}{3.218813in}}%
\pgfpathlineto{\pgfqpoint{11.016062in}{3.227570in}}%
\pgfpathlineto{\pgfqpoint{11.018344in}{3.238516in}}%
\pgfpathlineto{\pgfqpoint{11.022908in}{3.249463in}}%
\pgfpathlineto{\pgfqpoint{11.025190in}{3.254936in}}%
\pgfpathlineto{\pgfqpoint{11.032036in}{3.265517in}}%
\pgfpathlineto{\pgfqpoint{11.034317in}{3.281207in}}%
\pgfpathlineto{\pgfqpoint{11.036599in}{3.269896in}}%
\pgfpathlineto{\pgfqpoint{11.038881in}{3.275369in}}%
\pgfpathlineto{\pgfqpoint{11.041163in}{3.288505in}}%
\pgfpathlineto{\pgfqpoint{11.048009in}{3.284491in}}%
\pgfpathlineto{\pgfqpoint{11.050291in}{3.299451in}}%
\pgfpathlineto{\pgfqpoint{11.052573in}{3.288140in}}%
\pgfpathlineto{\pgfqpoint{11.054855in}{3.302735in}}%
\pgfpathlineto{\pgfqpoint{11.057136in}{3.290329in}}%
\pgfpathlineto{\pgfqpoint{11.066264in}{3.315505in}}%
\pgfpathlineto{\pgfqpoint{11.068546in}{3.306019in}}%
\pgfpathlineto{\pgfqpoint{11.073110in}{3.317330in}}%
\pgfpathlineto{\pgfqpoint{11.079956in}{3.321343in}}%
\pgfpathlineto{\pgfqpoint{11.082237in}{3.330101in}}%
\pgfpathlineto{\pgfqpoint{11.084519in}{3.315141in}}%
\pgfpathlineto{\pgfqpoint{11.086801in}{3.320979in}}%
\pgfpathlineto{\pgfqpoint{11.095929in}{3.322803in}}%
\pgfpathlineto{\pgfqpoint{11.098211in}{3.330465in}}%
\pgfpathlineto{\pgfqpoint{11.100493in}{3.316600in}}%
\pgfpathlineto{\pgfqpoint{11.105057in}{3.332290in}}%
\pgfpathlineto{\pgfqpoint{11.111902in}{3.342871in}}%
\pgfpathlineto{\pgfqpoint{11.114184in}{3.340682in}}%
\pgfpathlineto{\pgfqpoint{11.116466in}{3.354547in}}%
\pgfpathlineto{\pgfqpoint{11.118748in}{3.372426in}}%
\pgfpathlineto{\pgfqpoint{11.121030in}{3.382278in}}%
\pgfpathlineto{\pgfqpoint{11.127876in}{3.376075in}}%
\pgfpathlineto{\pgfqpoint{11.130157in}{3.370967in}}%
\pgfpathlineto{\pgfqpoint{11.132439in}{3.345790in}}%
\pgfpathlineto{\pgfqpoint{11.137003in}{3.420225in}}%
\pgfpathlineto{\pgfqpoint{11.143849in}{3.418766in}}%
\pgfpathlineto{\pgfqpoint{11.146131in}{3.422415in}}%
\pgfpathlineto{\pgfqpoint{11.148413in}{3.407819in}}%
\pgfpathlineto{\pgfqpoint{11.150695in}{3.447226in}}%
\pgfpathlineto{\pgfqpoint{11.152977in}{3.458537in}}%
\pgfpathlineto{\pgfqpoint{11.159822in}{3.453794in}}%
\pgfpathlineto{\pgfqpoint{11.162104in}{3.466200in}}%
\pgfpathlineto{\pgfqpoint{11.164386in}{3.424604in}}%
\pgfpathlineto{\pgfqpoint{11.168950in}{3.432266in}}%
\pgfpathlineto{\pgfqpoint{11.175796in}{3.414752in}}%
\pgfpathlineto{\pgfqpoint{11.178078in}{3.438834in}}%
\pgfpathlineto{\pgfqpoint{11.180359in}{3.443942in}}%
\pgfpathlineto{\pgfqpoint{11.182641in}{3.432996in}}%
\pgfpathlineto{\pgfqpoint{11.184923in}{3.438104in}}%
\pgfpathlineto{\pgfqpoint{11.191769in}{3.428617in}}%
\pgfpathlineto{\pgfqpoint{11.194051in}{3.445767in}}%
\pgfpathlineto{\pgfqpoint{11.196333in}{3.457443in}}%
\pgfpathlineto{\pgfqpoint{11.198615in}{3.446496in}}%
\pgfpathlineto{\pgfqpoint{11.200897in}{3.450510in}}%
\pgfpathlineto{\pgfqpoint{11.207742in}{3.433726in}}%
\pgfpathlineto{\pgfqpoint{11.212306in}{3.399062in}}%
\pgfpathlineto{\pgfqpoint{11.214588in}{3.410374in}}%
\pgfpathlineto{\pgfqpoint{11.216870in}{3.396873in}}%
\pgfpathlineto{\pgfqpoint{11.225998in}{3.381183in}}%
\pgfpathlineto{\pgfqpoint{11.228279in}{3.346520in}}%
\pgfpathlineto{\pgfqpoint{11.232843in}{3.316600in}}%
\pgfpathlineto{\pgfqpoint{11.239689in}{3.321708in}}%
\pgfpathlineto{\pgfqpoint{11.241971in}{3.326452in}}%
\pgfpathlineto{\pgfqpoint{11.244253in}{3.311492in}}%
\pgfpathlineto{\pgfqpoint{11.246535in}{3.357466in}}%
\pgfpathlineto{\pgfqpoint{11.248817in}{3.365129in}}%
\pgfpathlineto{\pgfqpoint{11.255662in}{3.373156in}}%
\pgfpathlineto{\pgfqpoint{11.260226in}{3.355642in}}%
\pgfpathlineto{\pgfqpoint{11.262508in}{3.376075in}}%
\pgfpathlineto{\pgfqpoint{11.264790in}{3.387751in}}%
\pgfpathlineto{\pgfqpoint{11.271636in}{3.377899in}}%
\pgfpathlineto{\pgfqpoint{11.273918in}{3.401252in}}%
\pgfpathlineto{\pgfqpoint{11.276200in}{3.359656in}}%
\pgfpathlineto{\pgfqpoint{11.278481in}{3.329736in}}%
\pgfpathlineto{\pgfqpoint{11.280763in}{3.345061in}}%
\pgfpathlineto{\pgfqpoint{11.287609in}{3.334114in}}%
\pgfpathlineto{\pgfqpoint{11.289891in}{3.370602in}}%
\pgfpathlineto{\pgfqpoint{11.292173in}{3.376440in}}%
\pgfpathlineto{\pgfqpoint{11.294455in}{3.385927in}}%
\pgfpathlineto{\pgfqpoint{11.296737in}{3.367683in}}%
\pgfpathlineto{\pgfqpoint{11.303582in}{3.368413in}}%
\pgfpathlineto{\pgfqpoint{11.305864in}{3.375710in}}%
\pgfpathlineto{\pgfqpoint{11.308146in}{3.386292in}}%
\pgfpathlineto{\pgfqpoint{11.312710in}{3.394319in}}%
\pgfpathlineto{\pgfqpoint{11.319556in}{3.404900in}}%
\pgfpathlineto{\pgfqpoint{11.321838in}{3.399062in}}%
\pgfpathlineto{\pgfqpoint{11.324120in}{3.387386in}}%
\pgfpathlineto{\pgfqpoint{11.326401in}{3.405995in}}%
\pgfpathlineto{\pgfqpoint{11.328683in}{3.383738in}}%
\pgfpathlineto{\pgfqpoint{11.335529in}{3.371332in}}%
\pgfpathlineto{\pgfqpoint{11.337811in}{3.380089in}}%
\pgfpathlineto{\pgfqpoint{11.340093in}{3.392859in}}%
\pgfpathlineto{\pgfqpoint{11.342375in}{3.371697in}}%
\pgfpathlineto{\pgfqpoint{11.344657in}{3.370967in}}%
\pgfpathlineto{\pgfqpoint{11.353784in}{3.382278in}}%
\pgfpathlineto{\pgfqpoint{11.356066in}{3.383008in}}%
\pgfpathlineto{\pgfqpoint{11.360630in}{3.400887in}}%
\pgfpathlineto{\pgfqpoint{11.367476in}{3.413657in}}%
\pgfpathlineto{\pgfqpoint{11.372040in}{3.357831in}}%
\pgfpathlineto{\pgfqpoint{11.374322in}{3.372791in}}%
\pgfpathlineto{\pgfqpoint{11.376603in}{3.379359in}}%
\pgfpathlineto{\pgfqpoint{11.383449in}{3.378264in}}%
\pgfpathlineto{\pgfqpoint{11.385731in}{3.374980in}}%
\pgfpathlineto{\pgfqpoint{11.388013in}{3.370602in}}%
\pgfpathlineto{\pgfqpoint{11.392577in}{3.353453in}}%
\pgfpathlineto{\pgfqpoint{11.399423in}{3.362575in}}%
\pgfpathlineto{\pgfqpoint{11.401704in}{3.348709in}}%
\pgfpathlineto{\pgfqpoint{11.403986in}{3.341412in}}%
\pgfpathlineto{\pgfqpoint{11.406268in}{3.331925in}}%
\pgfpathlineto{\pgfqpoint{11.408550in}{3.308573in}}%
\pgfpathlineto{\pgfqpoint{11.415396in}{3.302735in}}%
\pgfpathlineto{\pgfqpoint{11.417678in}{3.314411in}}%
\pgfpathlineto{\pgfqpoint{11.419960in}{3.296532in}}%
\pgfpathlineto{\pgfqpoint{11.422242in}{3.289599in}}%
\pgfpathlineto{\pgfqpoint{11.424523in}{3.306019in}}%
\pgfpathlineto{\pgfqpoint{11.431369in}{3.287410in}}%
\pgfpathlineto{\pgfqpoint{11.433651in}{3.287410in}}%
\pgfpathlineto{\pgfqpoint{11.435933in}{3.276828in}}%
\pgfpathlineto{\pgfqpoint{11.438215in}{3.311857in}}%
\pgfpathlineto{\pgfqpoint{11.440497in}{3.299451in}}%
\pgfpathlineto{\pgfqpoint{11.449624in}{3.261504in}}%
\pgfpathlineto{\pgfqpoint{11.451906in}{3.282666in}}%
\pgfpathlineto{\pgfqpoint{11.454188in}{3.279018in}}%
\pgfpathlineto{\pgfqpoint{11.456470in}{3.270626in}}%
\pgfpathlineto{\pgfqpoint{11.463316in}{3.258585in}}%
\pgfpathlineto{\pgfqpoint{11.465598in}{3.274274in}}%
\pgfpathlineto{\pgfqpoint{11.467880in}{3.275734in}}%
\pgfpathlineto{\pgfqpoint{11.470162in}{3.298721in}}%
\pgfpathlineto{\pgfqpoint{11.472444in}{3.310032in}}%
\pgfpathlineto{\pgfqpoint{11.479289in}{3.324992in}}%
\pgfpathlineto{\pgfqpoint{11.481571in}{3.332655in}}%
\pgfpathlineto{\pgfqpoint{11.483853in}{3.336668in}}%
\pgfpathlineto{\pgfqpoint{11.486135in}{3.330830in}}%
\pgfpathlineto{\pgfqpoint{11.488417in}{3.310762in}}%
\pgfpathlineto{\pgfqpoint{11.495263in}{3.316235in}}%
\pgfpathlineto{\pgfqpoint{11.497545in}{3.296532in}}%
\pgfpathlineto{\pgfqpoint{11.499826in}{3.287410in}}%
\pgfpathlineto{\pgfqpoint{11.502108in}{3.307478in}}%
\pgfpathlineto{\pgfqpoint{11.504390in}{3.288869in}}%
\pgfpathlineto{\pgfqpoint{11.511236in}{3.279747in}}%
\pgfpathlineto{\pgfqpoint{11.513518in}{3.288505in}}%
\pgfpathlineto{\pgfqpoint{11.515800in}{3.283031in}}%
\pgfpathlineto{\pgfqpoint{11.518082in}{3.289234in}}%
\pgfpathlineto{\pgfqpoint{11.520364in}{3.291788in}}%
\pgfpathlineto{\pgfqpoint{11.527209in}{3.296897in}}%
\pgfpathlineto{\pgfqpoint{11.529491in}{3.277193in}}%
\pgfpathlineto{\pgfqpoint{11.531773in}{3.281207in}}%
\pgfpathlineto{\pgfqpoint{11.534055in}{3.299816in}}%
\pgfpathlineto{\pgfqpoint{11.536337in}{3.306384in}}%
\pgfpathlineto{\pgfqpoint{11.543183in}{3.298356in}}%
\pgfpathlineto{\pgfqpoint{11.545465in}{3.284491in}}%
\pgfpathlineto{\pgfqpoint{11.547746in}{3.310762in}}%
\pgfpathlineto{\pgfqpoint{11.550028in}{3.357101in}}%
\pgfpathlineto{\pgfqpoint{11.552310in}{3.388481in}}%
\pgfpathlineto{\pgfqpoint{11.559156in}{3.404536in}}%
\pgfpathlineto{\pgfqpoint{11.561438in}{3.420225in}}%
\pgfpathlineto{\pgfqpoint{11.566002in}{3.399792in}}%
\pgfpathlineto{\pgfqpoint{11.568284in}{3.408184in}}%
\pgfpathlineto{\pgfqpoint{11.575129in}{3.404171in}}%
\pgfpathlineto{\pgfqpoint{11.577411in}{3.418766in}}%
\pgfpathlineto{\pgfqpoint{11.579693in}{3.403076in}}%
\pgfpathlineto{\pgfqpoint{11.584257in}{3.401981in}}%
\pgfpathlineto{\pgfqpoint{11.591103in}{3.418401in}}%
\pgfpathlineto{\pgfqpoint{11.593385in}{3.391035in}}%
\pgfpathlineto{\pgfqpoint{11.595666in}{3.405265in}}%
\pgfpathlineto{\pgfqpoint{11.597948in}{3.392859in}}%
\pgfpathlineto{\pgfqpoint{11.600230in}{3.393954in}}%
\pgfpathlineto{\pgfqpoint{11.607076in}{3.387021in}}%
\pgfpathlineto{\pgfqpoint{11.609358in}{3.392495in}}%
\pgfpathlineto{\pgfqpoint{11.611640in}{3.387751in}}%
\pgfpathlineto{\pgfqpoint{11.613922in}{3.395778in}}%
\pgfpathlineto{\pgfqpoint{11.616204in}{3.396873in}}%
\pgfpathlineto{\pgfqpoint{11.623049in}{3.409644in}}%
\pgfpathlineto{\pgfqpoint{11.625331in}{3.410009in}}%
\pgfpathlineto{\pgfqpoint{11.627613in}{3.399427in}}%
\pgfpathlineto{\pgfqpoint{11.629895in}{3.398698in}}%
\pgfpathlineto{\pgfqpoint{11.632177in}{3.394319in}}%
\pgfpathlineto{\pgfqpoint{11.639023in}{3.388481in}}%
\pgfpathlineto{\pgfqpoint{11.641305in}{3.389576in}}%
\pgfpathlineto{\pgfqpoint{11.643587in}{3.386292in}}%
\pgfpathlineto{\pgfqpoint{11.648150in}{3.377535in}}%
\pgfpathlineto{\pgfqpoint{11.654996in}{3.367318in}}%
\pgfpathlineto{\pgfqpoint{11.657278in}{3.376805in}}%
\pgfpathlineto{\pgfqpoint{11.659560in}{3.370602in}}%
\pgfpathlineto{\pgfqpoint{11.661842in}{3.357466in}}%
\pgfpathlineto{\pgfqpoint{11.664124in}{3.373521in}}%
\pgfpathlineto{\pgfqpoint{11.670969in}{3.376075in}}%
\pgfpathlineto{\pgfqpoint{11.677815in}{3.335574in}}%
\pgfpathlineto{\pgfqpoint{11.680097in}{3.327546in}}%
\pgfpathlineto{\pgfqpoint{11.686943in}{3.339222in}}%
\pgfpathlineto{\pgfqpoint{11.689225in}{3.319519in}}%
\pgfpathlineto{\pgfqpoint{11.691507in}{3.345061in}}%
\pgfpathlineto{\pgfqpoint{11.693788in}{3.343966in}}%
\pgfpathlineto{\pgfqpoint{11.696070in}{3.334114in}}%
\pgfpathlineto{\pgfqpoint{11.702916in}{3.348344in}}%
\pgfpathlineto{\pgfqpoint{11.705198in}{3.359291in}}%
\pgfpathlineto{\pgfqpoint{11.709762in}{3.365129in}}%
\pgfpathlineto{\pgfqpoint{11.712044in}{3.364764in}}%
\pgfpathlineto{\pgfqpoint{11.721171in}{3.364034in}}%
\pgfpathlineto{\pgfqpoint{11.725735in}{3.360021in}}%
\pgfpathlineto{\pgfqpoint{11.728017in}{3.343601in}}%
\pgfpathlineto{\pgfqpoint{11.734863in}{3.351263in}}%
\pgfpathlineto{\pgfqpoint{11.737145in}{3.368413in}}%
\pgfpathlineto{\pgfqpoint{11.739427in}{3.360750in}}%
\pgfpathlineto{\pgfqpoint{11.741709in}{3.332290in}}%
\pgfpathlineto{\pgfqpoint{11.743990in}{3.339587in}}%
\pgfpathlineto{\pgfqpoint{11.750836in}{3.319154in}}%
\pgfpathlineto{\pgfqpoint{11.753118in}{3.320249in}}%
\pgfpathlineto{\pgfqpoint{11.755400in}{3.354182in}}%
\pgfpathlineto{\pgfqpoint{11.757682in}{3.361480in}}%
\pgfpathlineto{\pgfqpoint{11.759964in}{3.356372in}}%
\pgfpathlineto{\pgfqpoint{11.769091in}{3.338128in}}%
\pgfpathlineto{\pgfqpoint{11.771373in}{3.327546in}}%
\pgfpathlineto{\pgfqpoint{11.773655in}{3.342506in}}%
\pgfpathlineto{\pgfqpoint{11.775937in}{3.335939in}}%
\pgfpathlineto{\pgfqpoint{11.782783in}{3.337763in}}%
\pgfpathlineto{\pgfqpoint{11.785065in}{3.328276in}}%
\pgfpathlineto{\pgfqpoint{11.787347in}{3.337763in}}%
\pgfpathlineto{\pgfqpoint{11.789629in}{3.335939in}}%
\pgfpathlineto{\pgfqpoint{11.791910in}{3.347250in}}%
\pgfpathlineto{\pgfqpoint{11.798756in}{3.301275in}}%
\pgfpathlineto{\pgfqpoint{11.801038in}{3.312951in}}%
\pgfpathlineto{\pgfqpoint{11.803320in}{3.309303in}}%
\pgfpathlineto{\pgfqpoint{11.805602in}{3.308938in}}%
\pgfpathlineto{\pgfqpoint{11.807884in}{3.312586in}}%
\pgfpathlineto{\pgfqpoint{11.814730in}{3.315505in}}%
\pgfpathlineto{\pgfqpoint{11.817011in}{3.324263in}}%
\pgfpathlineto{\pgfqpoint{11.819293in}{3.328641in}}%
\pgfpathlineto{\pgfqpoint{11.821575in}{3.326817in}}%
\pgfpathlineto{\pgfqpoint{11.823857in}{3.304194in}}%
\pgfpathlineto{\pgfqpoint{11.832985in}{3.289234in}}%
\pgfpathlineto{\pgfqpoint{11.835267in}{3.304924in}}%
\pgfpathlineto{\pgfqpoint{11.837549in}{3.349804in}}%
\pgfpathlineto{\pgfqpoint{11.839831in}{3.325722in}}%
\pgfpathlineto{\pgfqpoint{11.846676in}{3.296167in}}%
\pgfpathlineto{\pgfqpoint{11.851240in}{3.299816in}}%
\pgfpathlineto{\pgfqpoint{11.853522in}{3.332290in}}%
\pgfpathlineto{\pgfqpoint{11.855804in}{3.334479in}}%
\pgfpathlineto{\pgfqpoint{11.862650in}{3.326817in}}%
\pgfpathlineto{\pgfqpoint{11.864932in}{3.343601in}}%
\pgfpathlineto{\pgfqpoint{11.867213in}{3.329006in}}%
\pgfpathlineto{\pgfqpoint{11.869495in}{3.331195in}}%
\pgfpathlineto{\pgfqpoint{11.871777in}{3.322438in}}%
\pgfpathlineto{\pgfqpoint{11.878623in}{3.319154in}}%
\pgfpathlineto{\pgfqpoint{11.885469in}{3.290694in}}%
\pgfpathlineto{\pgfqpoint{11.887751in}{3.292153in}}%
\pgfpathlineto{\pgfqpoint{11.894596in}{3.298356in}}%
\pgfpathlineto{\pgfqpoint{11.896878in}{3.309667in}}%
\pgfpathlineto{\pgfqpoint{11.899160in}{3.299816in}}%
\pgfpathlineto{\pgfqpoint{11.901442in}{3.324263in}}%
\pgfpathlineto{\pgfqpoint{11.903724in}{3.315505in}}%
\pgfpathlineto{\pgfqpoint{11.910570in}{3.318424in}}%
\pgfpathlineto{\pgfqpoint{11.912852in}{3.324263in}}%
\pgfpathlineto{\pgfqpoint{11.915133in}{3.318424in}}%
\pgfpathlineto{\pgfqpoint{11.917415in}{3.337763in}}%
\pgfpathlineto{\pgfqpoint{11.919697in}{3.330101in}}%
\pgfpathlineto{\pgfqpoint{11.926543in}{3.334479in}}%
\pgfpathlineto{\pgfqpoint{11.928825in}{3.339587in}}%
\pgfpathlineto{\pgfqpoint{11.931107in}{3.341412in}}%
\pgfpathlineto{\pgfqpoint{11.933389in}{3.349804in}}%
\pgfpathlineto{\pgfqpoint{11.935671in}{3.347615in}}%
\pgfpathlineto{\pgfqpoint{11.942516in}{3.349804in}}%
\pgfpathlineto{\pgfqpoint{11.944798in}{3.367683in}}%
\pgfpathlineto{\pgfqpoint{11.947080in}{3.361115in}}%
\pgfpathlineto{\pgfqpoint{11.951644in}{3.335209in}}%
\pgfpathlineto{\pgfqpoint{11.958490in}{3.340682in}}%
\pgfpathlineto{\pgfqpoint{11.960772in}{3.330830in}}%
\pgfpathlineto{\pgfqpoint{11.963054in}{3.335939in}}%
\pgfpathlineto{\pgfqpoint{11.965335in}{3.350169in}}%
\pgfpathlineto{\pgfqpoint{11.974463in}{3.361115in}}%
\pgfpathlineto{\pgfqpoint{11.976745in}{3.359291in}}%
\pgfpathlineto{\pgfqpoint{11.979027in}{3.345425in}}%
\pgfpathlineto{\pgfqpoint{11.981309in}{3.312222in}}%
\pgfpathlineto{\pgfqpoint{11.983591in}{3.302370in}}%
\pgfpathlineto{\pgfqpoint{11.992718in}{3.326452in}}%
\pgfpathlineto{\pgfqpoint{11.995000in}{3.325357in}}%
\pgfpathlineto{\pgfqpoint{11.997282in}{3.339222in}}%
\pgfpathlineto{\pgfqpoint{11.999564in}{3.336303in}}%
\pgfpathlineto{\pgfqpoint{12.008692in}{3.346520in}}%
\pgfpathlineto{\pgfqpoint{12.010974in}{3.364764in}}%
\pgfpathlineto{\pgfqpoint{12.013255in}{3.375710in}}%
\pgfpathlineto{\pgfqpoint{12.015537in}{3.375710in}}%
\pgfpathlineto{\pgfqpoint{12.022383in}{3.366223in}}%
\pgfpathlineto{\pgfqpoint{12.024665in}{3.357466in}}%
\pgfpathlineto{\pgfqpoint{12.026947in}{3.362210in}}%
\pgfpathlineto{\pgfqpoint{12.029229in}{3.360750in}}%
\pgfpathlineto{\pgfqpoint{12.031511in}{3.392495in}}%
\pgfpathlineto{\pgfqpoint{12.038356in}{3.394684in}}%
\pgfpathlineto{\pgfqpoint{12.040638in}{3.380454in}}%
\pgfpathlineto{\pgfqpoint{12.042920in}{3.395414in}}%
\pgfpathlineto{\pgfqpoint{12.047484in}{3.411833in}}%
\pgfpathlineto{\pgfqpoint{12.056612in}{3.408184in}}%
\pgfpathlineto{\pgfqpoint{12.058894in}{3.411833in}}%
\pgfpathlineto{\pgfqpoint{12.061175in}{3.411103in}}%
\pgfpathlineto{\pgfqpoint{12.063457in}{3.418036in}}%
\pgfpathlineto{\pgfqpoint{12.070303in}{3.420590in}}%
\pgfpathlineto{\pgfqpoint{12.072585in}{3.398698in}}%
\pgfpathlineto{\pgfqpoint{12.074867in}{3.394684in}}%
\pgfpathlineto{\pgfqpoint{12.079431in}{3.402711in}}%
\pgfpathlineto{\pgfqpoint{12.086276in}{3.406725in}}%
\pgfpathlineto{\pgfqpoint{12.088558in}{3.405630in}}%
\pgfpathlineto{\pgfqpoint{12.090840in}{3.401252in}}%
\pgfpathlineto{\pgfqpoint{12.093122in}{3.391765in}}%
\pgfpathlineto{\pgfqpoint{12.095404in}{3.395778in}}%
\pgfpathlineto{\pgfqpoint{12.102250in}{3.399427in}}%
\pgfpathlineto{\pgfqpoint{12.104532in}{3.396508in}}%
\pgfpathlineto{\pgfqpoint{12.106814in}{3.402711in}}%
\pgfpathlineto{\pgfqpoint{12.109096in}{3.404171in}}%
\pgfpathlineto{\pgfqpoint{12.111377in}{3.401617in}}%
\pgfpathlineto{\pgfqpoint{12.118223in}{3.411468in}}%
\pgfpathlineto{\pgfqpoint{12.120505in}{3.398698in}}%
\pgfpathlineto{\pgfqpoint{12.122787in}{3.402346in}}%
\pgfpathlineto{\pgfqpoint{12.125069in}{3.396508in}}%
\pgfpathlineto{\pgfqpoint{12.127351in}{3.399792in}}%
\pgfpathlineto{\pgfqpoint{12.134197in}{3.388481in}}%
\pgfpathlineto{\pgfqpoint{12.136478in}{3.401252in}}%
\pgfpathlineto{\pgfqpoint{12.138760in}{3.409644in}}%
\pgfpathlineto{\pgfqpoint{12.141042in}{3.411103in}}%
\pgfpathlineto{\pgfqpoint{12.150170in}{3.412198in}}%
\pgfpathlineto{\pgfqpoint{12.152452in}{3.398698in}}%
\pgfpathlineto{\pgfqpoint{12.154734in}{3.402711in}}%
\pgfpathlineto{\pgfqpoint{12.159297in}{3.443942in}}%
\pgfpathlineto{\pgfqpoint{12.166143in}{3.450145in}}%
\pgfpathlineto{\pgfqpoint{12.168425in}{3.455254in}}%
\pgfpathlineto{\pgfqpoint{12.170707in}{3.462551in}}%
\pgfpathlineto{\pgfqpoint{12.172989in}{3.443942in}}%
\pgfpathlineto{\pgfqpoint{12.175271in}{3.456713in}}%
\pgfpathlineto{\pgfqpoint{12.182117in}{3.455254in}}%
\pgfpathlineto{\pgfqpoint{12.184398in}{3.463281in}}%
\pgfpathlineto{\pgfqpoint{12.186680in}{3.461092in}}%
\pgfpathlineto{\pgfqpoint{12.188962in}{3.465105in}}%
\pgfpathlineto{\pgfqpoint{12.191244in}{3.472038in}}%
\pgfpathlineto{\pgfqpoint{12.198090in}{3.480795in}}%
\pgfpathlineto{\pgfqpoint{12.200372in}{3.492106in}}%
\pgfpathlineto{\pgfqpoint{12.202654in}{3.485903in}}%
\pgfpathlineto{\pgfqpoint{12.204936in}{3.446861in}}%
\pgfpathlineto{\pgfqpoint{12.207218in}{3.429712in}}%
\pgfpathlineto{\pgfqpoint{12.214063in}{3.441023in}}%
\pgfpathlineto{\pgfqpoint{12.220909in}{3.395414in}}%
\pgfpathlineto{\pgfqpoint{12.223191in}{3.396873in}}%
\pgfpathlineto{\pgfqpoint{12.230037in}{3.396143in}}%
\pgfpathlineto{\pgfqpoint{12.232319in}{3.400887in}}%
\pgfpathlineto{\pgfqpoint{12.236882in}{3.405630in}}%
\pgfpathlineto{\pgfqpoint{12.239164in}{3.399792in}}%
\pgfpathlineto{\pgfqpoint{12.246010in}{3.399427in}}%
\pgfpathlineto{\pgfqpoint{12.248292in}{3.396873in}}%
\pgfpathlineto{\pgfqpoint{12.250574in}{3.400522in}}%
\pgfpathlineto{\pgfqpoint{12.252856in}{3.401981in}}%
\pgfpathlineto{\pgfqpoint{12.255138in}{3.395049in}}%
\pgfpathlineto{\pgfqpoint{12.261983in}{3.410009in}}%
\pgfpathlineto{\pgfqpoint{12.264265in}{3.412928in}}%
\pgfpathlineto{\pgfqpoint{12.266547in}{3.418036in}}%
\pgfpathlineto{\pgfqpoint{12.268829in}{3.417671in}}%
\pgfpathlineto{\pgfqpoint{12.271111in}{3.429347in}}%
\pgfpathlineto{\pgfqpoint{12.280239in}{3.427888in}}%
\pgfpathlineto{\pgfqpoint{12.282520in}{3.431172in}}%
\pgfpathlineto{\pgfqpoint{12.284802in}{3.426428in}}%
\pgfpathlineto{\pgfqpoint{12.287084in}{3.432996in}}%
\pgfpathlineto{\pgfqpoint{12.293930in}{3.420955in}}%
\pgfpathlineto{\pgfqpoint{12.296212in}{3.402711in}}%
\pgfpathlineto{\pgfqpoint{12.298494in}{3.398333in}}%
\pgfpathlineto{\pgfqpoint{12.300776in}{3.405995in}}%
\pgfpathlineto{\pgfqpoint{12.303058in}{3.388116in}}%
\pgfpathlineto{\pgfqpoint{12.309903in}{3.392859in}}%
\pgfpathlineto{\pgfqpoint{12.312185in}{3.404536in}}%
\pgfpathlineto{\pgfqpoint{12.314467in}{3.411833in}}%
\pgfpathlineto{\pgfqpoint{12.316749in}{3.425698in}}%
\pgfpathlineto{\pgfqpoint{12.319031in}{3.445037in}}%
\pgfpathlineto{\pgfqpoint{12.325877in}{3.439929in}}%
\pgfpathlineto{\pgfqpoint{12.328159in}{3.432266in}}%
\pgfpathlineto{\pgfqpoint{12.330441in}{3.436645in}}%
\pgfpathlineto{\pgfqpoint{12.332722in}{3.423509in}}%
\pgfpathlineto{\pgfqpoint{12.335004in}{3.427888in}}%
\pgfpathlineto{\pgfqpoint{12.341850in}{3.427523in}}%
\pgfpathlineto{\pgfqpoint{12.344132in}{3.434455in}}%
\pgfpathlineto{\pgfqpoint{12.346414in}{3.418036in}}%
\pgfpathlineto{\pgfqpoint{12.348696in}{3.414022in}}%
\pgfpathlineto{\pgfqpoint{12.350978in}{3.426428in}}%
\pgfpathlineto{\pgfqpoint{12.357823in}{3.437010in}}%
\pgfpathlineto{\pgfqpoint{12.360105in}{3.425334in}}%
\pgfpathlineto{\pgfqpoint{12.362387in}{3.447226in}}%
\pgfpathlineto{\pgfqpoint{12.364669in}{3.419496in}}%
\pgfpathlineto{\pgfqpoint{12.366951in}{3.419860in}}%
\pgfpathlineto{\pgfqpoint{12.376079in}{3.398698in}}%
\pgfpathlineto{\pgfqpoint{12.380642in}{3.382643in}}%
\pgfpathlineto{\pgfqpoint{12.382924in}{3.394319in}}%
\pgfpathlineto{\pgfqpoint{12.389770in}{3.406360in}}%
\pgfpathlineto{\pgfqpoint{12.392052in}{3.412563in}}%
\pgfpathlineto{\pgfqpoint{12.394334in}{3.401252in}}%
\pgfpathlineto{\pgfqpoint{12.396616in}{3.398698in}}%
\pgfpathlineto{\pgfqpoint{12.398898in}{3.414752in}}%
\pgfpathlineto{\pgfqpoint{12.405743in}{3.434091in}}%
\pgfpathlineto{\pgfqpoint{12.408025in}{3.449780in}}%
\pgfpathlineto{\pgfqpoint{12.410307in}{3.446132in}}%
\pgfpathlineto{\pgfqpoint{12.412589in}{3.447956in}}%
\pgfpathlineto{\pgfqpoint{12.414871in}{3.458173in}}%
\pgfpathlineto{\pgfqpoint{12.421717in}{3.462551in}}%
\pgfpathlineto{\pgfqpoint{12.423999in}{3.460727in}}%
\pgfpathlineto{\pgfqpoint{12.426281in}{3.461092in}}%
\pgfpathlineto{\pgfqpoint{12.428563in}{3.459267in}}%
\pgfpathlineto{\pgfqpoint{12.430844in}{3.476416in}}%
\pgfpathlineto{\pgfqpoint{12.437690in}{3.472038in}}%
\pgfpathlineto{\pgfqpoint{12.439972in}{3.468754in}}%
\pgfpathlineto{\pgfqpoint{12.442254in}{3.474957in}}%
\pgfpathlineto{\pgfqpoint{12.446818in}{3.495025in}}%
\pgfpathlineto{\pgfqpoint{12.453663in}{3.492106in}}%
\pgfpathlineto{\pgfqpoint{12.455945in}{3.486998in}}%
\pgfpathlineto{\pgfqpoint{12.458227in}{3.465470in}}%
\pgfpathlineto{\pgfqpoint{12.460509in}{3.456713in}}%
\pgfpathlineto{\pgfqpoint{12.462791in}{3.457078in}}%
\pgfpathlineto{\pgfqpoint{12.469637in}{3.436280in}}%
\pgfpathlineto{\pgfqpoint{12.471919in}{3.431536in}}%
\pgfpathlineto{\pgfqpoint{12.474201in}{3.452334in}}%
\pgfpathlineto{\pgfqpoint{12.478764in}{3.468389in}}%
\pgfpathlineto{\pgfqpoint{12.485610in}{3.451970in}}%
\pgfpathlineto{\pgfqpoint{12.487892in}{3.424969in}}%
\pgfpathlineto{\pgfqpoint{12.490174in}{3.415482in}}%
\pgfpathlineto{\pgfqpoint{12.492456in}{3.415117in}}%
\pgfpathlineto{\pgfqpoint{12.494738in}{3.410009in}}%
\pgfpathlineto{\pgfqpoint{12.501584in}{3.418766in}}%
\pgfpathlineto{\pgfqpoint{12.503865in}{3.360385in}}%
\pgfpathlineto{\pgfqpoint{12.506147in}{3.338858in}}%
\pgfpathlineto{\pgfqpoint{12.508429in}{3.343966in}}%
\pgfpathlineto{\pgfqpoint{12.510711in}{3.320979in}}%
\pgfpathlineto{\pgfqpoint{12.517557in}{3.316600in}}%
\pgfpathlineto{\pgfqpoint{12.519839in}{3.319519in}}%
\pgfpathlineto{\pgfqpoint{12.522121in}{3.346520in}}%
\pgfpathlineto{\pgfqpoint{12.524403in}{3.364399in}}%
\pgfpathlineto{\pgfqpoint{12.526684in}{3.363304in}}%
\pgfpathlineto{\pgfqpoint{12.535812in}{3.382643in}}%
\pgfpathlineto{\pgfqpoint{12.538094in}{3.382643in}}%
\pgfpathlineto{\pgfqpoint{12.542658in}{3.388116in}}%
\pgfpathlineto{\pgfqpoint{12.549504in}{3.378994in}}%
\pgfpathlineto{\pgfqpoint{12.551785in}{3.372426in}}%
\pgfpathlineto{\pgfqpoint{12.554067in}{3.356372in}}%
\pgfpathlineto{\pgfqpoint{12.558631in}{3.361845in}}%
\pgfpathlineto{\pgfqpoint{12.565477in}{3.350169in}}%
\pgfpathlineto{\pgfqpoint{12.567759in}{3.364034in}}%
\pgfpathlineto{\pgfqpoint{12.570041in}{3.355277in}}%
\pgfpathlineto{\pgfqpoint{12.572323in}{3.384467in}}%
\pgfpathlineto{\pgfqpoint{12.574605in}{3.372061in}}%
\pgfpathlineto{\pgfqpoint{12.583732in}{3.384467in}}%
\pgfpathlineto{\pgfqpoint{12.586014in}{3.378264in}}%
\pgfpathlineto{\pgfqpoint{12.588296in}{3.382278in}}%
\pgfpathlineto{\pgfqpoint{12.590578in}{3.408549in}}%
\pgfpathlineto{\pgfqpoint{12.601987in}{3.416212in}}%
\pgfpathlineto{\pgfqpoint{12.604269in}{3.403076in}}%
\pgfpathlineto{\pgfqpoint{12.606551in}{3.383738in}}%
\pgfpathlineto{\pgfqpoint{12.613397in}{3.377899in}}%
\pgfpathlineto{\pgfqpoint{12.615679in}{3.360021in}}%
\pgfpathlineto{\pgfqpoint{12.617961in}{3.350899in}}%
\pgfpathlineto{\pgfqpoint{12.620243in}{3.352723in}}%
\pgfpathlineto{\pgfqpoint{12.622525in}{3.340682in}}%
\pgfpathlineto{\pgfqpoint{12.629370in}{3.378629in}}%
\pgfpathlineto{\pgfqpoint{12.631652in}{3.403806in}}%
\pgfpathlineto{\pgfqpoint{12.633934in}{3.403076in}}%
\pgfpathlineto{\pgfqpoint{12.636216in}{3.404900in}}%
\pgfpathlineto{\pgfqpoint{12.638498in}{3.448321in}}%
\pgfpathlineto{\pgfqpoint{12.645344in}{3.441388in}}%
\pgfpathlineto{\pgfqpoint{12.647626in}{3.453794in}}%
\pgfpathlineto{\pgfqpoint{12.649907in}{3.462916in}}%
\pgfpathlineto{\pgfqpoint{12.652189in}{3.454159in}}%
\pgfpathlineto{\pgfqpoint{12.654471in}{3.447956in}}%
\pgfpathlineto{\pgfqpoint{12.663599in}{3.443942in}}%
\pgfpathlineto{\pgfqpoint{12.665881in}{3.437010in}}%
\pgfpathlineto{\pgfqpoint{12.668163in}{3.435915in}}%
\pgfpathlineto{\pgfqpoint{12.670445in}{3.437739in}}%
\pgfpathlineto{\pgfqpoint{12.677290in}{3.431901in}}%
\pgfpathlineto{\pgfqpoint{12.679572in}{3.445037in}}%
\pgfpathlineto{\pgfqpoint{12.681854in}{3.444672in}}%
\pgfpathlineto{\pgfqpoint{12.684136in}{3.449780in}}%
\pgfpathlineto{\pgfqpoint{12.686418in}{3.451970in}}%
\pgfpathlineto{\pgfqpoint{12.693264in}{3.452699in}}%
\pgfpathlineto{\pgfqpoint{12.695546in}{3.454889in}}%
\pgfpathlineto{\pgfqpoint{12.697828in}{3.440658in}}%
\pgfpathlineto{\pgfqpoint{12.700109in}{3.436280in}}%
\pgfpathlineto{\pgfqpoint{12.702391in}{3.417671in}}%
\pgfpathlineto{\pgfqpoint{12.709237in}{3.415847in}}%
\pgfpathlineto{\pgfqpoint{12.711519in}{3.394319in}}%
\pgfpathlineto{\pgfqpoint{12.713801in}{3.399427in}}%
\pgfpathlineto{\pgfqpoint{12.716083in}{3.430807in}}%
\pgfpathlineto{\pgfqpoint{12.718365in}{3.433726in}}%
\pgfpathlineto{\pgfqpoint{12.725210in}{3.446861in}}%
\pgfpathlineto{\pgfqpoint{12.727492in}{3.437010in}}%
\pgfpathlineto{\pgfqpoint{12.729774in}{3.454889in}}%
\pgfpathlineto{\pgfqpoint{12.732056in}{3.447591in}}%
\pgfpathlineto{\pgfqpoint{12.734338in}{3.455254in}}%
\pgfpathlineto{\pgfqpoint{12.741184in}{3.457808in}}%
\pgfpathlineto{\pgfqpoint{12.743466in}{3.450875in}}%
\pgfpathlineto{\pgfqpoint{12.745748in}{3.430807in}}%
\pgfpathlineto{\pgfqpoint{12.748029in}{3.421320in}}%
\pgfpathlineto{\pgfqpoint{12.750311in}{3.425698in}}%
\pgfpathlineto{\pgfqpoint{12.757157in}{3.442118in}}%
\pgfpathlineto{\pgfqpoint{12.759439in}{3.427523in}}%
\pgfpathlineto{\pgfqpoint{12.761721in}{3.436280in}}%
\pgfpathlineto{\pgfqpoint{12.764003in}{3.452699in}}%
\pgfpathlineto{\pgfqpoint{12.773130in}{3.457443in}}%
\pgfpathlineto{\pgfqpoint{12.775412in}{3.446861in}}%
\pgfpathlineto{\pgfqpoint{12.777694in}{3.458902in}}%
\pgfpathlineto{\pgfqpoint{12.779976in}{3.455254in}}%
\pgfpathlineto{\pgfqpoint{12.782258in}{3.461821in}}%
\pgfpathlineto{\pgfqpoint{12.789104in}{3.456348in}}%
\pgfpathlineto{\pgfqpoint{12.791386in}{3.460362in}}%
\pgfpathlineto{\pgfqpoint{12.793668in}{3.466930in}}%
\pgfpathlineto{\pgfqpoint{12.795950in}{3.463281in}}%
\pgfpathlineto{\pgfqpoint{12.798231in}{3.451970in}}%
\pgfpathlineto{\pgfqpoint{12.805077in}{3.466565in}}%
\pgfpathlineto{\pgfqpoint{12.807359in}{3.460362in}}%
\pgfpathlineto{\pgfqpoint{12.811923in}{3.486633in}}%
\pgfpathlineto{\pgfqpoint{12.814205in}{3.485903in}}%
\pgfpathlineto{\pgfqpoint{12.821050in}{3.487728in}}%
\pgfpathlineto{\pgfqpoint{12.823332in}{3.501593in}}%
\pgfpathlineto{\pgfqpoint{12.825614in}{3.499404in}}%
\pgfpathlineto{\pgfqpoint{12.830178in}{3.497579in}}%
\pgfpathlineto{\pgfqpoint{12.837024in}{3.500863in}}%
\pgfpathlineto{\pgfqpoint{12.841588in}{3.474227in}}%
\pgfpathlineto{\pgfqpoint{12.843870in}{3.477146in}}%
\pgfpathlineto{\pgfqpoint{12.846151in}{3.489552in}}%
\pgfpathlineto{\pgfqpoint{12.852997in}{3.479335in}}%
\pgfpathlineto{\pgfqpoint{12.855279in}{3.473862in}}%
\pgfpathlineto{\pgfqpoint{12.857561in}{3.477146in}}%
\pgfpathlineto{\pgfqpoint{12.859843in}{3.484444in}}%
\pgfpathlineto{\pgfqpoint{12.862125in}{3.478971in}}%
\pgfpathlineto{\pgfqpoint{12.871252in}{3.471673in}}%
\pgfpathlineto{\pgfqpoint{12.873534in}{3.475687in}}%
\pgfpathlineto{\pgfqpoint{12.875816in}{3.482254in}}%
\pgfpathlineto{\pgfqpoint{12.878098in}{3.473497in}}%
\pgfpathlineto{\pgfqpoint{12.887226in}{3.467659in}}%
\pgfpathlineto{\pgfqpoint{12.889508in}{3.472403in}}%
\pgfpathlineto{\pgfqpoint{12.891790in}{3.471308in}}%
\pgfpathlineto{\pgfqpoint{12.894072in}{3.468389in}}%
\pgfpathlineto{\pgfqpoint{12.905481in}{3.457443in}}%
\pgfpathlineto{\pgfqpoint{12.910045in}{3.401981in}}%
\pgfpathlineto{\pgfqpoint{12.916891in}{3.408184in}}%
\pgfpathlineto{\pgfqpoint{12.919172in}{3.405265in}}%
\pgfpathlineto{\pgfqpoint{12.921454in}{3.409279in}}%
\pgfpathlineto{\pgfqpoint{12.923736in}{3.417671in}}%
\pgfpathlineto{\pgfqpoint{12.926018in}{3.402346in}}%
\pgfpathlineto{\pgfqpoint{12.932864in}{3.395049in}}%
\pgfpathlineto{\pgfqpoint{12.935146in}{3.407455in}}%
\pgfpathlineto{\pgfqpoint{12.937428in}{3.403076in}}%
\pgfpathlineto{\pgfqpoint{12.939710in}{3.418036in}}%
\pgfpathlineto{\pgfqpoint{12.941992in}{3.408914in}}%
\pgfpathlineto{\pgfqpoint{12.948837in}{3.410738in}}%
\pgfpathlineto{\pgfqpoint{12.951119in}{3.418036in}}%
\pgfpathlineto{\pgfqpoint{12.953401in}{3.403806in}}%
\pgfpathlineto{\pgfqpoint{12.957965in}{3.413657in}}%
\pgfpathlineto{\pgfqpoint{12.964811in}{3.388116in}}%
\pgfpathlineto{\pgfqpoint{12.967093in}{3.383373in}}%
\pgfpathlineto{\pgfqpoint{12.969374in}{3.395049in}}%
\pgfpathlineto{\pgfqpoint{12.971656in}{3.400887in}}%
\pgfpathlineto{\pgfqpoint{12.980784in}{3.396143in}}%
\pgfpathlineto{\pgfqpoint{12.983066in}{3.402346in}}%
\pgfpathlineto{\pgfqpoint{12.985348in}{3.398698in}}%
\pgfpathlineto{\pgfqpoint{12.987630in}{3.389211in}}%
\pgfpathlineto{\pgfqpoint{12.989912in}{3.411468in}}%
\pgfpathlineto{\pgfqpoint{12.996757in}{3.417671in}}%
\pgfpathlineto{\pgfqpoint{12.999039in}{3.424239in}}%
\pgfpathlineto{\pgfqpoint{13.001321in}{3.421685in}}%
\pgfpathlineto{\pgfqpoint{13.003603in}{3.437010in}}%
\pgfpathlineto{\pgfqpoint{13.005885in}{3.429712in}}%
\pgfpathlineto{\pgfqpoint{13.012731in}{3.445037in}}%
\pgfpathlineto{\pgfqpoint{13.015013in}{3.410738in}}%
\pgfpathlineto{\pgfqpoint{13.017294in}{3.395049in}}%
\pgfpathlineto{\pgfqpoint{13.019576in}{3.391765in}}%
\pgfpathlineto{\pgfqpoint{13.021858in}{3.382278in}}%
\pgfpathlineto{\pgfqpoint{13.028704in}{3.376075in}}%
\pgfpathlineto{\pgfqpoint{13.030986in}{3.377899in}}%
\pgfpathlineto{\pgfqpoint{13.033268in}{3.398333in}}%
\pgfpathlineto{\pgfqpoint{13.037832in}{3.405265in}}%
\pgfpathlineto{\pgfqpoint{13.044677in}{3.410738in}}%
\pgfpathlineto{\pgfqpoint{13.046959in}{3.401617in}}%
\pgfpathlineto{\pgfqpoint{13.049241in}{3.400522in}}%
\pgfpathlineto{\pgfqpoint{13.051523in}{3.400522in}}%
\pgfpathlineto{\pgfqpoint{13.053805in}{3.392130in}}%
\pgfpathlineto{\pgfqpoint{13.065215in}{3.437375in}}%
\pgfpathlineto{\pgfqpoint{13.069778in}{3.426428in}}%
\pgfpathlineto{\pgfqpoint{13.076624in}{3.427523in}}%
\pgfpathlineto{\pgfqpoint{13.078906in}{3.425698in}}%
\pgfpathlineto{\pgfqpoint{13.081188in}{3.424969in}}%
\pgfpathlineto{\pgfqpoint{13.083470in}{3.407819in}}%
\pgfpathlineto{\pgfqpoint{13.085752in}{3.384102in}}%
\pgfpathlineto{\pgfqpoint{13.092597in}{3.342871in}}%
\pgfpathlineto{\pgfqpoint{13.094879in}{3.305289in}}%
\pgfpathlineto{\pgfqpoint{13.097161in}{3.355642in}}%
\pgfpathlineto{\pgfqpoint{13.099443in}{3.387021in}}%
\pgfpathlineto{\pgfqpoint{13.101725in}{3.383373in}}%
\pgfpathlineto{\pgfqpoint{13.108571in}{3.381548in}}%
\pgfpathlineto{\pgfqpoint{13.110853in}{3.347615in}}%
\pgfpathlineto{\pgfqpoint{13.115416in}{3.372791in}}%
\pgfpathlineto{\pgfqpoint{13.117698in}{3.345425in}}%
\pgfpathlineto{\pgfqpoint{13.126826in}{3.377535in}}%
\pgfpathlineto{\pgfqpoint{13.129108in}{3.362575in}}%
\pgfpathlineto{\pgfqpoint{13.131390in}{3.364764in}}%
\pgfpathlineto{\pgfqpoint{13.133672in}{3.373156in}}%
\pgfpathlineto{\pgfqpoint{13.140517in}{3.370602in}}%
\pgfpathlineto{\pgfqpoint{13.142799in}{3.392495in}}%
\pgfpathlineto{\pgfqpoint{13.145081in}{3.387021in}}%
\pgfpathlineto{\pgfqpoint{13.147363in}{3.357831in}}%
\pgfpathlineto{\pgfqpoint{13.149645in}{3.337763in}}%
\pgfpathlineto{\pgfqpoint{13.156491in}{3.344696in}}%
\pgfpathlineto{\pgfqpoint{13.161055in}{3.319884in}}%
\pgfpathlineto{\pgfqpoint{13.165618in}{3.327182in}}%
\pgfpathlineto{\pgfqpoint{13.172464in}{3.312586in}}%
\pgfpathlineto{\pgfqpoint{13.174746in}{3.306019in}}%
\pgfpathlineto{\pgfqpoint{13.177028in}{3.305289in}}%
\pgfpathlineto{\pgfqpoint{13.179310in}{3.288869in}}%
\pgfpathlineto{\pgfqpoint{13.181592in}{3.285221in}}%
\pgfpathlineto{\pgfqpoint{13.188437in}{3.320249in}}%
\pgfpathlineto{\pgfqpoint{13.190719in}{3.322073in}}%
\pgfpathlineto{\pgfqpoint{13.195283in}{3.344696in}}%
\pgfpathlineto{\pgfqpoint{13.197565in}{3.342506in}}%
\pgfpathlineto{\pgfqpoint{13.206693in}{3.348709in}}%
\pgfpathlineto{\pgfqpoint{13.208975in}{3.337398in}}%
\pgfpathlineto{\pgfqpoint{13.211257in}{3.358561in}}%
\pgfpathlineto{\pgfqpoint{13.213538in}{3.359291in}}%
\pgfpathlineto{\pgfqpoint{13.220384in}{3.359291in}}%
\pgfpathlineto{\pgfqpoint{13.222666in}{3.376075in}}%
\pgfpathlineto{\pgfqpoint{13.224948in}{3.364399in}}%
\pgfpathlineto{\pgfqpoint{13.227230in}{3.396143in}}%
\pgfpathlineto{\pgfqpoint{13.229512in}{3.404171in}}%
\pgfpathlineto{\pgfqpoint{13.236358in}{3.410374in}}%
\pgfpathlineto{\pgfqpoint{13.238639in}{3.404536in}}%
\pgfpathlineto{\pgfqpoint{13.240921in}{3.414022in}}%
\pgfpathlineto{\pgfqpoint{13.243203in}{3.411833in}}%
\pgfpathlineto{\pgfqpoint{13.245485in}{3.426428in}}%
\pgfpathlineto{\pgfqpoint{13.252331in}{3.423509in}}%
\pgfpathlineto{\pgfqpoint{13.256895in}{3.403806in}}%
\pgfpathlineto{\pgfqpoint{13.259177in}{3.405630in}}%
\pgfpathlineto{\pgfqpoint{13.261459in}{3.392495in}}%
\pgfpathlineto{\pgfqpoint{13.268304in}{3.377899in}}%
\pgfpathlineto{\pgfqpoint{13.270586in}{3.371697in}}%
\pgfpathlineto{\pgfqpoint{13.272868in}{3.378264in}}%
\pgfpathlineto{\pgfqpoint{13.277432in}{3.344696in}}%
\pgfpathlineto{\pgfqpoint{13.284278in}{3.369872in}}%
\pgfpathlineto{\pgfqpoint{13.286559in}{3.370967in}}%
\pgfpathlineto{\pgfqpoint{13.288841in}{3.380089in}}%
\pgfpathlineto{\pgfqpoint{13.291123in}{3.392130in}}%
\pgfpathlineto{\pgfqpoint{13.293405in}{3.380454in}}%
\pgfpathlineto{\pgfqpoint{13.300251in}{3.368413in}}%
\pgfpathlineto{\pgfqpoint{13.302533in}{3.374251in}}%
\pgfpathlineto{\pgfqpoint{13.304815in}{3.366223in}}%
\pgfpathlineto{\pgfqpoint{13.309379in}{3.375710in}}%
\pgfpathlineto{\pgfqpoint{13.316224in}{3.382278in}}%
\pgfpathlineto{\pgfqpoint{13.318506in}{3.386292in}}%
\pgfpathlineto{\pgfqpoint{13.320788in}{3.366223in}}%
\pgfpathlineto{\pgfqpoint{13.323070in}{3.354912in}}%
\pgfpathlineto{\pgfqpoint{13.325352in}{3.390305in}}%
\pgfpathlineto{\pgfqpoint{13.332198in}{3.401252in}}%
\pgfpathlineto{\pgfqpoint{13.336761in}{3.379724in}}%
\pgfpathlineto{\pgfqpoint{13.339043in}{3.378264in}}%
\pgfpathlineto{\pgfqpoint{13.341325in}{3.362940in}}%
\pgfpathlineto{\pgfqpoint{13.348171in}{3.382278in}}%
\pgfpathlineto{\pgfqpoint{13.350453in}{3.385562in}}%
\pgfpathlineto{\pgfqpoint{13.352735in}{3.415117in}}%
\pgfpathlineto{\pgfqpoint{13.355017in}{3.402346in}}%
\pgfpathlineto{\pgfqpoint{13.357299in}{3.385927in}}%
\pgfpathlineto{\pgfqpoint{13.364144in}{3.396143in}}%
\pgfpathlineto{\pgfqpoint{13.366426in}{3.410009in}}%
\pgfpathlineto{\pgfqpoint{13.368708in}{3.428617in}}%
\pgfpathlineto{\pgfqpoint{13.370990in}{3.421320in}}%
\pgfpathlineto{\pgfqpoint{13.380118in}{3.422415in}}%
\pgfpathlineto{\pgfqpoint{13.382400in}{3.436645in}}%
\pgfpathlineto{\pgfqpoint{13.384681in}{3.423144in}}%
\pgfpathlineto{\pgfqpoint{13.386963in}{3.405995in}}%
\pgfpathlineto{\pgfqpoint{13.396091in}{3.395414in}}%
\pgfpathlineto{\pgfqpoint{13.398373in}{3.414752in}}%
\pgfpathlineto{\pgfqpoint{13.400655in}{3.401981in}}%
\pgfpathlineto{\pgfqpoint{13.402937in}{3.393954in}}%
\pgfpathlineto{\pgfqpoint{13.405219in}{3.380454in}}%
\pgfpathlineto{\pgfqpoint{13.412064in}{3.388481in}}%
\pgfpathlineto{\pgfqpoint{13.414346in}{3.383373in}}%
\pgfpathlineto{\pgfqpoint{13.416628in}{3.358926in}}%
\pgfpathlineto{\pgfqpoint{13.418910in}{3.381548in}}%
\pgfpathlineto{\pgfqpoint{13.421192in}{3.367683in}}%
\pgfpathlineto{\pgfqpoint{13.430320in}{3.381548in}}%
\pgfpathlineto{\pgfqpoint{13.432602in}{3.367683in}}%
\pgfpathlineto{\pgfqpoint{13.437165in}{3.449415in}}%
\pgfpathlineto{\pgfqpoint{13.444011in}{3.449051in}}%
\pgfpathlineto{\pgfqpoint{13.446293in}{3.486998in}}%
\pgfpathlineto{\pgfqpoint{13.448575in}{3.511445in}}%
\pgfpathlineto{\pgfqpoint{13.450857in}{3.510715in}}%
\pgfpathlineto{\pgfqpoint{13.453139in}{3.540635in}}%
\pgfpathlineto{\pgfqpoint{13.459984in}{3.565446in}}%
\pgfpathlineto{\pgfqpoint{13.462266in}{3.538810in}}%
\pgfpathlineto{\pgfqpoint{13.464548in}{3.561068in}}%
\pgfpathlineto{\pgfqpoint{13.466830in}{3.554865in}}%
\pgfpathlineto{\pgfqpoint{13.469112in}{3.572014in}}%
\pgfpathlineto{\pgfqpoint{13.475958in}{3.564717in}}%
\pgfpathlineto{\pgfqpoint{13.478240in}{3.546108in}}%
\pgfpathlineto{\pgfqpoint{13.480522in}{3.541000in}}%
\pgfpathlineto{\pgfqpoint{13.482803in}{3.522391in}}%
\pgfpathlineto{\pgfqpoint{13.485085in}{3.545013in}}%
\pgfpathlineto{\pgfqpoint{13.494213in}{3.549027in}}%
\pgfpathlineto{\pgfqpoint{13.496495in}{3.551581in}}%
\pgfpathlineto{\pgfqpoint{13.498777in}{3.570920in}}%
\pgfpathlineto{\pgfqpoint{13.501059in}{3.568366in}}%
\pgfpathlineto{\pgfqpoint{13.507904in}{3.574933in}}%
\pgfpathlineto{\pgfqpoint{13.510186in}{3.561068in}}%
\pgfpathlineto{\pgfqpoint{13.512468in}{3.567271in}}%
\pgfpathlineto{\pgfqpoint{13.514750in}{3.576028in}}%
\pgfpathlineto{\pgfqpoint{13.517032in}{3.573474in}}%
\pgfpathlineto{\pgfqpoint{13.523878in}{3.564352in}}%
\pgfpathlineto{\pgfqpoint{13.528442in}{3.607772in}}%
\pgfpathlineto{\pgfqpoint{13.530724in}{3.600475in}}%
\pgfpathlineto{\pgfqpoint{13.533005in}{3.597921in}}%
\pgfpathlineto{\pgfqpoint{13.539851in}{3.610691in}}%
\pgfpathlineto{\pgfqpoint{13.542133in}{3.618354in}}%
\pgfpathlineto{\pgfqpoint{13.544415in}{3.614705in}}%
\pgfpathlineto{\pgfqpoint{13.546697in}{3.613975in}}%
\pgfpathlineto{\pgfqpoint{13.548979in}{3.620543in}}%
\pgfpathlineto{\pgfqpoint{13.555824in}{3.620908in}}%
\pgfpathlineto{\pgfqpoint{13.558106in}{3.624922in}}%
\pgfpathlineto{\pgfqpoint{13.562670in}{3.654841in}}%
\pgfpathlineto{\pgfqpoint{13.564952in}{3.642801in}}%
\pgfpathlineto{\pgfqpoint{13.571798in}{3.649003in}}%
\pgfpathlineto{\pgfqpoint{13.574080in}{3.641706in}}%
\pgfpathlineto{\pgfqpoint{13.576362in}{3.632219in}}%
\pgfpathlineto{\pgfqpoint{13.578644in}{3.652652in}}%
\pgfpathlineto{\pgfqpoint{13.587771in}{3.647544in}}%
\pgfpathlineto{\pgfqpoint{13.590053in}{3.667977in}}%
\pgfpathlineto{\pgfqpoint{13.592335in}{3.667612in}}%
\pgfpathlineto{\pgfqpoint{13.594617in}{3.668707in}}%
\pgfpathlineto{\pgfqpoint{13.596899in}{3.666518in}}%
\pgfpathlineto{\pgfqpoint{13.603745in}{3.679288in}}%
\pgfpathlineto{\pgfqpoint{13.606026in}{3.669072in}}%
\pgfpathlineto{\pgfqpoint{13.608308in}{3.669072in}}%
\pgfpathlineto{\pgfqpoint{13.610590in}{3.621273in}}%
\pgfpathlineto{\pgfqpoint{13.612872in}{3.626746in}}%
\pgfpathlineto{\pgfqpoint{13.619718in}{3.608867in}}%
\pgfpathlineto{\pgfqpoint{13.622000in}{3.619448in}}%
\pgfpathlineto{\pgfqpoint{13.624282in}{3.598650in}}%
\pgfpathlineto{\pgfqpoint{13.626564in}{3.600840in}}%
\pgfpathlineto{\pgfqpoint{13.628846in}{3.600475in}}%
\pgfpathlineto{\pgfqpoint{13.635691in}{3.612516in}}%
\pgfpathlineto{\pgfqpoint{13.637973in}{3.623462in}}%
\pgfpathlineto{\pgfqpoint{13.640255in}{3.613245in}}%
\pgfpathlineto{\pgfqpoint{13.642537in}{3.558879in}}%
\pgfpathlineto{\pgfqpoint{13.644819in}{3.575298in}}%
\pgfpathlineto{\pgfqpoint{13.651665in}{3.581866in}}%
\pgfpathlineto{\pgfqpoint{13.653946in}{3.572014in}}%
\pgfpathlineto{\pgfqpoint{13.656228in}{3.611421in}}%
\pgfpathlineto{\pgfqpoint{13.658510in}{3.590258in}}%
\pgfpathlineto{\pgfqpoint{13.660792in}{3.587704in}}%
\pgfpathlineto{\pgfqpoint{13.667638in}{3.599745in}}%
\pgfpathlineto{\pgfqpoint{13.669920in}{3.579312in}}%
\pgfpathlineto{\pgfqpoint{13.672202in}{3.584420in}}%
\pgfpathlineto{\pgfqpoint{13.674484in}{3.584420in}}%
\pgfpathlineto{\pgfqpoint{13.676766in}{3.593177in}}%
\pgfpathlineto{\pgfqpoint{13.683611in}{3.592083in}}%
\pgfpathlineto{\pgfqpoint{13.685893in}{3.606678in}}%
\pgfpathlineto{\pgfqpoint{13.688175in}{3.594272in}}%
\pgfpathlineto{\pgfqpoint{13.690457in}{3.604488in}}%
\pgfpathlineto{\pgfqpoint{13.692739in}{3.587704in}}%
\pgfpathlineto{\pgfqpoint{13.699585in}{3.596826in}}%
\pgfpathlineto{\pgfqpoint{13.701867in}{3.586245in}}%
\pgfpathlineto{\pgfqpoint{13.704148in}{3.570190in}}%
\pgfpathlineto{\pgfqpoint{13.706430in}{3.546473in}}%
\pgfpathlineto{\pgfqpoint{13.708712in}{3.547203in}}%
\pgfpathlineto{\pgfqpoint{13.715558in}{3.530783in}}%
\pgfpathlineto{\pgfqpoint{13.717840in}{3.544648in}}%
\pgfpathlineto{\pgfqpoint{13.722404in}{3.563257in}}%
\pgfpathlineto{\pgfqpoint{13.724686in}{3.577487in}}%
\pgfpathlineto{\pgfqpoint{13.733813in}{3.586245in}}%
\pgfpathlineto{\pgfqpoint{13.736095in}{3.571649in}}%
\pgfpathlineto{\pgfqpoint{13.738377in}{3.582231in}}%
\pgfpathlineto{\pgfqpoint{13.740659in}{3.586974in}}%
\pgfpathlineto{\pgfqpoint{13.747505in}{3.580406in}}%
\pgfpathlineto{\pgfqpoint{13.749787in}{3.613245in}}%
\pgfpathlineto{\pgfqpoint{13.752068in}{3.605948in}}%
\pgfpathlineto{\pgfqpoint{13.754350in}{3.619448in}}%
\pgfpathlineto{\pgfqpoint{13.756632in}{3.642071in}}%
\pgfpathlineto{\pgfqpoint{13.763478in}{3.639152in}}%
\pgfpathlineto{\pgfqpoint{13.765760in}{3.652287in}}%
\pgfpathlineto{\pgfqpoint{13.768042in}{3.647544in}}%
\pgfpathlineto{\pgfqpoint{13.770324in}{3.667247in}}%
\pgfpathlineto{\pgfqpoint{13.772606in}{3.677099in}}%
\pgfpathlineto{\pgfqpoint{13.779451in}{3.676734in}}%
\pgfpathlineto{\pgfqpoint{13.781733in}{3.687316in}}%
\pgfpathlineto{\pgfqpoint{13.784015in}{3.685126in}}%
\pgfpathlineto{\pgfqpoint{13.786297in}{3.705195in}}%
\pgfpathlineto{\pgfqpoint{13.788579in}{3.697532in}}%
\pgfpathlineto{\pgfqpoint{13.797707in}{3.709938in}}%
\pgfpathlineto{\pgfqpoint{13.799989in}{3.717600in}}%
\pgfpathlineto{\pgfqpoint{13.802270in}{3.742047in}}%
\pgfpathlineto{\pgfqpoint{13.804552in}{3.754453in}}%
\pgfpathlineto{\pgfqpoint{13.813680in}{3.763940in}}%
\pgfpathlineto{\pgfqpoint{13.815962in}{3.773427in}}%
\pgfpathlineto{\pgfqpoint{13.818244in}{3.745331in}}%
\pgfpathlineto{\pgfqpoint{13.820526in}{3.761751in}}%
\pgfpathlineto{\pgfqpoint{13.827371in}{3.762845in}}%
\pgfpathlineto{\pgfqpoint{13.829653in}{3.748250in}}%
\pgfpathlineto{\pgfqpoint{13.831935in}{3.765034in}}%
\pgfpathlineto{\pgfqpoint{13.834217in}{3.759926in}}%
\pgfpathlineto{\pgfqpoint{13.836499in}{3.759926in}}%
\pgfpathlineto{\pgfqpoint{13.843345in}{3.762845in}}%
\pgfpathlineto{\pgfqpoint{13.845627in}{3.755548in}}%
\pgfpathlineto{\pgfqpoint{13.847909in}{3.752993in}}%
\pgfpathlineto{\pgfqpoint{13.850190in}{3.744966in}}%
\pgfpathlineto{\pgfqpoint{13.852472in}{3.768318in}}%
\pgfpathlineto{\pgfqpoint{13.859318in}{3.761021in}}%
\pgfpathlineto{\pgfqpoint{13.861600in}{3.727087in}}%
\pgfpathlineto{\pgfqpoint{13.863882in}{3.743507in}}%
\pgfpathlineto{\pgfqpoint{13.866164in}{3.728547in}}%
\pgfpathlineto{\pgfqpoint{13.868446in}{3.746061in}}%
\pgfpathlineto{\pgfqpoint{13.875291in}{3.717236in}}%
\pgfpathlineto{\pgfqpoint{13.877573in}{3.701181in}}%
\pgfpathlineto{\pgfqpoint{13.879855in}{3.698262in}}%
\pgfpathlineto{\pgfqpoint{13.882137in}{3.698992in}}%
\pgfpathlineto{\pgfqpoint{13.884419in}{3.689870in}}%
\pgfpathlineto{\pgfqpoint{13.891265in}{3.688410in}}%
\pgfpathlineto{\pgfqpoint{13.893547in}{3.690599in}}%
\pgfpathlineto{\pgfqpoint{13.895829in}{3.695343in}}%
\pgfpathlineto{\pgfqpoint{13.898111in}{3.696802in}}%
\pgfpathlineto{\pgfqpoint{13.900392in}{3.690235in}}%
\pgfpathlineto{\pgfqpoint{13.907238in}{3.688775in}}%
\pgfpathlineto{\pgfqpoint{13.909520in}{3.661774in}}%
\pgfpathlineto{\pgfqpoint{13.911802in}{3.674910in}}%
\pgfpathlineto{\pgfqpoint{13.914084in}{3.665423in}}%
\pgfpathlineto{\pgfqpoint{13.916366in}{3.651922in}}%
\pgfpathlineto{\pgfqpoint{13.923211in}{3.655206in}}%
\pgfpathlineto{\pgfqpoint{13.925493in}{3.657760in}}%
\pgfpathlineto{\pgfqpoint{13.927775in}{3.654477in}}%
\pgfpathlineto{\pgfqpoint{13.930057in}{3.661409in}}%
\pgfpathlineto{\pgfqpoint{13.932339in}{3.639881in}}%
\pgfpathlineto{\pgfqpoint{13.939185in}{3.653382in}}%
\pgfpathlineto{\pgfqpoint{13.941467in}{3.646084in}}%
\pgfpathlineto{\pgfqpoint{13.943749in}{3.648274in}}%
\pgfpathlineto{\pgfqpoint{13.946031in}{3.655571in}}%
\pgfpathlineto{\pgfqpoint{13.948312in}{3.665788in}}%
\pgfpathlineto{\pgfqpoint{13.959722in}{3.692059in}}%
\pgfpathlineto{\pgfqpoint{13.962004in}{3.688410in}}%
\pgfpathlineto{\pgfqpoint{13.964286in}{3.631854in}}%
\pgfpathlineto{\pgfqpoint{13.971132in}{3.655936in}}%
\pgfpathlineto{\pgfqpoint{13.973413in}{3.620178in}}%
\pgfpathlineto{\pgfqpoint{13.975695in}{3.621273in}}%
\pgfpathlineto{\pgfqpoint{13.977977in}{3.636962in}}%
\pgfpathlineto{\pgfqpoint{13.980259in}{3.633679in}}%
\pgfpathlineto{\pgfqpoint{13.987105in}{3.612151in}}%
\pgfpathlineto{\pgfqpoint{13.989387in}{3.614340in}}%
\pgfpathlineto{\pgfqpoint{13.993951in}{3.648639in}}%
\pgfpathlineto{\pgfqpoint{13.996233in}{3.655571in}}%
\pgfpathlineto{\pgfqpoint{14.003078in}{3.642436in}}%
\pgfpathlineto{\pgfqpoint{14.005360in}{3.653382in}}%
\pgfpathlineto{\pgfqpoint{14.007642in}{3.639517in}}%
\pgfpathlineto{\pgfqpoint{14.009924in}{3.641341in}}%
\pgfpathlineto{\pgfqpoint{14.012206in}{3.636962in}}%
\pgfpathlineto{\pgfqpoint{14.019052in}{3.633679in}}%
\pgfpathlineto{\pgfqpoint{14.021333in}{3.613975in}}%
\pgfpathlineto{\pgfqpoint{14.023615in}{3.600840in}}%
\pgfpathlineto{\pgfqpoint{14.025897in}{3.600475in}}%
\pgfpathlineto{\pgfqpoint{14.028179in}{3.589528in}}%
\pgfpathlineto{\pgfqpoint{14.035025in}{3.598285in}}%
\pgfpathlineto{\pgfqpoint{14.037307in}{3.588799in}}%
\pgfpathlineto{\pgfqpoint{14.039589in}{3.601934in}}%
\pgfpathlineto{\pgfqpoint{14.044153in}{3.601204in}}%
\pgfpathlineto{\pgfqpoint{14.050998in}{3.605948in}}%
\pgfpathlineto{\pgfqpoint{14.053280in}{3.600840in}}%
\pgfpathlineto{\pgfqpoint{14.055562in}{3.604488in}}%
\pgfpathlineto{\pgfqpoint{14.057844in}{3.564352in}}%
\pgfpathlineto{\pgfqpoint{14.060126in}{3.534067in}}%
\pgfpathlineto{\pgfqpoint{14.066972in}{3.534432in}}%
\pgfpathlineto{\pgfqpoint{14.069254in}{3.522391in}}%
\pgfpathlineto{\pgfqpoint{14.071535in}{3.515823in}}%
\pgfpathlineto{\pgfqpoint{14.073817in}{3.545013in}}%
\pgfpathlineto{\pgfqpoint{14.076099in}{3.534432in}}%
\pgfpathlineto{\pgfqpoint{14.082945in}{3.530783in}}%
\pgfpathlineto{\pgfqpoint{14.085227in}{3.516918in}}%
\pgfpathlineto{\pgfqpoint{14.087509in}{3.493566in}}%
\pgfpathlineto{\pgfqpoint{14.089791in}{3.491376in}}%
\pgfpathlineto{\pgfqpoint{14.092073in}{3.497944in}}%
\pgfpathlineto{\pgfqpoint{14.098918in}{3.510350in}}%
\pgfpathlineto{\pgfqpoint{14.103482in}{3.523121in}}%
\pgfpathlineto{\pgfqpoint{14.105764in}{3.485538in}}%
\pgfpathlineto{\pgfqpoint{14.108046in}{3.485538in}}%
\pgfpathlineto{\pgfqpoint{14.114892in}{3.469119in}}%
\pgfpathlineto{\pgfqpoint{14.117174in}{3.507431in}}%
\pgfpathlineto{\pgfqpoint{14.119455in}{3.525310in}}%
\pgfpathlineto{\pgfqpoint{14.121737in}{3.522391in}}%
\pgfpathlineto{\pgfqpoint{14.124019in}{3.530053in}}%
\pgfpathlineto{\pgfqpoint{14.130865in}{3.537716in}}%
\pgfpathlineto{\pgfqpoint{14.133147in}{3.575663in}}%
\pgfpathlineto{\pgfqpoint{14.135429in}{3.599380in}}%
\pgfpathlineto{\pgfqpoint{14.139993in}{3.613610in}}%
\pgfpathlineto{\pgfqpoint{14.146838in}{3.628205in}}%
\pgfpathlineto{\pgfqpoint{14.149120in}{3.623097in}}%
\pgfpathlineto{\pgfqpoint{14.151402in}{3.588799in}}%
\pgfpathlineto{\pgfqpoint{14.153684in}{3.588069in}}%
\pgfpathlineto{\pgfqpoint{14.155966in}{3.585880in}}%
\pgfpathlineto{\pgfqpoint{14.162812in}{3.584055in}}%
\pgfpathlineto{\pgfqpoint{14.165094in}{3.603759in}}%
\pgfpathlineto{\pgfqpoint{14.167376in}{3.636598in}}%
\pgfpathlineto{\pgfqpoint{14.169657in}{3.628570in}}%
\pgfpathlineto{\pgfqpoint{14.171939in}{3.640246in}}%
\pgfpathlineto{\pgfqpoint{14.178785in}{3.648639in}}%
\pgfpathlineto{\pgfqpoint{14.181067in}{3.667977in}}%
\pgfpathlineto{\pgfqpoint{14.183349in}{3.644625in}}%
\pgfpathlineto{\pgfqpoint{14.185631in}{3.650463in}}%
\pgfpathlineto{\pgfqpoint{14.187913in}{3.665058in}}%
\pgfpathlineto{\pgfqpoint{14.197040in}{3.692424in}}%
\pgfpathlineto{\pgfqpoint{14.199322in}{3.687680in}}%
\pgfpathlineto{\pgfqpoint{14.201604in}{3.709573in}}%
\pgfpathlineto{\pgfqpoint{14.203886in}{3.710668in}}%
\pgfpathlineto{\pgfqpoint{14.213014in}{3.709208in}}%
\pgfpathlineto{\pgfqpoint{14.215296in}{3.703005in}}%
\pgfpathlineto{\pgfqpoint{14.217577in}{3.712492in}}%
\pgfpathlineto{\pgfqpoint{14.219859in}{3.700816in}}%
\pgfpathlineto{\pgfqpoint{14.228987in}{3.739493in}}%
\pgfpathlineto{\pgfqpoint{14.231269in}{3.737669in}}%
\pgfpathlineto{\pgfqpoint{14.233551in}{3.741682in}}%
\pgfpathlineto{\pgfqpoint{14.235833in}{3.715411in}}%
\pgfpathlineto{\pgfqpoint{14.242678in}{3.696437in}}%
\pgfpathlineto{\pgfqpoint{14.244960in}{3.698992in}}%
\pgfpathlineto{\pgfqpoint{14.247242in}{3.689505in}}%
\pgfpathlineto{\pgfqpoint{14.249524in}{3.696437in}}%
\pgfpathlineto{\pgfqpoint{14.251806in}{3.692424in}}%
\pgfpathlineto{\pgfqpoint{14.260934in}{3.698627in}}%
\pgfpathlineto{\pgfqpoint{14.263216in}{3.682572in}}%
\pgfpathlineto{\pgfqpoint{14.265498in}{3.686221in}}%
\pgfpathlineto{\pgfqpoint{14.267779in}{3.697897in}}%
\pgfpathlineto{\pgfqpoint{14.274625in}{3.687680in}}%
\pgfpathlineto{\pgfqpoint{14.276907in}{3.613245in}}%
\pgfpathlineto{\pgfqpoint{14.279189in}{3.601934in}}%
\pgfpathlineto{\pgfqpoint{14.281471in}{3.580771in}}%
\pgfpathlineto{\pgfqpoint{14.283753in}{3.596096in}}%
\pgfpathlineto{\pgfqpoint{14.290599in}{3.588799in}}%
\pgfpathlineto{\pgfqpoint{14.292880in}{3.577123in}}%
\pgfpathlineto{\pgfqpoint{14.295162in}{3.557054in}}%
\pgfpathlineto{\pgfqpoint{14.297444in}{3.553406in}}%
\pgfpathlineto{\pgfqpoint{14.299726in}{3.563257in}}%
\pgfpathlineto{\pgfqpoint{14.306572in}{3.545013in}}%
\pgfpathlineto{\pgfqpoint{14.308854in}{3.545378in}}%
\pgfpathlineto{\pgfqpoint{14.311136in}{3.556325in}}%
\pgfpathlineto{\pgfqpoint{14.313418in}{3.570555in}}%
\pgfpathlineto{\pgfqpoint{14.315699in}{3.576028in}}%
\pgfpathlineto{\pgfqpoint{14.322545in}{3.562163in}}%
\pgfpathlineto{\pgfqpoint{14.324827in}{3.553041in}}%
\pgfpathlineto{\pgfqpoint{14.327109in}{3.546838in}}%
\pgfpathlineto{\pgfqpoint{14.329391in}{3.559244in}}%
\pgfpathlineto{\pgfqpoint{14.331673in}{3.582961in}}%
\pgfpathlineto{\pgfqpoint{14.340800in}{3.590623in}}%
\pgfpathlineto{\pgfqpoint{14.343082in}{3.598650in}}%
\pgfpathlineto{\pgfqpoint{14.345364in}{3.619448in}}%
\pgfpathlineto{\pgfqpoint{14.347646in}{3.628935in}}%
\pgfpathlineto{\pgfqpoint{14.354492in}{3.607407in}}%
\pgfpathlineto{\pgfqpoint{14.356774in}{3.597191in}}%
\pgfpathlineto{\pgfqpoint{14.361338in}{3.608502in}}%
\pgfpathlineto{\pgfqpoint{14.363620in}{3.612151in}}%
\pgfpathlineto{\pgfqpoint{14.370465in}{3.610326in}}%
\pgfpathlineto{\pgfqpoint{14.372747in}{3.590988in}}%
\pgfpathlineto{\pgfqpoint{14.375029in}{3.581866in}}%
\pgfpathlineto{\pgfqpoint{14.377311in}{3.585880in}}%
\pgfpathlineto{\pgfqpoint{14.379593in}{3.588069in}}%
\pgfpathlineto{\pgfqpoint{14.386439in}{3.592083in}}%
\pgfpathlineto{\pgfqpoint{14.388720in}{3.588434in}}%
\pgfpathlineto{\pgfqpoint{14.391002in}{3.613975in}}%
\pgfpathlineto{\pgfqpoint{14.393284in}{3.610691in}}%
\pgfpathlineto{\pgfqpoint{14.395566in}{3.622002in}}%
\pgfpathlineto{\pgfqpoint{14.402412in}{3.617259in}}%
\pgfpathlineto{\pgfqpoint{14.404694in}{3.614340in}}%
\pgfpathlineto{\pgfqpoint{14.406976in}{3.599745in}}%
\pgfpathlineto{\pgfqpoint{14.409258in}{3.597556in}}%
\pgfpathlineto{\pgfqpoint{14.411540in}{3.599015in}}%
\pgfpathlineto{\pgfqpoint{14.418385in}{3.581136in}}%
\pgfpathlineto{\pgfqpoint{14.420667in}{3.586609in}}%
\pgfpathlineto{\pgfqpoint{14.422949in}{3.581136in}}%
\pgfpathlineto{\pgfqpoint{14.425231in}{3.578582in}}%
\pgfpathlineto{\pgfqpoint{14.427513in}{3.568730in}}%
\pgfpathlineto{\pgfqpoint{14.436641in}{3.586974in}}%
\pgfpathlineto{\pgfqpoint{14.438922in}{3.577487in}}%
\pgfpathlineto{\pgfqpoint{14.441204in}{3.577123in}}%
\pgfpathlineto{\pgfqpoint{14.443486in}{3.584420in}}%
\pgfpathlineto{\pgfqpoint{14.450332in}{3.580771in}}%
\pgfpathlineto{\pgfqpoint{14.452614in}{3.585880in}}%
\pgfpathlineto{\pgfqpoint{14.454896in}{3.593177in}}%
\pgfpathlineto{\pgfqpoint{14.457178in}{3.583325in}}%
\pgfpathlineto{\pgfqpoint{14.466305in}{3.589528in}}%
\pgfpathlineto{\pgfqpoint{14.468587in}{3.603029in}}%
\pgfpathlineto{\pgfqpoint{14.470869in}{3.593907in}}%
\pgfpathlineto{\pgfqpoint{14.473151in}{3.576393in}}%
\pgfpathlineto{\pgfqpoint{14.475433in}{3.538081in}}%
\pgfpathlineto{\pgfqpoint{14.482279in}{3.531513in}}%
\pgfpathlineto{\pgfqpoint{14.484561in}{3.519837in}}%
\pgfpathlineto{\pgfqpoint{14.486842in}{3.541729in}}%
\pgfpathlineto{\pgfqpoint{14.491406in}{3.493931in}}%
\pgfpathlineto{\pgfqpoint{14.498252in}{3.492836in}}%
\pgfpathlineto{\pgfqpoint{14.500534in}{3.493931in}}%
\pgfpathlineto{\pgfqpoint{14.502816in}{3.501958in}}%
\pgfpathlineto{\pgfqpoint{14.505098in}{3.492836in}}%
\pgfpathlineto{\pgfqpoint{14.507380in}{3.519837in}}%
\pgfpathlineto{\pgfqpoint{14.514225in}{3.517648in}}%
\pgfpathlineto{\pgfqpoint{14.516507in}{3.510715in}}%
\pgfpathlineto{\pgfqpoint{14.518789in}{3.509620in}}%
\pgfpathlineto{\pgfqpoint{14.521071in}{3.497579in}}%
\pgfpathlineto{\pgfqpoint{14.523353in}{3.491741in}}%
\pgfpathlineto{\pgfqpoint{14.530199in}{3.476416in}}%
\pgfpathlineto{\pgfqpoint{14.532481in}{3.474227in}}%
\pgfpathlineto{\pgfqpoint{14.534763in}{3.446861in}}%
\pgfpathlineto{\pgfqpoint{14.537044in}{3.465470in}}%
\pgfpathlineto{\pgfqpoint{14.539326in}{3.477876in}}%
\pgfpathlineto{\pgfqpoint{14.546172in}{3.479700in}}%
\pgfpathlineto{\pgfqpoint{14.548454in}{3.479700in}}%
\pgfpathlineto{\pgfqpoint{14.550736in}{3.465470in}}%
\pgfpathlineto{\pgfqpoint{14.553018in}{3.474227in}}%
\pgfpathlineto{\pgfqpoint{14.555300in}{3.474592in}}%
\pgfpathlineto{\pgfqpoint{14.564427in}{3.503417in}}%
\pgfpathlineto{\pgfqpoint{14.566709in}{3.518012in}}%
\pgfpathlineto{\pgfqpoint{14.568991in}{3.513634in}}%
\pgfpathlineto{\pgfqpoint{14.571273in}{3.511445in}}%
\pgfpathlineto{\pgfqpoint{14.578119in}{3.509255in}}%
\pgfpathlineto{\pgfqpoint{14.582683in}{3.513269in}}%
\pgfpathlineto{\pgfqpoint{14.584964in}{3.503052in}}%
\pgfpathlineto{\pgfqpoint{14.587246in}{3.520567in}}%
\pgfpathlineto{\pgfqpoint{14.594092in}{3.536256in}}%
\pgfpathlineto{\pgfqpoint{14.596374in}{3.512174in}}%
\pgfpathlineto{\pgfqpoint{14.598656in}{3.519837in}}%
\pgfpathlineto{\pgfqpoint{14.600938in}{3.518012in}}%
\pgfpathlineto{\pgfqpoint{14.603220in}{3.517648in}}%
\pgfpathlineto{\pgfqpoint{14.610065in}{3.515823in}}%
\pgfpathlineto{\pgfqpoint{14.614629in}{3.477511in}}%
\pgfpathlineto{\pgfqpoint{14.616911in}{3.477876in}}%
\pgfpathlineto{\pgfqpoint{14.619193in}{3.476781in}}%
\pgfpathlineto{\pgfqpoint{14.626039in}{3.488822in}}%
\pgfpathlineto{\pgfqpoint{14.628321in}{3.458902in}}%
\pgfpathlineto{\pgfqpoint{14.630603in}{3.458902in}}%
\pgfpathlineto{\pgfqpoint{14.632885in}{3.444672in}}%
\pgfpathlineto{\pgfqpoint{14.635166in}{3.452699in}}%
\pgfpathlineto{\pgfqpoint{14.642012in}{3.464011in}}%
\pgfpathlineto{\pgfqpoint{14.646576in}{3.452334in}}%
\pgfpathlineto{\pgfqpoint{14.648858in}{3.434091in}}%
\pgfpathlineto{\pgfqpoint{14.651140in}{3.432631in}}%
\pgfpathlineto{\pgfqpoint{14.657986in}{3.423509in}}%
\pgfpathlineto{\pgfqpoint{14.660267in}{3.412928in}}%
\pgfpathlineto{\pgfqpoint{14.664831in}{3.432996in}}%
\pgfpathlineto{\pgfqpoint{14.667113in}{3.435550in}}%
\pgfpathlineto{\pgfqpoint{14.673959in}{3.438834in}}%
\pgfpathlineto{\pgfqpoint{14.676241in}{3.428617in}}%
\pgfpathlineto{\pgfqpoint{14.678523in}{3.431536in}}%
\pgfpathlineto{\pgfqpoint{14.680805in}{3.457443in}}%
\pgfpathlineto{\pgfqpoint{14.683086in}{3.457808in}}%
\pgfpathlineto{\pgfqpoint{14.689932in}{3.440294in}}%
\pgfpathlineto{\pgfqpoint{14.692214in}{3.449415in}}%
\pgfpathlineto{\pgfqpoint{14.694496in}{3.463281in}}%
\pgfpathlineto{\pgfqpoint{14.696778in}{3.577123in}}%
\pgfpathlineto{\pgfqpoint{14.699060in}{3.581501in}}%
\pgfpathlineto{\pgfqpoint{14.705906in}{3.596826in}}%
\pgfpathlineto{\pgfqpoint{14.708187in}{3.613245in}}%
\pgfpathlineto{\pgfqpoint{14.710469in}{3.590623in}}%
\pgfpathlineto{\pgfqpoint{14.715033in}{3.613975in}}%
\pgfpathlineto{\pgfqpoint{14.721879in}{3.612151in}}%
\pgfpathlineto{\pgfqpoint{14.724161in}{3.603394in}}%
\pgfpathlineto{\pgfqpoint{14.726443in}{3.590988in}}%
\pgfpathlineto{\pgfqpoint{14.728725in}{3.584055in}}%
\pgfpathlineto{\pgfqpoint{14.731007in}{3.585515in}}%
\pgfpathlineto{\pgfqpoint{14.737852in}{3.609597in}}%
\pgfpathlineto{\pgfqpoint{14.740134in}{3.599380in}}%
\pgfpathlineto{\pgfqpoint{14.742416in}{3.597191in}}%
\pgfpathlineto{\pgfqpoint{14.744698in}{3.580042in}}%
\pgfpathlineto{\pgfqpoint{14.746980in}{3.573109in}}%
\pgfpathlineto{\pgfqpoint{14.753826in}{3.588069in}}%
\pgfpathlineto{\pgfqpoint{14.756107in}{3.595366in}}%
\pgfpathlineto{\pgfqpoint{14.758389in}{3.592812in}}%
\pgfpathlineto{\pgfqpoint{14.760671in}{3.593907in}}%
\pgfpathlineto{\pgfqpoint{14.762953in}{3.606313in}}%
\pgfpathlineto{\pgfqpoint{14.769799in}{3.603759in}}%
\pgfpathlineto{\pgfqpoint{14.772081in}{3.600475in}}%
\pgfpathlineto{\pgfqpoint{14.774363in}{3.587704in}}%
\pgfpathlineto{\pgfqpoint{14.776645in}{3.582596in}}%
\pgfpathlineto{\pgfqpoint{14.778927in}{3.580771in}}%
\pgfpathlineto{\pgfqpoint{14.788054in}{3.562163in}}%
\pgfpathlineto{\pgfqpoint{14.790336in}{3.547203in}}%
\pgfpathlineto{\pgfqpoint{14.792618in}{3.523486in}}%
\pgfpathlineto{\pgfqpoint{14.794900in}{3.520567in}}%
\pgfpathlineto{\pgfqpoint{14.801746in}{3.526769in}}%
\pgfpathlineto{\pgfqpoint{14.806309in}{3.558514in}}%
\pgfpathlineto{\pgfqpoint{14.808591in}{3.556325in}}%
\pgfpathlineto{\pgfqpoint{14.810873in}{3.578947in}}%
\pgfpathlineto{\pgfqpoint{14.817719in}{3.586609in}}%
\pgfpathlineto{\pgfqpoint{14.820001in}{3.628205in}}%
\pgfpathlineto{\pgfqpoint{14.822283in}{3.632949in}}%
\pgfpathlineto{\pgfqpoint{14.824565in}{3.615070in}}%
\pgfpathlineto{\pgfqpoint{14.826847in}{3.646814in}}%
\pgfpathlineto{\pgfqpoint{14.833692in}{3.646814in}}%
\pgfpathlineto{\pgfqpoint{14.835974in}{3.634043in}}%
\pgfpathlineto{\pgfqpoint{14.838256in}{3.634043in}}%
\pgfpathlineto{\pgfqpoint{14.840538in}{3.630760in}}%
\pgfpathlineto{\pgfqpoint{14.842820in}{3.633314in}}%
\pgfpathlineto{\pgfqpoint{14.849666in}{3.628935in}}%
\pgfpathlineto{\pgfqpoint{14.851948in}{3.645355in}}%
\pgfpathlineto{\pgfqpoint{14.854229in}{3.646814in}}%
\pgfpathlineto{\pgfqpoint{14.856511in}{3.642436in}}%
\pgfpathlineto{\pgfqpoint{14.858793in}{3.630030in}}%
\pgfpathlineto{\pgfqpoint{14.867921in}{3.641341in}}%
\pgfpathlineto{\pgfqpoint{14.870203in}{3.631854in}}%
\pgfpathlineto{\pgfqpoint{14.874767in}{3.597921in}}%
\pgfpathlineto{\pgfqpoint{14.881612in}{3.605948in}}%
\pgfpathlineto{\pgfqpoint{14.886176in}{3.624557in}}%
\pgfpathlineto{\pgfqpoint{14.888458in}{3.643530in}}%
\pgfpathlineto{\pgfqpoint{14.890740in}{3.654477in}}%
\pgfpathlineto{\pgfqpoint{14.897586in}{3.636233in}}%
\pgfpathlineto{\pgfqpoint{14.899868in}{3.634408in}}%
\pgfpathlineto{\pgfqpoint{14.902150in}{3.624192in}}%
\pgfpathlineto{\pgfqpoint{14.904431in}{3.632949in}}%
\pgfpathlineto{\pgfqpoint{14.906713in}{3.632219in}}%
\pgfpathlineto{\pgfqpoint{14.913559in}{3.597191in}}%
\pgfpathlineto{\pgfqpoint{14.915841in}{3.598285in}}%
\pgfpathlineto{\pgfqpoint{14.918123in}{3.597191in}}%
\pgfpathlineto{\pgfqpoint{14.920405in}{3.584420in}}%
\pgfpathlineto{\pgfqpoint{14.922687in}{3.583325in}}%
\pgfpathlineto{\pgfqpoint{14.929532in}{3.519472in}}%
\pgfpathlineto{\pgfqpoint{14.931814in}{3.520567in}}%
\pgfpathlineto{\pgfqpoint{14.934096in}{3.516918in}}%
\pgfpathlineto{\pgfqpoint{14.936378in}{3.503782in}}%
\pgfpathlineto{\pgfqpoint{14.938660in}{3.497579in}}%
\pgfpathlineto{\pgfqpoint{14.945506in}{3.492836in}}%
\pgfpathlineto{\pgfqpoint{14.947788in}{3.474957in}}%
\pgfpathlineto{\pgfqpoint{14.950070in}{3.471308in}}%
\pgfpathlineto{\pgfqpoint{14.954633in}{3.515458in}}%
\pgfpathlineto{\pgfqpoint{14.961479in}{3.542094in}}%
\pgfpathlineto{\pgfqpoint{14.963761in}{3.541365in}}%
\pgfpathlineto{\pgfqpoint{14.966043in}{3.572379in}}%
\pgfpathlineto{\pgfqpoint{14.970607in}{3.569460in}}%
\pgfpathlineto{\pgfqpoint{14.977452in}{3.591353in}}%
\pgfpathlineto{\pgfqpoint{14.984298in}{3.700451in}}%
\pgfpathlineto{\pgfqpoint{14.986580in}{3.712492in}}%
\pgfpathlineto{\pgfqpoint{14.993426in}{3.728547in}}%
\pgfpathlineto{\pgfqpoint{14.995708in}{3.701181in}}%
\pgfpathlineto{\pgfqpoint{15.000272in}{3.684397in}}%
\pgfpathlineto{\pgfqpoint{15.002553in}{3.707019in}}%
\pgfpathlineto{\pgfqpoint{15.009399in}{3.732560in}}%
\pgfpathlineto{\pgfqpoint{15.011681in}{3.778170in}}%
\pgfpathlineto{\pgfqpoint{15.013963in}{3.767953in}}%
\pgfpathlineto{\pgfqpoint{15.016245in}{3.749345in}}%
\pgfpathlineto{\pgfqpoint{15.018527in}{3.760291in}}%
\pgfpathlineto{\pgfqpoint{15.025373in}{3.779994in}}%
\pgfpathlineto{\pgfqpoint{15.027654in}{3.765764in}}%
\pgfpathlineto{\pgfqpoint{15.029936in}{3.763940in}}%
\pgfpathlineto{\pgfqpoint{15.034500in}{3.778170in}}%
\pgfpathlineto{\pgfqpoint{15.043628in}{3.778900in}}%
\pgfpathlineto{\pgfqpoint{15.045910in}{3.781089in}}%
\pgfpathlineto{\pgfqpoint{15.048192in}{3.786197in}}%
\pgfpathlineto{\pgfqpoint{15.050473in}{3.769413in}}%
\pgfpathlineto{\pgfqpoint{15.050473in}{3.769413in}}%
\pgfusepath{stroke}%
\end{pgfscope}%
\begin{pgfscope}%
\pgfsetrectcap%
\pgfsetmiterjoin%
\pgfsetlinewidth{0.803000pt}%
\definecolor{currentstroke}{rgb}{1.000000,1.000000,1.000000}%
\pgfsetstrokecolor{currentstroke}%
\pgfsetdash{}{0pt}%
\pgfpathmoveto{\pgfqpoint{9.810417in}{2.907317in}}%
\pgfpathlineto{\pgfqpoint{9.810417in}{3.828049in}}%
\pgfusepath{stroke}%
\end{pgfscope}%
\begin{pgfscope}%
\pgfsetrectcap%
\pgfsetmiterjoin%
\pgfsetlinewidth{0.803000pt}%
\definecolor{currentstroke}{rgb}{1.000000,1.000000,1.000000}%
\pgfsetstrokecolor{currentstroke}%
\pgfsetdash{}{0pt}%
\pgfpathmoveto{\pgfqpoint{15.300000in}{2.907317in}}%
\pgfpathlineto{\pgfqpoint{15.300000in}{3.828049in}}%
\pgfusepath{stroke}%
\end{pgfscope}%
\begin{pgfscope}%
\pgfsetrectcap%
\pgfsetmiterjoin%
\pgfsetlinewidth{0.803000pt}%
\definecolor{currentstroke}{rgb}{1.000000,1.000000,1.000000}%
\pgfsetstrokecolor{currentstroke}%
\pgfsetdash{}{0pt}%
\pgfpathmoveto{\pgfqpoint{9.810417in}{2.907317in}}%
\pgfpathlineto{\pgfqpoint{15.300000in}{2.907317in}}%
\pgfusepath{stroke}%
\end{pgfscope}%
\begin{pgfscope}%
\pgfsetrectcap%
\pgfsetmiterjoin%
\pgfsetlinewidth{0.803000pt}%
\definecolor{currentstroke}{rgb}{1.000000,1.000000,1.000000}%
\pgfsetstrokecolor{currentstroke}%
\pgfsetdash{}{0pt}%
\pgfpathmoveto{\pgfqpoint{9.810417in}{3.828049in}}%
\pgfpathlineto{\pgfqpoint{15.300000in}{3.828049in}}%
\pgfusepath{stroke}%
\end{pgfscope}%
\begin{pgfscope}%
\definecolor{textcolor}{rgb}{0.150000,0.150000,0.150000}%
\pgfsetstrokecolor{textcolor}%
\pgfsetfillcolor{textcolor}%
\pgftext[x=12.555208in,y=3.911382in,,base]{\color{textcolor}\rmfamily\fontsize{12.000000}{14.400000}\selectfont VZ}%
\end{pgfscope}%
\begin{pgfscope}%
\pgfsetbuttcap%
\pgfsetmiterjoin%
\definecolor{currentfill}{rgb}{0.917647,0.917647,0.949020}%
\pgfsetfillcolor{currentfill}%
\pgfsetlinewidth{0.000000pt}%
\definecolor{currentstroke}{rgb}{0.000000,0.000000,0.000000}%
\pgfsetstrokecolor{currentstroke}%
\pgfsetstrokeopacity{0.000000}%
\pgfsetdash{}{0pt}%
\pgfpathmoveto{\pgfqpoint{2.125000in}{1.250000in}}%
\pgfpathlineto{\pgfqpoint{7.614583in}{1.250000in}}%
\pgfpathlineto{\pgfqpoint{7.614583in}{2.170732in}}%
\pgfpathlineto{\pgfqpoint{2.125000in}{2.170732in}}%
\pgfpathclose%
\pgfusepath{fill}%
\end{pgfscope}%
\begin{pgfscope}%
\pgfpathrectangle{\pgfqpoint{2.125000in}{1.250000in}}{\pgfqpoint{5.489583in}{0.920732in}}%
\pgfusepath{clip}%
\pgfsetroundcap%
\pgfsetroundjoin%
\pgfsetlinewidth{0.803000pt}%
\definecolor{currentstroke}{rgb}{1.000000,1.000000,1.000000}%
\pgfsetstrokecolor{currentstroke}%
\pgfsetdash{}{0pt}%
\pgfpathmoveto{\pgfqpoint{2.369963in}{1.250000in}}%
\pgfpathlineto{\pgfqpoint{2.369963in}{2.170732in}}%
\pgfusepath{stroke}%
\end{pgfscope}%
\begin{pgfscope}%
\definecolor{textcolor}{rgb}{0.150000,0.150000,0.150000}%
\pgfsetstrokecolor{textcolor}%
\pgfsetfillcolor{textcolor}%
\pgftext[x=2.369963in,y=1.152778in,,top]{\color{textcolor}\rmfamily\fontsize{10.000000}{12.000000}\selectfont 2012}%
\end{pgfscope}%
\begin{pgfscope}%
\pgfpathrectangle{\pgfqpoint{2.125000in}{1.250000in}}{\pgfqpoint{5.489583in}{0.920732in}}%
\pgfusepath{clip}%
\pgfsetroundcap%
\pgfsetroundjoin%
\pgfsetlinewidth{0.803000pt}%
\definecolor{currentstroke}{rgb}{1.000000,1.000000,1.000000}%
\pgfsetstrokecolor{currentstroke}%
\pgfsetdash{}{0pt}%
\pgfpathmoveto{\pgfqpoint{3.205141in}{1.250000in}}%
\pgfpathlineto{\pgfqpoint{3.205141in}{2.170732in}}%
\pgfusepath{stroke}%
\end{pgfscope}%
\begin{pgfscope}%
\definecolor{textcolor}{rgb}{0.150000,0.150000,0.150000}%
\pgfsetstrokecolor{textcolor}%
\pgfsetfillcolor{textcolor}%
\pgftext[x=3.205141in,y=1.152778in,,top]{\color{textcolor}\rmfamily\fontsize{10.000000}{12.000000}\selectfont 2013}%
\end{pgfscope}%
\begin{pgfscope}%
\pgfpathrectangle{\pgfqpoint{2.125000in}{1.250000in}}{\pgfqpoint{5.489583in}{0.920732in}}%
\pgfusepath{clip}%
\pgfsetroundcap%
\pgfsetroundjoin%
\pgfsetlinewidth{0.803000pt}%
\definecolor{currentstroke}{rgb}{1.000000,1.000000,1.000000}%
\pgfsetstrokecolor{currentstroke}%
\pgfsetdash{}{0pt}%
\pgfpathmoveto{\pgfqpoint{4.038037in}{1.250000in}}%
\pgfpathlineto{\pgfqpoint{4.038037in}{2.170732in}}%
\pgfusepath{stroke}%
\end{pgfscope}%
\begin{pgfscope}%
\definecolor{textcolor}{rgb}{0.150000,0.150000,0.150000}%
\pgfsetstrokecolor{textcolor}%
\pgfsetfillcolor{textcolor}%
\pgftext[x=4.038037in,y=1.152778in,,top]{\color{textcolor}\rmfamily\fontsize{10.000000}{12.000000}\selectfont 2014}%
\end{pgfscope}%
\begin{pgfscope}%
\pgfpathrectangle{\pgfqpoint{2.125000in}{1.250000in}}{\pgfqpoint{5.489583in}{0.920732in}}%
\pgfusepath{clip}%
\pgfsetroundcap%
\pgfsetroundjoin%
\pgfsetlinewidth{0.803000pt}%
\definecolor{currentstroke}{rgb}{1.000000,1.000000,1.000000}%
\pgfsetstrokecolor{currentstroke}%
\pgfsetdash{}{0pt}%
\pgfpathmoveto{\pgfqpoint{4.870933in}{1.250000in}}%
\pgfpathlineto{\pgfqpoint{4.870933in}{2.170732in}}%
\pgfusepath{stroke}%
\end{pgfscope}%
\begin{pgfscope}%
\definecolor{textcolor}{rgb}{0.150000,0.150000,0.150000}%
\pgfsetstrokecolor{textcolor}%
\pgfsetfillcolor{textcolor}%
\pgftext[x=4.870933in,y=1.152778in,,top]{\color{textcolor}\rmfamily\fontsize{10.000000}{12.000000}\selectfont 2015}%
\end{pgfscope}%
\begin{pgfscope}%
\pgfpathrectangle{\pgfqpoint{2.125000in}{1.250000in}}{\pgfqpoint{5.489583in}{0.920732in}}%
\pgfusepath{clip}%
\pgfsetroundcap%
\pgfsetroundjoin%
\pgfsetlinewidth{0.803000pt}%
\definecolor{currentstroke}{rgb}{1.000000,1.000000,1.000000}%
\pgfsetstrokecolor{currentstroke}%
\pgfsetdash{}{0pt}%
\pgfpathmoveto{\pgfqpoint{5.703829in}{1.250000in}}%
\pgfpathlineto{\pgfqpoint{5.703829in}{2.170732in}}%
\pgfusepath{stroke}%
\end{pgfscope}%
\begin{pgfscope}%
\definecolor{textcolor}{rgb}{0.150000,0.150000,0.150000}%
\pgfsetstrokecolor{textcolor}%
\pgfsetfillcolor{textcolor}%
\pgftext[x=5.703829in,y=1.152778in,,top]{\color{textcolor}\rmfamily\fontsize{10.000000}{12.000000}\selectfont 2016}%
\end{pgfscope}%
\begin{pgfscope}%
\pgfpathrectangle{\pgfqpoint{2.125000in}{1.250000in}}{\pgfqpoint{5.489583in}{0.920732in}}%
\pgfusepath{clip}%
\pgfsetroundcap%
\pgfsetroundjoin%
\pgfsetlinewidth{0.803000pt}%
\definecolor{currentstroke}{rgb}{1.000000,1.000000,1.000000}%
\pgfsetstrokecolor{currentstroke}%
\pgfsetdash{}{0pt}%
\pgfpathmoveto{\pgfqpoint{6.539007in}{1.250000in}}%
\pgfpathlineto{\pgfqpoint{6.539007in}{2.170732in}}%
\pgfusepath{stroke}%
\end{pgfscope}%
\begin{pgfscope}%
\definecolor{textcolor}{rgb}{0.150000,0.150000,0.150000}%
\pgfsetstrokecolor{textcolor}%
\pgfsetfillcolor{textcolor}%
\pgftext[x=6.539007in,y=1.152778in,,top]{\color{textcolor}\rmfamily\fontsize{10.000000}{12.000000}\selectfont 2017}%
\end{pgfscope}%
\begin{pgfscope}%
\pgfpathrectangle{\pgfqpoint{2.125000in}{1.250000in}}{\pgfqpoint{5.489583in}{0.920732in}}%
\pgfusepath{clip}%
\pgfsetroundcap%
\pgfsetroundjoin%
\pgfsetlinewidth{0.803000pt}%
\definecolor{currentstroke}{rgb}{1.000000,1.000000,1.000000}%
\pgfsetstrokecolor{currentstroke}%
\pgfsetdash{}{0pt}%
\pgfpathmoveto{\pgfqpoint{7.371903in}{1.250000in}}%
\pgfpathlineto{\pgfqpoint{7.371903in}{2.170732in}}%
\pgfusepath{stroke}%
\end{pgfscope}%
\begin{pgfscope}%
\definecolor{textcolor}{rgb}{0.150000,0.150000,0.150000}%
\pgfsetstrokecolor{textcolor}%
\pgfsetfillcolor{textcolor}%
\pgftext[x=7.371903in,y=1.152778in,,top]{\color{textcolor}\rmfamily\fontsize{10.000000}{12.000000}\selectfont 2018}%
\end{pgfscope}%
\begin{pgfscope}%
\pgfpathrectangle{\pgfqpoint{2.125000in}{1.250000in}}{\pgfqpoint{5.489583in}{0.920732in}}%
\pgfusepath{clip}%
\pgfsetroundcap%
\pgfsetroundjoin%
\pgfsetlinewidth{0.803000pt}%
\definecolor{currentstroke}{rgb}{1.000000,1.000000,1.000000}%
\pgfsetstrokecolor{currentstroke}%
\pgfsetdash{}{0pt}%
\pgfpathmoveto{\pgfqpoint{2.125000in}{1.550553in}}%
\pgfpathlineto{\pgfqpoint{7.614583in}{1.550553in}}%
\pgfusepath{stroke}%
\end{pgfscope}%
\begin{pgfscope}%
\definecolor{textcolor}{rgb}{0.150000,0.150000,0.150000}%
\pgfsetstrokecolor{textcolor}%
\pgfsetfillcolor{textcolor}%
\pgftext[x=1.851047in,y=1.497791in,left,base]{\color{textcolor}\rmfamily\fontsize{10.000000}{12.000000}\selectfont 50}%
\end{pgfscope}%
\begin{pgfscope}%
\pgfpathrectangle{\pgfqpoint{2.125000in}{1.250000in}}{\pgfqpoint{5.489583in}{0.920732in}}%
\pgfusepath{clip}%
\pgfsetroundcap%
\pgfsetroundjoin%
\pgfsetlinewidth{0.803000pt}%
\definecolor{currentstroke}{rgb}{1.000000,1.000000,1.000000}%
\pgfsetstrokecolor{currentstroke}%
\pgfsetdash{}{0pt}%
\pgfpathmoveto{\pgfqpoint{2.125000in}{2.007946in}}%
\pgfpathlineto{\pgfqpoint{7.614583in}{2.007946in}}%
\pgfusepath{stroke}%
\end{pgfscope}%
\begin{pgfscope}%
\definecolor{textcolor}{rgb}{0.150000,0.150000,0.150000}%
\pgfsetstrokecolor{textcolor}%
\pgfsetfillcolor{textcolor}%
\pgftext[x=1.762682in,y=1.955184in,left,base]{\color{textcolor}\rmfamily\fontsize{10.000000}{12.000000}\selectfont 100}%
\end{pgfscope}%
\begin{pgfscope}%
\pgfpathrectangle{\pgfqpoint{2.125000in}{1.250000in}}{\pgfqpoint{5.489583in}{0.920732in}}%
\pgfusepath{clip}%
\pgfsetroundcap%
\pgfsetroundjoin%
\pgfsetlinewidth{1.505625pt}%
\definecolor{currentstroke}{rgb}{0.121569,0.466667,0.705882}%
\pgfsetstrokecolor{currentstroke}%
\pgfsetdash{}{0pt}%
\pgfpathmoveto{\pgfqpoint{2.374527in}{1.299627in}}%
\pgfpathlineto{\pgfqpoint{2.376808in}{1.295876in}}%
\pgfpathlineto{\pgfqpoint{2.379090in}{1.297432in}}%
\pgfpathlineto{\pgfqpoint{2.381372in}{1.295053in}}%
\pgfpathlineto{\pgfqpoint{2.392782in}{1.291851in}}%
\pgfpathlineto{\pgfqpoint{2.395064in}{1.296334in}}%
\pgfpathlineto{\pgfqpoint{2.397346in}{1.295053in}}%
\pgfpathlineto{\pgfqpoint{2.406473in}{1.298712in}}%
\pgfpathlineto{\pgfqpoint{2.408755in}{1.301182in}}%
\pgfpathlineto{\pgfqpoint{2.413319in}{1.294779in}}%
\pgfpathlineto{\pgfqpoint{2.420165in}{1.292766in}}%
\pgfpathlineto{\pgfqpoint{2.422447in}{1.295602in}}%
\pgfpathlineto{\pgfqpoint{2.424728in}{1.294687in}}%
\pgfpathlineto{\pgfqpoint{2.429292in}{1.295694in}}%
\pgfpathlineto{\pgfqpoint{2.436138in}{1.293407in}}%
\pgfpathlineto{\pgfqpoint{2.438420in}{1.294870in}}%
\pgfpathlineto{\pgfqpoint{2.440702in}{1.298621in}}%
\pgfpathlineto{\pgfqpoint{2.442984in}{1.305756in}}%
\pgfpathlineto{\pgfqpoint{2.445266in}{1.307677in}}%
\pgfpathlineto{\pgfqpoint{2.452111in}{1.308318in}}%
\pgfpathlineto{\pgfqpoint{2.454393in}{1.307586in}}%
\pgfpathlineto{\pgfqpoint{2.456675in}{1.310330in}}%
\pgfpathlineto{\pgfqpoint{2.458957in}{1.318472in}}%
\pgfpathlineto{\pgfqpoint{2.461239in}{1.321490in}}%
\pgfpathlineto{\pgfqpoint{2.468085in}{1.319112in}}%
\pgfpathlineto{\pgfqpoint{2.470367in}{1.324143in}}%
\pgfpathlineto{\pgfqpoint{2.472649in}{1.325607in}}%
\pgfpathlineto{\pgfqpoint{2.474930in}{1.323046in}}%
\pgfpathlineto{\pgfqpoint{2.477212in}{1.325424in}}%
\pgfpathlineto{\pgfqpoint{2.486340in}{1.323869in}}%
\pgfpathlineto{\pgfqpoint{2.488622in}{1.328168in}}%
\pgfpathlineto{\pgfqpoint{2.490904in}{1.328443in}}%
\pgfpathlineto{\pgfqpoint{2.493186in}{1.330547in}}%
\pgfpathlineto{\pgfqpoint{2.500031in}{1.329175in}}%
\pgfpathlineto{\pgfqpoint{2.502313in}{1.333383in}}%
\pgfpathlineto{\pgfqpoint{2.504595in}{1.328168in}}%
\pgfpathlineto{\pgfqpoint{2.506877in}{1.329815in}}%
\pgfpathlineto{\pgfqpoint{2.509159in}{1.327803in}}%
\pgfpathlineto{\pgfqpoint{2.516005in}{1.327985in}}%
\pgfpathlineto{\pgfqpoint{2.518287in}{1.325150in}}%
\pgfpathlineto{\pgfqpoint{2.520569in}{1.326705in}}%
\pgfpathlineto{\pgfqpoint{2.522850in}{1.331462in}}%
\pgfpathlineto{\pgfqpoint{2.525132in}{1.329815in}}%
\pgfpathlineto{\pgfqpoint{2.531978in}{1.328534in}}%
\pgfpathlineto{\pgfqpoint{2.534260in}{1.329998in}}%
\pgfpathlineto{\pgfqpoint{2.536542in}{1.328992in}}%
\pgfpathlineto{\pgfqpoint{2.538824in}{1.329449in}}%
\pgfpathlineto{\pgfqpoint{2.541106in}{1.328809in}}%
\pgfpathlineto{\pgfqpoint{2.547951in}{1.333200in}}%
\pgfpathlineto{\pgfqpoint{2.550233in}{1.328534in}}%
\pgfpathlineto{\pgfqpoint{2.552515in}{1.328992in}}%
\pgfpathlineto{\pgfqpoint{2.554797in}{1.330089in}}%
\pgfpathlineto{\pgfqpoint{2.557079in}{1.333108in}}%
\pgfpathlineto{\pgfqpoint{2.563925in}{1.335670in}}%
\pgfpathlineto{\pgfqpoint{2.570771in}{1.333566in}}%
\pgfpathlineto{\pgfqpoint{2.573052in}{1.331462in}}%
\pgfpathlineto{\pgfqpoint{2.579898in}{1.333474in}}%
\pgfpathlineto{\pgfqpoint{2.582180in}{1.336310in}}%
\pgfpathlineto{\pgfqpoint{2.584462in}{1.333474in}}%
\pgfpathlineto{\pgfqpoint{2.586744in}{1.337591in}}%
\pgfpathlineto{\pgfqpoint{2.595871in}{1.334572in}}%
\pgfpathlineto{\pgfqpoint{2.598153in}{1.328992in}}%
\pgfpathlineto{\pgfqpoint{2.600435in}{1.330272in}}%
\pgfpathlineto{\pgfqpoint{2.602717in}{1.337591in}}%
\pgfpathlineto{\pgfqpoint{2.604999in}{1.341890in}}%
\pgfpathlineto{\pgfqpoint{2.611845in}{1.337225in}}%
\pgfpathlineto{\pgfqpoint{2.614127in}{1.339786in}}%
\pgfpathlineto{\pgfqpoint{2.620972in}{1.337591in}}%
\pgfpathlineto{\pgfqpoint{2.627818in}{1.331736in}}%
\pgfpathlineto{\pgfqpoint{2.630100in}{1.333383in}}%
\pgfpathlineto{\pgfqpoint{2.632382in}{1.339146in}}%
\pgfpathlineto{\pgfqpoint{2.634664in}{1.341982in}}%
\pgfpathlineto{\pgfqpoint{2.636946in}{1.342622in}}%
\pgfpathlineto{\pgfqpoint{2.643792in}{1.341524in}}%
\pgfpathlineto{\pgfqpoint{2.646073in}{1.341799in}}%
\pgfpathlineto{\pgfqpoint{2.648355in}{1.339969in}}%
\pgfpathlineto{\pgfqpoint{2.650637in}{1.328260in}}%
\pgfpathlineto{\pgfqpoint{2.652919in}{1.331096in}}%
\pgfpathlineto{\pgfqpoint{2.659765in}{1.332559in}}%
\pgfpathlineto{\pgfqpoint{2.662047in}{1.331828in}}%
\pgfpathlineto{\pgfqpoint{2.664329in}{1.330364in}}%
\pgfpathlineto{\pgfqpoint{2.666611in}{1.331828in}}%
\pgfpathlineto{\pgfqpoint{2.668893in}{1.331370in}}%
\pgfpathlineto{\pgfqpoint{2.675738in}{1.328900in}}%
\pgfpathlineto{\pgfqpoint{2.678020in}{1.328717in}}%
\pgfpathlineto{\pgfqpoint{2.680302in}{1.332376in}}%
\pgfpathlineto{\pgfqpoint{2.684866in}{1.322405in}}%
\pgfpathlineto{\pgfqpoint{2.691712in}{1.329724in}}%
\pgfpathlineto{\pgfqpoint{2.693993in}{1.334115in}}%
\pgfpathlineto{\pgfqpoint{2.698557in}{1.336950in}}%
\pgfpathlineto{\pgfqpoint{2.700839in}{1.336127in}}%
\pgfpathlineto{\pgfqpoint{2.709967in}{1.337957in}}%
\pgfpathlineto{\pgfqpoint{2.716813in}{1.321582in}}%
\pgfpathlineto{\pgfqpoint{2.723658in}{1.325516in}}%
\pgfpathlineto{\pgfqpoint{2.725940in}{1.324784in}}%
\pgfpathlineto{\pgfqpoint{2.728222in}{1.329907in}}%
\pgfpathlineto{\pgfqpoint{2.730504in}{1.330913in}}%
\pgfpathlineto{\pgfqpoint{2.732786in}{1.330455in}}%
\pgfpathlineto{\pgfqpoint{2.739632in}{1.329998in}}%
\pgfpathlineto{\pgfqpoint{2.741914in}{1.331645in}}%
\pgfpathlineto{\pgfqpoint{2.744195in}{1.326979in}}%
\pgfpathlineto{\pgfqpoint{2.748759in}{1.334115in}}%
\pgfpathlineto{\pgfqpoint{2.755605in}{1.337591in}}%
\pgfpathlineto{\pgfqpoint{2.760169in}{1.342622in}}%
\pgfpathlineto{\pgfqpoint{2.762451in}{1.336310in}}%
\pgfpathlineto{\pgfqpoint{2.764733in}{1.347470in}}%
\pgfpathlineto{\pgfqpoint{2.771578in}{1.339969in}}%
\pgfpathlineto{\pgfqpoint{2.773860in}{1.343811in}}%
\pgfpathlineto{\pgfqpoint{2.776142in}{1.344360in}}%
\pgfpathlineto{\pgfqpoint{2.778424in}{1.340518in}}%
\pgfpathlineto{\pgfqpoint{2.780706in}{1.344818in}}%
\pgfpathlineto{\pgfqpoint{2.787552in}{1.350855in}}%
\pgfpathlineto{\pgfqpoint{2.789834in}{1.350306in}}%
\pgfpathlineto{\pgfqpoint{2.794397in}{1.351678in}}%
\pgfpathlineto{\pgfqpoint{2.796679in}{1.348111in}}%
\pgfpathlineto{\pgfqpoint{2.803525in}{1.344818in}}%
\pgfpathlineto{\pgfqpoint{2.808089in}{1.338871in}}%
\pgfpathlineto{\pgfqpoint{2.812653in}{1.345732in}}%
\pgfpathlineto{\pgfqpoint{2.821780in}{1.354148in}}%
\pgfpathlineto{\pgfqpoint{2.824062in}{1.353325in}}%
\pgfpathlineto{\pgfqpoint{2.826344in}{1.348385in}}%
\pgfpathlineto{\pgfqpoint{2.828626in}{1.349117in}}%
\pgfpathlineto{\pgfqpoint{2.835472in}{1.346464in}}%
\pgfpathlineto{\pgfqpoint{2.837754in}{1.342805in}}%
\pgfpathlineto{\pgfqpoint{2.840036in}{1.341890in}}%
\pgfpathlineto{\pgfqpoint{2.842317in}{1.351130in}}%
\pgfpathlineto{\pgfqpoint{2.844599in}{1.355978in}}%
\pgfpathlineto{\pgfqpoint{2.851445in}{1.359454in}}%
\pgfpathlineto{\pgfqpoint{2.856009in}{1.352593in}}%
\pgfpathlineto{\pgfqpoint{2.860573in}{1.359729in}}%
\pgfpathlineto{\pgfqpoint{2.867418in}{1.359820in}}%
\pgfpathlineto{\pgfqpoint{2.869700in}{1.358539in}}%
\pgfpathlineto{\pgfqpoint{2.871982in}{1.360094in}}%
\pgfpathlineto{\pgfqpoint{2.874264in}{1.354697in}}%
\pgfpathlineto{\pgfqpoint{2.876546in}{1.355886in}}%
\pgfpathlineto{\pgfqpoint{2.883392in}{1.354331in}}%
\pgfpathlineto{\pgfqpoint{2.885674in}{1.357533in}}%
\pgfpathlineto{\pgfqpoint{2.887956in}{1.357625in}}%
\pgfpathlineto{\pgfqpoint{2.890237in}{1.358905in}}%
\pgfpathlineto{\pgfqpoint{2.892519in}{1.356893in}}%
\pgfpathlineto{\pgfqpoint{2.899365in}{1.355795in}}%
\pgfpathlineto{\pgfqpoint{2.901647in}{1.354423in}}%
\pgfpathlineto{\pgfqpoint{2.903929in}{1.355612in}}%
\pgfpathlineto{\pgfqpoint{2.906211in}{1.353691in}}%
\pgfpathlineto{\pgfqpoint{2.908493in}{1.352685in}}%
\pgfpathlineto{\pgfqpoint{2.915338in}{1.354789in}}%
\pgfpathlineto{\pgfqpoint{2.917620in}{1.354240in}}%
\pgfpathlineto{\pgfqpoint{2.919902in}{1.355521in}}%
\pgfpathlineto{\pgfqpoint{2.922184in}{1.352776in}}%
\pgfpathlineto{\pgfqpoint{2.924466in}{1.355978in}}%
\pgfpathlineto{\pgfqpoint{2.933594in}{1.356527in}}%
\pgfpathlineto{\pgfqpoint{2.935876in}{1.354606in}}%
\pgfpathlineto{\pgfqpoint{2.938158in}{1.358539in}}%
\pgfpathlineto{\pgfqpoint{2.940439in}{1.358905in}}%
\pgfpathlineto{\pgfqpoint{2.947285in}{1.356801in}}%
\pgfpathlineto{\pgfqpoint{2.949567in}{1.362564in}}%
\pgfpathlineto{\pgfqpoint{2.954131in}{1.369791in}}%
\pgfpathlineto{\pgfqpoint{2.956413in}{1.368236in}}%
\pgfpathlineto{\pgfqpoint{2.965540in}{1.367047in}}%
\pgfpathlineto{\pgfqpoint{2.967822in}{1.369334in}}%
\pgfpathlineto{\pgfqpoint{2.970104in}{1.368968in}}%
\pgfpathlineto{\pgfqpoint{2.972386in}{1.369791in}}%
\pgfpathlineto{\pgfqpoint{2.979232in}{1.367230in}}%
\pgfpathlineto{\pgfqpoint{2.981514in}{1.368876in}}%
\pgfpathlineto{\pgfqpoint{2.983796in}{1.364668in}}%
\pgfpathlineto{\pgfqpoint{2.986078in}{1.367596in}}%
\pgfpathlineto{\pgfqpoint{2.988359in}{1.368328in}}%
\pgfpathlineto{\pgfqpoint{2.995205in}{1.373359in}}%
\pgfpathlineto{\pgfqpoint{2.997487in}{1.371804in}}%
\pgfpathlineto{\pgfqpoint{2.999769in}{1.377018in}}%
\pgfpathlineto{\pgfqpoint{3.002051in}{1.379396in}}%
\pgfpathlineto{\pgfqpoint{3.004333in}{1.380586in}}%
\pgfpathlineto{\pgfqpoint{3.011179in}{1.377384in}}%
\pgfpathlineto{\pgfqpoint{3.013460in}{1.373359in}}%
\pgfpathlineto{\pgfqpoint{3.015742in}{1.374731in}}%
\pgfpathlineto{\pgfqpoint{3.018024in}{1.378116in}}%
\pgfpathlineto{\pgfqpoint{3.020306in}{1.378207in}}%
\pgfpathlineto{\pgfqpoint{3.027152in}{1.379854in}}%
\pgfpathlineto{\pgfqpoint{3.031716in}{1.385251in}}%
\pgfpathlineto{\pgfqpoint{3.033998in}{1.383879in}}%
\pgfpathlineto{\pgfqpoint{3.036280in}{1.379945in}}%
\pgfpathlineto{\pgfqpoint{3.043125in}{1.378024in}}%
\pgfpathlineto{\pgfqpoint{3.045407in}{1.373176in}}%
\pgfpathlineto{\pgfqpoint{3.047689in}{1.372810in}}%
\pgfpathlineto{\pgfqpoint{3.052253in}{1.376561in}}%
\pgfpathlineto{\pgfqpoint{3.063662in}{1.377475in}}%
\pgfpathlineto{\pgfqpoint{3.065944in}{1.387995in}}%
\pgfpathlineto{\pgfqpoint{3.068226in}{1.386989in}}%
\pgfpathlineto{\pgfqpoint{3.075072in}{1.383147in}}%
\pgfpathlineto{\pgfqpoint{3.077354in}{1.387904in}}%
\pgfpathlineto{\pgfqpoint{3.079636in}{1.385251in}}%
\pgfpathlineto{\pgfqpoint{3.081918in}{1.384245in}}%
\pgfpathlineto{\pgfqpoint{3.084200in}{1.386074in}}%
\pgfpathlineto{\pgfqpoint{3.091045in}{1.386715in}}%
\pgfpathlineto{\pgfqpoint{3.093327in}{1.385068in}}%
\pgfpathlineto{\pgfqpoint{3.095609in}{1.382598in}}%
\pgfpathlineto{\pgfqpoint{3.097891in}{1.382690in}}%
\pgfpathlineto{\pgfqpoint{3.100173in}{1.388361in}}%
\pgfpathlineto{\pgfqpoint{3.109301in}{1.395863in}}%
\pgfpathlineto{\pgfqpoint{3.111582in}{1.396503in}}%
\pgfpathlineto{\pgfqpoint{3.116146in}{1.399522in}}%
\pgfpathlineto{\pgfqpoint{3.125274in}{1.396777in}}%
\pgfpathlineto{\pgfqpoint{3.127556in}{1.397784in}}%
\pgfpathlineto{\pgfqpoint{3.129838in}{1.399613in}}%
\pgfpathlineto{\pgfqpoint{3.132120in}{1.402815in}}%
\pgfpathlineto{\pgfqpoint{3.138965in}{1.400619in}}%
\pgfpathlineto{\pgfqpoint{3.141247in}{1.398424in}}%
\pgfpathlineto{\pgfqpoint{3.145811in}{1.400254in}}%
\pgfpathlineto{\pgfqpoint{3.157221in}{1.401168in}}%
\pgfpathlineto{\pgfqpoint{3.159502in}{1.398973in}}%
\pgfpathlineto{\pgfqpoint{3.164066in}{1.396777in}}%
\pgfpathlineto{\pgfqpoint{3.170912in}{1.401626in}}%
\pgfpathlineto{\pgfqpoint{3.173194in}{1.404919in}}%
\pgfpathlineto{\pgfqpoint{3.175476in}{1.401626in}}%
\pgfpathlineto{\pgfqpoint{3.177758in}{1.408487in}}%
\pgfpathlineto{\pgfqpoint{3.180040in}{1.405010in}}%
\pgfpathlineto{\pgfqpoint{3.186885in}{1.405834in}}%
\pgfpathlineto{\pgfqpoint{3.191449in}{1.402906in}}%
\pgfpathlineto{\pgfqpoint{3.193731in}{1.402175in}}%
\pgfpathlineto{\pgfqpoint{3.196013in}{1.400619in}}%
\pgfpathlineto{\pgfqpoint{3.202859in}{1.406657in}}%
\pgfpathlineto{\pgfqpoint{3.207423in}{1.414524in}}%
\pgfpathlineto{\pgfqpoint{3.209704in}{1.414799in}}%
\pgfpathlineto{\pgfqpoint{3.211986in}{1.417360in}}%
\pgfpathlineto{\pgfqpoint{3.218832in}{1.419738in}}%
\pgfpathlineto{\pgfqpoint{3.221114in}{1.422757in}}%
\pgfpathlineto{\pgfqpoint{3.223396in}{1.427789in}}%
\pgfpathlineto{\pgfqpoint{3.225678in}{1.425136in}}%
\pgfpathlineto{\pgfqpoint{3.227960in}{1.426508in}}%
\pgfpathlineto{\pgfqpoint{3.241651in}{1.423855in}}%
\pgfpathlineto{\pgfqpoint{3.243933in}{1.420470in}}%
\pgfpathlineto{\pgfqpoint{3.259906in}{1.423764in}}%
\pgfpathlineto{\pgfqpoint{3.266752in}{1.416537in}}%
\pgfpathlineto{\pgfqpoint{3.269034in}{1.417177in}}%
\pgfpathlineto{\pgfqpoint{3.271316in}{1.413792in}}%
\pgfpathlineto{\pgfqpoint{3.273598in}{1.419738in}}%
\pgfpathlineto{\pgfqpoint{3.275880in}{1.421111in}}%
\pgfpathlineto{\pgfqpoint{3.282725in}{1.417177in}}%
\pgfpathlineto{\pgfqpoint{3.285007in}{1.423123in}}%
\pgfpathlineto{\pgfqpoint{3.287289in}{1.425776in}}%
\pgfpathlineto{\pgfqpoint{3.289571in}{1.418000in}}%
\pgfpathlineto{\pgfqpoint{3.291853in}{1.419556in}}%
\pgfpathlineto{\pgfqpoint{3.298699in}{1.416537in}}%
\pgfpathlineto{\pgfqpoint{3.300981in}{1.417452in}}%
\pgfpathlineto{\pgfqpoint{3.303263in}{1.415988in}}%
\pgfpathlineto{\pgfqpoint{3.305545in}{1.418549in}}%
\pgfpathlineto{\pgfqpoint{3.307826in}{1.422666in}}%
\pgfpathlineto{\pgfqpoint{3.316954in}{1.421934in}}%
\pgfpathlineto{\pgfqpoint{3.319236in}{1.417360in}}%
\pgfpathlineto{\pgfqpoint{3.323800in}{1.425685in}}%
\pgfpathlineto{\pgfqpoint{3.330645in}{1.418275in}}%
\pgfpathlineto{\pgfqpoint{3.335209in}{1.426599in}}%
\pgfpathlineto{\pgfqpoint{3.337491in}{1.424038in}}%
\pgfpathlineto{\pgfqpoint{3.339773in}{1.422849in}}%
\pgfpathlineto{\pgfqpoint{3.346619in}{1.425868in}}%
\pgfpathlineto{\pgfqpoint{3.351183in}{1.429069in}}%
\pgfpathlineto{\pgfqpoint{3.353465in}{1.427972in}}%
\pgfpathlineto{\pgfqpoint{3.362592in}{1.429344in}}%
\pgfpathlineto{\pgfqpoint{3.364874in}{1.426508in}}%
\pgfpathlineto{\pgfqpoint{3.367156in}{1.425593in}}%
\pgfpathlineto{\pgfqpoint{3.369438in}{1.428246in}}%
\pgfpathlineto{\pgfqpoint{3.371720in}{1.423855in}}%
\pgfpathlineto{\pgfqpoint{3.378566in}{1.422849in}}%
\pgfpathlineto{\pgfqpoint{3.380847in}{1.418549in}}%
\pgfpathlineto{\pgfqpoint{3.383129in}{1.425502in}}%
\pgfpathlineto{\pgfqpoint{3.385411in}{1.422300in}}%
\pgfpathlineto{\pgfqpoint{3.387693in}{1.427057in}}%
\pgfpathlineto{\pgfqpoint{3.394539in}{1.435198in}}%
\pgfpathlineto{\pgfqpoint{3.396821in}{1.442425in}}%
\pgfpathlineto{\pgfqpoint{3.401385in}{1.447456in}}%
\pgfpathlineto{\pgfqpoint{3.410512in}{1.442242in}}%
\pgfpathlineto{\pgfqpoint{3.412794in}{1.443431in}}%
\pgfpathlineto{\pgfqpoint{3.415076in}{1.436845in}}%
\pgfpathlineto{\pgfqpoint{3.417358in}{1.440413in}}%
\pgfpathlineto{\pgfqpoint{3.419640in}{1.437668in}}%
\pgfpathlineto{\pgfqpoint{3.426486in}{1.440047in}}%
\pgfpathlineto{\pgfqpoint{3.428767in}{1.436845in}}%
\pgfpathlineto{\pgfqpoint{3.431049in}{1.441510in}}%
\pgfpathlineto{\pgfqpoint{3.433331in}{1.442791in}}%
\pgfpathlineto{\pgfqpoint{3.435613in}{1.438857in}}%
\pgfpathlineto{\pgfqpoint{3.442459in}{1.429618in}}%
\pgfpathlineto{\pgfqpoint{3.444741in}{1.436753in}}%
\pgfpathlineto{\pgfqpoint{3.447023in}{1.431539in}}%
\pgfpathlineto{\pgfqpoint{3.449305in}{1.429801in}}%
\pgfpathlineto{\pgfqpoint{3.451587in}{1.435107in}}%
\pgfpathlineto{\pgfqpoint{3.458432in}{1.434284in}}%
\pgfpathlineto{\pgfqpoint{3.460714in}{1.438675in}}%
\pgfpathlineto{\pgfqpoint{3.465278in}{1.445170in}}%
\pgfpathlineto{\pgfqpoint{3.467560in}{1.441968in}}%
\pgfpathlineto{\pgfqpoint{3.474406in}{1.443340in}}%
\pgfpathlineto{\pgfqpoint{3.476688in}{1.444529in}}%
\pgfpathlineto{\pgfqpoint{3.478969in}{1.439406in}}%
\pgfpathlineto{\pgfqpoint{3.481251in}{1.458983in}}%
\pgfpathlineto{\pgfqpoint{3.483533in}{1.467673in}}%
\pgfpathlineto{\pgfqpoint{3.490379in}{1.466301in}}%
\pgfpathlineto{\pgfqpoint{3.492661in}{1.468131in}}%
\pgfpathlineto{\pgfqpoint{3.497225in}{1.465661in}}%
\pgfpathlineto{\pgfqpoint{3.499507in}{1.466027in}}%
\pgfpathlineto{\pgfqpoint{3.506352in}{1.466393in}}%
\pgfpathlineto{\pgfqpoint{3.508634in}{1.469320in}}%
\pgfpathlineto{\pgfqpoint{3.510916in}{1.475083in}}%
\pgfpathlineto{\pgfqpoint{3.513198in}{1.470966in}}%
\pgfpathlineto{\pgfqpoint{3.515480in}{1.480938in}}%
\pgfpathlineto{\pgfqpoint{3.522326in}{1.474717in}}%
\pgfpathlineto{\pgfqpoint{3.524608in}{1.474626in}}%
\pgfpathlineto{\pgfqpoint{3.526889in}{1.471698in}}%
\pgfpathlineto{\pgfqpoint{3.529171in}{1.467399in}}%
\pgfpathlineto{\pgfqpoint{3.531453in}{1.472339in}}%
\pgfpathlineto{\pgfqpoint{3.540581in}{1.471424in}}%
\pgfpathlineto{\pgfqpoint{3.542863in}{1.467490in}}%
\pgfpathlineto{\pgfqpoint{3.545145in}{1.473528in}}%
\pgfpathlineto{\pgfqpoint{3.547427in}{1.467490in}}%
\pgfpathlineto{\pgfqpoint{3.554272in}{1.471790in}}%
\pgfpathlineto{\pgfqpoint{3.556554in}{1.471790in}}%
\pgfpathlineto{\pgfqpoint{3.558836in}{1.465569in}}%
\pgfpathlineto{\pgfqpoint{3.561118in}{1.470052in}}%
\pgfpathlineto{\pgfqpoint{3.563400in}{1.471241in}}%
\pgfpathlineto{\pgfqpoint{3.570246in}{1.476272in}}%
\pgfpathlineto{\pgfqpoint{3.572528in}{1.470692in}}%
\pgfpathlineto{\pgfqpoint{3.574810in}{1.469594in}}%
\pgfpathlineto{\pgfqpoint{3.577091in}{1.476089in}}%
\pgfpathlineto{\pgfqpoint{3.579373in}{1.473345in}}%
\pgfpathlineto{\pgfqpoint{3.586219in}{1.476272in}}%
\pgfpathlineto{\pgfqpoint{3.588501in}{1.479840in}}%
\pgfpathlineto{\pgfqpoint{3.590783in}{1.476913in}}%
\pgfpathlineto{\pgfqpoint{3.593065in}{1.468131in}}%
\pgfpathlineto{\pgfqpoint{3.595347in}{1.470326in}}%
\pgfpathlineto{\pgfqpoint{3.602192in}{1.467765in}}%
\pgfpathlineto{\pgfqpoint{3.609038in}{1.480297in}}%
\pgfpathlineto{\pgfqpoint{3.611320in}{1.477187in}}%
\pgfpathlineto{\pgfqpoint{3.618166in}{1.482310in}}%
\pgfpathlineto{\pgfqpoint{3.620448in}{1.482401in}}%
\pgfpathlineto{\pgfqpoint{3.627293in}{1.494019in}}%
\pgfpathlineto{\pgfqpoint{3.638703in}{1.485420in}}%
\pgfpathlineto{\pgfqpoint{3.640985in}{1.492281in}}%
\pgfpathlineto{\pgfqpoint{3.643267in}{1.493928in}}%
\pgfpathlineto{\pgfqpoint{3.650112in}{1.492921in}}%
\pgfpathlineto{\pgfqpoint{3.652394in}{1.491092in}}%
\pgfpathlineto{\pgfqpoint{3.654676in}{1.491641in}}%
\pgfpathlineto{\pgfqpoint{3.656958in}{1.494476in}}%
\pgfpathlineto{\pgfqpoint{3.659240in}{1.492189in}}%
\pgfpathlineto{\pgfqpoint{3.666086in}{1.494934in}}%
\pgfpathlineto{\pgfqpoint{3.668368in}{1.489445in}}%
\pgfpathlineto{\pgfqpoint{3.670650in}{1.485603in}}%
\pgfpathlineto{\pgfqpoint{3.672932in}{1.502069in}}%
\pgfpathlineto{\pgfqpoint{3.675213in}{1.499142in}}%
\pgfpathlineto{\pgfqpoint{3.684341in}{1.495391in}}%
\pgfpathlineto{\pgfqpoint{3.686623in}{1.465112in}}%
\pgfpathlineto{\pgfqpoint{3.688905in}{1.469594in}}%
\pgfpathlineto{\pgfqpoint{3.691187in}{1.479748in}}%
\pgfpathlineto{\pgfqpoint{3.698032in}{1.480938in}}%
\pgfpathlineto{\pgfqpoint{3.702596in}{1.474260in}}%
\pgfpathlineto{\pgfqpoint{3.704878in}{1.472979in}}%
\pgfpathlineto{\pgfqpoint{3.707160in}{1.470966in}}%
\pgfpathlineto{\pgfqpoint{3.714006in}{1.470875in}}%
\pgfpathlineto{\pgfqpoint{3.716288in}{1.469777in}}%
\pgfpathlineto{\pgfqpoint{3.718570in}{1.470692in}}%
\pgfpathlineto{\pgfqpoint{3.720852in}{1.461361in}}%
\pgfpathlineto{\pgfqpoint{3.723133in}{1.459623in}}%
\pgfpathlineto{\pgfqpoint{3.729979in}{1.463557in}}%
\pgfpathlineto{\pgfqpoint{3.732261in}{1.459806in}}%
\pgfpathlineto{\pgfqpoint{3.734543in}{1.470784in}}%
\pgfpathlineto{\pgfqpoint{3.736825in}{1.471881in}}%
\pgfpathlineto{\pgfqpoint{3.739107in}{1.472247in}}%
\pgfpathlineto{\pgfqpoint{3.745953in}{1.463557in}}%
\pgfpathlineto{\pgfqpoint{3.748234in}{1.461819in}}%
\pgfpathlineto{\pgfqpoint{3.750516in}{1.464014in}}%
\pgfpathlineto{\pgfqpoint{3.752798in}{1.464380in}}%
\pgfpathlineto{\pgfqpoint{3.755080in}{1.462367in}}%
\pgfpathlineto{\pgfqpoint{3.764208in}{1.467856in}}%
\pgfpathlineto{\pgfqpoint{3.766490in}{1.466118in}}%
\pgfpathlineto{\pgfqpoint{3.768772in}{1.466118in}}%
\pgfpathlineto{\pgfqpoint{3.771054in}{1.467124in}}%
\pgfpathlineto{\pgfqpoint{3.777899in}{1.471149in}}%
\pgfpathlineto{\pgfqpoint{3.780181in}{1.483865in}}%
\pgfpathlineto{\pgfqpoint{3.782463in}{1.487524in}}%
\pgfpathlineto{\pgfqpoint{3.784745in}{1.484871in}}%
\pgfpathlineto{\pgfqpoint{3.787027in}{1.493196in}}%
\pgfpathlineto{\pgfqpoint{3.793873in}{1.494019in}}%
\pgfpathlineto{\pgfqpoint{3.798436in}{1.503075in}}%
\pgfpathlineto{\pgfqpoint{3.800718in}{1.505362in}}%
\pgfpathlineto{\pgfqpoint{3.803000in}{1.514053in}}%
\pgfpathlineto{\pgfqpoint{3.809846in}{1.508564in}}%
\pgfpathlineto{\pgfqpoint{3.812128in}{1.502435in}}%
\pgfpathlineto{\pgfqpoint{3.814410in}{1.498684in}}%
\pgfpathlineto{\pgfqpoint{3.816692in}{1.502892in}}%
\pgfpathlineto{\pgfqpoint{3.818974in}{1.501795in}}%
\pgfpathlineto{\pgfqpoint{3.825819in}{1.497678in}}%
\pgfpathlineto{\pgfqpoint{3.828101in}{1.502161in}}%
\pgfpathlineto{\pgfqpoint{3.830383in}{1.499233in}}%
\pgfpathlineto{\pgfqpoint{3.832665in}{1.492464in}}%
\pgfpathlineto{\pgfqpoint{3.834947in}{1.496398in}}%
\pgfpathlineto{\pgfqpoint{3.841793in}{1.487616in}}%
\pgfpathlineto{\pgfqpoint{3.844075in}{1.479565in}}%
\pgfpathlineto{\pgfqpoint{3.846356in}{1.482310in}}%
\pgfpathlineto{\pgfqpoint{3.848638in}{1.493287in}}%
\pgfpathlineto{\pgfqpoint{3.850920in}{1.499965in}}%
\pgfpathlineto{\pgfqpoint{3.857766in}{1.502618in}}%
\pgfpathlineto{\pgfqpoint{3.860048in}{1.498227in}}%
\pgfpathlineto{\pgfqpoint{3.862330in}{1.507283in}}%
\pgfpathlineto{\pgfqpoint{3.866894in}{1.517438in}}%
\pgfpathlineto{\pgfqpoint{3.876021in}{1.516431in}}%
\pgfpathlineto{\pgfqpoint{3.878303in}{1.514144in}}%
\pgfpathlineto{\pgfqpoint{3.880585in}{1.522652in}}%
\pgfpathlineto{\pgfqpoint{3.882867in}{1.523018in}}%
\pgfpathlineto{\pgfqpoint{3.889713in}{1.523018in}}%
\pgfpathlineto{\pgfqpoint{3.891995in}{1.525488in}}%
\pgfpathlineto{\pgfqpoint{3.894276in}{1.524573in}}%
\pgfpathlineto{\pgfqpoint{3.896558in}{1.509479in}}%
\pgfpathlineto{\pgfqpoint{3.898840in}{1.514785in}}%
\pgfpathlineto{\pgfqpoint{3.905686in}{1.508930in}}%
\pgfpathlineto{\pgfqpoint{3.907968in}{1.510943in}}%
\pgfpathlineto{\pgfqpoint{3.910250in}{1.514419in}}%
\pgfpathlineto{\pgfqpoint{3.912532in}{1.508290in}}%
\pgfpathlineto{\pgfqpoint{3.914814in}{1.512864in}}%
\pgfpathlineto{\pgfqpoint{3.921659in}{1.514968in}}%
\pgfpathlineto{\pgfqpoint{3.923941in}{1.512864in}}%
\pgfpathlineto{\pgfqpoint{3.926223in}{1.519450in}}%
\pgfpathlineto{\pgfqpoint{3.928505in}{1.520273in}}%
\pgfpathlineto{\pgfqpoint{3.930787in}{1.524207in}}%
\pgfpathlineto{\pgfqpoint{3.937633in}{1.519725in}}%
\pgfpathlineto{\pgfqpoint{3.939915in}{1.514419in}}%
\pgfpathlineto{\pgfqpoint{3.942197in}{1.515974in}}%
\pgfpathlineto{\pgfqpoint{3.944478in}{1.523384in}}%
\pgfpathlineto{\pgfqpoint{3.946760in}{1.524481in}}%
\pgfpathlineto{\pgfqpoint{3.953606in}{1.524116in}}%
\pgfpathlineto{\pgfqpoint{3.955888in}{1.527500in}}%
\pgfpathlineto{\pgfqpoint{3.958170in}{1.528506in}}%
\pgfpathlineto{\pgfqpoint{3.962734in}{1.527317in}}%
\pgfpathlineto{\pgfqpoint{3.969579in}{1.530885in}}%
\pgfpathlineto{\pgfqpoint{3.971861in}{1.523658in}}%
\pgfpathlineto{\pgfqpoint{3.974143in}{1.525854in}}%
\pgfpathlineto{\pgfqpoint{3.976425in}{1.523567in}}%
\pgfpathlineto{\pgfqpoint{3.978707in}{1.523933in}}%
\pgfpathlineto{\pgfqpoint{3.985553in}{1.523384in}}%
\pgfpathlineto{\pgfqpoint{3.987835in}{1.518718in}}%
\pgfpathlineto{\pgfqpoint{3.990117in}{1.532074in}}%
\pgfpathlineto{\pgfqpoint{3.992398in}{1.527317in}}%
\pgfpathlineto{\pgfqpoint{3.994680in}{1.535642in}}%
\pgfpathlineto{\pgfqpoint{4.001526in}{1.536465in}}%
\pgfpathlineto{\pgfqpoint{4.003808in}{1.548266in}}%
\pgfpathlineto{\pgfqpoint{4.006090in}{1.552748in}}%
\pgfpathlineto{\pgfqpoint{4.008372in}{1.554303in}}%
\pgfpathlineto{\pgfqpoint{4.010654in}{1.554029in}}%
\pgfpathlineto{\pgfqpoint{4.024345in}{1.563085in}}%
\pgfpathlineto{\pgfqpoint{4.026627in}{1.561896in}}%
\pgfpathlineto{\pgfqpoint{4.033473in}{1.564549in}}%
\pgfpathlineto{\pgfqpoint{4.035755in}{1.568391in}}%
\pgfpathlineto{\pgfqpoint{4.040319in}{1.564823in}}%
\pgfpathlineto{\pgfqpoint{4.042600in}{1.565098in}}%
\pgfpathlineto{\pgfqpoint{4.049446in}{1.562262in}}%
\pgfpathlineto{\pgfqpoint{4.051728in}{1.565830in}}%
\pgfpathlineto{\pgfqpoint{4.054010in}{1.567385in}}%
\pgfpathlineto{\pgfqpoint{4.056292in}{1.566745in}}%
\pgfpathlineto{\pgfqpoint{4.065419in}{1.560341in}}%
\pgfpathlineto{\pgfqpoint{4.067701in}{1.568300in}}%
\pgfpathlineto{\pgfqpoint{4.069983in}{1.570678in}}%
\pgfpathlineto{\pgfqpoint{4.072265in}{1.566379in}}%
\pgfpathlineto{\pgfqpoint{4.074547in}{1.588608in}}%
\pgfpathlineto{\pgfqpoint{4.083675in}{1.588059in}}%
\pgfpathlineto{\pgfqpoint{4.085957in}{1.590346in}}%
\pgfpathlineto{\pgfqpoint{4.088239in}{1.580283in}}%
\pgfpathlineto{\pgfqpoint{4.090520in}{1.565281in}}%
\pgfpathlineto{\pgfqpoint{4.097366in}{1.554578in}}%
\pgfpathlineto{\pgfqpoint{4.099648in}{1.564732in}}%
\pgfpathlineto{\pgfqpoint{4.101930in}{1.556499in}}%
\pgfpathlineto{\pgfqpoint{4.104212in}{1.564549in}}%
\pgfpathlineto{\pgfqpoint{4.106494in}{1.552931in}}%
\pgfpathlineto{\pgfqpoint{4.113340in}{1.548723in}}%
\pgfpathlineto{\pgfqpoint{4.117903in}{1.553297in}}%
\pgfpathlineto{\pgfqpoint{4.122467in}{1.566470in}}%
\pgfpathlineto{\pgfqpoint{4.129313in}{1.563817in}}%
\pgfpathlineto{\pgfqpoint{4.131595in}{1.567659in}}%
\pgfpathlineto{\pgfqpoint{4.133877in}{1.574978in}}%
\pgfpathlineto{\pgfqpoint{4.136159in}{1.574795in}}%
\pgfpathlineto{\pgfqpoint{4.138441in}{1.578911in}}%
\pgfpathlineto{\pgfqpoint{4.147568in}{1.579094in}}%
\pgfpathlineto{\pgfqpoint{4.149850in}{1.574429in}}%
\pgfpathlineto{\pgfqpoint{4.152132in}{1.573514in}}%
\pgfpathlineto{\pgfqpoint{4.154414in}{1.573239in}}%
\pgfpathlineto{\pgfqpoint{4.163541in}{1.581290in}}%
\pgfpathlineto{\pgfqpoint{4.165823in}{1.579186in}}%
\pgfpathlineto{\pgfqpoint{4.168105in}{1.579552in}}%
\pgfpathlineto{\pgfqpoint{4.170387in}{1.578820in}}%
\pgfpathlineto{\pgfqpoint{4.177233in}{1.569123in}}%
\pgfpathlineto{\pgfqpoint{4.179515in}{1.577905in}}%
\pgfpathlineto{\pgfqpoint{4.181797in}{1.572050in}}%
\pgfpathlineto{\pgfqpoint{4.184079in}{1.574337in}}%
\pgfpathlineto{\pgfqpoint{4.186361in}{1.577996in}}%
\pgfpathlineto{\pgfqpoint{4.193206in}{1.577905in}}%
\pgfpathlineto{\pgfqpoint{4.195488in}{1.581473in}}%
\pgfpathlineto{\pgfqpoint{4.197770in}{1.579186in}}%
\pgfpathlineto{\pgfqpoint{4.200052in}{1.567751in}}%
\pgfpathlineto{\pgfqpoint{4.202334in}{1.567659in}}%
\pgfpathlineto{\pgfqpoint{4.209180in}{1.574154in}}%
\pgfpathlineto{\pgfqpoint{4.211462in}{1.579826in}}%
\pgfpathlineto{\pgfqpoint{4.216025in}{1.569946in}}%
\pgfpathlineto{\pgfqpoint{4.218307in}{1.573331in}}%
\pgfpathlineto{\pgfqpoint{4.225153in}{1.567659in}}%
\pgfpathlineto{\pgfqpoint{4.229717in}{1.556773in}}%
\pgfpathlineto{\pgfqpoint{4.231999in}{1.556956in}}%
\pgfpathlineto{\pgfqpoint{4.234281in}{1.549089in}}%
\pgfpathlineto{\pgfqpoint{4.241126in}{1.557139in}}%
\pgfpathlineto{\pgfqpoint{4.243408in}{1.554669in}}%
\pgfpathlineto{\pgfqpoint{4.245690in}{1.554578in}}%
\pgfpathlineto{\pgfqpoint{4.247972in}{1.555310in}}%
\pgfpathlineto{\pgfqpoint{4.250254in}{1.539575in}}%
\pgfpathlineto{\pgfqpoint{4.257100in}{1.530428in}}%
\pgfpathlineto{\pgfqpoint{4.259382in}{1.528598in}}%
\pgfpathlineto{\pgfqpoint{4.261663in}{1.539301in}}%
\pgfpathlineto{\pgfqpoint{4.266227in}{1.515791in}}%
\pgfpathlineto{\pgfqpoint{4.273073in}{1.525213in}}%
\pgfpathlineto{\pgfqpoint{4.275355in}{1.531800in}}%
\pgfpathlineto{\pgfqpoint{4.277637in}{1.543143in}}%
\pgfpathlineto{\pgfqpoint{4.279919in}{1.540124in}}%
\pgfpathlineto{\pgfqpoint{4.289046in}{1.542686in}}%
\pgfpathlineto{\pgfqpoint{4.291328in}{1.544515in}}%
\pgfpathlineto{\pgfqpoint{4.293610in}{1.542045in}}%
\pgfpathlineto{\pgfqpoint{4.295892in}{1.543235in}}%
\pgfpathlineto{\pgfqpoint{4.298174in}{1.520731in}}%
\pgfpathlineto{\pgfqpoint{4.305020in}{1.526128in}}%
\pgfpathlineto{\pgfqpoint{4.307302in}{1.528781in}}%
\pgfpathlineto{\pgfqpoint{4.309584in}{1.528689in}}%
\pgfpathlineto{\pgfqpoint{4.311865in}{1.536191in}}%
\pgfpathlineto{\pgfqpoint{4.314147in}{1.532532in}}%
\pgfpathlineto{\pgfqpoint{4.320993in}{1.538386in}}%
\pgfpathlineto{\pgfqpoint{4.323275in}{1.534544in}}%
\pgfpathlineto{\pgfqpoint{4.325557in}{1.541771in}}%
\pgfpathlineto{\pgfqpoint{4.327839in}{1.546528in}}%
\pgfpathlineto{\pgfqpoint{4.330121in}{1.546253in}}%
\pgfpathlineto{\pgfqpoint{4.339248in}{1.548998in}}%
\pgfpathlineto{\pgfqpoint{4.341530in}{1.547717in}}%
\pgfpathlineto{\pgfqpoint{4.343812in}{1.542503in}}%
\pgfpathlineto{\pgfqpoint{4.346094in}{1.547534in}}%
\pgfpathlineto{\pgfqpoint{4.352940in}{1.548723in}}%
\pgfpathlineto{\pgfqpoint{4.355222in}{1.543509in}}%
\pgfpathlineto{\pgfqpoint{4.357504in}{1.548174in}}%
\pgfpathlineto{\pgfqpoint{4.359785in}{1.546619in}}%
\pgfpathlineto{\pgfqpoint{4.362067in}{1.552474in}}%
\pgfpathlineto{\pgfqpoint{4.371195in}{1.557688in}}%
\pgfpathlineto{\pgfqpoint{4.373477in}{1.556224in}}%
\pgfpathlineto{\pgfqpoint{4.375759in}{1.558054in}}%
\pgfpathlineto{\pgfqpoint{4.378041in}{1.558420in}}%
\pgfpathlineto{\pgfqpoint{4.384886in}{1.555584in}}%
\pgfpathlineto{\pgfqpoint{4.387168in}{1.550827in}}%
\pgfpathlineto{\pgfqpoint{4.389450in}{1.551102in}}%
\pgfpathlineto{\pgfqpoint{4.394014in}{1.554486in}}%
\pgfpathlineto{\pgfqpoint{4.400860in}{1.553572in}}%
\pgfpathlineto{\pgfqpoint{4.403142in}{1.557231in}}%
\pgfpathlineto{\pgfqpoint{4.405424in}{1.553755in}}%
\pgfpathlineto{\pgfqpoint{4.407706in}{1.551834in}}%
\pgfpathlineto{\pgfqpoint{4.409987in}{1.550736in}}%
\pgfpathlineto{\pgfqpoint{4.416833in}{1.548540in}}%
\pgfpathlineto{\pgfqpoint{4.421397in}{1.550736in}}%
\pgfpathlineto{\pgfqpoint{4.425961in}{1.546894in}}%
\pgfpathlineto{\pgfqpoint{4.432807in}{1.546802in}}%
\pgfpathlineto{\pgfqpoint{4.435088in}{1.543235in}}%
\pgfpathlineto{\pgfqpoint{4.437370in}{1.545979in}}%
\pgfpathlineto{\pgfqpoint{4.439652in}{1.545704in}}%
\pgfpathlineto{\pgfqpoint{4.441934in}{1.546436in}}%
\pgfpathlineto{\pgfqpoint{4.448780in}{1.549547in}}%
\pgfpathlineto{\pgfqpoint{4.451062in}{1.557231in}}%
\pgfpathlineto{\pgfqpoint{4.453344in}{1.558511in}}%
\pgfpathlineto{\pgfqpoint{4.455626in}{1.562079in}}%
\pgfpathlineto{\pgfqpoint{4.464753in}{1.562537in}}%
\pgfpathlineto{\pgfqpoint{4.467035in}{1.559335in}}%
\pgfpathlineto{\pgfqpoint{4.469317in}{1.561347in}}%
\pgfpathlineto{\pgfqpoint{4.471599in}{1.559975in}}%
\pgfpathlineto{\pgfqpoint{4.480727in}{1.571867in}}%
\pgfpathlineto{\pgfqpoint{4.485290in}{1.575435in}}%
\pgfpathlineto{\pgfqpoint{4.487572in}{1.565006in}}%
\pgfpathlineto{\pgfqpoint{4.489854in}{1.570038in}}%
\pgfpathlineto{\pgfqpoint{4.496700in}{1.567934in}}%
\pgfpathlineto{\pgfqpoint{4.498982in}{1.572416in}}%
\pgfpathlineto{\pgfqpoint{4.501264in}{1.572233in}}%
\pgfpathlineto{\pgfqpoint{4.503546in}{1.575618in}}%
\pgfpathlineto{\pgfqpoint{4.505828in}{1.558328in}}%
\pgfpathlineto{\pgfqpoint{4.512673in}{1.557139in}}%
\pgfpathlineto{\pgfqpoint{4.514955in}{1.555584in}}%
\pgfpathlineto{\pgfqpoint{4.517237in}{1.556773in}}%
\pgfpathlineto{\pgfqpoint{4.519519in}{1.550187in}}%
\pgfpathlineto{\pgfqpoint{4.521801in}{1.551925in}}%
\pgfpathlineto{\pgfqpoint{4.528647in}{1.552657in}}%
\pgfpathlineto{\pgfqpoint{4.530928in}{1.549181in}}%
\pgfpathlineto{\pgfqpoint{4.533210in}{1.549455in}}%
\pgfpathlineto{\pgfqpoint{4.535492in}{1.545887in}}%
\pgfpathlineto{\pgfqpoint{4.537774in}{1.548998in}}%
\pgfpathlineto{\pgfqpoint{4.544620in}{1.549272in}}%
\pgfpathlineto{\pgfqpoint{4.546902in}{1.548632in}}%
\pgfpathlineto{\pgfqpoint{4.549184in}{1.554578in}}%
\pgfpathlineto{\pgfqpoint{4.551466in}{1.556956in}}%
\pgfpathlineto{\pgfqpoint{4.553748in}{1.551925in}}%
\pgfpathlineto{\pgfqpoint{4.560593in}{1.560890in}}%
\pgfpathlineto{\pgfqpoint{4.562875in}{1.562354in}}%
\pgfpathlineto{\pgfqpoint{4.565157in}{1.565006in}}%
\pgfpathlineto{\pgfqpoint{4.567439in}{1.564000in}}%
\pgfpathlineto{\pgfqpoint{4.569721in}{1.564732in}}%
\pgfpathlineto{\pgfqpoint{4.576567in}{1.564641in}}%
\pgfpathlineto{\pgfqpoint{4.578849in}{1.566196in}}%
\pgfpathlineto{\pgfqpoint{4.581130in}{1.567019in}}%
\pgfpathlineto{\pgfqpoint{4.585694in}{1.556956in}}%
\pgfpathlineto{\pgfqpoint{4.594822in}{1.561805in}}%
\pgfpathlineto{\pgfqpoint{4.597104in}{1.562171in}}%
\pgfpathlineto{\pgfqpoint{4.599386in}{1.560798in}}%
\pgfpathlineto{\pgfqpoint{4.601668in}{1.560615in}}%
\pgfpathlineto{\pgfqpoint{4.608513in}{1.564092in}}%
\pgfpathlineto{\pgfqpoint{4.610795in}{1.560798in}}%
\pgfpathlineto{\pgfqpoint{4.613077in}{1.566470in}}%
\pgfpathlineto{\pgfqpoint{4.615359in}{1.562262in}}%
\pgfpathlineto{\pgfqpoint{4.617641in}{1.560250in}}%
\pgfpathlineto{\pgfqpoint{4.624487in}{1.561622in}}%
\pgfpathlineto{\pgfqpoint{4.626769in}{1.567385in}}%
\pgfpathlineto{\pgfqpoint{4.629050in}{1.563543in}}%
\pgfpathlineto{\pgfqpoint{4.631332in}{1.565555in}}%
\pgfpathlineto{\pgfqpoint{4.633614in}{1.565098in}}%
\pgfpathlineto{\pgfqpoint{4.640460in}{1.559975in}}%
\pgfpathlineto{\pgfqpoint{4.642742in}{1.556865in}}%
\pgfpathlineto{\pgfqpoint{4.645024in}{1.561073in}}%
\pgfpathlineto{\pgfqpoint{4.647306in}{1.553023in}}%
\pgfpathlineto{\pgfqpoint{4.649588in}{1.555676in}}%
\pgfpathlineto{\pgfqpoint{4.656433in}{1.553480in}}%
\pgfpathlineto{\pgfqpoint{4.658715in}{1.558786in}}%
\pgfpathlineto{\pgfqpoint{4.660997in}{1.551468in}}%
\pgfpathlineto{\pgfqpoint{4.663279in}{1.550827in}}%
\pgfpathlineto{\pgfqpoint{4.665561in}{1.555859in}}%
\pgfpathlineto{\pgfqpoint{4.672407in}{1.555401in}}%
\pgfpathlineto{\pgfqpoint{4.674689in}{1.547260in}}%
\pgfpathlineto{\pgfqpoint{4.676971in}{1.556590in}}%
\pgfpathlineto{\pgfqpoint{4.679252in}{1.546436in}}%
\pgfpathlineto{\pgfqpoint{4.681534in}{1.540490in}}%
\pgfpathlineto{\pgfqpoint{4.688380in}{1.539027in}}%
\pgfpathlineto{\pgfqpoint{4.690662in}{1.535550in}}%
\pgfpathlineto{\pgfqpoint{4.692944in}{1.530153in}}%
\pgfpathlineto{\pgfqpoint{4.697508in}{1.542777in}}%
\pgfpathlineto{\pgfqpoint{4.704353in}{1.546802in}}%
\pgfpathlineto{\pgfqpoint{4.706635in}{1.558694in}}%
\pgfpathlineto{\pgfqpoint{4.708917in}{1.553480in}}%
\pgfpathlineto{\pgfqpoint{4.711199in}{1.560798in}}%
\pgfpathlineto{\pgfqpoint{4.713481in}{1.559060in}}%
\pgfpathlineto{\pgfqpoint{4.720327in}{1.558877in}}%
\pgfpathlineto{\pgfqpoint{4.722609in}{1.566104in}}%
\pgfpathlineto{\pgfqpoint{4.724891in}{1.561622in}}%
\pgfpathlineto{\pgfqpoint{4.727172in}{1.609648in}}%
\pgfpathlineto{\pgfqpoint{4.729454in}{1.620077in}}%
\pgfpathlineto{\pgfqpoint{4.736300in}{1.620259in}}%
\pgfpathlineto{\pgfqpoint{4.738582in}{1.623461in}}%
\pgfpathlineto{\pgfqpoint{4.740864in}{1.637732in}}%
\pgfpathlineto{\pgfqpoint{4.743146in}{1.638921in}}%
\pgfpathlineto{\pgfqpoint{4.745428in}{1.644044in}}%
\pgfpathlineto{\pgfqpoint{4.754555in}{1.638189in}}%
\pgfpathlineto{\pgfqpoint{4.756837in}{1.647246in}}%
\pgfpathlineto{\pgfqpoint{4.759119in}{1.645050in}}%
\pgfpathlineto{\pgfqpoint{4.761401in}{1.640476in}}%
\pgfpathlineto{\pgfqpoint{4.768247in}{1.642580in}}%
\pgfpathlineto{\pgfqpoint{4.772811in}{1.642672in}}%
\pgfpathlineto{\pgfqpoint{4.775093in}{1.646697in}}%
\pgfpathlineto{\pgfqpoint{4.777374in}{1.652277in}}%
\pgfpathlineto{\pgfqpoint{4.784220in}{1.653192in}}%
\pgfpathlineto{\pgfqpoint{4.786502in}{1.658955in}}%
\pgfpathlineto{\pgfqpoint{4.788784in}{1.658955in}}%
\pgfpathlineto{\pgfqpoint{4.793348in}{1.660967in}}%
\pgfpathlineto{\pgfqpoint{4.800194in}{1.660876in}}%
\pgfpathlineto{\pgfqpoint{4.804757in}{1.668469in}}%
\pgfpathlineto{\pgfqpoint{4.807039in}{1.667462in}}%
\pgfpathlineto{\pgfqpoint{4.809321in}{1.672311in}}%
\pgfpathlineto{\pgfqpoint{4.816167in}{1.671762in}}%
\pgfpathlineto{\pgfqpoint{4.818449in}{1.674049in}}%
\pgfpathlineto{\pgfqpoint{4.820731in}{1.668743in}}%
\pgfpathlineto{\pgfqpoint{4.823013in}{1.671853in}}%
\pgfpathlineto{\pgfqpoint{4.825294in}{1.657949in}}%
\pgfpathlineto{\pgfqpoint{4.832140in}{1.657766in}}%
\pgfpathlineto{\pgfqpoint{4.834422in}{1.650539in}}%
\pgfpathlineto{\pgfqpoint{4.838986in}{1.674140in}}%
\pgfpathlineto{\pgfqpoint{4.841268in}{1.668652in}}%
\pgfpathlineto{\pgfqpoint{4.850395in}{1.676519in}}%
\pgfpathlineto{\pgfqpoint{4.852677in}{1.681733in}}%
\pgfpathlineto{\pgfqpoint{4.857241in}{1.679538in}}%
\pgfpathlineto{\pgfqpoint{4.864087in}{1.676976in}}%
\pgfpathlineto{\pgfqpoint{4.866369in}{1.674964in}}%
\pgfpathlineto{\pgfqpoint{4.868651in}{1.669841in}}%
\pgfpathlineto{\pgfqpoint{4.873215in}{1.676061in}}%
\pgfpathlineto{\pgfqpoint{4.882342in}{1.659504in}}%
\pgfpathlineto{\pgfqpoint{4.886906in}{1.674781in}}%
\pgfpathlineto{\pgfqpoint{4.889188in}{1.666182in}}%
\pgfpathlineto{\pgfqpoint{4.896034in}{1.664992in}}%
\pgfpathlineto{\pgfqpoint{4.898316in}{1.666731in}}%
\pgfpathlineto{\pgfqpoint{4.900597in}{1.655204in}}%
\pgfpathlineto{\pgfqpoint{4.902879in}{1.649899in}}%
\pgfpathlineto{\pgfqpoint{4.905161in}{1.653924in}}%
\pgfpathlineto{\pgfqpoint{4.921135in}{1.661242in}}%
\pgfpathlineto{\pgfqpoint{4.927980in}{1.657400in}}%
\pgfpathlineto{\pgfqpoint{4.932544in}{1.634988in}}%
\pgfpathlineto{\pgfqpoint{4.934826in}{1.638555in}}%
\pgfpathlineto{\pgfqpoint{4.937108in}{1.653832in}}%
\pgfpathlineto{\pgfqpoint{4.943954in}{1.654747in}}%
\pgfpathlineto{\pgfqpoint{4.948517in}{1.675787in}}%
\pgfpathlineto{\pgfqpoint{4.950799in}{1.690972in}}%
\pgfpathlineto{\pgfqpoint{4.953081in}{1.681276in}}%
\pgfpathlineto{\pgfqpoint{4.962209in}{1.674964in}}%
\pgfpathlineto{\pgfqpoint{4.964491in}{1.682465in}}%
\pgfpathlineto{\pgfqpoint{4.966773in}{1.693351in}}%
\pgfpathlineto{\pgfqpoint{4.969055in}{1.690515in}}%
\pgfpathlineto{\pgfqpoint{4.978182in}{1.693259in}}%
\pgfpathlineto{\pgfqpoint{4.980464in}{1.689417in}}%
\pgfpathlineto{\pgfqpoint{4.982746in}{1.689326in}}%
\pgfpathlineto{\pgfqpoint{4.985028in}{1.697925in}}%
\pgfpathlineto{\pgfqpoint{4.991874in}{1.698016in}}%
\pgfpathlineto{\pgfqpoint{4.994156in}{1.696827in}}%
\pgfpathlineto{\pgfqpoint{4.998719in}{1.699663in}}%
\pgfpathlineto{\pgfqpoint{5.001001in}{1.694266in}}%
\pgfpathlineto{\pgfqpoint{5.007847in}{1.709726in}}%
\pgfpathlineto{\pgfqpoint{5.010129in}{1.703048in}}%
\pgfpathlineto{\pgfqpoint{5.012411in}{1.699663in}}%
\pgfpathlineto{\pgfqpoint{5.014693in}{1.700486in}}%
\pgfpathlineto{\pgfqpoint{5.016975in}{1.689875in}}%
\pgfpathlineto{\pgfqpoint{5.023820in}{1.694449in}}%
\pgfpathlineto{\pgfqpoint{5.026102in}{1.680727in}}%
\pgfpathlineto{\pgfqpoint{5.028384in}{1.679721in}}%
\pgfpathlineto{\pgfqpoint{5.030666in}{1.690424in}}%
\pgfpathlineto{\pgfqpoint{5.032948in}{1.680269in}}%
\pgfpathlineto{\pgfqpoint{5.039794in}{1.689143in}}%
\pgfpathlineto{\pgfqpoint{5.042076in}{1.679172in}}%
\pgfpathlineto{\pgfqpoint{5.044358in}{1.686124in}}%
\pgfpathlineto{\pgfqpoint{5.046639in}{1.685209in}}%
\pgfpathlineto{\pgfqpoint{5.048921in}{1.690515in}}%
\pgfpathlineto{\pgfqpoint{5.055767in}{1.687588in}}%
\pgfpathlineto{\pgfqpoint{5.058049in}{1.687771in}}%
\pgfpathlineto{\pgfqpoint{5.060331in}{1.675604in}}%
\pgfpathlineto{\pgfqpoint{5.062613in}{1.674415in}}%
\pgfpathlineto{\pgfqpoint{5.064895in}{1.673957in}}%
\pgfpathlineto{\pgfqpoint{5.071740in}{1.675055in}}%
\pgfpathlineto{\pgfqpoint{5.076304in}{1.670756in}}%
\pgfpathlineto{\pgfqpoint{5.078586in}{1.671762in}}%
\pgfpathlineto{\pgfqpoint{5.087714in}{1.670939in}}%
\pgfpathlineto{\pgfqpoint{5.092278in}{1.682465in}}%
\pgfpathlineto{\pgfqpoint{5.096841in}{1.681001in}}%
\pgfpathlineto{\pgfqpoint{5.103687in}{1.674598in}}%
\pgfpathlineto{\pgfqpoint{5.105969in}{1.673866in}}%
\pgfpathlineto{\pgfqpoint{5.108251in}{1.675238in}}%
\pgfpathlineto{\pgfqpoint{5.110533in}{1.674964in}}%
\pgfpathlineto{\pgfqpoint{5.112815in}{1.664901in}}%
\pgfpathlineto{\pgfqpoint{5.119660in}{1.666731in}}%
\pgfpathlineto{\pgfqpoint{5.121942in}{1.672311in}}%
\pgfpathlineto{\pgfqpoint{5.124224in}{1.695821in}}%
\pgfpathlineto{\pgfqpoint{5.128788in}{1.691155in}}%
\pgfpathlineto{\pgfqpoint{5.135634in}{1.687771in}}%
\pgfpathlineto{\pgfqpoint{5.137916in}{1.684843in}}%
\pgfpathlineto{\pgfqpoint{5.140198in}{1.689875in}}%
\pgfpathlineto{\pgfqpoint{5.142480in}{1.678440in}}%
\pgfpathlineto{\pgfqpoint{5.144761in}{1.675970in}}%
\pgfpathlineto{\pgfqpoint{5.151607in}{1.674323in}}%
\pgfpathlineto{\pgfqpoint{5.153889in}{1.677708in}}%
\pgfpathlineto{\pgfqpoint{5.156171in}{1.675147in}}%
\pgfpathlineto{\pgfqpoint{5.158453in}{1.683197in}}%
\pgfpathlineto{\pgfqpoint{5.160735in}{1.708811in}}%
\pgfpathlineto{\pgfqpoint{5.167581in}{1.704786in}}%
\pgfpathlineto{\pgfqpoint{5.169862in}{1.701858in}}%
\pgfpathlineto{\pgfqpoint{5.172144in}{1.702590in}}%
\pgfpathlineto{\pgfqpoint{5.174426in}{1.714574in}}%
\pgfpathlineto{\pgfqpoint{5.176708in}{1.710732in}}%
\pgfpathlineto{\pgfqpoint{5.185836in}{1.715946in}}%
\pgfpathlineto{\pgfqpoint{5.190400in}{1.708994in}}%
\pgfpathlineto{\pgfqpoint{5.192681in}{1.711189in}}%
\pgfpathlineto{\pgfqpoint{5.201809in}{1.701584in}}%
\pgfpathlineto{\pgfqpoint{5.204091in}{1.710000in}}%
\pgfpathlineto{\pgfqpoint{5.206373in}{1.710640in}}%
\pgfpathlineto{\pgfqpoint{5.208655in}{1.702865in}}%
\pgfpathlineto{\pgfqpoint{5.215501in}{1.706707in}}%
\pgfpathlineto{\pgfqpoint{5.220064in}{1.705243in}}%
\pgfpathlineto{\pgfqpoint{5.222346in}{1.698657in}}%
\pgfpathlineto{\pgfqpoint{5.224628in}{1.700120in}}%
\pgfpathlineto{\pgfqpoint{5.231474in}{1.693991in}}%
\pgfpathlineto{\pgfqpoint{5.233756in}{1.696370in}}%
\pgfpathlineto{\pgfqpoint{5.236038in}{1.711281in}}%
\pgfpathlineto{\pgfqpoint{5.238320in}{1.711372in}}%
\pgfpathlineto{\pgfqpoint{5.240602in}{1.708628in}}%
\pgfpathlineto{\pgfqpoint{5.247447in}{1.701858in}}%
\pgfpathlineto{\pgfqpoint{5.249729in}{1.705792in}}%
\pgfpathlineto{\pgfqpoint{5.252011in}{1.703505in}}%
\pgfpathlineto{\pgfqpoint{5.254293in}{1.710091in}}%
\pgfpathlineto{\pgfqpoint{5.256575in}{1.702956in}}%
\pgfpathlineto{\pgfqpoint{5.263421in}{1.706524in}}%
\pgfpathlineto{\pgfqpoint{5.265703in}{1.709451in}}%
\pgfpathlineto{\pgfqpoint{5.267984in}{1.704420in}}%
\pgfpathlineto{\pgfqpoint{5.270266in}{1.702316in}}%
\pgfpathlineto{\pgfqpoint{5.272548in}{1.703505in}}%
\pgfpathlineto{\pgfqpoint{5.279394in}{1.685484in}}%
\pgfpathlineto{\pgfqpoint{5.281676in}{1.689234in}}%
\pgfpathlineto{\pgfqpoint{5.283958in}{1.695455in}}%
\pgfpathlineto{\pgfqpoint{5.286240in}{1.698931in}}%
\pgfpathlineto{\pgfqpoint{5.295367in}{1.697833in}}%
\pgfpathlineto{\pgfqpoint{5.297649in}{1.694723in}}%
\pgfpathlineto{\pgfqpoint{5.299931in}{1.685575in}}%
\pgfpathlineto{\pgfqpoint{5.302213in}{1.688411in}}%
\pgfpathlineto{\pgfqpoint{5.304495in}{1.700578in}}%
\pgfpathlineto{\pgfqpoint{5.311341in}{1.710274in}}%
\pgfpathlineto{\pgfqpoint{5.313623in}{1.715489in}}%
\pgfpathlineto{\pgfqpoint{5.315904in}{1.714757in}}%
\pgfpathlineto{\pgfqpoint{5.318186in}{1.719605in}}%
\pgfpathlineto{\pgfqpoint{5.320468in}{1.722350in}}%
\pgfpathlineto{\pgfqpoint{5.327314in}{1.738541in}}%
\pgfpathlineto{\pgfqpoint{5.329596in}{1.732504in}}%
\pgfpathlineto{\pgfqpoint{5.331878in}{1.732046in}}%
\pgfpathlineto{\pgfqpoint{5.334160in}{1.730125in}}%
\pgfpathlineto{\pgfqpoint{5.336442in}{1.757203in}}%
\pgfpathlineto{\pgfqpoint{5.343287in}{1.748787in}}%
\pgfpathlineto{\pgfqpoint{5.345569in}{1.756654in}}%
\pgfpathlineto{\pgfqpoint{5.347851in}{1.767906in}}%
\pgfpathlineto{\pgfqpoint{5.350133in}{1.771199in}}%
\pgfpathlineto{\pgfqpoint{5.352415in}{1.761960in}}%
\pgfpathlineto{\pgfqpoint{5.359261in}{1.765802in}}%
\pgfpathlineto{\pgfqpoint{5.361543in}{1.761319in}}%
\pgfpathlineto{\pgfqpoint{5.363825in}{1.758758in}}%
\pgfpathlineto{\pgfqpoint{5.366106in}{1.746317in}}%
\pgfpathlineto{\pgfqpoint{5.368388in}{1.751897in}}%
\pgfpathlineto{\pgfqpoint{5.375234in}{1.753452in}}%
\pgfpathlineto{\pgfqpoint{5.377516in}{1.744396in}}%
\pgfpathlineto{\pgfqpoint{5.384362in}{1.753086in}}%
\pgfpathlineto{\pgfqpoint{5.393489in}{1.755282in}}%
\pgfpathlineto{\pgfqpoint{5.395771in}{1.754733in}}%
\pgfpathlineto{\pgfqpoint{5.398053in}{1.750708in}}%
\pgfpathlineto{\pgfqpoint{5.400335in}{1.726192in}}%
\pgfpathlineto{\pgfqpoint{5.407181in}{1.701035in}}%
\pgfpathlineto{\pgfqpoint{5.409463in}{1.688594in}}%
\pgfpathlineto{\pgfqpoint{5.411745in}{1.721709in}}%
\pgfpathlineto{\pgfqpoint{5.414026in}{1.736895in}}%
\pgfpathlineto{\pgfqpoint{5.416308in}{1.737444in}}%
\pgfpathlineto{\pgfqpoint{5.423154in}{1.727106in}}%
\pgfpathlineto{\pgfqpoint{5.425436in}{1.706341in}}%
\pgfpathlineto{\pgfqpoint{5.430000in}{1.719148in}}%
\pgfpathlineto{\pgfqpoint{5.432282in}{1.708079in}}%
\pgfpathlineto{\pgfqpoint{5.441409in}{1.720337in}}%
\pgfpathlineto{\pgfqpoint{5.443691in}{1.712012in}}%
\pgfpathlineto{\pgfqpoint{5.448255in}{1.722350in}}%
\pgfpathlineto{\pgfqpoint{5.455101in}{1.715672in}}%
\pgfpathlineto{\pgfqpoint{5.459665in}{1.724362in}}%
\pgfpathlineto{\pgfqpoint{5.461946in}{1.724088in}}%
\pgfpathlineto{\pgfqpoint{5.464228in}{1.713751in}}%
\pgfpathlineto{\pgfqpoint{5.471074in}{1.722533in}}%
\pgfpathlineto{\pgfqpoint{5.473356in}{1.717776in}}%
\pgfpathlineto{\pgfqpoint{5.475638in}{1.723996in}}%
\pgfpathlineto{\pgfqpoint{5.477920in}{1.717684in}}%
\pgfpathlineto{\pgfqpoint{5.480202in}{1.721709in}}%
\pgfpathlineto{\pgfqpoint{5.487047in}{1.690698in}}%
\pgfpathlineto{\pgfqpoint{5.491611in}{1.712561in}}%
\pgfpathlineto{\pgfqpoint{5.493893in}{1.715489in}}%
\pgfpathlineto{\pgfqpoint{5.496175in}{1.721526in}}%
\pgfpathlineto{\pgfqpoint{5.503021in}{1.735431in}}%
\pgfpathlineto{\pgfqpoint{5.505303in}{1.734059in}}%
\pgfpathlineto{\pgfqpoint{5.507585in}{1.744213in}}%
\pgfpathlineto{\pgfqpoint{5.509867in}{1.750251in}}%
\pgfpathlineto{\pgfqpoint{5.512148in}{1.750982in}}%
\pgfpathlineto{\pgfqpoint{5.518994in}{1.759947in}}%
\pgfpathlineto{\pgfqpoint{5.521276in}{1.760039in}}%
\pgfpathlineto{\pgfqpoint{5.523558in}{1.752903in}}%
\pgfpathlineto{\pgfqpoint{5.525840in}{1.763058in}}%
\pgfpathlineto{\pgfqpoint{5.528122in}{1.768912in}}%
\pgfpathlineto{\pgfqpoint{5.534968in}{1.777694in}}%
\pgfpathlineto{\pgfqpoint{5.539531in}{1.764155in}}%
\pgfpathlineto{\pgfqpoint{5.541813in}{1.772663in}}%
\pgfpathlineto{\pgfqpoint{5.544095in}{1.778426in}}%
\pgfpathlineto{\pgfqpoint{5.550941in}{1.788306in}}%
\pgfpathlineto{\pgfqpoint{5.553223in}{1.782451in}}%
\pgfpathlineto{\pgfqpoint{5.555505in}{1.794435in}}%
\pgfpathlineto{\pgfqpoint{5.557787in}{1.791233in}}%
\pgfpathlineto{\pgfqpoint{5.560068in}{1.783000in}}%
\pgfpathlineto{\pgfqpoint{5.566914in}{1.761960in}}%
\pgfpathlineto{\pgfqpoint{5.569196in}{1.785836in}}%
\pgfpathlineto{\pgfqpoint{5.571478in}{1.789861in}}%
\pgfpathlineto{\pgfqpoint{5.573760in}{1.797911in}}%
\pgfpathlineto{\pgfqpoint{5.576042in}{1.793428in}}%
\pgfpathlineto{\pgfqpoint{5.582888in}{1.787025in}}%
\pgfpathlineto{\pgfqpoint{5.585169in}{1.801113in}}%
\pgfpathlineto{\pgfqpoint{5.587451in}{1.798460in}}%
\pgfpathlineto{\pgfqpoint{5.589733in}{1.790684in}}%
\pgfpathlineto{\pgfqpoint{5.592015in}{1.788946in}}%
\pgfpathlineto{\pgfqpoint{5.598861in}{1.795990in}}%
\pgfpathlineto{\pgfqpoint{5.601143in}{1.795349in}}%
\pgfpathlineto{\pgfqpoint{5.603425in}{1.809895in}}%
\pgfpathlineto{\pgfqpoint{5.605707in}{1.807150in}}%
\pgfpathlineto{\pgfqpoint{5.607989in}{1.807425in}}%
\pgfpathlineto{\pgfqpoint{5.614834in}{1.806784in}}%
\pgfpathlineto{\pgfqpoint{5.619398in}{1.801844in}}%
\pgfpathlineto{\pgfqpoint{5.623962in}{1.804314in}}%
\pgfpathlineto{\pgfqpoint{5.630808in}{1.796996in}}%
\pgfpathlineto{\pgfqpoint{5.633090in}{1.804955in}}%
\pgfpathlineto{\pgfqpoint{5.637653in}{1.790501in}}%
\pgfpathlineto{\pgfqpoint{5.639935in}{1.809346in}}%
\pgfpathlineto{\pgfqpoint{5.646781in}{1.801753in}}%
\pgfpathlineto{\pgfqpoint{5.649063in}{1.797453in}}%
\pgfpathlineto{\pgfqpoint{5.651345in}{1.786933in}}%
\pgfpathlineto{\pgfqpoint{5.653627in}{1.789037in}}%
\pgfpathlineto{\pgfqpoint{5.655909in}{1.771108in}}%
\pgfpathlineto{\pgfqpoint{5.662754in}{1.777877in}}%
\pgfpathlineto{\pgfqpoint{5.665036in}{1.793520in}}%
\pgfpathlineto{\pgfqpoint{5.667318in}{1.804497in}}%
\pgfpathlineto{\pgfqpoint{5.669600in}{1.794069in}}%
\pgfpathlineto{\pgfqpoint{5.671882in}{1.773029in}}%
\pgfpathlineto{\pgfqpoint{5.681010in}{1.782451in}}%
\pgfpathlineto{\pgfqpoint{5.683291in}{1.792971in}}%
\pgfpathlineto{\pgfqpoint{5.685573in}{1.790318in}}%
\pgfpathlineto{\pgfqpoint{5.694701in}{1.792697in}}%
\pgfpathlineto{\pgfqpoint{5.696983in}{1.798734in}}%
\pgfpathlineto{\pgfqpoint{5.701547in}{1.783915in}}%
\pgfpathlineto{\pgfqpoint{5.710674in}{1.767448in}}%
\pgfpathlineto{\pgfqpoint{5.712956in}{1.772571in}}%
\pgfpathlineto{\pgfqpoint{5.715238in}{1.763606in}}%
\pgfpathlineto{\pgfqpoint{5.717520in}{1.750433in}}%
\pgfpathlineto{\pgfqpoint{5.719802in}{1.742383in}}%
\pgfpathlineto{\pgfqpoint{5.726648in}{1.751623in}}%
\pgfpathlineto{\pgfqpoint{5.728930in}{1.759124in}}%
\pgfpathlineto{\pgfqpoint{5.731212in}{1.744304in}}%
\pgfpathlineto{\pgfqpoint{5.733493in}{1.750525in}}%
\pgfpathlineto{\pgfqpoint{5.735775in}{1.732961in}}%
\pgfpathlineto{\pgfqpoint{5.744903in}{1.729119in}}%
\pgfpathlineto{\pgfqpoint{5.747185in}{1.722715in}}%
\pgfpathlineto{\pgfqpoint{5.749467in}{1.730583in}}%
\pgfpathlineto{\pgfqpoint{5.751749in}{1.740737in}}%
\pgfpathlineto{\pgfqpoint{5.758594in}{1.732229in}}%
\pgfpathlineto{\pgfqpoint{5.760876in}{1.733418in}}%
\pgfpathlineto{\pgfqpoint{5.763158in}{1.724819in}}%
\pgfpathlineto{\pgfqpoint{5.765440in}{1.710732in}}%
\pgfpathlineto{\pgfqpoint{5.767722in}{1.756654in}}%
\pgfpathlineto{\pgfqpoint{5.774568in}{1.755739in}}%
\pgfpathlineto{\pgfqpoint{5.776850in}{1.746957in}}%
\pgfpathlineto{\pgfqpoint{5.779132in}{1.755739in}}%
\pgfpathlineto{\pgfqpoint{5.781413in}{1.749519in}}%
\pgfpathlineto{\pgfqpoint{5.783695in}{1.730400in}}%
\pgfpathlineto{\pgfqpoint{5.790541in}{1.696827in}}%
\pgfpathlineto{\pgfqpoint{5.792823in}{1.701858in}}%
\pgfpathlineto{\pgfqpoint{5.795105in}{1.719331in}}%
\pgfpathlineto{\pgfqpoint{5.797387in}{1.704328in}}%
\pgfpathlineto{\pgfqpoint{5.799669in}{1.721709in}}%
\pgfpathlineto{\pgfqpoint{5.808796in}{1.727838in}}%
\pgfpathlineto{\pgfqpoint{5.811078in}{1.737352in}}%
\pgfpathlineto{\pgfqpoint{5.813360in}{1.730308in}}%
\pgfpathlineto{\pgfqpoint{5.815642in}{1.732870in}}%
\pgfpathlineto{\pgfqpoint{5.822488in}{1.746591in}}%
\pgfpathlineto{\pgfqpoint{5.824770in}{1.738450in}}%
\pgfpathlineto{\pgfqpoint{5.827052in}{1.735797in}}%
\pgfpathlineto{\pgfqpoint{5.829334in}{1.748421in}}%
\pgfpathlineto{\pgfqpoint{5.831615in}{1.743664in}}%
\pgfpathlineto{\pgfqpoint{5.838461in}{1.740554in}}%
\pgfpathlineto{\pgfqpoint{5.840743in}{1.760679in}}%
\pgfpathlineto{\pgfqpoint{5.845307in}{1.754184in}}%
\pgfpathlineto{\pgfqpoint{5.847589in}{1.754093in}}%
\pgfpathlineto{\pgfqpoint{5.854434in}{1.736529in}}%
\pgfpathlineto{\pgfqpoint{5.856716in}{1.724728in}}%
\pgfpathlineto{\pgfqpoint{5.858998in}{1.725277in}}%
\pgfpathlineto{\pgfqpoint{5.861280in}{1.721160in}}%
\pgfpathlineto{\pgfqpoint{5.863562in}{1.733784in}}%
\pgfpathlineto{\pgfqpoint{5.870408in}{1.732504in}}%
\pgfpathlineto{\pgfqpoint{5.874972in}{1.740371in}}%
\pgfpathlineto{\pgfqpoint{5.879535in}{1.752995in}}%
\pgfpathlineto{\pgfqpoint{5.886381in}{1.752903in}}%
\pgfpathlineto{\pgfqpoint{5.888663in}{1.745677in}}%
\pgfpathlineto{\pgfqpoint{5.890945in}{1.754184in}}%
\pgfpathlineto{\pgfqpoint{5.893227in}{1.756197in}}%
\pgfpathlineto{\pgfqpoint{5.902355in}{1.755739in}}%
\pgfpathlineto{\pgfqpoint{5.906918in}{1.779890in}}%
\pgfpathlineto{\pgfqpoint{5.909200in}{1.777145in}}%
\pgfpathlineto{\pgfqpoint{5.911482in}{1.787116in}}%
\pgfpathlineto{\pgfqpoint{5.918328in}{1.789220in}}%
\pgfpathlineto{\pgfqpoint{5.920610in}{1.781536in}}%
\pgfpathlineto{\pgfqpoint{5.922892in}{1.792605in}}%
\pgfpathlineto{\pgfqpoint{5.925174in}{1.787025in}}%
\pgfpathlineto{\pgfqpoint{5.927455in}{1.791050in}}%
\pgfpathlineto{\pgfqpoint{5.934301in}{1.789129in}}%
\pgfpathlineto{\pgfqpoint{5.936583in}{1.795441in}}%
\pgfpathlineto{\pgfqpoint{5.938865in}{1.806052in}}%
\pgfpathlineto{\pgfqpoint{5.941147in}{1.811633in}}%
\pgfpathlineto{\pgfqpoint{5.943429in}{1.809346in}}%
\pgfpathlineto{\pgfqpoint{5.950275in}{1.821604in}}%
\pgfpathlineto{\pgfqpoint{5.952556in}{1.815383in}}%
\pgfpathlineto{\pgfqpoint{5.954838in}{1.818951in}}%
\pgfpathlineto{\pgfqpoint{5.957120in}{1.815749in}}%
\pgfpathlineto{\pgfqpoint{5.959402in}{1.800655in}}%
\pgfpathlineto{\pgfqpoint{5.966248in}{1.792056in}}%
\pgfpathlineto{\pgfqpoint{5.968530in}{1.795532in}}%
\pgfpathlineto{\pgfqpoint{5.970812in}{1.797545in}}%
\pgfpathlineto{\pgfqpoint{5.973094in}{1.787940in}}%
\pgfpathlineto{\pgfqpoint{5.975376in}{1.784006in}}%
\pgfpathlineto{\pgfqpoint{5.982221in}{1.794892in}}%
\pgfpathlineto{\pgfqpoint{5.984503in}{1.783549in}}%
\pgfpathlineto{\pgfqpoint{5.986785in}{1.782451in}}%
\pgfpathlineto{\pgfqpoint{5.991349in}{1.788214in}}%
\pgfpathlineto{\pgfqpoint{5.998195in}{1.792697in}}%
\pgfpathlineto{\pgfqpoint{6.000477in}{1.801478in}}%
\pgfpathlineto{\pgfqpoint{6.002758in}{1.785744in}}%
\pgfpathlineto{\pgfqpoint{6.005040in}{1.791050in}}%
\pgfpathlineto{\pgfqpoint{6.007322in}{1.781536in}}%
\pgfpathlineto{\pgfqpoint{6.014168in}{1.790227in}}%
\pgfpathlineto{\pgfqpoint{6.016450in}{1.781170in}}%
\pgfpathlineto{\pgfqpoint{6.018732in}{1.786933in}}%
\pgfpathlineto{\pgfqpoint{6.021014in}{1.781994in}}%
\pgfpathlineto{\pgfqpoint{6.023296in}{1.789037in}}%
\pgfpathlineto{\pgfqpoint{6.030141in}{1.784921in}}%
\pgfpathlineto{\pgfqpoint{6.032423in}{1.804406in}}%
\pgfpathlineto{\pgfqpoint{6.034705in}{1.801478in}}%
\pgfpathlineto{\pgfqpoint{6.036987in}{1.800930in}}%
\pgfpathlineto{\pgfqpoint{6.039269in}{1.806876in}}%
\pgfpathlineto{\pgfqpoint{6.048397in}{1.800381in}}%
\pgfpathlineto{\pgfqpoint{6.050678in}{1.802942in}}%
\pgfpathlineto{\pgfqpoint{6.052960in}{1.809437in}}%
\pgfpathlineto{\pgfqpoint{6.055242in}{1.809346in}}%
\pgfpathlineto{\pgfqpoint{6.062088in}{1.814743in}}%
\pgfpathlineto{\pgfqpoint{6.064370in}{1.815292in}}%
\pgfpathlineto{\pgfqpoint{6.066652in}{1.824714in}}%
\pgfpathlineto{\pgfqpoint{6.068934in}{1.821238in}}%
\pgfpathlineto{\pgfqpoint{6.071216in}{1.811541in}}%
\pgfpathlineto{\pgfqpoint{6.078061in}{1.795624in}}%
\pgfpathlineto{\pgfqpoint{6.080343in}{1.797088in}}%
\pgfpathlineto{\pgfqpoint{6.082625in}{1.793520in}}%
\pgfpathlineto{\pgfqpoint{6.084907in}{1.795166in}}%
\pgfpathlineto{\pgfqpoint{6.087189in}{1.782908in}}%
\pgfpathlineto{\pgfqpoint{6.094035in}{1.786110in}}%
\pgfpathlineto{\pgfqpoint{6.096317in}{1.786019in}}%
\pgfpathlineto{\pgfqpoint{6.098599in}{1.778792in}}%
\pgfpathlineto{\pgfqpoint{6.100880in}{1.794069in}}%
\pgfpathlineto{\pgfqpoint{6.103162in}{1.765527in}}%
\pgfpathlineto{\pgfqpoint{6.110008in}{1.750251in}}%
\pgfpathlineto{\pgfqpoint{6.114572in}{1.780713in}}%
\pgfpathlineto{\pgfqpoint{6.116854in}{1.757660in}}%
\pgfpathlineto{\pgfqpoint{6.119136in}{1.760496in}}%
\pgfpathlineto{\pgfqpoint{6.128263in}{1.762143in}}%
\pgfpathlineto{\pgfqpoint{6.130545in}{1.756654in}}%
\pgfpathlineto{\pgfqpoint{6.132827in}{1.760771in}}%
\pgfpathlineto{\pgfqpoint{6.135109in}{1.777877in}}%
\pgfpathlineto{\pgfqpoint{6.141955in}{1.778700in}}%
\pgfpathlineto{\pgfqpoint{6.144237in}{1.787299in}}%
\pgfpathlineto{\pgfqpoint{6.146519in}{1.787116in}}%
\pgfpathlineto{\pgfqpoint{6.148800in}{1.793245in}}%
\pgfpathlineto{\pgfqpoint{6.151082in}{1.794709in}}%
\pgfpathlineto{\pgfqpoint{6.157928in}{1.794801in}}%
\pgfpathlineto{\pgfqpoint{6.160210in}{1.798460in}}%
\pgfpathlineto{\pgfqpoint{6.162492in}{1.804223in}}%
\pgfpathlineto{\pgfqpoint{6.164774in}{1.799100in}}%
\pgfpathlineto{\pgfqpoint{6.167056in}{1.809071in}}%
\pgfpathlineto{\pgfqpoint{6.173901in}{1.800381in}}%
\pgfpathlineto{\pgfqpoint{6.176183in}{1.796356in}}%
\pgfpathlineto{\pgfqpoint{6.178465in}{1.796722in}}%
\pgfpathlineto{\pgfqpoint{6.180747in}{1.802668in}}%
\pgfpathlineto{\pgfqpoint{6.183029in}{1.792422in}}%
\pgfpathlineto{\pgfqpoint{6.192157in}{1.794526in}}%
\pgfpathlineto{\pgfqpoint{6.194439in}{1.798368in}}%
\pgfpathlineto{\pgfqpoint{6.199002in}{1.811175in}}%
\pgfpathlineto{\pgfqpoint{6.205848in}{1.809437in}}%
\pgfpathlineto{\pgfqpoint{6.208130in}{1.810443in}}%
\pgfpathlineto{\pgfqpoint{6.210412in}{1.807059in}}%
\pgfpathlineto{\pgfqpoint{6.212694in}{1.810992in}}%
\pgfpathlineto{\pgfqpoint{6.214976in}{1.810077in}}%
\pgfpathlineto{\pgfqpoint{6.221821in}{1.818036in}}%
\pgfpathlineto{\pgfqpoint{6.224103in}{1.817213in}}%
\pgfpathlineto{\pgfqpoint{6.226385in}{1.818676in}}%
\pgfpathlineto{\pgfqpoint{6.228667in}{1.813554in}}%
\pgfpathlineto{\pgfqpoint{6.237795in}{1.820689in}}%
\pgfpathlineto{\pgfqpoint{6.240077in}{1.818311in}}%
\pgfpathlineto{\pgfqpoint{6.242359in}{1.813737in}}%
\pgfpathlineto{\pgfqpoint{6.244641in}{1.814011in}}%
\pgfpathlineto{\pgfqpoint{6.246922in}{1.816298in}}%
\pgfpathlineto{\pgfqpoint{6.253768in}{1.818951in}}%
\pgfpathlineto{\pgfqpoint{6.256050in}{1.821695in}}%
\pgfpathlineto{\pgfqpoint{6.258332in}{1.819225in}}%
\pgfpathlineto{\pgfqpoint{6.260614in}{1.822793in}}%
\pgfpathlineto{\pgfqpoint{6.262896in}{1.829105in}}%
\pgfpathlineto{\pgfqpoint{6.272023in}{1.833862in}}%
\pgfpathlineto{\pgfqpoint{6.274305in}{1.840265in}}%
\pgfpathlineto{\pgfqpoint{6.276587in}{1.837795in}}%
\pgfpathlineto{\pgfqpoint{6.278869in}{1.823067in}}%
\pgfpathlineto{\pgfqpoint{6.285715in}{1.837795in}}%
\pgfpathlineto{\pgfqpoint{6.287997in}{1.828099in}}%
\pgfpathlineto{\pgfqpoint{6.290279in}{1.824531in}}%
\pgfpathlineto{\pgfqpoint{6.292561in}{1.829197in}}%
\pgfpathlineto{\pgfqpoint{6.294843in}{1.829745in}}%
\pgfpathlineto{\pgfqpoint{6.301688in}{1.833679in}}%
\pgfpathlineto{\pgfqpoint{6.303970in}{1.833405in}}%
\pgfpathlineto{\pgfqpoint{6.306252in}{1.840082in}}%
\pgfpathlineto{\pgfqpoint{6.308534in}{1.841272in}}%
\pgfpathlineto{\pgfqpoint{6.310816in}{1.833953in}}%
\pgfpathlineto{\pgfqpoint{6.317662in}{1.827367in}}%
\pgfpathlineto{\pgfqpoint{6.319943in}{1.830569in}}%
\pgfpathlineto{\pgfqpoint{6.322225in}{1.837795in}}%
\pgfpathlineto{\pgfqpoint{6.324507in}{1.828373in}}%
\pgfpathlineto{\pgfqpoint{6.326789in}{1.835417in}}%
\pgfpathlineto{\pgfqpoint{6.333635in}{1.836789in}}%
\pgfpathlineto{\pgfqpoint{6.335917in}{1.835691in}}%
\pgfpathlineto{\pgfqpoint{6.338199in}{1.841180in}}%
\pgfpathlineto{\pgfqpoint{6.340481in}{1.841272in}}%
\pgfpathlineto{\pgfqpoint{6.342763in}{1.836972in}}%
\pgfpathlineto{\pgfqpoint{6.349608in}{1.839168in}}%
\pgfpathlineto{\pgfqpoint{6.351890in}{1.829471in}}%
\pgfpathlineto{\pgfqpoint{6.354172in}{1.831392in}}%
\pgfpathlineto{\pgfqpoint{6.356454in}{1.827916in}}%
\pgfpathlineto{\pgfqpoint{6.358736in}{1.833130in}}%
\pgfpathlineto{\pgfqpoint{6.365582in}{1.830477in}}%
\pgfpathlineto{\pgfqpoint{6.367864in}{1.825354in}}%
\pgfpathlineto{\pgfqpoint{6.370145in}{1.836423in}}%
\pgfpathlineto{\pgfqpoint{6.372427in}{1.833587in}}%
\pgfpathlineto{\pgfqpoint{6.374709in}{1.832215in}}%
\pgfpathlineto{\pgfqpoint{6.381555in}{1.839625in}}%
\pgfpathlineto{\pgfqpoint{6.383837in}{1.829379in}}%
\pgfpathlineto{\pgfqpoint{6.386119in}{1.826910in}}%
\pgfpathlineto{\pgfqpoint{6.388401in}{1.828373in}}%
\pgfpathlineto{\pgfqpoint{6.390683in}{1.831118in}}%
\pgfpathlineto{\pgfqpoint{6.397528in}{1.833679in}}%
\pgfpathlineto{\pgfqpoint{6.402092in}{1.816390in}}%
\pgfpathlineto{\pgfqpoint{6.404374in}{1.817030in}}%
\pgfpathlineto{\pgfqpoint{6.406656in}{1.814377in}}%
\pgfpathlineto{\pgfqpoint{6.413502in}{1.833679in}}%
\pgfpathlineto{\pgfqpoint{6.418065in}{1.840265in}}%
\pgfpathlineto{\pgfqpoint{6.420347in}{1.827916in}}%
\pgfpathlineto{\pgfqpoint{6.422629in}{1.828007in}}%
\pgfpathlineto{\pgfqpoint{6.429475in}{1.796630in}}%
\pgfpathlineto{\pgfqpoint{6.431757in}{1.798368in}}%
\pgfpathlineto{\pgfqpoint{6.434039in}{1.813371in}}%
\pgfpathlineto{\pgfqpoint{6.436321in}{1.822336in}}%
\pgfpathlineto{\pgfqpoint{6.438603in}{1.820049in}}%
\pgfpathlineto{\pgfqpoint{6.445448in}{1.827824in}}%
\pgfpathlineto{\pgfqpoint{6.447730in}{1.811999in}}%
\pgfpathlineto{\pgfqpoint{6.450012in}{1.808797in}}%
\pgfpathlineto{\pgfqpoint{6.454576in}{1.813828in}}%
\pgfpathlineto{\pgfqpoint{6.461422in}{1.804406in}}%
\pgfpathlineto{\pgfqpoint{6.463704in}{1.805046in}}%
\pgfpathlineto{\pgfqpoint{6.468267in}{1.771565in}}%
\pgfpathlineto{\pgfqpoint{6.470549in}{1.774218in}}%
\pgfpathlineto{\pgfqpoint{6.477395in}{1.788489in}}%
\pgfpathlineto{\pgfqpoint{6.479677in}{1.786659in}}%
\pgfpathlineto{\pgfqpoint{6.481959in}{1.806510in}}%
\pgfpathlineto{\pgfqpoint{6.484241in}{1.806235in}}%
\pgfpathlineto{\pgfqpoint{6.486523in}{1.804955in}}%
\pgfpathlineto{\pgfqpoint{6.493368in}{1.799192in}}%
\pgfpathlineto{\pgfqpoint{6.495650in}{1.805504in}}%
\pgfpathlineto{\pgfqpoint{6.497932in}{1.804863in}}%
\pgfpathlineto{\pgfqpoint{6.500214in}{1.808156in}}%
\pgfpathlineto{\pgfqpoint{6.502496in}{1.797819in}}%
\pgfpathlineto{\pgfqpoint{6.509342in}{1.795624in}}%
\pgfpathlineto{\pgfqpoint{6.511624in}{1.797911in}}%
\pgfpathlineto{\pgfqpoint{6.516187in}{1.793794in}}%
\pgfpathlineto{\pgfqpoint{6.518469in}{1.795990in}}%
\pgfpathlineto{\pgfqpoint{6.527597in}{1.797819in}}%
\pgfpathlineto{\pgfqpoint{6.529879in}{1.797362in}}%
\pgfpathlineto{\pgfqpoint{6.532161in}{1.797636in}}%
\pgfpathlineto{\pgfqpoint{6.534443in}{1.794892in}}%
\pgfpathlineto{\pgfqpoint{6.543570in}{1.808156in}}%
\pgfpathlineto{\pgfqpoint{6.545852in}{1.814011in}}%
\pgfpathlineto{\pgfqpoint{6.550416in}{1.832581in}}%
\pgfpathlineto{\pgfqpoint{6.557262in}{1.828373in}}%
\pgfpathlineto{\pgfqpoint{6.559544in}{1.824440in}}%
\pgfpathlineto{\pgfqpoint{6.561826in}{1.828831in}}%
\pgfpathlineto{\pgfqpoint{6.564108in}{1.824988in}}%
\pgfpathlineto{\pgfqpoint{6.566389in}{1.823159in}}%
\pgfpathlineto{\pgfqpoint{6.575517in}{1.823982in}}%
\pgfpathlineto{\pgfqpoint{6.577799in}{1.827001in}}%
\pgfpathlineto{\pgfqpoint{6.582363in}{1.829197in}}%
\pgfpathlineto{\pgfqpoint{6.589208in}{1.832032in}}%
\pgfpathlineto{\pgfqpoint{6.591490in}{1.841729in}}%
\pgfpathlineto{\pgfqpoint{6.593772in}{1.847767in}}%
\pgfpathlineto{\pgfqpoint{6.596054in}{1.841821in}}%
\pgfpathlineto{\pgfqpoint{6.598336in}{1.846577in}}%
\pgfpathlineto{\pgfqpoint{6.605182in}{1.845937in}}%
\pgfpathlineto{\pgfqpoint{6.607464in}{1.837064in}}%
\pgfpathlineto{\pgfqpoint{6.609746in}{1.834594in}}%
\pgfpathlineto{\pgfqpoint{6.612028in}{1.833313in}}%
\pgfpathlineto{\pgfqpoint{6.614309in}{1.867343in}}%
\pgfpathlineto{\pgfqpoint{6.623437in}{1.864690in}}%
\pgfpathlineto{\pgfqpoint{6.625719in}{1.858470in}}%
\pgfpathlineto{\pgfqpoint{6.628001in}{1.862861in}}%
\pgfpathlineto{\pgfqpoint{6.630283in}{1.865696in}}%
\pgfpathlineto{\pgfqpoint{6.637129in}{1.870636in}}%
\pgfpathlineto{\pgfqpoint{6.639410in}{1.874295in}}%
\pgfpathlineto{\pgfqpoint{6.641692in}{1.881980in}}%
\pgfpathlineto{\pgfqpoint{6.643974in}{1.880790in}}%
\pgfpathlineto{\pgfqpoint{6.646256in}{1.881248in}}%
\pgfpathlineto{\pgfqpoint{6.655384in}{1.885364in}}%
\pgfpathlineto{\pgfqpoint{6.657666in}{1.884358in}}%
\pgfpathlineto{\pgfqpoint{6.659948in}{1.887743in}}%
\pgfpathlineto{\pgfqpoint{6.662230in}{1.890030in}}%
\pgfpathlineto{\pgfqpoint{6.669075in}{1.886096in}}%
\pgfpathlineto{\pgfqpoint{6.671357in}{1.885547in}}%
\pgfpathlineto{\pgfqpoint{6.673639in}{1.895061in}}%
\pgfpathlineto{\pgfqpoint{6.675921in}{1.890945in}}%
\pgfpathlineto{\pgfqpoint{6.678203in}{1.893231in}}%
\pgfpathlineto{\pgfqpoint{6.685049in}{1.894604in}}%
\pgfpathlineto{\pgfqpoint{6.687330in}{1.895701in}}%
\pgfpathlineto{\pgfqpoint{6.689612in}{1.894787in}}%
\pgfpathlineto{\pgfqpoint{6.691894in}{1.896159in}}%
\pgfpathlineto{\pgfqpoint{6.694176in}{1.901739in}}%
\pgfpathlineto{\pgfqpoint{6.701022in}{1.905124in}}%
\pgfpathlineto{\pgfqpoint{6.703304in}{1.900092in}}%
\pgfpathlineto{\pgfqpoint{6.705586in}{1.903386in}}%
\pgfpathlineto{\pgfqpoint{6.707868in}{1.904483in}}%
\pgfpathlineto{\pgfqpoint{6.710150in}{1.906313in}}%
\pgfpathlineto{\pgfqpoint{6.716995in}{1.896799in}}%
\pgfpathlineto{\pgfqpoint{6.719277in}{1.886645in}}%
\pgfpathlineto{\pgfqpoint{6.721559in}{1.890853in}}%
\pgfpathlineto{\pgfqpoint{6.726123in}{1.896891in}}%
\pgfpathlineto{\pgfqpoint{6.732969in}{1.894695in}}%
\pgfpathlineto{\pgfqpoint{6.735251in}{1.896250in}}%
\pgfpathlineto{\pgfqpoint{6.737532in}{1.896799in}}%
\pgfpathlineto{\pgfqpoint{6.742096in}{1.893963in}}%
\pgfpathlineto{\pgfqpoint{6.748942in}{1.898354in}}%
\pgfpathlineto{\pgfqpoint{6.751224in}{1.893140in}}%
\pgfpathlineto{\pgfqpoint{6.753506in}{1.895427in}}%
\pgfpathlineto{\pgfqpoint{6.755788in}{1.895976in}}%
\pgfpathlineto{\pgfqpoint{6.758070in}{1.892774in}}%
\pgfpathlineto{\pgfqpoint{6.767197in}{1.893780in}}%
\pgfpathlineto{\pgfqpoint{6.769479in}{1.892225in}}%
\pgfpathlineto{\pgfqpoint{6.771761in}{1.893963in}}%
\pgfpathlineto{\pgfqpoint{6.780889in}{1.902471in}}%
\pgfpathlineto{\pgfqpoint{6.783171in}{1.901739in}}%
\pgfpathlineto{\pgfqpoint{6.785452in}{1.901922in}}%
\pgfpathlineto{\pgfqpoint{6.787734in}{1.914546in}}%
\pgfpathlineto{\pgfqpoint{6.790016in}{1.914546in}}%
\pgfpathlineto{\pgfqpoint{6.799144in}{1.923145in}}%
\pgfpathlineto{\pgfqpoint{6.805990in}{1.915186in}}%
\pgfpathlineto{\pgfqpoint{6.812835in}{1.915461in}}%
\pgfpathlineto{\pgfqpoint{6.815117in}{1.927079in}}%
\pgfpathlineto{\pgfqpoint{6.817399in}{1.926164in}}%
\pgfpathlineto{\pgfqpoint{6.819681in}{1.927719in}}%
\pgfpathlineto{\pgfqpoint{6.821963in}{1.922962in}}%
\pgfpathlineto{\pgfqpoint{6.828809in}{1.921407in}}%
\pgfpathlineto{\pgfqpoint{6.831091in}{1.922139in}}%
\pgfpathlineto{\pgfqpoint{6.833373in}{1.924426in}}%
\pgfpathlineto{\pgfqpoint{6.835654in}{1.923054in}}%
\pgfpathlineto{\pgfqpoint{6.837936in}{1.928725in}}%
\pgfpathlineto{\pgfqpoint{6.844782in}{1.933299in}}%
\pgfpathlineto{\pgfqpoint{6.847064in}{1.932567in}}%
\pgfpathlineto{\pgfqpoint{6.849346in}{1.921498in}}%
\pgfpathlineto{\pgfqpoint{6.851628in}{1.921041in}}%
\pgfpathlineto{\pgfqpoint{6.853910in}{1.927993in}}%
\pgfpathlineto{\pgfqpoint{6.860755in}{1.935495in}}%
\pgfpathlineto{\pgfqpoint{6.863037in}{1.940434in}}%
\pgfpathlineto{\pgfqpoint{6.865319in}{1.949033in}}%
\pgfpathlineto{\pgfqpoint{6.867601in}{1.951046in}}%
\pgfpathlineto{\pgfqpoint{6.869883in}{1.947753in}}%
\pgfpathlineto{\pgfqpoint{6.879011in}{1.948210in}}%
\pgfpathlineto{\pgfqpoint{6.881293in}{1.952784in}}%
\pgfpathlineto{\pgfqpoint{6.883574in}{1.954339in}}%
\pgfpathlineto{\pgfqpoint{6.885856in}{1.961109in}}%
\pgfpathlineto{\pgfqpoint{6.892702in}{1.964676in}}%
\pgfpathlineto{\pgfqpoint{6.894984in}{1.957815in}}%
\pgfpathlineto{\pgfqpoint{6.897266in}{1.960560in}}%
\pgfpathlineto{\pgfqpoint{6.899548in}{1.960560in}}%
\pgfpathlineto{\pgfqpoint{6.901830in}{1.946746in}}%
\pgfpathlineto{\pgfqpoint{6.908675in}{1.937141in}}%
\pgfpathlineto{\pgfqpoint{6.910957in}{1.951412in}}%
\pgfpathlineto{\pgfqpoint{6.913239in}{1.953607in}}%
\pgfpathlineto{\pgfqpoint{6.915521in}{1.943270in}}%
\pgfpathlineto{\pgfqpoint{6.917803in}{1.943270in}}%
\pgfpathlineto{\pgfqpoint{6.924649in}{1.948850in}}%
\pgfpathlineto{\pgfqpoint{6.926931in}{1.945100in}}%
\pgfpathlineto{\pgfqpoint{6.929213in}{1.946563in}}%
\pgfpathlineto{\pgfqpoint{6.931495in}{1.941258in}}%
\pgfpathlineto{\pgfqpoint{6.933776in}{1.955986in}}%
\pgfpathlineto{\pgfqpoint{6.940622in}{1.952693in}}%
\pgfpathlineto{\pgfqpoint{6.942904in}{1.949857in}}%
\pgfpathlineto{\pgfqpoint{6.945186in}{1.961840in}}%
\pgfpathlineto{\pgfqpoint{6.947468in}{1.945466in}}%
\pgfpathlineto{\pgfqpoint{6.949750in}{1.939703in}}%
\pgfpathlineto{\pgfqpoint{6.956595in}{1.935861in}}%
\pgfpathlineto{\pgfqpoint{6.961159in}{1.941807in}}%
\pgfpathlineto{\pgfqpoint{6.963441in}{1.934946in}}%
\pgfpathlineto{\pgfqpoint{6.965723in}{1.940983in}}%
\pgfpathlineto{\pgfqpoint{6.974851in}{1.954431in}}%
\pgfpathlineto{\pgfqpoint{6.977133in}{1.961383in}}%
\pgfpathlineto{\pgfqpoint{6.979415in}{1.959188in}}%
\pgfpathlineto{\pgfqpoint{6.981696in}{1.968152in}}%
\pgfpathlineto{\pgfqpoint{6.988542in}{1.967238in}}%
\pgfpathlineto{\pgfqpoint{6.993106in}{1.980045in}}%
\pgfpathlineto{\pgfqpoint{6.995388in}{1.978764in}}%
\pgfpathlineto{\pgfqpoint{6.997670in}{1.992211in}}%
\pgfpathlineto{\pgfqpoint{7.004516in}{1.999164in}}%
\pgfpathlineto{\pgfqpoint{7.006797in}{1.995779in}}%
\pgfpathlineto{\pgfqpoint{7.009079in}{2.003555in}}%
\pgfpathlineto{\pgfqpoint{7.011361in}{1.991937in}}%
\pgfpathlineto{\pgfqpoint{7.013643in}{1.988186in}}%
\pgfpathlineto{\pgfqpoint{7.020489in}{1.991845in}}%
\pgfpathlineto{\pgfqpoint{7.022771in}{2.003738in}}%
\pgfpathlineto{\pgfqpoint{7.025053in}{2.007397in}}%
\pgfpathlineto{\pgfqpoint{7.027335in}{2.001176in}}%
\pgfpathlineto{\pgfqpoint{7.029617in}{2.003921in}}%
\pgfpathlineto{\pgfqpoint{7.036462in}{2.009318in}}%
\pgfpathlineto{\pgfqpoint{7.038744in}{2.007397in}}%
\pgfpathlineto{\pgfqpoint{7.041026in}{2.004652in}}%
\pgfpathlineto{\pgfqpoint{7.043308in}{1.991297in}}%
\pgfpathlineto{\pgfqpoint{7.054717in}{2.019563in}}%
\pgfpathlineto{\pgfqpoint{7.056999in}{2.027339in}}%
\pgfpathlineto{\pgfqpoint{7.059281in}{2.016819in}}%
\pgfpathlineto{\pgfqpoint{7.061563in}{2.020021in}}%
\pgfpathlineto{\pgfqpoint{7.068409in}{2.026424in}}%
\pgfpathlineto{\pgfqpoint{7.070691in}{2.034108in}}%
\pgfpathlineto{\pgfqpoint{7.072973in}{2.026058in}}%
\pgfpathlineto{\pgfqpoint{7.075255in}{2.026424in}}%
\pgfpathlineto{\pgfqpoint{7.084382in}{2.031456in}}%
\pgfpathlineto{\pgfqpoint{7.088946in}{2.030998in}}%
\pgfpathlineto{\pgfqpoint{7.091228in}{2.029169in}}%
\pgfpathlineto{\pgfqpoint{7.093510in}{2.032553in}}%
\pgfpathlineto{\pgfqpoint{7.102638in}{2.024503in}}%
\pgfpathlineto{\pgfqpoint{7.104919in}{2.026058in}}%
\pgfpathlineto{\pgfqpoint{7.107201in}{2.038499in}}%
\pgfpathlineto{\pgfqpoint{7.109483in}{2.037402in}}%
\pgfpathlineto{\pgfqpoint{7.116329in}{2.052953in}}%
\pgfpathlineto{\pgfqpoint{7.118611in}{2.053502in}}%
\pgfpathlineto{\pgfqpoint{7.120893in}{2.050026in}}%
\pgfpathlineto{\pgfqpoint{7.123175in}{2.052313in}}%
\pgfpathlineto{\pgfqpoint{7.125457in}{2.045269in}}%
\pgfpathlineto{\pgfqpoint{7.132302in}{2.040969in}}%
\pgfpathlineto{\pgfqpoint{7.134584in}{2.046184in}}%
\pgfpathlineto{\pgfqpoint{7.136866in}{2.042250in}}%
\pgfpathlineto{\pgfqpoint{7.141430in}{2.047556in}}%
\pgfpathlineto{\pgfqpoint{7.148276in}{2.024595in}}%
\pgfpathlineto{\pgfqpoint{7.150558in}{2.023863in}}%
\pgfpathlineto{\pgfqpoint{7.152839in}{2.032462in}}%
\pgfpathlineto{\pgfqpoint{7.157403in}{2.044720in}}%
\pgfpathlineto{\pgfqpoint{7.164249in}{2.046458in}}%
\pgfpathlineto{\pgfqpoint{7.166531in}{2.047830in}}%
\pgfpathlineto{\pgfqpoint{7.168813in}{2.045360in}}%
\pgfpathlineto{\pgfqpoint{7.171095in}{2.053776in}}%
\pgfpathlineto{\pgfqpoint{7.173377in}{2.058167in}}%
\pgfpathlineto{\pgfqpoint{7.180222in}{2.060729in}}%
\pgfpathlineto{\pgfqpoint{7.182504in}{2.063382in}}%
\pgfpathlineto{\pgfqpoint{7.184786in}{2.073627in}}%
\pgfpathlineto{\pgfqpoint{7.187068in}{2.070608in}}%
\pgfpathlineto{\pgfqpoint{7.189350in}{2.075640in}}%
\pgfpathlineto{\pgfqpoint{7.196196in}{2.072347in}}%
\pgfpathlineto{\pgfqpoint{7.198478in}{2.065486in}}%
\pgfpathlineto{\pgfqpoint{7.200760in}{2.067864in}}%
\pgfpathlineto{\pgfqpoint{7.203041in}{2.060820in}}%
\pgfpathlineto{\pgfqpoint{7.205323in}{2.065577in}}%
\pgfpathlineto{\pgfqpoint{7.212169in}{2.065394in}}%
\pgfpathlineto{\pgfqpoint{7.216733in}{2.083141in}}%
\pgfpathlineto{\pgfqpoint{7.219015in}{2.085885in}}%
\pgfpathlineto{\pgfqpoint{7.221297in}{2.085062in}}%
\pgfpathlineto{\pgfqpoint{7.228142in}{2.088081in}}%
\pgfpathlineto{\pgfqpoint{7.230424in}{2.087532in}}%
\pgfpathlineto{\pgfqpoint{7.232706in}{2.097412in}}%
\pgfpathlineto{\pgfqpoint{7.234988in}{2.096588in}}%
\pgfpathlineto{\pgfqpoint{7.237270in}{2.100065in}}%
\pgfpathlineto{\pgfqpoint{7.246398in}{2.106651in}}%
\pgfpathlineto{\pgfqpoint{7.248680in}{2.110036in}}%
\pgfpathlineto{\pgfqpoint{7.253243in}{2.104730in}}%
\pgfpathlineto{\pgfqpoint{7.260089in}{2.100430in}}%
\pgfpathlineto{\pgfqpoint{7.262371in}{2.105645in}}%
\pgfpathlineto{\pgfqpoint{7.264653in}{2.090002in}}%
\pgfpathlineto{\pgfqpoint{7.266935in}{2.098692in}}%
\pgfpathlineto{\pgfqpoint{7.269217in}{2.087898in}}%
\pgfpathlineto{\pgfqpoint{7.276062in}{2.088996in}}%
\pgfpathlineto{\pgfqpoint{7.278344in}{2.102626in}}%
\pgfpathlineto{\pgfqpoint{7.280626in}{2.096954in}}%
\pgfpathlineto{\pgfqpoint{7.285190in}{2.107291in}}%
\pgfpathlineto{\pgfqpoint{7.292036in}{2.111042in}}%
\pgfpathlineto{\pgfqpoint{7.294318in}{2.119915in}}%
\pgfpathlineto{\pgfqpoint{7.296600in}{2.088264in}}%
\pgfpathlineto{\pgfqpoint{7.298882in}{2.112963in}}%
\pgfpathlineto{\pgfqpoint{7.301163in}{2.096131in}}%
\pgfpathlineto{\pgfqpoint{7.308009in}{2.066217in}}%
\pgfpathlineto{\pgfqpoint{7.312573in}{2.087166in}}%
\pgfpathlineto{\pgfqpoint{7.314855in}{2.102169in}}%
\pgfpathlineto{\pgfqpoint{7.317137in}{2.113054in}}%
\pgfpathlineto{\pgfqpoint{7.323982in}{2.110950in}}%
\pgfpathlineto{\pgfqpoint{7.326264in}{2.120830in}}%
\pgfpathlineto{\pgfqpoint{7.328546in}{2.119458in}}%
\pgfpathlineto{\pgfqpoint{7.330828in}{2.115982in}}%
\pgfpathlineto{\pgfqpoint{7.333110in}{2.124123in}}%
\pgfpathlineto{\pgfqpoint{7.339956in}{2.121653in}}%
\pgfpathlineto{\pgfqpoint{7.342238in}{2.108846in}}%
\pgfpathlineto{\pgfqpoint{7.344520in}{2.108664in}}%
\pgfpathlineto{\pgfqpoint{7.349083in}{2.113878in}}%
\pgfpathlineto{\pgfqpoint{7.358211in}{2.116531in}}%
\pgfpathlineto{\pgfqpoint{7.360493in}{2.125861in}}%
\pgfpathlineto{\pgfqpoint{7.362775in}{2.128880in}}%
\pgfpathlineto{\pgfqpoint{7.365057in}{2.125861in}}%
\pgfpathlineto{\pgfqpoint{7.365057in}{2.125861in}}%
\pgfusepath{stroke}%
\end{pgfscope}%
\begin{pgfscope}%
\pgfsetrectcap%
\pgfsetmiterjoin%
\pgfsetlinewidth{0.803000pt}%
\definecolor{currentstroke}{rgb}{1.000000,1.000000,1.000000}%
\pgfsetstrokecolor{currentstroke}%
\pgfsetdash{}{0pt}%
\pgfpathmoveto{\pgfqpoint{2.125000in}{1.250000in}}%
\pgfpathlineto{\pgfqpoint{2.125000in}{2.170732in}}%
\pgfusepath{stroke}%
\end{pgfscope}%
\begin{pgfscope}%
\pgfsetrectcap%
\pgfsetmiterjoin%
\pgfsetlinewidth{0.803000pt}%
\definecolor{currentstroke}{rgb}{1.000000,1.000000,1.000000}%
\pgfsetstrokecolor{currentstroke}%
\pgfsetdash{}{0pt}%
\pgfpathmoveto{\pgfqpoint{7.614583in}{1.250000in}}%
\pgfpathlineto{\pgfqpoint{7.614583in}{2.170732in}}%
\pgfusepath{stroke}%
\end{pgfscope}%
\begin{pgfscope}%
\pgfsetrectcap%
\pgfsetmiterjoin%
\pgfsetlinewidth{0.803000pt}%
\definecolor{currentstroke}{rgb}{1.000000,1.000000,1.000000}%
\pgfsetstrokecolor{currentstroke}%
\pgfsetdash{}{0pt}%
\pgfpathmoveto{\pgfqpoint{2.125000in}{1.250000in}}%
\pgfpathlineto{\pgfqpoint{7.614583in}{1.250000in}}%
\pgfusepath{stroke}%
\end{pgfscope}%
\begin{pgfscope}%
\pgfsetrectcap%
\pgfsetmiterjoin%
\pgfsetlinewidth{0.803000pt}%
\definecolor{currentstroke}{rgb}{1.000000,1.000000,1.000000}%
\pgfsetstrokecolor{currentstroke}%
\pgfsetdash{}{0pt}%
\pgfpathmoveto{\pgfqpoint{2.125000in}{2.170732in}}%
\pgfpathlineto{\pgfqpoint{7.614583in}{2.170732in}}%
\pgfusepath{stroke}%
\end{pgfscope}%
\begin{pgfscope}%
\definecolor{textcolor}{rgb}{0.150000,0.150000,0.150000}%
\pgfsetstrokecolor{textcolor}%
\pgfsetfillcolor{textcolor}%
\pgftext[x=4.869792in,y=2.254065in,,base]{\color{textcolor}\rmfamily\fontsize{12.000000}{14.400000}\selectfont V}%
\end{pgfscope}%
\begin{pgfscope}%
\pgfsetbuttcap%
\pgfsetmiterjoin%
\definecolor{currentfill}{rgb}{0.917647,0.917647,0.949020}%
\pgfsetfillcolor{currentfill}%
\pgfsetlinewidth{0.000000pt}%
\definecolor{currentstroke}{rgb}{0.000000,0.000000,0.000000}%
\pgfsetstrokecolor{currentstroke}%
\pgfsetstrokeopacity{0.000000}%
\pgfsetdash{}{0pt}%
\pgfpathmoveto{\pgfqpoint{9.810417in}{1.250000in}}%
\pgfpathlineto{\pgfqpoint{15.300000in}{1.250000in}}%
\pgfpathlineto{\pgfqpoint{15.300000in}{2.170732in}}%
\pgfpathlineto{\pgfqpoint{9.810417in}{2.170732in}}%
\pgfpathclose%
\pgfusepath{fill}%
\end{pgfscope}%
\begin{pgfscope}%
\pgfpathrectangle{\pgfqpoint{9.810417in}{1.250000in}}{\pgfqpoint{5.489583in}{0.920732in}}%
\pgfusepath{clip}%
\pgfsetroundcap%
\pgfsetroundjoin%
\pgfsetlinewidth{0.803000pt}%
\definecolor{currentstroke}{rgb}{1.000000,1.000000,1.000000}%
\pgfsetstrokecolor{currentstroke}%
\pgfsetdash{}{0pt}%
\pgfpathmoveto{\pgfqpoint{10.055379in}{1.250000in}}%
\pgfpathlineto{\pgfqpoint{10.055379in}{2.170732in}}%
\pgfusepath{stroke}%
\end{pgfscope}%
\begin{pgfscope}%
\definecolor{textcolor}{rgb}{0.150000,0.150000,0.150000}%
\pgfsetstrokecolor{textcolor}%
\pgfsetfillcolor{textcolor}%
\pgftext[x=10.055379in,y=1.152778in,,top]{\color{textcolor}\rmfamily\fontsize{10.000000}{12.000000}\selectfont 2012}%
\end{pgfscope}%
\begin{pgfscope}%
\pgfpathrectangle{\pgfqpoint{9.810417in}{1.250000in}}{\pgfqpoint{5.489583in}{0.920732in}}%
\pgfusepath{clip}%
\pgfsetroundcap%
\pgfsetroundjoin%
\pgfsetlinewidth{0.803000pt}%
\definecolor{currentstroke}{rgb}{1.000000,1.000000,1.000000}%
\pgfsetstrokecolor{currentstroke}%
\pgfsetdash{}{0pt}%
\pgfpathmoveto{\pgfqpoint{10.890557in}{1.250000in}}%
\pgfpathlineto{\pgfqpoint{10.890557in}{2.170732in}}%
\pgfusepath{stroke}%
\end{pgfscope}%
\begin{pgfscope}%
\definecolor{textcolor}{rgb}{0.150000,0.150000,0.150000}%
\pgfsetstrokecolor{textcolor}%
\pgfsetfillcolor{textcolor}%
\pgftext[x=10.890557in,y=1.152778in,,top]{\color{textcolor}\rmfamily\fontsize{10.000000}{12.000000}\selectfont 2013}%
\end{pgfscope}%
\begin{pgfscope}%
\pgfpathrectangle{\pgfqpoint{9.810417in}{1.250000in}}{\pgfqpoint{5.489583in}{0.920732in}}%
\pgfusepath{clip}%
\pgfsetroundcap%
\pgfsetroundjoin%
\pgfsetlinewidth{0.803000pt}%
\definecolor{currentstroke}{rgb}{1.000000,1.000000,1.000000}%
\pgfsetstrokecolor{currentstroke}%
\pgfsetdash{}{0pt}%
\pgfpathmoveto{\pgfqpoint{11.723453in}{1.250000in}}%
\pgfpathlineto{\pgfqpoint{11.723453in}{2.170732in}}%
\pgfusepath{stroke}%
\end{pgfscope}%
\begin{pgfscope}%
\definecolor{textcolor}{rgb}{0.150000,0.150000,0.150000}%
\pgfsetstrokecolor{textcolor}%
\pgfsetfillcolor{textcolor}%
\pgftext[x=11.723453in,y=1.152778in,,top]{\color{textcolor}\rmfamily\fontsize{10.000000}{12.000000}\selectfont 2014}%
\end{pgfscope}%
\begin{pgfscope}%
\pgfpathrectangle{\pgfqpoint{9.810417in}{1.250000in}}{\pgfqpoint{5.489583in}{0.920732in}}%
\pgfusepath{clip}%
\pgfsetroundcap%
\pgfsetroundjoin%
\pgfsetlinewidth{0.803000pt}%
\definecolor{currentstroke}{rgb}{1.000000,1.000000,1.000000}%
\pgfsetstrokecolor{currentstroke}%
\pgfsetdash{}{0pt}%
\pgfpathmoveto{\pgfqpoint{12.556349in}{1.250000in}}%
\pgfpathlineto{\pgfqpoint{12.556349in}{2.170732in}}%
\pgfusepath{stroke}%
\end{pgfscope}%
\begin{pgfscope}%
\definecolor{textcolor}{rgb}{0.150000,0.150000,0.150000}%
\pgfsetstrokecolor{textcolor}%
\pgfsetfillcolor{textcolor}%
\pgftext[x=12.556349in,y=1.152778in,,top]{\color{textcolor}\rmfamily\fontsize{10.000000}{12.000000}\selectfont 2015}%
\end{pgfscope}%
\begin{pgfscope}%
\pgfpathrectangle{\pgfqpoint{9.810417in}{1.250000in}}{\pgfqpoint{5.489583in}{0.920732in}}%
\pgfusepath{clip}%
\pgfsetroundcap%
\pgfsetroundjoin%
\pgfsetlinewidth{0.803000pt}%
\definecolor{currentstroke}{rgb}{1.000000,1.000000,1.000000}%
\pgfsetstrokecolor{currentstroke}%
\pgfsetdash{}{0pt}%
\pgfpathmoveto{\pgfqpoint{13.389245in}{1.250000in}}%
\pgfpathlineto{\pgfqpoint{13.389245in}{2.170732in}}%
\pgfusepath{stroke}%
\end{pgfscope}%
\begin{pgfscope}%
\definecolor{textcolor}{rgb}{0.150000,0.150000,0.150000}%
\pgfsetstrokecolor{textcolor}%
\pgfsetfillcolor{textcolor}%
\pgftext[x=13.389245in,y=1.152778in,,top]{\color{textcolor}\rmfamily\fontsize{10.000000}{12.000000}\selectfont 2016}%
\end{pgfscope}%
\begin{pgfscope}%
\pgfpathrectangle{\pgfqpoint{9.810417in}{1.250000in}}{\pgfqpoint{5.489583in}{0.920732in}}%
\pgfusepath{clip}%
\pgfsetroundcap%
\pgfsetroundjoin%
\pgfsetlinewidth{0.803000pt}%
\definecolor{currentstroke}{rgb}{1.000000,1.000000,1.000000}%
\pgfsetstrokecolor{currentstroke}%
\pgfsetdash{}{0pt}%
\pgfpathmoveto{\pgfqpoint{14.224423in}{1.250000in}}%
\pgfpathlineto{\pgfqpoint{14.224423in}{2.170732in}}%
\pgfusepath{stroke}%
\end{pgfscope}%
\begin{pgfscope}%
\definecolor{textcolor}{rgb}{0.150000,0.150000,0.150000}%
\pgfsetstrokecolor{textcolor}%
\pgfsetfillcolor{textcolor}%
\pgftext[x=14.224423in,y=1.152778in,,top]{\color{textcolor}\rmfamily\fontsize{10.000000}{12.000000}\selectfont 2017}%
\end{pgfscope}%
\begin{pgfscope}%
\pgfpathrectangle{\pgfqpoint{9.810417in}{1.250000in}}{\pgfqpoint{5.489583in}{0.920732in}}%
\pgfusepath{clip}%
\pgfsetroundcap%
\pgfsetroundjoin%
\pgfsetlinewidth{0.803000pt}%
\definecolor{currentstroke}{rgb}{1.000000,1.000000,1.000000}%
\pgfsetstrokecolor{currentstroke}%
\pgfsetdash{}{0pt}%
\pgfpathmoveto{\pgfqpoint{15.057319in}{1.250000in}}%
\pgfpathlineto{\pgfqpoint{15.057319in}{2.170732in}}%
\pgfusepath{stroke}%
\end{pgfscope}%
\begin{pgfscope}%
\definecolor{textcolor}{rgb}{0.150000,0.150000,0.150000}%
\pgfsetstrokecolor{textcolor}%
\pgfsetfillcolor{textcolor}%
\pgftext[x=15.057319in,y=1.152778in,,top]{\color{textcolor}\rmfamily\fontsize{10.000000}{12.000000}\selectfont 2018}%
\end{pgfscope}%
\begin{pgfscope}%
\pgfpathrectangle{\pgfqpoint{9.810417in}{1.250000in}}{\pgfqpoint{5.489583in}{0.920732in}}%
\pgfusepath{clip}%
\pgfsetroundcap%
\pgfsetroundjoin%
\pgfsetlinewidth{0.803000pt}%
\definecolor{currentstroke}{rgb}{1.000000,1.000000,1.000000}%
\pgfsetstrokecolor{currentstroke}%
\pgfsetdash{}{0pt}%
\pgfpathmoveto{\pgfqpoint{9.810417in}{1.453416in}}%
\pgfpathlineto{\pgfqpoint{15.300000in}{1.453416in}}%
\pgfusepath{stroke}%
\end{pgfscope}%
\begin{pgfscope}%
\definecolor{textcolor}{rgb}{0.150000,0.150000,0.150000}%
\pgfsetstrokecolor{textcolor}%
\pgfsetfillcolor{textcolor}%
\pgftext[x=9.536464in,y=1.400655in,left,base]{\color{textcolor}\rmfamily\fontsize{10.000000}{12.000000}\selectfont 50}%
\end{pgfscope}%
\begin{pgfscope}%
\pgfpathrectangle{\pgfqpoint{9.810417in}{1.250000in}}{\pgfqpoint{5.489583in}{0.920732in}}%
\pgfusepath{clip}%
\pgfsetroundcap%
\pgfsetroundjoin%
\pgfsetlinewidth{0.803000pt}%
\definecolor{currentstroke}{rgb}{1.000000,1.000000,1.000000}%
\pgfsetstrokecolor{currentstroke}%
\pgfsetdash{}{0pt}%
\pgfpathmoveto{\pgfqpoint{9.810417in}{1.974930in}}%
\pgfpathlineto{\pgfqpoint{15.300000in}{1.974930in}}%
\pgfusepath{stroke}%
\end{pgfscope}%
\begin{pgfscope}%
\definecolor{textcolor}{rgb}{0.150000,0.150000,0.150000}%
\pgfsetstrokecolor{textcolor}%
\pgfsetfillcolor{textcolor}%
\pgftext[x=9.448098in,y=1.922168in,left,base]{\color{textcolor}\rmfamily\fontsize{10.000000}{12.000000}\selectfont 100}%
\end{pgfscope}%
\begin{pgfscope}%
\pgfpathrectangle{\pgfqpoint{9.810417in}{1.250000in}}{\pgfqpoint{5.489583in}{0.920732in}}%
\pgfusepath{clip}%
\pgfsetroundcap%
\pgfsetroundjoin%
\pgfsetlinewidth{1.505625pt}%
\definecolor{currentstroke}{rgb}{0.121569,0.466667,0.705882}%
\pgfsetstrokecolor{currentstroke}%
\pgfsetdash{}{0pt}%
\pgfpathmoveto{\pgfqpoint{10.059943in}{1.291851in}}%
\pgfpathlineto{\pgfqpoint{10.066789in}{1.306871in}}%
\pgfpathlineto{\pgfqpoint{10.075917in}{1.304263in}}%
\pgfpathlineto{\pgfqpoint{10.078198in}{1.295502in}}%
\pgfpathlineto{\pgfqpoint{10.080480in}{1.295711in}}%
\pgfpathlineto{\pgfqpoint{10.082762in}{1.292686in}}%
\pgfpathlineto{\pgfqpoint{10.091890in}{1.293416in}}%
\pgfpathlineto{\pgfqpoint{10.096454in}{1.302386in}}%
\pgfpathlineto{\pgfqpoint{10.098736in}{1.301239in}}%
\pgfpathlineto{\pgfqpoint{10.107863in}{1.300613in}}%
\pgfpathlineto{\pgfqpoint{10.110145in}{1.303533in}}%
\pgfpathlineto{\pgfqpoint{10.112427in}{1.301552in}}%
\pgfpathlineto{\pgfqpoint{10.123837in}{1.297379in}}%
\pgfpathlineto{\pgfqpoint{10.126118in}{1.301343in}}%
\pgfpathlineto{\pgfqpoint{10.128400in}{1.297484in}}%
\pgfpathlineto{\pgfqpoint{10.130682in}{1.307705in}}%
\pgfpathlineto{\pgfqpoint{10.137528in}{1.311982in}}%
\pgfpathlineto{\pgfqpoint{10.139810in}{1.316884in}}%
\pgfpathlineto{\pgfqpoint{10.144374in}{1.322099in}}%
\pgfpathlineto{\pgfqpoint{10.146656in}{1.321265in}}%
\pgfpathlineto{\pgfqpoint{10.153501in}{1.324498in}}%
\pgfpathlineto{\pgfqpoint{10.155783in}{1.322725in}}%
\pgfpathlineto{\pgfqpoint{10.158065in}{1.319387in}}%
\pgfpathlineto{\pgfqpoint{10.160347in}{1.322204in}}%
\pgfpathlineto{\pgfqpoint{10.162629in}{1.324081in}}%
\pgfpathlineto{\pgfqpoint{10.171757in}{1.322412in}}%
\pgfpathlineto{\pgfqpoint{10.174039in}{1.319596in}}%
\pgfpathlineto{\pgfqpoint{10.176320in}{1.321578in}}%
\pgfpathlineto{\pgfqpoint{10.178602in}{1.320013in}}%
\pgfpathlineto{\pgfqpoint{10.185448in}{1.323142in}}%
\pgfpathlineto{\pgfqpoint{10.187730in}{1.325854in}}%
\pgfpathlineto{\pgfqpoint{10.190012in}{1.326376in}}%
\pgfpathlineto{\pgfqpoint{10.192294in}{1.330131in}}%
\pgfpathlineto{\pgfqpoint{10.194576in}{1.329818in}}%
\pgfpathlineto{\pgfqpoint{10.201421in}{1.333051in}}%
\pgfpathlineto{\pgfqpoint{10.203703in}{1.326480in}}%
\pgfpathlineto{\pgfqpoint{10.205985in}{1.324081in}}%
\pgfpathlineto{\pgfqpoint{10.210549in}{1.328775in}}%
\pgfpathlineto{\pgfqpoint{10.217395in}{1.329713in}}%
\pgfpathlineto{\pgfqpoint{10.219677in}{1.345359in}}%
\pgfpathlineto{\pgfqpoint{10.221959in}{1.340352in}}%
\pgfpathlineto{\pgfqpoint{10.224240in}{1.340248in}}%
\pgfpathlineto{\pgfqpoint{10.226522in}{1.337640in}}%
\pgfpathlineto{\pgfqpoint{10.233368in}{1.340039in}}%
\pgfpathlineto{\pgfqpoint{10.235650in}{1.338162in}}%
\pgfpathlineto{\pgfqpoint{10.240214in}{1.338579in}}%
\pgfpathlineto{\pgfqpoint{10.242496in}{1.342021in}}%
\pgfpathlineto{\pgfqpoint{10.249341in}{1.348801in}}%
\pgfpathlineto{\pgfqpoint{10.251623in}{1.346715in}}%
\pgfpathlineto{\pgfqpoint{10.256187in}{1.335971in}}%
\pgfpathlineto{\pgfqpoint{10.258469in}{1.343168in}}%
\pgfpathlineto{\pgfqpoint{10.265315in}{1.343794in}}%
\pgfpathlineto{\pgfqpoint{10.267597in}{1.338579in}}%
\pgfpathlineto{\pgfqpoint{10.269879in}{1.335241in}}%
\pgfpathlineto{\pgfqpoint{10.272161in}{1.336597in}}%
\pgfpathlineto{\pgfqpoint{10.281288in}{1.327523in}}%
\pgfpathlineto{\pgfqpoint{10.283570in}{1.316988in}}%
\pgfpathlineto{\pgfqpoint{10.285852in}{1.320535in}}%
\pgfpathlineto{\pgfqpoint{10.288134in}{1.327836in}}%
\pgfpathlineto{\pgfqpoint{10.290416in}{1.325020in}}%
\pgfpathlineto{\pgfqpoint{10.297261in}{1.323247in}}%
\pgfpathlineto{\pgfqpoint{10.299543in}{1.332842in}}%
\pgfpathlineto{\pgfqpoint{10.301825in}{1.331069in}}%
\pgfpathlineto{\pgfqpoint{10.304107in}{1.327210in}}%
\pgfpathlineto{\pgfqpoint{10.306389in}{1.329713in}}%
\pgfpathlineto{\pgfqpoint{10.313235in}{1.326584in}}%
\pgfpathlineto{\pgfqpoint{10.315517in}{1.328149in}}%
\pgfpathlineto{\pgfqpoint{10.320081in}{1.339205in}}%
\pgfpathlineto{\pgfqpoint{10.322362in}{1.339205in}}%
\pgfpathlineto{\pgfqpoint{10.329208in}{1.336910in}}%
\pgfpathlineto{\pgfqpoint{10.331490in}{1.343273in}}%
\pgfpathlineto{\pgfqpoint{10.333772in}{1.340978in}}%
\pgfpathlineto{\pgfqpoint{10.336054in}{1.343481in}}%
\pgfpathlineto{\pgfqpoint{10.338336in}{1.335241in}}%
\pgfpathlineto{\pgfqpoint{10.345182in}{1.343586in}}%
\pgfpathlineto{\pgfqpoint{10.347463in}{1.348071in}}%
\pgfpathlineto{\pgfqpoint{10.349745in}{1.354850in}}%
\pgfpathlineto{\pgfqpoint{10.354309in}{1.359961in}}%
\pgfpathlineto{\pgfqpoint{10.363437in}{1.354746in}}%
\pgfpathlineto{\pgfqpoint{10.365719in}{1.355372in}}%
\pgfpathlineto{\pgfqpoint{10.368001in}{1.348383in}}%
\pgfpathlineto{\pgfqpoint{10.370283in}{1.343481in}}%
\pgfpathlineto{\pgfqpoint{10.377128in}{1.348905in}}%
\pgfpathlineto{\pgfqpoint{10.379410in}{1.348905in}}%
\pgfpathlineto{\pgfqpoint{10.381692in}{1.347236in}}%
\pgfpathlineto{\pgfqpoint{10.383974in}{1.349427in}}%
\pgfpathlineto{\pgfqpoint{10.386256in}{1.349948in}}%
\pgfpathlineto{\pgfqpoint{10.395383in}{1.359127in}}%
\pgfpathlineto{\pgfqpoint{10.397665in}{1.356519in}}%
\pgfpathlineto{\pgfqpoint{10.399947in}{1.361317in}}%
\pgfpathlineto{\pgfqpoint{10.402229in}{1.349009in}}%
\pgfpathlineto{\pgfqpoint{10.409075in}{1.349114in}}%
\pgfpathlineto{\pgfqpoint{10.411357in}{1.353181in}}%
\pgfpathlineto{\pgfqpoint{10.413639in}{1.359648in}}%
\pgfpathlineto{\pgfqpoint{10.415921in}{1.360587in}}%
\pgfpathlineto{\pgfqpoint{10.418203in}{1.366324in}}%
\pgfpathlineto{\pgfqpoint{10.425048in}{1.362151in}}%
\pgfpathlineto{\pgfqpoint{10.427330in}{1.367575in}}%
\pgfpathlineto{\pgfqpoint{10.429612in}{1.366219in}}%
\pgfpathlineto{\pgfqpoint{10.431894in}{1.375189in}}%
\pgfpathlineto{\pgfqpoint{10.434176in}{1.374251in}}%
\pgfpathlineto{\pgfqpoint{10.441022in}{1.374355in}}%
\pgfpathlineto{\pgfqpoint{10.443304in}{1.378214in}}%
\pgfpathlineto{\pgfqpoint{10.445585in}{1.380300in}}%
\pgfpathlineto{\pgfqpoint{10.447867in}{1.377171in}}%
\pgfpathlineto{\pgfqpoint{10.450149in}{1.377901in}}%
\pgfpathlineto{\pgfqpoint{10.456995in}{1.370600in}}%
\pgfpathlineto{\pgfqpoint{10.461559in}{1.381656in}}%
\pgfpathlineto{\pgfqpoint{10.463841in}{1.381135in}}%
\pgfpathlineto{\pgfqpoint{10.466123in}{1.387497in}}%
\pgfpathlineto{\pgfqpoint{10.472968in}{1.389583in}}%
\pgfpathlineto{\pgfqpoint{10.475250in}{1.388436in}}%
\pgfpathlineto{\pgfqpoint{10.479814in}{1.383846in}}%
\pgfpathlineto{\pgfqpoint{10.482096in}{1.383221in}}%
\pgfpathlineto{\pgfqpoint{10.488942in}{1.382803in}}%
\pgfpathlineto{\pgfqpoint{10.491224in}{1.376858in}}%
\pgfpathlineto{\pgfqpoint{10.493505in}{1.376024in}}%
\pgfpathlineto{\pgfqpoint{10.495787in}{1.377275in}}%
\pgfpathlineto{\pgfqpoint{10.498069in}{1.384681in}}%
\pgfpathlineto{\pgfqpoint{10.504915in}{1.381552in}}%
\pgfpathlineto{\pgfqpoint{10.507197in}{1.395528in}}%
\pgfpathlineto{\pgfqpoint{10.509479in}{1.395528in}}%
\pgfpathlineto{\pgfqpoint{10.514043in}{1.388436in}}%
\pgfpathlineto{\pgfqpoint{10.520888in}{1.382699in}}%
\pgfpathlineto{\pgfqpoint{10.525452in}{1.385620in}}%
\pgfpathlineto{\pgfqpoint{10.527734in}{1.398866in}}%
\pgfpathlineto{\pgfqpoint{10.530016in}{1.401056in}}%
\pgfpathlineto{\pgfqpoint{10.536862in}{1.399805in}}%
\pgfpathlineto{\pgfqpoint{10.539144in}{1.393547in}}%
\pgfpathlineto{\pgfqpoint{10.541426in}{1.390522in}}%
\pgfpathlineto{\pgfqpoint{10.543707in}{1.392086in}}%
\pgfpathlineto{\pgfqpoint{10.545989in}{1.399492in}}%
\pgfpathlineto{\pgfqpoint{10.552835in}{1.398344in}}%
\pgfpathlineto{\pgfqpoint{10.555117in}{1.399805in}}%
\pgfpathlineto{\pgfqpoint{10.557399in}{1.406271in}}%
\pgfpathlineto{\pgfqpoint{10.559681in}{1.401265in}}%
\pgfpathlineto{\pgfqpoint{10.561963in}{1.398344in}}%
\pgfpathlineto{\pgfqpoint{10.568808in}{1.400326in}}%
\pgfpathlineto{\pgfqpoint{10.571090in}{1.398762in}}%
\pgfpathlineto{\pgfqpoint{10.573372in}{1.400639in}}%
\pgfpathlineto{\pgfqpoint{10.575654in}{1.403977in}}%
\pgfpathlineto{\pgfqpoint{10.577936in}{1.405959in}}%
\pgfpathlineto{\pgfqpoint{10.584782in}{1.405854in}}%
\pgfpathlineto{\pgfqpoint{10.587064in}{1.398240in}}%
\pgfpathlineto{\pgfqpoint{10.589346in}{1.398449in}}%
\pgfpathlineto{\pgfqpoint{10.591627in}{1.393859in}}%
\pgfpathlineto{\pgfqpoint{10.593909in}{1.397510in}}%
\pgfpathlineto{\pgfqpoint{10.603037in}{1.398136in}}%
\pgfpathlineto{\pgfqpoint{10.605319in}{1.401995in}}%
\pgfpathlineto{\pgfqpoint{10.607601in}{1.396154in}}%
\pgfpathlineto{\pgfqpoint{10.619010in}{1.398449in}}%
\pgfpathlineto{\pgfqpoint{10.623574in}{1.419101in}}%
\pgfpathlineto{\pgfqpoint{10.625856in}{1.417953in}}%
\pgfpathlineto{\pgfqpoint{10.632702in}{1.415867in}}%
\pgfpathlineto{\pgfqpoint{10.634984in}{1.416285in}}%
\pgfpathlineto{\pgfqpoint{10.637266in}{1.417640in}}%
\pgfpathlineto{\pgfqpoint{10.639548in}{1.426089in}}%
\pgfpathlineto{\pgfqpoint{10.641829in}{1.423690in}}%
\pgfpathlineto{\pgfqpoint{10.648675in}{1.421813in}}%
\pgfpathlineto{\pgfqpoint{10.650957in}{1.419518in}}%
\pgfpathlineto{\pgfqpoint{10.653239in}{1.427028in}}%
\pgfpathlineto{\pgfqpoint{10.655521in}{1.426610in}}%
\pgfpathlineto{\pgfqpoint{10.664648in}{1.429114in}}%
\pgfpathlineto{\pgfqpoint{10.666930in}{1.425463in}}%
\pgfpathlineto{\pgfqpoint{10.669212in}{1.419831in}}%
\pgfpathlineto{\pgfqpoint{10.671494in}{1.425880in}}%
\pgfpathlineto{\pgfqpoint{10.673776in}{1.423064in}}%
\pgfpathlineto{\pgfqpoint{10.680622in}{1.421082in}}%
\pgfpathlineto{\pgfqpoint{10.682904in}{1.417015in}}%
\pgfpathlineto{\pgfqpoint{10.685186in}{1.424837in}}%
\pgfpathlineto{\pgfqpoint{10.687468in}{1.426298in}}%
\pgfpathlineto{\pgfqpoint{10.689749in}{1.429531in}}%
\pgfpathlineto{\pgfqpoint{10.696595in}{1.423481in}}%
\pgfpathlineto{\pgfqpoint{10.698877in}{1.415867in}}%
\pgfpathlineto{\pgfqpoint{10.701159in}{1.413051in}}%
\pgfpathlineto{\pgfqpoint{10.703441in}{1.404811in}}%
\pgfpathlineto{\pgfqpoint{10.705723in}{1.407210in}}%
\pgfpathlineto{\pgfqpoint{10.712569in}{1.409088in}}%
\pgfpathlineto{\pgfqpoint{10.714850in}{1.413260in}}%
\pgfpathlineto{\pgfqpoint{10.717132in}{1.423168in}}%
\pgfpathlineto{\pgfqpoint{10.719414in}{1.424420in}}%
\pgfpathlineto{\pgfqpoint{10.721696in}{1.419518in}}%
\pgfpathlineto{\pgfqpoint{10.728542in}{1.418475in}}%
\pgfpathlineto{\pgfqpoint{10.730824in}{1.408775in}}%
\pgfpathlineto{\pgfqpoint{10.733106in}{1.407732in}}%
\pgfpathlineto{\pgfqpoint{10.735388in}{1.404081in}}%
\pgfpathlineto{\pgfqpoint{10.737670in}{1.402412in}}%
\pgfpathlineto{\pgfqpoint{10.749079in}{1.393338in}}%
\pgfpathlineto{\pgfqpoint{10.751361in}{1.399596in}}%
\pgfpathlineto{\pgfqpoint{10.753643in}{1.400326in}}%
\pgfpathlineto{\pgfqpoint{10.762770in}{1.406063in}}%
\pgfpathlineto{\pgfqpoint{10.765052in}{1.402412in}}%
\pgfpathlineto{\pgfqpoint{10.767334in}{1.401995in}}%
\pgfpathlineto{\pgfqpoint{10.769616in}{1.374042in}}%
\pgfpathlineto{\pgfqpoint{10.776462in}{1.377693in}}%
\pgfpathlineto{\pgfqpoint{10.778744in}{1.382490in}}%
\pgfpathlineto{\pgfqpoint{10.781026in}{1.375085in}}%
\pgfpathlineto{\pgfqpoint{10.783308in}{1.377901in}}%
\pgfpathlineto{\pgfqpoint{10.785590in}{1.377380in}}%
\pgfpathlineto{\pgfqpoint{10.792435in}{1.381969in}}%
\pgfpathlineto{\pgfqpoint{10.794717in}{1.386975in}}%
\pgfpathlineto{\pgfqpoint{10.796999in}{1.389270in}}%
\pgfpathlineto{\pgfqpoint{10.801563in}{1.394694in}}%
\pgfpathlineto{\pgfqpoint{10.808409in}{1.392504in}}%
\pgfpathlineto{\pgfqpoint{10.810691in}{1.388436in}}%
\pgfpathlineto{\pgfqpoint{10.815254in}{1.398970in}}%
\pgfpathlineto{\pgfqpoint{10.817536in}{1.398449in}}%
\pgfpathlineto{\pgfqpoint{10.824382in}{1.395007in}}%
\pgfpathlineto{\pgfqpoint{10.826664in}{1.395007in}}%
\pgfpathlineto{\pgfqpoint{10.831228in}{1.399909in}}%
\pgfpathlineto{\pgfqpoint{10.833510in}{1.401578in}}%
\pgfpathlineto{\pgfqpoint{10.840355in}{1.402204in}}%
\pgfpathlineto{\pgfqpoint{10.844919in}{1.405437in}}%
\pgfpathlineto{\pgfqpoint{10.849483in}{1.396154in}}%
\pgfpathlineto{\pgfqpoint{10.856329in}{1.401995in}}%
\pgfpathlineto{\pgfqpoint{10.858611in}{1.410965in}}%
\pgfpathlineto{\pgfqpoint{10.860892in}{1.408253in}}%
\pgfpathlineto{\pgfqpoint{10.863174in}{1.417745in}}%
\pgfpathlineto{\pgfqpoint{10.865456in}{1.408879in}}%
\pgfpathlineto{\pgfqpoint{10.876866in}{1.407419in}}%
\pgfpathlineto{\pgfqpoint{10.881430in}{1.400743in}}%
\pgfpathlineto{\pgfqpoint{10.888275in}{1.406793in}}%
\pgfpathlineto{\pgfqpoint{10.892839in}{1.419309in}}%
\pgfpathlineto{\pgfqpoint{10.895121in}{1.420352in}}%
\pgfpathlineto{\pgfqpoint{10.897403in}{1.429740in}}%
\pgfpathlineto{\pgfqpoint{10.904249in}{1.418058in}}%
\pgfpathlineto{\pgfqpoint{10.906531in}{1.416076in}}%
\pgfpathlineto{\pgfqpoint{10.911094in}{1.416389in}}%
\pgfpathlineto{\pgfqpoint{10.913376in}{1.414407in}}%
\pgfpathlineto{\pgfqpoint{10.920222in}{1.414511in}}%
\pgfpathlineto{\pgfqpoint{10.924786in}{1.423481in}}%
\pgfpathlineto{\pgfqpoint{10.927068in}{1.431826in}}%
\pgfpathlineto{\pgfqpoint{10.929350in}{1.431200in}}%
\pgfpathlineto{\pgfqpoint{10.938477in}{1.434850in}}%
\pgfpathlineto{\pgfqpoint{10.940759in}{1.446532in}}%
\pgfpathlineto{\pgfqpoint{10.943041in}{1.446532in}}%
\pgfpathlineto{\pgfqpoint{10.945323in}{1.450600in}}%
\pgfpathlineto{\pgfqpoint{10.952169in}{1.450391in}}%
\pgfpathlineto{\pgfqpoint{10.954451in}{1.446949in}}%
\pgfpathlineto{\pgfqpoint{10.956733in}{1.444968in}}%
\pgfpathlineto{\pgfqpoint{10.959014in}{1.445906in}}%
\pgfpathlineto{\pgfqpoint{10.961296in}{1.452582in}}%
\pgfpathlineto{\pgfqpoint{10.968142in}{1.446011in}}%
\pgfpathlineto{\pgfqpoint{10.970424in}{1.449766in}}%
\pgfpathlineto{\pgfqpoint{10.972706in}{1.451956in}}%
\pgfpathlineto{\pgfqpoint{10.974988in}{1.450391in}}%
\pgfpathlineto{\pgfqpoint{10.977270in}{1.453312in}}%
\pgfpathlineto{\pgfqpoint{10.984115in}{1.454146in}}%
\pgfpathlineto{\pgfqpoint{10.986397in}{1.456024in}}%
\pgfpathlineto{\pgfqpoint{10.988679in}{1.456128in}}%
\pgfpathlineto{\pgfqpoint{10.990961in}{1.455398in}}%
\pgfpathlineto{\pgfqpoint{10.993243in}{1.462386in}}%
\pgfpathlineto{\pgfqpoint{11.002371in}{1.463534in}}%
\pgfpathlineto{\pgfqpoint{11.004653in}{1.452686in}}%
\pgfpathlineto{\pgfqpoint{11.006935in}{1.448618in}}%
\pgfpathlineto{\pgfqpoint{11.009216in}{1.449348in}}%
\pgfpathlineto{\pgfqpoint{11.016062in}{1.443090in}}%
\pgfpathlineto{\pgfqpoint{11.018344in}{1.446011in}}%
\pgfpathlineto{\pgfqpoint{11.020626in}{1.451539in}}%
\pgfpathlineto{\pgfqpoint{11.022908in}{1.452582in}}%
\pgfpathlineto{\pgfqpoint{11.025190in}{1.459674in}}%
\pgfpathlineto{\pgfqpoint{11.032036in}{1.464159in}}%
\pgfpathlineto{\pgfqpoint{11.034317in}{1.470626in}}%
\pgfpathlineto{\pgfqpoint{11.036599in}{1.469479in}}%
\pgfpathlineto{\pgfqpoint{11.038881in}{1.469166in}}%
\pgfpathlineto{\pgfqpoint{11.041163in}{1.479388in}}%
\pgfpathlineto{\pgfqpoint{11.048009in}{1.481891in}}%
\pgfpathlineto{\pgfqpoint{11.050291in}{1.476676in}}%
\pgfpathlineto{\pgfqpoint{11.052573in}{1.478866in}}%
\pgfpathlineto{\pgfqpoint{11.054855in}{1.482725in}}%
\pgfpathlineto{\pgfqpoint{11.057136in}{1.481161in}}%
\pgfpathlineto{\pgfqpoint{11.063982in}{1.473964in}}%
\pgfpathlineto{\pgfqpoint{11.066264in}{1.469062in}}%
\pgfpathlineto{\pgfqpoint{11.068546in}{1.475007in}}%
\pgfpathlineto{\pgfqpoint{11.070828in}{1.469062in}}%
\pgfpathlineto{\pgfqpoint{11.073110in}{1.473547in}}%
\pgfpathlineto{\pgfqpoint{11.079956in}{1.468123in}}%
\pgfpathlineto{\pgfqpoint{11.082237in}{1.472086in}}%
\pgfpathlineto{\pgfqpoint{11.084519in}{1.470522in}}%
\pgfpathlineto{\pgfqpoint{11.086801in}{1.473755in}}%
\pgfpathlineto{\pgfqpoint{11.095929in}{1.472712in}}%
\pgfpathlineto{\pgfqpoint{11.098211in}{1.480013in}}%
\pgfpathlineto{\pgfqpoint{11.100493in}{1.478032in}}%
\pgfpathlineto{\pgfqpoint{11.102775in}{1.481265in}}%
\pgfpathlineto{\pgfqpoint{11.105057in}{1.482308in}}%
\pgfpathlineto{\pgfqpoint{11.114184in}{1.496076in}}%
\pgfpathlineto{\pgfqpoint{11.116466in}{1.505255in}}%
\pgfpathlineto{\pgfqpoint{11.118748in}{1.509531in}}%
\pgfpathlineto{\pgfqpoint{11.121030in}{1.509531in}}%
\pgfpathlineto{\pgfqpoint{11.127876in}{1.493573in}}%
\pgfpathlineto{\pgfqpoint{11.130157in}{1.511409in}}%
\pgfpathlineto{\pgfqpoint{11.132439in}{1.510678in}}%
\pgfpathlineto{\pgfqpoint{11.134721in}{1.504107in}}%
\pgfpathlineto{\pgfqpoint{11.137003in}{1.519127in}}%
\pgfpathlineto{\pgfqpoint{11.143849in}{1.523403in}}%
\pgfpathlineto{\pgfqpoint{11.146131in}{1.528931in}}%
\pgfpathlineto{\pgfqpoint{11.148413in}{1.522778in}}%
\pgfpathlineto{\pgfqpoint{11.150695in}{1.523299in}}%
\pgfpathlineto{\pgfqpoint{11.152977in}{1.522047in}}%
\pgfpathlineto{\pgfqpoint{11.159822in}{1.532895in}}%
\pgfpathlineto{\pgfqpoint{11.162104in}{1.531330in}}%
\pgfpathlineto{\pgfqpoint{11.164386in}{1.534877in}}%
\pgfpathlineto{\pgfqpoint{11.166668in}{1.541239in}}%
\pgfpathlineto{\pgfqpoint{11.168950in}{1.550001in}}%
\pgfpathlineto{\pgfqpoint{11.175796in}{1.552504in}}%
\pgfpathlineto{\pgfqpoint{11.178078in}{1.562100in}}%
\pgfpathlineto{\pgfqpoint{11.180359in}{1.561370in}}%
\pgfpathlineto{\pgfqpoint{11.184923in}{1.572947in}}%
\pgfpathlineto{\pgfqpoint{11.191769in}{1.574094in}}%
\pgfpathlineto{\pgfqpoint{11.196333in}{1.577432in}}%
\pgfpathlineto{\pgfqpoint{11.198615in}{1.565959in}}%
\pgfpathlineto{\pgfqpoint{11.200897in}{1.567002in}}%
\pgfpathlineto{\pgfqpoint{11.207742in}{1.562621in}}%
\pgfpathlineto{\pgfqpoint{11.214588in}{1.554173in}}%
\pgfpathlineto{\pgfqpoint{11.216870in}{1.556572in}}%
\pgfpathlineto{\pgfqpoint{11.225998in}{1.568045in}}%
\pgfpathlineto{\pgfqpoint{11.228279in}{1.563977in}}%
\pgfpathlineto{\pgfqpoint{11.232843in}{1.533625in}}%
\pgfpathlineto{\pgfqpoint{11.239689in}{1.540509in}}%
\pgfpathlineto{\pgfqpoint{11.241971in}{1.545724in}}%
\pgfpathlineto{\pgfqpoint{11.244253in}{1.534042in}}%
\pgfpathlineto{\pgfqpoint{11.246535in}{1.534147in}}%
\pgfpathlineto{\pgfqpoint{11.248817in}{1.550522in}}%
\pgfpathlineto{\pgfqpoint{11.255662in}{1.540822in}}%
\pgfpathlineto{\pgfqpoint{11.257944in}{1.540509in}}%
\pgfpathlineto{\pgfqpoint{11.260226in}{1.532791in}}%
\pgfpathlineto{\pgfqpoint{11.262508in}{1.545203in}}%
\pgfpathlineto{\pgfqpoint{11.264790in}{1.540509in}}%
\pgfpathlineto{\pgfqpoint{11.271636in}{1.547080in}}%
\pgfpathlineto{\pgfqpoint{11.273918in}{1.554694in}}%
\pgfpathlineto{\pgfqpoint{11.276200in}{1.545516in}}%
\pgfpathlineto{\pgfqpoint{11.278481in}{1.523090in}}%
\pgfpathlineto{\pgfqpoint{11.280763in}{1.530287in}}%
\pgfpathlineto{\pgfqpoint{11.287609in}{1.527471in}}%
\pgfpathlineto{\pgfqpoint{11.289891in}{1.528723in}}%
\pgfpathlineto{\pgfqpoint{11.294455in}{1.539675in}}%
\pgfpathlineto{\pgfqpoint{11.296737in}{1.534251in}}%
\pgfpathlineto{\pgfqpoint{11.303582in}{1.541761in}}%
\pgfpathlineto{\pgfqpoint{11.305864in}{1.535294in}}%
\pgfpathlineto{\pgfqpoint{11.308146in}{1.538632in}}%
\pgfpathlineto{\pgfqpoint{11.312710in}{1.540718in}}%
\pgfpathlineto{\pgfqpoint{11.321838in}{1.551356in}}%
\pgfpathlineto{\pgfqpoint{11.324120in}{1.551044in}}%
\pgfpathlineto{\pgfqpoint{11.326401in}{1.567002in}}%
\pgfpathlineto{\pgfqpoint{11.328683in}{1.570861in}}%
\pgfpathlineto{\pgfqpoint{11.335529in}{1.560848in}}%
\pgfpathlineto{\pgfqpoint{11.337811in}{1.552191in}}%
\pgfpathlineto{\pgfqpoint{11.340093in}{1.555320in}}%
\pgfpathlineto{\pgfqpoint{11.342375in}{1.559701in}}%
\pgfpathlineto{\pgfqpoint{11.344657in}{1.553443in}}%
\pgfpathlineto{\pgfqpoint{11.351502in}{1.546246in}}%
\pgfpathlineto{\pgfqpoint{11.353784in}{1.546663in}}%
\pgfpathlineto{\pgfqpoint{11.356066in}{1.548540in}}%
\pgfpathlineto{\pgfqpoint{11.358348in}{1.547602in}}%
\pgfpathlineto{\pgfqpoint{11.360630in}{1.551774in}}%
\pgfpathlineto{\pgfqpoint{11.367476in}{1.548227in}}%
\pgfpathlineto{\pgfqpoint{11.369758in}{1.544160in}}%
\pgfpathlineto{\pgfqpoint{11.372040in}{1.548540in}}%
\pgfpathlineto{\pgfqpoint{11.374322in}{1.555320in}}%
\pgfpathlineto{\pgfqpoint{11.376603in}{1.566376in}}%
\pgfpathlineto{\pgfqpoint{11.383449in}{1.561682in}}%
\pgfpathlineto{\pgfqpoint{11.385731in}{1.571487in}}%
\pgfpathlineto{\pgfqpoint{11.388013in}{1.560639in}}%
\pgfpathlineto{\pgfqpoint{11.390295in}{1.559075in}}%
\pgfpathlineto{\pgfqpoint{11.392577in}{1.549375in}}%
\pgfpathlineto{\pgfqpoint{11.399423in}{1.541761in}}%
\pgfpathlineto{\pgfqpoint{11.401704in}{1.542178in}}%
\pgfpathlineto{\pgfqpoint{11.403986in}{1.541969in}}%
\pgfpathlineto{\pgfqpoint{11.406268in}{1.526950in}}%
\pgfpathlineto{\pgfqpoint{11.408550in}{1.524968in}}%
\pgfpathlineto{\pgfqpoint{11.415396in}{1.521734in}}%
\pgfpathlineto{\pgfqpoint{11.417678in}{1.522152in}}%
\pgfpathlineto{\pgfqpoint{11.419960in}{1.515059in}}%
\pgfpathlineto{\pgfqpoint{11.422242in}{1.519857in}}%
\pgfpathlineto{\pgfqpoint{11.424523in}{1.520691in}}%
\pgfpathlineto{\pgfqpoint{11.431369in}{1.517145in}}%
\pgfpathlineto{\pgfqpoint{11.433651in}{1.510783in}}%
\pgfpathlineto{\pgfqpoint{11.435933in}{1.511930in}}%
\pgfpathlineto{\pgfqpoint{11.438215in}{1.514538in}}%
\pgfpathlineto{\pgfqpoint{11.440497in}{1.512139in}}%
\pgfpathlineto{\pgfqpoint{11.449624in}{1.513077in}}%
\pgfpathlineto{\pgfqpoint{11.454188in}{1.516624in}}%
\pgfpathlineto{\pgfqpoint{11.463316in}{1.519440in}}%
\pgfpathlineto{\pgfqpoint{11.472444in}{1.568045in}}%
\pgfpathlineto{\pgfqpoint{11.483853in}{1.572008in}}%
\pgfpathlineto{\pgfqpoint{11.486135in}{1.558762in}}%
\pgfpathlineto{\pgfqpoint{11.488417in}{1.551982in}}%
\pgfpathlineto{\pgfqpoint{11.495263in}{1.549583in}}%
\pgfpathlineto{\pgfqpoint{11.497545in}{1.545411in}}%
\pgfpathlineto{\pgfqpoint{11.499826in}{1.546663in}}%
\pgfpathlineto{\pgfqpoint{11.502108in}{1.554173in}}%
\pgfpathlineto{\pgfqpoint{11.504390in}{1.553755in}}%
\pgfpathlineto{\pgfqpoint{11.511236in}{1.547080in}}%
\pgfpathlineto{\pgfqpoint{11.513518in}{1.550313in}}%
\pgfpathlineto{\pgfqpoint{11.515800in}{1.550835in}}%
\pgfpathlineto{\pgfqpoint{11.518082in}{1.542595in}}%
\pgfpathlineto{\pgfqpoint{11.520364in}{1.554798in}}%
\pgfpathlineto{\pgfqpoint{11.527209in}{1.548019in}}%
\pgfpathlineto{\pgfqpoint{11.529491in}{1.542386in}}%
\pgfpathlineto{\pgfqpoint{11.531773in}{1.538527in}}%
\pgfpathlineto{\pgfqpoint{11.534055in}{1.557510in}}%
\pgfpathlineto{\pgfqpoint{11.536337in}{1.563456in}}%
\pgfpathlineto{\pgfqpoint{11.543183in}{1.569401in}}%
\pgfpathlineto{\pgfqpoint{11.545465in}{1.565646in}}%
\pgfpathlineto{\pgfqpoint{11.547746in}{1.564811in}}%
\pgfpathlineto{\pgfqpoint{11.550028in}{1.565437in}}%
\pgfpathlineto{\pgfqpoint{11.552310in}{1.572426in}}%
\pgfpathlineto{\pgfqpoint{11.559156in}{1.576806in}}%
\pgfpathlineto{\pgfqpoint{11.561438in}{1.590053in}}%
\pgfpathlineto{\pgfqpoint{11.563720in}{1.581709in}}%
\pgfpathlineto{\pgfqpoint{11.566002in}{1.590574in}}%
\pgfpathlineto{\pgfqpoint{11.568284in}{1.592556in}}%
\pgfpathlineto{\pgfqpoint{11.577411in}{1.589323in}}%
\pgfpathlineto{\pgfqpoint{11.579693in}{1.585046in}}%
\pgfpathlineto{\pgfqpoint{11.581975in}{1.586194in}}%
\pgfpathlineto{\pgfqpoint{11.584257in}{1.590157in}}%
\pgfpathlineto{\pgfqpoint{11.591103in}{1.588280in}}%
\pgfpathlineto{\pgfqpoint{11.593385in}{1.588697in}}%
\pgfpathlineto{\pgfqpoint{11.595666in}{1.590053in}}%
\pgfpathlineto{\pgfqpoint{11.597948in}{1.572426in}}%
\pgfpathlineto{\pgfqpoint{11.600230in}{1.586089in}}%
\pgfpathlineto{\pgfqpoint{11.607076in}{1.583795in}}%
\pgfpathlineto{\pgfqpoint{11.609358in}{1.578371in}}%
\pgfpathlineto{\pgfqpoint{11.613922in}{1.599857in}}%
\pgfpathlineto{\pgfqpoint{11.616204in}{1.599649in}}%
\pgfpathlineto{\pgfqpoint{11.623049in}{1.594851in}}%
\pgfpathlineto{\pgfqpoint{11.625331in}{1.591200in}}%
\pgfpathlineto{\pgfqpoint{11.627613in}{1.592243in}}%
\pgfpathlineto{\pgfqpoint{11.629895in}{1.599023in}}%
\pgfpathlineto{\pgfqpoint{11.632177in}{1.601526in}}%
\pgfpathlineto{\pgfqpoint{11.639023in}{1.597145in}}%
\pgfpathlineto{\pgfqpoint{11.641305in}{1.610913in}}%
\pgfpathlineto{\pgfqpoint{11.643587in}{1.606950in}}%
\pgfpathlineto{\pgfqpoint{11.648150in}{1.604759in}}%
\pgfpathlineto{\pgfqpoint{11.654996in}{1.608306in}}%
\pgfpathlineto{\pgfqpoint{11.657278in}{1.598710in}}%
\pgfpathlineto{\pgfqpoint{11.659560in}{1.599336in}}%
\pgfpathlineto{\pgfqpoint{11.661842in}{1.601839in}}%
\pgfpathlineto{\pgfqpoint{11.664124in}{1.613521in}}%
\pgfpathlineto{\pgfqpoint{11.670969in}{1.610183in}}%
\pgfpathlineto{\pgfqpoint{11.673251in}{1.614564in}}%
\pgfpathlineto{\pgfqpoint{11.675533in}{1.604551in}}%
\pgfpathlineto{\pgfqpoint{11.680097in}{1.604238in}}%
\pgfpathlineto{\pgfqpoint{11.686943in}{1.612791in}}%
\pgfpathlineto{\pgfqpoint{11.689225in}{1.614251in}}%
\pgfpathlineto{\pgfqpoint{11.691507in}{1.629062in}}%
\pgfpathlineto{\pgfqpoint{11.693788in}{1.636572in}}%
\pgfpathlineto{\pgfqpoint{11.696070in}{1.631044in}}%
\pgfpathlineto{\pgfqpoint{11.702916in}{1.639492in}}%
\pgfpathlineto{\pgfqpoint{11.705198in}{1.645020in}}%
\pgfpathlineto{\pgfqpoint{11.709762in}{1.652530in}}%
\pgfpathlineto{\pgfqpoint{11.712044in}{1.649922in}}%
\pgfpathlineto{\pgfqpoint{11.718889in}{1.668071in}}%
\pgfpathlineto{\pgfqpoint{11.721171in}{1.669636in}}%
\pgfpathlineto{\pgfqpoint{11.725735in}{1.668384in}}%
\pgfpathlineto{\pgfqpoint{11.728017in}{1.666819in}}%
\pgfpathlineto{\pgfqpoint{11.734863in}{1.664108in}}%
\pgfpathlineto{\pgfqpoint{11.737145in}{1.669114in}}%
\pgfpathlineto{\pgfqpoint{11.739427in}{1.658267in}}%
\pgfpathlineto{\pgfqpoint{11.741709in}{1.655138in}}%
\pgfpathlineto{\pgfqpoint{11.743990in}{1.659936in}}%
\pgfpathlineto{\pgfqpoint{11.750836in}{1.639492in}}%
\pgfpathlineto{\pgfqpoint{11.753118in}{1.650861in}}%
\pgfpathlineto{\pgfqpoint{11.755400in}{1.649192in}}%
\pgfpathlineto{\pgfqpoint{11.757682in}{1.648567in}}%
\pgfpathlineto{\pgfqpoint{11.759964in}{1.646272in}}%
\pgfpathlineto{\pgfqpoint{11.769091in}{1.648462in}}%
\pgfpathlineto{\pgfqpoint{11.771373in}{1.659101in}}%
\pgfpathlineto{\pgfqpoint{11.773655in}{1.654095in}}%
\pgfpathlineto{\pgfqpoint{11.775937in}{1.634173in}}%
\pgfpathlineto{\pgfqpoint{11.782783in}{1.629583in}}%
\pgfpathlineto{\pgfqpoint{11.785065in}{1.635633in}}%
\pgfpathlineto{\pgfqpoint{11.787347in}{1.620718in}}%
\pgfpathlineto{\pgfqpoint{11.789629in}{1.638971in}}%
\pgfpathlineto{\pgfqpoint{11.791910in}{1.633025in}}%
\pgfpathlineto{\pgfqpoint{11.798756in}{1.607784in}}%
\pgfpathlineto{\pgfqpoint{11.801038in}{1.618006in}}%
\pgfpathlineto{\pgfqpoint{11.803320in}{1.624890in}}%
\pgfpathlineto{\pgfqpoint{11.805602in}{1.661604in}}%
\pgfpathlineto{\pgfqpoint{11.807884in}{1.662647in}}%
\pgfpathlineto{\pgfqpoint{11.814730in}{1.675998in}}%
\pgfpathlineto{\pgfqpoint{11.817011in}{1.683091in}}%
\pgfpathlineto{\pgfqpoint{11.819293in}{1.684238in}}%
\pgfpathlineto{\pgfqpoint{11.821575in}{1.684134in}}%
\pgfpathlineto{\pgfqpoint{11.823857in}{1.696963in}}%
\pgfpathlineto{\pgfqpoint{11.832985in}{1.700405in}}%
\pgfpathlineto{\pgfqpoint{11.835267in}{1.693521in}}%
\pgfpathlineto{\pgfqpoint{11.837549in}{1.696650in}}%
\pgfpathlineto{\pgfqpoint{11.839831in}{1.705724in}}%
\pgfpathlineto{\pgfqpoint{11.846676in}{1.711461in}}%
\pgfpathlineto{\pgfqpoint{11.848958in}{1.706455in}}%
\pgfpathlineto{\pgfqpoint{11.851240in}{1.705203in}}%
\pgfpathlineto{\pgfqpoint{11.855804in}{1.712295in}}%
\pgfpathlineto{\pgfqpoint{11.862650in}{1.699258in}}%
\pgfpathlineto{\pgfqpoint{11.864932in}{1.720953in}}%
\pgfpathlineto{\pgfqpoint{11.867213in}{1.730340in}}%
\pgfpathlineto{\pgfqpoint{11.869495in}{1.736702in}}%
\pgfpathlineto{\pgfqpoint{11.871777in}{1.725750in}}%
\pgfpathlineto{\pgfqpoint{11.878623in}{1.723456in}}%
\pgfpathlineto{\pgfqpoint{11.880905in}{1.714590in}}%
\pgfpathlineto{\pgfqpoint{11.883187in}{1.717719in}}%
\pgfpathlineto{\pgfqpoint{11.885469in}{1.703743in}}%
\pgfpathlineto{\pgfqpoint{11.887751in}{1.705099in}}%
\pgfpathlineto{\pgfqpoint{11.894596in}{1.717823in}}%
\pgfpathlineto{\pgfqpoint{11.896878in}{1.723664in}}%
\pgfpathlineto{\pgfqpoint{11.899160in}{1.709479in}}%
\pgfpathlineto{\pgfqpoint{11.901442in}{1.712295in}}%
\pgfpathlineto{\pgfqpoint{11.903724in}{1.707810in}}%
\pgfpathlineto{\pgfqpoint{11.910570in}{1.699466in}}%
\pgfpathlineto{\pgfqpoint{11.912852in}{1.700092in}}%
\pgfpathlineto{\pgfqpoint{11.915133in}{1.691122in}}%
\pgfpathlineto{\pgfqpoint{11.917415in}{1.689766in}}%
\pgfpathlineto{\pgfqpoint{11.919697in}{1.694668in}}%
\pgfpathlineto{\pgfqpoint{11.926543in}{1.705099in}}%
\pgfpathlineto{\pgfqpoint{11.928825in}{1.719597in}}%
\pgfpathlineto{\pgfqpoint{11.931107in}{1.720535in}}%
\pgfpathlineto{\pgfqpoint{11.933389in}{1.720744in}}%
\pgfpathlineto{\pgfqpoint{11.935671in}{1.708541in}}%
\pgfpathlineto{\pgfqpoint{11.942516in}{1.696024in}}%
\pgfpathlineto{\pgfqpoint{11.944798in}{1.700301in}}%
\pgfpathlineto{\pgfqpoint{11.947080in}{1.708958in}}%
\pgfpathlineto{\pgfqpoint{11.949362in}{1.680379in}}%
\pgfpathlineto{\pgfqpoint{11.951644in}{1.675581in}}%
\pgfpathlineto{\pgfqpoint{11.958490in}{1.681422in}}%
\pgfpathlineto{\pgfqpoint{11.960772in}{1.681839in}}%
\pgfpathlineto{\pgfqpoint{11.965335in}{1.704368in}}%
\pgfpathlineto{\pgfqpoint{11.974463in}{1.695816in}}%
\pgfpathlineto{\pgfqpoint{11.976745in}{1.699153in}}%
\pgfpathlineto{\pgfqpoint{11.981309in}{1.700509in}}%
\pgfpathlineto{\pgfqpoint{11.983591in}{1.687367in}}%
\pgfpathlineto{\pgfqpoint{11.990436in}{1.682986in}}%
\pgfpathlineto{\pgfqpoint{11.995000in}{1.698110in}}%
\pgfpathlineto{\pgfqpoint{11.997282in}{1.700196in}}%
\pgfpathlineto{\pgfqpoint{11.999564in}{1.707393in}}%
\pgfpathlineto{\pgfqpoint{12.006410in}{1.716259in}}%
\pgfpathlineto{\pgfqpoint{12.008692in}{1.714382in}}%
\pgfpathlineto{\pgfqpoint{12.010974in}{1.707185in}}%
\pgfpathlineto{\pgfqpoint{12.013255in}{1.719910in}}%
\pgfpathlineto{\pgfqpoint{12.015537in}{1.723247in}}%
\pgfpathlineto{\pgfqpoint{12.022383in}{1.727837in}}%
\pgfpathlineto{\pgfqpoint{12.024665in}{1.724499in}}%
\pgfpathlineto{\pgfqpoint{12.026947in}{1.713338in}}%
\pgfpathlineto{\pgfqpoint{12.029229in}{1.705933in}}%
\pgfpathlineto{\pgfqpoint{12.038356in}{1.714590in}}%
\pgfpathlineto{\pgfqpoint{12.040638in}{1.715007in}}%
\pgfpathlineto{\pgfqpoint{12.042920in}{1.725542in}}%
\pgfpathlineto{\pgfqpoint{12.045202in}{1.727106in}}%
\pgfpathlineto{\pgfqpoint{12.047484in}{1.736494in}}%
\pgfpathlineto{\pgfqpoint{12.056612in}{1.740561in}}%
\pgfpathlineto{\pgfqpoint{12.058894in}{1.739206in}}%
\pgfpathlineto{\pgfqpoint{12.061175in}{1.743378in}}%
\pgfpathlineto{\pgfqpoint{12.063457in}{1.743169in}}%
\pgfpathlineto{\pgfqpoint{12.070303in}{1.745672in}}%
\pgfpathlineto{\pgfqpoint{12.072585in}{1.741917in}}%
\pgfpathlineto{\pgfqpoint{12.074867in}{1.745359in}}%
\pgfpathlineto{\pgfqpoint{12.077149in}{1.750575in}}%
\pgfpathlineto{\pgfqpoint{12.079431in}{1.748906in}}%
\pgfpathlineto{\pgfqpoint{12.086276in}{1.757354in}}%
\pgfpathlineto{\pgfqpoint{12.088558in}{1.750262in}}%
\pgfpathlineto{\pgfqpoint{12.090840in}{1.746090in}}%
\pgfpathlineto{\pgfqpoint{12.093122in}{1.731487in}}%
\pgfpathlineto{\pgfqpoint{12.095404in}{1.731487in}}%
\pgfpathlineto{\pgfqpoint{12.102250in}{1.736285in}}%
\pgfpathlineto{\pgfqpoint{12.104532in}{1.734929in}}%
\pgfpathlineto{\pgfqpoint{12.106814in}{1.738997in}}%
\pgfpathlineto{\pgfqpoint{12.109096in}{1.740874in}}%
\pgfpathlineto{\pgfqpoint{12.111377in}{1.731696in}}%
\pgfpathlineto{\pgfqpoint{12.120505in}{1.730340in}}%
\pgfpathlineto{\pgfqpoint{12.122787in}{1.742126in}}%
\pgfpathlineto{\pgfqpoint{12.125069in}{1.747445in}}%
\pgfpathlineto{\pgfqpoint{12.127351in}{1.755581in}}%
\pgfpathlineto{\pgfqpoint{12.134197in}{1.759857in}}%
\pgfpathlineto{\pgfqpoint{12.136478in}{1.766950in}}%
\pgfpathlineto{\pgfqpoint{12.138760in}{1.766637in}}%
\pgfpathlineto{\pgfqpoint{12.141042in}{1.770496in}}%
\pgfpathlineto{\pgfqpoint{12.150170in}{1.768097in}}%
\pgfpathlineto{\pgfqpoint{12.152452in}{1.761005in}}%
\pgfpathlineto{\pgfqpoint{12.154734in}{1.774147in}}%
\pgfpathlineto{\pgfqpoint{12.157016in}{1.770705in}}%
\pgfpathlineto{\pgfqpoint{12.159297in}{1.771018in}}%
\pgfpathlineto{\pgfqpoint{12.166143in}{1.769766in}}%
\pgfpathlineto{\pgfqpoint{12.168425in}{1.763821in}}%
\pgfpathlineto{\pgfqpoint{12.170707in}{1.756103in}}%
\pgfpathlineto{\pgfqpoint{12.172989in}{1.752973in}}%
\pgfpathlineto{\pgfqpoint{12.175271in}{1.760588in}}%
\pgfpathlineto{\pgfqpoint{12.182117in}{1.759857in}}%
\pgfpathlineto{\pgfqpoint{12.184398in}{1.764447in}}%
\pgfpathlineto{\pgfqpoint{12.186680in}{1.762778in}}%
\pgfpathlineto{\pgfqpoint{12.188962in}{1.770079in}}%
\pgfpathlineto{\pgfqpoint{12.191244in}{1.764551in}}%
\pgfpathlineto{\pgfqpoint{12.198090in}{1.773521in}}%
\pgfpathlineto{\pgfqpoint{12.200372in}{1.764342in}}%
\pgfpathlineto{\pgfqpoint{12.202654in}{1.774043in}}%
\pgfpathlineto{\pgfqpoint{12.204936in}{1.761213in}}%
\pgfpathlineto{\pgfqpoint{12.207218in}{1.756415in}}%
\pgfpathlineto{\pgfqpoint{12.214063in}{1.774356in}}%
\pgfpathlineto{\pgfqpoint{12.216345in}{1.769662in}}%
\pgfpathlineto{\pgfqpoint{12.218627in}{1.768097in}}%
\pgfpathlineto{\pgfqpoint{12.220909in}{1.757667in}}%
\pgfpathlineto{\pgfqpoint{12.223191in}{1.770601in}}%
\pgfpathlineto{\pgfqpoint{12.230037in}{1.776859in}}%
\pgfpathlineto{\pgfqpoint{12.232319in}{1.774043in}}%
\pgfpathlineto{\pgfqpoint{12.234600in}{1.777798in}}%
\pgfpathlineto{\pgfqpoint{12.236882in}{1.787602in}}%
\pgfpathlineto{\pgfqpoint{12.239164in}{1.794069in}}%
\pgfpathlineto{\pgfqpoint{12.246010in}{1.800744in}}%
\pgfpathlineto{\pgfqpoint{12.248292in}{1.801891in}}%
\pgfpathlineto{\pgfqpoint{12.250574in}{1.799492in}}%
\pgfpathlineto{\pgfqpoint{12.252856in}{1.804708in}}%
\pgfpathlineto{\pgfqpoint{12.255138in}{1.805751in}}%
\pgfpathlineto{\pgfqpoint{12.261983in}{1.804708in}}%
\pgfpathlineto{\pgfqpoint{12.264265in}{1.801161in}}%
\pgfpathlineto{\pgfqpoint{12.266547in}{1.804603in}}%
\pgfpathlineto{\pgfqpoint{12.268829in}{1.803247in}}%
\pgfpathlineto{\pgfqpoint{12.271111in}{1.799805in}}%
\pgfpathlineto{\pgfqpoint{12.280239in}{1.808775in}}%
\pgfpathlineto{\pgfqpoint{12.282520in}{1.810027in}}%
\pgfpathlineto{\pgfqpoint{12.284802in}{1.802309in}}%
\pgfpathlineto{\pgfqpoint{12.287084in}{1.810027in}}%
\pgfpathlineto{\pgfqpoint{12.293930in}{1.806376in}}%
\pgfpathlineto{\pgfqpoint{12.296212in}{1.797094in}}%
\pgfpathlineto{\pgfqpoint{12.298494in}{1.796363in}}%
\pgfpathlineto{\pgfqpoint{12.300776in}{1.800744in}}%
\pgfpathlineto{\pgfqpoint{12.303058in}{1.797824in}}%
\pgfpathlineto{\pgfqpoint{12.314467in}{1.804186in}}%
\pgfpathlineto{\pgfqpoint{12.316749in}{1.804290in}}%
\pgfpathlineto{\pgfqpoint{12.319031in}{1.805751in}}%
\pgfpathlineto{\pgfqpoint{12.325877in}{1.794173in}}%
\pgfpathlineto{\pgfqpoint{12.328159in}{1.784681in}}%
\pgfpathlineto{\pgfqpoint{12.330441in}{1.795738in}}%
\pgfpathlineto{\pgfqpoint{12.332722in}{1.782387in}}%
\pgfpathlineto{\pgfqpoint{12.335004in}{1.788854in}}%
\pgfpathlineto{\pgfqpoint{12.341850in}{1.789688in}}%
\pgfpathlineto{\pgfqpoint{12.344132in}{1.791670in}}%
\pgfpathlineto{\pgfqpoint{12.346414in}{1.776754in}}%
\pgfpathlineto{\pgfqpoint{12.348696in}{1.769975in}}%
\pgfpathlineto{\pgfqpoint{12.350978in}{1.786037in}}%
\pgfpathlineto{\pgfqpoint{12.357823in}{1.787080in}}%
\pgfpathlineto{\pgfqpoint{12.360105in}{1.773104in}}%
\pgfpathlineto{\pgfqpoint{12.362387in}{1.782700in}}%
\pgfpathlineto{\pgfqpoint{12.364669in}{1.759545in}}%
\pgfpathlineto{\pgfqpoint{12.366951in}{1.764968in}}%
\pgfpathlineto{\pgfqpoint{12.373797in}{1.742230in}}%
\pgfpathlineto{\pgfqpoint{12.376079in}{1.744421in}}%
\pgfpathlineto{\pgfqpoint{12.378361in}{1.724499in}}%
\pgfpathlineto{\pgfqpoint{12.380642in}{1.721265in}}%
\pgfpathlineto{\pgfqpoint{12.382924in}{1.741396in}}%
\pgfpathlineto{\pgfqpoint{12.389770in}{1.757771in}}%
\pgfpathlineto{\pgfqpoint{12.392052in}{1.777276in}}%
\pgfpathlineto{\pgfqpoint{12.394334in}{1.773000in}}%
\pgfpathlineto{\pgfqpoint{12.396616in}{1.781552in}}%
\pgfpathlineto{\pgfqpoint{12.398898in}{1.787602in}}%
\pgfpathlineto{\pgfqpoint{12.405743in}{1.786037in}}%
\pgfpathlineto{\pgfqpoint{12.408025in}{1.800327in}}%
\pgfpathlineto{\pgfqpoint{12.410307in}{1.796468in}}%
\pgfpathlineto{\pgfqpoint{12.412589in}{1.803143in}}%
\pgfpathlineto{\pgfqpoint{12.414871in}{1.814303in}}%
\pgfpathlineto{\pgfqpoint{12.421717in}{1.817537in}}%
\pgfpathlineto{\pgfqpoint{12.423999in}{1.804395in}}%
\pgfpathlineto{\pgfqpoint{12.426281in}{1.810653in}}%
\pgfpathlineto{\pgfqpoint{12.428563in}{1.820353in}}%
\pgfpathlineto{\pgfqpoint{12.430844in}{1.800953in}}%
\pgfpathlineto{\pgfqpoint{12.437690in}{1.798867in}}%
\pgfpathlineto{\pgfqpoint{12.439972in}{1.800848in}}%
\pgfpathlineto{\pgfqpoint{12.442254in}{1.800014in}}%
\pgfpathlineto{\pgfqpoint{12.444536in}{1.805751in}}%
\pgfpathlineto{\pgfqpoint{12.446818in}{1.808775in}}%
\pgfpathlineto{\pgfqpoint{12.455945in}{1.803665in}}%
\pgfpathlineto{\pgfqpoint{12.458227in}{1.799284in}}%
\pgfpathlineto{\pgfqpoint{12.460509in}{1.790418in}}%
\pgfpathlineto{\pgfqpoint{12.462791in}{1.790940in}}%
\pgfpathlineto{\pgfqpoint{12.469637in}{1.806585in}}%
\pgfpathlineto{\pgfqpoint{12.471919in}{1.816911in}}%
\pgfpathlineto{\pgfqpoint{12.478764in}{1.825255in}}%
\pgfpathlineto{\pgfqpoint{12.485610in}{1.827028in}}%
\pgfpathlineto{\pgfqpoint{12.487892in}{1.834538in}}%
\pgfpathlineto{\pgfqpoint{12.490174in}{1.830992in}}%
\pgfpathlineto{\pgfqpoint{12.492456in}{1.832244in}}%
\pgfpathlineto{\pgfqpoint{12.494738in}{1.837354in}}%
\pgfpathlineto{\pgfqpoint{12.501584in}{1.837667in}}%
\pgfpathlineto{\pgfqpoint{12.503865in}{1.829427in}}%
\pgfpathlineto{\pgfqpoint{12.506147in}{1.816702in}}%
\pgfpathlineto{\pgfqpoint{12.508429in}{1.829114in}}%
\pgfpathlineto{\pgfqpoint{12.510711in}{1.826611in}}%
\pgfpathlineto{\pgfqpoint{12.517557in}{1.820875in}}%
\pgfpathlineto{\pgfqpoint{12.519839in}{1.813678in}}%
\pgfpathlineto{\pgfqpoint{12.524403in}{1.837563in}}%
\pgfpathlineto{\pgfqpoint{12.526684in}{1.840275in}}%
\pgfpathlineto{\pgfqpoint{12.535812in}{1.857902in}}%
\pgfpathlineto{\pgfqpoint{12.538094in}{1.855294in}}%
\pgfpathlineto{\pgfqpoint{12.542658in}{1.861240in}}%
\pgfpathlineto{\pgfqpoint{12.549504in}{1.865829in}}%
\pgfpathlineto{\pgfqpoint{12.551785in}{1.858006in}}%
\pgfpathlineto{\pgfqpoint{12.554067in}{1.853000in}}%
\pgfpathlineto{\pgfqpoint{12.558631in}{1.848723in}}%
\pgfpathlineto{\pgfqpoint{12.567759in}{1.830575in}}%
\pgfpathlineto{\pgfqpoint{12.572323in}{1.849141in}}%
\pgfpathlineto{\pgfqpoint{12.574605in}{1.853626in}}%
\pgfpathlineto{\pgfqpoint{12.581450in}{1.855712in}}%
\pgfpathlineto{\pgfqpoint{12.583732in}{1.862804in}}%
\pgfpathlineto{\pgfqpoint{12.586014in}{1.853417in}}%
\pgfpathlineto{\pgfqpoint{12.588296in}{1.854564in}}%
\pgfpathlineto{\pgfqpoint{12.590578in}{1.862700in}}%
\pgfpathlineto{\pgfqpoint{12.599706in}{1.858423in}}%
\pgfpathlineto{\pgfqpoint{12.601987in}{1.852791in}}%
\pgfpathlineto{\pgfqpoint{12.604269in}{1.862387in}}%
\pgfpathlineto{\pgfqpoint{12.606551in}{1.858215in}}%
\pgfpathlineto{\pgfqpoint{12.613397in}{1.860614in}}%
\pgfpathlineto{\pgfqpoint{12.615679in}{1.850914in}}%
\pgfpathlineto{\pgfqpoint{12.617961in}{1.838189in}}%
\pgfpathlineto{\pgfqpoint{12.620243in}{1.843508in}}%
\pgfpathlineto{\pgfqpoint{12.622525in}{1.821396in}}%
\pgfpathlineto{\pgfqpoint{12.629370in}{1.830888in}}%
\pgfpathlineto{\pgfqpoint{12.631652in}{1.852165in}}%
\pgfpathlineto{\pgfqpoint{12.633934in}{1.922361in}}%
\pgfpathlineto{\pgfqpoint{12.636216in}{1.935607in}}%
\pgfpathlineto{\pgfqpoint{12.638498in}{1.929558in}}%
\pgfpathlineto{\pgfqpoint{12.645344in}{1.926742in}}%
\pgfpathlineto{\pgfqpoint{12.647626in}{1.928619in}}%
\pgfpathlineto{\pgfqpoint{12.649907in}{1.928098in}}%
\pgfpathlineto{\pgfqpoint{12.652189in}{1.944890in}}%
\pgfpathlineto{\pgfqpoint{12.654471in}{1.950627in}}%
\pgfpathlineto{\pgfqpoint{12.663599in}{1.950106in}}%
\pgfpathlineto{\pgfqpoint{12.665881in}{1.947602in}}%
\pgfpathlineto{\pgfqpoint{12.668163in}{1.947915in}}%
\pgfpathlineto{\pgfqpoint{12.670445in}{1.954382in}}%
\pgfpathlineto{\pgfqpoint{12.677290in}{1.958658in}}%
\pgfpathlineto{\pgfqpoint{12.679572in}{1.955529in}}%
\pgfpathlineto{\pgfqpoint{12.681854in}{1.964291in}}%
\pgfpathlineto{\pgfqpoint{12.684136in}{1.954486in}}%
\pgfpathlineto{\pgfqpoint{12.686418in}{1.949793in}}%
\pgfpathlineto{\pgfqpoint{12.695546in}{1.971905in}}%
\pgfpathlineto{\pgfqpoint{12.697828in}{1.964291in}}%
\pgfpathlineto{\pgfqpoint{12.700109in}{1.959076in}}%
\pgfpathlineto{\pgfqpoint{12.702391in}{1.947185in}}%
\pgfpathlineto{\pgfqpoint{12.709237in}{1.961266in}}%
\pgfpathlineto{\pgfqpoint{12.711519in}{1.940092in}}%
\pgfpathlineto{\pgfqpoint{12.713801in}{1.938111in}}%
\pgfpathlineto{\pgfqpoint{12.716083in}{1.979936in}}%
\pgfpathlineto{\pgfqpoint{12.718365in}{1.972843in}}%
\pgfpathlineto{\pgfqpoint{12.725210in}{1.981918in}}%
\pgfpathlineto{\pgfqpoint{12.727492in}{1.977954in}}%
\pgfpathlineto{\pgfqpoint{12.729774in}{1.987759in}}%
\pgfpathlineto{\pgfqpoint{12.732056in}{1.981918in}}%
\pgfpathlineto{\pgfqpoint{12.734338in}{1.992244in}}%
\pgfpathlineto{\pgfqpoint{12.741184in}{1.990262in}}%
\pgfpathlineto{\pgfqpoint{12.743466in}{1.979415in}}%
\pgfpathlineto{\pgfqpoint{12.745748in}{1.958763in}}%
\pgfpathlineto{\pgfqpoint{12.757157in}{1.969714in}}%
\pgfpathlineto{\pgfqpoint{12.759439in}{1.957615in}}%
\pgfpathlineto{\pgfqpoint{12.764003in}{1.968567in}}%
\pgfpathlineto{\pgfqpoint{12.773130in}{1.964916in}}%
\pgfpathlineto{\pgfqpoint{12.775412in}{1.962935in}}%
\pgfpathlineto{\pgfqpoint{12.777694in}{1.972009in}}%
\pgfpathlineto{\pgfqpoint{12.779976in}{1.976077in}}%
\pgfpathlineto{\pgfqpoint{12.782258in}{1.977850in}}%
\pgfpathlineto{\pgfqpoint{12.789104in}{1.973469in}}%
\pgfpathlineto{\pgfqpoint{12.791386in}{1.974930in}}%
\pgfpathlineto{\pgfqpoint{12.793668in}{1.978059in}}%
\pgfpathlineto{\pgfqpoint{12.795950in}{1.989010in}}%
\pgfpathlineto{\pgfqpoint{12.798231in}{1.975242in}}%
\pgfpathlineto{\pgfqpoint{12.805077in}{1.990262in}}%
\pgfpathlineto{\pgfqpoint{12.807359in}{1.984943in}}%
\pgfpathlineto{\pgfqpoint{12.809641in}{1.987446in}}%
\pgfpathlineto{\pgfqpoint{12.811923in}{1.997876in}}%
\pgfpathlineto{\pgfqpoint{12.814205in}{2.003091in}}%
\pgfpathlineto{\pgfqpoint{12.821050in}{2.009245in}}%
\pgfpathlineto{\pgfqpoint{12.823332in}{2.006846in}}%
\pgfpathlineto{\pgfqpoint{12.825614in}{2.005803in}}%
\pgfpathlineto{\pgfqpoint{12.827896in}{1.995164in}}%
\pgfpathlineto{\pgfqpoint{12.830178in}{2.012687in}}%
\pgfpathlineto{\pgfqpoint{12.837024in}{2.017694in}}%
\pgfpathlineto{\pgfqpoint{12.839306in}{2.015608in}}%
\pgfpathlineto{\pgfqpoint{12.841588in}{2.004864in}}%
\pgfpathlineto{\pgfqpoint{12.843870in}{2.000379in}}%
\pgfpathlineto{\pgfqpoint{12.846151in}{2.008724in}}%
\pgfpathlineto{\pgfqpoint{12.852997in}{1.993913in}}%
\pgfpathlineto{\pgfqpoint{12.855279in}{2.000171in}}%
\pgfpathlineto{\pgfqpoint{12.857561in}{1.999754in}}%
\pgfpathlineto{\pgfqpoint{12.859843in}{2.006950in}}%
\pgfpathlineto{\pgfqpoint{12.862125in}{2.010601in}}%
\pgfpathlineto{\pgfqpoint{12.868971in}{2.010914in}}%
\pgfpathlineto{\pgfqpoint{12.871252in}{2.013104in}}%
\pgfpathlineto{\pgfqpoint{12.873534in}{2.009558in}}%
\pgfpathlineto{\pgfqpoint{12.875816in}{2.011435in}}%
\pgfpathlineto{\pgfqpoint{12.878098in}{2.010184in}}%
\pgfpathlineto{\pgfqpoint{12.887226in}{2.002153in}}%
\pgfpathlineto{\pgfqpoint{12.889508in}{2.011227in}}%
\pgfpathlineto{\pgfqpoint{12.891790in}{2.012791in}}%
\pgfpathlineto{\pgfqpoint{12.894072in}{2.011227in}}%
\pgfpathlineto{\pgfqpoint{12.900917in}{2.017068in}}%
\pgfpathlineto{\pgfqpoint{12.903199in}{2.014982in}}%
\pgfpathlineto{\pgfqpoint{12.905481in}{2.019050in}}%
\pgfpathlineto{\pgfqpoint{12.907763in}{2.010601in}}%
\pgfpathlineto{\pgfqpoint{12.910045in}{2.010601in}}%
\pgfpathlineto{\pgfqpoint{12.916891in}{2.000692in}}%
\pgfpathlineto{\pgfqpoint{12.919172in}{1.993183in}}%
\pgfpathlineto{\pgfqpoint{12.921454in}{2.007681in}}%
\pgfpathlineto{\pgfqpoint{12.923736in}{2.013730in}}%
\pgfpathlineto{\pgfqpoint{12.926018in}{2.007159in}}%
\pgfpathlineto{\pgfqpoint{12.932864in}{2.009349in}}%
\pgfpathlineto{\pgfqpoint{12.935146in}{2.018007in}}%
\pgfpathlineto{\pgfqpoint{12.937428in}{2.022179in}}%
\pgfpathlineto{\pgfqpoint{12.939710in}{2.039076in}}%
\pgfpathlineto{\pgfqpoint{12.941992in}{2.033235in}}%
\pgfpathlineto{\pgfqpoint{12.948837in}{2.042205in}}%
\pgfpathlineto{\pgfqpoint{12.951119in}{2.050758in}}%
\pgfpathlineto{\pgfqpoint{12.953401in}{2.044499in}}%
\pgfpathlineto{\pgfqpoint{12.957965in}{2.056390in}}%
\pgfpathlineto{\pgfqpoint{12.964811in}{2.037511in}}%
\pgfpathlineto{\pgfqpoint{12.967093in}{2.048150in}}%
\pgfpathlineto{\pgfqpoint{12.969374in}{2.064317in}}%
\pgfpathlineto{\pgfqpoint{12.971656in}{2.062752in}}%
\pgfpathlineto{\pgfqpoint{12.980784in}{2.069949in}}%
\pgfpathlineto{\pgfqpoint{12.983066in}{2.083717in}}%
\pgfpathlineto{\pgfqpoint{12.985348in}{2.064943in}}%
\pgfpathlineto{\pgfqpoint{12.987630in}{2.069011in}}%
\pgfpathlineto{\pgfqpoint{12.989912in}{2.077250in}}%
\pgfpathlineto{\pgfqpoint{12.996757in}{2.093104in}}%
\pgfpathlineto{\pgfqpoint{12.999039in}{2.091123in}}%
\pgfpathlineto{\pgfqpoint{13.001321in}{2.095503in}}%
\pgfpathlineto{\pgfqpoint{13.003603in}{2.103118in}}%
\pgfpathlineto{\pgfqpoint{13.005885in}{2.101031in}}%
\pgfpathlineto{\pgfqpoint{13.012731in}{2.108124in}}%
\pgfpathlineto{\pgfqpoint{13.015013in}{2.105516in}}%
\pgfpathlineto{\pgfqpoint{13.017294in}{2.105621in}}%
\pgfpathlineto{\pgfqpoint{13.019576in}{2.100406in}}%
\pgfpathlineto{\pgfqpoint{13.021858in}{2.101553in}}%
\pgfpathlineto{\pgfqpoint{13.028704in}{2.095086in}}%
\pgfpathlineto{\pgfqpoint{13.030986in}{2.097068in}}%
\pgfpathlineto{\pgfqpoint{13.033268in}{2.110732in}}%
\pgfpathlineto{\pgfqpoint{13.035550in}{2.112505in}}%
\pgfpathlineto{\pgfqpoint{13.037832in}{2.112296in}}%
\pgfpathlineto{\pgfqpoint{13.044677in}{2.123248in}}%
\pgfpathlineto{\pgfqpoint{13.046959in}{2.128880in}}%
\pgfpathlineto{\pgfqpoint{13.049241in}{2.019154in}}%
\pgfpathlineto{\pgfqpoint{13.051523in}{1.999649in}}%
\pgfpathlineto{\pgfqpoint{13.053805in}{2.007472in}}%
\pgfpathlineto{\pgfqpoint{13.060651in}{2.023743in}}%
\pgfpathlineto{\pgfqpoint{13.062933in}{1.994226in}}%
\pgfpathlineto{\pgfqpoint{13.065215in}{1.984317in}}%
\pgfpathlineto{\pgfqpoint{13.067496in}{1.989532in}}%
\pgfpathlineto{\pgfqpoint{13.069778in}{1.985986in}}%
\pgfpathlineto{\pgfqpoint{13.076624in}{2.004551in}}%
\pgfpathlineto{\pgfqpoint{13.078906in}{1.983795in}}%
\pgfpathlineto{\pgfqpoint{13.081188in}{1.978997in}}%
\pgfpathlineto{\pgfqpoint{13.083470in}{1.915686in}}%
\pgfpathlineto{\pgfqpoint{13.092597in}{1.869897in}}%
\pgfpathlineto{\pgfqpoint{13.094879in}{1.875112in}}%
\pgfpathlineto{\pgfqpoint{13.099443in}{1.936859in}}%
\pgfpathlineto{\pgfqpoint{13.101725in}{1.939884in}}%
\pgfpathlineto{\pgfqpoint{13.108571in}{1.934043in}}%
\pgfpathlineto{\pgfqpoint{13.110853in}{1.910679in}}%
\pgfpathlineto{\pgfqpoint{13.113135in}{1.934147in}}%
\pgfpathlineto{\pgfqpoint{13.115416in}{1.935086in}}%
\pgfpathlineto{\pgfqpoint{13.117698in}{1.925073in}}%
\pgfpathlineto{\pgfqpoint{13.126826in}{1.955008in}}%
\pgfpathlineto{\pgfqpoint{13.129108in}{1.934356in}}%
\pgfpathlineto{\pgfqpoint{13.131390in}{1.941135in}}%
\pgfpathlineto{\pgfqpoint{13.133672in}{1.959597in}}%
\pgfpathlineto{\pgfqpoint{13.140517in}{1.953130in}}%
\pgfpathlineto{\pgfqpoint{13.142799in}{1.949271in}}%
\pgfpathlineto{\pgfqpoint{13.145081in}{1.954486in}}%
\pgfpathlineto{\pgfqpoint{13.147363in}{1.956885in}}%
\pgfpathlineto{\pgfqpoint{13.149645in}{1.943430in}}%
\pgfpathlineto{\pgfqpoint{13.156491in}{1.949062in}}%
\pgfpathlineto{\pgfqpoint{13.163337in}{1.921631in}}%
\pgfpathlineto{\pgfqpoint{13.165618in}{1.918502in}}%
\pgfpathlineto{\pgfqpoint{13.172464in}{1.900666in}}%
\pgfpathlineto{\pgfqpoint{13.174746in}{1.909845in}}%
\pgfpathlineto{\pgfqpoint{13.177028in}{1.937172in}}%
\pgfpathlineto{\pgfqpoint{13.179310in}{1.941761in}}%
\pgfpathlineto{\pgfqpoint{13.181592in}{1.944995in}}%
\pgfpathlineto{\pgfqpoint{13.188437in}{1.953443in}}%
\pgfpathlineto{\pgfqpoint{13.190719in}{1.952609in}}%
\pgfpathlineto{\pgfqpoint{13.193001in}{1.948854in}}%
\pgfpathlineto{\pgfqpoint{13.197565in}{1.970236in}}%
\pgfpathlineto{\pgfqpoint{13.206693in}{1.980353in}}%
\pgfpathlineto{\pgfqpoint{13.208975in}{1.971905in}}%
\pgfpathlineto{\pgfqpoint{13.211257in}{1.993183in}}%
\pgfpathlineto{\pgfqpoint{13.213538in}{1.996624in}}%
\pgfpathlineto{\pgfqpoint{13.222666in}{2.012270in}}%
\pgfpathlineto{\pgfqpoint{13.224948in}{2.014773in}}%
\pgfpathlineto{\pgfqpoint{13.227230in}{2.045855in}}%
\pgfpathlineto{\pgfqpoint{13.229512in}{2.044291in}}%
\pgfpathlineto{\pgfqpoint{13.236358in}{2.048567in}}%
\pgfpathlineto{\pgfqpoint{13.238639in}{2.050966in}}%
\pgfpathlineto{\pgfqpoint{13.243203in}{2.063482in}}%
\pgfpathlineto{\pgfqpoint{13.245485in}{2.050653in}}%
\pgfpathlineto{\pgfqpoint{13.254613in}{2.068385in}}%
\pgfpathlineto{\pgfqpoint{13.256895in}{2.045855in}}%
\pgfpathlineto{\pgfqpoint{13.259177in}{2.043352in}}%
\pgfpathlineto{\pgfqpoint{13.261459in}{2.069636in}}%
\pgfpathlineto{\pgfqpoint{13.268304in}{2.077042in}}%
\pgfpathlineto{\pgfqpoint{13.270586in}{2.086846in}}%
\pgfpathlineto{\pgfqpoint{13.272868in}{2.077981in}}%
\pgfpathlineto{\pgfqpoint{13.275150in}{2.074956in}}%
\pgfpathlineto{\pgfqpoint{13.277432in}{2.061501in}}%
\pgfpathlineto{\pgfqpoint{13.284278in}{2.072140in}}%
\pgfpathlineto{\pgfqpoint{13.286559in}{2.074226in}}%
\pgfpathlineto{\pgfqpoint{13.288841in}{2.093939in}}%
\pgfpathlineto{\pgfqpoint{13.291123in}{2.099571in}}%
\pgfpathlineto{\pgfqpoint{13.293405in}{2.112922in}}%
\pgfpathlineto{\pgfqpoint{13.300251in}{2.106559in}}%
\pgfpathlineto{\pgfqpoint{13.302533in}{2.092061in}}%
\pgfpathlineto{\pgfqpoint{13.304815in}{2.099154in}}%
\pgfpathlineto{\pgfqpoint{13.309379in}{2.064317in}}%
\pgfpathlineto{\pgfqpoint{13.316224in}{2.048046in}}%
\pgfpathlineto{\pgfqpoint{13.318506in}{2.066924in}}%
\pgfpathlineto{\pgfqpoint{13.320788in}{2.053261in}}%
\pgfpathlineto{\pgfqpoint{13.323070in}{2.032505in}}%
\pgfpathlineto{\pgfqpoint{13.325352in}{2.055555in}}%
\pgfpathlineto{\pgfqpoint{13.332198in}{2.051592in}}%
\pgfpathlineto{\pgfqpoint{13.334480in}{2.038241in}}%
\pgfpathlineto{\pgfqpoint{13.336761in}{2.028333in}}%
\pgfpathlineto{\pgfqpoint{13.339043in}{2.028333in}}%
\pgfpathlineto{\pgfqpoint{13.341325in}{2.001422in}}%
\pgfpathlineto{\pgfqpoint{13.348171in}{2.014356in}}%
\pgfpathlineto{\pgfqpoint{13.350453in}{2.042205in}}%
\pgfpathlineto{\pgfqpoint{13.352735in}{2.058372in}}%
\pgfpathlineto{\pgfqpoint{13.355017in}{2.040745in}}%
\pgfpathlineto{\pgfqpoint{13.357299in}{1.998293in}}%
\pgfpathlineto{\pgfqpoint{13.364144in}{1.987029in}}%
\pgfpathlineto{\pgfqpoint{13.366426in}{1.988593in}}%
\pgfpathlineto{\pgfqpoint{13.368708in}{1.976911in}}%
\pgfpathlineto{\pgfqpoint{13.370990in}{1.979832in}}%
\pgfpathlineto{\pgfqpoint{13.380118in}{1.993600in}}%
\pgfpathlineto{\pgfqpoint{13.382400in}{1.991931in}}%
\pgfpathlineto{\pgfqpoint{13.384681in}{1.984630in}}%
\pgfpathlineto{\pgfqpoint{13.386963in}{1.972113in}}%
\pgfpathlineto{\pgfqpoint{13.396091in}{1.951357in}}%
\pgfpathlineto{\pgfqpoint{13.398373in}{1.930705in}}%
\pgfpathlineto{\pgfqpoint{13.400655in}{1.925386in}}%
\pgfpathlineto{\pgfqpoint{13.402937in}{1.916937in}}%
\pgfpathlineto{\pgfqpoint{13.405219in}{1.914434in}}%
\pgfpathlineto{\pgfqpoint{13.412064in}{1.921005in}}%
\pgfpathlineto{\pgfqpoint{13.414346in}{1.936338in}}%
\pgfpathlineto{\pgfqpoint{13.416628in}{1.906820in}}%
\pgfpathlineto{\pgfqpoint{13.418910in}{1.912974in}}%
\pgfpathlineto{\pgfqpoint{13.421192in}{1.861448in}}%
\pgfpathlineto{\pgfqpoint{13.430320in}{1.862178in}}%
\pgfpathlineto{\pgfqpoint{13.432602in}{1.847993in}}%
\pgfpathlineto{\pgfqpoint{13.434883in}{1.862596in}}%
\pgfpathlineto{\pgfqpoint{13.437165in}{1.891175in}}%
\pgfpathlineto{\pgfqpoint{13.444011in}{1.875216in}}%
\pgfpathlineto{\pgfqpoint{13.446293in}{1.884916in}}%
\pgfpathlineto{\pgfqpoint{13.448575in}{1.865620in}}%
\pgfpathlineto{\pgfqpoint{13.450857in}{1.857798in}}%
\pgfpathlineto{\pgfqpoint{13.453139in}{1.880431in}}%
\pgfpathlineto{\pgfqpoint{13.459984in}{1.873860in}}%
\pgfpathlineto{\pgfqpoint{13.462266in}{1.853730in}}%
\pgfpathlineto{\pgfqpoint{13.464548in}{1.873756in}}%
\pgfpathlineto{\pgfqpoint{13.466830in}{1.876572in}}%
\pgfpathlineto{\pgfqpoint{13.469112in}{1.861448in}}%
\pgfpathlineto{\pgfqpoint{13.475958in}{1.843821in}}%
\pgfpathlineto{\pgfqpoint{13.478240in}{1.845803in}}%
\pgfpathlineto{\pgfqpoint{13.480522in}{1.811487in}}%
\pgfpathlineto{\pgfqpoint{13.482803in}{1.825881in}}%
\pgfpathlineto{\pgfqpoint{13.485085in}{1.834225in}}%
\pgfpathlineto{\pgfqpoint{13.494213in}{1.851644in}}%
\pgfpathlineto{\pgfqpoint{13.496495in}{1.877302in}}%
\pgfpathlineto{\pgfqpoint{13.498777in}{1.874069in}}%
\pgfpathlineto{\pgfqpoint{13.501059in}{1.872400in}}%
\pgfpathlineto{\pgfqpoint{13.507904in}{1.885855in}}%
\pgfpathlineto{\pgfqpoint{13.510186in}{1.876051in}}%
\pgfpathlineto{\pgfqpoint{13.512468in}{1.876572in}}%
\pgfpathlineto{\pgfqpoint{13.514750in}{1.878762in}}%
\pgfpathlineto{\pgfqpoint{13.517032in}{1.875425in}}%
\pgfpathlineto{\pgfqpoint{13.523878in}{1.877511in}}%
\pgfpathlineto{\pgfqpoint{13.526160in}{1.898580in}}%
\pgfpathlineto{\pgfqpoint{13.528442in}{1.892113in}}%
\pgfpathlineto{\pgfqpoint{13.530724in}{1.910158in}}%
\pgfpathlineto{\pgfqpoint{13.533005in}{1.906820in}}%
\pgfpathlineto{\pgfqpoint{13.539851in}{1.915790in}}%
\pgfpathlineto{\pgfqpoint{13.542133in}{1.900249in}}%
\pgfpathlineto{\pgfqpoint{13.544415in}{1.898684in}}%
\pgfpathlineto{\pgfqpoint{13.546697in}{1.892530in}}%
\pgfpathlineto{\pgfqpoint{13.548979in}{1.901396in}}%
\pgfpathlineto{\pgfqpoint{13.555824in}{1.910053in}}%
\pgfpathlineto{\pgfqpoint{13.558106in}{1.904421in}}%
\pgfpathlineto{\pgfqpoint{13.560388in}{1.906403in}}%
\pgfpathlineto{\pgfqpoint{13.562670in}{1.917876in}}%
\pgfpathlineto{\pgfqpoint{13.564952in}{1.913912in}}%
\pgfpathlineto{\pgfqpoint{13.571798in}{1.906611in}}%
\pgfpathlineto{\pgfqpoint{13.576362in}{1.890444in}}%
\pgfpathlineto{\pgfqpoint{13.578644in}{1.894304in}}%
\pgfpathlineto{\pgfqpoint{13.587771in}{1.902961in}}%
\pgfpathlineto{\pgfqpoint{13.590053in}{1.903587in}}%
\pgfpathlineto{\pgfqpoint{13.592335in}{1.911096in}}%
\pgfpathlineto{\pgfqpoint{13.594617in}{1.914956in}}%
\pgfpathlineto{\pgfqpoint{13.596899in}{1.912661in}}%
\pgfpathlineto{\pgfqpoint{13.603745in}{1.908802in}}%
\pgfpathlineto{\pgfqpoint{13.606026in}{1.892113in}}%
\pgfpathlineto{\pgfqpoint{13.608308in}{1.896911in}}%
\pgfpathlineto{\pgfqpoint{13.610590in}{1.883873in}}%
\pgfpathlineto{\pgfqpoint{13.612872in}{1.886377in}}%
\pgfpathlineto{\pgfqpoint{13.619718in}{1.884916in}}%
\pgfpathlineto{\pgfqpoint{13.622000in}{1.895555in}}%
\pgfpathlineto{\pgfqpoint{13.624282in}{1.916729in}}%
\pgfpathlineto{\pgfqpoint{13.626564in}{1.908280in}}%
\pgfpathlineto{\pgfqpoint{13.628846in}{1.907863in}}%
\pgfpathlineto{\pgfqpoint{13.640255in}{1.954173in}}%
\pgfpathlineto{\pgfqpoint{13.642537in}{1.950627in}}%
\pgfpathlineto{\pgfqpoint{13.644819in}{1.959180in}}%
\pgfpathlineto{\pgfqpoint{13.653946in}{1.970236in}}%
\pgfpathlineto{\pgfqpoint{13.656228in}{1.974095in}}%
\pgfpathlineto{\pgfqpoint{13.658510in}{1.961683in}}%
\pgfpathlineto{\pgfqpoint{13.660792in}{1.954069in}}%
\pgfpathlineto{\pgfqpoint{13.667638in}{1.965021in}}%
\pgfpathlineto{\pgfqpoint{13.669920in}{1.959284in}}%
\pgfpathlineto{\pgfqpoint{13.672202in}{1.958137in}}%
\pgfpathlineto{\pgfqpoint{13.674484in}{1.970653in}}%
\pgfpathlineto{\pgfqpoint{13.676766in}{1.976703in}}%
\pgfpathlineto{\pgfqpoint{13.683611in}{1.974721in}}%
\pgfpathlineto{\pgfqpoint{13.685893in}{1.987133in}}%
\pgfpathlineto{\pgfqpoint{13.688175in}{1.944473in}}%
\pgfpathlineto{\pgfqpoint{13.690457in}{1.938737in}}%
\pgfpathlineto{\pgfqpoint{13.692739in}{1.926950in}}%
\pgfpathlineto{\pgfqpoint{13.699585in}{1.925386in}}%
\pgfpathlineto{\pgfqpoint{13.701867in}{1.921214in}}%
\pgfpathlineto{\pgfqpoint{13.704148in}{1.911931in}}%
\pgfpathlineto{\pgfqpoint{13.706430in}{1.906090in}}%
\pgfpathlineto{\pgfqpoint{13.708712in}{1.919649in}}%
\pgfpathlineto{\pgfqpoint{13.715558in}{1.913704in}}%
\pgfpathlineto{\pgfqpoint{13.720122in}{1.920484in}}%
\pgfpathlineto{\pgfqpoint{13.722404in}{1.919962in}}%
\pgfpathlineto{\pgfqpoint{13.724686in}{1.924656in}}%
\pgfpathlineto{\pgfqpoint{13.733813in}{1.914121in}}%
\pgfpathlineto{\pgfqpoint{13.736095in}{1.907133in}}%
\pgfpathlineto{\pgfqpoint{13.738377in}{1.909115in}}%
\pgfpathlineto{\pgfqpoint{13.740659in}{1.909427in}}%
\pgfpathlineto{\pgfqpoint{13.747505in}{1.909740in}}%
\pgfpathlineto{\pgfqpoint{13.749787in}{1.905464in}}%
\pgfpathlineto{\pgfqpoint{13.752068in}{1.902439in}}%
\pgfpathlineto{\pgfqpoint{13.754350in}{1.900457in}}%
\pgfpathlineto{\pgfqpoint{13.756632in}{1.895451in}}%
\pgfpathlineto{\pgfqpoint{13.763478in}{1.897746in}}%
\pgfpathlineto{\pgfqpoint{13.765760in}{1.905985in}}%
\pgfpathlineto{\pgfqpoint{13.768042in}{1.904734in}}%
\pgfpathlineto{\pgfqpoint{13.770324in}{1.905777in}}%
\pgfpathlineto{\pgfqpoint{13.772606in}{1.911931in}}%
\pgfpathlineto{\pgfqpoint{13.779451in}{1.917563in}}%
\pgfpathlineto{\pgfqpoint{13.781733in}{1.910158in}}%
\pgfpathlineto{\pgfqpoint{13.784015in}{1.909845in}}%
\pgfpathlineto{\pgfqpoint{13.786297in}{1.912139in}}%
\pgfpathlineto{\pgfqpoint{13.788579in}{1.879493in}}%
\pgfpathlineto{\pgfqpoint{13.795425in}{1.866246in}}%
\pgfpathlineto{\pgfqpoint{13.797707in}{1.882726in}}%
\pgfpathlineto{\pgfqpoint{13.802270in}{1.900249in}}%
\pgfpathlineto{\pgfqpoint{13.804552in}{1.902335in}}%
\pgfpathlineto{\pgfqpoint{13.813680in}{1.898684in}}%
\pgfpathlineto{\pgfqpoint{13.818244in}{1.913182in}}%
\pgfpathlineto{\pgfqpoint{13.820526in}{1.925281in}}%
\pgfpathlineto{\pgfqpoint{13.827371in}{1.928828in}}%
\pgfpathlineto{\pgfqpoint{13.829653in}{1.931018in}}%
\pgfpathlineto{\pgfqpoint{13.831935in}{1.927785in}}%
\pgfpathlineto{\pgfqpoint{13.834217in}{1.928723in}}%
\pgfpathlineto{\pgfqpoint{13.836499in}{1.927055in}}%
\pgfpathlineto{\pgfqpoint{13.843345in}{1.930497in}}%
\pgfpathlineto{\pgfqpoint{13.845627in}{1.923717in}}%
\pgfpathlineto{\pgfqpoint{13.847909in}{1.911305in}}%
\pgfpathlineto{\pgfqpoint{13.850190in}{1.909219in}}%
\pgfpathlineto{\pgfqpoint{13.852472in}{1.906194in}}%
\pgfpathlineto{\pgfqpoint{13.859318in}{1.902961in}}%
\pgfpathlineto{\pgfqpoint{13.861600in}{1.895972in}}%
\pgfpathlineto{\pgfqpoint{13.866164in}{1.888254in}}%
\pgfpathlineto{\pgfqpoint{13.868446in}{1.888671in}}%
\pgfpathlineto{\pgfqpoint{13.875291in}{1.884603in}}%
\pgfpathlineto{\pgfqpoint{13.877573in}{1.879284in}}%
\pgfpathlineto{\pgfqpoint{13.879855in}{1.890027in}}%
\pgfpathlineto{\pgfqpoint{13.882137in}{1.880744in}}%
\pgfpathlineto{\pgfqpoint{13.884419in}{1.887420in}}%
\pgfpathlineto{\pgfqpoint{13.891265in}{1.886689in}}%
\pgfpathlineto{\pgfqpoint{13.893547in}{1.895868in}}%
\pgfpathlineto{\pgfqpoint{13.895829in}{1.907654in}}%
\pgfpathlineto{\pgfqpoint{13.898111in}{1.906820in}}%
\pgfpathlineto{\pgfqpoint{13.900392in}{1.897537in}}%
\pgfpathlineto{\pgfqpoint{13.907238in}{1.900145in}}%
\pgfpathlineto{\pgfqpoint{13.909520in}{1.897954in}}%
\pgfpathlineto{\pgfqpoint{13.911802in}{1.897850in}}%
\pgfpathlineto{\pgfqpoint{13.916366in}{1.893052in}}%
\pgfpathlineto{\pgfqpoint{13.923211in}{1.887837in}}%
\pgfpathlineto{\pgfqpoint{13.925493in}{1.888880in}}%
\pgfpathlineto{\pgfqpoint{13.927775in}{1.887315in}}%
\pgfpathlineto{\pgfqpoint{13.932339in}{1.881266in}}%
\pgfpathlineto{\pgfqpoint{13.939185in}{1.877824in}}%
\pgfpathlineto{\pgfqpoint{13.941467in}{1.877824in}}%
\pgfpathlineto{\pgfqpoint{13.943749in}{1.873756in}}%
\pgfpathlineto{\pgfqpoint{13.946031in}{1.871774in}}%
\pgfpathlineto{\pgfqpoint{13.948312in}{1.873339in}}%
\pgfpathlineto{\pgfqpoint{13.957440in}{1.867185in}}%
\pgfpathlineto{\pgfqpoint{13.959722in}{1.866350in}}%
\pgfpathlineto{\pgfqpoint{13.962004in}{1.870105in}}%
\pgfpathlineto{\pgfqpoint{13.964286in}{1.853417in}}%
\pgfpathlineto{\pgfqpoint{13.971132in}{1.865620in}}%
\pgfpathlineto{\pgfqpoint{13.973413in}{1.856233in}}%
\pgfpathlineto{\pgfqpoint{13.975695in}{1.851852in}}%
\pgfpathlineto{\pgfqpoint{13.977977in}{1.854251in}}%
\pgfpathlineto{\pgfqpoint{13.980259in}{1.854877in}}%
\pgfpathlineto{\pgfqpoint{13.987105in}{1.855503in}}%
\pgfpathlineto{\pgfqpoint{13.989387in}{1.858736in}}%
\pgfpathlineto{\pgfqpoint{13.991669in}{1.853104in}}%
\pgfpathlineto{\pgfqpoint{13.993951in}{1.863326in}}%
\pgfpathlineto{\pgfqpoint{13.996233in}{1.861970in}}%
\pgfpathlineto{\pgfqpoint{14.003078in}{1.848828in}}%
\pgfpathlineto{\pgfqpoint{14.005360in}{1.846429in}}%
\pgfpathlineto{\pgfqpoint{14.007642in}{1.851227in}}%
\pgfpathlineto{\pgfqpoint{14.009924in}{1.847263in}}%
\pgfpathlineto{\pgfqpoint{14.012206in}{1.857798in}}%
\pgfpathlineto{\pgfqpoint{14.019052in}{1.854147in}}%
\pgfpathlineto{\pgfqpoint{14.021333in}{1.855190in}}%
\pgfpathlineto{\pgfqpoint{14.023615in}{1.853730in}}%
\pgfpathlineto{\pgfqpoint{14.025897in}{1.857485in}}%
\pgfpathlineto{\pgfqpoint{14.028179in}{1.854147in}}%
\pgfpathlineto{\pgfqpoint{14.035025in}{1.854147in}}%
\pgfpathlineto{\pgfqpoint{14.037307in}{1.847889in}}%
\pgfpathlineto{\pgfqpoint{14.039589in}{1.843404in}}%
\pgfpathlineto{\pgfqpoint{14.041871in}{1.840483in}}%
\pgfpathlineto{\pgfqpoint{14.044153in}{1.842257in}}%
\pgfpathlineto{\pgfqpoint{14.050998in}{1.837563in}}%
\pgfpathlineto{\pgfqpoint{14.053280in}{1.841005in}}%
\pgfpathlineto{\pgfqpoint{14.055562in}{1.848515in}}%
\pgfpathlineto{\pgfqpoint{14.057844in}{1.849558in}}%
\pgfpathlineto{\pgfqpoint{14.060126in}{1.859571in}}%
\pgfpathlineto{\pgfqpoint{14.066972in}{1.862908in}}%
\pgfpathlineto{\pgfqpoint{14.069254in}{1.856755in}}%
\pgfpathlineto{\pgfqpoint{14.071535in}{1.864160in}}%
\pgfpathlineto{\pgfqpoint{14.073817in}{1.869375in}}%
\pgfpathlineto{\pgfqpoint{14.076099in}{1.867706in}}%
\pgfpathlineto{\pgfqpoint{14.087509in}{1.848410in}}%
\pgfpathlineto{\pgfqpoint{14.089791in}{1.862908in}}%
\pgfpathlineto{\pgfqpoint{14.092073in}{1.853730in}}%
\pgfpathlineto{\pgfqpoint{14.098918in}{1.873443in}}%
\pgfpathlineto{\pgfqpoint{14.101200in}{1.873026in}}%
\pgfpathlineto{\pgfqpoint{14.105764in}{1.878762in}}%
\pgfpathlineto{\pgfqpoint{14.108046in}{1.905881in}}%
\pgfpathlineto{\pgfqpoint{14.114892in}{1.908280in}}%
\pgfpathlineto{\pgfqpoint{14.117174in}{1.906090in}}%
\pgfpathlineto{\pgfqpoint{14.119455in}{1.920275in}}%
\pgfpathlineto{\pgfqpoint{14.121737in}{1.922778in}}%
\pgfpathlineto{\pgfqpoint{14.124019in}{1.911514in}}%
\pgfpathlineto{\pgfqpoint{14.130865in}{1.905360in}}%
\pgfpathlineto{\pgfqpoint{14.133147in}{1.906194in}}%
\pgfpathlineto{\pgfqpoint{14.135429in}{1.911722in}}%
\pgfpathlineto{\pgfqpoint{14.139993in}{1.917250in}}%
\pgfpathlineto{\pgfqpoint{14.146838in}{1.918710in}}%
\pgfpathlineto{\pgfqpoint{14.149120in}{1.925699in}}%
\pgfpathlineto{\pgfqpoint{14.151402in}{1.920275in}}%
\pgfpathlineto{\pgfqpoint{14.153684in}{1.918502in}}%
\pgfpathlineto{\pgfqpoint{14.155966in}{1.914121in}}%
\pgfpathlineto{\pgfqpoint{14.162812in}{1.928619in}}%
\pgfpathlineto{\pgfqpoint{14.165094in}{1.935607in}}%
\pgfpathlineto{\pgfqpoint{14.167376in}{1.948854in}}%
\pgfpathlineto{\pgfqpoint{14.169657in}{1.970653in}}%
\pgfpathlineto{\pgfqpoint{14.171939in}{1.985568in}}%
\pgfpathlineto{\pgfqpoint{14.181067in}{1.975347in}}%
\pgfpathlineto{\pgfqpoint{14.183349in}{1.977433in}}%
\pgfpathlineto{\pgfqpoint{14.185631in}{1.980770in}}%
\pgfpathlineto{\pgfqpoint{14.187913in}{1.975973in}}%
\pgfpathlineto{\pgfqpoint{14.194758in}{1.989949in}}%
\pgfpathlineto{\pgfqpoint{14.197040in}{1.991618in}}%
\pgfpathlineto{\pgfqpoint{14.199322in}{1.992557in}}%
\pgfpathlineto{\pgfqpoint{14.201604in}{1.991201in}}%
\pgfpathlineto{\pgfqpoint{14.203886in}{1.988489in}}%
\pgfpathlineto{\pgfqpoint{14.213014in}{1.988697in}}%
\pgfpathlineto{\pgfqpoint{14.215296in}{1.979936in}}%
\pgfpathlineto{\pgfqpoint{14.217577in}{1.982544in}}%
\pgfpathlineto{\pgfqpoint{14.219859in}{1.979102in}}%
\pgfpathlineto{\pgfqpoint{14.228987in}{1.997772in}}%
\pgfpathlineto{\pgfqpoint{14.231269in}{2.011435in}}%
\pgfpathlineto{\pgfqpoint{14.233551in}{2.010914in}}%
\pgfpathlineto{\pgfqpoint{14.235833in}{2.026977in}}%
\pgfpathlineto{\pgfqpoint{14.242678in}{2.020718in}}%
\pgfpathlineto{\pgfqpoint{14.244960in}{2.020927in}}%
\pgfpathlineto{\pgfqpoint{14.247242in}{2.031566in}}%
\pgfpathlineto{\pgfqpoint{14.249524in}{2.012374in}}%
\pgfpathlineto{\pgfqpoint{14.251806in}{2.017694in}}%
\pgfpathlineto{\pgfqpoint{14.260934in}{2.016755in}}%
\pgfpathlineto{\pgfqpoint{14.263216in}{2.018737in}}%
\pgfpathlineto{\pgfqpoint{14.265498in}{2.009871in}}%
\pgfpathlineto{\pgfqpoint{14.267779in}{2.013730in}}%
\pgfpathlineto{\pgfqpoint{14.274625in}{2.008202in}}%
\pgfpathlineto{\pgfqpoint{14.276907in}{2.016129in}}%
\pgfpathlineto{\pgfqpoint{14.279189in}{2.017485in}}%
\pgfpathlineto{\pgfqpoint{14.281471in}{2.017694in}}%
\pgfpathlineto{\pgfqpoint{14.283753in}{2.030106in}}%
\pgfpathlineto{\pgfqpoint{14.290599in}{2.046585in}}%
\pgfpathlineto{\pgfqpoint{14.292880in}{2.043769in}}%
\pgfpathlineto{\pgfqpoint{14.295162in}{2.050236in}}%
\pgfpathlineto{\pgfqpoint{14.297444in}{2.043352in}}%
\pgfpathlineto{\pgfqpoint{14.299726in}{2.040223in}}%
\pgfpathlineto{\pgfqpoint{14.306572in}{2.032922in}}%
\pgfpathlineto{\pgfqpoint{14.308854in}{2.027185in}}%
\pgfpathlineto{\pgfqpoint{14.311136in}{2.027185in}}%
\pgfpathlineto{\pgfqpoint{14.313418in}{2.032087in}}%
\pgfpathlineto{\pgfqpoint{14.315699in}{2.029793in}}%
\pgfpathlineto{\pgfqpoint{14.322545in}{2.033652in}}%
\pgfpathlineto{\pgfqpoint{14.324827in}{2.040536in}}%
\pgfpathlineto{\pgfqpoint{14.327109in}{2.038971in}}%
\pgfpathlineto{\pgfqpoint{14.329391in}{2.044291in}}%
\pgfpathlineto{\pgfqpoint{14.331673in}{2.037824in}}%
\pgfpathlineto{\pgfqpoint{14.340800in}{2.037303in}}%
\pgfpathlineto{\pgfqpoint{14.343082in}{2.038346in}}%
\pgfpathlineto{\pgfqpoint{14.345364in}{2.034486in}}%
\pgfpathlineto{\pgfqpoint{14.347646in}{2.040432in}}%
\pgfpathlineto{\pgfqpoint{14.354492in}{2.039493in}}%
\pgfpathlineto{\pgfqpoint{14.356774in}{2.038137in}}%
\pgfpathlineto{\pgfqpoint{14.359056in}{2.047628in}}%
\pgfpathlineto{\pgfqpoint{14.361338in}{2.043143in}}%
\pgfpathlineto{\pgfqpoint{14.363620in}{2.049610in}}%
\pgfpathlineto{\pgfqpoint{14.370465in}{2.043874in}}%
\pgfpathlineto{\pgfqpoint{14.372747in}{2.045855in}}%
\pgfpathlineto{\pgfqpoint{14.375029in}{2.045647in}}%
\pgfpathlineto{\pgfqpoint{14.377311in}{2.047524in}}%
\pgfpathlineto{\pgfqpoint{14.379593in}{2.046481in}}%
\pgfpathlineto{\pgfqpoint{14.386439in}{2.052426in}}%
\pgfpathlineto{\pgfqpoint{14.388720in}{2.060353in}}%
\pgfpathlineto{\pgfqpoint{14.391002in}{2.055973in}}%
\pgfpathlineto{\pgfqpoint{14.393284in}{2.054408in}}%
\pgfpathlineto{\pgfqpoint{14.395566in}{2.054825in}}%
\pgfpathlineto{\pgfqpoint{14.402412in}{2.064421in}}%
\pgfpathlineto{\pgfqpoint{14.404694in}{2.054825in}}%
\pgfpathlineto{\pgfqpoint{14.406976in}{2.058059in}}%
\pgfpathlineto{\pgfqpoint{14.409258in}{2.059728in}}%
\pgfpathlineto{\pgfqpoint{14.411540in}{2.058685in}}%
\pgfpathlineto{\pgfqpoint{14.418385in}{2.061084in}}%
\pgfpathlineto{\pgfqpoint{14.420667in}{2.067863in}}%
\pgfpathlineto{\pgfqpoint{14.422949in}{2.062648in}}%
\pgfpathlineto{\pgfqpoint{14.425231in}{2.069115in}}%
\pgfpathlineto{\pgfqpoint{14.427513in}{2.071305in}}%
\pgfpathlineto{\pgfqpoint{14.434359in}{2.069323in}}%
\pgfpathlineto{\pgfqpoint{14.436641in}{2.067759in}}%
\pgfpathlineto{\pgfqpoint{14.438922in}{2.067342in}}%
\pgfpathlineto{\pgfqpoint{14.441204in}{2.067863in}}%
\pgfpathlineto{\pgfqpoint{14.443486in}{2.063065in}}%
\pgfpathlineto{\pgfqpoint{14.450332in}{2.061605in}}%
\pgfpathlineto{\pgfqpoint{14.452614in}{2.068072in}}%
\pgfpathlineto{\pgfqpoint{14.454896in}{2.067759in}}%
\pgfpathlineto{\pgfqpoint{14.466305in}{2.075164in}}%
\pgfpathlineto{\pgfqpoint{14.468587in}{2.079337in}}%
\pgfpathlineto{\pgfqpoint{14.470869in}{2.074643in}}%
\pgfpathlineto{\pgfqpoint{14.473151in}{2.085282in}}%
\pgfpathlineto{\pgfqpoint{14.475433in}{2.081840in}}%
\pgfpathlineto{\pgfqpoint{14.482279in}{2.074330in}}%
\pgfpathlineto{\pgfqpoint{14.484561in}{2.089245in}}%
\pgfpathlineto{\pgfqpoint{14.486842in}{2.093209in}}%
\pgfpathlineto{\pgfqpoint{14.489124in}{2.095921in}}%
\pgfpathlineto{\pgfqpoint{14.491406in}{2.093417in}}%
\pgfpathlineto{\pgfqpoint{14.500534in}{2.081110in}}%
\pgfpathlineto{\pgfqpoint{14.502816in}{2.053469in}}%
\pgfpathlineto{\pgfqpoint{14.505098in}{2.048984in}}%
\pgfpathlineto{\pgfqpoint{14.507380in}{2.057224in}}%
\pgfpathlineto{\pgfqpoint{14.514225in}{2.051488in}}%
\pgfpathlineto{\pgfqpoint{14.516507in}{2.057954in}}%
\pgfpathlineto{\pgfqpoint{14.518789in}{2.033756in}}%
\pgfpathlineto{\pgfqpoint{14.521071in}{2.032922in}}%
\pgfpathlineto{\pgfqpoint{14.523353in}{2.034069in}}%
\pgfpathlineto{\pgfqpoint{14.530199in}{2.028437in}}%
\pgfpathlineto{\pgfqpoint{14.532481in}{2.016859in}}%
\pgfpathlineto{\pgfqpoint{14.534763in}{2.000066in}}%
\pgfpathlineto{\pgfqpoint{14.537044in}{2.003821in}}%
\pgfpathlineto{\pgfqpoint{14.539326in}{2.012270in}}%
\pgfpathlineto{\pgfqpoint{14.546172in}{2.013417in}}%
\pgfpathlineto{\pgfqpoint{14.548454in}{2.007263in}}%
\pgfpathlineto{\pgfqpoint{14.550736in}{2.014147in}}%
\pgfpathlineto{\pgfqpoint{14.553018in}{2.009767in}}%
\pgfpathlineto{\pgfqpoint{14.555300in}{2.021240in}}%
\pgfpathlineto{\pgfqpoint{14.564427in}{2.020510in}}%
\pgfpathlineto{\pgfqpoint{14.566709in}{2.016442in}}%
\pgfpathlineto{\pgfqpoint{14.568991in}{2.019154in}}%
\pgfpathlineto{\pgfqpoint{14.571273in}{2.008828in}}%
\pgfpathlineto{\pgfqpoint{14.578119in}{2.002257in}}%
\pgfpathlineto{\pgfqpoint{14.580401in}{1.991931in}}%
\pgfpathlineto{\pgfqpoint{14.582683in}{1.996207in}}%
\pgfpathlineto{\pgfqpoint{14.584964in}{1.980145in}}%
\pgfpathlineto{\pgfqpoint{14.587246in}{1.993183in}}%
\pgfpathlineto{\pgfqpoint{14.594092in}{2.007472in}}%
\pgfpathlineto{\pgfqpoint{14.598656in}{1.998398in}}%
\pgfpathlineto{\pgfqpoint{14.600938in}{1.996833in}}%
\pgfpathlineto{\pgfqpoint{14.603220in}{1.992035in}}%
\pgfpathlineto{\pgfqpoint{14.610065in}{1.990679in}}%
\pgfpathlineto{\pgfqpoint{14.612347in}{1.976285in}}%
\pgfpathlineto{\pgfqpoint{14.614629in}{1.984943in}}%
\pgfpathlineto{\pgfqpoint{14.616911in}{1.979102in}}%
\pgfpathlineto{\pgfqpoint{14.619193in}{1.980562in}}%
\pgfpathlineto{\pgfqpoint{14.626039in}{1.992661in}}%
\pgfpathlineto{\pgfqpoint{14.628321in}{1.991096in}}%
\pgfpathlineto{\pgfqpoint{14.630603in}{2.005595in}}%
\pgfpathlineto{\pgfqpoint{14.632885in}{1.994121in}}%
\pgfpathlineto{\pgfqpoint{14.635166in}{1.999545in}}%
\pgfpathlineto{\pgfqpoint{14.642012in}{2.011644in}}%
\pgfpathlineto{\pgfqpoint{14.646576in}{1.993287in}}%
\pgfpathlineto{\pgfqpoint{14.648858in}{1.978059in}}%
\pgfpathlineto{\pgfqpoint{14.651140in}{1.977746in}}%
\pgfpathlineto{\pgfqpoint{14.657986in}{1.980562in}}%
\pgfpathlineto{\pgfqpoint{14.660267in}{1.983065in}}%
\pgfpathlineto{\pgfqpoint{14.662549in}{1.988489in}}%
\pgfpathlineto{\pgfqpoint{14.664831in}{1.987654in}}%
\pgfpathlineto{\pgfqpoint{14.667113in}{1.995686in}}%
\pgfpathlineto{\pgfqpoint{14.673959in}{1.992661in}}%
\pgfpathlineto{\pgfqpoint{14.678523in}{2.013209in}}%
\pgfpathlineto{\pgfqpoint{14.680805in}{2.018737in}}%
\pgfpathlineto{\pgfqpoint{14.683086in}{2.015920in}}%
\pgfpathlineto{\pgfqpoint{14.689932in}{2.015086in}}%
\pgfpathlineto{\pgfqpoint{14.692214in}{2.009141in}}%
\pgfpathlineto{\pgfqpoint{14.694496in}{2.014460in}}%
\pgfpathlineto{\pgfqpoint{14.696778in}{2.045438in}}%
\pgfpathlineto{\pgfqpoint{14.699060in}{2.045021in}}%
\pgfpathlineto{\pgfqpoint{14.705906in}{2.044708in}}%
\pgfpathlineto{\pgfqpoint{14.708187in}{2.051592in}}%
\pgfpathlineto{\pgfqpoint{14.710469in}{2.031983in}}%
\pgfpathlineto{\pgfqpoint{14.712751in}{2.036468in}}%
\pgfpathlineto{\pgfqpoint{14.715033in}{2.022074in}}%
\pgfpathlineto{\pgfqpoint{14.721879in}{2.008515in}}%
\pgfpathlineto{\pgfqpoint{14.724161in}{2.014877in}}%
\pgfpathlineto{\pgfqpoint{14.726443in}{1.972843in}}%
\pgfpathlineto{\pgfqpoint{14.728725in}{1.957824in}}%
\pgfpathlineto{\pgfqpoint{14.731007in}{1.964291in}}%
\pgfpathlineto{\pgfqpoint{14.737852in}{1.958345in}}%
\pgfpathlineto{\pgfqpoint{14.740134in}{1.959493in}}%
\pgfpathlineto{\pgfqpoint{14.742416in}{1.966481in}}%
\pgfpathlineto{\pgfqpoint{14.746980in}{1.951253in}}%
\pgfpathlineto{\pgfqpoint{14.753826in}{1.956051in}}%
\pgfpathlineto{\pgfqpoint{14.756107in}{1.972218in}}%
\pgfpathlineto{\pgfqpoint{14.758389in}{1.959388in}}%
\pgfpathlineto{\pgfqpoint{14.760671in}{1.959597in}}%
\pgfpathlineto{\pgfqpoint{14.762953in}{1.968567in}}%
\pgfpathlineto{\pgfqpoint{14.772081in}{1.970236in}}%
\pgfpathlineto{\pgfqpoint{14.774363in}{1.973261in}}%
\pgfpathlineto{\pgfqpoint{14.776645in}{1.956364in}}%
\pgfpathlineto{\pgfqpoint{14.778927in}{1.959388in}}%
\pgfpathlineto{\pgfqpoint{14.788054in}{1.960431in}}%
\pgfpathlineto{\pgfqpoint{14.790336in}{1.959388in}}%
\pgfpathlineto{\pgfqpoint{14.792618in}{1.914434in}}%
\pgfpathlineto{\pgfqpoint{14.801746in}{1.914747in}}%
\pgfpathlineto{\pgfqpoint{14.806309in}{1.932270in}}%
\pgfpathlineto{\pgfqpoint{14.808591in}{1.922883in}}%
\pgfpathlineto{\pgfqpoint{14.810873in}{1.929245in}}%
\pgfpathlineto{\pgfqpoint{14.817719in}{1.924969in}}%
\pgfpathlineto{\pgfqpoint{14.820001in}{1.928306in}}%
\pgfpathlineto{\pgfqpoint{14.822283in}{1.936233in}}%
\pgfpathlineto{\pgfqpoint{14.826847in}{1.929975in}}%
\pgfpathlineto{\pgfqpoint{14.833692in}{1.939884in}}%
\pgfpathlineto{\pgfqpoint{14.835974in}{1.930288in}}%
\pgfpathlineto{\pgfqpoint{14.838256in}{1.936442in}}%
\pgfpathlineto{\pgfqpoint{14.840538in}{1.924447in}}%
\pgfpathlineto{\pgfqpoint{14.842820in}{1.929662in}}%
\pgfpathlineto{\pgfqpoint{14.849666in}{1.942804in}}%
\pgfpathlineto{\pgfqpoint{14.851948in}{1.952192in}}%
\pgfpathlineto{\pgfqpoint{14.854229in}{1.949793in}}%
\pgfpathlineto{\pgfqpoint{14.856511in}{1.945308in}}%
\pgfpathlineto{\pgfqpoint{14.858793in}{1.944890in}}%
\pgfpathlineto{\pgfqpoint{14.865639in}{1.939884in}}%
\pgfpathlineto{\pgfqpoint{14.867921in}{1.939884in}}%
\pgfpathlineto{\pgfqpoint{14.870203in}{1.929558in}}%
\pgfpathlineto{\pgfqpoint{14.872485in}{1.913078in}}%
\pgfpathlineto{\pgfqpoint{14.874767in}{1.917667in}}%
\pgfpathlineto{\pgfqpoint{14.883894in}{1.927576in}}%
\pgfpathlineto{\pgfqpoint{14.886176in}{1.926429in}}%
\pgfpathlineto{\pgfqpoint{14.888458in}{1.934147in}}%
\pgfpathlineto{\pgfqpoint{14.890740in}{1.938111in}}%
\pgfpathlineto{\pgfqpoint{14.897586in}{1.931018in}}%
\pgfpathlineto{\pgfqpoint{14.902150in}{1.921944in}}%
\pgfpathlineto{\pgfqpoint{14.904431in}{1.929558in}}%
\pgfpathlineto{\pgfqpoint{14.906713in}{1.927055in}}%
\pgfpathlineto{\pgfqpoint{14.913559in}{1.924343in}}%
\pgfpathlineto{\pgfqpoint{14.915841in}{1.922048in}}%
\pgfpathlineto{\pgfqpoint{14.918123in}{1.934356in}}%
\pgfpathlineto{\pgfqpoint{14.920405in}{1.927472in}}%
\pgfpathlineto{\pgfqpoint{14.922687in}{1.930392in}}%
\pgfpathlineto{\pgfqpoint{14.929532in}{1.950627in}}%
\pgfpathlineto{\pgfqpoint{14.931814in}{1.960431in}}%
\pgfpathlineto{\pgfqpoint{14.934096in}{1.956155in}}%
\pgfpathlineto{\pgfqpoint{14.936378in}{1.971279in}}%
\pgfpathlineto{\pgfqpoint{14.938660in}{1.992557in}}%
\pgfpathlineto{\pgfqpoint{14.945506in}{1.992139in}}%
\pgfpathlineto{\pgfqpoint{14.947788in}{1.976285in}}%
\pgfpathlineto{\pgfqpoint{14.950070in}{1.981501in}}%
\pgfpathlineto{\pgfqpoint{14.952351in}{1.980666in}}%
\pgfpathlineto{\pgfqpoint{14.954633in}{1.978997in}}%
\pgfpathlineto{\pgfqpoint{14.961479in}{1.972009in}}%
\pgfpathlineto{\pgfqpoint{14.963761in}{1.974512in}}%
\pgfpathlineto{\pgfqpoint{14.966043in}{1.971905in}}%
\pgfpathlineto{\pgfqpoint{14.970607in}{1.970862in}}%
\pgfpathlineto{\pgfqpoint{14.977452in}{1.972426in}}%
\pgfpathlineto{\pgfqpoint{14.979734in}{1.978684in}}%
\pgfpathlineto{\pgfqpoint{14.982016in}{1.997250in}}%
\pgfpathlineto{\pgfqpoint{14.984298in}{1.992974in}}%
\pgfpathlineto{\pgfqpoint{14.986580in}{1.997355in}}%
\pgfpathlineto{\pgfqpoint{14.993426in}{2.047628in}}%
\pgfpathlineto{\pgfqpoint{14.995708in}{2.017276in}}%
\pgfpathlineto{\pgfqpoint{14.997990in}{1.999441in}}%
\pgfpathlineto{\pgfqpoint{15.002553in}{1.995477in}}%
\pgfpathlineto{\pgfqpoint{15.009399in}{2.022074in}}%
\pgfpathlineto{\pgfqpoint{15.011681in}{2.028124in}}%
\pgfpathlineto{\pgfqpoint{15.013963in}{2.030001in}}%
\pgfpathlineto{\pgfqpoint{15.016245in}{2.060145in}}%
\pgfpathlineto{\pgfqpoint{15.018527in}{2.067342in}}%
\pgfpathlineto{\pgfqpoint{15.025373in}{2.064838in}}%
\pgfpathlineto{\pgfqpoint{15.027654in}{2.072870in}}%
\pgfpathlineto{\pgfqpoint{15.029936in}{2.051175in}}%
\pgfpathlineto{\pgfqpoint{15.032218in}{2.050027in}}%
\pgfpathlineto{\pgfqpoint{15.034500in}{2.040849in}}%
\pgfpathlineto{\pgfqpoint{15.043628in}{2.035216in}}%
\pgfpathlineto{\pgfqpoint{15.045910in}{2.030314in}}%
\pgfpathlineto{\pgfqpoint{15.048192in}{2.031566in}}%
\pgfpathlineto{\pgfqpoint{15.050473in}{2.028958in}}%
\pgfpathlineto{\pgfqpoint{15.050473in}{2.028958in}}%
\pgfusepath{stroke}%
\end{pgfscope}%
\begin{pgfscope}%
\pgfsetrectcap%
\pgfsetmiterjoin%
\pgfsetlinewidth{0.803000pt}%
\definecolor{currentstroke}{rgb}{1.000000,1.000000,1.000000}%
\pgfsetstrokecolor{currentstroke}%
\pgfsetdash{}{0pt}%
\pgfpathmoveto{\pgfqpoint{9.810417in}{1.250000in}}%
\pgfpathlineto{\pgfqpoint{9.810417in}{2.170732in}}%
\pgfusepath{stroke}%
\end{pgfscope}%
\begin{pgfscope}%
\pgfsetrectcap%
\pgfsetmiterjoin%
\pgfsetlinewidth{0.803000pt}%
\definecolor{currentstroke}{rgb}{1.000000,1.000000,1.000000}%
\pgfsetstrokecolor{currentstroke}%
\pgfsetdash{}{0pt}%
\pgfpathmoveto{\pgfqpoint{15.300000in}{1.250000in}}%
\pgfpathlineto{\pgfqpoint{15.300000in}{2.170732in}}%
\pgfusepath{stroke}%
\end{pgfscope}%
\begin{pgfscope}%
\pgfsetrectcap%
\pgfsetmiterjoin%
\pgfsetlinewidth{0.803000pt}%
\definecolor{currentstroke}{rgb}{1.000000,1.000000,1.000000}%
\pgfsetstrokecolor{currentstroke}%
\pgfsetdash{}{0pt}%
\pgfpathmoveto{\pgfqpoint{9.810417in}{1.250000in}}%
\pgfpathlineto{\pgfqpoint{15.300000in}{1.250000in}}%
\pgfusepath{stroke}%
\end{pgfscope}%
\begin{pgfscope}%
\pgfsetrectcap%
\pgfsetmiterjoin%
\pgfsetlinewidth{0.803000pt}%
\definecolor{currentstroke}{rgb}{1.000000,1.000000,1.000000}%
\pgfsetstrokecolor{currentstroke}%
\pgfsetdash{}{0pt}%
\pgfpathmoveto{\pgfqpoint{9.810417in}{2.170732in}}%
\pgfpathlineto{\pgfqpoint{15.300000in}{2.170732in}}%
\pgfusepath{stroke}%
\end{pgfscope}%
\begin{pgfscope}%
\definecolor{textcolor}{rgb}{0.150000,0.150000,0.150000}%
\pgfsetstrokecolor{textcolor}%
\pgfsetfillcolor{textcolor}%
\pgftext[x=12.555208in,y=2.254065in,,base]{\color{textcolor}\rmfamily\fontsize{12.000000}{14.400000}\selectfont DIS}%
\end{pgfscope}%
\end{pgfpicture}%
\makeatother%
\endgroup%

    \end{adjustbox}  
    \caption{Daily Stock Prices for all Stocks in the Data Set}
    \label{fig:adj_close_all}
\end{figure}{}







We see that the time series exhibits strong autocorrelation through the first lag. We therefore take first differences. 

The data then looks like this: 



Figure3: FD of Log closing prices. Entweder alle oder einige zum Beispiel
fig.savefig('/srv/shared/StatistischesPraktikumReport/Figures/all_log_adjclose_fd_and_qq.png')
Eventuell auch mit QQ-Plot

Visually the data looks like white noise. Also the mean of those values is close to zero

Table1: means of the values

Looking at the QQ-plot (explanation) we see that the distribution has fat tails: some of the values are more extreme than we would have expected if the differenced log-values were randomly distributed. 


\subsection{Predictions With Random Walks}
A random walk follows (and is the same as AR(1)??? MA(1))
XXXXXXXX
Predictions for period t + 1 are therefore exactly the value at time t. --> We do that on the FD of log-values scale but we could also do it on the real scale

Confidence Intervals are computed as
XXXXXXXX

Fitting: 
Our Code fits an RW model, prediicts the next value, compares it to the real value for the MSE and then adds the true value to the time series used for predicting the next value. (we could have done this simpler, by just taking the values of the previous period as prediction for the current one. 

Figure: Plot Real vs. Predicted values. (Real values? or FD of log-Values?
Table: MSE

\subsection{Predictions With AR(x) Models}
AR(x) models follow
XXXXXX

Die Frage ist so ein bisschen, ob wir versuchen sollen, die ganze Zeitreihe mit einem einzelnen Fit zu beschreiben, oder ob wir jede Periode ein neues Modell fitten sollten. 
Table: MSE
Figure: Plot Real vs. Predicted values. (Real values? or FD of log-Values?

\subsection{Predictions With ARMA Models}
ARMA() follows
XXXXXXX


Auto-ARMA


\subsection{Predictions With GARCH Models}
GARCH follows 
XXXXXXX

Idea behind GARCH

predictions
fitting
confidence intervals
MSE

Plot: Squared time series



\subsection{Summary of Prediction with Time Series Models}
Table: MSE all
Figure: All predicted values?


\section{Hybrid Prediction}

\subsection{ARMAX Predictions}
ARMAX works like this: 
XXXXXX

Predictions
Confidence Intervals


Predictors can be 
- Using weather forecasts --> ARMAX
- Number of Tweets?
- Sentiments from Machine Learning Algorithm
- Predictions made by the Algorithm



\subsection{Weighted Average of Predictions}
a) of different time series models
b) of time series and ML models
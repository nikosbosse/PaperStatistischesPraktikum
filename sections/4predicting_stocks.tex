\chapter{Predicting Stocks}\label{ch:predictions}


\section{Predictions Using Machine Learning}

\section{Predictions Using Time Series}
\subsection{Idea/Process and Evaluation}
irgendwas in der Richtung: wir benutzen Time Series Modelle, machen Predictions und gucken uns am Ende dann den Mean Squared Error an. Sinnvollerweise immer die ersten 10 Perioden verwerfen, um den MSE vergleichbar zu machen zwischen allen Gruppen, auch denen, bei denen die ersten paar Perioden nicht definiert sind. 	
In-Sample vs. Out of Sample Prediction?


\subsection{Theoretical Overview}
Time series predictions in this paper will be made using Random Walks, autoregressive (AR), moving average (MA) and generalized autoregressive conditional heteroscedasticity (GARCH) models. The following will first give a short theoretical overview. Then different models will be applied to a training data set of two chosen stocks. Then predictions will be made for the other stocks using the above mentioned techniques. 

\subsubsection{Random Walks}
Random walks serve as the baseline against which every prediction can be compared. Assuming a random walk as the underlying process implies that we know nothing about the future and can do no better than assuming tomorrow's stock price will on average be the same as today. Formally, a random walk follows \begin{align}
    y_t &= y_{t-1} + w_t \label{eg:rw1}\\
    \intertext{where $y_t$ is the value of the time series at time t and $w_t$ is a random realisation of a stationary white noise process with mean 0 and variance $\sigma^2$. We can expand equation \ref{eg:rw1} by allowing for a constant trend, a drift. A random walk with drift can be represented as}
    y_t &= \delta + y_{t-1} + w_t
\end{align}{}
where $\delta$ is a drift parameter. Predictions for period t + 1 are therefore exactly the value at time t. As we have already eliminated the trend by transforming the data to log-returns we will not use the drift representation here. If we did our analysis on the original stock values then a drift would be appropriate. Note that the random walk (with or without drift) is not a stationary process. 

\subsubsection{Autoregressive Models}
An autoregressive process of order p (AR(p)) implies that the current value of a time series can be described as a combination of the previous p values plus a random shock. As those previous values intern depend on previous values, the current value is indirectly influenced by its entire past. Formally, an AR(p) process follows 
\begin{align}
    y_t &= \psi_1 y_{t-1} + \psi_2 y_{t-2} + ... + \psi_p y_{t-p} + w_t \label{eq:AR(p)}
\intertext{where $y_t$ is stationary, $\psi_1, ..., \psi_p$ are constants and $w_t$ is white noise. The mean of $y_t$ is assumed to be zero. If the mean is $\mu$ instead of zero, equation \ref{eq:AR(p)} can be rewritten as}
    y_t - \mu &= \phi_1 (y_{t-1} - \mu) + \phi_2 (y_{t-2} - \mu) + ... + \phi_p (y_{t-p} - \mu) + w_t \label{eq:AR(p)} \\
\intertext{This can also be expressed as}
    y_t &= \alpha + \phi_1 y_{t-1} + \phi_2 y_{t-2} + ... + \phi_p y_{t-p} + w_t
\end{align}
\noindent with $\alpha$ = $\mu (1 - \phi_1 - ... - \phi_p)$.

\subsubsection{Moving Average Models}
A moving average process of order q implies that the current value of a time series consists of the average of the previous q observations plus a random shock. As the mean of the time series $\mu$ is constant this average can also be simply expressed as an average of the past random shocks $\{w_{t-1}, ... w_{t-q}\} $. In constrast to the AR(p) process, the shocks affect the future directly (and not only indirectly through past values) and only affect the next q values. Formally, the MA(q) process can be expressed as
\begin{align}
    y_t = \mu + w_t + \theta w_{t-1} + ... + \theta w_{t-q}
\end{align}{}
\noindent where $w_t$ represents white noise and $\theta_1, ..., \theta_q$ are parameters and q is the number of lags in the moving average. 


\subsubsection{Autoregressive Moving Average Models}
Autoregressive Moving Average Models of order p and q (ARMA(p,q)) form a combination of the above described AR(p) and MA(q) models. Formally, an ARMA(p,q) process follows
\begin{align}
    y_t = \phi_1 y_{t-1} + ... + \phi_p y_{t-p} + w_t + \theta_1 w_{t-1} + ... + \theta_q w_{t-q} \\
    \intertext{if the mean of $y_t$ is $\mu$, then the above results in}
    y_t = \alpha + \phi_1 y_{t-1} + ... + \phi_p y_{t-p} + w_t + \theta_1 w_{t-1} + ... + \theta_q w_{t-q}
\end{align}
\noindent with $\alpha = \mu (1 - \phi_1 - ... - \phi_p)$.


\subsubsection{GARCH Models}
The GARCH(p,q) model is specified as follows:
\begin{align}
    r_{t} &= \sigma_t  \upepsilon_t \label{eq:garch1}\\
\intertext{where $\upepsilon_t$ is Gaussian white noise with  $\upepsilon_t ~ \mathcal{N}(0,1)$ and}
    \sigma^2_t &= \alpha_0 + \underbrace{\alpha_1 r^2_{t-1} + ... + \alpha_p r^2_{t-p}}_\text{autoregressive part} + \underbrace{\beta_1 \sigma^2_{t-1} + ... + \beta_q \sigma^2_{t-q}}_\text{moving average part} \label{eq:garch2}
\end{align}{}
In equation \ref{eq:garch1} the returns $r_t$ are therefore modelled as white noise with mean zero and variance variance $\sigma^2_t$. When compared to white Gaussian noise with constant variance this can produce a leptokurtic (fat-tailed) distribution similar to what we observed in the QQ-Plots in figure XXXXX. Equation \ref{eq:garch1} is called the mean model of the GARCH(p,q) process. This mean model can also be altered as needed. The GARCH model can then be specified in the following way: 
\begin{equation}
    r_t = x_t + y_t
\end{equation}{}
where $x_t$ can be any constant mean, regression or time series process and $y_t$ is a GARCH process that satisfies equations \ref{eq:garch1} and \ref{eq:garch2}. In a similar way, the distribution of $\upepsilon_t$ can be altered. In praxis, researchers often assume a t-distribution instead of a standard normal distribution. 

A further expansion is the GJR-GARCH model (SOURCE) where the variance GJR-GARCH(1,1,1) is specified as follows: 

\begin{equation}
    \sigma^2_t = \omega + \alpha * \epsilon^2_{t-1} + \gamma \epsilon^2_{t-1} I(\epsilon^2_{t-1} < 0) + \beta \sigma^2_{t-1} 
\end{equation}{}

This means that negative shocks may have a different impact on future volatility than positive shocks, e.g. a sudden drop in a stock will cause the stock to be disproportionally more volatile in the near future. 



\subsection{Approaching the Training Data}
\subsubsection{Data Exploration}
To get a better feeling about our data and to avoid overfitting we try to explore the two stocks INTC and V. We apply different time series models to the entire time series and check their model fit. Figure \ref{fig:INTC_V_ACF_log_returns} shows the ACF and PACF for the log-returns of INTC and V. From looking at the plots we can presume that for V, AR and MA models of order one or two might be a reasonable try. For INTC it looks like there is very little information included as none of the lower order lags bear any significance. 

We start by applying a formal test for autocorrelation to the time series, the Box/Pierce and Ljung/Box tests. They have slightly different properties regarding their handling of very large and very small numbers of observations, but for both the null hypothesis is that there is no autocorrelation in the series. Figure \ref{fig:ljungbox} shows the p-values for the first 40 lags. The test suggests that for V there may be some significant autocorrelations that could be used for modelling. Compared to the plot of ACF and PACF the test even seems too optimistic. For INTC, the test confirms non-significance for the first few lags. While higher order lags are deemed to be significant by the test modelling a time series process with so many coefficients would be very prone to overfitting. 

\begin{figure}[h]
    \centering
    \figuretitle{ACF and PACF of log-returns of Stocks INTC and V}
    \begin{adjustbox}{width=.95\textwidth,center}
    %% Creator: Matplotlib, PGF backend
%%
%% To include the figure in your LaTeX document, write
%%   \input{<filename>.pgf}
%%
%% Make sure the required packages are loaded in your preamble
%%   \usepackage{pgf}
%%
%% Figures using additional raster images can only be included by \input if
%% they are in the same directory as the main LaTeX file. For loading figures
%% from other directories you can use the `import` package
%%   \usepackage{import}
%% and then include the figures with
%%   \import{<path to file>}{<filename>.pgf}
%%
%% Matplotlib used the following preamble
%%   \usepackage{fontspec}
%%   \setmainfont{DejaVuSerif.ttf}[Path=/opt/tljh/user/lib/python3.6/site-packages/matplotlib/mpl-data/fonts/ttf/]
%%   \setsansfont{DejaVuSans.ttf}[Path=/opt/tljh/user/lib/python3.6/site-packages/matplotlib/mpl-data/fonts/ttf/]
%%   \setmonofont{DejaVuSansMono.ttf}[Path=/opt/tljh/user/lib/python3.6/site-packages/matplotlib/mpl-data/fonts/ttf/]
%%
\begingroup%
\makeatletter%
\begin{pgfpicture}%
\pgfpathrectangle{\pgfpointorigin}{\pgfqpoint{12.143102in}{4.416596in}}%
\pgfusepath{use as bounding box, clip}%
\begin{pgfscope}%
\pgfsetbuttcap%
\pgfsetmiterjoin%
\definecolor{currentfill}{rgb}{1.000000,1.000000,1.000000}%
\pgfsetfillcolor{currentfill}%
\pgfsetlinewidth{0.000000pt}%
\definecolor{currentstroke}{rgb}{1.000000,1.000000,1.000000}%
\pgfsetstrokecolor{currentstroke}%
\pgfsetdash{}{0pt}%
\pgfpathmoveto{\pgfqpoint{0.000000in}{0.000000in}}%
\pgfpathlineto{\pgfqpoint{12.143102in}{0.000000in}}%
\pgfpathlineto{\pgfqpoint{12.143102in}{4.416596in}}%
\pgfpathlineto{\pgfqpoint{0.000000in}{4.416596in}}%
\pgfpathclose%
\pgfusepath{fill}%
\end{pgfscope}%
\begin{pgfscope}%
\pgfsetbuttcap%
\pgfsetmiterjoin%
\definecolor{currentfill}{rgb}{0.917647,0.917647,0.949020}%
\pgfsetfillcolor{currentfill}%
\pgfsetlinewidth{0.000000pt}%
\definecolor{currentstroke}{rgb}{0.000000,0.000000,0.000000}%
\pgfsetstrokecolor{currentstroke}%
\pgfsetstrokeopacity{0.000000}%
\pgfsetdash{}{0pt}%
\pgfpathmoveto{\pgfqpoint{0.418102in}{0.331635in}}%
\pgfpathlineto{\pgfqpoint{5.261852in}{0.331635in}}%
\pgfpathlineto{\pgfqpoint{5.261852in}{4.106635in}}%
\pgfpathlineto{\pgfqpoint{0.418102in}{4.106635in}}%
\pgfpathclose%
\pgfusepath{fill}%
\end{pgfscope}%
\begin{pgfscope}%
\pgfpathrectangle{\pgfqpoint{0.418102in}{0.331635in}}{\pgfqpoint{4.843750in}{3.775000in}}%
\pgfusepath{clip}%
\pgfsetroundcap%
\pgfsetroundjoin%
\pgfsetlinewidth{0.803000pt}%
\definecolor{currentstroke}{rgb}{1.000000,1.000000,1.000000}%
\pgfsetstrokecolor{currentstroke}%
\pgfsetdash{}{0pt}%
\pgfpathmoveto{\pgfqpoint{0.638272in}{0.331635in}}%
\pgfpathlineto{\pgfqpoint{0.638272in}{4.106635in}}%
\pgfusepath{stroke}%
\end{pgfscope}%
\begin{pgfscope}%
\definecolor{textcolor}{rgb}{0.150000,0.150000,0.150000}%
\pgfsetstrokecolor{textcolor}%
\pgfsetfillcolor{textcolor}%
\pgftext[x=0.638272in,y=0.234413in,,top]{\color{textcolor}\rmfamily\fontsize{10.000000}{12.000000}\selectfont 0}%
\end{pgfscope}%
\begin{pgfscope}%
\pgfpathrectangle{\pgfqpoint{0.418102in}{0.331635in}}{\pgfqpoint{4.843750in}{3.775000in}}%
\pgfusepath{clip}%
\pgfsetroundcap%
\pgfsetroundjoin%
\pgfsetlinewidth{0.803000pt}%
\definecolor{currentstroke}{rgb}{1.000000,1.000000,1.000000}%
\pgfsetstrokecolor{currentstroke}%
\pgfsetdash{}{0pt}%
\pgfpathmoveto{\pgfqpoint{1.712274in}{0.331635in}}%
\pgfpathlineto{\pgfqpoint{1.712274in}{4.106635in}}%
\pgfusepath{stroke}%
\end{pgfscope}%
\begin{pgfscope}%
\definecolor{textcolor}{rgb}{0.150000,0.150000,0.150000}%
\pgfsetstrokecolor{textcolor}%
\pgfsetfillcolor{textcolor}%
\pgftext[x=1.712274in,y=0.234413in,,top]{\color{textcolor}\rmfamily\fontsize{10.000000}{12.000000}\selectfont 5}%
\end{pgfscope}%
\begin{pgfscope}%
\pgfpathrectangle{\pgfqpoint{0.418102in}{0.331635in}}{\pgfqpoint{4.843750in}{3.775000in}}%
\pgfusepath{clip}%
\pgfsetroundcap%
\pgfsetroundjoin%
\pgfsetlinewidth{0.803000pt}%
\definecolor{currentstroke}{rgb}{1.000000,1.000000,1.000000}%
\pgfsetstrokecolor{currentstroke}%
\pgfsetdash{}{0pt}%
\pgfpathmoveto{\pgfqpoint{2.786277in}{0.331635in}}%
\pgfpathlineto{\pgfqpoint{2.786277in}{4.106635in}}%
\pgfusepath{stroke}%
\end{pgfscope}%
\begin{pgfscope}%
\definecolor{textcolor}{rgb}{0.150000,0.150000,0.150000}%
\pgfsetstrokecolor{textcolor}%
\pgfsetfillcolor{textcolor}%
\pgftext[x=2.786277in,y=0.234413in,,top]{\color{textcolor}\rmfamily\fontsize{10.000000}{12.000000}\selectfont 10}%
\end{pgfscope}%
\begin{pgfscope}%
\pgfpathrectangle{\pgfqpoint{0.418102in}{0.331635in}}{\pgfqpoint{4.843750in}{3.775000in}}%
\pgfusepath{clip}%
\pgfsetroundcap%
\pgfsetroundjoin%
\pgfsetlinewidth{0.803000pt}%
\definecolor{currentstroke}{rgb}{1.000000,1.000000,1.000000}%
\pgfsetstrokecolor{currentstroke}%
\pgfsetdash{}{0pt}%
\pgfpathmoveto{\pgfqpoint{3.860279in}{0.331635in}}%
\pgfpathlineto{\pgfqpoint{3.860279in}{4.106635in}}%
\pgfusepath{stroke}%
\end{pgfscope}%
\begin{pgfscope}%
\definecolor{textcolor}{rgb}{0.150000,0.150000,0.150000}%
\pgfsetstrokecolor{textcolor}%
\pgfsetfillcolor{textcolor}%
\pgftext[x=3.860279in,y=0.234413in,,top]{\color{textcolor}\rmfamily\fontsize{10.000000}{12.000000}\selectfont 15}%
\end{pgfscope}%
\begin{pgfscope}%
\pgfpathrectangle{\pgfqpoint{0.418102in}{0.331635in}}{\pgfqpoint{4.843750in}{3.775000in}}%
\pgfusepath{clip}%
\pgfsetroundcap%
\pgfsetroundjoin%
\pgfsetlinewidth{0.803000pt}%
\definecolor{currentstroke}{rgb}{1.000000,1.000000,1.000000}%
\pgfsetstrokecolor{currentstroke}%
\pgfsetdash{}{0pt}%
\pgfpathmoveto{\pgfqpoint{4.934281in}{0.331635in}}%
\pgfpathlineto{\pgfqpoint{4.934281in}{4.106635in}}%
\pgfusepath{stroke}%
\end{pgfscope}%
\begin{pgfscope}%
\definecolor{textcolor}{rgb}{0.150000,0.150000,0.150000}%
\pgfsetstrokecolor{textcolor}%
\pgfsetfillcolor{textcolor}%
\pgftext[x=4.934281in,y=0.234413in,,top]{\color{textcolor}\rmfamily\fontsize{10.000000}{12.000000}\selectfont 20}%
\end{pgfscope}%
\begin{pgfscope}%
\pgfpathrectangle{\pgfqpoint{0.418102in}{0.331635in}}{\pgfqpoint{4.843750in}{3.775000in}}%
\pgfusepath{clip}%
\pgfsetroundcap%
\pgfsetroundjoin%
\pgfsetlinewidth{0.803000pt}%
\definecolor{currentstroke}{rgb}{1.000000,1.000000,1.000000}%
\pgfsetstrokecolor{currentstroke}%
\pgfsetdash{}{0pt}%
\pgfpathmoveto{\pgfqpoint{0.418102in}{0.746785in}}%
\pgfpathlineto{\pgfqpoint{5.261852in}{0.746785in}}%
\pgfusepath{stroke}%
\end{pgfscope}%
\begin{pgfscope}%
\definecolor{textcolor}{rgb}{0.150000,0.150000,0.150000}%
\pgfsetstrokecolor{textcolor}%
\pgfsetfillcolor{textcolor}%
\pgftext[x=0.100000in,y=0.694023in,left,base]{\color{textcolor}\rmfamily\fontsize{10.000000}{12.000000}\selectfont 0.0}%
\end{pgfscope}%
\begin{pgfscope}%
\pgfpathrectangle{\pgfqpoint{0.418102in}{0.331635in}}{\pgfqpoint{4.843750in}{3.775000in}}%
\pgfusepath{clip}%
\pgfsetroundcap%
\pgfsetroundjoin%
\pgfsetlinewidth{0.803000pt}%
\definecolor{currentstroke}{rgb}{1.000000,1.000000,1.000000}%
\pgfsetstrokecolor{currentstroke}%
\pgfsetdash{}{0pt}%
\pgfpathmoveto{\pgfqpoint{0.418102in}{1.384437in}}%
\pgfpathlineto{\pgfqpoint{5.261852in}{1.384437in}}%
\pgfusepath{stroke}%
\end{pgfscope}%
\begin{pgfscope}%
\definecolor{textcolor}{rgb}{0.150000,0.150000,0.150000}%
\pgfsetstrokecolor{textcolor}%
\pgfsetfillcolor{textcolor}%
\pgftext[x=0.100000in,y=1.331675in,left,base]{\color{textcolor}\rmfamily\fontsize{10.000000}{12.000000}\selectfont 0.2}%
\end{pgfscope}%
\begin{pgfscope}%
\pgfpathrectangle{\pgfqpoint{0.418102in}{0.331635in}}{\pgfqpoint{4.843750in}{3.775000in}}%
\pgfusepath{clip}%
\pgfsetroundcap%
\pgfsetroundjoin%
\pgfsetlinewidth{0.803000pt}%
\definecolor{currentstroke}{rgb}{1.000000,1.000000,1.000000}%
\pgfsetstrokecolor{currentstroke}%
\pgfsetdash{}{0pt}%
\pgfpathmoveto{\pgfqpoint{0.418102in}{2.022089in}}%
\pgfpathlineto{\pgfqpoint{5.261852in}{2.022089in}}%
\pgfusepath{stroke}%
\end{pgfscope}%
\begin{pgfscope}%
\definecolor{textcolor}{rgb}{0.150000,0.150000,0.150000}%
\pgfsetstrokecolor{textcolor}%
\pgfsetfillcolor{textcolor}%
\pgftext[x=0.100000in,y=1.969327in,left,base]{\color{textcolor}\rmfamily\fontsize{10.000000}{12.000000}\selectfont 0.4}%
\end{pgfscope}%
\begin{pgfscope}%
\pgfpathrectangle{\pgfqpoint{0.418102in}{0.331635in}}{\pgfqpoint{4.843750in}{3.775000in}}%
\pgfusepath{clip}%
\pgfsetroundcap%
\pgfsetroundjoin%
\pgfsetlinewidth{0.803000pt}%
\definecolor{currentstroke}{rgb}{1.000000,1.000000,1.000000}%
\pgfsetstrokecolor{currentstroke}%
\pgfsetdash{}{0pt}%
\pgfpathmoveto{\pgfqpoint{0.418102in}{2.659740in}}%
\pgfpathlineto{\pgfqpoint{5.261852in}{2.659740in}}%
\pgfusepath{stroke}%
\end{pgfscope}%
\begin{pgfscope}%
\definecolor{textcolor}{rgb}{0.150000,0.150000,0.150000}%
\pgfsetstrokecolor{textcolor}%
\pgfsetfillcolor{textcolor}%
\pgftext[x=0.100000in,y=2.606979in,left,base]{\color{textcolor}\rmfamily\fontsize{10.000000}{12.000000}\selectfont 0.6}%
\end{pgfscope}%
\begin{pgfscope}%
\pgfpathrectangle{\pgfqpoint{0.418102in}{0.331635in}}{\pgfqpoint{4.843750in}{3.775000in}}%
\pgfusepath{clip}%
\pgfsetroundcap%
\pgfsetroundjoin%
\pgfsetlinewidth{0.803000pt}%
\definecolor{currentstroke}{rgb}{1.000000,1.000000,1.000000}%
\pgfsetstrokecolor{currentstroke}%
\pgfsetdash{}{0pt}%
\pgfpathmoveto{\pgfqpoint{0.418102in}{3.297392in}}%
\pgfpathlineto{\pgfqpoint{5.261852in}{3.297392in}}%
\pgfusepath{stroke}%
\end{pgfscope}%
\begin{pgfscope}%
\definecolor{textcolor}{rgb}{0.150000,0.150000,0.150000}%
\pgfsetstrokecolor{textcolor}%
\pgfsetfillcolor{textcolor}%
\pgftext[x=0.100000in,y=3.244631in,left,base]{\color{textcolor}\rmfamily\fontsize{10.000000}{12.000000}\selectfont 0.8}%
\end{pgfscope}%
\begin{pgfscope}%
\pgfpathrectangle{\pgfqpoint{0.418102in}{0.331635in}}{\pgfqpoint{4.843750in}{3.775000in}}%
\pgfusepath{clip}%
\pgfsetroundcap%
\pgfsetroundjoin%
\pgfsetlinewidth{0.803000pt}%
\definecolor{currentstroke}{rgb}{1.000000,1.000000,1.000000}%
\pgfsetstrokecolor{currentstroke}%
\pgfsetdash{}{0pt}%
\pgfpathmoveto{\pgfqpoint{0.418102in}{3.935044in}}%
\pgfpathlineto{\pgfqpoint{5.261852in}{3.935044in}}%
\pgfusepath{stroke}%
\end{pgfscope}%
\begin{pgfscope}%
\definecolor{textcolor}{rgb}{0.150000,0.150000,0.150000}%
\pgfsetstrokecolor{textcolor}%
\pgfsetfillcolor{textcolor}%
\pgftext[x=0.100000in,y=3.882283in,left,base]{\color{textcolor}\rmfamily\fontsize{10.000000}{12.000000}\selectfont 1.0}%
\end{pgfscope}%
\begin{pgfscope}%
\pgfpathrectangle{\pgfqpoint{0.418102in}{0.331635in}}{\pgfqpoint{4.843750in}{3.775000in}}%
\pgfusepath{clip}%
\pgfsetbuttcap%
\pgfsetroundjoin%
\definecolor{currentfill}{rgb}{0.121569,0.466667,0.705882}%
\pgfsetfillcolor{currentfill}%
\pgfsetfillopacity{0.250000}%
\pgfsetlinewidth{1.003750pt}%
\definecolor{currentstroke}{rgb}{1.000000,1.000000,1.000000}%
\pgfsetstrokecolor{currentstroke}%
\pgfsetstrokeopacity{0.250000}%
\pgfsetdash{}{0pt}%
\pgfpathmoveto{\pgfqpoint{0.745672in}{0.907702in}}%
\pgfpathlineto{\pgfqpoint{0.745672in}{0.585868in}}%
\pgfpathlineto{\pgfqpoint{1.067873in}{0.585838in}}%
\pgfpathlineto{\pgfqpoint{1.282673in}{0.585837in}}%
\pgfpathlineto{\pgfqpoint{1.497474in}{0.585837in}}%
\pgfpathlineto{\pgfqpoint{1.712274in}{0.585817in}}%
\pgfpathlineto{\pgfqpoint{1.927075in}{0.585816in}}%
\pgfpathlineto{\pgfqpoint{2.141875in}{0.585804in}}%
\pgfpathlineto{\pgfqpoint{2.356676in}{0.585721in}}%
\pgfpathlineto{\pgfqpoint{2.571476in}{0.585700in}}%
\pgfpathlineto{\pgfqpoint{2.786277in}{0.585487in}}%
\pgfpathlineto{\pgfqpoint{3.001077in}{0.585237in}}%
\pgfpathlineto{\pgfqpoint{3.215877in}{0.585088in}}%
\pgfpathlineto{\pgfqpoint{3.430678in}{0.585069in}}%
\pgfpathlineto{\pgfqpoint{3.645478in}{0.584677in}}%
\pgfpathlineto{\pgfqpoint{3.860279in}{0.584673in}}%
\pgfpathlineto{\pgfqpoint{4.075079in}{0.583743in}}%
\pgfpathlineto{\pgfqpoint{4.289880in}{0.583410in}}%
\pgfpathlineto{\pgfqpoint{4.504680in}{0.582993in}}%
\pgfpathlineto{\pgfqpoint{4.719481in}{0.582167in}}%
\pgfpathlineto{\pgfqpoint{5.041681in}{0.582164in}}%
\pgfpathlineto{\pgfqpoint{5.041681in}{0.911405in}}%
\pgfpathlineto{\pgfqpoint{5.041681in}{0.911405in}}%
\pgfpathlineto{\pgfqpoint{4.719481in}{0.911403in}}%
\pgfpathlineto{\pgfqpoint{4.504680in}{0.910577in}}%
\pgfpathlineto{\pgfqpoint{4.289880in}{0.910160in}}%
\pgfpathlineto{\pgfqpoint{4.075079in}{0.909827in}}%
\pgfpathlineto{\pgfqpoint{3.860279in}{0.908897in}}%
\pgfpathlineto{\pgfqpoint{3.645478in}{0.908893in}}%
\pgfpathlineto{\pgfqpoint{3.430678in}{0.908501in}}%
\pgfpathlineto{\pgfqpoint{3.215877in}{0.908482in}}%
\pgfpathlineto{\pgfqpoint{3.001077in}{0.908332in}}%
\pgfpathlineto{\pgfqpoint{2.786277in}{0.908083in}}%
\pgfpathlineto{\pgfqpoint{2.571476in}{0.907869in}}%
\pgfpathlineto{\pgfqpoint{2.356676in}{0.907848in}}%
\pgfpathlineto{\pgfqpoint{2.141875in}{0.907766in}}%
\pgfpathlineto{\pgfqpoint{1.927075in}{0.907754in}}%
\pgfpathlineto{\pgfqpoint{1.712274in}{0.907753in}}%
\pgfpathlineto{\pgfqpoint{1.497474in}{0.907733in}}%
\pgfpathlineto{\pgfqpoint{1.282673in}{0.907732in}}%
\pgfpathlineto{\pgfqpoint{1.067873in}{0.907732in}}%
\pgfpathlineto{\pgfqpoint{0.745672in}{0.907702in}}%
\pgfpathclose%
\pgfusepath{stroke,fill}%
\end{pgfscope}%
\begin{pgfscope}%
\pgfpathrectangle{\pgfqpoint{0.418102in}{0.331635in}}{\pgfqpoint{4.843750in}{3.775000in}}%
\pgfusepath{clip}%
\pgfsetbuttcap%
\pgfsetroundjoin%
\pgfsetlinewidth{1.505625pt}%
\definecolor{currentstroke}{rgb}{0.000000,0.000000,0.000000}%
\pgfsetstrokecolor{currentstroke}%
\pgfsetdash{}{0pt}%
\pgfpathmoveto{\pgfqpoint{0.638272in}{0.746785in}}%
\pgfpathlineto{\pgfqpoint{0.638272in}{3.935044in}}%
\pgfusepath{stroke}%
\end{pgfscope}%
\begin{pgfscope}%
\pgfpathrectangle{\pgfqpoint{0.418102in}{0.331635in}}{\pgfqpoint{4.843750in}{3.775000in}}%
\pgfusepath{clip}%
\pgfsetbuttcap%
\pgfsetroundjoin%
\pgfsetlinewidth{1.505625pt}%
\definecolor{currentstroke}{rgb}{0.000000,0.000000,0.000000}%
\pgfsetstrokecolor{currentstroke}%
\pgfsetdash{}{0pt}%
\pgfpathmoveto{\pgfqpoint{0.853073in}{0.746785in}}%
\pgfpathlineto{\pgfqpoint{0.853073in}{0.703053in}}%
\pgfusepath{stroke}%
\end{pgfscope}%
\begin{pgfscope}%
\pgfpathrectangle{\pgfqpoint{0.418102in}{0.331635in}}{\pgfqpoint{4.843750in}{3.775000in}}%
\pgfusepath{clip}%
\pgfsetbuttcap%
\pgfsetroundjoin%
\pgfsetlinewidth{1.505625pt}%
\definecolor{currentstroke}{rgb}{0.000000,0.000000,0.000000}%
\pgfsetstrokecolor{currentstroke}%
\pgfsetdash{}{0pt}%
\pgfpathmoveto{\pgfqpoint{1.067873in}{0.746785in}}%
\pgfpathlineto{\pgfqpoint{1.067873in}{0.752559in}}%
\pgfusepath{stroke}%
\end{pgfscope}%
\begin{pgfscope}%
\pgfpathrectangle{\pgfqpoint{0.418102in}{0.331635in}}{\pgfqpoint{4.843750in}{3.775000in}}%
\pgfusepath{clip}%
\pgfsetbuttcap%
\pgfsetroundjoin%
\pgfsetlinewidth{1.505625pt}%
\definecolor{currentstroke}{rgb}{0.000000,0.000000,0.000000}%
\pgfsetstrokecolor{currentstroke}%
\pgfsetdash{}{0pt}%
\pgfpathmoveto{\pgfqpoint{1.282673in}{0.746785in}}%
\pgfpathlineto{\pgfqpoint{1.282673in}{0.749973in}}%
\pgfusepath{stroke}%
\end{pgfscope}%
\begin{pgfscope}%
\pgfpathrectangle{\pgfqpoint{0.418102in}{0.331635in}}{\pgfqpoint{4.843750in}{3.775000in}}%
\pgfusepath{clip}%
\pgfsetbuttcap%
\pgfsetroundjoin%
\pgfsetlinewidth{1.505625pt}%
\definecolor{currentstroke}{rgb}{0.000000,0.000000,0.000000}%
\pgfsetstrokecolor{currentstroke}%
\pgfsetdash{}{0pt}%
\pgfpathmoveto{\pgfqpoint{1.497474in}{0.746785in}}%
\pgfpathlineto{\pgfqpoint{1.497474in}{0.710804in}}%
\pgfusepath{stroke}%
\end{pgfscope}%
\begin{pgfscope}%
\pgfpathrectangle{\pgfqpoint{0.418102in}{0.331635in}}{\pgfqpoint{4.843750in}{3.775000in}}%
\pgfusepath{clip}%
\pgfsetbuttcap%
\pgfsetroundjoin%
\pgfsetlinewidth{1.505625pt}%
\definecolor{currentstroke}{rgb}{0.000000,0.000000,0.000000}%
\pgfsetstrokecolor{currentstroke}%
\pgfsetdash{}{0pt}%
\pgfpathmoveto{\pgfqpoint{1.712274in}{0.746785in}}%
\pgfpathlineto{\pgfqpoint{1.712274in}{0.753166in}}%
\pgfusepath{stroke}%
\end{pgfscope}%
\begin{pgfscope}%
\pgfpathrectangle{\pgfqpoint{0.418102in}{0.331635in}}{\pgfqpoint{4.843750in}{3.775000in}}%
\pgfusepath{clip}%
\pgfsetbuttcap%
\pgfsetroundjoin%
\pgfsetlinewidth{1.505625pt}%
\definecolor{currentstroke}{rgb}{0.000000,0.000000,0.000000}%
\pgfsetstrokecolor{currentstroke}%
\pgfsetdash{}{0pt}%
\pgfpathmoveto{\pgfqpoint{1.927075in}{0.746785in}}%
\pgfpathlineto{\pgfqpoint{1.927075in}{0.718603in}}%
\pgfusepath{stroke}%
\end{pgfscope}%
\begin{pgfscope}%
\pgfpathrectangle{\pgfqpoint{0.418102in}{0.331635in}}{\pgfqpoint{4.843750in}{3.775000in}}%
\pgfusepath{clip}%
\pgfsetbuttcap%
\pgfsetroundjoin%
\pgfsetlinewidth{1.505625pt}%
\definecolor{currentstroke}{rgb}{0.000000,0.000000,0.000000}%
\pgfsetstrokecolor{currentstroke}%
\pgfsetdash{}{0pt}%
\pgfpathmoveto{\pgfqpoint{2.141875in}{0.746785in}}%
\pgfpathlineto{\pgfqpoint{2.141875in}{0.818891in}}%
\pgfusepath{stroke}%
\end{pgfscope}%
\begin{pgfscope}%
\pgfpathrectangle{\pgfqpoint{0.418102in}{0.331635in}}{\pgfqpoint{4.843750in}{3.775000in}}%
\pgfusepath{clip}%
\pgfsetbuttcap%
\pgfsetroundjoin%
\pgfsetlinewidth{1.505625pt}%
\definecolor{currentstroke}{rgb}{0.000000,0.000000,0.000000}%
\pgfsetstrokecolor{currentstroke}%
\pgfsetdash{}{0pt}%
\pgfpathmoveto{\pgfqpoint{2.356676in}{0.746785in}}%
\pgfpathlineto{\pgfqpoint{2.356676in}{0.783218in}}%
\pgfusepath{stroke}%
\end{pgfscope}%
\begin{pgfscope}%
\pgfpathrectangle{\pgfqpoint{0.418102in}{0.331635in}}{\pgfqpoint{4.843750in}{3.775000in}}%
\pgfusepath{clip}%
\pgfsetbuttcap%
\pgfsetroundjoin%
\pgfsetlinewidth{1.505625pt}%
\definecolor{currentstroke}{rgb}{0.000000,0.000000,0.000000}%
\pgfsetstrokecolor{currentstroke}%
\pgfsetdash{}{0pt}%
\pgfpathmoveto{\pgfqpoint{2.571476in}{0.746785in}}%
\pgfpathlineto{\pgfqpoint{2.571476in}{0.630682in}}%
\pgfusepath{stroke}%
\end{pgfscope}%
\begin{pgfscope}%
\pgfpathrectangle{\pgfqpoint{0.418102in}{0.331635in}}{\pgfqpoint{4.843750in}{3.775000in}}%
\pgfusepath{clip}%
\pgfsetbuttcap%
\pgfsetroundjoin%
\pgfsetlinewidth{1.505625pt}%
\definecolor{currentstroke}{rgb}{0.000000,0.000000,0.000000}%
\pgfsetstrokecolor{currentstroke}%
\pgfsetdash{}{0pt}%
\pgfpathmoveto{\pgfqpoint{2.786277in}{0.746785in}}%
\pgfpathlineto{\pgfqpoint{2.786277in}{0.872633in}}%
\pgfusepath{stroke}%
\end{pgfscope}%
\begin{pgfscope}%
\pgfpathrectangle{\pgfqpoint{0.418102in}{0.331635in}}{\pgfqpoint{4.843750in}{3.775000in}}%
\pgfusepath{clip}%
\pgfsetbuttcap%
\pgfsetroundjoin%
\pgfsetlinewidth{1.505625pt}%
\definecolor{currentstroke}{rgb}{0.000000,0.000000,0.000000}%
\pgfsetstrokecolor{currentstroke}%
\pgfsetdash{}{0pt}%
\pgfpathmoveto{\pgfqpoint{3.001077in}{0.746785in}}%
\pgfpathlineto{\pgfqpoint{3.001077in}{0.844296in}}%
\pgfusepath{stroke}%
\end{pgfscope}%
\begin{pgfscope}%
\pgfpathrectangle{\pgfqpoint{0.418102in}{0.331635in}}{\pgfqpoint{4.843750in}{3.775000in}}%
\pgfusepath{clip}%
\pgfsetbuttcap%
\pgfsetroundjoin%
\pgfsetlinewidth{1.505625pt}%
\definecolor{currentstroke}{rgb}{0.000000,0.000000,0.000000}%
\pgfsetstrokecolor{currentstroke}%
\pgfsetdash{}{0pt}%
\pgfpathmoveto{\pgfqpoint{3.215877in}{0.746785in}}%
\pgfpathlineto{\pgfqpoint{3.215877in}{0.781273in}}%
\pgfusepath{stroke}%
\end{pgfscope}%
\begin{pgfscope}%
\pgfpathrectangle{\pgfqpoint{0.418102in}{0.331635in}}{\pgfqpoint{4.843750in}{3.775000in}}%
\pgfusepath{clip}%
\pgfsetbuttcap%
\pgfsetroundjoin%
\pgfsetlinewidth{1.505625pt}%
\definecolor{currentstroke}{rgb}{0.000000,0.000000,0.000000}%
\pgfsetstrokecolor{currentstroke}%
\pgfsetdash{}{0pt}%
\pgfpathmoveto{\pgfqpoint{3.430678in}{0.746785in}}%
\pgfpathlineto{\pgfqpoint{3.430678in}{0.589026in}}%
\pgfusepath{stroke}%
\end{pgfscope}%
\begin{pgfscope}%
\pgfpathrectangle{\pgfqpoint{0.418102in}{0.331635in}}{\pgfqpoint{4.843750in}{3.775000in}}%
\pgfusepath{clip}%
\pgfsetbuttcap%
\pgfsetroundjoin%
\pgfsetlinewidth{1.505625pt}%
\definecolor{currentstroke}{rgb}{0.000000,0.000000,0.000000}%
\pgfsetstrokecolor{currentstroke}%
\pgfsetdash{}{0pt}%
\pgfpathmoveto{\pgfqpoint{3.645478in}{0.746785in}}%
\pgfpathlineto{\pgfqpoint{3.645478in}{0.763838in}}%
\pgfusepath{stroke}%
\end{pgfscope}%
\begin{pgfscope}%
\pgfpathrectangle{\pgfqpoint{0.418102in}{0.331635in}}{\pgfqpoint{4.843750in}{3.775000in}}%
\pgfusepath{clip}%
\pgfsetbuttcap%
\pgfsetroundjoin%
\pgfsetlinewidth{1.505625pt}%
\definecolor{currentstroke}{rgb}{0.000000,0.000000,0.000000}%
\pgfsetstrokecolor{currentstroke}%
\pgfsetdash{}{0pt}%
\pgfpathmoveto{\pgfqpoint{3.860279in}{0.746785in}}%
\pgfpathlineto{\pgfqpoint{3.860279in}{0.503226in}}%
\pgfusepath{stroke}%
\end{pgfscope}%
\begin{pgfscope}%
\pgfpathrectangle{\pgfqpoint{0.418102in}{0.331635in}}{\pgfqpoint{4.843750in}{3.775000in}}%
\pgfusepath{clip}%
\pgfsetbuttcap%
\pgfsetroundjoin%
\pgfsetlinewidth{1.505625pt}%
\definecolor{currentstroke}{rgb}{0.000000,0.000000,0.000000}%
\pgfsetstrokecolor{currentstroke}%
\pgfsetdash{}{0pt}%
\pgfpathmoveto{\pgfqpoint{4.075079in}{0.746785in}}%
\pgfpathlineto{\pgfqpoint{4.075079in}{0.600727in}}%
\pgfusepath{stroke}%
\end{pgfscope}%
\begin{pgfscope}%
\pgfpathrectangle{\pgfqpoint{0.418102in}{0.331635in}}{\pgfqpoint{4.843750in}{3.775000in}}%
\pgfusepath{clip}%
\pgfsetbuttcap%
\pgfsetroundjoin%
\pgfsetlinewidth{1.505625pt}%
\definecolor{currentstroke}{rgb}{0.000000,0.000000,0.000000}%
\pgfsetstrokecolor{currentstroke}%
\pgfsetdash{}{0pt}%
\pgfpathmoveto{\pgfqpoint{4.289880in}{0.746785in}}%
\pgfpathlineto{\pgfqpoint{4.289880in}{0.910445in}}%
\pgfusepath{stroke}%
\end{pgfscope}%
\begin{pgfscope}%
\pgfpathrectangle{\pgfqpoint{0.418102in}{0.331635in}}{\pgfqpoint{4.843750in}{3.775000in}}%
\pgfusepath{clip}%
\pgfsetbuttcap%
\pgfsetroundjoin%
\pgfsetlinewidth{1.505625pt}%
\definecolor{currentstroke}{rgb}{0.000000,0.000000,0.000000}%
\pgfsetstrokecolor{currentstroke}%
\pgfsetdash{}{0pt}%
\pgfpathmoveto{\pgfqpoint{4.504680in}{0.746785in}}%
\pgfpathlineto{\pgfqpoint{4.504680in}{0.977555in}}%
\pgfusepath{stroke}%
\end{pgfscope}%
\begin{pgfscope}%
\pgfpathrectangle{\pgfqpoint{0.418102in}{0.331635in}}{\pgfqpoint{4.843750in}{3.775000in}}%
\pgfusepath{clip}%
\pgfsetbuttcap%
\pgfsetroundjoin%
\pgfsetlinewidth{1.505625pt}%
\definecolor{currentstroke}{rgb}{0.000000,0.000000,0.000000}%
\pgfsetstrokecolor{currentstroke}%
\pgfsetdash{}{0pt}%
\pgfpathmoveto{\pgfqpoint{4.719481in}{0.746785in}}%
\pgfpathlineto{\pgfqpoint{4.719481in}{0.759369in}}%
\pgfusepath{stroke}%
\end{pgfscope}%
\begin{pgfscope}%
\pgfpathrectangle{\pgfqpoint{0.418102in}{0.331635in}}{\pgfqpoint{4.843750in}{3.775000in}}%
\pgfusepath{clip}%
\pgfsetbuttcap%
\pgfsetroundjoin%
\pgfsetlinewidth{1.505625pt}%
\definecolor{currentstroke}{rgb}{0.000000,0.000000,0.000000}%
\pgfsetstrokecolor{currentstroke}%
\pgfsetdash{}{0pt}%
\pgfpathmoveto{\pgfqpoint{4.934281in}{0.746785in}}%
\pgfpathlineto{\pgfqpoint{4.934281in}{0.758303in}}%
\pgfusepath{stroke}%
\end{pgfscope}%
\begin{pgfscope}%
\pgfpathrectangle{\pgfqpoint{0.418102in}{0.331635in}}{\pgfqpoint{4.843750in}{3.775000in}}%
\pgfusepath{clip}%
\pgfsetroundcap%
\pgfsetroundjoin%
\pgfsetlinewidth{1.505625pt}%
\definecolor{currentstroke}{rgb}{0.839216,0.152941,0.156863}%
\pgfsetstrokecolor{currentstroke}%
\pgfsetdash{}{0pt}%
\pgfpathmoveto{\pgfqpoint{0.418102in}{0.746785in}}%
\pgfpathlineto{\pgfqpoint{5.261852in}{0.746785in}}%
\pgfusepath{stroke}%
\end{pgfscope}%
\begin{pgfscope}%
\pgfpathrectangle{\pgfqpoint{0.418102in}{0.331635in}}{\pgfqpoint{4.843750in}{3.775000in}}%
\pgfusepath{clip}%
\pgfsetbuttcap%
\pgfsetroundjoin%
\definecolor{currentfill}{rgb}{0.839216,0.152941,0.156863}%
\pgfsetfillcolor{currentfill}%
\pgfsetlinewidth{1.003750pt}%
\definecolor{currentstroke}{rgb}{0.839216,0.152941,0.156863}%
\pgfsetstrokecolor{currentstroke}%
\pgfsetdash{}{0pt}%
\pgfsys@defobject{currentmarker}{\pgfqpoint{-0.034722in}{-0.034722in}}{\pgfqpoint{0.034722in}{0.034722in}}{%
\pgfpathmoveto{\pgfqpoint{0.000000in}{-0.034722in}}%
\pgfpathcurveto{\pgfqpoint{0.009208in}{-0.034722in}}{\pgfqpoint{0.018041in}{-0.031064in}}{\pgfqpoint{0.024552in}{-0.024552in}}%
\pgfpathcurveto{\pgfqpoint{0.031064in}{-0.018041in}}{\pgfqpoint{0.034722in}{-0.009208in}}{\pgfqpoint{0.034722in}{0.000000in}}%
\pgfpathcurveto{\pgfqpoint{0.034722in}{0.009208in}}{\pgfqpoint{0.031064in}{0.018041in}}{\pgfqpoint{0.024552in}{0.024552in}}%
\pgfpathcurveto{\pgfqpoint{0.018041in}{0.031064in}}{\pgfqpoint{0.009208in}{0.034722in}}{\pgfqpoint{0.000000in}{0.034722in}}%
\pgfpathcurveto{\pgfqpoint{-0.009208in}{0.034722in}}{\pgfqpoint{-0.018041in}{0.031064in}}{\pgfqpoint{-0.024552in}{0.024552in}}%
\pgfpathcurveto{\pgfqpoint{-0.031064in}{0.018041in}}{\pgfqpoint{-0.034722in}{0.009208in}}{\pgfqpoint{-0.034722in}{0.000000in}}%
\pgfpathcurveto{\pgfqpoint{-0.034722in}{-0.009208in}}{\pgfqpoint{-0.031064in}{-0.018041in}}{\pgfqpoint{-0.024552in}{-0.024552in}}%
\pgfpathcurveto{\pgfqpoint{-0.018041in}{-0.031064in}}{\pgfqpoint{-0.009208in}{-0.034722in}}{\pgfqpoint{0.000000in}{-0.034722in}}%
\pgfpathclose%
\pgfusepath{stroke,fill}%
}%
\begin{pgfscope}%
\pgfsys@transformshift{0.638272in}{3.935044in}%
\pgfsys@useobject{currentmarker}{}%
\end{pgfscope}%
\begin{pgfscope}%
\pgfsys@transformshift{0.853073in}{0.703053in}%
\pgfsys@useobject{currentmarker}{}%
\end{pgfscope}%
\begin{pgfscope}%
\pgfsys@transformshift{1.067873in}{0.752559in}%
\pgfsys@useobject{currentmarker}{}%
\end{pgfscope}%
\begin{pgfscope}%
\pgfsys@transformshift{1.282673in}{0.749973in}%
\pgfsys@useobject{currentmarker}{}%
\end{pgfscope}%
\begin{pgfscope}%
\pgfsys@transformshift{1.497474in}{0.710804in}%
\pgfsys@useobject{currentmarker}{}%
\end{pgfscope}%
\begin{pgfscope}%
\pgfsys@transformshift{1.712274in}{0.753166in}%
\pgfsys@useobject{currentmarker}{}%
\end{pgfscope}%
\begin{pgfscope}%
\pgfsys@transformshift{1.927075in}{0.718603in}%
\pgfsys@useobject{currentmarker}{}%
\end{pgfscope}%
\begin{pgfscope}%
\pgfsys@transformshift{2.141875in}{0.818891in}%
\pgfsys@useobject{currentmarker}{}%
\end{pgfscope}%
\begin{pgfscope}%
\pgfsys@transformshift{2.356676in}{0.783218in}%
\pgfsys@useobject{currentmarker}{}%
\end{pgfscope}%
\begin{pgfscope}%
\pgfsys@transformshift{2.571476in}{0.630682in}%
\pgfsys@useobject{currentmarker}{}%
\end{pgfscope}%
\begin{pgfscope}%
\pgfsys@transformshift{2.786277in}{0.872633in}%
\pgfsys@useobject{currentmarker}{}%
\end{pgfscope}%
\begin{pgfscope}%
\pgfsys@transformshift{3.001077in}{0.844296in}%
\pgfsys@useobject{currentmarker}{}%
\end{pgfscope}%
\begin{pgfscope}%
\pgfsys@transformshift{3.215877in}{0.781273in}%
\pgfsys@useobject{currentmarker}{}%
\end{pgfscope}%
\begin{pgfscope}%
\pgfsys@transformshift{3.430678in}{0.589026in}%
\pgfsys@useobject{currentmarker}{}%
\end{pgfscope}%
\begin{pgfscope}%
\pgfsys@transformshift{3.645478in}{0.763838in}%
\pgfsys@useobject{currentmarker}{}%
\end{pgfscope}%
\begin{pgfscope}%
\pgfsys@transformshift{3.860279in}{0.503226in}%
\pgfsys@useobject{currentmarker}{}%
\end{pgfscope}%
\begin{pgfscope}%
\pgfsys@transformshift{4.075079in}{0.600727in}%
\pgfsys@useobject{currentmarker}{}%
\end{pgfscope}%
\begin{pgfscope}%
\pgfsys@transformshift{4.289880in}{0.910445in}%
\pgfsys@useobject{currentmarker}{}%
\end{pgfscope}%
\begin{pgfscope}%
\pgfsys@transformshift{4.504680in}{0.977555in}%
\pgfsys@useobject{currentmarker}{}%
\end{pgfscope}%
\begin{pgfscope}%
\pgfsys@transformshift{4.719481in}{0.759369in}%
\pgfsys@useobject{currentmarker}{}%
\end{pgfscope}%
\begin{pgfscope}%
\pgfsys@transformshift{4.934281in}{0.758303in}%
\pgfsys@useobject{currentmarker}{}%
\end{pgfscope}%
\end{pgfscope}%
\begin{pgfscope}%
\pgfsetrectcap%
\pgfsetmiterjoin%
\pgfsetlinewidth{0.803000pt}%
\definecolor{currentstroke}{rgb}{1.000000,1.000000,1.000000}%
\pgfsetstrokecolor{currentstroke}%
\pgfsetdash{}{0pt}%
\pgfpathmoveto{\pgfqpoint{0.418102in}{0.331635in}}%
\pgfpathlineto{\pgfqpoint{0.418102in}{4.106635in}}%
\pgfusepath{stroke}%
\end{pgfscope}%
\begin{pgfscope}%
\pgfsetrectcap%
\pgfsetmiterjoin%
\pgfsetlinewidth{0.803000pt}%
\definecolor{currentstroke}{rgb}{1.000000,1.000000,1.000000}%
\pgfsetstrokecolor{currentstroke}%
\pgfsetdash{}{0pt}%
\pgfpathmoveto{\pgfqpoint{5.261852in}{0.331635in}}%
\pgfpathlineto{\pgfqpoint{5.261852in}{4.106635in}}%
\pgfusepath{stroke}%
\end{pgfscope}%
\begin{pgfscope}%
\pgfsetrectcap%
\pgfsetmiterjoin%
\pgfsetlinewidth{0.803000pt}%
\definecolor{currentstroke}{rgb}{1.000000,1.000000,1.000000}%
\pgfsetstrokecolor{currentstroke}%
\pgfsetdash{}{0pt}%
\pgfpathmoveto{\pgfqpoint{0.418102in}{0.331635in}}%
\pgfpathlineto{\pgfqpoint{5.261852in}{0.331635in}}%
\pgfusepath{stroke}%
\end{pgfscope}%
\begin{pgfscope}%
\pgfsetrectcap%
\pgfsetmiterjoin%
\pgfsetlinewidth{0.803000pt}%
\definecolor{currentstroke}{rgb}{1.000000,1.000000,1.000000}%
\pgfsetstrokecolor{currentstroke}%
\pgfsetdash{}{0pt}%
\pgfpathmoveto{\pgfqpoint{0.418102in}{4.106635in}}%
\pgfpathlineto{\pgfqpoint{5.261852in}{4.106635in}}%
\pgfusepath{stroke}%
\end{pgfscope}%
\begin{pgfscope}%
\definecolor{textcolor}{rgb}{0.150000,0.150000,0.150000}%
\pgfsetstrokecolor{textcolor}%
\pgfsetfillcolor{textcolor}%
\pgftext[x=2.839977in,y=4.189968in,,base]{\color{textcolor}\rmfamily\fontsize{12.000000}{14.400000}\selectfont Autocorrelation}%
\end{pgfscope}%
\begin{pgfscope}%
\pgfsetbuttcap%
\pgfsetmiterjoin%
\definecolor{currentfill}{rgb}{0.917647,0.917647,0.949020}%
\pgfsetfillcolor{currentfill}%
\pgfsetlinewidth{0.000000pt}%
\definecolor{currentstroke}{rgb}{0.000000,0.000000,0.000000}%
\pgfsetstrokecolor{currentstroke}%
\pgfsetstrokeopacity{0.000000}%
\pgfsetdash{}{0pt}%
\pgfpathmoveto{\pgfqpoint{7.199352in}{0.331635in}}%
\pgfpathlineto{\pgfqpoint{12.043102in}{0.331635in}}%
\pgfpathlineto{\pgfqpoint{12.043102in}{4.106635in}}%
\pgfpathlineto{\pgfqpoint{7.199352in}{4.106635in}}%
\pgfpathclose%
\pgfusepath{fill}%
\end{pgfscope}%
\begin{pgfscope}%
\pgfpathrectangle{\pgfqpoint{7.199352in}{0.331635in}}{\pgfqpoint{4.843750in}{3.775000in}}%
\pgfusepath{clip}%
\pgfsetroundcap%
\pgfsetroundjoin%
\pgfsetlinewidth{0.803000pt}%
\definecolor{currentstroke}{rgb}{1.000000,1.000000,1.000000}%
\pgfsetstrokecolor{currentstroke}%
\pgfsetdash{}{0pt}%
\pgfpathmoveto{\pgfqpoint{7.419522in}{0.331635in}}%
\pgfpathlineto{\pgfqpoint{7.419522in}{4.106635in}}%
\pgfusepath{stroke}%
\end{pgfscope}%
\begin{pgfscope}%
\definecolor{textcolor}{rgb}{0.150000,0.150000,0.150000}%
\pgfsetstrokecolor{textcolor}%
\pgfsetfillcolor{textcolor}%
\pgftext[x=7.419522in,y=0.234413in,,top]{\color{textcolor}\rmfamily\fontsize{10.000000}{12.000000}\selectfont 0}%
\end{pgfscope}%
\begin{pgfscope}%
\pgfpathrectangle{\pgfqpoint{7.199352in}{0.331635in}}{\pgfqpoint{4.843750in}{3.775000in}}%
\pgfusepath{clip}%
\pgfsetroundcap%
\pgfsetroundjoin%
\pgfsetlinewidth{0.803000pt}%
\definecolor{currentstroke}{rgb}{1.000000,1.000000,1.000000}%
\pgfsetstrokecolor{currentstroke}%
\pgfsetdash{}{0pt}%
\pgfpathmoveto{\pgfqpoint{8.493524in}{0.331635in}}%
\pgfpathlineto{\pgfqpoint{8.493524in}{4.106635in}}%
\pgfusepath{stroke}%
\end{pgfscope}%
\begin{pgfscope}%
\definecolor{textcolor}{rgb}{0.150000,0.150000,0.150000}%
\pgfsetstrokecolor{textcolor}%
\pgfsetfillcolor{textcolor}%
\pgftext[x=8.493524in,y=0.234413in,,top]{\color{textcolor}\rmfamily\fontsize{10.000000}{12.000000}\selectfont 5}%
\end{pgfscope}%
\begin{pgfscope}%
\pgfpathrectangle{\pgfqpoint{7.199352in}{0.331635in}}{\pgfqpoint{4.843750in}{3.775000in}}%
\pgfusepath{clip}%
\pgfsetroundcap%
\pgfsetroundjoin%
\pgfsetlinewidth{0.803000pt}%
\definecolor{currentstroke}{rgb}{1.000000,1.000000,1.000000}%
\pgfsetstrokecolor{currentstroke}%
\pgfsetdash{}{0pt}%
\pgfpathmoveto{\pgfqpoint{9.567527in}{0.331635in}}%
\pgfpathlineto{\pgfqpoint{9.567527in}{4.106635in}}%
\pgfusepath{stroke}%
\end{pgfscope}%
\begin{pgfscope}%
\definecolor{textcolor}{rgb}{0.150000,0.150000,0.150000}%
\pgfsetstrokecolor{textcolor}%
\pgfsetfillcolor{textcolor}%
\pgftext[x=9.567527in,y=0.234413in,,top]{\color{textcolor}\rmfamily\fontsize{10.000000}{12.000000}\selectfont 10}%
\end{pgfscope}%
\begin{pgfscope}%
\pgfpathrectangle{\pgfqpoint{7.199352in}{0.331635in}}{\pgfqpoint{4.843750in}{3.775000in}}%
\pgfusepath{clip}%
\pgfsetroundcap%
\pgfsetroundjoin%
\pgfsetlinewidth{0.803000pt}%
\definecolor{currentstroke}{rgb}{1.000000,1.000000,1.000000}%
\pgfsetstrokecolor{currentstroke}%
\pgfsetdash{}{0pt}%
\pgfpathmoveto{\pgfqpoint{10.641529in}{0.331635in}}%
\pgfpathlineto{\pgfqpoint{10.641529in}{4.106635in}}%
\pgfusepath{stroke}%
\end{pgfscope}%
\begin{pgfscope}%
\definecolor{textcolor}{rgb}{0.150000,0.150000,0.150000}%
\pgfsetstrokecolor{textcolor}%
\pgfsetfillcolor{textcolor}%
\pgftext[x=10.641529in,y=0.234413in,,top]{\color{textcolor}\rmfamily\fontsize{10.000000}{12.000000}\selectfont 15}%
\end{pgfscope}%
\begin{pgfscope}%
\pgfpathrectangle{\pgfqpoint{7.199352in}{0.331635in}}{\pgfqpoint{4.843750in}{3.775000in}}%
\pgfusepath{clip}%
\pgfsetroundcap%
\pgfsetroundjoin%
\pgfsetlinewidth{0.803000pt}%
\definecolor{currentstroke}{rgb}{1.000000,1.000000,1.000000}%
\pgfsetstrokecolor{currentstroke}%
\pgfsetdash{}{0pt}%
\pgfpathmoveto{\pgfqpoint{11.715531in}{0.331635in}}%
\pgfpathlineto{\pgfqpoint{11.715531in}{4.106635in}}%
\pgfusepath{stroke}%
\end{pgfscope}%
\begin{pgfscope}%
\definecolor{textcolor}{rgb}{0.150000,0.150000,0.150000}%
\pgfsetstrokecolor{textcolor}%
\pgfsetfillcolor{textcolor}%
\pgftext[x=11.715531in,y=0.234413in,,top]{\color{textcolor}\rmfamily\fontsize{10.000000}{12.000000}\selectfont 20}%
\end{pgfscope}%
\begin{pgfscope}%
\pgfpathrectangle{\pgfqpoint{7.199352in}{0.331635in}}{\pgfqpoint{4.843750in}{3.775000in}}%
\pgfusepath{clip}%
\pgfsetroundcap%
\pgfsetroundjoin%
\pgfsetlinewidth{0.803000pt}%
\definecolor{currentstroke}{rgb}{1.000000,1.000000,1.000000}%
\pgfsetstrokecolor{currentstroke}%
\pgfsetdash{}{0pt}%
\pgfpathmoveto{\pgfqpoint{7.199352in}{0.750949in}}%
\pgfpathlineto{\pgfqpoint{12.043102in}{0.750949in}}%
\pgfusepath{stroke}%
\end{pgfscope}%
\begin{pgfscope}%
\definecolor{textcolor}{rgb}{0.150000,0.150000,0.150000}%
\pgfsetstrokecolor{textcolor}%
\pgfsetfillcolor{textcolor}%
\pgftext[x=6.881250in,y=0.698188in,left,base]{\color{textcolor}\rmfamily\fontsize{10.000000}{12.000000}\selectfont 0.0}%
\end{pgfscope}%
\begin{pgfscope}%
\pgfpathrectangle{\pgfqpoint{7.199352in}{0.331635in}}{\pgfqpoint{4.843750in}{3.775000in}}%
\pgfusepath{clip}%
\pgfsetroundcap%
\pgfsetroundjoin%
\pgfsetlinewidth{0.803000pt}%
\definecolor{currentstroke}{rgb}{1.000000,1.000000,1.000000}%
\pgfsetstrokecolor{currentstroke}%
\pgfsetdash{}{0pt}%
\pgfpathmoveto{\pgfqpoint{7.199352in}{1.387768in}}%
\pgfpathlineto{\pgfqpoint{12.043102in}{1.387768in}}%
\pgfusepath{stroke}%
\end{pgfscope}%
\begin{pgfscope}%
\definecolor{textcolor}{rgb}{0.150000,0.150000,0.150000}%
\pgfsetstrokecolor{textcolor}%
\pgfsetfillcolor{textcolor}%
\pgftext[x=6.881250in,y=1.335007in,left,base]{\color{textcolor}\rmfamily\fontsize{10.000000}{12.000000}\selectfont 0.2}%
\end{pgfscope}%
\begin{pgfscope}%
\pgfpathrectangle{\pgfqpoint{7.199352in}{0.331635in}}{\pgfqpoint{4.843750in}{3.775000in}}%
\pgfusepath{clip}%
\pgfsetroundcap%
\pgfsetroundjoin%
\pgfsetlinewidth{0.803000pt}%
\definecolor{currentstroke}{rgb}{1.000000,1.000000,1.000000}%
\pgfsetstrokecolor{currentstroke}%
\pgfsetdash{}{0pt}%
\pgfpathmoveto{\pgfqpoint{7.199352in}{2.024587in}}%
\pgfpathlineto{\pgfqpoint{12.043102in}{2.024587in}}%
\pgfusepath{stroke}%
\end{pgfscope}%
\begin{pgfscope}%
\definecolor{textcolor}{rgb}{0.150000,0.150000,0.150000}%
\pgfsetstrokecolor{textcolor}%
\pgfsetfillcolor{textcolor}%
\pgftext[x=6.881250in,y=1.971826in,left,base]{\color{textcolor}\rmfamily\fontsize{10.000000}{12.000000}\selectfont 0.4}%
\end{pgfscope}%
\begin{pgfscope}%
\pgfpathrectangle{\pgfqpoint{7.199352in}{0.331635in}}{\pgfqpoint{4.843750in}{3.775000in}}%
\pgfusepath{clip}%
\pgfsetroundcap%
\pgfsetroundjoin%
\pgfsetlinewidth{0.803000pt}%
\definecolor{currentstroke}{rgb}{1.000000,1.000000,1.000000}%
\pgfsetstrokecolor{currentstroke}%
\pgfsetdash{}{0pt}%
\pgfpathmoveto{\pgfqpoint{7.199352in}{2.661406in}}%
\pgfpathlineto{\pgfqpoint{12.043102in}{2.661406in}}%
\pgfusepath{stroke}%
\end{pgfscope}%
\begin{pgfscope}%
\definecolor{textcolor}{rgb}{0.150000,0.150000,0.150000}%
\pgfsetstrokecolor{textcolor}%
\pgfsetfillcolor{textcolor}%
\pgftext[x=6.881250in,y=2.608645in,left,base]{\color{textcolor}\rmfamily\fontsize{10.000000}{12.000000}\selectfont 0.6}%
\end{pgfscope}%
\begin{pgfscope}%
\pgfpathrectangle{\pgfqpoint{7.199352in}{0.331635in}}{\pgfqpoint{4.843750in}{3.775000in}}%
\pgfusepath{clip}%
\pgfsetroundcap%
\pgfsetroundjoin%
\pgfsetlinewidth{0.803000pt}%
\definecolor{currentstroke}{rgb}{1.000000,1.000000,1.000000}%
\pgfsetstrokecolor{currentstroke}%
\pgfsetdash{}{0pt}%
\pgfpathmoveto{\pgfqpoint{7.199352in}{3.298225in}}%
\pgfpathlineto{\pgfqpoint{12.043102in}{3.298225in}}%
\pgfusepath{stroke}%
\end{pgfscope}%
\begin{pgfscope}%
\definecolor{textcolor}{rgb}{0.150000,0.150000,0.150000}%
\pgfsetstrokecolor{textcolor}%
\pgfsetfillcolor{textcolor}%
\pgftext[x=6.881250in,y=3.245464in,left,base]{\color{textcolor}\rmfamily\fontsize{10.000000}{12.000000}\selectfont 0.8}%
\end{pgfscope}%
\begin{pgfscope}%
\pgfpathrectangle{\pgfqpoint{7.199352in}{0.331635in}}{\pgfqpoint{4.843750in}{3.775000in}}%
\pgfusepath{clip}%
\pgfsetroundcap%
\pgfsetroundjoin%
\pgfsetlinewidth{0.803000pt}%
\definecolor{currentstroke}{rgb}{1.000000,1.000000,1.000000}%
\pgfsetstrokecolor{currentstroke}%
\pgfsetdash{}{0pt}%
\pgfpathmoveto{\pgfqpoint{7.199352in}{3.935044in}}%
\pgfpathlineto{\pgfqpoint{12.043102in}{3.935044in}}%
\pgfusepath{stroke}%
\end{pgfscope}%
\begin{pgfscope}%
\definecolor{textcolor}{rgb}{0.150000,0.150000,0.150000}%
\pgfsetstrokecolor{textcolor}%
\pgfsetfillcolor{textcolor}%
\pgftext[x=6.881250in,y=3.882283in,left,base]{\color{textcolor}\rmfamily\fontsize{10.000000}{12.000000}\selectfont 1.0}%
\end{pgfscope}%
\begin{pgfscope}%
\pgfpathrectangle{\pgfqpoint{7.199352in}{0.331635in}}{\pgfqpoint{4.843750in}{3.775000in}}%
\pgfusepath{clip}%
\pgfsetbuttcap%
\pgfsetroundjoin%
\definecolor{currentfill}{rgb}{0.121569,0.466667,0.705882}%
\pgfsetfillcolor{currentfill}%
\pgfsetfillopacity{0.250000}%
\pgfsetlinewidth{1.003750pt}%
\definecolor{currentstroke}{rgb}{1.000000,1.000000,1.000000}%
\pgfsetstrokecolor{currentstroke}%
\pgfsetstrokeopacity{0.250000}%
\pgfsetdash{}{0pt}%
\pgfpathmoveto{\pgfqpoint{7.526922in}{0.911656in}}%
\pgfpathlineto{\pgfqpoint{7.526922in}{0.590243in}}%
\pgfpathlineto{\pgfqpoint{7.849123in}{0.590243in}}%
\pgfpathlineto{\pgfqpoint{8.063923in}{0.590243in}}%
\pgfpathlineto{\pgfqpoint{8.278724in}{0.590243in}}%
\pgfpathlineto{\pgfqpoint{8.493524in}{0.590243in}}%
\pgfpathlineto{\pgfqpoint{8.708325in}{0.590243in}}%
\pgfpathlineto{\pgfqpoint{8.923125in}{0.590243in}}%
\pgfpathlineto{\pgfqpoint{9.137926in}{0.590243in}}%
\pgfpathlineto{\pgfqpoint{9.352726in}{0.590243in}}%
\pgfpathlineto{\pgfqpoint{9.567527in}{0.590243in}}%
\pgfpathlineto{\pgfqpoint{9.782327in}{0.590243in}}%
\pgfpathlineto{\pgfqpoint{9.997127in}{0.590243in}}%
\pgfpathlineto{\pgfqpoint{10.211928in}{0.590243in}}%
\pgfpathlineto{\pgfqpoint{10.426728in}{0.590243in}}%
\pgfpathlineto{\pgfqpoint{10.641529in}{0.590243in}}%
\pgfpathlineto{\pgfqpoint{10.856329in}{0.590243in}}%
\pgfpathlineto{\pgfqpoint{11.071130in}{0.590243in}}%
\pgfpathlineto{\pgfqpoint{11.285930in}{0.590243in}}%
\pgfpathlineto{\pgfqpoint{11.500731in}{0.590243in}}%
\pgfpathlineto{\pgfqpoint{11.822931in}{0.590243in}}%
\pgfpathlineto{\pgfqpoint{11.822931in}{0.911656in}}%
\pgfpathlineto{\pgfqpoint{11.822931in}{0.911656in}}%
\pgfpathlineto{\pgfqpoint{11.500731in}{0.911656in}}%
\pgfpathlineto{\pgfqpoint{11.285930in}{0.911656in}}%
\pgfpathlineto{\pgfqpoint{11.071130in}{0.911656in}}%
\pgfpathlineto{\pgfqpoint{10.856329in}{0.911656in}}%
\pgfpathlineto{\pgfqpoint{10.641529in}{0.911656in}}%
\pgfpathlineto{\pgfqpoint{10.426728in}{0.911656in}}%
\pgfpathlineto{\pgfqpoint{10.211928in}{0.911656in}}%
\pgfpathlineto{\pgfqpoint{9.997127in}{0.911656in}}%
\pgfpathlineto{\pgfqpoint{9.782327in}{0.911656in}}%
\pgfpathlineto{\pgfqpoint{9.567527in}{0.911656in}}%
\pgfpathlineto{\pgfqpoint{9.352726in}{0.911656in}}%
\pgfpathlineto{\pgfqpoint{9.137926in}{0.911656in}}%
\pgfpathlineto{\pgfqpoint{8.923125in}{0.911656in}}%
\pgfpathlineto{\pgfqpoint{8.708325in}{0.911656in}}%
\pgfpathlineto{\pgfqpoint{8.493524in}{0.911656in}}%
\pgfpathlineto{\pgfqpoint{8.278724in}{0.911656in}}%
\pgfpathlineto{\pgfqpoint{8.063923in}{0.911656in}}%
\pgfpathlineto{\pgfqpoint{7.849123in}{0.911656in}}%
\pgfpathlineto{\pgfqpoint{7.526922in}{0.911656in}}%
\pgfpathclose%
\pgfusepath{stroke,fill}%
\end{pgfscope}%
\begin{pgfscope}%
\pgfpathrectangle{\pgfqpoint{7.199352in}{0.331635in}}{\pgfqpoint{4.843750in}{3.775000in}}%
\pgfusepath{clip}%
\pgfsetbuttcap%
\pgfsetroundjoin%
\pgfsetlinewidth{1.505625pt}%
\definecolor{currentstroke}{rgb}{0.000000,0.000000,0.000000}%
\pgfsetstrokecolor{currentstroke}%
\pgfsetdash{}{0pt}%
\pgfpathmoveto{\pgfqpoint{7.419522in}{0.750949in}}%
\pgfpathlineto{\pgfqpoint{7.419522in}{3.935044in}}%
\pgfusepath{stroke}%
\end{pgfscope}%
\begin{pgfscope}%
\pgfpathrectangle{\pgfqpoint{7.199352in}{0.331635in}}{\pgfqpoint{4.843750in}{3.775000in}}%
\pgfusepath{clip}%
\pgfsetbuttcap%
\pgfsetroundjoin%
\pgfsetlinewidth{1.505625pt}%
\definecolor{currentstroke}{rgb}{0.000000,0.000000,0.000000}%
\pgfsetstrokecolor{currentstroke}%
\pgfsetdash{}{0pt}%
\pgfpathmoveto{\pgfqpoint{7.634323in}{0.750949in}}%
\pgfpathlineto{\pgfqpoint{7.634323in}{0.707246in}}%
\pgfusepath{stroke}%
\end{pgfscope}%
\begin{pgfscope}%
\pgfpathrectangle{\pgfqpoint{7.199352in}{0.331635in}}{\pgfqpoint{4.843750in}{3.775000in}}%
\pgfusepath{clip}%
\pgfsetbuttcap%
\pgfsetroundjoin%
\pgfsetlinewidth{1.505625pt}%
\definecolor{currentstroke}{rgb}{0.000000,0.000000,0.000000}%
\pgfsetstrokecolor{currentstroke}%
\pgfsetdash{}{0pt}%
\pgfpathmoveto{\pgfqpoint{7.849123in}{0.750949in}}%
\pgfpathlineto{\pgfqpoint{7.849123in}{0.756125in}}%
\pgfusepath{stroke}%
\end{pgfscope}%
\begin{pgfscope}%
\pgfpathrectangle{\pgfqpoint{7.199352in}{0.331635in}}{\pgfqpoint{4.843750in}{3.775000in}}%
\pgfusepath{clip}%
\pgfsetbuttcap%
\pgfsetroundjoin%
\pgfsetlinewidth{1.505625pt}%
\definecolor{currentstroke}{rgb}{0.000000,0.000000,0.000000}%
\pgfsetstrokecolor{currentstroke}%
\pgfsetdash{}{0pt}%
\pgfpathmoveto{\pgfqpoint{8.063923in}{0.750949in}}%
\pgfpathlineto{\pgfqpoint{8.063923in}{0.754290in}}%
\pgfusepath{stroke}%
\end{pgfscope}%
\begin{pgfscope}%
\pgfpathrectangle{\pgfqpoint{7.199352in}{0.331635in}}{\pgfqpoint{4.843750in}{3.775000in}}%
\pgfusepath{clip}%
\pgfsetbuttcap%
\pgfsetroundjoin%
\pgfsetlinewidth{1.505625pt}%
\definecolor{currentstroke}{rgb}{0.000000,0.000000,0.000000}%
\pgfsetstrokecolor{currentstroke}%
\pgfsetdash{}{0pt}%
\pgfpathmoveto{\pgfqpoint{8.278724in}{0.750949in}}%
\pgfpathlineto{\pgfqpoint{8.278724in}{0.714993in}}%
\pgfusepath{stroke}%
\end{pgfscope}%
\begin{pgfscope}%
\pgfpathrectangle{\pgfqpoint{7.199352in}{0.331635in}}{\pgfqpoint{4.843750in}{3.775000in}}%
\pgfusepath{clip}%
\pgfsetbuttcap%
\pgfsetroundjoin%
\pgfsetlinewidth{1.505625pt}%
\definecolor{currentstroke}{rgb}{0.000000,0.000000,0.000000}%
\pgfsetstrokecolor{currentstroke}%
\pgfsetdash{}{0pt}%
\pgfpathmoveto{\pgfqpoint{8.493524in}{0.750949in}}%
\pgfpathlineto{\pgfqpoint{8.493524in}{0.756347in}}%
\pgfusepath{stroke}%
\end{pgfscope}%
\begin{pgfscope}%
\pgfpathrectangle{\pgfqpoint{7.199352in}{0.331635in}}{\pgfqpoint{4.843750in}{3.775000in}}%
\pgfusepath{clip}%
\pgfsetbuttcap%
\pgfsetroundjoin%
\pgfsetlinewidth{1.505625pt}%
\definecolor{currentstroke}{rgb}{0.000000,0.000000,0.000000}%
\pgfsetstrokecolor{currentstroke}%
\pgfsetdash{}{0pt}%
\pgfpathmoveto{\pgfqpoint{8.708325in}{0.750949in}}%
\pgfpathlineto{\pgfqpoint{8.708325in}{0.722966in}}%
\pgfusepath{stroke}%
\end{pgfscope}%
\begin{pgfscope}%
\pgfpathrectangle{\pgfqpoint{7.199352in}{0.331635in}}{\pgfqpoint{4.843750in}{3.775000in}}%
\pgfusepath{clip}%
\pgfsetbuttcap%
\pgfsetroundjoin%
\pgfsetlinewidth{1.505625pt}%
\definecolor{currentstroke}{rgb}{0.000000,0.000000,0.000000}%
\pgfsetstrokecolor{currentstroke}%
\pgfsetdash{}{0pt}%
\pgfpathmoveto{\pgfqpoint{8.923125in}{0.750949in}}%
\pgfpathlineto{\pgfqpoint{8.923125in}{0.822605in}}%
\pgfusepath{stroke}%
\end{pgfscope}%
\begin{pgfscope}%
\pgfpathrectangle{\pgfqpoint{7.199352in}{0.331635in}}{\pgfqpoint{4.843750in}{3.775000in}}%
\pgfusepath{clip}%
\pgfsetbuttcap%
\pgfsetroundjoin%
\pgfsetlinewidth{1.505625pt}%
\definecolor{currentstroke}{rgb}{0.000000,0.000000,0.000000}%
\pgfsetstrokecolor{currentstroke}%
\pgfsetdash{}{0pt}%
\pgfpathmoveto{\pgfqpoint{9.137926in}{0.750949in}}%
\pgfpathlineto{\pgfqpoint{9.137926in}{0.789195in}}%
\pgfusepath{stroke}%
\end{pgfscope}%
\begin{pgfscope}%
\pgfpathrectangle{\pgfqpoint{7.199352in}{0.331635in}}{\pgfqpoint{4.843750in}{3.775000in}}%
\pgfusepath{clip}%
\pgfsetbuttcap%
\pgfsetroundjoin%
\pgfsetlinewidth{1.505625pt}%
\definecolor{currentstroke}{rgb}{0.000000,0.000000,0.000000}%
\pgfsetstrokecolor{currentstroke}%
\pgfsetdash{}{0pt}%
\pgfpathmoveto{\pgfqpoint{9.352726in}{0.750949in}}%
\pgfpathlineto{\pgfqpoint{9.352726in}{0.635144in}}%
\pgfusepath{stroke}%
\end{pgfscope}%
\begin{pgfscope}%
\pgfpathrectangle{\pgfqpoint{7.199352in}{0.331635in}}{\pgfqpoint{4.843750in}{3.775000in}}%
\pgfusepath{clip}%
\pgfsetbuttcap%
\pgfsetroundjoin%
\pgfsetlinewidth{1.505625pt}%
\definecolor{currentstroke}{rgb}{0.000000,0.000000,0.000000}%
\pgfsetstrokecolor{currentstroke}%
\pgfsetdash{}{0pt}%
\pgfpathmoveto{\pgfqpoint{9.567527in}{0.750949in}}%
\pgfpathlineto{\pgfqpoint{9.567527in}{0.873709in}}%
\pgfusepath{stroke}%
\end{pgfscope}%
\begin{pgfscope}%
\pgfpathrectangle{\pgfqpoint{7.199352in}{0.331635in}}{\pgfqpoint{4.843750in}{3.775000in}}%
\pgfusepath{clip}%
\pgfsetbuttcap%
\pgfsetroundjoin%
\pgfsetlinewidth{1.505625pt}%
\definecolor{currentstroke}{rgb}{0.000000,0.000000,0.000000}%
\pgfsetstrokecolor{currentstroke}%
\pgfsetdash{}{0pt}%
\pgfpathmoveto{\pgfqpoint{9.782327in}{0.750949in}}%
\pgfpathlineto{\pgfqpoint{9.782327in}{0.854722in}}%
\pgfusepath{stroke}%
\end{pgfscope}%
\begin{pgfscope}%
\pgfpathrectangle{\pgfqpoint{7.199352in}{0.331635in}}{\pgfqpoint{4.843750in}{3.775000in}}%
\pgfusepath{clip}%
\pgfsetbuttcap%
\pgfsetroundjoin%
\pgfsetlinewidth{1.505625pt}%
\definecolor{currentstroke}{rgb}{0.000000,0.000000,0.000000}%
\pgfsetstrokecolor{currentstroke}%
\pgfsetdash{}{0pt}%
\pgfpathmoveto{\pgfqpoint{9.997127in}{0.750949in}}%
\pgfpathlineto{\pgfqpoint{9.997127in}{0.788366in}}%
\pgfusepath{stroke}%
\end{pgfscope}%
\begin{pgfscope}%
\pgfpathrectangle{\pgfqpoint{7.199352in}{0.331635in}}{\pgfqpoint{4.843750in}{3.775000in}}%
\pgfusepath{clip}%
\pgfsetbuttcap%
\pgfsetroundjoin%
\pgfsetlinewidth{1.505625pt}%
\definecolor{currentstroke}{rgb}{0.000000,0.000000,0.000000}%
\pgfsetstrokecolor{currentstroke}%
\pgfsetdash{}{0pt}%
\pgfpathmoveto{\pgfqpoint{10.211928in}{0.750949in}}%
\pgfpathlineto{\pgfqpoint{10.211928in}{0.590081in}}%
\pgfusepath{stroke}%
\end{pgfscope}%
\begin{pgfscope}%
\pgfpathrectangle{\pgfqpoint{7.199352in}{0.331635in}}{\pgfqpoint{4.843750in}{3.775000in}}%
\pgfusepath{clip}%
\pgfsetbuttcap%
\pgfsetroundjoin%
\pgfsetlinewidth{1.505625pt}%
\definecolor{currentstroke}{rgb}{0.000000,0.000000,0.000000}%
\pgfsetstrokecolor{currentstroke}%
\pgfsetdash{}{0pt}%
\pgfpathmoveto{\pgfqpoint{10.426728in}{0.750949in}}%
\pgfpathlineto{\pgfqpoint{10.426728in}{0.765945in}}%
\pgfusepath{stroke}%
\end{pgfscope}%
\begin{pgfscope}%
\pgfpathrectangle{\pgfqpoint{7.199352in}{0.331635in}}{\pgfqpoint{4.843750in}{3.775000in}}%
\pgfusepath{clip}%
\pgfsetbuttcap%
\pgfsetroundjoin%
\pgfsetlinewidth{1.505625pt}%
\definecolor{currentstroke}{rgb}{0.000000,0.000000,0.000000}%
\pgfsetstrokecolor{currentstroke}%
\pgfsetdash{}{0pt}%
\pgfpathmoveto{\pgfqpoint{10.641529in}{0.750949in}}%
\pgfpathlineto{\pgfqpoint{10.641529in}{0.503226in}}%
\pgfusepath{stroke}%
\end{pgfscope}%
\begin{pgfscope}%
\pgfpathrectangle{\pgfqpoint{7.199352in}{0.331635in}}{\pgfqpoint{4.843750in}{3.775000in}}%
\pgfusepath{clip}%
\pgfsetbuttcap%
\pgfsetroundjoin%
\pgfsetlinewidth{1.505625pt}%
\definecolor{currentstroke}{rgb}{0.000000,0.000000,0.000000}%
\pgfsetstrokecolor{currentstroke}%
\pgfsetdash{}{0pt}%
\pgfpathmoveto{\pgfqpoint{10.856329in}{0.750949in}}%
\pgfpathlineto{\pgfqpoint{10.856329in}{0.602992in}}%
\pgfusepath{stroke}%
\end{pgfscope}%
\begin{pgfscope}%
\pgfpathrectangle{\pgfqpoint{7.199352in}{0.331635in}}{\pgfqpoint{4.843750in}{3.775000in}}%
\pgfusepath{clip}%
\pgfsetbuttcap%
\pgfsetroundjoin%
\pgfsetlinewidth{1.505625pt}%
\definecolor{currentstroke}{rgb}{0.000000,0.000000,0.000000}%
\pgfsetstrokecolor{currentstroke}%
\pgfsetdash{}{0pt}%
\pgfpathmoveto{\pgfqpoint{11.071130in}{0.750949in}}%
\pgfpathlineto{\pgfqpoint{11.071130in}{0.910279in}}%
\pgfusepath{stroke}%
\end{pgfscope}%
\begin{pgfscope}%
\pgfpathrectangle{\pgfqpoint{7.199352in}{0.331635in}}{\pgfqpoint{4.843750in}{3.775000in}}%
\pgfusepath{clip}%
\pgfsetbuttcap%
\pgfsetroundjoin%
\pgfsetlinewidth{1.505625pt}%
\definecolor{currentstroke}{rgb}{0.000000,0.000000,0.000000}%
\pgfsetstrokecolor{currentstroke}%
\pgfsetdash{}{0pt}%
\pgfpathmoveto{\pgfqpoint{11.285930in}{0.750949in}}%
\pgfpathlineto{\pgfqpoint{11.285930in}{0.984299in}}%
\pgfusepath{stroke}%
\end{pgfscope}%
\begin{pgfscope}%
\pgfpathrectangle{\pgfqpoint{7.199352in}{0.331635in}}{\pgfqpoint{4.843750in}{3.775000in}}%
\pgfusepath{clip}%
\pgfsetbuttcap%
\pgfsetroundjoin%
\pgfsetlinewidth{1.505625pt}%
\definecolor{currentstroke}{rgb}{0.000000,0.000000,0.000000}%
\pgfsetstrokecolor{currentstroke}%
\pgfsetdash{}{0pt}%
\pgfpathmoveto{\pgfqpoint{11.500731in}{0.750949in}}%
\pgfpathlineto{\pgfqpoint{11.500731in}{0.767355in}}%
\pgfusepath{stroke}%
\end{pgfscope}%
\begin{pgfscope}%
\pgfpathrectangle{\pgfqpoint{7.199352in}{0.331635in}}{\pgfqpoint{4.843750in}{3.775000in}}%
\pgfusepath{clip}%
\pgfsetbuttcap%
\pgfsetroundjoin%
\pgfsetlinewidth{1.505625pt}%
\definecolor{currentstroke}{rgb}{0.000000,0.000000,0.000000}%
\pgfsetstrokecolor{currentstroke}%
\pgfsetdash{}{0pt}%
\pgfpathmoveto{\pgfqpoint{11.715531in}{0.750949in}}%
\pgfpathlineto{\pgfqpoint{11.715531in}{0.768215in}}%
\pgfusepath{stroke}%
\end{pgfscope}%
\begin{pgfscope}%
\pgfpathrectangle{\pgfqpoint{7.199352in}{0.331635in}}{\pgfqpoint{4.843750in}{3.775000in}}%
\pgfusepath{clip}%
\pgfsetroundcap%
\pgfsetroundjoin%
\pgfsetlinewidth{1.505625pt}%
\definecolor{currentstroke}{rgb}{0.839216,0.152941,0.156863}%
\pgfsetstrokecolor{currentstroke}%
\pgfsetdash{}{0pt}%
\pgfpathmoveto{\pgfqpoint{7.199352in}{0.750949in}}%
\pgfpathlineto{\pgfqpoint{12.043102in}{0.750949in}}%
\pgfusepath{stroke}%
\end{pgfscope}%
\begin{pgfscope}%
\pgfpathrectangle{\pgfqpoint{7.199352in}{0.331635in}}{\pgfqpoint{4.843750in}{3.775000in}}%
\pgfusepath{clip}%
\pgfsetbuttcap%
\pgfsetroundjoin%
\definecolor{currentfill}{rgb}{0.839216,0.152941,0.156863}%
\pgfsetfillcolor{currentfill}%
\pgfsetlinewidth{1.003750pt}%
\definecolor{currentstroke}{rgb}{0.839216,0.152941,0.156863}%
\pgfsetstrokecolor{currentstroke}%
\pgfsetdash{}{0pt}%
\pgfsys@defobject{currentmarker}{\pgfqpoint{-0.034722in}{-0.034722in}}{\pgfqpoint{0.034722in}{0.034722in}}{%
\pgfpathmoveto{\pgfqpoint{0.000000in}{-0.034722in}}%
\pgfpathcurveto{\pgfqpoint{0.009208in}{-0.034722in}}{\pgfqpoint{0.018041in}{-0.031064in}}{\pgfqpoint{0.024552in}{-0.024552in}}%
\pgfpathcurveto{\pgfqpoint{0.031064in}{-0.018041in}}{\pgfqpoint{0.034722in}{-0.009208in}}{\pgfqpoint{0.034722in}{0.000000in}}%
\pgfpathcurveto{\pgfqpoint{0.034722in}{0.009208in}}{\pgfqpoint{0.031064in}{0.018041in}}{\pgfqpoint{0.024552in}{0.024552in}}%
\pgfpathcurveto{\pgfqpoint{0.018041in}{0.031064in}}{\pgfqpoint{0.009208in}{0.034722in}}{\pgfqpoint{0.000000in}{0.034722in}}%
\pgfpathcurveto{\pgfqpoint{-0.009208in}{0.034722in}}{\pgfqpoint{-0.018041in}{0.031064in}}{\pgfqpoint{-0.024552in}{0.024552in}}%
\pgfpathcurveto{\pgfqpoint{-0.031064in}{0.018041in}}{\pgfqpoint{-0.034722in}{0.009208in}}{\pgfqpoint{-0.034722in}{0.000000in}}%
\pgfpathcurveto{\pgfqpoint{-0.034722in}{-0.009208in}}{\pgfqpoint{-0.031064in}{-0.018041in}}{\pgfqpoint{-0.024552in}{-0.024552in}}%
\pgfpathcurveto{\pgfqpoint{-0.018041in}{-0.031064in}}{\pgfqpoint{-0.009208in}{-0.034722in}}{\pgfqpoint{0.000000in}{-0.034722in}}%
\pgfpathclose%
\pgfusepath{stroke,fill}%
}%
\begin{pgfscope}%
\pgfsys@transformshift{7.419522in}{3.935044in}%
\pgfsys@useobject{currentmarker}{}%
\end{pgfscope}%
\begin{pgfscope}%
\pgfsys@transformshift{7.634323in}{0.707246in}%
\pgfsys@useobject{currentmarker}{}%
\end{pgfscope}%
\begin{pgfscope}%
\pgfsys@transformshift{7.849123in}{0.756125in}%
\pgfsys@useobject{currentmarker}{}%
\end{pgfscope}%
\begin{pgfscope}%
\pgfsys@transformshift{8.063923in}{0.754290in}%
\pgfsys@useobject{currentmarker}{}%
\end{pgfscope}%
\begin{pgfscope}%
\pgfsys@transformshift{8.278724in}{0.714993in}%
\pgfsys@useobject{currentmarker}{}%
\end{pgfscope}%
\begin{pgfscope}%
\pgfsys@transformshift{8.493524in}{0.756347in}%
\pgfsys@useobject{currentmarker}{}%
\end{pgfscope}%
\begin{pgfscope}%
\pgfsys@transformshift{8.708325in}{0.722966in}%
\pgfsys@useobject{currentmarker}{}%
\end{pgfscope}%
\begin{pgfscope}%
\pgfsys@transformshift{8.923125in}{0.822605in}%
\pgfsys@useobject{currentmarker}{}%
\end{pgfscope}%
\begin{pgfscope}%
\pgfsys@transformshift{9.137926in}{0.789195in}%
\pgfsys@useobject{currentmarker}{}%
\end{pgfscope}%
\begin{pgfscope}%
\pgfsys@transformshift{9.352726in}{0.635144in}%
\pgfsys@useobject{currentmarker}{}%
\end{pgfscope}%
\begin{pgfscope}%
\pgfsys@transformshift{9.567527in}{0.873709in}%
\pgfsys@useobject{currentmarker}{}%
\end{pgfscope}%
\begin{pgfscope}%
\pgfsys@transformshift{9.782327in}{0.854722in}%
\pgfsys@useobject{currentmarker}{}%
\end{pgfscope}%
\begin{pgfscope}%
\pgfsys@transformshift{9.997127in}{0.788366in}%
\pgfsys@useobject{currentmarker}{}%
\end{pgfscope}%
\begin{pgfscope}%
\pgfsys@transformshift{10.211928in}{0.590081in}%
\pgfsys@useobject{currentmarker}{}%
\end{pgfscope}%
\begin{pgfscope}%
\pgfsys@transformshift{10.426728in}{0.765945in}%
\pgfsys@useobject{currentmarker}{}%
\end{pgfscope}%
\begin{pgfscope}%
\pgfsys@transformshift{10.641529in}{0.503226in}%
\pgfsys@useobject{currentmarker}{}%
\end{pgfscope}%
\begin{pgfscope}%
\pgfsys@transformshift{10.856329in}{0.602992in}%
\pgfsys@useobject{currentmarker}{}%
\end{pgfscope}%
\begin{pgfscope}%
\pgfsys@transformshift{11.071130in}{0.910279in}%
\pgfsys@useobject{currentmarker}{}%
\end{pgfscope}%
\begin{pgfscope}%
\pgfsys@transformshift{11.285930in}{0.984299in}%
\pgfsys@useobject{currentmarker}{}%
\end{pgfscope}%
\begin{pgfscope}%
\pgfsys@transformshift{11.500731in}{0.767355in}%
\pgfsys@useobject{currentmarker}{}%
\end{pgfscope}%
\begin{pgfscope}%
\pgfsys@transformshift{11.715531in}{0.768215in}%
\pgfsys@useobject{currentmarker}{}%
\end{pgfscope}%
\end{pgfscope}%
\begin{pgfscope}%
\pgfsetrectcap%
\pgfsetmiterjoin%
\pgfsetlinewidth{0.803000pt}%
\definecolor{currentstroke}{rgb}{1.000000,1.000000,1.000000}%
\pgfsetstrokecolor{currentstroke}%
\pgfsetdash{}{0pt}%
\pgfpathmoveto{\pgfqpoint{7.199352in}{0.331635in}}%
\pgfpathlineto{\pgfqpoint{7.199352in}{4.106635in}}%
\pgfusepath{stroke}%
\end{pgfscope}%
\begin{pgfscope}%
\pgfsetrectcap%
\pgfsetmiterjoin%
\pgfsetlinewidth{0.803000pt}%
\definecolor{currentstroke}{rgb}{1.000000,1.000000,1.000000}%
\pgfsetstrokecolor{currentstroke}%
\pgfsetdash{}{0pt}%
\pgfpathmoveto{\pgfqpoint{12.043102in}{0.331635in}}%
\pgfpathlineto{\pgfqpoint{12.043102in}{4.106635in}}%
\pgfusepath{stroke}%
\end{pgfscope}%
\begin{pgfscope}%
\pgfsetrectcap%
\pgfsetmiterjoin%
\pgfsetlinewidth{0.803000pt}%
\definecolor{currentstroke}{rgb}{1.000000,1.000000,1.000000}%
\pgfsetstrokecolor{currentstroke}%
\pgfsetdash{}{0pt}%
\pgfpathmoveto{\pgfqpoint{7.199352in}{0.331635in}}%
\pgfpathlineto{\pgfqpoint{12.043102in}{0.331635in}}%
\pgfusepath{stroke}%
\end{pgfscope}%
\begin{pgfscope}%
\pgfsetrectcap%
\pgfsetmiterjoin%
\pgfsetlinewidth{0.803000pt}%
\definecolor{currentstroke}{rgb}{1.000000,1.000000,1.000000}%
\pgfsetstrokecolor{currentstroke}%
\pgfsetdash{}{0pt}%
\pgfpathmoveto{\pgfqpoint{7.199352in}{4.106635in}}%
\pgfpathlineto{\pgfqpoint{12.043102in}{4.106635in}}%
\pgfusepath{stroke}%
\end{pgfscope}%
\begin{pgfscope}%
\definecolor{textcolor}{rgb}{0.150000,0.150000,0.150000}%
\pgfsetstrokecolor{textcolor}%
\pgfsetfillcolor{textcolor}%
\pgftext[x=9.621227in,y=4.189968in,,base]{\color{textcolor}\rmfamily\fontsize{12.000000}{14.400000}\selectfont Partial Autocorrelation}%
\end{pgfscope}%
\end{pgfpicture}%
\makeatother%
\endgroup%

    \end{adjustbox}
    
    \begin{adjustbox}{width=.95\textwidth,center}
    %% Creator: Matplotlib, PGF backend
%%
%% To include the figure in your LaTeX document, write
%%   \input{<filename>.pgf}
%%
%% Make sure the required packages are loaded in your preamble
%%   \usepackage{pgf}
%%
%% Figures using additional raster images can only be included by \input if
%% they are in the same directory as the main LaTeX file. For loading figures
%% from other directories you can use the `import` package
%%   \usepackage{import}
%% and then include the figures with
%%   \import{<path to file>}{<filename>.pgf}
%%
%% Matplotlib used the following preamble
%%   \usepackage{fontspec}
%%   \setmainfont{DejaVuSerif.ttf}[Path=/opt/tljh/user/lib/python3.6/site-packages/matplotlib/mpl-data/fonts/ttf/]
%%   \setsansfont{DejaVuSans.ttf}[Path=/opt/tljh/user/lib/python3.6/site-packages/matplotlib/mpl-data/fonts/ttf/]
%%   \setmonofont{DejaVuSansMono.ttf}[Path=/opt/tljh/user/lib/python3.6/site-packages/matplotlib/mpl-data/fonts/ttf/]
%%
\begingroup%
\makeatletter%
\begin{pgfpicture}%
\pgfpathrectangle{\pgfpointorigin}{\pgfqpoint{6.806467in}{2.151596in}}%
\pgfusepath{use as bounding box, clip}%
\begin{pgfscope}%
\pgfsetbuttcap%
\pgfsetmiterjoin%
\definecolor{currentfill}{rgb}{1.000000,1.000000,1.000000}%
\pgfsetfillcolor{currentfill}%
\pgfsetlinewidth{0.000000pt}%
\definecolor{currentstroke}{rgb}{1.000000,1.000000,1.000000}%
\pgfsetstrokecolor{currentstroke}%
\pgfsetdash{}{0pt}%
\pgfpathmoveto{\pgfqpoint{0.000000in}{0.000000in}}%
\pgfpathlineto{\pgfqpoint{6.806467in}{0.000000in}}%
\pgfpathlineto{\pgfqpoint{6.806467in}{2.151596in}}%
\pgfpathlineto{\pgfqpoint{0.000000in}{2.151596in}}%
\pgfpathclose%
\pgfusepath{fill}%
\end{pgfscope}%
\begin{pgfscope}%
\pgfsetbuttcap%
\pgfsetmiterjoin%
\definecolor{currentfill}{rgb}{0.917647,0.917647,0.949020}%
\pgfsetfillcolor{currentfill}%
\pgfsetlinewidth{0.000000pt}%
\definecolor{currentstroke}{rgb}{0.000000,0.000000,0.000000}%
\pgfsetstrokecolor{currentstroke}%
\pgfsetstrokeopacity{0.000000}%
\pgfsetdash{}{0pt}%
\pgfpathmoveto{\pgfqpoint{0.506467in}{0.331635in}}%
\pgfpathlineto{\pgfqpoint{3.089800in}{0.331635in}}%
\pgfpathlineto{\pgfqpoint{3.089800in}{1.841635in}}%
\pgfpathlineto{\pgfqpoint{0.506467in}{1.841635in}}%
\pgfpathclose%
\pgfusepath{fill}%
\end{pgfscope}%
\begin{pgfscope}%
\pgfpathrectangle{\pgfqpoint{0.506467in}{0.331635in}}{\pgfqpoint{2.583333in}{1.510000in}}%
\pgfusepath{clip}%
\pgfsetroundcap%
\pgfsetroundjoin%
\pgfsetlinewidth{0.803000pt}%
\definecolor{currentstroke}{rgb}{1.000000,1.000000,1.000000}%
\pgfsetstrokecolor{currentstroke}%
\pgfsetdash{}{0pt}%
\pgfpathmoveto{\pgfqpoint{0.623891in}{0.331635in}}%
\pgfpathlineto{\pgfqpoint{0.623891in}{1.841635in}}%
\pgfusepath{stroke}%
\end{pgfscope}%
\begin{pgfscope}%
\definecolor{textcolor}{rgb}{0.150000,0.150000,0.150000}%
\pgfsetstrokecolor{textcolor}%
\pgfsetfillcolor{textcolor}%
\pgftext[x=0.623891in,y=0.234413in,,top]{\color{textcolor}\rmfamily\fontsize{10.000000}{12.000000}\selectfont 0}%
\end{pgfscope}%
\begin{pgfscope}%
\pgfpathrectangle{\pgfqpoint{0.506467in}{0.331635in}}{\pgfqpoint{2.583333in}{1.510000in}}%
\pgfusepath{clip}%
\pgfsetroundcap%
\pgfsetroundjoin%
\pgfsetlinewidth{0.803000pt}%
\definecolor{currentstroke}{rgb}{1.000000,1.000000,1.000000}%
\pgfsetstrokecolor{currentstroke}%
\pgfsetdash{}{0pt}%
\pgfpathmoveto{\pgfqpoint{1.196693in}{0.331635in}}%
\pgfpathlineto{\pgfqpoint{1.196693in}{1.841635in}}%
\pgfusepath{stroke}%
\end{pgfscope}%
\begin{pgfscope}%
\definecolor{textcolor}{rgb}{0.150000,0.150000,0.150000}%
\pgfsetstrokecolor{textcolor}%
\pgfsetfillcolor{textcolor}%
\pgftext[x=1.196693in,y=0.234413in,,top]{\color{textcolor}\rmfamily\fontsize{10.000000}{12.000000}\selectfont 5}%
\end{pgfscope}%
\begin{pgfscope}%
\pgfpathrectangle{\pgfqpoint{0.506467in}{0.331635in}}{\pgfqpoint{2.583333in}{1.510000in}}%
\pgfusepath{clip}%
\pgfsetroundcap%
\pgfsetroundjoin%
\pgfsetlinewidth{0.803000pt}%
\definecolor{currentstroke}{rgb}{1.000000,1.000000,1.000000}%
\pgfsetstrokecolor{currentstroke}%
\pgfsetdash{}{0pt}%
\pgfpathmoveto{\pgfqpoint{1.769494in}{0.331635in}}%
\pgfpathlineto{\pgfqpoint{1.769494in}{1.841635in}}%
\pgfusepath{stroke}%
\end{pgfscope}%
\begin{pgfscope}%
\definecolor{textcolor}{rgb}{0.150000,0.150000,0.150000}%
\pgfsetstrokecolor{textcolor}%
\pgfsetfillcolor{textcolor}%
\pgftext[x=1.769494in,y=0.234413in,,top]{\color{textcolor}\rmfamily\fontsize{10.000000}{12.000000}\selectfont 10}%
\end{pgfscope}%
\begin{pgfscope}%
\pgfpathrectangle{\pgfqpoint{0.506467in}{0.331635in}}{\pgfqpoint{2.583333in}{1.510000in}}%
\pgfusepath{clip}%
\pgfsetroundcap%
\pgfsetroundjoin%
\pgfsetlinewidth{0.803000pt}%
\definecolor{currentstroke}{rgb}{1.000000,1.000000,1.000000}%
\pgfsetstrokecolor{currentstroke}%
\pgfsetdash{}{0pt}%
\pgfpathmoveto{\pgfqpoint{2.342295in}{0.331635in}}%
\pgfpathlineto{\pgfqpoint{2.342295in}{1.841635in}}%
\pgfusepath{stroke}%
\end{pgfscope}%
\begin{pgfscope}%
\definecolor{textcolor}{rgb}{0.150000,0.150000,0.150000}%
\pgfsetstrokecolor{textcolor}%
\pgfsetfillcolor{textcolor}%
\pgftext[x=2.342295in,y=0.234413in,,top]{\color{textcolor}\rmfamily\fontsize{10.000000}{12.000000}\selectfont 15}%
\end{pgfscope}%
\begin{pgfscope}%
\pgfpathrectangle{\pgfqpoint{0.506467in}{0.331635in}}{\pgfqpoint{2.583333in}{1.510000in}}%
\pgfusepath{clip}%
\pgfsetroundcap%
\pgfsetroundjoin%
\pgfsetlinewidth{0.803000pt}%
\definecolor{currentstroke}{rgb}{1.000000,1.000000,1.000000}%
\pgfsetstrokecolor{currentstroke}%
\pgfsetdash{}{0pt}%
\pgfpathmoveto{\pgfqpoint{2.915096in}{0.331635in}}%
\pgfpathlineto{\pgfqpoint{2.915096in}{1.841635in}}%
\pgfusepath{stroke}%
\end{pgfscope}%
\begin{pgfscope}%
\definecolor{textcolor}{rgb}{0.150000,0.150000,0.150000}%
\pgfsetstrokecolor{textcolor}%
\pgfsetfillcolor{textcolor}%
\pgftext[x=2.915096in,y=0.234413in,,top]{\color{textcolor}\rmfamily\fontsize{10.000000}{12.000000}\selectfont 20}%
\end{pgfscope}%
\begin{pgfscope}%
\pgfpathrectangle{\pgfqpoint{0.506467in}{0.331635in}}{\pgfqpoint{2.583333in}{1.510000in}}%
\pgfusepath{clip}%
\pgfsetroundcap%
\pgfsetroundjoin%
\pgfsetlinewidth{0.803000pt}%
\definecolor{currentstroke}{rgb}{1.000000,1.000000,1.000000}%
\pgfsetstrokecolor{currentstroke}%
\pgfsetdash{}{0pt}%
\pgfpathmoveto{\pgfqpoint{0.506467in}{0.504733in}}%
\pgfpathlineto{\pgfqpoint{3.089800in}{0.504733in}}%
\pgfusepath{stroke}%
\end{pgfscope}%
\begin{pgfscope}%
\definecolor{textcolor}{rgb}{0.150000,0.150000,0.150000}%
\pgfsetstrokecolor{textcolor}%
\pgfsetfillcolor{textcolor}%
\pgftext[x=0.100000in,y=0.451972in,left,base]{\color{textcolor}\rmfamily\fontsize{10.000000}{12.000000}\selectfont 0.00}%
\end{pgfscope}%
\begin{pgfscope}%
\pgfpathrectangle{\pgfqpoint{0.506467in}{0.331635in}}{\pgfqpoint{2.583333in}{1.510000in}}%
\pgfusepath{clip}%
\pgfsetroundcap%
\pgfsetroundjoin%
\pgfsetlinewidth{0.803000pt}%
\definecolor{currentstroke}{rgb}{1.000000,1.000000,1.000000}%
\pgfsetstrokecolor{currentstroke}%
\pgfsetdash{}{0pt}%
\pgfpathmoveto{\pgfqpoint{0.506467in}{0.821800in}}%
\pgfpathlineto{\pgfqpoint{3.089800in}{0.821800in}}%
\pgfusepath{stroke}%
\end{pgfscope}%
\begin{pgfscope}%
\definecolor{textcolor}{rgb}{0.150000,0.150000,0.150000}%
\pgfsetstrokecolor{textcolor}%
\pgfsetfillcolor{textcolor}%
\pgftext[x=0.100000in,y=0.769038in,left,base]{\color{textcolor}\rmfamily\fontsize{10.000000}{12.000000}\selectfont 0.25}%
\end{pgfscope}%
\begin{pgfscope}%
\pgfpathrectangle{\pgfqpoint{0.506467in}{0.331635in}}{\pgfqpoint{2.583333in}{1.510000in}}%
\pgfusepath{clip}%
\pgfsetroundcap%
\pgfsetroundjoin%
\pgfsetlinewidth{0.803000pt}%
\definecolor{currentstroke}{rgb}{1.000000,1.000000,1.000000}%
\pgfsetstrokecolor{currentstroke}%
\pgfsetdash{}{0pt}%
\pgfpathmoveto{\pgfqpoint{0.506467in}{1.138866in}}%
\pgfpathlineto{\pgfqpoint{3.089800in}{1.138866in}}%
\pgfusepath{stroke}%
\end{pgfscope}%
\begin{pgfscope}%
\definecolor{textcolor}{rgb}{0.150000,0.150000,0.150000}%
\pgfsetstrokecolor{textcolor}%
\pgfsetfillcolor{textcolor}%
\pgftext[x=0.100000in,y=1.086105in,left,base]{\color{textcolor}\rmfamily\fontsize{10.000000}{12.000000}\selectfont 0.50}%
\end{pgfscope}%
\begin{pgfscope}%
\pgfpathrectangle{\pgfqpoint{0.506467in}{0.331635in}}{\pgfqpoint{2.583333in}{1.510000in}}%
\pgfusepath{clip}%
\pgfsetroundcap%
\pgfsetroundjoin%
\pgfsetlinewidth{0.803000pt}%
\definecolor{currentstroke}{rgb}{1.000000,1.000000,1.000000}%
\pgfsetstrokecolor{currentstroke}%
\pgfsetdash{}{0pt}%
\pgfpathmoveto{\pgfqpoint{0.506467in}{1.455932in}}%
\pgfpathlineto{\pgfqpoint{3.089800in}{1.455932in}}%
\pgfusepath{stroke}%
\end{pgfscope}%
\begin{pgfscope}%
\definecolor{textcolor}{rgb}{0.150000,0.150000,0.150000}%
\pgfsetstrokecolor{textcolor}%
\pgfsetfillcolor{textcolor}%
\pgftext[x=0.100000in,y=1.403171in,left,base]{\color{textcolor}\rmfamily\fontsize{10.000000}{12.000000}\selectfont 0.75}%
\end{pgfscope}%
\begin{pgfscope}%
\pgfpathrectangle{\pgfqpoint{0.506467in}{0.331635in}}{\pgfqpoint{2.583333in}{1.510000in}}%
\pgfusepath{clip}%
\pgfsetroundcap%
\pgfsetroundjoin%
\pgfsetlinewidth{0.803000pt}%
\definecolor{currentstroke}{rgb}{1.000000,1.000000,1.000000}%
\pgfsetstrokecolor{currentstroke}%
\pgfsetdash{}{0pt}%
\pgfpathmoveto{\pgfqpoint{0.506467in}{1.772999in}}%
\pgfpathlineto{\pgfqpoint{3.089800in}{1.772999in}}%
\pgfusepath{stroke}%
\end{pgfscope}%
\begin{pgfscope}%
\definecolor{textcolor}{rgb}{0.150000,0.150000,0.150000}%
\pgfsetstrokecolor{textcolor}%
\pgfsetfillcolor{textcolor}%
\pgftext[x=0.100000in,y=1.720237in,left,base]{\color{textcolor}\rmfamily\fontsize{10.000000}{12.000000}\selectfont 1.00}%
\end{pgfscope}%
\begin{pgfscope}%
\pgfpathrectangle{\pgfqpoint{0.506467in}{0.331635in}}{\pgfqpoint{2.583333in}{1.510000in}}%
\pgfusepath{clip}%
\pgfsetbuttcap%
\pgfsetroundjoin%
\definecolor{currentfill}{rgb}{0.121569,0.466667,0.705882}%
\pgfsetfillcolor{currentfill}%
\pgfsetfillopacity{0.250000}%
\pgfsetlinewidth{1.003750pt}%
\definecolor{currentstroke}{rgb}{1.000000,1.000000,1.000000}%
\pgfsetstrokecolor{currentstroke}%
\pgfsetstrokeopacity{0.250000}%
\pgfsetdash{}{0pt}%
\pgfpathmoveto{\pgfqpoint{0.681171in}{0.568745in}}%
\pgfpathlineto{\pgfqpoint{0.681171in}{0.440722in}}%
\pgfpathlineto{\pgfqpoint{0.853012in}{0.440375in}}%
\pgfpathlineto{\pgfqpoint{0.967572in}{0.440283in}}%
\pgfpathlineto{\pgfqpoint{1.082132in}{0.440228in}}%
\pgfpathlineto{\pgfqpoint{1.196693in}{0.440062in}}%
\pgfpathlineto{\pgfqpoint{1.311253in}{0.439989in}}%
\pgfpathlineto{\pgfqpoint{1.425813in}{0.439980in}}%
\pgfpathlineto{\pgfqpoint{1.540373in}{0.439979in}}%
\pgfpathlineto{\pgfqpoint{1.654933in}{0.439704in}}%
\pgfpathlineto{\pgfqpoint{1.769494in}{0.439686in}}%
\pgfpathlineto{\pgfqpoint{1.884054in}{0.439627in}}%
\pgfpathlineto{\pgfqpoint{1.998614in}{0.439622in}}%
\pgfpathlineto{\pgfqpoint{2.113174in}{0.439618in}}%
\pgfpathlineto{\pgfqpoint{2.227735in}{0.439615in}}%
\pgfpathlineto{\pgfqpoint{2.342295in}{0.439610in}}%
\pgfpathlineto{\pgfqpoint{2.456855in}{0.439185in}}%
\pgfpathlineto{\pgfqpoint{2.571415in}{0.439185in}}%
\pgfpathlineto{\pgfqpoint{2.685976in}{0.439175in}}%
\pgfpathlineto{\pgfqpoint{2.800536in}{0.439165in}}%
\pgfpathlineto{\pgfqpoint{2.972376in}{0.439156in}}%
\pgfpathlineto{\pgfqpoint{2.972376in}{0.570311in}}%
\pgfpathlineto{\pgfqpoint{2.972376in}{0.570311in}}%
\pgfpathlineto{\pgfqpoint{2.800536in}{0.570302in}}%
\pgfpathlineto{\pgfqpoint{2.685976in}{0.570292in}}%
\pgfpathlineto{\pgfqpoint{2.571415in}{0.570282in}}%
\pgfpathlineto{\pgfqpoint{2.456855in}{0.570282in}}%
\pgfpathlineto{\pgfqpoint{2.342295in}{0.569857in}}%
\pgfpathlineto{\pgfqpoint{2.227735in}{0.569852in}}%
\pgfpathlineto{\pgfqpoint{2.113174in}{0.569849in}}%
\pgfpathlineto{\pgfqpoint{1.998614in}{0.569845in}}%
\pgfpathlineto{\pgfqpoint{1.884054in}{0.569840in}}%
\pgfpathlineto{\pgfqpoint{1.769494in}{0.569781in}}%
\pgfpathlineto{\pgfqpoint{1.654933in}{0.569763in}}%
\pgfpathlineto{\pgfqpoint{1.540373in}{0.569488in}}%
\pgfpathlineto{\pgfqpoint{1.425813in}{0.569487in}}%
\pgfpathlineto{\pgfqpoint{1.311253in}{0.569478in}}%
\pgfpathlineto{\pgfqpoint{1.196693in}{0.569405in}}%
\pgfpathlineto{\pgfqpoint{1.082132in}{0.569239in}}%
\pgfpathlineto{\pgfqpoint{0.967572in}{0.569184in}}%
\pgfpathlineto{\pgfqpoint{0.853012in}{0.569092in}}%
\pgfpathlineto{\pgfqpoint{0.681171in}{0.568745in}}%
\pgfpathclose%
\pgfusepath{stroke,fill}%
\end{pgfscope}%
\begin{pgfscope}%
\pgfpathrectangle{\pgfqpoint{0.506467in}{0.331635in}}{\pgfqpoint{2.583333in}{1.510000in}}%
\pgfusepath{clip}%
\pgfsetbuttcap%
\pgfsetroundjoin%
\pgfsetlinewidth{1.505625pt}%
\definecolor{currentstroke}{rgb}{0.000000,0.000000,0.000000}%
\pgfsetstrokecolor{currentstroke}%
\pgfsetdash{}{0pt}%
\pgfpathmoveto{\pgfqpoint{0.623891in}{0.504733in}}%
\pgfpathlineto{\pgfqpoint{0.623891in}{1.772999in}}%
\pgfusepath{stroke}%
\end{pgfscope}%
\begin{pgfscope}%
\pgfpathrectangle{\pgfqpoint{0.506467in}{0.331635in}}{\pgfqpoint{2.583333in}{1.510000in}}%
\pgfusepath{clip}%
\pgfsetbuttcap%
\pgfsetroundjoin%
\pgfsetlinewidth{1.505625pt}%
\definecolor{currentstroke}{rgb}{0.000000,0.000000,0.000000}%
\pgfsetstrokecolor{currentstroke}%
\pgfsetdash{}{0pt}%
\pgfpathmoveto{\pgfqpoint{0.738452in}{0.504733in}}%
\pgfpathlineto{\pgfqpoint{0.738452in}{0.411212in}}%
\pgfusepath{stroke}%
\end{pgfscope}%
\begin{pgfscope}%
\pgfpathrectangle{\pgfqpoint{0.506467in}{0.331635in}}{\pgfqpoint{2.583333in}{1.510000in}}%
\pgfusepath{clip}%
\pgfsetbuttcap%
\pgfsetroundjoin%
\pgfsetlinewidth{1.505625pt}%
\definecolor{currentstroke}{rgb}{0.000000,0.000000,0.000000}%
\pgfsetstrokecolor{currentstroke}%
\pgfsetdash{}{0pt}%
\pgfpathmoveto{\pgfqpoint{0.853012in}{0.504733in}}%
\pgfpathlineto{\pgfqpoint{0.853012in}{0.456503in}}%
\pgfusepath{stroke}%
\end{pgfscope}%
\begin{pgfscope}%
\pgfpathrectangle{\pgfqpoint{0.506467in}{0.331635in}}{\pgfqpoint{2.583333in}{1.510000in}}%
\pgfusepath{clip}%
\pgfsetbuttcap%
\pgfsetroundjoin%
\pgfsetlinewidth{1.505625pt}%
\definecolor{currentstroke}{rgb}{0.000000,0.000000,0.000000}%
\pgfsetstrokecolor{currentstroke}%
\pgfsetdash{}{0pt}%
\pgfpathmoveto{\pgfqpoint{0.967572in}{0.504733in}}%
\pgfpathlineto{\pgfqpoint{0.967572in}{0.467381in}}%
\pgfusepath{stroke}%
\end{pgfscope}%
\begin{pgfscope}%
\pgfpathrectangle{\pgfqpoint{0.506467in}{0.331635in}}{\pgfqpoint{2.583333in}{1.510000in}}%
\pgfusepath{clip}%
\pgfsetbuttcap%
\pgfsetroundjoin%
\pgfsetlinewidth{1.505625pt}%
\definecolor{currentstroke}{rgb}{0.000000,0.000000,0.000000}%
\pgfsetstrokecolor{currentstroke}%
\pgfsetdash{}{0pt}%
\pgfpathmoveto{\pgfqpoint{1.082132in}{0.504733in}}%
\pgfpathlineto{\pgfqpoint{1.082132in}{0.439844in}}%
\pgfusepath{stroke}%
\end{pgfscope}%
\begin{pgfscope}%
\pgfpathrectangle{\pgfqpoint{0.506467in}{0.331635in}}{\pgfqpoint{2.583333in}{1.510000in}}%
\pgfusepath{clip}%
\pgfsetbuttcap%
\pgfsetroundjoin%
\pgfsetlinewidth{1.505625pt}%
\definecolor{currentstroke}{rgb}{0.000000,0.000000,0.000000}%
\pgfsetstrokecolor{currentstroke}%
\pgfsetdash{}{0pt}%
\pgfpathmoveto{\pgfqpoint{1.196693in}{0.504733in}}%
\pgfpathlineto{\pgfqpoint{1.196693in}{0.461770in}}%
\pgfusepath{stroke}%
\end{pgfscope}%
\begin{pgfscope}%
\pgfpathrectangle{\pgfqpoint{0.506467in}{0.331635in}}{\pgfqpoint{2.583333in}{1.510000in}}%
\pgfusepath{clip}%
\pgfsetbuttcap%
\pgfsetroundjoin%
\pgfsetlinewidth{1.505625pt}%
\definecolor{currentstroke}{rgb}{0.000000,0.000000,0.000000}%
\pgfsetstrokecolor{currentstroke}%
\pgfsetdash{}{0pt}%
\pgfpathmoveto{\pgfqpoint{1.311253in}{0.504733in}}%
\pgfpathlineto{\pgfqpoint{1.311253in}{0.519947in}}%
\pgfusepath{stroke}%
\end{pgfscope}%
\begin{pgfscope}%
\pgfpathrectangle{\pgfqpoint{0.506467in}{0.331635in}}{\pgfqpoint{2.583333in}{1.510000in}}%
\pgfusepath{clip}%
\pgfsetbuttcap%
\pgfsetroundjoin%
\pgfsetlinewidth{1.505625pt}%
\definecolor{currentstroke}{rgb}{0.000000,0.000000,0.000000}%
\pgfsetstrokecolor{currentstroke}%
\pgfsetdash{}{0pt}%
\pgfpathmoveto{\pgfqpoint{1.425813in}{0.504733in}}%
\pgfpathlineto{\pgfqpoint{1.425813in}{0.499056in}}%
\pgfusepath{stroke}%
\end{pgfscope}%
\begin{pgfscope}%
\pgfpathrectangle{\pgfqpoint{0.506467in}{0.331635in}}{\pgfqpoint{2.583333in}{1.510000in}}%
\pgfusepath{clip}%
\pgfsetbuttcap%
\pgfsetroundjoin%
\pgfsetlinewidth{1.505625pt}%
\definecolor{currentstroke}{rgb}{0.000000,0.000000,0.000000}%
\pgfsetstrokecolor{currentstroke}%
\pgfsetdash{}{0pt}%
\pgfpathmoveto{\pgfqpoint{1.540373in}{0.504733in}}%
\pgfpathlineto{\pgfqpoint{1.540373in}{0.588348in}}%
\pgfusepath{stroke}%
\end{pgfscope}%
\begin{pgfscope}%
\pgfpathrectangle{\pgfqpoint{0.506467in}{0.331635in}}{\pgfqpoint{2.583333in}{1.510000in}}%
\pgfusepath{clip}%
\pgfsetbuttcap%
\pgfsetroundjoin%
\pgfsetlinewidth{1.505625pt}%
\definecolor{currentstroke}{rgb}{0.000000,0.000000,0.000000}%
\pgfsetstrokecolor{currentstroke}%
\pgfsetdash{}{0pt}%
\pgfpathmoveto{\pgfqpoint{1.654933in}{0.504733in}}%
\pgfpathlineto{\pgfqpoint{1.654933in}{0.483209in}}%
\pgfusepath{stroke}%
\end{pgfscope}%
\begin{pgfscope}%
\pgfpathrectangle{\pgfqpoint{0.506467in}{0.331635in}}{\pgfqpoint{2.583333in}{1.510000in}}%
\pgfusepath{clip}%
\pgfsetbuttcap%
\pgfsetroundjoin%
\pgfsetlinewidth{1.505625pt}%
\definecolor{currentstroke}{rgb}{0.000000,0.000000,0.000000}%
\pgfsetstrokecolor{currentstroke}%
\pgfsetdash{}{0pt}%
\pgfpathmoveto{\pgfqpoint{1.769494in}{0.504733in}}%
\pgfpathlineto{\pgfqpoint{1.769494in}{0.543585in}}%
\pgfusepath{stroke}%
\end{pgfscope}%
\begin{pgfscope}%
\pgfpathrectangle{\pgfqpoint{0.506467in}{0.331635in}}{\pgfqpoint{2.583333in}{1.510000in}}%
\pgfusepath{clip}%
\pgfsetbuttcap%
\pgfsetroundjoin%
\pgfsetlinewidth{1.505625pt}%
\definecolor{currentstroke}{rgb}{0.000000,0.000000,0.000000}%
\pgfsetstrokecolor{currentstroke}%
\pgfsetdash{}{0pt}%
\pgfpathmoveto{\pgfqpoint{1.884054in}{0.504733in}}%
\pgfpathlineto{\pgfqpoint{1.884054in}{0.516218in}}%
\pgfusepath{stroke}%
\end{pgfscope}%
\begin{pgfscope}%
\pgfpathrectangle{\pgfqpoint{0.506467in}{0.331635in}}{\pgfqpoint{2.583333in}{1.510000in}}%
\pgfusepath{clip}%
\pgfsetbuttcap%
\pgfsetroundjoin%
\pgfsetlinewidth{1.505625pt}%
\definecolor{currentstroke}{rgb}{0.000000,0.000000,0.000000}%
\pgfsetstrokecolor{currentstroke}%
\pgfsetdash{}{0pt}%
\pgfpathmoveto{\pgfqpoint{1.998614in}{0.504733in}}%
\pgfpathlineto{\pgfqpoint{1.998614in}{0.495343in}}%
\pgfusepath{stroke}%
\end{pgfscope}%
\begin{pgfscope}%
\pgfpathrectangle{\pgfqpoint{0.506467in}{0.331635in}}{\pgfqpoint{2.583333in}{1.510000in}}%
\pgfusepath{clip}%
\pgfsetbuttcap%
\pgfsetroundjoin%
\pgfsetlinewidth{1.505625pt}%
\definecolor{currentstroke}{rgb}{0.000000,0.000000,0.000000}%
\pgfsetstrokecolor{currentstroke}%
\pgfsetdash{}{0pt}%
\pgfpathmoveto{\pgfqpoint{2.113174in}{0.504733in}}%
\pgfpathlineto{\pgfqpoint{2.113174in}{0.495276in}}%
\pgfusepath{stroke}%
\end{pgfscope}%
\begin{pgfscope}%
\pgfpathrectangle{\pgfqpoint{0.506467in}{0.331635in}}{\pgfqpoint{2.583333in}{1.510000in}}%
\pgfusepath{clip}%
\pgfsetbuttcap%
\pgfsetroundjoin%
\pgfsetlinewidth{1.505625pt}%
\definecolor{currentstroke}{rgb}{0.000000,0.000000,0.000000}%
\pgfsetstrokecolor{currentstroke}%
\pgfsetdash{}{0pt}%
\pgfpathmoveto{\pgfqpoint{2.227735in}{0.504733in}}%
\pgfpathlineto{\pgfqpoint{2.227735in}{0.493785in}}%
\pgfusepath{stroke}%
\end{pgfscope}%
\begin{pgfscope}%
\pgfpathrectangle{\pgfqpoint{0.506467in}{0.331635in}}{\pgfqpoint{2.583333in}{1.510000in}}%
\pgfusepath{clip}%
\pgfsetbuttcap%
\pgfsetroundjoin%
\pgfsetlinewidth{1.505625pt}%
\definecolor{currentstroke}{rgb}{0.000000,0.000000,0.000000}%
\pgfsetstrokecolor{currentstroke}%
\pgfsetdash{}{0pt}%
\pgfpathmoveto{\pgfqpoint{2.342295in}{0.504733in}}%
\pgfpathlineto{\pgfqpoint{2.342295in}{0.400271in}}%
\pgfusepath{stroke}%
\end{pgfscope}%
\begin{pgfscope}%
\pgfpathrectangle{\pgfqpoint{0.506467in}{0.331635in}}{\pgfqpoint{2.583333in}{1.510000in}}%
\pgfusepath{clip}%
\pgfsetbuttcap%
\pgfsetroundjoin%
\pgfsetlinewidth{1.505625pt}%
\definecolor{currentstroke}{rgb}{0.000000,0.000000,0.000000}%
\pgfsetstrokecolor{currentstroke}%
\pgfsetdash{}{0pt}%
\pgfpathmoveto{\pgfqpoint{2.456855in}{0.504733in}}%
\pgfpathlineto{\pgfqpoint{2.456855in}{0.504034in}}%
\pgfusepath{stroke}%
\end{pgfscope}%
\begin{pgfscope}%
\pgfpathrectangle{\pgfqpoint{0.506467in}{0.331635in}}{\pgfqpoint{2.583333in}{1.510000in}}%
\pgfusepath{clip}%
\pgfsetbuttcap%
\pgfsetroundjoin%
\pgfsetlinewidth{1.505625pt}%
\definecolor{currentstroke}{rgb}{0.000000,0.000000,0.000000}%
\pgfsetstrokecolor{currentstroke}%
\pgfsetdash{}{0pt}%
\pgfpathmoveto{\pgfqpoint{2.571415in}{0.504733in}}%
\pgfpathlineto{\pgfqpoint{2.571415in}{0.520419in}}%
\pgfusepath{stroke}%
\end{pgfscope}%
\begin{pgfscope}%
\pgfpathrectangle{\pgfqpoint{0.506467in}{0.331635in}}{\pgfqpoint{2.583333in}{1.510000in}}%
\pgfusepath{clip}%
\pgfsetbuttcap%
\pgfsetroundjoin%
\pgfsetlinewidth{1.505625pt}%
\definecolor{currentstroke}{rgb}{0.000000,0.000000,0.000000}%
\pgfsetstrokecolor{currentstroke}%
\pgfsetdash{}{0pt}%
\pgfpathmoveto{\pgfqpoint{2.685976in}{0.504733in}}%
\pgfpathlineto{\pgfqpoint{2.685976in}{0.488435in}}%
\pgfusepath{stroke}%
\end{pgfscope}%
\begin{pgfscope}%
\pgfpathrectangle{\pgfqpoint{0.506467in}{0.331635in}}{\pgfqpoint{2.583333in}{1.510000in}}%
\pgfusepath{clip}%
\pgfsetbuttcap%
\pgfsetroundjoin%
\pgfsetlinewidth{1.505625pt}%
\definecolor{currentstroke}{rgb}{0.000000,0.000000,0.000000}%
\pgfsetstrokecolor{currentstroke}%
\pgfsetdash{}{0pt}%
\pgfpathmoveto{\pgfqpoint{2.800536in}{0.504733in}}%
\pgfpathlineto{\pgfqpoint{2.800536in}{0.520056in}}%
\pgfusepath{stroke}%
\end{pgfscope}%
\begin{pgfscope}%
\pgfpathrectangle{\pgfqpoint{0.506467in}{0.331635in}}{\pgfqpoint{2.583333in}{1.510000in}}%
\pgfusepath{clip}%
\pgfsetbuttcap%
\pgfsetroundjoin%
\pgfsetlinewidth{1.505625pt}%
\definecolor{currentstroke}{rgb}{0.000000,0.000000,0.000000}%
\pgfsetstrokecolor{currentstroke}%
\pgfsetdash{}{0pt}%
\pgfpathmoveto{\pgfqpoint{2.915096in}{0.504733in}}%
\pgfpathlineto{\pgfqpoint{2.915096in}{0.539542in}}%
\pgfusepath{stroke}%
\end{pgfscope}%
\begin{pgfscope}%
\pgfpathrectangle{\pgfqpoint{0.506467in}{0.331635in}}{\pgfqpoint{2.583333in}{1.510000in}}%
\pgfusepath{clip}%
\pgfsetroundcap%
\pgfsetroundjoin%
\pgfsetlinewidth{1.505625pt}%
\definecolor{currentstroke}{rgb}{0.737255,0.741176,0.133333}%
\pgfsetstrokecolor{currentstroke}%
\pgfsetdash{}{0pt}%
\pgfpathmoveto{\pgfqpoint{0.506467in}{0.504733in}}%
\pgfpathlineto{\pgfqpoint{3.089800in}{0.504733in}}%
\pgfusepath{stroke}%
\end{pgfscope}%
\begin{pgfscope}%
\pgfpathrectangle{\pgfqpoint{0.506467in}{0.331635in}}{\pgfqpoint{2.583333in}{1.510000in}}%
\pgfusepath{clip}%
\pgfsetbuttcap%
\pgfsetroundjoin%
\definecolor{currentfill}{rgb}{0.737255,0.741176,0.133333}%
\pgfsetfillcolor{currentfill}%
\pgfsetlinewidth{1.003750pt}%
\definecolor{currentstroke}{rgb}{0.737255,0.741176,0.133333}%
\pgfsetstrokecolor{currentstroke}%
\pgfsetdash{}{0pt}%
\pgfsys@defobject{currentmarker}{\pgfqpoint{-0.034722in}{-0.034722in}}{\pgfqpoint{0.034722in}{0.034722in}}{%
\pgfpathmoveto{\pgfqpoint{0.000000in}{-0.034722in}}%
\pgfpathcurveto{\pgfqpoint{0.009208in}{-0.034722in}}{\pgfqpoint{0.018041in}{-0.031064in}}{\pgfqpoint{0.024552in}{-0.024552in}}%
\pgfpathcurveto{\pgfqpoint{0.031064in}{-0.018041in}}{\pgfqpoint{0.034722in}{-0.009208in}}{\pgfqpoint{0.034722in}{0.000000in}}%
\pgfpathcurveto{\pgfqpoint{0.034722in}{0.009208in}}{\pgfqpoint{0.031064in}{0.018041in}}{\pgfqpoint{0.024552in}{0.024552in}}%
\pgfpathcurveto{\pgfqpoint{0.018041in}{0.031064in}}{\pgfqpoint{0.009208in}{0.034722in}}{\pgfqpoint{0.000000in}{0.034722in}}%
\pgfpathcurveto{\pgfqpoint{-0.009208in}{0.034722in}}{\pgfqpoint{-0.018041in}{0.031064in}}{\pgfqpoint{-0.024552in}{0.024552in}}%
\pgfpathcurveto{\pgfqpoint{-0.031064in}{0.018041in}}{\pgfqpoint{-0.034722in}{0.009208in}}{\pgfqpoint{-0.034722in}{0.000000in}}%
\pgfpathcurveto{\pgfqpoint{-0.034722in}{-0.009208in}}{\pgfqpoint{-0.031064in}{-0.018041in}}{\pgfqpoint{-0.024552in}{-0.024552in}}%
\pgfpathcurveto{\pgfqpoint{-0.018041in}{-0.031064in}}{\pgfqpoint{-0.009208in}{-0.034722in}}{\pgfqpoint{0.000000in}{-0.034722in}}%
\pgfpathclose%
\pgfusepath{stroke,fill}%
}%
\begin{pgfscope}%
\pgfsys@transformshift{0.623891in}{1.772999in}%
\pgfsys@useobject{currentmarker}{}%
\end{pgfscope}%
\begin{pgfscope}%
\pgfsys@transformshift{0.738452in}{0.411212in}%
\pgfsys@useobject{currentmarker}{}%
\end{pgfscope}%
\begin{pgfscope}%
\pgfsys@transformshift{0.853012in}{0.456503in}%
\pgfsys@useobject{currentmarker}{}%
\end{pgfscope}%
\begin{pgfscope}%
\pgfsys@transformshift{0.967572in}{0.467381in}%
\pgfsys@useobject{currentmarker}{}%
\end{pgfscope}%
\begin{pgfscope}%
\pgfsys@transformshift{1.082132in}{0.439844in}%
\pgfsys@useobject{currentmarker}{}%
\end{pgfscope}%
\begin{pgfscope}%
\pgfsys@transformshift{1.196693in}{0.461770in}%
\pgfsys@useobject{currentmarker}{}%
\end{pgfscope}%
\begin{pgfscope}%
\pgfsys@transformshift{1.311253in}{0.519947in}%
\pgfsys@useobject{currentmarker}{}%
\end{pgfscope}%
\begin{pgfscope}%
\pgfsys@transformshift{1.425813in}{0.499056in}%
\pgfsys@useobject{currentmarker}{}%
\end{pgfscope}%
\begin{pgfscope}%
\pgfsys@transformshift{1.540373in}{0.588348in}%
\pgfsys@useobject{currentmarker}{}%
\end{pgfscope}%
\begin{pgfscope}%
\pgfsys@transformshift{1.654933in}{0.483209in}%
\pgfsys@useobject{currentmarker}{}%
\end{pgfscope}%
\begin{pgfscope}%
\pgfsys@transformshift{1.769494in}{0.543585in}%
\pgfsys@useobject{currentmarker}{}%
\end{pgfscope}%
\begin{pgfscope}%
\pgfsys@transformshift{1.884054in}{0.516218in}%
\pgfsys@useobject{currentmarker}{}%
\end{pgfscope}%
\begin{pgfscope}%
\pgfsys@transformshift{1.998614in}{0.495343in}%
\pgfsys@useobject{currentmarker}{}%
\end{pgfscope}%
\begin{pgfscope}%
\pgfsys@transformshift{2.113174in}{0.495276in}%
\pgfsys@useobject{currentmarker}{}%
\end{pgfscope}%
\begin{pgfscope}%
\pgfsys@transformshift{2.227735in}{0.493785in}%
\pgfsys@useobject{currentmarker}{}%
\end{pgfscope}%
\begin{pgfscope}%
\pgfsys@transformshift{2.342295in}{0.400271in}%
\pgfsys@useobject{currentmarker}{}%
\end{pgfscope}%
\begin{pgfscope}%
\pgfsys@transformshift{2.456855in}{0.504034in}%
\pgfsys@useobject{currentmarker}{}%
\end{pgfscope}%
\begin{pgfscope}%
\pgfsys@transformshift{2.571415in}{0.520419in}%
\pgfsys@useobject{currentmarker}{}%
\end{pgfscope}%
\begin{pgfscope}%
\pgfsys@transformshift{2.685976in}{0.488435in}%
\pgfsys@useobject{currentmarker}{}%
\end{pgfscope}%
\begin{pgfscope}%
\pgfsys@transformshift{2.800536in}{0.520056in}%
\pgfsys@useobject{currentmarker}{}%
\end{pgfscope}%
\begin{pgfscope}%
\pgfsys@transformshift{2.915096in}{0.539542in}%
\pgfsys@useobject{currentmarker}{}%
\end{pgfscope}%
\end{pgfscope}%
\begin{pgfscope}%
\pgfsetrectcap%
\pgfsetmiterjoin%
\pgfsetlinewidth{0.803000pt}%
\definecolor{currentstroke}{rgb}{1.000000,1.000000,1.000000}%
\pgfsetstrokecolor{currentstroke}%
\pgfsetdash{}{0pt}%
\pgfpathmoveto{\pgfqpoint{0.506467in}{0.331635in}}%
\pgfpathlineto{\pgfqpoint{0.506467in}{1.841635in}}%
\pgfusepath{stroke}%
\end{pgfscope}%
\begin{pgfscope}%
\pgfsetrectcap%
\pgfsetmiterjoin%
\pgfsetlinewidth{0.803000pt}%
\definecolor{currentstroke}{rgb}{1.000000,1.000000,1.000000}%
\pgfsetstrokecolor{currentstroke}%
\pgfsetdash{}{0pt}%
\pgfpathmoveto{\pgfqpoint{3.089800in}{0.331635in}}%
\pgfpathlineto{\pgfqpoint{3.089800in}{1.841635in}}%
\pgfusepath{stroke}%
\end{pgfscope}%
\begin{pgfscope}%
\pgfsetrectcap%
\pgfsetmiterjoin%
\pgfsetlinewidth{0.803000pt}%
\definecolor{currentstroke}{rgb}{1.000000,1.000000,1.000000}%
\pgfsetstrokecolor{currentstroke}%
\pgfsetdash{}{0pt}%
\pgfpathmoveto{\pgfqpoint{0.506467in}{0.331635in}}%
\pgfpathlineto{\pgfqpoint{3.089800in}{0.331635in}}%
\pgfusepath{stroke}%
\end{pgfscope}%
\begin{pgfscope}%
\pgfsetrectcap%
\pgfsetmiterjoin%
\pgfsetlinewidth{0.803000pt}%
\definecolor{currentstroke}{rgb}{1.000000,1.000000,1.000000}%
\pgfsetstrokecolor{currentstroke}%
\pgfsetdash{}{0pt}%
\pgfpathmoveto{\pgfqpoint{0.506467in}{1.841635in}}%
\pgfpathlineto{\pgfqpoint{3.089800in}{1.841635in}}%
\pgfusepath{stroke}%
\end{pgfscope}%
\begin{pgfscope}%
\definecolor{textcolor}{rgb}{0.150000,0.150000,0.150000}%
\pgfsetstrokecolor{textcolor}%
\pgfsetfillcolor{textcolor}%
\pgftext[x=1.798134in,y=1.924968in,,base]{\color{textcolor}\rmfamily\fontsize{12.000000}{14.400000}\selectfont Autocorrelation}%
\end{pgfscope}%
\begin{pgfscope}%
\pgfsetbuttcap%
\pgfsetmiterjoin%
\definecolor{currentfill}{rgb}{0.917647,0.917647,0.949020}%
\pgfsetfillcolor{currentfill}%
\pgfsetlinewidth{0.000000pt}%
\definecolor{currentstroke}{rgb}{0.000000,0.000000,0.000000}%
\pgfsetstrokecolor{currentstroke}%
\pgfsetstrokeopacity{0.000000}%
\pgfsetdash{}{0pt}%
\pgfpathmoveto{\pgfqpoint{4.123134in}{0.331635in}}%
\pgfpathlineto{\pgfqpoint{6.706467in}{0.331635in}}%
\pgfpathlineto{\pgfqpoint{6.706467in}{1.841635in}}%
\pgfpathlineto{\pgfqpoint{4.123134in}{1.841635in}}%
\pgfpathclose%
\pgfusepath{fill}%
\end{pgfscope}%
\begin{pgfscope}%
\pgfpathrectangle{\pgfqpoint{4.123134in}{0.331635in}}{\pgfqpoint{2.583333in}{1.510000in}}%
\pgfusepath{clip}%
\pgfsetroundcap%
\pgfsetroundjoin%
\pgfsetlinewidth{0.803000pt}%
\definecolor{currentstroke}{rgb}{1.000000,1.000000,1.000000}%
\pgfsetstrokecolor{currentstroke}%
\pgfsetdash{}{0pt}%
\pgfpathmoveto{\pgfqpoint{4.240558in}{0.331635in}}%
\pgfpathlineto{\pgfqpoint{4.240558in}{1.841635in}}%
\pgfusepath{stroke}%
\end{pgfscope}%
\begin{pgfscope}%
\definecolor{textcolor}{rgb}{0.150000,0.150000,0.150000}%
\pgfsetstrokecolor{textcolor}%
\pgfsetfillcolor{textcolor}%
\pgftext[x=4.240558in,y=0.234413in,,top]{\color{textcolor}\rmfamily\fontsize{10.000000}{12.000000}\selectfont 0}%
\end{pgfscope}%
\begin{pgfscope}%
\pgfpathrectangle{\pgfqpoint{4.123134in}{0.331635in}}{\pgfqpoint{2.583333in}{1.510000in}}%
\pgfusepath{clip}%
\pgfsetroundcap%
\pgfsetroundjoin%
\pgfsetlinewidth{0.803000pt}%
\definecolor{currentstroke}{rgb}{1.000000,1.000000,1.000000}%
\pgfsetstrokecolor{currentstroke}%
\pgfsetdash{}{0pt}%
\pgfpathmoveto{\pgfqpoint{4.813359in}{0.331635in}}%
\pgfpathlineto{\pgfqpoint{4.813359in}{1.841635in}}%
\pgfusepath{stroke}%
\end{pgfscope}%
\begin{pgfscope}%
\definecolor{textcolor}{rgb}{0.150000,0.150000,0.150000}%
\pgfsetstrokecolor{textcolor}%
\pgfsetfillcolor{textcolor}%
\pgftext[x=4.813359in,y=0.234413in,,top]{\color{textcolor}\rmfamily\fontsize{10.000000}{12.000000}\selectfont 5}%
\end{pgfscope}%
\begin{pgfscope}%
\pgfpathrectangle{\pgfqpoint{4.123134in}{0.331635in}}{\pgfqpoint{2.583333in}{1.510000in}}%
\pgfusepath{clip}%
\pgfsetroundcap%
\pgfsetroundjoin%
\pgfsetlinewidth{0.803000pt}%
\definecolor{currentstroke}{rgb}{1.000000,1.000000,1.000000}%
\pgfsetstrokecolor{currentstroke}%
\pgfsetdash{}{0pt}%
\pgfpathmoveto{\pgfqpoint{5.386160in}{0.331635in}}%
\pgfpathlineto{\pgfqpoint{5.386160in}{1.841635in}}%
\pgfusepath{stroke}%
\end{pgfscope}%
\begin{pgfscope}%
\definecolor{textcolor}{rgb}{0.150000,0.150000,0.150000}%
\pgfsetstrokecolor{textcolor}%
\pgfsetfillcolor{textcolor}%
\pgftext[x=5.386160in,y=0.234413in,,top]{\color{textcolor}\rmfamily\fontsize{10.000000}{12.000000}\selectfont 10}%
\end{pgfscope}%
\begin{pgfscope}%
\pgfpathrectangle{\pgfqpoint{4.123134in}{0.331635in}}{\pgfqpoint{2.583333in}{1.510000in}}%
\pgfusepath{clip}%
\pgfsetroundcap%
\pgfsetroundjoin%
\pgfsetlinewidth{0.803000pt}%
\definecolor{currentstroke}{rgb}{1.000000,1.000000,1.000000}%
\pgfsetstrokecolor{currentstroke}%
\pgfsetdash{}{0pt}%
\pgfpathmoveto{\pgfqpoint{5.958962in}{0.331635in}}%
\pgfpathlineto{\pgfqpoint{5.958962in}{1.841635in}}%
\pgfusepath{stroke}%
\end{pgfscope}%
\begin{pgfscope}%
\definecolor{textcolor}{rgb}{0.150000,0.150000,0.150000}%
\pgfsetstrokecolor{textcolor}%
\pgfsetfillcolor{textcolor}%
\pgftext[x=5.958962in,y=0.234413in,,top]{\color{textcolor}\rmfamily\fontsize{10.000000}{12.000000}\selectfont 15}%
\end{pgfscope}%
\begin{pgfscope}%
\pgfpathrectangle{\pgfqpoint{4.123134in}{0.331635in}}{\pgfqpoint{2.583333in}{1.510000in}}%
\pgfusepath{clip}%
\pgfsetroundcap%
\pgfsetroundjoin%
\pgfsetlinewidth{0.803000pt}%
\definecolor{currentstroke}{rgb}{1.000000,1.000000,1.000000}%
\pgfsetstrokecolor{currentstroke}%
\pgfsetdash{}{0pt}%
\pgfpathmoveto{\pgfqpoint{6.531763in}{0.331635in}}%
\pgfpathlineto{\pgfqpoint{6.531763in}{1.841635in}}%
\pgfusepath{stroke}%
\end{pgfscope}%
\begin{pgfscope}%
\definecolor{textcolor}{rgb}{0.150000,0.150000,0.150000}%
\pgfsetstrokecolor{textcolor}%
\pgfsetfillcolor{textcolor}%
\pgftext[x=6.531763in,y=0.234413in,,top]{\color{textcolor}\rmfamily\fontsize{10.000000}{12.000000}\selectfont 20}%
\end{pgfscope}%
\begin{pgfscope}%
\pgfpathrectangle{\pgfqpoint{4.123134in}{0.331635in}}{\pgfqpoint{2.583333in}{1.510000in}}%
\pgfusepath{clip}%
\pgfsetroundcap%
\pgfsetroundjoin%
\pgfsetlinewidth{0.803000pt}%
\definecolor{currentstroke}{rgb}{1.000000,1.000000,1.000000}%
\pgfsetstrokecolor{currentstroke}%
\pgfsetdash{}{0pt}%
\pgfpathmoveto{\pgfqpoint{4.123134in}{0.504323in}}%
\pgfpathlineto{\pgfqpoint{6.706467in}{0.504323in}}%
\pgfusepath{stroke}%
\end{pgfscope}%
\begin{pgfscope}%
\definecolor{textcolor}{rgb}{0.150000,0.150000,0.150000}%
\pgfsetstrokecolor{textcolor}%
\pgfsetfillcolor{textcolor}%
\pgftext[x=3.716667in,y=0.451562in,left,base]{\color{textcolor}\rmfamily\fontsize{10.000000}{12.000000}\selectfont 0.00}%
\end{pgfscope}%
\begin{pgfscope}%
\pgfpathrectangle{\pgfqpoint{4.123134in}{0.331635in}}{\pgfqpoint{2.583333in}{1.510000in}}%
\pgfusepath{clip}%
\pgfsetroundcap%
\pgfsetroundjoin%
\pgfsetlinewidth{0.803000pt}%
\definecolor{currentstroke}{rgb}{1.000000,1.000000,1.000000}%
\pgfsetstrokecolor{currentstroke}%
\pgfsetdash{}{0pt}%
\pgfpathmoveto{\pgfqpoint{4.123134in}{0.821492in}}%
\pgfpathlineto{\pgfqpoint{6.706467in}{0.821492in}}%
\pgfusepath{stroke}%
\end{pgfscope}%
\begin{pgfscope}%
\definecolor{textcolor}{rgb}{0.150000,0.150000,0.150000}%
\pgfsetstrokecolor{textcolor}%
\pgfsetfillcolor{textcolor}%
\pgftext[x=3.716667in,y=0.768731in,left,base]{\color{textcolor}\rmfamily\fontsize{10.000000}{12.000000}\selectfont 0.25}%
\end{pgfscope}%
\begin{pgfscope}%
\pgfpathrectangle{\pgfqpoint{4.123134in}{0.331635in}}{\pgfqpoint{2.583333in}{1.510000in}}%
\pgfusepath{clip}%
\pgfsetroundcap%
\pgfsetroundjoin%
\pgfsetlinewidth{0.803000pt}%
\definecolor{currentstroke}{rgb}{1.000000,1.000000,1.000000}%
\pgfsetstrokecolor{currentstroke}%
\pgfsetdash{}{0pt}%
\pgfpathmoveto{\pgfqpoint{4.123134in}{1.138661in}}%
\pgfpathlineto{\pgfqpoint{6.706467in}{1.138661in}}%
\pgfusepath{stroke}%
\end{pgfscope}%
\begin{pgfscope}%
\definecolor{textcolor}{rgb}{0.150000,0.150000,0.150000}%
\pgfsetstrokecolor{textcolor}%
\pgfsetfillcolor{textcolor}%
\pgftext[x=3.716667in,y=1.085900in,left,base]{\color{textcolor}\rmfamily\fontsize{10.000000}{12.000000}\selectfont 0.50}%
\end{pgfscope}%
\begin{pgfscope}%
\pgfpathrectangle{\pgfqpoint{4.123134in}{0.331635in}}{\pgfqpoint{2.583333in}{1.510000in}}%
\pgfusepath{clip}%
\pgfsetroundcap%
\pgfsetroundjoin%
\pgfsetlinewidth{0.803000pt}%
\definecolor{currentstroke}{rgb}{1.000000,1.000000,1.000000}%
\pgfsetstrokecolor{currentstroke}%
\pgfsetdash{}{0pt}%
\pgfpathmoveto{\pgfqpoint{4.123134in}{1.455830in}}%
\pgfpathlineto{\pgfqpoint{6.706467in}{1.455830in}}%
\pgfusepath{stroke}%
\end{pgfscope}%
\begin{pgfscope}%
\definecolor{textcolor}{rgb}{0.150000,0.150000,0.150000}%
\pgfsetstrokecolor{textcolor}%
\pgfsetfillcolor{textcolor}%
\pgftext[x=3.716667in,y=1.403068in,left,base]{\color{textcolor}\rmfamily\fontsize{10.000000}{12.000000}\selectfont 0.75}%
\end{pgfscope}%
\begin{pgfscope}%
\pgfpathrectangle{\pgfqpoint{4.123134in}{0.331635in}}{\pgfqpoint{2.583333in}{1.510000in}}%
\pgfusepath{clip}%
\pgfsetroundcap%
\pgfsetroundjoin%
\pgfsetlinewidth{0.803000pt}%
\definecolor{currentstroke}{rgb}{1.000000,1.000000,1.000000}%
\pgfsetstrokecolor{currentstroke}%
\pgfsetdash{}{0pt}%
\pgfpathmoveto{\pgfqpoint{4.123134in}{1.772999in}}%
\pgfpathlineto{\pgfqpoint{6.706467in}{1.772999in}}%
\pgfusepath{stroke}%
\end{pgfscope}%
\begin{pgfscope}%
\definecolor{textcolor}{rgb}{0.150000,0.150000,0.150000}%
\pgfsetstrokecolor{textcolor}%
\pgfsetfillcolor{textcolor}%
\pgftext[x=3.716667in,y=1.720237in,left,base]{\color{textcolor}\rmfamily\fontsize{10.000000}{12.000000}\selectfont 1.00}%
\end{pgfscope}%
\begin{pgfscope}%
\pgfpathrectangle{\pgfqpoint{4.123134in}{0.331635in}}{\pgfqpoint{2.583333in}{1.510000in}}%
\pgfusepath{clip}%
\pgfsetbuttcap%
\pgfsetroundjoin%
\definecolor{currentfill}{rgb}{0.121569,0.466667,0.705882}%
\pgfsetfillcolor{currentfill}%
\pgfsetfillopacity{0.250000}%
\pgfsetlinewidth{1.003750pt}%
\definecolor{currentstroke}{rgb}{1.000000,1.000000,1.000000}%
\pgfsetstrokecolor{currentstroke}%
\pgfsetstrokeopacity{0.250000}%
\pgfsetdash{}{0pt}%
\pgfpathmoveto{\pgfqpoint{4.297838in}{0.568356in}}%
\pgfpathlineto{\pgfqpoint{4.297838in}{0.440291in}}%
\pgfpathlineto{\pgfqpoint{4.469678in}{0.440291in}}%
\pgfpathlineto{\pgfqpoint{4.584239in}{0.440291in}}%
\pgfpathlineto{\pgfqpoint{4.698799in}{0.440291in}}%
\pgfpathlineto{\pgfqpoint{4.813359in}{0.440291in}}%
\pgfpathlineto{\pgfqpoint{4.927919in}{0.440291in}}%
\pgfpathlineto{\pgfqpoint{5.042480in}{0.440291in}}%
\pgfpathlineto{\pgfqpoint{5.157040in}{0.440291in}}%
\pgfpathlineto{\pgfqpoint{5.271600in}{0.440291in}}%
\pgfpathlineto{\pgfqpoint{5.386160in}{0.440291in}}%
\pgfpathlineto{\pgfqpoint{5.500721in}{0.440291in}}%
\pgfpathlineto{\pgfqpoint{5.615281in}{0.440291in}}%
\pgfpathlineto{\pgfqpoint{5.729841in}{0.440291in}}%
\pgfpathlineto{\pgfqpoint{5.844401in}{0.440291in}}%
\pgfpathlineto{\pgfqpoint{5.958962in}{0.440291in}}%
\pgfpathlineto{\pgfqpoint{6.073522in}{0.440291in}}%
\pgfpathlineto{\pgfqpoint{6.188082in}{0.440291in}}%
\pgfpathlineto{\pgfqpoint{6.302642in}{0.440291in}}%
\pgfpathlineto{\pgfqpoint{6.417202in}{0.440291in}}%
\pgfpathlineto{\pgfqpoint{6.589043in}{0.440291in}}%
\pgfpathlineto{\pgfqpoint{6.589043in}{0.568356in}}%
\pgfpathlineto{\pgfqpoint{6.589043in}{0.568356in}}%
\pgfpathlineto{\pgfqpoint{6.417202in}{0.568356in}}%
\pgfpathlineto{\pgfqpoint{6.302642in}{0.568356in}}%
\pgfpathlineto{\pgfqpoint{6.188082in}{0.568356in}}%
\pgfpathlineto{\pgfqpoint{6.073522in}{0.568356in}}%
\pgfpathlineto{\pgfqpoint{5.958962in}{0.568356in}}%
\pgfpathlineto{\pgfqpoint{5.844401in}{0.568356in}}%
\pgfpathlineto{\pgfqpoint{5.729841in}{0.568356in}}%
\pgfpathlineto{\pgfqpoint{5.615281in}{0.568356in}}%
\pgfpathlineto{\pgfqpoint{5.500721in}{0.568356in}}%
\pgfpathlineto{\pgfqpoint{5.386160in}{0.568356in}}%
\pgfpathlineto{\pgfqpoint{5.271600in}{0.568356in}}%
\pgfpathlineto{\pgfqpoint{5.157040in}{0.568356in}}%
\pgfpathlineto{\pgfqpoint{5.042480in}{0.568356in}}%
\pgfpathlineto{\pgfqpoint{4.927919in}{0.568356in}}%
\pgfpathlineto{\pgfqpoint{4.813359in}{0.568356in}}%
\pgfpathlineto{\pgfqpoint{4.698799in}{0.568356in}}%
\pgfpathlineto{\pgfqpoint{4.584239in}{0.568356in}}%
\pgfpathlineto{\pgfqpoint{4.469678in}{0.568356in}}%
\pgfpathlineto{\pgfqpoint{4.297838in}{0.568356in}}%
\pgfpathclose%
\pgfusepath{stroke,fill}%
\end{pgfscope}%
\begin{pgfscope}%
\pgfpathrectangle{\pgfqpoint{4.123134in}{0.331635in}}{\pgfqpoint{2.583333in}{1.510000in}}%
\pgfusepath{clip}%
\pgfsetbuttcap%
\pgfsetroundjoin%
\pgfsetlinewidth{1.505625pt}%
\definecolor{currentstroke}{rgb}{0.000000,0.000000,0.000000}%
\pgfsetstrokecolor{currentstroke}%
\pgfsetdash{}{0pt}%
\pgfpathmoveto{\pgfqpoint{4.240558in}{0.504323in}}%
\pgfpathlineto{\pgfqpoint{4.240558in}{1.772999in}}%
\pgfusepath{stroke}%
\end{pgfscope}%
\begin{pgfscope}%
\pgfpathrectangle{\pgfqpoint{4.123134in}{0.331635in}}{\pgfqpoint{2.583333in}{1.510000in}}%
\pgfusepath{clip}%
\pgfsetbuttcap%
\pgfsetroundjoin%
\pgfsetlinewidth{1.505625pt}%
\definecolor{currentstroke}{rgb}{0.000000,0.000000,0.000000}%
\pgfsetstrokecolor{currentstroke}%
\pgfsetdash{}{0pt}%
\pgfpathmoveto{\pgfqpoint{4.355118in}{0.504323in}}%
\pgfpathlineto{\pgfqpoint{4.355118in}{0.410710in}}%
\pgfusepath{stroke}%
\end{pgfscope}%
\begin{pgfscope}%
\pgfpathrectangle{\pgfqpoint{4.123134in}{0.331635in}}{\pgfqpoint{2.583333in}{1.510000in}}%
\pgfusepath{clip}%
\pgfsetbuttcap%
\pgfsetroundjoin%
\pgfsetlinewidth{1.505625pt}%
\definecolor{currentstroke}{rgb}{0.000000,0.000000,0.000000}%
\pgfsetstrokecolor{currentstroke}%
\pgfsetdash{}{0pt}%
\pgfpathmoveto{\pgfqpoint{4.469678in}{0.504323in}}%
\pgfpathlineto{\pgfqpoint{4.469678in}{0.448803in}}%
\pgfusepath{stroke}%
\end{pgfscope}%
\begin{pgfscope}%
\pgfpathrectangle{\pgfqpoint{4.123134in}{0.331635in}}{\pgfqpoint{2.583333in}{1.510000in}}%
\pgfusepath{clip}%
\pgfsetbuttcap%
\pgfsetroundjoin%
\pgfsetlinewidth{1.505625pt}%
\definecolor{currentstroke}{rgb}{0.000000,0.000000,0.000000}%
\pgfsetstrokecolor{currentstroke}%
\pgfsetdash{}{0pt}%
\pgfpathmoveto{\pgfqpoint{4.584239in}{0.504323in}}%
\pgfpathlineto{\pgfqpoint{4.584239in}{0.458732in}}%
\pgfusepath{stroke}%
\end{pgfscope}%
\begin{pgfscope}%
\pgfpathrectangle{\pgfqpoint{4.123134in}{0.331635in}}{\pgfqpoint{2.583333in}{1.510000in}}%
\pgfusepath{clip}%
\pgfsetbuttcap%
\pgfsetroundjoin%
\pgfsetlinewidth{1.505625pt}%
\definecolor{currentstroke}{rgb}{0.000000,0.000000,0.000000}%
\pgfsetstrokecolor{currentstroke}%
\pgfsetdash{}{0pt}%
\pgfpathmoveto{\pgfqpoint{4.698799in}{0.504323in}}%
\pgfpathlineto{\pgfqpoint{4.698799in}{0.430045in}}%
\pgfusepath{stroke}%
\end{pgfscope}%
\begin{pgfscope}%
\pgfpathrectangle{\pgfqpoint{4.123134in}{0.331635in}}{\pgfqpoint{2.583333in}{1.510000in}}%
\pgfusepath{clip}%
\pgfsetbuttcap%
\pgfsetroundjoin%
\pgfsetlinewidth{1.505625pt}%
\definecolor{currentstroke}{rgb}{0.000000,0.000000,0.000000}%
\pgfsetstrokecolor{currentstroke}%
\pgfsetdash{}{0pt}%
\pgfpathmoveto{\pgfqpoint{4.813359in}{0.504323in}}%
\pgfpathlineto{\pgfqpoint{4.813359in}{0.445966in}}%
\pgfusepath{stroke}%
\end{pgfscope}%
\begin{pgfscope}%
\pgfpathrectangle{\pgfqpoint{4.123134in}{0.331635in}}{\pgfqpoint{2.583333in}{1.510000in}}%
\pgfusepath{clip}%
\pgfsetbuttcap%
\pgfsetroundjoin%
\pgfsetlinewidth{1.505625pt}%
\definecolor{currentstroke}{rgb}{0.000000,0.000000,0.000000}%
\pgfsetstrokecolor{currentstroke}%
\pgfsetdash{}{0pt}%
\pgfpathmoveto{\pgfqpoint{4.927919in}{0.504323in}}%
\pgfpathlineto{\pgfqpoint{4.927919in}{0.503755in}}%
\pgfusepath{stroke}%
\end{pgfscope}%
\begin{pgfscope}%
\pgfpathrectangle{\pgfqpoint{4.123134in}{0.331635in}}{\pgfqpoint{2.583333in}{1.510000in}}%
\pgfusepath{clip}%
\pgfsetbuttcap%
\pgfsetroundjoin%
\pgfsetlinewidth{1.505625pt}%
\definecolor{currentstroke}{rgb}{0.000000,0.000000,0.000000}%
\pgfsetstrokecolor{currentstroke}%
\pgfsetdash{}{0pt}%
\pgfpathmoveto{\pgfqpoint{5.042480in}{0.504323in}}%
\pgfpathlineto{\pgfqpoint{5.042480in}{0.490101in}}%
\pgfusepath{stroke}%
\end{pgfscope}%
\begin{pgfscope}%
\pgfpathrectangle{\pgfqpoint{4.123134in}{0.331635in}}{\pgfqpoint{2.583333in}{1.510000in}}%
\pgfusepath{clip}%
\pgfsetbuttcap%
\pgfsetroundjoin%
\pgfsetlinewidth{1.505625pt}%
\definecolor{currentstroke}{rgb}{0.000000,0.000000,0.000000}%
\pgfsetstrokecolor{currentstroke}%
\pgfsetdash{}{0pt}%
\pgfpathmoveto{\pgfqpoint{5.157040in}{0.504323in}}%
\pgfpathlineto{\pgfqpoint{5.157040in}{0.580994in}}%
\pgfusepath{stroke}%
\end{pgfscope}%
\begin{pgfscope}%
\pgfpathrectangle{\pgfqpoint{4.123134in}{0.331635in}}{\pgfqpoint{2.583333in}{1.510000in}}%
\pgfusepath{clip}%
\pgfsetbuttcap%
\pgfsetroundjoin%
\pgfsetlinewidth{1.505625pt}%
\definecolor{currentstroke}{rgb}{0.000000,0.000000,0.000000}%
\pgfsetstrokecolor{currentstroke}%
\pgfsetdash{}{0pt}%
\pgfpathmoveto{\pgfqpoint{5.271600in}{0.504323in}}%
\pgfpathlineto{\pgfqpoint{5.271600in}{0.489705in}}%
\pgfusepath{stroke}%
\end{pgfscope}%
\begin{pgfscope}%
\pgfpathrectangle{\pgfqpoint{4.123134in}{0.331635in}}{\pgfqpoint{2.583333in}{1.510000in}}%
\pgfusepath{clip}%
\pgfsetbuttcap%
\pgfsetroundjoin%
\pgfsetlinewidth{1.505625pt}%
\definecolor{currentstroke}{rgb}{0.000000,0.000000,0.000000}%
\pgfsetstrokecolor{currentstroke}%
\pgfsetdash{}{0pt}%
\pgfpathmoveto{\pgfqpoint{5.386160in}{0.504323in}}%
\pgfpathlineto{\pgfqpoint{5.386160in}{0.547158in}}%
\pgfusepath{stroke}%
\end{pgfscope}%
\begin{pgfscope}%
\pgfpathrectangle{\pgfqpoint{4.123134in}{0.331635in}}{\pgfqpoint{2.583333in}{1.510000in}}%
\pgfusepath{clip}%
\pgfsetbuttcap%
\pgfsetroundjoin%
\pgfsetlinewidth{1.505625pt}%
\definecolor{currentstroke}{rgb}{0.000000,0.000000,0.000000}%
\pgfsetstrokecolor{currentstroke}%
\pgfsetdash{}{0pt}%
\pgfpathmoveto{\pgfqpoint{5.500721in}{0.504323in}}%
\pgfpathlineto{\pgfqpoint{5.500721in}{0.526902in}}%
\pgfusepath{stroke}%
\end{pgfscope}%
\begin{pgfscope}%
\pgfpathrectangle{\pgfqpoint{4.123134in}{0.331635in}}{\pgfqpoint{2.583333in}{1.510000in}}%
\pgfusepath{clip}%
\pgfsetbuttcap%
\pgfsetroundjoin%
\pgfsetlinewidth{1.505625pt}%
\definecolor{currentstroke}{rgb}{0.000000,0.000000,0.000000}%
\pgfsetstrokecolor{currentstroke}%
\pgfsetdash{}{0pt}%
\pgfpathmoveto{\pgfqpoint{5.615281in}{0.504323in}}%
\pgfpathlineto{\pgfqpoint{5.615281in}{0.508693in}}%
\pgfusepath{stroke}%
\end{pgfscope}%
\begin{pgfscope}%
\pgfpathrectangle{\pgfqpoint{4.123134in}{0.331635in}}{\pgfqpoint{2.583333in}{1.510000in}}%
\pgfusepath{clip}%
\pgfsetbuttcap%
\pgfsetroundjoin%
\pgfsetlinewidth{1.505625pt}%
\definecolor{currentstroke}{rgb}{0.000000,0.000000,0.000000}%
\pgfsetstrokecolor{currentstroke}%
\pgfsetdash{}{0pt}%
\pgfpathmoveto{\pgfqpoint{5.729841in}{0.504323in}}%
\pgfpathlineto{\pgfqpoint{5.729841in}{0.503358in}}%
\pgfusepath{stroke}%
\end{pgfscope}%
\begin{pgfscope}%
\pgfpathrectangle{\pgfqpoint{4.123134in}{0.331635in}}{\pgfqpoint{2.583333in}{1.510000in}}%
\pgfusepath{clip}%
\pgfsetbuttcap%
\pgfsetroundjoin%
\pgfsetlinewidth{1.505625pt}%
\definecolor{currentstroke}{rgb}{0.000000,0.000000,0.000000}%
\pgfsetstrokecolor{currentstroke}%
\pgfsetdash{}{0pt}%
\pgfpathmoveto{\pgfqpoint{5.844401in}{0.504323in}}%
\pgfpathlineto{\pgfqpoint{5.844401in}{0.495101in}}%
\pgfusepath{stroke}%
\end{pgfscope}%
\begin{pgfscope}%
\pgfpathrectangle{\pgfqpoint{4.123134in}{0.331635in}}{\pgfqpoint{2.583333in}{1.510000in}}%
\pgfusepath{clip}%
\pgfsetbuttcap%
\pgfsetroundjoin%
\pgfsetlinewidth{1.505625pt}%
\definecolor{currentstroke}{rgb}{0.000000,0.000000,0.000000}%
\pgfsetstrokecolor{currentstroke}%
\pgfsetdash{}{0pt}%
\pgfpathmoveto{\pgfqpoint{5.958962in}{0.504323in}}%
\pgfpathlineto{\pgfqpoint{5.958962in}{0.400271in}}%
\pgfusepath{stroke}%
\end{pgfscope}%
\begin{pgfscope}%
\pgfpathrectangle{\pgfqpoint{4.123134in}{0.331635in}}{\pgfqpoint{2.583333in}{1.510000in}}%
\pgfusepath{clip}%
\pgfsetbuttcap%
\pgfsetroundjoin%
\pgfsetlinewidth{1.505625pt}%
\definecolor{currentstroke}{rgb}{0.000000,0.000000,0.000000}%
\pgfsetstrokecolor{currentstroke}%
\pgfsetdash{}{0pt}%
\pgfpathmoveto{\pgfqpoint{6.073522in}{0.504323in}}%
\pgfpathlineto{\pgfqpoint{6.073522in}{0.479252in}}%
\pgfusepath{stroke}%
\end{pgfscope}%
\begin{pgfscope}%
\pgfpathrectangle{\pgfqpoint{4.123134in}{0.331635in}}{\pgfqpoint{2.583333in}{1.510000in}}%
\pgfusepath{clip}%
\pgfsetbuttcap%
\pgfsetroundjoin%
\pgfsetlinewidth{1.505625pt}%
\definecolor{currentstroke}{rgb}{0.000000,0.000000,0.000000}%
\pgfsetstrokecolor{currentstroke}%
\pgfsetdash{}{0pt}%
\pgfpathmoveto{\pgfqpoint{6.188082in}{0.504323in}}%
\pgfpathlineto{\pgfqpoint{6.188082in}{0.508120in}}%
\pgfusepath{stroke}%
\end{pgfscope}%
\begin{pgfscope}%
\pgfpathrectangle{\pgfqpoint{4.123134in}{0.331635in}}{\pgfqpoint{2.583333in}{1.510000in}}%
\pgfusepath{clip}%
\pgfsetbuttcap%
\pgfsetroundjoin%
\pgfsetlinewidth{1.505625pt}%
\definecolor{currentstroke}{rgb}{0.000000,0.000000,0.000000}%
\pgfsetstrokecolor{currentstroke}%
\pgfsetdash{}{0pt}%
\pgfpathmoveto{\pgfqpoint{6.302642in}{0.504323in}}%
\pgfpathlineto{\pgfqpoint{6.302642in}{0.472725in}}%
\pgfusepath{stroke}%
\end{pgfscope}%
\begin{pgfscope}%
\pgfpathrectangle{\pgfqpoint{4.123134in}{0.331635in}}{\pgfqpoint{2.583333in}{1.510000in}}%
\pgfusepath{clip}%
\pgfsetbuttcap%
\pgfsetroundjoin%
\pgfsetlinewidth{1.505625pt}%
\definecolor{currentstroke}{rgb}{0.000000,0.000000,0.000000}%
\pgfsetstrokecolor{currentstroke}%
\pgfsetdash{}{0pt}%
\pgfpathmoveto{\pgfqpoint{6.417202in}{0.504323in}}%
\pgfpathlineto{\pgfqpoint{6.417202in}{0.502799in}}%
\pgfusepath{stroke}%
\end{pgfscope}%
\begin{pgfscope}%
\pgfpathrectangle{\pgfqpoint{4.123134in}{0.331635in}}{\pgfqpoint{2.583333in}{1.510000in}}%
\pgfusepath{clip}%
\pgfsetbuttcap%
\pgfsetroundjoin%
\pgfsetlinewidth{1.505625pt}%
\definecolor{currentstroke}{rgb}{0.000000,0.000000,0.000000}%
\pgfsetstrokecolor{currentstroke}%
\pgfsetdash{}{0pt}%
\pgfpathmoveto{\pgfqpoint{6.531763in}{0.504323in}}%
\pgfpathlineto{\pgfqpoint{6.531763in}{0.530890in}}%
\pgfusepath{stroke}%
\end{pgfscope}%
\begin{pgfscope}%
\pgfpathrectangle{\pgfqpoint{4.123134in}{0.331635in}}{\pgfqpoint{2.583333in}{1.510000in}}%
\pgfusepath{clip}%
\pgfsetroundcap%
\pgfsetroundjoin%
\pgfsetlinewidth{1.505625pt}%
\definecolor{currentstroke}{rgb}{0.737255,0.741176,0.133333}%
\pgfsetstrokecolor{currentstroke}%
\pgfsetdash{}{0pt}%
\pgfpathmoveto{\pgfqpoint{4.123134in}{0.504323in}}%
\pgfpathlineto{\pgfqpoint{6.706467in}{0.504323in}}%
\pgfusepath{stroke}%
\end{pgfscope}%
\begin{pgfscope}%
\pgfpathrectangle{\pgfqpoint{4.123134in}{0.331635in}}{\pgfqpoint{2.583333in}{1.510000in}}%
\pgfusepath{clip}%
\pgfsetbuttcap%
\pgfsetroundjoin%
\definecolor{currentfill}{rgb}{0.737255,0.741176,0.133333}%
\pgfsetfillcolor{currentfill}%
\pgfsetlinewidth{1.003750pt}%
\definecolor{currentstroke}{rgb}{0.737255,0.741176,0.133333}%
\pgfsetstrokecolor{currentstroke}%
\pgfsetdash{}{0pt}%
\pgfsys@defobject{currentmarker}{\pgfqpoint{-0.034722in}{-0.034722in}}{\pgfqpoint{0.034722in}{0.034722in}}{%
\pgfpathmoveto{\pgfqpoint{0.000000in}{-0.034722in}}%
\pgfpathcurveto{\pgfqpoint{0.009208in}{-0.034722in}}{\pgfqpoint{0.018041in}{-0.031064in}}{\pgfqpoint{0.024552in}{-0.024552in}}%
\pgfpathcurveto{\pgfqpoint{0.031064in}{-0.018041in}}{\pgfqpoint{0.034722in}{-0.009208in}}{\pgfqpoint{0.034722in}{0.000000in}}%
\pgfpathcurveto{\pgfqpoint{0.034722in}{0.009208in}}{\pgfqpoint{0.031064in}{0.018041in}}{\pgfqpoint{0.024552in}{0.024552in}}%
\pgfpathcurveto{\pgfqpoint{0.018041in}{0.031064in}}{\pgfqpoint{0.009208in}{0.034722in}}{\pgfqpoint{0.000000in}{0.034722in}}%
\pgfpathcurveto{\pgfqpoint{-0.009208in}{0.034722in}}{\pgfqpoint{-0.018041in}{0.031064in}}{\pgfqpoint{-0.024552in}{0.024552in}}%
\pgfpathcurveto{\pgfqpoint{-0.031064in}{0.018041in}}{\pgfqpoint{-0.034722in}{0.009208in}}{\pgfqpoint{-0.034722in}{0.000000in}}%
\pgfpathcurveto{\pgfqpoint{-0.034722in}{-0.009208in}}{\pgfqpoint{-0.031064in}{-0.018041in}}{\pgfqpoint{-0.024552in}{-0.024552in}}%
\pgfpathcurveto{\pgfqpoint{-0.018041in}{-0.031064in}}{\pgfqpoint{-0.009208in}{-0.034722in}}{\pgfqpoint{0.000000in}{-0.034722in}}%
\pgfpathclose%
\pgfusepath{stroke,fill}%
}%
\begin{pgfscope}%
\pgfsys@transformshift{4.240558in}{1.772999in}%
\pgfsys@useobject{currentmarker}{}%
\end{pgfscope}%
\begin{pgfscope}%
\pgfsys@transformshift{4.355118in}{0.410710in}%
\pgfsys@useobject{currentmarker}{}%
\end{pgfscope}%
\begin{pgfscope}%
\pgfsys@transformshift{4.469678in}{0.448803in}%
\pgfsys@useobject{currentmarker}{}%
\end{pgfscope}%
\begin{pgfscope}%
\pgfsys@transformshift{4.584239in}{0.458732in}%
\pgfsys@useobject{currentmarker}{}%
\end{pgfscope}%
\begin{pgfscope}%
\pgfsys@transformshift{4.698799in}{0.430045in}%
\pgfsys@useobject{currentmarker}{}%
\end{pgfscope}%
\begin{pgfscope}%
\pgfsys@transformshift{4.813359in}{0.445966in}%
\pgfsys@useobject{currentmarker}{}%
\end{pgfscope}%
\begin{pgfscope}%
\pgfsys@transformshift{4.927919in}{0.503755in}%
\pgfsys@useobject{currentmarker}{}%
\end{pgfscope}%
\begin{pgfscope}%
\pgfsys@transformshift{5.042480in}{0.490101in}%
\pgfsys@useobject{currentmarker}{}%
\end{pgfscope}%
\begin{pgfscope}%
\pgfsys@transformshift{5.157040in}{0.580994in}%
\pgfsys@useobject{currentmarker}{}%
\end{pgfscope}%
\begin{pgfscope}%
\pgfsys@transformshift{5.271600in}{0.489705in}%
\pgfsys@useobject{currentmarker}{}%
\end{pgfscope}%
\begin{pgfscope}%
\pgfsys@transformshift{5.386160in}{0.547158in}%
\pgfsys@useobject{currentmarker}{}%
\end{pgfscope}%
\begin{pgfscope}%
\pgfsys@transformshift{5.500721in}{0.526902in}%
\pgfsys@useobject{currentmarker}{}%
\end{pgfscope}%
\begin{pgfscope}%
\pgfsys@transformshift{5.615281in}{0.508693in}%
\pgfsys@useobject{currentmarker}{}%
\end{pgfscope}%
\begin{pgfscope}%
\pgfsys@transformshift{5.729841in}{0.503358in}%
\pgfsys@useobject{currentmarker}{}%
\end{pgfscope}%
\begin{pgfscope}%
\pgfsys@transformshift{5.844401in}{0.495101in}%
\pgfsys@useobject{currentmarker}{}%
\end{pgfscope}%
\begin{pgfscope}%
\pgfsys@transformshift{5.958962in}{0.400271in}%
\pgfsys@useobject{currentmarker}{}%
\end{pgfscope}%
\begin{pgfscope}%
\pgfsys@transformshift{6.073522in}{0.479252in}%
\pgfsys@useobject{currentmarker}{}%
\end{pgfscope}%
\begin{pgfscope}%
\pgfsys@transformshift{6.188082in}{0.508120in}%
\pgfsys@useobject{currentmarker}{}%
\end{pgfscope}%
\begin{pgfscope}%
\pgfsys@transformshift{6.302642in}{0.472725in}%
\pgfsys@useobject{currentmarker}{}%
\end{pgfscope}%
\begin{pgfscope}%
\pgfsys@transformshift{6.417202in}{0.502799in}%
\pgfsys@useobject{currentmarker}{}%
\end{pgfscope}%
\begin{pgfscope}%
\pgfsys@transformshift{6.531763in}{0.530890in}%
\pgfsys@useobject{currentmarker}{}%
\end{pgfscope}%
\end{pgfscope}%
\begin{pgfscope}%
\pgfsetrectcap%
\pgfsetmiterjoin%
\pgfsetlinewidth{0.803000pt}%
\definecolor{currentstroke}{rgb}{1.000000,1.000000,1.000000}%
\pgfsetstrokecolor{currentstroke}%
\pgfsetdash{}{0pt}%
\pgfpathmoveto{\pgfqpoint{4.123134in}{0.331635in}}%
\pgfpathlineto{\pgfqpoint{4.123134in}{1.841635in}}%
\pgfusepath{stroke}%
\end{pgfscope}%
\begin{pgfscope}%
\pgfsetrectcap%
\pgfsetmiterjoin%
\pgfsetlinewidth{0.803000pt}%
\definecolor{currentstroke}{rgb}{1.000000,1.000000,1.000000}%
\pgfsetstrokecolor{currentstroke}%
\pgfsetdash{}{0pt}%
\pgfpathmoveto{\pgfqpoint{6.706467in}{0.331635in}}%
\pgfpathlineto{\pgfqpoint{6.706467in}{1.841635in}}%
\pgfusepath{stroke}%
\end{pgfscope}%
\begin{pgfscope}%
\pgfsetrectcap%
\pgfsetmiterjoin%
\pgfsetlinewidth{0.803000pt}%
\definecolor{currentstroke}{rgb}{1.000000,1.000000,1.000000}%
\pgfsetstrokecolor{currentstroke}%
\pgfsetdash{}{0pt}%
\pgfpathmoveto{\pgfqpoint{4.123134in}{0.331635in}}%
\pgfpathlineto{\pgfqpoint{6.706467in}{0.331635in}}%
\pgfusepath{stroke}%
\end{pgfscope}%
\begin{pgfscope}%
\pgfsetrectcap%
\pgfsetmiterjoin%
\pgfsetlinewidth{0.803000pt}%
\definecolor{currentstroke}{rgb}{1.000000,1.000000,1.000000}%
\pgfsetstrokecolor{currentstroke}%
\pgfsetdash{}{0pt}%
\pgfpathmoveto{\pgfqpoint{4.123134in}{1.841635in}}%
\pgfpathlineto{\pgfqpoint{6.706467in}{1.841635in}}%
\pgfusepath{stroke}%
\end{pgfscope}%
\begin{pgfscope}%
\definecolor{textcolor}{rgb}{0.150000,0.150000,0.150000}%
\pgfsetstrokecolor{textcolor}%
\pgfsetfillcolor{textcolor}%
\pgftext[x=5.414800in,y=1.924968in,,base]{\color{textcolor}\rmfamily\fontsize{12.000000}{14.400000}\selectfont Partial Autocorrelation}%
\end{pgfscope}%
\end{pgfpicture}%
\makeatother%
\endgroup%

    \end{adjustbox} 
    
    \caption{}
    \label{fig:INTC_V_ACF_log_returns}
\end{figure}{}

\begin{figure}[h]
    \centering
    \figuretitle{Ljung-Box and Box-Pierce Test for Autocorrelation}
    \begin{adjustbox}{width=.95\textwidth,center}
    %% Creator: Matplotlib, PGF backend
%%
%% To include the figure in your LaTeX document, write
%%   \input{<filename>.pgf}
%%
%% Make sure the required packages are loaded in your preamble
%%   \usepackage{pgf}
%%
%% Figures using additional raster images can only be included by \input if
%% they are in the same directory as the main LaTeX file. For loading figures
%% from other directories you can use the `import` package
%%   \usepackage{import}
%% and then include the figures with
%%   \import{<path to file>}{<filename>.pgf}
%%
%% Matplotlib used the following preamble
%%   \usepackage{fontspec}
%%   \setmainfont{DejaVuSerif.ttf}[Path=/opt/tljh/user/lib/python3.6/site-packages/matplotlib/mpl-data/fonts/ttf/]
%%   \setsansfont{DejaVuSans.ttf}[Path=/opt/tljh/user/lib/python3.6/site-packages/matplotlib/mpl-data/fonts/ttf/]
%%   \setmonofont{DejaVuSansMono.ttf}[Path=/opt/tljh/user/lib/python3.6/site-packages/matplotlib/mpl-data/fonts/ttf/]
%%
\begingroup%
\makeatletter%
\begin{pgfpicture}%
\pgfpathrectangle{\pgfpointorigin}{\pgfqpoint{5.446435in}{3.684685in}}%
\pgfusepath{use as bounding box, clip}%
\begin{pgfscope}%
\pgfsetbuttcap%
\pgfsetmiterjoin%
\definecolor{currentfill}{rgb}{1.000000,1.000000,1.000000}%
\pgfsetfillcolor{currentfill}%
\pgfsetlinewidth{0.000000pt}%
\definecolor{currentstroke}{rgb}{1.000000,1.000000,1.000000}%
\pgfsetstrokecolor{currentstroke}%
\pgfsetdash{}{0pt}%
\pgfpathmoveto{\pgfqpoint{0.000000in}{0.000000in}}%
\pgfpathlineto{\pgfqpoint{5.446435in}{0.000000in}}%
\pgfpathlineto{\pgfqpoint{5.446435in}{3.684685in}}%
\pgfpathlineto{\pgfqpoint{0.000000in}{3.684685in}}%
\pgfpathclose%
\pgfusepath{fill}%
\end{pgfscope}%
\begin{pgfscope}%
\pgfsetbuttcap%
\pgfsetmiterjoin%
\definecolor{currentfill}{rgb}{0.917647,0.917647,0.949020}%
\pgfsetfillcolor{currentfill}%
\pgfsetlinewidth{0.000000pt}%
\definecolor{currentstroke}{rgb}{0.000000,0.000000,0.000000}%
\pgfsetstrokecolor{currentstroke}%
\pgfsetstrokeopacity{0.000000}%
\pgfsetdash{}{0pt}%
\pgfpathmoveto{\pgfqpoint{0.696435in}{0.523570in}}%
\pgfpathlineto{\pgfqpoint{5.346435in}{0.523570in}}%
\pgfpathlineto{\pgfqpoint{5.346435in}{3.543570in}}%
\pgfpathlineto{\pgfqpoint{0.696435in}{3.543570in}}%
\pgfpathclose%
\pgfusepath{fill}%
\end{pgfscope}%
\begin{pgfscope}%
\pgfpathrectangle{\pgfqpoint{0.696435in}{0.523570in}}{\pgfqpoint{4.650000in}{3.020000in}}%
\pgfusepath{clip}%
\pgfsetroundcap%
\pgfsetroundjoin%
\pgfsetlinewidth{0.803000pt}%
\definecolor{currentstroke}{rgb}{1.000000,1.000000,1.000000}%
\pgfsetstrokecolor{currentstroke}%
\pgfsetdash{}{0pt}%
\pgfpathmoveto{\pgfqpoint{0.907799in}{0.523570in}}%
\pgfpathlineto{\pgfqpoint{0.907799in}{3.543570in}}%
\pgfusepath{stroke}%
\end{pgfscope}%
\begin{pgfscope}%
\definecolor{textcolor}{rgb}{0.150000,0.150000,0.150000}%
\pgfsetstrokecolor{textcolor}%
\pgfsetfillcolor{textcolor}%
\pgftext[x=0.907799in,y=0.426348in,,top]{\color{textcolor}\rmfamily\fontsize{10.000000}{12.000000}\selectfont 0}%
\end{pgfscope}%
\begin{pgfscope}%
\pgfpathrectangle{\pgfqpoint{0.696435in}{0.523570in}}{\pgfqpoint{4.650000in}{3.020000in}}%
\pgfusepath{clip}%
\pgfsetroundcap%
\pgfsetroundjoin%
\pgfsetlinewidth{0.803000pt}%
\definecolor{currentstroke}{rgb}{1.000000,1.000000,1.000000}%
\pgfsetstrokecolor{currentstroke}%
\pgfsetdash{}{0pt}%
\pgfpathmoveto{\pgfqpoint{1.449757in}{0.523570in}}%
\pgfpathlineto{\pgfqpoint{1.449757in}{3.543570in}}%
\pgfusepath{stroke}%
\end{pgfscope}%
\begin{pgfscope}%
\definecolor{textcolor}{rgb}{0.150000,0.150000,0.150000}%
\pgfsetstrokecolor{textcolor}%
\pgfsetfillcolor{textcolor}%
\pgftext[x=1.449757in,y=0.426348in,,top]{\color{textcolor}\rmfamily\fontsize{10.000000}{12.000000}\selectfont 5}%
\end{pgfscope}%
\begin{pgfscope}%
\pgfpathrectangle{\pgfqpoint{0.696435in}{0.523570in}}{\pgfqpoint{4.650000in}{3.020000in}}%
\pgfusepath{clip}%
\pgfsetroundcap%
\pgfsetroundjoin%
\pgfsetlinewidth{0.803000pt}%
\definecolor{currentstroke}{rgb}{1.000000,1.000000,1.000000}%
\pgfsetstrokecolor{currentstroke}%
\pgfsetdash{}{0pt}%
\pgfpathmoveto{\pgfqpoint{1.991715in}{0.523570in}}%
\pgfpathlineto{\pgfqpoint{1.991715in}{3.543570in}}%
\pgfusepath{stroke}%
\end{pgfscope}%
\begin{pgfscope}%
\definecolor{textcolor}{rgb}{0.150000,0.150000,0.150000}%
\pgfsetstrokecolor{textcolor}%
\pgfsetfillcolor{textcolor}%
\pgftext[x=1.991715in,y=0.426348in,,top]{\color{textcolor}\rmfamily\fontsize{10.000000}{12.000000}\selectfont 10}%
\end{pgfscope}%
\begin{pgfscope}%
\pgfpathrectangle{\pgfqpoint{0.696435in}{0.523570in}}{\pgfqpoint{4.650000in}{3.020000in}}%
\pgfusepath{clip}%
\pgfsetroundcap%
\pgfsetroundjoin%
\pgfsetlinewidth{0.803000pt}%
\definecolor{currentstroke}{rgb}{1.000000,1.000000,1.000000}%
\pgfsetstrokecolor{currentstroke}%
\pgfsetdash{}{0pt}%
\pgfpathmoveto{\pgfqpoint{2.533673in}{0.523570in}}%
\pgfpathlineto{\pgfqpoint{2.533673in}{3.543570in}}%
\pgfusepath{stroke}%
\end{pgfscope}%
\begin{pgfscope}%
\definecolor{textcolor}{rgb}{0.150000,0.150000,0.150000}%
\pgfsetstrokecolor{textcolor}%
\pgfsetfillcolor{textcolor}%
\pgftext[x=2.533673in,y=0.426348in,,top]{\color{textcolor}\rmfamily\fontsize{10.000000}{12.000000}\selectfont 15}%
\end{pgfscope}%
\begin{pgfscope}%
\pgfpathrectangle{\pgfqpoint{0.696435in}{0.523570in}}{\pgfqpoint{4.650000in}{3.020000in}}%
\pgfusepath{clip}%
\pgfsetroundcap%
\pgfsetroundjoin%
\pgfsetlinewidth{0.803000pt}%
\definecolor{currentstroke}{rgb}{1.000000,1.000000,1.000000}%
\pgfsetstrokecolor{currentstroke}%
\pgfsetdash{}{0pt}%
\pgfpathmoveto{\pgfqpoint{3.075631in}{0.523570in}}%
\pgfpathlineto{\pgfqpoint{3.075631in}{3.543570in}}%
\pgfusepath{stroke}%
\end{pgfscope}%
\begin{pgfscope}%
\definecolor{textcolor}{rgb}{0.150000,0.150000,0.150000}%
\pgfsetstrokecolor{textcolor}%
\pgfsetfillcolor{textcolor}%
\pgftext[x=3.075631in,y=0.426348in,,top]{\color{textcolor}\rmfamily\fontsize{10.000000}{12.000000}\selectfont 20}%
\end{pgfscope}%
\begin{pgfscope}%
\pgfpathrectangle{\pgfqpoint{0.696435in}{0.523570in}}{\pgfqpoint{4.650000in}{3.020000in}}%
\pgfusepath{clip}%
\pgfsetroundcap%
\pgfsetroundjoin%
\pgfsetlinewidth{0.803000pt}%
\definecolor{currentstroke}{rgb}{1.000000,1.000000,1.000000}%
\pgfsetstrokecolor{currentstroke}%
\pgfsetdash{}{0pt}%
\pgfpathmoveto{\pgfqpoint{3.617589in}{0.523570in}}%
\pgfpathlineto{\pgfqpoint{3.617589in}{3.543570in}}%
\pgfusepath{stroke}%
\end{pgfscope}%
\begin{pgfscope}%
\definecolor{textcolor}{rgb}{0.150000,0.150000,0.150000}%
\pgfsetstrokecolor{textcolor}%
\pgfsetfillcolor{textcolor}%
\pgftext[x=3.617589in,y=0.426348in,,top]{\color{textcolor}\rmfamily\fontsize{10.000000}{12.000000}\selectfont 25}%
\end{pgfscope}%
\begin{pgfscope}%
\pgfpathrectangle{\pgfqpoint{0.696435in}{0.523570in}}{\pgfqpoint{4.650000in}{3.020000in}}%
\pgfusepath{clip}%
\pgfsetroundcap%
\pgfsetroundjoin%
\pgfsetlinewidth{0.803000pt}%
\definecolor{currentstroke}{rgb}{1.000000,1.000000,1.000000}%
\pgfsetstrokecolor{currentstroke}%
\pgfsetdash{}{0pt}%
\pgfpathmoveto{\pgfqpoint{4.159547in}{0.523570in}}%
\pgfpathlineto{\pgfqpoint{4.159547in}{3.543570in}}%
\pgfusepath{stroke}%
\end{pgfscope}%
\begin{pgfscope}%
\definecolor{textcolor}{rgb}{0.150000,0.150000,0.150000}%
\pgfsetstrokecolor{textcolor}%
\pgfsetfillcolor{textcolor}%
\pgftext[x=4.159547in,y=0.426348in,,top]{\color{textcolor}\rmfamily\fontsize{10.000000}{12.000000}\selectfont 30}%
\end{pgfscope}%
\begin{pgfscope}%
\pgfpathrectangle{\pgfqpoint{0.696435in}{0.523570in}}{\pgfqpoint{4.650000in}{3.020000in}}%
\pgfusepath{clip}%
\pgfsetroundcap%
\pgfsetroundjoin%
\pgfsetlinewidth{0.803000pt}%
\definecolor{currentstroke}{rgb}{1.000000,1.000000,1.000000}%
\pgfsetstrokecolor{currentstroke}%
\pgfsetdash{}{0pt}%
\pgfpathmoveto{\pgfqpoint{4.701505in}{0.523570in}}%
\pgfpathlineto{\pgfqpoint{4.701505in}{3.543570in}}%
\pgfusepath{stroke}%
\end{pgfscope}%
\begin{pgfscope}%
\definecolor{textcolor}{rgb}{0.150000,0.150000,0.150000}%
\pgfsetstrokecolor{textcolor}%
\pgfsetfillcolor{textcolor}%
\pgftext[x=4.701505in,y=0.426348in,,top]{\color{textcolor}\rmfamily\fontsize{10.000000}{12.000000}\selectfont 35}%
\end{pgfscope}%
\begin{pgfscope}%
\pgfpathrectangle{\pgfqpoint{0.696435in}{0.523570in}}{\pgfqpoint{4.650000in}{3.020000in}}%
\pgfusepath{clip}%
\pgfsetroundcap%
\pgfsetroundjoin%
\pgfsetlinewidth{0.803000pt}%
\definecolor{currentstroke}{rgb}{1.000000,1.000000,1.000000}%
\pgfsetstrokecolor{currentstroke}%
\pgfsetdash{}{0pt}%
\pgfpathmoveto{\pgfqpoint{5.243463in}{0.523570in}}%
\pgfpathlineto{\pgfqpoint{5.243463in}{3.543570in}}%
\pgfusepath{stroke}%
\end{pgfscope}%
\begin{pgfscope}%
\definecolor{textcolor}{rgb}{0.150000,0.150000,0.150000}%
\pgfsetstrokecolor{textcolor}%
\pgfsetfillcolor{textcolor}%
\pgftext[x=5.243463in,y=0.426348in,,top]{\color{textcolor}\rmfamily\fontsize{10.000000}{12.000000}\selectfont 40}%
\end{pgfscope}%
\begin{pgfscope}%
\definecolor{textcolor}{rgb}{0.150000,0.150000,0.150000}%
\pgfsetstrokecolor{textcolor}%
\pgfsetfillcolor{textcolor}%
\pgftext[x=3.021435in,y=0.236379in,,top]{\color{textcolor}\rmfamily\fontsize{10.000000}{12.000000}\selectfont lag}%
\end{pgfscope}%
\begin{pgfscope}%
\pgfpathrectangle{\pgfqpoint{0.696435in}{0.523570in}}{\pgfqpoint{4.650000in}{3.020000in}}%
\pgfusepath{clip}%
\pgfsetroundcap%
\pgfsetroundjoin%
\pgfsetlinewidth{0.803000pt}%
\definecolor{currentstroke}{rgb}{1.000000,1.000000,1.000000}%
\pgfsetstrokecolor{currentstroke}%
\pgfsetdash{}{0pt}%
\pgfpathmoveto{\pgfqpoint{0.696435in}{0.605763in}}%
\pgfpathlineto{\pgfqpoint{5.346435in}{0.605763in}}%
\pgfusepath{stroke}%
\end{pgfscope}%
\begin{pgfscope}%
\definecolor{textcolor}{rgb}{0.150000,0.150000,0.150000}%
\pgfsetstrokecolor{textcolor}%
\pgfsetfillcolor{textcolor}%
\pgftext[x=0.289968in,y=0.553002in,left,base]{\color{textcolor}\rmfamily\fontsize{10.000000}{12.000000}\selectfont 0.00}%
\end{pgfscope}%
\begin{pgfscope}%
\pgfpathrectangle{\pgfqpoint{0.696435in}{0.523570in}}{\pgfqpoint{4.650000in}{3.020000in}}%
\pgfusepath{clip}%
\pgfsetroundcap%
\pgfsetroundjoin%
\pgfsetlinewidth{0.803000pt}%
\definecolor{currentstroke}{rgb}{1.000000,1.000000,1.000000}%
\pgfsetstrokecolor{currentstroke}%
\pgfsetdash{}{0pt}%
\pgfpathmoveto{\pgfqpoint{0.696435in}{1.023786in}}%
\pgfpathlineto{\pgfqpoint{5.346435in}{1.023786in}}%
\pgfusepath{stroke}%
\end{pgfscope}%
\begin{pgfscope}%
\definecolor{textcolor}{rgb}{0.150000,0.150000,0.150000}%
\pgfsetstrokecolor{textcolor}%
\pgfsetfillcolor{textcolor}%
\pgftext[x=0.289968in,y=0.971025in,left,base]{\color{textcolor}\rmfamily\fontsize{10.000000}{12.000000}\selectfont 0.01}%
\end{pgfscope}%
\begin{pgfscope}%
\pgfpathrectangle{\pgfqpoint{0.696435in}{0.523570in}}{\pgfqpoint{4.650000in}{3.020000in}}%
\pgfusepath{clip}%
\pgfsetroundcap%
\pgfsetroundjoin%
\pgfsetlinewidth{0.803000pt}%
\definecolor{currentstroke}{rgb}{1.000000,1.000000,1.000000}%
\pgfsetstrokecolor{currentstroke}%
\pgfsetdash{}{0pt}%
\pgfpathmoveto{\pgfqpoint{0.696435in}{1.441809in}}%
\pgfpathlineto{\pgfqpoint{5.346435in}{1.441809in}}%
\pgfusepath{stroke}%
\end{pgfscope}%
\begin{pgfscope}%
\definecolor{textcolor}{rgb}{0.150000,0.150000,0.150000}%
\pgfsetstrokecolor{textcolor}%
\pgfsetfillcolor{textcolor}%
\pgftext[x=0.289968in,y=1.389048in,left,base]{\color{textcolor}\rmfamily\fontsize{10.000000}{12.000000}\selectfont 0.02}%
\end{pgfscope}%
\begin{pgfscope}%
\pgfpathrectangle{\pgfqpoint{0.696435in}{0.523570in}}{\pgfqpoint{4.650000in}{3.020000in}}%
\pgfusepath{clip}%
\pgfsetroundcap%
\pgfsetroundjoin%
\pgfsetlinewidth{0.803000pt}%
\definecolor{currentstroke}{rgb}{1.000000,1.000000,1.000000}%
\pgfsetstrokecolor{currentstroke}%
\pgfsetdash{}{0pt}%
\pgfpathmoveto{\pgfqpoint{0.696435in}{1.859832in}}%
\pgfpathlineto{\pgfqpoint{5.346435in}{1.859832in}}%
\pgfusepath{stroke}%
\end{pgfscope}%
\begin{pgfscope}%
\definecolor{textcolor}{rgb}{0.150000,0.150000,0.150000}%
\pgfsetstrokecolor{textcolor}%
\pgfsetfillcolor{textcolor}%
\pgftext[x=0.289968in,y=1.807070in,left,base]{\color{textcolor}\rmfamily\fontsize{10.000000}{12.000000}\selectfont 0.03}%
\end{pgfscope}%
\begin{pgfscope}%
\pgfpathrectangle{\pgfqpoint{0.696435in}{0.523570in}}{\pgfqpoint{4.650000in}{3.020000in}}%
\pgfusepath{clip}%
\pgfsetroundcap%
\pgfsetroundjoin%
\pgfsetlinewidth{0.803000pt}%
\definecolor{currentstroke}{rgb}{1.000000,1.000000,1.000000}%
\pgfsetstrokecolor{currentstroke}%
\pgfsetdash{}{0pt}%
\pgfpathmoveto{\pgfqpoint{0.696435in}{2.277855in}}%
\pgfpathlineto{\pgfqpoint{5.346435in}{2.277855in}}%
\pgfusepath{stroke}%
\end{pgfscope}%
\begin{pgfscope}%
\definecolor{textcolor}{rgb}{0.150000,0.150000,0.150000}%
\pgfsetstrokecolor{textcolor}%
\pgfsetfillcolor{textcolor}%
\pgftext[x=0.289968in,y=2.225093in,left,base]{\color{textcolor}\rmfamily\fontsize{10.000000}{12.000000}\selectfont 0.04}%
\end{pgfscope}%
\begin{pgfscope}%
\pgfpathrectangle{\pgfqpoint{0.696435in}{0.523570in}}{\pgfqpoint{4.650000in}{3.020000in}}%
\pgfusepath{clip}%
\pgfsetroundcap%
\pgfsetroundjoin%
\pgfsetlinewidth{0.803000pt}%
\definecolor{currentstroke}{rgb}{1.000000,1.000000,1.000000}%
\pgfsetstrokecolor{currentstroke}%
\pgfsetdash{}{0pt}%
\pgfpathmoveto{\pgfqpoint{0.696435in}{2.695878in}}%
\pgfpathlineto{\pgfqpoint{5.346435in}{2.695878in}}%
\pgfusepath{stroke}%
\end{pgfscope}%
\begin{pgfscope}%
\definecolor{textcolor}{rgb}{0.150000,0.150000,0.150000}%
\pgfsetstrokecolor{textcolor}%
\pgfsetfillcolor{textcolor}%
\pgftext[x=0.289968in,y=2.643116in,left,base]{\color{textcolor}\rmfamily\fontsize{10.000000}{12.000000}\selectfont 0.05}%
\end{pgfscope}%
\begin{pgfscope}%
\pgfpathrectangle{\pgfqpoint{0.696435in}{0.523570in}}{\pgfqpoint{4.650000in}{3.020000in}}%
\pgfusepath{clip}%
\pgfsetroundcap%
\pgfsetroundjoin%
\pgfsetlinewidth{0.803000pt}%
\definecolor{currentstroke}{rgb}{1.000000,1.000000,1.000000}%
\pgfsetstrokecolor{currentstroke}%
\pgfsetdash{}{0pt}%
\pgfpathmoveto{\pgfqpoint{0.696435in}{3.113900in}}%
\pgfpathlineto{\pgfqpoint{5.346435in}{3.113900in}}%
\pgfusepath{stroke}%
\end{pgfscope}%
\begin{pgfscope}%
\definecolor{textcolor}{rgb}{0.150000,0.150000,0.150000}%
\pgfsetstrokecolor{textcolor}%
\pgfsetfillcolor{textcolor}%
\pgftext[x=0.289968in,y=3.061139in,left,base]{\color{textcolor}\rmfamily\fontsize{10.000000}{12.000000}\selectfont 0.06}%
\end{pgfscope}%
\begin{pgfscope}%
\pgfpathrectangle{\pgfqpoint{0.696435in}{0.523570in}}{\pgfqpoint{4.650000in}{3.020000in}}%
\pgfusepath{clip}%
\pgfsetroundcap%
\pgfsetroundjoin%
\pgfsetlinewidth{0.803000pt}%
\definecolor{currentstroke}{rgb}{1.000000,1.000000,1.000000}%
\pgfsetstrokecolor{currentstroke}%
\pgfsetdash{}{0pt}%
\pgfpathmoveto{\pgfqpoint{0.696435in}{3.531923in}}%
\pgfpathlineto{\pgfqpoint{5.346435in}{3.531923in}}%
\pgfusepath{stroke}%
\end{pgfscope}%
\begin{pgfscope}%
\definecolor{textcolor}{rgb}{0.150000,0.150000,0.150000}%
\pgfsetstrokecolor{textcolor}%
\pgfsetfillcolor{textcolor}%
\pgftext[x=0.289968in,y=3.479162in,left,base]{\color{textcolor}\rmfamily\fontsize{10.000000}{12.000000}\selectfont 0.07}%
\end{pgfscope}%
\begin{pgfscope}%
\definecolor{textcolor}{rgb}{0.150000,0.150000,0.150000}%
\pgfsetstrokecolor{textcolor}%
\pgfsetfillcolor{textcolor}%
\pgftext[x=0.234413in,y=2.033570in,,bottom,rotate=90.000000]{\color{textcolor}\rmfamily\fontsize{10.000000}{12.000000}\selectfont p-value}%
\end{pgfscope}%
\begin{pgfscope}%
\pgfpathrectangle{\pgfqpoint{0.696435in}{0.523570in}}{\pgfqpoint{4.650000in}{3.020000in}}%
\pgfusepath{clip}%
\pgfsetroundcap%
\pgfsetroundjoin%
\pgfsetlinewidth{1.505625pt}%
\definecolor{currentstroke}{rgb}{0.121569,0.466667,0.705882}%
\pgfsetstrokecolor{currentstroke}%
\pgfsetdash{}{0pt}%
\pgfpathmoveto{\pgfqpoint{0.907799in}{0.780583in}}%
\pgfpathlineto{\pgfqpoint{1.016191in}{0.837014in}}%
\pgfpathlineto{\pgfqpoint{1.124582in}{0.960010in}}%
\pgfpathlineto{\pgfqpoint{1.232974in}{0.752383in}}%
\pgfpathlineto{\pgfqpoint{1.341365in}{0.764556in}}%
\pgfpathlineto{\pgfqpoint{1.449757in}{0.908977in}}%
\pgfpathlineto{\pgfqpoint{1.558149in}{1.175991in}}%
\pgfpathlineto{\pgfqpoint{1.666540in}{0.692372in}}%
\pgfpathlineto{\pgfqpoint{1.774932in}{0.744485in}}%
\pgfpathlineto{\pgfqpoint{1.883324in}{0.755218in}}%
\pgfpathlineto{\pgfqpoint{1.991715in}{0.854582in}}%
\pgfpathlineto{\pgfqpoint{2.100107in}{1.008127in}}%
\pgfpathlineto{\pgfqpoint{2.208498in}{1.230096in}}%
\pgfpathlineto{\pgfqpoint{2.316890in}{1.530868in}}%
\pgfpathlineto{\pgfqpoint{2.425282in}{0.660843in}}%
\pgfpathlineto{\pgfqpoint{2.533673in}{0.696887in}}%
\pgfpathlineto{\pgfqpoint{2.642065in}{0.741513in}}%
\pgfpathlineto{\pgfqpoint{2.750456in}{0.802085in}}%
\pgfpathlineto{\pgfqpoint{2.858848in}{0.885659in}}%
\pgfpathlineto{\pgfqpoint{2.967240in}{0.906492in}}%
\pgfpathlineto{\pgfqpoint{3.075631in}{1.036846in}}%
\pgfpathlineto{\pgfqpoint{3.184023in}{1.010567in}}%
\pgfpathlineto{\pgfqpoint{3.292414in}{1.183684in}}%
\pgfpathlineto{\pgfqpoint{3.400806in}{1.413353in}}%
\pgfpathlineto{\pgfqpoint{3.509198in}{1.711652in}}%
\pgfpathlineto{\pgfqpoint{3.617589in}{2.049959in}}%
\pgfpathlineto{\pgfqpoint{3.725981in}{2.480307in}}%
\pgfpathlineto{\pgfqpoint{3.834372in}{2.662037in}}%
\pgfpathlineto{\pgfqpoint{3.942764in}{3.239471in}}%
\pgfpathlineto{\pgfqpoint{4.051156in}{2.332387in}}%
\pgfpathlineto{\pgfqpoint{4.159547in}{2.643171in}}%
\pgfpathlineto{\pgfqpoint{4.267939in}{2.003165in}}%
\pgfpathlineto{\pgfqpoint{4.376331in}{1.132292in}}%
\pgfpathlineto{\pgfqpoint{4.484722in}{1.289658in}}%
\pgfpathlineto{\pgfqpoint{4.593114in}{1.383251in}}%
\pgfpathlineto{\pgfqpoint{4.701505in}{1.581711in}}%
\pgfpathlineto{\pgfqpoint{4.809897in}{1.823074in}}%
\pgfpathlineto{\pgfqpoint{4.918289in}{2.148997in}}%
\pgfpathlineto{\pgfqpoint{5.026680in}{1.826133in}}%
\pgfpathlineto{\pgfqpoint{5.135072in}{1.944291in}}%
\pgfusepath{stroke}%
\end{pgfscope}%
\begin{pgfscope}%
\pgfpathrectangle{\pgfqpoint{0.696435in}{0.523570in}}{\pgfqpoint{4.650000in}{3.020000in}}%
\pgfusepath{clip}%
\pgfsetroundcap%
\pgfsetroundjoin%
\pgfsetlinewidth{1.505625pt}%
\definecolor{currentstroke}{rgb}{1.000000,0.498039,0.054902}%
\pgfsetstrokecolor{currentstroke}%
\pgfsetdash{}{0pt}%
\pgfpathmoveto{\pgfqpoint{0.907799in}{0.782160in}}%
\pgfpathlineto{\pgfqpoint{1.016191in}{0.839583in}}%
\pgfpathlineto{\pgfqpoint{1.124582in}{0.964371in}}%
\pgfpathlineto{\pgfqpoint{1.232974in}{0.755148in}}%
\pgfpathlineto{\pgfqpoint{1.341365in}{0.767964in}}%
\pgfpathlineto{\pgfqpoint{1.449757in}{0.915254in}}%
\pgfpathlineto{\pgfqpoint{1.558149in}{1.187111in}}%
\pgfpathlineto{\pgfqpoint{1.666540in}{0.695625in}}%
\pgfpathlineto{\pgfqpoint{1.774932in}{0.749644in}}%
\pgfpathlineto{\pgfqpoint{1.883324in}{0.761242in}}%
\pgfpathlineto{\pgfqpoint{1.991715in}{0.864247in}}%
\pgfpathlineto{\pgfqpoint{2.100107in}{1.023107in}}%
\pgfpathlineto{\pgfqpoint{2.208498in}{1.252351in}}%
\pgfpathlineto{\pgfqpoint{2.316890in}{1.562519in}}%
\pgfpathlineto{\pgfqpoint{2.425282in}{0.665261in}}%
\pgfpathlineto{\pgfqpoint{2.533673in}{0.703920in}}%
\pgfpathlineto{\pgfqpoint{2.642065in}{0.751750in}}%
\pgfpathlineto{\pgfqpoint{2.750456in}{0.816567in}}%
\pgfpathlineto{\pgfqpoint{2.858848in}{0.905808in}}%
\pgfpathlineto{\pgfqpoint{2.967240in}{0.929215in}}%
\pgfpathlineto{\pgfqpoint{3.075631in}{1.068285in}}%
\pgfpathlineto{\pgfqpoint{3.184023in}{1.042716in}}%
\pgfpathlineto{\pgfqpoint{3.292414in}{1.227670in}}%
\pgfpathlineto{\pgfqpoint{3.400806in}{1.472180in}}%
\pgfpathlineto{\pgfqpoint{3.509198in}{1.788638in}}%
\pgfpathlineto{\pgfqpoint{3.617589in}{2.147004in}}%
\pgfpathlineto{\pgfqpoint{3.725981in}{2.601256in}}%
\pgfpathlineto{\pgfqpoint{3.834372in}{2.798612in}}%
\pgfpathlineto{\pgfqpoint{3.942764in}{3.406297in}}%
\pgfpathlineto{\pgfqpoint{4.051156in}{2.473569in}}%
\pgfpathlineto{\pgfqpoint{4.159547in}{2.808192in}}%
\pgfpathlineto{\pgfqpoint{4.267939in}{2.142305in}}%
\pgfpathlineto{\pgfqpoint{4.376331in}{1.207309in}}%
\pgfpathlineto{\pgfqpoint{4.484722in}{1.384126in}}%
\pgfpathlineto{\pgfqpoint{4.593114in}{1.491229in}}%
\pgfpathlineto{\pgfqpoint{4.701505in}{1.713600in}}%
\pgfpathlineto{\pgfqpoint{4.809897in}{1.982842in}}%
\pgfpathlineto{\pgfqpoint{4.918289in}{2.343696in}}%
\pgfpathlineto{\pgfqpoint{5.026680in}{2.000818in}}%
\pgfpathlineto{\pgfqpoint{5.135072in}{2.137825in}}%
\pgfusepath{stroke}%
\end{pgfscope}%
\begin{pgfscope}%
\pgfsetrectcap%
\pgfsetmiterjoin%
\pgfsetlinewidth{0.803000pt}%
\definecolor{currentstroke}{rgb}{1.000000,1.000000,1.000000}%
\pgfsetstrokecolor{currentstroke}%
\pgfsetdash{}{0pt}%
\pgfpathmoveto{\pgfqpoint{0.696435in}{0.523570in}}%
\pgfpathlineto{\pgfqpoint{0.696435in}{3.543570in}}%
\pgfusepath{stroke}%
\end{pgfscope}%
\begin{pgfscope}%
\pgfsetrectcap%
\pgfsetmiterjoin%
\pgfsetlinewidth{0.803000pt}%
\definecolor{currentstroke}{rgb}{1.000000,1.000000,1.000000}%
\pgfsetstrokecolor{currentstroke}%
\pgfsetdash{}{0pt}%
\pgfpathmoveto{\pgfqpoint{5.346435in}{0.523570in}}%
\pgfpathlineto{\pgfqpoint{5.346435in}{3.543570in}}%
\pgfusepath{stroke}%
\end{pgfscope}%
\begin{pgfscope}%
\pgfsetrectcap%
\pgfsetmiterjoin%
\pgfsetlinewidth{0.803000pt}%
\definecolor{currentstroke}{rgb}{1.000000,1.000000,1.000000}%
\pgfsetstrokecolor{currentstroke}%
\pgfsetdash{}{0pt}%
\pgfpathmoveto{\pgfqpoint{0.696435in}{0.523570in}}%
\pgfpathlineto{\pgfqpoint{5.346435in}{0.523570in}}%
\pgfusepath{stroke}%
\end{pgfscope}%
\begin{pgfscope}%
\pgfsetrectcap%
\pgfsetmiterjoin%
\pgfsetlinewidth{0.803000pt}%
\definecolor{currentstroke}{rgb}{1.000000,1.000000,1.000000}%
\pgfsetstrokecolor{currentstroke}%
\pgfsetdash{}{0pt}%
\pgfpathmoveto{\pgfqpoint{0.696435in}{3.543570in}}%
\pgfpathlineto{\pgfqpoint{5.346435in}{3.543570in}}%
\pgfusepath{stroke}%
\end{pgfscope}%
\begin{pgfscope}%
\pgfsetbuttcap%
\pgfsetmiterjoin%
\definecolor{currentfill}{rgb}{0.917647,0.917647,0.949020}%
\pgfsetfillcolor{currentfill}%
\pgfsetfillopacity{0.800000}%
\pgfsetlinewidth{1.003750pt}%
\definecolor{currentstroke}{rgb}{0.800000,0.800000,0.800000}%
\pgfsetstrokecolor{currentstroke}%
\pgfsetstrokeopacity{0.800000}%
\pgfsetdash{}{0pt}%
\pgfpathmoveto{\pgfqpoint{4.049832in}{3.022778in}}%
\pgfpathlineto{\pgfqpoint{5.249213in}{3.022778in}}%
\pgfpathquadraticcurveto{\pgfqpoint{5.276991in}{3.022778in}}{\pgfqpoint{5.276991in}{3.050556in}}%
\pgfpathlineto{\pgfqpoint{5.276991in}{3.446348in}}%
\pgfpathquadraticcurveto{\pgfqpoint{5.276991in}{3.474126in}}{\pgfqpoint{5.249213in}{3.474126in}}%
\pgfpathlineto{\pgfqpoint{4.049832in}{3.474126in}}%
\pgfpathquadraticcurveto{\pgfqpoint{4.022054in}{3.474126in}}{\pgfqpoint{4.022054in}{3.446348in}}%
\pgfpathlineto{\pgfqpoint{4.022054in}{3.050556in}}%
\pgfpathquadraticcurveto{\pgfqpoint{4.022054in}{3.022778in}}{\pgfqpoint{4.049832in}{3.022778in}}%
\pgfpathclose%
\pgfusepath{stroke,fill}%
\end{pgfscope}%
\begin{pgfscope}%
\pgfsetroundcap%
\pgfsetroundjoin%
\pgfsetlinewidth{1.505625pt}%
\definecolor{currentstroke}{rgb}{0.121569,0.466667,0.705882}%
\pgfsetstrokecolor{currentstroke}%
\pgfsetdash{}{0pt}%
\pgfpathmoveto{\pgfqpoint{4.077609in}{3.361658in}}%
\pgfpathlineto{\pgfqpoint{4.355387in}{3.361658in}}%
\pgfusepath{stroke}%
\end{pgfscope}%
\begin{pgfscope}%
\definecolor{textcolor}{rgb}{0.150000,0.150000,0.150000}%
\pgfsetstrokecolor{textcolor}%
\pgfsetfillcolor{textcolor}%
\pgftext[x=4.466498in,y=3.313047in,left,base]{\color{textcolor}\rmfamily\fontsize{10.000000}{12.000000}\selectfont Ljung-Box}%
\end{pgfscope}%
\begin{pgfscope}%
\pgfsetroundcap%
\pgfsetroundjoin%
\pgfsetlinewidth{1.505625pt}%
\definecolor{currentstroke}{rgb}{1.000000,0.498039,0.054902}%
\pgfsetstrokecolor{currentstroke}%
\pgfsetdash{}{0pt}%
\pgfpathmoveto{\pgfqpoint{4.077609in}{3.155834in}}%
\pgfpathlineto{\pgfqpoint{4.355387in}{3.155834in}}%
\pgfusepath{stroke}%
\end{pgfscope}%
\begin{pgfscope}%
\definecolor{textcolor}{rgb}{0.150000,0.150000,0.150000}%
\pgfsetstrokecolor{textcolor}%
\pgfsetfillcolor{textcolor}%
\pgftext[x=4.466498in,y=3.107223in,left,base]{\color{textcolor}\rmfamily\fontsize{10.000000}{12.000000}\selectfont Box-Pierce}%
\end{pgfscope}%
\end{pgfpicture}%
\makeatother%
\endgroup%

    %% Creator: Matplotlib, PGF backend
%%
%% To include the figure in your LaTeX document, write
%%   \input{<filename>.pgf}
%%
%% Make sure the required packages are loaded in your preamble
%%   \usepackage{pgf}
%%
%% Figures using additional raster images can only be included by \input if
%% they are in the same directory as the main LaTeX file. For loading figures
%% from other directories you can use the `import` package
%%   \usepackage{import}
%% and then include the figures with
%%   \import{<path to file>}{<filename>.pgf}
%%
%% Matplotlib used the following preamble
%%   \usepackage{fontspec}
%%   \setmainfont{DejaVuSerif.ttf}[Path=/opt/tljh/user/lib/python3.6/site-packages/matplotlib/mpl-data/fonts/ttf/]
%%   \setsansfont{DejaVuSans.ttf}[Path=/opt/tljh/user/lib/python3.6/site-packages/matplotlib/mpl-data/fonts/ttf/]
%%   \setmonofont{DejaVuSansMono.ttf}[Path=/opt/tljh/user/lib/python3.6/site-packages/matplotlib/mpl-data/fonts/ttf/]
%%
\begingroup%
\makeatletter%
\begin{pgfpicture}%
\pgfpathrectangle{\pgfpointorigin}{\pgfqpoint{6.996435in}{2.133570in}}%
\pgfusepath{use as bounding box, clip}%
\begin{pgfscope}%
\pgfsetbuttcap%
\pgfsetmiterjoin%
\definecolor{currentfill}{rgb}{1.000000,1.000000,1.000000}%
\pgfsetfillcolor{currentfill}%
\pgfsetlinewidth{0.000000pt}%
\definecolor{currentstroke}{rgb}{1.000000,1.000000,1.000000}%
\pgfsetstrokecolor{currentstroke}%
\pgfsetdash{}{0pt}%
\pgfpathmoveto{\pgfqpoint{0.000000in}{0.000000in}}%
\pgfpathlineto{\pgfqpoint{6.996435in}{0.000000in}}%
\pgfpathlineto{\pgfqpoint{6.996435in}{2.133570in}}%
\pgfpathlineto{\pgfqpoint{0.000000in}{2.133570in}}%
\pgfpathclose%
\pgfusepath{fill}%
\end{pgfscope}%
\begin{pgfscope}%
\pgfsetbuttcap%
\pgfsetmiterjoin%
\definecolor{currentfill}{rgb}{0.917647,0.917647,0.949020}%
\pgfsetfillcolor{currentfill}%
\pgfsetlinewidth{0.000000pt}%
\definecolor{currentstroke}{rgb}{0.000000,0.000000,0.000000}%
\pgfsetstrokecolor{currentstroke}%
\pgfsetstrokeopacity{0.000000}%
\pgfsetdash{}{0pt}%
\pgfpathmoveto{\pgfqpoint{0.696435in}{0.523570in}}%
\pgfpathlineto{\pgfqpoint{6.896435in}{0.523570in}}%
\pgfpathlineto{\pgfqpoint{6.896435in}{2.033570in}}%
\pgfpathlineto{\pgfqpoint{0.696435in}{2.033570in}}%
\pgfpathclose%
\pgfusepath{fill}%
\end{pgfscope}%
\begin{pgfscope}%
\pgfpathrectangle{\pgfqpoint{0.696435in}{0.523570in}}{\pgfqpoint{6.200000in}{1.510000in}}%
\pgfusepath{clip}%
\pgfsetroundcap%
\pgfsetroundjoin%
\pgfsetlinewidth{0.803000pt}%
\definecolor{currentstroke}{rgb}{1.000000,1.000000,1.000000}%
\pgfsetstrokecolor{currentstroke}%
\pgfsetdash{}{0pt}%
\pgfpathmoveto{\pgfqpoint{0.978254in}{0.523570in}}%
\pgfpathlineto{\pgfqpoint{0.978254in}{2.033570in}}%
\pgfusepath{stroke}%
\end{pgfscope}%
\begin{pgfscope}%
\definecolor{textcolor}{rgb}{0.150000,0.150000,0.150000}%
\pgfsetstrokecolor{textcolor}%
\pgfsetfillcolor{textcolor}%
\pgftext[x=0.978254in,y=0.426348in,,top]{\color{textcolor}\rmfamily\fontsize{10.000000}{12.000000}\selectfont 0}%
\end{pgfscope}%
\begin{pgfscope}%
\pgfpathrectangle{\pgfqpoint{0.696435in}{0.523570in}}{\pgfqpoint{6.200000in}{1.510000in}}%
\pgfusepath{clip}%
\pgfsetroundcap%
\pgfsetroundjoin%
\pgfsetlinewidth{0.803000pt}%
\definecolor{currentstroke}{rgb}{1.000000,1.000000,1.000000}%
\pgfsetstrokecolor{currentstroke}%
\pgfsetdash{}{0pt}%
\pgfpathmoveto{\pgfqpoint{1.700864in}{0.523570in}}%
\pgfpathlineto{\pgfqpoint{1.700864in}{2.033570in}}%
\pgfusepath{stroke}%
\end{pgfscope}%
\begin{pgfscope}%
\definecolor{textcolor}{rgb}{0.150000,0.150000,0.150000}%
\pgfsetstrokecolor{textcolor}%
\pgfsetfillcolor{textcolor}%
\pgftext[x=1.700864in,y=0.426348in,,top]{\color{textcolor}\rmfamily\fontsize{10.000000}{12.000000}\selectfont 5}%
\end{pgfscope}%
\begin{pgfscope}%
\pgfpathrectangle{\pgfqpoint{0.696435in}{0.523570in}}{\pgfqpoint{6.200000in}{1.510000in}}%
\pgfusepath{clip}%
\pgfsetroundcap%
\pgfsetroundjoin%
\pgfsetlinewidth{0.803000pt}%
\definecolor{currentstroke}{rgb}{1.000000,1.000000,1.000000}%
\pgfsetstrokecolor{currentstroke}%
\pgfsetdash{}{0pt}%
\pgfpathmoveto{\pgfqpoint{2.423475in}{0.523570in}}%
\pgfpathlineto{\pgfqpoint{2.423475in}{2.033570in}}%
\pgfusepath{stroke}%
\end{pgfscope}%
\begin{pgfscope}%
\definecolor{textcolor}{rgb}{0.150000,0.150000,0.150000}%
\pgfsetstrokecolor{textcolor}%
\pgfsetfillcolor{textcolor}%
\pgftext[x=2.423475in,y=0.426348in,,top]{\color{textcolor}\rmfamily\fontsize{10.000000}{12.000000}\selectfont 10}%
\end{pgfscope}%
\begin{pgfscope}%
\pgfpathrectangle{\pgfqpoint{0.696435in}{0.523570in}}{\pgfqpoint{6.200000in}{1.510000in}}%
\pgfusepath{clip}%
\pgfsetroundcap%
\pgfsetroundjoin%
\pgfsetlinewidth{0.803000pt}%
\definecolor{currentstroke}{rgb}{1.000000,1.000000,1.000000}%
\pgfsetstrokecolor{currentstroke}%
\pgfsetdash{}{0pt}%
\pgfpathmoveto{\pgfqpoint{3.146086in}{0.523570in}}%
\pgfpathlineto{\pgfqpoint{3.146086in}{2.033570in}}%
\pgfusepath{stroke}%
\end{pgfscope}%
\begin{pgfscope}%
\definecolor{textcolor}{rgb}{0.150000,0.150000,0.150000}%
\pgfsetstrokecolor{textcolor}%
\pgfsetfillcolor{textcolor}%
\pgftext[x=3.146086in,y=0.426348in,,top]{\color{textcolor}\rmfamily\fontsize{10.000000}{12.000000}\selectfont 15}%
\end{pgfscope}%
\begin{pgfscope}%
\pgfpathrectangle{\pgfqpoint{0.696435in}{0.523570in}}{\pgfqpoint{6.200000in}{1.510000in}}%
\pgfusepath{clip}%
\pgfsetroundcap%
\pgfsetroundjoin%
\pgfsetlinewidth{0.803000pt}%
\definecolor{currentstroke}{rgb}{1.000000,1.000000,1.000000}%
\pgfsetstrokecolor{currentstroke}%
\pgfsetdash{}{0pt}%
\pgfpathmoveto{\pgfqpoint{3.868696in}{0.523570in}}%
\pgfpathlineto{\pgfqpoint{3.868696in}{2.033570in}}%
\pgfusepath{stroke}%
\end{pgfscope}%
\begin{pgfscope}%
\definecolor{textcolor}{rgb}{0.150000,0.150000,0.150000}%
\pgfsetstrokecolor{textcolor}%
\pgfsetfillcolor{textcolor}%
\pgftext[x=3.868696in,y=0.426348in,,top]{\color{textcolor}\rmfamily\fontsize{10.000000}{12.000000}\selectfont 20}%
\end{pgfscope}%
\begin{pgfscope}%
\pgfpathrectangle{\pgfqpoint{0.696435in}{0.523570in}}{\pgfqpoint{6.200000in}{1.510000in}}%
\pgfusepath{clip}%
\pgfsetroundcap%
\pgfsetroundjoin%
\pgfsetlinewidth{0.803000pt}%
\definecolor{currentstroke}{rgb}{1.000000,1.000000,1.000000}%
\pgfsetstrokecolor{currentstroke}%
\pgfsetdash{}{0pt}%
\pgfpathmoveto{\pgfqpoint{4.591307in}{0.523570in}}%
\pgfpathlineto{\pgfqpoint{4.591307in}{2.033570in}}%
\pgfusepath{stroke}%
\end{pgfscope}%
\begin{pgfscope}%
\definecolor{textcolor}{rgb}{0.150000,0.150000,0.150000}%
\pgfsetstrokecolor{textcolor}%
\pgfsetfillcolor{textcolor}%
\pgftext[x=4.591307in,y=0.426348in,,top]{\color{textcolor}\rmfamily\fontsize{10.000000}{12.000000}\selectfont 25}%
\end{pgfscope}%
\begin{pgfscope}%
\pgfpathrectangle{\pgfqpoint{0.696435in}{0.523570in}}{\pgfqpoint{6.200000in}{1.510000in}}%
\pgfusepath{clip}%
\pgfsetroundcap%
\pgfsetroundjoin%
\pgfsetlinewidth{0.803000pt}%
\definecolor{currentstroke}{rgb}{1.000000,1.000000,1.000000}%
\pgfsetstrokecolor{currentstroke}%
\pgfsetdash{}{0pt}%
\pgfpathmoveto{\pgfqpoint{5.313918in}{0.523570in}}%
\pgfpathlineto{\pgfqpoint{5.313918in}{2.033570in}}%
\pgfusepath{stroke}%
\end{pgfscope}%
\begin{pgfscope}%
\definecolor{textcolor}{rgb}{0.150000,0.150000,0.150000}%
\pgfsetstrokecolor{textcolor}%
\pgfsetfillcolor{textcolor}%
\pgftext[x=5.313918in,y=0.426348in,,top]{\color{textcolor}\rmfamily\fontsize{10.000000}{12.000000}\selectfont 30}%
\end{pgfscope}%
\begin{pgfscope}%
\pgfpathrectangle{\pgfqpoint{0.696435in}{0.523570in}}{\pgfqpoint{6.200000in}{1.510000in}}%
\pgfusepath{clip}%
\pgfsetroundcap%
\pgfsetroundjoin%
\pgfsetlinewidth{0.803000pt}%
\definecolor{currentstroke}{rgb}{1.000000,1.000000,1.000000}%
\pgfsetstrokecolor{currentstroke}%
\pgfsetdash{}{0pt}%
\pgfpathmoveto{\pgfqpoint{6.036529in}{0.523570in}}%
\pgfpathlineto{\pgfqpoint{6.036529in}{2.033570in}}%
\pgfusepath{stroke}%
\end{pgfscope}%
\begin{pgfscope}%
\definecolor{textcolor}{rgb}{0.150000,0.150000,0.150000}%
\pgfsetstrokecolor{textcolor}%
\pgfsetfillcolor{textcolor}%
\pgftext[x=6.036529in,y=0.426348in,,top]{\color{textcolor}\rmfamily\fontsize{10.000000}{12.000000}\selectfont 35}%
\end{pgfscope}%
\begin{pgfscope}%
\pgfpathrectangle{\pgfqpoint{0.696435in}{0.523570in}}{\pgfqpoint{6.200000in}{1.510000in}}%
\pgfusepath{clip}%
\pgfsetroundcap%
\pgfsetroundjoin%
\pgfsetlinewidth{0.803000pt}%
\definecolor{currentstroke}{rgb}{1.000000,1.000000,1.000000}%
\pgfsetstrokecolor{currentstroke}%
\pgfsetdash{}{0pt}%
\pgfpathmoveto{\pgfqpoint{6.759139in}{0.523570in}}%
\pgfpathlineto{\pgfqpoint{6.759139in}{2.033570in}}%
\pgfusepath{stroke}%
\end{pgfscope}%
\begin{pgfscope}%
\definecolor{textcolor}{rgb}{0.150000,0.150000,0.150000}%
\pgfsetstrokecolor{textcolor}%
\pgfsetfillcolor{textcolor}%
\pgftext[x=6.759139in,y=0.426348in,,top]{\color{textcolor}\rmfamily\fontsize{10.000000}{12.000000}\selectfont 40}%
\end{pgfscope}%
\begin{pgfscope}%
\definecolor{textcolor}{rgb}{0.150000,0.150000,0.150000}%
\pgfsetstrokecolor{textcolor}%
\pgfsetfillcolor{textcolor}%
\pgftext[x=3.796435in,y=0.236379in,,top]{\color{textcolor}\rmfamily\fontsize{10.000000}{12.000000}\selectfont lag}%
\end{pgfscope}%
\begin{pgfscope}%
\pgfpathrectangle{\pgfqpoint{0.696435in}{0.523570in}}{\pgfqpoint{6.200000in}{1.510000in}}%
\pgfusepath{clip}%
\pgfsetroundcap%
\pgfsetroundjoin%
\pgfsetlinewidth{0.803000pt}%
\definecolor{currentstroke}{rgb}{1.000000,1.000000,1.000000}%
\pgfsetstrokecolor{currentstroke}%
\pgfsetdash{}{0pt}%
\pgfpathmoveto{\pgfqpoint{0.696435in}{0.580837in}}%
\pgfpathlineto{\pgfqpoint{6.896435in}{0.580837in}}%
\pgfusepath{stroke}%
\end{pgfscope}%
\begin{pgfscope}%
\definecolor{textcolor}{rgb}{0.150000,0.150000,0.150000}%
\pgfsetstrokecolor{textcolor}%
\pgfsetfillcolor{textcolor}%
\pgftext[x=0.289968in,y=0.528075in,left,base]{\color{textcolor}\rmfamily\fontsize{10.000000}{12.000000}\selectfont 0.00}%
\end{pgfscope}%
\begin{pgfscope}%
\pgfpathrectangle{\pgfqpoint{0.696435in}{0.523570in}}{\pgfqpoint{6.200000in}{1.510000in}}%
\pgfusepath{clip}%
\pgfsetroundcap%
\pgfsetroundjoin%
\pgfsetlinewidth{0.803000pt}%
\definecolor{currentstroke}{rgb}{1.000000,1.000000,1.000000}%
\pgfsetstrokecolor{currentstroke}%
\pgfsetdash{}{0pt}%
\pgfpathmoveto{\pgfqpoint{0.696435in}{0.928165in}}%
\pgfpathlineto{\pgfqpoint{6.896435in}{0.928165in}}%
\pgfusepath{stroke}%
\end{pgfscope}%
\begin{pgfscope}%
\definecolor{textcolor}{rgb}{0.150000,0.150000,0.150000}%
\pgfsetstrokecolor{textcolor}%
\pgfsetfillcolor{textcolor}%
\pgftext[x=0.289968in,y=0.875404in,left,base]{\color{textcolor}\rmfamily\fontsize{10.000000}{12.000000}\selectfont 0.25}%
\end{pgfscope}%
\begin{pgfscope}%
\pgfpathrectangle{\pgfqpoint{0.696435in}{0.523570in}}{\pgfqpoint{6.200000in}{1.510000in}}%
\pgfusepath{clip}%
\pgfsetroundcap%
\pgfsetroundjoin%
\pgfsetlinewidth{0.803000pt}%
\definecolor{currentstroke}{rgb}{1.000000,1.000000,1.000000}%
\pgfsetstrokecolor{currentstroke}%
\pgfsetdash{}{0pt}%
\pgfpathmoveto{\pgfqpoint{0.696435in}{1.275494in}}%
\pgfpathlineto{\pgfqpoint{6.896435in}{1.275494in}}%
\pgfusepath{stroke}%
\end{pgfscope}%
\begin{pgfscope}%
\definecolor{textcolor}{rgb}{0.150000,0.150000,0.150000}%
\pgfsetstrokecolor{textcolor}%
\pgfsetfillcolor{textcolor}%
\pgftext[x=0.289968in,y=1.222732in,left,base]{\color{textcolor}\rmfamily\fontsize{10.000000}{12.000000}\selectfont 0.50}%
\end{pgfscope}%
\begin{pgfscope}%
\pgfpathrectangle{\pgfqpoint{0.696435in}{0.523570in}}{\pgfqpoint{6.200000in}{1.510000in}}%
\pgfusepath{clip}%
\pgfsetroundcap%
\pgfsetroundjoin%
\pgfsetlinewidth{0.803000pt}%
\definecolor{currentstroke}{rgb}{1.000000,1.000000,1.000000}%
\pgfsetstrokecolor{currentstroke}%
\pgfsetdash{}{0pt}%
\pgfpathmoveto{\pgfqpoint{0.696435in}{1.622822in}}%
\pgfpathlineto{\pgfqpoint{6.896435in}{1.622822in}}%
\pgfusepath{stroke}%
\end{pgfscope}%
\begin{pgfscope}%
\definecolor{textcolor}{rgb}{0.150000,0.150000,0.150000}%
\pgfsetstrokecolor{textcolor}%
\pgfsetfillcolor{textcolor}%
\pgftext[x=0.289968in,y=1.570061in,left,base]{\color{textcolor}\rmfamily\fontsize{10.000000}{12.000000}\selectfont 0.75}%
\end{pgfscope}%
\begin{pgfscope}%
\pgfpathrectangle{\pgfqpoint{0.696435in}{0.523570in}}{\pgfqpoint{6.200000in}{1.510000in}}%
\pgfusepath{clip}%
\pgfsetroundcap%
\pgfsetroundjoin%
\pgfsetlinewidth{0.803000pt}%
\definecolor{currentstroke}{rgb}{1.000000,1.000000,1.000000}%
\pgfsetstrokecolor{currentstroke}%
\pgfsetdash{}{0pt}%
\pgfpathmoveto{\pgfqpoint{0.696435in}{1.970150in}}%
\pgfpathlineto{\pgfqpoint{6.896435in}{1.970150in}}%
\pgfusepath{stroke}%
\end{pgfscope}%
\begin{pgfscope}%
\definecolor{textcolor}{rgb}{0.150000,0.150000,0.150000}%
\pgfsetstrokecolor{textcolor}%
\pgfsetfillcolor{textcolor}%
\pgftext[x=0.289968in,y=1.917389in,left,base]{\color{textcolor}\rmfamily\fontsize{10.000000}{12.000000}\selectfont 1.00}%
\end{pgfscope}%
\begin{pgfscope}%
\definecolor{textcolor}{rgb}{0.150000,0.150000,0.150000}%
\pgfsetstrokecolor{textcolor}%
\pgfsetfillcolor{textcolor}%
\pgftext[x=0.234413in,y=1.278570in,,bottom,rotate=90.000000]{\color{textcolor}\rmfamily\fontsize{10.000000}{12.000000}\selectfont p-value}%
\end{pgfscope}%
\begin{pgfscope}%
\pgfpathrectangle{\pgfqpoint{0.696435in}{0.523570in}}{\pgfqpoint{6.200000in}{1.510000in}}%
\pgfusepath{clip}%
\pgfsetroundcap%
\pgfsetroundjoin%
\pgfsetlinewidth{1.505625pt}%
\definecolor{currentstroke}{rgb}{0.121569,0.466667,0.705882}%
\pgfsetstrokecolor{currentstroke}%
\pgfsetdash{}{0pt}%
\pgfpathmoveto{\pgfqpoint{0.978254in}{1.404124in}}%
\pgfpathlineto{\pgfqpoint{1.122776in}{1.781923in}}%
\pgfpathlineto{\pgfqpoint{1.267298in}{1.916511in}}%
\pgfpathlineto{\pgfqpoint{1.411820in}{1.935176in}}%
\pgfpathlineto{\pgfqpoint{1.556342in}{1.959597in}}%
\pgfpathlineto{\pgfqpoint{1.700864in}{1.964886in}}%
\pgfpathlineto{\pgfqpoint{1.845386in}{1.950610in}}%
\pgfpathlineto{\pgfqpoint{1.989909in}{1.957850in}}%
\pgfpathlineto{\pgfqpoint{2.134431in}{1.880620in}}%
\pgfpathlineto{\pgfqpoint{2.278953in}{1.717026in}}%
\pgfpathlineto{\pgfqpoint{2.423475in}{1.645785in}}%
\pgfpathlineto{\pgfqpoint{2.567997in}{1.717161in}}%
\pgfpathlineto{\pgfqpoint{2.712519in}{1.394166in}}%
\pgfpathlineto{\pgfqpoint{2.857041in}{1.495842in}}%
\pgfpathlineto{\pgfqpoint{3.001564in}{0.806517in}}%
\pgfpathlineto{\pgfqpoint{3.146086in}{0.722643in}}%
\pgfpathlineto{\pgfqpoint{3.290608in}{0.652175in}}%
\pgfpathlineto{\pgfqpoint{3.435130in}{0.592206in}}%
\pgfpathlineto{\pgfqpoint{3.579652in}{0.597626in}}%
\pgfpathlineto{\pgfqpoint{3.724174in}{0.605030in}}%
\pgfpathlineto{\pgfqpoint{3.868696in}{0.606488in}}%
\pgfpathlineto{\pgfqpoint{4.013219in}{0.604737in}}%
\pgfpathlineto{\pgfqpoint{4.157741in}{0.611998in}}%
\pgfpathlineto{\pgfqpoint{4.302263in}{0.618559in}}%
\pgfpathlineto{\pgfqpoint{4.446785in}{0.631357in}}%
\pgfpathlineto{\pgfqpoint{4.591307in}{0.618763in}}%
\pgfpathlineto{\pgfqpoint{4.735829in}{0.630880in}}%
\pgfpathlineto{\pgfqpoint{4.880352in}{0.645958in}}%
\pgfpathlineto{\pgfqpoint{5.024874in}{0.664800in}}%
\pgfpathlineto{\pgfqpoint{5.169396in}{0.675144in}}%
\pgfpathlineto{\pgfqpoint{5.313918in}{0.670688in}}%
\pgfpathlineto{\pgfqpoint{5.458440in}{0.693625in}}%
\pgfpathlineto{\pgfqpoint{5.602962in}{0.720018in}}%
\pgfpathlineto{\pgfqpoint{5.747484in}{0.733882in}}%
\pgfpathlineto{\pgfqpoint{5.892007in}{0.721668in}}%
\pgfpathlineto{\pgfqpoint{6.036529in}{0.751910in}}%
\pgfpathlineto{\pgfqpoint{6.181051in}{0.750177in}}%
\pgfpathlineto{\pgfqpoint{6.325573in}{0.773480in}}%
\pgfpathlineto{\pgfqpoint{6.470095in}{0.779094in}}%
\pgfpathlineto{\pgfqpoint{6.614617in}{0.815239in}}%
\pgfusepath{stroke}%
\end{pgfscope}%
\begin{pgfscope}%
\pgfpathrectangle{\pgfqpoint{0.696435in}{0.523570in}}{\pgfqpoint{6.200000in}{1.510000in}}%
\pgfusepath{clip}%
\pgfsetroundcap%
\pgfsetroundjoin%
\pgfsetlinewidth{1.505625pt}%
\definecolor{currentstroke}{rgb}{1.000000,0.498039,0.054902}%
\pgfsetstrokecolor{currentstroke}%
\pgfsetdash{}{0pt}%
\pgfpathmoveto{\pgfqpoint{0.978254in}{1.404635in}}%
\pgfpathlineto{\pgfqpoint{1.122776in}{1.782272in}}%
\pgfpathlineto{\pgfqpoint{1.267298in}{1.916663in}}%
\pgfpathlineto{\pgfqpoint{1.411820in}{1.935355in}}%
\pgfpathlineto{\pgfqpoint{1.556342in}{1.959665in}}%
\pgfpathlineto{\pgfqpoint{1.700864in}{1.964934in}}%
\pgfpathlineto{\pgfqpoint{1.845386in}{1.950887in}}%
\pgfpathlineto{\pgfqpoint{1.989909in}{1.958058in}}%
\pgfpathlineto{\pgfqpoint{2.134431in}{1.882361in}}%
\pgfpathlineto{\pgfqpoint{2.278953in}{1.721820in}}%
\pgfpathlineto{\pgfqpoint{2.423475in}{1.652109in}}%
\pgfpathlineto{\pgfqpoint{2.567997in}{1.722781in}}%
\pgfpathlineto{\pgfqpoint{2.712519in}{1.404902in}}%
\pgfpathlineto{\pgfqpoint{2.857041in}{1.506091in}}%
\pgfpathlineto{\pgfqpoint{3.001564in}{0.816525in}}%
\pgfpathlineto{\pgfqpoint{3.146086in}{0.730670in}}%
\pgfpathlineto{\pgfqpoint{3.290608in}{0.657494in}}%
\pgfpathlineto{\pgfqpoint{3.435130in}{0.593567in}}%
\pgfpathlineto{\pgfqpoint{3.579652in}{0.599547in}}%
\pgfpathlineto{\pgfqpoint{3.724174in}{0.607673in}}%
\pgfpathlineto{\pgfqpoint{3.868696in}{0.609353in}}%
\pgfpathlineto{\pgfqpoint{4.013219in}{0.607552in}}%
\pgfpathlineto{\pgfqpoint{4.157741in}{0.615559in}}%
\pgfpathlineto{\pgfqpoint{4.302263in}{0.622810in}}%
\pgfpathlineto{\pgfqpoint{4.446785in}{0.636789in}}%
\pgfpathlineto{\pgfqpoint{4.591307in}{0.623364in}}%
\pgfpathlineto{\pgfqpoint{4.735829in}{0.636689in}}%
\pgfpathlineto{\pgfqpoint{4.880352in}{0.653175in}}%
\pgfpathlineto{\pgfqpoint{5.024874in}{0.673647in}}%
\pgfpathlineto{\pgfqpoint{5.169396in}{0.685035in}}%
\pgfpathlineto{\pgfqpoint{5.313918in}{0.680663in}}%
\pgfpathlineto{\pgfqpoint{5.458440in}{0.705520in}}%
\pgfpathlineto{\pgfqpoint{5.602962in}{0.733955in}}%
\pgfpathlineto{\pgfqpoint{5.747484in}{0.749137in}}%
\pgfpathlineto{\pgfqpoint{5.892007in}{0.736831in}}%
\pgfpathlineto{\pgfqpoint{6.036529in}{0.769355in}}%
\pgfpathlineto{\pgfqpoint{6.181051in}{0.768204in}}%
\pgfpathlineto{\pgfqpoint{6.325573in}{0.793417in}}%
\pgfpathlineto{\pgfqpoint{6.470095in}{0.800070in}}%
\pgfpathlineto{\pgfqpoint{6.614617in}{0.838682in}}%
\pgfusepath{stroke}%
\end{pgfscope}%
\begin{pgfscope}%
\pgfsetrectcap%
\pgfsetmiterjoin%
\pgfsetlinewidth{0.803000pt}%
\definecolor{currentstroke}{rgb}{1.000000,1.000000,1.000000}%
\pgfsetstrokecolor{currentstroke}%
\pgfsetdash{}{0pt}%
\pgfpathmoveto{\pgfqpoint{0.696435in}{0.523570in}}%
\pgfpathlineto{\pgfqpoint{0.696435in}{2.033570in}}%
\pgfusepath{stroke}%
\end{pgfscope}%
\begin{pgfscope}%
\pgfsetrectcap%
\pgfsetmiterjoin%
\pgfsetlinewidth{0.803000pt}%
\definecolor{currentstroke}{rgb}{1.000000,1.000000,1.000000}%
\pgfsetstrokecolor{currentstroke}%
\pgfsetdash{}{0pt}%
\pgfpathmoveto{\pgfqpoint{6.896435in}{0.523570in}}%
\pgfpathlineto{\pgfqpoint{6.896435in}{2.033570in}}%
\pgfusepath{stroke}%
\end{pgfscope}%
\begin{pgfscope}%
\pgfsetrectcap%
\pgfsetmiterjoin%
\pgfsetlinewidth{0.803000pt}%
\definecolor{currentstroke}{rgb}{1.000000,1.000000,1.000000}%
\pgfsetstrokecolor{currentstroke}%
\pgfsetdash{}{0pt}%
\pgfpathmoveto{\pgfqpoint{0.696435in}{0.523570in}}%
\pgfpathlineto{\pgfqpoint{6.896435in}{0.523570in}}%
\pgfusepath{stroke}%
\end{pgfscope}%
\begin{pgfscope}%
\pgfsetrectcap%
\pgfsetmiterjoin%
\pgfsetlinewidth{0.803000pt}%
\definecolor{currentstroke}{rgb}{1.000000,1.000000,1.000000}%
\pgfsetstrokecolor{currentstroke}%
\pgfsetdash{}{0pt}%
\pgfpathmoveto{\pgfqpoint{0.696435in}{2.033570in}}%
\pgfpathlineto{\pgfqpoint{6.896435in}{2.033570in}}%
\pgfusepath{stroke}%
\end{pgfscope}%
\begin{pgfscope}%
\pgfsetbuttcap%
\pgfsetmiterjoin%
\definecolor{currentfill}{rgb}{0.917647,0.917647,0.949020}%
\pgfsetfillcolor{currentfill}%
\pgfsetfillopacity{0.800000}%
\pgfsetlinewidth{1.003750pt}%
\definecolor{currentstroke}{rgb}{0.800000,0.800000,0.800000}%
\pgfsetstrokecolor{currentstroke}%
\pgfsetstrokeopacity{0.800000}%
\pgfsetdash{}{0pt}%
\pgfpathmoveto{\pgfqpoint{5.599832in}{1.512778in}}%
\pgfpathlineto{\pgfqpoint{6.799213in}{1.512778in}}%
\pgfpathquadraticcurveto{\pgfqpoint{6.826991in}{1.512778in}}{\pgfqpoint{6.826991in}{1.540556in}}%
\pgfpathlineto{\pgfqpoint{6.826991in}{1.936348in}}%
\pgfpathquadraticcurveto{\pgfqpoint{6.826991in}{1.964126in}}{\pgfqpoint{6.799213in}{1.964126in}}%
\pgfpathlineto{\pgfqpoint{5.599832in}{1.964126in}}%
\pgfpathquadraticcurveto{\pgfqpoint{5.572054in}{1.964126in}}{\pgfqpoint{5.572054in}{1.936348in}}%
\pgfpathlineto{\pgfqpoint{5.572054in}{1.540556in}}%
\pgfpathquadraticcurveto{\pgfqpoint{5.572054in}{1.512778in}}{\pgfqpoint{5.599832in}{1.512778in}}%
\pgfpathclose%
\pgfusepath{stroke,fill}%
\end{pgfscope}%
\begin{pgfscope}%
\pgfsetroundcap%
\pgfsetroundjoin%
\pgfsetlinewidth{1.505625pt}%
\definecolor{currentstroke}{rgb}{0.121569,0.466667,0.705882}%
\pgfsetstrokecolor{currentstroke}%
\pgfsetdash{}{0pt}%
\pgfpathmoveto{\pgfqpoint{5.627609in}{1.851658in}}%
\pgfpathlineto{\pgfqpoint{5.905387in}{1.851658in}}%
\pgfusepath{stroke}%
\end{pgfscope}%
\begin{pgfscope}%
\definecolor{textcolor}{rgb}{0.150000,0.150000,0.150000}%
\pgfsetstrokecolor{textcolor}%
\pgfsetfillcolor{textcolor}%
\pgftext[x=6.016498in,y=1.803047in,left,base]{\color{textcolor}\rmfamily\fontsize{10.000000}{12.000000}\selectfont Ljung-Box}%
\end{pgfscope}%
\begin{pgfscope}%
\pgfsetroundcap%
\pgfsetroundjoin%
\pgfsetlinewidth{1.505625pt}%
\definecolor{currentstroke}{rgb}{1.000000,0.498039,0.054902}%
\pgfsetstrokecolor{currentstroke}%
\pgfsetdash{}{0pt}%
\pgfpathmoveto{\pgfqpoint{5.627609in}{1.645834in}}%
\pgfpathlineto{\pgfqpoint{5.905387in}{1.645834in}}%
\pgfusepath{stroke}%
\end{pgfscope}%
\begin{pgfscope}%
\definecolor{textcolor}{rgb}{0.150000,0.150000,0.150000}%
\pgfsetstrokecolor{textcolor}%
\pgfsetfillcolor{textcolor}%
\pgftext[x=6.016498in,y=1.597223in,left,base]{\color{textcolor}\rmfamily\fontsize{10.000000}{12.000000}\selectfont Box-Pierce}%
\end{pgfscope}%
\end{pgfpicture}%
\makeatother%
\endgroup%

    \end{adjustbox}
    \caption{}
    \label{fig:ljungbox}
\end{figure}{}

\subsubsection{ARMA models}
We start by fitting ARMA models to the time series. This turned out to be prone to numerical instability. While the BIC could always be calculated, for some of the models standard errors could not be computed as the software was not able to invert the Hessian matrix. The underlying problem is exacerbated when dealing with GARCH models, as all residuals are squared. The problem was eventually alleviated by multiplying our log-returns by 100 (corresponding to an approximate percentage interpretation). Table \ref{tab:bic_arma} shows the BIC that were obtained from fitting different ARMA(p,q) models to the series of log-returns. 

\begin{table}
    \centering
    \figuretitle{BIC for different combinations of ARMA(p,q) for V}
\begin{adjustbox}{width=.95\textwidth,center}
\begin{tabular}{lrrr}
\toprule
(p,q)   &          0     &       1       &     2 \\
\midrule
0 & -8870.52 & -8872.19 & -8868.22 \\
1 & -8871.40 & \textbf{-8872.93} & -8865.64 \\
2 & -8866.97 & -8865.65 & -8858.30 \\
3 & -8861.59 & -8859.10 & -8851.28 \\
4 & -8859.42 & -8854.27 & -8854.01 \\ 
\bottomrule
\end{tabular}
\quad
\begin{tabular}{lrrr}
\toprule
(p,q)   &          0     &       1       &     2 \\
\midrule
0 & 5027.88  & 5026.20  & 5030.17 \\
1 & 5026.99  & \textbf{5025.46}  & 5032.75 \\
2  & 5031.42  & 5032.75  & 5039.26 \\
3  & 5036.81  & 5039.29  & 5037.07 \\
4  & 5038.98  & 5044.13  & 5044.39 \\
\bottomrule
\end{tabular}
\end{adjustbox}

\hspace{6ex}
\newline
\hspace{6ex}
\newline
\figuretitle{BIC for different combinations of ARMA(p,q) for INTC}
\begin{adjustbox}{width=.95\textwidth,center}
\begin{tabular}{lrrr}
\toprule
(p,q)   &          0     &       1       &     2 \\
\midrule
0 & \textbf{-8692.98} & -8685.94 & -8678.63 \\
1 & -8685.94 & -8678.62 & -8671.31 \\
2 & -8678.63 & -8671.30 & -8672.74 \\
3 & -8671.31 & -8664.08 & -8664.26 \\
4 & -8664.18 & -8656.92 & -8652.19 \\
\bottomrule
\end{tabular}
\quad
\begin{tabular}{lrrr}
\toprule
(p,q)   &          0     &       1       &     2 \\
\midrule
0 & \textbf{5205.42} & 5212.45 & 5219.77 \\
1 & 5212.45 & 5219.77 & 5227.09 \\
2 & 5219.77 & 5227.09 & 5226.21 \\
3 & 5227.09 & 5234.32 & 5238.91 \\
4 & 5234.21 & 5241.47 & 5246.20 \\
\bottomrule
\end{tabular}
\end{adjustbox}
    \caption{BIC presented for different combinations of ARMA(p,q) fit to the log-returns of V (top) and INTC (bottom). On the right side, those returns were multiplied by 100 in order to allow for comparison with the GARCH models later on.}
    \label{tab:bic_arma}
\end{table}{}

For V the best model according to BIC is ARMA(1,1). However the difference to ARMA(0,0) seems as good as negligible. In order to avoid the peril of reading too much into random chance, it may be more prudent to stick with a constant mean model (ARMA(0,0)). For INTC the lowest BIC is indeed reached by ARMA(0,0) which corresponds to our observation that there is no significant lower-level autocorrelation. 

Tables \ref{tab:V_ARMA11_log_returns}, \ref{} show the results of fitting an  ARMA(1,1) and an ARMA(0,0) model to the log-returns of V. The change in AIC is larger than the change in BIC as the latter more heavily penalizes the number of parameters included in the estimation. Interestingly, even though the difference in BIC is quite small, the AR(1) and MA(1) are estimated very significantly and are high in relative magnitude. However, note that their effects go in opposing directions in similar magnitude and may as well cancel each other out. This interpretation may be somewhat plausible as the individual effects of the AR(1) and MA(1) terms are about an order of magnitude smaller (albeit still significant) and go in the same direction when estimating ARMA(1,0) and ARMA(0,1) models separately (-0.0736 for the AR(1) and -0.0809 for the MA(1) term).
Table \ref{tab:INTC_ARMA00_log_returns} shows the result of fitting an ARMA(0,0) model to the log-returns of INTC. Not even the mean is significant, which corresponds to the development of the stock prices of INTC over the time period. As could be seen back in figure \ref{fig:Daily Stock Prices for all Stocks in the Data Set}, Intel has hardly gained in value from 2012 to 2017. Also when exploratively trying to fit an ARMA(1,0), ARMA(0,1) or ARMA(1,1) model none of the coefficients reach significance (p-values all > 0.5). 

\begin{table}[h]
    \centering
    \figuretitle{Results for an ARMA(1,1) process fit to the log-returns of V}
    \begin{center}
\begin{tabular}{lclc}
\toprule
\textbf{Dep. Variable:}     &        log\_returns       & \textbf{  No. Observations:  } &            1509            \\
\textbf{Model:}             &         ARMA(1, 1)        & \textbf{  Log Likelihood     } &          4451.108          \\
\textbf{Method:}            &          css-mle          & \textbf{  S.D. of innovations} &           0.013            \\
\textbf{Date:}              &      Fri, 30 Aug 2019     & \textbf{  AIC                } &         -8894.215          \\
\textbf{Time:}              &          14:48:01         & \textbf{  BIC                } &         -8872.938          \\
\textbf{Sample:}            &             0             & \textbf{  HQIC               } &         -8886.291          \\
\bottomrule
\end{tabular}
\begin{tabular}{lcccccc}
                            & \textbf{coef} & \textbf{std err} & \textbf{z} & \textbf{P$> |$z$|$} & \textbf{[0.025} & \textbf{0.975]}  \\
\midrule
\textbf{const}              &       0.0011  &        0.000     &     4.194  &         0.000        &        0.001    &        0.002     \\
\textbf{ar.L1.log\_returns} &       0.6156  &        0.116     &     5.307  &         0.000        &        0.388    &        0.843     \\
\textbf{ma.L1.log\_returns} &      -0.6997  &        0.105     &    -6.681  &         0.000        &       -0.905    &       -0.494     \\
\bottomrule
\end{tabular}
\begin{tabular}{lcccc}
              & \textbf{            Real} & \textbf{         Imaginary} & \textbf{         Modulus} & \textbf{        Frequency}  \\
\midrule
\textbf{AR.1} &                1.6243     &                +0.0000j     &                1.6243     &                0.0000       \\
\textbf{MA.1} &                1.4293     &                +0.0000j     &                1.4293     &                0.0000       \\
\bottomrule
\end{tabular}
%\caption{ARMA Model Results}
\end{center}

    \caption{}
    \label{tab:V_ARMA11_log_returns}
\end{table}{}

\begin{table}[h!]
    \centering
    \figuretitle{Results for an ARMA(0,0) process fit to the log-returns of V}
    \begin{center}
\begin{tabular}{lclc}
\toprule
\textbf{Dep. Variable:} &   log\_returns   & \textbf{  No. Observations:  } &    1509     \\
\textbf{Model:}         &    ARMA(0, 0)    & \textbf{  Log Likelihood     } &  4442.581   \\
\textbf{Method:}        &       css        & \textbf{  S.D. of innovations} &   0.013     \\
\textbf{Date:}          & Wed, 04 Sep 2019 & \textbf{  AIC                } & -8881.162   \\
\textbf{Time:}          &     10:03:08     & \textbf{  BIC                } & -8870.524   \\
\textbf{Sample:}        &        0         & \textbf{  HQIC               } & -8877.200   \\
\bottomrule
\end{tabular}
\begin{tabular}{lcccccc}
               & \textbf{coef} & \textbf{std err} & \textbf{z} & \textbf{P$> |$z$|$} & \textbf{[0.025} & \textbf{0.975]}  \\
\midrule
\textbf{const} &       0.0011  &        0.000     &     3.253  &         0.001        &        0.000    &        0.002     \\
\bottomrule
\end{tabular}
%\caption{ARMA Model Results}
\end{center}

    \caption{}
    \label{tab:V_ARMA11_log_returns}
\end{table}{}

\begin{table}[h!]
    \centering
    \figuretitle{Results for an ARMA(0,0) process fit to the log-returns of INTC}
    \begin{center}
\begin{tabular}{lclc}
\toprule
\textbf{Dep. Variable:} &   log\_returns   & \textbf{  No. Observations:  } &    1509     \\
\textbf{Model:}         &    ARMA(0, 0)    & \textbf{  Log Likelihood     } &  4353.809   \\
\textbf{Method:}        &       css        & \textbf{  S.D. of innovations} &   0.014     \\
\textbf{Date:}          & Tue, 03 Sep 2019 & \textbf{  AIC                } & -8703.618   \\
\textbf{Time:}          &     14:04:11     & \textbf{  BIC                } & -8692.979   \\
\textbf{Sample:}        &        0         & \textbf{  HQIC               } & -8699.656   \\
\bottomrule
\end{tabular}
\begin{tabular}{lcccccc}
               & \textbf{coef} & \textbf{std err} & \textbf{z} & \textbf{P$> |$z$|$} & \textbf{[0.025} & \textbf{0.975]}  \\
\midrule
\textbf{const} &       0.0005  &        0.000     &     1.574  &         0.116        &       -0.000    &        0.001     \\
\bottomrule
\end{tabular}
%\caption{ARMA Model Results}
\end{center}

    \caption{}
    \label{tab:INTC_ARMA00_log_returns}
\end{table}{}


--> make predictions
--> calculate MSE
--> make test for trivial forecasts

\subsubsection{ARMAX models}
We proceed in our analysis by adding our sentiments generated by the machine learning model and the ravenpack sentiments as external regressors. To make matters more complicated our stock market observations don't match the external data. We have about a third as many analyst reports (and therefore sentiment scores) as stock market observations. In order to obtain an estimable model we decided to discard all observations on days where no analyst reports existed. This also implied that we had to confine our analysis to a constant mean model (ARMA(0,0) as autoregressive and moving-average components do not make sense when many observations in the time series are missing. As explained previously, this should not be problematic. The model hence reduces to a regression with an intercept and the sentiment data as independent variables. The Ravenpack data, on the other side, usually had multiple entries per day with only few days missing. We therefore aggregated sentiments to obtain one single observation per day. 
%To allow for better comparison, we also confined ourselves to 

\textit{ARMAX Sentiments Analyst Reports}
The fact that we had to omit two-thirds of all data points changes the BIC considerably. To be able to compare the models according to BIC we refit the ARMA(0,0) model to the reduced data and obtained a BIC of -3001.07 for V and -4533.63 for INTC (the difference owing to the different number of observations). Table \ref{tab:V_result_ARMAX00_sentiment} shows the results for V, table \ref{tab:INTC_result_ARMAX00_sentiment} shows the results for INTC. Even though sent\_mean is not too far from significance for V (and is even closer when only including sent\_mean, with a p-value of 0.067), all models fare worse in terms of BIC than the baseline. 

\begin{table}[h]
    \centering
    \figuretitle{ARMAX(0,0) with analyst report sentiments fit to the log-returns of V}
    \begin{center}
\begin{tabular}{lclc}
\toprule
\textbf{Dep. Variable:} &   log\_returns   & \textbf{  No. Observations:  } &    535      \\
\textbf{Model:}         &    ARMA(0, 0)    & \textbf{  Log Likelihood     } &  1508.493   \\
\textbf{Method:}        &       css        & \textbf{  S.D. of innovations} &   0.014     \\
\textbf{Date:}          & Tue, 03 Sep 2019 & \textbf{  AIC                } & -3010.985   \\
\textbf{Time:}          &     13:18:14     & \textbf{  BIC                } & -2998.139   \\
\textbf{Sample:}        &        0         & \textbf{  HQIC               } & -3005.959   \\
\bottomrule
\end{tabular}
\begin{tabular}{lcccccc}
                     & \textbf{coef} & \textbf{std err} & \textbf{z} & \textbf{P$> |$z$|$} & \textbf{[0.025} & \textbf{0.975]}  \\
\midrule
\textbf{const}       &       0.0048  &        0.002     &     3.087  &         0.002        &        0.002    &        0.008     \\
\textbf{sent1\_mean} &      -0.0050  &        0.003     &    -1.834  &         0.067        &       -0.010    &        0.000     \\
\bottomrule
\end{tabular}
%\caption{ARMA Model Results}
\end{center}

    \caption{}
    \label{tab:V_result_ARMAX00_sentiment}
\end{table}{}

\begin{table}[h!]
    \centering
    \figuretitle{ARMAX(0,0) with analyst report sentiments fit to the log-returns of INTC}
    \begin{center}
\begin{tabular}{lclc}
\toprule
\textbf{Dep. Variable:} &   log\_returns   & \textbf{  No. Observations:  } &    821      \\
\textbf{Model:}         &    ARMA(0, 0)    & \textbf{  Log Likelihood     } &  2273.910   \\
\textbf{Method:}        &       css        & \textbf{  S.D. of innovations} &   0.015     \\
\textbf{Date:}          & Tue, 03 Sep 2019 & \textbf{  AIC                } & -4541.819   \\
\textbf{Time:}          &     13:18:14     & \textbf{  BIC                } & -4527.688   \\
\textbf{Sample:}        &        0         & \textbf{  HQIC               } & -4536.397   \\
\bottomrule
\end{tabular}
\begin{tabular}{lcccccc}
                     & \textbf{coef} & \textbf{std err} & \textbf{z} & \textbf{P$> |$z$|$} & \textbf{[0.025} & \textbf{0.975]}  \\
\midrule
\textbf{const}       &       0.0020  &        0.001     &     1.470  &         0.142        &       -0.001    &        0.005     \\
\textbf{sent1\_mean} &      -0.0023  &        0.003     &    -0.879  &         0.380        &       -0.007    &        0.003     \\
\bottomrule
\end{tabular}
%\caption{ARMA Model Results}
\end{center}

    \caption{}
    \label{tab:INTC_result_ARMAX00_sentiment}
\end{table}{}

\textit{ARMAX Sentiments Ravenpack}
In the Ravenpack data for V, only 21 observations out of 1509 were missing. We decided it might still be interesting to continue having a look at ARMA(1,1) even though the estimation is now mildly distorted by the fact that some values are missing. We computed a new baseline BIC as we did in the previous analysis. The BIC for an ARMA(0,0) model is now -8734.35 and for an ARMA(1,1) model -8735.23. For INTC there were observations for every day of the time series, so the baseline BIC stayed -8692.98. Table \ref{tab:V_result_ARMAX00_ravenpack} shows the result for the ARMAX(0,0) model fit with ravenpack sentiments to V, table \ref{tab:INTC_result_ARMAX00_ravenpack}. We have tried different combinations of regressors to include in the model but only show the full model to avoid redundancy. Also the results for ARMAX(1,1) for V are not shown but look very similar. In all cases the external information worsened BIC and did not improve the fit. For the following GARCH analysis we will therefore not include external information. 

\begin{table}[h]
    \centering
    \figuretitle{ARMAX(0,0) with Ravenpack sentiments fit to the log-returns of V}
    \begin{center}
\begin{tabular}{lclc}
\toprule
\textbf{Dep. Variable:} &   log\_returns   & \textbf{  No. Observations:  } &    1488     \\
\textbf{Model:}         &    ARMA(0, 0)    & \textbf{  Log Likelihood     } &  4375.394   \\
\textbf{Method:}        &       css        & \textbf{  S.D. of innovations} &   0.013     \\
\textbf{Date:}          & Wed, 04 Sep 2019 & \textbf{  AIC                } & -8736.788   \\
\textbf{Time:}          &     17:23:40     & \textbf{  BIC                } & -8699.652   \\
\textbf{Sample:}        &        0         & \textbf{  HQIC               } & -8722.948   \\
\bottomrule
\end{tabular}
\begin{tabular}{lcccccc}
                   & \textbf{coef} & \textbf{std err} & \textbf{z} & \textbf{P$> |$z$|$} & \textbf{[0.025} & \textbf{0.975]}  \\
\midrule
\textbf{const}     &       0.0124  &        0.012     &     1.061  &         0.289        &       -0.010    &        0.035     \\
\textbf{relevance} &      -0.0001  &        0.000     &    -1.037  &         0.300        &       -0.000    &        0.000     \\
\textbf{aes\_min}  &   -1.536e-05  &     2.42e-05     &    -0.634  &         0.526        &    -6.29e-05    &     3.22e-05     \\
\textbf{aes\_max}  &    1.781e-05  &     3.35e-05     &     0.531  &         0.595        &    -4.79e-05    &     8.35e-05     \\
\textbf{count}     &    5.652e-06  &     1.41e-05     &     0.402  &         0.688        &    -2.19e-05    &     3.32e-05     \\
\textbf{aev\_min}  &    2.091e-07  &     2.53e-06     &     0.083  &         0.934        &    -4.76e-06    &     5.18e-06     \\
\bottomrule
\end{tabular}
%\caption{ARMA Model Results}
\end{center}

    \caption{}
    \label{tab:V_result_ARMAX00_ravenpack}
\end{table}{}

\begin{table}[h]
    \centering
    \figuretitle{ARMAX(1,1) with Ravenpack sentiments fit to the log-returns of V}
    \begin{center}
\begin{tabular}{lclc}
\toprule
\textbf{Dep. Variable:}     &        log\_returns       & \textbf{  No. Observations:  } &            1488            \\
\textbf{Model:}             &         ARMA(1, 1)        & \textbf{  Log Likelihood     } &          4383.305          \\
\textbf{Method:}            &          css-mle          & \textbf{  S.D. of innovations} &           0.013            \\
\textbf{Date:}              &      Wed, 04 Sep 2019     & \textbf{  AIC                } &         -8748.610          \\
\textbf{Time:}              &          17:23:41         & \textbf{  BIC                } &         -8700.863          \\
\textbf{Sample:}            &             0             & \textbf{  HQIC               } &         -8730.816          \\
\bottomrule
\end{tabular}
\begin{tabular}{lcccccc}
                            & \textbf{coef} & \textbf{std err} & \textbf{z} & \textbf{P$> |$z$|$} & \textbf{[0.025} & \textbf{0.975]}  \\
\midrule
\textbf{const}              &       0.0124  &        0.011     &     1.126  &         0.260        &       -0.009    &        0.034     \\
\textbf{relevance}          &      -0.0001  &        0.000     &    -1.136  &         0.256        &       -0.000    &     9.28e-05     \\
\textbf{aes\_min}           &   -1.653e-05  &     2.21e-05     &    -0.747  &         0.455        &    -5.99e-05    &     2.69e-05     \\
\textbf{aes\_max}           &     2.38e-05  &      3.2e-05     &     0.744  &         0.457        &    -3.89e-05    &     8.65e-05     \\
\textbf{count}              &    6.773e-06  &     1.36e-05     &     0.497  &         0.620        &       -2e-05    &     3.35e-05     \\
\textbf{aev\_min}           &    3.643e-07  &      2.4e-06     &     0.152  &         0.880        &    -4.35e-06    &     5.08e-06     \\
\textbf{ar.L1.log\_returns} &       0.5833  &        0.149     &     3.907  &         0.000        &        0.291    &        0.876     \\
\textbf{ma.L1.log\_returns} &      -0.6563  &        0.143     &    -4.585  &         0.000        &       -0.937    &       -0.376     \\
\bottomrule
\end{tabular}
\begin{tabular}{lcccc}
              & \textbf{            Real} & \textbf{         Imaginary} & \textbf{         Modulus} & \textbf{        Frequency}  \\
\midrule
\textbf{AR.1} &                1.7145     &                +0.0000j     &                1.7145     &                0.0000       \\
\textbf{MA.1} &                1.5237     &                +0.0000j     &                1.5237     &                0.0000       \\
\bottomrule
\end{tabular}
%\caption{ARMA Model Results}
\end{center}

    \caption{}
    \label{tab:V_result_ARMAX00_sentiment}
\end{table}{}

\begin{table}[h]
    \centering
    \figuretitle{ARMAX(0,0) with Ravenpack sentiments fit to the log-returns of INTC}
    \begin{center}
\begin{tabular}{lclc}
\toprule
\textbf{Dep. Variable:} &   log\_returns   & \textbf{  No. Observations:  } &    1509     \\
\textbf{Model:}         &    ARMA(0, 0)    & \textbf{  Log Likelihood     } &  4356.793   \\
\textbf{Method:}        &       css        & \textbf{  S.D. of innovations} &   0.013     \\
\textbf{Date:}          & Wed, 04 Sep 2019 & \textbf{  AIC                } & -8699.585   \\
\textbf{Time:}          &     17:23:41     & \textbf{  BIC                } & -8662.351   \\
\textbf{Sample:}        &        0         & \textbf{  HQIC               } & -8685.718   \\
\bottomrule
\end{tabular}
\begin{tabular}{lcccccc}
                   & \textbf{coef} & \textbf{std err} & \textbf{z} & \textbf{P$> |$z$|$} & \textbf{[0.025} & \textbf{0.975]}  \\
\midrule
\textbf{const}     &      -0.0324  &        0.019     &    -1.747  &         0.081        &       -0.069    &        0.004     \\
\textbf{relevance} &       0.0003  &        0.000     &     1.670  &         0.095        &    -5.63e-05    &        0.001     \\
\textbf{aes\_min}  &    4.252e-05  &     2.68e-05     &     1.586  &         0.113        &       -1e-05    &     9.51e-05     \\
\textbf{aes\_max}  &   -4.529e-06  &     3.08e-05     &    -0.147  &         0.883        &    -6.49e-05    &     5.59e-05     \\
\textbf{count}     &   -1.395e-06  &     6.02e-06     &    -0.232  &         0.817        &    -1.32e-05    &     1.04e-05     \\
\textbf{aev\_min}  &   -1.217e-06  &     1.83e-06     &    -0.667  &         0.505        &    -4.79e-06    &     2.36e-06     \\
\bottomrule
\end{tabular}
%\caption{ARMA Model Results}
\end{center}

    \caption{}
    \label{tab:INTC_result_ARMAX00_ravenpack}
\end{table}{}

\subsubsection{GARCH models}
Financial data often exhibit conditional heteroskedasticity. We therefore want to see whether GARCH models are able to improve the fit. Figure \ref{fig:V_INTC_squared} shows the squared log-returns of V and INTC as well as their ACF and PACF. For both the first lag visually seems to be significant. 

\begin{figure}[h]
    \centering
    \figuretitle{Squared log-returns V and INTC}
    \begin{adjustbox}{width=.95\textwidth,center}
    %% Creator: Matplotlib, PGF backend
%%
%% To include the figure in your LaTeX document, write
%%   \input{<filename>.pgf}
%%
%% Make sure the required packages are loaded in your preamble
%%   \usepackage{pgf}
%%
%% Figures using additional raster images can only be included by \input if
%% they are in the same directory as the main LaTeX file. For loading figures
%% from other directories you can use the `import` package
%%   \usepackage{import}
%% and then include the figures with
%%   \import{<path to file>}{<filename>.pgf}
%%
%% Matplotlib used the following preamble
%%   \usepackage{fontspec}
%%   \setmainfont{DejaVuSerif.ttf}[Path=/opt/tljh/user/lib/python3.6/site-packages/matplotlib/mpl-data/fonts/ttf/]
%%   \setsansfont{DejaVuSans.ttf}[Path=/opt/tljh/user/lib/python3.6/site-packages/matplotlib/mpl-data/fonts/ttf/]
%%   \setmonofont{DejaVuSansMono.ttf}[Path=/opt/tljh/user/lib/python3.6/site-packages/matplotlib/mpl-data/fonts/ttf/]
%%
\begingroup%
\makeatletter%
\begin{pgfpicture}%
\pgfpathrectangle{\pgfpointorigin}{\pgfqpoint{3.922887in}{1.941635in}}%
\pgfusepath{use as bounding box, clip}%
\begin{pgfscope}%
\pgfsetbuttcap%
\pgfsetmiterjoin%
\definecolor{currentfill}{rgb}{1.000000,1.000000,1.000000}%
\pgfsetfillcolor{currentfill}%
\pgfsetlinewidth{0.000000pt}%
\definecolor{currentstroke}{rgb}{1.000000,1.000000,1.000000}%
\pgfsetstrokecolor{currentstroke}%
\pgfsetdash{}{0pt}%
\pgfpathmoveto{\pgfqpoint{0.000000in}{0.000000in}}%
\pgfpathlineto{\pgfqpoint{3.922887in}{0.000000in}}%
\pgfpathlineto{\pgfqpoint{3.922887in}{1.941635in}}%
\pgfpathlineto{\pgfqpoint{0.000000in}{1.941635in}}%
\pgfpathclose%
\pgfusepath{fill}%
\end{pgfscope}%
\begin{pgfscope}%
\pgfsetbuttcap%
\pgfsetmiterjoin%
\definecolor{currentfill}{rgb}{0.917647,0.917647,0.949020}%
\pgfsetfillcolor{currentfill}%
\pgfsetlinewidth{0.000000pt}%
\definecolor{currentstroke}{rgb}{0.000000,0.000000,0.000000}%
\pgfsetstrokecolor{currentstroke}%
\pgfsetstrokeopacity{0.000000}%
\pgfsetdash{}{0pt}%
\pgfpathmoveto{\pgfqpoint{0.683198in}{0.331635in}}%
\pgfpathlineto{\pgfqpoint{3.783198in}{0.331635in}}%
\pgfpathlineto{\pgfqpoint{3.783198in}{1.841635in}}%
\pgfpathlineto{\pgfqpoint{0.683198in}{1.841635in}}%
\pgfpathclose%
\pgfusepath{fill}%
\end{pgfscope}%
\begin{pgfscope}%
\pgfpathrectangle{\pgfqpoint{0.683198in}{0.331635in}}{\pgfqpoint{3.100000in}{1.510000in}}%
\pgfusepath{clip}%
\pgfsetroundcap%
\pgfsetroundjoin%
\pgfsetlinewidth{0.803000pt}%
\definecolor{currentstroke}{rgb}{1.000000,1.000000,1.000000}%
\pgfsetstrokecolor{currentstroke}%
\pgfsetdash{}{0pt}%
\pgfpathmoveto{\pgfqpoint{0.820239in}{0.331635in}}%
\pgfpathlineto{\pgfqpoint{0.820239in}{1.841635in}}%
\pgfusepath{stroke}%
\end{pgfscope}%
\begin{pgfscope}%
\definecolor{textcolor}{rgb}{0.150000,0.150000,0.150000}%
\pgfsetstrokecolor{textcolor}%
\pgfsetfillcolor{textcolor}%
\pgftext[x=0.820239in,y=0.234413in,,top]{\color{textcolor}\rmfamily\fontsize{10.000000}{12.000000}\selectfont 2012}%
\end{pgfscope}%
\begin{pgfscope}%
\pgfpathrectangle{\pgfqpoint{0.683198in}{0.331635in}}{\pgfqpoint{3.100000in}{1.510000in}}%
\pgfusepath{clip}%
\pgfsetroundcap%
\pgfsetroundjoin%
\pgfsetlinewidth{0.803000pt}%
\definecolor{currentstroke}{rgb}{1.000000,1.000000,1.000000}%
\pgfsetstrokecolor{currentstroke}%
\pgfsetdash{}{0pt}%
\pgfpathmoveto{\pgfqpoint{1.292085in}{0.331635in}}%
\pgfpathlineto{\pgfqpoint{1.292085in}{1.841635in}}%
\pgfusepath{stroke}%
\end{pgfscope}%
\begin{pgfscope}%
\definecolor{textcolor}{rgb}{0.150000,0.150000,0.150000}%
\pgfsetstrokecolor{textcolor}%
\pgfsetfillcolor{textcolor}%
\pgftext[x=1.292085in,y=0.234413in,,top]{\color{textcolor}\rmfamily\fontsize{10.000000}{12.000000}\selectfont 2013}%
\end{pgfscope}%
\begin{pgfscope}%
\pgfpathrectangle{\pgfqpoint{0.683198in}{0.331635in}}{\pgfqpoint{3.100000in}{1.510000in}}%
\pgfusepath{clip}%
\pgfsetroundcap%
\pgfsetroundjoin%
\pgfsetlinewidth{0.803000pt}%
\definecolor{currentstroke}{rgb}{1.000000,1.000000,1.000000}%
\pgfsetstrokecolor{currentstroke}%
\pgfsetdash{}{0pt}%
\pgfpathmoveto{\pgfqpoint{1.762641in}{0.331635in}}%
\pgfpathlineto{\pgfqpoint{1.762641in}{1.841635in}}%
\pgfusepath{stroke}%
\end{pgfscope}%
\begin{pgfscope}%
\definecolor{textcolor}{rgb}{0.150000,0.150000,0.150000}%
\pgfsetstrokecolor{textcolor}%
\pgfsetfillcolor{textcolor}%
\pgftext[x=1.762641in,y=0.234413in,,top]{\color{textcolor}\rmfamily\fontsize{10.000000}{12.000000}\selectfont 2014}%
\end{pgfscope}%
\begin{pgfscope}%
\pgfpathrectangle{\pgfqpoint{0.683198in}{0.331635in}}{\pgfqpoint{3.100000in}{1.510000in}}%
\pgfusepath{clip}%
\pgfsetroundcap%
\pgfsetroundjoin%
\pgfsetlinewidth{0.803000pt}%
\definecolor{currentstroke}{rgb}{1.000000,1.000000,1.000000}%
\pgfsetstrokecolor{currentstroke}%
\pgfsetdash{}{0pt}%
\pgfpathmoveto{\pgfqpoint{2.233198in}{0.331635in}}%
\pgfpathlineto{\pgfqpoint{2.233198in}{1.841635in}}%
\pgfusepath{stroke}%
\end{pgfscope}%
\begin{pgfscope}%
\definecolor{textcolor}{rgb}{0.150000,0.150000,0.150000}%
\pgfsetstrokecolor{textcolor}%
\pgfsetfillcolor{textcolor}%
\pgftext[x=2.233198in,y=0.234413in,,top]{\color{textcolor}\rmfamily\fontsize{10.000000}{12.000000}\selectfont 2015}%
\end{pgfscope}%
\begin{pgfscope}%
\pgfpathrectangle{\pgfqpoint{0.683198in}{0.331635in}}{\pgfqpoint{3.100000in}{1.510000in}}%
\pgfusepath{clip}%
\pgfsetroundcap%
\pgfsetroundjoin%
\pgfsetlinewidth{0.803000pt}%
\definecolor{currentstroke}{rgb}{1.000000,1.000000,1.000000}%
\pgfsetstrokecolor{currentstroke}%
\pgfsetdash{}{0pt}%
\pgfpathmoveto{\pgfqpoint{2.703754in}{0.331635in}}%
\pgfpathlineto{\pgfqpoint{2.703754in}{1.841635in}}%
\pgfusepath{stroke}%
\end{pgfscope}%
\begin{pgfscope}%
\definecolor{textcolor}{rgb}{0.150000,0.150000,0.150000}%
\pgfsetstrokecolor{textcolor}%
\pgfsetfillcolor{textcolor}%
\pgftext[x=2.703754in,y=0.234413in,,top]{\color{textcolor}\rmfamily\fontsize{10.000000}{12.000000}\selectfont 2016}%
\end{pgfscope}%
\begin{pgfscope}%
\pgfpathrectangle{\pgfqpoint{0.683198in}{0.331635in}}{\pgfqpoint{3.100000in}{1.510000in}}%
\pgfusepath{clip}%
\pgfsetroundcap%
\pgfsetroundjoin%
\pgfsetlinewidth{0.803000pt}%
\definecolor{currentstroke}{rgb}{1.000000,1.000000,1.000000}%
\pgfsetstrokecolor{currentstroke}%
\pgfsetdash{}{0pt}%
\pgfpathmoveto{\pgfqpoint{3.175600in}{0.331635in}}%
\pgfpathlineto{\pgfqpoint{3.175600in}{1.841635in}}%
\pgfusepath{stroke}%
\end{pgfscope}%
\begin{pgfscope}%
\definecolor{textcolor}{rgb}{0.150000,0.150000,0.150000}%
\pgfsetstrokecolor{textcolor}%
\pgfsetfillcolor{textcolor}%
\pgftext[x=3.175600in,y=0.234413in,,top]{\color{textcolor}\rmfamily\fontsize{10.000000}{12.000000}\selectfont 2017}%
\end{pgfscope}%
\begin{pgfscope}%
\pgfpathrectangle{\pgfqpoint{0.683198in}{0.331635in}}{\pgfqpoint{3.100000in}{1.510000in}}%
\pgfusepath{clip}%
\pgfsetroundcap%
\pgfsetroundjoin%
\pgfsetlinewidth{0.803000pt}%
\definecolor{currentstroke}{rgb}{1.000000,1.000000,1.000000}%
\pgfsetstrokecolor{currentstroke}%
\pgfsetdash{}{0pt}%
\pgfpathmoveto{\pgfqpoint{3.646156in}{0.331635in}}%
\pgfpathlineto{\pgfqpoint{3.646156in}{1.841635in}}%
\pgfusepath{stroke}%
\end{pgfscope}%
\begin{pgfscope}%
\definecolor{textcolor}{rgb}{0.150000,0.150000,0.150000}%
\pgfsetstrokecolor{textcolor}%
\pgfsetfillcolor{textcolor}%
\pgftext[x=3.646156in,y=0.234413in,,top]{\color{textcolor}\rmfamily\fontsize{10.000000}{12.000000}\selectfont 2018}%
\end{pgfscope}%
\begin{pgfscope}%
\pgfpathrectangle{\pgfqpoint{0.683198in}{0.331635in}}{\pgfqpoint{3.100000in}{1.510000in}}%
\pgfusepath{clip}%
\pgfsetroundcap%
\pgfsetroundjoin%
\pgfsetlinewidth{0.803000pt}%
\definecolor{currentstroke}{rgb}{1.000000,1.000000,1.000000}%
\pgfsetstrokecolor{currentstroke}%
\pgfsetdash{}{0pt}%
\pgfpathmoveto{\pgfqpoint{0.683198in}{0.400271in}}%
\pgfpathlineto{\pgfqpoint{3.783198in}{0.400271in}}%
\pgfusepath{stroke}%
\end{pgfscope}%
\begin{pgfscope}%
\definecolor{textcolor}{rgb}{0.150000,0.150000,0.150000}%
\pgfsetstrokecolor{textcolor}%
\pgfsetfillcolor{textcolor}%
\pgftext[x=0.100000in,y=0.347510in,left,base]{\color{textcolor}\rmfamily\fontsize{10.000000}{12.000000}\selectfont 0.0000}%
\end{pgfscope}%
\begin{pgfscope}%
\pgfpathrectangle{\pgfqpoint{0.683198in}{0.331635in}}{\pgfqpoint{3.100000in}{1.510000in}}%
\pgfusepath{clip}%
\pgfsetroundcap%
\pgfsetroundjoin%
\pgfsetlinewidth{0.803000pt}%
\definecolor{currentstroke}{rgb}{1.000000,1.000000,1.000000}%
\pgfsetstrokecolor{currentstroke}%
\pgfsetdash{}{0pt}%
\pgfpathmoveto{\pgfqpoint{0.683198in}{0.760926in}}%
\pgfpathlineto{\pgfqpoint{3.783198in}{0.760926in}}%
\pgfusepath{stroke}%
\end{pgfscope}%
\begin{pgfscope}%
\definecolor{textcolor}{rgb}{0.150000,0.150000,0.150000}%
\pgfsetstrokecolor{textcolor}%
\pgfsetfillcolor{textcolor}%
\pgftext[x=0.100000in,y=0.708165in,left,base]{\color{textcolor}\rmfamily\fontsize{10.000000}{12.000000}\selectfont 0.0025}%
\end{pgfscope}%
\begin{pgfscope}%
\pgfpathrectangle{\pgfqpoint{0.683198in}{0.331635in}}{\pgfqpoint{3.100000in}{1.510000in}}%
\pgfusepath{clip}%
\pgfsetroundcap%
\pgfsetroundjoin%
\pgfsetlinewidth{0.803000pt}%
\definecolor{currentstroke}{rgb}{1.000000,1.000000,1.000000}%
\pgfsetstrokecolor{currentstroke}%
\pgfsetdash{}{0pt}%
\pgfpathmoveto{\pgfqpoint{0.683198in}{1.121581in}}%
\pgfpathlineto{\pgfqpoint{3.783198in}{1.121581in}}%
\pgfusepath{stroke}%
\end{pgfscope}%
\begin{pgfscope}%
\definecolor{textcolor}{rgb}{0.150000,0.150000,0.150000}%
\pgfsetstrokecolor{textcolor}%
\pgfsetfillcolor{textcolor}%
\pgftext[x=0.100000in,y=1.068820in,left,base]{\color{textcolor}\rmfamily\fontsize{10.000000}{12.000000}\selectfont 0.0050}%
\end{pgfscope}%
\begin{pgfscope}%
\pgfpathrectangle{\pgfqpoint{0.683198in}{0.331635in}}{\pgfqpoint{3.100000in}{1.510000in}}%
\pgfusepath{clip}%
\pgfsetroundcap%
\pgfsetroundjoin%
\pgfsetlinewidth{0.803000pt}%
\definecolor{currentstroke}{rgb}{1.000000,1.000000,1.000000}%
\pgfsetstrokecolor{currentstroke}%
\pgfsetdash{}{0pt}%
\pgfpathmoveto{\pgfqpoint{0.683198in}{1.482236in}}%
\pgfpathlineto{\pgfqpoint{3.783198in}{1.482236in}}%
\pgfusepath{stroke}%
\end{pgfscope}%
\begin{pgfscope}%
\definecolor{textcolor}{rgb}{0.150000,0.150000,0.150000}%
\pgfsetstrokecolor{textcolor}%
\pgfsetfillcolor{textcolor}%
\pgftext[x=0.100000in,y=1.429475in,left,base]{\color{textcolor}\rmfamily\fontsize{10.000000}{12.000000}\selectfont 0.0075}%
\end{pgfscope}%
\begin{pgfscope}%
\pgfpathrectangle{\pgfqpoint{0.683198in}{0.331635in}}{\pgfqpoint{3.100000in}{1.510000in}}%
\pgfusepath{clip}%
\pgfsetroundcap%
\pgfsetroundjoin%
\pgfsetlinewidth{1.505625pt}%
\definecolor{currentstroke}{rgb}{0.737255,0.741176,0.133333}%
\pgfsetstrokecolor{currentstroke}%
\pgfsetdash{}{0pt}%
\pgfpathmoveto{\pgfqpoint{0.824107in}{0.446565in}}%
\pgfpathlineto{\pgfqpoint{0.825396in}{0.408720in}}%
\pgfpathlineto{\pgfqpoint{0.826685in}{0.420113in}}%
\pgfpathlineto{\pgfqpoint{0.830553in}{0.409973in}}%
\pgfpathlineto{\pgfqpoint{0.831842in}{0.400392in}}%
\pgfpathlineto{\pgfqpoint{0.833131in}{0.407123in}}%
\pgfpathlineto{\pgfqpoint{0.834420in}{0.469389in}}%
\pgfpathlineto{\pgfqpoint{0.835709in}{0.405261in}}%
\pgfpathlineto{\pgfqpoint{0.840866in}{0.444656in}}%
\pgfpathlineto{\pgfqpoint{0.842155in}{0.422475in}}%
\pgfpathlineto{\pgfqpoint{0.843445in}{0.427524in}}%
\pgfpathlineto{\pgfqpoint{0.844734in}{0.444776in}}%
\pgfpathlineto{\pgfqpoint{0.848601in}{0.414830in}}%
\pgfpathlineto{\pgfqpoint{0.849891in}{0.429060in}}%
\pgfpathlineto{\pgfqpoint{0.851180in}{0.403238in}}%
\pgfpathlineto{\pgfqpoint{0.853758in}{0.400744in}}%
\pgfpathlineto{\pgfqpoint{0.857626in}{0.420458in}}%
\pgfpathlineto{\pgfqpoint{0.858915in}{0.408937in}}%
\pgfpathlineto{\pgfqpoint{0.860204in}{0.449369in}}%
\pgfpathlineto{\pgfqpoint{0.861493in}{0.568847in}}%
\pgfpathlineto{\pgfqpoint{0.862783in}{0.411975in}}%
\pgfpathlineto{\pgfqpoint{0.866650in}{0.401556in}}%
\pgfpathlineto{\pgfqpoint{0.867939in}{0.401950in}}%
\pgfpathlineto{\pgfqpoint{0.869229in}{0.423663in}}%
\pgfpathlineto{\pgfqpoint{0.870518in}{0.596163in}}%
\pgfpathlineto{\pgfqpoint{0.871807in}{0.424368in}}%
\pgfpathlineto{\pgfqpoint{0.875675in}{0.416141in}}%
\pgfpathlineto{\pgfqpoint{0.876964in}{0.470464in}}%
\pgfpathlineto{\pgfqpoint{0.878253in}{0.406787in}}%
\pgfpathlineto{\pgfqpoint{0.879542in}{0.419324in}}%
\pgfpathlineto{\pgfqpoint{0.885988in}{0.407591in}}%
\pgfpathlineto{\pgfqpoint{0.887277in}{0.451712in}}%
\pgfpathlineto{\pgfqpoint{0.888567in}{0.400359in}}%
\pgfpathlineto{\pgfqpoint{0.889856in}{0.412755in}}%
\pgfpathlineto{\pgfqpoint{0.893723in}{0.405131in}}%
\pgfpathlineto{\pgfqpoint{0.896302in}{0.469907in}}%
\pgfpathlineto{\pgfqpoint{0.897591in}{0.407321in}}%
\pgfpathlineto{\pgfqpoint{0.898880in}{0.411803in}}%
\pgfpathlineto{\pgfqpoint{0.902748in}{0.400359in}}%
\pgfpathlineto{\pgfqpoint{0.904037in}{0.421634in}}%
\pgfpathlineto{\pgfqpoint{0.905326in}{0.407511in}}%
\pgfpathlineto{\pgfqpoint{0.906615in}{0.456843in}}%
\pgfpathlineto{\pgfqpoint{0.907904in}{0.407228in}}%
\pgfpathlineto{\pgfqpoint{0.911772in}{0.404532in}}%
\pgfpathlineto{\pgfqpoint{0.913061in}{0.406547in}}%
\pgfpathlineto{\pgfqpoint{0.915640in}{0.401053in}}%
\pgfpathlineto{\pgfqpoint{0.916929in}{0.401336in}}%
\pgfpathlineto{\pgfqpoint{0.920796in}{0.447542in}}%
\pgfpathlineto{\pgfqpoint{0.922086in}{0.456017in}}%
\pgfpathlineto{\pgfqpoint{0.923375in}{0.400817in}}%
\pgfpathlineto{\pgfqpoint{0.924664in}{0.403392in}}%
\pgfpathlineto{\pgfqpoint{0.925953in}{0.423467in}}%
\pgfpathlineto{\pgfqpoint{0.929821in}{0.416586in}}%
\pgfpathlineto{\pgfqpoint{0.931110in}{0.400787in}}%
\pgfpathlineto{\pgfqpoint{0.932399in}{0.402783in}}%
\pgfpathlineto{\pgfqpoint{0.933688in}{0.401295in}}%
\pgfpathlineto{\pgfqpoint{0.934978in}{0.410498in}}%
\pgfpathlineto{\pgfqpoint{0.938845in}{0.409593in}}%
\pgfpathlineto{\pgfqpoint{0.940134in}{0.420186in}}%
\pgfpathlineto{\pgfqpoint{0.941424in}{0.420186in}}%
\pgfpathlineto{\pgfqpoint{0.942713in}{0.442014in}}%
\pgfpathlineto{\pgfqpoint{0.947870in}{0.422618in}}%
\pgfpathlineto{\pgfqpoint{0.949159in}{0.479413in}}%
\pgfpathlineto{\pgfqpoint{0.950448in}{0.404516in}}%
\pgfpathlineto{\pgfqpoint{0.951737in}{0.533977in}}%
\pgfpathlineto{\pgfqpoint{0.953026in}{0.444264in}}%
\pgfpathlineto{\pgfqpoint{0.956894in}{0.452148in}}%
\pgfpathlineto{\pgfqpoint{0.958183in}{0.416042in}}%
\pgfpathlineto{\pgfqpoint{0.959472in}{0.401249in}}%
\pgfpathlineto{\pgfqpoint{0.960761in}{0.403670in}}%
\pgfpathlineto{\pgfqpoint{0.962051in}{0.400595in}}%
\pgfpathlineto{\pgfqpoint{0.965918in}{0.485319in}}%
\pgfpathlineto{\pgfqpoint{0.967207in}{0.407117in}}%
\pgfpathlineto{\pgfqpoint{0.968497in}{0.481599in}}%
\pgfpathlineto{\pgfqpoint{0.969786in}{0.419282in}}%
\pgfpathlineto{\pgfqpoint{0.971075in}{0.401227in}}%
\pgfpathlineto{\pgfqpoint{0.974943in}{0.403085in}}%
\pgfpathlineto{\pgfqpoint{0.976232in}{0.400448in}}%
\pgfpathlineto{\pgfqpoint{0.977521in}{0.408164in}}%
\pgfpathlineto{\pgfqpoint{0.978810in}{0.736668in}}%
\pgfpathlineto{\pgfqpoint{0.980099in}{0.419747in}}%
\pgfpathlineto{\pgfqpoint{0.985256in}{0.401628in}}%
\pgfpathlineto{\pgfqpoint{0.986545in}{0.405747in}}%
\pgfpathlineto{\pgfqpoint{0.987835in}{0.405747in}}%
\pgfpathlineto{\pgfqpoint{0.989124in}{0.400804in}}%
\pgfpathlineto{\pgfqpoint{0.992991in}{0.415992in}}%
\pgfpathlineto{\pgfqpoint{0.994281in}{0.400359in}}%
\pgfpathlineto{\pgfqpoint{0.996859in}{0.468084in}}%
\pgfpathlineto{\pgfqpoint{0.998148in}{0.463660in}}%
\pgfpathlineto{\pgfqpoint{1.002016in}{0.543178in}}%
\pgfpathlineto{\pgfqpoint{1.003305in}{0.449217in}}%
\pgfpathlineto{\pgfqpoint{1.004594in}{0.406260in}}%
\pgfpathlineto{\pgfqpoint{1.007173in}{0.401927in}}%
\pgfpathlineto{\pgfqpoint{1.012329in}{0.408415in}}%
\pgfpathlineto{\pgfqpoint{1.013618in}{0.479784in}}%
\pgfpathlineto{\pgfqpoint{1.014908in}{0.456449in}}%
\pgfpathlineto{\pgfqpoint{1.016197in}{0.495768in}}%
\pgfpathlineto{\pgfqpoint{1.020064in}{0.440505in}}%
\pgfpathlineto{\pgfqpoint{1.021354in}{0.401712in}}%
\pgfpathlineto{\pgfqpoint{1.022643in}{0.471992in}}%
\pgfpathlineto{\pgfqpoint{1.023932in}{0.402422in}}%
\pgfpathlineto{\pgfqpoint{1.025221in}{0.400808in}}%
\pgfpathlineto{\pgfqpoint{1.029089in}{0.400810in}}%
\pgfpathlineto{\pgfqpoint{1.030378in}{0.407218in}}%
\pgfpathlineto{\pgfqpoint{1.031667in}{0.454541in}}%
\pgfpathlineto{\pgfqpoint{1.032956in}{0.427018in}}%
\pgfpathlineto{\pgfqpoint{1.034246in}{0.437658in}}%
\pgfpathlineto{\pgfqpoint{1.038113in}{0.429959in}}%
\pgfpathlineto{\pgfqpoint{1.039402in}{0.415995in}}%
\pgfpathlineto{\pgfqpoint{1.040692in}{0.414597in}}%
\pgfpathlineto{\pgfqpoint{1.041981in}{0.495305in}}%
\pgfpathlineto{\pgfqpoint{1.043270in}{0.691664in}}%
\pgfpathlineto{\pgfqpoint{1.047138in}{0.529968in}}%
\pgfpathlineto{\pgfqpoint{1.048427in}{0.434796in}}%
\pgfpathlineto{\pgfqpoint{1.049716in}{0.400964in}}%
\pgfpathlineto{\pgfqpoint{1.051005in}{0.434644in}}%
\pgfpathlineto{\pgfqpoint{1.052294in}{0.443236in}}%
\pgfpathlineto{\pgfqpoint{1.056162in}{0.481590in}}%
\pgfpathlineto{\pgfqpoint{1.057451in}{0.400929in}}%
\pgfpathlineto{\pgfqpoint{1.060030in}{0.404369in}}%
\pgfpathlineto{\pgfqpoint{1.061319in}{0.428210in}}%
\pgfpathlineto{\pgfqpoint{1.065186in}{0.424727in}}%
\pgfpathlineto{\pgfqpoint{1.066476in}{0.417667in}}%
\pgfpathlineto{\pgfqpoint{1.067765in}{0.424523in}}%
\pgfpathlineto{\pgfqpoint{1.069054in}{0.436768in}}%
\pgfpathlineto{\pgfqpoint{1.070343in}{0.419934in}}%
\pgfpathlineto{\pgfqpoint{1.074211in}{0.485915in}}%
\pgfpathlineto{\pgfqpoint{1.075500in}{0.410596in}}%
\pgfpathlineto{\pgfqpoint{1.076789in}{0.401716in}}%
\pgfpathlineto{\pgfqpoint{1.078078in}{0.453436in}}%
\pgfpathlineto{\pgfqpoint{1.079367in}{0.401457in}}%
\pgfpathlineto{\pgfqpoint{1.083235in}{0.415976in}}%
\pgfpathlineto{\pgfqpoint{1.084524in}{0.430905in}}%
\pgfpathlineto{\pgfqpoint{1.085813in}{0.402221in}}%
\pgfpathlineto{\pgfqpoint{1.087103in}{0.592733in}}%
\pgfpathlineto{\pgfqpoint{1.088392in}{0.450428in}}%
\pgfpathlineto{\pgfqpoint{1.092259in}{0.425248in}}%
\pgfpathlineto{\pgfqpoint{1.093549in}{0.427966in}}%
\pgfpathlineto{\pgfqpoint{1.094838in}{0.422037in}}%
\pgfpathlineto{\pgfqpoint{1.096127in}{0.428650in}}%
\pgfpathlineto{\pgfqpoint{1.097416in}{0.425196in}}%
\pgfpathlineto{\pgfqpoint{1.101284in}{0.400288in}}%
\pgfpathlineto{\pgfqpoint{1.102573in}{0.403624in}}%
\pgfpathlineto{\pgfqpoint{1.103862in}{0.405210in}}%
\pgfpathlineto{\pgfqpoint{1.105151in}{0.460631in}}%
\pgfpathlineto{\pgfqpoint{1.106441in}{0.403249in}}%
\pgfpathlineto{\pgfqpoint{1.110308in}{0.405370in}}%
\pgfpathlineto{\pgfqpoint{1.111597in}{0.421749in}}%
\pgfpathlineto{\pgfqpoint{1.112887in}{0.400289in}}%
\pgfpathlineto{\pgfqpoint{1.114176in}{0.403647in}}%
\pgfpathlineto{\pgfqpoint{1.115465in}{0.409412in}}%
\pgfpathlineto{\pgfqpoint{1.119333in}{0.402387in}}%
\pgfpathlineto{\pgfqpoint{1.120622in}{0.404241in}}%
\pgfpathlineto{\pgfqpoint{1.121911in}{0.403255in}}%
\pgfpathlineto{\pgfqpoint{1.123200in}{0.408079in}}%
\pgfpathlineto{\pgfqpoint{1.124489in}{0.402438in}}%
\pgfpathlineto{\pgfqpoint{1.128357in}{0.408904in}}%
\pgfpathlineto{\pgfqpoint{1.129646in}{0.400715in}}%
\pgfpathlineto{\pgfqpoint{1.130935in}{0.403259in}}%
\pgfpathlineto{\pgfqpoint{1.133514in}{0.422006in}}%
\pgfpathlineto{\pgfqpoint{1.138670in}{0.400901in}}%
\pgfpathlineto{\pgfqpoint{1.139960in}{0.408025in}}%
\pgfpathlineto{\pgfqpoint{1.141249in}{0.432532in}}%
\pgfpathlineto{\pgfqpoint{1.142538in}{0.400546in}}%
\pgfpathlineto{\pgfqpoint{1.146406in}{0.409412in}}%
\pgfpathlineto{\pgfqpoint{1.147695in}{0.467915in}}%
\pgfpathlineto{\pgfqpoint{1.148984in}{0.432085in}}%
\pgfpathlineto{\pgfqpoint{1.151562in}{0.404345in}}%
\pgfpathlineto{\pgfqpoint{1.156719in}{0.400529in}}%
\pgfpathlineto{\pgfqpoint{1.158008in}{0.410273in}}%
\pgfpathlineto{\pgfqpoint{1.159298in}{0.400526in}}%
\pgfpathlineto{\pgfqpoint{1.160587in}{0.401556in}}%
\pgfpathlineto{\pgfqpoint{1.164454in}{0.413703in}}%
\pgfpathlineto{\pgfqpoint{1.165744in}{0.406056in}}%
\pgfpathlineto{\pgfqpoint{1.167033in}{0.434486in}}%
\pgfpathlineto{\pgfqpoint{1.169611in}{0.401570in}}%
\pgfpathlineto{\pgfqpoint{1.173479in}{0.446063in}}%
\pgfpathlineto{\pgfqpoint{1.174768in}{0.404241in}}%
\pgfpathlineto{\pgfqpoint{1.176057in}{0.448275in}}%
\pgfpathlineto{\pgfqpoint{1.177346in}{0.410348in}}%
\pgfpathlineto{\pgfqpoint{1.178636in}{0.402759in}}%
\pgfpathlineto{\pgfqpoint{1.182503in}{0.417403in}}%
\pgfpathlineto{\pgfqpoint{1.183792in}{0.429694in}}%
\pgfpathlineto{\pgfqpoint{1.185082in}{0.403723in}}%
\pgfpathlineto{\pgfqpoint{1.186371in}{0.419828in}}%
\pgfpathlineto{\pgfqpoint{1.187660in}{0.400331in}}%
\pgfpathlineto{\pgfqpoint{1.191528in}{0.404553in}}%
\pgfpathlineto{\pgfqpoint{1.192817in}{0.411721in}}%
\pgfpathlineto{\pgfqpoint{1.194106in}{0.414944in}}%
\pgfpathlineto{\pgfqpoint{1.195395in}{0.403478in}}%
\pgfpathlineto{\pgfqpoint{1.196684in}{0.427112in}}%
\pgfpathlineto{\pgfqpoint{1.200552in}{0.407444in}}%
\pgfpathlineto{\pgfqpoint{1.201841in}{0.442921in}}%
\pgfpathlineto{\pgfqpoint{1.203130in}{0.400411in}}%
\pgfpathlineto{\pgfqpoint{1.204419in}{0.409117in}}%
\pgfpathlineto{\pgfqpoint{1.205709in}{0.404651in}}%
\pgfpathlineto{\pgfqpoint{1.212155in}{0.401774in}}%
\pgfpathlineto{\pgfqpoint{1.213444in}{0.587946in}}%
\pgfpathlineto{\pgfqpoint{1.214733in}{0.401963in}}%
\pgfpathlineto{\pgfqpoint{1.218601in}{0.425342in}}%
\pgfpathlineto{\pgfqpoint{1.219890in}{0.440058in}}%
\pgfpathlineto{\pgfqpoint{1.221179in}{0.412094in}}%
\pgfpathlineto{\pgfqpoint{1.222468in}{0.401994in}}%
\pgfpathlineto{\pgfqpoint{1.223757in}{0.405397in}}%
\pgfpathlineto{\pgfqpoint{1.227625in}{0.400961in}}%
\pgfpathlineto{\pgfqpoint{1.230203in}{0.409958in}}%
\pgfpathlineto{\pgfqpoint{1.231493in}{0.400286in}}%
\pgfpathlineto{\pgfqpoint{1.232782in}{0.452976in}}%
\pgfpathlineto{\pgfqpoint{1.236649in}{0.459587in}}%
\pgfpathlineto{\pgfqpoint{1.237939in}{0.403671in}}%
\pgfpathlineto{\pgfqpoint{1.239228in}{0.400918in}}%
\pgfpathlineto{\pgfqpoint{1.241806in}{0.414460in}}%
\pgfpathlineto{\pgfqpoint{1.246963in}{0.401581in}}%
\pgfpathlineto{\pgfqpoint{1.248252in}{0.401856in}}%
\pgfpathlineto{\pgfqpoint{1.249541in}{0.405460in}}%
\pgfpathlineto{\pgfqpoint{1.250831in}{0.415902in}}%
\pgfpathlineto{\pgfqpoint{1.254698in}{0.407597in}}%
\pgfpathlineto{\pgfqpoint{1.255987in}{0.407702in}}%
\pgfpathlineto{\pgfqpoint{1.257276in}{0.401838in}}%
\pgfpathlineto{\pgfqpoint{1.259855in}{0.400323in}}%
\pgfpathlineto{\pgfqpoint{1.265012in}{0.400732in}}%
\pgfpathlineto{\pgfqpoint{1.266301in}{0.407070in}}%
\pgfpathlineto{\pgfqpoint{1.268879in}{0.401110in}}%
\pgfpathlineto{\pgfqpoint{1.272747in}{0.435219in}}%
\pgfpathlineto{\pgfqpoint{1.274036in}{0.416589in}}%
\pgfpathlineto{\pgfqpoint{1.275325in}{0.415690in}}%
\pgfpathlineto{\pgfqpoint{1.276614in}{0.468413in}}%
\pgfpathlineto{\pgfqpoint{1.277904in}{0.418051in}}%
\pgfpathlineto{\pgfqpoint{1.281771in}{0.401277in}}%
\pgfpathlineto{\pgfqpoint{1.284350in}{0.413073in}}%
\pgfpathlineto{\pgfqpoint{1.285639in}{0.401081in}}%
\pgfpathlineto{\pgfqpoint{1.286928in}{0.403954in}}%
\pgfpathlineto{\pgfqpoint{1.293374in}{0.489148in}}%
\pgfpathlineto{\pgfqpoint{1.294663in}{0.400377in}}%
\pgfpathlineto{\pgfqpoint{1.295952in}{0.409374in}}%
\pgfpathlineto{\pgfqpoint{1.299820in}{0.408001in}}%
\pgfpathlineto{\pgfqpoint{1.301109in}{0.412519in}}%
\pgfpathlineto{\pgfqpoint{1.302398in}{0.433473in}}%
\pgfpathlineto{\pgfqpoint{1.303688in}{0.409436in}}%
\pgfpathlineto{\pgfqpoint{1.304977in}{0.402416in}}%
\pgfpathlineto{\pgfqpoint{1.308844in}{0.401594in}}%
\pgfpathlineto{\pgfqpoint{1.310134in}{0.400546in}}%
\pgfpathlineto{\pgfqpoint{1.312712in}{0.400813in}}%
\pgfpathlineto{\pgfqpoint{1.314001in}{0.415583in}}%
\pgfpathlineto{\pgfqpoint{1.320447in}{0.400271in}}%
\pgfpathlineto{\pgfqpoint{1.321736in}{0.401621in}}%
\pgfpathlineto{\pgfqpoint{1.323025in}{0.400815in}}%
\pgfpathlineto{\pgfqpoint{1.326893in}{0.470940in}}%
\pgfpathlineto{\pgfqpoint{1.328182in}{0.400838in}}%
\pgfpathlineto{\pgfqpoint{1.329471in}{0.416225in}}%
\pgfpathlineto{\pgfqpoint{1.330761in}{0.449118in}}%
\pgfpathlineto{\pgfqpoint{1.332050in}{0.402815in}}%
\pgfpathlineto{\pgfqpoint{1.335917in}{0.421337in}}%
\pgfpathlineto{\pgfqpoint{1.337207in}{0.448111in}}%
\pgfpathlineto{\pgfqpoint{1.338496in}{0.409548in}}%
\pgfpathlineto{\pgfqpoint{1.339785in}{0.481219in}}%
\pgfpathlineto{\pgfqpoint{1.341074in}{0.403571in}}%
\pgfpathlineto{\pgfqpoint{1.344942in}{0.413534in}}%
\pgfpathlineto{\pgfqpoint{1.346231in}{0.401669in}}%
\pgfpathlineto{\pgfqpoint{1.347520in}{0.403232in}}%
\pgfpathlineto{\pgfqpoint{1.348809in}{0.408675in}}%
\pgfpathlineto{\pgfqpoint{1.350099in}{0.424170in}}%
\pgfpathlineto{\pgfqpoint{1.355255in}{0.400987in}}%
\pgfpathlineto{\pgfqpoint{1.356545in}{0.428666in}}%
\pgfpathlineto{\pgfqpoint{1.357834in}{0.424331in}}%
\pgfpathlineto{\pgfqpoint{1.359123in}{0.422719in}}%
\pgfpathlineto{\pgfqpoint{1.362991in}{0.473738in}}%
\pgfpathlineto{\pgfqpoint{1.364280in}{0.423174in}}%
\pgfpathlineto{\pgfqpoint{1.365569in}{0.423605in}}%
\pgfpathlineto{\pgfqpoint{1.366858in}{0.408874in}}%
\pgfpathlineto{\pgfqpoint{1.368147in}{0.402147in}}%
\pgfpathlineto{\pgfqpoint{1.372015in}{0.412290in}}%
\pgfpathlineto{\pgfqpoint{1.373304in}{0.403797in}}%
\pgfpathlineto{\pgfqpoint{1.374593in}{0.403386in}}%
\pgfpathlineto{\pgfqpoint{1.377172in}{0.400368in}}%
\pgfpathlineto{\pgfqpoint{1.381039in}{0.401819in}}%
\pgfpathlineto{\pgfqpoint{1.382328in}{0.411336in}}%
\pgfpathlineto{\pgfqpoint{1.383618in}{0.401156in}}%
\pgfpathlineto{\pgfqpoint{1.384907in}{0.409411in}}%
\pgfpathlineto{\pgfqpoint{1.386196in}{0.425441in}}%
\pgfpathlineto{\pgfqpoint{1.390064in}{0.401615in}}%
\pgfpathlineto{\pgfqpoint{1.391353in}{0.425199in}}%
\pgfpathlineto{\pgfqpoint{1.392642in}{0.464929in}}%
\pgfpathlineto{\pgfqpoint{1.393931in}{0.413828in}}%
\pgfpathlineto{\pgfqpoint{1.399088in}{0.484225in}}%
\pgfpathlineto{\pgfqpoint{1.400377in}{0.463505in}}%
\pgfpathlineto{\pgfqpoint{1.401666in}{0.407454in}}%
\pgfpathlineto{\pgfqpoint{1.402956in}{0.407354in}}%
\pgfpathlineto{\pgfqpoint{1.408112in}{0.432084in}}%
\pgfpathlineto{\pgfqpoint{1.409402in}{0.402212in}}%
\pgfpathlineto{\pgfqpoint{1.410691in}{0.453866in}}%
\pgfpathlineto{\pgfqpoint{1.411980in}{0.415706in}}%
\pgfpathlineto{\pgfqpoint{1.413269in}{0.409382in}}%
\pgfpathlineto{\pgfqpoint{1.417137in}{0.407122in}}%
\pgfpathlineto{\pgfqpoint{1.418426in}{0.412715in}}%
\pgfpathlineto{\pgfqpoint{1.419715in}{0.426582in}}%
\pgfpathlineto{\pgfqpoint{1.421004in}{0.402220in}}%
\pgfpathlineto{\pgfqpoint{1.422294in}{0.417941in}}%
\pgfpathlineto{\pgfqpoint{1.426161in}{0.508569in}}%
\pgfpathlineto{\pgfqpoint{1.428740in}{0.432930in}}%
\pgfpathlineto{\pgfqpoint{1.430029in}{0.404108in}}%
\pgfpathlineto{\pgfqpoint{1.431318in}{0.435645in}}%
\pgfpathlineto{\pgfqpoint{1.435185in}{0.401309in}}%
\pgfpathlineto{\pgfqpoint{1.436475in}{0.424936in}}%
\pgfpathlineto{\pgfqpoint{1.437764in}{0.410567in}}%
\pgfpathlineto{\pgfqpoint{1.439053in}{0.414525in}}%
\pgfpathlineto{\pgfqpoint{1.444210in}{0.402501in}}%
\pgfpathlineto{\pgfqpoint{1.445499in}{0.401934in}}%
\pgfpathlineto{\pgfqpoint{1.446788in}{0.431471in}}%
\pgfpathlineto{\pgfqpoint{1.448077in}{0.837763in}}%
\pgfpathlineto{\pgfqpoint{1.449367in}{0.478339in}}%
\pgfpathlineto{\pgfqpoint{1.453234in}{0.402221in}}%
\pgfpathlineto{\pgfqpoint{1.454523in}{0.404087in}}%
\pgfpathlineto{\pgfqpoint{1.455813in}{0.401731in}}%
\pgfpathlineto{\pgfqpoint{1.457102in}{0.402228in}}%
\pgfpathlineto{\pgfqpoint{1.458391in}{0.400411in}}%
\pgfpathlineto{\pgfqpoint{1.462259in}{0.400411in}}%
\pgfpathlineto{\pgfqpoint{1.463548in}{0.409102in}}%
\pgfpathlineto{\pgfqpoint{1.464837in}{0.433720in}}%
\pgfpathlineto{\pgfqpoint{1.466126in}{0.417262in}}%
\pgfpathlineto{\pgfqpoint{1.467415in}{0.498449in}}%
\pgfpathlineto{\pgfqpoint{1.471283in}{0.438107in}}%
\pgfpathlineto{\pgfqpoint{1.472572in}{0.400280in}}%
\pgfpathlineto{\pgfqpoint{1.476440in}{0.425151in}}%
\pgfpathlineto{\pgfqpoint{1.481597in}{0.401115in}}%
\pgfpathlineto{\pgfqpoint{1.484175in}{0.437312in}}%
\pgfpathlineto{\pgfqpoint{1.485464in}{0.437312in}}%
\pgfpathlineto{\pgfqpoint{1.489332in}{0.419142in}}%
\pgfpathlineto{\pgfqpoint{1.490621in}{0.400271in}}%
\pgfpathlineto{\pgfqpoint{1.491910in}{0.439975in}}%
\pgfpathlineto{\pgfqpoint{1.494488in}{0.401707in}}%
\pgfpathlineto{\pgfqpoint{1.499645in}{0.431418in}}%
\pgfpathlineto{\pgfqpoint{1.500934in}{0.401498in}}%
\pgfpathlineto{\pgfqpoint{1.502224in}{0.442611in}}%
\pgfpathlineto{\pgfqpoint{1.503513in}{0.407756in}}%
\pgfpathlineto{\pgfqpoint{1.507380in}{0.408783in}}%
\pgfpathlineto{\pgfqpoint{1.508670in}{0.412702in}}%
\pgfpathlineto{\pgfqpoint{1.509959in}{0.408626in}}%
\pgfpathlineto{\pgfqpoint{1.511248in}{0.477811in}}%
\pgfpathlineto{\pgfqpoint{1.512537in}{0.405203in}}%
\pgfpathlineto{\pgfqpoint{1.516405in}{0.406990in}}%
\pgfpathlineto{\pgfqpoint{1.517694in}{0.421581in}}%
\pgfpathlineto{\pgfqpoint{1.518983in}{0.421072in}}%
\pgfpathlineto{\pgfqpoint{1.520272in}{0.411427in}}%
\pgfpathlineto{\pgfqpoint{1.521562in}{0.409685in}}%
\pgfpathlineto{\pgfqpoint{1.525429in}{0.425676in}}%
\pgfpathlineto{\pgfqpoint{1.526718in}{0.400279in}}%
\pgfpathlineto{\pgfqpoint{1.528008in}{0.413575in}}%
\pgfpathlineto{\pgfqpoint{1.530586in}{0.457104in}}%
\pgfpathlineto{\pgfqpoint{1.534454in}{0.429646in}}%
\pgfpathlineto{\pgfqpoint{1.535743in}{0.403386in}}%
\pgfpathlineto{\pgfqpoint{1.537032in}{0.401401in}}%
\pgfpathlineto{\pgfqpoint{1.538321in}{0.443765in}}%
\pgfpathlineto{\pgfqpoint{1.539610in}{0.402459in}}%
\pgfpathlineto{\pgfqpoint{1.543478in}{0.401027in}}%
\pgfpathlineto{\pgfqpoint{1.544767in}{0.403315in}}%
\pgfpathlineto{\pgfqpoint{1.546056in}{0.400546in}}%
\pgfpathlineto{\pgfqpoint{1.547346in}{0.407546in}}%
\pgfpathlineto{\pgfqpoint{1.548635in}{0.404996in}}%
\pgfpathlineto{\pgfqpoint{1.552502in}{0.407067in}}%
\pgfpathlineto{\pgfqpoint{1.553792in}{0.428569in}}%
\pgfpathlineto{\pgfqpoint{1.555081in}{0.414009in}}%
\pgfpathlineto{\pgfqpoint{1.556370in}{0.647429in}}%
\pgfpathlineto{\pgfqpoint{1.557659in}{0.408214in}}%
\pgfpathlineto{\pgfqpoint{1.561527in}{0.406069in}}%
\pgfpathlineto{\pgfqpoint{1.562816in}{0.401346in}}%
\pgfpathlineto{\pgfqpoint{1.564105in}{1.286399in}}%
\pgfpathlineto{\pgfqpoint{1.565394in}{0.421034in}}%
\pgfpathlineto{\pgfqpoint{1.566683in}{0.502771in}}%
\pgfpathlineto{\pgfqpoint{1.570551in}{0.401636in}}%
\pgfpathlineto{\pgfqpoint{1.571840in}{0.412002in}}%
\pgfpathlineto{\pgfqpoint{1.573129in}{0.410398in}}%
\pgfpathlineto{\pgfqpoint{1.574419in}{0.401911in}}%
\pgfpathlineto{\pgfqpoint{1.575708in}{0.404355in}}%
\pgfpathlineto{\pgfqpoint{1.579575in}{0.400280in}}%
\pgfpathlineto{\pgfqpoint{1.580865in}{0.401497in}}%
\pgfpathlineto{\pgfqpoint{1.582154in}{0.401123in}}%
\pgfpathlineto{\pgfqpoint{1.583443in}{0.490888in}}%
\pgfpathlineto{\pgfqpoint{1.584732in}{0.403511in}}%
\pgfpathlineto{\pgfqpoint{1.588600in}{0.416765in}}%
\pgfpathlineto{\pgfqpoint{1.589889in}{0.415259in}}%
\pgfpathlineto{\pgfqpoint{1.591178in}{0.526192in}}%
\pgfpathlineto{\pgfqpoint{1.592467in}{0.401490in}}%
\pgfpathlineto{\pgfqpoint{1.593757in}{0.400406in}}%
\pgfpathlineto{\pgfqpoint{1.597624in}{0.476442in}}%
\pgfpathlineto{\pgfqpoint{1.598913in}{0.403817in}}%
\pgfpathlineto{\pgfqpoint{1.600203in}{0.405373in}}%
\pgfpathlineto{\pgfqpoint{1.601492in}{0.400412in}}%
\pgfpathlineto{\pgfqpoint{1.602781in}{0.404547in}}%
\pgfpathlineto{\pgfqpoint{1.607938in}{0.431779in}}%
\pgfpathlineto{\pgfqpoint{1.609227in}{0.403399in}}%
\pgfpathlineto{\pgfqpoint{1.610516in}{0.400280in}}%
\pgfpathlineto{\pgfqpoint{1.611805in}{0.401139in}}%
\pgfpathlineto{\pgfqpoint{1.615673in}{0.416854in}}%
\pgfpathlineto{\pgfqpoint{1.616962in}{0.558649in}}%
\pgfpathlineto{\pgfqpoint{1.618251in}{0.412843in}}%
\pgfpathlineto{\pgfqpoint{1.619540in}{0.406862in}}%
\pgfpathlineto{\pgfqpoint{1.620830in}{0.464245in}}%
\pgfpathlineto{\pgfqpoint{1.624697in}{0.400883in}}%
\pgfpathlineto{\pgfqpoint{1.625986in}{0.419640in}}%
\pgfpathlineto{\pgfqpoint{1.627276in}{0.417049in}}%
\pgfpathlineto{\pgfqpoint{1.628565in}{0.404399in}}%
\pgfpathlineto{\pgfqpoint{1.629854in}{0.463262in}}%
\pgfpathlineto{\pgfqpoint{1.633722in}{0.425204in}}%
\pgfpathlineto{\pgfqpoint{1.635011in}{0.432245in}}%
\pgfpathlineto{\pgfqpoint{1.636300in}{0.412537in}}%
\pgfpathlineto{\pgfqpoint{1.637589in}{0.415694in}}%
\pgfpathlineto{\pgfqpoint{1.638878in}{0.401146in}}%
\pgfpathlineto{\pgfqpoint{1.644035in}{0.417838in}}%
\pgfpathlineto{\pgfqpoint{1.645324in}{0.408210in}}%
\pgfpathlineto{\pgfqpoint{1.646614in}{0.440064in}}%
\pgfpathlineto{\pgfqpoint{1.647903in}{0.413536in}}%
\pgfpathlineto{\pgfqpoint{1.651770in}{0.470441in}}%
\pgfpathlineto{\pgfqpoint{1.653060in}{0.460402in}}%
\pgfpathlineto{\pgfqpoint{1.654349in}{0.407517in}}%
\pgfpathlineto{\pgfqpoint{1.655638in}{0.512220in}}%
\pgfpathlineto{\pgfqpoint{1.656927in}{0.439901in}}%
\pgfpathlineto{\pgfqpoint{1.660795in}{0.406383in}}%
\pgfpathlineto{\pgfqpoint{1.662084in}{0.417087in}}%
\pgfpathlineto{\pgfqpoint{1.663373in}{0.469587in}}%
\pgfpathlineto{\pgfqpoint{1.664662in}{0.427047in}}%
\pgfpathlineto{\pgfqpoint{1.665952in}{0.417264in}}%
\pgfpathlineto{\pgfqpoint{1.669819in}{0.400945in}}%
\pgfpathlineto{\pgfqpoint{1.671108in}{0.400278in}}%
\pgfpathlineto{\pgfqpoint{1.672398in}{0.404517in}}%
\pgfpathlineto{\pgfqpoint{1.673687in}{0.458167in}}%
\pgfpathlineto{\pgfqpoint{1.674976in}{0.400330in}}%
\pgfpathlineto{\pgfqpoint{1.678843in}{0.400271in}}%
\pgfpathlineto{\pgfqpoint{1.680133in}{0.405021in}}%
\pgfpathlineto{\pgfqpoint{1.681422in}{0.400797in}}%
\pgfpathlineto{\pgfqpoint{1.682711in}{0.585993in}}%
\pgfpathlineto{\pgfqpoint{1.684000in}{0.423478in}}%
\pgfpathlineto{\pgfqpoint{1.687868in}{0.427682in}}%
\pgfpathlineto{\pgfqpoint{1.689157in}{0.403644in}}%
\pgfpathlineto{\pgfqpoint{1.690446in}{0.410202in}}%
\pgfpathlineto{\pgfqpoint{1.691735in}{0.432283in}}%
\pgfpathlineto{\pgfqpoint{1.693025in}{0.418345in}}%
\pgfpathlineto{\pgfqpoint{1.696892in}{0.403581in}}%
\pgfpathlineto{\pgfqpoint{1.698181in}{0.403890in}}%
\pgfpathlineto{\pgfqpoint{1.699471in}{0.436329in}}%
\pgfpathlineto{\pgfqpoint{1.700760in}{0.400697in}}%
\pgfpathlineto{\pgfqpoint{1.702049in}{0.412432in}}%
\pgfpathlineto{\pgfqpoint{1.705917in}{0.415441in}}%
\pgfpathlineto{\pgfqpoint{1.707206in}{0.423723in}}%
\pgfpathlineto{\pgfqpoint{1.708495in}{0.402236in}}%
\pgfpathlineto{\pgfqpoint{1.709784in}{0.443944in}}%
\pgfpathlineto{\pgfqpoint{1.711073in}{0.401211in}}%
\pgfpathlineto{\pgfqpoint{1.714941in}{0.400376in}}%
\pgfpathlineto{\pgfqpoint{1.716230in}{0.409127in}}%
\pgfpathlineto{\pgfqpoint{1.717519in}{0.401046in}}%
\pgfpathlineto{\pgfqpoint{1.720098in}{0.401354in}}%
\pgfpathlineto{\pgfqpoint{1.723965in}{0.409962in}}%
\pgfpathlineto{\pgfqpoint{1.725255in}{0.439354in}}%
\pgfpathlineto{\pgfqpoint{1.726544in}{0.403709in}}%
\pgfpathlineto{\pgfqpoint{1.727833in}{0.404015in}}%
\pgfpathlineto{\pgfqpoint{1.729122in}{0.400330in}}%
\pgfpathlineto{\pgfqpoint{1.732990in}{0.400507in}}%
\pgfpathlineto{\pgfqpoint{1.734279in}{0.416802in}}%
\pgfpathlineto{\pgfqpoint{1.735568in}{0.536525in}}%
\pgfpathlineto{\pgfqpoint{1.736857in}{0.417452in}}%
\pgfpathlineto{\pgfqpoint{1.738146in}{0.452461in}}%
\pgfpathlineto{\pgfqpoint{1.742014in}{0.400771in}}%
\pgfpathlineto{\pgfqpoint{1.743303in}{0.500126in}}%
\pgfpathlineto{\pgfqpoint{1.744592in}{0.413609in}}%
\pgfpathlineto{\pgfqpoint{1.745882in}{0.401923in}}%
\pgfpathlineto{\pgfqpoint{1.747171in}{0.400294in}}%
\pgfpathlineto{\pgfqpoint{1.751038in}{0.413279in}}%
\pgfpathlineto{\pgfqpoint{1.752328in}{0.401699in}}%
\pgfpathlineto{\pgfqpoint{1.754906in}{0.406653in}}%
\pgfpathlineto{\pgfqpoint{1.756195in}{0.401063in}}%
\pgfpathlineto{\pgfqpoint{1.760063in}{0.404879in}}%
\pgfpathlineto{\pgfqpoint{1.761352in}{0.409356in}}%
\pgfpathlineto{\pgfqpoint{1.763930in}{0.408489in}}%
\pgfpathlineto{\pgfqpoint{1.765220in}{0.400358in}}%
\pgfpathlineto{\pgfqpoint{1.769087in}{0.405526in}}%
\pgfpathlineto{\pgfqpoint{1.770376in}{0.408575in}}%
\pgfpathlineto{\pgfqpoint{1.771666in}{0.401654in}}%
\pgfpathlineto{\pgfqpoint{1.772955in}{0.400536in}}%
\pgfpathlineto{\pgfqpoint{1.774244in}{0.402027in}}%
\pgfpathlineto{\pgfqpoint{1.778112in}{0.415123in}}%
\pgfpathlineto{\pgfqpoint{1.779401in}{0.442510in}}%
\pgfpathlineto{\pgfqpoint{1.780690in}{0.403606in}}%
\pgfpathlineto{\pgfqpoint{1.781979in}{0.411609in}}%
\pgfpathlineto{\pgfqpoint{1.783268in}{0.705072in}}%
\pgfpathlineto{\pgfqpoint{1.788425in}{0.400449in}}%
\pgfpathlineto{\pgfqpoint{1.789714in}{0.403105in}}%
\pgfpathlineto{\pgfqpoint{1.792293in}{0.540127in}}%
\pgfpathlineto{\pgfqpoint{1.796160in}{0.476337in}}%
\pgfpathlineto{\pgfqpoint{1.797449in}{0.467599in}}%
\pgfpathlineto{\pgfqpoint{1.798739in}{0.444164in}}%
\pgfpathlineto{\pgfqpoint{1.800028in}{0.442229in}}%
\pgfpathlineto{\pgfqpoint{1.801317in}{0.490391in}}%
\pgfpathlineto{\pgfqpoint{1.805185in}{0.411978in}}%
\pgfpathlineto{\pgfqpoint{1.806474in}{0.403899in}}%
\pgfpathlineto{\pgfqpoint{1.807763in}{0.403863in}}%
\pgfpathlineto{\pgfqpoint{1.809052in}{0.435403in}}%
\pgfpathlineto{\pgfqpoint{1.810341in}{0.422690in}}%
\pgfpathlineto{\pgfqpoint{1.814209in}{0.404532in}}%
\pgfpathlineto{\pgfqpoint{1.815498in}{0.409385in}}%
\pgfpathlineto{\pgfqpoint{1.816787in}{0.435015in}}%
\pgfpathlineto{\pgfqpoint{1.818077in}{0.400318in}}%
\pgfpathlineto{\pgfqpoint{1.819366in}{0.411221in}}%
\pgfpathlineto{\pgfqpoint{1.824523in}{0.400276in}}%
\pgfpathlineto{\pgfqpoint{1.825812in}{0.413739in}}%
\pgfpathlineto{\pgfqpoint{1.827101in}{0.400795in}}%
\pgfpathlineto{\pgfqpoint{1.828390in}{0.400292in}}%
\pgfpathlineto{\pgfqpoint{1.832258in}{0.422864in}}%
\pgfpathlineto{\pgfqpoint{1.833547in}{0.402740in}}%
\pgfpathlineto{\pgfqpoint{1.834836in}{0.403211in}}%
\pgfpathlineto{\pgfqpoint{1.836125in}{0.400353in}}%
\pgfpathlineto{\pgfqpoint{1.837415in}{0.400599in}}%
\pgfpathlineto{\pgfqpoint{1.841282in}{0.458013in}}%
\pgfpathlineto{\pgfqpoint{1.842571in}{0.447628in}}%
\pgfpathlineto{\pgfqpoint{1.845150in}{0.403298in}}%
\pgfpathlineto{\pgfqpoint{1.846439in}{0.408576in}}%
\pgfpathlineto{\pgfqpoint{1.850307in}{0.400277in}}%
\pgfpathlineto{\pgfqpoint{1.851596in}{0.408052in}}%
\pgfpathlineto{\pgfqpoint{1.852885in}{0.403460in}}%
\pgfpathlineto{\pgfqpoint{1.854174in}{0.482294in}}%
\pgfpathlineto{\pgfqpoint{1.855463in}{0.400271in}}%
\pgfpathlineto{\pgfqpoint{1.859331in}{0.427007in}}%
\pgfpathlineto{\pgfqpoint{1.860620in}{0.419516in}}%
\pgfpathlineto{\pgfqpoint{1.861909in}{0.418887in}}%
\pgfpathlineto{\pgfqpoint{1.864488in}{0.407125in}}%
\pgfpathlineto{\pgfqpoint{1.868355in}{0.420044in}}%
\pgfpathlineto{\pgfqpoint{1.869644in}{0.417316in}}%
\pgfpathlineto{\pgfqpoint{1.870934in}{0.422350in}}%
\pgfpathlineto{\pgfqpoint{1.872223in}{0.400294in}}%
\pgfpathlineto{\pgfqpoint{1.873512in}{0.443612in}}%
\pgfpathlineto{\pgfqpoint{1.877380in}{0.445609in}}%
\pgfpathlineto{\pgfqpoint{1.878669in}{0.404393in}}%
\pgfpathlineto{\pgfqpoint{1.879958in}{0.400277in}}%
\pgfpathlineto{\pgfqpoint{1.881247in}{0.400635in}}%
\pgfpathlineto{\pgfqpoint{1.882536in}{0.573855in}}%
\pgfpathlineto{\pgfqpoint{1.886404in}{0.462294in}}%
\pgfpathlineto{\pgfqpoint{1.887693in}{0.402815in}}%
\pgfpathlineto{\pgfqpoint{1.888982in}{0.485582in}}%
\pgfpathlineto{\pgfqpoint{1.890272in}{0.524798in}}%
\pgfpathlineto{\pgfqpoint{1.891561in}{0.487705in}}%
\pgfpathlineto{\pgfqpoint{1.895428in}{0.469245in}}%
\pgfpathlineto{\pgfqpoint{1.896718in}{0.433391in}}%
\pgfpathlineto{\pgfqpoint{1.898007in}{0.494582in}}%
\pgfpathlineto{\pgfqpoint{1.899296in}{0.406827in}}%
\pgfpathlineto{\pgfqpoint{1.904453in}{0.404995in}}%
\pgfpathlineto{\pgfqpoint{1.905742in}{0.402658in}}%
\pgfpathlineto{\pgfqpoint{1.907031in}{0.404627in}}%
\pgfpathlineto{\pgfqpoint{1.908320in}{0.401284in}}%
\pgfpathlineto{\pgfqpoint{1.909610in}{0.777767in}}%
\pgfpathlineto{\pgfqpoint{1.913477in}{0.422269in}}%
\pgfpathlineto{\pgfqpoint{1.914766in}{0.405670in}}%
\pgfpathlineto{\pgfqpoint{1.916055in}{0.400278in}}%
\pgfpathlineto{\pgfqpoint{1.917345in}{0.442463in}}%
\pgfpathlineto{\pgfqpoint{1.918634in}{0.410223in}}%
\pgfpathlineto{\pgfqpoint{1.922501in}{0.425622in}}%
\pgfpathlineto{\pgfqpoint{1.923791in}{0.411139in}}%
\pgfpathlineto{\pgfqpoint{1.925080in}{0.438431in}}%
\pgfpathlineto{\pgfqpoint{1.927658in}{0.400295in}}%
\pgfpathlineto{\pgfqpoint{1.931526in}{0.402861in}}%
\pgfpathlineto{\pgfqpoint{1.932815in}{0.400744in}}%
\pgfpathlineto{\pgfqpoint{1.934104in}{0.401586in}}%
\pgfpathlineto{\pgfqpoint{1.935393in}{0.419537in}}%
\pgfpathlineto{\pgfqpoint{1.936683in}{0.418870in}}%
\pgfpathlineto{\pgfqpoint{1.940550in}{0.401260in}}%
\pgfpathlineto{\pgfqpoint{1.941839in}{0.419443in}}%
\pgfpathlineto{\pgfqpoint{1.943129in}{0.415638in}}%
\pgfpathlineto{\pgfqpoint{1.944418in}{0.401967in}}%
\pgfpathlineto{\pgfqpoint{1.945707in}{0.424079in}}%
\pgfpathlineto{\pgfqpoint{1.950864in}{0.418705in}}%
\pgfpathlineto{\pgfqpoint{1.952153in}{0.401712in}}%
\pgfpathlineto{\pgfqpoint{1.953442in}{0.402521in}}%
\pgfpathlineto{\pgfqpoint{1.954731in}{0.400361in}}%
\pgfpathlineto{\pgfqpoint{1.958599in}{0.405679in}}%
\pgfpathlineto{\pgfqpoint{1.959888in}{0.415739in}}%
\pgfpathlineto{\pgfqpoint{1.961177in}{0.400323in}}%
\pgfpathlineto{\pgfqpoint{1.962467in}{0.402135in}}%
\pgfpathlineto{\pgfqpoint{1.963756in}{0.402333in}}%
\pgfpathlineto{\pgfqpoint{1.967623in}{0.400841in}}%
\pgfpathlineto{\pgfqpoint{1.968913in}{0.408891in}}%
\pgfpathlineto{\pgfqpoint{1.970202in}{0.408026in}}%
\pgfpathlineto{\pgfqpoint{1.971491in}{0.402798in}}%
\pgfpathlineto{\pgfqpoint{1.972780in}{0.401102in}}%
\pgfpathlineto{\pgfqpoint{1.976648in}{0.403903in}}%
\pgfpathlineto{\pgfqpoint{1.977937in}{0.401413in}}%
\pgfpathlineto{\pgfqpoint{1.979226in}{0.400851in}}%
\pgfpathlineto{\pgfqpoint{1.980515in}{0.403083in}}%
\pgfpathlineto{\pgfqpoint{1.985672in}{0.400277in}}%
\pgfpathlineto{\pgfqpoint{1.986961in}{0.409290in}}%
\pgfpathlineto{\pgfqpoint{1.989540in}{0.400325in}}%
\pgfpathlineto{\pgfqpoint{1.990829in}{0.400649in}}%
\pgfpathlineto{\pgfqpoint{1.994696in}{0.407036in}}%
\pgfpathlineto{\pgfqpoint{1.995986in}{0.439655in}}%
\pgfpathlineto{\pgfqpoint{1.997275in}{0.401370in}}%
\pgfpathlineto{\pgfqpoint{1.998564in}{0.409149in}}%
\pgfpathlineto{\pgfqpoint{2.003721in}{0.400409in}}%
\pgfpathlineto{\pgfqpoint{2.005010in}{0.407048in}}%
\pgfpathlineto{\pgfqpoint{2.006299in}{0.402718in}}%
\pgfpathlineto{\pgfqpoint{2.007588in}{0.401357in}}%
\pgfpathlineto{\pgfqpoint{2.008878in}{0.406650in}}%
\pgfpathlineto{\pgfqpoint{2.012745in}{0.449861in}}%
\pgfpathlineto{\pgfqpoint{2.014034in}{0.401793in}}%
\pgfpathlineto{\pgfqpoint{2.015324in}{0.402576in}}%
\pgfpathlineto{\pgfqpoint{2.016613in}{0.468205in}}%
\pgfpathlineto{\pgfqpoint{2.017902in}{0.416545in}}%
\pgfpathlineto{\pgfqpoint{2.021770in}{0.403100in}}%
\pgfpathlineto{\pgfqpoint{2.023059in}{0.412531in}}%
\pgfpathlineto{\pgfqpoint{2.024348in}{0.400277in}}%
\pgfpathlineto{\pgfqpoint{2.025637in}{0.407060in}}%
\pgfpathlineto{\pgfqpoint{2.026926in}{0.590925in}}%
\pgfpathlineto{\pgfqpoint{2.030794in}{0.401219in}}%
\pgfpathlineto{\pgfqpoint{2.032083in}{0.402100in}}%
\pgfpathlineto{\pgfqpoint{2.033372in}{0.401378in}}%
\pgfpathlineto{\pgfqpoint{2.034661in}{0.429890in}}%
\pgfpathlineto{\pgfqpoint{2.035951in}{0.402356in}}%
\pgfpathlineto{\pgfqpoint{2.039818in}{0.400639in}}%
\pgfpathlineto{\pgfqpoint{2.041107in}{0.408614in}}%
\pgfpathlineto{\pgfqpoint{2.042397in}{0.400324in}}%
\pgfpathlineto{\pgfqpoint{2.043686in}{0.409185in}}%
\pgfpathlineto{\pgfqpoint{2.048843in}{0.400295in}}%
\pgfpathlineto{\pgfqpoint{2.050132in}{0.400481in}}%
\pgfpathlineto{\pgfqpoint{2.051421in}{0.424608in}}%
\pgfpathlineto{\pgfqpoint{2.052710in}{0.404095in}}%
\pgfpathlineto{\pgfqpoint{2.057867in}{0.455575in}}%
\pgfpathlineto{\pgfqpoint{2.059156in}{0.401683in}}%
\pgfpathlineto{\pgfqpoint{2.060445in}{0.404870in}}%
\pgfpathlineto{\pgfqpoint{2.061735in}{0.400931in}}%
\pgfpathlineto{\pgfqpoint{2.063024in}{0.400620in}}%
\pgfpathlineto{\pgfqpoint{2.066891in}{0.400293in}}%
\pgfpathlineto{\pgfqpoint{2.068181in}{0.401841in}}%
\pgfpathlineto{\pgfqpoint{2.069470in}{0.400812in}}%
\pgfpathlineto{\pgfqpoint{2.070759in}{0.420570in}}%
\pgfpathlineto{\pgfqpoint{2.072048in}{0.413654in}}%
\pgfpathlineto{\pgfqpoint{2.077205in}{0.415334in}}%
\pgfpathlineto{\pgfqpoint{2.078494in}{0.400409in}}%
\pgfpathlineto{\pgfqpoint{2.079783in}{0.401685in}}%
\pgfpathlineto{\pgfqpoint{2.081073in}{0.400277in}}%
\pgfpathlineto{\pgfqpoint{2.084940in}{0.407801in}}%
\pgfpathlineto{\pgfqpoint{2.086229in}{0.407006in}}%
\pgfpathlineto{\pgfqpoint{2.087519in}{0.421294in}}%
\pgfpathlineto{\pgfqpoint{2.090097in}{0.402945in}}%
\pgfpathlineto{\pgfqpoint{2.093965in}{0.401356in}}%
\pgfpathlineto{\pgfqpoint{2.095254in}{0.422594in}}%
\pgfpathlineto{\pgfqpoint{2.096543in}{0.409844in}}%
\pgfpathlineto{\pgfqpoint{2.097832in}{0.402674in}}%
\pgfpathlineto{\pgfqpoint{2.099121in}{0.400358in}}%
\pgfpathlineto{\pgfqpoint{2.102989in}{0.418127in}}%
\pgfpathlineto{\pgfqpoint{2.104278in}{0.406739in}}%
\pgfpathlineto{\pgfqpoint{2.105567in}{0.412598in}}%
\pgfpathlineto{\pgfqpoint{2.106856in}{0.443838in}}%
\pgfpathlineto{\pgfqpoint{2.108146in}{0.405058in}}%
\pgfpathlineto{\pgfqpoint{2.112013in}{0.403547in}}%
\pgfpathlineto{\pgfqpoint{2.114592in}{0.436577in}}%
\pgfpathlineto{\pgfqpoint{2.115881in}{0.400554in}}%
\pgfpathlineto{\pgfqpoint{2.117170in}{0.416945in}}%
\pgfpathlineto{\pgfqpoint{2.121038in}{0.400413in}}%
\pgfpathlineto{\pgfqpoint{2.122327in}{0.445973in}}%
\pgfpathlineto{\pgfqpoint{2.124905in}{0.471276in}}%
\pgfpathlineto{\pgfqpoint{2.126194in}{0.425498in}}%
\pgfpathlineto{\pgfqpoint{2.130062in}{0.401825in}}%
\pgfpathlineto{\pgfqpoint{2.131351in}{0.409135in}}%
\pgfpathlineto{\pgfqpoint{2.132640in}{0.421335in}}%
\pgfpathlineto{\pgfqpoint{2.135219in}{0.431779in}}%
\pgfpathlineto{\pgfqpoint{2.139086in}{0.411763in}}%
\pgfpathlineto{\pgfqpoint{2.140376in}{0.497144in}}%
\pgfpathlineto{\pgfqpoint{2.141665in}{0.418625in}}%
\pgfpathlineto{\pgfqpoint{2.142954in}{0.436262in}}%
\pgfpathlineto{\pgfqpoint{2.144243in}{0.402277in}}%
\pgfpathlineto{\pgfqpoint{2.148111in}{0.400294in}}%
\pgfpathlineto{\pgfqpoint{2.149400in}{0.434571in}}%
\pgfpathlineto{\pgfqpoint{2.150689in}{0.413390in}}%
\pgfpathlineto{\pgfqpoint{2.151978in}{1.772999in}}%
\pgfpathlineto{\pgfqpoint{2.153268in}{0.458081in}}%
\pgfpathlineto{\pgfqpoint{2.157135in}{0.400289in}}%
\pgfpathlineto{\pgfqpoint{2.158424in}{0.405279in}}%
\pgfpathlineto{\pgfqpoint{2.159713in}{0.503606in}}%
\pgfpathlineto{\pgfqpoint{2.161003in}{0.400960in}}%
\pgfpathlineto{\pgfqpoint{2.162292in}{0.412897in}}%
\pgfpathlineto{\pgfqpoint{2.166159in}{0.411575in}}%
\pgfpathlineto{\pgfqpoint{2.167449in}{0.400857in}}%
\pgfpathlineto{\pgfqpoint{2.168738in}{0.440359in}}%
\pgfpathlineto{\pgfqpoint{2.170027in}{0.402746in}}%
\pgfpathlineto{\pgfqpoint{2.171316in}{0.410293in}}%
\pgfpathlineto{\pgfqpoint{2.176473in}{0.400287in}}%
\pgfpathlineto{\pgfqpoint{2.177762in}{0.400336in}}%
\pgfpathlineto{\pgfqpoint{2.180341in}{0.414829in}}%
\pgfpathlineto{\pgfqpoint{2.184208in}{0.400658in}}%
\pgfpathlineto{\pgfqpoint{2.185497in}{0.415437in}}%
\pgfpathlineto{\pgfqpoint{2.186787in}{0.400271in}}%
\pgfpathlineto{\pgfqpoint{2.189365in}{0.402095in}}%
\pgfpathlineto{\pgfqpoint{2.193233in}{0.400271in}}%
\pgfpathlineto{\pgfqpoint{2.194522in}{0.407490in}}%
\pgfpathlineto{\pgfqpoint{2.195811in}{0.405586in}}%
\pgfpathlineto{\pgfqpoint{2.197100in}{0.400715in}}%
\pgfpathlineto{\pgfqpoint{2.198389in}{0.410496in}}%
\pgfpathlineto{\pgfqpoint{2.202257in}{0.400401in}}%
\pgfpathlineto{\pgfqpoint{2.203546in}{0.402523in}}%
\pgfpathlineto{\pgfqpoint{2.204835in}{0.412452in}}%
\pgfpathlineto{\pgfqpoint{2.206125in}{0.404473in}}%
\pgfpathlineto{\pgfqpoint{2.207414in}{0.485852in}}%
\pgfpathlineto{\pgfqpoint{2.211281in}{0.400287in}}%
\pgfpathlineto{\pgfqpoint{2.212571in}{0.424283in}}%
\pgfpathlineto{\pgfqpoint{2.213860in}{0.470778in}}%
\pgfpathlineto{\pgfqpoint{2.215149in}{0.454702in}}%
\pgfpathlineto{\pgfqpoint{2.216438in}{0.413307in}}%
\pgfpathlineto{\pgfqpoint{2.220306in}{0.413743in}}%
\pgfpathlineto{\pgfqpoint{2.221595in}{0.402504in}}%
\pgfpathlineto{\pgfqpoint{2.222884in}{0.411727in}}%
\pgfpathlineto{\pgfqpoint{2.225462in}{0.402292in}}%
\pgfpathlineto{\pgfqpoint{2.229330in}{0.403246in}}%
\pgfpathlineto{\pgfqpoint{2.230619in}{0.401843in}}%
\pgfpathlineto{\pgfqpoint{2.231908in}{0.411587in}}%
\pgfpathlineto{\pgfqpoint{2.234487in}{0.416441in}}%
\pgfpathlineto{\pgfqpoint{2.238354in}{0.471699in}}%
\pgfpathlineto{\pgfqpoint{2.239644in}{0.406272in}}%
\pgfpathlineto{\pgfqpoint{2.240933in}{0.425931in}}%
\pgfpathlineto{\pgfqpoint{2.242222in}{0.425862in}}%
\pgfpathlineto{\pgfqpoint{2.243511in}{0.432369in}}%
\pgfpathlineto{\pgfqpoint{2.247379in}{0.400896in}}%
\pgfpathlineto{\pgfqpoint{2.248668in}{0.401604in}}%
\pgfpathlineto{\pgfqpoint{2.249957in}{0.459895in}}%
\pgfpathlineto{\pgfqpoint{2.251246in}{0.413288in}}%
\pgfpathlineto{\pgfqpoint{2.252536in}{0.407780in}}%
\pgfpathlineto{\pgfqpoint{2.257692in}{0.408012in}}%
\pgfpathlineto{\pgfqpoint{2.258982in}{0.400730in}}%
\pgfpathlineto{\pgfqpoint{2.260271in}{0.400456in}}%
\pgfpathlineto{\pgfqpoint{2.261560in}{0.401359in}}%
\pgfpathlineto{\pgfqpoint{2.265428in}{0.406934in}}%
\pgfpathlineto{\pgfqpoint{2.266717in}{0.469025in}}%
\pgfpathlineto{\pgfqpoint{2.268006in}{0.451048in}}%
\pgfpathlineto{\pgfqpoint{2.269295in}{0.406826in}}%
\pgfpathlineto{\pgfqpoint{2.270584in}{0.509360in}}%
\pgfpathlineto{\pgfqpoint{2.274452in}{0.400656in}}%
\pgfpathlineto{\pgfqpoint{2.275741in}{0.443470in}}%
\pgfpathlineto{\pgfqpoint{2.277030in}{0.454408in}}%
\pgfpathlineto{\pgfqpoint{2.278319in}{0.496073in}}%
\pgfpathlineto{\pgfqpoint{2.279609in}{0.438238in}}%
\pgfpathlineto{\pgfqpoint{2.284765in}{0.401996in}}%
\pgfpathlineto{\pgfqpoint{2.287344in}{0.448753in}}%
\pgfpathlineto{\pgfqpoint{2.288633in}{0.403517in}}%
\pgfpathlineto{\pgfqpoint{2.293790in}{0.403312in}}%
\pgfpathlineto{\pgfqpoint{2.295079in}{0.406241in}}%
\pgfpathlineto{\pgfqpoint{2.296368in}{0.400271in}}%
\pgfpathlineto{\pgfqpoint{2.297657in}{0.429939in}}%
\pgfpathlineto{\pgfqpoint{2.301525in}{0.400271in}}%
\pgfpathlineto{\pgfqpoint{2.304103in}{0.400832in}}%
\pgfpathlineto{\pgfqpoint{2.305393in}{0.401341in}}%
\pgfpathlineto{\pgfqpoint{2.306682in}{0.411833in}}%
\pgfpathlineto{\pgfqpoint{2.310549in}{0.493579in}}%
\pgfpathlineto{\pgfqpoint{2.311839in}{0.416961in}}%
\pgfpathlineto{\pgfqpoint{2.313128in}{0.404998in}}%
\pgfpathlineto{\pgfqpoint{2.314417in}{0.400538in}}%
\pgfpathlineto{\pgfqpoint{2.315706in}{0.445231in}}%
\pgfpathlineto{\pgfqpoint{2.319574in}{0.409048in}}%
\pgfpathlineto{\pgfqpoint{2.320863in}{0.478408in}}%
\pgfpathlineto{\pgfqpoint{2.322152in}{0.400697in}}%
\pgfpathlineto{\pgfqpoint{2.323441in}{0.447588in}}%
\pgfpathlineto{\pgfqpoint{2.324731in}{0.442050in}}%
\pgfpathlineto{\pgfqpoint{2.328598in}{0.432824in}}%
\pgfpathlineto{\pgfqpoint{2.329887in}{0.442219in}}%
\pgfpathlineto{\pgfqpoint{2.332466in}{0.400616in}}%
\pgfpathlineto{\pgfqpoint{2.333755in}{0.411786in}}%
\pgfpathlineto{\pgfqpoint{2.338912in}{0.400285in}}%
\pgfpathlineto{\pgfqpoint{2.340201in}{0.462103in}}%
\pgfpathlineto{\pgfqpoint{2.341490in}{0.400876in}}%
\pgfpathlineto{\pgfqpoint{2.342779in}{0.400361in}}%
\pgfpathlineto{\pgfqpoint{2.346647in}{0.400705in}}%
\pgfpathlineto{\pgfqpoint{2.347936in}{0.402339in}}%
\pgfpathlineto{\pgfqpoint{2.349225in}{0.402022in}}%
\pgfpathlineto{\pgfqpoint{2.350514in}{0.400710in}}%
\pgfpathlineto{\pgfqpoint{2.355671in}{0.400565in}}%
\pgfpathlineto{\pgfqpoint{2.356960in}{0.412360in}}%
\pgfpathlineto{\pgfqpoint{2.358250in}{0.416563in}}%
\pgfpathlineto{\pgfqpoint{2.359539in}{0.400397in}}%
\pgfpathlineto{\pgfqpoint{2.360828in}{0.400621in}}%
\pgfpathlineto{\pgfqpoint{2.364696in}{0.417627in}}%
\pgfpathlineto{\pgfqpoint{2.365985in}{0.400501in}}%
\pgfpathlineto{\pgfqpoint{2.367274in}{0.400973in}}%
\pgfpathlineto{\pgfqpoint{2.368563in}{0.400304in}}%
\pgfpathlineto{\pgfqpoint{2.369852in}{0.443519in}}%
\pgfpathlineto{\pgfqpoint{2.373720in}{0.401604in}}%
\pgfpathlineto{\pgfqpoint{2.375009in}{0.413837in}}%
\pgfpathlineto{\pgfqpoint{2.376298in}{0.631152in}}%
\pgfpathlineto{\pgfqpoint{2.377588in}{0.402896in}}%
\pgfpathlineto{\pgfqpoint{2.378877in}{0.402215in}}%
\pgfpathlineto{\pgfqpoint{2.382744in}{0.404684in}}%
\pgfpathlineto{\pgfqpoint{2.384034in}{0.403795in}}%
\pgfpathlineto{\pgfqpoint{2.385323in}{0.410643in}}%
\pgfpathlineto{\pgfqpoint{2.386612in}{0.454431in}}%
\pgfpathlineto{\pgfqpoint{2.387901in}{0.402859in}}%
\pgfpathlineto{\pgfqpoint{2.391769in}{0.401429in}}%
\pgfpathlineto{\pgfqpoint{2.393058in}{0.404891in}}%
\pgfpathlineto{\pgfqpoint{2.394347in}{0.403062in}}%
\pgfpathlineto{\pgfqpoint{2.395636in}{0.427581in}}%
\pgfpathlineto{\pgfqpoint{2.396925in}{0.661576in}}%
\pgfpathlineto{\pgfqpoint{2.400793in}{0.406497in}}%
\pgfpathlineto{\pgfqpoint{2.403371in}{0.400431in}}%
\pgfpathlineto{\pgfqpoint{2.404661in}{0.455138in}}%
\pgfpathlineto{\pgfqpoint{2.405950in}{0.405837in}}%
\pgfpathlineto{\pgfqpoint{2.409817in}{0.405837in}}%
\pgfpathlineto{\pgfqpoint{2.411107in}{0.401072in}}%
\pgfpathlineto{\pgfqpoint{2.412396in}{0.404567in}}%
\pgfpathlineto{\pgfqpoint{2.413685in}{0.405349in}}%
\pgfpathlineto{\pgfqpoint{2.414974in}{0.402104in}}%
\pgfpathlineto{\pgfqpoint{2.420131in}{0.435094in}}%
\pgfpathlineto{\pgfqpoint{2.421420in}{0.427570in}}%
\pgfpathlineto{\pgfqpoint{2.422709in}{0.400386in}}%
\pgfpathlineto{\pgfqpoint{2.423999in}{0.423508in}}%
\pgfpathlineto{\pgfqpoint{2.429155in}{0.400429in}}%
\pgfpathlineto{\pgfqpoint{2.430445in}{0.400533in}}%
\pgfpathlineto{\pgfqpoint{2.431734in}{0.416739in}}%
\pgfpathlineto{\pgfqpoint{2.433023in}{0.401013in}}%
\pgfpathlineto{\pgfqpoint{2.436891in}{0.415177in}}%
\pgfpathlineto{\pgfqpoint{2.438180in}{0.402530in}}%
\pgfpathlineto{\pgfqpoint{2.439469in}{0.486550in}}%
\pgfpathlineto{\pgfqpoint{2.440758in}{0.400275in}}%
\pgfpathlineto{\pgfqpoint{2.442047in}{0.403135in}}%
\pgfpathlineto{\pgfqpoint{2.445915in}{0.417492in}}%
\pgfpathlineto{\pgfqpoint{2.447204in}{0.405998in}}%
\pgfpathlineto{\pgfqpoint{2.448493in}{0.402295in}}%
\pgfpathlineto{\pgfqpoint{2.449783in}{0.417405in}}%
\pgfpathlineto{\pgfqpoint{2.451072in}{0.420355in}}%
\pgfpathlineto{\pgfqpoint{2.454939in}{0.405195in}}%
\pgfpathlineto{\pgfqpoint{2.456228in}{0.403551in}}%
\pgfpathlineto{\pgfqpoint{2.457518in}{0.409643in}}%
\pgfpathlineto{\pgfqpoint{2.458807in}{0.402144in}}%
\pgfpathlineto{\pgfqpoint{2.460096in}{0.400822in}}%
\pgfpathlineto{\pgfqpoint{2.463964in}{0.530252in}}%
\pgfpathlineto{\pgfqpoint{2.465253in}{0.406321in}}%
\pgfpathlineto{\pgfqpoint{2.466542in}{0.415863in}}%
\pgfpathlineto{\pgfqpoint{2.467831in}{0.405062in}}%
\pgfpathlineto{\pgfqpoint{2.472988in}{0.400747in}}%
\pgfpathlineto{\pgfqpoint{2.474277in}{0.404118in}}%
\pgfpathlineto{\pgfqpoint{2.475566in}{0.434928in}}%
\pgfpathlineto{\pgfqpoint{2.476856in}{0.403571in}}%
\pgfpathlineto{\pgfqpoint{2.478145in}{0.459511in}}%
\pgfpathlineto{\pgfqpoint{2.482012in}{0.437248in}}%
\pgfpathlineto{\pgfqpoint{2.483302in}{0.410157in}}%
\pgfpathlineto{\pgfqpoint{2.484591in}{0.400472in}}%
\pgfpathlineto{\pgfqpoint{2.485880in}{0.409006in}}%
\pgfpathlineto{\pgfqpoint{2.487169in}{0.403036in}}%
\pgfpathlineto{\pgfqpoint{2.491037in}{0.493675in}}%
\pgfpathlineto{\pgfqpoint{2.492326in}{0.413052in}}%
\pgfpathlineto{\pgfqpoint{2.493615in}{0.400345in}}%
\pgfpathlineto{\pgfqpoint{2.494904in}{0.401583in}}%
\pgfpathlineto{\pgfqpoint{2.496194in}{0.649354in}}%
\pgfpathlineto{\pgfqpoint{2.500061in}{0.423301in}}%
\pgfpathlineto{\pgfqpoint{2.501350in}{0.420381in}}%
\pgfpathlineto{\pgfqpoint{2.502640in}{0.441190in}}%
\pgfpathlineto{\pgfqpoint{2.503929in}{0.403701in}}%
\pgfpathlineto{\pgfqpoint{2.505218in}{0.426969in}}%
\pgfpathlineto{\pgfqpoint{2.509086in}{0.404795in}}%
\pgfpathlineto{\pgfqpoint{2.510375in}{0.406478in}}%
\pgfpathlineto{\pgfqpoint{2.511664in}{0.402405in}}%
\pgfpathlineto{\pgfqpoint{2.512953in}{0.451774in}}%
\pgfpathlineto{\pgfqpoint{2.514242in}{0.410741in}}%
\pgfpathlineto{\pgfqpoint{2.518110in}{0.401076in}}%
\pgfpathlineto{\pgfqpoint{2.519399in}{0.427864in}}%
\pgfpathlineto{\pgfqpoint{2.520688in}{0.402662in}}%
\pgfpathlineto{\pgfqpoint{2.521977in}{0.404338in}}%
\pgfpathlineto{\pgfqpoint{2.523267in}{0.402459in}}%
\pgfpathlineto{\pgfqpoint{2.529713in}{0.400407in}}%
\pgfpathlineto{\pgfqpoint{2.531002in}{0.405418in}}%
\pgfpathlineto{\pgfqpoint{2.532291in}{0.609147in}}%
\pgfpathlineto{\pgfqpoint{2.536159in}{0.638196in}}%
\pgfpathlineto{\pgfqpoint{2.537448in}{0.461232in}}%
\pgfpathlineto{\pgfqpoint{2.538737in}{0.821735in}}%
\pgfpathlineto{\pgfqpoint{2.540026in}{0.482721in}}%
\pgfpathlineto{\pgfqpoint{2.541315in}{0.400376in}}%
\pgfpathlineto{\pgfqpoint{2.545183in}{0.438120in}}%
\pgfpathlineto{\pgfqpoint{2.546472in}{0.560757in}}%
\pgfpathlineto{\pgfqpoint{2.547761in}{0.413337in}}%
\pgfpathlineto{\pgfqpoint{2.549051in}{0.418319in}}%
\pgfpathlineto{\pgfqpoint{2.550340in}{0.445559in}}%
\pgfpathlineto{\pgfqpoint{2.555497in}{0.455796in}}%
\pgfpathlineto{\pgfqpoint{2.556786in}{0.426104in}}%
\pgfpathlineto{\pgfqpoint{2.558075in}{0.412314in}}%
\pgfpathlineto{\pgfqpoint{2.559364in}{0.408286in}}%
\pgfpathlineto{\pgfqpoint{2.563232in}{0.416744in}}%
\pgfpathlineto{\pgfqpoint{2.564521in}{0.407719in}}%
\pgfpathlineto{\pgfqpoint{2.565810in}{0.406745in}}%
\pgfpathlineto{\pgfqpoint{2.567099in}{0.400299in}}%
\pgfpathlineto{\pgfqpoint{2.568389in}{0.439756in}}%
\pgfpathlineto{\pgfqpoint{2.572256in}{0.428840in}}%
\pgfpathlineto{\pgfqpoint{2.573545in}{0.408599in}}%
\pgfpathlineto{\pgfqpoint{2.574835in}{0.414480in}}%
\pgfpathlineto{\pgfqpoint{2.576124in}{0.414903in}}%
\pgfpathlineto{\pgfqpoint{2.577413in}{0.406243in}}%
\pgfpathlineto{\pgfqpoint{2.581280in}{0.770688in}}%
\pgfpathlineto{\pgfqpoint{2.582570in}{0.456466in}}%
\pgfpathlineto{\pgfqpoint{2.583859in}{0.438381in}}%
\pgfpathlineto{\pgfqpoint{2.585148in}{0.403488in}}%
\pgfpathlineto{\pgfqpoint{2.586437in}{0.413758in}}%
\pgfpathlineto{\pgfqpoint{2.590305in}{0.469579in}}%
\pgfpathlineto{\pgfqpoint{2.591594in}{0.400933in}}%
\pgfpathlineto{\pgfqpoint{2.592883in}{0.436019in}}%
\pgfpathlineto{\pgfqpoint{2.594172in}{0.412598in}}%
\pgfpathlineto{\pgfqpoint{2.595462in}{0.400451in}}%
\pgfpathlineto{\pgfqpoint{2.599329in}{0.426777in}}%
\pgfpathlineto{\pgfqpoint{2.600618in}{0.400274in}}%
\pgfpathlineto{\pgfqpoint{2.603197in}{0.434019in}}%
\pgfpathlineto{\pgfqpoint{2.604486in}{0.411224in}}%
\pgfpathlineto{\pgfqpoint{2.608354in}{0.424388in}}%
\pgfpathlineto{\pgfqpoint{2.609643in}{0.412685in}}%
\pgfpathlineto{\pgfqpoint{2.612221in}{0.423240in}}%
\pgfpathlineto{\pgfqpoint{2.613510in}{0.410592in}}%
\pgfpathlineto{\pgfqpoint{2.617378in}{0.429917in}}%
\pgfpathlineto{\pgfqpoint{2.618667in}{0.410621in}}%
\pgfpathlineto{\pgfqpoint{2.619956in}{0.443254in}}%
\pgfpathlineto{\pgfqpoint{2.621246in}{0.403301in}}%
\pgfpathlineto{\pgfqpoint{2.622535in}{0.420636in}}%
\pgfpathlineto{\pgfqpoint{2.627692in}{0.576913in}}%
\pgfpathlineto{\pgfqpoint{2.628981in}{0.405128in}}%
\pgfpathlineto{\pgfqpoint{2.630270in}{0.419366in}}%
\pgfpathlineto{\pgfqpoint{2.631559in}{0.406405in}}%
\pgfpathlineto{\pgfqpoint{2.635427in}{0.411793in}}%
\pgfpathlineto{\pgfqpoint{2.636716in}{0.457957in}}%
\pgfpathlineto{\pgfqpoint{2.638005in}{0.402310in}}%
\pgfpathlineto{\pgfqpoint{2.639294in}{0.418052in}}%
\pgfpathlineto{\pgfqpoint{2.640583in}{0.401172in}}%
\pgfpathlineto{\pgfqpoint{2.644451in}{0.414950in}}%
\pgfpathlineto{\pgfqpoint{2.645740in}{0.400392in}}%
\pgfpathlineto{\pgfqpoint{2.647029in}{0.460329in}}%
\pgfpathlineto{\pgfqpoint{2.648319in}{0.402401in}}%
\pgfpathlineto{\pgfqpoint{2.649608in}{0.400309in}}%
\pgfpathlineto{\pgfqpoint{2.653475in}{0.400423in}}%
\pgfpathlineto{\pgfqpoint{2.654765in}{0.402566in}}%
\pgfpathlineto{\pgfqpoint{2.656054in}{0.401435in}}%
\pgfpathlineto{\pgfqpoint{2.658632in}{0.402023in}}%
\pgfpathlineto{\pgfqpoint{2.663789in}{0.418988in}}%
\pgfpathlineto{\pgfqpoint{2.665078in}{0.417726in}}%
\pgfpathlineto{\pgfqpoint{2.666367in}{0.413413in}}%
\pgfpathlineto{\pgfqpoint{2.667657in}{0.503136in}}%
\pgfpathlineto{\pgfqpoint{2.671524in}{0.416706in}}%
\pgfpathlineto{\pgfqpoint{2.672813in}{0.405630in}}%
\pgfpathlineto{\pgfqpoint{2.674103in}{0.433040in}}%
\pgfpathlineto{\pgfqpoint{2.675392in}{0.401598in}}%
\pgfpathlineto{\pgfqpoint{2.676681in}{0.497842in}}%
\pgfpathlineto{\pgfqpoint{2.680549in}{0.414169in}}%
\pgfpathlineto{\pgfqpoint{2.681838in}{0.474095in}}%
\pgfpathlineto{\pgfqpoint{2.683127in}{0.435265in}}%
\pgfpathlineto{\pgfqpoint{2.684416in}{0.431828in}}%
\pgfpathlineto{\pgfqpoint{2.685705in}{0.534669in}}%
\pgfpathlineto{\pgfqpoint{2.689573in}{0.418118in}}%
\pgfpathlineto{\pgfqpoint{2.690862in}{0.401293in}}%
\pgfpathlineto{\pgfqpoint{2.692151in}{0.433465in}}%
\pgfpathlineto{\pgfqpoint{2.693441in}{0.402358in}}%
\pgfpathlineto{\pgfqpoint{2.698597in}{0.401950in}}%
\pgfpathlineto{\pgfqpoint{2.701176in}{0.417620in}}%
\pgfpathlineto{\pgfqpoint{2.702465in}{0.415160in}}%
\pgfpathlineto{\pgfqpoint{2.707622in}{0.484471in}}%
\pgfpathlineto{\pgfqpoint{2.708911in}{0.408265in}}%
\pgfpathlineto{\pgfqpoint{2.710200in}{0.425279in}}%
\pgfpathlineto{\pgfqpoint{2.711489in}{0.457234in}}%
\pgfpathlineto{\pgfqpoint{2.712778in}{0.422747in}}%
\pgfpathlineto{\pgfqpoint{2.716646in}{0.429738in}}%
\pgfpathlineto{\pgfqpoint{2.717935in}{0.418386in}}%
\pgfpathlineto{\pgfqpoint{2.719224in}{0.472642in}}%
\pgfpathlineto{\pgfqpoint{2.720514in}{0.413349in}}%
\pgfpathlineto{\pgfqpoint{2.721803in}{0.506378in}}%
\pgfpathlineto{\pgfqpoint{2.726960in}{0.405520in}}%
\pgfpathlineto{\pgfqpoint{2.730827in}{0.436408in}}%
\pgfpathlineto{\pgfqpoint{2.734695in}{0.425573in}}%
\pgfpathlineto{\pgfqpoint{2.735984in}{0.400771in}}%
\pgfpathlineto{\pgfqpoint{2.737273in}{0.426723in}}%
\pgfpathlineto{\pgfqpoint{2.738562in}{0.473878in}}%
\pgfpathlineto{\pgfqpoint{2.739852in}{1.144540in}}%
\pgfpathlineto{\pgfqpoint{2.743719in}{0.400605in}}%
\pgfpathlineto{\pgfqpoint{2.745008in}{0.425496in}}%
\pgfpathlineto{\pgfqpoint{2.746298in}{0.425496in}}%
\pgfpathlineto{\pgfqpoint{2.747587in}{0.413147in}}%
\pgfpathlineto{\pgfqpoint{2.752744in}{0.821738in}}%
\pgfpathlineto{\pgfqpoint{2.754033in}{0.409879in}}%
\pgfpathlineto{\pgfqpoint{2.755322in}{0.516155in}}%
\pgfpathlineto{\pgfqpoint{2.756611in}{0.485363in}}%
\pgfpathlineto{\pgfqpoint{2.757900in}{0.514048in}}%
\pgfpathlineto{\pgfqpoint{2.763057in}{0.413895in}}%
\pgfpathlineto{\pgfqpoint{2.764346in}{0.432299in}}%
\pgfpathlineto{\pgfqpoint{2.766925in}{0.402600in}}%
\pgfpathlineto{\pgfqpoint{2.770792in}{0.465435in}}%
\pgfpathlineto{\pgfqpoint{2.772081in}{0.423014in}}%
\pgfpathlineto{\pgfqpoint{2.773371in}{0.402899in}}%
\pgfpathlineto{\pgfqpoint{2.774660in}{0.455825in}}%
\pgfpathlineto{\pgfqpoint{2.775949in}{0.408250in}}%
\pgfpathlineto{\pgfqpoint{2.779817in}{0.403402in}}%
\pgfpathlineto{\pgfqpoint{2.781106in}{0.535846in}}%
\pgfpathlineto{\pgfqpoint{2.782395in}{0.403243in}}%
\pgfpathlineto{\pgfqpoint{2.783684in}{0.404251in}}%
\pgfpathlineto{\pgfqpoint{2.784973in}{0.400274in}}%
\pgfpathlineto{\pgfqpoint{2.788841in}{0.505218in}}%
\pgfpathlineto{\pgfqpoint{2.791419in}{0.400381in}}%
\pgfpathlineto{\pgfqpoint{2.792709in}{0.406448in}}%
\pgfpathlineto{\pgfqpoint{2.793998in}{0.457581in}}%
\pgfpathlineto{\pgfqpoint{2.797865in}{0.400851in}}%
\pgfpathlineto{\pgfqpoint{2.799155in}{0.405221in}}%
\pgfpathlineto{\pgfqpoint{2.800444in}{0.406162in}}%
\pgfpathlineto{\pgfqpoint{2.801733in}{0.416358in}}%
\pgfpathlineto{\pgfqpoint{2.803022in}{0.411405in}}%
\pgfpathlineto{\pgfqpoint{2.806890in}{0.400274in}}%
\pgfpathlineto{\pgfqpoint{2.808179in}{0.417822in}}%
\pgfpathlineto{\pgfqpoint{2.809468in}{0.424546in}}%
\pgfpathlineto{\pgfqpoint{2.810757in}{0.401608in}}%
\pgfpathlineto{\pgfqpoint{2.815914in}{0.400371in}}%
\pgfpathlineto{\pgfqpoint{2.817203in}{0.443990in}}%
\pgfpathlineto{\pgfqpoint{2.818492in}{0.449354in}}%
\pgfpathlineto{\pgfqpoint{2.819782in}{0.402591in}}%
\pgfpathlineto{\pgfqpoint{2.821071in}{0.430577in}}%
\pgfpathlineto{\pgfqpoint{2.824938in}{0.401597in}}%
\pgfpathlineto{\pgfqpoint{2.827517in}{0.437087in}}%
\pgfpathlineto{\pgfqpoint{2.828806in}{0.409554in}}%
\pgfpathlineto{\pgfqpoint{2.830095in}{0.405112in}}%
\pgfpathlineto{\pgfqpoint{2.833963in}{0.401371in}}%
\pgfpathlineto{\pgfqpoint{2.835252in}{0.412065in}}%
\pgfpathlineto{\pgfqpoint{2.836541in}{0.432810in}}%
\pgfpathlineto{\pgfqpoint{2.837830in}{0.408781in}}%
\pgfpathlineto{\pgfqpoint{2.839120in}{0.401627in}}%
\pgfpathlineto{\pgfqpoint{2.842987in}{0.441937in}}%
\pgfpathlineto{\pgfqpoint{2.844276in}{0.411228in}}%
\pgfpathlineto{\pgfqpoint{2.845566in}{0.403785in}}%
\pgfpathlineto{\pgfqpoint{2.846855in}{0.403100in}}%
\pgfpathlineto{\pgfqpoint{2.848144in}{0.463969in}}%
\pgfpathlineto{\pgfqpoint{2.853301in}{0.403833in}}%
\pgfpathlineto{\pgfqpoint{2.854590in}{0.401456in}}%
\pgfpathlineto{\pgfqpoint{2.855879in}{0.427546in}}%
\pgfpathlineto{\pgfqpoint{2.857168in}{0.404935in}}%
\pgfpathlineto{\pgfqpoint{2.861036in}{0.435637in}}%
\pgfpathlineto{\pgfqpoint{2.862325in}{0.438697in}}%
\pgfpathlineto{\pgfqpoint{2.863614in}{0.400638in}}%
\pgfpathlineto{\pgfqpoint{2.866193in}{0.403534in}}%
\pgfpathlineto{\pgfqpoint{2.870060in}{0.406008in}}%
\pgfpathlineto{\pgfqpoint{2.871350in}{0.422789in}}%
\pgfpathlineto{\pgfqpoint{2.872639in}{0.473280in}}%
\pgfpathlineto{\pgfqpoint{2.873928in}{0.408698in}}%
\pgfpathlineto{\pgfqpoint{2.875217in}{0.427530in}}%
\pgfpathlineto{\pgfqpoint{2.879085in}{0.423043in}}%
\pgfpathlineto{\pgfqpoint{2.880374in}{0.425014in}}%
\pgfpathlineto{\pgfqpoint{2.881663in}{0.410339in}}%
\pgfpathlineto{\pgfqpoint{2.882952in}{0.407659in}}%
\pgfpathlineto{\pgfqpoint{2.884241in}{0.415247in}}%
\pgfpathlineto{\pgfqpoint{2.888109in}{0.405365in}}%
\pgfpathlineto{\pgfqpoint{2.889398in}{0.510873in}}%
\pgfpathlineto{\pgfqpoint{2.890687in}{0.402581in}}%
\pgfpathlineto{\pgfqpoint{2.891977in}{0.400358in}}%
\pgfpathlineto{\pgfqpoint{2.893266in}{0.410397in}}%
\pgfpathlineto{\pgfqpoint{2.898423in}{0.412362in}}%
\pgfpathlineto{\pgfqpoint{2.899712in}{0.402162in}}%
\pgfpathlineto{\pgfqpoint{2.901001in}{0.412275in}}%
\pgfpathlineto{\pgfqpoint{2.902290in}{0.400274in}}%
\pgfpathlineto{\pgfqpoint{2.906158in}{0.408426in}}%
\pgfpathlineto{\pgfqpoint{2.907447in}{0.400355in}}%
\pgfpathlineto{\pgfqpoint{2.908736in}{0.424115in}}%
\pgfpathlineto{\pgfqpoint{2.910025in}{0.403554in}}%
\pgfpathlineto{\pgfqpoint{2.911315in}{0.425792in}}%
\pgfpathlineto{\pgfqpoint{2.915182in}{0.473749in}}%
\pgfpathlineto{\pgfqpoint{2.916471in}{0.400979in}}%
\pgfpathlineto{\pgfqpoint{2.917761in}{0.404007in}}%
\pgfpathlineto{\pgfqpoint{2.919050in}{0.401069in}}%
\pgfpathlineto{\pgfqpoint{2.920339in}{0.444495in}}%
\pgfpathlineto{\pgfqpoint{2.924207in}{0.403199in}}%
\pgfpathlineto{\pgfqpoint{2.925496in}{0.400274in}}%
\pgfpathlineto{\pgfqpoint{2.926785in}{0.415776in}}%
\pgfpathlineto{\pgfqpoint{2.928074in}{0.469684in}}%
\pgfpathlineto{\pgfqpoint{2.929363in}{0.648697in}}%
\pgfpathlineto{\pgfqpoint{2.933231in}{0.476685in}}%
\pgfpathlineto{\pgfqpoint{2.934520in}{0.486952in}}%
\pgfpathlineto{\pgfqpoint{2.935809in}{0.462284in}}%
\pgfpathlineto{\pgfqpoint{2.937098in}{0.567199in}}%
\pgfpathlineto{\pgfqpoint{2.938388in}{0.402729in}}%
\pgfpathlineto{\pgfqpoint{2.943544in}{0.401250in}}%
\pgfpathlineto{\pgfqpoint{2.944834in}{0.410090in}}%
\pgfpathlineto{\pgfqpoint{2.946123in}{0.405563in}}%
\pgfpathlineto{\pgfqpoint{2.947412in}{0.492892in}}%
\pgfpathlineto{\pgfqpoint{2.951280in}{0.400529in}}%
\pgfpathlineto{\pgfqpoint{2.952569in}{0.422274in}}%
\pgfpathlineto{\pgfqpoint{2.953858in}{0.400274in}}%
\pgfpathlineto{\pgfqpoint{2.955147in}{0.411125in}}%
\pgfpathlineto{\pgfqpoint{2.956436in}{0.400902in}}%
\pgfpathlineto{\pgfqpoint{2.960304in}{0.400274in}}%
\pgfpathlineto{\pgfqpoint{2.962882in}{0.409552in}}%
\pgfpathlineto{\pgfqpoint{2.964172in}{0.407568in}}%
\pgfpathlineto{\pgfqpoint{2.965461in}{0.428732in}}%
\pgfpathlineto{\pgfqpoint{2.969328in}{0.421851in}}%
\pgfpathlineto{\pgfqpoint{2.970618in}{0.404984in}}%
\pgfpathlineto{\pgfqpoint{2.971907in}{0.400311in}}%
\pgfpathlineto{\pgfqpoint{2.973196in}{0.410518in}}%
\pgfpathlineto{\pgfqpoint{2.974485in}{0.430881in}}%
\pgfpathlineto{\pgfqpoint{2.978353in}{0.401360in}}%
\pgfpathlineto{\pgfqpoint{2.979642in}{0.400281in}}%
\pgfpathlineto{\pgfqpoint{2.980931in}{0.404589in}}%
\pgfpathlineto{\pgfqpoint{2.982220in}{0.413833in}}%
\pgfpathlineto{\pgfqpoint{2.987377in}{0.401121in}}%
\pgfpathlineto{\pgfqpoint{2.988666in}{0.400556in}}%
\pgfpathlineto{\pgfqpoint{2.989956in}{0.403508in}}%
\pgfpathlineto{\pgfqpoint{2.991245in}{0.404639in}}%
\pgfpathlineto{\pgfqpoint{2.992534in}{0.400507in}}%
\pgfpathlineto{\pgfqpoint{2.996401in}{0.417903in}}%
\pgfpathlineto{\pgfqpoint{2.997691in}{0.400502in}}%
\pgfpathlineto{\pgfqpoint{2.998980in}{0.400937in}}%
\pgfpathlineto{\pgfqpoint{3.000269in}{0.407535in}}%
\pgfpathlineto{\pgfqpoint{3.001558in}{0.401202in}}%
\pgfpathlineto{\pgfqpoint{3.005426in}{0.407758in}}%
\pgfpathlineto{\pgfqpoint{3.006715in}{0.401706in}}%
\pgfpathlineto{\pgfqpoint{3.008004in}{0.406063in}}%
\pgfpathlineto{\pgfqpoint{3.009293in}{0.400292in}}%
\pgfpathlineto{\pgfqpoint{3.010583in}{0.401609in}}%
\pgfpathlineto{\pgfqpoint{3.015739in}{0.402197in}}%
\pgfpathlineto{\pgfqpoint{3.017029in}{0.401819in}}%
\pgfpathlineto{\pgfqpoint{3.018318in}{0.403747in}}%
\pgfpathlineto{\pgfqpoint{3.019607in}{0.410697in}}%
\pgfpathlineto{\pgfqpoint{3.024764in}{0.406278in}}%
\pgfpathlineto{\pgfqpoint{3.026053in}{0.410993in}}%
\pgfpathlineto{\pgfqpoint{3.027342in}{0.401858in}}%
\pgfpathlineto{\pgfqpoint{3.028631in}{0.457294in}}%
\pgfpathlineto{\pgfqpoint{3.032499in}{0.457294in}}%
\pgfpathlineto{\pgfqpoint{3.035077in}{0.403875in}}%
\pgfpathlineto{\pgfqpoint{3.036367in}{0.406122in}}%
\pgfpathlineto{\pgfqpoint{3.037656in}{0.400352in}}%
\pgfpathlineto{\pgfqpoint{3.041523in}{0.404376in}}%
\pgfpathlineto{\pgfqpoint{3.042813in}{0.400291in}}%
\pgfpathlineto{\pgfqpoint{3.044102in}{0.411941in}}%
\pgfpathlineto{\pgfqpoint{3.045391in}{0.400696in}}%
\pgfpathlineto{\pgfqpoint{3.046680in}{0.414604in}}%
\pgfpathlineto{\pgfqpoint{3.050548in}{0.411492in}}%
\pgfpathlineto{\pgfqpoint{3.051837in}{0.402856in}}%
\pgfpathlineto{\pgfqpoint{3.054415in}{0.423279in}}%
\pgfpathlineto{\pgfqpoint{3.055705in}{0.413085in}}%
\pgfpathlineto{\pgfqpoint{3.059572in}{0.400833in}}%
\pgfpathlineto{\pgfqpoint{3.060861in}{0.400642in}}%
\pgfpathlineto{\pgfqpoint{3.062150in}{0.408382in}}%
\pgfpathlineto{\pgfqpoint{3.063440in}{0.400274in}}%
\pgfpathlineto{\pgfqpoint{3.064729in}{0.405077in}}%
\pgfpathlineto{\pgfqpoint{3.068596in}{0.401426in}}%
\pgfpathlineto{\pgfqpoint{3.069886in}{0.425039in}}%
\pgfpathlineto{\pgfqpoint{3.071175in}{0.401254in}}%
\pgfpathlineto{\pgfqpoint{3.072464in}{0.403327in}}%
\pgfpathlineto{\pgfqpoint{3.073753in}{0.407505in}}%
\pgfpathlineto{\pgfqpoint{3.077621in}{0.402268in}}%
\pgfpathlineto{\pgfqpoint{3.078910in}{0.407305in}}%
\pgfpathlineto{\pgfqpoint{3.080199in}{0.432846in}}%
\pgfpathlineto{\pgfqpoint{3.081488in}{0.402386in}}%
\pgfpathlineto{\pgfqpoint{3.082778in}{0.400705in}}%
\pgfpathlineto{\pgfqpoint{3.086645in}{0.414316in}}%
\pgfpathlineto{\pgfqpoint{3.087934in}{0.427908in}}%
\pgfpathlineto{\pgfqpoint{3.089224in}{0.401905in}}%
\pgfpathlineto{\pgfqpoint{3.090513in}{0.400920in}}%
\pgfpathlineto{\pgfqpoint{3.091802in}{0.402148in}}%
\pgfpathlineto{\pgfqpoint{3.095670in}{0.402008in}}%
\pgfpathlineto{\pgfqpoint{3.096959in}{0.417166in}}%
\pgfpathlineto{\pgfqpoint{3.098248in}{0.424049in}}%
\pgfpathlineto{\pgfqpoint{3.099537in}{0.400385in}}%
\pgfpathlineto{\pgfqpoint{3.100826in}{0.402090in}}%
\pgfpathlineto{\pgfqpoint{3.104694in}{0.500237in}}%
\pgfpathlineto{\pgfqpoint{3.105983in}{0.403280in}}%
\pgfpathlineto{\pgfqpoint{3.107272in}{0.402940in}}%
\pgfpathlineto{\pgfqpoint{3.108562in}{0.439872in}}%
\pgfpathlineto{\pgfqpoint{3.109851in}{0.400274in}}%
\pgfpathlineto{\pgfqpoint{3.113718in}{0.675747in}}%
\pgfpathlineto{\pgfqpoint{3.115008in}{0.401062in}}%
\pgfpathlineto{\pgfqpoint{3.116297in}{0.465172in}}%
\pgfpathlineto{\pgfqpoint{3.117586in}{0.421963in}}%
\pgfpathlineto{\pgfqpoint{3.118875in}{0.401587in}}%
\pgfpathlineto{\pgfqpoint{3.122743in}{0.416649in}}%
\pgfpathlineto{\pgfqpoint{3.124032in}{0.468875in}}%
\pgfpathlineto{\pgfqpoint{3.125321in}{0.403154in}}%
\pgfpathlineto{\pgfqpoint{3.127899in}{0.407372in}}%
\pgfpathlineto{\pgfqpoint{3.131767in}{0.425328in}}%
\pgfpathlineto{\pgfqpoint{3.133056in}{0.400389in}}%
\pgfpathlineto{\pgfqpoint{3.134345in}{0.479503in}}%
\pgfpathlineto{\pgfqpoint{3.135635in}{0.489055in}}%
\pgfpathlineto{\pgfqpoint{3.136924in}{0.402475in}}%
\pgfpathlineto{\pgfqpoint{3.140791in}{0.462490in}}%
\pgfpathlineto{\pgfqpoint{3.142081in}{0.401276in}}%
\pgfpathlineto{\pgfqpoint{3.143370in}{0.514459in}}%
\pgfpathlineto{\pgfqpoint{3.144659in}{0.400293in}}%
\pgfpathlineto{\pgfqpoint{3.145948in}{0.400675in}}%
\pgfpathlineto{\pgfqpoint{3.149816in}{0.409833in}}%
\pgfpathlineto{\pgfqpoint{3.151105in}{0.411404in}}%
\pgfpathlineto{\pgfqpoint{3.152394in}{0.400357in}}%
\pgfpathlineto{\pgfqpoint{3.153683in}{0.403355in}}%
\pgfpathlineto{\pgfqpoint{3.154973in}{0.430954in}}%
\pgfpathlineto{\pgfqpoint{3.158840in}{0.401800in}}%
\pgfpathlineto{\pgfqpoint{3.160129in}{0.401925in}}%
\pgfpathlineto{\pgfqpoint{3.162708in}{0.401572in}}%
\pgfpathlineto{\pgfqpoint{3.163997in}{0.401688in}}%
\pgfpathlineto{\pgfqpoint{3.169154in}{0.401249in}}%
\pgfpathlineto{\pgfqpoint{3.170443in}{0.400332in}}%
\pgfpathlineto{\pgfqpoint{3.171732in}{0.400293in}}%
\pgfpathlineto{\pgfqpoint{3.173021in}{0.402476in}}%
\pgfpathlineto{\pgfqpoint{3.178178in}{0.451002in}}%
\pgfpathlineto{\pgfqpoint{3.179467in}{0.409892in}}%
\pgfpathlineto{\pgfqpoint{3.182046in}{0.427438in}}%
\pgfpathlineto{\pgfqpoint{3.185913in}{0.404781in}}%
\pgfpathlineto{\pgfqpoint{3.187202in}{0.404435in}}%
\pgfpathlineto{\pgfqpoint{3.188492in}{0.405456in}}%
\pgfpathlineto{\pgfqpoint{3.189781in}{0.404238in}}%
\pgfpathlineto{\pgfqpoint{3.191070in}{0.401178in}}%
\pgfpathlineto{\pgfqpoint{3.196227in}{0.400455in}}%
\pgfpathlineto{\pgfqpoint{3.197516in}{0.402583in}}%
\pgfpathlineto{\pgfqpoint{3.198805in}{0.400711in}}%
\pgfpathlineto{\pgfqpoint{3.200094in}{0.400542in}}%
\pgfpathlineto{\pgfqpoint{3.203962in}{0.402275in}}%
\pgfpathlineto{\pgfqpoint{3.205251in}{0.424868in}}%
\pgfpathlineto{\pgfqpoint{3.206540in}{0.409608in}}%
\pgfpathlineto{\pgfqpoint{3.207830in}{0.409326in}}%
\pgfpathlineto{\pgfqpoint{3.209119in}{0.406076in}}%
\pgfpathlineto{\pgfqpoint{3.212986in}{0.400376in}}%
\pgfpathlineto{\pgfqpoint{3.214276in}{0.420615in}}%
\pgfpathlineto{\pgfqpoint{3.215565in}{0.401755in}}%
\pgfpathlineto{\pgfqpoint{3.216854in}{0.400704in}}%
\pgfpathlineto{\pgfqpoint{3.218143in}{0.691013in}}%
\pgfpathlineto{\pgfqpoint{3.222011in}{0.401538in}}%
\pgfpathlineto{\pgfqpoint{3.223300in}{0.400322in}}%
\pgfpathlineto{\pgfqpoint{3.224589in}{0.409477in}}%
\pgfpathlineto{\pgfqpoint{3.227168in}{0.402228in}}%
\pgfpathlineto{\pgfqpoint{3.231035in}{0.405934in}}%
\pgfpathlineto{\pgfqpoint{3.232324in}{0.403623in}}%
\pgfpathlineto{\pgfqpoint{3.233614in}{0.414134in}}%
\pgfpathlineto{\pgfqpoint{3.234903in}{0.400601in}}%
\pgfpathlineto{\pgfqpoint{3.236192in}{0.400320in}}%
\pgfpathlineto{\pgfqpoint{3.241349in}{0.404198in}}%
\pgfpathlineto{\pgfqpoint{3.242638in}{0.400550in}}%
\pgfpathlineto{\pgfqpoint{3.243927in}{0.403052in}}%
\pgfpathlineto{\pgfqpoint{3.245216in}{0.401373in}}%
\pgfpathlineto{\pgfqpoint{3.249084in}{0.403651in}}%
\pgfpathlineto{\pgfqpoint{3.250373in}{0.400341in}}%
\pgfpathlineto{\pgfqpoint{3.251662in}{0.420485in}}%
\pgfpathlineto{\pgfqpoint{3.252951in}{0.404104in}}%
\pgfpathlineto{\pgfqpoint{3.254241in}{0.401554in}}%
\pgfpathlineto{\pgfqpoint{3.260687in}{0.400460in}}%
\pgfpathlineto{\pgfqpoint{3.261976in}{0.400695in}}%
\pgfpathlineto{\pgfqpoint{3.263265in}{0.407210in}}%
\pgfpathlineto{\pgfqpoint{3.267133in}{0.402796in}}%
\pgfpathlineto{\pgfqpoint{3.268422in}{0.405861in}}%
\pgfpathlineto{\pgfqpoint{3.271000in}{0.400537in}}%
\pgfpathlineto{\pgfqpoint{3.272289in}{0.400934in}}%
\pgfpathlineto{\pgfqpoint{3.276157in}{0.420311in}}%
\pgfpathlineto{\pgfqpoint{3.277446in}{0.423244in}}%
\pgfpathlineto{\pgfqpoint{3.278735in}{0.404145in}}%
\pgfpathlineto{\pgfqpoint{3.280025in}{0.402336in}}%
\pgfpathlineto{\pgfqpoint{3.281314in}{0.402320in}}%
\pgfpathlineto{\pgfqpoint{3.289049in}{0.400814in}}%
\pgfpathlineto{\pgfqpoint{3.290338in}{0.400641in}}%
\pgfpathlineto{\pgfqpoint{3.295495in}{0.406377in}}%
\pgfpathlineto{\pgfqpoint{3.296784in}{0.401358in}}%
\pgfpathlineto{\pgfqpoint{3.298073in}{0.400339in}}%
\pgfpathlineto{\pgfqpoint{3.299362in}{0.402452in}}%
\pgfpathlineto{\pgfqpoint{3.304519in}{0.400273in}}%
\pgfpathlineto{\pgfqpoint{3.307098in}{0.400884in}}%
\pgfpathlineto{\pgfqpoint{3.312254in}{0.416430in}}%
\pgfpathlineto{\pgfqpoint{3.313544in}{0.400390in}}%
\pgfpathlineto{\pgfqpoint{3.314833in}{0.400279in}}%
\pgfpathlineto{\pgfqpoint{3.316122in}{0.434980in}}%
\pgfpathlineto{\pgfqpoint{3.317411in}{0.400271in}}%
\pgfpathlineto{\pgfqpoint{3.321279in}{0.408750in}}%
\pgfpathlineto{\pgfqpoint{3.322568in}{0.401373in}}%
\pgfpathlineto{\pgfqpoint{3.323857in}{0.401654in}}%
\pgfpathlineto{\pgfqpoint{3.325146in}{0.401129in}}%
\pgfpathlineto{\pgfqpoint{3.326436in}{0.402714in}}%
\pgfpathlineto{\pgfqpoint{3.330303in}{0.400300in}}%
\pgfpathlineto{\pgfqpoint{3.331592in}{0.427858in}}%
\pgfpathlineto{\pgfqpoint{3.332882in}{0.400413in}}%
\pgfpathlineto{\pgfqpoint{3.334171in}{0.400775in}}%
\pgfpathlineto{\pgfqpoint{3.335460in}{0.404999in}}%
\pgfpathlineto{\pgfqpoint{3.339328in}{0.400780in}}%
\pgfpathlineto{\pgfqpoint{3.340617in}{0.400384in}}%
\pgfpathlineto{\pgfqpoint{3.341906in}{0.401369in}}%
\pgfpathlineto{\pgfqpoint{3.343195in}{0.400666in}}%
\pgfpathlineto{\pgfqpoint{3.344484in}{0.406983in}}%
\pgfpathlineto{\pgfqpoint{3.348352in}{0.404413in}}%
\pgfpathlineto{\pgfqpoint{3.349641in}{0.400381in}}%
\pgfpathlineto{\pgfqpoint{3.350930in}{0.425347in}}%
\pgfpathlineto{\pgfqpoint{3.352220in}{0.400316in}}%
\pgfpathlineto{\pgfqpoint{3.353509in}{0.410389in}}%
\pgfpathlineto{\pgfqpoint{3.357376in}{0.411568in}}%
\pgfpathlineto{\pgfqpoint{3.358665in}{0.405403in}}%
\pgfpathlineto{\pgfqpoint{3.359955in}{0.414712in}}%
\pgfpathlineto{\pgfqpoint{3.361244in}{0.401144in}}%
\pgfpathlineto{\pgfqpoint{3.362533in}{0.402532in}}%
\pgfpathlineto{\pgfqpoint{3.367690in}{0.400313in}}%
\pgfpathlineto{\pgfqpoint{3.368979in}{0.404556in}}%
\pgfpathlineto{\pgfqpoint{3.370268in}{0.400690in}}%
\pgfpathlineto{\pgfqpoint{3.371557in}{0.409142in}}%
\pgfpathlineto{\pgfqpoint{3.375425in}{0.402832in}}%
\pgfpathlineto{\pgfqpoint{3.376714in}{0.409308in}}%
\pgfpathlineto{\pgfqpoint{3.378003in}{0.401629in}}%
\pgfpathlineto{\pgfqpoint{3.379293in}{0.400271in}}%
\pgfpathlineto{\pgfqpoint{3.380582in}{0.437063in}}%
\pgfpathlineto{\pgfqpoint{3.384449in}{0.418798in}}%
\pgfpathlineto{\pgfqpoint{3.385739in}{0.440943in}}%
\pgfpathlineto{\pgfqpoint{3.387028in}{0.401216in}}%
\pgfpathlineto{\pgfqpoint{3.388317in}{0.421782in}}%
\pgfpathlineto{\pgfqpoint{3.389606in}{0.400271in}}%
\pgfpathlineto{\pgfqpoint{3.393474in}{0.406465in}}%
\pgfpathlineto{\pgfqpoint{3.396052in}{0.400698in}}%
\pgfpathlineto{\pgfqpoint{3.397341in}{0.405898in}}%
\pgfpathlineto{\pgfqpoint{3.398631in}{0.442626in}}%
\pgfpathlineto{\pgfqpoint{3.402498in}{0.402271in}}%
\pgfpathlineto{\pgfqpoint{3.403787in}{0.401955in}}%
\pgfpathlineto{\pgfqpoint{3.406366in}{0.452081in}}%
\pgfpathlineto{\pgfqpoint{3.407655in}{0.406931in}}%
\pgfpathlineto{\pgfqpoint{3.411523in}{0.403410in}}%
\pgfpathlineto{\pgfqpoint{3.415390in}{0.409552in}}%
\pgfpathlineto{\pgfqpoint{3.416679in}{0.407660in}}%
\pgfpathlineto{\pgfqpoint{3.420547in}{0.422272in}}%
\pgfpathlineto{\pgfqpoint{3.421836in}{0.401845in}}%
\pgfpathlineto{\pgfqpoint{3.423125in}{0.409624in}}%
\pgfpathlineto{\pgfqpoint{3.424414in}{0.401123in}}%
\pgfpathlineto{\pgfqpoint{3.425704in}{0.415304in}}%
\pgfpathlineto{\pgfqpoint{3.429571in}{0.400430in}}%
\pgfpathlineto{\pgfqpoint{3.430860in}{0.408883in}}%
\pgfpathlineto{\pgfqpoint{3.432150in}{0.407022in}}%
\pgfpathlineto{\pgfqpoint{3.433439in}{0.400573in}}%
\pgfpathlineto{\pgfqpoint{3.434728in}{0.433129in}}%
\pgfpathlineto{\pgfqpoint{3.439885in}{0.402298in}}%
\pgfpathlineto{\pgfqpoint{3.441174in}{0.410668in}}%
\pgfpathlineto{\pgfqpoint{3.442463in}{0.423763in}}%
\pgfpathlineto{\pgfqpoint{3.443752in}{0.402801in}}%
\pgfpathlineto{\pgfqpoint{3.447620in}{0.402680in}}%
\pgfpathlineto{\pgfqpoint{3.448909in}{0.424893in}}%
\pgfpathlineto{\pgfqpoint{3.450198in}{0.402716in}}%
\pgfpathlineto{\pgfqpoint{3.451488in}{0.407015in}}%
\pgfpathlineto{\pgfqpoint{3.452777in}{0.401503in}}%
\pgfpathlineto{\pgfqpoint{3.456644in}{0.405323in}}%
\pgfpathlineto{\pgfqpoint{3.457934in}{0.400850in}}%
\pgfpathlineto{\pgfqpoint{3.459223in}{0.401668in}}%
\pgfpathlineto{\pgfqpoint{3.460512in}{0.431800in}}%
\pgfpathlineto{\pgfqpoint{3.461801in}{0.405643in}}%
\pgfpathlineto{\pgfqpoint{3.465669in}{0.444895in}}%
\pgfpathlineto{\pgfqpoint{3.466958in}{0.408265in}}%
\pgfpathlineto{\pgfqpoint{3.468247in}{0.410142in}}%
\pgfpathlineto{\pgfqpoint{3.469536in}{0.418506in}}%
\pgfpathlineto{\pgfqpoint{3.470826in}{0.401906in}}%
\pgfpathlineto{\pgfqpoint{3.474693in}{0.407130in}}%
\pgfpathlineto{\pgfqpoint{3.475982in}{0.410000in}}%
\pgfpathlineto{\pgfqpoint{3.477271in}{0.410953in}}%
\pgfpathlineto{\pgfqpoint{3.478561in}{0.400294in}}%
\pgfpathlineto{\pgfqpoint{3.479850in}{0.400506in}}%
\pgfpathlineto{\pgfqpoint{3.483717in}{0.402824in}}%
\pgfpathlineto{\pgfqpoint{3.485007in}{0.400273in}}%
\pgfpathlineto{\pgfqpoint{3.486296in}{0.400293in}}%
\pgfpathlineto{\pgfqpoint{3.487585in}{0.400880in}}%
\pgfpathlineto{\pgfqpoint{3.488874in}{0.402156in}}%
\pgfpathlineto{\pgfqpoint{3.494031in}{0.410745in}}%
\pgfpathlineto{\pgfqpoint{3.495320in}{0.400628in}}%
\pgfpathlineto{\pgfqpoint{3.496609in}{0.426036in}}%
\pgfpathlineto{\pgfqpoint{3.497899in}{0.400501in}}%
\pgfpathlineto{\pgfqpoint{3.501766in}{0.438430in}}%
\pgfpathlineto{\pgfqpoint{3.503055in}{0.400319in}}%
\pgfpathlineto{\pgfqpoint{3.504345in}{0.402075in}}%
\pgfpathlineto{\pgfqpoint{3.505634in}{0.401096in}}%
\pgfpathlineto{\pgfqpoint{3.506923in}{0.408132in}}%
\pgfpathlineto{\pgfqpoint{3.510791in}{0.403235in}}%
\pgfpathlineto{\pgfqpoint{3.512080in}{0.404626in}}%
\pgfpathlineto{\pgfqpoint{3.514658in}{0.401820in}}%
\pgfpathlineto{\pgfqpoint{3.515947in}{0.401039in}}%
\pgfpathlineto{\pgfqpoint{3.519815in}{0.485383in}}%
\pgfpathlineto{\pgfqpoint{3.521104in}{0.400361in}}%
\pgfpathlineto{\pgfqpoint{3.522393in}{0.412249in}}%
\pgfpathlineto{\pgfqpoint{3.523683in}{0.406761in}}%
\pgfpathlineto{\pgfqpoint{3.524972in}{0.405956in}}%
\pgfpathlineto{\pgfqpoint{3.528839in}{0.400753in}}%
\pgfpathlineto{\pgfqpoint{3.530129in}{0.400571in}}%
\pgfpathlineto{\pgfqpoint{3.531418in}{0.401242in}}%
\pgfpathlineto{\pgfqpoint{3.532707in}{0.411233in}}%
\pgfpathlineto{\pgfqpoint{3.533996in}{0.403407in}}%
\pgfpathlineto{\pgfqpoint{3.537864in}{0.401217in}}%
\pgfpathlineto{\pgfqpoint{3.539153in}{0.401432in}}%
\pgfpathlineto{\pgfqpoint{3.540442in}{0.416235in}}%
\pgfpathlineto{\pgfqpoint{3.541731in}{0.401647in}}%
\pgfpathlineto{\pgfqpoint{3.543020in}{0.403948in}}%
\pgfpathlineto{\pgfqpoint{3.546888in}{0.401813in}}%
\pgfpathlineto{\pgfqpoint{3.548177in}{0.407424in}}%
\pgfpathlineto{\pgfqpoint{3.549466in}{0.401070in}}%
\pgfpathlineto{\pgfqpoint{3.550756in}{0.407883in}}%
\pgfpathlineto{\pgfqpoint{3.552045in}{0.403886in}}%
\pgfpathlineto{\pgfqpoint{3.555912in}{0.400276in}}%
\pgfpathlineto{\pgfqpoint{3.557202in}{0.409887in}}%
\pgfpathlineto{\pgfqpoint{3.558491in}{0.414290in}}%
\pgfpathlineto{\pgfqpoint{3.559780in}{0.401455in}}%
\pgfpathlineto{\pgfqpoint{3.561069in}{0.400371in}}%
\pgfpathlineto{\pgfqpoint{3.564937in}{0.401528in}}%
\pgfpathlineto{\pgfqpoint{3.566226in}{0.400315in}}%
\pgfpathlineto{\pgfqpoint{3.567515in}{0.414414in}}%
\pgfpathlineto{\pgfqpoint{3.568804in}{0.400369in}}%
\pgfpathlineto{\pgfqpoint{3.570094in}{0.402002in}}%
\pgfpathlineto{\pgfqpoint{3.573961in}{0.403866in}}%
\pgfpathlineto{\pgfqpoint{3.575250in}{0.400613in}}%
\pgfpathlineto{\pgfqpoint{3.576540in}{0.401880in}}%
\pgfpathlineto{\pgfqpoint{3.579118in}{0.401069in}}%
\pgfpathlineto{\pgfqpoint{3.582986in}{0.402896in}}%
\pgfpathlineto{\pgfqpoint{3.584275in}{0.404128in}}%
\pgfpathlineto{\pgfqpoint{3.585564in}{0.435349in}}%
\pgfpathlineto{\pgfqpoint{3.586853in}{0.411402in}}%
\pgfpathlineto{\pgfqpoint{3.588142in}{0.417411in}}%
\pgfpathlineto{\pgfqpoint{3.592010in}{0.400478in}}%
\pgfpathlineto{\pgfqpoint{3.593299in}{0.426650in}}%
\pgfpathlineto{\pgfqpoint{3.594588in}{0.404864in}}%
\pgfpathlineto{\pgfqpoint{3.597167in}{0.415458in}}%
\pgfpathlineto{\pgfqpoint{3.601034in}{0.402243in}}%
\pgfpathlineto{\pgfqpoint{3.602323in}{0.411171in}}%
\pgfpathlineto{\pgfqpoint{3.603613in}{0.542122in}}%
\pgfpathlineto{\pgfqpoint{3.606191in}{0.440346in}}%
\pgfpathlineto{\pgfqpoint{3.610059in}{0.532101in}}%
\pgfpathlineto{\pgfqpoint{3.611348in}{0.416426in}}%
\pgfpathlineto{\pgfqpoint{3.612637in}{0.416651in}}%
\pgfpathlineto{\pgfqpoint{3.613926in}{0.432739in}}%
\pgfpathlineto{\pgfqpoint{3.615215in}{0.416932in}}%
\pgfpathlineto{\pgfqpoint{3.619083in}{0.400888in}}%
\pgfpathlineto{\pgfqpoint{3.620372in}{0.413772in}}%
\pgfpathlineto{\pgfqpoint{3.621661in}{0.400530in}}%
\pgfpathlineto{\pgfqpoint{3.622951in}{0.402026in}}%
\pgfpathlineto{\pgfqpoint{3.624240in}{0.409367in}}%
\pgfpathlineto{\pgfqpoint{3.628107in}{0.401043in}}%
\pgfpathlineto{\pgfqpoint{3.629397in}{0.422987in}}%
\pgfpathlineto{\pgfqpoint{3.630686in}{0.400276in}}%
\pgfpathlineto{\pgfqpoint{3.631975in}{0.401256in}}%
\pgfpathlineto{\pgfqpoint{3.638421in}{0.401246in}}%
\pgfpathlineto{\pgfqpoint{3.639710in}{0.412190in}}%
\pgfpathlineto{\pgfqpoint{3.640999in}{0.401504in}}%
\pgfpathlineto{\pgfqpoint{3.642289in}{0.401504in}}%
\pgfpathlineto{\pgfqpoint{3.642289in}{0.401504in}}%
\pgfusepath{stroke}%
\end{pgfscope}%
\begin{pgfscope}%
\pgfsetrectcap%
\pgfsetmiterjoin%
\pgfsetlinewidth{0.803000pt}%
\definecolor{currentstroke}{rgb}{1.000000,1.000000,1.000000}%
\pgfsetstrokecolor{currentstroke}%
\pgfsetdash{}{0pt}%
\pgfpathmoveto{\pgfqpoint{0.683198in}{0.331635in}}%
\pgfpathlineto{\pgfqpoint{0.683198in}{1.841635in}}%
\pgfusepath{stroke}%
\end{pgfscope}%
\begin{pgfscope}%
\pgfsetrectcap%
\pgfsetmiterjoin%
\pgfsetlinewidth{0.803000pt}%
\definecolor{currentstroke}{rgb}{1.000000,1.000000,1.000000}%
\pgfsetstrokecolor{currentstroke}%
\pgfsetdash{}{0pt}%
\pgfpathmoveto{\pgfqpoint{3.783198in}{0.331635in}}%
\pgfpathlineto{\pgfqpoint{3.783198in}{1.841635in}}%
\pgfusepath{stroke}%
\end{pgfscope}%
\begin{pgfscope}%
\pgfsetrectcap%
\pgfsetmiterjoin%
\pgfsetlinewidth{0.803000pt}%
\definecolor{currentstroke}{rgb}{1.000000,1.000000,1.000000}%
\pgfsetstrokecolor{currentstroke}%
\pgfsetdash{}{0pt}%
\pgfpathmoveto{\pgfqpoint{0.683198in}{0.331635in}}%
\pgfpathlineto{\pgfqpoint{3.783198in}{0.331635in}}%
\pgfusepath{stroke}%
\end{pgfscope}%
\begin{pgfscope}%
\pgfsetrectcap%
\pgfsetmiterjoin%
\pgfsetlinewidth{0.803000pt}%
\definecolor{currentstroke}{rgb}{1.000000,1.000000,1.000000}%
\pgfsetstrokecolor{currentstroke}%
\pgfsetdash{}{0pt}%
\pgfpathmoveto{\pgfqpoint{0.683198in}{1.841635in}}%
\pgfpathlineto{\pgfqpoint{3.783198in}{1.841635in}}%
\pgfusepath{stroke}%
\end{pgfscope}%
\end{pgfpicture}%
\makeatother%
\endgroup%

    %% Creator: Matplotlib, PGF backend
%%
%% To include the figure in your LaTeX document, write
%%   \input{<filename>.pgf}
%%
%% Make sure the required packages are loaded in your preamble
%%   \usepackage{pgf}
%%
%% Figures using additional raster images can only be included by \input if
%% they are in the same directory as the main LaTeX file. For loading figures
%% from other directories you can use the `import` package
%%   \usepackage{import}
%% and then include the figures with
%%   \import{<path to file>}{<filename>.pgf}
%%
%% Matplotlib used the following preamble
%%   \usepackage{fontspec}
%%   \setmainfont{DejaVuSerif.ttf}[Path=/opt/tljh/user/lib/python3.6/site-packages/matplotlib/mpl-data/fonts/ttf/]
%%   \setsansfont{DejaVuSans.ttf}[Path=/opt/tljh/user/lib/python3.6/site-packages/matplotlib/mpl-data/fonts/ttf/]
%%   \setmonofont{DejaVuSansMono.ttf}[Path=/opt/tljh/user/lib/python3.6/site-packages/matplotlib/mpl-data/fonts/ttf/]
%%
\begingroup%
\makeatletter%
\begin{pgfpicture}%
\pgfpathrectangle{\pgfpointorigin}{\pgfqpoint{3.834522in}{1.941635in}}%
\pgfusepath{use as bounding box, clip}%
\begin{pgfscope}%
\pgfsetbuttcap%
\pgfsetmiterjoin%
\definecolor{currentfill}{rgb}{1.000000,1.000000,1.000000}%
\pgfsetfillcolor{currentfill}%
\pgfsetlinewidth{0.000000pt}%
\definecolor{currentstroke}{rgb}{1.000000,1.000000,1.000000}%
\pgfsetstrokecolor{currentstroke}%
\pgfsetdash{}{0pt}%
\pgfpathmoveto{\pgfqpoint{0.000000in}{0.000000in}}%
\pgfpathlineto{\pgfqpoint{3.834522in}{0.000000in}}%
\pgfpathlineto{\pgfqpoint{3.834522in}{1.941635in}}%
\pgfpathlineto{\pgfqpoint{0.000000in}{1.941635in}}%
\pgfpathclose%
\pgfusepath{fill}%
\end{pgfscope}%
\begin{pgfscope}%
\pgfsetbuttcap%
\pgfsetmiterjoin%
\definecolor{currentfill}{rgb}{0.917647,0.917647,0.949020}%
\pgfsetfillcolor{currentfill}%
\pgfsetlinewidth{0.000000pt}%
\definecolor{currentstroke}{rgb}{0.000000,0.000000,0.000000}%
\pgfsetstrokecolor{currentstroke}%
\pgfsetstrokeopacity{0.000000}%
\pgfsetdash{}{0pt}%
\pgfpathmoveto{\pgfqpoint{0.594832in}{0.331635in}}%
\pgfpathlineto{\pgfqpoint{3.694832in}{0.331635in}}%
\pgfpathlineto{\pgfqpoint{3.694832in}{1.841635in}}%
\pgfpathlineto{\pgfqpoint{0.594832in}{1.841635in}}%
\pgfpathclose%
\pgfusepath{fill}%
\end{pgfscope}%
\begin{pgfscope}%
\pgfpathrectangle{\pgfqpoint{0.594832in}{0.331635in}}{\pgfqpoint{3.100000in}{1.510000in}}%
\pgfusepath{clip}%
\pgfsetroundcap%
\pgfsetroundjoin%
\pgfsetlinewidth{0.803000pt}%
\definecolor{currentstroke}{rgb}{1.000000,1.000000,1.000000}%
\pgfsetstrokecolor{currentstroke}%
\pgfsetdash{}{0pt}%
\pgfpathmoveto{\pgfqpoint{0.731874in}{0.331635in}}%
\pgfpathlineto{\pgfqpoint{0.731874in}{1.841635in}}%
\pgfusepath{stroke}%
\end{pgfscope}%
\begin{pgfscope}%
\definecolor{textcolor}{rgb}{0.150000,0.150000,0.150000}%
\pgfsetstrokecolor{textcolor}%
\pgfsetfillcolor{textcolor}%
\pgftext[x=0.731874in,y=0.234413in,,top]{\color{textcolor}\rmfamily\fontsize{10.000000}{12.000000}\selectfont 2012}%
\end{pgfscope}%
\begin{pgfscope}%
\pgfpathrectangle{\pgfqpoint{0.594832in}{0.331635in}}{\pgfqpoint{3.100000in}{1.510000in}}%
\pgfusepath{clip}%
\pgfsetroundcap%
\pgfsetroundjoin%
\pgfsetlinewidth{0.803000pt}%
\definecolor{currentstroke}{rgb}{1.000000,1.000000,1.000000}%
\pgfsetstrokecolor{currentstroke}%
\pgfsetdash{}{0pt}%
\pgfpathmoveto{\pgfqpoint{1.203719in}{0.331635in}}%
\pgfpathlineto{\pgfqpoint{1.203719in}{1.841635in}}%
\pgfusepath{stroke}%
\end{pgfscope}%
\begin{pgfscope}%
\definecolor{textcolor}{rgb}{0.150000,0.150000,0.150000}%
\pgfsetstrokecolor{textcolor}%
\pgfsetfillcolor{textcolor}%
\pgftext[x=1.203719in,y=0.234413in,,top]{\color{textcolor}\rmfamily\fontsize{10.000000}{12.000000}\selectfont 2013}%
\end{pgfscope}%
\begin{pgfscope}%
\pgfpathrectangle{\pgfqpoint{0.594832in}{0.331635in}}{\pgfqpoint{3.100000in}{1.510000in}}%
\pgfusepath{clip}%
\pgfsetroundcap%
\pgfsetroundjoin%
\pgfsetlinewidth{0.803000pt}%
\definecolor{currentstroke}{rgb}{1.000000,1.000000,1.000000}%
\pgfsetstrokecolor{currentstroke}%
\pgfsetdash{}{0pt}%
\pgfpathmoveto{\pgfqpoint{1.674276in}{0.331635in}}%
\pgfpathlineto{\pgfqpoint{1.674276in}{1.841635in}}%
\pgfusepath{stroke}%
\end{pgfscope}%
\begin{pgfscope}%
\definecolor{textcolor}{rgb}{0.150000,0.150000,0.150000}%
\pgfsetstrokecolor{textcolor}%
\pgfsetfillcolor{textcolor}%
\pgftext[x=1.674276in,y=0.234413in,,top]{\color{textcolor}\rmfamily\fontsize{10.000000}{12.000000}\selectfont 2014}%
\end{pgfscope}%
\begin{pgfscope}%
\pgfpathrectangle{\pgfqpoint{0.594832in}{0.331635in}}{\pgfqpoint{3.100000in}{1.510000in}}%
\pgfusepath{clip}%
\pgfsetroundcap%
\pgfsetroundjoin%
\pgfsetlinewidth{0.803000pt}%
\definecolor{currentstroke}{rgb}{1.000000,1.000000,1.000000}%
\pgfsetstrokecolor{currentstroke}%
\pgfsetdash{}{0pt}%
\pgfpathmoveto{\pgfqpoint{2.144832in}{0.331635in}}%
\pgfpathlineto{\pgfqpoint{2.144832in}{1.841635in}}%
\pgfusepath{stroke}%
\end{pgfscope}%
\begin{pgfscope}%
\definecolor{textcolor}{rgb}{0.150000,0.150000,0.150000}%
\pgfsetstrokecolor{textcolor}%
\pgfsetfillcolor{textcolor}%
\pgftext[x=2.144832in,y=0.234413in,,top]{\color{textcolor}\rmfamily\fontsize{10.000000}{12.000000}\selectfont 2015}%
\end{pgfscope}%
\begin{pgfscope}%
\pgfpathrectangle{\pgfqpoint{0.594832in}{0.331635in}}{\pgfqpoint{3.100000in}{1.510000in}}%
\pgfusepath{clip}%
\pgfsetroundcap%
\pgfsetroundjoin%
\pgfsetlinewidth{0.803000pt}%
\definecolor{currentstroke}{rgb}{1.000000,1.000000,1.000000}%
\pgfsetstrokecolor{currentstroke}%
\pgfsetdash{}{0pt}%
\pgfpathmoveto{\pgfqpoint{2.615389in}{0.331635in}}%
\pgfpathlineto{\pgfqpoint{2.615389in}{1.841635in}}%
\pgfusepath{stroke}%
\end{pgfscope}%
\begin{pgfscope}%
\definecolor{textcolor}{rgb}{0.150000,0.150000,0.150000}%
\pgfsetstrokecolor{textcolor}%
\pgfsetfillcolor{textcolor}%
\pgftext[x=2.615389in,y=0.234413in,,top]{\color{textcolor}\rmfamily\fontsize{10.000000}{12.000000}\selectfont 2016}%
\end{pgfscope}%
\begin{pgfscope}%
\pgfpathrectangle{\pgfqpoint{0.594832in}{0.331635in}}{\pgfqpoint{3.100000in}{1.510000in}}%
\pgfusepath{clip}%
\pgfsetroundcap%
\pgfsetroundjoin%
\pgfsetlinewidth{0.803000pt}%
\definecolor{currentstroke}{rgb}{1.000000,1.000000,1.000000}%
\pgfsetstrokecolor{currentstroke}%
\pgfsetdash{}{0pt}%
\pgfpathmoveto{\pgfqpoint{3.087234in}{0.331635in}}%
\pgfpathlineto{\pgfqpoint{3.087234in}{1.841635in}}%
\pgfusepath{stroke}%
\end{pgfscope}%
\begin{pgfscope}%
\definecolor{textcolor}{rgb}{0.150000,0.150000,0.150000}%
\pgfsetstrokecolor{textcolor}%
\pgfsetfillcolor{textcolor}%
\pgftext[x=3.087234in,y=0.234413in,,top]{\color{textcolor}\rmfamily\fontsize{10.000000}{12.000000}\selectfont 2017}%
\end{pgfscope}%
\begin{pgfscope}%
\pgfpathrectangle{\pgfqpoint{0.594832in}{0.331635in}}{\pgfqpoint{3.100000in}{1.510000in}}%
\pgfusepath{clip}%
\pgfsetroundcap%
\pgfsetroundjoin%
\pgfsetlinewidth{0.803000pt}%
\definecolor{currentstroke}{rgb}{1.000000,1.000000,1.000000}%
\pgfsetstrokecolor{currentstroke}%
\pgfsetdash{}{0pt}%
\pgfpathmoveto{\pgfqpoint{3.557791in}{0.331635in}}%
\pgfpathlineto{\pgfqpoint{3.557791in}{1.841635in}}%
\pgfusepath{stroke}%
\end{pgfscope}%
\begin{pgfscope}%
\definecolor{textcolor}{rgb}{0.150000,0.150000,0.150000}%
\pgfsetstrokecolor{textcolor}%
\pgfsetfillcolor{textcolor}%
\pgftext[x=3.557791in,y=0.234413in,,top]{\color{textcolor}\rmfamily\fontsize{10.000000}{12.000000}\selectfont 2018}%
\end{pgfscope}%
\begin{pgfscope}%
\pgfpathrectangle{\pgfqpoint{0.594832in}{0.331635in}}{\pgfqpoint{3.100000in}{1.510000in}}%
\pgfusepath{clip}%
\pgfsetroundcap%
\pgfsetroundjoin%
\pgfsetlinewidth{0.803000pt}%
\definecolor{currentstroke}{rgb}{1.000000,1.000000,1.000000}%
\pgfsetstrokecolor{currentstroke}%
\pgfsetdash{}{0pt}%
\pgfpathmoveto{\pgfqpoint{0.594832in}{0.400271in}}%
\pgfpathlineto{\pgfqpoint{3.694832in}{0.400271in}}%
\pgfusepath{stroke}%
\end{pgfscope}%
\begin{pgfscope}%
\definecolor{textcolor}{rgb}{0.150000,0.150000,0.150000}%
\pgfsetstrokecolor{textcolor}%
\pgfsetfillcolor{textcolor}%
\pgftext[x=0.100000in,y=0.347510in,left,base]{\color{textcolor}\rmfamily\fontsize{10.000000}{12.000000}\selectfont 0.000}%
\end{pgfscope}%
\begin{pgfscope}%
\pgfpathrectangle{\pgfqpoint{0.594832in}{0.331635in}}{\pgfqpoint{3.100000in}{1.510000in}}%
\pgfusepath{clip}%
\pgfsetroundcap%
\pgfsetroundjoin%
\pgfsetlinewidth{0.803000pt}%
\definecolor{currentstroke}{rgb}{1.000000,1.000000,1.000000}%
\pgfsetstrokecolor{currentstroke}%
\pgfsetdash{}{0pt}%
\pgfpathmoveto{\pgfqpoint{0.594832in}{0.702715in}}%
\pgfpathlineto{\pgfqpoint{3.694832in}{0.702715in}}%
\pgfusepath{stroke}%
\end{pgfscope}%
\begin{pgfscope}%
\definecolor{textcolor}{rgb}{0.150000,0.150000,0.150000}%
\pgfsetstrokecolor{textcolor}%
\pgfsetfillcolor{textcolor}%
\pgftext[x=0.100000in,y=0.649954in,left,base]{\color{textcolor}\rmfamily\fontsize{10.000000}{12.000000}\selectfont 0.002}%
\end{pgfscope}%
\begin{pgfscope}%
\pgfpathrectangle{\pgfqpoint{0.594832in}{0.331635in}}{\pgfqpoint{3.100000in}{1.510000in}}%
\pgfusepath{clip}%
\pgfsetroundcap%
\pgfsetroundjoin%
\pgfsetlinewidth{0.803000pt}%
\definecolor{currentstroke}{rgb}{1.000000,1.000000,1.000000}%
\pgfsetstrokecolor{currentstroke}%
\pgfsetdash{}{0pt}%
\pgfpathmoveto{\pgfqpoint{0.594832in}{1.005159in}}%
\pgfpathlineto{\pgfqpoint{3.694832in}{1.005159in}}%
\pgfusepath{stroke}%
\end{pgfscope}%
\begin{pgfscope}%
\definecolor{textcolor}{rgb}{0.150000,0.150000,0.150000}%
\pgfsetstrokecolor{textcolor}%
\pgfsetfillcolor{textcolor}%
\pgftext[x=0.100000in,y=0.952398in,left,base]{\color{textcolor}\rmfamily\fontsize{10.000000}{12.000000}\selectfont 0.004}%
\end{pgfscope}%
\begin{pgfscope}%
\pgfpathrectangle{\pgfqpoint{0.594832in}{0.331635in}}{\pgfqpoint{3.100000in}{1.510000in}}%
\pgfusepath{clip}%
\pgfsetroundcap%
\pgfsetroundjoin%
\pgfsetlinewidth{0.803000pt}%
\definecolor{currentstroke}{rgb}{1.000000,1.000000,1.000000}%
\pgfsetstrokecolor{currentstroke}%
\pgfsetdash{}{0pt}%
\pgfpathmoveto{\pgfqpoint{0.594832in}{1.307603in}}%
\pgfpathlineto{\pgfqpoint{3.694832in}{1.307603in}}%
\pgfusepath{stroke}%
\end{pgfscope}%
\begin{pgfscope}%
\definecolor{textcolor}{rgb}{0.150000,0.150000,0.150000}%
\pgfsetstrokecolor{textcolor}%
\pgfsetfillcolor{textcolor}%
\pgftext[x=0.100000in,y=1.254841in,left,base]{\color{textcolor}\rmfamily\fontsize{10.000000}{12.000000}\selectfont 0.006}%
\end{pgfscope}%
\begin{pgfscope}%
\pgfpathrectangle{\pgfqpoint{0.594832in}{0.331635in}}{\pgfqpoint{3.100000in}{1.510000in}}%
\pgfusepath{clip}%
\pgfsetroundcap%
\pgfsetroundjoin%
\pgfsetlinewidth{0.803000pt}%
\definecolor{currentstroke}{rgb}{1.000000,1.000000,1.000000}%
\pgfsetstrokecolor{currentstroke}%
\pgfsetdash{}{0pt}%
\pgfpathmoveto{\pgfqpoint{0.594832in}{1.610047in}}%
\pgfpathlineto{\pgfqpoint{3.694832in}{1.610047in}}%
\pgfusepath{stroke}%
\end{pgfscope}%
\begin{pgfscope}%
\definecolor{textcolor}{rgb}{0.150000,0.150000,0.150000}%
\pgfsetstrokecolor{textcolor}%
\pgfsetfillcolor{textcolor}%
\pgftext[x=0.100000in,y=1.557285in,left,base]{\color{textcolor}\rmfamily\fontsize{10.000000}{12.000000}\selectfont 0.008}%
\end{pgfscope}%
\begin{pgfscope}%
\pgfpathrectangle{\pgfqpoint{0.594832in}{0.331635in}}{\pgfqpoint{3.100000in}{1.510000in}}%
\pgfusepath{clip}%
\pgfsetroundcap%
\pgfsetroundjoin%
\pgfsetlinewidth{1.505625pt}%
\definecolor{currentstroke}{rgb}{0.839216,0.152941,0.156863}%
\pgfsetstrokecolor{currentstroke}%
\pgfsetdash{}{0pt}%
\pgfpathmoveto{\pgfqpoint{0.735741in}{0.480358in}}%
\pgfpathlineto{\pgfqpoint{0.737031in}{0.420483in}}%
\pgfpathlineto{\pgfqpoint{0.738320in}{0.405743in}}%
\pgfpathlineto{\pgfqpoint{0.742187in}{0.412545in}}%
\pgfpathlineto{\pgfqpoint{0.743477in}{0.403299in}}%
\pgfpathlineto{\pgfqpoint{0.744766in}{0.410934in}}%
\pgfpathlineto{\pgfqpoint{0.746055in}{0.400858in}}%
\pgfpathlineto{\pgfqpoint{0.747344in}{0.486950in}}%
\pgfpathlineto{\pgfqpoint{0.752501in}{0.402748in}}%
\pgfpathlineto{\pgfqpoint{0.753790in}{0.428208in}}%
\pgfpathlineto{\pgfqpoint{0.755079in}{0.413791in}}%
\pgfpathlineto{\pgfqpoint{0.756369in}{0.525732in}}%
\pgfpathlineto{\pgfqpoint{0.760236in}{0.423652in}}%
\pgfpathlineto{\pgfqpoint{0.761525in}{0.407903in}}%
\pgfpathlineto{\pgfqpoint{0.762815in}{0.400271in}}%
\pgfpathlineto{\pgfqpoint{0.764104in}{0.405149in}}%
\pgfpathlineto{\pgfqpoint{0.765393in}{0.400305in}}%
\pgfpathlineto{\pgfqpoint{0.769261in}{0.400305in}}%
\pgfpathlineto{\pgfqpoint{0.770550in}{0.423585in}}%
\pgfpathlineto{\pgfqpoint{0.771839in}{0.404474in}}%
\pgfpathlineto{\pgfqpoint{0.773128in}{0.401137in}}%
\pgfpathlineto{\pgfqpoint{0.774417in}{0.444503in}}%
\pgfpathlineto{\pgfqpoint{0.778285in}{0.400305in}}%
\pgfpathlineto{\pgfqpoint{0.779574in}{0.401922in}}%
\pgfpathlineto{\pgfqpoint{0.780863in}{0.409963in}}%
\pgfpathlineto{\pgfqpoint{0.782152in}{0.400305in}}%
\pgfpathlineto{\pgfqpoint{0.783442in}{0.405923in}}%
\pgfpathlineto{\pgfqpoint{0.787309in}{0.400271in}}%
\pgfpathlineto{\pgfqpoint{0.788598in}{0.401915in}}%
\pgfpathlineto{\pgfqpoint{0.791177in}{0.413718in}}%
\pgfpathlineto{\pgfqpoint{0.792466in}{0.457876in}}%
\pgfpathlineto{\pgfqpoint{0.797623in}{0.408531in}}%
\pgfpathlineto{\pgfqpoint{0.798912in}{0.438445in}}%
\pgfpathlineto{\pgfqpoint{0.800201in}{0.401483in}}%
\pgfpathlineto{\pgfqpoint{0.801490in}{0.400575in}}%
\pgfpathlineto{\pgfqpoint{0.805358in}{0.407789in}}%
\pgfpathlineto{\pgfqpoint{0.806647in}{0.425944in}}%
\pgfpathlineto{\pgfqpoint{0.807936in}{0.425944in}}%
\pgfpathlineto{\pgfqpoint{0.809226in}{0.400404in}}%
\pgfpathlineto{\pgfqpoint{0.810515in}{0.401100in}}%
\pgfpathlineto{\pgfqpoint{0.814382in}{0.430469in}}%
\pgfpathlineto{\pgfqpoint{0.815672in}{0.401120in}}%
\pgfpathlineto{\pgfqpoint{0.816961in}{0.419561in}}%
\pgfpathlineto{\pgfqpoint{0.818250in}{0.401467in}}%
\pgfpathlineto{\pgfqpoint{0.819539in}{0.412186in}}%
\pgfpathlineto{\pgfqpoint{0.823407in}{0.401880in}}%
\pgfpathlineto{\pgfqpoint{0.824696in}{0.451990in}}%
\pgfpathlineto{\pgfqpoint{0.825985in}{0.400398in}}%
\pgfpathlineto{\pgfqpoint{0.827274in}{0.416913in}}%
\pgfpathlineto{\pgfqpoint{0.828564in}{0.400396in}}%
\pgfpathlineto{\pgfqpoint{0.835010in}{0.400396in}}%
\pgfpathlineto{\pgfqpoint{0.836299in}{0.403365in}}%
\pgfpathlineto{\pgfqpoint{0.837588in}{0.400395in}}%
\pgfpathlineto{\pgfqpoint{0.841456in}{0.419337in}}%
\pgfpathlineto{\pgfqpoint{0.842745in}{0.400271in}}%
\pgfpathlineto{\pgfqpoint{0.844034in}{0.429667in}}%
\pgfpathlineto{\pgfqpoint{0.845323in}{0.424285in}}%
\pgfpathlineto{\pgfqpoint{0.846612in}{0.400544in}}%
\pgfpathlineto{\pgfqpoint{0.850480in}{0.413521in}}%
\pgfpathlineto{\pgfqpoint{0.851769in}{0.414820in}}%
\pgfpathlineto{\pgfqpoint{0.853058in}{0.406258in}}%
\pgfpathlineto{\pgfqpoint{0.854347in}{0.403972in}}%
\pgfpathlineto{\pgfqpoint{0.859504in}{0.419510in}}%
\pgfpathlineto{\pgfqpoint{0.860793in}{0.418400in}}%
\pgfpathlineto{\pgfqpoint{0.862083in}{0.432383in}}%
\pgfpathlineto{\pgfqpoint{0.863372in}{0.475826in}}%
\pgfpathlineto{\pgfqpoint{0.864661in}{0.429068in}}%
\pgfpathlineto{\pgfqpoint{0.868529in}{0.419049in}}%
\pgfpathlineto{\pgfqpoint{0.869818in}{0.401013in}}%
\pgfpathlineto{\pgfqpoint{0.871107in}{0.453438in}}%
\pgfpathlineto{\pgfqpoint{0.872396in}{0.412668in}}%
\pgfpathlineto{\pgfqpoint{0.873685in}{0.401809in}}%
\pgfpathlineto{\pgfqpoint{0.877553in}{0.404828in}}%
\pgfpathlineto{\pgfqpoint{0.878842in}{0.404141in}}%
\pgfpathlineto{\pgfqpoint{0.880131in}{0.458548in}}%
\pgfpathlineto{\pgfqpoint{0.882710in}{0.405331in}}%
\pgfpathlineto{\pgfqpoint{0.886577in}{0.400301in}}%
\pgfpathlineto{\pgfqpoint{0.887867in}{0.456742in}}%
\pgfpathlineto{\pgfqpoint{0.889156in}{0.409469in}}%
\pgfpathlineto{\pgfqpoint{0.890445in}{0.431389in}}%
\pgfpathlineto{\pgfqpoint{0.891734in}{0.480478in}}%
\pgfpathlineto{\pgfqpoint{0.895602in}{0.404666in}}%
\pgfpathlineto{\pgfqpoint{0.896891in}{0.430176in}}%
\pgfpathlineto{\pgfqpoint{0.898180in}{0.406497in}}%
\pgfpathlineto{\pgfqpoint{0.899469in}{0.400782in}}%
\pgfpathlineto{\pgfqpoint{0.900759in}{0.430449in}}%
\pgfpathlineto{\pgfqpoint{0.904626in}{0.476294in}}%
\pgfpathlineto{\pgfqpoint{0.905915in}{0.404210in}}%
\pgfpathlineto{\pgfqpoint{0.907204in}{0.430130in}}%
\pgfpathlineto{\pgfqpoint{0.908494in}{0.421551in}}%
\pgfpathlineto{\pgfqpoint{0.909783in}{0.403733in}}%
\pgfpathlineto{\pgfqpoint{0.913650in}{0.401970in}}%
\pgfpathlineto{\pgfqpoint{0.914940in}{0.403743in}}%
\pgfpathlineto{\pgfqpoint{0.916229in}{0.479111in}}%
\pgfpathlineto{\pgfqpoint{0.917518in}{0.410737in}}%
\pgfpathlineto{\pgfqpoint{0.918807in}{0.402025in}}%
\pgfpathlineto{\pgfqpoint{0.923964in}{0.427861in}}%
\pgfpathlineto{\pgfqpoint{0.925253in}{0.400583in}}%
\pgfpathlineto{\pgfqpoint{0.926542in}{0.418789in}}%
\pgfpathlineto{\pgfqpoint{0.927832in}{0.514352in}}%
\pgfpathlineto{\pgfqpoint{0.931699in}{0.402674in}}%
\pgfpathlineto{\pgfqpoint{0.932988in}{0.435941in}}%
\pgfpathlineto{\pgfqpoint{0.934278in}{0.493013in}}%
\pgfpathlineto{\pgfqpoint{0.935567in}{0.403767in}}%
\pgfpathlineto{\pgfqpoint{0.936856in}{0.447510in}}%
\pgfpathlineto{\pgfqpoint{0.940724in}{0.437776in}}%
\pgfpathlineto{\pgfqpoint{0.942013in}{0.460764in}}%
\pgfpathlineto{\pgfqpoint{0.943302in}{0.400406in}}%
\pgfpathlineto{\pgfqpoint{0.944591in}{0.440705in}}%
\pgfpathlineto{\pgfqpoint{0.945880in}{0.427218in}}%
\pgfpathlineto{\pgfqpoint{0.949748in}{0.401406in}}%
\pgfpathlineto{\pgfqpoint{0.951037in}{0.401807in}}%
\pgfpathlineto{\pgfqpoint{0.952326in}{0.404033in}}%
\pgfpathlineto{\pgfqpoint{0.953616in}{0.580367in}}%
\pgfpathlineto{\pgfqpoint{0.954905in}{0.412131in}}%
\pgfpathlineto{\pgfqpoint{0.958772in}{0.569996in}}%
\pgfpathlineto{\pgfqpoint{0.960062in}{0.400830in}}%
\pgfpathlineto{\pgfqpoint{0.961351in}{0.410291in}}%
\pgfpathlineto{\pgfqpoint{0.962640in}{0.433814in}}%
\pgfpathlineto{\pgfqpoint{0.963929in}{0.549809in}}%
\pgfpathlineto{\pgfqpoint{0.967797in}{0.400305in}}%
\pgfpathlineto{\pgfqpoint{0.971664in}{0.419363in}}%
\pgfpathlineto{\pgfqpoint{0.972953in}{0.435120in}}%
\pgfpathlineto{\pgfqpoint{0.976821in}{0.400306in}}%
\pgfpathlineto{\pgfqpoint{0.978110in}{0.485138in}}%
\pgfpathlineto{\pgfqpoint{0.979399in}{0.406428in}}%
\pgfpathlineto{\pgfqpoint{0.980689in}{0.502000in}}%
\pgfpathlineto{\pgfqpoint{0.981978in}{0.463859in}}%
\pgfpathlineto{\pgfqpoint{0.985845in}{0.403996in}}%
\pgfpathlineto{\pgfqpoint{0.987135in}{0.415096in}}%
\pgfpathlineto{\pgfqpoint{0.988424in}{0.555074in}}%
\pgfpathlineto{\pgfqpoint{0.989713in}{0.405257in}}%
\pgfpathlineto{\pgfqpoint{0.991002in}{0.466006in}}%
\pgfpathlineto{\pgfqpoint{0.994870in}{0.416447in}}%
\pgfpathlineto{\pgfqpoint{0.996159in}{0.415244in}}%
\pgfpathlineto{\pgfqpoint{0.997448in}{0.404033in}}%
\pgfpathlineto{\pgfqpoint{1.000027in}{0.460149in}}%
\pgfpathlineto{\pgfqpoint{1.005183in}{0.400843in}}%
\pgfpathlineto{\pgfqpoint{1.006473in}{0.411761in}}%
\pgfpathlineto{\pgfqpoint{1.007762in}{0.400412in}}%
\pgfpathlineto{\pgfqpoint{1.009051in}{0.467044in}}%
\pgfpathlineto{\pgfqpoint{1.012919in}{0.401921in}}%
\pgfpathlineto{\pgfqpoint{1.014208in}{0.407768in}}%
\pgfpathlineto{\pgfqpoint{1.015497in}{0.402381in}}%
\pgfpathlineto{\pgfqpoint{1.016786in}{0.402365in}}%
\pgfpathlineto{\pgfqpoint{1.018075in}{0.407554in}}%
\pgfpathlineto{\pgfqpoint{1.021943in}{0.407554in}}%
\pgfpathlineto{\pgfqpoint{1.023232in}{0.409766in}}%
\pgfpathlineto{\pgfqpoint{1.024521in}{0.409918in}}%
\pgfpathlineto{\pgfqpoint{1.025810in}{0.421055in}}%
\pgfpathlineto{\pgfqpoint{1.027100in}{0.414909in}}%
\pgfpathlineto{\pgfqpoint{1.030967in}{0.402425in}}%
\pgfpathlineto{\pgfqpoint{1.032256in}{0.403019in}}%
\pgfpathlineto{\pgfqpoint{1.033546in}{0.433494in}}%
\pgfpathlineto{\pgfqpoint{1.034835in}{0.513351in}}%
\pgfpathlineto{\pgfqpoint{1.036124in}{0.403996in}}%
\pgfpathlineto{\pgfqpoint{1.039992in}{0.401623in}}%
\pgfpathlineto{\pgfqpoint{1.041281in}{0.406594in}}%
\pgfpathlineto{\pgfqpoint{1.043859in}{0.442514in}}%
\pgfpathlineto{\pgfqpoint{1.045148in}{0.478349in}}%
\pgfpathlineto{\pgfqpoint{1.050305in}{0.442005in}}%
\pgfpathlineto{\pgfqpoint{1.051594in}{0.400427in}}%
\pgfpathlineto{\pgfqpoint{1.054173in}{0.604120in}}%
\pgfpathlineto{\pgfqpoint{1.058040in}{0.632405in}}%
\pgfpathlineto{\pgfqpoint{1.059330in}{0.401812in}}%
\pgfpathlineto{\pgfqpoint{1.060619in}{0.406452in}}%
\pgfpathlineto{\pgfqpoint{1.061908in}{0.408675in}}%
\pgfpathlineto{\pgfqpoint{1.063197in}{0.400314in}}%
\pgfpathlineto{\pgfqpoint{1.068354in}{0.401337in}}%
\pgfpathlineto{\pgfqpoint{1.069643in}{0.414179in}}%
\pgfpathlineto{\pgfqpoint{1.070932in}{0.400445in}}%
\pgfpathlineto{\pgfqpoint{1.072222in}{0.400965in}}%
\pgfpathlineto{\pgfqpoint{1.076089in}{0.430045in}}%
\pgfpathlineto{\pgfqpoint{1.077378in}{0.420196in}}%
\pgfpathlineto{\pgfqpoint{1.078668in}{0.403955in}}%
\pgfpathlineto{\pgfqpoint{1.079957in}{0.454665in}}%
\pgfpathlineto{\pgfqpoint{1.081246in}{0.454665in}}%
\pgfpathlineto{\pgfqpoint{1.085113in}{0.403155in}}%
\pgfpathlineto{\pgfqpoint{1.086403in}{0.402461in}}%
\pgfpathlineto{\pgfqpoint{1.087692in}{0.426253in}}%
\pgfpathlineto{\pgfqpoint{1.088981in}{0.401922in}}%
\pgfpathlineto{\pgfqpoint{1.090270in}{0.413444in}}%
\pgfpathlineto{\pgfqpoint{1.094138in}{0.409190in}}%
\pgfpathlineto{\pgfqpoint{1.095427in}{0.513414in}}%
\pgfpathlineto{\pgfqpoint{1.096716in}{0.406168in}}%
\pgfpathlineto{\pgfqpoint{1.098005in}{0.402684in}}%
\pgfpathlineto{\pgfqpoint{1.099295in}{0.413043in}}%
\pgfpathlineto{\pgfqpoint{1.103162in}{0.420182in}}%
\pgfpathlineto{\pgfqpoint{1.104451in}{0.519876in}}%
\pgfpathlineto{\pgfqpoint{1.105741in}{0.496877in}}%
\pgfpathlineto{\pgfqpoint{1.107030in}{0.405192in}}%
\pgfpathlineto{\pgfqpoint{1.108319in}{0.451894in}}%
\pgfpathlineto{\pgfqpoint{1.113476in}{0.405283in}}%
\pgfpathlineto{\pgfqpoint{1.114765in}{0.405283in}}%
\pgfpathlineto{\pgfqpoint{1.116054in}{0.416436in}}%
\pgfpathlineto{\pgfqpoint{1.117343in}{0.421789in}}%
\pgfpathlineto{\pgfqpoint{1.123789in}{0.433349in}}%
\pgfpathlineto{\pgfqpoint{1.125079in}{0.525765in}}%
\pgfpathlineto{\pgfqpoint{1.126368in}{0.412381in}}%
\pgfpathlineto{\pgfqpoint{1.130235in}{0.400271in}}%
\pgfpathlineto{\pgfqpoint{1.131525in}{0.404157in}}%
\pgfpathlineto{\pgfqpoint{1.132814in}{0.618397in}}%
\pgfpathlineto{\pgfqpoint{1.134103in}{0.402831in}}%
\pgfpathlineto{\pgfqpoint{1.135392in}{0.400481in}}%
\pgfpathlineto{\pgfqpoint{1.139260in}{0.400745in}}%
\pgfpathlineto{\pgfqpoint{1.140549in}{0.486725in}}%
\pgfpathlineto{\pgfqpoint{1.143127in}{0.402320in}}%
\pgfpathlineto{\pgfqpoint{1.144416in}{0.409779in}}%
\pgfpathlineto{\pgfqpoint{1.148284in}{0.401663in}}%
\pgfpathlineto{\pgfqpoint{1.149573in}{0.614599in}}%
\pgfpathlineto{\pgfqpoint{1.150862in}{0.408950in}}%
\pgfpathlineto{\pgfqpoint{1.153441in}{0.453907in}}%
\pgfpathlineto{\pgfqpoint{1.158598in}{0.401190in}}%
\pgfpathlineto{\pgfqpoint{1.159887in}{0.409873in}}%
\pgfpathlineto{\pgfqpoint{1.161176in}{0.522971in}}%
\pgfpathlineto{\pgfqpoint{1.162465in}{0.400807in}}%
\pgfpathlineto{\pgfqpoint{1.166333in}{0.400509in}}%
\pgfpathlineto{\pgfqpoint{1.167622in}{0.471699in}}%
\pgfpathlineto{\pgfqpoint{1.168911in}{0.406012in}}%
\pgfpathlineto{\pgfqpoint{1.170200in}{0.435824in}}%
\pgfpathlineto{\pgfqpoint{1.171490in}{0.400271in}}%
\pgfpathlineto{\pgfqpoint{1.175357in}{0.402295in}}%
\pgfpathlineto{\pgfqpoint{1.176646in}{0.516408in}}%
\pgfpathlineto{\pgfqpoint{1.177936in}{0.400485in}}%
\pgfpathlineto{\pgfqpoint{1.179225in}{0.412363in}}%
\pgfpathlineto{\pgfqpoint{1.180514in}{0.401137in}}%
\pgfpathlineto{\pgfqpoint{1.184382in}{0.400756in}}%
\pgfpathlineto{\pgfqpoint{1.185671in}{0.454300in}}%
\pgfpathlineto{\pgfqpoint{1.186960in}{0.406496in}}%
\pgfpathlineto{\pgfqpoint{1.188249in}{0.402118in}}%
\pgfpathlineto{\pgfqpoint{1.189538in}{0.423253in}}%
\pgfpathlineto{\pgfqpoint{1.193406in}{0.405579in}}%
\pgfpathlineto{\pgfqpoint{1.195984in}{0.400271in}}%
\pgfpathlineto{\pgfqpoint{1.197274in}{0.406774in}}%
\pgfpathlineto{\pgfqpoint{1.198563in}{0.429285in}}%
\pgfpathlineto{\pgfqpoint{1.202430in}{0.456133in}}%
\pgfpathlineto{\pgfqpoint{1.205009in}{0.598646in}}%
\pgfpathlineto{\pgfqpoint{1.206298in}{0.401519in}}%
\pgfpathlineto{\pgfqpoint{1.207587in}{0.408796in}}%
\pgfpathlineto{\pgfqpoint{1.211455in}{0.402751in}}%
\pgfpathlineto{\pgfqpoint{1.212744in}{0.408855in}}%
\pgfpathlineto{\pgfqpoint{1.214033in}{0.445538in}}%
\pgfpathlineto{\pgfqpoint{1.215322in}{0.438417in}}%
\pgfpathlineto{\pgfqpoint{1.216611in}{0.412422in}}%
\pgfpathlineto{\pgfqpoint{1.220479in}{0.400271in}}%
\pgfpathlineto{\pgfqpoint{1.221768in}{0.404101in}}%
\pgfpathlineto{\pgfqpoint{1.223057in}{0.415512in}}%
\pgfpathlineto{\pgfqpoint{1.224347in}{0.500524in}}%
\pgfpathlineto{\pgfqpoint{1.225636in}{1.046442in}}%
\pgfpathlineto{\pgfqpoint{1.230793in}{0.402092in}}%
\pgfpathlineto{\pgfqpoint{1.232082in}{0.401544in}}%
\pgfpathlineto{\pgfqpoint{1.233371in}{0.408966in}}%
\pgfpathlineto{\pgfqpoint{1.234660in}{0.400323in}}%
\pgfpathlineto{\pgfqpoint{1.238528in}{0.402798in}}%
\pgfpathlineto{\pgfqpoint{1.239817in}{0.418607in}}%
\pgfpathlineto{\pgfqpoint{1.241106in}{0.402723in}}%
\pgfpathlineto{\pgfqpoint{1.242395in}{0.437171in}}%
\pgfpathlineto{\pgfqpoint{1.243685in}{0.434508in}}%
\pgfpathlineto{\pgfqpoint{1.247552in}{0.413162in}}%
\pgfpathlineto{\pgfqpoint{1.248841in}{0.420366in}}%
\pgfpathlineto{\pgfqpoint{1.250131in}{0.413102in}}%
\pgfpathlineto{\pgfqpoint{1.251420in}{0.410267in}}%
\pgfpathlineto{\pgfqpoint{1.252709in}{0.411739in}}%
\pgfpathlineto{\pgfqpoint{1.256577in}{0.400725in}}%
\pgfpathlineto{\pgfqpoint{1.257866in}{0.408717in}}%
\pgfpathlineto{\pgfqpoint{1.259155in}{0.401508in}}%
\pgfpathlineto{\pgfqpoint{1.260444in}{0.400469in}}%
\pgfpathlineto{\pgfqpoint{1.261733in}{0.404296in}}%
\pgfpathlineto{\pgfqpoint{1.266890in}{0.400471in}}%
\pgfpathlineto{\pgfqpoint{1.269468in}{0.485195in}}%
\pgfpathlineto{\pgfqpoint{1.270758in}{0.410836in}}%
\pgfpathlineto{\pgfqpoint{1.274625in}{0.412406in}}%
\pgfpathlineto{\pgfqpoint{1.275914in}{0.445253in}}%
\pgfpathlineto{\pgfqpoint{1.277204in}{0.440818in}}%
\pgfpathlineto{\pgfqpoint{1.278493in}{0.401087in}}%
\pgfpathlineto{\pgfqpoint{1.279782in}{0.408845in}}%
\pgfpathlineto{\pgfqpoint{1.283650in}{0.420182in}}%
\pgfpathlineto{\pgfqpoint{1.284939in}{0.417845in}}%
\pgfpathlineto{\pgfqpoint{1.286228in}{0.419320in}}%
\pgfpathlineto{\pgfqpoint{1.287517in}{0.407007in}}%
\pgfpathlineto{\pgfqpoint{1.288806in}{0.432142in}}%
\pgfpathlineto{\pgfqpoint{1.292674in}{0.404127in}}%
\pgfpathlineto{\pgfqpoint{1.293963in}{0.401031in}}%
\pgfpathlineto{\pgfqpoint{1.296542in}{0.400319in}}%
\pgfpathlineto{\pgfqpoint{1.297831in}{0.423555in}}%
\pgfpathlineto{\pgfqpoint{1.301698in}{0.405170in}}%
\pgfpathlineto{\pgfqpoint{1.302988in}{0.405226in}}%
\pgfpathlineto{\pgfqpoint{1.304277in}{0.400719in}}%
\pgfpathlineto{\pgfqpoint{1.305566in}{0.406319in}}%
\pgfpathlineto{\pgfqpoint{1.306855in}{0.426528in}}%
\pgfpathlineto{\pgfqpoint{1.310723in}{0.409950in}}%
\pgfpathlineto{\pgfqpoint{1.312012in}{0.526050in}}%
\pgfpathlineto{\pgfqpoint{1.313301in}{0.401443in}}%
\pgfpathlineto{\pgfqpoint{1.314590in}{0.400271in}}%
\pgfpathlineto{\pgfqpoint{1.319747in}{0.452105in}}%
\pgfpathlineto{\pgfqpoint{1.321036in}{0.400465in}}%
\pgfpathlineto{\pgfqpoint{1.322325in}{0.457291in}}%
\pgfpathlineto{\pgfqpoint{1.323615in}{0.403475in}}%
\pgfpathlineto{\pgfqpoint{1.324904in}{0.414815in}}%
\pgfpathlineto{\pgfqpoint{1.328771in}{0.408796in}}%
\pgfpathlineto{\pgfqpoint{1.330061in}{0.541844in}}%
\pgfpathlineto{\pgfqpoint{1.331350in}{0.481434in}}%
\pgfpathlineto{\pgfqpoint{1.332639in}{0.456416in}}%
\pgfpathlineto{\pgfqpoint{1.333928in}{0.408225in}}%
\pgfpathlineto{\pgfqpoint{1.337796in}{0.427950in}}%
\pgfpathlineto{\pgfqpoint{1.339085in}{0.492269in}}%
\pgfpathlineto{\pgfqpoint{1.340374in}{0.400318in}}%
\pgfpathlineto{\pgfqpoint{1.341663in}{0.428821in}}%
\pgfpathlineto{\pgfqpoint{1.342953in}{0.413173in}}%
\pgfpathlineto{\pgfqpoint{1.348109in}{0.470287in}}%
\pgfpathlineto{\pgfqpoint{1.349399in}{0.421582in}}%
\pgfpathlineto{\pgfqpoint{1.350688in}{0.421582in}}%
\pgfpathlineto{\pgfqpoint{1.351977in}{0.400434in}}%
\pgfpathlineto{\pgfqpoint{1.355845in}{0.436323in}}%
\pgfpathlineto{\pgfqpoint{1.357134in}{0.409079in}}%
\pgfpathlineto{\pgfqpoint{1.358423in}{0.400892in}}%
\pgfpathlineto{\pgfqpoint{1.359712in}{0.404121in}}%
\pgfpathlineto{\pgfqpoint{1.361001in}{0.401646in}}%
\pgfpathlineto{\pgfqpoint{1.364869in}{0.400882in}}%
\pgfpathlineto{\pgfqpoint{1.366158in}{0.415409in}}%
\pgfpathlineto{\pgfqpoint{1.367447in}{0.402660in}}%
\pgfpathlineto{\pgfqpoint{1.368737in}{0.403269in}}%
\pgfpathlineto{\pgfqpoint{1.370026in}{0.405545in}}%
\pgfpathlineto{\pgfqpoint{1.373893in}{0.445650in}}%
\pgfpathlineto{\pgfqpoint{1.375183in}{0.415500in}}%
\pgfpathlineto{\pgfqpoint{1.376472in}{0.434366in}}%
\pgfpathlineto{\pgfqpoint{1.379050in}{0.402703in}}%
\pgfpathlineto{\pgfqpoint{1.382918in}{0.400876in}}%
\pgfpathlineto{\pgfqpoint{1.385496in}{0.402114in}}%
\pgfpathlineto{\pgfqpoint{1.386785in}{0.400422in}}%
\pgfpathlineto{\pgfqpoint{1.388074in}{0.404871in}}%
\pgfpathlineto{\pgfqpoint{1.393231in}{0.407711in}}%
\pgfpathlineto{\pgfqpoint{1.394520in}{0.409844in}}%
\pgfpathlineto{\pgfqpoint{1.395810in}{0.401201in}}%
\pgfpathlineto{\pgfqpoint{1.397099in}{0.401201in}}%
\pgfpathlineto{\pgfqpoint{1.400966in}{0.628616in}}%
\pgfpathlineto{\pgfqpoint{1.402256in}{0.403687in}}%
\pgfpathlineto{\pgfqpoint{1.403545in}{0.505865in}}%
\pgfpathlineto{\pgfqpoint{1.404834in}{0.400846in}}%
\pgfpathlineto{\pgfqpoint{1.406123in}{0.401173in}}%
\pgfpathlineto{\pgfqpoint{1.409991in}{0.443821in}}%
\pgfpathlineto{\pgfqpoint{1.411280in}{0.422383in}}%
\pgfpathlineto{\pgfqpoint{1.412569in}{0.416226in}}%
\pgfpathlineto{\pgfqpoint{1.413858in}{0.469535in}}%
\pgfpathlineto{\pgfqpoint{1.415148in}{0.401149in}}%
\pgfpathlineto{\pgfqpoint{1.419015in}{0.408132in}}%
\pgfpathlineto{\pgfqpoint{1.421594in}{0.452505in}}%
\pgfpathlineto{\pgfqpoint{1.422883in}{0.562543in}}%
\pgfpathlineto{\pgfqpoint{1.424172in}{0.400309in}}%
\pgfpathlineto{\pgfqpoint{1.428040in}{0.503851in}}%
\pgfpathlineto{\pgfqpoint{1.429329in}{0.424539in}}%
\pgfpathlineto{\pgfqpoint{1.430618in}{0.404885in}}%
\pgfpathlineto{\pgfqpoint{1.431907in}{0.400612in}}%
\pgfpathlineto{\pgfqpoint{1.437064in}{0.429882in}}%
\pgfpathlineto{\pgfqpoint{1.438353in}{0.407832in}}%
\pgfpathlineto{\pgfqpoint{1.439642in}{0.400620in}}%
\pgfpathlineto{\pgfqpoint{1.442221in}{0.424175in}}%
\pgfpathlineto{\pgfqpoint{1.446088in}{0.603344in}}%
\pgfpathlineto{\pgfqpoint{1.447377in}{0.400923in}}%
\pgfpathlineto{\pgfqpoint{1.448667in}{0.403561in}}%
\pgfpathlineto{\pgfqpoint{1.449956in}{0.546099in}}%
\pgfpathlineto{\pgfqpoint{1.451245in}{0.402140in}}%
\pgfpathlineto{\pgfqpoint{1.455113in}{0.400615in}}%
\pgfpathlineto{\pgfqpoint{1.456402in}{0.425727in}}%
\pgfpathlineto{\pgfqpoint{1.457691in}{0.402660in}}%
\pgfpathlineto{\pgfqpoint{1.458980in}{0.625165in}}%
\pgfpathlineto{\pgfqpoint{1.460269in}{0.412071in}}%
\pgfpathlineto{\pgfqpoint{1.464137in}{0.420440in}}%
\pgfpathlineto{\pgfqpoint{1.465426in}{0.400440in}}%
\pgfpathlineto{\pgfqpoint{1.466715in}{0.409702in}}%
\pgfpathlineto{\pgfqpoint{1.468005in}{0.405274in}}%
\pgfpathlineto{\pgfqpoint{1.469294in}{0.412047in}}%
\pgfpathlineto{\pgfqpoint{1.473161in}{0.400433in}}%
\pgfpathlineto{\pgfqpoint{1.474451in}{0.406063in}}%
\pgfpathlineto{\pgfqpoint{1.475740in}{0.400912in}}%
\pgfpathlineto{\pgfqpoint{1.477029in}{0.405156in}}%
\pgfpathlineto{\pgfqpoint{1.478318in}{0.400312in}}%
\pgfpathlineto{\pgfqpoint{1.482186in}{0.401736in}}%
\pgfpathlineto{\pgfqpoint{1.483475in}{0.404374in}}%
\pgfpathlineto{\pgfqpoint{1.484764in}{0.403627in}}%
\pgfpathlineto{\pgfqpoint{1.486053in}{0.418833in}}%
\pgfpathlineto{\pgfqpoint{1.487343in}{0.401333in}}%
\pgfpathlineto{\pgfqpoint{1.491210in}{0.405364in}}%
\pgfpathlineto{\pgfqpoint{1.492499in}{0.404478in}}%
\pgfpathlineto{\pgfqpoint{1.493789in}{0.401326in}}%
\pgfpathlineto{\pgfqpoint{1.495078in}{0.491493in}}%
\pgfpathlineto{\pgfqpoint{1.496367in}{0.404714in}}%
\pgfpathlineto{\pgfqpoint{1.500235in}{0.442483in}}%
\pgfpathlineto{\pgfqpoint{1.501524in}{0.417368in}}%
\pgfpathlineto{\pgfqpoint{1.502813in}{0.436390in}}%
\pgfpathlineto{\pgfqpoint{1.504102in}{0.402401in}}%
\pgfpathlineto{\pgfqpoint{1.505391in}{0.411260in}}%
\pgfpathlineto{\pgfqpoint{1.509259in}{0.409924in}}%
\pgfpathlineto{\pgfqpoint{1.510548in}{0.401833in}}%
\pgfpathlineto{\pgfqpoint{1.511837in}{0.403045in}}%
\pgfpathlineto{\pgfqpoint{1.513126in}{0.416009in}}%
\pgfpathlineto{\pgfqpoint{1.514416in}{0.402438in}}%
\pgfpathlineto{\pgfqpoint{1.519572in}{0.403099in}}%
\pgfpathlineto{\pgfqpoint{1.520862in}{0.495047in}}%
\pgfpathlineto{\pgfqpoint{1.522151in}{0.400649in}}%
\pgfpathlineto{\pgfqpoint{1.523440in}{0.401778in}}%
\pgfpathlineto{\pgfqpoint{1.527308in}{0.416783in}}%
\pgfpathlineto{\pgfqpoint{1.528597in}{0.401737in}}%
\pgfpathlineto{\pgfqpoint{1.529886in}{0.408287in}}%
\pgfpathlineto{\pgfqpoint{1.531175in}{0.409613in}}%
\pgfpathlineto{\pgfqpoint{1.532464in}{0.587064in}}%
\pgfpathlineto{\pgfqpoint{1.536332in}{0.401249in}}%
\pgfpathlineto{\pgfqpoint{1.537621in}{0.435024in}}%
\pgfpathlineto{\pgfqpoint{1.538910in}{0.406657in}}%
\pgfpathlineto{\pgfqpoint{1.540200in}{0.400309in}}%
\pgfpathlineto{\pgfqpoint{1.541489in}{0.405705in}}%
\pgfpathlineto{\pgfqpoint{1.545356in}{0.405770in}}%
\pgfpathlineto{\pgfqpoint{1.546646in}{0.401650in}}%
\pgfpathlineto{\pgfqpoint{1.547935in}{0.400271in}}%
\pgfpathlineto{\pgfqpoint{1.550513in}{0.451921in}}%
\pgfpathlineto{\pgfqpoint{1.554381in}{0.401289in}}%
\pgfpathlineto{\pgfqpoint{1.555670in}{0.402894in}}%
\pgfpathlineto{\pgfqpoint{1.556959in}{0.401297in}}%
\pgfpathlineto{\pgfqpoint{1.558248in}{0.424148in}}%
\pgfpathlineto{\pgfqpoint{1.559538in}{0.413744in}}%
\pgfpathlineto{\pgfqpoint{1.563405in}{0.400313in}}%
\pgfpathlineto{\pgfqpoint{1.564694in}{0.435409in}}%
\pgfpathlineto{\pgfqpoint{1.565983in}{0.403691in}}%
\pgfpathlineto{\pgfqpoint{1.567273in}{0.476245in}}%
\pgfpathlineto{\pgfqpoint{1.568562in}{0.407017in}}%
\pgfpathlineto{\pgfqpoint{1.572429in}{0.410338in}}%
\pgfpathlineto{\pgfqpoint{1.573719in}{0.401249in}}%
\pgfpathlineto{\pgfqpoint{1.575008in}{0.426427in}}%
\pgfpathlineto{\pgfqpoint{1.577586in}{0.400609in}}%
\pgfpathlineto{\pgfqpoint{1.581454in}{0.416672in}}%
\pgfpathlineto{\pgfqpoint{1.582743in}{0.401600in}}%
\pgfpathlineto{\pgfqpoint{1.584032in}{0.427627in}}%
\pgfpathlineto{\pgfqpoint{1.585321in}{0.400613in}}%
\pgfpathlineto{\pgfqpoint{1.586611in}{0.456840in}}%
\pgfpathlineto{\pgfqpoint{1.590478in}{0.403902in}}%
\pgfpathlineto{\pgfqpoint{1.591767in}{0.406338in}}%
\pgfpathlineto{\pgfqpoint{1.593057in}{0.400307in}}%
\pgfpathlineto{\pgfqpoint{1.594346in}{0.400593in}}%
\pgfpathlineto{\pgfqpoint{1.595635in}{0.405458in}}%
\pgfpathlineto{\pgfqpoint{1.600792in}{0.400271in}}%
\pgfpathlineto{\pgfqpoint{1.602081in}{0.413306in}}%
\pgfpathlineto{\pgfqpoint{1.603370in}{0.409501in}}%
\pgfpathlineto{\pgfqpoint{1.604659in}{0.400417in}}%
\pgfpathlineto{\pgfqpoint{1.608527in}{0.402042in}}%
\pgfpathlineto{\pgfqpoint{1.609816in}{0.417519in}}%
\pgfpathlineto{\pgfqpoint{1.611105in}{0.407135in}}%
\pgfpathlineto{\pgfqpoint{1.612395in}{0.410407in}}%
\pgfpathlineto{\pgfqpoint{1.613684in}{0.404527in}}%
\pgfpathlineto{\pgfqpoint{1.617551in}{0.401527in}}%
\pgfpathlineto{\pgfqpoint{1.620130in}{0.405266in}}%
\pgfpathlineto{\pgfqpoint{1.621419in}{0.510589in}}%
\pgfpathlineto{\pgfqpoint{1.622708in}{0.870394in}}%
\pgfpathlineto{\pgfqpoint{1.626576in}{0.403985in}}%
\pgfpathlineto{\pgfqpoint{1.627865in}{0.402669in}}%
\pgfpathlineto{\pgfqpoint{1.629154in}{0.416688in}}%
\pgfpathlineto{\pgfqpoint{1.631732in}{0.401195in}}%
\pgfpathlineto{\pgfqpoint{1.636889in}{0.406645in}}%
\pgfpathlineto{\pgfqpoint{1.638178in}{0.409911in}}%
\pgfpathlineto{\pgfqpoint{1.639468in}{0.474259in}}%
\pgfpathlineto{\pgfqpoint{1.640757in}{0.477453in}}%
\pgfpathlineto{\pgfqpoint{1.644624in}{0.403027in}}%
\pgfpathlineto{\pgfqpoint{1.645914in}{0.403027in}}%
\pgfpathlineto{\pgfqpoint{1.647203in}{0.440409in}}%
\pgfpathlineto{\pgfqpoint{1.648492in}{0.400835in}}%
\pgfpathlineto{\pgfqpoint{1.649781in}{0.408239in}}%
\pgfpathlineto{\pgfqpoint{1.653649in}{0.407215in}}%
\pgfpathlineto{\pgfqpoint{1.654938in}{0.410358in}}%
\pgfpathlineto{\pgfqpoint{1.656227in}{0.460135in}}%
\pgfpathlineto{\pgfqpoint{1.657516in}{0.400305in}}%
\pgfpathlineto{\pgfqpoint{1.658806in}{0.401908in}}%
\pgfpathlineto{\pgfqpoint{1.662673in}{0.416329in}}%
\pgfpathlineto{\pgfqpoint{1.663962in}{0.403540in}}%
\pgfpathlineto{\pgfqpoint{1.666541in}{0.417299in}}%
\pgfpathlineto{\pgfqpoint{1.667830in}{0.402862in}}%
\pgfpathlineto{\pgfqpoint{1.671698in}{0.414298in}}%
\pgfpathlineto{\pgfqpoint{1.672987in}{0.403407in}}%
\pgfpathlineto{\pgfqpoint{1.675565in}{0.407343in}}%
\pgfpathlineto{\pgfqpoint{1.676854in}{0.400303in}}%
\pgfpathlineto{\pgfqpoint{1.680722in}{0.423651in}}%
\pgfpathlineto{\pgfqpoint{1.682011in}{0.404181in}}%
\pgfpathlineto{\pgfqpoint{1.683300in}{0.405736in}}%
\pgfpathlineto{\pgfqpoint{1.684589in}{0.403540in}}%
\pgfpathlineto{\pgfqpoint{1.685879in}{0.410822in}}%
\pgfpathlineto{\pgfqpoint{1.689746in}{0.400401in}}%
\pgfpathlineto{\pgfqpoint{1.691035in}{0.625146in}}%
\pgfpathlineto{\pgfqpoint{1.692325in}{0.406105in}}%
\pgfpathlineto{\pgfqpoint{1.693614in}{0.403868in}}%
\pgfpathlineto{\pgfqpoint{1.694903in}{0.507046in}}%
\pgfpathlineto{\pgfqpoint{1.700060in}{0.415673in}}%
\pgfpathlineto{\pgfqpoint{1.701349in}{0.417457in}}%
\pgfpathlineto{\pgfqpoint{1.702638in}{0.408741in}}%
\pgfpathlineto{\pgfqpoint{1.703927in}{0.424882in}}%
\pgfpathlineto{\pgfqpoint{1.707795in}{0.401952in}}%
\pgfpathlineto{\pgfqpoint{1.710373in}{0.412633in}}%
\pgfpathlineto{\pgfqpoint{1.711663in}{0.401133in}}%
\pgfpathlineto{\pgfqpoint{1.712952in}{0.410291in}}%
\pgfpathlineto{\pgfqpoint{1.716819in}{0.489802in}}%
\pgfpathlineto{\pgfqpoint{1.718109in}{0.404736in}}%
\pgfpathlineto{\pgfqpoint{1.719398in}{0.401611in}}%
\pgfpathlineto{\pgfqpoint{1.720687in}{0.458816in}}%
\pgfpathlineto{\pgfqpoint{1.721976in}{0.413105in}}%
\pgfpathlineto{\pgfqpoint{1.725844in}{0.401992in}}%
\pgfpathlineto{\pgfqpoint{1.727133in}{0.408087in}}%
\pgfpathlineto{\pgfqpoint{1.728422in}{0.401956in}}%
\pgfpathlineto{\pgfqpoint{1.729711in}{0.406025in}}%
\pgfpathlineto{\pgfqpoint{1.731001in}{0.401115in}}%
\pgfpathlineto{\pgfqpoint{1.736157in}{0.400271in}}%
\pgfpathlineto{\pgfqpoint{1.737447in}{0.416743in}}%
\pgfpathlineto{\pgfqpoint{1.738736in}{0.413897in}}%
\pgfpathlineto{\pgfqpoint{1.740025in}{0.425187in}}%
\pgfpathlineto{\pgfqpoint{1.743892in}{0.411392in}}%
\pgfpathlineto{\pgfqpoint{1.745182in}{0.400305in}}%
\pgfpathlineto{\pgfqpoint{1.746471in}{0.407881in}}%
\pgfpathlineto{\pgfqpoint{1.747760in}{0.400574in}}%
\pgfpathlineto{\pgfqpoint{1.749049in}{0.400271in}}%
\pgfpathlineto{\pgfqpoint{1.752917in}{0.416743in}}%
\pgfpathlineto{\pgfqpoint{1.754206in}{0.403045in}}%
\pgfpathlineto{\pgfqpoint{1.755495in}{0.403045in}}%
\pgfpathlineto{\pgfqpoint{1.756784in}{0.404411in}}%
\pgfpathlineto{\pgfqpoint{1.758074in}{0.400305in}}%
\pgfpathlineto{\pgfqpoint{1.761941in}{0.410018in}}%
\pgfpathlineto{\pgfqpoint{1.763230in}{0.403633in}}%
\pgfpathlineto{\pgfqpoint{1.764520in}{0.400575in}}%
\pgfpathlineto{\pgfqpoint{1.765809in}{0.408959in}}%
\pgfpathlineto{\pgfqpoint{1.767098in}{0.401506in}}%
\pgfpathlineto{\pgfqpoint{1.770966in}{0.410130in}}%
\pgfpathlineto{\pgfqpoint{1.772255in}{0.403639in}}%
\pgfpathlineto{\pgfqpoint{1.773544in}{0.409882in}}%
\pgfpathlineto{\pgfqpoint{1.774833in}{0.440033in}}%
\pgfpathlineto{\pgfqpoint{1.776122in}{0.415887in}}%
\pgfpathlineto{\pgfqpoint{1.779990in}{0.400794in}}%
\pgfpathlineto{\pgfqpoint{1.781279in}{0.427418in}}%
\pgfpathlineto{\pgfqpoint{1.782568in}{0.401837in}}%
\pgfpathlineto{\pgfqpoint{1.783858in}{0.401429in}}%
\pgfpathlineto{\pgfqpoint{1.785147in}{0.423479in}}%
\pgfpathlineto{\pgfqpoint{1.789014in}{0.408263in}}%
\pgfpathlineto{\pgfqpoint{1.790304in}{0.407197in}}%
\pgfpathlineto{\pgfqpoint{1.791593in}{0.402235in}}%
\pgfpathlineto{\pgfqpoint{1.792882in}{0.458727in}}%
\pgfpathlineto{\pgfqpoint{1.794171in}{0.413451in}}%
\pgfpathlineto{\pgfqpoint{1.798039in}{0.423629in}}%
\pgfpathlineto{\pgfqpoint{1.799328in}{0.437809in}}%
\pgfpathlineto{\pgfqpoint{1.800617in}{0.401295in}}%
\pgfpathlineto{\pgfqpoint{1.801906in}{0.464227in}}%
\pgfpathlineto{\pgfqpoint{1.803195in}{0.414716in}}%
\pgfpathlineto{\pgfqpoint{1.807063in}{0.432615in}}%
\pgfpathlineto{\pgfqpoint{1.808352in}{0.409680in}}%
\pgfpathlineto{\pgfqpoint{1.810931in}{0.402568in}}%
\pgfpathlineto{\pgfqpoint{1.816087in}{0.402085in}}%
\pgfpathlineto{\pgfqpoint{1.817377in}{0.402584in}}%
\pgfpathlineto{\pgfqpoint{1.818666in}{0.402112in}}%
\pgfpathlineto{\pgfqpoint{1.819955in}{0.400271in}}%
\pgfpathlineto{\pgfqpoint{1.821244in}{0.452134in}}%
\pgfpathlineto{\pgfqpoint{1.825112in}{0.401347in}}%
\pgfpathlineto{\pgfqpoint{1.828979in}{0.411969in}}%
\pgfpathlineto{\pgfqpoint{1.830269in}{0.400744in}}%
\pgfpathlineto{\pgfqpoint{1.835425in}{0.400390in}}%
\pgfpathlineto{\pgfqpoint{1.836715in}{0.406883in}}%
\pgfpathlineto{\pgfqpoint{1.838004in}{0.400534in}}%
\pgfpathlineto{\pgfqpoint{1.839293in}{0.400535in}}%
\pgfpathlineto{\pgfqpoint{1.844450in}{0.401697in}}%
\pgfpathlineto{\pgfqpoint{1.845739in}{0.403185in}}%
\pgfpathlineto{\pgfqpoint{1.847028in}{0.423505in}}%
\pgfpathlineto{\pgfqpoint{1.848317in}{0.408008in}}%
\pgfpathlineto{\pgfqpoint{1.852185in}{0.411166in}}%
\pgfpathlineto{\pgfqpoint{1.853474in}{0.400271in}}%
\pgfpathlineto{\pgfqpoint{1.854763in}{0.405299in}}%
\pgfpathlineto{\pgfqpoint{1.856053in}{0.400745in}}%
\pgfpathlineto{\pgfqpoint{1.857342in}{0.404524in}}%
\pgfpathlineto{\pgfqpoint{1.862499in}{0.437743in}}%
\pgfpathlineto{\pgfqpoint{1.863788in}{0.406634in}}%
\pgfpathlineto{\pgfqpoint{1.865077in}{0.401644in}}%
\pgfpathlineto{\pgfqpoint{1.866366in}{0.426756in}}%
\pgfpathlineto{\pgfqpoint{1.870234in}{0.400953in}}%
\pgfpathlineto{\pgfqpoint{1.871523in}{0.431395in}}%
\pgfpathlineto{\pgfqpoint{1.872812in}{0.400936in}}%
\pgfpathlineto{\pgfqpoint{1.874101in}{0.400936in}}%
\pgfpathlineto{\pgfqpoint{1.875390in}{0.450722in}}%
\pgfpathlineto{\pgfqpoint{1.879258in}{0.412769in}}%
\pgfpathlineto{\pgfqpoint{1.880547in}{0.420466in}}%
\pgfpathlineto{\pgfqpoint{1.881836in}{0.417670in}}%
\pgfpathlineto{\pgfqpoint{1.883126in}{0.400375in}}%
\pgfpathlineto{\pgfqpoint{1.884415in}{1.061854in}}%
\pgfpathlineto{\pgfqpoint{1.888282in}{0.403533in}}%
\pgfpathlineto{\pgfqpoint{1.889572in}{0.400836in}}%
\pgfpathlineto{\pgfqpoint{1.890861in}{0.400362in}}%
\pgfpathlineto{\pgfqpoint{1.892150in}{0.404690in}}%
\pgfpathlineto{\pgfqpoint{1.893439in}{0.402505in}}%
\pgfpathlineto{\pgfqpoint{1.897307in}{0.400360in}}%
\pgfpathlineto{\pgfqpoint{1.899885in}{0.422339in}}%
\pgfpathlineto{\pgfqpoint{1.901174in}{0.401638in}}%
\pgfpathlineto{\pgfqpoint{1.902464in}{0.403874in}}%
\pgfpathlineto{\pgfqpoint{1.906331in}{0.400462in}}%
\pgfpathlineto{\pgfqpoint{1.907620in}{0.401310in}}%
\pgfpathlineto{\pgfqpoint{1.908910in}{0.400271in}}%
\pgfpathlineto{\pgfqpoint{1.910199in}{0.404395in}}%
\pgfpathlineto{\pgfqpoint{1.915356in}{0.402372in}}%
\pgfpathlineto{\pgfqpoint{1.916645in}{0.408769in}}%
\pgfpathlineto{\pgfqpoint{1.917934in}{0.401637in}}%
\pgfpathlineto{\pgfqpoint{1.919223in}{0.421800in}}%
\pgfpathlineto{\pgfqpoint{1.920512in}{0.400292in}}%
\pgfpathlineto{\pgfqpoint{1.924380in}{0.409369in}}%
\pgfpathlineto{\pgfqpoint{1.925669in}{0.407610in}}%
\pgfpathlineto{\pgfqpoint{1.926958in}{1.591255in}}%
\pgfpathlineto{\pgfqpoint{1.928247in}{0.517128in}}%
\pgfpathlineto{\pgfqpoint{1.929537in}{0.400271in}}%
\pgfpathlineto{\pgfqpoint{1.933404in}{0.417263in}}%
\pgfpathlineto{\pgfqpoint{1.934693in}{0.468251in}}%
\pgfpathlineto{\pgfqpoint{1.935983in}{0.410840in}}%
\pgfpathlineto{\pgfqpoint{1.937272in}{0.408586in}}%
\pgfpathlineto{\pgfqpoint{1.938561in}{0.400271in}}%
\pgfpathlineto{\pgfqpoint{1.943718in}{0.400549in}}%
\pgfpathlineto{\pgfqpoint{1.945007in}{0.403659in}}%
\pgfpathlineto{\pgfqpoint{1.946296in}{0.428170in}}%
\pgfpathlineto{\pgfqpoint{1.947585in}{0.403272in}}%
\pgfpathlineto{\pgfqpoint{1.951453in}{0.413152in}}%
\pgfpathlineto{\pgfqpoint{1.952742in}{0.536815in}}%
\pgfpathlineto{\pgfqpoint{1.954031in}{0.400346in}}%
\pgfpathlineto{\pgfqpoint{1.955321in}{0.404471in}}%
\pgfpathlineto{\pgfqpoint{1.956610in}{0.400948in}}%
\pgfpathlineto{\pgfqpoint{1.960477in}{0.424384in}}%
\pgfpathlineto{\pgfqpoint{1.961767in}{0.402102in}}%
\pgfpathlineto{\pgfqpoint{1.963056in}{0.525346in}}%
\pgfpathlineto{\pgfqpoint{1.964345in}{0.403663in}}%
\pgfpathlineto{\pgfqpoint{1.965634in}{0.407180in}}%
\pgfpathlineto{\pgfqpoint{1.969502in}{0.407784in}}%
\pgfpathlineto{\pgfqpoint{1.970791in}{0.400882in}}%
\pgfpathlineto{\pgfqpoint{1.972080in}{0.403585in}}%
\pgfpathlineto{\pgfqpoint{1.973369in}{0.452065in}}%
\pgfpathlineto{\pgfqpoint{1.974659in}{0.405556in}}%
\pgfpathlineto{\pgfqpoint{1.978526in}{0.402264in}}%
\pgfpathlineto{\pgfqpoint{1.979815in}{0.400288in}}%
\pgfpathlineto{\pgfqpoint{1.981105in}{0.400288in}}%
\pgfpathlineto{\pgfqpoint{1.982394in}{0.402664in}}%
\pgfpathlineto{\pgfqpoint{1.983683in}{0.409028in}}%
\pgfpathlineto{\pgfqpoint{1.988840in}{0.415204in}}%
\pgfpathlineto{\pgfqpoint{1.990129in}{0.400271in}}%
\pgfpathlineto{\pgfqpoint{1.991418in}{0.414230in}}%
\pgfpathlineto{\pgfqpoint{1.992707in}{0.401321in}}%
\pgfpathlineto{\pgfqpoint{1.996575in}{0.413894in}}%
\pgfpathlineto{\pgfqpoint{1.997864in}{0.422505in}}%
\pgfpathlineto{\pgfqpoint{1.999153in}{0.401910in}}%
\pgfpathlineto{\pgfqpoint{2.000442in}{0.400271in}}%
\pgfpathlineto{\pgfqpoint{2.001732in}{0.420509in}}%
\pgfpathlineto{\pgfqpoint{2.005599in}{0.401092in}}%
\pgfpathlineto{\pgfqpoint{2.006888in}{0.419464in}}%
\pgfpathlineto{\pgfqpoint{2.008178in}{0.400534in}}%
\pgfpathlineto{\pgfqpoint{2.009467in}{0.404977in}}%
\pgfpathlineto{\pgfqpoint{2.010756in}{0.415991in}}%
\pgfpathlineto{\pgfqpoint{2.014624in}{0.401614in}}%
\pgfpathlineto{\pgfqpoint{2.015913in}{0.410749in}}%
\pgfpathlineto{\pgfqpoint{2.017202in}{0.413401in}}%
\pgfpathlineto{\pgfqpoint{2.018491in}{0.447709in}}%
\pgfpathlineto{\pgfqpoint{2.019780in}{0.402344in}}%
\pgfpathlineto{\pgfqpoint{2.023648in}{0.450949in}}%
\pgfpathlineto{\pgfqpoint{2.024937in}{0.401079in}}%
\pgfpathlineto{\pgfqpoint{2.026226in}{0.488026in}}%
\pgfpathlineto{\pgfqpoint{2.027516in}{0.429832in}}%
\pgfpathlineto{\pgfqpoint{2.028805in}{0.435834in}}%
\pgfpathlineto{\pgfqpoint{2.032672in}{0.401117in}}%
\pgfpathlineto{\pgfqpoint{2.033962in}{0.455289in}}%
\pgfpathlineto{\pgfqpoint{2.035251in}{0.485831in}}%
\pgfpathlineto{\pgfqpoint{2.036540in}{0.456751in}}%
\pgfpathlineto{\pgfqpoint{2.037829in}{0.809117in}}%
\pgfpathlineto{\pgfqpoint{2.041697in}{0.430609in}}%
\pgfpathlineto{\pgfqpoint{2.044275in}{0.512316in}}%
\pgfpathlineto{\pgfqpoint{2.045564in}{0.430249in}}%
\pgfpathlineto{\pgfqpoint{2.046853in}{0.445978in}}%
\pgfpathlineto{\pgfqpoint{2.050721in}{0.406112in}}%
\pgfpathlineto{\pgfqpoint{2.052010in}{0.554345in}}%
\pgfpathlineto{\pgfqpoint{2.053299in}{0.416281in}}%
\pgfpathlineto{\pgfqpoint{2.055878in}{0.434421in}}%
\pgfpathlineto{\pgfqpoint{2.059745in}{0.400344in}}%
\pgfpathlineto{\pgfqpoint{2.061035in}{0.439762in}}%
\pgfpathlineto{\pgfqpoint{2.062324in}{0.404209in}}%
\pgfpathlineto{\pgfqpoint{2.063613in}{0.644138in}}%
\pgfpathlineto{\pgfqpoint{2.064902in}{0.678175in}}%
\pgfpathlineto{\pgfqpoint{2.068770in}{0.411876in}}%
\pgfpathlineto{\pgfqpoint{2.070059in}{0.407033in}}%
\pgfpathlineto{\pgfqpoint{2.071348in}{0.439586in}}%
\pgfpathlineto{\pgfqpoint{2.072637in}{0.400704in}}%
\pgfpathlineto{\pgfqpoint{2.073927in}{0.407948in}}%
\pgfpathlineto{\pgfqpoint{2.077794in}{0.414148in}}%
\pgfpathlineto{\pgfqpoint{2.079083in}{0.400557in}}%
\pgfpathlineto{\pgfqpoint{2.080373in}{0.400912in}}%
\pgfpathlineto{\pgfqpoint{2.081662in}{0.413091in}}%
\pgfpathlineto{\pgfqpoint{2.082951in}{0.409417in}}%
\pgfpathlineto{\pgfqpoint{2.086819in}{0.411767in}}%
\pgfpathlineto{\pgfqpoint{2.088108in}{0.428228in}}%
\pgfpathlineto{\pgfqpoint{2.089397in}{0.417250in}}%
\pgfpathlineto{\pgfqpoint{2.090686in}{0.714043in}}%
\pgfpathlineto{\pgfqpoint{2.091975in}{0.415120in}}%
\pgfpathlineto{\pgfqpoint{2.095843in}{0.450058in}}%
\pgfpathlineto{\pgfqpoint{2.097132in}{0.400812in}}%
\pgfpathlineto{\pgfqpoint{2.098421in}{0.438648in}}%
\pgfpathlineto{\pgfqpoint{2.101000in}{0.414094in}}%
\pgfpathlineto{\pgfqpoint{2.104867in}{0.400971in}}%
\pgfpathlineto{\pgfqpoint{2.106156in}{0.419638in}}%
\pgfpathlineto{\pgfqpoint{2.107446in}{0.403025in}}%
\pgfpathlineto{\pgfqpoint{2.108735in}{0.400328in}}%
\pgfpathlineto{\pgfqpoint{2.110024in}{0.405329in}}%
\pgfpathlineto{\pgfqpoint{2.113892in}{0.425158in}}%
\pgfpathlineto{\pgfqpoint{2.115181in}{0.410776in}}%
\pgfpathlineto{\pgfqpoint{2.116470in}{0.425012in}}%
\pgfpathlineto{\pgfqpoint{2.117759in}{0.409516in}}%
\pgfpathlineto{\pgfqpoint{2.119048in}{0.425260in}}%
\pgfpathlineto{\pgfqpoint{2.122916in}{0.412182in}}%
\pgfpathlineto{\pgfqpoint{2.124205in}{0.415148in}}%
\pgfpathlineto{\pgfqpoint{2.125494in}{0.453681in}}%
\pgfpathlineto{\pgfqpoint{2.126784in}{0.470433in}}%
\pgfpathlineto{\pgfqpoint{2.128073in}{0.447971in}}%
\pgfpathlineto{\pgfqpoint{2.131940in}{0.478122in}}%
\pgfpathlineto{\pgfqpoint{2.133230in}{0.405949in}}%
\pgfpathlineto{\pgfqpoint{2.134519in}{0.400271in}}%
\pgfpathlineto{\pgfqpoint{2.137097in}{0.401678in}}%
\pgfpathlineto{\pgfqpoint{2.143543in}{0.425182in}}%
\pgfpathlineto{\pgfqpoint{2.146122in}{0.400811in}}%
\pgfpathlineto{\pgfqpoint{2.149989in}{0.419874in}}%
\pgfpathlineto{\pgfqpoint{2.151278in}{0.454542in}}%
\pgfpathlineto{\pgfqpoint{2.152568in}{0.466016in}}%
\pgfpathlineto{\pgfqpoint{2.153857in}{0.452346in}}%
\pgfpathlineto{\pgfqpoint{2.155146in}{0.400799in}}%
\pgfpathlineto{\pgfqpoint{2.159014in}{0.403151in}}%
\pgfpathlineto{\pgfqpoint{2.160303in}{0.401470in}}%
\pgfpathlineto{\pgfqpoint{2.161592in}{0.402790in}}%
\pgfpathlineto{\pgfqpoint{2.162881in}{0.403217in}}%
\pgfpathlineto{\pgfqpoint{2.164170in}{0.408200in}}%
\pgfpathlineto{\pgfqpoint{2.170616in}{0.417635in}}%
\pgfpathlineto{\pgfqpoint{2.173195in}{0.423783in}}%
\pgfpathlineto{\pgfqpoint{2.177062in}{0.447765in}}%
\pgfpathlineto{\pgfqpoint{2.178351in}{0.725824in}}%
\pgfpathlineto{\pgfqpoint{2.179641in}{0.422460in}}%
\pgfpathlineto{\pgfqpoint{2.180930in}{0.424977in}}%
\pgfpathlineto{\pgfqpoint{2.182219in}{0.582205in}}%
\pgfpathlineto{\pgfqpoint{2.187376in}{0.400550in}}%
\pgfpathlineto{\pgfqpoint{2.188665in}{0.403201in}}%
\pgfpathlineto{\pgfqpoint{2.189954in}{0.415649in}}%
\pgfpathlineto{\pgfqpoint{2.191243in}{0.456295in}}%
\pgfpathlineto{\pgfqpoint{2.195111in}{0.418470in}}%
\pgfpathlineto{\pgfqpoint{2.196400in}{0.479109in}}%
\pgfpathlineto{\pgfqpoint{2.197689in}{0.403185in}}%
\pgfpathlineto{\pgfqpoint{2.198979in}{0.444550in}}%
\pgfpathlineto{\pgfqpoint{2.200268in}{0.407600in}}%
\pgfpathlineto{\pgfqpoint{2.205425in}{0.418050in}}%
\pgfpathlineto{\pgfqpoint{2.206714in}{0.427788in}}%
\pgfpathlineto{\pgfqpoint{2.208003in}{0.400870in}}%
\pgfpathlineto{\pgfqpoint{2.209292in}{0.405636in}}%
\pgfpathlineto{\pgfqpoint{2.213160in}{0.454775in}}%
\pgfpathlineto{\pgfqpoint{2.214449in}{0.454775in}}%
\pgfpathlineto{\pgfqpoint{2.217027in}{0.411806in}}%
\pgfpathlineto{\pgfqpoint{2.218317in}{0.421609in}}%
\pgfpathlineto{\pgfqpoint{2.222184in}{0.487021in}}%
\pgfpathlineto{\pgfqpoint{2.223473in}{0.400540in}}%
\pgfpathlineto{\pgfqpoint{2.224762in}{0.400288in}}%
\pgfpathlineto{\pgfqpoint{2.227341in}{0.438724in}}%
\pgfpathlineto{\pgfqpoint{2.231208in}{0.431926in}}%
\pgfpathlineto{\pgfqpoint{2.232498in}{0.552485in}}%
\pgfpathlineto{\pgfqpoint{2.233787in}{0.459909in}}%
\pgfpathlineto{\pgfqpoint{2.235076in}{0.756893in}}%
\pgfpathlineto{\pgfqpoint{2.236365in}{0.402747in}}%
\pgfpathlineto{\pgfqpoint{2.240233in}{0.401580in}}%
\pgfpathlineto{\pgfqpoint{2.242811in}{0.414212in}}%
\pgfpathlineto{\pgfqpoint{2.244100in}{0.403740in}}%
\pgfpathlineto{\pgfqpoint{2.245390in}{0.450889in}}%
\pgfpathlineto{\pgfqpoint{2.249257in}{0.402267in}}%
\pgfpathlineto{\pgfqpoint{2.250546in}{0.426568in}}%
\pgfpathlineto{\pgfqpoint{2.251836in}{0.532430in}}%
\pgfpathlineto{\pgfqpoint{2.253125in}{0.406535in}}%
\pgfpathlineto{\pgfqpoint{2.254414in}{0.978246in}}%
\pgfpathlineto{\pgfqpoint{2.258282in}{0.444873in}}%
\pgfpathlineto{\pgfqpoint{2.259571in}{0.405998in}}%
\pgfpathlineto{\pgfqpoint{2.260860in}{0.432642in}}%
\pgfpathlineto{\pgfqpoint{2.262149in}{0.400271in}}%
\pgfpathlineto{\pgfqpoint{2.267306in}{0.408423in}}%
\pgfpathlineto{\pgfqpoint{2.268595in}{0.408305in}}%
\pgfpathlineto{\pgfqpoint{2.269884in}{0.400590in}}%
\pgfpathlineto{\pgfqpoint{2.271174in}{0.400989in}}%
\pgfpathlineto{\pgfqpoint{2.272463in}{0.472955in}}%
\pgfpathlineto{\pgfqpoint{2.276330in}{0.406503in}}%
\pgfpathlineto{\pgfqpoint{2.277620in}{0.408873in}}%
\pgfpathlineto{\pgfqpoint{2.278909in}{0.662700in}}%
\pgfpathlineto{\pgfqpoint{2.280198in}{0.400434in}}%
\pgfpathlineto{\pgfqpoint{2.281487in}{0.422645in}}%
\pgfpathlineto{\pgfqpoint{2.285355in}{0.409973in}}%
\pgfpathlineto{\pgfqpoint{2.286644in}{0.412682in}}%
\pgfpathlineto{\pgfqpoint{2.287933in}{0.409994in}}%
\pgfpathlineto{\pgfqpoint{2.289222in}{0.416853in}}%
\pgfpathlineto{\pgfqpoint{2.290511in}{0.411086in}}%
\pgfpathlineto{\pgfqpoint{2.294379in}{0.425858in}}%
\pgfpathlineto{\pgfqpoint{2.295668in}{0.438691in}}%
\pgfpathlineto{\pgfqpoint{2.296957in}{0.402855in}}%
\pgfpathlineto{\pgfqpoint{2.298247in}{0.416658in}}%
\pgfpathlineto{\pgfqpoint{2.299536in}{0.506491in}}%
\pgfpathlineto{\pgfqpoint{2.303403in}{0.400289in}}%
\pgfpathlineto{\pgfqpoint{2.304693in}{0.442842in}}%
\pgfpathlineto{\pgfqpoint{2.307271in}{0.400290in}}%
\pgfpathlineto{\pgfqpoint{2.308560in}{0.445695in}}%
\pgfpathlineto{\pgfqpoint{2.312428in}{0.402063in}}%
\pgfpathlineto{\pgfqpoint{2.313717in}{0.427993in}}%
\pgfpathlineto{\pgfqpoint{2.316295in}{0.415285in}}%
\pgfpathlineto{\pgfqpoint{2.317585in}{0.400342in}}%
\pgfpathlineto{\pgfqpoint{2.321452in}{0.424134in}}%
\pgfpathlineto{\pgfqpoint{2.322741in}{0.409448in}}%
\pgfpathlineto{\pgfqpoint{2.326609in}{0.401658in}}%
\pgfpathlineto{\pgfqpoint{2.331766in}{0.416943in}}%
\pgfpathlineto{\pgfqpoint{2.333055in}{0.450467in}}%
\pgfpathlineto{\pgfqpoint{2.334344in}{0.411600in}}%
\pgfpathlineto{\pgfqpoint{2.335633in}{0.426505in}}%
\pgfpathlineto{\pgfqpoint{2.339501in}{0.438148in}}%
\pgfpathlineto{\pgfqpoint{2.340790in}{0.455583in}}%
\pgfpathlineto{\pgfqpoint{2.343369in}{0.425136in}}%
\pgfpathlineto{\pgfqpoint{2.344658in}{0.433206in}}%
\pgfpathlineto{\pgfqpoint{2.348525in}{0.444683in}}%
\pgfpathlineto{\pgfqpoint{2.349814in}{0.400586in}}%
\pgfpathlineto{\pgfqpoint{2.351104in}{0.448564in}}%
\pgfpathlineto{\pgfqpoint{2.352393in}{0.400442in}}%
\pgfpathlineto{\pgfqpoint{2.353682in}{0.442806in}}%
\pgfpathlineto{\pgfqpoint{2.357550in}{0.400975in}}%
\pgfpathlineto{\pgfqpoint{2.361417in}{0.427081in}}%
\pgfpathlineto{\pgfqpoint{2.362706in}{0.414776in}}%
\pgfpathlineto{\pgfqpoint{2.366574in}{0.405639in}}%
\pgfpathlineto{\pgfqpoint{2.367863in}{0.404447in}}%
\pgfpathlineto{\pgfqpoint{2.369152in}{0.405075in}}%
\pgfpathlineto{\pgfqpoint{2.370442in}{0.401194in}}%
\pgfpathlineto{\pgfqpoint{2.371731in}{0.543463in}}%
\pgfpathlineto{\pgfqpoint{2.375598in}{0.464187in}}%
\pgfpathlineto{\pgfqpoint{2.376888in}{0.400458in}}%
\pgfpathlineto{\pgfqpoint{2.378177in}{0.409498in}}%
\pgfpathlineto{\pgfqpoint{2.379466in}{0.422954in}}%
\pgfpathlineto{\pgfqpoint{2.384623in}{0.444560in}}%
\pgfpathlineto{\pgfqpoint{2.385912in}{0.403351in}}%
\pgfpathlineto{\pgfqpoint{2.388490in}{0.456546in}}%
\pgfpathlineto{\pgfqpoint{2.389780in}{0.410310in}}%
\pgfpathlineto{\pgfqpoint{2.393647in}{0.455645in}}%
\pgfpathlineto{\pgfqpoint{2.394936in}{0.401339in}}%
\pgfpathlineto{\pgfqpoint{2.396226in}{0.400468in}}%
\pgfpathlineto{\pgfqpoint{2.397515in}{0.408083in}}%
\pgfpathlineto{\pgfqpoint{2.398804in}{0.431746in}}%
\pgfpathlineto{\pgfqpoint{2.402672in}{0.424661in}}%
\pgfpathlineto{\pgfqpoint{2.403961in}{0.426843in}}%
\pgfpathlineto{\pgfqpoint{2.405250in}{0.402610in}}%
\pgfpathlineto{\pgfqpoint{2.406539in}{0.400295in}}%
\pgfpathlineto{\pgfqpoint{2.407828in}{0.453150in}}%
\pgfpathlineto{\pgfqpoint{2.411696in}{0.415364in}}%
\pgfpathlineto{\pgfqpoint{2.412985in}{0.468518in}}%
\pgfpathlineto{\pgfqpoint{2.414274in}{0.400843in}}%
\pgfpathlineto{\pgfqpoint{2.415563in}{0.402127in}}%
\pgfpathlineto{\pgfqpoint{2.416853in}{0.400478in}}%
\pgfpathlineto{\pgfqpoint{2.420720in}{0.402122in}}%
\pgfpathlineto{\pgfqpoint{2.422009in}{0.401384in}}%
\pgfpathlineto{\pgfqpoint{2.423299in}{0.410179in}}%
\pgfpathlineto{\pgfqpoint{2.424588in}{0.402508in}}%
\pgfpathlineto{\pgfqpoint{2.425877in}{0.403000in}}%
\pgfpathlineto{\pgfqpoint{2.429745in}{0.502305in}}%
\pgfpathlineto{\pgfqpoint{2.431034in}{0.479465in}}%
\pgfpathlineto{\pgfqpoint{2.432323in}{0.441209in}}%
\pgfpathlineto{\pgfqpoint{2.433612in}{0.460348in}}%
\pgfpathlineto{\pgfqpoint{2.434901in}{0.404083in}}%
\pgfpathlineto{\pgfqpoint{2.438769in}{0.400831in}}%
\pgfpathlineto{\pgfqpoint{2.440058in}{0.405330in}}%
\pgfpathlineto{\pgfqpoint{2.441347in}{0.455612in}}%
\pgfpathlineto{\pgfqpoint{2.443926in}{0.595838in}}%
\pgfpathlineto{\pgfqpoint{2.447793in}{0.421522in}}%
\pgfpathlineto{\pgfqpoint{2.449083in}{0.432445in}}%
\pgfpathlineto{\pgfqpoint{2.450372in}{0.838621in}}%
\pgfpathlineto{\pgfqpoint{2.451661in}{0.436327in}}%
\pgfpathlineto{\pgfqpoint{2.452950in}{0.492475in}}%
\pgfpathlineto{\pgfqpoint{2.456818in}{0.403091in}}%
\pgfpathlineto{\pgfqpoint{2.458107in}{0.497742in}}%
\pgfpathlineto{\pgfqpoint{2.459396in}{0.516600in}}%
\pgfpathlineto{\pgfqpoint{2.460685in}{0.440348in}}%
\pgfpathlineto{\pgfqpoint{2.461975in}{0.457247in}}%
\pgfpathlineto{\pgfqpoint{2.467131in}{0.574207in}}%
\pgfpathlineto{\pgfqpoint{2.468420in}{0.411859in}}%
\pgfpathlineto{\pgfqpoint{2.469710in}{0.400360in}}%
\pgfpathlineto{\pgfqpoint{2.470999in}{0.407371in}}%
\pgfpathlineto{\pgfqpoint{2.474866in}{0.401341in}}%
\pgfpathlineto{\pgfqpoint{2.476156in}{0.421052in}}%
\pgfpathlineto{\pgfqpoint{2.477445in}{0.400463in}}%
\pgfpathlineto{\pgfqpoint{2.478734in}{0.400805in}}%
\pgfpathlineto{\pgfqpoint{2.480023in}{0.484511in}}%
\pgfpathlineto{\pgfqpoint{2.483891in}{0.404045in}}%
\pgfpathlineto{\pgfqpoint{2.485180in}{0.444018in}}%
\pgfpathlineto{\pgfqpoint{2.486469in}{0.401097in}}%
\pgfpathlineto{\pgfqpoint{2.489048in}{0.419640in}}%
\pgfpathlineto{\pgfqpoint{2.492915in}{0.400636in}}%
\pgfpathlineto{\pgfqpoint{2.494204in}{0.441811in}}%
\pgfpathlineto{\pgfqpoint{2.495494in}{0.537487in}}%
\pgfpathlineto{\pgfqpoint{2.496783in}{0.403281in}}%
\pgfpathlineto{\pgfqpoint{2.498072in}{0.442078in}}%
\pgfpathlineto{\pgfqpoint{2.501940in}{0.479041in}}%
\pgfpathlineto{\pgfqpoint{2.503229in}{0.442414in}}%
\pgfpathlineto{\pgfqpoint{2.504518in}{0.450090in}}%
\pgfpathlineto{\pgfqpoint{2.505807in}{0.406097in}}%
\pgfpathlineto{\pgfqpoint{2.507096in}{0.421173in}}%
\pgfpathlineto{\pgfqpoint{2.510964in}{0.400929in}}%
\pgfpathlineto{\pgfqpoint{2.512253in}{0.404392in}}%
\pgfpathlineto{\pgfqpoint{2.513542in}{0.483424in}}%
\pgfpathlineto{\pgfqpoint{2.514832in}{0.400553in}}%
\pgfpathlineto{\pgfqpoint{2.516121in}{0.412074in}}%
\pgfpathlineto{\pgfqpoint{2.519988in}{0.441145in}}%
\pgfpathlineto{\pgfqpoint{2.521278in}{0.403569in}}%
\pgfpathlineto{\pgfqpoint{2.522567in}{0.400881in}}%
\pgfpathlineto{\pgfqpoint{2.523856in}{0.539614in}}%
\pgfpathlineto{\pgfqpoint{2.525145in}{0.432148in}}%
\pgfpathlineto{\pgfqpoint{2.529013in}{0.412545in}}%
\pgfpathlineto{\pgfqpoint{2.530302in}{0.401856in}}%
\pgfpathlineto{\pgfqpoint{2.531591in}{0.407236in}}%
\pgfpathlineto{\pgfqpoint{2.532880in}{0.459804in}}%
\pgfpathlineto{\pgfqpoint{2.534169in}{0.403962in}}%
\pgfpathlineto{\pgfqpoint{2.538037in}{0.408191in}}%
\pgfpathlineto{\pgfqpoint{2.539326in}{0.405504in}}%
\pgfpathlineto{\pgfqpoint{2.540615in}{0.401296in}}%
\pgfpathlineto{\pgfqpoint{2.541905in}{0.403416in}}%
\pgfpathlineto{\pgfqpoint{2.543194in}{0.403445in}}%
\pgfpathlineto{\pgfqpoint{2.547061in}{0.433704in}}%
\pgfpathlineto{\pgfqpoint{2.548351in}{0.402694in}}%
\pgfpathlineto{\pgfqpoint{2.550929in}{0.429692in}}%
\pgfpathlineto{\pgfqpoint{2.552218in}{0.412377in}}%
\pgfpathlineto{\pgfqpoint{2.556086in}{0.400289in}}%
\pgfpathlineto{\pgfqpoint{2.557375in}{0.441249in}}%
\pgfpathlineto{\pgfqpoint{2.558664in}{0.438309in}}%
\pgfpathlineto{\pgfqpoint{2.559953in}{0.574073in}}%
\pgfpathlineto{\pgfqpoint{2.561243in}{0.416323in}}%
\pgfpathlineto{\pgfqpoint{2.565110in}{0.404264in}}%
\pgfpathlineto{\pgfqpoint{2.566399in}{0.402175in}}%
\pgfpathlineto{\pgfqpoint{2.570267in}{0.400271in}}%
\pgfpathlineto{\pgfqpoint{2.574135in}{0.412466in}}%
\pgfpathlineto{\pgfqpoint{2.575424in}{0.413117in}}%
\pgfpathlineto{\pgfqpoint{2.576713in}{0.408336in}}%
\pgfpathlineto{\pgfqpoint{2.578002in}{0.479475in}}%
\pgfpathlineto{\pgfqpoint{2.579291in}{0.503028in}}%
\pgfpathlineto{\pgfqpoint{2.583159in}{0.400515in}}%
\pgfpathlineto{\pgfqpoint{2.584448in}{0.407029in}}%
\pgfpathlineto{\pgfqpoint{2.585737in}{0.400656in}}%
\pgfpathlineto{\pgfqpoint{2.587026in}{0.400518in}}%
\pgfpathlineto{\pgfqpoint{2.588316in}{0.431943in}}%
\pgfpathlineto{\pgfqpoint{2.592183in}{0.405383in}}%
\pgfpathlineto{\pgfqpoint{2.593472in}{0.463218in}}%
\pgfpathlineto{\pgfqpoint{2.594762in}{0.402087in}}%
\pgfpathlineto{\pgfqpoint{2.596051in}{0.418794in}}%
\pgfpathlineto{\pgfqpoint{2.597340in}{0.539515in}}%
\pgfpathlineto{\pgfqpoint{2.601208in}{0.418844in}}%
\pgfpathlineto{\pgfqpoint{2.602497in}{0.430600in}}%
\pgfpathlineto{\pgfqpoint{2.603786in}{0.409100in}}%
\pgfpathlineto{\pgfqpoint{2.605075in}{0.400332in}}%
\pgfpathlineto{\pgfqpoint{2.610232in}{0.400515in}}%
\pgfpathlineto{\pgfqpoint{2.611521in}{0.432115in}}%
\pgfpathlineto{\pgfqpoint{2.612810in}{0.425529in}}%
\pgfpathlineto{\pgfqpoint{2.614100in}{0.435882in}}%
\pgfpathlineto{\pgfqpoint{2.619256in}{0.428337in}}%
\pgfpathlineto{\pgfqpoint{2.620546in}{0.403447in}}%
\pgfpathlineto{\pgfqpoint{2.621835in}{0.477252in}}%
\pgfpathlineto{\pgfqpoint{2.623124in}{0.622194in}}%
\pgfpathlineto{\pgfqpoint{2.624413in}{0.415891in}}%
\pgfpathlineto{\pgfqpoint{2.629570in}{0.456047in}}%
\pgfpathlineto{\pgfqpoint{2.630859in}{0.485330in}}%
\pgfpathlineto{\pgfqpoint{2.632148in}{0.497937in}}%
\pgfpathlineto{\pgfqpoint{2.633438in}{1.772999in}}%
\pgfpathlineto{\pgfqpoint{2.638594in}{0.400607in}}%
\pgfpathlineto{\pgfqpoint{2.639884in}{0.407898in}}%
\pgfpathlineto{\pgfqpoint{2.641173in}{0.401036in}}%
\pgfpathlineto{\pgfqpoint{2.642462in}{0.412363in}}%
\pgfpathlineto{\pgfqpoint{2.647619in}{0.419193in}}%
\pgfpathlineto{\pgfqpoint{2.648908in}{0.402797in}}%
\pgfpathlineto{\pgfqpoint{2.650197in}{0.404358in}}%
\pgfpathlineto{\pgfqpoint{2.651486in}{0.581102in}}%
\pgfpathlineto{\pgfqpoint{2.655354in}{0.406584in}}%
\pgfpathlineto{\pgfqpoint{2.656643in}{0.571916in}}%
\pgfpathlineto{\pgfqpoint{2.657932in}{0.407898in}}%
\pgfpathlineto{\pgfqpoint{2.659221in}{0.433809in}}%
\pgfpathlineto{\pgfqpoint{2.660511in}{0.495326in}}%
\pgfpathlineto{\pgfqpoint{2.664378in}{0.409023in}}%
\pgfpathlineto{\pgfqpoint{2.665667in}{0.400293in}}%
\pgfpathlineto{\pgfqpoint{2.666957in}{0.461138in}}%
\pgfpathlineto{\pgfqpoint{2.668246in}{0.400294in}}%
\pgfpathlineto{\pgfqpoint{2.669535in}{0.432976in}}%
\pgfpathlineto{\pgfqpoint{2.674692in}{0.404024in}}%
\pgfpathlineto{\pgfqpoint{2.675981in}{0.483243in}}%
\pgfpathlineto{\pgfqpoint{2.677270in}{0.400609in}}%
\pgfpathlineto{\pgfqpoint{2.678559in}{0.491848in}}%
\pgfpathlineto{\pgfqpoint{2.682427in}{0.473378in}}%
\pgfpathlineto{\pgfqpoint{2.685005in}{0.428499in}}%
\pgfpathlineto{\pgfqpoint{2.686295in}{0.432476in}}%
\pgfpathlineto{\pgfqpoint{2.687584in}{0.405582in}}%
\pgfpathlineto{\pgfqpoint{2.691451in}{0.407768in}}%
\pgfpathlineto{\pgfqpoint{2.692741in}{0.502980in}}%
\pgfpathlineto{\pgfqpoint{2.694030in}{0.405324in}}%
\pgfpathlineto{\pgfqpoint{2.695319in}{0.400448in}}%
\pgfpathlineto{\pgfqpoint{2.696608in}{0.400760in}}%
\pgfpathlineto{\pgfqpoint{2.700476in}{0.415415in}}%
\pgfpathlineto{\pgfqpoint{2.701765in}{0.423994in}}%
\pgfpathlineto{\pgfqpoint{2.703054in}{0.416592in}}%
\pgfpathlineto{\pgfqpoint{2.704343in}{0.422208in}}%
\pgfpathlineto{\pgfqpoint{2.705632in}{0.439311in}}%
\pgfpathlineto{\pgfqpoint{2.712078in}{0.400563in}}%
\pgfpathlineto{\pgfqpoint{2.713368in}{0.412486in}}%
\pgfpathlineto{\pgfqpoint{2.714657in}{0.472035in}}%
\pgfpathlineto{\pgfqpoint{2.719814in}{0.400341in}}%
\pgfpathlineto{\pgfqpoint{2.721103in}{0.415162in}}%
\pgfpathlineto{\pgfqpoint{2.722392in}{0.402443in}}%
\pgfpathlineto{\pgfqpoint{2.727549in}{0.400343in}}%
\pgfpathlineto{\pgfqpoint{2.728838in}{0.434586in}}%
\pgfpathlineto{\pgfqpoint{2.730127in}{0.415826in}}%
\pgfpathlineto{\pgfqpoint{2.731416in}{0.419112in}}%
\pgfpathlineto{\pgfqpoint{2.732706in}{0.401684in}}%
\pgfpathlineto{\pgfqpoint{2.736573in}{0.429913in}}%
\pgfpathlineto{\pgfqpoint{2.737862in}{0.401724in}}%
\pgfpathlineto{\pgfqpoint{2.739152in}{0.404852in}}%
\pgfpathlineto{\pgfqpoint{2.740441in}{0.441960in}}%
\pgfpathlineto{\pgfqpoint{2.741730in}{0.401446in}}%
\pgfpathlineto{\pgfqpoint{2.745598in}{0.400436in}}%
\pgfpathlineto{\pgfqpoint{2.749465in}{0.416399in}}%
\pgfpathlineto{\pgfqpoint{2.750754in}{0.417860in}}%
\pgfpathlineto{\pgfqpoint{2.755911in}{0.400564in}}%
\pgfpathlineto{\pgfqpoint{2.757200in}{0.423737in}}%
\pgfpathlineto{\pgfqpoint{2.758490in}{0.400432in}}%
\pgfpathlineto{\pgfqpoint{2.759779in}{0.416567in}}%
\pgfpathlineto{\pgfqpoint{2.763646in}{0.409195in}}%
\pgfpathlineto{\pgfqpoint{2.764935in}{0.400290in}}%
\pgfpathlineto{\pgfqpoint{2.766225in}{0.417921in}}%
\pgfpathlineto{\pgfqpoint{2.768803in}{0.509601in}}%
\pgfpathlineto{\pgfqpoint{2.772671in}{0.418048in}}%
\pgfpathlineto{\pgfqpoint{2.773960in}{0.410694in}}%
\pgfpathlineto{\pgfqpoint{2.775249in}{0.410869in}}%
\pgfpathlineto{\pgfqpoint{2.776538in}{0.400775in}}%
\pgfpathlineto{\pgfqpoint{2.777827in}{0.403657in}}%
\pgfpathlineto{\pgfqpoint{2.781695in}{0.409999in}}%
\pgfpathlineto{\pgfqpoint{2.782984in}{0.419523in}}%
\pgfpathlineto{\pgfqpoint{2.784273in}{0.401543in}}%
\pgfpathlineto{\pgfqpoint{2.785563in}{0.414939in}}%
\pgfpathlineto{\pgfqpoint{2.786852in}{0.404234in}}%
\pgfpathlineto{\pgfqpoint{2.790719in}{0.438595in}}%
\pgfpathlineto{\pgfqpoint{2.792009in}{0.428793in}}%
\pgfpathlineto{\pgfqpoint{2.793298in}{0.400291in}}%
\pgfpathlineto{\pgfqpoint{2.795876in}{0.446682in}}%
\pgfpathlineto{\pgfqpoint{2.799744in}{0.401239in}}%
\pgfpathlineto{\pgfqpoint{2.801033in}{0.510983in}}%
\pgfpathlineto{\pgfqpoint{2.802322in}{0.416887in}}%
\pgfpathlineto{\pgfqpoint{2.803611in}{0.402092in}}%
\pgfpathlineto{\pgfqpoint{2.804901in}{0.401158in}}%
\pgfpathlineto{\pgfqpoint{2.810057in}{0.400344in}}%
\pgfpathlineto{\pgfqpoint{2.812636in}{0.401721in}}%
\pgfpathlineto{\pgfqpoint{2.813925in}{0.403301in}}%
\pgfpathlineto{\pgfqpoint{2.817793in}{0.400918in}}%
\pgfpathlineto{\pgfqpoint{2.819082in}{0.406045in}}%
\pgfpathlineto{\pgfqpoint{2.820371in}{0.400289in}}%
\pgfpathlineto{\pgfqpoint{2.826817in}{0.402786in}}%
\pgfpathlineto{\pgfqpoint{2.828106in}{0.400428in}}%
\pgfpathlineto{\pgfqpoint{2.829395in}{0.441080in}}%
\pgfpathlineto{\pgfqpoint{2.830684in}{0.401152in}}%
\pgfpathlineto{\pgfqpoint{2.831974in}{0.400915in}}%
\pgfpathlineto{\pgfqpoint{2.835841in}{0.425708in}}%
\pgfpathlineto{\pgfqpoint{2.837130in}{0.403663in}}%
\pgfpathlineto{\pgfqpoint{2.838420in}{0.400427in}}%
\pgfpathlineto{\pgfqpoint{2.839709in}{0.469468in}}%
\pgfpathlineto{\pgfqpoint{2.840998in}{0.701510in}}%
\pgfpathlineto{\pgfqpoint{2.846155in}{0.435001in}}%
\pgfpathlineto{\pgfqpoint{2.847444in}{0.483827in}}%
\pgfpathlineto{\pgfqpoint{2.848733in}{0.510226in}}%
\pgfpathlineto{\pgfqpoint{2.850022in}{0.400690in}}%
\pgfpathlineto{\pgfqpoint{2.855179in}{0.400877in}}%
\pgfpathlineto{\pgfqpoint{2.856468in}{0.411562in}}%
\pgfpathlineto{\pgfqpoint{2.857758in}{0.407523in}}%
\pgfpathlineto{\pgfqpoint{2.859047in}{0.487572in}}%
\pgfpathlineto{\pgfqpoint{2.862914in}{0.418070in}}%
\pgfpathlineto{\pgfqpoint{2.864204in}{0.440785in}}%
\pgfpathlineto{\pgfqpoint{2.865493in}{0.400801in}}%
\pgfpathlineto{\pgfqpoint{2.866782in}{0.405003in}}%
\pgfpathlineto{\pgfqpoint{2.868071in}{0.402370in}}%
\pgfpathlineto{\pgfqpoint{2.871939in}{0.400330in}}%
\pgfpathlineto{\pgfqpoint{2.873228in}{0.401455in}}%
\pgfpathlineto{\pgfqpoint{2.874517in}{0.436131in}}%
\pgfpathlineto{\pgfqpoint{2.875806in}{0.652745in}}%
\pgfpathlineto{\pgfqpoint{2.877096in}{0.419911in}}%
\pgfpathlineto{\pgfqpoint{2.880963in}{0.400406in}}%
\pgfpathlineto{\pgfqpoint{2.882252in}{0.420508in}}%
\pgfpathlineto{\pgfqpoint{2.884831in}{0.400806in}}%
\pgfpathlineto{\pgfqpoint{2.886120in}{0.401222in}}%
\pgfpathlineto{\pgfqpoint{2.889987in}{0.400271in}}%
\pgfpathlineto{\pgfqpoint{2.891277in}{0.411163in}}%
\pgfpathlineto{\pgfqpoint{2.892566in}{0.400649in}}%
\pgfpathlineto{\pgfqpoint{2.893855in}{0.414656in}}%
\pgfpathlineto{\pgfqpoint{2.895144in}{0.420332in}}%
\pgfpathlineto{\pgfqpoint{2.899012in}{0.400633in}}%
\pgfpathlineto{\pgfqpoint{2.900301in}{0.402025in}}%
\pgfpathlineto{\pgfqpoint{2.901590in}{0.419327in}}%
\pgfpathlineto{\pgfqpoint{2.902879in}{0.403173in}}%
\pgfpathlineto{\pgfqpoint{2.904169in}{0.401750in}}%
\pgfpathlineto{\pgfqpoint{2.908036in}{0.414388in}}%
\pgfpathlineto{\pgfqpoint{2.909325in}{0.411579in}}%
\pgfpathlineto{\pgfqpoint{2.911904in}{0.400503in}}%
\pgfpathlineto{\pgfqpoint{2.913193in}{0.409261in}}%
\pgfpathlineto{\pgfqpoint{2.918350in}{0.400498in}}%
\pgfpathlineto{\pgfqpoint{2.919639in}{0.407806in}}%
\pgfpathlineto{\pgfqpoint{2.920928in}{0.400789in}}%
\pgfpathlineto{\pgfqpoint{2.922217in}{0.403939in}}%
\pgfpathlineto{\pgfqpoint{2.926085in}{0.409832in}}%
\pgfpathlineto{\pgfqpoint{2.927374in}{0.403010in}}%
\pgfpathlineto{\pgfqpoint{2.928663in}{0.404754in}}%
\pgfpathlineto{\pgfqpoint{2.931242in}{0.400612in}}%
\pgfpathlineto{\pgfqpoint{2.936399in}{0.428700in}}%
\pgfpathlineto{\pgfqpoint{2.937688in}{0.401880in}}%
\pgfpathlineto{\pgfqpoint{2.938977in}{0.400285in}}%
\pgfpathlineto{\pgfqpoint{2.940266in}{0.518983in}}%
\pgfpathlineto{\pgfqpoint{2.946712in}{0.400285in}}%
\pgfpathlineto{\pgfqpoint{2.948001in}{0.503287in}}%
\pgfpathlineto{\pgfqpoint{2.949290in}{0.534177in}}%
\pgfpathlineto{\pgfqpoint{2.954447in}{0.400323in}}%
\pgfpathlineto{\pgfqpoint{2.955736in}{0.410991in}}%
\pgfpathlineto{\pgfqpoint{2.957026in}{0.401292in}}%
\pgfpathlineto{\pgfqpoint{2.958315in}{0.414095in}}%
\pgfpathlineto{\pgfqpoint{2.962182in}{0.432785in}}%
\pgfpathlineto{\pgfqpoint{2.963472in}{0.431506in}}%
\pgfpathlineto{\pgfqpoint{2.964761in}{0.407607in}}%
\pgfpathlineto{\pgfqpoint{2.966050in}{0.401806in}}%
\pgfpathlineto{\pgfqpoint{2.967339in}{0.420401in}}%
\pgfpathlineto{\pgfqpoint{2.971207in}{0.401281in}}%
\pgfpathlineto{\pgfqpoint{2.972496in}{0.401789in}}%
\pgfpathlineto{\pgfqpoint{2.973785in}{0.422198in}}%
\pgfpathlineto{\pgfqpoint{2.975074in}{0.400872in}}%
\pgfpathlineto{\pgfqpoint{2.976364in}{0.400381in}}%
\pgfpathlineto{\pgfqpoint{2.980231in}{0.400871in}}%
\pgfpathlineto{\pgfqpoint{2.981520in}{0.461567in}}%
\pgfpathlineto{\pgfqpoint{2.982810in}{0.402437in}}%
\pgfpathlineto{\pgfqpoint{2.984099in}{0.402802in}}%
\pgfpathlineto{\pgfqpoint{2.985388in}{0.425055in}}%
\pgfpathlineto{\pgfqpoint{2.989256in}{0.403128in}}%
\pgfpathlineto{\pgfqpoint{2.990545in}{0.423554in}}%
\pgfpathlineto{\pgfqpoint{2.991834in}{0.966633in}}%
\pgfpathlineto{\pgfqpoint{2.993123in}{0.401173in}}%
\pgfpathlineto{\pgfqpoint{2.994412in}{0.409167in}}%
\pgfpathlineto{\pgfqpoint{2.998280in}{0.401701in}}%
\pgfpathlineto{\pgfqpoint{3.000858in}{0.404451in}}%
\pgfpathlineto{\pgfqpoint{3.002148in}{0.401730in}}%
\pgfpathlineto{\pgfqpoint{3.003437in}{0.400799in}}%
\pgfpathlineto{\pgfqpoint{3.007304in}{0.402378in}}%
\pgfpathlineto{\pgfqpoint{3.008593in}{0.416309in}}%
\pgfpathlineto{\pgfqpoint{3.009883in}{0.401221in}}%
\pgfpathlineto{\pgfqpoint{3.011172in}{0.421906in}}%
\pgfpathlineto{\pgfqpoint{3.012461in}{0.414049in}}%
\pgfpathlineto{\pgfqpoint{3.016329in}{0.550025in}}%
\pgfpathlineto{\pgfqpoint{3.017618in}{0.400634in}}%
\pgfpathlineto{\pgfqpoint{3.018907in}{0.400286in}}%
\pgfpathlineto{\pgfqpoint{3.020196in}{0.407975in}}%
\pgfpathlineto{\pgfqpoint{3.021485in}{0.401733in}}%
\pgfpathlineto{\pgfqpoint{3.025353in}{0.402378in}}%
\pgfpathlineto{\pgfqpoint{3.026642in}{0.423478in}}%
\pgfpathlineto{\pgfqpoint{3.027931in}{0.400975in}}%
\pgfpathlineto{\pgfqpoint{3.029221in}{0.404408in}}%
\pgfpathlineto{\pgfqpoint{3.030510in}{0.400971in}}%
\pgfpathlineto{\pgfqpoint{3.034377in}{0.400400in}}%
\pgfpathlineto{\pgfqpoint{3.035667in}{0.431351in}}%
\pgfpathlineto{\pgfqpoint{3.036956in}{0.409722in}}%
\pgfpathlineto{\pgfqpoint{3.039534in}{0.407046in}}%
\pgfpathlineto{\pgfqpoint{3.043402in}{0.400951in}}%
\pgfpathlineto{\pgfqpoint{3.044691in}{0.405298in}}%
\pgfpathlineto{\pgfqpoint{3.045980in}{0.446575in}}%
\pgfpathlineto{\pgfqpoint{3.047269in}{0.513083in}}%
\pgfpathlineto{\pgfqpoint{3.048559in}{0.420997in}}%
\pgfpathlineto{\pgfqpoint{3.052426in}{0.406827in}}%
\pgfpathlineto{\pgfqpoint{3.053715in}{0.414327in}}%
\pgfpathlineto{\pgfqpoint{3.055005in}{0.475757in}}%
\pgfpathlineto{\pgfqpoint{3.056294in}{0.404735in}}%
\pgfpathlineto{\pgfqpoint{3.057583in}{0.400764in}}%
\pgfpathlineto{\pgfqpoint{3.061451in}{0.405172in}}%
\pgfpathlineto{\pgfqpoint{3.062740in}{0.480526in}}%
\pgfpathlineto{\pgfqpoint{3.064029in}{0.407749in}}%
\pgfpathlineto{\pgfqpoint{3.065318in}{0.407141in}}%
\pgfpathlineto{\pgfqpoint{3.066607in}{0.426739in}}%
\pgfpathlineto{\pgfqpoint{3.070475in}{0.438284in}}%
\pgfpathlineto{\pgfqpoint{3.071764in}{0.411719in}}%
\pgfpathlineto{\pgfqpoint{3.074342in}{0.400476in}}%
\pgfpathlineto{\pgfqpoint{3.075632in}{0.400386in}}%
\pgfpathlineto{\pgfqpoint{3.080788in}{0.401546in}}%
\pgfpathlineto{\pgfqpoint{3.082078in}{0.421885in}}%
\pgfpathlineto{\pgfqpoint{3.083367in}{0.400388in}}%
\pgfpathlineto{\pgfqpoint{3.084656in}{0.418249in}}%
\pgfpathlineto{\pgfqpoint{3.089813in}{0.412913in}}%
\pgfpathlineto{\pgfqpoint{3.091102in}{0.404517in}}%
\pgfpathlineto{\pgfqpoint{3.092391in}{0.400601in}}%
\pgfpathlineto{\pgfqpoint{3.093680in}{0.402167in}}%
\pgfpathlineto{\pgfqpoint{3.097548in}{0.402154in}}%
\pgfpathlineto{\pgfqpoint{3.098837in}{0.400911in}}%
\pgfpathlineto{\pgfqpoint{3.100126in}{0.419942in}}%
\pgfpathlineto{\pgfqpoint{3.101416in}{0.407081in}}%
\pgfpathlineto{\pgfqpoint{3.102705in}{0.401099in}}%
\pgfpathlineto{\pgfqpoint{3.109151in}{0.400478in}}%
\pgfpathlineto{\pgfqpoint{3.110440in}{0.404480in}}%
\pgfpathlineto{\pgfqpoint{3.111729in}{0.416105in}}%
\pgfpathlineto{\pgfqpoint{3.115597in}{0.403562in}}%
\pgfpathlineto{\pgfqpoint{3.116886in}{0.479044in}}%
\pgfpathlineto{\pgfqpoint{3.118175in}{0.403820in}}%
\pgfpathlineto{\pgfqpoint{3.119464in}{0.406778in}}%
\pgfpathlineto{\pgfqpoint{3.120754in}{0.418893in}}%
\pgfpathlineto{\pgfqpoint{3.125910in}{0.440024in}}%
\pgfpathlineto{\pgfqpoint{3.127199in}{0.409738in}}%
\pgfpathlineto{\pgfqpoint{3.128489in}{0.402826in}}%
\pgfpathlineto{\pgfqpoint{3.129778in}{0.401566in}}%
\pgfpathlineto{\pgfqpoint{3.133645in}{0.407758in}}%
\pgfpathlineto{\pgfqpoint{3.134935in}{0.401107in}}%
\pgfpathlineto{\pgfqpoint{3.136224in}{0.400389in}}%
\pgfpathlineto{\pgfqpoint{3.137513in}{0.501277in}}%
\pgfpathlineto{\pgfqpoint{3.138802in}{0.401934in}}%
\pgfpathlineto{\pgfqpoint{3.142670in}{0.425434in}}%
\pgfpathlineto{\pgfqpoint{3.143959in}{0.402533in}}%
\pgfpathlineto{\pgfqpoint{3.145248in}{0.401879in}}%
\pgfpathlineto{\pgfqpoint{3.146537in}{0.415429in}}%
\pgfpathlineto{\pgfqpoint{3.147827in}{0.400738in}}%
\pgfpathlineto{\pgfqpoint{3.152983in}{0.400478in}}%
\pgfpathlineto{\pgfqpoint{3.154273in}{0.423321in}}%
\pgfpathlineto{\pgfqpoint{3.155562in}{0.401590in}}%
\pgfpathlineto{\pgfqpoint{3.156851in}{0.414455in}}%
\pgfpathlineto{\pgfqpoint{3.160719in}{0.400323in}}%
\pgfpathlineto{\pgfqpoint{3.162008in}{0.411225in}}%
\pgfpathlineto{\pgfqpoint{3.163297in}{0.408542in}}%
\pgfpathlineto{\pgfqpoint{3.164586in}{0.400325in}}%
\pgfpathlineto{\pgfqpoint{3.165875in}{0.400285in}}%
\pgfpathlineto{\pgfqpoint{3.169743in}{0.413225in}}%
\pgfpathlineto{\pgfqpoint{3.171032in}{0.406233in}}%
\pgfpathlineto{\pgfqpoint{3.172321in}{0.404174in}}%
\pgfpathlineto{\pgfqpoint{3.173611in}{0.405143in}}%
\pgfpathlineto{\pgfqpoint{3.174900in}{0.401355in}}%
\pgfpathlineto{\pgfqpoint{3.178767in}{0.469001in}}%
\pgfpathlineto{\pgfqpoint{3.180057in}{0.400327in}}%
\pgfpathlineto{\pgfqpoint{3.181346in}{0.400954in}}%
\pgfpathlineto{\pgfqpoint{3.182635in}{0.400397in}}%
\pgfpathlineto{\pgfqpoint{3.183924in}{0.402619in}}%
\pgfpathlineto{\pgfqpoint{3.187792in}{0.403370in}}%
\pgfpathlineto{\pgfqpoint{3.189081in}{0.419253in}}%
\pgfpathlineto{\pgfqpoint{3.190370in}{0.413620in}}%
\pgfpathlineto{\pgfqpoint{3.191659in}{0.401389in}}%
\pgfpathlineto{\pgfqpoint{3.192948in}{0.401951in}}%
\pgfpathlineto{\pgfqpoint{3.196816in}{0.406968in}}%
\pgfpathlineto{\pgfqpoint{3.198105in}{0.405736in}}%
\pgfpathlineto{\pgfqpoint{3.199394in}{0.400394in}}%
\pgfpathlineto{\pgfqpoint{3.200684in}{0.404183in}}%
\pgfpathlineto{\pgfqpoint{3.201973in}{0.412284in}}%
\pgfpathlineto{\pgfqpoint{3.205840in}{0.401116in}}%
\pgfpathlineto{\pgfqpoint{3.207130in}{0.401859in}}%
\pgfpathlineto{\pgfqpoint{3.208419in}{0.400599in}}%
\pgfpathlineto{\pgfqpoint{3.209708in}{0.404545in}}%
\pgfpathlineto{\pgfqpoint{3.210997in}{0.400271in}}%
\pgfpathlineto{\pgfqpoint{3.214865in}{0.406731in}}%
\pgfpathlineto{\pgfqpoint{3.216154in}{0.400608in}}%
\pgfpathlineto{\pgfqpoint{3.217443in}{0.401907in}}%
\pgfpathlineto{\pgfqpoint{3.218732in}{0.417063in}}%
\pgfpathlineto{\pgfqpoint{3.223889in}{0.406342in}}%
\pgfpathlineto{\pgfqpoint{3.225178in}{0.410906in}}%
\pgfpathlineto{\pgfqpoint{3.226468in}{0.402536in}}%
\pgfpathlineto{\pgfqpoint{3.227757in}{0.408552in}}%
\pgfpathlineto{\pgfqpoint{3.229046in}{0.402485in}}%
\pgfpathlineto{\pgfqpoint{3.232914in}{0.420910in}}%
\pgfpathlineto{\pgfqpoint{3.234203in}{0.402101in}}%
\pgfpathlineto{\pgfqpoint{3.235492in}{0.400587in}}%
\pgfpathlineto{\pgfqpoint{3.236781in}{0.427785in}}%
\pgfpathlineto{\pgfqpoint{3.238070in}{0.583473in}}%
\pgfpathlineto{\pgfqpoint{3.241938in}{0.403223in}}%
\pgfpathlineto{\pgfqpoint{3.243227in}{0.449568in}}%
\pgfpathlineto{\pgfqpoint{3.244516in}{0.409382in}}%
\pgfpathlineto{\pgfqpoint{3.245805in}{0.402375in}}%
\pgfpathlineto{\pgfqpoint{3.247095in}{0.400321in}}%
\pgfpathlineto{\pgfqpoint{3.250962in}{0.409461in}}%
\pgfpathlineto{\pgfqpoint{3.252251in}{0.403539in}}%
\pgfpathlineto{\pgfqpoint{3.253541in}{0.415244in}}%
\pgfpathlineto{\pgfqpoint{3.254830in}{0.412149in}}%
\pgfpathlineto{\pgfqpoint{3.256119in}{0.403281in}}%
\pgfpathlineto{\pgfqpoint{3.259987in}{0.401357in}}%
\pgfpathlineto{\pgfqpoint{3.261276in}{0.404578in}}%
\pgfpathlineto{\pgfqpoint{3.262565in}{0.472278in}}%
\pgfpathlineto{\pgfqpoint{3.263854in}{0.404242in}}%
\pgfpathlineto{\pgfqpoint{3.265143in}{0.404202in}}%
\pgfpathlineto{\pgfqpoint{3.269011in}{0.415756in}}%
\pgfpathlineto{\pgfqpoint{3.270300in}{0.401343in}}%
\pgfpathlineto{\pgfqpoint{3.271589in}{0.408455in}}%
\pgfpathlineto{\pgfqpoint{3.272879in}{0.402460in}}%
\pgfpathlineto{\pgfqpoint{3.274168in}{0.400271in}}%
\pgfpathlineto{\pgfqpoint{3.279325in}{0.401099in}}%
\pgfpathlineto{\pgfqpoint{3.281903in}{0.400284in}}%
\pgfpathlineto{\pgfqpoint{3.283192in}{0.404460in}}%
\pgfpathlineto{\pgfqpoint{3.287060in}{0.400323in}}%
\pgfpathlineto{\pgfqpoint{3.288349in}{0.405440in}}%
\pgfpathlineto{\pgfqpoint{3.289638in}{0.402460in}}%
\pgfpathlineto{\pgfqpoint{3.290927in}{0.405925in}}%
\pgfpathlineto{\pgfqpoint{3.292217in}{0.469641in}}%
\pgfpathlineto{\pgfqpoint{3.296084in}{0.400325in}}%
\pgfpathlineto{\pgfqpoint{3.297373in}{0.402864in}}%
\pgfpathlineto{\pgfqpoint{3.298663in}{0.414759in}}%
\pgfpathlineto{\pgfqpoint{3.301241in}{0.401376in}}%
\pgfpathlineto{\pgfqpoint{3.305108in}{0.410906in}}%
\pgfpathlineto{\pgfqpoint{3.306398in}{0.451248in}}%
\pgfpathlineto{\pgfqpoint{3.307687in}{0.410528in}}%
\pgfpathlineto{\pgfqpoint{3.310265in}{0.403968in}}%
\pgfpathlineto{\pgfqpoint{3.314133in}{0.402033in}}%
\pgfpathlineto{\pgfqpoint{3.316711in}{0.440130in}}%
\pgfpathlineto{\pgfqpoint{3.318000in}{0.457113in}}%
\pgfpathlineto{\pgfqpoint{3.319290in}{0.405682in}}%
\pgfpathlineto{\pgfqpoint{3.323157in}{0.411225in}}%
\pgfpathlineto{\pgfqpoint{3.325736in}{0.501986in}}%
\pgfpathlineto{\pgfqpoint{3.327025in}{0.466218in}}%
\pgfpathlineto{\pgfqpoint{3.328314in}{0.408847in}}%
\pgfpathlineto{\pgfqpoint{3.332182in}{0.407473in}}%
\pgfpathlineto{\pgfqpoint{3.334760in}{0.414301in}}%
\pgfpathlineto{\pgfqpoint{3.336049in}{0.400286in}}%
\pgfpathlineto{\pgfqpoint{3.337338in}{0.425465in}}%
\pgfpathlineto{\pgfqpoint{3.342495in}{0.400628in}}%
\pgfpathlineto{\pgfqpoint{3.343784in}{0.400399in}}%
\pgfpathlineto{\pgfqpoint{3.345074in}{0.404848in}}%
\pgfpathlineto{\pgfqpoint{3.346363in}{0.400328in}}%
\pgfpathlineto{\pgfqpoint{3.350230in}{0.406514in}}%
\pgfpathlineto{\pgfqpoint{3.352809in}{0.400961in}}%
\pgfpathlineto{\pgfqpoint{3.354098in}{0.406427in}}%
\pgfpathlineto{\pgfqpoint{3.355387in}{0.414337in}}%
\pgfpathlineto{\pgfqpoint{3.359255in}{0.403319in}}%
\pgfpathlineto{\pgfqpoint{3.360544in}{0.490904in}}%
\pgfpathlineto{\pgfqpoint{3.361833in}{0.410255in}}%
\pgfpathlineto{\pgfqpoint{3.363122in}{0.401795in}}%
\pgfpathlineto{\pgfqpoint{3.364412in}{0.404359in}}%
\pgfpathlineto{\pgfqpoint{3.368279in}{0.402407in}}%
\pgfpathlineto{\pgfqpoint{3.369568in}{0.400322in}}%
\pgfpathlineto{\pgfqpoint{3.370857in}{0.403896in}}%
\pgfpathlineto{\pgfqpoint{3.372147in}{0.423635in}}%
\pgfpathlineto{\pgfqpoint{3.373436in}{0.408986in}}%
\pgfpathlineto{\pgfqpoint{3.377303in}{0.426233in}}%
\pgfpathlineto{\pgfqpoint{3.379882in}{0.404470in}}%
\pgfpathlineto{\pgfqpoint{3.381171in}{0.449630in}}%
\pgfpathlineto{\pgfqpoint{3.382460in}{0.403324in}}%
\pgfpathlineto{\pgfqpoint{3.386328in}{0.401378in}}%
\pgfpathlineto{\pgfqpoint{3.387617in}{0.409608in}}%
\pgfpathlineto{\pgfqpoint{3.388906in}{0.400285in}}%
\pgfpathlineto{\pgfqpoint{3.391485in}{0.400494in}}%
\pgfpathlineto{\pgfqpoint{3.395352in}{0.400327in}}%
\pgfpathlineto{\pgfqpoint{3.396641in}{0.401160in}}%
\pgfpathlineto{\pgfqpoint{3.397931in}{0.403374in}}%
\pgfpathlineto{\pgfqpoint{3.399220in}{0.404218in}}%
\pgfpathlineto{\pgfqpoint{3.400509in}{0.400326in}}%
\pgfpathlineto{\pgfqpoint{3.405666in}{0.400938in}}%
\pgfpathlineto{\pgfqpoint{3.406955in}{0.467539in}}%
\pgfpathlineto{\pgfqpoint{3.408244in}{0.406069in}}%
\pgfpathlineto{\pgfqpoint{3.413401in}{0.440421in}}%
\pgfpathlineto{\pgfqpoint{3.414690in}{0.411921in}}%
\pgfpathlineto{\pgfqpoint{3.417269in}{0.402743in}}%
\pgfpathlineto{\pgfqpoint{3.418558in}{0.431222in}}%
\pgfpathlineto{\pgfqpoint{3.422425in}{0.400271in}}%
\pgfpathlineto{\pgfqpoint{3.423715in}{0.406142in}}%
\pgfpathlineto{\pgfqpoint{3.425004in}{0.402995in}}%
\pgfpathlineto{\pgfqpoint{3.426293in}{0.402016in}}%
\pgfpathlineto{\pgfqpoint{3.427582in}{0.400320in}}%
\pgfpathlineto{\pgfqpoint{3.431450in}{0.400320in}}%
\pgfpathlineto{\pgfqpoint{3.432739in}{0.411071in}}%
\pgfpathlineto{\pgfqpoint{3.434028in}{0.400699in}}%
\pgfpathlineto{\pgfqpoint{3.435317in}{0.409496in}}%
\pgfpathlineto{\pgfqpoint{3.436606in}{0.406952in}}%
\pgfpathlineto{\pgfqpoint{3.440474in}{0.493332in}}%
\pgfpathlineto{\pgfqpoint{3.441763in}{0.411402in}}%
\pgfpathlineto{\pgfqpoint{3.443052in}{0.400444in}}%
\pgfpathlineto{\pgfqpoint{3.444342in}{0.404150in}}%
\pgfpathlineto{\pgfqpoint{3.445631in}{0.401135in}}%
\pgfpathlineto{\pgfqpoint{3.449498in}{0.405391in}}%
\pgfpathlineto{\pgfqpoint{3.450788in}{0.404500in}}%
\pgfpathlineto{\pgfqpoint{3.452077in}{0.411948in}}%
\pgfpathlineto{\pgfqpoint{3.453366in}{0.401584in}}%
\pgfpathlineto{\pgfqpoint{3.454655in}{0.423014in}}%
\pgfpathlineto{\pgfqpoint{3.458523in}{0.400949in}}%
\pgfpathlineto{\pgfqpoint{3.459812in}{0.400366in}}%
\pgfpathlineto{\pgfqpoint{3.461101in}{0.420472in}}%
\pgfpathlineto{\pgfqpoint{3.462390in}{0.402601in}}%
\pgfpathlineto{\pgfqpoint{3.463680in}{0.410828in}}%
\pgfpathlineto{\pgfqpoint{3.467547in}{0.414889in}}%
\pgfpathlineto{\pgfqpoint{3.468836in}{0.401710in}}%
\pgfpathlineto{\pgfqpoint{3.470126in}{0.403163in}}%
\pgfpathlineto{\pgfqpoint{3.471415in}{0.430248in}}%
\pgfpathlineto{\pgfqpoint{3.472704in}{1.165285in}}%
\pgfpathlineto{\pgfqpoint{3.476572in}{0.400348in}}%
\pgfpathlineto{\pgfqpoint{3.477861in}{0.495038in}}%
\pgfpathlineto{\pgfqpoint{3.479150in}{0.506061in}}%
\pgfpathlineto{\pgfqpoint{3.480439in}{0.410667in}}%
\pgfpathlineto{\pgfqpoint{3.481728in}{0.441069in}}%
\pgfpathlineto{\pgfqpoint{3.485596in}{0.428835in}}%
\pgfpathlineto{\pgfqpoint{3.486885in}{0.400755in}}%
\pgfpathlineto{\pgfqpoint{3.488174in}{0.400755in}}%
\pgfpathlineto{\pgfqpoint{3.489463in}{0.411299in}}%
\pgfpathlineto{\pgfqpoint{3.490753in}{0.437521in}}%
\pgfpathlineto{\pgfqpoint{3.494620in}{0.402299in}}%
\pgfpathlineto{\pgfqpoint{3.495909in}{0.401224in}}%
\pgfpathlineto{\pgfqpoint{3.497199in}{0.412319in}}%
\pgfpathlineto{\pgfqpoint{3.498488in}{0.402850in}}%
\pgfpathlineto{\pgfqpoint{3.499777in}{0.476537in}}%
\pgfpathlineto{\pgfqpoint{3.503645in}{0.400280in}}%
\pgfpathlineto{\pgfqpoint{3.504934in}{0.408184in}}%
\pgfpathlineto{\pgfqpoint{3.506223in}{0.406722in}}%
\pgfpathlineto{\pgfqpoint{3.508801in}{0.400941in}}%
\pgfpathlineto{\pgfqpoint{3.512669in}{0.405456in}}%
\pgfpathlineto{\pgfqpoint{3.513958in}{0.404662in}}%
\pgfpathlineto{\pgfqpoint{3.515247in}{0.446271in}}%
\pgfpathlineto{\pgfqpoint{3.516537in}{0.460807in}}%
\pgfpathlineto{\pgfqpoint{3.517826in}{0.402126in}}%
\pgfpathlineto{\pgfqpoint{3.521693in}{0.403270in}}%
\pgfpathlineto{\pgfqpoint{3.522983in}{0.485721in}}%
\pgfpathlineto{\pgfqpoint{3.524272in}{0.400280in}}%
\pgfpathlineto{\pgfqpoint{3.525561in}{0.411077in}}%
\pgfpathlineto{\pgfqpoint{3.526850in}{0.405798in}}%
\pgfpathlineto{\pgfqpoint{3.532007in}{0.409211in}}%
\pgfpathlineto{\pgfqpoint{3.533296in}{0.400280in}}%
\pgfpathlineto{\pgfqpoint{3.534585in}{0.400703in}}%
\pgfpathlineto{\pgfqpoint{3.535875in}{0.531978in}}%
\pgfpathlineto{\pgfqpoint{3.539742in}{0.613091in}}%
\pgfpathlineto{\pgfqpoint{3.541031in}{0.441829in}}%
\pgfpathlineto{\pgfqpoint{3.542321in}{0.418727in}}%
\pgfpathlineto{\pgfqpoint{3.543610in}{0.443159in}}%
\pgfpathlineto{\pgfqpoint{3.544899in}{0.400544in}}%
\pgfpathlineto{\pgfqpoint{3.550056in}{0.426981in}}%
\pgfpathlineto{\pgfqpoint{3.551345in}{0.400302in}}%
\pgfpathlineto{\pgfqpoint{3.552634in}{0.401209in}}%
\pgfpathlineto{\pgfqpoint{3.553923in}{0.400550in}}%
\pgfpathlineto{\pgfqpoint{3.553923in}{0.400550in}}%
\pgfusepath{stroke}%
\end{pgfscope}%
\begin{pgfscope}%
\pgfsetrectcap%
\pgfsetmiterjoin%
\pgfsetlinewidth{0.803000pt}%
\definecolor{currentstroke}{rgb}{1.000000,1.000000,1.000000}%
\pgfsetstrokecolor{currentstroke}%
\pgfsetdash{}{0pt}%
\pgfpathmoveto{\pgfqpoint{0.594832in}{0.331635in}}%
\pgfpathlineto{\pgfqpoint{0.594832in}{1.841635in}}%
\pgfusepath{stroke}%
\end{pgfscope}%
\begin{pgfscope}%
\pgfsetrectcap%
\pgfsetmiterjoin%
\pgfsetlinewidth{0.803000pt}%
\definecolor{currentstroke}{rgb}{1.000000,1.000000,1.000000}%
\pgfsetstrokecolor{currentstroke}%
\pgfsetdash{}{0pt}%
\pgfpathmoveto{\pgfqpoint{3.694832in}{0.331635in}}%
\pgfpathlineto{\pgfqpoint{3.694832in}{1.841635in}}%
\pgfusepath{stroke}%
\end{pgfscope}%
\begin{pgfscope}%
\pgfsetrectcap%
\pgfsetmiterjoin%
\pgfsetlinewidth{0.803000pt}%
\definecolor{currentstroke}{rgb}{1.000000,1.000000,1.000000}%
\pgfsetstrokecolor{currentstroke}%
\pgfsetdash{}{0pt}%
\pgfpathmoveto{\pgfqpoint{0.594832in}{0.331635in}}%
\pgfpathlineto{\pgfqpoint{3.694832in}{0.331635in}}%
\pgfusepath{stroke}%
\end{pgfscope}%
\begin{pgfscope}%
\pgfsetrectcap%
\pgfsetmiterjoin%
\pgfsetlinewidth{0.803000pt}%
\definecolor{currentstroke}{rgb}{1.000000,1.000000,1.000000}%
\pgfsetstrokecolor{currentstroke}%
\pgfsetdash{}{0pt}%
\pgfpathmoveto{\pgfqpoint{0.594832in}{1.841635in}}%
\pgfpathlineto{\pgfqpoint{3.694832in}{1.841635in}}%
\pgfusepath{stroke}%
\end{pgfscope}%
\end{pgfpicture}%
\makeatother%
\endgroup%

    \end{adjustbox}
    \hspace{3ex}
    \figuretitle{ACF and PACF of Squared log-returns V and INTC}
    \begin{adjustbox}{width=.95\textwidth,center}
    %% Creator: Matplotlib, PGF backend
%%
%% To include the figure in your LaTeX document, write
%%   \input{<filename>.pgf}
%%
%% Make sure the required packages are loaded in your preamble
%%   \usepackage{pgf}
%%
%% Figures using additional raster images can only be included by \input if
%% they are in the same directory as the main LaTeX file. For loading figures
%% from other directories you can use the `import` package
%%   \usepackage{import}
%% and then include the figures with
%%   \import{<path to file>}{<filename>.pgf}
%%
%% Matplotlib used the following preamble
%%   \usepackage{fontspec}
%%   \setmainfont{DejaVuSerif.ttf}[Path=/opt/tljh/user/lib/python3.6/site-packages/matplotlib/mpl-data/fonts/ttf/]
%%   \setsansfont{DejaVuSans.ttf}[Path=/opt/tljh/user/lib/python3.6/site-packages/matplotlib/mpl-data/fonts/ttf/]
%%   \setmonofont{DejaVuSansMono.ttf}[Path=/opt/tljh/user/lib/python3.6/site-packages/matplotlib/mpl-data/fonts/ttf/]
%%
\begingroup%
\makeatletter%
\begin{pgfpicture}%
\pgfpathrectangle{\pgfpointorigin}{\pgfqpoint{6.806467in}{2.151596in}}%
\pgfusepath{use as bounding box, clip}%
\begin{pgfscope}%
\pgfsetbuttcap%
\pgfsetmiterjoin%
\definecolor{currentfill}{rgb}{1.000000,1.000000,1.000000}%
\pgfsetfillcolor{currentfill}%
\pgfsetlinewidth{0.000000pt}%
\definecolor{currentstroke}{rgb}{1.000000,1.000000,1.000000}%
\pgfsetstrokecolor{currentstroke}%
\pgfsetdash{}{0pt}%
\pgfpathmoveto{\pgfqpoint{0.000000in}{0.000000in}}%
\pgfpathlineto{\pgfqpoint{6.806467in}{0.000000in}}%
\pgfpathlineto{\pgfqpoint{6.806467in}{2.151596in}}%
\pgfpathlineto{\pgfqpoint{0.000000in}{2.151596in}}%
\pgfpathclose%
\pgfusepath{fill}%
\end{pgfscope}%
\begin{pgfscope}%
\pgfsetbuttcap%
\pgfsetmiterjoin%
\definecolor{currentfill}{rgb}{0.917647,0.917647,0.949020}%
\pgfsetfillcolor{currentfill}%
\pgfsetlinewidth{0.000000pt}%
\definecolor{currentstroke}{rgb}{0.000000,0.000000,0.000000}%
\pgfsetstrokecolor{currentstroke}%
\pgfsetstrokeopacity{0.000000}%
\pgfsetdash{}{0pt}%
\pgfpathmoveto{\pgfqpoint{0.506467in}{0.331635in}}%
\pgfpathlineto{\pgfqpoint{3.089800in}{0.331635in}}%
\pgfpathlineto{\pgfqpoint{3.089800in}{1.841635in}}%
\pgfpathlineto{\pgfqpoint{0.506467in}{1.841635in}}%
\pgfpathclose%
\pgfusepath{fill}%
\end{pgfscope}%
\begin{pgfscope}%
\pgfpathrectangle{\pgfqpoint{0.506467in}{0.331635in}}{\pgfqpoint{2.583333in}{1.510000in}}%
\pgfusepath{clip}%
\pgfsetroundcap%
\pgfsetroundjoin%
\pgfsetlinewidth{0.803000pt}%
\definecolor{currentstroke}{rgb}{1.000000,1.000000,1.000000}%
\pgfsetstrokecolor{currentstroke}%
\pgfsetdash{}{0pt}%
\pgfpathmoveto{\pgfqpoint{0.623891in}{0.331635in}}%
\pgfpathlineto{\pgfqpoint{0.623891in}{1.841635in}}%
\pgfusepath{stroke}%
\end{pgfscope}%
\begin{pgfscope}%
\definecolor{textcolor}{rgb}{0.150000,0.150000,0.150000}%
\pgfsetstrokecolor{textcolor}%
\pgfsetfillcolor{textcolor}%
\pgftext[x=0.623891in,y=0.234413in,,top]{\color{textcolor}\rmfamily\fontsize{10.000000}{12.000000}\selectfont 0}%
\end{pgfscope}%
\begin{pgfscope}%
\pgfpathrectangle{\pgfqpoint{0.506467in}{0.331635in}}{\pgfqpoint{2.583333in}{1.510000in}}%
\pgfusepath{clip}%
\pgfsetroundcap%
\pgfsetroundjoin%
\pgfsetlinewidth{0.803000pt}%
\definecolor{currentstroke}{rgb}{1.000000,1.000000,1.000000}%
\pgfsetstrokecolor{currentstroke}%
\pgfsetdash{}{0pt}%
\pgfpathmoveto{\pgfqpoint{1.196693in}{0.331635in}}%
\pgfpathlineto{\pgfqpoint{1.196693in}{1.841635in}}%
\pgfusepath{stroke}%
\end{pgfscope}%
\begin{pgfscope}%
\definecolor{textcolor}{rgb}{0.150000,0.150000,0.150000}%
\pgfsetstrokecolor{textcolor}%
\pgfsetfillcolor{textcolor}%
\pgftext[x=1.196693in,y=0.234413in,,top]{\color{textcolor}\rmfamily\fontsize{10.000000}{12.000000}\selectfont 5}%
\end{pgfscope}%
\begin{pgfscope}%
\pgfpathrectangle{\pgfqpoint{0.506467in}{0.331635in}}{\pgfqpoint{2.583333in}{1.510000in}}%
\pgfusepath{clip}%
\pgfsetroundcap%
\pgfsetroundjoin%
\pgfsetlinewidth{0.803000pt}%
\definecolor{currentstroke}{rgb}{1.000000,1.000000,1.000000}%
\pgfsetstrokecolor{currentstroke}%
\pgfsetdash{}{0pt}%
\pgfpathmoveto{\pgfqpoint{1.769494in}{0.331635in}}%
\pgfpathlineto{\pgfqpoint{1.769494in}{1.841635in}}%
\pgfusepath{stroke}%
\end{pgfscope}%
\begin{pgfscope}%
\definecolor{textcolor}{rgb}{0.150000,0.150000,0.150000}%
\pgfsetstrokecolor{textcolor}%
\pgfsetfillcolor{textcolor}%
\pgftext[x=1.769494in,y=0.234413in,,top]{\color{textcolor}\rmfamily\fontsize{10.000000}{12.000000}\selectfont 10}%
\end{pgfscope}%
\begin{pgfscope}%
\pgfpathrectangle{\pgfqpoint{0.506467in}{0.331635in}}{\pgfqpoint{2.583333in}{1.510000in}}%
\pgfusepath{clip}%
\pgfsetroundcap%
\pgfsetroundjoin%
\pgfsetlinewidth{0.803000pt}%
\definecolor{currentstroke}{rgb}{1.000000,1.000000,1.000000}%
\pgfsetstrokecolor{currentstroke}%
\pgfsetdash{}{0pt}%
\pgfpathmoveto{\pgfqpoint{2.342295in}{0.331635in}}%
\pgfpathlineto{\pgfqpoint{2.342295in}{1.841635in}}%
\pgfusepath{stroke}%
\end{pgfscope}%
\begin{pgfscope}%
\definecolor{textcolor}{rgb}{0.150000,0.150000,0.150000}%
\pgfsetstrokecolor{textcolor}%
\pgfsetfillcolor{textcolor}%
\pgftext[x=2.342295in,y=0.234413in,,top]{\color{textcolor}\rmfamily\fontsize{10.000000}{12.000000}\selectfont 15}%
\end{pgfscope}%
\begin{pgfscope}%
\pgfpathrectangle{\pgfqpoint{0.506467in}{0.331635in}}{\pgfqpoint{2.583333in}{1.510000in}}%
\pgfusepath{clip}%
\pgfsetroundcap%
\pgfsetroundjoin%
\pgfsetlinewidth{0.803000pt}%
\definecolor{currentstroke}{rgb}{1.000000,1.000000,1.000000}%
\pgfsetstrokecolor{currentstroke}%
\pgfsetdash{}{0pt}%
\pgfpathmoveto{\pgfqpoint{2.915096in}{0.331635in}}%
\pgfpathlineto{\pgfqpoint{2.915096in}{1.841635in}}%
\pgfusepath{stroke}%
\end{pgfscope}%
\begin{pgfscope}%
\definecolor{textcolor}{rgb}{0.150000,0.150000,0.150000}%
\pgfsetstrokecolor{textcolor}%
\pgfsetfillcolor{textcolor}%
\pgftext[x=2.915096in,y=0.234413in,,top]{\color{textcolor}\rmfamily\fontsize{10.000000}{12.000000}\selectfont 20}%
\end{pgfscope}%
\begin{pgfscope}%
\pgfpathrectangle{\pgfqpoint{0.506467in}{0.331635in}}{\pgfqpoint{2.583333in}{1.510000in}}%
\pgfusepath{clip}%
\pgfsetroundcap%
\pgfsetroundjoin%
\pgfsetlinewidth{0.803000pt}%
\definecolor{currentstroke}{rgb}{1.000000,1.000000,1.000000}%
\pgfsetstrokecolor{currentstroke}%
\pgfsetdash{}{0pt}%
\pgfpathmoveto{\pgfqpoint{0.506467in}{0.467674in}}%
\pgfpathlineto{\pgfqpoint{3.089800in}{0.467674in}}%
\pgfusepath{stroke}%
\end{pgfscope}%
\begin{pgfscope}%
\definecolor{textcolor}{rgb}{0.150000,0.150000,0.150000}%
\pgfsetstrokecolor{textcolor}%
\pgfsetfillcolor{textcolor}%
\pgftext[x=0.100000in,y=0.414913in,left,base]{\color{textcolor}\rmfamily\fontsize{10.000000}{12.000000}\selectfont 0.00}%
\end{pgfscope}%
\begin{pgfscope}%
\pgfpathrectangle{\pgfqpoint{0.506467in}{0.331635in}}{\pgfqpoint{2.583333in}{1.510000in}}%
\pgfusepath{clip}%
\pgfsetroundcap%
\pgfsetroundjoin%
\pgfsetlinewidth{0.803000pt}%
\definecolor{currentstroke}{rgb}{1.000000,1.000000,1.000000}%
\pgfsetstrokecolor{currentstroke}%
\pgfsetdash{}{0pt}%
\pgfpathmoveto{\pgfqpoint{0.506467in}{0.794005in}}%
\pgfpathlineto{\pgfqpoint{3.089800in}{0.794005in}}%
\pgfusepath{stroke}%
\end{pgfscope}%
\begin{pgfscope}%
\definecolor{textcolor}{rgb}{0.150000,0.150000,0.150000}%
\pgfsetstrokecolor{textcolor}%
\pgfsetfillcolor{textcolor}%
\pgftext[x=0.100000in,y=0.741244in,left,base]{\color{textcolor}\rmfamily\fontsize{10.000000}{12.000000}\selectfont 0.25}%
\end{pgfscope}%
\begin{pgfscope}%
\pgfpathrectangle{\pgfqpoint{0.506467in}{0.331635in}}{\pgfqpoint{2.583333in}{1.510000in}}%
\pgfusepath{clip}%
\pgfsetroundcap%
\pgfsetroundjoin%
\pgfsetlinewidth{0.803000pt}%
\definecolor{currentstroke}{rgb}{1.000000,1.000000,1.000000}%
\pgfsetstrokecolor{currentstroke}%
\pgfsetdash{}{0pt}%
\pgfpathmoveto{\pgfqpoint{0.506467in}{1.120336in}}%
\pgfpathlineto{\pgfqpoint{3.089800in}{1.120336in}}%
\pgfusepath{stroke}%
\end{pgfscope}%
\begin{pgfscope}%
\definecolor{textcolor}{rgb}{0.150000,0.150000,0.150000}%
\pgfsetstrokecolor{textcolor}%
\pgfsetfillcolor{textcolor}%
\pgftext[x=0.100000in,y=1.067575in,left,base]{\color{textcolor}\rmfamily\fontsize{10.000000}{12.000000}\selectfont 0.50}%
\end{pgfscope}%
\begin{pgfscope}%
\pgfpathrectangle{\pgfqpoint{0.506467in}{0.331635in}}{\pgfqpoint{2.583333in}{1.510000in}}%
\pgfusepath{clip}%
\pgfsetroundcap%
\pgfsetroundjoin%
\pgfsetlinewidth{0.803000pt}%
\definecolor{currentstroke}{rgb}{1.000000,1.000000,1.000000}%
\pgfsetstrokecolor{currentstroke}%
\pgfsetdash{}{0pt}%
\pgfpathmoveto{\pgfqpoint{0.506467in}{1.446668in}}%
\pgfpathlineto{\pgfqpoint{3.089800in}{1.446668in}}%
\pgfusepath{stroke}%
\end{pgfscope}%
\begin{pgfscope}%
\definecolor{textcolor}{rgb}{0.150000,0.150000,0.150000}%
\pgfsetstrokecolor{textcolor}%
\pgfsetfillcolor{textcolor}%
\pgftext[x=0.100000in,y=1.393906in,left,base]{\color{textcolor}\rmfamily\fontsize{10.000000}{12.000000}\selectfont 0.75}%
\end{pgfscope}%
\begin{pgfscope}%
\pgfpathrectangle{\pgfqpoint{0.506467in}{0.331635in}}{\pgfqpoint{2.583333in}{1.510000in}}%
\pgfusepath{clip}%
\pgfsetroundcap%
\pgfsetroundjoin%
\pgfsetlinewidth{0.803000pt}%
\definecolor{currentstroke}{rgb}{1.000000,1.000000,1.000000}%
\pgfsetstrokecolor{currentstroke}%
\pgfsetdash{}{0pt}%
\pgfpathmoveto{\pgfqpoint{0.506467in}{1.772999in}}%
\pgfpathlineto{\pgfqpoint{3.089800in}{1.772999in}}%
\pgfusepath{stroke}%
\end{pgfscope}%
\begin{pgfscope}%
\definecolor{textcolor}{rgb}{0.150000,0.150000,0.150000}%
\pgfsetstrokecolor{textcolor}%
\pgfsetfillcolor{textcolor}%
\pgftext[x=0.100000in,y=1.720237in,left,base]{\color{textcolor}\rmfamily\fontsize{10.000000}{12.000000}\selectfont 1.00}%
\end{pgfscope}%
\begin{pgfscope}%
\pgfpathrectangle{\pgfqpoint{0.506467in}{0.331635in}}{\pgfqpoint{2.583333in}{1.510000in}}%
\pgfusepath{clip}%
\pgfsetbuttcap%
\pgfsetroundjoin%
\definecolor{currentfill}{rgb}{0.121569,0.466667,0.705882}%
\pgfsetfillcolor{currentfill}%
\pgfsetfillopacity{0.250000}%
\pgfsetlinewidth{1.003750pt}%
\definecolor{currentstroke}{rgb}{1.000000,1.000000,1.000000}%
\pgfsetstrokecolor{currentstroke}%
\pgfsetstrokeopacity{0.250000}%
\pgfsetdash{}{0pt}%
\pgfpathmoveto{\pgfqpoint{0.681171in}{0.533556in}}%
\pgfpathlineto{\pgfqpoint{0.681171in}{0.401792in}}%
\pgfpathlineto{\pgfqpoint{0.853012in}{0.401526in}}%
\pgfpathlineto{\pgfqpoint{0.967572in}{0.401418in}}%
\pgfpathlineto{\pgfqpoint{1.082132in}{0.401389in}}%
\pgfpathlineto{\pgfqpoint{1.196693in}{0.400960in}}%
\pgfpathlineto{\pgfqpoint{1.311253in}{0.400887in}}%
\pgfpathlineto{\pgfqpoint{1.425813in}{0.400590in}}%
\pgfpathlineto{\pgfqpoint{1.540373in}{0.400586in}}%
\pgfpathlineto{\pgfqpoint{1.654933in}{0.400586in}}%
\pgfpathlineto{\pgfqpoint{1.769494in}{0.400501in}}%
\pgfpathlineto{\pgfqpoint{1.884054in}{0.400498in}}%
\pgfpathlineto{\pgfqpoint{1.998614in}{0.400469in}}%
\pgfpathlineto{\pgfqpoint{2.113174in}{0.400468in}}%
\pgfpathlineto{\pgfqpoint{2.227735in}{0.400466in}}%
\pgfpathlineto{\pgfqpoint{2.342295in}{0.400466in}}%
\pgfpathlineto{\pgfqpoint{2.456855in}{0.400351in}}%
\pgfpathlineto{\pgfqpoint{2.571415in}{0.400351in}}%
\pgfpathlineto{\pgfqpoint{2.685976in}{0.400351in}}%
\pgfpathlineto{\pgfqpoint{2.800536in}{0.400272in}}%
\pgfpathlineto{\pgfqpoint{2.972376in}{0.400271in}}%
\pgfpathlineto{\pgfqpoint{2.972376in}{0.535077in}}%
\pgfpathlineto{\pgfqpoint{2.972376in}{0.535077in}}%
\pgfpathlineto{\pgfqpoint{2.800536in}{0.535076in}}%
\pgfpathlineto{\pgfqpoint{2.685976in}{0.534998in}}%
\pgfpathlineto{\pgfqpoint{2.571415in}{0.534997in}}%
\pgfpathlineto{\pgfqpoint{2.456855in}{0.534997in}}%
\pgfpathlineto{\pgfqpoint{2.342295in}{0.534883in}}%
\pgfpathlineto{\pgfqpoint{2.227735in}{0.534882in}}%
\pgfpathlineto{\pgfqpoint{2.113174in}{0.534880in}}%
\pgfpathlineto{\pgfqpoint{1.998614in}{0.534879in}}%
\pgfpathlineto{\pgfqpoint{1.884054in}{0.534850in}}%
\pgfpathlineto{\pgfqpoint{1.769494in}{0.534848in}}%
\pgfpathlineto{\pgfqpoint{1.654933in}{0.534763in}}%
\pgfpathlineto{\pgfqpoint{1.540373in}{0.534762in}}%
\pgfpathlineto{\pgfqpoint{1.425813in}{0.534759in}}%
\pgfpathlineto{\pgfqpoint{1.311253in}{0.534461in}}%
\pgfpathlineto{\pgfqpoint{1.196693in}{0.534389in}}%
\pgfpathlineto{\pgfqpoint{1.082132in}{0.533959in}}%
\pgfpathlineto{\pgfqpoint{0.967572in}{0.533930in}}%
\pgfpathlineto{\pgfqpoint{0.853012in}{0.533823in}}%
\pgfpathlineto{\pgfqpoint{0.681171in}{0.533556in}}%
\pgfpathclose%
\pgfusepath{stroke,fill}%
\end{pgfscope}%
\begin{pgfscope}%
\pgfpathrectangle{\pgfqpoint{0.506467in}{0.331635in}}{\pgfqpoint{2.583333in}{1.510000in}}%
\pgfusepath{clip}%
\pgfsetbuttcap%
\pgfsetroundjoin%
\pgfsetlinewidth{1.505625pt}%
\definecolor{currentstroke}{rgb}{0.000000,0.000000,0.000000}%
\pgfsetstrokecolor{currentstroke}%
\pgfsetdash{}{0pt}%
\pgfpathmoveto{\pgfqpoint{0.623891in}{0.467674in}}%
\pgfpathlineto{\pgfqpoint{0.623891in}{1.772999in}}%
\pgfusepath{stroke}%
\end{pgfscope}%
\begin{pgfscope}%
\pgfpathrectangle{\pgfqpoint{0.506467in}{0.331635in}}{\pgfqpoint{2.583333in}{1.510000in}}%
\pgfusepath{clip}%
\pgfsetbuttcap%
\pgfsetroundjoin%
\pgfsetlinewidth{1.505625pt}%
\definecolor{currentstroke}{rgb}{0.000000,0.000000,0.000000}%
\pgfsetstrokecolor{currentstroke}%
\pgfsetdash{}{0pt}%
\pgfpathmoveto{\pgfqpoint{0.738452in}{0.467674in}}%
\pgfpathlineto{\pgfqpoint{0.738452in}{0.550775in}}%
\pgfusepath{stroke}%
\end{pgfscope}%
\begin{pgfscope}%
\pgfpathrectangle{\pgfqpoint{0.506467in}{0.331635in}}{\pgfqpoint{2.583333in}{1.510000in}}%
\pgfusepath{clip}%
\pgfsetbuttcap%
\pgfsetroundjoin%
\pgfsetlinewidth{1.505625pt}%
\definecolor{currentstroke}{rgb}{0.000000,0.000000,0.000000}%
\pgfsetstrokecolor{currentstroke}%
\pgfsetdash{}{0pt}%
\pgfpathmoveto{\pgfqpoint{0.853012in}{0.467674in}}%
\pgfpathlineto{\pgfqpoint{0.853012in}{0.520600in}}%
\pgfusepath{stroke}%
\end{pgfscope}%
\begin{pgfscope}%
\pgfpathrectangle{\pgfqpoint{0.506467in}{0.331635in}}{\pgfqpoint{2.583333in}{1.510000in}}%
\pgfusepath{clip}%
\pgfsetbuttcap%
\pgfsetroundjoin%
\pgfsetlinewidth{1.505625pt}%
\definecolor{currentstroke}{rgb}{0.000000,0.000000,0.000000}%
\pgfsetstrokecolor{currentstroke}%
\pgfsetdash{}{0pt}%
\pgfpathmoveto{\pgfqpoint{0.967572in}{0.467674in}}%
\pgfpathlineto{\pgfqpoint{0.967572in}{0.495129in}}%
\pgfusepath{stroke}%
\end{pgfscope}%
\begin{pgfscope}%
\pgfpathrectangle{\pgfqpoint{0.506467in}{0.331635in}}{\pgfqpoint{2.583333in}{1.510000in}}%
\pgfusepath{clip}%
\pgfsetbuttcap%
\pgfsetroundjoin%
\pgfsetlinewidth{1.505625pt}%
\definecolor{currentstroke}{rgb}{0.000000,0.000000,0.000000}%
\pgfsetstrokecolor{currentstroke}%
\pgfsetdash{}{0pt}%
\pgfpathmoveto{\pgfqpoint{1.082132in}{0.467674in}}%
\pgfpathlineto{\pgfqpoint{1.082132in}{0.573568in}}%
\pgfusepath{stroke}%
\end{pgfscope}%
\begin{pgfscope}%
\pgfpathrectangle{\pgfqpoint{0.506467in}{0.331635in}}{\pgfqpoint{2.583333in}{1.510000in}}%
\pgfusepath{clip}%
\pgfsetbuttcap%
\pgfsetroundjoin%
\pgfsetlinewidth{1.505625pt}%
\definecolor{currentstroke}{rgb}{0.000000,0.000000,0.000000}%
\pgfsetstrokecolor{currentstroke}%
\pgfsetdash{}{0pt}%
\pgfpathmoveto{\pgfqpoint{1.196693in}{0.467674in}}%
\pgfpathlineto{\pgfqpoint{1.196693in}{0.511267in}}%
\pgfusepath{stroke}%
\end{pgfscope}%
\begin{pgfscope}%
\pgfpathrectangle{\pgfqpoint{0.506467in}{0.331635in}}{\pgfqpoint{2.583333in}{1.510000in}}%
\pgfusepath{clip}%
\pgfsetbuttcap%
\pgfsetroundjoin%
\pgfsetlinewidth{1.505625pt}%
\definecolor{currentstroke}{rgb}{0.000000,0.000000,0.000000}%
\pgfsetstrokecolor{currentstroke}%
\pgfsetdash{}{0pt}%
\pgfpathmoveto{\pgfqpoint{1.311253in}{0.467674in}}%
\pgfpathlineto{\pgfqpoint{1.311253in}{0.556033in}}%
\pgfusepath{stroke}%
\end{pgfscope}%
\begin{pgfscope}%
\pgfpathrectangle{\pgfqpoint{0.506467in}{0.331635in}}{\pgfqpoint{2.583333in}{1.510000in}}%
\pgfusepath{clip}%
\pgfsetbuttcap%
\pgfsetroundjoin%
\pgfsetlinewidth{1.505625pt}%
\definecolor{currentstroke}{rgb}{0.000000,0.000000,0.000000}%
\pgfsetstrokecolor{currentstroke}%
\pgfsetdash{}{0pt}%
\pgfpathmoveto{\pgfqpoint{1.425813in}{0.467674in}}%
\pgfpathlineto{\pgfqpoint{1.425813in}{0.477615in}}%
\pgfusepath{stroke}%
\end{pgfscope}%
\begin{pgfscope}%
\pgfpathrectangle{\pgfqpoint{0.506467in}{0.331635in}}{\pgfqpoint{2.583333in}{1.510000in}}%
\pgfusepath{clip}%
\pgfsetbuttcap%
\pgfsetroundjoin%
\pgfsetlinewidth{1.505625pt}%
\definecolor{currentstroke}{rgb}{0.000000,0.000000,0.000000}%
\pgfsetstrokecolor{currentstroke}%
\pgfsetdash{}{0pt}%
\pgfpathmoveto{\pgfqpoint{1.540373in}{0.467674in}}%
\pgfpathlineto{\pgfqpoint{1.540373in}{0.471763in}}%
\pgfusepath{stroke}%
\end{pgfscope}%
\begin{pgfscope}%
\pgfpathrectangle{\pgfqpoint{0.506467in}{0.331635in}}{\pgfqpoint{2.583333in}{1.510000in}}%
\pgfusepath{clip}%
\pgfsetbuttcap%
\pgfsetroundjoin%
\pgfsetlinewidth{1.505625pt}%
\definecolor{currentstroke}{rgb}{0.000000,0.000000,0.000000}%
\pgfsetstrokecolor{currentstroke}%
\pgfsetdash{}{0pt}%
\pgfpathmoveto{\pgfqpoint{1.654933in}{0.467674in}}%
\pgfpathlineto{\pgfqpoint{1.654933in}{0.514965in}}%
\pgfusepath{stroke}%
\end{pgfscope}%
\begin{pgfscope}%
\pgfpathrectangle{\pgfqpoint{0.506467in}{0.331635in}}{\pgfqpoint{2.583333in}{1.510000in}}%
\pgfusepath{clip}%
\pgfsetbuttcap%
\pgfsetroundjoin%
\pgfsetlinewidth{1.505625pt}%
\definecolor{currentstroke}{rgb}{0.000000,0.000000,0.000000}%
\pgfsetstrokecolor{currentstroke}%
\pgfsetdash{}{0pt}%
\pgfpathmoveto{\pgfqpoint{1.769494in}{0.467674in}}%
\pgfpathlineto{\pgfqpoint{1.769494in}{0.475836in}}%
\pgfusepath{stroke}%
\end{pgfscope}%
\begin{pgfscope}%
\pgfpathrectangle{\pgfqpoint{0.506467in}{0.331635in}}{\pgfqpoint{2.583333in}{1.510000in}}%
\pgfusepath{clip}%
\pgfsetbuttcap%
\pgfsetroundjoin%
\pgfsetlinewidth{1.505625pt}%
\definecolor{currentstroke}{rgb}{0.000000,0.000000,0.000000}%
\pgfsetstrokecolor{currentstroke}%
\pgfsetdash{}{0pt}%
\pgfpathmoveto{\pgfqpoint{1.884054in}{0.467674in}}%
\pgfpathlineto{\pgfqpoint{1.884054in}{0.495331in}}%
\pgfusepath{stroke}%
\end{pgfscope}%
\begin{pgfscope}%
\pgfpathrectangle{\pgfqpoint{0.506467in}{0.331635in}}{\pgfqpoint{2.583333in}{1.510000in}}%
\pgfusepath{clip}%
\pgfsetbuttcap%
\pgfsetroundjoin%
\pgfsetlinewidth{1.505625pt}%
\definecolor{currentstroke}{rgb}{0.000000,0.000000,0.000000}%
\pgfsetstrokecolor{currentstroke}%
\pgfsetdash{}{0pt}%
\pgfpathmoveto{\pgfqpoint{1.998614in}{0.467674in}}%
\pgfpathlineto{\pgfqpoint{1.998614in}{0.472993in}}%
\pgfusepath{stroke}%
\end{pgfscope}%
\begin{pgfscope}%
\pgfpathrectangle{\pgfqpoint{0.506467in}{0.331635in}}{\pgfqpoint{2.583333in}{1.510000in}}%
\pgfusepath{clip}%
\pgfsetbuttcap%
\pgfsetroundjoin%
\pgfsetlinewidth{1.505625pt}%
\definecolor{currentstroke}{rgb}{0.000000,0.000000,0.000000}%
\pgfsetstrokecolor{currentstroke}%
\pgfsetdash{}{0pt}%
\pgfpathmoveto{\pgfqpoint{2.113174in}{0.467674in}}%
\pgfpathlineto{\pgfqpoint{2.113174in}{0.460501in}}%
\pgfusepath{stroke}%
\end{pgfscope}%
\begin{pgfscope}%
\pgfpathrectangle{\pgfqpoint{0.506467in}{0.331635in}}{\pgfqpoint{2.583333in}{1.510000in}}%
\pgfusepath{clip}%
\pgfsetbuttcap%
\pgfsetroundjoin%
\pgfsetlinewidth{1.505625pt}%
\definecolor{currentstroke}{rgb}{0.000000,0.000000,0.000000}%
\pgfsetstrokecolor{currentstroke}%
\pgfsetdash{}{0pt}%
\pgfpathmoveto{\pgfqpoint{2.227735in}{0.467674in}}%
\pgfpathlineto{\pgfqpoint{2.227735in}{0.471877in}}%
\pgfusepath{stroke}%
\end{pgfscope}%
\begin{pgfscope}%
\pgfpathrectangle{\pgfqpoint{0.506467in}{0.331635in}}{\pgfqpoint{2.583333in}{1.510000in}}%
\pgfusepath{clip}%
\pgfsetbuttcap%
\pgfsetroundjoin%
\pgfsetlinewidth{1.505625pt}%
\definecolor{currentstroke}{rgb}{0.000000,0.000000,0.000000}%
\pgfsetstrokecolor{currentstroke}%
\pgfsetdash{}{0pt}%
\pgfpathmoveto{\pgfqpoint{2.342295in}{0.467674in}}%
\pgfpathlineto{\pgfqpoint{2.342295in}{0.522638in}}%
\pgfusepath{stroke}%
\end{pgfscope}%
\begin{pgfscope}%
\pgfpathrectangle{\pgfqpoint{0.506467in}{0.331635in}}{\pgfqpoint{2.583333in}{1.510000in}}%
\pgfusepath{clip}%
\pgfsetbuttcap%
\pgfsetroundjoin%
\pgfsetlinewidth{1.505625pt}%
\definecolor{currentstroke}{rgb}{0.000000,0.000000,0.000000}%
\pgfsetstrokecolor{currentstroke}%
\pgfsetdash{}{0pt}%
\pgfpathmoveto{\pgfqpoint{2.456855in}{0.467674in}}%
\pgfpathlineto{\pgfqpoint{2.456855in}{0.467509in}}%
\pgfusepath{stroke}%
\end{pgfscope}%
\begin{pgfscope}%
\pgfpathrectangle{\pgfqpoint{0.506467in}{0.331635in}}{\pgfqpoint{2.583333in}{1.510000in}}%
\pgfusepath{clip}%
\pgfsetbuttcap%
\pgfsetroundjoin%
\pgfsetlinewidth{1.505625pt}%
\definecolor{currentstroke}{rgb}{0.000000,0.000000,0.000000}%
\pgfsetstrokecolor{currentstroke}%
\pgfsetdash{}{0pt}%
\pgfpathmoveto{\pgfqpoint{2.571415in}{0.467674in}}%
\pgfpathlineto{\pgfqpoint{2.571415in}{0.465208in}}%
\pgfusepath{stroke}%
\end{pgfscope}%
\begin{pgfscope}%
\pgfpathrectangle{\pgfqpoint{0.506467in}{0.331635in}}{\pgfqpoint{2.583333in}{1.510000in}}%
\pgfusepath{clip}%
\pgfsetbuttcap%
\pgfsetroundjoin%
\pgfsetlinewidth{1.505625pt}%
\definecolor{currentstroke}{rgb}{0.000000,0.000000,0.000000}%
\pgfsetstrokecolor{currentstroke}%
\pgfsetdash{}{0pt}%
\pgfpathmoveto{\pgfqpoint{2.685976in}{0.467674in}}%
\pgfpathlineto{\pgfqpoint{2.685976in}{0.513217in}}%
\pgfusepath{stroke}%
\end{pgfscope}%
\begin{pgfscope}%
\pgfpathrectangle{\pgfqpoint{0.506467in}{0.331635in}}{\pgfqpoint{2.583333in}{1.510000in}}%
\pgfusepath{clip}%
\pgfsetbuttcap%
\pgfsetroundjoin%
\pgfsetlinewidth{1.505625pt}%
\definecolor{currentstroke}{rgb}{0.000000,0.000000,0.000000}%
\pgfsetstrokecolor{currentstroke}%
\pgfsetdash{}{0pt}%
\pgfpathmoveto{\pgfqpoint{2.800536in}{0.467674in}}%
\pgfpathlineto{\pgfqpoint{2.800536in}{0.473129in}}%
\pgfusepath{stroke}%
\end{pgfscope}%
\begin{pgfscope}%
\pgfpathrectangle{\pgfqpoint{0.506467in}{0.331635in}}{\pgfqpoint{2.583333in}{1.510000in}}%
\pgfusepath{clip}%
\pgfsetbuttcap%
\pgfsetroundjoin%
\pgfsetlinewidth{1.505625pt}%
\definecolor{currentstroke}{rgb}{0.000000,0.000000,0.000000}%
\pgfsetstrokecolor{currentstroke}%
\pgfsetdash{}{0pt}%
\pgfpathmoveto{\pgfqpoint{2.915096in}{0.467674in}}%
\pgfpathlineto{\pgfqpoint{2.915096in}{0.457582in}}%
\pgfusepath{stroke}%
\end{pgfscope}%
\begin{pgfscope}%
\pgfpathrectangle{\pgfqpoint{0.506467in}{0.331635in}}{\pgfqpoint{2.583333in}{1.510000in}}%
\pgfusepath{clip}%
\pgfsetroundcap%
\pgfsetroundjoin%
\pgfsetlinewidth{1.505625pt}%
\definecolor{currentstroke}{rgb}{0.737255,0.741176,0.133333}%
\pgfsetstrokecolor{currentstroke}%
\pgfsetdash{}{0pt}%
\pgfpathmoveto{\pgfqpoint{0.506467in}{0.467674in}}%
\pgfpathlineto{\pgfqpoint{3.089800in}{0.467674in}}%
\pgfusepath{stroke}%
\end{pgfscope}%
\begin{pgfscope}%
\pgfpathrectangle{\pgfqpoint{0.506467in}{0.331635in}}{\pgfqpoint{2.583333in}{1.510000in}}%
\pgfusepath{clip}%
\pgfsetbuttcap%
\pgfsetroundjoin%
\definecolor{currentfill}{rgb}{0.737255,0.741176,0.133333}%
\pgfsetfillcolor{currentfill}%
\pgfsetlinewidth{1.003750pt}%
\definecolor{currentstroke}{rgb}{0.737255,0.741176,0.133333}%
\pgfsetstrokecolor{currentstroke}%
\pgfsetdash{}{0pt}%
\pgfsys@defobject{currentmarker}{\pgfqpoint{-0.034722in}{-0.034722in}}{\pgfqpoint{0.034722in}{0.034722in}}{%
\pgfpathmoveto{\pgfqpoint{0.000000in}{-0.034722in}}%
\pgfpathcurveto{\pgfqpoint{0.009208in}{-0.034722in}}{\pgfqpoint{0.018041in}{-0.031064in}}{\pgfqpoint{0.024552in}{-0.024552in}}%
\pgfpathcurveto{\pgfqpoint{0.031064in}{-0.018041in}}{\pgfqpoint{0.034722in}{-0.009208in}}{\pgfqpoint{0.034722in}{0.000000in}}%
\pgfpathcurveto{\pgfqpoint{0.034722in}{0.009208in}}{\pgfqpoint{0.031064in}{0.018041in}}{\pgfqpoint{0.024552in}{0.024552in}}%
\pgfpathcurveto{\pgfqpoint{0.018041in}{0.031064in}}{\pgfqpoint{0.009208in}{0.034722in}}{\pgfqpoint{0.000000in}{0.034722in}}%
\pgfpathcurveto{\pgfqpoint{-0.009208in}{0.034722in}}{\pgfqpoint{-0.018041in}{0.031064in}}{\pgfqpoint{-0.024552in}{0.024552in}}%
\pgfpathcurveto{\pgfqpoint{-0.031064in}{0.018041in}}{\pgfqpoint{-0.034722in}{0.009208in}}{\pgfqpoint{-0.034722in}{0.000000in}}%
\pgfpathcurveto{\pgfqpoint{-0.034722in}{-0.009208in}}{\pgfqpoint{-0.031064in}{-0.018041in}}{\pgfqpoint{-0.024552in}{-0.024552in}}%
\pgfpathcurveto{\pgfqpoint{-0.018041in}{-0.031064in}}{\pgfqpoint{-0.009208in}{-0.034722in}}{\pgfqpoint{0.000000in}{-0.034722in}}%
\pgfpathclose%
\pgfusepath{stroke,fill}%
}%
\begin{pgfscope}%
\pgfsys@transformshift{0.623891in}{1.772999in}%
\pgfsys@useobject{currentmarker}{}%
\end{pgfscope}%
\begin{pgfscope}%
\pgfsys@transformshift{0.738452in}{0.550775in}%
\pgfsys@useobject{currentmarker}{}%
\end{pgfscope}%
\begin{pgfscope}%
\pgfsys@transformshift{0.853012in}{0.520600in}%
\pgfsys@useobject{currentmarker}{}%
\end{pgfscope}%
\begin{pgfscope}%
\pgfsys@transformshift{0.967572in}{0.495129in}%
\pgfsys@useobject{currentmarker}{}%
\end{pgfscope}%
\begin{pgfscope}%
\pgfsys@transformshift{1.082132in}{0.573568in}%
\pgfsys@useobject{currentmarker}{}%
\end{pgfscope}%
\begin{pgfscope}%
\pgfsys@transformshift{1.196693in}{0.511267in}%
\pgfsys@useobject{currentmarker}{}%
\end{pgfscope}%
\begin{pgfscope}%
\pgfsys@transformshift{1.311253in}{0.556033in}%
\pgfsys@useobject{currentmarker}{}%
\end{pgfscope}%
\begin{pgfscope}%
\pgfsys@transformshift{1.425813in}{0.477615in}%
\pgfsys@useobject{currentmarker}{}%
\end{pgfscope}%
\begin{pgfscope}%
\pgfsys@transformshift{1.540373in}{0.471763in}%
\pgfsys@useobject{currentmarker}{}%
\end{pgfscope}%
\begin{pgfscope}%
\pgfsys@transformshift{1.654933in}{0.514965in}%
\pgfsys@useobject{currentmarker}{}%
\end{pgfscope}%
\begin{pgfscope}%
\pgfsys@transformshift{1.769494in}{0.475836in}%
\pgfsys@useobject{currentmarker}{}%
\end{pgfscope}%
\begin{pgfscope}%
\pgfsys@transformshift{1.884054in}{0.495331in}%
\pgfsys@useobject{currentmarker}{}%
\end{pgfscope}%
\begin{pgfscope}%
\pgfsys@transformshift{1.998614in}{0.472993in}%
\pgfsys@useobject{currentmarker}{}%
\end{pgfscope}%
\begin{pgfscope}%
\pgfsys@transformshift{2.113174in}{0.460501in}%
\pgfsys@useobject{currentmarker}{}%
\end{pgfscope}%
\begin{pgfscope}%
\pgfsys@transformshift{2.227735in}{0.471877in}%
\pgfsys@useobject{currentmarker}{}%
\end{pgfscope}%
\begin{pgfscope}%
\pgfsys@transformshift{2.342295in}{0.522638in}%
\pgfsys@useobject{currentmarker}{}%
\end{pgfscope}%
\begin{pgfscope}%
\pgfsys@transformshift{2.456855in}{0.467509in}%
\pgfsys@useobject{currentmarker}{}%
\end{pgfscope}%
\begin{pgfscope}%
\pgfsys@transformshift{2.571415in}{0.465208in}%
\pgfsys@useobject{currentmarker}{}%
\end{pgfscope}%
\begin{pgfscope}%
\pgfsys@transformshift{2.685976in}{0.513217in}%
\pgfsys@useobject{currentmarker}{}%
\end{pgfscope}%
\begin{pgfscope}%
\pgfsys@transformshift{2.800536in}{0.473129in}%
\pgfsys@useobject{currentmarker}{}%
\end{pgfscope}%
\begin{pgfscope}%
\pgfsys@transformshift{2.915096in}{0.457582in}%
\pgfsys@useobject{currentmarker}{}%
\end{pgfscope}%
\end{pgfscope}%
\begin{pgfscope}%
\pgfsetrectcap%
\pgfsetmiterjoin%
\pgfsetlinewidth{0.803000pt}%
\definecolor{currentstroke}{rgb}{1.000000,1.000000,1.000000}%
\pgfsetstrokecolor{currentstroke}%
\pgfsetdash{}{0pt}%
\pgfpathmoveto{\pgfqpoint{0.506467in}{0.331635in}}%
\pgfpathlineto{\pgfqpoint{0.506467in}{1.841635in}}%
\pgfusepath{stroke}%
\end{pgfscope}%
\begin{pgfscope}%
\pgfsetrectcap%
\pgfsetmiterjoin%
\pgfsetlinewidth{0.803000pt}%
\definecolor{currentstroke}{rgb}{1.000000,1.000000,1.000000}%
\pgfsetstrokecolor{currentstroke}%
\pgfsetdash{}{0pt}%
\pgfpathmoveto{\pgfqpoint{3.089800in}{0.331635in}}%
\pgfpathlineto{\pgfqpoint{3.089800in}{1.841635in}}%
\pgfusepath{stroke}%
\end{pgfscope}%
\begin{pgfscope}%
\pgfsetrectcap%
\pgfsetmiterjoin%
\pgfsetlinewidth{0.803000pt}%
\definecolor{currentstroke}{rgb}{1.000000,1.000000,1.000000}%
\pgfsetstrokecolor{currentstroke}%
\pgfsetdash{}{0pt}%
\pgfpathmoveto{\pgfqpoint{0.506467in}{0.331635in}}%
\pgfpathlineto{\pgfqpoint{3.089800in}{0.331635in}}%
\pgfusepath{stroke}%
\end{pgfscope}%
\begin{pgfscope}%
\pgfsetrectcap%
\pgfsetmiterjoin%
\pgfsetlinewidth{0.803000pt}%
\definecolor{currentstroke}{rgb}{1.000000,1.000000,1.000000}%
\pgfsetstrokecolor{currentstroke}%
\pgfsetdash{}{0pt}%
\pgfpathmoveto{\pgfqpoint{0.506467in}{1.841635in}}%
\pgfpathlineto{\pgfqpoint{3.089800in}{1.841635in}}%
\pgfusepath{stroke}%
\end{pgfscope}%
\begin{pgfscope}%
\definecolor{textcolor}{rgb}{0.150000,0.150000,0.150000}%
\pgfsetstrokecolor{textcolor}%
\pgfsetfillcolor{textcolor}%
\pgftext[x=1.798134in,y=1.924968in,,base]{\color{textcolor}\rmfamily\fontsize{12.000000}{14.400000}\selectfont Autocorrelation V}%
\end{pgfscope}%
\begin{pgfscope}%
\pgfsetbuttcap%
\pgfsetmiterjoin%
\definecolor{currentfill}{rgb}{0.917647,0.917647,0.949020}%
\pgfsetfillcolor{currentfill}%
\pgfsetlinewidth{0.000000pt}%
\definecolor{currentstroke}{rgb}{0.000000,0.000000,0.000000}%
\pgfsetstrokecolor{currentstroke}%
\pgfsetstrokeopacity{0.000000}%
\pgfsetdash{}{0pt}%
\pgfpathmoveto{\pgfqpoint{4.123134in}{0.331635in}}%
\pgfpathlineto{\pgfqpoint{6.706467in}{0.331635in}}%
\pgfpathlineto{\pgfqpoint{6.706467in}{1.841635in}}%
\pgfpathlineto{\pgfqpoint{4.123134in}{1.841635in}}%
\pgfpathclose%
\pgfusepath{fill}%
\end{pgfscope}%
\begin{pgfscope}%
\pgfpathrectangle{\pgfqpoint{4.123134in}{0.331635in}}{\pgfqpoint{2.583333in}{1.510000in}}%
\pgfusepath{clip}%
\pgfsetroundcap%
\pgfsetroundjoin%
\pgfsetlinewidth{0.803000pt}%
\definecolor{currentstroke}{rgb}{1.000000,1.000000,1.000000}%
\pgfsetstrokecolor{currentstroke}%
\pgfsetdash{}{0pt}%
\pgfpathmoveto{\pgfqpoint{4.240558in}{0.331635in}}%
\pgfpathlineto{\pgfqpoint{4.240558in}{1.841635in}}%
\pgfusepath{stroke}%
\end{pgfscope}%
\begin{pgfscope}%
\definecolor{textcolor}{rgb}{0.150000,0.150000,0.150000}%
\pgfsetstrokecolor{textcolor}%
\pgfsetfillcolor{textcolor}%
\pgftext[x=4.240558in,y=0.234413in,,top]{\color{textcolor}\rmfamily\fontsize{10.000000}{12.000000}\selectfont 0}%
\end{pgfscope}%
\begin{pgfscope}%
\pgfpathrectangle{\pgfqpoint{4.123134in}{0.331635in}}{\pgfqpoint{2.583333in}{1.510000in}}%
\pgfusepath{clip}%
\pgfsetroundcap%
\pgfsetroundjoin%
\pgfsetlinewidth{0.803000pt}%
\definecolor{currentstroke}{rgb}{1.000000,1.000000,1.000000}%
\pgfsetstrokecolor{currentstroke}%
\pgfsetdash{}{0pt}%
\pgfpathmoveto{\pgfqpoint{4.813359in}{0.331635in}}%
\pgfpathlineto{\pgfqpoint{4.813359in}{1.841635in}}%
\pgfusepath{stroke}%
\end{pgfscope}%
\begin{pgfscope}%
\definecolor{textcolor}{rgb}{0.150000,0.150000,0.150000}%
\pgfsetstrokecolor{textcolor}%
\pgfsetfillcolor{textcolor}%
\pgftext[x=4.813359in,y=0.234413in,,top]{\color{textcolor}\rmfamily\fontsize{10.000000}{12.000000}\selectfont 5}%
\end{pgfscope}%
\begin{pgfscope}%
\pgfpathrectangle{\pgfqpoint{4.123134in}{0.331635in}}{\pgfqpoint{2.583333in}{1.510000in}}%
\pgfusepath{clip}%
\pgfsetroundcap%
\pgfsetroundjoin%
\pgfsetlinewidth{0.803000pt}%
\definecolor{currentstroke}{rgb}{1.000000,1.000000,1.000000}%
\pgfsetstrokecolor{currentstroke}%
\pgfsetdash{}{0pt}%
\pgfpathmoveto{\pgfqpoint{5.386160in}{0.331635in}}%
\pgfpathlineto{\pgfqpoint{5.386160in}{1.841635in}}%
\pgfusepath{stroke}%
\end{pgfscope}%
\begin{pgfscope}%
\definecolor{textcolor}{rgb}{0.150000,0.150000,0.150000}%
\pgfsetstrokecolor{textcolor}%
\pgfsetfillcolor{textcolor}%
\pgftext[x=5.386160in,y=0.234413in,,top]{\color{textcolor}\rmfamily\fontsize{10.000000}{12.000000}\selectfont 10}%
\end{pgfscope}%
\begin{pgfscope}%
\pgfpathrectangle{\pgfqpoint{4.123134in}{0.331635in}}{\pgfqpoint{2.583333in}{1.510000in}}%
\pgfusepath{clip}%
\pgfsetroundcap%
\pgfsetroundjoin%
\pgfsetlinewidth{0.803000pt}%
\definecolor{currentstroke}{rgb}{1.000000,1.000000,1.000000}%
\pgfsetstrokecolor{currentstroke}%
\pgfsetdash{}{0pt}%
\pgfpathmoveto{\pgfqpoint{5.958962in}{0.331635in}}%
\pgfpathlineto{\pgfqpoint{5.958962in}{1.841635in}}%
\pgfusepath{stroke}%
\end{pgfscope}%
\begin{pgfscope}%
\definecolor{textcolor}{rgb}{0.150000,0.150000,0.150000}%
\pgfsetstrokecolor{textcolor}%
\pgfsetfillcolor{textcolor}%
\pgftext[x=5.958962in,y=0.234413in,,top]{\color{textcolor}\rmfamily\fontsize{10.000000}{12.000000}\selectfont 15}%
\end{pgfscope}%
\begin{pgfscope}%
\pgfpathrectangle{\pgfqpoint{4.123134in}{0.331635in}}{\pgfqpoint{2.583333in}{1.510000in}}%
\pgfusepath{clip}%
\pgfsetroundcap%
\pgfsetroundjoin%
\pgfsetlinewidth{0.803000pt}%
\definecolor{currentstroke}{rgb}{1.000000,1.000000,1.000000}%
\pgfsetstrokecolor{currentstroke}%
\pgfsetdash{}{0pt}%
\pgfpathmoveto{\pgfqpoint{6.531763in}{0.331635in}}%
\pgfpathlineto{\pgfqpoint{6.531763in}{1.841635in}}%
\pgfusepath{stroke}%
\end{pgfscope}%
\begin{pgfscope}%
\definecolor{textcolor}{rgb}{0.150000,0.150000,0.150000}%
\pgfsetstrokecolor{textcolor}%
\pgfsetfillcolor{textcolor}%
\pgftext[x=6.531763in,y=0.234413in,,top]{\color{textcolor}\rmfamily\fontsize{10.000000}{12.000000}\selectfont 20}%
\end{pgfscope}%
\begin{pgfscope}%
\pgfpathrectangle{\pgfqpoint{4.123134in}{0.331635in}}{\pgfqpoint{2.583333in}{1.510000in}}%
\pgfusepath{clip}%
\pgfsetroundcap%
\pgfsetroundjoin%
\pgfsetlinewidth{0.803000pt}%
\definecolor{currentstroke}{rgb}{1.000000,1.000000,1.000000}%
\pgfsetstrokecolor{currentstroke}%
\pgfsetdash{}{0pt}%
\pgfpathmoveto{\pgfqpoint{4.123134in}{0.466226in}}%
\pgfpathlineto{\pgfqpoint{6.706467in}{0.466226in}}%
\pgfusepath{stroke}%
\end{pgfscope}%
\begin{pgfscope}%
\definecolor{textcolor}{rgb}{0.150000,0.150000,0.150000}%
\pgfsetstrokecolor{textcolor}%
\pgfsetfillcolor{textcolor}%
\pgftext[x=3.716667in,y=0.413465in,left,base]{\color{textcolor}\rmfamily\fontsize{10.000000}{12.000000}\selectfont 0.00}%
\end{pgfscope}%
\begin{pgfscope}%
\pgfpathrectangle{\pgfqpoint{4.123134in}{0.331635in}}{\pgfqpoint{2.583333in}{1.510000in}}%
\pgfusepath{clip}%
\pgfsetroundcap%
\pgfsetroundjoin%
\pgfsetlinewidth{0.803000pt}%
\definecolor{currentstroke}{rgb}{1.000000,1.000000,1.000000}%
\pgfsetstrokecolor{currentstroke}%
\pgfsetdash{}{0pt}%
\pgfpathmoveto{\pgfqpoint{4.123134in}{0.792919in}}%
\pgfpathlineto{\pgfqpoint{6.706467in}{0.792919in}}%
\pgfusepath{stroke}%
\end{pgfscope}%
\begin{pgfscope}%
\definecolor{textcolor}{rgb}{0.150000,0.150000,0.150000}%
\pgfsetstrokecolor{textcolor}%
\pgfsetfillcolor{textcolor}%
\pgftext[x=3.716667in,y=0.740158in,left,base]{\color{textcolor}\rmfamily\fontsize{10.000000}{12.000000}\selectfont 0.25}%
\end{pgfscope}%
\begin{pgfscope}%
\pgfpathrectangle{\pgfqpoint{4.123134in}{0.331635in}}{\pgfqpoint{2.583333in}{1.510000in}}%
\pgfusepath{clip}%
\pgfsetroundcap%
\pgfsetroundjoin%
\pgfsetlinewidth{0.803000pt}%
\definecolor{currentstroke}{rgb}{1.000000,1.000000,1.000000}%
\pgfsetstrokecolor{currentstroke}%
\pgfsetdash{}{0pt}%
\pgfpathmoveto{\pgfqpoint{4.123134in}{1.119612in}}%
\pgfpathlineto{\pgfqpoint{6.706467in}{1.119612in}}%
\pgfusepath{stroke}%
\end{pgfscope}%
\begin{pgfscope}%
\definecolor{textcolor}{rgb}{0.150000,0.150000,0.150000}%
\pgfsetstrokecolor{textcolor}%
\pgfsetfillcolor{textcolor}%
\pgftext[x=3.716667in,y=1.066851in,left,base]{\color{textcolor}\rmfamily\fontsize{10.000000}{12.000000}\selectfont 0.50}%
\end{pgfscope}%
\begin{pgfscope}%
\pgfpathrectangle{\pgfqpoint{4.123134in}{0.331635in}}{\pgfqpoint{2.583333in}{1.510000in}}%
\pgfusepath{clip}%
\pgfsetroundcap%
\pgfsetroundjoin%
\pgfsetlinewidth{0.803000pt}%
\definecolor{currentstroke}{rgb}{1.000000,1.000000,1.000000}%
\pgfsetstrokecolor{currentstroke}%
\pgfsetdash{}{0pt}%
\pgfpathmoveto{\pgfqpoint{4.123134in}{1.446306in}}%
\pgfpathlineto{\pgfqpoint{6.706467in}{1.446306in}}%
\pgfusepath{stroke}%
\end{pgfscope}%
\begin{pgfscope}%
\definecolor{textcolor}{rgb}{0.150000,0.150000,0.150000}%
\pgfsetstrokecolor{textcolor}%
\pgfsetfillcolor{textcolor}%
\pgftext[x=3.716667in,y=1.393544in,left,base]{\color{textcolor}\rmfamily\fontsize{10.000000}{12.000000}\selectfont 0.75}%
\end{pgfscope}%
\begin{pgfscope}%
\pgfpathrectangle{\pgfqpoint{4.123134in}{0.331635in}}{\pgfqpoint{2.583333in}{1.510000in}}%
\pgfusepath{clip}%
\pgfsetroundcap%
\pgfsetroundjoin%
\pgfsetlinewidth{0.803000pt}%
\definecolor{currentstroke}{rgb}{1.000000,1.000000,1.000000}%
\pgfsetstrokecolor{currentstroke}%
\pgfsetdash{}{0pt}%
\pgfpathmoveto{\pgfqpoint{4.123134in}{1.772999in}}%
\pgfpathlineto{\pgfqpoint{6.706467in}{1.772999in}}%
\pgfusepath{stroke}%
\end{pgfscope}%
\begin{pgfscope}%
\definecolor{textcolor}{rgb}{0.150000,0.150000,0.150000}%
\pgfsetstrokecolor{textcolor}%
\pgfsetfillcolor{textcolor}%
\pgftext[x=3.716667in,y=1.720237in,left,base]{\color{textcolor}\rmfamily\fontsize{10.000000}{12.000000}\selectfont 1.00}%
\end{pgfscope}%
\begin{pgfscope}%
\pgfpathrectangle{\pgfqpoint{4.123134in}{0.331635in}}{\pgfqpoint{2.583333in}{1.510000in}}%
\pgfusepath{clip}%
\pgfsetbuttcap%
\pgfsetroundjoin%
\definecolor{currentfill}{rgb}{0.121569,0.466667,0.705882}%
\pgfsetfillcolor{currentfill}%
\pgfsetfillopacity{0.250000}%
\pgfsetlinewidth{1.003750pt}%
\definecolor{currentstroke}{rgb}{1.000000,1.000000,1.000000}%
\pgfsetstrokecolor{currentstroke}%
\pgfsetstrokeopacity{0.250000}%
\pgfsetdash{}{0pt}%
\pgfpathmoveto{\pgfqpoint{4.297838in}{0.532181in}}%
\pgfpathlineto{\pgfqpoint{4.297838in}{0.400271in}}%
\pgfpathlineto{\pgfqpoint{4.469678in}{0.400271in}}%
\pgfpathlineto{\pgfqpoint{4.584239in}{0.400271in}}%
\pgfpathlineto{\pgfqpoint{4.698799in}{0.400271in}}%
\pgfpathlineto{\pgfqpoint{4.813359in}{0.400271in}}%
\pgfpathlineto{\pgfqpoint{4.927919in}{0.400271in}}%
\pgfpathlineto{\pgfqpoint{5.042480in}{0.400271in}}%
\pgfpathlineto{\pgfqpoint{5.157040in}{0.400271in}}%
\pgfpathlineto{\pgfqpoint{5.271600in}{0.400271in}}%
\pgfpathlineto{\pgfqpoint{5.386160in}{0.400271in}}%
\pgfpathlineto{\pgfqpoint{5.500721in}{0.400271in}}%
\pgfpathlineto{\pgfqpoint{5.615281in}{0.400271in}}%
\pgfpathlineto{\pgfqpoint{5.729841in}{0.400271in}}%
\pgfpathlineto{\pgfqpoint{5.844401in}{0.400271in}}%
\pgfpathlineto{\pgfqpoint{5.958962in}{0.400271in}}%
\pgfpathlineto{\pgfqpoint{6.073522in}{0.400271in}}%
\pgfpathlineto{\pgfqpoint{6.188082in}{0.400271in}}%
\pgfpathlineto{\pgfqpoint{6.302642in}{0.400271in}}%
\pgfpathlineto{\pgfqpoint{6.417202in}{0.400271in}}%
\pgfpathlineto{\pgfqpoint{6.589043in}{0.400271in}}%
\pgfpathlineto{\pgfqpoint{6.589043in}{0.532181in}}%
\pgfpathlineto{\pgfqpoint{6.589043in}{0.532181in}}%
\pgfpathlineto{\pgfqpoint{6.417202in}{0.532181in}}%
\pgfpathlineto{\pgfqpoint{6.302642in}{0.532181in}}%
\pgfpathlineto{\pgfqpoint{6.188082in}{0.532181in}}%
\pgfpathlineto{\pgfqpoint{6.073522in}{0.532181in}}%
\pgfpathlineto{\pgfqpoint{5.958962in}{0.532181in}}%
\pgfpathlineto{\pgfqpoint{5.844401in}{0.532181in}}%
\pgfpathlineto{\pgfqpoint{5.729841in}{0.532181in}}%
\pgfpathlineto{\pgfqpoint{5.615281in}{0.532181in}}%
\pgfpathlineto{\pgfqpoint{5.500721in}{0.532181in}}%
\pgfpathlineto{\pgfqpoint{5.386160in}{0.532181in}}%
\pgfpathlineto{\pgfqpoint{5.271600in}{0.532181in}}%
\pgfpathlineto{\pgfqpoint{5.157040in}{0.532181in}}%
\pgfpathlineto{\pgfqpoint{5.042480in}{0.532181in}}%
\pgfpathlineto{\pgfqpoint{4.927919in}{0.532181in}}%
\pgfpathlineto{\pgfqpoint{4.813359in}{0.532181in}}%
\pgfpathlineto{\pgfqpoint{4.698799in}{0.532181in}}%
\pgfpathlineto{\pgfqpoint{4.584239in}{0.532181in}}%
\pgfpathlineto{\pgfqpoint{4.469678in}{0.532181in}}%
\pgfpathlineto{\pgfqpoint{4.297838in}{0.532181in}}%
\pgfpathclose%
\pgfusepath{stroke,fill}%
\end{pgfscope}%
\begin{pgfscope}%
\pgfpathrectangle{\pgfqpoint{4.123134in}{0.331635in}}{\pgfqpoint{2.583333in}{1.510000in}}%
\pgfusepath{clip}%
\pgfsetbuttcap%
\pgfsetroundjoin%
\pgfsetlinewidth{1.505625pt}%
\definecolor{currentstroke}{rgb}{0.000000,0.000000,0.000000}%
\pgfsetstrokecolor{currentstroke}%
\pgfsetdash{}{0pt}%
\pgfpathmoveto{\pgfqpoint{4.240558in}{0.466226in}}%
\pgfpathlineto{\pgfqpoint{4.240558in}{1.772999in}}%
\pgfusepath{stroke}%
\end{pgfscope}%
\begin{pgfscope}%
\pgfpathrectangle{\pgfqpoint{4.123134in}{0.331635in}}{\pgfqpoint{2.583333in}{1.510000in}}%
\pgfusepath{clip}%
\pgfsetbuttcap%
\pgfsetroundjoin%
\pgfsetlinewidth{1.505625pt}%
\definecolor{currentstroke}{rgb}{0.000000,0.000000,0.000000}%
\pgfsetstrokecolor{currentstroke}%
\pgfsetdash{}{0pt}%
\pgfpathmoveto{\pgfqpoint{4.355118in}{0.466226in}}%
\pgfpathlineto{\pgfqpoint{4.355118in}{0.549474in}}%
\pgfusepath{stroke}%
\end{pgfscope}%
\begin{pgfscope}%
\pgfpathrectangle{\pgfqpoint{4.123134in}{0.331635in}}{\pgfqpoint{2.583333in}{1.510000in}}%
\pgfusepath{clip}%
\pgfsetbuttcap%
\pgfsetroundjoin%
\pgfsetlinewidth{1.505625pt}%
\definecolor{currentstroke}{rgb}{0.000000,0.000000,0.000000}%
\pgfsetstrokecolor{currentstroke}%
\pgfsetdash{}{0pt}%
\pgfpathmoveto{\pgfqpoint{4.469678in}{0.466226in}}%
\pgfpathlineto{\pgfqpoint{4.469678in}{0.514172in}}%
\pgfusepath{stroke}%
\end{pgfscope}%
\begin{pgfscope}%
\pgfpathrectangle{\pgfqpoint{4.123134in}{0.331635in}}{\pgfqpoint{2.583333in}{1.510000in}}%
\pgfusepath{clip}%
\pgfsetbuttcap%
\pgfsetroundjoin%
\pgfsetlinewidth{1.505625pt}%
\definecolor{currentstroke}{rgb}{0.000000,0.000000,0.000000}%
\pgfsetstrokecolor{currentstroke}%
\pgfsetdash{}{0pt}%
\pgfpathmoveto{\pgfqpoint{4.584239in}{0.466226in}}%
\pgfpathlineto{\pgfqpoint{4.584239in}{0.487572in}}%
\pgfusepath{stroke}%
\end{pgfscope}%
\begin{pgfscope}%
\pgfpathrectangle{\pgfqpoint{4.123134in}{0.331635in}}{\pgfqpoint{2.583333in}{1.510000in}}%
\pgfusepath{clip}%
\pgfsetbuttcap%
\pgfsetroundjoin%
\pgfsetlinewidth{1.505625pt}%
\definecolor{currentstroke}{rgb}{0.000000,0.000000,0.000000}%
\pgfsetstrokecolor{currentstroke}%
\pgfsetdash{}{0pt}%
\pgfpathmoveto{\pgfqpoint{4.698799in}{0.466226in}}%
\pgfpathlineto{\pgfqpoint{4.698799in}{0.568170in}}%
\pgfusepath{stroke}%
\end{pgfscope}%
\begin{pgfscope}%
\pgfpathrectangle{\pgfqpoint{4.123134in}{0.331635in}}{\pgfqpoint{2.583333in}{1.510000in}}%
\pgfusepath{clip}%
\pgfsetbuttcap%
\pgfsetroundjoin%
\pgfsetlinewidth{1.505625pt}%
\definecolor{currentstroke}{rgb}{0.000000,0.000000,0.000000}%
\pgfsetstrokecolor{currentstroke}%
\pgfsetdash{}{0pt}%
\pgfpathmoveto{\pgfqpoint{4.813359in}{0.466226in}}%
\pgfpathlineto{\pgfqpoint{4.813359in}{0.496022in}}%
\pgfusepath{stroke}%
\end{pgfscope}%
\begin{pgfscope}%
\pgfpathrectangle{\pgfqpoint{4.123134in}{0.331635in}}{\pgfqpoint{2.583333in}{1.510000in}}%
\pgfusepath{clip}%
\pgfsetbuttcap%
\pgfsetroundjoin%
\pgfsetlinewidth{1.505625pt}%
\definecolor{currentstroke}{rgb}{0.000000,0.000000,0.000000}%
\pgfsetstrokecolor{currentstroke}%
\pgfsetdash{}{0pt}%
\pgfpathmoveto{\pgfqpoint{4.927919in}{0.466226in}}%
\pgfpathlineto{\pgfqpoint{4.927919in}{0.543724in}}%
\pgfusepath{stroke}%
\end{pgfscope}%
\begin{pgfscope}%
\pgfpathrectangle{\pgfqpoint{4.123134in}{0.331635in}}{\pgfqpoint{2.583333in}{1.510000in}}%
\pgfusepath{clip}%
\pgfsetbuttcap%
\pgfsetroundjoin%
\pgfsetlinewidth{1.505625pt}%
\definecolor{currentstroke}{rgb}{0.000000,0.000000,0.000000}%
\pgfsetstrokecolor{currentstroke}%
\pgfsetdash{}{0pt}%
\pgfpathmoveto{\pgfqpoint{5.042480in}{0.466226in}}%
\pgfpathlineto{\pgfqpoint{5.042480in}{0.460796in}}%
\pgfusepath{stroke}%
\end{pgfscope}%
\begin{pgfscope}%
\pgfpathrectangle{\pgfqpoint{4.123134in}{0.331635in}}{\pgfqpoint{2.583333in}{1.510000in}}%
\pgfusepath{clip}%
\pgfsetbuttcap%
\pgfsetroundjoin%
\pgfsetlinewidth{1.505625pt}%
\definecolor{currentstroke}{rgb}{0.000000,0.000000,0.000000}%
\pgfsetstrokecolor{currentstroke}%
\pgfsetdash{}{0pt}%
\pgfpathmoveto{\pgfqpoint{5.157040in}{0.466226in}}%
\pgfpathlineto{\pgfqpoint{5.157040in}{0.455345in}}%
\pgfusepath{stroke}%
\end{pgfscope}%
\begin{pgfscope}%
\pgfpathrectangle{\pgfqpoint{4.123134in}{0.331635in}}{\pgfqpoint{2.583333in}{1.510000in}}%
\pgfusepath{clip}%
\pgfsetbuttcap%
\pgfsetroundjoin%
\pgfsetlinewidth{1.505625pt}%
\definecolor{currentstroke}{rgb}{0.000000,0.000000,0.000000}%
\pgfsetstrokecolor{currentstroke}%
\pgfsetdash{}{0pt}%
\pgfpathmoveto{\pgfqpoint{5.271600in}{0.466226in}}%
\pgfpathlineto{\pgfqpoint{5.271600in}{0.506889in}}%
\pgfusepath{stroke}%
\end{pgfscope}%
\begin{pgfscope}%
\pgfpathrectangle{\pgfqpoint{4.123134in}{0.331635in}}{\pgfqpoint{2.583333in}{1.510000in}}%
\pgfusepath{clip}%
\pgfsetbuttcap%
\pgfsetroundjoin%
\pgfsetlinewidth{1.505625pt}%
\definecolor{currentstroke}{rgb}{0.000000,0.000000,0.000000}%
\pgfsetstrokecolor{currentstroke}%
\pgfsetdash{}{0pt}%
\pgfpathmoveto{\pgfqpoint{5.386160in}{0.466226in}}%
\pgfpathlineto{\pgfqpoint{5.386160in}{0.455719in}}%
\pgfusepath{stroke}%
\end{pgfscope}%
\begin{pgfscope}%
\pgfpathrectangle{\pgfqpoint{4.123134in}{0.331635in}}{\pgfqpoint{2.583333in}{1.510000in}}%
\pgfusepath{clip}%
\pgfsetbuttcap%
\pgfsetroundjoin%
\pgfsetlinewidth{1.505625pt}%
\definecolor{currentstroke}{rgb}{0.000000,0.000000,0.000000}%
\pgfsetstrokecolor{currentstroke}%
\pgfsetdash{}{0pt}%
\pgfpathmoveto{\pgfqpoint{5.500721in}{0.466226in}}%
\pgfpathlineto{\pgfqpoint{5.500721in}{0.487421in}}%
\pgfusepath{stroke}%
\end{pgfscope}%
\begin{pgfscope}%
\pgfpathrectangle{\pgfqpoint{4.123134in}{0.331635in}}{\pgfqpoint{2.583333in}{1.510000in}}%
\pgfusepath{clip}%
\pgfsetbuttcap%
\pgfsetroundjoin%
\pgfsetlinewidth{1.505625pt}%
\definecolor{currentstroke}{rgb}{0.000000,0.000000,0.000000}%
\pgfsetstrokecolor{currentstroke}%
\pgfsetdash{}{0pt}%
\pgfpathmoveto{\pgfqpoint{5.615281in}{0.466226in}}%
\pgfpathlineto{\pgfqpoint{5.615281in}{0.463088in}}%
\pgfusepath{stroke}%
\end{pgfscope}%
\begin{pgfscope}%
\pgfpathrectangle{\pgfqpoint{4.123134in}{0.331635in}}{\pgfqpoint{2.583333in}{1.510000in}}%
\pgfusepath{clip}%
\pgfsetbuttcap%
\pgfsetroundjoin%
\pgfsetlinewidth{1.505625pt}%
\definecolor{currentstroke}{rgb}{0.000000,0.000000,0.000000}%
\pgfsetstrokecolor{currentstroke}%
\pgfsetdash{}{0pt}%
\pgfpathmoveto{\pgfqpoint{5.729841in}{0.466226in}}%
\pgfpathlineto{\pgfqpoint{5.729841in}{0.450497in}}%
\pgfusepath{stroke}%
\end{pgfscope}%
\begin{pgfscope}%
\pgfpathrectangle{\pgfqpoint{4.123134in}{0.331635in}}{\pgfqpoint{2.583333in}{1.510000in}}%
\pgfusepath{clip}%
\pgfsetbuttcap%
\pgfsetroundjoin%
\pgfsetlinewidth{1.505625pt}%
\definecolor{currentstroke}{rgb}{0.000000,0.000000,0.000000}%
\pgfsetstrokecolor{currentstroke}%
\pgfsetdash{}{0pt}%
\pgfpathmoveto{\pgfqpoint{5.844401in}{0.466226in}}%
\pgfpathlineto{\pgfqpoint{5.844401in}{0.469976in}}%
\pgfusepath{stroke}%
\end{pgfscope}%
\begin{pgfscope}%
\pgfpathrectangle{\pgfqpoint{4.123134in}{0.331635in}}{\pgfqpoint{2.583333in}{1.510000in}}%
\pgfusepath{clip}%
\pgfsetbuttcap%
\pgfsetroundjoin%
\pgfsetlinewidth{1.505625pt}%
\definecolor{currentstroke}{rgb}{0.000000,0.000000,0.000000}%
\pgfsetstrokecolor{currentstroke}%
\pgfsetdash{}{0pt}%
\pgfpathmoveto{\pgfqpoint{5.958962in}{0.466226in}}%
\pgfpathlineto{\pgfqpoint{5.958962in}{0.513799in}}%
\pgfusepath{stroke}%
\end{pgfscope}%
\begin{pgfscope}%
\pgfpathrectangle{\pgfqpoint{4.123134in}{0.331635in}}{\pgfqpoint{2.583333in}{1.510000in}}%
\pgfusepath{clip}%
\pgfsetbuttcap%
\pgfsetroundjoin%
\pgfsetlinewidth{1.505625pt}%
\definecolor{currentstroke}{rgb}{0.000000,0.000000,0.000000}%
\pgfsetstrokecolor{currentstroke}%
\pgfsetdash{}{0pt}%
\pgfpathmoveto{\pgfqpoint{6.073522in}{0.466226in}}%
\pgfpathlineto{\pgfqpoint{6.073522in}{0.459066in}}%
\pgfusepath{stroke}%
\end{pgfscope}%
\begin{pgfscope}%
\pgfpathrectangle{\pgfqpoint{4.123134in}{0.331635in}}{\pgfqpoint{2.583333in}{1.510000in}}%
\pgfusepath{clip}%
\pgfsetbuttcap%
\pgfsetroundjoin%
\pgfsetlinewidth{1.505625pt}%
\definecolor{currentstroke}{rgb}{0.000000,0.000000,0.000000}%
\pgfsetstrokecolor{currentstroke}%
\pgfsetdash{}{0pt}%
\pgfpathmoveto{\pgfqpoint{6.188082in}{0.466226in}}%
\pgfpathlineto{\pgfqpoint{6.188082in}{0.459984in}}%
\pgfusepath{stroke}%
\end{pgfscope}%
\begin{pgfscope}%
\pgfpathrectangle{\pgfqpoint{4.123134in}{0.331635in}}{\pgfqpoint{2.583333in}{1.510000in}}%
\pgfusepath{clip}%
\pgfsetbuttcap%
\pgfsetroundjoin%
\pgfsetlinewidth{1.505625pt}%
\definecolor{currentstroke}{rgb}{0.000000,0.000000,0.000000}%
\pgfsetstrokecolor{currentstroke}%
\pgfsetdash{}{0pt}%
\pgfpathmoveto{\pgfqpoint{6.302642in}{0.466226in}}%
\pgfpathlineto{\pgfqpoint{6.302642in}{0.511292in}}%
\pgfusepath{stroke}%
\end{pgfscope}%
\begin{pgfscope}%
\pgfpathrectangle{\pgfqpoint{4.123134in}{0.331635in}}{\pgfqpoint{2.583333in}{1.510000in}}%
\pgfusepath{clip}%
\pgfsetbuttcap%
\pgfsetroundjoin%
\pgfsetlinewidth{1.505625pt}%
\definecolor{currentstroke}{rgb}{0.000000,0.000000,0.000000}%
\pgfsetstrokecolor{currentstroke}%
\pgfsetdash{}{0pt}%
\pgfpathmoveto{\pgfqpoint{6.417202in}{0.466226in}}%
\pgfpathlineto{\pgfqpoint{6.417202in}{0.460740in}}%
\pgfusepath{stroke}%
\end{pgfscope}%
\begin{pgfscope}%
\pgfpathrectangle{\pgfqpoint{4.123134in}{0.331635in}}{\pgfqpoint{2.583333in}{1.510000in}}%
\pgfusepath{clip}%
\pgfsetbuttcap%
\pgfsetroundjoin%
\pgfsetlinewidth{1.505625pt}%
\definecolor{currentstroke}{rgb}{0.000000,0.000000,0.000000}%
\pgfsetstrokecolor{currentstroke}%
\pgfsetdash{}{0pt}%
\pgfpathmoveto{\pgfqpoint{6.531763in}{0.466226in}}%
\pgfpathlineto{\pgfqpoint{6.531763in}{0.449443in}}%
\pgfusepath{stroke}%
\end{pgfscope}%
\begin{pgfscope}%
\pgfpathrectangle{\pgfqpoint{4.123134in}{0.331635in}}{\pgfqpoint{2.583333in}{1.510000in}}%
\pgfusepath{clip}%
\pgfsetroundcap%
\pgfsetroundjoin%
\pgfsetlinewidth{1.505625pt}%
\definecolor{currentstroke}{rgb}{0.737255,0.741176,0.133333}%
\pgfsetstrokecolor{currentstroke}%
\pgfsetdash{}{0pt}%
\pgfpathmoveto{\pgfqpoint{4.123134in}{0.466226in}}%
\pgfpathlineto{\pgfqpoint{6.706467in}{0.466226in}}%
\pgfusepath{stroke}%
\end{pgfscope}%
\begin{pgfscope}%
\pgfpathrectangle{\pgfqpoint{4.123134in}{0.331635in}}{\pgfqpoint{2.583333in}{1.510000in}}%
\pgfusepath{clip}%
\pgfsetbuttcap%
\pgfsetroundjoin%
\definecolor{currentfill}{rgb}{0.737255,0.741176,0.133333}%
\pgfsetfillcolor{currentfill}%
\pgfsetlinewidth{1.003750pt}%
\definecolor{currentstroke}{rgb}{0.737255,0.741176,0.133333}%
\pgfsetstrokecolor{currentstroke}%
\pgfsetdash{}{0pt}%
\pgfsys@defobject{currentmarker}{\pgfqpoint{-0.034722in}{-0.034722in}}{\pgfqpoint{0.034722in}{0.034722in}}{%
\pgfpathmoveto{\pgfqpoint{0.000000in}{-0.034722in}}%
\pgfpathcurveto{\pgfqpoint{0.009208in}{-0.034722in}}{\pgfqpoint{0.018041in}{-0.031064in}}{\pgfqpoint{0.024552in}{-0.024552in}}%
\pgfpathcurveto{\pgfqpoint{0.031064in}{-0.018041in}}{\pgfqpoint{0.034722in}{-0.009208in}}{\pgfqpoint{0.034722in}{0.000000in}}%
\pgfpathcurveto{\pgfqpoint{0.034722in}{0.009208in}}{\pgfqpoint{0.031064in}{0.018041in}}{\pgfqpoint{0.024552in}{0.024552in}}%
\pgfpathcurveto{\pgfqpoint{0.018041in}{0.031064in}}{\pgfqpoint{0.009208in}{0.034722in}}{\pgfqpoint{0.000000in}{0.034722in}}%
\pgfpathcurveto{\pgfqpoint{-0.009208in}{0.034722in}}{\pgfqpoint{-0.018041in}{0.031064in}}{\pgfqpoint{-0.024552in}{0.024552in}}%
\pgfpathcurveto{\pgfqpoint{-0.031064in}{0.018041in}}{\pgfqpoint{-0.034722in}{0.009208in}}{\pgfqpoint{-0.034722in}{0.000000in}}%
\pgfpathcurveto{\pgfqpoint{-0.034722in}{-0.009208in}}{\pgfqpoint{-0.031064in}{-0.018041in}}{\pgfqpoint{-0.024552in}{-0.024552in}}%
\pgfpathcurveto{\pgfqpoint{-0.018041in}{-0.031064in}}{\pgfqpoint{-0.009208in}{-0.034722in}}{\pgfqpoint{0.000000in}{-0.034722in}}%
\pgfpathclose%
\pgfusepath{stroke,fill}%
}%
\begin{pgfscope}%
\pgfsys@transformshift{4.240558in}{1.772999in}%
\pgfsys@useobject{currentmarker}{}%
\end{pgfscope}%
\begin{pgfscope}%
\pgfsys@transformshift{4.355118in}{0.549474in}%
\pgfsys@useobject{currentmarker}{}%
\end{pgfscope}%
\begin{pgfscope}%
\pgfsys@transformshift{4.469678in}{0.514172in}%
\pgfsys@useobject{currentmarker}{}%
\end{pgfscope}%
\begin{pgfscope}%
\pgfsys@transformshift{4.584239in}{0.487572in}%
\pgfsys@useobject{currentmarker}{}%
\end{pgfscope}%
\begin{pgfscope}%
\pgfsys@transformshift{4.698799in}{0.568170in}%
\pgfsys@useobject{currentmarker}{}%
\end{pgfscope}%
\begin{pgfscope}%
\pgfsys@transformshift{4.813359in}{0.496022in}%
\pgfsys@useobject{currentmarker}{}%
\end{pgfscope}%
\begin{pgfscope}%
\pgfsys@transformshift{4.927919in}{0.543724in}%
\pgfsys@useobject{currentmarker}{}%
\end{pgfscope}%
\begin{pgfscope}%
\pgfsys@transformshift{5.042480in}{0.460796in}%
\pgfsys@useobject{currentmarker}{}%
\end{pgfscope}%
\begin{pgfscope}%
\pgfsys@transformshift{5.157040in}{0.455345in}%
\pgfsys@useobject{currentmarker}{}%
\end{pgfscope}%
\begin{pgfscope}%
\pgfsys@transformshift{5.271600in}{0.506889in}%
\pgfsys@useobject{currentmarker}{}%
\end{pgfscope}%
\begin{pgfscope}%
\pgfsys@transformshift{5.386160in}{0.455719in}%
\pgfsys@useobject{currentmarker}{}%
\end{pgfscope}%
\begin{pgfscope}%
\pgfsys@transformshift{5.500721in}{0.487421in}%
\pgfsys@useobject{currentmarker}{}%
\end{pgfscope}%
\begin{pgfscope}%
\pgfsys@transformshift{5.615281in}{0.463088in}%
\pgfsys@useobject{currentmarker}{}%
\end{pgfscope}%
\begin{pgfscope}%
\pgfsys@transformshift{5.729841in}{0.450497in}%
\pgfsys@useobject{currentmarker}{}%
\end{pgfscope}%
\begin{pgfscope}%
\pgfsys@transformshift{5.844401in}{0.469976in}%
\pgfsys@useobject{currentmarker}{}%
\end{pgfscope}%
\begin{pgfscope}%
\pgfsys@transformshift{5.958962in}{0.513799in}%
\pgfsys@useobject{currentmarker}{}%
\end{pgfscope}%
\begin{pgfscope}%
\pgfsys@transformshift{6.073522in}{0.459066in}%
\pgfsys@useobject{currentmarker}{}%
\end{pgfscope}%
\begin{pgfscope}%
\pgfsys@transformshift{6.188082in}{0.459984in}%
\pgfsys@useobject{currentmarker}{}%
\end{pgfscope}%
\begin{pgfscope}%
\pgfsys@transformshift{6.302642in}{0.511292in}%
\pgfsys@useobject{currentmarker}{}%
\end{pgfscope}%
\begin{pgfscope}%
\pgfsys@transformshift{6.417202in}{0.460740in}%
\pgfsys@useobject{currentmarker}{}%
\end{pgfscope}%
\begin{pgfscope}%
\pgfsys@transformshift{6.531763in}{0.449443in}%
\pgfsys@useobject{currentmarker}{}%
\end{pgfscope}%
\end{pgfscope}%
\begin{pgfscope}%
\pgfsetrectcap%
\pgfsetmiterjoin%
\pgfsetlinewidth{0.803000pt}%
\definecolor{currentstroke}{rgb}{1.000000,1.000000,1.000000}%
\pgfsetstrokecolor{currentstroke}%
\pgfsetdash{}{0pt}%
\pgfpathmoveto{\pgfqpoint{4.123134in}{0.331635in}}%
\pgfpathlineto{\pgfqpoint{4.123134in}{1.841635in}}%
\pgfusepath{stroke}%
\end{pgfscope}%
\begin{pgfscope}%
\pgfsetrectcap%
\pgfsetmiterjoin%
\pgfsetlinewidth{0.803000pt}%
\definecolor{currentstroke}{rgb}{1.000000,1.000000,1.000000}%
\pgfsetstrokecolor{currentstroke}%
\pgfsetdash{}{0pt}%
\pgfpathmoveto{\pgfqpoint{6.706467in}{0.331635in}}%
\pgfpathlineto{\pgfqpoint{6.706467in}{1.841635in}}%
\pgfusepath{stroke}%
\end{pgfscope}%
\begin{pgfscope}%
\pgfsetrectcap%
\pgfsetmiterjoin%
\pgfsetlinewidth{0.803000pt}%
\definecolor{currentstroke}{rgb}{1.000000,1.000000,1.000000}%
\pgfsetstrokecolor{currentstroke}%
\pgfsetdash{}{0pt}%
\pgfpathmoveto{\pgfqpoint{4.123134in}{0.331635in}}%
\pgfpathlineto{\pgfqpoint{6.706467in}{0.331635in}}%
\pgfusepath{stroke}%
\end{pgfscope}%
\begin{pgfscope}%
\pgfsetrectcap%
\pgfsetmiterjoin%
\pgfsetlinewidth{0.803000pt}%
\definecolor{currentstroke}{rgb}{1.000000,1.000000,1.000000}%
\pgfsetstrokecolor{currentstroke}%
\pgfsetdash{}{0pt}%
\pgfpathmoveto{\pgfqpoint{4.123134in}{1.841635in}}%
\pgfpathlineto{\pgfqpoint{6.706467in}{1.841635in}}%
\pgfusepath{stroke}%
\end{pgfscope}%
\begin{pgfscope}%
\definecolor{textcolor}{rgb}{0.150000,0.150000,0.150000}%
\pgfsetstrokecolor{textcolor}%
\pgfsetfillcolor{textcolor}%
\pgftext[x=5.414800in,y=1.924968in,,base]{\color{textcolor}\rmfamily\fontsize{12.000000}{14.400000}\selectfont Partial Autocorrelation V}%
\end{pgfscope}%
\end{pgfpicture}%
\makeatother%
\endgroup%

    \end{adjustbox}
    \begin{adjustbox}{width=.95\textwidth,center}
    %% Creator: Matplotlib, PGF backend
%%
%% To include the figure in your LaTeX document, write
%%   \input{<filename>.pgf}
%%
%% Make sure the required packages are loaded in your preamble
%%   \usepackage{pgf}
%%
%% Figures using additional raster images can only be included by \input if
%% they are in the same directory as the main LaTeX file. For loading figures
%% from other directories you can use the `import` package
%%   \usepackage{import}
%% and then include the figures with
%%   \import{<path to file>}{<filename>.pgf}
%%
%% Matplotlib used the following preamble
%%   \usepackage{fontspec}
%%   \setmainfont{DejaVuSerif.ttf}[Path=/opt/tljh/user/lib/python3.6/site-packages/matplotlib/mpl-data/fonts/ttf/]
%%   \setsansfont{DejaVuSans.ttf}[Path=/opt/tljh/user/lib/python3.6/site-packages/matplotlib/mpl-data/fonts/ttf/]
%%   \setmonofont{DejaVuSansMono.ttf}[Path=/opt/tljh/user/lib/python3.6/site-packages/matplotlib/mpl-data/fonts/ttf/]
%%
\begingroup%
\makeatletter%
\begin{pgfpicture}%
\pgfpathrectangle{\pgfpointorigin}{\pgfqpoint{6.806467in}{2.151596in}}%
\pgfusepath{use as bounding box, clip}%
\begin{pgfscope}%
\pgfsetbuttcap%
\pgfsetmiterjoin%
\definecolor{currentfill}{rgb}{1.000000,1.000000,1.000000}%
\pgfsetfillcolor{currentfill}%
\pgfsetlinewidth{0.000000pt}%
\definecolor{currentstroke}{rgb}{1.000000,1.000000,1.000000}%
\pgfsetstrokecolor{currentstroke}%
\pgfsetdash{}{0pt}%
\pgfpathmoveto{\pgfqpoint{0.000000in}{0.000000in}}%
\pgfpathlineto{\pgfqpoint{6.806467in}{0.000000in}}%
\pgfpathlineto{\pgfqpoint{6.806467in}{2.151596in}}%
\pgfpathlineto{\pgfqpoint{0.000000in}{2.151596in}}%
\pgfpathclose%
\pgfusepath{fill}%
\end{pgfscope}%
\begin{pgfscope}%
\pgfsetbuttcap%
\pgfsetmiterjoin%
\definecolor{currentfill}{rgb}{0.917647,0.917647,0.949020}%
\pgfsetfillcolor{currentfill}%
\pgfsetlinewidth{0.000000pt}%
\definecolor{currentstroke}{rgb}{0.000000,0.000000,0.000000}%
\pgfsetstrokecolor{currentstroke}%
\pgfsetstrokeopacity{0.000000}%
\pgfsetdash{}{0pt}%
\pgfpathmoveto{\pgfqpoint{0.506467in}{0.331635in}}%
\pgfpathlineto{\pgfqpoint{3.089800in}{0.331635in}}%
\pgfpathlineto{\pgfqpoint{3.089800in}{1.841635in}}%
\pgfpathlineto{\pgfqpoint{0.506467in}{1.841635in}}%
\pgfpathclose%
\pgfusepath{fill}%
\end{pgfscope}%
\begin{pgfscope}%
\pgfpathrectangle{\pgfqpoint{0.506467in}{0.331635in}}{\pgfqpoint{2.583333in}{1.510000in}}%
\pgfusepath{clip}%
\pgfsetroundcap%
\pgfsetroundjoin%
\pgfsetlinewidth{0.803000pt}%
\definecolor{currentstroke}{rgb}{1.000000,1.000000,1.000000}%
\pgfsetstrokecolor{currentstroke}%
\pgfsetdash{}{0pt}%
\pgfpathmoveto{\pgfqpoint{0.623891in}{0.331635in}}%
\pgfpathlineto{\pgfqpoint{0.623891in}{1.841635in}}%
\pgfusepath{stroke}%
\end{pgfscope}%
\begin{pgfscope}%
\definecolor{textcolor}{rgb}{0.150000,0.150000,0.150000}%
\pgfsetstrokecolor{textcolor}%
\pgfsetfillcolor{textcolor}%
\pgftext[x=0.623891in,y=0.234413in,,top]{\color{textcolor}\rmfamily\fontsize{10.000000}{12.000000}\selectfont 0}%
\end{pgfscope}%
\begin{pgfscope}%
\pgfpathrectangle{\pgfqpoint{0.506467in}{0.331635in}}{\pgfqpoint{2.583333in}{1.510000in}}%
\pgfusepath{clip}%
\pgfsetroundcap%
\pgfsetroundjoin%
\pgfsetlinewidth{0.803000pt}%
\definecolor{currentstroke}{rgb}{1.000000,1.000000,1.000000}%
\pgfsetstrokecolor{currentstroke}%
\pgfsetdash{}{0pt}%
\pgfpathmoveto{\pgfqpoint{1.196693in}{0.331635in}}%
\pgfpathlineto{\pgfqpoint{1.196693in}{1.841635in}}%
\pgfusepath{stroke}%
\end{pgfscope}%
\begin{pgfscope}%
\definecolor{textcolor}{rgb}{0.150000,0.150000,0.150000}%
\pgfsetstrokecolor{textcolor}%
\pgfsetfillcolor{textcolor}%
\pgftext[x=1.196693in,y=0.234413in,,top]{\color{textcolor}\rmfamily\fontsize{10.000000}{12.000000}\selectfont 5}%
\end{pgfscope}%
\begin{pgfscope}%
\pgfpathrectangle{\pgfqpoint{0.506467in}{0.331635in}}{\pgfqpoint{2.583333in}{1.510000in}}%
\pgfusepath{clip}%
\pgfsetroundcap%
\pgfsetroundjoin%
\pgfsetlinewidth{0.803000pt}%
\definecolor{currentstroke}{rgb}{1.000000,1.000000,1.000000}%
\pgfsetstrokecolor{currentstroke}%
\pgfsetdash{}{0pt}%
\pgfpathmoveto{\pgfqpoint{1.769494in}{0.331635in}}%
\pgfpathlineto{\pgfqpoint{1.769494in}{1.841635in}}%
\pgfusepath{stroke}%
\end{pgfscope}%
\begin{pgfscope}%
\definecolor{textcolor}{rgb}{0.150000,0.150000,0.150000}%
\pgfsetstrokecolor{textcolor}%
\pgfsetfillcolor{textcolor}%
\pgftext[x=1.769494in,y=0.234413in,,top]{\color{textcolor}\rmfamily\fontsize{10.000000}{12.000000}\selectfont 10}%
\end{pgfscope}%
\begin{pgfscope}%
\pgfpathrectangle{\pgfqpoint{0.506467in}{0.331635in}}{\pgfqpoint{2.583333in}{1.510000in}}%
\pgfusepath{clip}%
\pgfsetroundcap%
\pgfsetroundjoin%
\pgfsetlinewidth{0.803000pt}%
\definecolor{currentstroke}{rgb}{1.000000,1.000000,1.000000}%
\pgfsetstrokecolor{currentstroke}%
\pgfsetdash{}{0pt}%
\pgfpathmoveto{\pgfqpoint{2.342295in}{0.331635in}}%
\pgfpathlineto{\pgfqpoint{2.342295in}{1.841635in}}%
\pgfusepath{stroke}%
\end{pgfscope}%
\begin{pgfscope}%
\definecolor{textcolor}{rgb}{0.150000,0.150000,0.150000}%
\pgfsetstrokecolor{textcolor}%
\pgfsetfillcolor{textcolor}%
\pgftext[x=2.342295in,y=0.234413in,,top]{\color{textcolor}\rmfamily\fontsize{10.000000}{12.000000}\selectfont 15}%
\end{pgfscope}%
\begin{pgfscope}%
\pgfpathrectangle{\pgfqpoint{0.506467in}{0.331635in}}{\pgfqpoint{2.583333in}{1.510000in}}%
\pgfusepath{clip}%
\pgfsetroundcap%
\pgfsetroundjoin%
\pgfsetlinewidth{0.803000pt}%
\definecolor{currentstroke}{rgb}{1.000000,1.000000,1.000000}%
\pgfsetstrokecolor{currentstroke}%
\pgfsetdash{}{0pt}%
\pgfpathmoveto{\pgfqpoint{2.915096in}{0.331635in}}%
\pgfpathlineto{\pgfqpoint{2.915096in}{1.841635in}}%
\pgfusepath{stroke}%
\end{pgfscope}%
\begin{pgfscope}%
\definecolor{textcolor}{rgb}{0.150000,0.150000,0.150000}%
\pgfsetstrokecolor{textcolor}%
\pgfsetfillcolor{textcolor}%
\pgftext[x=2.915096in,y=0.234413in,,top]{\color{textcolor}\rmfamily\fontsize{10.000000}{12.000000}\selectfont 20}%
\end{pgfscope}%
\begin{pgfscope}%
\pgfpathrectangle{\pgfqpoint{0.506467in}{0.331635in}}{\pgfqpoint{2.583333in}{1.510000in}}%
\pgfusepath{clip}%
\pgfsetroundcap%
\pgfsetroundjoin%
\pgfsetlinewidth{0.803000pt}%
\definecolor{currentstroke}{rgb}{1.000000,1.000000,1.000000}%
\pgfsetstrokecolor{currentstroke}%
\pgfsetdash{}{0pt}%
\pgfpathmoveto{\pgfqpoint{0.506467in}{0.466954in}}%
\pgfpathlineto{\pgfqpoint{3.089800in}{0.466954in}}%
\pgfusepath{stroke}%
\end{pgfscope}%
\begin{pgfscope}%
\definecolor{textcolor}{rgb}{0.150000,0.150000,0.150000}%
\pgfsetstrokecolor{textcolor}%
\pgfsetfillcolor{textcolor}%
\pgftext[x=0.100000in,y=0.414193in,left,base]{\color{textcolor}\rmfamily\fontsize{10.000000}{12.000000}\selectfont 0.00}%
\end{pgfscope}%
\begin{pgfscope}%
\pgfpathrectangle{\pgfqpoint{0.506467in}{0.331635in}}{\pgfqpoint{2.583333in}{1.510000in}}%
\pgfusepath{clip}%
\pgfsetroundcap%
\pgfsetroundjoin%
\pgfsetlinewidth{0.803000pt}%
\definecolor{currentstroke}{rgb}{1.000000,1.000000,1.000000}%
\pgfsetstrokecolor{currentstroke}%
\pgfsetdash{}{0pt}%
\pgfpathmoveto{\pgfqpoint{0.506467in}{0.793465in}}%
\pgfpathlineto{\pgfqpoint{3.089800in}{0.793465in}}%
\pgfusepath{stroke}%
\end{pgfscope}%
\begin{pgfscope}%
\definecolor{textcolor}{rgb}{0.150000,0.150000,0.150000}%
\pgfsetstrokecolor{textcolor}%
\pgfsetfillcolor{textcolor}%
\pgftext[x=0.100000in,y=0.740704in,left,base]{\color{textcolor}\rmfamily\fontsize{10.000000}{12.000000}\selectfont 0.25}%
\end{pgfscope}%
\begin{pgfscope}%
\pgfpathrectangle{\pgfqpoint{0.506467in}{0.331635in}}{\pgfqpoint{2.583333in}{1.510000in}}%
\pgfusepath{clip}%
\pgfsetroundcap%
\pgfsetroundjoin%
\pgfsetlinewidth{0.803000pt}%
\definecolor{currentstroke}{rgb}{1.000000,1.000000,1.000000}%
\pgfsetstrokecolor{currentstroke}%
\pgfsetdash{}{0pt}%
\pgfpathmoveto{\pgfqpoint{0.506467in}{1.119977in}}%
\pgfpathlineto{\pgfqpoint{3.089800in}{1.119977in}}%
\pgfusepath{stroke}%
\end{pgfscope}%
\begin{pgfscope}%
\definecolor{textcolor}{rgb}{0.150000,0.150000,0.150000}%
\pgfsetstrokecolor{textcolor}%
\pgfsetfillcolor{textcolor}%
\pgftext[x=0.100000in,y=1.067215in,left,base]{\color{textcolor}\rmfamily\fontsize{10.000000}{12.000000}\selectfont 0.50}%
\end{pgfscope}%
\begin{pgfscope}%
\pgfpathrectangle{\pgfqpoint{0.506467in}{0.331635in}}{\pgfqpoint{2.583333in}{1.510000in}}%
\pgfusepath{clip}%
\pgfsetroundcap%
\pgfsetroundjoin%
\pgfsetlinewidth{0.803000pt}%
\definecolor{currentstroke}{rgb}{1.000000,1.000000,1.000000}%
\pgfsetstrokecolor{currentstroke}%
\pgfsetdash{}{0pt}%
\pgfpathmoveto{\pgfqpoint{0.506467in}{1.446488in}}%
\pgfpathlineto{\pgfqpoint{3.089800in}{1.446488in}}%
\pgfusepath{stroke}%
\end{pgfscope}%
\begin{pgfscope}%
\definecolor{textcolor}{rgb}{0.150000,0.150000,0.150000}%
\pgfsetstrokecolor{textcolor}%
\pgfsetfillcolor{textcolor}%
\pgftext[x=0.100000in,y=1.393726in,left,base]{\color{textcolor}\rmfamily\fontsize{10.000000}{12.000000}\selectfont 0.75}%
\end{pgfscope}%
\begin{pgfscope}%
\pgfpathrectangle{\pgfqpoint{0.506467in}{0.331635in}}{\pgfqpoint{2.583333in}{1.510000in}}%
\pgfusepath{clip}%
\pgfsetroundcap%
\pgfsetroundjoin%
\pgfsetlinewidth{0.803000pt}%
\definecolor{currentstroke}{rgb}{1.000000,1.000000,1.000000}%
\pgfsetstrokecolor{currentstroke}%
\pgfsetdash{}{0pt}%
\pgfpathmoveto{\pgfqpoint{0.506467in}{1.772999in}}%
\pgfpathlineto{\pgfqpoint{3.089800in}{1.772999in}}%
\pgfusepath{stroke}%
\end{pgfscope}%
\begin{pgfscope}%
\definecolor{textcolor}{rgb}{0.150000,0.150000,0.150000}%
\pgfsetstrokecolor{textcolor}%
\pgfsetfillcolor{textcolor}%
\pgftext[x=0.100000in,y=1.720237in,left,base]{\color{textcolor}\rmfamily\fontsize{10.000000}{12.000000}\selectfont 1.00}%
\end{pgfscope}%
\begin{pgfscope}%
\pgfpathrectangle{\pgfqpoint{0.506467in}{0.331635in}}{\pgfqpoint{2.583333in}{1.510000in}}%
\pgfusepath{clip}%
\pgfsetbuttcap%
\pgfsetroundjoin%
\definecolor{currentfill}{rgb}{0.121569,0.466667,0.705882}%
\pgfsetfillcolor{currentfill}%
\pgfsetfillopacity{0.250000}%
\pgfsetlinewidth{1.003750pt}%
\definecolor{currentstroke}{rgb}{1.000000,1.000000,1.000000}%
\pgfsetstrokecolor{currentstroke}%
\pgfsetstrokeopacity{0.250000}%
\pgfsetdash{}{0pt}%
\pgfpathmoveto{\pgfqpoint{0.681171in}{0.532873in}}%
\pgfpathlineto{\pgfqpoint{0.681171in}{0.401036in}}%
\pgfpathlineto{\pgfqpoint{0.853012in}{0.400862in}}%
\pgfpathlineto{\pgfqpoint{0.967572in}{0.400817in}}%
\pgfpathlineto{\pgfqpoint{1.082132in}{0.400780in}}%
\pgfpathlineto{\pgfqpoint{1.196693in}{0.400774in}}%
\pgfpathlineto{\pgfqpoint{1.311253in}{0.400774in}}%
\pgfpathlineto{\pgfqpoint{1.425813in}{0.400750in}}%
\pgfpathlineto{\pgfqpoint{1.540373in}{0.400742in}}%
\pgfpathlineto{\pgfqpoint{1.654933in}{0.400738in}}%
\pgfpathlineto{\pgfqpoint{1.769494in}{0.400679in}}%
\pgfpathlineto{\pgfqpoint{1.884054in}{0.400669in}}%
\pgfpathlineto{\pgfqpoint{1.998614in}{0.400509in}}%
\pgfpathlineto{\pgfqpoint{2.113174in}{0.400431in}}%
\pgfpathlineto{\pgfqpoint{2.227735in}{0.400424in}}%
\pgfpathlineto{\pgfqpoint{2.342295in}{0.400315in}}%
\pgfpathlineto{\pgfqpoint{2.456855in}{0.400314in}}%
\pgfpathlineto{\pgfqpoint{2.571415in}{0.400286in}}%
\pgfpathlineto{\pgfqpoint{2.685976in}{0.400272in}}%
\pgfpathlineto{\pgfqpoint{2.800536in}{0.400272in}}%
\pgfpathlineto{\pgfqpoint{2.972376in}{0.400271in}}%
\pgfpathlineto{\pgfqpoint{2.972376in}{0.533637in}}%
\pgfpathlineto{\pgfqpoint{2.972376in}{0.533637in}}%
\pgfpathlineto{\pgfqpoint{2.800536in}{0.533637in}}%
\pgfpathlineto{\pgfqpoint{2.685976in}{0.533637in}}%
\pgfpathlineto{\pgfqpoint{2.571415in}{0.533623in}}%
\pgfpathlineto{\pgfqpoint{2.456855in}{0.533595in}}%
\pgfpathlineto{\pgfqpoint{2.342295in}{0.533594in}}%
\pgfpathlineto{\pgfqpoint{2.227735in}{0.533485in}}%
\pgfpathlineto{\pgfqpoint{2.113174in}{0.533478in}}%
\pgfpathlineto{\pgfqpoint{1.998614in}{0.533400in}}%
\pgfpathlineto{\pgfqpoint{1.884054in}{0.533240in}}%
\pgfpathlineto{\pgfqpoint{1.769494in}{0.533230in}}%
\pgfpathlineto{\pgfqpoint{1.654933in}{0.533171in}}%
\pgfpathlineto{\pgfqpoint{1.540373in}{0.533167in}}%
\pgfpathlineto{\pgfqpoint{1.425813in}{0.533159in}}%
\pgfpathlineto{\pgfqpoint{1.311253in}{0.533135in}}%
\pgfpathlineto{\pgfqpoint{1.196693in}{0.533135in}}%
\pgfpathlineto{\pgfqpoint{1.082132in}{0.533129in}}%
\pgfpathlineto{\pgfqpoint{0.967572in}{0.533091in}}%
\pgfpathlineto{\pgfqpoint{0.853012in}{0.533047in}}%
\pgfpathlineto{\pgfqpoint{0.681171in}{0.532873in}}%
\pgfpathclose%
\pgfusepath{stroke,fill}%
\end{pgfscope}%
\begin{pgfscope}%
\pgfpathrectangle{\pgfqpoint{0.506467in}{0.331635in}}{\pgfqpoint{2.583333in}{1.510000in}}%
\pgfusepath{clip}%
\pgfsetbuttcap%
\pgfsetroundjoin%
\pgfsetlinewidth{1.505625pt}%
\definecolor{currentstroke}{rgb}{0.000000,0.000000,0.000000}%
\pgfsetstrokecolor{currentstroke}%
\pgfsetdash{}{0pt}%
\pgfpathmoveto{\pgfqpoint{0.623891in}{0.466954in}}%
\pgfpathlineto{\pgfqpoint{0.623891in}{1.772999in}}%
\pgfusepath{stroke}%
\end{pgfscope}%
\begin{pgfscope}%
\pgfpathrectangle{\pgfqpoint{0.506467in}{0.331635in}}{\pgfqpoint{2.583333in}{1.510000in}}%
\pgfusepath{clip}%
\pgfsetbuttcap%
\pgfsetroundjoin%
\pgfsetlinewidth{1.505625pt}%
\definecolor{currentstroke}{rgb}{0.000000,0.000000,0.000000}%
\pgfsetstrokecolor{currentstroke}%
\pgfsetdash{}{0pt}%
\pgfpathmoveto{\pgfqpoint{0.738452in}{0.466954in}}%
\pgfpathlineto{\pgfqpoint{0.738452in}{0.534223in}}%
\pgfusepath{stroke}%
\end{pgfscope}%
\begin{pgfscope}%
\pgfpathrectangle{\pgfqpoint{0.506467in}{0.331635in}}{\pgfqpoint{2.583333in}{1.510000in}}%
\pgfusepath{clip}%
\pgfsetbuttcap%
\pgfsetroundjoin%
\pgfsetlinewidth{1.505625pt}%
\definecolor{currentstroke}{rgb}{0.000000,0.000000,0.000000}%
\pgfsetstrokecolor{currentstroke}%
\pgfsetdash{}{0pt}%
\pgfpathmoveto{\pgfqpoint{0.853012in}{0.466954in}}%
\pgfpathlineto{\pgfqpoint{0.853012in}{0.500834in}}%
\pgfusepath{stroke}%
\end{pgfscope}%
\begin{pgfscope}%
\pgfpathrectangle{\pgfqpoint{0.506467in}{0.331635in}}{\pgfqpoint{2.583333in}{1.510000in}}%
\pgfusepath{clip}%
\pgfsetbuttcap%
\pgfsetroundjoin%
\pgfsetlinewidth{1.505625pt}%
\definecolor{currentstroke}{rgb}{0.000000,0.000000,0.000000}%
\pgfsetstrokecolor{currentstroke}%
\pgfsetdash{}{0pt}%
\pgfpathmoveto{\pgfqpoint{0.967572in}{0.466954in}}%
\pgfpathlineto{\pgfqpoint{0.967572in}{0.498081in}}%
\pgfusepath{stroke}%
\end{pgfscope}%
\begin{pgfscope}%
\pgfpathrectangle{\pgfqpoint{0.506467in}{0.331635in}}{\pgfqpoint{2.583333in}{1.510000in}}%
\pgfusepath{clip}%
\pgfsetbuttcap%
\pgfsetroundjoin%
\pgfsetlinewidth{1.505625pt}%
\definecolor{currentstroke}{rgb}{0.000000,0.000000,0.000000}%
\pgfsetstrokecolor{currentstroke}%
\pgfsetdash{}{0pt}%
\pgfpathmoveto{\pgfqpoint{1.082132in}{0.466954in}}%
\pgfpathlineto{\pgfqpoint{1.082132in}{0.479279in}}%
\pgfusepath{stroke}%
\end{pgfscope}%
\begin{pgfscope}%
\pgfpathrectangle{\pgfqpoint{0.506467in}{0.331635in}}{\pgfqpoint{2.583333in}{1.510000in}}%
\pgfusepath{clip}%
\pgfsetbuttcap%
\pgfsetroundjoin%
\pgfsetlinewidth{1.505625pt}%
\definecolor{currentstroke}{rgb}{0.000000,0.000000,0.000000}%
\pgfsetstrokecolor{currentstroke}%
\pgfsetdash{}{0pt}%
\pgfpathmoveto{\pgfqpoint{1.196693in}{0.466954in}}%
\pgfpathlineto{\pgfqpoint{1.196693in}{0.468063in}}%
\pgfusepath{stroke}%
\end{pgfscope}%
\begin{pgfscope}%
\pgfpathrectangle{\pgfqpoint{0.506467in}{0.331635in}}{\pgfqpoint{2.583333in}{1.510000in}}%
\pgfusepath{clip}%
\pgfsetbuttcap%
\pgfsetroundjoin%
\pgfsetlinewidth{1.505625pt}%
\definecolor{currentstroke}{rgb}{0.000000,0.000000,0.000000}%
\pgfsetstrokecolor{currentstroke}%
\pgfsetdash{}{0pt}%
\pgfpathmoveto{\pgfqpoint{1.311253in}{0.466954in}}%
\pgfpathlineto{\pgfqpoint{1.311253in}{0.491989in}}%
\pgfusepath{stroke}%
\end{pgfscope}%
\begin{pgfscope}%
\pgfpathrectangle{\pgfqpoint{0.506467in}{0.331635in}}{\pgfqpoint{2.583333in}{1.510000in}}%
\pgfusepath{clip}%
\pgfsetbuttcap%
\pgfsetroundjoin%
\pgfsetlinewidth{1.505625pt}%
\definecolor{currentstroke}{rgb}{0.000000,0.000000,0.000000}%
\pgfsetstrokecolor{currentstroke}%
\pgfsetdash{}{0pt}%
\pgfpathmoveto{\pgfqpoint{1.425813in}{0.466954in}}%
\pgfpathlineto{\pgfqpoint{1.425813in}{0.452265in}}%
\pgfusepath{stroke}%
\end{pgfscope}%
\begin{pgfscope}%
\pgfpathrectangle{\pgfqpoint{0.506467in}{0.331635in}}{\pgfqpoint{2.583333in}{1.510000in}}%
\pgfusepath{clip}%
\pgfsetbuttcap%
\pgfsetroundjoin%
\pgfsetlinewidth{1.505625pt}%
\definecolor{currentstroke}{rgb}{0.000000,0.000000,0.000000}%
\pgfsetstrokecolor{currentstroke}%
\pgfsetdash{}{0pt}%
\pgfpathmoveto{\pgfqpoint{1.540373in}{0.466954in}}%
\pgfpathlineto{\pgfqpoint{1.540373in}{0.457182in}}%
\pgfusepath{stroke}%
\end{pgfscope}%
\begin{pgfscope}%
\pgfpathrectangle{\pgfqpoint{0.506467in}{0.331635in}}{\pgfqpoint{2.583333in}{1.510000in}}%
\pgfusepath{clip}%
\pgfsetbuttcap%
\pgfsetroundjoin%
\pgfsetlinewidth{1.505625pt}%
\definecolor{currentstroke}{rgb}{0.000000,0.000000,0.000000}%
\pgfsetstrokecolor{currentstroke}%
\pgfsetdash{}{0pt}%
\pgfpathmoveto{\pgfqpoint{1.654933in}{0.466954in}}%
\pgfpathlineto{\pgfqpoint{1.654933in}{0.506058in}}%
\pgfusepath{stroke}%
\end{pgfscope}%
\begin{pgfscope}%
\pgfpathrectangle{\pgfqpoint{0.506467in}{0.331635in}}{\pgfqpoint{2.583333in}{1.510000in}}%
\pgfusepath{clip}%
\pgfsetbuttcap%
\pgfsetroundjoin%
\pgfsetlinewidth{1.505625pt}%
\definecolor{currentstroke}{rgb}{0.000000,0.000000,0.000000}%
\pgfsetstrokecolor{currentstroke}%
\pgfsetdash{}{0pt}%
\pgfpathmoveto{\pgfqpoint{1.769494in}{0.466954in}}%
\pgfpathlineto{\pgfqpoint{1.769494in}{0.450638in}}%
\pgfusepath{stroke}%
\end{pgfscope}%
\begin{pgfscope}%
\pgfpathrectangle{\pgfqpoint{0.506467in}{0.331635in}}{\pgfqpoint{2.583333in}{1.510000in}}%
\pgfusepath{clip}%
\pgfsetbuttcap%
\pgfsetroundjoin%
\pgfsetlinewidth{1.505625pt}%
\definecolor{currentstroke}{rgb}{0.000000,0.000000,0.000000}%
\pgfsetstrokecolor{currentstroke}%
\pgfsetdash{}{0pt}%
\pgfpathmoveto{\pgfqpoint{1.884054in}{0.466954in}}%
\pgfpathlineto{\pgfqpoint{1.884054in}{0.531602in}}%
\pgfusepath{stroke}%
\end{pgfscope}%
\begin{pgfscope}%
\pgfpathrectangle{\pgfqpoint{0.506467in}{0.331635in}}{\pgfqpoint{2.583333in}{1.510000in}}%
\pgfusepath{clip}%
\pgfsetbuttcap%
\pgfsetroundjoin%
\pgfsetlinewidth{1.505625pt}%
\definecolor{currentstroke}{rgb}{0.000000,0.000000,0.000000}%
\pgfsetstrokecolor{currentstroke}%
\pgfsetdash{}{0pt}%
\pgfpathmoveto{\pgfqpoint{1.998614in}{0.466954in}}%
\pgfpathlineto{\pgfqpoint{1.998614in}{0.512094in}}%
\pgfusepath{stroke}%
\end{pgfscope}%
\begin{pgfscope}%
\pgfpathrectangle{\pgfqpoint{0.506467in}{0.331635in}}{\pgfqpoint{2.583333in}{1.510000in}}%
\pgfusepath{clip}%
\pgfsetbuttcap%
\pgfsetroundjoin%
\pgfsetlinewidth{1.505625pt}%
\definecolor{currentstroke}{rgb}{0.000000,0.000000,0.000000}%
\pgfsetstrokecolor{currentstroke}%
\pgfsetdash{}{0pt}%
\pgfpathmoveto{\pgfqpoint{2.113174in}{0.466954in}}%
\pgfpathlineto{\pgfqpoint{2.113174in}{0.480143in}}%
\pgfusepath{stroke}%
\end{pgfscope}%
\begin{pgfscope}%
\pgfpathrectangle{\pgfqpoint{0.506467in}{0.331635in}}{\pgfqpoint{2.583333in}{1.510000in}}%
\pgfusepath{clip}%
\pgfsetbuttcap%
\pgfsetroundjoin%
\pgfsetlinewidth{1.505625pt}%
\definecolor{currentstroke}{rgb}{0.000000,0.000000,0.000000}%
\pgfsetstrokecolor{currentstroke}%
\pgfsetdash{}{0pt}%
\pgfpathmoveto{\pgfqpoint{2.227735in}{0.466954in}}%
\pgfpathlineto{\pgfqpoint{2.227735in}{0.520374in}}%
\pgfusepath{stroke}%
\end{pgfscope}%
\begin{pgfscope}%
\pgfpathrectangle{\pgfqpoint{0.506467in}{0.331635in}}{\pgfqpoint{2.583333in}{1.510000in}}%
\pgfusepath{clip}%
\pgfsetbuttcap%
\pgfsetroundjoin%
\pgfsetlinewidth{1.505625pt}%
\definecolor{currentstroke}{rgb}{0.000000,0.000000,0.000000}%
\pgfsetstrokecolor{currentstroke}%
\pgfsetdash{}{0pt}%
\pgfpathmoveto{\pgfqpoint{2.342295in}{0.466954in}}%
\pgfpathlineto{\pgfqpoint{2.342295in}{0.463454in}}%
\pgfusepath{stroke}%
\end{pgfscope}%
\begin{pgfscope}%
\pgfpathrectangle{\pgfqpoint{0.506467in}{0.331635in}}{\pgfqpoint{2.583333in}{1.510000in}}%
\pgfusepath{clip}%
\pgfsetbuttcap%
\pgfsetroundjoin%
\pgfsetlinewidth{1.505625pt}%
\definecolor{currentstroke}{rgb}{0.000000,0.000000,0.000000}%
\pgfsetstrokecolor{currentstroke}%
\pgfsetdash{}{0pt}%
\pgfpathmoveto{\pgfqpoint{2.456855in}{0.466954in}}%
\pgfpathlineto{\pgfqpoint{2.456855in}{0.439680in}}%
\pgfusepath{stroke}%
\end{pgfscope}%
\begin{pgfscope}%
\pgfpathrectangle{\pgfqpoint{0.506467in}{0.331635in}}{\pgfqpoint{2.583333in}{1.510000in}}%
\pgfusepath{clip}%
\pgfsetbuttcap%
\pgfsetroundjoin%
\pgfsetlinewidth{1.505625pt}%
\definecolor{currentstroke}{rgb}{0.000000,0.000000,0.000000}%
\pgfsetstrokecolor{currentstroke}%
\pgfsetdash{}{0pt}%
\pgfpathmoveto{\pgfqpoint{2.571415in}{0.466954in}}%
\pgfpathlineto{\pgfqpoint{2.571415in}{0.447681in}}%
\pgfusepath{stroke}%
\end{pgfscope}%
\begin{pgfscope}%
\pgfpathrectangle{\pgfqpoint{0.506467in}{0.331635in}}{\pgfqpoint{2.583333in}{1.510000in}}%
\pgfusepath{clip}%
\pgfsetbuttcap%
\pgfsetroundjoin%
\pgfsetlinewidth{1.505625pt}%
\definecolor{currentstroke}{rgb}{0.000000,0.000000,0.000000}%
\pgfsetstrokecolor{currentstroke}%
\pgfsetdash{}{0pt}%
\pgfpathmoveto{\pgfqpoint{2.685976in}{0.466954in}}%
\pgfpathlineto{\pgfqpoint{2.685976in}{0.468569in}}%
\pgfusepath{stroke}%
\end{pgfscope}%
\begin{pgfscope}%
\pgfpathrectangle{\pgfqpoint{0.506467in}{0.331635in}}{\pgfqpoint{2.583333in}{1.510000in}}%
\pgfusepath{clip}%
\pgfsetbuttcap%
\pgfsetroundjoin%
\pgfsetlinewidth{1.505625pt}%
\definecolor{currentstroke}{rgb}{0.000000,0.000000,0.000000}%
\pgfsetstrokecolor{currentstroke}%
\pgfsetdash{}{0pt}%
\pgfpathmoveto{\pgfqpoint{2.800536in}{0.466954in}}%
\pgfpathlineto{\pgfqpoint{2.800536in}{0.468972in}}%
\pgfusepath{stroke}%
\end{pgfscope}%
\begin{pgfscope}%
\pgfpathrectangle{\pgfqpoint{0.506467in}{0.331635in}}{\pgfqpoint{2.583333in}{1.510000in}}%
\pgfusepath{clip}%
\pgfsetbuttcap%
\pgfsetroundjoin%
\pgfsetlinewidth{1.505625pt}%
\definecolor{currentstroke}{rgb}{0.000000,0.000000,0.000000}%
\pgfsetstrokecolor{currentstroke}%
\pgfsetdash{}{0pt}%
\pgfpathmoveto{\pgfqpoint{2.915096in}{0.466954in}}%
\pgfpathlineto{\pgfqpoint{2.915096in}{0.464769in}}%
\pgfusepath{stroke}%
\end{pgfscope}%
\begin{pgfscope}%
\pgfpathrectangle{\pgfqpoint{0.506467in}{0.331635in}}{\pgfqpoint{2.583333in}{1.510000in}}%
\pgfusepath{clip}%
\pgfsetroundcap%
\pgfsetroundjoin%
\pgfsetlinewidth{1.505625pt}%
\definecolor{currentstroke}{rgb}{0.839216,0.152941,0.156863}%
\pgfsetstrokecolor{currentstroke}%
\pgfsetdash{}{0pt}%
\pgfpathmoveto{\pgfqpoint{0.506467in}{0.466954in}}%
\pgfpathlineto{\pgfqpoint{3.089800in}{0.466954in}}%
\pgfusepath{stroke}%
\end{pgfscope}%
\begin{pgfscope}%
\pgfpathrectangle{\pgfqpoint{0.506467in}{0.331635in}}{\pgfqpoint{2.583333in}{1.510000in}}%
\pgfusepath{clip}%
\pgfsetbuttcap%
\pgfsetroundjoin%
\definecolor{currentfill}{rgb}{0.839216,0.152941,0.156863}%
\pgfsetfillcolor{currentfill}%
\pgfsetlinewidth{1.003750pt}%
\definecolor{currentstroke}{rgb}{0.839216,0.152941,0.156863}%
\pgfsetstrokecolor{currentstroke}%
\pgfsetdash{}{0pt}%
\pgfsys@defobject{currentmarker}{\pgfqpoint{-0.034722in}{-0.034722in}}{\pgfqpoint{0.034722in}{0.034722in}}{%
\pgfpathmoveto{\pgfqpoint{0.000000in}{-0.034722in}}%
\pgfpathcurveto{\pgfqpoint{0.009208in}{-0.034722in}}{\pgfqpoint{0.018041in}{-0.031064in}}{\pgfqpoint{0.024552in}{-0.024552in}}%
\pgfpathcurveto{\pgfqpoint{0.031064in}{-0.018041in}}{\pgfqpoint{0.034722in}{-0.009208in}}{\pgfqpoint{0.034722in}{0.000000in}}%
\pgfpathcurveto{\pgfqpoint{0.034722in}{0.009208in}}{\pgfqpoint{0.031064in}{0.018041in}}{\pgfqpoint{0.024552in}{0.024552in}}%
\pgfpathcurveto{\pgfqpoint{0.018041in}{0.031064in}}{\pgfqpoint{0.009208in}{0.034722in}}{\pgfqpoint{0.000000in}{0.034722in}}%
\pgfpathcurveto{\pgfqpoint{-0.009208in}{0.034722in}}{\pgfqpoint{-0.018041in}{0.031064in}}{\pgfqpoint{-0.024552in}{0.024552in}}%
\pgfpathcurveto{\pgfqpoint{-0.031064in}{0.018041in}}{\pgfqpoint{-0.034722in}{0.009208in}}{\pgfqpoint{-0.034722in}{0.000000in}}%
\pgfpathcurveto{\pgfqpoint{-0.034722in}{-0.009208in}}{\pgfqpoint{-0.031064in}{-0.018041in}}{\pgfqpoint{-0.024552in}{-0.024552in}}%
\pgfpathcurveto{\pgfqpoint{-0.018041in}{-0.031064in}}{\pgfqpoint{-0.009208in}{-0.034722in}}{\pgfqpoint{0.000000in}{-0.034722in}}%
\pgfpathclose%
\pgfusepath{stroke,fill}%
}%
\begin{pgfscope}%
\pgfsys@transformshift{0.623891in}{1.772999in}%
\pgfsys@useobject{currentmarker}{}%
\end{pgfscope}%
\begin{pgfscope}%
\pgfsys@transformshift{0.738452in}{0.534223in}%
\pgfsys@useobject{currentmarker}{}%
\end{pgfscope}%
\begin{pgfscope}%
\pgfsys@transformshift{0.853012in}{0.500834in}%
\pgfsys@useobject{currentmarker}{}%
\end{pgfscope}%
\begin{pgfscope}%
\pgfsys@transformshift{0.967572in}{0.498081in}%
\pgfsys@useobject{currentmarker}{}%
\end{pgfscope}%
\begin{pgfscope}%
\pgfsys@transformshift{1.082132in}{0.479279in}%
\pgfsys@useobject{currentmarker}{}%
\end{pgfscope}%
\begin{pgfscope}%
\pgfsys@transformshift{1.196693in}{0.468063in}%
\pgfsys@useobject{currentmarker}{}%
\end{pgfscope}%
\begin{pgfscope}%
\pgfsys@transformshift{1.311253in}{0.491989in}%
\pgfsys@useobject{currentmarker}{}%
\end{pgfscope}%
\begin{pgfscope}%
\pgfsys@transformshift{1.425813in}{0.452265in}%
\pgfsys@useobject{currentmarker}{}%
\end{pgfscope}%
\begin{pgfscope}%
\pgfsys@transformshift{1.540373in}{0.457182in}%
\pgfsys@useobject{currentmarker}{}%
\end{pgfscope}%
\begin{pgfscope}%
\pgfsys@transformshift{1.654933in}{0.506058in}%
\pgfsys@useobject{currentmarker}{}%
\end{pgfscope}%
\begin{pgfscope}%
\pgfsys@transformshift{1.769494in}{0.450638in}%
\pgfsys@useobject{currentmarker}{}%
\end{pgfscope}%
\begin{pgfscope}%
\pgfsys@transformshift{1.884054in}{0.531602in}%
\pgfsys@useobject{currentmarker}{}%
\end{pgfscope}%
\begin{pgfscope}%
\pgfsys@transformshift{1.998614in}{0.512094in}%
\pgfsys@useobject{currentmarker}{}%
\end{pgfscope}%
\begin{pgfscope}%
\pgfsys@transformshift{2.113174in}{0.480143in}%
\pgfsys@useobject{currentmarker}{}%
\end{pgfscope}%
\begin{pgfscope}%
\pgfsys@transformshift{2.227735in}{0.520374in}%
\pgfsys@useobject{currentmarker}{}%
\end{pgfscope}%
\begin{pgfscope}%
\pgfsys@transformshift{2.342295in}{0.463454in}%
\pgfsys@useobject{currentmarker}{}%
\end{pgfscope}%
\begin{pgfscope}%
\pgfsys@transformshift{2.456855in}{0.439680in}%
\pgfsys@useobject{currentmarker}{}%
\end{pgfscope}%
\begin{pgfscope}%
\pgfsys@transformshift{2.571415in}{0.447681in}%
\pgfsys@useobject{currentmarker}{}%
\end{pgfscope}%
\begin{pgfscope}%
\pgfsys@transformshift{2.685976in}{0.468569in}%
\pgfsys@useobject{currentmarker}{}%
\end{pgfscope}%
\begin{pgfscope}%
\pgfsys@transformshift{2.800536in}{0.468972in}%
\pgfsys@useobject{currentmarker}{}%
\end{pgfscope}%
\begin{pgfscope}%
\pgfsys@transformshift{2.915096in}{0.464769in}%
\pgfsys@useobject{currentmarker}{}%
\end{pgfscope}%
\end{pgfscope}%
\begin{pgfscope}%
\pgfsetrectcap%
\pgfsetmiterjoin%
\pgfsetlinewidth{0.803000pt}%
\definecolor{currentstroke}{rgb}{1.000000,1.000000,1.000000}%
\pgfsetstrokecolor{currentstroke}%
\pgfsetdash{}{0pt}%
\pgfpathmoveto{\pgfqpoint{0.506467in}{0.331635in}}%
\pgfpathlineto{\pgfqpoint{0.506467in}{1.841635in}}%
\pgfusepath{stroke}%
\end{pgfscope}%
\begin{pgfscope}%
\pgfsetrectcap%
\pgfsetmiterjoin%
\pgfsetlinewidth{0.803000pt}%
\definecolor{currentstroke}{rgb}{1.000000,1.000000,1.000000}%
\pgfsetstrokecolor{currentstroke}%
\pgfsetdash{}{0pt}%
\pgfpathmoveto{\pgfqpoint{3.089800in}{0.331635in}}%
\pgfpathlineto{\pgfqpoint{3.089800in}{1.841635in}}%
\pgfusepath{stroke}%
\end{pgfscope}%
\begin{pgfscope}%
\pgfsetrectcap%
\pgfsetmiterjoin%
\pgfsetlinewidth{0.803000pt}%
\definecolor{currentstroke}{rgb}{1.000000,1.000000,1.000000}%
\pgfsetstrokecolor{currentstroke}%
\pgfsetdash{}{0pt}%
\pgfpathmoveto{\pgfqpoint{0.506467in}{0.331635in}}%
\pgfpathlineto{\pgfqpoint{3.089800in}{0.331635in}}%
\pgfusepath{stroke}%
\end{pgfscope}%
\begin{pgfscope}%
\pgfsetrectcap%
\pgfsetmiterjoin%
\pgfsetlinewidth{0.803000pt}%
\definecolor{currentstroke}{rgb}{1.000000,1.000000,1.000000}%
\pgfsetstrokecolor{currentstroke}%
\pgfsetdash{}{0pt}%
\pgfpathmoveto{\pgfqpoint{0.506467in}{1.841635in}}%
\pgfpathlineto{\pgfqpoint{3.089800in}{1.841635in}}%
\pgfusepath{stroke}%
\end{pgfscope}%
\begin{pgfscope}%
\definecolor{textcolor}{rgb}{0.150000,0.150000,0.150000}%
\pgfsetstrokecolor{textcolor}%
\pgfsetfillcolor{textcolor}%
\pgftext[x=1.798134in,y=1.924968in,,base]{\color{textcolor}\rmfamily\fontsize{12.000000}{14.400000}\selectfont Autocorrelation INTC\^2}%
\end{pgfscope}%
\begin{pgfscope}%
\pgfsetbuttcap%
\pgfsetmiterjoin%
\definecolor{currentfill}{rgb}{0.917647,0.917647,0.949020}%
\pgfsetfillcolor{currentfill}%
\pgfsetlinewidth{0.000000pt}%
\definecolor{currentstroke}{rgb}{0.000000,0.000000,0.000000}%
\pgfsetstrokecolor{currentstroke}%
\pgfsetstrokeopacity{0.000000}%
\pgfsetdash{}{0pt}%
\pgfpathmoveto{\pgfqpoint{4.123134in}{0.331635in}}%
\pgfpathlineto{\pgfqpoint{6.706467in}{0.331635in}}%
\pgfpathlineto{\pgfqpoint{6.706467in}{1.841635in}}%
\pgfpathlineto{\pgfqpoint{4.123134in}{1.841635in}}%
\pgfpathclose%
\pgfusepath{fill}%
\end{pgfscope}%
\begin{pgfscope}%
\pgfpathrectangle{\pgfqpoint{4.123134in}{0.331635in}}{\pgfqpoint{2.583333in}{1.510000in}}%
\pgfusepath{clip}%
\pgfsetroundcap%
\pgfsetroundjoin%
\pgfsetlinewidth{0.803000pt}%
\definecolor{currentstroke}{rgb}{1.000000,1.000000,1.000000}%
\pgfsetstrokecolor{currentstroke}%
\pgfsetdash{}{0pt}%
\pgfpathmoveto{\pgfqpoint{4.240558in}{0.331635in}}%
\pgfpathlineto{\pgfqpoint{4.240558in}{1.841635in}}%
\pgfusepath{stroke}%
\end{pgfscope}%
\begin{pgfscope}%
\definecolor{textcolor}{rgb}{0.150000,0.150000,0.150000}%
\pgfsetstrokecolor{textcolor}%
\pgfsetfillcolor{textcolor}%
\pgftext[x=4.240558in,y=0.234413in,,top]{\color{textcolor}\rmfamily\fontsize{10.000000}{12.000000}\selectfont 0}%
\end{pgfscope}%
\begin{pgfscope}%
\pgfpathrectangle{\pgfqpoint{4.123134in}{0.331635in}}{\pgfqpoint{2.583333in}{1.510000in}}%
\pgfusepath{clip}%
\pgfsetroundcap%
\pgfsetroundjoin%
\pgfsetlinewidth{0.803000pt}%
\definecolor{currentstroke}{rgb}{1.000000,1.000000,1.000000}%
\pgfsetstrokecolor{currentstroke}%
\pgfsetdash{}{0pt}%
\pgfpathmoveto{\pgfqpoint{4.813359in}{0.331635in}}%
\pgfpathlineto{\pgfqpoint{4.813359in}{1.841635in}}%
\pgfusepath{stroke}%
\end{pgfscope}%
\begin{pgfscope}%
\definecolor{textcolor}{rgb}{0.150000,0.150000,0.150000}%
\pgfsetstrokecolor{textcolor}%
\pgfsetfillcolor{textcolor}%
\pgftext[x=4.813359in,y=0.234413in,,top]{\color{textcolor}\rmfamily\fontsize{10.000000}{12.000000}\selectfont 5}%
\end{pgfscope}%
\begin{pgfscope}%
\pgfpathrectangle{\pgfqpoint{4.123134in}{0.331635in}}{\pgfqpoint{2.583333in}{1.510000in}}%
\pgfusepath{clip}%
\pgfsetroundcap%
\pgfsetroundjoin%
\pgfsetlinewidth{0.803000pt}%
\definecolor{currentstroke}{rgb}{1.000000,1.000000,1.000000}%
\pgfsetstrokecolor{currentstroke}%
\pgfsetdash{}{0pt}%
\pgfpathmoveto{\pgfqpoint{5.386160in}{0.331635in}}%
\pgfpathlineto{\pgfqpoint{5.386160in}{1.841635in}}%
\pgfusepath{stroke}%
\end{pgfscope}%
\begin{pgfscope}%
\definecolor{textcolor}{rgb}{0.150000,0.150000,0.150000}%
\pgfsetstrokecolor{textcolor}%
\pgfsetfillcolor{textcolor}%
\pgftext[x=5.386160in,y=0.234413in,,top]{\color{textcolor}\rmfamily\fontsize{10.000000}{12.000000}\selectfont 10}%
\end{pgfscope}%
\begin{pgfscope}%
\pgfpathrectangle{\pgfqpoint{4.123134in}{0.331635in}}{\pgfqpoint{2.583333in}{1.510000in}}%
\pgfusepath{clip}%
\pgfsetroundcap%
\pgfsetroundjoin%
\pgfsetlinewidth{0.803000pt}%
\definecolor{currentstroke}{rgb}{1.000000,1.000000,1.000000}%
\pgfsetstrokecolor{currentstroke}%
\pgfsetdash{}{0pt}%
\pgfpathmoveto{\pgfqpoint{5.958962in}{0.331635in}}%
\pgfpathlineto{\pgfqpoint{5.958962in}{1.841635in}}%
\pgfusepath{stroke}%
\end{pgfscope}%
\begin{pgfscope}%
\definecolor{textcolor}{rgb}{0.150000,0.150000,0.150000}%
\pgfsetstrokecolor{textcolor}%
\pgfsetfillcolor{textcolor}%
\pgftext[x=5.958962in,y=0.234413in,,top]{\color{textcolor}\rmfamily\fontsize{10.000000}{12.000000}\selectfont 15}%
\end{pgfscope}%
\begin{pgfscope}%
\pgfpathrectangle{\pgfqpoint{4.123134in}{0.331635in}}{\pgfqpoint{2.583333in}{1.510000in}}%
\pgfusepath{clip}%
\pgfsetroundcap%
\pgfsetroundjoin%
\pgfsetlinewidth{0.803000pt}%
\definecolor{currentstroke}{rgb}{1.000000,1.000000,1.000000}%
\pgfsetstrokecolor{currentstroke}%
\pgfsetdash{}{0pt}%
\pgfpathmoveto{\pgfqpoint{6.531763in}{0.331635in}}%
\pgfpathlineto{\pgfqpoint{6.531763in}{1.841635in}}%
\pgfusepath{stroke}%
\end{pgfscope}%
\begin{pgfscope}%
\definecolor{textcolor}{rgb}{0.150000,0.150000,0.150000}%
\pgfsetstrokecolor{textcolor}%
\pgfsetfillcolor{textcolor}%
\pgftext[x=6.531763in,y=0.234413in,,top]{\color{textcolor}\rmfamily\fontsize{10.000000}{12.000000}\selectfont 20}%
\end{pgfscope}%
\begin{pgfscope}%
\pgfpathrectangle{\pgfqpoint{4.123134in}{0.331635in}}{\pgfqpoint{2.583333in}{1.510000in}}%
\pgfusepath{clip}%
\pgfsetroundcap%
\pgfsetroundjoin%
\pgfsetlinewidth{0.803000pt}%
\definecolor{currentstroke}{rgb}{1.000000,1.000000,1.000000}%
\pgfsetstrokecolor{currentstroke}%
\pgfsetdash{}{0pt}%
\pgfpathmoveto{\pgfqpoint{4.123134in}{0.466226in}}%
\pgfpathlineto{\pgfqpoint{6.706467in}{0.466226in}}%
\pgfusepath{stroke}%
\end{pgfscope}%
\begin{pgfscope}%
\definecolor{textcolor}{rgb}{0.150000,0.150000,0.150000}%
\pgfsetstrokecolor{textcolor}%
\pgfsetfillcolor{textcolor}%
\pgftext[x=3.716667in,y=0.413465in,left,base]{\color{textcolor}\rmfamily\fontsize{10.000000}{12.000000}\selectfont 0.00}%
\end{pgfscope}%
\begin{pgfscope}%
\pgfpathrectangle{\pgfqpoint{4.123134in}{0.331635in}}{\pgfqpoint{2.583333in}{1.510000in}}%
\pgfusepath{clip}%
\pgfsetroundcap%
\pgfsetroundjoin%
\pgfsetlinewidth{0.803000pt}%
\definecolor{currentstroke}{rgb}{1.000000,1.000000,1.000000}%
\pgfsetstrokecolor{currentstroke}%
\pgfsetdash{}{0pt}%
\pgfpathmoveto{\pgfqpoint{4.123134in}{0.792919in}}%
\pgfpathlineto{\pgfqpoint{6.706467in}{0.792919in}}%
\pgfusepath{stroke}%
\end{pgfscope}%
\begin{pgfscope}%
\definecolor{textcolor}{rgb}{0.150000,0.150000,0.150000}%
\pgfsetstrokecolor{textcolor}%
\pgfsetfillcolor{textcolor}%
\pgftext[x=3.716667in,y=0.740158in,left,base]{\color{textcolor}\rmfamily\fontsize{10.000000}{12.000000}\selectfont 0.25}%
\end{pgfscope}%
\begin{pgfscope}%
\pgfpathrectangle{\pgfqpoint{4.123134in}{0.331635in}}{\pgfqpoint{2.583333in}{1.510000in}}%
\pgfusepath{clip}%
\pgfsetroundcap%
\pgfsetroundjoin%
\pgfsetlinewidth{0.803000pt}%
\definecolor{currentstroke}{rgb}{1.000000,1.000000,1.000000}%
\pgfsetstrokecolor{currentstroke}%
\pgfsetdash{}{0pt}%
\pgfpathmoveto{\pgfqpoint{4.123134in}{1.119612in}}%
\pgfpathlineto{\pgfqpoint{6.706467in}{1.119612in}}%
\pgfusepath{stroke}%
\end{pgfscope}%
\begin{pgfscope}%
\definecolor{textcolor}{rgb}{0.150000,0.150000,0.150000}%
\pgfsetstrokecolor{textcolor}%
\pgfsetfillcolor{textcolor}%
\pgftext[x=3.716667in,y=1.066851in,left,base]{\color{textcolor}\rmfamily\fontsize{10.000000}{12.000000}\selectfont 0.50}%
\end{pgfscope}%
\begin{pgfscope}%
\pgfpathrectangle{\pgfqpoint{4.123134in}{0.331635in}}{\pgfqpoint{2.583333in}{1.510000in}}%
\pgfusepath{clip}%
\pgfsetroundcap%
\pgfsetroundjoin%
\pgfsetlinewidth{0.803000pt}%
\definecolor{currentstroke}{rgb}{1.000000,1.000000,1.000000}%
\pgfsetstrokecolor{currentstroke}%
\pgfsetdash{}{0pt}%
\pgfpathmoveto{\pgfqpoint{4.123134in}{1.446306in}}%
\pgfpathlineto{\pgfqpoint{6.706467in}{1.446306in}}%
\pgfusepath{stroke}%
\end{pgfscope}%
\begin{pgfscope}%
\definecolor{textcolor}{rgb}{0.150000,0.150000,0.150000}%
\pgfsetstrokecolor{textcolor}%
\pgfsetfillcolor{textcolor}%
\pgftext[x=3.716667in,y=1.393544in,left,base]{\color{textcolor}\rmfamily\fontsize{10.000000}{12.000000}\selectfont 0.75}%
\end{pgfscope}%
\begin{pgfscope}%
\pgfpathrectangle{\pgfqpoint{4.123134in}{0.331635in}}{\pgfqpoint{2.583333in}{1.510000in}}%
\pgfusepath{clip}%
\pgfsetroundcap%
\pgfsetroundjoin%
\pgfsetlinewidth{0.803000pt}%
\definecolor{currentstroke}{rgb}{1.000000,1.000000,1.000000}%
\pgfsetstrokecolor{currentstroke}%
\pgfsetdash{}{0pt}%
\pgfpathmoveto{\pgfqpoint{4.123134in}{1.772999in}}%
\pgfpathlineto{\pgfqpoint{6.706467in}{1.772999in}}%
\pgfusepath{stroke}%
\end{pgfscope}%
\begin{pgfscope}%
\definecolor{textcolor}{rgb}{0.150000,0.150000,0.150000}%
\pgfsetstrokecolor{textcolor}%
\pgfsetfillcolor{textcolor}%
\pgftext[x=3.716667in,y=1.720237in,left,base]{\color{textcolor}\rmfamily\fontsize{10.000000}{12.000000}\selectfont 1.00}%
\end{pgfscope}%
\begin{pgfscope}%
\pgfpathrectangle{\pgfqpoint{4.123134in}{0.331635in}}{\pgfqpoint{2.583333in}{1.510000in}}%
\pgfusepath{clip}%
\pgfsetbuttcap%
\pgfsetroundjoin%
\definecolor{currentfill}{rgb}{0.121569,0.466667,0.705882}%
\pgfsetfillcolor{currentfill}%
\pgfsetfillopacity{0.250000}%
\pgfsetlinewidth{1.003750pt}%
\definecolor{currentstroke}{rgb}{1.000000,1.000000,1.000000}%
\pgfsetstrokecolor{currentstroke}%
\pgfsetstrokeopacity{0.250000}%
\pgfsetdash{}{0pt}%
\pgfpathmoveto{\pgfqpoint{4.297838in}{0.532181in}}%
\pgfpathlineto{\pgfqpoint{4.297838in}{0.400271in}}%
\pgfpathlineto{\pgfqpoint{4.469678in}{0.400271in}}%
\pgfpathlineto{\pgfqpoint{4.584239in}{0.400271in}}%
\pgfpathlineto{\pgfqpoint{4.698799in}{0.400271in}}%
\pgfpathlineto{\pgfqpoint{4.813359in}{0.400271in}}%
\pgfpathlineto{\pgfqpoint{4.927919in}{0.400271in}}%
\pgfpathlineto{\pgfqpoint{5.042480in}{0.400271in}}%
\pgfpathlineto{\pgfqpoint{5.157040in}{0.400271in}}%
\pgfpathlineto{\pgfqpoint{5.271600in}{0.400271in}}%
\pgfpathlineto{\pgfqpoint{5.386160in}{0.400271in}}%
\pgfpathlineto{\pgfqpoint{5.500721in}{0.400271in}}%
\pgfpathlineto{\pgfqpoint{5.615281in}{0.400271in}}%
\pgfpathlineto{\pgfqpoint{5.729841in}{0.400271in}}%
\pgfpathlineto{\pgfqpoint{5.844401in}{0.400271in}}%
\pgfpathlineto{\pgfqpoint{5.958962in}{0.400271in}}%
\pgfpathlineto{\pgfqpoint{6.073522in}{0.400271in}}%
\pgfpathlineto{\pgfqpoint{6.188082in}{0.400271in}}%
\pgfpathlineto{\pgfqpoint{6.302642in}{0.400271in}}%
\pgfpathlineto{\pgfqpoint{6.417202in}{0.400271in}}%
\pgfpathlineto{\pgfqpoint{6.589043in}{0.400271in}}%
\pgfpathlineto{\pgfqpoint{6.589043in}{0.532181in}}%
\pgfpathlineto{\pgfqpoint{6.589043in}{0.532181in}}%
\pgfpathlineto{\pgfqpoint{6.417202in}{0.532181in}}%
\pgfpathlineto{\pgfqpoint{6.302642in}{0.532181in}}%
\pgfpathlineto{\pgfqpoint{6.188082in}{0.532181in}}%
\pgfpathlineto{\pgfqpoint{6.073522in}{0.532181in}}%
\pgfpathlineto{\pgfqpoint{5.958962in}{0.532181in}}%
\pgfpathlineto{\pgfqpoint{5.844401in}{0.532181in}}%
\pgfpathlineto{\pgfqpoint{5.729841in}{0.532181in}}%
\pgfpathlineto{\pgfqpoint{5.615281in}{0.532181in}}%
\pgfpathlineto{\pgfqpoint{5.500721in}{0.532181in}}%
\pgfpathlineto{\pgfqpoint{5.386160in}{0.532181in}}%
\pgfpathlineto{\pgfqpoint{5.271600in}{0.532181in}}%
\pgfpathlineto{\pgfqpoint{5.157040in}{0.532181in}}%
\pgfpathlineto{\pgfqpoint{5.042480in}{0.532181in}}%
\pgfpathlineto{\pgfqpoint{4.927919in}{0.532181in}}%
\pgfpathlineto{\pgfqpoint{4.813359in}{0.532181in}}%
\pgfpathlineto{\pgfqpoint{4.698799in}{0.532181in}}%
\pgfpathlineto{\pgfqpoint{4.584239in}{0.532181in}}%
\pgfpathlineto{\pgfqpoint{4.469678in}{0.532181in}}%
\pgfpathlineto{\pgfqpoint{4.297838in}{0.532181in}}%
\pgfpathclose%
\pgfusepath{stroke,fill}%
\end{pgfscope}%
\begin{pgfscope}%
\pgfpathrectangle{\pgfqpoint{4.123134in}{0.331635in}}{\pgfqpoint{2.583333in}{1.510000in}}%
\pgfusepath{clip}%
\pgfsetbuttcap%
\pgfsetroundjoin%
\pgfsetlinewidth{1.505625pt}%
\definecolor{currentstroke}{rgb}{0.000000,0.000000,0.000000}%
\pgfsetstrokecolor{currentstroke}%
\pgfsetdash{}{0pt}%
\pgfpathmoveto{\pgfqpoint{4.240558in}{0.466226in}}%
\pgfpathlineto{\pgfqpoint{4.240558in}{1.772999in}}%
\pgfusepath{stroke}%
\end{pgfscope}%
\begin{pgfscope}%
\pgfpathrectangle{\pgfqpoint{4.123134in}{0.331635in}}{\pgfqpoint{2.583333in}{1.510000in}}%
\pgfusepath{clip}%
\pgfsetbuttcap%
\pgfsetroundjoin%
\pgfsetlinewidth{1.505625pt}%
\definecolor{currentstroke}{rgb}{0.000000,0.000000,0.000000}%
\pgfsetstrokecolor{currentstroke}%
\pgfsetdash{}{0pt}%
\pgfpathmoveto{\pgfqpoint{4.355118in}{0.466226in}}%
\pgfpathlineto{\pgfqpoint{4.355118in}{0.533577in}}%
\pgfusepath{stroke}%
\end{pgfscope}%
\begin{pgfscope}%
\pgfpathrectangle{\pgfqpoint{4.123134in}{0.331635in}}{\pgfqpoint{2.583333in}{1.510000in}}%
\pgfusepath{clip}%
\pgfsetbuttcap%
\pgfsetroundjoin%
\pgfsetlinewidth{1.505625pt}%
\definecolor{currentstroke}{rgb}{0.000000,0.000000,0.000000}%
\pgfsetstrokecolor{currentstroke}%
\pgfsetdash{}{0pt}%
\pgfpathmoveto{\pgfqpoint{4.469678in}{0.466226in}}%
\pgfpathlineto{\pgfqpoint{4.469678in}{0.496780in}}%
\pgfusepath{stroke}%
\end{pgfscope}%
\begin{pgfscope}%
\pgfpathrectangle{\pgfqpoint{4.123134in}{0.331635in}}{\pgfqpoint{2.583333in}{1.510000in}}%
\pgfusepath{clip}%
\pgfsetbuttcap%
\pgfsetroundjoin%
\pgfsetlinewidth{1.505625pt}%
\definecolor{currentstroke}{rgb}{0.000000,0.000000,0.000000}%
\pgfsetstrokecolor{currentstroke}%
\pgfsetdash{}{0pt}%
\pgfpathmoveto{\pgfqpoint{4.584239in}{0.466226in}}%
\pgfpathlineto{\pgfqpoint{4.584239in}{0.494239in}}%
\pgfusepath{stroke}%
\end{pgfscope}%
\begin{pgfscope}%
\pgfpathrectangle{\pgfqpoint{4.123134in}{0.331635in}}{\pgfqpoint{2.583333in}{1.510000in}}%
\pgfusepath{clip}%
\pgfsetbuttcap%
\pgfsetroundjoin%
\pgfsetlinewidth{1.505625pt}%
\definecolor{currentstroke}{rgb}{0.000000,0.000000,0.000000}%
\pgfsetstrokecolor{currentstroke}%
\pgfsetdash{}{0pt}%
\pgfpathmoveto{\pgfqpoint{4.698799in}{0.466226in}}%
\pgfpathlineto{\pgfqpoint{4.698799in}{0.474866in}}%
\pgfusepath{stroke}%
\end{pgfscope}%
\begin{pgfscope}%
\pgfpathrectangle{\pgfqpoint{4.123134in}{0.331635in}}{\pgfqpoint{2.583333in}{1.510000in}}%
\pgfusepath{clip}%
\pgfsetbuttcap%
\pgfsetroundjoin%
\pgfsetlinewidth{1.505625pt}%
\definecolor{currentstroke}{rgb}{0.000000,0.000000,0.000000}%
\pgfsetstrokecolor{currentstroke}%
\pgfsetdash{}{0pt}%
\pgfpathmoveto{\pgfqpoint{4.813359in}{0.466226in}}%
\pgfpathlineto{\pgfqpoint{4.813359in}{0.464867in}}%
\pgfusepath{stroke}%
\end{pgfscope}%
\begin{pgfscope}%
\pgfpathrectangle{\pgfqpoint{4.123134in}{0.331635in}}{\pgfqpoint{2.583333in}{1.510000in}}%
\pgfusepath{clip}%
\pgfsetbuttcap%
\pgfsetroundjoin%
\pgfsetlinewidth{1.505625pt}%
\definecolor{currentstroke}{rgb}{0.000000,0.000000,0.000000}%
\pgfsetstrokecolor{currentstroke}%
\pgfsetdash{}{0pt}%
\pgfpathmoveto{\pgfqpoint{4.927919in}{0.466226in}}%
\pgfpathlineto{\pgfqpoint{4.927919in}{0.490319in}}%
\pgfusepath{stroke}%
\end{pgfscope}%
\begin{pgfscope}%
\pgfpathrectangle{\pgfqpoint{4.123134in}{0.331635in}}{\pgfqpoint{2.583333in}{1.510000in}}%
\pgfusepath{clip}%
\pgfsetbuttcap%
\pgfsetroundjoin%
\pgfsetlinewidth{1.505625pt}%
\definecolor{currentstroke}{rgb}{0.000000,0.000000,0.000000}%
\pgfsetstrokecolor{currentstroke}%
\pgfsetdash{}{0pt}%
\pgfpathmoveto{\pgfqpoint{5.042480in}{0.466226in}}%
\pgfpathlineto{\pgfqpoint{5.042480in}{0.448487in}}%
\pgfusepath{stroke}%
\end{pgfscope}%
\begin{pgfscope}%
\pgfpathrectangle{\pgfqpoint{4.123134in}{0.331635in}}{\pgfqpoint{2.583333in}{1.510000in}}%
\pgfusepath{clip}%
\pgfsetbuttcap%
\pgfsetroundjoin%
\pgfsetlinewidth{1.505625pt}%
\definecolor{currentstroke}{rgb}{0.000000,0.000000,0.000000}%
\pgfsetstrokecolor{currentstroke}%
\pgfsetdash{}{0pt}%
\pgfpathmoveto{\pgfqpoint{5.157040in}{0.466226in}}%
\pgfpathlineto{\pgfqpoint{5.157040in}{0.456753in}}%
\pgfusepath{stroke}%
\end{pgfscope}%
\begin{pgfscope}%
\pgfpathrectangle{\pgfqpoint{4.123134in}{0.331635in}}{\pgfqpoint{2.583333in}{1.510000in}}%
\pgfusepath{clip}%
\pgfsetbuttcap%
\pgfsetroundjoin%
\pgfsetlinewidth{1.505625pt}%
\definecolor{currentstroke}{rgb}{0.000000,0.000000,0.000000}%
\pgfsetstrokecolor{currentstroke}%
\pgfsetdash{}{0pt}%
\pgfpathmoveto{\pgfqpoint{5.271600in}{0.466226in}}%
\pgfpathlineto{\pgfqpoint{5.271600in}{0.506399in}}%
\pgfusepath{stroke}%
\end{pgfscope}%
\begin{pgfscope}%
\pgfpathrectangle{\pgfqpoint{4.123134in}{0.331635in}}{\pgfqpoint{2.583333in}{1.510000in}}%
\pgfusepath{clip}%
\pgfsetbuttcap%
\pgfsetroundjoin%
\pgfsetlinewidth{1.505625pt}%
\definecolor{currentstroke}{rgb}{0.000000,0.000000,0.000000}%
\pgfsetstrokecolor{currentstroke}%
\pgfsetdash{}{0pt}%
\pgfpathmoveto{\pgfqpoint{5.386160in}{0.466226in}}%
\pgfpathlineto{\pgfqpoint{5.386160in}{0.446503in}}%
\pgfusepath{stroke}%
\end{pgfscope}%
\begin{pgfscope}%
\pgfpathrectangle{\pgfqpoint{4.123134in}{0.331635in}}{\pgfqpoint{2.583333in}{1.510000in}}%
\pgfusepath{clip}%
\pgfsetbuttcap%
\pgfsetroundjoin%
\pgfsetlinewidth{1.505625pt}%
\definecolor{currentstroke}{rgb}{0.000000,0.000000,0.000000}%
\pgfsetstrokecolor{currentstroke}%
\pgfsetdash{}{0pt}%
\pgfpathmoveto{\pgfqpoint{5.500721in}{0.466226in}}%
\pgfpathlineto{\pgfqpoint{5.500721in}{0.532393in}}%
\pgfusepath{stroke}%
\end{pgfscope}%
\begin{pgfscope}%
\pgfpathrectangle{\pgfqpoint{4.123134in}{0.331635in}}{\pgfqpoint{2.583333in}{1.510000in}}%
\pgfusepath{clip}%
\pgfsetbuttcap%
\pgfsetroundjoin%
\pgfsetlinewidth{1.505625pt}%
\definecolor{currentstroke}{rgb}{0.000000,0.000000,0.000000}%
\pgfsetstrokecolor{currentstroke}%
\pgfsetdash{}{0pt}%
\pgfpathmoveto{\pgfqpoint{5.615281in}{0.466226in}}%
\pgfpathlineto{\pgfqpoint{5.615281in}{0.504187in}}%
\pgfusepath{stroke}%
\end{pgfscope}%
\begin{pgfscope}%
\pgfpathrectangle{\pgfqpoint{4.123134in}{0.331635in}}{\pgfqpoint{2.583333in}{1.510000in}}%
\pgfusepath{clip}%
\pgfsetbuttcap%
\pgfsetroundjoin%
\pgfsetlinewidth{1.505625pt}%
\definecolor{currentstroke}{rgb}{0.000000,0.000000,0.000000}%
\pgfsetstrokecolor{currentstroke}%
\pgfsetdash{}{0pt}%
\pgfpathmoveto{\pgfqpoint{5.729841in}{0.466226in}}%
\pgfpathlineto{\pgfqpoint{5.729841in}{0.473132in}}%
\pgfusepath{stroke}%
\end{pgfscope}%
\begin{pgfscope}%
\pgfpathrectangle{\pgfqpoint{4.123134in}{0.331635in}}{\pgfqpoint{2.583333in}{1.510000in}}%
\pgfusepath{clip}%
\pgfsetbuttcap%
\pgfsetroundjoin%
\pgfsetlinewidth{1.505625pt}%
\definecolor{currentstroke}{rgb}{0.000000,0.000000,0.000000}%
\pgfsetstrokecolor{currentstroke}%
\pgfsetdash{}{0pt}%
\pgfpathmoveto{\pgfqpoint{5.844401in}{0.466226in}}%
\pgfpathlineto{\pgfqpoint{5.844401in}{0.515473in}}%
\pgfusepath{stroke}%
\end{pgfscope}%
\begin{pgfscope}%
\pgfpathrectangle{\pgfqpoint{4.123134in}{0.331635in}}{\pgfqpoint{2.583333in}{1.510000in}}%
\pgfusepath{clip}%
\pgfsetbuttcap%
\pgfsetroundjoin%
\pgfsetlinewidth{1.505625pt}%
\definecolor{currentstroke}{rgb}{0.000000,0.000000,0.000000}%
\pgfsetstrokecolor{currentstroke}%
\pgfsetdash{}{0pt}%
\pgfpathmoveto{\pgfqpoint{5.958962in}{0.466226in}}%
\pgfpathlineto{\pgfqpoint{5.958962in}{0.452321in}}%
\pgfusepath{stroke}%
\end{pgfscope}%
\begin{pgfscope}%
\pgfpathrectangle{\pgfqpoint{4.123134in}{0.331635in}}{\pgfqpoint{2.583333in}{1.510000in}}%
\pgfusepath{clip}%
\pgfsetbuttcap%
\pgfsetroundjoin%
\pgfsetlinewidth{1.505625pt}%
\definecolor{currentstroke}{rgb}{0.000000,0.000000,0.000000}%
\pgfsetstrokecolor{currentstroke}%
\pgfsetdash{}{0pt}%
\pgfpathmoveto{\pgfqpoint{6.073522in}{0.466226in}}%
\pgfpathlineto{\pgfqpoint{6.073522in}{0.437586in}}%
\pgfusepath{stroke}%
\end{pgfscope}%
\begin{pgfscope}%
\pgfpathrectangle{\pgfqpoint{4.123134in}{0.331635in}}{\pgfqpoint{2.583333in}{1.510000in}}%
\pgfusepath{clip}%
\pgfsetbuttcap%
\pgfsetroundjoin%
\pgfsetlinewidth{1.505625pt}%
\definecolor{currentstroke}{rgb}{0.000000,0.000000,0.000000}%
\pgfsetstrokecolor{currentstroke}%
\pgfsetdash{}{0pt}%
\pgfpathmoveto{\pgfqpoint{6.188082in}{0.466226in}}%
\pgfpathlineto{\pgfqpoint{6.188082in}{0.445261in}}%
\pgfusepath{stroke}%
\end{pgfscope}%
\begin{pgfscope}%
\pgfpathrectangle{\pgfqpoint{4.123134in}{0.331635in}}{\pgfqpoint{2.583333in}{1.510000in}}%
\pgfusepath{clip}%
\pgfsetbuttcap%
\pgfsetroundjoin%
\pgfsetlinewidth{1.505625pt}%
\definecolor{currentstroke}{rgb}{0.000000,0.000000,0.000000}%
\pgfsetstrokecolor{currentstroke}%
\pgfsetdash{}{0pt}%
\pgfpathmoveto{\pgfqpoint{6.302642in}{0.466226in}}%
\pgfpathlineto{\pgfqpoint{6.302642in}{0.469069in}}%
\pgfusepath{stroke}%
\end{pgfscope}%
\begin{pgfscope}%
\pgfpathrectangle{\pgfqpoint{4.123134in}{0.331635in}}{\pgfqpoint{2.583333in}{1.510000in}}%
\pgfusepath{clip}%
\pgfsetbuttcap%
\pgfsetroundjoin%
\pgfsetlinewidth{1.505625pt}%
\definecolor{currentstroke}{rgb}{0.000000,0.000000,0.000000}%
\pgfsetstrokecolor{currentstroke}%
\pgfsetdash{}{0pt}%
\pgfpathmoveto{\pgfqpoint{6.417202in}{0.466226in}}%
\pgfpathlineto{\pgfqpoint{6.417202in}{0.473487in}}%
\pgfusepath{stroke}%
\end{pgfscope}%
\begin{pgfscope}%
\pgfpathrectangle{\pgfqpoint{4.123134in}{0.331635in}}{\pgfqpoint{2.583333in}{1.510000in}}%
\pgfusepath{clip}%
\pgfsetbuttcap%
\pgfsetroundjoin%
\pgfsetlinewidth{1.505625pt}%
\definecolor{currentstroke}{rgb}{0.000000,0.000000,0.000000}%
\pgfsetstrokecolor{currentstroke}%
\pgfsetdash{}{0pt}%
\pgfpathmoveto{\pgfqpoint{6.531763in}{0.466226in}}%
\pgfpathlineto{\pgfqpoint{6.531763in}{0.459594in}}%
\pgfusepath{stroke}%
\end{pgfscope}%
\begin{pgfscope}%
\pgfpathrectangle{\pgfqpoint{4.123134in}{0.331635in}}{\pgfqpoint{2.583333in}{1.510000in}}%
\pgfusepath{clip}%
\pgfsetroundcap%
\pgfsetroundjoin%
\pgfsetlinewidth{1.505625pt}%
\definecolor{currentstroke}{rgb}{0.839216,0.152941,0.156863}%
\pgfsetstrokecolor{currentstroke}%
\pgfsetdash{}{0pt}%
\pgfpathmoveto{\pgfqpoint{4.123134in}{0.466226in}}%
\pgfpathlineto{\pgfqpoint{6.706467in}{0.466226in}}%
\pgfusepath{stroke}%
\end{pgfscope}%
\begin{pgfscope}%
\pgfpathrectangle{\pgfqpoint{4.123134in}{0.331635in}}{\pgfqpoint{2.583333in}{1.510000in}}%
\pgfusepath{clip}%
\pgfsetbuttcap%
\pgfsetroundjoin%
\definecolor{currentfill}{rgb}{0.839216,0.152941,0.156863}%
\pgfsetfillcolor{currentfill}%
\pgfsetlinewidth{1.003750pt}%
\definecolor{currentstroke}{rgb}{0.839216,0.152941,0.156863}%
\pgfsetstrokecolor{currentstroke}%
\pgfsetdash{}{0pt}%
\pgfsys@defobject{currentmarker}{\pgfqpoint{-0.034722in}{-0.034722in}}{\pgfqpoint{0.034722in}{0.034722in}}{%
\pgfpathmoveto{\pgfqpoint{0.000000in}{-0.034722in}}%
\pgfpathcurveto{\pgfqpoint{0.009208in}{-0.034722in}}{\pgfqpoint{0.018041in}{-0.031064in}}{\pgfqpoint{0.024552in}{-0.024552in}}%
\pgfpathcurveto{\pgfqpoint{0.031064in}{-0.018041in}}{\pgfqpoint{0.034722in}{-0.009208in}}{\pgfqpoint{0.034722in}{0.000000in}}%
\pgfpathcurveto{\pgfqpoint{0.034722in}{0.009208in}}{\pgfqpoint{0.031064in}{0.018041in}}{\pgfqpoint{0.024552in}{0.024552in}}%
\pgfpathcurveto{\pgfqpoint{0.018041in}{0.031064in}}{\pgfqpoint{0.009208in}{0.034722in}}{\pgfqpoint{0.000000in}{0.034722in}}%
\pgfpathcurveto{\pgfqpoint{-0.009208in}{0.034722in}}{\pgfqpoint{-0.018041in}{0.031064in}}{\pgfqpoint{-0.024552in}{0.024552in}}%
\pgfpathcurveto{\pgfqpoint{-0.031064in}{0.018041in}}{\pgfqpoint{-0.034722in}{0.009208in}}{\pgfqpoint{-0.034722in}{0.000000in}}%
\pgfpathcurveto{\pgfqpoint{-0.034722in}{-0.009208in}}{\pgfqpoint{-0.031064in}{-0.018041in}}{\pgfqpoint{-0.024552in}{-0.024552in}}%
\pgfpathcurveto{\pgfqpoint{-0.018041in}{-0.031064in}}{\pgfqpoint{-0.009208in}{-0.034722in}}{\pgfqpoint{0.000000in}{-0.034722in}}%
\pgfpathclose%
\pgfusepath{stroke,fill}%
}%
\begin{pgfscope}%
\pgfsys@transformshift{4.240558in}{1.772999in}%
\pgfsys@useobject{currentmarker}{}%
\end{pgfscope}%
\begin{pgfscope}%
\pgfsys@transformshift{4.355118in}{0.533577in}%
\pgfsys@useobject{currentmarker}{}%
\end{pgfscope}%
\begin{pgfscope}%
\pgfsys@transformshift{4.469678in}{0.496780in}%
\pgfsys@useobject{currentmarker}{}%
\end{pgfscope}%
\begin{pgfscope}%
\pgfsys@transformshift{4.584239in}{0.494239in}%
\pgfsys@useobject{currentmarker}{}%
\end{pgfscope}%
\begin{pgfscope}%
\pgfsys@transformshift{4.698799in}{0.474866in}%
\pgfsys@useobject{currentmarker}{}%
\end{pgfscope}%
\begin{pgfscope}%
\pgfsys@transformshift{4.813359in}{0.464867in}%
\pgfsys@useobject{currentmarker}{}%
\end{pgfscope}%
\begin{pgfscope}%
\pgfsys@transformshift{4.927919in}{0.490319in}%
\pgfsys@useobject{currentmarker}{}%
\end{pgfscope}%
\begin{pgfscope}%
\pgfsys@transformshift{5.042480in}{0.448487in}%
\pgfsys@useobject{currentmarker}{}%
\end{pgfscope}%
\begin{pgfscope}%
\pgfsys@transformshift{5.157040in}{0.456753in}%
\pgfsys@useobject{currentmarker}{}%
\end{pgfscope}%
\begin{pgfscope}%
\pgfsys@transformshift{5.271600in}{0.506399in}%
\pgfsys@useobject{currentmarker}{}%
\end{pgfscope}%
\begin{pgfscope}%
\pgfsys@transformshift{5.386160in}{0.446503in}%
\pgfsys@useobject{currentmarker}{}%
\end{pgfscope}%
\begin{pgfscope}%
\pgfsys@transformshift{5.500721in}{0.532393in}%
\pgfsys@useobject{currentmarker}{}%
\end{pgfscope}%
\begin{pgfscope}%
\pgfsys@transformshift{5.615281in}{0.504187in}%
\pgfsys@useobject{currentmarker}{}%
\end{pgfscope}%
\begin{pgfscope}%
\pgfsys@transformshift{5.729841in}{0.473132in}%
\pgfsys@useobject{currentmarker}{}%
\end{pgfscope}%
\begin{pgfscope}%
\pgfsys@transformshift{5.844401in}{0.515473in}%
\pgfsys@useobject{currentmarker}{}%
\end{pgfscope}%
\begin{pgfscope}%
\pgfsys@transformshift{5.958962in}{0.452321in}%
\pgfsys@useobject{currentmarker}{}%
\end{pgfscope}%
\begin{pgfscope}%
\pgfsys@transformshift{6.073522in}{0.437586in}%
\pgfsys@useobject{currentmarker}{}%
\end{pgfscope}%
\begin{pgfscope}%
\pgfsys@transformshift{6.188082in}{0.445261in}%
\pgfsys@useobject{currentmarker}{}%
\end{pgfscope}%
\begin{pgfscope}%
\pgfsys@transformshift{6.302642in}{0.469069in}%
\pgfsys@useobject{currentmarker}{}%
\end{pgfscope}%
\begin{pgfscope}%
\pgfsys@transformshift{6.417202in}{0.473487in}%
\pgfsys@useobject{currentmarker}{}%
\end{pgfscope}%
\begin{pgfscope}%
\pgfsys@transformshift{6.531763in}{0.459594in}%
\pgfsys@useobject{currentmarker}{}%
\end{pgfscope}%
\end{pgfscope}%
\begin{pgfscope}%
\pgfsetrectcap%
\pgfsetmiterjoin%
\pgfsetlinewidth{0.803000pt}%
\definecolor{currentstroke}{rgb}{1.000000,1.000000,1.000000}%
\pgfsetstrokecolor{currentstroke}%
\pgfsetdash{}{0pt}%
\pgfpathmoveto{\pgfqpoint{4.123134in}{0.331635in}}%
\pgfpathlineto{\pgfqpoint{4.123134in}{1.841635in}}%
\pgfusepath{stroke}%
\end{pgfscope}%
\begin{pgfscope}%
\pgfsetrectcap%
\pgfsetmiterjoin%
\pgfsetlinewidth{0.803000pt}%
\definecolor{currentstroke}{rgb}{1.000000,1.000000,1.000000}%
\pgfsetstrokecolor{currentstroke}%
\pgfsetdash{}{0pt}%
\pgfpathmoveto{\pgfqpoint{6.706467in}{0.331635in}}%
\pgfpathlineto{\pgfqpoint{6.706467in}{1.841635in}}%
\pgfusepath{stroke}%
\end{pgfscope}%
\begin{pgfscope}%
\pgfsetrectcap%
\pgfsetmiterjoin%
\pgfsetlinewidth{0.803000pt}%
\definecolor{currentstroke}{rgb}{1.000000,1.000000,1.000000}%
\pgfsetstrokecolor{currentstroke}%
\pgfsetdash{}{0pt}%
\pgfpathmoveto{\pgfqpoint{4.123134in}{0.331635in}}%
\pgfpathlineto{\pgfqpoint{6.706467in}{0.331635in}}%
\pgfusepath{stroke}%
\end{pgfscope}%
\begin{pgfscope}%
\pgfsetrectcap%
\pgfsetmiterjoin%
\pgfsetlinewidth{0.803000pt}%
\definecolor{currentstroke}{rgb}{1.000000,1.000000,1.000000}%
\pgfsetstrokecolor{currentstroke}%
\pgfsetdash{}{0pt}%
\pgfpathmoveto{\pgfqpoint{4.123134in}{1.841635in}}%
\pgfpathlineto{\pgfqpoint{6.706467in}{1.841635in}}%
\pgfusepath{stroke}%
\end{pgfscope}%
\begin{pgfscope}%
\definecolor{textcolor}{rgb}{0.150000,0.150000,0.150000}%
\pgfsetstrokecolor{textcolor}%
\pgfsetfillcolor{textcolor}%
\pgftext[x=5.414800in,y=1.924968in,,base]{\color{textcolor}\rmfamily\fontsize{12.000000}{14.400000}\selectfont Partial Autocorrelation INTC\^2}%
\end{pgfscope}%
\end{pgfpicture}%
\makeatother%
\endgroup%

    \end{adjustbox}
    \caption{Squared log-returns and ACF and PACF for squared log-returns for V (top-left and middle) and INTC (top right and bottom). }
    \label{fig:V_INTC_squared}
\end{figure}{}

We formally test for the presence of a GARCH effect, i.e. autocorrelation in the squared residuals of the time series using the Lagrange Multiplier Test proposed by Engle (SOURCE). For both tests, the null hypothesis of 'No ARCH effect' cannot be rejected with p-values of 0.1160 (V) and 0.2693 (INTC). We nevertheless proceed to fit a GARCH model to the data. There is extensive literature on the optimal order for a GARCH model (e.g. does anything beat GARCH(1,1), does anything not beat GARCH(1,1), does anyone need GARCH(1,1)?). 

For simplicity, we will stick to a GARCH(1,1) model with a constant mean model (the mean is modeled by an ARMA(0,0) process). While we have tried other models such as for example GARCH(3,1) they did not perform significantly better than GARCH(1,1). To avoid issues with numerical instability all log-returns are multiplied by 100 and the new current best BIC become 5025.46 for V and 5205.42 for INTC as shown on the right hand side of \ref{tab:bic_arma}. Table \ref{tab:V_result_GARCH11_100} shows the result for a GARCH(1,1) fit to V, table \ref{tab:INTC_result_GARCH11_100} shows the result for INTC. In both cases the fit improves and the BIC drops considerably. 

\begin{table}[h]
    \centering
    \figuretitle{Results for GARCH(1,1) with constant mean fit to the log-returns of V}
    \begin{center}
\begin{tabular}{lclc}
\toprule
\textbf{Dep. Variable:} &    log\_returns    & \textbf{  R-squared:         } &    -0.000   \\
\textbf{Mean Model:}    &   Constant Mean    & \textbf{  Adj. R-squared:    } &    -0.000   \\
\textbf{Vol Model:}     &       GARCH        & \textbf{  Log-Likelihood:    } &   -2459.70  \\
\textbf{Distribution:}  &       Normal       & \textbf{  AIC:               } &    4927.40  \\
\textbf{Method:}        & Maximum Likelihood & \textbf{  BIC:               } &    4948.68  \\
\textbf{}               &                    & \textbf{  No. Observations:  } &    1509     \\
\textbf{Date:}          &  Wed, Sep 04 2019  & \textbf{  Df Residuals:      } &    1505     \\
\bottomrule
\end{tabular}
\begin{tabular}{lccccc}
            & \textbf{coef} & \textbf{std err} & \textbf{t} & \textbf{P$> |$t$|$} & \textbf{95.0\% Conf. Int.}  \\
\midrule
\textbf{mu} &       0.1199  &    2.971e-02     &     4.034  &      5.473e-05       &    [6.164e-02,  0.178]      \\
                  & \textbf{coef} & \textbf{std err} & \textbf{t} & \textbf{P$> |$t$|$} & \textbf{95.0\% Conf. Int.}  \\
\midrule
\textbf{omega}    &       0.0811  &    5.148e-02     &     1.576  &          0.115       &    [-1.979e-02,  0.182]     \\
\textbf{alpha[1]} &       0.0985  &    3.488e-02     &     2.823  &      4.758e-03       &    [3.010e-02,  0.167]      \\
\textbf{beta[1]}  &       0.8585  &    5.388e-02     &    15.933  &      3.718e-57       &     [  0.753,  0.964]       \\
\bottomrule
\end{tabular}
%\caption{Constant Mean - GARCH Model Results}
\end{center}

Covariance estimator: robust

    \caption{}
    \label{tab:V_result_GARCH11_100}
\end{table}{}

\begin{table}[h]
    \centering
    \figuretitle{Results for GARCH(1,1) with constant mean fit to the log-returns of V}
    \begin{center}
\begin{tabular}{lclc}
\toprule
\textbf{Dep. Variable:} &    log\_returns    & \textbf{  R-squared:         } &    -0.000   \\
\textbf{Mean Model:}    &   Constant Mean    & \textbf{  Adj. R-squared:    } &    -0.000   \\
\textbf{Vol Model:}     &       GARCH        & \textbf{  Log-Likelihood:    } &   -2570.24  \\
\textbf{Distribution:}  &       Normal       & \textbf{  AIC:               } &    5148.47  \\
\textbf{Method:}        & Maximum Likelihood & \textbf{  BIC:               } &    5169.75  \\
\textbf{}               &                    & \textbf{  No. Observations:  } &    1509     \\
\textbf{Date:}          &  Wed, Sep 04 2019  & \textbf{  Df Residuals:      } &    1505     \\
\bottomrule
\end{tabular}
\begin{tabular}{lccccc}
            & \textbf{coef} & \textbf{std err} & \textbf{t} & \textbf{P$> |$t$|$} & \textbf{95.0\% Conf. Int.}  \\
\midrule
\textbf{mu} &       0.0565  &    3.422e-02     &     1.650  &      9.897e-02       &    [-1.061e-02,  0.124]     \\
                  & \textbf{coef} & \textbf{std err} & \textbf{t} & \textbf{P$> |$t$|$} & \textbf{95.0\% Conf. Int.}  \\
\midrule
\textbf{omega}    &       0.9185  &        0.289     &     3.181  &      1.469e-03       &     [  0.353,  1.484]       \\
\textbf{alpha[1]} &       0.2295  &    9.374e-02     &     2.448  &      1.436e-02       &    [4.576e-02,  0.413]      \\
\textbf{beta[1]}  &       0.2977  &        0.168     &     1.772  &      7.642e-02       &    [-3.161e-02,  0.627]     \\
\bottomrule
\end{tabular}
%\caption{Constant Mean - GARCH Model Results}
\end{center}

Covariance estimator: robust

    \caption{}
    \label{tab:INTC_result_GARCH11_100}
\end{table}{}


test different mean models for GARCH. 




3) add TGARCH?






Fitting: 
Our Code fits an RW model, prediicts the next value, compares it to the real value for the MSE and then adds the true value to the time series used for predicting the next value. (we could have done this simpler, by just taking the values of the previous period as prediction for the current one. 

Figure: Plot Real vs. Predicted values. (Real values? or FD of log-Values?
Table: MSE





\subsection{Summary of Prediction with Time Series Models}
Table: MSE all
Figure: All predicted values?


\section{Hybrid Prediction}

The movement of time series data for financial data is influenced bei external effects and information. To leverage this we tried to add information from news sources to our trading strategies. For the hybrid prediction we predict sentiment scores on news sources. The original aim was to use financial news data to predict stock price movement and volatility for trading strategies. To achieve this, large amounts of text data would need to be preprocessed and analyzed regarding their connections to specific stocks, their topic and sentiment. The news data would need to be as precise as possible, because […] mention that an effect on the stocks an only be measured up to 20min after the news appear. Other sources say that… .
%
As we were not able to acquire access to a reliable and precise news sources, we tried to implement our approach on the available analyst reports regarding specific stocks. The problem with these reports is, that they are more an indicator of performance over the past month and a prediction about the future performance and don´t cover sudden events. The reports also cluster around (meetings?) with long stretches of no or very few reports in between. This makes it unlikely that they are valuable for trading strategies.
%
The goal was to identify the connection of specific articles to listed companies and compute a sentiment score for the article. There are many ways to calculate sentiment scores, like positive and negative, from text data. Many of these require a supervised approach … . Language is very context specific (…) making it unpractical to use other, labelled training data sets, than financial news data(…). With the use of intra day trading and news data one could have also used the movement or volatility of the period close after the news release to get a rough estimate of the impact certain news have. Such an approach was chosen by \citet{robertson2007news}. Our data is only inter day and does not allow for a classification in that way. 

A common approach for unsupervised 


Analyst report data beschreiben...

To get reliable sentiment scores text data has to be preprocessed. The preprocessing was done using \texttt{R} \citep{Rproject}. At first words where converted to lowercase and tokenized using the R package \textit{tidytext} \citep{tidytext}. Next all the stop words where removed using the stop word library from the \textit{tidytext} package, as well as a custom set. In the next step all links to websites, hyper-references, numbers and words with numbers are removed as well. 
The last step is lemmatizing the words using the \textit{textstem} package \citep{textstem}. Lemmatizing words means reducing them to their inflectional forms. Commonly stemming is also applied, because words sometimes have derivationally related forms. This was not done to have more flexibility for the later applied text analysis. Additionally we could have also used the term frequency–inverse document frequency (tf-idf) matrix (ZITIEREN) for further reductions in the number of words. The issue here would have been that highly informative words for the stock sentiment could have been removed. 

\subsection{ARMAX Predictions}
ARMAX works like this: 
XXXXXX

Predictions
Confidence Intervals


Predictors can be 
- Using weather forecasts --> ARMAX
- Number of Tweets?
- Sentiments from Machine Learning Algorithm
- Predictions made by the Algorithm



\subsection{Weighted Average of Predictions}
a) of different time series models
b) of time series and ML models

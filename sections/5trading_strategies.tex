\chapter{Trading Strategies}\label{ch:predictions}

\section{Trading Strategies - Introduction and Theory}
\subsection{Idea and Process}
We compare different Trading Strategies. With the Goal of Predicting Stocks this is interesting in and of itself. For the purpose of the project it also serves as a baseline to compare our Trading Strategy against. 

\subsection{Mean Reversion Models and Momentum Trading - Theoretical Background}

A vast body of scientific literature has tried to develop models that allow to understand and explain movements in stock markets. An even vaster community of traders has tried to implement these theories to do actual forecasting. While many of the theories are much more complex, two basic ideas can be summarized as "Momentum Based Trading" and "Mean Reversion Based Trading". The former theory hypothesizes that stocks that do well now will likely continue to do so in the future while the latter states that especially good or past performance is an exception and that stocks will eventually return to their average performance. A third strategy, Pairs Trading, will be detailed later. 

In 1985 [Bondt and Thaler] were one of the early scholars to analyze mean reversion behaviour when they examined the hypothesis that markets tend to overreact. They looked at monthly returns of assets listed on the New York Stock Exchange in between 1926 and 1982 and constructed portfolios  of winners and losers that were updated every three years. Winners and losers were those stocks that had performed the best / worst over the previous years. Their idea was that "if stock prices systematically overshoot, then their reversal should be predictable from past return data alone, with no use of any accounting data such as earnings. [...] Extreme movements in stock prices will be followed by subsequent price movements in the opposite direction." Indeed they observed that the portfolio of losers outperformed the market significantly. While they focus on a very long time horizon of three years, Jagadeesh(1991) suggests that this behaviour could also be observed in the shorter term:  "These papers show that contrarian strategies that select stocks based on their returns in the previous week or month generate significant abnormal returns." 
While mean reversion behaviour has sometimes been observed in practice, it is not trivial to derive it from economic theory. Many economic theories like the famous model from Fama and French describe the return of the stocks of a company as the result of market properties like the baseline market return rate and inherent properties of that company, like for example its book-to-market-ratio [Fama and French 1993]. If such a relationship exists than this in turn implies that the actual observed stock movements should be random fluctuations around some much more slowly changing true return rate. However, it is unclear whether this mean reversion should happen in the form of a pendulum that swings forth and back or more in the form of coin tosses reverting to their equilibrium by flooding past observations with new random ones. 
Something about multiplicative connections?

On the opposite side of the spectrum of trading strategies lies the idea of momentum. [Japateesh and CX] in 1993 were one of the first to describe "that strategies which buy stocks that have performed well in the past and sell stocks that have performed poorly in the past generate significant positive returns over 3- to 12-month holding periods." [Japateesh and CX] looked at portfolios and saw mean reversion over 12 to 48 months. Thaler and Bondt observe mean reversion over the time frame of 36 months. Many scholares agree that mean reversion and momentum behaviour are not necessarily contradictions, but that the time frame determines in which way a stock will behave [Balvers and Wu]. 


PAIRS TRADING



\section{Trading Strategies - Implementation}
\subsection{Mean Reversion Portfolio}
While their analysis is focused on monthly returns over a much longer time frame, the basic idea that was replicated in our trading strategy was the same. While they found excessive returns of that strategy from 1932 to 1977 our implementation was unforturnately much less successful.

\begin{figure}
    \centering
    \begin{adjustbox}{width=.7\textwidth,center}
        %% Creator: Matplotlib, PGF backend
%%
%% To include the figure in your LaTeX document, write
%%   \input{<filename>.pgf}
%%
%% Make sure the required packages are loaded in your preamble
%%   \usepackage{pgf}
%%
%% Figures using additional raster images can only be included by \input if
%% they are in the same directory as the main LaTeX file. For loading figures
%% from other directories you can use the `import` package
%%   \usepackage{import}
%% and then include the figures with
%%   \import{<path to file>}{<filename>.pgf}
%%
%% Matplotlib used the following preamble
%%   \usepackage{fontspec}
%%   \setmainfont{DejaVuSerif.ttf}[Path=/opt/tljh/user/lib/python3.6/site-packages/matplotlib/mpl-data/fonts/ttf/]
%%   \setsansfont{DejaVuSans.ttf}[Path=/opt/tljh/user/lib/python3.6/site-packages/matplotlib/mpl-data/fonts/ttf/]
%%   \setmonofont{DejaVuSansMono.ttf}[Path=/opt/tljh/user/lib/python3.6/site-packages/matplotlib/mpl-data/fonts/ttf/]
%%
\begingroup%
\makeatletter%
\begin{pgfpicture}%
\pgfpathrectangle{\pgfpointorigin}{\pgfqpoint{6.806467in}{2.713314in}}%
\pgfusepath{use as bounding box, clip}%
\begin{pgfscope}%
\pgfsetbuttcap%
\pgfsetmiterjoin%
\definecolor{currentfill}{rgb}{1.000000,1.000000,1.000000}%
\pgfsetfillcolor{currentfill}%
\pgfsetlinewidth{0.000000pt}%
\definecolor{currentstroke}{rgb}{1.000000,1.000000,1.000000}%
\pgfsetstrokecolor{currentstroke}%
\pgfsetdash{}{0pt}%
\pgfpathmoveto{\pgfqpoint{0.000000in}{0.000000in}}%
\pgfpathlineto{\pgfqpoint{6.806467in}{0.000000in}}%
\pgfpathlineto{\pgfqpoint{6.806467in}{2.713314in}}%
\pgfpathlineto{\pgfqpoint{0.000000in}{2.713314in}}%
\pgfpathclose%
\pgfusepath{fill}%
\end{pgfscope}%
\begin{pgfscope}%
\pgfsetbuttcap%
\pgfsetmiterjoin%
\definecolor{currentfill}{rgb}{0.917647,0.917647,0.949020}%
\pgfsetfillcolor{currentfill}%
\pgfsetlinewidth{0.000000pt}%
\definecolor{currentstroke}{rgb}{0.000000,0.000000,0.000000}%
\pgfsetstrokecolor{currentstroke}%
\pgfsetstrokeopacity{0.000000}%
\pgfsetdash{}{0pt}%
\pgfpathmoveto{\pgfqpoint{0.506467in}{0.331635in}}%
\pgfpathlineto{\pgfqpoint{6.706467in}{0.331635in}}%
\pgfpathlineto{\pgfqpoint{6.706467in}{2.596635in}}%
\pgfpathlineto{\pgfqpoint{0.506467in}{2.596635in}}%
\pgfpathclose%
\pgfusepath{fill}%
\end{pgfscope}%
\begin{pgfscope}%
\pgfpathrectangle{\pgfqpoint{0.506467in}{0.331635in}}{\pgfqpoint{6.200000in}{2.265000in}}%
\pgfusepath{clip}%
\pgfsetroundcap%
\pgfsetroundjoin%
\pgfsetlinewidth{0.803000pt}%
\definecolor{currentstroke}{rgb}{1.000000,1.000000,1.000000}%
\pgfsetstrokecolor{currentstroke}%
\pgfsetdash{}{0pt}%
\pgfpathmoveto{\pgfqpoint{0.783131in}{0.331635in}}%
\pgfpathlineto{\pgfqpoint{0.783131in}{2.596635in}}%
\pgfusepath{stroke}%
\end{pgfscope}%
\begin{pgfscope}%
\definecolor{textcolor}{rgb}{0.150000,0.150000,0.150000}%
\pgfsetstrokecolor{textcolor}%
\pgfsetfillcolor{textcolor}%
\pgftext[x=0.783131in,y=0.234413in,,top]{\color{textcolor}\rmfamily\fontsize{10.000000}{12.000000}\selectfont 2012}%
\end{pgfscope}%
\begin{pgfscope}%
\pgfpathrectangle{\pgfqpoint{0.506467in}{0.331635in}}{\pgfqpoint{6.200000in}{2.265000in}}%
\pgfusepath{clip}%
\pgfsetroundcap%
\pgfsetroundjoin%
\pgfsetlinewidth{0.803000pt}%
\definecolor{currentstroke}{rgb}{1.000000,1.000000,1.000000}%
\pgfsetstrokecolor{currentstroke}%
\pgfsetdash{}{0pt}%
\pgfpathmoveto{\pgfqpoint{1.726391in}{0.331635in}}%
\pgfpathlineto{\pgfqpoint{1.726391in}{2.596635in}}%
\pgfusepath{stroke}%
\end{pgfscope}%
\begin{pgfscope}%
\definecolor{textcolor}{rgb}{0.150000,0.150000,0.150000}%
\pgfsetstrokecolor{textcolor}%
\pgfsetfillcolor{textcolor}%
\pgftext[x=1.726391in,y=0.234413in,,top]{\color{textcolor}\rmfamily\fontsize{10.000000}{12.000000}\selectfont 2013}%
\end{pgfscope}%
\begin{pgfscope}%
\pgfpathrectangle{\pgfqpoint{0.506467in}{0.331635in}}{\pgfqpoint{6.200000in}{2.265000in}}%
\pgfusepath{clip}%
\pgfsetroundcap%
\pgfsetroundjoin%
\pgfsetlinewidth{0.803000pt}%
\definecolor{currentstroke}{rgb}{1.000000,1.000000,1.000000}%
\pgfsetstrokecolor{currentstroke}%
\pgfsetdash{}{0pt}%
\pgfpathmoveto{\pgfqpoint{2.667073in}{0.331635in}}%
\pgfpathlineto{\pgfqpoint{2.667073in}{2.596635in}}%
\pgfusepath{stroke}%
\end{pgfscope}%
\begin{pgfscope}%
\definecolor{textcolor}{rgb}{0.150000,0.150000,0.150000}%
\pgfsetstrokecolor{textcolor}%
\pgfsetfillcolor{textcolor}%
\pgftext[x=2.667073in,y=0.234413in,,top]{\color{textcolor}\rmfamily\fontsize{10.000000}{12.000000}\selectfont 2014}%
\end{pgfscope}%
\begin{pgfscope}%
\pgfpathrectangle{\pgfqpoint{0.506467in}{0.331635in}}{\pgfqpoint{6.200000in}{2.265000in}}%
\pgfusepath{clip}%
\pgfsetroundcap%
\pgfsetroundjoin%
\pgfsetlinewidth{0.803000pt}%
\definecolor{currentstroke}{rgb}{1.000000,1.000000,1.000000}%
\pgfsetstrokecolor{currentstroke}%
\pgfsetdash{}{0pt}%
\pgfpathmoveto{\pgfqpoint{3.607756in}{0.331635in}}%
\pgfpathlineto{\pgfqpoint{3.607756in}{2.596635in}}%
\pgfusepath{stroke}%
\end{pgfscope}%
\begin{pgfscope}%
\definecolor{textcolor}{rgb}{0.150000,0.150000,0.150000}%
\pgfsetstrokecolor{textcolor}%
\pgfsetfillcolor{textcolor}%
\pgftext[x=3.607756in,y=0.234413in,,top]{\color{textcolor}\rmfamily\fontsize{10.000000}{12.000000}\selectfont 2015}%
\end{pgfscope}%
\begin{pgfscope}%
\pgfpathrectangle{\pgfqpoint{0.506467in}{0.331635in}}{\pgfqpoint{6.200000in}{2.265000in}}%
\pgfusepath{clip}%
\pgfsetroundcap%
\pgfsetroundjoin%
\pgfsetlinewidth{0.803000pt}%
\definecolor{currentstroke}{rgb}{1.000000,1.000000,1.000000}%
\pgfsetstrokecolor{currentstroke}%
\pgfsetdash{}{0pt}%
\pgfpathmoveto{\pgfqpoint{4.548438in}{0.331635in}}%
\pgfpathlineto{\pgfqpoint{4.548438in}{2.596635in}}%
\pgfusepath{stroke}%
\end{pgfscope}%
\begin{pgfscope}%
\definecolor{textcolor}{rgb}{0.150000,0.150000,0.150000}%
\pgfsetstrokecolor{textcolor}%
\pgfsetfillcolor{textcolor}%
\pgftext[x=4.548438in,y=0.234413in,,top]{\color{textcolor}\rmfamily\fontsize{10.000000}{12.000000}\selectfont 2016}%
\end{pgfscope}%
\begin{pgfscope}%
\pgfpathrectangle{\pgfqpoint{0.506467in}{0.331635in}}{\pgfqpoint{6.200000in}{2.265000in}}%
\pgfusepath{clip}%
\pgfsetroundcap%
\pgfsetroundjoin%
\pgfsetlinewidth{0.803000pt}%
\definecolor{currentstroke}{rgb}{1.000000,1.000000,1.000000}%
\pgfsetstrokecolor{currentstroke}%
\pgfsetdash{}{0pt}%
\pgfpathmoveto{\pgfqpoint{5.491698in}{0.331635in}}%
\pgfpathlineto{\pgfqpoint{5.491698in}{2.596635in}}%
\pgfusepath{stroke}%
\end{pgfscope}%
\begin{pgfscope}%
\definecolor{textcolor}{rgb}{0.150000,0.150000,0.150000}%
\pgfsetstrokecolor{textcolor}%
\pgfsetfillcolor{textcolor}%
\pgftext[x=5.491698in,y=0.234413in,,top]{\color{textcolor}\rmfamily\fontsize{10.000000}{12.000000}\selectfont 2017}%
\end{pgfscope}%
\begin{pgfscope}%
\pgfpathrectangle{\pgfqpoint{0.506467in}{0.331635in}}{\pgfqpoint{6.200000in}{2.265000in}}%
\pgfusepath{clip}%
\pgfsetroundcap%
\pgfsetroundjoin%
\pgfsetlinewidth{0.803000pt}%
\definecolor{currentstroke}{rgb}{1.000000,1.000000,1.000000}%
\pgfsetstrokecolor{currentstroke}%
\pgfsetdash{}{0pt}%
\pgfpathmoveto{\pgfqpoint{6.432381in}{0.331635in}}%
\pgfpathlineto{\pgfqpoint{6.432381in}{2.596635in}}%
\pgfusepath{stroke}%
\end{pgfscope}%
\begin{pgfscope}%
\definecolor{textcolor}{rgb}{0.150000,0.150000,0.150000}%
\pgfsetstrokecolor{textcolor}%
\pgfsetfillcolor{textcolor}%
\pgftext[x=6.432381in,y=0.234413in,,top]{\color{textcolor}\rmfamily\fontsize{10.000000}{12.000000}\selectfont 2018}%
\end{pgfscope}%
\begin{pgfscope}%
\pgfpathrectangle{\pgfqpoint{0.506467in}{0.331635in}}{\pgfqpoint{6.200000in}{2.265000in}}%
\pgfusepath{clip}%
\pgfsetroundcap%
\pgfsetroundjoin%
\pgfsetlinewidth{0.803000pt}%
\definecolor{currentstroke}{rgb}{1.000000,1.000000,1.000000}%
\pgfsetstrokecolor{currentstroke}%
\pgfsetdash{}{0pt}%
\pgfpathmoveto{\pgfqpoint{0.506467in}{0.574286in}}%
\pgfpathlineto{\pgfqpoint{6.706467in}{0.574286in}}%
\pgfusepath{stroke}%
\end{pgfscope}%
\begin{pgfscope}%
\definecolor{textcolor}{rgb}{0.150000,0.150000,0.150000}%
\pgfsetstrokecolor{textcolor}%
\pgfsetfillcolor{textcolor}%
\pgftext[x=0.100000in,y=0.521524in,left,base]{\color{textcolor}\rmfamily\fontsize{10.000000}{12.000000}\selectfont 1.00}%
\end{pgfscope}%
\begin{pgfscope}%
\pgfpathrectangle{\pgfqpoint{0.506467in}{0.331635in}}{\pgfqpoint{6.200000in}{2.265000in}}%
\pgfusepath{clip}%
\pgfsetroundcap%
\pgfsetroundjoin%
\pgfsetlinewidth{0.803000pt}%
\definecolor{currentstroke}{rgb}{1.000000,1.000000,1.000000}%
\pgfsetstrokecolor{currentstroke}%
\pgfsetdash{}{0pt}%
\pgfpathmoveto{\pgfqpoint{0.506467in}{0.905330in}}%
\pgfpathlineto{\pgfqpoint{6.706467in}{0.905330in}}%
\pgfusepath{stroke}%
\end{pgfscope}%
\begin{pgfscope}%
\definecolor{textcolor}{rgb}{0.150000,0.150000,0.150000}%
\pgfsetstrokecolor{textcolor}%
\pgfsetfillcolor{textcolor}%
\pgftext[x=0.100000in,y=0.852569in,left,base]{\color{textcolor}\rmfamily\fontsize{10.000000}{12.000000}\selectfont 1.25}%
\end{pgfscope}%
\begin{pgfscope}%
\pgfpathrectangle{\pgfqpoint{0.506467in}{0.331635in}}{\pgfqpoint{6.200000in}{2.265000in}}%
\pgfusepath{clip}%
\pgfsetroundcap%
\pgfsetroundjoin%
\pgfsetlinewidth{0.803000pt}%
\definecolor{currentstroke}{rgb}{1.000000,1.000000,1.000000}%
\pgfsetstrokecolor{currentstroke}%
\pgfsetdash{}{0pt}%
\pgfpathmoveto{\pgfqpoint{0.506467in}{1.236375in}}%
\pgfpathlineto{\pgfqpoint{6.706467in}{1.236375in}}%
\pgfusepath{stroke}%
\end{pgfscope}%
\begin{pgfscope}%
\definecolor{textcolor}{rgb}{0.150000,0.150000,0.150000}%
\pgfsetstrokecolor{textcolor}%
\pgfsetfillcolor{textcolor}%
\pgftext[x=0.100000in,y=1.183613in,left,base]{\color{textcolor}\rmfamily\fontsize{10.000000}{12.000000}\selectfont 1.50}%
\end{pgfscope}%
\begin{pgfscope}%
\pgfpathrectangle{\pgfqpoint{0.506467in}{0.331635in}}{\pgfqpoint{6.200000in}{2.265000in}}%
\pgfusepath{clip}%
\pgfsetroundcap%
\pgfsetroundjoin%
\pgfsetlinewidth{0.803000pt}%
\definecolor{currentstroke}{rgb}{1.000000,1.000000,1.000000}%
\pgfsetstrokecolor{currentstroke}%
\pgfsetdash{}{0pt}%
\pgfpathmoveto{\pgfqpoint{0.506467in}{1.567419in}}%
\pgfpathlineto{\pgfqpoint{6.706467in}{1.567419in}}%
\pgfusepath{stroke}%
\end{pgfscope}%
\begin{pgfscope}%
\definecolor{textcolor}{rgb}{0.150000,0.150000,0.150000}%
\pgfsetstrokecolor{textcolor}%
\pgfsetfillcolor{textcolor}%
\pgftext[x=0.100000in,y=1.514658in,left,base]{\color{textcolor}\rmfamily\fontsize{10.000000}{12.000000}\selectfont 1.75}%
\end{pgfscope}%
\begin{pgfscope}%
\pgfpathrectangle{\pgfqpoint{0.506467in}{0.331635in}}{\pgfqpoint{6.200000in}{2.265000in}}%
\pgfusepath{clip}%
\pgfsetroundcap%
\pgfsetroundjoin%
\pgfsetlinewidth{0.803000pt}%
\definecolor{currentstroke}{rgb}{1.000000,1.000000,1.000000}%
\pgfsetstrokecolor{currentstroke}%
\pgfsetdash{}{0pt}%
\pgfpathmoveto{\pgfqpoint{0.506467in}{1.898464in}}%
\pgfpathlineto{\pgfqpoint{6.706467in}{1.898464in}}%
\pgfusepath{stroke}%
\end{pgfscope}%
\begin{pgfscope}%
\definecolor{textcolor}{rgb}{0.150000,0.150000,0.150000}%
\pgfsetstrokecolor{textcolor}%
\pgfsetfillcolor{textcolor}%
\pgftext[x=0.100000in,y=1.845702in,left,base]{\color{textcolor}\rmfamily\fontsize{10.000000}{12.000000}\selectfont 2.00}%
\end{pgfscope}%
\begin{pgfscope}%
\pgfpathrectangle{\pgfqpoint{0.506467in}{0.331635in}}{\pgfqpoint{6.200000in}{2.265000in}}%
\pgfusepath{clip}%
\pgfsetroundcap%
\pgfsetroundjoin%
\pgfsetlinewidth{0.803000pt}%
\definecolor{currentstroke}{rgb}{1.000000,1.000000,1.000000}%
\pgfsetstrokecolor{currentstroke}%
\pgfsetdash{}{0pt}%
\pgfpathmoveto{\pgfqpoint{0.506467in}{2.229508in}}%
\pgfpathlineto{\pgfqpoint{6.706467in}{2.229508in}}%
\pgfusepath{stroke}%
\end{pgfscope}%
\begin{pgfscope}%
\definecolor{textcolor}{rgb}{0.150000,0.150000,0.150000}%
\pgfsetstrokecolor{textcolor}%
\pgfsetfillcolor{textcolor}%
\pgftext[x=0.100000in,y=2.176747in,left,base]{\color{textcolor}\rmfamily\fontsize{10.000000}{12.000000}\selectfont 2.25}%
\end{pgfscope}%
\begin{pgfscope}%
\pgfpathrectangle{\pgfqpoint{0.506467in}{0.331635in}}{\pgfqpoint{6.200000in}{2.265000in}}%
\pgfusepath{clip}%
\pgfsetroundcap%
\pgfsetroundjoin%
\pgfsetlinewidth{0.803000pt}%
\definecolor{currentstroke}{rgb}{1.000000,1.000000,1.000000}%
\pgfsetstrokecolor{currentstroke}%
\pgfsetdash{}{0pt}%
\pgfpathmoveto{\pgfqpoint{0.506467in}{2.560553in}}%
\pgfpathlineto{\pgfqpoint{6.706467in}{2.560553in}}%
\pgfusepath{stroke}%
\end{pgfscope}%
\begin{pgfscope}%
\definecolor{textcolor}{rgb}{0.150000,0.150000,0.150000}%
\pgfsetstrokecolor{textcolor}%
\pgfsetfillcolor{textcolor}%
\pgftext[x=0.100000in,y=2.507791in,left,base]{\color{textcolor}\rmfamily\fontsize{10.000000}{12.000000}\selectfont 2.50}%
\end{pgfscope}%
\begin{pgfscope}%
\pgfpathrectangle{\pgfqpoint{0.506467in}{0.331635in}}{\pgfqpoint{6.200000in}{2.265000in}}%
\pgfusepath{clip}%
\pgfsetroundcap%
\pgfsetroundjoin%
\pgfsetlinewidth{1.505625pt}%
\definecolor{currentstroke}{rgb}{0.121569,0.466667,0.705882}%
\pgfsetstrokecolor{currentstroke}%
\pgfsetdash{}{0pt}%
\pgfpathmoveto{\pgfqpoint{0.788285in}{0.574286in}}%
\pgfpathlineto{\pgfqpoint{0.793440in}{0.580332in}}%
\pgfpathlineto{\pgfqpoint{0.796017in}{0.575356in}}%
\pgfpathlineto{\pgfqpoint{0.803749in}{0.578843in}}%
\pgfpathlineto{\pgfqpoint{0.806326in}{0.583427in}}%
\pgfpathlineto{\pgfqpoint{0.808903in}{0.582453in}}%
\pgfpathlineto{\pgfqpoint{0.811480in}{0.589768in}}%
\pgfpathlineto{\pgfqpoint{0.814057in}{0.581286in}}%
\pgfpathlineto{\pgfqpoint{0.824366in}{0.587588in}}%
\pgfpathlineto{\pgfqpoint{0.826943in}{0.599090in}}%
\pgfpathlineto{\pgfqpoint{0.829521in}{0.602193in}}%
\pgfpathlineto{\pgfqpoint{0.832098in}{0.600206in}}%
\pgfpathlineto{\pgfqpoint{0.839830in}{0.592477in}}%
\pgfpathlineto{\pgfqpoint{0.842407in}{0.592994in}}%
\pgfpathlineto{\pgfqpoint{0.844984in}{0.599796in}}%
\pgfpathlineto{\pgfqpoint{0.847561in}{0.598436in}}%
\pgfpathlineto{\pgfqpoint{0.850138in}{0.596004in}}%
\pgfpathlineto{\pgfqpoint{0.857870in}{0.590095in}}%
\pgfpathlineto{\pgfqpoint{0.860447in}{0.591048in}}%
\pgfpathlineto{\pgfqpoint{0.863024in}{0.602088in}}%
\pgfpathlineto{\pgfqpoint{0.865602in}{0.605275in}}%
\pgfpathlineto{\pgfqpoint{0.868179in}{0.619682in}}%
\pgfpathlineto{\pgfqpoint{0.875910in}{0.621292in}}%
\pgfpathlineto{\pgfqpoint{0.878488in}{0.623766in}}%
\pgfpathlineto{\pgfqpoint{0.881065in}{0.629163in}}%
\pgfpathlineto{\pgfqpoint{0.883642in}{0.639947in}}%
\pgfpathlineto{\pgfqpoint{0.886219in}{0.633754in}}%
\pgfpathlineto{\pgfqpoint{0.893951in}{0.641744in}}%
\pgfpathlineto{\pgfqpoint{0.896528in}{0.643115in}}%
\pgfpathlineto{\pgfqpoint{0.899105in}{0.635567in}}%
\pgfpathlineto{\pgfqpoint{0.904260in}{0.655285in}}%
\pgfpathlineto{\pgfqpoint{0.914569in}{0.653133in}}%
\pgfpathlineto{\pgfqpoint{0.917146in}{0.651427in}}%
\pgfpathlineto{\pgfqpoint{0.922300in}{0.659901in}}%
\pgfpathlineto{\pgfqpoint{0.930032in}{0.661372in}}%
\pgfpathlineto{\pgfqpoint{0.932609in}{0.668657in}}%
\pgfpathlineto{\pgfqpoint{0.935186in}{0.661402in}}%
\pgfpathlineto{\pgfqpoint{0.937764in}{0.664856in}}%
\pgfpathlineto{\pgfqpoint{0.940341in}{0.662665in}}%
\pgfpathlineto{\pgfqpoint{0.948072in}{0.660182in}}%
\pgfpathlineto{\pgfqpoint{0.950650in}{0.640882in}}%
\pgfpathlineto{\pgfqpoint{0.953227in}{0.649494in}}%
\pgfpathlineto{\pgfqpoint{0.955804in}{0.663347in}}%
\pgfpathlineto{\pgfqpoint{0.966113in}{0.668621in}}%
\pgfpathlineto{\pgfqpoint{0.968690in}{0.692952in}}%
\pgfpathlineto{\pgfqpoint{0.971267in}{0.696795in}}%
\pgfpathlineto{\pgfqpoint{0.973845in}{0.704630in}}%
\pgfpathlineto{\pgfqpoint{0.976422in}{0.699237in}}%
\pgfpathlineto{\pgfqpoint{0.984153in}{0.703985in}}%
\pgfpathlineto{\pgfqpoint{0.986731in}{0.695034in}}%
\pgfpathlineto{\pgfqpoint{0.989308in}{0.694265in}}%
\pgfpathlineto{\pgfqpoint{0.991885in}{0.692614in}}%
\pgfpathlineto{\pgfqpoint{0.994462in}{0.693452in}}%
\pgfpathlineto{\pgfqpoint{1.002194in}{0.709642in}}%
\pgfpathlineto{\pgfqpoint{1.004771in}{0.704327in}}%
\pgfpathlineto{\pgfqpoint{1.007348in}{0.696531in}}%
\pgfpathlineto{\pgfqpoint{1.009926in}{0.693716in}}%
\pgfpathlineto{\pgfqpoint{1.012503in}{0.699408in}}%
\pgfpathlineto{\pgfqpoint{1.020234in}{0.704264in}}%
\pgfpathlineto{\pgfqpoint{1.022812in}{0.700145in}}%
\pgfpathlineto{\pgfqpoint{1.025389in}{0.689166in}}%
\pgfpathlineto{\pgfqpoint{1.027966in}{0.690501in}}%
\pgfpathlineto{\pgfqpoint{1.038275in}{0.672643in}}%
\pgfpathlineto{\pgfqpoint{1.040852in}{0.647125in}}%
\pgfpathlineto{\pgfqpoint{1.043429in}{0.660722in}}%
\pgfpathlineto{\pgfqpoint{1.046007in}{0.680981in}}%
\pgfpathlineto{\pgfqpoint{1.048584in}{0.668108in}}%
\pgfpathlineto{\pgfqpoint{1.056315in}{0.672850in}}%
\pgfpathlineto{\pgfqpoint{1.058893in}{0.688765in}}%
\pgfpathlineto{\pgfqpoint{1.061470in}{0.679254in}}%
\pgfpathlineto{\pgfqpoint{1.064047in}{0.674052in}}%
\pgfpathlineto{\pgfqpoint{1.066624in}{0.683518in}}%
\pgfpathlineto{\pgfqpoint{1.074356in}{0.669740in}}%
\pgfpathlineto{\pgfqpoint{1.079510in}{0.695744in}}%
\pgfpathlineto{\pgfqpoint{1.082087in}{0.710981in}}%
\pgfpathlineto{\pgfqpoint{1.084665in}{0.711430in}}%
\pgfpathlineto{\pgfqpoint{1.092396in}{0.707350in}}%
\pgfpathlineto{\pgfqpoint{1.094974in}{0.716831in}}%
\pgfpathlineto{\pgfqpoint{1.097551in}{0.717504in}}%
\pgfpathlineto{\pgfqpoint{1.100128in}{0.707752in}}%
\pgfpathlineto{\pgfqpoint{1.102705in}{0.692472in}}%
\pgfpathlineto{\pgfqpoint{1.110437in}{0.694315in}}%
\pgfpathlineto{\pgfqpoint{1.113014in}{0.691786in}}%
\pgfpathlineto{\pgfqpoint{1.115591in}{0.681215in}}%
\pgfpathlineto{\pgfqpoint{1.118168in}{0.687549in}}%
\pgfpathlineto{\pgfqpoint{1.120746in}{0.689914in}}%
\pgfpathlineto{\pgfqpoint{1.128477in}{0.671973in}}%
\pgfpathlineto{\pgfqpoint{1.131055in}{0.668413in}}%
\pgfpathlineto{\pgfqpoint{1.133632in}{0.671501in}}%
\pgfpathlineto{\pgfqpoint{1.136209in}{0.655626in}}%
\pgfpathlineto{\pgfqpoint{1.138786in}{0.646140in}}%
\pgfpathlineto{\pgfqpoint{1.146518in}{0.659943in}}%
\pgfpathlineto{\pgfqpoint{1.149095in}{0.661993in}}%
\pgfpathlineto{\pgfqpoint{1.151672in}{0.656948in}}%
\pgfpathlineto{\pgfqpoint{1.154249in}{0.662120in}}%
\pgfpathlineto{\pgfqpoint{1.156827in}{0.658003in}}%
\pgfpathlineto{\pgfqpoint{1.167135in}{0.674870in}}%
\pgfpathlineto{\pgfqpoint{1.169713in}{0.657756in}}%
\pgfpathlineto{\pgfqpoint{1.172290in}{0.658205in}}%
\pgfpathlineto{\pgfqpoint{1.174867in}{0.624860in}}%
\pgfpathlineto{\pgfqpoint{1.182599in}{0.624402in}}%
\pgfpathlineto{\pgfqpoint{1.185176in}{0.626821in}}%
\pgfpathlineto{\pgfqpoint{1.187753in}{0.657458in}}%
\pgfpathlineto{\pgfqpoint{1.190330in}{0.664708in}}%
\pgfpathlineto{\pgfqpoint{1.192908in}{0.676097in}}%
\pgfpathlineto{\pgfqpoint{1.200639in}{0.664514in}}%
\pgfpathlineto{\pgfqpoint{1.203216in}{0.682426in}}%
\pgfpathlineto{\pgfqpoint{1.205794in}{0.675378in}}%
\pgfpathlineto{\pgfqpoint{1.208371in}{0.693504in}}%
\pgfpathlineto{\pgfqpoint{1.210948in}{0.704769in}}%
\pgfpathlineto{\pgfqpoint{1.218680in}{0.705296in}}%
\pgfpathlineto{\pgfqpoint{1.221257in}{0.717305in}}%
\pgfpathlineto{\pgfqpoint{1.223834in}{0.715588in}}%
\pgfpathlineto{\pgfqpoint{1.226411in}{0.693884in}}%
\pgfpathlineto{\pgfqpoint{1.228989in}{0.709001in}}%
\pgfpathlineto{\pgfqpoint{1.236720in}{0.686438in}}%
\pgfpathlineto{\pgfqpoint{1.239297in}{0.693127in}}%
\pgfpathlineto{\pgfqpoint{1.241875in}{0.705976in}}%
\pgfpathlineto{\pgfqpoint{1.244452in}{0.699914in}}%
\pgfpathlineto{\pgfqpoint{1.247029in}{0.732698in}}%
\pgfpathlineto{\pgfqpoint{1.254761in}{0.737642in}}%
\pgfpathlineto{\pgfqpoint{1.257338in}{0.741682in}}%
\pgfpathlineto{\pgfqpoint{1.262492in}{0.736438in}}%
\pgfpathlineto{\pgfqpoint{1.265070in}{0.724108in}}%
\pgfpathlineto{\pgfqpoint{1.272801in}{0.724303in}}%
\pgfpathlineto{\pgfqpoint{1.275378in}{0.713171in}}%
\pgfpathlineto{\pgfqpoint{1.277956in}{0.705685in}}%
\pgfpathlineto{\pgfqpoint{1.280533in}{0.700658in}}%
\pgfpathlineto{\pgfqpoint{1.283110in}{0.725153in}}%
\pgfpathlineto{\pgfqpoint{1.290842in}{0.726752in}}%
\pgfpathlineto{\pgfqpoint{1.295996in}{0.752811in}}%
\pgfpathlineto{\pgfqpoint{1.301151in}{0.728958in}}%
\pgfpathlineto{\pgfqpoint{1.308882in}{0.720153in}}%
\pgfpathlineto{\pgfqpoint{1.311459in}{0.706946in}}%
\pgfpathlineto{\pgfqpoint{1.314037in}{0.711395in}}%
\pgfpathlineto{\pgfqpoint{1.316614in}{0.741826in}}%
\pgfpathlineto{\pgfqpoint{1.319191in}{0.761853in}}%
\pgfpathlineto{\pgfqpoint{1.326923in}{0.761216in}}%
\pgfpathlineto{\pgfqpoint{1.329500in}{0.752556in}}%
\pgfpathlineto{\pgfqpoint{1.332077in}{0.748675in}}%
\pgfpathlineto{\pgfqpoint{1.334654in}{0.741098in}}%
\pgfpathlineto{\pgfqpoint{1.337232in}{0.767757in}}%
\pgfpathlineto{\pgfqpoint{1.344963in}{0.766808in}}%
\pgfpathlineto{\pgfqpoint{1.347540in}{0.772527in}}%
\pgfpathlineto{\pgfqpoint{1.350118in}{0.774880in}}%
\pgfpathlineto{\pgfqpoint{1.352695in}{0.766938in}}%
\pgfpathlineto{\pgfqpoint{1.355272in}{0.770881in}}%
\pgfpathlineto{\pgfqpoint{1.363004in}{0.766760in}}%
\pgfpathlineto{\pgfqpoint{1.368158in}{0.769674in}}%
\pgfpathlineto{\pgfqpoint{1.370735in}{0.779629in}}%
\pgfpathlineto{\pgfqpoint{1.373312in}{0.780585in}}%
\pgfpathlineto{\pgfqpoint{1.381044in}{0.773883in}}%
\pgfpathlineto{\pgfqpoint{1.383621in}{0.763383in}}%
\pgfpathlineto{\pgfqpoint{1.386199in}{0.762021in}}%
\pgfpathlineto{\pgfqpoint{1.388776in}{0.749513in}}%
\pgfpathlineto{\pgfqpoint{1.391353in}{0.761488in}}%
\pgfpathlineto{\pgfqpoint{1.404239in}{0.761529in}}%
\pgfpathlineto{\pgfqpoint{1.406816in}{0.748986in}}%
\pgfpathlineto{\pgfqpoint{1.409393in}{0.762558in}}%
\pgfpathlineto{\pgfqpoint{1.422280in}{0.757972in}}%
\pgfpathlineto{\pgfqpoint{1.424857in}{0.782863in}}%
\pgfpathlineto{\pgfqpoint{1.435166in}{0.766277in}}%
\pgfpathlineto{\pgfqpoint{1.440320in}{0.776731in}}%
\pgfpathlineto{\pgfqpoint{1.442897in}{0.799366in}}%
\pgfpathlineto{\pgfqpoint{1.445474in}{0.801748in}}%
\pgfpathlineto{\pgfqpoint{1.455783in}{0.798023in}}%
\pgfpathlineto{\pgfqpoint{1.458361in}{0.803792in}}%
\pgfpathlineto{\pgfqpoint{1.460938in}{0.804428in}}%
\pgfpathlineto{\pgfqpoint{1.463515in}{0.803873in}}%
\pgfpathlineto{\pgfqpoint{1.471247in}{0.799335in}}%
\pgfpathlineto{\pgfqpoint{1.473824in}{0.791775in}}%
\pgfpathlineto{\pgfqpoint{1.476401in}{0.781987in}}%
\pgfpathlineto{\pgfqpoint{1.478978in}{0.794563in}}%
\pgfpathlineto{\pgfqpoint{1.481555in}{0.790443in}}%
\pgfpathlineto{\pgfqpoint{1.489287in}{0.799289in}}%
\pgfpathlineto{\pgfqpoint{1.491864in}{0.795024in}}%
\pgfpathlineto{\pgfqpoint{1.499596in}{0.820975in}}%
\pgfpathlineto{\pgfqpoint{1.507328in}{0.812627in}}%
\pgfpathlineto{\pgfqpoint{1.509905in}{0.791054in}}%
\pgfpathlineto{\pgfqpoint{1.512482in}{0.781433in}}%
\pgfpathlineto{\pgfqpoint{1.515059in}{0.778362in}}%
\pgfpathlineto{\pgfqpoint{1.517636in}{0.773740in}}%
\pgfpathlineto{\pgfqpoint{1.525368in}{0.780349in}}%
\pgfpathlineto{\pgfqpoint{1.530522in}{0.812153in}}%
\pgfpathlineto{\pgfqpoint{1.533100in}{0.813911in}}%
\pgfpathlineto{\pgfqpoint{1.535677in}{0.788987in}}%
\pgfpathlineto{\pgfqpoint{1.543409in}{0.783784in}}%
\pgfpathlineto{\pgfqpoint{1.545986in}{0.757841in}}%
\pgfpathlineto{\pgfqpoint{1.548563in}{0.758721in}}%
\pgfpathlineto{\pgfqpoint{1.551140in}{0.764041in}}%
\pgfpathlineto{\pgfqpoint{1.553717in}{0.767783in}}%
\pgfpathlineto{\pgfqpoint{1.566603in}{0.761671in}}%
\pgfpathlineto{\pgfqpoint{1.569181in}{0.785848in}}%
\pgfpathlineto{\pgfqpoint{1.571758in}{0.777689in}}%
\pgfpathlineto{\pgfqpoint{1.579489in}{0.775029in}}%
\pgfpathlineto{\pgfqpoint{1.582067in}{0.788426in}}%
\pgfpathlineto{\pgfqpoint{1.584644in}{0.758389in}}%
\pgfpathlineto{\pgfqpoint{1.587221in}{0.744921in}}%
\pgfpathlineto{\pgfqpoint{1.589798in}{0.738903in}}%
\pgfpathlineto{\pgfqpoint{1.597530in}{0.740806in}}%
\pgfpathlineto{\pgfqpoint{1.600107in}{0.733495in}}%
\pgfpathlineto{\pgfqpoint{1.602684in}{0.710988in}}%
\pgfpathlineto{\pgfqpoint{1.605262in}{0.711650in}}%
\pgfpathlineto{\pgfqpoint{1.615570in}{0.742960in}}%
\pgfpathlineto{\pgfqpoint{1.618148in}{0.743249in}}%
\pgfpathlineto{\pgfqpoint{1.620725in}{0.746521in}}%
\pgfpathlineto{\pgfqpoint{1.625879in}{0.767202in}}%
\pgfpathlineto{\pgfqpoint{1.633611in}{0.761627in}}%
\pgfpathlineto{\pgfqpoint{1.636188in}{0.753357in}}%
\pgfpathlineto{\pgfqpoint{1.638765in}{0.768767in}}%
\pgfpathlineto{\pgfqpoint{1.641343in}{0.769068in}}%
\pgfpathlineto{\pgfqpoint{1.643920in}{0.774089in}}%
\pgfpathlineto{\pgfqpoint{1.651651in}{0.767254in}}%
\pgfpathlineto{\pgfqpoint{1.654229in}{0.768100in}}%
\pgfpathlineto{\pgfqpoint{1.659383in}{0.783794in}}%
\pgfpathlineto{\pgfqpoint{1.661960in}{0.789120in}}%
\pgfpathlineto{\pgfqpoint{1.669692in}{0.788263in}}%
\pgfpathlineto{\pgfqpoint{1.672269in}{0.801884in}}%
\pgfpathlineto{\pgfqpoint{1.674846in}{0.805034in}}%
\pgfpathlineto{\pgfqpoint{1.677424in}{0.795219in}}%
\pgfpathlineto{\pgfqpoint{1.680001in}{0.788179in}}%
\pgfpathlineto{\pgfqpoint{1.687732in}{0.798204in}}%
\pgfpathlineto{\pgfqpoint{1.690310in}{0.810488in}}%
\pgfpathlineto{\pgfqpoint{1.692887in}{0.797587in}}%
\pgfpathlineto{\pgfqpoint{1.695464in}{0.811891in}}%
\pgfpathlineto{\pgfqpoint{1.698041in}{0.797553in}}%
\pgfpathlineto{\pgfqpoint{1.705773in}{0.794335in}}%
\pgfpathlineto{\pgfqpoint{1.710927in}{0.789808in}}%
\pgfpathlineto{\pgfqpoint{1.713505in}{0.784690in}}%
\pgfpathlineto{\pgfqpoint{1.716082in}{0.769727in}}%
\pgfpathlineto{\pgfqpoint{1.723813in}{0.792325in}}%
\pgfpathlineto{\pgfqpoint{1.728968in}{0.827985in}}%
\pgfpathlineto{\pgfqpoint{1.731545in}{0.825292in}}%
\pgfpathlineto{\pgfqpoint{1.734122in}{0.836209in}}%
\pgfpathlineto{\pgfqpoint{1.741854in}{0.833873in}}%
\pgfpathlineto{\pgfqpoint{1.744431in}{0.826698in}}%
\pgfpathlineto{\pgfqpoint{1.749586in}{0.847076in}}%
\pgfpathlineto{\pgfqpoint{1.752163in}{0.847921in}}%
\pgfpathlineto{\pgfqpoint{1.759894in}{0.848576in}}%
\pgfpathlineto{\pgfqpoint{1.762472in}{0.847030in}}%
\pgfpathlineto{\pgfqpoint{1.765049in}{0.847383in}}%
\pgfpathlineto{\pgfqpoint{1.767626in}{0.861878in}}%
\pgfpathlineto{\pgfqpoint{1.770203in}{0.857873in}}%
\pgfpathlineto{\pgfqpoint{1.780512in}{0.860290in}}%
\pgfpathlineto{\pgfqpoint{1.783089in}{0.864700in}}%
\pgfpathlineto{\pgfqpoint{1.785666in}{0.866746in}}%
\pgfpathlineto{\pgfqpoint{1.788244in}{0.882487in}}%
\pgfpathlineto{\pgfqpoint{1.795975in}{0.880855in}}%
\pgfpathlineto{\pgfqpoint{1.798553in}{0.891949in}}%
\pgfpathlineto{\pgfqpoint{1.801130in}{0.885009in}}%
\pgfpathlineto{\pgfqpoint{1.803707in}{0.881337in}}%
\pgfpathlineto{\pgfqpoint{1.806284in}{0.903858in}}%
\pgfpathlineto{\pgfqpoint{1.814016in}{0.891058in}}%
\pgfpathlineto{\pgfqpoint{1.816593in}{0.905289in}}%
\pgfpathlineto{\pgfqpoint{1.819170in}{0.910249in}}%
\pgfpathlineto{\pgfqpoint{1.821747in}{0.907695in}}%
\pgfpathlineto{\pgfqpoint{1.824325in}{0.910682in}}%
\pgfpathlineto{\pgfqpoint{1.832056in}{0.908753in}}%
\pgfpathlineto{\pgfqpoint{1.837211in}{0.923200in}}%
\pgfpathlineto{\pgfqpoint{1.839788in}{0.925139in}}%
\pgfpathlineto{\pgfqpoint{1.842365in}{0.929424in}}%
\pgfpathlineto{\pgfqpoint{1.852674in}{0.940125in}}%
\pgfpathlineto{\pgfqpoint{1.855251in}{0.924933in}}%
\pgfpathlineto{\pgfqpoint{1.857828in}{0.920226in}}%
\pgfpathlineto{\pgfqpoint{1.860406in}{0.931984in}}%
\pgfpathlineto{\pgfqpoint{1.868137in}{0.909495in}}%
\pgfpathlineto{\pgfqpoint{1.870714in}{0.921626in}}%
\pgfpathlineto{\pgfqpoint{1.873292in}{0.940393in}}%
\pgfpathlineto{\pgfqpoint{1.875869in}{0.936541in}}%
\pgfpathlineto{\pgfqpoint{1.878446in}{0.941516in}}%
\pgfpathlineto{\pgfqpoint{1.886178in}{0.948643in}}%
\pgfpathlineto{\pgfqpoint{1.888755in}{0.968466in}}%
\pgfpathlineto{\pgfqpoint{1.891332in}{0.971642in}}%
\pgfpathlineto{\pgfqpoint{1.893909in}{0.973709in}}%
\pgfpathlineto{\pgfqpoint{1.896487in}{0.981758in}}%
\pgfpathlineto{\pgfqpoint{1.904218in}{0.987167in}}%
\pgfpathlineto{\pgfqpoint{1.906795in}{0.982629in}}%
\pgfpathlineto{\pgfqpoint{1.909373in}{0.981273in}}%
\pgfpathlineto{\pgfqpoint{1.911950in}{0.992074in}}%
\pgfpathlineto{\pgfqpoint{1.914527in}{0.983786in}}%
\pgfpathlineto{\pgfqpoint{1.924836in}{0.973758in}}%
\pgfpathlineto{\pgfqpoint{1.927413in}{0.985506in}}%
\pgfpathlineto{\pgfqpoint{1.929990in}{0.973947in}}%
\pgfpathlineto{\pgfqpoint{1.932568in}{0.989945in}}%
\pgfpathlineto{\pgfqpoint{1.940299in}{0.985378in}}%
\pgfpathlineto{\pgfqpoint{1.942876in}{1.003757in}}%
\pgfpathlineto{\pgfqpoint{1.945454in}{1.001633in}}%
\pgfpathlineto{\pgfqpoint{1.948031in}{1.009400in}}%
\pgfpathlineto{\pgfqpoint{1.958340in}{1.002952in}}%
\pgfpathlineto{\pgfqpoint{1.960917in}{1.016242in}}%
\pgfpathlineto{\pgfqpoint{1.963494in}{0.997103in}}%
\pgfpathlineto{\pgfqpoint{1.966071in}{1.005550in}}%
\pgfpathlineto{\pgfqpoint{1.968649in}{0.996327in}}%
\pgfpathlineto{\pgfqpoint{1.976380in}{1.007022in}}%
\pgfpathlineto{\pgfqpoint{1.978957in}{1.011573in}}%
\pgfpathlineto{\pgfqpoint{1.981535in}{1.034774in}}%
\pgfpathlineto{\pgfqpoint{1.984112in}{1.040365in}}%
\pgfpathlineto{\pgfqpoint{1.986689in}{1.037849in}}%
\pgfpathlineto{\pgfqpoint{1.994421in}{1.005223in}}%
\pgfpathlineto{\pgfqpoint{1.996998in}{1.029118in}}%
\pgfpathlineto{\pgfqpoint{1.999575in}{1.013593in}}%
\pgfpathlineto{\pgfqpoint{2.002152in}{1.016788in}}%
\pgfpathlineto{\pgfqpoint{2.004730in}{1.039988in}}%
\pgfpathlineto{\pgfqpoint{2.012461in}{1.040850in}}%
\pgfpathlineto{\pgfqpoint{2.015038in}{1.057731in}}%
\pgfpathlineto{\pgfqpoint{2.017616in}{1.047672in}}%
\pgfpathlineto{\pgfqpoint{2.020193in}{1.050105in}}%
\pgfpathlineto{\pgfqpoint{2.022770in}{1.048083in}}%
\pgfpathlineto{\pgfqpoint{2.033079in}{1.061061in}}%
\pgfpathlineto{\pgfqpoint{2.035656in}{1.050932in}}%
\pgfpathlineto{\pgfqpoint{2.040811in}{1.096764in}}%
\pgfpathlineto{\pgfqpoint{2.048542in}{1.091331in}}%
\pgfpathlineto{\pgfqpoint{2.051119in}{1.104683in}}%
\pgfpathlineto{\pgfqpoint{2.053697in}{1.110521in}}%
\pgfpathlineto{\pgfqpoint{2.056274in}{1.111622in}}%
\pgfpathlineto{\pgfqpoint{2.058851in}{1.120185in}}%
\pgfpathlineto{\pgfqpoint{2.066583in}{1.114316in}}%
\pgfpathlineto{\pgfqpoint{2.069160in}{1.129067in}}%
\pgfpathlineto{\pgfqpoint{2.071737in}{1.148744in}}%
\pgfpathlineto{\pgfqpoint{2.074314in}{1.135055in}}%
\pgfpathlineto{\pgfqpoint{2.076891in}{1.151833in}}%
\pgfpathlineto{\pgfqpoint{2.084623in}{1.147369in}}%
\pgfpathlineto{\pgfqpoint{2.087200in}{1.148840in}}%
\pgfpathlineto{\pgfqpoint{2.089778in}{1.140020in}}%
\pgfpathlineto{\pgfqpoint{2.092355in}{1.133530in}}%
\pgfpathlineto{\pgfqpoint{2.094932in}{1.140041in}}%
\pgfpathlineto{\pgfqpoint{2.105241in}{1.148490in}}%
\pgfpathlineto{\pgfqpoint{2.107818in}{1.130276in}}%
\pgfpathlineto{\pgfqpoint{2.110395in}{1.130238in}}%
\pgfpathlineto{\pgfqpoint{2.112972in}{1.104066in}}%
\pgfpathlineto{\pgfqpoint{2.120704in}{1.125309in}}%
\pgfpathlineto{\pgfqpoint{2.123281in}{1.122779in}}%
\pgfpathlineto{\pgfqpoint{2.125859in}{1.095581in}}%
\pgfpathlineto{\pgfqpoint{2.128436in}{1.109788in}}%
\pgfpathlineto{\pgfqpoint{2.131013in}{1.135139in}}%
\pgfpathlineto{\pgfqpoint{2.138745in}{1.138950in}}%
\pgfpathlineto{\pgfqpoint{2.141322in}{1.123932in}}%
\pgfpathlineto{\pgfqpoint{2.143899in}{1.105567in}}%
\pgfpathlineto{\pgfqpoint{2.146476in}{1.133512in}}%
\pgfpathlineto{\pgfqpoint{2.149053in}{1.123257in}}%
\pgfpathlineto{\pgfqpoint{2.156785in}{1.138207in}}%
\pgfpathlineto{\pgfqpoint{2.159362in}{1.161514in}}%
\pgfpathlineto{\pgfqpoint{2.161940in}{1.130323in}}%
\pgfpathlineto{\pgfqpoint{2.164517in}{1.082329in}}%
\pgfpathlineto{\pgfqpoint{2.167094in}{1.097340in}}%
\pgfpathlineto{\pgfqpoint{2.174826in}{1.078470in}}%
\pgfpathlineto{\pgfqpoint{2.179980in}{1.115399in}}%
\pgfpathlineto{\pgfqpoint{2.182557in}{1.125824in}}%
\pgfpathlineto{\pgfqpoint{2.185134in}{1.113562in}}%
\pgfpathlineto{\pgfqpoint{2.192866in}{1.126902in}}%
\pgfpathlineto{\pgfqpoint{2.195443in}{1.117237in}}%
\pgfpathlineto{\pgfqpoint{2.198020in}{1.125489in}}%
\pgfpathlineto{\pgfqpoint{2.203175in}{1.149353in}}%
\pgfpathlineto{\pgfqpoint{2.210907in}{1.150917in}}%
\pgfpathlineto{\pgfqpoint{2.213484in}{1.160069in}}%
\pgfpathlineto{\pgfqpoint{2.216061in}{1.155909in}}%
\pgfpathlineto{\pgfqpoint{2.218638in}{1.186281in}}%
\pgfpathlineto{\pgfqpoint{2.221215in}{1.190456in}}%
\pgfpathlineto{\pgfqpoint{2.228947in}{1.184295in}}%
\pgfpathlineto{\pgfqpoint{2.231524in}{1.181339in}}%
\pgfpathlineto{\pgfqpoint{2.234101in}{1.181759in}}%
\pgfpathlineto{\pgfqpoint{2.236679in}{1.170913in}}%
\pgfpathlineto{\pgfqpoint{2.239256in}{1.185479in}}%
\pgfpathlineto{\pgfqpoint{2.246988in}{1.185581in}}%
\pgfpathlineto{\pgfqpoint{2.249565in}{1.187312in}}%
\pgfpathlineto{\pgfqpoint{2.252142in}{1.186759in}}%
\pgfpathlineto{\pgfqpoint{2.254719in}{1.199369in}}%
\pgfpathlineto{\pgfqpoint{2.257296in}{1.202195in}}%
\pgfpathlineto{\pgfqpoint{2.265028in}{1.199133in}}%
\pgfpathlineto{\pgfqpoint{2.267605in}{1.196891in}}%
\pgfpathlineto{\pgfqpoint{2.270182in}{1.176323in}}%
\pgfpathlineto{\pgfqpoint{2.272760in}{1.196926in}}%
\pgfpathlineto{\pgfqpoint{2.275337in}{1.208691in}}%
\pgfpathlineto{\pgfqpoint{2.283068in}{1.202732in}}%
\pgfpathlineto{\pgfqpoint{2.288223in}{1.192486in}}%
\pgfpathlineto{\pgfqpoint{2.290800in}{1.192233in}}%
\pgfpathlineto{\pgfqpoint{2.293377in}{1.180010in}}%
\pgfpathlineto{\pgfqpoint{2.301109in}{1.179338in}}%
\pgfpathlineto{\pgfqpoint{2.303686in}{1.182309in}}%
\pgfpathlineto{\pgfqpoint{2.306263in}{1.170536in}}%
\pgfpathlineto{\pgfqpoint{2.308841in}{1.141087in}}%
\pgfpathlineto{\pgfqpoint{2.311418in}{1.134095in}}%
\pgfpathlineto{\pgfqpoint{2.319149in}{1.134830in}}%
\pgfpathlineto{\pgfqpoint{2.321727in}{1.133093in}}%
\pgfpathlineto{\pgfqpoint{2.324304in}{1.122323in}}%
\pgfpathlineto{\pgfqpoint{2.326881in}{1.130669in}}%
\pgfpathlineto{\pgfqpoint{2.329458in}{1.136095in}}%
\pgfpathlineto{\pgfqpoint{2.337190in}{1.118779in}}%
\pgfpathlineto{\pgfqpoint{2.339767in}{1.096333in}}%
\pgfpathlineto{\pgfqpoint{2.342344in}{1.096066in}}%
\pgfpathlineto{\pgfqpoint{2.344922in}{1.102613in}}%
\pgfpathlineto{\pgfqpoint{2.347499in}{1.098740in}}%
\pgfpathlineto{\pgfqpoint{2.357808in}{1.101739in}}%
\pgfpathlineto{\pgfqpoint{2.360385in}{1.115747in}}%
\pgfpathlineto{\pgfqpoint{2.362962in}{1.117714in}}%
\pgfpathlineto{\pgfqpoint{2.365539in}{1.116453in}}%
\pgfpathlineto{\pgfqpoint{2.373271in}{1.130226in}}%
\pgfpathlineto{\pgfqpoint{2.375848in}{1.155642in}}%
\pgfpathlineto{\pgfqpoint{2.378425in}{1.169857in}}%
\pgfpathlineto{\pgfqpoint{2.381003in}{1.172143in}}%
\pgfpathlineto{\pgfqpoint{2.383580in}{1.190326in}}%
\pgfpathlineto{\pgfqpoint{2.391311in}{1.204043in}}%
\pgfpathlineto{\pgfqpoint{2.396466in}{1.233967in}}%
\pgfpathlineto{\pgfqpoint{2.399043in}{1.230407in}}%
\pgfpathlineto{\pgfqpoint{2.401620in}{1.214544in}}%
\pgfpathlineto{\pgfqpoint{2.409352in}{1.210830in}}%
\pgfpathlineto{\pgfqpoint{2.414506in}{1.189948in}}%
\pgfpathlineto{\pgfqpoint{2.417084in}{1.198180in}}%
\pgfpathlineto{\pgfqpoint{2.419661in}{1.184465in}}%
\pgfpathlineto{\pgfqpoint{2.427392in}{1.168250in}}%
\pgfpathlineto{\pgfqpoint{2.429970in}{1.177854in}}%
\pgfpathlineto{\pgfqpoint{2.432547in}{1.168787in}}%
\pgfpathlineto{\pgfqpoint{2.435124in}{1.153103in}}%
\pgfpathlineto{\pgfqpoint{2.437701in}{1.165733in}}%
\pgfpathlineto{\pgfqpoint{2.445433in}{1.151872in}}%
\pgfpathlineto{\pgfqpoint{2.448010in}{1.131290in}}%
\pgfpathlineto{\pgfqpoint{2.450587in}{1.135223in}}%
\pgfpathlineto{\pgfqpoint{2.453165in}{1.179947in}}%
\pgfpathlineto{\pgfqpoint{2.455742in}{1.199759in}}%
\pgfpathlineto{\pgfqpoint{2.463473in}{1.207210in}}%
\pgfpathlineto{\pgfqpoint{2.466051in}{1.190582in}}%
\pgfpathlineto{\pgfqpoint{2.468628in}{1.214182in}}%
\pgfpathlineto{\pgfqpoint{2.471205in}{1.248678in}}%
\pgfpathlineto{\pgfqpoint{2.473782in}{1.264161in}}%
\pgfpathlineto{\pgfqpoint{2.481514in}{1.272233in}}%
\pgfpathlineto{\pgfqpoint{2.484091in}{1.282575in}}%
\pgfpathlineto{\pgfqpoint{2.486668in}{1.272019in}}%
\pgfpathlineto{\pgfqpoint{2.491823in}{1.291681in}}%
\pgfpathlineto{\pgfqpoint{2.499554in}{1.295278in}}%
\pgfpathlineto{\pgfqpoint{2.502132in}{1.308824in}}%
\pgfpathlineto{\pgfqpoint{2.504709in}{1.299722in}}%
\pgfpathlineto{\pgfqpoint{2.507286in}{1.287481in}}%
\pgfpathlineto{\pgfqpoint{2.509863in}{1.299254in}}%
\pgfpathlineto{\pgfqpoint{2.517595in}{1.297478in}}%
\pgfpathlineto{\pgfqpoint{2.520172in}{1.293408in}}%
\pgfpathlineto{\pgfqpoint{2.522749in}{1.311312in}}%
\pgfpathlineto{\pgfqpoint{2.525326in}{1.288237in}}%
\pgfpathlineto{\pgfqpoint{2.527904in}{1.309254in}}%
\pgfpathlineto{\pgfqpoint{2.538213in}{1.305816in}}%
\pgfpathlineto{\pgfqpoint{2.543367in}{1.328514in}}%
\pgfpathlineto{\pgfqpoint{2.545944in}{1.337808in}}%
\pgfpathlineto{\pgfqpoint{2.553676in}{1.336204in}}%
\pgfpathlineto{\pgfqpoint{2.556253in}{1.332722in}}%
\pgfpathlineto{\pgfqpoint{2.558830in}{1.330684in}}%
\pgfpathlineto{\pgfqpoint{2.561407in}{1.349698in}}%
\pgfpathlineto{\pgfqpoint{2.563985in}{1.344230in}}%
\pgfpathlineto{\pgfqpoint{2.571716in}{1.342421in}}%
\pgfpathlineto{\pgfqpoint{2.574293in}{1.348853in}}%
\pgfpathlineto{\pgfqpoint{2.576871in}{1.353601in}}%
\pgfpathlineto{\pgfqpoint{2.582025in}{1.347708in}}%
\pgfpathlineto{\pgfqpoint{2.589757in}{1.333265in}}%
\pgfpathlineto{\pgfqpoint{2.592334in}{1.322127in}}%
\pgfpathlineto{\pgfqpoint{2.597488in}{1.319558in}}%
\pgfpathlineto{\pgfqpoint{2.600066in}{1.352387in}}%
\pgfpathlineto{\pgfqpoint{2.607797in}{1.354585in}}%
\pgfpathlineto{\pgfqpoint{2.610374in}{1.343544in}}%
\pgfpathlineto{\pgfqpoint{2.615529in}{1.311235in}}%
\pgfpathlineto{\pgfqpoint{2.618106in}{1.314666in}}%
\pgfpathlineto{\pgfqpoint{2.628415in}{1.331371in}}%
\pgfpathlineto{\pgfqpoint{2.630992in}{1.373172in}}%
\pgfpathlineto{\pgfqpoint{2.636147in}{1.377615in}}%
\pgfpathlineto{\pgfqpoint{2.643878in}{1.387642in}}%
\pgfpathlineto{\pgfqpoint{2.646455in}{1.396895in}}%
\pgfpathlineto{\pgfqpoint{2.651610in}{1.412991in}}%
\pgfpathlineto{\pgfqpoint{2.654187in}{1.413276in}}%
\pgfpathlineto{\pgfqpoint{2.661919in}{1.423302in}}%
\pgfpathlineto{\pgfqpoint{2.664496in}{1.430356in}}%
\pgfpathlineto{\pgfqpoint{2.669650in}{1.409549in}}%
\pgfpathlineto{\pgfqpoint{2.672228in}{1.410217in}}%
\pgfpathlineto{\pgfqpoint{2.679959in}{1.404941in}}%
\pgfpathlineto{\pgfqpoint{2.682536in}{1.419299in}}%
\pgfpathlineto{\pgfqpoint{2.685114in}{1.409538in}}%
\pgfpathlineto{\pgfqpoint{2.687691in}{1.403329in}}%
\pgfpathlineto{\pgfqpoint{2.690268in}{1.403619in}}%
\pgfpathlineto{\pgfqpoint{2.698000in}{1.380645in}}%
\pgfpathlineto{\pgfqpoint{2.700577in}{1.405378in}}%
\pgfpathlineto{\pgfqpoint{2.703154in}{1.422840in}}%
\pgfpathlineto{\pgfqpoint{2.705731in}{1.417326in}}%
\pgfpathlineto{\pgfqpoint{2.708309in}{1.421044in}}%
\pgfpathlineto{\pgfqpoint{2.718617in}{1.412591in}}%
\pgfpathlineto{\pgfqpoint{2.721195in}{1.412681in}}%
\pgfpathlineto{\pgfqpoint{2.723772in}{1.389981in}}%
\pgfpathlineto{\pgfqpoint{2.726349in}{1.346106in}}%
\pgfpathlineto{\pgfqpoint{2.734081in}{1.335906in}}%
\pgfpathlineto{\pgfqpoint{2.736658in}{1.351310in}}%
\pgfpathlineto{\pgfqpoint{2.739235in}{1.331786in}}%
\pgfpathlineto{\pgfqpoint{2.741812in}{1.343724in}}%
\pgfpathlineto{\pgfqpoint{2.752121in}{1.272492in}}%
\pgfpathlineto{\pgfqpoint{2.754698in}{1.285195in}}%
\pgfpathlineto{\pgfqpoint{2.757276in}{1.287332in}}%
\pgfpathlineto{\pgfqpoint{2.759853in}{1.323629in}}%
\pgfpathlineto{\pgfqpoint{2.762430in}{1.343095in}}%
\pgfpathlineto{\pgfqpoint{2.770162in}{1.353916in}}%
\pgfpathlineto{\pgfqpoint{2.772739in}{1.375510in}}%
\pgfpathlineto{\pgfqpoint{2.777893in}{1.380370in}}%
\pgfpathlineto{\pgfqpoint{2.780470in}{1.394695in}}%
\pgfpathlineto{\pgfqpoint{2.790779in}{1.386579in}}%
\pgfpathlineto{\pgfqpoint{2.793357in}{1.378209in}}%
\pgfpathlineto{\pgfqpoint{2.795934in}{1.392510in}}%
\pgfpathlineto{\pgfqpoint{2.798511in}{1.386565in}}%
\pgfpathlineto{\pgfqpoint{2.806243in}{1.395559in}}%
\pgfpathlineto{\pgfqpoint{2.811397in}{1.395824in}}%
\pgfpathlineto{\pgfqpoint{2.813974in}{1.409849in}}%
\pgfpathlineto{\pgfqpoint{2.816551in}{1.417434in}}%
\pgfpathlineto{\pgfqpoint{2.824283in}{1.390891in}}%
\pgfpathlineto{\pgfqpoint{2.826860in}{1.424369in}}%
\pgfpathlineto{\pgfqpoint{2.829438in}{1.420420in}}%
\pgfpathlineto{\pgfqpoint{2.832015in}{1.432760in}}%
\pgfpathlineto{\pgfqpoint{2.834592in}{1.433166in}}%
\pgfpathlineto{\pgfqpoint{2.842324in}{1.431323in}}%
\pgfpathlineto{\pgfqpoint{2.844901in}{1.421231in}}%
\pgfpathlineto{\pgfqpoint{2.847478in}{1.419430in}}%
\pgfpathlineto{\pgfqpoint{2.850055in}{1.388638in}}%
\pgfpathlineto{\pgfqpoint{2.852632in}{1.382388in}}%
\pgfpathlineto{\pgfqpoint{2.862941in}{1.419942in}}%
\pgfpathlineto{\pgfqpoint{2.865519in}{1.401334in}}%
\pgfpathlineto{\pgfqpoint{2.868096in}{1.411340in}}%
\pgfpathlineto{\pgfqpoint{2.870673in}{1.414260in}}%
\pgfpathlineto{\pgfqpoint{2.878405in}{1.407977in}}%
\pgfpathlineto{\pgfqpoint{2.880982in}{1.423555in}}%
\pgfpathlineto{\pgfqpoint{2.883559in}{1.408792in}}%
\pgfpathlineto{\pgfqpoint{2.886136in}{1.412526in}}%
\pgfpathlineto{\pgfqpoint{2.888713in}{1.418126in}}%
\pgfpathlineto{\pgfqpoint{2.896445in}{1.436556in}}%
\pgfpathlineto{\pgfqpoint{2.899022in}{1.446423in}}%
\pgfpathlineto{\pgfqpoint{2.901599in}{1.449197in}}%
\pgfpathlineto{\pgfqpoint{2.904177in}{1.458406in}}%
\pgfpathlineto{\pgfqpoint{2.906754in}{1.434935in}}%
\pgfpathlineto{\pgfqpoint{2.914486in}{1.417064in}}%
\pgfpathlineto{\pgfqpoint{2.917063in}{1.425175in}}%
\pgfpathlineto{\pgfqpoint{2.919640in}{1.446134in}}%
\pgfpathlineto{\pgfqpoint{2.922217in}{1.399330in}}%
\pgfpathlineto{\pgfqpoint{2.924794in}{1.381726in}}%
\pgfpathlineto{\pgfqpoint{2.932526in}{1.400627in}}%
\pgfpathlineto{\pgfqpoint{2.935103in}{1.414293in}}%
\pgfpathlineto{\pgfqpoint{2.937680in}{1.441332in}}%
\pgfpathlineto{\pgfqpoint{2.940258in}{1.449800in}}%
\pgfpathlineto{\pgfqpoint{2.950567in}{1.453765in}}%
\pgfpathlineto{\pgfqpoint{2.953144in}{1.457040in}}%
\pgfpathlineto{\pgfqpoint{2.958298in}{1.445375in}}%
\pgfpathlineto{\pgfqpoint{2.960875in}{1.422192in}}%
\pgfpathlineto{\pgfqpoint{2.971184in}{1.444120in}}%
\pgfpathlineto{\pgfqpoint{2.973761in}{1.452437in}}%
\pgfpathlineto{\pgfqpoint{2.976339in}{1.452297in}}%
\pgfpathlineto{\pgfqpoint{2.978916in}{1.445862in}}%
\pgfpathlineto{\pgfqpoint{2.986647in}{1.455022in}}%
\pgfpathlineto{\pgfqpoint{2.989225in}{1.440746in}}%
\pgfpathlineto{\pgfqpoint{2.991802in}{1.464389in}}%
\pgfpathlineto{\pgfqpoint{2.994379in}{1.471679in}}%
\pgfpathlineto{\pgfqpoint{3.004688in}{1.484678in}}%
\pgfpathlineto{\pgfqpoint{3.007265in}{1.483019in}}%
\pgfpathlineto{\pgfqpoint{3.009842in}{1.471642in}}%
\pgfpathlineto{\pgfqpoint{3.012420in}{1.454332in}}%
\pgfpathlineto{\pgfqpoint{3.014997in}{1.458567in}}%
\pgfpathlineto{\pgfqpoint{3.022728in}{1.464952in}}%
\pgfpathlineto{\pgfqpoint{3.025306in}{1.449216in}}%
\pgfpathlineto{\pgfqpoint{3.027883in}{1.469717in}}%
\pgfpathlineto{\pgfqpoint{3.030460in}{1.472527in}}%
\pgfpathlineto{\pgfqpoint{3.033037in}{1.484065in}}%
\pgfpathlineto{\pgfqpoint{3.043346in}{1.497286in}}%
\pgfpathlineto{\pgfqpoint{3.045923in}{1.497363in}}%
\pgfpathlineto{\pgfqpoint{3.051078in}{1.513150in}}%
\pgfpathlineto{\pgfqpoint{3.058809in}{1.515974in}}%
\pgfpathlineto{\pgfqpoint{3.063964in}{1.509069in}}%
\pgfpathlineto{\pgfqpoint{3.069118in}{1.539368in}}%
\pgfpathlineto{\pgfqpoint{3.076850in}{1.547235in}}%
\pgfpathlineto{\pgfqpoint{3.079427in}{1.548262in}}%
\pgfpathlineto{\pgfqpoint{3.082004in}{1.535148in}}%
\pgfpathlineto{\pgfqpoint{3.084582in}{1.518892in}}%
\pgfpathlineto{\pgfqpoint{3.087159in}{1.535365in}}%
\pgfpathlineto{\pgfqpoint{3.097468in}{1.534227in}}%
\pgfpathlineto{\pgfqpoint{3.100045in}{1.542985in}}%
\pgfpathlineto{\pgfqpoint{3.102622in}{1.549470in}}%
\pgfpathlineto{\pgfqpoint{3.105199in}{1.552210in}}%
\pgfpathlineto{\pgfqpoint{3.112931in}{1.545512in}}%
\pgfpathlineto{\pgfqpoint{3.115508in}{1.533885in}}%
\pgfpathlineto{\pgfqpoint{3.118085in}{1.546328in}}%
\pgfpathlineto{\pgfqpoint{3.120663in}{1.541249in}}%
\pgfpathlineto{\pgfqpoint{3.123240in}{1.549610in}}%
\pgfpathlineto{\pgfqpoint{3.130971in}{1.543755in}}%
\pgfpathlineto{\pgfqpoint{3.133549in}{1.562310in}}%
\pgfpathlineto{\pgfqpoint{3.136126in}{1.566012in}}%
\pgfpathlineto{\pgfqpoint{3.138703in}{1.575446in}}%
\pgfpathlineto{\pgfqpoint{3.149012in}{1.573537in}}%
\pgfpathlineto{\pgfqpoint{3.151589in}{1.557564in}}%
\pgfpathlineto{\pgfqpoint{3.154166in}{1.567730in}}%
\pgfpathlineto{\pgfqpoint{3.156744in}{1.568069in}}%
\pgfpathlineto{\pgfqpoint{3.159321in}{1.574955in}}%
\pgfpathlineto{\pgfqpoint{3.167052in}{1.586607in}}%
\pgfpathlineto{\pgfqpoint{3.169630in}{1.582686in}}%
\pgfpathlineto{\pgfqpoint{3.172207in}{1.607284in}}%
\pgfpathlineto{\pgfqpoint{3.174784in}{1.571228in}}%
\pgfpathlineto{\pgfqpoint{3.177361in}{1.585424in}}%
\pgfpathlineto{\pgfqpoint{3.185093in}{1.576557in}}%
\pgfpathlineto{\pgfqpoint{3.187670in}{1.587029in}}%
\pgfpathlineto{\pgfqpoint{3.190247in}{1.578966in}}%
\pgfpathlineto{\pgfqpoint{3.192824in}{1.581296in}}%
\pgfpathlineto{\pgfqpoint{3.195402in}{1.565735in}}%
\pgfpathlineto{\pgfqpoint{3.203133in}{1.567013in}}%
\pgfpathlineto{\pgfqpoint{3.205711in}{1.555317in}}%
\pgfpathlineto{\pgfqpoint{3.208288in}{1.555382in}}%
\pgfpathlineto{\pgfqpoint{3.210865in}{1.513025in}}%
\pgfpathlineto{\pgfqpoint{3.213442in}{1.511068in}}%
\pgfpathlineto{\pgfqpoint{3.221174in}{1.521784in}}%
\pgfpathlineto{\pgfqpoint{3.223751in}{1.503948in}}%
\pgfpathlineto{\pgfqpoint{3.226328in}{1.509055in}}%
\pgfpathlineto{\pgfqpoint{3.228905in}{1.496376in}}%
\pgfpathlineto{\pgfqpoint{3.231483in}{1.517884in}}%
\pgfpathlineto{\pgfqpoint{3.239214in}{1.522907in}}%
\pgfpathlineto{\pgfqpoint{3.241792in}{1.519527in}}%
\pgfpathlineto{\pgfqpoint{3.244369in}{1.540184in}}%
\pgfpathlineto{\pgfqpoint{3.246946in}{1.546770in}}%
\pgfpathlineto{\pgfqpoint{3.249523in}{1.539282in}}%
\pgfpathlineto{\pgfqpoint{3.257255in}{1.566635in}}%
\pgfpathlineto{\pgfqpoint{3.259832in}{1.571629in}}%
\pgfpathlineto{\pgfqpoint{3.264986in}{1.595056in}}%
\pgfpathlineto{\pgfqpoint{3.267564in}{1.587438in}}%
\pgfpathlineto{\pgfqpoint{3.275295in}{1.593526in}}%
\pgfpathlineto{\pgfqpoint{3.277872in}{1.590431in}}%
\pgfpathlineto{\pgfqpoint{3.280450in}{1.590885in}}%
\pgfpathlineto{\pgfqpoint{3.283027in}{1.583801in}}%
\pgfpathlineto{\pgfqpoint{3.285604in}{1.585094in}}%
\pgfpathlineto{\pgfqpoint{3.295913in}{1.588055in}}%
\pgfpathlineto{\pgfqpoint{3.298490in}{1.591522in}}%
\pgfpathlineto{\pgfqpoint{3.301067in}{1.591168in}}%
\pgfpathlineto{\pgfqpoint{3.303645in}{1.597436in}}%
\pgfpathlineto{\pgfqpoint{3.311376in}{1.595301in}}%
\pgfpathlineto{\pgfqpoint{3.313953in}{1.578994in}}%
\pgfpathlineto{\pgfqpoint{3.316531in}{1.587892in}}%
\pgfpathlineto{\pgfqpoint{3.319108in}{1.587093in}}%
\pgfpathlineto{\pgfqpoint{3.321685in}{1.574699in}}%
\pgfpathlineto{\pgfqpoint{3.329417in}{1.579254in}}%
\pgfpathlineto{\pgfqpoint{3.331994in}{1.596688in}}%
\pgfpathlineto{\pgfqpoint{3.334571in}{1.599448in}}%
\pgfpathlineto{\pgfqpoint{3.337148in}{1.614358in}}%
\pgfpathlineto{\pgfqpoint{3.339726in}{1.616764in}}%
\pgfpathlineto{\pgfqpoint{3.347457in}{1.600538in}}%
\pgfpathlineto{\pgfqpoint{3.350034in}{1.584427in}}%
\pgfpathlineto{\pgfqpoint{3.352612in}{1.599348in}}%
\pgfpathlineto{\pgfqpoint{3.355189in}{1.564028in}}%
\pgfpathlineto{\pgfqpoint{3.357766in}{1.575999in}}%
\pgfpathlineto{\pgfqpoint{3.365498in}{1.573942in}}%
\pgfpathlineto{\pgfqpoint{3.368075in}{1.577009in}}%
\pgfpathlineto{\pgfqpoint{3.370652in}{1.538409in}}%
\pgfpathlineto{\pgfqpoint{3.373229in}{1.528887in}}%
\pgfpathlineto{\pgfqpoint{3.375807in}{1.557055in}}%
\pgfpathlineto{\pgfqpoint{3.383538in}{1.555993in}}%
\pgfpathlineto{\pgfqpoint{3.386115in}{1.516608in}}%
\pgfpathlineto{\pgfqpoint{3.388693in}{1.559442in}}%
\pgfpathlineto{\pgfqpoint{3.391270in}{1.512844in}}%
\pgfpathlineto{\pgfqpoint{3.393847in}{1.485955in}}%
\pgfpathlineto{\pgfqpoint{3.401579in}{1.453444in}}%
\pgfpathlineto{\pgfqpoint{3.404156in}{1.457421in}}%
\pgfpathlineto{\pgfqpoint{3.406733in}{1.437943in}}%
\pgfpathlineto{\pgfqpoint{3.409310in}{1.430159in}}%
\pgfpathlineto{\pgfqpoint{3.411888in}{1.471963in}}%
\pgfpathlineto{\pgfqpoint{3.419619in}{1.491218in}}%
\pgfpathlineto{\pgfqpoint{3.422196in}{1.529095in}}%
\pgfpathlineto{\pgfqpoint{3.424774in}{1.512153in}}%
\pgfpathlineto{\pgfqpoint{3.429928in}{1.566332in}}%
\pgfpathlineto{\pgfqpoint{3.437660in}{1.575248in}}%
\pgfpathlineto{\pgfqpoint{3.440237in}{1.605903in}}%
\pgfpathlineto{\pgfqpoint{3.442814in}{1.603068in}}%
\pgfpathlineto{\pgfqpoint{3.445391in}{1.628360in}}%
\pgfpathlineto{\pgfqpoint{3.447969in}{1.659535in}}%
\pgfpathlineto{\pgfqpoint{3.455700in}{1.661961in}}%
\pgfpathlineto{\pgfqpoint{3.463432in}{1.694269in}}%
\pgfpathlineto{\pgfqpoint{3.466009in}{1.695307in}}%
\pgfpathlineto{\pgfqpoint{3.473741in}{1.694281in}}%
\pgfpathlineto{\pgfqpoint{3.476318in}{1.689901in}}%
\pgfpathlineto{\pgfqpoint{3.478895in}{1.697091in}}%
\pgfpathlineto{\pgfqpoint{3.481472in}{1.699001in}}%
\pgfpathlineto{\pgfqpoint{3.484049in}{1.696239in}}%
\pgfpathlineto{\pgfqpoint{3.491781in}{1.696894in}}%
\pgfpathlineto{\pgfqpoint{3.494358in}{1.710889in}}%
\pgfpathlineto{\pgfqpoint{3.496936in}{1.703618in}}%
\pgfpathlineto{\pgfqpoint{3.502090in}{1.718439in}}%
\pgfpathlineto{\pgfqpoint{3.509822in}{1.718831in}}%
\pgfpathlineto{\pgfqpoint{3.512399in}{1.728942in}}%
\pgfpathlineto{\pgfqpoint{3.514976in}{1.735822in}}%
\pgfpathlineto{\pgfqpoint{3.520130in}{1.751927in}}%
\pgfpathlineto{\pgfqpoint{3.527862in}{1.738529in}}%
\pgfpathlineto{\pgfqpoint{3.530439in}{1.750877in}}%
\pgfpathlineto{\pgfqpoint{3.533017in}{1.748035in}}%
\pgfpathlineto{\pgfqpoint{3.535594in}{1.748596in}}%
\pgfpathlineto{\pgfqpoint{3.538171in}{1.755997in}}%
\pgfpathlineto{\pgfqpoint{3.545903in}{1.756583in}}%
\pgfpathlineto{\pgfqpoint{3.548480in}{1.744314in}}%
\pgfpathlineto{\pgfqpoint{3.551057in}{1.713373in}}%
\pgfpathlineto{\pgfqpoint{3.553634in}{1.728276in}}%
\pgfpathlineto{\pgfqpoint{3.556211in}{1.688566in}}%
\pgfpathlineto{\pgfqpoint{3.563943in}{1.675460in}}%
\pgfpathlineto{\pgfqpoint{3.566520in}{1.666611in}}%
\pgfpathlineto{\pgfqpoint{3.569098in}{1.702919in}}%
\pgfpathlineto{\pgfqpoint{3.571675in}{1.757832in}}%
\pgfpathlineto{\pgfqpoint{3.574252in}{1.750225in}}%
\pgfpathlineto{\pgfqpoint{3.581984in}{1.777667in}}%
\pgfpathlineto{\pgfqpoint{3.584561in}{1.780277in}}%
\pgfpathlineto{\pgfqpoint{3.587138in}{1.781744in}}%
\pgfpathlineto{\pgfqpoint{3.592292in}{1.786021in}}%
\pgfpathlineto{\pgfqpoint{3.600024in}{1.779484in}}%
\pgfpathlineto{\pgfqpoint{3.602601in}{1.767874in}}%
\pgfpathlineto{\pgfqpoint{3.605178in}{1.741577in}}%
\pgfpathlineto{\pgfqpoint{3.610333in}{1.740106in}}%
\pgfpathlineto{\pgfqpoint{3.618065in}{1.702259in}}%
\pgfpathlineto{\pgfqpoint{3.620642in}{1.678366in}}%
\pgfpathlineto{\pgfqpoint{3.623219in}{1.704149in}}%
\pgfpathlineto{\pgfqpoint{3.625796in}{1.741008in}}%
\pgfpathlineto{\pgfqpoint{3.628373in}{1.719934in}}%
\pgfpathlineto{\pgfqpoint{3.636105in}{1.715412in}}%
\pgfpathlineto{\pgfqpoint{3.638682in}{1.716943in}}%
\pgfpathlineto{\pgfqpoint{3.641259in}{1.695088in}}%
\pgfpathlineto{\pgfqpoint{3.643837in}{1.683074in}}%
\pgfpathlineto{\pgfqpoint{3.646414in}{1.710553in}}%
\pgfpathlineto{\pgfqpoint{3.656723in}{1.712080in}}%
\pgfpathlineto{\pgfqpoint{3.661877in}{1.734866in}}%
\pgfpathlineto{\pgfqpoint{3.664454in}{1.717627in}}%
\pgfpathlineto{\pgfqpoint{3.672186in}{1.706482in}}%
\pgfpathlineto{\pgfqpoint{3.677340in}{1.641657in}}%
\pgfpathlineto{\pgfqpoint{3.679918in}{1.657082in}}%
\pgfpathlineto{\pgfqpoint{3.682495in}{1.623768in}}%
\pgfpathlineto{\pgfqpoint{3.690226in}{1.655798in}}%
\pgfpathlineto{\pgfqpoint{3.692804in}{1.690889in}}%
\pgfpathlineto{\pgfqpoint{3.695381in}{1.704787in}}%
\pgfpathlineto{\pgfqpoint{3.697958in}{1.736773in}}%
\pgfpathlineto{\pgfqpoint{3.700535in}{1.727286in}}%
\pgfpathlineto{\pgfqpoint{3.708267in}{1.714384in}}%
\pgfpathlineto{\pgfqpoint{3.710844in}{1.729114in}}%
\pgfpathlineto{\pgfqpoint{3.713421in}{1.733751in}}%
\pgfpathlineto{\pgfqpoint{3.715999in}{1.730958in}}%
\pgfpathlineto{\pgfqpoint{3.728885in}{1.742002in}}%
\pgfpathlineto{\pgfqpoint{3.731462in}{1.741468in}}%
\pgfpathlineto{\pgfqpoint{3.734039in}{1.736656in}}%
\pgfpathlineto{\pgfqpoint{3.736616in}{1.753781in}}%
\pgfpathlineto{\pgfqpoint{3.744348in}{1.751817in}}%
\pgfpathlineto{\pgfqpoint{3.746925in}{1.763755in}}%
\pgfpathlineto{\pgfqpoint{3.749502in}{1.772118in}}%
\pgfpathlineto{\pgfqpoint{3.752080in}{1.774929in}}%
\pgfpathlineto{\pgfqpoint{3.754657in}{1.761133in}}%
\pgfpathlineto{\pgfqpoint{3.762388in}{1.788620in}}%
\pgfpathlineto{\pgfqpoint{3.764966in}{1.775601in}}%
\pgfpathlineto{\pgfqpoint{3.767543in}{1.756884in}}%
\pgfpathlineto{\pgfqpoint{3.770120in}{1.758567in}}%
\pgfpathlineto{\pgfqpoint{3.772697in}{1.718342in}}%
\pgfpathlineto{\pgfqpoint{3.780429in}{1.732946in}}%
\pgfpathlineto{\pgfqpoint{3.783006in}{1.679432in}}%
\pgfpathlineto{\pgfqpoint{3.785583in}{1.683227in}}%
\pgfpathlineto{\pgfqpoint{3.788161in}{1.716083in}}%
\pgfpathlineto{\pgfqpoint{3.790738in}{1.695376in}}%
\pgfpathlineto{\pgfqpoint{3.798469in}{1.727038in}}%
\pgfpathlineto{\pgfqpoint{3.801047in}{1.707246in}}%
\pgfpathlineto{\pgfqpoint{3.803624in}{1.734384in}}%
\pgfpathlineto{\pgfqpoint{3.806201in}{1.722571in}}%
\pgfpathlineto{\pgfqpoint{3.808778in}{1.744457in}}%
\pgfpathlineto{\pgfqpoint{3.816510in}{1.741083in}}%
\pgfpathlineto{\pgfqpoint{3.819087in}{1.725226in}}%
\pgfpathlineto{\pgfqpoint{3.821664in}{1.680237in}}%
\pgfpathlineto{\pgfqpoint{3.824242in}{1.670241in}}%
\pgfpathlineto{\pgfqpoint{3.826819in}{1.689925in}}%
\pgfpathlineto{\pgfqpoint{3.834550in}{1.705102in}}%
\pgfpathlineto{\pgfqpoint{3.837128in}{1.685805in}}%
\pgfpathlineto{\pgfqpoint{3.839705in}{1.678404in}}%
\pgfpathlineto{\pgfqpoint{3.842282in}{1.690645in}}%
\pgfpathlineto{\pgfqpoint{3.852591in}{1.700443in}}%
\pgfpathlineto{\pgfqpoint{3.855168in}{1.697338in}}%
\pgfpathlineto{\pgfqpoint{3.860323in}{1.719866in}}%
\pgfpathlineto{\pgfqpoint{3.862900in}{1.759189in}}%
\pgfpathlineto{\pgfqpoint{3.870631in}{1.736587in}}%
\pgfpathlineto{\pgfqpoint{3.873209in}{1.738307in}}%
\pgfpathlineto{\pgfqpoint{3.875786in}{1.750615in}}%
\pgfpathlineto{\pgfqpoint{3.878363in}{1.751111in}}%
\pgfpathlineto{\pgfqpoint{3.880940in}{1.712520in}}%
\pgfpathlineto{\pgfqpoint{3.888672in}{1.728660in}}%
\pgfpathlineto{\pgfqpoint{3.891249in}{1.724117in}}%
\pgfpathlineto{\pgfqpoint{3.893826in}{1.746549in}}%
\pgfpathlineto{\pgfqpoint{3.896403in}{1.734923in}}%
\pgfpathlineto{\pgfqpoint{3.898981in}{1.730362in}}%
\pgfpathlineto{\pgfqpoint{3.906712in}{1.728133in}}%
\pgfpathlineto{\pgfqpoint{3.909290in}{1.736017in}}%
\pgfpathlineto{\pgfqpoint{3.911867in}{1.728836in}}%
\pgfpathlineto{\pgfqpoint{3.914444in}{1.711331in}}%
\pgfpathlineto{\pgfqpoint{3.917021in}{1.734178in}}%
\pgfpathlineto{\pgfqpoint{3.924753in}{1.741713in}}%
\pgfpathlineto{\pgfqpoint{3.927330in}{1.724124in}}%
\pgfpathlineto{\pgfqpoint{3.929907in}{1.714044in}}%
\pgfpathlineto{\pgfqpoint{3.932484in}{1.725103in}}%
\pgfpathlineto{\pgfqpoint{3.935062in}{1.763867in}}%
\pgfpathlineto{\pgfqpoint{3.942793in}{1.748610in}}%
\pgfpathlineto{\pgfqpoint{3.945371in}{1.740858in}}%
\pgfpathlineto{\pgfqpoint{3.947948in}{1.751460in}}%
\pgfpathlineto{\pgfqpoint{3.950525in}{1.778979in}}%
\pgfpathlineto{\pgfqpoint{3.953102in}{1.776033in}}%
\pgfpathlineto{\pgfqpoint{3.960834in}{1.781380in}}%
\pgfpathlineto{\pgfqpoint{3.963411in}{1.787546in}}%
\pgfpathlineto{\pgfqpoint{3.965988in}{1.786400in}}%
\pgfpathlineto{\pgfqpoint{3.968565in}{1.787508in}}%
\pgfpathlineto{\pgfqpoint{3.971143in}{1.780223in}}%
\pgfpathlineto{\pgfqpoint{3.981451in}{1.756146in}}%
\pgfpathlineto{\pgfqpoint{3.984029in}{1.772369in}}%
\pgfpathlineto{\pgfqpoint{3.986606in}{1.775645in}}%
\pgfpathlineto{\pgfqpoint{3.989183in}{1.761090in}}%
\pgfpathlineto{\pgfqpoint{3.996915in}{1.758958in}}%
\pgfpathlineto{\pgfqpoint{3.999492in}{1.754892in}}%
\pgfpathlineto{\pgfqpoint{4.002069in}{1.759421in}}%
\pgfpathlineto{\pgfqpoint{4.004646in}{1.731012in}}%
\pgfpathlineto{\pgfqpoint{4.007224in}{1.717740in}}%
\pgfpathlineto{\pgfqpoint{4.014955in}{1.704402in}}%
\pgfpathlineto{\pgfqpoint{4.017532in}{1.709934in}}%
\pgfpathlineto{\pgfqpoint{4.020110in}{1.739822in}}%
\pgfpathlineto{\pgfqpoint{4.022687in}{1.745808in}}%
\pgfpathlineto{\pgfqpoint{4.025264in}{1.724618in}}%
\pgfpathlineto{\pgfqpoint{4.032996in}{1.704973in}}%
\pgfpathlineto{\pgfqpoint{4.035573in}{1.719963in}}%
\pgfpathlineto{\pgfqpoint{4.038150in}{1.730106in}}%
\pgfpathlineto{\pgfqpoint{4.040727in}{1.759468in}}%
\pgfpathlineto{\pgfqpoint{4.043305in}{1.742750in}}%
\pgfpathlineto{\pgfqpoint{4.051036in}{1.755258in}}%
\pgfpathlineto{\pgfqpoint{4.053613in}{1.757222in}}%
\pgfpathlineto{\pgfqpoint{4.056191in}{1.737080in}}%
\pgfpathlineto{\pgfqpoint{4.058768in}{1.729376in}}%
\pgfpathlineto{\pgfqpoint{4.061345in}{1.728392in}}%
\pgfpathlineto{\pgfqpoint{4.069077in}{1.679570in}}%
\pgfpathlineto{\pgfqpoint{4.071654in}{1.681374in}}%
\pgfpathlineto{\pgfqpoint{4.074231in}{1.699559in}}%
\pgfpathlineto{\pgfqpoint{4.076808in}{1.702539in}}%
\pgfpathlineto{\pgfqpoint{4.087117in}{1.691260in}}%
\pgfpathlineto{\pgfqpoint{4.089694in}{1.703376in}}%
\pgfpathlineto{\pgfqpoint{4.092272in}{1.668671in}}%
\pgfpathlineto{\pgfqpoint{4.094849in}{1.669742in}}%
\pgfpathlineto{\pgfqpoint{4.097426in}{1.695974in}}%
\pgfpathlineto{\pgfqpoint{4.105158in}{1.725056in}}%
\pgfpathlineto{\pgfqpoint{4.107735in}{1.729819in}}%
\pgfpathlineto{\pgfqpoint{4.110312in}{1.730079in}}%
\pgfpathlineto{\pgfqpoint{4.112889in}{1.745779in}}%
\pgfpathlineto{\pgfqpoint{4.115467in}{1.739769in}}%
\pgfpathlineto{\pgfqpoint{4.123198in}{1.747223in}}%
\pgfpathlineto{\pgfqpoint{4.125775in}{1.709912in}}%
\pgfpathlineto{\pgfqpoint{4.128353in}{1.700001in}}%
\pgfpathlineto{\pgfqpoint{4.130930in}{1.677327in}}%
\pgfpathlineto{\pgfqpoint{4.133507in}{1.665129in}}%
\pgfpathlineto{\pgfqpoint{4.141239in}{1.655738in}}%
\pgfpathlineto{\pgfqpoint{4.146393in}{1.691923in}}%
\pgfpathlineto{\pgfqpoint{4.151548in}{1.681930in}}%
\pgfpathlineto{\pgfqpoint{4.159279in}{1.679365in}}%
\pgfpathlineto{\pgfqpoint{4.161856in}{1.674948in}}%
\pgfpathlineto{\pgfqpoint{4.167011in}{1.638560in}}%
\pgfpathlineto{\pgfqpoint{4.169588in}{1.649049in}}%
\pgfpathlineto{\pgfqpoint{4.177320in}{1.684928in}}%
\pgfpathlineto{\pgfqpoint{4.179897in}{1.657461in}}%
\pgfpathlineto{\pgfqpoint{4.182474in}{1.663550in}}%
\pgfpathlineto{\pgfqpoint{4.185051in}{1.653851in}}%
\pgfpathlineto{\pgfqpoint{4.187628in}{1.660080in}}%
\pgfpathlineto{\pgfqpoint{4.195360in}{1.672472in}}%
\pgfpathlineto{\pgfqpoint{4.197937in}{1.661831in}}%
\pgfpathlineto{\pgfqpoint{4.200515in}{1.643179in}}%
\pgfpathlineto{\pgfqpoint{4.205669in}{1.538283in}}%
\pgfpathlineto{\pgfqpoint{4.213401in}{1.471754in}}%
\pgfpathlineto{\pgfqpoint{4.215978in}{1.436416in}}%
\pgfpathlineto{\pgfqpoint{4.218555in}{1.523068in}}%
\pgfpathlineto{\pgfqpoint{4.221132in}{1.567875in}}%
\pgfpathlineto{\pgfqpoint{4.223709in}{1.571689in}}%
\pgfpathlineto{\pgfqpoint{4.231441in}{1.552347in}}%
\pgfpathlineto{\pgfqpoint{4.234018in}{1.490120in}}%
\pgfpathlineto{\pgfqpoint{4.236596in}{1.529815in}}%
\pgfpathlineto{\pgfqpoint{4.239173in}{1.539744in}}%
\pgfpathlineto{\pgfqpoint{4.241750in}{1.503001in}}%
\pgfpathlineto{\pgfqpoint{4.252059in}{1.562182in}}%
\pgfpathlineto{\pgfqpoint{4.254636in}{1.529243in}}%
\pgfpathlineto{\pgfqpoint{4.257213in}{1.534636in}}%
\pgfpathlineto{\pgfqpoint{4.259790in}{1.550553in}}%
\pgfpathlineto{\pgfqpoint{4.267522in}{1.540630in}}%
\pgfpathlineto{\pgfqpoint{4.270099in}{1.570742in}}%
\pgfpathlineto{\pgfqpoint{4.272677in}{1.587903in}}%
\pgfpathlineto{\pgfqpoint{4.275254in}{1.576579in}}%
\pgfpathlineto{\pgfqpoint{4.277831in}{1.536896in}}%
\pgfpathlineto{\pgfqpoint{4.285563in}{1.552391in}}%
\pgfpathlineto{\pgfqpoint{4.288140in}{1.526918in}}%
\pgfpathlineto{\pgfqpoint{4.290717in}{1.521280in}}%
\pgfpathlineto{\pgfqpoint{4.293294in}{1.512344in}}%
\pgfpathlineto{\pgfqpoint{4.295871in}{1.522612in}}%
\pgfpathlineto{\pgfqpoint{4.303603in}{1.488539in}}%
\pgfpathlineto{\pgfqpoint{4.308757in}{1.541863in}}%
\pgfpathlineto{\pgfqpoint{4.311335in}{1.535504in}}%
\pgfpathlineto{\pgfqpoint{4.313912in}{1.556681in}}%
\pgfpathlineto{\pgfqpoint{4.321644in}{1.612469in}}%
\pgfpathlineto{\pgfqpoint{4.324221in}{1.616166in}}%
\pgfpathlineto{\pgfqpoint{4.326798in}{1.641472in}}%
\pgfpathlineto{\pgfqpoint{4.329375in}{1.659969in}}%
\pgfpathlineto{\pgfqpoint{4.331952in}{1.664444in}}%
\pgfpathlineto{\pgfqpoint{4.339684in}{1.672459in}}%
\pgfpathlineto{\pgfqpoint{4.344838in}{1.648240in}}%
\pgfpathlineto{\pgfqpoint{4.347416in}{1.676589in}}%
\pgfpathlineto{\pgfqpoint{4.349993in}{1.695084in}}%
\pgfpathlineto{\pgfqpoint{4.357725in}{1.701039in}}%
\pgfpathlineto{\pgfqpoint{4.360302in}{1.707525in}}%
\pgfpathlineto{\pgfqpoint{4.362879in}{1.708842in}}%
\pgfpathlineto{\pgfqpoint{4.365456in}{1.749199in}}%
\pgfpathlineto{\pgfqpoint{4.368033in}{1.773656in}}%
\pgfpathlineto{\pgfqpoint{4.375765in}{1.776843in}}%
\pgfpathlineto{\pgfqpoint{4.378342in}{1.768999in}}%
\pgfpathlineto{\pgfqpoint{4.380919in}{1.780446in}}%
\pgfpathlineto{\pgfqpoint{4.383497in}{1.776628in}}%
\pgfpathlineto{\pgfqpoint{4.386074in}{1.760124in}}%
\pgfpathlineto{\pgfqpoint{4.393805in}{1.773572in}}%
\pgfpathlineto{\pgfqpoint{4.396383in}{1.785930in}}%
\pgfpathlineto{\pgfqpoint{4.398960in}{1.782199in}}%
\pgfpathlineto{\pgfqpoint{4.401537in}{1.784101in}}%
\pgfpathlineto{\pgfqpoint{4.404114in}{1.785264in}}%
\pgfpathlineto{\pgfqpoint{4.411846in}{1.764976in}}%
\pgfpathlineto{\pgfqpoint{4.414423in}{1.775071in}}%
\pgfpathlineto{\pgfqpoint{4.417000in}{1.776595in}}%
\pgfpathlineto{\pgfqpoint{4.422155in}{1.727287in}}%
\pgfpathlineto{\pgfqpoint{4.429886in}{1.754204in}}%
\pgfpathlineto{\pgfqpoint{4.432464in}{1.753817in}}%
\pgfpathlineto{\pgfqpoint{4.435041in}{1.782558in}}%
\pgfpathlineto{\pgfqpoint{4.437618in}{1.797522in}}%
\pgfpathlineto{\pgfqpoint{4.440195in}{1.804407in}}%
\pgfpathlineto{\pgfqpoint{4.447927in}{1.798494in}}%
\pgfpathlineto{\pgfqpoint{4.450504in}{1.789389in}}%
\pgfpathlineto{\pgfqpoint{4.453081in}{1.784850in}}%
\pgfpathlineto{\pgfqpoint{4.458236in}{1.780155in}}%
\pgfpathlineto{\pgfqpoint{4.465967in}{1.763228in}}%
\pgfpathlineto{\pgfqpoint{4.468545in}{1.785890in}}%
\pgfpathlineto{\pgfqpoint{4.473699in}{1.738551in}}%
\pgfpathlineto{\pgfqpoint{4.476276in}{1.790820in}}%
\pgfpathlineto{\pgfqpoint{4.484008in}{1.789216in}}%
\pgfpathlineto{\pgfqpoint{4.486585in}{1.766697in}}%
\pgfpathlineto{\pgfqpoint{4.489162in}{1.759146in}}%
\pgfpathlineto{\pgfqpoint{4.491740in}{1.764780in}}%
\pgfpathlineto{\pgfqpoint{4.494317in}{1.729909in}}%
\pgfpathlineto{\pgfqpoint{4.502048in}{1.746727in}}%
\pgfpathlineto{\pgfqpoint{4.504626in}{1.764687in}}%
\pgfpathlineto{\pgfqpoint{4.507203in}{1.800986in}}%
\pgfpathlineto{\pgfqpoint{4.509780in}{1.770941in}}%
\pgfpathlineto{\pgfqpoint{4.512357in}{1.716642in}}%
\pgfpathlineto{\pgfqpoint{4.520089in}{1.730320in}}%
\pgfpathlineto{\pgfqpoint{4.522666in}{1.750248in}}%
\pgfpathlineto{\pgfqpoint{4.525243in}{1.774736in}}%
\pgfpathlineto{\pgfqpoint{4.527821in}{1.772969in}}%
\pgfpathlineto{\pgfqpoint{4.538129in}{1.774863in}}%
\pgfpathlineto{\pgfqpoint{4.540707in}{1.797965in}}%
\pgfpathlineto{\pgfqpoint{4.543284in}{1.780471in}}%
\pgfpathlineto{\pgfqpoint{4.545861in}{1.757625in}}%
\pgfpathlineto{\pgfqpoint{4.556170in}{1.714347in}}%
\pgfpathlineto{\pgfqpoint{4.558747in}{1.713206in}}%
\pgfpathlineto{\pgfqpoint{4.563902in}{1.631073in}}%
\pgfpathlineto{\pgfqpoint{4.566479in}{1.606893in}}%
\pgfpathlineto{\pgfqpoint{4.574210in}{1.620866in}}%
\pgfpathlineto{\pgfqpoint{4.576788in}{1.634545in}}%
\pgfpathlineto{\pgfqpoint{4.579365in}{1.590310in}}%
\pgfpathlineto{\pgfqpoint{4.581942in}{1.622446in}}%
\pgfpathlineto{\pgfqpoint{4.584519in}{1.552313in}}%
\pgfpathlineto{\pgfqpoint{4.594828in}{1.557778in}}%
\pgfpathlineto{\pgfqpoint{4.597405in}{1.537160in}}%
\pgfpathlineto{\pgfqpoint{4.599982in}{1.561535in}}%
\pgfpathlineto{\pgfqpoint{4.602560in}{1.558340in}}%
\pgfpathlineto{\pgfqpoint{4.610291in}{1.537665in}}%
\pgfpathlineto{\pgfqpoint{4.612869in}{1.582616in}}%
\pgfpathlineto{\pgfqpoint{4.615446in}{1.576572in}}%
\pgfpathlineto{\pgfqpoint{4.618023in}{1.573098in}}%
\pgfpathlineto{\pgfqpoint{4.620600in}{1.638359in}}%
\pgfpathlineto{\pgfqpoint{4.628332in}{1.634420in}}%
\pgfpathlineto{\pgfqpoint{4.630909in}{1.594338in}}%
\pgfpathlineto{\pgfqpoint{4.633486in}{1.622832in}}%
\pgfpathlineto{\pgfqpoint{4.636063in}{1.635043in}}%
\pgfpathlineto{\pgfqpoint{4.638641in}{1.607087in}}%
\pgfpathlineto{\pgfqpoint{4.646372in}{1.582685in}}%
\pgfpathlineto{\pgfqpoint{4.648950in}{1.586789in}}%
\pgfpathlineto{\pgfqpoint{4.651527in}{1.572051in}}%
\pgfpathlineto{\pgfqpoint{4.654104in}{1.541277in}}%
\pgfpathlineto{\pgfqpoint{4.656681in}{1.582331in}}%
\pgfpathlineto{\pgfqpoint{4.666990in}{1.604504in}}%
\pgfpathlineto{\pgfqpoint{4.669567in}{1.637540in}}%
\pgfpathlineto{\pgfqpoint{4.672144in}{1.640604in}}%
\pgfpathlineto{\pgfqpoint{4.674722in}{1.637069in}}%
\pgfpathlineto{\pgfqpoint{4.682453in}{1.674502in}}%
\pgfpathlineto{\pgfqpoint{4.685030in}{1.651699in}}%
\pgfpathlineto{\pgfqpoint{4.687608in}{1.658975in}}%
\pgfpathlineto{\pgfqpoint{4.690185in}{1.696484in}}%
\pgfpathlineto{\pgfqpoint{4.692762in}{1.689135in}}%
\pgfpathlineto{\pgfqpoint{4.700494in}{1.675554in}}%
\pgfpathlineto{\pgfqpoint{4.703071in}{1.717548in}}%
\pgfpathlineto{\pgfqpoint{4.708225in}{1.736370in}}%
\pgfpathlineto{\pgfqpoint{4.710803in}{1.743358in}}%
\pgfpathlineto{\pgfqpoint{4.718534in}{1.745102in}}%
\pgfpathlineto{\pgfqpoint{4.721111in}{1.732615in}}%
\pgfpathlineto{\pgfqpoint{4.723689in}{1.732998in}}%
\pgfpathlineto{\pgfqpoint{4.726266in}{1.727851in}}%
\pgfpathlineto{\pgfqpoint{4.728843in}{1.750631in}}%
\pgfpathlineto{\pgfqpoint{4.736575in}{1.747911in}}%
\pgfpathlineto{\pgfqpoint{4.739152in}{1.749196in}}%
\pgfpathlineto{\pgfqpoint{4.741729in}{1.758490in}}%
\pgfpathlineto{\pgfqpoint{4.744306in}{1.784063in}}%
\pgfpathlineto{\pgfqpoint{4.746884in}{1.799515in}}%
\pgfpathlineto{\pgfqpoint{4.754615in}{1.796715in}}%
\pgfpathlineto{\pgfqpoint{4.757192in}{1.784943in}}%
\pgfpathlineto{\pgfqpoint{4.759770in}{1.785589in}}%
\pgfpathlineto{\pgfqpoint{4.762347in}{1.789083in}}%
\pgfpathlineto{\pgfqpoint{4.772656in}{1.794253in}}%
\pgfpathlineto{\pgfqpoint{4.777810in}{1.824253in}}%
\pgfpathlineto{\pgfqpoint{4.780387in}{1.822316in}}%
\pgfpathlineto{\pgfqpoint{4.782965in}{1.833045in}}%
\pgfpathlineto{\pgfqpoint{4.790696in}{1.820163in}}%
\pgfpathlineto{\pgfqpoint{4.793273in}{1.805442in}}%
\pgfpathlineto{\pgfqpoint{4.795851in}{1.817432in}}%
\pgfpathlineto{\pgfqpoint{4.798428in}{1.793574in}}%
\pgfpathlineto{\pgfqpoint{4.801005in}{1.801069in}}%
\pgfpathlineto{\pgfqpoint{4.808737in}{1.798327in}}%
\pgfpathlineto{\pgfqpoint{4.813891in}{1.837948in}}%
\pgfpathlineto{\pgfqpoint{4.816468in}{1.837303in}}%
\pgfpathlineto{\pgfqpoint{4.819046in}{1.834544in}}%
\pgfpathlineto{\pgfqpoint{4.826777in}{1.858137in}}%
\pgfpathlineto{\pgfqpoint{4.829354in}{1.870450in}}%
\pgfpathlineto{\pgfqpoint{4.831932in}{1.877813in}}%
\pgfpathlineto{\pgfqpoint{4.834509in}{1.864541in}}%
\pgfpathlineto{\pgfqpoint{4.837086in}{1.861369in}}%
\pgfpathlineto{\pgfqpoint{4.844818in}{1.857939in}}%
\pgfpathlineto{\pgfqpoint{4.847395in}{1.850632in}}%
\pgfpathlineto{\pgfqpoint{4.849972in}{1.867936in}}%
\pgfpathlineto{\pgfqpoint{4.852549in}{1.844375in}}%
\pgfpathlineto{\pgfqpoint{4.855127in}{1.831989in}}%
\pgfpathlineto{\pgfqpoint{4.862858in}{1.851519in}}%
\pgfpathlineto{\pgfqpoint{4.865435in}{1.829126in}}%
\pgfpathlineto{\pgfqpoint{4.868013in}{1.815592in}}%
\pgfpathlineto{\pgfqpoint{4.870590in}{1.817770in}}%
\pgfpathlineto{\pgfqpoint{4.873167in}{1.833608in}}%
\pgfpathlineto{\pgfqpoint{4.880899in}{1.827233in}}%
\pgfpathlineto{\pgfqpoint{4.883476in}{1.858784in}}%
\pgfpathlineto{\pgfqpoint{4.886053in}{1.833583in}}%
\pgfpathlineto{\pgfqpoint{4.888630in}{1.830901in}}%
\pgfpathlineto{\pgfqpoint{4.891207in}{1.807232in}}%
\pgfpathlineto{\pgfqpoint{4.898939in}{1.827150in}}%
\pgfpathlineto{\pgfqpoint{4.901516in}{1.798512in}}%
\pgfpathlineto{\pgfqpoint{4.904094in}{1.794939in}}%
\pgfpathlineto{\pgfqpoint{4.906671in}{1.774254in}}%
\pgfpathlineto{\pgfqpoint{4.909248in}{1.793611in}}%
\pgfpathlineto{\pgfqpoint{4.916980in}{1.787237in}}%
\pgfpathlineto{\pgfqpoint{4.919557in}{1.822758in}}%
\pgfpathlineto{\pgfqpoint{4.922134in}{1.836893in}}%
\pgfpathlineto{\pgfqpoint{4.924711in}{1.834893in}}%
\pgfpathlineto{\pgfqpoint{4.927288in}{1.845108in}}%
\pgfpathlineto{\pgfqpoint{4.940175in}{1.839481in}}%
\pgfpathlineto{\pgfqpoint{4.942752in}{1.850914in}}%
\pgfpathlineto{\pgfqpoint{4.945329in}{1.847942in}}%
\pgfpathlineto{\pgfqpoint{4.953061in}{1.861016in}}%
\pgfpathlineto{\pgfqpoint{4.955638in}{1.867061in}}%
\pgfpathlineto{\pgfqpoint{4.958215in}{1.875090in}}%
\pgfpathlineto{\pgfqpoint{4.960792in}{1.876210in}}%
\pgfpathlineto{\pgfqpoint{4.963369in}{1.867950in}}%
\pgfpathlineto{\pgfqpoint{4.971101in}{1.849563in}}%
\pgfpathlineto{\pgfqpoint{4.973678in}{1.852751in}}%
\pgfpathlineto{\pgfqpoint{4.976256in}{1.844679in}}%
\pgfpathlineto{\pgfqpoint{4.978833in}{1.860622in}}%
\pgfpathlineto{\pgfqpoint{4.981410in}{1.854298in}}%
\pgfpathlineto{\pgfqpoint{4.989142in}{1.871059in}}%
\pgfpathlineto{\pgfqpoint{4.991719in}{1.873090in}}%
\pgfpathlineto{\pgfqpoint{4.994296in}{1.867289in}}%
\pgfpathlineto{\pgfqpoint{4.996873in}{1.902410in}}%
\pgfpathlineto{\pgfqpoint{4.999450in}{1.818377in}}%
\pgfpathlineto{\pgfqpoint{5.007182in}{1.780666in}}%
\pgfpathlineto{\pgfqpoint{5.014914in}{1.895580in}}%
\pgfpathlineto{\pgfqpoint{5.017491in}{1.899867in}}%
\pgfpathlineto{\pgfqpoint{5.027800in}{1.892212in}}%
\pgfpathlineto{\pgfqpoint{5.030377in}{1.900711in}}%
\pgfpathlineto{\pgfqpoint{5.032954in}{1.904124in}}%
\pgfpathlineto{\pgfqpoint{5.035531in}{1.945206in}}%
\pgfpathlineto{\pgfqpoint{5.043263in}{1.954911in}}%
\pgfpathlineto{\pgfqpoint{5.045840in}{1.968331in}}%
\pgfpathlineto{\pgfqpoint{5.048417in}{1.974031in}}%
\pgfpathlineto{\pgfqpoint{5.050995in}{1.984929in}}%
\pgfpathlineto{\pgfqpoint{5.053572in}{1.988637in}}%
\pgfpathlineto{\pgfqpoint{5.061304in}{1.990585in}}%
\pgfpathlineto{\pgfqpoint{5.063881in}{1.996669in}}%
\pgfpathlineto{\pgfqpoint{5.066458in}{2.000015in}}%
\pgfpathlineto{\pgfqpoint{5.069035in}{1.976685in}}%
\pgfpathlineto{\pgfqpoint{5.071612in}{1.985317in}}%
\pgfpathlineto{\pgfqpoint{5.079344in}{1.974303in}}%
\pgfpathlineto{\pgfqpoint{5.081921in}{1.971760in}}%
\pgfpathlineto{\pgfqpoint{5.084498in}{1.967834in}}%
\pgfpathlineto{\pgfqpoint{5.087076in}{1.966429in}}%
\pgfpathlineto{\pgfqpoint{5.089653in}{1.968538in}}%
\pgfpathlineto{\pgfqpoint{5.097384in}{1.963503in}}%
\pgfpathlineto{\pgfqpoint{5.099962in}{1.951913in}}%
\pgfpathlineto{\pgfqpoint{5.102539in}{1.953332in}}%
\pgfpathlineto{\pgfqpoint{5.105116in}{1.956438in}}%
\pgfpathlineto{\pgfqpoint{5.107693in}{1.974666in}}%
\pgfpathlineto{\pgfqpoint{5.115425in}{1.973933in}}%
\pgfpathlineto{\pgfqpoint{5.118002in}{1.976097in}}%
\pgfpathlineto{\pgfqpoint{5.120579in}{1.973923in}}%
\pgfpathlineto{\pgfqpoint{5.123157in}{1.987997in}}%
\pgfpathlineto{\pgfqpoint{5.125734in}{1.980219in}}%
\pgfpathlineto{\pgfqpoint{5.133465in}{1.987884in}}%
\pgfpathlineto{\pgfqpoint{5.136043in}{1.974021in}}%
\pgfpathlineto{\pgfqpoint{5.138620in}{1.984801in}}%
\pgfpathlineto{\pgfqpoint{5.143774in}{1.979213in}}%
\pgfpathlineto{\pgfqpoint{5.151506in}{1.975391in}}%
\pgfpathlineto{\pgfqpoint{5.154083in}{1.978416in}}%
\pgfpathlineto{\pgfqpoint{5.156660in}{1.968910in}}%
\pgfpathlineto{\pgfqpoint{5.159238in}{1.969290in}}%
\pgfpathlineto{\pgfqpoint{5.161815in}{1.965119in}}%
\pgfpathlineto{\pgfqpoint{5.169546in}{1.981335in}}%
\pgfpathlineto{\pgfqpoint{5.174701in}{1.971047in}}%
\pgfpathlineto{\pgfqpoint{5.177278in}{1.975075in}}%
\pgfpathlineto{\pgfqpoint{5.179855in}{1.984052in}}%
\pgfpathlineto{\pgfqpoint{5.190164in}{1.990801in}}%
\pgfpathlineto{\pgfqpoint{5.192741in}{1.990976in}}%
\pgfpathlineto{\pgfqpoint{5.195319in}{1.990062in}}%
\pgfpathlineto{\pgfqpoint{5.197896in}{1.927860in}}%
\pgfpathlineto{\pgfqpoint{5.205627in}{1.967694in}}%
\pgfpathlineto{\pgfqpoint{5.208205in}{1.925262in}}%
\pgfpathlineto{\pgfqpoint{5.210782in}{1.915986in}}%
\pgfpathlineto{\pgfqpoint{5.213359in}{1.941201in}}%
\pgfpathlineto{\pgfqpoint{5.215936in}{1.937785in}}%
\pgfpathlineto{\pgfqpoint{5.223668in}{1.934777in}}%
\pgfpathlineto{\pgfqpoint{5.226245in}{1.938652in}}%
\pgfpathlineto{\pgfqpoint{5.231400in}{1.974296in}}%
\pgfpathlineto{\pgfqpoint{5.233977in}{1.954894in}}%
\pgfpathlineto{\pgfqpoint{5.241708in}{1.932491in}}%
\pgfpathlineto{\pgfqpoint{5.244286in}{1.950980in}}%
\pgfpathlineto{\pgfqpoint{5.246863in}{1.960364in}}%
\pgfpathlineto{\pgfqpoint{5.249440in}{1.933049in}}%
\pgfpathlineto{\pgfqpoint{5.252017in}{1.953968in}}%
\pgfpathlineto{\pgfqpoint{5.259749in}{1.950091in}}%
\pgfpathlineto{\pgfqpoint{5.262326in}{1.936372in}}%
\pgfpathlineto{\pgfqpoint{5.264903in}{1.947887in}}%
\pgfpathlineto{\pgfqpoint{5.267481in}{1.935820in}}%
\pgfpathlineto{\pgfqpoint{5.270058in}{1.928195in}}%
\pgfpathlineto{\pgfqpoint{5.277789in}{1.924929in}}%
\pgfpathlineto{\pgfqpoint{5.280367in}{1.897004in}}%
\pgfpathlineto{\pgfqpoint{5.282944in}{1.897922in}}%
\pgfpathlineto{\pgfqpoint{5.285521in}{1.894925in}}%
\pgfpathlineto{\pgfqpoint{5.288098in}{1.900900in}}%
\pgfpathlineto{\pgfqpoint{5.295830in}{1.895674in}}%
\pgfpathlineto{\pgfqpoint{5.298407in}{1.891621in}}%
\pgfpathlineto{\pgfqpoint{5.300984in}{1.882240in}}%
\pgfpathlineto{\pgfqpoint{5.303561in}{1.895701in}}%
\pgfpathlineto{\pgfqpoint{5.306139in}{1.885115in}}%
\pgfpathlineto{\pgfqpoint{5.313870in}{1.892642in}}%
\pgfpathlineto{\pgfqpoint{5.316448in}{1.887120in}}%
\pgfpathlineto{\pgfqpoint{5.319025in}{1.890264in}}%
\pgfpathlineto{\pgfqpoint{5.321602in}{1.891352in}}%
\pgfpathlineto{\pgfqpoint{5.324179in}{1.897744in}}%
\pgfpathlineto{\pgfqpoint{5.331911in}{1.896147in}}%
\pgfpathlineto{\pgfqpoint{5.337065in}{1.866162in}}%
\pgfpathlineto{\pgfqpoint{5.339642in}{1.865049in}}%
\pgfpathlineto{\pgfqpoint{5.342220in}{1.858130in}}%
\pgfpathlineto{\pgfqpoint{5.349951in}{1.912649in}}%
\pgfpathlineto{\pgfqpoint{5.352529in}{1.923049in}}%
\pgfpathlineto{\pgfqpoint{5.355106in}{1.940174in}}%
\pgfpathlineto{\pgfqpoint{5.357683in}{1.944990in}}%
\pgfpathlineto{\pgfqpoint{5.360260in}{1.959776in}}%
\pgfpathlineto{\pgfqpoint{5.367992in}{1.938462in}}%
\pgfpathlineto{\pgfqpoint{5.370569in}{1.954047in}}%
\pgfpathlineto{\pgfqpoint{5.375723in}{1.966531in}}%
\pgfpathlineto{\pgfqpoint{5.378301in}{1.954707in}}%
\pgfpathlineto{\pgfqpoint{5.386032in}{1.961219in}}%
\pgfpathlineto{\pgfqpoint{5.388609in}{1.965424in}}%
\pgfpathlineto{\pgfqpoint{5.391187in}{1.977490in}}%
\pgfpathlineto{\pgfqpoint{5.396341in}{1.994741in}}%
\pgfpathlineto{\pgfqpoint{5.404073in}{1.984063in}}%
\pgfpathlineto{\pgfqpoint{5.406650in}{1.979015in}}%
\pgfpathlineto{\pgfqpoint{5.409227in}{1.951502in}}%
\pgfpathlineto{\pgfqpoint{5.411804in}{1.943445in}}%
\pgfpathlineto{\pgfqpoint{5.414382in}{1.947867in}}%
\pgfpathlineto{\pgfqpoint{5.422113in}{1.956281in}}%
\pgfpathlineto{\pgfqpoint{5.424690in}{1.965318in}}%
\pgfpathlineto{\pgfqpoint{5.427268in}{2.013720in}}%
\pgfpathlineto{\pgfqpoint{5.429845in}{2.015482in}}%
\pgfpathlineto{\pgfqpoint{5.432422in}{2.036727in}}%
\pgfpathlineto{\pgfqpoint{5.440154in}{2.045131in}}%
\pgfpathlineto{\pgfqpoint{5.442731in}{2.056675in}}%
\pgfpathlineto{\pgfqpoint{5.445308in}{2.038930in}}%
\pgfpathlineto{\pgfqpoint{5.447885in}{2.044541in}}%
\pgfpathlineto{\pgfqpoint{5.450463in}{2.045807in}}%
\pgfpathlineto{\pgfqpoint{5.458194in}{2.063881in}}%
\pgfpathlineto{\pgfqpoint{5.460771in}{2.072623in}}%
\pgfpathlineto{\pgfqpoint{5.463349in}{2.066906in}}%
\pgfpathlineto{\pgfqpoint{5.465926in}{2.067723in}}%
\pgfpathlineto{\pgfqpoint{5.468503in}{2.072980in}}%
\pgfpathlineto{\pgfqpoint{5.478812in}{2.074407in}}%
\pgfpathlineto{\pgfqpoint{5.481389in}{2.056267in}}%
\pgfpathlineto{\pgfqpoint{5.483966in}{2.060643in}}%
\pgfpathlineto{\pgfqpoint{5.486544in}{2.049588in}}%
\pgfpathlineto{\pgfqpoint{5.496852in}{2.078540in}}%
\pgfpathlineto{\pgfqpoint{5.499430in}{2.088639in}}%
\pgfpathlineto{\pgfqpoint{5.502007in}{2.091918in}}%
\pgfpathlineto{\pgfqpoint{5.504584in}{2.100628in}}%
\pgfpathlineto{\pgfqpoint{5.514893in}{2.084569in}}%
\pgfpathlineto{\pgfqpoint{5.517470in}{2.091428in}}%
\pgfpathlineto{\pgfqpoint{5.520047in}{2.082180in}}%
\pgfpathlineto{\pgfqpoint{5.522625in}{2.080606in}}%
\pgfpathlineto{\pgfqpoint{5.532933in}{2.084772in}}%
\pgfpathlineto{\pgfqpoint{5.535511in}{2.090337in}}%
\pgfpathlineto{\pgfqpoint{5.538088in}{2.083579in}}%
\pgfpathlineto{\pgfqpoint{5.540665in}{2.090822in}}%
\pgfpathlineto{\pgfqpoint{5.548397in}{2.076059in}}%
\pgfpathlineto{\pgfqpoint{5.553551in}{2.084182in}}%
\pgfpathlineto{\pgfqpoint{5.556128in}{2.070398in}}%
\pgfpathlineto{\pgfqpoint{5.558706in}{2.081791in}}%
\pgfpathlineto{\pgfqpoint{5.566437in}{2.076261in}}%
\pgfpathlineto{\pgfqpoint{5.569014in}{2.063238in}}%
\pgfpathlineto{\pgfqpoint{5.571592in}{2.055377in}}%
\pgfpathlineto{\pgfqpoint{5.574169in}{2.053754in}}%
\pgfpathlineto{\pgfqpoint{5.576746in}{2.078019in}}%
\pgfpathlineto{\pgfqpoint{5.584478in}{2.071036in}}%
\pgfpathlineto{\pgfqpoint{5.587055in}{2.073501in}}%
\pgfpathlineto{\pgfqpoint{5.589632in}{2.073249in}}%
\pgfpathlineto{\pgfqpoint{5.592209in}{2.078171in}}%
\pgfpathlineto{\pgfqpoint{5.594786in}{2.085448in}}%
\pgfpathlineto{\pgfqpoint{5.605095in}{2.108589in}}%
\pgfpathlineto{\pgfqpoint{5.607673in}{2.129430in}}%
\pgfpathlineto{\pgfqpoint{5.610250in}{2.140411in}}%
\pgfpathlineto{\pgfqpoint{5.612827in}{2.146076in}}%
\pgfpathlineto{\pgfqpoint{5.623136in}{2.156842in}}%
\pgfpathlineto{\pgfqpoint{5.625713in}{2.155987in}}%
\pgfpathlineto{\pgfqpoint{5.628290in}{2.165560in}}%
\pgfpathlineto{\pgfqpoint{5.630867in}{2.178461in}}%
\pgfpathlineto{\pgfqpoint{5.638599in}{2.170715in}}%
\pgfpathlineto{\pgfqpoint{5.641176in}{2.162765in}}%
\pgfpathlineto{\pgfqpoint{5.643754in}{2.192373in}}%
\pgfpathlineto{\pgfqpoint{5.646331in}{2.178459in}}%
\pgfpathlineto{\pgfqpoint{5.648908in}{2.177417in}}%
\pgfpathlineto{\pgfqpoint{5.656640in}{2.167752in}}%
\pgfpathlineto{\pgfqpoint{5.659217in}{2.167981in}}%
\pgfpathlineto{\pgfqpoint{5.661794in}{2.161236in}}%
\pgfpathlineto{\pgfqpoint{5.664371in}{2.170339in}}%
\pgfpathlineto{\pgfqpoint{5.666948in}{2.185090in}}%
\pgfpathlineto{\pgfqpoint{5.674680in}{2.180591in}}%
\pgfpathlineto{\pgfqpoint{5.677257in}{2.175781in}}%
\pgfpathlineto{\pgfqpoint{5.679835in}{2.190733in}}%
\pgfpathlineto{\pgfqpoint{5.682412in}{2.187958in}}%
\pgfpathlineto{\pgfqpoint{5.684989in}{2.194982in}}%
\pgfpathlineto{\pgfqpoint{5.692721in}{2.193572in}}%
\pgfpathlineto{\pgfqpoint{5.695298in}{2.168398in}}%
\pgfpathlineto{\pgfqpoint{5.697875in}{2.169414in}}%
\pgfpathlineto{\pgfqpoint{5.703029in}{2.169166in}}%
\pgfpathlineto{\pgfqpoint{5.710761in}{2.161626in}}%
\pgfpathlineto{\pgfqpoint{5.713338in}{2.176137in}}%
\pgfpathlineto{\pgfqpoint{5.715915in}{2.168488in}}%
\pgfpathlineto{\pgfqpoint{5.718493in}{2.176074in}}%
\pgfpathlineto{\pgfqpoint{5.721070in}{2.173902in}}%
\pgfpathlineto{\pgfqpoint{5.728802in}{2.175249in}}%
\pgfpathlineto{\pgfqpoint{5.731379in}{2.177548in}}%
\pgfpathlineto{\pgfqpoint{5.736533in}{2.171135in}}%
\pgfpathlineto{\pgfqpoint{5.739110in}{2.170875in}}%
\pgfpathlineto{\pgfqpoint{5.746842in}{2.166267in}}%
\pgfpathlineto{\pgfqpoint{5.749419in}{2.169835in}}%
\pgfpathlineto{\pgfqpoint{5.751996in}{2.166305in}}%
\pgfpathlineto{\pgfqpoint{5.754574in}{2.153148in}}%
\pgfpathlineto{\pgfqpoint{5.764883in}{2.173359in}}%
\pgfpathlineto{\pgfqpoint{5.767460in}{2.170492in}}%
\pgfpathlineto{\pgfqpoint{5.770037in}{2.164513in}}%
\pgfpathlineto{\pgfqpoint{5.772614in}{2.195593in}}%
\pgfpathlineto{\pgfqpoint{5.775191in}{2.180840in}}%
\pgfpathlineto{\pgfqpoint{5.782923in}{2.199955in}}%
\pgfpathlineto{\pgfqpoint{5.785500in}{2.208636in}}%
\pgfpathlineto{\pgfqpoint{5.788077in}{2.206990in}}%
\pgfpathlineto{\pgfqpoint{5.790655in}{2.207693in}}%
\pgfpathlineto{\pgfqpoint{5.793232in}{2.184216in}}%
\pgfpathlineto{\pgfqpoint{5.800963in}{2.179827in}}%
\pgfpathlineto{\pgfqpoint{5.803541in}{2.191732in}}%
\pgfpathlineto{\pgfqpoint{5.808695in}{2.192329in}}%
\pgfpathlineto{\pgfqpoint{5.811272in}{2.197817in}}%
\pgfpathlineto{\pgfqpoint{5.819004in}{2.188825in}}%
\pgfpathlineto{\pgfqpoint{5.821581in}{2.186959in}}%
\pgfpathlineto{\pgfqpoint{5.824158in}{2.173145in}}%
\pgfpathlineto{\pgfqpoint{5.826736in}{2.167187in}}%
\pgfpathlineto{\pgfqpoint{5.829313in}{2.159133in}}%
\pgfpathlineto{\pgfqpoint{5.837044in}{2.169375in}}%
\pgfpathlineto{\pgfqpoint{5.839622in}{2.168159in}}%
\pgfpathlineto{\pgfqpoint{5.842199in}{2.126910in}}%
\pgfpathlineto{\pgfqpoint{5.844776in}{2.133081in}}%
\pgfpathlineto{\pgfqpoint{5.847353in}{2.155819in}}%
\pgfpathlineto{\pgfqpoint{5.855085in}{2.169670in}}%
\pgfpathlineto{\pgfqpoint{5.857662in}{2.172417in}}%
\pgfpathlineto{\pgfqpoint{5.860239in}{2.172757in}}%
\pgfpathlineto{\pgfqpoint{5.862817in}{2.179319in}}%
\pgfpathlineto{\pgfqpoint{5.865394in}{2.181734in}}%
\pgfpathlineto{\pgfqpoint{5.875703in}{2.186441in}}%
\pgfpathlineto{\pgfqpoint{5.880857in}{2.208486in}}%
\pgfpathlineto{\pgfqpoint{5.883434in}{2.220317in}}%
\pgfpathlineto{\pgfqpoint{5.891166in}{2.220082in}}%
\pgfpathlineto{\pgfqpoint{5.893743in}{2.210611in}}%
\pgfpathlineto{\pgfqpoint{5.896320in}{2.213386in}}%
\pgfpathlineto{\pgfqpoint{5.898898in}{2.206559in}}%
\pgfpathlineto{\pgfqpoint{5.901475in}{2.214238in}}%
\pgfpathlineto{\pgfqpoint{5.909206in}{2.227836in}}%
\pgfpathlineto{\pgfqpoint{5.911784in}{2.227433in}}%
\pgfpathlineto{\pgfqpoint{5.916938in}{2.238961in}}%
\pgfpathlineto{\pgfqpoint{5.919515in}{2.248560in}}%
\pgfpathlineto{\pgfqpoint{5.927247in}{2.255642in}}%
\pgfpathlineto{\pgfqpoint{5.929824in}{2.234764in}}%
\pgfpathlineto{\pgfqpoint{5.934979in}{2.221406in}}%
\pgfpathlineto{\pgfqpoint{5.937556in}{2.228651in}}%
\pgfpathlineto{\pgfqpoint{5.945287in}{2.233833in}}%
\pgfpathlineto{\pgfqpoint{5.947865in}{2.208722in}}%
\pgfpathlineto{\pgfqpoint{5.950442in}{2.223484in}}%
\pgfpathlineto{\pgfqpoint{5.953019in}{2.191123in}}%
\pgfpathlineto{\pgfqpoint{5.955596in}{2.197369in}}%
\pgfpathlineto{\pgfqpoint{5.963328in}{2.212510in}}%
\pgfpathlineto{\pgfqpoint{5.968482in}{2.217394in}}%
\pgfpathlineto{\pgfqpoint{5.971060in}{2.177477in}}%
\pgfpathlineto{\pgfqpoint{5.973637in}{2.186690in}}%
\pgfpathlineto{\pgfqpoint{5.981368in}{2.184215in}}%
\pgfpathlineto{\pgfqpoint{5.983946in}{2.187668in}}%
\pgfpathlineto{\pgfqpoint{5.986523in}{2.206732in}}%
\pgfpathlineto{\pgfqpoint{5.989100in}{2.208292in}}%
\pgfpathlineto{\pgfqpoint{5.991677in}{2.222286in}}%
\pgfpathlineto{\pgfqpoint{5.999409in}{2.219929in}}%
\pgfpathlineto{\pgfqpoint{6.004563in}{2.245520in}}%
\pgfpathlineto{\pgfqpoint{6.007140in}{2.251593in}}%
\pgfpathlineto{\pgfqpoint{6.017449in}{2.225287in}}%
\pgfpathlineto{\pgfqpoint{6.020027in}{2.206515in}}%
\pgfpathlineto{\pgfqpoint{6.022604in}{2.211103in}}%
\pgfpathlineto{\pgfqpoint{6.025181in}{2.241418in}}%
\pgfpathlineto{\pgfqpoint{6.027758in}{2.244842in}}%
\pgfpathlineto{\pgfqpoint{6.035490in}{2.258064in}}%
\pgfpathlineto{\pgfqpoint{6.038067in}{2.276747in}}%
\pgfpathlineto{\pgfqpoint{6.040644in}{2.278151in}}%
\pgfpathlineto{\pgfqpoint{6.043221in}{2.291728in}}%
\pgfpathlineto{\pgfqpoint{6.045799in}{2.290785in}}%
\pgfpathlineto{\pgfqpoint{6.053530in}{2.281502in}}%
\pgfpathlineto{\pgfqpoint{6.056108in}{2.277223in}}%
\pgfpathlineto{\pgfqpoint{6.058685in}{2.265534in}}%
\pgfpathlineto{\pgfqpoint{6.061262in}{2.240457in}}%
\pgfpathlineto{\pgfqpoint{6.063839in}{2.238727in}}%
\pgfpathlineto{\pgfqpoint{6.071571in}{2.261391in}}%
\pgfpathlineto{\pgfqpoint{6.074148in}{2.259642in}}%
\pgfpathlineto{\pgfqpoint{6.076725in}{2.274543in}}%
\pgfpathlineto{\pgfqpoint{6.079302in}{2.239008in}}%
\pgfpathlineto{\pgfqpoint{6.081880in}{2.226636in}}%
\pgfpathlineto{\pgfqpoint{6.089611in}{2.235113in}}%
\pgfpathlineto{\pgfqpoint{6.092188in}{2.247560in}}%
\pgfpathlineto{\pgfqpoint{6.094766in}{2.234064in}}%
\pgfpathlineto{\pgfqpoint{6.097343in}{2.228562in}}%
\pgfpathlineto{\pgfqpoint{6.099920in}{2.235464in}}%
\pgfpathlineto{\pgfqpoint{6.107652in}{2.237334in}}%
\pgfpathlineto{\pgfqpoint{6.110229in}{2.247083in}}%
\pgfpathlineto{\pgfqpoint{6.112806in}{2.245168in}}%
\pgfpathlineto{\pgfqpoint{6.115383in}{2.250555in}}%
\pgfpathlineto{\pgfqpoint{6.117961in}{2.251932in}}%
\pgfpathlineto{\pgfqpoint{6.128269in}{2.215132in}}%
\pgfpathlineto{\pgfqpoint{6.130847in}{2.219693in}}%
\pgfpathlineto{\pgfqpoint{6.133424in}{2.199545in}}%
\pgfpathlineto{\pgfqpoint{6.136001in}{2.190173in}}%
\pgfpathlineto{\pgfqpoint{6.146310in}{2.231042in}}%
\pgfpathlineto{\pgfqpoint{6.148887in}{2.241193in}}%
\pgfpathlineto{\pgfqpoint{6.154042in}{2.266864in}}%
\pgfpathlineto{\pgfqpoint{6.161773in}{2.276851in}}%
\pgfpathlineto{\pgfqpoint{6.164350in}{2.294992in}}%
\pgfpathlineto{\pgfqpoint{6.166928in}{2.288407in}}%
\pgfpathlineto{\pgfqpoint{6.169505in}{2.283489in}}%
\pgfpathlineto{\pgfqpoint{6.172082in}{2.290106in}}%
\pgfpathlineto{\pgfqpoint{6.179814in}{2.287019in}}%
\pgfpathlineto{\pgfqpoint{6.184968in}{2.281218in}}%
\pgfpathlineto{\pgfqpoint{6.187545in}{2.278035in}}%
\pgfpathlineto{\pgfqpoint{6.190123in}{2.286954in}}%
\pgfpathlineto{\pgfqpoint{6.197854in}{2.317113in}}%
\pgfpathlineto{\pgfqpoint{6.200431in}{2.338096in}}%
\pgfpathlineto{\pgfqpoint{6.203009in}{2.336603in}}%
\pgfpathlineto{\pgfqpoint{6.205586in}{2.341707in}}%
\pgfpathlineto{\pgfqpoint{6.208163in}{2.342311in}}%
\pgfpathlineto{\pgfqpoint{6.221049in}{2.328664in}}%
\pgfpathlineto{\pgfqpoint{6.223626in}{2.324091in}}%
\pgfpathlineto{\pgfqpoint{6.226204in}{2.333054in}}%
\pgfpathlineto{\pgfqpoint{6.233935in}{2.340515in}}%
\pgfpathlineto{\pgfqpoint{6.239090in}{2.352993in}}%
\pgfpathlineto{\pgfqpoint{6.241667in}{2.364589in}}%
\pgfpathlineto{\pgfqpoint{6.244244in}{2.372141in}}%
\pgfpathlineto{\pgfqpoint{6.251976in}{2.350258in}}%
\pgfpathlineto{\pgfqpoint{6.254553in}{2.361032in}}%
\pgfpathlineto{\pgfqpoint{6.257130in}{2.356032in}}%
\pgfpathlineto{\pgfqpoint{6.262285in}{2.380064in}}%
\pgfpathlineto{\pgfqpoint{6.270016in}{2.355498in}}%
\pgfpathlineto{\pgfqpoint{6.272593in}{2.358004in}}%
\pgfpathlineto{\pgfqpoint{6.275171in}{2.375961in}}%
\pgfpathlineto{\pgfqpoint{6.277748in}{2.377912in}}%
\pgfpathlineto{\pgfqpoint{6.280325in}{2.378692in}}%
\pgfpathlineto{\pgfqpoint{6.288057in}{2.370854in}}%
\pgfpathlineto{\pgfqpoint{6.290634in}{2.377472in}}%
\pgfpathlineto{\pgfqpoint{6.293211in}{2.377909in}}%
\pgfpathlineto{\pgfqpoint{6.295788in}{2.361376in}}%
\pgfpathlineto{\pgfqpoint{6.298365in}{2.365546in}}%
\pgfpathlineto{\pgfqpoint{6.306097in}{2.344764in}}%
\pgfpathlineto{\pgfqpoint{6.308674in}{2.325248in}}%
\pgfpathlineto{\pgfqpoint{6.311252in}{2.314198in}}%
\pgfpathlineto{\pgfqpoint{6.313829in}{2.331745in}}%
\pgfpathlineto{\pgfqpoint{6.316406in}{2.317020in}}%
\pgfpathlineto{\pgfqpoint{6.324138in}{2.319308in}}%
\pgfpathlineto{\pgfqpoint{6.326715in}{2.332923in}}%
\pgfpathlineto{\pgfqpoint{6.329292in}{2.331779in}}%
\pgfpathlineto{\pgfqpoint{6.334446in}{2.336565in}}%
\pgfpathlineto{\pgfqpoint{6.342178in}{2.347222in}}%
\pgfpathlineto{\pgfqpoint{6.344755in}{2.381556in}}%
\pgfpathlineto{\pgfqpoint{6.347333in}{2.388436in}}%
\pgfpathlineto{\pgfqpoint{6.349910in}{2.424444in}}%
\pgfpathlineto{\pgfqpoint{6.352487in}{2.411665in}}%
\pgfpathlineto{\pgfqpoint{6.360219in}{2.421154in}}%
\pgfpathlineto{\pgfqpoint{6.362796in}{2.401197in}}%
\pgfpathlineto{\pgfqpoint{6.367950in}{2.403367in}}%
\pgfpathlineto{\pgfqpoint{6.370527in}{2.412705in}}%
\pgfpathlineto{\pgfqpoint{6.378259in}{2.430970in}}%
\pgfpathlineto{\pgfqpoint{6.380836in}{2.446308in}}%
\pgfpathlineto{\pgfqpoint{6.383414in}{2.446739in}}%
\pgfpathlineto{\pgfqpoint{6.385991in}{2.441188in}}%
\pgfpathlineto{\pgfqpoint{6.388568in}{2.475983in}}%
\pgfpathlineto{\pgfqpoint{6.396300in}{2.493680in}}%
\pgfpathlineto{\pgfqpoint{6.398877in}{2.489288in}}%
\pgfpathlineto{\pgfqpoint{6.401454in}{2.478274in}}%
\pgfpathlineto{\pgfqpoint{6.404031in}{2.473392in}}%
\pgfpathlineto{\pgfqpoint{6.416917in}{2.470043in}}%
\pgfpathlineto{\pgfqpoint{6.422072in}{2.480106in}}%
\pgfpathlineto{\pgfqpoint{6.424649in}{2.470848in}}%
\pgfpathlineto{\pgfqpoint{6.424649in}{2.470848in}}%
\pgfusepath{stroke}%
\end{pgfscope}%
\begin{pgfscope}%
\pgfpathrectangle{\pgfqpoint{0.506467in}{0.331635in}}{\pgfqpoint{6.200000in}{2.265000in}}%
\pgfusepath{clip}%
\pgfsetroundcap%
\pgfsetroundjoin%
\pgfsetlinewidth{1.505625pt}%
\definecolor{currentstroke}{rgb}{1.000000,0.498039,0.054902}%
\pgfsetstrokecolor{currentstroke}%
\pgfsetdash{}{0pt}%
\pgfpathmoveto{\pgfqpoint{0.788285in}{0.574286in}}%
\pgfpathlineto{\pgfqpoint{0.790862in}{0.578546in}}%
\pgfpathlineto{\pgfqpoint{0.793440in}{0.569348in}}%
\pgfpathlineto{\pgfqpoint{0.796017in}{0.565016in}}%
\pgfpathlineto{\pgfqpoint{0.803749in}{0.560783in}}%
\pgfpathlineto{\pgfqpoint{0.806326in}{0.560642in}}%
\pgfpathlineto{\pgfqpoint{0.808903in}{0.555509in}}%
\pgfpathlineto{\pgfqpoint{0.811480in}{0.553916in}}%
\pgfpathlineto{\pgfqpoint{0.814057in}{0.547385in}}%
\pgfpathlineto{\pgfqpoint{0.824366in}{0.547875in}}%
\pgfpathlineto{\pgfqpoint{0.826943in}{0.550729in}}%
\pgfpathlineto{\pgfqpoint{0.829521in}{0.544822in}}%
\pgfpathlineto{\pgfqpoint{0.832098in}{0.528667in}}%
\pgfpathlineto{\pgfqpoint{0.839830in}{0.523571in}}%
\pgfpathlineto{\pgfqpoint{0.842407in}{0.510046in}}%
\pgfpathlineto{\pgfqpoint{0.844984in}{0.513425in}}%
\pgfpathlineto{\pgfqpoint{0.857870in}{0.515581in}}%
\pgfpathlineto{\pgfqpoint{0.860447in}{0.519966in}}%
\pgfpathlineto{\pgfqpoint{0.863024in}{0.512155in}}%
\pgfpathlineto{\pgfqpoint{0.865602in}{0.502075in}}%
\pgfpathlineto{\pgfqpoint{0.868179in}{0.503656in}}%
\pgfpathlineto{\pgfqpoint{0.875910in}{0.504350in}}%
\pgfpathlineto{\pgfqpoint{0.878488in}{0.501622in}}%
\pgfpathlineto{\pgfqpoint{0.881065in}{0.504255in}}%
\pgfpathlineto{\pgfqpoint{0.883642in}{0.490657in}}%
\pgfpathlineto{\pgfqpoint{0.886219in}{0.482213in}}%
\pgfpathlineto{\pgfqpoint{0.893951in}{0.489016in}}%
\pgfpathlineto{\pgfqpoint{0.896528in}{0.498620in}}%
\pgfpathlineto{\pgfqpoint{0.899105in}{0.490410in}}%
\pgfpathlineto{\pgfqpoint{0.901683in}{0.497355in}}%
\pgfpathlineto{\pgfqpoint{0.914569in}{0.494708in}}%
\pgfpathlineto{\pgfqpoint{0.917146in}{0.490397in}}%
\pgfpathlineto{\pgfqpoint{0.919723in}{0.488239in}}%
\pgfpathlineto{\pgfqpoint{0.922300in}{0.490330in}}%
\pgfpathlineto{\pgfqpoint{0.930032in}{0.486938in}}%
\pgfpathlineto{\pgfqpoint{0.932609in}{0.494227in}}%
\pgfpathlineto{\pgfqpoint{0.935186in}{0.499384in}}%
\pgfpathlineto{\pgfqpoint{0.937764in}{0.508975in}}%
\pgfpathlineto{\pgfqpoint{0.940341in}{0.512385in}}%
\pgfpathlineto{\pgfqpoint{0.950650in}{0.515655in}}%
\pgfpathlineto{\pgfqpoint{0.953227in}{0.518406in}}%
\pgfpathlineto{\pgfqpoint{0.955804in}{0.513854in}}%
\pgfpathlineto{\pgfqpoint{0.958381in}{0.518624in}}%
\pgfpathlineto{\pgfqpoint{0.966113in}{0.519008in}}%
\pgfpathlineto{\pgfqpoint{0.968690in}{0.524730in}}%
\pgfpathlineto{\pgfqpoint{0.971267in}{0.526262in}}%
\pgfpathlineto{\pgfqpoint{0.973845in}{0.517741in}}%
\pgfpathlineto{\pgfqpoint{0.976422in}{0.514654in}}%
\pgfpathlineto{\pgfqpoint{0.984153in}{0.512718in}}%
\pgfpathlineto{\pgfqpoint{0.986731in}{0.515753in}}%
\pgfpathlineto{\pgfqpoint{0.989308in}{0.514101in}}%
\pgfpathlineto{\pgfqpoint{0.991885in}{0.509923in}}%
\pgfpathlineto{\pgfqpoint{0.994462in}{0.504264in}}%
\pgfpathlineto{\pgfqpoint{1.002194in}{0.499130in}}%
\pgfpathlineto{\pgfqpoint{1.004771in}{0.496268in}}%
\pgfpathlineto{\pgfqpoint{1.007348in}{0.498810in}}%
\pgfpathlineto{\pgfqpoint{1.009926in}{0.504882in}}%
\pgfpathlineto{\pgfqpoint{1.012503in}{0.508345in}}%
\pgfpathlineto{\pgfqpoint{1.020234in}{0.514178in}}%
\pgfpathlineto{\pgfqpoint{1.022812in}{0.510688in}}%
\pgfpathlineto{\pgfqpoint{1.025389in}{0.515318in}}%
\pgfpathlineto{\pgfqpoint{1.027966in}{0.526396in}}%
\pgfpathlineto{\pgfqpoint{1.038275in}{0.527359in}}%
\pgfpathlineto{\pgfqpoint{1.040852in}{0.520963in}}%
\pgfpathlineto{\pgfqpoint{1.043429in}{0.523438in}}%
\pgfpathlineto{\pgfqpoint{1.046007in}{0.515465in}}%
\pgfpathlineto{\pgfqpoint{1.048584in}{0.509895in}}%
\pgfpathlineto{\pgfqpoint{1.056315in}{0.511745in}}%
\pgfpathlineto{\pgfqpoint{1.058893in}{0.520956in}}%
\pgfpathlineto{\pgfqpoint{1.061470in}{0.519314in}}%
\pgfpathlineto{\pgfqpoint{1.066624in}{0.510195in}}%
\pgfpathlineto{\pgfqpoint{1.074356in}{0.513179in}}%
\pgfpathlineto{\pgfqpoint{1.076933in}{0.509169in}}%
\pgfpathlineto{\pgfqpoint{1.079510in}{0.516977in}}%
\pgfpathlineto{\pgfqpoint{1.082087in}{0.517807in}}%
\pgfpathlineto{\pgfqpoint{1.084665in}{0.503194in}}%
\pgfpathlineto{\pgfqpoint{1.092396in}{0.498650in}}%
\pgfpathlineto{\pgfqpoint{1.097551in}{0.501513in}}%
\pgfpathlineto{\pgfqpoint{1.100128in}{0.490550in}}%
\pgfpathlineto{\pgfqpoint{1.102705in}{0.494411in}}%
\pgfpathlineto{\pgfqpoint{1.110437in}{0.497380in}}%
\pgfpathlineto{\pgfqpoint{1.115591in}{0.487217in}}%
\pgfpathlineto{\pgfqpoint{1.120746in}{0.493317in}}%
\pgfpathlineto{\pgfqpoint{1.128477in}{0.498490in}}%
\pgfpathlineto{\pgfqpoint{1.131055in}{0.494473in}}%
\pgfpathlineto{\pgfqpoint{1.133632in}{0.507171in}}%
\pgfpathlineto{\pgfqpoint{1.136209in}{0.506433in}}%
\pgfpathlineto{\pgfqpoint{1.138786in}{0.498425in}}%
\pgfpathlineto{\pgfqpoint{1.146518in}{0.512710in}}%
\pgfpathlineto{\pgfqpoint{1.151672in}{0.497959in}}%
\pgfpathlineto{\pgfqpoint{1.154249in}{0.499132in}}%
\pgfpathlineto{\pgfqpoint{1.156827in}{0.498166in}}%
\pgfpathlineto{\pgfqpoint{1.167135in}{0.496218in}}%
\pgfpathlineto{\pgfqpoint{1.169713in}{0.499351in}}%
\pgfpathlineto{\pgfqpoint{1.172290in}{0.497638in}}%
\pgfpathlineto{\pgfqpoint{1.174867in}{0.503601in}}%
\pgfpathlineto{\pgfqpoint{1.182599in}{0.495269in}}%
\pgfpathlineto{\pgfqpoint{1.185176in}{0.496031in}}%
\pgfpathlineto{\pgfqpoint{1.187753in}{0.498784in}}%
\pgfpathlineto{\pgfqpoint{1.190330in}{0.494213in}}%
\pgfpathlineto{\pgfqpoint{1.192908in}{0.505320in}}%
\pgfpathlineto{\pgfqpoint{1.200639in}{0.507728in}}%
\pgfpathlineto{\pgfqpoint{1.203216in}{0.513873in}}%
\pgfpathlineto{\pgfqpoint{1.205794in}{0.516848in}}%
\pgfpathlineto{\pgfqpoint{1.208371in}{0.510370in}}%
\pgfpathlineto{\pgfqpoint{1.210948in}{0.520947in}}%
\pgfpathlineto{\pgfqpoint{1.218680in}{0.526294in}}%
\pgfpathlineto{\pgfqpoint{1.221257in}{0.520595in}}%
\pgfpathlineto{\pgfqpoint{1.223834in}{0.507445in}}%
\pgfpathlineto{\pgfqpoint{1.228989in}{0.530721in}}%
\pgfpathlineto{\pgfqpoint{1.236720in}{0.535930in}}%
\pgfpathlineto{\pgfqpoint{1.239297in}{0.539312in}}%
\pgfpathlineto{\pgfqpoint{1.241875in}{0.538051in}}%
\pgfpathlineto{\pgfqpoint{1.244452in}{0.534468in}}%
\pgfpathlineto{\pgfqpoint{1.247029in}{0.544477in}}%
\pgfpathlineto{\pgfqpoint{1.254761in}{0.552157in}}%
\pgfpathlineto{\pgfqpoint{1.257338in}{0.551987in}}%
\pgfpathlineto{\pgfqpoint{1.262492in}{0.549480in}}%
\pgfpathlineto{\pgfqpoint{1.272801in}{0.546930in}}%
\pgfpathlineto{\pgfqpoint{1.275378in}{0.542570in}}%
\pgfpathlineto{\pgfqpoint{1.277956in}{0.543036in}}%
\pgfpathlineto{\pgfqpoint{1.280533in}{0.550793in}}%
\pgfpathlineto{\pgfqpoint{1.283110in}{0.552268in}}%
\pgfpathlineto{\pgfqpoint{1.290842in}{0.563817in}}%
\pgfpathlineto{\pgfqpoint{1.293419in}{0.573176in}}%
\pgfpathlineto{\pgfqpoint{1.295996in}{0.569615in}}%
\pgfpathlineto{\pgfqpoint{1.298573in}{0.553718in}}%
\pgfpathlineto{\pgfqpoint{1.301151in}{0.561912in}}%
\pgfpathlineto{\pgfqpoint{1.311459in}{0.551984in}}%
\pgfpathlineto{\pgfqpoint{1.314037in}{0.550487in}}%
\pgfpathlineto{\pgfqpoint{1.316614in}{0.556701in}}%
\pgfpathlineto{\pgfqpoint{1.319191in}{0.554941in}}%
\pgfpathlineto{\pgfqpoint{1.326923in}{0.553104in}}%
\pgfpathlineto{\pgfqpoint{1.329500in}{0.558455in}}%
\pgfpathlineto{\pgfqpoint{1.332077in}{0.551810in}}%
\pgfpathlineto{\pgfqpoint{1.334654in}{0.556493in}}%
\pgfpathlineto{\pgfqpoint{1.337232in}{0.549036in}}%
\pgfpathlineto{\pgfqpoint{1.350118in}{0.548008in}}%
\pgfpathlineto{\pgfqpoint{1.352695in}{0.550692in}}%
\pgfpathlineto{\pgfqpoint{1.355272in}{0.544433in}}%
\pgfpathlineto{\pgfqpoint{1.363004in}{0.548754in}}%
\pgfpathlineto{\pgfqpoint{1.365581in}{0.547934in}}%
\pgfpathlineto{\pgfqpoint{1.373312in}{0.534084in}}%
\pgfpathlineto{\pgfqpoint{1.381044in}{0.533966in}}%
\pgfpathlineto{\pgfqpoint{1.386199in}{0.528946in}}%
\pgfpathlineto{\pgfqpoint{1.388776in}{0.521734in}}%
\pgfpathlineto{\pgfqpoint{1.391353in}{0.518876in}}%
\pgfpathlineto{\pgfqpoint{1.401662in}{0.525361in}}%
\pgfpathlineto{\pgfqpoint{1.404239in}{0.533895in}}%
\pgfpathlineto{\pgfqpoint{1.406816in}{0.530580in}}%
\pgfpathlineto{\pgfqpoint{1.409393in}{0.538595in}}%
\pgfpathlineto{\pgfqpoint{1.419702in}{0.540375in}}%
\pgfpathlineto{\pgfqpoint{1.422280in}{0.545784in}}%
\pgfpathlineto{\pgfqpoint{1.424857in}{0.534890in}}%
\pgfpathlineto{\pgfqpoint{1.427434in}{0.540847in}}%
\pgfpathlineto{\pgfqpoint{1.435166in}{0.533906in}}%
\pgfpathlineto{\pgfqpoint{1.437743in}{0.535299in}}%
\pgfpathlineto{\pgfqpoint{1.440320in}{0.531284in}}%
\pgfpathlineto{\pgfqpoint{1.442897in}{0.529478in}}%
\pgfpathlineto{\pgfqpoint{1.445474in}{0.521962in}}%
\pgfpathlineto{\pgfqpoint{1.453206in}{0.523119in}}%
\pgfpathlineto{\pgfqpoint{1.455783in}{0.516797in}}%
\pgfpathlineto{\pgfqpoint{1.463515in}{0.503142in}}%
\pgfpathlineto{\pgfqpoint{1.471247in}{0.505951in}}%
\pgfpathlineto{\pgfqpoint{1.473824in}{0.508194in}}%
\pgfpathlineto{\pgfqpoint{1.476401in}{0.515107in}}%
\pgfpathlineto{\pgfqpoint{1.478978in}{0.513555in}}%
\pgfpathlineto{\pgfqpoint{1.481555in}{0.519716in}}%
\pgfpathlineto{\pgfqpoint{1.489287in}{0.509239in}}%
\pgfpathlineto{\pgfqpoint{1.491864in}{0.511427in}}%
\pgfpathlineto{\pgfqpoint{1.494441in}{0.518822in}}%
\pgfpathlineto{\pgfqpoint{1.497019in}{0.515405in}}%
\pgfpathlineto{\pgfqpoint{1.499596in}{0.515092in}}%
\pgfpathlineto{\pgfqpoint{1.507328in}{0.520969in}}%
\pgfpathlineto{\pgfqpoint{1.509905in}{0.519421in}}%
\pgfpathlineto{\pgfqpoint{1.512482in}{0.519694in}}%
\pgfpathlineto{\pgfqpoint{1.515059in}{0.518307in}}%
\pgfpathlineto{\pgfqpoint{1.517636in}{0.518793in}}%
\pgfpathlineto{\pgfqpoint{1.525368in}{0.513310in}}%
\pgfpathlineto{\pgfqpoint{1.527945in}{0.505685in}}%
\pgfpathlineto{\pgfqpoint{1.530522in}{0.518112in}}%
\pgfpathlineto{\pgfqpoint{1.533100in}{0.508469in}}%
\pgfpathlineto{\pgfqpoint{1.535677in}{0.505803in}}%
\pgfpathlineto{\pgfqpoint{1.543409in}{0.500805in}}%
\pgfpathlineto{\pgfqpoint{1.545986in}{0.491887in}}%
\pgfpathlineto{\pgfqpoint{1.548563in}{0.488285in}}%
\pgfpathlineto{\pgfqpoint{1.551140in}{0.482866in}}%
\pgfpathlineto{\pgfqpoint{1.553717in}{0.484469in}}%
\pgfpathlineto{\pgfqpoint{1.566603in}{0.487502in}}%
\pgfpathlineto{\pgfqpoint{1.569181in}{0.484503in}}%
\pgfpathlineto{\pgfqpoint{1.571758in}{0.485460in}}%
\pgfpathlineto{\pgfqpoint{1.579489in}{0.482052in}}%
\pgfpathlineto{\pgfqpoint{1.582067in}{0.483065in}}%
\pgfpathlineto{\pgfqpoint{1.587221in}{0.469512in}}%
\pgfpathlineto{\pgfqpoint{1.589798in}{0.485366in}}%
\pgfpathlineto{\pgfqpoint{1.597530in}{0.493460in}}%
\pgfpathlineto{\pgfqpoint{1.600107in}{0.482458in}}%
\pgfpathlineto{\pgfqpoint{1.602684in}{0.483836in}}%
\pgfpathlineto{\pgfqpoint{1.605262in}{0.487250in}}%
\pgfpathlineto{\pgfqpoint{1.607839in}{0.486027in}}%
\pgfpathlineto{\pgfqpoint{1.615570in}{0.488064in}}%
\pgfpathlineto{\pgfqpoint{1.618148in}{0.479060in}}%
\pgfpathlineto{\pgfqpoint{1.620725in}{0.475490in}}%
\pgfpathlineto{\pgfqpoint{1.625879in}{0.474560in}}%
\pgfpathlineto{\pgfqpoint{1.633611in}{0.465347in}}%
\pgfpathlineto{\pgfqpoint{1.636188in}{0.454977in}}%
\pgfpathlineto{\pgfqpoint{1.638765in}{0.460140in}}%
\pgfpathlineto{\pgfqpoint{1.641343in}{0.456901in}}%
\pgfpathlineto{\pgfqpoint{1.643920in}{0.458092in}}%
\pgfpathlineto{\pgfqpoint{1.651651in}{0.458910in}}%
\pgfpathlineto{\pgfqpoint{1.654229in}{0.463277in}}%
\pgfpathlineto{\pgfqpoint{1.661960in}{0.479512in}}%
\pgfpathlineto{\pgfqpoint{1.669692in}{0.473789in}}%
\pgfpathlineto{\pgfqpoint{1.672269in}{0.477530in}}%
\pgfpathlineto{\pgfqpoint{1.677424in}{0.474580in}}%
\pgfpathlineto{\pgfqpoint{1.680001in}{0.476873in}}%
\pgfpathlineto{\pgfqpoint{1.687732in}{0.479177in}}%
\pgfpathlineto{\pgfqpoint{1.690310in}{0.477857in}}%
\pgfpathlineto{\pgfqpoint{1.692887in}{0.459745in}}%
\pgfpathlineto{\pgfqpoint{1.695464in}{0.466933in}}%
\pgfpathlineto{\pgfqpoint{1.698041in}{0.470092in}}%
\pgfpathlineto{\pgfqpoint{1.705773in}{0.469256in}}%
\pgfpathlineto{\pgfqpoint{1.713505in}{0.473503in}}%
\pgfpathlineto{\pgfqpoint{1.716082in}{0.476950in}}%
\pgfpathlineto{\pgfqpoint{1.723813in}{0.476981in}}%
\pgfpathlineto{\pgfqpoint{1.728968in}{0.474441in}}%
\pgfpathlineto{\pgfqpoint{1.734122in}{0.467297in}}%
\pgfpathlineto{\pgfqpoint{1.744431in}{0.477115in}}%
\pgfpathlineto{\pgfqpoint{1.747008in}{0.475171in}}%
\pgfpathlineto{\pgfqpoint{1.749586in}{0.476962in}}%
\pgfpathlineto{\pgfqpoint{1.759894in}{0.473879in}}%
\pgfpathlineto{\pgfqpoint{1.767626in}{0.456582in}}%
\pgfpathlineto{\pgfqpoint{1.770203in}{0.468153in}}%
\pgfpathlineto{\pgfqpoint{1.780512in}{0.462851in}}%
\pgfpathlineto{\pgfqpoint{1.783089in}{0.455142in}}%
\pgfpathlineto{\pgfqpoint{1.785666in}{0.456788in}}%
\pgfpathlineto{\pgfqpoint{1.788244in}{0.453553in}}%
\pgfpathlineto{\pgfqpoint{1.795975in}{0.454651in}}%
\pgfpathlineto{\pgfqpoint{1.798553in}{0.452141in}}%
\pgfpathlineto{\pgfqpoint{1.801130in}{0.447708in}}%
\pgfpathlineto{\pgfqpoint{1.803707in}{0.458601in}}%
\pgfpathlineto{\pgfqpoint{1.806284in}{0.465037in}}%
\pgfpathlineto{\pgfqpoint{1.814016in}{0.462394in}}%
\pgfpathlineto{\pgfqpoint{1.816593in}{0.465202in}}%
\pgfpathlineto{\pgfqpoint{1.819170in}{0.464928in}}%
\pgfpathlineto{\pgfqpoint{1.821747in}{0.465672in}}%
\pgfpathlineto{\pgfqpoint{1.824325in}{0.470979in}}%
\pgfpathlineto{\pgfqpoint{1.832056in}{0.471131in}}%
\pgfpathlineto{\pgfqpoint{1.834634in}{0.469985in}}%
\pgfpathlineto{\pgfqpoint{1.839788in}{0.473798in}}%
\pgfpathlineto{\pgfqpoint{1.842365in}{0.477562in}}%
\pgfpathlineto{\pgfqpoint{1.852674in}{0.480179in}}%
\pgfpathlineto{\pgfqpoint{1.855251in}{0.476243in}}%
\pgfpathlineto{\pgfqpoint{1.857828in}{0.465446in}}%
\pgfpathlineto{\pgfqpoint{1.860406in}{0.465962in}}%
\pgfpathlineto{\pgfqpoint{1.868137in}{0.468477in}}%
\pgfpathlineto{\pgfqpoint{1.870714in}{0.472401in}}%
\pgfpathlineto{\pgfqpoint{1.873292in}{0.469030in}}%
\pgfpathlineto{\pgfqpoint{1.875869in}{0.470100in}}%
\pgfpathlineto{\pgfqpoint{1.878446in}{0.469656in}}%
\pgfpathlineto{\pgfqpoint{1.886178in}{0.464528in}}%
\pgfpathlineto{\pgfqpoint{1.888755in}{0.469584in}}%
\pgfpathlineto{\pgfqpoint{1.891332in}{0.468881in}}%
\pgfpathlineto{\pgfqpoint{1.893909in}{0.470015in}}%
\pgfpathlineto{\pgfqpoint{1.896487in}{0.474582in}}%
\pgfpathlineto{\pgfqpoint{1.904218in}{0.475463in}}%
\pgfpathlineto{\pgfqpoint{1.906795in}{0.476385in}}%
\pgfpathlineto{\pgfqpoint{1.909373in}{0.481607in}}%
\pgfpathlineto{\pgfqpoint{1.922259in}{0.486826in}}%
\pgfpathlineto{\pgfqpoint{1.924836in}{0.478261in}}%
\pgfpathlineto{\pgfqpoint{1.927413in}{0.489200in}}%
\pgfpathlineto{\pgfqpoint{1.932568in}{0.499145in}}%
\pgfpathlineto{\pgfqpoint{1.940299in}{0.491112in}}%
\pgfpathlineto{\pgfqpoint{1.945454in}{0.484082in}}%
\pgfpathlineto{\pgfqpoint{1.948031in}{0.485241in}}%
\pgfpathlineto{\pgfqpoint{1.958340in}{0.485745in}}%
\pgfpathlineto{\pgfqpoint{1.960917in}{0.479361in}}%
\pgfpathlineto{\pgfqpoint{1.963494in}{0.478502in}}%
\pgfpathlineto{\pgfqpoint{1.966071in}{0.481409in}}%
\pgfpathlineto{\pgfqpoint{1.968649in}{0.488599in}}%
\pgfpathlineto{\pgfqpoint{1.981535in}{0.480424in}}%
\pgfpathlineto{\pgfqpoint{1.984112in}{0.489012in}}%
\pgfpathlineto{\pgfqpoint{1.986689in}{0.482908in}}%
\pgfpathlineto{\pgfqpoint{1.994421in}{0.475636in}}%
\pgfpathlineto{\pgfqpoint{1.996998in}{0.488673in}}%
\pgfpathlineto{\pgfqpoint{1.999575in}{0.481316in}}%
\pgfpathlineto{\pgfqpoint{2.002152in}{0.483858in}}%
\pgfpathlineto{\pgfqpoint{2.004730in}{0.479094in}}%
\pgfpathlineto{\pgfqpoint{2.012461in}{0.476496in}}%
\pgfpathlineto{\pgfqpoint{2.015038in}{0.472783in}}%
\pgfpathlineto{\pgfqpoint{2.017616in}{0.461784in}}%
\pgfpathlineto{\pgfqpoint{2.020193in}{0.466218in}}%
\pgfpathlineto{\pgfqpoint{2.022770in}{0.462747in}}%
\pgfpathlineto{\pgfqpoint{2.030502in}{0.464502in}}%
\pgfpathlineto{\pgfqpoint{2.033079in}{0.469017in}}%
\pgfpathlineto{\pgfqpoint{2.035656in}{0.467403in}}%
\pgfpathlineto{\pgfqpoint{2.038233in}{0.479107in}}%
\pgfpathlineto{\pgfqpoint{2.040811in}{0.469921in}}%
\pgfpathlineto{\pgfqpoint{2.048542in}{0.466712in}}%
\pgfpathlineto{\pgfqpoint{2.053697in}{0.470697in}}%
\pgfpathlineto{\pgfqpoint{2.056274in}{0.476601in}}%
\pgfpathlineto{\pgfqpoint{2.058851in}{0.475210in}}%
\pgfpathlineto{\pgfqpoint{2.066583in}{0.473371in}}%
\pgfpathlineto{\pgfqpoint{2.069160in}{0.470895in}}%
\pgfpathlineto{\pgfqpoint{2.071737in}{0.470246in}}%
\pgfpathlineto{\pgfqpoint{2.074314in}{0.468352in}}%
\pgfpathlineto{\pgfqpoint{2.076891in}{0.465369in}}%
\pgfpathlineto{\pgfqpoint{2.084623in}{0.464998in}}%
\pgfpathlineto{\pgfqpoint{2.087200in}{0.459748in}}%
\pgfpathlineto{\pgfqpoint{2.089778in}{0.458268in}}%
\pgfpathlineto{\pgfqpoint{2.094932in}{0.464212in}}%
\pgfpathlineto{\pgfqpoint{2.105241in}{0.467510in}}%
\pgfpathlineto{\pgfqpoint{2.107818in}{0.455782in}}%
\pgfpathlineto{\pgfqpoint{2.110395in}{0.454778in}}%
\pgfpathlineto{\pgfqpoint{2.120704in}{0.443973in}}%
\pgfpathlineto{\pgfqpoint{2.123281in}{0.437638in}}%
\pgfpathlineto{\pgfqpoint{2.125859in}{0.445517in}}%
\pgfpathlineto{\pgfqpoint{2.131013in}{0.437723in}}%
\pgfpathlineto{\pgfqpoint{2.138745in}{0.450499in}}%
\pgfpathlineto{\pgfqpoint{2.141322in}{0.456856in}}%
\pgfpathlineto{\pgfqpoint{2.143899in}{0.454654in}}%
\pgfpathlineto{\pgfqpoint{2.146476in}{0.455374in}}%
\pgfpathlineto{\pgfqpoint{2.149053in}{0.447508in}}%
\pgfpathlineto{\pgfqpoint{2.156785in}{0.453875in}}%
\pgfpathlineto{\pgfqpoint{2.159362in}{0.458265in}}%
\pgfpathlineto{\pgfqpoint{2.161940in}{0.460811in}}%
\pgfpathlineto{\pgfqpoint{2.164517in}{0.460627in}}%
\pgfpathlineto{\pgfqpoint{2.167094in}{0.463356in}}%
\pgfpathlineto{\pgfqpoint{2.174826in}{0.458007in}}%
\pgfpathlineto{\pgfqpoint{2.179980in}{0.462547in}}%
\pgfpathlineto{\pgfqpoint{2.182557in}{0.464533in}}%
\pgfpathlineto{\pgfqpoint{2.185134in}{0.467760in}}%
\pgfpathlineto{\pgfqpoint{2.192866in}{0.471320in}}%
\pgfpathlineto{\pgfqpoint{2.195443in}{0.468958in}}%
\pgfpathlineto{\pgfqpoint{2.198020in}{0.467809in}}%
\pgfpathlineto{\pgfqpoint{2.203175in}{0.467363in}}%
\pgfpathlineto{\pgfqpoint{2.210907in}{0.474636in}}%
\pgfpathlineto{\pgfqpoint{2.213484in}{0.469240in}}%
\pgfpathlineto{\pgfqpoint{2.216061in}{0.466796in}}%
\pgfpathlineto{\pgfqpoint{2.218638in}{0.462124in}}%
\pgfpathlineto{\pgfqpoint{2.221215in}{0.467895in}}%
\pgfpathlineto{\pgfqpoint{2.228947in}{0.465471in}}%
\pgfpathlineto{\pgfqpoint{2.231524in}{0.463975in}}%
\pgfpathlineto{\pgfqpoint{2.234101in}{0.465245in}}%
\pgfpathlineto{\pgfqpoint{2.236679in}{0.445825in}}%
\pgfpathlineto{\pgfqpoint{2.239256in}{0.443298in}}%
\pgfpathlineto{\pgfqpoint{2.246988in}{0.434590in}}%
\pgfpathlineto{\pgfqpoint{2.249565in}{0.439808in}}%
\pgfpathlineto{\pgfqpoint{2.252142in}{0.437293in}}%
\pgfpathlineto{\pgfqpoint{2.254719in}{0.447069in}}%
\pgfpathlineto{\pgfqpoint{2.257296in}{0.449946in}}%
\pgfpathlineto{\pgfqpoint{2.265028in}{0.448946in}}%
\pgfpathlineto{\pgfqpoint{2.267605in}{0.454468in}}%
\pgfpathlineto{\pgfqpoint{2.270182in}{0.451601in}}%
\pgfpathlineto{\pgfqpoint{2.272760in}{0.456990in}}%
\pgfpathlineto{\pgfqpoint{2.275337in}{0.460169in}}%
\pgfpathlineto{\pgfqpoint{2.283068in}{0.461008in}}%
\pgfpathlineto{\pgfqpoint{2.288223in}{0.469027in}}%
\pgfpathlineto{\pgfqpoint{2.290800in}{0.469669in}}%
\pgfpathlineto{\pgfqpoint{2.293377in}{0.472312in}}%
\pgfpathlineto{\pgfqpoint{2.301109in}{0.465435in}}%
\pgfpathlineto{\pgfqpoint{2.303686in}{0.473627in}}%
\pgfpathlineto{\pgfqpoint{2.306263in}{0.483863in}}%
\pgfpathlineto{\pgfqpoint{2.308841in}{0.488985in}}%
\pgfpathlineto{\pgfqpoint{2.311418in}{0.492498in}}%
\pgfpathlineto{\pgfqpoint{2.319149in}{0.494771in}}%
\pgfpathlineto{\pgfqpoint{2.321727in}{0.493843in}}%
\pgfpathlineto{\pgfqpoint{2.324304in}{0.508855in}}%
\pgfpathlineto{\pgfqpoint{2.326881in}{0.505880in}}%
\pgfpathlineto{\pgfqpoint{2.329458in}{0.513996in}}%
\pgfpathlineto{\pgfqpoint{2.337190in}{0.521262in}}%
\pgfpathlineto{\pgfqpoint{2.339767in}{0.529192in}}%
\pgfpathlineto{\pgfqpoint{2.342344in}{0.530959in}}%
\pgfpathlineto{\pgfqpoint{2.344922in}{0.544470in}}%
\pgfpathlineto{\pgfqpoint{2.347499in}{0.544366in}}%
\pgfpathlineto{\pgfqpoint{2.357808in}{0.541449in}}%
\pgfpathlineto{\pgfqpoint{2.360385in}{0.547672in}}%
\pgfpathlineto{\pgfqpoint{2.362962in}{0.547706in}}%
\pgfpathlineto{\pgfqpoint{2.365539in}{0.549029in}}%
\pgfpathlineto{\pgfqpoint{2.373271in}{0.540438in}}%
\pgfpathlineto{\pgfqpoint{2.375848in}{0.549416in}}%
\pgfpathlineto{\pgfqpoint{2.378425in}{0.542280in}}%
\pgfpathlineto{\pgfqpoint{2.381003in}{0.537702in}}%
\pgfpathlineto{\pgfqpoint{2.383580in}{0.539516in}}%
\pgfpathlineto{\pgfqpoint{2.391311in}{0.546309in}}%
\pgfpathlineto{\pgfqpoint{2.393889in}{0.552288in}}%
\pgfpathlineto{\pgfqpoint{2.396466in}{0.552092in}}%
\pgfpathlineto{\pgfqpoint{2.399043in}{0.546011in}}%
\pgfpathlineto{\pgfqpoint{2.401620in}{0.536678in}}%
\pgfpathlineto{\pgfqpoint{2.409352in}{0.547734in}}%
\pgfpathlineto{\pgfqpoint{2.411929in}{0.541707in}}%
\pgfpathlineto{\pgfqpoint{2.414506in}{0.538663in}}%
\pgfpathlineto{\pgfqpoint{2.417084in}{0.539923in}}%
\pgfpathlineto{\pgfqpoint{2.419661in}{0.537495in}}%
\pgfpathlineto{\pgfqpoint{2.427392in}{0.541117in}}%
\pgfpathlineto{\pgfqpoint{2.429970in}{0.540645in}}%
\pgfpathlineto{\pgfqpoint{2.432547in}{0.534509in}}%
\pgfpathlineto{\pgfqpoint{2.435124in}{0.535330in}}%
\pgfpathlineto{\pgfqpoint{2.437701in}{0.543450in}}%
\pgfpathlineto{\pgfqpoint{2.445433in}{0.553935in}}%
\pgfpathlineto{\pgfqpoint{2.448010in}{0.553951in}}%
\pgfpathlineto{\pgfqpoint{2.450587in}{0.557250in}}%
\pgfpathlineto{\pgfqpoint{2.453165in}{0.563620in}}%
\pgfpathlineto{\pgfqpoint{2.455742in}{0.561378in}}%
\pgfpathlineto{\pgfqpoint{2.466051in}{0.554446in}}%
\pgfpathlineto{\pgfqpoint{2.468628in}{0.554523in}}%
\pgfpathlineto{\pgfqpoint{2.471205in}{0.543863in}}%
\pgfpathlineto{\pgfqpoint{2.473782in}{0.537705in}}%
\pgfpathlineto{\pgfqpoint{2.481514in}{0.524700in}}%
\pgfpathlineto{\pgfqpoint{2.484091in}{0.532304in}}%
\pgfpathlineto{\pgfqpoint{2.486668in}{0.531014in}}%
\pgfpathlineto{\pgfqpoint{2.489245in}{0.535010in}}%
\pgfpathlineto{\pgfqpoint{2.491823in}{0.533300in}}%
\pgfpathlineto{\pgfqpoint{2.499554in}{0.537447in}}%
\pgfpathlineto{\pgfqpoint{2.502132in}{0.532276in}}%
\pgfpathlineto{\pgfqpoint{2.504709in}{0.534703in}}%
\pgfpathlineto{\pgfqpoint{2.507286in}{0.534728in}}%
\pgfpathlineto{\pgfqpoint{2.509863in}{0.536294in}}%
\pgfpathlineto{\pgfqpoint{2.517595in}{0.544836in}}%
\pgfpathlineto{\pgfqpoint{2.520172in}{0.553314in}}%
\pgfpathlineto{\pgfqpoint{2.522749in}{0.551360in}}%
\pgfpathlineto{\pgfqpoint{2.525326in}{0.546974in}}%
\pgfpathlineto{\pgfqpoint{2.527904in}{0.548753in}}%
\pgfpathlineto{\pgfqpoint{2.535635in}{0.549862in}}%
\pgfpathlineto{\pgfqpoint{2.538213in}{0.547785in}}%
\pgfpathlineto{\pgfqpoint{2.540790in}{0.547886in}}%
\pgfpathlineto{\pgfqpoint{2.543367in}{0.546251in}}%
\pgfpathlineto{\pgfqpoint{2.545944in}{0.550941in}}%
\pgfpathlineto{\pgfqpoint{2.553676in}{0.555328in}}%
\pgfpathlineto{\pgfqpoint{2.556253in}{0.549416in}}%
\pgfpathlineto{\pgfqpoint{2.558830in}{0.550756in}}%
\pgfpathlineto{\pgfqpoint{2.561407in}{0.553876in}}%
\pgfpathlineto{\pgfqpoint{2.563985in}{0.572898in}}%
\pgfpathlineto{\pgfqpoint{2.571716in}{0.572840in}}%
\pgfpathlineto{\pgfqpoint{2.574293in}{0.581363in}}%
\pgfpathlineto{\pgfqpoint{2.582025in}{0.580306in}}%
\pgfpathlineto{\pgfqpoint{2.589757in}{0.594387in}}%
\pgfpathlineto{\pgfqpoint{2.592334in}{0.603731in}}%
\pgfpathlineto{\pgfqpoint{2.594911in}{0.609148in}}%
\pgfpathlineto{\pgfqpoint{2.597488in}{0.600941in}}%
\pgfpathlineto{\pgfqpoint{2.600066in}{0.598777in}}%
\pgfpathlineto{\pgfqpoint{2.607797in}{0.596470in}}%
\pgfpathlineto{\pgfqpoint{2.610374in}{0.598903in}}%
\pgfpathlineto{\pgfqpoint{2.612952in}{0.622927in}}%
\pgfpathlineto{\pgfqpoint{2.615529in}{0.631371in}}%
\pgfpathlineto{\pgfqpoint{2.618106in}{0.634934in}}%
\pgfpathlineto{\pgfqpoint{2.625838in}{0.638701in}}%
\pgfpathlineto{\pgfqpoint{2.628415in}{0.634855in}}%
\pgfpathlineto{\pgfqpoint{2.630992in}{0.632800in}}%
\pgfpathlineto{\pgfqpoint{2.633569in}{0.632464in}}%
\pgfpathlineto{\pgfqpoint{2.636147in}{0.634847in}}%
\pgfpathlineto{\pgfqpoint{2.643878in}{0.639245in}}%
\pgfpathlineto{\pgfqpoint{2.646455in}{0.633883in}}%
\pgfpathlineto{\pgfqpoint{2.651610in}{0.632518in}}%
\pgfpathlineto{\pgfqpoint{2.654187in}{0.634222in}}%
\pgfpathlineto{\pgfqpoint{2.661919in}{0.646194in}}%
\pgfpathlineto{\pgfqpoint{2.664496in}{0.644452in}}%
\pgfpathlineto{\pgfqpoint{2.669650in}{0.647495in}}%
\pgfpathlineto{\pgfqpoint{2.672228in}{0.653064in}}%
\pgfpathlineto{\pgfqpoint{2.679959in}{0.652266in}}%
\pgfpathlineto{\pgfqpoint{2.682536in}{0.642559in}}%
\pgfpathlineto{\pgfqpoint{2.685114in}{0.643432in}}%
\pgfpathlineto{\pgfqpoint{2.687691in}{0.645215in}}%
\pgfpathlineto{\pgfqpoint{2.690268in}{0.646256in}}%
\pgfpathlineto{\pgfqpoint{2.698000in}{0.647201in}}%
\pgfpathlineto{\pgfqpoint{2.700577in}{0.638782in}}%
\pgfpathlineto{\pgfqpoint{2.703154in}{0.647250in}}%
\pgfpathlineto{\pgfqpoint{2.705731in}{0.646008in}}%
\pgfpathlineto{\pgfqpoint{2.708309in}{0.655775in}}%
\pgfpathlineto{\pgfqpoint{2.718617in}{0.650324in}}%
\pgfpathlineto{\pgfqpoint{2.721195in}{0.646755in}}%
\pgfpathlineto{\pgfqpoint{2.723772in}{0.648070in}}%
\pgfpathlineto{\pgfqpoint{2.726349in}{0.642138in}}%
\pgfpathlineto{\pgfqpoint{2.734081in}{0.643159in}}%
\pgfpathlineto{\pgfqpoint{2.736658in}{0.648873in}}%
\pgfpathlineto{\pgfqpoint{2.739235in}{0.657777in}}%
\pgfpathlineto{\pgfqpoint{2.744390in}{0.680980in}}%
\pgfpathlineto{\pgfqpoint{2.752121in}{0.692692in}}%
\pgfpathlineto{\pgfqpoint{2.754698in}{0.691035in}}%
\pgfpathlineto{\pgfqpoint{2.759853in}{0.669174in}}%
\pgfpathlineto{\pgfqpoint{2.762430in}{0.666170in}}%
\pgfpathlineto{\pgfqpoint{2.770162in}{0.663884in}}%
\pgfpathlineto{\pgfqpoint{2.772739in}{0.667817in}}%
\pgfpathlineto{\pgfqpoint{2.775316in}{0.673329in}}%
\pgfpathlineto{\pgfqpoint{2.777893in}{0.675972in}}%
\pgfpathlineto{\pgfqpoint{2.780470in}{0.672783in}}%
\pgfpathlineto{\pgfqpoint{2.790779in}{0.673605in}}%
\pgfpathlineto{\pgfqpoint{2.793357in}{0.685301in}}%
\pgfpathlineto{\pgfqpoint{2.795934in}{0.672699in}}%
\pgfpathlineto{\pgfqpoint{2.798511in}{0.681097in}}%
\pgfpathlineto{\pgfqpoint{2.806243in}{0.670201in}}%
\pgfpathlineto{\pgfqpoint{2.808820in}{0.670910in}}%
\pgfpathlineto{\pgfqpoint{2.811397in}{0.672818in}}%
\pgfpathlineto{\pgfqpoint{2.813974in}{0.673614in}}%
\pgfpathlineto{\pgfqpoint{2.816551in}{0.671461in}}%
\pgfpathlineto{\pgfqpoint{2.824283in}{0.666876in}}%
\pgfpathlineto{\pgfqpoint{2.826860in}{0.667649in}}%
\pgfpathlineto{\pgfqpoint{2.829438in}{0.666851in}}%
\pgfpathlineto{\pgfqpoint{2.832015in}{0.662190in}}%
\pgfpathlineto{\pgfqpoint{2.834592in}{0.659851in}}%
\pgfpathlineto{\pgfqpoint{2.842324in}{0.659662in}}%
\pgfpathlineto{\pgfqpoint{2.844901in}{0.650257in}}%
\pgfpathlineto{\pgfqpoint{2.847478in}{0.652547in}}%
\pgfpathlineto{\pgfqpoint{2.852632in}{0.649417in}}%
\pgfpathlineto{\pgfqpoint{2.862941in}{0.653841in}}%
\pgfpathlineto{\pgfqpoint{2.865519in}{0.657209in}}%
\pgfpathlineto{\pgfqpoint{2.868096in}{0.650595in}}%
\pgfpathlineto{\pgfqpoint{2.870673in}{0.656871in}}%
\pgfpathlineto{\pgfqpoint{2.878405in}{0.662187in}}%
\pgfpathlineto{\pgfqpoint{2.883559in}{0.648287in}}%
\pgfpathlineto{\pgfqpoint{2.886136in}{0.648013in}}%
\pgfpathlineto{\pgfqpoint{2.888713in}{0.656585in}}%
\pgfpathlineto{\pgfqpoint{2.896445in}{0.657608in}}%
\pgfpathlineto{\pgfqpoint{2.899022in}{0.660351in}}%
\pgfpathlineto{\pgfqpoint{2.904177in}{0.661259in}}%
\pgfpathlineto{\pgfqpoint{2.906754in}{0.661193in}}%
\pgfpathlineto{\pgfqpoint{2.914486in}{0.645107in}}%
\pgfpathlineto{\pgfqpoint{2.917063in}{0.633956in}}%
\pgfpathlineto{\pgfqpoint{2.919640in}{0.645859in}}%
\pgfpathlineto{\pgfqpoint{2.922217in}{0.663784in}}%
\pgfpathlineto{\pgfqpoint{2.924794in}{0.662486in}}%
\pgfpathlineto{\pgfqpoint{2.932526in}{0.671141in}}%
\pgfpathlineto{\pgfqpoint{2.935103in}{0.670534in}}%
\pgfpathlineto{\pgfqpoint{2.937680in}{0.667822in}}%
\pgfpathlineto{\pgfqpoint{2.940258in}{0.672195in}}%
\pgfpathlineto{\pgfqpoint{2.950567in}{0.679508in}}%
\pgfpathlineto{\pgfqpoint{2.953144in}{0.679483in}}%
\pgfpathlineto{\pgfqpoint{2.955721in}{0.675473in}}%
\pgfpathlineto{\pgfqpoint{2.958298in}{0.664997in}}%
\pgfpathlineto{\pgfqpoint{2.960875in}{0.662540in}}%
\pgfpathlineto{\pgfqpoint{2.968607in}{0.658969in}}%
\pgfpathlineto{\pgfqpoint{2.971184in}{0.669020in}}%
\pgfpathlineto{\pgfqpoint{2.973761in}{0.668697in}}%
\pgfpathlineto{\pgfqpoint{2.976339in}{0.664544in}}%
\pgfpathlineto{\pgfqpoint{2.978916in}{0.667757in}}%
\pgfpathlineto{\pgfqpoint{2.986647in}{0.669116in}}%
\pgfpathlineto{\pgfqpoint{2.989225in}{0.665305in}}%
\pgfpathlineto{\pgfqpoint{2.991802in}{0.670181in}}%
\pgfpathlineto{\pgfqpoint{2.994379in}{0.669264in}}%
\pgfpathlineto{\pgfqpoint{2.996956in}{0.669051in}}%
\pgfpathlineto{\pgfqpoint{3.004688in}{0.675112in}}%
\pgfpathlineto{\pgfqpoint{3.007265in}{0.671021in}}%
\pgfpathlineto{\pgfqpoint{3.009842in}{0.671828in}}%
\pgfpathlineto{\pgfqpoint{3.012420in}{0.664545in}}%
\pgfpathlineto{\pgfqpoint{3.014997in}{0.650867in}}%
\pgfpathlineto{\pgfqpoint{3.022728in}{0.653740in}}%
\pgfpathlineto{\pgfqpoint{3.025306in}{0.650838in}}%
\pgfpathlineto{\pgfqpoint{3.027883in}{0.653448in}}%
\pgfpathlineto{\pgfqpoint{3.030460in}{0.652183in}}%
\pgfpathlineto{\pgfqpoint{3.033037in}{0.656630in}}%
\pgfpathlineto{\pgfqpoint{3.043346in}{0.649763in}}%
\pgfpathlineto{\pgfqpoint{3.048501in}{0.650146in}}%
\pgfpathlineto{\pgfqpoint{3.051078in}{0.652369in}}%
\pgfpathlineto{\pgfqpoint{3.058809in}{0.655313in}}%
\pgfpathlineto{\pgfqpoint{3.061387in}{0.647503in}}%
\pgfpathlineto{\pgfqpoint{3.066541in}{0.650743in}}%
\pgfpathlineto{\pgfqpoint{3.069118in}{0.647954in}}%
\pgfpathlineto{\pgfqpoint{3.076850in}{0.652555in}}%
\pgfpathlineto{\pgfqpoint{3.079427in}{0.665173in}}%
\pgfpathlineto{\pgfqpoint{3.082004in}{0.667318in}}%
\pgfpathlineto{\pgfqpoint{3.084582in}{0.667718in}}%
\pgfpathlineto{\pgfqpoint{3.087159in}{0.642988in}}%
\pgfpathlineto{\pgfqpoint{3.094890in}{0.638933in}}%
\pgfpathlineto{\pgfqpoint{3.097468in}{0.641902in}}%
\pgfpathlineto{\pgfqpoint{3.100045in}{0.642460in}}%
\pgfpathlineto{\pgfqpoint{3.102622in}{0.646111in}}%
\pgfpathlineto{\pgfqpoint{3.105199in}{0.640924in}}%
\pgfpathlineto{\pgfqpoint{3.112931in}{0.640738in}}%
\pgfpathlineto{\pgfqpoint{3.115508in}{0.636963in}}%
\pgfpathlineto{\pgfqpoint{3.118085in}{0.629676in}}%
\pgfpathlineto{\pgfqpoint{3.120663in}{0.625469in}}%
\pgfpathlineto{\pgfqpoint{3.123240in}{0.625772in}}%
\pgfpathlineto{\pgfqpoint{3.130971in}{0.625045in}}%
\pgfpathlineto{\pgfqpoint{3.136126in}{0.617691in}}%
\pgfpathlineto{\pgfqpoint{3.138703in}{0.617443in}}%
\pgfpathlineto{\pgfqpoint{3.149012in}{0.622177in}}%
\pgfpathlineto{\pgfqpoint{3.151589in}{0.615159in}}%
\pgfpathlineto{\pgfqpoint{3.154166in}{0.610490in}}%
\pgfpathlineto{\pgfqpoint{3.156744in}{0.609566in}}%
\pgfpathlineto{\pgfqpoint{3.159321in}{0.610464in}}%
\pgfpathlineto{\pgfqpoint{3.167052in}{0.609152in}}%
\pgfpathlineto{\pgfqpoint{3.169630in}{0.602950in}}%
\pgfpathlineto{\pgfqpoint{3.172207in}{0.563272in}}%
\pgfpathlineto{\pgfqpoint{3.174784in}{0.570142in}}%
\pgfpathlineto{\pgfqpoint{3.177361in}{0.570028in}}%
\pgfpathlineto{\pgfqpoint{3.185093in}{0.570780in}}%
\pgfpathlineto{\pgfqpoint{3.187670in}{0.564107in}}%
\pgfpathlineto{\pgfqpoint{3.190247in}{0.563942in}}%
\pgfpathlineto{\pgfqpoint{3.192824in}{0.557373in}}%
\pgfpathlineto{\pgfqpoint{3.195402in}{0.570992in}}%
\pgfpathlineto{\pgfqpoint{3.203133in}{0.568085in}}%
\pgfpathlineto{\pgfqpoint{3.205711in}{0.564770in}}%
\pgfpathlineto{\pgfqpoint{3.208288in}{0.565478in}}%
\pgfpathlineto{\pgfqpoint{3.210865in}{0.562393in}}%
\pgfpathlineto{\pgfqpoint{3.213442in}{0.544212in}}%
\pgfpathlineto{\pgfqpoint{3.221174in}{0.551135in}}%
\pgfpathlineto{\pgfqpoint{3.223751in}{0.560042in}}%
\pgfpathlineto{\pgfqpoint{3.226328in}{0.553646in}}%
\pgfpathlineto{\pgfqpoint{3.231483in}{0.555777in}}%
\pgfpathlineto{\pgfqpoint{3.239214in}{0.562424in}}%
\pgfpathlineto{\pgfqpoint{3.241792in}{0.560333in}}%
\pgfpathlineto{\pgfqpoint{3.244369in}{0.556890in}}%
\pgfpathlineto{\pgfqpoint{3.246946in}{0.559130in}}%
\pgfpathlineto{\pgfqpoint{3.249523in}{0.557067in}}%
\pgfpathlineto{\pgfqpoint{3.257255in}{0.564137in}}%
\pgfpathlineto{\pgfqpoint{3.259832in}{0.564702in}}%
\pgfpathlineto{\pgfqpoint{3.262409in}{0.562246in}}%
\pgfpathlineto{\pgfqpoint{3.264986in}{0.564119in}}%
\pgfpathlineto{\pgfqpoint{3.267564in}{0.566974in}}%
\pgfpathlineto{\pgfqpoint{3.275295in}{0.568450in}}%
\pgfpathlineto{\pgfqpoint{3.277872in}{0.567957in}}%
\pgfpathlineto{\pgfqpoint{3.283027in}{0.572083in}}%
\pgfpathlineto{\pgfqpoint{3.285604in}{0.571996in}}%
\pgfpathlineto{\pgfqpoint{3.295913in}{0.582230in}}%
\pgfpathlineto{\pgfqpoint{3.298490in}{0.582689in}}%
\pgfpathlineto{\pgfqpoint{3.301067in}{0.586995in}}%
\pgfpathlineto{\pgfqpoint{3.303645in}{0.588635in}}%
\pgfpathlineto{\pgfqpoint{3.311376in}{0.587527in}}%
\pgfpathlineto{\pgfqpoint{3.313953in}{0.585696in}}%
\pgfpathlineto{\pgfqpoint{3.316531in}{0.582450in}}%
\pgfpathlineto{\pgfqpoint{3.319108in}{0.588600in}}%
\pgfpathlineto{\pgfqpoint{3.321685in}{0.592562in}}%
\pgfpathlineto{\pgfqpoint{3.329417in}{0.592541in}}%
\pgfpathlineto{\pgfqpoint{3.331994in}{0.598637in}}%
\pgfpathlineto{\pgfqpoint{3.334571in}{0.599596in}}%
\pgfpathlineto{\pgfqpoint{3.337148in}{0.595330in}}%
\pgfpathlineto{\pgfqpoint{3.339726in}{0.596497in}}%
\pgfpathlineto{\pgfqpoint{3.347457in}{0.594353in}}%
\pgfpathlineto{\pgfqpoint{3.350034in}{0.590157in}}%
\pgfpathlineto{\pgfqpoint{3.352612in}{0.593985in}}%
\pgfpathlineto{\pgfqpoint{3.355189in}{0.592223in}}%
\pgfpathlineto{\pgfqpoint{3.357766in}{0.597693in}}%
\pgfpathlineto{\pgfqpoint{3.365498in}{0.596664in}}%
\pgfpathlineto{\pgfqpoint{3.370652in}{0.602800in}}%
\pgfpathlineto{\pgfqpoint{3.373229in}{0.597933in}}%
\pgfpathlineto{\pgfqpoint{3.375807in}{0.601531in}}%
\pgfpathlineto{\pgfqpoint{3.383538in}{0.601895in}}%
\pgfpathlineto{\pgfqpoint{3.386115in}{0.603811in}}%
\pgfpathlineto{\pgfqpoint{3.388693in}{0.606886in}}%
\pgfpathlineto{\pgfqpoint{3.391270in}{0.604568in}}%
\pgfpathlineto{\pgfqpoint{3.393847in}{0.603411in}}%
\pgfpathlineto{\pgfqpoint{3.401579in}{0.610768in}}%
\pgfpathlineto{\pgfqpoint{3.404156in}{0.611390in}}%
\pgfpathlineto{\pgfqpoint{3.406733in}{0.620549in}}%
\pgfpathlineto{\pgfqpoint{3.409310in}{0.619704in}}%
\pgfpathlineto{\pgfqpoint{3.411888in}{0.621428in}}%
\pgfpathlineto{\pgfqpoint{3.419619in}{0.615233in}}%
\pgfpathlineto{\pgfqpoint{3.422196in}{0.610450in}}%
\pgfpathlineto{\pgfqpoint{3.424774in}{0.612391in}}%
\pgfpathlineto{\pgfqpoint{3.427351in}{0.633334in}}%
\pgfpathlineto{\pgfqpoint{3.429928in}{0.634668in}}%
\pgfpathlineto{\pgfqpoint{3.437660in}{0.632213in}}%
\pgfpathlineto{\pgfqpoint{3.442814in}{0.642674in}}%
\pgfpathlineto{\pgfqpoint{3.445391in}{0.687335in}}%
\pgfpathlineto{\pgfqpoint{3.447969in}{0.695380in}}%
\pgfpathlineto{\pgfqpoint{3.455700in}{0.690982in}}%
\pgfpathlineto{\pgfqpoint{3.458277in}{0.688414in}}%
\pgfpathlineto{\pgfqpoint{3.460855in}{0.688137in}}%
\pgfpathlineto{\pgfqpoint{3.463432in}{0.690784in}}%
\pgfpathlineto{\pgfqpoint{3.466009in}{0.704836in}}%
\pgfpathlineto{\pgfqpoint{3.473741in}{0.710315in}}%
\pgfpathlineto{\pgfqpoint{3.476318in}{0.711069in}}%
\pgfpathlineto{\pgfqpoint{3.478895in}{0.713001in}}%
\pgfpathlineto{\pgfqpoint{3.481472in}{0.710343in}}%
\pgfpathlineto{\pgfqpoint{3.484049in}{0.701757in}}%
\pgfpathlineto{\pgfqpoint{3.491781in}{0.701177in}}%
\pgfpathlineto{\pgfqpoint{3.494358in}{0.692060in}}%
\pgfpathlineto{\pgfqpoint{3.496936in}{0.690186in}}%
\pgfpathlineto{\pgfqpoint{3.499513in}{0.703297in}}%
\pgfpathlineto{\pgfqpoint{3.502090in}{0.703522in}}%
\pgfpathlineto{\pgfqpoint{3.514976in}{0.725538in}}%
\pgfpathlineto{\pgfqpoint{3.520130in}{0.718981in}}%
\pgfpathlineto{\pgfqpoint{3.527862in}{0.717002in}}%
\pgfpathlineto{\pgfqpoint{3.530439in}{0.717320in}}%
\pgfpathlineto{\pgfqpoint{3.533017in}{0.717067in}}%
\pgfpathlineto{\pgfqpoint{3.535594in}{0.725115in}}%
\pgfpathlineto{\pgfqpoint{3.538171in}{0.724552in}}%
\pgfpathlineto{\pgfqpoint{3.545903in}{0.718704in}}%
\pgfpathlineto{\pgfqpoint{3.548480in}{0.708610in}}%
\pgfpathlineto{\pgfqpoint{3.551057in}{0.705804in}}%
\pgfpathlineto{\pgfqpoint{3.553634in}{0.706358in}}%
\pgfpathlineto{\pgfqpoint{3.556211in}{0.697543in}}%
\pgfpathlineto{\pgfqpoint{3.563943in}{0.697345in}}%
\pgfpathlineto{\pgfqpoint{3.566520in}{0.688250in}}%
\pgfpathlineto{\pgfqpoint{3.569098in}{0.699592in}}%
\pgfpathlineto{\pgfqpoint{3.574252in}{0.703204in}}%
\pgfpathlineto{\pgfqpoint{3.581984in}{0.709296in}}%
\pgfpathlineto{\pgfqpoint{3.584561in}{0.710413in}}%
\pgfpathlineto{\pgfqpoint{3.587138in}{0.713103in}}%
\pgfpathlineto{\pgfqpoint{3.592292in}{0.716201in}}%
\pgfpathlineto{\pgfqpoint{3.600024in}{0.712792in}}%
\pgfpathlineto{\pgfqpoint{3.602601in}{0.712668in}}%
\pgfpathlineto{\pgfqpoint{3.605178in}{0.713518in}}%
\pgfpathlineto{\pgfqpoint{3.610333in}{0.712715in}}%
\pgfpathlineto{\pgfqpoint{3.618065in}{0.715457in}}%
\pgfpathlineto{\pgfqpoint{3.620642in}{0.707211in}}%
\pgfpathlineto{\pgfqpoint{3.623219in}{0.715777in}}%
\pgfpathlineto{\pgfqpoint{3.628373in}{0.724475in}}%
\pgfpathlineto{\pgfqpoint{3.636105in}{0.723743in}}%
\pgfpathlineto{\pgfqpoint{3.638682in}{0.721970in}}%
\pgfpathlineto{\pgfqpoint{3.643837in}{0.712129in}}%
\pgfpathlineto{\pgfqpoint{3.646414in}{0.708261in}}%
\pgfpathlineto{\pgfqpoint{3.656723in}{0.711530in}}%
\pgfpathlineto{\pgfqpoint{3.659300in}{0.711271in}}%
\pgfpathlineto{\pgfqpoint{3.661877in}{0.703017in}}%
\pgfpathlineto{\pgfqpoint{3.664454in}{0.706574in}}%
\pgfpathlineto{\pgfqpoint{3.672186in}{0.705628in}}%
\pgfpathlineto{\pgfqpoint{3.674763in}{0.697187in}}%
\pgfpathlineto{\pgfqpoint{3.677340in}{0.696205in}}%
\pgfpathlineto{\pgfqpoint{3.679918in}{0.690356in}}%
\pgfpathlineto{\pgfqpoint{3.682495in}{0.702597in}}%
\pgfpathlineto{\pgfqpoint{3.690226in}{0.707793in}}%
\pgfpathlineto{\pgfqpoint{3.692804in}{0.712979in}}%
\pgfpathlineto{\pgfqpoint{3.695381in}{0.687571in}}%
\pgfpathlineto{\pgfqpoint{3.697958in}{0.682441in}}%
\pgfpathlineto{\pgfqpoint{3.700535in}{0.693204in}}%
\pgfpathlineto{\pgfqpoint{3.708267in}{0.688424in}}%
\pgfpathlineto{\pgfqpoint{3.715999in}{0.704070in}}%
\pgfpathlineto{\pgfqpoint{3.718576in}{0.696767in}}%
\pgfpathlineto{\pgfqpoint{3.728885in}{0.699847in}}%
\pgfpathlineto{\pgfqpoint{3.731462in}{0.703066in}}%
\pgfpathlineto{\pgfqpoint{3.734039in}{0.713029in}}%
\pgfpathlineto{\pgfqpoint{3.736616in}{0.713673in}}%
\pgfpathlineto{\pgfqpoint{3.744348in}{0.712281in}}%
\pgfpathlineto{\pgfqpoint{3.749502in}{0.728493in}}%
\pgfpathlineto{\pgfqpoint{3.752080in}{0.718686in}}%
\pgfpathlineto{\pgfqpoint{3.754657in}{0.721010in}}%
\pgfpathlineto{\pgfqpoint{3.764966in}{0.732957in}}%
\pgfpathlineto{\pgfqpoint{3.767543in}{0.734816in}}%
\pgfpathlineto{\pgfqpoint{3.770120in}{0.741110in}}%
\pgfpathlineto{\pgfqpoint{3.772697in}{0.743507in}}%
\pgfpathlineto{\pgfqpoint{3.780429in}{0.739322in}}%
\pgfpathlineto{\pgfqpoint{3.783006in}{0.743047in}}%
\pgfpathlineto{\pgfqpoint{3.785583in}{0.756654in}}%
\pgfpathlineto{\pgfqpoint{3.788161in}{0.777402in}}%
\pgfpathlineto{\pgfqpoint{3.790738in}{0.780643in}}%
\pgfpathlineto{\pgfqpoint{3.798469in}{0.788367in}}%
\pgfpathlineto{\pgfqpoint{3.801047in}{0.791803in}}%
\pgfpathlineto{\pgfqpoint{3.803624in}{0.796665in}}%
\pgfpathlineto{\pgfqpoint{3.808778in}{0.787754in}}%
\pgfpathlineto{\pgfqpoint{3.816510in}{0.787196in}}%
\pgfpathlineto{\pgfqpoint{3.819087in}{0.790888in}}%
\pgfpathlineto{\pgfqpoint{3.821664in}{0.791835in}}%
\pgfpathlineto{\pgfqpoint{3.824242in}{0.800549in}}%
\pgfpathlineto{\pgfqpoint{3.826819in}{0.773196in}}%
\pgfpathlineto{\pgfqpoint{3.834550in}{0.774432in}}%
\pgfpathlineto{\pgfqpoint{3.837128in}{0.781444in}}%
\pgfpathlineto{\pgfqpoint{3.842282in}{0.767791in}}%
\pgfpathlineto{\pgfqpoint{3.852591in}{0.769801in}}%
\pgfpathlineto{\pgfqpoint{3.855168in}{0.773525in}}%
\pgfpathlineto{\pgfqpoint{3.857745in}{0.772721in}}%
\pgfpathlineto{\pgfqpoint{3.860323in}{0.783821in}}%
\pgfpathlineto{\pgfqpoint{3.862900in}{0.749986in}}%
\pgfpathlineto{\pgfqpoint{3.870631in}{0.757117in}}%
\pgfpathlineto{\pgfqpoint{3.873209in}{0.756695in}}%
\pgfpathlineto{\pgfqpoint{3.875786in}{0.776048in}}%
\pgfpathlineto{\pgfqpoint{3.878363in}{0.774110in}}%
\pgfpathlineto{\pgfqpoint{3.880940in}{0.794896in}}%
\pgfpathlineto{\pgfqpoint{3.888672in}{0.802088in}}%
\pgfpathlineto{\pgfqpoint{3.891249in}{0.799893in}}%
\pgfpathlineto{\pgfqpoint{3.893826in}{0.791316in}}%
\pgfpathlineto{\pgfqpoint{3.898981in}{0.785403in}}%
\pgfpathlineto{\pgfqpoint{3.906712in}{0.788752in}}%
\pgfpathlineto{\pgfqpoint{3.909290in}{0.784279in}}%
\pgfpathlineto{\pgfqpoint{3.911867in}{0.786899in}}%
\pgfpathlineto{\pgfqpoint{3.914444in}{0.795071in}}%
\pgfpathlineto{\pgfqpoint{3.917021in}{0.794619in}}%
\pgfpathlineto{\pgfqpoint{3.924753in}{0.792443in}}%
\pgfpathlineto{\pgfqpoint{3.927330in}{0.796995in}}%
\pgfpathlineto{\pgfqpoint{3.929907in}{0.796047in}}%
\pgfpathlineto{\pgfqpoint{3.932484in}{0.792634in}}%
\pgfpathlineto{\pgfqpoint{3.935062in}{0.777935in}}%
\pgfpathlineto{\pgfqpoint{3.942793in}{0.781037in}}%
\pgfpathlineto{\pgfqpoint{3.945371in}{0.792942in}}%
\pgfpathlineto{\pgfqpoint{3.947948in}{0.799076in}}%
\pgfpathlineto{\pgfqpoint{3.950525in}{0.797732in}}%
\pgfpathlineto{\pgfqpoint{3.953102in}{0.802154in}}%
\pgfpathlineto{\pgfqpoint{3.960834in}{0.801598in}}%
\pgfpathlineto{\pgfqpoint{3.963411in}{0.802631in}}%
\pgfpathlineto{\pgfqpoint{3.965988in}{0.809608in}}%
\pgfpathlineto{\pgfqpoint{3.968565in}{0.803093in}}%
\pgfpathlineto{\pgfqpoint{3.971143in}{0.803874in}}%
\pgfpathlineto{\pgfqpoint{3.981451in}{0.809509in}}%
\pgfpathlineto{\pgfqpoint{3.984029in}{0.814998in}}%
\pgfpathlineto{\pgfqpoint{3.986606in}{0.812666in}}%
\pgfpathlineto{\pgfqpoint{3.989183in}{0.807415in}}%
\pgfpathlineto{\pgfqpoint{3.996915in}{0.814048in}}%
\pgfpathlineto{\pgfqpoint{3.999492in}{0.807909in}}%
\pgfpathlineto{\pgfqpoint{4.002069in}{0.794824in}}%
\pgfpathlineto{\pgfqpoint{4.004646in}{0.796223in}}%
\pgfpathlineto{\pgfqpoint{4.007224in}{0.793025in}}%
\pgfpathlineto{\pgfqpoint{4.014955in}{0.792729in}}%
\pgfpathlineto{\pgfqpoint{4.017532in}{0.788577in}}%
\pgfpathlineto{\pgfqpoint{4.020110in}{0.786848in}}%
\pgfpathlineto{\pgfqpoint{4.022687in}{0.790038in}}%
\pgfpathlineto{\pgfqpoint{4.025264in}{0.792213in}}%
\pgfpathlineto{\pgfqpoint{4.032996in}{0.794444in}}%
\pgfpathlineto{\pgfqpoint{4.035573in}{0.790901in}}%
\pgfpathlineto{\pgfqpoint{4.038150in}{0.790055in}}%
\pgfpathlineto{\pgfqpoint{4.040727in}{0.792590in}}%
\pgfpathlineto{\pgfqpoint{4.043305in}{0.789993in}}%
\pgfpathlineto{\pgfqpoint{4.051036in}{0.793168in}}%
\pgfpathlineto{\pgfqpoint{4.053613in}{0.788421in}}%
\pgfpathlineto{\pgfqpoint{4.056191in}{0.790896in}}%
\pgfpathlineto{\pgfqpoint{4.058768in}{0.786078in}}%
\pgfpathlineto{\pgfqpoint{4.061345in}{0.785713in}}%
\pgfpathlineto{\pgfqpoint{4.069077in}{0.788164in}}%
\pgfpathlineto{\pgfqpoint{4.071654in}{0.793135in}}%
\pgfpathlineto{\pgfqpoint{4.074231in}{0.786346in}}%
\pgfpathlineto{\pgfqpoint{4.076808in}{0.786278in}}%
\pgfpathlineto{\pgfqpoint{4.087117in}{0.789755in}}%
\pgfpathlineto{\pgfqpoint{4.089694in}{0.777793in}}%
\pgfpathlineto{\pgfqpoint{4.092272in}{0.776336in}}%
\pgfpathlineto{\pgfqpoint{4.094849in}{0.784404in}}%
\pgfpathlineto{\pgfqpoint{4.097426in}{0.786196in}}%
\pgfpathlineto{\pgfqpoint{4.105158in}{0.788214in}}%
\pgfpathlineto{\pgfqpoint{4.107735in}{0.785902in}}%
\pgfpathlineto{\pgfqpoint{4.110312in}{0.787726in}}%
\pgfpathlineto{\pgfqpoint{4.112889in}{0.785905in}}%
\pgfpathlineto{\pgfqpoint{4.115467in}{0.782919in}}%
\pgfpathlineto{\pgfqpoint{4.123198in}{0.770610in}}%
\pgfpathlineto{\pgfqpoint{4.125775in}{0.773946in}}%
\pgfpathlineto{\pgfqpoint{4.128353in}{0.766096in}}%
\pgfpathlineto{\pgfqpoint{4.130930in}{0.774845in}}%
\pgfpathlineto{\pgfqpoint{4.133507in}{0.779963in}}%
\pgfpathlineto{\pgfqpoint{4.141239in}{0.793697in}}%
\pgfpathlineto{\pgfqpoint{4.143816in}{0.788703in}}%
\pgfpathlineto{\pgfqpoint{4.146393in}{0.794235in}}%
\pgfpathlineto{\pgfqpoint{4.148970in}{0.790418in}}%
\pgfpathlineto{\pgfqpoint{4.151548in}{0.788685in}}%
\pgfpathlineto{\pgfqpoint{4.161856in}{0.788339in}}%
\pgfpathlineto{\pgfqpoint{4.164434in}{0.818241in}}%
\pgfpathlineto{\pgfqpoint{4.167011in}{0.806439in}}%
\pgfpathlineto{\pgfqpoint{4.169588in}{0.817619in}}%
\pgfpathlineto{\pgfqpoint{4.177320in}{0.820938in}}%
\pgfpathlineto{\pgfqpoint{4.179897in}{0.820975in}}%
\pgfpathlineto{\pgfqpoint{4.182474in}{0.823396in}}%
\pgfpathlineto{\pgfqpoint{4.185051in}{0.833196in}}%
\pgfpathlineto{\pgfqpoint{4.195360in}{0.836901in}}%
\pgfpathlineto{\pgfqpoint{4.197937in}{0.843994in}}%
\pgfpathlineto{\pgfqpoint{4.200515in}{0.840151in}}%
\pgfpathlineto{\pgfqpoint{4.203092in}{0.834251in}}%
\pgfpathlineto{\pgfqpoint{4.205669in}{0.836726in}}%
\pgfpathlineto{\pgfqpoint{4.213401in}{0.840009in}}%
\pgfpathlineto{\pgfqpoint{4.215978in}{0.842610in}}%
\pgfpathlineto{\pgfqpoint{4.218555in}{0.841723in}}%
\pgfpathlineto{\pgfqpoint{4.221132in}{0.839880in}}%
\pgfpathlineto{\pgfqpoint{4.223709in}{0.835590in}}%
\pgfpathlineto{\pgfqpoint{4.231441in}{0.831323in}}%
\pgfpathlineto{\pgfqpoint{4.234018in}{0.834108in}}%
\pgfpathlineto{\pgfqpoint{4.236596in}{0.838554in}}%
\pgfpathlineto{\pgfqpoint{4.239173in}{0.840712in}}%
\pgfpathlineto{\pgfqpoint{4.241750in}{0.841417in}}%
\pgfpathlineto{\pgfqpoint{4.252059in}{0.845264in}}%
\pgfpathlineto{\pgfqpoint{4.254636in}{0.844815in}}%
\pgfpathlineto{\pgfqpoint{4.257213in}{0.845262in}}%
\pgfpathlineto{\pgfqpoint{4.259790in}{0.839661in}}%
\pgfpathlineto{\pgfqpoint{4.267522in}{0.844536in}}%
\pgfpathlineto{\pgfqpoint{4.270099in}{0.845050in}}%
\pgfpathlineto{\pgfqpoint{4.272677in}{0.836202in}}%
\pgfpathlineto{\pgfqpoint{4.275254in}{0.835280in}}%
\pgfpathlineto{\pgfqpoint{4.277831in}{0.832165in}}%
\pgfpathlineto{\pgfqpoint{4.285563in}{0.829214in}}%
\pgfpathlineto{\pgfqpoint{4.288140in}{0.826629in}}%
\pgfpathlineto{\pgfqpoint{4.290717in}{0.822713in}}%
\pgfpathlineto{\pgfqpoint{4.293294in}{0.830487in}}%
\pgfpathlineto{\pgfqpoint{4.295871in}{0.826698in}}%
\pgfpathlineto{\pgfqpoint{4.303603in}{0.829429in}}%
\pgfpathlineto{\pgfqpoint{4.308757in}{0.825876in}}%
\pgfpathlineto{\pgfqpoint{4.311335in}{0.821006in}}%
\pgfpathlineto{\pgfqpoint{4.313912in}{0.821096in}}%
\pgfpathlineto{\pgfqpoint{4.321644in}{0.817284in}}%
\pgfpathlineto{\pgfqpoint{4.324221in}{0.806547in}}%
\pgfpathlineto{\pgfqpoint{4.326798in}{0.801944in}}%
\pgfpathlineto{\pgfqpoint{4.329375in}{0.804099in}}%
\pgfpathlineto{\pgfqpoint{4.331952in}{0.801389in}}%
\pgfpathlineto{\pgfqpoint{4.339684in}{0.800340in}}%
\pgfpathlineto{\pgfqpoint{4.344838in}{0.790103in}}%
\pgfpathlineto{\pgfqpoint{4.347416in}{0.800496in}}%
\pgfpathlineto{\pgfqpoint{4.349993in}{0.801642in}}%
\pgfpathlineto{\pgfqpoint{4.357725in}{0.796927in}}%
\pgfpathlineto{\pgfqpoint{4.360302in}{0.820202in}}%
\pgfpathlineto{\pgfqpoint{4.365456in}{0.799338in}}%
\pgfpathlineto{\pgfqpoint{4.368033in}{0.808640in}}%
\pgfpathlineto{\pgfqpoint{4.375765in}{0.810373in}}%
\pgfpathlineto{\pgfqpoint{4.378342in}{0.812474in}}%
\pgfpathlineto{\pgfqpoint{4.383497in}{0.822685in}}%
\pgfpathlineto{\pgfqpoint{4.386074in}{0.820679in}}%
\pgfpathlineto{\pgfqpoint{4.393805in}{0.824842in}}%
\pgfpathlineto{\pgfqpoint{4.396383in}{0.836443in}}%
\pgfpathlineto{\pgfqpoint{4.398960in}{0.832523in}}%
\pgfpathlineto{\pgfqpoint{4.401537in}{0.825741in}}%
\pgfpathlineto{\pgfqpoint{4.404114in}{0.822261in}}%
\pgfpathlineto{\pgfqpoint{4.411846in}{0.816907in}}%
\pgfpathlineto{\pgfqpoint{4.417000in}{0.801796in}}%
\pgfpathlineto{\pgfqpoint{4.419578in}{0.808462in}}%
\pgfpathlineto{\pgfqpoint{4.422155in}{0.818570in}}%
\pgfpathlineto{\pgfqpoint{4.429886in}{0.826289in}}%
\pgfpathlineto{\pgfqpoint{4.432464in}{0.833091in}}%
\pgfpathlineto{\pgfqpoint{4.437618in}{0.830308in}}%
\pgfpathlineto{\pgfqpoint{4.440195in}{0.834726in}}%
\pgfpathlineto{\pgfqpoint{4.447927in}{0.835121in}}%
\pgfpathlineto{\pgfqpoint{4.450504in}{0.832188in}}%
\pgfpathlineto{\pgfqpoint{4.453081in}{0.840094in}}%
\pgfpathlineto{\pgfqpoint{4.458236in}{0.850895in}}%
\pgfpathlineto{\pgfqpoint{4.465967in}{0.844553in}}%
\pgfpathlineto{\pgfqpoint{4.468545in}{0.849419in}}%
\pgfpathlineto{\pgfqpoint{4.471122in}{0.846219in}}%
\pgfpathlineto{\pgfqpoint{4.473699in}{0.845838in}}%
\pgfpathlineto{\pgfqpoint{4.476276in}{0.847539in}}%
\pgfpathlineto{\pgfqpoint{4.486585in}{0.846834in}}%
\pgfpathlineto{\pgfqpoint{4.489162in}{0.850792in}}%
\pgfpathlineto{\pgfqpoint{4.491740in}{0.850214in}}%
\pgfpathlineto{\pgfqpoint{4.494317in}{0.853857in}}%
\pgfpathlineto{\pgfqpoint{4.502048in}{0.861175in}}%
\pgfpathlineto{\pgfqpoint{4.504626in}{0.893197in}}%
\pgfpathlineto{\pgfqpoint{4.507203in}{0.894825in}}%
\pgfpathlineto{\pgfqpoint{4.509780in}{0.890086in}}%
\pgfpathlineto{\pgfqpoint{4.512357in}{0.873284in}}%
\pgfpathlineto{\pgfqpoint{4.522666in}{0.867455in}}%
\pgfpathlineto{\pgfqpoint{4.525243in}{0.859959in}}%
\pgfpathlineto{\pgfqpoint{4.527821in}{0.860831in}}%
\pgfpathlineto{\pgfqpoint{4.538129in}{0.859539in}}%
\pgfpathlineto{\pgfqpoint{4.540707in}{0.864377in}}%
\pgfpathlineto{\pgfqpoint{4.543284in}{0.866850in}}%
\pgfpathlineto{\pgfqpoint{4.545861in}{0.864059in}}%
\pgfpathlineto{\pgfqpoint{4.556170in}{0.866580in}}%
\pgfpathlineto{\pgfqpoint{4.558747in}{0.855945in}}%
\pgfpathlineto{\pgfqpoint{4.561324in}{0.851486in}}%
\pgfpathlineto{\pgfqpoint{4.563902in}{0.850688in}}%
\pgfpathlineto{\pgfqpoint{4.566479in}{0.844431in}}%
\pgfpathlineto{\pgfqpoint{4.574210in}{0.841337in}}%
\pgfpathlineto{\pgfqpoint{4.576788in}{0.833020in}}%
\pgfpathlineto{\pgfqpoint{4.579365in}{0.843583in}}%
\pgfpathlineto{\pgfqpoint{4.581942in}{0.839638in}}%
\pgfpathlineto{\pgfqpoint{4.584519in}{0.862331in}}%
\pgfpathlineto{\pgfqpoint{4.597405in}{0.860227in}}%
\pgfpathlineto{\pgfqpoint{4.599982in}{0.867215in}}%
\pgfpathlineto{\pgfqpoint{4.602560in}{0.816458in}}%
\pgfpathlineto{\pgfqpoint{4.610291in}{0.820032in}}%
\pgfpathlineto{\pgfqpoint{4.612869in}{0.816877in}}%
\pgfpathlineto{\pgfqpoint{4.618023in}{0.789523in}}%
\pgfpathlineto{\pgfqpoint{4.620600in}{0.803983in}}%
\pgfpathlineto{\pgfqpoint{4.628332in}{0.815249in}}%
\pgfpathlineto{\pgfqpoint{4.630909in}{0.821427in}}%
\pgfpathlineto{\pgfqpoint{4.633486in}{0.805972in}}%
\pgfpathlineto{\pgfqpoint{4.636063in}{0.818101in}}%
\pgfpathlineto{\pgfqpoint{4.638641in}{0.819001in}}%
\pgfpathlineto{\pgfqpoint{4.646372in}{0.798967in}}%
\pgfpathlineto{\pgfqpoint{4.648950in}{0.803839in}}%
\pgfpathlineto{\pgfqpoint{4.651527in}{0.786686in}}%
\pgfpathlineto{\pgfqpoint{4.654104in}{0.813695in}}%
\pgfpathlineto{\pgfqpoint{4.656681in}{0.827105in}}%
\pgfpathlineto{\pgfqpoint{4.666990in}{0.824345in}}%
\pgfpathlineto{\pgfqpoint{4.669567in}{0.808242in}}%
\pgfpathlineto{\pgfqpoint{4.672144in}{0.819920in}}%
\pgfpathlineto{\pgfqpoint{4.674722in}{0.818806in}}%
\pgfpathlineto{\pgfqpoint{4.682453in}{0.820302in}}%
\pgfpathlineto{\pgfqpoint{4.685030in}{0.826025in}}%
\pgfpathlineto{\pgfqpoint{4.687608in}{0.834602in}}%
\pgfpathlineto{\pgfqpoint{4.690185in}{0.822335in}}%
\pgfpathlineto{\pgfqpoint{4.692762in}{0.824754in}}%
\pgfpathlineto{\pgfqpoint{4.700494in}{0.825329in}}%
\pgfpathlineto{\pgfqpoint{4.703071in}{0.806123in}}%
\pgfpathlineto{\pgfqpoint{4.705648in}{0.813939in}}%
\pgfpathlineto{\pgfqpoint{4.708225in}{0.819618in}}%
\pgfpathlineto{\pgfqpoint{4.710803in}{0.818926in}}%
\pgfpathlineto{\pgfqpoint{4.718534in}{0.828164in}}%
\pgfpathlineto{\pgfqpoint{4.721111in}{0.819929in}}%
\pgfpathlineto{\pgfqpoint{4.723689in}{0.823026in}}%
\pgfpathlineto{\pgfqpoint{4.726266in}{0.813475in}}%
\pgfpathlineto{\pgfqpoint{4.728843in}{0.810382in}}%
\pgfpathlineto{\pgfqpoint{4.736575in}{0.812534in}}%
\pgfpathlineto{\pgfqpoint{4.739152in}{0.820436in}}%
\pgfpathlineto{\pgfqpoint{4.741729in}{0.821328in}}%
\pgfpathlineto{\pgfqpoint{4.744306in}{0.820494in}}%
\pgfpathlineto{\pgfqpoint{4.746884in}{0.825576in}}%
\pgfpathlineto{\pgfqpoint{4.754615in}{0.828298in}}%
\pgfpathlineto{\pgfqpoint{4.757192in}{0.826502in}}%
\pgfpathlineto{\pgfqpoint{4.759770in}{0.829194in}}%
\pgfpathlineto{\pgfqpoint{4.762347in}{0.828766in}}%
\pgfpathlineto{\pgfqpoint{4.772656in}{0.825452in}}%
\pgfpathlineto{\pgfqpoint{4.775233in}{0.833062in}}%
\pgfpathlineto{\pgfqpoint{4.777810in}{0.831829in}}%
\pgfpathlineto{\pgfqpoint{4.782965in}{0.845837in}}%
\pgfpathlineto{\pgfqpoint{4.790696in}{0.841495in}}%
\pgfpathlineto{\pgfqpoint{4.793273in}{0.843830in}}%
\pgfpathlineto{\pgfqpoint{4.795851in}{0.847407in}}%
\pgfpathlineto{\pgfqpoint{4.798428in}{0.843080in}}%
\pgfpathlineto{\pgfqpoint{4.801005in}{0.848295in}}%
\pgfpathlineto{\pgfqpoint{4.808737in}{0.846830in}}%
\pgfpathlineto{\pgfqpoint{4.811314in}{0.840361in}}%
\pgfpathlineto{\pgfqpoint{4.813891in}{0.822362in}}%
\pgfpathlineto{\pgfqpoint{4.816468in}{0.825134in}}%
\pgfpathlineto{\pgfqpoint{4.819046in}{0.824268in}}%
\pgfpathlineto{\pgfqpoint{4.829354in}{0.825416in}}%
\pgfpathlineto{\pgfqpoint{4.831932in}{0.820220in}}%
\pgfpathlineto{\pgfqpoint{4.834509in}{0.800359in}}%
\pgfpathlineto{\pgfqpoint{4.837086in}{0.804073in}}%
\pgfpathlineto{\pgfqpoint{4.844818in}{0.791772in}}%
\pgfpathlineto{\pgfqpoint{4.847395in}{0.801472in}}%
\pgfpathlineto{\pgfqpoint{4.849972in}{0.806224in}}%
\pgfpathlineto{\pgfqpoint{4.852549in}{0.815786in}}%
\pgfpathlineto{\pgfqpoint{4.855127in}{0.806088in}}%
\pgfpathlineto{\pgfqpoint{4.862858in}{0.808111in}}%
\pgfpathlineto{\pgfqpoint{4.865435in}{0.804362in}}%
\pgfpathlineto{\pgfqpoint{4.868013in}{0.797372in}}%
\pgfpathlineto{\pgfqpoint{4.870590in}{0.794960in}}%
\pgfpathlineto{\pgfqpoint{4.873167in}{0.799737in}}%
\pgfpathlineto{\pgfqpoint{4.880899in}{0.803214in}}%
\pgfpathlineto{\pgfqpoint{4.883476in}{0.811044in}}%
\pgfpathlineto{\pgfqpoint{4.886053in}{0.812349in}}%
\pgfpathlineto{\pgfqpoint{4.891207in}{0.822978in}}%
\pgfpathlineto{\pgfqpoint{4.898939in}{0.823044in}}%
\pgfpathlineto{\pgfqpoint{4.901516in}{0.825536in}}%
\pgfpathlineto{\pgfqpoint{4.904094in}{0.834396in}}%
\pgfpathlineto{\pgfqpoint{4.909248in}{0.840752in}}%
\pgfpathlineto{\pgfqpoint{4.916980in}{0.845309in}}%
\pgfpathlineto{\pgfqpoint{4.919557in}{0.833207in}}%
\pgfpathlineto{\pgfqpoint{4.922134in}{0.834341in}}%
\pgfpathlineto{\pgfqpoint{4.924711in}{0.833357in}}%
\pgfpathlineto{\pgfqpoint{4.927288in}{0.829358in}}%
\pgfpathlineto{\pgfqpoint{4.937597in}{0.828168in}}%
\pgfpathlineto{\pgfqpoint{4.940175in}{0.832034in}}%
\pgfpathlineto{\pgfqpoint{4.945329in}{0.829675in}}%
\pgfpathlineto{\pgfqpoint{4.953061in}{0.833405in}}%
\pgfpathlineto{\pgfqpoint{4.955638in}{0.837515in}}%
\pgfpathlineto{\pgfqpoint{4.958215in}{0.839470in}}%
\pgfpathlineto{\pgfqpoint{4.960792in}{0.842887in}}%
\pgfpathlineto{\pgfqpoint{4.963369in}{0.829210in}}%
\pgfpathlineto{\pgfqpoint{4.971101in}{0.811551in}}%
\pgfpathlineto{\pgfqpoint{4.973678in}{0.793544in}}%
\pgfpathlineto{\pgfqpoint{4.978833in}{0.785706in}}%
\pgfpathlineto{\pgfqpoint{4.981410in}{0.784288in}}%
\pgfpathlineto{\pgfqpoint{4.989142in}{0.787448in}}%
\pgfpathlineto{\pgfqpoint{4.994296in}{0.791060in}}%
\pgfpathlineto{\pgfqpoint{4.999450in}{0.812477in}}%
\pgfpathlineto{\pgfqpoint{5.007182in}{0.784979in}}%
\pgfpathlineto{\pgfqpoint{5.009759in}{0.786281in}}%
\pgfpathlineto{\pgfqpoint{5.012336in}{0.785497in}}%
\pgfpathlineto{\pgfqpoint{5.014914in}{0.779482in}}%
\pgfpathlineto{\pgfqpoint{5.017491in}{0.782384in}}%
\pgfpathlineto{\pgfqpoint{5.027800in}{0.768852in}}%
\pgfpathlineto{\pgfqpoint{5.030377in}{0.774561in}}%
\pgfpathlineto{\pgfqpoint{5.032954in}{0.772374in}}%
\pgfpathlineto{\pgfqpoint{5.035531in}{0.766195in}}%
\pgfpathlineto{\pgfqpoint{5.043263in}{0.762024in}}%
\pgfpathlineto{\pgfqpoint{5.045840in}{0.751164in}}%
\pgfpathlineto{\pgfqpoint{5.048417in}{0.753699in}}%
\pgfpathlineto{\pgfqpoint{5.050995in}{0.758251in}}%
\pgfpathlineto{\pgfqpoint{5.053572in}{0.757048in}}%
\pgfpathlineto{\pgfqpoint{5.061304in}{0.757893in}}%
\pgfpathlineto{\pgfqpoint{5.063881in}{0.760758in}}%
\pgfpathlineto{\pgfqpoint{5.066458in}{0.752776in}}%
\pgfpathlineto{\pgfqpoint{5.069035in}{0.767452in}}%
\pgfpathlineto{\pgfqpoint{5.071612in}{0.775183in}}%
\pgfpathlineto{\pgfqpoint{5.079344in}{0.772759in}}%
\pgfpathlineto{\pgfqpoint{5.081921in}{0.766439in}}%
\pgfpathlineto{\pgfqpoint{5.084498in}{0.774818in}}%
\pgfpathlineto{\pgfqpoint{5.087076in}{0.779506in}}%
\pgfpathlineto{\pgfqpoint{5.089653in}{0.785518in}}%
\pgfpathlineto{\pgfqpoint{5.097384in}{0.787215in}}%
\pgfpathlineto{\pgfqpoint{5.099962in}{0.779594in}}%
\pgfpathlineto{\pgfqpoint{5.102539in}{0.784530in}}%
\pgfpathlineto{\pgfqpoint{5.105116in}{0.787859in}}%
\pgfpathlineto{\pgfqpoint{5.107693in}{0.784546in}}%
\pgfpathlineto{\pgfqpoint{5.115425in}{0.782127in}}%
\pgfpathlineto{\pgfqpoint{5.118002in}{0.785851in}}%
\pgfpathlineto{\pgfqpoint{5.120579in}{0.773823in}}%
\pgfpathlineto{\pgfqpoint{5.123157in}{0.778291in}}%
\pgfpathlineto{\pgfqpoint{5.133465in}{0.770269in}}%
\pgfpathlineto{\pgfqpoint{5.136043in}{0.755269in}}%
\pgfpathlineto{\pgfqpoint{5.138620in}{0.762555in}}%
\pgfpathlineto{\pgfqpoint{5.141197in}{0.764523in}}%
\pgfpathlineto{\pgfqpoint{5.143774in}{0.765223in}}%
\pgfpathlineto{\pgfqpoint{5.151506in}{0.762780in}}%
\pgfpathlineto{\pgfqpoint{5.154083in}{0.762787in}}%
\pgfpathlineto{\pgfqpoint{5.156660in}{0.763644in}}%
\pgfpathlineto{\pgfqpoint{5.159238in}{0.762997in}}%
\pgfpathlineto{\pgfqpoint{5.161815in}{0.766720in}}%
\pgfpathlineto{\pgfqpoint{5.169546in}{0.766333in}}%
\pgfpathlineto{\pgfqpoint{5.172124in}{0.771339in}}%
\pgfpathlineto{\pgfqpoint{5.174701in}{0.766239in}}%
\pgfpathlineto{\pgfqpoint{5.177278in}{0.771632in}}%
\pgfpathlineto{\pgfqpoint{5.192741in}{0.771154in}}%
\pgfpathlineto{\pgfqpoint{5.195319in}{0.772133in}}%
\pgfpathlineto{\pgfqpoint{5.197896in}{0.768210in}}%
\pgfpathlineto{\pgfqpoint{5.205627in}{0.770933in}}%
\pgfpathlineto{\pgfqpoint{5.208205in}{0.767121in}}%
\pgfpathlineto{\pgfqpoint{5.210782in}{0.764945in}}%
\pgfpathlineto{\pgfqpoint{5.213359in}{0.764885in}}%
\pgfpathlineto{\pgfqpoint{5.215936in}{0.750474in}}%
\pgfpathlineto{\pgfqpoint{5.223668in}{0.758373in}}%
\pgfpathlineto{\pgfqpoint{5.226245in}{0.758798in}}%
\pgfpathlineto{\pgfqpoint{5.228822in}{0.762120in}}%
\pgfpathlineto{\pgfqpoint{5.231400in}{0.764196in}}%
\pgfpathlineto{\pgfqpoint{5.233977in}{0.761051in}}%
\pgfpathlineto{\pgfqpoint{5.244286in}{0.768540in}}%
\pgfpathlineto{\pgfqpoint{5.246863in}{0.768249in}}%
\pgfpathlineto{\pgfqpoint{5.252017in}{0.779540in}}%
\pgfpathlineto{\pgfqpoint{5.259749in}{0.784697in}}%
\pgfpathlineto{\pgfqpoint{5.264903in}{0.774246in}}%
\pgfpathlineto{\pgfqpoint{5.267481in}{0.789266in}}%
\pgfpathlineto{\pgfqpoint{5.270058in}{0.783654in}}%
\pgfpathlineto{\pgfqpoint{5.277789in}{0.785670in}}%
\pgfpathlineto{\pgfqpoint{5.280367in}{0.793545in}}%
\pgfpathlineto{\pgfqpoint{5.282944in}{0.793348in}}%
\pgfpathlineto{\pgfqpoint{5.285521in}{0.789667in}}%
\pgfpathlineto{\pgfqpoint{5.288098in}{0.791836in}}%
\pgfpathlineto{\pgfqpoint{5.295830in}{0.796392in}}%
\pgfpathlineto{\pgfqpoint{5.298407in}{0.807471in}}%
\pgfpathlineto{\pgfqpoint{5.300984in}{0.832634in}}%
\pgfpathlineto{\pgfqpoint{5.303561in}{0.794616in}}%
\pgfpathlineto{\pgfqpoint{5.306139in}{0.785993in}}%
\pgfpathlineto{\pgfqpoint{5.313870in}{0.786796in}}%
\pgfpathlineto{\pgfqpoint{5.316448in}{0.814513in}}%
\pgfpathlineto{\pgfqpoint{5.319025in}{0.813314in}}%
\pgfpathlineto{\pgfqpoint{5.321602in}{0.820576in}}%
\pgfpathlineto{\pgfqpoint{5.324179in}{0.832608in}}%
\pgfpathlineto{\pgfqpoint{5.331911in}{0.831653in}}%
\pgfpathlineto{\pgfqpoint{5.334488in}{0.833546in}}%
\pgfpathlineto{\pgfqpoint{5.339642in}{0.830318in}}%
\pgfpathlineto{\pgfqpoint{5.342220in}{0.833524in}}%
\pgfpathlineto{\pgfqpoint{5.349951in}{0.833447in}}%
\pgfpathlineto{\pgfqpoint{5.352529in}{0.834233in}}%
\pgfpathlineto{\pgfqpoint{5.355106in}{0.853588in}}%
\pgfpathlineto{\pgfqpoint{5.357683in}{0.843453in}}%
\pgfpathlineto{\pgfqpoint{5.360260in}{0.840775in}}%
\pgfpathlineto{\pgfqpoint{5.367992in}{0.831803in}}%
\pgfpathlineto{\pgfqpoint{5.370569in}{0.832414in}}%
\pgfpathlineto{\pgfqpoint{5.373146in}{0.834501in}}%
\pgfpathlineto{\pgfqpoint{5.375723in}{0.830889in}}%
\pgfpathlineto{\pgfqpoint{5.378301in}{0.834146in}}%
\pgfpathlineto{\pgfqpoint{5.386032in}{0.836110in}}%
\pgfpathlineto{\pgfqpoint{5.388609in}{0.845323in}}%
\pgfpathlineto{\pgfqpoint{5.391187in}{0.841930in}}%
\pgfpathlineto{\pgfqpoint{5.396341in}{0.844017in}}%
\pgfpathlineto{\pgfqpoint{5.404073in}{0.842953in}}%
\pgfpathlineto{\pgfqpoint{5.406650in}{0.843328in}}%
\pgfpathlineto{\pgfqpoint{5.409227in}{0.848832in}}%
\pgfpathlineto{\pgfqpoint{5.411804in}{0.839047in}}%
\pgfpathlineto{\pgfqpoint{5.414382in}{0.849651in}}%
\pgfpathlineto{\pgfqpoint{5.422113in}{0.856791in}}%
\pgfpathlineto{\pgfqpoint{5.424690in}{0.855805in}}%
\pgfpathlineto{\pgfqpoint{5.427268in}{0.856187in}}%
\pgfpathlineto{\pgfqpoint{5.432422in}{0.865415in}}%
\pgfpathlineto{\pgfqpoint{5.440154in}{0.856196in}}%
\pgfpathlineto{\pgfqpoint{5.442731in}{0.859068in}}%
\pgfpathlineto{\pgfqpoint{5.445308in}{0.859427in}}%
\pgfpathlineto{\pgfqpoint{5.447885in}{0.853476in}}%
\pgfpathlineto{\pgfqpoint{5.450463in}{0.860947in}}%
\pgfpathlineto{\pgfqpoint{5.458194in}{0.858940in}}%
\pgfpathlineto{\pgfqpoint{5.463349in}{0.861535in}}%
\pgfpathlineto{\pgfqpoint{5.465926in}{0.864686in}}%
\pgfpathlineto{\pgfqpoint{5.468503in}{0.868859in}}%
\pgfpathlineto{\pgfqpoint{5.478812in}{0.870987in}}%
\pgfpathlineto{\pgfqpoint{5.481389in}{0.874704in}}%
\pgfpathlineto{\pgfqpoint{5.483966in}{0.873675in}}%
\pgfpathlineto{\pgfqpoint{5.486544in}{0.876814in}}%
\pgfpathlineto{\pgfqpoint{5.496852in}{0.879276in}}%
\pgfpathlineto{\pgfqpoint{5.499430in}{0.878477in}}%
\pgfpathlineto{\pgfqpoint{5.502007in}{0.887355in}}%
\pgfpathlineto{\pgfqpoint{5.504584in}{0.885618in}}%
\pgfpathlineto{\pgfqpoint{5.512316in}{0.885683in}}%
\pgfpathlineto{\pgfqpoint{5.514893in}{0.882023in}}%
\pgfpathlineto{\pgfqpoint{5.517470in}{0.886628in}}%
\pgfpathlineto{\pgfqpoint{5.520047in}{0.897782in}}%
\pgfpathlineto{\pgfqpoint{5.522625in}{0.900882in}}%
\pgfpathlineto{\pgfqpoint{5.532933in}{0.900089in}}%
\pgfpathlineto{\pgfqpoint{5.538088in}{0.906045in}}%
\pgfpathlineto{\pgfqpoint{5.540665in}{0.901820in}}%
\pgfpathlineto{\pgfqpoint{5.548397in}{0.894302in}}%
\pgfpathlineto{\pgfqpoint{5.550974in}{0.880234in}}%
\pgfpathlineto{\pgfqpoint{5.556128in}{0.883921in}}%
\pgfpathlineto{\pgfqpoint{5.558706in}{0.892418in}}%
\pgfpathlineto{\pgfqpoint{5.569014in}{0.882299in}}%
\pgfpathlineto{\pgfqpoint{5.571592in}{0.884584in}}%
\pgfpathlineto{\pgfqpoint{5.574169in}{0.888162in}}%
\pgfpathlineto{\pgfqpoint{5.576746in}{0.889450in}}%
\pgfpathlineto{\pgfqpoint{5.584478in}{0.889057in}}%
\pgfpathlineto{\pgfqpoint{5.587055in}{0.886425in}}%
\pgfpathlineto{\pgfqpoint{5.589632in}{0.886181in}}%
\pgfpathlineto{\pgfqpoint{5.592209in}{0.881103in}}%
\pgfpathlineto{\pgfqpoint{5.594786in}{0.880026in}}%
\pgfpathlineto{\pgfqpoint{5.602518in}{0.881031in}}%
\pgfpathlineto{\pgfqpoint{5.605095in}{0.878335in}}%
\pgfpathlineto{\pgfqpoint{5.607673in}{0.889951in}}%
\pgfpathlineto{\pgfqpoint{5.610250in}{0.897299in}}%
\pgfpathlineto{\pgfqpoint{5.612827in}{0.899222in}}%
\pgfpathlineto{\pgfqpoint{5.623136in}{0.896278in}}%
\pgfpathlineto{\pgfqpoint{5.625713in}{0.893118in}}%
\pgfpathlineto{\pgfqpoint{5.628290in}{0.885584in}}%
\pgfpathlineto{\pgfqpoint{5.630867in}{0.880590in}}%
\pgfpathlineto{\pgfqpoint{5.638599in}{0.883270in}}%
\pgfpathlineto{\pgfqpoint{5.641176in}{0.880325in}}%
\pgfpathlineto{\pgfqpoint{5.643754in}{0.871279in}}%
\pgfpathlineto{\pgfqpoint{5.646331in}{0.881321in}}%
\pgfpathlineto{\pgfqpoint{5.648908in}{0.879274in}}%
\pgfpathlineto{\pgfqpoint{5.656640in}{0.877670in}}%
\pgfpathlineto{\pgfqpoint{5.659217in}{0.881526in}}%
\pgfpathlineto{\pgfqpoint{5.661794in}{0.882288in}}%
\pgfpathlineto{\pgfqpoint{5.664371in}{0.877718in}}%
\pgfpathlineto{\pgfqpoint{5.666948in}{0.884991in}}%
\pgfpathlineto{\pgfqpoint{5.674680in}{0.890824in}}%
\pgfpathlineto{\pgfqpoint{5.677257in}{0.886270in}}%
\pgfpathlineto{\pgfqpoint{5.679835in}{0.895011in}}%
\pgfpathlineto{\pgfqpoint{5.682412in}{0.899669in}}%
\pgfpathlineto{\pgfqpoint{5.684989in}{0.909998in}}%
\pgfpathlineto{\pgfqpoint{5.692721in}{0.907226in}}%
\pgfpathlineto{\pgfqpoint{5.695298in}{0.903233in}}%
\pgfpathlineto{\pgfqpoint{5.697875in}{0.910732in}}%
\pgfpathlineto{\pgfqpoint{5.700452in}{0.908529in}}%
\pgfpathlineto{\pgfqpoint{5.703029in}{0.902804in}}%
\pgfpathlineto{\pgfqpoint{5.710761in}{0.910929in}}%
\pgfpathlineto{\pgfqpoint{5.715915in}{0.915251in}}%
\pgfpathlineto{\pgfqpoint{5.718493in}{0.913921in}}%
\pgfpathlineto{\pgfqpoint{5.721070in}{0.913551in}}%
\pgfpathlineto{\pgfqpoint{5.728802in}{0.916173in}}%
\pgfpathlineto{\pgfqpoint{5.731379in}{0.914807in}}%
\pgfpathlineto{\pgfqpoint{5.733956in}{0.917580in}}%
\pgfpathlineto{\pgfqpoint{5.736533in}{0.916064in}}%
\pgfpathlineto{\pgfqpoint{5.746842in}{0.917784in}}%
\pgfpathlineto{\pgfqpoint{5.749419in}{0.915318in}}%
\pgfpathlineto{\pgfqpoint{5.751996in}{0.908535in}}%
\pgfpathlineto{\pgfqpoint{5.754574in}{0.907664in}}%
\pgfpathlineto{\pgfqpoint{5.764883in}{0.908507in}}%
\pgfpathlineto{\pgfqpoint{5.767460in}{0.919925in}}%
\pgfpathlineto{\pgfqpoint{5.770037in}{0.917808in}}%
\pgfpathlineto{\pgfqpoint{5.775191in}{0.894177in}}%
\pgfpathlineto{\pgfqpoint{5.782923in}{0.882845in}}%
\pgfpathlineto{\pgfqpoint{5.785500in}{0.881788in}}%
\pgfpathlineto{\pgfqpoint{5.788077in}{0.893838in}}%
\pgfpathlineto{\pgfqpoint{5.790655in}{0.895101in}}%
\pgfpathlineto{\pgfqpoint{5.793232in}{0.901754in}}%
\pgfpathlineto{\pgfqpoint{5.800963in}{0.905646in}}%
\pgfpathlineto{\pgfqpoint{5.806118in}{0.882978in}}%
\pgfpathlineto{\pgfqpoint{5.808695in}{0.876664in}}%
\pgfpathlineto{\pgfqpoint{5.811272in}{0.882921in}}%
\pgfpathlineto{\pgfqpoint{5.819004in}{0.883969in}}%
\pgfpathlineto{\pgfqpoint{5.821581in}{0.886988in}}%
\pgfpathlineto{\pgfqpoint{5.826736in}{0.888066in}}%
\pgfpathlineto{\pgfqpoint{5.829313in}{0.891111in}}%
\pgfpathlineto{\pgfqpoint{5.837044in}{0.880940in}}%
\pgfpathlineto{\pgfqpoint{5.839622in}{0.874554in}}%
\pgfpathlineto{\pgfqpoint{5.842199in}{0.871285in}}%
\pgfpathlineto{\pgfqpoint{5.847353in}{0.875634in}}%
\pgfpathlineto{\pgfqpoint{5.855085in}{0.878678in}}%
\pgfpathlineto{\pgfqpoint{5.857662in}{0.876282in}}%
\pgfpathlineto{\pgfqpoint{5.860239in}{0.880735in}}%
\pgfpathlineto{\pgfqpoint{5.862817in}{0.875515in}}%
\pgfpathlineto{\pgfqpoint{5.865394in}{0.878957in}}%
\pgfpathlineto{\pgfqpoint{5.875703in}{0.876479in}}%
\pgfpathlineto{\pgfqpoint{5.878280in}{0.867772in}}%
\pgfpathlineto{\pgfqpoint{5.880857in}{0.877174in}}%
\pgfpathlineto{\pgfqpoint{5.883434in}{0.877849in}}%
\pgfpathlineto{\pgfqpoint{5.891166in}{0.874829in}}%
\pgfpathlineto{\pgfqpoint{5.893743in}{0.871851in}}%
\pgfpathlineto{\pgfqpoint{5.896320in}{0.874512in}}%
\pgfpathlineto{\pgfqpoint{5.898898in}{0.879130in}}%
\pgfpathlineto{\pgfqpoint{5.901475in}{0.892263in}}%
\pgfpathlineto{\pgfqpoint{5.909206in}{0.867755in}}%
\pgfpathlineto{\pgfqpoint{5.911784in}{0.883876in}}%
\pgfpathlineto{\pgfqpoint{5.914361in}{0.886636in}}%
\pgfpathlineto{\pgfqpoint{5.916938in}{0.875652in}}%
\pgfpathlineto{\pgfqpoint{5.919515in}{0.868932in}}%
\pgfpathlineto{\pgfqpoint{5.927247in}{0.876781in}}%
\pgfpathlineto{\pgfqpoint{5.929824in}{0.876301in}}%
\pgfpathlineto{\pgfqpoint{5.932401in}{0.867276in}}%
\pgfpathlineto{\pgfqpoint{5.934979in}{0.862858in}}%
\pgfpathlineto{\pgfqpoint{5.937556in}{0.859919in}}%
\pgfpathlineto{\pgfqpoint{5.945287in}{0.863574in}}%
\pgfpathlineto{\pgfqpoint{5.947865in}{0.857175in}}%
\pgfpathlineto{\pgfqpoint{5.950442in}{0.845434in}}%
\pgfpathlineto{\pgfqpoint{5.953019in}{0.856257in}}%
\pgfpathlineto{\pgfqpoint{5.955596in}{0.853183in}}%
\pgfpathlineto{\pgfqpoint{5.963328in}{0.852500in}}%
\pgfpathlineto{\pgfqpoint{5.968482in}{0.874109in}}%
\pgfpathlineto{\pgfqpoint{5.971060in}{0.874880in}}%
\pgfpathlineto{\pgfqpoint{5.973637in}{0.871894in}}%
\pgfpathlineto{\pgfqpoint{5.981368in}{0.868683in}}%
\pgfpathlineto{\pgfqpoint{5.983946in}{0.871270in}}%
\pgfpathlineto{\pgfqpoint{5.986523in}{0.870579in}}%
\pgfpathlineto{\pgfqpoint{5.989100in}{0.868384in}}%
\pgfpathlineto{\pgfqpoint{5.991677in}{0.871950in}}%
\pgfpathlineto{\pgfqpoint{5.999409in}{0.875222in}}%
\pgfpathlineto{\pgfqpoint{6.001986in}{0.880779in}}%
\pgfpathlineto{\pgfqpoint{6.004563in}{0.878976in}}%
\pgfpathlineto{\pgfqpoint{6.007140in}{0.887007in}}%
\pgfpathlineto{\pgfqpoint{6.009718in}{0.879849in}}%
\pgfpathlineto{\pgfqpoint{6.017449in}{0.863343in}}%
\pgfpathlineto{\pgfqpoint{6.020027in}{0.863826in}}%
\pgfpathlineto{\pgfqpoint{6.022604in}{0.855632in}}%
\pgfpathlineto{\pgfqpoint{6.025181in}{0.826654in}}%
\pgfpathlineto{\pgfqpoint{6.027758in}{0.828525in}}%
\pgfpathlineto{\pgfqpoint{6.035490in}{0.828582in}}%
\pgfpathlineto{\pgfqpoint{6.040644in}{0.819399in}}%
\pgfpathlineto{\pgfqpoint{6.043221in}{0.817540in}}%
\pgfpathlineto{\pgfqpoint{6.045799in}{0.817465in}}%
\pgfpathlineto{\pgfqpoint{6.053530in}{0.815068in}}%
\pgfpathlineto{\pgfqpoint{6.056108in}{0.816492in}}%
\pgfpathlineto{\pgfqpoint{6.058685in}{0.827339in}}%
\pgfpathlineto{\pgfqpoint{6.061262in}{0.826965in}}%
\pgfpathlineto{\pgfqpoint{6.063839in}{0.828295in}}%
\pgfpathlineto{\pgfqpoint{6.071571in}{0.830105in}}%
\pgfpathlineto{\pgfqpoint{6.074148in}{0.826063in}}%
\pgfpathlineto{\pgfqpoint{6.076725in}{0.827382in}}%
\pgfpathlineto{\pgfqpoint{6.079302in}{0.825578in}}%
\pgfpathlineto{\pgfqpoint{6.081880in}{0.822673in}}%
\pgfpathlineto{\pgfqpoint{6.089611in}{0.819870in}}%
\pgfpathlineto{\pgfqpoint{6.092188in}{0.813538in}}%
\pgfpathlineto{\pgfqpoint{6.094766in}{0.822556in}}%
\pgfpathlineto{\pgfqpoint{6.097343in}{0.828124in}}%
\pgfpathlineto{\pgfqpoint{6.099920in}{0.830367in}}%
\pgfpathlineto{\pgfqpoint{6.107652in}{0.831449in}}%
\pgfpathlineto{\pgfqpoint{6.110229in}{0.819183in}}%
\pgfpathlineto{\pgfqpoint{6.112806in}{0.814593in}}%
\pgfpathlineto{\pgfqpoint{6.115383in}{0.814842in}}%
\pgfpathlineto{\pgfqpoint{6.117961in}{0.810069in}}%
\pgfpathlineto{\pgfqpoint{6.128269in}{0.793958in}}%
\pgfpathlineto{\pgfqpoint{6.130847in}{0.791531in}}%
\pgfpathlineto{\pgfqpoint{6.133424in}{0.803273in}}%
\pgfpathlineto{\pgfqpoint{6.136001in}{0.798104in}}%
\pgfpathlineto{\pgfqpoint{6.146310in}{0.806088in}}%
\pgfpathlineto{\pgfqpoint{6.148887in}{0.801504in}}%
\pgfpathlineto{\pgfqpoint{6.151464in}{0.811678in}}%
\pgfpathlineto{\pgfqpoint{6.154042in}{0.817235in}}%
\pgfpathlineto{\pgfqpoint{6.161773in}{0.822302in}}%
\pgfpathlineto{\pgfqpoint{6.164350in}{0.824913in}}%
\pgfpathlineto{\pgfqpoint{6.166928in}{0.822421in}}%
\pgfpathlineto{\pgfqpoint{6.169505in}{0.812164in}}%
\pgfpathlineto{\pgfqpoint{6.172082in}{0.805804in}}%
\pgfpathlineto{\pgfqpoint{6.179814in}{0.807949in}}%
\pgfpathlineto{\pgfqpoint{6.182391in}{0.815677in}}%
\pgfpathlineto{\pgfqpoint{6.184968in}{0.813887in}}%
\pgfpathlineto{\pgfqpoint{6.187545in}{0.813726in}}%
\pgfpathlineto{\pgfqpoint{6.190123in}{0.812601in}}%
\pgfpathlineto{\pgfqpoint{6.197854in}{0.811827in}}%
\pgfpathlineto{\pgfqpoint{6.203009in}{0.812717in}}%
\pgfpathlineto{\pgfqpoint{6.205586in}{0.815117in}}%
\pgfpathlineto{\pgfqpoint{6.208163in}{0.812723in}}%
\pgfpathlineto{\pgfqpoint{6.215895in}{0.797692in}}%
\pgfpathlineto{\pgfqpoint{6.218472in}{0.797889in}}%
\pgfpathlineto{\pgfqpoint{6.221049in}{0.781778in}}%
\pgfpathlineto{\pgfqpoint{6.226204in}{0.770264in}}%
\pgfpathlineto{\pgfqpoint{6.233935in}{0.774187in}}%
\pgfpathlineto{\pgfqpoint{6.239090in}{0.770522in}}%
\pgfpathlineto{\pgfqpoint{6.241667in}{0.776124in}}%
\pgfpathlineto{\pgfqpoint{6.244244in}{0.757561in}}%
\pgfpathlineto{\pgfqpoint{6.251976in}{0.778688in}}%
\pgfpathlineto{\pgfqpoint{6.254553in}{0.774895in}}%
\pgfpathlineto{\pgfqpoint{6.257130in}{0.766605in}}%
\pgfpathlineto{\pgfqpoint{6.259707in}{0.772834in}}%
\pgfpathlineto{\pgfqpoint{6.262285in}{0.737811in}}%
\pgfpathlineto{\pgfqpoint{6.270016in}{0.739553in}}%
\pgfpathlineto{\pgfqpoint{6.272593in}{0.735679in}}%
\pgfpathlineto{\pgfqpoint{6.275171in}{0.723563in}}%
\pgfpathlineto{\pgfqpoint{6.277748in}{0.718714in}}%
\pgfpathlineto{\pgfqpoint{6.280325in}{0.720727in}}%
\pgfpathlineto{\pgfqpoint{6.288057in}{0.724495in}}%
\pgfpathlineto{\pgfqpoint{6.290634in}{0.720127in}}%
\pgfpathlineto{\pgfqpoint{6.293211in}{0.717805in}}%
\pgfpathlineto{\pgfqpoint{6.295788in}{0.722524in}}%
\pgfpathlineto{\pgfqpoint{6.298365in}{0.711652in}}%
\pgfpathlineto{\pgfqpoint{6.306097in}{0.739982in}}%
\pgfpathlineto{\pgfqpoint{6.308674in}{0.724613in}}%
\pgfpathlineto{\pgfqpoint{6.311252in}{0.743343in}}%
\pgfpathlineto{\pgfqpoint{6.313829in}{0.750412in}}%
\pgfpathlineto{\pgfqpoint{6.316406in}{0.745513in}}%
\pgfpathlineto{\pgfqpoint{6.324138in}{0.737409in}}%
\pgfpathlineto{\pgfqpoint{6.326715in}{0.731154in}}%
\pgfpathlineto{\pgfqpoint{6.329292in}{0.749268in}}%
\pgfpathlineto{\pgfqpoint{6.334446in}{0.751836in}}%
\pgfpathlineto{\pgfqpoint{6.342178in}{0.753055in}}%
\pgfpathlineto{\pgfqpoint{6.344755in}{0.749438in}}%
\pgfpathlineto{\pgfqpoint{6.347333in}{0.737306in}}%
\pgfpathlineto{\pgfqpoint{6.349910in}{0.748148in}}%
\pgfpathlineto{\pgfqpoint{6.352487in}{0.751534in}}%
\pgfpathlineto{\pgfqpoint{6.360219in}{0.741088in}}%
\pgfpathlineto{\pgfqpoint{6.362796in}{0.753680in}}%
\pgfpathlineto{\pgfqpoint{6.365373in}{0.739812in}}%
\pgfpathlineto{\pgfqpoint{6.370527in}{0.735796in}}%
\pgfpathlineto{\pgfqpoint{6.378259in}{0.742173in}}%
\pgfpathlineto{\pgfqpoint{6.380836in}{0.739484in}}%
\pgfpathlineto{\pgfqpoint{6.383414in}{0.748822in}}%
\pgfpathlineto{\pgfqpoint{6.385991in}{0.745017in}}%
\pgfpathlineto{\pgfqpoint{6.388568in}{0.743430in}}%
\pgfpathlineto{\pgfqpoint{6.396300in}{0.726517in}}%
\pgfpathlineto{\pgfqpoint{6.398877in}{0.718680in}}%
\pgfpathlineto{\pgfqpoint{6.401454in}{0.718588in}}%
\pgfpathlineto{\pgfqpoint{6.404031in}{0.723894in}}%
\pgfpathlineto{\pgfqpoint{6.406608in}{0.721129in}}%
\pgfpathlineto{\pgfqpoint{6.419494in}{0.716519in}}%
\pgfpathlineto{\pgfqpoint{6.422072in}{0.713614in}}%
\pgfpathlineto{\pgfqpoint{6.424649in}{0.717870in}}%
\pgfpathlineto{\pgfqpoint{6.424649in}{0.717870in}}%
\pgfusepath{stroke}%
\end{pgfscope}%
\begin{pgfscope}%
\pgfsetrectcap%
\pgfsetmiterjoin%
\pgfsetlinewidth{0.803000pt}%
\definecolor{currentstroke}{rgb}{1.000000,1.000000,1.000000}%
\pgfsetstrokecolor{currentstroke}%
\pgfsetdash{}{0pt}%
\pgfpathmoveto{\pgfqpoint{0.506467in}{0.331635in}}%
\pgfpathlineto{\pgfqpoint{0.506467in}{2.596635in}}%
\pgfusepath{stroke}%
\end{pgfscope}%
\begin{pgfscope}%
\pgfsetrectcap%
\pgfsetmiterjoin%
\pgfsetlinewidth{0.803000pt}%
\definecolor{currentstroke}{rgb}{1.000000,1.000000,1.000000}%
\pgfsetstrokecolor{currentstroke}%
\pgfsetdash{}{0pt}%
\pgfpathmoveto{\pgfqpoint{6.706467in}{0.331635in}}%
\pgfpathlineto{\pgfqpoint{6.706467in}{2.596635in}}%
\pgfusepath{stroke}%
\end{pgfscope}%
\begin{pgfscope}%
\pgfsetrectcap%
\pgfsetmiterjoin%
\pgfsetlinewidth{0.803000pt}%
\definecolor{currentstroke}{rgb}{1.000000,1.000000,1.000000}%
\pgfsetstrokecolor{currentstroke}%
\pgfsetdash{}{0pt}%
\pgfpathmoveto{\pgfqpoint{0.506467in}{0.331635in}}%
\pgfpathlineto{\pgfqpoint{6.706467in}{0.331635in}}%
\pgfusepath{stroke}%
\end{pgfscope}%
\begin{pgfscope}%
\pgfsetrectcap%
\pgfsetmiterjoin%
\pgfsetlinewidth{0.803000pt}%
\definecolor{currentstroke}{rgb}{1.000000,1.000000,1.000000}%
\pgfsetstrokecolor{currentstroke}%
\pgfsetdash{}{0pt}%
\pgfpathmoveto{\pgfqpoint{0.506467in}{2.596635in}}%
\pgfpathlineto{\pgfqpoint{6.706467in}{2.596635in}}%
\pgfusepath{stroke}%
\end{pgfscope}%
\begin{pgfscope}%
\pgfsetbuttcap%
\pgfsetmiterjoin%
\definecolor{currentfill}{rgb}{0.917647,0.917647,0.949020}%
\pgfsetfillcolor{currentfill}%
\pgfsetfillopacity{0.800000}%
\pgfsetlinewidth{1.003750pt}%
\definecolor{currentstroke}{rgb}{0.800000,0.800000,0.800000}%
\pgfsetstrokecolor{currentstroke}%
\pgfsetstrokeopacity{0.800000}%
\pgfsetdash{}{0pt}%
\pgfpathmoveto{\pgfqpoint{0.603689in}{2.075843in}}%
\pgfpathlineto{\pgfqpoint{1.653670in}{2.075843in}}%
\pgfpathquadraticcurveto{\pgfqpoint{1.681448in}{2.075843in}}{\pgfqpoint{1.681448in}{2.103621in}}%
\pgfpathlineto{\pgfqpoint{1.681448in}{2.499413in}}%
\pgfpathquadraticcurveto{\pgfqpoint{1.681448in}{2.527191in}}{\pgfqpoint{1.653670in}{2.527191in}}%
\pgfpathlineto{\pgfqpoint{0.603689in}{2.527191in}}%
\pgfpathquadraticcurveto{\pgfqpoint{0.575912in}{2.527191in}}{\pgfqpoint{0.575912in}{2.499413in}}%
\pgfpathlineto{\pgfqpoint{0.575912in}{2.103621in}}%
\pgfpathquadraticcurveto{\pgfqpoint{0.575912in}{2.075843in}}{\pgfqpoint{0.603689in}{2.075843in}}%
\pgfpathclose%
\pgfusepath{stroke,fill}%
\end{pgfscope}%
\begin{pgfscope}%
\pgfsetroundcap%
\pgfsetroundjoin%
\pgfsetlinewidth{1.505625pt}%
\definecolor{currentstroke}{rgb}{0.121569,0.466667,0.705882}%
\pgfsetstrokecolor{currentstroke}%
\pgfsetdash{}{0pt}%
\pgfpathmoveto{\pgfqpoint{0.631467in}{2.414723in}}%
\pgfpathlineto{\pgfqpoint{0.909245in}{2.414723in}}%
\pgfusepath{stroke}%
\end{pgfscope}%
\begin{pgfscope}%
\definecolor{textcolor}{rgb}{0.150000,0.150000,0.150000}%
\pgfsetstrokecolor{textcolor}%
\pgfsetfillcolor{textcolor}%
\pgftext[x=1.020356in,y=2.366112in,left,base]{\color{textcolor}\rmfamily\fontsize{10.000000}{12.000000}\selectfont Baseline}%
\end{pgfscope}%
\begin{pgfscope}%
\pgfsetroundcap%
\pgfsetroundjoin%
\pgfsetlinewidth{1.505625pt}%
\definecolor{currentstroke}{rgb}{1.000000,0.498039,0.054902}%
\pgfsetstrokecolor{currentstroke}%
\pgfsetdash{}{0pt}%
\pgfpathmoveto{\pgfqpoint{0.631467in}{2.210866in}}%
\pgfpathlineto{\pgfqpoint{0.909245in}{2.210866in}}%
\pgfusepath{stroke}%
\end{pgfscope}%
\begin{pgfscope}%
\definecolor{textcolor}{rgb}{0.150000,0.150000,0.150000}%
\pgfsetstrokecolor{textcolor}%
\pgfsetfillcolor{textcolor}%
\pgftext[x=1.020356in,y=2.162255in,left,base]{\color{textcolor}\rmfamily\fontsize{10.000000}{12.000000}\selectfont Strategy}%
\end{pgfscope}%
\end{pgfpicture}%
\makeatother%
\endgroup%

    \end{adjustbox}  
    \caption{Caption}
    \label{fig:mean reversion}
\end{figure}{}

Implementation algorithm: rank portfolio everyday along their returns from the last day. Then buy the two worst performers and sell the two best performers. Interestingly, reversing that does not produce better results. The time frame is obviously not right. 

\subsection{Mean Reversion - single stocks}

\subsubsection{Cumulative Mean - Stock prices}
Idea: 
Plot: Cumulative Mean vs. Actual Time Series
--> We see that the cumulative mean does not capture the time series well. Trend is always behind the current development, since we have a trend

\begin{figure}
    \centering
    \begin{minipage}[b]{0.49\textwidth}
        \centering
            \begin{adjustbox}{width=\textwidth,center}
                %% Creator: Matplotlib, PGF backend
%%
%% To include the figure in your LaTeX document, write
%%   \input{<filename>.pgf}
%%
%% Make sure the required packages are loaded in your preamble
%%   \usepackage{pgf}
%%
%% Figures using additional raster images can only be included by \input if
%% they are in the same directory as the main LaTeX file. For loading figures
%% from other directories you can use the `import` package
%%   \usepackage{import}
%% and then include the figures with
%%   \import{<path to file>}{<filename>.pgf}
%%
%% Matplotlib used the following preamble
%%   \usepackage{fontspec}
%%   \setmainfont{DejaVuSerif.ttf}[Path=/opt/tljh/user/lib/python3.6/site-packages/matplotlib/mpl-data/fonts/ttf/]
%%   \setsansfont{DejaVuSans.ttf}[Path=/opt/tljh/user/lib/python3.6/site-packages/matplotlib/mpl-data/fonts/ttf/]
%%   \setmonofont{DejaVuSansMono.ttf}[Path=/opt/tljh/user/lib/python3.6/site-packages/matplotlib/mpl-data/fonts/ttf/]
%%
\begingroup%
\makeatletter%
\begin{pgfpicture}%
\pgfpathrectangle{\pgfpointorigin}{\pgfqpoint{6.819881in}{9.701596in}}%
\pgfusepath{use as bounding box, clip}%
\begin{pgfscope}%
\pgfsetbuttcap%
\pgfsetmiterjoin%
\definecolor{currentfill}{rgb}{1.000000,1.000000,1.000000}%
\pgfsetfillcolor{currentfill}%
\pgfsetlinewidth{0.000000pt}%
\definecolor{currentstroke}{rgb}{1.000000,1.000000,1.000000}%
\pgfsetstrokecolor{currentstroke}%
\pgfsetdash{}{0pt}%
\pgfpathmoveto{\pgfqpoint{0.000000in}{0.000000in}}%
\pgfpathlineto{\pgfqpoint{6.819881in}{0.000000in}}%
\pgfpathlineto{\pgfqpoint{6.819881in}{9.701596in}}%
\pgfpathlineto{\pgfqpoint{0.000000in}{9.701596in}}%
\pgfpathclose%
\pgfusepath{fill}%
\end{pgfscope}%
\begin{pgfscope}%
\pgfsetbuttcap%
\pgfsetmiterjoin%
\definecolor{currentfill}{rgb}{0.917647,0.917647,0.949020}%
\pgfsetfillcolor{currentfill}%
\pgfsetlinewidth{0.000000pt}%
\definecolor{currentstroke}{rgb}{0.000000,0.000000,0.000000}%
\pgfsetstrokecolor{currentstroke}%
\pgfsetstrokeopacity{0.000000}%
\pgfsetdash{}{0pt}%
\pgfpathmoveto{\pgfqpoint{0.462318in}{8.286757in}}%
\pgfpathlineto{\pgfqpoint{3.157970in}{8.286757in}}%
\pgfpathlineto{\pgfqpoint{3.157970in}{9.391635in}}%
\pgfpathlineto{\pgfqpoint{0.462318in}{9.391635in}}%
\pgfpathclose%
\pgfusepath{fill}%
\end{pgfscope}%
\begin{pgfscope}%
\pgfpathrectangle{\pgfqpoint{0.462318in}{8.286757in}}{\pgfqpoint{2.695652in}{1.104878in}}%
\pgfusepath{clip}%
\pgfsetroundcap%
\pgfsetroundjoin%
\pgfsetlinewidth{0.803000pt}%
\definecolor{currentstroke}{rgb}{1.000000,1.000000,1.000000}%
\pgfsetstrokecolor{currentstroke}%
\pgfsetdash{}{0pt}%
\pgfpathmoveto{\pgfqpoint{0.582607in}{8.286757in}}%
\pgfpathlineto{\pgfqpoint{0.582607in}{9.391635in}}%
\pgfusepath{stroke}%
\end{pgfscope}%
\begin{pgfscope}%
\definecolor{textcolor}{rgb}{0.150000,0.150000,0.150000}%
\pgfsetstrokecolor{textcolor}%
\pgfsetfillcolor{textcolor}%
\pgftext[x=0.582607in,y=8.189535in,,top]{\color{textcolor}\rmfamily\fontsize{10.000000}{12.000000}\selectfont 2012}%
\end{pgfscope}%
\begin{pgfscope}%
\pgfpathrectangle{\pgfqpoint{0.462318in}{8.286757in}}{\pgfqpoint{2.695652in}{1.104878in}}%
\pgfusepath{clip}%
\pgfsetroundcap%
\pgfsetroundjoin%
\pgfsetlinewidth{0.803000pt}%
\definecolor{currentstroke}{rgb}{1.000000,1.000000,1.000000}%
\pgfsetstrokecolor{currentstroke}%
\pgfsetdash{}{0pt}%
\pgfpathmoveto{\pgfqpoint{0.992720in}{8.286757in}}%
\pgfpathlineto{\pgfqpoint{0.992720in}{9.391635in}}%
\pgfusepath{stroke}%
\end{pgfscope}%
\begin{pgfscope}%
\definecolor{textcolor}{rgb}{0.150000,0.150000,0.150000}%
\pgfsetstrokecolor{textcolor}%
\pgfsetfillcolor{textcolor}%
\pgftext[x=0.992720in,y=8.189535in,,top]{\color{textcolor}\rmfamily\fontsize{10.000000}{12.000000}\selectfont 2013}%
\end{pgfscope}%
\begin{pgfscope}%
\pgfpathrectangle{\pgfqpoint{0.462318in}{8.286757in}}{\pgfqpoint{2.695652in}{1.104878in}}%
\pgfusepath{clip}%
\pgfsetroundcap%
\pgfsetroundjoin%
\pgfsetlinewidth{0.803000pt}%
\definecolor{currentstroke}{rgb}{1.000000,1.000000,1.000000}%
\pgfsetstrokecolor{currentstroke}%
\pgfsetdash{}{0pt}%
\pgfpathmoveto{\pgfqpoint{1.401712in}{8.286757in}}%
\pgfpathlineto{\pgfqpoint{1.401712in}{9.391635in}}%
\pgfusepath{stroke}%
\end{pgfscope}%
\begin{pgfscope}%
\definecolor{textcolor}{rgb}{0.150000,0.150000,0.150000}%
\pgfsetstrokecolor{textcolor}%
\pgfsetfillcolor{textcolor}%
\pgftext[x=1.401712in,y=8.189535in,,top]{\color{textcolor}\rmfamily\fontsize{10.000000}{12.000000}\selectfont 2014}%
\end{pgfscope}%
\begin{pgfscope}%
\pgfpathrectangle{\pgfqpoint{0.462318in}{8.286757in}}{\pgfqpoint{2.695652in}{1.104878in}}%
\pgfusepath{clip}%
\pgfsetroundcap%
\pgfsetroundjoin%
\pgfsetlinewidth{0.803000pt}%
\definecolor{currentstroke}{rgb}{1.000000,1.000000,1.000000}%
\pgfsetstrokecolor{currentstroke}%
\pgfsetdash{}{0pt}%
\pgfpathmoveto{\pgfqpoint{1.810705in}{8.286757in}}%
\pgfpathlineto{\pgfqpoint{1.810705in}{9.391635in}}%
\pgfusepath{stroke}%
\end{pgfscope}%
\begin{pgfscope}%
\definecolor{textcolor}{rgb}{0.150000,0.150000,0.150000}%
\pgfsetstrokecolor{textcolor}%
\pgfsetfillcolor{textcolor}%
\pgftext[x=1.810705in,y=8.189535in,,top]{\color{textcolor}\rmfamily\fontsize{10.000000}{12.000000}\selectfont 2015}%
\end{pgfscope}%
\begin{pgfscope}%
\pgfpathrectangle{\pgfqpoint{0.462318in}{8.286757in}}{\pgfqpoint{2.695652in}{1.104878in}}%
\pgfusepath{clip}%
\pgfsetroundcap%
\pgfsetroundjoin%
\pgfsetlinewidth{0.803000pt}%
\definecolor{currentstroke}{rgb}{1.000000,1.000000,1.000000}%
\pgfsetstrokecolor{currentstroke}%
\pgfsetdash{}{0pt}%
\pgfpathmoveto{\pgfqpoint{2.219697in}{8.286757in}}%
\pgfpathlineto{\pgfqpoint{2.219697in}{9.391635in}}%
\pgfusepath{stroke}%
\end{pgfscope}%
\begin{pgfscope}%
\definecolor{textcolor}{rgb}{0.150000,0.150000,0.150000}%
\pgfsetstrokecolor{textcolor}%
\pgfsetfillcolor{textcolor}%
\pgftext[x=2.219697in,y=8.189535in,,top]{\color{textcolor}\rmfamily\fontsize{10.000000}{12.000000}\selectfont 2016}%
\end{pgfscope}%
\begin{pgfscope}%
\pgfpathrectangle{\pgfqpoint{0.462318in}{8.286757in}}{\pgfqpoint{2.695652in}{1.104878in}}%
\pgfusepath{clip}%
\pgfsetroundcap%
\pgfsetroundjoin%
\pgfsetlinewidth{0.803000pt}%
\definecolor{currentstroke}{rgb}{1.000000,1.000000,1.000000}%
\pgfsetstrokecolor{currentstroke}%
\pgfsetdash{}{0pt}%
\pgfpathmoveto{\pgfqpoint{2.629810in}{8.286757in}}%
\pgfpathlineto{\pgfqpoint{2.629810in}{9.391635in}}%
\pgfusepath{stroke}%
\end{pgfscope}%
\begin{pgfscope}%
\definecolor{textcolor}{rgb}{0.150000,0.150000,0.150000}%
\pgfsetstrokecolor{textcolor}%
\pgfsetfillcolor{textcolor}%
\pgftext[x=2.629810in,y=8.189535in,,top]{\color{textcolor}\rmfamily\fontsize{10.000000}{12.000000}\selectfont 2017}%
\end{pgfscope}%
\begin{pgfscope}%
\pgfpathrectangle{\pgfqpoint{0.462318in}{8.286757in}}{\pgfqpoint{2.695652in}{1.104878in}}%
\pgfusepath{clip}%
\pgfsetroundcap%
\pgfsetroundjoin%
\pgfsetlinewidth{0.803000pt}%
\definecolor{currentstroke}{rgb}{1.000000,1.000000,1.000000}%
\pgfsetstrokecolor{currentstroke}%
\pgfsetdash{}{0pt}%
\pgfpathmoveto{\pgfqpoint{3.038802in}{8.286757in}}%
\pgfpathlineto{\pgfqpoint{3.038802in}{9.391635in}}%
\pgfusepath{stroke}%
\end{pgfscope}%
\begin{pgfscope}%
\definecolor{textcolor}{rgb}{0.150000,0.150000,0.150000}%
\pgfsetstrokecolor{textcolor}%
\pgfsetfillcolor{textcolor}%
\pgftext[x=3.038802in,y=8.189535in,,top]{\color{textcolor}\rmfamily\fontsize{10.000000}{12.000000}\selectfont 2018}%
\end{pgfscope}%
\begin{pgfscope}%
\pgfpathrectangle{\pgfqpoint{0.462318in}{8.286757in}}{\pgfqpoint{2.695652in}{1.104878in}}%
\pgfusepath{clip}%
\pgfsetroundcap%
\pgfsetroundjoin%
\pgfsetlinewidth{0.803000pt}%
\definecolor{currentstroke}{rgb}{1.000000,1.000000,1.000000}%
\pgfsetstrokecolor{currentstroke}%
\pgfsetdash{}{0pt}%
\pgfpathmoveto{\pgfqpoint{0.462318in}{8.532436in}}%
\pgfpathlineto{\pgfqpoint{3.157970in}{8.532436in}}%
\pgfusepath{stroke}%
\end{pgfscope}%
\begin{pgfscope}%
\definecolor{textcolor}{rgb}{0.150000,0.150000,0.150000}%
\pgfsetstrokecolor{textcolor}%
\pgfsetfillcolor{textcolor}%
\pgftext[x=0.100000in,y=8.479675in,left,base]{\color{textcolor}\rmfamily\fontsize{10.000000}{12.000000}\selectfont 100}%
\end{pgfscope}%
\begin{pgfscope}%
\pgfpathrectangle{\pgfqpoint{0.462318in}{8.286757in}}{\pgfqpoint{2.695652in}{1.104878in}}%
\pgfusepath{clip}%
\pgfsetroundcap%
\pgfsetroundjoin%
\pgfsetlinewidth{0.803000pt}%
\definecolor{currentstroke}{rgb}{1.000000,1.000000,1.000000}%
\pgfsetstrokecolor{currentstroke}%
\pgfsetdash{}{0pt}%
\pgfpathmoveto{\pgfqpoint{0.462318in}{9.149411in}}%
\pgfpathlineto{\pgfqpoint{3.157970in}{9.149411in}}%
\pgfusepath{stroke}%
\end{pgfscope}%
\begin{pgfscope}%
\definecolor{textcolor}{rgb}{0.150000,0.150000,0.150000}%
\pgfsetstrokecolor{textcolor}%
\pgfsetfillcolor{textcolor}%
\pgftext[x=0.100000in,y=9.096649in,left,base]{\color{textcolor}\rmfamily\fontsize{10.000000}{12.000000}\selectfont 200}%
\end{pgfscope}%
\begin{pgfscope}%
\pgfpathrectangle{\pgfqpoint{0.462318in}{8.286757in}}{\pgfqpoint{2.695652in}{1.104878in}}%
\pgfusepath{clip}%
\pgfsetroundcap%
\pgfsetroundjoin%
\pgfsetlinewidth{1.505625pt}%
\definecolor{currentstroke}{rgb}{0.121569,0.466667,0.705882}%
\pgfsetstrokecolor{currentstroke}%
\pgfsetdash{}{0pt}%
\pgfpathmoveto{\pgfqpoint{0.584848in}{8.337534in}}%
\pgfpathlineto{\pgfqpoint{0.585968in}{8.341051in}}%
\pgfpathlineto{\pgfqpoint{0.588209in}{8.336979in}}%
\pgfpathlineto{\pgfqpoint{0.591571in}{8.339508in}}%
\pgfpathlineto{\pgfqpoint{0.592692in}{8.341668in}}%
\pgfpathlineto{\pgfqpoint{0.593812in}{8.338953in}}%
\pgfpathlineto{\pgfqpoint{0.594933in}{8.341544in}}%
\pgfpathlineto{\pgfqpoint{0.596053in}{8.338089in}}%
\pgfpathlineto{\pgfqpoint{0.600535in}{8.341298in}}%
\pgfpathlineto{\pgfqpoint{0.602776in}{8.349256in}}%
\pgfpathlineto{\pgfqpoint{0.603897in}{8.348454in}}%
\pgfpathlineto{\pgfqpoint{0.607258in}{8.348269in}}%
\pgfpathlineto{\pgfqpoint{0.608379in}{8.349873in}}%
\pgfpathlineto{\pgfqpoint{0.609499in}{8.352650in}}%
\pgfpathlineto{\pgfqpoint{0.610620in}{8.358264in}}%
\pgfpathlineto{\pgfqpoint{0.611740in}{8.357647in}}%
\pgfpathlineto{\pgfqpoint{0.615102in}{8.357030in}}%
\pgfpathlineto{\pgfqpoint{0.616223in}{8.353822in}}%
\pgfpathlineto{\pgfqpoint{0.617343in}{8.357092in}}%
\pgfpathlineto{\pgfqpoint{0.618464in}{8.357462in}}%
\pgfpathlineto{\pgfqpoint{0.619584in}{8.359005in}}%
\pgfpathlineto{\pgfqpoint{0.622946in}{8.358141in}}%
\pgfpathlineto{\pgfqpoint{0.624066in}{8.359807in}}%
\pgfpathlineto{\pgfqpoint{0.626307in}{8.360485in}}%
\pgfpathlineto{\pgfqpoint{0.627428in}{8.356043in}}%
\pgfpathlineto{\pgfqpoint{0.630789in}{8.360485in}}%
\pgfpathlineto{\pgfqpoint{0.631910in}{8.360300in}}%
\pgfpathlineto{\pgfqpoint{0.633031in}{8.358326in}}%
\pgfpathlineto{\pgfqpoint{0.634151in}{8.361596in}}%
\pgfpathlineto{\pgfqpoint{0.635272in}{8.361102in}}%
\pgfpathlineto{\pgfqpoint{0.639754in}{8.361349in}}%
\pgfpathlineto{\pgfqpoint{0.643115in}{8.364372in}}%
\pgfpathlineto{\pgfqpoint{0.646477in}{8.363694in}}%
\pgfpathlineto{\pgfqpoint{0.648718in}{8.361349in}}%
\pgfpathlineto{\pgfqpoint{0.649838in}{8.360794in}}%
\pgfpathlineto{\pgfqpoint{0.650959in}{8.360917in}}%
\pgfpathlineto{\pgfqpoint{0.654321in}{8.358573in}}%
\pgfpathlineto{\pgfqpoint{0.655441in}{8.347837in}}%
\pgfpathlineto{\pgfqpoint{0.656562in}{8.350490in}}%
\pgfpathlineto{\pgfqpoint{0.657682in}{8.356784in}}%
\pgfpathlineto{\pgfqpoint{0.658803in}{8.357277in}}%
\pgfpathlineto{\pgfqpoint{0.662164in}{8.361041in}}%
\pgfpathlineto{\pgfqpoint{0.663285in}{8.367210in}}%
\pgfpathlineto{\pgfqpoint{0.664405in}{8.367766in}}%
\pgfpathlineto{\pgfqpoint{0.665526in}{8.373565in}}%
\pgfpathlineto{\pgfqpoint{0.666646in}{8.371282in}}%
\pgfpathlineto{\pgfqpoint{0.670008in}{8.372208in}}%
\pgfpathlineto{\pgfqpoint{0.673369in}{8.366285in}}%
\pgfpathlineto{\pgfqpoint{0.674490in}{8.365730in}}%
\pgfpathlineto{\pgfqpoint{0.677852in}{8.369061in}}%
\pgfpathlineto{\pgfqpoint{0.678972in}{8.369185in}}%
\pgfpathlineto{\pgfqpoint{0.680093in}{8.365668in}}%
\pgfpathlineto{\pgfqpoint{0.682334in}{8.369493in}}%
\pgfpathlineto{\pgfqpoint{0.685695in}{8.369617in}}%
\pgfpathlineto{\pgfqpoint{0.686816in}{8.367396in}}%
\pgfpathlineto{\pgfqpoint{0.689057in}{8.359622in}}%
\pgfpathlineto{\pgfqpoint{0.693539in}{8.354871in}}%
\pgfpathlineto{\pgfqpoint{0.694659in}{8.345987in}}%
\pgfpathlineto{\pgfqpoint{0.695780in}{8.349873in}}%
\pgfpathlineto{\pgfqpoint{0.696901in}{8.357586in}}%
\pgfpathlineto{\pgfqpoint{0.698021in}{8.351601in}}%
\pgfpathlineto{\pgfqpoint{0.701383in}{8.355241in}}%
\pgfpathlineto{\pgfqpoint{0.702503in}{8.360547in}}%
\pgfpathlineto{\pgfqpoint{0.704744in}{8.357277in}}%
\pgfpathlineto{\pgfqpoint{0.705865in}{8.360732in}}%
\pgfpathlineto{\pgfqpoint{0.709226in}{8.358943in}}%
\pgfpathlineto{\pgfqpoint{0.710347in}{8.365853in}}%
\pgfpathlineto{\pgfqpoint{0.711467in}{8.367457in}}%
\pgfpathlineto{\pgfqpoint{0.712588in}{8.370110in}}%
\pgfpathlineto{\pgfqpoint{0.719311in}{8.371097in}}%
\pgfpathlineto{\pgfqpoint{0.720432in}{8.370419in}}%
\pgfpathlineto{\pgfqpoint{0.721552in}{8.366779in}}%
\pgfpathlineto{\pgfqpoint{0.724914in}{8.363447in}}%
\pgfpathlineto{\pgfqpoint{0.727155in}{8.359622in}}%
\pgfpathlineto{\pgfqpoint{0.728275in}{8.358943in}}%
\pgfpathlineto{\pgfqpoint{0.729396in}{8.356907in}}%
\pgfpathlineto{\pgfqpoint{0.732757in}{8.352341in}}%
\pgfpathlineto{\pgfqpoint{0.733878in}{8.352095in}}%
\pgfpathlineto{\pgfqpoint{0.734998in}{8.352897in}}%
\pgfpathlineto{\pgfqpoint{0.737240in}{8.343457in}}%
\pgfpathlineto{\pgfqpoint{0.740601in}{8.348331in}}%
\pgfpathlineto{\pgfqpoint{0.741722in}{8.346603in}}%
\pgfpathlineto{\pgfqpoint{0.742842in}{8.350059in}}%
\pgfpathlineto{\pgfqpoint{0.743963in}{8.350984in}}%
\pgfpathlineto{\pgfqpoint{0.745083in}{8.349935in}}%
\pgfpathlineto{\pgfqpoint{0.749565in}{8.354809in}}%
\pgfpathlineto{\pgfqpoint{0.750686in}{8.348269in}}%
\pgfpathlineto{\pgfqpoint{0.751806in}{8.348084in}}%
\pgfpathlineto{\pgfqpoint{0.752927in}{8.340064in}}%
\pgfpathlineto{\pgfqpoint{0.757409in}{8.338336in}}%
\pgfpathlineto{\pgfqpoint{0.758530in}{8.349195in}}%
\pgfpathlineto{\pgfqpoint{0.760771in}{8.356228in}}%
\pgfpathlineto{\pgfqpoint{0.764132in}{8.352033in}}%
\pgfpathlineto{\pgfqpoint{0.765253in}{8.359930in}}%
\pgfpathlineto{\pgfqpoint{0.766373in}{8.356907in}}%
\pgfpathlineto{\pgfqpoint{0.768614in}{8.363570in}}%
\pgfpathlineto{\pgfqpoint{0.771976in}{8.362953in}}%
\pgfpathlineto{\pgfqpoint{0.773096in}{8.365545in}}%
\pgfpathlineto{\pgfqpoint{0.774217in}{8.364126in}}%
\pgfpathlineto{\pgfqpoint{0.775337in}{8.359930in}}%
\pgfpathlineto{\pgfqpoint{0.776458in}{8.360485in}}%
\pgfpathlineto{\pgfqpoint{0.779820in}{8.355426in}}%
\pgfpathlineto{\pgfqpoint{0.780940in}{8.357154in}}%
\pgfpathlineto{\pgfqpoint{0.782061in}{8.362151in}}%
\pgfpathlineto{\pgfqpoint{0.783181in}{8.362151in}}%
\pgfpathlineto{\pgfqpoint{0.784302in}{8.374676in}}%
\pgfpathlineto{\pgfqpoint{0.787663in}{8.373010in}}%
\pgfpathlineto{\pgfqpoint{0.788784in}{8.375169in}}%
\pgfpathlineto{\pgfqpoint{0.791025in}{8.374491in}}%
\pgfpathlineto{\pgfqpoint{0.792145in}{8.371529in}}%
\pgfpathlineto{\pgfqpoint{0.795507in}{8.371406in}}%
\pgfpathlineto{\pgfqpoint{0.797748in}{8.365668in}}%
\pgfpathlineto{\pgfqpoint{0.798869in}{8.358326in}}%
\pgfpathlineto{\pgfqpoint{0.799989in}{8.364372in}}%
\pgfpathlineto{\pgfqpoint{0.803351in}{8.366964in}}%
\pgfpathlineto{\pgfqpoint{0.804471in}{8.371468in}}%
\pgfpathlineto{\pgfqpoint{0.805592in}{8.381154in}}%
\pgfpathlineto{\pgfqpoint{0.806712in}{8.380969in}}%
\pgfpathlineto{\pgfqpoint{0.807833in}{8.376650in}}%
\pgfpathlineto{\pgfqpoint{0.811194in}{8.373442in}}%
\pgfpathlineto{\pgfqpoint{0.812315in}{8.367704in}}%
\pgfpathlineto{\pgfqpoint{0.813435in}{8.370295in}}%
\pgfpathlineto{\pgfqpoint{0.815676in}{8.385473in}}%
\pgfpathlineto{\pgfqpoint{0.820159in}{8.383005in}}%
\pgfpathlineto{\pgfqpoint{0.821279in}{8.382882in}}%
\pgfpathlineto{\pgfqpoint{0.822400in}{8.377144in}}%
\pgfpathlineto{\pgfqpoint{0.823520in}{8.385411in}}%
\pgfpathlineto{\pgfqpoint{0.826882in}{8.383930in}}%
\pgfpathlineto{\pgfqpoint{0.828002in}{8.385411in}}%
\pgfpathlineto{\pgfqpoint{0.830243in}{8.384856in}}%
\pgfpathlineto{\pgfqpoint{0.831364in}{8.388434in}}%
\pgfpathlineto{\pgfqpoint{0.834725in}{8.388990in}}%
\pgfpathlineto{\pgfqpoint{0.835846in}{8.388496in}}%
\pgfpathlineto{\pgfqpoint{0.836966in}{8.389730in}}%
\pgfpathlineto{\pgfqpoint{0.838087in}{8.395900in}}%
\pgfpathlineto{\pgfqpoint{0.839208in}{8.398429in}}%
\pgfpathlineto{\pgfqpoint{0.842569in}{8.396640in}}%
\pgfpathlineto{\pgfqpoint{0.843690in}{8.393185in}}%
\pgfpathlineto{\pgfqpoint{0.844810in}{8.393494in}}%
\pgfpathlineto{\pgfqpoint{0.845931in}{8.389853in}}%
\pgfpathlineto{\pgfqpoint{0.847051in}{8.394234in}}%
\pgfpathlineto{\pgfqpoint{0.850413in}{8.393000in}}%
\pgfpathlineto{\pgfqpoint{0.851533in}{8.391519in}}%
\pgfpathlineto{\pgfqpoint{0.852654in}{8.392198in}}%
\pgfpathlineto{\pgfqpoint{0.853774in}{8.388743in}}%
\pgfpathlineto{\pgfqpoint{0.854895in}{8.393062in}}%
\pgfpathlineto{\pgfqpoint{0.859377in}{8.388311in}}%
\pgfpathlineto{\pgfqpoint{0.860498in}{8.388681in}}%
\pgfpathlineto{\pgfqpoint{0.861618in}{8.396578in}}%
\pgfpathlineto{\pgfqpoint{0.862739in}{8.394172in}}%
\pgfpathlineto{\pgfqpoint{0.866100in}{8.383128in}}%
\pgfpathlineto{\pgfqpoint{0.867221in}{8.385720in}}%
\pgfpathlineto{\pgfqpoint{0.868341in}{8.383807in}}%
\pgfpathlineto{\pgfqpoint{0.869462in}{8.390285in}}%
\pgfpathlineto{\pgfqpoint{0.870582in}{8.400157in}}%
\pgfpathlineto{\pgfqpoint{0.873944in}{8.399170in}}%
\pgfpathlineto{\pgfqpoint{0.875064in}{8.397319in}}%
\pgfpathlineto{\pgfqpoint{0.876185in}{8.398368in}}%
\pgfpathlineto{\pgfqpoint{0.877305in}{8.398121in}}%
\pgfpathlineto{\pgfqpoint{0.878426in}{8.396208in}}%
\pgfpathlineto{\pgfqpoint{0.881788in}{8.398923in}}%
\pgfpathlineto{\pgfqpoint{0.882908in}{8.394234in}}%
\pgfpathlineto{\pgfqpoint{0.884029in}{8.393000in}}%
\pgfpathlineto{\pgfqpoint{0.885149in}{8.394172in}}%
\pgfpathlineto{\pgfqpoint{0.886270in}{8.392136in}}%
\pgfpathlineto{\pgfqpoint{0.891872in}{8.399170in}}%
\pgfpathlineto{\pgfqpoint{0.894113in}{8.405216in}}%
\pgfpathlineto{\pgfqpoint{0.897475in}{8.407375in}}%
\pgfpathlineto{\pgfqpoint{0.898595in}{8.399972in}}%
\pgfpathlineto{\pgfqpoint{0.900836in}{8.394234in}}%
\pgfpathlineto{\pgfqpoint{0.901957in}{8.393864in}}%
\pgfpathlineto{\pgfqpoint{0.905319in}{8.394049in}}%
\pgfpathlineto{\pgfqpoint{0.906439in}{8.401453in}}%
\pgfpathlineto{\pgfqpoint{0.907560in}{8.404476in}}%
\pgfpathlineto{\pgfqpoint{0.908680in}{8.404106in}}%
\pgfpathlineto{\pgfqpoint{0.909801in}{8.394789in}}%
\pgfpathlineto{\pgfqpoint{0.913162in}{8.392691in}}%
\pgfpathlineto{\pgfqpoint{0.914283in}{8.373133in}}%
\pgfpathlineto{\pgfqpoint{0.915403in}{8.371344in}}%
\pgfpathlineto{\pgfqpoint{0.916524in}{8.368383in}}%
\pgfpathlineto{\pgfqpoint{0.917644in}{8.369493in}}%
\pgfpathlineto{\pgfqpoint{0.923247in}{8.367272in}}%
\pgfpathlineto{\pgfqpoint{0.924368in}{8.375786in}}%
\pgfpathlineto{\pgfqpoint{0.925488in}{8.374367in}}%
\pgfpathlineto{\pgfqpoint{0.928850in}{8.377576in}}%
\pgfpathlineto{\pgfqpoint{0.929970in}{8.384116in}}%
\pgfpathlineto{\pgfqpoint{0.932211in}{8.372146in}}%
\pgfpathlineto{\pgfqpoint{0.933332in}{8.373504in}}%
\pgfpathlineto{\pgfqpoint{0.936693in}{8.375416in}}%
\pgfpathlineto{\pgfqpoint{0.937814in}{8.374923in}}%
\pgfpathlineto{\pgfqpoint{0.938934in}{8.365791in}}%
\pgfpathlineto{\pgfqpoint{0.941175in}{8.371899in}}%
\pgfpathlineto{\pgfqpoint{0.944537in}{8.377452in}}%
\pgfpathlineto{\pgfqpoint{0.945658in}{8.377637in}}%
\pgfpathlineto{\pgfqpoint{0.946778in}{8.377205in}}%
\pgfpathlineto{\pgfqpoint{0.949019in}{8.384177in}}%
\pgfpathlineto{\pgfqpoint{0.952381in}{8.383190in}}%
\pgfpathlineto{\pgfqpoint{0.953501in}{8.384362in}}%
\pgfpathlineto{\pgfqpoint{0.954622in}{8.387571in}}%
\pgfpathlineto{\pgfqpoint{0.955742in}{8.386090in}}%
\pgfpathlineto{\pgfqpoint{0.956863in}{8.387694in}}%
\pgfpathlineto{\pgfqpoint{0.961345in}{8.383375in}}%
\pgfpathlineto{\pgfqpoint{0.962465in}{8.386707in}}%
\pgfpathlineto{\pgfqpoint{0.963586in}{8.387694in}}%
\pgfpathlineto{\pgfqpoint{0.964707in}{8.390594in}}%
\pgfpathlineto{\pgfqpoint{0.968068in}{8.392445in}}%
\pgfpathlineto{\pgfqpoint{0.969189in}{8.401823in}}%
\pgfpathlineto{\pgfqpoint{0.972550in}{8.394542in}}%
\pgfpathlineto{\pgfqpoint{0.975912in}{8.398491in}}%
\pgfpathlineto{\pgfqpoint{0.977032in}{8.402748in}}%
\pgfpathlineto{\pgfqpoint{0.978153in}{8.398429in}}%
\pgfpathlineto{\pgfqpoint{0.979273in}{8.404167in}}%
\pgfpathlineto{\pgfqpoint{0.980394in}{8.398800in}}%
\pgfpathlineto{\pgfqpoint{0.983756in}{8.399293in}}%
\pgfpathlineto{\pgfqpoint{0.985997in}{8.398676in}}%
\pgfpathlineto{\pgfqpoint{0.987117in}{8.396455in}}%
\pgfpathlineto{\pgfqpoint{0.988238in}{8.391951in}}%
\pgfpathlineto{\pgfqpoint{0.991599in}{8.397504in}}%
\pgfpathlineto{\pgfqpoint{0.993840in}{8.407561in}}%
\pgfpathlineto{\pgfqpoint{0.994961in}{8.407005in}}%
\pgfpathlineto{\pgfqpoint{0.996081in}{8.410584in}}%
\pgfpathlineto{\pgfqpoint{1.000563in}{8.411262in}}%
\pgfpathlineto{\pgfqpoint{1.002804in}{8.418481in}}%
\pgfpathlineto{\pgfqpoint{1.003925in}{8.415334in}}%
\pgfpathlineto{\pgfqpoint{1.009528in}{8.422183in}}%
\pgfpathlineto{\pgfqpoint{1.011769in}{8.428106in}}%
\pgfpathlineto{\pgfqpoint{1.018492in}{8.432918in}}%
\pgfpathlineto{\pgfqpoint{1.019612in}{8.437731in}}%
\pgfpathlineto{\pgfqpoint{1.022974in}{8.438039in}}%
\pgfpathlineto{\pgfqpoint{1.024094in}{8.444024in}}%
\pgfpathlineto{\pgfqpoint{1.025215in}{8.438779in}}%
\pgfpathlineto{\pgfqpoint{1.026336in}{8.437484in}}%
\pgfpathlineto{\pgfqpoint{1.027456in}{8.442728in}}%
\pgfpathlineto{\pgfqpoint{1.030818in}{8.438656in}}%
\pgfpathlineto{\pgfqpoint{1.031938in}{8.442358in}}%
\pgfpathlineto{\pgfqpoint{1.033059in}{8.448589in}}%
\pgfpathlineto{\pgfqpoint{1.034179in}{8.446183in}}%
\pgfpathlineto{\pgfqpoint{1.035300in}{8.448466in}}%
\pgfpathlineto{\pgfqpoint{1.038661in}{8.448281in}}%
\pgfpathlineto{\pgfqpoint{1.039782in}{8.452600in}}%
\pgfpathlineto{\pgfqpoint{1.040902in}{8.452785in}}%
\pgfpathlineto{\pgfqpoint{1.042023in}{8.452353in}}%
\pgfpathlineto{\pgfqpoint{1.043143in}{8.454759in}}%
\pgfpathlineto{\pgfqpoint{1.047626in}{8.459695in}}%
\pgfpathlineto{\pgfqpoint{1.048746in}{8.454327in}}%
\pgfpathlineto{\pgfqpoint{1.049867in}{8.452044in}}%
\pgfpathlineto{\pgfqpoint{1.050987in}{8.456363in}}%
\pgfpathlineto{\pgfqpoint{1.054349in}{8.446985in}}%
\pgfpathlineto{\pgfqpoint{1.055469in}{8.449947in}}%
\pgfpathlineto{\pgfqpoint{1.056590in}{8.456487in}}%
\pgfpathlineto{\pgfqpoint{1.057710in}{8.458769in}}%
\pgfpathlineto{\pgfqpoint{1.058831in}{8.457536in}}%
\pgfpathlineto{\pgfqpoint{1.062192in}{8.455006in}}%
\pgfpathlineto{\pgfqpoint{1.063313in}{8.461114in}}%
\pgfpathlineto{\pgfqpoint{1.064433in}{8.462225in}}%
\pgfpathlineto{\pgfqpoint{1.065554in}{8.461546in}}%
\pgfpathlineto{\pgfqpoint{1.066675in}{8.467716in}}%
\pgfpathlineto{\pgfqpoint{1.070036in}{8.468209in}}%
\pgfpathlineto{\pgfqpoint{1.071157in}{8.464631in}}%
\pgfpathlineto{\pgfqpoint{1.072277in}{8.464446in}}%
\pgfpathlineto{\pgfqpoint{1.073398in}{8.469320in}}%
\pgfpathlineto{\pgfqpoint{1.074518in}{8.471294in}}%
\pgfpathlineto{\pgfqpoint{1.079000in}{8.464939in}}%
\pgfpathlineto{\pgfqpoint{1.080121in}{8.467407in}}%
\pgfpathlineto{\pgfqpoint{1.081241in}{8.463644in}}%
\pgfpathlineto{\pgfqpoint{1.082362in}{8.471417in}}%
\pgfpathlineto{\pgfqpoint{1.085723in}{8.464878in}}%
\pgfpathlineto{\pgfqpoint{1.086844in}{8.469567in}}%
\pgfpathlineto{\pgfqpoint{1.087965in}{8.465494in}}%
\pgfpathlineto{\pgfqpoint{1.089085in}{8.470800in}}%
\pgfpathlineto{\pgfqpoint{1.093567in}{8.467345in}}%
\pgfpathlineto{\pgfqpoint{1.094688in}{8.471911in}}%
\pgfpathlineto{\pgfqpoint{1.095808in}{8.467530in}}%
\pgfpathlineto{\pgfqpoint{1.096929in}{8.468271in}}%
\pgfpathlineto{\pgfqpoint{1.098049in}{8.468024in}}%
\pgfpathlineto{\pgfqpoint{1.101411in}{8.467901in}}%
\pgfpathlineto{\pgfqpoint{1.102531in}{8.468950in}}%
\pgfpathlineto{\pgfqpoint{1.103652in}{8.478019in}}%
\pgfpathlineto{\pgfqpoint{1.104772in}{8.480981in}}%
\pgfpathlineto{\pgfqpoint{1.109255in}{8.467962in}}%
\pgfpathlineto{\pgfqpoint{1.110375in}{8.469998in}}%
\pgfpathlineto{\pgfqpoint{1.112616in}{8.463890in}}%
\pgfpathlineto{\pgfqpoint{1.113737in}{8.467716in}}%
\pgfpathlineto{\pgfqpoint{1.117098in}{8.468147in}}%
\pgfpathlineto{\pgfqpoint{1.118219in}{8.476477in}}%
\pgfpathlineto{\pgfqpoint{1.119339in}{8.478945in}}%
\pgfpathlineto{\pgfqpoint{1.120460in}{8.463335in}}%
\pgfpathlineto{\pgfqpoint{1.121580in}{8.457659in}}%
\pgfpathlineto{\pgfqpoint{1.124942in}{8.457844in}}%
\pgfpathlineto{\pgfqpoint{1.126062in}{8.462471in}}%
\pgfpathlineto{\pgfqpoint{1.127183in}{8.461608in}}%
\pgfpathlineto{\pgfqpoint{1.129424in}{8.478759in}}%
\pgfpathlineto{\pgfqpoint{1.132786in}{8.478821in}}%
\pgfpathlineto{\pgfqpoint{1.135027in}{8.480672in}}%
\pgfpathlineto{\pgfqpoint{1.136147in}{8.489680in}}%
\pgfpathlineto{\pgfqpoint{1.137268in}{8.492580in}}%
\pgfpathlineto{\pgfqpoint{1.141750in}{8.493197in}}%
\pgfpathlineto{\pgfqpoint{1.142870in}{8.497947in}}%
\pgfpathlineto{\pgfqpoint{1.143991in}{8.495665in}}%
\pgfpathlineto{\pgfqpoint{1.145111in}{8.497330in}}%
\pgfpathlineto{\pgfqpoint{1.148473in}{8.499058in}}%
\pgfpathlineto{\pgfqpoint{1.149594in}{8.500600in}}%
\pgfpathlineto{\pgfqpoint{1.151835in}{8.495541in}}%
\pgfpathlineto{\pgfqpoint{1.152955in}{8.494801in}}%
\pgfpathlineto{\pgfqpoint{1.157437in}{8.501711in}}%
\pgfpathlineto{\pgfqpoint{1.158558in}{8.499305in}}%
\pgfpathlineto{\pgfqpoint{1.159678in}{8.500847in}}%
\pgfpathlineto{\pgfqpoint{1.160799in}{8.494801in}}%
\pgfpathlineto{\pgfqpoint{1.164160in}{8.496590in}}%
\pgfpathlineto{\pgfqpoint{1.165281in}{8.493505in}}%
\pgfpathlineto{\pgfqpoint{1.166401in}{8.485978in}}%
\pgfpathlineto{\pgfqpoint{1.167522in}{8.486410in}}%
\pgfpathlineto{\pgfqpoint{1.168642in}{8.499181in}}%
\pgfpathlineto{\pgfqpoint{1.172004in}{8.497639in}}%
\pgfpathlineto{\pgfqpoint{1.173125in}{8.494554in}}%
\pgfpathlineto{\pgfqpoint{1.174245in}{8.488261in}}%
\pgfpathlineto{\pgfqpoint{1.175366in}{8.499675in}}%
\pgfpathlineto{\pgfqpoint{1.176486in}{8.498811in}}%
\pgfpathlineto{\pgfqpoint{1.179848in}{8.503438in}}%
\pgfpathlineto{\pgfqpoint{1.180968in}{8.508991in}}%
\pgfpathlineto{\pgfqpoint{1.182089in}{8.501649in}}%
\pgfpathlineto{\pgfqpoint{1.183209in}{8.486965in}}%
\pgfpathlineto{\pgfqpoint{1.184330in}{8.491222in}}%
\pgfpathlineto{\pgfqpoint{1.187691in}{8.480240in}}%
\pgfpathlineto{\pgfqpoint{1.188812in}{8.484127in}}%
\pgfpathlineto{\pgfqpoint{1.189933in}{8.491716in}}%
\pgfpathlineto{\pgfqpoint{1.191053in}{8.494616in}}%
\pgfpathlineto{\pgfqpoint{1.192174in}{8.489988in}}%
\pgfpathlineto{\pgfqpoint{1.195535in}{8.489742in}}%
\pgfpathlineto{\pgfqpoint{1.196656in}{8.486718in}}%
\pgfpathlineto{\pgfqpoint{1.197776in}{8.490482in}}%
\pgfpathlineto{\pgfqpoint{1.200017in}{8.501464in}}%
\pgfpathlineto{\pgfqpoint{1.203379in}{8.504549in}}%
\pgfpathlineto{\pgfqpoint{1.204499in}{8.510966in}}%
\pgfpathlineto{\pgfqpoint{1.205620in}{8.511397in}}%
\pgfpathlineto{\pgfqpoint{1.207861in}{8.520405in}}%
\pgfpathlineto{\pgfqpoint{1.211223in}{8.518863in}}%
\pgfpathlineto{\pgfqpoint{1.212343in}{8.516148in}}%
\pgfpathlineto{\pgfqpoint{1.213464in}{8.517752in}}%
\pgfpathlineto{\pgfqpoint{1.215705in}{8.525896in}}%
\pgfpathlineto{\pgfqpoint{1.219066in}{8.526390in}}%
\pgfpathlineto{\pgfqpoint{1.220187in}{8.528858in}}%
\pgfpathlineto{\pgfqpoint{1.221307in}{8.526637in}}%
\pgfpathlineto{\pgfqpoint{1.223548in}{8.529660in}}%
\pgfpathlineto{\pgfqpoint{1.226910in}{8.528117in}}%
\pgfpathlineto{\pgfqpoint{1.228030in}{8.529290in}}%
\pgfpathlineto{\pgfqpoint{1.229151in}{8.532436in}}%
\pgfpathlineto{\pgfqpoint{1.230271in}{8.537557in}}%
\pgfpathlineto{\pgfqpoint{1.234754in}{8.534966in}}%
\pgfpathlineto{\pgfqpoint{1.235874in}{8.532806in}}%
\pgfpathlineto{\pgfqpoint{1.236995in}{8.534534in}}%
\pgfpathlineto{\pgfqpoint{1.238115in}{8.538976in}}%
\pgfpathlineto{\pgfqpoint{1.239236in}{8.537064in}}%
\pgfpathlineto{\pgfqpoint{1.242597in}{8.537372in}}%
\pgfpathlineto{\pgfqpoint{1.243718in}{8.538914in}}%
\pgfpathlineto{\pgfqpoint{1.244838in}{8.533423in}}%
\pgfpathlineto{\pgfqpoint{1.245959in}{8.524107in}}%
\pgfpathlineto{\pgfqpoint{1.247079in}{8.524354in}}%
\pgfpathlineto{\pgfqpoint{1.251561in}{8.521824in}}%
\pgfpathlineto{\pgfqpoint{1.252682in}{8.514667in}}%
\pgfpathlineto{\pgfqpoint{1.253803in}{8.521269in}}%
\pgfpathlineto{\pgfqpoint{1.254923in}{8.519788in}}%
\pgfpathlineto{\pgfqpoint{1.258285in}{8.519418in}}%
\pgfpathlineto{\pgfqpoint{1.259405in}{8.510966in}}%
\pgfpathlineto{\pgfqpoint{1.262767in}{8.515469in}}%
\pgfpathlineto{\pgfqpoint{1.267249in}{8.513557in}}%
\pgfpathlineto{\pgfqpoint{1.268369in}{8.520590in}}%
\pgfpathlineto{\pgfqpoint{1.270610in}{8.523182in}}%
\pgfpathlineto{\pgfqpoint{1.275093in}{8.536693in}}%
\pgfpathlineto{\pgfqpoint{1.276213in}{8.542555in}}%
\pgfpathlineto{\pgfqpoint{1.277334in}{8.539902in}}%
\pgfpathlineto{\pgfqpoint{1.278454in}{8.541999in}}%
\pgfpathlineto{\pgfqpoint{1.281816in}{8.545331in}}%
\pgfpathlineto{\pgfqpoint{1.282936in}{8.549033in}}%
\pgfpathlineto{\pgfqpoint{1.284057in}{8.556190in}}%
\pgfpathlineto{\pgfqpoint{1.285177in}{8.557670in}}%
\pgfpathlineto{\pgfqpoint{1.286298in}{8.549465in}}%
\pgfpathlineto{\pgfqpoint{1.289659in}{8.555264in}}%
\pgfpathlineto{\pgfqpoint{1.290780in}{8.553598in}}%
\pgfpathlineto{\pgfqpoint{1.291900in}{8.550452in}}%
\pgfpathlineto{\pgfqpoint{1.293021in}{8.552858in}}%
\pgfpathlineto{\pgfqpoint{1.294142in}{8.550575in}}%
\pgfpathlineto{\pgfqpoint{1.297503in}{8.546256in}}%
\pgfpathlineto{\pgfqpoint{1.298624in}{8.547367in}}%
\pgfpathlineto{\pgfqpoint{1.300865in}{8.541753in}}%
\pgfpathlineto{\pgfqpoint{1.301985in}{8.546256in}}%
\pgfpathlineto{\pgfqpoint{1.305347in}{8.542493in}}%
\pgfpathlineto{\pgfqpoint{1.306467in}{8.534411in}}%
\pgfpathlineto{\pgfqpoint{1.307588in}{8.536570in}}%
\pgfpathlineto{\pgfqpoint{1.309829in}{8.553228in}}%
\pgfpathlineto{\pgfqpoint{1.313190in}{8.556807in}}%
\pgfpathlineto{\pgfqpoint{1.314311in}{8.548478in}}%
\pgfpathlineto{\pgfqpoint{1.315432in}{8.554339in}}%
\pgfpathlineto{\pgfqpoint{1.316552in}{8.563223in}}%
\pgfpathlineto{\pgfqpoint{1.317673in}{8.564396in}}%
\pgfpathlineto{\pgfqpoint{1.321034in}{8.566555in}}%
\pgfpathlineto{\pgfqpoint{1.322155in}{8.569455in}}%
\pgfpathlineto{\pgfqpoint{1.323275in}{8.566308in}}%
\pgfpathlineto{\pgfqpoint{1.324396in}{8.567851in}}%
\pgfpathlineto{\pgfqpoint{1.325516in}{8.572725in}}%
\pgfpathlineto{\pgfqpoint{1.328878in}{8.575378in}}%
\pgfpathlineto{\pgfqpoint{1.329998in}{8.577290in}}%
\pgfpathlineto{\pgfqpoint{1.331119in}{8.574699in}}%
\pgfpathlineto{\pgfqpoint{1.332239in}{8.580313in}}%
\pgfpathlineto{\pgfqpoint{1.333360in}{8.580560in}}%
\pgfpathlineto{\pgfqpoint{1.336722in}{8.582781in}}%
\pgfpathlineto{\pgfqpoint{1.337842in}{8.581732in}}%
\pgfpathlineto{\pgfqpoint{1.338963in}{8.587038in}}%
\pgfpathlineto{\pgfqpoint{1.340083in}{8.583275in}}%
\pgfpathlineto{\pgfqpoint{1.341204in}{8.591604in}}%
\pgfpathlineto{\pgfqpoint{1.344565in}{8.591419in}}%
\pgfpathlineto{\pgfqpoint{1.346806in}{8.594751in}}%
\pgfpathlineto{\pgfqpoint{1.347927in}{8.601105in}}%
\pgfpathlineto{\pgfqpoint{1.353529in}{8.602463in}}%
\pgfpathlineto{\pgfqpoint{1.354650in}{8.601229in}}%
\pgfpathlineto{\pgfqpoint{1.356891in}{8.611039in}}%
\pgfpathlineto{\pgfqpoint{1.360253in}{8.612149in}}%
\pgfpathlineto{\pgfqpoint{1.362494in}{8.624180in}}%
\pgfpathlineto{\pgfqpoint{1.364735in}{8.624242in}}%
\pgfpathlineto{\pgfqpoint{1.369217in}{8.587532in}}%
\pgfpathlineto{\pgfqpoint{1.370337in}{8.586792in}}%
\pgfpathlineto{\pgfqpoint{1.371458in}{8.588766in}}%
\pgfpathlineto{\pgfqpoint{1.372578in}{8.598206in}}%
\pgfpathlineto{\pgfqpoint{1.375940in}{8.598021in}}%
\pgfpathlineto{\pgfqpoint{1.378181in}{8.588458in}}%
\pgfpathlineto{\pgfqpoint{1.380422in}{8.586668in}}%
\pgfpathlineto{\pgfqpoint{1.383784in}{8.593147in}}%
\pgfpathlineto{\pgfqpoint{1.386025in}{8.636396in}}%
\pgfpathlineto{\pgfqpoint{1.388266in}{8.641271in}}%
\pgfpathlineto{\pgfqpoint{1.391627in}{8.641702in}}%
\pgfpathlineto{\pgfqpoint{1.392748in}{8.642690in}}%
\pgfpathlineto{\pgfqpoint{1.394989in}{8.649600in}}%
\pgfpathlineto{\pgfqpoint{1.396109in}{8.655214in}}%
\pgfpathlineto{\pgfqpoint{1.399471in}{8.655584in}}%
\pgfpathlineto{\pgfqpoint{1.400592in}{8.660027in}}%
\pgfpathlineto{\pgfqpoint{1.402833in}{8.648736in}}%
\pgfpathlineto{\pgfqpoint{1.403953in}{8.650463in}}%
\pgfpathlineto{\pgfqpoint{1.407315in}{8.646083in}}%
\pgfpathlineto{\pgfqpoint{1.408435in}{8.646206in}}%
\pgfpathlineto{\pgfqpoint{1.409556in}{8.640777in}}%
\pgfpathlineto{\pgfqpoint{1.411797in}{8.638432in}}%
\pgfpathlineto{\pgfqpoint{1.415158in}{8.630473in}}%
\pgfpathlineto{\pgfqpoint{1.416279in}{8.644911in}}%
\pgfpathlineto{\pgfqpoint{1.417400in}{8.650402in}}%
\pgfpathlineto{\pgfqpoint{1.418520in}{8.648921in}}%
\pgfpathlineto{\pgfqpoint{1.419641in}{8.644417in}}%
\pgfpathlineto{\pgfqpoint{1.424123in}{8.642690in}}%
\pgfpathlineto{\pgfqpoint{1.425243in}{8.639975in}}%
\pgfpathlineto{\pgfqpoint{1.426364in}{8.630597in}}%
\pgfpathlineto{\pgfqpoint{1.427484in}{8.606782in}}%
\pgfpathlineto{\pgfqpoint{1.430846in}{8.600057in}}%
\pgfpathlineto{\pgfqpoint{1.433087in}{8.606905in}}%
\pgfpathlineto{\pgfqpoint{1.434207in}{8.595244in}}%
\pgfpathlineto{\pgfqpoint{1.435328in}{8.595985in}}%
\pgfpathlineto{\pgfqpoint{1.438690in}{8.573218in}}%
\pgfpathlineto{\pgfqpoint{1.439810in}{8.588211in}}%
\pgfpathlineto{\pgfqpoint{1.440931in}{8.591604in}}%
\pgfpathlineto{\pgfqpoint{1.443172in}{8.607337in}}%
\pgfpathlineto{\pgfqpoint{1.446533in}{8.604005in}}%
\pgfpathlineto{\pgfqpoint{1.447654in}{8.610792in}}%
\pgfpathlineto{\pgfqpoint{1.448774in}{8.612458in}}%
\pgfpathlineto{\pgfqpoint{1.449895in}{8.610854in}}%
\pgfpathlineto{\pgfqpoint{1.451015in}{8.621466in}}%
\pgfpathlineto{\pgfqpoint{1.455497in}{8.619738in}}%
\pgfpathlineto{\pgfqpoint{1.456618in}{8.613136in}}%
\pgfpathlineto{\pgfqpoint{1.457738in}{8.618442in}}%
\pgfpathlineto{\pgfqpoint{1.458859in}{8.618504in}}%
\pgfpathlineto{\pgfqpoint{1.462221in}{8.621898in}}%
\pgfpathlineto{\pgfqpoint{1.463341in}{8.625784in}}%
\pgfpathlineto{\pgfqpoint{1.464462in}{8.625414in}}%
\pgfpathlineto{\pgfqpoint{1.465582in}{8.633312in}}%
\pgfpathlineto{\pgfqpoint{1.466703in}{8.635409in}}%
\pgfpathlineto{\pgfqpoint{1.470064in}{8.621959in}}%
\pgfpathlineto{\pgfqpoint{1.471185in}{8.624427in}}%
\pgfpathlineto{\pgfqpoint{1.472305in}{8.630720in}}%
\pgfpathlineto{\pgfqpoint{1.473426in}{8.631954in}}%
\pgfpathlineto{\pgfqpoint{1.474546in}{8.632078in}}%
\pgfpathlineto{\pgfqpoint{1.477908in}{8.629178in}}%
\pgfpathlineto{\pgfqpoint{1.479029in}{8.623563in}}%
\pgfpathlineto{\pgfqpoint{1.480149in}{8.623872in}}%
\pgfpathlineto{\pgfqpoint{1.482390in}{8.609250in}}%
\pgfpathlineto{\pgfqpoint{1.486872in}{8.624797in}}%
\pgfpathlineto{\pgfqpoint{1.487993in}{8.616715in}}%
\pgfpathlineto{\pgfqpoint{1.490234in}{8.626833in}}%
\pgfpathlineto{\pgfqpoint{1.493595in}{8.623070in}}%
\pgfpathlineto{\pgfqpoint{1.494716in}{8.631831in}}%
\pgfpathlineto{\pgfqpoint{1.495836in}{8.626648in}}%
\pgfpathlineto{\pgfqpoint{1.496957in}{8.625291in}}%
\pgfpathlineto{\pgfqpoint{1.498077in}{8.632571in}}%
\pgfpathlineto{\pgfqpoint{1.501439in}{8.640407in}}%
\pgfpathlineto{\pgfqpoint{1.502560in}{8.645034in}}%
\pgfpathlineto{\pgfqpoint{1.503680in}{8.642196in}}%
\pgfpathlineto{\pgfqpoint{1.504801in}{8.642936in}}%
\pgfpathlineto{\pgfqpoint{1.505921in}{8.641456in}}%
\pgfpathlineto{\pgfqpoint{1.509283in}{8.633620in}}%
\pgfpathlineto{\pgfqpoint{1.510403in}{8.636026in}}%
\pgfpathlineto{\pgfqpoint{1.511524in}{8.641332in}}%
\pgfpathlineto{\pgfqpoint{1.513765in}{8.622885in}}%
\pgfpathlineto{\pgfqpoint{1.517126in}{8.626957in}}%
\pgfpathlineto{\pgfqpoint{1.518247in}{8.632016in}}%
\pgfpathlineto{\pgfqpoint{1.519367in}{8.646330in}}%
\pgfpathlineto{\pgfqpoint{1.520488in}{8.651451in}}%
\pgfpathlineto{\pgfqpoint{1.526091in}{8.657620in}}%
\pgfpathlineto{\pgfqpoint{1.529452in}{8.645157in}}%
\pgfpathlineto{\pgfqpoint{1.533934in}{8.650463in}}%
\pgfpathlineto{\pgfqpoint{1.536175in}{8.667924in}}%
\pgfpathlineto{\pgfqpoint{1.537296in}{8.664222in}}%
\pgfpathlineto{\pgfqpoint{1.540658in}{8.666813in}}%
\pgfpathlineto{\pgfqpoint{1.541778in}{8.659780in}}%
\pgfpathlineto{\pgfqpoint{1.542899in}{8.669651in}}%
\pgfpathlineto{\pgfqpoint{1.544019in}{8.667986in}}%
\pgfpathlineto{\pgfqpoint{1.548501in}{8.678659in}}%
\pgfpathlineto{\pgfqpoint{1.549622in}{8.676561in}}%
\pgfpathlineto{\pgfqpoint{1.551863in}{8.668788in}}%
\pgfpathlineto{\pgfqpoint{1.556345in}{8.672489in}}%
\pgfpathlineto{\pgfqpoint{1.557465in}{8.664777in}}%
\pgfpathlineto{\pgfqpoint{1.558586in}{8.671687in}}%
\pgfpathlineto{\pgfqpoint{1.559706in}{8.669836in}}%
\pgfpathlineto{\pgfqpoint{1.560827in}{8.674279in}}%
\pgfpathlineto{\pgfqpoint{1.566430in}{8.675821in}}%
\pgfpathlineto{\pgfqpoint{1.567550in}{8.681004in}}%
\pgfpathlineto{\pgfqpoint{1.568671in}{8.681867in}}%
\pgfpathlineto{\pgfqpoint{1.572032in}{8.680634in}}%
\pgfpathlineto{\pgfqpoint{1.573153in}{8.683718in}}%
\pgfpathlineto{\pgfqpoint{1.574273in}{8.680325in}}%
\pgfpathlineto{\pgfqpoint{1.576514in}{8.693096in}}%
\pgfpathlineto{\pgfqpoint{1.579876in}{8.696737in}}%
\pgfpathlineto{\pgfqpoint{1.580996in}{8.694886in}}%
\pgfpathlineto{\pgfqpoint{1.582117in}{8.691862in}}%
\pgfpathlineto{\pgfqpoint{1.583238in}{8.684644in}}%
\pgfpathlineto{\pgfqpoint{1.584358in}{8.686248in}}%
\pgfpathlineto{\pgfqpoint{1.587720in}{8.686001in}}%
\pgfpathlineto{\pgfqpoint{1.588840in}{8.687976in}}%
\pgfpathlineto{\pgfqpoint{1.589961in}{8.691431in}}%
\pgfpathlineto{\pgfqpoint{1.591081in}{8.692294in}}%
\pgfpathlineto{\pgfqpoint{1.592202in}{8.695873in}}%
\pgfpathlineto{\pgfqpoint{1.595563in}{8.690135in}}%
\pgfpathlineto{\pgfqpoint{1.596684in}{8.684767in}}%
\pgfpathlineto{\pgfqpoint{1.597804in}{8.687790in}}%
\pgfpathlineto{\pgfqpoint{1.598925in}{8.688222in}}%
\pgfpathlineto{\pgfqpoint{1.600045in}{8.687976in}}%
\pgfpathlineto{\pgfqpoint{1.603407in}{8.685569in}}%
\pgfpathlineto{\pgfqpoint{1.605648in}{8.696305in}}%
\pgfpathlineto{\pgfqpoint{1.606769in}{8.697230in}}%
\pgfpathlineto{\pgfqpoint{1.611251in}{8.694577in}}%
\pgfpathlineto{\pgfqpoint{1.612371in}{8.692665in}}%
\pgfpathlineto{\pgfqpoint{1.613492in}{8.693343in}}%
\pgfpathlineto{\pgfqpoint{1.614612in}{8.689086in}}%
\pgfpathlineto{\pgfqpoint{1.615733in}{8.691307in}}%
\pgfpathlineto{\pgfqpoint{1.619094in}{8.695071in}}%
\pgfpathlineto{\pgfqpoint{1.620215in}{8.695379in}}%
\pgfpathlineto{\pgfqpoint{1.621335in}{8.701302in}}%
\pgfpathlineto{\pgfqpoint{1.622456in}{8.687173in}}%
\pgfpathlineto{\pgfqpoint{1.623577in}{8.694145in}}%
\pgfpathlineto{\pgfqpoint{1.626938in}{8.691307in}}%
\pgfpathlineto{\pgfqpoint{1.628059in}{8.695688in}}%
\pgfpathlineto{\pgfqpoint{1.629179in}{8.693343in}}%
\pgfpathlineto{\pgfqpoint{1.630300in}{8.695749in}}%
\pgfpathlineto{\pgfqpoint{1.631420in}{8.695688in}}%
\pgfpathlineto{\pgfqpoint{1.634782in}{8.697909in}}%
\pgfpathlineto{\pgfqpoint{1.635902in}{8.689765in}}%
\pgfpathlineto{\pgfqpoint{1.637023in}{8.688284in}}%
\pgfpathlineto{\pgfqpoint{1.638143in}{8.672921in}}%
\pgfpathlineto{\pgfqpoint{1.639264in}{8.668726in}}%
\pgfpathlineto{\pgfqpoint{1.642625in}{8.672181in}}%
\pgfpathlineto{\pgfqpoint{1.643746in}{8.666998in}}%
\pgfpathlineto{\pgfqpoint{1.644867in}{8.665764in}}%
\pgfpathlineto{\pgfqpoint{1.645987in}{8.663482in}}%
\pgfpathlineto{\pgfqpoint{1.647108in}{8.672736in}}%
\pgfpathlineto{\pgfqpoint{1.650469in}{8.671255in}}%
\pgfpathlineto{\pgfqpoint{1.651590in}{8.672860in}}%
\pgfpathlineto{\pgfqpoint{1.653831in}{8.680757in}}%
\pgfpathlineto{\pgfqpoint{1.654951in}{8.678474in}}%
\pgfpathlineto{\pgfqpoint{1.658313in}{8.690320in}}%
\pgfpathlineto{\pgfqpoint{1.659433in}{8.690999in}}%
\pgfpathlineto{\pgfqpoint{1.660554in}{8.697539in}}%
\pgfpathlineto{\pgfqpoint{1.661674in}{8.696983in}}%
\pgfpathlineto{\pgfqpoint{1.662795in}{8.695009in}}%
\pgfpathlineto{\pgfqpoint{1.666157in}{8.698094in}}%
\pgfpathlineto{\pgfqpoint{1.667277in}{8.697539in}}%
\pgfpathlineto{\pgfqpoint{1.668398in}{8.693960in}}%
\pgfpathlineto{\pgfqpoint{1.675121in}{8.695441in}}%
\pgfpathlineto{\pgfqpoint{1.677362in}{8.692171in}}%
\pgfpathlineto{\pgfqpoint{1.678482in}{8.695379in}}%
\pgfpathlineto{\pgfqpoint{1.681844in}{8.699081in}}%
\pgfpathlineto{\pgfqpoint{1.682964in}{8.696798in}}%
\pgfpathlineto{\pgfqpoint{1.684085in}{8.697724in}}%
\pgfpathlineto{\pgfqpoint{1.686326in}{8.693960in}}%
\pgfpathlineto{\pgfqpoint{1.689688in}{8.696860in}}%
\pgfpathlineto{\pgfqpoint{1.691929in}{8.701672in}}%
\pgfpathlineto{\pgfqpoint{1.693049in}{8.709631in}}%
\pgfpathlineto{\pgfqpoint{1.694170in}{8.708829in}}%
\pgfpathlineto{\pgfqpoint{1.697531in}{8.703523in}}%
\pgfpathlineto{\pgfqpoint{1.698652in}{8.696305in}}%
\pgfpathlineto{\pgfqpoint{1.699772in}{8.699019in}}%
\pgfpathlineto{\pgfqpoint{1.700893in}{8.686063in}}%
\pgfpathlineto{\pgfqpoint{1.705375in}{8.684520in}}%
\pgfpathlineto{\pgfqpoint{1.706496in}{8.681744in}}%
\pgfpathlineto{\pgfqpoint{1.707616in}{8.668232in}}%
\pgfpathlineto{\pgfqpoint{1.708737in}{8.665456in}}%
\pgfpathlineto{\pgfqpoint{1.709857in}{8.673538in}}%
\pgfpathlineto{\pgfqpoint{1.713219in}{8.674464in}}%
\pgfpathlineto{\pgfqpoint{1.714339in}{8.660088in}}%
\pgfpathlineto{\pgfqpoint{1.715460in}{8.680263in}}%
\pgfpathlineto{\pgfqpoint{1.716580in}{8.665271in}}%
\pgfpathlineto{\pgfqpoint{1.717701in}{8.639296in}}%
\pgfpathlineto{\pgfqpoint{1.721062in}{8.634237in}}%
\pgfpathlineto{\pgfqpoint{1.722183in}{8.641147in}}%
\pgfpathlineto{\pgfqpoint{1.723303in}{8.641394in}}%
\pgfpathlineto{\pgfqpoint{1.724424in}{8.645898in}}%
\pgfpathlineto{\pgfqpoint{1.725544in}{8.658608in}}%
\pgfpathlineto{\pgfqpoint{1.728906in}{8.659656in}}%
\pgfpathlineto{\pgfqpoint{1.730027in}{8.677672in}}%
\pgfpathlineto{\pgfqpoint{1.731147in}{8.666937in}}%
\pgfpathlineto{\pgfqpoint{1.733388in}{8.719133in}}%
\pgfpathlineto{\pgfqpoint{1.736750in}{8.724377in}}%
\pgfpathlineto{\pgfqpoint{1.737870in}{8.732459in}}%
\pgfpathlineto{\pgfqpoint{1.738991in}{8.732213in}}%
\pgfpathlineto{\pgfqpoint{1.740111in}{8.738074in}}%
\pgfpathlineto{\pgfqpoint{1.741232in}{8.747143in}}%
\pgfpathlineto{\pgfqpoint{1.744593in}{8.744552in}}%
\pgfpathlineto{\pgfqpoint{1.745714in}{8.754053in}}%
\pgfpathlineto{\pgfqpoint{1.747955in}{8.759421in}}%
\pgfpathlineto{\pgfqpoint{1.749076in}{8.762136in}}%
\pgfpathlineto{\pgfqpoint{1.752437in}{8.768182in}}%
\pgfpathlineto{\pgfqpoint{1.753558in}{8.765159in}}%
\pgfpathlineto{\pgfqpoint{1.756919in}{8.774599in}}%
\pgfpathlineto{\pgfqpoint{1.760281in}{8.773735in}}%
\pgfpathlineto{\pgfqpoint{1.761401in}{8.781632in}}%
\pgfpathlineto{\pgfqpoint{1.762522in}{8.779103in}}%
\pgfpathlineto{\pgfqpoint{1.764763in}{8.786321in}}%
\pgfpathlineto{\pgfqpoint{1.768125in}{8.784779in}}%
\pgfpathlineto{\pgfqpoint{1.769245in}{8.774907in}}%
\pgfpathlineto{\pgfqpoint{1.770366in}{8.776265in}}%
\pgfpathlineto{\pgfqpoint{1.772607in}{8.785951in}}%
\pgfpathlineto{\pgfqpoint{1.775968in}{8.775463in}}%
\pgfpathlineto{\pgfqpoint{1.778209in}{8.797735in}}%
\pgfpathlineto{\pgfqpoint{1.780450in}{8.797797in}}%
\pgfpathlineto{\pgfqpoint{1.783812in}{8.790517in}}%
\pgfpathlineto{\pgfqpoint{1.784932in}{8.789961in}}%
\pgfpathlineto{\pgfqpoint{1.786053in}{8.775894in}}%
\pgfpathlineto{\pgfqpoint{1.787173in}{8.780830in}}%
\pgfpathlineto{\pgfqpoint{1.788294in}{8.769786in}}%
\pgfpathlineto{\pgfqpoint{1.791656in}{8.768306in}}%
\pgfpathlineto{\pgfqpoint{1.793897in}{8.788727in}}%
\pgfpathlineto{\pgfqpoint{1.795017in}{8.814270in}}%
\pgfpathlineto{\pgfqpoint{1.796138in}{8.815257in}}%
\pgfpathlineto{\pgfqpoint{1.799499in}{8.825006in}}%
\pgfpathlineto{\pgfqpoint{1.800620in}{8.822846in}}%
\pgfpathlineto{\pgfqpoint{1.801740in}{8.823278in}}%
\pgfpathlineto{\pgfqpoint{1.803981in}{8.819514in}}%
\pgfpathlineto{\pgfqpoint{1.807343in}{8.821921in}}%
\pgfpathlineto{\pgfqpoint{1.808463in}{8.817232in}}%
\pgfpathlineto{\pgfqpoint{1.809584in}{8.808964in}}%
\pgfpathlineto{\pgfqpoint{1.811825in}{8.807545in}}%
\pgfpathlineto{\pgfqpoint{1.816307in}{8.778115in}}%
\pgfpathlineto{\pgfqpoint{1.817428in}{8.784347in}}%
\pgfpathlineto{\pgfqpoint{1.818548in}{8.805201in}}%
\pgfpathlineto{\pgfqpoint{1.819669in}{8.794280in}}%
\pgfpathlineto{\pgfqpoint{1.823030in}{8.789468in}}%
\pgfpathlineto{\pgfqpoint{1.824151in}{8.788851in}}%
\pgfpathlineto{\pgfqpoint{1.825271in}{8.784594in}}%
\pgfpathlineto{\pgfqpoint{1.826392in}{8.783607in}}%
\pgfpathlineto{\pgfqpoint{1.827512in}{8.796316in}}%
\pgfpathlineto{\pgfqpoint{1.831995in}{8.796008in}}%
\pgfpathlineto{\pgfqpoint{1.833115in}{8.799278in}}%
\pgfpathlineto{\pgfqpoint{1.834236in}{8.817478in}}%
\pgfpathlineto{\pgfqpoint{1.835356in}{8.807298in}}%
\pgfpathlineto{\pgfqpoint{1.838718in}{8.808532in}}%
\pgfpathlineto{\pgfqpoint{1.839838in}{8.805201in}}%
\pgfpathlineto{\pgfqpoint{1.840959in}{8.806867in}}%
\pgfpathlineto{\pgfqpoint{1.842079in}{8.818774in}}%
\pgfpathlineto{\pgfqpoint{1.843200in}{8.797982in}}%
\pgfpathlineto{\pgfqpoint{1.846561in}{8.809519in}}%
\pgfpathlineto{\pgfqpoint{1.847682in}{8.817787in}}%
\pgfpathlineto{\pgfqpoint{1.848802in}{8.811617in}}%
\pgfpathlineto{\pgfqpoint{1.849923in}{8.821057in}}%
\pgfpathlineto{\pgfqpoint{1.854405in}{8.811802in}}%
\pgfpathlineto{\pgfqpoint{1.855526in}{8.817108in}}%
\pgfpathlineto{\pgfqpoint{1.856646in}{8.815381in}}%
\pgfpathlineto{\pgfqpoint{1.857767in}{8.823216in}}%
\pgfpathlineto{\pgfqpoint{1.858887in}{8.823401in}}%
\pgfpathlineto{\pgfqpoint{1.863369in}{8.828707in}}%
\pgfpathlineto{\pgfqpoint{1.864490in}{8.831360in}}%
\pgfpathlineto{\pgfqpoint{1.865610in}{8.829633in}}%
\pgfpathlineto{\pgfqpoint{1.866731in}{8.835309in}}%
\pgfpathlineto{\pgfqpoint{1.871213in}{8.840800in}}%
\pgfpathlineto{\pgfqpoint{1.872334in}{8.839504in}}%
\pgfpathlineto{\pgfqpoint{1.873454in}{8.843453in}}%
\pgfpathlineto{\pgfqpoint{1.874575in}{8.838209in}}%
\pgfpathlineto{\pgfqpoint{1.877936in}{8.848327in}}%
\pgfpathlineto{\pgfqpoint{1.879057in}{8.836111in}}%
\pgfpathlineto{\pgfqpoint{1.880177in}{8.830065in}}%
\pgfpathlineto{\pgfqpoint{1.881298in}{8.832286in}}%
\pgfpathlineto{\pgfqpoint{1.882418in}{8.814702in}}%
\pgfpathlineto{\pgfqpoint{1.885780in}{8.825746in}}%
\pgfpathlineto{\pgfqpoint{1.886900in}{8.803226in}}%
\pgfpathlineto{\pgfqpoint{1.888021in}{8.800327in}}%
\pgfpathlineto{\pgfqpoint{1.889141in}{8.815319in}}%
\pgfpathlineto{\pgfqpoint{1.890262in}{8.805879in}}%
\pgfpathlineto{\pgfqpoint{1.893624in}{8.824820in}}%
\pgfpathlineto{\pgfqpoint{1.894744in}{8.814023in}}%
\pgfpathlineto{\pgfqpoint{1.895865in}{8.826301in}}%
\pgfpathlineto{\pgfqpoint{1.896985in}{8.821859in}}%
\pgfpathlineto{\pgfqpoint{1.898106in}{8.826363in}}%
\pgfpathlineto{\pgfqpoint{1.901467in}{8.824389in}}%
\pgfpathlineto{\pgfqpoint{1.902588in}{8.825437in}}%
\pgfpathlineto{\pgfqpoint{1.903708in}{8.805633in}}%
\pgfpathlineto{\pgfqpoint{1.904829in}{8.805016in}}%
\pgfpathlineto{\pgfqpoint{1.909311in}{8.823957in}}%
\pgfpathlineto{\pgfqpoint{1.910431in}{8.817972in}}%
\pgfpathlineto{\pgfqpoint{1.911552in}{8.804645in}}%
\pgfpathlineto{\pgfqpoint{1.912673in}{8.806188in}}%
\pgfpathlineto{\pgfqpoint{1.918275in}{8.824820in}}%
\pgfpathlineto{\pgfqpoint{1.919396in}{8.825191in}}%
\pgfpathlineto{\pgfqpoint{1.921637in}{8.829571in}}%
\pgfpathlineto{\pgfqpoint{1.924998in}{8.822846in}}%
\pgfpathlineto{\pgfqpoint{1.926119in}{8.823648in}}%
\pgfpathlineto{\pgfqpoint{1.927239in}{8.826116in}}%
\pgfpathlineto{\pgfqpoint{1.928360in}{8.823093in}}%
\pgfpathlineto{\pgfqpoint{1.929480in}{8.800203in}}%
\pgfpathlineto{\pgfqpoint{1.932842in}{8.815566in}}%
\pgfpathlineto{\pgfqpoint{1.933963in}{8.812728in}}%
\pgfpathlineto{\pgfqpoint{1.935083in}{8.816430in}}%
\pgfpathlineto{\pgfqpoint{1.936204in}{8.789036in}}%
\pgfpathlineto{\pgfqpoint{1.937324in}{8.785334in}}%
\pgfpathlineto{\pgfqpoint{1.940686in}{8.779843in}}%
\pgfpathlineto{\pgfqpoint{1.941806in}{8.781756in}}%
\pgfpathlineto{\pgfqpoint{1.942927in}{8.774290in}}%
\pgfpathlineto{\pgfqpoint{1.944047in}{8.771144in}}%
\pgfpathlineto{\pgfqpoint{1.945168in}{8.778177in}}%
\pgfpathlineto{\pgfqpoint{1.948529in}{8.785519in}}%
\pgfpathlineto{\pgfqpoint{1.949650in}{8.779411in}}%
\pgfpathlineto{\pgfqpoint{1.950770in}{8.777930in}}%
\pgfpathlineto{\pgfqpoint{1.951891in}{8.783236in}}%
\pgfpathlineto{\pgfqpoint{1.953011in}{8.794157in}}%
\pgfpathlineto{\pgfqpoint{1.956373in}{8.790332in}}%
\pgfpathlineto{\pgfqpoint{1.957494in}{8.791257in}}%
\pgfpathlineto{\pgfqpoint{1.960855in}{8.808903in}}%
\pgfpathlineto{\pgfqpoint{1.964217in}{8.806558in}}%
\pgfpathlineto{\pgfqpoint{1.965337in}{8.807915in}}%
\pgfpathlineto{\pgfqpoint{1.966458in}{8.806867in}}%
\pgfpathlineto{\pgfqpoint{1.967578in}{8.807730in}}%
\pgfpathlineto{\pgfqpoint{1.968699in}{8.801807in}}%
\pgfpathlineto{\pgfqpoint{1.973181in}{8.794157in}}%
\pgfpathlineto{\pgfqpoint{1.974302in}{8.801992in}}%
\pgfpathlineto{\pgfqpoint{1.975422in}{8.801190in}}%
\pgfpathlineto{\pgfqpoint{1.976543in}{8.791319in}}%
\pgfpathlineto{\pgfqpoint{1.979904in}{8.790763in}}%
\pgfpathlineto{\pgfqpoint{1.981025in}{8.791319in}}%
\pgfpathlineto{\pgfqpoint{1.982145in}{8.797303in}}%
\pgfpathlineto{\pgfqpoint{1.984386in}{8.780337in}}%
\pgfpathlineto{\pgfqpoint{1.987748in}{8.777622in}}%
\pgfpathlineto{\pgfqpoint{1.988868in}{8.779781in}}%
\pgfpathlineto{\pgfqpoint{1.989989in}{8.791134in}}%
\pgfpathlineto{\pgfqpoint{1.991109in}{8.795638in}}%
\pgfpathlineto{\pgfqpoint{1.992230in}{8.785643in}}%
\pgfpathlineto{\pgfqpoint{1.995592in}{8.773673in}}%
\pgfpathlineto{\pgfqpoint{1.997833in}{8.779596in}}%
\pgfpathlineto{\pgfqpoint{1.998953in}{8.794342in}}%
\pgfpathlineto{\pgfqpoint{2.000074in}{8.790640in}}%
\pgfpathlineto{\pgfqpoint{2.004556in}{8.795514in}}%
\pgfpathlineto{\pgfqpoint{2.006797in}{8.773920in}}%
\pgfpathlineto{\pgfqpoint{2.007917in}{8.780398in}}%
\pgfpathlineto{\pgfqpoint{2.011279in}{8.763123in}}%
\pgfpathlineto{\pgfqpoint{2.012399in}{8.765036in}}%
\pgfpathlineto{\pgfqpoint{2.013520in}{8.772439in}}%
\pgfpathlineto{\pgfqpoint{2.014640in}{8.770959in}}%
\pgfpathlineto{\pgfqpoint{2.019123in}{8.769293in}}%
\pgfpathlineto{\pgfqpoint{2.020243in}{8.771267in}}%
\pgfpathlineto{\pgfqpoint{2.021364in}{8.756275in}}%
\pgfpathlineto{\pgfqpoint{2.023605in}{8.768614in}}%
\pgfpathlineto{\pgfqpoint{2.028087in}{8.778671in}}%
\pgfpathlineto{\pgfqpoint{2.029207in}{8.774414in}}%
\pgfpathlineto{\pgfqpoint{2.030328in}{8.780830in}}%
\pgfpathlineto{\pgfqpoint{2.031448in}{8.778177in}}%
\pgfpathlineto{\pgfqpoint{2.034810in}{8.780522in}}%
\pgfpathlineto{\pgfqpoint{2.035931in}{8.773056in}}%
\pgfpathlineto{\pgfqpoint{2.037051in}{8.771144in}}%
\pgfpathlineto{\pgfqpoint{2.038172in}{8.738567in}}%
\pgfpathlineto{\pgfqpoint{2.042654in}{8.734249in}}%
\pgfpathlineto{\pgfqpoint{2.043774in}{8.747452in}}%
\pgfpathlineto{\pgfqpoint{2.044895in}{8.749303in}}%
\pgfpathlineto{\pgfqpoint{2.046015in}{8.749981in}}%
\pgfpathlineto{\pgfqpoint{2.047136in}{8.748748in}}%
\pgfpathlineto{\pgfqpoint{2.050497in}{8.742022in}}%
\pgfpathlineto{\pgfqpoint{2.051618in}{8.743565in}}%
\pgfpathlineto{\pgfqpoint{2.052738in}{8.746897in}}%
\pgfpathlineto{\pgfqpoint{2.053859in}{8.737272in}}%
\pgfpathlineto{\pgfqpoint{2.054979in}{8.735236in}}%
\pgfpathlineto{\pgfqpoint{2.058341in}{8.748192in}}%
\pgfpathlineto{\pgfqpoint{2.059462in}{8.732953in}}%
\pgfpathlineto{\pgfqpoint{2.060582in}{8.733200in}}%
\pgfpathlineto{\pgfqpoint{2.061703in}{8.727030in}}%
\pgfpathlineto{\pgfqpoint{2.062823in}{8.731904in}}%
\pgfpathlineto{\pgfqpoint{2.066185in}{8.737148in}}%
\pgfpathlineto{\pgfqpoint{2.067305in}{8.732027in}}%
\pgfpathlineto{\pgfqpoint{2.068426in}{8.724192in}}%
\pgfpathlineto{\pgfqpoint{2.070667in}{8.703215in}}%
\pgfpathlineto{\pgfqpoint{2.075149in}{8.678597in}}%
\pgfpathlineto{\pgfqpoint{2.076269in}{8.706732in}}%
\pgfpathlineto{\pgfqpoint{2.077390in}{8.713271in}}%
\pgfpathlineto{\pgfqpoint{2.078511in}{8.714999in}}%
\pgfpathlineto{\pgfqpoint{2.081872in}{8.703523in}}%
\pgfpathlineto{\pgfqpoint{2.082993in}{8.683348in}}%
\pgfpathlineto{\pgfqpoint{2.084113in}{8.698526in}}%
\pgfpathlineto{\pgfqpoint{2.085234in}{8.701302in}}%
\pgfpathlineto{\pgfqpoint{2.086354in}{8.690752in}}%
\pgfpathlineto{\pgfqpoint{2.090836in}{8.710680in}}%
\pgfpathlineto{\pgfqpoint{2.091957in}{8.696428in}}%
\pgfpathlineto{\pgfqpoint{2.093077in}{8.695996in}}%
\pgfpathlineto{\pgfqpoint{2.094198in}{8.698649in}}%
\pgfpathlineto{\pgfqpoint{2.097559in}{8.696120in}}%
\pgfpathlineto{\pgfqpoint{2.098680in}{8.711606in}}%
\pgfpathlineto{\pgfqpoint{2.099801in}{8.714937in}}%
\pgfpathlineto{\pgfqpoint{2.100921in}{8.708027in}}%
\pgfpathlineto{\pgfqpoint{2.102042in}{8.689518in}}%
\pgfpathlineto{\pgfqpoint{2.105403in}{8.691862in}}%
\pgfpathlineto{\pgfqpoint{2.106524in}{8.680695in}}%
\pgfpathlineto{\pgfqpoint{2.107644in}{8.678659in}}%
\pgfpathlineto{\pgfqpoint{2.108765in}{8.678227in}}%
\pgfpathlineto{\pgfqpoint{2.109885in}{8.689209in}}%
\pgfpathlineto{\pgfqpoint{2.113247in}{8.682670in}}%
\pgfpathlineto{\pgfqpoint{2.114367in}{8.700253in}}%
\pgfpathlineto{\pgfqpoint{2.115488in}{8.701487in}}%
\pgfpathlineto{\pgfqpoint{2.116608in}{8.696120in}}%
\pgfpathlineto{\pgfqpoint{2.117729in}{8.709385in}}%
\pgfpathlineto{\pgfqpoint{2.121091in}{8.726722in}}%
\pgfpathlineto{\pgfqpoint{2.122211in}{8.723760in}}%
\pgfpathlineto{\pgfqpoint{2.124452in}{8.744182in}}%
\pgfpathlineto{\pgfqpoint{2.125573in}{8.746526in}}%
\pgfpathlineto{\pgfqpoint{2.128934in}{8.747452in}}%
\pgfpathlineto{\pgfqpoint{2.131175in}{8.737765in}}%
\pgfpathlineto{\pgfqpoint{2.132296in}{8.742578in}}%
\pgfpathlineto{\pgfqpoint{2.133416in}{8.739863in}}%
\pgfpathlineto{\pgfqpoint{2.136778in}{8.736038in}}%
\pgfpathlineto{\pgfqpoint{2.139019in}{8.746095in}}%
\pgfpathlineto{\pgfqpoint{2.140140in}{8.780337in}}%
\pgfpathlineto{\pgfqpoint{2.141260in}{8.779226in}}%
\pgfpathlineto{\pgfqpoint{2.145742in}{8.784409in}}%
\pgfpathlineto{\pgfqpoint{2.146863in}{8.791874in}}%
\pgfpathlineto{\pgfqpoint{2.149104in}{8.787062in}}%
\pgfpathlineto{\pgfqpoint{2.152465in}{8.802116in}}%
\pgfpathlineto{\pgfqpoint{2.153586in}{8.795638in}}%
\pgfpathlineto{\pgfqpoint{2.156947in}{8.798414in}}%
\pgfpathlineto{\pgfqpoint{2.160309in}{8.788481in}}%
\pgfpathlineto{\pgfqpoint{2.161430in}{8.789159in}}%
\pgfpathlineto{\pgfqpoint{2.162550in}{8.797242in}}%
\pgfpathlineto{\pgfqpoint{2.163671in}{8.782002in}}%
\pgfpathlineto{\pgfqpoint{2.164791in}{8.778424in}}%
\pgfpathlineto{\pgfqpoint{2.168153in}{8.792614in}}%
\pgfpathlineto{\pgfqpoint{2.169273in}{8.786074in}}%
\pgfpathlineto{\pgfqpoint{2.171514in}{8.799339in}}%
\pgfpathlineto{\pgfqpoint{2.172635in}{8.802486in}}%
\pgfpathlineto{\pgfqpoint{2.175996in}{8.800820in}}%
\pgfpathlineto{\pgfqpoint{2.177117in}{8.795823in}}%
\pgfpathlineto{\pgfqpoint{2.178237in}{8.795144in}}%
\pgfpathlineto{\pgfqpoint{2.180479in}{8.797118in}}%
\pgfpathlineto{\pgfqpoint{2.183840in}{8.789283in}}%
\pgfpathlineto{\pgfqpoint{2.184961in}{8.791072in}}%
\pgfpathlineto{\pgfqpoint{2.187202in}{8.777005in}}%
\pgfpathlineto{\pgfqpoint{2.188322in}{8.798414in}}%
\pgfpathlineto{\pgfqpoint{2.191684in}{8.795823in}}%
\pgfpathlineto{\pgfqpoint{2.193925in}{8.786321in}}%
\pgfpathlineto{\pgfqpoint{2.195045in}{8.794527in}}%
\pgfpathlineto{\pgfqpoint{2.196166in}{8.779535in}}%
\pgfpathlineto{\pgfqpoint{2.199527in}{8.795144in}}%
\pgfpathlineto{\pgfqpoint{2.200648in}{8.742146in}}%
\pgfpathlineto{\pgfqpoint{2.201769in}{8.752264in}}%
\pgfpathlineto{\pgfqpoint{2.202889in}{8.746156in}}%
\pgfpathlineto{\pgfqpoint{2.204010in}{8.735359in}}%
\pgfpathlineto{\pgfqpoint{2.207371in}{8.738506in}}%
\pgfpathlineto{\pgfqpoint{2.209612in}{8.759051in}}%
\pgfpathlineto{\pgfqpoint{2.215215in}{8.759359in}}%
\pgfpathlineto{\pgfqpoint{2.216335in}{8.768676in}}%
\pgfpathlineto{\pgfqpoint{2.223059in}{8.734804in}}%
\pgfpathlineto{\pgfqpoint{2.224179in}{8.738382in}}%
\pgfpathlineto{\pgfqpoint{2.226420in}{8.702166in}}%
\pgfpathlineto{\pgfqpoint{2.227541in}{8.699513in}}%
\pgfpathlineto{\pgfqpoint{2.230902in}{8.699328in}}%
\pgfpathlineto{\pgfqpoint{2.232023in}{8.701549in}}%
\pgfpathlineto{\pgfqpoint{2.233143in}{8.689580in}}%
\pgfpathlineto{\pgfqpoint{2.234264in}{8.703338in}}%
\pgfpathlineto{\pgfqpoint{2.235384in}{8.689456in}}%
\pgfpathlineto{\pgfqpoint{2.239866in}{8.687790in}}%
\pgfpathlineto{\pgfqpoint{2.240987in}{8.679770in}}%
\pgfpathlineto{\pgfqpoint{2.242107in}{8.684274in}}%
\pgfpathlineto{\pgfqpoint{2.243228in}{8.694084in}}%
\pgfpathlineto{\pgfqpoint{2.246590in}{8.683163in}}%
\pgfpathlineto{\pgfqpoint{2.247710in}{8.723452in}}%
\pgfpathlineto{\pgfqpoint{2.248831in}{8.727709in}}%
\pgfpathlineto{\pgfqpoint{2.249951in}{8.737580in}}%
\pgfpathlineto{\pgfqpoint{2.251072in}{8.758126in}}%
\pgfpathlineto{\pgfqpoint{2.255554in}{8.740665in}}%
\pgfpathlineto{\pgfqpoint{2.256674in}{8.766640in}}%
\pgfpathlineto{\pgfqpoint{2.257795in}{8.771761in}}%
\pgfpathlineto{\pgfqpoint{2.258915in}{8.771946in}}%
\pgfpathlineto{\pgfqpoint{2.262277in}{8.774352in}}%
\pgfpathlineto{\pgfqpoint{2.263398in}{8.779041in}}%
\pgfpathlineto{\pgfqpoint{2.265639in}{8.762383in}}%
\pgfpathlineto{\pgfqpoint{2.266759in}{8.780892in}}%
\pgfpathlineto{\pgfqpoint{2.271241in}{8.789715in}}%
\pgfpathlineto{\pgfqpoint{2.272362in}{8.795823in}}%
\pgfpathlineto{\pgfqpoint{2.273482in}{8.796748in}}%
\pgfpathlineto{\pgfqpoint{2.274603in}{8.794836in}}%
\pgfpathlineto{\pgfqpoint{2.277964in}{8.801869in}}%
\pgfpathlineto{\pgfqpoint{2.279085in}{8.793293in}}%
\pgfpathlineto{\pgfqpoint{2.280205in}{8.799216in}}%
\pgfpathlineto{\pgfqpoint{2.281326in}{8.809149in}}%
\pgfpathlineto{\pgfqpoint{2.282446in}{8.805077in}}%
\pgfpathlineto{\pgfqpoint{2.285808in}{8.797242in}}%
\pgfpathlineto{\pgfqpoint{2.286929in}{8.812543in}}%
\pgfpathlineto{\pgfqpoint{2.288049in}{8.811494in}}%
\pgfpathlineto{\pgfqpoint{2.289170in}{8.811309in}}%
\pgfpathlineto{\pgfqpoint{2.290290in}{8.815196in}}%
\pgfpathlineto{\pgfqpoint{2.293652in}{8.818034in}}%
\pgfpathlineto{\pgfqpoint{2.294772in}{8.815936in}}%
\pgfpathlineto{\pgfqpoint{2.297013in}{8.814209in}}%
\pgfpathlineto{\pgfqpoint{2.298134in}{8.825376in}}%
\pgfpathlineto{\pgfqpoint{2.301495in}{8.825006in}}%
\pgfpathlineto{\pgfqpoint{2.304857in}{8.835926in}}%
\pgfpathlineto{\pgfqpoint{2.305978in}{8.844749in}}%
\pgfpathlineto{\pgfqpoint{2.309339in}{8.842034in}}%
\pgfpathlineto{\pgfqpoint{2.310460in}{8.842404in}}%
\pgfpathlineto{\pgfqpoint{2.311580in}{8.838826in}}%
\pgfpathlineto{\pgfqpoint{2.312701in}{8.839875in}}%
\pgfpathlineto{\pgfqpoint{2.317183in}{8.850116in}}%
\pgfpathlineto{\pgfqpoint{2.318303in}{8.839504in}}%
\pgfpathlineto{\pgfqpoint{2.319424in}{8.852769in}}%
\pgfpathlineto{\pgfqpoint{2.320544in}{8.852091in}}%
\pgfpathlineto{\pgfqpoint{2.321665in}{8.857150in}}%
\pgfpathlineto{\pgfqpoint{2.325027in}{8.850795in}}%
\pgfpathlineto{\pgfqpoint{2.326147in}{8.847463in}}%
\pgfpathlineto{\pgfqpoint{2.327268in}{8.853078in}}%
\pgfpathlineto{\pgfqpoint{2.328388in}{8.855052in}}%
\pgfpathlineto{\pgfqpoint{2.329509in}{8.852091in}}%
\pgfpathlineto{\pgfqpoint{2.332870in}{8.851844in}}%
\pgfpathlineto{\pgfqpoint{2.333991in}{8.859926in}}%
\pgfpathlineto{\pgfqpoint{2.335111in}{8.863073in}}%
\pgfpathlineto{\pgfqpoint{2.336232in}{8.860667in}}%
\pgfpathlineto{\pgfqpoint{2.337352in}{8.864183in}}%
\pgfpathlineto{\pgfqpoint{2.341834in}{8.869058in}}%
\pgfpathlineto{\pgfqpoint{2.342955in}{8.865047in}}%
\pgfpathlineto{\pgfqpoint{2.344075in}{8.863505in}}%
\pgfpathlineto{\pgfqpoint{2.345196in}{8.863505in}}%
\pgfpathlineto{\pgfqpoint{2.348558in}{8.861901in}}%
\pgfpathlineto{\pgfqpoint{2.349678in}{8.849561in}}%
\pgfpathlineto{\pgfqpoint{2.350799in}{8.858446in}}%
\pgfpathlineto{\pgfqpoint{2.351919in}{8.854188in}}%
\pgfpathlineto{\pgfqpoint{2.356401in}{8.861716in}}%
\pgfpathlineto{\pgfqpoint{2.357522in}{8.859618in}}%
\pgfpathlineto{\pgfqpoint{2.358642in}{8.855052in}}%
\pgfpathlineto{\pgfqpoint{2.359763in}{8.858507in}}%
\pgfpathlineto{\pgfqpoint{2.360883in}{8.864800in}}%
\pgfpathlineto{\pgfqpoint{2.364245in}{8.862703in}}%
\pgfpathlineto{\pgfqpoint{2.365365in}{8.872513in}}%
\pgfpathlineto{\pgfqpoint{2.366486in}{8.869736in}}%
\pgfpathlineto{\pgfqpoint{2.367607in}{8.871896in}}%
\pgfpathlineto{\pgfqpoint{2.368727in}{8.861592in}}%
\pgfpathlineto{\pgfqpoint{2.372089in}{8.868502in}}%
\pgfpathlineto{\pgfqpoint{2.373209in}{8.858692in}}%
\pgfpathlineto{\pgfqpoint{2.374330in}{8.859371in}}%
\pgfpathlineto{\pgfqpoint{2.375450in}{8.849746in}}%
\pgfpathlineto{\pgfqpoint{2.376571in}{8.849129in}}%
\pgfpathlineto{\pgfqpoint{2.379932in}{8.855361in}}%
\pgfpathlineto{\pgfqpoint{2.382173in}{8.877325in}}%
\pgfpathlineto{\pgfqpoint{2.383294in}{8.871217in}}%
\pgfpathlineto{\pgfqpoint{2.384414in}{8.871094in}}%
\pgfpathlineto{\pgfqpoint{2.388897in}{8.867885in}}%
\pgfpathlineto{\pgfqpoint{2.390017in}{8.869983in}}%
\pgfpathlineto{\pgfqpoint{2.391138in}{8.866281in}}%
\pgfpathlineto{\pgfqpoint{2.392258in}{8.868194in}}%
\pgfpathlineto{\pgfqpoint{2.397861in}{8.885407in}}%
\pgfpathlineto{\pgfqpoint{2.400102in}{8.869243in}}%
\pgfpathlineto{\pgfqpoint{2.403463in}{8.861160in}}%
\pgfpathlineto{\pgfqpoint{2.404584in}{8.863875in}}%
\pgfpathlineto{\pgfqpoint{2.405704in}{8.864924in}}%
\pgfpathlineto{\pgfqpoint{2.406825in}{8.877017in}}%
\pgfpathlineto{\pgfqpoint{2.407946in}{8.871402in}}%
\pgfpathlineto{\pgfqpoint{2.411307in}{8.884359in}}%
\pgfpathlineto{\pgfqpoint{2.412428in}{8.885161in}}%
\pgfpathlineto{\pgfqpoint{2.413548in}{8.884359in}}%
\pgfpathlineto{\pgfqpoint{2.414669in}{8.900647in}}%
\pgfpathlineto{\pgfqpoint{2.415789in}{8.872389in}}%
\pgfpathlineto{\pgfqpoint{2.419151in}{8.861469in}}%
\pgfpathlineto{\pgfqpoint{2.422512in}{8.906385in}}%
\pgfpathlineto{\pgfqpoint{2.423633in}{8.908729in}}%
\pgfpathlineto{\pgfqpoint{2.428115in}{8.909223in}}%
\pgfpathlineto{\pgfqpoint{2.430356in}{8.904965in}}%
\pgfpathlineto{\pgfqpoint{2.431477in}{8.917675in}}%
\pgfpathlineto{\pgfqpoint{2.434838in}{8.922734in}}%
\pgfpathlineto{\pgfqpoint{2.435959in}{8.929213in}}%
\pgfpathlineto{\pgfqpoint{2.437079in}{8.929521in}}%
\pgfpathlineto{\pgfqpoint{2.438200in}{8.939084in}}%
\pgfpathlineto{\pgfqpoint{2.439320in}{8.941922in}}%
\pgfpathlineto{\pgfqpoint{2.442682in}{8.940256in}}%
\pgfpathlineto{\pgfqpoint{2.444923in}{8.941984in}}%
\pgfpathlineto{\pgfqpoint{2.446043in}{8.935444in}}%
\pgfpathlineto{\pgfqpoint{2.447164in}{8.936493in}}%
\pgfpathlineto{\pgfqpoint{2.450526in}{8.931866in}}%
\pgfpathlineto{\pgfqpoint{2.451646in}{8.920760in}}%
\pgfpathlineto{\pgfqpoint{2.452767in}{8.924215in}}%
\pgfpathlineto{\pgfqpoint{2.453887in}{8.922549in}}%
\pgfpathlineto{\pgfqpoint{2.455008in}{8.924709in}}%
\pgfpathlineto{\pgfqpoint{2.458369in}{8.924277in}}%
\pgfpathlineto{\pgfqpoint{2.460610in}{8.924832in}}%
\pgfpathlineto{\pgfqpoint{2.461731in}{8.921562in}}%
\pgfpathlineto{\pgfqpoint{2.462851in}{8.925881in}}%
\pgfpathlineto{\pgfqpoint{2.466213in}{8.925943in}}%
\pgfpathlineto{\pgfqpoint{2.467333in}{8.924894in}}%
\pgfpathlineto{\pgfqpoint{2.468454in}{8.927300in}}%
\pgfpathlineto{\pgfqpoint{2.469575in}{8.939639in}}%
\pgfpathlineto{\pgfqpoint{2.470695in}{8.935506in}}%
\pgfpathlineto{\pgfqpoint{2.474057in}{8.937172in}}%
\pgfpathlineto{\pgfqpoint{2.475177in}{8.929706in}}%
\pgfpathlineto{\pgfqpoint{2.476298in}{8.939578in}}%
\pgfpathlineto{\pgfqpoint{2.477418in}{8.935629in}}%
\pgfpathlineto{\pgfqpoint{2.478539in}{8.938097in}}%
\pgfpathlineto{\pgfqpoint{2.481900in}{8.935012in}}%
\pgfpathlineto{\pgfqpoint{2.483021in}{8.939022in}}%
\pgfpathlineto{\pgfqpoint{2.484141in}{8.936986in}}%
\pgfpathlineto{\pgfqpoint{2.485262in}{8.937912in}}%
\pgfpathlineto{\pgfqpoint{2.486382in}{8.936986in}}%
\pgfpathlineto{\pgfqpoint{2.489744in}{8.943156in}}%
\pgfpathlineto{\pgfqpoint{2.490865in}{8.941182in}}%
\pgfpathlineto{\pgfqpoint{2.491985in}{8.935999in}}%
\pgfpathlineto{\pgfqpoint{2.494226in}{8.944945in}}%
\pgfpathlineto{\pgfqpoint{2.498708in}{8.942909in}}%
\pgfpathlineto{\pgfqpoint{2.499829in}{8.938899in}}%
\pgfpathlineto{\pgfqpoint{2.500949in}{8.940873in}}%
\pgfpathlineto{\pgfqpoint{2.502070in}{8.915516in}}%
\pgfpathlineto{\pgfqpoint{2.505431in}{8.929521in}}%
\pgfpathlineto{\pgfqpoint{2.506552in}{8.917922in}}%
\pgfpathlineto{\pgfqpoint{2.507672in}{8.915392in}}%
\pgfpathlineto{\pgfqpoint{2.508793in}{8.920883in}}%
\pgfpathlineto{\pgfqpoint{2.509913in}{8.912184in}}%
\pgfpathlineto{\pgfqpoint{2.513275in}{8.921624in}}%
\pgfpathlineto{\pgfqpoint{2.514396in}{8.926683in}}%
\pgfpathlineto{\pgfqpoint{2.515516in}{8.937974in}}%
\pgfpathlineto{\pgfqpoint{2.516637in}{8.939454in}}%
\pgfpathlineto{\pgfqpoint{2.517757in}{8.925449in}}%
\pgfpathlineto{\pgfqpoint{2.521119in}{8.917243in}}%
\pgfpathlineto{\pgfqpoint{2.522239in}{8.919279in}}%
\pgfpathlineto{\pgfqpoint{2.523360in}{8.926621in}}%
\pgfpathlineto{\pgfqpoint{2.524480in}{8.913973in}}%
\pgfpathlineto{\pgfqpoint{2.525601in}{8.918847in}}%
\pgfpathlineto{\pgfqpoint{2.528962in}{8.912061in}}%
\pgfpathlineto{\pgfqpoint{2.530083in}{8.892934in}}%
\pgfpathlineto{\pgfqpoint{2.531204in}{8.897006in}}%
\pgfpathlineto{\pgfqpoint{2.532324in}{8.892749in}}%
\pgfpathlineto{\pgfqpoint{2.533445in}{8.890960in}}%
\pgfpathlineto{\pgfqpoint{2.536806in}{8.889850in}}%
\pgfpathlineto{\pgfqpoint{2.537927in}{8.881582in}}%
\pgfpathlineto{\pgfqpoint{2.539047in}{8.881767in}}%
\pgfpathlineto{\pgfqpoint{2.541288in}{8.885284in}}%
\pgfpathlineto{\pgfqpoint{2.544650in}{8.884605in}}%
\pgfpathlineto{\pgfqpoint{2.545770in}{8.883125in}}%
\pgfpathlineto{\pgfqpoint{2.548011in}{8.882569in}}%
\pgfpathlineto{\pgfqpoint{2.549132in}{8.880533in}}%
\pgfpathlineto{\pgfqpoint{2.552494in}{8.890590in}}%
\pgfpathlineto{\pgfqpoint{2.553614in}{8.861901in}}%
\pgfpathlineto{\pgfqpoint{2.554735in}{8.863505in}}%
\pgfpathlineto{\pgfqpoint{2.555855in}{8.859248in}}%
\pgfpathlineto{\pgfqpoint{2.556976in}{8.859309in}}%
\pgfpathlineto{\pgfqpoint{2.560337in}{8.856595in}}%
\pgfpathlineto{\pgfqpoint{2.561458in}{8.850672in}}%
\pgfpathlineto{\pgfqpoint{2.563699in}{8.865356in}}%
\pgfpathlineto{\pgfqpoint{2.564819in}{8.863320in}}%
\pgfpathlineto{\pgfqpoint{2.569301in}{8.889233in}}%
\pgfpathlineto{\pgfqpoint{2.570422in}{8.885592in}}%
\pgfpathlineto{\pgfqpoint{2.571542in}{8.907742in}}%
\pgfpathlineto{\pgfqpoint{2.572663in}{8.912307in}}%
\pgfpathlineto{\pgfqpoint{2.576025in}{8.899968in}}%
\pgfpathlineto{\pgfqpoint{2.577145in}{8.907495in}}%
\pgfpathlineto{\pgfqpoint{2.578266in}{8.901079in}}%
\pgfpathlineto{\pgfqpoint{2.579386in}{8.905459in}}%
\pgfpathlineto{\pgfqpoint{2.580507in}{8.906570in}}%
\pgfpathlineto{\pgfqpoint{2.583868in}{8.898302in}}%
\pgfpathlineto{\pgfqpoint{2.586109in}{8.902498in}}%
\pgfpathlineto{\pgfqpoint{2.588350in}{8.909593in}}%
\pgfpathlineto{\pgfqpoint{2.591712in}{8.903917in}}%
\pgfpathlineto{\pgfqpoint{2.592832in}{8.905459in}}%
\pgfpathlineto{\pgfqpoint{2.593953in}{8.899536in}}%
\pgfpathlineto{\pgfqpoint{2.595074in}{8.904657in}}%
\pgfpathlineto{\pgfqpoint{2.596194in}{8.903485in}}%
\pgfpathlineto{\pgfqpoint{2.599556in}{8.898857in}}%
\pgfpathlineto{\pgfqpoint{2.600676in}{8.899906in}}%
\pgfpathlineto{\pgfqpoint{2.601797in}{8.924277in}}%
\pgfpathlineto{\pgfqpoint{2.602917in}{8.923290in}}%
\pgfpathlineto{\pgfqpoint{2.604038in}{8.938220in}}%
\pgfpathlineto{\pgfqpoint{2.607399in}{8.944822in}}%
\pgfpathlineto{\pgfqpoint{2.608520in}{8.940195in}}%
\pgfpathlineto{\pgfqpoint{2.609640in}{8.927423in}}%
\pgfpathlineto{\pgfqpoint{2.610761in}{8.924092in}}%
\pgfpathlineto{\pgfqpoint{2.611881in}{8.932297in}}%
\pgfpathlineto{\pgfqpoint{2.615243in}{8.936369in}}%
\pgfpathlineto{\pgfqpoint{2.616364in}{8.939146in}}%
\pgfpathlineto{\pgfqpoint{2.617484in}{8.937912in}}%
\pgfpathlineto{\pgfqpoint{2.618605in}{8.942292in}}%
\pgfpathlineto{\pgfqpoint{2.619725in}{8.939701in}}%
\pgfpathlineto{\pgfqpoint{2.624207in}{8.940688in}}%
\pgfpathlineto{\pgfqpoint{2.625328in}{8.935876in}}%
\pgfpathlineto{\pgfqpoint{2.627569in}{8.938714in}}%
\pgfpathlineto{\pgfqpoint{2.632051in}{8.935691in}}%
\pgfpathlineto{\pgfqpoint{2.633171in}{8.937295in}}%
\pgfpathlineto{\pgfqpoint{2.634292in}{8.933778in}}%
\pgfpathlineto{\pgfqpoint{2.635413in}{8.936740in}}%
\pgfpathlineto{\pgfqpoint{2.638774in}{8.931249in}}%
\pgfpathlineto{\pgfqpoint{2.639895in}{8.927300in}}%
\pgfpathlineto{\pgfqpoint{2.641015in}{8.934827in}}%
\pgfpathlineto{\pgfqpoint{2.642136in}{8.932236in}}%
\pgfpathlineto{\pgfqpoint{2.647738in}{8.931187in}}%
\pgfpathlineto{\pgfqpoint{2.648859in}{8.938220in}}%
\pgfpathlineto{\pgfqpoint{2.649979in}{8.939331in}}%
\pgfpathlineto{\pgfqpoint{2.651100in}{8.938220in}}%
\pgfpathlineto{\pgfqpoint{2.654461in}{8.938344in}}%
\pgfpathlineto{\pgfqpoint{2.655582in}{8.923783in}}%
\pgfpathlineto{\pgfqpoint{2.656703in}{8.928164in}}%
\pgfpathlineto{\pgfqpoint{2.657823in}{8.928657in}}%
\pgfpathlineto{\pgfqpoint{2.658944in}{8.932421in}}%
\pgfpathlineto{\pgfqpoint{2.663426in}{8.917182in}}%
\pgfpathlineto{\pgfqpoint{2.664546in}{8.919218in}}%
\pgfpathlineto{\pgfqpoint{2.665667in}{8.913541in}}%
\pgfpathlineto{\pgfqpoint{2.666787in}{8.918477in}}%
\pgfpathlineto{\pgfqpoint{2.670149in}{8.918786in}}%
\pgfpathlineto{\pgfqpoint{2.671269in}{8.922611in}}%
\pgfpathlineto{\pgfqpoint{2.673510in}{8.936493in}}%
\pgfpathlineto{\pgfqpoint{2.674631in}{8.941182in}}%
\pgfpathlineto{\pgfqpoint{2.679113in}{8.956051in}}%
\pgfpathlineto{\pgfqpoint{2.681354in}{8.973265in}}%
\pgfpathlineto{\pgfqpoint{2.682475in}{8.970612in}}%
\pgfpathlineto{\pgfqpoint{2.686957in}{8.973018in}}%
\pgfpathlineto{\pgfqpoint{2.688077in}{8.988442in}}%
\pgfpathlineto{\pgfqpoint{2.689198in}{8.995105in}}%
\pgfpathlineto{\pgfqpoint{2.690318in}{8.996339in}}%
\pgfpathlineto{\pgfqpoint{2.693680in}{8.993440in}}%
\pgfpathlineto{\pgfqpoint{2.694800in}{8.990231in}}%
\pgfpathlineto{\pgfqpoint{2.695921in}{9.010468in}}%
\pgfpathlineto{\pgfqpoint{2.697042in}{9.010653in}}%
\pgfpathlineto{\pgfqpoint{2.698162in}{9.007322in}}%
\pgfpathlineto{\pgfqpoint{2.701524in}{9.004915in}}%
\pgfpathlineto{\pgfqpoint{2.702644in}{9.006026in}}%
\pgfpathlineto{\pgfqpoint{2.704885in}{9.010715in}}%
\pgfpathlineto{\pgfqpoint{2.706006in}{9.018242in}}%
\pgfpathlineto{\pgfqpoint{2.709367in}{9.020031in}}%
\pgfpathlineto{\pgfqpoint{2.710488in}{9.013183in}}%
\pgfpathlineto{\pgfqpoint{2.711608in}{9.018180in}}%
\pgfpathlineto{\pgfqpoint{2.712729in}{9.013059in}}%
\pgfpathlineto{\pgfqpoint{2.713849in}{9.024905in}}%
\pgfpathlineto{\pgfqpoint{2.717211in}{9.028669in}}%
\pgfpathlineto{\pgfqpoint{2.718332in}{9.023548in}}%
\pgfpathlineto{\pgfqpoint{2.719452in}{9.024103in}}%
\pgfpathlineto{\pgfqpoint{2.720573in}{9.023610in}}%
\pgfpathlineto{\pgfqpoint{2.721693in}{9.019970in}}%
\pgfpathlineto{\pgfqpoint{2.725055in}{9.013738in}}%
\pgfpathlineto{\pgfqpoint{2.726175in}{9.017008in}}%
\pgfpathlineto{\pgfqpoint{2.727296in}{9.015219in}}%
\pgfpathlineto{\pgfqpoint{2.728416in}{9.018674in}}%
\pgfpathlineto{\pgfqpoint{2.729537in}{9.018982in}}%
\pgfpathlineto{\pgfqpoint{2.732898in}{9.015466in}}%
\pgfpathlineto{\pgfqpoint{2.734019in}{9.012381in}}%
\pgfpathlineto{\pgfqpoint{2.735139in}{9.012689in}}%
\pgfpathlineto{\pgfqpoint{2.736260in}{9.010653in}}%
\pgfpathlineto{\pgfqpoint{2.737381in}{9.011208in}}%
\pgfpathlineto{\pgfqpoint{2.740742in}{9.009604in}}%
\pgfpathlineto{\pgfqpoint{2.741863in}{9.011702in}}%
\pgfpathlineto{\pgfqpoint{2.742983in}{9.009543in}}%
\pgfpathlineto{\pgfqpoint{2.744104in}{9.003496in}}%
\pgfpathlineto{\pgfqpoint{2.748586in}{9.013368in}}%
\pgfpathlineto{\pgfqpoint{2.749706in}{9.012566in}}%
\pgfpathlineto{\pgfqpoint{2.750827in}{9.010283in}}%
\pgfpathlineto{\pgfqpoint{2.751947in}{9.017995in}}%
\pgfpathlineto{\pgfqpoint{2.753068in}{9.019908in}}%
\pgfpathlineto{\pgfqpoint{2.757550in}{9.040885in}}%
\pgfpathlineto{\pgfqpoint{2.758671in}{9.040145in}}%
\pgfpathlineto{\pgfqpoint{2.759791in}{9.046499in}}%
\pgfpathlineto{\pgfqpoint{2.760912in}{9.044895in}}%
\pgfpathlineto{\pgfqpoint{2.764273in}{9.039034in}}%
\pgfpathlineto{\pgfqpoint{2.767635in}{9.066119in}}%
\pgfpathlineto{\pgfqpoint{2.768755in}{9.065502in}}%
\pgfpathlineto{\pgfqpoint{2.772117in}{9.061183in}}%
\pgfpathlineto{\pgfqpoint{2.773237in}{9.057173in}}%
\pgfpathlineto{\pgfqpoint{2.774358in}{9.049584in}}%
\pgfpathlineto{\pgfqpoint{2.775478in}{9.050201in}}%
\pgfpathlineto{\pgfqpoint{2.776599in}{9.049152in}}%
\pgfpathlineto{\pgfqpoint{2.781081in}{9.056926in}}%
\pgfpathlineto{\pgfqpoint{2.782202in}{9.045882in}}%
\pgfpathlineto{\pgfqpoint{2.784443in}{9.051497in}}%
\pgfpathlineto{\pgfqpoint{2.787804in}{9.067230in}}%
\pgfpathlineto{\pgfqpoint{2.788925in}{9.062664in}}%
\pgfpathlineto{\pgfqpoint{2.790045in}{9.061307in}}%
\pgfpathlineto{\pgfqpoint{2.792286in}{9.079754in}}%
\pgfpathlineto{\pgfqpoint{2.796768in}{9.089996in}}%
\pgfpathlineto{\pgfqpoint{2.797889in}{9.101780in}}%
\pgfpathlineto{\pgfqpoint{2.799009in}{9.101102in}}%
\pgfpathlineto{\pgfqpoint{2.800130in}{9.114737in}}%
\pgfpathlineto{\pgfqpoint{2.803492in}{9.111960in}}%
\pgfpathlineto{\pgfqpoint{2.804612in}{9.107210in}}%
\pgfpathlineto{\pgfqpoint{2.805733in}{9.104927in}}%
\pgfpathlineto{\pgfqpoint{2.807974in}{9.116032in}}%
\pgfpathlineto{\pgfqpoint{2.811335in}{9.118747in}}%
\pgfpathlineto{\pgfqpoint{2.813576in}{9.133925in}}%
\pgfpathlineto{\pgfqpoint{2.814697in}{9.141328in}}%
\pgfpathlineto{\pgfqpoint{2.815817in}{9.152681in}}%
\pgfpathlineto{\pgfqpoint{2.820300in}{9.153359in}}%
\pgfpathlineto{\pgfqpoint{2.822541in}{9.146758in}}%
\pgfpathlineto{\pgfqpoint{2.823661in}{9.150706in}}%
\pgfpathlineto{\pgfqpoint{2.827023in}{9.148979in}}%
\pgfpathlineto{\pgfqpoint{2.828143in}{9.132567in}}%
\pgfpathlineto{\pgfqpoint{2.829264in}{9.137441in}}%
\pgfpathlineto{\pgfqpoint{2.830384in}{9.121400in}}%
\pgfpathlineto{\pgfqpoint{2.831505in}{9.123374in}}%
\pgfpathlineto{\pgfqpoint{2.834866in}{9.132876in}}%
\pgfpathlineto{\pgfqpoint{2.837107in}{9.132444in}}%
\pgfpathlineto{\pgfqpoint{2.838228in}{9.122387in}}%
\pgfpathlineto{\pgfqpoint{2.839348in}{9.131457in}}%
\pgfpathlineto{\pgfqpoint{2.842710in}{9.136701in}}%
\pgfpathlineto{\pgfqpoint{2.843831in}{9.131889in}}%
\pgfpathlineto{\pgfqpoint{2.844951in}{9.141390in}}%
\pgfpathlineto{\pgfqpoint{2.846072in}{9.140218in}}%
\pgfpathlineto{\pgfqpoint{2.847192in}{9.144167in}}%
\pgfpathlineto{\pgfqpoint{2.850554in}{9.143611in}}%
\pgfpathlineto{\pgfqpoint{2.851674in}{9.141452in}}%
\pgfpathlineto{\pgfqpoint{2.852795in}{9.146079in}}%
\pgfpathlineto{\pgfqpoint{2.853915in}{9.148053in}}%
\pgfpathlineto{\pgfqpoint{2.855036in}{9.140588in}}%
\pgfpathlineto{\pgfqpoint{2.858397in}{9.133863in}}%
\pgfpathlineto{\pgfqpoint{2.859518in}{9.072289in}}%
\pgfpathlineto{\pgfqpoint{2.860638in}{9.070191in}}%
\pgfpathlineto{\pgfqpoint{2.861759in}{9.076114in}}%
\pgfpathlineto{\pgfqpoint{2.862880in}{9.074202in}}%
\pgfpathlineto{\pgfqpoint{2.866241in}{9.082654in}}%
\pgfpathlineto{\pgfqpoint{2.869603in}{9.120043in}}%
\pgfpathlineto{\pgfqpoint{2.870723in}{9.120228in}}%
\pgfpathlineto{\pgfqpoint{2.874085in}{9.118994in}}%
\pgfpathlineto{\pgfqpoint{2.875205in}{9.113133in}}%
\pgfpathlineto{\pgfqpoint{2.876326in}{9.113441in}}%
\pgfpathlineto{\pgfqpoint{2.878567in}{9.110541in}}%
\pgfpathlineto{\pgfqpoint{2.881929in}{9.118624in}}%
\pgfpathlineto{\pgfqpoint{2.883049in}{9.117513in}}%
\pgfpathlineto{\pgfqpoint{2.884170in}{9.121770in}}%
\pgfpathlineto{\pgfqpoint{2.886411in}{9.096351in}}%
\pgfpathlineto{\pgfqpoint{2.889772in}{9.102274in}}%
\pgfpathlineto{\pgfqpoint{2.890893in}{9.107518in}}%
\pgfpathlineto{\pgfqpoint{2.892013in}{9.098079in}}%
\pgfpathlineto{\pgfqpoint{2.893134in}{9.094870in}}%
\pgfpathlineto{\pgfqpoint{2.894254in}{9.094932in}}%
\pgfpathlineto{\pgfqpoint{2.897616in}{9.096906in}}%
\pgfpathlineto{\pgfqpoint{2.898736in}{9.099559in}}%
\pgfpathlineto{\pgfqpoint{2.900977in}{9.107703in}}%
\pgfpathlineto{\pgfqpoint{2.902098in}{9.103323in}}%
\pgfpathlineto{\pgfqpoint{2.906580in}{9.088269in}}%
\pgfpathlineto{\pgfqpoint{2.907701in}{9.094500in}}%
\pgfpathlineto{\pgfqpoint{2.909942in}{9.115724in}}%
\pgfpathlineto{\pgfqpoint{2.913303in}{9.138305in}}%
\pgfpathlineto{\pgfqpoint{2.914424in}{9.138799in}}%
\pgfpathlineto{\pgfqpoint{2.915544in}{9.137873in}}%
\pgfpathlineto{\pgfqpoint{2.917785in}{9.160393in}}%
\pgfpathlineto{\pgfqpoint{2.921147in}{9.162799in}}%
\pgfpathlineto{\pgfqpoint{2.922267in}{9.161627in}}%
\pgfpathlineto{\pgfqpoint{2.923388in}{9.143858in}}%
\pgfpathlineto{\pgfqpoint{2.924509in}{9.143426in}}%
\pgfpathlineto{\pgfqpoint{2.925629in}{9.145339in}}%
\pgfpathlineto{\pgfqpoint{2.928991in}{9.145154in}}%
\pgfpathlineto{\pgfqpoint{2.930111in}{9.146881in}}%
\pgfpathlineto{\pgfqpoint{2.931232in}{9.137812in}}%
\pgfpathlineto{\pgfqpoint{2.932352in}{9.138244in}}%
\pgfpathlineto{\pgfqpoint{2.933473in}{9.140280in}}%
\pgfpathlineto{\pgfqpoint{2.936834in}{9.157000in}}%
\pgfpathlineto{\pgfqpoint{2.939075in}{9.178902in}}%
\pgfpathlineto{\pgfqpoint{2.940196in}{9.178038in}}%
\pgfpathlineto{\pgfqpoint{2.941316in}{9.178902in}}%
\pgfpathlineto{\pgfqpoint{2.945799in}{9.180260in}}%
\pgfpathlineto{\pgfqpoint{2.946919in}{9.178840in}}%
\pgfpathlineto{\pgfqpoint{2.948040in}{9.185134in}}%
\pgfpathlineto{\pgfqpoint{2.949160in}{9.185936in}}%
\pgfpathlineto{\pgfqpoint{2.952522in}{9.191735in}}%
\pgfpathlineto{\pgfqpoint{2.953642in}{9.186121in}}%
\pgfpathlineto{\pgfqpoint{2.954763in}{9.189144in}}%
\pgfpathlineto{\pgfqpoint{2.955883in}{9.194820in}}%
\pgfpathlineto{\pgfqpoint{2.957004in}{9.206913in}}%
\pgfpathlineto{\pgfqpoint{2.960365in}{9.208270in}}%
\pgfpathlineto{\pgfqpoint{2.961486in}{9.284713in}}%
\pgfpathlineto{\pgfqpoint{2.962606in}{9.302420in}}%
\pgfpathlineto{\pgfqpoint{2.963727in}{9.274718in}}%
\pgfpathlineto{\pgfqpoint{2.964848in}{9.285269in}}%
\pgfpathlineto{\pgfqpoint{2.969330in}{9.258677in}}%
\pgfpathlineto{\pgfqpoint{2.970450in}{9.258615in}}%
\pgfpathlineto{\pgfqpoint{2.971571in}{9.270585in}}%
\pgfpathlineto{\pgfqpoint{2.972691in}{9.270523in}}%
\pgfpathlineto{\pgfqpoint{2.976053in}{9.259417in}}%
\pgfpathlineto{\pgfqpoint{2.978294in}{9.256579in}}%
\pgfpathlineto{\pgfqpoint{2.980535in}{9.242697in}}%
\pgfpathlineto{\pgfqpoint{2.983896in}{9.247201in}}%
\pgfpathlineto{\pgfqpoint{2.985017in}{9.253679in}}%
\pgfpathlineto{\pgfqpoint{2.986138in}{9.242389in}}%
\pgfpathlineto{\pgfqpoint{2.987258in}{9.254173in}}%
\pgfpathlineto{\pgfqpoint{2.988379in}{9.253865in}}%
\pgfpathlineto{\pgfqpoint{2.991740in}{9.266266in}}%
\pgfpathlineto{\pgfqpoint{2.992861in}{9.281443in}}%
\pgfpathlineto{\pgfqpoint{2.993981in}{9.273608in}}%
\pgfpathlineto{\pgfqpoint{2.996222in}{9.272436in}}%
\pgfpathlineto{\pgfqpoint{2.999584in}{9.287798in}}%
\pgfpathlineto{\pgfqpoint{3.001825in}{9.311305in}}%
\pgfpathlineto{\pgfqpoint{3.002945in}{9.341413in}}%
\pgfpathlineto{\pgfqpoint{3.004066in}{9.329752in}}%
\pgfpathlineto{\pgfqpoint{3.007428in}{9.318647in}}%
\pgfpathlineto{\pgfqpoint{3.008548in}{9.312786in}}%
\pgfpathlineto{\pgfqpoint{3.009669in}{9.315192in}}%
\pgfpathlineto{\pgfqpoint{3.010789in}{9.324755in}}%
\pgfpathlineto{\pgfqpoint{3.011910in}{9.312045in}}%
\pgfpathlineto{\pgfqpoint{3.015271in}{9.318894in}}%
\pgfpathlineto{\pgfqpoint{3.016392in}{9.302914in}}%
\pgfpathlineto{\pgfqpoint{3.017512in}{9.317783in}}%
\pgfpathlineto{\pgfqpoint{3.018633in}{9.311737in}}%
\pgfpathlineto{\pgfqpoint{3.019753in}{9.311243in}}%
\pgfpathlineto{\pgfqpoint{3.023115in}{9.313711in}}%
\pgfpathlineto{\pgfqpoint{3.024235in}{9.313526in}}%
\pgfpathlineto{\pgfqpoint{3.026477in}{9.292364in}}%
\pgfpathlineto{\pgfqpoint{3.027597in}{9.292117in}}%
\pgfpathlineto{\pgfqpoint{3.032079in}{9.296312in}}%
\pgfpathlineto{\pgfqpoint{3.033200in}{9.300693in}}%
\pgfpathlineto{\pgfqpoint{3.035441in}{9.295819in}}%
\pgfpathlineto{\pgfqpoint{3.035441in}{9.295819in}}%
\pgfusepath{stroke}%
\end{pgfscope}%
\begin{pgfscope}%
\pgfpathrectangle{\pgfqpoint{0.462318in}{8.286757in}}{\pgfqpoint{2.695652in}{1.104878in}}%
\pgfusepath{clip}%
\pgfsetroundcap%
\pgfsetroundjoin%
\pgfsetlinewidth{1.505625pt}%
\definecolor{currentstroke}{rgb}{1.000000,0.647059,0.000000}%
\pgfsetstrokecolor{currentstroke}%
\pgfsetdash{}{0pt}%
\pgfpathmoveto{\pgfqpoint{0.584848in}{8.337534in}}%
\pgfpathlineto{\pgfqpoint{0.585968in}{8.339292in}}%
\pgfpathlineto{\pgfqpoint{0.593812in}{8.339261in}}%
\pgfpathlineto{\pgfqpoint{0.596053in}{8.339385in}}%
\pgfpathlineto{\pgfqpoint{0.601656in}{8.340120in}}%
\pgfpathlineto{\pgfqpoint{0.603897in}{8.341464in}}%
\pgfpathlineto{\pgfqpoint{0.608379in}{8.342478in}}%
\pgfpathlineto{\pgfqpoint{0.611740in}{8.344763in}}%
\pgfpathlineto{\pgfqpoint{0.617343in}{8.346366in}}%
\pgfpathlineto{\pgfqpoint{0.619584in}{8.347398in}}%
\pgfpathlineto{\pgfqpoint{0.624066in}{8.348324in}}%
\pgfpathlineto{\pgfqpoint{0.627428in}{8.349459in}}%
\pgfpathlineto{\pgfqpoint{0.633031in}{8.350451in}}%
\pgfpathlineto{\pgfqpoint{0.635272in}{8.351111in}}%
\pgfpathlineto{\pgfqpoint{0.641995in}{8.352028in}}%
\pgfpathlineto{\pgfqpoint{0.643115in}{8.352361in}}%
\pgfpathlineto{\pgfqpoint{0.670008in}{8.355290in}}%
\pgfpathlineto{\pgfqpoint{0.674490in}{8.356151in}}%
\pgfpathlineto{\pgfqpoint{0.698021in}{8.357189in}}%
\pgfpathlineto{\pgfqpoint{0.711467in}{8.357536in}}%
\pgfpathlineto{\pgfqpoint{0.721552in}{8.358557in}}%
\pgfpathlineto{\pgfqpoint{0.736119in}{8.358326in}}%
\pgfpathlineto{\pgfqpoint{0.742842in}{8.357873in}}%
\pgfpathlineto{\pgfqpoint{0.767494in}{8.356906in}}%
\pgfpathlineto{\pgfqpoint{0.829123in}{8.360639in}}%
\pgfpathlineto{\pgfqpoint{0.854895in}{8.363775in}}%
\pgfpathlineto{\pgfqpoint{0.868341in}{8.364763in}}%
\pgfpathlineto{\pgfqpoint{0.886270in}{8.366758in}}%
\pgfpathlineto{\pgfqpoint{0.897475in}{8.367831in}}%
\pgfpathlineto{\pgfqpoint{0.909801in}{8.369176in}}%
\pgfpathlineto{\pgfqpoint{0.937814in}{8.369584in}}%
\pgfpathlineto{\pgfqpoint{0.953501in}{8.369870in}}%
\pgfpathlineto{\pgfqpoint{1.017371in}{8.374718in}}%
\pgfpathlineto{\pgfqpoint{1.027456in}{8.376368in}}%
\pgfpathlineto{\pgfqpoint{1.034179in}{8.377347in}}%
\pgfpathlineto{\pgfqpoint{1.043143in}{8.378926in}}%
\pgfpathlineto{\pgfqpoint{1.049867in}{8.379730in}}%
\pgfpathlineto{\pgfqpoint{1.054349in}{8.380231in}}%
\pgfpathlineto{\pgfqpoint{1.066675in}{8.382624in}}%
\pgfpathlineto{\pgfqpoint{1.073398in}{8.383745in}}%
\pgfpathlineto{\pgfqpoint{1.074518in}{8.384035in}}%
\pgfpathlineto{\pgfqpoint{1.081241in}{8.385104in}}%
\pgfpathlineto{\pgfqpoint{1.082362in}{8.385386in}}%
\pgfpathlineto{\pgfqpoint{1.093567in}{8.386708in}}%
\pgfpathlineto{\pgfqpoint{1.098049in}{8.387752in}}%
\pgfpathlineto{\pgfqpoint{1.103652in}{8.388544in}}%
\pgfpathlineto{\pgfqpoint{1.105893in}{8.389112in}}%
\pgfpathlineto{\pgfqpoint{1.112616in}{8.390074in}}%
\pgfpathlineto{\pgfqpoint{1.121580in}{8.391504in}}%
\pgfpathlineto{\pgfqpoint{1.128304in}{8.392358in}}%
\pgfpathlineto{\pgfqpoint{1.137268in}{8.393964in}}%
\pgfpathlineto{\pgfqpoint{1.142870in}{8.394845in}}%
\pgfpathlineto{\pgfqpoint{1.145111in}{8.395434in}}%
\pgfpathlineto{\pgfqpoint{1.150714in}{8.396327in}}%
\pgfpathlineto{\pgfqpoint{1.152955in}{8.396892in}}%
\pgfpathlineto{\pgfqpoint{1.159678in}{8.397774in}}%
\pgfpathlineto{\pgfqpoint{1.165281in}{8.398593in}}%
\pgfpathlineto{\pgfqpoint{1.176486in}{8.400686in}}%
\pgfpathlineto{\pgfqpoint{1.183209in}{8.401768in}}%
\pgfpathlineto{\pgfqpoint{1.192174in}{8.403162in}}%
\pgfpathlineto{\pgfqpoint{1.203379in}{8.404368in}}%
\pgfpathlineto{\pgfqpoint{1.215705in}{8.407001in}}%
\pgfpathlineto{\pgfqpoint{1.221307in}{8.407924in}}%
\pgfpathlineto{\pgfqpoint{1.223548in}{8.408538in}}%
\pgfpathlineto{\pgfqpoint{1.229151in}{8.409458in}}%
\pgfpathlineto{\pgfqpoint{1.231392in}{8.410100in}}%
\pgfpathlineto{\pgfqpoint{1.236995in}{8.411028in}}%
\pgfpathlineto{\pgfqpoint{1.239236in}{8.411658in}}%
\pgfpathlineto{\pgfqpoint{1.244838in}{8.412581in}}%
\pgfpathlineto{\pgfqpoint{1.251561in}{8.413661in}}%
\pgfpathlineto{\pgfqpoint{1.262767in}{8.415620in}}%
\pgfpathlineto{\pgfqpoint{1.269490in}{8.416355in}}%
\pgfpathlineto{\pgfqpoint{1.278454in}{8.418038in}}%
\pgfpathlineto{\pgfqpoint{1.284057in}{8.418960in}}%
\pgfpathlineto{\pgfqpoint{1.286298in}{8.419583in}}%
\pgfpathlineto{\pgfqpoint{1.291900in}{8.420504in}}%
\pgfpathlineto{\pgfqpoint{1.294142in}{8.421104in}}%
\pgfpathlineto{\pgfqpoint{1.300865in}{8.422229in}}%
\pgfpathlineto{\pgfqpoint{1.305347in}{8.422781in}}%
\pgfpathlineto{\pgfqpoint{1.317673in}{8.425330in}}%
\pgfpathlineto{\pgfqpoint{1.323275in}{8.426267in}}%
\pgfpathlineto{\pgfqpoint{1.325516in}{8.426897in}}%
\pgfpathlineto{\pgfqpoint{1.331119in}{8.427868in}}%
\pgfpathlineto{\pgfqpoint{1.333360in}{8.428529in}}%
\pgfpathlineto{\pgfqpoint{1.338963in}{8.429531in}}%
\pgfpathlineto{\pgfqpoint{1.341204in}{8.430207in}}%
\pgfpathlineto{\pgfqpoint{1.346806in}{8.431248in}}%
\pgfpathlineto{\pgfqpoint{1.349047in}{8.431968in}}%
\pgfpathlineto{\pgfqpoint{1.354650in}{8.433043in}}%
\pgfpathlineto{\pgfqpoint{1.356891in}{8.433782in}}%
\pgfpathlineto{\pgfqpoint{1.362494in}{8.434932in}}%
\pgfpathlineto{\pgfqpoint{1.386025in}{8.439559in}}%
\pgfpathlineto{\pgfqpoint{1.388266in}{8.440369in}}%
\pgfpathlineto{\pgfqpoint{1.394989in}{8.441597in}}%
\pgfpathlineto{\pgfqpoint{1.396109in}{8.442024in}}%
\pgfpathlineto{\pgfqpoint{1.402833in}{8.443293in}}%
\pgfpathlineto{\pgfqpoint{1.403953in}{8.443704in}}%
\pgfpathlineto{\pgfqpoint{1.409556in}{8.444891in}}%
\pgfpathlineto{\pgfqpoint{1.411797in}{8.445655in}}%
\pgfpathlineto{\pgfqpoint{1.417400in}{8.446805in}}%
\pgfpathlineto{\pgfqpoint{1.419641in}{8.447582in}}%
\pgfpathlineto{\pgfqpoint{1.426364in}{8.448686in}}%
\pgfpathlineto{\pgfqpoint{1.427484in}{8.448991in}}%
\pgfpathlineto{\pgfqpoint{1.433087in}{8.449883in}}%
\pgfpathlineto{\pgfqpoint{1.435328in}{8.450440in}}%
\pgfpathlineto{\pgfqpoint{1.440931in}{8.451204in}}%
\pgfpathlineto{\pgfqpoint{1.451015in}{8.453283in}}%
\pgfpathlineto{\pgfqpoint{1.457738in}{8.454200in}}%
\pgfpathlineto{\pgfqpoint{1.458859in}{8.454506in}}%
\pgfpathlineto{\pgfqpoint{1.464462in}{8.455449in}}%
\pgfpathlineto{\pgfqpoint{1.466703in}{8.456110in}}%
\pgfpathlineto{\pgfqpoint{1.472305in}{8.457043in}}%
\pgfpathlineto{\pgfqpoint{1.474546in}{8.457683in}}%
\pgfpathlineto{\pgfqpoint{1.480149in}{8.458599in}}%
\pgfpathlineto{\pgfqpoint{1.490234in}{8.460622in}}%
\pgfpathlineto{\pgfqpoint{1.495836in}{8.461515in}}%
\pgfpathlineto{\pgfqpoint{1.505921in}{8.463700in}}%
\pgfpathlineto{\pgfqpoint{1.511524in}{8.464612in}}%
\pgfpathlineto{\pgfqpoint{1.513765in}{8.465180in}}%
\pgfpathlineto{\pgfqpoint{1.519367in}{8.466067in}}%
\pgfpathlineto{\pgfqpoint{1.520488in}{8.466389in}}%
\pgfpathlineto{\pgfqpoint{1.527211in}{8.467369in}}%
\pgfpathlineto{\pgfqpoint{1.529452in}{8.467982in}}%
\pgfpathlineto{\pgfqpoint{1.535055in}{8.468931in}}%
\pgfpathlineto{\pgfqpoint{1.537296in}{8.469603in}}%
\pgfpathlineto{\pgfqpoint{1.542899in}{8.470601in}}%
\pgfpathlineto{\pgfqpoint{1.545140in}{8.471273in}}%
\pgfpathlineto{\pgfqpoint{1.550742in}{8.472306in}}%
\pgfpathlineto{\pgfqpoint{1.552983in}{8.472967in}}%
\pgfpathlineto{\pgfqpoint{1.558586in}{8.473952in}}%
\pgfpathlineto{\pgfqpoint{1.560827in}{8.474611in}}%
\pgfpathlineto{\pgfqpoint{1.567550in}{8.475619in}}%
\pgfpathlineto{\pgfqpoint{1.568671in}{8.475960in}}%
\pgfpathlineto{\pgfqpoint{1.574273in}{8.476975in}}%
\pgfpathlineto{\pgfqpoint{1.576514in}{8.477675in}}%
\pgfpathlineto{\pgfqpoint{1.582117in}{8.478736in}}%
\pgfpathlineto{\pgfqpoint{1.584358in}{8.479408in}}%
\pgfpathlineto{\pgfqpoint{1.589961in}{8.480423in}}%
\pgfpathlineto{\pgfqpoint{1.592202in}{8.481112in}}%
\pgfpathlineto{\pgfqpoint{1.597804in}{8.482107in}}%
\pgfpathlineto{\pgfqpoint{1.600045in}{8.482766in}}%
\pgfpathlineto{\pgfqpoint{1.605648in}{8.483763in}}%
\pgfpathlineto{\pgfqpoint{1.606769in}{8.484102in}}%
\pgfpathlineto{\pgfqpoint{1.613492in}{8.485096in}}%
\pgfpathlineto{\pgfqpoint{1.615733in}{8.485743in}}%
\pgfpathlineto{\pgfqpoint{1.621335in}{8.486740in}}%
\pgfpathlineto{\pgfqpoint{1.623577in}{8.487378in}}%
\pgfpathlineto{\pgfqpoint{1.629179in}{8.488341in}}%
\pgfpathlineto{\pgfqpoint{1.631420in}{8.488985in}}%
\pgfpathlineto{\pgfqpoint{1.637023in}{8.489926in}}%
\pgfpathlineto{\pgfqpoint{1.639264in}{8.490483in}}%
\pgfpathlineto{\pgfqpoint{1.645987in}{8.491565in}}%
\pgfpathlineto{\pgfqpoint{1.654951in}{8.493241in}}%
\pgfpathlineto{\pgfqpoint{1.660554in}{8.494147in}}%
\pgfpathlineto{\pgfqpoint{1.662795in}{8.494755in}}%
\pgfpathlineto{\pgfqpoint{1.668398in}{8.495662in}}%
\pgfpathlineto{\pgfqpoint{1.670639in}{8.496256in}}%
\pgfpathlineto{\pgfqpoint{1.677362in}{8.497137in}}%
\pgfpathlineto{\pgfqpoint{1.678482in}{8.497431in}}%
\pgfpathlineto{\pgfqpoint{1.684085in}{8.498321in}}%
\pgfpathlineto{\pgfqpoint{1.686326in}{8.498901in}}%
\pgfpathlineto{\pgfqpoint{1.691929in}{8.499783in}}%
\pgfpathlineto{\pgfqpoint{1.694170in}{8.500397in}}%
\pgfpathlineto{\pgfqpoint{1.700893in}{8.501537in}}%
\pgfpathlineto{\pgfqpoint{1.706496in}{8.502330in}}%
\pgfpathlineto{\pgfqpoint{1.717701in}{8.504204in}}%
\pgfpathlineto{\pgfqpoint{1.724424in}{8.504982in}}%
\pgfpathlineto{\pgfqpoint{1.733388in}{8.506468in}}%
\pgfpathlineto{\pgfqpoint{1.738991in}{8.507409in}}%
\pgfpathlineto{\pgfqpoint{1.741232in}{8.508069in}}%
\pgfpathlineto{\pgfqpoint{1.746834in}{8.509091in}}%
\pgfpathlineto{\pgfqpoint{1.749076in}{8.509792in}}%
\pgfpathlineto{\pgfqpoint{1.754678in}{8.510864in}}%
\pgfpathlineto{\pgfqpoint{1.756919in}{8.511589in}}%
\pgfpathlineto{\pgfqpoint{1.762522in}{8.512691in}}%
\pgfpathlineto{\pgfqpoint{1.764763in}{8.513437in}}%
\pgfpathlineto{\pgfqpoint{1.770366in}{8.514525in}}%
\pgfpathlineto{\pgfqpoint{1.779330in}{8.516391in}}%
\pgfpathlineto{\pgfqpoint{1.780450in}{8.516772in}}%
\pgfpathlineto{\pgfqpoint{1.786053in}{8.517862in}}%
\pgfpathlineto{\pgfqpoint{1.788294in}{8.518556in}}%
\pgfpathlineto{\pgfqpoint{1.793897in}{8.519605in}}%
\pgfpathlineto{\pgfqpoint{1.796138in}{8.520395in}}%
\pgfpathlineto{\pgfqpoint{1.801740in}{8.521608in}}%
\pgfpathlineto{\pgfqpoint{1.825271in}{8.526335in}}%
\pgfpathlineto{\pgfqpoint{1.827512in}{8.527024in}}%
\pgfpathlineto{\pgfqpoint{1.834236in}{8.528107in}}%
\pgfpathlineto{\pgfqpoint{1.835356in}{8.528470in}}%
\pgfpathlineto{\pgfqpoint{1.840959in}{8.529552in}}%
\pgfpathlineto{\pgfqpoint{1.843200in}{8.530273in}}%
\pgfpathlineto{\pgfqpoint{1.848802in}{8.531364in}}%
\pgfpathlineto{\pgfqpoint{1.851044in}{8.532105in}}%
\pgfpathlineto{\pgfqpoint{1.856646in}{8.533189in}}%
\pgfpathlineto{\pgfqpoint{1.858887in}{8.533929in}}%
\pgfpathlineto{\pgfqpoint{1.864490in}{8.534682in}}%
\pgfpathlineto{\pgfqpoint{1.866731in}{8.535438in}}%
\pgfpathlineto{\pgfqpoint{1.872334in}{8.536593in}}%
\pgfpathlineto{\pgfqpoint{1.874575in}{8.537360in}}%
\pgfpathlineto{\pgfqpoint{1.880177in}{8.538494in}}%
\pgfpathlineto{\pgfqpoint{1.882418in}{8.539208in}}%
\pgfpathlineto{\pgfqpoint{1.888021in}{8.540221in}}%
\pgfpathlineto{\pgfqpoint{1.890262in}{8.540895in}}%
\pgfpathlineto{\pgfqpoint{1.895865in}{8.541940in}}%
\pgfpathlineto{\pgfqpoint{1.898106in}{8.542638in}}%
\pgfpathlineto{\pgfqpoint{1.903708in}{8.543659in}}%
\pgfpathlineto{\pgfqpoint{1.905949in}{8.544308in}}%
\pgfpathlineto{\pgfqpoint{1.911552in}{8.545305in}}%
\pgfpathlineto{\pgfqpoint{1.912673in}{8.545624in}}%
\pgfpathlineto{\pgfqpoint{1.919396in}{8.546641in}}%
\pgfpathlineto{\pgfqpoint{1.921637in}{8.547328in}}%
\pgfpathlineto{\pgfqpoint{1.927239in}{8.548335in}}%
\pgfpathlineto{\pgfqpoint{1.929480in}{8.548971in}}%
\pgfpathlineto{\pgfqpoint{1.935083in}{8.549933in}}%
\pgfpathlineto{\pgfqpoint{1.937324in}{8.550503in}}%
\pgfpathlineto{\pgfqpoint{1.944047in}{8.551586in}}%
\pgfpathlineto{\pgfqpoint{1.953011in}{8.553235in}}%
\pgfpathlineto{\pgfqpoint{1.958614in}{8.554087in}}%
\pgfpathlineto{\pgfqpoint{1.960855in}{8.554688in}}%
\pgfpathlineto{\pgfqpoint{1.966458in}{8.555579in}}%
\pgfpathlineto{\pgfqpoint{1.968699in}{8.556164in}}%
\pgfpathlineto{\pgfqpoint{1.975422in}{8.557016in}}%
\pgfpathlineto{\pgfqpoint{1.984386in}{8.558638in}}%
\pgfpathlineto{\pgfqpoint{1.991109in}{8.559689in}}%
\pgfpathlineto{\pgfqpoint{2.000074in}{8.561231in}}%
\pgfpathlineto{\pgfqpoint{2.006797in}{8.562262in}}%
\pgfpathlineto{\pgfqpoint{2.014640in}{8.563445in}}%
\pgfpathlineto{\pgfqpoint{2.022484in}{8.564355in}}%
\pgfpathlineto{\pgfqpoint{2.031448in}{8.565784in}}%
\pgfpathlineto{\pgfqpoint{2.042654in}{8.567057in}}%
\pgfpathlineto{\pgfqpoint{2.054979in}{8.568821in}}%
\pgfpathlineto{\pgfqpoint{2.067305in}{8.570094in}}%
\pgfpathlineto{\pgfqpoint{2.078511in}{8.571262in}}%
\pgfpathlineto{\pgfqpoint{2.097559in}{8.572620in}}%
\pgfpathlineto{\pgfqpoint{2.109885in}{8.573779in}}%
\pgfpathlineto{\pgfqpoint{2.122211in}{8.574756in}}%
\pgfpathlineto{\pgfqpoint{2.133416in}{8.576156in}}%
\pgfpathlineto{\pgfqpoint{2.141260in}{8.577098in}}%
\pgfpathlineto{\pgfqpoint{2.147983in}{8.577971in}}%
\pgfpathlineto{\pgfqpoint{2.156947in}{8.579322in}}%
\pgfpathlineto{\pgfqpoint{2.163671in}{8.580185in}}%
\pgfpathlineto{\pgfqpoint{2.172635in}{8.581483in}}%
\pgfpathlineto{\pgfqpoint{2.184961in}{8.582783in}}%
\pgfpathlineto{\pgfqpoint{2.196166in}{8.584439in}}%
\pgfpathlineto{\pgfqpoint{2.209612in}{8.585778in}}%
\pgfpathlineto{\pgfqpoint{2.218576in}{8.586650in}}%
\pgfpathlineto{\pgfqpoint{2.233143in}{8.587630in}}%
\pgfpathlineto{\pgfqpoint{2.243228in}{8.588231in}}%
\pgfpathlineto{\pgfqpoint{2.254433in}{8.589056in}}%
\pgfpathlineto{\pgfqpoint{2.282446in}{8.592411in}}%
\pgfpathlineto{\pgfqpoint{2.289170in}{8.593234in}}%
\pgfpathlineto{\pgfqpoint{2.298134in}{8.594510in}}%
\pgfpathlineto{\pgfqpoint{2.304857in}{8.595401in}}%
\pgfpathlineto{\pgfqpoint{2.312701in}{8.596559in}}%
\pgfpathlineto{\pgfqpoint{2.320544in}{8.597504in}}%
\pgfpathlineto{\pgfqpoint{2.329509in}{8.598930in}}%
\pgfpathlineto{\pgfqpoint{2.336232in}{8.599896in}}%
\pgfpathlineto{\pgfqpoint{2.345196in}{8.601368in}}%
\pgfpathlineto{\pgfqpoint{2.351919in}{8.602305in}}%
\pgfpathlineto{\pgfqpoint{2.360883in}{8.603716in}}%
\pgfpathlineto{\pgfqpoint{2.367607in}{8.604684in}}%
\pgfpathlineto{\pgfqpoint{2.376571in}{8.606061in}}%
\pgfpathlineto{\pgfqpoint{2.383294in}{8.607009in}}%
\pgfpathlineto{\pgfqpoint{2.384414in}{8.607247in}}%
\pgfpathlineto{\pgfqpoint{2.392258in}{8.608186in}}%
\pgfpathlineto{\pgfqpoint{2.398981in}{8.609163in}}%
\pgfpathlineto{\pgfqpoint{2.407946in}{8.610547in}}%
\pgfpathlineto{\pgfqpoint{2.414669in}{8.611535in}}%
\pgfpathlineto{\pgfqpoint{2.423633in}{8.612984in}}%
\pgfpathlineto{\pgfqpoint{2.430356in}{8.613761in}}%
\pgfpathlineto{\pgfqpoint{2.439320in}{8.615424in}}%
\pgfpathlineto{\pgfqpoint{2.444923in}{8.616278in}}%
\pgfpathlineto{\pgfqpoint{2.450526in}{8.617111in}}%
\pgfpathlineto{\pgfqpoint{2.462851in}{8.619498in}}%
\pgfpathlineto{\pgfqpoint{2.469575in}{8.620567in}}%
\pgfpathlineto{\pgfqpoint{2.478539in}{8.622190in}}%
\pgfpathlineto{\pgfqpoint{2.485262in}{8.623267in}}%
\pgfpathlineto{\pgfqpoint{2.494226in}{8.624885in}}%
\pgfpathlineto{\pgfqpoint{2.500949in}{8.625689in}}%
\pgfpathlineto{\pgfqpoint{2.509913in}{8.627172in}}%
\pgfpathlineto{\pgfqpoint{2.516637in}{8.628196in}}%
\pgfpathlineto{\pgfqpoint{2.525601in}{8.629662in}}%
\pgfpathlineto{\pgfqpoint{2.532324in}{8.630559in}}%
\pgfpathlineto{\pgfqpoint{2.541288in}{8.631829in}}%
\pgfpathlineto{\pgfqpoint{2.552494in}{8.633078in}}%
\pgfpathlineto{\pgfqpoint{2.564819in}{8.634752in}}%
\pgfpathlineto{\pgfqpoint{2.571542in}{8.635589in}}%
\pgfpathlineto{\pgfqpoint{2.580507in}{8.636906in}}%
\pgfpathlineto{\pgfqpoint{2.592832in}{8.638199in}}%
\pgfpathlineto{\pgfqpoint{2.604038in}{8.639957in}}%
\pgfpathlineto{\pgfqpoint{2.610761in}{8.640900in}}%
\pgfpathlineto{\pgfqpoint{2.619725in}{8.642321in}}%
\pgfpathlineto{\pgfqpoint{2.632051in}{8.643495in}}%
\pgfpathlineto{\pgfqpoint{2.643256in}{8.645324in}}%
\pgfpathlineto{\pgfqpoint{2.651100in}{8.646241in}}%
\pgfpathlineto{\pgfqpoint{2.657823in}{8.647130in}}%
\pgfpathlineto{\pgfqpoint{2.666787in}{8.648409in}}%
\pgfpathlineto{\pgfqpoint{2.673510in}{8.649277in}}%
\pgfpathlineto{\pgfqpoint{2.682475in}{8.650719in}}%
\pgfpathlineto{\pgfqpoint{2.689198in}{8.651495in}}%
\pgfpathlineto{\pgfqpoint{2.698162in}{8.653110in}}%
\pgfpathlineto{\pgfqpoint{2.704885in}{8.654197in}}%
\pgfpathlineto{\pgfqpoint{2.713849in}{8.655863in}}%
\pgfpathlineto{\pgfqpoint{2.720573in}{8.656987in}}%
\pgfpathlineto{\pgfqpoint{2.729537in}{8.658624in}}%
\pgfpathlineto{\pgfqpoint{2.736260in}{8.659694in}}%
\pgfpathlineto{\pgfqpoint{2.744104in}{8.661009in}}%
\pgfpathlineto{\pgfqpoint{2.750827in}{8.661799in}}%
\pgfpathlineto{\pgfqpoint{2.760912in}{8.663751in}}%
\pgfpathlineto{\pgfqpoint{2.766514in}{8.664610in}}%
\pgfpathlineto{\pgfqpoint{2.768755in}{8.665207in}}%
\pgfpathlineto{\pgfqpoint{2.774358in}{8.666077in}}%
\pgfpathlineto{\pgfqpoint{2.783322in}{8.667784in}}%
\pgfpathlineto{\pgfqpoint{2.792286in}{8.669542in}}%
\pgfpathlineto{\pgfqpoint{2.799009in}{8.670484in}}%
\pgfpathlineto{\pgfqpoint{2.800130in}{8.670810in}}%
\pgfpathlineto{\pgfqpoint{2.805733in}{8.671771in}}%
\pgfpathlineto{\pgfqpoint{2.807974in}{8.672416in}}%
\pgfpathlineto{\pgfqpoint{2.813576in}{8.673411in}}%
\pgfpathlineto{\pgfqpoint{2.815817in}{8.674101in}}%
\pgfpathlineto{\pgfqpoint{2.821420in}{8.675144in}}%
\pgfpathlineto{\pgfqpoint{2.823661in}{8.675831in}}%
\pgfpathlineto{\pgfqpoint{2.829264in}{8.676839in}}%
\pgfpathlineto{\pgfqpoint{2.831505in}{8.677483in}}%
\pgfpathlineto{\pgfqpoint{2.842710in}{8.679117in}}%
\pgfpathlineto{\pgfqpoint{2.847192in}{8.680440in}}%
\pgfpathlineto{\pgfqpoint{2.852795in}{8.681436in}}%
\pgfpathlineto{\pgfqpoint{2.855036in}{8.682099in}}%
\pgfpathlineto{\pgfqpoint{2.861759in}{8.683258in}}%
\pgfpathlineto{\pgfqpoint{2.870723in}{8.685034in}}%
\pgfpathlineto{\pgfqpoint{2.876326in}{8.685949in}}%
\pgfpathlineto{\pgfqpoint{2.878567in}{8.686552in}}%
\pgfpathlineto{\pgfqpoint{2.884170in}{8.687469in}}%
\pgfpathlineto{\pgfqpoint{2.886411in}{8.688054in}}%
\pgfpathlineto{\pgfqpoint{2.892013in}{8.688930in}}%
\pgfpathlineto{\pgfqpoint{2.902098in}{8.690947in}}%
\pgfpathlineto{\pgfqpoint{2.908821in}{8.691799in}}%
\pgfpathlineto{\pgfqpoint{2.909942in}{8.692096in}}%
\pgfpathlineto{\pgfqpoint{2.915544in}{8.693029in}}%
\pgfpathlineto{\pgfqpoint{2.917785in}{8.693672in}}%
\pgfpathlineto{\pgfqpoint{2.923388in}{8.694636in}}%
\pgfpathlineto{\pgfqpoint{2.925629in}{8.695261in}}%
\pgfpathlineto{\pgfqpoint{2.931232in}{8.696191in}}%
\pgfpathlineto{\pgfqpoint{2.933473in}{8.696804in}}%
\pgfpathlineto{\pgfqpoint{2.939075in}{8.697779in}}%
\pgfpathlineto{\pgfqpoint{2.941316in}{8.698442in}}%
\pgfpathlineto{\pgfqpoint{2.946919in}{8.699435in}}%
\pgfpathlineto{\pgfqpoint{2.949160in}{8.700102in}}%
\pgfpathlineto{\pgfqpoint{2.954763in}{8.701108in}}%
\pgfpathlineto{\pgfqpoint{2.957004in}{8.701792in}}%
\pgfpathlineto{\pgfqpoint{2.961486in}{8.702536in}}%
\pgfpathlineto{\pgfqpoint{2.964848in}{8.703733in}}%
\pgfpathlineto{\pgfqpoint{2.970450in}{8.704870in}}%
\pgfpathlineto{\pgfqpoint{2.972691in}{8.705639in}}%
\pgfpathlineto{\pgfqpoint{2.978294in}{8.706763in}}%
\pgfpathlineto{\pgfqpoint{2.980535in}{8.707493in}}%
\pgfpathlineto{\pgfqpoint{2.986138in}{8.708589in}}%
\pgfpathlineto{\pgfqpoint{2.988379in}{8.709326in}}%
\pgfpathlineto{\pgfqpoint{2.993981in}{8.710467in}}%
\pgfpathlineto{\pgfqpoint{3.002945in}{8.712453in}}%
\pgfpathlineto{\pgfqpoint{3.004066in}{8.712868in}}%
\pgfpathlineto{\pgfqpoint{3.009669in}{8.714079in}}%
\pgfpathlineto{\pgfqpoint{3.011910in}{8.714887in}}%
\pgfpathlineto{\pgfqpoint{3.017512in}{8.716085in}}%
\pgfpathlineto{\pgfqpoint{3.019753in}{8.716879in}}%
\pgfpathlineto{\pgfqpoint{3.025356in}{8.718064in}}%
\pgfpathlineto{\pgfqpoint{3.027597in}{8.718827in}}%
\pgfpathlineto{\pgfqpoint{3.033200in}{8.719597in}}%
\pgfpathlineto{\pgfqpoint{3.035441in}{8.720362in}}%
\pgfpathlineto{\pgfqpoint{3.035441in}{8.720362in}}%
\pgfusepath{stroke}%
\end{pgfscope}%
\begin{pgfscope}%
\pgfsetrectcap%
\pgfsetmiterjoin%
\pgfsetlinewidth{0.803000pt}%
\definecolor{currentstroke}{rgb}{1.000000,1.000000,1.000000}%
\pgfsetstrokecolor{currentstroke}%
\pgfsetdash{}{0pt}%
\pgfpathmoveto{\pgfqpoint{0.462318in}{8.286757in}}%
\pgfpathlineto{\pgfqpoint{0.462318in}{9.391635in}}%
\pgfusepath{stroke}%
\end{pgfscope}%
\begin{pgfscope}%
\pgfsetrectcap%
\pgfsetmiterjoin%
\pgfsetlinewidth{0.803000pt}%
\definecolor{currentstroke}{rgb}{1.000000,1.000000,1.000000}%
\pgfsetstrokecolor{currentstroke}%
\pgfsetdash{}{0pt}%
\pgfpathmoveto{\pgfqpoint{3.157970in}{8.286757in}}%
\pgfpathlineto{\pgfqpoint{3.157970in}{9.391635in}}%
\pgfusepath{stroke}%
\end{pgfscope}%
\begin{pgfscope}%
\pgfsetrectcap%
\pgfsetmiterjoin%
\pgfsetlinewidth{0.803000pt}%
\definecolor{currentstroke}{rgb}{1.000000,1.000000,1.000000}%
\pgfsetstrokecolor{currentstroke}%
\pgfsetdash{}{0pt}%
\pgfpathmoveto{\pgfqpoint{0.462318in}{8.286757in}}%
\pgfpathlineto{\pgfqpoint{3.157970in}{8.286757in}}%
\pgfusepath{stroke}%
\end{pgfscope}%
\begin{pgfscope}%
\pgfsetrectcap%
\pgfsetmiterjoin%
\pgfsetlinewidth{0.803000pt}%
\definecolor{currentstroke}{rgb}{1.000000,1.000000,1.000000}%
\pgfsetstrokecolor{currentstroke}%
\pgfsetdash{}{0pt}%
\pgfpathmoveto{\pgfqpoint{0.462318in}{9.391635in}}%
\pgfpathlineto{\pgfqpoint{3.157970in}{9.391635in}}%
\pgfusepath{stroke}%
\end{pgfscope}%
\begin{pgfscope}%
\definecolor{textcolor}{rgb}{0.150000,0.150000,0.150000}%
\pgfsetstrokecolor{textcolor}%
\pgfsetfillcolor{textcolor}%
\pgftext[x=1.810144in,y=9.474968in,,base]{\color{textcolor}\rmfamily\fontsize{12.000000}{14.400000}\selectfont MMM}%
\end{pgfscope}%
\begin{pgfscope}%
\pgfsetbuttcap%
\pgfsetmiterjoin%
\definecolor{currentfill}{rgb}{0.917647,0.917647,0.949020}%
\pgfsetfillcolor{currentfill}%
\pgfsetlinewidth{0.000000pt}%
\definecolor{currentstroke}{rgb}{0.000000,0.000000,0.000000}%
\pgfsetstrokecolor{currentstroke}%
\pgfsetstrokeopacity{0.000000}%
\pgfsetdash{}{0pt}%
\pgfpathmoveto{\pgfqpoint{3.966666in}{8.286757in}}%
\pgfpathlineto{\pgfqpoint{6.662318in}{8.286757in}}%
\pgfpathlineto{\pgfqpoint{6.662318in}{9.391635in}}%
\pgfpathlineto{\pgfqpoint{3.966666in}{9.391635in}}%
\pgfpathclose%
\pgfusepath{fill}%
\end{pgfscope}%
\begin{pgfscope}%
\pgfpathrectangle{\pgfqpoint{3.966666in}{8.286757in}}{\pgfqpoint{2.695652in}{1.104878in}}%
\pgfusepath{clip}%
\pgfsetroundcap%
\pgfsetroundjoin%
\pgfsetlinewidth{0.803000pt}%
\definecolor{currentstroke}{rgb}{1.000000,1.000000,1.000000}%
\pgfsetstrokecolor{currentstroke}%
\pgfsetdash{}{0pt}%
\pgfpathmoveto{\pgfqpoint{4.086955in}{8.286757in}}%
\pgfpathlineto{\pgfqpoint{4.086955in}{9.391635in}}%
\pgfusepath{stroke}%
\end{pgfscope}%
\begin{pgfscope}%
\definecolor{textcolor}{rgb}{0.150000,0.150000,0.150000}%
\pgfsetstrokecolor{textcolor}%
\pgfsetfillcolor{textcolor}%
\pgftext[x=4.086955in,y=8.189535in,,top]{\color{textcolor}\rmfamily\fontsize{10.000000}{12.000000}\selectfont 2012}%
\end{pgfscope}%
\begin{pgfscope}%
\pgfpathrectangle{\pgfqpoint{3.966666in}{8.286757in}}{\pgfqpoint{2.695652in}{1.104878in}}%
\pgfusepath{clip}%
\pgfsetroundcap%
\pgfsetroundjoin%
\pgfsetlinewidth{0.803000pt}%
\definecolor{currentstroke}{rgb}{1.000000,1.000000,1.000000}%
\pgfsetstrokecolor{currentstroke}%
\pgfsetdash{}{0pt}%
\pgfpathmoveto{\pgfqpoint{4.497068in}{8.286757in}}%
\pgfpathlineto{\pgfqpoint{4.497068in}{9.391635in}}%
\pgfusepath{stroke}%
\end{pgfscope}%
\begin{pgfscope}%
\definecolor{textcolor}{rgb}{0.150000,0.150000,0.150000}%
\pgfsetstrokecolor{textcolor}%
\pgfsetfillcolor{textcolor}%
\pgftext[x=4.497068in,y=8.189535in,,top]{\color{textcolor}\rmfamily\fontsize{10.000000}{12.000000}\selectfont 2013}%
\end{pgfscope}%
\begin{pgfscope}%
\pgfpathrectangle{\pgfqpoint{3.966666in}{8.286757in}}{\pgfqpoint{2.695652in}{1.104878in}}%
\pgfusepath{clip}%
\pgfsetroundcap%
\pgfsetroundjoin%
\pgfsetlinewidth{0.803000pt}%
\definecolor{currentstroke}{rgb}{1.000000,1.000000,1.000000}%
\pgfsetstrokecolor{currentstroke}%
\pgfsetdash{}{0pt}%
\pgfpathmoveto{\pgfqpoint{4.906060in}{8.286757in}}%
\pgfpathlineto{\pgfqpoint{4.906060in}{9.391635in}}%
\pgfusepath{stroke}%
\end{pgfscope}%
\begin{pgfscope}%
\definecolor{textcolor}{rgb}{0.150000,0.150000,0.150000}%
\pgfsetstrokecolor{textcolor}%
\pgfsetfillcolor{textcolor}%
\pgftext[x=4.906060in,y=8.189535in,,top]{\color{textcolor}\rmfamily\fontsize{10.000000}{12.000000}\selectfont 2014}%
\end{pgfscope}%
\begin{pgfscope}%
\pgfpathrectangle{\pgfqpoint{3.966666in}{8.286757in}}{\pgfqpoint{2.695652in}{1.104878in}}%
\pgfusepath{clip}%
\pgfsetroundcap%
\pgfsetroundjoin%
\pgfsetlinewidth{0.803000pt}%
\definecolor{currentstroke}{rgb}{1.000000,1.000000,1.000000}%
\pgfsetstrokecolor{currentstroke}%
\pgfsetdash{}{0pt}%
\pgfpathmoveto{\pgfqpoint{5.315052in}{8.286757in}}%
\pgfpathlineto{\pgfqpoint{5.315052in}{9.391635in}}%
\pgfusepath{stroke}%
\end{pgfscope}%
\begin{pgfscope}%
\definecolor{textcolor}{rgb}{0.150000,0.150000,0.150000}%
\pgfsetstrokecolor{textcolor}%
\pgfsetfillcolor{textcolor}%
\pgftext[x=5.315052in,y=8.189535in,,top]{\color{textcolor}\rmfamily\fontsize{10.000000}{12.000000}\selectfont 2015}%
\end{pgfscope}%
\begin{pgfscope}%
\pgfpathrectangle{\pgfqpoint{3.966666in}{8.286757in}}{\pgfqpoint{2.695652in}{1.104878in}}%
\pgfusepath{clip}%
\pgfsetroundcap%
\pgfsetroundjoin%
\pgfsetlinewidth{0.803000pt}%
\definecolor{currentstroke}{rgb}{1.000000,1.000000,1.000000}%
\pgfsetstrokecolor{currentstroke}%
\pgfsetdash{}{0pt}%
\pgfpathmoveto{\pgfqpoint{5.724045in}{8.286757in}}%
\pgfpathlineto{\pgfqpoint{5.724045in}{9.391635in}}%
\pgfusepath{stroke}%
\end{pgfscope}%
\begin{pgfscope}%
\definecolor{textcolor}{rgb}{0.150000,0.150000,0.150000}%
\pgfsetstrokecolor{textcolor}%
\pgfsetfillcolor{textcolor}%
\pgftext[x=5.724045in,y=8.189535in,,top]{\color{textcolor}\rmfamily\fontsize{10.000000}{12.000000}\selectfont 2016}%
\end{pgfscope}%
\begin{pgfscope}%
\pgfpathrectangle{\pgfqpoint{3.966666in}{8.286757in}}{\pgfqpoint{2.695652in}{1.104878in}}%
\pgfusepath{clip}%
\pgfsetroundcap%
\pgfsetroundjoin%
\pgfsetlinewidth{0.803000pt}%
\definecolor{currentstroke}{rgb}{1.000000,1.000000,1.000000}%
\pgfsetstrokecolor{currentstroke}%
\pgfsetdash{}{0pt}%
\pgfpathmoveto{\pgfqpoint{6.134158in}{8.286757in}}%
\pgfpathlineto{\pgfqpoint{6.134158in}{9.391635in}}%
\pgfusepath{stroke}%
\end{pgfscope}%
\begin{pgfscope}%
\definecolor{textcolor}{rgb}{0.150000,0.150000,0.150000}%
\pgfsetstrokecolor{textcolor}%
\pgfsetfillcolor{textcolor}%
\pgftext[x=6.134158in,y=8.189535in,,top]{\color{textcolor}\rmfamily\fontsize{10.000000}{12.000000}\selectfont 2017}%
\end{pgfscope}%
\begin{pgfscope}%
\pgfpathrectangle{\pgfqpoint{3.966666in}{8.286757in}}{\pgfqpoint{2.695652in}{1.104878in}}%
\pgfusepath{clip}%
\pgfsetroundcap%
\pgfsetroundjoin%
\pgfsetlinewidth{0.803000pt}%
\definecolor{currentstroke}{rgb}{1.000000,1.000000,1.000000}%
\pgfsetstrokecolor{currentstroke}%
\pgfsetdash{}{0pt}%
\pgfpathmoveto{\pgfqpoint{6.543150in}{8.286757in}}%
\pgfpathlineto{\pgfqpoint{6.543150in}{9.391635in}}%
\pgfusepath{stroke}%
\end{pgfscope}%
\begin{pgfscope}%
\definecolor{textcolor}{rgb}{0.150000,0.150000,0.150000}%
\pgfsetstrokecolor{textcolor}%
\pgfsetfillcolor{textcolor}%
\pgftext[x=6.543150in,y=8.189535in,,top]{\color{textcolor}\rmfamily\fontsize{10.000000}{12.000000}\selectfont 2018}%
\end{pgfscope}%
\begin{pgfscope}%
\pgfpathrectangle{\pgfqpoint{3.966666in}{8.286757in}}{\pgfqpoint{2.695652in}{1.104878in}}%
\pgfusepath{clip}%
\pgfsetroundcap%
\pgfsetroundjoin%
\pgfsetlinewidth{0.803000pt}%
\definecolor{currentstroke}{rgb}{1.000000,1.000000,1.000000}%
\pgfsetstrokecolor{currentstroke}%
\pgfsetdash{}{0pt}%
\pgfpathmoveto{\pgfqpoint{3.966666in}{8.648327in}}%
\pgfpathlineto{\pgfqpoint{6.662318in}{8.648327in}}%
\pgfusepath{stroke}%
\end{pgfscope}%
\begin{pgfscope}%
\definecolor{textcolor}{rgb}{0.150000,0.150000,0.150000}%
\pgfsetstrokecolor{textcolor}%
\pgfsetfillcolor{textcolor}%
\pgftext[x=3.692713in,y=8.595566in,left,base]{\color{textcolor}\rmfamily\fontsize{10.000000}{12.000000}\selectfont 60}%
\end{pgfscope}%
\begin{pgfscope}%
\pgfpathrectangle{\pgfqpoint{3.966666in}{8.286757in}}{\pgfqpoint{2.695652in}{1.104878in}}%
\pgfusepath{clip}%
\pgfsetroundcap%
\pgfsetroundjoin%
\pgfsetlinewidth{0.803000pt}%
\definecolor{currentstroke}{rgb}{1.000000,1.000000,1.000000}%
\pgfsetstrokecolor{currentstroke}%
\pgfsetdash{}{0pt}%
\pgfpathmoveto{\pgfqpoint{3.966666in}{9.020754in}}%
\pgfpathlineto{\pgfqpoint{6.662318in}{9.020754in}}%
\pgfusepath{stroke}%
\end{pgfscope}%
\begin{pgfscope}%
\definecolor{textcolor}{rgb}{0.150000,0.150000,0.150000}%
\pgfsetstrokecolor{textcolor}%
\pgfsetfillcolor{textcolor}%
\pgftext[x=3.692713in,y=8.967992in,left,base]{\color{textcolor}\rmfamily\fontsize{10.000000}{12.000000}\selectfont 80}%
\end{pgfscope}%
\begin{pgfscope}%
\pgfpathrectangle{\pgfqpoint{3.966666in}{8.286757in}}{\pgfqpoint{2.695652in}{1.104878in}}%
\pgfusepath{clip}%
\pgfsetroundcap%
\pgfsetroundjoin%
\pgfsetlinewidth{1.505625pt}%
\definecolor{currentstroke}{rgb}{1.000000,0.498039,0.054902}%
\pgfsetstrokecolor{currentstroke}%
\pgfsetdash{}{0pt}%
\pgfpathmoveto{\pgfqpoint{4.089196in}{8.336979in}}%
\pgfpathlineto{\pgfqpoint{4.090316in}{8.337537in}}%
\pgfpathlineto{\pgfqpoint{4.091437in}{8.346848in}}%
\pgfpathlineto{\pgfqpoint{4.092557in}{8.337910in}}%
\pgfpathlineto{\pgfqpoint{4.095919in}{8.339958in}}%
\pgfpathlineto{\pgfqpoint{4.098160in}{8.349269in}}%
\pgfpathlineto{\pgfqpoint{4.099280in}{8.361000in}}%
\pgfpathlineto{\pgfqpoint{4.104883in}{8.370497in}}%
\pgfpathlineto{\pgfqpoint{4.107124in}{8.382787in}}%
\pgfpathlineto{\pgfqpoint{4.108245in}{8.367518in}}%
\pgfpathlineto{\pgfqpoint{4.112727in}{8.353924in}}%
\pgfpathlineto{\pgfqpoint{4.113847in}{8.369752in}}%
\pgfpathlineto{\pgfqpoint{4.116088in}{8.364352in}}%
\pgfpathlineto{\pgfqpoint{4.119450in}{8.352248in}}%
\pgfpathlineto{\pgfqpoint{4.120570in}{8.369194in}}%
\pgfpathlineto{\pgfqpoint{4.122811in}{8.386325in}}%
\pgfpathlineto{\pgfqpoint{4.123932in}{8.404574in}}%
\pgfpathlineto{\pgfqpoint{4.127294in}{8.397126in}}%
\pgfpathlineto{\pgfqpoint{4.128414in}{8.402526in}}%
\pgfpathlineto{\pgfqpoint{4.129535in}{8.394332in}}%
\pgfpathlineto{\pgfqpoint{4.130655in}{8.405505in}}%
\pgfpathlineto{\pgfqpoint{4.131776in}{8.397126in}}%
\pgfpathlineto{\pgfqpoint{4.135137in}{8.401408in}}%
\pgfpathlineto{\pgfqpoint{4.136258in}{8.399733in}}%
\pgfpathlineto{\pgfqpoint{4.137378in}{8.392284in}}%
\pgfpathlineto{\pgfqpoint{4.138499in}{8.414816in}}%
\pgfpathlineto{\pgfqpoint{4.145222in}{8.413512in}}%
\pgfpathlineto{\pgfqpoint{4.146343in}{8.411464in}}%
\pgfpathlineto{\pgfqpoint{4.147463in}{8.422637in}}%
\pgfpathlineto{\pgfqpoint{4.150825in}{8.436789in}}%
\pgfpathlineto{\pgfqpoint{4.151945in}{8.429713in}}%
\pgfpathlineto{\pgfqpoint{4.153066in}{8.415188in}}%
\pgfpathlineto{\pgfqpoint{4.154186in}{8.426547in}}%
\pgfpathlineto{\pgfqpoint{4.155307in}{8.416864in}}%
\pgfpathlineto{\pgfqpoint{4.158668in}{8.416492in}}%
\pgfpathlineto{\pgfqpoint{4.159789in}{8.396008in}}%
\pgfpathlineto{\pgfqpoint{4.163150in}{8.420402in}}%
\pgfpathlineto{\pgfqpoint{4.166512in}{8.413140in}}%
\pgfpathlineto{\pgfqpoint{4.169874in}{8.479246in}}%
\pgfpathlineto{\pgfqpoint{4.170994in}{8.476452in}}%
\pgfpathlineto{\pgfqpoint{4.174356in}{8.488370in}}%
\pgfpathlineto{\pgfqpoint{4.175476in}{8.482411in}}%
\pgfpathlineto{\pgfqpoint{4.176597in}{8.484646in}}%
\pgfpathlineto{\pgfqpoint{4.177717in}{8.488743in}}%
\pgfpathlineto{\pgfqpoint{4.178838in}{8.487998in}}%
\pgfpathlineto{\pgfqpoint{4.182199in}{8.511647in}}%
\pgfpathlineto{\pgfqpoint{4.183320in}{8.504384in}}%
\pgfpathlineto{\pgfqpoint{4.184440in}{8.518350in}}%
\pgfpathlineto{\pgfqpoint{4.185561in}{8.498798in}}%
\pgfpathlineto{\pgfqpoint{4.186682in}{8.498239in}}%
\pgfpathlineto{\pgfqpoint{4.190043in}{8.500846in}}%
\pgfpathlineto{\pgfqpoint{4.191164in}{8.510529in}}%
\pgfpathlineto{\pgfqpoint{4.192284in}{8.493398in}}%
\pgfpathlineto{\pgfqpoint{4.193405in}{8.505688in}}%
\pgfpathlineto{\pgfqpoint{4.197887in}{8.489860in}}%
\pgfpathlineto{\pgfqpoint{4.199007in}{8.473473in}}%
\pgfpathlineto{\pgfqpoint{4.201248in}{8.504571in}}%
\pgfpathlineto{\pgfqpoint{4.202369in}{8.491908in}}%
\pgfpathlineto{\pgfqpoint{4.205730in}{8.501033in}}%
\pgfpathlineto{\pgfqpoint{4.206851in}{8.506991in}}%
\pgfpathlineto{\pgfqpoint{4.207972in}{8.504571in}}%
\pgfpathlineto{\pgfqpoint{4.209092in}{8.496750in}}%
\pgfpathlineto{\pgfqpoint{4.210213in}{8.494701in}}%
\pgfpathlineto{\pgfqpoint{4.213574in}{8.492653in}}%
\pgfpathlineto{\pgfqpoint{4.214695in}{8.497681in}}%
\pgfpathlineto{\pgfqpoint{4.216936in}{8.530641in}}%
\pgfpathlineto{\pgfqpoint{4.218056in}{8.540324in}}%
\pgfpathlineto{\pgfqpoint{4.221418in}{8.541068in}}%
\pgfpathlineto{\pgfqpoint{4.222538in}{8.553731in}}%
\pgfpathlineto{\pgfqpoint{4.223659in}{8.555221in}}%
\pgfpathlineto{\pgfqpoint{4.224779in}{8.551869in}}%
\pgfpathlineto{\pgfqpoint{4.225900in}{8.539206in}}%
\pgfpathlineto{\pgfqpoint{4.229262in}{8.539206in}}%
\pgfpathlineto{\pgfqpoint{4.230382in}{8.535668in}}%
\pgfpathlineto{\pgfqpoint{4.231503in}{8.528406in}}%
\pgfpathlineto{\pgfqpoint{4.232623in}{8.527847in}}%
\pgfpathlineto{\pgfqpoint{4.233744in}{8.531572in}}%
\pgfpathlineto{\pgfqpoint{4.238226in}{8.502895in}}%
\pgfpathlineto{\pgfqpoint{4.239346in}{8.493770in}}%
\pgfpathlineto{\pgfqpoint{4.240467in}{8.464535in}}%
\pgfpathlineto{\pgfqpoint{4.241587in}{8.460438in}}%
\pgfpathlineto{\pgfqpoint{4.244949in}{8.475894in}}%
\pgfpathlineto{\pgfqpoint{4.246069in}{8.476266in}}%
\pgfpathlineto{\pgfqpoint{4.247190in}{8.469935in}}%
\pgfpathlineto{\pgfqpoint{4.248311in}{8.476080in}}%
\pgfpathlineto{\pgfqpoint{4.249431in}{8.467142in}}%
\pgfpathlineto{\pgfqpoint{4.253913in}{8.479804in}}%
\pgfpathlineto{\pgfqpoint{4.255034in}{8.461369in}}%
\pgfpathlineto{\pgfqpoint{4.256154in}{8.467514in}}%
\pgfpathlineto{\pgfqpoint{4.257275in}{8.427292in}}%
\pgfpathlineto{\pgfqpoint{4.260636in}{8.428596in}}%
\pgfpathlineto{\pgfqpoint{4.261757in}{8.436603in}}%
\pgfpathlineto{\pgfqpoint{4.262877in}{8.460066in}}%
\pgfpathlineto{\pgfqpoint{4.263998in}{8.457645in}}%
\pgfpathlineto{\pgfqpoint{4.265118in}{8.468073in}}%
\pgfpathlineto{\pgfqpoint{4.268480in}{8.455410in}}%
\pgfpathlineto{\pgfqpoint{4.269601in}{8.478501in}}%
\pgfpathlineto{\pgfqpoint{4.270721in}{8.455410in}}%
\pgfpathlineto{\pgfqpoint{4.271842in}{8.454479in}}%
\pgfpathlineto{\pgfqpoint{4.272962in}{8.475149in}}%
\pgfpathlineto{\pgfqpoint{4.276324in}{8.467887in}}%
\pgfpathlineto{\pgfqpoint{4.277444in}{8.486136in}}%
\pgfpathlineto{\pgfqpoint{4.278565in}{8.494515in}}%
\pgfpathlineto{\pgfqpoint{4.279685in}{8.475708in}}%
\pgfpathlineto{\pgfqpoint{4.280806in}{8.483715in}}%
\pgfpathlineto{\pgfqpoint{4.284167in}{8.471238in}}%
\pgfpathlineto{\pgfqpoint{4.285288in}{8.472356in}}%
\pgfpathlineto{\pgfqpoint{4.286408in}{8.485204in}}%
\pgfpathlineto{\pgfqpoint{4.287529in}{8.482039in}}%
\pgfpathlineto{\pgfqpoint{4.288649in}{8.507550in}}%
\pgfpathlineto{\pgfqpoint{4.292011in}{8.521516in}}%
\pgfpathlineto{\pgfqpoint{4.293132in}{8.531013in}}%
\pgfpathlineto{\pgfqpoint{4.295373in}{8.528406in}}%
\pgfpathlineto{\pgfqpoint{4.296493in}{8.517792in}}%
\pgfpathlineto{\pgfqpoint{4.299855in}{8.514254in}}%
\pgfpathlineto{\pgfqpoint{4.300975in}{8.513881in}}%
\pgfpathlineto{\pgfqpoint{4.302096in}{8.508667in}}%
\pgfpathlineto{\pgfqpoint{4.303216in}{8.489301in}}%
\pgfpathlineto{\pgfqpoint{4.304337in}{8.506060in}}%
\pgfpathlineto{\pgfqpoint{4.307698in}{8.517978in}}%
\pgfpathlineto{\pgfqpoint{4.308819in}{8.518723in}}%
\pgfpathlineto{\pgfqpoint{4.309940in}{8.512205in}}%
\pgfpathlineto{\pgfqpoint{4.311060in}{8.477570in}}%
\pgfpathlineto{\pgfqpoint{4.312181in}{8.470494in}}%
\pgfpathlineto{\pgfqpoint{4.315542in}{8.469004in}}%
\pgfpathlineto{\pgfqpoint{4.316663in}{8.467328in}}%
\pgfpathlineto{\pgfqpoint{4.317783in}{8.474404in}}%
\pgfpathlineto{\pgfqpoint{4.318904in}{8.503267in}}%
\pgfpathlineto{\pgfqpoint{4.320024in}{8.516116in}}%
\pgfpathlineto{\pgfqpoint{4.323386in}{8.512205in}}%
\pgfpathlineto{\pgfqpoint{4.324506in}{8.502336in}}%
\pgfpathlineto{\pgfqpoint{4.325627in}{8.487067in}}%
\pgfpathlineto{\pgfqpoint{4.326747in}{8.481853in}}%
\pgfpathlineto{\pgfqpoint{4.327868in}{8.500660in}}%
\pgfpathlineto{\pgfqpoint{4.331230in}{8.492094in}}%
\pgfpathlineto{\pgfqpoint{4.332350in}{8.501219in}}%
\pgfpathlineto{\pgfqpoint{4.333471in}{8.505688in}}%
\pgfpathlineto{\pgfqpoint{4.334591in}{8.481480in}}%
\pgfpathlineto{\pgfqpoint{4.335712in}{8.471052in}}%
\pgfpathlineto{\pgfqpoint{4.339073in}{8.475708in}}%
\pgfpathlineto{\pgfqpoint{4.340194in}{8.474963in}}%
\pgfpathlineto{\pgfqpoint{4.343555in}{8.500288in}}%
\pgfpathlineto{\pgfqpoint{4.348037in}{8.483715in}}%
\pgfpathlineto{\pgfqpoint{4.349158in}{8.487439in}}%
\pgfpathlineto{\pgfqpoint{4.350278in}{8.480735in}}%
\pgfpathlineto{\pgfqpoint{4.351399in}{8.498612in}}%
\pgfpathlineto{\pgfqpoint{4.354761in}{8.497495in}}%
\pgfpathlineto{\pgfqpoint{4.355881in}{8.499915in}}%
\pgfpathlineto{\pgfqpoint{4.357002in}{8.497495in}}%
\pgfpathlineto{\pgfqpoint{4.358122in}{8.493398in}}%
\pgfpathlineto{\pgfqpoint{4.359243in}{8.512392in}}%
\pgfpathlineto{\pgfqpoint{4.363725in}{8.517606in}}%
\pgfpathlineto{\pgfqpoint{4.364845in}{8.493584in}}%
\pgfpathlineto{\pgfqpoint{4.367086in}{8.502709in}}%
\pgfpathlineto{\pgfqpoint{4.370448in}{8.499170in}}%
\pgfpathlineto{\pgfqpoint{4.371569in}{8.494329in}}%
\pgfpathlineto{\pgfqpoint{4.372689in}{8.495074in}}%
\pgfpathlineto{\pgfqpoint{4.373810in}{8.524868in}}%
\pgfpathlineto{\pgfqpoint{4.374930in}{8.528592in}}%
\pgfpathlineto{\pgfqpoint{4.378292in}{8.527102in}}%
\pgfpathlineto{\pgfqpoint{4.379412in}{8.518537in}}%
\pgfpathlineto{\pgfqpoint{4.380533in}{8.518537in}}%
\pgfpathlineto{\pgfqpoint{4.382774in}{8.504943in}}%
\pgfpathlineto{\pgfqpoint{4.386135in}{8.501777in}}%
\pgfpathlineto{\pgfqpoint{4.387256in}{8.492653in}}%
\pgfpathlineto{\pgfqpoint{4.388376in}{8.475894in}}%
\pgfpathlineto{\pgfqpoint{4.390617in}{8.487998in}}%
\pgfpathlineto{\pgfqpoint{4.393979in}{8.502709in}}%
\pgfpathlineto{\pgfqpoint{4.395100in}{8.493398in}}%
\pgfpathlineto{\pgfqpoint{4.396220in}{8.500660in}}%
\pgfpathlineto{\pgfqpoint{4.397341in}{8.516861in}}%
\pgfpathlineto{\pgfqpoint{4.398461in}{8.520213in}}%
\pgfpathlineto{\pgfqpoint{4.401823in}{8.524495in}}%
\pgfpathlineto{\pgfqpoint{4.402943in}{8.514812in}}%
\pgfpathlineto{\pgfqpoint{4.404064in}{8.510157in}}%
\pgfpathlineto{\pgfqpoint{4.405184in}{8.518723in}}%
\pgfpathlineto{\pgfqpoint{4.406305in}{8.508854in}}%
\pgfpathlineto{\pgfqpoint{4.409666in}{8.503826in}}%
\pgfpathlineto{\pgfqpoint{4.411907in}{8.533806in}}%
\pgfpathlineto{\pgfqpoint{4.413028in}{8.504198in}}%
\pgfpathlineto{\pgfqpoint{4.414149in}{8.491536in}}%
\pgfpathlineto{\pgfqpoint{4.417510in}{8.489487in}}%
\pgfpathlineto{\pgfqpoint{4.418631in}{8.466397in}}%
\pgfpathlineto{\pgfqpoint{4.419751in}{8.463976in}}%
\pgfpathlineto{\pgfqpoint{4.421992in}{8.472728in}}%
\pgfpathlineto{\pgfqpoint{4.427595in}{8.476452in}}%
\pgfpathlineto{\pgfqpoint{4.428715in}{8.491350in}}%
\pgfpathlineto{\pgfqpoint{4.433198in}{8.482970in}}%
\pgfpathlineto{\pgfqpoint{4.434318in}{8.496936in}}%
\pgfpathlineto{\pgfqpoint{4.435439in}{8.469749in}}%
\pgfpathlineto{\pgfqpoint{4.436559in}{8.469563in}}%
\pgfpathlineto{\pgfqpoint{4.437680in}{8.474032in}}%
\pgfpathlineto{\pgfqpoint{4.441041in}{8.468818in}}%
\pgfpathlineto{\pgfqpoint{4.443282in}{8.437161in}}%
\pgfpathlineto{\pgfqpoint{4.444403in}{8.436975in}}%
\pgfpathlineto{\pgfqpoint{4.445523in}{8.448148in}}%
\pgfpathlineto{\pgfqpoint{4.448885in}{8.463976in}}%
\pgfpathlineto{\pgfqpoint{4.450005in}{8.474590in}}%
\pgfpathlineto{\pgfqpoint{4.451126in}{8.476639in}}%
\pgfpathlineto{\pgfqpoint{4.453367in}{8.485577in}}%
\pgfpathlineto{\pgfqpoint{4.456729in}{8.471611in}}%
\pgfpathlineto{\pgfqpoint{4.457849in}{8.450569in}}%
\pgfpathlineto{\pgfqpoint{4.458970in}{8.468631in}}%
\pgfpathlineto{\pgfqpoint{4.460090in}{8.475521in}}%
\pgfpathlineto{\pgfqpoint{4.461211in}{8.475149in}}%
\pgfpathlineto{\pgfqpoint{4.464572in}{8.476825in}}%
\pgfpathlineto{\pgfqpoint{4.465693in}{8.474218in}}%
\pgfpathlineto{\pgfqpoint{4.466813in}{8.483715in}}%
\pgfpathlineto{\pgfqpoint{4.467934in}{8.478873in}}%
\pgfpathlineto{\pgfqpoint{4.469054in}{8.487253in}}%
\pgfpathlineto{\pgfqpoint{4.472416in}{8.489674in}}%
\pgfpathlineto{\pgfqpoint{4.473536in}{8.494888in}}%
\pgfpathlineto{\pgfqpoint{4.474657in}{8.505129in}}%
\pgfpathlineto{\pgfqpoint{4.475778in}{8.506060in}}%
\pgfpathlineto{\pgfqpoint{4.476898in}{8.487811in}}%
\pgfpathlineto{\pgfqpoint{4.480260in}{8.498426in}}%
\pgfpathlineto{\pgfqpoint{4.481380in}{8.507736in}}%
\pgfpathlineto{\pgfqpoint{4.482501in}{8.490232in}}%
\pgfpathlineto{\pgfqpoint{4.483621in}{8.500660in}}%
\pgfpathlineto{\pgfqpoint{4.484742in}{8.504757in}}%
\pgfpathlineto{\pgfqpoint{4.488103in}{8.502709in}}%
\pgfpathlineto{\pgfqpoint{4.490344in}{8.496005in}}%
\pgfpathlineto{\pgfqpoint{4.491465in}{8.488556in}}%
\pgfpathlineto{\pgfqpoint{4.492585in}{8.488184in}}%
\pgfpathlineto{\pgfqpoint{4.495947in}{8.501964in}}%
\pgfpathlineto{\pgfqpoint{4.498188in}{8.526916in}}%
\pgfpathlineto{\pgfqpoint{4.499309in}{8.531013in}}%
\pgfpathlineto{\pgfqpoint{4.500429in}{8.541441in}}%
\pgfpathlineto{\pgfqpoint{4.503791in}{8.545724in}}%
\pgfpathlineto{\pgfqpoint{4.504911in}{8.551496in}}%
\pgfpathlineto{\pgfqpoint{4.506032in}{8.552427in}}%
\pgfpathlineto{\pgfqpoint{4.508273in}{8.569000in}}%
\pgfpathlineto{\pgfqpoint{4.511634in}{8.568442in}}%
\pgfpathlineto{\pgfqpoint{4.512755in}{8.560993in}}%
\pgfpathlineto{\pgfqpoint{4.513875in}{8.558573in}}%
\pgfpathlineto{\pgfqpoint{4.514996in}{8.560621in}}%
\pgfpathlineto{\pgfqpoint{4.516117in}{8.544234in}}%
\pgfpathlineto{\pgfqpoint{4.520599in}{8.537903in}}%
\pgfpathlineto{\pgfqpoint{4.521719in}{8.529896in}}%
\pgfpathlineto{\pgfqpoint{4.523960in}{8.539579in}}%
\pgfpathlineto{\pgfqpoint{4.527322in}{8.532316in}}%
\pgfpathlineto{\pgfqpoint{4.528442in}{8.538648in}}%
\pgfpathlineto{\pgfqpoint{4.529563in}{8.536599in}}%
\pgfpathlineto{\pgfqpoint{4.530683in}{8.527847in}}%
\pgfpathlineto{\pgfqpoint{4.531804in}{8.546469in}}%
\pgfpathlineto{\pgfqpoint{4.535165in}{8.538089in}}%
\pgfpathlineto{\pgfqpoint{4.536286in}{8.559131in}}%
\pgfpathlineto{\pgfqpoint{4.537407in}{8.557455in}}%
\pgfpathlineto{\pgfqpoint{4.538527in}{8.584270in}}%
\pgfpathlineto{\pgfqpoint{4.539648in}{8.578497in}}%
\pgfpathlineto{\pgfqpoint{4.543009in}{8.581663in}}%
\pgfpathlineto{\pgfqpoint{4.544130in}{8.585387in}}%
\pgfpathlineto{\pgfqpoint{4.545250in}{8.583525in}}%
\pgfpathlineto{\pgfqpoint{4.546371in}{8.587622in}}%
\pgfpathlineto{\pgfqpoint{4.547491in}{8.576635in}}%
\pgfpathlineto{\pgfqpoint{4.551973in}{8.586691in}}%
\pgfpathlineto{\pgfqpoint{4.554214in}{8.574214in}}%
\pgfpathlineto{\pgfqpoint{4.555335in}{8.591532in}}%
\pgfpathlineto{\pgfqpoint{4.559817in}{8.580546in}}%
\pgfpathlineto{\pgfqpoint{4.560938in}{8.590415in}}%
\pgfpathlineto{\pgfqpoint{4.562058in}{8.584456in}}%
\pgfpathlineto{\pgfqpoint{4.563179in}{8.588180in}}%
\pgfpathlineto{\pgfqpoint{4.566540in}{8.596932in}}%
\pgfpathlineto{\pgfqpoint{4.567661in}{8.617788in}}%
\pgfpathlineto{\pgfqpoint{4.568781in}{8.626913in}}%
\pgfpathlineto{\pgfqpoint{4.569902in}{8.625237in}}%
\pgfpathlineto{\pgfqpoint{4.571022in}{8.627658in}}%
\pgfpathlineto{\pgfqpoint{4.574384in}{8.641251in}}%
\pgfpathlineto{\pgfqpoint{4.575504in}{8.637713in}}%
\pgfpathlineto{\pgfqpoint{4.576625in}{8.637899in}}%
\pgfpathlineto{\pgfqpoint{4.577746in}{8.639203in}}%
\pgfpathlineto{\pgfqpoint{4.578866in}{8.651307in}}%
\pgfpathlineto{\pgfqpoint{4.582228in}{8.645907in}}%
\pgfpathlineto{\pgfqpoint{4.583348in}{8.634920in}}%
\pgfpathlineto{\pgfqpoint{4.584469in}{8.649631in}}%
\pgfpathlineto{\pgfqpoint{4.585589in}{8.639762in}}%
\pgfpathlineto{\pgfqpoint{4.586710in}{8.653355in}}%
\pgfpathlineto{\pgfqpoint{4.590071in}{8.651493in}}%
\pgfpathlineto{\pgfqpoint{4.591192in}{8.669556in}}%
\pgfpathlineto{\pgfqpoint{4.592312in}{8.669183in}}%
\pgfpathlineto{\pgfqpoint{4.593433in}{8.674397in}}%
\pgfpathlineto{\pgfqpoint{4.597915in}{8.671045in}}%
\pgfpathlineto{\pgfqpoint{4.599036in}{8.677563in}}%
\pgfpathlineto{\pgfqpoint{4.600156in}{8.657266in}}%
\pgfpathlineto{\pgfqpoint{4.601277in}{8.665459in}}%
\pgfpathlineto{\pgfqpoint{4.602397in}{8.641065in}}%
\pgfpathlineto{\pgfqpoint{4.605759in}{8.646279in}}%
\pgfpathlineto{\pgfqpoint{4.606879in}{8.639762in}}%
\pgfpathlineto{\pgfqpoint{4.608000in}{8.642927in}}%
\pgfpathlineto{\pgfqpoint{4.609120in}{8.648700in}}%
\pgfpathlineto{\pgfqpoint{4.610241in}{8.647582in}}%
\pgfpathlineto{\pgfqpoint{4.613602in}{8.620768in}}%
\pgfpathlineto{\pgfqpoint{4.614723in}{8.629147in}}%
\pgfpathlineto{\pgfqpoint{4.615843in}{8.621326in}}%
\pgfpathlineto{\pgfqpoint{4.616964in}{8.636782in}}%
\pgfpathlineto{\pgfqpoint{4.618084in}{8.674025in}}%
\pgfpathlineto{\pgfqpoint{4.621446in}{8.664342in}}%
\pgfpathlineto{\pgfqpoint{4.622567in}{8.677749in}}%
\pgfpathlineto{\pgfqpoint{4.623687in}{8.677190in}}%
\pgfpathlineto{\pgfqpoint{4.624808in}{8.689294in}}%
\pgfpathlineto{\pgfqpoint{4.625928in}{8.682777in}}%
\pgfpathlineto{\pgfqpoint{4.629290in}{8.680728in}}%
\pgfpathlineto{\pgfqpoint{4.630410in}{8.693950in}}%
\pgfpathlineto{\pgfqpoint{4.631531in}{8.691715in}}%
\pgfpathlineto{\pgfqpoint{4.633772in}{8.724861in}}%
\pgfpathlineto{\pgfqpoint{4.637133in}{8.722068in}}%
\pgfpathlineto{\pgfqpoint{4.638254in}{8.725047in}}%
\pgfpathlineto{\pgfqpoint{4.639374in}{8.726351in}}%
\pgfpathlineto{\pgfqpoint{4.644977in}{8.717412in}}%
\pgfpathlineto{\pgfqpoint{4.647218in}{8.768249in}}%
\pgfpathlineto{\pgfqpoint{4.648339in}{8.758938in}}%
\pgfpathlineto{\pgfqpoint{4.649459in}{8.777559in}}%
\pgfpathlineto{\pgfqpoint{4.652821in}{8.795808in}}%
\pgfpathlineto{\pgfqpoint{4.653941in}{8.807912in}}%
\pgfpathlineto{\pgfqpoint{4.655062in}{8.796553in}}%
\pgfpathlineto{\pgfqpoint{4.656182in}{8.800836in}}%
\pgfpathlineto{\pgfqpoint{4.657303in}{8.810705in}}%
\pgfpathlineto{\pgfqpoint{4.661785in}{8.825789in}}%
\pgfpathlineto{\pgfqpoint{4.662906in}{8.820202in}}%
\pgfpathlineto{\pgfqpoint{4.664026in}{8.825416in}}%
\pgfpathlineto{\pgfqpoint{4.665147in}{8.818154in}}%
\pgfpathlineto{\pgfqpoint{4.668508in}{8.831003in}}%
\pgfpathlineto{\pgfqpoint{4.669629in}{8.824113in}}%
\pgfpathlineto{\pgfqpoint{4.670749in}{8.801953in}}%
\pgfpathlineto{\pgfqpoint{4.672990in}{8.857631in}}%
\pgfpathlineto{\pgfqpoint{4.676352in}{8.861728in}}%
\pgfpathlineto{\pgfqpoint{4.678593in}{8.801208in}}%
\pgfpathlineto{\pgfqpoint{4.679713in}{8.809588in}}%
\pgfpathlineto{\pgfqpoint{4.680834in}{8.771601in}}%
\pgfpathlineto{\pgfqpoint{4.684196in}{8.786311in}}%
\pgfpathlineto{\pgfqpoint{4.685316in}{8.805678in}}%
\pgfpathlineto{\pgfqpoint{4.686437in}{8.793015in}}%
\pgfpathlineto{\pgfqpoint{4.687557in}{8.770669in}}%
\pgfpathlineto{\pgfqpoint{4.688678in}{8.777373in}}%
\pgfpathlineto{\pgfqpoint{4.692039in}{8.755214in}}%
\pgfpathlineto{\pgfqpoint{4.695401in}{8.808098in}}%
\pgfpathlineto{\pgfqpoint{4.696521in}{8.801953in}}%
\pgfpathlineto{\pgfqpoint{4.699883in}{8.816850in}}%
\pgfpathlineto{\pgfqpoint{4.701003in}{8.803443in}}%
\pgfpathlineto{\pgfqpoint{4.702124in}{8.802698in}}%
\pgfpathlineto{\pgfqpoint{4.704365in}{8.832306in}}%
\pgfpathlineto{\pgfqpoint{4.707727in}{8.844782in}}%
\pgfpathlineto{\pgfqpoint{4.708847in}{8.854465in}}%
\pgfpathlineto{\pgfqpoint{4.709968in}{8.832679in}}%
\pgfpathlineto{\pgfqpoint{4.711088in}{8.842920in}}%
\pgfpathlineto{\pgfqpoint{4.712209in}{8.866756in}}%
\pgfpathlineto{\pgfqpoint{4.715570in}{8.858562in}}%
\pgfpathlineto{\pgfqpoint{4.716691in}{8.865638in}}%
\pgfpathlineto{\pgfqpoint{4.717811in}{8.840686in}}%
\pgfpathlineto{\pgfqpoint{4.718932in}{8.793015in}}%
\pgfpathlineto{\pgfqpoint{4.720052in}{8.793946in}}%
\pgfpathlineto{\pgfqpoint{4.723414in}{8.805119in}}%
\pgfpathlineto{\pgfqpoint{4.724535in}{8.799905in}}%
\pgfpathlineto{\pgfqpoint{4.725655in}{8.816106in}}%
\pgfpathlineto{\pgfqpoint{4.726776in}{8.822995in}}%
\pgfpathlineto{\pgfqpoint{4.727896in}{8.815733in}}%
\pgfpathlineto{\pgfqpoint{4.731258in}{8.810892in}}%
\pgfpathlineto{\pgfqpoint{4.732378in}{8.813126in}}%
\pgfpathlineto{\pgfqpoint{4.733499in}{8.788918in}}%
\pgfpathlineto{\pgfqpoint{4.734619in}{8.820575in}}%
\pgfpathlineto{\pgfqpoint{4.739101in}{8.825789in}}%
\pgfpathlineto{\pgfqpoint{4.740222in}{8.823740in}}%
\pgfpathlineto{\pgfqpoint{4.741342in}{8.812381in}}%
\pgfpathlineto{\pgfqpoint{4.742463in}{8.830258in}}%
\pgfpathlineto{\pgfqpoint{4.743584in}{8.818526in}}%
\pgfpathlineto{\pgfqpoint{4.746945in}{8.818340in}}%
\pgfpathlineto{\pgfqpoint{4.748066in}{8.830816in}}%
\pgfpathlineto{\pgfqpoint{4.749186in}{8.824858in}}%
\pgfpathlineto{\pgfqpoint{4.750307in}{8.807912in}}%
\pgfpathlineto{\pgfqpoint{4.751427in}{8.812754in}}%
\pgfpathlineto{\pgfqpoint{4.754789in}{8.798601in}}%
\pgfpathlineto{\pgfqpoint{4.755909in}{8.797484in}}%
\pgfpathlineto{\pgfqpoint{4.757030in}{8.783332in}}%
\pgfpathlineto{\pgfqpoint{4.758150in}{8.790408in}}%
\pgfpathlineto{\pgfqpoint{4.759271in}{8.786870in}}%
\pgfpathlineto{\pgfqpoint{4.762632in}{8.786311in}}%
\pgfpathlineto{\pgfqpoint{4.763753in}{8.757262in}}%
\pgfpathlineto{\pgfqpoint{4.764874in}{8.758938in}}%
\pgfpathlineto{\pgfqpoint{4.765994in}{8.762476in}}%
\pgfpathlineto{\pgfqpoint{4.767115in}{8.757262in}}%
\pgfpathlineto{\pgfqpoint{4.771597in}{8.766014in}}%
\pgfpathlineto{\pgfqpoint{4.773838in}{8.789477in}}%
\pgfpathlineto{\pgfqpoint{4.774958in}{8.781842in}}%
\pgfpathlineto{\pgfqpoint{4.778320in}{8.787056in}}%
\pgfpathlineto{\pgfqpoint{4.780561in}{8.812940in}}%
\pgfpathlineto{\pgfqpoint{4.781681in}{8.814802in}}%
\pgfpathlineto{\pgfqpoint{4.782802in}{8.814988in}}%
\pgfpathlineto{\pgfqpoint{4.786164in}{8.820202in}}%
\pgfpathlineto{\pgfqpoint{4.788405in}{8.855024in}}%
\pgfpathlineto{\pgfqpoint{4.789525in}{8.853721in}}%
\pgfpathlineto{\pgfqpoint{4.794007in}{8.834541in}}%
\pgfpathlineto{\pgfqpoint{4.795128in}{8.828209in}}%
\pgfpathlineto{\pgfqpoint{4.796248in}{8.826720in}}%
\pgfpathlineto{\pgfqpoint{4.797369in}{8.832492in}}%
\pgfpathlineto{\pgfqpoint{4.798489in}{8.825044in}}%
\pgfpathlineto{\pgfqpoint{4.801851in}{8.818713in}}%
\pgfpathlineto{\pgfqpoint{4.802971in}{8.825789in}}%
\pgfpathlineto{\pgfqpoint{4.804092in}{8.806609in}}%
\pgfpathlineto{\pgfqpoint{4.805213in}{8.797112in}}%
\pgfpathlineto{\pgfqpoint{4.806333in}{8.801953in}}%
\pgfpathlineto{\pgfqpoint{4.809695in}{8.778490in}}%
\pgfpathlineto{\pgfqpoint{4.810815in}{8.766200in}}%
\pgfpathlineto{\pgfqpoint{4.811936in}{8.766200in}}%
\pgfpathlineto{\pgfqpoint{4.813056in}{8.807912in}}%
\pgfpathlineto{\pgfqpoint{4.814177in}{8.820575in}}%
\pgfpathlineto{\pgfqpoint{4.817538in}{8.832120in}}%
\pgfpathlineto{\pgfqpoint{4.818659in}{8.818154in}}%
\pgfpathlineto{\pgfqpoint{4.819779in}{8.836403in}}%
\pgfpathlineto{\pgfqpoint{4.820900in}{8.903253in}}%
\pgfpathlineto{\pgfqpoint{4.822020in}{8.908281in}}%
\pgfpathlineto{\pgfqpoint{4.825382in}{8.906233in}}%
\pgfpathlineto{\pgfqpoint{4.826503in}{8.914240in}}%
\pgfpathlineto{\pgfqpoint{4.827623in}{8.910143in}}%
\pgfpathlineto{\pgfqpoint{4.828744in}{8.914799in}}%
\pgfpathlineto{\pgfqpoint{4.829864in}{8.944034in}}%
\pgfpathlineto{\pgfqpoint{4.833226in}{8.946455in}}%
\pgfpathlineto{\pgfqpoint{4.834346in}{8.961911in}}%
\pgfpathlineto{\pgfqpoint{4.835467in}{8.952414in}}%
\pgfpathlineto{\pgfqpoint{4.836587in}{8.930068in}}%
\pgfpathlineto{\pgfqpoint{4.837708in}{8.936213in}}%
\pgfpathlineto{\pgfqpoint{4.842190in}{8.931744in}}%
\pgfpathlineto{\pgfqpoint{4.843310in}{8.935654in}}%
\pgfpathlineto{\pgfqpoint{4.844431in}{8.914985in}}%
\pgfpathlineto{\pgfqpoint{4.845551in}{8.929696in}}%
\pgfpathlineto{\pgfqpoint{4.848913in}{8.923364in}}%
\pgfpathlineto{\pgfqpoint{4.850034in}{8.917033in}}%
\pgfpathlineto{\pgfqpoint{4.852275in}{8.931744in}}%
\pgfpathlineto{\pgfqpoint{4.853395in}{8.947200in}}%
\pgfpathlineto{\pgfqpoint{4.856757in}{8.939379in}}%
\pgfpathlineto{\pgfqpoint{4.857877in}{8.941427in}}%
\pgfpathlineto{\pgfqpoint{4.858998in}{8.937889in}}%
\pgfpathlineto{\pgfqpoint{4.860118in}{8.966938in}}%
\pgfpathlineto{\pgfqpoint{4.861239in}{8.965262in}}%
\pgfpathlineto{\pgfqpoint{4.864600in}{8.977925in}}%
\pgfpathlineto{\pgfqpoint{4.866842in}{8.994870in}}%
\pgfpathlineto{\pgfqpoint{4.869083in}{8.998595in}}%
\pgfpathlineto{\pgfqpoint{4.872444in}{8.989843in}}%
\pgfpathlineto{\pgfqpoint{4.873565in}{8.978111in}}%
\pgfpathlineto{\pgfqpoint{4.874685in}{8.976249in}}%
\pgfpathlineto{\pgfqpoint{4.875806in}{8.976994in}}%
\pgfpathlineto{\pgfqpoint{4.876926in}{9.000829in}}%
\pgfpathlineto{\pgfqpoint{4.880288in}{8.997664in}}%
\pgfpathlineto{\pgfqpoint{4.881408in}{8.989843in}}%
\pgfpathlineto{\pgfqpoint{4.882529in}{8.966007in}}%
\pgfpathlineto{\pgfqpoint{4.883649in}{8.955766in}}%
\pgfpathlineto{\pgfqpoint{4.884770in}{8.962283in}}%
\pgfpathlineto{\pgfqpoint{4.888132in}{8.976621in}}%
\pgfpathlineto{\pgfqpoint{4.889252in}{8.969545in}}%
\pgfpathlineto{\pgfqpoint{4.890373in}{9.001760in}}%
\pgfpathlineto{\pgfqpoint{4.891493in}{9.009023in}}%
\pgfpathlineto{\pgfqpoint{4.892614in}{9.028761in}}%
\pgfpathlineto{\pgfqpoint{4.895975in}{9.041424in}}%
\pgfpathlineto{\pgfqpoint{4.897096in}{9.047941in}}%
\pgfpathlineto{\pgfqpoint{4.903819in}{9.063397in}}%
\pgfpathlineto{\pgfqpoint{4.904939in}{9.082763in}}%
\pgfpathlineto{\pgfqpoint{4.907180in}{9.060976in}}%
\pgfpathlineto{\pgfqpoint{4.908301in}{9.066004in}}%
\pgfpathlineto{\pgfqpoint{4.911663in}{9.065259in}}%
\pgfpathlineto{\pgfqpoint{4.912783in}{9.059486in}}%
\pgfpathlineto{\pgfqpoint{4.913904in}{9.064142in}}%
\pgfpathlineto{\pgfqpoint{4.919506in}{9.022802in}}%
\pgfpathlineto{\pgfqpoint{4.920627in}{9.025037in}}%
\pgfpathlineto{\pgfqpoint{4.921747in}{9.044403in}}%
\pgfpathlineto{\pgfqpoint{4.922868in}{9.036210in}}%
\pgfpathlineto{\pgfqpoint{4.923988in}{9.090956in}}%
\pgfpathlineto{\pgfqpoint{4.928471in}{9.084439in}}%
\pgfpathlineto{\pgfqpoint{4.929591in}{9.093750in}}%
\pgfpathlineto{\pgfqpoint{4.931832in}{9.022057in}}%
\pgfpathlineto{\pgfqpoint{4.935194in}{9.001015in}}%
\pgfpathlineto{\pgfqpoint{4.936314in}{9.016657in}}%
\pgfpathlineto{\pgfqpoint{4.937435in}{8.998036in}}%
\pgfpathlineto{\pgfqpoint{4.938555in}{9.016285in}}%
\pgfpathlineto{\pgfqpoint{4.939676in}{8.988911in}}%
\pgfpathlineto{\pgfqpoint{4.943037in}{8.951855in}}%
\pgfpathlineto{\pgfqpoint{4.944158in}{8.971407in}}%
\pgfpathlineto{\pgfqpoint{4.945278in}{8.966566in}}%
\pgfpathlineto{\pgfqpoint{4.947519in}{9.022802in}}%
\pgfpathlineto{\pgfqpoint{4.953122in}{9.057438in}}%
\pgfpathlineto{\pgfqpoint{4.954243in}{9.055390in}}%
\pgfpathlineto{\pgfqpoint{4.955363in}{9.057252in}}%
\pgfpathlineto{\pgfqpoint{4.959845in}{9.057438in}}%
\pgfpathlineto{\pgfqpoint{4.960966in}{9.054459in}}%
\pgfpathlineto{\pgfqpoint{4.962086in}{9.057438in}}%
\pgfpathlineto{\pgfqpoint{4.963207in}{9.052969in}}%
\pgfpathlineto{\pgfqpoint{4.966568in}{9.072894in}}%
\pgfpathlineto{\pgfqpoint{4.967689in}{9.072707in}}%
\pgfpathlineto{\pgfqpoint{4.968809in}{9.069728in}}%
\pgfpathlineto{\pgfqpoint{4.969930in}{9.079225in}}%
\pgfpathlineto{\pgfqpoint{4.971051in}{9.096357in}}%
\pgfpathlineto{\pgfqpoint{4.974412in}{9.073825in}}%
\pgfpathlineto{\pgfqpoint{4.975533in}{9.119075in}}%
\pgfpathlineto{\pgfqpoint{4.976653in}{9.111067in}}%
\pgfpathlineto{\pgfqpoint{4.977774in}{9.134716in}}%
\pgfpathlineto{\pgfqpoint{4.978894in}{9.140489in}}%
\pgfpathlineto{\pgfqpoint{4.982256in}{9.137696in}}%
\pgfpathlineto{\pgfqpoint{4.984497in}{9.124102in}}%
\pgfpathlineto{\pgfqpoint{4.985617in}{9.086115in}}%
\pgfpathlineto{\pgfqpoint{4.986738in}{9.077177in}}%
\pgfpathlineto{\pgfqpoint{4.991220in}{9.101571in}}%
\pgfpathlineto{\pgfqpoint{4.992341in}{9.086860in}}%
\pgfpathlineto{\pgfqpoint{4.993461in}{9.103246in}}%
\pgfpathlineto{\pgfqpoint{4.999064in}{9.088722in}}%
\pgfpathlineto{\pgfqpoint{5.000184in}{9.068425in}}%
\pgfpathlineto{\pgfqpoint{5.002425in}{9.082204in}}%
\pgfpathlineto{\pgfqpoint{5.005787in}{9.074756in}}%
\pgfpathlineto{\pgfqpoint{5.006907in}{9.094308in}}%
\pgfpathlineto{\pgfqpoint{5.008028in}{9.085184in}}%
\pgfpathlineto{\pgfqpoint{5.009148in}{9.095053in}}%
\pgfpathlineto{\pgfqpoint{5.010269in}{9.063955in}}%
\pgfpathlineto{\pgfqpoint{5.013631in}{9.019823in}}%
\pgfpathlineto{\pgfqpoint{5.014751in}{9.017961in}}%
\pgfpathlineto{\pgfqpoint{5.015872in}{9.056321in}}%
\pgfpathlineto{\pgfqpoint{5.016992in}{8.998408in}}%
\pgfpathlineto{\pgfqpoint{5.018113in}{8.984442in}}%
\pgfpathlineto{\pgfqpoint{5.021474in}{9.000829in}}%
\pgfpathlineto{\pgfqpoint{5.022595in}{9.010140in}}%
\pgfpathlineto{\pgfqpoint{5.023715in}{9.033603in}}%
\pgfpathlineto{\pgfqpoint{5.024836in}{9.013305in}}%
\pgfpathlineto{\pgfqpoint{5.029318in}{9.020940in}}%
\pgfpathlineto{\pgfqpoint{5.030438in}{9.027830in}}%
\pgfpathlineto{\pgfqpoint{5.031559in}{9.028947in}}%
\pgfpathlineto{\pgfqpoint{5.032680in}{9.033789in}}%
\pgfpathlineto{\pgfqpoint{5.033800in}{9.027271in}}%
\pgfpathlineto{\pgfqpoint{5.037162in}{9.027644in}}%
\pgfpathlineto{\pgfqpoint{5.038282in}{9.040306in}}%
\pgfpathlineto{\pgfqpoint{5.039403in}{9.033975in}}%
\pgfpathlineto{\pgfqpoint{5.040523in}{9.023547in}}%
\pgfpathlineto{\pgfqpoint{5.041644in}{9.025409in}}%
\pgfpathlineto{\pgfqpoint{5.045005in}{9.032858in}}%
\pgfpathlineto{\pgfqpoint{5.046126in}{9.012933in}}%
\pgfpathlineto{\pgfqpoint{5.047246in}{9.043472in}}%
\pgfpathlineto{\pgfqpoint{5.048367in}{9.054459in}}%
\pgfpathlineto{\pgfqpoint{5.049487in}{9.058369in}}%
\pgfpathlineto{\pgfqpoint{5.052849in}{9.072335in}}%
\pgfpathlineto{\pgfqpoint{5.055090in}{9.051852in}}%
\pgfpathlineto{\pgfqpoint{5.056211in}{9.036955in}}%
\pgfpathlineto{\pgfqpoint{5.057331in}{9.035279in}}%
\pgfpathlineto{\pgfqpoint{5.060693in}{9.045893in}}%
\pgfpathlineto{\pgfqpoint{5.061813in}{9.028202in}}%
\pgfpathlineto{\pgfqpoint{5.062934in}{9.041424in}}%
\pgfpathlineto{\pgfqpoint{5.064054in}{9.046451in}}%
\pgfpathlineto{\pgfqpoint{5.069657in}{9.102129in}}%
\pgfpathlineto{\pgfqpoint{5.070777in}{9.096543in}}%
\pgfpathlineto{\pgfqpoint{5.073019in}{9.103991in}}%
\pgfpathlineto{\pgfqpoint{5.076380in}{9.110695in}}%
\pgfpathlineto{\pgfqpoint{5.077501in}{9.107902in}}%
\pgfpathlineto{\pgfqpoint{5.078621in}{9.109391in}}%
\pgfpathlineto{\pgfqpoint{5.079742in}{9.126337in}}%
\pgfpathlineto{\pgfqpoint{5.080862in}{9.162648in}}%
\pgfpathlineto{\pgfqpoint{5.084224in}{9.174008in}}%
\pgfpathlineto{\pgfqpoint{5.087585in}{9.160042in}}%
\pgfpathlineto{\pgfqpoint{5.088706in}{9.161531in}}%
\pgfpathlineto{\pgfqpoint{5.092067in}{9.153338in}}%
\pgfpathlineto{\pgfqpoint{5.093188in}{9.158366in}}%
\pgfpathlineto{\pgfqpoint{5.094309in}{9.173821in}}%
\pgfpathlineto{\pgfqpoint{5.095429in}{9.165442in}}%
\pgfpathlineto{\pgfqpoint{5.096550in}{9.173449in}}%
\pgfpathlineto{\pgfqpoint{5.099911in}{9.172890in}}%
\pgfpathlineto{\pgfqpoint{5.101032in}{9.154455in}}%
\pgfpathlineto{\pgfqpoint{5.102152in}{9.157993in}}%
\pgfpathlineto{\pgfqpoint{5.103273in}{9.152221in}}%
\pgfpathlineto{\pgfqpoint{5.104393in}{9.163021in}}%
\pgfpathlineto{\pgfqpoint{5.107755in}{9.161904in}}%
\pgfpathlineto{\pgfqpoint{5.108875in}{9.170283in}}%
\pgfpathlineto{\pgfqpoint{5.109996in}{9.167862in}}%
\pgfpathlineto{\pgfqpoint{5.111116in}{9.178663in}}%
\pgfpathlineto{\pgfqpoint{5.115599in}{9.171028in}}%
\pgfpathlineto{\pgfqpoint{5.116719in}{9.156876in}}%
\pgfpathlineto{\pgfqpoint{5.117840in}{9.164138in}}%
\pgfpathlineto{\pgfqpoint{5.118960in}{9.159297in}}%
\pgfpathlineto{\pgfqpoint{5.120081in}{9.159483in}}%
\pgfpathlineto{\pgfqpoint{5.123442in}{9.160786in}}%
\pgfpathlineto{\pgfqpoint{5.124563in}{9.159297in}}%
\pgfpathlineto{\pgfqpoint{5.125683in}{9.159483in}}%
\pgfpathlineto{\pgfqpoint{5.126804in}{9.133972in}}%
\pgfpathlineto{\pgfqpoint{5.127924in}{9.143282in}}%
\pgfpathlineto{\pgfqpoint{5.131286in}{9.132110in}}%
\pgfpathlineto{\pgfqpoint{5.132406in}{9.140675in}}%
\pgfpathlineto{\pgfqpoint{5.134647in}{9.136765in}}%
\pgfpathlineto{\pgfqpoint{5.135768in}{9.115723in}}%
\pgfpathlineto{\pgfqpoint{5.139130in}{9.114605in}}%
\pgfpathlineto{\pgfqpoint{5.140250in}{9.111998in}}%
\pgfpathlineto{\pgfqpoint{5.141371in}{9.098219in}}%
\pgfpathlineto{\pgfqpoint{5.143612in}{9.021685in}}%
\pgfpathlineto{\pgfqpoint{5.146973in}{9.029506in}}%
\pgfpathlineto{\pgfqpoint{5.148094in}{9.019823in}}%
\pgfpathlineto{\pgfqpoint{5.149214in}{9.020382in}}%
\pgfpathlineto{\pgfqpoint{5.150335in}{9.013864in}}%
\pgfpathlineto{\pgfqpoint{5.151455in}{9.038817in}}%
\pgfpathlineto{\pgfqpoint{5.154817in}{9.029878in}}%
\pgfpathlineto{\pgfqpoint{5.155938in}{9.032113in}}%
\pgfpathlineto{\pgfqpoint{5.157058in}{9.037699in}}%
\pgfpathlineto{\pgfqpoint{5.158179in}{9.035465in}}%
\pgfpathlineto{\pgfqpoint{5.159299in}{9.023920in}}%
\pgfpathlineto{\pgfqpoint{5.162661in}{9.033603in}}%
\pgfpathlineto{\pgfqpoint{5.163781in}{9.050176in}}%
\pgfpathlineto{\pgfqpoint{5.164902in}{9.056507in}}%
\pgfpathlineto{\pgfqpoint{5.166022in}{9.067680in}}%
\pgfpathlineto{\pgfqpoint{5.167143in}{9.063211in}}%
\pgfpathlineto{\pgfqpoint{5.170504in}{9.075501in}}%
\pgfpathlineto{\pgfqpoint{5.171625in}{9.067866in}}%
\pgfpathlineto{\pgfqpoint{5.172745in}{9.069356in}}%
\pgfpathlineto{\pgfqpoint{5.173866in}{9.065631in}}%
\pgfpathlineto{\pgfqpoint{5.174986in}{9.074756in}}%
\pgfpathlineto{\pgfqpoint{5.179469in}{9.077735in}}%
\pgfpathlineto{\pgfqpoint{5.180589in}{9.084998in}}%
\pgfpathlineto{\pgfqpoint{5.181710in}{9.076432in}}%
\pgfpathlineto{\pgfqpoint{5.182830in}{9.075687in}}%
\pgfpathlineto{\pgfqpoint{5.186192in}{9.064142in}}%
\pgfpathlineto{\pgfqpoint{5.187312in}{9.046265in}}%
\pgfpathlineto{\pgfqpoint{5.188433in}{9.055017in}}%
\pgfpathlineto{\pgfqpoint{5.189553in}{9.055203in}}%
\pgfpathlineto{\pgfqpoint{5.190674in}{9.041796in}}%
\pgfpathlineto{\pgfqpoint{5.194035in}{9.037327in}}%
\pgfpathlineto{\pgfqpoint{5.197397in}{9.084253in}}%
\pgfpathlineto{\pgfqpoint{5.198518in}{9.077363in}}%
\pgfpathlineto{\pgfqpoint{5.201879in}{9.065073in}}%
\pgfpathlineto{\pgfqpoint{5.203000in}{9.053341in}}%
\pgfpathlineto{\pgfqpoint{5.204120in}{9.057066in}}%
\pgfpathlineto{\pgfqpoint{5.205241in}{9.026713in}}%
\pgfpathlineto{\pgfqpoint{5.206361in}{9.054459in}}%
\pgfpathlineto{\pgfqpoint{5.209723in}{9.047196in}}%
\pgfpathlineto{\pgfqpoint{5.210843in}{9.040120in}}%
\pgfpathlineto{\pgfqpoint{5.211964in}{9.014609in}}%
\pgfpathlineto{\pgfqpoint{5.213084in}{9.015726in}}%
\pgfpathlineto{\pgfqpoint{5.214205in}{9.038072in}}%
\pgfpathlineto{\pgfqpoint{5.217567in}{9.035837in}}%
\pgfpathlineto{\pgfqpoint{5.218687in}{9.006788in}}%
\pgfpathlineto{\pgfqpoint{5.219808in}{9.042355in}}%
\pgfpathlineto{\pgfqpoint{5.222049in}{9.000457in}}%
\pgfpathlineto{\pgfqpoint{5.225410in}{8.962283in}}%
\pgfpathlineto{\pgfqpoint{5.226531in}{8.961538in}}%
\pgfpathlineto{\pgfqpoint{5.227651in}{8.930254in}}%
\pgfpathlineto{\pgfqpoint{5.228772in}{8.918337in}}%
\pgfpathlineto{\pgfqpoint{5.229892in}{8.958745in}}%
\pgfpathlineto{\pgfqpoint{5.233254in}{8.983511in}}%
\pgfpathlineto{\pgfqpoint{5.234374in}{9.011816in}}%
\pgfpathlineto{\pgfqpoint{5.235495in}{8.982766in}}%
\pgfpathlineto{\pgfqpoint{5.236615in}{9.011257in}}%
\pgfpathlineto{\pgfqpoint{5.237736in}{9.024851in}}%
\pgfpathlineto{\pgfqpoint{5.241098in}{9.028947in}}%
\pgfpathlineto{\pgfqpoint{5.242218in}{9.052783in}}%
\pgfpathlineto{\pgfqpoint{5.244459in}{9.065259in}}%
\pgfpathlineto{\pgfqpoint{5.245580in}{9.086301in}}%
\pgfpathlineto{\pgfqpoint{5.248941in}{9.101757in}}%
\pgfpathlineto{\pgfqpoint{5.250062in}{9.111067in}}%
\pgfpathlineto{\pgfqpoint{5.251182in}{9.128944in}}%
\pgfpathlineto{\pgfqpoint{5.252303in}{9.114419in}}%
\pgfpathlineto{\pgfqpoint{5.253423in}{9.126151in}}%
\pgfpathlineto{\pgfqpoint{5.256785in}{9.128571in}}%
\pgfpathlineto{\pgfqpoint{5.257905in}{9.117212in}}%
\pgfpathlineto{\pgfqpoint{5.259026in}{9.113861in}}%
\pgfpathlineto{\pgfqpoint{5.261267in}{9.098777in}}%
\pgfpathlineto{\pgfqpoint{5.264629in}{9.089280in}}%
\pgfpathlineto{\pgfqpoint{5.265749in}{9.097101in}}%
\pgfpathlineto{\pgfqpoint{5.266870in}{9.095798in}}%
\pgfpathlineto{\pgfqpoint{5.267990in}{9.097846in}}%
\pgfpathlineto{\pgfqpoint{5.269111in}{9.093936in}}%
\pgfpathlineto{\pgfqpoint{5.272472in}{9.105109in}}%
\pgfpathlineto{\pgfqpoint{5.273593in}{9.111254in}}%
\pgfpathlineto{\pgfqpoint{5.274713in}{9.112185in}}%
\pgfpathlineto{\pgfqpoint{5.276954in}{9.128944in}}%
\pgfpathlineto{\pgfqpoint{5.280316in}{9.124102in}}%
\pgfpathlineto{\pgfqpoint{5.281437in}{9.138999in}}%
\pgfpathlineto{\pgfqpoint{5.282557in}{9.108647in}}%
\pgfpathlineto{\pgfqpoint{5.284798in}{9.132854in}}%
\pgfpathlineto{\pgfqpoint{5.288160in}{9.148682in}}%
\pgfpathlineto{\pgfqpoint{5.289280in}{9.145144in}}%
\pgfpathlineto{\pgfqpoint{5.290401in}{9.134716in}}%
\pgfpathlineto{\pgfqpoint{5.291521in}{9.141234in}}%
\pgfpathlineto{\pgfqpoint{5.292642in}{9.104364in}}%
\pgfpathlineto{\pgfqpoint{5.296003in}{9.087791in}}%
\pgfpathlineto{\pgfqpoint{5.297124in}{9.056693in}}%
\pgfpathlineto{\pgfqpoint{5.299365in}{9.141979in}}%
\pgfpathlineto{\pgfqpoint{5.300486in}{9.137323in}}%
\pgfpathlineto{\pgfqpoint{5.306088in}{9.157621in}}%
\pgfpathlineto{\pgfqpoint{5.308329in}{9.161345in}}%
\pgfpathlineto{\pgfqpoint{5.312811in}{9.160973in}}%
\pgfpathlineto{\pgfqpoint{5.313932in}{9.139744in}}%
\pgfpathlineto{\pgfqpoint{5.316173in}{9.139372in}}%
\pgfpathlineto{\pgfqpoint{5.319534in}{9.096729in}}%
\pgfpathlineto{\pgfqpoint{5.320655in}{9.063397in}}%
\pgfpathlineto{\pgfqpoint{5.322896in}{9.119075in}}%
\pgfpathlineto{\pgfqpoint{5.324017in}{9.098964in}}%
\pgfpathlineto{\pgfqpoint{5.328499in}{9.078294in}}%
\pgfpathlineto{\pgfqpoint{5.330740in}{9.020195in}}%
\pgfpathlineto{\pgfqpoint{5.331860in}{9.022989in}}%
\pgfpathlineto{\pgfqpoint{5.337463in}{9.051293in}}%
\pgfpathlineto{\pgfqpoint{5.338583in}{8.994125in}}%
\pgfpathlineto{\pgfqpoint{5.343066in}{8.975504in}}%
\pgfpathlineto{\pgfqpoint{5.345307in}{8.948317in}}%
\pgfpathlineto{\pgfqpoint{5.346427in}{8.952972in}}%
\pgfpathlineto{\pgfqpoint{5.347548in}{8.930254in}}%
\pgfpathlineto{\pgfqpoint{5.350909in}{8.955207in}}%
\pgfpathlineto{\pgfqpoint{5.352030in}{8.982953in}}%
\pgfpathlineto{\pgfqpoint{5.353150in}{8.980904in}}%
\pgfpathlineto{\pgfqpoint{5.354271in}{9.000270in}}%
\pgfpathlineto{\pgfqpoint{5.355391in}{9.005112in}}%
\pgfpathlineto{\pgfqpoint{5.358753in}{9.004740in}}%
\pgfpathlineto{\pgfqpoint{5.359873in}{9.019637in}}%
\pgfpathlineto{\pgfqpoint{5.360994in}{9.022430in}}%
\pgfpathlineto{\pgfqpoint{5.362115in}{8.926530in}}%
\pgfpathlineto{\pgfqpoint{5.363235in}{8.885004in}}%
\pgfpathlineto{\pgfqpoint{5.367717in}{8.902322in}}%
\pgfpathlineto{\pgfqpoint{5.368838in}{8.914426in}}%
\pgfpathlineto{\pgfqpoint{5.369958in}{8.890591in}}%
\pgfpathlineto{\pgfqpoint{5.371079in}{8.915357in}}%
\pgfpathlineto{\pgfqpoint{5.374440in}{8.923551in}}%
\pgfpathlineto{\pgfqpoint{5.375561in}{8.933234in}}%
\pgfpathlineto{\pgfqpoint{5.377802in}{8.974573in}}%
\pgfpathlineto{\pgfqpoint{5.378922in}{8.945896in}}%
\pgfpathlineto{\pgfqpoint{5.382284in}{8.953531in}}%
\pgfpathlineto{\pgfqpoint{5.383405in}{8.951483in}}%
\pgfpathlineto{\pgfqpoint{5.384525in}{8.929137in}}%
\pgfpathlineto{\pgfqpoint{5.385646in}{8.938261in}}%
\pgfpathlineto{\pgfqpoint{5.386766in}{8.923737in}}%
\pgfpathlineto{\pgfqpoint{5.390128in}{8.927089in}}%
\pgfpathlineto{\pgfqpoint{5.391248in}{8.902881in}}%
\pgfpathlineto{\pgfqpoint{5.392369in}{8.908840in}}%
\pgfpathlineto{\pgfqpoint{5.393489in}{8.945338in}}%
\pgfpathlineto{\pgfqpoint{5.394610in}{8.928765in}}%
\pgfpathlineto{\pgfqpoint{5.397971in}{8.944220in}}%
\pgfpathlineto{\pgfqpoint{5.399092in}{8.936586in}}%
\pgfpathlineto{\pgfqpoint{5.400212in}{8.950552in}}%
\pgfpathlineto{\pgfqpoint{5.401333in}{8.944965in}}%
\pgfpathlineto{\pgfqpoint{5.402453in}{8.965076in}}%
\pgfpathlineto{\pgfqpoint{5.405815in}{8.956510in}}%
\pgfpathlineto{\pgfqpoint{5.406936in}{8.941986in}}%
\pgfpathlineto{\pgfqpoint{5.410297in}{8.883142in}}%
\pgfpathlineto{\pgfqpoint{5.413659in}{8.884446in}}%
\pgfpathlineto{\pgfqpoint{5.414779in}{8.890218in}}%
\pgfpathlineto{\pgfqpoint{5.417020in}{8.917778in}}%
\pgfpathlineto{\pgfqpoint{5.421502in}{8.916474in}}%
\pgfpathlineto{\pgfqpoint{5.422623in}{8.893943in}}%
\pgfpathlineto{\pgfqpoint{5.424864in}{8.908281in}}%
\pgfpathlineto{\pgfqpoint{5.425985in}{8.915730in}}%
\pgfpathlineto{\pgfqpoint{5.429346in}{8.910143in}}%
\pgfpathlineto{\pgfqpoint{5.431587in}{8.918523in}}%
\pgfpathlineto{\pgfqpoint{5.432708in}{8.938820in}}%
\pgfpathlineto{\pgfqpoint{5.433828in}{8.876252in}}%
\pgfpathlineto{\pgfqpoint{5.437190in}{8.874763in}}%
\pgfpathlineto{\pgfqpoint{5.438310in}{8.875694in}}%
\pgfpathlineto{\pgfqpoint{5.439431in}{8.895805in}}%
\pgfpathlineto{\pgfqpoint{5.441672in}{8.887984in}}%
\pgfpathlineto{\pgfqpoint{5.445034in}{8.879232in}}%
\pgfpathlineto{\pgfqpoint{5.446154in}{8.879232in}}%
\pgfpathlineto{\pgfqpoint{5.447275in}{8.873459in}}%
\pgfpathlineto{\pgfqpoint{5.449516in}{8.882584in}}%
\pgfpathlineto{\pgfqpoint{5.452877in}{8.893012in}}%
\pgfpathlineto{\pgfqpoint{5.453998in}{8.885377in}}%
\pgfpathlineto{\pgfqpoint{5.455118in}{8.885563in}}%
\pgfpathlineto{\pgfqpoint{5.457359in}{8.905115in}}%
\pgfpathlineto{\pgfqpoint{5.460721in}{8.917778in}}%
\pgfpathlineto{\pgfqpoint{5.461841in}{8.906978in}}%
\pgfpathlineto{\pgfqpoint{5.464082in}{8.936027in}}%
\pgfpathlineto{\pgfqpoint{5.465203in}{8.926716in}}%
\pgfpathlineto{\pgfqpoint{5.468565in}{8.925971in}}%
\pgfpathlineto{\pgfqpoint{5.469685in}{8.946641in}}%
\pgfpathlineto{\pgfqpoint{5.470806in}{8.939565in}}%
\pgfpathlineto{\pgfqpoint{5.471926in}{8.936213in}}%
\pgfpathlineto{\pgfqpoint{5.473047in}{8.944593in}}%
\pgfpathlineto{\pgfqpoint{5.477529in}{8.926158in}}%
\pgfpathlineto{\pgfqpoint{5.478649in}{8.924854in}}%
\pgfpathlineto{\pgfqpoint{5.479770in}{8.924482in}}%
\pgfpathlineto{\pgfqpoint{5.480890in}{8.918150in}}%
\pgfpathlineto{\pgfqpoint{5.484252in}{8.913681in}}%
\pgfpathlineto{\pgfqpoint{5.486493in}{8.933420in}}%
\pgfpathlineto{\pgfqpoint{5.487614in}{8.911819in}}%
\pgfpathlineto{\pgfqpoint{5.488734in}{8.912378in}}%
\pgfpathlineto{\pgfqpoint{5.492096in}{8.902136in}}%
\pgfpathlineto{\pgfqpoint{5.493216in}{8.908467in}}%
\pgfpathlineto{\pgfqpoint{5.494337in}{8.926158in}}%
\pgfpathlineto{\pgfqpoint{5.495457in}{8.928206in}}%
\pgfpathlineto{\pgfqpoint{5.496578in}{8.914799in}}%
\pgfpathlineto{\pgfqpoint{5.499939in}{8.909957in}}%
\pgfpathlineto{\pgfqpoint{5.501060in}{8.911819in}}%
\pgfpathlineto{\pgfqpoint{5.502180in}{8.928578in}}%
\pgfpathlineto{\pgfqpoint{5.503301in}{8.936399in}}%
\pgfpathlineto{\pgfqpoint{5.504421in}{8.926158in}}%
\pgfpathlineto{\pgfqpoint{5.507783in}{8.944593in}}%
\pgfpathlineto{\pgfqpoint{5.508904in}{8.946641in}}%
\pgfpathlineto{\pgfqpoint{5.511145in}{8.920757in}}%
\pgfpathlineto{\pgfqpoint{5.512265in}{8.920757in}}%
\pgfpathlineto{\pgfqpoint{5.515627in}{8.884632in}}%
\pgfpathlineto{\pgfqpoint{5.516747in}{8.888356in}}%
\pgfpathlineto{\pgfqpoint{5.517868in}{8.900274in}}%
\pgfpathlineto{\pgfqpoint{5.518988in}{8.896922in}}%
\pgfpathlineto{\pgfqpoint{5.523470in}{8.886122in}}%
\pgfpathlineto{\pgfqpoint{5.524591in}{8.885377in}}%
\pgfpathlineto{\pgfqpoint{5.525711in}{8.857072in}}%
\pgfpathlineto{\pgfqpoint{5.526832in}{8.864149in}}%
\pgfpathlineto{\pgfqpoint{5.527953in}{8.881094in}}%
\pgfpathlineto{\pgfqpoint{5.532435in}{8.910143in}}%
\pgfpathlineto{\pgfqpoint{5.533555in}{8.902695in}}%
\pgfpathlineto{\pgfqpoint{5.535796in}{8.914426in}}%
\pgfpathlineto{\pgfqpoint{5.539158in}{8.915916in}}%
\pgfpathlineto{\pgfqpoint{5.540278in}{8.909771in}}%
\pgfpathlineto{\pgfqpoint{5.541399in}{8.910516in}}%
\pgfpathlineto{\pgfqpoint{5.543640in}{8.856514in}}%
\pgfpathlineto{\pgfqpoint{5.547001in}{8.839382in}}%
\pgfpathlineto{\pgfqpoint{5.548122in}{8.842734in}}%
\pgfpathlineto{\pgfqpoint{5.550363in}{8.860424in}}%
\pgfpathlineto{\pgfqpoint{5.551484in}{8.859307in}}%
\pgfpathlineto{\pgfqpoint{5.554845in}{8.858004in}}%
\pgfpathlineto{\pgfqpoint{5.555966in}{8.853348in}}%
\pgfpathlineto{\pgfqpoint{5.557086in}{8.851114in}}%
\pgfpathlineto{\pgfqpoint{5.558207in}{8.840872in}}%
\pgfpathlineto{\pgfqpoint{5.559327in}{8.923178in}}%
\pgfpathlineto{\pgfqpoint{5.562689in}{8.950179in}}%
\pgfpathlineto{\pgfqpoint{5.563809in}{8.951483in}}%
\pgfpathlineto{\pgfqpoint{5.566050in}{8.941427in}}%
\pgfpathlineto{\pgfqpoint{5.567171in}{8.944034in}}%
\pgfpathlineto{\pgfqpoint{5.570533in}{8.945524in}}%
\pgfpathlineto{\pgfqpoint{5.571653in}{8.949807in}}%
\pgfpathlineto{\pgfqpoint{5.572774in}{8.944779in}}%
\pgfpathlineto{\pgfqpoint{5.575015in}{8.876252in}}%
\pgfpathlineto{\pgfqpoint{5.579497in}{8.814988in}}%
\pgfpathlineto{\pgfqpoint{5.581738in}{8.873645in}}%
\pgfpathlineto{\pgfqpoint{5.582858in}{8.869549in}}%
\pgfpathlineto{\pgfqpoint{5.586220in}{8.870852in}}%
\pgfpathlineto{\pgfqpoint{5.587340in}{8.823368in}}%
\pgfpathlineto{\pgfqpoint{5.588461in}{8.839755in}}%
\pgfpathlineto{\pgfqpoint{5.589582in}{8.845341in}}%
\pgfpathlineto{\pgfqpoint{5.590702in}{8.824671in}}%
\pgfpathlineto{\pgfqpoint{5.595184in}{8.849438in}}%
\pgfpathlineto{\pgfqpoint{5.596305in}{8.842920in}}%
\pgfpathlineto{\pgfqpoint{5.598546in}{8.849624in}}%
\pgfpathlineto{\pgfqpoint{5.601907in}{8.843293in}}%
\pgfpathlineto{\pgfqpoint{5.604148in}{8.881280in}}%
\pgfpathlineto{\pgfqpoint{5.605269in}{8.876811in}}%
\pgfpathlineto{\pgfqpoint{5.606389in}{8.857445in}}%
\pgfpathlineto{\pgfqpoint{5.609751in}{8.871225in}}%
\pgfpathlineto{\pgfqpoint{5.610872in}{8.853162in}}%
\pgfpathlineto{\pgfqpoint{5.611992in}{8.851858in}}%
\pgfpathlineto{\pgfqpoint{5.613113in}{8.835472in}}%
\pgfpathlineto{\pgfqpoint{5.614233in}{8.842548in}}%
\pgfpathlineto{\pgfqpoint{5.617595in}{8.811823in}}%
\pgfpathlineto{\pgfqpoint{5.618715in}{8.807726in}}%
\pgfpathlineto{\pgfqpoint{5.619836in}{8.825602in}}%
\pgfpathlineto{\pgfqpoint{5.620956in}{8.821506in}}%
\pgfpathlineto{\pgfqpoint{5.622077in}{8.830444in}}%
\pgfpathlineto{\pgfqpoint{5.625438in}{8.880908in}}%
\pgfpathlineto{\pgfqpoint{5.626559in}{8.874204in}}%
\pgfpathlineto{\pgfqpoint{5.627679in}{8.884073in}}%
\pgfpathlineto{\pgfqpoint{5.628800in}{8.884073in}}%
\pgfpathlineto{\pgfqpoint{5.629921in}{8.886680in}}%
\pgfpathlineto{\pgfqpoint{5.633282in}{8.886308in}}%
\pgfpathlineto{\pgfqpoint{5.635523in}{8.866197in}}%
\pgfpathlineto{\pgfqpoint{5.637764in}{8.884446in}}%
\pgfpathlineto{\pgfqpoint{5.642246in}{8.879977in}}%
\pgfpathlineto{\pgfqpoint{5.643367in}{8.872342in}}%
\pgfpathlineto{\pgfqpoint{5.644487in}{8.801953in}}%
\pgfpathlineto{\pgfqpoint{5.645608in}{8.838637in}}%
\pgfpathlineto{\pgfqpoint{5.650090in}{8.828396in}}%
\pgfpathlineto{\pgfqpoint{5.651211in}{8.836217in}}%
\pgfpathlineto{\pgfqpoint{5.652331in}{8.832120in}}%
\pgfpathlineto{\pgfqpoint{5.653452in}{8.815361in}}%
\pgfpathlineto{\pgfqpoint{5.656813in}{8.827278in}}%
\pgfpathlineto{\pgfqpoint{5.659054in}{8.829513in}}%
\pgfpathlineto{\pgfqpoint{5.660175in}{8.827092in}}%
\pgfpathlineto{\pgfqpoint{5.661295in}{8.833423in}}%
\pgfpathlineto{\pgfqpoint{5.664657in}{8.818154in}}%
\pgfpathlineto{\pgfqpoint{5.665777in}{8.817223in}}%
\pgfpathlineto{\pgfqpoint{5.666898in}{8.809216in}}%
\pgfpathlineto{\pgfqpoint{5.669139in}{8.779235in}}%
\pgfpathlineto{\pgfqpoint{5.672501in}{8.787242in}}%
\pgfpathlineto{\pgfqpoint{5.673621in}{8.777559in}}%
\pgfpathlineto{\pgfqpoint{5.675862in}{8.806236in}}%
\pgfpathlineto{\pgfqpoint{5.676983in}{8.800464in}}%
\pgfpathlineto{\pgfqpoint{5.680344in}{8.797484in}}%
\pgfpathlineto{\pgfqpoint{5.681465in}{8.786684in}}%
\pgfpathlineto{\pgfqpoint{5.684826in}{8.790594in}}%
\pgfpathlineto{\pgfqpoint{5.688188in}{8.786870in}}%
\pgfpathlineto{\pgfqpoint{5.689308in}{8.796739in}}%
\pgfpathlineto{\pgfqpoint{5.691549in}{8.765456in}}%
\pgfpathlineto{\pgfqpoint{5.692670in}{8.777373in}}%
\pgfpathlineto{\pgfqpoint{5.696032in}{8.768807in}}%
\pgfpathlineto{\pgfqpoint{5.697152in}{8.756517in}}%
\pgfpathlineto{\pgfqpoint{5.698273in}{8.755586in}}%
\pgfpathlineto{\pgfqpoint{5.699393in}{8.760055in}}%
\pgfpathlineto{\pgfqpoint{5.700514in}{8.738082in}}%
\pgfpathlineto{\pgfqpoint{5.703875in}{8.737710in}}%
\pgfpathlineto{\pgfqpoint{5.704996in}{8.760800in}}%
\pgfpathlineto{\pgfqpoint{5.706116in}{8.770669in}}%
\pgfpathlineto{\pgfqpoint{5.708357in}{8.720764in}}%
\pgfpathlineto{\pgfqpoint{5.711719in}{8.730075in}}%
\pgfpathlineto{\pgfqpoint{5.712840in}{8.737896in}}%
\pgfpathlineto{\pgfqpoint{5.713960in}{8.757635in}}%
\pgfpathlineto{\pgfqpoint{5.715081in}{8.760986in}}%
\pgfpathlineto{\pgfqpoint{5.719563in}{8.753910in}}%
\pgfpathlineto{\pgfqpoint{5.720683in}{8.767690in}}%
\pgfpathlineto{\pgfqpoint{5.721804in}{8.760986in}}%
\pgfpathlineto{\pgfqpoint{5.722924in}{8.750186in}}%
\pgfpathlineto{\pgfqpoint{5.727406in}{8.715923in}}%
\pgfpathlineto{\pgfqpoint{5.728527in}{8.697674in}}%
\pgfpathlineto{\pgfqpoint{5.729647in}{8.665273in}}%
\pgfpathlineto{\pgfqpoint{5.730768in}{8.655031in}}%
\pgfpathlineto{\pgfqpoint{5.731888in}{8.651307in}}%
\pgfpathlineto{\pgfqpoint{5.735250in}{8.658755in}}%
\pgfpathlineto{\pgfqpoint{5.736371in}{8.664900in}}%
\pgfpathlineto{\pgfqpoint{5.737491in}{8.637527in}}%
\pgfpathlineto{\pgfqpoint{5.738612in}{8.645348in}}%
\pgfpathlineto{\pgfqpoint{5.739732in}{8.638644in}}%
\pgfpathlineto{\pgfqpoint{5.744214in}{8.633989in}}%
\pgfpathlineto{\pgfqpoint{5.745335in}{8.640693in}}%
\pgfpathlineto{\pgfqpoint{5.746455in}{8.633989in}}%
\pgfpathlineto{\pgfqpoint{5.747576in}{8.500474in}}%
\pgfpathlineto{\pgfqpoint{5.750937in}{8.499729in}}%
\pgfpathlineto{\pgfqpoint{5.752058in}{8.501033in}}%
\pgfpathlineto{\pgfqpoint{5.753178in}{8.490977in}}%
\pgfpathlineto{\pgfqpoint{5.754299in}{8.462114in}}%
\pgfpathlineto{\pgfqpoint{5.755420in}{8.472914in}}%
\pgfpathlineto{\pgfqpoint{5.758781in}{8.494143in}}%
\pgfpathlineto{\pgfqpoint{5.759902in}{8.475894in}}%
\pgfpathlineto{\pgfqpoint{5.762143in}{8.488556in}}%
\pgfpathlineto{\pgfqpoint{5.763263in}{8.481480in}}%
\pgfpathlineto{\pgfqpoint{5.766625in}{8.453548in}}%
\pgfpathlineto{\pgfqpoint{5.767745in}{8.457645in}}%
\pgfpathlineto{\pgfqpoint{5.768866in}{8.451686in}}%
\pgfpathlineto{\pgfqpoint{5.769986in}{8.430830in}}%
\pgfpathlineto{\pgfqpoint{5.771107in}{8.458204in}}%
\pgfpathlineto{\pgfqpoint{5.775589in}{8.467328in}}%
\pgfpathlineto{\pgfqpoint{5.777830in}{8.484460in}}%
\pgfpathlineto{\pgfqpoint{5.778951in}{8.494329in}}%
\pgfpathlineto{\pgfqpoint{5.782312in}{8.510529in}}%
\pgfpathlineto{\pgfqpoint{5.784553in}{8.493025in}}%
\pgfpathlineto{\pgfqpoint{5.785674in}{8.506247in}}%
\pgfpathlineto{\pgfqpoint{5.786794in}{8.506060in}}%
\pgfpathlineto{\pgfqpoint{5.790156in}{8.509598in}}%
\pgfpathlineto{\pgfqpoint{5.791276in}{8.531013in}}%
\pgfpathlineto{\pgfqpoint{5.792397in}{8.536786in}}%
\pgfpathlineto{\pgfqpoint{5.793517in}{8.553731in}}%
\pgfpathlineto{\pgfqpoint{5.794638in}{8.557269in}}%
\pgfpathlineto{\pgfqpoint{5.798000in}{8.569745in}}%
\pgfpathlineto{\pgfqpoint{5.799120in}{8.577380in}}%
\pgfpathlineto{\pgfqpoint{5.801361in}{8.565462in}}%
\pgfpathlineto{\pgfqpoint{5.802482in}{8.577939in}}%
\pgfpathlineto{\pgfqpoint{5.805843in}{8.579801in}}%
\pgfpathlineto{\pgfqpoint{5.806964in}{8.573842in}}%
\pgfpathlineto{\pgfqpoint{5.809205in}{8.588925in}}%
\pgfpathlineto{\pgfqpoint{5.810325in}{8.608850in}}%
\pgfpathlineto{\pgfqpoint{5.813687in}{8.608664in}}%
\pgfpathlineto{\pgfqpoint{5.814807in}{8.598236in}}%
\pgfpathlineto{\pgfqpoint{5.815928in}{8.598608in}}%
\pgfpathlineto{\pgfqpoint{5.817049in}{8.595629in}}%
\pgfpathlineto{\pgfqpoint{5.821531in}{8.592277in}}%
\pgfpathlineto{\pgfqpoint{5.822651in}{8.598050in}}%
\pgfpathlineto{\pgfqpoint{5.823772in}{8.592463in}}%
\pgfpathlineto{\pgfqpoint{5.824892in}{8.612016in}}%
\pgfpathlineto{\pgfqpoint{5.826013in}{8.606802in}}%
\pgfpathlineto{\pgfqpoint{5.829374in}{8.598422in}}%
\pgfpathlineto{\pgfqpoint{5.830495in}{8.590787in}}%
\pgfpathlineto{\pgfqpoint{5.831615in}{8.590415in}}%
\pgfpathlineto{\pgfqpoint{5.832736in}{8.572352in}}%
\pgfpathlineto{\pgfqpoint{5.833856in}{8.583711in}}%
\pgfpathlineto{\pgfqpoint{5.837218in}{8.588925in}}%
\pgfpathlineto{\pgfqpoint{5.838339in}{8.604567in}}%
\pgfpathlineto{\pgfqpoint{5.839459in}{8.630823in}}%
\pgfpathlineto{\pgfqpoint{5.840580in}{8.636968in}}%
\pgfpathlineto{\pgfqpoint{5.841700in}{8.630451in}}%
\pgfpathlineto{\pgfqpoint{5.845062in}{8.638458in}}%
\pgfpathlineto{\pgfqpoint{5.848423in}{8.691715in}}%
\pgfpathlineto{\pgfqpoint{5.849544in}{8.697488in}}%
\pgfpathlineto{\pgfqpoint{5.852905in}{8.693391in}}%
\pgfpathlineto{\pgfqpoint{5.854026in}{8.702329in}}%
\pgfpathlineto{\pgfqpoint{5.855146in}{8.702702in}}%
\pgfpathlineto{\pgfqpoint{5.857388in}{8.688549in}}%
\pgfpathlineto{\pgfqpoint{5.860749in}{8.693019in}}%
\pgfpathlineto{\pgfqpoint{5.862990in}{8.667880in}}%
\pgfpathlineto{\pgfqpoint{5.864111in}{8.661921in}}%
\pgfpathlineto{\pgfqpoint{5.865231in}{8.672349in}}%
\pgfpathlineto{\pgfqpoint{5.868593in}{8.663038in}}%
\pgfpathlineto{\pgfqpoint{5.869713in}{8.678121in}}%
\pgfpathlineto{\pgfqpoint{5.873075in}{8.665459in}}%
\pgfpathlineto{\pgfqpoint{5.876436in}{8.664528in}}%
\pgfpathlineto{\pgfqpoint{5.877557in}{8.641437in}}%
\pgfpathlineto{\pgfqpoint{5.878678in}{8.654845in}}%
\pgfpathlineto{\pgfqpoint{5.879798in}{8.641065in}}%
\pgfpathlineto{\pgfqpoint{5.880919in}{8.661921in}}%
\pgfpathlineto{\pgfqpoint{5.884280in}{8.656148in}}%
\pgfpathlineto{\pgfqpoint{5.885401in}{8.678680in}}%
\pgfpathlineto{\pgfqpoint{5.886521in}{8.686501in}}%
\pgfpathlineto{\pgfqpoint{5.887642in}{8.685011in}}%
\pgfpathlineto{\pgfqpoint{5.888762in}{8.690225in}}%
\pgfpathlineto{\pgfqpoint{5.893244in}{8.694508in}}%
\pgfpathlineto{\pgfqpoint{5.894365in}{8.697115in}}%
\pgfpathlineto{\pgfqpoint{5.895485in}{8.705867in}}%
\pgfpathlineto{\pgfqpoint{5.896606in}{8.689667in}}%
\pgfpathlineto{\pgfqpoint{5.899968in}{8.697674in}}%
\pgfpathlineto{\pgfqpoint{5.901088in}{8.696743in}}%
\pgfpathlineto{\pgfqpoint{5.902209in}{8.703260in}}%
\pgfpathlineto{\pgfqpoint{5.907811in}{8.657452in}}%
\pgfpathlineto{\pgfqpoint{5.908932in}{8.611457in}}%
\pgfpathlineto{\pgfqpoint{5.911173in}{8.626727in}}%
\pgfpathlineto{\pgfqpoint{5.912293in}{8.625423in}}%
\pgfpathlineto{\pgfqpoint{5.915655in}{8.633244in}}%
\pgfpathlineto{\pgfqpoint{5.916775in}{8.633058in}}%
\pgfpathlineto{\pgfqpoint{5.917896in}{8.627099in}}%
\pgfpathlineto{\pgfqpoint{5.919017in}{8.650003in}}%
\pgfpathlineto{\pgfqpoint{5.920137in}{8.593581in}}%
\pgfpathlineto{\pgfqpoint{5.923499in}{8.551310in}}%
\pgfpathlineto{\pgfqpoint{5.924619in}{8.555407in}}%
\pgfpathlineto{\pgfqpoint{5.926860in}{8.611457in}}%
\pgfpathlineto{\pgfqpoint{5.927981in}{8.610154in}}%
\pgfpathlineto{\pgfqpoint{5.932463in}{8.582780in}}%
\pgfpathlineto{\pgfqpoint{5.934704in}{8.594512in}}%
\pgfpathlineto{\pgfqpoint{5.935824in}{8.624306in}}%
\pgfpathlineto{\pgfqpoint{5.939186in}{8.636223in}}%
\pgfpathlineto{\pgfqpoint{5.940307in}{8.651307in}}%
\pgfpathlineto{\pgfqpoint{5.941427in}{8.652983in}}%
\pgfpathlineto{\pgfqpoint{5.942548in}{8.662107in}}%
\pgfpathlineto{\pgfqpoint{5.943668in}{8.665087in}}%
\pgfpathlineto{\pgfqpoint{5.947030in}{8.668811in}}%
\pgfpathlineto{\pgfqpoint{5.948150in}{8.671790in}}%
\pgfpathlineto{\pgfqpoint{5.949271in}{8.677563in}}%
\pgfpathlineto{\pgfqpoint{5.950391in}{8.658941in}}%
\pgfpathlineto{\pgfqpoint{5.951512in}{8.674025in}}%
\pgfpathlineto{\pgfqpoint{5.955994in}{8.675514in}}%
\pgfpathlineto{\pgfqpoint{5.958235in}{8.683149in}}%
\pgfpathlineto{\pgfqpoint{5.959355in}{8.677190in}}%
\pgfpathlineto{\pgfqpoint{5.962717in}{8.671418in}}%
\pgfpathlineto{\pgfqpoint{5.963838in}{8.659686in}}%
\pgfpathlineto{\pgfqpoint{5.964958in}{8.666018in}}%
\pgfpathlineto{\pgfqpoint{5.966079in}{8.667880in}}%
\pgfpathlineto{\pgfqpoint{5.967199in}{8.695998in}}%
\pgfpathlineto{\pgfqpoint{5.970561in}{8.701584in}}%
\pgfpathlineto{\pgfqpoint{5.972802in}{8.682218in}}%
\pgfpathlineto{\pgfqpoint{5.973922in}{8.695253in}}%
\pgfpathlineto{\pgfqpoint{5.975043in}{8.693577in}}%
\pgfpathlineto{\pgfqpoint{5.978404in}{8.698046in}}%
\pgfpathlineto{\pgfqpoint{5.979525in}{8.692087in}}%
\pgfpathlineto{\pgfqpoint{5.980646in}{8.698977in}}%
\pgfpathlineto{\pgfqpoint{5.981766in}{8.698046in}}%
\pgfpathlineto{\pgfqpoint{5.982887in}{8.696184in}}%
\pgfpathlineto{\pgfqpoint{5.986248in}{8.693205in}}%
\pgfpathlineto{\pgfqpoint{5.987369in}{8.698977in}}%
\pgfpathlineto{\pgfqpoint{5.988489in}{8.689667in}}%
\pgfpathlineto{\pgfqpoint{5.990730in}{8.683149in}}%
\pgfpathlineto{\pgfqpoint{5.994092in}{8.695998in}}%
\pgfpathlineto{\pgfqpoint{5.995212in}{8.694881in}}%
\pgfpathlineto{\pgfqpoint{5.996333in}{8.697115in}}%
\pgfpathlineto{\pgfqpoint{5.997453in}{8.684267in}}%
\pgfpathlineto{\pgfqpoint{5.998574in}{8.690225in}}%
\pgfpathlineto{\pgfqpoint{6.003056in}{8.700281in}}%
\pgfpathlineto{\pgfqpoint{6.004177in}{8.707916in}}%
\pgfpathlineto{\pgfqpoint{6.005297in}{8.708847in}}%
\pgfpathlineto{\pgfqpoint{6.006418in}{8.688922in}}%
\pgfpathlineto{\pgfqpoint{6.009779in}{8.702888in}}%
\pgfpathlineto{\pgfqpoint{6.012020in}{8.659686in}}%
\pgfpathlineto{\pgfqpoint{6.013141in}{8.666018in}}%
\pgfpathlineto{\pgfqpoint{6.014261in}{8.663038in}}%
\pgfpathlineto{\pgfqpoint{6.017623in}{8.670114in}}%
\pgfpathlineto{\pgfqpoint{6.018743in}{8.663969in}}%
\pgfpathlineto{\pgfqpoint{6.020984in}{8.680170in}}%
\pgfpathlineto{\pgfqpoint{6.022105in}{8.666390in}}%
\pgfpathlineto{\pgfqpoint{6.025467in}{8.658755in}}%
\pgfpathlineto{\pgfqpoint{6.026587in}{8.674025in}}%
\pgfpathlineto{\pgfqpoint{6.027708in}{8.672907in}}%
\pgfpathlineto{\pgfqpoint{6.028828in}{8.657824in}}%
\pgfpathlineto{\pgfqpoint{6.029949in}{8.669742in}}%
\pgfpathlineto{\pgfqpoint{6.033310in}{8.665645in}}%
\pgfpathlineto{\pgfqpoint{6.034431in}{8.667507in}}%
\pgfpathlineto{\pgfqpoint{6.035551in}{8.681101in}}%
\pgfpathlineto{\pgfqpoint{6.036672in}{8.637899in}}%
\pgfpathlineto{\pgfqpoint{6.037792in}{8.634734in}}%
\pgfpathlineto{\pgfqpoint{6.041154in}{8.637155in}}%
\pgfpathlineto{\pgfqpoint{6.042274in}{8.618719in}}%
\pgfpathlineto{\pgfqpoint{6.043395in}{8.615554in}}%
\pgfpathlineto{\pgfqpoint{6.045636in}{8.605871in}}%
\pgfpathlineto{\pgfqpoint{6.048998in}{8.601402in}}%
\pgfpathlineto{\pgfqpoint{6.050118in}{8.604753in}}%
\pgfpathlineto{\pgfqpoint{6.051239in}{8.625609in}}%
\pgfpathlineto{\pgfqpoint{6.052359in}{8.724489in}}%
\pgfpathlineto{\pgfqpoint{6.053480in}{8.734730in}}%
\pgfpathlineto{\pgfqpoint{6.056841in}{8.729889in}}%
\pgfpathlineto{\pgfqpoint{6.057962in}{8.723744in}}%
\pgfpathlineto{\pgfqpoint{6.059082in}{8.724861in}}%
\pgfpathlineto{\pgfqpoint{6.060203in}{8.727096in}}%
\pgfpathlineto{\pgfqpoint{6.061323in}{8.718530in}}%
\pgfpathlineto{\pgfqpoint{6.064685in}{8.717971in}}%
\pgfpathlineto{\pgfqpoint{6.065806in}{8.714992in}}%
\pgfpathlineto{\pgfqpoint{6.066926in}{8.700653in}}%
\pgfpathlineto{\pgfqpoint{6.068047in}{8.698605in}}%
\pgfpathlineto{\pgfqpoint{6.069167in}{8.701771in}}%
\pgfpathlineto{\pgfqpoint{6.072529in}{8.728399in}}%
\pgfpathlineto{\pgfqpoint{6.073649in}{8.729516in}}%
\pgfpathlineto{\pgfqpoint{6.075890in}{8.783891in}}%
\pgfpathlineto{\pgfqpoint{6.077011in}{8.790967in}}%
\pgfpathlineto{\pgfqpoint{6.080372in}{8.825230in}}%
\pgfpathlineto{\pgfqpoint{6.081493in}{8.826161in}}%
\pgfpathlineto{\pgfqpoint{6.082613in}{8.812009in}}%
\pgfpathlineto{\pgfqpoint{6.083734in}{8.813685in}}%
\pgfpathlineto{\pgfqpoint{6.084855in}{8.799905in}}%
\pgfpathlineto{\pgfqpoint{6.089337in}{8.812754in}}%
\pgfpathlineto{\pgfqpoint{6.090457in}{8.833423in}}%
\pgfpathlineto{\pgfqpoint{6.092698in}{8.833051in}}%
\pgfpathlineto{\pgfqpoint{6.096060in}{8.820016in}}%
\pgfpathlineto{\pgfqpoint{6.097180in}{8.808657in}}%
\pgfpathlineto{\pgfqpoint{6.099421in}{8.827092in}}%
\pgfpathlineto{\pgfqpoint{6.100542in}{8.815174in}}%
\pgfpathlineto{\pgfqpoint{6.103903in}{8.818154in}}%
\pgfpathlineto{\pgfqpoint{6.105024in}{8.823182in}}%
\pgfpathlineto{\pgfqpoint{6.106145in}{8.858562in}}%
\pgfpathlineto{\pgfqpoint{6.107265in}{8.869735in}}%
\pgfpathlineto{\pgfqpoint{6.108386in}{8.867128in}}%
\pgfpathlineto{\pgfqpoint{6.111747in}{8.845900in}}%
\pgfpathlineto{\pgfqpoint{6.113988in}{8.854652in}}%
\pgfpathlineto{\pgfqpoint{6.115109in}{8.870294in}}%
\pgfpathlineto{\pgfqpoint{6.116229in}{8.871225in}}%
\pgfpathlineto{\pgfqpoint{6.119591in}{8.863217in}}%
\pgfpathlineto{\pgfqpoint{6.121832in}{8.876997in}}%
\pgfpathlineto{\pgfqpoint{6.122952in}{8.863776in}}%
\pgfpathlineto{\pgfqpoint{6.124073in}{8.870852in}}%
\pgfpathlineto{\pgfqpoint{6.128555in}{8.871038in}}%
\pgfpathlineto{\pgfqpoint{6.130796in}{8.852045in}}%
\pgfpathlineto{\pgfqpoint{6.131917in}{8.854838in}}%
\pgfpathlineto{\pgfqpoint{6.136399in}{8.877556in}}%
\pgfpathlineto{\pgfqpoint{6.137519in}{8.899715in}}%
\pgfpathlineto{\pgfqpoint{6.138640in}{8.882770in}}%
\pgfpathlineto{\pgfqpoint{6.143122in}{8.892453in}}%
\pgfpathlineto{\pgfqpoint{6.144242in}{8.906605in}}%
\pgfpathlineto{\pgfqpoint{6.145363in}{8.911261in}}%
\pgfpathlineto{\pgfqpoint{6.146484in}{8.910702in}}%
\pgfpathlineto{\pgfqpoint{6.147604in}{8.906047in}}%
\pgfpathlineto{\pgfqpoint{6.152086in}{8.905674in}}%
\pgfpathlineto{\pgfqpoint{6.153207in}{8.921688in}}%
\pgfpathlineto{\pgfqpoint{6.155448in}{8.898598in}}%
\pgfpathlineto{\pgfqpoint{6.158809in}{8.894501in}}%
\pgfpathlineto{\pgfqpoint{6.159930in}{8.920571in}}%
\pgfpathlineto{\pgfqpoint{6.161050in}{8.910888in}}%
\pgfpathlineto{\pgfqpoint{6.162171in}{8.911633in}}%
\pgfpathlineto{\pgfqpoint{6.163291in}{8.910143in}}%
\pgfpathlineto{\pgfqpoint{6.166653in}{8.917964in}}%
\pgfpathlineto{\pgfqpoint{6.167774in}{8.901764in}}%
\pgfpathlineto{\pgfqpoint{6.168894in}{8.908654in}}%
\pgfpathlineto{\pgfqpoint{6.170015in}{8.904184in}}%
\pgfpathlineto{\pgfqpoint{6.171135in}{8.931558in}}%
\pgfpathlineto{\pgfqpoint{6.175617in}{8.925785in}}%
\pgfpathlineto{\pgfqpoint{6.176738in}{8.927275in}}%
\pgfpathlineto{\pgfqpoint{6.178979in}{8.939565in}}%
\pgfpathlineto{\pgfqpoint{6.182340in}{8.947200in}}%
\pgfpathlineto{\pgfqpoint{6.183461in}{8.956138in}}%
\pgfpathlineto{\pgfqpoint{6.184581in}{8.959676in}}%
\pgfpathlineto{\pgfqpoint{6.185702in}{8.958000in}}%
\pgfpathlineto{\pgfqpoint{6.186822in}{8.961538in}}%
\pgfpathlineto{\pgfqpoint{6.191305in}{8.966193in}}%
\pgfpathlineto{\pgfqpoint{6.192425in}{8.964331in}}%
\pgfpathlineto{\pgfqpoint{6.193546in}{8.967683in}}%
\pgfpathlineto{\pgfqpoint{6.194666in}{8.962469in}}%
\pgfpathlineto{\pgfqpoint{6.198028in}{8.969732in}}%
\pgfpathlineto{\pgfqpoint{6.199148in}{8.967869in}}%
\pgfpathlineto{\pgfqpoint{6.200269in}{9.001202in}}%
\pgfpathlineto{\pgfqpoint{6.201389in}{8.968614in}}%
\pgfpathlineto{\pgfqpoint{6.202510in}{8.964704in}}%
\pgfpathlineto{\pgfqpoint{6.205871in}{8.957814in}}%
\pgfpathlineto{\pgfqpoint{6.206992in}{8.959304in}}%
\pgfpathlineto{\pgfqpoint{6.208113in}{8.949620in}}%
\pgfpathlineto{\pgfqpoint{6.209233in}{8.954276in}}%
\pgfpathlineto{\pgfqpoint{6.210354in}{8.955579in}}%
\pgfpathlineto{\pgfqpoint{6.213715in}{8.952600in}}%
\pgfpathlineto{\pgfqpoint{6.214836in}{8.961166in}}%
\pgfpathlineto{\pgfqpoint{6.215956in}{8.952972in}}%
\pgfpathlineto{\pgfqpoint{6.217077in}{8.962655in}}%
\pgfpathlineto{\pgfqpoint{6.218197in}{8.953345in}}%
\pgfpathlineto{\pgfqpoint{6.221559in}{8.946082in}}%
\pgfpathlineto{\pgfqpoint{6.222679in}{8.922433in}}%
\pgfpathlineto{\pgfqpoint{6.224920in}{8.928020in}}%
\pgfpathlineto{\pgfqpoint{6.226041in}{8.934537in}}%
\pgfpathlineto{\pgfqpoint{6.229403in}{8.923737in}}%
\pgfpathlineto{\pgfqpoint{6.230523in}{8.942358in}}%
\pgfpathlineto{\pgfqpoint{6.231644in}{8.935282in}}%
\pgfpathlineto{\pgfqpoint{6.232764in}{8.952600in}}%
\pgfpathlineto{\pgfqpoint{6.233885in}{8.950738in}}%
\pgfpathlineto{\pgfqpoint{6.237246in}{8.941427in}}%
\pgfpathlineto{\pgfqpoint{6.239487in}{8.932303in}}%
\pgfpathlineto{\pgfqpoint{6.240608in}{8.935096in}}%
\pgfpathlineto{\pgfqpoint{6.241728in}{8.932489in}}%
\pgfpathlineto{\pgfqpoint{6.245090in}{8.927461in}}%
\pgfpathlineto{\pgfqpoint{6.246210in}{8.923364in}}%
\pgfpathlineto{\pgfqpoint{6.248451in}{8.896922in}}%
\pgfpathlineto{\pgfqpoint{6.252934in}{8.912564in}}%
\pgfpathlineto{\pgfqpoint{6.254054in}{8.896736in}}%
\pgfpathlineto{\pgfqpoint{6.255175in}{8.892453in}}%
\pgfpathlineto{\pgfqpoint{6.256295in}{8.973083in}}%
\pgfpathlineto{\pgfqpoint{6.257416in}{8.965262in}}%
\pgfpathlineto{\pgfqpoint{6.261898in}{8.984070in}}%
\pgfpathlineto{\pgfqpoint{6.263018in}{8.982022in}}%
\pgfpathlineto{\pgfqpoint{6.264139in}{8.978670in}}%
\pgfpathlineto{\pgfqpoint{6.265259in}{8.959117in}}%
\pgfpathlineto{\pgfqpoint{6.268621in}{8.958745in}}%
\pgfpathlineto{\pgfqpoint{6.269742in}{8.964331in}}%
\pgfpathlineto{\pgfqpoint{6.271983in}{8.942544in}}%
\pgfpathlineto{\pgfqpoint{6.273103in}{8.942358in}}%
\pgfpathlineto{\pgfqpoint{6.276465in}{8.939565in}}%
\pgfpathlineto{\pgfqpoint{6.278706in}{8.948317in}}%
\pgfpathlineto{\pgfqpoint{6.280947in}{8.927461in}}%
\pgfpathlineto{\pgfqpoint{6.284308in}{8.942544in}}%
\pgfpathlineto{\pgfqpoint{6.285429in}{8.939006in}}%
\pgfpathlineto{\pgfqpoint{6.286549in}{8.907164in}}%
\pgfpathlineto{\pgfqpoint{6.287670in}{8.907350in}}%
\pgfpathlineto{\pgfqpoint{6.288790in}{8.914985in}}%
\pgfpathlineto{\pgfqpoint{6.292152in}{8.918150in}}%
\pgfpathlineto{\pgfqpoint{6.293273in}{8.922247in}}%
\pgfpathlineto{\pgfqpoint{6.294393in}{8.920757in}}%
\pgfpathlineto{\pgfqpoint{6.295514in}{8.926530in}}%
\pgfpathlineto{\pgfqpoint{6.296634in}{8.926902in}}%
\pgfpathlineto{\pgfqpoint{6.302237in}{8.917592in}}%
\pgfpathlineto{\pgfqpoint{6.303357in}{8.941613in}}%
\pgfpathlineto{\pgfqpoint{6.304478in}{8.945524in}}%
\pgfpathlineto{\pgfqpoint{6.307839in}{8.954090in}}%
\pgfpathlineto{\pgfqpoint{6.308960in}{8.951855in}}%
\pgfpathlineto{\pgfqpoint{6.310080in}{8.969173in}}%
\pgfpathlineto{\pgfqpoint{6.311201in}{8.971780in}}%
\pgfpathlineto{\pgfqpoint{6.312322in}{8.978297in}}%
\pgfpathlineto{\pgfqpoint{6.315683in}{8.975690in}}%
\pgfpathlineto{\pgfqpoint{6.317924in}{8.987794in}}%
\pgfpathlineto{\pgfqpoint{6.319045in}{8.985187in}}%
\pgfpathlineto{\pgfqpoint{6.320165in}{8.998781in}}%
\pgfpathlineto{\pgfqpoint{6.323527in}{9.006602in}}%
\pgfpathlineto{\pgfqpoint{6.324647in}{9.017961in}}%
\pgfpathlineto{\pgfqpoint{6.325768in}{9.012561in}}%
\pgfpathlineto{\pgfqpoint{6.326888in}{9.013119in}}%
\pgfpathlineto{\pgfqpoint{6.328009in}{9.012747in}}%
\pgfpathlineto{\pgfqpoint{6.332491in}{9.028202in}}%
\pgfpathlineto{\pgfqpoint{6.333612in}{9.044217in}}%
\pgfpathlineto{\pgfqpoint{6.334732in}{9.038072in}}%
\pgfpathlineto{\pgfqpoint{6.335853in}{9.049058in}}%
\pgfpathlineto{\pgfqpoint{6.339214in}{9.064514in}}%
\pgfpathlineto{\pgfqpoint{6.341455in}{9.066562in}}%
\pgfpathlineto{\pgfqpoint{6.342576in}{9.045707in}}%
\pgfpathlineto{\pgfqpoint{6.343696in}{9.056507in}}%
\pgfpathlineto{\pgfqpoint{6.347058in}{9.055948in}}%
\pgfpathlineto{\pgfqpoint{6.348178in}{9.053341in}}%
\pgfpathlineto{\pgfqpoint{6.350419in}{9.075314in}}%
\pgfpathlineto{\pgfqpoint{6.351540in}{9.073639in}}%
\pgfpathlineto{\pgfqpoint{6.354902in}{9.072521in}}%
\pgfpathlineto{\pgfqpoint{6.357143in}{9.085370in}}%
\pgfpathlineto{\pgfqpoint{6.358263in}{9.074942in}}%
\pgfpathlineto{\pgfqpoint{6.359384in}{9.079225in}}%
\pgfpathlineto{\pgfqpoint{6.362745in}{9.068611in}}%
\pgfpathlineto{\pgfqpoint{6.363866in}{9.075873in}}%
\pgfpathlineto{\pgfqpoint{6.364986in}{9.074011in}}%
\pgfpathlineto{\pgfqpoint{6.366107in}{9.047755in}}%
\pgfpathlineto{\pgfqpoint{6.367227in}{9.064887in}}%
\pgfpathlineto{\pgfqpoint{6.370589in}{9.072707in}}%
\pgfpathlineto{\pgfqpoint{6.371709in}{9.072894in}}%
\pgfpathlineto{\pgfqpoint{6.372830in}{9.074011in}}%
\pgfpathlineto{\pgfqpoint{6.373951in}{9.078294in}}%
\pgfpathlineto{\pgfqpoint{6.375071in}{9.086115in}}%
\pgfpathlineto{\pgfqpoint{6.378433in}{9.083880in}}%
\pgfpathlineto{\pgfqpoint{6.379553in}{9.085556in}}%
\pgfpathlineto{\pgfqpoint{6.380674in}{9.081087in}}%
\pgfpathlineto{\pgfqpoint{6.381794in}{9.060604in}}%
\pgfpathlineto{\pgfqpoint{6.382915in}{9.055762in}}%
\pgfpathlineto{\pgfqpoint{6.386276in}{9.076990in}}%
\pgfpathlineto{\pgfqpoint{6.387397in}{9.101012in}}%
\pgfpathlineto{\pgfqpoint{6.388517in}{9.111812in}}%
\pgfpathlineto{\pgfqpoint{6.390758in}{9.075501in}}%
\pgfpathlineto{\pgfqpoint{6.395241in}{9.074011in}}%
\pgfpathlineto{\pgfqpoint{6.396361in}{9.073639in}}%
\pgfpathlineto{\pgfqpoint{6.398602in}{9.076990in}}%
\pgfpathlineto{\pgfqpoint{6.403084in}{9.076059in}}%
\pgfpathlineto{\pgfqpoint{6.406446in}{9.089094in}}%
\pgfpathlineto{\pgfqpoint{6.410928in}{9.074197in}}%
\pgfpathlineto{\pgfqpoint{6.412048in}{9.072707in}}%
\pgfpathlineto{\pgfqpoint{6.413169in}{9.057810in}}%
\pgfpathlineto{\pgfqpoint{6.414290in}{9.055017in}}%
\pgfpathlineto{\pgfqpoint{6.417651in}{9.081087in}}%
\pgfpathlineto{\pgfqpoint{6.418772in}{9.096543in}}%
\pgfpathlineto{\pgfqpoint{6.419892in}{9.098219in}}%
\pgfpathlineto{\pgfqpoint{6.421013in}{9.090025in}}%
\pgfpathlineto{\pgfqpoint{6.422133in}{9.104550in}}%
\pgfpathlineto{\pgfqpoint{6.425495in}{9.120378in}}%
\pgfpathlineto{\pgfqpoint{6.426615in}{9.140862in}}%
\pgfpathlineto{\pgfqpoint{6.427736in}{9.130620in}}%
\pgfpathlineto{\pgfqpoint{6.429977in}{9.130061in}}%
\pgfpathlineto{\pgfqpoint{6.433338in}{9.126523in}}%
\pgfpathlineto{\pgfqpoint{6.434459in}{9.135275in}}%
\pgfpathlineto{\pgfqpoint{6.436700in}{9.161531in}}%
\pgfpathlineto{\pgfqpoint{6.437821in}{9.167304in}}%
\pgfpathlineto{\pgfqpoint{6.441182in}{9.168794in}}%
\pgfpathlineto{\pgfqpoint{6.442303in}{9.184808in}}%
\pgfpathlineto{\pgfqpoint{6.443423in}{9.177173in}}%
\pgfpathlineto{\pgfqpoint{6.445664in}{9.193374in}}%
\pgfpathlineto{\pgfqpoint{6.449026in}{9.195981in}}%
\pgfpathlineto{\pgfqpoint{6.450146in}{9.199519in}}%
\pgfpathlineto{\pgfqpoint{6.451267in}{9.200822in}}%
\pgfpathlineto{\pgfqpoint{6.452387in}{9.194491in}}%
\pgfpathlineto{\pgfqpoint{6.453508in}{9.217209in}}%
\pgfpathlineto{\pgfqpoint{6.457990in}{9.195981in}}%
\pgfpathlineto{\pgfqpoint{6.459111in}{9.203057in}}%
\pgfpathlineto{\pgfqpoint{6.460231in}{9.199705in}}%
\pgfpathlineto{\pgfqpoint{6.461352in}{9.203243in}}%
\pgfpathlineto{\pgfqpoint{6.464713in}{9.208457in}}%
\pgfpathlineto{\pgfqpoint{6.465834in}{9.235272in}}%
\pgfpathlineto{\pgfqpoint{6.466954in}{9.229313in}}%
\pgfpathlineto{\pgfqpoint{6.468075in}{9.268604in}}%
\pgfpathlineto{\pgfqpoint{6.469195in}{9.270466in}}%
\pgfpathlineto{\pgfqpoint{6.472557in}{9.257245in}}%
\pgfpathlineto{\pgfqpoint{6.474798in}{9.270466in}}%
\pgfpathlineto{\pgfqpoint{6.475919in}{9.273818in}}%
\pgfpathlineto{\pgfqpoint{6.477039in}{9.282011in}}%
\pgfpathlineto{\pgfqpoint{6.480401in}{9.279404in}}%
\pgfpathlineto{\pgfqpoint{6.481521in}{9.262831in}}%
\pgfpathlineto{\pgfqpoint{6.482642in}{9.258362in}}%
\pgfpathlineto{\pgfqpoint{6.483762in}{9.233410in}}%
\pgfpathlineto{\pgfqpoint{6.484883in}{9.229127in}}%
\pgfpathlineto{\pgfqpoint{6.488244in}{9.236017in}}%
\pgfpathlineto{\pgfqpoint{6.489365in}{9.233596in}}%
\pgfpathlineto{\pgfqpoint{6.490485in}{9.224471in}}%
\pgfpathlineto{\pgfqpoint{6.491606in}{9.229871in}}%
\pgfpathlineto{\pgfqpoint{6.492726in}{9.232292in}}%
\pgfpathlineto{\pgfqpoint{6.496088in}{9.236948in}}%
\pgfpathlineto{\pgfqpoint{6.497209in}{9.245327in}}%
\pgfpathlineto{\pgfqpoint{6.498329in}{9.234713in}}%
\pgfpathlineto{\pgfqpoint{6.500570in}{9.228382in}}%
\pgfpathlineto{\pgfqpoint{6.503932in}{9.228196in}}%
\pgfpathlineto{\pgfqpoint{6.506173in}{9.285177in}}%
\pgfpathlineto{\pgfqpoint{6.507293in}{9.305288in}}%
\pgfpathlineto{\pgfqpoint{6.508414in}{9.308081in}}%
\pgfpathlineto{\pgfqpoint{6.511775in}{9.321302in}}%
\pgfpathlineto{\pgfqpoint{6.512896in}{9.323351in}}%
\pgfpathlineto{\pgfqpoint{6.514016in}{9.314412in}}%
\pgfpathlineto{\pgfqpoint{6.515137in}{9.321116in}}%
\pgfpathlineto{\pgfqpoint{6.516257in}{9.320557in}}%
\pgfpathlineto{\pgfqpoint{6.519619in}{9.328751in}}%
\pgfpathlineto{\pgfqpoint{6.520740in}{9.335454in}}%
\pgfpathlineto{\pgfqpoint{6.521860in}{9.306591in}}%
\pgfpathlineto{\pgfqpoint{6.522981in}{9.295046in}}%
\pgfpathlineto{\pgfqpoint{6.524101in}{9.319999in}}%
\pgfpathlineto{\pgfqpoint{6.527463in}{9.341041in}}%
\pgfpathlineto{\pgfqpoint{6.529704in}{9.319813in}}%
\pgfpathlineto{\pgfqpoint{6.530824in}{9.319626in}}%
\pgfpathlineto{\pgfqpoint{6.531945in}{9.323909in}}%
\pgfpathlineto{\pgfqpoint{6.536427in}{9.320930in}}%
\pgfpathlineto{\pgfqpoint{6.538668in}{9.341413in}}%
\pgfpathlineto{\pgfqpoint{6.539789in}{9.334337in}}%
\pgfpathlineto{\pgfqpoint{6.539789in}{9.334337in}}%
\pgfusepath{stroke}%
\end{pgfscope}%
\begin{pgfscope}%
\pgfpathrectangle{\pgfqpoint{3.966666in}{8.286757in}}{\pgfqpoint{2.695652in}{1.104878in}}%
\pgfusepath{clip}%
\pgfsetroundcap%
\pgfsetroundjoin%
\pgfsetlinewidth{1.505625pt}%
\definecolor{currentstroke}{rgb}{1.000000,0.647059,0.000000}%
\pgfsetstrokecolor{currentstroke}%
\pgfsetdash{}{0pt}%
\pgfpathmoveto{\pgfqpoint{4.089196in}{8.336979in}}%
\pgfpathlineto{\pgfqpoint{4.090316in}{8.337258in}}%
\pgfpathlineto{\pgfqpoint{4.091437in}{8.340455in}}%
\pgfpathlineto{\pgfqpoint{4.092557in}{8.339818in}}%
\pgfpathlineto{\pgfqpoint{4.095919in}{8.339846in}}%
\pgfpathlineto{\pgfqpoint{4.098160in}{8.341873in}}%
\pgfpathlineto{\pgfqpoint{4.100401in}{8.346331in}}%
\pgfpathlineto{\pgfqpoint{4.104883in}{8.348747in}}%
\pgfpathlineto{\pgfqpoint{4.108245in}{8.354927in}}%
\pgfpathlineto{\pgfqpoint{4.112727in}{8.355029in}}%
\pgfpathlineto{\pgfqpoint{4.114968in}{8.356575in}}%
\pgfpathlineto{\pgfqpoint{4.116088in}{8.357007in}}%
\pgfpathlineto{\pgfqpoint{4.119450in}{8.356756in}}%
\pgfpathlineto{\pgfqpoint{4.121691in}{8.358322in}}%
\pgfpathlineto{\pgfqpoint{4.123932in}{8.361551in}}%
\pgfpathlineto{\pgfqpoint{4.127294in}{8.363033in}}%
\pgfpathlineto{\pgfqpoint{4.130655in}{8.367228in}}%
\pgfpathlineto{\pgfqpoint{4.131776in}{8.368296in}}%
\pgfpathlineto{\pgfqpoint{4.135137in}{8.369438in}}%
\pgfpathlineto{\pgfqpoint{4.139619in}{8.373792in}}%
\pgfpathlineto{\pgfqpoint{4.144101in}{8.374988in}}%
\pgfpathlineto{\pgfqpoint{4.147463in}{8.378303in}}%
\pgfpathlineto{\pgfqpoint{4.150825in}{8.379842in}}%
\pgfpathlineto{\pgfqpoint{4.154186in}{8.383060in}}%
\pgfpathlineto{\pgfqpoint{4.155307in}{8.383865in}}%
\pgfpathlineto{\pgfqpoint{4.160909in}{8.385324in}}%
\pgfpathlineto{\pgfqpoint{4.163150in}{8.386725in}}%
\pgfpathlineto{\pgfqpoint{4.166512in}{8.387276in}}%
\pgfpathlineto{\pgfqpoint{4.168753in}{8.389938in}}%
\pgfpathlineto{\pgfqpoint{4.170994in}{8.393319in}}%
\pgfpathlineto{\pgfqpoint{4.174356in}{8.395112in}}%
\pgfpathlineto{\pgfqpoint{4.178838in}{8.401487in}}%
\pgfpathlineto{\pgfqpoint{4.182199in}{8.403386in}}%
\pgfpathlineto{\pgfqpoint{4.186682in}{8.409938in}}%
\pgfpathlineto{\pgfqpoint{4.190043in}{8.411381in}}%
\pgfpathlineto{\pgfqpoint{4.193405in}{8.415555in}}%
\pgfpathlineto{\pgfqpoint{4.199007in}{8.417500in}}%
\pgfpathlineto{\pgfqpoint{4.202369in}{8.420743in}}%
\pgfpathlineto{\pgfqpoint{4.205730in}{8.421858in}}%
\pgfpathlineto{\pgfqpoint{4.210213in}{8.426011in}}%
\pgfpathlineto{\pgfqpoint{4.213574in}{8.426876in}}%
\pgfpathlineto{\pgfqpoint{4.216936in}{8.430213in}}%
\pgfpathlineto{\pgfqpoint{4.218056in}{8.431573in}}%
\pgfpathlineto{\pgfqpoint{4.221418in}{8.432908in}}%
\pgfpathlineto{\pgfqpoint{4.225900in}{8.438355in}}%
\pgfpathlineto{\pgfqpoint{4.229262in}{8.439514in}}%
\pgfpathlineto{\pgfqpoint{4.233744in}{8.443530in}}%
\pgfpathlineto{\pgfqpoint{4.238226in}{8.444894in}}%
\pgfpathlineto{\pgfqpoint{4.240467in}{8.445616in}}%
\pgfpathlineto{\pgfqpoint{4.257275in}{8.447573in}}%
\pgfpathlineto{\pgfqpoint{4.265118in}{8.447693in}}%
\pgfpathlineto{\pgfqpoint{4.272962in}{8.448393in}}%
\pgfpathlineto{\pgfqpoint{4.277444in}{8.448882in}}%
\pgfpathlineto{\pgfqpoint{4.280806in}{8.449776in}}%
\pgfpathlineto{\pgfqpoint{4.287529in}{8.450677in}}%
\pgfpathlineto{\pgfqpoint{4.288649in}{8.451132in}}%
\pgfpathlineto{\pgfqpoint{4.293132in}{8.452315in}}%
\pgfpathlineto{\pgfqpoint{4.312181in}{8.456974in}}%
\pgfpathlineto{\pgfqpoint{4.318904in}{8.457576in}}%
\pgfpathlineto{\pgfqpoint{4.320024in}{8.457982in}}%
\pgfpathlineto{\pgfqpoint{4.332350in}{8.459781in}}%
\pgfpathlineto{\pgfqpoint{4.335712in}{8.460293in}}%
\pgfpathlineto{\pgfqpoint{4.342435in}{8.460868in}}%
\pgfpathlineto{\pgfqpoint{4.343555in}{8.461116in}}%
\pgfpathlineto{\pgfqpoint{4.354761in}{8.462147in}}%
\pgfpathlineto{\pgfqpoint{4.359243in}{8.463062in}}%
\pgfpathlineto{\pgfqpoint{4.372689in}{8.464530in}}%
\pgfpathlineto{\pgfqpoint{4.374930in}{8.465228in}}%
\pgfpathlineto{\pgfqpoint{4.381653in}{8.466416in}}%
\pgfpathlineto{\pgfqpoint{4.387256in}{8.466958in}}%
\pgfpathlineto{\pgfqpoint{4.396220in}{8.467701in}}%
\pgfpathlineto{\pgfqpoint{4.405184in}{8.469219in}}%
\pgfpathlineto{\pgfqpoint{4.414149in}{8.470442in}}%
\pgfpathlineto{\pgfqpoint{4.442162in}{8.470799in}}%
\pgfpathlineto{\pgfqpoint{4.448885in}{8.470363in}}%
\pgfpathlineto{\pgfqpoint{4.481380in}{8.471444in}}%
\pgfpathlineto{\pgfqpoint{4.504911in}{8.473565in}}%
\pgfpathlineto{\pgfqpoint{4.516117in}{8.476178in}}%
\pgfpathlineto{\pgfqpoint{4.527322in}{8.477279in}}%
\pgfpathlineto{\pgfqpoint{4.569902in}{8.486502in}}%
\pgfpathlineto{\pgfqpoint{4.571022in}{8.486979in}}%
\pgfpathlineto{\pgfqpoint{4.575504in}{8.488002in}}%
\pgfpathlineto{\pgfqpoint{4.578866in}{8.489545in}}%
\pgfpathlineto{\pgfqpoint{4.583348in}{8.490541in}}%
\pgfpathlineto{\pgfqpoint{4.586710in}{8.492080in}}%
\pgfpathlineto{\pgfqpoint{4.591192in}{8.493174in}}%
\pgfpathlineto{\pgfqpoint{4.593433in}{8.494327in}}%
\pgfpathlineto{\pgfqpoint{4.599036in}{8.495480in}}%
\pgfpathlineto{\pgfqpoint{4.602397in}{8.496996in}}%
\pgfpathlineto{\pgfqpoint{4.606879in}{8.497917in}}%
\pgfpathlineto{\pgfqpoint{4.610241in}{8.499309in}}%
\pgfpathlineto{\pgfqpoint{4.615843in}{8.500465in}}%
\pgfpathlineto{\pgfqpoint{4.618084in}{8.501418in}}%
\pgfpathlineto{\pgfqpoint{4.622567in}{8.502456in}}%
\pgfpathlineto{\pgfqpoint{4.625928in}{8.504098in}}%
\pgfpathlineto{\pgfqpoint{4.630410in}{8.505202in}}%
\pgfpathlineto{\pgfqpoint{4.633772in}{8.507027in}}%
\pgfpathlineto{\pgfqpoint{4.638254in}{8.508312in}}%
\pgfpathlineto{\pgfqpoint{4.641616in}{8.510219in}}%
\pgfpathlineto{\pgfqpoint{4.646098in}{8.511517in}}%
\pgfpathlineto{\pgfqpoint{4.649459in}{8.513749in}}%
\pgfpathlineto{\pgfqpoint{4.652821in}{8.514564in}}%
\pgfpathlineto{\pgfqpoint{4.657303in}{8.517872in}}%
\pgfpathlineto{\pgfqpoint{4.661785in}{8.518749in}}%
\pgfpathlineto{\pgfqpoint{4.665147in}{8.521313in}}%
\pgfpathlineto{\pgfqpoint{4.669629in}{8.523033in}}%
\pgfpathlineto{\pgfqpoint{4.672990in}{8.525589in}}%
\pgfpathlineto{\pgfqpoint{4.677472in}{8.527369in}}%
\pgfpathlineto{\pgfqpoint{4.680834in}{8.529568in}}%
\pgfpathlineto{\pgfqpoint{4.685316in}{8.531024in}}%
\pgfpathlineto{\pgfqpoint{4.688678in}{8.533051in}}%
\pgfpathlineto{\pgfqpoint{4.693160in}{8.534304in}}%
\pgfpathlineto{\pgfqpoint{4.696521in}{8.536428in}}%
\pgfpathlineto{\pgfqpoint{4.701003in}{8.537884in}}%
\pgfpathlineto{\pgfqpoint{4.704365in}{8.539363in}}%
\pgfpathlineto{\pgfqpoint{4.708847in}{8.540996in}}%
\pgfpathlineto{\pgfqpoint{4.712209in}{8.543397in}}%
\pgfpathlineto{\pgfqpoint{4.716691in}{8.545052in}}%
\pgfpathlineto{\pgfqpoint{4.720052in}{8.547095in}}%
\pgfpathlineto{\pgfqpoint{4.724535in}{8.548405in}}%
\pgfpathlineto{\pgfqpoint{4.727896in}{8.550465in}}%
\pgfpathlineto{\pgfqpoint{4.732378in}{8.551789in}}%
\pgfpathlineto{\pgfqpoint{4.735740in}{8.553738in}}%
\pgfpathlineto{\pgfqpoint{4.740222in}{8.555094in}}%
\pgfpathlineto{\pgfqpoint{4.743584in}{8.557068in}}%
\pgfpathlineto{\pgfqpoint{4.748066in}{8.558390in}}%
\pgfpathlineto{\pgfqpoint{4.751427in}{8.560278in}}%
\pgfpathlineto{\pgfqpoint{4.755909in}{8.561437in}}%
\pgfpathlineto{\pgfqpoint{4.759271in}{8.563075in}}%
\pgfpathlineto{\pgfqpoint{4.763753in}{8.564081in}}%
\pgfpathlineto{\pgfqpoint{4.767115in}{8.565484in}}%
\pgfpathlineto{\pgfqpoint{4.772717in}{8.566468in}}%
\pgfpathlineto{\pgfqpoint{4.774958in}{8.567506in}}%
\pgfpathlineto{\pgfqpoint{4.779440in}{8.568580in}}%
\pgfpathlineto{\pgfqpoint{4.782802in}{8.570306in}}%
\pgfpathlineto{\pgfqpoint{4.787284in}{8.571523in}}%
\pgfpathlineto{\pgfqpoint{4.790646in}{8.573476in}}%
\pgfpathlineto{\pgfqpoint{4.795128in}{8.574665in}}%
\pgfpathlineto{\pgfqpoint{4.798489in}{8.576404in}}%
\pgfpathlineto{\pgfqpoint{4.802971in}{8.577524in}}%
\pgfpathlineto{\pgfqpoint{4.806333in}{8.579047in}}%
\pgfpathlineto{\pgfqpoint{4.811936in}{8.580336in}}%
\pgfpathlineto{\pgfqpoint{4.814177in}{8.581383in}}%
\pgfpathlineto{\pgfqpoint{4.818659in}{8.582469in}}%
\pgfpathlineto{\pgfqpoint{4.822020in}{8.584461in}}%
\pgfpathlineto{\pgfqpoint{4.826503in}{8.585896in}}%
\pgfpathlineto{\pgfqpoint{4.829864in}{8.588109in}}%
\pgfpathlineto{\pgfqpoint{4.834346in}{8.589704in}}%
\pgfpathlineto{\pgfqpoint{4.837708in}{8.591976in}}%
\pgfpathlineto{\pgfqpoint{4.842190in}{8.593442in}}%
\pgfpathlineto{\pgfqpoint{4.845551in}{8.595583in}}%
\pgfpathlineto{\pgfqpoint{4.850034in}{8.596967in}}%
\pgfpathlineto{\pgfqpoint{4.853395in}{8.599111in}}%
\pgfpathlineto{\pgfqpoint{4.857877in}{8.600551in}}%
\pgfpathlineto{\pgfqpoint{4.861239in}{8.602791in}}%
\pgfpathlineto{\pgfqpoint{4.865721in}{8.604375in}}%
\pgfpathlineto{\pgfqpoint{4.866842in}{8.605189in}}%
\pgfpathlineto{\pgfqpoint{4.874685in}{8.608333in}}%
\pgfpathlineto{\pgfqpoint{4.876926in}{8.609900in}}%
\pgfpathlineto{\pgfqpoint{4.881408in}{8.611473in}}%
\pgfpathlineto{\pgfqpoint{4.884770in}{8.613610in}}%
\pgfpathlineto{\pgfqpoint{4.889252in}{8.615069in}}%
\pgfpathlineto{\pgfqpoint{4.892614in}{8.617477in}}%
\pgfpathlineto{\pgfqpoint{4.895975in}{8.618330in}}%
\pgfpathlineto{\pgfqpoint{4.897096in}{8.619192in}}%
\pgfpathlineto{\pgfqpoint{4.907180in}{8.623608in}}%
\pgfpathlineto{\pgfqpoint{4.908301in}{8.624486in}}%
\pgfpathlineto{\pgfqpoint{4.912783in}{8.626217in}}%
\pgfpathlineto{\pgfqpoint{4.916145in}{8.628751in}}%
\pgfpathlineto{\pgfqpoint{4.920627in}{8.630298in}}%
\pgfpathlineto{\pgfqpoint{4.923988in}{8.632789in}}%
\pgfpathlineto{\pgfqpoint{4.928471in}{8.633666in}}%
\pgfpathlineto{\pgfqpoint{4.931832in}{8.636127in}}%
\pgfpathlineto{\pgfqpoint{4.936314in}{8.637561in}}%
\pgfpathlineto{\pgfqpoint{4.939676in}{8.639646in}}%
\pgfpathlineto{\pgfqpoint{4.944158in}{8.640873in}}%
\pgfpathlineto{\pgfqpoint{4.947519in}{8.642894in}}%
\pgfpathlineto{\pgfqpoint{4.952002in}{8.644422in}}%
\pgfpathlineto{\pgfqpoint{4.955363in}{8.646743in}}%
\pgfpathlineto{\pgfqpoint{4.959845in}{8.647512in}}%
\pgfpathlineto{\pgfqpoint{4.963207in}{8.649788in}}%
\pgfpathlineto{\pgfqpoint{4.967689in}{8.651358in}}%
\pgfpathlineto{\pgfqpoint{4.971051in}{8.653740in}}%
\pgfpathlineto{\pgfqpoint{4.974412in}{8.654514in}}%
\pgfpathlineto{\pgfqpoint{4.978894in}{8.657964in}}%
\pgfpathlineto{\pgfqpoint{4.983376in}{8.659696in}}%
\pgfpathlineto{\pgfqpoint{4.986738in}{8.662066in}}%
\pgfpathlineto{\pgfqpoint{4.991220in}{8.663643in}}%
\pgfpathlineto{\pgfqpoint{4.994582in}{8.665976in}}%
\pgfpathlineto{\pgfqpoint{4.999064in}{8.667494in}}%
\pgfpathlineto{\pgfqpoint{5.002425in}{8.669669in}}%
\pgfpathlineto{\pgfqpoint{5.006907in}{8.671140in}}%
\pgfpathlineto{\pgfqpoint{5.010269in}{8.673311in}}%
\pgfpathlineto{\pgfqpoint{5.014751in}{8.674525in}}%
\pgfpathlineto{\pgfqpoint{5.018113in}{8.676301in}}%
\pgfpathlineto{\pgfqpoint{5.022595in}{8.677448in}}%
\pgfpathlineto{\pgfqpoint{5.024836in}{8.678649in}}%
\pgfpathlineto{\pgfqpoint{5.030438in}{8.679846in}}%
\pgfpathlineto{\pgfqpoint{5.033800in}{8.681654in}}%
\pgfpathlineto{\pgfqpoint{5.038282in}{8.682862in}}%
\pgfpathlineto{\pgfqpoint{5.041644in}{8.684627in}}%
\pgfpathlineto{\pgfqpoint{5.046126in}{8.685778in}}%
\pgfpathlineto{\pgfqpoint{5.049487in}{8.687638in}}%
\pgfpathlineto{\pgfqpoint{5.053970in}{8.688919in}}%
\pgfpathlineto{\pgfqpoint{5.057331in}{8.690693in}}%
\pgfpathlineto{\pgfqpoint{5.061813in}{8.691851in}}%
\pgfpathlineto{\pgfqpoint{5.065175in}{8.693631in}}%
\pgfpathlineto{\pgfqpoint{5.070777in}{8.694976in}}%
\pgfpathlineto{\pgfqpoint{5.073019in}{8.696322in}}%
\pgfpathlineto{\pgfqpoint{5.077501in}{8.697683in}}%
\pgfpathlineto{\pgfqpoint{5.080862in}{8.699823in}}%
\pgfpathlineto{\pgfqpoint{5.085344in}{8.701364in}}%
\pgfpathlineto{\pgfqpoint{5.088706in}{8.703613in}}%
\pgfpathlineto{\pgfqpoint{5.093188in}{8.705079in}}%
\pgfpathlineto{\pgfqpoint{5.096550in}{8.707333in}}%
\pgfpathlineto{\pgfqpoint{5.101032in}{8.708801in}}%
\pgfpathlineto{\pgfqpoint{5.104393in}{8.710955in}}%
\pgfpathlineto{\pgfqpoint{5.108875in}{8.712407in}}%
\pgfpathlineto{\pgfqpoint{5.111116in}{8.713873in}}%
\pgfpathlineto{\pgfqpoint{5.116719in}{8.715299in}}%
\pgfpathlineto{\pgfqpoint{5.120081in}{8.717408in}}%
\pgfpathlineto{\pgfqpoint{5.124563in}{8.718800in}}%
\pgfpathlineto{\pgfqpoint{5.127924in}{8.720804in}}%
\pgfpathlineto{\pgfqpoint{5.132406in}{8.722100in}}%
\pgfpathlineto{\pgfqpoint{5.135768in}{8.724003in}}%
\pgfpathlineto{\pgfqpoint{5.140250in}{8.725208in}}%
\pgfpathlineto{\pgfqpoint{5.143612in}{8.726737in}}%
\pgfpathlineto{\pgfqpoint{5.148094in}{8.727652in}}%
\pgfpathlineto{\pgfqpoint{5.151455in}{8.729013in}}%
\pgfpathlineto{\pgfqpoint{5.155938in}{8.729934in}}%
\pgfpathlineto{\pgfqpoint{5.159299in}{8.731311in}}%
\pgfpathlineto{\pgfqpoint{5.163781in}{8.732250in}}%
\pgfpathlineto{\pgfqpoint{5.167143in}{8.733742in}}%
\pgfpathlineto{\pgfqpoint{5.171625in}{8.734757in}}%
\pgfpathlineto{\pgfqpoint{5.174986in}{8.736260in}}%
\pgfpathlineto{\pgfqpoint{5.180589in}{8.737289in}}%
\pgfpathlineto{\pgfqpoint{5.182830in}{8.738295in}}%
\pgfpathlineto{\pgfqpoint{5.187312in}{8.739234in}}%
\pgfpathlineto{\pgfqpoint{5.190674in}{8.740613in}}%
\pgfpathlineto{\pgfqpoint{5.195156in}{8.741510in}}%
\pgfpathlineto{\pgfqpoint{5.198518in}{8.742980in}}%
\pgfpathlineto{\pgfqpoint{5.203000in}{8.743903in}}%
\pgfpathlineto{\pgfqpoint{5.206361in}{8.745221in}}%
\pgfpathlineto{\pgfqpoint{5.211964in}{8.746474in}}%
\pgfpathlineto{\pgfqpoint{5.214205in}{8.747283in}}%
\pgfpathlineto{\pgfqpoint{5.219808in}{8.748495in}}%
\pgfpathlineto{\pgfqpoint{5.222049in}{8.749239in}}%
\pgfpathlineto{\pgfqpoint{5.228772in}{8.750344in}}%
\pgfpathlineto{\pgfqpoint{5.237736in}{8.752421in}}%
\pgfpathlineto{\pgfqpoint{5.242218in}{8.753234in}}%
\pgfpathlineto{\pgfqpoint{5.245580in}{8.754567in}}%
\pgfpathlineto{\pgfqpoint{5.250062in}{8.755551in}}%
\pgfpathlineto{\pgfqpoint{5.253423in}{8.757087in}}%
\pgfpathlineto{\pgfqpoint{5.257905in}{8.758103in}}%
\pgfpathlineto{\pgfqpoint{5.261267in}{8.759547in}}%
\pgfpathlineto{\pgfqpoint{5.265749in}{8.760467in}}%
\pgfpathlineto{\pgfqpoint{5.269111in}{8.761849in}}%
\pgfpathlineto{\pgfqpoint{5.273593in}{8.762798in}}%
\pgfpathlineto{\pgfqpoint{5.276954in}{8.763776in}}%
\pgfpathlineto{\pgfqpoint{5.281437in}{8.764778in}}%
\pgfpathlineto{\pgfqpoint{5.284798in}{8.766224in}}%
\pgfpathlineto{\pgfqpoint{5.289280in}{8.767254in}}%
\pgfpathlineto{\pgfqpoint{5.292642in}{8.768708in}}%
\pgfpathlineto{\pgfqpoint{5.298244in}{8.769960in}}%
\pgfpathlineto{\pgfqpoint{5.300486in}{8.770950in}}%
\pgfpathlineto{\pgfqpoint{5.304968in}{8.771966in}}%
\pgfpathlineto{\pgfqpoint{5.308329in}{8.772998in}}%
\pgfpathlineto{\pgfqpoint{5.312811in}{8.774029in}}%
\pgfpathlineto{\pgfqpoint{5.316173in}{8.774997in}}%
\pgfpathlineto{\pgfqpoint{5.321776in}{8.776227in}}%
\pgfpathlineto{\pgfqpoint{5.324017in}{8.777102in}}%
\pgfpathlineto{\pgfqpoint{5.329619in}{8.778244in}}%
\pgfpathlineto{\pgfqpoint{5.331860in}{8.778880in}}%
\pgfpathlineto{\pgfqpoint{5.338583in}{8.779860in}}%
\pgfpathlineto{\pgfqpoint{5.343066in}{8.780387in}}%
\pgfpathlineto{\pgfqpoint{5.355391in}{8.782560in}}%
\pgfpathlineto{\pgfqpoint{5.375561in}{8.784749in}}%
\pgfpathlineto{\pgfqpoint{5.386766in}{8.786371in}}%
\pgfpathlineto{\pgfqpoint{5.397971in}{8.787415in}}%
\pgfpathlineto{\pgfqpoint{5.410297in}{8.789019in}}%
\pgfpathlineto{\pgfqpoint{5.423744in}{8.789975in}}%
\pgfpathlineto{\pgfqpoint{5.433828in}{8.791005in}}%
\pgfpathlineto{\pgfqpoint{5.448395in}{8.791982in}}%
\pgfpathlineto{\pgfqpoint{5.515627in}{8.797924in}}%
\pgfpathlineto{\pgfqpoint{5.527953in}{8.798687in}}%
\pgfpathlineto{\pgfqpoint{5.541399in}{8.799679in}}%
\pgfpathlineto{\pgfqpoint{5.555966in}{8.800235in}}%
\pgfpathlineto{\pgfqpoint{5.570533in}{8.801433in}}%
\pgfpathlineto{\pgfqpoint{5.575015in}{8.801948in}}%
\pgfpathlineto{\pgfqpoint{5.604148in}{8.802802in}}%
\pgfpathlineto{\pgfqpoint{5.617595in}{8.803204in}}%
\pgfpathlineto{\pgfqpoint{5.633282in}{8.803782in}}%
\pgfpathlineto{\pgfqpoint{5.651211in}{8.804432in}}%
\pgfpathlineto{\pgfqpoint{5.675862in}{8.804541in}}%
\pgfpathlineto{\pgfqpoint{5.707237in}{8.803920in}}%
\pgfpathlineto{\pgfqpoint{5.719563in}{8.803559in}}%
\pgfpathlineto{\pgfqpoint{5.736371in}{8.802524in}}%
\pgfpathlineto{\pgfqpoint{5.767745in}{8.797529in}}%
\pgfpathlineto{\pgfqpoint{5.771107in}{8.796512in}}%
\pgfpathlineto{\pgfqpoint{5.777830in}{8.795585in}}%
\pgfpathlineto{\pgfqpoint{5.778951in}{8.795295in}}%
\pgfpathlineto{\pgfqpoint{5.784553in}{8.794450in}}%
\pgfpathlineto{\pgfqpoint{5.790156in}{8.793625in}}%
\pgfpathlineto{\pgfqpoint{5.794638in}{8.792676in}}%
\pgfpathlineto{\pgfqpoint{5.805843in}{8.791430in}}%
\pgfpathlineto{\pgfqpoint{5.815928in}{8.790129in}}%
\pgfpathlineto{\pgfqpoint{5.823772in}{8.789396in}}%
\pgfpathlineto{\pgfqpoint{5.839459in}{8.787616in}}%
\pgfpathlineto{\pgfqpoint{5.849544in}{8.786802in}}%
\pgfpathlineto{\pgfqpoint{5.868593in}{8.785764in}}%
\pgfpathlineto{\pgfqpoint{5.888762in}{8.784272in}}%
\pgfpathlineto{\pgfqpoint{5.907811in}{8.783443in}}%
\pgfpathlineto{\pgfqpoint{5.920137in}{8.782172in}}%
\pgfpathlineto{\pgfqpoint{5.934704in}{8.780787in}}%
\pgfpathlineto{\pgfqpoint{5.943668in}{8.780093in}}%
\pgfpathlineto{\pgfqpoint{5.963838in}{8.778975in}}%
\pgfpathlineto{\pgfqpoint{5.975043in}{8.778341in}}%
\pgfpathlineto{\pgfqpoint{5.997453in}{8.777329in}}%
\pgfpathlineto{\pgfqpoint{6.029949in}{8.775659in}}%
\pgfpathlineto{\pgfqpoint{6.043395in}{8.774789in}}%
\pgfpathlineto{\pgfqpoint{6.053480in}{8.774032in}}%
\pgfpathlineto{\pgfqpoint{6.103903in}{8.774034in}}%
\pgfpathlineto{\pgfqpoint{6.146484in}{8.776039in}}%
\pgfpathlineto{\pgfqpoint{6.163291in}{8.777078in}}%
\pgfpathlineto{\pgfqpoint{6.177858in}{8.778079in}}%
\pgfpathlineto{\pgfqpoint{6.194666in}{8.779470in}}%
\pgfpathlineto{\pgfqpoint{6.206992in}{8.780494in}}%
\pgfpathlineto{\pgfqpoint{6.224920in}{8.782014in}}%
\pgfpathlineto{\pgfqpoint{6.241728in}{8.783308in}}%
\pgfpathlineto{\pgfqpoint{6.260777in}{8.784393in}}%
\pgfpathlineto{\pgfqpoint{6.273103in}{8.785586in}}%
\pgfpathlineto{\pgfqpoint{6.292152in}{8.786754in}}%
\pgfpathlineto{\pgfqpoint{6.357143in}{8.793025in}}%
\pgfpathlineto{\pgfqpoint{6.367227in}{8.794405in}}%
\pgfpathlineto{\pgfqpoint{6.378433in}{8.795613in}}%
\pgfpathlineto{\pgfqpoint{6.390758in}{8.797431in}}%
\pgfpathlineto{\pgfqpoint{6.401964in}{8.798601in}}%
\pgfpathlineto{\pgfqpoint{6.406446in}{8.799400in}}%
\pgfpathlineto{\pgfqpoint{6.417651in}{8.800339in}}%
\pgfpathlineto{\pgfqpoint{6.429977in}{8.802309in}}%
\pgfpathlineto{\pgfqpoint{6.436700in}{8.803253in}}%
\pgfpathlineto{\pgfqpoint{6.445664in}{8.804809in}}%
\pgfpathlineto{\pgfqpoint{6.452387in}{8.805889in}}%
\pgfpathlineto{\pgfqpoint{6.453508in}{8.806172in}}%
\pgfpathlineto{\pgfqpoint{6.460231in}{8.807250in}}%
\pgfpathlineto{\pgfqpoint{6.469195in}{8.809005in}}%
\pgfpathlineto{\pgfqpoint{6.474798in}{8.809935in}}%
\pgfpathlineto{\pgfqpoint{6.477039in}{8.810571in}}%
\pgfpathlineto{\pgfqpoint{6.482642in}{8.811500in}}%
\pgfpathlineto{\pgfqpoint{6.484883in}{8.812068in}}%
\pgfpathlineto{\pgfqpoint{6.490485in}{8.812919in}}%
\pgfpathlineto{\pgfqpoint{6.498329in}{8.814344in}}%
\pgfpathlineto{\pgfqpoint{6.507293in}{8.815846in}}%
\pgfpathlineto{\pgfqpoint{6.508414in}{8.816176in}}%
\pgfpathlineto{\pgfqpoint{6.514016in}{8.817188in}}%
\pgfpathlineto{\pgfqpoint{6.516257in}{8.817862in}}%
\pgfpathlineto{\pgfqpoint{6.521860in}{8.818874in}}%
\pgfpathlineto{\pgfqpoint{6.524101in}{8.819526in}}%
\pgfpathlineto{\pgfqpoint{6.529704in}{8.820546in}}%
\pgfpathlineto{\pgfqpoint{6.531945in}{8.821212in}}%
\pgfpathlineto{\pgfqpoint{6.538668in}{8.822227in}}%
\pgfpathlineto{\pgfqpoint{6.539789in}{8.822566in}}%
\pgfpathlineto{\pgfqpoint{6.539789in}{8.822566in}}%
\pgfusepath{stroke}%
\end{pgfscope}%
\begin{pgfscope}%
\pgfsetrectcap%
\pgfsetmiterjoin%
\pgfsetlinewidth{0.803000pt}%
\definecolor{currentstroke}{rgb}{1.000000,1.000000,1.000000}%
\pgfsetstrokecolor{currentstroke}%
\pgfsetdash{}{0pt}%
\pgfpathmoveto{\pgfqpoint{3.966666in}{8.286757in}}%
\pgfpathlineto{\pgfqpoint{3.966666in}{9.391635in}}%
\pgfusepath{stroke}%
\end{pgfscope}%
\begin{pgfscope}%
\pgfsetrectcap%
\pgfsetmiterjoin%
\pgfsetlinewidth{0.803000pt}%
\definecolor{currentstroke}{rgb}{1.000000,1.000000,1.000000}%
\pgfsetstrokecolor{currentstroke}%
\pgfsetdash{}{0pt}%
\pgfpathmoveto{\pgfqpoint{6.662318in}{8.286757in}}%
\pgfpathlineto{\pgfqpoint{6.662318in}{9.391635in}}%
\pgfusepath{stroke}%
\end{pgfscope}%
\begin{pgfscope}%
\pgfsetrectcap%
\pgfsetmiterjoin%
\pgfsetlinewidth{0.803000pt}%
\definecolor{currentstroke}{rgb}{1.000000,1.000000,1.000000}%
\pgfsetstrokecolor{currentstroke}%
\pgfsetdash{}{0pt}%
\pgfpathmoveto{\pgfqpoint{3.966666in}{8.286757in}}%
\pgfpathlineto{\pgfqpoint{6.662318in}{8.286757in}}%
\pgfusepath{stroke}%
\end{pgfscope}%
\begin{pgfscope}%
\pgfsetrectcap%
\pgfsetmiterjoin%
\pgfsetlinewidth{0.803000pt}%
\definecolor{currentstroke}{rgb}{1.000000,1.000000,1.000000}%
\pgfsetstrokecolor{currentstroke}%
\pgfsetdash{}{0pt}%
\pgfpathmoveto{\pgfqpoint{3.966666in}{9.391635in}}%
\pgfpathlineto{\pgfqpoint{6.662318in}{9.391635in}}%
\pgfusepath{stroke}%
\end{pgfscope}%
\begin{pgfscope}%
\definecolor{textcolor}{rgb}{0.150000,0.150000,0.150000}%
\pgfsetstrokecolor{textcolor}%
\pgfsetfillcolor{textcolor}%
\pgftext[x=5.314492in,y=9.474968in,,base]{\color{textcolor}\rmfamily\fontsize{12.000000}{14.400000}\selectfont AXP}%
\end{pgfscope}%
\begin{pgfscope}%
\pgfsetbuttcap%
\pgfsetmiterjoin%
\definecolor{currentfill}{rgb}{0.917647,0.917647,0.949020}%
\pgfsetfillcolor{currentfill}%
\pgfsetlinewidth{0.000000pt}%
\definecolor{currentstroke}{rgb}{0.000000,0.000000,0.000000}%
\pgfsetstrokecolor{currentstroke}%
\pgfsetstrokeopacity{0.000000}%
\pgfsetdash{}{0pt}%
\pgfpathmoveto{\pgfqpoint{0.462318in}{6.297976in}}%
\pgfpathlineto{\pgfqpoint{3.157970in}{6.297976in}}%
\pgfpathlineto{\pgfqpoint{3.157970in}{7.402855in}}%
\pgfpathlineto{\pgfqpoint{0.462318in}{7.402855in}}%
\pgfpathclose%
\pgfusepath{fill}%
\end{pgfscope}%
\begin{pgfscope}%
\pgfpathrectangle{\pgfqpoint{0.462318in}{6.297976in}}{\pgfqpoint{2.695652in}{1.104878in}}%
\pgfusepath{clip}%
\pgfsetroundcap%
\pgfsetroundjoin%
\pgfsetlinewidth{0.803000pt}%
\definecolor{currentstroke}{rgb}{1.000000,1.000000,1.000000}%
\pgfsetstrokecolor{currentstroke}%
\pgfsetdash{}{0pt}%
\pgfpathmoveto{\pgfqpoint{0.582607in}{6.297976in}}%
\pgfpathlineto{\pgfqpoint{0.582607in}{7.402855in}}%
\pgfusepath{stroke}%
\end{pgfscope}%
\begin{pgfscope}%
\definecolor{textcolor}{rgb}{0.150000,0.150000,0.150000}%
\pgfsetstrokecolor{textcolor}%
\pgfsetfillcolor{textcolor}%
\pgftext[x=0.582607in,y=6.200754in,,top]{\color{textcolor}\rmfamily\fontsize{10.000000}{12.000000}\selectfont 2012}%
\end{pgfscope}%
\begin{pgfscope}%
\pgfpathrectangle{\pgfqpoint{0.462318in}{6.297976in}}{\pgfqpoint{2.695652in}{1.104878in}}%
\pgfusepath{clip}%
\pgfsetroundcap%
\pgfsetroundjoin%
\pgfsetlinewidth{0.803000pt}%
\definecolor{currentstroke}{rgb}{1.000000,1.000000,1.000000}%
\pgfsetstrokecolor{currentstroke}%
\pgfsetdash{}{0pt}%
\pgfpathmoveto{\pgfqpoint{0.992720in}{6.297976in}}%
\pgfpathlineto{\pgfqpoint{0.992720in}{7.402855in}}%
\pgfusepath{stroke}%
\end{pgfscope}%
\begin{pgfscope}%
\definecolor{textcolor}{rgb}{0.150000,0.150000,0.150000}%
\pgfsetstrokecolor{textcolor}%
\pgfsetfillcolor{textcolor}%
\pgftext[x=0.992720in,y=6.200754in,,top]{\color{textcolor}\rmfamily\fontsize{10.000000}{12.000000}\selectfont 2013}%
\end{pgfscope}%
\begin{pgfscope}%
\pgfpathrectangle{\pgfqpoint{0.462318in}{6.297976in}}{\pgfqpoint{2.695652in}{1.104878in}}%
\pgfusepath{clip}%
\pgfsetroundcap%
\pgfsetroundjoin%
\pgfsetlinewidth{0.803000pt}%
\definecolor{currentstroke}{rgb}{1.000000,1.000000,1.000000}%
\pgfsetstrokecolor{currentstroke}%
\pgfsetdash{}{0pt}%
\pgfpathmoveto{\pgfqpoint{1.401712in}{6.297976in}}%
\pgfpathlineto{\pgfqpoint{1.401712in}{7.402855in}}%
\pgfusepath{stroke}%
\end{pgfscope}%
\begin{pgfscope}%
\definecolor{textcolor}{rgb}{0.150000,0.150000,0.150000}%
\pgfsetstrokecolor{textcolor}%
\pgfsetfillcolor{textcolor}%
\pgftext[x=1.401712in,y=6.200754in,,top]{\color{textcolor}\rmfamily\fontsize{10.000000}{12.000000}\selectfont 2014}%
\end{pgfscope}%
\begin{pgfscope}%
\pgfpathrectangle{\pgfqpoint{0.462318in}{6.297976in}}{\pgfqpoint{2.695652in}{1.104878in}}%
\pgfusepath{clip}%
\pgfsetroundcap%
\pgfsetroundjoin%
\pgfsetlinewidth{0.803000pt}%
\definecolor{currentstroke}{rgb}{1.000000,1.000000,1.000000}%
\pgfsetstrokecolor{currentstroke}%
\pgfsetdash{}{0pt}%
\pgfpathmoveto{\pgfqpoint{1.810705in}{6.297976in}}%
\pgfpathlineto{\pgfqpoint{1.810705in}{7.402855in}}%
\pgfusepath{stroke}%
\end{pgfscope}%
\begin{pgfscope}%
\definecolor{textcolor}{rgb}{0.150000,0.150000,0.150000}%
\pgfsetstrokecolor{textcolor}%
\pgfsetfillcolor{textcolor}%
\pgftext[x=1.810705in,y=6.200754in,,top]{\color{textcolor}\rmfamily\fontsize{10.000000}{12.000000}\selectfont 2015}%
\end{pgfscope}%
\begin{pgfscope}%
\pgfpathrectangle{\pgfqpoint{0.462318in}{6.297976in}}{\pgfqpoint{2.695652in}{1.104878in}}%
\pgfusepath{clip}%
\pgfsetroundcap%
\pgfsetroundjoin%
\pgfsetlinewidth{0.803000pt}%
\definecolor{currentstroke}{rgb}{1.000000,1.000000,1.000000}%
\pgfsetstrokecolor{currentstroke}%
\pgfsetdash{}{0pt}%
\pgfpathmoveto{\pgfqpoint{2.219697in}{6.297976in}}%
\pgfpathlineto{\pgfqpoint{2.219697in}{7.402855in}}%
\pgfusepath{stroke}%
\end{pgfscope}%
\begin{pgfscope}%
\definecolor{textcolor}{rgb}{0.150000,0.150000,0.150000}%
\pgfsetstrokecolor{textcolor}%
\pgfsetfillcolor{textcolor}%
\pgftext[x=2.219697in,y=6.200754in,,top]{\color{textcolor}\rmfamily\fontsize{10.000000}{12.000000}\selectfont 2016}%
\end{pgfscope}%
\begin{pgfscope}%
\pgfpathrectangle{\pgfqpoint{0.462318in}{6.297976in}}{\pgfqpoint{2.695652in}{1.104878in}}%
\pgfusepath{clip}%
\pgfsetroundcap%
\pgfsetroundjoin%
\pgfsetlinewidth{0.803000pt}%
\definecolor{currentstroke}{rgb}{1.000000,1.000000,1.000000}%
\pgfsetstrokecolor{currentstroke}%
\pgfsetdash{}{0pt}%
\pgfpathmoveto{\pgfqpoint{2.629810in}{6.297976in}}%
\pgfpathlineto{\pgfqpoint{2.629810in}{7.402855in}}%
\pgfusepath{stroke}%
\end{pgfscope}%
\begin{pgfscope}%
\definecolor{textcolor}{rgb}{0.150000,0.150000,0.150000}%
\pgfsetstrokecolor{textcolor}%
\pgfsetfillcolor{textcolor}%
\pgftext[x=2.629810in,y=6.200754in,,top]{\color{textcolor}\rmfamily\fontsize{10.000000}{12.000000}\selectfont 2017}%
\end{pgfscope}%
\begin{pgfscope}%
\pgfpathrectangle{\pgfqpoint{0.462318in}{6.297976in}}{\pgfqpoint{2.695652in}{1.104878in}}%
\pgfusepath{clip}%
\pgfsetroundcap%
\pgfsetroundjoin%
\pgfsetlinewidth{0.803000pt}%
\definecolor{currentstroke}{rgb}{1.000000,1.000000,1.000000}%
\pgfsetstrokecolor{currentstroke}%
\pgfsetdash{}{0pt}%
\pgfpathmoveto{\pgfqpoint{3.038802in}{6.297976in}}%
\pgfpathlineto{\pgfqpoint{3.038802in}{7.402855in}}%
\pgfusepath{stroke}%
\end{pgfscope}%
\begin{pgfscope}%
\definecolor{textcolor}{rgb}{0.150000,0.150000,0.150000}%
\pgfsetstrokecolor{textcolor}%
\pgfsetfillcolor{textcolor}%
\pgftext[x=3.038802in,y=6.200754in,,top]{\color{textcolor}\rmfamily\fontsize{10.000000}{12.000000}\selectfont 2018}%
\end{pgfscope}%
\begin{pgfscope}%
\pgfpathrectangle{\pgfqpoint{0.462318in}{6.297976in}}{\pgfqpoint{2.695652in}{1.104878in}}%
\pgfusepath{clip}%
\pgfsetroundcap%
\pgfsetroundjoin%
\pgfsetlinewidth{0.803000pt}%
\definecolor{currentstroke}{rgb}{1.000000,1.000000,1.000000}%
\pgfsetstrokecolor{currentstroke}%
\pgfsetdash{}{0pt}%
\pgfpathmoveto{\pgfqpoint{0.462318in}{6.431512in}}%
\pgfpathlineto{\pgfqpoint{3.157970in}{6.431512in}}%
\pgfusepath{stroke}%
\end{pgfscope}%
\begin{pgfscope}%
\definecolor{textcolor}{rgb}{0.150000,0.150000,0.150000}%
\pgfsetstrokecolor{textcolor}%
\pgfsetfillcolor{textcolor}%
\pgftext[x=0.188365in,y=6.378751in,left,base]{\color{textcolor}\rmfamily\fontsize{10.000000}{12.000000}\selectfont 15}%
\end{pgfscope}%
\begin{pgfscope}%
\pgfpathrectangle{\pgfqpoint{0.462318in}{6.297976in}}{\pgfqpoint{2.695652in}{1.104878in}}%
\pgfusepath{clip}%
\pgfsetroundcap%
\pgfsetroundjoin%
\pgfsetlinewidth{0.803000pt}%
\definecolor{currentstroke}{rgb}{1.000000,1.000000,1.000000}%
\pgfsetstrokecolor{currentstroke}%
\pgfsetdash{}{0pt}%
\pgfpathmoveto{\pgfqpoint{0.462318in}{6.764769in}}%
\pgfpathlineto{\pgfqpoint{3.157970in}{6.764769in}}%
\pgfusepath{stroke}%
\end{pgfscope}%
\begin{pgfscope}%
\definecolor{textcolor}{rgb}{0.150000,0.150000,0.150000}%
\pgfsetstrokecolor{textcolor}%
\pgfsetfillcolor{textcolor}%
\pgftext[x=0.188365in,y=6.712007in,left,base]{\color{textcolor}\rmfamily\fontsize{10.000000}{12.000000}\selectfont 20}%
\end{pgfscope}%
\begin{pgfscope}%
\pgfpathrectangle{\pgfqpoint{0.462318in}{6.297976in}}{\pgfqpoint{2.695652in}{1.104878in}}%
\pgfusepath{clip}%
\pgfsetroundcap%
\pgfsetroundjoin%
\pgfsetlinewidth{0.803000pt}%
\definecolor{currentstroke}{rgb}{1.000000,1.000000,1.000000}%
\pgfsetstrokecolor{currentstroke}%
\pgfsetdash{}{0pt}%
\pgfpathmoveto{\pgfqpoint{0.462318in}{7.098025in}}%
\pgfpathlineto{\pgfqpoint{3.157970in}{7.098025in}}%
\pgfusepath{stroke}%
\end{pgfscope}%
\begin{pgfscope}%
\definecolor{textcolor}{rgb}{0.150000,0.150000,0.150000}%
\pgfsetstrokecolor{textcolor}%
\pgfsetfillcolor{textcolor}%
\pgftext[x=0.188365in,y=7.045263in,left,base]{\color{textcolor}\rmfamily\fontsize{10.000000}{12.000000}\selectfont 25}%
\end{pgfscope}%
\begin{pgfscope}%
\pgfpathrectangle{\pgfqpoint{0.462318in}{6.297976in}}{\pgfqpoint{2.695652in}{1.104878in}}%
\pgfusepath{clip}%
\pgfsetroundcap%
\pgfsetroundjoin%
\pgfsetlinewidth{1.505625pt}%
\definecolor{currentstroke}{rgb}{0.172549,0.627451,0.172549}%
\pgfsetstrokecolor{currentstroke}%
\pgfsetdash{}{0pt}%
\pgfpathmoveto{\pgfqpoint{0.584848in}{6.350864in}}%
\pgfpathlineto{\pgfqpoint{0.585968in}{6.360862in}}%
\pgfpathlineto{\pgfqpoint{0.587089in}{6.360195in}}%
\pgfpathlineto{\pgfqpoint{0.588209in}{6.365528in}}%
\pgfpathlineto{\pgfqpoint{0.591571in}{6.376192in}}%
\pgfpathlineto{\pgfqpoint{0.592692in}{6.368860in}}%
\pgfpathlineto{\pgfqpoint{0.593812in}{6.376858in}}%
\pgfpathlineto{\pgfqpoint{0.594933in}{6.379524in}}%
\pgfpathlineto{\pgfqpoint{0.596053in}{6.374859in}}%
\pgfpathlineto{\pgfqpoint{0.600535in}{6.370193in}}%
\pgfpathlineto{\pgfqpoint{0.601656in}{6.384190in}}%
\pgfpathlineto{\pgfqpoint{0.602776in}{6.390188in}}%
\pgfpathlineto{\pgfqpoint{0.603897in}{6.390188in}}%
\pgfpathlineto{\pgfqpoint{0.607258in}{6.380191in}}%
\pgfpathlineto{\pgfqpoint{0.608379in}{6.374859in}}%
\pgfpathlineto{\pgfqpoint{0.609499in}{6.389522in}}%
\pgfpathlineto{\pgfqpoint{0.611740in}{6.384190in}}%
\pgfpathlineto{\pgfqpoint{0.615102in}{6.378191in}}%
\pgfpathlineto{\pgfqpoint{0.616223in}{6.368194in}}%
\pgfpathlineto{\pgfqpoint{0.617343in}{6.371526in}}%
\pgfpathlineto{\pgfqpoint{0.618464in}{6.370193in}}%
\pgfpathlineto{\pgfqpoint{0.619584in}{6.384190in}}%
\pgfpathlineto{\pgfqpoint{0.622946in}{6.385523in}}%
\pgfpathlineto{\pgfqpoint{0.624066in}{6.392188in}}%
\pgfpathlineto{\pgfqpoint{0.625187in}{6.394854in}}%
\pgfpathlineto{\pgfqpoint{0.626307in}{6.389522in}}%
\pgfpathlineto{\pgfqpoint{0.627428in}{6.376858in}}%
\pgfpathlineto{\pgfqpoint{0.630789in}{6.386189in}}%
\pgfpathlineto{\pgfqpoint{0.631910in}{6.380191in}}%
\pgfpathlineto{\pgfqpoint{0.633031in}{6.370860in}}%
\pgfpathlineto{\pgfqpoint{0.635272in}{6.396854in}}%
\pgfpathlineto{\pgfqpoint{0.639754in}{6.403519in}}%
\pgfpathlineto{\pgfqpoint{0.640874in}{6.402186in}}%
\pgfpathlineto{\pgfqpoint{0.641995in}{6.406851in}}%
\pgfpathlineto{\pgfqpoint{0.646477in}{6.394854in}}%
\pgfpathlineto{\pgfqpoint{0.647597in}{6.399520in}}%
\pgfpathlineto{\pgfqpoint{0.648718in}{6.394188in}}%
\pgfpathlineto{\pgfqpoint{0.649838in}{6.397520in}}%
\pgfpathlineto{\pgfqpoint{0.650959in}{6.390188in}}%
\pgfpathlineto{\pgfqpoint{0.654321in}{6.383523in}}%
\pgfpathlineto{\pgfqpoint{0.655441in}{6.362195in}}%
\pgfpathlineto{\pgfqpoint{0.657682in}{6.392855in}}%
\pgfpathlineto{\pgfqpoint{0.658803in}{6.393521in}}%
\pgfpathlineto{\pgfqpoint{0.662164in}{6.398187in}}%
\pgfpathlineto{\pgfqpoint{0.663285in}{6.421515in}}%
\pgfpathlineto{\pgfqpoint{0.664405in}{6.431512in}}%
\pgfpathlineto{\pgfqpoint{0.665526in}{6.450175in}}%
\pgfpathlineto{\pgfqpoint{0.666646in}{6.452174in}}%
\pgfpathlineto{\pgfqpoint{0.670008in}{6.452841in}}%
\pgfpathlineto{\pgfqpoint{0.671128in}{6.445509in}}%
\pgfpathlineto{\pgfqpoint{0.672249in}{6.445509in}}%
\pgfpathlineto{\pgfqpoint{0.673369in}{6.434178in}}%
\pgfpathlineto{\pgfqpoint{0.674490in}{6.430846in}}%
\pgfpathlineto{\pgfqpoint{0.677852in}{6.444176in}}%
\pgfpathlineto{\pgfqpoint{0.678972in}{6.444176in}}%
\pgfpathlineto{\pgfqpoint{0.681213in}{6.439510in}}%
\pgfpathlineto{\pgfqpoint{0.682334in}{6.445509in}}%
\pgfpathlineto{\pgfqpoint{0.685695in}{6.442843in}}%
\pgfpathlineto{\pgfqpoint{0.686816in}{6.440177in}}%
\pgfpathlineto{\pgfqpoint{0.689057in}{6.416182in}}%
\pgfpathlineto{\pgfqpoint{0.693539in}{6.401519in}}%
\pgfpathlineto{\pgfqpoint{0.694659in}{6.378191in}}%
\pgfpathlineto{\pgfqpoint{0.696901in}{6.406851in}}%
\pgfpathlineto{\pgfqpoint{0.698021in}{6.385523in}}%
\pgfpathlineto{\pgfqpoint{0.701383in}{6.386189in}}%
\pgfpathlineto{\pgfqpoint{0.702503in}{6.408851in}}%
\pgfpathlineto{\pgfqpoint{0.703624in}{6.396187in}}%
\pgfpathlineto{\pgfqpoint{0.704744in}{6.398187in}}%
\pgfpathlineto{\pgfqpoint{0.705865in}{6.409517in}}%
\pgfpathlineto{\pgfqpoint{0.709226in}{6.394854in}}%
\pgfpathlineto{\pgfqpoint{0.710347in}{6.418849in}}%
\pgfpathlineto{\pgfqpoint{0.711467in}{6.414183in}}%
\pgfpathlineto{\pgfqpoint{0.713708in}{6.430846in}}%
\pgfpathlineto{\pgfqpoint{0.717070in}{6.420848in}}%
\pgfpathlineto{\pgfqpoint{0.718191in}{6.431512in}}%
\pgfpathlineto{\pgfqpoint{0.719311in}{6.430179in}}%
\pgfpathlineto{\pgfqpoint{0.720432in}{6.422181in}}%
\pgfpathlineto{\pgfqpoint{0.721552in}{6.408851in}}%
\pgfpathlineto{\pgfqpoint{0.724914in}{6.407518in}}%
\pgfpathlineto{\pgfqpoint{0.726034in}{6.404185in}}%
\pgfpathlineto{\pgfqpoint{0.727155in}{6.386856in}}%
\pgfpathlineto{\pgfqpoint{0.728275in}{6.396187in}}%
\pgfpathlineto{\pgfqpoint{0.729396in}{6.392188in}}%
\pgfpathlineto{\pgfqpoint{0.733878in}{6.360862in}}%
\pgfpathlineto{\pgfqpoint{0.734998in}{6.391522in}}%
\pgfpathlineto{\pgfqpoint{0.736119in}{6.385523in}}%
\pgfpathlineto{\pgfqpoint{0.740601in}{6.397520in}}%
\pgfpathlineto{\pgfqpoint{0.741722in}{6.400186in}}%
\pgfpathlineto{\pgfqpoint{0.742842in}{6.400186in}}%
\pgfpathlineto{\pgfqpoint{0.743963in}{6.404185in}}%
\pgfpathlineto{\pgfqpoint{0.745083in}{6.401519in}}%
\pgfpathlineto{\pgfqpoint{0.749565in}{6.408851in}}%
\pgfpathlineto{\pgfqpoint{0.750686in}{6.393521in}}%
\pgfpathlineto{\pgfqpoint{0.751806in}{6.396187in}}%
\pgfpathlineto{\pgfqpoint{0.752927in}{6.368194in}}%
\pgfpathlineto{\pgfqpoint{0.756288in}{6.348198in}}%
\pgfpathlineto{\pgfqpoint{0.757409in}{6.352864in}}%
\pgfpathlineto{\pgfqpoint{0.758530in}{6.385523in}}%
\pgfpathlineto{\pgfqpoint{0.759650in}{6.391522in}}%
\pgfpathlineto{\pgfqpoint{0.760771in}{6.401519in}}%
\pgfpathlineto{\pgfqpoint{0.764132in}{6.396854in}}%
\pgfpathlineto{\pgfqpoint{0.765253in}{6.415516in}}%
\pgfpathlineto{\pgfqpoint{0.766373in}{6.410184in}}%
\pgfpathlineto{\pgfqpoint{0.768614in}{6.444843in}}%
\pgfpathlineto{\pgfqpoint{0.771976in}{6.432179in}}%
\pgfpathlineto{\pgfqpoint{0.773096in}{6.444843in}}%
\pgfpathlineto{\pgfqpoint{0.774217in}{6.449508in}}%
\pgfpathlineto{\pgfqpoint{0.775337in}{6.429513in}}%
\pgfpathlineto{\pgfqpoint{0.776458in}{6.443510in}}%
\pgfpathlineto{\pgfqpoint{0.779820in}{6.428846in}}%
\pgfpathlineto{\pgfqpoint{0.782061in}{6.460172in}}%
\pgfpathlineto{\pgfqpoint{0.783181in}{6.463505in}}%
\pgfpathlineto{\pgfqpoint{0.784302in}{6.496164in}}%
\pgfpathlineto{\pgfqpoint{0.787663in}{6.478168in}}%
\pgfpathlineto{\pgfqpoint{0.791025in}{6.470170in}}%
\pgfpathlineto{\pgfqpoint{0.792145in}{6.453507in}}%
\pgfpathlineto{\pgfqpoint{0.795507in}{6.455507in}}%
\pgfpathlineto{\pgfqpoint{0.796627in}{6.434178in}}%
\pgfpathlineto{\pgfqpoint{0.797748in}{6.436844in}}%
\pgfpathlineto{\pgfqpoint{0.798869in}{6.424847in}}%
\pgfpathlineto{\pgfqpoint{0.799989in}{6.441510in}}%
\pgfpathlineto{\pgfqpoint{0.803351in}{6.432179in}}%
\pgfpathlineto{\pgfqpoint{0.805592in}{6.445509in}}%
\pgfpathlineto{\pgfqpoint{0.806712in}{6.442843in}}%
\pgfpathlineto{\pgfqpoint{0.811194in}{6.458173in}}%
\pgfpathlineto{\pgfqpoint{0.812315in}{6.451508in}}%
\pgfpathlineto{\pgfqpoint{0.813435in}{6.453507in}}%
\pgfpathlineto{\pgfqpoint{0.815676in}{6.500163in}}%
\pgfpathlineto{\pgfqpoint{0.821279in}{6.490832in}}%
\pgfpathlineto{\pgfqpoint{0.822400in}{6.480168in}}%
\pgfpathlineto{\pgfqpoint{0.823520in}{6.502163in}}%
\pgfpathlineto{\pgfqpoint{0.826882in}{6.502829in}}%
\pgfpathlineto{\pgfqpoint{0.828002in}{6.510827in}}%
\pgfpathlineto{\pgfqpoint{0.829123in}{6.504829in}}%
\pgfpathlineto{\pgfqpoint{0.831364in}{6.509494in}}%
\pgfpathlineto{\pgfqpoint{0.835846in}{6.501496in}}%
\pgfpathlineto{\pgfqpoint{0.836966in}{6.502163in}}%
\pgfpathlineto{\pgfqpoint{0.838087in}{6.506828in}}%
\pgfpathlineto{\pgfqpoint{0.839208in}{6.504162in}}%
\pgfpathlineto{\pgfqpoint{0.842569in}{6.500830in}}%
\pgfpathlineto{\pgfqpoint{0.844810in}{6.493498in}}%
\pgfpathlineto{\pgfqpoint{0.845931in}{6.486166in}}%
\pgfpathlineto{\pgfqpoint{0.847051in}{6.494164in}}%
\pgfpathlineto{\pgfqpoint{0.850413in}{6.496831in}}%
\pgfpathlineto{\pgfqpoint{0.851533in}{6.494831in}}%
\pgfpathlineto{\pgfqpoint{0.852654in}{6.495497in}}%
\pgfpathlineto{\pgfqpoint{0.853774in}{6.486166in}}%
\pgfpathlineto{\pgfqpoint{0.854895in}{6.489499in}}%
\pgfpathlineto{\pgfqpoint{0.859377in}{6.479501in}}%
\pgfpathlineto{\pgfqpoint{0.860498in}{6.486833in}}%
\pgfpathlineto{\pgfqpoint{0.861618in}{6.520158in}}%
\pgfpathlineto{\pgfqpoint{0.862739in}{6.534822in}}%
\pgfpathlineto{\pgfqpoint{0.866100in}{6.528823in}}%
\pgfpathlineto{\pgfqpoint{0.867221in}{6.534822in}}%
\pgfpathlineto{\pgfqpoint{0.868341in}{6.550152in}}%
\pgfpathlineto{\pgfqpoint{0.870582in}{6.560816in}}%
\pgfpathlineto{\pgfqpoint{0.873944in}{6.558150in}}%
\pgfpathlineto{\pgfqpoint{0.878426in}{6.591475in}}%
\pgfpathlineto{\pgfqpoint{0.882908in}{6.580145in}}%
\pgfpathlineto{\pgfqpoint{0.884029in}{6.569480in}}%
\pgfpathlineto{\pgfqpoint{0.885149in}{6.601473in}}%
\pgfpathlineto{\pgfqpoint{0.886270in}{6.600806in}}%
\pgfpathlineto{\pgfqpoint{0.889631in}{6.605472in}}%
\pgfpathlineto{\pgfqpoint{0.890752in}{6.604806in}}%
\pgfpathlineto{\pgfqpoint{0.891872in}{6.610804in}}%
\pgfpathlineto{\pgfqpoint{0.892993in}{6.612804in}}%
\pgfpathlineto{\pgfqpoint{0.894113in}{6.621468in}}%
\pgfpathlineto{\pgfqpoint{0.897475in}{6.611471in}}%
\pgfpathlineto{\pgfqpoint{0.899716in}{6.586143in}}%
\pgfpathlineto{\pgfqpoint{0.900836in}{6.590142in}}%
\pgfpathlineto{\pgfqpoint{0.901957in}{6.588809in}}%
\pgfpathlineto{\pgfqpoint{0.905319in}{6.596807in}}%
\pgfpathlineto{\pgfqpoint{0.906439in}{6.596807in}}%
\pgfpathlineto{\pgfqpoint{0.907560in}{6.610804in}}%
\pgfpathlineto{\pgfqpoint{0.908680in}{6.605472in}}%
\pgfpathlineto{\pgfqpoint{0.909801in}{6.565481in}}%
\pgfpathlineto{\pgfqpoint{0.913162in}{6.548819in}}%
\pgfpathlineto{\pgfqpoint{0.914283in}{6.526824in}}%
\pgfpathlineto{\pgfqpoint{0.915403in}{6.526157in}}%
\pgfpathlineto{\pgfqpoint{0.916524in}{6.526157in}}%
\pgfpathlineto{\pgfqpoint{0.917644in}{6.518159in}}%
\pgfpathlineto{\pgfqpoint{0.923247in}{6.515493in}}%
\pgfpathlineto{\pgfqpoint{0.924368in}{6.530156in}}%
\pgfpathlineto{\pgfqpoint{0.925488in}{6.528823in}}%
\pgfpathlineto{\pgfqpoint{0.928850in}{6.533489in}}%
\pgfpathlineto{\pgfqpoint{0.929970in}{6.542820in}}%
\pgfpathlineto{\pgfqpoint{0.932211in}{6.506828in}}%
\pgfpathlineto{\pgfqpoint{0.933332in}{6.512827in}}%
\pgfpathlineto{\pgfqpoint{0.936693in}{6.506828in}}%
\pgfpathlineto{\pgfqpoint{0.937814in}{6.496164in}}%
\pgfpathlineto{\pgfqpoint{0.938934in}{6.461505in}}%
\pgfpathlineto{\pgfqpoint{0.940055in}{6.464171in}}%
\pgfpathlineto{\pgfqpoint{0.941175in}{6.468837in}}%
\pgfpathlineto{\pgfqpoint{0.944537in}{6.494831in}}%
\pgfpathlineto{\pgfqpoint{0.945658in}{6.492831in}}%
\pgfpathlineto{\pgfqpoint{0.946778in}{6.496164in}}%
\pgfpathlineto{\pgfqpoint{0.949019in}{6.514826in}}%
\pgfpathlineto{\pgfqpoint{0.952381in}{6.515493in}}%
\pgfpathlineto{\pgfqpoint{0.953501in}{6.506162in}}%
\pgfpathlineto{\pgfqpoint{0.954622in}{6.520158in}}%
\pgfpathlineto{\pgfqpoint{0.955742in}{6.519492in}}%
\pgfpathlineto{\pgfqpoint{0.956863in}{6.519492in}}%
\pgfpathlineto{\pgfqpoint{0.960224in}{6.503496in}}%
\pgfpathlineto{\pgfqpoint{0.961345in}{6.505495in}}%
\pgfpathlineto{\pgfqpoint{0.962465in}{6.524158in}}%
\pgfpathlineto{\pgfqpoint{0.964707in}{6.536155in}}%
\pgfpathlineto{\pgfqpoint{0.968068in}{6.532822in}}%
\pgfpathlineto{\pgfqpoint{0.969189in}{6.538821in}}%
\pgfpathlineto{\pgfqpoint{0.970309in}{6.552818in}}%
\pgfpathlineto{\pgfqpoint{0.971430in}{6.544819in}}%
\pgfpathlineto{\pgfqpoint{0.972550in}{6.544819in}}%
\pgfpathlineto{\pgfqpoint{0.975912in}{6.560816in}}%
\pgfpathlineto{\pgfqpoint{0.977032in}{6.548152in}}%
\pgfpathlineto{\pgfqpoint{0.978153in}{6.512827in}}%
\pgfpathlineto{\pgfqpoint{0.979273in}{6.524824in}}%
\pgfpathlineto{\pgfqpoint{0.980394in}{6.516159in}}%
\pgfpathlineto{\pgfqpoint{0.985997in}{6.510827in}}%
\pgfpathlineto{\pgfqpoint{0.987117in}{6.506162in}}%
\pgfpathlineto{\pgfqpoint{0.988238in}{6.493498in}}%
\pgfpathlineto{\pgfqpoint{0.993840in}{6.540154in}}%
\pgfpathlineto{\pgfqpoint{0.994961in}{6.527490in}}%
\pgfpathlineto{\pgfqpoint{0.996081in}{6.532822in}}%
\pgfpathlineto{\pgfqpoint{0.999443in}{6.529490in}}%
\pgfpathlineto{\pgfqpoint{1.000563in}{6.517492in}}%
\pgfpathlineto{\pgfqpoint{1.001684in}{6.520158in}}%
\pgfpathlineto{\pgfqpoint{1.002804in}{6.531489in}}%
\pgfpathlineto{\pgfqpoint{1.003925in}{6.529490in}}%
\pgfpathlineto{\pgfqpoint{1.007287in}{6.528823in}}%
\pgfpathlineto{\pgfqpoint{1.008407in}{6.532822in}}%
\pgfpathlineto{\pgfqpoint{1.009528in}{6.528823in}}%
\pgfpathlineto{\pgfqpoint{1.010648in}{6.538154in}}%
\pgfpathlineto{\pgfqpoint{1.011769in}{6.576812in}}%
\pgfpathlineto{\pgfqpoint{1.016251in}{6.574812in}}%
\pgfpathlineto{\pgfqpoint{1.017371in}{6.571480in}}%
\pgfpathlineto{\pgfqpoint{1.018492in}{6.576812in}}%
\pgfpathlineto{\pgfqpoint{1.019612in}{6.589476in}}%
\pgfpathlineto{\pgfqpoint{1.022974in}{6.600140in}}%
\pgfpathlineto{\pgfqpoint{1.024094in}{6.600140in}}%
\pgfpathlineto{\pgfqpoint{1.025215in}{6.586143in}}%
\pgfpathlineto{\pgfqpoint{1.026336in}{6.588809in}}%
\pgfpathlineto{\pgfqpoint{1.027456in}{6.606805in}}%
\pgfpathlineto{\pgfqpoint{1.030818in}{6.590809in}}%
\pgfpathlineto{\pgfqpoint{1.031938in}{6.602806in}}%
\pgfpathlineto{\pgfqpoint{1.033059in}{6.597474in}}%
\pgfpathlineto{\pgfqpoint{1.034179in}{6.599473in}}%
\pgfpathlineto{\pgfqpoint{1.035300in}{6.600140in}}%
\pgfpathlineto{\pgfqpoint{1.038661in}{6.598140in}}%
\pgfpathlineto{\pgfqpoint{1.039782in}{6.604806in}}%
\pgfpathlineto{\pgfqpoint{1.040902in}{6.646796in}}%
\pgfpathlineto{\pgfqpoint{1.042023in}{6.647462in}}%
\pgfpathlineto{\pgfqpoint{1.043143in}{6.641464in}}%
\pgfpathlineto{\pgfqpoint{1.047626in}{6.665458in}}%
\pgfpathlineto{\pgfqpoint{1.048746in}{6.647462in}}%
\pgfpathlineto{\pgfqpoint{1.049867in}{6.649462in}}%
\pgfpathlineto{\pgfqpoint{1.050987in}{6.656794in}}%
\pgfpathlineto{\pgfqpoint{1.054349in}{6.626134in}}%
\pgfpathlineto{\pgfqpoint{1.056590in}{6.655461in}}%
\pgfpathlineto{\pgfqpoint{1.057710in}{6.647462in}}%
\pgfpathlineto{\pgfqpoint{1.058831in}{6.646129in}}%
\pgfpathlineto{\pgfqpoint{1.062192in}{6.650128in}}%
\pgfpathlineto{\pgfqpoint{1.063313in}{6.666791in}}%
\pgfpathlineto{\pgfqpoint{1.064433in}{6.671457in}}%
\pgfpathlineto{\pgfqpoint{1.065554in}{6.671457in}}%
\pgfpathlineto{\pgfqpoint{1.066675in}{6.676789in}}%
\pgfpathlineto{\pgfqpoint{1.070036in}{6.668791in}}%
\pgfpathlineto{\pgfqpoint{1.071157in}{6.657460in}}%
\pgfpathlineto{\pgfqpoint{1.072277in}{6.662126in}}%
\pgfpathlineto{\pgfqpoint{1.073398in}{6.672123in}}%
\pgfpathlineto{\pgfqpoint{1.074518in}{6.659460in}}%
\pgfpathlineto{\pgfqpoint{1.077880in}{6.649462in}}%
\pgfpathlineto{\pgfqpoint{1.079000in}{6.652794in}}%
\pgfpathlineto{\pgfqpoint{1.080121in}{6.660126in}}%
\pgfpathlineto{\pgfqpoint{1.081241in}{6.651461in}}%
\pgfpathlineto{\pgfqpoint{1.082362in}{6.655461in}}%
\pgfpathlineto{\pgfqpoint{1.085723in}{6.648795in}}%
\pgfpathlineto{\pgfqpoint{1.086844in}{6.642130in}}%
\pgfpathlineto{\pgfqpoint{1.087965in}{6.641464in}}%
\pgfpathlineto{\pgfqpoint{1.089085in}{6.642130in}}%
\pgfpathlineto{\pgfqpoint{1.093567in}{6.640131in}}%
\pgfpathlineto{\pgfqpoint{1.094688in}{6.654128in}}%
\pgfpathlineto{\pgfqpoint{1.095808in}{6.636132in}}%
\pgfpathlineto{\pgfqpoint{1.096929in}{6.640131in}}%
\pgfpathlineto{\pgfqpoint{1.098049in}{6.632799in}}%
\pgfpathlineto{\pgfqpoint{1.101411in}{6.642130in}}%
\pgfpathlineto{\pgfqpoint{1.102531in}{6.639464in}}%
\pgfpathlineto{\pgfqpoint{1.103652in}{6.666791in}}%
\pgfpathlineto{\pgfqpoint{1.104772in}{6.666791in}}%
\pgfpathlineto{\pgfqpoint{1.105893in}{6.660126in}}%
\pgfpathlineto{\pgfqpoint{1.109255in}{6.626134in}}%
\pgfpathlineto{\pgfqpoint{1.110375in}{6.641464in}}%
\pgfpathlineto{\pgfqpoint{1.111496in}{6.623468in}}%
\pgfpathlineto{\pgfqpoint{1.112616in}{6.618802in}}%
\pgfpathlineto{\pgfqpoint{1.113737in}{6.570813in}}%
\pgfpathlineto{\pgfqpoint{1.117098in}{6.549485in}}%
\pgfpathlineto{\pgfqpoint{1.118219in}{6.557483in}}%
\pgfpathlineto{\pgfqpoint{1.119339in}{6.581478in}}%
\pgfpathlineto{\pgfqpoint{1.120460in}{6.581478in}}%
\pgfpathlineto{\pgfqpoint{1.121580in}{6.594808in}}%
\pgfpathlineto{\pgfqpoint{1.126062in}{6.598807in}}%
\pgfpathlineto{\pgfqpoint{1.127183in}{6.591475in}}%
\pgfpathlineto{\pgfqpoint{1.129424in}{6.613470in}}%
\pgfpathlineto{\pgfqpoint{1.132786in}{6.614137in}}%
\pgfpathlineto{\pgfqpoint{1.133906in}{6.619469in}}%
\pgfpathlineto{\pgfqpoint{1.135027in}{6.636798in}}%
\pgfpathlineto{\pgfqpoint{1.136147in}{6.624801in}}%
\pgfpathlineto{\pgfqpoint{1.137268in}{6.630800in}}%
\pgfpathlineto{\pgfqpoint{1.140629in}{6.628134in}}%
\pgfpathlineto{\pgfqpoint{1.145111in}{6.660126in}}%
\pgfpathlineto{\pgfqpoint{1.148473in}{6.666125in}}%
\pgfpathlineto{\pgfqpoint{1.149594in}{6.670790in}}%
\pgfpathlineto{\pgfqpoint{1.150714in}{6.681455in}}%
\pgfpathlineto{\pgfqpoint{1.152955in}{6.664125in}}%
\pgfpathlineto{\pgfqpoint{1.157437in}{6.667458in}}%
\pgfpathlineto{\pgfqpoint{1.158558in}{6.669457in}}%
\pgfpathlineto{\pgfqpoint{1.159678in}{6.667458in}}%
\pgfpathlineto{\pgfqpoint{1.160799in}{6.652794in}}%
\pgfpathlineto{\pgfqpoint{1.164160in}{6.669457in}}%
\pgfpathlineto{\pgfqpoint{1.165281in}{6.670790in}}%
\pgfpathlineto{\pgfqpoint{1.166401in}{6.652794in}}%
\pgfpathlineto{\pgfqpoint{1.167522in}{6.656127in}}%
\pgfpathlineto{\pgfqpoint{1.168642in}{6.681455in}}%
\pgfpathlineto{\pgfqpoint{1.172004in}{6.676789in}}%
\pgfpathlineto{\pgfqpoint{1.173125in}{6.666791in}}%
\pgfpathlineto{\pgfqpoint{1.174245in}{6.662126in}}%
\pgfpathlineto{\pgfqpoint{1.175366in}{6.671457in}}%
\pgfpathlineto{\pgfqpoint{1.176486in}{6.663459in}}%
\pgfpathlineto{\pgfqpoint{1.179848in}{6.676789in}}%
\pgfpathlineto{\pgfqpoint{1.180968in}{6.706115in}}%
\pgfpathlineto{\pgfqpoint{1.182089in}{6.687453in}}%
\pgfpathlineto{\pgfqpoint{1.183209in}{6.658793in}}%
\pgfpathlineto{\pgfqpoint{1.184330in}{6.664792in}}%
\pgfpathlineto{\pgfqpoint{1.187691in}{6.642130in}}%
\pgfpathlineto{\pgfqpoint{1.189933in}{6.658793in}}%
\pgfpathlineto{\pgfqpoint{1.191053in}{6.662792in}}%
\pgfpathlineto{\pgfqpoint{1.192174in}{6.656127in}}%
\pgfpathlineto{\pgfqpoint{1.195535in}{6.663459in}}%
\pgfpathlineto{\pgfqpoint{1.196656in}{6.640797in}}%
\pgfpathlineto{\pgfqpoint{1.197776in}{6.640797in}}%
\pgfpathlineto{\pgfqpoint{1.200017in}{6.658127in}}%
\pgfpathlineto{\pgfqpoint{1.203379in}{6.662792in}}%
\pgfpathlineto{\pgfqpoint{1.204499in}{6.678788in}}%
\pgfpathlineto{\pgfqpoint{1.205620in}{6.674123in}}%
\pgfpathlineto{\pgfqpoint{1.206740in}{6.695451in}}%
\pgfpathlineto{\pgfqpoint{1.207861in}{6.686120in}}%
\pgfpathlineto{\pgfqpoint{1.211223in}{6.678788in}}%
\pgfpathlineto{\pgfqpoint{1.212343in}{6.668791in}}%
\pgfpathlineto{\pgfqpoint{1.214584in}{6.678788in}}%
\pgfpathlineto{\pgfqpoint{1.215705in}{6.736775in}}%
\pgfpathlineto{\pgfqpoint{1.219066in}{6.744107in}}%
\pgfpathlineto{\pgfqpoint{1.221307in}{6.731443in}}%
\pgfpathlineto{\pgfqpoint{1.222428in}{6.734776in}}%
\pgfpathlineto{\pgfqpoint{1.223548in}{6.732776in}}%
\pgfpathlineto{\pgfqpoint{1.226910in}{6.724111in}}%
\pgfpathlineto{\pgfqpoint{1.228030in}{6.724111in}}%
\pgfpathlineto{\pgfqpoint{1.229151in}{6.718113in}}%
\pgfpathlineto{\pgfqpoint{1.230271in}{6.731443in}}%
\pgfpathlineto{\pgfqpoint{1.231392in}{6.735442in}}%
\pgfpathlineto{\pgfqpoint{1.234754in}{6.726111in}}%
\pgfpathlineto{\pgfqpoint{1.235874in}{6.714780in}}%
\pgfpathlineto{\pgfqpoint{1.236995in}{6.716780in}}%
\pgfpathlineto{\pgfqpoint{1.238115in}{6.716113in}}%
\pgfpathlineto{\pgfqpoint{1.239236in}{6.711448in}}%
\pgfpathlineto{\pgfqpoint{1.242597in}{6.712781in}}%
\pgfpathlineto{\pgfqpoint{1.243718in}{6.709448in}}%
\pgfpathlineto{\pgfqpoint{1.244838in}{6.702116in}}%
\pgfpathlineto{\pgfqpoint{1.247079in}{6.696118in}}%
\pgfpathlineto{\pgfqpoint{1.250441in}{6.690786in}}%
\pgfpathlineto{\pgfqpoint{1.252682in}{6.678122in}}%
\pgfpathlineto{\pgfqpoint{1.253803in}{6.686787in}}%
\pgfpathlineto{\pgfqpoint{1.254923in}{6.686787in}}%
\pgfpathlineto{\pgfqpoint{1.258285in}{6.678122in}}%
\pgfpathlineto{\pgfqpoint{1.259405in}{6.655461in}}%
\pgfpathlineto{\pgfqpoint{1.260526in}{6.656127in}}%
\pgfpathlineto{\pgfqpoint{1.261646in}{6.651461in}}%
\pgfpathlineto{\pgfqpoint{1.262767in}{6.653461in}}%
\pgfpathlineto{\pgfqpoint{1.267249in}{6.648795in}}%
\pgfpathlineto{\pgfqpoint{1.268369in}{6.654794in}}%
\pgfpathlineto{\pgfqpoint{1.269490in}{6.654128in}}%
\pgfpathlineto{\pgfqpoint{1.270610in}{6.654128in}}%
\pgfpathlineto{\pgfqpoint{1.273972in}{6.666125in}}%
\pgfpathlineto{\pgfqpoint{1.275093in}{6.691452in}}%
\pgfpathlineto{\pgfqpoint{1.276213in}{6.703449in}}%
\pgfpathlineto{\pgfqpoint{1.277334in}{6.690786in}}%
\pgfpathlineto{\pgfqpoint{1.278454in}{6.686787in}}%
\pgfpathlineto{\pgfqpoint{1.281816in}{6.706115in}}%
\pgfpathlineto{\pgfqpoint{1.284057in}{6.744107in}}%
\pgfpathlineto{\pgfqpoint{1.285177in}{6.732776in}}%
\pgfpathlineto{\pgfqpoint{1.286298in}{6.708782in}}%
\pgfpathlineto{\pgfqpoint{1.289659in}{6.723445in}}%
\pgfpathlineto{\pgfqpoint{1.290780in}{6.725444in}}%
\pgfpathlineto{\pgfqpoint{1.291900in}{6.720779in}}%
\pgfpathlineto{\pgfqpoint{1.293021in}{6.721445in}}%
\pgfpathlineto{\pgfqpoint{1.294142in}{6.710781in}}%
\pgfpathlineto{\pgfqpoint{1.297503in}{6.702783in}}%
\pgfpathlineto{\pgfqpoint{1.299744in}{6.726111in}}%
\pgfpathlineto{\pgfqpoint{1.300865in}{6.713447in}}%
\pgfpathlineto{\pgfqpoint{1.301985in}{6.710781in}}%
\pgfpathlineto{\pgfqpoint{1.305347in}{6.705449in}}%
\pgfpathlineto{\pgfqpoint{1.306467in}{6.690786in}}%
\pgfpathlineto{\pgfqpoint{1.307588in}{6.685454in}}%
\pgfpathlineto{\pgfqpoint{1.308708in}{6.721445in}}%
\pgfpathlineto{\pgfqpoint{1.309829in}{6.729443in}}%
\pgfpathlineto{\pgfqpoint{1.313190in}{6.728777in}}%
\pgfpathlineto{\pgfqpoint{1.314311in}{6.718113in}}%
\pgfpathlineto{\pgfqpoint{1.315432in}{6.727444in}}%
\pgfpathlineto{\pgfqpoint{1.316552in}{6.744773in}}%
\pgfpathlineto{\pgfqpoint{1.317673in}{6.790763in}}%
\pgfpathlineto{\pgfqpoint{1.321034in}{6.822089in}}%
\pgfpathlineto{\pgfqpoint{1.322155in}{6.816090in}}%
\pgfpathlineto{\pgfqpoint{1.323275in}{6.798761in}}%
\pgfpathlineto{\pgfqpoint{1.324396in}{6.811424in}}%
\pgfpathlineto{\pgfqpoint{1.325516in}{6.808092in}}%
\pgfpathlineto{\pgfqpoint{1.328878in}{6.819423in}}%
\pgfpathlineto{\pgfqpoint{1.331119in}{6.834086in}}%
\pgfpathlineto{\pgfqpoint{1.332239in}{6.822089in}}%
\pgfpathlineto{\pgfqpoint{1.333360in}{6.843417in}}%
\pgfpathlineto{\pgfqpoint{1.336722in}{6.837418in}}%
\pgfpathlineto{\pgfqpoint{1.337842in}{6.836752in}}%
\pgfpathlineto{\pgfqpoint{1.338963in}{6.862746in}}%
\pgfpathlineto{\pgfqpoint{1.340083in}{6.846750in}}%
\pgfpathlineto{\pgfqpoint{1.341204in}{6.870744in}}%
\pgfpathlineto{\pgfqpoint{1.344565in}{6.868745in}}%
\pgfpathlineto{\pgfqpoint{1.345686in}{6.870744in}}%
\pgfpathlineto{\pgfqpoint{1.346806in}{6.876076in}}%
\pgfpathlineto{\pgfqpoint{1.347927in}{6.867412in}}%
\pgfpathlineto{\pgfqpoint{1.349047in}{6.878742in}}%
\pgfpathlineto{\pgfqpoint{1.352409in}{6.879409in}}%
\pgfpathlineto{\pgfqpoint{1.353529in}{6.869411in}}%
\pgfpathlineto{\pgfqpoint{1.355771in}{6.863412in}}%
\pgfpathlineto{\pgfqpoint{1.356891in}{6.872077in}}%
\pgfpathlineto{\pgfqpoint{1.360253in}{6.853415in}}%
\pgfpathlineto{\pgfqpoint{1.362494in}{6.858747in}}%
\pgfpathlineto{\pgfqpoint{1.364735in}{6.850082in}}%
\pgfpathlineto{\pgfqpoint{1.368096in}{6.850082in}}%
\pgfpathlineto{\pgfqpoint{1.369217in}{6.844750in}}%
\pgfpathlineto{\pgfqpoint{1.370337in}{6.848749in}}%
\pgfpathlineto{\pgfqpoint{1.371458in}{6.838752in}}%
\pgfpathlineto{\pgfqpoint{1.372578in}{6.864746in}}%
\pgfpathlineto{\pgfqpoint{1.375940in}{6.878076in}}%
\pgfpathlineto{\pgfqpoint{1.377061in}{6.875410in}}%
\pgfpathlineto{\pgfqpoint{1.378181in}{6.845417in}}%
\pgfpathlineto{\pgfqpoint{1.379302in}{6.843417in}}%
\pgfpathlineto{\pgfqpoint{1.380422in}{6.859413in}}%
\pgfpathlineto{\pgfqpoint{1.384904in}{6.869411in}}%
\pgfpathlineto{\pgfqpoint{1.386025in}{6.889406in}}%
\pgfpathlineto{\pgfqpoint{1.387145in}{6.896738in}}%
\pgfpathlineto{\pgfqpoint{1.388266in}{6.898738in}}%
\pgfpathlineto{\pgfqpoint{1.391627in}{6.900737in}}%
\pgfpathlineto{\pgfqpoint{1.392748in}{6.912068in}}%
\pgfpathlineto{\pgfqpoint{1.394989in}{6.924065in}}%
\pgfpathlineto{\pgfqpoint{1.396109in}{6.924065in}}%
\pgfpathlineto{\pgfqpoint{1.399471in}{6.927398in}}%
\pgfpathlineto{\pgfqpoint{1.400592in}{6.934729in}}%
\pgfpathlineto{\pgfqpoint{1.402833in}{6.906069in}}%
\pgfpathlineto{\pgfqpoint{1.403953in}{6.905403in}}%
\pgfpathlineto{\pgfqpoint{1.407315in}{6.893406in}}%
\pgfpathlineto{\pgfqpoint{1.408435in}{6.894739in}}%
\pgfpathlineto{\pgfqpoint{1.409556in}{6.890740in}}%
\pgfpathlineto{\pgfqpoint{1.410676in}{6.891406in}}%
\pgfpathlineto{\pgfqpoint{1.411797in}{6.877409in}}%
\pgfpathlineto{\pgfqpoint{1.415158in}{6.864746in}}%
\pgfpathlineto{\pgfqpoint{1.416279in}{6.878076in}}%
\pgfpathlineto{\pgfqpoint{1.417400in}{6.898071in}}%
\pgfpathlineto{\pgfqpoint{1.418520in}{6.890073in}}%
\pgfpathlineto{\pgfqpoint{1.419641in}{6.856747in}}%
\pgfpathlineto{\pgfqpoint{1.424123in}{6.841418in}}%
\pgfpathlineto{\pgfqpoint{1.426364in}{6.816090in}}%
\pgfpathlineto{\pgfqpoint{1.427484in}{6.769434in}}%
\pgfpathlineto{\pgfqpoint{1.430846in}{6.776099in}}%
\pgfpathlineto{\pgfqpoint{1.431966in}{6.796761in}}%
\pgfpathlineto{\pgfqpoint{1.433087in}{6.788097in}}%
\pgfpathlineto{\pgfqpoint{1.434207in}{6.798761in}}%
\pgfpathlineto{\pgfqpoint{1.435328in}{6.779432in}}%
\pgfpathlineto{\pgfqpoint{1.438690in}{6.737442in}}%
\pgfpathlineto{\pgfqpoint{1.439810in}{6.749439in}}%
\pgfpathlineto{\pgfqpoint{1.440931in}{6.746773in}}%
\pgfpathlineto{\pgfqpoint{1.443172in}{6.782764in}}%
\pgfpathlineto{\pgfqpoint{1.446533in}{6.774766in}}%
\pgfpathlineto{\pgfqpoint{1.447654in}{6.795428in}}%
\pgfpathlineto{\pgfqpoint{1.448774in}{6.793429in}}%
\pgfpathlineto{\pgfqpoint{1.449895in}{6.796095in}}%
\pgfpathlineto{\pgfqpoint{1.451015in}{6.812091in}}%
\pgfpathlineto{\pgfqpoint{1.455497in}{6.807425in}}%
\pgfpathlineto{\pgfqpoint{1.456618in}{6.793429in}}%
\pgfpathlineto{\pgfqpoint{1.457738in}{6.790763in}}%
\pgfpathlineto{\pgfqpoint{1.458859in}{6.780765in}}%
\pgfpathlineto{\pgfqpoint{1.462221in}{6.799427in}}%
\pgfpathlineto{\pgfqpoint{1.463341in}{6.798761in}}%
\pgfpathlineto{\pgfqpoint{1.464462in}{6.800094in}}%
\pgfpathlineto{\pgfqpoint{1.465582in}{6.810758in}}%
\pgfpathlineto{\pgfqpoint{1.466703in}{6.809425in}}%
\pgfpathlineto{\pgfqpoint{1.470064in}{6.790763in}}%
\pgfpathlineto{\pgfqpoint{1.472305in}{6.834086in}}%
\pgfpathlineto{\pgfqpoint{1.473426in}{6.850082in}}%
\pgfpathlineto{\pgfqpoint{1.474546in}{6.845417in}}%
\pgfpathlineto{\pgfqpoint{1.477908in}{6.840085in}}%
\pgfpathlineto{\pgfqpoint{1.480149in}{6.825421in}}%
\pgfpathlineto{\pgfqpoint{1.482390in}{6.790096in}}%
\pgfpathlineto{\pgfqpoint{1.485752in}{6.807425in}}%
\pgfpathlineto{\pgfqpoint{1.486872in}{6.819423in}}%
\pgfpathlineto{\pgfqpoint{1.487993in}{6.799427in}}%
\pgfpathlineto{\pgfqpoint{1.489113in}{6.798761in}}%
\pgfpathlineto{\pgfqpoint{1.490234in}{6.805426in}}%
\pgfpathlineto{\pgfqpoint{1.493595in}{6.806092in}}%
\pgfpathlineto{\pgfqpoint{1.494716in}{6.822089in}}%
\pgfpathlineto{\pgfqpoint{1.495836in}{6.817423in}}%
\pgfpathlineto{\pgfqpoint{1.496957in}{6.828087in}}%
\pgfpathlineto{\pgfqpoint{1.498077in}{6.831420in}}%
\pgfpathlineto{\pgfqpoint{1.501439in}{6.832086in}}%
\pgfpathlineto{\pgfqpoint{1.502560in}{6.830753in}}%
\pgfpathlineto{\pgfqpoint{1.504801in}{6.850749in}}%
\pgfpathlineto{\pgfqpoint{1.505921in}{6.839418in}}%
\pgfpathlineto{\pgfqpoint{1.509283in}{6.830087in}}%
\pgfpathlineto{\pgfqpoint{1.510403in}{6.824755in}}%
\pgfpathlineto{\pgfqpoint{1.511524in}{6.835419in}}%
\pgfpathlineto{\pgfqpoint{1.512644in}{6.815424in}}%
\pgfpathlineto{\pgfqpoint{1.513765in}{6.807425in}}%
\pgfpathlineto{\pgfqpoint{1.518247in}{6.828087in}}%
\pgfpathlineto{\pgfqpoint{1.520488in}{6.868078in}}%
\pgfpathlineto{\pgfqpoint{1.524970in}{6.870078in}}%
\pgfpathlineto{\pgfqpoint{1.526091in}{6.869411in}}%
\pgfpathlineto{\pgfqpoint{1.527211in}{6.860746in}}%
\pgfpathlineto{\pgfqpoint{1.528332in}{6.862746in}}%
\pgfpathlineto{\pgfqpoint{1.529452in}{6.870744in}}%
\pgfpathlineto{\pgfqpoint{1.532814in}{6.880075in}}%
\pgfpathlineto{\pgfqpoint{1.533934in}{6.879409in}}%
\pgfpathlineto{\pgfqpoint{1.535055in}{6.886074in}}%
\pgfpathlineto{\pgfqpoint{1.537296in}{6.874743in}}%
\pgfpathlineto{\pgfqpoint{1.540658in}{6.869411in}}%
\pgfpathlineto{\pgfqpoint{1.541778in}{6.848083in}}%
\pgfpathlineto{\pgfqpoint{1.542899in}{6.866745in}}%
\pgfpathlineto{\pgfqpoint{1.544019in}{6.862079in}}%
\pgfpathlineto{\pgfqpoint{1.545140in}{6.860746in}}%
\pgfpathlineto{\pgfqpoint{1.549622in}{6.888073in}}%
\pgfpathlineto{\pgfqpoint{1.551863in}{6.870744in}}%
\pgfpathlineto{\pgfqpoint{1.552983in}{6.874077in}}%
\pgfpathlineto{\pgfqpoint{1.556345in}{6.870744in}}%
\pgfpathlineto{\pgfqpoint{1.557465in}{6.854081in}}%
\pgfpathlineto{\pgfqpoint{1.558586in}{6.864746in}}%
\pgfpathlineto{\pgfqpoint{1.559706in}{6.865412in}}%
\pgfpathlineto{\pgfqpoint{1.560827in}{6.865412in}}%
\pgfpathlineto{\pgfqpoint{1.565309in}{6.868745in}}%
\pgfpathlineto{\pgfqpoint{1.567550in}{6.878076in}}%
\pgfpathlineto{\pgfqpoint{1.568671in}{6.880742in}}%
\pgfpathlineto{\pgfqpoint{1.572032in}{6.882741in}}%
\pgfpathlineto{\pgfqpoint{1.573153in}{6.880742in}}%
\pgfpathlineto{\pgfqpoint{1.574273in}{6.868078in}}%
\pgfpathlineto{\pgfqpoint{1.575394in}{6.879409in}}%
\pgfpathlineto{\pgfqpoint{1.576514in}{6.902070in}}%
\pgfpathlineto{\pgfqpoint{1.579876in}{6.916067in}}%
\pgfpathlineto{\pgfqpoint{1.580996in}{6.914067in}}%
\pgfpathlineto{\pgfqpoint{1.583238in}{6.890073in}}%
\pgfpathlineto{\pgfqpoint{1.584358in}{6.894072in}}%
\pgfpathlineto{\pgfqpoint{1.587720in}{6.882741in}}%
\pgfpathlineto{\pgfqpoint{1.588840in}{6.885407in}}%
\pgfpathlineto{\pgfqpoint{1.589961in}{6.886074in}}%
\pgfpathlineto{\pgfqpoint{1.591081in}{6.900071in}}%
\pgfpathlineto{\pgfqpoint{1.592202in}{6.902737in}}%
\pgfpathlineto{\pgfqpoint{1.596684in}{6.881408in}}%
\pgfpathlineto{\pgfqpoint{1.598925in}{6.865412in}}%
\pgfpathlineto{\pgfqpoint{1.600045in}{6.873410in}}%
\pgfpathlineto{\pgfqpoint{1.603407in}{6.864746in}}%
\pgfpathlineto{\pgfqpoint{1.604528in}{6.871411in}}%
\pgfpathlineto{\pgfqpoint{1.606769in}{6.896738in}}%
\pgfpathlineto{\pgfqpoint{1.611251in}{6.890740in}}%
\pgfpathlineto{\pgfqpoint{1.612371in}{6.870078in}}%
\pgfpathlineto{\pgfqpoint{1.613492in}{6.867412in}}%
\pgfpathlineto{\pgfqpoint{1.614612in}{6.860746in}}%
\pgfpathlineto{\pgfqpoint{1.615733in}{6.879409in}}%
\pgfpathlineto{\pgfqpoint{1.619094in}{6.885407in}}%
\pgfpathlineto{\pgfqpoint{1.620215in}{6.882741in}}%
\pgfpathlineto{\pgfqpoint{1.621335in}{6.905403in}}%
\pgfpathlineto{\pgfqpoint{1.622456in}{6.882741in}}%
\pgfpathlineto{\pgfqpoint{1.626938in}{6.848749in}}%
\pgfpathlineto{\pgfqpoint{1.628059in}{6.850749in}}%
\pgfpathlineto{\pgfqpoint{1.629179in}{6.844750in}}%
\pgfpathlineto{\pgfqpoint{1.630300in}{6.846083in}}%
\pgfpathlineto{\pgfqpoint{1.631420in}{6.838085in}}%
\pgfpathlineto{\pgfqpoint{1.634782in}{6.827421in}}%
\pgfpathlineto{\pgfqpoint{1.635902in}{6.819423in}}%
\pgfpathlineto{\pgfqpoint{1.637023in}{6.830087in}}%
\pgfpathlineto{\pgfqpoint{1.638143in}{6.803426in}}%
\pgfpathlineto{\pgfqpoint{1.639264in}{6.814091in}}%
\pgfpathlineto{\pgfqpoint{1.642625in}{6.810091in}}%
\pgfpathlineto{\pgfqpoint{1.643746in}{6.796095in}}%
\pgfpathlineto{\pgfqpoint{1.644867in}{6.819423in}}%
\pgfpathlineto{\pgfqpoint{1.645987in}{6.822089in}}%
\pgfpathlineto{\pgfqpoint{1.647108in}{6.831420in}}%
\pgfpathlineto{\pgfqpoint{1.650469in}{6.838085in}}%
\pgfpathlineto{\pgfqpoint{1.651590in}{6.828087in}}%
\pgfpathlineto{\pgfqpoint{1.652710in}{6.840085in}}%
\pgfpathlineto{\pgfqpoint{1.653831in}{6.843417in}}%
\pgfpathlineto{\pgfqpoint{1.654951in}{6.830087in}}%
\pgfpathlineto{\pgfqpoint{1.658313in}{6.853415in}}%
\pgfpathlineto{\pgfqpoint{1.659433in}{6.852082in}}%
\pgfpathlineto{\pgfqpoint{1.660554in}{6.869411in}}%
\pgfpathlineto{\pgfqpoint{1.661674in}{6.873410in}}%
\pgfpathlineto{\pgfqpoint{1.662795in}{6.858080in}}%
\pgfpathlineto{\pgfqpoint{1.666157in}{6.860746in}}%
\pgfpathlineto{\pgfqpoint{1.667277in}{6.850082in}}%
\pgfpathlineto{\pgfqpoint{1.668398in}{6.856747in}}%
\pgfpathlineto{\pgfqpoint{1.669518in}{6.850082in}}%
\pgfpathlineto{\pgfqpoint{1.670639in}{6.848749in}}%
\pgfpathlineto{\pgfqpoint{1.675121in}{6.841418in}}%
\pgfpathlineto{\pgfqpoint{1.676241in}{6.846750in}}%
\pgfpathlineto{\pgfqpoint{1.677362in}{6.847416in}}%
\pgfpathlineto{\pgfqpoint{1.678482in}{6.855414in}}%
\pgfpathlineto{\pgfqpoint{1.681844in}{6.854081in}}%
\pgfpathlineto{\pgfqpoint{1.682964in}{6.844084in}}%
\pgfpathlineto{\pgfqpoint{1.685206in}{6.850749in}}%
\pgfpathlineto{\pgfqpoint{1.686326in}{6.842751in}}%
\pgfpathlineto{\pgfqpoint{1.689688in}{6.845417in}}%
\pgfpathlineto{\pgfqpoint{1.690808in}{6.861413in}}%
\pgfpathlineto{\pgfqpoint{1.691929in}{6.864079in}}%
\pgfpathlineto{\pgfqpoint{1.693049in}{6.873410in}}%
\pgfpathlineto{\pgfqpoint{1.694170in}{6.877409in}}%
\pgfpathlineto{\pgfqpoint{1.699772in}{6.858080in}}%
\pgfpathlineto{\pgfqpoint{1.700893in}{6.836752in}}%
\pgfpathlineto{\pgfqpoint{1.702013in}{6.841418in}}%
\pgfpathlineto{\pgfqpoint{1.705375in}{6.829420in}}%
\pgfpathlineto{\pgfqpoint{1.706496in}{6.840751in}}%
\pgfpathlineto{\pgfqpoint{1.707616in}{6.815424in}}%
\pgfpathlineto{\pgfqpoint{1.708737in}{6.813424in}}%
\pgfpathlineto{\pgfqpoint{1.709857in}{6.828754in}}%
\pgfpathlineto{\pgfqpoint{1.713219in}{6.818756in}}%
\pgfpathlineto{\pgfqpoint{1.714339in}{6.796095in}}%
\pgfpathlineto{\pgfqpoint{1.715460in}{6.820089in}}%
\pgfpathlineto{\pgfqpoint{1.717701in}{6.766768in}}%
\pgfpathlineto{\pgfqpoint{1.721062in}{6.748772in}}%
\pgfpathlineto{\pgfqpoint{1.723303in}{6.766768in}}%
\pgfpathlineto{\pgfqpoint{1.724424in}{6.765435in}}%
\pgfpathlineto{\pgfqpoint{1.725544in}{6.796761in}}%
\pgfpathlineto{\pgfqpoint{1.728906in}{6.808092in}}%
\pgfpathlineto{\pgfqpoint{1.730027in}{6.831420in}}%
\pgfpathlineto{\pgfqpoint{1.731147in}{6.817423in}}%
\pgfpathlineto{\pgfqpoint{1.733388in}{6.842084in}}%
\pgfpathlineto{\pgfqpoint{1.736750in}{6.835419in}}%
\pgfpathlineto{\pgfqpoint{1.737870in}{6.854748in}}%
\pgfpathlineto{\pgfqpoint{1.738991in}{6.842751in}}%
\pgfpathlineto{\pgfqpoint{1.740111in}{6.843417in}}%
\pgfpathlineto{\pgfqpoint{1.741232in}{6.851415in}}%
\pgfpathlineto{\pgfqpoint{1.744593in}{6.845417in}}%
\pgfpathlineto{\pgfqpoint{1.745714in}{6.845417in}}%
\pgfpathlineto{\pgfqpoint{1.746834in}{6.852082in}}%
\pgfpathlineto{\pgfqpoint{1.747955in}{6.881408in}}%
\pgfpathlineto{\pgfqpoint{1.749076in}{6.884074in}}%
\pgfpathlineto{\pgfqpoint{1.752437in}{6.887407in}}%
\pgfpathlineto{\pgfqpoint{1.753558in}{6.882741in}}%
\pgfpathlineto{\pgfqpoint{1.754678in}{6.890073in}}%
\pgfpathlineto{\pgfqpoint{1.755799in}{6.884741in}}%
\pgfpathlineto{\pgfqpoint{1.756919in}{6.886740in}}%
\pgfpathlineto{\pgfqpoint{1.760281in}{6.895405in}}%
\pgfpathlineto{\pgfqpoint{1.761401in}{6.917400in}}%
\pgfpathlineto{\pgfqpoint{1.763642in}{6.908069in}}%
\pgfpathlineto{\pgfqpoint{1.764763in}{6.916067in}}%
\pgfpathlineto{\pgfqpoint{1.768125in}{6.916733in}}%
\pgfpathlineto{\pgfqpoint{1.769245in}{6.908735in}}%
\pgfpathlineto{\pgfqpoint{1.770366in}{6.909402in}}%
\pgfpathlineto{\pgfqpoint{1.775968in}{6.862746in}}%
\pgfpathlineto{\pgfqpoint{1.777089in}{6.864079in}}%
\pgfpathlineto{\pgfqpoint{1.778209in}{6.882741in}}%
\pgfpathlineto{\pgfqpoint{1.779330in}{6.866745in}}%
\pgfpathlineto{\pgfqpoint{1.780450in}{6.862079in}}%
\pgfpathlineto{\pgfqpoint{1.784932in}{6.838752in}}%
\pgfpathlineto{\pgfqpoint{1.786053in}{6.821422in}}%
\pgfpathlineto{\pgfqpoint{1.787173in}{6.829420in}}%
\pgfpathlineto{\pgfqpoint{1.788294in}{6.800760in}}%
\pgfpathlineto{\pgfqpoint{1.792776in}{6.778765in}}%
\pgfpathlineto{\pgfqpoint{1.793897in}{6.788097in}}%
\pgfpathlineto{\pgfqpoint{1.796138in}{6.854081in}}%
\pgfpathlineto{\pgfqpoint{1.799499in}{6.858747in}}%
\pgfpathlineto{\pgfqpoint{1.800620in}{6.868745in}}%
\pgfpathlineto{\pgfqpoint{1.801740in}{6.865412in}}%
\pgfpathlineto{\pgfqpoint{1.807343in}{6.858747in}}%
\pgfpathlineto{\pgfqpoint{1.808463in}{6.851415in}}%
\pgfpathlineto{\pgfqpoint{1.809584in}{6.834752in}}%
\pgfpathlineto{\pgfqpoint{1.811825in}{6.822755in}}%
\pgfpathlineto{\pgfqpoint{1.815187in}{6.797428in}}%
\pgfpathlineto{\pgfqpoint{1.816307in}{6.768101in}}%
\pgfpathlineto{\pgfqpoint{1.817428in}{6.768768in}}%
\pgfpathlineto{\pgfqpoint{1.818548in}{6.784764in}}%
\pgfpathlineto{\pgfqpoint{1.819669in}{6.765435in}}%
\pgfpathlineto{\pgfqpoint{1.823030in}{6.762769in}}%
\pgfpathlineto{\pgfqpoint{1.826392in}{6.740774in}}%
\pgfpathlineto{\pgfqpoint{1.827512in}{6.741441in}}%
\pgfpathlineto{\pgfqpoint{1.831995in}{6.755437in}}%
\pgfpathlineto{\pgfqpoint{1.835356in}{6.790763in}}%
\pgfpathlineto{\pgfqpoint{1.838718in}{6.796761in}}%
\pgfpathlineto{\pgfqpoint{1.839838in}{6.785431in}}%
\pgfpathlineto{\pgfqpoint{1.840959in}{6.755437in}}%
\pgfpathlineto{\pgfqpoint{1.842079in}{6.768768in}}%
\pgfpathlineto{\pgfqpoint{1.843200in}{6.758103in}}%
\pgfpathlineto{\pgfqpoint{1.846561in}{6.775433in}}%
\pgfpathlineto{\pgfqpoint{1.847682in}{6.790096in}}%
\pgfpathlineto{\pgfqpoint{1.848802in}{6.772767in}}%
\pgfpathlineto{\pgfqpoint{1.849923in}{6.792096in}}%
\pgfpathlineto{\pgfqpoint{1.851044in}{6.792762in}}%
\pgfpathlineto{\pgfqpoint{1.854405in}{6.799427in}}%
\pgfpathlineto{\pgfqpoint{1.857767in}{6.813424in}}%
\pgfpathlineto{\pgfqpoint{1.858887in}{6.828087in}}%
\pgfpathlineto{\pgfqpoint{1.863369in}{6.828754in}}%
\pgfpathlineto{\pgfqpoint{1.864490in}{6.833419in}}%
\pgfpathlineto{\pgfqpoint{1.865610in}{6.832753in}}%
\pgfpathlineto{\pgfqpoint{1.866731in}{6.844084in}}%
\pgfpathlineto{\pgfqpoint{1.870092in}{6.842084in}}%
\pgfpathlineto{\pgfqpoint{1.871213in}{6.854081in}}%
\pgfpathlineto{\pgfqpoint{1.872334in}{6.883408in}}%
\pgfpathlineto{\pgfqpoint{1.873454in}{6.882075in}}%
\pgfpathlineto{\pgfqpoint{1.874575in}{6.888073in}}%
\pgfpathlineto{\pgfqpoint{1.877936in}{6.894739in}}%
\pgfpathlineto{\pgfqpoint{1.880177in}{6.869411in}}%
\pgfpathlineto{\pgfqpoint{1.881298in}{6.878076in}}%
\pgfpathlineto{\pgfqpoint{1.882418in}{6.856081in}}%
\pgfpathlineto{\pgfqpoint{1.885780in}{6.868078in}}%
\pgfpathlineto{\pgfqpoint{1.886900in}{6.842084in}}%
\pgfpathlineto{\pgfqpoint{1.888021in}{6.842751in}}%
\pgfpathlineto{\pgfqpoint{1.889141in}{6.854748in}}%
\pgfpathlineto{\pgfqpoint{1.890262in}{6.834752in}}%
\pgfpathlineto{\pgfqpoint{1.893624in}{6.857414in}}%
\pgfpathlineto{\pgfqpoint{1.894744in}{6.849416in}}%
\pgfpathlineto{\pgfqpoint{1.895865in}{6.868078in}}%
\pgfpathlineto{\pgfqpoint{1.896985in}{6.850749in}}%
\pgfpathlineto{\pgfqpoint{1.898106in}{6.854748in}}%
\pgfpathlineto{\pgfqpoint{1.901467in}{6.858747in}}%
\pgfpathlineto{\pgfqpoint{1.902588in}{6.847416in}}%
\pgfpathlineto{\pgfqpoint{1.903708in}{6.827421in}}%
\pgfpathlineto{\pgfqpoint{1.904829in}{6.821422in}}%
\pgfpathlineto{\pgfqpoint{1.905949in}{6.824755in}}%
\pgfpathlineto{\pgfqpoint{1.909311in}{6.838752in}}%
\pgfpathlineto{\pgfqpoint{1.910431in}{6.821422in}}%
\pgfpathlineto{\pgfqpoint{1.911552in}{6.823422in}}%
\pgfpathlineto{\pgfqpoint{1.912673in}{6.828754in}}%
\pgfpathlineto{\pgfqpoint{1.917155in}{6.842751in}}%
\pgfpathlineto{\pgfqpoint{1.918275in}{6.833419in}}%
\pgfpathlineto{\pgfqpoint{1.919396in}{6.832753in}}%
\pgfpathlineto{\pgfqpoint{1.920516in}{6.873410in}}%
\pgfpathlineto{\pgfqpoint{1.921637in}{7.028708in}}%
\pgfpathlineto{\pgfqpoint{1.924998in}{6.979386in}}%
\pgfpathlineto{\pgfqpoint{1.926119in}{6.985384in}}%
\pgfpathlineto{\pgfqpoint{1.928360in}{6.960057in}}%
\pgfpathlineto{\pgfqpoint{1.929480in}{6.958724in}}%
\pgfpathlineto{\pgfqpoint{1.932842in}{6.945394in}}%
\pgfpathlineto{\pgfqpoint{1.933963in}{6.923399in}}%
\pgfpathlineto{\pgfqpoint{1.935083in}{6.939395in}}%
\pgfpathlineto{\pgfqpoint{1.937324in}{6.933396in}}%
\pgfpathlineto{\pgfqpoint{1.940686in}{6.937395in}}%
\pgfpathlineto{\pgfqpoint{1.941806in}{6.951392in}}%
\pgfpathlineto{\pgfqpoint{1.942927in}{6.949393in}}%
\pgfpathlineto{\pgfqpoint{1.944047in}{6.948726in}}%
\pgfpathlineto{\pgfqpoint{1.945168in}{6.962056in}}%
\pgfpathlineto{\pgfqpoint{1.948529in}{6.959390in}}%
\pgfpathlineto{\pgfqpoint{1.949650in}{6.940061in}}%
\pgfpathlineto{\pgfqpoint{1.950770in}{6.934063in}}%
\pgfpathlineto{\pgfqpoint{1.953011in}{6.964722in}}%
\pgfpathlineto{\pgfqpoint{1.956373in}{6.940061in}}%
\pgfpathlineto{\pgfqpoint{1.957494in}{6.946060in}}%
\pgfpathlineto{\pgfqpoint{1.959735in}{6.967388in}}%
\pgfpathlineto{\pgfqpoint{1.960855in}{6.959390in}}%
\pgfpathlineto{\pgfqpoint{1.964217in}{6.962056in}}%
\pgfpathlineto{\pgfqpoint{1.965337in}{6.964056in}}%
\pgfpathlineto{\pgfqpoint{1.966458in}{6.980052in}}%
\pgfpathlineto{\pgfqpoint{1.967578in}{6.984718in}}%
\pgfpathlineto{\pgfqpoint{1.968699in}{6.982718in}}%
\pgfpathlineto{\pgfqpoint{1.973181in}{6.973387in}}%
\pgfpathlineto{\pgfqpoint{1.974302in}{6.973387in}}%
\pgfpathlineto{\pgfqpoint{1.975422in}{6.979386in}}%
\pgfpathlineto{\pgfqpoint{1.976543in}{6.959390in}}%
\pgfpathlineto{\pgfqpoint{1.979904in}{6.960057in}}%
\pgfpathlineto{\pgfqpoint{1.981025in}{6.962723in}}%
\pgfpathlineto{\pgfqpoint{1.982145in}{6.974054in}}%
\pgfpathlineto{\pgfqpoint{1.983266in}{6.958724in}}%
\pgfpathlineto{\pgfqpoint{1.984386in}{6.960723in}}%
\pgfpathlineto{\pgfqpoint{1.987748in}{6.958057in}}%
\pgfpathlineto{\pgfqpoint{1.988868in}{6.962723in}}%
\pgfpathlineto{\pgfqpoint{1.989989in}{6.979386in}}%
\pgfpathlineto{\pgfqpoint{1.992230in}{6.966055in}}%
\pgfpathlineto{\pgfqpoint{1.995592in}{6.956058in}}%
\pgfpathlineto{\pgfqpoint{1.996712in}{6.956724in}}%
\pgfpathlineto{\pgfqpoint{1.997833in}{6.959390in}}%
\pgfpathlineto{\pgfqpoint{1.998953in}{6.978053in}}%
\pgfpathlineto{\pgfqpoint{2.000074in}{6.970721in}}%
\pgfpathlineto{\pgfqpoint{2.003435in}{6.980719in}}%
\pgfpathlineto{\pgfqpoint{2.004556in}{6.988050in}}%
\pgfpathlineto{\pgfqpoint{2.006797in}{6.959390in}}%
\pgfpathlineto{\pgfqpoint{2.007917in}{6.962056in}}%
\pgfpathlineto{\pgfqpoint{2.011279in}{6.936729in}}%
\pgfpathlineto{\pgfqpoint{2.012399in}{6.932730in}}%
\pgfpathlineto{\pgfqpoint{2.014640in}{6.944727in}}%
\pgfpathlineto{\pgfqpoint{2.019123in}{6.918067in}}%
\pgfpathlineto{\pgfqpoint{2.020243in}{6.927398in}}%
\pgfpathlineto{\pgfqpoint{2.021364in}{6.894739in}}%
\pgfpathlineto{\pgfqpoint{2.022484in}{6.902070in}}%
\pgfpathlineto{\pgfqpoint{2.023605in}{6.916067in}}%
\pgfpathlineto{\pgfqpoint{2.026966in}{6.927398in}}%
\pgfpathlineto{\pgfqpoint{2.031448in}{6.970721in}}%
\pgfpathlineto{\pgfqpoint{2.034810in}{6.965389in}}%
\pgfpathlineto{\pgfqpoint{2.038172in}{6.915400in}}%
\pgfpathlineto{\pgfqpoint{2.039292in}{6.886740in}}%
\pgfpathlineto{\pgfqpoint{2.042654in}{6.898071in}}%
\pgfpathlineto{\pgfqpoint{2.044895in}{6.915400in}}%
\pgfpathlineto{\pgfqpoint{2.046015in}{6.907402in}}%
\pgfpathlineto{\pgfqpoint{2.047136in}{6.906736in}}%
\pgfpathlineto{\pgfqpoint{2.050497in}{6.893406in}}%
\pgfpathlineto{\pgfqpoint{2.051618in}{6.895405in}}%
\pgfpathlineto{\pgfqpoint{2.052738in}{6.906736in}}%
\pgfpathlineto{\pgfqpoint{2.053859in}{6.902737in}}%
\pgfpathlineto{\pgfqpoint{2.054979in}{6.888740in}}%
\pgfpathlineto{\pgfqpoint{2.058341in}{6.914067in}}%
\pgfpathlineto{\pgfqpoint{2.059462in}{6.884074in}}%
\pgfpathlineto{\pgfqpoint{2.060582in}{6.892739in}}%
\pgfpathlineto{\pgfqpoint{2.061703in}{6.888740in}}%
\pgfpathlineto{\pgfqpoint{2.062823in}{6.905403in}}%
\pgfpathlineto{\pgfqpoint{2.066185in}{6.912734in}}%
\pgfpathlineto{\pgfqpoint{2.067305in}{6.904736in}}%
\pgfpathlineto{\pgfqpoint{2.068426in}{6.885407in}}%
\pgfpathlineto{\pgfqpoint{2.070667in}{6.821422in}}%
\pgfpathlineto{\pgfqpoint{2.074028in}{6.780098in}}%
\pgfpathlineto{\pgfqpoint{2.075149in}{6.746773in}}%
\pgfpathlineto{\pgfqpoint{2.078511in}{6.853415in}}%
\pgfpathlineto{\pgfqpoint{2.081872in}{6.834086in}}%
\pgfpathlineto{\pgfqpoint{2.082993in}{6.780765in}}%
\pgfpathlineto{\pgfqpoint{2.084113in}{6.820089in}}%
\pgfpathlineto{\pgfqpoint{2.085234in}{6.816757in}}%
\pgfpathlineto{\pgfqpoint{2.086354in}{6.788097in}}%
\pgfpathlineto{\pgfqpoint{2.090836in}{6.842084in}}%
\pgfpathlineto{\pgfqpoint{2.091957in}{6.818756in}}%
\pgfpathlineto{\pgfqpoint{2.093077in}{6.826088in}}%
\pgfpathlineto{\pgfqpoint{2.094198in}{6.841418in}}%
\pgfpathlineto{\pgfqpoint{2.097559in}{6.831420in}}%
\pgfpathlineto{\pgfqpoint{2.099801in}{6.896738in}}%
\pgfpathlineto{\pgfqpoint{2.100921in}{6.876743in}}%
\pgfpathlineto{\pgfqpoint{2.102042in}{6.845417in}}%
\pgfpathlineto{\pgfqpoint{2.105403in}{6.862079in}}%
\pgfpathlineto{\pgfqpoint{2.107644in}{6.864746in}}%
\pgfpathlineto{\pgfqpoint{2.108765in}{6.852082in}}%
\pgfpathlineto{\pgfqpoint{2.109885in}{6.852082in}}%
\pgfpathlineto{\pgfqpoint{2.113247in}{6.817423in}}%
\pgfpathlineto{\pgfqpoint{2.114367in}{6.832086in}}%
\pgfpathlineto{\pgfqpoint{2.115488in}{6.869411in}}%
\pgfpathlineto{\pgfqpoint{2.116608in}{6.867412in}}%
\pgfpathlineto{\pgfqpoint{2.117729in}{6.883408in}}%
\pgfpathlineto{\pgfqpoint{2.124452in}{7.029374in}}%
\pgfpathlineto{\pgfqpoint{2.125573in}{7.032040in}}%
\pgfpathlineto{\pgfqpoint{2.128934in}{7.033373in}}%
\pgfpathlineto{\pgfqpoint{2.131175in}{7.005380in}}%
\pgfpathlineto{\pgfqpoint{2.132296in}{7.029374in}}%
\pgfpathlineto{\pgfqpoint{2.133416in}{7.084028in}}%
\pgfpathlineto{\pgfqpoint{2.136778in}{7.084695in}}%
\pgfpathlineto{\pgfqpoint{2.137898in}{7.072697in}}%
\pgfpathlineto{\pgfqpoint{2.139019in}{7.076697in}}%
\pgfpathlineto{\pgfqpoint{2.140140in}{7.118020in}}%
\pgfpathlineto{\pgfqpoint{2.141260in}{7.114021in}}%
\pgfpathlineto{\pgfqpoint{2.144622in}{7.116021in}}%
\pgfpathlineto{\pgfqpoint{2.147983in}{7.104024in}}%
\pgfpathlineto{\pgfqpoint{2.149104in}{7.080696in}}%
\pgfpathlineto{\pgfqpoint{2.153586in}{7.118687in}}%
\pgfpathlineto{\pgfqpoint{2.154706in}{7.116021in}}%
\pgfpathlineto{\pgfqpoint{2.155827in}{7.121353in}}%
\pgfpathlineto{\pgfqpoint{2.156947in}{7.137349in}}%
\pgfpathlineto{\pgfqpoint{2.160309in}{7.128018in}}%
\pgfpathlineto{\pgfqpoint{2.161430in}{7.148680in}}%
\pgfpathlineto{\pgfqpoint{2.162550in}{7.180006in}}%
\pgfpathlineto{\pgfqpoint{2.163671in}{7.151346in}}%
\pgfpathlineto{\pgfqpoint{2.164791in}{7.158011in}}%
\pgfpathlineto{\pgfqpoint{2.168153in}{7.162677in}}%
\pgfpathlineto{\pgfqpoint{2.169273in}{7.160011in}}%
\pgfpathlineto{\pgfqpoint{2.170394in}{7.171341in}}%
\pgfpathlineto{\pgfqpoint{2.171514in}{7.157345in}}%
\pgfpathlineto{\pgfqpoint{2.172635in}{7.179339in}}%
\pgfpathlineto{\pgfqpoint{2.175996in}{7.175340in}}%
\pgfpathlineto{\pgfqpoint{2.177117in}{7.179339in}}%
\pgfpathlineto{\pgfqpoint{2.178237in}{7.162677in}}%
\pgfpathlineto{\pgfqpoint{2.180479in}{7.162677in}}%
\pgfpathlineto{\pgfqpoint{2.183840in}{7.138682in}}%
\pgfpathlineto{\pgfqpoint{2.184961in}{7.151346in}}%
\pgfpathlineto{\pgfqpoint{2.186081in}{7.140015in}}%
\pgfpathlineto{\pgfqpoint{2.187202in}{7.143348in}}%
\pgfpathlineto{\pgfqpoint{2.188322in}{7.170008in}}%
\pgfpathlineto{\pgfqpoint{2.191684in}{7.163343in}}%
\pgfpathlineto{\pgfqpoint{2.192804in}{7.152679in}}%
\pgfpathlineto{\pgfqpoint{2.195045in}{7.178673in}}%
\pgfpathlineto{\pgfqpoint{2.196166in}{7.156678in}}%
\pgfpathlineto{\pgfqpoint{2.199527in}{7.156678in}}%
\pgfpathlineto{\pgfqpoint{2.200648in}{7.160011in}}%
\pgfpathlineto{\pgfqpoint{2.201769in}{7.198002in}}%
\pgfpathlineto{\pgfqpoint{2.204010in}{7.170675in}}%
\pgfpathlineto{\pgfqpoint{2.207371in}{7.178006in}}%
\pgfpathlineto{\pgfqpoint{2.208492in}{7.182672in}}%
\pgfpathlineto{\pgfqpoint{2.209612in}{7.209333in}}%
\pgfpathlineto{\pgfqpoint{2.210733in}{7.202667in}}%
\pgfpathlineto{\pgfqpoint{2.215215in}{7.206667in}}%
\pgfpathlineto{\pgfqpoint{2.216335in}{7.227995in}}%
\pgfpathlineto{\pgfqpoint{2.217456in}{7.215331in}}%
\pgfpathlineto{\pgfqpoint{2.218576in}{7.220663in}}%
\pgfpathlineto{\pgfqpoint{2.223059in}{7.195336in}}%
\pgfpathlineto{\pgfqpoint{2.224179in}{7.197335in}}%
\pgfpathlineto{\pgfqpoint{2.225300in}{7.169342in}}%
\pgfpathlineto{\pgfqpoint{2.226420in}{7.095359in}}%
\pgfpathlineto{\pgfqpoint{2.227541in}{7.066032in}}%
\pgfpathlineto{\pgfqpoint{2.232023in}{7.076697in}}%
\pgfpathlineto{\pgfqpoint{2.233143in}{7.053369in}}%
\pgfpathlineto{\pgfqpoint{2.234264in}{7.100691in}}%
\pgfpathlineto{\pgfqpoint{2.235384in}{7.068032in}}%
\pgfpathlineto{\pgfqpoint{2.239866in}{7.068032in}}%
\pgfpathlineto{\pgfqpoint{2.240987in}{7.040038in}}%
\pgfpathlineto{\pgfqpoint{2.242107in}{7.074030in}}%
\pgfpathlineto{\pgfqpoint{2.243228in}{7.053369in}}%
\pgfpathlineto{\pgfqpoint{2.246590in}{7.042038in}}%
\pgfpathlineto{\pgfqpoint{2.247710in}{7.057368in}}%
\pgfpathlineto{\pgfqpoint{2.248831in}{7.040038in}}%
\pgfpathlineto{\pgfqpoint{2.249951in}{7.052036in}}%
\pgfpathlineto{\pgfqpoint{2.251072in}{7.103357in}}%
\pgfpathlineto{\pgfqpoint{2.254433in}{7.076697in}}%
\pgfpathlineto{\pgfqpoint{2.255554in}{7.053369in}}%
\pgfpathlineto{\pgfqpoint{2.257795in}{7.107356in}}%
\pgfpathlineto{\pgfqpoint{2.258915in}{7.070698in}}%
\pgfpathlineto{\pgfqpoint{2.262277in}{7.049370in}}%
\pgfpathlineto{\pgfqpoint{2.263398in}{7.056035in}}%
\pgfpathlineto{\pgfqpoint{2.264518in}{7.057368in}}%
\pgfpathlineto{\pgfqpoint{2.265639in}{7.008046in}}%
\pgfpathlineto{\pgfqpoint{2.266759in}{7.054702in}}%
\pgfpathlineto{\pgfqpoint{2.271241in}{7.089360in}}%
\pgfpathlineto{\pgfqpoint{2.272362in}{7.116687in}}%
\pgfpathlineto{\pgfqpoint{2.273482in}{7.102024in}}%
\pgfpathlineto{\pgfqpoint{2.274603in}{7.098691in}}%
\pgfpathlineto{\pgfqpoint{2.277964in}{7.120686in}}%
\pgfpathlineto{\pgfqpoint{2.280205in}{7.094692in}}%
\pgfpathlineto{\pgfqpoint{2.281326in}{7.124019in}}%
\pgfpathlineto{\pgfqpoint{2.282446in}{7.134017in}}%
\pgfpathlineto{\pgfqpoint{2.285808in}{7.118687in}}%
\pgfpathlineto{\pgfqpoint{2.286929in}{7.161344in}}%
\pgfpathlineto{\pgfqpoint{2.288049in}{7.178673in}}%
\pgfpathlineto{\pgfqpoint{2.289170in}{7.181339in}}%
\pgfpathlineto{\pgfqpoint{2.290290in}{7.195336in}}%
\pgfpathlineto{\pgfqpoint{2.293652in}{7.185338in}}%
\pgfpathlineto{\pgfqpoint{2.294772in}{7.172008in}}%
\pgfpathlineto{\pgfqpoint{2.295893in}{7.171341in}}%
\pgfpathlineto{\pgfqpoint{2.297013in}{7.165343in}}%
\pgfpathlineto{\pgfqpoint{2.298134in}{7.188004in}}%
\pgfpathlineto{\pgfqpoint{2.301495in}{7.184005in}}%
\pgfpathlineto{\pgfqpoint{2.302616in}{7.184672in}}%
\pgfpathlineto{\pgfqpoint{2.303736in}{7.178673in}}%
\pgfpathlineto{\pgfqpoint{2.304857in}{7.223996in}}%
\pgfpathlineto{\pgfqpoint{2.305978in}{7.221996in}}%
\pgfpathlineto{\pgfqpoint{2.309339in}{7.231327in}}%
\pgfpathlineto{\pgfqpoint{2.310460in}{7.229994in}}%
\pgfpathlineto{\pgfqpoint{2.311580in}{7.230661in}}%
\pgfpathlineto{\pgfqpoint{2.312701in}{7.232661in}}%
\pgfpathlineto{\pgfqpoint{2.317183in}{7.254655in}}%
\pgfpathlineto{\pgfqpoint{2.318303in}{7.253989in}}%
\pgfpathlineto{\pgfqpoint{2.319424in}{7.274651in}}%
\pgfpathlineto{\pgfqpoint{2.320544in}{7.271985in}}%
\pgfpathlineto{\pgfqpoint{2.321665in}{7.279983in}}%
\pgfpathlineto{\pgfqpoint{2.326147in}{7.225329in}}%
\pgfpathlineto{\pgfqpoint{2.327268in}{7.220663in}}%
\pgfpathlineto{\pgfqpoint{2.328388in}{7.205333in}}%
\pgfpathlineto{\pgfqpoint{2.329509in}{7.213998in}}%
\pgfpathlineto{\pgfqpoint{2.332870in}{7.209333in}}%
\pgfpathlineto{\pgfqpoint{2.333991in}{7.215331in}}%
\pgfpathlineto{\pgfqpoint{2.335111in}{7.225329in}}%
\pgfpathlineto{\pgfqpoint{2.336232in}{7.227328in}}%
\pgfpathlineto{\pgfqpoint{2.340714in}{7.229994in}}%
\pgfpathlineto{\pgfqpoint{2.341834in}{7.235327in}}%
\pgfpathlineto{\pgfqpoint{2.342955in}{7.235327in}}%
\pgfpathlineto{\pgfqpoint{2.345196in}{7.212665in}}%
\pgfpathlineto{\pgfqpoint{2.348558in}{7.208000in}}%
\pgfpathlineto{\pgfqpoint{2.349678in}{7.220663in}}%
\pgfpathlineto{\pgfqpoint{2.350799in}{7.222663in}}%
\pgfpathlineto{\pgfqpoint{2.351919in}{7.220663in}}%
\pgfpathlineto{\pgfqpoint{2.353040in}{7.211999in}}%
\pgfpathlineto{\pgfqpoint{2.356401in}{7.219997in}}%
\pgfpathlineto{\pgfqpoint{2.357522in}{7.205333in}}%
\pgfpathlineto{\pgfqpoint{2.358642in}{7.172674in}}%
\pgfpathlineto{\pgfqpoint{2.359763in}{7.162010in}}%
\pgfpathlineto{\pgfqpoint{2.360883in}{7.175340in}}%
\pgfpathlineto{\pgfqpoint{2.364245in}{7.160677in}}%
\pgfpathlineto{\pgfqpoint{2.365365in}{7.196002in}}%
\pgfpathlineto{\pgfqpoint{2.366486in}{7.188004in}}%
\pgfpathlineto{\pgfqpoint{2.367607in}{7.174007in}}%
\pgfpathlineto{\pgfqpoint{2.368727in}{7.148013in}}%
\pgfpathlineto{\pgfqpoint{2.372089in}{7.166009in}}%
\pgfpathlineto{\pgfqpoint{2.373209in}{7.152012in}}%
\pgfpathlineto{\pgfqpoint{2.374330in}{7.146014in}}%
\pgfpathlineto{\pgfqpoint{2.375450in}{7.131351in}}%
\pgfpathlineto{\pgfqpoint{2.376571in}{7.143348in}}%
\pgfpathlineto{\pgfqpoint{2.379932in}{7.138682in}}%
\pgfpathlineto{\pgfqpoint{2.382173in}{7.174007in}}%
\pgfpathlineto{\pgfqpoint{2.383294in}{7.170008in}}%
\pgfpathlineto{\pgfqpoint{2.384414in}{7.175340in}}%
\pgfpathlineto{\pgfqpoint{2.388897in}{7.182006in}}%
\pgfpathlineto{\pgfqpoint{2.390017in}{7.174674in}}%
\pgfpathlineto{\pgfqpoint{2.391138in}{7.171341in}}%
\pgfpathlineto{\pgfqpoint{2.392258in}{7.165343in}}%
\pgfpathlineto{\pgfqpoint{2.395620in}{7.175340in}}%
\pgfpathlineto{\pgfqpoint{2.396740in}{7.176673in}}%
\pgfpathlineto{\pgfqpoint{2.397861in}{7.186671in}}%
\pgfpathlineto{\pgfqpoint{2.398981in}{7.182672in}}%
\pgfpathlineto{\pgfqpoint{2.400102in}{7.170675in}}%
\pgfpathlineto{\pgfqpoint{2.403463in}{7.158678in}}%
\pgfpathlineto{\pgfqpoint{2.404584in}{7.194003in}}%
\pgfpathlineto{\pgfqpoint{2.405704in}{7.202667in}}%
\pgfpathlineto{\pgfqpoint{2.406825in}{7.219330in}}%
\pgfpathlineto{\pgfqpoint{2.407946in}{7.216664in}}%
\pgfpathlineto{\pgfqpoint{2.412428in}{7.236660in}}%
\pgfpathlineto{\pgfqpoint{2.413548in}{7.227328in}}%
\pgfpathlineto{\pgfqpoint{2.414669in}{7.251323in}}%
\pgfpathlineto{\pgfqpoint{2.415789in}{7.171341in}}%
\pgfpathlineto{\pgfqpoint{2.419151in}{7.142015in}}%
\pgfpathlineto{\pgfqpoint{2.421392in}{7.213998in}}%
\pgfpathlineto{\pgfqpoint{2.422512in}{7.267986in}}%
\pgfpathlineto{\pgfqpoint{2.423633in}{7.268652in}}%
\pgfpathlineto{\pgfqpoint{2.428115in}{7.265986in}}%
\pgfpathlineto{\pgfqpoint{2.429236in}{7.283315in}}%
\pgfpathlineto{\pgfqpoint{2.430356in}{7.287981in}}%
\pgfpathlineto{\pgfqpoint{2.431477in}{7.309976in}}%
\pgfpathlineto{\pgfqpoint{2.434838in}{7.310642in}}%
\pgfpathlineto{\pgfqpoint{2.435959in}{7.313309in}}%
\pgfpathlineto{\pgfqpoint{2.437079in}{7.319307in}}%
\pgfpathlineto{\pgfqpoint{2.439320in}{7.349967in}}%
\pgfpathlineto{\pgfqpoint{2.442682in}{7.351300in}}%
\pgfpathlineto{\pgfqpoint{2.443802in}{7.352633in}}%
\pgfpathlineto{\pgfqpoint{2.446043in}{7.332637in}}%
\pgfpathlineto{\pgfqpoint{2.447164in}{7.301978in}}%
\pgfpathlineto{\pgfqpoint{2.451646in}{7.267319in}}%
\pgfpathlineto{\pgfqpoint{2.452767in}{7.256655in}}%
\pgfpathlineto{\pgfqpoint{2.453887in}{7.254655in}}%
\pgfpathlineto{\pgfqpoint{2.455008in}{7.247990in}}%
\pgfpathlineto{\pgfqpoint{2.458369in}{7.248657in}}%
\pgfpathlineto{\pgfqpoint{2.459490in}{7.242658in}}%
\pgfpathlineto{\pgfqpoint{2.462851in}{7.256655in}}%
\pgfpathlineto{\pgfqpoint{2.466213in}{7.255988in}}%
\pgfpathlineto{\pgfqpoint{2.467333in}{7.257321in}}%
\pgfpathlineto{\pgfqpoint{2.468454in}{7.255988in}}%
\pgfpathlineto{\pgfqpoint{2.469575in}{7.256655in}}%
\pgfpathlineto{\pgfqpoint{2.470695in}{7.253989in}}%
\pgfpathlineto{\pgfqpoint{2.474057in}{7.253989in}}%
\pgfpathlineto{\pgfqpoint{2.475177in}{7.251323in}}%
\pgfpathlineto{\pgfqpoint{2.476298in}{7.256655in}}%
\pgfpathlineto{\pgfqpoint{2.477418in}{7.265320in}}%
\pgfpathlineto{\pgfqpoint{2.478539in}{7.254655in}}%
\pgfpathlineto{\pgfqpoint{2.481900in}{7.258654in}}%
\pgfpathlineto{\pgfqpoint{2.483021in}{7.253322in}}%
\pgfpathlineto{\pgfqpoint{2.485262in}{7.251989in}}%
\pgfpathlineto{\pgfqpoint{2.486382in}{7.253322in}}%
\pgfpathlineto{\pgfqpoint{2.489744in}{7.261321in}}%
\pgfpathlineto{\pgfqpoint{2.490865in}{7.261321in}}%
\pgfpathlineto{\pgfqpoint{2.491985in}{7.253989in}}%
\pgfpathlineto{\pgfqpoint{2.493106in}{7.251989in}}%
\pgfpathlineto{\pgfqpoint{2.494226in}{7.256655in}}%
\pgfpathlineto{\pgfqpoint{2.498708in}{7.242658in}}%
\pgfpathlineto{\pgfqpoint{2.499829in}{7.243325in}}%
\pgfpathlineto{\pgfqpoint{2.500949in}{7.242658in}}%
\pgfpathlineto{\pgfqpoint{2.502070in}{7.188004in}}%
\pgfpathlineto{\pgfqpoint{2.505431in}{7.209999in}}%
\pgfpathlineto{\pgfqpoint{2.506552in}{7.172674in}}%
\pgfpathlineto{\pgfqpoint{2.507672in}{7.164010in}}%
\pgfpathlineto{\pgfqpoint{2.508793in}{7.180673in}}%
\pgfpathlineto{\pgfqpoint{2.509913in}{7.176673in}}%
\pgfpathlineto{\pgfqpoint{2.513275in}{7.162010in}}%
\pgfpathlineto{\pgfqpoint{2.516637in}{7.198002in}}%
\pgfpathlineto{\pgfqpoint{2.517757in}{7.188671in}}%
\pgfpathlineto{\pgfqpoint{2.521119in}{7.168009in}}%
\pgfpathlineto{\pgfqpoint{2.522239in}{7.188004in}}%
\pgfpathlineto{\pgfqpoint{2.523360in}{7.189337in}}%
\pgfpathlineto{\pgfqpoint{2.524480in}{7.168009in}}%
\pgfpathlineto{\pgfqpoint{2.525601in}{7.172674in}}%
\pgfpathlineto{\pgfqpoint{2.528962in}{7.174007in}}%
\pgfpathlineto{\pgfqpoint{2.530083in}{7.166009in}}%
\pgfpathlineto{\pgfqpoint{2.531204in}{7.166009in}}%
\pgfpathlineto{\pgfqpoint{2.533445in}{7.141348in}}%
\pgfpathlineto{\pgfqpoint{2.536806in}{7.128018in}}%
\pgfpathlineto{\pgfqpoint{2.537927in}{7.132017in}}%
\pgfpathlineto{\pgfqpoint{2.539047in}{7.130684in}}%
\pgfpathlineto{\pgfqpoint{2.540168in}{7.123352in}}%
\pgfpathlineto{\pgfqpoint{2.541288in}{7.130018in}}%
\pgfpathlineto{\pgfqpoint{2.544650in}{7.128018in}}%
\pgfpathlineto{\pgfqpoint{2.546891in}{7.140015in}}%
\pgfpathlineto{\pgfqpoint{2.548011in}{7.140682in}}%
\pgfpathlineto{\pgfqpoint{2.549132in}{7.135350in}}%
\pgfpathlineto{\pgfqpoint{2.552494in}{7.132017in}}%
\pgfpathlineto{\pgfqpoint{2.553614in}{7.116021in}}%
\pgfpathlineto{\pgfqpoint{2.554735in}{7.128685in}}%
\pgfpathlineto{\pgfqpoint{2.555855in}{7.114688in}}%
\pgfpathlineto{\pgfqpoint{2.556976in}{7.149346in}}%
\pgfpathlineto{\pgfqpoint{2.560337in}{7.142681in}}%
\pgfpathlineto{\pgfqpoint{2.561458in}{7.129351in}}%
\pgfpathlineto{\pgfqpoint{2.563699in}{7.094026in}}%
\pgfpathlineto{\pgfqpoint{2.564819in}{7.103357in}}%
\pgfpathlineto{\pgfqpoint{2.568181in}{7.154679in}}%
\pgfpathlineto{\pgfqpoint{2.569301in}{7.161344in}}%
\pgfpathlineto{\pgfqpoint{2.570422in}{7.173341in}}%
\pgfpathlineto{\pgfqpoint{2.571542in}{7.219330in}}%
\pgfpathlineto{\pgfqpoint{2.572663in}{7.237326in}}%
\pgfpathlineto{\pgfqpoint{2.576025in}{7.225329in}}%
\pgfpathlineto{\pgfqpoint{2.577145in}{7.239326in}}%
\pgfpathlineto{\pgfqpoint{2.578266in}{7.238659in}}%
\pgfpathlineto{\pgfqpoint{2.579386in}{7.241992in}}%
\pgfpathlineto{\pgfqpoint{2.580507in}{7.234660in}}%
\pgfpathlineto{\pgfqpoint{2.583868in}{7.246657in}}%
\pgfpathlineto{\pgfqpoint{2.586109in}{7.273984in}}%
\pgfpathlineto{\pgfqpoint{2.588350in}{7.279983in}}%
\pgfpathlineto{\pgfqpoint{2.591712in}{7.268652in}}%
\pgfpathlineto{\pgfqpoint{2.592832in}{7.257321in}}%
\pgfpathlineto{\pgfqpoint{2.593953in}{7.239992in}}%
\pgfpathlineto{\pgfqpoint{2.595074in}{7.277317in}}%
\pgfpathlineto{\pgfqpoint{2.596194in}{7.273984in}}%
\pgfpathlineto{\pgfqpoint{2.599556in}{7.260654in}}%
\pgfpathlineto{\pgfqpoint{2.600676in}{7.263987in}}%
\pgfpathlineto{\pgfqpoint{2.601797in}{7.289314in}}%
\pgfpathlineto{\pgfqpoint{2.602917in}{7.285315in}}%
\pgfpathlineto{\pgfqpoint{2.604038in}{7.299978in}}%
\pgfpathlineto{\pgfqpoint{2.607399in}{7.304644in}}%
\pgfpathlineto{\pgfqpoint{2.608520in}{7.297312in}}%
\pgfpathlineto{\pgfqpoint{2.610761in}{7.269319in}}%
\pgfpathlineto{\pgfqpoint{2.611881in}{7.297979in}}%
\pgfpathlineto{\pgfqpoint{2.615243in}{7.307976in}}%
\pgfpathlineto{\pgfqpoint{2.616364in}{7.327305in}}%
\pgfpathlineto{\pgfqpoint{2.618605in}{7.316641in}}%
\pgfpathlineto{\pgfqpoint{2.619725in}{7.319974in}}%
\pgfpathlineto{\pgfqpoint{2.624207in}{7.321307in}}%
\pgfpathlineto{\pgfqpoint{2.625328in}{7.309309in}}%
\pgfpathlineto{\pgfqpoint{2.626448in}{7.309976in}}%
\pgfpathlineto{\pgfqpoint{2.627569in}{7.303311in}}%
\pgfpathlineto{\pgfqpoint{2.633171in}{7.309309in}}%
\pgfpathlineto{\pgfqpoint{2.634292in}{7.298645in}}%
\pgfpathlineto{\pgfqpoint{2.635413in}{7.303977in}}%
\pgfpathlineto{\pgfqpoint{2.638774in}{7.295313in}}%
\pgfpathlineto{\pgfqpoint{2.639895in}{7.289981in}}%
\pgfpathlineto{\pgfqpoint{2.641015in}{7.295979in}}%
\pgfpathlineto{\pgfqpoint{2.642136in}{7.290647in}}%
\pgfpathlineto{\pgfqpoint{2.647738in}{7.283982in}}%
\pgfpathlineto{\pgfqpoint{2.649979in}{7.279983in}}%
\pgfpathlineto{\pgfqpoint{2.651100in}{7.239992in}}%
\pgfpathlineto{\pgfqpoint{2.654461in}{7.194003in}}%
\pgfpathlineto{\pgfqpoint{2.655582in}{7.208666in}}%
\pgfpathlineto{\pgfqpoint{2.656703in}{7.230661in}}%
\pgfpathlineto{\pgfqpoint{2.657823in}{7.227328in}}%
\pgfpathlineto{\pgfqpoint{2.658944in}{7.209333in}}%
\pgfpathlineto{\pgfqpoint{2.662305in}{7.206000in}}%
\pgfpathlineto{\pgfqpoint{2.663426in}{7.190670in}}%
\pgfpathlineto{\pgfqpoint{2.665667in}{7.189337in}}%
\pgfpathlineto{\pgfqpoint{2.666787in}{7.190670in}}%
\pgfpathlineto{\pgfqpoint{2.670149in}{7.188671in}}%
\pgfpathlineto{\pgfqpoint{2.672390in}{7.174674in}}%
\pgfpathlineto{\pgfqpoint{2.674631in}{7.192003in}}%
\pgfpathlineto{\pgfqpoint{2.677993in}{7.210666in}}%
\pgfpathlineto{\pgfqpoint{2.679113in}{7.225329in}}%
\pgfpathlineto{\pgfqpoint{2.681354in}{7.235327in}}%
\pgfpathlineto{\pgfqpoint{2.682475in}{7.230661in}}%
\pgfpathlineto{\pgfqpoint{2.686957in}{7.239326in}}%
\pgfpathlineto{\pgfqpoint{2.688077in}{7.227995in}}%
\pgfpathlineto{\pgfqpoint{2.689198in}{7.223996in}}%
\pgfpathlineto{\pgfqpoint{2.690318in}{7.233994in}}%
\pgfpathlineto{\pgfqpoint{2.693680in}{7.219330in}}%
\pgfpathlineto{\pgfqpoint{2.694800in}{7.211332in}}%
\pgfpathlineto{\pgfqpoint{2.695921in}{7.233994in}}%
\pgfpathlineto{\pgfqpoint{2.697042in}{7.233994in}}%
\pgfpathlineto{\pgfqpoint{2.698162in}{7.229994in}}%
\pgfpathlineto{\pgfqpoint{2.701524in}{7.222663in}}%
\pgfpathlineto{\pgfqpoint{2.704885in}{7.202667in}}%
\pgfpathlineto{\pgfqpoint{2.706006in}{7.239326in}}%
\pgfpathlineto{\pgfqpoint{2.709367in}{7.214665in}}%
\pgfpathlineto{\pgfqpoint{2.710488in}{7.195336in}}%
\pgfpathlineto{\pgfqpoint{2.711608in}{7.208666in}}%
\pgfpathlineto{\pgfqpoint{2.712729in}{7.208000in}}%
\pgfpathlineto{\pgfqpoint{2.713849in}{7.215331in}}%
\pgfpathlineto{\pgfqpoint{2.717211in}{7.207333in}}%
\pgfpathlineto{\pgfqpoint{2.718332in}{7.186671in}}%
\pgfpathlineto{\pgfqpoint{2.720573in}{7.200001in}}%
\pgfpathlineto{\pgfqpoint{2.721693in}{7.206000in}}%
\pgfpathlineto{\pgfqpoint{2.725055in}{7.189337in}}%
\pgfpathlineto{\pgfqpoint{2.726175in}{7.200001in}}%
\pgfpathlineto{\pgfqpoint{2.727296in}{7.204000in}}%
\pgfpathlineto{\pgfqpoint{2.728416in}{7.215331in}}%
\pgfpathlineto{\pgfqpoint{2.729537in}{7.210666in}}%
\pgfpathlineto{\pgfqpoint{2.732898in}{7.215331in}}%
\pgfpathlineto{\pgfqpoint{2.734019in}{7.223996in}}%
\pgfpathlineto{\pgfqpoint{2.736260in}{7.218664in}}%
\pgfpathlineto{\pgfqpoint{2.737381in}{7.221996in}}%
\pgfpathlineto{\pgfqpoint{2.740742in}{7.223329in}}%
\pgfpathlineto{\pgfqpoint{2.741863in}{7.225329in}}%
\pgfpathlineto{\pgfqpoint{2.744104in}{7.196669in}}%
\pgfpathlineto{\pgfqpoint{2.748586in}{7.201334in}}%
\pgfpathlineto{\pgfqpoint{2.751947in}{7.238659in}}%
\pgfpathlineto{\pgfqpoint{2.753068in}{7.196002in}}%
\pgfpathlineto{\pgfqpoint{2.756429in}{7.196002in}}%
\pgfpathlineto{\pgfqpoint{2.757550in}{7.190004in}}%
\pgfpathlineto{\pgfqpoint{2.759791in}{7.168009in}}%
\pgfpathlineto{\pgfqpoint{2.760912in}{7.162677in}}%
\pgfpathlineto{\pgfqpoint{2.764273in}{7.159344in}}%
\pgfpathlineto{\pgfqpoint{2.765394in}{7.162677in}}%
\pgfpathlineto{\pgfqpoint{2.766514in}{7.176673in}}%
\pgfpathlineto{\pgfqpoint{2.767635in}{7.175340in}}%
\pgfpathlineto{\pgfqpoint{2.768755in}{7.176007in}}%
\pgfpathlineto{\pgfqpoint{2.772117in}{7.167342in}}%
\pgfpathlineto{\pgfqpoint{2.773237in}{7.158678in}}%
\pgfpathlineto{\pgfqpoint{2.774358in}{7.145347in}}%
\pgfpathlineto{\pgfqpoint{2.775478in}{7.155345in}}%
\pgfpathlineto{\pgfqpoint{2.776599in}{7.119353in}}%
\pgfpathlineto{\pgfqpoint{2.779961in}{7.114021in}}%
\pgfpathlineto{\pgfqpoint{2.781081in}{7.106023in}}%
\pgfpathlineto{\pgfqpoint{2.782202in}{7.068032in}}%
\pgfpathlineto{\pgfqpoint{2.783322in}{7.072031in}}%
\pgfpathlineto{\pgfqpoint{2.784443in}{7.106690in}}%
\pgfpathlineto{\pgfqpoint{2.787804in}{7.114021in}}%
\pgfpathlineto{\pgfqpoint{2.788925in}{7.120020in}}%
\pgfpathlineto{\pgfqpoint{2.791166in}{7.072697in}}%
\pgfpathlineto{\pgfqpoint{2.792286in}{7.070698in}}%
\pgfpathlineto{\pgfqpoint{2.796768in}{7.065366in}}%
\pgfpathlineto{\pgfqpoint{2.797889in}{7.066699in}}%
\pgfpathlineto{\pgfqpoint{2.799009in}{7.086694in}}%
\pgfpathlineto{\pgfqpoint{2.800130in}{7.096025in}}%
\pgfpathlineto{\pgfqpoint{2.803492in}{7.102024in}}%
\pgfpathlineto{\pgfqpoint{2.804612in}{7.099358in}}%
\pgfpathlineto{\pgfqpoint{2.805733in}{7.084028in}}%
\pgfpathlineto{\pgfqpoint{2.806853in}{7.078696in}}%
\pgfpathlineto{\pgfqpoint{2.811335in}{7.159344in}}%
\pgfpathlineto{\pgfqpoint{2.812456in}{7.130018in}}%
\pgfpathlineto{\pgfqpoint{2.813576in}{7.144681in}}%
\pgfpathlineto{\pgfqpoint{2.815817in}{7.192670in}}%
\pgfpathlineto{\pgfqpoint{2.819179in}{7.180673in}}%
\pgfpathlineto{\pgfqpoint{2.821420in}{7.118687in}}%
\pgfpathlineto{\pgfqpoint{2.822541in}{7.104690in}}%
\pgfpathlineto{\pgfqpoint{2.823661in}{7.106023in}}%
\pgfpathlineto{\pgfqpoint{2.827023in}{7.108689in}}%
\pgfpathlineto{\pgfqpoint{2.828143in}{7.084028in}}%
\pgfpathlineto{\pgfqpoint{2.829264in}{7.076030in}}%
\pgfpathlineto{\pgfqpoint{2.830384in}{7.072697in}}%
\pgfpathlineto{\pgfqpoint{2.831505in}{7.072031in}}%
\pgfpathlineto{\pgfqpoint{2.834866in}{7.098691in}}%
\pgfpathlineto{\pgfqpoint{2.837107in}{7.092693in}}%
\pgfpathlineto{\pgfqpoint{2.838228in}{7.029374in}}%
\pgfpathlineto{\pgfqpoint{2.839348in}{7.020043in}}%
\pgfpathlineto{\pgfqpoint{2.842710in}{7.013378in}}%
\pgfpathlineto{\pgfqpoint{2.844951in}{7.046037in}}%
\pgfpathlineto{\pgfqpoint{2.846072in}{7.058701in}}%
\pgfpathlineto{\pgfqpoint{2.847192in}{7.058034in}}%
\pgfpathlineto{\pgfqpoint{2.850554in}{7.060700in}}%
\pgfpathlineto{\pgfqpoint{2.852795in}{7.068032in}}%
\pgfpathlineto{\pgfqpoint{2.853915in}{7.052702in}}%
\pgfpathlineto{\pgfqpoint{2.855036in}{7.005380in}}%
\pgfpathlineto{\pgfqpoint{2.858397in}{6.976053in}}%
\pgfpathlineto{\pgfqpoint{2.859518in}{6.976720in}}%
\pgfpathlineto{\pgfqpoint{2.861759in}{6.998048in}}%
\pgfpathlineto{\pgfqpoint{2.862880in}{6.982052in}}%
\pgfpathlineto{\pgfqpoint{2.866241in}{6.986717in}}%
\pgfpathlineto{\pgfqpoint{2.867362in}{6.976720in}}%
\pgfpathlineto{\pgfqpoint{2.868482in}{6.981385in}}%
\pgfpathlineto{\pgfqpoint{2.869603in}{6.996049in}}%
\pgfpathlineto{\pgfqpoint{2.870723in}{6.997382in}}%
\pgfpathlineto{\pgfqpoint{2.875205in}{6.984051in}}%
\pgfpathlineto{\pgfqpoint{2.876326in}{6.993382in}}%
\pgfpathlineto{\pgfqpoint{2.877446in}{6.968055in}}%
\pgfpathlineto{\pgfqpoint{2.878567in}{6.962056in}}%
\pgfpathlineto{\pgfqpoint{2.881929in}{6.972054in}}%
\pgfpathlineto{\pgfqpoint{2.883049in}{6.958724in}}%
\pgfpathlineto{\pgfqpoint{2.884170in}{6.956058in}}%
\pgfpathlineto{\pgfqpoint{2.886411in}{6.922732in}}%
\pgfpathlineto{\pgfqpoint{2.889772in}{6.918733in}}%
\pgfpathlineto{\pgfqpoint{2.890893in}{6.925398in}}%
\pgfpathlineto{\pgfqpoint{2.892013in}{6.912734in}}%
\pgfpathlineto{\pgfqpoint{2.893134in}{6.907402in}}%
\pgfpathlineto{\pgfqpoint{2.894254in}{6.918733in}}%
\pgfpathlineto{\pgfqpoint{2.897616in}{6.918067in}}%
\pgfpathlineto{\pgfqpoint{2.898736in}{6.916067in}}%
\pgfpathlineto{\pgfqpoint{2.899857in}{6.906069in}}%
\pgfpathlineto{\pgfqpoint{2.900977in}{6.922732in}}%
\pgfpathlineto{\pgfqpoint{2.902098in}{6.958724in}}%
\pgfpathlineto{\pgfqpoint{2.906580in}{6.934729in}}%
\pgfpathlineto{\pgfqpoint{2.907701in}{6.945394in}}%
\pgfpathlineto{\pgfqpoint{2.908821in}{6.890740in}}%
\pgfpathlineto{\pgfqpoint{2.909942in}{6.878076in}}%
\pgfpathlineto{\pgfqpoint{2.913303in}{6.872077in}}%
\pgfpathlineto{\pgfqpoint{2.916665in}{6.904736in}}%
\pgfpathlineto{\pgfqpoint{2.917785in}{6.899404in}}%
\pgfpathlineto{\pgfqpoint{2.921147in}{6.932063in}}%
\pgfpathlineto{\pgfqpoint{2.922267in}{6.916067in}}%
\pgfpathlineto{\pgfqpoint{2.923388in}{6.923399in}}%
\pgfpathlineto{\pgfqpoint{2.924509in}{6.950059in}}%
\pgfpathlineto{\pgfqpoint{2.925629in}{6.957391in}}%
\pgfpathlineto{\pgfqpoint{2.928991in}{6.972054in}}%
\pgfpathlineto{\pgfqpoint{2.930111in}{6.960723in}}%
\pgfpathlineto{\pgfqpoint{2.931232in}{6.926731in}}%
\pgfpathlineto{\pgfqpoint{2.933473in}{6.914734in}}%
\pgfpathlineto{\pgfqpoint{2.936834in}{6.938728in}}%
\pgfpathlineto{\pgfqpoint{2.937955in}{6.952725in}}%
\pgfpathlineto{\pgfqpoint{2.939075in}{6.933396in}}%
\pgfpathlineto{\pgfqpoint{2.940196in}{6.936729in}}%
\pgfpathlineto{\pgfqpoint{2.941316in}{6.927398in}}%
\pgfpathlineto{\pgfqpoint{2.944678in}{6.868745in}}%
\pgfpathlineto{\pgfqpoint{2.945799in}{6.864746in}}%
\pgfpathlineto{\pgfqpoint{2.946919in}{6.846750in}}%
\pgfpathlineto{\pgfqpoint{2.948040in}{6.845417in}}%
\pgfpathlineto{\pgfqpoint{2.949160in}{6.841418in}}%
\pgfpathlineto{\pgfqpoint{2.952522in}{6.864746in}}%
\pgfpathlineto{\pgfqpoint{2.953642in}{6.854081in}}%
\pgfpathlineto{\pgfqpoint{2.954763in}{6.850082in}}%
\pgfpathlineto{\pgfqpoint{2.957004in}{6.893406in}}%
\pgfpathlineto{\pgfqpoint{2.962606in}{6.750772in}}%
\pgfpathlineto{\pgfqpoint{2.963727in}{6.739441in}}%
\pgfpathlineto{\pgfqpoint{2.964848in}{6.706782in}}%
\pgfpathlineto{\pgfqpoint{2.968209in}{6.683454in}}%
\pgfpathlineto{\pgfqpoint{2.970450in}{6.659460in}}%
\pgfpathlineto{\pgfqpoint{2.971571in}{6.654794in}}%
\pgfpathlineto{\pgfqpoint{2.972691in}{6.666791in}}%
\pgfpathlineto{\pgfqpoint{2.976053in}{6.666125in}}%
\pgfpathlineto{\pgfqpoint{2.977173in}{6.671457in}}%
\pgfpathlineto{\pgfqpoint{2.979414in}{6.658127in}}%
\pgfpathlineto{\pgfqpoint{2.980535in}{6.688786in}}%
\pgfpathlineto{\pgfqpoint{2.983896in}{6.598140in}}%
\pgfpathlineto{\pgfqpoint{2.985017in}{6.529490in}}%
\pgfpathlineto{\pgfqpoint{2.986138in}{6.551485in}}%
\pgfpathlineto{\pgfqpoint{2.987258in}{6.550818in}}%
\pgfpathlineto{\pgfqpoint{2.988379in}{6.548819in}}%
\pgfpathlineto{\pgfqpoint{2.991740in}{6.534822in}}%
\pgfpathlineto{\pgfqpoint{2.992861in}{6.525491in}}%
\pgfpathlineto{\pgfqpoint{2.993981in}{6.544819in}}%
\pgfpathlineto{\pgfqpoint{2.996222in}{6.547485in}}%
\pgfpathlineto{\pgfqpoint{2.999584in}{6.542820in}}%
\pgfpathlineto{\pgfqpoint{3.000704in}{6.560816in}}%
\pgfpathlineto{\pgfqpoint{3.001825in}{6.565481in}}%
\pgfpathlineto{\pgfqpoint{3.002945in}{6.553484in}}%
\pgfpathlineto{\pgfqpoint{3.004066in}{6.528157in}}%
\pgfpathlineto{\pgfqpoint{3.007428in}{6.532822in}}%
\pgfpathlineto{\pgfqpoint{3.008548in}{6.520825in}}%
\pgfpathlineto{\pgfqpoint{3.010789in}{6.525491in}}%
\pgfpathlineto{\pgfqpoint{3.011910in}{6.525491in}}%
\pgfpathlineto{\pgfqpoint{3.015271in}{6.522158in}}%
\pgfpathlineto{\pgfqpoint{3.016392in}{6.538154in}}%
\pgfpathlineto{\pgfqpoint{3.018633in}{6.521491in}}%
\pgfpathlineto{\pgfqpoint{3.019753in}{6.532156in}}%
\pgfpathlineto{\pgfqpoint{3.023115in}{6.528823in}}%
\pgfpathlineto{\pgfqpoint{3.025356in}{6.509494in}}%
\pgfpathlineto{\pgfqpoint{3.027597in}{6.512827in}}%
\pgfpathlineto{\pgfqpoint{3.032079in}{6.515493in}}%
\pgfpathlineto{\pgfqpoint{3.034320in}{6.511494in}}%
\pgfpathlineto{\pgfqpoint{3.035441in}{6.516826in}}%
\pgfpathlineto{\pgfqpoint{3.035441in}{6.516826in}}%
\pgfusepath{stroke}%
\end{pgfscope}%
\begin{pgfscope}%
\pgfpathrectangle{\pgfqpoint{0.462318in}{6.297976in}}{\pgfqpoint{2.695652in}{1.104878in}}%
\pgfusepath{clip}%
\pgfsetroundcap%
\pgfsetroundjoin%
\pgfsetlinewidth{1.505625pt}%
\definecolor{currentstroke}{rgb}{1.000000,0.647059,0.000000}%
\pgfsetstrokecolor{currentstroke}%
\pgfsetdash{}{0pt}%
\pgfpathmoveto{\pgfqpoint{0.584848in}{6.350864in}}%
\pgfpathlineto{\pgfqpoint{0.585968in}{6.355863in}}%
\pgfpathlineto{\pgfqpoint{0.596053in}{6.368194in}}%
\pgfpathlineto{\pgfqpoint{0.600535in}{6.368394in}}%
\pgfpathlineto{\pgfqpoint{0.603897in}{6.372962in}}%
\pgfpathlineto{\pgfqpoint{0.608379in}{6.373570in}}%
\pgfpathlineto{\pgfqpoint{0.610620in}{6.375251in}}%
\pgfpathlineto{\pgfqpoint{0.611740in}{6.375747in}}%
\pgfpathlineto{\pgfqpoint{0.616223in}{6.375492in}}%
\pgfpathlineto{\pgfqpoint{0.619584in}{6.375467in}}%
\pgfpathlineto{\pgfqpoint{0.624066in}{6.376538in}}%
\pgfpathlineto{\pgfqpoint{0.626307in}{6.377698in}}%
\pgfpathlineto{\pgfqpoint{0.634151in}{6.377983in}}%
\pgfpathlineto{\pgfqpoint{0.635272in}{6.378555in}}%
\pgfpathlineto{\pgfqpoint{0.640874in}{6.379943in}}%
\pgfpathlineto{\pgfqpoint{0.643115in}{6.381308in}}%
\pgfpathlineto{\pgfqpoint{0.648718in}{6.382424in}}%
\pgfpathlineto{\pgfqpoint{0.650959in}{6.382968in}}%
\pgfpathlineto{\pgfqpoint{0.662164in}{6.383218in}}%
\pgfpathlineto{\pgfqpoint{0.665526in}{6.386229in}}%
\pgfpathlineto{\pgfqpoint{0.666646in}{6.387497in}}%
\pgfpathlineto{\pgfqpoint{0.670008in}{6.388730in}}%
\pgfpathlineto{\pgfqpoint{0.674490in}{6.392258in}}%
\pgfpathlineto{\pgfqpoint{0.678972in}{6.394018in}}%
\pgfpathlineto{\pgfqpoint{0.682334in}{6.396359in}}%
\pgfpathlineto{\pgfqpoint{0.686816in}{6.397770in}}%
\pgfpathlineto{\pgfqpoint{0.689057in}{6.398520in}}%
\pgfpathlineto{\pgfqpoint{0.698021in}{6.398121in}}%
\pgfpathlineto{\pgfqpoint{0.705865in}{6.398230in}}%
\pgfpathlineto{\pgfqpoint{0.711467in}{6.398651in}}%
\pgfpathlineto{\pgfqpoint{0.720432in}{6.400610in}}%
\pgfpathlineto{\pgfqpoint{0.727155in}{6.400665in}}%
\pgfpathlineto{\pgfqpoint{0.732757in}{6.400201in}}%
\pgfpathlineto{\pgfqpoint{0.734998in}{6.399690in}}%
\pgfpathlineto{\pgfqpoint{0.740601in}{6.399410in}}%
\pgfpathlineto{\pgfqpoint{0.752927in}{6.399196in}}%
\pgfpathlineto{\pgfqpoint{0.764132in}{6.398127in}}%
\pgfpathlineto{\pgfqpoint{0.773096in}{6.399719in}}%
\pgfpathlineto{\pgfqpoint{0.776458in}{6.400747in}}%
\pgfpathlineto{\pgfqpoint{0.780940in}{6.401323in}}%
\pgfpathlineto{\pgfqpoint{0.788784in}{6.404211in}}%
\pgfpathlineto{\pgfqpoint{0.799989in}{6.406354in}}%
\pgfpathlineto{\pgfqpoint{0.805592in}{6.407065in}}%
\pgfpathlineto{\pgfqpoint{0.820159in}{6.410882in}}%
\pgfpathlineto{\pgfqpoint{0.823520in}{6.412497in}}%
\pgfpathlineto{\pgfqpoint{0.828002in}{6.413746in}}%
\pgfpathlineto{\pgfqpoint{0.831364in}{6.415568in}}%
\pgfpathlineto{\pgfqpoint{0.835846in}{6.416687in}}%
\pgfpathlineto{\pgfqpoint{0.839208in}{6.418341in}}%
\pgfpathlineto{\pgfqpoint{0.843690in}{6.419341in}}%
\pgfpathlineto{\pgfqpoint{0.847051in}{6.420657in}}%
\pgfpathlineto{\pgfqpoint{0.851533in}{6.421563in}}%
\pgfpathlineto{\pgfqpoint{0.854895in}{6.422785in}}%
\pgfpathlineto{\pgfqpoint{0.860498in}{6.423491in}}%
\pgfpathlineto{\pgfqpoint{0.862739in}{6.424693in}}%
\pgfpathlineto{\pgfqpoint{0.867221in}{6.425917in}}%
\pgfpathlineto{\pgfqpoint{0.870582in}{6.428109in}}%
\pgfpathlineto{\pgfqpoint{0.875064in}{6.429605in}}%
\pgfpathlineto{\pgfqpoint{0.878426in}{6.432153in}}%
\pgfpathlineto{\pgfqpoint{0.882908in}{6.433768in}}%
\pgfpathlineto{\pgfqpoint{0.886270in}{6.436270in}}%
\pgfpathlineto{\pgfqpoint{0.889631in}{6.437165in}}%
\pgfpathlineto{\pgfqpoint{0.894113in}{6.440799in}}%
\pgfpathlineto{\pgfqpoint{0.898595in}{6.442470in}}%
\pgfpathlineto{\pgfqpoint{0.901957in}{6.444681in}}%
\pgfpathlineto{\pgfqpoint{0.906439in}{6.446202in}}%
\pgfpathlineto{\pgfqpoint{0.909801in}{6.448385in}}%
\pgfpathlineto{\pgfqpoint{0.915403in}{6.449631in}}%
\pgfpathlineto{\pgfqpoint{0.917644in}{6.450328in}}%
\pgfpathlineto{\pgfqpoint{0.924368in}{6.451019in}}%
\pgfpathlineto{\pgfqpoint{0.925488in}{6.451388in}}%
\pgfpathlineto{\pgfqpoint{0.931091in}{6.452517in}}%
\pgfpathlineto{\pgfqpoint{0.937814in}{6.453492in}}%
\pgfpathlineto{\pgfqpoint{0.946778in}{6.454195in}}%
\pgfpathlineto{\pgfqpoint{0.962465in}{6.456517in}}%
\pgfpathlineto{\pgfqpoint{0.964707in}{6.457174in}}%
\pgfpathlineto{\pgfqpoint{0.970309in}{6.458237in}}%
\pgfpathlineto{\pgfqpoint{0.972550in}{6.458959in}}%
\pgfpathlineto{\pgfqpoint{1.002804in}{6.463259in}}%
\pgfpathlineto{\pgfqpoint{1.011769in}{6.464990in}}%
\pgfpathlineto{\pgfqpoint{1.017371in}{6.465806in}}%
\pgfpathlineto{\pgfqpoint{1.019612in}{6.466685in}}%
\pgfpathlineto{\pgfqpoint{1.024094in}{6.467677in}}%
\pgfpathlineto{\pgfqpoint{1.027456in}{6.469070in}}%
\pgfpathlineto{\pgfqpoint{1.031938in}{6.470002in}}%
\pgfpathlineto{\pgfqpoint{1.035300in}{6.471400in}}%
\pgfpathlineto{\pgfqpoint{1.039782in}{6.472332in}}%
\pgfpathlineto{\pgfqpoint{1.043143in}{6.474171in}}%
\pgfpathlineto{\pgfqpoint{1.048746in}{6.475455in}}%
\pgfpathlineto{\pgfqpoint{1.050987in}{6.476698in}}%
\pgfpathlineto{\pgfqpoint{1.055469in}{6.477779in}}%
\pgfpathlineto{\pgfqpoint{1.058831in}{6.479552in}}%
\pgfpathlineto{\pgfqpoint{1.063313in}{6.480773in}}%
\pgfpathlineto{\pgfqpoint{1.066675in}{6.482723in}}%
\pgfpathlineto{\pgfqpoint{1.071157in}{6.483934in}}%
\pgfpathlineto{\pgfqpoint{1.074518in}{6.485735in}}%
\pgfpathlineto{\pgfqpoint{1.079000in}{6.486826in}}%
\pgfpathlineto{\pgfqpoint{1.082362in}{6.488482in}}%
\pgfpathlineto{\pgfqpoint{1.086844in}{6.489501in}}%
\pgfpathlineto{\pgfqpoint{1.089085in}{6.490484in}}%
\pgfpathlineto{\pgfqpoint{1.094688in}{6.491488in}}%
\pgfpathlineto{\pgfqpoint{1.098049in}{6.492867in}}%
\pgfpathlineto{\pgfqpoint{1.102531in}{6.493801in}}%
\pgfpathlineto{\pgfqpoint{1.105893in}{6.495402in}}%
\pgfpathlineto{\pgfqpoint{1.110375in}{6.496261in}}%
\pgfpathlineto{\pgfqpoint{1.113737in}{6.497259in}}%
\pgfpathlineto{\pgfqpoint{1.120460in}{6.498113in}}%
\pgfpathlineto{\pgfqpoint{1.121580in}{6.498406in}}%
\pgfpathlineto{\pgfqpoint{1.127183in}{6.499286in}}%
\pgfpathlineto{\pgfqpoint{1.137268in}{6.501773in}}%
\pgfpathlineto{\pgfqpoint{1.141750in}{6.502537in}}%
\pgfpathlineto{\pgfqpoint{1.145111in}{6.503845in}}%
\pgfpathlineto{\pgfqpoint{1.149594in}{6.504794in}}%
\pgfpathlineto{\pgfqpoint{1.152955in}{6.506228in}}%
\pgfpathlineto{\pgfqpoint{1.158558in}{6.507150in}}%
\pgfpathlineto{\pgfqpoint{1.160799in}{6.508014in}}%
\pgfpathlineto{\pgfqpoint{1.166401in}{6.509328in}}%
\pgfpathlineto{\pgfqpoint{1.168642in}{6.510216in}}%
\pgfpathlineto{\pgfqpoint{1.174245in}{6.511529in}}%
\pgfpathlineto{\pgfqpoint{1.176486in}{6.512386in}}%
\pgfpathlineto{\pgfqpoint{1.180968in}{6.513364in}}%
\pgfpathlineto{\pgfqpoint{1.184330in}{6.514640in}}%
\pgfpathlineto{\pgfqpoint{1.189933in}{6.515738in}}%
\pgfpathlineto{\pgfqpoint{1.192174in}{6.516507in}}%
\pgfpathlineto{\pgfqpoint{1.203379in}{6.518310in}}%
\pgfpathlineto{\pgfqpoint{1.207861in}{6.520037in}}%
\pgfpathlineto{\pgfqpoint{1.213464in}{6.521232in}}%
\pgfpathlineto{\pgfqpoint{1.223548in}{6.524912in}}%
\pgfpathlineto{\pgfqpoint{1.228030in}{6.525921in}}%
\pgfpathlineto{\pgfqpoint{1.231392in}{6.527447in}}%
\pgfpathlineto{\pgfqpoint{1.235874in}{6.528412in}}%
\pgfpathlineto{\pgfqpoint{1.239236in}{6.529799in}}%
\pgfpathlineto{\pgfqpoint{1.243718in}{6.530694in}}%
\pgfpathlineto{\pgfqpoint{1.247079in}{6.531932in}}%
\pgfpathlineto{\pgfqpoint{1.252682in}{6.533043in}}%
\pgfpathlineto{\pgfqpoint{1.254923in}{6.533787in}}%
\pgfpathlineto{\pgfqpoint{1.261646in}{6.535001in}}%
\pgfpathlineto{\pgfqpoint{1.262767in}{6.535284in}}%
\pgfpathlineto{\pgfqpoint{1.269490in}{6.536120in}}%
\pgfpathlineto{\pgfqpoint{1.278454in}{6.538171in}}%
\pgfpathlineto{\pgfqpoint{1.282936in}{6.538992in}}%
\pgfpathlineto{\pgfqpoint{1.286298in}{6.540308in}}%
\pgfpathlineto{\pgfqpoint{1.291900in}{6.541570in}}%
\pgfpathlineto{\pgfqpoint{1.294142in}{6.542368in}}%
\pgfpathlineto{\pgfqpoint{1.298624in}{6.543133in}}%
\pgfpathlineto{\pgfqpoint{1.301985in}{6.544311in}}%
\pgfpathlineto{\pgfqpoint{1.307588in}{6.545320in}}%
\pgfpathlineto{\pgfqpoint{1.309829in}{6.546126in}}%
\pgfpathlineto{\pgfqpoint{1.315432in}{6.547317in}}%
\pgfpathlineto{\pgfqpoint{1.317673in}{6.548292in}}%
\pgfpathlineto{\pgfqpoint{1.322155in}{6.549485in}}%
\pgfpathlineto{\pgfqpoint{1.325516in}{6.551170in}}%
\pgfpathlineto{\pgfqpoint{1.329998in}{6.552353in}}%
\pgfpathlineto{\pgfqpoint{1.333360in}{6.554177in}}%
\pgfpathlineto{\pgfqpoint{1.337842in}{6.555396in}}%
\pgfpathlineto{\pgfqpoint{1.341204in}{6.557353in}}%
\pgfpathlineto{\pgfqpoint{1.345686in}{6.558685in}}%
\pgfpathlineto{\pgfqpoint{1.349047in}{6.560690in}}%
\pgfpathlineto{\pgfqpoint{1.353529in}{6.562014in}}%
\pgfpathlineto{\pgfqpoint{1.356891in}{6.563933in}}%
\pgfpathlineto{\pgfqpoint{1.361373in}{6.565147in}}%
\pgfpathlineto{\pgfqpoint{1.364735in}{6.566350in}}%
\pgfpathlineto{\pgfqpoint{1.369217in}{6.567514in}}%
\pgfpathlineto{\pgfqpoint{1.372578in}{6.569262in}}%
\pgfpathlineto{\pgfqpoint{1.377061in}{6.570523in}}%
\pgfpathlineto{\pgfqpoint{1.380422in}{6.572227in}}%
\pgfpathlineto{\pgfqpoint{1.384904in}{6.573427in}}%
\pgfpathlineto{\pgfqpoint{1.388266in}{6.575372in}}%
\pgfpathlineto{\pgfqpoint{1.392748in}{6.576701in}}%
\pgfpathlineto{\pgfqpoint{1.402833in}{6.580146in}}%
\pgfpathlineto{\pgfqpoint{1.403953in}{6.580791in}}%
\pgfpathlineto{\pgfqpoint{1.408435in}{6.582030in}}%
\pgfpathlineto{\pgfqpoint{1.411797in}{6.583824in}}%
\pgfpathlineto{\pgfqpoint{1.416279in}{6.584950in}}%
\pgfpathlineto{\pgfqpoint{1.419641in}{6.586681in}}%
\pgfpathlineto{\pgfqpoint{1.425243in}{6.587638in}}%
\pgfpathlineto{\pgfqpoint{1.427484in}{6.588430in}}%
\pgfpathlineto{\pgfqpoint{1.433087in}{6.589573in}}%
\pgfpathlineto{\pgfqpoint{1.435328in}{6.590336in}}%
\pgfpathlineto{\pgfqpoint{1.440931in}{6.591216in}}%
\pgfpathlineto{\pgfqpoint{1.443172in}{6.591916in}}%
\pgfpathlineto{\pgfqpoint{1.448774in}{6.593023in}}%
\pgfpathlineto{\pgfqpoint{1.451015in}{6.593815in}}%
\pgfpathlineto{\pgfqpoint{1.457738in}{6.594953in}}%
\pgfpathlineto{\pgfqpoint{1.458859in}{6.595299in}}%
\pgfpathlineto{\pgfqpoint{1.464462in}{6.596433in}}%
\pgfpathlineto{\pgfqpoint{1.466703in}{6.597222in}}%
\pgfpathlineto{\pgfqpoint{1.471185in}{6.597986in}}%
\pgfpathlineto{\pgfqpoint{1.474546in}{6.599331in}}%
\pgfpathlineto{\pgfqpoint{1.480149in}{6.600604in}}%
\pgfpathlineto{\pgfqpoint{1.482390in}{6.601312in}}%
\pgfpathlineto{\pgfqpoint{1.487993in}{6.602434in}}%
\pgfpathlineto{\pgfqpoint{1.490234in}{6.603151in}}%
\pgfpathlineto{\pgfqpoint{1.495836in}{6.604287in}}%
\pgfpathlineto{\pgfqpoint{1.498077in}{6.605089in}}%
\pgfpathlineto{\pgfqpoint{1.503680in}{6.606306in}}%
\pgfpathlineto{\pgfqpoint{1.505921in}{6.607148in}}%
\pgfpathlineto{\pgfqpoint{1.511524in}{6.608322in}}%
\pgfpathlineto{\pgfqpoint{1.513765in}{6.609032in}}%
\pgfpathlineto{\pgfqpoint{1.519367in}{6.610193in}}%
\pgfpathlineto{\pgfqpoint{1.520488in}{6.610641in}}%
\pgfpathlineto{\pgfqpoint{1.526091in}{6.611538in}}%
\pgfpathlineto{\pgfqpoint{1.529452in}{6.612845in}}%
\pgfpathlineto{\pgfqpoint{1.533934in}{6.613761in}}%
\pgfpathlineto{\pgfqpoint{1.537296in}{6.615124in}}%
\pgfpathlineto{\pgfqpoint{1.542899in}{6.616378in}}%
\pgfpathlineto{\pgfqpoint{1.545140in}{6.617208in}}%
\pgfpathlineto{\pgfqpoint{1.549622in}{6.618114in}}%
\pgfpathlineto{\pgfqpoint{1.552983in}{6.619406in}}%
\pgfpathlineto{\pgfqpoint{1.558586in}{6.620627in}}%
\pgfpathlineto{\pgfqpoint{1.560827in}{6.621442in}}%
\pgfpathlineto{\pgfqpoint{1.566430in}{6.622271in}}%
\pgfpathlineto{\pgfqpoint{1.568671in}{6.623121in}}%
\pgfpathlineto{\pgfqpoint{1.574273in}{6.624375in}}%
\pgfpathlineto{\pgfqpoint{1.576514in}{6.625248in}}%
\pgfpathlineto{\pgfqpoint{1.580996in}{6.626195in}}%
\pgfpathlineto{\pgfqpoint{1.584358in}{6.627505in}}%
\pgfpathlineto{\pgfqpoint{1.589961in}{6.628754in}}%
\pgfpathlineto{\pgfqpoint{1.592202in}{6.629633in}}%
\pgfpathlineto{\pgfqpoint{1.597804in}{6.630840in}}%
\pgfpathlineto{\pgfqpoint{1.600045in}{6.631604in}}%
\pgfpathlineto{\pgfqpoint{1.605648in}{6.632757in}}%
\pgfpathlineto{\pgfqpoint{1.606769in}{6.633176in}}%
\pgfpathlineto{\pgfqpoint{1.613492in}{6.634329in}}%
\pgfpathlineto{\pgfqpoint{1.615733in}{6.635073in}}%
\pgfpathlineto{\pgfqpoint{1.621335in}{6.636279in}}%
\pgfpathlineto{\pgfqpoint{1.623577in}{6.637038in}}%
\pgfpathlineto{\pgfqpoint{1.629179in}{6.638024in}}%
\pgfpathlineto{\pgfqpoint{1.631420in}{6.638658in}}%
\pgfpathlineto{\pgfqpoint{1.638143in}{6.639778in}}%
\pgfpathlineto{\pgfqpoint{1.642625in}{6.640308in}}%
\pgfpathlineto{\pgfqpoint{1.670639in}{6.646072in}}%
\pgfpathlineto{\pgfqpoint{1.677362in}{6.646961in}}%
\pgfpathlineto{\pgfqpoint{1.678482in}{6.647270in}}%
\pgfpathlineto{\pgfqpoint{1.684085in}{6.648162in}}%
\pgfpathlineto{\pgfqpoint{1.686326in}{6.648748in}}%
\pgfpathlineto{\pgfqpoint{1.691929in}{6.649665in}}%
\pgfpathlineto{\pgfqpoint{1.694170in}{6.650327in}}%
\pgfpathlineto{\pgfqpoint{1.699772in}{6.651254in}}%
\pgfpathlineto{\pgfqpoint{1.706496in}{6.652331in}}%
\pgfpathlineto{\pgfqpoint{1.716580in}{6.653939in}}%
\pgfpathlineto{\pgfqpoint{1.725544in}{6.654904in}}%
\pgfpathlineto{\pgfqpoint{1.732268in}{6.655849in}}%
\pgfpathlineto{\pgfqpoint{1.741232in}{6.657440in}}%
\pgfpathlineto{\pgfqpoint{1.746834in}{6.658237in}}%
\pgfpathlineto{\pgfqpoint{1.749076in}{6.658863in}}%
\pgfpathlineto{\pgfqpoint{1.754678in}{6.659811in}}%
\pgfpathlineto{\pgfqpoint{1.756919in}{6.660436in}}%
\pgfpathlineto{\pgfqpoint{1.762522in}{6.661460in}}%
\pgfpathlineto{\pgfqpoint{1.764763in}{6.662149in}}%
\pgfpathlineto{\pgfqpoint{1.775968in}{6.663752in}}%
\pgfpathlineto{\pgfqpoint{1.783812in}{6.665110in}}%
\pgfpathlineto{\pgfqpoint{1.788294in}{6.665959in}}%
\pgfpathlineto{\pgfqpoint{1.795017in}{6.666648in}}%
\pgfpathlineto{\pgfqpoint{1.801740in}{6.667689in}}%
\pgfpathlineto{\pgfqpoint{1.826392in}{6.670017in}}%
\pgfpathlineto{\pgfqpoint{1.871213in}{6.674179in}}%
\pgfpathlineto{\pgfqpoint{1.881298in}{6.676007in}}%
\pgfpathlineto{\pgfqpoint{1.890262in}{6.677305in}}%
\pgfpathlineto{\pgfqpoint{1.896985in}{6.678193in}}%
\pgfpathlineto{\pgfqpoint{1.905949in}{6.679380in}}%
\pgfpathlineto{\pgfqpoint{1.919396in}{6.680680in}}%
\pgfpathlineto{\pgfqpoint{1.968699in}{6.690995in}}%
\pgfpathlineto{\pgfqpoint{1.975422in}{6.691992in}}%
\pgfpathlineto{\pgfqpoint{1.976543in}{6.692305in}}%
\pgfpathlineto{\pgfqpoint{1.982145in}{6.693259in}}%
\pgfpathlineto{\pgfqpoint{1.984386in}{6.693878in}}%
\pgfpathlineto{\pgfqpoint{1.989989in}{6.694826in}}%
\pgfpathlineto{\pgfqpoint{1.992230in}{6.695460in}}%
\pgfpathlineto{\pgfqpoint{1.997833in}{6.696364in}}%
\pgfpathlineto{\pgfqpoint{2.000074in}{6.697002in}}%
\pgfpathlineto{\pgfqpoint{2.005676in}{6.697975in}}%
\pgfpathlineto{\pgfqpoint{2.007917in}{6.698575in}}%
\pgfpathlineto{\pgfqpoint{2.019123in}{6.699911in}}%
\pgfpathlineto{\pgfqpoint{2.023605in}{6.700861in}}%
\pgfpathlineto{\pgfqpoint{2.029207in}{6.701657in}}%
\pgfpathlineto{\pgfqpoint{2.034810in}{6.702544in}}%
\pgfpathlineto{\pgfqpoint{2.039292in}{6.703524in}}%
\pgfpathlineto{\pgfqpoint{2.046015in}{6.704429in}}%
\pgfpathlineto{\pgfqpoint{2.054979in}{6.705719in}}%
\pgfpathlineto{\pgfqpoint{2.066185in}{6.706996in}}%
\pgfpathlineto{\pgfqpoint{2.070667in}{6.707693in}}%
\pgfpathlineto{\pgfqpoint{2.084113in}{6.708545in}}%
\pgfpathlineto{\pgfqpoint{2.109885in}{6.710891in}}%
\pgfpathlineto{\pgfqpoint{2.117729in}{6.711649in}}%
\pgfpathlineto{\pgfqpoint{2.123332in}{6.712523in}}%
\pgfpathlineto{\pgfqpoint{2.125573in}{6.713194in}}%
\pgfpathlineto{\pgfqpoint{2.131175in}{6.714160in}}%
\pgfpathlineto{\pgfqpoint{2.141260in}{6.716851in}}%
\pgfpathlineto{\pgfqpoint{2.146863in}{6.718082in}}%
\pgfpathlineto{\pgfqpoint{2.149104in}{6.718858in}}%
\pgfpathlineto{\pgfqpoint{2.154706in}{6.720085in}}%
\pgfpathlineto{\pgfqpoint{2.156947in}{6.720929in}}%
\pgfpathlineto{\pgfqpoint{2.161430in}{6.721789in}}%
\pgfpathlineto{\pgfqpoint{2.164791in}{6.723149in}}%
\pgfpathlineto{\pgfqpoint{2.169273in}{6.724046in}}%
\pgfpathlineto{\pgfqpoint{2.172635in}{6.725411in}}%
\pgfpathlineto{\pgfqpoint{2.177117in}{6.726332in}}%
\pgfpathlineto{\pgfqpoint{2.180479in}{6.727220in}}%
\pgfpathlineto{\pgfqpoint{2.184961in}{6.728068in}}%
\pgfpathlineto{\pgfqpoint{2.188322in}{6.729353in}}%
\pgfpathlineto{\pgfqpoint{2.192804in}{6.730219in}}%
\pgfpathlineto{\pgfqpoint{2.196166in}{6.731542in}}%
\pgfpathlineto{\pgfqpoint{2.200648in}{6.732400in}}%
\pgfpathlineto{\pgfqpoint{2.204010in}{6.733760in}}%
\pgfpathlineto{\pgfqpoint{2.208492in}{6.734653in}}%
\pgfpathlineto{\pgfqpoint{2.210733in}{6.735594in}}%
\pgfpathlineto{\pgfqpoint{2.216335in}{6.736553in}}%
\pgfpathlineto{\pgfqpoint{2.218576in}{6.737510in}}%
\pgfpathlineto{\pgfqpoint{2.224179in}{6.738421in}}%
\pgfpathlineto{\pgfqpoint{2.227541in}{6.739524in}}%
\pgfpathlineto{\pgfqpoint{2.233143in}{6.740495in}}%
\pgfpathlineto{\pgfqpoint{2.235384in}{6.741172in}}%
\pgfpathlineto{\pgfqpoint{2.242107in}{6.742113in}}%
\pgfpathlineto{\pgfqpoint{2.243228in}{6.742418in}}%
\pgfpathlineto{\pgfqpoint{2.248831in}{6.743310in}}%
\pgfpathlineto{\pgfqpoint{2.258915in}{6.745580in}}%
\pgfpathlineto{\pgfqpoint{2.264518in}{6.746477in}}%
\pgfpathlineto{\pgfqpoint{2.266759in}{6.747027in}}%
\pgfpathlineto{\pgfqpoint{2.273482in}{6.748055in}}%
\pgfpathlineto{\pgfqpoint{2.274603in}{6.748393in}}%
\pgfpathlineto{\pgfqpoint{2.280205in}{6.749429in}}%
\pgfpathlineto{\pgfqpoint{2.282446in}{6.750156in}}%
\pgfpathlineto{\pgfqpoint{2.286929in}{6.750902in}}%
\pgfpathlineto{\pgfqpoint{2.290290in}{6.752144in}}%
\pgfpathlineto{\pgfqpoint{2.295893in}{6.753353in}}%
\pgfpathlineto{\pgfqpoint{2.298134in}{6.754156in}}%
\pgfpathlineto{\pgfqpoint{2.303736in}{6.755372in}}%
\pgfpathlineto{\pgfqpoint{2.305978in}{6.756255in}}%
\pgfpathlineto{\pgfqpoint{2.310460in}{6.757149in}}%
\pgfpathlineto{\pgfqpoint{2.312701in}{6.758042in}}%
\pgfpathlineto{\pgfqpoint{2.318303in}{6.758974in}}%
\pgfpathlineto{\pgfqpoint{2.321665in}{6.760425in}}%
\pgfpathlineto{\pgfqpoint{2.326147in}{6.761308in}}%
\pgfpathlineto{\pgfqpoint{2.329509in}{6.762572in}}%
\pgfpathlineto{\pgfqpoint{2.333991in}{6.763408in}}%
\pgfpathlineto{\pgfqpoint{2.337352in}{6.764698in}}%
\pgfpathlineto{\pgfqpoint{2.341834in}{6.765565in}}%
\pgfpathlineto{\pgfqpoint{2.345196in}{6.766836in}}%
\pgfpathlineto{\pgfqpoint{2.349678in}{6.767661in}}%
\pgfpathlineto{\pgfqpoint{2.353040in}{6.768904in}}%
\pgfpathlineto{\pgfqpoint{2.358642in}{6.770087in}}%
\pgfpathlineto{\pgfqpoint{2.360883in}{6.770817in}}%
\pgfpathlineto{\pgfqpoint{2.366486in}{6.771941in}}%
\pgfpathlineto{\pgfqpoint{2.368727in}{6.772650in}}%
\pgfpathlineto{\pgfqpoint{2.374330in}{6.773691in}}%
\pgfpathlineto{\pgfqpoint{2.376571in}{6.774350in}}%
\pgfpathlineto{\pgfqpoint{2.382173in}{6.775389in}}%
\pgfpathlineto{\pgfqpoint{2.384414in}{6.776107in}}%
\pgfpathlineto{\pgfqpoint{2.391138in}{6.777186in}}%
\pgfpathlineto{\pgfqpoint{2.392258in}{6.777535in}}%
\pgfpathlineto{\pgfqpoint{2.397861in}{6.778617in}}%
\pgfpathlineto{\pgfqpoint{2.400102in}{6.779330in}}%
\pgfpathlineto{\pgfqpoint{2.405704in}{6.780417in}}%
\pgfpathlineto{\pgfqpoint{2.407946in}{6.781197in}}%
\pgfpathlineto{\pgfqpoint{2.413548in}{6.782397in}}%
\pgfpathlineto{\pgfqpoint{2.415789in}{6.783158in}}%
\pgfpathlineto{\pgfqpoint{2.421392in}{6.784207in}}%
\pgfpathlineto{\pgfqpoint{2.423633in}{6.785062in}}%
\pgfpathlineto{\pgfqpoint{2.429236in}{6.785925in}}%
\pgfpathlineto{\pgfqpoint{2.431477in}{6.786829in}}%
\pgfpathlineto{\pgfqpoint{2.435959in}{6.787752in}}%
\pgfpathlineto{\pgfqpoint{2.439320in}{6.789190in}}%
\pgfpathlineto{\pgfqpoint{2.443802in}{6.790175in}}%
\pgfpathlineto{\pgfqpoint{2.447164in}{6.791578in}}%
\pgfpathlineto{\pgfqpoint{2.452767in}{6.792820in}}%
\pgfpathlineto{\pgfqpoint{2.455008in}{6.793616in}}%
\pgfpathlineto{\pgfqpoint{2.460610in}{6.794793in}}%
\pgfpathlineto{\pgfqpoint{2.462851in}{6.795586in}}%
\pgfpathlineto{\pgfqpoint{2.468454in}{6.796779in}}%
\pgfpathlineto{\pgfqpoint{2.470695in}{6.797569in}}%
\pgfpathlineto{\pgfqpoint{2.476298in}{6.798745in}}%
\pgfpathlineto{\pgfqpoint{2.478539in}{6.799536in}}%
\pgfpathlineto{\pgfqpoint{2.484141in}{6.800705in}}%
\pgfpathlineto{\pgfqpoint{2.486382in}{6.801477in}}%
\pgfpathlineto{\pgfqpoint{2.491985in}{6.802646in}}%
\pgfpathlineto{\pgfqpoint{2.494226in}{6.803414in}}%
\pgfpathlineto{\pgfqpoint{2.499829in}{6.804160in}}%
\pgfpathlineto{\pgfqpoint{2.502070in}{6.804857in}}%
\pgfpathlineto{\pgfqpoint{2.507672in}{6.805814in}}%
\pgfpathlineto{\pgfqpoint{2.509913in}{6.806443in}}%
\pgfpathlineto{\pgfqpoint{2.515516in}{6.807374in}}%
\pgfpathlineto{\pgfqpoint{2.517757in}{6.808022in}}%
\pgfpathlineto{\pgfqpoint{2.523360in}{6.808962in}}%
\pgfpathlineto{\pgfqpoint{2.525601in}{6.809567in}}%
\pgfpathlineto{\pgfqpoint{2.531204in}{6.810466in}}%
\pgfpathlineto{\pgfqpoint{2.533445in}{6.811027in}}%
\pgfpathlineto{\pgfqpoint{2.540168in}{6.812082in}}%
\pgfpathlineto{\pgfqpoint{2.549132in}{6.813683in}}%
\pgfpathlineto{\pgfqpoint{2.555855in}{6.814702in}}%
\pgfpathlineto{\pgfqpoint{2.560337in}{6.815247in}}%
\pgfpathlineto{\pgfqpoint{2.572663in}{6.817730in}}%
\pgfpathlineto{\pgfqpoint{2.578266in}{6.818748in}}%
\pgfpathlineto{\pgfqpoint{2.580507in}{6.819430in}}%
\pgfpathlineto{\pgfqpoint{2.586109in}{6.820506in}}%
\pgfpathlineto{\pgfqpoint{2.600676in}{6.823373in}}%
\pgfpathlineto{\pgfqpoint{2.604038in}{6.824502in}}%
\pgfpathlineto{\pgfqpoint{2.609640in}{6.825635in}}%
\pgfpathlineto{\pgfqpoint{2.611881in}{6.826368in}}%
\pgfpathlineto{\pgfqpoint{2.617484in}{6.827548in}}%
\pgfpathlineto{\pgfqpoint{2.619725in}{6.828330in}}%
\pgfpathlineto{\pgfqpoint{2.625328in}{6.829106in}}%
\pgfpathlineto{\pgfqpoint{2.627569in}{6.829865in}}%
\pgfpathlineto{\pgfqpoint{2.633171in}{6.830625in}}%
\pgfpathlineto{\pgfqpoint{2.635413in}{6.831371in}}%
\pgfpathlineto{\pgfqpoint{2.641015in}{6.832468in}}%
\pgfpathlineto{\pgfqpoint{2.643256in}{6.833190in}}%
\pgfpathlineto{\pgfqpoint{2.649979in}{6.834250in}}%
\pgfpathlineto{\pgfqpoint{2.651100in}{6.834569in}}%
\pgfpathlineto{\pgfqpoint{2.656703in}{6.835456in}}%
\pgfpathlineto{\pgfqpoint{2.658944in}{6.836056in}}%
\pgfpathlineto{\pgfqpoint{2.665667in}{6.837174in}}%
\pgfpathlineto{\pgfqpoint{2.671269in}{6.837993in}}%
\pgfpathlineto{\pgfqpoint{2.682475in}{6.840300in}}%
\pgfpathlineto{\pgfqpoint{2.689198in}{6.841204in}}%
\pgfpathlineto{\pgfqpoint{2.690318in}{6.841508in}}%
\pgfpathlineto{\pgfqpoint{2.695921in}{6.842386in}}%
\pgfpathlineto{\pgfqpoint{2.698162in}{6.842985in}}%
\pgfpathlineto{\pgfqpoint{2.704885in}{6.844119in}}%
\pgfpathlineto{\pgfqpoint{2.706006in}{6.844422in}}%
\pgfpathlineto{\pgfqpoint{2.712729in}{6.845529in}}%
\pgfpathlineto{\pgfqpoint{2.717211in}{6.846087in}}%
\pgfpathlineto{\pgfqpoint{2.729537in}{6.848505in}}%
\pgfpathlineto{\pgfqpoint{2.735139in}{6.849348in}}%
\pgfpathlineto{\pgfqpoint{2.742983in}{6.850742in}}%
\pgfpathlineto{\pgfqpoint{2.744104in}{6.851003in}}%
\pgfpathlineto{\pgfqpoint{2.750827in}{6.851817in}}%
\pgfpathlineto{\pgfqpoint{2.753068in}{6.852365in}}%
\pgfpathlineto{\pgfqpoint{2.759791in}{6.853354in}}%
\pgfpathlineto{\pgfqpoint{2.768755in}{6.854762in}}%
\pgfpathlineto{\pgfqpoint{2.779961in}{6.856044in}}%
\pgfpathlineto{\pgfqpoint{2.792286in}{6.857604in}}%
\pgfpathlineto{\pgfqpoint{2.803492in}{6.858431in}}%
\pgfpathlineto{\pgfqpoint{2.823661in}{6.861185in}}%
\pgfpathlineto{\pgfqpoint{2.843831in}{6.862790in}}%
\pgfpathlineto{\pgfqpoint{2.858397in}{6.863952in}}%
\pgfpathlineto{\pgfqpoint{2.886411in}{6.865438in}}%
\pgfpathlineto{\pgfqpoint{2.932352in}{6.866501in}}%
\pgfpathlineto{\pgfqpoint{2.957004in}{6.866738in}}%
\pgfpathlineto{\pgfqpoint{2.969330in}{6.866095in}}%
\pgfpathlineto{\pgfqpoint{2.980535in}{6.865013in}}%
\pgfpathlineto{\pgfqpoint{2.987258in}{6.864182in}}%
\pgfpathlineto{\pgfqpoint{2.993981in}{6.863304in}}%
\pgfpathlineto{\pgfqpoint{3.009669in}{6.861365in}}%
\pgfpathlineto{\pgfqpoint{3.019753in}{6.859808in}}%
\pgfpathlineto{\pgfqpoint{3.026477in}{6.858896in}}%
\pgfpathlineto{\pgfqpoint{3.027597in}{6.858666in}}%
\pgfpathlineto{\pgfqpoint{3.035441in}{6.857753in}}%
\pgfpathlineto{\pgfqpoint{3.035441in}{6.857753in}}%
\pgfusepath{stroke}%
\end{pgfscope}%
\begin{pgfscope}%
\pgfsetrectcap%
\pgfsetmiterjoin%
\pgfsetlinewidth{0.803000pt}%
\definecolor{currentstroke}{rgb}{1.000000,1.000000,1.000000}%
\pgfsetstrokecolor{currentstroke}%
\pgfsetdash{}{0pt}%
\pgfpathmoveto{\pgfqpoint{0.462318in}{6.297976in}}%
\pgfpathlineto{\pgfqpoint{0.462318in}{7.402855in}}%
\pgfusepath{stroke}%
\end{pgfscope}%
\begin{pgfscope}%
\pgfsetrectcap%
\pgfsetmiterjoin%
\pgfsetlinewidth{0.803000pt}%
\definecolor{currentstroke}{rgb}{1.000000,1.000000,1.000000}%
\pgfsetstrokecolor{currentstroke}%
\pgfsetdash{}{0pt}%
\pgfpathmoveto{\pgfqpoint{3.157970in}{6.297976in}}%
\pgfpathlineto{\pgfqpoint{3.157970in}{7.402855in}}%
\pgfusepath{stroke}%
\end{pgfscope}%
\begin{pgfscope}%
\pgfsetrectcap%
\pgfsetmiterjoin%
\pgfsetlinewidth{0.803000pt}%
\definecolor{currentstroke}{rgb}{1.000000,1.000000,1.000000}%
\pgfsetstrokecolor{currentstroke}%
\pgfsetdash{}{0pt}%
\pgfpathmoveto{\pgfqpoint{0.462318in}{6.297976in}}%
\pgfpathlineto{\pgfqpoint{3.157970in}{6.297976in}}%
\pgfusepath{stroke}%
\end{pgfscope}%
\begin{pgfscope}%
\pgfsetrectcap%
\pgfsetmiterjoin%
\pgfsetlinewidth{0.803000pt}%
\definecolor{currentstroke}{rgb}{1.000000,1.000000,1.000000}%
\pgfsetstrokecolor{currentstroke}%
\pgfsetdash{}{0pt}%
\pgfpathmoveto{\pgfqpoint{0.462318in}{7.402855in}}%
\pgfpathlineto{\pgfqpoint{3.157970in}{7.402855in}}%
\pgfusepath{stroke}%
\end{pgfscope}%
\begin{pgfscope}%
\definecolor{textcolor}{rgb}{0.150000,0.150000,0.150000}%
\pgfsetstrokecolor{textcolor}%
\pgfsetfillcolor{textcolor}%
\pgftext[x=1.810144in,y=7.486188in,,base]{\color{textcolor}\rmfamily\fontsize{12.000000}{14.400000}\selectfont GE}%
\end{pgfscope}%
\begin{pgfscope}%
\pgfsetbuttcap%
\pgfsetmiterjoin%
\definecolor{currentfill}{rgb}{0.917647,0.917647,0.949020}%
\pgfsetfillcolor{currentfill}%
\pgfsetlinewidth{0.000000pt}%
\definecolor{currentstroke}{rgb}{0.000000,0.000000,0.000000}%
\pgfsetstrokecolor{currentstroke}%
\pgfsetstrokeopacity{0.000000}%
\pgfsetdash{}{0pt}%
\pgfpathmoveto{\pgfqpoint{3.966666in}{6.297976in}}%
\pgfpathlineto{\pgfqpoint{6.662318in}{6.297976in}}%
\pgfpathlineto{\pgfqpoint{6.662318in}{7.402855in}}%
\pgfpathlineto{\pgfqpoint{3.966666in}{7.402855in}}%
\pgfpathclose%
\pgfusepath{fill}%
\end{pgfscope}%
\begin{pgfscope}%
\pgfpathrectangle{\pgfqpoint{3.966666in}{6.297976in}}{\pgfqpoint{2.695652in}{1.104878in}}%
\pgfusepath{clip}%
\pgfsetroundcap%
\pgfsetroundjoin%
\pgfsetlinewidth{0.803000pt}%
\definecolor{currentstroke}{rgb}{1.000000,1.000000,1.000000}%
\pgfsetstrokecolor{currentstroke}%
\pgfsetdash{}{0pt}%
\pgfpathmoveto{\pgfqpoint{4.086955in}{6.297976in}}%
\pgfpathlineto{\pgfqpoint{4.086955in}{7.402855in}}%
\pgfusepath{stroke}%
\end{pgfscope}%
\begin{pgfscope}%
\definecolor{textcolor}{rgb}{0.150000,0.150000,0.150000}%
\pgfsetstrokecolor{textcolor}%
\pgfsetfillcolor{textcolor}%
\pgftext[x=4.086955in,y=6.200754in,,top]{\color{textcolor}\rmfamily\fontsize{10.000000}{12.000000}\selectfont 2012}%
\end{pgfscope}%
\begin{pgfscope}%
\pgfpathrectangle{\pgfqpoint{3.966666in}{6.297976in}}{\pgfqpoint{2.695652in}{1.104878in}}%
\pgfusepath{clip}%
\pgfsetroundcap%
\pgfsetroundjoin%
\pgfsetlinewidth{0.803000pt}%
\definecolor{currentstroke}{rgb}{1.000000,1.000000,1.000000}%
\pgfsetstrokecolor{currentstroke}%
\pgfsetdash{}{0pt}%
\pgfpathmoveto{\pgfqpoint{4.497068in}{6.297976in}}%
\pgfpathlineto{\pgfqpoint{4.497068in}{7.402855in}}%
\pgfusepath{stroke}%
\end{pgfscope}%
\begin{pgfscope}%
\definecolor{textcolor}{rgb}{0.150000,0.150000,0.150000}%
\pgfsetstrokecolor{textcolor}%
\pgfsetfillcolor{textcolor}%
\pgftext[x=4.497068in,y=6.200754in,,top]{\color{textcolor}\rmfamily\fontsize{10.000000}{12.000000}\selectfont 2013}%
\end{pgfscope}%
\begin{pgfscope}%
\pgfpathrectangle{\pgfqpoint{3.966666in}{6.297976in}}{\pgfqpoint{2.695652in}{1.104878in}}%
\pgfusepath{clip}%
\pgfsetroundcap%
\pgfsetroundjoin%
\pgfsetlinewidth{0.803000pt}%
\definecolor{currentstroke}{rgb}{1.000000,1.000000,1.000000}%
\pgfsetstrokecolor{currentstroke}%
\pgfsetdash{}{0pt}%
\pgfpathmoveto{\pgfqpoint{4.906060in}{6.297976in}}%
\pgfpathlineto{\pgfqpoint{4.906060in}{7.402855in}}%
\pgfusepath{stroke}%
\end{pgfscope}%
\begin{pgfscope}%
\definecolor{textcolor}{rgb}{0.150000,0.150000,0.150000}%
\pgfsetstrokecolor{textcolor}%
\pgfsetfillcolor{textcolor}%
\pgftext[x=4.906060in,y=6.200754in,,top]{\color{textcolor}\rmfamily\fontsize{10.000000}{12.000000}\selectfont 2014}%
\end{pgfscope}%
\begin{pgfscope}%
\pgfpathrectangle{\pgfqpoint{3.966666in}{6.297976in}}{\pgfqpoint{2.695652in}{1.104878in}}%
\pgfusepath{clip}%
\pgfsetroundcap%
\pgfsetroundjoin%
\pgfsetlinewidth{0.803000pt}%
\definecolor{currentstroke}{rgb}{1.000000,1.000000,1.000000}%
\pgfsetstrokecolor{currentstroke}%
\pgfsetdash{}{0pt}%
\pgfpathmoveto{\pgfqpoint{5.315052in}{6.297976in}}%
\pgfpathlineto{\pgfqpoint{5.315052in}{7.402855in}}%
\pgfusepath{stroke}%
\end{pgfscope}%
\begin{pgfscope}%
\definecolor{textcolor}{rgb}{0.150000,0.150000,0.150000}%
\pgfsetstrokecolor{textcolor}%
\pgfsetfillcolor{textcolor}%
\pgftext[x=5.315052in,y=6.200754in,,top]{\color{textcolor}\rmfamily\fontsize{10.000000}{12.000000}\selectfont 2015}%
\end{pgfscope}%
\begin{pgfscope}%
\pgfpathrectangle{\pgfqpoint{3.966666in}{6.297976in}}{\pgfqpoint{2.695652in}{1.104878in}}%
\pgfusepath{clip}%
\pgfsetroundcap%
\pgfsetroundjoin%
\pgfsetlinewidth{0.803000pt}%
\definecolor{currentstroke}{rgb}{1.000000,1.000000,1.000000}%
\pgfsetstrokecolor{currentstroke}%
\pgfsetdash{}{0pt}%
\pgfpathmoveto{\pgfqpoint{5.724045in}{6.297976in}}%
\pgfpathlineto{\pgfqpoint{5.724045in}{7.402855in}}%
\pgfusepath{stroke}%
\end{pgfscope}%
\begin{pgfscope}%
\definecolor{textcolor}{rgb}{0.150000,0.150000,0.150000}%
\pgfsetstrokecolor{textcolor}%
\pgfsetfillcolor{textcolor}%
\pgftext[x=5.724045in,y=6.200754in,,top]{\color{textcolor}\rmfamily\fontsize{10.000000}{12.000000}\selectfont 2016}%
\end{pgfscope}%
\begin{pgfscope}%
\pgfpathrectangle{\pgfqpoint{3.966666in}{6.297976in}}{\pgfqpoint{2.695652in}{1.104878in}}%
\pgfusepath{clip}%
\pgfsetroundcap%
\pgfsetroundjoin%
\pgfsetlinewidth{0.803000pt}%
\definecolor{currentstroke}{rgb}{1.000000,1.000000,1.000000}%
\pgfsetstrokecolor{currentstroke}%
\pgfsetdash{}{0pt}%
\pgfpathmoveto{\pgfqpoint{6.134158in}{6.297976in}}%
\pgfpathlineto{\pgfqpoint{6.134158in}{7.402855in}}%
\pgfusepath{stroke}%
\end{pgfscope}%
\begin{pgfscope}%
\definecolor{textcolor}{rgb}{0.150000,0.150000,0.150000}%
\pgfsetstrokecolor{textcolor}%
\pgfsetfillcolor{textcolor}%
\pgftext[x=6.134158in,y=6.200754in,,top]{\color{textcolor}\rmfamily\fontsize{10.000000}{12.000000}\selectfont 2017}%
\end{pgfscope}%
\begin{pgfscope}%
\pgfpathrectangle{\pgfqpoint{3.966666in}{6.297976in}}{\pgfqpoint{2.695652in}{1.104878in}}%
\pgfusepath{clip}%
\pgfsetroundcap%
\pgfsetroundjoin%
\pgfsetlinewidth{0.803000pt}%
\definecolor{currentstroke}{rgb}{1.000000,1.000000,1.000000}%
\pgfsetstrokecolor{currentstroke}%
\pgfsetdash{}{0pt}%
\pgfpathmoveto{\pgfqpoint{6.543150in}{6.297976in}}%
\pgfpathlineto{\pgfqpoint{6.543150in}{7.402855in}}%
\pgfusepath{stroke}%
\end{pgfscope}%
\begin{pgfscope}%
\definecolor{textcolor}{rgb}{0.150000,0.150000,0.150000}%
\pgfsetstrokecolor{textcolor}%
\pgfsetfillcolor{textcolor}%
\pgftext[x=6.543150in,y=6.200754in,,top]{\color{textcolor}\rmfamily\fontsize{10.000000}{12.000000}\selectfont 2018}%
\end{pgfscope}%
\begin{pgfscope}%
\pgfpathrectangle{\pgfqpoint{3.966666in}{6.297976in}}{\pgfqpoint{2.695652in}{1.104878in}}%
\pgfusepath{clip}%
\pgfsetroundcap%
\pgfsetroundjoin%
\pgfsetlinewidth{0.803000pt}%
\definecolor{currentstroke}{rgb}{1.000000,1.000000,1.000000}%
\pgfsetstrokecolor{currentstroke}%
\pgfsetdash{}{0pt}%
\pgfpathmoveto{\pgfqpoint{3.966666in}{6.490772in}}%
\pgfpathlineto{\pgfqpoint{6.662318in}{6.490772in}}%
\pgfusepath{stroke}%
\end{pgfscope}%
\begin{pgfscope}%
\definecolor{textcolor}{rgb}{0.150000,0.150000,0.150000}%
\pgfsetstrokecolor{textcolor}%
\pgfsetfillcolor{textcolor}%
\pgftext[x=3.692713in,y=6.438010in,left,base]{\color{textcolor}\rmfamily\fontsize{10.000000}{12.000000}\selectfont 20}%
\end{pgfscope}%
\begin{pgfscope}%
\pgfpathrectangle{\pgfqpoint{3.966666in}{6.297976in}}{\pgfqpoint{2.695652in}{1.104878in}}%
\pgfusepath{clip}%
\pgfsetroundcap%
\pgfsetroundjoin%
\pgfsetlinewidth{0.803000pt}%
\definecolor{currentstroke}{rgb}{1.000000,1.000000,1.000000}%
\pgfsetstrokecolor{currentstroke}%
\pgfsetdash{}{0pt}%
\pgfpathmoveto{\pgfqpoint{3.966666in}{7.166476in}}%
\pgfpathlineto{\pgfqpoint{6.662318in}{7.166476in}}%
\pgfusepath{stroke}%
\end{pgfscope}%
\begin{pgfscope}%
\definecolor{textcolor}{rgb}{0.150000,0.150000,0.150000}%
\pgfsetstrokecolor{textcolor}%
\pgfsetfillcolor{textcolor}%
\pgftext[x=3.692713in,y=7.113715in,left,base]{\color{textcolor}\rmfamily\fontsize{10.000000}{12.000000}\selectfont 40}%
\end{pgfscope}%
\begin{pgfscope}%
\pgfpathrectangle{\pgfqpoint{3.966666in}{6.297976in}}{\pgfqpoint{2.695652in}{1.104878in}}%
\pgfusepath{clip}%
\pgfsetroundcap%
\pgfsetroundjoin%
\pgfsetlinewidth{1.505625pt}%
\definecolor{currentstroke}{rgb}{0.839216,0.152941,0.156863}%
\pgfsetstrokecolor{currentstroke}%
\pgfsetdash{}{0pt}%
\pgfpathmoveto{\pgfqpoint{4.089196in}{6.468136in}}%
\pgfpathlineto{\pgfqpoint{4.091437in}{6.491110in}}%
\pgfpathlineto{\pgfqpoint{4.092557in}{6.487055in}}%
\pgfpathlineto{\pgfqpoint{4.097039in}{6.496177in}}%
\pgfpathlineto{\pgfqpoint{4.098160in}{6.501921in}}%
\pgfpathlineto{\pgfqpoint{4.099280in}{6.500570in}}%
\pgfpathlineto{\pgfqpoint{4.100401in}{6.484353in}}%
\pgfpathlineto{\pgfqpoint{4.104883in}{6.481650in}}%
\pgfpathlineto{\pgfqpoint{4.107124in}{6.497191in}}%
\pgfpathlineto{\pgfqpoint{4.108245in}{6.517124in}}%
\pgfpathlineto{\pgfqpoint{4.111606in}{6.525908in}}%
\pgfpathlineto{\pgfqpoint{4.112727in}{6.530976in}}%
\pgfpathlineto{\pgfqpoint{4.113847in}{6.530976in}}%
\pgfpathlineto{\pgfqpoint{4.114968in}{6.526922in}}%
\pgfpathlineto{\pgfqpoint{4.116088in}{6.526584in}}%
\pgfpathlineto{\pgfqpoint{4.119450in}{6.526922in}}%
\pgfpathlineto{\pgfqpoint{4.120570in}{6.518138in}}%
\pgfpathlineto{\pgfqpoint{4.121691in}{6.521854in}}%
\pgfpathlineto{\pgfqpoint{4.122811in}{6.520165in}}%
\pgfpathlineto{\pgfqpoint{4.123932in}{6.532328in}}%
\pgfpathlineto{\pgfqpoint{4.127294in}{6.531990in}}%
\pgfpathlineto{\pgfqpoint{4.128414in}{6.529625in}}%
\pgfpathlineto{\pgfqpoint{4.129535in}{6.535368in}}%
\pgfpathlineto{\pgfqpoint{4.130655in}{6.535706in}}%
\pgfpathlineto{\pgfqpoint{4.131776in}{6.531314in}}%
\pgfpathlineto{\pgfqpoint{4.135137in}{6.531314in}}%
\pgfpathlineto{\pgfqpoint{4.136258in}{6.533679in}}%
\pgfpathlineto{\pgfqpoint{4.137378in}{6.528273in}}%
\pgfpathlineto{\pgfqpoint{4.138499in}{6.535030in}}%
\pgfpathlineto{\pgfqpoint{4.139619in}{6.549220in}}%
\pgfpathlineto{\pgfqpoint{4.144101in}{6.543815in}}%
\pgfpathlineto{\pgfqpoint{4.145222in}{6.532328in}}%
\pgfpathlineto{\pgfqpoint{4.146343in}{6.530301in}}%
\pgfpathlineto{\pgfqpoint{4.147463in}{6.531314in}}%
\pgfpathlineto{\pgfqpoint{4.150825in}{6.536382in}}%
\pgfpathlineto{\pgfqpoint{4.151945in}{6.545842in}}%
\pgfpathlineto{\pgfqpoint{4.153066in}{6.536382in}}%
\pgfpathlineto{\pgfqpoint{4.154186in}{6.535706in}}%
\pgfpathlineto{\pgfqpoint{4.155307in}{6.537395in}}%
\pgfpathlineto{\pgfqpoint{4.158668in}{6.527260in}}%
\pgfpathlineto{\pgfqpoint{4.159789in}{6.528949in}}%
\pgfpathlineto{\pgfqpoint{4.160909in}{6.537058in}}%
\pgfpathlineto{\pgfqpoint{4.162030in}{6.535030in}}%
\pgfpathlineto{\pgfqpoint{4.163150in}{6.541450in}}%
\pgfpathlineto{\pgfqpoint{4.166512in}{6.539085in}}%
\pgfpathlineto{\pgfqpoint{4.167633in}{6.552599in}}%
\pgfpathlineto{\pgfqpoint{4.168753in}{6.551923in}}%
\pgfpathlineto{\pgfqpoint{4.169874in}{6.559694in}}%
\pgfpathlineto{\pgfqpoint{4.170994in}{6.559018in}}%
\pgfpathlineto{\pgfqpoint{4.175476in}{6.559694in}}%
\pgfpathlineto{\pgfqpoint{4.176597in}{6.560369in}}%
\pgfpathlineto{\pgfqpoint{4.177717in}{6.563748in}}%
\pgfpathlineto{\pgfqpoint{4.178838in}{6.563072in}}%
\pgfpathlineto{\pgfqpoint{4.182199in}{6.571518in}}%
\pgfpathlineto{\pgfqpoint{4.183320in}{6.571518in}}%
\pgfpathlineto{\pgfqpoint{4.184440in}{6.561045in}}%
\pgfpathlineto{\pgfqpoint{4.185561in}{6.570505in}}%
\pgfpathlineto{\pgfqpoint{4.186682in}{6.569491in}}%
\pgfpathlineto{\pgfqpoint{4.190043in}{6.576586in}}%
\pgfpathlineto{\pgfqpoint{4.192284in}{6.564424in}}%
\pgfpathlineto{\pgfqpoint{4.193405in}{6.568140in}}%
\pgfpathlineto{\pgfqpoint{4.197887in}{6.559694in}}%
\pgfpathlineto{\pgfqpoint{4.199007in}{6.551585in}}%
\pgfpathlineto{\pgfqpoint{4.200128in}{6.562396in}}%
\pgfpathlineto{\pgfqpoint{4.201248in}{6.579289in}}%
\pgfpathlineto{\pgfqpoint{4.202369in}{6.568816in}}%
\pgfpathlineto{\pgfqpoint{4.206851in}{6.578951in}}%
\pgfpathlineto{\pgfqpoint{4.207972in}{6.564761in}}%
\pgfpathlineto{\pgfqpoint{4.209092in}{6.558004in}}%
\pgfpathlineto{\pgfqpoint{4.210213in}{6.555639in}}%
\pgfpathlineto{\pgfqpoint{4.213574in}{6.551585in}}%
\pgfpathlineto{\pgfqpoint{4.214695in}{6.547869in}}%
\pgfpathlineto{\pgfqpoint{4.216936in}{6.572194in}}%
\pgfpathlineto{\pgfqpoint{4.218056in}{6.576586in}}%
\pgfpathlineto{\pgfqpoint{4.221418in}{6.576924in}}%
\pgfpathlineto{\pgfqpoint{4.222538in}{6.591790in}}%
\pgfpathlineto{\pgfqpoint{4.223659in}{6.597871in}}%
\pgfpathlineto{\pgfqpoint{4.224779in}{6.586722in}}%
\pgfpathlineto{\pgfqpoint{4.225900in}{6.569154in}}%
\pgfpathlineto{\pgfqpoint{4.229262in}{6.565099in}}%
\pgfpathlineto{\pgfqpoint{4.230382in}{6.554626in}}%
\pgfpathlineto{\pgfqpoint{4.231503in}{6.549896in}}%
\pgfpathlineto{\pgfqpoint{4.232623in}{6.551247in}}%
\pgfpathlineto{\pgfqpoint{4.233744in}{6.561721in}}%
\pgfpathlineto{\pgfqpoint{4.238226in}{6.541450in}}%
\pgfpathlineto{\pgfqpoint{4.240467in}{6.522868in}}%
\pgfpathlineto{\pgfqpoint{4.241587in}{6.519489in}}%
\pgfpathlineto{\pgfqpoint{4.244949in}{6.521854in}}%
\pgfpathlineto{\pgfqpoint{4.246069in}{6.518476in}}%
\pgfpathlineto{\pgfqpoint{4.247190in}{6.502597in}}%
\pgfpathlineto{\pgfqpoint{4.248311in}{6.508340in}}%
\pgfpathlineto{\pgfqpoint{4.253913in}{6.520165in}}%
\pgfpathlineto{\pgfqpoint{4.255034in}{6.521179in}}%
\pgfpathlineto{\pgfqpoint{4.256154in}{6.513408in}}%
\pgfpathlineto{\pgfqpoint{4.257275in}{6.494488in}}%
\pgfpathlineto{\pgfqpoint{4.260636in}{6.491785in}}%
\pgfpathlineto{\pgfqpoint{4.261757in}{6.502259in}}%
\pgfpathlineto{\pgfqpoint{4.262877in}{6.519489in}}%
\pgfpathlineto{\pgfqpoint{4.263998in}{6.516111in}}%
\pgfpathlineto{\pgfqpoint{4.265118in}{6.528611in}}%
\pgfpathlineto{\pgfqpoint{4.268480in}{6.517462in}}%
\pgfpathlineto{\pgfqpoint{4.269601in}{6.531652in}}%
\pgfpathlineto{\pgfqpoint{4.270721in}{6.532328in}}%
\pgfpathlineto{\pgfqpoint{4.272962in}{6.553950in}}%
\pgfpathlineto{\pgfqpoint{4.276324in}{6.555977in}}%
\pgfpathlineto{\pgfqpoint{4.277444in}{6.558342in}}%
\pgfpathlineto{\pgfqpoint{4.278565in}{6.562059in}}%
\pgfpathlineto{\pgfqpoint{4.279685in}{6.536720in}}%
\pgfpathlineto{\pgfqpoint{4.280806in}{6.543139in}}%
\pgfpathlineto{\pgfqpoint{4.284167in}{6.519151in}}%
\pgfpathlineto{\pgfqpoint{4.285288in}{6.517800in}}%
\pgfpathlineto{\pgfqpoint{4.286408in}{6.523543in}}%
\pgfpathlineto{\pgfqpoint{4.287529in}{6.513070in}}%
\pgfpathlineto{\pgfqpoint{4.288649in}{6.535368in}}%
\pgfpathlineto{\pgfqpoint{4.292011in}{6.535706in}}%
\pgfpathlineto{\pgfqpoint{4.293132in}{6.540774in}}%
\pgfpathlineto{\pgfqpoint{4.295373in}{6.532665in}}%
\pgfpathlineto{\pgfqpoint{4.296493in}{6.521854in}}%
\pgfpathlineto{\pgfqpoint{4.299855in}{6.522192in}}%
\pgfpathlineto{\pgfqpoint{4.300975in}{6.505637in}}%
\pgfpathlineto{\pgfqpoint{4.302096in}{6.501245in}}%
\pgfpathlineto{\pgfqpoint{4.303216in}{6.483677in}}%
\pgfpathlineto{\pgfqpoint{4.304337in}{6.497529in}}%
\pgfpathlineto{\pgfqpoint{4.307698in}{6.494150in}}%
\pgfpathlineto{\pgfqpoint{4.308819in}{6.500907in}}%
\pgfpathlineto{\pgfqpoint{4.309940in}{6.523206in}}%
\pgfpathlineto{\pgfqpoint{4.311060in}{6.519151in}}%
\pgfpathlineto{\pgfqpoint{4.312181in}{6.504624in}}%
\pgfpathlineto{\pgfqpoint{4.315542in}{6.497529in}}%
\pgfpathlineto{\pgfqpoint{4.316663in}{6.490772in}}%
\pgfpathlineto{\pgfqpoint{4.317783in}{6.494150in}}%
\pgfpathlineto{\pgfqpoint{4.320024in}{6.518138in}}%
\pgfpathlineto{\pgfqpoint{4.324506in}{6.509692in}}%
\pgfpathlineto{\pgfqpoint{4.325627in}{6.515773in}}%
\pgfpathlineto{\pgfqpoint{4.326747in}{6.515097in}}%
\pgfpathlineto{\pgfqpoint{4.327868in}{6.529963in}}%
\pgfpathlineto{\pgfqpoint{4.331230in}{6.532328in}}%
\pgfpathlineto{\pgfqpoint{4.333471in}{6.540098in}}%
\pgfpathlineto{\pgfqpoint{4.334591in}{6.542801in}}%
\pgfpathlineto{\pgfqpoint{4.335712in}{6.547869in}}%
\pgfpathlineto{\pgfqpoint{4.339073in}{6.542801in}}%
\pgfpathlineto{\pgfqpoint{4.341314in}{6.531314in}}%
\pgfpathlineto{\pgfqpoint{4.342435in}{6.539760in}}%
\pgfpathlineto{\pgfqpoint{4.343555in}{6.532665in}}%
\pgfpathlineto{\pgfqpoint{4.346917in}{6.529963in}}%
\pgfpathlineto{\pgfqpoint{4.348037in}{6.526922in}}%
\pgfpathlineto{\pgfqpoint{4.349158in}{6.516449in}}%
\pgfpathlineto{\pgfqpoint{4.350278in}{6.497529in}}%
\pgfpathlineto{\pgfqpoint{4.351399in}{6.494150in}}%
\pgfpathlineto{\pgfqpoint{4.354761in}{6.492123in}}%
\pgfpathlineto{\pgfqpoint{4.355881in}{6.496515in}}%
\pgfpathlineto{\pgfqpoint{4.358122in}{6.476582in}}%
\pgfpathlineto{\pgfqpoint{4.359243in}{6.491785in}}%
\pgfpathlineto{\pgfqpoint{4.363725in}{6.480636in}}%
\pgfpathlineto{\pgfqpoint{4.364845in}{6.479961in}}%
\pgfpathlineto{\pgfqpoint{4.365966in}{6.499218in}}%
\pgfpathlineto{\pgfqpoint{4.367086in}{6.474555in}}%
\pgfpathlineto{\pgfqpoint{4.370448in}{6.449216in}}%
\pgfpathlineto{\pgfqpoint{4.371569in}{6.451243in}}%
\pgfpathlineto{\pgfqpoint{4.372689in}{6.447189in}}%
\pgfpathlineto{\pgfqpoint{4.373810in}{6.451919in}}%
\pgfpathlineto{\pgfqpoint{4.374930in}{6.452257in}}%
\pgfpathlineto{\pgfqpoint{4.378292in}{6.450567in}}%
\pgfpathlineto{\pgfqpoint{4.379412in}{6.452257in}}%
\pgfpathlineto{\pgfqpoint{4.380533in}{6.446175in}}%
\pgfpathlineto{\pgfqpoint{4.381653in}{6.446851in}}%
\pgfpathlineto{\pgfqpoint{4.382774in}{6.445500in}}%
\pgfpathlineto{\pgfqpoint{4.386135in}{6.436715in}}%
\pgfpathlineto{\pgfqpoint{4.387256in}{6.429621in}}%
\pgfpathlineto{\pgfqpoint{4.388376in}{6.432661in}}%
\pgfpathlineto{\pgfqpoint{4.389497in}{6.444486in}}%
\pgfpathlineto{\pgfqpoint{4.390617in}{6.432661in}}%
\pgfpathlineto{\pgfqpoint{4.393979in}{6.435364in}}%
\pgfpathlineto{\pgfqpoint{4.395100in}{6.437729in}}%
\pgfpathlineto{\pgfqpoint{4.396220in}{6.429621in}}%
\pgfpathlineto{\pgfqpoint{4.397341in}{6.427593in}}%
\pgfpathlineto{\pgfqpoint{4.398461in}{6.433337in}}%
\pgfpathlineto{\pgfqpoint{4.401823in}{6.428607in}}%
\pgfpathlineto{\pgfqpoint{4.402943in}{6.412052in}}%
\pgfpathlineto{\pgfqpoint{4.406305in}{6.400565in}}%
\pgfpathlineto{\pgfqpoint{4.409666in}{6.407322in}}%
\pgfpathlineto{\pgfqpoint{4.410787in}{6.424215in}}%
\pgfpathlineto{\pgfqpoint{4.411907in}{6.409012in}}%
\pgfpathlineto{\pgfqpoint{4.413028in}{6.405633in}}%
\pgfpathlineto{\pgfqpoint{4.414149in}{6.394822in}}%
\pgfpathlineto{\pgfqpoint{4.417510in}{6.400227in}}%
\pgfpathlineto{\pgfqpoint{4.418631in}{6.403606in}}%
\pgfpathlineto{\pgfqpoint{4.419751in}{6.400227in}}%
\pgfpathlineto{\pgfqpoint{4.421992in}{6.413404in}}%
\pgfpathlineto{\pgfqpoint{4.427595in}{6.404620in}}%
\pgfpathlineto{\pgfqpoint{4.428715in}{6.421850in}}%
\pgfpathlineto{\pgfqpoint{4.429836in}{6.416444in}}%
\pgfpathlineto{\pgfqpoint{4.433198in}{6.416444in}}%
\pgfpathlineto{\pgfqpoint{4.434318in}{6.413404in}}%
\pgfpathlineto{\pgfqpoint{4.435439in}{6.391105in}}%
\pgfpathlineto{\pgfqpoint{4.436559in}{6.388740in}}%
\pgfpathlineto{\pgfqpoint{4.437680in}{6.388065in}}%
\pgfpathlineto{\pgfqpoint{4.441041in}{6.387051in}}%
\pgfpathlineto{\pgfqpoint{4.443282in}{6.364753in}}%
\pgfpathlineto{\pgfqpoint{4.444403in}{6.366780in}}%
\pgfpathlineto{\pgfqpoint{4.445523in}{6.371172in}}%
\pgfpathlineto{\pgfqpoint{4.448885in}{6.372861in}}%
\pgfpathlineto{\pgfqpoint{4.450005in}{6.352252in}}%
\pgfpathlineto{\pgfqpoint{4.451126in}{6.348198in}}%
\pgfpathlineto{\pgfqpoint{4.453367in}{6.358334in}}%
\pgfpathlineto{\pgfqpoint{4.457849in}{6.364077in}}%
\pgfpathlineto{\pgfqpoint{4.458970in}{6.368469in}}%
\pgfpathlineto{\pgfqpoint{4.460090in}{6.352928in}}%
\pgfpathlineto{\pgfqpoint{4.461211in}{6.353942in}}%
\pgfpathlineto{\pgfqpoint{4.464572in}{6.353266in}}%
\pgfpathlineto{\pgfqpoint{4.465693in}{6.365091in}}%
\pgfpathlineto{\pgfqpoint{4.466813in}{6.361712in}}%
\pgfpathlineto{\pgfqpoint{4.467934in}{6.370159in}}%
\pgfpathlineto{\pgfqpoint{4.469054in}{6.370159in}}%
\pgfpathlineto{\pgfqpoint{4.472416in}{6.368131in}}%
\pgfpathlineto{\pgfqpoint{4.473536in}{6.383673in}}%
\pgfpathlineto{\pgfqpoint{4.474657in}{6.384348in}}%
\pgfpathlineto{\pgfqpoint{4.475778in}{6.379281in}}%
\pgfpathlineto{\pgfqpoint{4.476898in}{6.380632in}}%
\pgfpathlineto{\pgfqpoint{4.480260in}{6.381646in}}%
\pgfpathlineto{\pgfqpoint{4.481380in}{6.392457in}}%
\pgfpathlineto{\pgfqpoint{4.482501in}{6.396173in}}%
\pgfpathlineto{\pgfqpoint{4.483621in}{6.394146in}}%
\pgfpathlineto{\pgfqpoint{4.484742in}{6.387051in}}%
\pgfpathlineto{\pgfqpoint{4.488103in}{6.383673in}}%
\pgfpathlineto{\pgfqpoint{4.490344in}{6.383673in}}%
\pgfpathlineto{\pgfqpoint{4.491465in}{6.379956in}}%
\pgfpathlineto{\pgfqpoint{4.492585in}{6.372186in}}%
\pgfpathlineto{\pgfqpoint{4.495947in}{6.382997in}}%
\pgfpathlineto{\pgfqpoint{4.498188in}{6.403944in}}%
\pgfpathlineto{\pgfqpoint{4.499309in}{6.402255in}}%
\pgfpathlineto{\pgfqpoint{4.500429in}{6.397862in}}%
\pgfpathlineto{\pgfqpoint{4.503791in}{6.400227in}}%
\pgfpathlineto{\pgfqpoint{4.504911in}{6.395835in}}%
\pgfpathlineto{\pgfqpoint{4.507152in}{6.415431in}}%
\pgfpathlineto{\pgfqpoint{4.508273in}{6.420836in}}%
\pgfpathlineto{\pgfqpoint{4.511634in}{6.420836in}}%
\pgfpathlineto{\pgfqpoint{4.512755in}{6.417796in}}%
\pgfpathlineto{\pgfqpoint{4.513875in}{6.423877in}}%
\pgfpathlineto{\pgfqpoint{4.514996in}{6.439756in}}%
\pgfpathlineto{\pgfqpoint{4.516117in}{6.400227in}}%
\pgfpathlineto{\pgfqpoint{4.520599in}{6.398200in}}%
\pgfpathlineto{\pgfqpoint{4.521719in}{6.396511in}}%
\pgfpathlineto{\pgfqpoint{4.522840in}{6.392119in}}%
\pgfpathlineto{\pgfqpoint{4.523960in}{6.392457in}}%
\pgfpathlineto{\pgfqpoint{4.527322in}{6.394822in}}%
\pgfpathlineto{\pgfqpoint{4.528442in}{6.401241in}}%
\pgfpathlineto{\pgfqpoint{4.529563in}{6.403606in}}%
\pgfpathlineto{\pgfqpoint{4.530683in}{6.394484in}}%
\pgfpathlineto{\pgfqpoint{4.531804in}{6.403268in}}%
\pgfpathlineto{\pgfqpoint{4.535165in}{6.397862in}}%
\pgfpathlineto{\pgfqpoint{4.536286in}{6.404620in}}%
\pgfpathlineto{\pgfqpoint{4.538527in}{6.394484in}}%
\pgfpathlineto{\pgfqpoint{4.539648in}{6.399552in}}%
\pgfpathlineto{\pgfqpoint{4.543009in}{6.400565in}}%
\pgfpathlineto{\pgfqpoint{4.544130in}{6.404957in}}%
\pgfpathlineto{\pgfqpoint{4.545250in}{6.406647in}}%
\pgfpathlineto{\pgfqpoint{4.546371in}{6.405971in}}%
\pgfpathlineto{\pgfqpoint{4.547491in}{6.402930in}}%
\pgfpathlineto{\pgfqpoint{4.551973in}{6.402255in}}%
\pgfpathlineto{\pgfqpoint{4.554214in}{6.378605in}}%
\pgfpathlineto{\pgfqpoint{4.555335in}{6.383335in}}%
\pgfpathlineto{\pgfqpoint{4.558697in}{6.378267in}}%
\pgfpathlineto{\pgfqpoint{4.560938in}{6.397525in}}%
\pgfpathlineto{\pgfqpoint{4.562058in}{6.396173in}}%
\pgfpathlineto{\pgfqpoint{4.563179in}{6.400565in}}%
\pgfpathlineto{\pgfqpoint{4.566540in}{6.407322in}}%
\pgfpathlineto{\pgfqpoint{4.569902in}{6.424553in}}%
\pgfpathlineto{\pgfqpoint{4.571022in}{6.415769in}}%
\pgfpathlineto{\pgfqpoint{4.574384in}{6.418809in}}%
\pgfpathlineto{\pgfqpoint{4.575504in}{6.417458in}}%
\pgfpathlineto{\pgfqpoint{4.576625in}{6.418134in}}%
\pgfpathlineto{\pgfqpoint{4.577746in}{6.417796in}}%
\pgfpathlineto{\pgfqpoint{4.578866in}{6.410363in}}%
\pgfpathlineto{\pgfqpoint{4.582228in}{6.406984in}}%
\pgfpathlineto{\pgfqpoint{4.583348in}{6.403606in}}%
\pgfpathlineto{\pgfqpoint{4.584469in}{6.404620in}}%
\pgfpathlineto{\pgfqpoint{4.585589in}{6.400903in}}%
\pgfpathlineto{\pgfqpoint{4.586710in}{6.408674in}}%
\pgfpathlineto{\pgfqpoint{4.590071in}{6.403944in}}%
\pgfpathlineto{\pgfqpoint{4.591192in}{6.421174in}}%
\pgfpathlineto{\pgfqpoint{4.592312in}{6.422864in}}%
\pgfpathlineto{\pgfqpoint{4.593433in}{6.422864in}}%
\pgfpathlineto{\pgfqpoint{4.597915in}{6.411714in}}%
\pgfpathlineto{\pgfqpoint{4.599036in}{6.412390in}}%
\pgfpathlineto{\pgfqpoint{4.600156in}{6.400903in}}%
\pgfpathlineto{\pgfqpoint{4.601277in}{6.403606in}}%
\pgfpathlineto{\pgfqpoint{4.602397in}{6.397862in}}%
\pgfpathlineto{\pgfqpoint{4.605759in}{6.402255in}}%
\pgfpathlineto{\pgfqpoint{4.608000in}{6.434688in}}%
\pgfpathlineto{\pgfqpoint{4.609120in}{6.422864in}}%
\pgfpathlineto{\pgfqpoint{4.610241in}{6.418471in}}%
\pgfpathlineto{\pgfqpoint{4.613602in}{6.410363in}}%
\pgfpathlineto{\pgfqpoint{4.614723in}{6.425228in}}%
\pgfpathlineto{\pgfqpoint{4.615843in}{6.425566in}}%
\pgfpathlineto{\pgfqpoint{4.618084in}{6.439756in}}%
\pgfpathlineto{\pgfqpoint{4.621446in}{6.451919in}}%
\pgfpathlineto{\pgfqpoint{4.623687in}{6.473541in}}%
\pgfpathlineto{\pgfqpoint{4.624808in}{6.465771in}}%
\pgfpathlineto{\pgfqpoint{4.625928in}{6.466446in}}%
\pgfpathlineto{\pgfqpoint{4.633772in}{6.488407in}}%
\pgfpathlineto{\pgfqpoint{4.637133in}{6.487055in}}%
\pgfpathlineto{\pgfqpoint{4.638254in}{6.493812in}}%
\pgfpathlineto{\pgfqpoint{4.640495in}{6.499556in}}%
\pgfpathlineto{\pgfqpoint{4.641616in}{6.503610in}}%
\pgfpathlineto{\pgfqpoint{4.644977in}{6.491785in}}%
\pgfpathlineto{\pgfqpoint{4.646098in}{6.485028in}}%
\pgfpathlineto{\pgfqpoint{4.647218in}{6.495164in}}%
\pgfpathlineto{\pgfqpoint{4.648339in}{6.487731in}}%
\pgfpathlineto{\pgfqpoint{4.649459in}{6.490434in}}%
\pgfpathlineto{\pgfqpoint{4.652821in}{6.491785in}}%
\pgfpathlineto{\pgfqpoint{4.653941in}{6.493812in}}%
\pgfpathlineto{\pgfqpoint{4.655062in}{6.491448in}}%
\pgfpathlineto{\pgfqpoint{4.656182in}{6.490772in}}%
\pgfpathlineto{\pgfqpoint{4.657303in}{6.487055in}}%
\pgfpathlineto{\pgfqpoint{4.661785in}{6.491785in}}%
\pgfpathlineto{\pgfqpoint{4.662906in}{6.497191in}}%
\pgfpathlineto{\pgfqpoint{4.664026in}{6.495502in}}%
\pgfpathlineto{\pgfqpoint{4.665147in}{6.497191in}}%
\pgfpathlineto{\pgfqpoint{4.668508in}{6.524219in}}%
\pgfpathlineto{\pgfqpoint{4.669629in}{6.527598in}}%
\pgfpathlineto{\pgfqpoint{4.670749in}{6.509016in}}%
\pgfpathlineto{\pgfqpoint{4.672990in}{6.505975in}}%
\pgfpathlineto{\pgfqpoint{4.676352in}{6.517800in}}%
\pgfpathlineto{\pgfqpoint{4.678593in}{6.502259in}}%
\pgfpathlineto{\pgfqpoint{4.679713in}{6.517124in}}%
\pgfpathlineto{\pgfqpoint{4.680834in}{6.515435in}}%
\pgfpathlineto{\pgfqpoint{4.684196in}{6.520503in}}%
\pgfpathlineto{\pgfqpoint{4.685316in}{6.530638in}}%
\pgfpathlineto{\pgfqpoint{4.686437in}{6.517462in}}%
\pgfpathlineto{\pgfqpoint{4.687557in}{6.494826in}}%
\pgfpathlineto{\pgfqpoint{4.688678in}{6.495164in}}%
\pgfpathlineto{\pgfqpoint{4.692039in}{6.477596in}}%
\pgfpathlineto{\pgfqpoint{4.693160in}{6.486042in}}%
\pgfpathlineto{\pgfqpoint{4.694280in}{6.489758in}}%
\pgfpathlineto{\pgfqpoint{4.695401in}{6.490772in}}%
\pgfpathlineto{\pgfqpoint{4.696521in}{6.495840in}}%
\pgfpathlineto{\pgfqpoint{4.701003in}{6.481650in}}%
\pgfpathlineto{\pgfqpoint{4.702124in}{6.482663in}}%
\pgfpathlineto{\pgfqpoint{4.704365in}{6.491110in}}%
\pgfpathlineto{\pgfqpoint{4.707727in}{6.466784in}}%
\pgfpathlineto{\pgfqpoint{4.708847in}{6.465433in}}%
\pgfpathlineto{\pgfqpoint{4.709968in}{6.468474in}}%
\pgfpathlineto{\pgfqpoint{4.711088in}{6.489083in}}%
\pgfpathlineto{\pgfqpoint{4.712209in}{6.486718in}}%
\pgfpathlineto{\pgfqpoint{4.715570in}{6.487731in}}%
\pgfpathlineto{\pgfqpoint{4.716691in}{6.496515in}}%
\pgfpathlineto{\pgfqpoint{4.717811in}{6.493812in}}%
\pgfpathlineto{\pgfqpoint{4.718932in}{6.468136in}}%
\pgfpathlineto{\pgfqpoint{4.720052in}{6.462392in}}%
\pgfpathlineto{\pgfqpoint{4.723414in}{6.454959in}}%
\pgfpathlineto{\pgfqpoint{4.724535in}{6.454284in}}%
\pgfpathlineto{\pgfqpoint{4.727896in}{6.468811in}}%
\pgfpathlineto{\pgfqpoint{4.731258in}{6.468136in}}%
\pgfpathlineto{\pgfqpoint{4.732378in}{6.472190in}}%
\pgfpathlineto{\pgfqpoint{4.733499in}{6.470839in}}%
\pgfpathlineto{\pgfqpoint{4.734619in}{6.467122in}}%
\pgfpathlineto{\pgfqpoint{4.735740in}{6.467460in}}%
\pgfpathlineto{\pgfqpoint{4.739101in}{6.465433in}}%
\pgfpathlineto{\pgfqpoint{4.741342in}{6.459014in}}%
\pgfpathlineto{\pgfqpoint{4.742463in}{6.451919in}}%
\pgfpathlineto{\pgfqpoint{4.743584in}{6.453608in}}%
\pgfpathlineto{\pgfqpoint{4.746945in}{6.457324in}}%
\pgfpathlineto{\pgfqpoint{4.748066in}{6.453946in}}%
\pgfpathlineto{\pgfqpoint{4.749186in}{6.455635in}}%
\pgfpathlineto{\pgfqpoint{4.750307in}{6.440094in}}%
\pgfpathlineto{\pgfqpoint{4.751427in}{6.436715in}}%
\pgfpathlineto{\pgfqpoint{4.754789in}{6.447189in}}%
\pgfpathlineto{\pgfqpoint{4.755909in}{6.453946in}}%
\pgfpathlineto{\pgfqpoint{4.757030in}{6.444148in}}%
\pgfpathlineto{\pgfqpoint{4.758150in}{6.446513in}}%
\pgfpathlineto{\pgfqpoint{4.759271in}{6.451919in}}%
\pgfpathlineto{\pgfqpoint{4.763753in}{6.444824in}}%
\pgfpathlineto{\pgfqpoint{4.764874in}{6.447527in}}%
\pgfpathlineto{\pgfqpoint{4.765994in}{6.441108in}}%
\pgfpathlineto{\pgfqpoint{4.767115in}{6.438743in}}%
\pgfpathlineto{\pgfqpoint{4.771597in}{6.441445in}}%
\pgfpathlineto{\pgfqpoint{4.772717in}{6.457324in}}%
\pgfpathlineto{\pgfqpoint{4.773838in}{6.456311in}}%
\pgfpathlineto{\pgfqpoint{4.779440in}{6.467122in}}%
\pgfpathlineto{\pgfqpoint{4.781681in}{6.457324in}}%
\pgfpathlineto{\pgfqpoint{4.782802in}{6.480298in}}%
\pgfpathlineto{\pgfqpoint{4.786164in}{6.478609in}}%
\pgfpathlineto{\pgfqpoint{4.787284in}{6.488745in}}%
\pgfpathlineto{\pgfqpoint{4.788405in}{6.493137in}}%
\pgfpathlineto{\pgfqpoint{4.789525in}{6.493475in}}%
\pgfpathlineto{\pgfqpoint{4.790646in}{6.489420in}}%
\pgfpathlineto{\pgfqpoint{4.794007in}{6.485366in}}%
\pgfpathlineto{\pgfqpoint{4.795128in}{6.487393in}}%
\pgfpathlineto{\pgfqpoint{4.796248in}{6.487393in}}%
\pgfpathlineto{\pgfqpoint{4.798489in}{6.467122in}}%
\pgfpathlineto{\pgfqpoint{4.801851in}{6.465433in}}%
\pgfpathlineto{\pgfqpoint{4.802971in}{6.462730in}}%
\pgfpathlineto{\pgfqpoint{4.804092in}{6.464419in}}%
\pgfpathlineto{\pgfqpoint{4.805213in}{6.456311in}}%
\pgfpathlineto{\pgfqpoint{4.806333in}{6.462392in}}%
\pgfpathlineto{\pgfqpoint{4.809695in}{6.462730in}}%
\pgfpathlineto{\pgfqpoint{4.810815in}{6.452932in}}%
\pgfpathlineto{\pgfqpoint{4.811936in}{6.455973in}}%
\pgfpathlineto{\pgfqpoint{4.813056in}{6.470501in}}%
\pgfpathlineto{\pgfqpoint{4.814177in}{6.474893in}}%
\pgfpathlineto{\pgfqpoint{4.817538in}{6.480298in}}%
\pgfpathlineto{\pgfqpoint{4.818659in}{6.478609in}}%
\pgfpathlineto{\pgfqpoint{4.820900in}{6.493812in}}%
\pgfpathlineto{\pgfqpoint{4.822020in}{6.492799in}}%
\pgfpathlineto{\pgfqpoint{4.825382in}{6.499894in}}%
\pgfpathlineto{\pgfqpoint{4.826503in}{6.497867in}}%
\pgfpathlineto{\pgfqpoint{4.827623in}{6.488745in}}%
\pgfpathlineto{\pgfqpoint{4.828744in}{6.489758in}}%
\pgfpathlineto{\pgfqpoint{4.829864in}{6.502935in}}%
\pgfpathlineto{\pgfqpoint{4.833226in}{6.506313in}}%
\pgfpathlineto{\pgfqpoint{4.834346in}{6.510705in}}%
\pgfpathlineto{\pgfqpoint{4.835467in}{6.510367in}}%
\pgfpathlineto{\pgfqpoint{4.836587in}{6.509354in}}%
\pgfpathlineto{\pgfqpoint{4.837708in}{6.505299in}}%
\pgfpathlineto{\pgfqpoint{4.841069in}{6.503272in}}%
\pgfpathlineto{\pgfqpoint{4.842190in}{6.503272in}}%
\pgfpathlineto{\pgfqpoint{4.843310in}{6.509692in}}%
\pgfpathlineto{\pgfqpoint{4.844431in}{6.504286in}}%
\pgfpathlineto{\pgfqpoint{4.848913in}{6.507327in}}%
\pgfpathlineto{\pgfqpoint{4.851154in}{6.519489in}}%
\pgfpathlineto{\pgfqpoint{4.852275in}{6.513746in}}%
\pgfpathlineto{\pgfqpoint{4.853395in}{6.517462in}}%
\pgfpathlineto{\pgfqpoint{4.856757in}{6.519489in}}%
\pgfpathlineto{\pgfqpoint{4.857877in}{6.522530in}}%
\pgfpathlineto{\pgfqpoint{4.858998in}{6.518476in}}%
\pgfpathlineto{\pgfqpoint{4.860118in}{6.537733in}}%
\pgfpathlineto{\pgfqpoint{4.861239in}{6.498542in}}%
\pgfpathlineto{\pgfqpoint{4.864600in}{6.495164in}}%
\pgfpathlineto{\pgfqpoint{4.865721in}{6.492461in}}%
\pgfpathlineto{\pgfqpoint{4.866842in}{6.499556in}}%
\pgfpathlineto{\pgfqpoint{4.869083in}{6.497867in}}%
\pgfpathlineto{\pgfqpoint{4.872444in}{6.493812in}}%
\pgfpathlineto{\pgfqpoint{4.873565in}{6.489420in}}%
\pgfpathlineto{\pgfqpoint{4.874685in}{6.494826in}}%
\pgfpathlineto{\pgfqpoint{4.876926in}{6.525908in}}%
\pgfpathlineto{\pgfqpoint{4.880288in}{6.528949in}}%
\pgfpathlineto{\pgfqpoint{4.881408in}{6.525908in}}%
\pgfpathlineto{\pgfqpoint{4.882529in}{6.514421in}}%
\pgfpathlineto{\pgfqpoint{4.883649in}{6.515773in}}%
\pgfpathlineto{\pgfqpoint{4.884770in}{6.510705in}}%
\pgfpathlineto{\pgfqpoint{4.888132in}{6.515435in}}%
\pgfpathlineto{\pgfqpoint{4.889252in}{6.521179in}}%
\pgfpathlineto{\pgfqpoint{4.890373in}{6.535368in}}%
\pgfpathlineto{\pgfqpoint{4.891493in}{6.535030in}}%
\pgfpathlineto{\pgfqpoint{4.892614in}{6.532665in}}%
\pgfpathlineto{\pgfqpoint{4.897096in}{6.543477in}}%
\pgfpathlineto{\pgfqpoint{4.899337in}{6.551247in}}%
\pgfpathlineto{\pgfqpoint{4.900457in}{6.548207in}}%
\pgfpathlineto{\pgfqpoint{4.904939in}{6.558680in}}%
\pgfpathlineto{\pgfqpoint{4.907180in}{6.553612in}}%
\pgfpathlineto{\pgfqpoint{4.908301in}{6.553274in}}%
\pgfpathlineto{\pgfqpoint{4.911663in}{6.544152in}}%
\pgfpathlineto{\pgfqpoint{4.912783in}{6.547869in}}%
\pgfpathlineto{\pgfqpoint{4.915024in}{6.540098in}}%
\pgfpathlineto{\pgfqpoint{4.916145in}{6.546180in}}%
\pgfpathlineto{\pgfqpoint{4.919506in}{6.545504in}}%
\pgfpathlineto{\pgfqpoint{4.920627in}{6.574221in}}%
\pgfpathlineto{\pgfqpoint{4.921747in}{6.578951in}}%
\pgfpathlineto{\pgfqpoint{4.922868in}{6.575235in}}%
\pgfpathlineto{\pgfqpoint{4.923988in}{6.555302in}}%
\pgfpathlineto{\pgfqpoint{4.928471in}{6.547869in}}%
\pgfpathlineto{\pgfqpoint{4.931832in}{6.525571in}}%
\pgfpathlineto{\pgfqpoint{4.935194in}{6.523206in}}%
\pgfpathlineto{\pgfqpoint{4.936314in}{6.528273in}}%
\pgfpathlineto{\pgfqpoint{4.937435in}{6.521854in}}%
\pgfpathlineto{\pgfqpoint{4.938555in}{6.523543in}}%
\pgfpathlineto{\pgfqpoint{4.944158in}{6.497191in}}%
\pgfpathlineto{\pgfqpoint{4.945278in}{6.495164in}}%
\pgfpathlineto{\pgfqpoint{4.946399in}{6.508678in}}%
\pgfpathlineto{\pgfqpoint{4.947519in}{6.515097in}}%
\pgfpathlineto{\pgfqpoint{4.950881in}{6.517462in}}%
\pgfpathlineto{\pgfqpoint{4.952002in}{6.522530in}}%
\pgfpathlineto{\pgfqpoint{4.953122in}{6.524895in}}%
\pgfpathlineto{\pgfqpoint{4.954243in}{6.529287in}}%
\pgfpathlineto{\pgfqpoint{4.955363in}{6.530976in}}%
\pgfpathlineto{\pgfqpoint{4.959845in}{6.530976in}}%
\pgfpathlineto{\pgfqpoint{4.960966in}{6.523543in}}%
\pgfpathlineto{\pgfqpoint{4.962086in}{6.530301in}}%
\pgfpathlineto{\pgfqpoint{4.963207in}{6.521179in}}%
\pgfpathlineto{\pgfqpoint{4.966568in}{6.527260in}}%
\pgfpathlineto{\pgfqpoint{4.967689in}{6.526922in}}%
\pgfpathlineto{\pgfqpoint{4.968809in}{6.531990in}}%
\pgfpathlineto{\pgfqpoint{4.969930in}{6.530976in}}%
\pgfpathlineto{\pgfqpoint{4.971051in}{6.530976in}}%
\pgfpathlineto{\pgfqpoint{4.974412in}{6.523543in}}%
\pgfpathlineto{\pgfqpoint{4.975533in}{6.526584in}}%
\pgfpathlineto{\pgfqpoint{4.976653in}{6.523543in}}%
\pgfpathlineto{\pgfqpoint{4.977774in}{6.527260in}}%
\pgfpathlineto{\pgfqpoint{4.978894in}{6.527598in}}%
\pgfpathlineto{\pgfqpoint{4.982256in}{6.533341in}}%
\pgfpathlineto{\pgfqpoint{4.983376in}{6.529963in}}%
\pgfpathlineto{\pgfqpoint{4.984497in}{6.530976in}}%
\pgfpathlineto{\pgfqpoint{4.985617in}{6.525571in}}%
\pgfpathlineto{\pgfqpoint{4.986738in}{6.523543in}}%
\pgfpathlineto{\pgfqpoint{4.990099in}{6.529287in}}%
\pgfpathlineto{\pgfqpoint{4.991220in}{6.532665in}}%
\pgfpathlineto{\pgfqpoint{4.992341in}{6.538409in}}%
\pgfpathlineto{\pgfqpoint{4.993461in}{6.550234in}}%
\pgfpathlineto{\pgfqpoint{4.994582in}{6.542801in}}%
\pgfpathlineto{\pgfqpoint{4.997943in}{6.541450in}}%
\pgfpathlineto{\pgfqpoint{4.999064in}{6.551247in}}%
\pgfpathlineto{\pgfqpoint{5.001305in}{6.546855in}}%
\pgfpathlineto{\pgfqpoint{5.002425in}{6.555977in}}%
\pgfpathlineto{\pgfqpoint{5.005787in}{6.561383in}}%
\pgfpathlineto{\pgfqpoint{5.006907in}{6.566451in}}%
\pgfpathlineto{\pgfqpoint{5.008028in}{6.563748in}}%
\pgfpathlineto{\pgfqpoint{5.009148in}{6.578613in}}%
\pgfpathlineto{\pgfqpoint{5.010269in}{6.571518in}}%
\pgfpathlineto{\pgfqpoint{5.013631in}{6.580978in}}%
\pgfpathlineto{\pgfqpoint{5.014751in}{6.593141in}}%
\pgfpathlineto{\pgfqpoint{5.015872in}{6.595168in}}%
\pgfpathlineto{\pgfqpoint{5.016992in}{6.579289in}}%
\pgfpathlineto{\pgfqpoint{5.018113in}{6.571856in}}%
\pgfpathlineto{\pgfqpoint{5.021474in}{6.583005in}}%
\pgfpathlineto{\pgfqpoint{5.023715in}{6.593817in}}%
\pgfpathlineto{\pgfqpoint{5.024836in}{6.596857in}}%
\pgfpathlineto{\pgfqpoint{5.029318in}{6.594155in}}%
\pgfpathlineto{\pgfqpoint{5.031559in}{6.588411in}}%
\pgfpathlineto{\pgfqpoint{5.032680in}{6.588411in}}%
\pgfpathlineto{\pgfqpoint{5.033800in}{6.574221in}}%
\pgfpathlineto{\pgfqpoint{5.037162in}{6.576248in}}%
\pgfpathlineto{\pgfqpoint{5.039403in}{6.586722in}}%
\pgfpathlineto{\pgfqpoint{5.040523in}{6.579965in}}%
\pgfpathlineto{\pgfqpoint{5.041644in}{6.578613in}}%
\pgfpathlineto{\pgfqpoint{5.045005in}{6.578276in}}%
\pgfpathlineto{\pgfqpoint{5.046126in}{6.578951in}}%
\pgfpathlineto{\pgfqpoint{5.047246in}{6.584019in}}%
\pgfpathlineto{\pgfqpoint{5.049487in}{6.581992in}}%
\pgfpathlineto{\pgfqpoint{5.052849in}{6.584019in}}%
\pgfpathlineto{\pgfqpoint{5.053970in}{6.586384in}}%
\pgfpathlineto{\pgfqpoint{5.055090in}{6.583005in}}%
\pgfpathlineto{\pgfqpoint{5.057331in}{6.568140in}}%
\pgfpathlineto{\pgfqpoint{5.060693in}{6.574559in}}%
\pgfpathlineto{\pgfqpoint{5.061813in}{6.574559in}}%
\pgfpathlineto{\pgfqpoint{5.062934in}{6.578951in}}%
\pgfpathlineto{\pgfqpoint{5.064054in}{6.577600in}}%
\pgfpathlineto{\pgfqpoint{5.065175in}{6.581654in}}%
\pgfpathlineto{\pgfqpoint{5.069657in}{6.593817in}}%
\pgfpathlineto{\pgfqpoint{5.070777in}{6.598885in}}%
\pgfpathlineto{\pgfqpoint{5.071898in}{6.601249in}}%
\pgfpathlineto{\pgfqpoint{5.073019in}{6.611723in}}%
\pgfpathlineto{\pgfqpoint{5.076380in}{6.610034in}}%
\pgfpathlineto{\pgfqpoint{5.077501in}{6.621521in}}%
\pgfpathlineto{\pgfqpoint{5.078621in}{6.619831in}}%
\pgfpathlineto{\pgfqpoint{5.079742in}{6.621521in}}%
\pgfpathlineto{\pgfqpoint{5.080862in}{6.636386in}}%
\pgfpathlineto{\pgfqpoint{5.084224in}{6.628953in}}%
\pgfpathlineto{\pgfqpoint{5.085344in}{6.638413in}}%
\pgfpathlineto{\pgfqpoint{5.086465in}{6.629629in}}%
\pgfpathlineto{\pgfqpoint{5.087585in}{6.630305in}}%
\pgfpathlineto{\pgfqpoint{5.088706in}{6.686050in}}%
\pgfpathlineto{\pgfqpoint{5.092067in}{6.690105in}}%
\pgfpathlineto{\pgfqpoint{5.093188in}{6.688415in}}%
\pgfpathlineto{\pgfqpoint{5.094309in}{6.687740in}}%
\pgfpathlineto{\pgfqpoint{5.096550in}{6.695848in}}%
\pgfpathlineto{\pgfqpoint{5.099911in}{6.696524in}}%
\pgfpathlineto{\pgfqpoint{5.102152in}{6.715444in}}%
\pgfpathlineto{\pgfqpoint{5.103273in}{6.712741in}}%
\pgfpathlineto{\pgfqpoint{5.104393in}{6.717133in}}%
\pgfpathlineto{\pgfqpoint{5.107755in}{6.716119in}}%
\pgfpathlineto{\pgfqpoint{5.108875in}{6.718484in}}%
\pgfpathlineto{\pgfqpoint{5.109996in}{6.718484in}}%
\pgfpathlineto{\pgfqpoint{5.111116in}{6.723214in}}%
\pgfpathlineto{\pgfqpoint{5.115599in}{6.719836in}}%
\pgfpathlineto{\pgfqpoint{5.116719in}{6.713079in}}%
\pgfpathlineto{\pgfqpoint{5.117840in}{6.715781in}}%
\pgfpathlineto{\pgfqpoint{5.118960in}{6.726593in}}%
\pgfpathlineto{\pgfqpoint{5.120081in}{6.726255in}}%
\pgfpathlineto{\pgfqpoint{5.123442in}{6.733350in}}%
\pgfpathlineto{\pgfqpoint{5.124563in}{6.739769in}}%
\pgfpathlineto{\pgfqpoint{5.125683in}{6.825583in}}%
\pgfpathlineto{\pgfqpoint{5.126804in}{6.797879in}}%
\pgfpathlineto{\pgfqpoint{5.127924in}{6.797879in}}%
\pgfpathlineto{\pgfqpoint{5.131286in}{6.808353in}}%
\pgfpathlineto{\pgfqpoint{5.132406in}{6.829638in}}%
\pgfpathlineto{\pgfqpoint{5.134647in}{6.813759in}}%
\pgfpathlineto{\pgfqpoint{5.139130in}{6.813421in}}%
\pgfpathlineto{\pgfqpoint{5.140250in}{6.812069in}}%
\pgfpathlineto{\pgfqpoint{5.141371in}{6.816799in}}%
\pgfpathlineto{\pgfqpoint{5.142491in}{6.803285in}}%
\pgfpathlineto{\pgfqpoint{5.143612in}{6.798893in}}%
\pgfpathlineto{\pgfqpoint{5.146973in}{6.808015in}}%
\pgfpathlineto{\pgfqpoint{5.148094in}{6.778622in}}%
\pgfpathlineto{\pgfqpoint{5.149214in}{6.779298in}}%
\pgfpathlineto{\pgfqpoint{5.150335in}{6.774230in}}%
\pgfpathlineto{\pgfqpoint{5.151455in}{6.772203in}}%
\pgfpathlineto{\pgfqpoint{5.155938in}{6.787744in}}%
\pgfpathlineto{\pgfqpoint{5.157058in}{6.816123in}}%
\pgfpathlineto{\pgfqpoint{5.158179in}{6.811394in}}%
\pgfpathlineto{\pgfqpoint{5.159299in}{6.818151in}}%
\pgfpathlineto{\pgfqpoint{5.162661in}{6.825245in}}%
\pgfpathlineto{\pgfqpoint{5.163781in}{6.823218in}}%
\pgfpathlineto{\pgfqpoint{5.164902in}{6.827948in}}%
\pgfpathlineto{\pgfqpoint{5.166022in}{6.846868in}}%
\pgfpathlineto{\pgfqpoint{5.167143in}{6.840787in}}%
\pgfpathlineto{\pgfqpoint{5.170504in}{6.837070in}}%
\pgfpathlineto{\pgfqpoint{5.172745in}{6.836395in}}%
\pgfpathlineto{\pgfqpoint{5.173866in}{6.832340in}}%
\pgfpathlineto{\pgfqpoint{5.174986in}{6.840111in}}%
\pgfpathlineto{\pgfqpoint{5.179469in}{6.829975in}}%
\pgfpathlineto{\pgfqpoint{5.180589in}{6.829975in}}%
\pgfpathlineto{\pgfqpoint{5.181710in}{6.839773in}}%
\pgfpathlineto{\pgfqpoint{5.182830in}{6.842476in}}%
\pgfpathlineto{\pgfqpoint{5.186192in}{6.852274in}}%
\pgfpathlineto{\pgfqpoint{5.187312in}{6.839773in}}%
\pgfpathlineto{\pgfqpoint{5.188433in}{6.843152in}}%
\pgfpathlineto{\pgfqpoint{5.189553in}{6.843152in}}%
\pgfpathlineto{\pgfqpoint{5.190674in}{6.831327in}}%
\pgfpathlineto{\pgfqpoint{5.194035in}{6.828962in}}%
\pgfpathlineto{\pgfqpoint{5.195156in}{6.840449in}}%
\pgfpathlineto{\pgfqpoint{5.196276in}{6.841800in}}%
\pgfpathlineto{\pgfqpoint{5.197397in}{6.847544in}}%
\pgfpathlineto{\pgfqpoint{5.198518in}{6.837070in}}%
\pgfpathlineto{\pgfqpoint{5.201879in}{6.834030in}}%
\pgfpathlineto{\pgfqpoint{5.203000in}{6.825583in}}%
\pgfpathlineto{\pgfqpoint{5.204120in}{6.835043in}}%
\pgfpathlineto{\pgfqpoint{5.205241in}{6.817137in}}%
\pgfpathlineto{\pgfqpoint{5.206361in}{6.820853in}}%
\pgfpathlineto{\pgfqpoint{5.209723in}{6.839435in}}%
\pgfpathlineto{\pgfqpoint{5.210843in}{6.837070in}}%
\pgfpathlineto{\pgfqpoint{5.213084in}{6.798893in}}%
\pgfpathlineto{\pgfqpoint{5.214205in}{6.814096in}}%
\pgfpathlineto{\pgfqpoint{5.217567in}{6.816461in}}%
\pgfpathlineto{\pgfqpoint{5.218687in}{6.797542in}}%
\pgfpathlineto{\pgfqpoint{5.219808in}{6.821191in}}%
\pgfpathlineto{\pgfqpoint{5.220928in}{6.801934in}}%
\pgfpathlineto{\pgfqpoint{5.222049in}{6.751932in}}%
\pgfpathlineto{\pgfqpoint{5.225410in}{6.738755in}}%
\pgfpathlineto{\pgfqpoint{5.226531in}{6.758689in}}%
\pgfpathlineto{\pgfqpoint{5.228772in}{6.720511in}}%
\pgfpathlineto{\pgfqpoint{5.229892in}{6.736390in}}%
\pgfpathlineto{\pgfqpoint{5.233254in}{6.742134in}}%
\pgfpathlineto{\pgfqpoint{5.234374in}{6.772203in}}%
\pgfpathlineto{\pgfqpoint{5.235495in}{6.762405in}}%
\pgfpathlineto{\pgfqpoint{5.237736in}{6.789095in}}%
\pgfpathlineto{\pgfqpoint{5.241098in}{6.789771in}}%
\pgfpathlineto{\pgfqpoint{5.242218in}{6.805650in}}%
\pgfpathlineto{\pgfqpoint{5.243339in}{6.810718in}}%
\pgfpathlineto{\pgfqpoint{5.244459in}{6.771527in}}%
\pgfpathlineto{\pgfqpoint{5.245580in}{6.813421in}}%
\pgfpathlineto{\pgfqpoint{5.248941in}{6.822205in}}%
\pgfpathlineto{\pgfqpoint{5.250062in}{6.828962in}}%
\pgfpathlineto{\pgfqpoint{5.251182in}{6.812745in}}%
\pgfpathlineto{\pgfqpoint{5.252303in}{6.814434in}}%
\pgfpathlineto{\pgfqpoint{5.253423in}{6.807339in}}%
\pgfpathlineto{\pgfqpoint{5.256785in}{6.797879in}}%
\pgfpathlineto{\pgfqpoint{5.259026in}{6.801258in}}%
\pgfpathlineto{\pgfqpoint{5.261267in}{6.818151in}}%
\pgfpathlineto{\pgfqpoint{5.264629in}{6.826935in}}%
\pgfpathlineto{\pgfqpoint{5.265749in}{6.840787in}}%
\pgfpathlineto{\pgfqpoint{5.266870in}{6.829975in}}%
\pgfpathlineto{\pgfqpoint{5.267990in}{6.877275in}}%
\pgfpathlineto{\pgfqpoint{5.269111in}{6.866801in}}%
\pgfpathlineto{\pgfqpoint{5.272472in}{6.886059in}}%
\pgfpathlineto{\pgfqpoint{5.273593in}{6.888086in}}%
\pgfpathlineto{\pgfqpoint{5.274713in}{6.905316in}}%
\pgfpathlineto{\pgfqpoint{5.276954in}{6.915790in}}%
\pgfpathlineto{\pgfqpoint{5.280316in}{6.913425in}}%
\pgfpathlineto{\pgfqpoint{5.281437in}{6.925925in}}%
\pgfpathlineto{\pgfqpoint{5.282557in}{6.921196in}}%
\pgfpathlineto{\pgfqpoint{5.283678in}{6.921871in}}%
\pgfpathlineto{\pgfqpoint{5.284798in}{6.928290in}}%
\pgfpathlineto{\pgfqpoint{5.288160in}{6.914101in}}%
\pgfpathlineto{\pgfqpoint{5.289280in}{6.904979in}}%
\pgfpathlineto{\pgfqpoint{5.290401in}{6.891127in}}%
\pgfpathlineto{\pgfqpoint{5.291521in}{6.899573in}}%
\pgfpathlineto{\pgfqpoint{5.292642in}{6.885721in}}%
\pgfpathlineto{\pgfqpoint{5.296003in}{6.876261in}}%
\pgfpathlineto{\pgfqpoint{5.297124in}{6.865788in}}%
\pgfpathlineto{\pgfqpoint{5.299365in}{6.909033in}}%
\pgfpathlineto{\pgfqpoint{5.300486in}{6.889775in}}%
\pgfpathlineto{\pgfqpoint{5.304968in}{6.921196in}}%
\pgfpathlineto{\pgfqpoint{5.306088in}{6.921196in}}%
\pgfpathlineto{\pgfqpoint{5.308329in}{6.924574in}}%
\pgfpathlineto{\pgfqpoint{5.311691in}{6.913763in}}%
\pgfpathlineto{\pgfqpoint{5.313932in}{6.887410in}}%
\pgfpathlineto{\pgfqpoint{5.316173in}{6.889437in}}%
\pgfpathlineto{\pgfqpoint{5.319534in}{6.877275in}}%
\pgfpathlineto{\pgfqpoint{5.320655in}{6.857341in}}%
\pgfpathlineto{\pgfqpoint{5.322896in}{6.899235in}}%
\pgfpathlineto{\pgfqpoint{5.324017in}{6.901262in}}%
\pgfpathlineto{\pgfqpoint{5.327378in}{6.896532in}}%
\pgfpathlineto{\pgfqpoint{5.329619in}{6.889100in}}%
\pgfpathlineto{\pgfqpoint{5.330740in}{6.884370in}}%
\pgfpathlineto{\pgfqpoint{5.331860in}{6.892140in}}%
\pgfpathlineto{\pgfqpoint{5.336342in}{6.881329in}}%
\pgfpathlineto{\pgfqpoint{5.338583in}{6.905654in}}%
\pgfpathlineto{\pgfqpoint{5.339704in}{6.892140in}}%
\pgfpathlineto{\pgfqpoint{5.343066in}{6.873221in}}%
\pgfpathlineto{\pgfqpoint{5.344186in}{6.825245in}}%
\pgfpathlineto{\pgfqpoint{5.345307in}{6.813083in}}%
\pgfpathlineto{\pgfqpoint{5.346427in}{6.825921in}}%
\pgfpathlineto{\pgfqpoint{5.347548in}{6.791460in}}%
\pgfpathlineto{\pgfqpoint{5.350909in}{6.809366in}}%
\pgfpathlineto{\pgfqpoint{5.352030in}{6.810718in}}%
\pgfpathlineto{\pgfqpoint{5.353150in}{6.815110in}}%
\pgfpathlineto{\pgfqpoint{5.354271in}{6.825245in}}%
\pgfpathlineto{\pgfqpoint{5.355391in}{6.805988in}}%
\pgfpathlineto{\pgfqpoint{5.358753in}{6.795177in}}%
\pgfpathlineto{\pgfqpoint{5.359873in}{6.817813in}}%
\pgfpathlineto{\pgfqpoint{5.360994in}{6.813421in}}%
\pgfpathlineto{\pgfqpoint{5.362115in}{6.830651in}}%
\pgfpathlineto{\pgfqpoint{5.363235in}{6.837746in}}%
\pgfpathlineto{\pgfqpoint{5.367717in}{6.848895in}}%
\pgfpathlineto{\pgfqpoint{5.368838in}{6.835043in}}%
\pgfpathlineto{\pgfqpoint{5.369958in}{6.833016in}}%
\pgfpathlineto{\pgfqpoint{5.371079in}{6.839097in}}%
\pgfpathlineto{\pgfqpoint{5.374440in}{6.819840in}}%
\pgfpathlineto{\pgfqpoint{5.375561in}{6.839097in}}%
\pgfpathlineto{\pgfqpoint{5.377802in}{6.816461in}}%
\pgfpathlineto{\pgfqpoint{5.378922in}{6.804637in}}%
\pgfpathlineto{\pgfqpoint{5.382284in}{6.828624in}}%
\pgfpathlineto{\pgfqpoint{5.383405in}{6.829975in}}%
\pgfpathlineto{\pgfqpoint{5.384525in}{6.830313in}}%
\pgfpathlineto{\pgfqpoint{5.386766in}{6.802947in}}%
\pgfpathlineto{\pgfqpoint{5.390128in}{6.788757in}}%
\pgfpathlineto{\pgfqpoint{5.391248in}{6.758351in}}%
\pgfpathlineto{\pgfqpoint{5.392369in}{6.777270in}}%
\pgfpathlineto{\pgfqpoint{5.393489in}{6.731660in}}%
\pgfpathlineto{\pgfqpoint{5.394610in}{6.735377in}}%
\pgfpathlineto{\pgfqpoint{5.397971in}{6.732674in}}%
\pgfpathlineto{\pgfqpoint{5.399092in}{6.725579in}}%
\pgfpathlineto{\pgfqpoint{5.400212in}{6.734363in}}%
\pgfpathlineto{\pgfqpoint{5.401333in}{6.729971in}}%
\pgfpathlineto{\pgfqpoint{5.402453in}{6.746864in}}%
\pgfpathlineto{\pgfqpoint{5.405815in}{6.743485in}}%
\pgfpathlineto{\pgfqpoint{5.406936in}{6.731323in}}%
\pgfpathlineto{\pgfqpoint{5.408056in}{6.704632in}}%
\pgfpathlineto{\pgfqpoint{5.409177in}{6.710376in}}%
\pgfpathlineto{\pgfqpoint{5.410297in}{6.767473in}}%
\pgfpathlineto{\pgfqpoint{5.414779in}{6.745512in}}%
\pgfpathlineto{\pgfqpoint{5.415900in}{6.731998in}}%
\pgfpathlineto{\pgfqpoint{5.417020in}{6.731998in}}%
\pgfpathlineto{\pgfqpoint{5.421502in}{6.738755in}}%
\pgfpathlineto{\pgfqpoint{5.422623in}{6.745512in}}%
\pgfpathlineto{\pgfqpoint{5.423744in}{6.746864in}}%
\pgfpathlineto{\pgfqpoint{5.424864in}{6.744837in}}%
\pgfpathlineto{\pgfqpoint{5.425985in}{6.765446in}}%
\pgfpathlineto{\pgfqpoint{5.429346in}{6.759364in}}%
\pgfpathlineto{\pgfqpoint{5.430467in}{6.752269in}}%
\pgfpathlineto{\pgfqpoint{5.431587in}{6.792136in}}%
\pgfpathlineto{\pgfqpoint{5.432708in}{6.793150in}}%
\pgfpathlineto{\pgfqpoint{5.433828in}{6.781325in}}%
\pgfpathlineto{\pgfqpoint{5.437190in}{6.789095in}}%
\pgfpathlineto{\pgfqpoint{5.438310in}{6.780311in}}%
\pgfpathlineto{\pgfqpoint{5.439431in}{6.788082in}}%
\pgfpathlineto{\pgfqpoint{5.441672in}{6.769838in}}%
\pgfpathlineto{\pgfqpoint{5.445034in}{6.782338in}}%
\pgfpathlineto{\pgfqpoint{5.446154in}{6.797879in}}%
\pgfpathlineto{\pgfqpoint{5.447275in}{6.793825in}}%
\pgfpathlineto{\pgfqpoint{5.448395in}{6.783690in}}%
\pgfpathlineto{\pgfqpoint{5.449516in}{6.809704in}}%
\pgfpathlineto{\pgfqpoint{5.452877in}{6.810042in}}%
\pgfpathlineto{\pgfqpoint{5.455118in}{6.780987in}}%
\pgfpathlineto{\pgfqpoint{5.456239in}{6.781325in}}%
\pgfpathlineto{\pgfqpoint{5.457359in}{6.798217in}}%
\pgfpathlineto{\pgfqpoint{5.460721in}{6.794839in}}%
\pgfpathlineto{\pgfqpoint{5.461841in}{6.781663in}}%
\pgfpathlineto{\pgfqpoint{5.464082in}{6.803285in}}%
\pgfpathlineto{\pgfqpoint{5.465203in}{6.803961in}}%
\pgfpathlineto{\pgfqpoint{5.468565in}{6.816461in}}%
\pgfpathlineto{\pgfqpoint{5.469685in}{6.808691in}}%
\pgfpathlineto{\pgfqpoint{5.471926in}{6.820853in}}%
\pgfpathlineto{\pgfqpoint{5.473047in}{6.817813in}}%
\pgfpathlineto{\pgfqpoint{5.477529in}{6.807339in}}%
\pgfpathlineto{\pgfqpoint{5.478649in}{6.825583in}}%
\pgfpathlineto{\pgfqpoint{5.479770in}{6.834368in}}%
\pgfpathlineto{\pgfqpoint{5.480890in}{6.847882in}}%
\pgfpathlineto{\pgfqpoint{5.484252in}{6.831665in}}%
\pgfpathlineto{\pgfqpoint{5.486493in}{6.796190in}}%
\pgfpathlineto{\pgfqpoint{5.488734in}{6.769500in}}%
\pgfpathlineto{\pgfqpoint{5.492096in}{6.753283in}}%
\pgfpathlineto{\pgfqpoint{5.493216in}{6.751932in}}%
\pgfpathlineto{\pgfqpoint{5.494337in}{6.768824in}}%
\pgfpathlineto{\pgfqpoint{5.495457in}{6.769838in}}%
\pgfpathlineto{\pgfqpoint{5.496578in}{6.753959in}}%
\pgfpathlineto{\pgfqpoint{5.499939in}{6.755986in}}%
\pgfpathlineto{\pgfqpoint{5.502180in}{6.772878in}}%
\pgfpathlineto{\pgfqpoint{5.503301in}{6.785717in}}%
\pgfpathlineto{\pgfqpoint{5.504421in}{6.776257in}}%
\pgfpathlineto{\pgfqpoint{5.507783in}{6.782000in}}%
\pgfpathlineto{\pgfqpoint{5.510024in}{6.771527in}}%
\pgfpathlineto{\pgfqpoint{5.511145in}{6.773892in}}%
\pgfpathlineto{\pgfqpoint{5.512265in}{6.744837in}}%
\pgfpathlineto{\pgfqpoint{5.515627in}{6.725917in}}%
\pgfpathlineto{\pgfqpoint{5.516747in}{6.726931in}}%
\pgfpathlineto{\pgfqpoint{5.517868in}{6.719836in}}%
\pgfpathlineto{\pgfqpoint{5.518988in}{6.730985in}}%
\pgfpathlineto{\pgfqpoint{5.524591in}{6.711389in}}%
\pgfpathlineto{\pgfqpoint{5.526832in}{6.682334in}}%
\pgfpathlineto{\pgfqpoint{5.527953in}{6.689429in}}%
\pgfpathlineto{\pgfqpoint{5.531314in}{6.706322in}}%
\pgfpathlineto{\pgfqpoint{5.532435in}{6.703957in}}%
\pgfpathlineto{\pgfqpoint{5.533555in}{6.704970in}}%
\pgfpathlineto{\pgfqpoint{5.534676in}{6.711389in}}%
\pgfpathlineto{\pgfqpoint{5.535796in}{6.698551in}}%
\pgfpathlineto{\pgfqpoint{5.539158in}{6.687402in}}%
\pgfpathlineto{\pgfqpoint{5.540278in}{6.675915in}}%
\pgfpathlineto{\pgfqpoint{5.541399in}{6.672536in}}%
\pgfpathlineto{\pgfqpoint{5.542519in}{6.672198in}}%
\pgfpathlineto{\pgfqpoint{5.543640in}{6.656319in}}%
\pgfpathlineto{\pgfqpoint{5.547001in}{6.664766in}}%
\pgfpathlineto{\pgfqpoint{5.548122in}{6.683010in}}%
\pgfpathlineto{\pgfqpoint{5.549243in}{6.684699in}}%
\pgfpathlineto{\pgfqpoint{5.550363in}{6.681658in}}%
\pgfpathlineto{\pgfqpoint{5.554845in}{6.685713in}}%
\pgfpathlineto{\pgfqpoint{5.555966in}{6.688078in}}%
\pgfpathlineto{\pgfqpoint{5.557086in}{6.695172in}}%
\pgfpathlineto{\pgfqpoint{5.559327in}{6.688078in}}%
\pgfpathlineto{\pgfqpoint{5.562689in}{6.711051in}}%
\pgfpathlineto{\pgfqpoint{5.563809in}{6.690780in}}%
\pgfpathlineto{\pgfqpoint{5.564930in}{6.705308in}}%
\pgfpathlineto{\pgfqpoint{5.566050in}{6.687740in}}%
\pgfpathlineto{\pgfqpoint{5.567171in}{6.692132in}}%
\pgfpathlineto{\pgfqpoint{5.570533in}{6.693821in}}%
\pgfpathlineto{\pgfqpoint{5.571653in}{6.688753in}}%
\pgfpathlineto{\pgfqpoint{5.572774in}{6.672198in}}%
\pgfpathlineto{\pgfqpoint{5.575015in}{6.617804in}}%
\pgfpathlineto{\pgfqpoint{5.578376in}{6.608344in}}%
\pgfpathlineto{\pgfqpoint{5.579497in}{6.596857in}}%
\pgfpathlineto{\pgfqpoint{5.580617in}{6.640102in}}%
\pgfpathlineto{\pgfqpoint{5.581738in}{6.652941in}}%
\pgfpathlineto{\pgfqpoint{5.582858in}{6.673888in}}%
\pgfpathlineto{\pgfqpoint{5.586220in}{6.677604in}}%
\pgfpathlineto{\pgfqpoint{5.587340in}{6.655982in}}%
\pgfpathlineto{\pgfqpoint{5.589582in}{6.693821in}}%
\pgfpathlineto{\pgfqpoint{5.590702in}{6.676928in}}%
\pgfpathlineto{\pgfqpoint{5.595184in}{6.706659in}}%
\pgfpathlineto{\pgfqpoint{5.596305in}{6.698889in}}%
\pgfpathlineto{\pgfqpoint{5.597425in}{6.699564in}}%
\pgfpathlineto{\pgfqpoint{5.598546in}{6.705646in}}%
\pgfpathlineto{\pgfqpoint{5.601907in}{6.703281in}}%
\pgfpathlineto{\pgfqpoint{5.603028in}{6.713754in}}%
\pgfpathlineto{\pgfqpoint{5.604148in}{6.714768in}}%
\pgfpathlineto{\pgfqpoint{5.605269in}{6.713079in}}%
\pgfpathlineto{\pgfqpoint{5.606389in}{6.692132in}}%
\pgfpathlineto{\pgfqpoint{5.609751in}{6.696524in}}%
\pgfpathlineto{\pgfqpoint{5.610872in}{6.681658in}}%
\pgfpathlineto{\pgfqpoint{5.611992in}{6.683685in}}%
\pgfpathlineto{\pgfqpoint{5.613113in}{6.675915in}}%
\pgfpathlineto{\pgfqpoint{5.614233in}{6.685713in}}%
\pgfpathlineto{\pgfqpoint{5.617595in}{6.684361in}}%
\pgfpathlineto{\pgfqpoint{5.618715in}{6.698889in}}%
\pgfpathlineto{\pgfqpoint{5.619836in}{6.725917in}}%
\pgfpathlineto{\pgfqpoint{5.620956in}{6.721863in}}%
\pgfpathlineto{\pgfqpoint{5.622077in}{6.737066in}}%
\pgfpathlineto{\pgfqpoint{5.625438in}{6.758351in}}%
\pgfpathlineto{\pgfqpoint{5.627679in}{6.791798in}}%
\pgfpathlineto{\pgfqpoint{5.628800in}{6.797879in}}%
\pgfpathlineto{\pgfqpoint{5.629921in}{6.786392in}}%
\pgfpathlineto{\pgfqpoint{5.633282in}{6.788420in}}%
\pgfpathlineto{\pgfqpoint{5.634403in}{6.783352in}}%
\pgfpathlineto{\pgfqpoint{5.635523in}{6.806326in}}%
\pgfpathlineto{\pgfqpoint{5.636644in}{6.804974in}}%
\pgfpathlineto{\pgfqpoint{5.637764in}{6.813759in}}%
\pgfpathlineto{\pgfqpoint{5.641126in}{6.830313in}}%
\pgfpathlineto{\pgfqpoint{5.642246in}{6.825583in}}%
\pgfpathlineto{\pgfqpoint{5.643367in}{6.823556in}}%
\pgfpathlineto{\pgfqpoint{5.644487in}{6.854639in}}%
\pgfpathlineto{\pgfqpoint{5.645608in}{6.869842in}}%
\pgfpathlineto{\pgfqpoint{5.650090in}{6.857004in}}%
\pgfpathlineto{\pgfqpoint{5.651211in}{6.864099in}}%
\pgfpathlineto{\pgfqpoint{5.652331in}{6.843490in}}%
\pgfpathlineto{\pgfqpoint{5.653452in}{6.838422in}}%
\pgfpathlineto{\pgfqpoint{5.656813in}{6.845854in}}%
\pgfpathlineto{\pgfqpoint{5.657934in}{6.851936in}}%
\pgfpathlineto{\pgfqpoint{5.659054in}{6.854639in}}%
\pgfpathlineto{\pgfqpoint{5.665777in}{6.825921in}}%
\pgfpathlineto{\pgfqpoint{5.669139in}{6.792474in}}%
\pgfpathlineto{\pgfqpoint{5.672501in}{6.792136in}}%
\pgfpathlineto{\pgfqpoint{5.674742in}{6.824232in}}%
\pgfpathlineto{\pgfqpoint{5.675862in}{6.859031in}}%
\pgfpathlineto{\pgfqpoint{5.676983in}{6.869842in}}%
\pgfpathlineto{\pgfqpoint{5.680344in}{6.864436in}}%
\pgfpathlineto{\pgfqpoint{5.681465in}{6.860720in}}%
\pgfpathlineto{\pgfqpoint{5.682585in}{6.863761in}}%
\pgfpathlineto{\pgfqpoint{5.684826in}{6.863761in}}%
\pgfpathlineto{\pgfqpoint{5.688188in}{6.873221in}}%
\pgfpathlineto{\pgfqpoint{5.689308in}{6.883018in}}%
\pgfpathlineto{\pgfqpoint{5.690429in}{6.875248in}}%
\pgfpathlineto{\pgfqpoint{5.691549in}{6.851260in}}%
\pgfpathlineto{\pgfqpoint{5.692670in}{6.878626in}}%
\pgfpathlineto{\pgfqpoint{5.696032in}{6.879978in}}%
\pgfpathlineto{\pgfqpoint{5.697152in}{6.872883in}}%
\pgfpathlineto{\pgfqpoint{5.698273in}{6.874572in}}%
\pgfpathlineto{\pgfqpoint{5.699393in}{6.873221in}}%
\pgfpathlineto{\pgfqpoint{5.700514in}{6.858017in}}%
\pgfpathlineto{\pgfqpoint{5.703875in}{6.864099in}}%
\pgfpathlineto{\pgfqpoint{5.704996in}{6.885721in}}%
\pgfpathlineto{\pgfqpoint{5.706116in}{6.889437in}}%
\pgfpathlineto{\pgfqpoint{5.707237in}{6.877613in}}%
\pgfpathlineto{\pgfqpoint{5.708357in}{6.845854in}}%
\pgfpathlineto{\pgfqpoint{5.711719in}{6.857341in}}%
\pgfpathlineto{\pgfqpoint{5.713960in}{6.880315in}}%
\pgfpathlineto{\pgfqpoint{5.715081in}{6.879640in}}%
\pgfpathlineto{\pgfqpoint{5.719563in}{6.878288in}}%
\pgfpathlineto{\pgfqpoint{5.720683in}{6.893829in}}%
\pgfpathlineto{\pgfqpoint{5.722924in}{6.863761in}}%
\pgfpathlineto{\pgfqpoint{5.728527in}{6.844841in}}%
\pgfpathlineto{\pgfqpoint{5.729647in}{6.821867in}}%
\pgfpathlineto{\pgfqpoint{5.730768in}{6.784028in}}%
\pgfpathlineto{\pgfqpoint{5.731888in}{6.774230in}}%
\pgfpathlineto{\pgfqpoint{5.735250in}{6.790785in}}%
\pgfpathlineto{\pgfqpoint{5.736371in}{6.809704in}}%
\pgfpathlineto{\pgfqpoint{5.737491in}{6.786392in}}%
\pgfpathlineto{\pgfqpoint{5.738612in}{6.811394in}}%
\pgfpathlineto{\pgfqpoint{5.739732in}{6.720849in}}%
\pgfpathlineto{\pgfqpoint{5.744214in}{6.722201in}}%
\pgfpathlineto{\pgfqpoint{5.745335in}{6.715781in}}%
\pgfpathlineto{\pgfqpoint{5.746455in}{6.717808in}}%
\pgfpathlineto{\pgfqpoint{5.747576in}{6.725917in}}%
\pgfpathlineto{\pgfqpoint{5.750937in}{6.716119in}}%
\pgfpathlineto{\pgfqpoint{5.752058in}{6.726255in}}%
\pgfpathlineto{\pgfqpoint{5.753178in}{6.722538in}}%
\pgfpathlineto{\pgfqpoint{5.754299in}{6.727268in}}%
\pgfpathlineto{\pgfqpoint{5.755420in}{6.759364in}}%
\pgfpathlineto{\pgfqpoint{5.758781in}{6.753283in}}%
\pgfpathlineto{\pgfqpoint{5.759902in}{6.722201in}}%
\pgfpathlineto{\pgfqpoint{5.761022in}{6.715781in}}%
\pgfpathlineto{\pgfqpoint{5.762143in}{6.729295in}}%
\pgfpathlineto{\pgfqpoint{5.763263in}{6.706659in}}%
\pgfpathlineto{\pgfqpoint{5.766625in}{6.699902in}}%
\pgfpathlineto{\pgfqpoint{5.767745in}{6.699564in}}%
\pgfpathlineto{\pgfqpoint{5.768866in}{6.681996in}}%
\pgfpathlineto{\pgfqpoint{5.769986in}{6.681658in}}%
\pgfpathlineto{\pgfqpoint{5.771107in}{6.694497in}}%
\pgfpathlineto{\pgfqpoint{5.775589in}{6.698889in}}%
\pgfpathlineto{\pgfqpoint{5.776710in}{6.719836in}}%
\pgfpathlineto{\pgfqpoint{5.777830in}{6.718484in}}%
\pgfpathlineto{\pgfqpoint{5.778951in}{6.696524in}}%
\pgfpathlineto{\pgfqpoint{5.782312in}{6.716119in}}%
\pgfpathlineto{\pgfqpoint{5.783433in}{6.699227in}}%
\pgfpathlineto{\pgfqpoint{5.786794in}{6.729971in}}%
\pgfpathlineto{\pgfqpoint{5.790156in}{6.723552in}}%
\pgfpathlineto{\pgfqpoint{5.791276in}{6.747539in}}%
\pgfpathlineto{\pgfqpoint{5.792397in}{6.752945in}}%
\pgfpathlineto{\pgfqpoint{5.794638in}{6.755648in}}%
\pgfpathlineto{\pgfqpoint{5.798000in}{6.765108in}}%
\pgfpathlineto{\pgfqpoint{5.799120in}{6.753283in}}%
\pgfpathlineto{\pgfqpoint{5.801361in}{6.774568in}}%
\pgfpathlineto{\pgfqpoint{5.802482in}{6.790109in}}%
\pgfpathlineto{\pgfqpoint{5.805843in}{6.779973in}}%
\pgfpathlineto{\pgfqpoint{5.806964in}{6.786730in}}%
\pgfpathlineto{\pgfqpoint{5.808084in}{6.788082in}}%
\pgfpathlineto{\pgfqpoint{5.809205in}{6.796866in}}%
\pgfpathlineto{\pgfqpoint{5.810325in}{6.818488in}}%
\pgfpathlineto{\pgfqpoint{5.813687in}{6.808015in}}%
\pgfpathlineto{\pgfqpoint{5.814807in}{6.807339in}}%
\pgfpathlineto{\pgfqpoint{5.815928in}{6.797542in}}%
\pgfpathlineto{\pgfqpoint{5.817049in}{6.793825in}}%
\pgfpathlineto{\pgfqpoint{5.821531in}{6.794501in}}%
\pgfpathlineto{\pgfqpoint{5.823772in}{6.819502in}}%
\pgfpathlineto{\pgfqpoint{5.824892in}{6.808353in}}%
\pgfpathlineto{\pgfqpoint{5.826013in}{6.811394in}}%
\pgfpathlineto{\pgfqpoint{5.830495in}{6.794501in}}%
\pgfpathlineto{\pgfqpoint{5.831615in}{6.799907in}}%
\pgfpathlineto{\pgfqpoint{5.832736in}{6.783690in}}%
\pgfpathlineto{\pgfqpoint{5.833856in}{6.786392in}}%
\pgfpathlineto{\pgfqpoint{5.837218in}{6.787406in}}%
\pgfpathlineto{\pgfqpoint{5.839459in}{6.801596in}}%
\pgfpathlineto{\pgfqpoint{5.841700in}{6.780987in}}%
\pgfpathlineto{\pgfqpoint{5.845062in}{6.786730in}}%
\pgfpathlineto{\pgfqpoint{5.846182in}{6.785379in}}%
\pgfpathlineto{\pgfqpoint{5.847303in}{6.797542in}}%
\pgfpathlineto{\pgfqpoint{5.848423in}{6.796528in}}%
\pgfpathlineto{\pgfqpoint{5.849544in}{6.786392in}}%
\pgfpathlineto{\pgfqpoint{5.852905in}{6.778960in}}%
\pgfpathlineto{\pgfqpoint{5.854026in}{6.779298in}}%
\pgfpathlineto{\pgfqpoint{5.855146in}{6.789771in}}%
\pgfpathlineto{\pgfqpoint{5.857388in}{6.744837in}}%
\pgfpathlineto{\pgfqpoint{5.860749in}{6.754972in}}%
\pgfpathlineto{\pgfqpoint{5.862990in}{6.739431in}}%
\pgfpathlineto{\pgfqpoint{5.864111in}{6.741120in}}%
\pgfpathlineto{\pgfqpoint{5.865231in}{6.745512in}}%
\pgfpathlineto{\pgfqpoint{5.868593in}{6.738080in}}%
\pgfpathlineto{\pgfqpoint{5.869713in}{6.748553in}}%
\pgfpathlineto{\pgfqpoint{5.870834in}{6.745850in}}%
\pgfpathlineto{\pgfqpoint{5.871954in}{6.736728in}}%
\pgfpathlineto{\pgfqpoint{5.876436in}{6.756324in}}%
\pgfpathlineto{\pgfqpoint{5.877557in}{6.743485in}}%
\pgfpathlineto{\pgfqpoint{5.878678in}{6.743823in}}%
\pgfpathlineto{\pgfqpoint{5.879798in}{6.732674in}}%
\pgfpathlineto{\pgfqpoint{5.880919in}{6.748891in}}%
\pgfpathlineto{\pgfqpoint{5.884280in}{6.751256in}}%
\pgfpathlineto{\pgfqpoint{5.885401in}{6.776933in}}%
\pgfpathlineto{\pgfqpoint{5.886521in}{6.787068in}}%
\pgfpathlineto{\pgfqpoint{5.888762in}{6.792812in}}%
\pgfpathlineto{\pgfqpoint{5.893244in}{6.793487in}}%
\pgfpathlineto{\pgfqpoint{5.895485in}{6.798555in}}%
\pgfpathlineto{\pgfqpoint{5.896606in}{6.794163in}}%
\pgfpathlineto{\pgfqpoint{5.899968in}{6.796190in}}%
\pgfpathlineto{\pgfqpoint{5.901088in}{6.802272in}}%
\pgfpathlineto{\pgfqpoint{5.902209in}{6.802609in}}%
\pgfpathlineto{\pgfqpoint{5.903329in}{6.804299in}}%
\pgfpathlineto{\pgfqpoint{5.904450in}{6.807339in}}%
\pgfpathlineto{\pgfqpoint{5.907811in}{6.811394in}}%
\pgfpathlineto{\pgfqpoint{5.908932in}{6.810380in}}%
\pgfpathlineto{\pgfqpoint{5.910052in}{6.794163in}}%
\pgfpathlineto{\pgfqpoint{5.912293in}{6.798555in}}%
\pgfpathlineto{\pgfqpoint{5.916775in}{6.816123in}}%
\pgfpathlineto{\pgfqpoint{5.917896in}{6.815110in}}%
\pgfpathlineto{\pgfqpoint{5.919017in}{6.836732in}}%
\pgfpathlineto{\pgfqpoint{5.920137in}{6.792136in}}%
\pgfpathlineto{\pgfqpoint{5.923499in}{6.766459in}}%
\pgfpathlineto{\pgfqpoint{5.924619in}{6.780987in}}%
\pgfpathlineto{\pgfqpoint{5.926860in}{6.830989in}}%
\pgfpathlineto{\pgfqpoint{5.927981in}{6.829300in}}%
\pgfpathlineto{\pgfqpoint{5.932463in}{6.827273in}}%
\pgfpathlineto{\pgfqpoint{5.934704in}{6.843152in}}%
\pgfpathlineto{\pgfqpoint{5.935824in}{6.868153in}}%
\pgfpathlineto{\pgfqpoint{5.939186in}{6.879640in}}%
\pgfpathlineto{\pgfqpoint{5.940307in}{6.897208in}}%
\pgfpathlineto{\pgfqpoint{5.941427in}{6.899235in}}%
\pgfpathlineto{\pgfqpoint{5.942548in}{6.905316in}}%
\pgfpathlineto{\pgfqpoint{5.943668in}{6.901262in}}%
\pgfpathlineto{\pgfqpoint{5.947030in}{6.900587in}}%
\pgfpathlineto{\pgfqpoint{5.948150in}{6.903627in}}%
\pgfpathlineto{\pgfqpoint{5.949271in}{6.920520in}}%
\pgfpathlineto{\pgfqpoint{5.950391in}{6.876261in}}%
\pgfpathlineto{\pgfqpoint{5.951512in}{6.888424in}}%
\pgfpathlineto{\pgfqpoint{5.954873in}{6.889437in}}%
\pgfpathlineto{\pgfqpoint{5.955994in}{6.901938in}}%
\pgfpathlineto{\pgfqpoint{5.957114in}{6.893829in}}%
\pgfpathlineto{\pgfqpoint{5.958235in}{6.891802in}}%
\pgfpathlineto{\pgfqpoint{5.959355in}{6.894505in}}%
\pgfpathlineto{\pgfqpoint{5.962717in}{6.894505in}}%
\pgfpathlineto{\pgfqpoint{5.963838in}{6.885383in}}%
\pgfpathlineto{\pgfqpoint{5.964958in}{6.883694in}}%
\pgfpathlineto{\pgfqpoint{5.967199in}{6.906668in}}%
\pgfpathlineto{\pgfqpoint{5.970561in}{6.908357in}}%
\pgfpathlineto{\pgfqpoint{5.971681in}{6.904641in}}%
\pgfpathlineto{\pgfqpoint{5.972802in}{6.892478in}}%
\pgfpathlineto{\pgfqpoint{5.973922in}{6.897208in}}%
\pgfpathlineto{\pgfqpoint{5.975043in}{6.893829in}}%
\pgfpathlineto{\pgfqpoint{5.978404in}{6.904303in}}%
\pgfpathlineto{\pgfqpoint{5.979525in}{6.913763in}}%
\pgfpathlineto{\pgfqpoint{5.980646in}{6.907681in}}%
\pgfpathlineto{\pgfqpoint{5.981766in}{6.906330in}}%
\pgfpathlineto{\pgfqpoint{5.982887in}{6.914776in}}%
\pgfpathlineto{\pgfqpoint{5.987369in}{6.919844in}}%
\pgfpathlineto{\pgfqpoint{5.988489in}{6.912074in}}%
\pgfpathlineto{\pgfqpoint{5.989610in}{6.910046in}}%
\pgfpathlineto{\pgfqpoint{5.990730in}{6.915452in}}%
\pgfpathlineto{\pgfqpoint{5.994092in}{6.924236in}}%
\pgfpathlineto{\pgfqpoint{5.998574in}{6.940791in}}%
\pgfpathlineto{\pgfqpoint{6.003056in}{6.956332in}}%
\pgfpathlineto{\pgfqpoint{6.004177in}{6.952616in}}%
\pgfpathlineto{\pgfqpoint{6.005297in}{6.952278in}}%
\pgfpathlineto{\pgfqpoint{6.006418in}{6.920858in}}%
\pgfpathlineto{\pgfqpoint{6.009779in}{6.940791in}}%
\pgfpathlineto{\pgfqpoint{6.010900in}{6.926263in}}%
\pgfpathlineto{\pgfqpoint{6.012020in}{6.926601in}}%
\pgfpathlineto{\pgfqpoint{6.014261in}{6.990455in}}%
\pgfpathlineto{\pgfqpoint{6.017623in}{6.974576in}}%
\pgfpathlineto{\pgfqpoint{6.018743in}{6.973900in}}%
\pgfpathlineto{\pgfqpoint{6.019864in}{6.983698in}}%
\pgfpathlineto{\pgfqpoint{6.020984in}{6.986739in}}%
\pgfpathlineto{\pgfqpoint{6.022105in}{6.975590in}}%
\pgfpathlineto{\pgfqpoint{6.025467in}{6.958697in}}%
\pgfpathlineto{\pgfqpoint{6.026587in}{6.975252in}}%
\pgfpathlineto{\pgfqpoint{6.027708in}{6.983360in}}%
\pgfpathlineto{\pgfqpoint{6.028828in}{6.979644in}}%
\pgfpathlineto{\pgfqpoint{6.029949in}{6.993158in}}%
\pgfpathlineto{\pgfqpoint{6.033310in}{6.990117in}}%
\pgfpathlineto{\pgfqpoint{6.034431in}{6.986401in}}%
\pgfpathlineto{\pgfqpoint{6.035551in}{7.000591in}}%
\pgfpathlineto{\pgfqpoint{6.036672in}{7.002956in}}%
\pgfpathlineto{\pgfqpoint{6.037792in}{7.003969in}}%
\pgfpathlineto{\pgfqpoint{6.041154in}{7.001604in}}%
\pgfpathlineto{\pgfqpoint{6.042274in}{6.977955in}}%
\pgfpathlineto{\pgfqpoint{6.044516in}{6.968833in}}%
\pgfpathlineto{\pgfqpoint{6.045636in}{6.983698in}}%
\pgfpathlineto{\pgfqpoint{6.048998in}{6.978630in}}%
\pgfpathlineto{\pgfqpoint{6.050118in}{6.993158in}}%
\pgfpathlineto{\pgfqpoint{6.051239in}{6.923223in}}%
\pgfpathlineto{\pgfqpoint{6.052359in}{6.920520in}}%
\pgfpathlineto{\pgfqpoint{6.053480in}{6.912074in}}%
\pgfpathlineto{\pgfqpoint{6.056841in}{6.915452in}}%
\pgfpathlineto{\pgfqpoint{6.060203in}{6.901262in}}%
\pgfpathlineto{\pgfqpoint{6.061323in}{6.899235in}}%
\pgfpathlineto{\pgfqpoint{6.064685in}{6.903289in}}%
\pgfpathlineto{\pgfqpoint{6.065806in}{6.892140in}}%
\pgfpathlineto{\pgfqpoint{6.066926in}{6.894843in}}%
\pgfpathlineto{\pgfqpoint{6.069167in}{6.871869in}}%
\pgfpathlineto{\pgfqpoint{6.072529in}{6.905654in}}%
\pgfpathlineto{\pgfqpoint{6.073649in}{6.907344in}}%
\pgfpathlineto{\pgfqpoint{6.074770in}{6.907681in}}%
\pgfpathlineto{\pgfqpoint{6.075890in}{6.899911in}}%
\pgfpathlineto{\pgfqpoint{6.077011in}{6.903289in}}%
\pgfpathlineto{\pgfqpoint{6.080372in}{6.899235in}}%
\pgfpathlineto{\pgfqpoint{6.081493in}{6.912749in}}%
\pgfpathlineto{\pgfqpoint{6.082613in}{6.910384in}}%
\pgfpathlineto{\pgfqpoint{6.083734in}{6.916128in}}%
\pgfpathlineto{\pgfqpoint{6.084855in}{6.913763in}}%
\pgfpathlineto{\pgfqpoint{6.088216in}{6.914776in}}%
\pgfpathlineto{\pgfqpoint{6.089337in}{6.930655in}}%
\pgfpathlineto{\pgfqpoint{6.090457in}{6.921871in}}%
\pgfpathlineto{\pgfqpoint{6.092698in}{6.929304in}}%
\pgfpathlineto{\pgfqpoint{6.096060in}{6.931669in}}%
\pgfpathlineto{\pgfqpoint{6.097180in}{6.925250in}}%
\pgfpathlineto{\pgfqpoint{6.098301in}{6.905992in}}%
\pgfpathlineto{\pgfqpoint{6.099421in}{6.876599in}}%
\pgfpathlineto{\pgfqpoint{6.100542in}{6.889100in}}%
\pgfpathlineto{\pgfqpoint{6.103903in}{6.896194in}}%
\pgfpathlineto{\pgfqpoint{6.105024in}{6.906668in}}%
\pgfpathlineto{\pgfqpoint{6.106145in}{6.931331in}}%
\pgfpathlineto{\pgfqpoint{6.107265in}{6.937412in}}%
\pgfpathlineto{\pgfqpoint{6.111747in}{6.945859in}}%
\pgfpathlineto{\pgfqpoint{6.112868in}{6.972211in}}%
\pgfpathlineto{\pgfqpoint{6.113988in}{6.964103in}}%
\pgfpathlineto{\pgfqpoint{6.115109in}{6.971873in}}%
\pgfpathlineto{\pgfqpoint{6.116229in}{6.956670in}}%
\pgfpathlineto{\pgfqpoint{6.119591in}{6.974914in}}%
\pgfpathlineto{\pgfqpoint{6.120711in}{6.985050in}}%
\pgfpathlineto{\pgfqpoint{6.121832in}{6.977617in}}%
\pgfpathlineto{\pgfqpoint{6.122952in}{6.976265in}}%
\pgfpathlineto{\pgfqpoint{6.128555in}{6.980658in}}%
\pgfpathlineto{\pgfqpoint{6.129676in}{6.966806in}}%
\pgfpathlineto{\pgfqpoint{6.130796in}{6.967819in}}%
\pgfpathlineto{\pgfqpoint{6.131917in}{6.955319in}}%
\pgfpathlineto{\pgfqpoint{6.136399in}{6.965792in}}%
\pgfpathlineto{\pgfqpoint{6.137519in}{6.959711in}}%
\pgfpathlineto{\pgfqpoint{6.138640in}{6.958021in}}%
\pgfpathlineto{\pgfqpoint{6.139760in}{6.962076in}}%
\pgfpathlineto{\pgfqpoint{6.143122in}{6.966130in}}%
\pgfpathlineto{\pgfqpoint{6.144242in}{6.963765in}}%
\pgfpathlineto{\pgfqpoint{6.145363in}{6.976941in}}%
\pgfpathlineto{\pgfqpoint{6.146484in}{6.969171in}}%
\pgfpathlineto{\pgfqpoint{6.147604in}{6.971873in}}%
\pgfpathlineto{\pgfqpoint{6.152086in}{6.972211in}}%
\pgfpathlineto{\pgfqpoint{6.153207in}{6.970860in}}%
\pgfpathlineto{\pgfqpoint{6.154327in}{6.964778in}}%
\pgfpathlineto{\pgfqpoint{6.155448in}{6.976603in}}%
\pgfpathlineto{\pgfqpoint{6.158809in}{6.971198in}}%
\pgfpathlineto{\pgfqpoint{6.159930in}{6.997888in}}%
\pgfpathlineto{\pgfqpoint{6.161050in}{7.003631in}}%
\pgfpathlineto{\pgfqpoint{6.162171in}{6.995861in}}%
\pgfpathlineto{\pgfqpoint{6.163291in}{7.009037in}}%
\pgfpathlineto{\pgfqpoint{6.166653in}{6.991469in}}%
\pgfpathlineto{\pgfqpoint{6.167774in}{6.972549in}}%
\pgfpathlineto{\pgfqpoint{6.168894in}{6.963427in}}%
\pgfpathlineto{\pgfqpoint{6.171135in}{6.971536in}}%
\pgfpathlineto{\pgfqpoint{6.174497in}{6.963427in}}%
\pgfpathlineto{\pgfqpoint{6.175617in}{6.966130in}}%
\pgfpathlineto{\pgfqpoint{6.176738in}{6.967143in}}%
\pgfpathlineto{\pgfqpoint{6.177858in}{6.937750in}}%
\pgfpathlineto{\pgfqpoint{6.178979in}{6.934034in}}%
\pgfpathlineto{\pgfqpoint{6.184581in}{6.956670in}}%
\pgfpathlineto{\pgfqpoint{6.185702in}{6.968157in}}%
\pgfpathlineto{\pgfqpoint{6.186822in}{6.970184in}}%
\pgfpathlineto{\pgfqpoint{6.191305in}{6.971536in}}%
\pgfpathlineto{\pgfqpoint{6.192425in}{6.957346in}}%
\pgfpathlineto{\pgfqpoint{6.193546in}{6.960724in}}%
\pgfpathlineto{\pgfqpoint{6.194666in}{6.971873in}}%
\pgfpathlineto{\pgfqpoint{6.198028in}{6.971198in}}%
\pgfpathlineto{\pgfqpoint{6.200269in}{6.952954in}}%
\pgfpathlineto{\pgfqpoint{6.202510in}{6.951940in}}%
\pgfpathlineto{\pgfqpoint{6.205871in}{6.941467in}}%
\pgfpathlineto{\pgfqpoint{6.206992in}{6.948562in}}%
\pgfpathlineto{\pgfqpoint{6.208113in}{6.942818in}}%
\pgfpathlineto{\pgfqpoint{6.209233in}{6.949237in}}%
\pgfpathlineto{\pgfqpoint{6.210354in}{6.952278in}}%
\pgfpathlineto{\pgfqpoint{6.213715in}{6.928290in}}%
\pgfpathlineto{\pgfqpoint{6.214836in}{6.928966in}}%
\pgfpathlineto{\pgfqpoint{6.215956in}{6.926601in}}%
\pgfpathlineto{\pgfqpoint{6.217077in}{6.927615in}}%
\pgfpathlineto{\pgfqpoint{6.218197in}{6.932007in}}%
\pgfpathlineto{\pgfqpoint{6.221559in}{6.937075in}}%
\pgfpathlineto{\pgfqpoint{6.222679in}{6.924574in}}%
\pgfpathlineto{\pgfqpoint{6.223800in}{6.935047in}}%
\pgfpathlineto{\pgfqpoint{6.226041in}{6.928290in}}%
\pgfpathlineto{\pgfqpoint{6.229403in}{6.935723in}}%
\pgfpathlineto{\pgfqpoint{6.230523in}{6.942480in}}%
\pgfpathlineto{\pgfqpoint{6.231644in}{6.941467in}}%
\pgfpathlineto{\pgfqpoint{6.232764in}{6.947210in}}%
\pgfpathlineto{\pgfqpoint{6.233885in}{6.957346in}}%
\pgfpathlineto{\pgfqpoint{6.237246in}{6.960049in}}%
\pgfpathlineto{\pgfqpoint{6.238367in}{6.963765in}}%
\pgfpathlineto{\pgfqpoint{6.239487in}{6.962076in}}%
\pgfpathlineto{\pgfqpoint{6.240608in}{6.955994in}}%
\pgfpathlineto{\pgfqpoint{6.241728in}{6.955994in}}%
\pgfpathlineto{\pgfqpoint{6.247331in}{6.943156in}}%
\pgfpathlineto{\pgfqpoint{6.248451in}{6.931331in}}%
\pgfpathlineto{\pgfqpoint{6.252934in}{6.938426in}}%
\pgfpathlineto{\pgfqpoint{6.254054in}{6.947886in}}%
\pgfpathlineto{\pgfqpoint{6.255175in}{6.952278in}}%
\pgfpathlineto{\pgfqpoint{6.257416in}{6.965116in}}%
\pgfpathlineto{\pgfqpoint{6.261898in}{6.982685in}}%
\pgfpathlineto{\pgfqpoint{6.263018in}{6.984374in}}%
\pgfpathlineto{\pgfqpoint{6.264139in}{7.000253in}}%
\pgfpathlineto{\pgfqpoint{6.265259in}{6.959711in}}%
\pgfpathlineto{\pgfqpoint{6.268621in}{6.964778in}}%
\pgfpathlineto{\pgfqpoint{6.269742in}{6.985725in}}%
\pgfpathlineto{\pgfqpoint{6.270862in}{6.994847in}}%
\pgfpathlineto{\pgfqpoint{6.271983in}{6.990455in}}%
\pgfpathlineto{\pgfqpoint{6.273103in}{6.989780in}}%
\pgfpathlineto{\pgfqpoint{6.276465in}{6.980658in}}%
\pgfpathlineto{\pgfqpoint{6.277585in}{6.975252in}}%
\pgfpathlineto{\pgfqpoint{6.279826in}{6.953629in}}%
\pgfpathlineto{\pgfqpoint{6.280947in}{6.948562in}}%
\pgfpathlineto{\pgfqpoint{6.284308in}{6.951602in}}%
\pgfpathlineto{\pgfqpoint{6.285429in}{6.957684in}}%
\pgfpathlineto{\pgfqpoint{6.286549in}{6.933020in}}%
\pgfpathlineto{\pgfqpoint{6.288790in}{6.944507in}}%
\pgfpathlineto{\pgfqpoint{6.293273in}{6.959035in}}%
\pgfpathlineto{\pgfqpoint{6.295514in}{6.971873in}}%
\pgfpathlineto{\pgfqpoint{6.296634in}{6.971873in}}%
\pgfpathlineto{\pgfqpoint{6.301116in}{6.969171in}}%
\pgfpathlineto{\pgfqpoint{6.302237in}{6.967143in}}%
\pgfpathlineto{\pgfqpoint{6.303357in}{6.967481in}}%
\pgfpathlineto{\pgfqpoint{6.304478in}{6.973563in}}%
\pgfpathlineto{\pgfqpoint{6.307839in}{6.974238in}}%
\pgfpathlineto{\pgfqpoint{6.308960in}{6.967481in}}%
\pgfpathlineto{\pgfqpoint{6.310080in}{6.971873in}}%
\pgfpathlineto{\pgfqpoint{6.311201in}{6.978968in}}%
\pgfpathlineto{\pgfqpoint{6.312322in}{6.954305in}}%
\pgfpathlineto{\pgfqpoint{6.315683in}{6.954981in}}%
\pgfpathlineto{\pgfqpoint{6.316804in}{6.959711in}}%
\pgfpathlineto{\pgfqpoint{6.319045in}{6.941467in}}%
\pgfpathlineto{\pgfqpoint{6.320165in}{6.938426in}}%
\pgfpathlineto{\pgfqpoint{6.323527in}{6.947886in}}%
\pgfpathlineto{\pgfqpoint{6.324647in}{6.927277in}}%
\pgfpathlineto{\pgfqpoint{6.326888in}{6.911060in}}%
\pgfpathlineto{\pgfqpoint{6.328009in}{6.905654in}}%
\pgfpathlineto{\pgfqpoint{6.331370in}{6.901938in}}%
\pgfpathlineto{\pgfqpoint{6.332491in}{6.888424in}}%
\pgfpathlineto{\pgfqpoint{6.333612in}{6.905992in}}%
\pgfpathlineto{\pgfqpoint{6.334732in}{6.885045in}}%
\pgfpathlineto{\pgfqpoint{6.335853in}{6.891465in}}%
\pgfpathlineto{\pgfqpoint{6.339214in}{6.882343in}}%
\pgfpathlineto{\pgfqpoint{6.341455in}{6.910384in}}%
\pgfpathlineto{\pgfqpoint{6.342576in}{6.887748in}}%
\pgfpathlineto{\pgfqpoint{6.343696in}{6.895857in}}%
\pgfpathlineto{\pgfqpoint{6.347058in}{6.888424in}}%
\pgfpathlineto{\pgfqpoint{6.349299in}{6.907681in}}%
\pgfpathlineto{\pgfqpoint{6.350419in}{6.907344in}}%
\pgfpathlineto{\pgfqpoint{6.351540in}{6.921533in}}%
\pgfpathlineto{\pgfqpoint{6.354902in}{6.914776in}}%
\pgfpathlineto{\pgfqpoint{6.357143in}{6.917479in}}%
\pgfpathlineto{\pgfqpoint{6.358263in}{6.923560in}}%
\pgfpathlineto{\pgfqpoint{6.359384in}{6.922885in}}%
\pgfpathlineto{\pgfqpoint{6.362745in}{6.915790in}}%
\pgfpathlineto{\pgfqpoint{6.363866in}{6.921196in}}%
\pgfpathlineto{\pgfqpoint{6.364986in}{6.923560in}}%
\pgfpathlineto{\pgfqpoint{6.366107in}{6.930655in}}%
\pgfpathlineto{\pgfqpoint{6.367227in}{6.941467in}}%
\pgfpathlineto{\pgfqpoint{6.370589in}{6.946534in}}%
\pgfpathlineto{\pgfqpoint{6.371709in}{6.974576in}}%
\pgfpathlineto{\pgfqpoint{6.372830in}{6.984036in}}%
\pgfpathlineto{\pgfqpoint{6.373951in}{6.987752in}}%
\pgfpathlineto{\pgfqpoint{6.375071in}{6.981671in}}%
\pgfpathlineto{\pgfqpoint{6.378433in}{6.986063in}}%
\pgfpathlineto{\pgfqpoint{6.379553in}{6.985387in}}%
\pgfpathlineto{\pgfqpoint{6.380674in}{6.991131in}}%
\pgfpathlineto{\pgfqpoint{6.382915in}{6.967819in}}%
\pgfpathlineto{\pgfqpoint{6.386276in}{6.983022in}}%
\pgfpathlineto{\pgfqpoint{6.390758in}{6.940453in}}%
\pgfpathlineto{\pgfqpoint{6.394120in}{6.937412in}}%
\pgfpathlineto{\pgfqpoint{6.395241in}{6.928628in}}%
\pgfpathlineto{\pgfqpoint{6.396361in}{6.928966in}}%
\pgfpathlineto{\pgfqpoint{6.397482in}{6.930655in}}%
\pgfpathlineto{\pgfqpoint{6.398602in}{6.929304in}}%
\pgfpathlineto{\pgfqpoint{6.401964in}{6.928628in}}%
\pgfpathlineto{\pgfqpoint{6.403084in}{6.931331in}}%
\pgfpathlineto{\pgfqpoint{6.405325in}{6.942142in}}%
\pgfpathlineto{\pgfqpoint{6.406446in}{6.942818in}}%
\pgfpathlineto{\pgfqpoint{6.410928in}{6.940453in}}%
\pgfpathlineto{\pgfqpoint{6.412048in}{6.964441in}}%
\pgfpathlineto{\pgfqpoint{6.413169in}{6.957346in}}%
\pgfpathlineto{\pgfqpoint{6.414290in}{6.946197in}}%
\pgfpathlineto{\pgfqpoint{6.417651in}{6.964778in}}%
\pgfpathlineto{\pgfqpoint{6.419892in}{6.982685in}}%
\pgfpathlineto{\pgfqpoint{6.421013in}{6.987415in}}%
\pgfpathlineto{\pgfqpoint{6.422133in}{7.004307in}}%
\pgfpathlineto{\pgfqpoint{6.425495in}{7.004307in}}%
\pgfpathlineto{\pgfqpoint{6.426615in}{7.011740in}}%
\pgfpathlineto{\pgfqpoint{6.427736in}{7.006672in}}%
\pgfpathlineto{\pgfqpoint{6.428856in}{7.010726in}}%
\pgfpathlineto{\pgfqpoint{6.429977in}{7.010051in}}%
\pgfpathlineto{\pgfqpoint{6.433338in}{7.009375in}}%
\pgfpathlineto{\pgfqpoint{6.434459in}{7.019511in}}%
\pgfpathlineto{\pgfqpoint{6.435580in}{7.021538in}}%
\pgfpathlineto{\pgfqpoint{6.442303in}{7.080662in}}%
\pgfpathlineto{\pgfqpoint{6.443423in}{7.079310in}}%
\pgfpathlineto{\pgfqpoint{6.444544in}{7.085730in}}%
\pgfpathlineto{\pgfqpoint{6.445664in}{7.088770in}}%
\pgfpathlineto{\pgfqpoint{6.449026in}{7.096203in}}%
\pgfpathlineto{\pgfqpoint{6.450146in}{7.089446in}}%
\pgfpathlineto{\pgfqpoint{6.451267in}{7.078297in}}%
\pgfpathlineto{\pgfqpoint{6.452387in}{7.074580in}}%
\pgfpathlineto{\pgfqpoint{6.453508in}{7.090122in}}%
\pgfpathlineto{\pgfqpoint{6.457990in}{7.093838in}}%
\pgfpathlineto{\pgfqpoint{6.459111in}{7.108703in}}%
\pgfpathlineto{\pgfqpoint{6.460231in}{7.103636in}}%
\pgfpathlineto{\pgfqpoint{6.461352in}{7.114447in}}%
\pgfpathlineto{\pgfqpoint{6.465834in}{7.131340in}}%
\pgfpathlineto{\pgfqpoint{6.466954in}{7.125596in}}%
\pgfpathlineto{\pgfqpoint{6.468075in}{7.144178in}}%
\pgfpathlineto{\pgfqpoint{6.469195in}{7.242155in}}%
\pgfpathlineto{\pgfqpoint{6.472557in}{7.241142in}}%
\pgfpathlineto{\pgfqpoint{6.474798in}{7.316483in}}%
\pgfpathlineto{\pgfqpoint{6.475919in}{7.328983in}}%
\pgfpathlineto{\pgfqpoint{6.477039in}{7.304320in}}%
\pgfpathlineto{\pgfqpoint{6.480401in}{7.324929in}}%
\pgfpathlineto{\pgfqpoint{6.481521in}{7.327632in}}%
\pgfpathlineto{\pgfqpoint{6.482642in}{7.324929in}}%
\pgfpathlineto{\pgfqpoint{6.483762in}{7.312091in}}%
\pgfpathlineto{\pgfqpoint{6.484883in}{7.288779in}}%
\pgfpathlineto{\pgfqpoint{6.488244in}{7.294184in}}%
\pgfpathlineto{\pgfqpoint{6.489365in}{7.297901in}}%
\pgfpathlineto{\pgfqpoint{6.490485in}{7.284724in}}%
\pgfpathlineto{\pgfqpoint{6.491606in}{7.290806in}}%
\pgfpathlineto{\pgfqpoint{6.492726in}{7.258034in}}%
\pgfpathlineto{\pgfqpoint{6.496088in}{7.257696in}}%
\pgfpathlineto{\pgfqpoint{6.497209in}{7.268170in}}%
\pgfpathlineto{\pgfqpoint{6.498329in}{7.258710in}}%
\pgfpathlineto{\pgfqpoint{6.500570in}{7.261751in}}%
\pgfpathlineto{\pgfqpoint{6.503932in}{7.253304in}}%
\pgfpathlineto{\pgfqpoint{6.505052in}{7.261075in}}%
\pgfpathlineto{\pgfqpoint{6.506173in}{7.236074in}}%
\pgfpathlineto{\pgfqpoint{6.507293in}{7.264791in}}%
\pgfpathlineto{\pgfqpoint{6.508414in}{7.259723in}}%
\pgfpathlineto{\pgfqpoint{6.511775in}{7.253304in}}%
\pgfpathlineto{\pgfqpoint{6.512896in}{7.219519in}}%
\pgfpathlineto{\pgfqpoint{6.514016in}{7.219857in}}%
\pgfpathlineto{\pgfqpoint{6.515137in}{7.208032in}}%
\pgfpathlineto{\pgfqpoint{6.516257in}{7.216478in}}%
\pgfpathlineto{\pgfqpoint{6.519619in}{7.226614in}}%
\pgfpathlineto{\pgfqpoint{6.520740in}{7.215803in}}%
\pgfpathlineto{\pgfqpoint{6.521860in}{7.216140in}}%
\pgfpathlineto{\pgfqpoint{6.522981in}{7.213776in}}%
\pgfpathlineto{\pgfqpoint{6.524101in}{7.255669in}}%
\pgfpathlineto{\pgfqpoint{6.529704in}{7.352633in}}%
\pgfpathlineto{\pgfqpoint{6.530824in}{7.326956in}}%
\pgfpathlineto{\pgfqpoint{6.531945in}{7.324929in}}%
\pgfpathlineto{\pgfqpoint{6.536427in}{7.304996in}}%
\pgfpathlineto{\pgfqpoint{6.537547in}{7.305671in}}%
\pgfpathlineto{\pgfqpoint{6.538668in}{7.309388in}}%
\pgfpathlineto{\pgfqpoint{6.539789in}{7.307361in}}%
\pgfpathlineto{\pgfqpoint{6.539789in}{7.307361in}}%
\pgfusepath{stroke}%
\end{pgfscope}%
\begin{pgfscope}%
\pgfpathrectangle{\pgfqpoint{3.966666in}{6.297976in}}{\pgfqpoint{2.695652in}{1.104878in}}%
\pgfusepath{clip}%
\pgfsetroundcap%
\pgfsetroundjoin%
\pgfsetlinewidth{1.505625pt}%
\definecolor{currentstroke}{rgb}{1.000000,0.647059,0.000000}%
\pgfsetstrokecolor{currentstroke}%
\pgfsetdash{}{0pt}%
\pgfpathmoveto{\pgfqpoint{4.089196in}{6.468136in}}%
\pgfpathlineto{\pgfqpoint{4.091437in}{6.480861in}}%
\pgfpathlineto{\pgfqpoint{4.092557in}{6.482410in}}%
\pgfpathlineto{\pgfqpoint{4.095919in}{6.484555in}}%
\pgfpathlineto{\pgfqpoint{4.099280in}{6.490181in}}%
\pgfpathlineto{\pgfqpoint{4.100401in}{6.489533in}}%
\pgfpathlineto{\pgfqpoint{4.106004in}{6.488929in}}%
\pgfpathlineto{\pgfqpoint{4.107124in}{6.489617in}}%
\pgfpathlineto{\pgfqpoint{4.108245in}{6.491733in}}%
\pgfpathlineto{\pgfqpoint{4.111606in}{6.494174in}}%
\pgfpathlineto{\pgfqpoint{4.114968in}{6.500430in}}%
\pgfpathlineto{\pgfqpoint{4.116088in}{6.501883in}}%
\pgfpathlineto{\pgfqpoint{4.120570in}{6.503948in}}%
\pgfpathlineto{\pgfqpoint{4.123932in}{6.506666in}}%
\pgfpathlineto{\pgfqpoint{4.128414in}{6.508597in}}%
\pgfpathlineto{\pgfqpoint{4.131776in}{6.511333in}}%
\pgfpathlineto{\pgfqpoint{4.136258in}{6.512743in}}%
\pgfpathlineto{\pgfqpoint{4.138499in}{6.513925in}}%
\pgfpathlineto{\pgfqpoint{4.139619in}{6.514995in}}%
\pgfpathlineto{\pgfqpoint{4.146343in}{6.516702in}}%
\pgfpathlineto{\pgfqpoint{4.147463in}{6.517097in}}%
\pgfpathlineto{\pgfqpoint{4.151945in}{6.518328in}}%
\pgfpathlineto{\pgfqpoint{4.155307in}{6.519626in}}%
\pgfpathlineto{\pgfqpoint{4.160909in}{6.520390in}}%
\pgfpathlineto{\pgfqpoint{4.169874in}{6.523477in}}%
\pgfpathlineto{\pgfqpoint{4.170994in}{6.524161in}}%
\pgfpathlineto{\pgfqpoint{4.175476in}{6.525470in}}%
\pgfpathlineto{\pgfqpoint{4.178838in}{6.527414in}}%
\pgfpathlineto{\pgfqpoint{4.183320in}{6.528909in}}%
\pgfpathlineto{\pgfqpoint{4.186682in}{6.530753in}}%
\pgfpathlineto{\pgfqpoint{4.191164in}{6.532069in}}%
\pgfpathlineto{\pgfqpoint{4.193405in}{6.533106in}}%
\pgfpathlineto{\pgfqpoint{4.200128in}{6.534183in}}%
\pgfpathlineto{\pgfqpoint{4.202369in}{6.535306in}}%
\pgfpathlineto{\pgfqpoint{4.206851in}{6.536479in}}%
\pgfpathlineto{\pgfqpoint{4.210213in}{6.537387in}}%
\pgfpathlineto{\pgfqpoint{4.215815in}{6.538016in}}%
\pgfpathlineto{\pgfqpoint{4.218056in}{6.538914in}}%
\pgfpathlineto{\pgfqpoint{4.222538in}{6.540009in}}%
\pgfpathlineto{\pgfqpoint{4.225900in}{6.541564in}}%
\pgfpathlineto{\pgfqpoint{4.239346in}{6.542287in}}%
\pgfpathlineto{\pgfqpoint{4.253913in}{6.540191in}}%
\pgfpathlineto{\pgfqpoint{4.270721in}{6.537772in}}%
\pgfpathlineto{\pgfqpoint{4.286408in}{6.538074in}}%
\pgfpathlineto{\pgfqpoint{4.288649in}{6.537852in}}%
\pgfpathlineto{\pgfqpoint{4.300975in}{6.537331in}}%
\pgfpathlineto{\pgfqpoint{4.309940in}{6.535701in}}%
\pgfpathlineto{\pgfqpoint{4.333471in}{6.533626in}}%
\pgfpathlineto{\pgfqpoint{4.346917in}{6.533846in}}%
\pgfpathlineto{\pgfqpoint{4.351399in}{6.533234in}}%
\pgfpathlineto{\pgfqpoint{4.357002in}{6.532496in}}%
\pgfpathlineto{\pgfqpoint{4.359243in}{6.531924in}}%
\pgfpathlineto{\pgfqpoint{4.365966in}{6.531133in}}%
\pgfpathlineto{\pgfqpoint{4.382774in}{6.526353in}}%
\pgfpathlineto{\pgfqpoint{4.387256in}{6.525346in}}%
\pgfpathlineto{\pgfqpoint{4.390617in}{6.523930in}}%
\pgfpathlineto{\pgfqpoint{4.395100in}{6.523010in}}%
\pgfpathlineto{\pgfqpoint{4.398461in}{6.521567in}}%
\pgfpathlineto{\pgfqpoint{4.402943in}{6.520529in}}%
\pgfpathlineto{\pgfqpoint{4.406305in}{6.518778in}}%
\pgfpathlineto{\pgfqpoint{4.410787in}{6.517748in}}%
\pgfpathlineto{\pgfqpoint{4.414149in}{6.516054in}}%
\pgfpathlineto{\pgfqpoint{4.418631in}{6.514941in}}%
\pgfpathlineto{\pgfqpoint{4.421992in}{6.513379in}}%
\pgfpathlineto{\pgfqpoint{4.428715in}{6.512425in}}%
\pgfpathlineto{\pgfqpoint{4.429836in}{6.511970in}}%
\pgfpathlineto{\pgfqpoint{4.434318in}{6.511059in}}%
\pgfpathlineto{\pgfqpoint{4.437680in}{6.509368in}}%
\pgfpathlineto{\pgfqpoint{4.442162in}{6.508184in}}%
\pgfpathlineto{\pgfqpoint{4.445523in}{6.506275in}}%
\pgfpathlineto{\pgfqpoint{4.450005in}{6.504986in}}%
\pgfpathlineto{\pgfqpoint{4.451126in}{6.504286in}}%
\pgfpathlineto{\pgfqpoint{4.460090in}{6.501164in}}%
\pgfpathlineto{\pgfqpoint{4.461211in}{6.500524in}}%
\pgfpathlineto{\pgfqpoint{4.465693in}{6.499306in}}%
\pgfpathlineto{\pgfqpoint{4.469054in}{6.497621in}}%
\pgfpathlineto{\pgfqpoint{4.473536in}{6.496594in}}%
\pgfpathlineto{\pgfqpoint{4.476898in}{6.495154in}}%
\pgfpathlineto{\pgfqpoint{4.482501in}{6.493857in}}%
\pgfpathlineto{\pgfqpoint{4.484742in}{6.493014in}}%
\pgfpathlineto{\pgfqpoint{4.491465in}{6.491676in}}%
\pgfpathlineto{\pgfqpoint{4.492585in}{6.491197in}}%
\pgfpathlineto{\pgfqpoint{4.503791in}{6.489351in}}%
\pgfpathlineto{\pgfqpoint{4.508273in}{6.488114in}}%
\pgfpathlineto{\pgfqpoint{4.514996in}{6.487159in}}%
\pgfpathlineto{\pgfqpoint{4.516117in}{6.486828in}}%
\pgfpathlineto{\pgfqpoint{4.522840in}{6.485799in}}%
\pgfpathlineto{\pgfqpoint{4.523960in}{6.485450in}}%
\pgfpathlineto{\pgfqpoint{4.529563in}{6.484499in}}%
\pgfpathlineto{\pgfqpoint{4.531804in}{6.483869in}}%
\pgfpathlineto{\pgfqpoint{4.537407in}{6.482961in}}%
\pgfpathlineto{\pgfqpoint{4.539648in}{6.482340in}}%
\pgfpathlineto{\pgfqpoint{4.546371in}{6.481233in}}%
\pgfpathlineto{\pgfqpoint{4.547491in}{6.480955in}}%
\pgfpathlineto{\pgfqpoint{4.554214in}{6.480008in}}%
\pgfpathlineto{\pgfqpoint{4.555335in}{6.479670in}}%
\pgfpathlineto{\pgfqpoint{4.562058in}{6.478433in}}%
\pgfpathlineto{\pgfqpoint{4.566540in}{6.477923in}}%
\pgfpathlineto{\pgfqpoint{4.575504in}{6.476726in}}%
\pgfpathlineto{\pgfqpoint{4.590071in}{6.474721in}}%
\pgfpathlineto{\pgfqpoint{4.602397in}{6.473120in}}%
\pgfpathlineto{\pgfqpoint{4.632651in}{6.471609in}}%
\pgfpathlineto{\pgfqpoint{4.717811in}{6.474882in}}%
\pgfpathlineto{\pgfqpoint{4.763753in}{6.473620in}}%
\pgfpathlineto{\pgfqpoint{4.781681in}{6.473111in}}%
\pgfpathlineto{\pgfqpoint{4.856757in}{6.474790in}}%
\pgfpathlineto{\pgfqpoint{4.866842in}{6.475292in}}%
\pgfpathlineto{\pgfqpoint{4.889252in}{6.476236in}}%
\pgfpathlineto{\pgfqpoint{4.908301in}{6.477760in}}%
\pgfpathlineto{\pgfqpoint{4.920627in}{6.478733in}}%
\pgfpathlineto{\pgfqpoint{4.923988in}{6.479265in}}%
\pgfpathlineto{\pgfqpoint{4.954243in}{6.480670in}}%
\pgfpathlineto{\pgfqpoint{4.971051in}{6.481553in}}%
\pgfpathlineto{\pgfqpoint{4.991220in}{6.482555in}}%
\pgfpathlineto{\pgfqpoint{5.056211in}{6.488921in}}%
\pgfpathlineto{\pgfqpoint{5.073019in}{6.490527in}}%
\pgfpathlineto{\pgfqpoint{5.079742in}{6.491366in}}%
\pgfpathlineto{\pgfqpoint{5.088706in}{6.492832in}}%
\pgfpathlineto{\pgfqpoint{5.094309in}{6.493783in}}%
\pgfpathlineto{\pgfqpoint{5.096550in}{6.494429in}}%
\pgfpathlineto{\pgfqpoint{5.102152in}{6.495446in}}%
\pgfpathlineto{\pgfqpoint{5.104393in}{6.496148in}}%
\pgfpathlineto{\pgfqpoint{5.109996in}{6.497207in}}%
\pgfpathlineto{\pgfqpoint{5.111116in}{6.497566in}}%
\pgfpathlineto{\pgfqpoint{5.117840in}{6.498604in}}%
\pgfpathlineto{\pgfqpoint{5.120081in}{6.499323in}}%
\pgfpathlineto{\pgfqpoint{5.124563in}{6.500069in}}%
\pgfpathlineto{\pgfqpoint{5.127924in}{6.501510in}}%
\pgfpathlineto{\pgfqpoint{5.132406in}{6.502501in}}%
\pgfpathlineto{\pgfqpoint{5.135768in}{6.503962in}}%
\pgfpathlineto{\pgfqpoint{5.140250in}{6.504918in}}%
\pgfpathlineto{\pgfqpoint{5.143612in}{6.506311in}}%
\pgfpathlineto{\pgfqpoint{5.149214in}{6.507611in}}%
\pgfpathlineto{\pgfqpoint{5.151455in}{6.508423in}}%
\pgfpathlineto{\pgfqpoint{5.155938in}{6.509269in}}%
\pgfpathlineto{\pgfqpoint{5.159299in}{6.510662in}}%
\pgfpathlineto{\pgfqpoint{5.163781in}{6.511611in}}%
\pgfpathlineto{\pgfqpoint{5.167143in}{6.513088in}}%
\pgfpathlineto{\pgfqpoint{5.171625in}{6.514060in}}%
\pgfpathlineto{\pgfqpoint{5.174986in}{6.515505in}}%
\pgfpathlineto{\pgfqpoint{5.180589in}{6.516443in}}%
\pgfpathlineto{\pgfqpoint{5.182830in}{6.517407in}}%
\pgfpathlineto{\pgfqpoint{5.187312in}{6.518381in}}%
\pgfpathlineto{\pgfqpoint{5.190674in}{6.519801in}}%
\pgfpathlineto{\pgfqpoint{5.195156in}{6.520727in}}%
\pgfpathlineto{\pgfqpoint{5.198518in}{6.522139in}}%
\pgfpathlineto{\pgfqpoint{5.203000in}{6.523037in}}%
\pgfpathlineto{\pgfqpoint{5.206361in}{6.524351in}}%
\pgfpathlineto{\pgfqpoint{5.210843in}{6.525261in}}%
\pgfpathlineto{\pgfqpoint{5.214205in}{6.526487in}}%
\pgfpathlineto{\pgfqpoint{5.219808in}{6.527717in}}%
\pgfpathlineto{\pgfqpoint{5.222049in}{6.528431in}}%
\pgfpathlineto{\pgfqpoint{5.227651in}{6.529352in}}%
\pgfpathlineto{\pgfqpoint{5.237736in}{6.531600in}}%
\pgfpathlineto{\pgfqpoint{5.243339in}{6.532741in}}%
\pgfpathlineto{\pgfqpoint{5.253423in}{6.535445in}}%
\pgfpathlineto{\pgfqpoint{5.259026in}{6.536544in}}%
\pgfpathlineto{\pgfqpoint{5.261267in}{6.537312in}}%
\pgfpathlineto{\pgfqpoint{5.266870in}{6.538532in}}%
\pgfpathlineto{\pgfqpoint{5.269111in}{6.539448in}}%
\pgfpathlineto{\pgfqpoint{5.273593in}{6.540400in}}%
\pgfpathlineto{\pgfqpoint{5.276954in}{6.541412in}}%
\pgfpathlineto{\pgfqpoint{5.281437in}{6.542442in}}%
\pgfpathlineto{\pgfqpoint{5.284798in}{6.543995in}}%
\pgfpathlineto{\pgfqpoint{5.289280in}{6.544984in}}%
\pgfpathlineto{\pgfqpoint{5.292642in}{6.546388in}}%
\pgfpathlineto{\pgfqpoint{5.297124in}{6.547260in}}%
\pgfpathlineto{\pgfqpoint{5.300486in}{6.548656in}}%
\pgfpathlineto{\pgfqpoint{5.304968in}{6.549642in}}%
\pgfpathlineto{\pgfqpoint{5.308329in}{6.550636in}}%
\pgfpathlineto{\pgfqpoint{5.312811in}{6.551584in}}%
\pgfpathlineto{\pgfqpoint{5.316173in}{6.552476in}}%
\pgfpathlineto{\pgfqpoint{5.321776in}{6.553738in}}%
\pgfpathlineto{\pgfqpoint{5.324017in}{6.554650in}}%
\pgfpathlineto{\pgfqpoint{5.328499in}{6.555543in}}%
\pgfpathlineto{\pgfqpoint{5.331860in}{6.556849in}}%
\pgfpathlineto{\pgfqpoint{5.337463in}{6.557710in}}%
\pgfpathlineto{\pgfqpoint{5.339704in}{6.558598in}}%
\pgfpathlineto{\pgfqpoint{5.345307in}{6.559680in}}%
\pgfpathlineto{\pgfqpoint{5.347548in}{6.560324in}}%
\pgfpathlineto{\pgfqpoint{5.353150in}{6.561294in}}%
\pgfpathlineto{\pgfqpoint{5.355391in}{6.561947in}}%
\pgfpathlineto{\pgfqpoint{5.360994in}{6.562894in}}%
\pgfpathlineto{\pgfqpoint{5.363235in}{6.563586in}}%
\pgfpathlineto{\pgfqpoint{5.369958in}{6.564636in}}%
\pgfpathlineto{\pgfqpoint{5.371079in}{6.564984in}}%
\pgfpathlineto{\pgfqpoint{5.376681in}{6.565982in}}%
\pgfpathlineto{\pgfqpoint{5.378922in}{6.566599in}}%
\pgfpathlineto{\pgfqpoint{5.384525in}{6.567590in}}%
\pgfpathlineto{\pgfqpoint{5.386766in}{6.568200in}}%
\pgfpathlineto{\pgfqpoint{5.397971in}{6.569587in}}%
\pgfpathlineto{\pgfqpoint{5.410297in}{6.571392in}}%
\pgfpathlineto{\pgfqpoint{5.422623in}{6.572634in}}%
\pgfpathlineto{\pgfqpoint{5.433828in}{6.574513in}}%
\pgfpathlineto{\pgfqpoint{5.440551in}{6.575521in}}%
\pgfpathlineto{\pgfqpoint{5.449516in}{6.577055in}}%
\pgfpathlineto{\pgfqpoint{5.456239in}{6.578075in}}%
\pgfpathlineto{\pgfqpoint{5.465203in}{6.579618in}}%
\pgfpathlineto{\pgfqpoint{5.471926in}{6.580726in}}%
\pgfpathlineto{\pgfqpoint{5.473047in}{6.581004in}}%
\pgfpathlineto{\pgfqpoint{5.479770in}{6.581851in}}%
\pgfpathlineto{\pgfqpoint{5.480890in}{6.582162in}}%
\pgfpathlineto{\pgfqpoint{5.487614in}{6.583203in}}%
\pgfpathlineto{\pgfqpoint{5.496578in}{6.584436in}}%
\pgfpathlineto{\pgfqpoint{5.503301in}{6.585287in}}%
\pgfpathlineto{\pgfqpoint{5.512265in}{6.586559in}}%
\pgfpathlineto{\pgfqpoint{5.531314in}{6.587961in}}%
\pgfpathlineto{\pgfqpoint{5.543640in}{6.588957in}}%
\pgfpathlineto{\pgfqpoint{5.557086in}{6.589792in}}%
\pgfpathlineto{\pgfqpoint{5.573894in}{6.590975in}}%
\pgfpathlineto{\pgfqpoint{5.597425in}{6.592058in}}%
\pgfpathlineto{\pgfqpoint{5.611992in}{6.593100in}}%
\pgfpathlineto{\pgfqpoint{5.634403in}{6.595315in}}%
\pgfpathlineto{\pgfqpoint{5.645608in}{6.597261in}}%
\pgfpathlineto{\pgfqpoint{5.652331in}{6.598337in}}%
\pgfpathlineto{\pgfqpoint{5.661295in}{6.599881in}}%
\pgfpathlineto{\pgfqpoint{5.672501in}{6.601171in}}%
\pgfpathlineto{\pgfqpoint{5.684826in}{6.603210in}}%
\pgfpathlineto{\pgfqpoint{5.691549in}{6.604294in}}%
\pgfpathlineto{\pgfqpoint{5.698273in}{6.605393in}}%
\pgfpathlineto{\pgfqpoint{5.708357in}{6.607253in}}%
\pgfpathlineto{\pgfqpoint{5.713960in}{6.608040in}}%
\pgfpathlineto{\pgfqpoint{5.715081in}{6.608311in}}%
\pgfpathlineto{\pgfqpoint{5.721804in}{6.609134in}}%
\pgfpathlineto{\pgfqpoint{5.722924in}{6.609387in}}%
\pgfpathlineto{\pgfqpoint{5.735250in}{6.610582in}}%
\pgfpathlineto{\pgfqpoint{5.739732in}{6.611257in}}%
\pgfpathlineto{\pgfqpoint{5.754299in}{6.612120in}}%
\pgfpathlineto{\pgfqpoint{5.763263in}{6.612814in}}%
\pgfpathlineto{\pgfqpoint{5.784553in}{6.613835in}}%
\pgfpathlineto{\pgfqpoint{5.888762in}{6.623344in}}%
\pgfpathlineto{\pgfqpoint{5.901088in}{6.624278in}}%
\pgfpathlineto{\pgfqpoint{5.912293in}{6.625553in}}%
\pgfpathlineto{\pgfqpoint{5.924619in}{6.626652in}}%
\pgfpathlineto{\pgfqpoint{6.018743in}{6.640265in}}%
\pgfpathlineto{\pgfqpoint{6.022105in}{6.641127in}}%
\pgfpathlineto{\pgfqpoint{6.027708in}{6.641960in}}%
\pgfpathlineto{\pgfqpoint{6.037792in}{6.644012in}}%
\pgfpathlineto{\pgfqpoint{6.044516in}{6.645130in}}%
\pgfpathlineto{\pgfqpoint{6.050118in}{6.645975in}}%
\pgfpathlineto{\pgfqpoint{6.061323in}{6.647719in}}%
\pgfpathlineto{\pgfqpoint{6.072529in}{6.648918in}}%
\pgfpathlineto{\pgfqpoint{6.084855in}{6.650812in}}%
\pgfpathlineto{\pgfqpoint{6.097180in}{6.652146in}}%
\pgfpathlineto{\pgfqpoint{6.108386in}{6.653807in}}%
\pgfpathlineto{\pgfqpoint{6.115109in}{6.654799in}}%
\pgfpathlineto{\pgfqpoint{6.124073in}{6.656330in}}%
\pgfpathlineto{\pgfqpoint{6.136399in}{6.657565in}}%
\pgfpathlineto{\pgfqpoint{6.147604in}{6.659512in}}%
\pgfpathlineto{\pgfqpoint{6.155448in}{6.660492in}}%
\pgfpathlineto{\pgfqpoint{6.162171in}{6.661533in}}%
\pgfpathlineto{\pgfqpoint{6.166653in}{6.662063in}}%
\pgfpathlineto{\pgfqpoint{6.177858in}{6.663942in}}%
\pgfpathlineto{\pgfqpoint{6.186822in}{6.665295in}}%
\pgfpathlineto{\pgfqpoint{6.194666in}{6.666222in}}%
\pgfpathlineto{\pgfqpoint{6.201389in}{6.667125in}}%
\pgfpathlineto{\pgfqpoint{6.210354in}{6.668415in}}%
\pgfpathlineto{\pgfqpoint{6.221559in}{6.669612in}}%
\pgfpathlineto{\pgfqpoint{6.233885in}{6.671444in}}%
\pgfpathlineto{\pgfqpoint{6.240608in}{6.672317in}}%
\pgfpathlineto{\pgfqpoint{6.248451in}{6.673344in}}%
\pgfpathlineto{\pgfqpoint{6.256295in}{6.674173in}}%
\pgfpathlineto{\pgfqpoint{6.265259in}{6.675537in}}%
\pgfpathlineto{\pgfqpoint{6.271983in}{6.676455in}}%
\pgfpathlineto{\pgfqpoint{6.280947in}{6.677755in}}%
\pgfpathlineto{\pgfqpoint{6.292152in}{6.678947in}}%
\pgfpathlineto{\pgfqpoint{6.296634in}{6.679796in}}%
\pgfpathlineto{\pgfqpoint{6.307839in}{6.680861in}}%
\pgfpathlineto{\pgfqpoint{6.320165in}{6.682673in}}%
\pgfpathlineto{\pgfqpoint{6.332491in}{6.683848in}}%
\pgfpathlineto{\pgfqpoint{6.343696in}{6.684910in}}%
\pgfpathlineto{\pgfqpoint{6.354902in}{6.685863in}}%
\pgfpathlineto{\pgfqpoint{6.367227in}{6.687389in}}%
\pgfpathlineto{\pgfqpoint{6.373951in}{6.688202in}}%
\pgfpathlineto{\pgfqpoint{6.382915in}{6.689449in}}%
\pgfpathlineto{\pgfqpoint{6.395241in}{6.690750in}}%
\pgfpathlineto{\pgfqpoint{6.406446in}{6.692113in}}%
\pgfpathlineto{\pgfqpoint{6.417651in}{6.693029in}}%
\pgfpathlineto{\pgfqpoint{6.429977in}{6.694942in}}%
\pgfpathlineto{\pgfqpoint{6.436700in}{6.695842in}}%
\pgfpathlineto{\pgfqpoint{6.445664in}{6.697406in}}%
\pgfpathlineto{\pgfqpoint{6.452387in}{6.698470in}}%
\pgfpathlineto{\pgfqpoint{6.461352in}{6.700122in}}%
\pgfpathlineto{\pgfqpoint{6.466954in}{6.700999in}}%
\pgfpathlineto{\pgfqpoint{6.484883in}{6.705755in}}%
\pgfpathlineto{\pgfqpoint{6.490485in}{6.706945in}}%
\pgfpathlineto{\pgfqpoint{6.492726in}{6.707711in}}%
\pgfpathlineto{\pgfqpoint{6.498329in}{6.708831in}}%
\pgfpathlineto{\pgfqpoint{6.512896in}{6.711739in}}%
\pgfpathlineto{\pgfqpoint{6.516257in}{6.712749in}}%
\pgfpathlineto{\pgfqpoint{6.521860in}{6.713764in}}%
\pgfpathlineto{\pgfqpoint{6.531945in}{6.716504in}}%
\pgfpathlineto{\pgfqpoint{6.537547in}{6.717285in}}%
\pgfpathlineto{\pgfqpoint{6.539789in}{6.718069in}}%
\pgfpathlineto{\pgfqpoint{6.539789in}{6.718069in}}%
\pgfusepath{stroke}%
\end{pgfscope}%
\begin{pgfscope}%
\pgfsetrectcap%
\pgfsetmiterjoin%
\pgfsetlinewidth{0.803000pt}%
\definecolor{currentstroke}{rgb}{1.000000,1.000000,1.000000}%
\pgfsetstrokecolor{currentstroke}%
\pgfsetdash{}{0pt}%
\pgfpathmoveto{\pgfqpoint{3.966666in}{6.297976in}}%
\pgfpathlineto{\pgfqpoint{3.966666in}{7.402855in}}%
\pgfusepath{stroke}%
\end{pgfscope}%
\begin{pgfscope}%
\pgfsetrectcap%
\pgfsetmiterjoin%
\pgfsetlinewidth{0.803000pt}%
\definecolor{currentstroke}{rgb}{1.000000,1.000000,1.000000}%
\pgfsetstrokecolor{currentstroke}%
\pgfsetdash{}{0pt}%
\pgfpathmoveto{\pgfqpoint{6.662318in}{6.297976in}}%
\pgfpathlineto{\pgfqpoint{6.662318in}{7.402855in}}%
\pgfusepath{stroke}%
\end{pgfscope}%
\begin{pgfscope}%
\pgfsetrectcap%
\pgfsetmiterjoin%
\pgfsetlinewidth{0.803000pt}%
\definecolor{currentstroke}{rgb}{1.000000,1.000000,1.000000}%
\pgfsetstrokecolor{currentstroke}%
\pgfsetdash{}{0pt}%
\pgfpathmoveto{\pgfqpoint{3.966666in}{6.297976in}}%
\pgfpathlineto{\pgfqpoint{6.662318in}{6.297976in}}%
\pgfusepath{stroke}%
\end{pgfscope}%
\begin{pgfscope}%
\pgfsetrectcap%
\pgfsetmiterjoin%
\pgfsetlinewidth{0.803000pt}%
\definecolor{currentstroke}{rgb}{1.000000,1.000000,1.000000}%
\pgfsetstrokecolor{currentstroke}%
\pgfsetdash{}{0pt}%
\pgfpathmoveto{\pgfqpoint{3.966666in}{7.402855in}}%
\pgfpathlineto{\pgfqpoint{6.662318in}{7.402855in}}%
\pgfusepath{stroke}%
\end{pgfscope}%
\begin{pgfscope}%
\definecolor{textcolor}{rgb}{0.150000,0.150000,0.150000}%
\pgfsetstrokecolor{textcolor}%
\pgfsetfillcolor{textcolor}%
\pgftext[x=5.314492in,y=7.486188in,,base]{\color{textcolor}\rmfamily\fontsize{12.000000}{14.400000}\selectfont INTC}%
\end{pgfscope}%
\begin{pgfscope}%
\pgfsetbuttcap%
\pgfsetmiterjoin%
\definecolor{currentfill}{rgb}{0.917647,0.917647,0.949020}%
\pgfsetfillcolor{currentfill}%
\pgfsetlinewidth{0.000000pt}%
\definecolor{currentstroke}{rgb}{0.000000,0.000000,0.000000}%
\pgfsetstrokecolor{currentstroke}%
\pgfsetstrokeopacity{0.000000}%
\pgfsetdash{}{0pt}%
\pgfpathmoveto{\pgfqpoint{0.462318in}{4.309196in}}%
\pgfpathlineto{\pgfqpoint{3.157970in}{4.309196in}}%
\pgfpathlineto{\pgfqpoint{3.157970in}{5.414074in}}%
\pgfpathlineto{\pgfqpoint{0.462318in}{5.414074in}}%
\pgfpathclose%
\pgfusepath{fill}%
\end{pgfscope}%
\begin{pgfscope}%
\pgfpathrectangle{\pgfqpoint{0.462318in}{4.309196in}}{\pgfqpoint{2.695652in}{1.104878in}}%
\pgfusepath{clip}%
\pgfsetroundcap%
\pgfsetroundjoin%
\pgfsetlinewidth{0.803000pt}%
\definecolor{currentstroke}{rgb}{1.000000,1.000000,1.000000}%
\pgfsetstrokecolor{currentstroke}%
\pgfsetdash{}{0pt}%
\pgfpathmoveto{\pgfqpoint{0.582607in}{4.309196in}}%
\pgfpathlineto{\pgfqpoint{0.582607in}{5.414074in}}%
\pgfusepath{stroke}%
\end{pgfscope}%
\begin{pgfscope}%
\definecolor{textcolor}{rgb}{0.150000,0.150000,0.150000}%
\pgfsetstrokecolor{textcolor}%
\pgfsetfillcolor{textcolor}%
\pgftext[x=0.582607in,y=4.211974in,,top]{\color{textcolor}\rmfamily\fontsize{10.000000}{12.000000}\selectfont 2012}%
\end{pgfscope}%
\begin{pgfscope}%
\pgfpathrectangle{\pgfqpoint{0.462318in}{4.309196in}}{\pgfqpoint{2.695652in}{1.104878in}}%
\pgfusepath{clip}%
\pgfsetroundcap%
\pgfsetroundjoin%
\pgfsetlinewidth{0.803000pt}%
\definecolor{currentstroke}{rgb}{1.000000,1.000000,1.000000}%
\pgfsetstrokecolor{currentstroke}%
\pgfsetdash{}{0pt}%
\pgfpathmoveto{\pgfqpoint{0.992720in}{4.309196in}}%
\pgfpathlineto{\pgfqpoint{0.992720in}{5.414074in}}%
\pgfusepath{stroke}%
\end{pgfscope}%
\begin{pgfscope}%
\definecolor{textcolor}{rgb}{0.150000,0.150000,0.150000}%
\pgfsetstrokecolor{textcolor}%
\pgfsetfillcolor{textcolor}%
\pgftext[x=0.992720in,y=4.211974in,,top]{\color{textcolor}\rmfamily\fontsize{10.000000}{12.000000}\selectfont 2013}%
\end{pgfscope}%
\begin{pgfscope}%
\pgfpathrectangle{\pgfqpoint{0.462318in}{4.309196in}}{\pgfqpoint{2.695652in}{1.104878in}}%
\pgfusepath{clip}%
\pgfsetroundcap%
\pgfsetroundjoin%
\pgfsetlinewidth{0.803000pt}%
\definecolor{currentstroke}{rgb}{1.000000,1.000000,1.000000}%
\pgfsetstrokecolor{currentstroke}%
\pgfsetdash{}{0pt}%
\pgfpathmoveto{\pgfqpoint{1.401712in}{4.309196in}}%
\pgfpathlineto{\pgfqpoint{1.401712in}{5.414074in}}%
\pgfusepath{stroke}%
\end{pgfscope}%
\begin{pgfscope}%
\definecolor{textcolor}{rgb}{0.150000,0.150000,0.150000}%
\pgfsetstrokecolor{textcolor}%
\pgfsetfillcolor{textcolor}%
\pgftext[x=1.401712in,y=4.211974in,,top]{\color{textcolor}\rmfamily\fontsize{10.000000}{12.000000}\selectfont 2014}%
\end{pgfscope}%
\begin{pgfscope}%
\pgfpathrectangle{\pgfqpoint{0.462318in}{4.309196in}}{\pgfqpoint{2.695652in}{1.104878in}}%
\pgfusepath{clip}%
\pgfsetroundcap%
\pgfsetroundjoin%
\pgfsetlinewidth{0.803000pt}%
\definecolor{currentstroke}{rgb}{1.000000,1.000000,1.000000}%
\pgfsetstrokecolor{currentstroke}%
\pgfsetdash{}{0pt}%
\pgfpathmoveto{\pgfqpoint{1.810705in}{4.309196in}}%
\pgfpathlineto{\pgfqpoint{1.810705in}{5.414074in}}%
\pgfusepath{stroke}%
\end{pgfscope}%
\begin{pgfscope}%
\definecolor{textcolor}{rgb}{0.150000,0.150000,0.150000}%
\pgfsetstrokecolor{textcolor}%
\pgfsetfillcolor{textcolor}%
\pgftext[x=1.810705in,y=4.211974in,,top]{\color{textcolor}\rmfamily\fontsize{10.000000}{12.000000}\selectfont 2015}%
\end{pgfscope}%
\begin{pgfscope}%
\pgfpathrectangle{\pgfqpoint{0.462318in}{4.309196in}}{\pgfqpoint{2.695652in}{1.104878in}}%
\pgfusepath{clip}%
\pgfsetroundcap%
\pgfsetroundjoin%
\pgfsetlinewidth{0.803000pt}%
\definecolor{currentstroke}{rgb}{1.000000,1.000000,1.000000}%
\pgfsetstrokecolor{currentstroke}%
\pgfsetdash{}{0pt}%
\pgfpathmoveto{\pgfqpoint{2.219697in}{4.309196in}}%
\pgfpathlineto{\pgfqpoint{2.219697in}{5.414074in}}%
\pgfusepath{stroke}%
\end{pgfscope}%
\begin{pgfscope}%
\definecolor{textcolor}{rgb}{0.150000,0.150000,0.150000}%
\pgfsetstrokecolor{textcolor}%
\pgfsetfillcolor{textcolor}%
\pgftext[x=2.219697in,y=4.211974in,,top]{\color{textcolor}\rmfamily\fontsize{10.000000}{12.000000}\selectfont 2016}%
\end{pgfscope}%
\begin{pgfscope}%
\pgfpathrectangle{\pgfqpoint{0.462318in}{4.309196in}}{\pgfqpoint{2.695652in}{1.104878in}}%
\pgfusepath{clip}%
\pgfsetroundcap%
\pgfsetroundjoin%
\pgfsetlinewidth{0.803000pt}%
\definecolor{currentstroke}{rgb}{1.000000,1.000000,1.000000}%
\pgfsetstrokecolor{currentstroke}%
\pgfsetdash{}{0pt}%
\pgfpathmoveto{\pgfqpoint{2.629810in}{4.309196in}}%
\pgfpathlineto{\pgfqpoint{2.629810in}{5.414074in}}%
\pgfusepath{stroke}%
\end{pgfscope}%
\begin{pgfscope}%
\definecolor{textcolor}{rgb}{0.150000,0.150000,0.150000}%
\pgfsetstrokecolor{textcolor}%
\pgfsetfillcolor{textcolor}%
\pgftext[x=2.629810in,y=4.211974in,,top]{\color{textcolor}\rmfamily\fontsize{10.000000}{12.000000}\selectfont 2017}%
\end{pgfscope}%
\begin{pgfscope}%
\pgfpathrectangle{\pgfqpoint{0.462318in}{4.309196in}}{\pgfqpoint{2.695652in}{1.104878in}}%
\pgfusepath{clip}%
\pgfsetroundcap%
\pgfsetroundjoin%
\pgfsetlinewidth{0.803000pt}%
\definecolor{currentstroke}{rgb}{1.000000,1.000000,1.000000}%
\pgfsetstrokecolor{currentstroke}%
\pgfsetdash{}{0pt}%
\pgfpathmoveto{\pgfqpoint{3.038802in}{4.309196in}}%
\pgfpathlineto{\pgfqpoint{3.038802in}{5.414074in}}%
\pgfusepath{stroke}%
\end{pgfscope}%
\begin{pgfscope}%
\definecolor{textcolor}{rgb}{0.150000,0.150000,0.150000}%
\pgfsetstrokecolor{textcolor}%
\pgfsetfillcolor{textcolor}%
\pgftext[x=3.038802in,y=4.211974in,,top]{\color{textcolor}\rmfamily\fontsize{10.000000}{12.000000}\selectfont 2018}%
\end{pgfscope}%
\begin{pgfscope}%
\pgfpathrectangle{\pgfqpoint{0.462318in}{4.309196in}}{\pgfqpoint{2.695652in}{1.104878in}}%
\pgfusepath{clip}%
\pgfsetroundcap%
\pgfsetroundjoin%
\pgfsetlinewidth{0.803000pt}%
\definecolor{currentstroke}{rgb}{1.000000,1.000000,1.000000}%
\pgfsetstrokecolor{currentstroke}%
\pgfsetdash{}{0pt}%
\pgfpathmoveto{\pgfqpoint{0.462318in}{4.355911in}}%
\pgfpathlineto{\pgfqpoint{3.157970in}{4.355911in}}%
\pgfusepath{stroke}%
\end{pgfscope}%
\begin{pgfscope}%
\definecolor{textcolor}{rgb}{0.150000,0.150000,0.150000}%
\pgfsetstrokecolor{textcolor}%
\pgfsetfillcolor{textcolor}%
\pgftext[x=0.188365in,y=4.303150in,left,base]{\color{textcolor}\rmfamily\fontsize{10.000000}{12.000000}\selectfont 50}%
\end{pgfscope}%
\begin{pgfscope}%
\pgfpathrectangle{\pgfqpoint{0.462318in}{4.309196in}}{\pgfqpoint{2.695652in}{1.104878in}}%
\pgfusepath{clip}%
\pgfsetroundcap%
\pgfsetroundjoin%
\pgfsetlinewidth{0.803000pt}%
\definecolor{currentstroke}{rgb}{1.000000,1.000000,1.000000}%
\pgfsetstrokecolor{currentstroke}%
\pgfsetdash{}{0pt}%
\pgfpathmoveto{\pgfqpoint{0.462318in}{4.940293in}}%
\pgfpathlineto{\pgfqpoint{3.157970in}{4.940293in}}%
\pgfusepath{stroke}%
\end{pgfscope}%
\begin{pgfscope}%
\definecolor{textcolor}{rgb}{0.150000,0.150000,0.150000}%
\pgfsetstrokecolor{textcolor}%
\pgfsetfillcolor{textcolor}%
\pgftext[x=0.100000in,y=4.887531in,left,base]{\color{textcolor}\rmfamily\fontsize{10.000000}{12.000000}\selectfont 100}%
\end{pgfscope}%
\begin{pgfscope}%
\pgfpathrectangle{\pgfqpoint{0.462318in}{4.309196in}}{\pgfqpoint{2.695652in}{1.104878in}}%
\pgfusepath{clip}%
\pgfsetroundcap%
\pgfsetroundjoin%
\pgfsetlinewidth{1.505625pt}%
\definecolor{currentstroke}{rgb}{0.580392,0.403922,0.741176}%
\pgfsetstrokecolor{currentstroke}%
\pgfsetdash{}{0pt}%
\pgfpathmoveto{\pgfqpoint{0.584848in}{4.387001in}}%
\pgfpathlineto{\pgfqpoint{0.585968in}{4.383260in}}%
\pgfpathlineto{\pgfqpoint{0.587089in}{4.382559in}}%
\pgfpathlineto{\pgfqpoint{0.588209in}{4.377183in}}%
\pgfpathlineto{\pgfqpoint{0.591571in}{4.378118in}}%
\pgfpathlineto{\pgfqpoint{0.592692in}{4.380689in}}%
\pgfpathlineto{\pgfqpoint{0.593812in}{4.379988in}}%
\pgfpathlineto{\pgfqpoint{0.596053in}{4.381157in}}%
\pgfpathlineto{\pgfqpoint{0.600535in}{4.379871in}}%
\pgfpathlineto{\pgfqpoint{0.601656in}{4.381390in}}%
\pgfpathlineto{\pgfqpoint{0.602776in}{4.380572in}}%
\pgfpathlineto{\pgfqpoint{0.603897in}{4.381274in}}%
\pgfpathlineto{\pgfqpoint{0.607258in}{4.378819in}}%
\pgfpathlineto{\pgfqpoint{0.608379in}{4.378819in}}%
\pgfpathlineto{\pgfqpoint{0.609499in}{4.380806in}}%
\pgfpathlineto{\pgfqpoint{0.610620in}{4.385364in}}%
\pgfpathlineto{\pgfqpoint{0.611740in}{4.383962in}}%
\pgfpathlineto{\pgfqpoint{0.615102in}{4.385364in}}%
\pgfpathlineto{\pgfqpoint{0.616223in}{4.387234in}}%
\pgfpathlineto{\pgfqpoint{0.618464in}{4.384312in}}%
\pgfpathlineto{\pgfqpoint{0.619584in}{4.384780in}}%
\pgfpathlineto{\pgfqpoint{0.622946in}{4.380572in}}%
\pgfpathlineto{\pgfqpoint{0.624066in}{4.381157in}}%
\pgfpathlineto{\pgfqpoint{0.625187in}{4.381040in}}%
\pgfpathlineto{\pgfqpoint{0.627428in}{4.375079in}}%
\pgfpathlineto{\pgfqpoint{0.630789in}{4.375780in}}%
\pgfpathlineto{\pgfqpoint{0.631910in}{4.375079in}}%
\pgfpathlineto{\pgfqpoint{0.633031in}{4.375547in}}%
\pgfpathlineto{\pgfqpoint{0.634151in}{4.378001in}}%
\pgfpathlineto{\pgfqpoint{0.635272in}{4.378702in}}%
\pgfpathlineto{\pgfqpoint{0.643115in}{4.379053in}}%
\pgfpathlineto{\pgfqpoint{0.646477in}{4.378936in}}%
\pgfpathlineto{\pgfqpoint{0.647597in}{4.385832in}}%
\pgfpathlineto{\pgfqpoint{0.648718in}{4.384897in}}%
\pgfpathlineto{\pgfqpoint{0.649838in}{4.382559in}}%
\pgfpathlineto{\pgfqpoint{0.650959in}{4.381975in}}%
\pgfpathlineto{\pgfqpoint{0.654321in}{4.383260in}}%
\pgfpathlineto{\pgfqpoint{0.655441in}{4.378001in}}%
\pgfpathlineto{\pgfqpoint{0.656562in}{4.377534in}}%
\pgfpathlineto{\pgfqpoint{0.657682in}{4.382676in}}%
\pgfpathlineto{\pgfqpoint{0.658803in}{4.381741in}}%
\pgfpathlineto{\pgfqpoint{0.662164in}{4.385130in}}%
\pgfpathlineto{\pgfqpoint{0.663285in}{4.387234in}}%
\pgfpathlineto{\pgfqpoint{0.664405in}{4.384897in}}%
\pgfpathlineto{\pgfqpoint{0.665526in}{4.384780in}}%
\pgfpathlineto{\pgfqpoint{0.670008in}{4.386065in}}%
\pgfpathlineto{\pgfqpoint{0.673369in}{4.379053in}}%
\pgfpathlineto{\pgfqpoint{0.674490in}{4.379871in}}%
\pgfpathlineto{\pgfqpoint{0.680093in}{4.390039in}}%
\pgfpathlineto{\pgfqpoint{0.681213in}{4.389221in}}%
\pgfpathlineto{\pgfqpoint{0.682334in}{4.393195in}}%
\pgfpathlineto{\pgfqpoint{0.685695in}{4.395532in}}%
\pgfpathlineto{\pgfqpoint{0.689057in}{4.387351in}}%
\pgfpathlineto{\pgfqpoint{0.693539in}{4.383377in}}%
\pgfpathlineto{\pgfqpoint{0.694659in}{4.376599in}}%
\pgfpathlineto{\pgfqpoint{0.695780in}{4.375897in}}%
\pgfpathlineto{\pgfqpoint{0.696901in}{4.376131in}}%
\pgfpathlineto{\pgfqpoint{0.698021in}{4.370404in}}%
\pgfpathlineto{\pgfqpoint{0.701383in}{4.374495in}}%
\pgfpathlineto{\pgfqpoint{0.702503in}{4.376832in}}%
\pgfpathlineto{\pgfqpoint{0.703624in}{4.367716in}}%
\pgfpathlineto{\pgfqpoint{0.704744in}{4.365612in}}%
\pgfpathlineto{\pgfqpoint{0.705865in}{4.371923in}}%
\pgfpathlineto{\pgfqpoint{0.709226in}{4.368768in}}%
\pgfpathlineto{\pgfqpoint{0.710347in}{4.372508in}}%
\pgfpathlineto{\pgfqpoint{0.711467in}{4.378819in}}%
\pgfpathlineto{\pgfqpoint{0.712588in}{4.381741in}}%
\pgfpathlineto{\pgfqpoint{0.719311in}{4.387234in}}%
\pgfpathlineto{\pgfqpoint{0.720432in}{4.387351in}}%
\pgfpathlineto{\pgfqpoint{0.721552in}{4.381741in}}%
\pgfpathlineto{\pgfqpoint{0.724914in}{4.382209in}}%
\pgfpathlineto{\pgfqpoint{0.726034in}{4.383962in}}%
\pgfpathlineto{\pgfqpoint{0.727155in}{4.377300in}}%
\pgfpathlineto{\pgfqpoint{0.728275in}{4.380105in}}%
\pgfpathlineto{\pgfqpoint{0.729396in}{4.377884in}}%
\pgfpathlineto{\pgfqpoint{0.732757in}{4.374027in}}%
\pgfpathlineto{\pgfqpoint{0.733878in}{4.370988in}}%
\pgfpathlineto{\pgfqpoint{0.734998in}{4.371923in}}%
\pgfpathlineto{\pgfqpoint{0.737240in}{4.368534in}}%
\pgfpathlineto{\pgfqpoint{0.741722in}{4.370170in}}%
\pgfpathlineto{\pgfqpoint{0.742842in}{4.367833in}}%
\pgfpathlineto{\pgfqpoint{0.743963in}{4.372040in}}%
\pgfpathlineto{\pgfqpoint{0.745083in}{4.366430in}}%
\pgfpathlineto{\pgfqpoint{0.749565in}{4.367132in}}%
\pgfpathlineto{\pgfqpoint{0.750686in}{4.363508in}}%
\pgfpathlineto{\pgfqpoint{0.751806in}{4.365612in}}%
\pgfpathlineto{\pgfqpoint{0.752927in}{4.359418in}}%
\pgfpathlineto{\pgfqpoint{0.756288in}{4.364794in}}%
\pgfpathlineto{\pgfqpoint{0.757409in}{4.363508in}}%
\pgfpathlineto{\pgfqpoint{0.758530in}{4.369118in}}%
\pgfpathlineto{\pgfqpoint{0.759650in}{4.369118in}}%
\pgfpathlineto{\pgfqpoint{0.760771in}{4.370872in}}%
\pgfpathlineto{\pgfqpoint{0.764132in}{4.362690in}}%
\pgfpathlineto{\pgfqpoint{0.768614in}{4.399740in}}%
\pgfpathlineto{\pgfqpoint{0.771976in}{4.402428in}}%
\pgfpathlineto{\pgfqpoint{0.774217in}{4.409090in}}%
\pgfpathlineto{\pgfqpoint{0.775337in}{4.403363in}}%
\pgfpathlineto{\pgfqpoint{0.776458in}{4.405584in}}%
\pgfpathlineto{\pgfqpoint{0.780940in}{4.403597in}}%
\pgfpathlineto{\pgfqpoint{0.782061in}{4.407571in}}%
\pgfpathlineto{\pgfqpoint{0.783181in}{4.408623in}}%
\pgfpathlineto{\pgfqpoint{0.784302in}{4.414466in}}%
\pgfpathlineto{\pgfqpoint{0.787663in}{4.418674in}}%
\pgfpathlineto{\pgfqpoint{0.788784in}{4.419025in}}%
\pgfpathlineto{\pgfqpoint{0.792145in}{4.415285in}}%
\pgfpathlineto{\pgfqpoint{0.797748in}{4.417739in}}%
\pgfpathlineto{\pgfqpoint{0.798869in}{4.415869in}}%
\pgfpathlineto{\pgfqpoint{0.799989in}{4.424401in}}%
\pgfpathlineto{\pgfqpoint{0.803351in}{4.422882in}}%
\pgfpathlineto{\pgfqpoint{0.805592in}{4.431647in}}%
\pgfpathlineto{\pgfqpoint{0.806712in}{4.433167in}}%
\pgfpathlineto{\pgfqpoint{0.807833in}{4.424635in}}%
\pgfpathlineto{\pgfqpoint{0.811194in}{4.419726in}}%
\pgfpathlineto{\pgfqpoint{0.812315in}{4.412480in}}%
\pgfpathlineto{\pgfqpoint{0.813435in}{4.414233in}}%
\pgfpathlineto{\pgfqpoint{0.815676in}{4.433167in}}%
\pgfpathlineto{\pgfqpoint{0.819038in}{4.432465in}}%
\pgfpathlineto{\pgfqpoint{0.820159in}{4.430245in}}%
\pgfpathlineto{\pgfqpoint{0.821279in}{4.431764in}}%
\pgfpathlineto{\pgfqpoint{0.822400in}{4.422882in}}%
\pgfpathlineto{\pgfqpoint{0.823520in}{4.429310in}}%
\pgfpathlineto{\pgfqpoint{0.826882in}{4.426622in}}%
\pgfpathlineto{\pgfqpoint{0.828002in}{4.421362in}}%
\pgfpathlineto{\pgfqpoint{0.829123in}{4.421947in}}%
\pgfpathlineto{\pgfqpoint{0.830243in}{4.421713in}}%
\pgfpathlineto{\pgfqpoint{0.831364in}{4.424752in}}%
\pgfpathlineto{\pgfqpoint{0.834725in}{4.422998in}}%
\pgfpathlineto{\pgfqpoint{0.835846in}{4.424752in}}%
\pgfpathlineto{\pgfqpoint{0.839208in}{4.416804in}}%
\pgfpathlineto{\pgfqpoint{0.842569in}{4.415752in}}%
\pgfpathlineto{\pgfqpoint{0.843690in}{4.416570in}}%
\pgfpathlineto{\pgfqpoint{0.845931in}{4.416220in}}%
\pgfpathlineto{\pgfqpoint{0.847051in}{4.420661in}}%
\pgfpathlineto{\pgfqpoint{0.850413in}{4.419609in}}%
\pgfpathlineto{\pgfqpoint{0.851533in}{4.419843in}}%
\pgfpathlineto{\pgfqpoint{0.853774in}{4.416921in}}%
\pgfpathlineto{\pgfqpoint{0.854895in}{4.419025in}}%
\pgfpathlineto{\pgfqpoint{0.860498in}{4.417505in}}%
\pgfpathlineto{\pgfqpoint{0.861618in}{4.422998in}}%
\pgfpathlineto{\pgfqpoint{0.862739in}{4.423349in}}%
\pgfpathlineto{\pgfqpoint{0.866100in}{4.426271in}}%
\pgfpathlineto{\pgfqpoint{0.867221in}{4.426505in}}%
\pgfpathlineto{\pgfqpoint{0.868341in}{4.425920in}}%
\pgfpathlineto{\pgfqpoint{0.869462in}{4.433985in}}%
\pgfpathlineto{\pgfqpoint{0.870582in}{4.429076in}}%
\pgfpathlineto{\pgfqpoint{0.873944in}{4.426972in}}%
\pgfpathlineto{\pgfqpoint{0.875064in}{4.429777in}}%
\pgfpathlineto{\pgfqpoint{0.876185in}{4.430245in}}%
\pgfpathlineto{\pgfqpoint{0.878426in}{4.434686in}}%
\pgfpathlineto{\pgfqpoint{0.881788in}{4.434102in}}%
\pgfpathlineto{\pgfqpoint{0.882908in}{4.437257in}}%
\pgfpathlineto{\pgfqpoint{0.884029in}{4.434102in}}%
\pgfpathlineto{\pgfqpoint{0.885149in}{4.434219in}}%
\pgfpathlineto{\pgfqpoint{0.886270in}{4.433284in}}%
\pgfpathlineto{\pgfqpoint{0.889631in}{4.435387in}}%
\pgfpathlineto{\pgfqpoint{0.890752in}{4.433751in}}%
\pgfpathlineto{\pgfqpoint{0.891872in}{4.434102in}}%
\pgfpathlineto{\pgfqpoint{0.894113in}{4.440413in}}%
\pgfpathlineto{\pgfqpoint{0.897475in}{4.438309in}}%
\pgfpathlineto{\pgfqpoint{0.898595in}{4.428492in}}%
\pgfpathlineto{\pgfqpoint{0.900836in}{4.424284in}}%
\pgfpathlineto{\pgfqpoint{0.901957in}{4.424284in}}%
\pgfpathlineto{\pgfqpoint{0.905319in}{4.430245in}}%
\pgfpathlineto{\pgfqpoint{0.906439in}{4.439361in}}%
\pgfpathlineto{\pgfqpoint{0.908680in}{4.467996in}}%
\pgfpathlineto{\pgfqpoint{0.909801in}{4.461568in}}%
\pgfpathlineto{\pgfqpoint{0.913162in}{4.460866in}}%
\pgfpathlineto{\pgfqpoint{0.914283in}{4.452334in}}%
\pgfpathlineto{\pgfqpoint{0.915403in}{4.450815in}}%
\pgfpathlineto{\pgfqpoint{0.916524in}{4.454672in}}%
\pgfpathlineto{\pgfqpoint{0.917644in}{4.452334in}}%
\pgfpathlineto{\pgfqpoint{0.923247in}{4.451633in}}%
\pgfpathlineto{\pgfqpoint{0.924368in}{4.458178in}}%
\pgfpathlineto{\pgfqpoint{0.925488in}{4.452334in}}%
\pgfpathlineto{\pgfqpoint{0.928850in}{4.451282in}}%
\pgfpathlineto{\pgfqpoint{0.929970in}{4.453386in}}%
\pgfpathlineto{\pgfqpoint{0.932211in}{4.440413in}}%
\pgfpathlineto{\pgfqpoint{0.933332in}{4.442517in}}%
\pgfpathlineto{\pgfqpoint{0.936693in}{4.440647in}}%
\pgfpathlineto{\pgfqpoint{0.940055in}{4.434803in}}%
\pgfpathlineto{\pgfqpoint{0.941175in}{4.435972in}}%
\pgfpathlineto{\pgfqpoint{0.944537in}{4.436556in}}%
\pgfpathlineto{\pgfqpoint{0.945658in}{4.440530in}}%
\pgfpathlineto{\pgfqpoint{0.946778in}{4.439829in}}%
\pgfpathlineto{\pgfqpoint{0.949019in}{4.445439in}}%
\pgfpathlineto{\pgfqpoint{0.952381in}{4.440880in}}%
\pgfpathlineto{\pgfqpoint{0.953501in}{4.438192in}}%
\pgfpathlineto{\pgfqpoint{0.954622in}{4.442750in}}%
\pgfpathlineto{\pgfqpoint{0.955742in}{4.442166in}}%
\pgfpathlineto{\pgfqpoint{0.956863in}{4.447075in}}%
\pgfpathlineto{\pgfqpoint{0.960224in}{4.446257in}}%
\pgfpathlineto{\pgfqpoint{0.962465in}{4.449412in}}%
\pgfpathlineto{\pgfqpoint{0.963586in}{4.450114in}}%
\pgfpathlineto{\pgfqpoint{0.964707in}{4.454087in}}%
\pgfpathlineto{\pgfqpoint{0.968068in}{4.455490in}}%
\pgfpathlineto{\pgfqpoint{0.969189in}{4.460282in}}%
\pgfpathlineto{\pgfqpoint{0.972550in}{4.456425in}}%
\pgfpathlineto{\pgfqpoint{0.977032in}{4.458879in}}%
\pgfpathlineto{\pgfqpoint{0.978153in}{4.455841in}}%
\pgfpathlineto{\pgfqpoint{0.979273in}{4.456893in}}%
\pgfpathlineto{\pgfqpoint{0.980394in}{4.452334in}}%
\pgfpathlineto{\pgfqpoint{0.983756in}{4.449880in}}%
\pgfpathlineto{\pgfqpoint{0.985997in}{4.451282in}}%
\pgfpathlineto{\pgfqpoint{0.987117in}{4.450581in}}%
\pgfpathlineto{\pgfqpoint{0.988238in}{4.444620in}}%
\pgfpathlineto{\pgfqpoint{0.991599in}{4.450698in}}%
\pgfpathlineto{\pgfqpoint{0.993840in}{4.457828in}}%
\pgfpathlineto{\pgfqpoint{0.994961in}{4.456893in}}%
\pgfpathlineto{\pgfqpoint{0.996081in}{4.464723in}}%
\pgfpathlineto{\pgfqpoint{0.999443in}{4.463204in}}%
\pgfpathlineto{\pgfqpoint{1.000563in}{4.463321in}}%
\pgfpathlineto{\pgfqpoint{1.003925in}{4.472437in}}%
\pgfpathlineto{\pgfqpoint{1.007287in}{4.474541in}}%
\pgfpathlineto{\pgfqpoint{1.008407in}{4.472671in}}%
\pgfpathlineto{\pgfqpoint{1.009528in}{4.474658in}}%
\pgfpathlineto{\pgfqpoint{1.011769in}{4.480969in}}%
\pgfpathlineto{\pgfqpoint{1.016251in}{4.475710in}}%
\pgfpathlineto{\pgfqpoint{1.018492in}{4.479683in}}%
\pgfpathlineto{\pgfqpoint{1.019612in}{4.487631in}}%
\pgfpathlineto{\pgfqpoint{1.022974in}{4.484709in}}%
\pgfpathlineto{\pgfqpoint{1.024094in}{4.492423in}}%
\pgfpathlineto{\pgfqpoint{1.026336in}{4.487631in}}%
\pgfpathlineto{\pgfqpoint{1.027456in}{4.490202in}}%
\pgfpathlineto{\pgfqpoint{1.030818in}{4.489501in}}%
\pgfpathlineto{\pgfqpoint{1.033059in}{4.501890in}}%
\pgfpathlineto{\pgfqpoint{1.034179in}{4.498734in}}%
\pgfpathlineto{\pgfqpoint{1.035300in}{4.502825in}}%
\pgfpathlineto{\pgfqpoint{1.038661in}{4.502124in}}%
\pgfpathlineto{\pgfqpoint{1.039782in}{4.505864in}}%
\pgfpathlineto{\pgfqpoint{1.040902in}{4.504461in}}%
\pgfpathlineto{\pgfqpoint{1.042023in}{4.505981in}}%
\pgfpathlineto{\pgfqpoint{1.043143in}{4.509370in}}%
\pgfpathlineto{\pgfqpoint{1.047626in}{4.517084in}}%
\pgfpathlineto{\pgfqpoint{1.048746in}{4.514162in}}%
\pgfpathlineto{\pgfqpoint{1.049867in}{4.516266in}}%
\pgfpathlineto{\pgfqpoint{1.050987in}{4.516149in}}%
\pgfpathlineto{\pgfqpoint{1.054349in}{4.509487in}}%
\pgfpathlineto{\pgfqpoint{1.055469in}{4.511240in}}%
\pgfpathlineto{\pgfqpoint{1.056590in}{4.516850in}}%
\pgfpathlineto{\pgfqpoint{1.057710in}{4.514746in}}%
\pgfpathlineto{\pgfqpoint{1.058831in}{4.520590in}}%
\pgfpathlineto{\pgfqpoint{1.062192in}{4.525382in}}%
\pgfpathlineto{\pgfqpoint{1.063313in}{4.529940in}}%
\pgfpathlineto{\pgfqpoint{1.064433in}{4.527252in}}%
\pgfpathlineto{\pgfqpoint{1.066675in}{4.535083in}}%
\pgfpathlineto{\pgfqpoint{1.072277in}{4.538589in}}%
\pgfpathlineto{\pgfqpoint{1.073398in}{4.543965in}}%
\pgfpathlineto{\pgfqpoint{1.074518in}{4.544900in}}%
\pgfpathlineto{\pgfqpoint{1.077880in}{4.541160in}}%
\pgfpathlineto{\pgfqpoint{1.079000in}{4.541628in}}%
\pgfpathlineto{\pgfqpoint{1.080121in}{4.547355in}}%
\pgfpathlineto{\pgfqpoint{1.081241in}{4.543147in}}%
\pgfpathlineto{\pgfqpoint{1.082362in}{4.550277in}}%
\pgfpathlineto{\pgfqpoint{1.085723in}{4.549692in}}%
\pgfpathlineto{\pgfqpoint{1.086844in}{4.561029in}}%
\pgfpathlineto{\pgfqpoint{1.089085in}{4.567691in}}%
\pgfpathlineto{\pgfqpoint{1.093567in}{4.571665in}}%
\pgfpathlineto{\pgfqpoint{1.094688in}{4.579028in}}%
\pgfpathlineto{\pgfqpoint{1.095808in}{4.572951in}}%
\pgfpathlineto{\pgfqpoint{1.096929in}{4.576340in}}%
\pgfpathlineto{\pgfqpoint{1.101411in}{4.563601in}}%
\pgfpathlineto{\pgfqpoint{1.105893in}{4.579496in}}%
\pgfpathlineto{\pgfqpoint{1.109255in}{4.569444in}}%
\pgfpathlineto{\pgfqpoint{1.110375in}{4.586391in}}%
\pgfpathlineto{\pgfqpoint{1.111496in}{4.590833in}}%
\pgfpathlineto{\pgfqpoint{1.112616in}{4.583820in}}%
\pgfpathlineto{\pgfqpoint{1.113737in}{4.596560in}}%
\pgfpathlineto{\pgfqpoint{1.117098in}{4.599949in}}%
\pgfpathlineto{\pgfqpoint{1.118219in}{4.606027in}}%
\pgfpathlineto{\pgfqpoint{1.119339in}{4.595625in}}%
\pgfpathlineto{\pgfqpoint{1.120460in}{4.603689in}}%
\pgfpathlineto{\pgfqpoint{1.121580in}{4.602754in}}%
\pgfpathlineto{\pgfqpoint{1.124942in}{4.607195in}}%
\pgfpathlineto{\pgfqpoint{1.126062in}{4.603806in}}%
\pgfpathlineto{\pgfqpoint{1.127183in}{4.594105in}}%
\pgfpathlineto{\pgfqpoint{1.129424in}{4.608949in}}%
\pgfpathlineto{\pgfqpoint{1.132786in}{4.598430in}}%
\pgfpathlineto{\pgfqpoint{1.133906in}{4.606728in}}%
\pgfpathlineto{\pgfqpoint{1.135027in}{4.606027in}}%
\pgfpathlineto{\pgfqpoint{1.136147in}{4.603105in}}%
\pgfpathlineto{\pgfqpoint{1.137268in}{4.609065in}}%
\pgfpathlineto{\pgfqpoint{1.140629in}{4.609884in}}%
\pgfpathlineto{\pgfqpoint{1.142870in}{4.627415in}}%
\pgfpathlineto{\pgfqpoint{1.143991in}{4.625545in}}%
\pgfpathlineto{\pgfqpoint{1.145111in}{4.631739in}}%
\pgfpathlineto{\pgfqpoint{1.148473in}{4.631038in}}%
\pgfpathlineto{\pgfqpoint{1.149594in}{4.636648in}}%
\pgfpathlineto{\pgfqpoint{1.150714in}{4.635363in}}%
\pgfpathlineto{\pgfqpoint{1.152955in}{4.625779in}}%
\pgfpathlineto{\pgfqpoint{1.157437in}{4.633493in}}%
\pgfpathlineto{\pgfqpoint{1.158558in}{4.614208in}}%
\pgfpathlineto{\pgfqpoint{1.159678in}{4.617597in}}%
\pgfpathlineto{\pgfqpoint{1.160799in}{4.599715in}}%
\pgfpathlineto{\pgfqpoint{1.164160in}{4.604975in}}%
\pgfpathlineto{\pgfqpoint{1.166401in}{4.594806in}}%
\pgfpathlineto{\pgfqpoint{1.168642in}{4.606962in}}%
\pgfpathlineto{\pgfqpoint{1.172004in}{4.609182in}}%
\pgfpathlineto{\pgfqpoint{1.173125in}{4.604741in}}%
\pgfpathlineto{\pgfqpoint{1.174245in}{4.595391in}}%
\pgfpathlineto{\pgfqpoint{1.175366in}{4.606962in}}%
\pgfpathlineto{\pgfqpoint{1.176486in}{4.606962in}}%
\pgfpathlineto{\pgfqpoint{1.179848in}{4.613974in}}%
\pgfpathlineto{\pgfqpoint{1.180968in}{4.621221in}}%
\pgfpathlineto{\pgfqpoint{1.182089in}{4.606494in}}%
\pgfpathlineto{\pgfqpoint{1.183209in}{4.584405in}}%
\pgfpathlineto{\pgfqpoint{1.184330in}{4.590131in}}%
\pgfpathlineto{\pgfqpoint{1.187691in}{4.604040in}}%
\pgfpathlineto{\pgfqpoint{1.188812in}{4.611286in}}%
\pgfpathlineto{\pgfqpoint{1.189933in}{4.627415in}}%
\pgfpathlineto{\pgfqpoint{1.191053in}{4.624610in}}%
\pgfpathlineto{\pgfqpoint{1.192174in}{4.616312in}}%
\pgfpathlineto{\pgfqpoint{1.195535in}{4.623909in}}%
\pgfpathlineto{\pgfqpoint{1.196656in}{4.623324in}}%
\pgfpathlineto{\pgfqpoint{1.197776in}{4.625311in}}%
\pgfpathlineto{\pgfqpoint{1.200017in}{4.636064in}}%
\pgfpathlineto{\pgfqpoint{1.204499in}{4.645998in}}%
\pgfpathlineto{\pgfqpoint{1.207861in}{4.656868in}}%
\pgfpathlineto{\pgfqpoint{1.211223in}{4.660958in}}%
\pgfpathlineto{\pgfqpoint{1.212343in}{4.660958in}}%
\pgfpathlineto{\pgfqpoint{1.213464in}{4.658270in}}%
\pgfpathlineto{\pgfqpoint{1.214584in}{4.658738in}}%
\pgfpathlineto{\pgfqpoint{1.215705in}{4.678957in}}%
\pgfpathlineto{\pgfqpoint{1.219066in}{4.679425in}}%
\pgfpathlineto{\pgfqpoint{1.220187in}{4.680827in}}%
\pgfpathlineto{\pgfqpoint{1.221307in}{4.680243in}}%
\pgfpathlineto{\pgfqpoint{1.223548in}{4.684918in}}%
\pgfpathlineto{\pgfqpoint{1.226910in}{4.688658in}}%
\pgfpathlineto{\pgfqpoint{1.228030in}{4.688191in}}%
\pgfpathlineto{\pgfqpoint{1.230271in}{4.694151in}}%
\pgfpathlineto{\pgfqpoint{1.231392in}{4.700229in}}%
\pgfpathlineto{\pgfqpoint{1.234754in}{4.694385in}}%
\pgfpathlineto{\pgfqpoint{1.235874in}{4.694502in}}%
\pgfpathlineto{\pgfqpoint{1.236995in}{4.692983in}}%
\pgfpathlineto{\pgfqpoint{1.238115in}{4.689827in}}%
\pgfpathlineto{\pgfqpoint{1.239236in}{4.680243in}}%
\pgfpathlineto{\pgfqpoint{1.242597in}{4.676737in}}%
\pgfpathlineto{\pgfqpoint{1.243718in}{4.686671in}}%
\pgfpathlineto{\pgfqpoint{1.244838in}{4.663880in}}%
\pgfpathlineto{\pgfqpoint{1.245959in}{4.652543in}}%
\pgfpathlineto{\pgfqpoint{1.247079in}{4.650790in}}%
\pgfpathlineto{\pgfqpoint{1.250441in}{4.661426in}}%
\pgfpathlineto{\pgfqpoint{1.253803in}{4.639921in}}%
\pgfpathlineto{\pgfqpoint{1.254923in}{4.647985in}}%
\pgfpathlineto{\pgfqpoint{1.258285in}{4.639219in}}%
\pgfpathlineto{\pgfqpoint{1.259405in}{4.625779in}}%
\pgfpathlineto{\pgfqpoint{1.260526in}{4.629285in}}%
\pgfpathlineto{\pgfqpoint{1.261646in}{4.629636in}}%
\pgfpathlineto{\pgfqpoint{1.262767in}{4.628116in}}%
\pgfpathlineto{\pgfqpoint{1.267249in}{4.628233in}}%
\pgfpathlineto{\pgfqpoint{1.268369in}{4.632908in}}%
\pgfpathlineto{\pgfqpoint{1.273972in}{4.639453in}}%
\pgfpathlineto{\pgfqpoint{1.276213in}{4.656050in}}%
\pgfpathlineto{\pgfqpoint{1.277334in}{4.653829in}}%
\pgfpathlineto{\pgfqpoint{1.278454in}{4.649505in}}%
\pgfpathlineto{\pgfqpoint{1.281816in}{4.654063in}}%
\pgfpathlineto{\pgfqpoint{1.282936in}{4.654413in}}%
\pgfpathlineto{\pgfqpoint{1.284057in}{4.662828in}}%
\pgfpathlineto{\pgfqpoint{1.285177in}{4.664348in}}%
\pgfpathlineto{\pgfqpoint{1.286298in}{4.660491in}}%
\pgfpathlineto{\pgfqpoint{1.289659in}{4.654647in}}%
\pgfpathlineto{\pgfqpoint{1.291900in}{4.634778in}}%
\pgfpathlineto{\pgfqpoint{1.293021in}{4.634661in}}%
\pgfpathlineto{\pgfqpoint{1.294142in}{4.631272in}}%
\pgfpathlineto{\pgfqpoint{1.297503in}{4.630921in}}%
\pgfpathlineto{\pgfqpoint{1.298624in}{4.638635in}}%
\pgfpathlineto{\pgfqpoint{1.299744in}{4.636882in}}%
\pgfpathlineto{\pgfqpoint{1.300865in}{4.629752in}}%
\pgfpathlineto{\pgfqpoint{1.301985in}{4.636999in}}%
\pgfpathlineto{\pgfqpoint{1.305347in}{4.629869in}}%
\pgfpathlineto{\pgfqpoint{1.306467in}{4.620169in}}%
\pgfpathlineto{\pgfqpoint{1.307588in}{4.623675in}}%
\pgfpathlineto{\pgfqpoint{1.309829in}{4.658270in}}%
\pgfpathlineto{\pgfqpoint{1.314311in}{4.662945in}}%
\pgfpathlineto{\pgfqpoint{1.316552in}{4.683165in}}%
\pgfpathlineto{\pgfqpoint{1.317673in}{4.679892in}}%
\pgfpathlineto{\pgfqpoint{1.321034in}{4.675568in}}%
\pgfpathlineto{\pgfqpoint{1.322155in}{4.687139in}}%
\pgfpathlineto{\pgfqpoint{1.323275in}{4.684567in}}%
\pgfpathlineto{\pgfqpoint{1.324396in}{4.687022in}}%
\pgfpathlineto{\pgfqpoint{1.325516in}{4.684451in}}%
\pgfpathlineto{\pgfqpoint{1.328878in}{4.687372in}}%
\pgfpathlineto{\pgfqpoint{1.329998in}{4.694853in}}%
\pgfpathlineto{\pgfqpoint{1.332239in}{4.689593in}}%
\pgfpathlineto{\pgfqpoint{1.333360in}{4.697073in}}%
\pgfpathlineto{\pgfqpoint{1.336722in}{4.693684in}}%
\pgfpathlineto{\pgfqpoint{1.337842in}{4.691580in}}%
\pgfpathlineto{\pgfqpoint{1.338963in}{4.693801in}}%
\pgfpathlineto{\pgfqpoint{1.340083in}{4.690411in}}%
\pgfpathlineto{\pgfqpoint{1.341204in}{4.703852in}}%
\pgfpathlineto{\pgfqpoint{1.344565in}{4.706190in}}%
\pgfpathlineto{\pgfqpoint{1.345686in}{4.698943in}}%
\pgfpathlineto{\pgfqpoint{1.346806in}{4.696839in}}%
\pgfpathlineto{\pgfqpoint{1.349047in}{4.707241in}}%
\pgfpathlineto{\pgfqpoint{1.352409in}{4.706306in}}%
\pgfpathlineto{\pgfqpoint{1.354650in}{4.714722in}}%
\pgfpathlineto{\pgfqpoint{1.355771in}{4.715189in}}%
\pgfpathlineto{\pgfqpoint{1.356891in}{4.722318in}}%
\pgfpathlineto{\pgfqpoint{1.360253in}{4.726175in}}%
\pgfpathlineto{\pgfqpoint{1.361373in}{4.720448in}}%
\pgfpathlineto{\pgfqpoint{1.362494in}{4.719630in}}%
\pgfpathlineto{\pgfqpoint{1.369217in}{4.709579in}}%
\pgfpathlineto{\pgfqpoint{1.370337in}{4.706190in}}%
\pgfpathlineto{\pgfqpoint{1.371458in}{4.699528in}}%
\pgfpathlineto{\pgfqpoint{1.372578in}{4.714254in}}%
\pgfpathlineto{\pgfqpoint{1.375940in}{4.714254in}}%
\pgfpathlineto{\pgfqpoint{1.377061in}{4.711215in}}%
\pgfpathlineto{\pgfqpoint{1.378181in}{4.700930in}}%
\pgfpathlineto{\pgfqpoint{1.379302in}{4.681529in}}%
\pgfpathlineto{\pgfqpoint{1.380422in}{4.683399in}}%
\pgfpathlineto{\pgfqpoint{1.383784in}{4.683632in}}%
\pgfpathlineto{\pgfqpoint{1.384904in}{4.676503in}}%
\pgfpathlineto{\pgfqpoint{1.386025in}{4.696255in}}%
\pgfpathlineto{\pgfqpoint{1.387145in}{4.689710in}}%
\pgfpathlineto{\pgfqpoint{1.388266in}{4.690762in}}%
\pgfpathlineto{\pgfqpoint{1.392748in}{4.690528in}}%
\pgfpathlineto{\pgfqpoint{1.394989in}{4.695320in}}%
\pgfpathlineto{\pgfqpoint{1.396109in}{4.693333in}}%
\pgfpathlineto{\pgfqpoint{1.399471in}{4.692866in}}%
\pgfpathlineto{\pgfqpoint{1.400592in}{4.685736in}}%
\pgfpathlineto{\pgfqpoint{1.402833in}{4.680243in}}%
\pgfpathlineto{\pgfqpoint{1.403953in}{4.688424in}}%
\pgfpathlineto{\pgfqpoint{1.407315in}{4.693216in}}%
\pgfpathlineto{\pgfqpoint{1.408435in}{4.712735in}}%
\pgfpathlineto{\pgfqpoint{1.409556in}{4.711449in}}%
\pgfpathlineto{\pgfqpoint{1.410676in}{4.717176in}}%
\pgfpathlineto{\pgfqpoint{1.411797in}{4.717293in}}%
\pgfpathlineto{\pgfqpoint{1.415158in}{4.714838in}}%
\pgfpathlineto{\pgfqpoint{1.416279in}{4.717059in}}%
\pgfpathlineto{\pgfqpoint{1.417400in}{4.717877in}}%
\pgfpathlineto{\pgfqpoint{1.418520in}{4.716241in}}%
\pgfpathlineto{\pgfqpoint{1.419641in}{4.720448in}}%
\pgfpathlineto{\pgfqpoint{1.424123in}{4.710163in}}%
\pgfpathlineto{\pgfqpoint{1.425243in}{4.713085in}}%
\pgfpathlineto{\pgfqpoint{1.427484in}{4.676035in}}%
\pgfpathlineto{\pgfqpoint{1.430846in}{4.669374in}}%
\pgfpathlineto{\pgfqpoint{1.431966in}{4.670893in}}%
\pgfpathlineto{\pgfqpoint{1.433087in}{4.658972in}}%
\pgfpathlineto{\pgfqpoint{1.434207in}{4.664932in}}%
\pgfpathlineto{\pgfqpoint{1.435328in}{4.654647in}}%
\pgfpathlineto{\pgfqpoint{1.438690in}{4.637817in}}%
\pgfpathlineto{\pgfqpoint{1.439810in}{4.636181in}}%
\pgfpathlineto{\pgfqpoint{1.440931in}{4.642726in}}%
\pgfpathlineto{\pgfqpoint{1.443172in}{4.670309in}}%
\pgfpathlineto{\pgfqpoint{1.446533in}{4.680594in}}%
\pgfpathlineto{\pgfqpoint{1.447654in}{4.699528in}}%
\pgfpathlineto{\pgfqpoint{1.448774in}{4.694034in}}%
\pgfpathlineto{\pgfqpoint{1.451015in}{4.697424in}}%
\pgfpathlineto{\pgfqpoint{1.455497in}{4.691580in}}%
\pgfpathlineto{\pgfqpoint{1.456618in}{4.686321in}}%
\pgfpathlineto{\pgfqpoint{1.457738in}{4.693684in}}%
\pgfpathlineto{\pgfqpoint{1.458859in}{4.691697in}}%
\pgfpathlineto{\pgfqpoint{1.462221in}{4.687489in}}%
\pgfpathlineto{\pgfqpoint{1.464462in}{4.687489in}}%
\pgfpathlineto{\pgfqpoint{1.465582in}{4.690061in}}%
\pgfpathlineto{\pgfqpoint{1.466703in}{4.697658in}}%
\pgfpathlineto{\pgfqpoint{1.470064in}{4.692048in}}%
\pgfpathlineto{\pgfqpoint{1.471185in}{4.709930in}}%
\pgfpathlineto{\pgfqpoint{1.472305in}{4.702450in}}%
\pgfpathlineto{\pgfqpoint{1.474546in}{4.709813in}}%
\pgfpathlineto{\pgfqpoint{1.479029in}{4.711449in}}%
\pgfpathlineto{\pgfqpoint{1.480149in}{4.712618in}}%
\pgfpathlineto{\pgfqpoint{1.481270in}{4.706540in}}%
\pgfpathlineto{\pgfqpoint{1.482390in}{4.704670in}}%
\pgfpathlineto{\pgfqpoint{1.485752in}{4.715890in}}%
\pgfpathlineto{\pgfqpoint{1.486872in}{4.716007in}}%
\pgfpathlineto{\pgfqpoint{1.487993in}{4.712501in}}%
\pgfpathlineto{\pgfqpoint{1.489113in}{4.717760in}}%
\pgfpathlineto{\pgfqpoint{1.490234in}{4.735993in}}%
\pgfpathlineto{\pgfqpoint{1.493595in}{4.728630in}}%
\pgfpathlineto{\pgfqpoint{1.494716in}{4.750603in}}%
\pgfpathlineto{\pgfqpoint{1.495836in}{4.747213in}}%
\pgfpathlineto{\pgfqpoint{1.501439in}{4.759135in}}%
\pgfpathlineto{\pgfqpoint{1.502560in}{4.756213in}}%
\pgfpathlineto{\pgfqpoint{1.503680in}{4.759135in}}%
\pgfpathlineto{\pgfqpoint{1.504801in}{4.759485in}}%
\pgfpathlineto{\pgfqpoint{1.505921in}{4.761005in}}%
\pgfpathlineto{\pgfqpoint{1.509283in}{4.755745in}}%
\pgfpathlineto{\pgfqpoint{1.510403in}{4.757381in}}%
\pgfpathlineto{\pgfqpoint{1.511524in}{4.766381in}}%
\pgfpathlineto{\pgfqpoint{1.512644in}{4.742187in}}%
\pgfpathlineto{\pgfqpoint{1.513765in}{4.745460in}}%
\pgfpathlineto{\pgfqpoint{1.517126in}{4.748148in}}%
\pgfpathlineto{\pgfqpoint{1.518247in}{4.768835in}}%
\pgfpathlineto{\pgfqpoint{1.519367in}{4.764394in}}%
\pgfpathlineto{\pgfqpoint{1.520488in}{4.766498in}}%
\pgfpathlineto{\pgfqpoint{1.526091in}{4.778770in}}%
\pgfpathlineto{\pgfqpoint{1.527211in}{4.779120in}}%
\pgfpathlineto{\pgfqpoint{1.529452in}{4.774796in}}%
\pgfpathlineto{\pgfqpoint{1.532814in}{4.790340in}}%
\pgfpathlineto{\pgfqpoint{1.533934in}{4.787302in}}%
\pgfpathlineto{\pgfqpoint{1.535055in}{4.789873in}}%
\pgfpathlineto{\pgfqpoint{1.536175in}{4.782276in}}%
\pgfpathlineto{\pgfqpoint{1.537296in}{4.770004in}}%
\pgfpathlineto{\pgfqpoint{1.540658in}{4.776900in}}%
\pgfpathlineto{\pgfqpoint{1.541778in}{4.771991in}}%
\pgfpathlineto{\pgfqpoint{1.542899in}{4.786133in}}%
\pgfpathlineto{\pgfqpoint{1.544019in}{4.781925in}}%
\pgfpathlineto{\pgfqpoint{1.545140in}{4.786133in}}%
\pgfpathlineto{\pgfqpoint{1.548501in}{4.782159in}}%
\pgfpathlineto{\pgfqpoint{1.549622in}{4.787185in}}%
\pgfpathlineto{\pgfqpoint{1.552983in}{4.782744in}}%
\pgfpathlineto{\pgfqpoint{1.556345in}{4.783328in}}%
\pgfpathlineto{\pgfqpoint{1.557465in}{4.779471in}}%
\pgfpathlineto{\pgfqpoint{1.559706in}{4.793613in}}%
\pgfpathlineto{\pgfqpoint{1.560827in}{4.793847in}}%
\pgfpathlineto{\pgfqpoint{1.565309in}{4.792094in}}%
\pgfpathlineto{\pgfqpoint{1.566430in}{4.786951in}}%
\pgfpathlineto{\pgfqpoint{1.567550in}{4.791626in}}%
\pgfpathlineto{\pgfqpoint{1.568671in}{4.798756in}}%
\pgfpathlineto{\pgfqpoint{1.574273in}{4.811028in}}%
\pgfpathlineto{\pgfqpoint{1.575394in}{4.816521in}}%
\pgfpathlineto{\pgfqpoint{1.576514in}{4.816170in}}%
\pgfpathlineto{\pgfqpoint{1.579876in}{4.816521in}}%
\pgfpathlineto{\pgfqpoint{1.580996in}{4.825403in}}%
\pgfpathlineto{\pgfqpoint{1.583238in}{4.809508in}}%
\pgfpathlineto{\pgfqpoint{1.584358in}{4.809508in}}%
\pgfpathlineto{\pgfqpoint{1.587720in}{4.808690in}}%
\pgfpathlineto{\pgfqpoint{1.588840in}{4.803431in}}%
\pgfpathlineto{\pgfqpoint{1.591081in}{4.822481in}}%
\pgfpathlineto{\pgfqpoint{1.592202in}{4.837325in}}%
\pgfpathlineto{\pgfqpoint{1.596684in}{4.830429in}}%
\pgfpathlineto{\pgfqpoint{1.597804in}{4.842234in}}%
\pgfpathlineto{\pgfqpoint{1.598925in}{4.841182in}}%
\pgfpathlineto{\pgfqpoint{1.600045in}{4.834403in}}%
\pgfpathlineto{\pgfqpoint{1.603407in}{4.830663in}}%
\pgfpathlineto{\pgfqpoint{1.604528in}{4.843402in}}%
\pgfpathlineto{\pgfqpoint{1.605648in}{4.843285in}}%
\pgfpathlineto{\pgfqpoint{1.606769in}{4.838844in}}%
\pgfpathlineto{\pgfqpoint{1.611251in}{4.849480in}}%
\pgfpathlineto{\pgfqpoint{1.612371in}{4.841883in}}%
\pgfpathlineto{\pgfqpoint{1.613492in}{4.845039in}}%
\pgfpathlineto{\pgfqpoint{1.614612in}{4.842701in}}%
\pgfpathlineto{\pgfqpoint{1.615733in}{4.835572in}}%
\pgfpathlineto{\pgfqpoint{1.619094in}{4.838377in}}%
\pgfpathlineto{\pgfqpoint{1.621335in}{4.806353in}}%
\pgfpathlineto{\pgfqpoint{1.622456in}{4.787652in}}%
\pgfpathlineto{\pgfqpoint{1.623577in}{4.802145in}}%
\pgfpathlineto{\pgfqpoint{1.626938in}{4.796769in}}%
\pgfpathlineto{\pgfqpoint{1.628059in}{4.808924in}}%
\pgfpathlineto{\pgfqpoint{1.629179in}{4.806119in}}%
\pgfpathlineto{\pgfqpoint{1.630300in}{4.806236in}}%
\pgfpathlineto{\pgfqpoint{1.631420in}{4.805301in}}%
\pgfpathlineto{\pgfqpoint{1.634782in}{4.805301in}}%
\pgfpathlineto{\pgfqpoint{1.635902in}{4.803781in}}%
\pgfpathlineto{\pgfqpoint{1.637023in}{4.807171in}}%
\pgfpathlineto{\pgfqpoint{1.638143in}{4.784847in}}%
\pgfpathlineto{\pgfqpoint{1.639264in}{4.782977in}}%
\pgfpathlineto{\pgfqpoint{1.642625in}{4.785549in}}%
\pgfpathlineto{\pgfqpoint{1.643746in}{4.782159in}}%
\pgfpathlineto{\pgfqpoint{1.644867in}{4.791159in}}%
\pgfpathlineto{\pgfqpoint{1.645987in}{4.783211in}}%
\pgfpathlineto{\pgfqpoint{1.647108in}{4.794899in}}%
\pgfpathlineto{\pgfqpoint{1.650469in}{4.795717in}}%
\pgfpathlineto{\pgfqpoint{1.651590in}{4.790340in}}%
\pgfpathlineto{\pgfqpoint{1.652710in}{4.801561in}}%
\pgfpathlineto{\pgfqpoint{1.653831in}{4.804366in}}%
\pgfpathlineto{\pgfqpoint{1.654951in}{4.795834in}}%
\pgfpathlineto{\pgfqpoint{1.659433in}{4.813949in}}%
\pgfpathlineto{\pgfqpoint{1.660554in}{4.816404in}}%
\pgfpathlineto{\pgfqpoint{1.661674in}{4.826338in}}%
\pgfpathlineto{\pgfqpoint{1.662795in}{4.822365in}}%
\pgfpathlineto{\pgfqpoint{1.666157in}{4.823650in}}%
\pgfpathlineto{\pgfqpoint{1.667277in}{4.825871in}}%
\pgfpathlineto{\pgfqpoint{1.669518in}{4.820845in}}%
\pgfpathlineto{\pgfqpoint{1.670639in}{4.828793in}}%
\pgfpathlineto{\pgfqpoint{1.675121in}{4.825053in}}%
\pgfpathlineto{\pgfqpoint{1.676241in}{4.829143in}}%
\pgfpathlineto{\pgfqpoint{1.677362in}{4.829962in}}%
\pgfpathlineto{\pgfqpoint{1.678482in}{4.835805in}}%
\pgfpathlineto{\pgfqpoint{1.681844in}{4.832065in}}%
\pgfpathlineto{\pgfqpoint{1.682964in}{4.829494in}}%
\pgfpathlineto{\pgfqpoint{1.684085in}{4.841649in}}%
\pgfpathlineto{\pgfqpoint{1.685206in}{4.837208in}}%
\pgfpathlineto{\pgfqpoint{1.686326in}{4.837442in}}%
\pgfpathlineto{\pgfqpoint{1.689688in}{4.838844in}}%
\pgfpathlineto{\pgfqpoint{1.690808in}{4.850766in}}%
\pgfpathlineto{\pgfqpoint{1.691929in}{4.853921in}}%
\pgfpathlineto{\pgfqpoint{1.694170in}{4.872271in}}%
\pgfpathlineto{\pgfqpoint{1.697531in}{4.871102in}}%
\pgfpathlineto{\pgfqpoint{1.698652in}{4.866778in}}%
\pgfpathlineto{\pgfqpoint{1.699772in}{4.878816in}}%
\pgfpathlineto{\pgfqpoint{1.700893in}{4.863154in}}%
\pgfpathlineto{\pgfqpoint{1.702013in}{4.863154in}}%
\pgfpathlineto{\pgfqpoint{1.705375in}{4.857427in}}%
\pgfpathlineto{\pgfqpoint{1.706496in}{4.857895in}}%
\pgfpathlineto{\pgfqpoint{1.707616in}{4.834637in}}%
\pgfpathlineto{\pgfqpoint{1.708737in}{4.830078in}}%
\pgfpathlineto{\pgfqpoint{1.709857in}{4.843052in}}%
\pgfpathlineto{\pgfqpoint{1.713219in}{4.840364in}}%
\pgfpathlineto{\pgfqpoint{1.714339in}{4.815118in}}%
\pgfpathlineto{\pgfqpoint{1.715460in}{4.840831in}}%
\pgfpathlineto{\pgfqpoint{1.716580in}{4.811963in}}%
\pgfpathlineto{\pgfqpoint{1.717701in}{4.803314in}}%
\pgfpathlineto{\pgfqpoint{1.721062in}{4.781809in}}%
\pgfpathlineto{\pgfqpoint{1.722183in}{4.760303in}}%
\pgfpathlineto{\pgfqpoint{1.723303in}{4.772575in}}%
\pgfpathlineto{\pgfqpoint{1.724424in}{4.757966in}}%
\pgfpathlineto{\pgfqpoint{1.725544in}{4.777484in}}%
\pgfpathlineto{\pgfqpoint{1.728906in}{4.782627in}}%
\pgfpathlineto{\pgfqpoint{1.732268in}{4.817573in}}%
\pgfpathlineto{\pgfqpoint{1.733388in}{4.822715in}}%
\pgfpathlineto{\pgfqpoint{1.736750in}{4.832299in}}%
\pgfpathlineto{\pgfqpoint{1.738991in}{4.847493in}}%
\pgfpathlineto{\pgfqpoint{1.740111in}{4.862570in}}%
\pgfpathlineto{\pgfqpoint{1.741232in}{4.870050in}}%
\pgfpathlineto{\pgfqpoint{1.744593in}{4.866778in}}%
\pgfpathlineto{\pgfqpoint{1.745714in}{4.878699in}}%
\pgfpathlineto{\pgfqpoint{1.747955in}{4.882673in}}%
\pgfpathlineto{\pgfqpoint{1.749076in}{4.874375in}}%
\pgfpathlineto{\pgfqpoint{1.753558in}{4.881621in}}%
\pgfpathlineto{\pgfqpoint{1.754678in}{4.879985in}}%
\pgfpathlineto{\pgfqpoint{1.755799in}{4.883257in}}%
\pgfpathlineto{\pgfqpoint{1.756919in}{4.873907in}}%
\pgfpathlineto{\pgfqpoint{1.760281in}{4.875426in}}%
\pgfpathlineto{\pgfqpoint{1.761401in}{4.880803in}}%
\pgfpathlineto{\pgfqpoint{1.762522in}{4.880101in}}%
\pgfpathlineto{\pgfqpoint{1.763642in}{4.874141in}}%
\pgfpathlineto{\pgfqpoint{1.764763in}{4.877998in}}%
\pgfpathlineto{\pgfqpoint{1.769245in}{4.866193in}}%
\pgfpathlineto{\pgfqpoint{1.772607in}{4.882088in}}%
\pgfpathlineto{\pgfqpoint{1.775968in}{4.879751in}}%
\pgfpathlineto{\pgfqpoint{1.777089in}{4.884776in}}%
\pgfpathlineto{\pgfqpoint{1.778209in}{4.876595in}}%
\pgfpathlineto{\pgfqpoint{1.779330in}{4.874959in}}%
\pgfpathlineto{\pgfqpoint{1.780450in}{4.884776in}}%
\pgfpathlineto{\pgfqpoint{1.783812in}{4.884776in}}%
\pgfpathlineto{\pgfqpoint{1.784932in}{4.879985in}}%
\pgfpathlineto{\pgfqpoint{1.786053in}{4.861401in}}%
\pgfpathlineto{\pgfqpoint{1.787173in}{4.866310in}}%
\pgfpathlineto{\pgfqpoint{1.788294in}{4.842818in}}%
\pgfpathlineto{\pgfqpoint{1.791656in}{4.838026in}}%
\pgfpathlineto{\pgfqpoint{1.792776in}{4.825754in}}%
\pgfpathlineto{\pgfqpoint{1.793897in}{4.839195in}}%
\pgfpathlineto{\pgfqpoint{1.795017in}{4.867245in}}%
\pgfpathlineto{\pgfqpoint{1.796138in}{4.854389in}}%
\pgfpathlineto{\pgfqpoint{1.799499in}{4.866544in}}%
\pgfpathlineto{\pgfqpoint{1.800620in}{4.841299in}}%
\pgfpathlineto{\pgfqpoint{1.803981in}{4.849363in}}%
\pgfpathlineto{\pgfqpoint{1.808463in}{4.852402in}}%
\pgfpathlineto{\pgfqpoint{1.809584in}{4.844337in}}%
\pgfpathlineto{\pgfqpoint{1.811825in}{4.843753in}}%
\pgfpathlineto{\pgfqpoint{1.815187in}{4.836273in}}%
\pgfpathlineto{\pgfqpoint{1.816307in}{4.831013in}}%
\pgfpathlineto{\pgfqpoint{1.817428in}{4.854506in}}%
\pgfpathlineto{\pgfqpoint{1.818548in}{4.862921in}}%
\pgfpathlineto{\pgfqpoint{1.819669in}{4.848077in}}%
\pgfpathlineto{\pgfqpoint{1.823030in}{4.844454in}}%
\pgfpathlineto{\pgfqpoint{1.824151in}{4.846207in}}%
\pgfpathlineto{\pgfqpoint{1.825271in}{4.838494in}}%
\pgfpathlineto{\pgfqpoint{1.826392in}{4.822949in}}%
\pgfpathlineto{\pgfqpoint{1.827512in}{4.838844in}}%
\pgfpathlineto{\pgfqpoint{1.831995in}{4.810677in}}%
\pgfpathlineto{\pgfqpoint{1.833115in}{4.816871in}}%
\pgfpathlineto{\pgfqpoint{1.834236in}{4.836039in}}%
\pgfpathlineto{\pgfqpoint{1.835356in}{4.820027in}}%
\pgfpathlineto{\pgfqpoint{1.838718in}{4.820611in}}%
\pgfpathlineto{\pgfqpoint{1.839838in}{4.818858in}}%
\pgfpathlineto{\pgfqpoint{1.840959in}{4.812547in}}%
\pgfpathlineto{\pgfqpoint{1.842079in}{4.821780in}}%
\pgfpathlineto{\pgfqpoint{1.843200in}{4.798872in}}%
\pgfpathlineto{\pgfqpoint{1.846561in}{4.805885in}}%
\pgfpathlineto{\pgfqpoint{1.847682in}{4.822598in}}%
\pgfpathlineto{\pgfqpoint{1.848802in}{4.811378in}}%
\pgfpathlineto{\pgfqpoint{1.849923in}{4.822598in}}%
\pgfpathlineto{\pgfqpoint{1.851044in}{4.808690in}}%
\pgfpathlineto{\pgfqpoint{1.854405in}{4.795132in}}%
\pgfpathlineto{\pgfqpoint{1.855526in}{4.800976in}}%
\pgfpathlineto{\pgfqpoint{1.856646in}{4.801327in}}%
\pgfpathlineto{\pgfqpoint{1.857767in}{4.781458in}}%
\pgfpathlineto{\pgfqpoint{1.858887in}{4.793496in}}%
\pgfpathlineto{\pgfqpoint{1.863369in}{4.801911in}}%
\pgfpathlineto{\pgfqpoint{1.864490in}{4.797002in}}%
\pgfpathlineto{\pgfqpoint{1.865610in}{4.804599in}}%
\pgfpathlineto{\pgfqpoint{1.866731in}{4.807288in}}%
\pgfpathlineto{\pgfqpoint{1.870092in}{4.806469in}}%
\pgfpathlineto{\pgfqpoint{1.872334in}{4.817105in}}%
\pgfpathlineto{\pgfqpoint{1.873454in}{4.833585in}}%
\pgfpathlineto{\pgfqpoint{1.874575in}{4.830546in}}%
\pgfpathlineto{\pgfqpoint{1.877936in}{4.837909in}}%
\pgfpathlineto{\pgfqpoint{1.880177in}{4.821663in}}%
\pgfpathlineto{\pgfqpoint{1.881298in}{4.830663in}}%
\pgfpathlineto{\pgfqpoint{1.882418in}{4.805768in}}%
\pgfpathlineto{\pgfqpoint{1.885780in}{4.811378in}}%
\pgfpathlineto{\pgfqpoint{1.888021in}{4.787302in}}%
\pgfpathlineto{\pgfqpoint{1.889141in}{4.802846in}}%
\pgfpathlineto{\pgfqpoint{1.890262in}{4.796418in}}%
\pgfpathlineto{\pgfqpoint{1.893624in}{4.815586in}}%
\pgfpathlineto{\pgfqpoint{1.894744in}{4.803431in}}%
\pgfpathlineto{\pgfqpoint{1.895865in}{4.819676in}}%
\pgfpathlineto{\pgfqpoint{1.896985in}{4.822014in}}%
\pgfpathlineto{\pgfqpoint{1.898106in}{4.829377in}}%
\pgfpathlineto{\pgfqpoint{1.901467in}{4.835338in}}%
\pgfpathlineto{\pgfqpoint{1.902588in}{4.824819in}}%
\pgfpathlineto{\pgfqpoint{1.903708in}{4.808106in}}%
\pgfpathlineto{\pgfqpoint{1.904829in}{4.806002in}}%
\pgfpathlineto{\pgfqpoint{1.905949in}{4.808106in}}%
\pgfpathlineto{\pgfqpoint{1.909311in}{4.820611in}}%
\pgfpathlineto{\pgfqpoint{1.910431in}{4.810794in}}%
\pgfpathlineto{\pgfqpoint{1.911552in}{4.795834in}}%
\pgfpathlineto{\pgfqpoint{1.912673in}{4.800859in}}%
\pgfpathlineto{\pgfqpoint{1.917155in}{4.795951in}}%
\pgfpathlineto{\pgfqpoint{1.918275in}{4.805651in}}%
\pgfpathlineto{\pgfqpoint{1.919396in}{4.806236in}}%
\pgfpathlineto{\pgfqpoint{1.921637in}{4.825871in}}%
\pgfpathlineto{\pgfqpoint{1.924998in}{4.810326in}}%
\pgfpathlineto{\pgfqpoint{1.926119in}{4.809976in}}%
\pgfpathlineto{\pgfqpoint{1.927239in}{4.810794in}}%
\pgfpathlineto{\pgfqpoint{1.928360in}{4.802379in}}%
\pgfpathlineto{\pgfqpoint{1.929480in}{4.800275in}}%
\pgfpathlineto{\pgfqpoint{1.932842in}{4.806820in}}%
\pgfpathlineto{\pgfqpoint{1.936204in}{4.809391in}}%
\pgfpathlineto{\pgfqpoint{1.937324in}{4.815703in}}%
\pgfpathlineto{\pgfqpoint{1.940686in}{4.810560in}}%
\pgfpathlineto{\pgfqpoint{1.941806in}{4.812196in}}%
\pgfpathlineto{\pgfqpoint{1.942927in}{4.808573in}}%
\pgfpathlineto{\pgfqpoint{1.944047in}{4.796301in}}%
\pgfpathlineto{\pgfqpoint{1.945168in}{4.806002in}}%
\pgfpathlineto{\pgfqpoint{1.948529in}{4.808223in}}%
\pgfpathlineto{\pgfqpoint{1.949650in}{4.799457in}}%
\pgfpathlineto{\pgfqpoint{1.950770in}{4.796067in}}%
\pgfpathlineto{\pgfqpoint{1.951891in}{4.801093in}}%
\pgfpathlineto{\pgfqpoint{1.953011in}{4.819793in}}%
\pgfpathlineto{\pgfqpoint{1.956373in}{4.815235in}}%
\pgfpathlineto{\pgfqpoint{1.957494in}{4.809508in}}%
\pgfpathlineto{\pgfqpoint{1.958614in}{4.810326in}}%
\pgfpathlineto{\pgfqpoint{1.959735in}{4.823533in}}%
\pgfpathlineto{\pgfqpoint{1.960855in}{4.828325in}}%
\pgfpathlineto{\pgfqpoint{1.965337in}{4.845506in}}%
\pgfpathlineto{\pgfqpoint{1.967578in}{4.837325in}}%
\pgfpathlineto{\pgfqpoint{1.968699in}{4.826221in}}%
\pgfpathlineto{\pgfqpoint{1.973181in}{4.820611in}}%
\pgfpathlineto{\pgfqpoint{1.974302in}{4.823884in}}%
\pgfpathlineto{\pgfqpoint{1.975422in}{4.824118in}}%
\pgfpathlineto{\pgfqpoint{1.976543in}{4.813599in}}%
\pgfpathlineto{\pgfqpoint{1.981025in}{4.811963in}}%
\pgfpathlineto{\pgfqpoint{1.982145in}{4.813014in}}%
\pgfpathlineto{\pgfqpoint{1.984386in}{4.797470in}}%
\pgfpathlineto{\pgfqpoint{1.987748in}{4.790925in}}%
\pgfpathlineto{\pgfqpoint{1.988868in}{4.793496in}}%
\pgfpathlineto{\pgfqpoint{1.991109in}{4.804249in}}%
\pgfpathlineto{\pgfqpoint{1.992230in}{4.795016in}}%
\pgfpathlineto{\pgfqpoint{1.995592in}{4.786016in}}%
\pgfpathlineto{\pgfqpoint{1.996712in}{4.795016in}}%
\pgfpathlineto{\pgfqpoint{1.997833in}{4.798872in}}%
\pgfpathlineto{\pgfqpoint{1.998953in}{4.815936in}}%
\pgfpathlineto{\pgfqpoint{2.000074in}{4.810677in}}%
\pgfpathlineto{\pgfqpoint{2.003435in}{4.813014in}}%
\pgfpathlineto{\pgfqpoint{2.006797in}{4.802963in}}%
\pgfpathlineto{\pgfqpoint{2.007917in}{4.808339in}}%
\pgfpathlineto{\pgfqpoint{2.011279in}{4.788003in}}%
\pgfpathlineto{\pgfqpoint{2.012399in}{4.785665in}}%
\pgfpathlineto{\pgfqpoint{2.013520in}{4.796184in}}%
\pgfpathlineto{\pgfqpoint{2.014640in}{4.795951in}}%
\pgfpathlineto{\pgfqpoint{2.019123in}{4.793379in}}%
\pgfpathlineto{\pgfqpoint{2.020243in}{4.800859in}}%
\pgfpathlineto{\pgfqpoint{2.021364in}{4.788938in}}%
\pgfpathlineto{\pgfqpoint{2.022484in}{4.795483in}}%
\pgfpathlineto{\pgfqpoint{2.023605in}{4.807288in}}%
\pgfpathlineto{\pgfqpoint{2.026966in}{4.815001in}}%
\pgfpathlineto{\pgfqpoint{2.028087in}{4.809859in}}%
\pgfpathlineto{\pgfqpoint{2.030328in}{4.823650in}}%
\pgfpathlineto{\pgfqpoint{2.031448in}{4.813014in}}%
\pgfpathlineto{\pgfqpoint{2.034810in}{4.815936in}}%
\pgfpathlineto{\pgfqpoint{2.035931in}{4.815703in}}%
\pgfpathlineto{\pgfqpoint{2.037051in}{4.813949in}}%
\pgfpathlineto{\pgfqpoint{2.038172in}{4.814183in}}%
\pgfpathlineto{\pgfqpoint{2.039292in}{4.803314in}}%
\pgfpathlineto{\pgfqpoint{2.042654in}{4.794197in}}%
\pgfpathlineto{\pgfqpoint{2.044895in}{4.809274in}}%
\pgfpathlineto{\pgfqpoint{2.046015in}{4.810443in}}%
\pgfpathlineto{\pgfqpoint{2.047136in}{4.814300in}}%
\pgfpathlineto{\pgfqpoint{2.050497in}{4.812313in}}%
\pgfpathlineto{\pgfqpoint{2.051618in}{4.810093in}}%
\pgfpathlineto{\pgfqpoint{2.052738in}{4.817573in}}%
\pgfpathlineto{\pgfqpoint{2.053859in}{4.802496in}}%
\pgfpathlineto{\pgfqpoint{2.054979in}{4.800158in}}%
\pgfpathlineto{\pgfqpoint{2.058341in}{4.809976in}}%
\pgfpathlineto{\pgfqpoint{2.059462in}{4.801794in}}%
\pgfpathlineto{\pgfqpoint{2.061703in}{4.796301in}}%
\pgfpathlineto{\pgfqpoint{2.066185in}{4.810794in}}%
\pgfpathlineto{\pgfqpoint{2.067305in}{4.805534in}}%
\pgfpathlineto{\pgfqpoint{2.068426in}{4.804950in}}%
\pgfpathlineto{\pgfqpoint{2.069546in}{4.799574in}}%
\pgfpathlineto{\pgfqpoint{2.070667in}{4.773510in}}%
\pgfpathlineto{\pgfqpoint{2.074028in}{4.744759in}}%
\pgfpathlineto{\pgfqpoint{2.075149in}{4.722903in}}%
\pgfpathlineto{\pgfqpoint{2.076269in}{4.768835in}}%
\pgfpathlineto{\pgfqpoint{2.077390in}{4.780406in}}%
\pgfpathlineto{\pgfqpoint{2.078511in}{4.769420in}}%
\pgfpathlineto{\pgfqpoint{2.081872in}{4.756914in}}%
\pgfpathlineto{\pgfqpoint{2.082993in}{4.737162in}}%
\pgfpathlineto{\pgfqpoint{2.084113in}{4.750369in}}%
\pgfpathlineto{\pgfqpoint{2.085234in}{4.742889in}}%
\pgfpathlineto{\pgfqpoint{2.086354in}{4.728980in}}%
\pgfpathlineto{\pgfqpoint{2.090836in}{4.756563in}}%
\pgfpathlineto{\pgfqpoint{2.091957in}{4.738564in}}%
\pgfpathlineto{\pgfqpoint{2.093077in}{4.743707in}}%
\pgfpathlineto{\pgfqpoint{2.094198in}{4.745928in}}%
\pgfpathlineto{\pgfqpoint{2.097559in}{4.750018in}}%
\pgfpathlineto{\pgfqpoint{2.098680in}{4.761355in}}%
\pgfpathlineto{\pgfqpoint{2.100921in}{4.766030in}}%
\pgfpathlineto{\pgfqpoint{2.102042in}{4.750836in}}%
\pgfpathlineto{\pgfqpoint{2.105403in}{4.748031in}}%
\pgfpathlineto{\pgfqpoint{2.106524in}{4.749200in}}%
\pgfpathlineto{\pgfqpoint{2.107644in}{4.746629in}}%
\pgfpathlineto{\pgfqpoint{2.108765in}{4.741252in}}%
\pgfpathlineto{\pgfqpoint{2.109885in}{4.725708in}}%
\pgfpathlineto{\pgfqpoint{2.113247in}{4.729565in}}%
\pgfpathlineto{\pgfqpoint{2.114367in}{4.747096in}}%
\pgfpathlineto{\pgfqpoint{2.115488in}{4.750369in}}%
\pgfpathlineto{\pgfqpoint{2.116608in}{4.748499in}}%
\pgfpathlineto{\pgfqpoint{2.117729in}{4.756446in}}%
\pgfpathlineto{\pgfqpoint{2.121091in}{4.765095in}}%
\pgfpathlineto{\pgfqpoint{2.122211in}{4.751070in}}%
\pgfpathlineto{\pgfqpoint{2.123332in}{4.767433in}}%
\pgfpathlineto{\pgfqpoint{2.124452in}{4.768485in}}%
\pgfpathlineto{\pgfqpoint{2.125573in}{4.771523in}}%
\pgfpathlineto{\pgfqpoint{2.128934in}{4.778068in}}%
\pgfpathlineto{\pgfqpoint{2.130055in}{4.772342in}}%
\pgfpathlineto{\pgfqpoint{2.131175in}{4.762758in}}%
\pgfpathlineto{\pgfqpoint{2.132296in}{4.790224in}}%
\pgfpathlineto{\pgfqpoint{2.133416in}{4.801677in}}%
\pgfpathlineto{\pgfqpoint{2.136778in}{4.798522in}}%
\pgfpathlineto{\pgfqpoint{2.137898in}{4.794782in}}%
\pgfpathlineto{\pgfqpoint{2.139019in}{4.795249in}}%
\pgfpathlineto{\pgfqpoint{2.140140in}{4.815118in}}%
\pgfpathlineto{\pgfqpoint{2.141260in}{4.823300in}}%
\pgfpathlineto{\pgfqpoint{2.144622in}{4.819443in}}%
\pgfpathlineto{\pgfqpoint{2.146863in}{4.825170in}}%
\pgfpathlineto{\pgfqpoint{2.147983in}{4.834403in}}%
\pgfpathlineto{\pgfqpoint{2.149104in}{4.830897in}}%
\pgfpathlineto{\pgfqpoint{2.152465in}{4.842818in}}%
\pgfpathlineto{\pgfqpoint{2.153586in}{4.840597in}}%
\pgfpathlineto{\pgfqpoint{2.154706in}{4.840480in}}%
\pgfpathlineto{\pgfqpoint{2.155827in}{4.844571in}}%
\pgfpathlineto{\pgfqpoint{2.160309in}{4.828910in}}%
\pgfpathlineto{\pgfqpoint{2.162550in}{4.839545in}}%
\pgfpathlineto{\pgfqpoint{2.163671in}{4.822949in}}%
\pgfpathlineto{\pgfqpoint{2.164791in}{4.818858in}}%
\pgfpathlineto{\pgfqpoint{2.169273in}{4.835805in}}%
\pgfpathlineto{\pgfqpoint{2.170394in}{4.848077in}}%
\pgfpathlineto{\pgfqpoint{2.171514in}{4.846207in}}%
\pgfpathlineto{\pgfqpoint{2.172635in}{4.854038in}}%
\pgfpathlineto{\pgfqpoint{2.175996in}{4.857077in}}%
\pgfpathlineto{\pgfqpoint{2.177117in}{4.849363in}}%
\pgfpathlineto{\pgfqpoint{2.178237in}{4.848545in}}%
\pgfpathlineto{\pgfqpoint{2.180479in}{4.852869in}}%
\pgfpathlineto{\pgfqpoint{2.183840in}{4.840948in}}%
\pgfpathlineto{\pgfqpoint{2.184961in}{4.852752in}}%
\pgfpathlineto{\pgfqpoint{2.186081in}{4.849480in}}%
\pgfpathlineto{\pgfqpoint{2.187202in}{4.836156in}}%
\pgfpathlineto{\pgfqpoint{2.188322in}{4.858947in}}%
\pgfpathlineto{\pgfqpoint{2.191684in}{4.862921in}}%
\pgfpathlineto{\pgfqpoint{2.192804in}{4.853337in}}%
\pgfpathlineto{\pgfqpoint{2.193925in}{4.850532in}}%
\pgfpathlineto{\pgfqpoint{2.195045in}{4.855674in}}%
\pgfpathlineto{\pgfqpoint{2.196166in}{4.845623in}}%
\pgfpathlineto{\pgfqpoint{2.199527in}{4.850532in}}%
\pgfpathlineto{\pgfqpoint{2.201769in}{4.883257in}}%
\pgfpathlineto{\pgfqpoint{2.204010in}{4.848428in}}%
\pgfpathlineto{\pgfqpoint{2.207371in}{4.844571in}}%
\pgfpathlineto{\pgfqpoint{2.209612in}{4.864674in}}%
\pgfpathlineto{\pgfqpoint{2.210733in}{4.867128in}}%
\pgfpathlineto{\pgfqpoint{2.215215in}{4.861869in}}%
\pgfpathlineto{\pgfqpoint{2.216335in}{4.870401in}}%
\pgfpathlineto{\pgfqpoint{2.217456in}{4.867713in}}%
\pgfpathlineto{\pgfqpoint{2.218576in}{4.856609in}}%
\pgfpathlineto{\pgfqpoint{2.223059in}{4.832883in}}%
\pgfpathlineto{\pgfqpoint{2.224179in}{4.837325in}}%
\pgfpathlineto{\pgfqpoint{2.225300in}{4.831948in}}%
\pgfpathlineto{\pgfqpoint{2.227541in}{4.808339in}}%
\pgfpathlineto{\pgfqpoint{2.230902in}{4.802145in}}%
\pgfpathlineto{\pgfqpoint{2.232023in}{4.809274in}}%
\pgfpathlineto{\pgfqpoint{2.233143in}{4.796301in}}%
\pgfpathlineto{\pgfqpoint{2.234264in}{4.816053in}}%
\pgfpathlineto{\pgfqpoint{2.235384in}{4.796184in}}%
\pgfpathlineto{\pgfqpoint{2.239866in}{4.801444in}}%
\pgfpathlineto{\pgfqpoint{2.240987in}{4.782977in}}%
\pgfpathlineto{\pgfqpoint{2.242107in}{4.784964in}}%
\pgfpathlineto{\pgfqpoint{2.243228in}{4.793496in}}%
\pgfpathlineto{\pgfqpoint{2.246590in}{4.789756in}}%
\pgfpathlineto{\pgfqpoint{2.247710in}{4.840247in}}%
\pgfpathlineto{\pgfqpoint{2.248831in}{4.850649in}}%
\pgfpathlineto{\pgfqpoint{2.249951in}{4.851817in}}%
\pgfpathlineto{\pgfqpoint{2.251072in}{4.874725in}}%
\pgfpathlineto{\pgfqpoint{2.254433in}{4.874024in}}%
\pgfpathlineto{\pgfqpoint{2.255554in}{4.863856in}}%
\pgfpathlineto{\pgfqpoint{2.256674in}{4.871569in}}%
\pgfpathlineto{\pgfqpoint{2.257795in}{4.868998in}}%
\pgfpathlineto{\pgfqpoint{2.258915in}{4.833585in}}%
\pgfpathlineto{\pgfqpoint{2.262277in}{4.848895in}}%
\pgfpathlineto{\pgfqpoint{2.263398in}{4.848662in}}%
\pgfpathlineto{\pgfqpoint{2.264518in}{4.846090in}}%
\pgfpathlineto{\pgfqpoint{2.265639in}{4.845740in}}%
\pgfpathlineto{\pgfqpoint{2.272362in}{4.854272in}}%
\pgfpathlineto{\pgfqpoint{2.273482in}{4.872621in}}%
\pgfpathlineto{\pgfqpoint{2.274603in}{4.879751in}}%
\pgfpathlineto{\pgfqpoint{2.277964in}{4.886062in}}%
\pgfpathlineto{\pgfqpoint{2.279085in}{4.878933in}}%
\pgfpathlineto{\pgfqpoint{2.280205in}{4.888283in}}%
\pgfpathlineto{\pgfqpoint{2.281326in}{4.903360in}}%
\pgfpathlineto{\pgfqpoint{2.282446in}{4.896932in}}%
\pgfpathlineto{\pgfqpoint{2.285808in}{4.890854in}}%
\pgfpathlineto{\pgfqpoint{2.286929in}{4.912242in}}%
\pgfpathlineto{\pgfqpoint{2.288049in}{4.910372in}}%
\pgfpathlineto{\pgfqpoint{2.289170in}{4.906282in}}%
\pgfpathlineto{\pgfqpoint{2.290290in}{4.904645in}}%
\pgfpathlineto{\pgfqpoint{2.293652in}{4.907217in}}%
\pgfpathlineto{\pgfqpoint{2.294772in}{4.901607in}}%
\pgfpathlineto{\pgfqpoint{2.295893in}{4.908269in}}%
\pgfpathlineto{\pgfqpoint{2.297013in}{4.911424in}}%
\pgfpathlineto{\pgfqpoint{2.298134in}{4.917502in}}%
\pgfpathlineto{\pgfqpoint{2.301495in}{4.917034in}}%
\pgfpathlineto{\pgfqpoint{2.302616in}{4.918086in}}%
\pgfpathlineto{\pgfqpoint{2.303736in}{4.914346in}}%
\pgfpathlineto{\pgfqpoint{2.304857in}{4.907217in}}%
\pgfpathlineto{\pgfqpoint{2.305978in}{4.915281in}}%
\pgfpathlineto{\pgfqpoint{2.309339in}{4.912944in}}%
\pgfpathlineto{\pgfqpoint{2.310460in}{4.913996in}}%
\pgfpathlineto{\pgfqpoint{2.311580in}{4.925917in}}%
\pgfpathlineto{\pgfqpoint{2.312701in}{4.923930in}}%
\pgfpathlineto{\pgfqpoint{2.317183in}{4.922995in}}%
\pgfpathlineto{\pgfqpoint{2.318303in}{4.932696in}}%
\pgfpathlineto{\pgfqpoint{2.319424in}{4.931059in}}%
\pgfpathlineto{\pgfqpoint{2.320544in}{4.922761in}}%
\pgfpathlineto{\pgfqpoint{2.321665in}{4.933280in}}%
\pgfpathlineto{\pgfqpoint{2.325027in}{4.926852in}}%
\pgfpathlineto{\pgfqpoint{2.327268in}{4.935735in}}%
\pgfpathlineto{\pgfqpoint{2.329509in}{4.932345in}}%
\pgfpathlineto{\pgfqpoint{2.332870in}{4.930943in}}%
\pgfpathlineto{\pgfqpoint{2.333991in}{4.937605in}}%
\pgfpathlineto{\pgfqpoint{2.335111in}{4.940526in}}%
\pgfpathlineto{\pgfqpoint{2.336232in}{4.940176in}}%
\pgfpathlineto{\pgfqpoint{2.337352in}{4.943799in}}%
\pgfpathlineto{\pgfqpoint{2.340714in}{4.951747in}}%
\pgfpathlineto{\pgfqpoint{2.342955in}{4.979914in}}%
\pgfpathlineto{\pgfqpoint{2.344075in}{4.979914in}}%
\pgfpathlineto{\pgfqpoint{2.345196in}{4.977226in}}%
\pgfpathlineto{\pgfqpoint{2.348558in}{4.979213in}}%
\pgfpathlineto{\pgfqpoint{2.349678in}{4.972784in}}%
\pgfpathlineto{\pgfqpoint{2.350799in}{4.971382in}}%
\pgfpathlineto{\pgfqpoint{2.353040in}{4.964019in}}%
\pgfpathlineto{\pgfqpoint{2.356401in}{4.971148in}}%
\pgfpathlineto{\pgfqpoint{2.357522in}{4.970447in}}%
\pgfpathlineto{\pgfqpoint{2.358642in}{4.965538in}}%
\pgfpathlineto{\pgfqpoint{2.359763in}{4.972551in}}%
\pgfpathlineto{\pgfqpoint{2.360883in}{4.971031in}}%
\pgfpathlineto{\pgfqpoint{2.364245in}{4.981433in}}%
\pgfpathlineto{\pgfqpoint{2.365365in}{4.991601in}}%
\pgfpathlineto{\pgfqpoint{2.366486in}{4.988329in}}%
\pgfpathlineto{\pgfqpoint{2.367607in}{4.986926in}}%
\pgfpathlineto{\pgfqpoint{2.368727in}{4.979797in}}%
\pgfpathlineto{\pgfqpoint{2.372089in}{4.989147in}}%
\pgfpathlineto{\pgfqpoint{2.373209in}{4.982602in}}%
\pgfpathlineto{\pgfqpoint{2.374330in}{4.980031in}}%
\pgfpathlineto{\pgfqpoint{2.375450in}{4.972200in}}%
\pgfpathlineto{\pgfqpoint{2.376571in}{4.978511in}}%
\pgfpathlineto{\pgfqpoint{2.379932in}{4.973369in}}%
\pgfpathlineto{\pgfqpoint{2.382173in}{4.986108in}}%
\pgfpathlineto{\pgfqpoint{2.383294in}{4.981316in}}%
\pgfpathlineto{\pgfqpoint{2.384414in}{4.983069in}}%
\pgfpathlineto{\pgfqpoint{2.388897in}{4.979096in}}%
\pgfpathlineto{\pgfqpoint{2.390017in}{4.980031in}}%
\pgfpathlineto{\pgfqpoint{2.391138in}{4.998380in}}%
\pgfpathlineto{\pgfqpoint{2.392258in}{5.001185in}}%
\pgfpathlineto{\pgfqpoint{2.395620in}{5.012055in}}%
\pgfpathlineto{\pgfqpoint{2.396740in}{5.011587in}}%
\pgfpathlineto{\pgfqpoint{2.397861in}{5.012405in}}%
\pgfpathlineto{\pgfqpoint{2.398981in}{5.025262in}}%
\pgfpathlineto{\pgfqpoint{2.400102in}{5.025379in}}%
\pgfpathlineto{\pgfqpoint{2.403463in}{5.023041in}}%
\pgfpathlineto{\pgfqpoint{2.404584in}{5.026547in}}%
\pgfpathlineto{\pgfqpoint{2.405704in}{5.018950in}}%
\pgfpathlineto{\pgfqpoint{2.406825in}{5.021405in}}%
\pgfpathlineto{\pgfqpoint{2.407946in}{5.008899in}}%
\pgfpathlineto{\pgfqpoint{2.411307in}{5.020353in}}%
\pgfpathlineto{\pgfqpoint{2.412428in}{5.016496in}}%
\pgfpathlineto{\pgfqpoint{2.413548in}{5.019418in}}%
\pgfpathlineto{\pgfqpoint{2.414669in}{5.029352in}}%
\pgfpathlineto{\pgfqpoint{2.415789in}{5.010535in}}%
\pgfpathlineto{\pgfqpoint{2.419151in}{5.020470in}}%
\pgfpathlineto{\pgfqpoint{2.422512in}{5.071311in}}%
\pgfpathlineto{\pgfqpoint{2.423633in}{5.071194in}}%
\pgfpathlineto{\pgfqpoint{2.429236in}{5.085687in}}%
\pgfpathlineto{\pgfqpoint{2.430356in}{5.084284in}}%
\pgfpathlineto{\pgfqpoint{2.431477in}{5.087907in}}%
\pgfpathlineto{\pgfqpoint{2.437079in}{5.089544in}}%
\pgfpathlineto{\pgfqpoint{2.438200in}{5.091414in}}%
\pgfpathlineto{\pgfqpoint{2.439320in}{5.089544in}}%
\pgfpathlineto{\pgfqpoint{2.442682in}{5.091063in}}%
\pgfpathlineto{\pgfqpoint{2.443802in}{5.113620in}}%
\pgfpathlineto{\pgfqpoint{2.444923in}{5.112451in}}%
\pgfpathlineto{\pgfqpoint{2.446043in}{5.112568in}}%
\pgfpathlineto{\pgfqpoint{2.447164in}{5.111283in}}%
\pgfpathlineto{\pgfqpoint{2.450526in}{5.109763in}}%
\pgfpathlineto{\pgfqpoint{2.451646in}{5.112568in}}%
\pgfpathlineto{\pgfqpoint{2.453887in}{5.105205in}}%
\pgfpathlineto{\pgfqpoint{2.455008in}{5.113386in}}%
\pgfpathlineto{\pgfqpoint{2.458369in}{5.115256in}}%
\pgfpathlineto{\pgfqpoint{2.459490in}{5.109413in}}%
\pgfpathlineto{\pgfqpoint{2.460610in}{5.099128in}}%
\pgfpathlineto{\pgfqpoint{2.461731in}{5.098777in}}%
\pgfpathlineto{\pgfqpoint{2.462851in}{5.102868in}}%
\pgfpathlineto{\pgfqpoint{2.467333in}{5.094102in}}%
\pgfpathlineto{\pgfqpoint{2.468454in}{5.093401in}}%
\pgfpathlineto{\pgfqpoint{2.469575in}{5.097725in}}%
\pgfpathlineto{\pgfqpoint{2.470695in}{5.091881in}}%
\pgfpathlineto{\pgfqpoint{2.474057in}{5.082181in}}%
\pgfpathlineto{\pgfqpoint{2.475177in}{5.060909in}}%
\pgfpathlineto{\pgfqpoint{2.476298in}{5.071428in}}%
\pgfpathlineto{\pgfqpoint{2.477418in}{5.065117in}}%
\pgfpathlineto{\pgfqpoint{2.478539in}{5.065117in}}%
\pgfpathlineto{\pgfqpoint{2.481900in}{5.056585in}}%
\pgfpathlineto{\pgfqpoint{2.483021in}{5.059974in}}%
\pgfpathlineto{\pgfqpoint{2.484141in}{5.052026in}}%
\pgfpathlineto{\pgfqpoint{2.485262in}{5.050507in}}%
\pgfpathlineto{\pgfqpoint{2.486382in}{5.055650in}}%
\pgfpathlineto{\pgfqpoint{2.489744in}{5.065117in}}%
\pgfpathlineto{\pgfqpoint{2.490865in}{5.060208in}}%
\pgfpathlineto{\pgfqpoint{2.491985in}{5.058805in}}%
\pgfpathlineto{\pgfqpoint{2.493106in}{5.056000in}}%
\pgfpathlineto{\pgfqpoint{2.494226in}{5.058572in}}%
\pgfpathlineto{\pgfqpoint{2.498708in}{5.063247in}}%
\pgfpathlineto{\pgfqpoint{2.500949in}{5.060208in}}%
\pgfpathlineto{\pgfqpoint{2.502070in}{5.046884in}}%
\pgfpathlineto{\pgfqpoint{2.505431in}{5.056935in}}%
\pgfpathlineto{\pgfqpoint{2.506552in}{5.040222in}}%
\pgfpathlineto{\pgfqpoint{2.507672in}{5.042910in}}%
\pgfpathlineto{\pgfqpoint{2.508793in}{5.051208in}}%
\pgfpathlineto{\pgfqpoint{2.509913in}{5.047118in}}%
\pgfpathlineto{\pgfqpoint{2.513275in}{5.040689in}}%
\pgfpathlineto{\pgfqpoint{2.514396in}{5.043845in}}%
\pgfpathlineto{\pgfqpoint{2.516637in}{5.060091in}}%
\pgfpathlineto{\pgfqpoint{2.517757in}{5.053078in}}%
\pgfpathlineto{\pgfqpoint{2.521119in}{5.041975in}}%
\pgfpathlineto{\pgfqpoint{2.522239in}{5.057520in}}%
\pgfpathlineto{\pgfqpoint{2.523360in}{5.059390in}}%
\pgfpathlineto{\pgfqpoint{2.524480in}{5.036482in}}%
\pgfpathlineto{\pgfqpoint{2.525601in}{5.045832in}}%
\pgfpathlineto{\pgfqpoint{2.528962in}{5.053078in}}%
\pgfpathlineto{\pgfqpoint{2.530083in}{5.053195in}}%
\pgfpathlineto{\pgfqpoint{2.531204in}{5.057169in}}%
\pgfpathlineto{\pgfqpoint{2.532324in}{5.052494in}}%
\pgfpathlineto{\pgfqpoint{2.533445in}{5.057753in}}%
\pgfpathlineto{\pgfqpoint{2.536806in}{5.063831in}}%
\pgfpathlineto{\pgfqpoint{2.537927in}{5.040456in}}%
\pgfpathlineto{\pgfqpoint{2.540168in}{5.047235in}}%
\pgfpathlineto{\pgfqpoint{2.541288in}{5.039638in}}%
\pgfpathlineto{\pgfqpoint{2.544650in}{5.049689in}}%
\pgfpathlineto{\pgfqpoint{2.545770in}{5.016496in}}%
\pgfpathlineto{\pgfqpoint{2.546891in}{5.007613in}}%
\pgfpathlineto{\pgfqpoint{2.548011in}{5.010652in}}%
\pgfpathlineto{\pgfqpoint{2.549132in}{4.995225in}}%
\pgfpathlineto{\pgfqpoint{2.552494in}{4.996978in}}%
\pgfpathlineto{\pgfqpoint{2.553614in}{5.000835in}}%
\pgfpathlineto{\pgfqpoint{2.554735in}{5.007263in}}%
\pgfpathlineto{\pgfqpoint{2.555855in}{5.019535in}}%
\pgfpathlineto{\pgfqpoint{2.556976in}{5.015561in}}%
\pgfpathlineto{\pgfqpoint{2.560337in}{5.022690in}}%
\pgfpathlineto{\pgfqpoint{2.562578in}{5.010535in}}%
\pgfpathlineto{\pgfqpoint{2.564819in}{5.013224in}}%
\pgfpathlineto{\pgfqpoint{2.569301in}{5.034144in}}%
\pgfpathlineto{\pgfqpoint{2.570422in}{5.069324in}}%
\pgfpathlineto{\pgfqpoint{2.576025in}{5.029236in}}%
\pgfpathlineto{\pgfqpoint{2.577145in}{5.026314in}}%
\pgfpathlineto{\pgfqpoint{2.578266in}{5.026664in}}%
\pgfpathlineto{\pgfqpoint{2.579386in}{5.029002in}}%
\pgfpathlineto{\pgfqpoint{2.580507in}{5.024561in}}%
\pgfpathlineto{\pgfqpoint{2.583868in}{5.020587in}}%
\pgfpathlineto{\pgfqpoint{2.584989in}{4.996043in}}%
\pgfpathlineto{\pgfqpoint{2.586109in}{4.999666in}}%
\pgfpathlineto{\pgfqpoint{2.588350in}{5.011120in}}%
\pgfpathlineto{\pgfqpoint{2.591712in}{5.000250in}}%
\pgfpathlineto{\pgfqpoint{2.592832in}{4.993238in}}%
\pgfpathlineto{\pgfqpoint{2.593953in}{4.980381in}}%
\pgfpathlineto{\pgfqpoint{2.595074in}{4.981316in}}%
\pgfpathlineto{\pgfqpoint{2.596194in}{4.987628in}}%
\pgfpathlineto{\pgfqpoint{2.599556in}{4.987394in}}%
\pgfpathlineto{\pgfqpoint{2.600676in}{4.988679in}}%
\pgfpathlineto{\pgfqpoint{2.601797in}{4.978278in}}%
\pgfpathlineto{\pgfqpoint{2.602917in}{4.976992in}}%
\pgfpathlineto{\pgfqpoint{2.604038in}{4.990783in}}%
\pgfpathlineto{\pgfqpoint{2.608520in}{5.030287in}}%
\pgfpathlineto{\pgfqpoint{2.609640in}{5.020470in}}%
\pgfpathlineto{\pgfqpoint{2.610761in}{5.030287in}}%
\pgfpathlineto{\pgfqpoint{2.611881in}{5.030171in}}%
\pgfpathlineto{\pgfqpoint{2.615243in}{5.031690in}}%
\pgfpathlineto{\pgfqpoint{2.617484in}{5.023976in}}%
\pgfpathlineto{\pgfqpoint{2.618605in}{5.025379in}}%
\pgfpathlineto{\pgfqpoint{2.619725in}{5.030989in}}%
\pgfpathlineto{\pgfqpoint{2.624207in}{5.030521in}}%
\pgfpathlineto{\pgfqpoint{2.625328in}{5.021639in}}%
\pgfpathlineto{\pgfqpoint{2.626448in}{5.025963in}}%
\pgfpathlineto{\pgfqpoint{2.627569in}{5.022924in}}%
\pgfpathlineto{\pgfqpoint{2.632051in}{5.029703in}}%
\pgfpathlineto{\pgfqpoint{2.633171in}{5.027716in}}%
\pgfpathlineto{\pgfqpoint{2.634292in}{5.040806in}}%
\pgfpathlineto{\pgfqpoint{2.635413in}{5.034729in}}%
\pgfpathlineto{\pgfqpoint{2.638774in}{5.034495in}}%
\pgfpathlineto{\pgfqpoint{2.639895in}{5.033209in}}%
\pgfpathlineto{\pgfqpoint{2.641015in}{5.017665in}}%
\pgfpathlineto{\pgfqpoint{2.642136in}{5.016496in}}%
\pgfpathlineto{\pgfqpoint{2.643256in}{5.016262in}}%
\pgfpathlineto{\pgfqpoint{2.647738in}{5.019184in}}%
\pgfpathlineto{\pgfqpoint{2.648859in}{5.017314in}}%
\pgfpathlineto{\pgfqpoint{2.649979in}{5.011938in}}%
\pgfpathlineto{\pgfqpoint{2.651100in}{5.011353in}}%
\pgfpathlineto{\pgfqpoint{2.654461in}{5.008782in}}%
\pgfpathlineto{\pgfqpoint{2.655582in}{4.985407in}}%
\pgfpathlineto{\pgfqpoint{2.656703in}{4.996744in}}%
\pgfpathlineto{\pgfqpoint{2.657823in}{4.986225in}}%
\pgfpathlineto{\pgfqpoint{2.658944in}{5.003055in}}%
\pgfpathlineto{\pgfqpoint{2.662305in}{5.000250in}}%
\pgfpathlineto{\pgfqpoint{2.663426in}{5.001536in}}%
\pgfpathlineto{\pgfqpoint{2.664546in}{5.001419in}}%
\pgfpathlineto{\pgfqpoint{2.665667in}{5.005042in}}%
\pgfpathlineto{\pgfqpoint{2.666787in}{5.005860in}}%
\pgfpathlineto{\pgfqpoint{2.670149in}{5.003172in}}%
\pgfpathlineto{\pgfqpoint{2.671269in}{5.004107in}}%
\pgfpathlineto{\pgfqpoint{2.672390in}{5.003172in}}%
\pgfpathlineto{\pgfqpoint{2.673510in}{5.010652in}}%
\pgfpathlineto{\pgfqpoint{2.674631in}{5.023158in}}%
\pgfpathlineto{\pgfqpoint{2.677993in}{5.030171in}}%
\pgfpathlineto{\pgfqpoint{2.679113in}{5.035430in}}%
\pgfpathlineto{\pgfqpoint{2.682475in}{5.062545in}}%
\pgfpathlineto{\pgfqpoint{2.686957in}{5.071194in}}%
\pgfpathlineto{\pgfqpoint{2.688077in}{5.069675in}}%
\pgfpathlineto{\pgfqpoint{2.690318in}{5.113386in}}%
\pgfpathlineto{\pgfqpoint{2.693680in}{5.109763in}}%
\pgfpathlineto{\pgfqpoint{2.694800in}{5.107660in}}%
\pgfpathlineto{\pgfqpoint{2.695921in}{5.125775in}}%
\pgfpathlineto{\pgfqpoint{2.697042in}{5.123204in}}%
\pgfpathlineto{\pgfqpoint{2.698162in}{5.124957in}}%
\pgfpathlineto{\pgfqpoint{2.701524in}{5.124139in}}%
\pgfpathlineto{\pgfqpoint{2.702644in}{5.125425in}}%
\pgfpathlineto{\pgfqpoint{2.703765in}{5.128347in}}%
\pgfpathlineto{\pgfqpoint{2.704885in}{5.148566in}}%
\pgfpathlineto{\pgfqpoint{2.706006in}{5.151371in}}%
\pgfpathlineto{\pgfqpoint{2.709367in}{5.156514in}}%
\pgfpathlineto{\pgfqpoint{2.710488in}{5.160604in}}%
\pgfpathlineto{\pgfqpoint{2.711608in}{5.181525in}}%
\pgfpathlineto{\pgfqpoint{2.713849in}{5.171708in}}%
\pgfpathlineto{\pgfqpoint{2.717211in}{5.171708in}}%
\pgfpathlineto{\pgfqpoint{2.720573in}{5.147982in}}%
\pgfpathlineto{\pgfqpoint{2.721693in}{5.143424in}}%
\pgfpathlineto{\pgfqpoint{2.725055in}{5.146930in}}%
\pgfpathlineto{\pgfqpoint{2.726175in}{5.145411in}}%
\pgfpathlineto{\pgfqpoint{2.727296in}{5.137346in}}%
\pgfpathlineto{\pgfqpoint{2.728416in}{5.134424in}}%
\pgfpathlineto{\pgfqpoint{2.729537in}{5.133255in}}%
\pgfpathlineto{\pgfqpoint{2.732898in}{5.134775in}}%
\pgfpathlineto{\pgfqpoint{2.734019in}{5.134658in}}%
\pgfpathlineto{\pgfqpoint{2.735139in}{5.136060in}}%
\pgfpathlineto{\pgfqpoint{2.736260in}{5.138749in}}%
\pgfpathlineto{\pgfqpoint{2.737381in}{5.137346in}}%
\pgfpathlineto{\pgfqpoint{2.741863in}{5.129632in}}%
\pgfpathlineto{\pgfqpoint{2.742983in}{5.142606in}}%
\pgfpathlineto{\pgfqpoint{2.744104in}{5.138047in}}%
\pgfpathlineto{\pgfqpoint{2.748586in}{5.146112in}}%
\pgfpathlineto{\pgfqpoint{2.749706in}{5.103452in}}%
\pgfpathlineto{\pgfqpoint{2.750827in}{5.098543in}}%
\pgfpathlineto{\pgfqpoint{2.751947in}{5.103919in}}%
\pgfpathlineto{\pgfqpoint{2.753068in}{5.102751in}}%
\pgfpathlineto{\pgfqpoint{2.758671in}{5.121918in}}%
\pgfpathlineto{\pgfqpoint{2.759791in}{5.124373in}}%
\pgfpathlineto{\pgfqpoint{2.760912in}{5.121451in}}%
\pgfpathlineto{\pgfqpoint{2.764273in}{5.120048in}}%
\pgfpathlineto{\pgfqpoint{2.765394in}{5.124022in}}%
\pgfpathlineto{\pgfqpoint{2.766514in}{5.119932in}}%
\pgfpathlineto{\pgfqpoint{2.767635in}{5.126710in}}%
\pgfpathlineto{\pgfqpoint{2.768755in}{5.121918in}}%
\pgfpathlineto{\pgfqpoint{2.773237in}{5.118646in}}%
\pgfpathlineto{\pgfqpoint{2.774358in}{5.114205in}}%
\pgfpathlineto{\pgfqpoint{2.776599in}{5.123321in}}%
\pgfpathlineto{\pgfqpoint{2.781081in}{5.168435in}}%
\pgfpathlineto{\pgfqpoint{2.782202in}{5.156514in}}%
\pgfpathlineto{\pgfqpoint{2.783322in}{5.159786in}}%
\pgfpathlineto{\pgfqpoint{2.784443in}{5.160020in}}%
\pgfpathlineto{\pgfqpoint{2.787804in}{5.162942in}}%
\pgfpathlineto{\pgfqpoint{2.788925in}{5.165747in}}%
\pgfpathlineto{\pgfqpoint{2.790045in}{5.165630in}}%
\pgfpathlineto{\pgfqpoint{2.791166in}{5.175097in}}%
\pgfpathlineto{\pgfqpoint{2.792286in}{5.168435in}}%
\pgfpathlineto{\pgfqpoint{2.796768in}{5.170539in}}%
\pgfpathlineto{\pgfqpoint{2.797889in}{5.183045in}}%
\pgfpathlineto{\pgfqpoint{2.799009in}{5.188889in}}%
\pgfpathlineto{\pgfqpoint{2.800130in}{5.203147in}}%
\pgfpathlineto{\pgfqpoint{2.803492in}{5.206303in}}%
\pgfpathlineto{\pgfqpoint{2.804612in}{5.211446in}}%
\pgfpathlineto{\pgfqpoint{2.805733in}{5.210511in}}%
\pgfpathlineto{\pgfqpoint{2.806853in}{5.208290in}}%
\pgfpathlineto{\pgfqpoint{2.807974in}{5.219159in}}%
\pgfpathlineto{\pgfqpoint{2.811335in}{5.222315in}}%
\pgfpathlineto{\pgfqpoint{2.812456in}{5.224536in}}%
\pgfpathlineto{\pgfqpoint{2.813576in}{5.232834in}}%
\pgfpathlineto{\pgfqpoint{2.814697in}{5.235756in}}%
\pgfpathlineto{\pgfqpoint{2.815817in}{5.250132in}}%
\pgfpathlineto{\pgfqpoint{2.819179in}{5.247093in}}%
\pgfpathlineto{\pgfqpoint{2.820300in}{5.248729in}}%
\pgfpathlineto{\pgfqpoint{2.821420in}{5.256326in}}%
\pgfpathlineto{\pgfqpoint{2.822541in}{5.268832in}}%
\pgfpathlineto{\pgfqpoint{2.823661in}{5.273039in}}%
\pgfpathlineto{\pgfqpoint{2.827023in}{5.272104in}}%
\pgfpathlineto{\pgfqpoint{2.830384in}{5.231315in}}%
\pgfpathlineto{\pgfqpoint{2.831505in}{5.227458in}}%
\pgfpathlineto{\pgfqpoint{2.834866in}{5.234237in}}%
\pgfpathlineto{\pgfqpoint{2.837107in}{5.242301in}}%
\pgfpathlineto{\pgfqpoint{2.838228in}{5.230029in}}%
\pgfpathlineto{\pgfqpoint{2.839348in}{5.230263in}}%
\pgfpathlineto{\pgfqpoint{2.843831in}{5.215887in}}%
\pgfpathlineto{\pgfqpoint{2.844951in}{5.226873in}}%
\pgfpathlineto{\pgfqpoint{2.846072in}{5.222783in}}%
\pgfpathlineto{\pgfqpoint{2.847192in}{5.230964in}}%
\pgfpathlineto{\pgfqpoint{2.850554in}{5.225938in}}%
\pgfpathlineto{\pgfqpoint{2.851674in}{5.251417in}}%
\pgfpathlineto{\pgfqpoint{2.852795in}{5.259599in}}%
\pgfpathlineto{\pgfqpoint{2.853915in}{5.274559in}}%
\pgfpathlineto{\pgfqpoint{2.855036in}{5.260767in}}%
\pgfpathlineto{\pgfqpoint{2.859518in}{5.223016in}}%
\pgfpathlineto{\pgfqpoint{2.860638in}{5.212731in}}%
\pgfpathlineto{\pgfqpoint{2.861759in}{5.211446in}}%
\pgfpathlineto{\pgfqpoint{2.862880in}{5.222666in}}%
\pgfpathlineto{\pgfqpoint{2.866241in}{5.232250in}}%
\pgfpathlineto{\pgfqpoint{2.867362in}{5.229912in}}%
\pgfpathlineto{\pgfqpoint{2.868482in}{5.226055in}}%
\pgfpathlineto{\pgfqpoint{2.869603in}{5.239145in}}%
\pgfpathlineto{\pgfqpoint{2.870723in}{5.237275in}}%
\pgfpathlineto{\pgfqpoint{2.874085in}{5.233769in}}%
\pgfpathlineto{\pgfqpoint{2.875205in}{5.227224in}}%
\pgfpathlineto{\pgfqpoint{2.876326in}{5.237860in}}%
\pgfpathlineto{\pgfqpoint{2.877446in}{5.236340in}}%
\pgfpathlineto{\pgfqpoint{2.878567in}{5.236457in}}%
\pgfpathlineto{\pgfqpoint{2.881929in}{5.240548in}}%
\pgfpathlineto{\pgfqpoint{2.883049in}{5.239496in}}%
\pgfpathlineto{\pgfqpoint{2.884170in}{5.248145in}}%
\pgfpathlineto{\pgfqpoint{2.885290in}{5.235756in}}%
\pgfpathlineto{\pgfqpoint{2.886411in}{5.231198in}}%
\pgfpathlineto{\pgfqpoint{2.889772in}{5.240314in}}%
\pgfpathlineto{\pgfqpoint{2.890893in}{5.253989in}}%
\pgfpathlineto{\pgfqpoint{2.892013in}{5.233068in}}%
\pgfpathlineto{\pgfqpoint{2.893134in}{5.234120in}}%
\pgfpathlineto{\pgfqpoint{2.894254in}{5.230029in}}%
\pgfpathlineto{\pgfqpoint{2.897616in}{5.230613in}}%
\pgfpathlineto{\pgfqpoint{2.898736in}{5.235873in}}%
\pgfpathlineto{\pgfqpoint{2.899857in}{5.223250in}}%
\pgfpathlineto{\pgfqpoint{2.900977in}{5.237626in}}%
\pgfpathlineto{\pgfqpoint{2.902098in}{5.222783in}}%
\pgfpathlineto{\pgfqpoint{2.906580in}{5.210277in}}%
\pgfpathlineto{\pgfqpoint{2.907701in}{5.218809in}}%
\pgfpathlineto{\pgfqpoint{2.908821in}{5.235639in}}%
\pgfpathlineto{\pgfqpoint{2.909942in}{5.222198in}}%
\pgfpathlineto{\pgfqpoint{2.913303in}{5.246976in}}%
\pgfpathlineto{\pgfqpoint{2.914424in}{5.240548in}}%
\pgfpathlineto{\pgfqpoint{2.915544in}{5.238561in}}%
\pgfpathlineto{\pgfqpoint{2.916665in}{5.257729in}}%
\pgfpathlineto{\pgfqpoint{2.921147in}{5.270936in}}%
\pgfpathlineto{\pgfqpoint{2.922267in}{5.269183in}}%
\pgfpathlineto{\pgfqpoint{2.924509in}{5.230730in}}%
\pgfpathlineto{\pgfqpoint{2.925629in}{5.226756in}}%
\pgfpathlineto{\pgfqpoint{2.928991in}{5.224302in}}%
\pgfpathlineto{\pgfqpoint{2.930111in}{5.221848in}}%
\pgfpathlineto{\pgfqpoint{2.931232in}{5.208641in}}%
\pgfpathlineto{\pgfqpoint{2.932352in}{5.205485in}}%
\pgfpathlineto{\pgfqpoint{2.933473in}{5.211446in}}%
\pgfpathlineto{\pgfqpoint{2.936834in}{5.224886in}}%
\pgfpathlineto{\pgfqpoint{2.939075in}{5.243353in}}%
\pgfpathlineto{\pgfqpoint{2.940196in}{5.246742in}}%
\pgfpathlineto{\pgfqpoint{2.941316in}{5.247093in}}%
\pgfpathlineto{\pgfqpoint{2.944678in}{5.249547in}}%
\pgfpathlineto{\pgfqpoint{2.945799in}{5.254573in}}%
\pgfpathlineto{\pgfqpoint{2.946919in}{5.285078in}}%
\pgfpathlineto{\pgfqpoint{2.948040in}{5.287065in}}%
\pgfpathlineto{\pgfqpoint{2.949160in}{5.282623in}}%
\pgfpathlineto{\pgfqpoint{2.952522in}{5.279234in}}%
\pgfpathlineto{\pgfqpoint{2.953642in}{5.330893in}}%
\pgfpathlineto{\pgfqpoint{2.954763in}{5.329724in}}%
\pgfpathlineto{\pgfqpoint{2.955883in}{5.344801in}}%
\pgfpathlineto{\pgfqpoint{2.960365in}{5.362216in}}%
\pgfpathlineto{\pgfqpoint{2.961486in}{5.340360in}}%
\pgfpathlineto{\pgfqpoint{2.962606in}{5.348308in}}%
\pgfpathlineto{\pgfqpoint{2.963727in}{5.342230in}}%
\pgfpathlineto{\pgfqpoint{2.964848in}{5.341880in}}%
\pgfpathlineto{\pgfqpoint{2.969330in}{5.315582in}}%
\pgfpathlineto{\pgfqpoint{2.970450in}{5.321894in}}%
\pgfpathlineto{\pgfqpoint{2.971571in}{5.321426in}}%
\pgfpathlineto{\pgfqpoint{2.972691in}{5.323062in}}%
\pgfpathlineto{\pgfqpoint{2.976053in}{5.319439in}}%
\pgfpathlineto{\pgfqpoint{2.977173in}{5.319556in}}%
\pgfpathlineto{\pgfqpoint{2.978294in}{5.336737in}}%
\pgfpathlineto{\pgfqpoint{2.980535in}{5.317336in}}%
\pgfpathlineto{\pgfqpoint{2.983896in}{5.319439in}}%
\pgfpathlineto{\pgfqpoint{2.987258in}{5.309622in}}%
\pgfpathlineto{\pgfqpoint{2.988379in}{5.300038in}}%
\pgfpathlineto{\pgfqpoint{2.991740in}{5.299220in}}%
\pgfpathlineto{\pgfqpoint{2.992861in}{5.303778in}}%
\pgfpathlineto{\pgfqpoint{2.993981in}{5.292090in}}%
\pgfpathlineto{\pgfqpoint{2.999584in}{5.310440in}}%
\pgfpathlineto{\pgfqpoint{3.000704in}{5.331828in}}%
\pgfpathlineto{\pgfqpoint{3.001825in}{5.329491in}}%
\pgfpathlineto{\pgfqpoint{3.002945in}{5.324231in}}%
\pgfpathlineto{\pgfqpoint{3.004066in}{5.331478in}}%
\pgfpathlineto{\pgfqpoint{3.007428in}{5.320608in}}%
\pgfpathlineto{\pgfqpoint{3.008548in}{5.327971in}}%
\pgfpathlineto{\pgfqpoint{3.009669in}{5.343516in}}%
\pgfpathlineto{\pgfqpoint{3.010789in}{5.331711in}}%
\pgfpathlineto{\pgfqpoint{3.011910in}{5.338256in}}%
\pgfpathlineto{\pgfqpoint{3.015271in}{5.344334in}}%
\pgfpathlineto{\pgfqpoint{3.016392in}{5.360580in}}%
\pgfpathlineto{\pgfqpoint{3.017512in}{5.363852in}}%
\pgfpathlineto{\pgfqpoint{3.018633in}{5.350061in}}%
\pgfpathlineto{\pgfqpoint{3.019753in}{5.359060in}}%
\pgfpathlineto{\pgfqpoint{3.023115in}{5.351697in}}%
\pgfpathlineto{\pgfqpoint{3.024235in}{5.351463in}}%
\pgfpathlineto{\pgfqpoint{3.025356in}{5.344568in}}%
\pgfpathlineto{\pgfqpoint{3.026477in}{5.343516in}}%
\pgfpathlineto{\pgfqpoint{3.027597in}{5.332997in}}%
\pgfpathlineto{\pgfqpoint{3.032079in}{5.332646in}}%
\pgfpathlineto{\pgfqpoint{3.033200in}{5.338023in}}%
\pgfpathlineto{\pgfqpoint{3.034320in}{5.337906in}}%
\pgfpathlineto{\pgfqpoint{3.035441in}{5.328556in}}%
\pgfpathlineto{\pgfqpoint{3.035441in}{5.328556in}}%
\pgfusepath{stroke}%
\end{pgfscope}%
\begin{pgfscope}%
\pgfpathrectangle{\pgfqpoint{0.462318in}{4.309196in}}{\pgfqpoint{2.695652in}{1.104878in}}%
\pgfusepath{clip}%
\pgfsetroundcap%
\pgfsetroundjoin%
\pgfsetlinewidth{1.505625pt}%
\definecolor{currentstroke}{rgb}{1.000000,0.647059,0.000000}%
\pgfsetstrokecolor{currentstroke}%
\pgfsetdash{}{0pt}%
\pgfpathmoveto{\pgfqpoint{0.584848in}{4.387001in}}%
\pgfpathlineto{\pgfqpoint{0.587089in}{4.384273in}}%
\pgfpathlineto{\pgfqpoint{0.588209in}{4.382501in}}%
\pgfpathlineto{\pgfqpoint{0.594933in}{4.381215in}}%
\pgfpathlineto{\pgfqpoint{0.615102in}{4.381427in}}%
\pgfpathlineto{\pgfqpoint{0.619584in}{4.382117in}}%
\pgfpathlineto{\pgfqpoint{0.630789in}{4.381382in}}%
\pgfpathlineto{\pgfqpoint{0.635272in}{4.380831in}}%
\pgfpathlineto{\pgfqpoint{0.662164in}{4.380974in}}%
\pgfpathlineto{\pgfqpoint{0.671128in}{4.381458in}}%
\pgfpathlineto{\pgfqpoint{0.682334in}{4.382026in}}%
\pgfpathlineto{\pgfqpoint{0.695780in}{4.382385in}}%
\pgfpathlineto{\pgfqpoint{0.717070in}{4.381187in}}%
\pgfpathlineto{\pgfqpoint{0.729396in}{4.381333in}}%
\pgfpathlineto{\pgfqpoint{0.741722in}{4.380583in}}%
\pgfpathlineto{\pgfqpoint{0.752927in}{4.379611in}}%
\pgfpathlineto{\pgfqpoint{0.768614in}{4.379218in}}%
\pgfpathlineto{\pgfqpoint{0.775337in}{4.380096in}}%
\pgfpathlineto{\pgfqpoint{0.784302in}{4.381405in}}%
\pgfpathlineto{\pgfqpoint{0.795507in}{4.382783in}}%
\pgfpathlineto{\pgfqpoint{0.807833in}{4.385452in}}%
\pgfpathlineto{\pgfqpoint{0.814556in}{4.386363in}}%
\pgfpathlineto{\pgfqpoint{0.815676in}{4.386688in}}%
\pgfpathlineto{\pgfqpoint{0.822400in}{4.387841in}}%
\pgfpathlineto{\pgfqpoint{0.823520in}{4.388119in}}%
\pgfpathlineto{\pgfqpoint{0.834725in}{4.389478in}}%
\pgfpathlineto{\pgfqpoint{0.839208in}{4.390272in}}%
\pgfpathlineto{\pgfqpoint{0.850413in}{4.391262in}}%
\pgfpathlineto{\pgfqpoint{0.854895in}{4.391908in}}%
\pgfpathlineto{\pgfqpoint{0.866100in}{4.392758in}}%
\pgfpathlineto{\pgfqpoint{0.878426in}{4.394592in}}%
\pgfpathlineto{\pgfqpoint{0.885149in}{4.395454in}}%
\pgfpathlineto{\pgfqpoint{0.894113in}{4.396702in}}%
\pgfpathlineto{\pgfqpoint{0.906439in}{4.397876in}}%
\pgfpathlineto{\pgfqpoint{0.917644in}{4.400140in}}%
\pgfpathlineto{\pgfqpoint{0.928850in}{4.401144in}}%
\pgfpathlineto{\pgfqpoint{0.933332in}{4.401971in}}%
\pgfpathlineto{\pgfqpoint{0.945658in}{4.403094in}}%
\pgfpathlineto{\pgfqpoint{0.956863in}{4.404289in}}%
\pgfpathlineto{\pgfqpoint{0.964707in}{4.405254in}}%
\pgfpathlineto{\pgfqpoint{0.971430in}{4.406135in}}%
\pgfpathlineto{\pgfqpoint{0.980394in}{4.407369in}}%
\pgfpathlineto{\pgfqpoint{0.994961in}{4.408600in}}%
\pgfpathlineto{\pgfqpoint{1.003925in}{4.409954in}}%
\pgfpathlineto{\pgfqpoint{1.010648in}{4.410946in}}%
\pgfpathlineto{\pgfqpoint{1.011769in}{4.411213in}}%
\pgfpathlineto{\pgfqpoint{1.018492in}{4.411961in}}%
\pgfpathlineto{\pgfqpoint{1.019612in}{4.412244in}}%
\pgfpathlineto{\pgfqpoint{1.025215in}{4.413095in}}%
\pgfpathlineto{\pgfqpoint{1.035300in}{4.415167in}}%
\pgfpathlineto{\pgfqpoint{1.040902in}{4.416121in}}%
\pgfpathlineto{\pgfqpoint{1.043143in}{4.416770in}}%
\pgfpathlineto{\pgfqpoint{1.049867in}{4.417813in}}%
\pgfpathlineto{\pgfqpoint{1.050987in}{4.418157in}}%
\pgfpathlineto{\pgfqpoint{1.056590in}{4.419136in}}%
\pgfpathlineto{\pgfqpoint{1.058831in}{4.419813in}}%
\pgfpathlineto{\pgfqpoint{1.064433in}{4.420913in}}%
\pgfpathlineto{\pgfqpoint{1.066675in}{4.421669in}}%
\pgfpathlineto{\pgfqpoint{1.072277in}{4.422839in}}%
\pgfpathlineto{\pgfqpoint{1.074518in}{4.423647in}}%
\pgfpathlineto{\pgfqpoint{1.080121in}{4.424829in}}%
\pgfpathlineto{\pgfqpoint{1.082362in}{4.425625in}}%
\pgfpathlineto{\pgfqpoint{1.086844in}{4.426468in}}%
\pgfpathlineto{\pgfqpoint{1.089085in}{4.427371in}}%
\pgfpathlineto{\pgfqpoint{1.094688in}{4.428319in}}%
\pgfpathlineto{\pgfqpoint{1.098049in}{4.429707in}}%
\pgfpathlineto{\pgfqpoint{1.102531in}{4.430564in}}%
\pgfpathlineto{\pgfqpoint{1.105893in}{4.431926in}}%
\pgfpathlineto{\pgfqpoint{1.110375in}{4.432832in}}%
\pgfpathlineto{\pgfqpoint{1.113737in}{4.434287in}}%
\pgfpathlineto{\pgfqpoint{1.118219in}{4.435319in}}%
\pgfpathlineto{\pgfqpoint{1.121580in}{4.436822in}}%
\pgfpathlineto{\pgfqpoint{1.126062in}{4.437838in}}%
\pgfpathlineto{\pgfqpoint{1.129424in}{4.439309in}}%
\pgfpathlineto{\pgfqpoint{1.133906in}{4.440278in}}%
\pgfpathlineto{\pgfqpoint{1.137268in}{4.441741in}}%
\pgfpathlineto{\pgfqpoint{1.141750in}{4.442756in}}%
\pgfpathlineto{\pgfqpoint{1.145111in}{4.444369in}}%
\pgfpathlineto{\pgfqpoint{1.149594in}{4.445461in}}%
\pgfpathlineto{\pgfqpoint{1.152955in}{4.447045in}}%
\pgfpathlineto{\pgfqpoint{1.158558in}{4.448049in}}%
\pgfpathlineto{\pgfqpoint{1.160799in}{4.448956in}}%
\pgfpathlineto{\pgfqpoint{1.166401in}{4.450222in}}%
\pgfpathlineto{\pgfqpoint{1.168642in}{4.451083in}}%
\pgfpathlineto{\pgfqpoint{1.174245in}{4.452343in}}%
\pgfpathlineto{\pgfqpoint{1.176486in}{4.453193in}}%
\pgfpathlineto{\pgfqpoint{1.180968in}{4.454091in}}%
\pgfpathlineto{\pgfqpoint{1.184330in}{4.455226in}}%
\pgfpathlineto{\pgfqpoint{1.188812in}{4.456048in}}%
\pgfpathlineto{\pgfqpoint{1.192174in}{4.457385in}}%
\pgfpathlineto{\pgfqpoint{1.196656in}{4.458269in}}%
\pgfpathlineto{\pgfqpoint{1.200017in}{4.459181in}}%
\pgfpathlineto{\pgfqpoint{1.204499in}{4.460157in}}%
\pgfpathlineto{\pgfqpoint{1.207861in}{4.461671in}}%
\pgfpathlineto{\pgfqpoint{1.212343in}{4.462706in}}%
\pgfpathlineto{\pgfqpoint{1.215705in}{4.464273in}}%
\pgfpathlineto{\pgfqpoint{1.220187in}{4.465380in}}%
\pgfpathlineto{\pgfqpoint{1.223548in}{4.467037in}}%
\pgfpathlineto{\pgfqpoint{1.228030in}{4.468158in}}%
\pgfpathlineto{\pgfqpoint{1.231392in}{4.469870in}}%
\pgfpathlineto{\pgfqpoint{1.235874in}{4.470993in}}%
\pgfpathlineto{\pgfqpoint{1.239236in}{4.472606in}}%
\pgfpathlineto{\pgfqpoint{1.243718in}{4.473639in}}%
\pgfpathlineto{\pgfqpoint{1.247079in}{4.474978in}}%
\pgfpathlineto{\pgfqpoint{1.252682in}{4.476282in}}%
\pgfpathlineto{\pgfqpoint{1.254923in}{4.477094in}}%
\pgfpathlineto{\pgfqpoint{1.260526in}{4.478207in}}%
\pgfpathlineto{\pgfqpoint{1.262767in}{4.478928in}}%
\pgfpathlineto{\pgfqpoint{1.269490in}{4.480017in}}%
\pgfpathlineto{\pgfqpoint{1.270610in}{4.480386in}}%
\pgfpathlineto{\pgfqpoint{1.276213in}{4.481571in}}%
\pgfpathlineto{\pgfqpoint{1.278454in}{4.482367in}}%
\pgfpathlineto{\pgfqpoint{1.284057in}{4.483587in}}%
\pgfpathlineto{\pgfqpoint{1.286298in}{4.484414in}}%
\pgfpathlineto{\pgfqpoint{1.291900in}{4.485523in}}%
\pgfpathlineto{\pgfqpoint{1.294142in}{4.486198in}}%
\pgfpathlineto{\pgfqpoint{1.299744in}{4.487216in}}%
\pgfpathlineto{\pgfqpoint{1.301985in}{4.487877in}}%
\pgfpathlineto{\pgfqpoint{1.307588in}{4.488798in}}%
\pgfpathlineto{\pgfqpoint{1.309829in}{4.489520in}}%
\pgfpathlineto{\pgfqpoint{1.315432in}{4.490699in}}%
\pgfpathlineto{\pgfqpoint{1.317673in}{4.491543in}}%
\pgfpathlineto{\pgfqpoint{1.322155in}{4.492380in}}%
\pgfpathlineto{\pgfqpoint{1.325516in}{4.493646in}}%
\pgfpathlineto{\pgfqpoint{1.329998in}{4.494507in}}%
\pgfpathlineto{\pgfqpoint{1.333360in}{4.495794in}}%
\pgfpathlineto{\pgfqpoint{1.337842in}{4.496642in}}%
\pgfpathlineto{\pgfqpoint{1.341204in}{4.497923in}}%
\pgfpathlineto{\pgfqpoint{1.346806in}{4.499217in}}%
\pgfpathlineto{\pgfqpoint{1.349047in}{4.500088in}}%
\pgfpathlineto{\pgfqpoint{1.353529in}{4.500970in}}%
\pgfpathlineto{\pgfqpoint{1.356891in}{4.502332in}}%
\pgfpathlineto{\pgfqpoint{1.361373in}{4.503254in}}%
\pgfpathlineto{\pgfqpoint{1.364735in}{4.504147in}}%
\pgfpathlineto{\pgfqpoint{1.370337in}{4.505420in}}%
\pgfpathlineto{\pgfqpoint{1.372578in}{4.506249in}}%
\pgfpathlineto{\pgfqpoint{1.378181in}{4.507492in}}%
\pgfpathlineto{\pgfqpoint{1.380422in}{4.508205in}}%
\pgfpathlineto{\pgfqpoint{1.386025in}{4.509281in}}%
\pgfpathlineto{\pgfqpoint{1.388266in}{4.510011in}}%
\pgfpathlineto{\pgfqpoint{1.394989in}{4.511105in}}%
\pgfpathlineto{\pgfqpoint{1.396109in}{4.511469in}}%
\pgfpathlineto{\pgfqpoint{1.407315in}{4.513218in}}%
\pgfpathlineto{\pgfqpoint{1.411797in}{4.514801in}}%
\pgfpathlineto{\pgfqpoint{1.417400in}{4.515984in}}%
\pgfpathlineto{\pgfqpoint{1.419641in}{4.516771in}}%
\pgfpathlineto{\pgfqpoint{1.426364in}{4.517874in}}%
\pgfpathlineto{\pgfqpoint{1.427484in}{4.518179in}}%
\pgfpathlineto{\pgfqpoint{1.434207in}{4.519312in}}%
\pgfpathlineto{\pgfqpoint{1.438690in}{4.519797in}}%
\pgfpathlineto{\pgfqpoint{1.496957in}{4.531731in}}%
\pgfpathlineto{\pgfqpoint{1.498077in}{4.532121in}}%
\pgfpathlineto{\pgfqpoint{1.503680in}{4.533321in}}%
\pgfpathlineto{\pgfqpoint{1.505921in}{4.534122in}}%
\pgfpathlineto{\pgfqpoint{1.511524in}{4.535310in}}%
\pgfpathlineto{\pgfqpoint{1.513765in}{4.536039in}}%
\pgfpathlineto{\pgfqpoint{1.519367in}{4.537210in}}%
\pgfpathlineto{\pgfqpoint{1.520488in}{4.537608in}}%
\pgfpathlineto{\pgfqpoint{1.526091in}{4.538439in}}%
\pgfpathlineto{\pgfqpoint{1.529452in}{4.539670in}}%
\pgfpathlineto{\pgfqpoint{1.533934in}{4.540525in}}%
\pgfpathlineto{\pgfqpoint{1.537296in}{4.541754in}}%
\pgfpathlineto{\pgfqpoint{1.542899in}{4.542959in}}%
\pgfpathlineto{\pgfqpoint{1.545140in}{4.543775in}}%
\pgfpathlineto{\pgfqpoint{1.550742in}{4.544993in}}%
\pgfpathlineto{\pgfqpoint{1.552983in}{4.545793in}}%
\pgfpathlineto{\pgfqpoint{1.558586in}{4.546985in}}%
\pgfpathlineto{\pgfqpoint{1.560827in}{4.547806in}}%
\pgfpathlineto{\pgfqpoint{1.566430in}{4.548608in}}%
\pgfpathlineto{\pgfqpoint{1.568671in}{4.549423in}}%
\pgfpathlineto{\pgfqpoint{1.573153in}{4.550273in}}%
\pgfpathlineto{\pgfqpoint{1.576514in}{4.551572in}}%
\pgfpathlineto{\pgfqpoint{1.580996in}{4.552453in}}%
\pgfpathlineto{\pgfqpoint{1.584358in}{4.553720in}}%
\pgfpathlineto{\pgfqpoint{1.589961in}{4.554955in}}%
\pgfpathlineto{\pgfqpoint{1.592202in}{4.555842in}}%
\pgfpathlineto{\pgfqpoint{1.596684in}{4.556728in}}%
\pgfpathlineto{\pgfqpoint{1.600045in}{4.558084in}}%
\pgfpathlineto{\pgfqpoint{1.604528in}{4.558974in}}%
\pgfpathlineto{\pgfqpoint{1.606769in}{4.559871in}}%
\pgfpathlineto{\pgfqpoint{1.612371in}{4.560776in}}%
\pgfpathlineto{\pgfqpoint{1.615733in}{4.562103in}}%
\pgfpathlineto{\pgfqpoint{1.621335in}{4.563320in}}%
\pgfpathlineto{\pgfqpoint{1.623577in}{4.564045in}}%
\pgfpathlineto{\pgfqpoint{1.629179in}{4.565166in}}%
\pgfpathlineto{\pgfqpoint{1.631420in}{4.565913in}}%
\pgfpathlineto{\pgfqpoint{1.637023in}{4.567024in}}%
\pgfpathlineto{\pgfqpoint{1.639264in}{4.567692in}}%
\pgfpathlineto{\pgfqpoint{1.644867in}{4.568698in}}%
\pgfpathlineto{\pgfqpoint{1.647108in}{4.569372in}}%
\pgfpathlineto{\pgfqpoint{1.652710in}{4.570406in}}%
\pgfpathlineto{\pgfqpoint{1.654951in}{4.571103in}}%
\pgfpathlineto{\pgfqpoint{1.660554in}{4.572203in}}%
\pgfpathlineto{\pgfqpoint{1.662795in}{4.572963in}}%
\pgfpathlineto{\pgfqpoint{1.668398in}{4.574094in}}%
\pgfpathlineto{\pgfqpoint{1.670639in}{4.574843in}}%
\pgfpathlineto{\pgfqpoint{1.676241in}{4.575595in}}%
\pgfpathlineto{\pgfqpoint{1.678482in}{4.576360in}}%
\pgfpathlineto{\pgfqpoint{1.684085in}{4.577505in}}%
\pgfpathlineto{\pgfqpoint{1.686326in}{4.578271in}}%
\pgfpathlineto{\pgfqpoint{1.691929in}{4.579459in}}%
\pgfpathlineto{\pgfqpoint{1.694170in}{4.580307in}}%
\pgfpathlineto{\pgfqpoint{1.698652in}{4.581150in}}%
\pgfpathlineto{\pgfqpoint{1.702013in}{4.582402in}}%
\pgfpathlineto{\pgfqpoint{1.707616in}{4.583564in}}%
\pgfpathlineto{\pgfqpoint{1.709857in}{4.584294in}}%
\pgfpathlineto{\pgfqpoint{1.715460in}{4.585362in}}%
\pgfpathlineto{\pgfqpoint{1.717701in}{4.585999in}}%
\pgfpathlineto{\pgfqpoint{1.724424in}{4.587037in}}%
\pgfpathlineto{\pgfqpoint{1.733388in}{4.588839in}}%
\pgfpathlineto{\pgfqpoint{1.738991in}{4.589898in}}%
\pgfpathlineto{\pgfqpoint{1.741232in}{4.590673in}}%
\pgfpathlineto{\pgfqpoint{1.746834in}{4.591866in}}%
\pgfpathlineto{\pgfqpoint{1.749076in}{4.592664in}}%
\pgfpathlineto{\pgfqpoint{1.754678in}{4.593863in}}%
\pgfpathlineto{\pgfqpoint{1.756919in}{4.594651in}}%
\pgfpathlineto{\pgfqpoint{1.762522in}{4.595825in}}%
\pgfpathlineto{\pgfqpoint{1.764763in}{4.596595in}}%
\pgfpathlineto{\pgfqpoint{1.770366in}{4.597711in}}%
\pgfpathlineto{\pgfqpoint{1.784932in}{4.600777in}}%
\pgfpathlineto{\pgfqpoint{1.788294in}{4.601812in}}%
\pgfpathlineto{\pgfqpoint{1.793897in}{4.602749in}}%
\pgfpathlineto{\pgfqpoint{1.796138in}{4.603440in}}%
\pgfpathlineto{\pgfqpoint{1.807343in}{4.605084in}}%
\pgfpathlineto{\pgfqpoint{1.811825in}{4.606044in}}%
\pgfpathlineto{\pgfqpoint{1.817428in}{4.606973in}}%
\pgfpathlineto{\pgfqpoint{1.819669in}{4.607627in}}%
\pgfpathlineto{\pgfqpoint{1.825271in}{4.608552in}}%
\pgfpathlineto{\pgfqpoint{1.827512in}{4.609134in}}%
\pgfpathlineto{\pgfqpoint{1.834236in}{4.609962in}}%
\pgfpathlineto{\pgfqpoint{1.842079in}{4.611313in}}%
\pgfpathlineto{\pgfqpoint{1.851044in}{4.612856in}}%
\pgfpathlineto{\pgfqpoint{1.857767in}{4.613785in}}%
\pgfpathlineto{\pgfqpoint{1.858887in}{4.614014in}}%
\pgfpathlineto{\pgfqpoint{1.866731in}{4.614972in}}%
\pgfpathlineto{\pgfqpoint{1.873454in}{4.615994in}}%
\pgfpathlineto{\pgfqpoint{1.881298in}{4.617336in}}%
\pgfpathlineto{\pgfqpoint{1.890262in}{4.618705in}}%
\pgfpathlineto{\pgfqpoint{1.896985in}{4.619679in}}%
\pgfpathlineto{\pgfqpoint{1.903708in}{4.620688in}}%
\pgfpathlineto{\pgfqpoint{1.912673in}{4.622057in}}%
\pgfpathlineto{\pgfqpoint{1.920516in}{4.622956in}}%
\pgfpathlineto{\pgfqpoint{1.927239in}{4.623883in}}%
\pgfpathlineto{\pgfqpoint{1.937324in}{4.625427in}}%
\pgfpathlineto{\pgfqpoint{1.944047in}{4.626295in}}%
\pgfpathlineto{\pgfqpoint{1.953011in}{4.627569in}}%
\pgfpathlineto{\pgfqpoint{1.959735in}{4.628454in}}%
\pgfpathlineto{\pgfqpoint{1.968699in}{4.629919in}}%
\pgfpathlineto{\pgfqpoint{1.979904in}{4.631022in}}%
\pgfpathlineto{\pgfqpoint{1.992230in}{4.632790in}}%
\pgfpathlineto{\pgfqpoint{2.000074in}{4.633758in}}%
\pgfpathlineto{\pgfqpoint{2.011279in}{4.634926in}}%
\pgfpathlineto{\pgfqpoint{2.023605in}{4.636377in}}%
\pgfpathlineto{\pgfqpoint{2.034810in}{4.637585in}}%
\pgfpathlineto{\pgfqpoint{2.047136in}{4.639295in}}%
\pgfpathlineto{\pgfqpoint{2.058341in}{4.640417in}}%
\pgfpathlineto{\pgfqpoint{2.070667in}{4.641977in}}%
\pgfpathlineto{\pgfqpoint{2.084113in}{4.642947in}}%
\pgfpathlineto{\pgfqpoint{2.102042in}{4.644206in}}%
\pgfpathlineto{\pgfqpoint{2.117729in}{4.645266in}}%
\pgfpathlineto{\pgfqpoint{2.131175in}{4.646290in}}%
\pgfpathlineto{\pgfqpoint{2.149104in}{4.648360in}}%
\pgfpathlineto{\pgfqpoint{2.160309in}{4.649543in}}%
\pgfpathlineto{\pgfqpoint{2.172635in}{4.651267in}}%
\pgfpathlineto{\pgfqpoint{2.184961in}{4.652479in}}%
\pgfpathlineto{\pgfqpoint{2.196166in}{4.654083in}}%
\pgfpathlineto{\pgfqpoint{2.202889in}{4.654941in}}%
\pgfpathlineto{\pgfqpoint{2.210733in}{4.655945in}}%
\pgfpathlineto{\pgfqpoint{2.223059in}{4.656948in}}%
\pgfpathlineto{\pgfqpoint{2.234264in}{4.658195in}}%
\pgfpathlineto{\pgfqpoint{2.243228in}{4.658850in}}%
\pgfpathlineto{\pgfqpoint{2.249951in}{4.659530in}}%
\pgfpathlineto{\pgfqpoint{2.258915in}{4.660724in}}%
\pgfpathlineto{\pgfqpoint{2.272362in}{4.661995in}}%
\pgfpathlineto{\pgfqpoint{2.282446in}{4.663501in}}%
\pgfpathlineto{\pgfqpoint{2.289170in}{4.664422in}}%
\pgfpathlineto{\pgfqpoint{2.298134in}{4.665811in}}%
\pgfpathlineto{\pgfqpoint{2.304857in}{4.666750in}}%
\pgfpathlineto{\pgfqpoint{2.312701in}{4.667934in}}%
\pgfpathlineto{\pgfqpoint{2.320544in}{4.668907in}}%
\pgfpathlineto{\pgfqpoint{2.329509in}{4.670379in}}%
\pgfpathlineto{\pgfqpoint{2.336232in}{4.671370in}}%
\pgfpathlineto{\pgfqpoint{2.345196in}{4.673009in}}%
\pgfpathlineto{\pgfqpoint{2.351919in}{4.674112in}}%
\pgfpathlineto{\pgfqpoint{2.360883in}{4.675732in}}%
\pgfpathlineto{\pgfqpoint{2.366486in}{4.676584in}}%
\pgfpathlineto{\pgfqpoint{2.372089in}{4.677427in}}%
\pgfpathlineto{\pgfqpoint{2.376571in}{4.678518in}}%
\pgfpathlineto{\pgfqpoint{2.383294in}{4.679607in}}%
\pgfpathlineto{\pgfqpoint{2.384414in}{4.679881in}}%
\pgfpathlineto{\pgfqpoint{2.391138in}{4.680707in}}%
\pgfpathlineto{\pgfqpoint{2.400102in}{4.682501in}}%
\pgfpathlineto{\pgfqpoint{2.405704in}{4.683413in}}%
\pgfpathlineto{\pgfqpoint{2.407946in}{4.684004in}}%
\pgfpathlineto{\pgfqpoint{2.413548in}{4.684897in}}%
\pgfpathlineto{\pgfqpoint{2.415789in}{4.685491in}}%
\pgfpathlineto{\pgfqpoint{2.421392in}{4.686423in}}%
\pgfpathlineto{\pgfqpoint{2.423633in}{4.687103in}}%
\pgfpathlineto{\pgfqpoint{2.430356in}{4.688152in}}%
\pgfpathlineto{\pgfqpoint{2.431477in}{4.688503in}}%
\pgfpathlineto{\pgfqpoint{2.437079in}{4.689558in}}%
\pgfpathlineto{\pgfqpoint{2.439320in}{4.690261in}}%
\pgfpathlineto{\pgfqpoint{2.444923in}{4.691351in}}%
\pgfpathlineto{\pgfqpoint{2.447164in}{4.692085in}}%
\pgfpathlineto{\pgfqpoint{2.452767in}{4.693176in}}%
\pgfpathlineto{\pgfqpoint{2.455008in}{4.693899in}}%
\pgfpathlineto{\pgfqpoint{2.460610in}{4.694976in}}%
\pgfpathlineto{\pgfqpoint{2.462851in}{4.695678in}}%
\pgfpathlineto{\pgfqpoint{2.468454in}{4.696711in}}%
\pgfpathlineto{\pgfqpoint{2.470695in}{4.697396in}}%
\pgfpathlineto{\pgfqpoint{2.476298in}{4.698361in}}%
\pgfpathlineto{\pgfqpoint{2.478539in}{4.698990in}}%
\pgfpathlineto{\pgfqpoint{2.484141in}{4.699906in}}%
\pgfpathlineto{\pgfqpoint{2.486382in}{4.700510in}}%
\pgfpathlineto{\pgfqpoint{2.491985in}{4.701432in}}%
\pgfpathlineto{\pgfqpoint{2.494226in}{4.702037in}}%
\pgfpathlineto{\pgfqpoint{2.500949in}{4.702952in}}%
\pgfpathlineto{\pgfqpoint{2.502070in}{4.703244in}}%
\pgfpathlineto{\pgfqpoint{2.507672in}{4.704115in}}%
\pgfpathlineto{\pgfqpoint{2.509913in}{4.704697in}}%
\pgfpathlineto{\pgfqpoint{2.515516in}{4.705560in}}%
\pgfpathlineto{\pgfqpoint{2.517757in}{4.706150in}}%
\pgfpathlineto{\pgfqpoint{2.523360in}{4.707022in}}%
\pgfpathlineto{\pgfqpoint{2.533445in}{4.709027in}}%
\pgfpathlineto{\pgfqpoint{2.540168in}{4.710156in}}%
\pgfpathlineto{\pgfqpoint{2.545770in}{4.710965in}}%
\pgfpathlineto{\pgfqpoint{2.556976in}{4.712912in}}%
\pgfpathlineto{\pgfqpoint{2.563699in}{4.713904in}}%
\pgfpathlineto{\pgfqpoint{2.572663in}{4.715515in}}%
\pgfpathlineto{\pgfqpoint{2.579386in}{4.716532in}}%
\pgfpathlineto{\pgfqpoint{2.586109in}{4.717484in}}%
\pgfpathlineto{\pgfqpoint{2.608520in}{4.720377in}}%
\pgfpathlineto{\pgfqpoint{2.619725in}{4.722336in}}%
\pgfpathlineto{\pgfqpoint{2.632051in}{4.723543in}}%
\pgfpathlineto{\pgfqpoint{2.643256in}{4.725463in}}%
\pgfpathlineto{\pgfqpoint{2.654461in}{4.726596in}}%
\pgfpathlineto{\pgfqpoint{2.666787in}{4.728506in}}%
\pgfpathlineto{\pgfqpoint{2.673510in}{4.729368in}}%
\pgfpathlineto{\pgfqpoint{2.682475in}{4.730819in}}%
\pgfpathlineto{\pgfqpoint{2.689198in}{4.731624in}}%
\pgfpathlineto{\pgfqpoint{2.690318in}{4.731919in}}%
\pgfpathlineto{\pgfqpoint{2.695921in}{4.732803in}}%
\pgfpathlineto{\pgfqpoint{2.698162in}{4.733405in}}%
\pgfpathlineto{\pgfqpoint{2.703765in}{4.734309in}}%
\pgfpathlineto{\pgfqpoint{2.706006in}{4.734946in}}%
\pgfpathlineto{\pgfqpoint{2.711608in}{4.735935in}}%
\pgfpathlineto{\pgfqpoint{2.713849in}{4.736603in}}%
\pgfpathlineto{\pgfqpoint{2.719452in}{4.737576in}}%
\pgfpathlineto{\pgfqpoint{2.721693in}{4.738196in}}%
\pgfpathlineto{\pgfqpoint{2.727296in}{4.739118in}}%
\pgfpathlineto{\pgfqpoint{2.729537in}{4.739716in}}%
\pgfpathlineto{\pgfqpoint{2.735139in}{4.740613in}}%
\pgfpathlineto{\pgfqpoint{2.737381in}{4.741213in}}%
\pgfpathlineto{\pgfqpoint{2.742983in}{4.742101in}}%
\pgfpathlineto{\pgfqpoint{2.744104in}{4.742399in}}%
\pgfpathlineto{\pgfqpoint{2.750827in}{4.743241in}}%
\pgfpathlineto{\pgfqpoint{2.760912in}{4.745188in}}%
\pgfpathlineto{\pgfqpoint{2.766514in}{4.746029in}}%
\pgfpathlineto{\pgfqpoint{2.768755in}{4.746592in}}%
\pgfpathlineto{\pgfqpoint{2.775478in}{4.747693in}}%
\pgfpathlineto{\pgfqpoint{2.784443in}{4.749497in}}%
\pgfpathlineto{\pgfqpoint{2.790045in}{4.750415in}}%
\pgfpathlineto{\pgfqpoint{2.792286in}{4.751035in}}%
\pgfpathlineto{\pgfqpoint{2.799009in}{4.751981in}}%
\pgfpathlineto{\pgfqpoint{2.800130in}{4.752312in}}%
\pgfpathlineto{\pgfqpoint{2.805733in}{4.753316in}}%
\pgfpathlineto{\pgfqpoint{2.807974in}{4.753989in}}%
\pgfpathlineto{\pgfqpoint{2.813576in}{4.755023in}}%
\pgfpathlineto{\pgfqpoint{2.815817in}{4.755734in}}%
\pgfpathlineto{\pgfqpoint{2.821420in}{4.756813in}}%
\pgfpathlineto{\pgfqpoint{2.823661in}{4.757560in}}%
\pgfpathlineto{\pgfqpoint{2.829264in}{4.758647in}}%
\pgfpathlineto{\pgfqpoint{2.831505in}{4.759327in}}%
\pgfpathlineto{\pgfqpoint{2.838228in}{4.760358in}}%
\pgfpathlineto{\pgfqpoint{2.839348in}{4.760697in}}%
\pgfpathlineto{\pgfqpoint{2.844951in}{4.761690in}}%
\pgfpathlineto{\pgfqpoint{2.847192in}{4.762358in}}%
\pgfpathlineto{\pgfqpoint{2.852795in}{4.763397in}}%
\pgfpathlineto{\pgfqpoint{2.855036in}{4.764119in}}%
\pgfpathlineto{\pgfqpoint{2.860638in}{4.765104in}}%
\pgfpathlineto{\pgfqpoint{2.862880in}{4.765749in}}%
\pgfpathlineto{\pgfqpoint{2.868482in}{4.766739in}}%
\pgfpathlineto{\pgfqpoint{2.870723in}{4.767409in}}%
\pgfpathlineto{\pgfqpoint{2.876326in}{4.768400in}}%
\pgfpathlineto{\pgfqpoint{2.878567in}{4.769062in}}%
\pgfpathlineto{\pgfqpoint{2.884170in}{4.770067in}}%
\pgfpathlineto{\pgfqpoint{2.886411in}{4.770721in}}%
\pgfpathlineto{\pgfqpoint{2.892013in}{4.771717in}}%
\pgfpathlineto{\pgfqpoint{2.894254in}{4.772365in}}%
\pgfpathlineto{\pgfqpoint{2.899857in}{4.773328in}}%
\pgfpathlineto{\pgfqpoint{2.902098in}{4.773968in}}%
\pgfpathlineto{\pgfqpoint{2.908821in}{4.774908in}}%
\pgfpathlineto{\pgfqpoint{2.909942in}{4.775220in}}%
\pgfpathlineto{\pgfqpoint{2.915544in}{4.776197in}}%
\pgfpathlineto{\pgfqpoint{2.917785in}{4.776869in}}%
\pgfpathlineto{\pgfqpoint{2.923388in}{4.777882in}}%
\pgfpathlineto{\pgfqpoint{2.925629in}{4.778507in}}%
\pgfpathlineto{\pgfqpoint{2.931232in}{4.779421in}}%
\pgfpathlineto{\pgfqpoint{2.941316in}{4.781597in}}%
\pgfpathlineto{\pgfqpoint{2.946919in}{4.782591in}}%
\pgfpathlineto{\pgfqpoint{2.949160in}{4.783281in}}%
\pgfpathlineto{\pgfqpoint{2.954763in}{4.784370in}}%
\pgfpathlineto{\pgfqpoint{2.957004in}{4.785140in}}%
\pgfpathlineto{\pgfqpoint{2.962606in}{4.786298in}}%
\pgfpathlineto{\pgfqpoint{2.964848in}{4.787057in}}%
\pgfpathlineto{\pgfqpoint{2.970450in}{4.788145in}}%
\pgfpathlineto{\pgfqpoint{2.972691in}{4.788871in}}%
\pgfpathlineto{\pgfqpoint{2.978294in}{4.789962in}}%
\pgfpathlineto{\pgfqpoint{2.980535in}{4.790683in}}%
\pgfpathlineto{\pgfqpoint{2.986138in}{4.791749in}}%
\pgfpathlineto{\pgfqpoint{2.988379in}{4.792441in}}%
\pgfpathlineto{\pgfqpoint{2.993981in}{4.793464in}}%
\pgfpathlineto{\pgfqpoint{3.002945in}{4.795230in}}%
\pgfpathlineto{\pgfqpoint{3.004066in}{4.795590in}}%
\pgfpathlineto{\pgfqpoint{3.009669in}{4.796665in}}%
\pgfpathlineto{\pgfqpoint{3.011910in}{4.797385in}}%
\pgfpathlineto{\pgfqpoint{3.017512in}{4.798504in}}%
\pgfpathlineto{\pgfqpoint{3.019753in}{4.799246in}}%
\pgfpathlineto{\pgfqpoint{3.025356in}{4.800343in}}%
\pgfpathlineto{\pgfqpoint{3.027597in}{4.801058in}}%
\pgfpathlineto{\pgfqpoint{3.034320in}{4.802123in}}%
\pgfpathlineto{\pgfqpoint{3.035441in}{4.802472in}}%
\pgfpathlineto{\pgfqpoint{3.035441in}{4.802472in}}%
\pgfusepath{stroke}%
\end{pgfscope}%
\begin{pgfscope}%
\pgfsetrectcap%
\pgfsetmiterjoin%
\pgfsetlinewidth{0.803000pt}%
\definecolor{currentstroke}{rgb}{1.000000,1.000000,1.000000}%
\pgfsetstrokecolor{currentstroke}%
\pgfsetdash{}{0pt}%
\pgfpathmoveto{\pgfqpoint{0.462318in}{4.309196in}}%
\pgfpathlineto{\pgfqpoint{0.462318in}{5.414074in}}%
\pgfusepath{stroke}%
\end{pgfscope}%
\begin{pgfscope}%
\pgfsetrectcap%
\pgfsetmiterjoin%
\pgfsetlinewidth{0.803000pt}%
\definecolor{currentstroke}{rgb}{1.000000,1.000000,1.000000}%
\pgfsetstrokecolor{currentstroke}%
\pgfsetdash{}{0pt}%
\pgfpathmoveto{\pgfqpoint{3.157970in}{4.309196in}}%
\pgfpathlineto{\pgfqpoint{3.157970in}{5.414074in}}%
\pgfusepath{stroke}%
\end{pgfscope}%
\begin{pgfscope}%
\pgfsetrectcap%
\pgfsetmiterjoin%
\pgfsetlinewidth{0.803000pt}%
\definecolor{currentstroke}{rgb}{1.000000,1.000000,1.000000}%
\pgfsetstrokecolor{currentstroke}%
\pgfsetdash{}{0pt}%
\pgfpathmoveto{\pgfqpoint{0.462318in}{4.309196in}}%
\pgfpathlineto{\pgfqpoint{3.157970in}{4.309196in}}%
\pgfusepath{stroke}%
\end{pgfscope}%
\begin{pgfscope}%
\pgfsetrectcap%
\pgfsetmiterjoin%
\pgfsetlinewidth{0.803000pt}%
\definecolor{currentstroke}{rgb}{1.000000,1.000000,1.000000}%
\pgfsetstrokecolor{currentstroke}%
\pgfsetdash{}{0pt}%
\pgfpathmoveto{\pgfqpoint{0.462318in}{5.414074in}}%
\pgfpathlineto{\pgfqpoint{3.157970in}{5.414074in}}%
\pgfusepath{stroke}%
\end{pgfscope}%
\begin{pgfscope}%
\definecolor{textcolor}{rgb}{0.150000,0.150000,0.150000}%
\pgfsetstrokecolor{textcolor}%
\pgfsetfillcolor{textcolor}%
\pgftext[x=1.810144in,y=5.497407in,,base]{\color{textcolor}\rmfamily\fontsize{12.000000}{14.400000}\selectfont JNJ}%
\end{pgfscope}%
\begin{pgfscope}%
\pgfsetbuttcap%
\pgfsetmiterjoin%
\definecolor{currentfill}{rgb}{0.917647,0.917647,0.949020}%
\pgfsetfillcolor{currentfill}%
\pgfsetlinewidth{0.000000pt}%
\definecolor{currentstroke}{rgb}{0.000000,0.000000,0.000000}%
\pgfsetstrokecolor{currentstroke}%
\pgfsetstrokeopacity{0.000000}%
\pgfsetdash{}{0pt}%
\pgfpathmoveto{\pgfqpoint{3.966666in}{4.309196in}}%
\pgfpathlineto{\pgfqpoint{6.662318in}{4.309196in}}%
\pgfpathlineto{\pgfqpoint{6.662318in}{5.414074in}}%
\pgfpathlineto{\pgfqpoint{3.966666in}{5.414074in}}%
\pgfpathclose%
\pgfusepath{fill}%
\end{pgfscope}%
\begin{pgfscope}%
\pgfpathrectangle{\pgfqpoint{3.966666in}{4.309196in}}{\pgfqpoint{2.695652in}{1.104878in}}%
\pgfusepath{clip}%
\pgfsetroundcap%
\pgfsetroundjoin%
\pgfsetlinewidth{0.803000pt}%
\definecolor{currentstroke}{rgb}{1.000000,1.000000,1.000000}%
\pgfsetstrokecolor{currentstroke}%
\pgfsetdash{}{0pt}%
\pgfpathmoveto{\pgfqpoint{4.086955in}{4.309196in}}%
\pgfpathlineto{\pgfqpoint{4.086955in}{5.414074in}}%
\pgfusepath{stroke}%
\end{pgfscope}%
\begin{pgfscope}%
\definecolor{textcolor}{rgb}{0.150000,0.150000,0.150000}%
\pgfsetstrokecolor{textcolor}%
\pgfsetfillcolor{textcolor}%
\pgftext[x=4.086955in,y=4.211974in,,top]{\color{textcolor}\rmfamily\fontsize{10.000000}{12.000000}\selectfont 2012}%
\end{pgfscope}%
\begin{pgfscope}%
\pgfpathrectangle{\pgfqpoint{3.966666in}{4.309196in}}{\pgfqpoint{2.695652in}{1.104878in}}%
\pgfusepath{clip}%
\pgfsetroundcap%
\pgfsetroundjoin%
\pgfsetlinewidth{0.803000pt}%
\definecolor{currentstroke}{rgb}{1.000000,1.000000,1.000000}%
\pgfsetstrokecolor{currentstroke}%
\pgfsetdash{}{0pt}%
\pgfpathmoveto{\pgfqpoint{4.497068in}{4.309196in}}%
\pgfpathlineto{\pgfqpoint{4.497068in}{5.414074in}}%
\pgfusepath{stroke}%
\end{pgfscope}%
\begin{pgfscope}%
\definecolor{textcolor}{rgb}{0.150000,0.150000,0.150000}%
\pgfsetstrokecolor{textcolor}%
\pgfsetfillcolor{textcolor}%
\pgftext[x=4.497068in,y=4.211974in,,top]{\color{textcolor}\rmfamily\fontsize{10.000000}{12.000000}\selectfont 2013}%
\end{pgfscope}%
\begin{pgfscope}%
\pgfpathrectangle{\pgfqpoint{3.966666in}{4.309196in}}{\pgfqpoint{2.695652in}{1.104878in}}%
\pgfusepath{clip}%
\pgfsetroundcap%
\pgfsetroundjoin%
\pgfsetlinewidth{0.803000pt}%
\definecolor{currentstroke}{rgb}{1.000000,1.000000,1.000000}%
\pgfsetstrokecolor{currentstroke}%
\pgfsetdash{}{0pt}%
\pgfpathmoveto{\pgfqpoint{4.906060in}{4.309196in}}%
\pgfpathlineto{\pgfqpoint{4.906060in}{5.414074in}}%
\pgfusepath{stroke}%
\end{pgfscope}%
\begin{pgfscope}%
\definecolor{textcolor}{rgb}{0.150000,0.150000,0.150000}%
\pgfsetstrokecolor{textcolor}%
\pgfsetfillcolor{textcolor}%
\pgftext[x=4.906060in,y=4.211974in,,top]{\color{textcolor}\rmfamily\fontsize{10.000000}{12.000000}\selectfont 2014}%
\end{pgfscope}%
\begin{pgfscope}%
\pgfpathrectangle{\pgfqpoint{3.966666in}{4.309196in}}{\pgfqpoint{2.695652in}{1.104878in}}%
\pgfusepath{clip}%
\pgfsetroundcap%
\pgfsetroundjoin%
\pgfsetlinewidth{0.803000pt}%
\definecolor{currentstroke}{rgb}{1.000000,1.000000,1.000000}%
\pgfsetstrokecolor{currentstroke}%
\pgfsetdash{}{0pt}%
\pgfpathmoveto{\pgfqpoint{5.315052in}{4.309196in}}%
\pgfpathlineto{\pgfqpoint{5.315052in}{5.414074in}}%
\pgfusepath{stroke}%
\end{pgfscope}%
\begin{pgfscope}%
\definecolor{textcolor}{rgb}{0.150000,0.150000,0.150000}%
\pgfsetstrokecolor{textcolor}%
\pgfsetfillcolor{textcolor}%
\pgftext[x=5.315052in,y=4.211974in,,top]{\color{textcolor}\rmfamily\fontsize{10.000000}{12.000000}\selectfont 2015}%
\end{pgfscope}%
\begin{pgfscope}%
\pgfpathrectangle{\pgfqpoint{3.966666in}{4.309196in}}{\pgfqpoint{2.695652in}{1.104878in}}%
\pgfusepath{clip}%
\pgfsetroundcap%
\pgfsetroundjoin%
\pgfsetlinewidth{0.803000pt}%
\definecolor{currentstroke}{rgb}{1.000000,1.000000,1.000000}%
\pgfsetstrokecolor{currentstroke}%
\pgfsetdash{}{0pt}%
\pgfpathmoveto{\pgfqpoint{5.724045in}{4.309196in}}%
\pgfpathlineto{\pgfqpoint{5.724045in}{5.414074in}}%
\pgfusepath{stroke}%
\end{pgfscope}%
\begin{pgfscope}%
\definecolor{textcolor}{rgb}{0.150000,0.150000,0.150000}%
\pgfsetstrokecolor{textcolor}%
\pgfsetfillcolor{textcolor}%
\pgftext[x=5.724045in,y=4.211974in,,top]{\color{textcolor}\rmfamily\fontsize{10.000000}{12.000000}\selectfont 2016}%
\end{pgfscope}%
\begin{pgfscope}%
\pgfpathrectangle{\pgfqpoint{3.966666in}{4.309196in}}{\pgfqpoint{2.695652in}{1.104878in}}%
\pgfusepath{clip}%
\pgfsetroundcap%
\pgfsetroundjoin%
\pgfsetlinewidth{0.803000pt}%
\definecolor{currentstroke}{rgb}{1.000000,1.000000,1.000000}%
\pgfsetstrokecolor{currentstroke}%
\pgfsetdash{}{0pt}%
\pgfpathmoveto{\pgfqpoint{6.134158in}{4.309196in}}%
\pgfpathlineto{\pgfqpoint{6.134158in}{5.414074in}}%
\pgfusepath{stroke}%
\end{pgfscope}%
\begin{pgfscope}%
\definecolor{textcolor}{rgb}{0.150000,0.150000,0.150000}%
\pgfsetstrokecolor{textcolor}%
\pgfsetfillcolor{textcolor}%
\pgftext[x=6.134158in,y=4.211974in,,top]{\color{textcolor}\rmfamily\fontsize{10.000000}{12.000000}\selectfont 2017}%
\end{pgfscope}%
\begin{pgfscope}%
\pgfpathrectangle{\pgfqpoint{3.966666in}{4.309196in}}{\pgfqpoint{2.695652in}{1.104878in}}%
\pgfusepath{clip}%
\pgfsetroundcap%
\pgfsetroundjoin%
\pgfsetlinewidth{0.803000pt}%
\definecolor{currentstroke}{rgb}{1.000000,1.000000,1.000000}%
\pgfsetstrokecolor{currentstroke}%
\pgfsetdash{}{0pt}%
\pgfpathmoveto{\pgfqpoint{6.543150in}{4.309196in}}%
\pgfpathlineto{\pgfqpoint{6.543150in}{5.414074in}}%
\pgfusepath{stroke}%
\end{pgfscope}%
\begin{pgfscope}%
\definecolor{textcolor}{rgb}{0.150000,0.150000,0.150000}%
\pgfsetstrokecolor{textcolor}%
\pgfsetfillcolor{textcolor}%
\pgftext[x=6.543150in,y=4.211974in,,top]{\color{textcolor}\rmfamily\fontsize{10.000000}{12.000000}\selectfont 2018}%
\end{pgfscope}%
\begin{pgfscope}%
\pgfpathrectangle{\pgfqpoint{3.966666in}{4.309196in}}{\pgfqpoint{2.695652in}{1.104878in}}%
\pgfusepath{clip}%
\pgfsetroundcap%
\pgfsetroundjoin%
\pgfsetlinewidth{0.803000pt}%
\definecolor{currentstroke}{rgb}{1.000000,1.000000,1.000000}%
\pgfsetstrokecolor{currentstroke}%
\pgfsetdash{}{0pt}%
\pgfpathmoveto{\pgfqpoint{3.966666in}{4.671715in}}%
\pgfpathlineto{\pgfqpoint{6.662318in}{4.671715in}}%
\pgfusepath{stroke}%
\end{pgfscope}%
\begin{pgfscope}%
\definecolor{textcolor}{rgb}{0.150000,0.150000,0.150000}%
\pgfsetstrokecolor{textcolor}%
\pgfsetfillcolor{textcolor}%
\pgftext[x=3.692713in,y=4.618953in,left,base]{\color{textcolor}\rmfamily\fontsize{10.000000}{12.000000}\selectfont 60}%
\end{pgfscope}%
\begin{pgfscope}%
\pgfpathrectangle{\pgfqpoint{3.966666in}{4.309196in}}{\pgfqpoint{2.695652in}{1.104878in}}%
\pgfusepath{clip}%
\pgfsetroundcap%
\pgfsetroundjoin%
\pgfsetlinewidth{0.803000pt}%
\definecolor{currentstroke}{rgb}{1.000000,1.000000,1.000000}%
\pgfsetstrokecolor{currentstroke}%
\pgfsetdash{}{0pt}%
\pgfpathmoveto{\pgfqpoint{3.966666in}{5.155896in}}%
\pgfpathlineto{\pgfqpoint{6.662318in}{5.155896in}}%
\pgfusepath{stroke}%
\end{pgfscope}%
\begin{pgfscope}%
\definecolor{textcolor}{rgb}{0.150000,0.150000,0.150000}%
\pgfsetstrokecolor{textcolor}%
\pgfsetfillcolor{textcolor}%
\pgftext[x=3.692713in,y=5.103135in,left,base]{\color{textcolor}\rmfamily\fontsize{10.000000}{12.000000}\selectfont 80}%
\end{pgfscope}%
\begin{pgfscope}%
\pgfpathrectangle{\pgfqpoint{3.966666in}{4.309196in}}{\pgfqpoint{2.695652in}{1.104878in}}%
\pgfusepath{clip}%
\pgfsetroundcap%
\pgfsetroundjoin%
\pgfsetlinewidth{1.505625pt}%
\definecolor{currentstroke}{rgb}{0.549020,0.337255,0.294118}%
\pgfsetstrokecolor{currentstroke}%
\pgfsetdash{}{0pt}%
\pgfpathmoveto{\pgfqpoint{4.089196in}{4.484094in}}%
\pgfpathlineto{\pgfqpoint{4.090316in}{4.483368in}}%
\pgfpathlineto{\pgfqpoint{4.092557in}{4.475137in}}%
\pgfpathlineto{\pgfqpoint{4.095919in}{4.480463in}}%
\pgfpathlineto{\pgfqpoint{4.097039in}{4.474411in}}%
\pgfpathlineto{\pgfqpoint{4.098160in}{4.462306in}}%
\pgfpathlineto{\pgfqpoint{4.099280in}{4.464727in}}%
\pgfpathlineto{\pgfqpoint{4.100401in}{4.464727in}}%
\pgfpathlineto{\pgfqpoint{4.104883in}{4.473200in}}%
\pgfpathlineto{\pgfqpoint{4.106004in}{4.478768in}}%
\pgfpathlineto{\pgfqpoint{4.107124in}{4.479737in}}%
\pgfpathlineto{\pgfqpoint{4.108245in}{4.482642in}}%
\pgfpathlineto{\pgfqpoint{4.112727in}{4.449718in}}%
\pgfpathlineto{\pgfqpoint{4.113847in}{4.458917in}}%
\pgfpathlineto{\pgfqpoint{4.114968in}{4.455286in}}%
\pgfpathlineto{\pgfqpoint{4.116088in}{4.445844in}}%
\pgfpathlineto{\pgfqpoint{4.119450in}{4.425024in}}%
\pgfpathlineto{\pgfqpoint{4.120570in}{4.421877in}}%
\pgfpathlineto{\pgfqpoint{4.122811in}{4.427203in}}%
\pgfpathlineto{\pgfqpoint{4.123932in}{4.416551in}}%
\pgfpathlineto{\pgfqpoint{4.128414in}{4.434466in}}%
\pgfpathlineto{\pgfqpoint{4.129535in}{4.433255in}}%
\pgfpathlineto{\pgfqpoint{4.130655in}{4.440760in}}%
\pgfpathlineto{\pgfqpoint{4.131776in}{4.437855in}}%
\pgfpathlineto{\pgfqpoint{4.135137in}{4.444392in}}%
\pgfpathlineto{\pgfqpoint{4.136258in}{4.449233in}}%
\pgfpathlineto{\pgfqpoint{4.137378in}{4.450686in}}%
\pgfpathlineto{\pgfqpoint{4.138499in}{4.463033in}}%
\pgfpathlineto{\pgfqpoint{4.139619in}{4.457464in}}%
\pgfpathlineto{\pgfqpoint{4.144101in}{4.448023in}}%
\pgfpathlineto{\pgfqpoint{4.145222in}{4.448507in}}%
\pgfpathlineto{\pgfqpoint{4.146343in}{4.486273in}}%
\pgfpathlineto{\pgfqpoint{4.147463in}{4.491841in}}%
\pgfpathlineto{\pgfqpoint{4.150825in}{4.491599in}}%
\pgfpathlineto{\pgfqpoint{4.151945in}{4.504672in}}%
\pgfpathlineto{\pgfqpoint{4.153066in}{4.509272in}}%
\pgfpathlineto{\pgfqpoint{4.154186in}{4.490873in}}%
\pgfpathlineto{\pgfqpoint{4.155307in}{4.491115in}}%
\pgfpathlineto{\pgfqpoint{4.158668in}{4.496441in}}%
\pgfpathlineto{\pgfqpoint{4.159789in}{4.494262in}}%
\pgfpathlineto{\pgfqpoint{4.160909in}{4.489420in}}%
\pgfpathlineto{\pgfqpoint{4.162030in}{4.495473in}}%
\pgfpathlineto{\pgfqpoint{4.163150in}{4.495957in}}%
\pgfpathlineto{\pgfqpoint{4.167633in}{4.514598in}}%
\pgfpathlineto{\pgfqpoint{4.168753in}{4.513630in}}%
\pgfpathlineto{\pgfqpoint{4.169874in}{4.510240in}}%
\pgfpathlineto{\pgfqpoint{4.170994in}{4.502009in}}%
\pgfpathlineto{\pgfqpoint{4.176597in}{4.501041in}}%
\pgfpathlineto{\pgfqpoint{4.177717in}{4.507335in}}%
\pgfpathlineto{\pgfqpoint{4.178838in}{4.505641in}}%
\pgfpathlineto{\pgfqpoint{4.182199in}{4.506125in}}%
\pgfpathlineto{\pgfqpoint{4.183320in}{4.500315in}}%
\pgfpathlineto{\pgfqpoint{4.184440in}{4.501041in}}%
\pgfpathlineto{\pgfqpoint{4.185561in}{4.497409in}}%
\pgfpathlineto{\pgfqpoint{4.186682in}{4.501283in}}%
\pgfpathlineto{\pgfqpoint{4.190043in}{4.508061in}}%
\pgfpathlineto{\pgfqpoint{4.191164in}{4.499104in}}%
\pgfpathlineto{\pgfqpoint{4.192284in}{4.502251in}}%
\pgfpathlineto{\pgfqpoint{4.193405in}{4.503220in}}%
\pgfpathlineto{\pgfqpoint{4.197887in}{4.493778in}}%
\pgfpathlineto{\pgfqpoint{4.199007in}{4.484821in}}%
\pgfpathlineto{\pgfqpoint{4.200128in}{4.486757in}}%
\pgfpathlineto{\pgfqpoint{4.202369in}{4.474653in}}%
\pgfpathlineto{\pgfqpoint{4.206851in}{4.497652in}}%
\pgfpathlineto{\pgfqpoint{4.209092in}{4.489178in}}%
\pgfpathlineto{\pgfqpoint{4.210213in}{4.507093in}}%
\pgfpathlineto{\pgfqpoint{4.213574in}{4.490631in}}%
\pgfpathlineto{\pgfqpoint{4.215815in}{4.506125in}}%
\pgfpathlineto{\pgfqpoint{4.216936in}{4.505641in}}%
\pgfpathlineto{\pgfqpoint{4.218056in}{4.458917in}}%
\pgfpathlineto{\pgfqpoint{4.221418in}{4.443423in}}%
\pgfpathlineto{\pgfqpoint{4.222538in}{4.442213in}}%
\pgfpathlineto{\pgfqpoint{4.224779in}{4.460127in}}%
\pgfpathlineto{\pgfqpoint{4.225900in}{4.455770in}}%
\pgfpathlineto{\pgfqpoint{4.229262in}{4.455286in}}%
\pgfpathlineto{\pgfqpoint{4.230382in}{4.453591in}}%
\pgfpathlineto{\pgfqpoint{4.231503in}{4.444149in}}%
\pgfpathlineto{\pgfqpoint{4.232623in}{4.453107in}}%
\pgfpathlineto{\pgfqpoint{4.233744in}{4.444392in}}%
\pgfpathlineto{\pgfqpoint{4.237105in}{4.442455in}}%
\pgfpathlineto{\pgfqpoint{4.238226in}{4.445118in}}%
\pgfpathlineto{\pgfqpoint{4.239346in}{4.456012in}}%
\pgfpathlineto{\pgfqpoint{4.241587in}{4.441244in}}%
\pgfpathlineto{\pgfqpoint{4.244949in}{4.438581in}}%
\pgfpathlineto{\pgfqpoint{4.246069in}{4.433982in}}%
\pgfpathlineto{\pgfqpoint{4.247190in}{4.419456in}}%
\pgfpathlineto{\pgfqpoint{4.248311in}{4.422845in}}%
\pgfpathlineto{\pgfqpoint{4.249431in}{4.421393in}}%
\pgfpathlineto{\pgfqpoint{4.253913in}{4.430350in}}%
\pgfpathlineto{\pgfqpoint{4.255034in}{4.418004in}}%
\pgfpathlineto{\pgfqpoint{4.256154in}{4.417519in}}%
\pgfpathlineto{\pgfqpoint{4.257275in}{4.403236in}}%
\pgfpathlineto{\pgfqpoint{4.260636in}{4.400331in}}%
\pgfpathlineto{\pgfqpoint{4.261757in}{4.395973in}}%
\pgfpathlineto{\pgfqpoint{4.262877in}{4.408078in}}%
\pgfpathlineto{\pgfqpoint{4.263998in}{4.426477in}}%
\pgfpathlineto{\pgfqpoint{4.265118in}{4.426477in}}%
\pgfpathlineto{\pgfqpoint{4.268480in}{4.422361in}}%
\pgfpathlineto{\pgfqpoint{4.269601in}{4.426477in}}%
\pgfpathlineto{\pgfqpoint{4.270721in}{4.422845in}}%
\pgfpathlineto{\pgfqpoint{4.271842in}{4.434708in}}%
\pgfpathlineto{\pgfqpoint{4.272962in}{4.428898in}}%
\pgfpathlineto{\pgfqpoint{4.276324in}{4.417519in}}%
\pgfpathlineto{\pgfqpoint{4.277444in}{4.416067in}}%
\pgfpathlineto{\pgfqpoint{4.278565in}{4.380964in}}%
\pgfpathlineto{\pgfqpoint{4.279685in}{4.368617in}}%
\pgfpathlineto{\pgfqpoint{4.280806in}{4.370312in}}%
\pgfpathlineto{\pgfqpoint{4.284167in}{4.360144in}}%
\pgfpathlineto{\pgfqpoint{4.285288in}{4.359418in}}%
\pgfpathlineto{\pgfqpoint{4.286408in}{4.373217in}}%
\pgfpathlineto{\pgfqpoint{4.287529in}{4.379027in}}%
\pgfpathlineto{\pgfqpoint{4.288649in}{4.397426in}}%
\pgfpathlineto{\pgfqpoint{4.292011in}{4.396458in}}%
\pgfpathlineto{\pgfqpoint{4.293132in}{4.399605in}}%
\pgfpathlineto{\pgfqpoint{4.295373in}{4.399363in}}%
\pgfpathlineto{\pgfqpoint{4.296493in}{4.398152in}}%
\pgfpathlineto{\pgfqpoint{4.299855in}{4.403236in}}%
\pgfpathlineto{\pgfqpoint{4.300975in}{4.406867in}}%
\pgfpathlineto{\pgfqpoint{4.302096in}{4.400331in}}%
\pgfpathlineto{\pgfqpoint{4.304337in}{4.471506in}}%
\pgfpathlineto{\pgfqpoint{4.307698in}{4.465938in}}%
\pgfpathlineto{\pgfqpoint{4.308819in}{4.476348in}}%
\pgfpathlineto{\pgfqpoint{4.309940in}{4.477074in}}%
\pgfpathlineto{\pgfqpoint{4.311060in}{4.479011in}}%
\pgfpathlineto{\pgfqpoint{4.312181in}{4.475379in}}%
\pgfpathlineto{\pgfqpoint{4.315542in}{4.468601in}}%
\pgfpathlineto{\pgfqpoint{4.316663in}{4.461580in}}%
\pgfpathlineto{\pgfqpoint{4.317783in}{4.461580in}}%
\pgfpathlineto{\pgfqpoint{4.320024in}{4.482158in}}%
\pgfpathlineto{\pgfqpoint{4.323386in}{4.482400in}}%
\pgfpathlineto{\pgfqpoint{4.326747in}{4.451654in}}%
\pgfpathlineto{\pgfqpoint{4.327868in}{4.490147in}}%
\pgfpathlineto{\pgfqpoint{4.331230in}{4.496199in}}%
\pgfpathlineto{\pgfqpoint{4.333471in}{4.513872in}}%
\pgfpathlineto{\pgfqpoint{4.335712in}{4.514840in}}%
\pgfpathlineto{\pgfqpoint{4.339073in}{4.509272in}}%
\pgfpathlineto{\pgfqpoint{4.340194in}{4.514114in}}%
\pgfpathlineto{\pgfqpoint{4.341314in}{4.512419in}}%
\pgfpathlineto{\pgfqpoint{4.342435in}{4.519440in}}%
\pgfpathlineto{\pgfqpoint{4.343555in}{4.519440in}}%
\pgfpathlineto{\pgfqpoint{4.346917in}{4.514598in}}%
\pgfpathlineto{\pgfqpoint{4.348037in}{4.514840in}}%
\pgfpathlineto{\pgfqpoint{4.349158in}{4.516535in}}%
\pgfpathlineto{\pgfqpoint{4.350278in}{4.513145in}}%
\pgfpathlineto{\pgfqpoint{4.351399in}{4.519682in}}%
\pgfpathlineto{\pgfqpoint{4.354761in}{4.521376in}}%
\pgfpathlineto{\pgfqpoint{4.357002in}{4.516777in}}%
\pgfpathlineto{\pgfqpoint{4.358122in}{4.517019in}}%
\pgfpathlineto{\pgfqpoint{4.359243in}{4.523071in}}%
\pgfpathlineto{\pgfqpoint{4.363725in}{4.527187in}}%
\pgfpathlineto{\pgfqpoint{4.364845in}{4.525008in}}%
\pgfpathlineto{\pgfqpoint{4.365966in}{4.543407in}}%
\pgfpathlineto{\pgfqpoint{4.367086in}{4.548733in}}%
\pgfpathlineto{\pgfqpoint{4.370448in}{4.548733in}}%
\pgfpathlineto{\pgfqpoint{4.372689in}{4.540744in}}%
\pgfpathlineto{\pgfqpoint{4.373810in}{4.556480in}}%
\pgfpathlineto{\pgfqpoint{4.374930in}{4.561321in}}%
\pgfpathlineto{\pgfqpoint{4.378292in}{4.563016in}}%
\pgfpathlineto{\pgfqpoint{4.379412in}{4.562532in}}%
\pgfpathlineto{\pgfqpoint{4.380533in}{4.563258in}}%
\pgfpathlineto{\pgfqpoint{4.381653in}{4.569068in}}%
\pgfpathlineto{\pgfqpoint{4.382774in}{4.566405in}}%
\pgfpathlineto{\pgfqpoint{4.386135in}{4.572942in}}%
\pgfpathlineto{\pgfqpoint{4.387256in}{4.569552in}}%
\pgfpathlineto{\pgfqpoint{4.388376in}{4.563984in}}%
\pgfpathlineto{\pgfqpoint{4.389497in}{4.563984in}}%
\pgfpathlineto{\pgfqpoint{4.390617in}{4.565195in}}%
\pgfpathlineto{\pgfqpoint{4.393979in}{4.566889in}}%
\pgfpathlineto{\pgfqpoint{4.395100in}{4.554059in}}%
\pgfpathlineto{\pgfqpoint{4.397341in}{4.565437in}}%
\pgfpathlineto{\pgfqpoint{4.398461in}{4.570279in}}%
\pgfpathlineto{\pgfqpoint{4.401823in}{4.560111in}}%
\pgfpathlineto{\pgfqpoint{4.405184in}{4.538807in}}%
\pgfpathlineto{\pgfqpoint{4.406305in}{4.537597in}}%
\pgfpathlineto{\pgfqpoint{4.410787in}{4.558174in}}%
\pgfpathlineto{\pgfqpoint{4.411907in}{4.578268in}}%
\pgfpathlineto{\pgfqpoint{4.413028in}{4.578268in}}%
\pgfpathlineto{\pgfqpoint{4.414149in}{4.560837in}}%
\pgfpathlineto{\pgfqpoint{4.417510in}{4.559385in}}%
\pgfpathlineto{\pgfqpoint{4.418631in}{4.538565in}}%
\pgfpathlineto{\pgfqpoint{4.419751in}{4.551154in}}%
\pgfpathlineto{\pgfqpoint{4.420872in}{4.590130in}}%
\pgfpathlineto{\pgfqpoint{4.421992in}{4.577784in}}%
\pgfpathlineto{\pgfqpoint{4.427595in}{4.573910in}}%
\pgfpathlineto{\pgfqpoint{4.428715in}{4.573910in}}%
\pgfpathlineto{\pgfqpoint{4.429836in}{4.572942in}}%
\pgfpathlineto{\pgfqpoint{4.433198in}{4.562774in}}%
\pgfpathlineto{\pgfqpoint{4.434318in}{4.567858in}}%
\pgfpathlineto{\pgfqpoint{4.436559in}{4.528155in}}%
\pgfpathlineto{\pgfqpoint{4.437680in}{4.530092in}}%
\pgfpathlineto{\pgfqpoint{4.441041in}{4.531544in}}%
\pgfpathlineto{\pgfqpoint{4.444403in}{4.516777in}}%
\pgfpathlineto{\pgfqpoint{4.445523in}{4.526460in}}%
\pgfpathlineto{\pgfqpoint{4.450005in}{4.555511in}}%
\pgfpathlineto{\pgfqpoint{4.451126in}{4.558658in}}%
\pgfpathlineto{\pgfqpoint{4.453367in}{4.580689in}}%
\pgfpathlineto{\pgfqpoint{4.456729in}{4.578510in}}%
\pgfpathlineto{\pgfqpoint{4.457849in}{4.568826in}}%
\pgfpathlineto{\pgfqpoint{4.458970in}{4.577784in}}%
\pgfpathlineto{\pgfqpoint{4.460090in}{4.578994in}}%
\pgfpathlineto{\pgfqpoint{4.461211in}{4.585288in}}%
\pgfpathlineto{\pgfqpoint{4.464572in}{4.580447in}}%
\pgfpathlineto{\pgfqpoint{4.465693in}{4.575121in}}%
\pgfpathlineto{\pgfqpoint{4.466813in}{4.577057in}}%
\pgfpathlineto{\pgfqpoint{4.469054in}{4.594488in}}%
\pgfpathlineto{\pgfqpoint{4.472416in}{4.593035in}}%
\pgfpathlineto{\pgfqpoint{4.473536in}{4.601508in}}%
\pgfpathlineto{\pgfqpoint{4.474657in}{4.603687in}}%
\pgfpathlineto{\pgfqpoint{4.475778in}{4.591341in}}%
\pgfpathlineto{\pgfqpoint{4.476898in}{4.587225in}}%
\pgfpathlineto{\pgfqpoint{4.480260in}{4.587225in}}%
\pgfpathlineto{\pgfqpoint{4.481380in}{4.588193in}}%
\pgfpathlineto{\pgfqpoint{4.482501in}{4.575847in}}%
\pgfpathlineto{\pgfqpoint{4.483621in}{4.585288in}}%
\pgfpathlineto{\pgfqpoint{4.484742in}{4.563742in}}%
\pgfpathlineto{\pgfqpoint{4.488103in}{4.559627in}}%
\pgfpathlineto{\pgfqpoint{4.490344in}{4.549459in}}%
\pgfpathlineto{\pgfqpoint{4.491465in}{4.548975in}}%
\pgfpathlineto{\pgfqpoint{4.492585in}{4.532997in}}%
\pgfpathlineto{\pgfqpoint{4.495947in}{4.547522in}}%
\pgfpathlineto{\pgfqpoint{4.498188in}{4.576815in}}%
\pgfpathlineto{\pgfqpoint{4.499309in}{4.568100in}}%
\pgfpathlineto{\pgfqpoint{4.500429in}{4.571005in}}%
\pgfpathlineto{\pgfqpoint{4.504911in}{4.559627in}}%
\pgfpathlineto{\pgfqpoint{4.507152in}{4.574394in}}%
\pgfpathlineto{\pgfqpoint{4.508273in}{4.573426in}}%
\pgfpathlineto{\pgfqpoint{4.511634in}{4.581415in}}%
\pgfpathlineto{\pgfqpoint{4.512755in}{4.586257in}}%
\pgfpathlineto{\pgfqpoint{4.513875in}{4.586741in}}%
\pgfpathlineto{\pgfqpoint{4.516117in}{4.598603in}}%
\pgfpathlineto{\pgfqpoint{4.520599in}{4.598845in}}%
\pgfpathlineto{\pgfqpoint{4.521719in}{4.613371in}}%
\pgfpathlineto{\pgfqpoint{4.522840in}{4.608045in}}%
\pgfpathlineto{\pgfqpoint{4.523960in}{4.663968in}}%
\pgfpathlineto{\pgfqpoint{4.527322in}{4.674136in}}%
\pgfpathlineto{\pgfqpoint{4.528442in}{4.698345in}}%
\pgfpathlineto{\pgfqpoint{4.530683in}{4.701492in}}%
\pgfpathlineto{\pgfqpoint{4.531804in}{4.716502in}}%
\pgfpathlineto{\pgfqpoint{4.535165in}{4.703429in}}%
\pgfpathlineto{\pgfqpoint{4.537407in}{4.721101in}}%
\pgfpathlineto{\pgfqpoint{4.538527in}{4.721101in}}%
\pgfpathlineto{\pgfqpoint{4.539648in}{4.713112in}}%
\pgfpathlineto{\pgfqpoint{4.543009in}{4.714323in}}%
\pgfpathlineto{\pgfqpoint{4.544130in}{4.717712in}}%
\pgfpathlineto{\pgfqpoint{4.545250in}{4.729090in}}%
\pgfpathlineto{\pgfqpoint{4.546371in}{4.733448in}}%
\pgfpathlineto{\pgfqpoint{4.547491in}{4.728848in}}%
\pgfpathlineto{\pgfqpoint{4.551973in}{4.745310in}}%
\pgfpathlineto{\pgfqpoint{4.553094in}{4.739500in}}%
\pgfpathlineto{\pgfqpoint{4.555335in}{4.737563in}}%
\pgfpathlineto{\pgfqpoint{4.558697in}{4.716502in}}%
\pgfpathlineto{\pgfqpoint{4.559817in}{4.719649in}}%
\pgfpathlineto{\pgfqpoint{4.560938in}{4.732964in}}%
\pgfpathlineto{\pgfqpoint{4.562058in}{4.721585in}}%
\pgfpathlineto{\pgfqpoint{4.563179in}{4.727880in}}%
\pgfpathlineto{\pgfqpoint{4.566540in}{4.731511in}}%
\pgfpathlineto{\pgfqpoint{4.567661in}{4.738774in}}%
\pgfpathlineto{\pgfqpoint{4.568781in}{4.741921in}}%
\pgfpathlineto{\pgfqpoint{4.569902in}{4.735869in}}%
\pgfpathlineto{\pgfqpoint{4.571022in}{4.741437in}}%
\pgfpathlineto{\pgfqpoint{4.574384in}{4.744826in}}%
\pgfpathlineto{\pgfqpoint{4.575504in}{4.741195in}}%
\pgfpathlineto{\pgfqpoint{4.576625in}{4.733932in}}%
\pgfpathlineto{\pgfqpoint{4.577746in}{4.745552in}}%
\pgfpathlineto{\pgfqpoint{4.578866in}{4.724733in}}%
\pgfpathlineto{\pgfqpoint{4.582228in}{4.721343in}}%
\pgfpathlineto{\pgfqpoint{4.584469in}{4.749426in}}%
\pgfpathlineto{\pgfqpoint{4.585589in}{4.741921in}}%
\pgfpathlineto{\pgfqpoint{4.586710in}{4.743132in}}%
\pgfpathlineto{\pgfqpoint{4.590071in}{4.731511in}}%
\pgfpathlineto{\pgfqpoint{4.591192in}{4.745795in}}%
\pgfpathlineto{\pgfqpoint{4.592312in}{4.739016in}}%
\pgfpathlineto{\pgfqpoint{4.593433in}{4.739016in}}%
\pgfpathlineto{\pgfqpoint{4.597915in}{4.751605in}}%
\pgfpathlineto{\pgfqpoint{4.599036in}{4.776540in}}%
\pgfpathlineto{\pgfqpoint{4.600156in}{4.760078in}}%
\pgfpathlineto{\pgfqpoint{4.601277in}{4.768309in}}%
\pgfpathlineto{\pgfqpoint{4.602397in}{4.762015in}}%
\pgfpathlineto{\pgfqpoint{4.605759in}{4.773151in}}%
\pgfpathlineto{\pgfqpoint{4.606879in}{4.762741in}}%
\pgfpathlineto{\pgfqpoint{4.608000in}{4.782108in}}%
\pgfpathlineto{\pgfqpoint{4.610241in}{4.798570in}}%
\pgfpathlineto{\pgfqpoint{4.613602in}{4.790097in}}%
\pgfpathlineto{\pgfqpoint{4.614723in}{4.799055in}}%
\pgfpathlineto{\pgfqpoint{4.615843in}{4.778477in}}%
\pgfpathlineto{\pgfqpoint{4.616964in}{4.794455in}}%
\pgfpathlineto{\pgfqpoint{4.618084in}{4.825200in}}%
\pgfpathlineto{\pgfqpoint{4.621446in}{4.824716in}}%
\pgfpathlineto{\pgfqpoint{4.622567in}{4.847231in}}%
\pgfpathlineto{\pgfqpoint{4.623687in}{4.751363in}}%
\pgfpathlineto{\pgfqpoint{4.624808in}{4.740711in}}%
\pgfpathlineto{\pgfqpoint{4.625928in}{4.751121in}}%
\pgfpathlineto{\pgfqpoint{4.629290in}{4.762499in}}%
\pgfpathlineto{\pgfqpoint{4.630410in}{4.744342in}}%
\pgfpathlineto{\pgfqpoint{4.631531in}{4.748700in}}%
\pgfpathlineto{\pgfqpoint{4.633772in}{4.772667in}}%
\pgfpathlineto{\pgfqpoint{4.637133in}{4.763951in}}%
\pgfpathlineto{\pgfqpoint{4.638254in}{4.767583in}}%
\pgfpathlineto{\pgfqpoint{4.639374in}{4.777508in}}%
\pgfpathlineto{\pgfqpoint{4.640495in}{4.773151in}}%
\pgfpathlineto{\pgfqpoint{4.641616in}{4.784045in}}%
\pgfpathlineto{\pgfqpoint{4.644977in}{4.780656in}}%
\pgfpathlineto{\pgfqpoint{4.647218in}{4.822053in}}%
\pgfpathlineto{\pgfqpoint{4.648339in}{4.812612in}}%
\pgfpathlineto{\pgfqpoint{4.649459in}{4.808980in}}%
\pgfpathlineto{\pgfqpoint{4.653941in}{4.784771in}}%
\pgfpathlineto{\pgfqpoint{4.655062in}{4.785255in}}%
\pgfpathlineto{\pgfqpoint{4.656182in}{4.782834in}}%
\pgfpathlineto{\pgfqpoint{4.657303in}{4.846020in}}%
\pgfpathlineto{\pgfqpoint{4.661785in}{4.825685in}}%
\pgfpathlineto{\pgfqpoint{4.662906in}{4.786708in}}%
\pgfpathlineto{\pgfqpoint{4.664026in}{4.790581in}}%
\pgfpathlineto{\pgfqpoint{4.665147in}{4.744342in}}%
\pgfpathlineto{\pgfqpoint{4.668508in}{4.762015in}}%
\pgfpathlineto{\pgfqpoint{4.669629in}{4.756447in}}%
\pgfpathlineto{\pgfqpoint{4.670749in}{4.742163in}}%
\pgfpathlineto{\pgfqpoint{4.671870in}{4.745552in}}%
\pgfpathlineto{\pgfqpoint{4.672990in}{4.763951in}}%
\pgfpathlineto{\pgfqpoint{4.677472in}{4.771214in}}%
\pgfpathlineto{\pgfqpoint{4.678593in}{4.761046in}}%
\pgfpathlineto{\pgfqpoint{4.679713in}{4.777508in}}%
\pgfpathlineto{\pgfqpoint{4.680834in}{4.769519in}}%
\pgfpathlineto{\pgfqpoint{4.684196in}{4.787918in}}%
\pgfpathlineto{\pgfqpoint{4.685316in}{4.789613in}}%
\pgfpathlineto{\pgfqpoint{4.686437in}{4.761046in}}%
\pgfpathlineto{\pgfqpoint{4.687557in}{4.714323in}}%
\pgfpathlineto{\pgfqpoint{4.688678in}{4.757657in}}%
\pgfpathlineto{\pgfqpoint{4.692039in}{4.740711in}}%
\pgfpathlineto{\pgfqpoint{4.693160in}{4.742889in}}%
\pgfpathlineto{\pgfqpoint{4.694280in}{4.756931in}}%
\pgfpathlineto{\pgfqpoint{4.695401in}{4.762257in}}%
\pgfpathlineto{\pgfqpoint{4.696521in}{4.748700in}}%
\pgfpathlineto{\pgfqpoint{4.702124in}{4.780171in}}%
\pgfpathlineto{\pgfqpoint{4.704365in}{4.775572in}}%
\pgfpathlineto{\pgfqpoint{4.707727in}{4.784045in}}%
\pgfpathlineto{\pgfqpoint{4.708847in}{4.800023in}}%
\pgfpathlineto{\pgfqpoint{4.709968in}{4.804623in}}%
\pgfpathlineto{\pgfqpoint{4.712209in}{4.839484in}}%
\pgfpathlineto{\pgfqpoint{4.715570in}{4.838515in}}%
\pgfpathlineto{\pgfqpoint{4.717811in}{4.822295in}}%
\pgfpathlineto{\pgfqpoint{4.718932in}{4.826411in}}%
\pgfpathlineto{\pgfqpoint{4.720052in}{4.847957in}}%
\pgfpathlineto{\pgfqpoint{4.723414in}{4.845052in}}%
\pgfpathlineto{\pgfqpoint{4.724535in}{4.839484in}}%
\pgfpathlineto{\pgfqpoint{4.725655in}{4.825927in}}%
\pgfpathlineto{\pgfqpoint{4.726776in}{4.828832in}}%
\pgfpathlineto{\pgfqpoint{4.727896in}{4.828348in}}%
\pgfpathlineto{\pgfqpoint{4.731258in}{4.822295in}}%
\pgfpathlineto{\pgfqpoint{4.732378in}{4.829316in}}%
\pgfpathlineto{\pgfqpoint{4.733499in}{4.826411in}}%
\pgfpathlineto{\pgfqpoint{4.734619in}{4.853283in}}%
\pgfpathlineto{\pgfqpoint{4.735740in}{4.846262in}}%
\pgfpathlineto{\pgfqpoint{4.739101in}{4.848441in}}%
\pgfpathlineto{\pgfqpoint{4.741342in}{4.859819in}}%
\pgfpathlineto{\pgfqpoint{4.742463in}{4.863935in}}%
\pgfpathlineto{\pgfqpoint{4.743584in}{4.853283in}}%
\pgfpathlineto{\pgfqpoint{4.746945in}{4.853041in}}%
\pgfpathlineto{\pgfqpoint{4.748066in}{4.853767in}}%
\pgfpathlineto{\pgfqpoint{4.749186in}{4.845536in}}%
\pgfpathlineto{\pgfqpoint{4.751427in}{4.818422in}}%
\pgfpathlineto{\pgfqpoint{4.757030in}{4.808012in}}%
\pgfpathlineto{\pgfqpoint{4.759271in}{4.820601in}}%
\pgfpathlineto{\pgfqpoint{4.763753in}{4.779929in}}%
\pgfpathlineto{\pgfqpoint{4.764874in}{4.757415in}}%
\pgfpathlineto{\pgfqpoint{4.767115in}{4.778235in}}%
\pgfpathlineto{\pgfqpoint{4.771597in}{4.775572in}}%
\pgfpathlineto{\pgfqpoint{4.773838in}{4.763225in}}%
\pgfpathlineto{\pgfqpoint{4.774958in}{4.763467in}}%
\pgfpathlineto{\pgfqpoint{4.778320in}{4.783803in}}%
\pgfpathlineto{\pgfqpoint{4.779440in}{4.779445in}}%
\pgfpathlineto{\pgfqpoint{4.780561in}{4.785982in}}%
\pgfpathlineto{\pgfqpoint{4.781681in}{4.785740in}}%
\pgfpathlineto{\pgfqpoint{4.782802in}{4.801475in}}%
\pgfpathlineto{\pgfqpoint{4.786164in}{4.823748in}}%
\pgfpathlineto{\pgfqpoint{4.787284in}{4.817211in}}%
\pgfpathlineto{\pgfqpoint{4.788405in}{4.826411in}}%
\pgfpathlineto{\pgfqpoint{4.789525in}{4.823022in}}%
\pgfpathlineto{\pgfqpoint{4.790646in}{4.808254in}}%
\pgfpathlineto{\pgfqpoint{4.794007in}{4.806075in}}%
\pgfpathlineto{\pgfqpoint{4.796248in}{4.774845in}}%
\pgfpathlineto{\pgfqpoint{4.797369in}{4.781382in}}%
\pgfpathlineto{\pgfqpoint{4.798489in}{4.764678in}}%
\pgfpathlineto{\pgfqpoint{4.801851in}{4.732237in}}%
\pgfpathlineto{\pgfqpoint{4.802971in}{4.743616in}}%
\pgfpathlineto{\pgfqpoint{4.804092in}{4.739016in}}%
\pgfpathlineto{\pgfqpoint{4.805213in}{4.737321in}}%
\pgfpathlineto{\pgfqpoint{4.806333in}{4.740953in}}%
\pgfpathlineto{\pgfqpoint{4.809695in}{4.733448in}}%
\pgfpathlineto{\pgfqpoint{4.814177in}{4.790097in}}%
\pgfpathlineto{\pgfqpoint{4.817538in}{4.795181in}}%
\pgfpathlineto{\pgfqpoint{4.818659in}{4.772425in}}%
\pgfpathlineto{\pgfqpoint{4.820900in}{4.821327in}}%
\pgfpathlineto{\pgfqpoint{4.822020in}{4.821085in}}%
\pgfpathlineto{\pgfqpoint{4.825382in}{4.812370in}}%
\pgfpathlineto{\pgfqpoint{4.826503in}{4.840694in}}%
\pgfpathlineto{\pgfqpoint{4.827623in}{4.851346in}}%
\pgfpathlineto{\pgfqpoint{4.828744in}{4.845294in}}%
\pgfpathlineto{\pgfqpoint{4.829864in}{4.832947in}}%
\pgfpathlineto{\pgfqpoint{4.833226in}{4.859335in}}%
\pgfpathlineto{\pgfqpoint{4.834346in}{4.882576in}}%
\pgfpathlineto{\pgfqpoint{4.836587in}{4.848199in}}%
\pgfpathlineto{\pgfqpoint{4.837708in}{4.856188in}}%
\pgfpathlineto{\pgfqpoint{4.842190in}{4.861756in}}%
\pgfpathlineto{\pgfqpoint{4.843310in}{4.889596in}}%
\pgfpathlineto{\pgfqpoint{4.844431in}{4.879913in}}%
\pgfpathlineto{\pgfqpoint{4.845551in}{4.883544in}}%
\pgfpathlineto{\pgfqpoint{4.848913in}{4.878702in}}%
\pgfpathlineto{\pgfqpoint{4.851154in}{4.903638in}}%
\pgfpathlineto{\pgfqpoint{4.853395in}{4.930752in}}%
\pgfpathlineto{\pgfqpoint{4.856757in}{4.925184in}}%
\pgfpathlineto{\pgfqpoint{4.857877in}{4.920584in}}%
\pgfpathlineto{\pgfqpoint{4.858998in}{4.927605in}}%
\pgfpathlineto{\pgfqpoint{4.860118in}{4.927121in}}%
\pgfpathlineto{\pgfqpoint{4.861239in}{4.932931in}}%
\pgfpathlineto{\pgfqpoint{4.864600in}{4.942130in}}%
\pgfpathlineto{\pgfqpoint{4.865721in}{4.926636in}}%
\pgfpathlineto{\pgfqpoint{4.866842in}{4.919374in}}%
\pgfpathlineto{\pgfqpoint{4.869083in}{4.918163in}}%
\pgfpathlineto{\pgfqpoint{4.872444in}{4.900491in}}%
\pgfpathlineto{\pgfqpoint{4.873565in}{4.910174in}}%
\pgfpathlineto{\pgfqpoint{4.875806in}{4.887176in}}%
\pgfpathlineto{\pgfqpoint{4.876926in}{4.924215in}}%
\pgfpathlineto{\pgfqpoint{4.880288in}{4.929541in}}%
\pgfpathlineto{\pgfqpoint{4.881408in}{4.906543in}}%
\pgfpathlineto{\pgfqpoint{4.882529in}{4.914048in}}%
\pgfpathlineto{\pgfqpoint{4.883649in}{4.879429in}}%
\pgfpathlineto{\pgfqpoint{4.884770in}{4.880881in}}%
\pgfpathlineto{\pgfqpoint{4.888132in}{4.867082in}}%
\pgfpathlineto{\pgfqpoint{4.889252in}{4.851346in}}%
\pgfpathlineto{\pgfqpoint{4.890373in}{4.881123in}}%
\pgfpathlineto{\pgfqpoint{4.891493in}{4.871440in}}%
\pgfpathlineto{\pgfqpoint{4.892614in}{4.870229in}}%
\pgfpathlineto{\pgfqpoint{4.895975in}{4.859335in}}%
\pgfpathlineto{\pgfqpoint{4.897096in}{4.859335in}}%
\pgfpathlineto{\pgfqpoint{4.900457in}{4.873618in}}%
\pgfpathlineto{\pgfqpoint{4.903819in}{4.873376in}}%
\pgfpathlineto{\pgfqpoint{4.904939in}{4.861514in}}%
\pgfpathlineto{\pgfqpoint{4.907180in}{4.843841in}}%
\pgfpathlineto{\pgfqpoint{4.908301in}{4.842147in}}%
\pgfpathlineto{\pgfqpoint{4.911663in}{4.846020in}}%
\pgfpathlineto{\pgfqpoint{4.912783in}{4.861756in}}%
\pgfpathlineto{\pgfqpoint{4.913904in}{4.837789in}}%
\pgfpathlineto{\pgfqpoint{4.915024in}{4.841420in}}%
\pgfpathlineto{\pgfqpoint{4.916145in}{4.839000in}}%
\pgfpathlineto{\pgfqpoint{4.919506in}{4.833189in}}%
\pgfpathlineto{\pgfqpoint{4.920627in}{4.850620in}}%
\pgfpathlineto{\pgfqpoint{4.921747in}{4.848925in}}%
\pgfpathlineto{\pgfqpoint{4.922868in}{4.844326in}}%
\pgfpathlineto{\pgfqpoint{4.923988in}{4.830526in}}%
\pgfpathlineto{\pgfqpoint{4.928471in}{4.836579in}}%
\pgfpathlineto{\pgfqpoint{4.929591in}{4.829558in}}%
\pgfpathlineto{\pgfqpoint{4.930712in}{4.809464in}}%
\pgfpathlineto{\pgfqpoint{4.931832in}{4.828590in}}%
\pgfpathlineto{\pgfqpoint{4.935194in}{4.814064in}}%
\pgfpathlineto{\pgfqpoint{4.936314in}{4.827137in}}%
\pgfpathlineto{\pgfqpoint{4.938555in}{4.781624in}}%
\pgfpathlineto{\pgfqpoint{4.939676in}{4.776540in}}%
\pgfpathlineto{\pgfqpoint{4.943037in}{4.757899in}}%
\pgfpathlineto{\pgfqpoint{4.947519in}{4.790581in}}%
\pgfpathlineto{\pgfqpoint{4.950881in}{4.805107in}}%
\pgfpathlineto{\pgfqpoint{4.952002in}{4.821569in}}%
\pgfpathlineto{\pgfqpoint{4.953122in}{4.794213in}}%
\pgfpathlineto{\pgfqpoint{4.954243in}{4.800507in}}%
\pgfpathlineto{\pgfqpoint{4.955363in}{4.832947in}}%
\pgfpathlineto{\pgfqpoint{4.959845in}{4.803896in}}%
\pgfpathlineto{\pgfqpoint{4.960966in}{4.807528in}}%
\pgfpathlineto{\pgfqpoint{4.962086in}{4.802928in}}%
\pgfpathlineto{\pgfqpoint{4.963207in}{4.803896in}}%
\pgfpathlineto{\pgfqpoint{4.966568in}{4.801718in}}%
\pgfpathlineto{\pgfqpoint{4.967689in}{4.807044in}}%
\pgfpathlineto{\pgfqpoint{4.968809in}{4.801718in}}%
\pgfpathlineto{\pgfqpoint{4.971051in}{4.817938in}}%
\pgfpathlineto{\pgfqpoint{4.974412in}{4.793971in}}%
\pgfpathlineto{\pgfqpoint{4.975533in}{4.813822in}}%
\pgfpathlineto{\pgfqpoint{4.976653in}{4.800991in}}%
\pgfpathlineto{\pgfqpoint{4.978894in}{4.812370in}}%
\pgfpathlineto{\pgfqpoint{4.982256in}{4.815275in}}%
\pgfpathlineto{\pgfqpoint{4.984497in}{4.829800in}}%
\pgfpathlineto{\pgfqpoint{4.985617in}{4.828832in}}%
\pgfpathlineto{\pgfqpoint{4.986738in}{4.824474in}}%
\pgfpathlineto{\pgfqpoint{4.990099in}{4.841905in}}%
\pgfpathlineto{\pgfqpoint{4.991220in}{4.840452in}}%
\pgfpathlineto{\pgfqpoint{4.992341in}{4.820359in}}%
\pgfpathlineto{\pgfqpoint{4.994582in}{4.802202in}}%
\pgfpathlineto{\pgfqpoint{4.999064in}{4.841420in}}%
\pgfpathlineto{\pgfqpoint{5.000184in}{4.835126in}}%
\pgfpathlineto{\pgfqpoint{5.002425in}{4.840452in}}%
\pgfpathlineto{\pgfqpoint{5.005787in}{4.857398in}}%
\pgfpathlineto{\pgfqpoint{5.008028in}{4.847957in}}%
\pgfpathlineto{\pgfqpoint{5.009148in}{4.847231in}}%
\pgfpathlineto{\pgfqpoint{5.010269in}{4.840452in}}%
\pgfpathlineto{\pgfqpoint{5.013631in}{4.855220in}}%
\pgfpathlineto{\pgfqpoint{5.014751in}{4.872650in}}%
\pgfpathlineto{\pgfqpoint{5.015872in}{4.875555in}}%
\pgfpathlineto{\pgfqpoint{5.018113in}{4.860788in}}%
\pgfpathlineto{\pgfqpoint{5.022595in}{4.862240in}}%
\pgfpathlineto{\pgfqpoint{5.023715in}{4.878702in}}%
\pgfpathlineto{\pgfqpoint{5.024836in}{4.880881in}}%
\pgfpathlineto{\pgfqpoint{5.029318in}{4.877008in}}%
\pgfpathlineto{\pgfqpoint{5.031559in}{4.865629in}}%
\pgfpathlineto{\pgfqpoint{5.032680in}{4.881850in}}%
\pgfpathlineto{\pgfqpoint{5.033800in}{4.887176in}}%
\pgfpathlineto{\pgfqpoint{5.037162in}{4.918405in}}%
\pgfpathlineto{\pgfqpoint{5.038282in}{4.908237in}}%
\pgfpathlineto{\pgfqpoint{5.039403in}{4.910416in}}%
\pgfpathlineto{\pgfqpoint{5.040523in}{4.906059in}}%
\pgfpathlineto{\pgfqpoint{5.041644in}{4.897585in}}%
\pgfpathlineto{\pgfqpoint{5.045005in}{4.892986in}}%
\pgfpathlineto{\pgfqpoint{5.046126in}{4.881365in}}%
\pgfpathlineto{\pgfqpoint{5.047246in}{4.900975in}}%
\pgfpathlineto{\pgfqpoint{5.048367in}{4.902427in}}%
\pgfpathlineto{\pgfqpoint{5.049487in}{4.907269in}}%
\pgfpathlineto{\pgfqpoint{5.053970in}{4.891291in}}%
\pgfpathlineto{\pgfqpoint{5.057331in}{4.864903in}}%
\pgfpathlineto{\pgfqpoint{5.060693in}{4.856672in}}%
\pgfpathlineto{\pgfqpoint{5.062934in}{4.868292in}}%
\pgfpathlineto{\pgfqpoint{5.064054in}{4.871440in}}%
\pgfpathlineto{\pgfqpoint{5.065175in}{4.868777in}}%
\pgfpathlineto{\pgfqpoint{5.069657in}{4.859819in}}%
\pgfpathlineto{\pgfqpoint{5.070777in}{4.860304in}}%
\pgfpathlineto{\pgfqpoint{5.073019in}{4.874345in}}%
\pgfpathlineto{\pgfqpoint{5.076380in}{4.865629in}}%
\pgfpathlineto{\pgfqpoint{5.077501in}{4.856672in}}%
\pgfpathlineto{\pgfqpoint{5.078621in}{4.855462in}}%
\pgfpathlineto{\pgfqpoint{5.079742in}{4.860546in}}%
\pgfpathlineto{\pgfqpoint{5.080862in}{4.858851in}}%
\pgfpathlineto{\pgfqpoint{5.084224in}{4.860061in}}%
\pgfpathlineto{\pgfqpoint{5.085344in}{4.861272in}}%
\pgfpathlineto{\pgfqpoint{5.086465in}{4.859335in}}%
\pgfpathlineto{\pgfqpoint{5.087585in}{4.853283in}}%
\pgfpathlineto{\pgfqpoint{5.088706in}{4.850862in}}%
\pgfpathlineto{\pgfqpoint{5.092067in}{4.851830in}}%
\pgfpathlineto{\pgfqpoint{5.093188in}{4.849652in}}%
\pgfpathlineto{\pgfqpoint{5.094309in}{4.854009in}}%
\pgfpathlineto{\pgfqpoint{5.095429in}{4.863209in}}%
\pgfpathlineto{\pgfqpoint{5.096550in}{4.856672in}}%
\pgfpathlineto{\pgfqpoint{5.099911in}{4.848441in}}%
\pgfpathlineto{\pgfqpoint{5.101032in}{4.838031in}}%
\pgfpathlineto{\pgfqpoint{5.102152in}{4.844326in}}%
\pgfpathlineto{\pgfqpoint{5.103273in}{4.830042in}}%
\pgfpathlineto{\pgfqpoint{5.104393in}{4.838031in}}%
\pgfpathlineto{\pgfqpoint{5.107755in}{4.829316in}}%
\pgfpathlineto{\pgfqpoint{5.108875in}{4.843357in}}%
\pgfpathlineto{\pgfqpoint{5.111116in}{4.857883in}}%
\pgfpathlineto{\pgfqpoint{5.115599in}{4.861998in}}%
\pgfpathlineto{\pgfqpoint{5.116719in}{4.869745in}}%
\pgfpathlineto{\pgfqpoint{5.117840in}{4.892502in}}%
\pgfpathlineto{\pgfqpoint{5.118960in}{4.891291in}}%
\pgfpathlineto{\pgfqpoint{5.120081in}{4.882092in}}%
\pgfpathlineto{\pgfqpoint{5.123442in}{4.885239in}}%
\pgfpathlineto{\pgfqpoint{5.124563in}{4.884028in}}%
\pgfpathlineto{\pgfqpoint{5.125683in}{4.890807in}}%
\pgfpathlineto{\pgfqpoint{5.126804in}{4.879671in}}%
\pgfpathlineto{\pgfqpoint{5.127924in}{4.882576in}}%
\pgfpathlineto{\pgfqpoint{5.131286in}{4.877008in}}%
\pgfpathlineto{\pgfqpoint{5.133527in}{4.871198in}}%
\pgfpathlineto{\pgfqpoint{5.134647in}{4.876766in}}%
\pgfpathlineto{\pgfqpoint{5.135768in}{4.862240in}}%
\pgfpathlineto{\pgfqpoint{5.139130in}{4.855946in}}%
\pgfpathlineto{\pgfqpoint{5.142491in}{4.816001in}}%
\pgfpathlineto{\pgfqpoint{5.143612in}{4.864177in}}%
\pgfpathlineto{\pgfqpoint{5.146973in}{4.855220in}}%
\pgfpathlineto{\pgfqpoint{5.148094in}{4.859093in}}%
\pgfpathlineto{\pgfqpoint{5.149214in}{4.893712in}}%
\pgfpathlineto{\pgfqpoint{5.150335in}{4.874103in}}%
\pgfpathlineto{\pgfqpoint{5.151455in}{4.890807in}}%
\pgfpathlineto{\pgfqpoint{5.154817in}{4.901943in}}%
\pgfpathlineto{\pgfqpoint{5.155938in}{4.900733in}}%
\pgfpathlineto{\pgfqpoint{5.157058in}{4.901943in}}%
\pgfpathlineto{\pgfqpoint{5.158179in}{4.911627in}}%
\pgfpathlineto{\pgfqpoint{5.159299in}{4.907995in}}%
\pgfpathlineto{\pgfqpoint{5.164902in}{4.929299in}}%
\pgfpathlineto{\pgfqpoint{5.166022in}{4.938983in}}%
\pgfpathlineto{\pgfqpoint{5.167143in}{4.941404in}}%
\pgfpathlineto{\pgfqpoint{5.170504in}{4.944309in}}%
\pgfpathlineto{\pgfqpoint{5.171625in}{4.941162in}}%
\pgfpathlineto{\pgfqpoint{5.172745in}{4.939709in}}%
\pgfpathlineto{\pgfqpoint{5.173866in}{4.933899in}}%
\pgfpathlineto{\pgfqpoint{5.174986in}{4.935594in}}%
\pgfpathlineto{\pgfqpoint{5.179469in}{4.932931in}}%
\pgfpathlineto{\pgfqpoint{5.180589in}{4.931236in}}%
\pgfpathlineto{\pgfqpoint{5.181710in}{4.947698in}}%
\pgfpathlineto{\pgfqpoint{5.182830in}{4.949151in}}%
\pgfpathlineto{\pgfqpoint{5.186192in}{4.939951in}}%
\pgfpathlineto{\pgfqpoint{5.187312in}{4.932931in}}%
\pgfpathlineto{\pgfqpoint{5.188433in}{4.946488in}}%
\pgfpathlineto{\pgfqpoint{5.189553in}{4.943341in}}%
\pgfpathlineto{\pgfqpoint{5.190674in}{4.938499in}}%
\pgfpathlineto{\pgfqpoint{5.195156in}{4.955445in}}%
\pgfpathlineto{\pgfqpoint{5.197397in}{4.957866in}}%
\pgfpathlineto{\pgfqpoint{5.198518in}{4.963676in}}%
\pgfpathlineto{\pgfqpoint{5.201879in}{4.970697in}}%
\pgfpathlineto{\pgfqpoint{5.203000in}{4.962950in}}%
\pgfpathlineto{\pgfqpoint{5.204120in}{4.979412in}}%
\pgfpathlineto{\pgfqpoint{5.205241in}{4.960771in}}%
\pgfpathlineto{\pgfqpoint{5.206361in}{4.965855in}}%
\pgfpathlineto{\pgfqpoint{5.209723in}{4.962950in}}%
\pgfpathlineto{\pgfqpoint{5.211964in}{4.936078in}}%
\pgfpathlineto{\pgfqpoint{5.213084in}{4.934383in}}%
\pgfpathlineto{\pgfqpoint{5.214205in}{4.949635in}}%
\pgfpathlineto{\pgfqpoint{5.217567in}{4.945035in}}%
\pgfpathlineto{\pgfqpoint{5.218687in}{4.936562in}}%
\pgfpathlineto{\pgfqpoint{5.219808in}{4.957624in}}%
\pgfpathlineto{\pgfqpoint{5.220928in}{4.946972in}}%
\pgfpathlineto{\pgfqpoint{5.222049in}{4.968034in}}%
\pgfpathlineto{\pgfqpoint{5.225410in}{4.940920in}}%
\pgfpathlineto{\pgfqpoint{5.226531in}{4.944551in}}%
\pgfpathlineto{\pgfqpoint{5.228772in}{4.917437in}}%
\pgfpathlineto{\pgfqpoint{5.229892in}{4.938741in}}%
\pgfpathlineto{\pgfqpoint{5.235495in}{4.971907in}}%
\pgfpathlineto{\pgfqpoint{5.236615in}{4.951088in}}%
\pgfpathlineto{\pgfqpoint{5.237736in}{4.991275in}}%
\pgfpathlineto{\pgfqpoint{5.241098in}{5.007737in}}%
\pgfpathlineto{\pgfqpoint{5.242218in}{5.018631in}}%
\pgfpathlineto{\pgfqpoint{5.243339in}{5.020083in}}%
\pgfpathlineto{\pgfqpoint{5.245580in}{5.035335in}}%
\pgfpathlineto{\pgfqpoint{5.248941in}{5.037514in}}%
\pgfpathlineto{\pgfqpoint{5.250062in}{5.063660in}}%
\pgfpathlineto{\pgfqpoint{5.251182in}{5.071165in}}%
\pgfpathlineto{\pgfqpoint{5.252303in}{5.069228in}}%
\pgfpathlineto{\pgfqpoint{5.253423in}{5.074070in}}%
\pgfpathlineto{\pgfqpoint{5.256785in}{5.080606in}}%
\pgfpathlineto{\pgfqpoint{5.257905in}{5.085206in}}%
\pgfpathlineto{\pgfqpoint{5.259026in}{5.081332in}}%
\pgfpathlineto{\pgfqpoint{5.261267in}{5.052766in}}%
\pgfpathlineto{\pgfqpoint{5.264629in}{5.047198in}}%
\pgfpathlineto{\pgfqpoint{5.265749in}{5.049376in}}%
\pgfpathlineto{\pgfqpoint{5.266870in}{5.065596in}}%
\pgfpathlineto{\pgfqpoint{5.267990in}{5.060270in}}%
\pgfpathlineto{\pgfqpoint{5.269111in}{5.062933in}}%
\pgfpathlineto{\pgfqpoint{5.272472in}{5.052766in}}%
\pgfpathlineto{\pgfqpoint{5.273593in}{5.067049in}}%
\pgfpathlineto{\pgfqpoint{5.274713in}{5.068744in}}%
\pgfpathlineto{\pgfqpoint{5.276954in}{5.100942in}}%
\pgfpathlineto{\pgfqpoint{5.280316in}{5.093679in}}%
\pgfpathlineto{\pgfqpoint{5.281437in}{5.114257in}}%
\pgfpathlineto{\pgfqpoint{5.282557in}{5.091984in}}%
\pgfpathlineto{\pgfqpoint{5.283678in}{5.104089in}}%
\pgfpathlineto{\pgfqpoint{5.284798in}{5.099973in}}%
\pgfpathlineto{\pgfqpoint{5.288160in}{5.107962in}}%
\pgfpathlineto{\pgfqpoint{5.289280in}{5.106752in}}%
\pgfpathlineto{\pgfqpoint{5.290401in}{5.091984in}}%
\pgfpathlineto{\pgfqpoint{5.291521in}{5.100700in}}%
\pgfpathlineto{\pgfqpoint{5.292642in}{5.082785in}}%
\pgfpathlineto{\pgfqpoint{5.296003in}{5.075522in}}%
\pgfpathlineto{\pgfqpoint{5.297124in}{5.078669in}}%
\pgfpathlineto{\pgfqpoint{5.299365in}{5.133624in}}%
\pgfpathlineto{\pgfqpoint{5.300486in}{5.134834in}}%
\pgfpathlineto{\pgfqpoint{5.303847in}{5.146213in}}%
\pgfpathlineto{\pgfqpoint{5.304968in}{5.160496in}}%
\pgfpathlineto{\pgfqpoint{5.306088in}{5.157349in}}%
\pgfpathlineto{\pgfqpoint{5.308329in}{5.164127in}}%
\pgfpathlineto{\pgfqpoint{5.312811in}{5.142097in}}%
\pgfpathlineto{\pgfqpoint{5.313932in}{5.114741in}}%
\pgfpathlineto{\pgfqpoint{5.316173in}{5.101184in}}%
\pgfpathlineto{\pgfqpoint{5.319534in}{5.092226in}}%
\pgfpathlineto{\pgfqpoint{5.320655in}{5.083753in}}%
\pgfpathlineto{\pgfqpoint{5.321776in}{5.093437in}}%
\pgfpathlineto{\pgfqpoint{5.322896in}{5.114983in}}%
\pgfpathlineto{\pgfqpoint{5.324017in}{5.097310in}}%
\pgfpathlineto{\pgfqpoint{5.327378in}{5.090290in}}%
\pgfpathlineto{\pgfqpoint{5.328499in}{5.098279in}}%
\pgfpathlineto{\pgfqpoint{5.329619in}{5.091742in}}%
\pgfpathlineto{\pgfqpoint{5.330740in}{5.089079in}}%
\pgfpathlineto{\pgfqpoint{5.331860in}{5.118130in}}%
\pgfpathlineto{\pgfqpoint{5.336342in}{5.116920in}}%
\pgfpathlineto{\pgfqpoint{5.337463in}{5.120551in}}%
\pgfpathlineto{\pgfqpoint{5.338583in}{5.139434in}}%
\pgfpathlineto{\pgfqpoint{5.339704in}{5.106994in}}%
\pgfpathlineto{\pgfqpoint{5.343066in}{5.096584in}}%
\pgfpathlineto{\pgfqpoint{5.344186in}{5.031704in}}%
\pgfpathlineto{\pgfqpoint{5.345307in}{5.003621in}}%
\pgfpathlineto{\pgfqpoint{5.346427in}{5.014515in}}%
\pgfpathlineto{\pgfqpoint{5.347548in}{4.985707in}}%
\pgfpathlineto{\pgfqpoint{5.350909in}{5.003137in}}%
\pgfpathlineto{\pgfqpoint{5.352030in}{5.020568in}}%
\pgfpathlineto{\pgfqpoint{5.353150in}{5.017178in}}%
\pgfpathlineto{\pgfqpoint{5.354271in}{5.036303in}}%
\pgfpathlineto{\pgfqpoint{5.355391in}{5.013305in}}%
\pgfpathlineto{\pgfqpoint{5.358753in}{5.001685in}}%
\pgfpathlineto{\pgfqpoint{5.362115in}{5.022262in}}%
\pgfpathlineto{\pgfqpoint{5.363235in}{5.019357in}}%
\pgfpathlineto{\pgfqpoint{5.367717in}{5.010884in}}%
\pgfpathlineto{\pgfqpoint{5.368838in}{5.027104in}}%
\pgfpathlineto{\pgfqpoint{5.369958in}{5.005074in}}%
\pgfpathlineto{\pgfqpoint{5.371079in}{4.997811in}}%
\pgfpathlineto{\pgfqpoint{5.375561in}{5.011126in}}%
\pgfpathlineto{\pgfqpoint{5.376681in}{5.010158in}}%
\pgfpathlineto{\pgfqpoint{5.377802in}{5.004105in}}%
\pgfpathlineto{\pgfqpoint{5.378922in}{5.003379in}}%
\pgfpathlineto{\pgfqpoint{5.382284in}{5.009189in}}%
\pgfpathlineto{\pgfqpoint{5.383405in}{5.003863in}}%
\pgfpathlineto{\pgfqpoint{5.384525in}{4.986917in}}%
\pgfpathlineto{\pgfqpoint{5.385646in}{4.992727in}}%
\pgfpathlineto{\pgfqpoint{5.386766in}{4.951572in}}%
\pgfpathlineto{\pgfqpoint{5.390128in}{4.960529in}}%
\pgfpathlineto{\pgfqpoint{5.391248in}{4.928089in}}%
\pgfpathlineto{\pgfqpoint{5.392369in}{4.924942in}}%
\pgfpathlineto{\pgfqpoint{5.393489in}{4.939709in}}%
\pgfpathlineto{\pgfqpoint{5.394610in}{4.934141in}}%
\pgfpathlineto{\pgfqpoint{5.397971in}{4.970455in}}%
\pgfpathlineto{\pgfqpoint{5.399092in}{4.955445in}}%
\pgfpathlineto{\pgfqpoint{5.400212in}{4.974328in}}%
\pgfpathlineto{\pgfqpoint{5.401333in}{4.966581in}}%
\pgfpathlineto{\pgfqpoint{5.402453in}{4.995148in}}%
\pgfpathlineto{\pgfqpoint{5.405815in}{4.997569in}}%
\pgfpathlineto{\pgfqpoint{5.409177in}{4.940920in}}%
\pgfpathlineto{\pgfqpoint{5.413659in}{4.952782in}}%
\pgfpathlineto{\pgfqpoint{5.414779in}{4.936562in}}%
\pgfpathlineto{\pgfqpoint{5.415900in}{4.944309in}}%
\pgfpathlineto{\pgfqpoint{5.417020in}{4.946730in}}%
\pgfpathlineto{\pgfqpoint{5.421502in}{4.959561in}}%
\pgfpathlineto{\pgfqpoint{5.422623in}{4.945762in}}%
\pgfpathlineto{\pgfqpoint{5.423744in}{4.953751in}}%
\pgfpathlineto{\pgfqpoint{5.424864in}{4.956414in}}%
\pgfpathlineto{\pgfqpoint{5.425985in}{4.966097in}}%
\pgfpathlineto{\pgfqpoint{5.429346in}{4.967792in}}%
\pgfpathlineto{\pgfqpoint{5.430467in}{4.971181in}}%
\pgfpathlineto{\pgfqpoint{5.431587in}{4.969244in}}%
\pgfpathlineto{\pgfqpoint{5.432708in}{4.969244in}}%
\pgfpathlineto{\pgfqpoint{5.433828in}{4.948909in}}%
\pgfpathlineto{\pgfqpoint{5.437190in}{4.955929in}}%
\pgfpathlineto{\pgfqpoint{5.438310in}{4.960287in}}%
\pgfpathlineto{\pgfqpoint{5.439431in}{4.960529in}}%
\pgfpathlineto{\pgfqpoint{5.440551in}{4.929299in}}%
\pgfpathlineto{\pgfqpoint{5.441672in}{4.930510in}}%
\pgfpathlineto{\pgfqpoint{5.446154in}{4.918163in}}%
\pgfpathlineto{\pgfqpoint{5.448395in}{4.899038in}}%
\pgfpathlineto{\pgfqpoint{5.449516in}{4.915500in}}%
\pgfpathlineto{\pgfqpoint{5.452877in}{4.916711in}}%
\pgfpathlineto{\pgfqpoint{5.453998in}{4.910658in}}%
\pgfpathlineto{\pgfqpoint{5.455118in}{4.917679in}}%
\pgfpathlineto{\pgfqpoint{5.456239in}{4.913563in}}%
\pgfpathlineto{\pgfqpoint{5.457359in}{4.929541in}}%
\pgfpathlineto{\pgfqpoint{5.462962in}{4.902911in}}%
\pgfpathlineto{\pgfqpoint{5.465203in}{4.931478in}}%
\pgfpathlineto{\pgfqpoint{5.468565in}{4.924942in}}%
\pgfpathlineto{\pgfqpoint{5.469685in}{4.926878in}}%
\pgfpathlineto{\pgfqpoint{5.470806in}{4.919374in}}%
\pgfpathlineto{\pgfqpoint{5.471926in}{4.917921in}}%
\pgfpathlineto{\pgfqpoint{5.473047in}{4.908237in}}%
\pgfpathlineto{\pgfqpoint{5.477529in}{4.891049in}}%
\pgfpathlineto{\pgfqpoint{5.478649in}{4.896375in}}%
\pgfpathlineto{\pgfqpoint{5.479770in}{4.895165in}}%
\pgfpathlineto{\pgfqpoint{5.480890in}{4.875313in}}%
\pgfpathlineto{\pgfqpoint{5.484252in}{4.884997in}}%
\pgfpathlineto{\pgfqpoint{5.485372in}{4.878460in}}%
\pgfpathlineto{\pgfqpoint{5.486493in}{4.878944in}}%
\pgfpathlineto{\pgfqpoint{5.487614in}{4.870229in}}%
\pgfpathlineto{\pgfqpoint{5.488734in}{4.854978in}}%
\pgfpathlineto{\pgfqpoint{5.492096in}{4.860788in}}%
\pgfpathlineto{\pgfqpoint{5.494337in}{4.899522in}}%
\pgfpathlineto{\pgfqpoint{5.495457in}{4.896859in}}%
\pgfpathlineto{\pgfqpoint{5.496578in}{4.885481in}}%
\pgfpathlineto{\pgfqpoint{5.499939in}{4.869503in}}%
\pgfpathlineto{\pgfqpoint{5.503301in}{4.926636in}}%
\pgfpathlineto{\pgfqpoint{5.504421in}{4.920584in}}%
\pgfpathlineto{\pgfqpoint{5.507783in}{4.918889in}}%
\pgfpathlineto{\pgfqpoint{5.508904in}{4.904848in}}%
\pgfpathlineto{\pgfqpoint{5.510024in}{4.899038in}}%
\pgfpathlineto{\pgfqpoint{5.511145in}{4.896375in}}%
\pgfpathlineto{\pgfqpoint{5.512265in}{4.895407in}}%
\pgfpathlineto{\pgfqpoint{5.515627in}{4.873861in}}%
\pgfpathlineto{\pgfqpoint{5.516747in}{4.872166in}}%
\pgfpathlineto{\pgfqpoint{5.517868in}{4.903396in}}%
\pgfpathlineto{\pgfqpoint{5.518988in}{4.907753in}}%
\pgfpathlineto{\pgfqpoint{5.523470in}{4.910416in}}%
\pgfpathlineto{\pgfqpoint{5.524591in}{4.945519in}}%
\pgfpathlineto{\pgfqpoint{5.525711in}{4.930268in}}%
\pgfpathlineto{\pgfqpoint{5.526832in}{4.923247in}}%
\pgfpathlineto{\pgfqpoint{5.532435in}{4.952298in}}%
\pgfpathlineto{\pgfqpoint{5.534676in}{4.957866in}}%
\pgfpathlineto{\pgfqpoint{5.535796in}{4.956656in}}%
\pgfpathlineto{\pgfqpoint{5.539158in}{4.955445in}}%
\pgfpathlineto{\pgfqpoint{5.540278in}{4.943825in}}%
\pgfpathlineto{\pgfqpoint{5.542519in}{4.938015in}}%
\pgfpathlineto{\pgfqpoint{5.543640in}{4.929299in}}%
\pgfpathlineto{\pgfqpoint{5.547001in}{4.922521in}}%
\pgfpathlineto{\pgfqpoint{5.549243in}{4.936320in}}%
\pgfpathlineto{\pgfqpoint{5.550363in}{4.867566in}}%
\pgfpathlineto{\pgfqpoint{5.551484in}{4.852799in}}%
\pgfpathlineto{\pgfqpoint{5.554845in}{4.846504in}}%
\pgfpathlineto{\pgfqpoint{5.555966in}{4.836094in}}%
\pgfpathlineto{\pgfqpoint{5.557086in}{4.832947in}}%
\pgfpathlineto{\pgfqpoint{5.558207in}{4.832463in}}%
\pgfpathlineto{\pgfqpoint{5.559327in}{4.826895in}}%
\pgfpathlineto{\pgfqpoint{5.562689in}{4.846020in}}%
\pgfpathlineto{\pgfqpoint{5.563809in}{4.842631in}}%
\pgfpathlineto{\pgfqpoint{5.564930in}{4.846262in}}%
\pgfpathlineto{\pgfqpoint{5.566050in}{4.833189in}}%
\pgfpathlineto{\pgfqpoint{5.567171in}{4.829800in}}%
\pgfpathlineto{\pgfqpoint{5.570533in}{4.827863in}}%
\pgfpathlineto{\pgfqpoint{5.571653in}{4.819390in}}%
\pgfpathlineto{\pgfqpoint{5.572774in}{4.797844in}}%
\pgfpathlineto{\pgfqpoint{5.573894in}{4.793486in}}%
\pgfpathlineto{\pgfqpoint{5.575015in}{4.749426in}}%
\pgfpathlineto{\pgfqpoint{5.579497in}{4.676557in}}%
\pgfpathlineto{\pgfqpoint{5.580617in}{4.729332in}}%
\pgfpathlineto{\pgfqpoint{5.581738in}{4.741679in}}%
\pgfpathlineto{\pgfqpoint{5.582858in}{4.735869in}}%
\pgfpathlineto{\pgfqpoint{5.586220in}{4.724491in}}%
\pgfpathlineto{\pgfqpoint{5.587340in}{4.686724in}}%
\pgfpathlineto{\pgfqpoint{5.588461in}{4.706092in}}%
\pgfpathlineto{\pgfqpoint{5.589582in}{4.708513in}}%
\pgfpathlineto{\pgfqpoint{5.590702in}{4.683819in}}%
\pgfpathlineto{\pgfqpoint{5.595184in}{4.709723in}}%
\pgfpathlineto{\pgfqpoint{5.596305in}{4.677767in}}%
\pgfpathlineto{\pgfqpoint{5.597425in}{4.674378in}}%
\pgfpathlineto{\pgfqpoint{5.598546in}{4.676557in}}%
\pgfpathlineto{\pgfqpoint{5.601907in}{4.668810in}}%
\pgfpathlineto{\pgfqpoint{5.603028in}{4.698345in}}%
\pgfpathlineto{\pgfqpoint{5.604148in}{4.712144in}}%
\pgfpathlineto{\pgfqpoint{5.605269in}{4.715291in}}%
\pgfpathlineto{\pgfqpoint{5.606389in}{4.708755in}}%
\pgfpathlineto{\pgfqpoint{5.609751in}{4.724006in}}%
\pgfpathlineto{\pgfqpoint{5.610872in}{4.714081in}}%
\pgfpathlineto{\pgfqpoint{5.611992in}{4.715775in}}%
\pgfpathlineto{\pgfqpoint{5.614233in}{4.767099in}}%
\pgfpathlineto{\pgfqpoint{5.617595in}{4.747731in}}%
\pgfpathlineto{\pgfqpoint{5.618715in}{4.758625in}}%
\pgfpathlineto{\pgfqpoint{5.619836in}{4.751363in}}%
\pgfpathlineto{\pgfqpoint{5.620956in}{4.751605in}}%
\pgfpathlineto{\pgfqpoint{5.622077in}{4.761773in}}%
\pgfpathlineto{\pgfqpoint{5.627679in}{4.789371in}}%
\pgfpathlineto{\pgfqpoint{5.628800in}{4.803896in}}%
\pgfpathlineto{\pgfqpoint{5.629921in}{4.805591in}}%
\pgfpathlineto{\pgfqpoint{5.633282in}{4.802444in}}%
\pgfpathlineto{\pgfqpoint{5.634403in}{4.797602in}}%
\pgfpathlineto{\pgfqpoint{5.636644in}{4.800991in}}%
\pgfpathlineto{\pgfqpoint{5.637764in}{4.814548in}}%
\pgfpathlineto{\pgfqpoint{5.641126in}{4.820116in}}%
\pgfpathlineto{\pgfqpoint{5.642246in}{4.804381in}}%
\pgfpathlineto{\pgfqpoint{5.643367in}{4.800749in}}%
\pgfpathlineto{\pgfqpoint{5.644487in}{4.827863in}}%
\pgfpathlineto{\pgfqpoint{5.645608in}{4.874587in}}%
\pgfpathlineto{\pgfqpoint{5.648969in}{4.884513in}}%
\pgfpathlineto{\pgfqpoint{5.650090in}{4.880155in}}%
\pgfpathlineto{\pgfqpoint{5.651211in}{4.863451in}}%
\pgfpathlineto{\pgfqpoint{5.652331in}{4.874345in}}%
\pgfpathlineto{\pgfqpoint{5.653452in}{4.860546in}}%
\pgfpathlineto{\pgfqpoint{5.656813in}{4.865387in}}%
\pgfpathlineto{\pgfqpoint{5.657934in}{4.875071in}}%
\pgfpathlineto{\pgfqpoint{5.659054in}{4.875313in}}%
\pgfpathlineto{\pgfqpoint{5.661295in}{4.843115in}}%
\pgfpathlineto{\pgfqpoint{5.664657in}{4.839484in}}%
\pgfpathlineto{\pgfqpoint{5.666898in}{4.851588in}}%
\pgfpathlineto{\pgfqpoint{5.669139in}{4.808738in}}%
\pgfpathlineto{\pgfqpoint{5.672501in}{4.837305in}}%
\pgfpathlineto{\pgfqpoint{5.673621in}{4.832463in}}%
\pgfpathlineto{\pgfqpoint{5.674742in}{4.850378in}}%
\pgfpathlineto{\pgfqpoint{5.675862in}{4.857156in}}%
\pgfpathlineto{\pgfqpoint{5.676983in}{4.848683in}}%
\pgfpathlineto{\pgfqpoint{5.680344in}{4.851830in}}%
\pgfpathlineto{\pgfqpoint{5.681465in}{4.862240in}}%
\pgfpathlineto{\pgfqpoint{5.682585in}{4.850378in}}%
\pgfpathlineto{\pgfqpoint{5.684826in}{4.846020in}}%
\pgfpathlineto{\pgfqpoint{5.688188in}{4.827621in}}%
\pgfpathlineto{\pgfqpoint{5.689308in}{4.851104in}}%
\pgfpathlineto{\pgfqpoint{5.690429in}{4.847957in}}%
\pgfpathlineto{\pgfqpoint{5.691549in}{4.846989in}}%
\pgfpathlineto{\pgfqpoint{5.692670in}{4.891775in}}%
\pgfpathlineto{\pgfqpoint{5.696032in}{4.903396in}}%
\pgfpathlineto{\pgfqpoint{5.697152in}{4.890081in}}%
\pgfpathlineto{\pgfqpoint{5.698273in}{4.889112in}}%
\pgfpathlineto{\pgfqpoint{5.699393in}{4.891049in}}%
\pgfpathlineto{\pgfqpoint{5.700514in}{4.890807in}}%
\pgfpathlineto{\pgfqpoint{5.703875in}{4.901701in}}%
\pgfpathlineto{\pgfqpoint{5.706116in}{4.959803in}}%
\pgfpathlineto{\pgfqpoint{5.707237in}{4.944551in}}%
\pgfpathlineto{\pgfqpoint{5.708357in}{4.898312in}}%
\pgfpathlineto{\pgfqpoint{5.711719in}{4.915742in}}%
\pgfpathlineto{\pgfqpoint{5.712840in}{4.929784in}}%
\pgfpathlineto{\pgfqpoint{5.713960in}{4.936804in}}%
\pgfpathlineto{\pgfqpoint{5.715081in}{4.933899in}}%
\pgfpathlineto{\pgfqpoint{5.719563in}{4.936804in}}%
\pgfpathlineto{\pgfqpoint{5.720683in}{4.946246in}}%
\pgfpathlineto{\pgfqpoint{5.721804in}{4.939951in}}%
\pgfpathlineto{\pgfqpoint{5.722924in}{4.925668in}}%
\pgfpathlineto{\pgfqpoint{5.727406in}{4.903396in}}%
\pgfpathlineto{\pgfqpoint{5.728527in}{4.908722in}}%
\pgfpathlineto{\pgfqpoint{5.730768in}{4.877734in}}%
\pgfpathlineto{\pgfqpoint{5.731888in}{4.851830in}}%
\pgfpathlineto{\pgfqpoint{5.735250in}{4.866840in}}%
\pgfpathlineto{\pgfqpoint{5.736371in}{4.863451in}}%
\pgfpathlineto{\pgfqpoint{5.737491in}{4.849167in}}%
\pgfpathlineto{\pgfqpoint{5.738612in}{4.855704in}}%
\pgfpathlineto{\pgfqpoint{5.739732in}{4.830526in}}%
\pgfpathlineto{\pgfqpoint{5.744214in}{4.868050in}}%
\pgfpathlineto{\pgfqpoint{5.745335in}{4.862966in}}%
\pgfpathlineto{\pgfqpoint{5.747576in}{4.896133in}}%
\pgfpathlineto{\pgfqpoint{5.750937in}{4.885239in}}%
\pgfpathlineto{\pgfqpoint{5.752058in}{4.927605in}}%
\pgfpathlineto{\pgfqpoint{5.753178in}{4.927363in}}%
\pgfpathlineto{\pgfqpoint{5.754299in}{4.949393in}}%
\pgfpathlineto{\pgfqpoint{5.755420in}{4.990064in}}%
\pgfpathlineto{\pgfqpoint{5.758781in}{4.977718in}}%
\pgfpathlineto{\pgfqpoint{5.759902in}{4.958108in}}%
\pgfpathlineto{\pgfqpoint{5.761022in}{4.977233in}}%
\pgfpathlineto{\pgfqpoint{5.762143in}{4.968518in}}%
\pgfpathlineto{\pgfqpoint{5.766625in}{5.010158in}}%
\pgfpathlineto{\pgfqpoint{5.767745in}{5.010642in}}%
\pgfpathlineto{\pgfqpoint{5.768866in}{4.988612in}}%
\pgfpathlineto{\pgfqpoint{5.769986in}{4.951330in}}%
\pgfpathlineto{\pgfqpoint{5.771107in}{4.974812in}}%
\pgfpathlineto{\pgfqpoint{5.775589in}{4.985222in}}%
\pgfpathlineto{\pgfqpoint{5.776710in}{5.006526in}}%
\pgfpathlineto{\pgfqpoint{5.777830in}{4.996359in}}%
\pgfpathlineto{\pgfqpoint{5.778951in}{4.992243in}}%
\pgfpathlineto{\pgfqpoint{5.782312in}{4.999506in}}%
\pgfpathlineto{\pgfqpoint{5.784553in}{4.987159in}}%
\pgfpathlineto{\pgfqpoint{5.785674in}{5.004105in}}%
\pgfpathlineto{\pgfqpoint{5.786794in}{4.977233in}}%
\pgfpathlineto{\pgfqpoint{5.790156in}{4.959803in}}%
\pgfpathlineto{\pgfqpoint{5.793517in}{5.015000in}}%
\pgfpathlineto{\pgfqpoint{5.794638in}{5.029041in}}%
\pgfpathlineto{\pgfqpoint{5.798000in}{5.020568in}}%
\pgfpathlineto{\pgfqpoint{5.799120in}{5.019841in}}%
\pgfpathlineto{\pgfqpoint{5.800241in}{5.017662in}}%
\pgfpathlineto{\pgfqpoint{5.802482in}{4.991275in}}%
\pgfpathlineto{\pgfqpoint{5.805843in}{4.978686in}}%
\pgfpathlineto{\pgfqpoint{5.806964in}{4.981833in}}%
\pgfpathlineto{\pgfqpoint{5.808084in}{4.982559in}}%
\pgfpathlineto{\pgfqpoint{5.809205in}{5.013063in}}%
\pgfpathlineto{\pgfqpoint{5.810325in}{5.021778in}}%
\pgfpathlineto{\pgfqpoint{5.813687in}{5.025409in}}%
\pgfpathlineto{\pgfqpoint{5.814807in}{5.012821in}}%
\pgfpathlineto{\pgfqpoint{5.817049in}{5.015968in}}%
\pgfpathlineto{\pgfqpoint{5.821531in}{5.010158in}}%
\pgfpathlineto{\pgfqpoint{5.822651in}{5.014031in}}%
\pgfpathlineto{\pgfqpoint{5.823772in}{5.011610in}}%
\pgfpathlineto{\pgfqpoint{5.824892in}{5.003379in}}%
\pgfpathlineto{\pgfqpoint{5.826013in}{5.030009in}}%
\pgfpathlineto{\pgfqpoint{5.829374in}{5.022988in}}%
\pgfpathlineto{\pgfqpoint{5.830495in}{5.022020in}}%
\pgfpathlineto{\pgfqpoint{5.831615in}{5.036061in}}%
\pgfpathlineto{\pgfqpoint{5.832736in}{5.023715in}}%
\pgfpathlineto{\pgfqpoint{5.833856in}{5.022746in}}%
\pgfpathlineto{\pgfqpoint{5.837218in}{5.012579in}}%
\pgfpathlineto{\pgfqpoint{5.838339in}{5.014757in}}%
\pgfpathlineto{\pgfqpoint{5.839459in}{5.006768in}}%
\pgfpathlineto{\pgfqpoint{5.845062in}{5.029525in}}%
\pgfpathlineto{\pgfqpoint{5.846182in}{5.039209in}}%
\pgfpathlineto{\pgfqpoint{5.847303in}{5.001442in}}%
\pgfpathlineto{\pgfqpoint{5.848423in}{4.984980in}}%
\pgfpathlineto{\pgfqpoint{5.852905in}{4.998537in}}%
\pgfpathlineto{\pgfqpoint{5.854026in}{4.957866in}}%
\pgfpathlineto{\pgfqpoint{5.855146in}{4.965129in}}%
\pgfpathlineto{\pgfqpoint{5.856267in}{4.962466in}}%
\pgfpathlineto{\pgfqpoint{5.857388in}{4.970213in}}%
\pgfpathlineto{\pgfqpoint{5.860749in}{4.988854in}}%
\pgfpathlineto{\pgfqpoint{5.861870in}{4.991759in}}%
\pgfpathlineto{\pgfqpoint{5.862990in}{5.002653in}}%
\pgfpathlineto{\pgfqpoint{5.864111in}{4.996116in}}%
\pgfpathlineto{\pgfqpoint{5.865231in}{5.014273in}}%
\pgfpathlineto{\pgfqpoint{5.868593in}{5.014031in}}%
\pgfpathlineto{\pgfqpoint{5.869713in}{5.021778in}}%
\pgfpathlineto{\pgfqpoint{5.870834in}{5.014515in}}%
\pgfpathlineto{\pgfqpoint{5.871954in}{5.020325in}}%
\pgfpathlineto{\pgfqpoint{5.873075in}{4.994422in}}%
\pgfpathlineto{\pgfqpoint{5.876436in}{5.003137in}}%
\pgfpathlineto{\pgfqpoint{5.878678in}{4.964403in}}%
\pgfpathlineto{\pgfqpoint{5.879798in}{4.971665in}}%
\pgfpathlineto{\pgfqpoint{5.880919in}{4.968034in}}%
\pgfpathlineto{\pgfqpoint{5.884280in}{4.971907in}}%
\pgfpathlineto{\pgfqpoint{5.886521in}{4.999990in}}%
\pgfpathlineto{\pgfqpoint{5.887642in}{4.994180in}}%
\pgfpathlineto{\pgfqpoint{5.888762in}{4.998779in}}%
\pgfpathlineto{\pgfqpoint{5.893244in}{4.990306in}}%
\pgfpathlineto{\pgfqpoint{5.894365in}{5.006768in}}%
\pgfpathlineto{\pgfqpoint{5.895485in}{5.010158in}}%
\pgfpathlineto{\pgfqpoint{5.896606in}{5.021536in}}%
\pgfpathlineto{\pgfqpoint{5.899968in}{5.028072in}}%
\pgfpathlineto{\pgfqpoint{5.901088in}{5.018389in}}%
\pgfpathlineto{\pgfqpoint{5.902209in}{5.025409in}}%
\pgfpathlineto{\pgfqpoint{5.903329in}{5.036788in}}%
\pgfpathlineto{\pgfqpoint{5.904450in}{5.037514in}}%
\pgfpathlineto{\pgfqpoint{5.907811in}{5.023715in}}%
\pgfpathlineto{\pgfqpoint{5.908932in}{5.040903in}}%
\pgfpathlineto{\pgfqpoint{5.910052in}{5.032188in}}%
\pgfpathlineto{\pgfqpoint{5.911173in}{5.042114in}}%
\pgfpathlineto{\pgfqpoint{5.912293in}{5.036061in}}%
\pgfpathlineto{\pgfqpoint{5.915655in}{5.034125in}}%
\pgfpathlineto{\pgfqpoint{5.916775in}{5.042114in}}%
\pgfpathlineto{\pgfqpoint{5.917896in}{5.045745in}}%
\pgfpathlineto{\pgfqpoint{5.919017in}{5.059544in}}%
\pgfpathlineto{\pgfqpoint{5.920137in}{5.016936in}}%
\pgfpathlineto{\pgfqpoint{5.923499in}{4.994422in}}%
\pgfpathlineto{\pgfqpoint{5.926860in}{5.069712in}}%
\pgfpathlineto{\pgfqpoint{5.927981in}{5.072133in}}%
\pgfpathlineto{\pgfqpoint{5.932463in}{5.086416in}}%
\pgfpathlineto{\pgfqpoint{5.933583in}{5.077459in}}%
\pgfpathlineto{\pgfqpoint{5.934704in}{5.073101in}}%
\pgfpathlineto{\pgfqpoint{5.935824in}{5.093679in}}%
\pgfpathlineto{\pgfqpoint{5.940307in}{5.093195in}}%
\pgfpathlineto{\pgfqpoint{5.941427in}{5.096342in}}%
\pgfpathlineto{\pgfqpoint{5.942548in}{5.095858in}}%
\pgfpathlineto{\pgfqpoint{5.943668in}{5.099005in}}%
\pgfpathlineto{\pgfqpoint{5.947030in}{5.097310in}}%
\pgfpathlineto{\pgfqpoint{5.948150in}{5.102394in}}%
\pgfpathlineto{\pgfqpoint{5.949271in}{5.098521in}}%
\pgfpathlineto{\pgfqpoint{5.950391in}{5.097068in}}%
\pgfpathlineto{\pgfqpoint{5.951512in}{5.107236in}}%
\pgfpathlineto{\pgfqpoint{5.954873in}{5.108931in}}%
\pgfpathlineto{\pgfqpoint{5.955994in}{5.097310in}}%
\pgfpathlineto{\pgfqpoint{5.957114in}{5.079638in}}%
\pgfpathlineto{\pgfqpoint{5.958235in}{5.087143in}}%
\pgfpathlineto{\pgfqpoint{5.959355in}{5.104331in}}%
\pgfpathlineto{\pgfqpoint{5.963838in}{5.130235in}}%
\pgfpathlineto{\pgfqpoint{5.964958in}{5.112804in}}%
\pgfpathlineto{\pgfqpoint{5.966079in}{5.114499in}}%
\pgfpathlineto{\pgfqpoint{5.967199in}{5.108689in}}%
\pgfpathlineto{\pgfqpoint{5.970561in}{5.108204in}}%
\pgfpathlineto{\pgfqpoint{5.972802in}{5.120309in}}%
\pgfpathlineto{\pgfqpoint{5.975043in}{5.136287in}}%
\pgfpathlineto{\pgfqpoint{5.978404in}{5.135803in}}%
\pgfpathlineto{\pgfqpoint{5.979525in}{5.126119in}}%
\pgfpathlineto{\pgfqpoint{5.981766in}{5.145244in}}%
\pgfpathlineto{\pgfqpoint{5.986248in}{5.132171in}}%
\pgfpathlineto{\pgfqpoint{5.987369in}{5.144276in}}%
\pgfpathlineto{\pgfqpoint{5.988489in}{5.142339in}}%
\pgfpathlineto{\pgfqpoint{5.989610in}{5.155412in}}%
\pgfpathlineto{\pgfqpoint{5.990730in}{5.148149in}}%
\pgfpathlineto{\pgfqpoint{5.994092in}{5.164127in}}%
\pgfpathlineto{\pgfqpoint{5.995212in}{5.147423in}}%
\pgfpathlineto{\pgfqpoint{5.996333in}{5.142339in}}%
\pgfpathlineto{\pgfqpoint{5.997453in}{5.164370in}}%
\pgfpathlineto{\pgfqpoint{5.998574in}{5.161949in}}%
\pgfpathlineto{\pgfqpoint{6.003056in}{5.171632in}}%
\pgfpathlineto{\pgfqpoint{6.004177in}{5.156381in}}%
\pgfpathlineto{\pgfqpoint{6.005297in}{5.152749in}}%
\pgfpathlineto{\pgfqpoint{6.006418in}{5.118856in}}%
\pgfpathlineto{\pgfqpoint{6.009779in}{5.162917in}}%
\pgfpathlineto{\pgfqpoint{6.010900in}{5.136529in}}%
\pgfpathlineto{\pgfqpoint{6.012020in}{5.135803in}}%
\pgfpathlineto{\pgfqpoint{6.013141in}{5.158801in}}%
\pgfpathlineto{\pgfqpoint{6.014261in}{5.158559in}}%
\pgfpathlineto{\pgfqpoint{6.017623in}{5.165580in}}%
\pgfpathlineto{\pgfqpoint{6.018743in}{5.170180in}}%
\pgfpathlineto{\pgfqpoint{6.019864in}{5.152991in}}%
\pgfpathlineto{\pgfqpoint{6.020984in}{5.179379in}}%
\pgfpathlineto{\pgfqpoint{6.022105in}{5.152265in}}%
\pgfpathlineto{\pgfqpoint{6.025467in}{5.154202in}}%
\pgfpathlineto{\pgfqpoint{6.026587in}{5.165338in}}%
\pgfpathlineto{\pgfqpoint{6.027708in}{5.189547in}}%
\pgfpathlineto{\pgfqpoint{6.028828in}{5.162675in}}%
\pgfpathlineto{\pgfqpoint{6.029949in}{5.196083in}}%
\pgfpathlineto{\pgfqpoint{6.034431in}{5.165338in}}%
\pgfpathlineto{\pgfqpoint{6.037792in}{5.201651in}}%
\pgfpathlineto{\pgfqpoint{6.041154in}{5.180832in}}%
\pgfpathlineto{\pgfqpoint{6.042274in}{5.169453in}}%
\pgfpathlineto{\pgfqpoint{6.043395in}{5.170180in}}%
\pgfpathlineto{\pgfqpoint{6.044516in}{5.162675in}}%
\pgfpathlineto{\pgfqpoint{6.045636in}{5.167033in}}%
\pgfpathlineto{\pgfqpoint{6.048998in}{5.153718in}}%
\pgfpathlineto{\pgfqpoint{6.050118in}{5.145486in}}%
\pgfpathlineto{\pgfqpoint{6.051239in}{5.117888in}}%
\pgfpathlineto{\pgfqpoint{6.053480in}{5.091016in}}%
\pgfpathlineto{\pgfqpoint{6.056841in}{5.085932in}}%
\pgfpathlineto{\pgfqpoint{6.057962in}{5.149602in}}%
\pgfpathlineto{\pgfqpoint{6.059082in}{5.159044in}}%
\pgfpathlineto{\pgfqpoint{6.060203in}{5.140887in}}%
\pgfpathlineto{\pgfqpoint{6.061323in}{5.146697in}}%
\pgfpathlineto{\pgfqpoint{6.064685in}{5.145729in}}%
\pgfpathlineto{\pgfqpoint{6.065806in}{5.146939in}}%
\pgfpathlineto{\pgfqpoint{6.068047in}{5.141371in}}%
\pgfpathlineto{\pgfqpoint{6.069167in}{5.107720in}}%
\pgfpathlineto{\pgfqpoint{6.072529in}{5.140403in}}%
\pgfpathlineto{\pgfqpoint{6.073649in}{5.160496in}}%
\pgfpathlineto{\pgfqpoint{6.074770in}{5.126603in}}%
\pgfpathlineto{\pgfqpoint{6.075890in}{5.060513in}}%
\pgfpathlineto{\pgfqpoint{6.077011in}{5.074312in}}%
\pgfpathlineto{\pgfqpoint{6.080372in}{5.061481in}}%
\pgfpathlineto{\pgfqpoint{6.081493in}{5.075280in}}%
\pgfpathlineto{\pgfqpoint{6.082613in}{5.065596in}}%
\pgfpathlineto{\pgfqpoint{6.083734in}{5.062933in}}%
\pgfpathlineto{\pgfqpoint{6.084855in}{5.039209in}}%
\pgfpathlineto{\pgfqpoint{6.089337in}{5.056155in}}%
\pgfpathlineto{\pgfqpoint{6.090457in}{5.054460in}}%
\pgfpathlineto{\pgfqpoint{6.092698in}{5.071649in}}%
\pgfpathlineto{\pgfqpoint{6.097180in}{5.059060in}}%
\pgfpathlineto{\pgfqpoint{6.099421in}{5.036061in}}%
\pgfpathlineto{\pgfqpoint{6.100542in}{5.048166in}}%
\pgfpathlineto{\pgfqpoint{6.103903in}{5.061239in}}%
\pgfpathlineto{\pgfqpoint{6.105024in}{5.059544in}}%
\pgfpathlineto{\pgfqpoint{6.106145in}{5.087627in}}%
\pgfpathlineto{\pgfqpoint{6.107265in}{5.072617in}}%
\pgfpathlineto{\pgfqpoint{6.108386in}{5.091984in}}%
\pgfpathlineto{\pgfqpoint{6.111747in}{5.108689in}}%
\pgfpathlineto{\pgfqpoint{6.112868in}{5.109899in}}%
\pgfpathlineto{\pgfqpoint{6.113988in}{5.091984in}}%
\pgfpathlineto{\pgfqpoint{6.115109in}{5.098763in}}%
\pgfpathlineto{\pgfqpoint{6.116229in}{5.098763in}}%
\pgfpathlineto{\pgfqpoint{6.119591in}{5.099489in}}%
\pgfpathlineto{\pgfqpoint{6.120711in}{5.096342in}}%
\pgfpathlineto{\pgfqpoint{6.121832in}{5.089806in}}%
\pgfpathlineto{\pgfqpoint{6.122952in}{5.094163in}}%
\pgfpathlineto{\pgfqpoint{6.124073in}{5.105057in}}%
\pgfpathlineto{\pgfqpoint{6.128555in}{5.097068in}}%
\pgfpathlineto{\pgfqpoint{6.129676in}{5.085206in}}%
\pgfpathlineto{\pgfqpoint{6.130796in}{5.091500in}}%
\pgfpathlineto{\pgfqpoint{6.131917in}{5.085448in}}%
\pgfpathlineto{\pgfqpoint{6.136399in}{5.088111in}}%
\pgfpathlineto{\pgfqpoint{6.137519in}{5.094889in}}%
\pgfpathlineto{\pgfqpoint{6.138640in}{5.107236in}}%
\pgfpathlineto{\pgfqpoint{6.139760in}{5.106510in}}%
\pgfpathlineto{\pgfqpoint{6.143122in}{5.092469in}}%
\pgfpathlineto{\pgfqpoint{6.144242in}{5.072375in}}%
\pgfpathlineto{\pgfqpoint{6.145363in}{5.078185in}}%
\pgfpathlineto{\pgfqpoint{6.146484in}{5.080122in}}%
\pgfpathlineto{\pgfqpoint{6.147604in}{5.083995in}}%
\pgfpathlineto{\pgfqpoint{6.153207in}{5.119341in}}%
\pgfpathlineto{\pgfqpoint{6.154327in}{5.114015in}}%
\pgfpathlineto{\pgfqpoint{6.155448in}{5.175506in}}%
\pgfpathlineto{\pgfqpoint{6.158809in}{5.164612in}}%
\pgfpathlineto{\pgfqpoint{6.159930in}{5.184705in}}%
\pgfpathlineto{\pgfqpoint{6.162171in}{5.156623in}}%
\pgfpathlineto{\pgfqpoint{6.163291in}{5.159286in}}%
\pgfpathlineto{\pgfqpoint{6.166653in}{5.160012in}}%
\pgfpathlineto{\pgfqpoint{6.167774in}{5.178895in}}%
\pgfpathlineto{\pgfqpoint{6.168894in}{5.172843in}}%
\pgfpathlineto{\pgfqpoint{6.170015in}{5.182526in}}%
\pgfpathlineto{\pgfqpoint{6.171135in}{5.174779in}}%
\pgfpathlineto{\pgfqpoint{6.174497in}{5.174537in}}%
\pgfpathlineto{\pgfqpoint{6.176738in}{5.195357in}}%
\pgfpathlineto{\pgfqpoint{6.177858in}{5.202862in}}%
\pgfpathlineto{\pgfqpoint{6.178979in}{5.187368in}}%
\pgfpathlineto{\pgfqpoint{6.182340in}{5.194873in}}%
\pgfpathlineto{\pgfqpoint{6.183461in}{5.184705in}}%
\pgfpathlineto{\pgfqpoint{6.184581in}{5.257817in}}%
\pgfpathlineto{\pgfqpoint{6.185702in}{5.250312in}}%
\pgfpathlineto{\pgfqpoint{6.186822in}{5.257090in}}%
\pgfpathlineto{\pgfqpoint{6.191305in}{5.269921in}}%
\pgfpathlineto{\pgfqpoint{6.194666in}{5.256122in}}%
\pgfpathlineto{\pgfqpoint{6.198028in}{5.252491in}}%
\pgfpathlineto{\pgfqpoint{6.199148in}{5.256606in}}%
\pgfpathlineto{\pgfqpoint{6.200269in}{5.269921in}}%
\pgfpathlineto{\pgfqpoint{6.202510in}{5.243775in}}%
\pgfpathlineto{\pgfqpoint{6.206992in}{5.239176in}}%
\pgfpathlineto{\pgfqpoint{6.208113in}{5.235786in}}%
\pgfpathlineto{\pgfqpoint{6.209233in}{5.240386in}}%
\pgfpathlineto{\pgfqpoint{6.210354in}{5.256606in}}%
\pgfpathlineto{\pgfqpoint{6.213715in}{5.261932in}}%
\pgfpathlineto{\pgfqpoint{6.214836in}{5.255154in}}%
\pgfpathlineto{\pgfqpoint{6.215956in}{5.264111in}}%
\pgfpathlineto{\pgfqpoint{6.217077in}{5.264837in}}%
\pgfpathlineto{\pgfqpoint{6.218197in}{5.255154in}}%
\pgfpathlineto{\pgfqpoint{6.221559in}{5.259995in}}%
\pgfpathlineto{\pgfqpoint{6.222679in}{5.259269in}}%
\pgfpathlineto{\pgfqpoint{6.226041in}{5.245470in}}%
\pgfpathlineto{\pgfqpoint{6.229403in}{5.243533in}}%
\pgfpathlineto{\pgfqpoint{6.230523in}{5.249585in}}%
\pgfpathlineto{\pgfqpoint{6.231644in}{5.246196in}}%
\pgfpathlineto{\pgfqpoint{6.233885in}{5.229250in}}%
\pgfpathlineto{\pgfqpoint{6.237246in}{5.225618in}}%
\pgfpathlineto{\pgfqpoint{6.238367in}{5.230702in}}%
\pgfpathlineto{\pgfqpoint{6.239487in}{5.231913in}}%
\pgfpathlineto{\pgfqpoint{6.240608in}{5.219324in}}%
\pgfpathlineto{\pgfqpoint{6.241728in}{5.215451in}}%
\pgfpathlineto{\pgfqpoint{6.245090in}{5.221261in}}%
\pgfpathlineto{\pgfqpoint{6.246210in}{5.228281in}}%
\pgfpathlineto{\pgfqpoint{6.247331in}{5.239660in}}%
\pgfpathlineto{\pgfqpoint{6.248451in}{5.233365in}}%
\pgfpathlineto{\pgfqpoint{6.252934in}{5.241354in}}%
\pgfpathlineto{\pgfqpoint{6.254054in}{5.250554in}}%
\pgfpathlineto{\pgfqpoint{6.257416in}{5.216903in}}%
\pgfpathlineto{\pgfqpoint{6.261898in}{5.248133in}}%
\pgfpathlineto{\pgfqpoint{6.263018in}{5.197052in}}%
\pgfpathlineto{\pgfqpoint{6.264139in}{5.196083in}}%
\pgfpathlineto{\pgfqpoint{6.265259in}{5.187852in}}%
\pgfpathlineto{\pgfqpoint{6.268621in}{5.182526in}}%
\pgfpathlineto{\pgfqpoint{6.269742in}{5.162917in}}%
\pgfpathlineto{\pgfqpoint{6.270862in}{5.167517in}}%
\pgfpathlineto{\pgfqpoint{6.271983in}{5.168001in}}%
\pgfpathlineto{\pgfqpoint{6.273103in}{5.169211in}}%
\pgfpathlineto{\pgfqpoint{6.276465in}{5.170422in}}%
\pgfpathlineto{\pgfqpoint{6.277585in}{5.167033in}}%
\pgfpathlineto{\pgfqpoint{6.278706in}{5.168969in}}%
\pgfpathlineto{\pgfqpoint{6.279826in}{5.161707in}}%
\pgfpathlineto{\pgfqpoint{6.280947in}{5.162191in}}%
\pgfpathlineto{\pgfqpoint{6.284308in}{5.165338in}}%
\pgfpathlineto{\pgfqpoint{6.285429in}{5.163401in}}%
\pgfpathlineto{\pgfqpoint{6.286549in}{5.163885in}}%
\pgfpathlineto{\pgfqpoint{6.287670in}{5.154928in}}%
\pgfpathlineto{\pgfqpoint{6.288790in}{5.163401in}}%
\pgfpathlineto{\pgfqpoint{6.292152in}{5.162675in}}%
\pgfpathlineto{\pgfqpoint{6.293273in}{5.159770in}}%
\pgfpathlineto{\pgfqpoint{6.296634in}{5.186158in}}%
\pgfpathlineto{\pgfqpoint{6.301116in}{5.189547in}}%
\pgfpathlineto{\pgfqpoint{6.302237in}{5.205041in}}%
\pgfpathlineto{\pgfqpoint{6.303357in}{5.206009in}}%
\pgfpathlineto{\pgfqpoint{6.304478in}{5.216177in}}%
\pgfpathlineto{\pgfqpoint{6.308960in}{5.221019in}}%
\pgfpathlineto{\pgfqpoint{6.310080in}{5.220292in}}%
\pgfpathlineto{\pgfqpoint{6.311201in}{5.199715in}}%
\pgfpathlineto{\pgfqpoint{6.312322in}{5.206493in}}%
\pgfpathlineto{\pgfqpoint{6.315683in}{5.208430in}}%
\pgfpathlineto{\pgfqpoint{6.316804in}{5.204314in}}%
\pgfpathlineto{\pgfqpoint{6.317924in}{5.213030in}}%
\pgfpathlineto{\pgfqpoint{6.319045in}{5.234092in}}%
\pgfpathlineto{\pgfqpoint{6.320165in}{5.240386in}}%
\pgfpathlineto{\pgfqpoint{6.323527in}{5.244986in}}%
\pgfpathlineto{\pgfqpoint{6.325768in}{5.234576in}}%
\pgfpathlineto{\pgfqpoint{6.326888in}{5.226103in}}%
\pgfpathlineto{\pgfqpoint{6.328009in}{5.235060in}}%
\pgfpathlineto{\pgfqpoint{6.331370in}{5.233607in}}%
\pgfpathlineto{\pgfqpoint{6.332491in}{5.216661in}}%
\pgfpathlineto{\pgfqpoint{6.333612in}{5.211335in}}%
\pgfpathlineto{\pgfqpoint{6.334732in}{5.180347in}}%
\pgfpathlineto{\pgfqpoint{6.335853in}{5.183737in}}%
\pgfpathlineto{\pgfqpoint{6.339214in}{5.197052in}}%
\pgfpathlineto{\pgfqpoint{6.341455in}{5.195599in}}%
\pgfpathlineto{\pgfqpoint{6.342576in}{5.188821in}}%
\pgfpathlineto{\pgfqpoint{6.343696in}{5.195115in}}%
\pgfpathlineto{\pgfqpoint{6.347058in}{5.182526in}}%
\pgfpathlineto{\pgfqpoint{6.348178in}{5.175264in}}%
\pgfpathlineto{\pgfqpoint{6.349299in}{5.179621in}}%
\pgfpathlineto{\pgfqpoint{6.350419in}{5.173811in}}%
\pgfpathlineto{\pgfqpoint{6.351540in}{5.182768in}}%
\pgfpathlineto{\pgfqpoint{6.354902in}{5.192936in}}%
\pgfpathlineto{\pgfqpoint{6.356022in}{5.217387in}}%
\pgfpathlineto{\pgfqpoint{6.358263in}{5.232155in}}%
\pgfpathlineto{\pgfqpoint{6.359384in}{5.232397in}}%
\pgfpathlineto{\pgfqpoint{6.362745in}{5.222713in}}%
\pgfpathlineto{\pgfqpoint{6.363866in}{5.244502in}}%
\pgfpathlineto{\pgfqpoint{6.364986in}{5.248133in}}%
\pgfpathlineto{\pgfqpoint{6.366107in}{5.279363in}}%
\pgfpathlineto{\pgfqpoint{6.367227in}{5.268711in}}%
\pgfpathlineto{\pgfqpoint{6.371709in}{5.289046in}}%
\pgfpathlineto{\pgfqpoint{6.372830in}{5.287594in}}%
\pgfpathlineto{\pgfqpoint{6.375071in}{5.279121in}}%
\pgfpathlineto{\pgfqpoint{6.380674in}{5.307687in}}%
\pgfpathlineto{\pgfqpoint{6.381794in}{5.303572in}}%
\pgfpathlineto{\pgfqpoint{6.382915in}{5.294372in}}%
\pgfpathlineto{\pgfqpoint{6.386276in}{5.301393in}}%
\pgfpathlineto{\pgfqpoint{6.387397in}{5.313982in}}%
\pgfpathlineto{\pgfqpoint{6.388517in}{5.319550in}}%
\pgfpathlineto{\pgfqpoint{6.389638in}{5.311077in}}%
\pgfpathlineto{\pgfqpoint{6.390758in}{5.320034in}}%
\pgfpathlineto{\pgfqpoint{6.394120in}{5.328991in}}%
\pgfpathlineto{\pgfqpoint{6.395241in}{5.327297in}}%
\pgfpathlineto{\pgfqpoint{6.397482in}{5.315676in}}%
\pgfpathlineto{\pgfqpoint{6.398602in}{5.321002in}}%
\pgfpathlineto{\pgfqpoint{6.401964in}{5.320034in}}%
\pgfpathlineto{\pgfqpoint{6.403084in}{5.316645in}}%
\pgfpathlineto{\pgfqpoint{6.404205in}{5.306477in}}%
\pgfpathlineto{\pgfqpoint{6.406446in}{5.321486in}}%
\pgfpathlineto{\pgfqpoint{6.410928in}{5.325844in}}%
\pgfpathlineto{\pgfqpoint{6.412048in}{5.325844in}}%
\pgfpathlineto{\pgfqpoint{6.413169in}{5.331412in}}%
\pgfpathlineto{\pgfqpoint{6.414290in}{5.328507in}}%
\pgfpathlineto{\pgfqpoint{6.417651in}{5.354653in}}%
\pgfpathlineto{\pgfqpoint{6.418772in}{5.343759in}}%
\pgfpathlineto{\pgfqpoint{6.419892in}{5.344727in}}%
\pgfpathlineto{\pgfqpoint{6.421013in}{5.344727in}}%
\pgfpathlineto{\pgfqpoint{6.422133in}{5.338191in}}%
\pgfpathlineto{\pgfqpoint{6.425495in}{5.335528in}}%
\pgfpathlineto{\pgfqpoint{6.426615in}{5.358768in}}%
\pgfpathlineto{\pgfqpoint{6.427736in}{5.363852in}}%
\pgfpathlineto{\pgfqpoint{6.428856in}{5.323907in}}%
\pgfpathlineto{\pgfqpoint{6.429977in}{5.314950in}}%
\pgfpathlineto{\pgfqpoint{6.433338in}{5.325844in}}%
\pgfpathlineto{\pgfqpoint{6.434459in}{5.324149in}}%
\pgfpathlineto{\pgfqpoint{6.435580in}{5.283720in}}%
\pgfpathlineto{\pgfqpoint{6.436700in}{5.284204in}}%
\pgfpathlineto{\pgfqpoint{6.437821in}{5.286141in}}%
\pgfpathlineto{\pgfqpoint{6.443423in}{5.319066in}}%
\pgfpathlineto{\pgfqpoint{6.444544in}{5.310108in}}%
\pgfpathlineto{\pgfqpoint{6.445664in}{5.316887in}}%
\pgfpathlineto{\pgfqpoint{6.449026in}{5.312045in}}%
\pgfpathlineto{\pgfqpoint{6.450146in}{5.300909in}}%
\pgfpathlineto{\pgfqpoint{6.451267in}{5.297277in}}%
\pgfpathlineto{\pgfqpoint{6.453508in}{5.333107in}}%
\pgfpathlineto{\pgfqpoint{6.456870in}{5.335286in}}%
\pgfpathlineto{\pgfqpoint{6.457990in}{5.327539in}}%
\pgfpathlineto{\pgfqpoint{6.459111in}{5.326812in}}%
\pgfpathlineto{\pgfqpoint{6.460231in}{5.315676in}}%
\pgfpathlineto{\pgfqpoint{6.461352in}{5.239176in}}%
\pgfpathlineto{\pgfqpoint{6.465834in}{5.210125in}}%
\pgfpathlineto{\pgfqpoint{6.466954in}{5.207462in}}%
\pgfpathlineto{\pgfqpoint{6.468075in}{5.221987in}}%
\pgfpathlineto{\pgfqpoint{6.469195in}{5.211577in}}%
\pgfpathlineto{\pgfqpoint{6.472557in}{5.193905in}}%
\pgfpathlineto{\pgfqpoint{6.473677in}{5.195599in}}%
\pgfpathlineto{\pgfqpoint{6.474798in}{5.208188in}}%
\pgfpathlineto{\pgfqpoint{6.475919in}{5.199473in}}%
\pgfpathlineto{\pgfqpoint{6.477039in}{5.200925in}}%
\pgfpathlineto{\pgfqpoint{6.480401in}{5.188821in}}%
\pgfpathlineto{\pgfqpoint{6.482642in}{5.223924in}}%
\pgfpathlineto{\pgfqpoint{6.483762in}{5.228524in}}%
\pgfpathlineto{\pgfqpoint{6.484883in}{5.237239in}}%
\pgfpathlineto{\pgfqpoint{6.488244in}{5.256364in}}%
\pgfpathlineto{\pgfqpoint{6.489365in}{5.253459in}}%
\pgfpathlineto{\pgfqpoint{6.490485in}{5.238691in}}%
\pgfpathlineto{\pgfqpoint{6.491606in}{5.262174in}}%
\pgfpathlineto{\pgfqpoint{6.492726in}{5.243291in}}%
\pgfpathlineto{\pgfqpoint{6.496088in}{5.239660in}}%
\pgfpathlineto{\pgfqpoint{6.497209in}{5.250070in}}%
\pgfpathlineto{\pgfqpoint{6.498329in}{5.241112in}}%
\pgfpathlineto{\pgfqpoint{6.500570in}{5.243775in}}%
\pgfpathlineto{\pgfqpoint{6.503932in}{5.255396in}}%
\pgfpathlineto{\pgfqpoint{6.505052in}{5.265563in}}%
\pgfpathlineto{\pgfqpoint{6.506173in}{5.265079in}}%
\pgfpathlineto{\pgfqpoint{6.508414in}{5.287594in}}%
\pgfpathlineto{\pgfqpoint{6.511775in}{5.311561in}}%
\pgfpathlineto{\pgfqpoint{6.512896in}{5.311319in}}%
\pgfpathlineto{\pgfqpoint{6.514016in}{5.307929in}}%
\pgfpathlineto{\pgfqpoint{6.515137in}{5.281541in}}%
\pgfpathlineto{\pgfqpoint{6.516257in}{5.287836in}}%
\pgfpathlineto{\pgfqpoint{6.519619in}{5.284689in}}%
\pgfpathlineto{\pgfqpoint{6.520740in}{5.275973in}}%
\pgfpathlineto{\pgfqpoint{6.521860in}{5.299456in}}%
\pgfpathlineto{\pgfqpoint{6.522981in}{5.302119in}}%
\pgfpathlineto{\pgfqpoint{6.524101in}{5.322455in}}%
\pgfpathlineto{\pgfqpoint{6.527463in}{5.322455in}}%
\pgfpathlineto{\pgfqpoint{6.529704in}{5.314224in}}%
\pgfpathlineto{\pgfqpoint{6.530824in}{5.317613in}}%
\pgfpathlineto{\pgfqpoint{6.531945in}{5.328023in}}%
\pgfpathlineto{\pgfqpoint{6.536427in}{5.336012in}}%
\pgfpathlineto{\pgfqpoint{6.537547in}{5.327297in}}%
\pgfpathlineto{\pgfqpoint{6.538668in}{5.326570in}}%
\pgfpathlineto{\pgfqpoint{6.539789in}{5.322455in}}%
\pgfpathlineto{\pgfqpoint{6.539789in}{5.322455in}}%
\pgfusepath{stroke}%
\end{pgfscope}%
\begin{pgfscope}%
\pgfpathrectangle{\pgfqpoint{3.966666in}{4.309196in}}{\pgfqpoint{2.695652in}{1.104878in}}%
\pgfusepath{clip}%
\pgfsetroundcap%
\pgfsetroundjoin%
\pgfsetlinewidth{1.505625pt}%
\definecolor{currentstroke}{rgb}{1.000000,0.647059,0.000000}%
\pgfsetstrokecolor{currentstroke}%
\pgfsetdash{}{0pt}%
\pgfpathmoveto{\pgfqpoint{4.089196in}{4.484094in}}%
\pgfpathlineto{\pgfqpoint{4.090316in}{4.483731in}}%
\pgfpathlineto{\pgfqpoint{4.092557in}{4.480160in}}%
\pgfpathlineto{\pgfqpoint{4.095919in}{4.480221in}}%
\pgfpathlineto{\pgfqpoint{4.097039in}{4.479253in}}%
\pgfpathlineto{\pgfqpoint{4.099280in}{4.475319in}}%
\pgfpathlineto{\pgfqpoint{4.100401in}{4.474142in}}%
\pgfpathlineto{\pgfqpoint{4.106004in}{4.474477in}}%
\pgfpathlineto{\pgfqpoint{4.108245in}{4.475510in}}%
\pgfpathlineto{\pgfqpoint{4.111606in}{4.474342in}}%
\pgfpathlineto{\pgfqpoint{4.113847in}{4.471839in}}%
\pgfpathlineto{\pgfqpoint{4.119450in}{4.467135in}}%
\pgfpathlineto{\pgfqpoint{4.122811in}{4.461349in}}%
\pgfpathlineto{\pgfqpoint{4.123932in}{4.459401in}}%
\pgfpathlineto{\pgfqpoint{4.128414in}{4.457261in}}%
\pgfpathlineto{\pgfqpoint{4.131776in}{4.455121in}}%
\pgfpathlineto{\pgfqpoint{4.139619in}{4.454794in}}%
\pgfpathlineto{\pgfqpoint{4.145222in}{4.454421in}}%
\pgfpathlineto{\pgfqpoint{4.147463in}{4.456293in}}%
\pgfpathlineto{\pgfqpoint{4.150825in}{4.457222in}}%
\pgfpathlineto{\pgfqpoint{4.154186in}{4.460470in}}%
\pgfpathlineto{\pgfqpoint{4.155307in}{4.461200in}}%
\pgfpathlineto{\pgfqpoint{4.159789in}{4.462752in}}%
\pgfpathlineto{\pgfqpoint{4.170994in}{4.469103in}}%
\pgfpathlineto{\pgfqpoint{4.175476in}{4.470295in}}%
\pgfpathlineto{\pgfqpoint{4.178838in}{4.472105in}}%
\pgfpathlineto{\pgfqpoint{4.184440in}{4.473624in}}%
\pgfpathlineto{\pgfqpoint{4.190043in}{4.474987in}}%
\pgfpathlineto{\pgfqpoint{4.193405in}{4.476193in}}%
\pgfpathlineto{\pgfqpoint{4.209092in}{4.477616in}}%
\pgfpathlineto{\pgfqpoint{4.210213in}{4.478004in}}%
\pgfpathlineto{\pgfqpoint{4.215815in}{4.478765in}}%
\pgfpathlineto{\pgfqpoint{4.216936in}{4.479101in}}%
\pgfpathlineto{\pgfqpoint{4.225900in}{4.477192in}}%
\pgfpathlineto{\pgfqpoint{4.231503in}{4.476309in}}%
\pgfpathlineto{\pgfqpoint{4.241587in}{4.474204in}}%
\pgfpathlineto{\pgfqpoint{4.246069in}{4.473430in}}%
\pgfpathlineto{\pgfqpoint{4.249431in}{4.471880in}}%
\pgfpathlineto{\pgfqpoint{4.255034in}{4.470953in}}%
\pgfpathlineto{\pgfqpoint{4.257275in}{4.469800in}}%
\pgfpathlineto{\pgfqpoint{4.261757in}{4.468460in}}%
\pgfpathlineto{\pgfqpoint{4.265118in}{4.467148in}}%
\pgfpathlineto{\pgfqpoint{4.270721in}{4.466000in}}%
\pgfpathlineto{\pgfqpoint{4.278565in}{4.463866in}}%
\pgfpathlineto{\pgfqpoint{4.280806in}{4.462292in}}%
\pgfpathlineto{\pgfqpoint{4.285288in}{4.460612in}}%
\pgfpathlineto{\pgfqpoint{4.288649in}{4.458754in}}%
\pgfpathlineto{\pgfqpoint{4.295373in}{4.457342in}}%
\pgfpathlineto{\pgfqpoint{4.296493in}{4.456883in}}%
\pgfpathlineto{\pgfqpoint{4.307698in}{4.455781in}}%
\pgfpathlineto{\pgfqpoint{4.316663in}{4.456513in}}%
\pgfpathlineto{\pgfqpoint{4.327868in}{4.457323in}}%
\pgfpathlineto{\pgfqpoint{4.333471in}{4.458259in}}%
\pgfpathlineto{\pgfqpoint{4.335712in}{4.458989in}}%
\pgfpathlineto{\pgfqpoint{4.341314in}{4.460001in}}%
\pgfpathlineto{\pgfqpoint{4.343555in}{4.460749in}}%
\pgfpathlineto{\pgfqpoint{4.349158in}{4.461759in}}%
\pgfpathlineto{\pgfqpoint{4.359243in}{4.464114in}}%
\pgfpathlineto{\pgfqpoint{4.364845in}{4.464839in}}%
\pgfpathlineto{\pgfqpoint{4.367086in}{4.465778in}}%
\pgfpathlineto{\pgfqpoint{4.372689in}{4.467119in}}%
\pgfpathlineto{\pgfqpoint{4.374930in}{4.468150in}}%
\pgfpathlineto{\pgfqpoint{4.379412in}{4.469202in}}%
\pgfpathlineto{\pgfqpoint{4.382774in}{4.470793in}}%
\pgfpathlineto{\pgfqpoint{4.387256in}{4.471879in}}%
\pgfpathlineto{\pgfqpoint{4.390617in}{4.473355in}}%
\pgfpathlineto{\pgfqpoint{4.396220in}{4.474728in}}%
\pgfpathlineto{\pgfqpoint{4.398461in}{4.475693in}}%
\pgfpathlineto{\pgfqpoint{4.404064in}{4.476850in}}%
\pgfpathlineto{\pgfqpoint{4.414149in}{4.479641in}}%
\pgfpathlineto{\pgfqpoint{4.419751in}{4.480662in}}%
\pgfpathlineto{\pgfqpoint{4.421992in}{4.481655in}}%
\pgfpathlineto{\pgfqpoint{4.428715in}{4.482534in}}%
\pgfpathlineto{\pgfqpoint{4.429836in}{4.482962in}}%
\pgfpathlineto{\pgfqpoint{4.436559in}{4.484253in}}%
\pgfpathlineto{\pgfqpoint{4.444403in}{4.485186in}}%
\pgfpathlineto{\pgfqpoint{4.460090in}{4.488272in}}%
\pgfpathlineto{\pgfqpoint{4.461211in}{4.488694in}}%
\pgfpathlineto{\pgfqpoint{4.466813in}{4.489838in}}%
\pgfpathlineto{\pgfqpoint{4.469054in}{4.490700in}}%
\pgfpathlineto{\pgfqpoint{4.473536in}{4.491599in}}%
\pgfpathlineto{\pgfqpoint{4.476898in}{4.492880in}}%
\pgfpathlineto{\pgfqpoint{4.482501in}{4.494002in}}%
\pgfpathlineto{\pgfqpoint{4.484742in}{4.494659in}}%
\pgfpathlineto{\pgfqpoint{4.499309in}{4.496330in}}%
\pgfpathlineto{\pgfqpoint{4.500429in}{4.496625in}}%
\pgfpathlineto{\pgfqpoint{4.506032in}{4.497400in}}%
\pgfpathlineto{\pgfqpoint{4.508273in}{4.497993in}}%
\pgfpathlineto{\pgfqpoint{4.513875in}{4.498991in}}%
\pgfpathlineto{\pgfqpoint{4.516117in}{4.499728in}}%
\pgfpathlineto{\pgfqpoint{4.521719in}{4.500531in}}%
\pgfpathlineto{\pgfqpoint{4.529563in}{4.503649in}}%
\pgfpathlineto{\pgfqpoint{4.531804in}{4.505159in}}%
\pgfpathlineto{\pgfqpoint{4.536286in}{4.506638in}}%
\pgfpathlineto{\pgfqpoint{4.539648in}{4.508932in}}%
\pgfpathlineto{\pgfqpoint{4.544130in}{4.510416in}}%
\pgfpathlineto{\pgfqpoint{4.547491in}{4.512757in}}%
\pgfpathlineto{\pgfqpoint{4.551973in}{4.513579in}}%
\pgfpathlineto{\pgfqpoint{4.555335in}{4.515940in}}%
\pgfpathlineto{\pgfqpoint{4.559817in}{4.517343in}}%
\pgfpathlineto{\pgfqpoint{4.563179in}{4.519510in}}%
\pgfpathlineto{\pgfqpoint{4.567661in}{4.520981in}}%
\pgfpathlineto{\pgfqpoint{4.571022in}{4.523199in}}%
\pgfpathlineto{\pgfqpoint{4.575504in}{4.524674in}}%
\pgfpathlineto{\pgfqpoint{4.578866in}{4.526768in}}%
\pgfpathlineto{\pgfqpoint{4.583348in}{4.528113in}}%
\pgfpathlineto{\pgfqpoint{4.586710in}{4.530238in}}%
\pgfpathlineto{\pgfqpoint{4.591192in}{4.531591in}}%
\pgfpathlineto{\pgfqpoint{4.593433in}{4.532930in}}%
\pgfpathlineto{\pgfqpoint{4.597915in}{4.533633in}}%
\pgfpathlineto{\pgfqpoint{4.602397in}{4.536593in}}%
\pgfpathlineto{\pgfqpoint{4.606879in}{4.538052in}}%
\pgfpathlineto{\pgfqpoint{4.610241in}{4.540418in}}%
\pgfpathlineto{\pgfqpoint{4.614723in}{4.541997in}}%
\pgfpathlineto{\pgfqpoint{4.618084in}{4.544373in}}%
\pgfpathlineto{\pgfqpoint{4.621446in}{4.545233in}}%
\pgfpathlineto{\pgfqpoint{4.624808in}{4.547371in}}%
\pgfpathlineto{\pgfqpoint{4.625928in}{4.547989in}}%
\pgfpathlineto{\pgfqpoint{4.630410in}{4.549226in}}%
\pgfpathlineto{\pgfqpoint{4.633772in}{4.551130in}}%
\pgfpathlineto{\pgfqpoint{4.638254in}{4.552404in}}%
\pgfpathlineto{\pgfqpoint{4.641616in}{4.554397in}}%
\pgfpathlineto{\pgfqpoint{4.646098in}{4.555771in}}%
\pgfpathlineto{\pgfqpoint{4.649459in}{4.558021in}}%
\pgfpathlineto{\pgfqpoint{4.653941in}{4.559345in}}%
\pgfpathlineto{\pgfqpoint{4.657303in}{4.561448in}}%
\pgfpathlineto{\pgfqpoint{4.662906in}{4.562839in}}%
\pgfpathlineto{\pgfqpoint{4.665147in}{4.563995in}}%
\pgfpathlineto{\pgfqpoint{4.669629in}{4.565091in}}%
\pgfpathlineto{\pgfqpoint{4.672990in}{4.566641in}}%
\pgfpathlineto{\pgfqpoint{4.677472in}{4.567772in}}%
\pgfpathlineto{\pgfqpoint{4.680834in}{4.569433in}}%
\pgfpathlineto{\pgfqpoint{4.685316in}{4.570632in}}%
\pgfpathlineto{\pgfqpoint{4.688678in}{4.572044in}}%
\pgfpathlineto{\pgfqpoint{4.693160in}{4.572959in}}%
\pgfpathlineto{\pgfqpoint{4.696521in}{4.574427in}}%
\pgfpathlineto{\pgfqpoint{4.701003in}{4.575486in}}%
\pgfpathlineto{\pgfqpoint{4.704365in}{4.576556in}}%
\pgfpathlineto{\pgfqpoint{4.708847in}{4.577691in}}%
\pgfpathlineto{\pgfqpoint{4.712209in}{4.579610in}}%
\pgfpathlineto{\pgfqpoint{4.716691in}{4.580927in}}%
\pgfpathlineto{\pgfqpoint{4.720052in}{4.582870in}}%
\pgfpathlineto{\pgfqpoint{4.724535in}{4.584200in}}%
\pgfpathlineto{\pgfqpoint{4.727896in}{4.586059in}}%
\pgfpathlineto{\pgfqpoint{4.732378in}{4.587273in}}%
\pgfpathlineto{\pgfqpoint{4.735740in}{4.589193in}}%
\pgfpathlineto{\pgfqpoint{4.740222in}{4.590507in}}%
\pgfpathlineto{\pgfqpoint{4.743584in}{4.592505in}}%
\pgfpathlineto{\pgfqpoint{4.748066in}{4.593794in}}%
\pgfpathlineto{\pgfqpoint{4.751427in}{4.595540in}}%
\pgfpathlineto{\pgfqpoint{4.755909in}{4.596595in}}%
\pgfpathlineto{\pgfqpoint{4.759271in}{4.598181in}}%
\pgfpathlineto{\pgfqpoint{4.764874in}{4.599465in}}%
\pgfpathlineto{\pgfqpoint{4.767115in}{4.600292in}}%
\pgfpathlineto{\pgfqpoint{4.772717in}{4.601114in}}%
\pgfpathlineto{\pgfqpoint{4.774958in}{4.601883in}}%
\pgfpathlineto{\pgfqpoint{4.779440in}{4.602731in}}%
\pgfpathlineto{\pgfqpoint{4.782802in}{4.604054in}}%
\pgfpathlineto{\pgfqpoint{4.787284in}{4.605063in}}%
\pgfpathlineto{\pgfqpoint{4.790646in}{4.606550in}}%
\pgfpathlineto{\pgfqpoint{4.796248in}{4.607825in}}%
\pgfpathlineto{\pgfqpoint{4.798489in}{4.608581in}}%
\pgfpathlineto{\pgfqpoint{4.804092in}{4.609465in}}%
\pgfpathlineto{\pgfqpoint{4.806333in}{4.610052in}}%
\pgfpathlineto{\pgfqpoint{4.811936in}{4.610972in}}%
\pgfpathlineto{\pgfqpoint{4.814177in}{4.611747in}}%
\pgfpathlineto{\pgfqpoint{4.819779in}{4.612929in}}%
\pgfpathlineto{\pgfqpoint{4.822020in}{4.613851in}}%
\pgfpathlineto{\pgfqpoint{4.826503in}{4.614788in}}%
\pgfpathlineto{\pgfqpoint{4.829864in}{4.616287in}}%
\pgfpathlineto{\pgfqpoint{4.834346in}{4.617397in}}%
\pgfpathlineto{\pgfqpoint{4.837708in}{4.618946in}}%
\pgfpathlineto{\pgfqpoint{4.842190in}{4.619989in}}%
\pgfpathlineto{\pgfqpoint{4.845551in}{4.621687in}}%
\pgfpathlineto{\pgfqpoint{4.850034in}{4.622806in}}%
\pgfpathlineto{\pgfqpoint{4.853395in}{4.624684in}}%
\pgfpathlineto{\pgfqpoint{4.857877in}{4.625942in}}%
\pgfpathlineto{\pgfqpoint{4.861239in}{4.627849in}}%
\pgfpathlineto{\pgfqpoint{4.865721in}{4.629129in}}%
\pgfpathlineto{\pgfqpoint{4.869083in}{4.630333in}}%
\pgfpathlineto{\pgfqpoint{4.873565in}{4.631472in}}%
\pgfpathlineto{\pgfqpoint{4.876926in}{4.633154in}}%
\pgfpathlineto{\pgfqpoint{4.881408in}{4.634322in}}%
\pgfpathlineto{\pgfqpoint{4.884770in}{4.635893in}}%
\pgfpathlineto{\pgfqpoint{4.889252in}{4.636799in}}%
\pgfpathlineto{\pgfqpoint{4.892614in}{4.638235in}}%
\pgfpathlineto{\pgfqpoint{4.899337in}{4.639583in}}%
\pgfpathlineto{\pgfqpoint{4.900457in}{4.640051in}}%
\pgfpathlineto{\pgfqpoint{4.907180in}{4.641360in}}%
\pgfpathlineto{\pgfqpoint{4.908301in}{4.641758in}}%
\pgfpathlineto{\pgfqpoint{4.913904in}{4.642982in}}%
\pgfpathlineto{\pgfqpoint{4.916145in}{4.643757in}}%
\pgfpathlineto{\pgfqpoint{4.921747in}{4.644932in}}%
\pgfpathlineto{\pgfqpoint{4.923988in}{4.645681in}}%
\pgfpathlineto{\pgfqpoint{4.930712in}{4.646722in}}%
\pgfpathlineto{\pgfqpoint{4.931832in}{4.647073in}}%
\pgfpathlineto{\pgfqpoint{4.937435in}{4.648028in}}%
\pgfpathlineto{\pgfqpoint{4.943037in}{4.648738in}}%
\pgfpathlineto{\pgfqpoint{4.963207in}{4.652370in}}%
\pgfpathlineto{\pgfqpoint{4.968809in}{4.653210in}}%
\pgfpathlineto{\pgfqpoint{4.971051in}{4.653800in}}%
\pgfpathlineto{\pgfqpoint{4.976653in}{4.654621in}}%
\pgfpathlineto{\pgfqpoint{4.986738in}{4.656719in}}%
\pgfpathlineto{\pgfqpoint{4.992341in}{4.657679in}}%
\pgfpathlineto{\pgfqpoint{5.002425in}{4.659806in}}%
\pgfpathlineto{\pgfqpoint{5.008028in}{4.660829in}}%
\pgfpathlineto{\pgfqpoint{5.010269in}{4.661475in}}%
\pgfpathlineto{\pgfqpoint{5.015872in}{4.662561in}}%
\pgfpathlineto{\pgfqpoint{5.018113in}{4.663265in}}%
\pgfpathlineto{\pgfqpoint{5.023715in}{4.664331in}}%
\pgfpathlineto{\pgfqpoint{5.024836in}{4.664707in}}%
\pgfpathlineto{\pgfqpoint{5.031559in}{4.665777in}}%
\pgfpathlineto{\pgfqpoint{5.033800in}{4.666530in}}%
\pgfpathlineto{\pgfqpoint{5.039403in}{4.667792in}}%
\pgfpathlineto{\pgfqpoint{5.041644in}{4.668591in}}%
\pgfpathlineto{\pgfqpoint{5.047246in}{4.669728in}}%
\pgfpathlineto{\pgfqpoint{5.049487in}{4.670524in}}%
\pgfpathlineto{\pgfqpoint{5.055090in}{4.671627in}}%
\pgfpathlineto{\pgfqpoint{5.057331in}{4.672283in}}%
\pgfpathlineto{\pgfqpoint{5.062934in}{4.673236in}}%
\pgfpathlineto{\pgfqpoint{5.065175in}{4.673891in}}%
\pgfpathlineto{\pgfqpoint{5.071898in}{4.674826in}}%
\pgfpathlineto{\pgfqpoint{5.073019in}{4.675156in}}%
\pgfpathlineto{\pgfqpoint{5.078621in}{4.676064in}}%
\pgfpathlineto{\pgfqpoint{5.080862in}{4.676667in}}%
\pgfpathlineto{\pgfqpoint{5.086465in}{4.677565in}}%
\pgfpathlineto{\pgfqpoint{5.088706in}{4.678132in}}%
\pgfpathlineto{\pgfqpoint{5.094309in}{4.678976in}}%
\pgfpathlineto{\pgfqpoint{5.096550in}{4.679559in}}%
\pgfpathlineto{\pgfqpoint{5.103273in}{4.680589in}}%
\pgfpathlineto{\pgfqpoint{5.111116in}{4.681885in}}%
\pgfpathlineto{\pgfqpoint{5.117840in}{4.682800in}}%
\pgfpathlineto{\pgfqpoint{5.120081in}{4.683443in}}%
\pgfpathlineto{\pgfqpoint{5.125683in}{4.684400in}}%
\pgfpathlineto{\pgfqpoint{5.127924in}{4.685016in}}%
\pgfpathlineto{\pgfqpoint{5.133527in}{4.685899in}}%
\pgfpathlineto{\pgfqpoint{5.135768in}{4.686469in}}%
\pgfpathlineto{\pgfqpoint{5.142491in}{4.687399in}}%
\pgfpathlineto{\pgfqpoint{5.151455in}{4.689100in}}%
\pgfpathlineto{\pgfqpoint{5.157058in}{4.690070in}}%
\pgfpathlineto{\pgfqpoint{5.159299in}{4.690737in}}%
\pgfpathlineto{\pgfqpoint{5.164902in}{4.691803in}}%
\pgfpathlineto{\pgfqpoint{5.167143in}{4.692551in}}%
\pgfpathlineto{\pgfqpoint{5.172745in}{4.693672in}}%
\pgfpathlineto{\pgfqpoint{5.174986in}{4.694393in}}%
\pgfpathlineto{\pgfqpoint{5.181710in}{4.695477in}}%
\pgfpathlineto{\pgfqpoint{5.182830in}{4.695854in}}%
\pgfpathlineto{\pgfqpoint{5.188433in}{4.696937in}}%
\pgfpathlineto{\pgfqpoint{5.190674in}{4.697656in}}%
\pgfpathlineto{\pgfqpoint{5.196276in}{4.698788in}}%
\pgfpathlineto{\pgfqpoint{5.198518in}{4.699556in}}%
\pgfpathlineto{\pgfqpoint{5.204120in}{4.700743in}}%
\pgfpathlineto{\pgfqpoint{5.206361in}{4.701506in}}%
\pgfpathlineto{\pgfqpoint{5.211964in}{4.702581in}}%
\pgfpathlineto{\pgfqpoint{5.214205in}{4.703272in}}%
\pgfpathlineto{\pgfqpoint{5.219808in}{4.704320in}}%
\pgfpathlineto{\pgfqpoint{5.222049in}{4.705046in}}%
\pgfpathlineto{\pgfqpoint{5.227651in}{4.706048in}}%
\pgfpathlineto{\pgfqpoint{5.237736in}{4.708523in}}%
\pgfpathlineto{\pgfqpoint{5.242218in}{4.709381in}}%
\pgfpathlineto{\pgfqpoint{5.245580in}{4.710721in}}%
\pgfpathlineto{\pgfqpoint{5.250062in}{4.711672in}}%
\pgfpathlineto{\pgfqpoint{5.253423in}{4.713175in}}%
\pgfpathlineto{\pgfqpoint{5.257905in}{4.714202in}}%
\pgfpathlineto{\pgfqpoint{5.261267in}{4.715661in}}%
\pgfpathlineto{\pgfqpoint{5.265749in}{4.716578in}}%
\pgfpathlineto{\pgfqpoint{5.269111in}{4.718006in}}%
\pgfpathlineto{\pgfqpoint{5.273593in}{4.718942in}}%
\pgfpathlineto{\pgfqpoint{5.276954in}{4.719942in}}%
\pgfpathlineto{\pgfqpoint{5.281437in}{4.720989in}}%
\pgfpathlineto{\pgfqpoint{5.284798in}{4.722526in}}%
\pgfpathlineto{\pgfqpoint{5.289280in}{4.723567in}}%
\pgfpathlineto{\pgfqpoint{5.292642in}{4.725056in}}%
\pgfpathlineto{\pgfqpoint{5.297124in}{4.726003in}}%
\pgfpathlineto{\pgfqpoint{5.300486in}{4.727605in}}%
\pgfpathlineto{\pgfqpoint{5.304968in}{4.728742in}}%
\pgfpathlineto{\pgfqpoint{5.308329in}{4.729892in}}%
\pgfpathlineto{\pgfqpoint{5.312811in}{4.730994in}}%
\pgfpathlineto{\pgfqpoint{5.316173in}{4.731992in}}%
\pgfpathlineto{\pgfqpoint{5.320655in}{4.732933in}}%
\pgfpathlineto{\pgfqpoint{5.324017in}{4.734389in}}%
\pgfpathlineto{\pgfqpoint{5.328499in}{4.735334in}}%
\pgfpathlineto{\pgfqpoint{5.331860in}{4.736763in}}%
\pgfpathlineto{\pgfqpoint{5.337463in}{4.737759in}}%
\pgfpathlineto{\pgfqpoint{5.339704in}{4.738761in}}%
\pgfpathlineto{\pgfqpoint{5.345307in}{4.739947in}}%
\pgfpathlineto{\pgfqpoint{5.347548in}{4.740619in}}%
\pgfpathlineto{\pgfqpoint{5.353150in}{4.741673in}}%
\pgfpathlineto{\pgfqpoint{5.355391in}{4.742400in}}%
\pgfpathlineto{\pgfqpoint{5.360994in}{4.743420in}}%
\pgfpathlineto{\pgfqpoint{5.363235in}{4.744127in}}%
\pgfpathlineto{\pgfqpoint{5.369958in}{4.745158in}}%
\pgfpathlineto{\pgfqpoint{5.371079in}{4.745478in}}%
\pgfpathlineto{\pgfqpoint{5.376681in}{4.746481in}}%
\pgfpathlineto{\pgfqpoint{5.378922in}{4.747130in}}%
\pgfpathlineto{\pgfqpoint{5.384525in}{4.748083in}}%
\pgfpathlineto{\pgfqpoint{5.386766in}{4.748645in}}%
\pgfpathlineto{\pgfqpoint{5.394610in}{4.749821in}}%
\pgfpathlineto{\pgfqpoint{5.401333in}{4.750896in}}%
\pgfpathlineto{\pgfqpoint{5.402453in}{4.751198in}}%
\pgfpathlineto{\pgfqpoint{5.409177in}{4.752270in}}%
\pgfpathlineto{\pgfqpoint{5.417020in}{4.753449in}}%
\pgfpathlineto{\pgfqpoint{5.424864in}{4.754426in}}%
\pgfpathlineto{\pgfqpoint{5.433828in}{4.755956in}}%
\pgfpathlineto{\pgfqpoint{5.440551in}{4.756898in}}%
\pgfpathlineto{\pgfqpoint{5.449516in}{4.758032in}}%
\pgfpathlineto{\pgfqpoint{5.460721in}{4.759165in}}%
\pgfpathlineto{\pgfqpoint{5.473047in}{4.760842in}}%
\pgfpathlineto{\pgfqpoint{5.485372in}{4.761724in}}%
\pgfpathlineto{\pgfqpoint{5.496578in}{4.762808in}}%
\pgfpathlineto{\pgfqpoint{5.504421in}{4.763616in}}%
\pgfpathlineto{\pgfqpoint{5.517868in}{4.764816in}}%
\pgfpathlineto{\pgfqpoint{5.527953in}{4.765898in}}%
\pgfpathlineto{\pgfqpoint{5.534676in}{4.766743in}}%
\pgfpathlineto{\pgfqpoint{5.543640in}{4.767931in}}%
\pgfpathlineto{\pgfqpoint{5.580617in}{4.769351in}}%
\pgfpathlineto{\pgfqpoint{5.637764in}{4.768239in}}%
\pgfpathlineto{\pgfqpoint{5.652331in}{4.768982in}}%
\pgfpathlineto{\pgfqpoint{5.669139in}{4.769899in}}%
\pgfpathlineto{\pgfqpoint{5.691549in}{4.770913in}}%
\pgfpathlineto{\pgfqpoint{5.715081in}{4.773054in}}%
\pgfpathlineto{\pgfqpoint{5.729647in}{4.774086in}}%
\pgfpathlineto{\pgfqpoint{5.739732in}{4.774654in}}%
\pgfpathlineto{\pgfqpoint{5.753178in}{4.775462in}}%
\pgfpathlineto{\pgfqpoint{5.771107in}{4.777811in}}%
\pgfpathlineto{\pgfqpoint{5.782312in}{4.778860in}}%
\pgfpathlineto{\pgfqpoint{5.794638in}{4.780713in}}%
\pgfpathlineto{\pgfqpoint{5.805843in}{4.781989in}}%
\pgfpathlineto{\pgfqpoint{5.817049in}{4.783694in}}%
\pgfpathlineto{\pgfqpoint{5.824892in}{4.784542in}}%
\pgfpathlineto{\pgfqpoint{5.833856in}{4.785893in}}%
\pgfpathlineto{\pgfqpoint{5.840580in}{4.786731in}}%
\pgfpathlineto{\pgfqpoint{5.849544in}{4.787969in}}%
\pgfpathlineto{\pgfqpoint{5.861870in}{4.789179in}}%
\pgfpathlineto{\pgfqpoint{5.873075in}{4.790786in}}%
\pgfpathlineto{\pgfqpoint{5.885401in}{4.791975in}}%
\pgfpathlineto{\pgfqpoint{5.888762in}{4.792532in}}%
\pgfpathlineto{\pgfqpoint{5.899968in}{4.793515in}}%
\pgfpathlineto{\pgfqpoint{5.912293in}{4.795433in}}%
\pgfpathlineto{\pgfqpoint{5.919017in}{4.796321in}}%
\pgfpathlineto{\pgfqpoint{5.927981in}{4.797601in}}%
\pgfpathlineto{\pgfqpoint{5.934704in}{4.798345in}}%
\pgfpathlineto{\pgfqpoint{5.943668in}{4.799906in}}%
\pgfpathlineto{\pgfqpoint{5.950391in}{4.800950in}}%
\pgfpathlineto{\pgfqpoint{5.957114in}{4.801985in}}%
\pgfpathlineto{\pgfqpoint{5.967199in}{4.803859in}}%
\pgfpathlineto{\pgfqpoint{5.973922in}{4.804942in}}%
\pgfpathlineto{\pgfqpoint{5.978404in}{4.805512in}}%
\pgfpathlineto{\pgfqpoint{5.990730in}{4.808092in}}%
\pgfpathlineto{\pgfqpoint{5.996333in}{4.808969in}}%
\pgfpathlineto{\pgfqpoint{5.998574in}{4.809571in}}%
\pgfpathlineto{\pgfqpoint{6.005297in}{4.810463in}}%
\pgfpathlineto{\pgfqpoint{6.014261in}{4.812158in}}%
\pgfpathlineto{\pgfqpoint{6.019864in}{4.813044in}}%
\pgfpathlineto{\pgfqpoint{6.022105in}{4.813637in}}%
\pgfpathlineto{\pgfqpoint{6.027708in}{4.814533in}}%
\pgfpathlineto{\pgfqpoint{6.037792in}{4.816663in}}%
\pgfpathlineto{\pgfqpoint{6.043395in}{4.817553in}}%
\pgfpathlineto{\pgfqpoint{6.045636in}{4.818129in}}%
\pgfpathlineto{\pgfqpoint{6.052359in}{4.819162in}}%
\pgfpathlineto{\pgfqpoint{6.061323in}{4.820692in}}%
\pgfpathlineto{\pgfqpoint{6.068047in}{4.821755in}}%
\pgfpathlineto{\pgfqpoint{6.077011in}{4.823175in}}%
\pgfpathlineto{\pgfqpoint{6.088216in}{4.824328in}}%
\pgfpathlineto{\pgfqpoint{6.098301in}{4.825466in}}%
\pgfpathlineto{\pgfqpoint{6.116229in}{4.827915in}}%
\pgfpathlineto{\pgfqpoint{6.122952in}{4.828768in}}%
\pgfpathlineto{\pgfqpoint{6.124073in}{4.828988in}}%
\pgfpathlineto{\pgfqpoint{6.136399in}{4.830023in}}%
\pgfpathlineto{\pgfqpoint{6.147604in}{4.831661in}}%
\pgfpathlineto{\pgfqpoint{6.155448in}{4.832599in}}%
\pgfpathlineto{\pgfqpoint{6.162171in}{4.833654in}}%
\pgfpathlineto{\pgfqpoint{6.171135in}{4.835236in}}%
\pgfpathlineto{\pgfqpoint{6.176738in}{4.836055in}}%
\pgfpathlineto{\pgfqpoint{6.178979in}{4.836614in}}%
\pgfpathlineto{\pgfqpoint{6.184581in}{4.837489in}}%
\pgfpathlineto{\pgfqpoint{6.186822in}{4.838133in}}%
\pgfpathlineto{\pgfqpoint{6.193546in}{4.839121in}}%
\pgfpathlineto{\pgfqpoint{6.194666in}{4.839443in}}%
\pgfpathlineto{\pgfqpoint{6.200269in}{4.840415in}}%
\pgfpathlineto{\pgfqpoint{6.202510in}{4.841042in}}%
\pgfpathlineto{\pgfqpoint{6.208113in}{4.841958in}}%
\pgfpathlineto{\pgfqpoint{6.210354in}{4.842581in}}%
\pgfpathlineto{\pgfqpoint{6.215956in}{4.843539in}}%
\pgfpathlineto{\pgfqpoint{6.218197in}{4.844175in}}%
\pgfpathlineto{\pgfqpoint{6.223800in}{4.845120in}}%
\pgfpathlineto{\pgfqpoint{6.226041in}{4.845733in}}%
\pgfpathlineto{\pgfqpoint{6.231644in}{4.846645in}}%
\pgfpathlineto{\pgfqpoint{6.233885in}{4.847230in}}%
\pgfpathlineto{\pgfqpoint{6.239487in}{4.848097in}}%
\pgfpathlineto{\pgfqpoint{6.248451in}{4.849804in}}%
\pgfpathlineto{\pgfqpoint{6.255175in}{4.850691in}}%
\pgfpathlineto{\pgfqpoint{6.257416in}{4.851252in}}%
\pgfpathlineto{\pgfqpoint{6.264139in}{4.852354in}}%
\pgfpathlineto{\pgfqpoint{6.271983in}{4.853551in}}%
\pgfpathlineto{\pgfqpoint{6.280947in}{4.854943in}}%
\pgfpathlineto{\pgfqpoint{6.287670in}{4.855850in}}%
\pgfpathlineto{\pgfqpoint{6.296634in}{4.857236in}}%
\pgfpathlineto{\pgfqpoint{6.303357in}{4.857992in}}%
\pgfpathlineto{\pgfqpoint{6.312322in}{4.859553in}}%
\pgfpathlineto{\pgfqpoint{6.319045in}{4.860589in}}%
\pgfpathlineto{\pgfqpoint{6.326888in}{4.861956in}}%
\pgfpathlineto{\pgfqpoint{6.332491in}{4.862753in}}%
\pgfpathlineto{\pgfqpoint{6.339214in}{4.863708in}}%
\pgfpathlineto{\pgfqpoint{6.356022in}{4.866037in}}%
\pgfpathlineto{\pgfqpoint{6.367227in}{4.868195in}}%
\pgfpathlineto{\pgfqpoint{6.372830in}{4.869088in}}%
\pgfpathlineto{\pgfqpoint{6.375071in}{4.869674in}}%
\pgfpathlineto{\pgfqpoint{6.380674in}{4.870593in}}%
\pgfpathlineto{\pgfqpoint{6.382915in}{4.871200in}}%
\pgfpathlineto{\pgfqpoint{6.388517in}{4.872133in}}%
\pgfpathlineto{\pgfqpoint{6.390758in}{4.872759in}}%
\pgfpathlineto{\pgfqpoint{6.396361in}{4.873715in}}%
\pgfpathlineto{\pgfqpoint{6.398602in}{4.874341in}}%
\pgfpathlineto{\pgfqpoint{6.404205in}{4.875267in}}%
\pgfpathlineto{\pgfqpoint{6.406446in}{4.875888in}}%
\pgfpathlineto{\pgfqpoint{6.413169in}{4.876836in}}%
\pgfpathlineto{\pgfqpoint{6.414290in}{4.877152in}}%
\pgfpathlineto{\pgfqpoint{6.419892in}{4.878136in}}%
\pgfpathlineto{\pgfqpoint{6.422133in}{4.878782in}}%
\pgfpathlineto{\pgfqpoint{6.427736in}{4.879770in}}%
\pgfpathlineto{\pgfqpoint{6.429977in}{4.880380in}}%
\pgfpathlineto{\pgfqpoint{6.436700in}{4.881554in}}%
\pgfpathlineto{\pgfqpoint{6.445664in}{4.883318in}}%
\pgfpathlineto{\pgfqpoint{6.451267in}{4.884184in}}%
\pgfpathlineto{\pgfqpoint{6.460231in}{4.885997in}}%
\pgfpathlineto{\pgfqpoint{6.468075in}{4.887134in}}%
\pgfpathlineto{\pgfqpoint{6.477039in}{4.888417in}}%
\pgfpathlineto{\pgfqpoint{6.483762in}{4.889297in}}%
\pgfpathlineto{\pgfqpoint{6.492726in}{4.890752in}}%
\pgfpathlineto{\pgfqpoint{6.503932in}{4.891947in}}%
\pgfpathlineto{\pgfqpoint{6.516257in}{4.894335in}}%
\pgfpathlineto{\pgfqpoint{6.522981in}{4.895393in}}%
\pgfpathlineto{\pgfqpoint{6.527463in}{4.895962in}}%
\pgfpathlineto{\pgfqpoint{6.531945in}{4.897087in}}%
\pgfpathlineto{\pgfqpoint{6.538668in}{4.897948in}}%
\pgfpathlineto{\pgfqpoint{6.539789in}{4.898230in}}%
\pgfpathlineto{\pgfqpoint{6.539789in}{4.898230in}}%
\pgfusepath{stroke}%
\end{pgfscope}%
\begin{pgfscope}%
\pgfsetrectcap%
\pgfsetmiterjoin%
\pgfsetlinewidth{0.803000pt}%
\definecolor{currentstroke}{rgb}{1.000000,1.000000,1.000000}%
\pgfsetstrokecolor{currentstroke}%
\pgfsetdash{}{0pt}%
\pgfpathmoveto{\pgfqpoint{3.966666in}{4.309196in}}%
\pgfpathlineto{\pgfqpoint{3.966666in}{5.414074in}}%
\pgfusepath{stroke}%
\end{pgfscope}%
\begin{pgfscope}%
\pgfsetrectcap%
\pgfsetmiterjoin%
\pgfsetlinewidth{0.803000pt}%
\definecolor{currentstroke}{rgb}{1.000000,1.000000,1.000000}%
\pgfsetstrokecolor{currentstroke}%
\pgfsetdash{}{0pt}%
\pgfpathmoveto{\pgfqpoint{6.662318in}{4.309196in}}%
\pgfpathlineto{\pgfqpoint{6.662318in}{5.414074in}}%
\pgfusepath{stroke}%
\end{pgfscope}%
\begin{pgfscope}%
\pgfsetrectcap%
\pgfsetmiterjoin%
\pgfsetlinewidth{0.803000pt}%
\definecolor{currentstroke}{rgb}{1.000000,1.000000,1.000000}%
\pgfsetstrokecolor{currentstroke}%
\pgfsetdash{}{0pt}%
\pgfpathmoveto{\pgfqpoint{3.966666in}{4.309196in}}%
\pgfpathlineto{\pgfqpoint{6.662318in}{4.309196in}}%
\pgfusepath{stroke}%
\end{pgfscope}%
\begin{pgfscope}%
\pgfsetrectcap%
\pgfsetmiterjoin%
\pgfsetlinewidth{0.803000pt}%
\definecolor{currentstroke}{rgb}{1.000000,1.000000,1.000000}%
\pgfsetstrokecolor{currentstroke}%
\pgfsetdash{}{0pt}%
\pgfpathmoveto{\pgfqpoint{3.966666in}{5.414074in}}%
\pgfpathlineto{\pgfqpoint{6.662318in}{5.414074in}}%
\pgfusepath{stroke}%
\end{pgfscope}%
\begin{pgfscope}%
\definecolor{textcolor}{rgb}{0.150000,0.150000,0.150000}%
\pgfsetstrokecolor{textcolor}%
\pgfsetfillcolor{textcolor}%
\pgftext[x=5.314492in,y=5.497407in,,base]{\color{textcolor}\rmfamily\fontsize{12.000000}{14.400000}\selectfont PG}%
\end{pgfscope}%
\begin{pgfscope}%
\pgfsetbuttcap%
\pgfsetmiterjoin%
\definecolor{currentfill}{rgb}{0.917647,0.917647,0.949020}%
\pgfsetfillcolor{currentfill}%
\pgfsetlinewidth{0.000000pt}%
\definecolor{currentstroke}{rgb}{0.000000,0.000000,0.000000}%
\pgfsetstrokecolor{currentstroke}%
\pgfsetstrokeopacity{0.000000}%
\pgfsetdash{}{0pt}%
\pgfpathmoveto{\pgfqpoint{0.462318in}{2.320415in}}%
\pgfpathlineto{\pgfqpoint{3.157970in}{2.320415in}}%
\pgfpathlineto{\pgfqpoint{3.157970in}{3.425294in}}%
\pgfpathlineto{\pgfqpoint{0.462318in}{3.425294in}}%
\pgfpathclose%
\pgfusepath{fill}%
\end{pgfscope}%
\begin{pgfscope}%
\pgfpathrectangle{\pgfqpoint{0.462318in}{2.320415in}}{\pgfqpoint{2.695652in}{1.104878in}}%
\pgfusepath{clip}%
\pgfsetroundcap%
\pgfsetroundjoin%
\pgfsetlinewidth{0.803000pt}%
\definecolor{currentstroke}{rgb}{1.000000,1.000000,1.000000}%
\pgfsetstrokecolor{currentstroke}%
\pgfsetdash{}{0pt}%
\pgfpathmoveto{\pgfqpoint{0.582607in}{2.320415in}}%
\pgfpathlineto{\pgfqpoint{0.582607in}{3.425294in}}%
\pgfusepath{stroke}%
\end{pgfscope}%
\begin{pgfscope}%
\definecolor{textcolor}{rgb}{0.150000,0.150000,0.150000}%
\pgfsetstrokecolor{textcolor}%
\pgfsetfillcolor{textcolor}%
\pgftext[x=0.582607in,y=2.223193in,,top]{\color{textcolor}\rmfamily\fontsize{10.000000}{12.000000}\selectfont 2012}%
\end{pgfscope}%
\begin{pgfscope}%
\pgfpathrectangle{\pgfqpoint{0.462318in}{2.320415in}}{\pgfqpoint{2.695652in}{1.104878in}}%
\pgfusepath{clip}%
\pgfsetroundcap%
\pgfsetroundjoin%
\pgfsetlinewidth{0.803000pt}%
\definecolor{currentstroke}{rgb}{1.000000,1.000000,1.000000}%
\pgfsetstrokecolor{currentstroke}%
\pgfsetdash{}{0pt}%
\pgfpathmoveto{\pgfqpoint{0.992720in}{2.320415in}}%
\pgfpathlineto{\pgfqpoint{0.992720in}{3.425294in}}%
\pgfusepath{stroke}%
\end{pgfscope}%
\begin{pgfscope}%
\definecolor{textcolor}{rgb}{0.150000,0.150000,0.150000}%
\pgfsetstrokecolor{textcolor}%
\pgfsetfillcolor{textcolor}%
\pgftext[x=0.992720in,y=2.223193in,,top]{\color{textcolor}\rmfamily\fontsize{10.000000}{12.000000}\selectfont 2013}%
\end{pgfscope}%
\begin{pgfscope}%
\pgfpathrectangle{\pgfqpoint{0.462318in}{2.320415in}}{\pgfqpoint{2.695652in}{1.104878in}}%
\pgfusepath{clip}%
\pgfsetroundcap%
\pgfsetroundjoin%
\pgfsetlinewidth{0.803000pt}%
\definecolor{currentstroke}{rgb}{1.000000,1.000000,1.000000}%
\pgfsetstrokecolor{currentstroke}%
\pgfsetdash{}{0pt}%
\pgfpathmoveto{\pgfqpoint{1.401712in}{2.320415in}}%
\pgfpathlineto{\pgfqpoint{1.401712in}{3.425294in}}%
\pgfusepath{stroke}%
\end{pgfscope}%
\begin{pgfscope}%
\definecolor{textcolor}{rgb}{0.150000,0.150000,0.150000}%
\pgfsetstrokecolor{textcolor}%
\pgfsetfillcolor{textcolor}%
\pgftext[x=1.401712in,y=2.223193in,,top]{\color{textcolor}\rmfamily\fontsize{10.000000}{12.000000}\selectfont 2014}%
\end{pgfscope}%
\begin{pgfscope}%
\pgfpathrectangle{\pgfqpoint{0.462318in}{2.320415in}}{\pgfqpoint{2.695652in}{1.104878in}}%
\pgfusepath{clip}%
\pgfsetroundcap%
\pgfsetroundjoin%
\pgfsetlinewidth{0.803000pt}%
\definecolor{currentstroke}{rgb}{1.000000,1.000000,1.000000}%
\pgfsetstrokecolor{currentstroke}%
\pgfsetdash{}{0pt}%
\pgfpathmoveto{\pgfqpoint{1.810705in}{2.320415in}}%
\pgfpathlineto{\pgfqpoint{1.810705in}{3.425294in}}%
\pgfusepath{stroke}%
\end{pgfscope}%
\begin{pgfscope}%
\definecolor{textcolor}{rgb}{0.150000,0.150000,0.150000}%
\pgfsetstrokecolor{textcolor}%
\pgfsetfillcolor{textcolor}%
\pgftext[x=1.810705in,y=2.223193in,,top]{\color{textcolor}\rmfamily\fontsize{10.000000}{12.000000}\selectfont 2015}%
\end{pgfscope}%
\begin{pgfscope}%
\pgfpathrectangle{\pgfqpoint{0.462318in}{2.320415in}}{\pgfqpoint{2.695652in}{1.104878in}}%
\pgfusepath{clip}%
\pgfsetroundcap%
\pgfsetroundjoin%
\pgfsetlinewidth{0.803000pt}%
\definecolor{currentstroke}{rgb}{1.000000,1.000000,1.000000}%
\pgfsetstrokecolor{currentstroke}%
\pgfsetdash{}{0pt}%
\pgfpathmoveto{\pgfqpoint{2.219697in}{2.320415in}}%
\pgfpathlineto{\pgfqpoint{2.219697in}{3.425294in}}%
\pgfusepath{stroke}%
\end{pgfscope}%
\begin{pgfscope}%
\definecolor{textcolor}{rgb}{0.150000,0.150000,0.150000}%
\pgfsetstrokecolor{textcolor}%
\pgfsetfillcolor{textcolor}%
\pgftext[x=2.219697in,y=2.223193in,,top]{\color{textcolor}\rmfamily\fontsize{10.000000}{12.000000}\selectfont 2016}%
\end{pgfscope}%
\begin{pgfscope}%
\pgfpathrectangle{\pgfqpoint{0.462318in}{2.320415in}}{\pgfqpoint{2.695652in}{1.104878in}}%
\pgfusepath{clip}%
\pgfsetroundcap%
\pgfsetroundjoin%
\pgfsetlinewidth{0.803000pt}%
\definecolor{currentstroke}{rgb}{1.000000,1.000000,1.000000}%
\pgfsetstrokecolor{currentstroke}%
\pgfsetdash{}{0pt}%
\pgfpathmoveto{\pgfqpoint{2.629810in}{2.320415in}}%
\pgfpathlineto{\pgfqpoint{2.629810in}{3.425294in}}%
\pgfusepath{stroke}%
\end{pgfscope}%
\begin{pgfscope}%
\definecolor{textcolor}{rgb}{0.150000,0.150000,0.150000}%
\pgfsetstrokecolor{textcolor}%
\pgfsetfillcolor{textcolor}%
\pgftext[x=2.629810in,y=2.223193in,,top]{\color{textcolor}\rmfamily\fontsize{10.000000}{12.000000}\selectfont 2017}%
\end{pgfscope}%
\begin{pgfscope}%
\pgfpathrectangle{\pgfqpoint{0.462318in}{2.320415in}}{\pgfqpoint{2.695652in}{1.104878in}}%
\pgfusepath{clip}%
\pgfsetroundcap%
\pgfsetroundjoin%
\pgfsetlinewidth{0.803000pt}%
\definecolor{currentstroke}{rgb}{1.000000,1.000000,1.000000}%
\pgfsetstrokecolor{currentstroke}%
\pgfsetdash{}{0pt}%
\pgfpathmoveto{\pgfqpoint{3.038802in}{2.320415in}}%
\pgfpathlineto{\pgfqpoint{3.038802in}{3.425294in}}%
\pgfusepath{stroke}%
\end{pgfscope}%
\begin{pgfscope}%
\definecolor{textcolor}{rgb}{0.150000,0.150000,0.150000}%
\pgfsetstrokecolor{textcolor}%
\pgfsetfillcolor{textcolor}%
\pgftext[x=3.038802in,y=2.223193in,,top]{\color{textcolor}\rmfamily\fontsize{10.000000}{12.000000}\selectfont 2018}%
\end{pgfscope}%
\begin{pgfscope}%
\pgfpathrectangle{\pgfqpoint{0.462318in}{2.320415in}}{\pgfqpoint{2.695652in}{1.104878in}}%
\pgfusepath{clip}%
\pgfsetroundcap%
\pgfsetroundjoin%
\pgfsetlinewidth{0.803000pt}%
\definecolor{currentstroke}{rgb}{1.000000,1.000000,1.000000}%
\pgfsetstrokecolor{currentstroke}%
\pgfsetdash{}{0pt}%
\pgfpathmoveto{\pgfqpoint{0.462318in}{2.611879in}}%
\pgfpathlineto{\pgfqpoint{3.157970in}{2.611879in}}%
\pgfusepath{stroke}%
\end{pgfscope}%
\begin{pgfscope}%
\definecolor{textcolor}{rgb}{0.150000,0.150000,0.150000}%
\pgfsetstrokecolor{textcolor}%
\pgfsetfillcolor{textcolor}%
\pgftext[x=0.188365in,y=2.559118in,left,base]{\color{textcolor}\rmfamily\fontsize{10.000000}{12.000000}\selectfont 75}%
\end{pgfscope}%
\begin{pgfscope}%
\pgfpathrectangle{\pgfqpoint{0.462318in}{2.320415in}}{\pgfqpoint{2.695652in}{1.104878in}}%
\pgfusepath{clip}%
\pgfsetroundcap%
\pgfsetroundjoin%
\pgfsetlinewidth{0.803000pt}%
\definecolor{currentstroke}{rgb}{1.000000,1.000000,1.000000}%
\pgfsetstrokecolor{currentstroke}%
\pgfsetdash{}{0pt}%
\pgfpathmoveto{\pgfqpoint{0.462318in}{3.008137in}}%
\pgfpathlineto{\pgfqpoint{3.157970in}{3.008137in}}%
\pgfusepath{stroke}%
\end{pgfscope}%
\begin{pgfscope}%
\definecolor{textcolor}{rgb}{0.150000,0.150000,0.150000}%
\pgfsetstrokecolor{textcolor}%
\pgfsetfillcolor{textcolor}%
\pgftext[x=0.100000in,y=2.955375in,left,base]{\color{textcolor}\rmfamily\fontsize{10.000000}{12.000000}\selectfont 100}%
\end{pgfscope}%
\begin{pgfscope}%
\pgfpathrectangle{\pgfqpoint{0.462318in}{2.320415in}}{\pgfqpoint{2.695652in}{1.104878in}}%
\pgfusepath{clip}%
\pgfsetroundcap%
\pgfsetroundjoin%
\pgfsetlinewidth{0.803000pt}%
\definecolor{currentstroke}{rgb}{1.000000,1.000000,1.000000}%
\pgfsetstrokecolor{currentstroke}%
\pgfsetdash{}{0pt}%
\pgfpathmoveto{\pgfqpoint{0.462318in}{3.404395in}}%
\pgfpathlineto{\pgfqpoint{3.157970in}{3.404395in}}%
\pgfusepath{stroke}%
\end{pgfscope}%
\begin{pgfscope}%
\definecolor{textcolor}{rgb}{0.150000,0.150000,0.150000}%
\pgfsetstrokecolor{textcolor}%
\pgfsetfillcolor{textcolor}%
\pgftext[x=0.100000in,y=3.351633in,left,base]{\color{textcolor}\rmfamily\fontsize{10.000000}{12.000000}\selectfont 125}%
\end{pgfscope}%
\begin{pgfscope}%
\pgfpathrectangle{\pgfqpoint{0.462318in}{2.320415in}}{\pgfqpoint{2.695652in}{1.104878in}}%
\pgfusepath{clip}%
\pgfsetroundcap%
\pgfsetroundjoin%
\pgfsetlinewidth{1.505625pt}%
\definecolor{currentstroke}{rgb}{0.890196,0.466667,0.760784}%
\pgfsetstrokecolor{currentstroke}%
\pgfsetdash{}{0pt}%
\pgfpathmoveto{\pgfqpoint{0.584848in}{2.409312in}}%
\pgfpathlineto{\pgfqpoint{0.585968in}{2.414384in}}%
\pgfpathlineto{\pgfqpoint{0.588209in}{2.399168in}}%
\pgfpathlineto{\pgfqpoint{0.591571in}{2.401862in}}%
\pgfpathlineto{\pgfqpoint{0.592692in}{2.427381in}}%
\pgfpathlineto{\pgfqpoint{0.594933in}{2.443390in}}%
\pgfpathlineto{\pgfqpoint{0.596053in}{2.428015in}}%
\pgfpathlineto{\pgfqpoint{0.600535in}{2.440696in}}%
\pgfpathlineto{\pgfqpoint{0.601656in}{2.448145in}}%
\pgfpathlineto{\pgfqpoint{0.603897in}{2.436099in}}%
\pgfpathlineto{\pgfqpoint{0.607258in}{2.438318in}}%
\pgfpathlineto{\pgfqpoint{0.608379in}{2.450523in}}%
\pgfpathlineto{\pgfqpoint{0.609499in}{2.448779in}}%
\pgfpathlineto{\pgfqpoint{0.610620in}{2.445609in}}%
\pgfpathlineto{\pgfqpoint{0.611740in}{2.448304in}}%
\pgfpathlineto{\pgfqpoint{0.615102in}{2.448145in}}%
\pgfpathlineto{\pgfqpoint{0.616223in}{2.457972in}}%
\pgfpathlineto{\pgfqpoint{0.617343in}{2.482699in}}%
\pgfpathlineto{\pgfqpoint{0.618464in}{2.480004in}}%
\pgfpathlineto{\pgfqpoint{0.619584in}{2.493636in}}%
\pgfpathlineto{\pgfqpoint{0.622946in}{2.487296in}}%
\pgfpathlineto{\pgfqpoint{0.624066in}{2.483491in}}%
\pgfpathlineto{\pgfqpoint{0.626307in}{2.529774in}}%
\pgfpathlineto{\pgfqpoint{0.627428in}{2.525970in}}%
\pgfpathlineto{\pgfqpoint{0.630789in}{2.544198in}}%
\pgfpathlineto{\pgfqpoint{0.631910in}{2.541028in}}%
\pgfpathlineto{\pgfqpoint{0.633031in}{2.520423in}}%
\pgfpathlineto{\pgfqpoint{0.635272in}{2.535956in}}%
\pgfpathlineto{\pgfqpoint{0.639754in}{2.538492in}}%
\pgfpathlineto{\pgfqpoint{0.640874in}{2.538334in}}%
\pgfpathlineto{\pgfqpoint{0.641995in}{2.532152in}}%
\pgfpathlineto{\pgfqpoint{0.643115in}{2.538492in}}%
\pgfpathlineto{\pgfqpoint{0.647597in}{2.531835in}}%
\pgfpathlineto{\pgfqpoint{0.650959in}{2.546100in}}%
\pgfpathlineto{\pgfqpoint{0.654321in}{2.529457in}}%
\pgfpathlineto{\pgfqpoint{0.655441in}{2.504255in}}%
\pgfpathlineto{\pgfqpoint{0.657682in}{2.532786in}}%
\pgfpathlineto{\pgfqpoint{0.658803in}{2.532944in}}%
\pgfpathlineto{\pgfqpoint{0.662164in}{2.538809in}}%
\pgfpathlineto{\pgfqpoint{0.663285in}{2.573046in}}%
\pgfpathlineto{\pgfqpoint{0.664405in}{2.576216in}}%
\pgfpathlineto{\pgfqpoint{0.665526in}{2.577325in}}%
\pgfpathlineto{\pgfqpoint{0.666646in}{2.558622in}}%
\pgfpathlineto{\pgfqpoint{0.670008in}{2.547210in}}%
\pgfpathlineto{\pgfqpoint{0.671128in}{2.529140in}}%
\pgfpathlineto{\pgfqpoint{0.674490in}{2.509803in}}%
\pgfpathlineto{\pgfqpoint{0.677852in}{2.532310in}}%
\pgfpathlineto{\pgfqpoint{0.678972in}{2.527080in}}%
\pgfpathlineto{\pgfqpoint{0.680093in}{2.505999in}}%
\pgfpathlineto{\pgfqpoint{0.682334in}{2.524861in}}%
\pgfpathlineto{\pgfqpoint{0.685695in}{2.522008in}}%
\pgfpathlineto{\pgfqpoint{0.687936in}{2.511388in}}%
\pgfpathlineto{\pgfqpoint{0.693539in}{2.488881in}}%
\pgfpathlineto{\pgfqpoint{0.694659in}{2.465581in}}%
\pgfpathlineto{\pgfqpoint{0.696901in}{2.501402in}}%
\pgfpathlineto{\pgfqpoint{0.698021in}{2.483174in}}%
\pgfpathlineto{\pgfqpoint{0.701383in}{2.484284in}}%
\pgfpathlineto{\pgfqpoint{0.702503in}{2.502353in}}%
\pgfpathlineto{\pgfqpoint{0.703624in}{2.501719in}}%
\pgfpathlineto{\pgfqpoint{0.704744in}{2.491100in}}%
\pgfpathlineto{\pgfqpoint{0.705865in}{2.499025in}}%
\pgfpathlineto{\pgfqpoint{0.709226in}{2.482540in}}%
\pgfpathlineto{\pgfqpoint{0.710347in}{2.483808in}}%
\pgfpathlineto{\pgfqpoint{0.711467in}{2.483333in}}%
\pgfpathlineto{\pgfqpoint{0.713708in}{2.512339in}}%
\pgfpathlineto{\pgfqpoint{0.717070in}{2.507584in}}%
\pgfpathlineto{\pgfqpoint{0.718191in}{2.504889in}}%
\pgfpathlineto{\pgfqpoint{0.719311in}{2.506474in}}%
\pgfpathlineto{\pgfqpoint{0.720432in}{2.497915in}}%
\pgfpathlineto{\pgfqpoint{0.721552in}{2.478895in}}%
\pgfpathlineto{\pgfqpoint{0.724914in}{2.471287in}}%
\pgfpathlineto{\pgfqpoint{0.726034in}{2.470177in}}%
\pgfpathlineto{\pgfqpoint{0.727155in}{2.446085in}}%
\pgfpathlineto{\pgfqpoint{0.728275in}{2.449889in}}%
\pgfpathlineto{\pgfqpoint{0.729396in}{2.448304in}}%
\pgfpathlineto{\pgfqpoint{0.732757in}{2.431978in}}%
\pgfpathlineto{\pgfqpoint{0.733878in}{2.434990in}}%
\pgfpathlineto{\pgfqpoint{0.737240in}{2.390767in}}%
\pgfpathlineto{\pgfqpoint{0.741722in}{2.411531in}}%
\pgfpathlineto{\pgfqpoint{0.742842in}{2.413116in}}%
\pgfpathlineto{\pgfqpoint{0.745083in}{2.399168in}}%
\pgfpathlineto{\pgfqpoint{0.749565in}{2.426906in}}%
\pgfpathlineto{\pgfqpoint{0.750686in}{2.408836in}}%
\pgfpathlineto{\pgfqpoint{0.751806in}{2.413750in}}%
\pgfpathlineto{\pgfqpoint{0.752927in}{2.385854in}}%
\pgfpathlineto{\pgfqpoint{0.756288in}{2.378245in}}%
\pgfpathlineto{\pgfqpoint{0.757409in}{2.370637in}}%
\pgfpathlineto{\pgfqpoint{0.759650in}{2.431027in}}%
\pgfpathlineto{\pgfqpoint{0.760771in}{2.432453in}}%
\pgfpathlineto{\pgfqpoint{0.765253in}{2.417079in}}%
\pgfpathlineto{\pgfqpoint{0.766373in}{2.406142in}}%
\pgfpathlineto{\pgfqpoint{0.768614in}{2.418822in}}%
\pgfpathlineto{\pgfqpoint{0.771976in}{2.425638in}}%
\pgfpathlineto{\pgfqpoint{0.773096in}{2.444500in}}%
\pgfpathlineto{\pgfqpoint{0.774217in}{2.437843in}}%
\pgfpathlineto{\pgfqpoint{0.775337in}{2.423260in}}%
\pgfpathlineto{\pgfqpoint{0.776458in}{2.428332in}}%
\pgfpathlineto{\pgfqpoint{0.779820in}{2.408995in}}%
\pgfpathlineto{\pgfqpoint{0.780940in}{2.407410in}}%
\pgfpathlineto{\pgfqpoint{0.782061in}{2.414543in}}%
\pgfpathlineto{\pgfqpoint{0.783181in}{2.393620in}}%
\pgfpathlineto{\pgfqpoint{0.784302in}{2.432770in}}%
\pgfpathlineto{\pgfqpoint{0.787663in}{2.426272in}}%
\pgfpathlineto{\pgfqpoint{0.788784in}{2.435782in}}%
\pgfpathlineto{\pgfqpoint{0.791025in}{2.430868in}}%
\pgfpathlineto{\pgfqpoint{0.792145in}{2.413592in}}%
\pgfpathlineto{\pgfqpoint{0.795507in}{2.416762in}}%
\pgfpathlineto{\pgfqpoint{0.796627in}{2.414384in}}%
\pgfpathlineto{\pgfqpoint{0.797748in}{2.392352in}}%
\pgfpathlineto{\pgfqpoint{0.798869in}{2.383476in}}%
\pgfpathlineto{\pgfqpoint{0.799989in}{2.406934in}}%
\pgfpathlineto{\pgfqpoint{0.803351in}{2.401862in}}%
\pgfpathlineto{\pgfqpoint{0.804471in}{2.406934in}}%
\pgfpathlineto{\pgfqpoint{0.806712in}{2.436733in}}%
\pgfpathlineto{\pgfqpoint{0.807833in}{2.415494in}}%
\pgfpathlineto{\pgfqpoint{0.811194in}{2.402655in}}%
\pgfpathlineto{\pgfqpoint{0.812315in}{2.384744in}}%
\pgfpathlineto{\pgfqpoint{0.813435in}{2.393779in}}%
\pgfpathlineto{\pgfqpoint{0.814556in}{2.398058in}}%
\pgfpathlineto{\pgfqpoint{0.815676in}{2.416128in}}%
\pgfpathlineto{\pgfqpoint{0.819038in}{2.425321in}}%
\pgfpathlineto{\pgfqpoint{0.820159in}{2.418188in}}%
\pgfpathlineto{\pgfqpoint{0.821279in}{2.422943in}}%
\pgfpathlineto{\pgfqpoint{0.822400in}{2.418188in}}%
\pgfpathlineto{\pgfqpoint{0.823520in}{2.449889in}}%
\pgfpathlineto{\pgfqpoint{0.826882in}{2.447828in}}%
\pgfpathlineto{\pgfqpoint{0.828002in}{2.467483in}}%
\pgfpathlineto{\pgfqpoint{0.830243in}{2.452900in}}%
\pgfpathlineto{\pgfqpoint{0.831364in}{2.464313in}}%
\pgfpathlineto{\pgfqpoint{0.834725in}{2.458765in}}%
\pgfpathlineto{\pgfqpoint{0.835846in}{2.462886in}}%
\pgfpathlineto{\pgfqpoint{0.838087in}{2.483333in}}%
\pgfpathlineto{\pgfqpoint{0.839208in}{2.504889in}}%
\pgfpathlineto{\pgfqpoint{0.842569in}{2.501085in}}%
\pgfpathlineto{\pgfqpoint{0.843690in}{2.490149in}}%
\pgfpathlineto{\pgfqpoint{0.844810in}{2.495379in}}%
\pgfpathlineto{\pgfqpoint{0.845931in}{2.489198in}}%
\pgfpathlineto{\pgfqpoint{0.847051in}{2.501085in}}%
\pgfpathlineto{\pgfqpoint{0.851533in}{2.508535in}}%
\pgfpathlineto{\pgfqpoint{0.852654in}{2.502195in}}%
\pgfpathlineto{\pgfqpoint{0.853774in}{2.487296in}}%
\pgfpathlineto{\pgfqpoint{0.854895in}{2.497915in}}%
\pgfpathlineto{\pgfqpoint{0.860498in}{2.473506in}}%
\pgfpathlineto{\pgfqpoint{0.861618in}{2.491892in}}%
\pgfpathlineto{\pgfqpoint{0.862739in}{2.492051in}}%
\pgfpathlineto{\pgfqpoint{0.866100in}{2.481272in}}%
\pgfpathlineto{\pgfqpoint{0.868341in}{2.484284in}}%
\pgfpathlineto{\pgfqpoint{0.870582in}{2.532944in}}%
\pgfpathlineto{\pgfqpoint{0.873944in}{2.528982in}}%
\pgfpathlineto{\pgfqpoint{0.875064in}{2.519947in}}%
\pgfpathlineto{\pgfqpoint{0.876185in}{2.523117in}}%
\pgfpathlineto{\pgfqpoint{0.877305in}{2.512339in}}%
\pgfpathlineto{\pgfqpoint{0.878426in}{2.510120in}}%
\pgfpathlineto{\pgfqpoint{0.881788in}{2.500293in}}%
\pgfpathlineto{\pgfqpoint{0.882908in}{2.483333in}}%
\pgfpathlineto{\pgfqpoint{0.885149in}{2.477785in}}%
\pgfpathlineto{\pgfqpoint{0.886270in}{2.476993in}}%
\pgfpathlineto{\pgfqpoint{0.890752in}{2.478261in}}%
\pgfpathlineto{\pgfqpoint{0.891872in}{2.480638in}}%
\pgfpathlineto{\pgfqpoint{0.897475in}{2.477944in}}%
\pgfpathlineto{\pgfqpoint{0.899716in}{2.446402in}}%
\pgfpathlineto{\pgfqpoint{0.900836in}{2.446719in}}%
\pgfpathlineto{\pgfqpoint{0.901957in}{2.445609in}}%
\pgfpathlineto{\pgfqpoint{0.905319in}{2.448145in}}%
\pgfpathlineto{\pgfqpoint{0.907560in}{2.480321in}}%
\pgfpathlineto{\pgfqpoint{0.908680in}{2.489673in}}%
\pgfpathlineto{\pgfqpoint{0.909801in}{2.472872in}}%
\pgfpathlineto{\pgfqpoint{0.913162in}{2.470811in}}%
\pgfpathlineto{\pgfqpoint{0.914283in}{2.460509in}}%
\pgfpathlineto{\pgfqpoint{0.915403in}{2.471762in}}%
\pgfpathlineto{\pgfqpoint{0.916524in}{2.463520in}}%
\pgfpathlineto{\pgfqpoint{0.917644in}{2.475725in}}%
\pgfpathlineto{\pgfqpoint{0.923247in}{2.475249in}}%
\pgfpathlineto{\pgfqpoint{0.924368in}{2.487296in}}%
\pgfpathlineto{\pgfqpoint{0.925488in}{2.473981in}}%
\pgfpathlineto{\pgfqpoint{0.928850in}{2.471762in}}%
\pgfpathlineto{\pgfqpoint{0.929970in}{2.499500in}}%
\pgfpathlineto{\pgfqpoint{0.932211in}{2.448304in}}%
\pgfpathlineto{\pgfqpoint{0.933332in}{2.444024in}}%
\pgfpathlineto{\pgfqpoint{0.936693in}{2.458923in}}%
\pgfpathlineto{\pgfqpoint{0.937814in}{2.459557in}}%
\pgfpathlineto{\pgfqpoint{0.938934in}{2.434990in}}%
\pgfpathlineto{\pgfqpoint{0.940055in}{2.437526in}}%
\pgfpathlineto{\pgfqpoint{0.946778in}{2.471287in}}%
\pgfpathlineto{\pgfqpoint{0.949019in}{2.488722in}}%
\pgfpathlineto{\pgfqpoint{0.952381in}{2.489673in}}%
\pgfpathlineto{\pgfqpoint{0.953501in}{2.491575in}}%
\pgfpathlineto{\pgfqpoint{0.954622in}{2.504731in}}%
\pgfpathlineto{\pgfqpoint{0.955742in}{2.504255in}}%
\pgfpathlineto{\pgfqpoint{0.956863in}{2.509010in}}%
\pgfpathlineto{\pgfqpoint{0.960224in}{2.504731in}}%
\pgfpathlineto{\pgfqpoint{0.961345in}{2.509327in}}%
\pgfpathlineto{\pgfqpoint{0.962465in}{2.510596in}}%
\pgfpathlineto{\pgfqpoint{0.963586in}{2.519313in}}%
\pgfpathlineto{\pgfqpoint{0.964707in}{2.520898in}}%
\pgfpathlineto{\pgfqpoint{0.968068in}{2.521215in}}%
\pgfpathlineto{\pgfqpoint{0.969189in}{2.523593in}}%
\pgfpathlineto{\pgfqpoint{0.970309in}{2.521057in}}%
\pgfpathlineto{\pgfqpoint{0.972550in}{2.507267in}}%
\pgfpathlineto{\pgfqpoint{0.975912in}{2.507425in}}%
\pgfpathlineto{\pgfqpoint{0.977032in}{2.537700in}}%
\pgfpathlineto{\pgfqpoint{0.978153in}{2.548636in}}%
\pgfpathlineto{\pgfqpoint{0.979273in}{2.550538in}}%
\pgfpathlineto{\pgfqpoint{0.980394in}{2.541979in}}%
\pgfpathlineto{\pgfqpoint{0.983756in}{2.537858in}}%
\pgfpathlineto{\pgfqpoint{0.987117in}{2.535481in}}%
\pgfpathlineto{\pgfqpoint{0.988238in}{2.518521in}}%
\pgfpathlineto{\pgfqpoint{0.991599in}{2.534688in}}%
\pgfpathlineto{\pgfqpoint{0.993840in}{2.561792in}}%
\pgfpathlineto{\pgfqpoint{0.994961in}{2.565913in}}%
\pgfpathlineto{\pgfqpoint{0.996081in}{2.574948in}}%
\pgfpathlineto{\pgfqpoint{0.999443in}{2.569400in}}%
\pgfpathlineto{\pgfqpoint{1.000563in}{2.555610in}}%
\pgfpathlineto{\pgfqpoint{1.001684in}{2.569242in}}%
\pgfpathlineto{\pgfqpoint{1.003925in}{2.577642in}}%
\pgfpathlineto{\pgfqpoint{1.008407in}{2.588262in}}%
\pgfpathlineto{\pgfqpoint{1.009528in}{2.583031in}}%
\pgfpathlineto{\pgfqpoint{1.011769in}{2.601735in}}%
\pgfpathlineto{\pgfqpoint{1.016251in}{2.608709in}}%
\pgfpathlineto{\pgfqpoint{1.018492in}{2.628363in}}%
\pgfpathlineto{\pgfqpoint{1.019612in}{2.640251in}}%
\pgfpathlineto{\pgfqpoint{1.022974in}{2.640410in}}%
\pgfpathlineto{\pgfqpoint{1.024094in}{2.639776in}}%
\pgfpathlineto{\pgfqpoint{1.025215in}{2.630899in}}%
\pgfpathlineto{\pgfqpoint{1.026336in}{2.610136in}}%
\pgfpathlineto{\pgfqpoint{1.027456in}{2.640885in}}%
\pgfpathlineto{\pgfqpoint{1.030818in}{2.636764in}}%
\pgfpathlineto{\pgfqpoint{1.031938in}{2.632960in}}%
\pgfpathlineto{\pgfqpoint{1.033059in}{2.634545in}}%
\pgfpathlineto{\pgfqpoint{1.034179in}{2.642312in}}%
\pgfpathlineto{\pgfqpoint{1.035300in}{2.644214in}}%
\pgfpathlineto{\pgfqpoint{1.038661in}{2.637715in}}%
\pgfpathlineto{\pgfqpoint{1.039782in}{2.642787in}}%
\pgfpathlineto{\pgfqpoint{1.040902in}{2.643580in}}%
\pgfpathlineto{\pgfqpoint{1.042023in}{2.646116in}}%
\pgfpathlineto{\pgfqpoint{1.043143in}{2.661015in}}%
\pgfpathlineto{\pgfqpoint{1.047626in}{2.664185in}}%
\pgfpathlineto{\pgfqpoint{1.049867in}{2.642312in}}%
\pgfpathlineto{\pgfqpoint{1.050987in}{2.657052in}}%
\pgfpathlineto{\pgfqpoint{1.054349in}{2.628046in}}%
\pgfpathlineto{\pgfqpoint{1.055469in}{2.639459in}}%
\pgfpathlineto{\pgfqpoint{1.056590in}{2.658162in}}%
\pgfpathlineto{\pgfqpoint{1.057710in}{2.657845in}}%
\pgfpathlineto{\pgfqpoint{1.058831in}{2.652139in}}%
\pgfpathlineto{\pgfqpoint{1.062192in}{2.638508in}}%
\pgfpathlineto{\pgfqpoint{1.063313in}{2.664185in}}%
\pgfpathlineto{\pgfqpoint{1.064433in}{2.664978in}}%
\pgfpathlineto{\pgfqpoint{1.066675in}{2.678609in}}%
\pgfpathlineto{\pgfqpoint{1.071157in}{2.693191in}}%
\pgfpathlineto{\pgfqpoint{1.072277in}{2.692399in}}%
\pgfpathlineto{\pgfqpoint{1.073398in}{2.697312in}}%
\pgfpathlineto{\pgfqpoint{1.077880in}{2.688119in}}%
\pgfpathlineto{\pgfqpoint{1.080121in}{2.697312in}}%
\pgfpathlineto{\pgfqpoint{1.081241in}{2.683522in}}%
\pgfpathlineto{\pgfqpoint{1.082362in}{2.699373in}}%
\pgfpathlineto{\pgfqpoint{1.086844in}{2.686534in}}%
\pgfpathlineto{\pgfqpoint{1.087965in}{2.685900in}}%
\pgfpathlineto{\pgfqpoint{1.089085in}{2.697154in}}%
\pgfpathlineto{\pgfqpoint{1.093567in}{2.690180in}}%
\pgfpathlineto{\pgfqpoint{1.094688in}{2.691131in}}%
\pgfpathlineto{\pgfqpoint{1.095808in}{2.693984in}}%
\pgfpathlineto{\pgfqpoint{1.096929in}{2.693508in}}%
\pgfpathlineto{\pgfqpoint{1.098049in}{2.686376in}}%
\pgfpathlineto{\pgfqpoint{1.101411in}{2.700958in}}%
\pgfpathlineto{\pgfqpoint{1.104772in}{2.729647in}}%
\pgfpathlineto{\pgfqpoint{1.105893in}{2.727903in}}%
\pgfpathlineto{\pgfqpoint{1.109255in}{2.698263in}}%
\pgfpathlineto{\pgfqpoint{1.110375in}{2.711895in}}%
\pgfpathlineto{\pgfqpoint{1.112616in}{2.671476in}}%
\pgfpathlineto{\pgfqpoint{1.113737in}{2.694301in}}%
\pgfpathlineto{\pgfqpoint{1.117098in}{2.699848in}}%
\pgfpathlineto{\pgfqpoint{1.119339in}{2.677658in}}%
\pgfpathlineto{\pgfqpoint{1.120460in}{2.679084in}}%
\pgfpathlineto{\pgfqpoint{1.121580in}{2.666087in}}%
\pgfpathlineto{\pgfqpoint{1.124942in}{2.672427in}}%
\pgfpathlineto{\pgfqpoint{1.127183in}{2.664661in}}%
\pgfpathlineto{\pgfqpoint{1.128304in}{2.674171in}}%
\pgfpathlineto{\pgfqpoint{1.129424in}{2.692716in}}%
\pgfpathlineto{\pgfqpoint{1.132786in}{2.697946in}}%
\pgfpathlineto{\pgfqpoint{1.136147in}{2.715223in}}%
\pgfpathlineto{\pgfqpoint{1.137268in}{2.721880in}}%
\pgfpathlineto{\pgfqpoint{1.140629in}{2.717918in}}%
\pgfpathlineto{\pgfqpoint{1.141750in}{2.731232in}}%
\pgfpathlineto{\pgfqpoint{1.142870in}{2.736938in}}%
\pgfpathlineto{\pgfqpoint{1.143991in}{2.728062in}}%
\pgfpathlineto{\pgfqpoint{1.145111in}{2.758019in}}%
\pgfpathlineto{\pgfqpoint{1.148473in}{2.756434in}}%
\pgfpathlineto{\pgfqpoint{1.149594in}{2.760714in}}%
\pgfpathlineto{\pgfqpoint{1.151835in}{2.732500in}}%
\pgfpathlineto{\pgfqpoint{1.152955in}{2.726318in}}%
\pgfpathlineto{\pgfqpoint{1.157437in}{2.739157in}}%
\pgfpathlineto{\pgfqpoint{1.158558in}{2.727428in}}%
\pgfpathlineto{\pgfqpoint{1.159678in}{2.738999in}}%
\pgfpathlineto{\pgfqpoint{1.160799in}{2.724416in}}%
\pgfpathlineto{\pgfqpoint{1.164160in}{2.729013in}}%
\pgfpathlineto{\pgfqpoint{1.167522in}{2.694935in}}%
\pgfpathlineto{\pgfqpoint{1.168642in}{2.718869in}}%
\pgfpathlineto{\pgfqpoint{1.172004in}{2.713955in}}%
\pgfpathlineto{\pgfqpoint{1.173125in}{2.707615in}}%
\pgfpathlineto{\pgfqpoint{1.174245in}{2.693984in}}%
\pgfpathlineto{\pgfqpoint{1.175366in}{2.716016in}}%
\pgfpathlineto{\pgfqpoint{1.176486in}{2.712370in}}%
\pgfpathlineto{\pgfqpoint{1.179848in}{2.725526in}}%
\pgfpathlineto{\pgfqpoint{1.180968in}{2.741852in}}%
\pgfpathlineto{\pgfqpoint{1.183209in}{2.689387in}}%
\pgfpathlineto{\pgfqpoint{1.184330in}{2.687168in}}%
\pgfpathlineto{\pgfqpoint{1.187691in}{2.677658in}}%
\pgfpathlineto{\pgfqpoint{1.188812in}{2.681620in}}%
\pgfpathlineto{\pgfqpoint{1.189933in}{2.698105in}}%
\pgfpathlineto{\pgfqpoint{1.191053in}{2.705396in}}%
\pgfpathlineto{\pgfqpoint{1.192174in}{2.697471in}}%
\pgfpathlineto{\pgfqpoint{1.195535in}{2.722197in}}%
\pgfpathlineto{\pgfqpoint{1.196656in}{2.709358in}}%
\pgfpathlineto{\pgfqpoint{1.200017in}{2.746765in}}%
\pgfpathlineto{\pgfqpoint{1.203379in}{2.752947in}}%
\pgfpathlineto{\pgfqpoint{1.204499in}{2.767212in}}%
\pgfpathlineto{\pgfqpoint{1.205620in}{2.764042in}}%
\pgfpathlineto{\pgfqpoint{1.206740in}{2.790037in}}%
\pgfpathlineto{\pgfqpoint{1.211223in}{2.797645in}}%
\pgfpathlineto{\pgfqpoint{1.212343in}{2.794158in}}%
\pgfpathlineto{\pgfqpoint{1.215705in}{2.828394in}}%
\pgfpathlineto{\pgfqpoint{1.219066in}{2.823322in}}%
\pgfpathlineto{\pgfqpoint{1.220187in}{2.864533in}}%
\pgfpathlineto{\pgfqpoint{1.222428in}{2.860095in}}%
\pgfpathlineto{\pgfqpoint{1.223548in}{2.862473in}}%
\pgfpathlineto{\pgfqpoint{1.226910in}{2.863899in}}%
\pgfpathlineto{\pgfqpoint{1.228030in}{2.870715in}}%
\pgfpathlineto{\pgfqpoint{1.229151in}{2.870715in}}%
\pgfpathlineto{\pgfqpoint{1.230271in}{2.892747in}}%
\pgfpathlineto{\pgfqpoint{1.231392in}{2.900830in}}%
\pgfpathlineto{\pgfqpoint{1.234754in}{2.885456in}}%
\pgfpathlineto{\pgfqpoint{1.235874in}{2.865643in}}%
\pgfpathlineto{\pgfqpoint{1.236995in}{2.877055in}}%
\pgfpathlineto{\pgfqpoint{1.238115in}{2.880225in}}%
\pgfpathlineto{\pgfqpoint{1.239236in}{2.871349in}}%
\pgfpathlineto{\pgfqpoint{1.242597in}{2.870239in}}%
\pgfpathlineto{\pgfqpoint{1.243718in}{2.887516in}}%
\pgfpathlineto{\pgfqpoint{1.244838in}{2.871032in}}%
\pgfpathlineto{\pgfqpoint{1.245959in}{2.842501in}}%
\pgfpathlineto{\pgfqpoint{1.247079in}{2.843769in}}%
\pgfpathlineto{\pgfqpoint{1.251561in}{2.834259in}}%
\pgfpathlineto{\pgfqpoint{1.252682in}{2.824749in}}%
\pgfpathlineto{\pgfqpoint{1.253803in}{2.842026in}}%
\pgfpathlineto{\pgfqpoint{1.258285in}{2.832991in}}%
\pgfpathlineto{\pgfqpoint{1.259405in}{2.800498in}}%
\pgfpathlineto{\pgfqpoint{1.260526in}{2.800973in}}%
\pgfpathlineto{\pgfqpoint{1.261646in}{2.807630in}}%
\pgfpathlineto{\pgfqpoint{1.262767in}{2.802558in}}%
\pgfpathlineto{\pgfqpoint{1.269490in}{2.852011in}}%
\pgfpathlineto{\pgfqpoint{1.270610in}{2.845830in}}%
\pgfpathlineto{\pgfqpoint{1.273972in}{2.862948in}}%
\pgfpathlineto{\pgfqpoint{1.276213in}{2.909231in}}%
\pgfpathlineto{\pgfqpoint{1.277334in}{2.909231in}}%
\pgfpathlineto{\pgfqpoint{1.278454in}{2.916839in}}%
\pgfpathlineto{\pgfqpoint{1.281816in}{2.934909in}}%
\pgfpathlineto{\pgfqpoint{1.285177in}{2.966609in}}%
\pgfpathlineto{\pgfqpoint{1.286298in}{2.933323in}}%
\pgfpathlineto{\pgfqpoint{1.289659in}{2.931104in}}%
\pgfpathlineto{\pgfqpoint{1.290780in}{2.939030in}}%
\pgfpathlineto{\pgfqpoint{1.291900in}{2.928885in}}%
\pgfpathlineto{\pgfqpoint{1.293021in}{2.934433in}}%
\pgfpathlineto{\pgfqpoint{1.294142in}{2.930312in}}%
\pgfpathlineto{\pgfqpoint{1.298624in}{2.902891in}}%
\pgfpathlineto{\pgfqpoint{1.300865in}{2.852170in}}%
\pgfpathlineto{\pgfqpoint{1.301985in}{2.860095in}}%
\pgfpathlineto{\pgfqpoint{1.305347in}{2.856766in}}%
\pgfpathlineto{\pgfqpoint{1.306467in}{2.839331in}}%
\pgfpathlineto{\pgfqpoint{1.307588in}{2.840441in}}%
\pgfpathlineto{\pgfqpoint{1.308708in}{2.882761in}}%
\pgfpathlineto{\pgfqpoint{1.309829in}{2.897819in}}%
\pgfpathlineto{\pgfqpoint{1.313190in}{2.896392in}}%
\pgfpathlineto{\pgfqpoint{1.314311in}{2.881176in}}%
\pgfpathlineto{\pgfqpoint{1.315432in}{2.889894in}}%
\pgfpathlineto{\pgfqpoint{1.316552in}{2.911767in}}%
\pgfpathlineto{\pgfqpoint{1.317673in}{2.907963in}}%
\pgfpathlineto{\pgfqpoint{1.321034in}{2.906219in}}%
\pgfpathlineto{\pgfqpoint{1.322155in}{2.885773in}}%
\pgfpathlineto{\pgfqpoint{1.323275in}{2.889260in}}%
\pgfpathlineto{\pgfqpoint{1.325516in}{2.904951in}}%
\pgfpathlineto{\pgfqpoint{1.328878in}{2.883870in}}%
\pgfpathlineto{\pgfqpoint{1.329998in}{2.889735in}}%
\pgfpathlineto{\pgfqpoint{1.331119in}{2.883236in}}%
\pgfpathlineto{\pgfqpoint{1.332239in}{2.887358in}}%
\pgfpathlineto{\pgfqpoint{1.333360in}{2.904793in}}%
\pgfpathlineto{\pgfqpoint{1.336722in}{2.911292in}}%
\pgfpathlineto{\pgfqpoint{1.337842in}{2.907170in}}%
\pgfpathlineto{\pgfqpoint{1.338963in}{2.920802in}}%
\pgfpathlineto{\pgfqpoint{1.340083in}{2.901464in}}%
\pgfpathlineto{\pgfqpoint{1.341204in}{2.919058in}}%
\pgfpathlineto{\pgfqpoint{1.344565in}{2.912560in}}%
\pgfpathlineto{\pgfqpoint{1.345686in}{2.904000in}}%
\pgfpathlineto{\pgfqpoint{1.346806in}{2.912243in}}%
\pgfpathlineto{\pgfqpoint{1.347927in}{2.929361in}}%
\pgfpathlineto{\pgfqpoint{1.349047in}{2.927934in}}%
\pgfpathlineto{\pgfqpoint{1.352409in}{2.935384in}}%
\pgfpathlineto{\pgfqpoint{1.353529in}{2.934909in}}%
\pgfpathlineto{\pgfqpoint{1.354650in}{2.931263in}}%
\pgfpathlineto{\pgfqpoint{1.355771in}{2.944260in}}%
\pgfpathlineto{\pgfqpoint{1.356891in}{2.950283in}}%
\pgfpathlineto{\pgfqpoint{1.360253in}{2.951710in}}%
\pgfpathlineto{\pgfqpoint{1.362494in}{2.968353in}}%
\pgfpathlineto{\pgfqpoint{1.364735in}{2.959318in}}%
\pgfpathlineto{\pgfqpoint{1.368096in}{2.951710in}}%
\pgfpathlineto{\pgfqpoint{1.370337in}{2.932689in}}%
\pgfpathlineto{\pgfqpoint{1.371458in}{2.934274in}}%
\pgfpathlineto{\pgfqpoint{1.372578in}{2.962964in}}%
\pgfpathlineto{\pgfqpoint{1.375940in}{2.963756in}}%
\pgfpathlineto{\pgfqpoint{1.377061in}{2.961062in}}%
\pgfpathlineto{\pgfqpoint{1.378181in}{2.928885in}}%
\pgfpathlineto{\pgfqpoint{1.380422in}{2.910658in}}%
\pgfpathlineto{\pgfqpoint{1.383784in}{2.927142in}}%
\pgfpathlineto{\pgfqpoint{1.384904in}{2.914462in}}%
\pgfpathlineto{\pgfqpoint{1.386025in}{2.944736in}}%
\pgfpathlineto{\pgfqpoint{1.387145in}{2.940456in}}%
\pgfpathlineto{\pgfqpoint{1.388266in}{2.956623in}}%
\pgfpathlineto{\pgfqpoint{1.391627in}{2.958842in}}%
\pgfpathlineto{\pgfqpoint{1.394989in}{2.984678in}}%
\pgfpathlineto{\pgfqpoint{1.396109in}{2.986264in}}%
\pgfpathlineto{\pgfqpoint{1.399471in}{2.985313in}}%
\pgfpathlineto{\pgfqpoint{1.400592in}{3.000053in}}%
\pgfpathlineto{\pgfqpoint{1.402833in}{2.981984in}}%
\pgfpathlineto{\pgfqpoint{1.403953in}{2.987690in}}%
\pgfpathlineto{\pgfqpoint{1.407315in}{2.986105in}}%
\pgfpathlineto{\pgfqpoint{1.408435in}{2.996091in}}%
\pgfpathlineto{\pgfqpoint{1.409556in}{2.997993in}}%
\pgfpathlineto{\pgfqpoint{1.410676in}{2.998785in}}%
\pgfpathlineto{\pgfqpoint{1.411797in}{3.000529in}}%
\pgfpathlineto{\pgfqpoint{1.416279in}{2.983886in}}%
\pgfpathlineto{\pgfqpoint{1.417400in}{3.003857in}}%
\pgfpathlineto{\pgfqpoint{1.418520in}{3.005918in}}%
\pgfpathlineto{\pgfqpoint{1.419641in}{3.005759in}}%
\pgfpathlineto{\pgfqpoint{1.424123in}{3.016538in}}%
\pgfpathlineto{\pgfqpoint{1.425243in}{3.032229in}}%
\pgfpathlineto{\pgfqpoint{1.426364in}{3.014477in}}%
\pgfpathlineto{\pgfqpoint{1.427484in}{2.972315in}}%
\pgfpathlineto{\pgfqpoint{1.430846in}{2.999895in}}%
\pgfpathlineto{\pgfqpoint{1.431966in}{3.001004in}}%
\pgfpathlineto{\pgfqpoint{1.433087in}{2.992762in}}%
\pgfpathlineto{\pgfqpoint{1.434207in}{3.012258in}}%
\pgfpathlineto{\pgfqpoint{1.435328in}{3.003223in}}%
\pgfpathlineto{\pgfqpoint{1.440931in}{2.918424in}}%
\pgfpathlineto{\pgfqpoint{1.443172in}{2.957574in}}%
\pgfpathlineto{\pgfqpoint{1.446533in}{2.969621in}}%
\pgfpathlineto{\pgfqpoint{1.447654in}{2.988483in}}%
\pgfpathlineto{\pgfqpoint{1.449895in}{3.001480in}}%
\pgfpathlineto{\pgfqpoint{1.451015in}{3.009405in}}%
\pgfpathlineto{\pgfqpoint{1.455497in}{3.007503in}}%
\pgfpathlineto{\pgfqpoint{1.456618in}{3.011783in}}%
\pgfpathlineto{\pgfqpoint{1.457738in}{3.025731in}}%
\pgfpathlineto{\pgfqpoint{1.462221in}{3.045861in}}%
\pgfpathlineto{\pgfqpoint{1.463341in}{3.037460in}}%
\pgfpathlineto{\pgfqpoint{1.464462in}{3.040630in}}%
\pgfpathlineto{\pgfqpoint{1.466703in}{3.053310in}}%
\pgfpathlineto{\pgfqpoint{1.470064in}{3.049665in}}%
\pgfpathlineto{\pgfqpoint{1.471185in}{3.061236in}}%
\pgfpathlineto{\pgfqpoint{1.472305in}{3.058699in}}%
\pgfpathlineto{\pgfqpoint{1.473426in}{3.063613in}}%
\pgfpathlineto{\pgfqpoint{1.474546in}{3.071221in}}%
\pgfpathlineto{\pgfqpoint{1.477908in}{3.063613in}}%
\pgfpathlineto{\pgfqpoint{1.479029in}{3.033973in}}%
\pgfpathlineto{\pgfqpoint{1.480149in}{3.036351in}}%
\pgfpathlineto{\pgfqpoint{1.481270in}{2.995774in}}%
\pgfpathlineto{\pgfqpoint{1.482390in}{2.991653in}}%
\pgfpathlineto{\pgfqpoint{1.485752in}{3.016855in}}%
\pgfpathlineto{\pgfqpoint{1.486872in}{3.020659in}}%
\pgfpathlineto{\pgfqpoint{1.487993in}{3.011307in}}%
\pgfpathlineto{\pgfqpoint{1.489113in}{3.007978in}}%
\pgfpathlineto{\pgfqpoint{1.490234in}{3.019074in}}%
\pgfpathlineto{\pgfqpoint{1.493595in}{3.007661in}}%
\pgfpathlineto{\pgfqpoint{1.494716in}{3.027950in}}%
\pgfpathlineto{\pgfqpoint{1.496957in}{3.008454in}}%
\pgfpathlineto{\pgfqpoint{1.498077in}{3.022402in}}%
\pgfpathlineto{\pgfqpoint{1.501439in}{3.050774in}}%
\pgfpathlineto{\pgfqpoint{1.502560in}{3.065832in}}%
\pgfpathlineto{\pgfqpoint{1.503680in}{3.092936in}}%
\pgfpathlineto{\pgfqpoint{1.504801in}{3.091510in}}%
\pgfpathlineto{\pgfqpoint{1.505921in}{3.069319in}}%
\pgfpathlineto{\pgfqpoint{1.510403in}{3.035875in}}%
\pgfpathlineto{\pgfqpoint{1.511524in}{3.053469in}}%
\pgfpathlineto{\pgfqpoint{1.512644in}{3.018915in}}%
\pgfpathlineto{\pgfqpoint{1.513765in}{3.010198in}}%
\pgfpathlineto{\pgfqpoint{1.517126in}{3.024304in}}%
\pgfpathlineto{\pgfqpoint{1.518247in}{3.036826in}}%
\pgfpathlineto{\pgfqpoint{1.519367in}{3.067893in}}%
\pgfpathlineto{\pgfqpoint{1.520488in}{3.074867in}}%
\pgfpathlineto{\pgfqpoint{1.524970in}{3.071063in}}%
\pgfpathlineto{\pgfqpoint{1.527211in}{3.089925in}}%
\pgfpathlineto{\pgfqpoint{1.528332in}{3.080573in}}%
\pgfpathlineto{\pgfqpoint{1.529452in}{3.055846in}}%
\pgfpathlineto{\pgfqpoint{1.532814in}{3.062662in}}%
\pgfpathlineto{\pgfqpoint{1.533934in}{3.060760in}}%
\pgfpathlineto{\pgfqpoint{1.535055in}{3.071538in}}%
\pgfpathlineto{\pgfqpoint{1.536175in}{3.050140in}}%
\pgfpathlineto{\pgfqpoint{1.537296in}{3.046336in}}%
\pgfpathlineto{\pgfqpoint{1.540658in}{3.050457in}}%
\pgfpathlineto{\pgfqpoint{1.541778in}{3.039204in}}%
\pgfpathlineto{\pgfqpoint{1.542899in}{3.051725in}}%
\pgfpathlineto{\pgfqpoint{1.544019in}{3.053310in}}%
\pgfpathlineto{\pgfqpoint{1.545140in}{3.052835in}}%
\pgfpathlineto{\pgfqpoint{1.548501in}{3.075659in}}%
\pgfpathlineto{\pgfqpoint{1.549622in}{3.077244in}}%
\pgfpathlineto{\pgfqpoint{1.550742in}{3.065515in}}%
\pgfpathlineto{\pgfqpoint{1.552983in}{3.027157in}}%
\pgfpathlineto{\pgfqpoint{1.556345in}{3.033339in}}%
\pgfpathlineto{\pgfqpoint{1.557465in}{3.006552in}}%
\pgfpathlineto{\pgfqpoint{1.558586in}{3.030803in}}%
\pgfpathlineto{\pgfqpoint{1.559706in}{3.033656in}}%
\pgfpathlineto{\pgfqpoint{1.560827in}{3.040630in}}%
\pgfpathlineto{\pgfqpoint{1.566430in}{3.046336in}}%
\pgfpathlineto{\pgfqpoint{1.567550in}{3.052042in}}%
\pgfpathlineto{\pgfqpoint{1.568671in}{3.050140in}}%
\pgfpathlineto{\pgfqpoint{1.573153in}{3.072648in}}%
\pgfpathlineto{\pgfqpoint{1.574273in}{3.062979in}}%
\pgfpathlineto{\pgfqpoint{1.576514in}{3.087706in}}%
\pgfpathlineto{\pgfqpoint{1.579876in}{3.104348in}}%
\pgfpathlineto{\pgfqpoint{1.583238in}{3.059016in}}%
\pgfpathlineto{\pgfqpoint{1.584358in}{3.058224in}}%
\pgfpathlineto{\pgfqpoint{1.587720in}{3.059333in}}%
\pgfpathlineto{\pgfqpoint{1.588840in}{3.061236in}}%
\pgfpathlineto{\pgfqpoint{1.589961in}{3.064247in}}%
\pgfpathlineto{\pgfqpoint{1.592202in}{3.075025in}}%
\pgfpathlineto{\pgfqpoint{1.595563in}{3.062979in}}%
\pgfpathlineto{\pgfqpoint{1.596684in}{3.042691in}}%
\pgfpathlineto{\pgfqpoint{1.597804in}{3.048714in}}%
\pgfpathlineto{\pgfqpoint{1.598925in}{3.043642in}}%
\pgfpathlineto{\pgfqpoint{1.600045in}{3.055212in}}%
\pgfpathlineto{\pgfqpoint{1.603407in}{3.039362in}}%
\pgfpathlineto{\pgfqpoint{1.604528in}{3.046178in}}%
\pgfpathlineto{\pgfqpoint{1.605648in}{3.035083in}}%
\pgfpathlineto{\pgfqpoint{1.606769in}{3.040472in}}%
\pgfpathlineto{\pgfqpoint{1.611251in}{3.034448in}}%
\pgfpathlineto{\pgfqpoint{1.612371in}{3.020025in}}%
\pgfpathlineto{\pgfqpoint{1.614612in}{3.012258in}}%
\pgfpathlineto{\pgfqpoint{1.615733in}{3.020976in}}%
\pgfpathlineto{\pgfqpoint{1.619094in}{3.031437in}}%
\pgfpathlineto{\pgfqpoint{1.620215in}{3.030803in}}%
\pgfpathlineto{\pgfqpoint{1.621335in}{3.024146in}}%
\pgfpathlineto{\pgfqpoint{1.622456in}{3.001797in}}%
\pgfpathlineto{\pgfqpoint{1.623577in}{3.013051in}}%
\pgfpathlineto{\pgfqpoint{1.626938in}{3.004808in}}%
\pgfpathlineto{\pgfqpoint{1.629179in}{2.959318in}}%
\pgfpathlineto{\pgfqpoint{1.630300in}{2.948381in}}%
\pgfpathlineto{\pgfqpoint{1.631420in}{2.947430in}}%
\pgfpathlineto{\pgfqpoint{1.634782in}{2.949015in}}%
\pgfpathlineto{\pgfqpoint{1.637023in}{2.912401in}}%
\pgfpathlineto{\pgfqpoint{1.638143in}{2.895124in}}%
\pgfpathlineto{\pgfqpoint{1.639264in}{2.889577in}}%
\pgfpathlineto{\pgfqpoint{1.642625in}{2.893856in}}%
\pgfpathlineto{\pgfqpoint{1.643746in}{2.893381in}}%
\pgfpathlineto{\pgfqpoint{1.644867in}{2.876104in}}%
\pgfpathlineto{\pgfqpoint{1.645987in}{2.882127in}}%
\pgfpathlineto{\pgfqpoint{1.647108in}{2.906536in}}%
\pgfpathlineto{\pgfqpoint{1.650469in}{2.903208in}}%
\pgfpathlineto{\pgfqpoint{1.651590in}{2.891796in}}%
\pgfpathlineto{\pgfqpoint{1.652710in}{2.909231in}}%
\pgfpathlineto{\pgfqpoint{1.653831in}{2.912243in}}%
\pgfpathlineto{\pgfqpoint{1.654951in}{2.910341in}}%
\pgfpathlineto{\pgfqpoint{1.660554in}{2.967719in}}%
\pgfpathlineto{\pgfqpoint{1.661674in}{2.972632in}}%
\pgfpathlineto{\pgfqpoint{1.662795in}{2.963281in}}%
\pgfpathlineto{\pgfqpoint{1.666157in}{2.969304in}}%
\pgfpathlineto{\pgfqpoint{1.667277in}{2.967243in}}%
\pgfpathlineto{\pgfqpoint{1.668398in}{2.959001in}}%
\pgfpathlineto{\pgfqpoint{1.669518in}{2.959318in}}%
\pgfpathlineto{\pgfqpoint{1.670639in}{2.943309in}}%
\pgfpathlineto{\pgfqpoint{1.676241in}{2.960586in}}%
\pgfpathlineto{\pgfqpoint{1.677362in}{2.960745in}}%
\pgfpathlineto{\pgfqpoint{1.678482in}{2.953453in}}%
\pgfpathlineto{\pgfqpoint{1.681844in}{2.952027in}}%
\pgfpathlineto{\pgfqpoint{1.682964in}{2.953136in}}%
\pgfpathlineto{\pgfqpoint{1.684085in}{2.950759in}}%
\pgfpathlineto{\pgfqpoint{1.685206in}{2.950759in}}%
\pgfpathlineto{\pgfqpoint{1.686326in}{2.948540in}}%
\pgfpathlineto{\pgfqpoint{1.689688in}{2.948223in}}%
\pgfpathlineto{\pgfqpoint{1.690808in}{2.951234in}}%
\pgfpathlineto{\pgfqpoint{1.691929in}{2.944102in}}%
\pgfpathlineto{\pgfqpoint{1.693049in}{2.950917in}}%
\pgfpathlineto{\pgfqpoint{1.694170in}{2.949966in}}%
\pgfpathlineto{\pgfqpoint{1.697531in}{2.922070in}}%
\pgfpathlineto{\pgfqpoint{1.698652in}{2.907329in}}%
\pgfpathlineto{\pgfqpoint{1.699772in}{2.916522in}}%
\pgfpathlineto{\pgfqpoint{1.700893in}{2.894173in}}%
\pgfpathlineto{\pgfqpoint{1.702013in}{2.904793in}}%
\pgfpathlineto{\pgfqpoint{1.705375in}{2.902574in}}%
\pgfpathlineto{\pgfqpoint{1.706496in}{2.909865in}}%
\pgfpathlineto{\pgfqpoint{1.707616in}{2.885139in}}%
\pgfpathlineto{\pgfqpoint{1.708737in}{2.876579in}}%
\pgfpathlineto{\pgfqpoint{1.709857in}{2.893222in}}%
\pgfpathlineto{\pgfqpoint{1.713219in}{2.890369in}}%
\pgfpathlineto{\pgfqpoint{1.714339in}{2.851060in}}%
\pgfpathlineto{\pgfqpoint{1.715460in}{2.868020in}}%
\pgfpathlineto{\pgfqpoint{1.716580in}{2.830138in}}%
\pgfpathlineto{\pgfqpoint{1.717701in}{2.830138in}}%
\pgfpathlineto{\pgfqpoint{1.721062in}{2.821262in}}%
\pgfpathlineto{\pgfqpoint{1.722183in}{2.832674in}}%
\pgfpathlineto{\pgfqpoint{1.723303in}{2.819360in}}%
\pgfpathlineto{\pgfqpoint{1.724424in}{2.820152in}}%
\pgfpathlineto{\pgfqpoint{1.725544in}{2.852487in}}%
\pgfpathlineto{\pgfqpoint{1.728906in}{2.851853in}}%
\pgfpathlineto{\pgfqpoint{1.730027in}{2.858827in}}%
\pgfpathlineto{\pgfqpoint{1.731147in}{2.847573in}}%
\pgfpathlineto{\pgfqpoint{1.732268in}{2.875787in}}%
\pgfpathlineto{\pgfqpoint{1.733388in}{2.884822in}}%
\pgfpathlineto{\pgfqpoint{1.736750in}{2.890052in}}%
\pgfpathlineto{\pgfqpoint{1.737870in}{2.919217in}}%
\pgfpathlineto{\pgfqpoint{1.738991in}{2.913352in}}%
\pgfpathlineto{\pgfqpoint{1.741232in}{2.929519in}}%
\pgfpathlineto{\pgfqpoint{1.744593in}{2.919692in}}%
\pgfpathlineto{\pgfqpoint{1.745714in}{2.927934in}}%
\pgfpathlineto{\pgfqpoint{1.747955in}{2.951868in}}%
\pgfpathlineto{\pgfqpoint{1.749076in}{2.958842in}}%
\pgfpathlineto{\pgfqpoint{1.752437in}{2.957733in}}%
\pgfpathlineto{\pgfqpoint{1.753558in}{2.947272in}}%
\pgfpathlineto{\pgfqpoint{1.754678in}{2.953929in}}%
\pgfpathlineto{\pgfqpoint{1.755799in}{2.953929in}}%
\pgfpathlineto{\pgfqpoint{1.756919in}{2.944260in}}%
\pgfpathlineto{\pgfqpoint{1.760281in}{2.942992in}}%
\pgfpathlineto{\pgfqpoint{1.761401in}{2.963281in}}%
\pgfpathlineto{\pgfqpoint{1.762522in}{2.961379in}}%
\pgfpathlineto{\pgfqpoint{1.763642in}{2.963598in}}%
\pgfpathlineto{\pgfqpoint{1.764763in}{2.984520in}}%
\pgfpathlineto{\pgfqpoint{1.768125in}{2.963122in}}%
\pgfpathlineto{\pgfqpoint{1.769245in}{3.004491in}}%
\pgfpathlineto{\pgfqpoint{1.770366in}{2.982618in}}%
\pgfpathlineto{\pgfqpoint{1.772607in}{2.981350in}}%
\pgfpathlineto{\pgfqpoint{1.775968in}{2.975961in}}%
\pgfpathlineto{\pgfqpoint{1.777089in}{2.975802in}}%
\pgfpathlineto{\pgfqpoint{1.778209in}{2.994347in}}%
\pgfpathlineto{\pgfqpoint{1.779330in}{2.997359in}}%
\pgfpathlineto{\pgfqpoint{1.780450in}{2.998627in}}%
\pgfpathlineto{\pgfqpoint{1.783812in}{3.023512in}}%
\pgfpathlineto{\pgfqpoint{1.784932in}{3.050933in}}%
\pgfpathlineto{\pgfqpoint{1.786053in}{3.029852in}}%
\pgfpathlineto{\pgfqpoint{1.787173in}{3.037460in}}%
\pgfpathlineto{\pgfqpoint{1.788294in}{3.010673in}}%
\pgfpathlineto{\pgfqpoint{1.791656in}{3.007344in}}%
\pgfpathlineto{\pgfqpoint{1.793897in}{3.035558in}}%
\pgfpathlineto{\pgfqpoint{1.795017in}{3.078829in}}%
\pgfpathlineto{\pgfqpoint{1.796138in}{3.059492in}}%
\pgfpathlineto{\pgfqpoint{1.799499in}{3.081365in}}%
\pgfpathlineto{\pgfqpoint{1.800620in}{3.082316in}}%
\pgfpathlineto{\pgfqpoint{1.801740in}{3.078037in}}%
\pgfpathlineto{\pgfqpoint{1.803981in}{3.082792in}}%
\pgfpathlineto{\pgfqpoint{1.807343in}{3.077403in}}%
\pgfpathlineto{\pgfqpoint{1.808463in}{3.068051in}}%
\pgfpathlineto{\pgfqpoint{1.809584in}{3.051091in}}%
\pgfpathlineto{\pgfqpoint{1.811825in}{3.051567in}}%
\pgfpathlineto{\pgfqpoint{1.815187in}{3.024463in}}%
\pgfpathlineto{\pgfqpoint{1.816307in}{3.001797in}}%
\pgfpathlineto{\pgfqpoint{1.817428in}{3.018915in}}%
\pgfpathlineto{\pgfqpoint{1.818548in}{3.046178in}}%
\pgfpathlineto{\pgfqpoint{1.819669in}{3.037143in}}%
\pgfpathlineto{\pgfqpoint{1.823030in}{3.043325in}}%
\pgfpathlineto{\pgfqpoint{1.824151in}{3.041740in}}%
\pgfpathlineto{\pgfqpoint{1.825271in}{3.030169in}}%
\pgfpathlineto{\pgfqpoint{1.826392in}{3.030169in}}%
\pgfpathlineto{\pgfqpoint{1.827512in}{3.067417in}}%
\pgfpathlineto{\pgfqpoint{1.831995in}{3.086913in}}%
\pgfpathlineto{\pgfqpoint{1.834236in}{3.128282in}}%
\pgfpathlineto{\pgfqpoint{1.838718in}{3.104190in}}%
\pgfpathlineto{\pgfqpoint{1.839838in}{3.109896in}}%
\pgfpathlineto{\pgfqpoint{1.840959in}{3.078195in}}%
\pgfpathlineto{\pgfqpoint{1.842079in}{3.071538in}}%
\pgfpathlineto{\pgfqpoint{1.843200in}{3.047921in}}%
\pgfpathlineto{\pgfqpoint{1.846561in}{3.073440in}}%
\pgfpathlineto{\pgfqpoint{1.847682in}{3.106092in}}%
\pgfpathlineto{\pgfqpoint{1.848802in}{3.090559in}}%
\pgfpathlineto{\pgfqpoint{1.849923in}{3.123369in}}%
\pgfpathlineto{\pgfqpoint{1.851044in}{3.119089in}}%
\pgfpathlineto{\pgfqpoint{1.854405in}{3.112274in}}%
\pgfpathlineto{\pgfqpoint{1.855526in}{3.113383in}}%
\pgfpathlineto{\pgfqpoint{1.856646in}{3.112432in}}%
\pgfpathlineto{\pgfqpoint{1.857767in}{3.125112in}}%
\pgfpathlineto{\pgfqpoint{1.858887in}{3.148729in}}%
\pgfpathlineto{\pgfqpoint{1.863369in}{3.149839in}}%
\pgfpathlineto{\pgfqpoint{1.865610in}{3.172346in}}%
\pgfpathlineto{\pgfqpoint{1.866731in}{3.189465in}}%
\pgfpathlineto{\pgfqpoint{1.870092in}{3.183917in}}%
\pgfpathlineto{\pgfqpoint{1.871213in}{3.185027in}}%
\pgfpathlineto{\pgfqpoint{1.872334in}{3.179003in}}%
\pgfpathlineto{\pgfqpoint{1.874575in}{3.158239in}}%
\pgfpathlineto{\pgfqpoint{1.877936in}{3.176467in}}%
\pgfpathlineto{\pgfqpoint{1.879057in}{3.155228in}}%
\pgfpathlineto{\pgfqpoint{1.880177in}{3.145876in}}%
\pgfpathlineto{\pgfqpoint{1.881298in}{3.142231in}}%
\pgfpathlineto{\pgfqpoint{1.882418in}{3.122893in}}%
\pgfpathlineto{\pgfqpoint{1.885780in}{3.155862in}}%
\pgfpathlineto{\pgfqpoint{1.886900in}{3.094046in}}%
\pgfpathlineto{\pgfqpoint{1.888021in}{3.107360in}}%
\pgfpathlineto{\pgfqpoint{1.889141in}{3.148729in}}%
\pgfpathlineto{\pgfqpoint{1.890262in}{3.113066in}}%
\pgfpathlineto{\pgfqpoint{1.893624in}{3.131928in}}%
\pgfpathlineto{\pgfqpoint{1.894744in}{3.129075in}}%
\pgfpathlineto{\pgfqpoint{1.895865in}{3.135415in}}%
\pgfpathlineto{\pgfqpoint{1.896985in}{3.122259in}}%
\pgfpathlineto{\pgfqpoint{1.898106in}{3.123369in}}%
\pgfpathlineto{\pgfqpoint{1.901467in}{3.112274in}}%
\pgfpathlineto{\pgfqpoint{1.902588in}{3.115602in}}%
\pgfpathlineto{\pgfqpoint{1.903708in}{3.080573in}}%
\pgfpathlineto{\pgfqpoint{1.904829in}{3.074708in}}%
\pgfpathlineto{\pgfqpoint{1.905949in}{3.086913in}}%
\pgfpathlineto{\pgfqpoint{1.909311in}{3.114334in}}%
\pgfpathlineto{\pgfqpoint{1.911552in}{3.072965in}}%
\pgfpathlineto{\pgfqpoint{1.912673in}{3.090083in}}%
\pgfpathlineto{\pgfqpoint{1.917155in}{3.100544in}}%
\pgfpathlineto{\pgfqpoint{1.918275in}{3.095314in}}%
\pgfpathlineto{\pgfqpoint{1.919396in}{3.100227in}}%
\pgfpathlineto{\pgfqpoint{1.920516in}{3.100861in}}%
\pgfpathlineto{\pgfqpoint{1.921637in}{3.109103in}}%
\pgfpathlineto{\pgfqpoint{1.924998in}{3.094204in}}%
\pgfpathlineto{\pgfqpoint{1.927239in}{3.100703in}}%
\pgfpathlineto{\pgfqpoint{1.928360in}{3.095155in}}%
\pgfpathlineto{\pgfqpoint{1.929480in}{3.061394in}}%
\pgfpathlineto{\pgfqpoint{1.933963in}{3.087547in}}%
\pgfpathlineto{\pgfqpoint{1.935083in}{3.087706in}}%
\pgfpathlineto{\pgfqpoint{1.936204in}{3.091985in}}%
\pgfpathlineto{\pgfqpoint{1.937324in}{3.076293in}}%
\pgfpathlineto{\pgfqpoint{1.940686in}{3.070429in}}%
\pgfpathlineto{\pgfqpoint{1.941806in}{3.075184in}}%
\pgfpathlineto{\pgfqpoint{1.942927in}{3.065198in}}%
\pgfpathlineto{\pgfqpoint{1.944047in}{3.042057in}}%
\pgfpathlineto{\pgfqpoint{1.945168in}{3.066149in}}%
\pgfpathlineto{\pgfqpoint{1.948529in}{3.080414in}}%
\pgfpathlineto{\pgfqpoint{1.949650in}{3.060919in}}%
\pgfpathlineto{\pgfqpoint{1.950770in}{3.060919in}}%
\pgfpathlineto{\pgfqpoint{1.951891in}{3.074708in}}%
\pgfpathlineto{\pgfqpoint{1.953011in}{3.108628in}}%
\pgfpathlineto{\pgfqpoint{1.957494in}{3.093412in}}%
\pgfpathlineto{\pgfqpoint{1.958614in}{3.102763in}}%
\pgfpathlineto{\pgfqpoint{1.959735in}{3.128124in}}%
\pgfpathlineto{\pgfqpoint{1.960855in}{3.118772in}}%
\pgfpathlineto{\pgfqpoint{1.964217in}{3.119089in}}%
\pgfpathlineto{\pgfqpoint{1.965337in}{3.127014in}}%
\pgfpathlineto{\pgfqpoint{1.966458in}{3.124161in}}%
\pgfpathlineto{\pgfqpoint{1.967578in}{3.127807in}}%
\pgfpathlineto{\pgfqpoint{1.973181in}{3.095631in}}%
\pgfpathlineto{\pgfqpoint{1.974302in}{3.106567in}}%
\pgfpathlineto{\pgfqpoint{1.975422in}{3.107360in}}%
\pgfpathlineto{\pgfqpoint{1.976543in}{3.099910in}}%
\pgfpathlineto{\pgfqpoint{1.979904in}{3.097374in}}%
\pgfpathlineto{\pgfqpoint{1.981025in}{3.102288in}}%
\pgfpathlineto{\pgfqpoint{1.982145in}{3.119089in}}%
\pgfpathlineto{\pgfqpoint{1.983266in}{3.099435in}}%
\pgfpathlineto{\pgfqpoint{1.984386in}{3.097374in}}%
\pgfpathlineto{\pgfqpoint{1.987748in}{3.085645in}}%
\pgfpathlineto{\pgfqpoint{1.988868in}{3.087389in}}%
\pgfpathlineto{\pgfqpoint{1.991109in}{3.116395in}}%
\pgfpathlineto{\pgfqpoint{1.992230in}{3.106092in}}%
\pgfpathlineto{\pgfqpoint{1.995592in}{3.063296in}}%
\pgfpathlineto{\pgfqpoint{1.996712in}{3.067893in}}%
\pgfpathlineto{\pgfqpoint{1.997833in}{3.069795in}}%
\pgfpathlineto{\pgfqpoint{1.998953in}{3.081682in}}%
\pgfpathlineto{\pgfqpoint{2.000074in}{3.067259in}}%
\pgfpathlineto{\pgfqpoint{2.003435in}{3.072806in}}%
\pgfpathlineto{\pgfqpoint{2.004556in}{3.072172in}}%
\pgfpathlineto{\pgfqpoint{2.006797in}{3.044434in}}%
\pgfpathlineto{\pgfqpoint{2.007917in}{3.047604in}}%
\pgfpathlineto{\pgfqpoint{2.012399in}{3.010515in}}%
\pgfpathlineto{\pgfqpoint{2.013520in}{3.008929in}}%
\pgfpathlineto{\pgfqpoint{2.014640in}{2.991019in}}%
\pgfpathlineto{\pgfqpoint{2.019123in}{2.988166in}}%
\pgfpathlineto{\pgfqpoint{2.020243in}{2.998627in}}%
\pgfpathlineto{\pgfqpoint{2.021364in}{2.978180in}}%
\pgfpathlineto{\pgfqpoint{2.022484in}{2.981825in}}%
\pgfpathlineto{\pgfqpoint{2.023605in}{2.999895in}}%
\pgfpathlineto{\pgfqpoint{2.026966in}{3.019866in}}%
\pgfpathlineto{\pgfqpoint{2.028087in}{3.018915in}}%
\pgfpathlineto{\pgfqpoint{2.029207in}{3.015111in}}%
\pgfpathlineto{\pgfqpoint{2.030328in}{3.015270in}}%
\pgfpathlineto{\pgfqpoint{2.031448in}{3.007820in}}%
\pgfpathlineto{\pgfqpoint{2.034810in}{3.004174in}}%
\pgfpathlineto{\pgfqpoint{2.035931in}{2.892905in}}%
\pgfpathlineto{\pgfqpoint{2.037051in}{2.876421in}}%
\pgfpathlineto{\pgfqpoint{2.038172in}{2.870398in}}%
\pgfpathlineto{\pgfqpoint{2.039292in}{2.844245in}}%
\pgfpathlineto{\pgfqpoint{2.042654in}{2.838063in}}%
\pgfpathlineto{\pgfqpoint{2.043774in}{2.839490in}}%
\pgfpathlineto{\pgfqpoint{2.044895in}{2.845037in}}%
\pgfpathlineto{\pgfqpoint{2.046015in}{2.864533in}}%
\pgfpathlineto{\pgfqpoint{2.047136in}{2.858668in}}%
\pgfpathlineto{\pgfqpoint{2.052738in}{2.838063in}}%
\pgfpathlineto{\pgfqpoint{2.053859in}{2.839490in}}%
\pgfpathlineto{\pgfqpoint{2.054979in}{2.829662in}}%
\pgfpathlineto{\pgfqpoint{2.058341in}{2.847732in}}%
\pgfpathlineto{\pgfqpoint{2.059462in}{2.834259in}}%
\pgfpathlineto{\pgfqpoint{2.060582in}{2.844562in}}%
\pgfpathlineto{\pgfqpoint{2.061703in}{2.840124in}}%
\pgfpathlineto{\pgfqpoint{2.062823in}{2.844720in}}%
\pgfpathlineto{\pgfqpoint{2.066185in}{2.854230in}}%
\pgfpathlineto{\pgfqpoint{2.067305in}{2.855023in}}%
\pgfpathlineto{\pgfqpoint{2.068426in}{2.837746in}}%
\pgfpathlineto{\pgfqpoint{2.070667in}{2.762774in}}%
\pgfpathlineto{\pgfqpoint{2.074028in}{2.732024in}}%
\pgfpathlineto{\pgfqpoint{2.075149in}{2.702226in}}%
\pgfpathlineto{\pgfqpoint{2.077390in}{2.766578in}}%
\pgfpathlineto{\pgfqpoint{2.078511in}{2.766103in}}%
\pgfpathlineto{\pgfqpoint{2.081872in}{2.742644in}}%
\pgfpathlineto{\pgfqpoint{2.082993in}{2.715382in}}%
\pgfpathlineto{\pgfqpoint{2.084113in}{2.737255in}}%
\pgfpathlineto{\pgfqpoint{2.085234in}{2.746924in}}%
\pgfpathlineto{\pgfqpoint{2.086354in}{2.729330in}}%
\pgfpathlineto{\pgfqpoint{2.090836in}{2.760080in}}%
\pgfpathlineto{\pgfqpoint{2.091957in}{2.746131in}}%
\pgfpathlineto{\pgfqpoint{2.093077in}{2.739633in}}%
\pgfpathlineto{\pgfqpoint{2.094198in}{2.753105in}}%
\pgfpathlineto{\pgfqpoint{2.097559in}{2.745814in}}%
\pgfpathlineto{\pgfqpoint{2.099801in}{2.776247in}}%
\pgfpathlineto{\pgfqpoint{2.100921in}{2.767212in}}%
\pgfpathlineto{\pgfqpoint{2.102042in}{2.734878in}}%
\pgfpathlineto{\pgfqpoint{2.105403in}{2.741852in}}%
\pgfpathlineto{\pgfqpoint{2.106524in}{2.693508in}}%
\pgfpathlineto{\pgfqpoint{2.107644in}{2.675756in}}%
\pgfpathlineto{\pgfqpoint{2.108765in}{2.673695in}}%
\pgfpathlineto{\pgfqpoint{2.109885in}{2.680511in}}%
\pgfpathlineto{\pgfqpoint{2.113247in}{2.673854in}}%
\pgfpathlineto{\pgfqpoint{2.115488in}{2.704920in}}%
\pgfpathlineto{\pgfqpoint{2.116608in}{2.695886in}}%
\pgfpathlineto{\pgfqpoint{2.117729in}{2.716174in}}%
\pgfpathlineto{\pgfqpoint{2.121091in}{2.751996in}}%
\pgfpathlineto{\pgfqpoint{2.122211in}{2.756275in}}%
\pgfpathlineto{\pgfqpoint{2.125573in}{2.796852in}}%
\pgfpathlineto{\pgfqpoint{2.128934in}{2.797645in}}%
\pgfpathlineto{\pgfqpoint{2.130055in}{2.781160in}}%
\pgfpathlineto{\pgfqpoint{2.131175in}{2.750728in}}%
\pgfpathlineto{\pgfqpoint{2.132296in}{2.765310in}}%
\pgfpathlineto{\pgfqpoint{2.133416in}{2.762616in}}%
\pgfpathlineto{\pgfqpoint{2.136778in}{2.748984in}}%
\pgfpathlineto{\pgfqpoint{2.139019in}{2.834259in}}%
\pgfpathlineto{\pgfqpoint{2.140140in}{2.860412in}}%
\pgfpathlineto{\pgfqpoint{2.141260in}{2.872458in}}%
\pgfpathlineto{\pgfqpoint{2.144622in}{2.866752in}}%
\pgfpathlineto{\pgfqpoint{2.145742in}{2.847415in}}%
\pgfpathlineto{\pgfqpoint{2.146863in}{2.853755in}}%
\pgfpathlineto{\pgfqpoint{2.147983in}{2.849792in}}%
\pgfpathlineto{\pgfqpoint{2.149104in}{2.840599in}}%
\pgfpathlineto{\pgfqpoint{2.152465in}{2.854706in}}%
\pgfpathlineto{\pgfqpoint{2.154706in}{2.868337in}}%
\pgfpathlineto{\pgfqpoint{2.155827in}{2.874994in}}%
\pgfpathlineto{\pgfqpoint{2.156947in}{2.874994in}}%
\pgfpathlineto{\pgfqpoint{2.162550in}{2.847573in}}%
\pgfpathlineto{\pgfqpoint{2.163671in}{2.861522in}}%
\pgfpathlineto{\pgfqpoint{2.164791in}{2.821737in}}%
\pgfpathlineto{\pgfqpoint{2.168153in}{2.841392in}}%
\pgfpathlineto{\pgfqpoint{2.169273in}{2.837271in}}%
\pgfpathlineto{\pgfqpoint{2.170394in}{2.839648in}}%
\pgfpathlineto{\pgfqpoint{2.171514in}{2.848207in}}%
\pgfpathlineto{\pgfqpoint{2.172635in}{2.846939in}}%
\pgfpathlineto{\pgfqpoint{2.175996in}{2.845196in}}%
\pgfpathlineto{\pgfqpoint{2.177117in}{2.834576in}}%
\pgfpathlineto{\pgfqpoint{2.178237in}{2.833308in}}%
\pgfpathlineto{\pgfqpoint{2.183840in}{2.815556in}}%
\pgfpathlineto{\pgfqpoint{2.184961in}{2.824907in}}%
\pgfpathlineto{\pgfqpoint{2.186081in}{2.803985in}}%
\pgfpathlineto{\pgfqpoint{2.187202in}{2.796060in}}%
\pgfpathlineto{\pgfqpoint{2.188322in}{2.810959in}}%
\pgfpathlineto{\pgfqpoint{2.191684in}{2.813495in}}%
\pgfpathlineto{\pgfqpoint{2.192804in}{2.789244in}}%
\pgfpathlineto{\pgfqpoint{2.193925in}{2.788135in}}%
\pgfpathlineto{\pgfqpoint{2.195045in}{2.784013in}}%
\pgfpathlineto{\pgfqpoint{2.196166in}{2.775613in}}%
\pgfpathlineto{\pgfqpoint{2.199527in}{2.771809in}}%
\pgfpathlineto{\pgfqpoint{2.200648in}{2.775137in}}%
\pgfpathlineto{\pgfqpoint{2.201769in}{2.802241in}}%
\pgfpathlineto{\pgfqpoint{2.204010in}{2.761823in}}%
\pgfpathlineto{\pgfqpoint{2.207371in}{2.780209in}}%
\pgfpathlineto{\pgfqpoint{2.209612in}{2.819201in}}%
\pgfpathlineto{\pgfqpoint{2.210733in}{2.819201in}}%
\pgfpathlineto{\pgfqpoint{2.215215in}{2.816190in}}%
\pgfpathlineto{\pgfqpoint{2.216335in}{2.833308in}}%
\pgfpathlineto{\pgfqpoint{2.217456in}{2.827919in}}%
\pgfpathlineto{\pgfqpoint{2.218576in}{2.815873in}}%
\pgfpathlineto{\pgfqpoint{2.223059in}{2.808581in}}%
\pgfpathlineto{\pgfqpoint{2.224179in}{2.810801in}}%
\pgfpathlineto{\pgfqpoint{2.225300in}{2.773077in}}%
\pgfpathlineto{\pgfqpoint{2.227541in}{2.733609in}}%
\pgfpathlineto{\pgfqpoint{2.232023in}{2.735036in}}%
\pgfpathlineto{\pgfqpoint{2.233143in}{2.710944in}}%
\pgfpathlineto{\pgfqpoint{2.234264in}{2.713480in}}%
\pgfpathlineto{\pgfqpoint{2.235384in}{2.664502in}}%
\pgfpathlineto{\pgfqpoint{2.239866in}{2.658796in}}%
\pgfpathlineto{\pgfqpoint{2.240987in}{2.652931in}}%
\pgfpathlineto{\pgfqpoint{2.243228in}{2.674963in}}%
\pgfpathlineto{\pgfqpoint{2.246590in}{2.653882in}}%
\pgfpathlineto{\pgfqpoint{2.247710in}{2.664819in}}%
\pgfpathlineto{\pgfqpoint{2.248831in}{2.667038in}}%
\pgfpathlineto{\pgfqpoint{2.249951in}{2.676390in}}%
\pgfpathlineto{\pgfqpoint{2.251072in}{2.694301in}}%
\pgfpathlineto{\pgfqpoint{2.254433in}{2.692399in}}%
\pgfpathlineto{\pgfqpoint{2.255554in}{2.661491in}}%
\pgfpathlineto{\pgfqpoint{2.256674in}{2.669257in}}%
\pgfpathlineto{\pgfqpoint{2.257795in}{2.700641in}}%
\pgfpathlineto{\pgfqpoint{2.258915in}{2.696520in}}%
\pgfpathlineto{\pgfqpoint{2.262277in}{2.681145in}}%
\pgfpathlineto{\pgfqpoint{2.263398in}{2.687802in}}%
\pgfpathlineto{\pgfqpoint{2.264518in}{2.683839in}}%
\pgfpathlineto{\pgfqpoint{2.265639in}{2.650395in}}%
\pgfpathlineto{\pgfqpoint{2.266759in}{2.669099in}}%
\pgfpathlineto{\pgfqpoint{2.271241in}{2.676865in}}%
\pgfpathlineto{\pgfqpoint{2.272362in}{2.710151in}}%
\pgfpathlineto{\pgfqpoint{2.273482in}{2.713638in}}%
\pgfpathlineto{\pgfqpoint{2.274603in}{2.711736in}}%
\pgfpathlineto{\pgfqpoint{2.277964in}{2.772126in}}%
\pgfpathlineto{\pgfqpoint{2.279085in}{2.760872in}}%
\pgfpathlineto{\pgfqpoint{2.280205in}{2.790354in}}%
\pgfpathlineto{\pgfqpoint{2.281326in}{2.855498in}}%
\pgfpathlineto{\pgfqpoint{2.285808in}{2.834259in}}%
\pgfpathlineto{\pgfqpoint{2.286929in}{2.811276in}}%
\pgfpathlineto{\pgfqpoint{2.289170in}{2.826809in}}%
\pgfpathlineto{\pgfqpoint{2.290290in}{2.839807in}}%
\pgfpathlineto{\pgfqpoint{2.294772in}{2.838380in}}%
\pgfpathlineto{\pgfqpoint{2.297013in}{2.827760in}}%
\pgfpathlineto{\pgfqpoint{2.298134in}{2.836161in}}%
\pgfpathlineto{\pgfqpoint{2.301495in}{2.837271in}}%
\pgfpathlineto{\pgfqpoint{2.302616in}{2.829504in}}%
\pgfpathlineto{\pgfqpoint{2.304857in}{2.866752in}}%
\pgfpathlineto{\pgfqpoint{2.305978in}{2.869764in}}%
\pgfpathlineto{\pgfqpoint{2.309339in}{2.871507in}}%
\pgfpathlineto{\pgfqpoint{2.310460in}{2.864058in}}%
\pgfpathlineto{\pgfqpoint{2.311580in}{2.870556in}}%
\pgfpathlineto{\pgfqpoint{2.317183in}{2.867703in}}%
\pgfpathlineto{\pgfqpoint{2.318303in}{2.885456in}}%
\pgfpathlineto{\pgfqpoint{2.319424in}{2.887358in}}%
\pgfpathlineto{\pgfqpoint{2.321665in}{2.883236in}}%
\pgfpathlineto{\pgfqpoint{2.325027in}{2.887516in}}%
\pgfpathlineto{\pgfqpoint{2.326147in}{2.881651in}}%
\pgfpathlineto{\pgfqpoint{2.329509in}{2.902891in}}%
\pgfpathlineto{\pgfqpoint{2.332870in}{2.914937in}}%
\pgfpathlineto{\pgfqpoint{2.333991in}{2.928568in}}%
\pgfpathlineto{\pgfqpoint{2.335111in}{2.950917in}}%
\pgfpathlineto{\pgfqpoint{2.336232in}{2.952027in}}%
\pgfpathlineto{\pgfqpoint{2.337352in}{2.950442in}}%
\pgfpathlineto{\pgfqpoint{2.340714in}{2.958525in}}%
\pgfpathlineto{\pgfqpoint{2.341834in}{2.956623in}}%
\pgfpathlineto{\pgfqpoint{2.342955in}{2.964390in}}%
\pgfpathlineto{\pgfqpoint{2.344075in}{2.962964in}}%
\pgfpathlineto{\pgfqpoint{2.345196in}{2.966926in}}%
\pgfpathlineto{\pgfqpoint{2.348558in}{2.959635in}}%
\pgfpathlineto{\pgfqpoint{2.349678in}{2.953770in}}%
\pgfpathlineto{\pgfqpoint{2.350799in}{2.969621in}}%
\pgfpathlineto{\pgfqpoint{2.351919in}{2.945370in}}%
\pgfpathlineto{\pgfqpoint{2.353040in}{2.947430in}}%
\pgfpathlineto{\pgfqpoint{2.356401in}{2.947430in}}%
\pgfpathlineto{\pgfqpoint{2.358642in}{2.893381in}}%
\pgfpathlineto{\pgfqpoint{2.359763in}{2.887833in}}%
\pgfpathlineto{\pgfqpoint{2.360883in}{2.900038in}}%
\pgfpathlineto{\pgfqpoint{2.364245in}{2.884980in}}%
\pgfpathlineto{\pgfqpoint{2.365365in}{2.915254in}}%
\pgfpathlineto{\pgfqpoint{2.366486in}{2.907012in}}%
\pgfpathlineto{\pgfqpoint{2.367607in}{2.904951in}}%
\pgfpathlineto{\pgfqpoint{2.368727in}{2.887516in}}%
\pgfpathlineto{\pgfqpoint{2.372089in}{2.910182in}}%
\pgfpathlineto{\pgfqpoint{2.373209in}{2.884504in}}%
\pgfpathlineto{\pgfqpoint{2.374330in}{2.882919in}}%
\pgfpathlineto{\pgfqpoint{2.375450in}{2.871349in}}%
\pgfpathlineto{\pgfqpoint{2.376571in}{2.880066in}}%
\pgfpathlineto{\pgfqpoint{2.379932in}{2.877213in}}%
\pgfpathlineto{\pgfqpoint{2.381053in}{2.891954in}}%
\pgfpathlineto{\pgfqpoint{2.382173in}{2.897977in}}%
\pgfpathlineto{\pgfqpoint{2.383294in}{2.899404in}}%
\pgfpathlineto{\pgfqpoint{2.384414in}{2.904476in}}%
\pgfpathlineto{\pgfqpoint{2.388897in}{2.901781in}}%
\pgfpathlineto{\pgfqpoint{2.390017in}{2.898453in}}%
\pgfpathlineto{\pgfqpoint{2.391138in}{2.904951in}}%
\pgfpathlineto{\pgfqpoint{2.392258in}{2.899721in}}%
\pgfpathlineto{\pgfqpoint{2.395620in}{2.913352in}}%
\pgfpathlineto{\pgfqpoint{2.396740in}{2.914620in}}%
\pgfpathlineto{\pgfqpoint{2.397861in}{2.923813in}}%
\pgfpathlineto{\pgfqpoint{2.398981in}{2.927459in}}%
\pgfpathlineto{\pgfqpoint{2.400102in}{2.922704in}}%
\pgfpathlineto{\pgfqpoint{2.403463in}{2.911133in}}%
\pgfpathlineto{\pgfqpoint{2.404584in}{2.910816in}}%
\pgfpathlineto{\pgfqpoint{2.405704in}{2.901306in}}%
\pgfpathlineto{\pgfqpoint{2.406825in}{2.909865in}}%
\pgfpathlineto{\pgfqpoint{2.407946in}{2.910975in}}%
\pgfpathlineto{\pgfqpoint{2.411307in}{2.918266in}}%
\pgfpathlineto{\pgfqpoint{2.412428in}{2.916364in}}%
\pgfpathlineto{\pgfqpoint{2.413548in}{2.913194in}}%
\pgfpathlineto{\pgfqpoint{2.414669in}{2.927617in}}%
\pgfpathlineto{\pgfqpoint{2.415789in}{2.877055in}}%
\pgfpathlineto{\pgfqpoint{2.419151in}{2.852328in}}%
\pgfpathlineto{\pgfqpoint{2.422512in}{2.930787in}}%
\pgfpathlineto{\pgfqpoint{2.423633in}{2.933482in}}%
\pgfpathlineto{\pgfqpoint{2.428115in}{2.902415in}}%
\pgfpathlineto{\pgfqpoint{2.430356in}{2.922228in}}%
\pgfpathlineto{\pgfqpoint{2.431477in}{2.947113in}}%
\pgfpathlineto{\pgfqpoint{2.434838in}{2.951551in}}%
\pgfpathlineto{\pgfqpoint{2.437079in}{2.968194in}}%
\pgfpathlineto{\pgfqpoint{2.438200in}{2.968828in}}%
\pgfpathlineto{\pgfqpoint{2.439320in}{2.974217in}}%
\pgfpathlineto{\pgfqpoint{2.442682in}{2.974059in}}%
\pgfpathlineto{\pgfqpoint{2.443802in}{2.975961in}}%
\pgfpathlineto{\pgfqpoint{2.444923in}{2.982142in}}%
\pgfpathlineto{\pgfqpoint{2.446043in}{2.979289in}}%
\pgfpathlineto{\pgfqpoint{2.447164in}{2.968670in}}%
\pgfpathlineto{\pgfqpoint{2.450526in}{2.961696in}}%
\pgfpathlineto{\pgfqpoint{2.451646in}{3.009246in}}%
\pgfpathlineto{\pgfqpoint{2.453887in}{3.004808in}}%
\pgfpathlineto{\pgfqpoint{2.455008in}{3.005759in}}%
\pgfpathlineto{\pgfqpoint{2.458369in}{2.995298in}}%
\pgfpathlineto{\pgfqpoint{2.459490in}{2.985471in}}%
\pgfpathlineto{\pgfqpoint{2.460610in}{2.985788in}}%
\pgfpathlineto{\pgfqpoint{2.461731in}{2.986898in}}%
\pgfpathlineto{\pgfqpoint{2.462851in}{3.007186in}}%
\pgfpathlineto{\pgfqpoint{2.466213in}{3.007820in}}%
\pgfpathlineto{\pgfqpoint{2.467333in}{3.015904in}}%
\pgfpathlineto{\pgfqpoint{2.468454in}{3.012734in}}%
\pgfpathlineto{\pgfqpoint{2.469575in}{3.028267in}}%
\pgfpathlineto{\pgfqpoint{2.470695in}{3.023670in}}%
\pgfpathlineto{\pgfqpoint{2.474057in}{3.035717in}}%
\pgfpathlineto{\pgfqpoint{2.475177in}{3.027950in}}%
\pgfpathlineto{\pgfqpoint{2.477418in}{3.040313in}}%
\pgfpathlineto{\pgfqpoint{2.481900in}{3.027950in}}%
\pgfpathlineto{\pgfqpoint{2.483021in}{3.020659in}}%
\pgfpathlineto{\pgfqpoint{2.484141in}{3.020025in}}%
\pgfpathlineto{\pgfqpoint{2.485262in}{3.016538in}}%
\pgfpathlineto{\pgfqpoint{2.486382in}{3.010356in}}%
\pgfpathlineto{\pgfqpoint{2.489744in}{3.020183in}}%
\pgfpathlineto{\pgfqpoint{2.491985in}{2.997359in}}%
\pgfpathlineto{\pgfqpoint{2.494226in}{3.004808in}}%
\pgfpathlineto{\pgfqpoint{2.499829in}{2.988800in}}%
\pgfpathlineto{\pgfqpoint{2.500949in}{2.987373in}}%
\pgfpathlineto{\pgfqpoint{2.502070in}{2.941883in}}%
\pgfpathlineto{\pgfqpoint{2.505431in}{2.961854in}}%
\pgfpathlineto{\pgfqpoint{2.506552in}{2.936652in}}%
\pgfpathlineto{\pgfqpoint{2.507672in}{2.926825in}}%
\pgfpathlineto{\pgfqpoint{2.508793in}{2.942358in}}%
\pgfpathlineto{\pgfqpoint{2.509913in}{2.903683in}}%
\pgfpathlineto{\pgfqpoint{2.513275in}{2.908755in}}%
\pgfpathlineto{\pgfqpoint{2.514396in}{2.905744in}}%
\pgfpathlineto{\pgfqpoint{2.516637in}{2.946638in}}%
\pgfpathlineto{\pgfqpoint{2.517757in}{2.940139in}}%
\pgfpathlineto{\pgfqpoint{2.521119in}{2.935226in}}%
\pgfpathlineto{\pgfqpoint{2.522239in}{2.936969in}}%
\pgfpathlineto{\pgfqpoint{2.523360in}{2.936969in}}%
\pgfpathlineto{\pgfqpoint{2.524480in}{2.917949in}}%
\pgfpathlineto{\pgfqpoint{2.525601in}{2.925874in}}%
\pgfpathlineto{\pgfqpoint{2.528962in}{2.938237in}}%
\pgfpathlineto{\pgfqpoint{2.530083in}{2.923496in}}%
\pgfpathlineto{\pgfqpoint{2.531204in}{2.935543in}}%
\pgfpathlineto{\pgfqpoint{2.532324in}{2.933006in}}%
\pgfpathlineto{\pgfqpoint{2.533445in}{2.910816in}}%
\pgfpathlineto{\pgfqpoint{2.536806in}{2.901940in}}%
\pgfpathlineto{\pgfqpoint{2.537927in}{2.884029in}}%
\pgfpathlineto{\pgfqpoint{2.539047in}{2.886407in}}%
\pgfpathlineto{\pgfqpoint{2.540168in}{2.900038in}}%
\pgfpathlineto{\pgfqpoint{2.541288in}{2.904634in}}%
\pgfpathlineto{\pgfqpoint{2.544650in}{2.897977in}}%
\pgfpathlineto{\pgfqpoint{2.545770in}{2.902098in}}%
\pgfpathlineto{\pgfqpoint{2.546891in}{2.898770in}}%
\pgfpathlineto{\pgfqpoint{2.549132in}{2.882602in}}%
\pgfpathlineto{\pgfqpoint{2.552494in}{2.895124in}}%
\pgfpathlineto{\pgfqpoint{2.553614in}{2.922387in}}%
\pgfpathlineto{\pgfqpoint{2.554735in}{2.917156in}}%
\pgfpathlineto{\pgfqpoint{2.555855in}{2.903366in}}%
\pgfpathlineto{\pgfqpoint{2.556976in}{2.929519in}}%
\pgfpathlineto{\pgfqpoint{2.560337in}{2.934750in}}%
\pgfpathlineto{\pgfqpoint{2.561458in}{2.932055in}}%
\pgfpathlineto{\pgfqpoint{2.563699in}{2.917790in}}%
\pgfpathlineto{\pgfqpoint{2.564819in}{2.921911in}}%
\pgfpathlineto{\pgfqpoint{2.569301in}{2.952185in}}%
\pgfpathlineto{\pgfqpoint{2.570422in}{2.973425in}}%
\pgfpathlineto{\pgfqpoint{2.571542in}{3.026682in}}%
\pgfpathlineto{\pgfqpoint{2.572663in}{3.033339in}}%
\pgfpathlineto{\pgfqpoint{2.577145in}{3.015904in}}%
\pgfpathlineto{\pgfqpoint{2.578266in}{3.014953in}}%
\pgfpathlineto{\pgfqpoint{2.580507in}{3.008137in}}%
\pgfpathlineto{\pgfqpoint{2.584989in}{3.014636in}}%
\pgfpathlineto{\pgfqpoint{2.586109in}{3.032071in}}%
\pgfpathlineto{\pgfqpoint{2.588350in}{3.042057in}}%
\pgfpathlineto{\pgfqpoint{2.591712in}{3.037143in}}%
\pgfpathlineto{\pgfqpoint{2.592832in}{3.043008in}}%
\pgfpathlineto{\pgfqpoint{2.593953in}{3.026365in}}%
\pgfpathlineto{\pgfqpoint{2.595074in}{3.022719in}}%
\pgfpathlineto{\pgfqpoint{2.596194in}{3.033656in}}%
\pgfpathlineto{\pgfqpoint{2.599556in}{3.021768in}}%
\pgfpathlineto{\pgfqpoint{2.600676in}{3.021927in}}%
\pgfpathlineto{\pgfqpoint{2.601797in}{3.055529in}}%
\pgfpathlineto{\pgfqpoint{2.602917in}{3.036826in}}%
\pgfpathlineto{\pgfqpoint{2.604038in}{3.057114in}}%
\pgfpathlineto{\pgfqpoint{2.607399in}{3.065991in}}%
\pgfpathlineto{\pgfqpoint{2.608520in}{3.064881in}}%
\pgfpathlineto{\pgfqpoint{2.610761in}{3.032388in}}%
\pgfpathlineto{\pgfqpoint{2.611881in}{3.038253in}}%
\pgfpathlineto{\pgfqpoint{2.615243in}{3.072489in}}%
\pgfpathlineto{\pgfqpoint{2.617484in}{3.067576in}}%
\pgfpathlineto{\pgfqpoint{2.618605in}{3.067100in}}%
\pgfpathlineto{\pgfqpoint{2.619725in}{3.069953in}}%
\pgfpathlineto{\pgfqpoint{2.624207in}{3.074867in}}%
\pgfpathlineto{\pgfqpoint{2.625328in}{3.062979in}}%
\pgfpathlineto{\pgfqpoint{2.626448in}{3.068368in}}%
\pgfpathlineto{\pgfqpoint{2.627569in}{3.054578in}}%
\pgfpathlineto{\pgfqpoint{2.632051in}{3.072648in}}%
\pgfpathlineto{\pgfqpoint{2.633171in}{3.073599in}}%
\pgfpathlineto{\pgfqpoint{2.634292in}{3.080256in}}%
\pgfpathlineto{\pgfqpoint{2.635413in}{3.098167in}}%
\pgfpathlineto{\pgfqpoint{2.639895in}{3.078988in}}%
\pgfpathlineto{\pgfqpoint{2.642136in}{3.072489in}}%
\pgfpathlineto{\pgfqpoint{2.643256in}{3.063455in}}%
\pgfpathlineto{\pgfqpoint{2.647738in}{3.059809in}}%
\pgfpathlineto{\pgfqpoint{2.649979in}{3.071063in}}%
\pgfpathlineto{\pgfqpoint{2.651100in}{3.072014in}}%
\pgfpathlineto{\pgfqpoint{2.654461in}{3.065357in}}%
\pgfpathlineto{\pgfqpoint{2.655582in}{3.084218in}}%
\pgfpathlineto{\pgfqpoint{2.658944in}{3.055688in}}%
\pgfpathlineto{\pgfqpoint{2.662305in}{3.047921in}}%
\pgfpathlineto{\pgfqpoint{2.663426in}{3.055371in}}%
\pgfpathlineto{\pgfqpoint{2.664546in}{3.033180in}}%
\pgfpathlineto{\pgfqpoint{2.665667in}{3.035875in}}%
\pgfpathlineto{\pgfqpoint{2.666787in}{3.055054in}}%
\pgfpathlineto{\pgfqpoint{2.670149in}{3.070270in}}%
\pgfpathlineto{\pgfqpoint{2.671269in}{3.078829in}}%
\pgfpathlineto{\pgfqpoint{2.672390in}{3.066149in}}%
\pgfpathlineto{\pgfqpoint{2.673510in}{3.061553in}}%
\pgfpathlineto{\pgfqpoint{2.674631in}{3.075818in}}%
\pgfpathlineto{\pgfqpoint{2.677993in}{3.090242in}}%
\pgfpathlineto{\pgfqpoint{2.679113in}{3.083584in}}%
\pgfpathlineto{\pgfqpoint{2.680234in}{3.098484in}}%
\pgfpathlineto{\pgfqpoint{2.681354in}{3.101020in}}%
\pgfpathlineto{\pgfqpoint{2.682475in}{3.102129in}}%
\pgfpathlineto{\pgfqpoint{2.686957in}{3.107360in}}%
\pgfpathlineto{\pgfqpoint{2.688077in}{3.109738in}}%
\pgfpathlineto{\pgfqpoint{2.689198in}{3.101178in}}%
\pgfpathlineto{\pgfqpoint{2.690318in}{3.106726in}}%
\pgfpathlineto{\pgfqpoint{2.693680in}{3.111957in}}%
\pgfpathlineto{\pgfqpoint{2.694800in}{3.108152in}}%
\pgfpathlineto{\pgfqpoint{2.695921in}{3.125112in}}%
\pgfpathlineto{\pgfqpoint{2.697042in}{3.110213in}}%
\pgfpathlineto{\pgfqpoint{2.698162in}{3.104982in}}%
\pgfpathlineto{\pgfqpoint{2.701524in}{3.094838in}}%
\pgfpathlineto{\pgfqpoint{2.702644in}{3.104031in}}%
\pgfpathlineto{\pgfqpoint{2.703765in}{3.096106in}}%
\pgfpathlineto{\pgfqpoint{2.706006in}{3.101971in}}%
\pgfpathlineto{\pgfqpoint{2.709367in}{3.104507in}}%
\pgfpathlineto{\pgfqpoint{2.710488in}{3.097374in}}%
\pgfpathlineto{\pgfqpoint{2.711608in}{3.116078in}}%
\pgfpathlineto{\pgfqpoint{2.712729in}{3.104982in}}%
\pgfpathlineto{\pgfqpoint{2.713849in}{3.121625in}}%
\pgfpathlineto{\pgfqpoint{2.717211in}{3.122576in}}%
\pgfpathlineto{\pgfqpoint{2.718332in}{3.102288in}}%
\pgfpathlineto{\pgfqpoint{2.719452in}{3.098801in}}%
\pgfpathlineto{\pgfqpoint{2.721693in}{3.096899in}}%
\pgfpathlineto{\pgfqpoint{2.725055in}{3.097374in}}%
\pgfpathlineto{\pgfqpoint{2.726175in}{3.111006in}}%
\pgfpathlineto{\pgfqpoint{2.727296in}{3.100861in}}%
\pgfpathlineto{\pgfqpoint{2.728416in}{3.106567in}}%
\pgfpathlineto{\pgfqpoint{2.729537in}{3.103080in}}%
\pgfpathlineto{\pgfqpoint{2.732898in}{3.098801in}}%
\pgfpathlineto{\pgfqpoint{2.734019in}{3.114651in}}%
\pgfpathlineto{\pgfqpoint{2.735139in}{3.107360in}}%
\pgfpathlineto{\pgfqpoint{2.736260in}{3.106092in}}%
\pgfpathlineto{\pgfqpoint{2.737381in}{3.114651in}}%
\pgfpathlineto{\pgfqpoint{2.740742in}{3.113066in}}%
\pgfpathlineto{\pgfqpoint{2.741863in}{3.117029in}}%
\pgfpathlineto{\pgfqpoint{2.742983in}{3.103556in}}%
\pgfpathlineto{\pgfqpoint{2.744104in}{3.101178in}}%
\pgfpathlineto{\pgfqpoint{2.749706in}{3.123210in}}%
\pgfpathlineto{\pgfqpoint{2.750827in}{3.113700in}}%
\pgfpathlineto{\pgfqpoint{2.753068in}{3.144608in}}%
\pgfpathlineto{\pgfqpoint{2.757550in}{3.172822in}}%
\pgfpathlineto{\pgfqpoint{2.758671in}{3.192793in}}%
\pgfpathlineto{\pgfqpoint{2.759791in}{3.201352in}}%
\pgfpathlineto{\pgfqpoint{2.760912in}{3.204522in}}%
\pgfpathlineto{\pgfqpoint{2.764273in}{3.203571in}}%
\pgfpathlineto{\pgfqpoint{2.765394in}{3.209436in}}%
\pgfpathlineto{\pgfqpoint{2.767635in}{3.234638in}}%
\pgfpathlineto{\pgfqpoint{2.768755in}{3.239869in}}%
\pgfpathlineto{\pgfqpoint{2.772117in}{3.235906in}}%
\pgfpathlineto{\pgfqpoint{2.773237in}{3.241612in}}%
\pgfpathlineto{\pgfqpoint{2.774358in}{3.234955in}}%
\pgfpathlineto{\pgfqpoint{2.775478in}{3.238918in}}%
\pgfpathlineto{\pgfqpoint{2.776599in}{3.232102in}}%
\pgfpathlineto{\pgfqpoint{2.779961in}{3.233370in}}%
\pgfpathlineto{\pgfqpoint{2.781081in}{3.240661in}}%
\pgfpathlineto{\pgfqpoint{2.782202in}{3.223226in}}%
\pgfpathlineto{\pgfqpoint{2.783322in}{3.220056in}}%
\pgfpathlineto{\pgfqpoint{2.784443in}{3.247001in}}%
\pgfpathlineto{\pgfqpoint{2.787804in}{3.254451in}}%
\pgfpathlineto{\pgfqpoint{2.788925in}{3.260633in}}%
\pgfpathlineto{\pgfqpoint{2.790045in}{3.260791in}}%
\pgfpathlineto{\pgfqpoint{2.791166in}{3.263327in}}%
\pgfpathlineto{\pgfqpoint{2.792286in}{3.257304in}}%
\pgfpathlineto{\pgfqpoint{2.796768in}{3.248586in}}%
\pgfpathlineto{\pgfqpoint{2.797889in}{3.248745in}}%
\pgfpathlineto{\pgfqpoint{2.800130in}{3.261425in}}%
\pgfpathlineto{\pgfqpoint{2.803492in}{3.244307in}}%
\pgfpathlineto{\pgfqpoint{2.804612in}{3.231309in}}%
\pgfpathlineto{\pgfqpoint{2.805733in}{3.226237in}}%
\pgfpathlineto{\pgfqpoint{2.806853in}{3.228615in}}%
\pgfpathlineto{\pgfqpoint{2.807974in}{3.239076in}}%
\pgfpathlineto{\pgfqpoint{2.811335in}{3.227664in}}%
\pgfpathlineto{\pgfqpoint{2.812456in}{3.230358in}}%
\pgfpathlineto{\pgfqpoint{2.813576in}{3.229724in}}%
\pgfpathlineto{\pgfqpoint{2.814697in}{3.240820in}}%
\pgfpathlineto{\pgfqpoint{2.815817in}{3.235906in}}%
\pgfpathlineto{\pgfqpoint{2.819179in}{3.257304in}}%
\pgfpathlineto{\pgfqpoint{2.820300in}{3.252232in}}%
\pgfpathlineto{\pgfqpoint{2.821420in}{3.255877in}}%
\pgfpathlineto{\pgfqpoint{2.822541in}{3.262218in}}%
\pgfpathlineto{\pgfqpoint{2.823661in}{3.262852in}}%
\pgfpathlineto{\pgfqpoint{2.827023in}{3.257621in}}%
\pgfpathlineto{\pgfqpoint{2.828143in}{3.253500in}}%
\pgfpathlineto{\pgfqpoint{2.829264in}{3.267131in}}%
\pgfpathlineto{\pgfqpoint{2.830384in}{3.254292in}}%
\pgfpathlineto{\pgfqpoint{2.831505in}{3.261267in}}%
\pgfpathlineto{\pgfqpoint{2.834866in}{3.260791in}}%
\pgfpathlineto{\pgfqpoint{2.837107in}{3.271569in}}%
\pgfpathlineto{\pgfqpoint{2.838228in}{3.257938in}}%
\pgfpathlineto{\pgfqpoint{2.839348in}{3.268716in}}%
\pgfpathlineto{\pgfqpoint{2.842710in}{3.275532in}}%
\pgfpathlineto{\pgfqpoint{2.843831in}{3.282823in}}%
\pgfpathlineto{\pgfqpoint{2.844951in}{3.285359in}}%
\pgfpathlineto{\pgfqpoint{2.846072in}{3.275690in}}%
\pgfpathlineto{\pgfqpoint{2.847192in}{3.280762in}}%
\pgfpathlineto{\pgfqpoint{2.850554in}{3.276324in}}%
\pgfpathlineto{\pgfqpoint{2.851674in}{3.269033in}}%
\pgfpathlineto{\pgfqpoint{2.852795in}{3.275849in}}%
\pgfpathlineto{\pgfqpoint{2.853915in}{3.266339in}}%
\pgfpathlineto{\pgfqpoint{2.855036in}{3.282030in}}%
\pgfpathlineto{\pgfqpoint{2.858397in}{3.276641in}}%
\pgfpathlineto{\pgfqpoint{2.859518in}{3.235748in}}%
\pgfpathlineto{\pgfqpoint{2.861759in}{3.209753in}}%
\pgfpathlineto{\pgfqpoint{2.862880in}{3.212448in}}%
\pgfpathlineto{\pgfqpoint{2.866241in}{3.208009in}}%
\pgfpathlineto{\pgfqpoint{2.867362in}{3.212606in}}%
\pgfpathlineto{\pgfqpoint{2.869603in}{3.246050in}}%
\pgfpathlineto{\pgfqpoint{2.870723in}{3.251915in}}%
\pgfpathlineto{\pgfqpoint{2.874085in}{3.207217in}}%
\pgfpathlineto{\pgfqpoint{2.875205in}{3.202303in}}%
\pgfpathlineto{\pgfqpoint{2.876326in}{3.187404in}}%
\pgfpathlineto{\pgfqpoint{2.877446in}{3.180747in}}%
\pgfpathlineto{\pgfqpoint{2.878567in}{3.182649in}}%
\pgfpathlineto{\pgfqpoint{2.881929in}{3.186612in}}%
\pgfpathlineto{\pgfqpoint{2.883049in}{3.168701in}}%
\pgfpathlineto{\pgfqpoint{2.884170in}{3.210229in}}%
\pgfpathlineto{\pgfqpoint{2.885290in}{3.181381in}}%
\pgfpathlineto{\pgfqpoint{2.886411in}{3.172029in}}%
\pgfpathlineto{\pgfqpoint{2.889772in}{3.169018in}}%
\pgfpathlineto{\pgfqpoint{2.890893in}{3.175199in}}%
\pgfpathlineto{\pgfqpoint{2.892013in}{3.195488in}}%
\pgfpathlineto{\pgfqpoint{2.893134in}{3.169335in}}%
\pgfpathlineto{\pgfqpoint{2.894254in}{3.165848in}}%
\pgfpathlineto{\pgfqpoint{2.897616in}{3.169810in}}%
\pgfpathlineto{\pgfqpoint{2.898736in}{3.220848in}}%
\pgfpathlineto{\pgfqpoint{2.899857in}{3.234479in}}%
\pgfpathlineto{\pgfqpoint{2.900977in}{3.236223in}}%
\pgfpathlineto{\pgfqpoint{2.907701in}{3.083109in}}%
\pgfpathlineto{\pgfqpoint{2.908821in}{3.088657in}}%
\pgfpathlineto{\pgfqpoint{2.909942in}{3.082158in}}%
\pgfpathlineto{\pgfqpoint{2.913303in}{3.083584in}}%
\pgfpathlineto{\pgfqpoint{2.914424in}{3.086913in}}%
\pgfpathlineto{\pgfqpoint{2.915544in}{3.093253in}}%
\pgfpathlineto{\pgfqpoint{2.916665in}{3.136525in}}%
\pgfpathlineto{\pgfqpoint{2.917785in}{3.135732in}}%
\pgfpathlineto{\pgfqpoint{2.921147in}{3.130977in}}%
\pgfpathlineto{\pgfqpoint{2.923388in}{3.151107in}}%
\pgfpathlineto{\pgfqpoint{2.925629in}{3.164897in}}%
\pgfpathlineto{\pgfqpoint{2.928991in}{3.153960in}}%
\pgfpathlineto{\pgfqpoint{2.930111in}{3.160459in}}%
\pgfpathlineto{\pgfqpoint{2.931232in}{3.197707in}}%
\pgfpathlineto{\pgfqpoint{2.932352in}{3.177418in}}%
\pgfpathlineto{\pgfqpoint{2.933473in}{3.181064in}}%
\pgfpathlineto{\pgfqpoint{2.936834in}{3.203888in}}%
\pgfpathlineto{\pgfqpoint{2.937955in}{3.205790in}}%
\pgfpathlineto{\pgfqpoint{2.939075in}{3.204522in}}%
\pgfpathlineto{\pgfqpoint{2.940196in}{3.212765in}}%
\pgfpathlineto{\pgfqpoint{2.941316in}{3.213716in}}%
\pgfpathlineto{\pgfqpoint{2.944678in}{3.219739in}}%
\pgfpathlineto{\pgfqpoint{2.946919in}{3.206424in}}%
\pgfpathlineto{\pgfqpoint{2.948040in}{3.222592in}}%
\pgfpathlineto{\pgfqpoint{2.949160in}{3.221007in}}%
\pgfpathlineto{\pgfqpoint{2.952522in}{3.225445in}}%
\pgfpathlineto{\pgfqpoint{2.953642in}{3.230834in}}%
\pgfpathlineto{\pgfqpoint{2.954763in}{3.227981in}}%
\pgfpathlineto{\pgfqpoint{2.955883in}{3.232736in}}%
\pgfpathlineto{\pgfqpoint{2.957004in}{3.254609in}}%
\pgfpathlineto{\pgfqpoint{2.960365in}{3.253975in}}%
\pgfpathlineto{\pgfqpoint{2.962606in}{3.225128in}}%
\pgfpathlineto{\pgfqpoint{2.963727in}{3.239393in}}%
\pgfpathlineto{\pgfqpoint{2.964848in}{3.226396in}}%
\pgfpathlineto{\pgfqpoint{2.968209in}{3.237967in}}%
\pgfpathlineto{\pgfqpoint{2.969330in}{3.236857in}}%
\pgfpathlineto{\pgfqpoint{2.970450in}{3.242246in}}%
\pgfpathlineto{\pgfqpoint{2.971571in}{3.262535in}}%
\pgfpathlineto{\pgfqpoint{2.972691in}{3.256670in}}%
\pgfpathlineto{\pgfqpoint{2.976053in}{3.246367in}}%
\pgfpathlineto{\pgfqpoint{2.977173in}{3.251756in}}%
\pgfpathlineto{\pgfqpoint{2.978294in}{3.244782in}}%
\pgfpathlineto{\pgfqpoint{2.979414in}{3.215776in}}%
\pgfpathlineto{\pgfqpoint{2.980535in}{3.211814in}}%
\pgfpathlineto{\pgfqpoint{2.983896in}{3.196122in}}%
\pgfpathlineto{\pgfqpoint{2.985017in}{3.222275in}}%
\pgfpathlineto{\pgfqpoint{2.986138in}{3.203730in}}%
\pgfpathlineto{\pgfqpoint{2.987258in}{3.219263in}}%
\pgfpathlineto{\pgfqpoint{2.988379in}{3.198499in}}%
\pgfpathlineto{\pgfqpoint{2.991740in}{3.196122in}}%
\pgfpathlineto{\pgfqpoint{2.992861in}{3.206266in}}%
\pgfpathlineto{\pgfqpoint{2.993981in}{3.201511in}}%
\pgfpathlineto{\pgfqpoint{2.999584in}{3.207534in}}%
\pgfpathlineto{\pgfqpoint{3.001825in}{3.223384in}}%
\pgfpathlineto{\pgfqpoint{3.002945in}{3.273471in}}%
\pgfpathlineto{\pgfqpoint{3.004066in}{3.253183in}}%
\pgfpathlineto{\pgfqpoint{3.007428in}{3.251915in}}%
\pgfpathlineto{\pgfqpoint{3.008548in}{3.255719in}}%
\pgfpathlineto{\pgfqpoint{3.011910in}{3.294235in}}%
\pgfpathlineto{\pgfqpoint{3.016392in}{3.304379in}}%
\pgfpathlineto{\pgfqpoint{3.017512in}{3.316901in}}%
\pgfpathlineto{\pgfqpoint{3.018633in}{3.308659in}}%
\pgfpathlineto{\pgfqpoint{3.019753in}{3.345432in}}%
\pgfpathlineto{\pgfqpoint{3.023115in}{3.353515in}}%
\pgfpathlineto{\pgfqpoint{3.024235in}{3.354625in}}%
\pgfpathlineto{\pgfqpoint{3.026477in}{3.362709in}}%
\pgfpathlineto{\pgfqpoint{3.027597in}{3.361441in}}%
\pgfpathlineto{\pgfqpoint{3.032079in}{3.360173in}}%
\pgfpathlineto{\pgfqpoint{3.034320in}{3.375072in}}%
\pgfpathlineto{\pgfqpoint{3.035441in}{3.366671in}}%
\pgfpathlineto{\pgfqpoint{3.035441in}{3.366671in}}%
\pgfusepath{stroke}%
\end{pgfscope}%
\begin{pgfscope}%
\pgfpathrectangle{\pgfqpoint{0.462318in}{2.320415in}}{\pgfqpoint{2.695652in}{1.104878in}}%
\pgfusepath{clip}%
\pgfsetroundcap%
\pgfsetroundjoin%
\pgfsetlinewidth{1.505625pt}%
\definecolor{currentstroke}{rgb}{1.000000,0.647059,0.000000}%
\pgfsetstrokecolor{currentstroke}%
\pgfsetdash{}{0pt}%
\pgfpathmoveto{\pgfqpoint{0.584848in}{2.409312in}}%
\pgfpathlineto{\pgfqpoint{0.585968in}{2.411848in}}%
\pgfpathlineto{\pgfqpoint{0.588209in}{2.406934in}}%
\pgfpathlineto{\pgfqpoint{0.591571in}{2.405920in}}%
\pgfpathlineto{\pgfqpoint{0.594933in}{2.417079in}}%
\pgfpathlineto{\pgfqpoint{0.596053in}{2.418294in}}%
\pgfpathlineto{\pgfqpoint{0.600535in}{2.420534in}}%
\pgfpathlineto{\pgfqpoint{0.602776in}{2.424674in}}%
\pgfpathlineto{\pgfqpoint{0.603897in}{2.425552in}}%
\pgfpathlineto{\pgfqpoint{0.607258in}{2.426464in}}%
\pgfpathlineto{\pgfqpoint{0.610620in}{2.430318in}}%
\pgfpathlineto{\pgfqpoint{0.611740in}{2.431318in}}%
\pgfpathlineto{\pgfqpoint{0.615102in}{2.432203in}}%
\pgfpathlineto{\pgfqpoint{0.616223in}{2.433492in}}%
\pgfpathlineto{\pgfqpoint{0.619584in}{2.440268in}}%
\pgfpathlineto{\pgfqpoint{0.622946in}{2.442228in}}%
\pgfpathlineto{\pgfqpoint{0.625187in}{2.446146in}}%
\pgfpathlineto{\pgfqpoint{0.627428in}{2.451983in}}%
\pgfpathlineto{\pgfqpoint{0.630789in}{2.455163in}}%
\pgfpathlineto{\pgfqpoint{0.633031in}{2.460038in}}%
\pgfpathlineto{\pgfqpoint{0.635272in}{2.464418in}}%
\pgfpathlineto{\pgfqpoint{0.639754in}{2.466597in}}%
\pgfpathlineto{\pgfqpoint{0.643115in}{2.472251in}}%
\pgfpathlineto{\pgfqpoint{0.646477in}{2.473877in}}%
\pgfpathlineto{\pgfqpoint{0.650959in}{2.480084in}}%
\pgfpathlineto{\pgfqpoint{0.655441in}{2.481755in}}%
\pgfpathlineto{\pgfqpoint{0.658803in}{2.484743in}}%
\pgfpathlineto{\pgfqpoint{0.662164in}{2.485869in}}%
\pgfpathlineto{\pgfqpoint{0.666646in}{2.492441in}}%
\pgfpathlineto{\pgfqpoint{0.685695in}{2.497734in}}%
\pgfpathlineto{\pgfqpoint{0.689057in}{2.498345in}}%
\pgfpathlineto{\pgfqpoint{0.694659in}{2.497724in}}%
\pgfpathlineto{\pgfqpoint{0.698021in}{2.497335in}}%
\pgfpathlineto{\pgfqpoint{0.713708in}{2.496958in}}%
\pgfpathlineto{\pgfqpoint{0.721552in}{2.497086in}}%
\pgfpathlineto{\pgfqpoint{0.726034in}{2.496487in}}%
\pgfpathlineto{\pgfqpoint{0.729396in}{2.494892in}}%
\pgfpathlineto{\pgfqpoint{0.733878in}{2.493571in}}%
\pgfpathlineto{\pgfqpoint{0.737240in}{2.490812in}}%
\pgfpathlineto{\pgfqpoint{0.741722in}{2.489162in}}%
\pgfpathlineto{\pgfqpoint{0.745083in}{2.486691in}}%
\pgfpathlineto{\pgfqpoint{0.749565in}{2.486105in}}%
\pgfpathlineto{\pgfqpoint{0.752927in}{2.483725in}}%
\pgfpathlineto{\pgfqpoint{0.756288in}{2.482730in}}%
\pgfpathlineto{\pgfqpoint{0.758530in}{2.480997in}}%
\pgfpathlineto{\pgfqpoint{0.760771in}{2.480101in}}%
\pgfpathlineto{\pgfqpoint{0.765253in}{2.479007in}}%
\pgfpathlineto{\pgfqpoint{0.768614in}{2.477286in}}%
\pgfpathlineto{\pgfqpoint{0.774217in}{2.476237in}}%
\pgfpathlineto{\pgfqpoint{0.776458in}{2.475396in}}%
\pgfpathlineto{\pgfqpoint{0.780940in}{2.474294in}}%
\pgfpathlineto{\pgfqpoint{0.784302in}{2.472839in}}%
\pgfpathlineto{\pgfqpoint{0.791025in}{2.471858in}}%
\pgfpathlineto{\pgfqpoint{0.792145in}{2.471406in}}%
\pgfpathlineto{\pgfqpoint{0.796627in}{2.470554in}}%
\pgfpathlineto{\pgfqpoint{0.799989in}{2.468845in}}%
\pgfpathlineto{\pgfqpoint{0.806712in}{2.467352in}}%
\pgfpathlineto{\pgfqpoint{0.807833in}{2.466979in}}%
\pgfpathlineto{\pgfqpoint{0.812315in}{2.465939in}}%
\pgfpathlineto{\pgfqpoint{0.815676in}{2.464621in}}%
\pgfpathlineto{\pgfqpoint{0.821279in}{2.463754in}}%
\pgfpathlineto{\pgfqpoint{0.823520in}{2.463355in}}%
\pgfpathlineto{\pgfqpoint{0.838087in}{2.463350in}}%
\pgfpathlineto{\pgfqpoint{0.842569in}{2.463845in}}%
\pgfpathlineto{\pgfqpoint{0.854895in}{2.465647in}}%
\pgfpathlineto{\pgfqpoint{0.868341in}{2.466355in}}%
\pgfpathlineto{\pgfqpoint{0.878426in}{2.468377in}}%
\pgfpathlineto{\pgfqpoint{0.899716in}{2.468963in}}%
\pgfpathlineto{\pgfqpoint{0.907560in}{2.468682in}}%
\pgfpathlineto{\pgfqpoint{0.914283in}{2.468775in}}%
\pgfpathlineto{\pgfqpoint{0.938934in}{2.468641in}}%
\pgfpathlineto{\pgfqpoint{0.945658in}{2.468326in}}%
\pgfpathlineto{\pgfqpoint{0.955742in}{2.468938in}}%
\pgfpathlineto{\pgfqpoint{0.964707in}{2.470046in}}%
\pgfpathlineto{\pgfqpoint{0.975912in}{2.471178in}}%
\pgfpathlineto{\pgfqpoint{0.983756in}{2.472644in}}%
\pgfpathlineto{\pgfqpoint{0.994961in}{2.474298in}}%
\pgfpathlineto{\pgfqpoint{0.996081in}{2.474695in}}%
\pgfpathlineto{\pgfqpoint{1.001684in}{2.475751in}}%
\pgfpathlineto{\pgfqpoint{1.003925in}{2.476530in}}%
\pgfpathlineto{\pgfqpoint{1.009528in}{2.477785in}}%
\pgfpathlineto{\pgfqpoint{1.011769in}{2.478700in}}%
\pgfpathlineto{\pgfqpoint{1.017371in}{2.479712in}}%
\pgfpathlineto{\pgfqpoint{1.019612in}{2.480870in}}%
\pgfpathlineto{\pgfqpoint{1.024094in}{2.482054in}}%
\pgfpathlineto{\pgfqpoint{1.027456in}{2.483656in}}%
\pgfpathlineto{\pgfqpoint{1.031938in}{2.484759in}}%
\pgfpathlineto{\pgfqpoint{1.035300in}{2.486445in}}%
\pgfpathlineto{\pgfqpoint{1.039782in}{2.487547in}}%
\pgfpathlineto{\pgfqpoint{1.043143in}{2.489278in}}%
\pgfpathlineto{\pgfqpoint{1.048746in}{2.490468in}}%
\pgfpathlineto{\pgfqpoint{1.050987in}{2.491582in}}%
\pgfpathlineto{\pgfqpoint{1.055469in}{2.492569in}}%
\pgfpathlineto{\pgfqpoint{1.058831in}{2.494254in}}%
\pgfpathlineto{\pgfqpoint{1.063313in}{2.495327in}}%
\pgfpathlineto{\pgfqpoint{1.066675in}{2.497122in}}%
\pgfpathlineto{\pgfqpoint{1.071157in}{2.498427in}}%
\pgfpathlineto{\pgfqpoint{1.074518in}{2.500385in}}%
\pgfpathlineto{\pgfqpoint{1.079000in}{2.501640in}}%
\pgfpathlineto{\pgfqpoint{1.082362in}{2.503520in}}%
\pgfpathlineto{\pgfqpoint{1.086844in}{2.504715in}}%
\pgfpathlineto{\pgfqpoint{1.089085in}{2.505920in}}%
\pgfpathlineto{\pgfqpoint{1.094688in}{2.507104in}}%
\pgfpathlineto{\pgfqpoint{1.098049in}{2.508859in}}%
\pgfpathlineto{\pgfqpoint{1.102531in}{2.510103in}}%
\pgfpathlineto{\pgfqpoint{1.105893in}{2.512128in}}%
\pgfpathlineto{\pgfqpoint{1.110375in}{2.513326in}}%
\pgfpathlineto{\pgfqpoint{1.113737in}{2.514911in}}%
\pgfpathlineto{\pgfqpoint{1.118219in}{2.516011in}}%
\pgfpathlineto{\pgfqpoint{1.121580in}{2.517450in}}%
\pgfpathlineto{\pgfqpoint{1.126062in}{2.518370in}}%
\pgfpathlineto{\pgfqpoint{1.129424in}{2.519792in}}%
\pgfpathlineto{\pgfqpoint{1.133906in}{2.520865in}}%
\pgfpathlineto{\pgfqpoint{1.137268in}{2.522584in}}%
\pgfpathlineto{\pgfqpoint{1.141750in}{2.523766in}}%
\pgfpathlineto{\pgfqpoint{1.145111in}{2.525655in}}%
\pgfpathlineto{\pgfqpoint{1.149594in}{2.526997in}}%
\pgfpathlineto{\pgfqpoint{1.152955in}{2.528771in}}%
\pgfpathlineto{\pgfqpoint{1.158558in}{2.529933in}}%
\pgfpathlineto{\pgfqpoint{1.160799in}{2.531073in}}%
\pgfpathlineto{\pgfqpoint{1.165281in}{2.532148in}}%
\pgfpathlineto{\pgfqpoint{1.168642in}{2.533585in}}%
\pgfpathlineto{\pgfqpoint{1.173125in}{2.534566in}}%
\pgfpathlineto{\pgfqpoint{1.176486in}{2.535991in}}%
\pgfpathlineto{\pgfqpoint{1.180968in}{2.537072in}}%
\pgfpathlineto{\pgfqpoint{1.184330in}{2.538372in}}%
\pgfpathlineto{\pgfqpoint{1.188812in}{2.539134in}}%
\pgfpathlineto{\pgfqpoint{1.192174in}{2.540427in}}%
\pgfpathlineto{\pgfqpoint{1.196656in}{2.541359in}}%
\pgfpathlineto{\pgfqpoint{1.200017in}{2.542381in}}%
\pgfpathlineto{\pgfqpoint{1.204499in}{2.543527in}}%
\pgfpathlineto{\pgfqpoint{1.207861in}{2.545395in}}%
\pgfpathlineto{\pgfqpoint{1.212343in}{2.546696in}}%
\pgfpathlineto{\pgfqpoint{1.215705in}{2.548777in}}%
\pgfpathlineto{\pgfqpoint{1.219066in}{2.549483in}}%
\pgfpathlineto{\pgfqpoint{1.223548in}{2.552668in}}%
\pgfpathlineto{\pgfqpoint{1.228030in}{2.554261in}}%
\pgfpathlineto{\pgfqpoint{1.231392in}{2.556778in}}%
\pgfpathlineto{\pgfqpoint{1.235874in}{2.558372in}}%
\pgfpathlineto{\pgfqpoint{1.239236in}{2.560738in}}%
\pgfpathlineto{\pgfqpoint{1.243718in}{2.562309in}}%
\pgfpathlineto{\pgfqpoint{1.247079in}{2.564442in}}%
\pgfpathlineto{\pgfqpoint{1.251561in}{2.565766in}}%
\pgfpathlineto{\pgfqpoint{1.254923in}{2.567725in}}%
\pgfpathlineto{\pgfqpoint{1.259405in}{2.568925in}}%
\pgfpathlineto{\pgfqpoint{1.262767in}{2.570611in}}%
\pgfpathlineto{\pgfqpoint{1.268369in}{2.571906in}}%
\pgfpathlineto{\pgfqpoint{1.270610in}{2.573219in}}%
\pgfpathlineto{\pgfqpoint{1.275093in}{2.574644in}}%
\pgfpathlineto{\pgfqpoint{1.278454in}{2.577012in}}%
\pgfpathlineto{\pgfqpoint{1.281816in}{2.577849in}}%
\pgfpathlineto{\pgfqpoint{1.286298in}{2.581288in}}%
\pgfpathlineto{\pgfqpoint{1.290780in}{2.582919in}}%
\pgfpathlineto{\pgfqpoint{1.294142in}{2.585310in}}%
\pgfpathlineto{\pgfqpoint{1.298624in}{2.586771in}}%
\pgfpathlineto{\pgfqpoint{1.301985in}{2.588630in}}%
\pgfpathlineto{\pgfqpoint{1.306467in}{2.589799in}}%
\pgfpathlineto{\pgfqpoint{1.309829in}{2.591704in}}%
\pgfpathlineto{\pgfqpoint{1.314311in}{2.593027in}}%
\pgfpathlineto{\pgfqpoint{1.317673in}{2.595086in}}%
\pgfpathlineto{\pgfqpoint{1.322155in}{2.596411in}}%
\pgfpathlineto{\pgfqpoint{1.325516in}{2.598389in}}%
\pgfpathlineto{\pgfqpoint{1.329998in}{2.599645in}}%
\pgfpathlineto{\pgfqpoint{1.333360in}{2.601542in}}%
\pgfpathlineto{\pgfqpoint{1.337842in}{2.602869in}}%
\pgfpathlineto{\pgfqpoint{1.341204in}{2.604866in}}%
\pgfpathlineto{\pgfqpoint{1.345686in}{2.606160in}}%
\pgfpathlineto{\pgfqpoint{1.349047in}{2.608175in}}%
\pgfpathlineto{\pgfqpoint{1.353529in}{2.609554in}}%
\pgfpathlineto{\pgfqpoint{1.356891in}{2.611645in}}%
\pgfpathlineto{\pgfqpoint{1.361373in}{2.613083in}}%
\pgfpathlineto{\pgfqpoint{1.362494in}{2.613823in}}%
\pgfpathlineto{\pgfqpoint{1.371458in}{2.617229in}}%
\pgfpathlineto{\pgfqpoint{1.372578in}{2.617941in}}%
\pgfpathlineto{\pgfqpoint{1.377061in}{2.619352in}}%
\pgfpathlineto{\pgfqpoint{1.380422in}{2.621190in}}%
\pgfpathlineto{\pgfqpoint{1.384904in}{2.622406in}}%
\pgfpathlineto{\pgfqpoint{1.388266in}{2.624371in}}%
\pgfpathlineto{\pgfqpoint{1.392748in}{2.625732in}}%
\pgfpathlineto{\pgfqpoint{1.402833in}{2.629330in}}%
\pgfpathlineto{\pgfqpoint{1.403953in}{2.630041in}}%
\pgfpathlineto{\pgfqpoint{1.408435in}{2.631468in}}%
\pgfpathlineto{\pgfqpoint{1.411797in}{2.633635in}}%
\pgfpathlineto{\pgfqpoint{1.416279in}{2.635010in}}%
\pgfpathlineto{\pgfqpoint{1.419641in}{2.637171in}}%
\pgfpathlineto{\pgfqpoint{1.424123in}{2.637907in}}%
\pgfpathlineto{\pgfqpoint{1.427484in}{2.640041in}}%
\pgfpathlineto{\pgfqpoint{1.431966in}{2.641427in}}%
\pgfpathlineto{\pgfqpoint{1.435328in}{2.643500in}}%
\pgfpathlineto{\pgfqpoint{1.439810in}{2.644638in}}%
\pgfpathlineto{\pgfqpoint{1.443172in}{2.646308in}}%
\pgfpathlineto{\pgfqpoint{1.447654in}{2.647564in}}%
\pgfpathlineto{\pgfqpoint{1.451015in}{2.649557in}}%
\pgfpathlineto{\pgfqpoint{1.456618in}{2.650903in}}%
\pgfpathlineto{\pgfqpoint{1.458859in}{2.652308in}}%
\pgfpathlineto{\pgfqpoint{1.463341in}{2.653753in}}%
\pgfpathlineto{\pgfqpoint{1.466703in}{2.655930in}}%
\pgfpathlineto{\pgfqpoint{1.471185in}{2.657399in}}%
\pgfpathlineto{\pgfqpoint{1.474546in}{2.659632in}}%
\pgfpathlineto{\pgfqpoint{1.479029in}{2.661049in}}%
\pgfpathlineto{\pgfqpoint{1.482390in}{2.662935in}}%
\pgfpathlineto{\pgfqpoint{1.486872in}{2.664219in}}%
\pgfpathlineto{\pgfqpoint{1.490234in}{2.666097in}}%
\pgfpathlineto{\pgfqpoint{1.494716in}{2.667355in}}%
\pgfpathlineto{\pgfqpoint{1.498077in}{2.669220in}}%
\pgfpathlineto{\pgfqpoint{1.502560in}{2.670599in}}%
\pgfpathlineto{\pgfqpoint{1.505921in}{2.672790in}}%
\pgfpathlineto{\pgfqpoint{1.510403in}{2.674076in}}%
\pgfpathlineto{\pgfqpoint{1.513765in}{2.675930in}}%
\pgfpathlineto{\pgfqpoint{1.518247in}{2.677166in}}%
\pgfpathlineto{\pgfqpoint{1.520488in}{2.678534in}}%
\pgfpathlineto{\pgfqpoint{1.526091in}{2.679914in}}%
\pgfpathlineto{\pgfqpoint{1.529452in}{2.681956in}}%
\pgfpathlineto{\pgfqpoint{1.533934in}{2.683259in}}%
\pgfpathlineto{\pgfqpoint{1.537296in}{2.685167in}}%
\pgfpathlineto{\pgfqpoint{1.541778in}{2.686391in}}%
\pgfpathlineto{\pgfqpoint{1.545140in}{2.688250in}}%
\pgfpathlineto{\pgfqpoint{1.549622in}{2.689559in}}%
\pgfpathlineto{\pgfqpoint{1.552983in}{2.691349in}}%
\pgfpathlineto{\pgfqpoint{1.557465in}{2.692448in}}%
\pgfpathlineto{\pgfqpoint{1.560827in}{2.694158in}}%
\pgfpathlineto{\pgfqpoint{1.566430in}{2.695324in}}%
\pgfpathlineto{\pgfqpoint{1.568671in}{2.696500in}}%
\pgfpathlineto{\pgfqpoint{1.573153in}{2.697730in}}%
\pgfpathlineto{\pgfqpoint{1.576514in}{2.699590in}}%
\pgfpathlineto{\pgfqpoint{1.580996in}{2.700891in}}%
\pgfpathlineto{\pgfqpoint{1.584358in}{2.702669in}}%
\pgfpathlineto{\pgfqpoint{1.588840in}{2.703829in}}%
\pgfpathlineto{\pgfqpoint{1.592202in}{2.705598in}}%
\pgfpathlineto{\pgfqpoint{1.596684in}{2.706714in}}%
\pgfpathlineto{\pgfqpoint{1.600045in}{2.708358in}}%
\pgfpathlineto{\pgfqpoint{1.604528in}{2.709425in}}%
\pgfpathlineto{\pgfqpoint{1.606769in}{2.710469in}}%
\pgfpathlineto{\pgfqpoint{1.612371in}{2.711473in}}%
\pgfpathlineto{\pgfqpoint{1.615733in}{2.712917in}}%
\pgfpathlineto{\pgfqpoint{1.620215in}{2.713918in}}%
\pgfpathlineto{\pgfqpoint{1.623577in}{2.715322in}}%
\pgfpathlineto{\pgfqpoint{1.629179in}{2.716558in}}%
\pgfpathlineto{\pgfqpoint{1.631420in}{2.717276in}}%
\pgfpathlineto{\pgfqpoint{1.638143in}{2.718531in}}%
\pgfpathlineto{\pgfqpoint{1.647108in}{2.720107in}}%
\pgfpathlineto{\pgfqpoint{1.652710in}{2.720935in}}%
\pgfpathlineto{\pgfqpoint{1.662795in}{2.723308in}}%
\pgfpathlineto{\pgfqpoint{1.668398in}{2.724396in}}%
\pgfpathlineto{\pgfqpoint{1.670639in}{2.725074in}}%
\pgfpathlineto{\pgfqpoint{1.677362in}{2.726123in}}%
\pgfpathlineto{\pgfqpoint{1.678482in}{2.726460in}}%
\pgfpathlineto{\pgfqpoint{1.684085in}{2.727461in}}%
\pgfpathlineto{\pgfqpoint{1.686326in}{2.728117in}}%
\pgfpathlineto{\pgfqpoint{1.691929in}{2.729085in}}%
\pgfpathlineto{\pgfqpoint{1.694170in}{2.729733in}}%
\pgfpathlineto{\pgfqpoint{1.700893in}{2.730782in}}%
\pgfpathlineto{\pgfqpoint{1.707616in}{2.731765in}}%
\pgfpathlineto{\pgfqpoint{1.716580in}{2.732940in}}%
\pgfpathlineto{\pgfqpoint{1.753558in}{2.737962in}}%
\pgfpathlineto{\pgfqpoint{1.756919in}{2.738845in}}%
\pgfpathlineto{\pgfqpoint{1.762522in}{2.739742in}}%
\pgfpathlineto{\pgfqpoint{1.764763in}{2.740385in}}%
\pgfpathlineto{\pgfqpoint{1.770366in}{2.741383in}}%
\pgfpathlineto{\pgfqpoint{1.783812in}{2.743764in}}%
\pgfpathlineto{\pgfqpoint{1.788294in}{2.745319in}}%
\pgfpathlineto{\pgfqpoint{1.793897in}{2.746437in}}%
\pgfpathlineto{\pgfqpoint{1.796138in}{2.747301in}}%
\pgfpathlineto{\pgfqpoint{1.800620in}{2.748195in}}%
\pgfpathlineto{\pgfqpoint{1.803981in}{2.749079in}}%
\pgfpathlineto{\pgfqpoint{1.809584in}{2.750338in}}%
\pgfpathlineto{\pgfqpoint{1.825271in}{2.753670in}}%
\pgfpathlineto{\pgfqpoint{1.827512in}{2.754442in}}%
\pgfpathlineto{\pgfqpoint{1.833115in}{2.755336in}}%
\pgfpathlineto{\pgfqpoint{1.835356in}{2.756298in}}%
\pgfpathlineto{\pgfqpoint{1.839838in}{2.757208in}}%
\pgfpathlineto{\pgfqpoint{1.843200in}{2.758404in}}%
\pgfpathlineto{\pgfqpoint{1.847682in}{2.759258in}}%
\pgfpathlineto{\pgfqpoint{1.851044in}{2.760613in}}%
\pgfpathlineto{\pgfqpoint{1.855526in}{2.761515in}}%
\pgfpathlineto{\pgfqpoint{1.858887in}{2.762920in}}%
\pgfpathlineto{\pgfqpoint{1.864490in}{2.763916in}}%
\pgfpathlineto{\pgfqpoint{1.866731in}{2.764975in}}%
\pgfpathlineto{\pgfqpoint{1.871213in}{2.766037in}}%
\pgfpathlineto{\pgfqpoint{1.874575in}{2.767560in}}%
\pgfpathlineto{\pgfqpoint{1.879057in}{2.768562in}}%
\pgfpathlineto{\pgfqpoint{1.882418in}{2.769947in}}%
\pgfpathlineto{\pgfqpoint{1.888021in}{2.771255in}}%
\pgfpathlineto{\pgfqpoint{1.890262in}{2.772151in}}%
\pgfpathlineto{\pgfqpoint{1.894744in}{2.773041in}}%
\pgfpathlineto{\pgfqpoint{1.898106in}{2.774355in}}%
\pgfpathlineto{\pgfqpoint{1.903708in}{2.775570in}}%
\pgfpathlineto{\pgfqpoint{1.905949in}{2.776321in}}%
\pgfpathlineto{\pgfqpoint{1.911552in}{2.777485in}}%
\pgfpathlineto{\pgfqpoint{1.912673in}{2.777868in}}%
\pgfpathlineto{\pgfqpoint{1.918275in}{2.778649in}}%
\pgfpathlineto{\pgfqpoint{1.921637in}{2.779834in}}%
\pgfpathlineto{\pgfqpoint{1.927239in}{2.780990in}}%
\pgfpathlineto{\pgfqpoint{1.929480in}{2.781709in}}%
\pgfpathlineto{\pgfqpoint{1.935083in}{2.782807in}}%
\pgfpathlineto{\pgfqpoint{1.937324in}{2.783532in}}%
\pgfpathlineto{\pgfqpoint{1.942927in}{2.784562in}}%
\pgfpathlineto{\pgfqpoint{1.953011in}{2.786939in}}%
\pgfpathlineto{\pgfqpoint{1.958614in}{2.788044in}}%
\pgfpathlineto{\pgfqpoint{1.960855in}{2.788836in}}%
\pgfpathlineto{\pgfqpoint{1.966458in}{2.790017in}}%
\pgfpathlineto{\pgfqpoint{1.968699in}{2.790802in}}%
\pgfpathlineto{\pgfqpoint{1.974302in}{2.791528in}}%
\pgfpathlineto{\pgfqpoint{1.976543in}{2.792258in}}%
\pgfpathlineto{\pgfqpoint{1.982145in}{2.793354in}}%
\pgfpathlineto{\pgfqpoint{1.984386in}{2.794063in}}%
\pgfpathlineto{\pgfqpoint{1.989989in}{2.795099in}}%
\pgfpathlineto{\pgfqpoint{1.992230in}{2.795829in}}%
\pgfpathlineto{\pgfqpoint{1.997833in}{2.796766in}}%
\pgfpathlineto{\pgfqpoint{2.000074in}{2.797403in}}%
\pgfpathlineto{\pgfqpoint{2.005676in}{2.798326in}}%
\pgfpathlineto{\pgfqpoint{2.007917in}{2.798892in}}%
\pgfpathlineto{\pgfqpoint{2.020243in}{2.800279in}}%
\pgfpathlineto{\pgfqpoint{2.034810in}{2.802342in}}%
\pgfpathlineto{\pgfqpoint{2.047136in}{2.802907in}}%
\pgfpathlineto{\pgfqpoint{2.082993in}{2.802930in}}%
\pgfpathlineto{\pgfqpoint{2.102042in}{2.802244in}}%
\pgfpathlineto{\pgfqpoint{2.113247in}{2.801527in}}%
\pgfpathlineto{\pgfqpoint{2.123332in}{2.800972in}}%
\pgfpathlineto{\pgfqpoint{2.146863in}{2.801083in}}%
\pgfpathlineto{\pgfqpoint{2.180479in}{2.802101in}}%
\pgfpathlineto{\pgfqpoint{2.232023in}{2.801800in}}%
\pgfpathlineto{\pgfqpoint{2.274603in}{2.798772in}}%
\pgfpathlineto{\pgfqpoint{2.298134in}{2.799107in}}%
\pgfpathlineto{\pgfqpoint{2.319424in}{2.799836in}}%
\pgfpathlineto{\pgfqpoint{2.392258in}{2.804570in}}%
\pgfpathlineto{\pgfqpoint{2.411307in}{2.805651in}}%
\pgfpathlineto{\pgfqpoint{2.423633in}{2.806428in}}%
\pgfpathlineto{\pgfqpoint{2.437079in}{2.807235in}}%
\pgfpathlineto{\pgfqpoint{2.455008in}{2.809081in}}%
\pgfpathlineto{\pgfqpoint{2.467333in}{2.810221in}}%
\pgfpathlineto{\pgfqpoint{2.478539in}{2.811730in}}%
\pgfpathlineto{\pgfqpoint{2.490865in}{2.812961in}}%
\pgfpathlineto{\pgfqpoint{2.502070in}{2.813997in}}%
\pgfpathlineto{\pgfqpoint{2.524480in}{2.815366in}}%
\pgfpathlineto{\pgfqpoint{2.541288in}{2.816259in}}%
\pgfpathlineto{\pgfqpoint{2.561458in}{2.817173in}}%
\pgfpathlineto{\pgfqpoint{2.600676in}{2.820769in}}%
\pgfpathlineto{\pgfqpoint{2.611881in}{2.822238in}}%
\pgfpathlineto{\pgfqpoint{2.624207in}{2.823424in}}%
\pgfpathlineto{\pgfqpoint{2.627569in}{2.823993in}}%
\pgfpathlineto{\pgfqpoint{2.638774in}{2.825012in}}%
\pgfpathlineto{\pgfqpoint{2.643256in}{2.825793in}}%
\pgfpathlineto{\pgfqpoint{2.654461in}{2.826741in}}%
\pgfpathlineto{\pgfqpoint{2.664546in}{2.828015in}}%
\pgfpathlineto{\pgfqpoint{2.682475in}{2.830325in}}%
\pgfpathlineto{\pgfqpoint{2.693680in}{2.831394in}}%
\pgfpathlineto{\pgfqpoint{2.706006in}{2.833281in}}%
\pgfpathlineto{\pgfqpoint{2.712729in}{2.834113in}}%
\pgfpathlineto{\pgfqpoint{2.721693in}{2.835356in}}%
\pgfpathlineto{\pgfqpoint{2.732898in}{2.836572in}}%
\pgfpathlineto{\pgfqpoint{2.744104in}{2.838216in}}%
\pgfpathlineto{\pgfqpoint{2.751947in}{2.839065in}}%
\pgfpathlineto{\pgfqpoint{2.760912in}{2.840593in}}%
\pgfpathlineto{\pgfqpoint{2.766514in}{2.841424in}}%
\pgfpathlineto{\pgfqpoint{2.768755in}{2.842013in}}%
\pgfpathlineto{\pgfqpoint{2.774358in}{2.842894in}}%
\pgfpathlineto{\pgfqpoint{2.776599in}{2.843476in}}%
\pgfpathlineto{\pgfqpoint{2.782202in}{2.844339in}}%
\pgfpathlineto{\pgfqpoint{2.792286in}{2.846439in}}%
\pgfpathlineto{\pgfqpoint{2.799009in}{2.847331in}}%
\pgfpathlineto{\pgfqpoint{2.800130in}{2.847634in}}%
\pgfpathlineto{\pgfqpoint{2.806853in}{2.848761in}}%
\pgfpathlineto{\pgfqpoint{2.807974in}{2.849046in}}%
\pgfpathlineto{\pgfqpoint{2.813576in}{2.849878in}}%
\pgfpathlineto{\pgfqpoint{2.823661in}{2.851923in}}%
\pgfpathlineto{\pgfqpoint{2.829264in}{2.852808in}}%
\pgfpathlineto{\pgfqpoint{2.834866in}{2.853688in}}%
\pgfpathlineto{\pgfqpoint{2.844951in}{2.855501in}}%
\pgfpathlineto{\pgfqpoint{2.847192in}{2.856109in}}%
\pgfpathlineto{\pgfqpoint{2.852795in}{2.857007in}}%
\pgfpathlineto{\pgfqpoint{2.858397in}{2.857904in}}%
\pgfpathlineto{\pgfqpoint{2.862880in}{2.858937in}}%
\pgfpathlineto{\pgfqpoint{2.869603in}{2.859980in}}%
\pgfpathlineto{\pgfqpoint{2.870723in}{2.860258in}}%
\pgfpathlineto{\pgfqpoint{2.877446in}{2.861206in}}%
\pgfpathlineto{\pgfqpoint{2.886411in}{2.862571in}}%
\pgfpathlineto{\pgfqpoint{2.893134in}{2.863456in}}%
\pgfpathlineto{\pgfqpoint{2.902098in}{2.864897in}}%
\pgfpathlineto{\pgfqpoint{2.914424in}{2.865834in}}%
\pgfpathlineto{\pgfqpoint{2.987258in}{2.876654in}}%
\pgfpathlineto{\pgfqpoint{2.993981in}{2.877527in}}%
\pgfpathlineto{\pgfqpoint{3.008548in}{2.879448in}}%
\pgfpathlineto{\pgfqpoint{3.019753in}{2.881710in}}%
\pgfpathlineto{\pgfqpoint{3.025356in}{2.882655in}}%
\pgfpathlineto{\pgfqpoint{3.027597in}{2.883292in}}%
\pgfpathlineto{\pgfqpoint{3.034320in}{2.884255in}}%
\pgfpathlineto{\pgfqpoint{3.035441in}{2.884575in}}%
\pgfpathlineto{\pgfqpoint{3.035441in}{2.884575in}}%
\pgfusepath{stroke}%
\end{pgfscope}%
\begin{pgfscope}%
\pgfsetrectcap%
\pgfsetmiterjoin%
\pgfsetlinewidth{0.803000pt}%
\definecolor{currentstroke}{rgb}{1.000000,1.000000,1.000000}%
\pgfsetstrokecolor{currentstroke}%
\pgfsetdash{}{0pt}%
\pgfpathmoveto{\pgfqpoint{0.462318in}{2.320415in}}%
\pgfpathlineto{\pgfqpoint{0.462318in}{3.425294in}}%
\pgfusepath{stroke}%
\end{pgfscope}%
\begin{pgfscope}%
\pgfsetrectcap%
\pgfsetmiterjoin%
\pgfsetlinewidth{0.803000pt}%
\definecolor{currentstroke}{rgb}{1.000000,1.000000,1.000000}%
\pgfsetstrokecolor{currentstroke}%
\pgfsetdash{}{0pt}%
\pgfpathmoveto{\pgfqpoint{3.157970in}{2.320415in}}%
\pgfpathlineto{\pgfqpoint{3.157970in}{3.425294in}}%
\pgfusepath{stroke}%
\end{pgfscope}%
\begin{pgfscope}%
\pgfsetrectcap%
\pgfsetmiterjoin%
\pgfsetlinewidth{0.803000pt}%
\definecolor{currentstroke}{rgb}{1.000000,1.000000,1.000000}%
\pgfsetstrokecolor{currentstroke}%
\pgfsetdash{}{0pt}%
\pgfpathmoveto{\pgfqpoint{0.462318in}{2.320415in}}%
\pgfpathlineto{\pgfqpoint{3.157970in}{2.320415in}}%
\pgfusepath{stroke}%
\end{pgfscope}%
\begin{pgfscope}%
\pgfsetrectcap%
\pgfsetmiterjoin%
\pgfsetlinewidth{0.803000pt}%
\definecolor{currentstroke}{rgb}{1.000000,1.000000,1.000000}%
\pgfsetstrokecolor{currentstroke}%
\pgfsetdash{}{0pt}%
\pgfpathmoveto{\pgfqpoint{0.462318in}{3.425294in}}%
\pgfpathlineto{\pgfqpoint{3.157970in}{3.425294in}}%
\pgfusepath{stroke}%
\end{pgfscope}%
\begin{pgfscope}%
\definecolor{textcolor}{rgb}{0.150000,0.150000,0.150000}%
\pgfsetstrokecolor{textcolor}%
\pgfsetfillcolor{textcolor}%
\pgftext[x=1.810144in,y=3.508627in,,base]{\color{textcolor}\rmfamily\fontsize{12.000000}{14.400000}\selectfont UTX}%
\end{pgfscope}%
\begin{pgfscope}%
\pgfsetbuttcap%
\pgfsetmiterjoin%
\definecolor{currentfill}{rgb}{0.917647,0.917647,0.949020}%
\pgfsetfillcolor{currentfill}%
\pgfsetlinewidth{0.000000pt}%
\definecolor{currentstroke}{rgb}{0.000000,0.000000,0.000000}%
\pgfsetstrokecolor{currentstroke}%
\pgfsetstrokeopacity{0.000000}%
\pgfsetdash{}{0pt}%
\pgfpathmoveto{\pgfqpoint{3.966666in}{2.320415in}}%
\pgfpathlineto{\pgfqpoint{6.662318in}{2.320415in}}%
\pgfpathlineto{\pgfqpoint{6.662318in}{3.425294in}}%
\pgfpathlineto{\pgfqpoint{3.966666in}{3.425294in}}%
\pgfpathclose%
\pgfusepath{fill}%
\end{pgfscope}%
\begin{pgfscope}%
\pgfpathrectangle{\pgfqpoint{3.966666in}{2.320415in}}{\pgfqpoint{2.695652in}{1.104878in}}%
\pgfusepath{clip}%
\pgfsetroundcap%
\pgfsetroundjoin%
\pgfsetlinewidth{0.803000pt}%
\definecolor{currentstroke}{rgb}{1.000000,1.000000,1.000000}%
\pgfsetstrokecolor{currentstroke}%
\pgfsetdash{}{0pt}%
\pgfpathmoveto{\pgfqpoint{4.086955in}{2.320415in}}%
\pgfpathlineto{\pgfqpoint{4.086955in}{3.425294in}}%
\pgfusepath{stroke}%
\end{pgfscope}%
\begin{pgfscope}%
\definecolor{textcolor}{rgb}{0.150000,0.150000,0.150000}%
\pgfsetstrokecolor{textcolor}%
\pgfsetfillcolor{textcolor}%
\pgftext[x=4.086955in,y=2.223193in,,top]{\color{textcolor}\rmfamily\fontsize{10.000000}{12.000000}\selectfont 2012}%
\end{pgfscope}%
\begin{pgfscope}%
\pgfpathrectangle{\pgfqpoint{3.966666in}{2.320415in}}{\pgfqpoint{2.695652in}{1.104878in}}%
\pgfusepath{clip}%
\pgfsetroundcap%
\pgfsetroundjoin%
\pgfsetlinewidth{0.803000pt}%
\definecolor{currentstroke}{rgb}{1.000000,1.000000,1.000000}%
\pgfsetstrokecolor{currentstroke}%
\pgfsetdash{}{0pt}%
\pgfpathmoveto{\pgfqpoint{4.497068in}{2.320415in}}%
\pgfpathlineto{\pgfqpoint{4.497068in}{3.425294in}}%
\pgfusepath{stroke}%
\end{pgfscope}%
\begin{pgfscope}%
\definecolor{textcolor}{rgb}{0.150000,0.150000,0.150000}%
\pgfsetstrokecolor{textcolor}%
\pgfsetfillcolor{textcolor}%
\pgftext[x=4.497068in,y=2.223193in,,top]{\color{textcolor}\rmfamily\fontsize{10.000000}{12.000000}\selectfont 2013}%
\end{pgfscope}%
\begin{pgfscope}%
\pgfpathrectangle{\pgfqpoint{3.966666in}{2.320415in}}{\pgfqpoint{2.695652in}{1.104878in}}%
\pgfusepath{clip}%
\pgfsetroundcap%
\pgfsetroundjoin%
\pgfsetlinewidth{0.803000pt}%
\definecolor{currentstroke}{rgb}{1.000000,1.000000,1.000000}%
\pgfsetstrokecolor{currentstroke}%
\pgfsetdash{}{0pt}%
\pgfpathmoveto{\pgfqpoint{4.906060in}{2.320415in}}%
\pgfpathlineto{\pgfqpoint{4.906060in}{3.425294in}}%
\pgfusepath{stroke}%
\end{pgfscope}%
\begin{pgfscope}%
\definecolor{textcolor}{rgb}{0.150000,0.150000,0.150000}%
\pgfsetstrokecolor{textcolor}%
\pgfsetfillcolor{textcolor}%
\pgftext[x=4.906060in,y=2.223193in,,top]{\color{textcolor}\rmfamily\fontsize{10.000000}{12.000000}\selectfont 2014}%
\end{pgfscope}%
\begin{pgfscope}%
\pgfpathrectangle{\pgfqpoint{3.966666in}{2.320415in}}{\pgfqpoint{2.695652in}{1.104878in}}%
\pgfusepath{clip}%
\pgfsetroundcap%
\pgfsetroundjoin%
\pgfsetlinewidth{0.803000pt}%
\definecolor{currentstroke}{rgb}{1.000000,1.000000,1.000000}%
\pgfsetstrokecolor{currentstroke}%
\pgfsetdash{}{0pt}%
\pgfpathmoveto{\pgfqpoint{5.315052in}{2.320415in}}%
\pgfpathlineto{\pgfqpoint{5.315052in}{3.425294in}}%
\pgfusepath{stroke}%
\end{pgfscope}%
\begin{pgfscope}%
\definecolor{textcolor}{rgb}{0.150000,0.150000,0.150000}%
\pgfsetstrokecolor{textcolor}%
\pgfsetfillcolor{textcolor}%
\pgftext[x=5.315052in,y=2.223193in,,top]{\color{textcolor}\rmfamily\fontsize{10.000000}{12.000000}\selectfont 2015}%
\end{pgfscope}%
\begin{pgfscope}%
\pgfpathrectangle{\pgfqpoint{3.966666in}{2.320415in}}{\pgfqpoint{2.695652in}{1.104878in}}%
\pgfusepath{clip}%
\pgfsetroundcap%
\pgfsetroundjoin%
\pgfsetlinewidth{0.803000pt}%
\definecolor{currentstroke}{rgb}{1.000000,1.000000,1.000000}%
\pgfsetstrokecolor{currentstroke}%
\pgfsetdash{}{0pt}%
\pgfpathmoveto{\pgfqpoint{5.724045in}{2.320415in}}%
\pgfpathlineto{\pgfqpoint{5.724045in}{3.425294in}}%
\pgfusepath{stroke}%
\end{pgfscope}%
\begin{pgfscope}%
\definecolor{textcolor}{rgb}{0.150000,0.150000,0.150000}%
\pgfsetstrokecolor{textcolor}%
\pgfsetfillcolor{textcolor}%
\pgftext[x=5.724045in,y=2.223193in,,top]{\color{textcolor}\rmfamily\fontsize{10.000000}{12.000000}\selectfont 2016}%
\end{pgfscope}%
\begin{pgfscope}%
\pgfpathrectangle{\pgfqpoint{3.966666in}{2.320415in}}{\pgfqpoint{2.695652in}{1.104878in}}%
\pgfusepath{clip}%
\pgfsetroundcap%
\pgfsetroundjoin%
\pgfsetlinewidth{0.803000pt}%
\definecolor{currentstroke}{rgb}{1.000000,1.000000,1.000000}%
\pgfsetstrokecolor{currentstroke}%
\pgfsetdash{}{0pt}%
\pgfpathmoveto{\pgfqpoint{6.134158in}{2.320415in}}%
\pgfpathlineto{\pgfqpoint{6.134158in}{3.425294in}}%
\pgfusepath{stroke}%
\end{pgfscope}%
\begin{pgfscope}%
\definecolor{textcolor}{rgb}{0.150000,0.150000,0.150000}%
\pgfsetstrokecolor{textcolor}%
\pgfsetfillcolor{textcolor}%
\pgftext[x=6.134158in,y=2.223193in,,top]{\color{textcolor}\rmfamily\fontsize{10.000000}{12.000000}\selectfont 2017}%
\end{pgfscope}%
\begin{pgfscope}%
\pgfpathrectangle{\pgfqpoint{3.966666in}{2.320415in}}{\pgfqpoint{2.695652in}{1.104878in}}%
\pgfusepath{clip}%
\pgfsetroundcap%
\pgfsetroundjoin%
\pgfsetlinewidth{0.803000pt}%
\definecolor{currentstroke}{rgb}{1.000000,1.000000,1.000000}%
\pgfsetstrokecolor{currentstroke}%
\pgfsetdash{}{0pt}%
\pgfpathmoveto{\pgfqpoint{6.543150in}{2.320415in}}%
\pgfpathlineto{\pgfqpoint{6.543150in}{3.425294in}}%
\pgfusepath{stroke}%
\end{pgfscope}%
\begin{pgfscope}%
\definecolor{textcolor}{rgb}{0.150000,0.150000,0.150000}%
\pgfsetstrokecolor{textcolor}%
\pgfsetfillcolor{textcolor}%
\pgftext[x=6.543150in,y=2.223193in,,top]{\color{textcolor}\rmfamily\fontsize{10.000000}{12.000000}\selectfont 2018}%
\end{pgfscope}%
\begin{pgfscope}%
\pgfpathrectangle{\pgfqpoint{3.966666in}{2.320415in}}{\pgfqpoint{2.695652in}{1.104878in}}%
\pgfusepath{clip}%
\pgfsetroundcap%
\pgfsetroundjoin%
\pgfsetlinewidth{0.803000pt}%
\definecolor{currentstroke}{rgb}{1.000000,1.000000,1.000000}%
\pgfsetstrokecolor{currentstroke}%
\pgfsetdash{}{0pt}%
\pgfpathmoveto{\pgfqpoint{3.966666in}{2.523448in}}%
\pgfpathlineto{\pgfqpoint{6.662318in}{2.523448in}}%
\pgfusepath{stroke}%
\end{pgfscope}%
\begin{pgfscope}%
\definecolor{textcolor}{rgb}{0.150000,0.150000,0.150000}%
\pgfsetstrokecolor{textcolor}%
\pgfsetfillcolor{textcolor}%
\pgftext[x=3.692713in,y=2.470686in,left,base]{\color{textcolor}\rmfamily\fontsize{10.000000}{12.000000}\selectfont 30}%
\end{pgfscope}%
\begin{pgfscope}%
\pgfpathrectangle{\pgfqpoint{3.966666in}{2.320415in}}{\pgfqpoint{2.695652in}{1.104878in}}%
\pgfusepath{clip}%
\pgfsetroundcap%
\pgfsetroundjoin%
\pgfsetlinewidth{0.803000pt}%
\definecolor{currentstroke}{rgb}{1.000000,1.000000,1.000000}%
\pgfsetstrokecolor{currentstroke}%
\pgfsetdash{}{0pt}%
\pgfpathmoveto{\pgfqpoint{3.966666in}{2.961301in}}%
\pgfpathlineto{\pgfqpoint{6.662318in}{2.961301in}}%
\pgfusepath{stroke}%
\end{pgfscope}%
\begin{pgfscope}%
\definecolor{textcolor}{rgb}{0.150000,0.150000,0.150000}%
\pgfsetstrokecolor{textcolor}%
\pgfsetfillcolor{textcolor}%
\pgftext[x=3.692713in,y=2.908539in,left,base]{\color{textcolor}\rmfamily\fontsize{10.000000}{12.000000}\selectfont 40}%
\end{pgfscope}%
\begin{pgfscope}%
\pgfpathrectangle{\pgfqpoint{3.966666in}{2.320415in}}{\pgfqpoint{2.695652in}{1.104878in}}%
\pgfusepath{clip}%
\pgfsetroundcap%
\pgfsetroundjoin%
\pgfsetlinewidth{0.803000pt}%
\definecolor{currentstroke}{rgb}{1.000000,1.000000,1.000000}%
\pgfsetstrokecolor{currentstroke}%
\pgfsetdash{}{0pt}%
\pgfpathmoveto{\pgfqpoint{3.966666in}{3.399154in}}%
\pgfpathlineto{\pgfqpoint{6.662318in}{3.399154in}}%
\pgfusepath{stroke}%
\end{pgfscope}%
\begin{pgfscope}%
\definecolor{textcolor}{rgb}{0.150000,0.150000,0.150000}%
\pgfsetstrokecolor{textcolor}%
\pgfsetfillcolor{textcolor}%
\pgftext[x=3.692713in,y=3.346392in,left,base]{\color{textcolor}\rmfamily\fontsize{10.000000}{12.000000}\selectfont 50}%
\end{pgfscope}%
\begin{pgfscope}%
\pgfpathrectangle{\pgfqpoint{3.966666in}{2.320415in}}{\pgfqpoint{2.695652in}{1.104878in}}%
\pgfusepath{clip}%
\pgfsetroundcap%
\pgfsetroundjoin%
\pgfsetlinewidth{1.505625pt}%
\definecolor{currentstroke}{rgb}{0.498039,0.498039,0.498039}%
\pgfsetstrokecolor{currentstroke}%
\pgfsetdash{}{0pt}%
\pgfpathmoveto{\pgfqpoint{4.089196in}{2.433250in}}%
\pgfpathlineto{\pgfqpoint{4.091437in}{2.409168in}}%
\pgfpathlineto{\pgfqpoint{4.092557in}{2.405665in}}%
\pgfpathlineto{\pgfqpoint{4.095919in}{2.406979in}}%
\pgfpathlineto{\pgfqpoint{4.097039in}{2.413109in}}%
\pgfpathlineto{\pgfqpoint{4.098160in}{2.423617in}}%
\pgfpathlineto{\pgfqpoint{4.100401in}{2.424055in}}%
\pgfpathlineto{\pgfqpoint{4.104883in}{2.427120in}}%
\pgfpathlineto{\pgfqpoint{4.108245in}{2.425807in}}%
\pgfpathlineto{\pgfqpoint{4.111606in}{2.407855in}}%
\pgfpathlineto{\pgfqpoint{4.112727in}{2.389027in}}%
\pgfpathlineto{\pgfqpoint{4.113847in}{2.385524in}}%
\pgfpathlineto{\pgfqpoint{4.114968in}{2.374578in}}%
\pgfpathlineto{\pgfqpoint{4.116088in}{2.370637in}}%
\pgfpathlineto{\pgfqpoint{4.119450in}{2.383335in}}%
\pgfpathlineto{\pgfqpoint{4.120570in}{2.384649in}}%
\pgfpathlineto{\pgfqpoint{4.121691in}{2.389027in}}%
\pgfpathlineto{\pgfqpoint{4.122811in}{2.381584in}}%
\pgfpathlineto{\pgfqpoint{4.123932in}{2.390341in}}%
\pgfpathlineto{\pgfqpoint{4.127294in}{2.399536in}}%
\pgfpathlineto{\pgfqpoint{4.128414in}{2.392968in}}%
\pgfpathlineto{\pgfqpoint{4.130655in}{2.392968in}}%
\pgfpathlineto{\pgfqpoint{4.131776in}{2.385524in}}%
\pgfpathlineto{\pgfqpoint{4.135137in}{2.399536in}}%
\pgfpathlineto{\pgfqpoint{4.136258in}{2.396471in}}%
\pgfpathlineto{\pgfqpoint{4.137378in}{2.389903in}}%
\pgfpathlineto{\pgfqpoint{4.138499in}{2.396908in}}%
\pgfpathlineto{\pgfqpoint{4.139619in}{2.409606in}}%
\pgfpathlineto{\pgfqpoint{4.144101in}{2.410482in}}%
\pgfpathlineto{\pgfqpoint{4.145222in}{2.401725in}}%
\pgfpathlineto{\pgfqpoint{4.146343in}{2.399536in}}%
\pgfpathlineto{\pgfqpoint{4.151945in}{2.399973in}}%
\pgfpathlineto{\pgfqpoint{4.153066in}{2.398660in}}%
\pgfpathlineto{\pgfqpoint{4.155307in}{2.416174in}}%
\pgfpathlineto{\pgfqpoint{4.158668in}{2.426682in}}%
\pgfpathlineto{\pgfqpoint{4.159789in}{2.417050in}}%
\pgfpathlineto{\pgfqpoint{4.160909in}{2.422742in}}%
\pgfpathlineto{\pgfqpoint{4.162030in}{2.433250in}}%
\pgfpathlineto{\pgfqpoint{4.163150in}{2.429747in}}%
\pgfpathlineto{\pgfqpoint{4.166512in}{2.436753in}}%
\pgfpathlineto{\pgfqpoint{4.167633in}{2.441569in}}%
\pgfpathlineto{\pgfqpoint{4.168753in}{2.441132in}}%
\pgfpathlineto{\pgfqpoint{4.169874in}{2.443321in}}%
\pgfpathlineto{\pgfqpoint{4.174356in}{2.446824in}}%
\pgfpathlineto{\pgfqpoint{4.175476in}{2.446386in}}%
\pgfpathlineto{\pgfqpoint{4.176597in}{2.450764in}}%
\pgfpathlineto{\pgfqpoint{4.177717in}{2.447261in}}%
\pgfpathlineto{\pgfqpoint{4.178838in}{2.439818in}}%
\pgfpathlineto{\pgfqpoint{4.182199in}{2.436753in}}%
\pgfpathlineto{\pgfqpoint{4.184440in}{2.403476in}}%
\pgfpathlineto{\pgfqpoint{4.185561in}{2.397784in}}%
\pgfpathlineto{\pgfqpoint{4.186682in}{2.402600in}}%
\pgfpathlineto{\pgfqpoint{4.190043in}{2.411795in}}%
\pgfpathlineto{\pgfqpoint{4.192284in}{2.407417in}}%
\pgfpathlineto{\pgfqpoint{4.193405in}{2.400411in}}%
\pgfpathlineto{\pgfqpoint{4.197887in}{2.393843in}}%
\pgfpathlineto{\pgfqpoint{4.199007in}{2.373264in}}%
\pgfpathlineto{\pgfqpoint{4.200128in}{2.391216in}}%
\pgfpathlineto{\pgfqpoint{4.201248in}{2.396908in}}%
\pgfpathlineto{\pgfqpoint{4.202369in}{2.387713in}}%
\pgfpathlineto{\pgfqpoint{4.205730in}{2.392968in}}%
\pgfpathlineto{\pgfqpoint{4.206851in}{2.402600in}}%
\pgfpathlineto{\pgfqpoint{4.207972in}{2.400411in}}%
\pgfpathlineto{\pgfqpoint{4.210213in}{2.434126in}}%
\pgfpathlineto{\pgfqpoint{4.213574in}{2.428872in}}%
\pgfpathlineto{\pgfqpoint{4.214695in}{2.458646in}}%
\pgfpathlineto{\pgfqpoint{4.215815in}{2.457770in}}%
\pgfpathlineto{\pgfqpoint{4.216936in}{2.478787in}}%
\pgfpathlineto{\pgfqpoint{4.218056in}{2.481414in}}%
\pgfpathlineto{\pgfqpoint{4.221418in}{2.486230in}}%
\pgfpathlineto{\pgfqpoint{4.222538in}{2.491922in}}%
\pgfpathlineto{\pgfqpoint{4.224779in}{2.494550in}}%
\pgfpathlineto{\pgfqpoint{4.225900in}{2.482290in}}%
\pgfpathlineto{\pgfqpoint{4.229262in}{2.491047in}}%
\pgfpathlineto{\pgfqpoint{4.230382in}{2.491485in}}%
\pgfpathlineto{\pgfqpoint{4.231503in}{2.482290in}}%
\pgfpathlineto{\pgfqpoint{4.232623in}{2.491485in}}%
\pgfpathlineto{\pgfqpoint{4.233744in}{2.510750in}}%
\pgfpathlineto{\pgfqpoint{4.237105in}{2.502431in}}%
\pgfpathlineto{\pgfqpoint{4.238226in}{2.507247in}}%
\pgfpathlineto{\pgfqpoint{4.239346in}{2.501993in}}%
\pgfpathlineto{\pgfqpoint{4.240467in}{2.517756in}}%
\pgfpathlineto{\pgfqpoint{4.241587in}{2.522572in}}%
\pgfpathlineto{\pgfqpoint{4.244949in}{2.516442in}}%
\pgfpathlineto{\pgfqpoint{4.246069in}{2.518194in}}%
\pgfpathlineto{\pgfqpoint{4.247190in}{2.514691in}}%
\pgfpathlineto{\pgfqpoint{4.249431in}{2.519945in}}%
\pgfpathlineto{\pgfqpoint{4.253913in}{2.529578in}}%
\pgfpathlineto{\pgfqpoint{4.255034in}{2.519069in}}%
\pgfpathlineto{\pgfqpoint{4.256154in}{2.526075in}}%
\pgfpathlineto{\pgfqpoint{4.257275in}{2.506809in}}%
\pgfpathlineto{\pgfqpoint{4.260636in}{2.516442in}}%
\pgfpathlineto{\pgfqpoint{4.261757in}{2.512064in}}%
\pgfpathlineto{\pgfqpoint{4.262877in}{2.530016in}}%
\pgfpathlineto{\pgfqpoint{4.263998in}{2.526075in}}%
\pgfpathlineto{\pgfqpoint{4.265118in}{2.551470in}}%
\pgfpathlineto{\pgfqpoint{4.268480in}{2.554973in}}%
\pgfpathlineto{\pgfqpoint{4.269601in}{2.567233in}}%
\pgfpathlineto{\pgfqpoint{4.270721in}{2.568547in}}%
\pgfpathlineto{\pgfqpoint{4.271842in}{2.592629in}}%
\pgfpathlineto{\pgfqpoint{4.272962in}{2.586499in}}%
\pgfpathlineto{\pgfqpoint{4.276324in}{2.594818in}}%
\pgfpathlineto{\pgfqpoint{4.277444in}{2.592191in}}%
\pgfpathlineto{\pgfqpoint{4.278565in}{2.578617in}}%
\pgfpathlineto{\pgfqpoint{4.279685in}{2.579493in}}%
\pgfpathlineto{\pgfqpoint{4.280806in}{2.599196in}}%
\pgfpathlineto{\pgfqpoint{4.284167in}{2.589564in}}%
\pgfpathlineto{\pgfqpoint{4.285288in}{2.595694in}}%
\pgfpathlineto{\pgfqpoint{4.286408in}{2.593942in}}%
\pgfpathlineto{\pgfqpoint{4.287529in}{2.599634in}}%
\pgfpathlineto{\pgfqpoint{4.288649in}{2.614521in}}%
\pgfpathlineto{\pgfqpoint{4.292011in}{2.630284in}}%
\pgfpathlineto{\pgfqpoint{4.293132in}{2.630722in}}%
\pgfpathlineto{\pgfqpoint{4.295373in}{2.628970in}}%
\pgfpathlineto{\pgfqpoint{4.296493in}{2.629846in}}%
\pgfpathlineto{\pgfqpoint{4.299855in}{2.639917in}}%
\pgfpathlineto{\pgfqpoint{4.300975in}{2.638603in}}%
\pgfpathlineto{\pgfqpoint{4.302096in}{2.645171in}}%
\pgfpathlineto{\pgfqpoint{4.303216in}{2.637727in}}%
\pgfpathlineto{\pgfqpoint{4.304337in}{2.655242in}}%
\pgfpathlineto{\pgfqpoint{4.307698in}{2.656993in}}%
\pgfpathlineto{\pgfqpoint{4.309940in}{2.676696in}}%
\pgfpathlineto{\pgfqpoint{4.311060in}{2.633787in}}%
\pgfpathlineto{\pgfqpoint{4.315542in}{2.627657in}}%
\pgfpathlineto{\pgfqpoint{4.316663in}{2.608829in}}%
\pgfpathlineto{\pgfqpoint{4.317783in}{2.607516in}}%
\pgfpathlineto{\pgfqpoint{4.320024in}{2.645171in}}%
\pgfpathlineto{\pgfqpoint{4.323386in}{2.646922in}}%
\pgfpathlineto{\pgfqpoint{4.324506in}{2.652614in}}%
\pgfpathlineto{\pgfqpoint{4.325627in}{2.655242in}}%
\pgfpathlineto{\pgfqpoint{4.326747in}{2.635976in}}%
\pgfpathlineto{\pgfqpoint{4.327868in}{2.631160in}}%
\pgfpathlineto{\pgfqpoint{4.331230in}{2.638603in}}%
\pgfpathlineto{\pgfqpoint{4.333471in}{2.621965in}}%
\pgfpathlineto{\pgfqpoint{4.334591in}{2.627219in}}%
\pgfpathlineto{\pgfqpoint{4.335712in}{2.635538in}}%
\pgfpathlineto{\pgfqpoint{4.339073in}{2.625030in}}%
\pgfpathlineto{\pgfqpoint{4.340194in}{2.626781in}}%
\pgfpathlineto{\pgfqpoint{4.342435in}{2.620213in}}%
\pgfpathlineto{\pgfqpoint{4.343555in}{2.618462in}}%
\pgfpathlineto{\pgfqpoint{4.346917in}{2.606640in}}%
\pgfpathlineto{\pgfqpoint{4.348037in}{2.580807in}}%
\pgfpathlineto{\pgfqpoint{4.350278in}{2.560228in}}%
\pgfpathlineto{\pgfqpoint{4.351399in}{2.590002in}}%
\pgfpathlineto{\pgfqpoint{4.355881in}{2.573801in}}%
\pgfpathlineto{\pgfqpoint{4.357002in}{2.587374in}}%
\pgfpathlineto{\pgfqpoint{4.358122in}{2.576866in}}%
\pgfpathlineto{\pgfqpoint{4.364845in}{2.610143in}}%
\pgfpathlineto{\pgfqpoint{4.365966in}{2.621089in}}%
\pgfpathlineto{\pgfqpoint{4.367086in}{2.607516in}}%
\pgfpathlineto{\pgfqpoint{4.370448in}{2.618462in}}%
\pgfpathlineto{\pgfqpoint{4.371569in}{2.624154in}}%
\pgfpathlineto{\pgfqpoint{4.373810in}{2.667064in}}%
\pgfpathlineto{\pgfqpoint{4.374930in}{2.633349in}}%
\pgfpathlineto{\pgfqpoint{4.378292in}{2.635100in}}%
\pgfpathlineto{\pgfqpoint{4.381653in}{2.663999in}}%
\pgfpathlineto{\pgfqpoint{4.382774in}{2.668815in}}%
\pgfpathlineto{\pgfqpoint{4.386135in}{2.670129in}}%
\pgfpathlineto{\pgfqpoint{4.387256in}{2.667939in}}%
\pgfpathlineto{\pgfqpoint{4.388376in}{2.667064in}}%
\pgfpathlineto{\pgfqpoint{4.389497in}{2.672756in}}%
\pgfpathlineto{\pgfqpoint{4.390617in}{2.666626in}}%
\pgfpathlineto{\pgfqpoint{4.395100in}{2.675821in}}%
\pgfpathlineto{\pgfqpoint{4.396220in}{2.689394in}}%
\pgfpathlineto{\pgfqpoint{4.397341in}{2.720482in}}%
\pgfpathlineto{\pgfqpoint{4.398461in}{2.730552in}}%
\pgfpathlineto{\pgfqpoint{4.401823in}{2.714790in}}%
\pgfpathlineto{\pgfqpoint{4.406305in}{2.651739in}}%
\pgfpathlineto{\pgfqpoint{4.409666in}{2.647798in}}%
\pgfpathlineto{\pgfqpoint{4.410787in}{2.634225in}}%
\pgfpathlineto{\pgfqpoint{4.411907in}{2.655242in}}%
\pgfpathlineto{\pgfqpoint{4.413028in}{2.689394in}}%
\pgfpathlineto{\pgfqpoint{4.414149in}{2.669253in}}%
\pgfpathlineto{\pgfqpoint{4.417510in}{2.656555in}}%
\pgfpathlineto{\pgfqpoint{4.418631in}{2.634225in}}%
\pgfpathlineto{\pgfqpoint{4.419751in}{2.639479in}}%
\pgfpathlineto{\pgfqpoint{4.420872in}{2.639917in}}%
\pgfpathlineto{\pgfqpoint{4.421992in}{2.655242in}}%
\pgfpathlineto{\pgfqpoint{4.427595in}{2.652614in}}%
\pgfpathlineto{\pgfqpoint{4.428715in}{2.668815in}}%
\pgfpathlineto{\pgfqpoint{4.429836in}{2.648674in}}%
\pgfpathlineto{\pgfqpoint{4.433198in}{2.638165in}}%
\pgfpathlineto{\pgfqpoint{4.434318in}{2.641668in}}%
\pgfpathlineto{\pgfqpoint{4.436559in}{2.586937in}}%
\pgfpathlineto{\pgfqpoint{4.437680in}{2.587812in}}%
\pgfpathlineto{\pgfqpoint{4.442162in}{2.584747in}}%
\pgfpathlineto{\pgfqpoint{4.443282in}{2.575115in}}%
\pgfpathlineto{\pgfqpoint{4.445523in}{2.547968in}}%
\pgfpathlineto{\pgfqpoint{4.448885in}{2.593504in}}%
\pgfpathlineto{\pgfqpoint{4.450005in}{2.593504in}}%
\pgfpathlineto{\pgfqpoint{4.453367in}{2.624154in}}%
\pgfpathlineto{\pgfqpoint{4.456729in}{2.609267in}}%
\pgfpathlineto{\pgfqpoint{4.457849in}{2.598759in}}%
\pgfpathlineto{\pgfqpoint{4.460090in}{2.632035in}}%
\pgfpathlineto{\pgfqpoint{4.461211in}{2.635538in}}%
\pgfpathlineto{\pgfqpoint{4.464572in}{2.635100in}}%
\pgfpathlineto{\pgfqpoint{4.465693in}{2.621089in}}%
\pgfpathlineto{\pgfqpoint{4.467934in}{2.646485in}}%
\pgfpathlineto{\pgfqpoint{4.469054in}{2.645171in}}%
\pgfpathlineto{\pgfqpoint{4.472416in}{2.632911in}}%
\pgfpathlineto{\pgfqpoint{4.474657in}{2.657431in}}%
\pgfpathlineto{\pgfqpoint{4.476898in}{2.638603in}}%
\pgfpathlineto{\pgfqpoint{4.480260in}{2.635538in}}%
\pgfpathlineto{\pgfqpoint{4.481380in}{2.627657in}}%
\pgfpathlineto{\pgfqpoint{4.482501in}{2.611894in}}%
\pgfpathlineto{\pgfqpoint{4.483621in}{2.625905in}}%
\pgfpathlineto{\pgfqpoint{4.484742in}{2.618024in}}%
\pgfpathlineto{\pgfqpoint{4.488103in}{2.617148in}}%
\pgfpathlineto{\pgfqpoint{4.490344in}{2.614083in}}%
\pgfpathlineto{\pgfqpoint{4.491465in}{2.614959in}}%
\pgfpathlineto{\pgfqpoint{4.492585in}{2.596131in}}%
\pgfpathlineto{\pgfqpoint{4.495947in}{2.608391in}}%
\pgfpathlineto{\pgfqpoint{4.498188in}{2.640355in}}%
\pgfpathlineto{\pgfqpoint{4.499309in}{2.633787in}}%
\pgfpathlineto{\pgfqpoint{4.500429in}{2.641668in}}%
\pgfpathlineto{\pgfqpoint{4.503791in}{2.653928in}}%
\pgfpathlineto{\pgfqpoint{4.504911in}{2.618900in}}%
\pgfpathlineto{\pgfqpoint{4.506032in}{2.615835in}}%
\pgfpathlineto{\pgfqpoint{4.507152in}{2.635100in}}%
\pgfpathlineto{\pgfqpoint{4.508273in}{2.625468in}}%
\pgfpathlineto{\pgfqpoint{4.511634in}{2.602261in}}%
\pgfpathlineto{\pgfqpoint{4.513875in}{2.566795in}}%
\pgfpathlineto{\pgfqpoint{4.516117in}{2.600510in}}%
\pgfpathlineto{\pgfqpoint{4.520599in}{2.613646in}}%
\pgfpathlineto{\pgfqpoint{4.522840in}{2.602261in}}%
\pgfpathlineto{\pgfqpoint{4.523960in}{2.604889in}}%
\pgfpathlineto{\pgfqpoint{4.527322in}{2.608391in}}%
\pgfpathlineto{\pgfqpoint{4.528442in}{2.632035in}}%
\pgfpathlineto{\pgfqpoint{4.529563in}{2.635538in}}%
\pgfpathlineto{\pgfqpoint{4.530683in}{2.635538in}}%
\pgfpathlineto{\pgfqpoint{4.531804in}{2.666626in}}%
\pgfpathlineto{\pgfqpoint{4.535165in}{2.665312in}}%
\pgfpathlineto{\pgfqpoint{4.536286in}{2.666626in}}%
\pgfpathlineto{\pgfqpoint{4.537407in}{2.672756in}}%
\pgfpathlineto{\pgfqpoint{4.538527in}{2.663561in}}%
\pgfpathlineto{\pgfqpoint{4.539648in}{2.660058in}}%
\pgfpathlineto{\pgfqpoint{4.543009in}{2.658744in}}%
\pgfpathlineto{\pgfqpoint{4.545250in}{2.665312in}}%
\pgfpathlineto{\pgfqpoint{4.546371in}{2.659182in}}%
\pgfpathlineto{\pgfqpoint{4.547491in}{2.661372in}}%
\pgfpathlineto{\pgfqpoint{4.551973in}{2.664874in}}%
\pgfpathlineto{\pgfqpoint{4.553094in}{2.678448in}}%
\pgfpathlineto{\pgfqpoint{4.555335in}{2.694211in}}%
\pgfpathlineto{\pgfqpoint{4.558697in}{2.704719in}}%
\pgfpathlineto{\pgfqpoint{4.560938in}{2.725298in}}%
\pgfpathlineto{\pgfqpoint{4.563179in}{2.737558in}}%
\pgfpathlineto{\pgfqpoint{4.566540in}{2.750256in}}%
\pgfpathlineto{\pgfqpoint{4.567661in}{2.769083in}}%
\pgfpathlineto{\pgfqpoint{4.568781in}{2.755510in}}%
\pgfpathlineto{\pgfqpoint{4.569902in}{2.762078in}}%
\pgfpathlineto{\pgfqpoint{4.571022in}{2.777840in}}%
\pgfpathlineto{\pgfqpoint{4.574384in}{2.773024in}}%
\pgfpathlineto{\pgfqpoint{4.575504in}{2.790976in}}%
\pgfpathlineto{\pgfqpoint{4.576625in}{2.777403in}}%
\pgfpathlineto{\pgfqpoint{4.577746in}{2.794917in}}%
\pgfpathlineto{\pgfqpoint{4.578866in}{2.780030in}}%
\pgfpathlineto{\pgfqpoint{4.583348in}{2.810242in}}%
\pgfpathlineto{\pgfqpoint{4.584469in}{2.798857in}}%
\pgfpathlineto{\pgfqpoint{4.586710in}{2.812431in}}%
\pgfpathlineto{\pgfqpoint{4.590071in}{2.817247in}}%
\pgfpathlineto{\pgfqpoint{4.591192in}{2.827756in}}%
\pgfpathlineto{\pgfqpoint{4.592312in}{2.809804in}}%
\pgfpathlineto{\pgfqpoint{4.593433in}{2.816809in}}%
\pgfpathlineto{\pgfqpoint{4.597915in}{2.818999in}}%
\pgfpathlineto{\pgfqpoint{4.599036in}{2.828194in}}%
\pgfpathlineto{\pgfqpoint{4.600156in}{2.811555in}}%
\pgfpathlineto{\pgfqpoint{4.602397in}{2.830383in}}%
\pgfpathlineto{\pgfqpoint{4.605759in}{2.843081in}}%
\pgfpathlineto{\pgfqpoint{4.606879in}{2.840453in}}%
\pgfpathlineto{\pgfqpoint{4.610241in}{2.890369in}}%
\pgfpathlineto{\pgfqpoint{4.613602in}{2.882925in}}%
\pgfpathlineto{\pgfqpoint{4.614723in}{2.876795in}}%
\pgfpathlineto{\pgfqpoint{4.615843in}{2.846583in}}%
\pgfpathlineto{\pgfqpoint{4.618084in}{2.935905in}}%
\pgfpathlineto{\pgfqpoint{4.621446in}{2.934154in}}%
\pgfpathlineto{\pgfqpoint{4.622567in}{2.938532in}}%
\pgfpathlineto{\pgfqpoint{4.623687in}{2.921018in}}%
\pgfpathlineto{\pgfqpoint{4.624808in}{2.968306in}}%
\pgfpathlineto{\pgfqpoint{4.625928in}{2.981880in}}%
\pgfpathlineto{\pgfqpoint{4.629290in}{2.976188in}}%
\pgfpathlineto{\pgfqpoint{4.630410in}{2.991075in}}%
\pgfpathlineto{\pgfqpoint{4.631531in}{2.941160in}}%
\pgfpathlineto{\pgfqpoint{4.633772in}{2.950354in}}%
\pgfpathlineto{\pgfqpoint{4.637133in}{2.929338in}}%
\pgfpathlineto{\pgfqpoint{4.638254in}{2.958236in}}%
\pgfpathlineto{\pgfqpoint{4.639374in}{2.964366in}}%
\pgfpathlineto{\pgfqpoint{4.640495in}{2.951230in}}%
\pgfpathlineto{\pgfqpoint{4.641616in}{2.957360in}}%
\pgfpathlineto{\pgfqpoint{4.644977in}{2.945976in}}%
\pgfpathlineto{\pgfqpoint{4.647218in}{2.980566in}}%
\pgfpathlineto{\pgfqpoint{4.648339in}{2.967431in}}%
\pgfpathlineto{\pgfqpoint{4.649459in}{2.972247in}}%
\pgfpathlineto{\pgfqpoint{4.652821in}{2.952106in}}%
\pgfpathlineto{\pgfqpoint{4.655062in}{2.910510in}}%
\pgfpathlineto{\pgfqpoint{4.656182in}{2.924083in}}%
\pgfpathlineto{\pgfqpoint{4.657303in}{2.907883in}}%
\pgfpathlineto{\pgfqpoint{4.661785in}{2.889055in}}%
\pgfpathlineto{\pgfqpoint{4.662906in}{2.847459in}}%
\pgfpathlineto{\pgfqpoint{4.665147in}{2.811555in}}%
\pgfpathlineto{\pgfqpoint{4.668508in}{2.817685in}}%
\pgfpathlineto{\pgfqpoint{4.669629in}{2.823377in}}%
\pgfpathlineto{\pgfqpoint{4.670749in}{2.805425in}}%
\pgfpathlineto{\pgfqpoint{4.671870in}{2.860595in}}%
\pgfpathlineto{\pgfqpoint{4.672990in}{2.869790in}}%
\pgfpathlineto{\pgfqpoint{4.676352in}{2.879422in}}%
\pgfpathlineto{\pgfqpoint{4.678593in}{2.858405in}}%
\pgfpathlineto{\pgfqpoint{4.680834in}{2.896936in}}%
\pgfpathlineto{\pgfqpoint{4.684196in}{2.885114in}}%
\pgfpathlineto{\pgfqpoint{4.685316in}{2.913137in}}%
\pgfpathlineto{\pgfqpoint{4.687557in}{2.827318in}}%
\pgfpathlineto{\pgfqpoint{4.688678in}{2.845708in}}%
\pgfpathlineto{\pgfqpoint{4.692039in}{2.832572in}}%
\pgfpathlineto{\pgfqpoint{4.693160in}{2.876357in}}%
\pgfpathlineto{\pgfqpoint{4.694280in}{2.883363in}}%
\pgfpathlineto{\pgfqpoint{4.695401in}{2.894747in}}%
\pgfpathlineto{\pgfqpoint{4.696521in}{2.872855in}}%
\pgfpathlineto{\pgfqpoint{4.699883in}{2.873730in}}%
\pgfpathlineto{\pgfqpoint{4.704365in}{2.904818in}}%
\pgfpathlineto{\pgfqpoint{4.707727in}{2.917516in}}%
\pgfpathlineto{\pgfqpoint{4.708847in}{2.910510in}}%
\pgfpathlineto{\pgfqpoint{4.709968in}{2.896499in}}%
\pgfpathlineto{\pgfqpoint{4.711088in}{2.918829in}}%
\pgfpathlineto{\pgfqpoint{4.712209in}{2.892120in}}%
\pgfpathlineto{\pgfqpoint{4.715570in}{2.877233in}}%
\pgfpathlineto{\pgfqpoint{4.717811in}{2.903066in}}%
\pgfpathlineto{\pgfqpoint{4.718932in}{2.877671in}}%
\pgfpathlineto{\pgfqpoint{4.720052in}{2.876795in}}%
\pgfpathlineto{\pgfqpoint{4.724535in}{2.890369in}}%
\pgfpathlineto{\pgfqpoint{4.725655in}{2.891244in}}%
\pgfpathlineto{\pgfqpoint{4.727896in}{2.912699in}}%
\pgfpathlineto{\pgfqpoint{4.731258in}{2.928024in}}%
\pgfpathlineto{\pgfqpoint{4.733499in}{2.861032in}}%
\pgfpathlineto{\pgfqpoint{4.734619in}{2.878984in}}%
\pgfpathlineto{\pgfqpoint{4.735740in}{2.886866in}}%
\pgfpathlineto{\pgfqpoint{4.739101in}{2.885552in}}%
\pgfpathlineto{\pgfqpoint{4.741342in}{2.876357in}}%
\pgfpathlineto{\pgfqpoint{4.743584in}{2.855778in}}%
\pgfpathlineto{\pgfqpoint{4.746945in}{2.866725in}}%
\pgfpathlineto{\pgfqpoint{4.749186in}{2.841329in}}%
\pgfpathlineto{\pgfqpoint{4.750307in}{2.829945in}}%
\pgfpathlineto{\pgfqpoint{4.751427in}{2.801922in}}%
\pgfpathlineto{\pgfqpoint{4.754789in}{2.794917in}}%
\pgfpathlineto{\pgfqpoint{4.755909in}{2.808928in}}%
\pgfpathlineto{\pgfqpoint{4.757030in}{2.787473in}}%
\pgfpathlineto{\pgfqpoint{4.758150in}{2.779154in}}%
\pgfpathlineto{\pgfqpoint{4.759271in}{2.798857in}}%
\pgfpathlineto{\pgfqpoint{4.762632in}{2.776527in}}%
\pgfpathlineto{\pgfqpoint{4.763753in}{2.776527in}}%
\pgfpathlineto{\pgfqpoint{4.764874in}{2.763829in}}%
\pgfpathlineto{\pgfqpoint{4.765994in}{2.805863in}}%
\pgfpathlineto{\pgfqpoint{4.767115in}{2.790976in}}%
\pgfpathlineto{\pgfqpoint{4.771597in}{2.745439in}}%
\pgfpathlineto{\pgfqpoint{4.772717in}{2.770835in}}%
\pgfpathlineto{\pgfqpoint{4.773838in}{2.766456in}}%
\pgfpathlineto{\pgfqpoint{4.774958in}{2.756386in}}%
\pgfpathlineto{\pgfqpoint{4.778320in}{2.741936in}}%
\pgfpathlineto{\pgfqpoint{4.779440in}{2.760764in}}%
\pgfpathlineto{\pgfqpoint{4.780561in}{2.762516in}}%
\pgfpathlineto{\pgfqpoint{4.781681in}{2.790100in}}%
\pgfpathlineto{\pgfqpoint{4.782802in}{2.803674in}}%
\pgfpathlineto{\pgfqpoint{4.786164in}{2.821626in}}%
\pgfpathlineto{\pgfqpoint{4.788405in}{2.835637in}}%
\pgfpathlineto{\pgfqpoint{4.789525in}{2.828631in}}%
\pgfpathlineto{\pgfqpoint{4.790646in}{2.804549in}}%
\pgfpathlineto{\pgfqpoint{4.794007in}{2.811117in}}%
\pgfpathlineto{\pgfqpoint{4.795128in}{2.787473in}}%
\pgfpathlineto{\pgfqpoint{4.796248in}{2.776527in}}%
\pgfpathlineto{\pgfqpoint{4.797369in}{2.800609in}}%
\pgfpathlineto{\pgfqpoint{4.798489in}{2.778278in}}%
\pgfpathlineto{\pgfqpoint{4.801851in}{2.767332in}}%
\pgfpathlineto{\pgfqpoint{4.802971in}{2.777840in}}%
\pgfpathlineto{\pgfqpoint{4.804092in}{2.771273in}}%
\pgfpathlineto{\pgfqpoint{4.805213in}{2.778716in}}%
\pgfpathlineto{\pgfqpoint{4.806333in}{2.781781in}}%
\pgfpathlineto{\pgfqpoint{4.809695in}{2.787911in}}%
\pgfpathlineto{\pgfqpoint{4.810815in}{2.764267in}}%
\pgfpathlineto{\pgfqpoint{4.811936in}{2.769083in}}%
\pgfpathlineto{\pgfqpoint{4.813056in}{2.791414in}}%
\pgfpathlineto{\pgfqpoint{4.814177in}{2.799295in}}%
\pgfpathlineto{\pgfqpoint{4.817538in}{2.789662in}}%
\pgfpathlineto{\pgfqpoint{4.818659in}{2.773024in}}%
\pgfpathlineto{\pgfqpoint{4.819779in}{2.804549in}}%
\pgfpathlineto{\pgfqpoint{4.822020in}{2.897812in}}%
\pgfpathlineto{\pgfqpoint{4.825382in}{2.917078in}}%
\pgfpathlineto{\pgfqpoint{4.826503in}{2.935905in}}%
\pgfpathlineto{\pgfqpoint{4.828744in}{2.911386in}}%
\pgfpathlineto{\pgfqpoint{4.829864in}{2.921456in}}%
\pgfpathlineto{\pgfqpoint{4.833226in}{2.916640in}}%
\pgfpathlineto{\pgfqpoint{4.834346in}{2.934154in}}%
\pgfpathlineto{\pgfqpoint{4.835467in}{2.915326in}}%
\pgfpathlineto{\pgfqpoint{4.837708in}{2.914013in}}%
\pgfpathlineto{\pgfqpoint{4.841069in}{2.933716in}}%
\pgfpathlineto{\pgfqpoint{4.842190in}{2.900877in}}%
\pgfpathlineto{\pgfqpoint{4.843310in}{2.917953in}}%
\pgfpathlineto{\pgfqpoint{4.844431in}{2.903066in}}%
\pgfpathlineto{\pgfqpoint{4.845551in}{2.904380in}}%
\pgfpathlineto{\pgfqpoint{4.848913in}{2.896061in}}%
\pgfpathlineto{\pgfqpoint{4.850034in}{2.902629in}}%
\pgfpathlineto{\pgfqpoint{4.851154in}{2.896936in}}%
\pgfpathlineto{\pgfqpoint{4.852275in}{2.906569in}}%
\pgfpathlineto{\pgfqpoint{4.853395in}{2.907883in}}%
\pgfpathlineto{\pgfqpoint{4.856757in}{2.923208in}}%
\pgfpathlineto{\pgfqpoint{4.857877in}{2.923645in}}%
\pgfpathlineto{\pgfqpoint{4.858998in}{2.910948in}}%
\pgfpathlineto{\pgfqpoint{4.860118in}{2.910072in}}%
\pgfpathlineto{\pgfqpoint{4.861239in}{2.904818in}}%
\pgfpathlineto{\pgfqpoint{4.864600in}{2.897812in}}%
\pgfpathlineto{\pgfqpoint{4.865721in}{2.899126in}}%
\pgfpathlineto{\pgfqpoint{4.866842in}{2.895185in}}%
\pgfpathlineto{\pgfqpoint{4.869083in}{2.884677in}}%
\pgfpathlineto{\pgfqpoint{4.872444in}{2.872417in}}%
\pgfpathlineto{\pgfqpoint{4.873565in}{2.883801in}}%
\pgfpathlineto{\pgfqpoint{4.874685in}{2.876357in}}%
\pgfpathlineto{\pgfqpoint{4.875806in}{2.860595in}}%
\pgfpathlineto{\pgfqpoint{4.876926in}{2.879860in}}%
\pgfpathlineto{\pgfqpoint{4.880288in}{2.882925in}}%
\pgfpathlineto{\pgfqpoint{4.883649in}{2.834323in}}%
\pgfpathlineto{\pgfqpoint{4.884770in}{2.824691in}}%
\pgfpathlineto{\pgfqpoint{4.888132in}{2.838702in}}%
\pgfpathlineto{\pgfqpoint{4.889252in}{2.815058in}}%
\pgfpathlineto{\pgfqpoint{4.890373in}{2.845708in}}%
\pgfpathlineto{\pgfqpoint{4.891493in}{2.844394in}}%
\pgfpathlineto{\pgfqpoint{4.892614in}{2.832572in}}%
\pgfpathlineto{\pgfqpoint{4.895975in}{2.849648in}}%
\pgfpathlineto{\pgfqpoint{4.897096in}{2.862784in}}%
\pgfpathlineto{\pgfqpoint{4.899337in}{2.869790in}}%
\pgfpathlineto{\pgfqpoint{4.900457in}{2.869352in}}%
\pgfpathlineto{\pgfqpoint{4.904939in}{2.868476in}}%
\pgfpathlineto{\pgfqpoint{4.907180in}{2.863660in}}%
\pgfpathlineto{\pgfqpoint{4.908301in}{2.843956in}}%
\pgfpathlineto{\pgfqpoint{4.911663in}{2.853151in}}%
\pgfpathlineto{\pgfqpoint{4.912783in}{2.873730in}}%
\pgfpathlineto{\pgfqpoint{4.913904in}{2.864535in}}%
\pgfpathlineto{\pgfqpoint{4.915024in}{2.830383in}}%
\pgfpathlineto{\pgfqpoint{4.916145in}{2.839140in}}%
\pgfpathlineto{\pgfqpoint{4.919506in}{2.814620in}}%
\pgfpathlineto{\pgfqpoint{4.920627in}{2.815934in}}%
\pgfpathlineto{\pgfqpoint{4.921747in}{2.856654in}}%
\pgfpathlineto{\pgfqpoint{4.922868in}{2.865411in}}%
\pgfpathlineto{\pgfqpoint{4.928471in}{2.837388in}}%
\pgfpathlineto{\pgfqpoint{4.929591in}{2.824691in}}%
\pgfpathlineto{\pgfqpoint{4.930712in}{2.842643in}}%
\pgfpathlineto{\pgfqpoint{4.931832in}{2.834761in}}%
\pgfpathlineto{\pgfqpoint{4.935194in}{2.836951in}}%
\pgfpathlineto{\pgfqpoint{4.936314in}{2.825566in}}%
\pgfpathlineto{\pgfqpoint{4.937435in}{2.836951in}}%
\pgfpathlineto{\pgfqpoint{4.938555in}{2.834761in}}%
\pgfpathlineto{\pgfqpoint{4.939676in}{2.848335in}}%
\pgfpathlineto{\pgfqpoint{4.943037in}{2.793165in}}%
\pgfpathlineto{\pgfqpoint{4.944158in}{2.807177in}}%
\pgfpathlineto{\pgfqpoint{4.945278in}{2.802798in}}%
\pgfpathlineto{\pgfqpoint{4.946399in}{2.802360in}}%
\pgfpathlineto{\pgfqpoint{4.947519in}{2.806739in}}%
\pgfpathlineto{\pgfqpoint{4.950881in}{2.810242in}}%
\pgfpathlineto{\pgfqpoint{4.952002in}{2.820750in}}%
\pgfpathlineto{\pgfqpoint{4.953122in}{2.826004in}}%
\pgfpathlineto{\pgfqpoint{4.954243in}{2.823815in}}%
\pgfpathlineto{\pgfqpoint{4.955363in}{2.796668in}}%
\pgfpathlineto{\pgfqpoint{4.959845in}{2.778716in}}%
\pgfpathlineto{\pgfqpoint{4.960966in}{2.797544in}}%
\pgfpathlineto{\pgfqpoint{4.962086in}{2.851400in}}%
\pgfpathlineto{\pgfqpoint{4.963207in}{2.822501in}}%
\pgfpathlineto{\pgfqpoint{4.966568in}{2.787035in}}%
\pgfpathlineto{\pgfqpoint{4.968809in}{2.791414in}}%
\pgfpathlineto{\pgfqpoint{4.969930in}{2.830383in}}%
\pgfpathlineto{\pgfqpoint{4.971051in}{2.833010in}}%
\pgfpathlineto{\pgfqpoint{4.974412in}{2.823815in}}%
\pgfpathlineto{\pgfqpoint{4.975533in}{2.843956in}}%
\pgfpathlineto{\pgfqpoint{4.976653in}{2.826442in}}%
\pgfpathlineto{\pgfqpoint{4.977774in}{2.829069in}}%
\pgfpathlineto{\pgfqpoint{4.978894in}{2.818561in}}%
\pgfpathlineto{\pgfqpoint{4.982256in}{2.814620in}}%
\pgfpathlineto{\pgfqpoint{4.985617in}{2.780468in}}%
\pgfpathlineto{\pgfqpoint{4.986738in}{2.782219in}}%
\pgfpathlineto{\pgfqpoint{4.990099in}{2.789662in}}%
\pgfpathlineto{\pgfqpoint{4.991220in}{2.803236in}}%
\pgfpathlineto{\pgfqpoint{4.992341in}{2.791414in}}%
\pgfpathlineto{\pgfqpoint{4.993461in}{2.820750in}}%
\pgfpathlineto{\pgfqpoint{4.994582in}{2.810242in}}%
\pgfpathlineto{\pgfqpoint{4.997943in}{2.813744in}}%
\pgfpathlineto{\pgfqpoint{4.999064in}{2.820750in}}%
\pgfpathlineto{\pgfqpoint{5.000184in}{2.813744in}}%
\pgfpathlineto{\pgfqpoint{5.001305in}{2.836951in}}%
\pgfpathlineto{\pgfqpoint{5.002425in}{2.827756in}}%
\pgfpathlineto{\pgfqpoint{5.005787in}{2.833010in}}%
\pgfpathlineto{\pgfqpoint{5.006907in}{2.839140in}}%
\pgfpathlineto{\pgfqpoint{5.008028in}{2.841329in}}%
\pgfpathlineto{\pgfqpoint{5.009148in}{2.851400in}}%
\pgfpathlineto{\pgfqpoint{5.010269in}{2.848773in}}%
\pgfpathlineto{\pgfqpoint{5.013631in}{2.851400in}}%
\pgfpathlineto{\pgfqpoint{5.014751in}{2.872855in}}%
\pgfpathlineto{\pgfqpoint{5.015872in}{2.864973in}}%
\pgfpathlineto{\pgfqpoint{5.018113in}{2.833886in}}%
\pgfpathlineto{\pgfqpoint{5.021474in}{2.840453in}}%
\pgfpathlineto{\pgfqpoint{5.022595in}{2.828631in}}%
\pgfpathlineto{\pgfqpoint{5.023715in}{2.834761in}}%
\pgfpathlineto{\pgfqpoint{5.024836in}{2.851838in}}%
\pgfpathlineto{\pgfqpoint{5.029318in}{2.864973in}}%
\pgfpathlineto{\pgfqpoint{5.030438in}{2.862784in}}%
\pgfpathlineto{\pgfqpoint{5.031559in}{2.846145in}}%
\pgfpathlineto{\pgfqpoint{5.032680in}{2.806301in}}%
\pgfpathlineto{\pgfqpoint{5.033800in}{2.794479in}}%
\pgfpathlineto{\pgfqpoint{5.038282in}{2.823377in}}%
\pgfpathlineto{\pgfqpoint{5.039403in}{2.822064in}}%
\pgfpathlineto{\pgfqpoint{5.040523in}{2.838702in}}%
\pgfpathlineto{\pgfqpoint{5.041644in}{2.835199in}}%
\pgfpathlineto{\pgfqpoint{5.046126in}{2.847459in}}%
\pgfpathlineto{\pgfqpoint{5.048367in}{2.882487in}}%
\pgfpathlineto{\pgfqpoint{5.049487in}{2.882487in}}%
\pgfpathlineto{\pgfqpoint{5.052849in}{2.871103in}}%
\pgfpathlineto{\pgfqpoint{5.053970in}{2.860595in}}%
\pgfpathlineto{\pgfqpoint{5.055090in}{2.866287in}}%
\pgfpathlineto{\pgfqpoint{5.056211in}{2.864535in}}%
\pgfpathlineto{\pgfqpoint{5.057331in}{2.902629in}}%
\pgfpathlineto{\pgfqpoint{5.060693in}{2.905256in}}%
\pgfpathlineto{\pgfqpoint{5.061813in}{2.888179in}}%
\pgfpathlineto{\pgfqpoint{5.064054in}{2.915764in}}%
\pgfpathlineto{\pgfqpoint{5.065175in}{2.925835in}}%
\pgfpathlineto{\pgfqpoint{5.069657in}{2.921456in}}%
\pgfpathlineto{\pgfqpoint{5.070777in}{2.925835in}}%
\pgfpathlineto{\pgfqpoint{5.071898in}{2.924959in}}%
\pgfpathlineto{\pgfqpoint{5.073019in}{2.933278in}}%
\pgfpathlineto{\pgfqpoint{5.076380in}{2.936343in}}%
\pgfpathlineto{\pgfqpoint{5.077501in}{2.910072in}}%
\pgfpathlineto{\pgfqpoint{5.078621in}{2.905256in}}%
\pgfpathlineto{\pgfqpoint{5.080862in}{2.914888in}}%
\pgfpathlineto{\pgfqpoint{5.084224in}{2.919705in}}%
\pgfpathlineto{\pgfqpoint{5.085344in}{2.918391in}}%
\pgfpathlineto{\pgfqpoint{5.086465in}{2.913137in}}%
\pgfpathlineto{\pgfqpoint{5.087585in}{2.901753in}}%
\pgfpathlineto{\pgfqpoint{5.088706in}{2.906569in}}%
\pgfpathlineto{\pgfqpoint{5.092067in}{2.910948in}}%
\pgfpathlineto{\pgfqpoint{5.093188in}{2.907445in}}%
\pgfpathlineto{\pgfqpoint{5.094309in}{2.914888in}}%
\pgfpathlineto{\pgfqpoint{5.095429in}{2.916640in}}%
\pgfpathlineto{\pgfqpoint{5.096550in}{2.913575in}}%
\pgfpathlineto{\pgfqpoint{5.099911in}{2.925397in}}%
\pgfpathlineto{\pgfqpoint{5.101032in}{2.910072in}}%
\pgfpathlineto{\pgfqpoint{5.102152in}{2.914451in}}%
\pgfpathlineto{\pgfqpoint{5.103273in}{2.907445in}}%
\pgfpathlineto{\pgfqpoint{5.104393in}{2.911386in}}%
\pgfpathlineto{\pgfqpoint{5.107755in}{2.897812in}}%
\pgfpathlineto{\pgfqpoint{5.109996in}{2.923208in}}%
\pgfpathlineto{\pgfqpoint{5.111116in}{2.924959in}}%
\pgfpathlineto{\pgfqpoint{5.115599in}{2.926273in}}%
\pgfpathlineto{\pgfqpoint{5.116719in}{2.910072in}}%
\pgfpathlineto{\pgfqpoint{5.117840in}{2.914888in}}%
\pgfpathlineto{\pgfqpoint{5.120081in}{2.964366in}}%
\pgfpathlineto{\pgfqpoint{5.123442in}{2.971809in}}%
\pgfpathlineto{\pgfqpoint{5.125683in}{2.986696in}}%
\pgfpathlineto{\pgfqpoint{5.126804in}{2.964366in}}%
\pgfpathlineto{\pgfqpoint{5.127924in}{2.979691in}}%
\pgfpathlineto{\pgfqpoint{5.131286in}{2.977939in}}%
\pgfpathlineto{\pgfqpoint{5.132406in}{2.987572in}}%
\pgfpathlineto{\pgfqpoint{5.133527in}{2.984945in}}%
\pgfpathlineto{\pgfqpoint{5.134647in}{2.989761in}}%
\pgfpathlineto{\pgfqpoint{5.135768in}{2.998080in}}%
\pgfpathlineto{\pgfqpoint{5.139130in}{3.008589in}}%
\pgfpathlineto{\pgfqpoint{5.140250in}{3.022162in}}%
\pgfpathlineto{\pgfqpoint{5.141371in}{3.014719in}}%
\pgfpathlineto{\pgfqpoint{5.142491in}{2.967869in}}%
\pgfpathlineto{\pgfqpoint{5.143612in}{2.947290in}}%
\pgfpathlineto{\pgfqpoint{5.146973in}{2.960863in}}%
\pgfpathlineto{\pgfqpoint{5.150335in}{2.906131in}}%
\pgfpathlineto{\pgfqpoint{5.151455in}{2.907883in}}%
\pgfpathlineto{\pgfqpoint{5.154817in}{2.907007in}}%
\pgfpathlineto{\pgfqpoint{5.157058in}{2.915764in}}%
\pgfpathlineto{\pgfqpoint{5.158179in}{2.918391in}}%
\pgfpathlineto{\pgfqpoint{5.159299in}{2.911386in}}%
\pgfpathlineto{\pgfqpoint{5.162661in}{2.910948in}}%
\pgfpathlineto{\pgfqpoint{5.163781in}{2.907883in}}%
\pgfpathlineto{\pgfqpoint{5.164902in}{2.912261in}}%
\pgfpathlineto{\pgfqpoint{5.166022in}{2.914013in}}%
\pgfpathlineto{\pgfqpoint{5.167143in}{2.905693in}}%
\pgfpathlineto{\pgfqpoint{5.172745in}{2.933278in}}%
\pgfpathlineto{\pgfqpoint{5.173866in}{2.932840in}}%
\pgfpathlineto{\pgfqpoint{5.174986in}{2.946852in}}%
\pgfpathlineto{\pgfqpoint{5.179469in}{2.945100in}}%
\pgfpathlineto{\pgfqpoint{5.180589in}{2.949041in}}%
\pgfpathlineto{\pgfqpoint{5.181710in}{2.943349in}}%
\pgfpathlineto{\pgfqpoint{5.182830in}{2.951230in}}%
\pgfpathlineto{\pgfqpoint{5.186192in}{2.936781in}}%
\pgfpathlineto{\pgfqpoint{5.187312in}{2.914888in}}%
\pgfpathlineto{\pgfqpoint{5.188433in}{2.909634in}}%
\pgfpathlineto{\pgfqpoint{5.189553in}{2.918829in}}%
\pgfpathlineto{\pgfqpoint{5.190674in}{2.897374in}}%
\pgfpathlineto{\pgfqpoint{5.194035in}{2.903066in}}%
\pgfpathlineto{\pgfqpoint{5.198518in}{2.965679in}}%
\pgfpathlineto{\pgfqpoint{5.201879in}{2.959549in}}%
\pgfpathlineto{\pgfqpoint{5.203000in}{2.950354in}}%
\pgfpathlineto{\pgfqpoint{5.204120in}{2.955609in}}%
\pgfpathlineto{\pgfqpoint{5.205241in}{2.939846in}}%
\pgfpathlineto{\pgfqpoint{5.206361in}{2.945100in}}%
\pgfpathlineto{\pgfqpoint{5.209723in}{2.944662in}}%
\pgfpathlineto{\pgfqpoint{5.210843in}{2.952982in}}%
\pgfpathlineto{\pgfqpoint{5.211964in}{2.933278in}}%
\pgfpathlineto{\pgfqpoint{5.213084in}{2.928462in}}%
\pgfpathlineto{\pgfqpoint{5.214205in}{2.943349in}}%
\pgfpathlineto{\pgfqpoint{5.217567in}{2.956047in}}%
\pgfpathlineto{\pgfqpoint{5.218687in}{2.942035in}}%
\pgfpathlineto{\pgfqpoint{5.219808in}{2.968306in}}%
\pgfpathlineto{\pgfqpoint{5.220928in}{2.935030in}}%
\pgfpathlineto{\pgfqpoint{5.222049in}{2.935467in}}%
\pgfpathlineto{\pgfqpoint{5.227651in}{2.899564in}}%
\pgfpathlineto{\pgfqpoint{5.228772in}{2.890806in}}%
\pgfpathlineto{\pgfqpoint{5.229892in}{2.904818in}}%
\pgfpathlineto{\pgfqpoint{5.233254in}{2.919267in}}%
\pgfpathlineto{\pgfqpoint{5.234374in}{2.926710in}}%
\pgfpathlineto{\pgfqpoint{5.235495in}{2.913137in}}%
\pgfpathlineto{\pgfqpoint{5.236615in}{2.910072in}}%
\pgfpathlineto{\pgfqpoint{5.237736in}{2.929338in}}%
\pgfpathlineto{\pgfqpoint{5.241098in}{2.952544in}}%
\pgfpathlineto{\pgfqpoint{5.242218in}{2.971371in}}%
\pgfpathlineto{\pgfqpoint{5.243339in}{2.966993in}}%
\pgfpathlineto{\pgfqpoint{5.244459in}{2.969182in}}%
\pgfpathlineto{\pgfqpoint{5.245580in}{2.981442in}}%
\pgfpathlineto{\pgfqpoint{5.248941in}{2.986696in}}%
\pgfpathlineto{\pgfqpoint{5.250062in}{2.984507in}}%
\pgfpathlineto{\pgfqpoint{5.251182in}{2.984945in}}%
\pgfpathlineto{\pgfqpoint{5.252303in}{2.982756in}}%
\pgfpathlineto{\pgfqpoint{5.253423in}{3.003335in}}%
\pgfpathlineto{\pgfqpoint{5.256785in}{2.998080in}}%
\pgfpathlineto{\pgfqpoint{5.257905in}{2.994140in}}%
\pgfpathlineto{\pgfqpoint{5.259026in}{3.001583in}}%
\pgfpathlineto{\pgfqpoint{5.261267in}{3.025665in}}%
\pgfpathlineto{\pgfqpoint{5.264629in}{3.022162in}}%
\pgfpathlineto{\pgfqpoint{5.265749in}{3.016032in}}%
\pgfpathlineto{\pgfqpoint{5.266870in}{2.990199in}}%
\pgfpathlineto{\pgfqpoint{5.267990in}{2.979691in}}%
\pgfpathlineto{\pgfqpoint{5.269111in}{2.980128in}}%
\pgfpathlineto{\pgfqpoint{5.273593in}{2.949479in}}%
\pgfpathlineto{\pgfqpoint{5.274713in}{2.974436in}}%
\pgfpathlineto{\pgfqpoint{5.276954in}{2.993702in}}%
\pgfpathlineto{\pgfqpoint{5.280316in}{2.973999in}}%
\pgfpathlineto{\pgfqpoint{5.281437in}{2.941597in}}%
\pgfpathlineto{\pgfqpoint{5.282557in}{2.930213in}}%
\pgfpathlineto{\pgfqpoint{5.283678in}{2.929775in}}%
\pgfpathlineto{\pgfqpoint{5.284798in}{2.923645in}}%
\pgfpathlineto{\pgfqpoint{5.288160in}{2.934154in}}%
\pgfpathlineto{\pgfqpoint{5.289280in}{2.864097in}}%
\pgfpathlineto{\pgfqpoint{5.290401in}{2.838264in}}%
\pgfpathlineto{\pgfqpoint{5.291521in}{2.844394in}}%
\pgfpathlineto{\pgfqpoint{5.292642in}{2.816809in}}%
\pgfpathlineto{\pgfqpoint{5.296003in}{2.811555in}}%
\pgfpathlineto{\pgfqpoint{5.297124in}{2.815058in}}%
\pgfpathlineto{\pgfqpoint{5.299365in}{2.868914in}}%
\pgfpathlineto{\pgfqpoint{5.300486in}{2.867600in}}%
\pgfpathlineto{\pgfqpoint{5.304968in}{2.890806in}}%
\pgfpathlineto{\pgfqpoint{5.306088in}{2.890806in}}%
\pgfpathlineto{\pgfqpoint{5.308329in}{2.897374in}}%
\pgfpathlineto{\pgfqpoint{5.311691in}{2.886428in}}%
\pgfpathlineto{\pgfqpoint{5.312811in}{2.878547in}}%
\pgfpathlineto{\pgfqpoint{5.313932in}{2.859281in}}%
\pgfpathlineto{\pgfqpoint{5.316173in}{2.865849in}}%
\pgfpathlineto{\pgfqpoint{5.319534in}{2.851838in}}%
\pgfpathlineto{\pgfqpoint{5.320655in}{2.868476in}}%
\pgfpathlineto{\pgfqpoint{5.321776in}{2.857968in}}%
\pgfpathlineto{\pgfqpoint{5.322896in}{2.892996in}}%
\pgfpathlineto{\pgfqpoint{5.324017in}{2.878109in}}%
\pgfpathlineto{\pgfqpoint{5.328499in}{2.892996in}}%
\pgfpathlineto{\pgfqpoint{5.329619in}{2.885552in}}%
\pgfpathlineto{\pgfqpoint{5.330740in}{2.890369in}}%
\pgfpathlineto{\pgfqpoint{5.331860in}{2.921894in}}%
\pgfpathlineto{\pgfqpoint{5.337463in}{2.931089in}}%
\pgfpathlineto{\pgfqpoint{5.339704in}{2.892120in}}%
\pgfpathlineto{\pgfqpoint{5.343066in}{2.885114in}}%
\pgfpathlineto{\pgfqpoint{5.345307in}{2.852713in}}%
\pgfpathlineto{\pgfqpoint{5.346427in}{2.854903in}}%
\pgfpathlineto{\pgfqpoint{5.347548in}{2.840453in}}%
\pgfpathlineto{\pgfqpoint{5.350909in}{2.885990in}}%
\pgfpathlineto{\pgfqpoint{5.352030in}{2.916202in}}%
\pgfpathlineto{\pgfqpoint{5.353150in}{2.915326in}}%
\pgfpathlineto{\pgfqpoint{5.354271in}{2.917516in}}%
\pgfpathlineto{\pgfqpoint{5.355391in}{2.969620in}}%
\pgfpathlineto{\pgfqpoint{5.358753in}{2.961301in}}%
\pgfpathlineto{\pgfqpoint{5.360994in}{2.987134in}}%
\pgfpathlineto{\pgfqpoint{5.363235in}{2.969182in}}%
\pgfpathlineto{\pgfqpoint{5.367717in}{2.964366in}}%
\pgfpathlineto{\pgfqpoint{5.368838in}{2.956047in}}%
\pgfpathlineto{\pgfqpoint{5.369958in}{2.954733in}}%
\pgfpathlineto{\pgfqpoint{5.371079in}{2.956922in}}%
\pgfpathlineto{\pgfqpoint{5.374440in}{2.949917in}}%
\pgfpathlineto{\pgfqpoint{5.375561in}{2.965679in}}%
\pgfpathlineto{\pgfqpoint{5.376681in}{2.965241in}}%
\pgfpathlineto{\pgfqpoint{5.377802in}{2.971371in}}%
\pgfpathlineto{\pgfqpoint{5.378922in}{2.973999in}}%
\pgfpathlineto{\pgfqpoint{5.382284in}{2.974874in}}%
\pgfpathlineto{\pgfqpoint{5.383405in}{2.977501in}}%
\pgfpathlineto{\pgfqpoint{5.384525in}{2.960425in}}%
\pgfpathlineto{\pgfqpoint{5.385646in}{2.955171in}}%
\pgfpathlineto{\pgfqpoint{5.386766in}{2.932840in}}%
\pgfpathlineto{\pgfqpoint{5.390128in}{2.930651in}}%
\pgfpathlineto{\pgfqpoint{5.391248in}{2.904818in}}%
\pgfpathlineto{\pgfqpoint{5.392369in}{2.910948in}}%
\pgfpathlineto{\pgfqpoint{5.393489in}{2.948603in}}%
\pgfpathlineto{\pgfqpoint{5.394610in}{2.952106in}}%
\pgfpathlineto{\pgfqpoint{5.397971in}{2.967869in}}%
\pgfpathlineto{\pgfqpoint{5.399092in}{2.956047in}}%
\pgfpathlineto{\pgfqpoint{5.400212in}{2.977501in}}%
\pgfpathlineto{\pgfqpoint{5.401333in}{2.968744in}}%
\pgfpathlineto{\pgfqpoint{5.402453in}{2.977939in}}%
\pgfpathlineto{\pgfqpoint{5.405815in}{2.981004in}}%
\pgfpathlineto{\pgfqpoint{5.406936in}{2.972685in}}%
\pgfpathlineto{\pgfqpoint{5.408056in}{2.948603in}}%
\pgfpathlineto{\pgfqpoint{5.409177in}{2.937219in}}%
\pgfpathlineto{\pgfqpoint{5.410297in}{2.942473in}}%
\pgfpathlineto{\pgfqpoint{5.413659in}{2.962177in}}%
\pgfpathlineto{\pgfqpoint{5.414779in}{2.944662in}}%
\pgfpathlineto{\pgfqpoint{5.415900in}{2.955171in}}%
\pgfpathlineto{\pgfqpoint{5.417020in}{2.974874in}}%
\pgfpathlineto{\pgfqpoint{5.421502in}{2.980566in}}%
\pgfpathlineto{\pgfqpoint{5.422623in}{2.967869in}}%
\pgfpathlineto{\pgfqpoint{5.423744in}{2.982318in}}%
\pgfpathlineto{\pgfqpoint{5.424864in}{2.977939in}}%
\pgfpathlineto{\pgfqpoint{5.425985in}{2.985821in}}%
\pgfpathlineto{\pgfqpoint{5.429346in}{2.979253in}}%
\pgfpathlineto{\pgfqpoint{5.430467in}{2.984069in}}%
\pgfpathlineto{\pgfqpoint{5.431587in}{2.991951in}}%
\pgfpathlineto{\pgfqpoint{5.432708in}{2.987572in}}%
\pgfpathlineto{\pgfqpoint{5.433828in}{2.973999in}}%
\pgfpathlineto{\pgfqpoint{5.437190in}{2.991513in}}%
\pgfpathlineto{\pgfqpoint{5.438310in}{2.984069in}}%
\pgfpathlineto{\pgfqpoint{5.440551in}{3.015595in}}%
\pgfpathlineto{\pgfqpoint{5.441672in}{3.014719in}}%
\pgfpathlineto{\pgfqpoint{5.445034in}{3.016908in}}%
\pgfpathlineto{\pgfqpoint{5.446154in}{3.033547in}}%
\pgfpathlineto{\pgfqpoint{5.448395in}{3.029606in}}%
\pgfpathlineto{\pgfqpoint{5.449516in}{3.028730in}}%
\pgfpathlineto{\pgfqpoint{5.452877in}{3.032671in}}%
\pgfpathlineto{\pgfqpoint{5.455118in}{3.000708in}}%
\pgfpathlineto{\pgfqpoint{5.456239in}{3.004210in}}%
\pgfpathlineto{\pgfqpoint{5.457359in}{3.019097in}}%
\pgfpathlineto{\pgfqpoint{5.460721in}{3.006838in}}%
\pgfpathlineto{\pgfqpoint{5.461841in}{3.000270in}}%
\pgfpathlineto{\pgfqpoint{5.462962in}{3.004210in}}%
\pgfpathlineto{\pgfqpoint{5.464082in}{3.012967in}}%
\pgfpathlineto{\pgfqpoint{5.465203in}{3.006400in}}%
\pgfpathlineto{\pgfqpoint{5.469685in}{2.997643in}}%
\pgfpathlineto{\pgfqpoint{5.470806in}{3.002459in}}%
\pgfpathlineto{\pgfqpoint{5.471926in}{3.010340in}}%
\pgfpathlineto{\pgfqpoint{5.473047in}{2.999832in}}%
\pgfpathlineto{\pgfqpoint{5.477529in}{2.992826in}}%
\pgfpathlineto{\pgfqpoint{5.478649in}{2.998518in}}%
\pgfpathlineto{\pgfqpoint{5.479770in}{2.997205in}}%
\pgfpathlineto{\pgfqpoint{5.480890in}{2.993702in}}%
\pgfpathlineto{\pgfqpoint{5.486493in}{2.980566in}}%
\pgfpathlineto{\pgfqpoint{5.488734in}{2.914013in}}%
\pgfpathlineto{\pgfqpoint{5.492096in}{2.921456in}}%
\pgfpathlineto{\pgfqpoint{5.493216in}{2.917953in}}%
\pgfpathlineto{\pgfqpoint{5.494337in}{2.922770in}}%
\pgfpathlineto{\pgfqpoint{5.495457in}{2.932840in}}%
\pgfpathlineto{\pgfqpoint{5.496578in}{2.914451in}}%
\pgfpathlineto{\pgfqpoint{5.499939in}{2.905693in}}%
\pgfpathlineto{\pgfqpoint{5.501060in}{2.920580in}}%
\pgfpathlineto{\pgfqpoint{5.502180in}{2.915326in}}%
\pgfpathlineto{\pgfqpoint{5.503301in}{2.933278in}}%
\pgfpathlineto{\pgfqpoint{5.504421in}{2.922332in}}%
\pgfpathlineto{\pgfqpoint{5.507783in}{2.924521in}}%
\pgfpathlineto{\pgfqpoint{5.508904in}{2.933278in}}%
\pgfpathlineto{\pgfqpoint{5.510024in}{2.916202in}}%
\pgfpathlineto{\pgfqpoint{5.512265in}{2.928024in}}%
\pgfpathlineto{\pgfqpoint{5.516747in}{2.891682in}}%
\pgfpathlineto{\pgfqpoint{5.517868in}{2.905693in}}%
\pgfpathlineto{\pgfqpoint{5.518988in}{2.912699in}}%
\pgfpathlineto{\pgfqpoint{5.523470in}{2.907007in}}%
\pgfpathlineto{\pgfqpoint{5.524591in}{2.914451in}}%
\pgfpathlineto{\pgfqpoint{5.525711in}{2.910072in}}%
\pgfpathlineto{\pgfqpoint{5.526832in}{2.898688in}}%
\pgfpathlineto{\pgfqpoint{5.527953in}{2.925397in}}%
\pgfpathlineto{\pgfqpoint{5.531314in}{2.932840in}}%
\pgfpathlineto{\pgfqpoint{5.532435in}{2.940722in}}%
\pgfpathlineto{\pgfqpoint{5.533555in}{2.937657in}}%
\pgfpathlineto{\pgfqpoint{5.534676in}{2.956047in}}%
\pgfpathlineto{\pgfqpoint{5.535796in}{2.947290in}}%
\pgfpathlineto{\pgfqpoint{5.539158in}{2.965679in}}%
\pgfpathlineto{\pgfqpoint{5.540278in}{2.924521in}}%
\pgfpathlineto{\pgfqpoint{5.541399in}{2.905693in}}%
\pgfpathlineto{\pgfqpoint{5.542519in}{2.901753in}}%
\pgfpathlineto{\pgfqpoint{5.543640in}{2.890369in}}%
\pgfpathlineto{\pgfqpoint{5.547001in}{2.882925in}}%
\pgfpathlineto{\pgfqpoint{5.548122in}{2.885114in}}%
\pgfpathlineto{\pgfqpoint{5.549243in}{2.909634in}}%
\pgfpathlineto{\pgfqpoint{5.551484in}{2.917953in}}%
\pgfpathlineto{\pgfqpoint{5.554845in}{2.924521in}}%
\pgfpathlineto{\pgfqpoint{5.555966in}{2.913575in}}%
\pgfpathlineto{\pgfqpoint{5.557086in}{2.912261in}}%
\pgfpathlineto{\pgfqpoint{5.558207in}{2.912261in}}%
\pgfpathlineto{\pgfqpoint{5.559327in}{2.902191in}}%
\pgfpathlineto{\pgfqpoint{5.564930in}{2.956484in}}%
\pgfpathlineto{\pgfqpoint{5.567171in}{2.943349in}}%
\pgfpathlineto{\pgfqpoint{5.570533in}{2.944662in}}%
\pgfpathlineto{\pgfqpoint{5.571653in}{2.942473in}}%
\pgfpathlineto{\pgfqpoint{5.572774in}{2.941597in}}%
\pgfpathlineto{\pgfqpoint{5.578376in}{2.843081in}}%
\pgfpathlineto{\pgfqpoint{5.579497in}{2.797982in}}%
\pgfpathlineto{\pgfqpoint{5.581738in}{2.896061in}}%
\pgfpathlineto{\pgfqpoint{5.582858in}{2.891682in}}%
\pgfpathlineto{\pgfqpoint{5.586220in}{2.889493in}}%
\pgfpathlineto{\pgfqpoint{5.587340in}{2.848773in}}%
\pgfpathlineto{\pgfqpoint{5.589582in}{2.878984in}}%
\pgfpathlineto{\pgfqpoint{5.590702in}{2.846145in}}%
\pgfpathlineto{\pgfqpoint{5.595184in}{2.884677in}}%
\pgfpathlineto{\pgfqpoint{5.596305in}{2.866725in}}%
\pgfpathlineto{\pgfqpoint{5.597425in}{2.869352in}}%
\pgfpathlineto{\pgfqpoint{5.598546in}{2.879422in}}%
\pgfpathlineto{\pgfqpoint{5.601907in}{2.876357in}}%
\pgfpathlineto{\pgfqpoint{5.603028in}{2.902629in}}%
\pgfpathlineto{\pgfqpoint{5.604148in}{2.896061in}}%
\pgfpathlineto{\pgfqpoint{5.606389in}{2.836951in}}%
\pgfpathlineto{\pgfqpoint{5.609751in}{2.845270in}}%
\pgfpathlineto{\pgfqpoint{5.611992in}{2.815496in}}%
\pgfpathlineto{\pgfqpoint{5.614233in}{2.824253in}}%
\pgfpathlineto{\pgfqpoint{5.619836in}{2.797982in}}%
\pgfpathlineto{\pgfqpoint{5.620956in}{2.778278in}}%
\pgfpathlineto{\pgfqpoint{5.622077in}{2.773900in}}%
\pgfpathlineto{\pgfqpoint{5.625438in}{2.815934in}}%
\pgfpathlineto{\pgfqpoint{5.626559in}{2.818123in}}%
\pgfpathlineto{\pgfqpoint{5.628800in}{2.845270in}}%
\pgfpathlineto{\pgfqpoint{5.629921in}{2.842643in}}%
\pgfpathlineto{\pgfqpoint{5.634403in}{2.850086in}}%
\pgfpathlineto{\pgfqpoint{5.635523in}{2.836513in}}%
\pgfpathlineto{\pgfqpoint{5.636644in}{2.861908in}}%
\pgfpathlineto{\pgfqpoint{5.637764in}{2.862784in}}%
\pgfpathlineto{\pgfqpoint{5.641126in}{2.862784in}}%
\pgfpathlineto{\pgfqpoint{5.642246in}{2.882925in}}%
\pgfpathlineto{\pgfqpoint{5.643367in}{2.868914in}}%
\pgfpathlineto{\pgfqpoint{5.644487in}{2.907007in}}%
\pgfpathlineto{\pgfqpoint{5.645608in}{2.916640in}}%
\pgfpathlineto{\pgfqpoint{5.648969in}{2.924083in}}%
\pgfpathlineto{\pgfqpoint{5.650090in}{2.917078in}}%
\pgfpathlineto{\pgfqpoint{5.651211in}{2.928462in}}%
\pgfpathlineto{\pgfqpoint{5.652331in}{2.925835in}}%
\pgfpathlineto{\pgfqpoint{5.653452in}{2.943349in}}%
\pgfpathlineto{\pgfqpoint{5.656813in}{2.939846in}}%
\pgfpathlineto{\pgfqpoint{5.659054in}{2.916202in}}%
\pgfpathlineto{\pgfqpoint{5.660175in}{2.918391in}}%
\pgfpathlineto{\pgfqpoint{5.661295in}{2.902629in}}%
\pgfpathlineto{\pgfqpoint{5.665777in}{2.877671in}}%
\pgfpathlineto{\pgfqpoint{5.666898in}{2.885552in}}%
\pgfpathlineto{\pgfqpoint{5.669139in}{2.845270in}}%
\pgfpathlineto{\pgfqpoint{5.672501in}{2.875482in}}%
\pgfpathlineto{\pgfqpoint{5.673621in}{2.876795in}}%
\pgfpathlineto{\pgfqpoint{5.675862in}{2.902191in}}%
\pgfpathlineto{\pgfqpoint{5.676983in}{2.888179in}}%
\pgfpathlineto{\pgfqpoint{5.680344in}{2.873730in}}%
\pgfpathlineto{\pgfqpoint{5.681465in}{2.880736in}}%
\pgfpathlineto{\pgfqpoint{5.682585in}{2.871103in}}%
\pgfpathlineto{\pgfqpoint{5.684826in}{2.882487in}}%
\pgfpathlineto{\pgfqpoint{5.688188in}{2.890369in}}%
\pgfpathlineto{\pgfqpoint{5.689308in}{2.895185in}}%
\pgfpathlineto{\pgfqpoint{5.691549in}{2.857530in}}%
\pgfpathlineto{\pgfqpoint{5.692670in}{2.900001in}}%
\pgfpathlineto{\pgfqpoint{5.696032in}{2.913137in}}%
\pgfpathlineto{\pgfqpoint{5.698273in}{2.887304in}}%
\pgfpathlineto{\pgfqpoint{5.699393in}{2.885552in}}%
\pgfpathlineto{\pgfqpoint{5.700514in}{2.867162in}}%
\pgfpathlineto{\pgfqpoint{5.704996in}{2.894309in}}%
\pgfpathlineto{\pgfqpoint{5.706116in}{2.929775in}}%
\pgfpathlineto{\pgfqpoint{5.708357in}{2.894747in}}%
\pgfpathlineto{\pgfqpoint{5.711719in}{2.907007in}}%
\pgfpathlineto{\pgfqpoint{5.713960in}{2.945976in}}%
\pgfpathlineto{\pgfqpoint{5.715081in}{2.937219in}}%
\pgfpathlineto{\pgfqpoint{5.719563in}{2.938532in}}%
\pgfpathlineto{\pgfqpoint{5.720683in}{2.955609in}}%
\pgfpathlineto{\pgfqpoint{5.722924in}{2.918829in}}%
\pgfpathlineto{\pgfqpoint{5.727406in}{2.906131in}}%
\pgfpathlineto{\pgfqpoint{5.728527in}{2.929338in}}%
\pgfpathlineto{\pgfqpoint{5.731888in}{2.888179in}}%
\pgfpathlineto{\pgfqpoint{5.735250in}{2.897812in}}%
\pgfpathlineto{\pgfqpoint{5.736371in}{2.891682in}}%
\pgfpathlineto{\pgfqpoint{5.737491in}{2.862346in}}%
\pgfpathlineto{\pgfqpoint{5.738612in}{2.889493in}}%
\pgfpathlineto{\pgfqpoint{5.739732in}{2.872855in}}%
\pgfpathlineto{\pgfqpoint{5.744214in}{2.889493in}}%
\pgfpathlineto{\pgfqpoint{5.745335in}{2.872855in}}%
\pgfpathlineto{\pgfqpoint{5.747576in}{2.970934in}}%
\pgfpathlineto{\pgfqpoint{5.750937in}{2.970496in}}%
\pgfpathlineto{\pgfqpoint{5.753178in}{3.045369in}}%
\pgfpathlineto{\pgfqpoint{5.754299in}{3.044493in}}%
\pgfpathlineto{\pgfqpoint{5.755420in}{3.080397in}}%
\pgfpathlineto{\pgfqpoint{5.758781in}{3.110171in}}%
\pgfpathlineto{\pgfqpoint{5.759902in}{3.078208in}}%
\pgfpathlineto{\pgfqpoint{5.761022in}{3.104917in}}%
\pgfpathlineto{\pgfqpoint{5.762143in}{3.097473in}}%
\pgfpathlineto{\pgfqpoint{5.763263in}{3.118052in}}%
\pgfpathlineto{\pgfqpoint{5.766625in}{3.109295in}}%
\pgfpathlineto{\pgfqpoint{5.767745in}{3.086965in}}%
\pgfpathlineto{\pgfqpoint{5.768866in}{3.080835in}}%
\pgfpathlineto{\pgfqpoint{5.769986in}{3.058504in}}%
\pgfpathlineto{\pgfqpoint{5.771107in}{3.085651in}}%
\pgfpathlineto{\pgfqpoint{5.775589in}{3.090467in}}%
\pgfpathlineto{\pgfqpoint{5.776710in}{3.093532in}}%
\pgfpathlineto{\pgfqpoint{5.777830in}{3.116739in}}%
\pgfpathlineto{\pgfqpoint{5.778951in}{3.113674in}}%
\pgfpathlineto{\pgfqpoint{5.782312in}{3.121555in}}%
\pgfpathlineto{\pgfqpoint{5.783433in}{3.104917in}}%
\pgfpathlineto{\pgfqpoint{5.785674in}{3.122869in}}%
\pgfpathlineto{\pgfqpoint{5.786794in}{3.119804in}}%
\pgfpathlineto{\pgfqpoint{5.790156in}{3.108857in}}%
\pgfpathlineto{\pgfqpoint{5.792397in}{3.160962in}}%
\pgfpathlineto{\pgfqpoint{5.793517in}{3.152205in}}%
\pgfpathlineto{\pgfqpoint{5.794638in}{3.149140in}}%
\pgfpathlineto{\pgfqpoint{5.798000in}{3.164465in}}%
\pgfpathlineto{\pgfqpoint{5.799120in}{3.173659in}}%
\pgfpathlineto{\pgfqpoint{5.800241in}{3.169281in}}%
\pgfpathlineto{\pgfqpoint{5.801361in}{3.168405in}}%
\pgfpathlineto{\pgfqpoint{5.802482in}{3.176287in}}%
\pgfpathlineto{\pgfqpoint{5.805843in}{3.176724in}}%
\pgfpathlineto{\pgfqpoint{5.806964in}{3.181541in}}%
\pgfpathlineto{\pgfqpoint{5.809205in}{3.217445in}}%
\pgfpathlineto{\pgfqpoint{5.810325in}{3.202996in}}%
\pgfpathlineto{\pgfqpoint{5.813687in}{3.210439in}}%
\pgfpathlineto{\pgfqpoint{5.815928in}{3.190298in}}%
\pgfpathlineto{\pgfqpoint{5.817049in}{3.214818in}}%
\pgfpathlineto{\pgfqpoint{5.821531in}{3.208688in}}%
\pgfpathlineto{\pgfqpoint{5.822651in}{3.233207in}}%
\pgfpathlineto{\pgfqpoint{5.823772in}{3.232770in}}%
\pgfpathlineto{\pgfqpoint{5.824892in}{3.234083in}}%
\pgfpathlineto{\pgfqpoint{5.826013in}{3.231456in}}%
\pgfpathlineto{\pgfqpoint{5.829374in}{3.246781in}}%
\pgfpathlineto{\pgfqpoint{5.830495in}{3.234521in}}%
\pgfpathlineto{\pgfqpoint{5.831615in}{3.234521in}}%
\pgfpathlineto{\pgfqpoint{5.832736in}{3.177162in}}%
\pgfpathlineto{\pgfqpoint{5.833856in}{3.183730in}}%
\pgfpathlineto{\pgfqpoint{5.837218in}{3.162275in}}%
\pgfpathlineto{\pgfqpoint{5.838339in}{3.174973in}}%
\pgfpathlineto{\pgfqpoint{5.839459in}{3.150015in}}%
\pgfpathlineto{\pgfqpoint{5.840580in}{3.152643in}}%
\pgfpathlineto{\pgfqpoint{5.841700in}{3.152205in}}%
\pgfpathlineto{\pgfqpoint{5.845062in}{3.166654in}}%
\pgfpathlineto{\pgfqpoint{5.846182in}{3.179789in}}%
\pgfpathlineto{\pgfqpoint{5.847303in}{3.167530in}}%
\pgfpathlineto{\pgfqpoint{5.848423in}{3.102289in}}%
\pgfpathlineto{\pgfqpoint{5.849544in}{3.121993in}}%
\pgfpathlineto{\pgfqpoint{5.852905in}{3.129874in}}%
\pgfpathlineto{\pgfqpoint{5.854026in}{3.118052in}}%
\pgfpathlineto{\pgfqpoint{5.855146in}{3.165340in}}%
\pgfpathlineto{\pgfqpoint{5.856267in}{3.139945in}}%
\pgfpathlineto{\pgfqpoint{5.857388in}{3.136880in}}%
\pgfpathlineto{\pgfqpoint{5.860749in}{3.151329in}}%
\pgfpathlineto{\pgfqpoint{5.861870in}{3.126809in}}%
\pgfpathlineto{\pgfqpoint{5.862990in}{3.132939in}}%
\pgfpathlineto{\pgfqpoint{5.864111in}{3.132939in}}%
\pgfpathlineto{\pgfqpoint{5.865231in}{3.143448in}}%
\pgfpathlineto{\pgfqpoint{5.868593in}{3.142134in}}%
\pgfpathlineto{\pgfqpoint{5.869713in}{3.159648in}}%
\pgfpathlineto{\pgfqpoint{5.870834in}{3.144761in}}%
\pgfpathlineto{\pgfqpoint{5.871954in}{3.157021in}}%
\pgfpathlineto{\pgfqpoint{5.873075in}{3.136880in}}%
\pgfpathlineto{\pgfqpoint{5.876436in}{3.147826in}}%
\pgfpathlineto{\pgfqpoint{5.877557in}{3.135128in}}%
\pgfpathlineto{\pgfqpoint{5.879798in}{3.087402in}}%
\pgfpathlineto{\pgfqpoint{5.880919in}{3.088278in}}%
\pgfpathlineto{\pgfqpoint{5.884280in}{3.068575in}}%
\pgfpathlineto{\pgfqpoint{5.886521in}{3.095722in}}%
\pgfpathlineto{\pgfqpoint{5.888762in}{3.124620in}}%
\pgfpathlineto{\pgfqpoint{5.893244in}{3.135128in}}%
\pgfpathlineto{\pgfqpoint{5.894365in}{3.117614in}}%
\pgfpathlineto{\pgfqpoint{5.895485in}{3.130312in}}%
\pgfpathlineto{\pgfqpoint{5.896606in}{3.136004in}}%
\pgfpathlineto{\pgfqpoint{5.899968in}{3.128123in}}%
\pgfpathlineto{\pgfqpoint{5.901088in}{3.167530in}}%
\pgfpathlineto{\pgfqpoint{5.902209in}{3.158772in}}%
\pgfpathlineto{\pgfqpoint{5.903329in}{3.174973in}}%
\pgfpathlineto{\pgfqpoint{5.904450in}{3.202120in}}%
\pgfpathlineto{\pgfqpoint{5.907811in}{3.198617in}}%
\pgfpathlineto{\pgfqpoint{5.908932in}{3.214380in}}%
\pgfpathlineto{\pgfqpoint{5.910052in}{3.208688in}}%
\pgfpathlineto{\pgfqpoint{5.911173in}{3.232332in}}%
\pgfpathlineto{\pgfqpoint{5.912293in}{3.244154in}}%
\pgfpathlineto{\pgfqpoint{5.915655in}{3.243716in}}%
\pgfpathlineto{\pgfqpoint{5.916775in}{3.256414in}}%
\pgfpathlineto{\pgfqpoint{5.917896in}{3.253787in}}%
\pgfpathlineto{\pgfqpoint{5.919017in}{3.277868in}}%
\pgfpathlineto{\pgfqpoint{5.920137in}{3.268674in}}%
\pgfpathlineto{\pgfqpoint{5.924619in}{3.283561in}}%
\pgfpathlineto{\pgfqpoint{5.925740in}{3.292755in}}%
\pgfpathlineto{\pgfqpoint{5.927981in}{3.336979in}}%
\pgfpathlineto{\pgfqpoint{5.932463in}{3.348363in}}%
\pgfpathlineto{\pgfqpoint{5.933583in}{3.359747in}}%
\pgfpathlineto{\pgfqpoint{5.934704in}{3.326032in}}%
\pgfpathlineto{\pgfqpoint{5.935824in}{3.345736in}}%
\pgfpathlineto{\pgfqpoint{5.939186in}{3.347049in}}%
\pgfpathlineto{\pgfqpoint{5.940307in}{3.329535in}}%
\pgfpathlineto{\pgfqpoint{5.941427in}{3.349676in}}%
\pgfpathlineto{\pgfqpoint{5.942548in}{3.343546in}}%
\pgfpathlineto{\pgfqpoint{5.943668in}{3.343546in}}%
\pgfpathlineto{\pgfqpoint{5.947030in}{3.347049in}}%
\pgfpathlineto{\pgfqpoint{5.948150in}{3.338292in}}%
\pgfpathlineto{\pgfqpoint{5.949271in}{3.335227in}}%
\pgfpathlineto{\pgfqpoint{5.950391in}{3.325594in}}%
\pgfpathlineto{\pgfqpoint{5.951512in}{3.353617in}}%
\pgfpathlineto{\pgfqpoint{5.954873in}{3.344860in}}%
\pgfpathlineto{\pgfqpoint{5.955994in}{3.304140in}}%
\pgfpathlineto{\pgfqpoint{5.957114in}{3.323843in}}%
\pgfpathlineto{\pgfqpoint{5.958235in}{3.305891in}}%
\pgfpathlineto{\pgfqpoint{5.959355in}{3.326908in}}%
\pgfpathlineto{\pgfqpoint{5.962717in}{3.292318in}}%
\pgfpathlineto{\pgfqpoint{5.963838in}{3.273052in}}%
\pgfpathlineto{\pgfqpoint{5.964958in}{3.269549in}}%
\pgfpathlineto{\pgfqpoint{5.966079in}{3.270425in}}%
\pgfpathlineto{\pgfqpoint{5.967199in}{3.259479in}}%
\pgfpathlineto{\pgfqpoint{5.970561in}{3.257727in}}%
\pgfpathlineto{\pgfqpoint{5.971681in}{3.260354in}}%
\pgfpathlineto{\pgfqpoint{5.972802in}{3.266046in}}%
\pgfpathlineto{\pgfqpoint{5.973922in}{3.267798in}}%
\pgfpathlineto{\pgfqpoint{5.975043in}{3.259916in}}%
\pgfpathlineto{\pgfqpoint{5.978404in}{3.258165in}}%
\pgfpathlineto{\pgfqpoint{5.979525in}{3.225764in}}%
\pgfpathlineto{\pgfqpoint{5.980646in}{3.241527in}}%
\pgfpathlineto{\pgfqpoint{5.982887in}{3.213942in}}%
\pgfpathlineto{\pgfqpoint{5.986248in}{3.217883in}}%
\pgfpathlineto{\pgfqpoint{5.987369in}{3.220948in}}%
\pgfpathlineto{\pgfqpoint{5.988489in}{3.217007in}}%
\pgfpathlineto{\pgfqpoint{5.989610in}{3.225326in}}%
\pgfpathlineto{\pgfqpoint{5.990730in}{3.199493in}}%
\pgfpathlineto{\pgfqpoint{5.994092in}{3.215693in}}%
\pgfpathlineto{\pgfqpoint{5.995212in}{3.206936in}}%
\pgfpathlineto{\pgfqpoint{5.996333in}{3.209563in}}%
\pgfpathlineto{\pgfqpoint{5.998574in}{3.230580in}}%
\pgfpathlineto{\pgfqpoint{6.004177in}{3.262106in}}%
\pgfpathlineto{\pgfqpoint{6.005297in}{3.257727in}}%
\pgfpathlineto{\pgfqpoint{6.006418in}{3.189860in}}%
\pgfpathlineto{\pgfqpoint{6.009779in}{3.218758in}}%
\pgfpathlineto{\pgfqpoint{6.010900in}{3.175849in}}%
\pgfpathlineto{\pgfqpoint{6.012020in}{3.177162in}}%
\pgfpathlineto{\pgfqpoint{6.013141in}{3.195990in}}%
\pgfpathlineto{\pgfqpoint{6.014261in}{3.192049in}}%
\pgfpathlineto{\pgfqpoint{6.017623in}{3.166216in}}%
\pgfpathlineto{\pgfqpoint{6.018743in}{3.168843in}}%
\pgfpathlineto{\pgfqpoint{6.020984in}{3.210001in}}%
\pgfpathlineto{\pgfqpoint{6.022105in}{3.218320in}}%
\pgfpathlineto{\pgfqpoint{6.025467in}{3.202558in}}%
\pgfpathlineto{\pgfqpoint{6.026587in}{3.215693in}}%
\pgfpathlineto{\pgfqpoint{6.027708in}{3.199055in}}%
\pgfpathlineto{\pgfqpoint{6.028828in}{3.201244in}}%
\pgfpathlineto{\pgfqpoint{6.029949in}{3.195990in}}%
\pgfpathlineto{\pgfqpoint{6.033310in}{3.192049in}}%
\pgfpathlineto{\pgfqpoint{6.035551in}{3.152643in}}%
\pgfpathlineto{\pgfqpoint{6.036672in}{3.152205in}}%
\pgfpathlineto{\pgfqpoint{6.037792in}{3.139069in}}%
\pgfpathlineto{\pgfqpoint{6.041154in}{3.149578in}}%
\pgfpathlineto{\pgfqpoint{6.042274in}{3.138193in}}%
\pgfpathlineto{\pgfqpoint{6.043395in}{3.153956in}}%
\pgfpathlineto{\pgfqpoint{6.045636in}{3.153080in}}%
\pgfpathlineto{\pgfqpoint{6.048998in}{3.158772in}}%
\pgfpathlineto{\pgfqpoint{6.050118in}{3.152643in}}%
\pgfpathlineto{\pgfqpoint{6.051239in}{3.157021in}}%
\pgfpathlineto{\pgfqpoint{6.053480in}{3.072515in}}%
\pgfpathlineto{\pgfqpoint{6.056841in}{3.072953in}}%
\pgfpathlineto{\pgfqpoint{6.059082in}{3.050623in}}%
\pgfpathlineto{\pgfqpoint{6.060203in}{3.085651in}}%
\pgfpathlineto{\pgfqpoint{6.061323in}{3.072953in}}%
\pgfpathlineto{\pgfqpoint{6.064685in}{3.068575in}}%
\pgfpathlineto{\pgfqpoint{6.065806in}{3.051936in}}%
\pgfpathlineto{\pgfqpoint{6.066926in}{3.023914in}}%
\pgfpathlineto{\pgfqpoint{6.068047in}{3.021287in}}%
\pgfpathlineto{\pgfqpoint{6.069167in}{3.029168in}}%
\pgfpathlineto{\pgfqpoint{6.073649in}{3.051499in}}%
\pgfpathlineto{\pgfqpoint{6.074770in}{3.059380in}}%
\pgfpathlineto{\pgfqpoint{6.075890in}{3.014281in}}%
\pgfpathlineto{\pgfqpoint{6.077011in}{3.014281in}}%
\pgfpathlineto{\pgfqpoint{6.080372in}{2.994578in}}%
\pgfpathlineto{\pgfqpoint{6.081493in}{3.040552in}}%
\pgfpathlineto{\pgfqpoint{6.082613in}{3.062007in}}%
\pgfpathlineto{\pgfqpoint{6.083734in}{3.058504in}}%
\pgfpathlineto{\pgfqpoint{6.084855in}{3.067699in}}%
\pgfpathlineto{\pgfqpoint{6.088216in}{3.076894in}}%
\pgfpathlineto{\pgfqpoint{6.090457in}{3.150891in}}%
\pgfpathlineto{\pgfqpoint{6.092698in}{3.167967in}}%
\pgfpathlineto{\pgfqpoint{6.096060in}{3.185481in}}%
\pgfpathlineto{\pgfqpoint{6.097180in}{3.179352in}}%
\pgfpathlineto{\pgfqpoint{6.098301in}{3.138193in}}%
\pgfpathlineto{\pgfqpoint{6.099421in}{3.137318in}}%
\pgfpathlineto{\pgfqpoint{6.100542in}{3.134691in}}%
\pgfpathlineto{\pgfqpoint{6.103903in}{3.132501in}}%
\pgfpathlineto{\pgfqpoint{6.105024in}{3.156145in}}%
\pgfpathlineto{\pgfqpoint{6.106145in}{3.195552in}}%
\pgfpathlineto{\pgfqpoint{6.107265in}{3.185919in}}%
\pgfpathlineto{\pgfqpoint{6.108386in}{3.199931in}}%
\pgfpathlineto{\pgfqpoint{6.111747in}{3.210001in}}%
\pgfpathlineto{\pgfqpoint{6.112868in}{3.233207in}}%
\pgfpathlineto{\pgfqpoint{6.113988in}{3.205185in}}%
\pgfpathlineto{\pgfqpoint{6.115109in}{3.212191in}}%
\pgfpathlineto{\pgfqpoint{6.116229in}{3.229705in}}%
\pgfpathlineto{\pgfqpoint{6.120711in}{3.262544in}}%
\pgfpathlineto{\pgfqpoint{6.121832in}{3.256852in}}%
\pgfpathlineto{\pgfqpoint{6.122952in}{3.283123in}}%
\pgfpathlineto{\pgfqpoint{6.124073in}{3.284436in}}%
\pgfpathlineto{\pgfqpoint{6.128555in}{3.282685in}}%
\pgfpathlineto{\pgfqpoint{6.129676in}{3.275241in}}%
\pgfpathlineto{\pgfqpoint{6.130796in}{3.286626in}}%
\pgfpathlineto{\pgfqpoint{6.131917in}{3.272614in}}%
\pgfpathlineto{\pgfqpoint{6.136399in}{3.319027in}}%
\pgfpathlineto{\pgfqpoint{6.137519in}{3.316837in}}%
\pgfpathlineto{\pgfqpoint{6.138640in}{3.321654in}}%
\pgfpathlineto{\pgfqpoint{6.139760in}{3.290128in}}%
\pgfpathlineto{\pgfqpoint{6.143122in}{3.267360in}}%
\pgfpathlineto{\pgfqpoint{6.144242in}{3.270425in}}%
\pgfpathlineto{\pgfqpoint{6.145363in}{3.259041in}}%
\pgfpathlineto{\pgfqpoint{6.146484in}{3.267360in}}%
\pgfpathlineto{\pgfqpoint{6.147604in}{3.262544in}}%
\pgfpathlineto{\pgfqpoint{6.152086in}{3.269987in}}%
\pgfpathlineto{\pgfqpoint{6.153207in}{3.250722in}}%
\pgfpathlineto{\pgfqpoint{6.154327in}{3.255100in}}%
\pgfpathlineto{\pgfqpoint{6.155448in}{3.269111in}}%
\pgfpathlineto{\pgfqpoint{6.158809in}{3.256852in}}%
\pgfpathlineto{\pgfqpoint{6.159930in}{3.167530in}}%
\pgfpathlineto{\pgfqpoint{6.161050in}{3.153956in}}%
\pgfpathlineto{\pgfqpoint{6.162171in}{3.128561in}}%
\pgfpathlineto{\pgfqpoint{6.163291in}{3.146950in}}%
\pgfpathlineto{\pgfqpoint{6.166653in}{3.138193in}}%
\pgfpathlineto{\pgfqpoint{6.167774in}{3.124182in}}%
\pgfpathlineto{\pgfqpoint{6.168894in}{3.100100in}}%
\pgfpathlineto{\pgfqpoint{6.170015in}{3.095722in}}%
\pgfpathlineto{\pgfqpoint{6.171135in}{3.107544in}}%
\pgfpathlineto{\pgfqpoint{6.174497in}{3.085651in}}%
\pgfpathlineto{\pgfqpoint{6.175617in}{3.086089in}}%
\pgfpathlineto{\pgfqpoint{6.178979in}{3.122869in}}%
\pgfpathlineto{\pgfqpoint{6.182340in}{3.106230in}}%
\pgfpathlineto{\pgfqpoint{6.184581in}{3.087840in}}%
\pgfpathlineto{\pgfqpoint{6.185702in}{3.102727in}}%
\pgfpathlineto{\pgfqpoint{6.186822in}{3.131188in}}%
\pgfpathlineto{\pgfqpoint{6.191305in}{3.140383in}}%
\pgfpathlineto{\pgfqpoint{6.192425in}{3.150015in}}%
\pgfpathlineto{\pgfqpoint{6.193546in}{3.174973in}}%
\pgfpathlineto{\pgfqpoint{6.194666in}{3.186357in}}%
\pgfpathlineto{\pgfqpoint{6.199148in}{3.148264in}}%
\pgfpathlineto{\pgfqpoint{6.202510in}{3.166216in}}%
\pgfpathlineto{\pgfqpoint{6.205871in}{3.164027in}}%
\pgfpathlineto{\pgfqpoint{6.206992in}{3.140820in}}%
\pgfpathlineto{\pgfqpoint{6.208113in}{3.129874in}}%
\pgfpathlineto{\pgfqpoint{6.210354in}{3.137318in}}%
\pgfpathlineto{\pgfqpoint{6.213715in}{3.142134in}}%
\pgfpathlineto{\pgfqpoint{6.214836in}{3.137756in}}%
\pgfpathlineto{\pgfqpoint{6.215956in}{3.168405in}}%
\pgfpathlineto{\pgfqpoint{6.217077in}{3.164465in}}%
\pgfpathlineto{\pgfqpoint{6.218197in}{3.178038in}}%
\pgfpathlineto{\pgfqpoint{6.221559in}{3.172346in}}%
\pgfpathlineto{\pgfqpoint{6.222679in}{3.168843in}}%
\pgfpathlineto{\pgfqpoint{6.223800in}{3.151329in}}%
\pgfpathlineto{\pgfqpoint{6.224920in}{3.148702in}}%
\pgfpathlineto{\pgfqpoint{6.226041in}{3.150453in}}%
\pgfpathlineto{\pgfqpoint{6.229403in}{3.128998in}}%
\pgfpathlineto{\pgfqpoint{6.230523in}{3.135566in}}%
\pgfpathlineto{\pgfqpoint{6.231644in}{3.128998in}}%
\pgfpathlineto{\pgfqpoint{6.232764in}{3.125933in}}%
\pgfpathlineto{\pgfqpoint{6.233885in}{3.114111in}}%
\pgfpathlineto{\pgfqpoint{6.238367in}{3.136004in}}%
\pgfpathlineto{\pgfqpoint{6.239487in}{3.124620in}}%
\pgfpathlineto{\pgfqpoint{6.240608in}{3.124182in}}%
\pgfpathlineto{\pgfqpoint{6.241728in}{3.132939in}}%
\pgfpathlineto{\pgfqpoint{6.245090in}{3.128561in}}%
\pgfpathlineto{\pgfqpoint{6.247331in}{3.143448in}}%
\pgfpathlineto{\pgfqpoint{6.248451in}{3.131626in}}%
\pgfpathlineto{\pgfqpoint{6.252934in}{3.139069in}}%
\pgfpathlineto{\pgfqpoint{6.254054in}{3.155270in}}%
\pgfpathlineto{\pgfqpoint{6.255175in}{3.144323in}}%
\pgfpathlineto{\pgfqpoint{6.256295in}{3.123306in}}%
\pgfpathlineto{\pgfqpoint{6.257416in}{3.077332in}}%
\pgfpathlineto{\pgfqpoint{6.260777in}{3.069450in}}%
\pgfpathlineto{\pgfqpoint{6.261898in}{3.055439in}}%
\pgfpathlineto{\pgfqpoint{6.263018in}{3.081710in}}%
\pgfpathlineto{\pgfqpoint{6.265259in}{3.024352in}}%
\pgfpathlineto{\pgfqpoint{6.268621in}{3.023038in}}%
\pgfpathlineto{\pgfqpoint{6.269742in}{3.024352in}}%
\pgfpathlineto{\pgfqpoint{6.270862in}{3.033984in}}%
\pgfpathlineto{\pgfqpoint{6.271983in}{3.023038in}}%
\pgfpathlineto{\pgfqpoint{6.273103in}{3.055439in}}%
\pgfpathlineto{\pgfqpoint{6.276465in}{3.052812in}}%
\pgfpathlineto{\pgfqpoint{6.277585in}{3.044493in}}%
\pgfpathlineto{\pgfqpoint{6.278706in}{3.043179in}}%
\pgfpathlineto{\pgfqpoint{6.279826in}{3.028730in}}%
\pgfpathlineto{\pgfqpoint{6.284308in}{3.003335in}}%
\pgfpathlineto{\pgfqpoint{6.285429in}{3.000708in}}%
\pgfpathlineto{\pgfqpoint{6.286549in}{2.967869in}}%
\pgfpathlineto{\pgfqpoint{6.288790in}{3.005086in}}%
\pgfpathlineto{\pgfqpoint{6.292152in}{3.007275in}}%
\pgfpathlineto{\pgfqpoint{6.293273in}{3.007275in}}%
\pgfpathlineto{\pgfqpoint{6.294393in}{2.990199in}}%
\pgfpathlineto{\pgfqpoint{6.295514in}{3.000708in}}%
\pgfpathlineto{\pgfqpoint{6.296634in}{3.001145in}}%
\pgfpathlineto{\pgfqpoint{6.301116in}{3.035736in}}%
\pgfpathlineto{\pgfqpoint{6.302237in}{3.053250in}}%
\pgfpathlineto{\pgfqpoint{6.304478in}{3.045369in}}%
\pgfpathlineto{\pgfqpoint{6.307839in}{3.042741in}}%
\pgfpathlineto{\pgfqpoint{6.310080in}{3.047558in}}%
\pgfpathlineto{\pgfqpoint{6.311201in}{3.035298in}}%
\pgfpathlineto{\pgfqpoint{6.312322in}{3.056315in}}%
\pgfpathlineto{\pgfqpoint{6.315683in}{3.075143in}}%
\pgfpathlineto{\pgfqpoint{6.316804in}{3.046244in}}%
\pgfpathlineto{\pgfqpoint{6.317924in}{3.055439in}}%
\pgfpathlineto{\pgfqpoint{6.319045in}{3.053250in}}%
\pgfpathlineto{\pgfqpoint{6.320165in}{3.052812in}}%
\pgfpathlineto{\pgfqpoint{6.323527in}{3.050623in}}%
\pgfpathlineto{\pgfqpoint{6.325768in}{3.004648in}}%
\pgfpathlineto{\pgfqpoint{6.326888in}{3.005086in}}%
\pgfpathlineto{\pgfqpoint{6.328009in}{3.003773in}}%
\pgfpathlineto{\pgfqpoint{6.331370in}{3.018222in}}%
\pgfpathlineto{\pgfqpoint{6.332491in}{2.982318in}}%
\pgfpathlineto{\pgfqpoint{6.333612in}{2.982318in}}%
\pgfpathlineto{\pgfqpoint{6.334732in}{2.965241in}}%
\pgfpathlineto{\pgfqpoint{6.335853in}{2.974874in}}%
\pgfpathlineto{\pgfqpoint{6.339214in}{2.988448in}}%
\pgfpathlineto{\pgfqpoint{6.341455in}{2.974436in}}%
\pgfpathlineto{\pgfqpoint{6.342576in}{2.952544in}}%
\pgfpathlineto{\pgfqpoint{6.343696in}{2.950792in}}%
\pgfpathlineto{\pgfqpoint{6.347058in}{2.939846in}}%
\pgfpathlineto{\pgfqpoint{6.348178in}{2.927148in}}%
\pgfpathlineto{\pgfqpoint{6.350419in}{2.951230in}}%
\pgfpathlineto{\pgfqpoint{6.351540in}{2.954295in}}%
\pgfpathlineto{\pgfqpoint{6.354902in}{2.958236in}}%
\pgfpathlineto{\pgfqpoint{6.356022in}{2.945976in}}%
\pgfpathlineto{\pgfqpoint{6.357143in}{2.949479in}}%
\pgfpathlineto{\pgfqpoint{6.358263in}{2.980566in}}%
\pgfpathlineto{\pgfqpoint{6.359384in}{2.981004in}}%
\pgfpathlineto{\pgfqpoint{6.362745in}{2.959987in}}%
\pgfpathlineto{\pgfqpoint{6.363866in}{2.970934in}}%
\pgfpathlineto{\pgfqpoint{6.364986in}{2.987572in}}%
\pgfpathlineto{\pgfqpoint{6.366107in}{3.124182in}}%
\pgfpathlineto{\pgfqpoint{6.370589in}{3.147826in}}%
\pgfpathlineto{\pgfqpoint{6.371709in}{3.167530in}}%
\pgfpathlineto{\pgfqpoint{6.372830in}{3.140383in}}%
\pgfpathlineto{\pgfqpoint{6.375071in}{3.168405in}}%
\pgfpathlineto{\pgfqpoint{6.378433in}{3.166216in}}%
\pgfpathlineto{\pgfqpoint{6.381794in}{3.132501in}}%
\pgfpathlineto{\pgfqpoint{6.382915in}{3.134253in}}%
\pgfpathlineto{\pgfqpoint{6.386276in}{3.163151in}}%
\pgfpathlineto{\pgfqpoint{6.387397in}{3.150891in}}%
\pgfpathlineto{\pgfqpoint{6.388517in}{3.148264in}}%
\pgfpathlineto{\pgfqpoint{6.389638in}{3.127685in}}%
\pgfpathlineto{\pgfqpoint{6.390758in}{3.119366in}}%
\pgfpathlineto{\pgfqpoint{6.395241in}{3.146075in}}%
\pgfpathlineto{\pgfqpoint{6.396361in}{3.143010in}}%
\pgfpathlineto{\pgfqpoint{6.397482in}{3.144323in}}%
\pgfpathlineto{\pgfqpoint{6.398602in}{3.159210in}}%
\pgfpathlineto{\pgfqpoint{6.401964in}{3.156145in}}%
\pgfpathlineto{\pgfqpoint{6.403084in}{3.152205in}}%
\pgfpathlineto{\pgfqpoint{6.404205in}{3.136880in}}%
\pgfpathlineto{\pgfqpoint{6.405325in}{3.130750in}}%
\pgfpathlineto{\pgfqpoint{6.406446in}{3.128561in}}%
\pgfpathlineto{\pgfqpoint{6.410928in}{3.106230in}}%
\pgfpathlineto{\pgfqpoint{6.412048in}{3.088278in}}%
\pgfpathlineto{\pgfqpoint{6.413169in}{3.059818in}}%
\pgfpathlineto{\pgfqpoint{6.414290in}{3.056315in}}%
\pgfpathlineto{\pgfqpoint{6.417651in}{3.063758in}}%
\pgfpathlineto{\pgfqpoint{6.419892in}{3.101852in}}%
\pgfpathlineto{\pgfqpoint{6.421013in}{3.099224in}}%
\pgfpathlineto{\pgfqpoint{6.422133in}{3.126371in}}%
\pgfpathlineto{\pgfqpoint{6.425495in}{3.135566in}}%
\pgfpathlineto{\pgfqpoint{6.426615in}{3.185481in}}%
\pgfpathlineto{\pgfqpoint{6.427736in}{3.191174in}}%
\pgfpathlineto{\pgfqpoint{6.428856in}{3.169719in}}%
\pgfpathlineto{\pgfqpoint{6.429977in}{3.207812in}}%
\pgfpathlineto{\pgfqpoint{6.433338in}{3.207812in}}%
\pgfpathlineto{\pgfqpoint{6.434459in}{3.192487in}}%
\pgfpathlineto{\pgfqpoint{6.435580in}{3.192487in}}%
\pgfpathlineto{\pgfqpoint{6.436700in}{3.188546in}}%
\pgfpathlineto{\pgfqpoint{6.437821in}{3.191611in}}%
\pgfpathlineto{\pgfqpoint{6.441182in}{3.186357in}}%
\pgfpathlineto{\pgfqpoint{6.442303in}{3.206061in}}%
\pgfpathlineto{\pgfqpoint{6.443423in}{3.207812in}}%
\pgfpathlineto{\pgfqpoint{6.444544in}{3.202558in}}%
\pgfpathlineto{\pgfqpoint{6.445664in}{3.187671in}}%
\pgfpathlineto{\pgfqpoint{6.450146in}{3.201244in}}%
\pgfpathlineto{\pgfqpoint{6.451267in}{3.189860in}}%
\pgfpathlineto{\pgfqpoint{6.453508in}{3.149140in}}%
\pgfpathlineto{\pgfqpoint{6.456870in}{3.158772in}}%
\pgfpathlineto{\pgfqpoint{6.459111in}{3.181103in}}%
\pgfpathlineto{\pgfqpoint{6.461352in}{3.217007in}}%
\pgfpathlineto{\pgfqpoint{6.464713in}{3.195114in}}%
\pgfpathlineto{\pgfqpoint{6.465834in}{3.192925in}}%
\pgfpathlineto{\pgfqpoint{6.466954in}{3.180665in}}%
\pgfpathlineto{\pgfqpoint{6.468075in}{3.191174in}}%
\pgfpathlineto{\pgfqpoint{6.469195in}{3.190298in}}%
\pgfpathlineto{\pgfqpoint{6.472557in}{3.148264in}}%
\pgfpathlineto{\pgfqpoint{6.473677in}{3.149578in}}%
\pgfpathlineto{\pgfqpoint{6.474798in}{3.148264in}}%
\pgfpathlineto{\pgfqpoint{6.475919in}{3.132939in}}%
\pgfpathlineto{\pgfqpoint{6.477039in}{3.131626in}}%
\pgfpathlineto{\pgfqpoint{6.480401in}{3.055001in}}%
\pgfpathlineto{\pgfqpoint{6.481521in}{3.056315in}}%
\pgfpathlineto{\pgfqpoint{6.482642in}{3.051936in}}%
\pgfpathlineto{\pgfqpoint{6.483762in}{3.036174in}}%
\pgfpathlineto{\pgfqpoint{6.484883in}{3.028730in}}%
\pgfpathlineto{\pgfqpoint{6.488244in}{3.023038in}}%
\pgfpathlineto{\pgfqpoint{6.489365in}{3.001583in}}%
\pgfpathlineto{\pgfqpoint{6.490485in}{2.997205in}}%
\pgfpathlineto{\pgfqpoint{6.492726in}{3.050185in}}%
\pgfpathlineto{\pgfqpoint{6.496088in}{3.082148in}}%
\pgfpathlineto{\pgfqpoint{6.497209in}{3.081273in}}%
\pgfpathlineto{\pgfqpoint{6.498329in}{3.118490in}}%
\pgfpathlineto{\pgfqpoint{6.500570in}{3.114987in}}%
\pgfpathlineto{\pgfqpoint{6.503932in}{3.141258in}}%
\pgfpathlineto{\pgfqpoint{6.507293in}{3.272176in}}%
\pgfpathlineto{\pgfqpoint{6.508414in}{3.286626in}}%
\pgfpathlineto{\pgfqpoint{6.511775in}{3.305891in}}%
\pgfpathlineto{\pgfqpoint{6.512896in}{3.273052in}}%
\pgfpathlineto{\pgfqpoint{6.515137in}{3.252911in}}%
\pgfpathlineto{\pgfqpoint{6.516257in}{3.280058in}}%
\pgfpathlineto{\pgfqpoint{6.519619in}{3.310707in}}%
\pgfpathlineto{\pgfqpoint{6.520740in}{3.365439in}}%
\pgfpathlineto{\pgfqpoint{6.521860in}{3.353179in}}%
\pgfpathlineto{\pgfqpoint{6.522981in}{3.330849in}}%
\pgfpathlineto{\pgfqpoint{6.524101in}{3.343984in}}%
\pgfpathlineto{\pgfqpoint{6.527463in}{3.367628in}}%
\pgfpathlineto{\pgfqpoint{6.528583in}{3.350552in}}%
\pgfpathlineto{\pgfqpoint{6.529704in}{3.348363in}}%
\pgfpathlineto{\pgfqpoint{6.531945in}{3.365439in}}%
\pgfpathlineto{\pgfqpoint{6.536427in}{3.366315in}}%
\pgfpathlineto{\pgfqpoint{6.537547in}{3.368942in}}%
\pgfpathlineto{\pgfqpoint{6.538668in}{3.375072in}}%
\pgfpathlineto{\pgfqpoint{6.539789in}{3.354931in}}%
\pgfpathlineto{\pgfqpoint{6.539789in}{3.354931in}}%
\pgfusepath{stroke}%
\end{pgfscope}%
\begin{pgfscope}%
\pgfpathrectangle{\pgfqpoint{3.966666in}{2.320415in}}{\pgfqpoint{2.695652in}{1.104878in}}%
\pgfusepath{clip}%
\pgfsetroundcap%
\pgfsetroundjoin%
\pgfsetlinewidth{1.505625pt}%
\definecolor{currentstroke}{rgb}{1.000000,0.647059,0.000000}%
\pgfsetstrokecolor{currentstroke}%
\pgfsetdash{}{0pt}%
\pgfpathmoveto{\pgfqpoint{4.089196in}{2.433250in}}%
\pgfpathlineto{\pgfqpoint{4.091437in}{2.419969in}}%
\pgfpathlineto{\pgfqpoint{4.092557in}{2.416393in}}%
\pgfpathlineto{\pgfqpoint{4.097039in}{2.414277in}}%
\pgfpathlineto{\pgfqpoint{4.100401in}{2.417487in}}%
\pgfpathlineto{\pgfqpoint{4.106004in}{2.419199in}}%
\pgfpathlineto{\pgfqpoint{4.108245in}{2.420283in}}%
\pgfpathlineto{\pgfqpoint{4.111606in}{2.419395in}}%
\pgfpathlineto{\pgfqpoint{4.116088in}{2.410628in}}%
\pgfpathlineto{\pgfqpoint{4.119450in}{2.409191in}}%
\pgfpathlineto{\pgfqpoint{4.122811in}{2.405904in}}%
\pgfpathlineto{\pgfqpoint{4.123932in}{2.405228in}}%
\pgfpathlineto{\pgfqpoint{4.128414in}{2.404510in}}%
\pgfpathlineto{\pgfqpoint{4.131776in}{2.403007in}}%
\pgfpathlineto{\pgfqpoint{4.137378in}{2.402261in}}%
\pgfpathlineto{\pgfqpoint{4.139619in}{2.402322in}}%
\pgfpathlineto{\pgfqpoint{4.147463in}{2.402376in}}%
\pgfpathlineto{\pgfqpoint{4.155307in}{2.402642in}}%
\pgfpathlineto{\pgfqpoint{4.160909in}{2.403943in}}%
\pgfpathlineto{\pgfqpoint{4.163150in}{2.405116in}}%
\pgfpathlineto{\pgfqpoint{4.167633in}{2.406505in}}%
\pgfpathlineto{\pgfqpoint{4.170994in}{2.408604in}}%
\pgfpathlineto{\pgfqpoint{4.175476in}{2.410012in}}%
\pgfpathlineto{\pgfqpoint{4.178838in}{2.411903in}}%
\pgfpathlineto{\pgfqpoint{4.184440in}{2.412248in}}%
\pgfpathlineto{\pgfqpoint{4.186682in}{2.411859in}}%
\pgfpathlineto{\pgfqpoint{4.197887in}{2.411318in}}%
\pgfpathlineto{\pgfqpoint{4.200128in}{2.410475in}}%
\pgfpathlineto{\pgfqpoint{4.207972in}{2.409506in}}%
\pgfpathlineto{\pgfqpoint{4.213574in}{2.410158in}}%
\pgfpathlineto{\pgfqpoint{4.233744in}{2.421693in}}%
\pgfpathlineto{\pgfqpoint{4.237105in}{2.422570in}}%
\pgfpathlineto{\pgfqpoint{4.241587in}{2.426313in}}%
\pgfpathlineto{\pgfqpoint{4.244949in}{2.427242in}}%
\pgfpathlineto{\pgfqpoint{4.249431in}{2.430827in}}%
\pgfpathlineto{\pgfqpoint{4.253913in}{2.431795in}}%
\pgfpathlineto{\pgfqpoint{4.257275in}{2.434238in}}%
\pgfpathlineto{\pgfqpoint{4.261757in}{2.435734in}}%
\pgfpathlineto{\pgfqpoint{4.265118in}{2.438465in}}%
\pgfpathlineto{\pgfqpoint{4.268480in}{2.439514in}}%
\pgfpathlineto{\pgfqpoint{4.272962in}{2.444356in}}%
\pgfpathlineto{\pgfqpoint{4.276324in}{2.445654in}}%
\pgfpathlineto{\pgfqpoint{4.280806in}{2.450378in}}%
\pgfpathlineto{\pgfqpoint{4.284167in}{2.451528in}}%
\pgfpathlineto{\pgfqpoint{4.288649in}{2.456309in}}%
\pgfpathlineto{\pgfqpoint{4.292011in}{2.457690in}}%
\pgfpathlineto{\pgfqpoint{4.293132in}{2.459052in}}%
\pgfpathlineto{\pgfqpoint{4.295373in}{2.460380in}}%
\pgfpathlineto{\pgfqpoint{4.296493in}{2.461694in}}%
\pgfpathlineto{\pgfqpoint{4.299855in}{2.463065in}}%
\pgfpathlineto{\pgfqpoint{4.304337in}{2.468471in}}%
\pgfpathlineto{\pgfqpoint{4.307698in}{2.469868in}}%
\pgfpathlineto{\pgfqpoint{4.312181in}{2.475136in}}%
\pgfpathlineto{\pgfqpoint{4.316663in}{2.477166in}}%
\pgfpathlineto{\pgfqpoint{4.320024in}{2.480307in}}%
\pgfpathlineto{\pgfqpoint{4.323386in}{2.481456in}}%
\pgfpathlineto{\pgfqpoint{4.327868in}{2.485813in}}%
\pgfpathlineto{\pgfqpoint{4.331230in}{2.486832in}}%
\pgfpathlineto{\pgfqpoint{4.335712in}{2.490524in}}%
\pgfpathlineto{\pgfqpoint{4.339073in}{2.491391in}}%
\pgfpathlineto{\pgfqpoint{4.343555in}{2.494676in}}%
\pgfpathlineto{\pgfqpoint{4.349158in}{2.496377in}}%
\pgfpathlineto{\pgfqpoint{4.359243in}{2.499768in}}%
\pgfpathlineto{\pgfqpoint{4.364845in}{2.501038in}}%
\pgfpathlineto{\pgfqpoint{4.367086in}{2.502347in}}%
\pgfpathlineto{\pgfqpoint{4.371569in}{2.503707in}}%
\pgfpathlineto{\pgfqpoint{4.374930in}{2.506145in}}%
\pgfpathlineto{\pgfqpoint{4.379412in}{2.507637in}}%
\pgfpathlineto{\pgfqpoint{4.382774in}{2.510188in}}%
\pgfpathlineto{\pgfqpoint{4.387256in}{2.511905in}}%
\pgfpathlineto{\pgfqpoint{4.390617in}{2.514409in}}%
\pgfpathlineto{\pgfqpoint{4.395100in}{2.516099in}}%
\pgfpathlineto{\pgfqpoint{4.398461in}{2.519167in}}%
\pgfpathlineto{\pgfqpoint{4.401823in}{2.520175in}}%
\pgfpathlineto{\pgfqpoint{4.406305in}{2.523362in}}%
\pgfpathlineto{\pgfqpoint{4.410787in}{2.524538in}}%
\pgfpathlineto{\pgfqpoint{4.414149in}{2.526707in}}%
\pgfpathlineto{\pgfqpoint{4.418631in}{2.527865in}}%
\pgfpathlineto{\pgfqpoint{4.421992in}{2.529553in}}%
\pgfpathlineto{\pgfqpoint{4.428715in}{2.530802in}}%
\pgfpathlineto{\pgfqpoint{4.429836in}{2.531360in}}%
\pgfpathlineto{\pgfqpoint{4.435439in}{2.532723in}}%
\pgfpathlineto{\pgfqpoint{4.442162in}{2.533703in}}%
\pgfpathlineto{\pgfqpoint{4.466813in}{2.538320in}}%
\pgfpathlineto{\pgfqpoint{4.469054in}{2.539235in}}%
\pgfpathlineto{\pgfqpoint{4.473536in}{2.540081in}}%
\pgfpathlineto{\pgfqpoint{4.476898in}{2.541433in}}%
\pgfpathlineto{\pgfqpoint{4.482501in}{2.542465in}}%
\pgfpathlineto{\pgfqpoint{4.484742in}{2.543114in}}%
\pgfpathlineto{\pgfqpoint{4.498188in}{2.544836in}}%
\pgfpathlineto{\pgfqpoint{4.500429in}{2.545571in}}%
\pgfpathlineto{\pgfqpoint{4.507152in}{2.546899in}}%
\pgfpathlineto{\pgfqpoint{4.508273in}{2.547204in}}%
\pgfpathlineto{\pgfqpoint{4.522840in}{2.548656in}}%
\pgfpathlineto{\pgfqpoint{4.538527in}{2.552151in}}%
\pgfpathlineto{\pgfqpoint{4.539648in}{2.552541in}}%
\pgfpathlineto{\pgfqpoint{4.545250in}{2.553716in}}%
\pgfpathlineto{\pgfqpoint{4.547491in}{2.554472in}}%
\pgfpathlineto{\pgfqpoint{4.553094in}{2.555297in}}%
\pgfpathlineto{\pgfqpoint{4.555335in}{2.556236in}}%
\pgfpathlineto{\pgfqpoint{4.559817in}{2.557313in}}%
\pgfpathlineto{\pgfqpoint{4.563179in}{2.559107in}}%
\pgfpathlineto{\pgfqpoint{4.567661in}{2.560476in}}%
\pgfpathlineto{\pgfqpoint{4.571022in}{2.562550in}}%
\pgfpathlineto{\pgfqpoint{4.575504in}{2.564023in}}%
\pgfpathlineto{\pgfqpoint{4.578866in}{2.566216in}}%
\pgfpathlineto{\pgfqpoint{4.583348in}{2.567805in}}%
\pgfpathlineto{\pgfqpoint{4.586710in}{2.570136in}}%
\pgfpathlineto{\pgfqpoint{4.591192in}{2.571775in}}%
\pgfpathlineto{\pgfqpoint{4.593433in}{2.573333in}}%
\pgfpathlineto{\pgfqpoint{4.597915in}{2.574123in}}%
\pgfpathlineto{\pgfqpoint{4.602397in}{2.577283in}}%
\pgfpathlineto{\pgfqpoint{4.606879in}{2.578952in}}%
\pgfpathlineto{\pgfqpoint{4.610241in}{2.581730in}}%
\pgfpathlineto{\pgfqpoint{4.613602in}{2.582669in}}%
\pgfpathlineto{\pgfqpoint{4.618084in}{2.586423in}}%
\pgfpathlineto{\pgfqpoint{4.621446in}{2.587490in}}%
\pgfpathlineto{\pgfqpoint{4.625928in}{2.591913in}}%
\pgfpathlineto{\pgfqpoint{4.629290in}{2.593074in}}%
\pgfpathlineto{\pgfqpoint{4.633772in}{2.597420in}}%
\pgfpathlineto{\pgfqpoint{4.637133in}{2.598408in}}%
\pgfpathlineto{\pgfqpoint{4.641616in}{2.602636in}}%
\pgfpathlineto{\pgfqpoint{4.644977in}{2.603643in}}%
\pgfpathlineto{\pgfqpoint{4.649459in}{2.607910in}}%
\pgfpathlineto{\pgfqpoint{4.652821in}{2.608905in}}%
\pgfpathlineto{\pgfqpoint{4.657303in}{2.612440in}}%
\pgfpathlineto{\pgfqpoint{4.662906in}{2.613893in}}%
\pgfpathlineto{\pgfqpoint{4.665147in}{2.615066in}}%
\pgfpathlineto{\pgfqpoint{4.669629in}{2.616220in}}%
\pgfpathlineto{\pgfqpoint{4.680834in}{2.621691in}}%
\pgfpathlineto{\pgfqpoint{4.685316in}{2.623207in}}%
\pgfpathlineto{\pgfqpoint{4.688678in}{2.625013in}}%
\pgfpathlineto{\pgfqpoint{4.693160in}{2.626250in}}%
\pgfpathlineto{\pgfqpoint{4.696521in}{2.628315in}}%
\pgfpathlineto{\pgfqpoint{4.701003in}{2.629644in}}%
\pgfpathlineto{\pgfqpoint{4.702124in}{2.630348in}}%
\pgfpathlineto{\pgfqpoint{4.709968in}{2.633256in}}%
\pgfpathlineto{\pgfqpoint{4.712209in}{2.634677in}}%
\pgfpathlineto{\pgfqpoint{4.716691in}{2.635965in}}%
\pgfpathlineto{\pgfqpoint{4.720052in}{2.637897in}}%
\pgfpathlineto{\pgfqpoint{4.724535in}{2.639184in}}%
\pgfpathlineto{\pgfqpoint{4.727896in}{2.641190in}}%
\pgfpathlineto{\pgfqpoint{4.732378in}{2.642553in}}%
\pgfpathlineto{\pgfqpoint{4.735740in}{2.644310in}}%
\pgfpathlineto{\pgfqpoint{4.740222in}{2.645506in}}%
\pgfpathlineto{\pgfqpoint{4.743584in}{2.647147in}}%
\pgfpathlineto{\pgfqpoint{4.748066in}{2.648191in}}%
\pgfpathlineto{\pgfqpoint{4.751427in}{2.649486in}}%
\pgfpathlineto{\pgfqpoint{4.757030in}{2.650564in}}%
\pgfpathlineto{\pgfqpoint{4.759271in}{2.651234in}}%
\pgfpathlineto{\pgfqpoint{4.764874in}{2.652107in}}%
\pgfpathlineto{\pgfqpoint{4.767115in}{2.652807in}}%
\pgfpathlineto{\pgfqpoint{4.773838in}{2.653578in}}%
\pgfpathlineto{\pgfqpoint{4.782802in}{2.655203in}}%
\pgfpathlineto{\pgfqpoint{4.787284in}{2.656000in}}%
\pgfpathlineto{\pgfqpoint{4.790646in}{2.657159in}}%
\pgfpathlineto{\pgfqpoint{4.797369in}{2.658414in}}%
\pgfpathlineto{\pgfqpoint{4.798489in}{2.658688in}}%
\pgfpathlineto{\pgfqpoint{4.805213in}{2.659732in}}%
\pgfpathlineto{\pgfqpoint{4.809695in}{2.660297in}}%
\pgfpathlineto{\pgfqpoint{4.826503in}{2.664348in}}%
\pgfpathlineto{\pgfqpoint{4.829864in}{2.666021in}}%
\pgfpathlineto{\pgfqpoint{4.834346in}{2.667151in}}%
\pgfpathlineto{\pgfqpoint{4.837708in}{2.668758in}}%
\pgfpathlineto{\pgfqpoint{4.842190in}{2.669829in}}%
\pgfpathlineto{\pgfqpoint{4.845551in}{2.671362in}}%
\pgfpathlineto{\pgfqpoint{4.850034in}{2.672335in}}%
\pgfpathlineto{\pgfqpoint{4.853395in}{2.673806in}}%
\pgfpathlineto{\pgfqpoint{4.857877in}{2.674859in}}%
\pgfpathlineto{\pgfqpoint{4.861239in}{2.676329in}}%
\pgfpathlineto{\pgfqpoint{4.865721in}{2.677257in}}%
\pgfpathlineto{\pgfqpoint{4.869083in}{2.678141in}}%
\pgfpathlineto{\pgfqpoint{4.873565in}{2.678969in}}%
\pgfpathlineto{\pgfqpoint{4.876926in}{2.680162in}}%
\pgfpathlineto{\pgfqpoint{4.882529in}{2.681296in}}%
\pgfpathlineto{\pgfqpoint{4.884770in}{2.681900in}}%
\pgfpathlineto{\pgfqpoint{4.890373in}{2.682818in}}%
\pgfpathlineto{\pgfqpoint{4.892614in}{2.683446in}}%
\pgfpathlineto{\pgfqpoint{4.899337in}{2.684512in}}%
\pgfpathlineto{\pgfqpoint{4.900457in}{2.684882in}}%
\pgfpathlineto{\pgfqpoint{4.911663in}{2.686612in}}%
\pgfpathlineto{\pgfqpoint{4.916145in}{2.687911in}}%
\pgfpathlineto{\pgfqpoint{4.921747in}{2.688738in}}%
\pgfpathlineto{\pgfqpoint{4.923988in}{2.689414in}}%
\pgfpathlineto{\pgfqpoint{4.930712in}{2.690258in}}%
\pgfpathlineto{\pgfqpoint{4.935194in}{2.690819in}}%
\pgfpathlineto{\pgfqpoint{4.944158in}{2.692345in}}%
\pgfpathlineto{\pgfqpoint{4.955363in}{2.694128in}}%
\pgfpathlineto{\pgfqpoint{4.962086in}{2.694773in}}%
\pgfpathlineto{\pgfqpoint{4.966568in}{2.695181in}}%
\pgfpathlineto{\pgfqpoint{4.994582in}{2.699068in}}%
\pgfpathlineto{\pgfqpoint{5.001305in}{2.699939in}}%
\pgfpathlineto{\pgfqpoint{5.010269in}{2.701424in}}%
\pgfpathlineto{\pgfqpoint{5.015872in}{2.702275in}}%
\pgfpathlineto{\pgfqpoint{5.021474in}{2.702999in}}%
\pgfpathlineto{\pgfqpoint{5.032680in}{2.704679in}}%
\pgfpathlineto{\pgfqpoint{5.061813in}{2.709248in}}%
\pgfpathlineto{\pgfqpoint{5.065175in}{2.710279in}}%
\pgfpathlineto{\pgfqpoint{5.071898in}{2.711341in}}%
\pgfpathlineto{\pgfqpoint{5.073019in}{2.711708in}}%
\pgfpathlineto{\pgfqpoint{5.078621in}{2.712722in}}%
\pgfpathlineto{\pgfqpoint{5.080862in}{2.713377in}}%
\pgfpathlineto{\pgfqpoint{5.086465in}{2.714374in}}%
\pgfpathlineto{\pgfqpoint{5.088706in}{2.714991in}}%
\pgfpathlineto{\pgfqpoint{5.094309in}{2.715943in}}%
\pgfpathlineto{\pgfqpoint{5.096550in}{2.716586in}}%
\pgfpathlineto{\pgfqpoint{5.102152in}{2.717549in}}%
\pgfpathlineto{\pgfqpoint{5.104393in}{2.718163in}}%
\pgfpathlineto{\pgfqpoint{5.109996in}{2.719086in}}%
\pgfpathlineto{\pgfqpoint{5.111116in}{2.719413in}}%
\pgfpathlineto{\pgfqpoint{5.117840in}{2.720351in}}%
\pgfpathlineto{\pgfqpoint{5.120081in}{2.721084in}}%
\pgfpathlineto{\pgfqpoint{5.125683in}{2.722298in}}%
\pgfpathlineto{\pgfqpoint{5.127924in}{2.723079in}}%
\pgfpathlineto{\pgfqpoint{5.133527in}{2.724296in}}%
\pgfpathlineto{\pgfqpoint{5.135768in}{2.725134in}}%
\pgfpathlineto{\pgfqpoint{5.140250in}{2.726032in}}%
\pgfpathlineto{\pgfqpoint{5.143612in}{2.727190in}}%
\pgfpathlineto{\pgfqpoint{5.150335in}{2.728450in}}%
\pgfpathlineto{\pgfqpoint{5.159299in}{2.730123in}}%
\pgfpathlineto{\pgfqpoint{5.166022in}{2.731216in}}%
\pgfpathlineto{\pgfqpoint{5.174986in}{2.732983in}}%
\pgfpathlineto{\pgfqpoint{5.181710in}{2.733933in}}%
\pgfpathlineto{\pgfqpoint{5.182830in}{2.734256in}}%
\pgfpathlineto{\pgfqpoint{5.189553in}{2.735354in}}%
\pgfpathlineto{\pgfqpoint{5.198518in}{2.737022in}}%
\pgfpathlineto{\pgfqpoint{5.204120in}{2.737976in}}%
\pgfpathlineto{\pgfqpoint{5.206361in}{2.738570in}}%
\pgfpathlineto{\pgfqpoint{5.211964in}{2.739461in}}%
\pgfpathlineto{\pgfqpoint{5.222049in}{2.741513in}}%
\pgfpathlineto{\pgfqpoint{5.228772in}{2.742439in}}%
\pgfpathlineto{\pgfqpoint{5.237736in}{2.743920in}}%
\pgfpathlineto{\pgfqpoint{5.243339in}{2.744847in}}%
\pgfpathlineto{\pgfqpoint{5.245580in}{2.745493in}}%
\pgfpathlineto{\pgfqpoint{5.251182in}{2.746498in}}%
\pgfpathlineto{\pgfqpoint{5.253423in}{2.747185in}}%
\pgfpathlineto{\pgfqpoint{5.259026in}{2.748229in}}%
\pgfpathlineto{\pgfqpoint{5.261267in}{2.748982in}}%
\pgfpathlineto{\pgfqpoint{5.266870in}{2.750058in}}%
\pgfpathlineto{\pgfqpoint{5.269111in}{2.750689in}}%
\pgfpathlineto{\pgfqpoint{5.280316in}{2.752181in}}%
\pgfpathlineto{\pgfqpoint{5.284798in}{2.753153in}}%
\pgfpathlineto{\pgfqpoint{5.327378in}{2.756735in}}%
\pgfpathlineto{\pgfqpoint{5.339704in}{2.758301in}}%
\pgfpathlineto{\pgfqpoint{5.352030in}{2.759320in}}%
\pgfpathlineto{\pgfqpoint{5.371079in}{2.762357in}}%
\pgfpathlineto{\pgfqpoint{5.377802in}{2.763371in}}%
\pgfpathlineto{\pgfqpoint{5.384525in}{2.764418in}}%
\pgfpathlineto{\pgfqpoint{5.390128in}{2.765075in}}%
\pgfpathlineto{\pgfqpoint{5.456239in}{2.775848in}}%
\pgfpathlineto{\pgfqpoint{5.457359in}{2.776137in}}%
\pgfpathlineto{\pgfqpoint{5.464082in}{2.777225in}}%
\pgfpathlineto{\pgfqpoint{5.471926in}{2.778553in}}%
\pgfpathlineto{\pgfqpoint{5.473047in}{2.778812in}}%
\pgfpathlineto{\pgfqpoint{5.479770in}{2.779575in}}%
\pgfpathlineto{\pgfqpoint{5.486493in}{2.780535in}}%
\pgfpathlineto{\pgfqpoint{5.496578in}{2.781695in}}%
\pgfpathlineto{\pgfqpoint{5.507783in}{2.782649in}}%
\pgfpathlineto{\pgfqpoint{5.518988in}{2.783837in}}%
\pgfpathlineto{\pgfqpoint{5.531314in}{2.784723in}}%
\pgfpathlineto{\pgfqpoint{5.543640in}{2.786184in}}%
\pgfpathlineto{\pgfqpoint{5.555966in}{2.787119in}}%
\pgfpathlineto{\pgfqpoint{5.575015in}{2.789160in}}%
\pgfpathlineto{\pgfqpoint{5.595184in}{2.790045in}}%
\pgfpathlineto{\pgfqpoint{5.614233in}{2.790955in}}%
\pgfpathlineto{\pgfqpoint{5.641126in}{2.791558in}}%
\pgfpathlineto{\pgfqpoint{5.676983in}{2.794227in}}%
\pgfpathlineto{\pgfqpoint{5.697152in}{2.795236in}}%
\pgfpathlineto{\pgfqpoint{5.731888in}{2.797681in}}%
\pgfpathlineto{\pgfqpoint{5.750937in}{2.798727in}}%
\pgfpathlineto{\pgfqpoint{5.771107in}{2.802529in}}%
\pgfpathlineto{\pgfqpoint{5.777830in}{2.803389in}}%
\pgfpathlineto{\pgfqpoint{5.778951in}{2.803688in}}%
\pgfpathlineto{\pgfqpoint{5.784553in}{2.804578in}}%
\pgfpathlineto{\pgfqpoint{5.786794in}{2.805185in}}%
\pgfpathlineto{\pgfqpoint{5.792397in}{2.806131in}}%
\pgfpathlineto{\pgfqpoint{5.794638in}{2.806788in}}%
\pgfpathlineto{\pgfqpoint{5.800241in}{2.807821in}}%
\pgfpathlineto{\pgfqpoint{5.802482in}{2.808513in}}%
\pgfpathlineto{\pgfqpoint{5.808084in}{2.809586in}}%
\pgfpathlineto{\pgfqpoint{5.810325in}{2.810342in}}%
\pgfpathlineto{\pgfqpoint{5.815928in}{2.811445in}}%
\pgfpathlineto{\pgfqpoint{5.817049in}{2.811825in}}%
\pgfpathlineto{\pgfqpoint{5.822651in}{2.812593in}}%
\pgfpathlineto{\pgfqpoint{5.826013in}{2.813773in}}%
\pgfpathlineto{\pgfqpoint{5.831615in}{2.814963in}}%
\pgfpathlineto{\pgfqpoint{5.833856in}{2.815645in}}%
\pgfpathlineto{\pgfqpoint{5.839459in}{2.816612in}}%
\pgfpathlineto{\pgfqpoint{5.841700in}{2.817235in}}%
\pgfpathlineto{\pgfqpoint{5.847303in}{2.818217in}}%
\pgfpathlineto{\pgfqpoint{5.857388in}{2.820227in}}%
\pgfpathlineto{\pgfqpoint{5.864111in}{2.821384in}}%
\pgfpathlineto{\pgfqpoint{5.865231in}{2.821679in}}%
\pgfpathlineto{\pgfqpoint{5.870834in}{2.822574in}}%
\pgfpathlineto{\pgfqpoint{5.873075in}{2.823165in}}%
\pgfpathlineto{\pgfqpoint{5.879798in}{2.824248in}}%
\pgfpathlineto{\pgfqpoint{5.888762in}{2.825714in}}%
\pgfpathlineto{\pgfqpoint{5.895485in}{2.826530in}}%
\pgfpathlineto{\pgfqpoint{5.904450in}{2.828328in}}%
\pgfpathlineto{\pgfqpoint{5.910052in}{2.829342in}}%
\pgfpathlineto{\pgfqpoint{5.912293in}{2.830071in}}%
\pgfpathlineto{\pgfqpoint{5.917896in}{2.831195in}}%
\pgfpathlineto{\pgfqpoint{5.920137in}{2.831979in}}%
\pgfpathlineto{\pgfqpoint{5.925740in}{2.833183in}}%
\pgfpathlineto{\pgfqpoint{5.927981in}{2.834060in}}%
\pgfpathlineto{\pgfqpoint{5.933583in}{2.834978in}}%
\pgfpathlineto{\pgfqpoint{5.935824in}{2.835859in}}%
\pgfpathlineto{\pgfqpoint{5.940307in}{2.836742in}}%
\pgfpathlineto{\pgfqpoint{5.943668in}{2.838080in}}%
\pgfpathlineto{\pgfqpoint{5.948150in}{2.838963in}}%
\pgfpathlineto{\pgfqpoint{5.951512in}{2.840270in}}%
\pgfpathlineto{\pgfqpoint{5.957114in}{2.841534in}}%
\pgfpathlineto{\pgfqpoint{5.959355in}{2.842359in}}%
\pgfpathlineto{\pgfqpoint{5.964958in}{2.843492in}}%
\pgfpathlineto{\pgfqpoint{5.967199in}{2.844221in}}%
\pgfpathlineto{\pgfqpoint{5.972802in}{2.845301in}}%
\pgfpathlineto{\pgfqpoint{5.975043in}{2.846022in}}%
\pgfpathlineto{\pgfqpoint{5.980646in}{2.847042in}}%
\pgfpathlineto{\pgfqpoint{5.982887in}{2.847685in}}%
\pgfpathlineto{\pgfqpoint{5.988489in}{2.848637in}}%
\pgfpathlineto{\pgfqpoint{5.990730in}{2.849259in}}%
\pgfpathlineto{\pgfqpoint{5.996333in}{2.850182in}}%
\pgfpathlineto{\pgfqpoint{5.998574in}{2.850819in}}%
\pgfpathlineto{\pgfqpoint{6.005297in}{2.851855in}}%
\pgfpathlineto{\pgfqpoint{6.010900in}{2.852726in}}%
\pgfpathlineto{\pgfqpoint{6.022105in}{2.854994in}}%
\pgfpathlineto{\pgfqpoint{6.027708in}{2.855876in}}%
\pgfpathlineto{\pgfqpoint{6.029949in}{2.856450in}}%
\pgfpathlineto{\pgfqpoint{6.036672in}{2.857483in}}%
\pgfpathlineto{\pgfqpoint{6.045636in}{2.858929in}}%
\pgfpathlineto{\pgfqpoint{6.057962in}{2.860389in}}%
\pgfpathlineto{\pgfqpoint{6.069167in}{2.861635in}}%
\pgfpathlineto{\pgfqpoint{6.082613in}{2.862765in}}%
\pgfpathlineto{\pgfqpoint{6.122952in}{2.869045in}}%
\pgfpathlineto{\pgfqpoint{6.124073in}{2.869376in}}%
\pgfpathlineto{\pgfqpoint{6.130796in}{2.870359in}}%
\pgfpathlineto{\pgfqpoint{6.131917in}{2.870679in}}%
\pgfpathlineto{\pgfqpoint{6.138640in}{2.871746in}}%
\pgfpathlineto{\pgfqpoint{6.139760in}{2.872078in}}%
\pgfpathlineto{\pgfqpoint{6.145363in}{2.873011in}}%
\pgfpathlineto{\pgfqpoint{6.147604in}{2.873630in}}%
\pgfpathlineto{\pgfqpoint{6.154327in}{2.874539in}}%
\pgfpathlineto{\pgfqpoint{6.155448in}{2.874849in}}%
\pgfpathlineto{\pgfqpoint{6.168894in}{2.876583in}}%
\pgfpathlineto{\pgfqpoint{6.178979in}{2.877810in}}%
\pgfpathlineto{\pgfqpoint{6.191305in}{2.878891in}}%
\pgfpathlineto{\pgfqpoint{6.202510in}{2.880639in}}%
\pgfpathlineto{\pgfqpoint{6.213715in}{2.881837in}}%
\pgfpathlineto{\pgfqpoint{6.226041in}{2.883741in}}%
\pgfpathlineto{\pgfqpoint{6.237246in}{2.884848in}}%
\pgfpathlineto{\pgfqpoint{6.248451in}{2.886336in}}%
\pgfpathlineto{\pgfqpoint{6.260777in}{2.887377in}}%
\pgfpathlineto{\pgfqpoint{6.273103in}{2.888411in}}%
\pgfpathlineto{\pgfqpoint{6.302237in}{2.889992in}}%
\pgfpathlineto{\pgfqpoint{6.320165in}{2.891393in}}%
\pgfpathlineto{\pgfqpoint{6.371709in}{2.893716in}}%
\pgfpathlineto{\pgfqpoint{6.382915in}{2.895164in}}%
\pgfpathlineto{\pgfqpoint{6.395241in}{2.896380in}}%
\pgfpathlineto{\pgfqpoint{6.406446in}{2.897768in}}%
\pgfpathlineto{\pgfqpoint{6.421013in}{2.898797in}}%
\pgfpathlineto{\pgfqpoint{6.445664in}{2.901966in}}%
\pgfpathlineto{\pgfqpoint{6.456870in}{2.903102in}}%
\pgfpathlineto{\pgfqpoint{6.469195in}{2.904872in}}%
\pgfpathlineto{\pgfqpoint{6.497209in}{2.906770in}}%
\pgfpathlineto{\pgfqpoint{6.522981in}{2.910493in}}%
\pgfpathlineto{\pgfqpoint{6.531945in}{2.912268in}}%
\pgfpathlineto{\pgfqpoint{6.538668in}{2.913179in}}%
\pgfpathlineto{\pgfqpoint{6.539789in}{2.913472in}}%
\pgfpathlineto{\pgfqpoint{6.539789in}{2.913472in}}%
\pgfusepath{stroke}%
\end{pgfscope}%
\begin{pgfscope}%
\pgfsetrectcap%
\pgfsetmiterjoin%
\pgfsetlinewidth{0.803000pt}%
\definecolor{currentstroke}{rgb}{1.000000,1.000000,1.000000}%
\pgfsetstrokecolor{currentstroke}%
\pgfsetdash{}{0pt}%
\pgfpathmoveto{\pgfqpoint{3.966666in}{2.320415in}}%
\pgfpathlineto{\pgfqpoint{3.966666in}{3.425294in}}%
\pgfusepath{stroke}%
\end{pgfscope}%
\begin{pgfscope}%
\pgfsetrectcap%
\pgfsetmiterjoin%
\pgfsetlinewidth{0.803000pt}%
\definecolor{currentstroke}{rgb}{1.000000,1.000000,1.000000}%
\pgfsetstrokecolor{currentstroke}%
\pgfsetdash{}{0pt}%
\pgfpathmoveto{\pgfqpoint{6.662318in}{2.320415in}}%
\pgfpathlineto{\pgfqpoint{6.662318in}{3.425294in}}%
\pgfusepath{stroke}%
\end{pgfscope}%
\begin{pgfscope}%
\pgfsetrectcap%
\pgfsetmiterjoin%
\pgfsetlinewidth{0.803000pt}%
\definecolor{currentstroke}{rgb}{1.000000,1.000000,1.000000}%
\pgfsetstrokecolor{currentstroke}%
\pgfsetdash{}{0pt}%
\pgfpathmoveto{\pgfqpoint{3.966666in}{2.320415in}}%
\pgfpathlineto{\pgfqpoint{6.662318in}{2.320415in}}%
\pgfusepath{stroke}%
\end{pgfscope}%
\begin{pgfscope}%
\pgfsetrectcap%
\pgfsetmiterjoin%
\pgfsetlinewidth{0.803000pt}%
\definecolor{currentstroke}{rgb}{1.000000,1.000000,1.000000}%
\pgfsetstrokecolor{currentstroke}%
\pgfsetdash{}{0pt}%
\pgfpathmoveto{\pgfqpoint{3.966666in}{3.425294in}}%
\pgfpathlineto{\pgfqpoint{6.662318in}{3.425294in}}%
\pgfusepath{stroke}%
\end{pgfscope}%
\begin{pgfscope}%
\definecolor{textcolor}{rgb}{0.150000,0.150000,0.150000}%
\pgfsetstrokecolor{textcolor}%
\pgfsetfillcolor{textcolor}%
\pgftext[x=5.314492in,y=3.508627in,,base]{\color{textcolor}\rmfamily\fontsize{12.000000}{14.400000}\selectfont VZ}%
\end{pgfscope}%
\begin{pgfscope}%
\pgfsetbuttcap%
\pgfsetmiterjoin%
\definecolor{currentfill}{rgb}{0.917647,0.917647,0.949020}%
\pgfsetfillcolor{currentfill}%
\pgfsetlinewidth{0.000000pt}%
\definecolor{currentstroke}{rgb}{0.000000,0.000000,0.000000}%
\pgfsetstrokecolor{currentstroke}%
\pgfsetstrokeopacity{0.000000}%
\pgfsetdash{}{0pt}%
\pgfpathmoveto{\pgfqpoint{0.462318in}{0.331635in}}%
\pgfpathlineto{\pgfqpoint{3.157970in}{0.331635in}}%
\pgfpathlineto{\pgfqpoint{3.157970in}{1.436513in}}%
\pgfpathlineto{\pgfqpoint{0.462318in}{1.436513in}}%
\pgfpathclose%
\pgfusepath{fill}%
\end{pgfscope}%
\begin{pgfscope}%
\pgfpathrectangle{\pgfqpoint{0.462318in}{0.331635in}}{\pgfqpoint{2.695652in}{1.104878in}}%
\pgfusepath{clip}%
\pgfsetroundcap%
\pgfsetroundjoin%
\pgfsetlinewidth{0.803000pt}%
\definecolor{currentstroke}{rgb}{1.000000,1.000000,1.000000}%
\pgfsetstrokecolor{currentstroke}%
\pgfsetdash{}{0pt}%
\pgfpathmoveto{\pgfqpoint{0.582607in}{0.331635in}}%
\pgfpathlineto{\pgfqpoint{0.582607in}{1.436513in}}%
\pgfusepath{stroke}%
\end{pgfscope}%
\begin{pgfscope}%
\definecolor{textcolor}{rgb}{0.150000,0.150000,0.150000}%
\pgfsetstrokecolor{textcolor}%
\pgfsetfillcolor{textcolor}%
\pgftext[x=0.582607in,y=0.234413in,,top]{\color{textcolor}\rmfamily\fontsize{10.000000}{12.000000}\selectfont 2012}%
\end{pgfscope}%
\begin{pgfscope}%
\pgfpathrectangle{\pgfqpoint{0.462318in}{0.331635in}}{\pgfqpoint{2.695652in}{1.104878in}}%
\pgfusepath{clip}%
\pgfsetroundcap%
\pgfsetroundjoin%
\pgfsetlinewidth{0.803000pt}%
\definecolor{currentstroke}{rgb}{1.000000,1.000000,1.000000}%
\pgfsetstrokecolor{currentstroke}%
\pgfsetdash{}{0pt}%
\pgfpathmoveto{\pgfqpoint{0.992720in}{0.331635in}}%
\pgfpathlineto{\pgfqpoint{0.992720in}{1.436513in}}%
\pgfusepath{stroke}%
\end{pgfscope}%
\begin{pgfscope}%
\definecolor{textcolor}{rgb}{0.150000,0.150000,0.150000}%
\pgfsetstrokecolor{textcolor}%
\pgfsetfillcolor{textcolor}%
\pgftext[x=0.992720in,y=0.234413in,,top]{\color{textcolor}\rmfamily\fontsize{10.000000}{12.000000}\selectfont 2013}%
\end{pgfscope}%
\begin{pgfscope}%
\pgfpathrectangle{\pgfqpoint{0.462318in}{0.331635in}}{\pgfqpoint{2.695652in}{1.104878in}}%
\pgfusepath{clip}%
\pgfsetroundcap%
\pgfsetroundjoin%
\pgfsetlinewidth{0.803000pt}%
\definecolor{currentstroke}{rgb}{1.000000,1.000000,1.000000}%
\pgfsetstrokecolor{currentstroke}%
\pgfsetdash{}{0pt}%
\pgfpathmoveto{\pgfqpoint{1.401712in}{0.331635in}}%
\pgfpathlineto{\pgfqpoint{1.401712in}{1.436513in}}%
\pgfusepath{stroke}%
\end{pgfscope}%
\begin{pgfscope}%
\definecolor{textcolor}{rgb}{0.150000,0.150000,0.150000}%
\pgfsetstrokecolor{textcolor}%
\pgfsetfillcolor{textcolor}%
\pgftext[x=1.401712in,y=0.234413in,,top]{\color{textcolor}\rmfamily\fontsize{10.000000}{12.000000}\selectfont 2014}%
\end{pgfscope}%
\begin{pgfscope}%
\pgfpathrectangle{\pgfqpoint{0.462318in}{0.331635in}}{\pgfqpoint{2.695652in}{1.104878in}}%
\pgfusepath{clip}%
\pgfsetroundcap%
\pgfsetroundjoin%
\pgfsetlinewidth{0.803000pt}%
\definecolor{currentstroke}{rgb}{1.000000,1.000000,1.000000}%
\pgfsetstrokecolor{currentstroke}%
\pgfsetdash{}{0pt}%
\pgfpathmoveto{\pgfqpoint{1.810705in}{0.331635in}}%
\pgfpathlineto{\pgfqpoint{1.810705in}{1.436513in}}%
\pgfusepath{stroke}%
\end{pgfscope}%
\begin{pgfscope}%
\definecolor{textcolor}{rgb}{0.150000,0.150000,0.150000}%
\pgfsetstrokecolor{textcolor}%
\pgfsetfillcolor{textcolor}%
\pgftext[x=1.810705in,y=0.234413in,,top]{\color{textcolor}\rmfamily\fontsize{10.000000}{12.000000}\selectfont 2015}%
\end{pgfscope}%
\begin{pgfscope}%
\pgfpathrectangle{\pgfqpoint{0.462318in}{0.331635in}}{\pgfqpoint{2.695652in}{1.104878in}}%
\pgfusepath{clip}%
\pgfsetroundcap%
\pgfsetroundjoin%
\pgfsetlinewidth{0.803000pt}%
\definecolor{currentstroke}{rgb}{1.000000,1.000000,1.000000}%
\pgfsetstrokecolor{currentstroke}%
\pgfsetdash{}{0pt}%
\pgfpathmoveto{\pgfqpoint{2.219697in}{0.331635in}}%
\pgfpathlineto{\pgfqpoint{2.219697in}{1.436513in}}%
\pgfusepath{stroke}%
\end{pgfscope}%
\begin{pgfscope}%
\definecolor{textcolor}{rgb}{0.150000,0.150000,0.150000}%
\pgfsetstrokecolor{textcolor}%
\pgfsetfillcolor{textcolor}%
\pgftext[x=2.219697in,y=0.234413in,,top]{\color{textcolor}\rmfamily\fontsize{10.000000}{12.000000}\selectfont 2016}%
\end{pgfscope}%
\begin{pgfscope}%
\pgfpathrectangle{\pgfqpoint{0.462318in}{0.331635in}}{\pgfqpoint{2.695652in}{1.104878in}}%
\pgfusepath{clip}%
\pgfsetroundcap%
\pgfsetroundjoin%
\pgfsetlinewidth{0.803000pt}%
\definecolor{currentstroke}{rgb}{1.000000,1.000000,1.000000}%
\pgfsetstrokecolor{currentstroke}%
\pgfsetdash{}{0pt}%
\pgfpathmoveto{\pgfqpoint{2.629810in}{0.331635in}}%
\pgfpathlineto{\pgfqpoint{2.629810in}{1.436513in}}%
\pgfusepath{stroke}%
\end{pgfscope}%
\begin{pgfscope}%
\definecolor{textcolor}{rgb}{0.150000,0.150000,0.150000}%
\pgfsetstrokecolor{textcolor}%
\pgfsetfillcolor{textcolor}%
\pgftext[x=2.629810in,y=0.234413in,,top]{\color{textcolor}\rmfamily\fontsize{10.000000}{12.000000}\selectfont 2017}%
\end{pgfscope}%
\begin{pgfscope}%
\pgfpathrectangle{\pgfqpoint{0.462318in}{0.331635in}}{\pgfqpoint{2.695652in}{1.104878in}}%
\pgfusepath{clip}%
\pgfsetroundcap%
\pgfsetroundjoin%
\pgfsetlinewidth{0.803000pt}%
\definecolor{currentstroke}{rgb}{1.000000,1.000000,1.000000}%
\pgfsetstrokecolor{currentstroke}%
\pgfsetdash{}{0pt}%
\pgfpathmoveto{\pgfqpoint{3.038802in}{0.331635in}}%
\pgfpathlineto{\pgfqpoint{3.038802in}{1.436513in}}%
\pgfusepath{stroke}%
\end{pgfscope}%
\begin{pgfscope}%
\definecolor{textcolor}{rgb}{0.150000,0.150000,0.150000}%
\pgfsetstrokecolor{textcolor}%
\pgfsetfillcolor{textcolor}%
\pgftext[x=3.038802in,y=0.234413in,,top]{\color{textcolor}\rmfamily\fontsize{10.000000}{12.000000}\selectfont 2018}%
\end{pgfscope}%
\begin{pgfscope}%
\pgfpathrectangle{\pgfqpoint{0.462318in}{0.331635in}}{\pgfqpoint{2.695652in}{1.104878in}}%
\pgfusepath{clip}%
\pgfsetroundcap%
\pgfsetroundjoin%
\pgfsetlinewidth{0.803000pt}%
\definecolor{currentstroke}{rgb}{1.000000,1.000000,1.000000}%
\pgfsetstrokecolor{currentstroke}%
\pgfsetdash{}{0pt}%
\pgfpathmoveto{\pgfqpoint{0.462318in}{0.693070in}}%
\pgfpathlineto{\pgfqpoint{3.157970in}{0.693070in}}%
\pgfusepath{stroke}%
\end{pgfscope}%
\begin{pgfscope}%
\definecolor{textcolor}{rgb}{0.150000,0.150000,0.150000}%
\pgfsetstrokecolor{textcolor}%
\pgfsetfillcolor{textcolor}%
\pgftext[x=0.188365in,y=0.640308in,left,base]{\color{textcolor}\rmfamily\fontsize{10.000000}{12.000000}\selectfont 50}%
\end{pgfscope}%
\begin{pgfscope}%
\pgfpathrectangle{\pgfqpoint{0.462318in}{0.331635in}}{\pgfqpoint{2.695652in}{1.104878in}}%
\pgfusepath{clip}%
\pgfsetroundcap%
\pgfsetroundjoin%
\pgfsetlinewidth{0.803000pt}%
\definecolor{currentstroke}{rgb}{1.000000,1.000000,1.000000}%
\pgfsetstrokecolor{currentstroke}%
\pgfsetdash{}{0pt}%
\pgfpathmoveto{\pgfqpoint{0.462318in}{1.242722in}}%
\pgfpathlineto{\pgfqpoint{3.157970in}{1.242722in}}%
\pgfusepath{stroke}%
\end{pgfscope}%
\begin{pgfscope}%
\definecolor{textcolor}{rgb}{0.150000,0.150000,0.150000}%
\pgfsetstrokecolor{textcolor}%
\pgfsetfillcolor{textcolor}%
\pgftext[x=0.100000in,y=1.189961in,left,base]{\color{textcolor}\rmfamily\fontsize{10.000000}{12.000000}\selectfont 100}%
\end{pgfscope}%
\begin{pgfscope}%
\pgfpathrectangle{\pgfqpoint{0.462318in}{0.331635in}}{\pgfqpoint{2.695652in}{1.104878in}}%
\pgfusepath{clip}%
\pgfsetroundcap%
\pgfsetroundjoin%
\pgfsetlinewidth{1.505625pt}%
\definecolor{currentstroke}{rgb}{0.737255,0.741176,0.133333}%
\pgfsetstrokecolor{currentstroke}%
\pgfsetdash{}{0pt}%
\pgfpathmoveto{\pgfqpoint{0.584848in}{0.391091in}}%
\pgfpathlineto{\pgfqpoint{0.585968in}{0.386694in}}%
\pgfpathlineto{\pgfqpoint{0.587089in}{0.388562in}}%
\pgfpathlineto{\pgfqpoint{0.588209in}{0.385704in}}%
\pgfpathlineto{\pgfqpoint{0.593812in}{0.381857in}}%
\pgfpathlineto{\pgfqpoint{0.594933in}{0.387133in}}%
\pgfpathlineto{\pgfqpoint{0.596053in}{0.385704in}}%
\pgfpathlineto{\pgfqpoint{0.600535in}{0.389992in}}%
\pgfpathlineto{\pgfqpoint{0.601656in}{0.393070in}}%
\pgfpathlineto{\pgfqpoint{0.603897in}{0.385375in}}%
\pgfpathlineto{\pgfqpoint{0.607258in}{0.382956in}}%
\pgfpathlineto{\pgfqpoint{0.608379in}{0.386364in}}%
\pgfpathlineto{\pgfqpoint{0.609499in}{0.385265in}}%
\pgfpathlineto{\pgfqpoint{0.611740in}{0.386474in}}%
\pgfpathlineto{\pgfqpoint{0.615102in}{0.383616in}}%
\pgfpathlineto{\pgfqpoint{0.616223in}{0.385484in}}%
\pgfpathlineto{\pgfqpoint{0.617343in}{0.389992in}}%
\pgfpathlineto{\pgfqpoint{0.618464in}{0.398566in}}%
\pgfpathlineto{\pgfqpoint{0.619584in}{0.400875in}}%
\pgfpathlineto{\pgfqpoint{0.622946in}{0.401644in}}%
\pgfpathlineto{\pgfqpoint{0.624066in}{0.400765in}}%
\pgfpathlineto{\pgfqpoint{0.625187in}{0.404063in}}%
\pgfpathlineto{\pgfqpoint{0.626307in}{0.413846in}}%
\pgfpathlineto{\pgfqpoint{0.627428in}{0.417364in}}%
\pgfpathlineto{\pgfqpoint{0.630789in}{0.414506in}}%
\pgfpathlineto{\pgfqpoint{0.631910in}{0.420552in}}%
\pgfpathlineto{\pgfqpoint{0.633031in}{0.422421in}}%
\pgfpathlineto{\pgfqpoint{0.634151in}{0.419233in}}%
\pgfpathlineto{\pgfqpoint{0.635272in}{0.422201in}}%
\pgfpathlineto{\pgfqpoint{0.639754in}{0.420222in}}%
\pgfpathlineto{\pgfqpoint{0.640874in}{0.425499in}}%
\pgfpathlineto{\pgfqpoint{0.641995in}{0.425719in}}%
\pgfpathlineto{\pgfqpoint{0.643115in}{0.428357in}}%
\pgfpathlineto{\pgfqpoint{0.646477in}{0.426708in}}%
\pgfpathlineto{\pgfqpoint{0.647597in}{0.431765in}}%
\pgfpathlineto{\pgfqpoint{0.648718in}{0.425499in}}%
\pgfpathlineto{\pgfqpoint{0.649838in}{0.427478in}}%
\pgfpathlineto{\pgfqpoint{0.650959in}{0.424949in}}%
\pgfpathlineto{\pgfqpoint{0.654321in}{0.425169in}}%
\pgfpathlineto{\pgfqpoint{0.655441in}{0.421761in}}%
\pgfpathlineto{\pgfqpoint{0.656562in}{0.423740in}}%
\pgfpathlineto{\pgfqpoint{0.657682in}{0.429347in}}%
\pgfpathlineto{\pgfqpoint{0.658803in}{0.427368in}}%
\pgfpathlineto{\pgfqpoint{0.662164in}{0.425829in}}%
\pgfpathlineto{\pgfqpoint{0.663285in}{0.427698in}}%
\pgfpathlineto{\pgfqpoint{0.664405in}{0.426379in}}%
\pgfpathlineto{\pgfqpoint{0.665526in}{0.427038in}}%
\pgfpathlineto{\pgfqpoint{0.666646in}{0.426269in}}%
\pgfpathlineto{\pgfqpoint{0.670008in}{0.431435in}}%
\pgfpathlineto{\pgfqpoint{0.671128in}{0.425829in}}%
\pgfpathlineto{\pgfqpoint{0.672249in}{0.426379in}}%
\pgfpathlineto{\pgfqpoint{0.673369in}{0.427698in}}%
\pgfpathlineto{\pgfqpoint{0.674490in}{0.431325in}}%
\pgfpathlineto{\pgfqpoint{0.677852in}{0.434403in}}%
\pgfpathlineto{\pgfqpoint{0.678972in}{0.433854in}}%
\pgfpathlineto{\pgfqpoint{0.682334in}{0.429457in}}%
\pgfpathlineto{\pgfqpoint{0.685695in}{0.431765in}}%
\pgfpathlineto{\pgfqpoint{0.686816in}{0.435173in}}%
\pgfpathlineto{\pgfqpoint{0.687936in}{0.431765in}}%
\pgfpathlineto{\pgfqpoint{0.689057in}{0.436712in}}%
\pgfpathlineto{\pgfqpoint{0.693539in}{0.433084in}}%
\pgfpathlineto{\pgfqpoint{0.694659in}{0.426379in}}%
\pgfpathlineto{\pgfqpoint{0.695780in}{0.427918in}}%
\pgfpathlineto{\pgfqpoint{0.698021in}{0.441879in}}%
\pgfpathlineto{\pgfqpoint{0.701383in}{0.436272in}}%
\pgfpathlineto{\pgfqpoint{0.702503in}{0.439350in}}%
\pgfpathlineto{\pgfqpoint{0.705865in}{0.436712in}}%
\pgfpathlineto{\pgfqpoint{0.709226in}{0.429676in}}%
\pgfpathlineto{\pgfqpoint{0.710347in}{0.431655in}}%
\pgfpathlineto{\pgfqpoint{0.711467in}{0.438581in}}%
\pgfpathlineto{\pgfqpoint{0.712588in}{0.441989in}}%
\pgfpathlineto{\pgfqpoint{0.713708in}{0.442758in}}%
\pgfpathlineto{\pgfqpoint{0.717070in}{0.441439in}}%
\pgfpathlineto{\pgfqpoint{0.718191in}{0.441769in}}%
\pgfpathlineto{\pgfqpoint{0.719311in}{0.439570in}}%
\pgfpathlineto{\pgfqpoint{0.720432in}{0.425609in}}%
\pgfpathlineto{\pgfqpoint{0.721552in}{0.428907in}}%
\pgfpathlineto{\pgfqpoint{0.724914in}{0.430666in}}%
\pgfpathlineto{\pgfqpoint{0.726034in}{0.429786in}}%
\pgfpathlineto{\pgfqpoint{0.727155in}{0.428028in}}%
\pgfpathlineto{\pgfqpoint{0.728275in}{0.429786in}}%
\pgfpathlineto{\pgfqpoint{0.729396in}{0.429237in}}%
\pgfpathlineto{\pgfqpoint{0.732757in}{0.426269in}}%
\pgfpathlineto{\pgfqpoint{0.733878in}{0.426049in}}%
\pgfpathlineto{\pgfqpoint{0.734998in}{0.430446in}}%
\pgfpathlineto{\pgfqpoint{0.737240in}{0.418464in}}%
\pgfpathlineto{\pgfqpoint{0.740601in}{0.427258in}}%
\pgfpathlineto{\pgfqpoint{0.741722in}{0.432535in}}%
\pgfpathlineto{\pgfqpoint{0.743963in}{0.435943in}}%
\pgfpathlineto{\pgfqpoint{0.745083in}{0.434953in}}%
\pgfpathlineto{\pgfqpoint{0.749565in}{0.437152in}}%
\pgfpathlineto{\pgfqpoint{0.752927in}{0.417584in}}%
\pgfpathlineto{\pgfqpoint{0.756288in}{0.422201in}}%
\pgfpathlineto{\pgfqpoint{0.757409in}{0.421322in}}%
\pgfpathlineto{\pgfqpoint{0.758530in}{0.427588in}}%
\pgfpathlineto{\pgfqpoint{0.759650in}{0.428687in}}%
\pgfpathlineto{\pgfqpoint{0.760771in}{0.428137in}}%
\pgfpathlineto{\pgfqpoint{0.764132in}{0.427588in}}%
\pgfpathlineto{\pgfqpoint{0.765253in}{0.429567in}}%
\pgfpathlineto{\pgfqpoint{0.766373in}{0.424070in}}%
\pgfpathlineto{\pgfqpoint{0.768614in}{0.432535in}}%
\pgfpathlineto{\pgfqpoint{0.771976in}{0.436712in}}%
\pgfpathlineto{\pgfqpoint{0.774217in}{0.442758in}}%
\pgfpathlineto{\pgfqpoint{0.775337in}{0.435173in}}%
\pgfpathlineto{\pgfqpoint{0.776458in}{0.448585in}}%
\pgfpathlineto{\pgfqpoint{0.779820in}{0.439570in}}%
\pgfpathlineto{\pgfqpoint{0.780940in}{0.444187in}}%
\pgfpathlineto{\pgfqpoint{0.782061in}{0.444847in}}%
\pgfpathlineto{\pgfqpoint{0.783181in}{0.440230in}}%
\pgfpathlineto{\pgfqpoint{0.784302in}{0.445397in}}%
\pgfpathlineto{\pgfqpoint{0.787663in}{0.452652in}}%
\pgfpathlineto{\pgfqpoint{0.788784in}{0.451992in}}%
\pgfpathlineto{\pgfqpoint{0.791025in}{0.453641in}}%
\pgfpathlineto{\pgfqpoint{0.792145in}{0.449354in}}%
\pgfpathlineto{\pgfqpoint{0.795507in}{0.445397in}}%
\pgfpathlineto{\pgfqpoint{0.797748in}{0.438251in}}%
\pgfpathlineto{\pgfqpoint{0.799989in}{0.446496in}}%
\pgfpathlineto{\pgfqpoint{0.804471in}{0.456609in}}%
\pgfpathlineto{\pgfqpoint{0.805592in}{0.455620in}}%
\pgfpathlineto{\pgfqpoint{0.806712in}{0.449684in}}%
\pgfpathlineto{\pgfqpoint{0.807833in}{0.450563in}}%
\pgfpathlineto{\pgfqpoint{0.811194in}{0.447375in}}%
\pgfpathlineto{\pgfqpoint{0.812315in}{0.442978in}}%
\pgfpathlineto{\pgfqpoint{0.813435in}{0.441879in}}%
\pgfpathlineto{\pgfqpoint{0.815676in}{0.458808in}}%
\pgfpathlineto{\pgfqpoint{0.819038in}{0.462985in}}%
\pgfpathlineto{\pgfqpoint{0.821279in}{0.454741in}}%
\pgfpathlineto{\pgfqpoint{0.823520in}{0.463315in}}%
\pgfpathlineto{\pgfqpoint{0.826882in}{0.463425in}}%
\pgfpathlineto{\pgfqpoint{0.828002in}{0.461886in}}%
\pgfpathlineto{\pgfqpoint{0.829123in}{0.463755in}}%
\pgfpathlineto{\pgfqpoint{0.830243in}{0.457269in}}%
\pgfpathlineto{\pgfqpoint{0.831364in}{0.458698in}}%
\pgfpathlineto{\pgfqpoint{0.834725in}{0.456829in}}%
\pgfpathlineto{\pgfqpoint{0.835846in}{0.460677in}}%
\pgfpathlineto{\pgfqpoint{0.836966in}{0.460787in}}%
\pgfpathlineto{\pgfqpoint{0.838087in}{0.462326in}}%
\pgfpathlineto{\pgfqpoint{0.839208in}{0.459797in}}%
\pgfpathlineto{\pgfqpoint{0.842569in}{0.458588in}}%
\pgfpathlineto{\pgfqpoint{0.843690in}{0.456939in}}%
\pgfpathlineto{\pgfqpoint{0.844810in}{0.458368in}}%
\pgfpathlineto{\pgfqpoint{0.847051in}{0.454851in}}%
\pgfpathlineto{\pgfqpoint{0.850413in}{0.457269in}}%
\pgfpathlineto{\pgfqpoint{0.851533in}{0.456719in}}%
\pgfpathlineto{\pgfqpoint{0.852654in}{0.458148in}}%
\pgfpathlineto{\pgfqpoint{0.853774in}{0.454960in}}%
\pgfpathlineto{\pgfqpoint{0.854895in}{0.458808in}}%
\pgfpathlineto{\pgfqpoint{0.859377in}{0.459468in}}%
\pgfpathlineto{\pgfqpoint{0.860498in}{0.457159in}}%
\pgfpathlineto{\pgfqpoint{0.861618in}{0.461886in}}%
\pgfpathlineto{\pgfqpoint{0.862739in}{0.462326in}}%
\pgfpathlineto{\pgfqpoint{0.866100in}{0.459797in}}%
\pgfpathlineto{\pgfqpoint{0.868341in}{0.471560in}}%
\pgfpathlineto{\pgfqpoint{0.869462in}{0.475298in}}%
\pgfpathlineto{\pgfqpoint{0.870582in}{0.473539in}}%
\pgfpathlineto{\pgfqpoint{0.875064in}{0.472110in}}%
\pgfpathlineto{\pgfqpoint{0.876185in}{0.474858in}}%
\pgfpathlineto{\pgfqpoint{0.877305in}{0.474418in}}%
\pgfpathlineto{\pgfqpoint{0.878426in}{0.475408in}}%
\pgfpathlineto{\pgfqpoint{0.881788in}{0.472220in}}%
\pgfpathlineto{\pgfqpoint{0.882908in}{0.474308in}}%
\pgfpathlineto{\pgfqpoint{0.884029in}{0.469251in}}%
\pgfpathlineto{\pgfqpoint{0.885149in}{0.472659in}}%
\pgfpathlineto{\pgfqpoint{0.886270in}{0.473649in}}%
\pgfpathlineto{\pgfqpoint{0.889631in}{0.479585in}}%
\pgfpathlineto{\pgfqpoint{0.890752in}{0.477826in}}%
\pgfpathlineto{\pgfqpoint{0.891872in}{0.483982in}}%
\pgfpathlineto{\pgfqpoint{0.894113in}{0.488269in}}%
\pgfpathlineto{\pgfqpoint{0.897475in}{0.484532in}}%
\pgfpathlineto{\pgfqpoint{0.898595in}{0.479695in}}%
\pgfpathlineto{\pgfqpoint{0.899716in}{0.481344in}}%
\pgfpathlineto{\pgfqpoint{0.900836in}{0.485301in}}%
\pgfpathlineto{\pgfqpoint{0.901957in}{0.485521in}}%
\pgfpathlineto{\pgfqpoint{0.905319in}{0.487390in}}%
\pgfpathlineto{\pgfqpoint{0.907560in}{0.493986in}}%
\pgfpathlineto{\pgfqpoint{0.908680in}{0.492337in}}%
\pgfpathlineto{\pgfqpoint{0.909801in}{0.487610in}}%
\pgfpathlineto{\pgfqpoint{0.913162in}{0.485191in}}%
\pgfpathlineto{\pgfqpoint{0.914283in}{0.479365in}}%
\pgfpathlineto{\pgfqpoint{0.915403in}{0.479035in}}%
\pgfpathlineto{\pgfqpoint{0.917644in}{0.483542in}}%
\pgfpathlineto{\pgfqpoint{0.923247in}{0.484642in}}%
\pgfpathlineto{\pgfqpoint{0.924368in}{0.497174in}}%
\pgfpathlineto{\pgfqpoint{0.925488in}{0.495965in}}%
\pgfpathlineto{\pgfqpoint{0.928850in}{0.491347in}}%
\pgfpathlineto{\pgfqpoint{0.929970in}{0.497174in}}%
\pgfpathlineto{\pgfqpoint{0.931091in}{0.493986in}}%
\pgfpathlineto{\pgfqpoint{0.932211in}{0.492777in}}%
\pgfpathlineto{\pgfqpoint{0.933332in}{0.494865in}}%
\pgfpathlineto{\pgfqpoint{0.936693in}{0.495635in}}%
\pgfpathlineto{\pgfqpoint{0.938934in}{0.490798in}}%
\pgfpathlineto{\pgfqpoint{0.940055in}{0.490908in}}%
\pgfpathlineto{\pgfqpoint{0.941175in}{0.497614in}}%
\pgfpathlineto{\pgfqpoint{0.945658in}{0.506628in}}%
\pgfpathlineto{\pgfqpoint{0.946778in}{0.507397in}}%
\pgfpathlineto{\pgfqpoint{0.949019in}{0.511025in}}%
\pgfpathlineto{\pgfqpoint{0.953501in}{0.507727in}}%
\pgfpathlineto{\pgfqpoint{0.954622in}{0.508936in}}%
\pgfpathlineto{\pgfqpoint{0.955742in}{0.511135in}}%
\pgfpathlineto{\pgfqpoint{0.956863in}{0.514983in}}%
\pgfpathlineto{\pgfqpoint{0.960224in}{0.512344in}}%
\pgfpathlineto{\pgfqpoint{0.961345in}{0.509706in}}%
\pgfpathlineto{\pgfqpoint{0.963586in}{0.511904in}}%
\pgfpathlineto{\pgfqpoint{0.969189in}{0.513004in}}%
\pgfpathlineto{\pgfqpoint{0.971430in}{0.508716in}}%
\pgfpathlineto{\pgfqpoint{0.972550in}{0.507837in}}%
\pgfpathlineto{\pgfqpoint{0.975912in}{0.513553in}}%
\pgfpathlineto{\pgfqpoint{0.977032in}{0.517511in}}%
\pgfpathlineto{\pgfqpoint{0.978153in}{0.513663in}}%
\pgfpathlineto{\pgfqpoint{0.979273in}{0.521798in}}%
\pgfpathlineto{\pgfqpoint{0.980394in}{0.517621in}}%
\pgfpathlineto{\pgfqpoint{0.983756in}{0.518610in}}%
\pgfpathlineto{\pgfqpoint{0.988238in}{0.512344in}}%
\pgfpathlineto{\pgfqpoint{0.991599in}{0.519600in}}%
\pgfpathlineto{\pgfqpoint{0.993840in}{0.529054in}}%
\pgfpathlineto{\pgfqpoint{0.994961in}{0.529383in}}%
\pgfpathlineto{\pgfqpoint{0.996081in}{0.532461in}}%
\pgfpathlineto{\pgfqpoint{0.999443in}{0.535320in}}%
\pgfpathlineto{\pgfqpoint{1.000563in}{0.538947in}}%
\pgfpathlineto{\pgfqpoint{1.001684in}{0.544994in}}%
\pgfpathlineto{\pgfqpoint{1.002804in}{0.541806in}}%
\pgfpathlineto{\pgfqpoint{1.003925in}{0.543345in}}%
\pgfpathlineto{\pgfqpoint{1.009528in}{0.541036in}}%
\pgfpathlineto{\pgfqpoint{1.010648in}{0.540267in}}%
\pgfpathlineto{\pgfqpoint{1.011769in}{0.536199in}}%
\pgfpathlineto{\pgfqpoint{1.019612in}{0.540157in}}%
\pgfpathlineto{\pgfqpoint{1.022974in}{0.531472in}}%
\pgfpathlineto{\pgfqpoint{1.024094in}{0.532242in}}%
\pgfpathlineto{\pgfqpoint{1.025215in}{0.528174in}}%
\pgfpathlineto{\pgfqpoint{1.026336in}{0.535320in}}%
\pgfpathlineto{\pgfqpoint{1.027456in}{0.536969in}}%
\pgfpathlineto{\pgfqpoint{1.030818in}{0.532242in}}%
\pgfpathlineto{\pgfqpoint{1.031938in}{0.539387in}}%
\pgfpathlineto{\pgfqpoint{1.033059in}{0.542575in}}%
\pgfpathlineto{\pgfqpoint{1.034179in}{0.533231in}}%
\pgfpathlineto{\pgfqpoint{1.035300in}{0.535100in}}%
\pgfpathlineto{\pgfqpoint{1.038661in}{0.531362in}}%
\pgfpathlineto{\pgfqpoint{1.039782in}{0.532571in}}%
\pgfpathlineto{\pgfqpoint{1.040902in}{0.530812in}}%
\pgfpathlineto{\pgfqpoint{1.042023in}{0.533781in}}%
\pgfpathlineto{\pgfqpoint{1.043143in}{0.538837in}}%
\pgfpathlineto{\pgfqpoint{1.047626in}{0.537958in}}%
\pgfpathlineto{\pgfqpoint{1.048746in}{0.532461in}}%
\pgfpathlineto{\pgfqpoint{1.050987in}{0.542465in}}%
\pgfpathlineto{\pgfqpoint{1.054349in}{0.533561in}}%
\pgfpathlineto{\pgfqpoint{1.056590in}{0.543564in}}%
\pgfpathlineto{\pgfqpoint{1.057710in}{0.540486in}}%
\pgfpathlineto{\pgfqpoint{1.058831in}{0.539057in}}%
\pgfpathlineto{\pgfqpoint{1.063313in}{0.544664in}}%
\pgfpathlineto{\pgfqpoint{1.064433in}{0.546533in}}%
\pgfpathlineto{\pgfqpoint{1.065554in}{0.545213in}}%
\pgfpathlineto{\pgfqpoint{1.070036in}{0.546862in}}%
\pgfpathlineto{\pgfqpoint{1.071157in}{0.543345in}}%
\pgfpathlineto{\pgfqpoint{1.072277in}{0.542355in}}%
\pgfpathlineto{\pgfqpoint{1.073398in}{0.545543in}}%
\pgfpathlineto{\pgfqpoint{1.074518in}{0.540267in}}%
\pgfpathlineto{\pgfqpoint{1.077880in}{0.539057in}}%
\pgfpathlineto{\pgfqpoint{1.079000in}{0.533891in}}%
\pgfpathlineto{\pgfqpoint{1.080121in}{0.542245in}}%
\pgfpathlineto{\pgfqpoint{1.081241in}{0.538398in}}%
\pgfpathlineto{\pgfqpoint{1.082362in}{0.544114in}}%
\pgfpathlineto{\pgfqpoint{1.085723in}{0.553898in}}%
\pgfpathlineto{\pgfqpoint{1.086844in}{0.562582in}}%
\pgfpathlineto{\pgfqpoint{1.089085in}{0.568519in}}%
\pgfpathlineto{\pgfqpoint{1.093567in}{0.562253in}}%
\pgfpathlineto{\pgfqpoint{1.094688in}{0.563792in}}%
\pgfpathlineto{\pgfqpoint{1.095808in}{0.555767in}}%
\pgfpathlineto{\pgfqpoint{1.096929in}{0.560054in}}%
\pgfpathlineto{\pgfqpoint{1.098049in}{0.556756in}}%
\pgfpathlineto{\pgfqpoint{1.101411in}{0.559614in}}%
\pgfpathlineto{\pgfqpoint{1.102531in}{0.555767in}}%
\pgfpathlineto{\pgfqpoint{1.103652in}{0.561373in}}%
\pgfpathlineto{\pgfqpoint{1.104772in}{0.562912in}}%
\pgfpathlineto{\pgfqpoint{1.105893in}{0.558295in}}%
\pgfpathlineto{\pgfqpoint{1.109255in}{0.547082in}}%
\pgfpathlineto{\pgfqpoint{1.110375in}{0.555657in}}%
\pgfpathlineto{\pgfqpoint{1.111496in}{0.549501in}}%
\pgfpathlineto{\pgfqpoint{1.112616in}{0.547412in}}%
\pgfpathlineto{\pgfqpoint{1.113737in}{0.553788in}}%
\pgfpathlineto{\pgfqpoint{1.117098in}{0.552689in}}%
\pgfpathlineto{\pgfqpoint{1.119339in}{0.561593in}}%
\pgfpathlineto{\pgfqpoint{1.120460in}{0.565770in}}%
\pgfpathlineto{\pgfqpoint{1.121580in}{0.562033in}}%
\pgfpathlineto{\pgfqpoint{1.124942in}{0.563682in}}%
\pgfpathlineto{\pgfqpoint{1.126062in}{0.565111in}}%
\pgfpathlineto{\pgfqpoint{1.127183in}{0.558955in}}%
\pgfpathlineto{\pgfqpoint{1.128304in}{0.582480in}}%
\pgfpathlineto{\pgfqpoint{1.129424in}{0.592813in}}%
\pgfpathlineto{\pgfqpoint{1.132786in}{0.591164in}}%
\pgfpathlineto{\pgfqpoint{1.133906in}{0.593473in}}%
\pgfpathlineto{\pgfqpoint{1.136147in}{0.590395in}}%
\pgfpathlineto{\pgfqpoint{1.137268in}{0.590835in}}%
\pgfpathlineto{\pgfqpoint{1.140629in}{0.591274in}}%
\pgfpathlineto{\pgfqpoint{1.141750in}{0.594792in}}%
\pgfpathlineto{\pgfqpoint{1.142870in}{0.601718in}}%
\pgfpathlineto{\pgfqpoint{1.143991in}{0.596771in}}%
\pgfpathlineto{\pgfqpoint{1.145111in}{0.608753in}}%
\pgfpathlineto{\pgfqpoint{1.148473in}{0.601278in}}%
\pgfpathlineto{\pgfqpoint{1.149594in}{0.601168in}}%
\pgfpathlineto{\pgfqpoint{1.151835in}{0.592483in}}%
\pgfpathlineto{\pgfqpoint{1.152955in}{0.598420in}}%
\pgfpathlineto{\pgfqpoint{1.157437in}{0.597320in}}%
\pgfpathlineto{\pgfqpoint{1.158558in}{0.592593in}}%
\pgfpathlineto{\pgfqpoint{1.159678in}{0.599849in}}%
\pgfpathlineto{\pgfqpoint{1.160799in}{0.592593in}}%
\pgfpathlineto{\pgfqpoint{1.164160in}{0.597760in}}%
\pgfpathlineto{\pgfqpoint{1.165281in}{0.597760in}}%
\pgfpathlineto{\pgfqpoint{1.166401in}{0.590285in}}%
\pgfpathlineto{\pgfqpoint{1.167522in}{0.595671in}}%
\pgfpathlineto{\pgfqpoint{1.168642in}{0.597101in}}%
\pgfpathlineto{\pgfqpoint{1.172004in}{0.603147in}}%
\pgfpathlineto{\pgfqpoint{1.173125in}{0.596441in}}%
\pgfpathlineto{\pgfqpoint{1.174245in}{0.595122in}}%
\pgfpathlineto{\pgfqpoint{1.175366in}{0.602927in}}%
\pgfpathlineto{\pgfqpoint{1.176486in}{0.599629in}}%
\pgfpathlineto{\pgfqpoint{1.179848in}{0.603147in}}%
\pgfpathlineto{\pgfqpoint{1.180968in}{0.607434in}}%
\pgfpathlineto{\pgfqpoint{1.182089in}{0.603916in}}%
\pgfpathlineto{\pgfqpoint{1.183209in}{0.593363in}}%
\pgfpathlineto{\pgfqpoint{1.184330in}{0.596001in}}%
\pgfpathlineto{\pgfqpoint{1.187691in}{0.592923in}}%
\pgfpathlineto{\pgfqpoint{1.191053in}{0.607984in}}%
\pgfpathlineto{\pgfqpoint{1.192174in}{0.604246in}}%
\pgfpathlineto{\pgfqpoint{1.195535in}{0.610402in}}%
\pgfpathlineto{\pgfqpoint{1.196656in}{0.610512in}}%
\pgfpathlineto{\pgfqpoint{1.200017in}{0.624473in}}%
\pgfpathlineto{\pgfqpoint{1.205620in}{0.614140in}}%
\pgfpathlineto{\pgfqpoint{1.206740in}{0.622385in}}%
\pgfpathlineto{\pgfqpoint{1.207861in}{0.624253in}}%
\pgfpathlineto{\pgfqpoint{1.211223in}{0.623154in}}%
\pgfpathlineto{\pgfqpoint{1.212343in}{0.620955in}}%
\pgfpathlineto{\pgfqpoint{1.213464in}{0.621615in}}%
\pgfpathlineto{\pgfqpoint{1.214584in}{0.625023in}}%
\pgfpathlineto{\pgfqpoint{1.215705in}{0.622275in}}%
\pgfpathlineto{\pgfqpoint{1.219066in}{0.625573in}}%
\pgfpathlineto{\pgfqpoint{1.221307in}{0.614250in}}%
\pgfpathlineto{\pgfqpoint{1.222428in}{0.634147in}}%
\pgfpathlineto{\pgfqpoint{1.223548in}{0.630519in}}%
\pgfpathlineto{\pgfqpoint{1.228030in}{0.626122in}}%
\pgfpathlineto{\pgfqpoint{1.229151in}{0.589735in}}%
\pgfpathlineto{\pgfqpoint{1.230271in}{0.595122in}}%
\pgfpathlineto{\pgfqpoint{1.231392in}{0.607324in}}%
\pgfpathlineto{\pgfqpoint{1.234754in}{0.608753in}}%
\pgfpathlineto{\pgfqpoint{1.236995in}{0.600728in}}%
\pgfpathlineto{\pgfqpoint{1.239236in}{0.596771in}}%
\pgfpathlineto{\pgfqpoint{1.242597in}{0.596661in}}%
\pgfpathlineto{\pgfqpoint{1.243718in}{0.595342in}}%
\pgfpathlineto{\pgfqpoint{1.244838in}{0.596441in}}%
\pgfpathlineto{\pgfqpoint{1.245959in}{0.585228in}}%
\pgfpathlineto{\pgfqpoint{1.247079in}{0.583139in}}%
\pgfpathlineto{\pgfqpoint{1.250441in}{0.587866in}}%
\pgfpathlineto{\pgfqpoint{1.251561in}{0.583359in}}%
\pgfpathlineto{\pgfqpoint{1.252682in}{0.596551in}}%
\pgfpathlineto{\pgfqpoint{1.253803in}{0.597870in}}%
\pgfpathlineto{\pgfqpoint{1.254923in}{0.598310in}}%
\pgfpathlineto{\pgfqpoint{1.259405in}{0.585778in}}%
\pgfpathlineto{\pgfqpoint{1.260526in}{0.588416in}}%
\pgfpathlineto{\pgfqpoint{1.261646in}{0.588856in}}%
\pgfpathlineto{\pgfqpoint{1.262767in}{0.586437in}}%
\pgfpathlineto{\pgfqpoint{1.267249in}{0.593033in}}%
\pgfpathlineto{\pgfqpoint{1.268369in}{0.590944in}}%
\pgfpathlineto{\pgfqpoint{1.269490in}{0.591054in}}%
\pgfpathlineto{\pgfqpoint{1.270610in}{0.592154in}}%
\pgfpathlineto{\pgfqpoint{1.273972in}{0.596991in}}%
\pgfpathlineto{\pgfqpoint{1.275093in}{0.612271in}}%
\pgfpathlineto{\pgfqpoint{1.276213in}{0.616668in}}%
\pgfpathlineto{\pgfqpoint{1.277334in}{0.613480in}}%
\pgfpathlineto{\pgfqpoint{1.278454in}{0.623484in}}%
\pgfpathlineto{\pgfqpoint{1.281816in}{0.624473in}}%
\pgfpathlineto{\pgfqpoint{1.284057in}{0.635356in}}%
\pgfpathlineto{\pgfqpoint{1.285177in}{0.637995in}}%
\pgfpathlineto{\pgfqpoint{1.286298in}{0.648438in}}%
\pgfpathlineto{\pgfqpoint{1.289659in}{0.641842in}}%
\pgfpathlineto{\pgfqpoint{1.291900in}{0.629970in}}%
\pgfpathlineto{\pgfqpoint{1.293021in}{0.635027in}}%
\pgfpathlineto{\pgfqpoint{1.294142in}{0.633817in}}%
\pgfpathlineto{\pgfqpoint{1.297503in}{0.628870in}}%
\pgfpathlineto{\pgfqpoint{1.298624in}{0.634257in}}%
\pgfpathlineto{\pgfqpoint{1.299744in}{0.630629in}}%
\pgfpathlineto{\pgfqpoint{1.300865in}{0.622604in}}%
\pgfpathlineto{\pgfqpoint{1.301985in}{0.627222in}}%
\pgfpathlineto{\pgfqpoint{1.305347in}{0.616668in}}%
\pgfpathlineto{\pgfqpoint{1.306467in}{0.607104in}}%
\pgfpathlineto{\pgfqpoint{1.307588in}{0.610402in}}%
\pgfpathlineto{\pgfqpoint{1.309829in}{0.631619in}}%
\pgfpathlineto{\pgfqpoint{1.313190in}{0.634807in}}%
\pgfpathlineto{\pgfqpoint{1.314311in}{0.629530in}}%
\pgfpathlineto{\pgfqpoint{1.316552in}{0.647119in}}%
\pgfpathlineto{\pgfqpoint{1.317673in}{0.652615in}}%
\pgfpathlineto{\pgfqpoint{1.322155in}{0.651406in}}%
\pgfpathlineto{\pgfqpoint{1.323275in}{0.648658in}}%
\pgfpathlineto{\pgfqpoint{1.324396in}{0.658881in}}%
\pgfpathlineto{\pgfqpoint{1.325516in}{0.659211in}}%
\pgfpathlineto{\pgfqpoint{1.328878in}{0.659211in}}%
\pgfpathlineto{\pgfqpoint{1.329998in}{0.662179in}}%
\pgfpathlineto{\pgfqpoint{1.331119in}{0.661190in}}%
\pgfpathlineto{\pgfqpoint{1.332239in}{0.642942in}}%
\pgfpathlineto{\pgfqpoint{1.333360in}{0.649318in}}%
\pgfpathlineto{\pgfqpoint{1.336722in}{0.642392in}}%
\pgfpathlineto{\pgfqpoint{1.337842in}{0.644810in}}%
\pgfpathlineto{\pgfqpoint{1.338963in}{0.648988in}}%
\pgfpathlineto{\pgfqpoint{1.340083in}{0.641512in}}%
\pgfpathlineto{\pgfqpoint{1.341204in}{0.647119in}}%
\pgfpathlineto{\pgfqpoint{1.344565in}{0.649537in}}%
\pgfpathlineto{\pgfqpoint{1.345686in}{0.647009in}}%
\pgfpathlineto{\pgfqpoint{1.346806in}{0.655034in}}%
\pgfpathlineto{\pgfqpoint{1.347927in}{0.655913in}}%
\pgfpathlineto{\pgfqpoint{1.349047in}{0.660640in}}%
\pgfpathlineto{\pgfqpoint{1.352409in}{0.655364in}}%
\pgfpathlineto{\pgfqpoint{1.353529in}{0.648878in}}%
\pgfpathlineto{\pgfqpoint{1.354650in}{0.650747in}}%
\pgfpathlineto{\pgfqpoint{1.355771in}{0.659651in}}%
\pgfpathlineto{\pgfqpoint{1.356891in}{0.660970in}}%
\pgfpathlineto{\pgfqpoint{1.360253in}{0.660530in}}%
\pgfpathlineto{\pgfqpoint{1.361373in}{0.664598in}}%
\pgfpathlineto{\pgfqpoint{1.362494in}{0.665807in}}%
\pgfpathlineto{\pgfqpoint{1.364735in}{0.664378in}}%
\pgfpathlineto{\pgfqpoint{1.368096in}{0.668665in}}%
\pgfpathlineto{\pgfqpoint{1.369217in}{0.660091in}}%
\pgfpathlineto{\pgfqpoint{1.370337in}{0.662619in}}%
\pgfpathlineto{\pgfqpoint{1.371458in}{0.659981in}}%
\pgfpathlineto{\pgfqpoint{1.372578in}{0.660311in}}%
\pgfpathlineto{\pgfqpoint{1.375940in}{0.659651in}}%
\pgfpathlineto{\pgfqpoint{1.377061in}{0.654154in}}%
\pgfpathlineto{\pgfqpoint{1.378181in}{0.670094in}}%
\pgfpathlineto{\pgfqpoint{1.379302in}{0.664378in}}%
\pgfpathlineto{\pgfqpoint{1.380422in}{0.674382in}}%
\pgfpathlineto{\pgfqpoint{1.383784in}{0.675371in}}%
\pgfpathlineto{\pgfqpoint{1.384904in}{0.689552in}}%
\pgfpathlineto{\pgfqpoint{1.386025in}{0.694829in}}%
\pgfpathlineto{\pgfqpoint{1.387145in}{0.696698in}}%
\pgfpathlineto{\pgfqpoint{1.388266in}{0.696478in}}%
\pgfpathlineto{\pgfqpoint{1.394989in}{0.707251in}}%
\pgfpathlineto{\pgfqpoint{1.396109in}{0.705932in}}%
\pgfpathlineto{\pgfqpoint{1.399471in}{0.709120in}}%
\pgfpathlineto{\pgfqpoint{1.400592in}{0.713627in}}%
\pgfpathlineto{\pgfqpoint{1.402833in}{0.709340in}}%
\pgfpathlineto{\pgfqpoint{1.403953in}{0.709779in}}%
\pgfpathlineto{\pgfqpoint{1.407315in}{0.706371in}}%
\pgfpathlineto{\pgfqpoint{1.408435in}{0.710659in}}%
\pgfpathlineto{\pgfqpoint{1.409556in}{0.712418in}}%
\pgfpathlineto{\pgfqpoint{1.410676in}{0.711648in}}%
\pgfpathlineto{\pgfqpoint{1.415158in}{0.703953in}}%
\pgfpathlineto{\pgfqpoint{1.416279in}{0.713627in}}%
\pgfpathlineto{\pgfqpoint{1.417400in}{0.716375in}}%
\pgfpathlineto{\pgfqpoint{1.418520in}{0.711318in}}%
\pgfpathlineto{\pgfqpoint{1.419641in}{0.738031in}}%
\pgfpathlineto{\pgfqpoint{1.424123in}{0.737372in}}%
\pgfpathlineto{\pgfqpoint{1.425243in}{0.740010in}}%
\pgfpathlineto{\pgfqpoint{1.427484in}{0.709999in}}%
\pgfpathlineto{\pgfqpoint{1.430846in}{0.697137in}}%
\pgfpathlineto{\pgfqpoint{1.431966in}{0.709230in}}%
\pgfpathlineto{\pgfqpoint{1.433087in}{0.699446in}}%
\pgfpathlineto{\pgfqpoint{1.434207in}{0.709010in}}%
\pgfpathlineto{\pgfqpoint{1.435328in}{0.695049in}}%
\pgfpathlineto{\pgfqpoint{1.438690in}{0.690102in}}%
\pgfpathlineto{\pgfqpoint{1.440931in}{0.695598in}}%
\pgfpathlineto{\pgfqpoint{1.443172in}{0.711318in}}%
\pgfpathlineto{\pgfqpoint{1.446533in}{0.708240in}}%
\pgfpathlineto{\pgfqpoint{1.447654in}{0.712747in}}%
\pgfpathlineto{\pgfqpoint{1.448774in}{0.721652in}}%
\pgfpathlineto{\pgfqpoint{1.449895in}{0.721322in}}%
\pgfpathlineto{\pgfqpoint{1.451015in}{0.726379in}}%
\pgfpathlineto{\pgfqpoint{1.455497in}{0.726489in}}%
\pgfpathlineto{\pgfqpoint{1.456618in}{0.720882in}}%
\pgfpathlineto{\pgfqpoint{1.457738in}{0.719783in}}%
\pgfpathlineto{\pgfqpoint{1.458859in}{0.719563in}}%
\pgfpathlineto{\pgfqpoint{1.463341in}{0.729237in}}%
\pgfpathlineto{\pgfqpoint{1.464462in}{0.726599in}}%
\pgfpathlineto{\pgfqpoint{1.465582in}{0.727038in}}%
\pgfpathlineto{\pgfqpoint{1.466703in}{0.726159in}}%
\pgfpathlineto{\pgfqpoint{1.470064in}{0.714616in}}%
\pgfpathlineto{\pgfqpoint{1.471185in}{0.725060in}}%
\pgfpathlineto{\pgfqpoint{1.472305in}{0.718134in}}%
\pgfpathlineto{\pgfqpoint{1.473426in}{0.720772in}}%
\pgfpathlineto{\pgfqpoint{1.474546in}{0.725170in}}%
\pgfpathlineto{\pgfqpoint{1.477908in}{0.725060in}}%
\pgfpathlineto{\pgfqpoint{1.479029in}{0.729347in}}%
\pgfpathlineto{\pgfqpoint{1.480149in}{0.726599in}}%
\pgfpathlineto{\pgfqpoint{1.481270in}{0.712857in}}%
\pgfpathlineto{\pgfqpoint{1.482390in}{0.712857in}}%
\pgfpathlineto{\pgfqpoint{1.485752in}{0.720662in}}%
\pgfpathlineto{\pgfqpoint{1.486872in}{0.727368in}}%
\pgfpathlineto{\pgfqpoint{1.489113in}{0.715606in}}%
\pgfpathlineto{\pgfqpoint{1.490234in}{0.719563in}}%
\pgfpathlineto{\pgfqpoint{1.493595in}{0.712857in}}%
\pgfpathlineto{\pgfqpoint{1.495836in}{0.699776in}}%
\pgfpathlineto{\pgfqpoint{1.496957in}{0.699995in}}%
\pgfpathlineto{\pgfqpoint{1.498077in}{0.690432in}}%
\pgfpathlineto{\pgfqpoint{1.501439in}{0.700215in}}%
\pgfpathlineto{\pgfqpoint{1.502560in}{0.697247in}}%
\pgfpathlineto{\pgfqpoint{1.503680in}{0.697137in}}%
\pgfpathlineto{\pgfqpoint{1.504801in}{0.698017in}}%
\pgfpathlineto{\pgfqpoint{1.505921in}{0.679109in}}%
\pgfpathlineto{\pgfqpoint{1.510403in}{0.665917in}}%
\pgfpathlineto{\pgfqpoint{1.511524in}{0.678779in}}%
\pgfpathlineto{\pgfqpoint{1.513765in}{0.650637in}}%
\pgfpathlineto{\pgfqpoint{1.517126in}{0.661850in}}%
\pgfpathlineto{\pgfqpoint{1.518247in}{0.669765in}}%
\pgfpathlineto{\pgfqpoint{1.519367in}{0.683396in}}%
\pgfpathlineto{\pgfqpoint{1.520488in}{0.679768in}}%
\pgfpathlineto{\pgfqpoint{1.524970in}{0.682846in}}%
\pgfpathlineto{\pgfqpoint{1.526091in}{0.685045in}}%
\pgfpathlineto{\pgfqpoint{1.527211in}{0.682077in}}%
\pgfpathlineto{\pgfqpoint{1.528332in}{0.683506in}}%
\pgfpathlineto{\pgfqpoint{1.529452in}{0.656573in}}%
\pgfpathlineto{\pgfqpoint{1.535055in}{0.666027in}}%
\pgfpathlineto{\pgfqpoint{1.536175in}{0.675041in}}%
\pgfpathlineto{\pgfqpoint{1.537296in}{0.670644in}}%
\pgfpathlineto{\pgfqpoint{1.540658in}{0.677680in}}%
\pgfpathlineto{\pgfqpoint{1.541778in}{0.673063in}}%
\pgfpathlineto{\pgfqpoint{1.544019in}{0.687353in}}%
\pgfpathlineto{\pgfqpoint{1.545140in}{0.687134in}}%
\pgfpathlineto{\pgfqpoint{1.549622in}{0.690432in}}%
\pgfpathlineto{\pgfqpoint{1.550742in}{0.688783in}}%
\pgfpathlineto{\pgfqpoint{1.551863in}{0.682517in}}%
\pgfpathlineto{\pgfqpoint{1.552983in}{0.688673in}}%
\pgfpathlineto{\pgfqpoint{1.556345in}{0.690102in}}%
\pgfpathlineto{\pgfqpoint{1.557465in}{0.683836in}}%
\pgfpathlineto{\pgfqpoint{1.558586in}{0.689442in}}%
\pgfpathlineto{\pgfqpoint{1.559706in}{0.687573in}}%
\pgfpathlineto{\pgfqpoint{1.560827in}{0.694609in}}%
\pgfpathlineto{\pgfqpoint{1.565309in}{0.700875in}}%
\pgfpathlineto{\pgfqpoint{1.566430in}{0.699116in}}%
\pgfpathlineto{\pgfqpoint{1.567550in}{0.701315in}}%
\pgfpathlineto{\pgfqpoint{1.568671in}{0.701754in}}%
\pgfpathlineto{\pgfqpoint{1.572032in}{0.698347in}}%
\pgfpathlineto{\pgfqpoint{1.573153in}{0.692630in}}%
\pgfpathlineto{\pgfqpoint{1.574273in}{0.692960in}}%
\pgfpathlineto{\pgfqpoint{1.576514in}{0.697027in}}%
\pgfpathlineto{\pgfqpoint{1.579876in}{0.695928in}}%
\pgfpathlineto{\pgfqpoint{1.580996in}{0.700215in}}%
\pgfpathlineto{\pgfqpoint{1.583238in}{0.693839in}}%
\pgfpathlineto{\pgfqpoint{1.584358in}{0.692520in}}%
\pgfpathlineto{\pgfqpoint{1.587720in}{0.689772in}}%
\pgfpathlineto{\pgfqpoint{1.589961in}{0.692410in}}%
\pgfpathlineto{\pgfqpoint{1.592202in}{0.687903in}}%
\pgfpathlineto{\pgfqpoint{1.595563in}{0.687793in}}%
\pgfpathlineto{\pgfqpoint{1.596684in}{0.683506in}}%
\pgfpathlineto{\pgfqpoint{1.597804in}{0.686804in}}%
\pgfpathlineto{\pgfqpoint{1.598925in}{0.686474in}}%
\pgfpathlineto{\pgfqpoint{1.600045in}{0.687353in}}%
\pgfpathlineto{\pgfqpoint{1.603407in}{0.691091in}}%
\pgfpathlineto{\pgfqpoint{1.604528in}{0.700215in}}%
\pgfpathlineto{\pgfqpoint{1.605648in}{0.701754in}}%
\pgfpathlineto{\pgfqpoint{1.606769in}{0.706152in}}%
\pgfpathlineto{\pgfqpoint{1.611251in}{0.706701in}}%
\pgfpathlineto{\pgfqpoint{1.612371in}{0.702854in}}%
\pgfpathlineto{\pgfqpoint{1.613492in}{0.705162in}}%
\pgfpathlineto{\pgfqpoint{1.614612in}{0.703623in}}%
\pgfpathlineto{\pgfqpoint{1.619094in}{0.717914in}}%
\pgfpathlineto{\pgfqpoint{1.621335in}{0.722091in}}%
\pgfpathlineto{\pgfqpoint{1.622456in}{0.709669in}}%
\pgfpathlineto{\pgfqpoint{1.623577in}{0.715716in}}%
\pgfpathlineto{\pgfqpoint{1.626938in}{0.713187in}}%
\pgfpathlineto{\pgfqpoint{1.628059in}{0.718464in}}%
\pgfpathlineto{\pgfqpoint{1.629179in}{0.718354in}}%
\pgfpathlineto{\pgfqpoint{1.630300in}{0.722311in}}%
\pgfpathlineto{\pgfqpoint{1.631420in}{0.701644in}}%
\pgfpathlineto{\pgfqpoint{1.634782in}{0.700215in}}%
\pgfpathlineto{\pgfqpoint{1.635902in}{0.698237in}}%
\pgfpathlineto{\pgfqpoint{1.637023in}{0.699776in}}%
\pgfpathlineto{\pgfqpoint{1.638143in}{0.691861in}}%
\pgfpathlineto{\pgfqpoint{1.639264in}{0.693949in}}%
\pgfpathlineto{\pgfqpoint{1.642625in}{0.694829in}}%
\pgfpathlineto{\pgfqpoint{1.643746in}{0.690651in}}%
\pgfpathlineto{\pgfqpoint{1.644867in}{0.690981in}}%
\pgfpathlineto{\pgfqpoint{1.645987in}{0.686694in}}%
\pgfpathlineto{\pgfqpoint{1.647108in}{0.690432in}}%
\pgfpathlineto{\pgfqpoint{1.650469in}{0.690651in}}%
\pgfpathlineto{\pgfqpoint{1.651590in}{0.689992in}}%
\pgfpathlineto{\pgfqpoint{1.652710in}{0.697137in}}%
\pgfpathlineto{\pgfqpoint{1.653831in}{0.699995in}}%
\pgfpathlineto{\pgfqpoint{1.654951in}{0.693839in}}%
\pgfpathlineto{\pgfqpoint{1.659433in}{0.706481in}}%
\pgfpathlineto{\pgfqpoint{1.660554in}{0.709669in}}%
\pgfpathlineto{\pgfqpoint{1.661674in}{0.708460in}}%
\pgfpathlineto{\pgfqpoint{1.662795in}{0.709340in}}%
\pgfpathlineto{\pgfqpoint{1.666157in}{0.709120in}}%
\pgfpathlineto{\pgfqpoint{1.668398in}{0.712088in}}%
\pgfpathlineto{\pgfqpoint{1.670639in}{0.699995in}}%
\pgfpathlineto{\pgfqpoint{1.676241in}{0.706262in}}%
\pgfpathlineto{\pgfqpoint{1.677362in}{0.704503in}}%
\pgfpathlineto{\pgfqpoint{1.678482in}{0.704393in}}%
\pgfpathlineto{\pgfqpoint{1.681844in}{0.708460in}}%
\pgfpathlineto{\pgfqpoint{1.682964in}{0.704613in}}%
\pgfpathlineto{\pgfqpoint{1.684085in}{0.711428in}}%
\pgfpathlineto{\pgfqpoint{1.685206in}{0.706371in}}%
\pgfpathlineto{\pgfqpoint{1.686326in}{0.703953in}}%
\pgfpathlineto{\pgfqpoint{1.689688in}{0.705492in}}%
\pgfpathlineto{\pgfqpoint{1.690808in}{0.712528in}}%
\pgfpathlineto{\pgfqpoint{1.691929in}{0.707910in}}%
\pgfpathlineto{\pgfqpoint{1.693049in}{0.710219in}}%
\pgfpathlineto{\pgfqpoint{1.694170in}{0.709779in}}%
\pgfpathlineto{\pgfqpoint{1.697531in}{0.703513in}}%
\pgfpathlineto{\pgfqpoint{1.698652in}{0.699776in}}%
\pgfpathlineto{\pgfqpoint{1.699772in}{0.704942in}}%
\pgfpathlineto{\pgfqpoint{1.700893in}{0.695268in}}%
\pgfpathlineto{\pgfqpoint{1.702013in}{0.698456in}}%
\pgfpathlineto{\pgfqpoint{1.705375in}{0.695818in}}%
\pgfpathlineto{\pgfqpoint{1.706496in}{0.702194in}}%
\pgfpathlineto{\pgfqpoint{1.707616in}{0.693400in}}%
\pgfpathlineto{\pgfqpoint{1.708737in}{0.692630in}}%
\pgfpathlineto{\pgfqpoint{1.709857in}{0.698566in}}%
\pgfpathlineto{\pgfqpoint{1.713219in}{0.698017in}}%
\pgfpathlineto{\pgfqpoint{1.714339in}{0.688233in}}%
\pgfpathlineto{\pgfqpoint{1.715460in}{0.699556in}}%
\pgfpathlineto{\pgfqpoint{1.717701in}{0.680208in}}%
\pgfpathlineto{\pgfqpoint{1.721062in}{0.678449in}}%
\pgfpathlineto{\pgfqpoint{1.722183in}{0.674272in}}%
\pgfpathlineto{\pgfqpoint{1.723303in}{0.667896in}}%
\pgfpathlineto{\pgfqpoint{1.725544in}{0.682956in}}%
\pgfpathlineto{\pgfqpoint{1.728906in}{0.687793in}}%
\pgfpathlineto{\pgfqpoint{1.730027in}{0.702084in}}%
\pgfpathlineto{\pgfqpoint{1.731147in}{0.695818in}}%
\pgfpathlineto{\pgfqpoint{1.732268in}{0.704613in}}%
\pgfpathlineto{\pgfqpoint{1.733388in}{0.702524in}}%
\pgfpathlineto{\pgfqpoint{1.736750in}{0.702304in}}%
\pgfpathlineto{\pgfqpoint{1.737870in}{0.710989in}}%
\pgfpathlineto{\pgfqpoint{1.738991in}{0.705602in}}%
\pgfpathlineto{\pgfqpoint{1.740111in}{0.763205in}}%
\pgfpathlineto{\pgfqpoint{1.741232in}{0.775738in}}%
\pgfpathlineto{\pgfqpoint{1.744593in}{0.775957in}}%
\pgfpathlineto{\pgfqpoint{1.745714in}{0.779695in}}%
\pgfpathlineto{\pgfqpoint{1.746834in}{0.796954in}}%
\pgfpathlineto{\pgfqpoint{1.747955in}{0.798383in}}%
\pgfpathlineto{\pgfqpoint{1.749076in}{0.804539in}}%
\pgfpathlineto{\pgfqpoint{1.753558in}{0.797394in}}%
\pgfpathlineto{\pgfqpoint{1.754678in}{0.808387in}}%
\pgfpathlineto{\pgfqpoint{1.755799in}{0.805639in}}%
\pgfpathlineto{\pgfqpoint{1.756919in}{0.800142in}}%
\pgfpathlineto{\pgfqpoint{1.760281in}{0.802671in}}%
\pgfpathlineto{\pgfqpoint{1.761401in}{0.802451in}}%
\pgfpathlineto{\pgfqpoint{1.762522in}{0.802890in}}%
\pgfpathlineto{\pgfqpoint{1.764763in}{0.814323in}}%
\pgfpathlineto{\pgfqpoint{1.768125in}{0.815422in}}%
\pgfpathlineto{\pgfqpoint{1.769245in}{0.822348in}}%
\pgfpathlineto{\pgfqpoint{1.770366in}{0.822348in}}%
\pgfpathlineto{\pgfqpoint{1.772607in}{0.824767in}}%
\pgfpathlineto{\pgfqpoint{1.775968in}{0.824767in}}%
\pgfpathlineto{\pgfqpoint{1.778209in}{0.833781in}}%
\pgfpathlineto{\pgfqpoint{1.779330in}{0.832572in}}%
\pgfpathlineto{\pgfqpoint{1.780450in}{0.838398in}}%
\pgfpathlineto{\pgfqpoint{1.783812in}{0.837738in}}%
\pgfpathlineto{\pgfqpoint{1.784932in}{0.840487in}}%
\pgfpathlineto{\pgfqpoint{1.786053in}{0.834111in}}%
\pgfpathlineto{\pgfqpoint{1.787173in}{0.837848in}}%
\pgfpathlineto{\pgfqpoint{1.788294in}{0.821139in}}%
\pgfpathlineto{\pgfqpoint{1.791656in}{0.820919in}}%
\pgfpathlineto{\pgfqpoint{1.792776in}{0.812234in}}%
\pgfpathlineto{\pgfqpoint{1.795017in}{0.840597in}}%
\pgfpathlineto{\pgfqpoint{1.796138in}{0.834001in}}%
\pgfpathlineto{\pgfqpoint{1.800620in}{0.843455in}}%
\pgfpathlineto{\pgfqpoint{1.801740in}{0.849721in}}%
\pgfpathlineto{\pgfqpoint{1.808463in}{0.841586in}}%
\pgfpathlineto{\pgfqpoint{1.809584in}{0.835430in}}%
\pgfpathlineto{\pgfqpoint{1.811825in}{0.842795in}}%
\pgfpathlineto{\pgfqpoint{1.816307in}{0.823008in}}%
\pgfpathlineto{\pgfqpoint{1.818548in}{0.841366in}}%
\pgfpathlineto{\pgfqpoint{1.819669in}{0.831033in}}%
\pgfpathlineto{\pgfqpoint{1.823030in}{0.829603in}}%
\pgfpathlineto{\pgfqpoint{1.824151in}{0.831692in}}%
\pgfpathlineto{\pgfqpoint{1.825271in}{0.817841in}}%
\pgfpathlineto{\pgfqpoint{1.826392in}{0.811465in}}%
\pgfpathlineto{\pgfqpoint{1.827512in}{0.816302in}}%
\pgfpathlineto{\pgfqpoint{1.835356in}{0.825096in}}%
\pgfpathlineto{\pgfqpoint{1.838718in}{0.820479in}}%
\pgfpathlineto{\pgfqpoint{1.840959in}{0.793546in}}%
\pgfpathlineto{\pgfqpoint{1.842079in}{0.797944in}}%
\pgfpathlineto{\pgfqpoint{1.843200in}{0.816192in}}%
\pgfpathlineto{\pgfqpoint{1.846561in}{0.817291in}}%
\pgfpathlineto{\pgfqpoint{1.848802in}{0.842465in}}%
\pgfpathlineto{\pgfqpoint{1.849923in}{0.860714in}}%
\pgfpathlineto{\pgfqpoint{1.851044in}{0.849171in}}%
\pgfpathlineto{\pgfqpoint{1.855526in}{0.841586in}}%
\pgfpathlineto{\pgfqpoint{1.857767in}{0.863572in}}%
\pgfpathlineto{\pgfqpoint{1.858887in}{0.860164in}}%
\pgfpathlineto{\pgfqpoint{1.863369in}{0.863462in}}%
\pgfpathlineto{\pgfqpoint{1.864490in}{0.858845in}}%
\pgfpathlineto{\pgfqpoint{1.865610in}{0.858845in}}%
\pgfpathlineto{\pgfqpoint{1.866731in}{0.869178in}}%
\pgfpathlineto{\pgfqpoint{1.870092in}{0.869178in}}%
\pgfpathlineto{\pgfqpoint{1.871213in}{0.867749in}}%
\pgfpathlineto{\pgfqpoint{1.873454in}{0.871157in}}%
\pgfpathlineto{\pgfqpoint{1.874575in}{0.864671in}}%
\pgfpathlineto{\pgfqpoint{1.877936in}{0.883250in}}%
\pgfpathlineto{\pgfqpoint{1.880177in}{0.871157in}}%
\pgfpathlineto{\pgfqpoint{1.881298in}{0.872147in}}%
\pgfpathlineto{\pgfqpoint{1.882418in}{0.859395in}}%
\pgfpathlineto{\pgfqpoint{1.885780in}{0.865001in}}%
\pgfpathlineto{\pgfqpoint{1.886900in}{0.848402in}}%
\pgfpathlineto{\pgfqpoint{1.888021in}{0.847192in}}%
\pgfpathlineto{\pgfqpoint{1.889141in}{0.860054in}}%
\pgfpathlineto{\pgfqpoint{1.890262in}{0.847962in}}%
\pgfpathlineto{\pgfqpoint{1.893624in}{0.858625in}}%
\pgfpathlineto{\pgfqpoint{1.894744in}{0.846533in}}%
\pgfpathlineto{\pgfqpoint{1.895865in}{0.854997in}}%
\pgfpathlineto{\pgfqpoint{1.896985in}{0.853898in}}%
\pgfpathlineto{\pgfqpoint{1.898106in}{0.860274in}}%
\pgfpathlineto{\pgfqpoint{1.901467in}{0.856756in}}%
\pgfpathlineto{\pgfqpoint{1.902588in}{0.856976in}}%
\pgfpathlineto{\pgfqpoint{1.903708in}{0.842355in}}%
\pgfpathlineto{\pgfqpoint{1.905949in}{0.840377in}}%
\pgfpathlineto{\pgfqpoint{1.909311in}{0.841586in}}%
\pgfpathlineto{\pgfqpoint{1.911552in}{0.836529in}}%
\pgfpathlineto{\pgfqpoint{1.912673in}{0.837738in}}%
\pgfpathlineto{\pgfqpoint{1.917155in}{0.836749in}}%
\pgfpathlineto{\pgfqpoint{1.919396in}{0.850600in}}%
\pgfpathlineto{\pgfqpoint{1.921637in}{0.848841in}}%
\pgfpathlineto{\pgfqpoint{1.924998in}{0.841146in}}%
\pgfpathlineto{\pgfqpoint{1.926119in}{0.840267in}}%
\pgfpathlineto{\pgfqpoint{1.927239in}{0.841806in}}%
\pgfpathlineto{\pgfqpoint{1.928360in}{0.841476in}}%
\pgfpathlineto{\pgfqpoint{1.929480in}{0.829494in}}%
\pgfpathlineto{\pgfqpoint{1.932842in}{0.831582in}}%
\pgfpathlineto{\pgfqpoint{1.933963in}{0.838288in}}%
\pgfpathlineto{\pgfqpoint{1.935083in}{0.866650in}}%
\pgfpathlineto{\pgfqpoint{1.937324in}{0.860934in}}%
\pgfpathlineto{\pgfqpoint{1.940686in}{0.856976in}}%
\pgfpathlineto{\pgfqpoint{1.941806in}{0.853458in}}%
\pgfpathlineto{\pgfqpoint{1.942927in}{0.859505in}}%
\pgfpathlineto{\pgfqpoint{1.944047in}{0.845763in}}%
\pgfpathlineto{\pgfqpoint{1.945168in}{0.842795in}}%
\pgfpathlineto{\pgfqpoint{1.948529in}{0.840816in}}%
\pgfpathlineto{\pgfqpoint{1.949650in}{0.844774in}}%
\pgfpathlineto{\pgfqpoint{1.950770in}{0.841696in}}%
\pgfpathlineto{\pgfqpoint{1.951891in}{0.851370in}}%
\pgfpathlineto{\pgfqpoint{1.953011in}{0.882150in}}%
\pgfpathlineto{\pgfqpoint{1.956373in}{0.877313in}}%
\pgfpathlineto{\pgfqpoint{1.957494in}{0.873905in}}%
\pgfpathlineto{\pgfqpoint{1.958614in}{0.874675in}}%
\pgfpathlineto{\pgfqpoint{1.959735in}{0.889076in}}%
\pgfpathlineto{\pgfqpoint{1.960855in}{0.884459in}}%
\pgfpathlineto{\pgfqpoint{1.965337in}{0.890835in}}%
\pgfpathlineto{\pgfqpoint{1.967578in}{0.882370in}}%
\pgfpathlineto{\pgfqpoint{1.968699in}{0.885008in}}%
\pgfpathlineto{\pgfqpoint{1.973181in}{0.873576in}}%
\pgfpathlineto{\pgfqpoint{1.974302in}{0.883689in}}%
\pgfpathlineto{\pgfqpoint{1.975422in}{0.884349in}}%
\pgfpathlineto{\pgfqpoint{1.976543in}{0.875005in}}%
\pgfpathlineto{\pgfqpoint{1.979904in}{0.879622in}}%
\pgfpathlineto{\pgfqpoint{1.982145in}{0.877863in}}%
\pgfpathlineto{\pgfqpoint{1.983266in}{0.870058in}}%
\pgfpathlineto{\pgfqpoint{1.984386in}{0.871707in}}%
\pgfpathlineto{\pgfqpoint{1.987748in}{0.864342in}}%
\pgfpathlineto{\pgfqpoint{1.988868in}{0.867200in}}%
\pgfpathlineto{\pgfqpoint{1.989989in}{0.885118in}}%
\pgfpathlineto{\pgfqpoint{1.991109in}{0.885228in}}%
\pgfpathlineto{\pgfqpoint{1.992230in}{0.881930in}}%
\pgfpathlineto{\pgfqpoint{1.995592in}{0.873905in}}%
\pgfpathlineto{\pgfqpoint{1.996712in}{0.878523in}}%
\pgfpathlineto{\pgfqpoint{1.997833in}{0.875774in}}%
\pgfpathlineto{\pgfqpoint{1.998953in}{0.883799in}}%
\pgfpathlineto{\pgfqpoint{2.000074in}{0.875115in}}%
\pgfpathlineto{\pgfqpoint{2.003435in}{0.879402in}}%
\pgfpathlineto{\pgfqpoint{2.004556in}{0.882920in}}%
\pgfpathlineto{\pgfqpoint{2.005676in}{0.876984in}}%
\pgfpathlineto{\pgfqpoint{2.006797in}{0.874345in}}%
\pgfpathlineto{\pgfqpoint{2.007917in}{0.875774in}}%
\pgfpathlineto{\pgfqpoint{2.011279in}{0.854118in}}%
\pgfpathlineto{\pgfqpoint{2.012399in}{0.858735in}}%
\pgfpathlineto{\pgfqpoint{2.014640in}{0.870388in}}%
\pgfpathlineto{\pgfqpoint{2.019123in}{0.869069in}}%
\pgfpathlineto{\pgfqpoint{2.020243in}{0.865331in}}%
\pgfpathlineto{\pgfqpoint{2.021364in}{0.854228in}}%
\pgfpathlineto{\pgfqpoint{2.022484in}{0.857636in}}%
\pgfpathlineto{\pgfqpoint{2.023605in}{0.872257in}}%
\pgfpathlineto{\pgfqpoint{2.026966in}{0.884019in}}%
\pgfpathlineto{\pgfqpoint{2.028087in}{0.890175in}}%
\pgfpathlineto{\pgfqpoint{2.029207in}{0.889296in}}%
\pgfpathlineto{\pgfqpoint{2.034810in}{0.917878in}}%
\pgfpathlineto{\pgfqpoint{2.035931in}{0.910622in}}%
\pgfpathlineto{\pgfqpoint{2.037051in}{0.910073in}}%
\pgfpathlineto{\pgfqpoint{2.038172in}{0.907764in}}%
\pgfpathlineto{\pgfqpoint{2.039292in}{0.940194in}}%
\pgfpathlineto{\pgfqpoint{2.042654in}{0.930190in}}%
\pgfpathlineto{\pgfqpoint{2.046015in}{0.957013in}}%
\pgfpathlineto{\pgfqpoint{2.047136in}{0.946020in}}%
\pgfpathlineto{\pgfqpoint{2.050497in}{0.950527in}}%
\pgfpathlineto{\pgfqpoint{2.052738in}{0.942172in}}%
\pgfpathlineto{\pgfqpoint{2.053859in}{0.927222in}}%
\pgfpathlineto{\pgfqpoint{2.054979in}{0.933927in}}%
\pgfpathlineto{\pgfqpoint{2.058341in}{0.935796in}}%
\pgfpathlineto{\pgfqpoint{2.059462in}{0.924913in}}%
\pgfpathlineto{\pgfqpoint{2.062823in}{0.935357in}}%
\pgfpathlineto{\pgfqpoint{2.067305in}{0.937995in}}%
\pgfpathlineto{\pgfqpoint{2.068426in}{0.937225in}}%
\pgfpathlineto{\pgfqpoint{2.069546in}{0.932498in}}%
\pgfpathlineto{\pgfqpoint{2.070667in}{0.903037in}}%
\pgfpathlineto{\pgfqpoint{2.075149in}{0.857966in}}%
\pgfpathlineto{\pgfqpoint{2.076269in}{0.897650in}}%
\pgfpathlineto{\pgfqpoint{2.077390in}{0.915899in}}%
\pgfpathlineto{\pgfqpoint{2.078511in}{0.916558in}}%
\pgfpathlineto{\pgfqpoint{2.081872in}{0.904136in}}%
\pgfpathlineto{\pgfqpoint{2.082993in}{0.879182in}}%
\pgfpathlineto{\pgfqpoint{2.085234in}{0.894572in}}%
\pgfpathlineto{\pgfqpoint{2.086354in}{0.881381in}}%
\pgfpathlineto{\pgfqpoint{2.090836in}{0.896001in}}%
\pgfpathlineto{\pgfqpoint{2.091957in}{0.885998in}}%
\pgfpathlineto{\pgfqpoint{2.094198in}{0.898420in}}%
\pgfpathlineto{\pgfqpoint{2.097559in}{0.890395in}}%
\pgfpathlineto{\pgfqpoint{2.099801in}{0.900838in}}%
\pgfpathlineto{\pgfqpoint{2.100921in}{0.900509in}}%
\pgfpathlineto{\pgfqpoint{2.102042in}{0.888086in}}%
\pgfpathlineto{\pgfqpoint{2.105403in}{0.898640in}}%
\pgfpathlineto{\pgfqpoint{2.106524in}{0.892923in}}%
\pgfpathlineto{\pgfqpoint{2.107644in}{0.900399in}}%
\pgfpathlineto{\pgfqpoint{2.108765in}{0.892813in}}%
\pgfpathlineto{\pgfqpoint{2.109885in}{0.897650in}}%
\pgfpathlineto{\pgfqpoint{2.113247in}{0.860384in}}%
\pgfpathlineto{\pgfqpoint{2.115488in}{0.886657in}}%
\pgfpathlineto{\pgfqpoint{2.116608in}{0.890175in}}%
\pgfpathlineto{\pgfqpoint{2.117729in}{0.897431in}}%
\pgfpathlineto{\pgfqpoint{2.121091in}{0.914140in}}%
\pgfpathlineto{\pgfqpoint{2.122211in}{0.912491in}}%
\pgfpathlineto{\pgfqpoint{2.124452in}{0.931949in}}%
\pgfpathlineto{\pgfqpoint{2.125573in}{0.932828in}}%
\pgfpathlineto{\pgfqpoint{2.128934in}{0.943601in}}%
\pgfpathlineto{\pgfqpoint{2.130055in}{0.943711in}}%
\pgfpathlineto{\pgfqpoint{2.131175in}{0.935137in}}%
\pgfpathlineto{\pgfqpoint{2.133416in}{0.954375in}}%
\pgfpathlineto{\pgfqpoint{2.136778in}{0.964928in}}%
\pgfpathlineto{\pgfqpoint{2.139019in}{0.948548in}}%
\pgfpathlineto{\pgfqpoint{2.141260in}{0.965697in}}%
\pgfpathlineto{\pgfqpoint{2.144622in}{0.977570in}}%
\pgfpathlineto{\pgfqpoint{2.145742in}{0.970534in}}%
\pgfpathlineto{\pgfqpoint{2.146863in}{0.984935in}}%
\pgfpathlineto{\pgfqpoint{2.147983in}{0.981088in}}%
\pgfpathlineto{\pgfqpoint{2.152465in}{0.946020in}}%
\pgfpathlineto{\pgfqpoint{2.153586in}{0.974602in}}%
\pgfpathlineto{\pgfqpoint{2.154706in}{0.979439in}}%
\pgfpathlineto{\pgfqpoint{2.155827in}{0.989113in}}%
\pgfpathlineto{\pgfqpoint{2.156947in}{0.983616in}}%
\pgfpathlineto{\pgfqpoint{2.160309in}{0.976141in}}%
\pgfpathlineto{\pgfqpoint{2.161430in}{0.992960in}}%
\pgfpathlineto{\pgfqpoint{2.162550in}{0.989772in}}%
\pgfpathlineto{\pgfqpoint{2.163671in}{0.980428in}}%
\pgfpathlineto{\pgfqpoint{2.164791in}{0.978339in}}%
\pgfpathlineto{\pgfqpoint{2.168153in}{0.986804in}}%
\pgfpathlineto{\pgfqpoint{2.169273in}{0.986035in}}%
\pgfpathlineto{\pgfqpoint{2.170394in}{1.003404in}}%
\pgfpathlineto{\pgfqpoint{2.171514in}{1.000106in}}%
\pgfpathlineto{\pgfqpoint{2.172635in}{1.000545in}}%
\pgfpathlineto{\pgfqpoint{2.175996in}{0.999666in}}%
\pgfpathlineto{\pgfqpoint{2.178237in}{0.993840in}}%
\pgfpathlineto{\pgfqpoint{2.180479in}{0.996808in}}%
\pgfpathlineto{\pgfqpoint{2.183840in}{0.987903in}}%
\pgfpathlineto{\pgfqpoint{2.184961in}{0.997577in}}%
\pgfpathlineto{\pgfqpoint{2.187202in}{0.980208in}}%
\pgfpathlineto{\pgfqpoint{2.188322in}{1.002854in}}%
\pgfpathlineto{\pgfqpoint{2.191684in}{0.993730in}}%
\pgfpathlineto{\pgfqpoint{2.192804in}{0.988563in}}%
\pgfpathlineto{\pgfqpoint{2.193925in}{0.975921in}}%
\pgfpathlineto{\pgfqpoint{2.195045in}{0.978449in}}%
\pgfpathlineto{\pgfqpoint{2.196166in}{0.957013in}}%
\pgfpathlineto{\pgfqpoint{2.199527in}{0.965038in}}%
\pgfpathlineto{\pgfqpoint{2.201769in}{0.997028in}}%
\pgfpathlineto{\pgfqpoint{2.202889in}{0.984495in}}%
\pgfpathlineto{\pgfqpoint{2.204010in}{0.959211in}}%
\pgfpathlineto{\pgfqpoint{2.208492in}{0.970534in}}%
\pgfpathlineto{\pgfqpoint{2.209612in}{0.983176in}}%
\pgfpathlineto{\pgfqpoint{2.210733in}{0.979988in}}%
\pgfpathlineto{\pgfqpoint{2.215215in}{0.982847in}}%
\pgfpathlineto{\pgfqpoint{2.216335in}{0.990102in}}%
\pgfpathlineto{\pgfqpoint{2.218576in}{0.972403in}}%
\pgfpathlineto{\pgfqpoint{2.223059in}{0.952616in}}%
\pgfpathlineto{\pgfqpoint{2.224179in}{0.958662in}}%
\pgfpathlineto{\pgfqpoint{2.227541in}{0.922385in}}%
\pgfpathlineto{\pgfqpoint{2.230902in}{0.933598in}}%
\pgfpathlineto{\pgfqpoint{2.232023in}{0.942502in}}%
\pgfpathlineto{\pgfqpoint{2.233143in}{0.924803in}}%
\pgfpathlineto{\pgfqpoint{2.234264in}{0.932279in}}%
\pgfpathlineto{\pgfqpoint{2.235384in}{0.911172in}}%
\pgfpathlineto{\pgfqpoint{2.239866in}{0.906555in}}%
\pgfpathlineto{\pgfqpoint{2.240987in}{0.898970in}}%
\pgfpathlineto{\pgfqpoint{2.243228in}{0.920516in}}%
\pgfpathlineto{\pgfqpoint{2.246590in}{0.910292in}}%
\pgfpathlineto{\pgfqpoint{2.247710in}{0.911722in}}%
\pgfpathlineto{\pgfqpoint{2.248831in}{0.901388in}}%
\pgfpathlineto{\pgfqpoint{2.249951in}{0.884459in}}%
\pgfpathlineto{\pgfqpoint{2.251072in}{0.939644in}}%
\pgfpathlineto{\pgfqpoint{2.254433in}{0.938435in}}%
\pgfpathlineto{\pgfqpoint{2.255554in}{0.927991in}}%
\pgfpathlineto{\pgfqpoint{2.256674in}{0.938435in}}%
\pgfpathlineto{\pgfqpoint{2.257795in}{0.930959in}}%
\pgfpathlineto{\pgfqpoint{2.258915in}{0.908094in}}%
\pgfpathlineto{\pgfqpoint{2.262277in}{0.867859in}}%
\pgfpathlineto{\pgfqpoint{2.263398in}{0.873796in}}%
\pgfpathlineto{\pgfqpoint{2.264518in}{0.894792in}}%
\pgfpathlineto{\pgfqpoint{2.265639in}{0.876764in}}%
\pgfpathlineto{\pgfqpoint{2.266759in}{0.897650in}}%
\pgfpathlineto{\pgfqpoint{2.271241in}{0.905016in}}%
\pgfpathlineto{\pgfqpoint{2.272362in}{0.916449in}}%
\pgfpathlineto{\pgfqpoint{2.273482in}{0.907984in}}%
\pgfpathlineto{\pgfqpoint{2.274603in}{0.911062in}}%
\pgfpathlineto{\pgfqpoint{2.277964in}{0.927552in}}%
\pgfpathlineto{\pgfqpoint{2.279085in}{0.917768in}}%
\pgfpathlineto{\pgfqpoint{2.280205in}{0.914470in}}%
\pgfpathlineto{\pgfqpoint{2.281326in}{0.929750in}}%
\pgfpathlineto{\pgfqpoint{2.282446in}{0.923924in}}%
\pgfpathlineto{\pgfqpoint{2.285808in}{0.920296in}}%
\pgfpathlineto{\pgfqpoint{2.286929in}{0.944481in}}%
\pgfpathlineto{\pgfqpoint{2.289170in}{0.936676in}}%
\pgfpathlineto{\pgfqpoint{2.290290in}{0.936566in}}%
\pgfpathlineto{\pgfqpoint{2.293652in}{0.915459in}}%
\pgfpathlineto{\pgfqpoint{2.294772in}{0.901278in}}%
\pgfpathlineto{\pgfqpoint{2.295893in}{0.901938in}}%
\pgfpathlineto{\pgfqpoint{2.297013in}{0.896991in}}%
\pgfpathlineto{\pgfqpoint{2.298134in}{0.912161in}}%
\pgfpathlineto{\pgfqpoint{2.301495in}{0.910622in}}%
\pgfpathlineto{\pgfqpoint{2.303736in}{0.920076in}}%
\pgfpathlineto{\pgfqpoint{2.305978in}{0.935247in}}%
\pgfpathlineto{\pgfqpoint{2.309339in}{0.935137in}}%
\pgfpathlineto{\pgfqpoint{2.310460in}{0.926452in}}%
\pgfpathlineto{\pgfqpoint{2.311580in}{0.936676in}}%
\pgfpathlineto{\pgfqpoint{2.312701in}{0.939094in}}%
\pgfpathlineto{\pgfqpoint{2.317183in}{0.938435in}}%
\pgfpathlineto{\pgfqpoint{2.319424in}{0.967456in}}%
\pgfpathlineto{\pgfqpoint{2.320544in}{0.964158in}}%
\pgfpathlineto{\pgfqpoint{2.321665in}{0.976141in}}%
\pgfpathlineto{\pgfqpoint{2.325027in}{0.978669in}}%
\pgfpathlineto{\pgfqpoint{2.326147in}{0.969435in}}%
\pgfpathlineto{\pgfqpoint{2.327268in}{0.982737in}}%
\pgfpathlineto{\pgfqpoint{2.328388in}{0.976031in}}%
\pgfpathlineto{\pgfqpoint{2.329509in}{0.980868in}}%
\pgfpathlineto{\pgfqpoint{2.332870in}{0.978559in}}%
\pgfpathlineto{\pgfqpoint{2.333991in}{0.986144in}}%
\pgfpathlineto{\pgfqpoint{2.336232in}{1.005492in}}%
\pgfpathlineto{\pgfqpoint{2.337352in}{1.002854in}}%
\pgfpathlineto{\pgfqpoint{2.340714in}{1.017585in}}%
\pgfpathlineto{\pgfqpoint{2.341834in}{1.009999in}}%
\pgfpathlineto{\pgfqpoint{2.342955in}{1.014287in}}%
\pgfpathlineto{\pgfqpoint{2.344075in}{1.010439in}}%
\pgfpathlineto{\pgfqpoint{2.345196in}{0.992410in}}%
\pgfpathlineto{\pgfqpoint{2.348558in}{0.982077in}}%
\pgfpathlineto{\pgfqpoint{2.350799in}{0.988673in}}%
\pgfpathlineto{\pgfqpoint{2.351919in}{0.977130in}}%
\pgfpathlineto{\pgfqpoint{2.353040in}{0.972403in}}%
\pgfpathlineto{\pgfqpoint{2.356401in}{0.985485in}}%
\pgfpathlineto{\pgfqpoint{2.357522in}{0.971853in}}%
\pgfpathlineto{\pgfqpoint{2.358642in}{0.970534in}}%
\pgfpathlineto{\pgfqpoint{2.360883in}{0.977570in}}%
\pgfpathlineto{\pgfqpoint{2.364245in}{0.982847in}}%
\pgfpathlineto{\pgfqpoint{2.365365in}{0.993400in}}%
\pgfpathlineto{\pgfqpoint{2.366486in}{0.974492in}}%
\pgfpathlineto{\pgfqpoint{2.367607in}{0.980868in}}%
\pgfpathlineto{\pgfqpoint{2.368727in}{0.969435in}}%
\pgfpathlineto{\pgfqpoint{2.372089in}{0.979878in}}%
\pgfpathlineto{\pgfqpoint{2.373209in}{0.968995in}}%
\pgfpathlineto{\pgfqpoint{2.374330in}{0.975921in}}%
\pgfpathlineto{\pgfqpoint{2.375450in}{0.969985in}}%
\pgfpathlineto{\pgfqpoint{2.376571in}{0.978449in}}%
\pgfpathlineto{\pgfqpoint{2.379932in}{0.973502in}}%
\pgfpathlineto{\pgfqpoint{2.381053in}{0.996808in}}%
\pgfpathlineto{\pgfqpoint{2.382173in}{0.993400in}}%
\pgfpathlineto{\pgfqpoint{2.383294in}{0.992740in}}%
\pgfpathlineto{\pgfqpoint{2.384414in}{0.999886in}}%
\pgfpathlineto{\pgfqpoint{2.388897in}{0.992081in}}%
\pgfpathlineto{\pgfqpoint{2.390017in}{0.995159in}}%
\pgfpathlineto{\pgfqpoint{2.391138in}{1.002964in}}%
\pgfpathlineto{\pgfqpoint{2.392258in}{1.002854in}}%
\pgfpathlineto{\pgfqpoint{2.395620in}{1.009340in}}%
\pgfpathlineto{\pgfqpoint{2.396740in}{1.009999in}}%
\pgfpathlineto{\pgfqpoint{2.397861in}{1.021212in}}%
\pgfpathlineto{\pgfqpoint{2.398981in}{1.017035in}}%
\pgfpathlineto{\pgfqpoint{2.400102in}{1.005492in}}%
\pgfpathlineto{\pgfqpoint{2.403463in}{0.986254in}}%
\pgfpathlineto{\pgfqpoint{2.404584in}{0.988123in}}%
\pgfpathlineto{\pgfqpoint{2.405704in}{0.983836in}}%
\pgfpathlineto{\pgfqpoint{2.406825in}{0.985815in}}%
\pgfpathlineto{\pgfqpoint{2.407946in}{0.971194in}}%
\pgfpathlineto{\pgfqpoint{2.411307in}{0.974932in}}%
\pgfpathlineto{\pgfqpoint{2.412428in}{0.974822in}}%
\pgfpathlineto{\pgfqpoint{2.413548in}{0.966247in}}%
\pgfpathlineto{\pgfqpoint{2.414669in}{0.984495in}}%
\pgfpathlineto{\pgfqpoint{2.415789in}{0.950307in}}%
\pgfpathlineto{\pgfqpoint{2.419151in}{0.931949in}}%
\pgfpathlineto{\pgfqpoint{2.421392in}{0.968446in}}%
\pgfpathlineto{\pgfqpoint{2.422512in}{0.940853in}}%
\pgfpathlineto{\pgfqpoint{2.423633in}{0.944151in}}%
\pgfpathlineto{\pgfqpoint{2.428115in}{0.946240in}}%
\pgfpathlineto{\pgfqpoint{2.429236in}{0.939644in}}%
\pgfpathlineto{\pgfqpoint{2.430356in}{0.944481in}}%
\pgfpathlineto{\pgfqpoint{2.431477in}{0.965038in}}%
\pgfpathlineto{\pgfqpoint{2.434838in}{0.966137in}}%
\pgfpathlineto{\pgfqpoint{2.435959in}{0.976361in}}%
\pgfpathlineto{\pgfqpoint{2.437079in}{0.976251in}}%
\pgfpathlineto{\pgfqpoint{2.438200in}{0.983506in}}%
\pgfpathlineto{\pgfqpoint{2.439320in}{0.985265in}}%
\pgfpathlineto{\pgfqpoint{2.442682in}{0.985375in}}%
\pgfpathlineto{\pgfqpoint{2.443802in}{0.989772in}}%
\pgfpathlineto{\pgfqpoint{2.444923in}{0.996588in}}%
\pgfpathlineto{\pgfqpoint{2.446043in}{0.990542in}}%
\pgfpathlineto{\pgfqpoint{2.447164in}{1.002524in}}%
\pgfpathlineto{\pgfqpoint{2.451646in}{0.987244in}}%
\pgfpathlineto{\pgfqpoint{2.452767in}{0.987683in}}%
\pgfpathlineto{\pgfqpoint{2.453887in}{0.994829in}}%
\pgfpathlineto{\pgfqpoint{2.455008in}{0.982517in}}%
\pgfpathlineto{\pgfqpoint{2.459490in}{0.985045in}}%
\pgfpathlineto{\pgfqpoint{2.460610in}{0.989662in}}%
\pgfpathlineto{\pgfqpoint{2.462851in}{1.005052in}}%
\pgfpathlineto{\pgfqpoint{2.466213in}{1.002964in}}%
\pgfpathlineto{\pgfqpoint{2.467333in}{1.004173in}}%
\pgfpathlineto{\pgfqpoint{2.468454in}{1.000106in}}%
\pgfpathlineto{\pgfqpoint{2.469575in}{1.004833in}}%
\pgfpathlineto{\pgfqpoint{2.470695in}{1.003733in}}%
\pgfpathlineto{\pgfqpoint{2.474057in}{1.013297in}}%
\pgfpathlineto{\pgfqpoint{2.475177in}{1.012198in}}%
\pgfpathlineto{\pgfqpoint{2.476298in}{1.014067in}}%
\pgfpathlineto{\pgfqpoint{2.477418in}{1.007911in}}%
\pgfpathlineto{\pgfqpoint{2.481900in}{1.016375in}}%
\pgfpathlineto{\pgfqpoint{2.483021in}{1.013627in}}%
\pgfpathlineto{\pgfqpoint{2.484141in}{1.008131in}}%
\pgfpathlineto{\pgfqpoint{2.485262in}{1.008460in}}%
\pgfpathlineto{\pgfqpoint{2.486382in}{1.011099in}}%
\pgfpathlineto{\pgfqpoint{2.489744in}{1.014397in}}%
\pgfpathlineto{\pgfqpoint{2.490865in}{1.017585in}}%
\pgfpathlineto{\pgfqpoint{2.491985in}{1.014726in}}%
\pgfpathlineto{\pgfqpoint{2.493106in}{1.019014in}}%
\pgfpathlineto{\pgfqpoint{2.494226in}{1.026489in}}%
\pgfpathlineto{\pgfqpoint{2.498708in}{1.032205in}}%
\pgfpathlineto{\pgfqpoint{2.499829in}{1.039900in}}%
\pgfpathlineto{\pgfqpoint{2.500949in}{1.036932in}}%
\pgfpathlineto{\pgfqpoint{2.502070in}{1.019343in}}%
\pgfpathlineto{\pgfqpoint{2.505431in}{1.036932in}}%
\pgfpathlineto{\pgfqpoint{2.506552in}{1.025390in}}%
\pgfpathlineto{\pgfqpoint{2.507672in}{1.020992in}}%
\pgfpathlineto{\pgfqpoint{2.508793in}{1.026599in}}%
\pgfpathlineto{\pgfqpoint{2.509913in}{1.027258in}}%
\pgfpathlineto{\pgfqpoint{2.513275in}{1.031985in}}%
\pgfpathlineto{\pgfqpoint{2.514396in}{1.031656in}}%
\pgfpathlineto{\pgfqpoint{2.515516in}{1.039681in}}%
\pgfpathlineto{\pgfqpoint{2.516637in}{1.041220in}}%
\pgfpathlineto{\pgfqpoint{2.517757in}{1.032315in}}%
\pgfpathlineto{\pgfqpoint{2.521119in}{1.024510in}}%
\pgfpathlineto{\pgfqpoint{2.522239in}{1.028248in}}%
\pgfpathlineto{\pgfqpoint{2.523360in}{1.036932in}}%
\pgfpathlineto{\pgfqpoint{2.524480in}{1.025719in}}%
\pgfpathlineto{\pgfqpoint{2.525601in}{1.034074in}}%
\pgfpathlineto{\pgfqpoint{2.528962in}{1.035833in}}%
\pgfpathlineto{\pgfqpoint{2.530083in}{1.034404in}}%
\pgfpathlineto{\pgfqpoint{2.531204in}{1.041110in}}%
\pgfpathlineto{\pgfqpoint{2.532324in}{1.041220in}}%
\pgfpathlineto{\pgfqpoint{2.533445in}{1.036053in}}%
\pgfpathlineto{\pgfqpoint{2.536806in}{1.038581in}}%
\pgfpathlineto{\pgfqpoint{2.537927in}{1.026929in}}%
\pgfpathlineto{\pgfqpoint{2.539047in}{1.029237in}}%
\pgfpathlineto{\pgfqpoint{2.540168in}{1.025170in}}%
\pgfpathlineto{\pgfqpoint{2.541288in}{1.031436in}}%
\pgfpathlineto{\pgfqpoint{2.544650in}{1.028138in}}%
\pgfpathlineto{\pgfqpoint{2.545770in}{1.021982in}}%
\pgfpathlineto{\pgfqpoint{2.546891in}{1.035283in}}%
\pgfpathlineto{\pgfqpoint{2.548011in}{1.031876in}}%
\pgfpathlineto{\pgfqpoint{2.549132in}{1.030336in}}%
\pgfpathlineto{\pgfqpoint{2.552494in}{1.039131in}}%
\pgfpathlineto{\pgfqpoint{2.553614in}{1.026819in}}%
\pgfpathlineto{\pgfqpoint{2.554735in}{1.023851in}}%
\pgfpathlineto{\pgfqpoint{2.555855in}{1.025719in}}%
\pgfpathlineto{\pgfqpoint{2.556976in}{1.028907in}}%
\pgfpathlineto{\pgfqpoint{2.560337in}{1.031985in}}%
\pgfpathlineto{\pgfqpoint{2.562578in}{1.011209in}}%
\pgfpathlineto{\pgfqpoint{2.563699in}{1.011978in}}%
\pgfpathlineto{\pgfqpoint{2.564819in}{1.008900in}}%
\pgfpathlineto{\pgfqpoint{2.569301in}{1.036053in}}%
\pgfpathlineto{\pgfqpoint{2.570422in}{1.039900in}}%
\pgfpathlineto{\pgfqpoint{2.571542in}{1.025170in}}%
\pgfpathlineto{\pgfqpoint{2.572663in}{1.025280in}}%
\pgfpathlineto{\pgfqpoint{2.576025in}{0.987574in}}%
\pgfpathlineto{\pgfqpoint{2.577145in}{0.989552in}}%
\pgfpathlineto{\pgfqpoint{2.579386in}{1.018354in}}%
\pgfpathlineto{\pgfqpoint{2.580507in}{1.015716in}}%
\pgfpathlineto{\pgfqpoint{2.583868in}{1.025060in}}%
\pgfpathlineto{\pgfqpoint{2.584989in}{1.006042in}}%
\pgfpathlineto{\pgfqpoint{2.586109in}{1.002194in}}%
\pgfpathlineto{\pgfqpoint{2.588350in}{1.008240in}}%
\pgfpathlineto{\pgfqpoint{2.591712in}{0.996918in}}%
\pgfpathlineto{\pgfqpoint{2.592832in}{0.997687in}}%
\pgfpathlineto{\pgfqpoint{2.595074in}{0.957453in}}%
\pgfpathlineto{\pgfqpoint{2.596194in}{0.960641in}}%
\pgfpathlineto{\pgfqpoint{2.599556in}{0.977790in}}%
\pgfpathlineto{\pgfqpoint{2.600676in}{0.975591in}}%
\pgfpathlineto{\pgfqpoint{2.601797in}{0.999336in}}%
\pgfpathlineto{\pgfqpoint{2.602917in}{0.999006in}}%
\pgfpathlineto{\pgfqpoint{2.604038in}{0.997577in}}%
\pgfpathlineto{\pgfqpoint{2.607399in}{0.990652in}}%
\pgfpathlineto{\pgfqpoint{2.608520in}{0.998127in}}%
\pgfpathlineto{\pgfqpoint{2.609640in}{0.997467in}}%
\pgfpathlineto{\pgfqpoint{2.610761in}{1.001425in}}%
\pgfpathlineto{\pgfqpoint{2.611881in}{0.989003in}}%
\pgfpathlineto{\pgfqpoint{2.615243in}{0.986254in}}%
\pgfpathlineto{\pgfqpoint{2.616364in}{0.989113in}}%
\pgfpathlineto{\pgfqpoint{2.618605in}{0.984166in}}%
\pgfpathlineto{\pgfqpoint{2.619725in}{0.986804in}}%
\pgfpathlineto{\pgfqpoint{2.624207in}{0.989003in}}%
\pgfpathlineto{\pgfqpoint{2.625328in}{0.988453in}}%
\pgfpathlineto{\pgfqpoint{2.626448in}{0.988783in}}%
\pgfpathlineto{\pgfqpoint{2.627569in}{0.985485in}}%
\pgfpathlineto{\pgfqpoint{2.632051in}{1.001425in}}%
\pgfpathlineto{\pgfqpoint{2.634292in}{1.018574in}}%
\pgfpathlineto{\pgfqpoint{2.635413in}{1.030666in}}%
\pgfpathlineto{\pgfqpoint{2.638774in}{1.025719in}}%
\pgfpathlineto{\pgfqpoint{2.639895in}{1.020992in}}%
\pgfpathlineto{\pgfqpoint{2.641015in}{1.026269in}}%
\pgfpathlineto{\pgfqpoint{2.642136in}{1.021652in}}%
\pgfpathlineto{\pgfqpoint{2.643256in}{1.019453in}}%
\pgfpathlineto{\pgfqpoint{2.647738in}{1.020443in}}%
\pgfpathlineto{\pgfqpoint{2.648859in}{1.023961in}}%
\pgfpathlineto{\pgfqpoint{2.651100in}{1.026709in}}%
\pgfpathlineto{\pgfqpoint{2.654461in}{1.030007in}}%
\pgfpathlineto{\pgfqpoint{2.656703in}{1.048915in}}%
\pgfpathlineto{\pgfqpoint{2.657823in}{1.041769in}}%
\pgfpathlineto{\pgfqpoint{2.658944in}{1.047486in}}%
\pgfpathlineto{\pgfqpoint{2.662305in}{1.046716in}}%
\pgfpathlineto{\pgfqpoint{2.663426in}{1.036053in}}%
\pgfpathlineto{\pgfqpoint{2.665667in}{1.031656in}}%
\pgfpathlineto{\pgfqpoint{2.666787in}{1.072440in}}%
\pgfpathlineto{\pgfqpoint{2.671269in}{1.069142in}}%
\pgfpathlineto{\pgfqpoint{2.672390in}{1.061777in}}%
\pgfpathlineto{\pgfqpoint{2.674631in}{1.070461in}}%
\pgfpathlineto{\pgfqpoint{2.677993in}{1.076287in}}%
\pgfpathlineto{\pgfqpoint{2.679113in}{1.080795in}}%
\pgfpathlineto{\pgfqpoint{2.680234in}{1.090029in}}%
\pgfpathlineto{\pgfqpoint{2.681354in}{1.088600in}}%
\pgfpathlineto{\pgfqpoint{2.682475in}{1.089149in}}%
\pgfpathlineto{\pgfqpoint{2.686957in}{1.094096in}}%
\pgfpathlineto{\pgfqpoint{2.688077in}{1.092777in}}%
\pgfpathlineto{\pgfqpoint{2.690318in}{1.099593in}}%
\pgfpathlineto{\pgfqpoint{2.694800in}{1.094316in}}%
\pgfpathlineto{\pgfqpoint{2.695921in}{1.105639in}}%
\pgfpathlineto{\pgfqpoint{2.697042in}{1.100692in}}%
\pgfpathlineto{\pgfqpoint{2.698162in}{1.103550in}}%
\pgfpathlineto{\pgfqpoint{2.701524in}{1.105089in}}%
\pgfpathlineto{\pgfqpoint{2.702644in}{1.106408in}}%
\pgfpathlineto{\pgfqpoint{2.703765in}{1.105309in}}%
\pgfpathlineto{\pgfqpoint{2.704885in}{1.106958in}}%
\pgfpathlineto{\pgfqpoint{2.706006in}{1.113664in}}%
\pgfpathlineto{\pgfqpoint{2.709367in}{1.117731in}}%
\pgfpathlineto{\pgfqpoint{2.710488in}{1.111685in}}%
\pgfpathlineto{\pgfqpoint{2.711608in}{1.115752in}}%
\pgfpathlineto{\pgfqpoint{2.713849in}{1.119160in}}%
\pgfpathlineto{\pgfqpoint{2.717211in}{1.107728in}}%
\pgfpathlineto{\pgfqpoint{2.718332in}{1.095635in}}%
\pgfpathlineto{\pgfqpoint{2.720573in}{1.104210in}}%
\pgfpathlineto{\pgfqpoint{2.721693in}{1.107837in}}%
\pgfpathlineto{\pgfqpoint{2.725055in}{1.105199in}}%
\pgfpathlineto{\pgfqpoint{2.726175in}{1.107068in}}%
\pgfpathlineto{\pgfqpoint{2.727296in}{1.107728in}}%
\pgfpathlineto{\pgfqpoint{2.729537in}{1.104320in}}%
\pgfpathlineto{\pgfqpoint{2.732898in}{1.109706in}}%
\pgfpathlineto{\pgfqpoint{2.734019in}{1.103440in}}%
\pgfpathlineto{\pgfqpoint{2.735139in}{1.106079in}}%
\pgfpathlineto{\pgfqpoint{2.736260in}{1.106738in}}%
\pgfpathlineto{\pgfqpoint{2.737381in}{1.103001in}}%
\pgfpathlineto{\pgfqpoint{2.741863in}{1.104100in}}%
\pgfpathlineto{\pgfqpoint{2.742983in}{1.102341in}}%
\pgfpathlineto{\pgfqpoint{2.744104in}{1.104320in}}%
\pgfpathlineto{\pgfqpoint{2.748586in}{1.114543in}}%
\pgfpathlineto{\pgfqpoint{2.749706in}{1.113664in}}%
\pgfpathlineto{\pgfqpoint{2.750827in}{1.113884in}}%
\pgfpathlineto{\pgfqpoint{2.751947in}{1.129054in}}%
\pgfpathlineto{\pgfqpoint{2.753068in}{1.129054in}}%
\pgfpathlineto{\pgfqpoint{2.757550in}{1.139388in}}%
\pgfpathlineto{\pgfqpoint{2.760912in}{1.129824in}}%
\pgfpathlineto{\pgfqpoint{2.764273in}{1.130263in}}%
\pgfpathlineto{\pgfqpoint{2.765394in}{1.144005in}}%
\pgfpathlineto{\pgfqpoint{2.766514in}{1.143015in}}%
\pgfpathlineto{\pgfqpoint{2.767635in}{1.144884in}}%
\pgfpathlineto{\pgfqpoint{2.768755in}{1.139168in}}%
\pgfpathlineto{\pgfqpoint{2.772117in}{1.137299in}}%
\pgfpathlineto{\pgfqpoint{2.773237in}{1.138178in}}%
\pgfpathlineto{\pgfqpoint{2.774358in}{1.140927in}}%
\pgfpathlineto{\pgfqpoint{2.775478in}{1.139278in}}%
\pgfpathlineto{\pgfqpoint{2.776599in}{1.146093in}}%
\pgfpathlineto{\pgfqpoint{2.779961in}{1.151480in}}%
\pgfpathlineto{\pgfqpoint{2.781081in}{1.150600in}}%
\pgfpathlineto{\pgfqpoint{2.782202in}{1.137409in}}%
\pgfpathlineto{\pgfqpoint{2.783322in}{1.136859in}}%
\pgfpathlineto{\pgfqpoint{2.784443in}{1.145214in}}%
\pgfpathlineto{\pgfqpoint{2.787804in}{1.154118in}}%
\pgfpathlineto{\pgfqpoint{2.788925in}{1.160164in}}%
\pgfpathlineto{\pgfqpoint{2.790045in}{1.170388in}}%
\pgfpathlineto{\pgfqpoint{2.791166in}{1.172916in}}%
\pgfpathlineto{\pgfqpoint{2.792286in}{1.168849in}}%
\pgfpathlineto{\pgfqpoint{2.796768in}{1.169399in}}%
\pgfpathlineto{\pgfqpoint{2.797889in}{1.175005in}}%
\pgfpathlineto{\pgfqpoint{2.799009in}{1.176764in}}%
\pgfpathlineto{\pgfqpoint{2.800130in}{1.184899in}}%
\pgfpathlineto{\pgfqpoint{2.803492in}{1.189296in}}%
\pgfpathlineto{\pgfqpoint{2.804612in}{1.181051in}}%
\pgfpathlineto{\pgfqpoint{2.805733in}{1.184239in}}%
\pgfpathlineto{\pgfqpoint{2.806853in}{1.184239in}}%
\pgfpathlineto{\pgfqpoint{2.807974in}{1.167750in}}%
\pgfpathlineto{\pgfqpoint{2.811335in}{1.156207in}}%
\pgfpathlineto{\pgfqpoint{2.812456in}{1.173356in}}%
\pgfpathlineto{\pgfqpoint{2.813576in}{1.175994in}}%
\pgfpathlineto{\pgfqpoint{2.814697in}{1.163462in}}%
\pgfpathlineto{\pgfqpoint{2.815817in}{1.163462in}}%
\pgfpathlineto{\pgfqpoint{2.819179in}{1.170168in}}%
\pgfpathlineto{\pgfqpoint{2.820300in}{1.165771in}}%
\pgfpathlineto{\pgfqpoint{2.821420in}{1.167530in}}%
\pgfpathlineto{\pgfqpoint{2.822541in}{1.161154in}}%
\pgfpathlineto{\pgfqpoint{2.823661in}{1.178743in}}%
\pgfpathlineto{\pgfqpoint{2.827023in}{1.174895in}}%
\pgfpathlineto{\pgfqpoint{2.828143in}{1.171377in}}%
\pgfpathlineto{\pgfqpoint{2.829264in}{1.185778in}}%
\pgfpathlineto{\pgfqpoint{2.830384in}{1.166211in}}%
\pgfpathlineto{\pgfqpoint{2.831505in}{1.159285in}}%
\pgfpathlineto{\pgfqpoint{2.834866in}{1.154558in}}%
\pgfpathlineto{\pgfqpoint{2.837107in}{1.161703in}}%
\pgfpathlineto{\pgfqpoint{2.838228in}{1.153569in}}%
\pgfpathlineto{\pgfqpoint{2.839348in}{1.160824in}}%
\pgfpathlineto{\pgfqpoint{2.843831in}{1.176874in}}%
\pgfpathlineto{\pgfqpoint{2.844951in}{1.185229in}}%
\pgfpathlineto{\pgfqpoint{2.846072in}{1.182700in}}%
\pgfpathlineto{\pgfqpoint{2.847192in}{1.193363in}}%
\pgfpathlineto{\pgfqpoint{2.850554in}{1.192264in}}%
\pgfpathlineto{\pgfqpoint{2.852795in}{1.207654in}}%
\pgfpathlineto{\pgfqpoint{2.853915in}{1.206115in}}%
\pgfpathlineto{\pgfqpoint{2.855036in}{1.222275in}}%
\pgfpathlineto{\pgfqpoint{2.858397in}{1.230630in}}%
\pgfpathlineto{\pgfqpoint{2.859518in}{1.226562in}}%
\pgfpathlineto{\pgfqpoint{2.860638in}{1.235797in}}%
\pgfpathlineto{\pgfqpoint{2.861759in}{1.221945in}}%
\pgfpathlineto{\pgfqpoint{2.862880in}{1.217438in}}%
\pgfpathlineto{\pgfqpoint{2.866241in}{1.221835in}}%
\pgfpathlineto{\pgfqpoint{2.867362in}{1.236016in}}%
\pgfpathlineto{\pgfqpoint{2.868482in}{1.240524in}}%
\pgfpathlineto{\pgfqpoint{2.869603in}{1.233048in}}%
\pgfpathlineto{\pgfqpoint{2.870723in}{1.236236in}}%
\pgfpathlineto{\pgfqpoint{2.874085in}{1.242722in}}%
\pgfpathlineto{\pgfqpoint{2.876326in}{1.237116in}}%
\pgfpathlineto{\pgfqpoint{2.877446in}{1.221066in}}%
\pgfpathlineto{\pgfqpoint{2.883049in}{1.255144in}}%
\pgfpathlineto{\pgfqpoint{2.884170in}{1.264378in}}%
\pgfpathlineto{\pgfqpoint{2.885290in}{1.251846in}}%
\pgfpathlineto{\pgfqpoint{2.886411in}{1.255584in}}%
\pgfpathlineto{\pgfqpoint{2.889772in}{1.263279in}}%
\pgfpathlineto{\pgfqpoint{2.890893in}{1.272513in}}%
\pgfpathlineto{\pgfqpoint{2.892013in}{1.262839in}}%
\pgfpathlineto{\pgfqpoint{2.893134in}{1.263279in}}%
\pgfpathlineto{\pgfqpoint{2.897616in}{1.269435in}}%
\pgfpathlineto{\pgfqpoint{2.899857in}{1.268886in}}%
\pgfpathlineto{\pgfqpoint{2.900977in}{1.266577in}}%
\pgfpathlineto{\pgfqpoint{2.902098in}{1.270644in}}%
\pgfpathlineto{\pgfqpoint{2.906580in}{1.261081in}}%
\pgfpathlineto{\pgfqpoint{2.907701in}{1.262839in}}%
\pgfpathlineto{\pgfqpoint{2.908821in}{1.277900in}}%
\pgfpathlineto{\pgfqpoint{2.909942in}{1.276471in}}%
\pgfpathlineto{\pgfqpoint{2.913303in}{1.295049in}}%
\pgfpathlineto{\pgfqpoint{2.914424in}{1.295709in}}%
\pgfpathlineto{\pgfqpoint{2.915544in}{1.291641in}}%
\pgfpathlineto{\pgfqpoint{2.916665in}{1.294389in}}%
\pgfpathlineto{\pgfqpoint{2.917785in}{1.285925in}}%
\pgfpathlineto{\pgfqpoint{2.921147in}{1.280758in}}%
\pgfpathlineto{\pgfqpoint{2.922267in}{1.287024in}}%
\pgfpathlineto{\pgfqpoint{2.923388in}{1.282297in}}%
\pgfpathlineto{\pgfqpoint{2.925629in}{1.288673in}}%
\pgfpathlineto{\pgfqpoint{2.928991in}{1.261190in}}%
\pgfpathlineto{\pgfqpoint{2.930111in}{1.260311in}}%
\pgfpathlineto{\pgfqpoint{2.932352in}{1.278120in}}%
\pgfpathlineto{\pgfqpoint{2.933473in}{1.285265in}}%
\pgfpathlineto{\pgfqpoint{2.936834in}{1.287354in}}%
\pgfpathlineto{\pgfqpoint{2.937955in}{1.289003in}}%
\pgfpathlineto{\pgfqpoint{2.939075in}{1.286035in}}%
\pgfpathlineto{\pgfqpoint{2.941316in}{1.301425in}}%
\pgfpathlineto{\pgfqpoint{2.944678in}{1.304393in}}%
\pgfpathlineto{\pgfqpoint{2.945799in}{1.307691in}}%
\pgfpathlineto{\pgfqpoint{2.946919in}{1.320003in}}%
\pgfpathlineto{\pgfqpoint{2.948040in}{1.316376in}}%
\pgfpathlineto{\pgfqpoint{2.949160in}{1.322312in}}%
\pgfpathlineto{\pgfqpoint{2.952522in}{1.318464in}}%
\pgfpathlineto{\pgfqpoint{2.953642in}{1.310219in}}%
\pgfpathlineto{\pgfqpoint{2.954763in}{1.312968in}}%
\pgfpathlineto{\pgfqpoint{2.955883in}{1.304503in}}%
\pgfpathlineto{\pgfqpoint{2.957004in}{1.310329in}}%
\pgfpathlineto{\pgfqpoint{2.960365in}{1.310110in}}%
\pgfpathlineto{\pgfqpoint{2.963727in}{1.334734in}}%
\pgfpathlineto{\pgfqpoint{2.964848in}{1.333745in}}%
\pgfpathlineto{\pgfqpoint{2.968209in}{1.337262in}}%
\pgfpathlineto{\pgfqpoint{2.969330in}{1.336603in}}%
\pgfpathlineto{\pgfqpoint{2.970450in}{1.348475in}}%
\pgfpathlineto{\pgfqpoint{2.971571in}{1.347486in}}%
\pgfpathlineto{\pgfqpoint{2.972691in}{1.351663in}}%
\pgfpathlineto{\pgfqpoint{2.977173in}{1.359578in}}%
\pgfpathlineto{\pgfqpoint{2.978294in}{1.363646in}}%
\pgfpathlineto{\pgfqpoint{2.980535in}{1.357270in}}%
\pgfpathlineto{\pgfqpoint{2.983896in}{1.352103in}}%
\pgfpathlineto{\pgfqpoint{2.985017in}{1.358369in}}%
\pgfpathlineto{\pgfqpoint{2.986138in}{1.339571in}}%
\pgfpathlineto{\pgfqpoint{2.987258in}{1.350124in}}%
\pgfpathlineto{\pgfqpoint{2.988379in}{1.337042in}}%
\pgfpathlineto{\pgfqpoint{2.991740in}{1.338472in}}%
\pgfpathlineto{\pgfqpoint{2.992861in}{1.354741in}}%
\pgfpathlineto{\pgfqpoint{2.993981in}{1.347926in}}%
\pgfpathlineto{\pgfqpoint{2.996222in}{1.360348in}}%
\pgfpathlineto{\pgfqpoint{2.999584in}{1.364855in}}%
\pgfpathlineto{\pgfqpoint{3.000704in}{1.375518in}}%
\pgfpathlineto{\pgfqpoint{3.001825in}{1.337482in}}%
\pgfpathlineto{\pgfqpoint{3.002945in}{1.367163in}}%
\pgfpathlineto{\pgfqpoint{3.004066in}{1.346936in}}%
\pgfpathlineto{\pgfqpoint{3.007428in}{1.311099in}}%
\pgfpathlineto{\pgfqpoint{3.010789in}{1.354192in}}%
\pgfpathlineto{\pgfqpoint{3.011910in}{1.367273in}}%
\pgfpathlineto{\pgfqpoint{3.015271in}{1.364745in}}%
\pgfpathlineto{\pgfqpoint{3.016392in}{1.376617in}}%
\pgfpathlineto{\pgfqpoint{3.017512in}{1.374968in}}%
\pgfpathlineto{\pgfqpoint{3.018633in}{1.370681in}}%
\pgfpathlineto{\pgfqpoint{3.019753in}{1.380465in}}%
\pgfpathlineto{\pgfqpoint{3.023115in}{1.377607in}}%
\pgfpathlineto{\pgfqpoint{3.024235in}{1.362217in}}%
\pgfpathlineto{\pgfqpoint{3.025356in}{1.361997in}}%
\pgfpathlineto{\pgfqpoint{3.027597in}{1.368263in}}%
\pgfpathlineto{\pgfqpoint{3.032079in}{1.371451in}}%
\pgfpathlineto{\pgfqpoint{3.033200in}{1.382664in}}%
\pgfpathlineto{\pgfqpoint{3.034320in}{1.386291in}}%
\pgfpathlineto{\pgfqpoint{3.035441in}{1.382664in}}%
\pgfpathlineto{\pgfqpoint{3.035441in}{1.382664in}}%
\pgfusepath{stroke}%
\end{pgfscope}%
\begin{pgfscope}%
\pgfpathrectangle{\pgfqpoint{0.462318in}{0.331635in}}{\pgfqpoint{2.695652in}{1.104878in}}%
\pgfusepath{clip}%
\pgfsetroundcap%
\pgfsetroundjoin%
\pgfsetlinewidth{1.505625pt}%
\definecolor{currentstroke}{rgb}{1.000000,0.647059,0.000000}%
\pgfsetstrokecolor{currentstroke}%
\pgfsetdash{}{0pt}%
\pgfpathmoveto{\pgfqpoint{0.584848in}{0.391091in}}%
\pgfpathlineto{\pgfqpoint{0.585968in}{0.388892in}}%
\pgfpathlineto{\pgfqpoint{0.587089in}{0.388782in}}%
\pgfpathlineto{\pgfqpoint{0.588209in}{0.388013in}}%
\pgfpathlineto{\pgfqpoint{0.592692in}{0.386547in}}%
\pgfpathlineto{\pgfqpoint{0.593812in}{0.385877in}}%
\pgfpathlineto{\pgfqpoint{0.603897in}{0.387083in}}%
\pgfpathlineto{\pgfqpoint{0.616223in}{0.386413in}}%
\pgfpathlineto{\pgfqpoint{0.618464in}{0.387128in}}%
\pgfpathlineto{\pgfqpoint{0.619584in}{0.387726in}}%
\pgfpathlineto{\pgfqpoint{0.624066in}{0.388804in}}%
\pgfpathlineto{\pgfqpoint{0.626307in}{0.390297in}}%
\pgfpathlineto{\pgfqpoint{0.627428in}{0.391264in}}%
\pgfpathlineto{\pgfqpoint{0.630789in}{0.392065in}}%
\pgfpathlineto{\pgfqpoint{0.635272in}{0.395585in}}%
\pgfpathlineto{\pgfqpoint{0.639754in}{0.396309in}}%
\pgfpathlineto{\pgfqpoint{0.643115in}{0.398759in}}%
\pgfpathlineto{\pgfqpoint{0.646477in}{0.399495in}}%
\pgfpathlineto{\pgfqpoint{0.650959in}{0.402155in}}%
\pgfpathlineto{\pgfqpoint{0.656562in}{0.403581in}}%
\pgfpathlineto{\pgfqpoint{0.658803in}{0.404636in}}%
\pgfpathlineto{\pgfqpoint{0.663285in}{0.405539in}}%
\pgfpathlineto{\pgfqpoint{0.666646in}{0.406752in}}%
\pgfpathlineto{\pgfqpoint{0.672249in}{0.407904in}}%
\pgfpathlineto{\pgfqpoint{0.682334in}{0.410580in}}%
\pgfpathlineto{\pgfqpoint{0.687936in}{0.411611in}}%
\pgfpathlineto{\pgfqpoint{0.689057in}{0.411991in}}%
\pgfpathlineto{\pgfqpoint{0.696901in}{0.413079in}}%
\pgfpathlineto{\pgfqpoint{0.698021in}{0.413484in}}%
\pgfpathlineto{\pgfqpoint{0.703624in}{0.414481in}}%
\pgfpathlineto{\pgfqpoint{0.705865in}{0.415072in}}%
\pgfpathlineto{\pgfqpoint{0.711467in}{0.415764in}}%
\pgfpathlineto{\pgfqpoint{0.713708in}{0.416421in}}%
\pgfpathlineto{\pgfqpoint{0.752927in}{0.419574in}}%
\pgfpathlineto{\pgfqpoint{0.768614in}{0.420217in}}%
\pgfpathlineto{\pgfqpoint{0.779820in}{0.421219in}}%
\pgfpathlineto{\pgfqpoint{0.792145in}{0.422867in}}%
\pgfpathlineto{\pgfqpoint{0.803351in}{0.423844in}}%
\pgfpathlineto{\pgfqpoint{0.807833in}{0.424687in}}%
\pgfpathlineto{\pgfqpoint{0.819038in}{0.425783in}}%
\pgfpathlineto{\pgfqpoint{0.831364in}{0.427788in}}%
\pgfpathlineto{\pgfqpoint{0.838087in}{0.428607in}}%
\pgfpathlineto{\pgfqpoint{0.847051in}{0.429662in}}%
\pgfpathlineto{\pgfqpoint{0.860498in}{0.430801in}}%
\pgfpathlineto{\pgfqpoint{0.878426in}{0.433374in}}%
\pgfpathlineto{\pgfqpoint{0.886270in}{0.434413in}}%
\pgfpathlineto{\pgfqpoint{0.892993in}{0.435405in}}%
\pgfpathlineto{\pgfqpoint{0.897475in}{0.435931in}}%
\pgfpathlineto{\pgfqpoint{0.909801in}{0.438199in}}%
\pgfpathlineto{\pgfqpoint{0.924368in}{0.439738in}}%
\pgfpathlineto{\pgfqpoint{0.929970in}{0.440514in}}%
\pgfpathlineto{\pgfqpoint{0.938934in}{0.441969in}}%
\pgfpathlineto{\pgfqpoint{0.949019in}{0.443598in}}%
\pgfpathlineto{\pgfqpoint{0.954622in}{0.444452in}}%
\pgfpathlineto{\pgfqpoint{0.960224in}{0.445340in}}%
\pgfpathlineto{\pgfqpoint{0.971430in}{0.447543in}}%
\pgfpathlineto{\pgfqpoint{0.980394in}{0.449203in}}%
\pgfpathlineto{\pgfqpoint{0.993840in}{0.450851in}}%
\pgfpathlineto{\pgfqpoint{0.996081in}{0.451484in}}%
\pgfpathlineto{\pgfqpoint{1.001684in}{0.452518in}}%
\pgfpathlineto{\pgfqpoint{1.003925in}{0.453217in}}%
\pgfpathlineto{\pgfqpoint{1.009528in}{0.454232in}}%
\pgfpathlineto{\pgfqpoint{1.011769in}{0.454871in}}%
\pgfpathlineto{\pgfqpoint{1.018492in}{0.455815in}}%
\pgfpathlineto{\pgfqpoint{1.019612in}{0.456131in}}%
\pgfpathlineto{\pgfqpoint{1.026336in}{0.457248in}}%
\pgfpathlineto{\pgfqpoint{1.027456in}{0.457541in}}%
\pgfpathlineto{\pgfqpoint{1.033059in}{0.458419in}}%
\pgfpathlineto{\pgfqpoint{1.039782in}{0.459490in}}%
\pgfpathlineto{\pgfqpoint{1.043143in}{0.460287in}}%
\pgfpathlineto{\pgfqpoint{1.049867in}{0.461084in}}%
\pgfpathlineto{\pgfqpoint{1.050987in}{0.461369in}}%
\pgfpathlineto{\pgfqpoint{1.056590in}{0.462170in}}%
\pgfpathlineto{\pgfqpoint{1.066675in}{0.464092in}}%
\pgfpathlineto{\pgfqpoint{1.073398in}{0.465165in}}%
\pgfpathlineto{\pgfqpoint{1.082362in}{0.466625in}}%
\pgfpathlineto{\pgfqpoint{1.087965in}{0.467538in}}%
\pgfpathlineto{\pgfqpoint{1.089085in}{0.467864in}}%
\pgfpathlineto{\pgfqpoint{1.095808in}{0.468753in}}%
\pgfpathlineto{\pgfqpoint{1.098049in}{0.469322in}}%
\pgfpathlineto{\pgfqpoint{1.103652in}{0.470168in}}%
\pgfpathlineto{\pgfqpoint{1.105893in}{0.470733in}}%
\pgfpathlineto{\pgfqpoint{1.112616in}{0.471710in}}%
\pgfpathlineto{\pgfqpoint{1.121580in}{0.473297in}}%
\pgfpathlineto{\pgfqpoint{1.127183in}{0.474102in}}%
\pgfpathlineto{\pgfqpoint{1.129424in}{0.474779in}}%
\pgfpathlineto{\pgfqpoint{1.135027in}{0.475822in}}%
\pgfpathlineto{\pgfqpoint{1.137268in}{0.476497in}}%
\pgfpathlineto{\pgfqpoint{1.142870in}{0.477542in}}%
\pgfpathlineto{\pgfqpoint{1.145111in}{0.478268in}}%
\pgfpathlineto{\pgfqpoint{1.150714in}{0.479317in}}%
\pgfpathlineto{\pgfqpoint{1.152955in}{0.479981in}}%
\pgfpathlineto{\pgfqpoint{1.159678in}{0.480972in}}%
\pgfpathlineto{\pgfqpoint{1.160799in}{0.481287in}}%
\pgfpathlineto{\pgfqpoint{1.166401in}{0.482245in}}%
\pgfpathlineto{\pgfqpoint{1.168642in}{0.482881in}}%
\pgfpathlineto{\pgfqpoint{1.174245in}{0.483837in}}%
\pgfpathlineto{\pgfqpoint{1.176486in}{0.484482in}}%
\pgfpathlineto{\pgfqpoint{1.182089in}{0.485466in}}%
\pgfpathlineto{\pgfqpoint{1.184330in}{0.486058in}}%
\pgfpathlineto{\pgfqpoint{1.189933in}{0.486964in}}%
\pgfpathlineto{\pgfqpoint{1.192174in}{0.487601in}}%
\pgfpathlineto{\pgfqpoint{1.197776in}{0.488591in}}%
\pgfpathlineto{\pgfqpoint{1.213464in}{0.491666in}}%
\pgfpathlineto{\pgfqpoint{1.215705in}{0.492346in}}%
\pgfpathlineto{\pgfqpoint{1.221307in}{0.493322in}}%
\pgfpathlineto{\pgfqpoint{1.223548in}{0.494029in}}%
\pgfpathlineto{\pgfqpoint{1.230271in}{0.495194in}}%
\pgfpathlineto{\pgfqpoint{1.235874in}{0.496032in}}%
\pgfpathlineto{\pgfqpoint{1.242597in}{0.497045in}}%
\pgfpathlineto{\pgfqpoint{1.250441in}{0.498176in}}%
\pgfpathlineto{\pgfqpoint{1.262767in}{0.500162in}}%
\pgfpathlineto{\pgfqpoint{1.273972in}{0.501257in}}%
\pgfpathlineto{\pgfqpoint{1.286298in}{0.503875in}}%
\pgfpathlineto{\pgfqpoint{1.291900in}{0.504782in}}%
\pgfpathlineto{\pgfqpoint{1.294142in}{0.505376in}}%
\pgfpathlineto{\pgfqpoint{1.299744in}{0.506234in}}%
\pgfpathlineto{\pgfqpoint{1.305347in}{0.507019in}}%
\pgfpathlineto{\pgfqpoint{1.341204in}{0.514038in}}%
\pgfpathlineto{\pgfqpoint{1.346806in}{0.514909in}}%
\pgfpathlineto{\pgfqpoint{1.349047in}{0.515517in}}%
\pgfpathlineto{\pgfqpoint{1.354650in}{0.516377in}}%
\pgfpathlineto{\pgfqpoint{1.356891in}{0.516980in}}%
\pgfpathlineto{\pgfqpoint{1.368096in}{0.518514in}}%
\pgfpathlineto{\pgfqpoint{1.372578in}{0.519684in}}%
\pgfpathlineto{\pgfqpoint{1.378181in}{0.520553in}}%
\pgfpathlineto{\pgfqpoint{1.388266in}{0.522867in}}%
\pgfpathlineto{\pgfqpoint{1.394989in}{0.523957in}}%
\pgfpathlineto{\pgfqpoint{1.396109in}{0.524321in}}%
\pgfpathlineto{\pgfqpoint{1.402833in}{0.525433in}}%
\pgfpathlineto{\pgfqpoint{1.403953in}{0.525798in}}%
\pgfpathlineto{\pgfqpoint{1.409556in}{0.526887in}}%
\pgfpathlineto{\pgfqpoint{1.411797in}{0.527609in}}%
\pgfpathlineto{\pgfqpoint{1.417400in}{0.528686in}}%
\pgfpathlineto{\pgfqpoint{1.419641in}{0.529448in}}%
\pgfpathlineto{\pgfqpoint{1.425243in}{0.530259in}}%
\pgfpathlineto{\pgfqpoint{1.427484in}{0.530988in}}%
\pgfpathlineto{\pgfqpoint{1.433087in}{0.531972in}}%
\pgfpathlineto{\pgfqpoint{1.435328in}{0.532623in}}%
\pgfpathlineto{\pgfqpoint{1.440931in}{0.533537in}}%
\pgfpathlineto{\pgfqpoint{1.443172in}{0.534197in}}%
\pgfpathlineto{\pgfqpoint{1.448774in}{0.535214in}}%
\pgfpathlineto{\pgfqpoint{1.451015in}{0.535921in}}%
\pgfpathlineto{\pgfqpoint{1.457738in}{0.536965in}}%
\pgfpathlineto{\pgfqpoint{1.458859in}{0.537305in}}%
\pgfpathlineto{\pgfqpoint{1.464462in}{0.538362in}}%
\pgfpathlineto{\pgfqpoint{1.466703in}{0.539057in}}%
\pgfpathlineto{\pgfqpoint{1.472305in}{0.540049in}}%
\pgfpathlineto{\pgfqpoint{1.474546in}{0.540717in}}%
\pgfpathlineto{\pgfqpoint{1.480149in}{0.541734in}}%
\pgfpathlineto{\pgfqpoint{1.482390in}{0.542354in}}%
\pgfpathlineto{\pgfqpoint{1.487993in}{0.543330in}}%
\pgfpathlineto{\pgfqpoint{1.490234in}{0.543955in}}%
\pgfpathlineto{\pgfqpoint{1.496957in}{0.545102in}}%
\pgfpathlineto{\pgfqpoint{1.505921in}{0.546675in}}%
\pgfpathlineto{\pgfqpoint{1.517126in}{0.547911in}}%
\pgfpathlineto{\pgfqpoint{1.520488in}{0.548587in}}%
\pgfpathlineto{\pgfqpoint{1.528332in}{0.549517in}}%
\pgfpathlineto{\pgfqpoint{1.537296in}{0.550712in}}%
\pgfpathlineto{\pgfqpoint{1.544019in}{0.551588in}}%
\pgfpathlineto{\pgfqpoint{1.552983in}{0.552959in}}%
\pgfpathlineto{\pgfqpoint{1.559706in}{0.553858in}}%
\pgfpathlineto{\pgfqpoint{1.560827in}{0.554092in}}%
\pgfpathlineto{\pgfqpoint{1.568671in}{0.555062in}}%
\pgfpathlineto{\pgfqpoint{1.575394in}{0.555979in}}%
\pgfpathlineto{\pgfqpoint{1.584358in}{0.557345in}}%
\pgfpathlineto{\pgfqpoint{1.591081in}{0.558208in}}%
\pgfpathlineto{\pgfqpoint{1.600045in}{0.559441in}}%
\pgfpathlineto{\pgfqpoint{1.611251in}{0.560565in}}%
\pgfpathlineto{\pgfqpoint{1.623577in}{0.562692in}}%
\pgfpathlineto{\pgfqpoint{1.630300in}{0.563659in}}%
\pgfpathlineto{\pgfqpoint{1.639264in}{0.564897in}}%
\pgfpathlineto{\pgfqpoint{1.650469in}{0.566050in}}%
\pgfpathlineto{\pgfqpoint{1.662795in}{0.567895in}}%
\pgfpathlineto{\pgfqpoint{1.675121in}{0.569143in}}%
\pgfpathlineto{\pgfqpoint{1.686326in}{0.570760in}}%
\pgfpathlineto{\pgfqpoint{1.697531in}{0.571966in}}%
\pgfpathlineto{\pgfqpoint{1.709857in}{0.573602in}}%
\pgfpathlineto{\pgfqpoint{1.723303in}{0.574865in}}%
\pgfpathlineto{\pgfqpoint{1.753558in}{0.579223in}}%
\pgfpathlineto{\pgfqpoint{1.756919in}{0.580158in}}%
\pgfpathlineto{\pgfqpoint{1.762522in}{0.581078in}}%
\pgfpathlineto{\pgfqpoint{1.764763in}{0.581709in}}%
\pgfpathlineto{\pgfqpoint{1.770366in}{0.582687in}}%
\pgfpathlineto{\pgfqpoint{1.784932in}{0.585395in}}%
\pgfpathlineto{\pgfqpoint{1.788294in}{0.586388in}}%
\pgfpathlineto{\pgfqpoint{1.793897in}{0.587329in}}%
\pgfpathlineto{\pgfqpoint{1.796138in}{0.587998in}}%
\pgfpathlineto{\pgfqpoint{1.801740in}{0.589025in}}%
\pgfpathlineto{\pgfqpoint{1.824151in}{0.592908in}}%
\pgfpathlineto{\pgfqpoint{1.827512in}{0.593779in}}%
\pgfpathlineto{\pgfqpoint{1.834236in}{0.594672in}}%
\pgfpathlineto{\pgfqpoint{1.835356in}{0.594972in}}%
\pgfpathlineto{\pgfqpoint{1.842079in}{0.596056in}}%
\pgfpathlineto{\pgfqpoint{1.851044in}{0.597902in}}%
\pgfpathlineto{\pgfqpoint{1.856646in}{0.598852in}}%
\pgfpathlineto{\pgfqpoint{1.858887in}{0.599523in}}%
\pgfpathlineto{\pgfqpoint{1.865610in}{0.600517in}}%
\pgfpathlineto{\pgfqpoint{1.866731in}{0.600858in}}%
\pgfpathlineto{\pgfqpoint{1.872334in}{0.601874in}}%
\pgfpathlineto{\pgfqpoint{1.874575in}{0.602545in}}%
\pgfpathlineto{\pgfqpoint{1.880177in}{0.603578in}}%
\pgfpathlineto{\pgfqpoint{1.882418in}{0.604235in}}%
\pgfpathlineto{\pgfqpoint{1.888021in}{0.605168in}}%
\pgfpathlineto{\pgfqpoint{1.890262in}{0.605788in}}%
\pgfpathlineto{\pgfqpoint{1.895865in}{0.606710in}}%
\pgfpathlineto{\pgfqpoint{1.898106in}{0.607330in}}%
\pgfpathlineto{\pgfqpoint{1.903708in}{0.608235in}}%
\pgfpathlineto{\pgfqpoint{1.905949in}{0.608806in}}%
\pgfpathlineto{\pgfqpoint{1.917155in}{0.610209in}}%
\pgfpathlineto{\pgfqpoint{1.927239in}{0.612202in}}%
\pgfpathlineto{\pgfqpoint{1.929480in}{0.612742in}}%
\pgfpathlineto{\pgfqpoint{1.935083in}{0.613584in}}%
\pgfpathlineto{\pgfqpoint{1.937324in}{0.614181in}}%
\pgfpathlineto{\pgfqpoint{1.942927in}{0.615053in}}%
\pgfpathlineto{\pgfqpoint{1.949650in}{0.616142in}}%
\pgfpathlineto{\pgfqpoint{1.960855in}{0.618557in}}%
\pgfpathlineto{\pgfqpoint{1.966458in}{0.619511in}}%
\pgfpathlineto{\pgfqpoint{1.968699in}{0.620131in}}%
\pgfpathlineto{\pgfqpoint{1.975422in}{0.621045in}}%
\pgfpathlineto{\pgfqpoint{1.976543in}{0.621342in}}%
\pgfpathlineto{\pgfqpoint{1.982145in}{0.622241in}}%
\pgfpathlineto{\pgfqpoint{1.984386in}{0.622818in}}%
\pgfpathlineto{\pgfqpoint{1.989989in}{0.623684in}}%
\pgfpathlineto{\pgfqpoint{1.992230in}{0.624284in}}%
\pgfpathlineto{\pgfqpoint{1.997833in}{0.625154in}}%
\pgfpathlineto{\pgfqpoint{2.000074in}{0.625738in}}%
\pgfpathlineto{\pgfqpoint{2.005676in}{0.626609in}}%
\pgfpathlineto{\pgfqpoint{2.007917in}{0.627177in}}%
\pgfpathlineto{\pgfqpoint{2.019123in}{0.628519in}}%
\pgfpathlineto{\pgfqpoint{2.031448in}{0.631047in}}%
\pgfpathlineto{\pgfqpoint{2.037051in}{0.631994in}}%
\pgfpathlineto{\pgfqpoint{2.047136in}{0.634382in}}%
\pgfpathlineto{\pgfqpoint{2.052738in}{0.635418in}}%
\pgfpathlineto{\pgfqpoint{2.054979in}{0.636070in}}%
\pgfpathlineto{\pgfqpoint{2.060582in}{0.637040in}}%
\pgfpathlineto{\pgfqpoint{2.062823in}{0.637692in}}%
\pgfpathlineto{\pgfqpoint{2.068426in}{0.638677in}}%
\pgfpathlineto{\pgfqpoint{2.070667in}{0.639287in}}%
\pgfpathlineto{\pgfqpoint{2.076269in}{0.640061in}}%
\pgfpathlineto{\pgfqpoint{2.078511in}{0.640662in}}%
\pgfpathlineto{\pgfqpoint{2.085234in}{0.641745in}}%
\pgfpathlineto{\pgfqpoint{2.086354in}{0.642004in}}%
\pgfpathlineto{\pgfqpoint{2.093077in}{0.642811in}}%
\pgfpathlineto{\pgfqpoint{2.100921in}{0.644175in}}%
\pgfpathlineto{\pgfqpoint{2.109885in}{0.645778in}}%
\pgfpathlineto{\pgfqpoint{2.116608in}{0.646763in}}%
\pgfpathlineto{\pgfqpoint{2.125573in}{0.648483in}}%
\pgfpathlineto{\pgfqpoint{2.131175in}{0.649405in}}%
\pgfpathlineto{\pgfqpoint{2.133416in}{0.650037in}}%
\pgfpathlineto{\pgfqpoint{2.139019in}{0.650999in}}%
\pgfpathlineto{\pgfqpoint{2.141260in}{0.651648in}}%
\pgfpathlineto{\pgfqpoint{2.146863in}{0.652665in}}%
\pgfpathlineto{\pgfqpoint{2.149104in}{0.653336in}}%
\pgfpathlineto{\pgfqpoint{2.154706in}{0.654308in}}%
\pgfpathlineto{\pgfqpoint{2.156947in}{0.654993in}}%
\pgfpathlineto{\pgfqpoint{2.162550in}{0.656016in}}%
\pgfpathlineto{\pgfqpoint{2.164791in}{0.656680in}}%
\pgfpathlineto{\pgfqpoint{2.170394in}{0.657710in}}%
\pgfpathlineto{\pgfqpoint{2.172635in}{0.658410in}}%
\pgfpathlineto{\pgfqpoint{2.178237in}{0.659443in}}%
\pgfpathlineto{\pgfqpoint{2.192804in}{0.662130in}}%
\pgfpathlineto{\pgfqpoint{2.196166in}{0.663061in}}%
\pgfpathlineto{\pgfqpoint{2.201769in}{0.664022in}}%
\pgfpathlineto{\pgfqpoint{2.204010in}{0.664639in}}%
\pgfpathlineto{\pgfqpoint{2.209612in}{0.665566in}}%
\pgfpathlineto{\pgfqpoint{2.210733in}{0.665880in}}%
\pgfpathlineto{\pgfqpoint{2.217456in}{0.666831in}}%
\pgfpathlineto{\pgfqpoint{2.218576in}{0.667135in}}%
\pgfpathlineto{\pgfqpoint{2.225300in}{0.667985in}}%
\pgfpathlineto{\pgfqpoint{2.235384in}{0.669780in}}%
\pgfpathlineto{\pgfqpoint{2.243228in}{0.670716in}}%
\pgfpathlineto{\pgfqpoint{2.249951in}{0.671619in}}%
\pgfpathlineto{\pgfqpoint{2.257795in}{0.672900in}}%
\pgfpathlineto{\pgfqpoint{2.264518in}{0.673725in}}%
\pgfpathlineto{\pgfqpoint{2.274603in}{0.675046in}}%
\pgfpathlineto{\pgfqpoint{2.281326in}{0.675995in}}%
\pgfpathlineto{\pgfqpoint{2.290290in}{0.677469in}}%
\pgfpathlineto{\pgfqpoint{2.301495in}{0.678771in}}%
\pgfpathlineto{\pgfqpoint{2.312701in}{0.680659in}}%
\pgfpathlineto{\pgfqpoint{2.319424in}{0.681424in}}%
\pgfpathlineto{\pgfqpoint{2.329509in}{0.683342in}}%
\pgfpathlineto{\pgfqpoint{2.335111in}{0.684191in}}%
\pgfpathlineto{\pgfqpoint{2.337352in}{0.684785in}}%
\pgfpathlineto{\pgfqpoint{2.342955in}{0.685699in}}%
\pgfpathlineto{\pgfqpoint{2.345196in}{0.686282in}}%
\pgfpathlineto{\pgfqpoint{2.351919in}{0.687376in}}%
\pgfpathlineto{\pgfqpoint{2.360883in}{0.688956in}}%
\pgfpathlineto{\pgfqpoint{2.367607in}{0.690028in}}%
\pgfpathlineto{\pgfqpoint{2.376571in}{0.691571in}}%
\pgfpathlineto{\pgfqpoint{2.382173in}{0.692375in}}%
\pgfpathlineto{\pgfqpoint{2.384414in}{0.692924in}}%
\pgfpathlineto{\pgfqpoint{2.391138in}{0.693744in}}%
\pgfpathlineto{\pgfqpoint{2.400102in}{0.695448in}}%
\pgfpathlineto{\pgfqpoint{2.406825in}{0.696485in}}%
\pgfpathlineto{\pgfqpoint{2.415789in}{0.697943in}}%
\pgfpathlineto{\pgfqpoint{2.422512in}{0.698828in}}%
\pgfpathlineto{\pgfqpoint{2.423633in}{0.699044in}}%
\pgfpathlineto{\pgfqpoint{2.431477in}{0.699924in}}%
\pgfpathlineto{\pgfqpoint{2.438200in}{0.700891in}}%
\pgfpathlineto{\pgfqpoint{2.447164in}{0.702414in}}%
\pgfpathlineto{\pgfqpoint{2.453887in}{0.703416in}}%
\pgfpathlineto{\pgfqpoint{2.462851in}{0.704907in}}%
\pgfpathlineto{\pgfqpoint{2.469575in}{0.705935in}}%
\pgfpathlineto{\pgfqpoint{2.478539in}{0.707501in}}%
\pgfpathlineto{\pgfqpoint{2.485262in}{0.708541in}}%
\pgfpathlineto{\pgfqpoint{2.494226in}{0.710116in}}%
\pgfpathlineto{\pgfqpoint{2.500949in}{0.710946in}}%
\pgfpathlineto{\pgfqpoint{2.509913in}{0.712542in}}%
\pgfpathlineto{\pgfqpoint{2.516637in}{0.713630in}}%
\pgfpathlineto{\pgfqpoint{2.525601in}{0.715220in}}%
\pgfpathlineto{\pgfqpoint{2.532324in}{0.716297in}}%
\pgfpathlineto{\pgfqpoint{2.541288in}{0.717866in}}%
\pgfpathlineto{\pgfqpoint{2.548011in}{0.718896in}}%
\pgfpathlineto{\pgfqpoint{2.556976in}{0.720428in}}%
\pgfpathlineto{\pgfqpoint{2.563699in}{0.721409in}}%
\pgfpathlineto{\pgfqpoint{2.572663in}{0.722910in}}%
\pgfpathlineto{\pgfqpoint{2.579386in}{0.723815in}}%
\pgfpathlineto{\pgfqpoint{2.586109in}{0.724750in}}%
\pgfpathlineto{\pgfqpoint{2.609640in}{0.727708in}}%
\pgfpathlineto{\pgfqpoint{2.619725in}{0.729167in}}%
\pgfpathlineto{\pgfqpoint{2.632051in}{0.730205in}}%
\pgfpathlineto{\pgfqpoint{2.643256in}{0.732044in}}%
\pgfpathlineto{\pgfqpoint{2.651100in}{0.732963in}}%
\pgfpathlineto{\pgfqpoint{2.657823in}{0.733929in}}%
\pgfpathlineto{\pgfqpoint{2.666787in}{0.735384in}}%
\pgfpathlineto{\pgfqpoint{2.673510in}{0.736416in}}%
\pgfpathlineto{\pgfqpoint{2.682475in}{0.738024in}}%
\pgfpathlineto{\pgfqpoint{2.689198in}{0.738851in}}%
\pgfpathlineto{\pgfqpoint{2.698162in}{0.740517in}}%
\pgfpathlineto{\pgfqpoint{2.703765in}{0.741358in}}%
\pgfpathlineto{\pgfqpoint{2.713849in}{0.743352in}}%
\pgfpathlineto{\pgfqpoint{2.720573in}{0.744444in}}%
\pgfpathlineto{\pgfqpoint{2.729537in}{0.746089in}}%
\pgfpathlineto{\pgfqpoint{2.736260in}{0.747177in}}%
\pgfpathlineto{\pgfqpoint{2.744104in}{0.748518in}}%
\pgfpathlineto{\pgfqpoint{2.750827in}{0.749342in}}%
\pgfpathlineto{\pgfqpoint{2.760912in}{0.751350in}}%
\pgfpathlineto{\pgfqpoint{2.766514in}{0.752216in}}%
\pgfpathlineto{\pgfqpoint{2.768755in}{0.752797in}}%
\pgfpathlineto{\pgfqpoint{2.774358in}{0.753656in}}%
\pgfpathlineto{\pgfqpoint{2.784443in}{0.755673in}}%
\pgfpathlineto{\pgfqpoint{2.790045in}{0.756571in}}%
\pgfpathlineto{\pgfqpoint{2.792286in}{0.757181in}}%
\pgfpathlineto{\pgfqpoint{2.799009in}{0.758098in}}%
\pgfpathlineto{\pgfqpoint{2.800130in}{0.758411in}}%
\pgfpathlineto{\pgfqpoint{2.805733in}{0.759348in}}%
\pgfpathlineto{\pgfqpoint{2.807974in}{0.759957in}}%
\pgfpathlineto{\pgfqpoint{2.813576in}{0.760851in}}%
\pgfpathlineto{\pgfqpoint{2.815817in}{0.761437in}}%
\pgfpathlineto{\pgfqpoint{2.821420in}{0.762323in}}%
\pgfpathlineto{\pgfqpoint{2.831505in}{0.764392in}}%
\pgfpathlineto{\pgfqpoint{2.842710in}{0.765820in}}%
\pgfpathlineto{\pgfqpoint{2.855036in}{0.768594in}}%
\pgfpathlineto{\pgfqpoint{2.860638in}{0.769585in}}%
\pgfpathlineto{\pgfqpoint{2.862880in}{0.770227in}}%
\pgfpathlineto{\pgfqpoint{2.868482in}{0.771214in}}%
\pgfpathlineto{\pgfqpoint{2.870723in}{0.771873in}}%
\pgfpathlineto{\pgfqpoint{2.876326in}{0.772869in}}%
\pgfpathlineto{\pgfqpoint{2.878567in}{0.773509in}}%
\pgfpathlineto{\pgfqpoint{2.884170in}{0.774531in}}%
\pgfpathlineto{\pgfqpoint{2.886411in}{0.775207in}}%
\pgfpathlineto{\pgfqpoint{2.892013in}{0.776244in}}%
\pgfpathlineto{\pgfqpoint{2.894254in}{0.776930in}}%
\pgfpathlineto{\pgfqpoint{2.899857in}{0.777967in}}%
\pgfpathlineto{\pgfqpoint{2.902098in}{0.778654in}}%
\pgfpathlineto{\pgfqpoint{2.908821in}{0.779679in}}%
\pgfpathlineto{\pgfqpoint{2.909942in}{0.780027in}}%
\pgfpathlineto{\pgfqpoint{2.915544in}{0.781102in}}%
\pgfpathlineto{\pgfqpoint{2.917785in}{0.781811in}}%
\pgfpathlineto{\pgfqpoint{2.923388in}{0.782857in}}%
\pgfpathlineto{\pgfqpoint{2.925629in}{0.783557in}}%
\pgfpathlineto{\pgfqpoint{2.931232in}{0.784555in}}%
\pgfpathlineto{\pgfqpoint{2.933473in}{0.785243in}}%
\pgfpathlineto{\pgfqpoint{2.939075in}{0.786283in}}%
\pgfpathlineto{\pgfqpoint{2.941316in}{0.786989in}}%
\pgfpathlineto{\pgfqpoint{2.946919in}{0.788069in}}%
\pgfpathlineto{\pgfqpoint{2.949160in}{0.788799in}}%
\pgfpathlineto{\pgfqpoint{2.954763in}{0.789879in}}%
\pgfpathlineto{\pgfqpoint{2.957004in}{0.790587in}}%
\pgfpathlineto{\pgfqpoint{2.962606in}{0.791673in}}%
\pgfpathlineto{\pgfqpoint{2.964848in}{0.792413in}}%
\pgfpathlineto{\pgfqpoint{2.970450in}{0.793533in}}%
\pgfpathlineto{\pgfqpoint{2.972691in}{0.794289in}}%
\pgfpathlineto{\pgfqpoint{2.978294in}{0.795441in}}%
\pgfpathlineto{\pgfqpoint{2.980535in}{0.796204in}}%
\pgfpathlineto{\pgfqpoint{2.986138in}{0.797328in}}%
\pgfpathlineto{\pgfqpoint{2.988379in}{0.798065in}}%
\pgfpathlineto{\pgfqpoint{2.993981in}{0.799175in}}%
\pgfpathlineto{\pgfqpoint{3.008548in}{0.802120in}}%
\pgfpathlineto{\pgfqpoint{3.011910in}{0.803224in}}%
\pgfpathlineto{\pgfqpoint{3.017512in}{0.804364in}}%
\pgfpathlineto{\pgfqpoint{3.019753in}{0.805125in}}%
\pgfpathlineto{\pgfqpoint{3.025356in}{0.806247in}}%
\pgfpathlineto{\pgfqpoint{3.027597in}{0.806992in}}%
\pgfpathlineto{\pgfqpoint{3.033200in}{0.807749in}}%
\pgfpathlineto{\pgfqpoint{3.035441in}{0.808513in}}%
\pgfpathlineto{\pgfqpoint{3.035441in}{0.808513in}}%
\pgfusepath{stroke}%
\end{pgfscope}%
\begin{pgfscope}%
\pgfsetrectcap%
\pgfsetmiterjoin%
\pgfsetlinewidth{0.803000pt}%
\definecolor{currentstroke}{rgb}{1.000000,1.000000,1.000000}%
\pgfsetstrokecolor{currentstroke}%
\pgfsetdash{}{0pt}%
\pgfpathmoveto{\pgfqpoint{0.462318in}{0.331635in}}%
\pgfpathlineto{\pgfqpoint{0.462318in}{1.436513in}}%
\pgfusepath{stroke}%
\end{pgfscope}%
\begin{pgfscope}%
\pgfsetrectcap%
\pgfsetmiterjoin%
\pgfsetlinewidth{0.803000pt}%
\definecolor{currentstroke}{rgb}{1.000000,1.000000,1.000000}%
\pgfsetstrokecolor{currentstroke}%
\pgfsetdash{}{0pt}%
\pgfpathmoveto{\pgfqpoint{3.157970in}{0.331635in}}%
\pgfpathlineto{\pgfqpoint{3.157970in}{1.436513in}}%
\pgfusepath{stroke}%
\end{pgfscope}%
\begin{pgfscope}%
\pgfsetrectcap%
\pgfsetmiterjoin%
\pgfsetlinewidth{0.803000pt}%
\definecolor{currentstroke}{rgb}{1.000000,1.000000,1.000000}%
\pgfsetstrokecolor{currentstroke}%
\pgfsetdash{}{0pt}%
\pgfpathmoveto{\pgfqpoint{0.462318in}{0.331635in}}%
\pgfpathlineto{\pgfqpoint{3.157970in}{0.331635in}}%
\pgfusepath{stroke}%
\end{pgfscope}%
\begin{pgfscope}%
\pgfsetrectcap%
\pgfsetmiterjoin%
\pgfsetlinewidth{0.803000pt}%
\definecolor{currentstroke}{rgb}{1.000000,1.000000,1.000000}%
\pgfsetstrokecolor{currentstroke}%
\pgfsetdash{}{0pt}%
\pgfpathmoveto{\pgfqpoint{0.462318in}{1.436513in}}%
\pgfpathlineto{\pgfqpoint{3.157970in}{1.436513in}}%
\pgfusepath{stroke}%
\end{pgfscope}%
\begin{pgfscope}%
\definecolor{textcolor}{rgb}{0.150000,0.150000,0.150000}%
\pgfsetstrokecolor{textcolor}%
\pgfsetfillcolor{textcolor}%
\pgftext[x=1.810144in,y=1.519846in,,base]{\color{textcolor}\rmfamily\fontsize{12.000000}{14.400000}\selectfont V}%
\end{pgfscope}%
\begin{pgfscope}%
\pgfsetbuttcap%
\pgfsetmiterjoin%
\definecolor{currentfill}{rgb}{0.917647,0.917647,0.949020}%
\pgfsetfillcolor{currentfill}%
\pgfsetlinewidth{0.000000pt}%
\definecolor{currentstroke}{rgb}{0.000000,0.000000,0.000000}%
\pgfsetstrokecolor{currentstroke}%
\pgfsetstrokeopacity{0.000000}%
\pgfsetdash{}{0pt}%
\pgfpathmoveto{\pgfqpoint{3.966666in}{0.331635in}}%
\pgfpathlineto{\pgfqpoint{6.662318in}{0.331635in}}%
\pgfpathlineto{\pgfqpoint{6.662318in}{1.436513in}}%
\pgfpathlineto{\pgfqpoint{3.966666in}{1.436513in}}%
\pgfpathclose%
\pgfusepath{fill}%
\end{pgfscope}%
\begin{pgfscope}%
\pgfpathrectangle{\pgfqpoint{3.966666in}{0.331635in}}{\pgfqpoint{2.695652in}{1.104878in}}%
\pgfusepath{clip}%
\pgfsetroundcap%
\pgfsetroundjoin%
\pgfsetlinewidth{0.803000pt}%
\definecolor{currentstroke}{rgb}{1.000000,1.000000,1.000000}%
\pgfsetstrokecolor{currentstroke}%
\pgfsetdash{}{0pt}%
\pgfpathmoveto{\pgfqpoint{4.086955in}{0.331635in}}%
\pgfpathlineto{\pgfqpoint{4.086955in}{1.436513in}}%
\pgfusepath{stroke}%
\end{pgfscope}%
\begin{pgfscope}%
\definecolor{textcolor}{rgb}{0.150000,0.150000,0.150000}%
\pgfsetstrokecolor{textcolor}%
\pgfsetfillcolor{textcolor}%
\pgftext[x=4.086955in,y=0.234413in,,top]{\color{textcolor}\rmfamily\fontsize{10.000000}{12.000000}\selectfont 2012}%
\end{pgfscope}%
\begin{pgfscope}%
\pgfpathrectangle{\pgfqpoint{3.966666in}{0.331635in}}{\pgfqpoint{2.695652in}{1.104878in}}%
\pgfusepath{clip}%
\pgfsetroundcap%
\pgfsetroundjoin%
\pgfsetlinewidth{0.803000pt}%
\definecolor{currentstroke}{rgb}{1.000000,1.000000,1.000000}%
\pgfsetstrokecolor{currentstroke}%
\pgfsetdash{}{0pt}%
\pgfpathmoveto{\pgfqpoint{4.497068in}{0.331635in}}%
\pgfpathlineto{\pgfqpoint{4.497068in}{1.436513in}}%
\pgfusepath{stroke}%
\end{pgfscope}%
\begin{pgfscope}%
\definecolor{textcolor}{rgb}{0.150000,0.150000,0.150000}%
\pgfsetstrokecolor{textcolor}%
\pgfsetfillcolor{textcolor}%
\pgftext[x=4.497068in,y=0.234413in,,top]{\color{textcolor}\rmfamily\fontsize{10.000000}{12.000000}\selectfont 2013}%
\end{pgfscope}%
\begin{pgfscope}%
\pgfpathrectangle{\pgfqpoint{3.966666in}{0.331635in}}{\pgfqpoint{2.695652in}{1.104878in}}%
\pgfusepath{clip}%
\pgfsetroundcap%
\pgfsetroundjoin%
\pgfsetlinewidth{0.803000pt}%
\definecolor{currentstroke}{rgb}{1.000000,1.000000,1.000000}%
\pgfsetstrokecolor{currentstroke}%
\pgfsetdash{}{0pt}%
\pgfpathmoveto{\pgfqpoint{4.906060in}{0.331635in}}%
\pgfpathlineto{\pgfqpoint{4.906060in}{1.436513in}}%
\pgfusepath{stroke}%
\end{pgfscope}%
\begin{pgfscope}%
\definecolor{textcolor}{rgb}{0.150000,0.150000,0.150000}%
\pgfsetstrokecolor{textcolor}%
\pgfsetfillcolor{textcolor}%
\pgftext[x=4.906060in,y=0.234413in,,top]{\color{textcolor}\rmfamily\fontsize{10.000000}{12.000000}\selectfont 2014}%
\end{pgfscope}%
\begin{pgfscope}%
\pgfpathrectangle{\pgfqpoint{3.966666in}{0.331635in}}{\pgfqpoint{2.695652in}{1.104878in}}%
\pgfusepath{clip}%
\pgfsetroundcap%
\pgfsetroundjoin%
\pgfsetlinewidth{0.803000pt}%
\definecolor{currentstroke}{rgb}{1.000000,1.000000,1.000000}%
\pgfsetstrokecolor{currentstroke}%
\pgfsetdash{}{0pt}%
\pgfpathmoveto{\pgfqpoint{5.315052in}{0.331635in}}%
\pgfpathlineto{\pgfqpoint{5.315052in}{1.436513in}}%
\pgfusepath{stroke}%
\end{pgfscope}%
\begin{pgfscope}%
\definecolor{textcolor}{rgb}{0.150000,0.150000,0.150000}%
\pgfsetstrokecolor{textcolor}%
\pgfsetfillcolor{textcolor}%
\pgftext[x=5.315052in,y=0.234413in,,top]{\color{textcolor}\rmfamily\fontsize{10.000000}{12.000000}\selectfont 2015}%
\end{pgfscope}%
\begin{pgfscope}%
\pgfpathrectangle{\pgfqpoint{3.966666in}{0.331635in}}{\pgfqpoint{2.695652in}{1.104878in}}%
\pgfusepath{clip}%
\pgfsetroundcap%
\pgfsetroundjoin%
\pgfsetlinewidth{0.803000pt}%
\definecolor{currentstroke}{rgb}{1.000000,1.000000,1.000000}%
\pgfsetstrokecolor{currentstroke}%
\pgfsetdash{}{0pt}%
\pgfpathmoveto{\pgfqpoint{5.724045in}{0.331635in}}%
\pgfpathlineto{\pgfqpoint{5.724045in}{1.436513in}}%
\pgfusepath{stroke}%
\end{pgfscope}%
\begin{pgfscope}%
\definecolor{textcolor}{rgb}{0.150000,0.150000,0.150000}%
\pgfsetstrokecolor{textcolor}%
\pgfsetfillcolor{textcolor}%
\pgftext[x=5.724045in,y=0.234413in,,top]{\color{textcolor}\rmfamily\fontsize{10.000000}{12.000000}\selectfont 2016}%
\end{pgfscope}%
\begin{pgfscope}%
\pgfpathrectangle{\pgfqpoint{3.966666in}{0.331635in}}{\pgfqpoint{2.695652in}{1.104878in}}%
\pgfusepath{clip}%
\pgfsetroundcap%
\pgfsetroundjoin%
\pgfsetlinewidth{0.803000pt}%
\definecolor{currentstroke}{rgb}{1.000000,1.000000,1.000000}%
\pgfsetstrokecolor{currentstroke}%
\pgfsetdash{}{0pt}%
\pgfpathmoveto{\pgfqpoint{6.134158in}{0.331635in}}%
\pgfpathlineto{\pgfqpoint{6.134158in}{1.436513in}}%
\pgfusepath{stroke}%
\end{pgfscope}%
\begin{pgfscope}%
\definecolor{textcolor}{rgb}{0.150000,0.150000,0.150000}%
\pgfsetstrokecolor{textcolor}%
\pgfsetfillcolor{textcolor}%
\pgftext[x=6.134158in,y=0.234413in,,top]{\color{textcolor}\rmfamily\fontsize{10.000000}{12.000000}\selectfont 2017}%
\end{pgfscope}%
\begin{pgfscope}%
\pgfpathrectangle{\pgfqpoint{3.966666in}{0.331635in}}{\pgfqpoint{2.695652in}{1.104878in}}%
\pgfusepath{clip}%
\pgfsetroundcap%
\pgfsetroundjoin%
\pgfsetlinewidth{0.803000pt}%
\definecolor{currentstroke}{rgb}{1.000000,1.000000,1.000000}%
\pgfsetstrokecolor{currentstroke}%
\pgfsetdash{}{0pt}%
\pgfpathmoveto{\pgfqpoint{6.543150in}{0.331635in}}%
\pgfpathlineto{\pgfqpoint{6.543150in}{1.436513in}}%
\pgfusepath{stroke}%
\end{pgfscope}%
\begin{pgfscope}%
\definecolor{textcolor}{rgb}{0.150000,0.150000,0.150000}%
\pgfsetstrokecolor{textcolor}%
\pgfsetfillcolor{textcolor}%
\pgftext[x=6.543150in,y=0.234413in,,top]{\color{textcolor}\rmfamily\fontsize{10.000000}{12.000000}\selectfont 2018}%
\end{pgfscope}%
\begin{pgfscope}%
\pgfpathrectangle{\pgfqpoint{3.966666in}{0.331635in}}{\pgfqpoint{2.695652in}{1.104878in}}%
\pgfusepath{clip}%
\pgfsetroundcap%
\pgfsetroundjoin%
\pgfsetlinewidth{0.803000pt}%
\definecolor{currentstroke}{rgb}{1.000000,1.000000,1.000000}%
\pgfsetstrokecolor{currentstroke}%
\pgfsetdash{}{0pt}%
\pgfpathmoveto{\pgfqpoint{3.966666in}{0.575735in}}%
\pgfpathlineto{\pgfqpoint{6.662318in}{0.575735in}}%
\pgfusepath{stroke}%
\end{pgfscope}%
\begin{pgfscope}%
\definecolor{textcolor}{rgb}{0.150000,0.150000,0.150000}%
\pgfsetstrokecolor{textcolor}%
\pgfsetfillcolor{textcolor}%
\pgftext[x=3.692713in,y=0.522973in,left,base]{\color{textcolor}\rmfamily\fontsize{10.000000}{12.000000}\selectfont 50}%
\end{pgfscope}%
\begin{pgfscope}%
\pgfpathrectangle{\pgfqpoint{3.966666in}{0.331635in}}{\pgfqpoint{2.695652in}{1.104878in}}%
\pgfusepath{clip}%
\pgfsetroundcap%
\pgfsetroundjoin%
\pgfsetlinewidth{0.803000pt}%
\definecolor{currentstroke}{rgb}{1.000000,1.000000,1.000000}%
\pgfsetstrokecolor{currentstroke}%
\pgfsetdash{}{0pt}%
\pgfpathmoveto{\pgfqpoint{3.966666in}{1.201550in}}%
\pgfpathlineto{\pgfqpoint{6.662318in}{1.201550in}}%
\pgfusepath{stroke}%
\end{pgfscope}%
\begin{pgfscope}%
\definecolor{textcolor}{rgb}{0.150000,0.150000,0.150000}%
\pgfsetstrokecolor{textcolor}%
\pgfsetfillcolor{textcolor}%
\pgftext[x=3.604348in,y=1.148789in,left,base]{\color{textcolor}\rmfamily\fontsize{10.000000}{12.000000}\selectfont 100}%
\end{pgfscope}%
\begin{pgfscope}%
\pgfpathrectangle{\pgfqpoint{3.966666in}{0.331635in}}{\pgfqpoint{2.695652in}{1.104878in}}%
\pgfusepath{clip}%
\pgfsetroundcap%
\pgfsetroundjoin%
\pgfsetlinewidth{1.505625pt}%
\definecolor{currentstroke}{rgb}{0.090196,0.745098,0.811765}%
\pgfsetstrokecolor{currentstroke}%
\pgfsetdash{}{0pt}%
\pgfpathmoveto{\pgfqpoint{4.089196in}{0.381857in}}%
\pgfpathlineto{\pgfqpoint{4.092557in}{0.399880in}}%
\pgfpathlineto{\pgfqpoint{4.097039in}{0.396751in}}%
\pgfpathlineto{\pgfqpoint{4.098160in}{0.386237in}}%
\pgfpathlineto{\pgfqpoint{4.099280in}{0.386488in}}%
\pgfpathlineto{\pgfqpoint{4.100401in}{0.382858in}}%
\pgfpathlineto{\pgfqpoint{4.104883in}{0.383734in}}%
\pgfpathlineto{\pgfqpoint{4.107124in}{0.394498in}}%
\pgfpathlineto{\pgfqpoint{4.108245in}{0.393121in}}%
\pgfpathlineto{\pgfqpoint{4.112727in}{0.392370in}}%
\pgfpathlineto{\pgfqpoint{4.113847in}{0.395875in}}%
\pgfpathlineto{\pgfqpoint{4.116088in}{0.392370in}}%
\pgfpathlineto{\pgfqpoint{4.120570in}{0.388490in}}%
\pgfpathlineto{\pgfqpoint{4.121691in}{0.393247in}}%
\pgfpathlineto{\pgfqpoint{4.122811in}{0.388616in}}%
\pgfpathlineto{\pgfqpoint{4.123932in}{0.400882in}}%
\pgfpathlineto{\pgfqpoint{4.127294in}{0.406013in}}%
\pgfpathlineto{\pgfqpoint{4.129535in}{0.415150in}}%
\pgfpathlineto{\pgfqpoint{4.130655in}{0.418154in}}%
\pgfpathlineto{\pgfqpoint{4.131776in}{0.417153in}}%
\pgfpathlineto{\pgfqpoint{4.135137in}{0.421033in}}%
\pgfpathlineto{\pgfqpoint{4.136258in}{0.418905in}}%
\pgfpathlineto{\pgfqpoint{4.137378in}{0.414900in}}%
\pgfpathlineto{\pgfqpoint{4.139619in}{0.420532in}}%
\pgfpathlineto{\pgfqpoint{4.144101in}{0.418530in}}%
\pgfpathlineto{\pgfqpoint{4.145222in}{0.415150in}}%
\pgfpathlineto{\pgfqpoint{4.146343in}{0.417528in}}%
\pgfpathlineto{\pgfqpoint{4.147463in}{0.415651in}}%
\pgfpathlineto{\pgfqpoint{4.150825in}{0.419406in}}%
\pgfpathlineto{\pgfqpoint{4.151945in}{0.422660in}}%
\pgfpathlineto{\pgfqpoint{4.153066in}{0.423286in}}%
\pgfpathlineto{\pgfqpoint{4.154186in}{0.427792in}}%
\pgfpathlineto{\pgfqpoint{4.155307in}{0.427416in}}%
\pgfpathlineto{\pgfqpoint{4.158668in}{0.431296in}}%
\pgfpathlineto{\pgfqpoint{4.159789in}{0.423411in}}%
\pgfpathlineto{\pgfqpoint{4.160909in}{0.420532in}}%
\pgfpathlineto{\pgfqpoint{4.163150in}{0.426164in}}%
\pgfpathlineto{\pgfqpoint{4.166512in}{0.427291in}}%
\pgfpathlineto{\pgfqpoint{4.167633in}{0.446065in}}%
\pgfpathlineto{\pgfqpoint{4.168753in}{0.440058in}}%
\pgfpathlineto{\pgfqpoint{4.169874in}{0.439932in}}%
\pgfpathlineto{\pgfqpoint{4.170994in}{0.436803in}}%
\pgfpathlineto{\pgfqpoint{4.174356in}{0.439682in}}%
\pgfpathlineto{\pgfqpoint{4.175476in}{0.437429in}}%
\pgfpathlineto{\pgfqpoint{4.177717in}{0.437930in}}%
\pgfpathlineto{\pgfqpoint{4.178838in}{0.442060in}}%
\pgfpathlineto{\pgfqpoint{4.182199in}{0.450196in}}%
\pgfpathlineto{\pgfqpoint{4.183320in}{0.447693in}}%
\pgfpathlineto{\pgfqpoint{4.185561in}{0.434801in}}%
\pgfpathlineto{\pgfqpoint{4.186682in}{0.443437in}}%
\pgfpathlineto{\pgfqpoint{4.190043in}{0.444188in}}%
\pgfpathlineto{\pgfqpoint{4.192284in}{0.433925in}}%
\pgfpathlineto{\pgfqpoint{4.193405in}{0.435552in}}%
\pgfpathlineto{\pgfqpoint{4.197887in}{0.424663in}}%
\pgfpathlineto{\pgfqpoint{4.199007in}{0.412021in}}%
\pgfpathlineto{\pgfqpoint{4.200128in}{0.416277in}}%
\pgfpathlineto{\pgfqpoint{4.201248in}{0.425038in}}%
\pgfpathlineto{\pgfqpoint{4.202369in}{0.421659in}}%
\pgfpathlineto{\pgfqpoint{4.205730in}{0.419531in}}%
\pgfpathlineto{\pgfqpoint{4.206851in}{0.431046in}}%
\pgfpathlineto{\pgfqpoint{4.207972in}{0.428918in}}%
\pgfpathlineto{\pgfqpoint{4.209092in}{0.424287in}}%
\pgfpathlineto{\pgfqpoint{4.210213in}{0.427291in}}%
\pgfpathlineto{\pgfqpoint{4.213574in}{0.423536in}}%
\pgfpathlineto{\pgfqpoint{4.214695in}{0.425414in}}%
\pgfpathlineto{\pgfqpoint{4.216936in}{0.438681in}}%
\pgfpathlineto{\pgfqpoint{4.218056in}{0.438681in}}%
\pgfpathlineto{\pgfqpoint{4.221418in}{0.435927in}}%
\pgfpathlineto{\pgfqpoint{4.222538in}{0.443562in}}%
\pgfpathlineto{\pgfqpoint{4.223659in}{0.440809in}}%
\pgfpathlineto{\pgfqpoint{4.224779in}{0.443813in}}%
\pgfpathlineto{\pgfqpoint{4.225900in}{0.433925in}}%
\pgfpathlineto{\pgfqpoint{4.229262in}{0.443938in}}%
\pgfpathlineto{\pgfqpoint{4.230382in}{0.449320in}}%
\pgfpathlineto{\pgfqpoint{4.231503in}{0.457455in}}%
\pgfpathlineto{\pgfqpoint{4.233744in}{0.463588in}}%
\pgfpathlineto{\pgfqpoint{4.238226in}{0.457330in}}%
\pgfpathlineto{\pgfqpoint{4.239346in}{0.458081in}}%
\pgfpathlineto{\pgfqpoint{4.241587in}{0.443813in}}%
\pgfpathlineto{\pgfqpoint{4.244949in}{0.450321in}}%
\pgfpathlineto{\pgfqpoint{4.246069in}{0.450321in}}%
\pgfpathlineto{\pgfqpoint{4.247190in}{0.448318in}}%
\pgfpathlineto{\pgfqpoint{4.248311in}{0.450947in}}%
\pgfpathlineto{\pgfqpoint{4.249431in}{0.451573in}}%
\pgfpathlineto{\pgfqpoint{4.253913in}{0.462587in}}%
\pgfpathlineto{\pgfqpoint{4.255034in}{0.459458in}}%
\pgfpathlineto{\pgfqpoint{4.256154in}{0.465215in}}%
\pgfpathlineto{\pgfqpoint{4.257275in}{0.450446in}}%
\pgfpathlineto{\pgfqpoint{4.260636in}{0.450571in}}%
\pgfpathlineto{\pgfqpoint{4.261757in}{0.455453in}}%
\pgfpathlineto{\pgfqpoint{4.262877in}{0.463213in}}%
\pgfpathlineto{\pgfqpoint{4.263998in}{0.464339in}}%
\pgfpathlineto{\pgfqpoint{4.265118in}{0.471223in}}%
\pgfpathlineto{\pgfqpoint{4.268480in}{0.466217in}}%
\pgfpathlineto{\pgfqpoint{4.269601in}{0.472725in}}%
\pgfpathlineto{\pgfqpoint{4.270721in}{0.471098in}}%
\pgfpathlineto{\pgfqpoint{4.271842in}{0.481862in}}%
\pgfpathlineto{\pgfqpoint{4.272962in}{0.480736in}}%
\pgfpathlineto{\pgfqpoint{4.276324in}{0.480861in}}%
\pgfpathlineto{\pgfqpoint{4.278565in}{0.487995in}}%
\pgfpathlineto{\pgfqpoint{4.279685in}{0.484240in}}%
\pgfpathlineto{\pgfqpoint{4.280806in}{0.485116in}}%
\pgfpathlineto{\pgfqpoint{4.284167in}{0.476355in}}%
\pgfpathlineto{\pgfqpoint{4.286408in}{0.489622in}}%
\pgfpathlineto{\pgfqpoint{4.287529in}{0.488996in}}%
\pgfpathlineto{\pgfqpoint{4.288649in}{0.496631in}}%
\pgfpathlineto{\pgfqpoint{4.292011in}{0.499135in}}%
\pgfpathlineto{\pgfqpoint{4.293132in}{0.497758in}}%
\pgfpathlineto{\pgfqpoint{4.295373in}{0.492251in}}%
\pgfpathlineto{\pgfqpoint{4.296493in}{0.491500in}}%
\pgfpathlineto{\pgfqpoint{4.299855in}{0.490999in}}%
\pgfpathlineto{\pgfqpoint{4.300975in}{0.483865in}}%
\pgfpathlineto{\pgfqpoint{4.302096in}{0.482863in}}%
\pgfpathlineto{\pgfqpoint{4.303216in}{0.484365in}}%
\pgfpathlineto{\pgfqpoint{4.304337in}{0.493252in}}%
\pgfpathlineto{\pgfqpoint{4.307698in}{0.489497in}}%
\pgfpathlineto{\pgfqpoint{4.308819in}{0.506269in}}%
\pgfpathlineto{\pgfqpoint{4.309940in}{0.506269in}}%
\pgfpathlineto{\pgfqpoint{4.312181in}{0.497758in}}%
\pgfpathlineto{\pgfqpoint{4.315542in}{0.490874in}}%
\pgfpathlineto{\pgfqpoint{4.317783in}{0.494378in}}%
\pgfpathlineto{\pgfqpoint{4.318904in}{0.510274in}}%
\pgfpathlineto{\pgfqpoint{4.320024in}{0.512903in}}%
\pgfpathlineto{\pgfqpoint{4.323386in}{0.511401in}}%
\pgfpathlineto{\pgfqpoint{4.324506in}{0.503891in}}%
\pgfpathlineto{\pgfqpoint{4.325627in}{0.500261in}}%
\pgfpathlineto{\pgfqpoint{4.326747in}{0.502139in}}%
\pgfpathlineto{\pgfqpoint{4.327868in}{0.511025in}}%
\pgfpathlineto{\pgfqpoint{4.331230in}{0.509648in}}%
\pgfpathlineto{\pgfqpoint{4.332350in}{0.511401in}}%
\pgfpathlineto{\pgfqpoint{4.333471in}{0.519161in}}%
\pgfpathlineto{\pgfqpoint{4.335712in}{0.509648in}}%
\pgfpathlineto{\pgfqpoint{4.339073in}{0.512026in}}%
\pgfpathlineto{\pgfqpoint{4.340194in}{0.510149in}}%
\pgfpathlineto{\pgfqpoint{4.341314in}{0.512402in}}%
\pgfpathlineto{\pgfqpoint{4.343555in}{0.518785in}}%
\pgfpathlineto{\pgfqpoint{4.346917in}{0.518660in}}%
\pgfpathlineto{\pgfqpoint{4.348037in}{0.509523in}}%
\pgfpathlineto{\pgfqpoint{4.349158in}{0.509774in}}%
\pgfpathlineto{\pgfqpoint{4.350278in}{0.504266in}}%
\pgfpathlineto{\pgfqpoint{4.351399in}{0.508647in}}%
\pgfpathlineto{\pgfqpoint{4.355881in}{0.509398in}}%
\pgfpathlineto{\pgfqpoint{4.357002in}{0.514029in}}%
\pgfpathlineto{\pgfqpoint{4.358122in}{0.507020in}}%
\pgfpathlineto{\pgfqpoint{4.363725in}{0.509774in}}%
\pgfpathlineto{\pgfqpoint{4.365966in}{0.534556in}}%
\pgfpathlineto{\pgfqpoint{4.367086in}{0.533179in}}%
\pgfpathlineto{\pgfqpoint{4.370448in}{0.530676in}}%
\pgfpathlineto{\pgfqpoint{4.371569in}{0.531176in}}%
\pgfpathlineto{\pgfqpoint{4.372689in}{0.532804in}}%
\pgfpathlineto{\pgfqpoint{4.373810in}{0.542942in}}%
\pgfpathlineto{\pgfqpoint{4.374930in}{0.540063in}}%
\pgfpathlineto{\pgfqpoint{4.378292in}{0.537810in}}%
\pgfpathlineto{\pgfqpoint{4.379412in}{0.535056in}}%
\pgfpathlineto{\pgfqpoint{4.380533in}{0.544068in}}%
\pgfpathlineto{\pgfqpoint{4.381653in}{0.543568in}}%
\pgfpathlineto{\pgfqpoint{4.386135in}{0.546571in}}%
\pgfpathlineto{\pgfqpoint{4.387256in}{0.542191in}}%
\pgfpathlineto{\pgfqpoint{4.388376in}{0.535432in}}%
\pgfpathlineto{\pgfqpoint{4.389497in}{0.542691in}}%
\pgfpathlineto{\pgfqpoint{4.390617in}{0.539312in}}%
\pgfpathlineto{\pgfqpoint{4.393979in}{0.536934in}}%
\pgfpathlineto{\pgfqpoint{4.395100in}{0.532053in}}%
\pgfpathlineto{\pgfqpoint{4.396220in}{0.541440in}}%
\pgfpathlineto{\pgfqpoint{4.397341in}{0.543192in}}%
\pgfpathlineto{\pgfqpoint{4.398461in}{0.547072in}}%
\pgfpathlineto{\pgfqpoint{4.401823in}{0.539813in}}%
\pgfpathlineto{\pgfqpoint{4.402943in}{0.530676in}}%
\pgfpathlineto{\pgfqpoint{4.404064in}{0.527296in}}%
\pgfpathlineto{\pgfqpoint{4.405184in}{0.517408in}}%
\pgfpathlineto{\pgfqpoint{4.406305in}{0.520287in}}%
\pgfpathlineto{\pgfqpoint{4.409666in}{0.522540in}}%
\pgfpathlineto{\pgfqpoint{4.410787in}{0.527547in}}%
\pgfpathlineto{\pgfqpoint{4.411907in}{0.539437in}}%
\pgfpathlineto{\pgfqpoint{4.413028in}{0.540939in}}%
\pgfpathlineto{\pgfqpoint{4.414149in}{0.535056in}}%
\pgfpathlineto{\pgfqpoint{4.417510in}{0.533805in}}%
\pgfpathlineto{\pgfqpoint{4.418631in}{0.522165in}}%
\pgfpathlineto{\pgfqpoint{4.419751in}{0.520913in}}%
\pgfpathlineto{\pgfqpoint{4.420872in}{0.516532in}}%
\pgfpathlineto{\pgfqpoint{4.427595in}{0.503641in}}%
\pgfpathlineto{\pgfqpoint{4.428715in}{0.511150in}}%
\pgfpathlineto{\pgfqpoint{4.429836in}{0.512026in}}%
\pgfpathlineto{\pgfqpoint{4.434318in}{0.518910in}}%
\pgfpathlineto{\pgfqpoint{4.435439in}{0.514530in}}%
\pgfpathlineto{\pgfqpoint{4.436559in}{0.514029in}}%
\pgfpathlineto{\pgfqpoint{4.437680in}{0.480485in}}%
\pgfpathlineto{\pgfqpoint{4.441041in}{0.484866in}}%
\pgfpathlineto{\pgfqpoint{4.442162in}{0.490624in}}%
\pgfpathlineto{\pgfqpoint{4.443282in}{0.481737in}}%
\pgfpathlineto{\pgfqpoint{4.444403in}{0.485116in}}%
\pgfpathlineto{\pgfqpoint{4.445523in}{0.484491in}}%
\pgfpathlineto{\pgfqpoint{4.448885in}{0.489998in}}%
\pgfpathlineto{\pgfqpoint{4.450005in}{0.496006in}}%
\pgfpathlineto{\pgfqpoint{4.453367in}{0.505268in}}%
\pgfpathlineto{\pgfqpoint{4.456729in}{0.502639in}}%
\pgfpathlineto{\pgfqpoint{4.457849in}{0.497758in}}%
\pgfpathlineto{\pgfqpoint{4.460090in}{0.510399in}}%
\pgfpathlineto{\pgfqpoint{4.461211in}{0.509774in}}%
\pgfpathlineto{\pgfqpoint{4.464572in}{0.505643in}}%
\pgfpathlineto{\pgfqpoint{4.465693in}{0.505643in}}%
\pgfpathlineto{\pgfqpoint{4.467934in}{0.511526in}}%
\pgfpathlineto{\pgfqpoint{4.469054in}{0.513528in}}%
\pgfpathlineto{\pgfqpoint{4.472416in}{0.514279in}}%
\pgfpathlineto{\pgfqpoint{4.474657in}{0.518159in}}%
\pgfpathlineto{\pgfqpoint{4.476898in}{0.507020in}}%
\pgfpathlineto{\pgfqpoint{4.480260in}{0.514029in}}%
\pgfpathlineto{\pgfqpoint{4.481380in}{0.524793in}}%
\pgfpathlineto{\pgfqpoint{4.482501in}{0.521539in}}%
\pgfpathlineto{\pgfqpoint{4.483621in}{0.532929in}}%
\pgfpathlineto{\pgfqpoint{4.484742in}{0.522290in}}%
\pgfpathlineto{\pgfqpoint{4.490344in}{0.520538in}}%
\pgfpathlineto{\pgfqpoint{4.492585in}{0.512527in}}%
\pgfpathlineto{\pgfqpoint{4.495947in}{0.519787in}}%
\pgfpathlineto{\pgfqpoint{4.498188in}{0.534806in}}%
\pgfpathlineto{\pgfqpoint{4.499309in}{0.536058in}}%
\pgfpathlineto{\pgfqpoint{4.500429in}{0.547322in}}%
\pgfpathlineto{\pgfqpoint{4.504911in}{0.530926in}}%
\pgfpathlineto{\pgfqpoint{4.507152in}{0.531302in}}%
\pgfpathlineto{\pgfqpoint{4.508273in}{0.528923in}}%
\pgfpathlineto{\pgfqpoint{4.511634in}{0.529049in}}%
\pgfpathlineto{\pgfqpoint{4.513875in}{0.539813in}}%
\pgfpathlineto{\pgfqpoint{4.514996in}{0.549826in}}%
\pgfpathlineto{\pgfqpoint{4.516117in}{0.549075in}}%
\pgfpathlineto{\pgfqpoint{4.520599in}{0.553455in}}%
\pgfpathlineto{\pgfqpoint{4.521719in}{0.567474in}}%
\pgfpathlineto{\pgfqpoint{4.522840in}{0.567474in}}%
\pgfpathlineto{\pgfqpoint{4.523960in}{0.572355in}}%
\pgfpathlineto{\pgfqpoint{4.527322in}{0.572105in}}%
\pgfpathlineto{\pgfqpoint{4.529563in}{0.565596in}}%
\pgfpathlineto{\pgfqpoint{4.530683in}{0.566723in}}%
\pgfpathlineto{\pgfqpoint{4.531804in}{0.574733in}}%
\pgfpathlineto{\pgfqpoint{4.535165in}{0.566848in}}%
\pgfpathlineto{\pgfqpoint{4.537407in}{0.573982in}}%
\pgfpathlineto{\pgfqpoint{4.538527in}{0.572105in}}%
\pgfpathlineto{\pgfqpoint{4.539648in}{0.575609in}}%
\pgfpathlineto{\pgfqpoint{4.543009in}{0.576611in}}%
\pgfpathlineto{\pgfqpoint{4.544130in}{0.578864in}}%
\pgfpathlineto{\pgfqpoint{4.545250in}{0.578989in}}%
\pgfpathlineto{\pgfqpoint{4.546371in}{0.578113in}}%
\pgfpathlineto{\pgfqpoint{4.547491in}{0.586499in}}%
\pgfpathlineto{\pgfqpoint{4.551973in}{0.587875in}}%
\pgfpathlineto{\pgfqpoint{4.553094in}{0.574858in}}%
\pgfpathlineto{\pgfqpoint{4.554214in}{0.569977in}}%
\pgfpathlineto{\pgfqpoint{4.555335in}{0.570853in}}%
\pgfpathlineto{\pgfqpoint{4.558697in}{0.563343in}}%
\pgfpathlineto{\pgfqpoint{4.559817in}{0.566848in}}%
\pgfpathlineto{\pgfqpoint{4.560938in}{0.573482in}}%
\pgfpathlineto{\pgfqpoint{4.562058in}{0.574733in}}%
\pgfpathlineto{\pgfqpoint{4.563179in}{0.583244in}}%
\pgfpathlineto{\pgfqpoint{4.566540in}{0.588626in}}%
\pgfpathlineto{\pgfqpoint{4.567661in}{0.596386in}}%
\pgfpathlineto{\pgfqpoint{4.568781in}{0.595010in}}%
\pgfpathlineto{\pgfqpoint{4.569902in}{0.594634in}}%
\pgfpathlineto{\pgfqpoint{4.571022in}{0.606900in}}%
\pgfpathlineto{\pgfqpoint{4.574384in}{0.609904in}}%
\pgfpathlineto{\pgfqpoint{4.575504in}{0.603646in}}%
\pgfpathlineto{\pgfqpoint{4.576625in}{0.606274in}}%
\pgfpathlineto{\pgfqpoint{4.577746in}{0.610905in}}%
\pgfpathlineto{\pgfqpoint{4.578866in}{0.609028in}}%
\pgfpathlineto{\pgfqpoint{4.582228in}{0.600392in}}%
\pgfpathlineto{\pgfqpoint{4.583348in}{0.594509in}}%
\pgfpathlineto{\pgfqpoint{4.584469in}{0.601643in}}%
\pgfpathlineto{\pgfqpoint{4.585589in}{0.594509in}}%
\pgfpathlineto{\pgfqpoint{4.586710in}{0.599891in}}%
\pgfpathlineto{\pgfqpoint{4.590071in}{0.593383in}}%
\pgfpathlineto{\pgfqpoint{4.591192in}{0.598139in}}%
\pgfpathlineto{\pgfqpoint{4.592312in}{0.596261in}}%
\pgfpathlineto{\pgfqpoint{4.593433in}{0.600141in}}%
\pgfpathlineto{\pgfqpoint{4.597915in}{0.598890in}}%
\pgfpathlineto{\pgfqpoint{4.599036in}{0.607651in}}%
\pgfpathlineto{\pgfqpoint{4.600156in}{0.605273in}}%
\pgfpathlineto{\pgfqpoint{4.601277in}{0.609153in}}%
\pgfpathlineto{\pgfqpoint{4.602397in}{0.610405in}}%
\pgfpathlineto{\pgfqpoint{4.606879in}{0.626926in}}%
\pgfpathlineto{\pgfqpoint{4.608000in}{0.637941in}}%
\pgfpathlineto{\pgfqpoint{4.609120in}{0.643072in}}%
\pgfpathlineto{\pgfqpoint{4.610241in}{0.643072in}}%
\pgfpathlineto{\pgfqpoint{4.613602in}{0.623922in}}%
\pgfpathlineto{\pgfqpoint{4.614723in}{0.645325in}}%
\pgfpathlineto{\pgfqpoint{4.615843in}{0.644449in}}%
\pgfpathlineto{\pgfqpoint{4.616964in}{0.636564in}}%
\pgfpathlineto{\pgfqpoint{4.618084in}{0.654587in}}%
\pgfpathlineto{\pgfqpoint{4.621446in}{0.659719in}}%
\pgfpathlineto{\pgfqpoint{4.622567in}{0.666353in}}%
\pgfpathlineto{\pgfqpoint{4.623687in}{0.658968in}}%
\pgfpathlineto{\pgfqpoint{4.624808in}{0.659594in}}%
\pgfpathlineto{\pgfqpoint{4.625928in}{0.658092in}}%
\pgfpathlineto{\pgfqpoint{4.629290in}{0.671109in}}%
\pgfpathlineto{\pgfqpoint{4.630410in}{0.669231in}}%
\pgfpathlineto{\pgfqpoint{4.631531in}{0.673487in}}%
\pgfpathlineto{\pgfqpoint{4.633772in}{0.691636in}}%
\pgfpathlineto{\pgfqpoint{4.637133in}{0.694640in}}%
\pgfpathlineto{\pgfqpoint{4.638254in}{0.706155in}}%
\pgfpathlineto{\pgfqpoint{4.639374in}{0.705278in}}%
\pgfpathlineto{\pgfqpoint{4.641616in}{0.719172in}}%
\pgfpathlineto{\pgfqpoint{4.644977in}{0.720548in}}%
\pgfpathlineto{\pgfqpoint{4.647218in}{0.724554in}}%
\pgfpathlineto{\pgfqpoint{4.648339in}{0.710786in}}%
\pgfpathlineto{\pgfqpoint{4.649459in}{0.712037in}}%
\pgfpathlineto{\pgfqpoint{4.652821in}{0.706780in}}%
\pgfpathlineto{\pgfqpoint{4.656182in}{0.696642in}}%
\pgfpathlineto{\pgfqpoint{4.657303in}{0.699521in}}%
\pgfpathlineto{\pgfqpoint{4.661785in}{0.713289in}}%
\pgfpathlineto{\pgfqpoint{4.662906in}{0.708407in}}%
\pgfpathlineto{\pgfqpoint{4.665147in}{0.671985in}}%
\pgfpathlineto{\pgfqpoint{4.668508in}{0.680246in}}%
\pgfpathlineto{\pgfqpoint{4.669629in}{0.686504in}}%
\pgfpathlineto{\pgfqpoint{4.670749in}{0.672486in}}%
\pgfpathlineto{\pgfqpoint{4.671870in}{0.672611in}}%
\pgfpathlineto{\pgfqpoint{4.672990in}{0.692261in}}%
\pgfpathlineto{\pgfqpoint{4.676352in}{0.680621in}}%
\pgfpathlineto{\pgfqpoint{4.677472in}{0.680246in}}%
\pgfpathlineto{\pgfqpoint{4.678593in}{0.670984in}}%
\pgfpathlineto{\pgfqpoint{4.679713in}{0.685878in}}%
\pgfpathlineto{\pgfqpoint{4.680834in}{0.680246in}}%
\pgfpathlineto{\pgfqpoint{4.684196in}{0.688131in}}%
\pgfpathlineto{\pgfqpoint{4.685316in}{0.697268in}}%
\pgfpathlineto{\pgfqpoint{4.686437in}{0.686254in}}%
\pgfpathlineto{\pgfqpoint{4.687557in}{0.659344in}}%
\pgfpathlineto{\pgfqpoint{4.688678in}{0.667980in}}%
\pgfpathlineto{\pgfqpoint{4.692039in}{0.664600in}}%
\pgfpathlineto{\pgfqpoint{4.693160in}{0.666102in}}%
\pgfpathlineto{\pgfqpoint{4.695401in}{0.679244in}}%
\pgfpathlineto{\pgfqpoint{4.696521in}{0.672736in}}%
\pgfpathlineto{\pgfqpoint{4.699883in}{0.681748in}}%
\pgfpathlineto{\pgfqpoint{4.701003in}{0.673988in}}%
\pgfpathlineto{\pgfqpoint{4.702124in}{0.677993in}}%
\pgfpathlineto{\pgfqpoint{4.704365in}{0.680496in}}%
\pgfpathlineto{\pgfqpoint{4.708847in}{0.693263in}}%
\pgfpathlineto{\pgfqpoint{4.709968in}{0.692887in}}%
\pgfpathlineto{\pgfqpoint{4.711088in}{0.712037in}}%
\pgfpathlineto{\pgfqpoint{4.712209in}{0.716668in}}%
\pgfpathlineto{\pgfqpoint{4.715570in}{0.704653in}}%
\pgfpathlineto{\pgfqpoint{4.716691in}{0.694264in}}%
\pgfpathlineto{\pgfqpoint{4.718932in}{0.703276in}}%
\pgfpathlineto{\pgfqpoint{4.720052in}{0.695766in}}%
\pgfpathlineto{\pgfqpoint{4.723414in}{0.687130in}}%
\pgfpathlineto{\pgfqpoint{4.724535in}{0.687630in}}%
\pgfpathlineto{\pgfqpoint{4.725655in}{0.689883in}}%
\pgfpathlineto{\pgfqpoint{4.726776in}{0.688757in}}%
\pgfpathlineto{\pgfqpoint{4.727896in}{0.693763in}}%
\pgfpathlineto{\pgfqpoint{4.731258in}{0.689508in}}%
\pgfpathlineto{\pgfqpoint{4.732378in}{0.684626in}}%
\pgfpathlineto{\pgfqpoint{4.733499in}{0.689883in}}%
\pgfpathlineto{\pgfqpoint{4.734619in}{0.698019in}}%
\pgfpathlineto{\pgfqpoint{4.735740in}{0.711286in}}%
\pgfpathlineto{\pgfqpoint{4.739101in}{0.705654in}}%
\pgfpathlineto{\pgfqpoint{4.740222in}{0.717419in}}%
\pgfpathlineto{\pgfqpoint{4.741342in}{0.704402in}}%
\pgfpathlineto{\pgfqpoint{4.742463in}{0.702525in}}%
\pgfpathlineto{\pgfqpoint{4.743584in}{0.690885in}}%
\pgfpathlineto{\pgfqpoint{4.746945in}{0.681748in}}%
\pgfpathlineto{\pgfqpoint{4.748066in}{0.682248in}}%
\pgfpathlineto{\pgfqpoint{4.749186in}{0.681998in}}%
\pgfpathlineto{\pgfqpoint{4.750307in}{0.663975in}}%
\pgfpathlineto{\pgfqpoint{4.751427in}{0.661596in}}%
\pgfpathlineto{\pgfqpoint{4.754789in}{0.657716in}}%
\pgfpathlineto{\pgfqpoint{4.755909in}{0.658217in}}%
\pgfpathlineto{\pgfqpoint{4.757030in}{0.649706in}}%
\pgfpathlineto{\pgfqpoint{4.758150in}{0.655463in}}%
\pgfpathlineto{\pgfqpoint{4.759271in}{0.656465in}}%
\pgfpathlineto{\pgfqpoint{4.762632in}{0.652209in}}%
\pgfpathlineto{\pgfqpoint{4.763753in}{0.644574in}}%
\pgfpathlineto{\pgfqpoint{4.764874in}{0.645951in}}%
\pgfpathlineto{\pgfqpoint{4.765994in}{0.649080in}}%
\pgfpathlineto{\pgfqpoint{4.767115in}{0.646201in}}%
\pgfpathlineto{\pgfqpoint{4.771597in}{0.647328in}}%
\pgfpathlineto{\pgfqpoint{4.774958in}{0.652585in}}%
\pgfpathlineto{\pgfqpoint{4.778320in}{0.654963in}}%
\pgfpathlineto{\pgfqpoint{4.782802in}{0.713289in}}%
\pgfpathlineto{\pgfqpoint{4.788405in}{0.718045in}}%
\pgfpathlineto{\pgfqpoint{4.790646in}{0.694014in}}%
\pgfpathlineto{\pgfqpoint{4.794007in}{0.691135in}}%
\pgfpathlineto{\pgfqpoint{4.795128in}{0.686128in}}%
\pgfpathlineto{\pgfqpoint{4.796248in}{0.687630in}}%
\pgfpathlineto{\pgfqpoint{4.797369in}{0.696642in}}%
\pgfpathlineto{\pgfqpoint{4.798489in}{0.696141in}}%
\pgfpathlineto{\pgfqpoint{4.801851in}{0.688131in}}%
\pgfpathlineto{\pgfqpoint{4.802971in}{0.692011in}}%
\pgfpathlineto{\pgfqpoint{4.804092in}{0.692637in}}%
\pgfpathlineto{\pgfqpoint{4.805213in}{0.682749in}}%
\pgfpathlineto{\pgfqpoint{4.806333in}{0.697393in}}%
\pgfpathlineto{\pgfqpoint{4.809695in}{0.689258in}}%
\pgfpathlineto{\pgfqpoint{4.811936in}{0.677868in}}%
\pgfpathlineto{\pgfqpoint{4.813056in}{0.700647in}}%
\pgfpathlineto{\pgfqpoint{4.814177in}{0.707782in}}%
\pgfpathlineto{\pgfqpoint{4.817538in}{0.714916in}}%
\pgfpathlineto{\pgfqpoint{4.818659in}{0.710410in}}%
\pgfpathlineto{\pgfqpoint{4.819779in}{0.709409in}}%
\pgfpathlineto{\pgfqpoint{4.820900in}{0.710160in}}%
\pgfpathlineto{\pgfqpoint{4.822020in}{0.718546in}}%
\pgfpathlineto{\pgfqpoint{4.825382in}{0.723803in}}%
\pgfpathlineto{\pgfqpoint{4.826503in}{0.739698in}}%
\pgfpathlineto{\pgfqpoint{4.827623in}{0.729685in}}%
\pgfpathlineto{\pgfqpoint{4.828744in}{0.740324in}}%
\pgfpathlineto{\pgfqpoint{4.829864in}{0.742702in}}%
\pgfpathlineto{\pgfqpoint{4.834346in}{0.738822in}}%
\pgfpathlineto{\pgfqpoint{4.835467in}{0.733690in}}%
\pgfpathlineto{\pgfqpoint{4.836587in}{0.735067in}}%
\pgfpathlineto{\pgfqpoint{4.837708in}{0.739823in}}%
\pgfpathlineto{\pgfqpoint{4.841069in}{0.737571in}}%
\pgfpathlineto{\pgfqpoint{4.842190in}{0.738071in}}%
\pgfpathlineto{\pgfqpoint{4.843310in}{0.739698in}}%
\pgfpathlineto{\pgfqpoint{4.844431in}{0.718546in}}%
\pgfpathlineto{\pgfqpoint{4.845551in}{0.734942in}}%
\pgfpathlineto{\pgfqpoint{4.848913in}{0.732188in}}%
\pgfpathlineto{\pgfqpoint{4.850034in}{0.725680in}}%
\pgfpathlineto{\pgfqpoint{4.852275in}{0.751464in}}%
\pgfpathlineto{\pgfqpoint{4.853395in}{0.751213in}}%
\pgfpathlineto{\pgfqpoint{4.856757in}{0.745456in}}%
\pgfpathlineto{\pgfqpoint{4.857877in}{0.741075in}}%
\pgfpathlineto{\pgfqpoint{4.858998in}{0.742327in}}%
\pgfpathlineto{\pgfqpoint{4.860118in}{0.750462in}}%
\pgfpathlineto{\pgfqpoint{4.861239in}{0.753466in}}%
\pgfpathlineto{\pgfqpoint{4.864600in}{0.748209in}}%
\pgfpathlineto{\pgfqpoint{4.865721in}{0.764731in}}%
\pgfpathlineto{\pgfqpoint{4.866842in}{0.759975in}}%
\pgfpathlineto{\pgfqpoint{4.869083in}{0.757346in}}%
\pgfpathlineto{\pgfqpoint{4.872444in}{0.761602in}}%
\pgfpathlineto{\pgfqpoint{4.873565in}{0.750087in}}%
\pgfpathlineto{\pgfqpoint{4.874685in}{0.750838in}}%
\pgfpathlineto{\pgfqpoint{4.875806in}{0.753842in}}%
\pgfpathlineto{\pgfqpoint{4.876926in}{0.767860in}}%
\pgfpathlineto{\pgfqpoint{4.880288in}{0.763855in}}%
\pgfpathlineto{\pgfqpoint{4.881408in}{0.769112in}}%
\pgfpathlineto{\pgfqpoint{4.882529in}{0.757096in}}%
\pgfpathlineto{\pgfqpoint{4.884770in}{0.756720in}}%
\pgfpathlineto{\pgfqpoint{4.889252in}{0.768736in}}%
\pgfpathlineto{\pgfqpoint{4.891493in}{0.795521in}}%
\pgfpathlineto{\pgfqpoint{4.892614in}{0.788887in}}%
\pgfpathlineto{\pgfqpoint{4.895975in}{0.799026in}}%
\pgfpathlineto{\pgfqpoint{4.897096in}{0.805659in}}%
\pgfpathlineto{\pgfqpoint{4.899337in}{0.814671in}}%
\pgfpathlineto{\pgfqpoint{4.900457in}{0.811542in}}%
\pgfpathlineto{\pgfqpoint{4.903819in}{0.833320in}}%
\pgfpathlineto{\pgfqpoint{4.904939in}{0.835198in}}%
\pgfpathlineto{\pgfqpoint{4.907180in}{0.833696in}}%
\pgfpathlineto{\pgfqpoint{4.908301in}{0.831818in}}%
\pgfpathlineto{\pgfqpoint{4.911663in}{0.828564in}}%
\pgfpathlineto{\pgfqpoint{4.912783in}{0.834572in}}%
\pgfpathlineto{\pgfqpoint{4.913904in}{0.821555in}}%
\pgfpathlineto{\pgfqpoint{4.915024in}{0.817800in}}%
\pgfpathlineto{\pgfqpoint{4.916145in}{0.823558in}}%
\pgfpathlineto{\pgfqpoint{4.919506in}{0.799026in}}%
\pgfpathlineto{\pgfqpoint{4.920627in}{0.812668in}}%
\pgfpathlineto{\pgfqpoint{4.921747in}{0.810666in}}%
\pgfpathlineto{\pgfqpoint{4.922868in}{0.809915in}}%
\pgfpathlineto{\pgfqpoint{4.923988in}{0.807161in}}%
\pgfpathlineto{\pgfqpoint{4.928471in}{0.809790in}}%
\pgfpathlineto{\pgfqpoint{4.929591in}{0.822556in}}%
\pgfpathlineto{\pgfqpoint{4.930712in}{0.816548in}}%
\pgfpathlineto{\pgfqpoint{4.931832in}{0.792642in}}%
\pgfpathlineto{\pgfqpoint{4.935194in}{0.787135in}}%
\pgfpathlineto{\pgfqpoint{4.936314in}{0.794395in}}%
\pgfpathlineto{\pgfqpoint{4.937435in}{0.776496in}}%
\pgfpathlineto{\pgfqpoint{4.938555in}{0.798400in}}%
\pgfpathlineto{\pgfqpoint{4.939676in}{0.791266in}}%
\pgfpathlineto{\pgfqpoint{4.943037in}{0.760976in}}%
\pgfpathlineto{\pgfqpoint{4.945278in}{0.781503in}}%
\pgfpathlineto{\pgfqpoint{4.946399in}{0.825560in}}%
\pgfpathlineto{\pgfqpoint{4.947519in}{0.826812in}}%
\pgfpathlineto{\pgfqpoint{4.950881in}{0.842833in}}%
\pgfpathlineto{\pgfqpoint{4.952002in}{0.851344in}}%
\pgfpathlineto{\pgfqpoint{4.953122in}{0.852721in}}%
\pgfpathlineto{\pgfqpoint{4.954243in}{0.852595in}}%
\pgfpathlineto{\pgfqpoint{4.955363in}{0.867991in}}%
\pgfpathlineto{\pgfqpoint{4.959845in}{0.872121in}}%
\pgfpathlineto{\pgfqpoint{4.960966in}{0.863860in}}%
\pgfpathlineto{\pgfqpoint{4.962086in}{0.867615in}}%
\pgfpathlineto{\pgfqpoint{4.963207in}{0.878504in}}%
\pgfpathlineto{\pgfqpoint{4.966568in}{0.885388in}}%
\pgfpathlineto{\pgfqpoint{4.967689in}{0.879380in}}%
\pgfpathlineto{\pgfqpoint{4.968809in}{0.877878in}}%
\pgfpathlineto{\pgfqpoint{4.971051in}{0.886390in}}%
\pgfpathlineto{\pgfqpoint{4.974412in}{0.870744in}}%
\pgfpathlineto{\pgfqpoint{4.975533in}{0.896778in}}%
\pgfpathlineto{\pgfqpoint{4.977774in}{0.915678in}}%
\pgfpathlineto{\pgfqpoint{4.978894in}{0.902536in}}%
\pgfpathlineto{\pgfqpoint{4.982256in}{0.899782in}}%
\pgfpathlineto{\pgfqpoint{4.983376in}{0.889143in}}%
\pgfpathlineto{\pgfqpoint{4.984497in}{0.892898in}}%
\pgfpathlineto{\pgfqpoint{4.985617in}{0.876126in}}%
\pgfpathlineto{\pgfqpoint{4.986738in}{0.877753in}}%
\pgfpathlineto{\pgfqpoint{4.991220in}{0.900032in}}%
\pgfpathlineto{\pgfqpoint{4.992341in}{0.883010in}}%
\pgfpathlineto{\pgfqpoint{4.993461in}{0.886390in}}%
\pgfpathlineto{\pgfqpoint{4.994582in}{0.881008in}}%
\pgfpathlineto{\pgfqpoint{4.997943in}{0.870994in}}%
\pgfpathlineto{\pgfqpoint{4.999064in}{0.871745in}}%
\pgfpathlineto{\pgfqpoint{5.000184in}{0.860981in}}%
\pgfpathlineto{\pgfqpoint{5.001305in}{0.859354in}}%
\pgfpathlineto{\pgfqpoint{5.002425in}{0.865237in}}%
\pgfpathlineto{\pgfqpoint{5.005787in}{0.877753in}}%
\pgfpathlineto{\pgfqpoint{5.006907in}{0.895151in}}%
\pgfpathlineto{\pgfqpoint{5.008028in}{0.896277in}}%
\pgfpathlineto{\pgfqpoint{5.009148in}{0.896528in}}%
\pgfpathlineto{\pgfqpoint{5.010269in}{0.881884in}}%
\pgfpathlineto{\pgfqpoint{5.013631in}{0.866864in}}%
\pgfpathlineto{\pgfqpoint{5.014751in}{0.871996in}}%
\pgfpathlineto{\pgfqpoint{5.015872in}{0.882384in}}%
\pgfpathlineto{\pgfqpoint{5.016992in}{0.848090in}}%
\pgfpathlineto{\pgfqpoint{5.018113in}{0.842332in}}%
\pgfpathlineto{\pgfqpoint{5.021474in}{0.849341in}}%
\pgfpathlineto{\pgfqpoint{5.022595in}{0.849842in}}%
\pgfpathlineto{\pgfqpoint{5.024836in}{0.876877in}}%
\pgfpathlineto{\pgfqpoint{5.029318in}{0.866614in}}%
\pgfpathlineto{\pgfqpoint{5.030438in}{0.870619in}}%
\pgfpathlineto{\pgfqpoint{5.032680in}{0.872246in}}%
\pgfpathlineto{\pgfqpoint{5.033800in}{0.856476in}}%
\pgfpathlineto{\pgfqpoint{5.037162in}{0.851219in}}%
\pgfpathlineto{\pgfqpoint{5.039403in}{0.869367in}}%
\pgfpathlineto{\pgfqpoint{5.040523in}{0.871871in}}%
\pgfpathlineto{\pgfqpoint{5.041644in}{0.880507in}}%
\pgfpathlineto{\pgfqpoint{5.045005in}{0.891146in}}%
\pgfpathlineto{\pgfqpoint{5.046126in}{0.888893in}}%
\pgfpathlineto{\pgfqpoint{5.047246in}{0.880257in}}%
\pgfpathlineto{\pgfqpoint{5.048367in}{0.895526in}}%
\pgfpathlineto{\pgfqpoint{5.049487in}{0.899532in}}%
\pgfpathlineto{\pgfqpoint{5.052849in}{0.905039in}}%
\pgfpathlineto{\pgfqpoint{5.053970in}{0.901034in}}%
\pgfpathlineto{\pgfqpoint{5.056211in}{0.878755in}}%
\pgfpathlineto{\pgfqpoint{5.060693in}{0.889143in}}%
\pgfpathlineto{\pgfqpoint{5.061813in}{0.889644in}}%
\pgfpathlineto{\pgfqpoint{5.062934in}{0.902285in}}%
\pgfpathlineto{\pgfqpoint{5.064054in}{0.904163in}}%
\pgfpathlineto{\pgfqpoint{5.065175in}{0.915427in}}%
\pgfpathlineto{\pgfqpoint{5.069657in}{0.920309in}}%
\pgfpathlineto{\pgfqpoint{5.070777in}{0.918682in}}%
\pgfpathlineto{\pgfqpoint{5.071898in}{0.923688in}}%
\pgfpathlineto{\pgfqpoint{5.073019in}{0.923438in}}%
\pgfpathlineto{\pgfqpoint{5.076380in}{0.926442in}}%
\pgfpathlineto{\pgfqpoint{5.077501in}{0.921936in}}%
\pgfpathlineto{\pgfqpoint{5.078621in}{0.926066in}}%
\pgfpathlineto{\pgfqpoint{5.079742in}{0.932324in}}%
\pgfpathlineto{\pgfqpoint{5.080862in}{0.930322in}}%
\pgfpathlineto{\pgfqpoint{5.084224in}{0.940460in}}%
\pgfpathlineto{\pgfqpoint{5.086465in}{0.926942in}}%
\pgfpathlineto{\pgfqpoint{5.087585in}{0.909420in}}%
\pgfpathlineto{\pgfqpoint{5.088706in}{0.909420in}}%
\pgfpathlineto{\pgfqpoint{5.092067in}{0.915177in}}%
\pgfpathlineto{\pgfqpoint{5.093188in}{0.913550in}}%
\pgfpathlineto{\pgfqpoint{5.094309in}{0.918431in}}%
\pgfpathlineto{\pgfqpoint{5.095429in}{0.920684in}}%
\pgfpathlineto{\pgfqpoint{5.096550in}{0.909670in}}%
\pgfpathlineto{\pgfqpoint{5.099911in}{0.908919in}}%
\pgfpathlineto{\pgfqpoint{5.101032in}{0.908043in}}%
\pgfpathlineto{\pgfqpoint{5.102152in}{0.922186in}}%
\pgfpathlineto{\pgfqpoint{5.103273in}{0.928570in}}%
\pgfpathlineto{\pgfqpoint{5.104393in}{0.938332in}}%
\pgfpathlineto{\pgfqpoint{5.107755in}{0.943464in}}%
\pgfpathlineto{\pgfqpoint{5.108875in}{0.951975in}}%
\pgfpathlineto{\pgfqpoint{5.109996in}{0.951600in}}%
\pgfpathlineto{\pgfqpoint{5.111116in}{0.956231in}}%
\pgfpathlineto{\pgfqpoint{5.115599in}{0.953352in}}%
\pgfpathlineto{\pgfqpoint{5.116719in}{0.944841in}}%
\pgfpathlineto{\pgfqpoint{5.117840in}{0.960611in}}%
\pgfpathlineto{\pgfqpoint{5.118960in}{0.956481in}}%
\pgfpathlineto{\pgfqpoint{5.120081in}{0.956856in}}%
\pgfpathlineto{\pgfqpoint{5.123442in}{0.955354in}}%
\pgfpathlineto{\pgfqpoint{5.126804in}{0.935203in}}%
\pgfpathlineto{\pgfqpoint{5.127924in}{0.944340in}}%
\pgfpathlineto{\pgfqpoint{5.131286in}{0.943464in}}%
\pgfpathlineto{\pgfqpoint{5.132406in}{0.948971in}}%
\pgfpathlineto{\pgfqpoint{5.133527in}{0.946969in}}%
\pgfpathlineto{\pgfqpoint{5.134647in}{0.955730in}}%
\pgfpathlineto{\pgfqpoint{5.135768in}{0.949096in}}%
\pgfpathlineto{\pgfqpoint{5.139130in}{0.959860in}}%
\pgfpathlineto{\pgfqpoint{5.140250in}{0.948846in}}%
\pgfpathlineto{\pgfqpoint{5.141371in}{0.960486in}}%
\pgfpathlineto{\pgfqpoint{5.142491in}{0.945091in}}%
\pgfpathlineto{\pgfqpoint{5.143612in}{0.939334in}}%
\pgfpathlineto{\pgfqpoint{5.146973in}{0.960862in}}%
\pgfpathlineto{\pgfqpoint{5.148094in}{0.955229in}}%
\pgfpathlineto{\pgfqpoint{5.149214in}{0.953352in}}%
\pgfpathlineto{\pgfqpoint{5.150335in}{0.940836in}}%
\pgfpathlineto{\pgfqpoint{5.151455in}{0.956356in}}%
\pgfpathlineto{\pgfqpoint{5.154817in}{0.963866in}}%
\pgfpathlineto{\pgfqpoint{5.155938in}{0.960486in}}%
\pgfpathlineto{\pgfqpoint{5.157058in}{0.964992in}}%
\pgfpathlineto{\pgfqpoint{5.159299in}{0.984517in}}%
\pgfpathlineto{\pgfqpoint{5.163781in}{0.993905in}}%
\pgfpathlineto{\pgfqpoint{5.164902in}{0.991026in}}%
\pgfpathlineto{\pgfqpoint{5.166022in}{0.997284in}}%
\pgfpathlineto{\pgfqpoint{5.167143in}{0.998536in}}%
\pgfpathlineto{\pgfqpoint{5.170504in}{0.997284in}}%
\pgfpathlineto{\pgfqpoint{5.171625in}{0.993029in}}%
\pgfpathlineto{\pgfqpoint{5.172745in}{0.997159in}}%
\pgfpathlineto{\pgfqpoint{5.173866in}{0.995532in}}%
\pgfpathlineto{\pgfqpoint{5.174986in}{0.991401in}}%
\pgfpathlineto{\pgfqpoint{5.180589in}{1.003667in}}%
\pgfpathlineto{\pgfqpoint{5.181710in}{0.994405in}}%
\pgfpathlineto{\pgfqpoint{5.182830in}{1.003667in}}%
\pgfpathlineto{\pgfqpoint{5.186192in}{0.999287in}}%
\pgfpathlineto{\pgfqpoint{5.187312in}{0.988147in}}%
\pgfpathlineto{\pgfqpoint{5.188433in}{0.987271in}}%
\pgfpathlineto{\pgfqpoint{5.189553in}{0.992528in}}%
\pgfpathlineto{\pgfqpoint{5.190674in}{0.989023in}}%
\pgfpathlineto{\pgfqpoint{5.196276in}{0.996658in}}%
\pgfpathlineto{\pgfqpoint{5.197397in}{0.996783in}}%
\pgfpathlineto{\pgfqpoint{5.198518in}{0.998536in}}%
\pgfpathlineto{\pgfqpoint{5.201879in}{0.984643in}}%
\pgfpathlineto{\pgfqpoint{5.203000in}{0.973253in}}%
\pgfpathlineto{\pgfqpoint{5.204120in}{0.986520in}}%
\pgfpathlineto{\pgfqpoint{5.205241in}{0.970499in}}%
\pgfpathlineto{\pgfqpoint{5.206361in}{0.978259in}}%
\pgfpathlineto{\pgfqpoint{5.209723in}{0.979261in}}%
\pgfpathlineto{\pgfqpoint{5.210843in}{0.981639in}}%
\pgfpathlineto{\pgfqpoint{5.211964in}{0.963740in}}%
\pgfpathlineto{\pgfqpoint{5.213084in}{0.955605in}}%
\pgfpathlineto{\pgfqpoint{5.214205in}{0.974880in}}%
\pgfpathlineto{\pgfqpoint{5.217567in}{0.976132in}}%
\pgfpathlineto{\pgfqpoint{5.218687in}{0.959360in}}%
\pgfpathlineto{\pgfqpoint{5.219808in}{0.970875in}}%
\pgfpathlineto{\pgfqpoint{5.220928in}{0.943088in}}%
\pgfpathlineto{\pgfqpoint{5.222049in}{0.949597in}}%
\pgfpathlineto{\pgfqpoint{5.225410in}{0.922311in}}%
\pgfpathlineto{\pgfqpoint{5.226531in}{0.924940in}}%
\pgfpathlineto{\pgfqpoint{5.227651in}{0.901034in}}%
\pgfpathlineto{\pgfqpoint{5.228772in}{0.897154in}}%
\pgfpathlineto{\pgfqpoint{5.229892in}{0.921310in}}%
\pgfpathlineto{\pgfqpoint{5.233254in}{0.940961in}}%
\pgfpathlineto{\pgfqpoint{5.234374in}{0.964366in}}%
\pgfpathlineto{\pgfqpoint{5.235495in}{0.959235in}}%
\pgfpathlineto{\pgfqpoint{5.237736in}{0.976757in}}%
\pgfpathlineto{\pgfqpoint{5.241098in}{0.974880in}}%
\pgfpathlineto{\pgfqpoint{5.242218in}{0.992027in}}%
\pgfpathlineto{\pgfqpoint{5.243339in}{0.987396in}}%
\pgfpathlineto{\pgfqpoint{5.244459in}{0.995407in}}%
\pgfpathlineto{\pgfqpoint{5.245580in}{1.008799in}}%
\pgfpathlineto{\pgfqpoint{5.248941in}{1.012679in}}%
\pgfpathlineto{\pgfqpoint{5.250062in}{0.996909in}}%
\pgfpathlineto{\pgfqpoint{5.251182in}{1.004418in}}%
\pgfpathlineto{\pgfqpoint{5.252303in}{1.016059in}}%
\pgfpathlineto{\pgfqpoint{5.253423in}{0.992778in}}%
\pgfpathlineto{\pgfqpoint{5.256785in}{0.990275in}}%
\pgfpathlineto{\pgfqpoint{5.257905in}{0.992653in}}%
\pgfpathlineto{\pgfqpoint{5.259026in}{0.991652in}}%
\pgfpathlineto{\pgfqpoint{5.261267in}{1.002165in}}%
\pgfpathlineto{\pgfqpoint{5.265749in}{0.996032in}}%
\pgfpathlineto{\pgfqpoint{5.266870in}{0.990776in}}%
\pgfpathlineto{\pgfqpoint{5.267990in}{0.980137in}}%
\pgfpathlineto{\pgfqpoint{5.269111in}{0.980763in}}%
\pgfpathlineto{\pgfqpoint{5.272472in}{0.999537in}}%
\pgfpathlineto{\pgfqpoint{5.273593in}{1.011928in}}%
\pgfpathlineto{\pgfqpoint{5.276954in}{1.021941in}}%
\pgfpathlineto{\pgfqpoint{5.280316in}{1.024069in}}%
\pgfpathlineto{\pgfqpoint{5.281437in}{1.033081in}}%
\pgfpathlineto{\pgfqpoint{5.282557in}{1.028825in}}%
\pgfpathlineto{\pgfqpoint{5.283678in}{1.030327in}}%
\pgfpathlineto{\pgfqpoint{5.284798in}{1.036460in}}%
\pgfpathlineto{\pgfqpoint{5.288160in}{1.036836in}}%
\pgfpathlineto{\pgfqpoint{5.289280in}{1.026948in}}%
\pgfpathlineto{\pgfqpoint{5.290401in}{1.011678in}}%
\pgfpathlineto{\pgfqpoint{5.291521in}{1.026572in}}%
\pgfpathlineto{\pgfqpoint{5.292642in}{1.023568in}}%
\pgfpathlineto{\pgfqpoint{5.296003in}{1.016684in}}%
\pgfpathlineto{\pgfqpoint{5.297124in}{1.008048in}}%
\pgfpathlineto{\pgfqpoint{5.299365in}{1.036711in}}%
\pgfpathlineto{\pgfqpoint{5.300486in}{1.039965in}}%
\pgfpathlineto{\pgfqpoint{5.304968in}{1.061117in}}%
\pgfpathlineto{\pgfqpoint{5.306088in}{1.057988in}}%
\pgfpathlineto{\pgfqpoint{5.308329in}{1.065123in}}%
\pgfpathlineto{\pgfqpoint{5.311691in}{1.070630in}}%
\pgfpathlineto{\pgfqpoint{5.313932in}{1.055235in}}%
\pgfpathlineto{\pgfqpoint{5.316173in}{1.050103in}}%
\pgfpathlineto{\pgfqpoint{5.320655in}{1.028325in}}%
\pgfpathlineto{\pgfqpoint{5.324017in}{1.055986in}}%
\pgfpathlineto{\pgfqpoint{5.327378in}{1.058489in}}%
\pgfpathlineto{\pgfqpoint{5.328499in}{1.067000in}}%
\pgfpathlineto{\pgfqpoint{5.329619in}{1.055735in}}%
\pgfpathlineto{\pgfqpoint{5.330740in}{1.057112in}}%
\pgfpathlineto{\pgfqpoint{5.331860in}{1.066875in}}%
\pgfpathlineto{\pgfqpoint{5.336342in}{1.061743in}}%
\pgfpathlineto{\pgfqpoint{5.337463in}{1.054984in}}%
\pgfpathlineto{\pgfqpoint{5.338583in}{1.066499in}}%
\pgfpathlineto{\pgfqpoint{5.339704in}{1.061493in}}%
\pgfpathlineto{\pgfqpoint{5.343066in}{1.064372in}}%
\pgfpathlineto{\pgfqpoint{5.345307in}{1.037462in}}%
\pgfpathlineto{\pgfqpoint{5.346427in}{1.043845in}}%
\pgfpathlineto{\pgfqpoint{5.347548in}{1.017310in}}%
\pgfpathlineto{\pgfqpoint{5.350909in}{1.028700in}}%
\pgfpathlineto{\pgfqpoint{5.352030in}{1.054233in}}%
\pgfpathlineto{\pgfqpoint{5.353150in}{1.138468in}}%
\pgfpathlineto{\pgfqpoint{5.354271in}{1.154364in}}%
\pgfpathlineto{\pgfqpoint{5.355391in}{1.147104in}}%
\pgfpathlineto{\pgfqpoint{5.358753in}{1.143725in}}%
\pgfpathlineto{\pgfqpoint{5.359873in}{1.145978in}}%
\pgfpathlineto{\pgfqpoint{5.360994in}{1.145352in}}%
\pgfpathlineto{\pgfqpoint{5.362115in}{1.165503in}}%
\pgfpathlineto{\pgfqpoint{5.363235in}{1.172387in}}%
\pgfpathlineto{\pgfqpoint{5.367717in}{1.171762in}}%
\pgfpathlineto{\pgfqpoint{5.368838in}{1.168758in}}%
\pgfpathlineto{\pgfqpoint{5.369958in}{1.169133in}}%
\pgfpathlineto{\pgfqpoint{5.371079in}{1.176893in}}%
\pgfpathlineto{\pgfqpoint{5.374440in}{1.182025in}}%
\pgfpathlineto{\pgfqpoint{5.375561in}{1.178270in}}%
\pgfpathlineto{\pgfqpoint{5.376681in}{1.188784in}}%
\pgfpathlineto{\pgfqpoint{5.377802in}{1.177018in}}%
\pgfpathlineto{\pgfqpoint{5.378922in}{1.171386in}}%
\pgfpathlineto{\pgfqpoint{5.383405in}{1.197921in}}%
\pgfpathlineto{\pgfqpoint{5.386766in}{1.168257in}}%
\pgfpathlineto{\pgfqpoint{5.390128in}{1.185154in}}%
\pgfpathlineto{\pgfqpoint{5.391248in}{1.159746in}}%
\pgfpathlineto{\pgfqpoint{5.392369in}{1.157368in}}%
\pgfpathlineto{\pgfqpoint{5.393489in}{1.207558in}}%
\pgfpathlineto{\pgfqpoint{5.394610in}{1.199047in}}%
\pgfpathlineto{\pgfqpoint{5.397971in}{1.209936in}}%
\pgfpathlineto{\pgfqpoint{5.399092in}{1.205180in}}%
\pgfpathlineto{\pgfqpoint{5.400212in}{1.216946in}}%
\pgfpathlineto{\pgfqpoint{5.401333in}{1.209936in}}%
\pgfpathlineto{\pgfqpoint{5.402453in}{1.222328in}}%
\pgfpathlineto{\pgfqpoint{5.405815in}{1.219949in}}%
\pgfpathlineto{\pgfqpoint{5.406936in}{1.206932in}}%
\pgfpathlineto{\pgfqpoint{5.408056in}{1.182150in}}%
\pgfpathlineto{\pgfqpoint{5.413659in}{1.195292in}}%
\pgfpathlineto{\pgfqpoint{5.414779in}{1.180773in}}%
\pgfpathlineto{\pgfqpoint{5.417020in}{1.193915in}}%
\pgfpathlineto{\pgfqpoint{5.421502in}{1.189535in}}%
\pgfpathlineto{\pgfqpoint{5.422623in}{1.187157in}}%
\pgfpathlineto{\pgfqpoint{5.423744in}{1.198046in}}%
\pgfpathlineto{\pgfqpoint{5.424864in}{1.202927in}}%
\pgfpathlineto{\pgfqpoint{5.425985in}{1.205055in}}%
\pgfpathlineto{\pgfqpoint{5.429346in}{1.199798in}}%
\pgfpathlineto{\pgfqpoint{5.430467in}{1.201550in}}%
\pgfpathlineto{\pgfqpoint{5.431587in}{1.205305in}}%
\pgfpathlineto{\pgfqpoint{5.432708in}{1.218447in}}%
\pgfpathlineto{\pgfqpoint{5.433828in}{1.201926in}}%
\pgfpathlineto{\pgfqpoint{5.437190in}{1.219949in}}%
\pgfpathlineto{\pgfqpoint{5.438310in}{1.213566in}}%
\pgfpathlineto{\pgfqpoint{5.439431in}{1.216570in}}%
\pgfpathlineto{\pgfqpoint{5.440551in}{1.229086in}}%
\pgfpathlineto{\pgfqpoint{5.441672in}{1.235345in}}%
\pgfpathlineto{\pgfqpoint{5.445034in}{1.242729in}}%
\pgfpathlineto{\pgfqpoint{5.446154in}{1.239850in}}%
\pgfpathlineto{\pgfqpoint{5.447275in}{1.238599in}}%
\pgfpathlineto{\pgfqpoint{5.448395in}{1.225832in}}%
\pgfpathlineto{\pgfqpoint{5.449516in}{1.246860in}}%
\pgfpathlineto{\pgfqpoint{5.452877in}{1.252867in}}%
\pgfpathlineto{\pgfqpoint{5.453998in}{1.250364in}}%
\pgfpathlineto{\pgfqpoint{5.455118in}{1.237472in}}%
\pgfpathlineto{\pgfqpoint{5.456239in}{1.232090in}}%
\pgfpathlineto{\pgfqpoint{5.457359in}{1.242103in}}%
\pgfpathlineto{\pgfqpoint{5.460721in}{1.224330in}}%
\pgfpathlineto{\pgfqpoint{5.461841in}{1.231840in}}%
\pgfpathlineto{\pgfqpoint{5.462962in}{1.231339in}}%
\pgfpathlineto{\pgfqpoint{5.465203in}{1.244356in}}%
\pgfpathlineto{\pgfqpoint{5.468565in}{1.244732in}}%
\pgfpathlineto{\pgfqpoint{5.469685in}{1.247360in}}%
\pgfpathlineto{\pgfqpoint{5.470806in}{1.243105in}}%
\pgfpathlineto{\pgfqpoint{5.471926in}{1.245358in}}%
\pgfpathlineto{\pgfqpoint{5.473047in}{1.243856in}}%
\pgfpathlineto{\pgfqpoint{5.477529in}{1.234218in}}%
\pgfpathlineto{\pgfqpoint{5.478649in}{1.245107in}}%
\pgfpathlineto{\pgfqpoint{5.479770in}{1.246985in}}%
\pgfpathlineto{\pgfqpoint{5.480890in}{1.245107in}}%
\pgfpathlineto{\pgfqpoint{5.484252in}{1.252116in}}%
\pgfpathlineto{\pgfqpoint{5.485372in}{1.249613in}}%
\pgfpathlineto{\pgfqpoint{5.486493in}{1.254494in}}%
\pgfpathlineto{\pgfqpoint{5.487614in}{1.244356in}}%
\pgfpathlineto{\pgfqpoint{5.488734in}{1.244356in}}%
\pgfpathlineto{\pgfqpoint{5.492096in}{1.232466in}}%
\pgfpathlineto{\pgfqpoint{5.493216in}{1.223454in}}%
\pgfpathlineto{\pgfqpoint{5.494337in}{1.240852in}}%
\pgfpathlineto{\pgfqpoint{5.495457in}{1.248111in}}%
\pgfpathlineto{\pgfqpoint{5.496578in}{1.240226in}}%
\pgfpathlineto{\pgfqpoint{5.499939in}{1.242854in}}%
\pgfpathlineto{\pgfqpoint{5.501060in}{1.253243in}}%
\pgfpathlineto{\pgfqpoint{5.502180in}{1.258249in}}%
\pgfpathlineto{\pgfqpoint{5.503301in}{1.278526in}}%
\pgfpathlineto{\pgfqpoint{5.504421in}{1.271517in}}%
\pgfpathlineto{\pgfqpoint{5.507783in}{1.282281in}}%
\pgfpathlineto{\pgfqpoint{5.508904in}{1.292544in}}%
\pgfpathlineto{\pgfqpoint{5.510024in}{1.285034in}}%
\pgfpathlineto{\pgfqpoint{5.512265in}{1.299303in}}%
\pgfpathlineto{\pgfqpoint{5.515627in}{1.276648in}}%
\pgfpathlineto{\pgfqpoint{5.516747in}{1.289415in}}%
\pgfpathlineto{\pgfqpoint{5.517868in}{1.308815in}}%
\pgfpathlineto{\pgfqpoint{5.518988in}{1.306938in}}%
\pgfpathlineto{\pgfqpoint{5.523470in}{1.315574in}}%
\pgfpathlineto{\pgfqpoint{5.524591in}{1.332096in}}%
\pgfpathlineto{\pgfqpoint{5.525711in}{1.309566in}}%
\pgfpathlineto{\pgfqpoint{5.526832in}{1.314448in}}%
\pgfpathlineto{\pgfqpoint{5.527953in}{1.324336in}}%
\pgfpathlineto{\pgfqpoint{5.531314in}{1.343360in}}%
\pgfpathlineto{\pgfqpoint{5.532435in}{1.340982in}}%
\pgfpathlineto{\pgfqpoint{5.533555in}{1.346239in}}%
\pgfpathlineto{\pgfqpoint{5.534676in}{1.355376in}}%
\pgfpathlineto{\pgfqpoint{5.535796in}{1.352873in}}%
\pgfpathlineto{\pgfqpoint{5.539158in}{1.361384in}}%
\pgfpathlineto{\pgfqpoint{5.540278in}{1.358255in}}%
\pgfpathlineto{\pgfqpoint{5.541399in}{1.358380in}}%
\pgfpathlineto{\pgfqpoint{5.542519in}{1.352122in}}%
\pgfpathlineto{\pgfqpoint{5.543640in}{1.353499in}}%
\pgfpathlineto{\pgfqpoint{5.547001in}{1.345738in}}%
\pgfpathlineto{\pgfqpoint{5.548122in}{1.348117in}}%
\pgfpathlineto{\pgfqpoint{5.549243in}{1.364513in}}%
\pgfpathlineto{\pgfqpoint{5.550363in}{1.366641in}}%
\pgfpathlineto{\pgfqpoint{5.551484in}{1.366390in}}%
\pgfpathlineto{\pgfqpoint{5.554845in}{1.379533in}}%
\pgfpathlineto{\pgfqpoint{5.555966in}{1.386291in}}%
\pgfpathlineto{\pgfqpoint{5.557086in}{1.254620in}}%
\pgfpathlineto{\pgfqpoint{5.558207in}{1.231214in}}%
\pgfpathlineto{\pgfqpoint{5.559327in}{1.240601in}}%
\pgfpathlineto{\pgfqpoint{5.562689in}{1.260127in}}%
\pgfpathlineto{\pgfqpoint{5.563809in}{1.224706in}}%
\pgfpathlineto{\pgfqpoint{5.564930in}{1.212815in}}%
\pgfpathlineto{\pgfqpoint{5.566050in}{1.219073in}}%
\pgfpathlineto{\pgfqpoint{5.567171in}{1.214818in}}%
\pgfpathlineto{\pgfqpoint{5.570533in}{1.237097in}}%
\pgfpathlineto{\pgfqpoint{5.571653in}{1.212189in}}%
\pgfpathlineto{\pgfqpoint{5.572774in}{1.206432in}}%
\pgfpathlineto{\pgfqpoint{5.573894in}{1.130458in}}%
\pgfpathlineto{\pgfqpoint{5.578376in}{1.075511in}}%
\pgfpathlineto{\pgfqpoint{5.579497in}{1.081769in}}%
\pgfpathlineto{\pgfqpoint{5.581738in}{1.155866in}}%
\pgfpathlineto{\pgfqpoint{5.582858in}{1.159496in}}%
\pgfpathlineto{\pgfqpoint{5.586220in}{1.152486in}}%
\pgfpathlineto{\pgfqpoint{5.587340in}{1.124450in}}%
\pgfpathlineto{\pgfqpoint{5.588461in}{1.152612in}}%
\pgfpathlineto{\pgfqpoint{5.589582in}{1.153738in}}%
\pgfpathlineto{\pgfqpoint{5.590702in}{1.141722in}}%
\pgfpathlineto{\pgfqpoint{5.595184in}{1.177644in}}%
\pgfpathlineto{\pgfqpoint{5.596305in}{1.152862in}}%
\pgfpathlineto{\pgfqpoint{5.597425in}{1.160998in}}%
\pgfpathlineto{\pgfqpoint{5.598546in}{1.183151in}}%
\pgfpathlineto{\pgfqpoint{5.601907in}{1.175391in}}%
\pgfpathlineto{\pgfqpoint{5.603028in}{1.170760in}}%
\pgfpathlineto{\pgfqpoint{5.604148in}{1.177018in}}%
\pgfpathlineto{\pgfqpoint{5.605269in}{1.179897in}}%
\pgfpathlineto{\pgfqpoint{5.606389in}{1.163751in}}%
\pgfpathlineto{\pgfqpoint{5.609751in}{1.170510in}}%
\pgfpathlineto{\pgfqpoint{5.613113in}{1.137592in}}%
\pgfpathlineto{\pgfqpoint{5.614233in}{1.133837in}}%
\pgfpathlineto{\pgfqpoint{5.617595in}{1.112434in}}%
\pgfpathlineto{\pgfqpoint{5.618715in}{1.123449in}}%
\pgfpathlineto{\pgfqpoint{5.619836in}{1.156241in}}%
\pgfpathlineto{\pgfqpoint{5.622077in}{1.165629in}}%
\pgfpathlineto{\pgfqpoint{5.625438in}{1.175767in}}%
\pgfpathlineto{\pgfqpoint{5.626559in}{1.174766in}}%
\pgfpathlineto{\pgfqpoint{5.627679in}{1.170260in}}%
\pgfpathlineto{\pgfqpoint{5.629921in}{1.195918in}}%
\pgfpathlineto{\pgfqpoint{5.634403in}{1.208059in}}%
\pgfpathlineto{\pgfqpoint{5.635523in}{1.197921in}}%
\pgfpathlineto{\pgfqpoint{5.636644in}{1.223454in}}%
\pgfpathlineto{\pgfqpoint{5.637764in}{1.227584in}}%
\pgfpathlineto{\pgfqpoint{5.643367in}{1.249363in}}%
\pgfpathlineto{\pgfqpoint{5.644487in}{1.286661in}}%
\pgfpathlineto{\pgfqpoint{5.645608in}{1.284784in}}%
\pgfpathlineto{\pgfqpoint{5.648969in}{1.289916in}}%
\pgfpathlineto{\pgfqpoint{5.650090in}{1.292794in}}%
\pgfpathlineto{\pgfqpoint{5.652331in}{1.307814in}}%
\pgfpathlineto{\pgfqpoint{5.653452in}{1.292419in}}%
\pgfpathlineto{\pgfqpoint{5.657934in}{1.313697in}}%
\pgfpathlineto{\pgfqpoint{5.659054in}{1.286661in}}%
\pgfpathlineto{\pgfqpoint{5.660175in}{1.283658in}}%
\pgfpathlineto{\pgfqpoint{5.661295in}{1.315199in}}%
\pgfpathlineto{\pgfqpoint{5.664657in}{1.324085in}}%
\pgfpathlineto{\pgfqpoint{5.665777in}{1.335851in}}%
\pgfpathlineto{\pgfqpoint{5.666898in}{1.325212in}}%
\pgfpathlineto{\pgfqpoint{5.668018in}{1.321582in}}%
\pgfpathlineto{\pgfqpoint{5.669139in}{1.305436in}}%
\pgfpathlineto{\pgfqpoint{5.673621in}{1.320706in}}%
\pgfpathlineto{\pgfqpoint{5.674742in}{1.344362in}}%
\pgfpathlineto{\pgfqpoint{5.675862in}{1.351120in}}%
\pgfpathlineto{\pgfqpoint{5.676983in}{1.367141in}}%
\pgfpathlineto{\pgfqpoint{5.680344in}{1.359506in}}%
\pgfpathlineto{\pgfqpoint{5.681465in}{1.342109in}}%
\pgfpathlineto{\pgfqpoint{5.682585in}{1.350620in}}%
\pgfpathlineto{\pgfqpoint{5.684826in}{1.308815in}}%
\pgfpathlineto{\pgfqpoint{5.688188in}{1.289290in}}%
\pgfpathlineto{\pgfqpoint{5.689308in}{1.311944in}}%
\pgfpathlineto{\pgfqpoint{5.690429in}{1.295548in}}%
\pgfpathlineto{\pgfqpoint{5.691549in}{1.270641in}}%
\pgfpathlineto{\pgfqpoint{5.692670in}{1.298302in}}%
\pgfpathlineto{\pgfqpoint{5.696032in}{1.293545in}}%
\pgfpathlineto{\pgfqpoint{5.698273in}{1.265634in}}%
\pgfpathlineto{\pgfqpoint{5.699393in}{1.265634in}}%
\pgfpathlineto{\pgfqpoint{5.700514in}{1.233342in}}%
\pgfpathlineto{\pgfqpoint{5.703875in}{1.248862in}}%
\pgfpathlineto{\pgfqpoint{5.706116in}{1.301681in}}%
\pgfpathlineto{\pgfqpoint{5.707237in}{1.280528in}}%
\pgfpathlineto{\pgfqpoint{5.708357in}{1.229587in}}%
\pgfpathlineto{\pgfqpoint{5.711719in}{1.216069in}}%
\pgfpathlineto{\pgfqpoint{5.712840in}{1.217947in}}%
\pgfpathlineto{\pgfqpoint{5.713960in}{1.203929in}}%
\pgfpathlineto{\pgfqpoint{5.715081in}{1.207433in}}%
\pgfpathlineto{\pgfqpoint{5.719563in}{1.223955in}}%
\pgfpathlineto{\pgfqpoint{5.720683in}{1.221952in}}%
\pgfpathlineto{\pgfqpoint{5.721804in}{1.213191in}}%
\pgfpathlineto{\pgfqpoint{5.722924in}{1.198171in}}%
\pgfpathlineto{\pgfqpoint{5.727406in}{1.173264in}}%
\pgfpathlineto{\pgfqpoint{5.728527in}{1.148481in}}%
\pgfpathlineto{\pgfqpoint{5.729647in}{1.142098in}}%
\pgfpathlineto{\pgfqpoint{5.730768in}{1.131960in}}%
\pgfpathlineto{\pgfqpoint{5.731888in}{1.128956in}}%
\pgfpathlineto{\pgfqpoint{5.735250in}{1.136841in}}%
\pgfpathlineto{\pgfqpoint{5.736371in}{1.155240in}}%
\pgfpathlineto{\pgfqpoint{5.737491in}{1.119819in}}%
\pgfpathlineto{\pgfqpoint{5.738612in}{1.127204in}}%
\pgfpathlineto{\pgfqpoint{5.739732in}{1.065373in}}%
\pgfpathlineto{\pgfqpoint{5.744214in}{1.066249in}}%
\pgfpathlineto{\pgfqpoint{5.745335in}{1.049227in}}%
\pgfpathlineto{\pgfqpoint{5.746455in}{1.066750in}}%
\pgfpathlineto{\pgfqpoint{5.747576in}{1.101044in}}%
\pgfpathlineto{\pgfqpoint{5.750937in}{1.081894in}}%
\pgfpathlineto{\pgfqpoint{5.752058in}{1.093535in}}%
\pgfpathlineto{\pgfqpoint{5.753178in}{1.070379in}}%
\pgfpathlineto{\pgfqpoint{5.754299in}{1.060992in}}%
\pgfpathlineto{\pgfqpoint{5.755420in}{1.088153in}}%
\pgfpathlineto{\pgfqpoint{5.758781in}{1.080267in}}%
\pgfpathlineto{\pgfqpoint{5.759902in}{1.056111in}}%
\pgfpathlineto{\pgfqpoint{5.761022in}{1.080142in}}%
\pgfpathlineto{\pgfqpoint{5.762143in}{1.083522in}}%
\pgfpathlineto{\pgfqpoint{5.763263in}{1.065373in}}%
\pgfpathlineto{\pgfqpoint{5.766625in}{1.044220in}}%
\pgfpathlineto{\pgfqpoint{5.767745in}{1.046598in}}%
\pgfpathlineto{\pgfqpoint{5.768866in}{1.005420in}}%
\pgfpathlineto{\pgfqpoint{5.771107in}{1.032705in}}%
\pgfpathlineto{\pgfqpoint{5.775589in}{1.053608in}}%
\pgfpathlineto{\pgfqpoint{5.776710in}{1.084398in}}%
\pgfpathlineto{\pgfqpoint{5.778951in}{1.078515in}}%
\pgfpathlineto{\pgfqpoint{5.782312in}{1.094661in}}%
\pgfpathlineto{\pgfqpoint{5.783433in}{1.082896in}}%
\pgfpathlineto{\pgfqpoint{5.784553in}{1.083522in}}%
\pgfpathlineto{\pgfqpoint{5.785674in}{1.086150in}}%
\pgfpathlineto{\pgfqpoint{5.786794in}{1.082145in}}%
\pgfpathlineto{\pgfqpoint{5.790156in}{1.084648in}}%
\pgfpathlineto{\pgfqpoint{5.791276in}{1.109931in}}%
\pgfpathlineto{\pgfqpoint{5.792397in}{1.102171in}}%
\pgfpathlineto{\pgfqpoint{5.793517in}{1.123824in}}%
\pgfpathlineto{\pgfqpoint{5.794638in}{1.119819in}}%
\pgfpathlineto{\pgfqpoint{5.798000in}{1.130583in}}%
\pgfpathlineto{\pgfqpoint{5.799120in}{1.111934in}}%
\pgfpathlineto{\pgfqpoint{5.800241in}{1.110056in}}%
\pgfpathlineto{\pgfqpoint{5.801361in}{1.102672in}}%
\pgfpathlineto{\pgfqpoint{5.802482in}{1.113310in}}%
\pgfpathlineto{\pgfqpoint{5.805843in}{1.123699in}}%
\pgfpathlineto{\pgfqpoint{5.806964in}{1.116940in}}%
\pgfpathlineto{\pgfqpoint{5.808084in}{1.119318in}}%
\pgfpathlineto{\pgfqpoint{5.809205in}{1.133086in}}%
\pgfpathlineto{\pgfqpoint{5.810325in}{1.128330in}}%
\pgfpathlineto{\pgfqpoint{5.813687in}{1.119569in}}%
\pgfpathlineto{\pgfqpoint{5.815928in}{1.100168in}}%
\pgfpathlineto{\pgfqpoint{5.817049in}{1.104799in}}%
\pgfpathlineto{\pgfqpoint{5.821531in}{1.115188in}}%
\pgfpathlineto{\pgfqpoint{5.822651in}{1.115939in}}%
\pgfpathlineto{\pgfqpoint{5.824892in}{1.129582in}}%
\pgfpathlineto{\pgfqpoint{5.826013in}{1.126828in}}%
\pgfpathlineto{\pgfqpoint{5.829374in}{1.122197in}}%
\pgfpathlineto{\pgfqpoint{5.830495in}{1.102171in}}%
\pgfpathlineto{\pgfqpoint{5.831615in}{1.107928in}}%
\pgfpathlineto{\pgfqpoint{5.832736in}{1.092283in}}%
\pgfpathlineto{\pgfqpoint{5.833856in}{1.095287in}}%
\pgfpathlineto{\pgfqpoint{5.837218in}{1.093535in}}%
\pgfpathlineto{\pgfqpoint{5.838339in}{1.106301in}}%
\pgfpathlineto{\pgfqpoint{5.839459in}{1.131709in}}%
\pgfpathlineto{\pgfqpoint{5.840580in}{1.121571in}}%
\pgfpathlineto{\pgfqpoint{5.841700in}{1.121071in}}%
\pgfpathlineto{\pgfqpoint{5.847303in}{1.176643in}}%
\pgfpathlineto{\pgfqpoint{5.848423in}{1.172387in}}%
\pgfpathlineto{\pgfqpoint{5.849544in}{1.182651in}}%
\pgfpathlineto{\pgfqpoint{5.854026in}{1.195918in}}%
\pgfpathlineto{\pgfqpoint{5.855146in}{1.200549in}}%
\pgfpathlineto{\pgfqpoint{5.857388in}{1.176518in}}%
\pgfpathlineto{\pgfqpoint{5.860749in}{1.189660in}}%
\pgfpathlineto{\pgfqpoint{5.861870in}{1.182776in}}%
\pgfpathlineto{\pgfqpoint{5.862990in}{1.181399in}}%
\pgfpathlineto{\pgfqpoint{5.864111in}{1.196419in}}%
\pgfpathlineto{\pgfqpoint{5.865231in}{1.203678in}}%
\pgfpathlineto{\pgfqpoint{5.868593in}{1.201300in}}%
\pgfpathlineto{\pgfqpoint{5.869713in}{1.216195in}}%
\pgfpathlineto{\pgfqpoint{5.870834in}{1.165003in}}%
\pgfpathlineto{\pgfqpoint{5.871954in}{1.158119in}}%
\pgfpathlineto{\pgfqpoint{5.873075in}{1.143975in}}%
\pgfpathlineto{\pgfqpoint{5.876436in}{1.142098in}}%
\pgfpathlineto{\pgfqpoint{5.877557in}{1.137091in}}%
\pgfpathlineto{\pgfqpoint{5.879798in}{1.118943in}}%
\pgfpathlineto{\pgfqpoint{5.880919in}{1.135214in}}%
\pgfpathlineto{\pgfqpoint{5.884280in}{1.128080in}}%
\pgfpathlineto{\pgfqpoint{5.886521in}{1.136215in}}%
\pgfpathlineto{\pgfqpoint{5.887642in}{1.135589in}}%
\pgfpathlineto{\pgfqpoint{5.888762in}{1.141222in}}%
\pgfpathlineto{\pgfqpoint{5.893244in}{1.128580in}}%
\pgfpathlineto{\pgfqpoint{5.894365in}{1.120194in}}%
\pgfpathlineto{\pgfqpoint{5.895485in}{1.122572in}}%
\pgfpathlineto{\pgfqpoint{5.896606in}{1.122948in}}%
\pgfpathlineto{\pgfqpoint{5.899968in}{1.123323in}}%
\pgfpathlineto{\pgfqpoint{5.902209in}{1.114562in}}%
\pgfpathlineto{\pgfqpoint{5.903329in}{1.112184in}}%
\pgfpathlineto{\pgfqpoint{5.904450in}{1.106176in}}%
\pgfpathlineto{\pgfqpoint{5.907811in}{1.108930in}}%
\pgfpathlineto{\pgfqpoint{5.908932in}{1.118818in}}%
\pgfpathlineto{\pgfqpoint{5.910052in}{1.117316in}}%
\pgfpathlineto{\pgfqpoint{5.911173in}{1.118567in}}%
\pgfpathlineto{\pgfqpoint{5.912293in}{1.125952in}}%
\pgfpathlineto{\pgfqpoint{5.915655in}{1.132711in}}%
\pgfpathlineto{\pgfqpoint{5.916775in}{1.123824in}}%
\pgfpathlineto{\pgfqpoint{5.917896in}{1.123449in}}%
\pgfpathlineto{\pgfqpoint{5.919017in}{1.126202in}}%
\pgfpathlineto{\pgfqpoint{5.920137in}{1.087026in}}%
\pgfpathlineto{\pgfqpoint{5.923499in}{1.071130in}}%
\pgfpathlineto{\pgfqpoint{5.925740in}{1.101921in}}%
\pgfpathlineto{\pgfqpoint{5.926860in}{1.111934in}}%
\pgfpathlineto{\pgfqpoint{5.927981in}{1.114437in}}%
\pgfpathlineto{\pgfqpoint{5.932463in}{1.110056in}}%
\pgfpathlineto{\pgfqpoint{5.935824in}{1.141973in}}%
\pgfpathlineto{\pgfqpoint{5.939186in}{1.146228in}}%
\pgfpathlineto{\pgfqpoint{5.940307in}{1.148857in}}%
\pgfpathlineto{\pgfqpoint{5.941427in}{1.144977in}}%
\pgfpathlineto{\pgfqpoint{5.942548in}{1.146103in}}%
\pgfpathlineto{\pgfqpoint{5.943668in}{1.144101in}}%
\pgfpathlineto{\pgfqpoint{5.947030in}{1.148231in}}%
\pgfpathlineto{\pgfqpoint{5.948150in}{1.140095in}}%
\pgfpathlineto{\pgfqpoint{5.949271in}{1.125201in}}%
\pgfpathlineto{\pgfqpoint{5.951512in}{1.119068in}}%
\pgfpathlineto{\pgfqpoint{5.954873in}{1.115188in}}%
\pgfpathlineto{\pgfqpoint{5.957114in}{1.102546in}}%
\pgfpathlineto{\pgfqpoint{5.958235in}{1.097540in}}%
\pgfpathlineto{\pgfqpoint{5.959355in}{1.098040in}}%
\pgfpathlineto{\pgfqpoint{5.962717in}{1.093159in}}%
\pgfpathlineto{\pgfqpoint{5.963838in}{1.086776in}}%
\pgfpathlineto{\pgfqpoint{5.964958in}{1.099668in}}%
\pgfpathlineto{\pgfqpoint{5.966079in}{1.088528in}}%
\pgfpathlineto{\pgfqpoint{5.967199in}{1.096539in}}%
\pgfpathlineto{\pgfqpoint{5.970561in}{1.095662in}}%
\pgfpathlineto{\pgfqpoint{5.972802in}{1.120820in}}%
\pgfpathlineto{\pgfqpoint{5.973922in}{1.119819in}}%
\pgfpathlineto{\pgfqpoint{5.975043in}{1.108679in}}%
\pgfpathlineto{\pgfqpoint{5.978404in}{1.111808in}}%
\pgfpathlineto{\pgfqpoint{5.979525in}{1.109180in}}%
\pgfpathlineto{\pgfqpoint{5.980646in}{1.109055in}}%
\pgfpathlineto{\pgfqpoint{5.986248in}{1.097039in}}%
\pgfpathlineto{\pgfqpoint{5.987369in}{1.098291in}}%
\pgfpathlineto{\pgfqpoint{5.988489in}{1.096413in}}%
\pgfpathlineto{\pgfqpoint{5.990730in}{1.089154in}}%
\pgfpathlineto{\pgfqpoint{5.994092in}{1.085024in}}%
\pgfpathlineto{\pgfqpoint{5.995212in}{1.085024in}}%
\pgfpathlineto{\pgfqpoint{5.996333in}{1.080142in}}%
\pgfpathlineto{\pgfqpoint{5.997453in}{1.077764in}}%
\pgfpathlineto{\pgfqpoint{5.998574in}{1.079642in}}%
\pgfpathlineto{\pgfqpoint{6.004177in}{1.071256in}}%
\pgfpathlineto{\pgfqpoint{6.005297in}{1.075761in}}%
\pgfpathlineto{\pgfqpoint{6.006418in}{1.055735in}}%
\pgfpathlineto{\pgfqpoint{6.009779in}{1.070379in}}%
\pgfpathlineto{\pgfqpoint{6.010900in}{1.059115in}}%
\pgfpathlineto{\pgfqpoint{6.012020in}{1.053858in}}%
\pgfpathlineto{\pgfqpoint{6.013141in}{1.056737in}}%
\pgfpathlineto{\pgfqpoint{6.014261in}{1.057488in}}%
\pgfpathlineto{\pgfqpoint{6.017623in}{1.058239in}}%
\pgfpathlineto{\pgfqpoint{6.018743in}{1.062119in}}%
\pgfpathlineto{\pgfqpoint{6.019864in}{1.055360in}}%
\pgfpathlineto{\pgfqpoint{6.020984in}{1.067626in}}%
\pgfpathlineto{\pgfqpoint{6.022105in}{1.065999in}}%
\pgfpathlineto{\pgfqpoint{6.026587in}{1.047349in}}%
\pgfpathlineto{\pgfqpoint{6.027708in}{1.053107in}}%
\pgfpathlineto{\pgfqpoint{6.028828in}{1.048351in}}%
\pgfpathlineto{\pgfqpoint{6.029949in}{1.060992in}}%
\pgfpathlineto{\pgfqpoint{6.033310in}{1.056611in}}%
\pgfpathlineto{\pgfqpoint{6.034431in}{1.057863in}}%
\pgfpathlineto{\pgfqpoint{6.035551in}{1.056111in}}%
\pgfpathlineto{\pgfqpoint{6.036672in}{1.060617in}}%
\pgfpathlineto{\pgfqpoint{6.037792in}{1.056611in}}%
\pgfpathlineto{\pgfqpoint{6.041154in}{1.056611in}}%
\pgfpathlineto{\pgfqpoint{6.043395in}{1.043720in}}%
\pgfpathlineto{\pgfqpoint{6.044516in}{1.040215in}}%
\pgfpathlineto{\pgfqpoint{6.045636in}{1.042343in}}%
\pgfpathlineto{\pgfqpoint{6.048998in}{1.036711in}}%
\pgfpathlineto{\pgfqpoint{6.050118in}{1.040841in}}%
\pgfpathlineto{\pgfqpoint{6.051239in}{1.049853in}}%
\pgfpathlineto{\pgfqpoint{6.052359in}{1.051104in}}%
\pgfpathlineto{\pgfqpoint{6.053480in}{1.063120in}}%
\pgfpathlineto{\pgfqpoint{6.056841in}{1.067125in}}%
\pgfpathlineto{\pgfqpoint{6.057962in}{1.059741in}}%
\pgfpathlineto{\pgfqpoint{6.060203in}{1.074885in}}%
\pgfpathlineto{\pgfqpoint{6.061323in}{1.072883in}}%
\pgfpathlineto{\pgfqpoint{6.066926in}{1.049728in}}%
\pgfpathlineto{\pgfqpoint{6.068047in}{1.067125in}}%
\pgfpathlineto{\pgfqpoint{6.069167in}{1.056111in}}%
\pgfpathlineto{\pgfqpoint{6.072529in}{1.079767in}}%
\pgfpathlineto{\pgfqpoint{6.073649in}{1.079266in}}%
\pgfpathlineto{\pgfqpoint{6.075890in}{1.086150in}}%
\pgfpathlineto{\pgfqpoint{6.077011in}{1.118692in}}%
\pgfpathlineto{\pgfqpoint{6.080372in}{1.121571in}}%
\pgfpathlineto{\pgfqpoint{6.081493in}{1.118943in}}%
\pgfpathlineto{\pgfqpoint{6.082613in}{1.135965in}}%
\pgfpathlineto{\pgfqpoint{6.083734in}{1.138969in}}%
\pgfpathlineto{\pgfqpoint{6.084855in}{1.125451in}}%
\pgfpathlineto{\pgfqpoint{6.088216in}{1.118067in}}%
\pgfpathlineto{\pgfqpoint{6.089337in}{1.119068in}}%
\pgfpathlineto{\pgfqpoint{6.090457in}{1.125702in}}%
\pgfpathlineto{\pgfqpoint{6.092698in}{1.132335in}}%
\pgfpathlineto{\pgfqpoint{6.096060in}{1.134087in}}%
\pgfpathlineto{\pgfqpoint{6.097180in}{1.142473in}}%
\pgfpathlineto{\pgfqpoint{6.098301in}{1.135965in}}%
\pgfpathlineto{\pgfqpoint{6.099421in}{1.133837in}}%
\pgfpathlineto{\pgfqpoint{6.100542in}{1.128580in}}%
\pgfpathlineto{\pgfqpoint{6.105024in}{1.154364in}}%
\pgfpathlineto{\pgfqpoint{6.106145in}{1.170260in}}%
\pgfpathlineto{\pgfqpoint{6.108386in}{1.214317in}}%
\pgfpathlineto{\pgfqpoint{6.112868in}{1.202051in}}%
\pgfpathlineto{\pgfqpoint{6.113988in}{1.204554in}}%
\pgfpathlineto{\pgfqpoint{6.115109in}{1.208560in}}%
\pgfpathlineto{\pgfqpoint{6.116229in}{1.202802in}}%
\pgfpathlineto{\pgfqpoint{6.119591in}{1.219574in}}%
\pgfpathlineto{\pgfqpoint{6.121832in}{1.222703in}}%
\pgfpathlineto{\pgfqpoint{6.122952in}{1.221076in}}%
\pgfpathlineto{\pgfqpoint{6.124073in}{1.217822in}}%
\pgfpathlineto{\pgfqpoint{6.128555in}{1.218072in}}%
\pgfpathlineto{\pgfqpoint{6.129676in}{1.207558in}}%
\pgfpathlineto{\pgfqpoint{6.130796in}{1.210687in}}%
\pgfpathlineto{\pgfqpoint{6.131917in}{1.206557in}}%
\pgfpathlineto{\pgfqpoint{6.136399in}{1.228961in}}%
\pgfpathlineto{\pgfqpoint{6.137519in}{1.245358in}}%
\pgfpathlineto{\pgfqpoint{6.138640in}{1.244732in}}%
\pgfpathlineto{\pgfqpoint{6.139760in}{1.264007in}}%
\pgfpathlineto{\pgfqpoint{6.143122in}{1.256497in}}%
\pgfpathlineto{\pgfqpoint{6.144242in}{1.256747in}}%
\pgfpathlineto{\pgfqpoint{6.145363in}{1.269514in}}%
\pgfpathlineto{\pgfqpoint{6.146484in}{1.246484in}}%
\pgfpathlineto{\pgfqpoint{6.147604in}{1.252867in}}%
\pgfpathlineto{\pgfqpoint{6.152086in}{1.251741in}}%
\pgfpathlineto{\pgfqpoint{6.153207in}{1.254119in}}%
\pgfpathlineto{\pgfqpoint{6.154327in}{1.243480in}}%
\pgfpathlineto{\pgfqpoint{6.155448in}{1.248111in}}%
\pgfpathlineto{\pgfqpoint{6.158809in}{1.241478in}}%
\pgfpathlineto{\pgfqpoint{6.159930in}{1.250990in}}%
\pgfpathlineto{\pgfqpoint{6.161050in}{1.252617in}}%
\pgfpathlineto{\pgfqpoint{6.162171in}{1.252867in}}%
\pgfpathlineto{\pgfqpoint{6.163291in}{1.267762in}}%
\pgfpathlineto{\pgfqpoint{6.166653in}{1.287538in}}%
\pgfpathlineto{\pgfqpoint{6.167774in}{1.284158in}}%
\pgfpathlineto{\pgfqpoint{6.168894in}{1.291918in}}%
\pgfpathlineto{\pgfqpoint{6.170015in}{1.283658in}}%
\pgfpathlineto{\pgfqpoint{6.171135in}{1.279903in}}%
\pgfpathlineto{\pgfqpoint{6.174497in}{1.271141in}}%
\pgfpathlineto{\pgfqpoint{6.175617in}{1.264257in}}%
\pgfpathlineto{\pgfqpoint{6.176738in}{1.264257in}}%
\pgfpathlineto{\pgfqpoint{6.177858in}{1.270140in}}%
\pgfpathlineto{\pgfqpoint{6.178979in}{1.267386in}}%
\pgfpathlineto{\pgfqpoint{6.182340in}{1.272017in}}%
\pgfpathlineto{\pgfqpoint{6.183461in}{1.280278in}}%
\pgfpathlineto{\pgfqpoint{6.184581in}{1.278401in}}%
\pgfpathlineto{\pgfqpoint{6.185702in}{1.284784in}}%
\pgfpathlineto{\pgfqpoint{6.186822in}{1.277024in}}%
\pgfpathlineto{\pgfqpoint{6.191305in}{1.276398in}}%
\pgfpathlineto{\pgfqpoint{6.192425in}{1.277650in}}%
\pgfpathlineto{\pgfqpoint{6.193546in}{1.273019in}}%
\pgfpathlineto{\pgfqpoint{6.194666in}{1.280153in}}%
\pgfpathlineto{\pgfqpoint{6.198028in}{1.279026in}}%
\pgfpathlineto{\pgfqpoint{6.199148in}{1.277399in}}%
\pgfpathlineto{\pgfqpoint{6.200269in}{1.288789in}}%
\pgfpathlineto{\pgfqpoint{6.201389in}{1.283407in}}%
\pgfpathlineto{\pgfqpoint{6.202510in}{1.291167in}}%
\pgfpathlineto{\pgfqpoint{6.205871in}{1.284283in}}%
\pgfpathlineto{\pgfqpoint{6.206992in}{1.286661in}}%
\pgfpathlineto{\pgfqpoint{6.208113in}{1.286411in}}%
\pgfpathlineto{\pgfqpoint{6.209233in}{1.288664in}}%
\pgfpathlineto{\pgfqpoint{6.210354in}{1.287412in}}%
\pgfpathlineto{\pgfqpoint{6.213715in}{1.294547in}}%
\pgfpathlineto{\pgfqpoint{6.214836in}{1.304059in}}%
\pgfpathlineto{\pgfqpoint{6.215956in}{1.298802in}}%
\pgfpathlineto{\pgfqpoint{6.217077in}{1.296925in}}%
\pgfpathlineto{\pgfqpoint{6.218197in}{1.297425in}}%
\pgfpathlineto{\pgfqpoint{6.221559in}{1.308940in}}%
\pgfpathlineto{\pgfqpoint{6.222679in}{1.297425in}}%
\pgfpathlineto{\pgfqpoint{6.224920in}{1.303308in}}%
\pgfpathlineto{\pgfqpoint{6.226041in}{1.302056in}}%
\pgfpathlineto{\pgfqpoint{6.229403in}{1.304935in}}%
\pgfpathlineto{\pgfqpoint{6.230523in}{1.313071in}}%
\pgfpathlineto{\pgfqpoint{6.231644in}{1.306813in}}%
\pgfpathlineto{\pgfqpoint{6.232764in}{1.314573in}}%
\pgfpathlineto{\pgfqpoint{6.233885in}{1.317201in}}%
\pgfpathlineto{\pgfqpoint{6.237246in}{1.314823in}}%
\pgfpathlineto{\pgfqpoint{6.238367in}{1.312946in}}%
\pgfpathlineto{\pgfqpoint{6.239487in}{1.312445in}}%
\pgfpathlineto{\pgfqpoint{6.240608in}{1.313071in}}%
\pgfpathlineto{\pgfqpoint{6.241728in}{1.307313in}}%
\pgfpathlineto{\pgfqpoint{6.245090in}{1.305561in}}%
\pgfpathlineto{\pgfqpoint{6.246210in}{1.313321in}}%
\pgfpathlineto{\pgfqpoint{6.247331in}{1.312946in}}%
\pgfpathlineto{\pgfqpoint{6.252934in}{1.321832in}}%
\pgfpathlineto{\pgfqpoint{6.254054in}{1.326839in}}%
\pgfpathlineto{\pgfqpoint{6.255175in}{1.321206in}}%
\pgfpathlineto{\pgfqpoint{6.256295in}{1.333973in}}%
\pgfpathlineto{\pgfqpoint{6.257416in}{1.329843in}}%
\pgfpathlineto{\pgfqpoint{6.260777in}{1.320831in}}%
\pgfpathlineto{\pgfqpoint{6.261898in}{1.338729in}}%
\pgfpathlineto{\pgfqpoint{6.264139in}{1.346740in}}%
\pgfpathlineto{\pgfqpoint{6.268621in}{1.333347in}}%
\pgfpathlineto{\pgfqpoint{6.269742in}{1.328967in}}%
\pgfpathlineto{\pgfqpoint{6.270862in}{1.295798in}}%
\pgfpathlineto{\pgfqpoint{6.271983in}{1.290416in}}%
\pgfpathlineto{\pgfqpoint{6.273103in}{1.300304in}}%
\pgfpathlineto{\pgfqpoint{6.276465in}{1.293420in}}%
\pgfpathlineto{\pgfqpoint{6.277585in}{1.301180in}}%
\pgfpathlineto{\pgfqpoint{6.278706in}{1.272142in}}%
\pgfpathlineto{\pgfqpoint{6.279826in}{1.271141in}}%
\pgfpathlineto{\pgfqpoint{6.280947in}{1.272518in}}%
\pgfpathlineto{\pgfqpoint{6.284308in}{1.265759in}}%
\pgfpathlineto{\pgfqpoint{6.286549in}{1.231715in}}%
\pgfpathlineto{\pgfqpoint{6.287670in}{1.236221in}}%
\pgfpathlineto{\pgfqpoint{6.288790in}{1.246359in}}%
\pgfpathlineto{\pgfqpoint{6.292152in}{1.247736in}}%
\pgfpathlineto{\pgfqpoint{6.293273in}{1.240351in}}%
\pgfpathlineto{\pgfqpoint{6.294393in}{1.248612in}}%
\pgfpathlineto{\pgfqpoint{6.295514in}{1.243355in}}%
\pgfpathlineto{\pgfqpoint{6.296634in}{1.257123in}}%
\pgfpathlineto{\pgfqpoint{6.301116in}{1.256247in}}%
\pgfpathlineto{\pgfqpoint{6.302237in}{1.251365in}}%
\pgfpathlineto{\pgfqpoint{6.303357in}{1.254620in}}%
\pgfpathlineto{\pgfqpoint{6.304478in}{1.242228in}}%
\pgfpathlineto{\pgfqpoint{6.307839in}{1.234343in}}%
\pgfpathlineto{\pgfqpoint{6.308960in}{1.221952in}}%
\pgfpathlineto{\pgfqpoint{6.310080in}{1.227084in}}%
\pgfpathlineto{\pgfqpoint{6.311201in}{1.207809in}}%
\pgfpathlineto{\pgfqpoint{6.312322in}{1.223454in}}%
\pgfpathlineto{\pgfqpoint{6.315683in}{1.240601in}}%
\pgfpathlineto{\pgfqpoint{6.317924in}{1.229712in}}%
\pgfpathlineto{\pgfqpoint{6.319045in}{1.227835in}}%
\pgfpathlineto{\pgfqpoint{6.320165in}{1.222077in}}%
\pgfpathlineto{\pgfqpoint{6.323527in}{1.220450in}}%
\pgfpathlineto{\pgfqpoint{6.324647in}{1.203178in}}%
\pgfpathlineto{\pgfqpoint{6.325768in}{1.213566in}}%
\pgfpathlineto{\pgfqpoint{6.326888in}{1.206557in}}%
\pgfpathlineto{\pgfqpoint{6.328009in}{1.208309in}}%
\pgfpathlineto{\pgfqpoint{6.331370in}{1.222828in}}%
\pgfpathlineto{\pgfqpoint{6.332491in}{1.220951in}}%
\pgfpathlineto{\pgfqpoint{6.333612in}{1.238348in}}%
\pgfpathlineto{\pgfqpoint{6.334732in}{1.224580in}}%
\pgfpathlineto{\pgfqpoint{6.335853in}{1.231089in}}%
\pgfpathlineto{\pgfqpoint{6.339214in}{1.245608in}}%
\pgfpathlineto{\pgfqpoint{6.341455in}{1.223579in}}%
\pgfpathlineto{\pgfqpoint{6.342576in}{1.205305in}}%
\pgfpathlineto{\pgfqpoint{6.343696in}{1.204930in}}%
\pgfpathlineto{\pgfqpoint{6.347058in}{1.208309in}}%
\pgfpathlineto{\pgfqpoint{6.348178in}{1.211313in}}%
\pgfpathlineto{\pgfqpoint{6.349299in}{1.217822in}}%
\pgfpathlineto{\pgfqpoint{6.350419in}{1.216820in}}%
\pgfpathlineto{\pgfqpoint{6.351540in}{1.226458in}}%
\pgfpathlineto{\pgfqpoint{6.354902in}{1.222828in}}%
\pgfpathlineto{\pgfqpoint{6.358263in}{1.254119in}}%
\pgfpathlineto{\pgfqpoint{6.359384in}{1.250740in}}%
\pgfpathlineto{\pgfqpoint{6.362745in}{1.249738in}}%
\pgfpathlineto{\pgfqpoint{6.363866in}{1.242604in}}%
\pgfpathlineto{\pgfqpoint{6.364986in}{1.248987in}}%
\pgfpathlineto{\pgfqpoint{6.366107in}{1.286161in}}%
\pgfpathlineto{\pgfqpoint{6.367227in}{1.285660in}}%
\pgfpathlineto{\pgfqpoint{6.370589in}{1.285285in}}%
\pgfpathlineto{\pgfqpoint{6.371709in}{1.293545in}}%
\pgfpathlineto{\pgfqpoint{6.372830in}{1.270015in}}%
\pgfpathlineto{\pgfqpoint{6.373951in}{1.275397in}}%
\pgfpathlineto{\pgfqpoint{6.375071in}{1.258124in}}%
\pgfpathlineto{\pgfqpoint{6.378433in}{1.241853in}}%
\pgfpathlineto{\pgfqpoint{6.379553in}{1.249488in}}%
\pgfpathlineto{\pgfqpoint{6.380674in}{1.199047in}}%
\pgfpathlineto{\pgfqpoint{6.381794in}{1.181024in}}%
\pgfpathlineto{\pgfqpoint{6.382915in}{1.188784in}}%
\pgfpathlineto{\pgfqpoint{6.386276in}{1.181650in}}%
\pgfpathlineto{\pgfqpoint{6.387397in}{1.183026in}}%
\pgfpathlineto{\pgfqpoint{6.388517in}{1.191412in}}%
\pgfpathlineto{\pgfqpoint{6.390758in}{1.173138in}}%
\pgfpathlineto{\pgfqpoint{6.394120in}{1.178896in}}%
\pgfpathlineto{\pgfqpoint{6.395241in}{1.198296in}}%
\pgfpathlineto{\pgfqpoint{6.396361in}{1.182901in}}%
\pgfpathlineto{\pgfqpoint{6.397482in}{1.183151in}}%
\pgfpathlineto{\pgfqpoint{6.398602in}{1.193915in}}%
\pgfpathlineto{\pgfqpoint{6.403084in}{1.195918in}}%
\pgfpathlineto{\pgfqpoint{6.404205in}{1.199548in}}%
\pgfpathlineto{\pgfqpoint{6.405325in}{1.179271in}}%
\pgfpathlineto{\pgfqpoint{6.406446in}{1.182901in}}%
\pgfpathlineto{\pgfqpoint{6.410928in}{1.184153in}}%
\pgfpathlineto{\pgfqpoint{6.412048in}{1.182901in}}%
\pgfpathlineto{\pgfqpoint{6.413169in}{1.128956in}}%
\pgfpathlineto{\pgfqpoint{6.417651in}{1.129331in}}%
\pgfpathlineto{\pgfqpoint{6.419892in}{1.150359in}}%
\pgfpathlineto{\pgfqpoint{6.421013in}{1.139094in}}%
\pgfpathlineto{\pgfqpoint{6.422133in}{1.146729in}}%
\pgfpathlineto{\pgfqpoint{6.425495in}{1.141597in}}%
\pgfpathlineto{\pgfqpoint{6.426615in}{1.145603in}}%
\pgfpathlineto{\pgfqpoint{6.427736in}{1.155115in}}%
\pgfpathlineto{\pgfqpoint{6.429977in}{1.147605in}}%
\pgfpathlineto{\pgfqpoint{6.433338in}{1.159496in}}%
\pgfpathlineto{\pgfqpoint{6.434459in}{1.147981in}}%
\pgfpathlineto{\pgfqpoint{6.435580in}{1.155365in}}%
\pgfpathlineto{\pgfqpoint{6.436700in}{1.140971in}}%
\pgfpathlineto{\pgfqpoint{6.442303in}{1.174265in}}%
\pgfpathlineto{\pgfqpoint{6.443423in}{1.171386in}}%
\pgfpathlineto{\pgfqpoint{6.444544in}{1.166004in}}%
\pgfpathlineto{\pgfqpoint{6.445664in}{1.165503in}}%
\pgfpathlineto{\pgfqpoint{6.449026in}{1.159496in}}%
\pgfpathlineto{\pgfqpoint{6.450146in}{1.159496in}}%
\pgfpathlineto{\pgfqpoint{6.451267in}{1.147104in}}%
\pgfpathlineto{\pgfqpoint{6.452387in}{1.127329in}}%
\pgfpathlineto{\pgfqpoint{6.453508in}{1.132836in}}%
\pgfpathlineto{\pgfqpoint{6.457990in}{1.144726in}}%
\pgfpathlineto{\pgfqpoint{6.459111in}{1.143350in}}%
\pgfpathlineto{\pgfqpoint{6.461352in}{1.157368in}}%
\pgfpathlineto{\pgfqpoint{6.464713in}{1.148857in}}%
\pgfpathlineto{\pgfqpoint{6.466954in}{1.137968in}}%
\pgfpathlineto{\pgfqpoint{6.468075in}{1.147104in}}%
\pgfpathlineto{\pgfqpoint{6.469195in}{1.144101in}}%
\pgfpathlineto{\pgfqpoint{6.472557in}{1.140846in}}%
\pgfpathlineto{\pgfqpoint{6.473677in}{1.138093in}}%
\pgfpathlineto{\pgfqpoint{6.474798in}{1.152862in}}%
\pgfpathlineto{\pgfqpoint{6.475919in}{1.144601in}}%
\pgfpathlineto{\pgfqpoint{6.477039in}{1.148106in}}%
\pgfpathlineto{\pgfqpoint{6.481521in}{1.184153in}}%
\pgfpathlineto{\pgfqpoint{6.482642in}{1.179021in}}%
\pgfpathlineto{\pgfqpoint{6.484883in}{1.222703in}}%
\pgfpathlineto{\pgfqpoint{6.488244in}{1.222202in}}%
\pgfpathlineto{\pgfqpoint{6.489365in}{1.203178in}}%
\pgfpathlineto{\pgfqpoint{6.490485in}{1.209436in}}%
\pgfpathlineto{\pgfqpoint{6.491606in}{1.208434in}}%
\pgfpathlineto{\pgfqpoint{6.492726in}{1.206432in}}%
\pgfpathlineto{\pgfqpoint{6.496088in}{1.198046in}}%
\pgfpathlineto{\pgfqpoint{6.497209in}{1.201050in}}%
\pgfpathlineto{\pgfqpoint{6.498329in}{1.197921in}}%
\pgfpathlineto{\pgfqpoint{6.500570in}{1.196669in}}%
\pgfpathlineto{\pgfqpoint{6.503932in}{1.198547in}}%
\pgfpathlineto{\pgfqpoint{6.505052in}{1.206056in}}%
\pgfpathlineto{\pgfqpoint{6.506173in}{1.228335in}}%
\pgfpathlineto{\pgfqpoint{6.507293in}{1.223204in}}%
\pgfpathlineto{\pgfqpoint{6.508414in}{1.228461in}}%
\pgfpathlineto{\pgfqpoint{6.511775in}{1.288789in}}%
\pgfpathlineto{\pgfqpoint{6.514016in}{1.230964in}}%
\pgfpathlineto{\pgfqpoint{6.516257in}{1.226208in}}%
\pgfpathlineto{\pgfqpoint{6.520740in}{1.265384in}}%
\pgfpathlineto{\pgfqpoint{6.521860in}{1.267637in}}%
\pgfpathlineto{\pgfqpoint{6.522981in}{1.303809in}}%
\pgfpathlineto{\pgfqpoint{6.524101in}{1.312445in}}%
\pgfpathlineto{\pgfqpoint{6.527463in}{1.309441in}}%
\pgfpathlineto{\pgfqpoint{6.528583in}{1.319079in}}%
\pgfpathlineto{\pgfqpoint{6.529704in}{1.293045in}}%
\pgfpathlineto{\pgfqpoint{6.530824in}{1.291668in}}%
\pgfpathlineto{\pgfqpoint{6.531945in}{1.280654in}}%
\pgfpathlineto{\pgfqpoint{6.536427in}{1.273895in}}%
\pgfpathlineto{\pgfqpoint{6.537547in}{1.268012in}}%
\pgfpathlineto{\pgfqpoint{6.538668in}{1.269514in}}%
\pgfpathlineto{\pgfqpoint{6.539789in}{1.266385in}}%
\pgfpathlineto{\pgfqpoint{6.539789in}{1.266385in}}%
\pgfusepath{stroke}%
\end{pgfscope}%
\begin{pgfscope}%
\pgfpathrectangle{\pgfqpoint{3.966666in}{0.331635in}}{\pgfqpoint{2.695652in}{1.104878in}}%
\pgfusepath{clip}%
\pgfsetroundcap%
\pgfsetroundjoin%
\pgfsetlinewidth{1.505625pt}%
\definecolor{currentstroke}{rgb}{1.000000,0.647059,0.000000}%
\pgfsetstrokecolor{currentstroke}%
\pgfsetdash{}{0pt}%
\pgfpathmoveto{\pgfqpoint{4.089196in}{0.381857in}}%
\pgfpathlineto{\pgfqpoint{4.092557in}{0.391213in}}%
\pgfpathlineto{\pgfqpoint{4.097039in}{0.393267in}}%
\pgfpathlineto{\pgfqpoint{4.100401in}{0.390576in}}%
\pgfpathlineto{\pgfqpoint{4.106004in}{0.389890in}}%
\pgfpathlineto{\pgfqpoint{4.108245in}{0.390493in}}%
\pgfpathlineto{\pgfqpoint{4.123932in}{0.391462in}}%
\pgfpathlineto{\pgfqpoint{4.127294in}{0.392068in}}%
\pgfpathlineto{\pgfqpoint{4.131776in}{0.395428in}}%
\pgfpathlineto{\pgfqpoint{4.136258in}{0.397064in}}%
\pgfpathlineto{\pgfqpoint{4.139619in}{0.398959in}}%
\pgfpathlineto{\pgfqpoint{4.145222in}{0.399980in}}%
\pgfpathlineto{\pgfqpoint{4.147463in}{0.400878in}}%
\pgfpathlineto{\pgfqpoint{4.151945in}{0.401912in}}%
\pgfpathlineto{\pgfqpoint{4.155307in}{0.403644in}}%
\pgfpathlineto{\pgfqpoint{4.162030in}{0.405477in}}%
\pgfpathlineto{\pgfqpoint{4.163150in}{0.405917in}}%
\pgfpathlineto{\pgfqpoint{4.166512in}{0.406363in}}%
\pgfpathlineto{\pgfqpoint{4.170994in}{0.409005in}}%
\pgfpathlineto{\pgfqpoint{4.175476in}{0.410100in}}%
\pgfpathlineto{\pgfqpoint{4.178838in}{0.411632in}}%
\pgfpathlineto{\pgfqpoint{4.183320in}{0.412897in}}%
\pgfpathlineto{\pgfqpoint{4.186682in}{0.414187in}}%
\pgfpathlineto{\pgfqpoint{4.202369in}{0.415928in}}%
\pgfpathlineto{\pgfqpoint{4.209092in}{0.416462in}}%
\pgfpathlineto{\pgfqpoint{4.231503in}{0.419868in}}%
\pgfpathlineto{\pgfqpoint{4.233744in}{0.420793in}}%
\pgfpathlineto{\pgfqpoint{4.239346in}{0.421986in}}%
\pgfpathlineto{\pgfqpoint{4.241587in}{0.422502in}}%
\pgfpathlineto{\pgfqpoint{4.248311in}{0.423601in}}%
\pgfpathlineto{\pgfqpoint{4.249431in}{0.423878in}}%
\pgfpathlineto{\pgfqpoint{4.256154in}{0.424990in}}%
\pgfpathlineto{\pgfqpoint{4.278565in}{0.430349in}}%
\pgfpathlineto{\pgfqpoint{4.280806in}{0.431254in}}%
\pgfpathlineto{\pgfqpoint{4.285288in}{0.432049in}}%
\pgfpathlineto{\pgfqpoint{4.288649in}{0.433482in}}%
\pgfpathlineto{\pgfqpoint{4.295373in}{0.434956in}}%
\pgfpathlineto{\pgfqpoint{4.296493in}{0.435395in}}%
\pgfpathlineto{\pgfqpoint{4.302096in}{0.436543in}}%
\pgfpathlineto{\pgfqpoint{4.304337in}{0.437323in}}%
\pgfpathlineto{\pgfqpoint{4.308819in}{0.438213in}}%
\pgfpathlineto{\pgfqpoint{4.312181in}{0.439592in}}%
\pgfpathlineto{\pgfqpoint{4.317783in}{0.440714in}}%
\pgfpathlineto{\pgfqpoint{4.320024in}{0.441699in}}%
\pgfpathlineto{\pgfqpoint{4.325627in}{0.442994in}}%
\pgfpathlineto{\pgfqpoint{4.327868in}{0.443848in}}%
\pgfpathlineto{\pgfqpoint{4.332350in}{0.444731in}}%
\pgfpathlineto{\pgfqpoint{4.335712in}{0.446080in}}%
\pgfpathlineto{\pgfqpoint{4.341314in}{0.447331in}}%
\pgfpathlineto{\pgfqpoint{4.343555in}{0.448214in}}%
\pgfpathlineto{\pgfqpoint{4.349158in}{0.449408in}}%
\pgfpathlineto{\pgfqpoint{4.351399in}{0.450103in}}%
\pgfpathlineto{\pgfqpoint{4.357002in}{0.451196in}}%
\pgfpathlineto{\pgfqpoint{4.359243in}{0.451861in}}%
\pgfpathlineto{\pgfqpoint{4.364845in}{0.452613in}}%
\pgfpathlineto{\pgfqpoint{4.367086in}{0.453552in}}%
\pgfpathlineto{\pgfqpoint{4.371569in}{0.454436in}}%
\pgfpathlineto{\pgfqpoint{4.374930in}{0.455855in}}%
\pgfpathlineto{\pgfqpoint{4.379412in}{0.456750in}}%
\pgfpathlineto{\pgfqpoint{4.382774in}{0.458181in}}%
\pgfpathlineto{\pgfqpoint{4.388376in}{0.459523in}}%
\pgfpathlineto{\pgfqpoint{4.390617in}{0.460390in}}%
\pgfpathlineto{\pgfqpoint{4.396220in}{0.461590in}}%
\pgfpathlineto{\pgfqpoint{4.398461in}{0.462456in}}%
\pgfpathlineto{\pgfqpoint{4.405184in}{0.463803in}}%
\pgfpathlineto{\pgfqpoint{4.414149in}{0.465788in}}%
\pgfpathlineto{\pgfqpoint{4.420872in}{0.466901in}}%
\pgfpathlineto{\pgfqpoint{4.421992in}{0.467130in}}%
\pgfpathlineto{\pgfqpoint{4.433198in}{0.467957in}}%
\pgfpathlineto{\pgfqpoint{4.437680in}{0.468680in}}%
\pgfpathlineto{\pgfqpoint{4.453367in}{0.469563in}}%
\pgfpathlineto{\pgfqpoint{4.465693in}{0.470638in}}%
\pgfpathlineto{\pgfqpoint{4.476898in}{0.472042in}}%
\pgfpathlineto{\pgfqpoint{4.483621in}{0.472882in}}%
\pgfpathlineto{\pgfqpoint{4.488103in}{0.473278in}}%
\pgfpathlineto{\pgfqpoint{4.504911in}{0.475210in}}%
\pgfpathlineto{\pgfqpoint{4.516117in}{0.477082in}}%
\pgfpathlineto{\pgfqpoint{4.522840in}{0.478048in}}%
\pgfpathlineto{\pgfqpoint{4.523960in}{0.478402in}}%
\pgfpathlineto{\pgfqpoint{4.529563in}{0.479403in}}%
\pgfpathlineto{\pgfqpoint{4.531804in}{0.480075in}}%
\pgfpathlineto{\pgfqpoint{4.537407in}{0.481064in}}%
\pgfpathlineto{\pgfqpoint{4.539648in}{0.481734in}}%
\pgfpathlineto{\pgfqpoint{4.545250in}{0.482767in}}%
\pgfpathlineto{\pgfqpoint{4.547491in}{0.483473in}}%
\pgfpathlineto{\pgfqpoint{4.554214in}{0.484463in}}%
\pgfpathlineto{\pgfqpoint{4.555335in}{0.484765in}}%
\pgfpathlineto{\pgfqpoint{4.560938in}{0.485628in}}%
\pgfpathlineto{\pgfqpoint{4.571022in}{0.488129in}}%
\pgfpathlineto{\pgfqpoint{4.576625in}{0.489317in}}%
\pgfpathlineto{\pgfqpoint{4.578866in}{0.490119in}}%
\pgfpathlineto{\pgfqpoint{4.584469in}{0.491192in}}%
\pgfpathlineto{\pgfqpoint{4.586710in}{0.491885in}}%
\pgfpathlineto{\pgfqpoint{4.592312in}{0.492895in}}%
\pgfpathlineto{\pgfqpoint{4.593433in}{0.493241in}}%
\pgfpathlineto{\pgfqpoint{4.600156in}{0.494302in}}%
\pgfpathlineto{\pgfqpoint{4.602397in}{0.495035in}}%
\pgfpathlineto{\pgfqpoint{4.606879in}{0.495856in}}%
\pgfpathlineto{\pgfqpoint{4.610241in}{0.497220in}}%
\pgfpathlineto{\pgfqpoint{4.614723in}{0.498073in}}%
\pgfpathlineto{\pgfqpoint{4.618084in}{0.499431in}}%
\pgfpathlineto{\pgfqpoint{4.622567in}{0.500432in}}%
\pgfpathlineto{\pgfqpoint{4.625928in}{0.501872in}}%
\pgfpathlineto{\pgfqpoint{4.630410in}{0.502886in}}%
\pgfpathlineto{\pgfqpoint{4.633772in}{0.504491in}}%
\pgfpathlineto{\pgfqpoint{4.638254in}{0.505654in}}%
\pgfpathlineto{\pgfqpoint{4.641616in}{0.507479in}}%
\pgfpathlineto{\pgfqpoint{4.646098in}{0.508729in}}%
\pgfpathlineto{\pgfqpoint{4.649459in}{0.510530in}}%
\pgfpathlineto{\pgfqpoint{4.653941in}{0.511651in}}%
\pgfpathlineto{\pgfqpoint{4.657303in}{0.513256in}}%
\pgfpathlineto{\pgfqpoint{4.662906in}{0.514379in}}%
\pgfpathlineto{\pgfqpoint{4.665147in}{0.515320in}}%
\pgfpathlineto{\pgfqpoint{4.669629in}{0.516264in}}%
\pgfpathlineto{\pgfqpoint{4.672990in}{0.517625in}}%
\pgfpathlineto{\pgfqpoint{4.677472in}{0.518527in}}%
\pgfpathlineto{\pgfqpoint{4.680834in}{0.519850in}}%
\pgfpathlineto{\pgfqpoint{4.685316in}{0.520794in}}%
\pgfpathlineto{\pgfqpoint{4.688678in}{0.522017in}}%
\pgfpathlineto{\pgfqpoint{4.694280in}{0.523195in}}%
\pgfpathlineto{\pgfqpoint{4.696521in}{0.524012in}}%
\pgfpathlineto{\pgfqpoint{4.702124in}{0.525237in}}%
\pgfpathlineto{\pgfqpoint{4.711088in}{0.527444in}}%
\pgfpathlineto{\pgfqpoint{4.712209in}{0.527938in}}%
\pgfpathlineto{\pgfqpoint{4.716691in}{0.528829in}}%
\pgfpathlineto{\pgfqpoint{4.720052in}{0.530145in}}%
\pgfpathlineto{\pgfqpoint{4.725655in}{0.531358in}}%
\pgfpathlineto{\pgfqpoint{4.727896in}{0.532171in}}%
\pgfpathlineto{\pgfqpoint{4.733499in}{0.533352in}}%
\pgfpathlineto{\pgfqpoint{4.735740in}{0.534213in}}%
\pgfpathlineto{\pgfqpoint{4.740222in}{0.535099in}}%
\pgfpathlineto{\pgfqpoint{4.743584in}{0.536321in}}%
\pgfpathlineto{\pgfqpoint{4.749186in}{0.537398in}}%
\pgfpathlineto{\pgfqpoint{4.751427in}{0.538013in}}%
\pgfpathlineto{\pgfqpoint{4.758150in}{0.539151in}}%
\pgfpathlineto{\pgfqpoint{4.759271in}{0.539435in}}%
\pgfpathlineto{\pgfqpoint{4.765994in}{0.540476in}}%
\pgfpathlineto{\pgfqpoint{4.767115in}{0.540729in}}%
\pgfpathlineto{\pgfqpoint{4.773838in}{0.541504in}}%
\pgfpathlineto{\pgfqpoint{4.798489in}{0.547023in}}%
\pgfpathlineto{\pgfqpoint{4.804092in}{0.548004in}}%
\pgfpathlineto{\pgfqpoint{4.810815in}{0.549265in}}%
\pgfpathlineto{\pgfqpoint{4.822020in}{0.552043in}}%
\pgfpathlineto{\pgfqpoint{4.827623in}{0.553224in}}%
\pgfpathlineto{\pgfqpoint{4.829864in}{0.554048in}}%
\pgfpathlineto{\pgfqpoint{4.835467in}{0.555243in}}%
\pgfpathlineto{\pgfqpoint{4.837708in}{0.556032in}}%
\pgfpathlineto{\pgfqpoint{4.843310in}{0.557209in}}%
\pgfpathlineto{\pgfqpoint{4.845551in}{0.557935in}}%
\pgfpathlineto{\pgfqpoint{4.851154in}{0.559049in}}%
\pgfpathlineto{\pgfqpoint{4.853395in}{0.559864in}}%
\pgfpathlineto{\pgfqpoint{4.858998in}{0.561020in}}%
\pgfpathlineto{\pgfqpoint{4.861239in}{0.561821in}}%
\pgfpathlineto{\pgfqpoint{4.865721in}{0.562633in}}%
\pgfpathlineto{\pgfqpoint{4.869083in}{0.563448in}}%
\pgfpathlineto{\pgfqpoint{4.874685in}{0.564631in}}%
\pgfpathlineto{\pgfqpoint{4.876926in}{0.565438in}}%
\pgfpathlineto{\pgfqpoint{4.882529in}{0.566652in}}%
\pgfpathlineto{\pgfqpoint{4.884770in}{0.567427in}}%
\pgfpathlineto{\pgfqpoint{4.889252in}{0.568240in}}%
\pgfpathlineto{\pgfqpoint{4.892614in}{0.569583in}}%
\pgfpathlineto{\pgfqpoint{4.899337in}{0.571007in}}%
\pgfpathlineto{\pgfqpoint{4.900457in}{0.571488in}}%
\pgfpathlineto{\pgfqpoint{4.907180in}{0.573054in}}%
\pgfpathlineto{\pgfqpoint{4.908301in}{0.573568in}}%
\pgfpathlineto{\pgfqpoint{4.912783in}{0.574588in}}%
\pgfpathlineto{\pgfqpoint{4.916145in}{0.576040in}}%
\pgfpathlineto{\pgfqpoint{4.920627in}{0.576939in}}%
\pgfpathlineto{\pgfqpoint{4.923988in}{0.578295in}}%
\pgfpathlineto{\pgfqpoint{4.929591in}{0.579217in}}%
\pgfpathlineto{\pgfqpoint{4.931832in}{0.580087in}}%
\pgfpathlineto{\pgfqpoint{4.937435in}{0.581273in}}%
\pgfpathlineto{\pgfqpoint{4.939676in}{0.582090in}}%
\pgfpathlineto{\pgfqpoint{4.945278in}{0.583172in}}%
\pgfpathlineto{\pgfqpoint{4.947519in}{0.584093in}}%
\pgfpathlineto{\pgfqpoint{4.952002in}{0.585085in}}%
\pgfpathlineto{\pgfqpoint{4.955363in}{0.586620in}}%
\pgfpathlineto{\pgfqpoint{4.960966in}{0.587672in}}%
\pgfpathlineto{\pgfqpoint{4.963207in}{0.588735in}}%
\pgfpathlineto{\pgfqpoint{4.967689in}{0.589824in}}%
\pgfpathlineto{\pgfqpoint{4.971051in}{0.591443in}}%
\pgfpathlineto{\pgfqpoint{4.975533in}{0.592518in}}%
\pgfpathlineto{\pgfqpoint{4.978894in}{0.594252in}}%
\pgfpathlineto{\pgfqpoint{4.983376in}{0.595346in}}%
\pgfpathlineto{\pgfqpoint{4.986738in}{0.596905in}}%
\pgfpathlineto{\pgfqpoint{4.991220in}{0.597987in}}%
\pgfpathlineto{\pgfqpoint{4.994582in}{0.599524in}}%
\pgfpathlineto{\pgfqpoint{4.999064in}{0.600497in}}%
\pgfpathlineto{\pgfqpoint{5.002425in}{0.601892in}}%
\pgfpathlineto{\pgfqpoint{5.006907in}{0.602901in}}%
\pgfpathlineto{\pgfqpoint{5.010269in}{0.604428in}}%
\pgfpathlineto{\pgfqpoint{5.014751in}{0.605360in}}%
\pgfpathlineto{\pgfqpoint{5.018113in}{0.606683in}}%
\pgfpathlineto{\pgfqpoint{5.022595in}{0.607529in}}%
\pgfpathlineto{\pgfqpoint{5.024836in}{0.608443in}}%
\pgfpathlineto{\pgfqpoint{5.030438in}{0.609344in}}%
\pgfpathlineto{\pgfqpoint{5.033800in}{0.610672in}}%
\pgfpathlineto{\pgfqpoint{5.038282in}{0.611515in}}%
\pgfpathlineto{\pgfqpoint{5.041644in}{0.612858in}}%
\pgfpathlineto{\pgfqpoint{5.046126in}{0.613801in}}%
\pgfpathlineto{\pgfqpoint{5.049487in}{0.615212in}}%
\pgfpathlineto{\pgfqpoint{5.053970in}{0.616183in}}%
\pgfpathlineto{\pgfqpoint{5.057331in}{0.617524in}}%
\pgfpathlineto{\pgfqpoint{5.061813in}{0.618433in}}%
\pgfpathlineto{\pgfqpoint{5.065175in}{0.619875in}}%
\pgfpathlineto{\pgfqpoint{5.070777in}{0.620869in}}%
\pgfpathlineto{\pgfqpoint{5.073019in}{0.621869in}}%
\pgfpathlineto{\pgfqpoint{5.077501in}{0.622866in}}%
\pgfpathlineto{\pgfqpoint{5.080862in}{0.624374in}}%
\pgfpathlineto{\pgfqpoint{5.085344in}{0.625393in}}%
\pgfpathlineto{\pgfqpoint{5.088706in}{0.626807in}}%
\pgfpathlineto{\pgfqpoint{5.093188in}{0.627739in}}%
\pgfpathlineto{\pgfqpoint{5.096550in}{0.629135in}}%
\pgfpathlineto{\pgfqpoint{5.101032in}{0.630033in}}%
\pgfpathlineto{\pgfqpoint{5.104393in}{0.631472in}}%
\pgfpathlineto{\pgfqpoint{5.108875in}{0.632481in}}%
\pgfpathlineto{\pgfqpoint{5.111116in}{0.633503in}}%
\pgfpathlineto{\pgfqpoint{5.116719in}{0.634503in}}%
\pgfpathlineto{\pgfqpoint{5.120081in}{0.636034in}}%
\pgfpathlineto{\pgfqpoint{5.124563in}{0.637026in}}%
\pgfpathlineto{\pgfqpoint{5.127924in}{0.638447in}}%
\pgfpathlineto{\pgfqpoint{5.132406in}{0.639407in}}%
\pgfpathlineto{\pgfqpoint{5.135768in}{0.640856in}}%
\pgfpathlineto{\pgfqpoint{5.140250in}{0.641827in}}%
\pgfpathlineto{\pgfqpoint{5.143612in}{0.643244in}}%
\pgfpathlineto{\pgfqpoint{5.148094in}{0.644211in}}%
\pgfpathlineto{\pgfqpoint{5.151455in}{0.645614in}}%
\pgfpathlineto{\pgfqpoint{5.155938in}{0.646580in}}%
\pgfpathlineto{\pgfqpoint{5.159299in}{0.648077in}}%
\pgfpathlineto{\pgfqpoint{5.163781in}{0.649121in}}%
\pgfpathlineto{\pgfqpoint{5.167143in}{0.650686in}}%
\pgfpathlineto{\pgfqpoint{5.171625in}{0.651721in}}%
\pgfpathlineto{\pgfqpoint{5.174986in}{0.653259in}}%
\pgfpathlineto{\pgfqpoint{5.180589in}{0.654301in}}%
\pgfpathlineto{\pgfqpoint{5.182830in}{0.655325in}}%
\pgfpathlineto{\pgfqpoint{5.187312in}{0.656328in}}%
\pgfpathlineto{\pgfqpoint{5.190674in}{0.657803in}}%
\pgfpathlineto{\pgfqpoint{5.195156in}{0.658794in}}%
\pgfpathlineto{\pgfqpoint{5.198518in}{0.660281in}}%
\pgfpathlineto{\pgfqpoint{5.203000in}{0.661211in}}%
\pgfpathlineto{\pgfqpoint{5.206361in}{0.662595in}}%
\pgfpathlineto{\pgfqpoint{5.210843in}{0.663516in}}%
\pgfpathlineto{\pgfqpoint{5.214205in}{0.664820in}}%
\pgfpathlineto{\pgfqpoint{5.218687in}{0.665692in}}%
\pgfpathlineto{\pgfqpoint{5.222049in}{0.666933in}}%
\pgfpathlineto{\pgfqpoint{5.227651in}{0.667999in}}%
\pgfpathlineto{\pgfqpoint{5.237736in}{0.670758in}}%
\pgfpathlineto{\pgfqpoint{5.242218in}{0.671639in}}%
\pgfpathlineto{\pgfqpoint{5.245580in}{0.673009in}}%
\pgfpathlineto{\pgfqpoint{5.250062in}{0.673937in}}%
\pgfpathlineto{\pgfqpoint{5.253423in}{0.675318in}}%
\pgfpathlineto{\pgfqpoint{5.257905in}{0.676196in}}%
\pgfpathlineto{\pgfqpoint{5.261267in}{0.677529in}}%
\pgfpathlineto{\pgfqpoint{5.265749in}{0.678410in}}%
\pgfpathlineto{\pgfqpoint{5.269111in}{0.679669in}}%
\pgfpathlineto{\pgfqpoint{5.273593in}{0.680562in}}%
\pgfpathlineto{\pgfqpoint{5.276954in}{0.681485in}}%
\pgfpathlineto{\pgfqpoint{5.281437in}{0.682431in}}%
\pgfpathlineto{\pgfqpoint{5.284798in}{0.683853in}}%
\pgfpathlineto{\pgfqpoint{5.289280in}{0.684795in}}%
\pgfpathlineto{\pgfqpoint{5.292642in}{0.686153in}}%
\pgfpathlineto{\pgfqpoint{5.297124in}{0.687030in}}%
\pgfpathlineto{\pgfqpoint{5.300486in}{0.688419in}}%
\pgfpathlineto{\pgfqpoint{5.304968in}{0.689407in}}%
\pgfpathlineto{\pgfqpoint{5.308329in}{0.690398in}}%
\pgfpathlineto{\pgfqpoint{5.312811in}{0.691395in}}%
\pgfpathlineto{\pgfqpoint{5.316173in}{0.692352in}}%
\pgfpathlineto{\pgfqpoint{5.320655in}{0.693248in}}%
\pgfpathlineto{\pgfqpoint{5.324017in}{0.694651in}}%
\pgfpathlineto{\pgfqpoint{5.328499in}{0.695617in}}%
\pgfpathlineto{\pgfqpoint{5.331860in}{0.697045in}}%
\pgfpathlineto{\pgfqpoint{5.337463in}{0.697987in}}%
\pgfpathlineto{\pgfqpoint{5.339704in}{0.698939in}}%
\pgfpathlineto{\pgfqpoint{5.344186in}{0.699872in}}%
\pgfpathlineto{\pgfqpoint{5.347548in}{0.701163in}}%
\pgfpathlineto{\pgfqpoint{5.352030in}{0.702040in}}%
\pgfpathlineto{\pgfqpoint{5.355391in}{0.703752in}}%
\pgfpathlineto{\pgfqpoint{5.359873in}{0.704882in}}%
\pgfpathlineto{\pgfqpoint{5.363235in}{0.706627in}}%
\pgfpathlineto{\pgfqpoint{5.368838in}{0.707807in}}%
\pgfpathlineto{\pgfqpoint{5.371079in}{0.708988in}}%
\pgfpathlineto{\pgfqpoint{5.375561in}{0.710181in}}%
\pgfpathlineto{\pgfqpoint{5.378922in}{0.711955in}}%
\pgfpathlineto{\pgfqpoint{5.383405in}{0.713170in}}%
\pgfpathlineto{\pgfqpoint{5.386766in}{0.714925in}}%
\pgfpathlineto{\pgfqpoint{5.391248in}{0.716069in}}%
\pgfpathlineto{\pgfqpoint{5.394610in}{0.717832in}}%
\pgfpathlineto{\pgfqpoint{5.399092in}{0.719048in}}%
\pgfpathlineto{\pgfqpoint{5.402453in}{0.720895in}}%
\pgfpathlineto{\pgfqpoint{5.406936in}{0.722111in}}%
\pgfpathlineto{\pgfqpoint{5.410297in}{0.723819in}}%
\pgfpathlineto{\pgfqpoint{5.414779in}{0.724958in}}%
\pgfpathlineto{\pgfqpoint{5.417020in}{0.726098in}}%
\pgfpathlineto{\pgfqpoint{5.422623in}{0.727227in}}%
\pgfpathlineto{\pgfqpoint{5.425985in}{0.728960in}}%
\pgfpathlineto{\pgfqpoint{5.430467in}{0.730105in}}%
\pgfpathlineto{\pgfqpoint{5.433828in}{0.731840in}}%
\pgfpathlineto{\pgfqpoint{5.438310in}{0.733010in}}%
\pgfpathlineto{\pgfqpoint{5.441672in}{0.734791in}}%
\pgfpathlineto{\pgfqpoint{5.446154in}{0.736006in}}%
\pgfpathlineto{\pgfqpoint{5.449516in}{0.737802in}}%
\pgfpathlineto{\pgfqpoint{5.453998in}{0.739027in}}%
\pgfpathlineto{\pgfqpoint{5.457359in}{0.740802in}}%
\pgfpathlineto{\pgfqpoint{5.461841in}{0.741957in}}%
\pgfpathlineto{\pgfqpoint{5.465203in}{0.743715in}}%
\pgfpathlineto{\pgfqpoint{5.469685in}{0.744899in}}%
\pgfpathlineto{\pgfqpoint{5.473047in}{0.746657in}}%
\pgfpathlineto{\pgfqpoint{5.478649in}{0.747811in}}%
\pgfpathlineto{\pgfqpoint{5.480890in}{0.748975in}}%
\pgfpathlineto{\pgfqpoint{5.485372in}{0.750145in}}%
\pgfpathlineto{\pgfqpoint{5.488734in}{0.751879in}}%
\pgfpathlineto{\pgfqpoint{5.493216in}{0.752982in}}%
\pgfpathlineto{\pgfqpoint{5.496578in}{0.754680in}}%
\pgfpathlineto{\pgfqpoint{5.501060in}{0.755817in}}%
\pgfpathlineto{\pgfqpoint{5.504421in}{0.757586in}}%
\pgfpathlineto{\pgfqpoint{5.508904in}{0.758800in}}%
\pgfpathlineto{\pgfqpoint{5.512265in}{0.760627in}}%
\pgfpathlineto{\pgfqpoint{5.516747in}{0.761817in}}%
\pgfpathlineto{\pgfqpoint{5.518988in}{0.763058in}}%
\pgfpathlineto{\pgfqpoint{5.524591in}{0.764330in}}%
\pgfpathlineto{\pgfqpoint{5.527953in}{0.766200in}}%
\pgfpathlineto{\pgfqpoint{5.532435in}{0.767499in}}%
\pgfpathlineto{\pgfqpoint{5.535796in}{0.769468in}}%
\pgfpathlineto{\pgfqpoint{5.540278in}{0.770791in}}%
\pgfpathlineto{\pgfqpoint{5.543640in}{0.772748in}}%
\pgfpathlineto{\pgfqpoint{5.548122in}{0.774029in}}%
\pgfpathlineto{\pgfqpoint{5.551484in}{0.776001in}}%
\pgfpathlineto{\pgfqpoint{5.555966in}{0.777347in}}%
\pgfpathlineto{\pgfqpoint{5.559327in}{0.778888in}}%
\pgfpathlineto{\pgfqpoint{5.563809in}{0.779910in}}%
\pgfpathlineto{\pgfqpoint{5.567171in}{0.781346in}}%
\pgfpathlineto{\pgfqpoint{5.571653in}{0.782318in}}%
\pgfpathlineto{\pgfqpoint{5.575015in}{0.783528in}}%
\pgfpathlineto{\pgfqpoint{5.580617in}{0.784538in}}%
\pgfpathlineto{\pgfqpoint{5.582858in}{0.785350in}}%
\pgfpathlineto{\pgfqpoint{5.588461in}{0.786513in}}%
\pgfpathlineto{\pgfqpoint{5.590702in}{0.787294in}}%
\pgfpathlineto{\pgfqpoint{5.596305in}{0.788109in}}%
\pgfpathlineto{\pgfqpoint{5.598546in}{0.788936in}}%
\pgfpathlineto{\pgfqpoint{5.604148in}{0.790176in}}%
\pgfpathlineto{\pgfqpoint{5.606389in}{0.790994in}}%
\pgfpathlineto{\pgfqpoint{5.611992in}{0.792174in}}%
\pgfpathlineto{\pgfqpoint{5.614233in}{0.792906in}}%
\pgfpathlineto{\pgfqpoint{5.619836in}{0.793982in}}%
\pgfpathlineto{\pgfqpoint{5.622077in}{0.794765in}}%
\pgfpathlineto{\pgfqpoint{5.627679in}{0.795965in}}%
\pgfpathlineto{\pgfqpoint{5.629921in}{0.796796in}}%
\pgfpathlineto{\pgfqpoint{5.634403in}{0.797658in}}%
\pgfpathlineto{\pgfqpoint{5.637764in}{0.798974in}}%
\pgfpathlineto{\pgfqpoint{5.642246in}{0.799906in}}%
\pgfpathlineto{\pgfqpoint{5.645608in}{0.801388in}}%
\pgfpathlineto{\pgfqpoint{5.650090in}{0.802407in}}%
\pgfpathlineto{\pgfqpoint{5.653452in}{0.803956in}}%
\pgfpathlineto{\pgfqpoint{5.657934in}{0.805005in}}%
\pgfpathlineto{\pgfqpoint{5.661295in}{0.806523in}}%
\pgfpathlineto{\pgfqpoint{5.665777in}{0.807601in}}%
\pgfpathlineto{\pgfqpoint{5.669139in}{0.809171in}}%
\pgfpathlineto{\pgfqpoint{5.673621in}{0.810217in}}%
\pgfpathlineto{\pgfqpoint{5.676983in}{0.811884in}}%
\pgfpathlineto{\pgfqpoint{5.681465in}{0.812982in}}%
\pgfpathlineto{\pgfqpoint{5.684826in}{0.814034in}}%
\pgfpathlineto{\pgfqpoint{5.689308in}{0.815022in}}%
\pgfpathlineto{\pgfqpoint{5.692670in}{0.816458in}}%
\pgfpathlineto{\pgfqpoint{5.697152in}{0.817406in}}%
\pgfpathlineto{\pgfqpoint{5.700514in}{0.818728in}}%
\pgfpathlineto{\pgfqpoint{5.704996in}{0.819626in}}%
\pgfpathlineto{\pgfqpoint{5.708357in}{0.820981in}}%
\pgfpathlineto{\pgfqpoint{5.713960in}{0.822155in}}%
\pgfpathlineto{\pgfqpoint{5.715081in}{0.822540in}}%
\pgfpathlineto{\pgfqpoint{5.720683in}{0.823337in}}%
\pgfpathlineto{\pgfqpoint{5.722924in}{0.824097in}}%
\pgfpathlineto{\pgfqpoint{5.729647in}{0.825080in}}%
\pgfpathlineto{\pgfqpoint{5.731888in}{0.825684in}}%
\pgfpathlineto{\pgfqpoint{5.737491in}{0.826606in}}%
\pgfpathlineto{\pgfqpoint{5.739732in}{0.827137in}}%
\pgfpathlineto{\pgfqpoint{5.747576in}{0.828093in}}%
\pgfpathlineto{\pgfqpoint{5.754299in}{0.829064in}}%
\pgfpathlineto{\pgfqpoint{5.762143in}{0.830272in}}%
\pgfpathlineto{\pgfqpoint{5.768866in}{0.831085in}}%
\pgfpathlineto{\pgfqpoint{5.778951in}{0.832400in}}%
\pgfpathlineto{\pgfqpoint{5.785674in}{0.833376in}}%
\pgfpathlineto{\pgfqpoint{5.794638in}{0.834922in}}%
\pgfpathlineto{\pgfqpoint{5.801361in}{0.835982in}}%
\pgfpathlineto{\pgfqpoint{5.810325in}{0.837605in}}%
\pgfpathlineto{\pgfqpoint{5.821531in}{0.838884in}}%
\pgfpathlineto{\pgfqpoint{5.833856in}{0.841183in}}%
\pgfpathlineto{\pgfqpoint{5.840580in}{0.842193in}}%
\pgfpathlineto{\pgfqpoint{5.849544in}{0.843970in}}%
\pgfpathlineto{\pgfqpoint{5.855146in}{0.844943in}}%
\pgfpathlineto{\pgfqpoint{5.857388in}{0.845561in}}%
\pgfpathlineto{\pgfqpoint{5.862990in}{0.846493in}}%
\pgfpathlineto{\pgfqpoint{5.865231in}{0.847140in}}%
\pgfpathlineto{\pgfqpoint{5.870834in}{0.848090in}}%
\pgfpathlineto{\pgfqpoint{5.873075in}{0.848642in}}%
\pgfpathlineto{\pgfqpoint{5.879798in}{0.849667in}}%
\pgfpathlineto{\pgfqpoint{5.888762in}{0.851211in}}%
\pgfpathlineto{\pgfqpoint{5.896606in}{0.852190in}}%
\pgfpathlineto{\pgfqpoint{5.903329in}{0.853140in}}%
\pgfpathlineto{\pgfqpoint{5.912293in}{0.854545in}}%
\pgfpathlineto{\pgfqpoint{5.919017in}{0.855511in}}%
\pgfpathlineto{\pgfqpoint{5.927981in}{0.856787in}}%
\pgfpathlineto{\pgfqpoint{5.935824in}{0.857731in}}%
\pgfpathlineto{\pgfqpoint{5.942548in}{0.858744in}}%
\pgfpathlineto{\pgfqpoint{5.951512in}{0.860181in}}%
\pgfpathlineto{\pgfqpoint{5.962717in}{0.861442in}}%
\pgfpathlineto{\pgfqpoint{5.975043in}{0.863311in}}%
\pgfpathlineto{\pgfqpoint{5.986248in}{0.864560in}}%
\pgfpathlineto{\pgfqpoint{5.998574in}{0.866263in}}%
\pgfpathlineto{\pgfqpoint{6.010900in}{0.867284in}}%
\pgfpathlineto{\pgfqpoint{6.022105in}{0.868578in}}%
\pgfpathlineto{\pgfqpoint{6.034431in}{0.869659in}}%
\pgfpathlineto{\pgfqpoint{6.045636in}{0.870861in}}%
\pgfpathlineto{\pgfqpoint{6.057962in}{0.871911in}}%
\pgfpathlineto{\pgfqpoint{6.069167in}{0.873164in}}%
\pgfpathlineto{\pgfqpoint{6.080372in}{0.874247in}}%
\pgfpathlineto{\pgfqpoint{6.092698in}{0.875885in}}%
\pgfpathlineto{\pgfqpoint{6.099421in}{0.876727in}}%
\pgfpathlineto{\pgfqpoint{6.108386in}{0.878134in}}%
\pgfpathlineto{\pgfqpoint{6.115109in}{0.879181in}}%
\pgfpathlineto{\pgfqpoint{6.124073in}{0.880800in}}%
\pgfpathlineto{\pgfqpoint{6.130796in}{0.881591in}}%
\pgfpathlineto{\pgfqpoint{6.131917in}{0.881849in}}%
\pgfpathlineto{\pgfqpoint{6.138640in}{0.882701in}}%
\pgfpathlineto{\pgfqpoint{6.139760in}{0.883003in}}%
\pgfpathlineto{\pgfqpoint{6.145363in}{0.883899in}}%
\pgfpathlineto{\pgfqpoint{6.147604in}{0.884476in}}%
\pgfpathlineto{\pgfqpoint{6.154327in}{0.885339in}}%
\pgfpathlineto{\pgfqpoint{6.155448in}{0.885625in}}%
\pgfpathlineto{\pgfqpoint{6.161050in}{0.886479in}}%
\pgfpathlineto{\pgfqpoint{6.171135in}{0.888620in}}%
\pgfpathlineto{\pgfqpoint{6.176738in}{0.889503in}}%
\pgfpathlineto{\pgfqpoint{6.178979in}{0.890093in}}%
\pgfpathlineto{\pgfqpoint{6.184581in}{0.890993in}}%
\pgfpathlineto{\pgfqpoint{6.186822in}{0.891597in}}%
\pgfpathlineto{\pgfqpoint{6.193546in}{0.892487in}}%
\pgfpathlineto{\pgfqpoint{6.194666in}{0.892787in}}%
\pgfpathlineto{\pgfqpoint{6.200269in}{0.893686in}}%
\pgfpathlineto{\pgfqpoint{6.202510in}{0.894291in}}%
\pgfpathlineto{\pgfqpoint{6.208113in}{0.895193in}}%
\pgfpathlineto{\pgfqpoint{6.210354in}{0.895795in}}%
\pgfpathlineto{\pgfqpoint{6.215956in}{0.896720in}}%
\pgfpathlineto{\pgfqpoint{6.218197in}{0.897331in}}%
\pgfpathlineto{\pgfqpoint{6.223800in}{0.898257in}}%
\pgfpathlineto{\pgfqpoint{6.226041in}{0.898872in}}%
\pgfpathlineto{\pgfqpoint{6.231644in}{0.899804in}}%
\pgfpathlineto{\pgfqpoint{6.233885in}{0.900434in}}%
\pgfpathlineto{\pgfqpoint{6.239487in}{0.901371in}}%
\pgfpathlineto{\pgfqpoint{6.241728in}{0.901988in}}%
\pgfpathlineto{\pgfqpoint{6.247331in}{0.902911in}}%
\pgfpathlineto{\pgfqpoint{6.248451in}{0.903221in}}%
\pgfpathlineto{\pgfqpoint{6.255175in}{0.904167in}}%
\pgfpathlineto{\pgfqpoint{6.257416in}{0.904808in}}%
\pgfpathlineto{\pgfqpoint{6.263018in}{0.905772in}}%
\pgfpathlineto{\pgfqpoint{6.265259in}{0.906429in}}%
\pgfpathlineto{\pgfqpoint{6.270862in}{0.907352in}}%
\pgfpathlineto{\pgfqpoint{6.278706in}{0.908778in}}%
\pgfpathlineto{\pgfqpoint{6.285429in}{0.909833in}}%
\pgfpathlineto{\pgfqpoint{6.296634in}{0.911800in}}%
\pgfpathlineto{\pgfqpoint{6.304478in}{0.912796in}}%
\pgfpathlineto{\pgfqpoint{6.311201in}{0.913703in}}%
\pgfpathlineto{\pgfqpoint{6.320165in}{0.915084in}}%
\pgfpathlineto{\pgfqpoint{6.331370in}{0.916378in}}%
\pgfpathlineto{\pgfqpoint{6.339214in}{0.917519in}}%
\pgfpathlineto{\pgfqpoint{6.356022in}{0.919668in}}%
\pgfpathlineto{\pgfqpoint{6.367227in}{0.921600in}}%
\pgfpathlineto{\pgfqpoint{6.373951in}{0.922623in}}%
\pgfpathlineto{\pgfqpoint{6.380674in}{0.923515in}}%
\pgfpathlineto{\pgfqpoint{6.390758in}{0.924796in}}%
\pgfpathlineto{\pgfqpoint{6.401964in}{0.925910in}}%
\pgfpathlineto{\pgfqpoint{6.406446in}{0.926648in}}%
\pgfpathlineto{\pgfqpoint{6.419892in}{0.927734in}}%
\pgfpathlineto{\pgfqpoint{6.437821in}{0.929563in}}%
\pgfpathlineto{\pgfqpoint{6.450146in}{0.930700in}}%
\pgfpathlineto{\pgfqpoint{6.461352in}{0.931865in}}%
\pgfpathlineto{\pgfqpoint{6.473677in}{0.932872in}}%
\pgfpathlineto{\pgfqpoint{6.492726in}{0.935116in}}%
\pgfpathlineto{\pgfqpoint{6.506173in}{0.936380in}}%
\pgfpathlineto{\pgfqpoint{6.516257in}{0.937801in}}%
\pgfpathlineto{\pgfqpoint{6.522981in}{0.938697in}}%
\pgfpathlineto{\pgfqpoint{6.531945in}{0.940142in}}%
\pgfpathlineto{\pgfqpoint{6.539789in}{0.941015in}}%
\pgfpathlineto{\pgfqpoint{6.539789in}{0.941015in}}%
\pgfusepath{stroke}%
\end{pgfscope}%
\begin{pgfscope}%
\pgfsetrectcap%
\pgfsetmiterjoin%
\pgfsetlinewidth{0.803000pt}%
\definecolor{currentstroke}{rgb}{1.000000,1.000000,1.000000}%
\pgfsetstrokecolor{currentstroke}%
\pgfsetdash{}{0pt}%
\pgfpathmoveto{\pgfqpoint{3.966666in}{0.331635in}}%
\pgfpathlineto{\pgfqpoint{3.966666in}{1.436513in}}%
\pgfusepath{stroke}%
\end{pgfscope}%
\begin{pgfscope}%
\pgfsetrectcap%
\pgfsetmiterjoin%
\pgfsetlinewidth{0.803000pt}%
\definecolor{currentstroke}{rgb}{1.000000,1.000000,1.000000}%
\pgfsetstrokecolor{currentstroke}%
\pgfsetdash{}{0pt}%
\pgfpathmoveto{\pgfqpoint{6.662318in}{0.331635in}}%
\pgfpathlineto{\pgfqpoint{6.662318in}{1.436513in}}%
\pgfusepath{stroke}%
\end{pgfscope}%
\begin{pgfscope}%
\pgfsetrectcap%
\pgfsetmiterjoin%
\pgfsetlinewidth{0.803000pt}%
\definecolor{currentstroke}{rgb}{1.000000,1.000000,1.000000}%
\pgfsetstrokecolor{currentstroke}%
\pgfsetdash{}{0pt}%
\pgfpathmoveto{\pgfqpoint{3.966666in}{0.331635in}}%
\pgfpathlineto{\pgfqpoint{6.662318in}{0.331635in}}%
\pgfusepath{stroke}%
\end{pgfscope}%
\begin{pgfscope}%
\pgfsetrectcap%
\pgfsetmiterjoin%
\pgfsetlinewidth{0.803000pt}%
\definecolor{currentstroke}{rgb}{1.000000,1.000000,1.000000}%
\pgfsetstrokecolor{currentstroke}%
\pgfsetdash{}{0pt}%
\pgfpathmoveto{\pgfqpoint{3.966666in}{1.436513in}}%
\pgfpathlineto{\pgfqpoint{6.662318in}{1.436513in}}%
\pgfusepath{stroke}%
\end{pgfscope}%
\begin{pgfscope}%
\definecolor{textcolor}{rgb}{0.150000,0.150000,0.150000}%
\pgfsetstrokecolor{textcolor}%
\pgfsetfillcolor{textcolor}%
\pgftext[x=5.314492in,y=1.519846in,,base]{\color{textcolor}\rmfamily\fontsize{12.000000}{14.400000}\selectfont DIS}%
\end{pgfscope}%
\end{pgfpicture}%
\makeatother%
\endgroup%

            \end{adjustbox}
        %\caption{Flower one.}
    \end{minipage}
    \hfill
    \begin{minipage}[b]{0.49\textwidth}
        \centering
        \begin{adjustbox}{width=\textwidth,center}
            %% Creator: Matplotlib, PGF backend
%%
%% To include the figure in your LaTeX document, write
%%   \input{<filename>.pgf}
%%
%% Make sure the required packages are loaded in your preamble
%%   \usepackage{pgf}
%%
%% Figures using additional raster images can only be included by \input if
%% they are in the same directory as the main LaTeX file. For loading figures
%% from other directories you can use the `import` package
%%   \usepackage{import}
%% and then include the figures with
%%   \import{<path to file>}{<filename>.pgf}
%%
%% Matplotlib used the following preamble
%%   \usepackage{fontspec}
%%   \setmainfont{DejaVuSerif.ttf}[Path=/opt/tljh/user/lib/python3.6/site-packages/matplotlib/mpl-data/fonts/ttf/]
%%   \setsansfont{DejaVuSans.ttf}[Path=/opt/tljh/user/lib/python3.6/site-packages/matplotlib/mpl-data/fonts/ttf/]
%%   \setmonofont{DejaVuSansMono.ttf}[Path=/opt/tljh/user/lib/python3.6/site-packages/matplotlib/mpl-data/fonts/ttf/]
%%
\begingroup%
\makeatletter%
\begin{pgfpicture}%
\pgfpathrectangle{\pgfpointorigin}{\pgfqpoint{6.754154in}{5.934781in}}%
\pgfusepath{use as bounding box, clip}%
\begin{pgfscope}%
\pgfsetbuttcap%
\pgfsetmiterjoin%
\definecolor{currentfill}{rgb}{1.000000,1.000000,1.000000}%
\pgfsetfillcolor{currentfill}%
\pgfsetlinewidth{0.000000pt}%
\definecolor{currentstroke}{rgb}{1.000000,1.000000,1.000000}%
\pgfsetstrokecolor{currentstroke}%
\pgfsetdash{}{0pt}%
\pgfpathmoveto{\pgfqpoint{0.000000in}{0.000000in}}%
\pgfpathlineto{\pgfqpoint{6.754154in}{0.000000in}}%
\pgfpathlineto{\pgfqpoint{6.754154in}{5.934781in}}%
\pgfpathlineto{\pgfqpoint{0.000000in}{5.934781in}}%
\pgfpathclose%
\pgfusepath{fill}%
\end{pgfscope}%
\begin{pgfscope}%
\pgfsetbuttcap%
\pgfsetmiterjoin%
\definecolor{currentfill}{rgb}{0.917647,0.917647,0.949020}%
\pgfsetfillcolor{currentfill}%
\pgfsetlinewidth{0.000000pt}%
\definecolor{currentstroke}{rgb}{0.000000,0.000000,0.000000}%
\pgfsetstrokecolor{currentstroke}%
\pgfsetstrokeopacity{0.000000}%
\pgfsetdash{}{0pt}%
\pgfpathmoveto{\pgfqpoint{0.320934in}{4.233896in}}%
\pgfpathlineto{\pgfqpoint{2.904267in}{4.233896in}}%
\pgfpathlineto{\pgfqpoint{2.904267in}{4.634781in}}%
\pgfpathlineto{\pgfqpoint{0.320934in}{4.634781in}}%
\pgfpathclose%
\pgfusepath{fill}%
\end{pgfscope}%
\begin{pgfscope}%
\pgfpathrectangle{\pgfqpoint{0.320934in}{4.233896in}}{\pgfqpoint{2.583333in}{0.400885in}}%
\pgfusepath{clip}%
\pgfsetroundcap%
\pgfsetroundjoin%
\pgfsetlinewidth{0.803000pt}%
\definecolor{currentstroke}{rgb}{1.000000,1.000000,1.000000}%
\pgfsetstrokecolor{currentstroke}%
\pgfsetdash{}{0pt}%
\pgfpathmoveto{\pgfqpoint{0.436210in}{4.233896in}}%
\pgfpathlineto{\pgfqpoint{0.436210in}{4.634781in}}%
\pgfusepath{stroke}%
\end{pgfscope}%
\begin{pgfscope}%
\definecolor{textcolor}{rgb}{0.150000,0.150000,0.150000}%
\pgfsetstrokecolor{textcolor}%
\pgfsetfillcolor{textcolor}%
\pgftext[x=0.436210in,y=4.136674in,,top]{\color{textcolor}\rmfamily\fontsize{14.000000}{16.800000}\selectfont 2012}%
\end{pgfscope}%
\begin{pgfscope}%
\pgfpathrectangle{\pgfqpoint{0.320934in}{4.233896in}}{\pgfqpoint{2.583333in}{0.400885in}}%
\pgfusepath{clip}%
\pgfsetroundcap%
\pgfsetroundjoin%
\pgfsetlinewidth{0.803000pt}%
\definecolor{currentstroke}{rgb}{1.000000,1.000000,1.000000}%
\pgfsetstrokecolor{currentstroke}%
\pgfsetdash{}{0pt}%
\pgfpathmoveto{\pgfqpoint{0.829235in}{4.233896in}}%
\pgfpathlineto{\pgfqpoint{0.829235in}{4.634781in}}%
\pgfusepath{stroke}%
\end{pgfscope}%
\begin{pgfscope}%
\definecolor{textcolor}{rgb}{0.150000,0.150000,0.150000}%
\pgfsetstrokecolor{textcolor}%
\pgfsetfillcolor{textcolor}%
\pgftext[x=0.829235in,y=4.136674in,,top]{\color{textcolor}\rmfamily\fontsize{14.000000}{16.800000}\selectfont 2013}%
\end{pgfscope}%
\begin{pgfscope}%
\pgfpathrectangle{\pgfqpoint{0.320934in}{4.233896in}}{\pgfqpoint{2.583333in}{0.400885in}}%
\pgfusepath{clip}%
\pgfsetroundcap%
\pgfsetroundjoin%
\pgfsetlinewidth{0.803000pt}%
\definecolor{currentstroke}{rgb}{1.000000,1.000000,1.000000}%
\pgfsetstrokecolor{currentstroke}%
\pgfsetdash{}{0pt}%
\pgfpathmoveto{\pgfqpoint{1.221186in}{4.233896in}}%
\pgfpathlineto{\pgfqpoint{1.221186in}{4.634781in}}%
\pgfusepath{stroke}%
\end{pgfscope}%
\begin{pgfscope}%
\definecolor{textcolor}{rgb}{0.150000,0.150000,0.150000}%
\pgfsetstrokecolor{textcolor}%
\pgfsetfillcolor{textcolor}%
\pgftext[x=1.221186in,y=4.136674in,,top]{\color{textcolor}\rmfamily\fontsize{14.000000}{16.800000}\selectfont 2014}%
\end{pgfscope}%
\begin{pgfscope}%
\pgfpathrectangle{\pgfqpoint{0.320934in}{4.233896in}}{\pgfqpoint{2.583333in}{0.400885in}}%
\pgfusepath{clip}%
\pgfsetroundcap%
\pgfsetroundjoin%
\pgfsetlinewidth{0.803000pt}%
\definecolor{currentstroke}{rgb}{1.000000,1.000000,1.000000}%
\pgfsetstrokecolor{currentstroke}%
\pgfsetdash{}{0pt}%
\pgfpathmoveto{\pgfqpoint{1.613137in}{4.233896in}}%
\pgfpathlineto{\pgfqpoint{1.613137in}{4.634781in}}%
\pgfusepath{stroke}%
\end{pgfscope}%
\begin{pgfscope}%
\definecolor{textcolor}{rgb}{0.150000,0.150000,0.150000}%
\pgfsetstrokecolor{textcolor}%
\pgfsetfillcolor{textcolor}%
\pgftext[x=1.613137in,y=4.136674in,,top]{\color{textcolor}\rmfamily\fontsize{14.000000}{16.800000}\selectfont 2015}%
\end{pgfscope}%
\begin{pgfscope}%
\pgfpathrectangle{\pgfqpoint{0.320934in}{4.233896in}}{\pgfqpoint{2.583333in}{0.400885in}}%
\pgfusepath{clip}%
\pgfsetroundcap%
\pgfsetroundjoin%
\pgfsetlinewidth{0.803000pt}%
\definecolor{currentstroke}{rgb}{1.000000,1.000000,1.000000}%
\pgfsetstrokecolor{currentstroke}%
\pgfsetdash{}{0pt}%
\pgfpathmoveto{\pgfqpoint{2.005088in}{4.233896in}}%
\pgfpathlineto{\pgfqpoint{2.005088in}{4.634781in}}%
\pgfusepath{stroke}%
\end{pgfscope}%
\begin{pgfscope}%
\definecolor{textcolor}{rgb}{0.150000,0.150000,0.150000}%
\pgfsetstrokecolor{textcolor}%
\pgfsetfillcolor{textcolor}%
\pgftext[x=2.005088in,y=4.136674in,,top]{\color{textcolor}\rmfamily\fontsize{14.000000}{16.800000}\selectfont 2016}%
\end{pgfscope}%
\begin{pgfscope}%
\pgfpathrectangle{\pgfqpoint{0.320934in}{4.233896in}}{\pgfqpoint{2.583333in}{0.400885in}}%
\pgfusepath{clip}%
\pgfsetroundcap%
\pgfsetroundjoin%
\pgfsetlinewidth{0.803000pt}%
\definecolor{currentstroke}{rgb}{1.000000,1.000000,1.000000}%
\pgfsetstrokecolor{currentstroke}%
\pgfsetdash{}{0pt}%
\pgfpathmoveto{\pgfqpoint{2.398113in}{4.233896in}}%
\pgfpathlineto{\pgfqpoint{2.398113in}{4.634781in}}%
\pgfusepath{stroke}%
\end{pgfscope}%
\begin{pgfscope}%
\definecolor{textcolor}{rgb}{0.150000,0.150000,0.150000}%
\pgfsetstrokecolor{textcolor}%
\pgfsetfillcolor{textcolor}%
\pgftext[x=2.398113in,y=4.136674in,,top]{\color{textcolor}\rmfamily\fontsize{14.000000}{16.800000}\selectfont 2017}%
\end{pgfscope}%
\begin{pgfscope}%
\pgfpathrectangle{\pgfqpoint{0.320934in}{4.233896in}}{\pgfqpoint{2.583333in}{0.400885in}}%
\pgfusepath{clip}%
\pgfsetroundcap%
\pgfsetroundjoin%
\pgfsetlinewidth{0.803000pt}%
\definecolor{currentstroke}{rgb}{1.000000,1.000000,1.000000}%
\pgfsetstrokecolor{currentstroke}%
\pgfsetdash{}{0pt}%
\pgfpathmoveto{\pgfqpoint{2.790064in}{4.233896in}}%
\pgfpathlineto{\pgfqpoint{2.790064in}{4.634781in}}%
\pgfusepath{stroke}%
\end{pgfscope}%
\begin{pgfscope}%
\definecolor{textcolor}{rgb}{0.150000,0.150000,0.150000}%
\pgfsetstrokecolor{textcolor}%
\pgfsetfillcolor{textcolor}%
\pgftext[x=2.790064in,y=4.136674in,,top]{\color{textcolor}\rmfamily\fontsize{14.000000}{16.800000}\selectfont 2018}%
\end{pgfscope}%
\begin{pgfscope}%
\pgfpathrectangle{\pgfqpoint{0.320934in}{4.233896in}}{\pgfqpoint{2.583333in}{0.400885in}}%
\pgfusepath{clip}%
\pgfsetroundcap%
\pgfsetroundjoin%
\pgfsetlinewidth{0.803000pt}%
\definecolor{currentstroke}{rgb}{1.000000,1.000000,1.000000}%
\pgfsetstrokecolor{currentstroke}%
\pgfsetdash{}{0pt}%
\pgfpathmoveto{\pgfqpoint{0.320934in}{4.344128in}}%
\pgfpathlineto{\pgfqpoint{2.904267in}{4.344128in}}%
\pgfusepath{stroke}%
\end{pgfscope}%
\begin{pgfscope}%
\definecolor{textcolor}{rgb}{0.150000,0.150000,0.150000}%
\pgfsetstrokecolor{textcolor}%
\pgfsetfillcolor{textcolor}%
\pgftext[x=0.100000in,y=4.270262in,left,base]{\color{textcolor}\rmfamily\fontsize{14.000000}{16.800000}\selectfont 1}%
\end{pgfscope}%
\begin{pgfscope}%
\pgfpathrectangle{\pgfqpoint{0.320934in}{4.233896in}}{\pgfqpoint{2.583333in}{0.400885in}}%
\pgfusepath{clip}%
\pgfsetroundcap%
\pgfsetroundjoin%
\pgfsetlinewidth{0.803000pt}%
\definecolor{currentstroke}{rgb}{1.000000,1.000000,1.000000}%
\pgfsetstrokecolor{currentstroke}%
\pgfsetdash{}{0pt}%
\pgfpathmoveto{\pgfqpoint{0.320934in}{4.458669in}}%
\pgfpathlineto{\pgfqpoint{2.904267in}{4.458669in}}%
\pgfusepath{stroke}%
\end{pgfscope}%
\begin{pgfscope}%
\definecolor{textcolor}{rgb}{0.150000,0.150000,0.150000}%
\pgfsetstrokecolor{textcolor}%
\pgfsetfillcolor{textcolor}%
\pgftext[x=0.100000in,y=4.384803in,left,base]{\color{textcolor}\rmfamily\fontsize{14.000000}{16.800000}\selectfont 2}%
\end{pgfscope}%
\begin{pgfscope}%
\pgfpathrectangle{\pgfqpoint{0.320934in}{4.233896in}}{\pgfqpoint{2.583333in}{0.400885in}}%
\pgfusepath{clip}%
\pgfsetroundcap%
\pgfsetroundjoin%
\pgfsetlinewidth{0.803000pt}%
\definecolor{currentstroke}{rgb}{1.000000,1.000000,1.000000}%
\pgfsetstrokecolor{currentstroke}%
\pgfsetdash{}{0pt}%
\pgfpathmoveto{\pgfqpoint{0.320934in}{4.573210in}}%
\pgfpathlineto{\pgfqpoint{2.904267in}{4.573210in}}%
\pgfusepath{stroke}%
\end{pgfscope}%
\begin{pgfscope}%
\definecolor{textcolor}{rgb}{0.150000,0.150000,0.150000}%
\pgfsetstrokecolor{textcolor}%
\pgfsetfillcolor{textcolor}%
\pgftext[x=0.100000in,y=4.499344in,left,base]{\color{textcolor}\rmfamily\fontsize{14.000000}{16.800000}\selectfont 3}%
\end{pgfscope}%
\begin{pgfscope}%
\pgfpathrectangle{\pgfqpoint{0.320934in}{4.233896in}}{\pgfqpoint{2.583333in}{0.400885in}}%
\pgfusepath{clip}%
\pgfsetroundcap%
\pgfsetroundjoin%
\pgfsetlinewidth{1.505625pt}%
\definecolor{currentstroke}{rgb}{0.000000,0.000000,0.000000}%
\pgfsetstrokecolor{currentstroke}%
\pgfsetdash{}{0pt}%
\pgfpathmoveto{\pgfqpoint{0.438358in}{4.344128in}}%
\pgfpathlineto{\pgfqpoint{0.439432in}{4.345083in}}%
\pgfpathlineto{\pgfqpoint{0.441580in}{4.343978in}}%
\pgfpathlineto{\pgfqpoint{0.445875in}{4.345250in}}%
\pgfpathlineto{\pgfqpoint{0.446949in}{4.344513in}}%
\pgfpathlineto{\pgfqpoint{0.448023in}{4.345217in}}%
\pgfpathlineto{\pgfqpoint{0.449096in}{4.344279in}}%
\pgfpathlineto{\pgfqpoint{0.453392in}{4.345150in}}%
\pgfpathlineto{\pgfqpoint{0.455539in}{4.347310in}}%
\pgfpathlineto{\pgfqpoint{0.456613in}{4.347092in}}%
\pgfpathlineto{\pgfqpoint{0.460909in}{4.347477in}}%
\pgfpathlineto{\pgfqpoint{0.461982in}{4.348230in}}%
\pgfpathlineto{\pgfqpoint{0.463056in}{4.349754in}}%
\pgfpathlineto{\pgfqpoint{0.468425in}{4.348549in}}%
\pgfpathlineto{\pgfqpoint{0.469499in}{4.349436in}}%
\pgfpathlineto{\pgfqpoint{0.478090in}{4.350357in}}%
\pgfpathlineto{\pgfqpoint{0.479164in}{4.349151in}}%
\pgfpathlineto{\pgfqpoint{0.483459in}{4.350307in}}%
\pgfpathlineto{\pgfqpoint{0.484533in}{4.349771in}}%
\pgfpathlineto{\pgfqpoint{0.485607in}{4.350658in}}%
\pgfpathlineto{\pgfqpoint{0.486681in}{4.350524in}}%
\pgfpathlineto{\pgfqpoint{0.493124in}{4.350993in}}%
\pgfpathlineto{\pgfqpoint{0.494198in}{4.351412in}}%
\pgfpathlineto{\pgfqpoint{0.499567in}{4.350591in}}%
\pgfpathlineto{\pgfqpoint{0.504936in}{4.349838in}}%
\pgfpathlineto{\pgfqpoint{0.506010in}{4.346924in}}%
\pgfpathlineto{\pgfqpoint{0.507084in}{4.347644in}}%
\pgfpathlineto{\pgfqpoint{0.508157in}{4.349352in}}%
\pgfpathlineto{\pgfqpoint{0.509231in}{4.349486in}}%
\pgfpathlineto{\pgfqpoint{0.512453in}{4.350508in}}%
\pgfpathlineto{\pgfqpoint{0.513527in}{4.352182in}}%
\pgfpathlineto{\pgfqpoint{0.514601in}{4.352333in}}%
\pgfpathlineto{\pgfqpoint{0.515674in}{4.353906in}}%
\pgfpathlineto{\pgfqpoint{0.516748in}{4.353287in}}%
\pgfpathlineto{\pgfqpoint{0.519970in}{4.353538in}}%
\pgfpathlineto{\pgfqpoint{0.524265in}{4.351780in}}%
\pgfpathlineto{\pgfqpoint{0.528560in}{4.352718in}}%
\pgfpathlineto{\pgfqpoint{0.529634in}{4.351763in}}%
\pgfpathlineto{\pgfqpoint{0.531782in}{4.352801in}}%
\pgfpathlineto{\pgfqpoint{0.535003in}{4.352835in}}%
\pgfpathlineto{\pgfqpoint{0.536077in}{4.352232in}}%
\pgfpathlineto{\pgfqpoint{0.538225in}{4.350122in}}%
\pgfpathlineto{\pgfqpoint{0.542520in}{4.348833in}}%
\pgfpathlineto{\pgfqpoint{0.543594in}{4.346422in}}%
\pgfpathlineto{\pgfqpoint{0.544668in}{4.347477in}}%
\pgfpathlineto{\pgfqpoint{0.545742in}{4.349570in}}%
\pgfpathlineto{\pgfqpoint{0.546816in}{4.347946in}}%
\pgfpathlineto{\pgfqpoint{0.550037in}{4.348934in}}%
\pgfpathlineto{\pgfqpoint{0.551111in}{4.350374in}}%
\pgfpathlineto{\pgfqpoint{0.553259in}{4.349486in}}%
\pgfpathlineto{\pgfqpoint{0.554333in}{4.350424in}}%
\pgfpathlineto{\pgfqpoint{0.557554in}{4.349938in}}%
\pgfpathlineto{\pgfqpoint{0.558628in}{4.351814in}}%
\pgfpathlineto{\pgfqpoint{0.561849in}{4.353019in}}%
\pgfpathlineto{\pgfqpoint{0.568292in}{4.353053in}}%
\pgfpathlineto{\pgfqpoint{0.569366in}{4.352065in}}%
\pgfpathlineto{\pgfqpoint{0.576883in}{4.349386in}}%
\pgfpathlineto{\pgfqpoint{0.581179in}{4.348080in}}%
\pgfpathlineto{\pgfqpoint{0.582252in}{4.348297in}}%
\pgfpathlineto{\pgfqpoint{0.584400in}{4.345736in}}%
\pgfpathlineto{\pgfqpoint{0.587622in}{4.347058in}}%
\pgfpathlineto{\pgfqpoint{0.588695in}{4.346590in}}%
\pgfpathlineto{\pgfqpoint{0.590843in}{4.347778in}}%
\pgfpathlineto{\pgfqpoint{0.591917in}{4.347494in}}%
\pgfpathlineto{\pgfqpoint{0.596212in}{4.348816in}}%
\pgfpathlineto{\pgfqpoint{0.597286in}{4.347042in}}%
\pgfpathlineto{\pgfqpoint{0.598360in}{4.346991in}}%
\pgfpathlineto{\pgfqpoint{0.599434in}{4.344815in}}%
\pgfpathlineto{\pgfqpoint{0.603729in}{4.344346in}}%
\pgfpathlineto{\pgfqpoint{0.604803in}{4.347293in}}%
\pgfpathlineto{\pgfqpoint{0.606951in}{4.349202in}}%
\pgfpathlineto{\pgfqpoint{0.610172in}{4.348063in}}%
\pgfpathlineto{\pgfqpoint{0.611246in}{4.350206in}}%
\pgfpathlineto{\pgfqpoint{0.612320in}{4.349386in}}%
\pgfpathlineto{\pgfqpoint{0.614468in}{4.351194in}}%
\pgfpathlineto{\pgfqpoint{0.617689in}{4.351027in}}%
\pgfpathlineto{\pgfqpoint{0.618763in}{4.351730in}}%
\pgfpathlineto{\pgfqpoint{0.619837in}{4.351345in}}%
\pgfpathlineto{\pgfqpoint{0.620911in}{4.350206in}}%
\pgfpathlineto{\pgfqpoint{0.621984in}{4.350357in}}%
\pgfpathlineto{\pgfqpoint{0.625206in}{4.348984in}}%
\pgfpathlineto{\pgfqpoint{0.626280in}{4.349453in}}%
\pgfpathlineto{\pgfqpoint{0.627354in}{4.350809in}}%
\pgfpathlineto{\pgfqpoint{0.628427in}{4.350809in}}%
\pgfpathlineto{\pgfqpoint{0.629501in}{4.354208in}}%
\pgfpathlineto{\pgfqpoint{0.632723in}{4.353756in}}%
\pgfpathlineto{\pgfqpoint{0.633797in}{4.354342in}}%
\pgfpathlineto{\pgfqpoint{0.635944in}{4.354158in}}%
\pgfpathlineto{\pgfqpoint{0.637018in}{4.353354in}}%
\pgfpathlineto{\pgfqpoint{0.640240in}{4.353320in}}%
\pgfpathlineto{\pgfqpoint{0.643461in}{4.349771in}}%
\pgfpathlineto{\pgfqpoint{0.644535in}{4.351412in}}%
\pgfpathlineto{\pgfqpoint{0.647757in}{4.352115in}}%
\pgfpathlineto{\pgfqpoint{0.648830in}{4.353337in}}%
\pgfpathlineto{\pgfqpoint{0.649904in}{4.355966in}}%
\pgfpathlineto{\pgfqpoint{0.650978in}{4.355916in}}%
\pgfpathlineto{\pgfqpoint{0.652052in}{4.354744in}}%
\pgfpathlineto{\pgfqpoint{0.655273in}{4.353873in}}%
\pgfpathlineto{\pgfqpoint{0.656347in}{4.352316in}}%
\pgfpathlineto{\pgfqpoint{0.657421in}{4.353019in}}%
\pgfpathlineto{\pgfqpoint{0.659569in}{4.357138in}}%
\pgfpathlineto{\pgfqpoint{0.664938in}{4.356435in}}%
\pgfpathlineto{\pgfqpoint{0.666012in}{4.354878in}}%
\pgfpathlineto{\pgfqpoint{0.667086in}{4.357121in}}%
\pgfpathlineto{\pgfqpoint{0.671381in}{4.357121in}}%
\pgfpathlineto{\pgfqpoint{0.673529in}{4.356970in}}%
\pgfpathlineto{\pgfqpoint{0.674602in}{4.357942in}}%
\pgfpathlineto{\pgfqpoint{0.679972in}{4.358293in}}%
\pgfpathlineto{\pgfqpoint{0.682119in}{4.360654in}}%
\pgfpathlineto{\pgfqpoint{0.685341in}{4.360168in}}%
\pgfpathlineto{\pgfqpoint{0.686415in}{4.359231in}}%
\pgfpathlineto{\pgfqpoint{0.687489in}{4.359315in}}%
\pgfpathlineto{\pgfqpoint{0.688562in}{4.358327in}}%
\pgfpathlineto{\pgfqpoint{0.689636in}{4.359515in}}%
\pgfpathlineto{\pgfqpoint{0.695005in}{4.358963in}}%
\pgfpathlineto{\pgfqpoint{0.696079in}{4.358025in}}%
\pgfpathlineto{\pgfqpoint{0.697153in}{4.359197in}}%
\pgfpathlineto{\pgfqpoint{0.702522in}{4.358009in}}%
\pgfpathlineto{\pgfqpoint{0.703596in}{4.360152in}}%
\pgfpathlineto{\pgfqpoint{0.704670in}{4.359499in}}%
\pgfpathlineto{\pgfqpoint{0.707891in}{4.356502in}}%
\pgfpathlineto{\pgfqpoint{0.708965in}{4.357205in}}%
\pgfpathlineto{\pgfqpoint{0.710039in}{4.356686in}}%
\pgfpathlineto{\pgfqpoint{0.712187in}{4.361123in}}%
\pgfpathlineto{\pgfqpoint{0.719704in}{4.360051in}}%
\pgfpathlineto{\pgfqpoint{0.722925in}{4.360788in}}%
\pgfpathlineto{\pgfqpoint{0.723999in}{4.359515in}}%
\pgfpathlineto{\pgfqpoint{0.725073in}{4.359181in}}%
\pgfpathlineto{\pgfqpoint{0.726147in}{4.359499in}}%
\pgfpathlineto{\pgfqpoint{0.727221in}{4.358946in}}%
\pgfpathlineto{\pgfqpoint{0.733664in}{4.361759in}}%
\pgfpathlineto{\pgfqpoint{0.734737in}{4.362496in}}%
\pgfpathlineto{\pgfqpoint{0.737959in}{4.363082in}}%
\pgfpathlineto{\pgfqpoint{0.740107in}{4.360152in}}%
\pgfpathlineto{\pgfqpoint{0.742254in}{4.359415in}}%
\pgfpathlineto{\pgfqpoint{0.745476in}{4.359465in}}%
\pgfpathlineto{\pgfqpoint{0.746550in}{4.361474in}}%
\pgfpathlineto{\pgfqpoint{0.747623in}{4.362295in}}%
\pgfpathlineto{\pgfqpoint{0.748697in}{4.362194in}}%
\pgfpathlineto{\pgfqpoint{0.749771in}{4.359666in}}%
\pgfpathlineto{\pgfqpoint{0.752993in}{4.359097in}}%
\pgfpathlineto{\pgfqpoint{0.754067in}{4.353789in}}%
\pgfpathlineto{\pgfqpoint{0.757288in}{4.352801in}}%
\pgfpathlineto{\pgfqpoint{0.762657in}{4.352199in}}%
\pgfpathlineto{\pgfqpoint{0.763731in}{4.354509in}}%
\pgfpathlineto{\pgfqpoint{0.764805in}{4.354124in}}%
\pgfpathlineto{\pgfqpoint{0.768026in}{4.354995in}}%
\pgfpathlineto{\pgfqpoint{0.769100in}{4.356770in}}%
\pgfpathlineto{\pgfqpoint{0.771248in}{4.353521in}}%
\pgfpathlineto{\pgfqpoint{0.775543in}{4.354409in}}%
\pgfpathlineto{\pgfqpoint{0.776617in}{4.354275in}}%
\pgfpathlineto{\pgfqpoint{0.777691in}{4.351797in}}%
\pgfpathlineto{\pgfqpoint{0.779839in}{4.353454in}}%
\pgfpathlineto{\pgfqpoint{0.784134in}{4.355011in}}%
\pgfpathlineto{\pgfqpoint{0.785208in}{4.354894in}}%
\pgfpathlineto{\pgfqpoint{0.787356in}{4.356786in}}%
\pgfpathlineto{\pgfqpoint{0.791651in}{4.356836in}}%
\pgfpathlineto{\pgfqpoint{0.792725in}{4.357707in}}%
\pgfpathlineto{\pgfqpoint{0.793799in}{4.357305in}}%
\pgfpathlineto{\pgfqpoint{0.794872in}{4.357741in}}%
\pgfpathlineto{\pgfqpoint{0.799168in}{4.356569in}}%
\pgfpathlineto{\pgfqpoint{0.801315in}{4.357741in}}%
\pgfpathlineto{\pgfqpoint{0.802389in}{4.358528in}}%
\pgfpathlineto{\pgfqpoint{0.805611in}{4.359030in}}%
\pgfpathlineto{\pgfqpoint{0.806685in}{4.361575in}}%
\pgfpathlineto{\pgfqpoint{0.809906in}{4.359599in}}%
\pgfpathlineto{\pgfqpoint{0.813128in}{4.360671in}}%
\pgfpathlineto{\pgfqpoint{0.814201in}{4.361826in}}%
\pgfpathlineto{\pgfqpoint{0.815275in}{4.360654in}}%
\pgfpathlineto{\pgfqpoint{0.816349in}{4.362211in}}%
\pgfpathlineto{\pgfqpoint{0.817423in}{4.360754in}}%
\pgfpathlineto{\pgfqpoint{0.822792in}{4.360721in}}%
\pgfpathlineto{\pgfqpoint{0.824940in}{4.358896in}}%
\pgfpathlineto{\pgfqpoint{0.828161in}{4.360403in}}%
\pgfpathlineto{\pgfqpoint{0.830309in}{4.363132in}}%
\pgfpathlineto{\pgfqpoint{0.831383in}{4.362981in}}%
\pgfpathlineto{\pgfqpoint{0.832457in}{4.363952in}}%
\pgfpathlineto{\pgfqpoint{0.836752in}{4.364137in}}%
\pgfpathlineto{\pgfqpoint{0.838900in}{4.366096in}}%
\pgfpathlineto{\pgfqpoint{0.839974in}{4.365242in}}%
\pgfpathlineto{\pgfqpoint{0.846417in}{4.367787in}}%
\pgfpathlineto{\pgfqpoint{0.847490in}{4.368708in}}%
\pgfpathlineto{\pgfqpoint{0.853934in}{4.370013in}}%
\pgfpathlineto{\pgfqpoint{0.855007in}{4.371319in}}%
\pgfpathlineto{\pgfqpoint{0.858229in}{4.371403in}}%
\pgfpathlineto{\pgfqpoint{0.859303in}{4.373027in}}%
\pgfpathlineto{\pgfqpoint{0.860377in}{4.371604in}}%
\pgfpathlineto{\pgfqpoint{0.861450in}{4.371252in}}%
\pgfpathlineto{\pgfqpoint{0.862524in}{4.372676in}}%
\pgfpathlineto{\pgfqpoint{0.865746in}{4.371571in}}%
\pgfpathlineto{\pgfqpoint{0.867893in}{4.374266in}}%
\pgfpathlineto{\pgfqpoint{0.868967in}{4.373613in}}%
\pgfpathlineto{\pgfqpoint{0.870041in}{4.374233in}}%
\pgfpathlineto{\pgfqpoint{0.873263in}{4.374183in}}%
\pgfpathlineto{\pgfqpoint{0.874336in}{4.375355in}}%
\pgfpathlineto{\pgfqpoint{0.877558in}{4.375941in}}%
\pgfpathlineto{\pgfqpoint{0.881853in}{4.377280in}}%
\pgfpathlineto{\pgfqpoint{0.884001in}{4.375204in}}%
\pgfpathlineto{\pgfqpoint{0.885075in}{4.376376in}}%
\pgfpathlineto{\pgfqpoint{0.888296in}{4.373831in}}%
\pgfpathlineto{\pgfqpoint{0.889370in}{4.374635in}}%
\pgfpathlineto{\pgfqpoint{0.890444in}{4.376409in}}%
\pgfpathlineto{\pgfqpoint{0.891518in}{4.377029in}}%
\pgfpathlineto{\pgfqpoint{0.895813in}{4.376008in}}%
\pgfpathlineto{\pgfqpoint{0.896887in}{4.377665in}}%
\pgfpathlineto{\pgfqpoint{0.899035in}{4.377782in}}%
\pgfpathlineto{\pgfqpoint{0.900109in}{4.379457in}}%
\pgfpathlineto{\pgfqpoint{0.903330in}{4.379591in}}%
\pgfpathlineto{\pgfqpoint{0.904404in}{4.378620in}}%
\pgfpathlineto{\pgfqpoint{0.905478in}{4.378569in}}%
\pgfpathlineto{\pgfqpoint{0.907625in}{4.380428in}}%
\pgfpathlineto{\pgfqpoint{0.911921in}{4.378703in}}%
\pgfpathlineto{\pgfqpoint{0.912995in}{4.379373in}}%
\pgfpathlineto{\pgfqpoint{0.914068in}{4.378352in}}%
\pgfpathlineto{\pgfqpoint{0.915142in}{4.380461in}}%
\pgfpathlineto{\pgfqpoint{0.918364in}{4.378687in}}%
\pgfpathlineto{\pgfqpoint{0.919438in}{4.379959in}}%
\pgfpathlineto{\pgfqpoint{0.920511in}{4.378854in}}%
\pgfpathlineto{\pgfqpoint{0.921585in}{4.380294in}}%
\pgfpathlineto{\pgfqpoint{0.925881in}{4.379356in}}%
\pgfpathlineto{\pgfqpoint{0.926955in}{4.380595in}}%
\pgfpathlineto{\pgfqpoint{0.928028in}{4.379406in}}%
\pgfpathlineto{\pgfqpoint{0.930176in}{4.379540in}}%
\pgfpathlineto{\pgfqpoint{0.934471in}{4.379792in}}%
\pgfpathlineto{\pgfqpoint{0.935545in}{4.382253in}}%
\pgfpathlineto{\pgfqpoint{0.936619in}{4.383057in}}%
\pgfpathlineto{\pgfqpoint{0.940914in}{4.379524in}}%
\pgfpathlineto{\pgfqpoint{0.941988in}{4.380076in}}%
\pgfpathlineto{\pgfqpoint{0.944136in}{4.378419in}}%
\pgfpathlineto{\pgfqpoint{0.945210in}{4.379457in}}%
\pgfpathlineto{\pgfqpoint{0.948431in}{4.379574in}}%
\pgfpathlineto{\pgfqpoint{0.949505in}{4.381834in}}%
\pgfpathlineto{\pgfqpoint{0.950579in}{4.382504in}}%
\pgfpathlineto{\pgfqpoint{0.951653in}{4.378268in}}%
\pgfpathlineto{\pgfqpoint{0.952727in}{4.376728in}}%
\pgfpathlineto{\pgfqpoint{0.955948in}{4.376778in}}%
\pgfpathlineto{\pgfqpoint{0.957022in}{4.378034in}}%
\pgfpathlineto{\pgfqpoint{0.958096in}{4.377799in}}%
\pgfpathlineto{\pgfqpoint{0.960244in}{4.382454in}}%
\pgfpathlineto{\pgfqpoint{0.965613in}{4.382973in}}%
\pgfpathlineto{\pgfqpoint{0.966687in}{4.385417in}}%
\pgfpathlineto{\pgfqpoint{0.967760in}{4.386204in}}%
\pgfpathlineto{\pgfqpoint{0.972056in}{4.386372in}}%
\pgfpathlineto{\pgfqpoint{0.973130in}{4.387661in}}%
\pgfpathlineto{\pgfqpoint{0.974203in}{4.387041in}}%
\pgfpathlineto{\pgfqpoint{0.975277in}{4.387494in}}%
\pgfpathlineto{\pgfqpoint{0.979573in}{4.388381in}}%
\pgfpathlineto{\pgfqpoint{0.981720in}{4.387008in}}%
\pgfpathlineto{\pgfqpoint{0.982794in}{4.386807in}}%
\pgfpathlineto{\pgfqpoint{0.987089in}{4.388682in}}%
\pgfpathlineto{\pgfqpoint{0.988163in}{4.388029in}}%
\pgfpathlineto{\pgfqpoint{0.989237in}{4.388448in}}%
\pgfpathlineto{\pgfqpoint{0.990311in}{4.386807in}}%
\pgfpathlineto{\pgfqpoint{0.993533in}{4.387293in}}%
\pgfpathlineto{\pgfqpoint{0.994606in}{4.386455in}}%
\pgfpathlineto{\pgfqpoint{0.995680in}{4.384413in}}%
\pgfpathlineto{\pgfqpoint{0.996754in}{4.384530in}}%
\pgfpathlineto{\pgfqpoint{0.997828in}{4.387996in}}%
\pgfpathlineto{\pgfqpoint{1.001049in}{4.387577in}}%
\pgfpathlineto{\pgfqpoint{1.002123in}{4.386740in}}%
\pgfpathlineto{\pgfqpoint{1.003197in}{4.385032in}}%
\pgfpathlineto{\pgfqpoint{1.004271in}{4.388130in}}%
\pgfpathlineto{\pgfqpoint{1.005345in}{4.387895in}}%
\pgfpathlineto{\pgfqpoint{1.008566in}{4.389151in}}%
\pgfpathlineto{\pgfqpoint{1.009640in}{4.390658in}}%
\pgfpathlineto{\pgfqpoint{1.010714in}{4.388666in}}%
\pgfpathlineto{\pgfqpoint{1.011788in}{4.384681in}}%
\pgfpathlineto{\pgfqpoint{1.012862in}{4.385836in}}%
\pgfpathlineto{\pgfqpoint{1.016083in}{4.382856in}}%
\pgfpathlineto{\pgfqpoint{1.017157in}{4.383910in}}%
\pgfpathlineto{\pgfqpoint{1.018231in}{4.385970in}}%
\pgfpathlineto{\pgfqpoint{1.019305in}{4.386757in}}%
\pgfpathlineto{\pgfqpoint{1.020378in}{4.385501in}}%
\pgfpathlineto{\pgfqpoint{1.023600in}{4.385434in}}%
\pgfpathlineto{\pgfqpoint{1.024674in}{4.384614in}}%
\pgfpathlineto{\pgfqpoint{1.025748in}{4.385635in}}%
\pgfpathlineto{\pgfqpoint{1.027895in}{4.388615in}}%
\pgfpathlineto{\pgfqpoint{1.031117in}{4.389452in}}%
\pgfpathlineto{\pgfqpoint{1.032191in}{4.391194in}}%
\pgfpathlineto{\pgfqpoint{1.033265in}{4.391311in}}%
\pgfpathlineto{\pgfqpoint{1.035412in}{4.393755in}}%
\pgfpathlineto{\pgfqpoint{1.038634in}{4.393337in}}%
\pgfpathlineto{\pgfqpoint{1.039708in}{4.392600in}}%
\pgfpathlineto{\pgfqpoint{1.040781in}{4.393036in}}%
\pgfpathlineto{\pgfqpoint{1.042929in}{4.395246in}}%
\pgfpathlineto{\pgfqpoint{1.046151in}{4.395380in}}%
\pgfpathlineto{\pgfqpoint{1.047224in}{4.396049in}}%
\pgfpathlineto{\pgfqpoint{1.048298in}{4.395447in}}%
\pgfpathlineto{\pgfqpoint{1.050446in}{4.396267in}}%
\pgfpathlineto{\pgfqpoint{1.054741in}{4.396167in}}%
\pgfpathlineto{\pgfqpoint{1.057963in}{4.398192in}}%
\pgfpathlineto{\pgfqpoint{1.063332in}{4.397590in}}%
\pgfpathlineto{\pgfqpoint{1.064406in}{4.398795in}}%
\pgfpathlineto{\pgfqpoint{1.065480in}{4.398276in}}%
\pgfpathlineto{\pgfqpoint{1.069775in}{4.398778in}}%
\pgfpathlineto{\pgfqpoint{1.070849in}{4.397288in}}%
\pgfpathlineto{\pgfqpoint{1.071923in}{4.394760in}}%
\pgfpathlineto{\pgfqpoint{1.072997in}{4.394827in}}%
\pgfpathlineto{\pgfqpoint{1.077292in}{4.394141in}}%
\pgfpathlineto{\pgfqpoint{1.078366in}{4.392198in}}%
\pgfpathlineto{\pgfqpoint{1.079440in}{4.393990in}}%
\pgfpathlineto{\pgfqpoint{1.080513in}{4.393588in}}%
\pgfpathlineto{\pgfqpoint{1.083735in}{4.393488in}}%
\pgfpathlineto{\pgfqpoint{1.084809in}{4.391194in}}%
\pgfpathlineto{\pgfqpoint{1.088030in}{4.392416in}}%
\pgfpathlineto{\pgfqpoint{1.092326in}{4.391897in}}%
\pgfpathlineto{\pgfqpoint{1.093399in}{4.393806in}}%
\pgfpathlineto{\pgfqpoint{1.095547in}{4.394509in}}%
\pgfpathlineto{\pgfqpoint{1.099843in}{4.398176in}}%
\pgfpathlineto{\pgfqpoint{1.100916in}{4.399766in}}%
\pgfpathlineto{\pgfqpoint{1.101990in}{4.399046in}}%
\pgfpathlineto{\pgfqpoint{1.103064in}{4.399616in}}%
\pgfpathlineto{\pgfqpoint{1.106286in}{4.400520in}}%
\pgfpathlineto{\pgfqpoint{1.107359in}{4.401524in}}%
\pgfpathlineto{\pgfqpoint{1.108433in}{4.403467in}}%
\pgfpathlineto{\pgfqpoint{1.109507in}{4.403868in}}%
\pgfpathlineto{\pgfqpoint{1.110581in}{4.401642in}}%
\pgfpathlineto{\pgfqpoint{1.113802in}{4.403215in}}%
\pgfpathlineto{\pgfqpoint{1.115950in}{4.401909in}}%
\pgfpathlineto{\pgfqpoint{1.117024in}{4.402562in}}%
\pgfpathlineto{\pgfqpoint{1.118098in}{4.401943in}}%
\pgfpathlineto{\pgfqpoint{1.121319in}{4.400771in}}%
\pgfpathlineto{\pgfqpoint{1.122393in}{4.401072in}}%
\pgfpathlineto{\pgfqpoint{1.124541in}{4.399549in}}%
\pgfpathlineto{\pgfqpoint{1.125615in}{4.400771in}}%
\pgfpathlineto{\pgfqpoint{1.128836in}{4.399750in}}%
\pgfpathlineto{\pgfqpoint{1.129910in}{4.397556in}}%
\pgfpathlineto{\pgfqpoint{1.130984in}{4.398142in}}%
\pgfpathlineto{\pgfqpoint{1.133132in}{4.402663in}}%
\pgfpathlineto{\pgfqpoint{1.136353in}{4.403634in}}%
\pgfpathlineto{\pgfqpoint{1.137427in}{4.401374in}}%
\pgfpathlineto{\pgfqpoint{1.140648in}{4.405693in}}%
\pgfpathlineto{\pgfqpoint{1.143870in}{4.406279in}}%
\pgfpathlineto{\pgfqpoint{1.144944in}{4.407066in}}%
\pgfpathlineto{\pgfqpoint{1.146018in}{4.406213in}}%
\pgfpathlineto{\pgfqpoint{1.147091in}{4.406631in}}%
\pgfpathlineto{\pgfqpoint{1.148165in}{4.407954in}}%
\pgfpathlineto{\pgfqpoint{1.152461in}{4.409193in}}%
\pgfpathlineto{\pgfqpoint{1.153534in}{4.408490in}}%
\pgfpathlineto{\pgfqpoint{1.154608in}{4.410013in}}%
\pgfpathlineto{\pgfqpoint{1.159977in}{4.410398in}}%
\pgfpathlineto{\pgfqpoint{1.161051in}{4.411838in}}%
\pgfpathlineto{\pgfqpoint{1.162125in}{4.410817in}}%
\pgfpathlineto{\pgfqpoint{1.163199in}{4.413077in}}%
\pgfpathlineto{\pgfqpoint{1.166421in}{4.413027in}}%
\pgfpathlineto{\pgfqpoint{1.168568in}{4.413931in}}%
\pgfpathlineto{\pgfqpoint{1.169642in}{4.415656in}}%
\pgfpathlineto{\pgfqpoint{1.176085in}{4.415689in}}%
\pgfpathlineto{\pgfqpoint{1.178233in}{4.418351in}}%
\pgfpathlineto{\pgfqpoint{1.181454in}{4.418653in}}%
\pgfpathlineto{\pgfqpoint{1.183602in}{4.421918in}}%
\pgfpathlineto{\pgfqpoint{1.185750in}{4.421934in}}%
\pgfpathlineto{\pgfqpoint{1.190045in}{4.411972in}}%
\pgfpathlineto{\pgfqpoint{1.191119in}{4.411771in}}%
\pgfpathlineto{\pgfqpoint{1.192193in}{4.412307in}}%
\pgfpathlineto{\pgfqpoint{1.193266in}{4.414869in}}%
\pgfpathlineto{\pgfqpoint{1.196488in}{4.414819in}}%
\pgfpathlineto{\pgfqpoint{1.198636in}{4.412223in}}%
\pgfpathlineto{\pgfqpoint{1.200783in}{4.411738in}}%
\pgfpathlineto{\pgfqpoint{1.204005in}{4.413496in}}%
\pgfpathlineto{\pgfqpoint{1.206153in}{4.425233in}}%
\pgfpathlineto{\pgfqpoint{1.208300in}{4.426556in}}%
\pgfpathlineto{\pgfqpoint{1.212596in}{4.426941in}}%
\pgfpathlineto{\pgfqpoint{1.215817in}{4.430340in}}%
\pgfpathlineto{\pgfqpoint{1.219039in}{4.430440in}}%
\pgfpathlineto{\pgfqpoint{1.220112in}{4.431646in}}%
\pgfpathlineto{\pgfqpoint{1.222260in}{4.428582in}}%
\pgfpathlineto{\pgfqpoint{1.223334in}{4.429050in}}%
\pgfpathlineto{\pgfqpoint{1.227629in}{4.427895in}}%
\pgfpathlineto{\pgfqpoint{1.228703in}{4.426422in}}%
\pgfpathlineto{\pgfqpoint{1.230851in}{4.425785in}}%
\pgfpathlineto{\pgfqpoint{1.234072in}{4.423626in}}%
\pgfpathlineto{\pgfqpoint{1.235146in}{4.427544in}}%
\pgfpathlineto{\pgfqpoint{1.236220in}{4.429034in}}%
\pgfpathlineto{\pgfqpoint{1.237294in}{4.428632in}}%
\pgfpathlineto{\pgfqpoint{1.238368in}{4.427410in}}%
\pgfpathlineto{\pgfqpoint{1.242663in}{4.426941in}}%
\pgfpathlineto{\pgfqpoint{1.243737in}{4.426204in}}%
\pgfpathlineto{\pgfqpoint{1.244811in}{4.423659in}}%
\pgfpathlineto{\pgfqpoint{1.245885in}{4.417196in}}%
\pgfpathlineto{\pgfqpoint{1.249106in}{4.415371in}}%
\pgfpathlineto{\pgfqpoint{1.251254in}{4.417230in}}%
\pgfpathlineto{\pgfqpoint{1.252328in}{4.414065in}}%
\pgfpathlineto{\pgfqpoint{1.253401in}{4.414266in}}%
\pgfpathlineto{\pgfqpoint{1.256623in}{4.408088in}}%
\pgfpathlineto{\pgfqpoint{1.257697in}{4.412156in}}%
\pgfpathlineto{\pgfqpoint{1.258771in}{4.413077in}}%
\pgfpathlineto{\pgfqpoint{1.260918in}{4.417347in}}%
\pgfpathlineto{\pgfqpoint{1.264140in}{4.416443in}}%
\pgfpathlineto{\pgfqpoint{1.265214in}{4.418284in}}%
\pgfpathlineto{\pgfqpoint{1.266287in}{4.418737in}}%
\pgfpathlineto{\pgfqpoint{1.267361in}{4.418301in}}%
\pgfpathlineto{\pgfqpoint{1.268435in}{4.421181in}}%
\pgfpathlineto{\pgfqpoint{1.272731in}{4.420712in}}%
\pgfpathlineto{\pgfqpoint{1.273804in}{4.418921in}}%
\pgfpathlineto{\pgfqpoint{1.274878in}{4.420361in}}%
\pgfpathlineto{\pgfqpoint{1.275952in}{4.420377in}}%
\pgfpathlineto{\pgfqpoint{1.279174in}{4.421298in}}%
\pgfpathlineto{\pgfqpoint{1.280247in}{4.422353in}}%
\pgfpathlineto{\pgfqpoint{1.281321in}{4.422253in}}%
\pgfpathlineto{\pgfqpoint{1.282395in}{4.424396in}}%
\pgfpathlineto{\pgfqpoint{1.283469in}{4.424965in}}%
\pgfpathlineto{\pgfqpoint{1.286690in}{4.421315in}}%
\pgfpathlineto{\pgfqpoint{1.287764in}{4.421985in}}%
\pgfpathlineto{\pgfqpoint{1.288838in}{4.423693in}}%
\pgfpathlineto{\pgfqpoint{1.290986in}{4.424061in}}%
\pgfpathlineto{\pgfqpoint{1.294207in}{4.423274in}}%
\pgfpathlineto{\pgfqpoint{1.295281in}{4.421750in}}%
\pgfpathlineto{\pgfqpoint{1.296355in}{4.421834in}}%
\pgfpathlineto{\pgfqpoint{1.298503in}{4.417866in}}%
\pgfpathlineto{\pgfqpoint{1.302798in}{4.422085in}}%
\pgfpathlineto{\pgfqpoint{1.303872in}{4.419892in}}%
\pgfpathlineto{\pgfqpoint{1.306020in}{4.422638in}}%
\pgfpathlineto{\pgfqpoint{1.309241in}{4.421616in}}%
\pgfpathlineto{\pgfqpoint{1.310315in}{4.423994in}}%
\pgfpathlineto{\pgfqpoint{1.311389in}{4.422587in}}%
\pgfpathlineto{\pgfqpoint{1.312463in}{4.422219in}}%
\pgfpathlineto{\pgfqpoint{1.313536in}{4.424195in}}%
\pgfpathlineto{\pgfqpoint{1.317832in}{4.427577in}}%
\pgfpathlineto{\pgfqpoint{1.318906in}{4.426807in}}%
\pgfpathlineto{\pgfqpoint{1.319979in}{4.427008in}}%
\pgfpathlineto{\pgfqpoint{1.321053in}{4.426606in}}%
\pgfpathlineto{\pgfqpoint{1.324275in}{4.424479in}}%
\pgfpathlineto{\pgfqpoint{1.325349in}{4.425132in}}%
\pgfpathlineto{\pgfqpoint{1.326422in}{4.426572in}}%
\pgfpathlineto{\pgfqpoint{1.328570in}{4.421566in}}%
\pgfpathlineto{\pgfqpoint{1.331792in}{4.422671in}}%
\pgfpathlineto{\pgfqpoint{1.332865in}{4.424044in}}%
\pgfpathlineto{\pgfqpoint{1.333939in}{4.427929in}}%
\pgfpathlineto{\pgfqpoint{1.335013in}{4.429318in}}%
\pgfpathlineto{\pgfqpoint{1.340382in}{4.430993in}}%
\pgfpathlineto{\pgfqpoint{1.343604in}{4.427610in}}%
\pgfpathlineto{\pgfqpoint{1.347899in}{4.429050in}}%
\pgfpathlineto{\pgfqpoint{1.350047in}{4.433789in}}%
\pgfpathlineto{\pgfqpoint{1.351121in}{4.432784in}}%
\pgfpathlineto{\pgfqpoint{1.354342in}{4.433487in}}%
\pgfpathlineto{\pgfqpoint{1.355416in}{4.431579in}}%
\pgfpathlineto{\pgfqpoint{1.356490in}{4.434258in}}%
\pgfpathlineto{\pgfqpoint{1.357564in}{4.433806in}}%
\pgfpathlineto{\pgfqpoint{1.361859in}{4.436702in}}%
\pgfpathlineto{\pgfqpoint{1.362933in}{4.436133in}}%
\pgfpathlineto{\pgfqpoint{1.365081in}{4.434023in}}%
\pgfpathlineto{\pgfqpoint{1.369376in}{4.435028in}}%
\pgfpathlineto{\pgfqpoint{1.370450in}{4.432935in}}%
\pgfpathlineto{\pgfqpoint{1.371524in}{4.434810in}}%
\pgfpathlineto{\pgfqpoint{1.372598in}{4.434308in}}%
\pgfpathlineto{\pgfqpoint{1.373671in}{4.435513in}}%
\pgfpathlineto{\pgfqpoint{1.379041in}{4.435932in}}%
\pgfpathlineto{\pgfqpoint{1.380114in}{4.437338in}}%
\pgfpathlineto{\pgfqpoint{1.381188in}{4.437573in}}%
\pgfpathlineto{\pgfqpoint{1.384410in}{4.437238in}}%
\pgfpathlineto{\pgfqpoint{1.385484in}{4.438075in}}%
\pgfpathlineto{\pgfqpoint{1.386557in}{4.437154in}}%
\pgfpathlineto{\pgfqpoint{1.388705in}{4.440620in}}%
\pgfpathlineto{\pgfqpoint{1.391927in}{4.441608in}}%
\pgfpathlineto{\pgfqpoint{1.394074in}{4.440285in}}%
\pgfpathlineto{\pgfqpoint{1.395148in}{4.438326in}}%
\pgfpathlineto{\pgfqpoint{1.396222in}{4.438762in}}%
\pgfpathlineto{\pgfqpoint{1.399443in}{4.438695in}}%
\pgfpathlineto{\pgfqpoint{1.403739in}{4.441373in}}%
\pgfpathlineto{\pgfqpoint{1.406960in}{4.439816in}}%
\pgfpathlineto{\pgfqpoint{1.408034in}{4.438360in}}%
\pgfpathlineto{\pgfqpoint{1.409108in}{4.439180in}}%
\pgfpathlineto{\pgfqpoint{1.411256in}{4.439230in}}%
\pgfpathlineto{\pgfqpoint{1.414477in}{4.438577in}}%
\pgfpathlineto{\pgfqpoint{1.416625in}{4.441491in}}%
\pgfpathlineto{\pgfqpoint{1.417699in}{4.441742in}}%
\pgfpathlineto{\pgfqpoint{1.424142in}{4.440687in}}%
\pgfpathlineto{\pgfqpoint{1.425216in}{4.439532in}}%
\pgfpathlineto{\pgfqpoint{1.426289in}{4.440134in}}%
\pgfpathlineto{\pgfqpoint{1.430585in}{4.441240in}}%
\pgfpathlineto{\pgfqpoint{1.431659in}{4.442847in}}%
\pgfpathlineto{\pgfqpoint{1.432732in}{4.439013in}}%
\pgfpathlineto{\pgfqpoint{1.433806in}{4.440905in}}%
\pgfpathlineto{\pgfqpoint{1.437028in}{4.440134in}}%
\pgfpathlineto{\pgfqpoint{1.438102in}{4.441323in}}%
\pgfpathlineto{\pgfqpoint{1.439175in}{4.440687in}}%
\pgfpathlineto{\pgfqpoint{1.440249in}{4.441340in}}%
\pgfpathlineto{\pgfqpoint{1.441323in}{4.441323in}}%
\pgfpathlineto{\pgfqpoint{1.444545in}{4.441926in}}%
\pgfpathlineto{\pgfqpoint{1.445619in}{4.439716in}}%
\pgfpathlineto{\pgfqpoint{1.446692in}{4.439314in}}%
\pgfpathlineto{\pgfqpoint{1.447766in}{4.435145in}}%
\pgfpathlineto{\pgfqpoint{1.448840in}{4.434006in}}%
\pgfpathlineto{\pgfqpoint{1.452062in}{4.434944in}}%
\pgfpathlineto{\pgfqpoint{1.453135in}{4.433538in}}%
\pgfpathlineto{\pgfqpoint{1.455283in}{4.432583in}}%
\pgfpathlineto{\pgfqpoint{1.456357in}{4.435095in}}%
\pgfpathlineto{\pgfqpoint{1.459578in}{4.434693in}}%
\pgfpathlineto{\pgfqpoint{1.460652in}{4.435128in}}%
\pgfpathlineto{\pgfqpoint{1.462800in}{4.437271in}}%
\pgfpathlineto{\pgfqpoint{1.463874in}{4.436652in}}%
\pgfpathlineto{\pgfqpoint{1.467095in}{4.439867in}}%
\pgfpathlineto{\pgfqpoint{1.468169in}{4.440051in}}%
\pgfpathlineto{\pgfqpoint{1.469243in}{4.441826in}}%
\pgfpathlineto{\pgfqpoint{1.471391in}{4.441139in}}%
\pgfpathlineto{\pgfqpoint{1.475686in}{4.441826in}}%
\pgfpathlineto{\pgfqpoint{1.476760in}{4.440854in}}%
\pgfpathlineto{\pgfqpoint{1.484277in}{4.440670in}}%
\pgfpathlineto{\pgfqpoint{1.485351in}{4.440369in}}%
\pgfpathlineto{\pgfqpoint{1.486424in}{4.441240in}}%
\pgfpathlineto{\pgfqpoint{1.489646in}{4.442244in}}%
\pgfpathlineto{\pgfqpoint{1.490720in}{4.441625in}}%
\pgfpathlineto{\pgfqpoint{1.491794in}{4.441876in}}%
\pgfpathlineto{\pgfqpoint{1.493941in}{4.440854in}}%
\pgfpathlineto{\pgfqpoint{1.498237in}{4.442194in}}%
\pgfpathlineto{\pgfqpoint{1.499310in}{4.442947in}}%
\pgfpathlineto{\pgfqpoint{1.500384in}{4.445107in}}%
\pgfpathlineto{\pgfqpoint{1.501458in}{4.444890in}}%
\pgfpathlineto{\pgfqpoint{1.504680in}{4.443450in}}%
\pgfpathlineto{\pgfqpoint{1.505753in}{4.441491in}}%
\pgfpathlineto{\pgfqpoint{1.506827in}{4.442227in}}%
\pgfpathlineto{\pgfqpoint{1.507901in}{4.438711in}}%
\pgfpathlineto{\pgfqpoint{1.512197in}{4.438293in}}%
\pgfpathlineto{\pgfqpoint{1.513270in}{4.437539in}}%
\pgfpathlineto{\pgfqpoint{1.514344in}{4.433872in}}%
\pgfpathlineto{\pgfqpoint{1.515418in}{4.433119in}}%
\pgfpathlineto{\pgfqpoint{1.516492in}{4.435312in}}%
\pgfpathlineto{\pgfqpoint{1.519713in}{4.435564in}}%
\pgfpathlineto{\pgfqpoint{1.520787in}{4.431662in}}%
\pgfpathlineto{\pgfqpoint{1.521861in}{4.437137in}}%
\pgfpathlineto{\pgfqpoint{1.522935in}{4.433069in}}%
\pgfpathlineto{\pgfqpoint{1.524009in}{4.426020in}}%
\pgfpathlineto{\pgfqpoint{1.527230in}{4.424647in}}%
\pgfpathlineto{\pgfqpoint{1.528304in}{4.426522in}}%
\pgfpathlineto{\pgfqpoint{1.529378in}{4.426589in}}%
\pgfpathlineto{\pgfqpoint{1.530452in}{4.427811in}}%
\pgfpathlineto{\pgfqpoint{1.531526in}{4.431261in}}%
\pgfpathlineto{\pgfqpoint{1.534747in}{4.431545in}}%
\pgfpathlineto{\pgfqpoint{1.535821in}{4.436434in}}%
\pgfpathlineto{\pgfqpoint{1.536895in}{4.433521in}}%
\pgfpathlineto{\pgfqpoint{1.539042in}{4.447686in}}%
\pgfpathlineto{\pgfqpoint{1.542264in}{4.449109in}}%
\pgfpathlineto{\pgfqpoint{1.543338in}{4.451302in}}%
\pgfpathlineto{\pgfqpoint{1.544412in}{4.451235in}}%
\pgfpathlineto{\pgfqpoint{1.546559in}{4.455287in}}%
\pgfpathlineto{\pgfqpoint{1.549781in}{4.454584in}}%
\pgfpathlineto{\pgfqpoint{1.550855in}{4.457162in}}%
\pgfpathlineto{\pgfqpoint{1.554076in}{4.459356in}}%
\pgfpathlineto{\pgfqpoint{1.557298in}{4.460997in}}%
\pgfpathlineto{\pgfqpoint{1.558372in}{4.460176in}}%
\pgfpathlineto{\pgfqpoint{1.561593in}{4.462738in}}%
\pgfpathlineto{\pgfqpoint{1.564815in}{4.462504in}}%
\pgfpathlineto{\pgfqpoint{1.565888in}{4.464647in}}%
\pgfpathlineto{\pgfqpoint{1.566962in}{4.463960in}}%
\pgfpathlineto{\pgfqpoint{1.569110in}{4.465919in}}%
\pgfpathlineto{\pgfqpoint{1.572331in}{4.465501in}}%
\pgfpathlineto{\pgfqpoint{1.573405in}{4.462822in}}%
\pgfpathlineto{\pgfqpoint{1.574479in}{4.463190in}}%
\pgfpathlineto{\pgfqpoint{1.576627in}{4.465819in}}%
\pgfpathlineto{\pgfqpoint{1.579848in}{4.462972in}}%
\pgfpathlineto{\pgfqpoint{1.581996in}{4.469017in}}%
\pgfpathlineto{\pgfqpoint{1.584144in}{4.469033in}}%
\pgfpathlineto{\pgfqpoint{1.588439in}{4.466907in}}%
\pgfpathlineto{\pgfqpoint{1.589513in}{4.463090in}}%
\pgfpathlineto{\pgfqpoint{1.590587in}{4.464429in}}%
\pgfpathlineto{\pgfqpoint{1.591661in}{4.461432in}}%
\pgfpathlineto{\pgfqpoint{1.594882in}{4.461030in}}%
\pgfpathlineto{\pgfqpoint{1.597030in}{4.466572in}}%
\pgfpathlineto{\pgfqpoint{1.598104in}{4.473504in}}%
\pgfpathlineto{\pgfqpoint{1.599177in}{4.473772in}}%
\pgfpathlineto{\pgfqpoint{1.602399in}{4.476417in}}%
\pgfpathlineto{\pgfqpoint{1.603473in}{4.475831in}}%
\pgfpathlineto{\pgfqpoint{1.604547in}{4.475948in}}%
\pgfpathlineto{\pgfqpoint{1.606694in}{4.474927in}}%
\pgfpathlineto{\pgfqpoint{1.609916in}{4.475580in}}%
\pgfpathlineto{\pgfqpoint{1.610990in}{4.474308in}}%
\pgfpathlineto{\pgfqpoint{1.612063in}{4.472064in}}%
\pgfpathlineto{\pgfqpoint{1.614211in}{4.471679in}}%
\pgfpathlineto{\pgfqpoint{1.618507in}{4.463692in}}%
\pgfpathlineto{\pgfqpoint{1.619580in}{4.465383in}}%
\pgfpathlineto{\pgfqpoint{1.620654in}{4.471043in}}%
\pgfpathlineto{\pgfqpoint{1.621728in}{4.468079in}}%
\pgfpathlineto{\pgfqpoint{1.628171in}{4.465182in}}%
\pgfpathlineto{\pgfqpoint{1.629245in}{4.468632in}}%
\pgfpathlineto{\pgfqpoint{1.633540in}{4.468548in}}%
\pgfpathlineto{\pgfqpoint{1.634614in}{4.469435in}}%
\pgfpathlineto{\pgfqpoint{1.635688in}{4.474375in}}%
\pgfpathlineto{\pgfqpoint{1.636762in}{4.471612in}}%
\pgfpathlineto{\pgfqpoint{1.639983in}{4.471947in}}%
\pgfpathlineto{\pgfqpoint{1.641057in}{4.471043in}}%
\pgfpathlineto{\pgfqpoint{1.642131in}{4.471495in}}%
\pgfpathlineto{\pgfqpoint{1.643205in}{4.474726in}}%
\pgfpathlineto{\pgfqpoint{1.644279in}{4.469084in}}%
\pgfpathlineto{\pgfqpoint{1.647500in}{4.472215in}}%
\pgfpathlineto{\pgfqpoint{1.648574in}{4.474458in}}%
\pgfpathlineto{\pgfqpoint{1.649648in}{4.472784in}}%
\pgfpathlineto{\pgfqpoint{1.650722in}{4.475346in}}%
\pgfpathlineto{\pgfqpoint{1.655017in}{4.472834in}}%
\pgfpathlineto{\pgfqpoint{1.656091in}{4.474274in}}%
\pgfpathlineto{\pgfqpoint{1.657165in}{4.473805in}}%
\pgfpathlineto{\pgfqpoint{1.658239in}{4.475932in}}%
\pgfpathlineto{\pgfqpoint{1.659312in}{4.475982in}}%
\pgfpathlineto{\pgfqpoint{1.665755in}{4.477673in}}%
\pgfpathlineto{\pgfqpoint{1.666829in}{4.479213in}}%
\pgfpathlineto{\pgfqpoint{1.673272in}{4.481423in}}%
\pgfpathlineto{\pgfqpoint{1.674346in}{4.480000in}}%
\pgfpathlineto{\pgfqpoint{1.677568in}{4.482746in}}%
\pgfpathlineto{\pgfqpoint{1.679715in}{4.477790in}}%
\pgfpathlineto{\pgfqpoint{1.680789in}{4.478393in}}%
\pgfpathlineto{\pgfqpoint{1.681863in}{4.473621in}}%
\pgfpathlineto{\pgfqpoint{1.685085in}{4.476618in}}%
\pgfpathlineto{\pgfqpoint{1.686158in}{4.470507in}}%
\pgfpathlineto{\pgfqpoint{1.687232in}{4.469720in}}%
\pgfpathlineto{\pgfqpoint{1.688306in}{4.473789in}}%
\pgfpathlineto{\pgfqpoint{1.689380in}{4.471227in}}%
\pgfpathlineto{\pgfqpoint{1.692601in}{4.476367in}}%
\pgfpathlineto{\pgfqpoint{1.693675in}{4.473437in}}%
\pgfpathlineto{\pgfqpoint{1.694749in}{4.476769in}}%
\pgfpathlineto{\pgfqpoint{1.695823in}{4.475563in}}%
\pgfpathlineto{\pgfqpoint{1.696897in}{4.476786in}}%
\pgfpathlineto{\pgfqpoint{1.701192in}{4.476534in}}%
\pgfpathlineto{\pgfqpoint{1.702266in}{4.471160in}}%
\pgfpathlineto{\pgfqpoint{1.703340in}{4.470992in}}%
\pgfpathlineto{\pgfqpoint{1.707635in}{4.476133in}}%
\pgfpathlineto{\pgfqpoint{1.708709in}{4.474508in}}%
\pgfpathlineto{\pgfqpoint{1.709783in}{4.470892in}}%
\pgfpathlineto{\pgfqpoint{1.710857in}{4.471311in}}%
\pgfpathlineto{\pgfqpoint{1.716226in}{4.476367in}}%
\pgfpathlineto{\pgfqpoint{1.717300in}{4.476467in}}%
\pgfpathlineto{\pgfqpoint{1.719447in}{4.477656in}}%
\pgfpathlineto{\pgfqpoint{1.722669in}{4.475831in}}%
\pgfpathlineto{\pgfqpoint{1.724817in}{4.476719in}}%
\pgfpathlineto{\pgfqpoint{1.725890in}{4.475898in}}%
\pgfpathlineto{\pgfqpoint{1.726964in}{4.469686in}}%
\pgfpathlineto{\pgfqpoint{1.730186in}{4.473856in}}%
\pgfpathlineto{\pgfqpoint{1.731260in}{4.473085in}}%
\pgfpathlineto{\pgfqpoint{1.732333in}{4.474090in}}%
\pgfpathlineto{\pgfqpoint{1.733407in}{4.466656in}}%
\pgfpathlineto{\pgfqpoint{1.734481in}{4.465651in}}%
\pgfpathlineto{\pgfqpoint{1.737703in}{4.464161in}}%
\pgfpathlineto{\pgfqpoint{1.738776in}{4.464680in}}%
\pgfpathlineto{\pgfqpoint{1.740924in}{4.461800in}}%
\pgfpathlineto{\pgfqpoint{1.741998in}{4.463709in}}%
\pgfpathlineto{\pgfqpoint{1.745219in}{4.465702in}}%
\pgfpathlineto{\pgfqpoint{1.746293in}{4.464044in}}%
\pgfpathlineto{\pgfqpoint{1.747367in}{4.463642in}}%
\pgfpathlineto{\pgfqpoint{1.748441in}{4.465082in}}%
\pgfpathlineto{\pgfqpoint{1.749515in}{4.468046in}}%
\pgfpathlineto{\pgfqpoint{1.752736in}{4.467007in}}%
\pgfpathlineto{\pgfqpoint{1.753810in}{4.467259in}}%
\pgfpathlineto{\pgfqpoint{1.757032in}{4.472047in}}%
\pgfpathlineto{\pgfqpoint{1.760253in}{4.471411in}}%
\pgfpathlineto{\pgfqpoint{1.761327in}{4.471779in}}%
\pgfpathlineto{\pgfqpoint{1.763475in}{4.471729in}}%
\pgfpathlineto{\pgfqpoint{1.764549in}{4.470122in}}%
\pgfpathlineto{\pgfqpoint{1.768844in}{4.468046in}}%
\pgfpathlineto{\pgfqpoint{1.769918in}{4.470172in}}%
\pgfpathlineto{\pgfqpoint{1.770992in}{4.469954in}}%
\pgfpathlineto{\pgfqpoint{1.772065in}{4.467275in}}%
\pgfpathlineto{\pgfqpoint{1.776361in}{4.467275in}}%
\pgfpathlineto{\pgfqpoint{1.777435in}{4.468899in}}%
\pgfpathlineto{\pgfqpoint{1.779582in}{4.464295in}}%
\pgfpathlineto{\pgfqpoint{1.782804in}{4.463558in}}%
\pgfpathlineto{\pgfqpoint{1.783878in}{4.464144in}}%
\pgfpathlineto{\pgfqpoint{1.784951in}{4.467225in}}%
\pgfpathlineto{\pgfqpoint{1.786025in}{4.468447in}}%
\pgfpathlineto{\pgfqpoint{1.787099in}{4.465735in}}%
\pgfpathlineto{\pgfqpoint{1.790321in}{4.462487in}}%
\pgfpathlineto{\pgfqpoint{1.792468in}{4.464094in}}%
\pgfpathlineto{\pgfqpoint{1.793542in}{4.468096in}}%
\pgfpathlineto{\pgfqpoint{1.794616in}{4.467091in}}%
\pgfpathlineto{\pgfqpoint{1.798911in}{4.468414in}}%
\pgfpathlineto{\pgfqpoint{1.801059in}{4.462554in}}%
\pgfpathlineto{\pgfqpoint{1.802133in}{4.464312in}}%
\pgfpathlineto{\pgfqpoint{1.805354in}{4.459624in}}%
\pgfpathlineto{\pgfqpoint{1.806428in}{4.460143in}}%
\pgfpathlineto{\pgfqpoint{1.807502in}{4.462152in}}%
\pgfpathlineto{\pgfqpoint{1.808576in}{4.461750in}}%
\pgfpathlineto{\pgfqpoint{1.812871in}{4.461298in}}%
\pgfpathlineto{\pgfqpoint{1.813945in}{4.461834in}}%
\pgfpathlineto{\pgfqpoint{1.815019in}{4.457765in}}%
\pgfpathlineto{\pgfqpoint{1.817167in}{4.461114in}}%
\pgfpathlineto{\pgfqpoint{1.821462in}{4.463843in}}%
\pgfpathlineto{\pgfqpoint{1.822536in}{4.462688in}}%
\pgfpathlineto{\pgfqpoint{1.823610in}{4.464429in}}%
\pgfpathlineto{\pgfqpoint{1.824684in}{4.463709in}}%
\pgfpathlineto{\pgfqpoint{1.827905in}{4.464345in}}%
\pgfpathlineto{\pgfqpoint{1.828979in}{4.462319in}}%
\pgfpathlineto{\pgfqpoint{1.830053in}{4.461800in}}%
\pgfpathlineto{\pgfqpoint{1.831127in}{4.452960in}}%
\pgfpathlineto{\pgfqpoint{1.835422in}{4.451788in}}%
\pgfpathlineto{\pgfqpoint{1.836496in}{4.455371in}}%
\pgfpathlineto{\pgfqpoint{1.838643in}{4.456057in}}%
\pgfpathlineto{\pgfqpoint{1.839717in}{4.455723in}}%
\pgfpathlineto{\pgfqpoint{1.842939in}{4.453897in}}%
\pgfpathlineto{\pgfqpoint{1.845086in}{4.455220in}}%
\pgfpathlineto{\pgfqpoint{1.846160in}{4.452608in}}%
\pgfpathlineto{\pgfqpoint{1.847234in}{4.452056in}}%
\pgfpathlineto{\pgfqpoint{1.850456in}{4.455572in}}%
\pgfpathlineto{\pgfqpoint{1.851529in}{4.451436in}}%
\pgfpathlineto{\pgfqpoint{1.852603in}{4.451503in}}%
\pgfpathlineto{\pgfqpoint{1.853677in}{4.449829in}}%
\pgfpathlineto{\pgfqpoint{1.854751in}{4.451152in}}%
\pgfpathlineto{\pgfqpoint{1.857973in}{4.452575in}}%
\pgfpathlineto{\pgfqpoint{1.860120in}{4.449059in}}%
\pgfpathlineto{\pgfqpoint{1.862268in}{4.443366in}}%
\pgfpathlineto{\pgfqpoint{1.866563in}{4.436685in}}%
\pgfpathlineto{\pgfqpoint{1.867637in}{4.444320in}}%
\pgfpathlineto{\pgfqpoint{1.868711in}{4.446095in}}%
\pgfpathlineto{\pgfqpoint{1.869785in}{4.446564in}}%
\pgfpathlineto{\pgfqpoint{1.873006in}{4.443450in}}%
\pgfpathlineto{\pgfqpoint{1.874080in}{4.437975in}}%
\pgfpathlineto{\pgfqpoint{1.875154in}{4.442093in}}%
\pgfpathlineto{\pgfqpoint{1.876228in}{4.442847in}}%
\pgfpathlineto{\pgfqpoint{1.877302in}{4.439984in}}%
\pgfpathlineto{\pgfqpoint{1.881597in}{4.445392in}}%
\pgfpathlineto{\pgfqpoint{1.882671in}{4.441524in}}%
\pgfpathlineto{\pgfqpoint{1.883745in}{4.441407in}}%
\pgfpathlineto{\pgfqpoint{1.884818in}{4.442127in}}%
\pgfpathlineto{\pgfqpoint{1.888040in}{4.441440in}}%
\pgfpathlineto{\pgfqpoint{1.889114in}{4.445643in}}%
\pgfpathlineto{\pgfqpoint{1.890188in}{4.446547in}}%
\pgfpathlineto{\pgfqpoint{1.891262in}{4.444672in}}%
\pgfpathlineto{\pgfqpoint{1.892335in}{4.439649in}}%
\pgfpathlineto{\pgfqpoint{1.895557in}{4.440285in}}%
\pgfpathlineto{\pgfqpoint{1.896631in}{4.437255in}}%
\pgfpathlineto{\pgfqpoint{1.898778in}{4.436585in}}%
\pgfpathlineto{\pgfqpoint{1.899852in}{4.439565in}}%
\pgfpathlineto{\pgfqpoint{1.903074in}{4.437790in}}%
\pgfpathlineto{\pgfqpoint{1.904148in}{4.442562in}}%
\pgfpathlineto{\pgfqpoint{1.905221in}{4.442897in}}%
\pgfpathlineto{\pgfqpoint{1.906295in}{4.441440in}}%
\pgfpathlineto{\pgfqpoint{1.907369in}{4.445040in}}%
\pgfpathlineto{\pgfqpoint{1.910591in}{4.449745in}}%
\pgfpathlineto{\pgfqpoint{1.911664in}{4.448941in}}%
\pgfpathlineto{\pgfqpoint{1.913812in}{4.454483in}}%
\pgfpathlineto{\pgfqpoint{1.914886in}{4.455120in}}%
\pgfpathlineto{\pgfqpoint{1.918107in}{4.455371in}}%
\pgfpathlineto{\pgfqpoint{1.920255in}{4.452742in}}%
\pgfpathlineto{\pgfqpoint{1.921329in}{4.454048in}}%
\pgfpathlineto{\pgfqpoint{1.922403in}{4.453311in}}%
\pgfpathlineto{\pgfqpoint{1.925624in}{4.452273in}}%
\pgfpathlineto{\pgfqpoint{1.927772in}{4.455003in}}%
\pgfpathlineto{\pgfqpoint{1.928846in}{4.464295in}}%
\pgfpathlineto{\pgfqpoint{1.929920in}{4.463994in}}%
\pgfpathlineto{\pgfqpoint{1.934215in}{4.465400in}}%
\pgfpathlineto{\pgfqpoint{1.935289in}{4.467426in}}%
\pgfpathlineto{\pgfqpoint{1.937437in}{4.466120in}}%
\pgfpathlineto{\pgfqpoint{1.940658in}{4.470205in}}%
\pgfpathlineto{\pgfqpoint{1.941732in}{4.468447in}}%
\pgfpathlineto{\pgfqpoint{1.944953in}{4.469201in}}%
\pgfpathlineto{\pgfqpoint{1.948175in}{4.466505in}}%
\pgfpathlineto{\pgfqpoint{1.949249in}{4.466689in}}%
\pgfpathlineto{\pgfqpoint{1.950323in}{4.468883in}}%
\pgfpathlineto{\pgfqpoint{1.951396in}{4.464747in}}%
\pgfpathlineto{\pgfqpoint{1.952470in}{4.463776in}}%
\pgfpathlineto{\pgfqpoint{1.955692in}{4.467627in}}%
\pgfpathlineto{\pgfqpoint{1.956766in}{4.465852in}}%
\pgfpathlineto{\pgfqpoint{1.958913in}{4.469452in}}%
\pgfpathlineto{\pgfqpoint{1.959987in}{4.470306in}}%
\pgfpathlineto{\pgfqpoint{1.963209in}{4.469854in}}%
\pgfpathlineto{\pgfqpoint{1.964283in}{4.468498in}}%
\pgfpathlineto{\pgfqpoint{1.965356in}{4.468313in}}%
\pgfpathlineto{\pgfqpoint{1.967504in}{4.468849in}}%
\pgfpathlineto{\pgfqpoint{1.970726in}{4.466723in}}%
\pgfpathlineto{\pgfqpoint{1.971799in}{4.467208in}}%
\pgfpathlineto{\pgfqpoint{1.973947in}{4.463391in}}%
\pgfpathlineto{\pgfqpoint{1.975021in}{4.469201in}}%
\pgfpathlineto{\pgfqpoint{1.978242in}{4.468498in}}%
\pgfpathlineto{\pgfqpoint{1.980390in}{4.465919in}}%
\pgfpathlineto{\pgfqpoint{1.981464in}{4.468146in}}%
\pgfpathlineto{\pgfqpoint{1.982538in}{4.464077in}}%
\pgfpathlineto{\pgfqpoint{1.985759in}{4.468313in}}%
\pgfpathlineto{\pgfqpoint{1.986833in}{4.453931in}}%
\pgfpathlineto{\pgfqpoint{1.987907in}{4.456677in}}%
\pgfpathlineto{\pgfqpoint{1.988981in}{4.455019in}}%
\pgfpathlineto{\pgfqpoint{1.990055in}{4.452089in}}%
\pgfpathlineto{\pgfqpoint{1.993276in}{4.452943in}}%
\pgfpathlineto{\pgfqpoint{1.996498in}{4.458552in}}%
\pgfpathlineto{\pgfqpoint{2.000793in}{4.458602in}}%
\pgfpathlineto{\pgfqpoint{2.001867in}{4.461131in}}%
\pgfpathlineto{\pgfqpoint{2.008310in}{4.451939in}}%
\pgfpathlineto{\pgfqpoint{2.009384in}{4.452910in}}%
\pgfpathlineto{\pgfqpoint{2.011531in}{4.443081in}}%
\pgfpathlineto{\pgfqpoint{2.012605in}{4.442361in}}%
\pgfpathlineto{\pgfqpoint{2.015827in}{4.442311in}}%
\pgfpathlineto{\pgfqpoint{2.016901in}{4.442914in}}%
\pgfpathlineto{\pgfqpoint{2.017974in}{4.439666in}}%
\pgfpathlineto{\pgfqpoint{2.019048in}{4.443399in}}%
\pgfpathlineto{\pgfqpoint{2.020122in}{4.439632in}}%
\pgfpathlineto{\pgfqpoint{2.024417in}{4.439180in}}%
\pgfpathlineto{\pgfqpoint{2.025491in}{4.437003in}}%
\pgfpathlineto{\pgfqpoint{2.026565in}{4.438226in}}%
\pgfpathlineto{\pgfqpoint{2.027639in}{4.440888in}}%
\pgfpathlineto{\pgfqpoint{2.030861in}{4.437924in}}%
\pgfpathlineto{\pgfqpoint{2.031934in}{4.448858in}}%
\pgfpathlineto{\pgfqpoint{2.033008in}{4.450013in}}%
\pgfpathlineto{\pgfqpoint{2.034082in}{4.452692in}}%
\pgfpathlineto{\pgfqpoint{2.035156in}{4.458267in}}%
\pgfpathlineto{\pgfqpoint{2.039451in}{4.453529in}}%
\pgfpathlineto{\pgfqpoint{2.040525in}{4.460578in}}%
\pgfpathlineto{\pgfqpoint{2.041599in}{4.461968in}}%
\pgfpathlineto{\pgfqpoint{2.045894in}{4.462671in}}%
\pgfpathlineto{\pgfqpoint{2.046968in}{4.463943in}}%
\pgfpathlineto{\pgfqpoint{2.049116in}{4.459423in}}%
\pgfpathlineto{\pgfqpoint{2.050190in}{4.464446in}}%
\pgfpathlineto{\pgfqpoint{2.054485in}{4.466840in}}%
\pgfpathlineto{\pgfqpoint{2.055559in}{4.468498in}}%
\pgfpathlineto{\pgfqpoint{2.056633in}{4.468749in}}%
\pgfpathlineto{\pgfqpoint{2.057706in}{4.468230in}}%
\pgfpathlineto{\pgfqpoint{2.060928in}{4.470138in}}%
\pgfpathlineto{\pgfqpoint{2.062002in}{4.467811in}}%
\pgfpathlineto{\pgfqpoint{2.063076in}{4.469419in}}%
\pgfpathlineto{\pgfqpoint{2.064150in}{4.472114in}}%
\pgfpathlineto{\pgfqpoint{2.065223in}{4.471009in}}%
\pgfpathlineto{\pgfqpoint{2.068445in}{4.468883in}}%
\pgfpathlineto{\pgfqpoint{2.069519in}{4.473035in}}%
\pgfpathlineto{\pgfqpoint{2.071666in}{4.472700in}}%
\pgfpathlineto{\pgfqpoint{2.072740in}{4.473755in}}%
\pgfpathlineto{\pgfqpoint{2.075962in}{4.474525in}}%
\pgfpathlineto{\pgfqpoint{2.078109in}{4.473789in}}%
\pgfpathlineto{\pgfqpoint{2.079183in}{4.473487in}}%
\pgfpathlineto{\pgfqpoint{2.080257in}{4.476518in}}%
\pgfpathlineto{\pgfqpoint{2.083479in}{4.476417in}}%
\pgfpathlineto{\pgfqpoint{2.086700in}{4.479381in}}%
\pgfpathlineto{\pgfqpoint{2.087774in}{4.481775in}}%
\pgfpathlineto{\pgfqpoint{2.094217in}{4.480452in}}%
\pgfpathlineto{\pgfqpoint{2.098512in}{4.483232in}}%
\pgfpathlineto{\pgfqpoint{2.099586in}{4.480352in}}%
\pgfpathlineto{\pgfqpoint{2.100660in}{4.483952in}}%
\pgfpathlineto{\pgfqpoint{2.101734in}{4.483768in}}%
\pgfpathlineto{\pgfqpoint{2.102808in}{4.485141in}}%
\pgfpathlineto{\pgfqpoint{2.107103in}{4.482512in}}%
\pgfpathlineto{\pgfqpoint{2.108177in}{4.484035in}}%
\pgfpathlineto{\pgfqpoint{2.109251in}{4.484571in}}%
\pgfpathlineto{\pgfqpoint{2.110325in}{4.483768in}}%
\pgfpathlineto{\pgfqpoint{2.113546in}{4.483701in}}%
\pgfpathlineto{\pgfqpoint{2.114620in}{4.485894in}}%
\pgfpathlineto{\pgfqpoint{2.115694in}{4.486748in}}%
\pgfpathlineto{\pgfqpoint{2.116768in}{4.486095in}}%
\pgfpathlineto{\pgfqpoint{2.117841in}{4.487049in}}%
\pgfpathlineto{\pgfqpoint{2.122137in}{4.488372in}}%
\pgfpathlineto{\pgfqpoint{2.124284in}{4.486865in}}%
\pgfpathlineto{\pgfqpoint{2.128580in}{4.486430in}}%
\pgfpathlineto{\pgfqpoint{2.129654in}{4.483081in}}%
\pgfpathlineto{\pgfqpoint{2.130728in}{4.485492in}}%
\pgfpathlineto{\pgfqpoint{2.131801in}{4.484337in}}%
\pgfpathlineto{\pgfqpoint{2.136097in}{4.486380in}}%
\pgfpathlineto{\pgfqpoint{2.137171in}{4.485810in}}%
\pgfpathlineto{\pgfqpoint{2.138244in}{4.484571in}}%
\pgfpathlineto{\pgfqpoint{2.139318in}{4.485509in}}%
\pgfpathlineto{\pgfqpoint{2.140392in}{4.487217in}}%
\pgfpathlineto{\pgfqpoint{2.143614in}{4.486647in}}%
\pgfpathlineto{\pgfqpoint{2.144687in}{4.489310in}}%
\pgfpathlineto{\pgfqpoint{2.145761in}{4.488556in}}%
\pgfpathlineto{\pgfqpoint{2.146835in}{4.489142in}}%
\pgfpathlineto{\pgfqpoint{2.147909in}{4.486346in}}%
\pgfpathlineto{\pgfqpoint{2.151130in}{4.488221in}}%
\pgfpathlineto{\pgfqpoint{2.152204in}{4.485559in}}%
\pgfpathlineto{\pgfqpoint{2.153278in}{4.485743in}}%
\pgfpathlineto{\pgfqpoint{2.154352in}{4.483131in}}%
\pgfpathlineto{\pgfqpoint{2.155426in}{4.482964in}}%
\pgfpathlineto{\pgfqpoint{2.158647in}{4.484655in}}%
\pgfpathlineto{\pgfqpoint{2.160795in}{4.490616in}}%
\pgfpathlineto{\pgfqpoint{2.161869in}{4.488958in}}%
\pgfpathlineto{\pgfqpoint{2.162943in}{4.488924in}}%
\pgfpathlineto{\pgfqpoint{2.167238in}{4.488054in}}%
\pgfpathlineto{\pgfqpoint{2.168312in}{4.488623in}}%
\pgfpathlineto{\pgfqpoint{2.169386in}{4.487619in}}%
\pgfpathlineto{\pgfqpoint{2.170460in}{4.488138in}}%
\pgfpathlineto{\pgfqpoint{2.175829in}{4.492809in}}%
\pgfpathlineto{\pgfqpoint{2.177976in}{4.488422in}}%
\pgfpathlineto{\pgfqpoint{2.181198in}{4.486229in}}%
\pgfpathlineto{\pgfqpoint{2.183346in}{4.487250in}}%
\pgfpathlineto{\pgfqpoint{2.184419in}{4.490532in}}%
\pgfpathlineto{\pgfqpoint{2.185493in}{4.489008in}}%
\pgfpathlineto{\pgfqpoint{2.188715in}{4.492524in}}%
\pgfpathlineto{\pgfqpoint{2.190862in}{4.492524in}}%
\pgfpathlineto{\pgfqpoint{2.191936in}{4.496945in}}%
\pgfpathlineto{\pgfqpoint{2.193010in}{4.489276in}}%
\pgfpathlineto{\pgfqpoint{2.196232in}{4.486313in}}%
\pgfpathlineto{\pgfqpoint{2.200527in}{4.499138in}}%
\pgfpathlineto{\pgfqpoint{2.204822in}{4.499272in}}%
\pgfpathlineto{\pgfqpoint{2.206970in}{4.498117in}}%
\pgfpathlineto{\pgfqpoint{2.208044in}{4.501566in}}%
\pgfpathlineto{\pgfqpoint{2.211265in}{4.502939in}}%
\pgfpathlineto{\pgfqpoint{2.212339in}{4.504697in}}%
\pgfpathlineto{\pgfqpoint{2.213413in}{4.504780in}}%
\pgfpathlineto{\pgfqpoint{2.214487in}{4.507376in}}%
\pgfpathlineto{\pgfqpoint{2.215561in}{4.508146in}}%
\pgfpathlineto{\pgfqpoint{2.219856in}{4.507895in}}%
\pgfpathlineto{\pgfqpoint{2.220930in}{4.508163in}}%
\pgfpathlineto{\pgfqpoint{2.222004in}{4.506388in}}%
\pgfpathlineto{\pgfqpoint{2.223078in}{4.506672in}}%
\pgfpathlineto{\pgfqpoint{2.226299in}{4.505417in}}%
\pgfpathlineto{\pgfqpoint{2.227373in}{4.502403in}}%
\pgfpathlineto{\pgfqpoint{2.228447in}{4.503340in}}%
\pgfpathlineto{\pgfqpoint{2.229521in}{4.502888in}}%
\pgfpathlineto{\pgfqpoint{2.230594in}{4.503474in}}%
\pgfpathlineto{\pgfqpoint{2.235964in}{4.503508in}}%
\pgfpathlineto{\pgfqpoint{2.237038in}{4.502621in}}%
\pgfpathlineto{\pgfqpoint{2.238111in}{4.503793in}}%
\pgfpathlineto{\pgfqpoint{2.243481in}{4.504178in}}%
\pgfpathlineto{\pgfqpoint{2.244554in}{4.507526in}}%
\pgfpathlineto{\pgfqpoint{2.245628in}{4.506405in}}%
\pgfpathlineto{\pgfqpoint{2.248850in}{4.506857in}}%
\pgfpathlineto{\pgfqpoint{2.249924in}{4.504831in}}%
\pgfpathlineto{\pgfqpoint{2.250997in}{4.507510in}}%
\pgfpathlineto{\pgfqpoint{2.252071in}{4.506438in}}%
\pgfpathlineto{\pgfqpoint{2.253145in}{4.507108in}}%
\pgfpathlineto{\pgfqpoint{2.256367in}{4.506271in}}%
\pgfpathlineto{\pgfqpoint{2.257440in}{4.507359in}}%
\pgfpathlineto{\pgfqpoint{2.258514in}{4.506806in}}%
\pgfpathlineto{\pgfqpoint{2.264957in}{4.507945in}}%
\pgfpathlineto{\pgfqpoint{2.266031in}{4.506538in}}%
\pgfpathlineto{\pgfqpoint{2.268179in}{4.508966in}}%
\pgfpathlineto{\pgfqpoint{2.272474in}{4.508414in}}%
\pgfpathlineto{\pgfqpoint{2.273548in}{4.507325in}}%
\pgfpathlineto{\pgfqpoint{2.274622in}{4.507861in}}%
\pgfpathlineto{\pgfqpoint{2.275696in}{4.500980in}}%
\pgfpathlineto{\pgfqpoint{2.278917in}{4.504780in}}%
\pgfpathlineto{\pgfqpoint{2.279991in}{4.501633in}}%
\pgfpathlineto{\pgfqpoint{2.281065in}{4.500946in}}%
\pgfpathlineto{\pgfqpoint{2.282139in}{4.502436in}}%
\pgfpathlineto{\pgfqpoint{2.283213in}{4.500076in}}%
\pgfpathlineto{\pgfqpoint{2.287508in}{4.504010in}}%
\pgfpathlineto{\pgfqpoint{2.288582in}{4.507074in}}%
\pgfpathlineto{\pgfqpoint{2.289656in}{4.507476in}}%
\pgfpathlineto{\pgfqpoint{2.290729in}{4.503675in}}%
\pgfpathlineto{\pgfqpoint{2.293951in}{4.501448in}}%
\pgfpathlineto{\pgfqpoint{2.295025in}{4.502001in}}%
\pgfpathlineto{\pgfqpoint{2.296099in}{4.503993in}}%
\pgfpathlineto{\pgfqpoint{2.297172in}{4.500561in}}%
\pgfpathlineto{\pgfqpoint{2.298246in}{4.501884in}}%
\pgfpathlineto{\pgfqpoint{2.301468in}{4.500042in}}%
\pgfpathlineto{\pgfqpoint{2.302542in}{4.494852in}}%
\pgfpathlineto{\pgfqpoint{2.303616in}{4.495957in}}%
\pgfpathlineto{\pgfqpoint{2.305763in}{4.494316in}}%
\pgfpathlineto{\pgfqpoint{2.308985in}{4.494014in}}%
\pgfpathlineto{\pgfqpoint{2.310059in}{4.491771in}}%
\pgfpathlineto{\pgfqpoint{2.312206in}{4.492290in}}%
\pgfpathlineto{\pgfqpoint{2.313280in}{4.492775in}}%
\pgfpathlineto{\pgfqpoint{2.319723in}{4.492039in}}%
\pgfpathlineto{\pgfqpoint{2.320797in}{4.491486in}}%
\pgfpathlineto{\pgfqpoint{2.324018in}{4.494215in}}%
\pgfpathlineto{\pgfqpoint{2.325092in}{4.486430in}}%
\pgfpathlineto{\pgfqpoint{2.326166in}{4.486865in}}%
\pgfpathlineto{\pgfqpoint{2.327240in}{4.485710in}}%
\pgfpathlineto{\pgfqpoint{2.328314in}{4.485727in}}%
\pgfpathlineto{\pgfqpoint{2.331535in}{4.484990in}}%
\pgfpathlineto{\pgfqpoint{2.332609in}{4.483382in}}%
\pgfpathlineto{\pgfqpoint{2.334757in}{4.487367in}}%
\pgfpathlineto{\pgfqpoint{2.335831in}{4.486815in}}%
\pgfpathlineto{\pgfqpoint{2.340126in}{4.493847in}}%
\pgfpathlineto{\pgfqpoint{2.341200in}{4.492859in}}%
\pgfpathlineto{\pgfqpoint{2.342274in}{4.498870in}}%
\pgfpathlineto{\pgfqpoint{2.343348in}{4.500109in}}%
\pgfpathlineto{\pgfqpoint{2.346569in}{4.496760in}}%
\pgfpathlineto{\pgfqpoint{2.347643in}{4.498803in}}%
\pgfpathlineto{\pgfqpoint{2.348717in}{4.497062in}}%
\pgfpathlineto{\pgfqpoint{2.349791in}{4.498251in}}%
\pgfpathlineto{\pgfqpoint{2.350864in}{4.498552in}}%
\pgfpathlineto{\pgfqpoint{2.354086in}{4.496308in}}%
\pgfpathlineto{\pgfqpoint{2.356234in}{4.497447in}}%
\pgfpathlineto{\pgfqpoint{2.358381in}{4.499372in}}%
\pgfpathlineto{\pgfqpoint{2.361603in}{4.497832in}}%
\pgfpathlineto{\pgfqpoint{2.362677in}{4.498251in}}%
\pgfpathlineto{\pgfqpoint{2.363750in}{4.496643in}}%
\pgfpathlineto{\pgfqpoint{2.364824in}{4.498033in}}%
\pgfpathlineto{\pgfqpoint{2.370193in}{4.496744in}}%
\pgfpathlineto{\pgfqpoint{2.371267in}{4.503357in}}%
\pgfpathlineto{\pgfqpoint{2.372341in}{4.503089in}}%
\pgfpathlineto{\pgfqpoint{2.373415in}{4.507141in}}%
\pgfpathlineto{\pgfqpoint{2.376637in}{4.508933in}}%
\pgfpathlineto{\pgfqpoint{2.377710in}{4.507677in}}%
\pgfpathlineto{\pgfqpoint{2.378784in}{4.504211in}}%
\pgfpathlineto{\pgfqpoint{2.379858in}{4.503307in}}%
\pgfpathlineto{\pgfqpoint{2.380932in}{4.505534in}}%
\pgfpathlineto{\pgfqpoint{2.388449in}{4.507543in}}%
\pgfpathlineto{\pgfqpoint{2.392744in}{4.507811in}}%
\pgfpathlineto{\pgfqpoint{2.393818in}{4.506505in}}%
\pgfpathlineto{\pgfqpoint{2.395966in}{4.507275in}}%
\pgfpathlineto{\pgfqpoint{2.400261in}{4.506455in}}%
\pgfpathlineto{\pgfqpoint{2.401335in}{4.506890in}}%
\pgfpathlineto{\pgfqpoint{2.402409in}{4.505936in}}%
\pgfpathlineto{\pgfqpoint{2.403482in}{4.506739in}}%
\pgfpathlineto{\pgfqpoint{2.407778in}{4.504178in}}%
\pgfpathlineto{\pgfqpoint{2.408852in}{4.506220in}}%
\pgfpathlineto{\pgfqpoint{2.410999in}{4.505433in}}%
\pgfpathlineto{\pgfqpoint{2.415295in}{4.505232in}}%
\pgfpathlineto{\pgfqpoint{2.416369in}{4.507141in}}%
\pgfpathlineto{\pgfqpoint{2.417442in}{4.507443in}}%
\pgfpathlineto{\pgfqpoint{2.418516in}{4.507141in}}%
\pgfpathlineto{\pgfqpoint{2.421738in}{4.507175in}}%
\pgfpathlineto{\pgfqpoint{2.422812in}{4.503223in}}%
\pgfpathlineto{\pgfqpoint{2.423885in}{4.504412in}}%
\pgfpathlineto{\pgfqpoint{2.424959in}{4.504546in}}%
\pgfpathlineto{\pgfqpoint{2.426033in}{4.505567in}}%
\pgfpathlineto{\pgfqpoint{2.430328in}{4.501432in}}%
\pgfpathlineto{\pgfqpoint{2.431402in}{4.501984in}}%
\pgfpathlineto{\pgfqpoint{2.432476in}{4.500444in}}%
\pgfpathlineto{\pgfqpoint{2.433550in}{4.501783in}}%
\pgfpathlineto{\pgfqpoint{2.436771in}{4.501867in}}%
\pgfpathlineto{\pgfqpoint{2.437845in}{4.502905in}}%
\pgfpathlineto{\pgfqpoint{2.441067in}{4.507945in}}%
\pgfpathlineto{\pgfqpoint{2.445362in}{4.511980in}}%
\pgfpathlineto{\pgfqpoint{2.447510in}{4.516651in}}%
\pgfpathlineto{\pgfqpoint{2.448584in}{4.515931in}}%
\pgfpathlineto{\pgfqpoint{2.452879in}{4.516584in}}%
\pgfpathlineto{\pgfqpoint{2.453953in}{4.520770in}}%
\pgfpathlineto{\pgfqpoint{2.455027in}{4.522579in}}%
\pgfpathlineto{\pgfqpoint{2.456101in}{4.522913in}}%
\pgfpathlineto{\pgfqpoint{2.459322in}{4.522126in}}%
\pgfpathlineto{\pgfqpoint{2.460396in}{4.521256in}}%
\pgfpathlineto{\pgfqpoint{2.461470in}{4.526748in}}%
\pgfpathlineto{\pgfqpoint{2.462544in}{4.526798in}}%
\pgfpathlineto{\pgfqpoint{2.463617in}{4.525894in}}%
\pgfpathlineto{\pgfqpoint{2.467913in}{4.525542in}}%
\pgfpathlineto{\pgfqpoint{2.470060in}{4.526815in}}%
\pgfpathlineto{\pgfqpoint{2.471134in}{4.528857in}}%
\pgfpathlineto{\pgfqpoint{2.474356in}{4.529343in}}%
\pgfpathlineto{\pgfqpoint{2.475430in}{4.527484in}}%
\pgfpathlineto{\pgfqpoint{2.476504in}{4.528841in}}%
\pgfpathlineto{\pgfqpoint{2.477577in}{4.527451in}}%
\pgfpathlineto{\pgfqpoint{2.478651in}{4.530666in}}%
\pgfpathlineto{\pgfqpoint{2.481873in}{4.531687in}}%
\pgfpathlineto{\pgfqpoint{2.482947in}{4.530297in}}%
\pgfpathlineto{\pgfqpoint{2.485094in}{4.530314in}}%
\pgfpathlineto{\pgfqpoint{2.486168in}{4.529326in}}%
\pgfpathlineto{\pgfqpoint{2.489390in}{4.527635in}}%
\pgfpathlineto{\pgfqpoint{2.490463in}{4.528522in}}%
\pgfpathlineto{\pgfqpoint{2.491537in}{4.528037in}}%
\pgfpathlineto{\pgfqpoint{2.492611in}{4.528974in}}%
\pgfpathlineto{\pgfqpoint{2.493685in}{4.529058in}}%
\pgfpathlineto{\pgfqpoint{2.501202in}{4.526949in}}%
\pgfpathlineto{\pgfqpoint{2.504423in}{4.526513in}}%
\pgfpathlineto{\pgfqpoint{2.505497in}{4.527083in}}%
\pgfpathlineto{\pgfqpoint{2.506571in}{4.526496in}}%
\pgfpathlineto{\pgfqpoint{2.507645in}{4.524856in}}%
\pgfpathlineto{\pgfqpoint{2.511940in}{4.527535in}}%
\pgfpathlineto{\pgfqpoint{2.514088in}{4.526697in}}%
\pgfpathlineto{\pgfqpoint{2.515162in}{4.528790in}}%
\pgfpathlineto{\pgfqpoint{2.516236in}{4.529309in}}%
\pgfpathlineto{\pgfqpoint{2.520531in}{4.535002in}}%
\pgfpathlineto{\pgfqpoint{2.521605in}{4.534801in}}%
\pgfpathlineto{\pgfqpoint{2.522679in}{4.536526in}}%
\pgfpathlineto{\pgfqpoint{2.526974in}{4.534500in}}%
\pgfpathlineto{\pgfqpoint{2.530195in}{4.541850in}}%
\pgfpathlineto{\pgfqpoint{2.531269in}{4.541683in}}%
\pgfpathlineto{\pgfqpoint{2.534491in}{4.540511in}}%
\pgfpathlineto{\pgfqpoint{2.535565in}{4.539422in}}%
\pgfpathlineto{\pgfqpoint{2.536638in}{4.537363in}}%
\pgfpathlineto{\pgfqpoint{2.538786in}{4.537246in}}%
\pgfpathlineto{\pgfqpoint{2.543081in}{4.539355in}}%
\pgfpathlineto{\pgfqpoint{2.544155in}{4.536358in}}%
\pgfpathlineto{\pgfqpoint{2.546303in}{4.537882in}}%
\pgfpathlineto{\pgfqpoint{2.549525in}{4.542151in}}%
\pgfpathlineto{\pgfqpoint{2.550598in}{4.540912in}}%
\pgfpathlineto{\pgfqpoint{2.551672in}{4.540544in}}%
\pgfpathlineto{\pgfqpoint{2.553820in}{4.545550in}}%
\pgfpathlineto{\pgfqpoint{2.558115in}{4.548330in}}%
\pgfpathlineto{\pgfqpoint{2.559189in}{4.551528in}}%
\pgfpathlineto{\pgfqpoint{2.560263in}{4.551344in}}%
\pgfpathlineto{\pgfqpoint{2.561337in}{4.555044in}}%
\pgfpathlineto{\pgfqpoint{2.564558in}{4.554290in}}%
\pgfpathlineto{\pgfqpoint{2.566706in}{4.552382in}}%
\pgfpathlineto{\pgfqpoint{2.568854in}{4.555395in}}%
\pgfpathlineto{\pgfqpoint{2.572075in}{4.556132in}}%
\pgfpathlineto{\pgfqpoint{2.574223in}{4.560251in}}%
\pgfpathlineto{\pgfqpoint{2.576370in}{4.565341in}}%
\pgfpathlineto{\pgfqpoint{2.580666in}{4.565525in}}%
\pgfpathlineto{\pgfqpoint{2.582814in}{4.563734in}}%
\pgfpathlineto{\pgfqpoint{2.583887in}{4.564805in}}%
\pgfpathlineto{\pgfqpoint{2.587109in}{4.564336in}}%
\pgfpathlineto{\pgfqpoint{2.588183in}{4.559883in}}%
\pgfpathlineto{\pgfqpoint{2.589257in}{4.561205in}}%
\pgfpathlineto{\pgfqpoint{2.590330in}{4.556852in}}%
\pgfpathlineto{\pgfqpoint{2.591404in}{4.557388in}}%
\pgfpathlineto{\pgfqpoint{2.594626in}{4.559966in}}%
\pgfpathlineto{\pgfqpoint{2.596773in}{4.559849in}}%
\pgfpathlineto{\pgfqpoint{2.597847in}{4.557120in}}%
\pgfpathlineto{\pgfqpoint{2.598921in}{4.559581in}}%
\pgfpathlineto{\pgfqpoint{2.602143in}{4.561004in}}%
\pgfpathlineto{\pgfqpoint{2.603216in}{4.559698in}}%
\pgfpathlineto{\pgfqpoint{2.604290in}{4.562277in}}%
\pgfpathlineto{\pgfqpoint{2.605364in}{4.561959in}}%
\pgfpathlineto{\pgfqpoint{2.606438in}{4.563030in}}%
\pgfpathlineto{\pgfqpoint{2.610733in}{4.562294in}}%
\pgfpathlineto{\pgfqpoint{2.612881in}{4.564085in}}%
\pgfpathlineto{\pgfqpoint{2.613955in}{4.562059in}}%
\pgfpathlineto{\pgfqpoint{2.617176in}{4.560234in}}%
\pgfpathlineto{\pgfqpoint{2.618250in}{4.543524in}}%
\pgfpathlineto{\pgfqpoint{2.619324in}{4.542955in}}%
\pgfpathlineto{\pgfqpoint{2.620398in}{4.544563in}}%
\pgfpathlineto{\pgfqpoint{2.621472in}{4.544043in}}%
\pgfpathlineto{\pgfqpoint{2.624693in}{4.546337in}}%
\pgfpathlineto{\pgfqpoint{2.627915in}{4.556484in}}%
\pgfpathlineto{\pgfqpoint{2.632210in}{4.556199in}}%
\pgfpathlineto{\pgfqpoint{2.633284in}{4.554609in}}%
\pgfpathlineto{\pgfqpoint{2.635432in}{4.554307in}}%
\pgfpathlineto{\pgfqpoint{2.636505in}{4.553905in}}%
\pgfpathlineto{\pgfqpoint{2.639727in}{4.556099in}}%
\pgfpathlineto{\pgfqpoint{2.640801in}{4.555797in}}%
\pgfpathlineto{\pgfqpoint{2.641875in}{4.556953in}}%
\pgfpathlineto{\pgfqpoint{2.644022in}{4.550054in}}%
\pgfpathlineto{\pgfqpoint{2.647244in}{4.551662in}}%
\pgfpathlineto{\pgfqpoint{2.648318in}{4.553085in}}%
\pgfpathlineto{\pgfqpoint{2.649392in}{4.550523in}}%
\pgfpathlineto{\pgfqpoint{2.650465in}{4.549652in}}%
\pgfpathlineto{\pgfqpoint{2.654761in}{4.550205in}}%
\pgfpathlineto{\pgfqpoint{2.656908in}{4.552147in}}%
\pgfpathlineto{\pgfqpoint{2.657982in}{4.553135in}}%
\pgfpathlineto{\pgfqpoint{2.663351in}{4.547861in}}%
\pgfpathlineto{\pgfqpoint{2.664425in}{4.549552in}}%
\pgfpathlineto{\pgfqpoint{2.666573in}{4.555312in}}%
\pgfpathlineto{\pgfqpoint{2.669794in}{4.561440in}}%
\pgfpathlineto{\pgfqpoint{2.671942in}{4.561323in}}%
\pgfpathlineto{\pgfqpoint{2.674090in}{4.567434in}}%
\pgfpathlineto{\pgfqpoint{2.678385in}{4.567769in}}%
\pgfpathlineto{\pgfqpoint{2.679459in}{4.562947in}}%
\pgfpathlineto{\pgfqpoint{2.680533in}{4.562829in}}%
\pgfpathlineto{\pgfqpoint{2.681607in}{4.563349in}}%
\pgfpathlineto{\pgfqpoint{2.685902in}{4.563767in}}%
\pgfpathlineto{\pgfqpoint{2.686976in}{4.561306in}}%
\pgfpathlineto{\pgfqpoint{2.689124in}{4.561976in}}%
\pgfpathlineto{\pgfqpoint{2.692345in}{4.566513in}}%
\pgfpathlineto{\pgfqpoint{2.694493in}{4.572457in}}%
\pgfpathlineto{\pgfqpoint{2.696640in}{4.572457in}}%
\pgfpathlineto{\pgfqpoint{2.702010in}{4.572440in}}%
\pgfpathlineto{\pgfqpoint{2.703083in}{4.574148in}}%
\pgfpathlineto{\pgfqpoint{2.704157in}{4.574366in}}%
\pgfpathlineto{\pgfqpoint{2.707379in}{4.575939in}}%
\pgfpathlineto{\pgfqpoint{2.708453in}{4.574416in}}%
\pgfpathlineto{\pgfqpoint{2.709526in}{4.575236in}}%
\pgfpathlineto{\pgfqpoint{2.710600in}{4.576777in}}%
\pgfpathlineto{\pgfqpoint{2.711674in}{4.580058in}}%
\pgfpathlineto{\pgfqpoint{2.714896in}{4.580427in}}%
\pgfpathlineto{\pgfqpoint{2.715969in}{4.601172in}}%
\pgfpathlineto{\pgfqpoint{2.717043in}{4.605977in}}%
\pgfpathlineto{\pgfqpoint{2.718117in}{4.598459in}}%
\pgfpathlineto{\pgfqpoint{2.719191in}{4.601322in}}%
\pgfpathlineto{\pgfqpoint{2.723486in}{4.594106in}}%
\pgfpathlineto{\pgfqpoint{2.724560in}{4.594089in}}%
\pgfpathlineto{\pgfqpoint{2.725634in}{4.597337in}}%
\pgfpathlineto{\pgfqpoint{2.726708in}{4.597321in}}%
\pgfpathlineto{\pgfqpoint{2.731003in}{4.593888in}}%
\pgfpathlineto{\pgfqpoint{2.732077in}{4.593537in}}%
\pgfpathlineto{\pgfqpoint{2.734225in}{4.589769in}}%
\pgfpathlineto{\pgfqpoint{2.737446in}{4.590992in}}%
\pgfpathlineto{\pgfqpoint{2.738520in}{4.592750in}}%
\pgfpathlineto{\pgfqpoint{2.739594in}{4.589686in}}%
\pgfpathlineto{\pgfqpoint{2.740668in}{4.592884in}}%
\pgfpathlineto{\pgfqpoint{2.741742in}{4.592800in}}%
\pgfpathlineto{\pgfqpoint{2.744963in}{4.596165in}}%
\pgfpathlineto{\pgfqpoint{2.746037in}{4.600284in}}%
\pgfpathlineto{\pgfqpoint{2.747111in}{4.598158in}}%
\pgfpathlineto{\pgfqpoint{2.749258in}{4.597840in}}%
\pgfpathlineto{\pgfqpoint{2.752480in}{4.602009in}}%
\pgfpathlineto{\pgfqpoint{2.754628in}{4.608388in}}%
\pgfpathlineto{\pgfqpoint{2.755702in}{4.616559in}}%
\pgfpathlineto{\pgfqpoint{2.756775in}{4.613394in}}%
\pgfpathlineto{\pgfqpoint{2.761071in}{4.608790in}}%
\pgfpathlineto{\pgfqpoint{2.762145in}{4.609443in}}%
\pgfpathlineto{\pgfqpoint{2.763218in}{4.612038in}}%
\pgfpathlineto{\pgfqpoint{2.764292in}{4.608589in}}%
\pgfpathlineto{\pgfqpoint{2.767514in}{4.610447in}}%
\pgfpathlineto{\pgfqpoint{2.768588in}{4.606111in}}%
\pgfpathlineto{\pgfqpoint{2.769661in}{4.610146in}}%
\pgfpathlineto{\pgfqpoint{2.770735in}{4.608505in}}%
\pgfpathlineto{\pgfqpoint{2.771809in}{4.608371in}}%
\pgfpathlineto{\pgfqpoint{2.776104in}{4.608991in}}%
\pgfpathlineto{\pgfqpoint{2.779326in}{4.603181in}}%
\pgfpathlineto{\pgfqpoint{2.783621in}{4.604319in}}%
\pgfpathlineto{\pgfqpoint{2.784695in}{4.605508in}}%
\pgfpathlineto{\pgfqpoint{2.786843in}{4.604185in}}%
\pgfpathlineto{\pgfqpoint{2.786843in}{4.604185in}}%
\pgfusepath{stroke}%
\end{pgfscope}%
\begin{pgfscope}%
\pgfpathrectangle{\pgfqpoint{0.320934in}{4.233896in}}{\pgfqpoint{2.583333in}{0.400885in}}%
\pgfusepath{clip}%
\pgfsetroundcap%
\pgfsetroundjoin%
\pgfsetlinewidth{1.505625pt}%
\definecolor{currentstroke}{rgb}{0.121569,0.466667,0.705882}%
\pgfsetstrokecolor{currentstroke}%
\pgfsetdash{}{0pt}%
\pgfpathmoveto{\pgfqpoint{0.438358in}{4.344128in}}%
\pgfpathlineto{\pgfqpoint{0.439432in}{4.343174in}}%
\pgfpathlineto{\pgfqpoint{0.440506in}{4.343174in}}%
\pgfpathlineto{\pgfqpoint{0.441580in}{4.342595in}}%
\pgfpathlineto{\pgfqpoint{0.444801in}{4.342595in}}%
\pgfpathlineto{\pgfqpoint{0.448023in}{4.341332in}}%
\pgfpathlineto{\pgfqpoint{0.449096in}{4.341332in}}%
\pgfpathlineto{\pgfqpoint{0.453392in}{4.340483in}}%
\pgfpathlineto{\pgfqpoint{0.455539in}{4.338430in}}%
\pgfpathlineto{\pgfqpoint{0.459835in}{4.338678in}}%
\pgfpathlineto{\pgfqpoint{0.461982in}{4.337579in}}%
\pgfpathlineto{\pgfqpoint{0.463056in}{4.336192in}}%
\pgfpathlineto{\pgfqpoint{0.468425in}{4.337266in}}%
\pgfpathlineto{\pgfqpoint{0.469499in}{4.336463in}}%
\pgfpathlineto{\pgfqpoint{0.479164in}{4.335644in}}%
\pgfpathlineto{\pgfqpoint{0.483459in}{4.334619in}}%
\pgfpathlineto{\pgfqpoint{0.484533in}{4.334619in}}%
\pgfpathlineto{\pgfqpoint{0.485607in}{4.333843in}}%
\pgfpathlineto{\pgfqpoint{0.490976in}{4.333901in}}%
\pgfpathlineto{\pgfqpoint{0.494198in}{4.333196in}}%
\pgfpathlineto{\pgfqpoint{0.512453in}{4.333696in}}%
\pgfpathlineto{\pgfqpoint{0.513527in}{4.332254in}}%
\pgfpathlineto{\pgfqpoint{0.514601in}{4.332128in}}%
\pgfpathlineto{\pgfqpoint{0.515674in}{4.330813in}}%
\pgfpathlineto{\pgfqpoint{0.516748in}{4.331318in}}%
\pgfpathlineto{\pgfqpoint{0.521044in}{4.331550in}}%
\pgfpathlineto{\pgfqpoint{0.524265in}{4.332439in}}%
\pgfpathlineto{\pgfqpoint{0.531782in}{4.330787in}}%
\pgfpathlineto{\pgfqpoint{0.538225in}{4.331254in}}%
\pgfpathlineto{\pgfqpoint{0.542520in}{4.331254in}}%
\pgfpathlineto{\pgfqpoint{0.543594in}{4.329198in}}%
\pgfpathlineto{\pgfqpoint{0.559702in}{4.328844in}}%
\pgfpathlineto{\pgfqpoint{0.561849in}{4.328221in}}%
\pgfpathlineto{\pgfqpoint{0.582252in}{4.328193in}}%
\pgfpathlineto{\pgfqpoint{0.584400in}{4.326065in}}%
\pgfpathlineto{\pgfqpoint{0.587622in}{4.327164in}}%
\pgfpathlineto{\pgfqpoint{0.589769in}{4.326775in}}%
\pgfpathlineto{\pgfqpoint{0.596212in}{4.326775in}}%
\pgfpathlineto{\pgfqpoint{0.597286in}{4.325328in}}%
\pgfpathlineto{\pgfqpoint{0.598360in}{4.325287in}}%
\pgfpathlineto{\pgfqpoint{0.599434in}{4.323513in}}%
\pgfpathlineto{\pgfqpoint{0.610172in}{4.323131in}}%
\pgfpathlineto{\pgfqpoint{0.628427in}{4.323131in}}%
\pgfpathlineto{\pgfqpoint{0.629501in}{4.320508in}}%
\pgfpathlineto{\pgfqpoint{0.633797in}{4.320407in}}%
\pgfpathlineto{\pgfqpoint{0.640240in}{4.321152in}}%
\pgfpathlineto{\pgfqpoint{0.641313in}{4.321884in}}%
\pgfpathlineto{\pgfqpoint{0.647757in}{4.321884in}}%
\pgfpathlineto{\pgfqpoint{0.648830in}{4.320963in}}%
\pgfpathlineto{\pgfqpoint{0.649904in}{4.319022in}}%
\pgfpathlineto{\pgfqpoint{0.650978in}{4.319057in}}%
\pgfpathlineto{\pgfqpoint{0.652052in}{4.319887in}}%
\pgfpathlineto{\pgfqpoint{0.656347in}{4.320516in}}%
\pgfpathlineto{\pgfqpoint{0.657421in}{4.319995in}}%
\pgfpathlineto{\pgfqpoint{0.659569in}{4.317024in}}%
\pgfpathlineto{\pgfqpoint{0.664938in}{4.317507in}}%
\pgfpathlineto{\pgfqpoint{0.666012in}{4.318586in}}%
\pgfpathlineto{\pgfqpoint{0.667086in}{4.316992in}}%
\pgfpathlineto{\pgfqpoint{0.673529in}{4.317094in}}%
\pgfpathlineto{\pgfqpoint{0.674602in}{4.316427in}}%
\pgfpathlineto{\pgfqpoint{0.679972in}{4.316189in}}%
\pgfpathlineto{\pgfqpoint{0.682119in}{4.314612in}}%
\pgfpathlineto{\pgfqpoint{0.687489in}{4.315484in}}%
\pgfpathlineto{\pgfqpoint{0.688562in}{4.316139in}}%
\pgfpathlineto{\pgfqpoint{0.689636in}{4.315339in}}%
\pgfpathlineto{\pgfqpoint{0.702522in}{4.316324in}}%
\pgfpathlineto{\pgfqpoint{0.703596in}{4.314876in}}%
\pgfpathlineto{\pgfqpoint{0.704670in}{4.315303in}}%
\pgfpathlineto{\pgfqpoint{0.707891in}{4.317280in}}%
\pgfpathlineto{\pgfqpoint{0.708965in}{4.316794in}}%
\pgfpathlineto{\pgfqpoint{0.710039in}{4.317149in}}%
\pgfpathlineto{\pgfqpoint{0.712187in}{4.314142in}}%
\pgfpathlineto{\pgfqpoint{0.723999in}{4.315174in}}%
\pgfpathlineto{\pgfqpoint{0.727221in}{4.315548in}}%
\pgfpathlineto{\pgfqpoint{0.737959in}{4.312869in}}%
\pgfpathlineto{\pgfqpoint{0.740107in}{4.314714in}}%
\pgfpathlineto{\pgfqpoint{0.742254in}{4.315195in}}%
\pgfpathlineto{\pgfqpoint{0.745476in}{4.315162in}}%
\pgfpathlineto{\pgfqpoint{0.747623in}{4.313314in}}%
\pgfpathlineto{\pgfqpoint{0.748697in}{4.313377in}}%
\pgfpathlineto{\pgfqpoint{0.749771in}{4.314975in}}%
\pgfpathlineto{\pgfqpoint{0.757288in}{4.315349in}}%
\pgfpathlineto{\pgfqpoint{0.791651in}{4.315349in}}%
\pgfpathlineto{\pgfqpoint{0.793799in}{4.314762in}}%
\pgfpathlineto{\pgfqpoint{0.794872in}{4.314472in}}%
\pgfpathlineto{\pgfqpoint{0.801315in}{4.314294in}}%
\pgfpathlineto{\pgfqpoint{0.802389in}{4.313774in}}%
\pgfpathlineto{\pgfqpoint{0.805611in}{4.313446in}}%
\pgfpathlineto{\pgfqpoint{0.806685in}{4.311797in}}%
\pgfpathlineto{\pgfqpoint{0.809906in}{4.313039in}}%
\pgfpathlineto{\pgfqpoint{0.816349in}{4.311366in}}%
\pgfpathlineto{\pgfqpoint{0.817423in}{4.312264in}}%
\pgfpathlineto{\pgfqpoint{0.822792in}{4.312285in}}%
\pgfpathlineto{\pgfqpoint{0.824940in}{4.313443in}}%
\pgfpathlineto{\pgfqpoint{0.828161in}{4.312466in}}%
\pgfpathlineto{\pgfqpoint{0.830309in}{4.310736in}}%
\pgfpathlineto{\pgfqpoint{0.831383in}{4.310828in}}%
\pgfpathlineto{\pgfqpoint{0.832457in}{4.310237in}}%
\pgfpathlineto{\pgfqpoint{0.836752in}{4.310126in}}%
\pgfpathlineto{\pgfqpoint{0.838900in}{4.308961in}}%
\pgfpathlineto{\pgfqpoint{0.839974in}{4.309458in}}%
\pgfpathlineto{\pgfqpoint{0.860377in}{4.305840in}}%
\pgfpathlineto{\pgfqpoint{0.861450in}{4.306029in}}%
\pgfpathlineto{\pgfqpoint{0.862524in}{4.305261in}}%
\pgfpathlineto{\pgfqpoint{0.865746in}{4.305846in}}%
\pgfpathlineto{\pgfqpoint{0.868967in}{4.304748in}}%
\pgfpathlineto{\pgfqpoint{0.870041in}{4.304425in}}%
\pgfpathlineto{\pgfqpoint{0.873263in}{4.304451in}}%
\pgfpathlineto{\pgfqpoint{0.875410in}{4.303819in}}%
\pgfpathlineto{\pgfqpoint{0.877558in}{4.303545in}}%
\pgfpathlineto{\pgfqpoint{0.881853in}{4.302868in}}%
\pgfpathlineto{\pgfqpoint{0.884001in}{4.303905in}}%
\pgfpathlineto{\pgfqpoint{0.885075in}{4.303306in}}%
\pgfpathlineto{\pgfqpoint{0.888296in}{4.304585in}}%
\pgfpathlineto{\pgfqpoint{0.892592in}{4.303110in}}%
\pgfpathlineto{\pgfqpoint{0.895813in}{4.303453in}}%
\pgfpathlineto{\pgfqpoint{0.897961in}{4.302468in}}%
\pgfpathlineto{\pgfqpoint{0.899035in}{4.302559in}}%
\pgfpathlineto{\pgfqpoint{0.900109in}{4.301734in}}%
\pgfpathlineto{\pgfqpoint{0.910847in}{4.301928in}}%
\pgfpathlineto{\pgfqpoint{0.914068in}{4.302249in}}%
\pgfpathlineto{\pgfqpoint{0.915142in}{4.301218in}}%
\pgfpathlineto{\pgfqpoint{0.918364in}{4.302061in}}%
\pgfpathlineto{\pgfqpoint{0.919438in}{4.301443in}}%
\pgfpathlineto{\pgfqpoint{0.920511in}{4.301971in}}%
\pgfpathlineto{\pgfqpoint{0.921585in}{4.301272in}}%
\pgfpathlineto{\pgfqpoint{0.925881in}{4.301718in}}%
\pgfpathlineto{\pgfqpoint{0.926955in}{4.301122in}}%
\pgfpathlineto{\pgfqpoint{0.928028in}{4.301685in}}%
\pgfpathlineto{\pgfqpoint{0.934471in}{4.301500in}}%
\pgfpathlineto{\pgfqpoint{0.936619in}{4.299949in}}%
\pgfpathlineto{\pgfqpoint{0.944136in}{4.302111in}}%
\pgfpathlineto{\pgfqpoint{0.945210in}{4.301605in}}%
\pgfpathlineto{\pgfqpoint{0.948431in}{4.301549in}}%
\pgfpathlineto{\pgfqpoint{0.950579in}{4.300152in}}%
\pgfpathlineto{\pgfqpoint{0.951653in}{4.302107in}}%
\pgfpathlineto{\pgfqpoint{0.952727in}{4.302859in}}%
\pgfpathlineto{\pgfqpoint{0.955948in}{4.302834in}}%
\pgfpathlineto{\pgfqpoint{0.957022in}{4.302209in}}%
\pgfpathlineto{\pgfqpoint{0.958096in}{4.302323in}}%
\pgfpathlineto{\pgfqpoint{0.960244in}{4.300074in}}%
\pgfpathlineto{\pgfqpoint{0.965613in}{4.299835in}}%
\pgfpathlineto{\pgfqpoint{0.967760in}{4.298367in}}%
\pgfpathlineto{\pgfqpoint{0.972056in}{4.298293in}}%
\pgfpathlineto{\pgfqpoint{0.973130in}{4.297728in}}%
\pgfpathlineto{\pgfqpoint{0.975277in}{4.297799in}}%
\pgfpathlineto{\pgfqpoint{0.980646in}{4.297774in}}%
\pgfpathlineto{\pgfqpoint{0.982794in}{4.298093in}}%
\pgfpathlineto{\pgfqpoint{0.989237in}{4.297374in}}%
\pgfpathlineto{\pgfqpoint{0.990311in}{4.298074in}}%
\pgfpathlineto{\pgfqpoint{0.994606in}{4.298225in}}%
\pgfpathlineto{\pgfqpoint{0.995680in}{4.299119in}}%
\pgfpathlineto{\pgfqpoint{0.996754in}{4.299066in}}%
\pgfpathlineto{\pgfqpoint{0.997828in}{4.297512in}}%
\pgfpathlineto{\pgfqpoint{1.002123in}{4.298052in}}%
\pgfpathlineto{\pgfqpoint{1.003197in}{4.298796in}}%
\pgfpathlineto{\pgfqpoint{1.004271in}{4.297417in}}%
\pgfpathlineto{\pgfqpoint{1.005345in}{4.297517in}}%
\pgfpathlineto{\pgfqpoint{1.010714in}{4.297168in}}%
\pgfpathlineto{\pgfqpoint{1.011788in}{4.298861in}}%
\pgfpathlineto{\pgfqpoint{1.012862in}{4.298345in}}%
\pgfpathlineto{\pgfqpoint{1.016083in}{4.299656in}}%
\pgfpathlineto{\pgfqpoint{1.019305in}{4.297900in}}%
\pgfpathlineto{\pgfqpoint{1.020378in}{4.298446in}}%
\pgfpathlineto{\pgfqpoint{1.025748in}{4.298382in}}%
\pgfpathlineto{\pgfqpoint{1.027895in}{4.297068in}}%
\pgfpathlineto{\pgfqpoint{1.031117in}{4.296713in}}%
\pgfpathlineto{\pgfqpoint{1.032191in}{4.295981in}}%
\pgfpathlineto{\pgfqpoint{1.033265in}{4.295933in}}%
\pgfpathlineto{\pgfqpoint{1.035412in}{4.294937in}}%
\pgfpathlineto{\pgfqpoint{1.041855in}{4.294739in}}%
\pgfpathlineto{\pgfqpoint{1.042929in}{4.294341in}}%
\pgfpathlineto{\pgfqpoint{1.055815in}{4.293650in}}%
\pgfpathlineto{\pgfqpoint{1.057963in}{4.293200in}}%
\pgfpathlineto{\pgfqpoint{1.063332in}{4.293427in}}%
\pgfpathlineto{\pgfqpoint{1.064406in}{4.292968in}}%
\pgfpathlineto{\pgfqpoint{1.065480in}{4.293163in}}%
\pgfpathlineto{\pgfqpoint{1.070849in}{4.293532in}}%
\pgfpathlineto{\pgfqpoint{1.071923in}{4.294496in}}%
\pgfpathlineto{\pgfqpoint{1.077292in}{4.294740in}}%
\pgfpathlineto{\pgfqpoint{1.078366in}{4.295509in}}%
\pgfpathlineto{\pgfqpoint{1.079440in}{4.294783in}}%
\pgfpathlineto{\pgfqpoint{1.086956in}{4.295548in}}%
\pgfpathlineto{\pgfqpoint{1.088030in}{4.295398in}}%
\pgfpathlineto{\pgfqpoint{1.092326in}{4.295608in}}%
\pgfpathlineto{\pgfqpoint{1.094473in}{4.294645in}}%
\pgfpathlineto{\pgfqpoint{1.095547in}{4.294553in}}%
\pgfpathlineto{\pgfqpoint{1.103064in}{4.292577in}}%
\pgfpathlineto{\pgfqpoint{1.107359in}{4.291874in}}%
\pgfpathlineto{\pgfqpoint{1.109507in}{4.291028in}}%
\pgfpathlineto{\pgfqpoint{1.110581in}{4.291813in}}%
\pgfpathlineto{\pgfqpoint{1.114876in}{4.291404in}}%
\pgfpathlineto{\pgfqpoint{1.117024in}{4.291473in}}%
\pgfpathlineto{\pgfqpoint{1.121319in}{4.292117in}}%
\pgfpathlineto{\pgfqpoint{1.123467in}{4.292227in}}%
\pgfpathlineto{\pgfqpoint{1.124541in}{4.292564in}}%
\pgfpathlineto{\pgfqpoint{1.125615in}{4.292111in}}%
\pgfpathlineto{\pgfqpoint{1.128836in}{4.292484in}}%
\pgfpathlineto{\pgfqpoint{1.129910in}{4.293295in}}%
\pgfpathlineto{\pgfqpoint{1.130984in}{4.293073in}}%
\pgfpathlineto{\pgfqpoint{1.133132in}{4.291392in}}%
\pgfpathlineto{\pgfqpoint{1.136353in}{4.291046in}}%
\pgfpathlineto{\pgfqpoint{1.137427in}{4.291844in}}%
\pgfpathlineto{\pgfqpoint{1.140648in}{4.290300in}}%
\pgfpathlineto{\pgfqpoint{1.153534in}{4.289341in}}%
\pgfpathlineto{\pgfqpoint{1.155682in}{4.288810in}}%
\pgfpathlineto{\pgfqpoint{1.168568in}{4.287557in}}%
\pgfpathlineto{\pgfqpoint{1.170716in}{4.286989in}}%
\pgfpathlineto{\pgfqpoint{1.181454in}{4.286098in}}%
\pgfpathlineto{\pgfqpoint{1.183602in}{4.285130in}}%
\pgfpathlineto{\pgfqpoint{1.185750in}{4.285125in}}%
\pgfpathlineto{\pgfqpoint{1.190045in}{4.288043in}}%
\pgfpathlineto{\pgfqpoint{1.192193in}{4.287935in}}%
\pgfpathlineto{\pgfqpoint{1.193266in}{4.287117in}}%
\pgfpathlineto{\pgfqpoint{1.197562in}{4.287590in}}%
\pgfpathlineto{\pgfqpoint{1.200783in}{4.288100in}}%
\pgfpathlineto{\pgfqpoint{1.204005in}{4.287535in}}%
\pgfpathlineto{\pgfqpoint{1.206153in}{4.283951in}}%
\pgfpathlineto{\pgfqpoint{1.208300in}{4.283584in}}%
\pgfpathlineto{\pgfqpoint{1.214743in}{4.282967in}}%
\pgfpathlineto{\pgfqpoint{1.215817in}{4.282558in}}%
\pgfpathlineto{\pgfqpoint{1.223334in}{4.282886in}}%
\pgfpathlineto{\pgfqpoint{1.234072in}{4.284364in}}%
\pgfpathlineto{\pgfqpoint{1.236220in}{4.282854in}}%
\pgfpathlineto{\pgfqpoint{1.243737in}{4.283618in}}%
\pgfpathlineto{\pgfqpoint{1.244811in}{4.284317in}}%
\pgfpathlineto{\pgfqpoint{1.245885in}{4.286140in}}%
\pgfpathlineto{\pgfqpoint{1.250180in}{4.286314in}}%
\pgfpathlineto{\pgfqpoint{1.251254in}{4.286121in}}%
\pgfpathlineto{\pgfqpoint{1.252328in}{4.287075in}}%
\pgfpathlineto{\pgfqpoint{1.253401in}{4.287012in}}%
\pgfpathlineto{\pgfqpoint{1.256623in}{4.288933in}}%
\pgfpathlineto{\pgfqpoint{1.257697in}{4.287580in}}%
\pgfpathlineto{\pgfqpoint{1.259844in}{4.286593in}}%
\pgfpathlineto{\pgfqpoint{1.260918in}{4.285961in}}%
\pgfpathlineto{\pgfqpoint{1.264140in}{4.286232in}}%
\pgfpathlineto{\pgfqpoint{1.266287in}{4.285539in}}%
\pgfpathlineto{\pgfqpoint{1.267361in}{4.285668in}}%
\pgfpathlineto{\pgfqpoint{1.268435in}{4.284812in}}%
\pgfpathlineto{\pgfqpoint{1.279174in}{4.284769in}}%
\pgfpathlineto{\pgfqpoint{1.283469in}{4.283724in}}%
\pgfpathlineto{\pgfqpoint{1.287764in}{4.284543in}}%
\pgfpathlineto{\pgfqpoint{1.289912in}{4.283961in}}%
\pgfpathlineto{\pgfqpoint{1.294207in}{4.284172in}}%
\pgfpathlineto{\pgfqpoint{1.298503in}{4.285723in}}%
\pgfpathlineto{\pgfqpoint{1.302798in}{4.284473in}}%
\pgfpathlineto{\pgfqpoint{1.303872in}{4.285098in}}%
\pgfpathlineto{\pgfqpoint{1.306020in}{4.284303in}}%
\pgfpathlineto{\pgfqpoint{1.309241in}{4.284592in}}%
\pgfpathlineto{\pgfqpoint{1.310315in}{4.283911in}}%
\pgfpathlineto{\pgfqpoint{1.312463in}{4.284409in}}%
\pgfpathlineto{\pgfqpoint{1.313536in}{4.283846in}}%
\pgfpathlineto{\pgfqpoint{1.319979in}{4.283064in}}%
\pgfpathlineto{\pgfqpoint{1.327496in}{4.283910in}}%
\pgfpathlineto{\pgfqpoint{1.328570in}{4.284551in}}%
\pgfpathlineto{\pgfqpoint{1.332865in}{4.283846in}}%
\pgfpathlineto{\pgfqpoint{1.335013in}{4.282390in}}%
\pgfpathlineto{\pgfqpoint{1.341456in}{4.282288in}}%
\pgfpathlineto{\pgfqpoint{1.343604in}{4.282836in}}%
\pgfpathlineto{\pgfqpoint{1.348973in}{4.281855in}}%
\pgfpathlineto{\pgfqpoint{1.350047in}{4.281208in}}%
\pgfpathlineto{\pgfqpoint{1.351121in}{4.281462in}}%
\pgfpathlineto{\pgfqpoint{1.369376in}{4.280871in}}%
\pgfpathlineto{\pgfqpoint{1.370450in}{4.281394in}}%
\pgfpathlineto{\pgfqpoint{1.372598in}{4.281042in}}%
\pgfpathlineto{\pgfqpoint{1.373671in}{4.280739in}}%
\pgfpathlineto{\pgfqpoint{1.386557in}{4.280330in}}%
\pgfpathlineto{\pgfqpoint{1.388705in}{4.279489in}}%
\pgfpathlineto{\pgfqpoint{1.394074in}{4.279566in}}%
\pgfpathlineto{\pgfqpoint{1.396222in}{4.279926in}}%
\pgfpathlineto{\pgfqpoint{1.402665in}{4.279532in}}%
\pgfpathlineto{\pgfqpoint{1.403739in}{4.279302in}}%
\pgfpathlineto{\pgfqpoint{1.411256in}{4.279805in}}%
\pgfpathlineto{\pgfqpoint{1.415551in}{4.279497in}}%
\pgfpathlineto{\pgfqpoint{1.417699in}{4.279204in}}%
\pgfpathlineto{\pgfqpoint{1.430585in}{4.279318in}}%
\pgfpathlineto{\pgfqpoint{1.431659in}{4.278940in}}%
\pgfpathlineto{\pgfqpoint{1.432732in}{4.279828in}}%
\pgfpathlineto{\pgfqpoint{1.433806in}{4.279374in}}%
\pgfpathlineto{\pgfqpoint{1.441323in}{4.279272in}}%
\pgfpathlineto{\pgfqpoint{1.444545in}{4.279131in}}%
\pgfpathlineto{\pgfqpoint{1.448840in}{4.281023in}}%
\pgfpathlineto{\pgfqpoint{1.453135in}{4.281137in}}%
\pgfpathlineto{\pgfqpoint{1.455283in}{4.281379in}}%
\pgfpathlineto{\pgfqpoint{1.456357in}{4.280738in}}%
\pgfpathlineto{\pgfqpoint{1.461726in}{4.280392in}}%
\pgfpathlineto{\pgfqpoint{1.470317in}{4.279132in}}%
\pgfpathlineto{\pgfqpoint{1.471391in}{4.279257in}}%
\pgfpathlineto{\pgfqpoint{1.478908in}{4.279303in}}%
\pgfpathlineto{\pgfqpoint{1.499310in}{4.278829in}}%
\pgfpathlineto{\pgfqpoint{1.501458in}{4.278380in}}%
\pgfpathlineto{\pgfqpoint{1.506827in}{4.278984in}}%
\pgfpathlineto{\pgfqpoint{1.507901in}{4.279801in}}%
\pgfpathlineto{\pgfqpoint{1.513270in}{4.280083in}}%
\pgfpathlineto{\pgfqpoint{1.515418in}{4.281163in}}%
\pgfpathlineto{\pgfqpoint{1.516492in}{4.280607in}}%
\pgfpathlineto{\pgfqpoint{1.519713in}{4.280545in}}%
\pgfpathlineto{\pgfqpoint{1.520787in}{4.281510in}}%
\pgfpathlineto{\pgfqpoint{1.521861in}{4.280103in}}%
\pgfpathlineto{\pgfqpoint{1.522935in}{4.281093in}}%
\pgfpathlineto{\pgfqpoint{1.524009in}{4.282877in}}%
\pgfpathlineto{\pgfqpoint{1.527230in}{4.283250in}}%
\pgfpathlineto{\pgfqpoint{1.529378in}{4.282716in}}%
\pgfpathlineto{\pgfqpoint{1.530452in}{4.282386in}}%
\pgfpathlineto{\pgfqpoint{1.531526in}{4.281468in}}%
\pgfpathlineto{\pgfqpoint{1.534747in}{4.281394in}}%
\pgfpathlineto{\pgfqpoint{1.535821in}{4.280140in}}%
\pgfpathlineto{\pgfqpoint{1.536895in}{4.280852in}}%
\pgfpathlineto{\pgfqpoint{1.539042in}{4.277402in}}%
\pgfpathlineto{\pgfqpoint{1.545486in}{4.276292in}}%
\pgfpathlineto{\pgfqpoint{1.546559in}{4.275777in}}%
\pgfpathlineto{\pgfqpoint{1.549781in}{4.275921in}}%
\pgfpathlineto{\pgfqpoint{1.551929in}{4.275201in}}%
\pgfpathlineto{\pgfqpoint{1.572331in}{4.273759in}}%
\pgfpathlineto{\pgfqpoint{1.574479in}{4.274190in}}%
\pgfpathlineto{\pgfqpoint{1.576627in}{4.273688in}}%
\pgfpathlineto{\pgfqpoint{1.579848in}{4.274219in}}%
\pgfpathlineto{\pgfqpoint{1.581996in}{4.273078in}}%
\pgfpathlineto{\pgfqpoint{1.588439in}{4.273461in}}%
\pgfpathlineto{\pgfqpoint{1.589513in}{4.274167in}}%
\pgfpathlineto{\pgfqpoint{1.590587in}{4.273911in}}%
\pgfpathlineto{\pgfqpoint{1.591661in}{4.274477in}}%
\pgfpathlineto{\pgfqpoint{1.594882in}{4.274555in}}%
\pgfpathlineto{\pgfqpoint{1.599177in}{4.272159in}}%
\pgfpathlineto{\pgfqpoint{1.604547in}{4.271778in}}%
\pgfpathlineto{\pgfqpoint{1.614211in}{4.272516in}}%
\pgfpathlineto{\pgfqpoint{1.619580in}{4.273633in}}%
\pgfpathlineto{\pgfqpoint{1.620654in}{4.272575in}}%
\pgfpathlineto{\pgfqpoint{1.621728in}{4.273103in}}%
\pgfpathlineto{\pgfqpoint{1.628171in}{4.273636in}}%
\pgfpathlineto{\pgfqpoint{1.629245in}{4.272991in}}%
\pgfpathlineto{\pgfqpoint{1.634614in}{4.272845in}}%
\pgfpathlineto{\pgfqpoint{1.635688in}{4.271954in}}%
\pgfpathlineto{\pgfqpoint{1.636762in}{4.272432in}}%
\pgfpathlineto{\pgfqpoint{1.643205in}{4.271879in}}%
\pgfpathlineto{\pgfqpoint{1.644279in}{4.272853in}}%
\pgfpathlineto{\pgfqpoint{1.651796in}{4.271854in}}%
\pgfpathlineto{\pgfqpoint{1.657165in}{4.271995in}}%
\pgfpathlineto{\pgfqpoint{1.659312in}{4.271617in}}%
\pgfpathlineto{\pgfqpoint{1.678641in}{4.271018in}}%
\pgfpathlineto{\pgfqpoint{1.681863in}{4.271987in}}%
\pgfpathlineto{\pgfqpoint{1.685085in}{4.271466in}}%
\pgfpathlineto{\pgfqpoint{1.686158in}{4.272502in}}%
\pgfpathlineto{\pgfqpoint{1.687232in}{4.272642in}}%
\pgfpathlineto{\pgfqpoint{1.688306in}{4.271913in}}%
\pgfpathlineto{\pgfqpoint{1.689380in}{4.272357in}}%
\pgfpathlineto{\pgfqpoint{1.692601in}{4.271447in}}%
\pgfpathlineto{\pgfqpoint{1.693675in}{4.271944in}}%
\pgfpathlineto{\pgfqpoint{1.694749in}{4.271365in}}%
\pgfpathlineto{\pgfqpoint{1.696897in}{4.271360in}}%
\pgfpathlineto{\pgfqpoint{1.701192in}{4.271402in}}%
\pgfpathlineto{\pgfqpoint{1.702266in}{4.272313in}}%
\pgfpathlineto{\pgfqpoint{1.704414in}{4.272102in}}%
\pgfpathlineto{\pgfqpoint{1.708709in}{4.271715in}}%
\pgfpathlineto{\pgfqpoint{1.709783in}{4.272337in}}%
\pgfpathlineto{\pgfqpoint{1.725890in}{4.271452in}}%
\pgfpathlineto{\pgfqpoint{1.726964in}{4.272508in}}%
\pgfpathlineto{\pgfqpoint{1.732333in}{4.271721in}}%
\pgfpathlineto{\pgfqpoint{1.733407in}{4.273002in}}%
\pgfpathlineto{\pgfqpoint{1.740924in}{4.273903in}}%
\pgfpathlineto{\pgfqpoint{1.741998in}{4.273539in}}%
\pgfpathlineto{\pgfqpoint{1.746293in}{4.273471in}}%
\pgfpathlineto{\pgfqpoint{1.748441in}{4.273276in}}%
\pgfpathlineto{\pgfqpoint{1.749515in}{4.272726in}}%
\pgfpathlineto{\pgfqpoint{1.754884in}{4.272533in}}%
\pgfpathlineto{\pgfqpoint{1.757032in}{4.272005in}}%
\pgfpathlineto{\pgfqpoint{1.778508in}{4.273033in}}%
\pgfpathlineto{\pgfqpoint{1.779582in}{4.273381in}}%
\pgfpathlineto{\pgfqpoint{1.783878in}{4.273408in}}%
\pgfpathlineto{\pgfqpoint{1.786025in}{4.272610in}}%
\pgfpathlineto{\pgfqpoint{1.787099in}{4.273099in}}%
\pgfpathlineto{\pgfqpoint{1.791395in}{4.273523in}}%
\pgfpathlineto{\pgfqpoint{1.792468in}{4.273394in}}%
\pgfpathlineto{\pgfqpoint{1.793542in}{4.272646in}}%
\pgfpathlineto{\pgfqpoint{1.794616in}{4.272828in}}%
\pgfpathlineto{\pgfqpoint{1.798911in}{4.272587in}}%
\pgfpathlineto{\pgfqpoint{1.801059in}{4.273655in}}%
\pgfpathlineto{\pgfqpoint{1.802133in}{4.273323in}}%
\pgfpathlineto{\pgfqpoint{1.806428in}{4.274096in}}%
\pgfpathlineto{\pgfqpoint{1.808576in}{4.273784in}}%
\pgfpathlineto{\pgfqpoint{1.813945in}{4.273768in}}%
\pgfpathlineto{\pgfqpoint{1.815019in}{4.274542in}}%
\pgfpathlineto{\pgfqpoint{1.820388in}{4.273454in}}%
\pgfpathlineto{\pgfqpoint{1.827905in}{4.273268in}}%
\pgfpathlineto{\pgfqpoint{1.830053in}{4.273743in}}%
\pgfpathlineto{\pgfqpoint{1.831127in}{4.275425in}}%
\pgfpathlineto{\pgfqpoint{1.835422in}{4.275665in}}%
\pgfpathlineto{\pgfqpoint{1.837570in}{4.274822in}}%
\pgfpathlineto{\pgfqpoint{1.839717in}{4.274852in}}%
\pgfpathlineto{\pgfqpoint{1.847234in}{4.275587in}}%
\pgfpathlineto{\pgfqpoint{1.850456in}{4.274860in}}%
\pgfpathlineto{\pgfqpoint{1.851529in}{4.275689in}}%
\pgfpathlineto{\pgfqpoint{1.927772in}{4.274464in}}%
\pgfpathlineto{\pgfqpoint{1.928846in}{4.272614in}}%
\pgfpathlineto{\pgfqpoint{1.934215in}{4.272411in}}%
\pgfpathlineto{\pgfqpoint{1.936363in}{4.272169in}}%
\pgfpathlineto{\pgfqpoint{1.937437in}{4.272277in}}%
\pgfpathlineto{\pgfqpoint{1.941732in}{4.271846in}}%
\pgfpathlineto{\pgfqpoint{1.944953in}{4.271713in}}%
\pgfpathlineto{\pgfqpoint{1.949249in}{4.272154in}}%
\pgfpathlineto{\pgfqpoint{1.950323in}{4.271760in}}%
\pgfpathlineto{\pgfqpoint{1.952470in}{4.272666in}}%
\pgfpathlineto{\pgfqpoint{1.959987in}{4.271479in}}%
\pgfpathlineto{\pgfqpoint{1.972873in}{4.272345in}}%
\pgfpathlineto{\pgfqpoint{1.973947in}{4.272707in}}%
\pgfpathlineto{\pgfqpoint{1.975021in}{4.271635in}}%
\pgfpathlineto{\pgfqpoint{1.981464in}{4.271814in}}%
\pgfpathlineto{\pgfqpoint{1.982538in}{4.272535in}}%
\pgfpathlineto{\pgfqpoint{1.990055in}{4.271243in}}%
\pgfpathlineto{\pgfqpoint{1.994350in}{4.271243in}}%
\pgfpathlineto{\pgfqpoint{1.996498in}{4.270624in}}%
\pgfpathlineto{\pgfqpoint{2.034082in}{4.270762in}}%
\pgfpathlineto{\pgfqpoint{2.035156in}{4.269733in}}%
\pgfpathlineto{\pgfqpoint{2.039451in}{4.269733in}}%
\pgfpathlineto{\pgfqpoint{2.040525in}{4.268470in}}%
\pgfpathlineto{\pgfqpoint{2.042673in}{4.268228in}}%
\pgfpathlineto{\pgfqpoint{2.048042in}{4.268204in}}%
\pgfpathlineto{\pgfqpoint{2.049116in}{4.268655in}}%
\pgfpathlineto{\pgfqpoint{2.050190in}{4.267801in}}%
\pgfpathlineto{\pgfqpoint{2.065223in}{4.266743in}}%
\pgfpathlineto{\pgfqpoint{2.068445in}{4.267070in}}%
\pgfpathlineto{\pgfqpoint{2.069519in}{4.266420in}}%
\pgfpathlineto{\pgfqpoint{2.079183in}{4.266350in}}%
\pgfpathlineto{\pgfqpoint{2.080257in}{4.265894in}}%
\pgfpathlineto{\pgfqpoint{2.086700in}{4.265476in}}%
\pgfpathlineto{\pgfqpoint{2.087774in}{4.265132in}}%
\pgfpathlineto{\pgfqpoint{2.099586in}{4.265323in}}%
\pgfpathlineto{\pgfqpoint{2.101734in}{4.264836in}}%
\pgfpathlineto{\pgfqpoint{2.102808in}{4.264646in}}%
\pgfpathlineto{\pgfqpoint{2.110325in}{4.264831in}}%
\pgfpathlineto{\pgfqpoint{2.117841in}{4.264378in}}%
\pgfpathlineto{\pgfqpoint{2.125358in}{4.264402in}}%
\pgfpathlineto{\pgfqpoint{2.143614in}{4.264417in}}%
\pgfpathlineto{\pgfqpoint{2.145761in}{4.264156in}}%
\pgfpathlineto{\pgfqpoint{2.151130in}{4.264195in}}%
\pgfpathlineto{\pgfqpoint{2.154352in}{4.264882in}}%
\pgfpathlineto{\pgfqpoint{2.158647in}{4.264670in}}%
\pgfpathlineto{\pgfqpoint{2.160795in}{4.263858in}}%
\pgfpathlineto{\pgfqpoint{2.162943in}{4.264080in}}%
\pgfpathlineto{\pgfqpoint{2.173681in}{4.263835in}}%
\pgfpathlineto{\pgfqpoint{2.176903in}{4.263821in}}%
\pgfpathlineto{\pgfqpoint{2.177976in}{4.264137in}}%
\pgfpathlineto{\pgfqpoint{2.183346in}{4.264291in}}%
\pgfpathlineto{\pgfqpoint{2.184419in}{4.263849in}}%
\pgfpathlineto{\pgfqpoint{2.185493in}{4.264049in}}%
\pgfpathlineto{\pgfqpoint{2.191936in}{4.263010in}}%
\pgfpathlineto{\pgfqpoint{2.193010in}{4.263969in}}%
\pgfpathlineto{\pgfqpoint{2.196232in}{4.264361in}}%
\pgfpathlineto{\pgfqpoint{2.200527in}{4.262681in}}%
\pgfpathlineto{\pgfqpoint{2.211265in}{4.262214in}}%
\pgfpathlineto{\pgfqpoint{2.215561in}{4.261600in}}%
\pgfpathlineto{\pgfqpoint{2.256367in}{4.261798in}}%
\pgfpathlineto{\pgfqpoint{2.260662in}{4.261735in}}%
\pgfpathlineto{\pgfqpoint{2.272474in}{4.261546in}}%
\pgfpathlineto{\pgfqpoint{2.283213in}{4.262496in}}%
\pgfpathlineto{\pgfqpoint{2.290729in}{4.262050in}}%
\pgfpathlineto{\pgfqpoint{2.313280in}{4.263366in}}%
\pgfpathlineto{\pgfqpoint{2.324018in}{4.263179in}}%
\pgfpathlineto{\pgfqpoint{2.325092in}{4.264167in}}%
\pgfpathlineto{\pgfqpoint{2.331535in}{4.264361in}}%
\pgfpathlineto{\pgfqpoint{2.333683in}{4.264266in}}%
\pgfpathlineto{\pgfqpoint{2.340126in}{4.263175in}}%
\pgfpathlineto{\pgfqpoint{2.341200in}{4.263300in}}%
\pgfpathlineto{\pgfqpoint{2.343348in}{4.262379in}}%
\pgfpathlineto{\pgfqpoint{2.348717in}{4.262744in}}%
\pgfpathlineto{\pgfqpoint{2.350864in}{4.262560in}}%
\pgfpathlineto{\pgfqpoint{2.356234in}{4.262693in}}%
\pgfpathlineto{\pgfqpoint{2.361603in}{4.262643in}}%
\pgfpathlineto{\pgfqpoint{2.372341in}{4.261985in}}%
\pgfpathlineto{\pgfqpoint{2.373415in}{4.261505in}}%
\pgfpathlineto{\pgfqpoint{2.377710in}{4.261441in}}%
\pgfpathlineto{\pgfqpoint{2.379858in}{4.261944in}}%
\pgfpathlineto{\pgfqpoint{2.380932in}{4.261681in}}%
\pgfpathlineto{\pgfqpoint{2.395966in}{4.261477in}}%
\pgfpathlineto{\pgfqpoint{2.418516in}{4.261487in}}%
\pgfpathlineto{\pgfqpoint{2.426033in}{4.261661in}}%
\pgfpathlineto{\pgfqpoint{2.433550in}{4.262100in}}%
\pgfpathlineto{\pgfqpoint{2.438919in}{4.261726in}}%
\pgfpathlineto{\pgfqpoint{2.447510in}{4.260404in}}%
\pgfpathlineto{\pgfqpoint{2.452879in}{4.260411in}}%
\pgfpathlineto{\pgfqpoint{2.455027in}{4.259773in}}%
\pgfpathlineto{\pgfqpoint{2.486168in}{4.259073in}}%
\pgfpathlineto{\pgfqpoint{2.501202in}{4.259306in}}%
\pgfpathlineto{\pgfqpoint{2.516236in}{4.259067in}}%
\pgfpathlineto{\pgfqpoint{2.523752in}{4.258407in}}%
\pgfpathlineto{\pgfqpoint{2.529122in}{4.258141in}}%
\pgfpathlineto{\pgfqpoint{2.531269in}{4.257884in}}%
\pgfpathlineto{\pgfqpoint{2.546303in}{4.258227in}}%
\pgfpathlineto{\pgfqpoint{2.561337in}{4.256699in}}%
\pgfpathlineto{\pgfqpoint{2.573149in}{4.256384in}}%
\pgfpathlineto{\pgfqpoint{2.581740in}{4.255907in}}%
\pgfpathlineto{\pgfqpoint{2.591404in}{4.256487in}}%
\pgfpathlineto{\pgfqpoint{2.602143in}{4.256188in}}%
\pgfpathlineto{\pgfqpoint{2.605364in}{4.256110in}}%
\pgfpathlineto{\pgfqpoint{2.612881in}{4.255940in}}%
\pgfpathlineto{\pgfqpoint{2.617176in}{4.256245in}}%
\pgfpathlineto{\pgfqpoint{2.618250in}{4.257592in}}%
\pgfpathlineto{\pgfqpoint{2.624693in}{4.257341in}}%
\pgfpathlineto{\pgfqpoint{2.628989in}{4.256467in}}%
\pgfpathlineto{\pgfqpoint{2.666573in}{4.256545in}}%
\pgfpathlineto{\pgfqpoint{2.674090in}{4.255564in}}%
\pgfpathlineto{\pgfqpoint{2.685902in}{4.255843in}}%
\pgfpathlineto{\pgfqpoint{2.689124in}{4.255983in}}%
\pgfpathlineto{\pgfqpoint{2.700936in}{4.255140in}}%
\pgfpathlineto{\pgfqpoint{2.710600in}{4.254847in}}%
\pgfpathlineto{\pgfqpoint{2.711674in}{4.254608in}}%
\pgfpathlineto{\pgfqpoint{2.714896in}{4.254582in}}%
\pgfpathlineto{\pgfqpoint{2.715969in}{4.253104in}}%
\pgfpathlineto{\pgfqpoint{2.717043in}{4.252800in}}%
\pgfpathlineto{\pgfqpoint{2.718117in}{4.253264in}}%
\pgfpathlineto{\pgfqpoint{2.719191in}{4.253080in}}%
\pgfpathlineto{\pgfqpoint{2.725634in}{4.253326in}}%
\pgfpathlineto{\pgfqpoint{2.753554in}{4.252849in}}%
\pgfpathlineto{\pgfqpoint{2.756775in}{4.252302in}}%
\pgfpathlineto{\pgfqpoint{2.771809in}{4.252592in}}%
\pgfpathlineto{\pgfqpoint{2.786843in}{4.252846in}}%
\pgfpathlineto{\pgfqpoint{2.786843in}{4.252846in}}%
\pgfusepath{stroke}%
\end{pgfscope}%
\begin{pgfscope}%
\pgfsetrectcap%
\pgfsetmiterjoin%
\pgfsetlinewidth{0.803000pt}%
\definecolor{currentstroke}{rgb}{1.000000,1.000000,1.000000}%
\pgfsetstrokecolor{currentstroke}%
\pgfsetdash{}{0pt}%
\pgfpathmoveto{\pgfqpoint{0.320934in}{4.233896in}}%
\pgfpathlineto{\pgfqpoint{0.320934in}{4.634781in}}%
\pgfusepath{stroke}%
\end{pgfscope}%
\begin{pgfscope}%
\pgfsetrectcap%
\pgfsetmiterjoin%
\pgfsetlinewidth{0.803000pt}%
\definecolor{currentstroke}{rgb}{1.000000,1.000000,1.000000}%
\pgfsetstrokecolor{currentstroke}%
\pgfsetdash{}{0pt}%
\pgfpathmoveto{\pgfqpoint{2.904267in}{4.233896in}}%
\pgfpathlineto{\pgfqpoint{2.904267in}{4.634781in}}%
\pgfusepath{stroke}%
\end{pgfscope}%
\begin{pgfscope}%
\pgfsetrectcap%
\pgfsetmiterjoin%
\pgfsetlinewidth{0.803000pt}%
\definecolor{currentstroke}{rgb}{1.000000,1.000000,1.000000}%
\pgfsetstrokecolor{currentstroke}%
\pgfsetdash{}{0pt}%
\pgfpathmoveto{\pgfqpoint{0.320934in}{4.233896in}}%
\pgfpathlineto{\pgfqpoint{2.904267in}{4.233896in}}%
\pgfusepath{stroke}%
\end{pgfscope}%
\begin{pgfscope}%
\pgfsetrectcap%
\pgfsetmiterjoin%
\pgfsetlinewidth{0.803000pt}%
\definecolor{currentstroke}{rgb}{1.000000,1.000000,1.000000}%
\pgfsetstrokecolor{currentstroke}%
\pgfsetdash{}{0pt}%
\pgfpathmoveto{\pgfqpoint{0.320934in}{4.634781in}}%
\pgfpathlineto{\pgfqpoint{2.904267in}{4.634781in}}%
\pgfusepath{stroke}%
\end{pgfscope}%
\begin{pgfscope}%
\definecolor{textcolor}{rgb}{0.150000,0.150000,0.150000}%
\pgfsetstrokecolor{textcolor}%
\pgfsetfillcolor{textcolor}%
\pgftext[x=1.612600in,y=4.718114in,,base]{\color{textcolor}\rmfamily\fontsize{16.800000}{20.160000}\selectfont MMM}%
\end{pgfscope}%
\begin{pgfscope}%
\pgfsetbuttcap%
\pgfsetmiterjoin%
\definecolor{currentfill}{rgb}{0.917647,0.917647,0.949020}%
\pgfsetfillcolor{currentfill}%
\pgfsetlinewidth{0.000000pt}%
\definecolor{currentstroke}{rgb}{0.000000,0.000000,0.000000}%
\pgfsetstrokecolor{currentstroke}%
\pgfsetstrokeopacity{0.000000}%
\pgfsetdash{}{0pt}%
\pgfpathmoveto{\pgfqpoint{3.937600in}{4.233896in}}%
\pgfpathlineto{\pgfqpoint{6.520934in}{4.233896in}}%
\pgfpathlineto{\pgfqpoint{6.520934in}{4.634781in}}%
\pgfpathlineto{\pgfqpoint{3.937600in}{4.634781in}}%
\pgfpathclose%
\pgfusepath{fill}%
\end{pgfscope}%
\begin{pgfscope}%
\pgfpathrectangle{\pgfqpoint{3.937600in}{4.233896in}}{\pgfqpoint{2.583333in}{0.400885in}}%
\pgfusepath{clip}%
\pgfsetroundcap%
\pgfsetroundjoin%
\pgfsetlinewidth{0.803000pt}%
\definecolor{currentstroke}{rgb}{1.000000,1.000000,1.000000}%
\pgfsetstrokecolor{currentstroke}%
\pgfsetdash{}{0pt}%
\pgfpathmoveto{\pgfqpoint{4.052877in}{4.233896in}}%
\pgfpathlineto{\pgfqpoint{4.052877in}{4.634781in}}%
\pgfusepath{stroke}%
\end{pgfscope}%
\begin{pgfscope}%
\definecolor{textcolor}{rgb}{0.150000,0.150000,0.150000}%
\pgfsetstrokecolor{textcolor}%
\pgfsetfillcolor{textcolor}%
\pgftext[x=4.052877in,y=4.136674in,,top]{\color{textcolor}\rmfamily\fontsize{14.000000}{16.800000}\selectfont 2012}%
\end{pgfscope}%
\begin{pgfscope}%
\pgfpathrectangle{\pgfqpoint{3.937600in}{4.233896in}}{\pgfqpoint{2.583333in}{0.400885in}}%
\pgfusepath{clip}%
\pgfsetroundcap%
\pgfsetroundjoin%
\pgfsetlinewidth{0.803000pt}%
\definecolor{currentstroke}{rgb}{1.000000,1.000000,1.000000}%
\pgfsetstrokecolor{currentstroke}%
\pgfsetdash{}{0pt}%
\pgfpathmoveto{\pgfqpoint{4.445902in}{4.233896in}}%
\pgfpathlineto{\pgfqpoint{4.445902in}{4.634781in}}%
\pgfusepath{stroke}%
\end{pgfscope}%
\begin{pgfscope}%
\definecolor{textcolor}{rgb}{0.150000,0.150000,0.150000}%
\pgfsetstrokecolor{textcolor}%
\pgfsetfillcolor{textcolor}%
\pgftext[x=4.445902in,y=4.136674in,,top]{\color{textcolor}\rmfamily\fontsize{14.000000}{16.800000}\selectfont 2013}%
\end{pgfscope}%
\begin{pgfscope}%
\pgfpathrectangle{\pgfqpoint{3.937600in}{4.233896in}}{\pgfqpoint{2.583333in}{0.400885in}}%
\pgfusepath{clip}%
\pgfsetroundcap%
\pgfsetroundjoin%
\pgfsetlinewidth{0.803000pt}%
\definecolor{currentstroke}{rgb}{1.000000,1.000000,1.000000}%
\pgfsetstrokecolor{currentstroke}%
\pgfsetdash{}{0pt}%
\pgfpathmoveto{\pgfqpoint{4.837853in}{4.233896in}}%
\pgfpathlineto{\pgfqpoint{4.837853in}{4.634781in}}%
\pgfusepath{stroke}%
\end{pgfscope}%
\begin{pgfscope}%
\definecolor{textcolor}{rgb}{0.150000,0.150000,0.150000}%
\pgfsetstrokecolor{textcolor}%
\pgfsetfillcolor{textcolor}%
\pgftext[x=4.837853in,y=4.136674in,,top]{\color{textcolor}\rmfamily\fontsize{14.000000}{16.800000}\selectfont 2014}%
\end{pgfscope}%
\begin{pgfscope}%
\pgfpathrectangle{\pgfqpoint{3.937600in}{4.233896in}}{\pgfqpoint{2.583333in}{0.400885in}}%
\pgfusepath{clip}%
\pgfsetroundcap%
\pgfsetroundjoin%
\pgfsetlinewidth{0.803000pt}%
\definecolor{currentstroke}{rgb}{1.000000,1.000000,1.000000}%
\pgfsetstrokecolor{currentstroke}%
\pgfsetdash{}{0pt}%
\pgfpathmoveto{\pgfqpoint{5.229804in}{4.233896in}}%
\pgfpathlineto{\pgfqpoint{5.229804in}{4.634781in}}%
\pgfusepath{stroke}%
\end{pgfscope}%
\begin{pgfscope}%
\definecolor{textcolor}{rgb}{0.150000,0.150000,0.150000}%
\pgfsetstrokecolor{textcolor}%
\pgfsetfillcolor{textcolor}%
\pgftext[x=5.229804in,y=4.136674in,,top]{\color{textcolor}\rmfamily\fontsize{14.000000}{16.800000}\selectfont 2015}%
\end{pgfscope}%
\begin{pgfscope}%
\pgfpathrectangle{\pgfqpoint{3.937600in}{4.233896in}}{\pgfqpoint{2.583333in}{0.400885in}}%
\pgfusepath{clip}%
\pgfsetroundcap%
\pgfsetroundjoin%
\pgfsetlinewidth{0.803000pt}%
\definecolor{currentstroke}{rgb}{1.000000,1.000000,1.000000}%
\pgfsetstrokecolor{currentstroke}%
\pgfsetdash{}{0pt}%
\pgfpathmoveto{\pgfqpoint{5.621755in}{4.233896in}}%
\pgfpathlineto{\pgfqpoint{5.621755in}{4.634781in}}%
\pgfusepath{stroke}%
\end{pgfscope}%
\begin{pgfscope}%
\definecolor{textcolor}{rgb}{0.150000,0.150000,0.150000}%
\pgfsetstrokecolor{textcolor}%
\pgfsetfillcolor{textcolor}%
\pgftext[x=5.621755in,y=4.136674in,,top]{\color{textcolor}\rmfamily\fontsize{14.000000}{16.800000}\selectfont 2016}%
\end{pgfscope}%
\begin{pgfscope}%
\pgfpathrectangle{\pgfqpoint{3.937600in}{4.233896in}}{\pgfqpoint{2.583333in}{0.400885in}}%
\pgfusepath{clip}%
\pgfsetroundcap%
\pgfsetroundjoin%
\pgfsetlinewidth{0.803000pt}%
\definecolor{currentstroke}{rgb}{1.000000,1.000000,1.000000}%
\pgfsetstrokecolor{currentstroke}%
\pgfsetdash{}{0pt}%
\pgfpathmoveto{\pgfqpoint{6.014780in}{4.233896in}}%
\pgfpathlineto{\pgfqpoint{6.014780in}{4.634781in}}%
\pgfusepath{stroke}%
\end{pgfscope}%
\begin{pgfscope}%
\definecolor{textcolor}{rgb}{0.150000,0.150000,0.150000}%
\pgfsetstrokecolor{textcolor}%
\pgfsetfillcolor{textcolor}%
\pgftext[x=6.014780in,y=4.136674in,,top]{\color{textcolor}\rmfamily\fontsize{14.000000}{16.800000}\selectfont 2017}%
\end{pgfscope}%
\begin{pgfscope}%
\pgfpathrectangle{\pgfqpoint{3.937600in}{4.233896in}}{\pgfqpoint{2.583333in}{0.400885in}}%
\pgfusepath{clip}%
\pgfsetroundcap%
\pgfsetroundjoin%
\pgfsetlinewidth{0.803000pt}%
\definecolor{currentstroke}{rgb}{1.000000,1.000000,1.000000}%
\pgfsetstrokecolor{currentstroke}%
\pgfsetdash{}{0pt}%
\pgfpathmoveto{\pgfqpoint{6.406731in}{4.233896in}}%
\pgfpathlineto{\pgfqpoint{6.406731in}{4.634781in}}%
\pgfusepath{stroke}%
\end{pgfscope}%
\begin{pgfscope}%
\definecolor{textcolor}{rgb}{0.150000,0.150000,0.150000}%
\pgfsetstrokecolor{textcolor}%
\pgfsetfillcolor{textcolor}%
\pgftext[x=6.406731in,y=4.136674in,,top]{\color{textcolor}\rmfamily\fontsize{14.000000}{16.800000}\selectfont 2018}%
\end{pgfscope}%
\begin{pgfscope}%
\pgfpathrectangle{\pgfqpoint{3.937600in}{4.233896in}}{\pgfqpoint{2.583333in}{0.400885in}}%
\pgfusepath{clip}%
\pgfsetroundcap%
\pgfsetroundjoin%
\pgfsetlinewidth{0.803000pt}%
\definecolor{currentstroke}{rgb}{1.000000,1.000000,1.000000}%
\pgfsetstrokecolor{currentstroke}%
\pgfsetdash{}{0pt}%
\pgfpathmoveto{\pgfqpoint{3.937600in}{4.384956in}}%
\pgfpathlineto{\pgfqpoint{6.520934in}{4.384956in}}%
\pgfusepath{stroke}%
\end{pgfscope}%
\begin{pgfscope}%
\definecolor{textcolor}{rgb}{0.150000,0.150000,0.150000}%
\pgfsetstrokecolor{textcolor}%
\pgfsetfillcolor{textcolor}%
\pgftext[x=3.716667in,y=4.311089in,left,base]{\color{textcolor}\rmfamily\fontsize{14.000000}{16.800000}\selectfont 1}%
\end{pgfscope}%
\begin{pgfscope}%
\pgfpathrectangle{\pgfqpoint{3.937600in}{4.233896in}}{\pgfqpoint{2.583333in}{0.400885in}}%
\pgfusepath{clip}%
\pgfsetroundcap%
\pgfsetroundjoin%
\pgfsetlinewidth{0.803000pt}%
\definecolor{currentstroke}{rgb}{1.000000,1.000000,1.000000}%
\pgfsetstrokecolor{currentstroke}%
\pgfsetdash{}{0pt}%
\pgfpathmoveto{\pgfqpoint{3.937600in}{4.570788in}}%
\pgfpathlineto{\pgfqpoint{6.520934in}{4.570788in}}%
\pgfusepath{stroke}%
\end{pgfscope}%
\begin{pgfscope}%
\definecolor{textcolor}{rgb}{0.150000,0.150000,0.150000}%
\pgfsetstrokecolor{textcolor}%
\pgfsetfillcolor{textcolor}%
\pgftext[x=3.716667in,y=4.496922in,left,base]{\color{textcolor}\rmfamily\fontsize{14.000000}{16.800000}\selectfont 2}%
\end{pgfscope}%
\begin{pgfscope}%
\pgfpathrectangle{\pgfqpoint{3.937600in}{4.233896in}}{\pgfqpoint{2.583333in}{0.400885in}}%
\pgfusepath{clip}%
\pgfsetroundcap%
\pgfsetroundjoin%
\pgfsetlinewidth{1.505625pt}%
\definecolor{currentstroke}{rgb}{0.000000,0.000000,0.000000}%
\pgfsetstrokecolor{currentstroke}%
\pgfsetdash{}{0pt}%
\pgfpathmoveto{\pgfqpoint{4.055025in}{4.384956in}}%
\pgfpathlineto{\pgfqpoint{4.056098in}{4.385084in}}%
\pgfpathlineto{\pgfqpoint{4.057172in}{4.387231in}}%
\pgfpathlineto{\pgfqpoint{4.058246in}{4.385170in}}%
\pgfpathlineto{\pgfqpoint{4.061468in}{4.385643in}}%
\pgfpathlineto{\pgfqpoint{4.063615in}{4.387789in}}%
\pgfpathlineto{\pgfqpoint{4.064689in}{4.390494in}}%
\pgfpathlineto{\pgfqpoint{4.070058in}{4.392684in}}%
\pgfpathlineto{\pgfqpoint{4.072206in}{4.395518in}}%
\pgfpathlineto{\pgfqpoint{4.073280in}{4.391997in}}%
\pgfpathlineto{\pgfqpoint{4.077575in}{4.388863in}}%
\pgfpathlineto{\pgfqpoint{4.078649in}{4.392512in}}%
\pgfpathlineto{\pgfqpoint{4.084018in}{4.388476in}}%
\pgfpathlineto{\pgfqpoint{4.086166in}{4.394230in}}%
\pgfpathlineto{\pgfqpoint{4.087240in}{4.396334in}}%
\pgfpathlineto{\pgfqpoint{4.088314in}{4.400542in}}%
\pgfpathlineto{\pgfqpoint{4.091535in}{4.398824in}}%
\pgfpathlineto{\pgfqpoint{4.092609in}{4.400069in}}%
\pgfpathlineto{\pgfqpoint{4.093683in}{4.398180in}}%
\pgfpathlineto{\pgfqpoint{4.094757in}{4.400756in}}%
\pgfpathlineto{\pgfqpoint{4.095831in}{4.398824in}}%
\pgfpathlineto{\pgfqpoint{4.099052in}{4.399812in}}%
\pgfpathlineto{\pgfqpoint{4.100126in}{4.399425in}}%
\pgfpathlineto{\pgfqpoint{4.101200in}{4.397708in}}%
\pgfpathlineto{\pgfqpoint{4.102274in}{4.402903in}}%
\pgfpathlineto{\pgfqpoint{4.109790in}{4.402130in}}%
\pgfpathlineto{\pgfqpoint{4.110864in}{4.404707in}}%
\pgfpathlineto{\pgfqpoint{4.114086in}{4.407970in}}%
\pgfpathlineto{\pgfqpoint{4.115160in}{4.406338in}}%
\pgfpathlineto{\pgfqpoint{4.116233in}{4.402989in}}%
\pgfpathlineto{\pgfqpoint{4.117307in}{4.405608in}}%
\pgfpathlineto{\pgfqpoint{4.118381in}{4.403376in}}%
\pgfpathlineto{\pgfqpoint{4.121603in}{4.403290in}}%
\pgfpathlineto{\pgfqpoint{4.122676in}{4.398567in}}%
\pgfpathlineto{\pgfqpoint{4.125898in}{4.404191in}}%
\pgfpathlineto{\pgfqpoint{4.129120in}{4.402517in}}%
\pgfpathlineto{\pgfqpoint{4.132341in}{4.417760in}}%
\pgfpathlineto{\pgfqpoint{4.133415in}{4.417115in}}%
\pgfpathlineto{\pgfqpoint{4.136636in}{4.419863in}}%
\pgfpathlineto{\pgfqpoint{4.137710in}{4.418489in}}%
\pgfpathlineto{\pgfqpoint{4.140932in}{4.419778in}}%
\pgfpathlineto{\pgfqpoint{4.144153in}{4.425231in}}%
\pgfpathlineto{\pgfqpoint{4.145227in}{4.423556in}}%
\pgfpathlineto{\pgfqpoint{4.146301in}{4.426776in}}%
\pgfpathlineto{\pgfqpoint{4.147375in}{4.422268in}}%
\pgfpathlineto{\pgfqpoint{4.148449in}{4.422139in}}%
\pgfpathlineto{\pgfqpoint{4.151670in}{4.422740in}}%
\pgfpathlineto{\pgfqpoint{4.152744in}{4.424973in}}%
\pgfpathlineto{\pgfqpoint{4.153818in}{4.421023in}}%
\pgfpathlineto{\pgfqpoint{4.154892in}{4.423857in}}%
\pgfpathlineto{\pgfqpoint{4.159187in}{4.420207in}}%
\pgfpathlineto{\pgfqpoint{4.160261in}{4.416428in}}%
\pgfpathlineto{\pgfqpoint{4.162409in}{4.423599in}}%
\pgfpathlineto{\pgfqpoint{4.163482in}{4.420679in}}%
\pgfpathlineto{\pgfqpoint{4.166704in}{4.422783in}}%
\pgfpathlineto{\pgfqpoint{4.167778in}{4.424157in}}%
\pgfpathlineto{\pgfqpoint{4.168852in}{4.423599in}}%
\pgfpathlineto{\pgfqpoint{4.169925in}{4.421796in}}%
\pgfpathlineto{\pgfqpoint{4.170999in}{4.421323in}}%
\pgfpathlineto{\pgfqpoint{4.174221in}{4.420851in}}%
\pgfpathlineto{\pgfqpoint{4.175295in}{4.422010in}}%
\pgfpathlineto{\pgfqpoint{4.177442in}{4.429610in}}%
\pgfpathlineto{\pgfqpoint{4.178516in}{4.431843in}}%
\pgfpathlineto{\pgfqpoint{4.181738in}{4.432015in}}%
\pgfpathlineto{\pgfqpoint{4.182811in}{4.434934in}}%
\pgfpathlineto{\pgfqpoint{4.183885in}{4.435278in}}%
\pgfpathlineto{\pgfqpoint{4.184959in}{4.434505in}}%
\pgfpathlineto{\pgfqpoint{4.186033in}{4.431585in}}%
\pgfpathlineto{\pgfqpoint{4.189254in}{4.431585in}}%
\pgfpathlineto{\pgfqpoint{4.190328in}{4.430770in}}%
\pgfpathlineto{\pgfqpoint{4.191402in}{4.429095in}}%
\pgfpathlineto{\pgfqpoint{4.192476in}{4.428966in}}%
\pgfpathlineto{\pgfqpoint{4.193550in}{4.429825in}}%
\pgfpathlineto{\pgfqpoint{4.198919in}{4.421109in}}%
\pgfpathlineto{\pgfqpoint{4.199993in}{4.414367in}}%
\pgfpathlineto{\pgfqpoint{4.201067in}{4.413423in}}%
\pgfpathlineto{\pgfqpoint{4.204288in}{4.416987in}}%
\pgfpathlineto{\pgfqpoint{4.205362in}{4.417073in}}%
\pgfpathlineto{\pgfqpoint{4.206436in}{4.415613in}}%
\pgfpathlineto{\pgfqpoint{4.207510in}{4.417030in}}%
\pgfpathlineto{\pgfqpoint{4.208584in}{4.414969in}}%
\pgfpathlineto{\pgfqpoint{4.212879in}{4.417888in}}%
\pgfpathlineto{\pgfqpoint{4.213953in}{4.413638in}}%
\pgfpathlineto{\pgfqpoint{4.215027in}{4.415054in}}%
\pgfpathlineto{\pgfqpoint{4.216100in}{4.405780in}}%
\pgfpathlineto{\pgfqpoint{4.219322in}{4.406081in}}%
\pgfpathlineto{\pgfqpoint{4.220396in}{4.407927in}}%
\pgfpathlineto{\pgfqpoint{4.221470in}{4.413337in}}%
\pgfpathlineto{\pgfqpoint{4.222543in}{4.412779in}}%
\pgfpathlineto{\pgfqpoint{4.223617in}{4.415183in}}%
\pgfpathlineto{\pgfqpoint{4.226839in}{4.412264in}}%
\pgfpathlineto{\pgfqpoint{4.227913in}{4.417588in}}%
\pgfpathlineto{\pgfqpoint{4.228986in}{4.412264in}}%
\pgfpathlineto{\pgfqpoint{4.230060in}{4.412049in}}%
\pgfpathlineto{\pgfqpoint{4.231134in}{4.416815in}}%
\pgfpathlineto{\pgfqpoint{4.234356in}{4.415140in}}%
\pgfpathlineto{\pgfqpoint{4.235430in}{4.419348in}}%
\pgfpathlineto{\pgfqpoint{4.236503in}{4.421280in}}%
\pgfpathlineto{\pgfqpoint{4.237577in}{4.416944in}}%
\pgfpathlineto{\pgfqpoint{4.238651in}{4.418790in}}%
\pgfpathlineto{\pgfqpoint{4.241873in}{4.415913in}}%
\pgfpathlineto{\pgfqpoint{4.242946in}{4.416171in}}%
\pgfpathlineto{\pgfqpoint{4.244020in}{4.419134in}}%
\pgfpathlineto{\pgfqpoint{4.245094in}{4.418404in}}%
\pgfpathlineto{\pgfqpoint{4.246168in}{4.424286in}}%
\pgfpathlineto{\pgfqpoint{4.249389in}{4.427506in}}%
\pgfpathlineto{\pgfqpoint{4.250463in}{4.429696in}}%
\pgfpathlineto{\pgfqpoint{4.252611in}{4.429095in}}%
\pgfpathlineto{\pgfqpoint{4.253685in}{4.426648in}}%
\pgfpathlineto{\pgfqpoint{4.257980in}{4.425746in}}%
\pgfpathlineto{\pgfqpoint{4.259054in}{4.424544in}}%
\pgfpathlineto{\pgfqpoint{4.260128in}{4.420078in}}%
\pgfpathlineto{\pgfqpoint{4.261202in}{4.423942in}}%
\pgfpathlineto{\pgfqpoint{4.264423in}{4.426690in}}%
\pgfpathlineto{\pgfqpoint{4.265497in}{4.426862in}}%
\pgfpathlineto{\pgfqpoint{4.266571in}{4.425359in}}%
\pgfpathlineto{\pgfqpoint{4.267645in}{4.417373in}}%
\pgfpathlineto{\pgfqpoint{4.268719in}{4.415741in}}%
\pgfpathlineto{\pgfqpoint{4.273014in}{4.415012in}}%
\pgfpathlineto{\pgfqpoint{4.274088in}{4.416643in}}%
\pgfpathlineto{\pgfqpoint{4.275162in}{4.423298in}}%
\pgfpathlineto{\pgfqpoint{4.276235in}{4.426261in}}%
\pgfpathlineto{\pgfqpoint{4.279457in}{4.425359in}}%
\pgfpathlineto{\pgfqpoint{4.282678in}{4.418361in}}%
\pgfpathlineto{\pgfqpoint{4.283752in}{4.422697in}}%
\pgfpathlineto{\pgfqpoint{4.286974in}{4.420722in}}%
\pgfpathlineto{\pgfqpoint{4.289121in}{4.423857in}}%
\pgfpathlineto{\pgfqpoint{4.290195in}{4.418275in}}%
\pgfpathlineto{\pgfqpoint{4.291269in}{4.415870in}}%
\pgfpathlineto{\pgfqpoint{4.295564in}{4.416772in}}%
\pgfpathlineto{\pgfqpoint{4.298786in}{4.422611in}}%
\pgfpathlineto{\pgfqpoint{4.303081in}{4.418790in}}%
\pgfpathlineto{\pgfqpoint{4.304155in}{4.419649in}}%
\pgfpathlineto{\pgfqpoint{4.305229in}{4.418103in}}%
\pgfpathlineto{\pgfqpoint{4.306303in}{4.422225in}}%
\pgfpathlineto{\pgfqpoint{4.309524in}{4.421967in}}%
\pgfpathlineto{\pgfqpoint{4.310598in}{4.422526in}}%
\pgfpathlineto{\pgfqpoint{4.312746in}{4.421023in}}%
\pgfpathlineto{\pgfqpoint{4.313820in}{4.425402in}}%
\pgfpathlineto{\pgfqpoint{4.318115in}{4.426605in}}%
\pgfpathlineto{\pgfqpoint{4.319189in}{4.421066in}}%
\pgfpathlineto{\pgfqpoint{4.321337in}{4.423170in}}%
\pgfpathlineto{\pgfqpoint{4.324558in}{4.422354in}}%
\pgfpathlineto{\pgfqpoint{4.325632in}{4.421237in}}%
\pgfpathlineto{\pgfqpoint{4.326706in}{4.421409in}}%
\pgfpathlineto{\pgfqpoint{4.327780in}{4.428279in}}%
\pgfpathlineto{\pgfqpoint{4.328853in}{4.429138in}}%
\pgfpathlineto{\pgfqpoint{4.332075in}{4.428794in}}%
\pgfpathlineto{\pgfqpoint{4.333149in}{4.426819in}}%
\pgfpathlineto{\pgfqpoint{4.334223in}{4.426819in}}%
\pgfpathlineto{\pgfqpoint{4.336370in}{4.423685in}}%
\pgfpathlineto{\pgfqpoint{4.339592in}{4.422955in}}%
\pgfpathlineto{\pgfqpoint{4.340666in}{4.420851in}}%
\pgfpathlineto{\pgfqpoint{4.341740in}{4.416987in}}%
\pgfpathlineto{\pgfqpoint{4.343887in}{4.419778in}}%
\pgfpathlineto{\pgfqpoint{4.347109in}{4.423170in}}%
\pgfpathlineto{\pgfqpoint{4.348183in}{4.421023in}}%
\pgfpathlineto{\pgfqpoint{4.349256in}{4.422697in}}%
\pgfpathlineto{\pgfqpoint{4.350330in}{4.426433in}}%
\pgfpathlineto{\pgfqpoint{4.351404in}{4.427206in}}%
\pgfpathlineto{\pgfqpoint{4.354626in}{4.428193in}}%
\pgfpathlineto{\pgfqpoint{4.356773in}{4.424887in}}%
\pgfpathlineto{\pgfqpoint{4.357847in}{4.426862in}}%
\pgfpathlineto{\pgfqpoint{4.358921in}{4.424587in}}%
\pgfpathlineto{\pgfqpoint{4.362142in}{4.423427in}}%
\pgfpathlineto{\pgfqpoint{4.364290in}{4.430340in}}%
\pgfpathlineto{\pgfqpoint{4.365364in}{4.423513in}}%
\pgfpathlineto{\pgfqpoint{4.366438in}{4.420593in}}%
\pgfpathlineto{\pgfqpoint{4.369659in}{4.420121in}}%
\pgfpathlineto{\pgfqpoint{4.370733in}{4.414797in}}%
\pgfpathlineto{\pgfqpoint{4.371807in}{4.414239in}}%
\pgfpathlineto{\pgfqpoint{4.373955in}{4.416257in}}%
\pgfpathlineto{\pgfqpoint{4.379324in}{4.417115in}}%
\pgfpathlineto{\pgfqpoint{4.380398in}{4.420550in}}%
\pgfpathlineto{\pgfqpoint{4.384693in}{4.418618in}}%
\pgfpathlineto{\pgfqpoint{4.385767in}{4.421839in}}%
\pgfpathlineto{\pgfqpoint{4.386841in}{4.415570in}}%
\pgfpathlineto{\pgfqpoint{4.387915in}{4.415527in}}%
\pgfpathlineto{\pgfqpoint{4.388988in}{4.416557in}}%
\pgfpathlineto{\pgfqpoint{4.392210in}{4.415355in}}%
\pgfpathlineto{\pgfqpoint{4.394358in}{4.408056in}}%
\pgfpathlineto{\pgfqpoint{4.395431in}{4.408013in}}%
\pgfpathlineto{\pgfqpoint{4.396505in}{4.410589in}}%
\pgfpathlineto{\pgfqpoint{4.399727in}{4.414239in}}%
\pgfpathlineto{\pgfqpoint{4.400801in}{4.416686in}}%
\pgfpathlineto{\pgfqpoint{4.401875in}{4.417158in}}%
\pgfpathlineto{\pgfqpoint{4.404022in}{4.419219in}}%
\pgfpathlineto{\pgfqpoint{4.407244in}{4.415999in}}%
\pgfpathlineto{\pgfqpoint{4.408318in}{4.411147in}}%
\pgfpathlineto{\pgfqpoint{4.409391in}{4.415312in}}%
\pgfpathlineto{\pgfqpoint{4.410465in}{4.416901in}}%
\pgfpathlineto{\pgfqpoint{4.411539in}{4.416815in}}%
\pgfpathlineto{\pgfqpoint{4.414761in}{4.417201in}}%
\pgfpathlineto{\pgfqpoint{4.415834in}{4.416600in}}%
\pgfpathlineto{\pgfqpoint{4.416908in}{4.418790in}}%
\pgfpathlineto{\pgfqpoint{4.417982in}{4.417674in}}%
\pgfpathlineto{\pgfqpoint{4.419056in}{4.419606in}}%
\pgfpathlineto{\pgfqpoint{4.422277in}{4.420164in}}%
\pgfpathlineto{\pgfqpoint{4.423351in}{4.421366in}}%
\pgfpathlineto{\pgfqpoint{4.424425in}{4.423728in}}%
\pgfpathlineto{\pgfqpoint{4.425499in}{4.423942in}}%
\pgfpathlineto{\pgfqpoint{4.426573in}{4.419735in}}%
\pgfpathlineto{\pgfqpoint{4.429794in}{4.422182in}}%
\pgfpathlineto{\pgfqpoint{4.430868in}{4.424329in}}%
\pgfpathlineto{\pgfqpoint{4.431942in}{4.420293in}}%
\pgfpathlineto{\pgfqpoint{4.433016in}{4.422697in}}%
\pgfpathlineto{\pgfqpoint{4.434090in}{4.423642in}}%
\pgfpathlineto{\pgfqpoint{4.437311in}{4.423170in}}%
\pgfpathlineto{\pgfqpoint{4.439459in}{4.421624in}}%
\pgfpathlineto{\pgfqpoint{4.440533in}{4.419906in}}%
\pgfpathlineto{\pgfqpoint{4.441607in}{4.419821in}}%
\pgfpathlineto{\pgfqpoint{4.444828in}{4.422998in}}%
\pgfpathlineto{\pgfqpoint{4.446976in}{4.428751in}}%
\pgfpathlineto{\pgfqpoint{4.448050in}{4.429696in}}%
\pgfpathlineto{\pgfqpoint{4.449123in}{4.432101in}}%
\pgfpathlineto{\pgfqpoint{4.452345in}{4.433088in}}%
\pgfpathlineto{\pgfqpoint{4.453419in}{4.434419in}}%
\pgfpathlineto{\pgfqpoint{4.454493in}{4.434634in}}%
\pgfpathlineto{\pgfqpoint{4.456640in}{4.438455in}}%
\pgfpathlineto{\pgfqpoint{4.459862in}{4.438326in}}%
\pgfpathlineto{\pgfqpoint{4.460936in}{4.436609in}}%
\pgfpathlineto{\pgfqpoint{4.462009in}{4.436051in}}%
\pgfpathlineto{\pgfqpoint{4.463083in}{4.436523in}}%
\pgfpathlineto{\pgfqpoint{4.464157in}{4.432745in}}%
\pgfpathlineto{\pgfqpoint{4.468452in}{4.431285in}}%
\pgfpathlineto{\pgfqpoint{4.469526in}{4.429438in}}%
\pgfpathlineto{\pgfqpoint{4.471674in}{4.431671in}}%
\pgfpathlineto{\pgfqpoint{4.474896in}{4.429997in}}%
\pgfpathlineto{\pgfqpoint{4.475969in}{4.431456in}}%
\pgfpathlineto{\pgfqpoint{4.477043in}{4.430984in}}%
\pgfpathlineto{\pgfqpoint{4.478117in}{4.428966in}}%
\pgfpathlineto{\pgfqpoint{4.479191in}{4.433260in}}%
\pgfpathlineto{\pgfqpoint{4.482412in}{4.431328in}}%
\pgfpathlineto{\pgfqpoint{4.483486in}{4.436180in}}%
\pgfpathlineto{\pgfqpoint{4.484560in}{4.435793in}}%
\pgfpathlineto{\pgfqpoint{4.485634in}{4.441976in}}%
\pgfpathlineto{\pgfqpoint{4.486708in}{4.440645in}}%
\pgfpathlineto{\pgfqpoint{4.489929in}{4.441375in}}%
\pgfpathlineto{\pgfqpoint{4.491003in}{4.442234in}}%
\pgfpathlineto{\pgfqpoint{4.492077in}{4.441804in}}%
\pgfpathlineto{\pgfqpoint{4.493151in}{4.442749in}}%
\pgfpathlineto{\pgfqpoint{4.494225in}{4.440216in}}%
\pgfpathlineto{\pgfqpoint{4.498520in}{4.442534in}}%
\pgfpathlineto{\pgfqpoint{4.500668in}{4.439658in}}%
\pgfpathlineto{\pgfqpoint{4.501741in}{4.443651in}}%
\pgfpathlineto{\pgfqpoint{4.506037in}{4.441117in}}%
\pgfpathlineto{\pgfqpoint{4.507111in}{4.443393in}}%
\pgfpathlineto{\pgfqpoint{4.508185in}{4.442019in}}%
\pgfpathlineto{\pgfqpoint{4.512480in}{4.444896in}}%
\pgfpathlineto{\pgfqpoint{4.513554in}{4.449705in}}%
\pgfpathlineto{\pgfqpoint{4.514628in}{4.451809in}}%
\pgfpathlineto{\pgfqpoint{4.515701in}{4.451422in}}%
\pgfpathlineto{\pgfqpoint{4.516775in}{4.451980in}}%
\pgfpathlineto{\pgfqpoint{4.519997in}{4.455115in}}%
\pgfpathlineto{\pgfqpoint{4.521071in}{4.454299in}}%
\pgfpathlineto{\pgfqpoint{4.523218in}{4.454643in}}%
\pgfpathlineto{\pgfqpoint{4.524292in}{4.457434in}}%
\pgfpathlineto{\pgfqpoint{4.527514in}{4.456188in}}%
\pgfpathlineto{\pgfqpoint{4.528587in}{4.453655in}}%
\pgfpathlineto{\pgfqpoint{4.529661in}{4.457047in}}%
\pgfpathlineto{\pgfqpoint{4.530735in}{4.454771in}}%
\pgfpathlineto{\pgfqpoint{4.531809in}{4.457906in}}%
\pgfpathlineto{\pgfqpoint{4.535030in}{4.457476in}}%
\pgfpathlineto{\pgfqpoint{4.536104in}{4.461641in}}%
\pgfpathlineto{\pgfqpoint{4.537178in}{4.461555in}}%
\pgfpathlineto{\pgfqpoint{4.538252in}{4.462758in}}%
\pgfpathlineto{\pgfqpoint{4.542547in}{4.461985in}}%
\pgfpathlineto{\pgfqpoint{4.543621in}{4.463488in}}%
\pgfpathlineto{\pgfqpoint{4.544695in}{4.458807in}}%
\pgfpathlineto{\pgfqpoint{4.545769in}{4.460697in}}%
\pgfpathlineto{\pgfqpoint{4.546843in}{4.455072in}}%
\pgfpathlineto{\pgfqpoint{4.550064in}{4.456274in}}%
\pgfpathlineto{\pgfqpoint{4.551138in}{4.454771in}}%
\pgfpathlineto{\pgfqpoint{4.554360in}{4.456575in}}%
\pgfpathlineto{\pgfqpoint{4.557581in}{4.450392in}}%
\pgfpathlineto{\pgfqpoint{4.558655in}{4.452324in}}%
\pgfpathlineto{\pgfqpoint{4.559729in}{4.450521in}}%
\pgfpathlineto{\pgfqpoint{4.560803in}{4.454084in}}%
\pgfpathlineto{\pgfqpoint{4.561876in}{4.462672in}}%
\pgfpathlineto{\pgfqpoint{4.565098in}{4.460439in}}%
\pgfpathlineto{\pgfqpoint{4.566172in}{4.463531in}}%
\pgfpathlineto{\pgfqpoint{4.567246in}{4.463402in}}%
\pgfpathlineto{\pgfqpoint{4.568319in}{4.466193in}}%
\pgfpathlineto{\pgfqpoint{4.569393in}{4.464690in}}%
\pgfpathlineto{\pgfqpoint{4.572615in}{4.464218in}}%
\pgfpathlineto{\pgfqpoint{4.573689in}{4.467266in}}%
\pgfpathlineto{\pgfqpoint{4.574763in}{4.466751in}}%
\pgfpathlineto{\pgfqpoint{4.576910in}{4.474394in}}%
\pgfpathlineto{\pgfqpoint{4.580132in}{4.473750in}}%
\pgfpathlineto{\pgfqpoint{4.582279in}{4.474737in}}%
\pgfpathlineto{\pgfqpoint{4.587649in}{4.472676in}}%
\pgfpathlineto{\pgfqpoint{4.589796in}{4.484398in}}%
\pgfpathlineto{\pgfqpoint{4.590870in}{4.482251in}}%
\pgfpathlineto{\pgfqpoint{4.591944in}{4.486545in}}%
\pgfpathlineto{\pgfqpoint{4.595165in}{4.490753in}}%
\pgfpathlineto{\pgfqpoint{4.596239in}{4.493544in}}%
\pgfpathlineto{\pgfqpoint{4.597313in}{4.490925in}}%
\pgfpathlineto{\pgfqpoint{4.598387in}{4.491912in}}%
\pgfpathlineto{\pgfqpoint{4.599461in}{4.494188in}}%
\pgfpathlineto{\pgfqpoint{4.603756in}{4.497666in}}%
\pgfpathlineto{\pgfqpoint{4.604830in}{4.496378in}}%
\pgfpathlineto{\pgfqpoint{4.605904in}{4.497580in}}%
\pgfpathlineto{\pgfqpoint{4.606978in}{4.495905in}}%
\pgfpathlineto{\pgfqpoint{4.610199in}{4.498868in}}%
\pgfpathlineto{\pgfqpoint{4.611273in}{4.497279in}}%
\pgfpathlineto{\pgfqpoint{4.612347in}{4.492170in}}%
\pgfpathlineto{\pgfqpoint{4.614495in}{4.505008in}}%
\pgfpathlineto{\pgfqpoint{4.617716in}{4.505953in}}%
\pgfpathlineto{\pgfqpoint{4.619864in}{4.491998in}}%
\pgfpathlineto{\pgfqpoint{4.620938in}{4.493930in}}%
\pgfpathlineto{\pgfqpoint{4.622011in}{4.485171in}}%
\pgfpathlineto{\pgfqpoint{4.625233in}{4.488563in}}%
\pgfpathlineto{\pgfqpoint{4.626307in}{4.493028in}}%
\pgfpathlineto{\pgfqpoint{4.627381in}{4.490109in}}%
\pgfpathlineto{\pgfqpoint{4.628454in}{4.484956in}}%
\pgfpathlineto{\pgfqpoint{4.629528in}{4.486502in}}%
\pgfpathlineto{\pgfqpoint{4.632750in}{4.481392in}}%
\pgfpathlineto{\pgfqpoint{4.635971in}{4.493587in}}%
\pgfpathlineto{\pgfqpoint{4.637045in}{4.492170in}}%
\pgfpathlineto{\pgfqpoint{4.640267in}{4.495605in}}%
\pgfpathlineto{\pgfqpoint{4.641340in}{4.492513in}}%
\pgfpathlineto{\pgfqpoint{4.642414in}{4.492341in}}%
\pgfpathlineto{\pgfqpoint{4.644562in}{4.499168in}}%
\pgfpathlineto{\pgfqpoint{4.647784in}{4.502045in}}%
\pgfpathlineto{\pgfqpoint{4.648857in}{4.504278in}}%
\pgfpathlineto{\pgfqpoint{4.649931in}{4.499254in}}%
\pgfpathlineto{\pgfqpoint{4.651005in}{4.501616in}}%
\pgfpathlineto{\pgfqpoint{4.652079in}{4.507112in}}%
\pgfpathlineto{\pgfqpoint{4.655300in}{4.505223in}}%
\pgfpathlineto{\pgfqpoint{4.656374in}{4.506854in}}%
\pgfpathlineto{\pgfqpoint{4.657448in}{4.501101in}}%
\pgfpathlineto{\pgfqpoint{4.658522in}{4.490109in}}%
\pgfpathlineto{\pgfqpoint{4.659596in}{4.490323in}}%
\pgfpathlineto{\pgfqpoint{4.662817in}{4.492900in}}%
\pgfpathlineto{\pgfqpoint{4.663891in}{4.491697in}}%
\pgfpathlineto{\pgfqpoint{4.664965in}{4.495433in}}%
\pgfpathlineto{\pgfqpoint{4.666039in}{4.497022in}}%
\pgfpathlineto{\pgfqpoint{4.667113in}{4.495347in}}%
\pgfpathlineto{\pgfqpoint{4.670334in}{4.494231in}}%
\pgfpathlineto{\pgfqpoint{4.671408in}{4.494746in}}%
\pgfpathlineto{\pgfqpoint{4.672482in}{4.489164in}}%
\pgfpathlineto{\pgfqpoint{4.673556in}{4.496463in}}%
\pgfpathlineto{\pgfqpoint{4.678925in}{4.497193in}}%
\pgfpathlineto{\pgfqpoint{4.679999in}{4.494574in}}%
\pgfpathlineto{\pgfqpoint{4.681073in}{4.498696in}}%
\pgfpathlineto{\pgfqpoint{4.682146in}{4.495991in}}%
\pgfpathlineto{\pgfqpoint{4.685368in}{4.495948in}}%
\pgfpathlineto{\pgfqpoint{4.686442in}{4.498825in}}%
\pgfpathlineto{\pgfqpoint{4.687516in}{4.497451in}}%
\pgfpathlineto{\pgfqpoint{4.688589in}{4.493544in}}%
\pgfpathlineto{\pgfqpoint{4.689663in}{4.494660in}}%
\pgfpathlineto{\pgfqpoint{4.692885in}{4.491397in}}%
\pgfpathlineto{\pgfqpoint{4.693959in}{4.491139in}}%
\pgfpathlineto{\pgfqpoint{4.695032in}{4.487876in}}%
\pgfpathlineto{\pgfqpoint{4.696106in}{4.489508in}}%
\pgfpathlineto{\pgfqpoint{4.697180in}{4.488692in}}%
\pgfpathlineto{\pgfqpoint{4.700402in}{4.488563in}}%
\pgfpathlineto{\pgfqpoint{4.701475in}{4.481865in}}%
\pgfpathlineto{\pgfqpoint{4.703623in}{4.483067in}}%
\pgfpathlineto{\pgfqpoint{4.704697in}{4.481865in}}%
\pgfpathlineto{\pgfqpoint{4.708992in}{4.483883in}}%
\pgfpathlineto{\pgfqpoint{4.711140in}{4.489293in}}%
\pgfpathlineto{\pgfqpoint{4.712214in}{4.487532in}}%
\pgfpathlineto{\pgfqpoint{4.715435in}{4.488735in}}%
\pgfpathlineto{\pgfqpoint{4.717583in}{4.494703in}}%
\pgfpathlineto{\pgfqpoint{4.722952in}{4.496378in}}%
\pgfpathlineto{\pgfqpoint{4.725100in}{4.504407in}}%
\pgfpathlineto{\pgfqpoint{4.726174in}{4.504106in}}%
\pgfpathlineto{\pgfqpoint{4.730469in}{4.499684in}}%
\pgfpathlineto{\pgfqpoint{4.731543in}{4.498224in}}%
\pgfpathlineto{\pgfqpoint{4.732617in}{4.497880in}}%
\pgfpathlineto{\pgfqpoint{4.733691in}{4.499211in}}%
\pgfpathlineto{\pgfqpoint{4.734764in}{4.497494in}}%
\pgfpathlineto{\pgfqpoint{4.737986in}{4.496034in}}%
\pgfpathlineto{\pgfqpoint{4.739060in}{4.497666in}}%
\pgfpathlineto{\pgfqpoint{4.741207in}{4.491053in}}%
\pgfpathlineto{\pgfqpoint{4.742281in}{4.492170in}}%
\pgfpathlineto{\pgfqpoint{4.747651in}{4.483926in}}%
\pgfpathlineto{\pgfqpoint{4.748724in}{4.493544in}}%
\pgfpathlineto{\pgfqpoint{4.749798in}{4.496463in}}%
\pgfpathlineto{\pgfqpoint{4.753020in}{4.499126in}}%
\pgfpathlineto{\pgfqpoint{4.754094in}{4.495905in}}%
\pgfpathlineto{\pgfqpoint{4.755167in}{4.500113in}}%
\pgfpathlineto{\pgfqpoint{4.756241in}{4.515528in}}%
\pgfpathlineto{\pgfqpoint{4.757315in}{4.516687in}}%
\pgfpathlineto{\pgfqpoint{4.760537in}{4.516215in}}%
\pgfpathlineto{\pgfqpoint{4.761610in}{4.518061in}}%
\pgfpathlineto{\pgfqpoint{4.762684in}{4.517116in}}%
\pgfpathlineto{\pgfqpoint{4.763758in}{4.518190in}}%
\pgfpathlineto{\pgfqpoint{4.764832in}{4.524931in}}%
\pgfpathlineto{\pgfqpoint{4.768053in}{4.525489in}}%
\pgfpathlineto{\pgfqpoint{4.769127in}{4.529053in}}%
\pgfpathlineto{\pgfqpoint{4.770201in}{4.526863in}}%
\pgfpathlineto{\pgfqpoint{4.771275in}{4.521710in}}%
\pgfpathlineto{\pgfqpoint{4.772349in}{4.523127in}}%
\pgfpathlineto{\pgfqpoint{4.776644in}{4.522097in}}%
\pgfpathlineto{\pgfqpoint{4.777718in}{4.522999in}}%
\pgfpathlineto{\pgfqpoint{4.778792in}{4.518233in}}%
\pgfpathlineto{\pgfqpoint{4.779866in}{4.521625in}}%
\pgfpathlineto{\pgfqpoint{4.783087in}{4.520165in}}%
\pgfpathlineto{\pgfqpoint{4.784161in}{4.518705in}}%
\pgfpathlineto{\pgfqpoint{4.786309in}{4.522097in}}%
\pgfpathlineto{\pgfqpoint{4.787383in}{4.525661in}}%
\pgfpathlineto{\pgfqpoint{4.790604in}{4.523857in}}%
\pgfpathlineto{\pgfqpoint{4.791678in}{4.524330in}}%
\pgfpathlineto{\pgfqpoint{4.792752in}{4.523514in}}%
\pgfpathlineto{\pgfqpoint{4.793826in}{4.530212in}}%
\pgfpathlineto{\pgfqpoint{4.794899in}{4.529826in}}%
\pgfpathlineto{\pgfqpoint{4.798121in}{4.532745in}}%
\pgfpathlineto{\pgfqpoint{4.800269in}{4.536653in}}%
\pgfpathlineto{\pgfqpoint{4.802416in}{4.537511in}}%
\pgfpathlineto{\pgfqpoint{4.805638in}{4.535493in}}%
\pgfpathlineto{\pgfqpoint{4.806712in}{4.532788in}}%
\pgfpathlineto{\pgfqpoint{4.807785in}{4.532359in}}%
\pgfpathlineto{\pgfqpoint{4.808859in}{4.532531in}}%
\pgfpathlineto{\pgfqpoint{4.809933in}{4.538027in}}%
\pgfpathlineto{\pgfqpoint{4.813155in}{4.537297in}}%
\pgfpathlineto{\pgfqpoint{4.814228in}{4.535493in}}%
\pgfpathlineto{\pgfqpoint{4.815302in}{4.529997in}}%
\pgfpathlineto{\pgfqpoint{4.816376in}{4.527636in}}%
\pgfpathlineto{\pgfqpoint{4.817450in}{4.529139in}}%
\pgfpathlineto{\pgfqpoint{4.820672in}{4.532445in}}%
\pgfpathlineto{\pgfqpoint{4.821745in}{4.530813in}}%
\pgfpathlineto{\pgfqpoint{4.822819in}{4.538241in}}%
\pgfpathlineto{\pgfqpoint{4.823893in}{4.539916in}}%
\pgfpathlineto{\pgfqpoint{4.824967in}{4.544467in}}%
\pgfpathlineto{\pgfqpoint{4.832484in}{4.550865in}}%
\pgfpathlineto{\pgfqpoint{4.835705in}{4.552454in}}%
\pgfpathlineto{\pgfqpoint{4.836779in}{4.556919in}}%
\pgfpathlineto{\pgfqpoint{4.838927in}{4.551895in}}%
\pgfpathlineto{\pgfqpoint{4.840001in}{4.553055in}}%
\pgfpathlineto{\pgfqpoint{4.843222in}{4.552883in}}%
\pgfpathlineto{\pgfqpoint{4.844296in}{4.551552in}}%
\pgfpathlineto{\pgfqpoint{4.845370in}{4.552625in}}%
\pgfpathlineto{\pgfqpoint{4.850739in}{4.543093in}}%
\pgfpathlineto{\pgfqpoint{4.851813in}{4.543608in}}%
\pgfpathlineto{\pgfqpoint{4.852887in}{4.548074in}}%
\pgfpathlineto{\pgfqpoint{4.853961in}{4.546185in}}%
\pgfpathlineto{\pgfqpoint{4.855034in}{4.558808in}}%
\pgfpathlineto{\pgfqpoint{4.859330in}{4.557305in}}%
\pgfpathlineto{\pgfqpoint{4.860404in}{4.559452in}}%
\pgfpathlineto{\pgfqpoint{4.862551in}{4.542921in}}%
\pgfpathlineto{\pgfqpoint{4.865773in}{4.538070in}}%
\pgfpathlineto{\pgfqpoint{4.866847in}{4.541676in}}%
\pgfpathlineto{\pgfqpoint{4.867920in}{4.537383in}}%
\pgfpathlineto{\pgfqpoint{4.868994in}{4.541590in}}%
\pgfpathlineto{\pgfqpoint{4.870068in}{4.535279in}}%
\pgfpathlineto{\pgfqpoint{4.873290in}{4.526734in}}%
\pgfpathlineto{\pgfqpoint{4.874363in}{4.531243in}}%
\pgfpathlineto{\pgfqpoint{4.875437in}{4.530126in}}%
\pgfpathlineto{\pgfqpoint{4.877585in}{4.543093in}}%
\pgfpathlineto{\pgfqpoint{4.882954in}{4.551080in}}%
\pgfpathlineto{\pgfqpoint{4.884028in}{4.550607in}}%
\pgfpathlineto{\pgfqpoint{4.885102in}{4.551037in}}%
\pgfpathlineto{\pgfqpoint{4.889397in}{4.551080in}}%
\pgfpathlineto{\pgfqpoint{4.890471in}{4.550393in}}%
\pgfpathlineto{\pgfqpoint{4.891545in}{4.551080in}}%
\pgfpathlineto{\pgfqpoint{4.892619in}{4.550049in}}%
\pgfpathlineto{\pgfqpoint{4.895840in}{4.554643in}}%
\pgfpathlineto{\pgfqpoint{4.896914in}{4.554600in}}%
\pgfpathlineto{\pgfqpoint{4.897988in}{4.553913in}}%
\pgfpathlineto{\pgfqpoint{4.899062in}{4.556103in}}%
\pgfpathlineto{\pgfqpoint{4.900136in}{4.560053in}}%
\pgfpathlineto{\pgfqpoint{4.903357in}{4.554858in}}%
\pgfpathlineto{\pgfqpoint{4.904431in}{4.565292in}}%
\pgfpathlineto{\pgfqpoint{4.905505in}{4.563445in}}%
\pgfpathlineto{\pgfqpoint{4.906579in}{4.568898in}}%
\pgfpathlineto{\pgfqpoint{4.907652in}{4.570230in}}%
\pgfpathlineto{\pgfqpoint{4.910874in}{4.569585in}}%
\pgfpathlineto{\pgfqpoint{4.913022in}{4.566451in}}%
\pgfpathlineto{\pgfqpoint{4.914095in}{4.557692in}}%
\pgfpathlineto{\pgfqpoint{4.915169in}{4.555631in}}%
\pgfpathlineto{\pgfqpoint{4.919465in}{4.561256in}}%
\pgfpathlineto{\pgfqpoint{4.920539in}{4.557864in}}%
\pgfpathlineto{\pgfqpoint{4.921612in}{4.561642in}}%
\pgfpathlineto{\pgfqpoint{4.926982in}{4.558293in}}%
\pgfpathlineto{\pgfqpoint{4.928055in}{4.553613in}}%
\pgfpathlineto{\pgfqpoint{4.930203in}{4.556790in}}%
\pgfpathlineto{\pgfqpoint{4.933425in}{4.555073in}}%
\pgfpathlineto{\pgfqpoint{4.934498in}{4.559581in}}%
\pgfpathlineto{\pgfqpoint{4.935572in}{4.557477in}}%
\pgfpathlineto{\pgfqpoint{4.936646in}{4.559753in}}%
\pgfpathlineto{\pgfqpoint{4.937720in}{4.552582in}}%
\pgfpathlineto{\pgfqpoint{4.940941in}{4.542406in}}%
\pgfpathlineto{\pgfqpoint{4.942015in}{4.541977in}}%
\pgfpathlineto{\pgfqpoint{4.943089in}{4.550822in}}%
\pgfpathlineto{\pgfqpoint{4.944163in}{4.537468in}}%
\pgfpathlineto{\pgfqpoint{4.945237in}{4.534248in}}%
\pgfpathlineto{\pgfqpoint{4.948458in}{4.538027in}}%
\pgfpathlineto{\pgfqpoint{4.949532in}{4.540173in}}%
\pgfpathlineto{\pgfqpoint{4.950606in}{4.545584in}}%
\pgfpathlineto{\pgfqpoint{4.951680in}{4.540903in}}%
\pgfpathlineto{\pgfqpoint{4.955975in}{4.542664in}}%
\pgfpathlineto{\pgfqpoint{4.957049in}{4.544253in}}%
\pgfpathlineto{\pgfqpoint{4.958123in}{4.544510in}}%
\pgfpathlineto{\pgfqpoint{4.959197in}{4.545627in}}%
\pgfpathlineto{\pgfqpoint{4.960271in}{4.544124in}}%
\pgfpathlineto{\pgfqpoint{4.963492in}{4.544210in}}%
\pgfpathlineto{\pgfqpoint{4.964566in}{4.547129in}}%
\pgfpathlineto{\pgfqpoint{4.965640in}{4.545669in}}%
\pgfpathlineto{\pgfqpoint{4.966714in}{4.543265in}}%
\pgfpathlineto{\pgfqpoint{4.971009in}{4.545412in}}%
\pgfpathlineto{\pgfqpoint{4.972083in}{4.540818in}}%
\pgfpathlineto{\pgfqpoint{4.973157in}{4.547859in}}%
\pgfpathlineto{\pgfqpoint{4.974230in}{4.550393in}}%
\pgfpathlineto{\pgfqpoint{4.978526in}{4.554515in}}%
\pgfpathlineto{\pgfqpoint{4.980673in}{4.549791in}}%
\pgfpathlineto{\pgfqpoint{4.981747in}{4.546356in}}%
\pgfpathlineto{\pgfqpoint{4.982821in}{4.545970in}}%
\pgfpathlineto{\pgfqpoint{4.986043in}{4.548417in}}%
\pgfpathlineto{\pgfqpoint{4.987116in}{4.544338in}}%
\pgfpathlineto{\pgfqpoint{4.988190in}{4.547387in}}%
\pgfpathlineto{\pgfqpoint{4.989264in}{4.548546in}}%
\pgfpathlineto{\pgfqpoint{4.994633in}{4.561384in}}%
\pgfpathlineto{\pgfqpoint{4.995707in}{4.560096in}}%
\pgfpathlineto{\pgfqpoint{4.997855in}{4.561814in}}%
\pgfpathlineto{\pgfqpoint{5.001076in}{4.563360in}}%
\pgfpathlineto{\pgfqpoint{5.002150in}{4.562716in}}%
\pgfpathlineto{\pgfqpoint{5.003224in}{4.563059in}}%
\pgfpathlineto{\pgfqpoint{5.004298in}{4.566966in}}%
\pgfpathlineto{\pgfqpoint{5.005372in}{4.575339in}}%
\pgfpathlineto{\pgfqpoint{5.008593in}{4.577958in}}%
\pgfpathlineto{\pgfqpoint{5.011815in}{4.574738in}}%
\pgfpathlineto{\pgfqpoint{5.012889in}{4.575081in}}%
\pgfpathlineto{\pgfqpoint{5.016110in}{4.573192in}}%
\pgfpathlineto{\pgfqpoint{5.017184in}{4.574351in}}%
\pgfpathlineto{\pgfqpoint{5.018258in}{4.577915in}}%
\pgfpathlineto{\pgfqpoint{5.019332in}{4.575983in}}%
\pgfpathlineto{\pgfqpoint{5.020405in}{4.577829in}}%
\pgfpathlineto{\pgfqpoint{5.023627in}{4.577701in}}%
\pgfpathlineto{\pgfqpoint{5.024701in}{4.573450in}}%
\pgfpathlineto{\pgfqpoint{5.025775in}{4.574266in}}%
\pgfpathlineto{\pgfqpoint{5.026849in}{4.572935in}}%
\pgfpathlineto{\pgfqpoint{5.027922in}{4.575425in}}%
\pgfpathlineto{\pgfqpoint{5.031144in}{4.575167in}}%
\pgfpathlineto{\pgfqpoint{5.032218in}{4.577099in}}%
\pgfpathlineto{\pgfqpoint{5.033292in}{4.576541in}}%
\pgfpathlineto{\pgfqpoint{5.034365in}{4.579032in}}%
\pgfpathlineto{\pgfqpoint{5.038661in}{4.577271in}}%
\pgfpathlineto{\pgfqpoint{5.039735in}{4.574008in}}%
\pgfpathlineto{\pgfqpoint{5.040808in}{4.575683in}}%
\pgfpathlineto{\pgfqpoint{5.041882in}{4.574566in}}%
\pgfpathlineto{\pgfqpoint{5.048325in}{4.574609in}}%
\pgfpathlineto{\pgfqpoint{5.049399in}{4.568727in}}%
\pgfpathlineto{\pgfqpoint{5.050473in}{4.570874in}}%
\pgfpathlineto{\pgfqpoint{5.053694in}{4.568297in}}%
\pgfpathlineto{\pgfqpoint{5.054768in}{4.570272in}}%
\pgfpathlineto{\pgfqpoint{5.056916in}{4.569371in}}%
\pgfpathlineto{\pgfqpoint{5.057990in}{4.564519in}}%
\pgfpathlineto{\pgfqpoint{5.062285in}{4.563660in}}%
\pgfpathlineto{\pgfqpoint{5.063359in}{4.560483in}}%
\pgfpathlineto{\pgfqpoint{5.065507in}{4.542836in}}%
\pgfpathlineto{\pgfqpoint{5.068728in}{4.544639in}}%
\pgfpathlineto{\pgfqpoint{5.069802in}{4.542406in}}%
\pgfpathlineto{\pgfqpoint{5.070876in}{4.542535in}}%
\pgfpathlineto{\pgfqpoint{5.071950in}{4.541032in}}%
\pgfpathlineto{\pgfqpoint{5.073024in}{4.546786in}}%
\pgfpathlineto{\pgfqpoint{5.076245in}{4.544725in}}%
\pgfpathlineto{\pgfqpoint{5.077319in}{4.545240in}}%
\pgfpathlineto{\pgfqpoint{5.078393in}{4.546528in}}%
\pgfpathlineto{\pgfqpoint{5.079467in}{4.546013in}}%
\pgfpathlineto{\pgfqpoint{5.080540in}{4.543351in}}%
\pgfpathlineto{\pgfqpoint{5.083762in}{4.545584in}}%
\pgfpathlineto{\pgfqpoint{5.084836in}{4.549405in}}%
\pgfpathlineto{\pgfqpoint{5.085910in}{4.550865in}}%
\pgfpathlineto{\pgfqpoint{5.086983in}{4.553441in}}%
\pgfpathlineto{\pgfqpoint{5.088057in}{4.552411in}}%
\pgfpathlineto{\pgfqpoint{5.091279in}{4.555244in}}%
\pgfpathlineto{\pgfqpoint{5.092353in}{4.553484in}}%
\pgfpathlineto{\pgfqpoint{5.093427in}{4.553828in}}%
\pgfpathlineto{\pgfqpoint{5.094500in}{4.552969in}}%
\pgfpathlineto{\pgfqpoint{5.095574in}{4.555073in}}%
\pgfpathlineto{\pgfqpoint{5.099870in}{4.555760in}}%
\pgfpathlineto{\pgfqpoint{5.100943in}{4.557434in}}%
\pgfpathlineto{\pgfqpoint{5.102017in}{4.555459in}}%
\pgfpathlineto{\pgfqpoint{5.103091in}{4.555287in}}%
\pgfpathlineto{\pgfqpoint{5.106313in}{4.552625in}}%
\pgfpathlineto{\pgfqpoint{5.107386in}{4.548503in}}%
\pgfpathlineto{\pgfqpoint{5.108460in}{4.550521in}}%
\pgfpathlineto{\pgfqpoint{5.109534in}{4.550564in}}%
\pgfpathlineto{\pgfqpoint{5.110608in}{4.547473in}}%
\pgfpathlineto{\pgfqpoint{5.113829in}{4.546442in}}%
\pgfpathlineto{\pgfqpoint{5.117051in}{4.557262in}}%
\pgfpathlineto{\pgfqpoint{5.118125in}{4.555674in}}%
\pgfpathlineto{\pgfqpoint{5.121346in}{4.552840in}}%
\pgfpathlineto{\pgfqpoint{5.122420in}{4.550135in}}%
\pgfpathlineto{\pgfqpoint{5.123494in}{4.550994in}}%
\pgfpathlineto{\pgfqpoint{5.124568in}{4.543995in}}%
\pgfpathlineto{\pgfqpoint{5.125642in}{4.550393in}}%
\pgfpathlineto{\pgfqpoint{5.128863in}{4.548718in}}%
\pgfpathlineto{\pgfqpoint{5.129937in}{4.547086in}}%
\pgfpathlineto{\pgfqpoint{5.131011in}{4.541204in}}%
\pgfpathlineto{\pgfqpoint{5.132085in}{4.541462in}}%
\pgfpathlineto{\pgfqpoint{5.133159in}{4.546614in}}%
\pgfpathlineto{\pgfqpoint{5.136380in}{4.546099in}}%
\pgfpathlineto{\pgfqpoint{5.137454in}{4.539401in}}%
\pgfpathlineto{\pgfqpoint{5.138528in}{4.547602in}}%
\pgfpathlineto{\pgfqpoint{5.140675in}{4.537941in}}%
\pgfpathlineto{\pgfqpoint{5.143897in}{4.529139in}}%
\pgfpathlineto{\pgfqpoint{5.144971in}{4.528967in}}%
\pgfpathlineto{\pgfqpoint{5.146045in}{4.521753in}}%
\pgfpathlineto{\pgfqpoint{5.147118in}{4.519005in}}%
\pgfpathlineto{\pgfqpoint{5.148192in}{4.528323in}}%
\pgfpathlineto{\pgfqpoint{5.151414in}{4.534033in}}%
\pgfpathlineto{\pgfqpoint{5.152488in}{4.540560in}}%
\pgfpathlineto{\pgfqpoint{5.153561in}{4.533862in}}%
\pgfpathlineto{\pgfqpoint{5.155709in}{4.543566in}}%
\pgfpathlineto{\pgfqpoint{5.158931in}{4.544510in}}%
\pgfpathlineto{\pgfqpoint{5.160004in}{4.550006in}}%
\pgfpathlineto{\pgfqpoint{5.162152in}{4.552883in}}%
\pgfpathlineto{\pgfqpoint{5.163226in}{4.557735in}}%
\pgfpathlineto{\pgfqpoint{5.166448in}{4.561299in}}%
\pgfpathlineto{\pgfqpoint{5.167521in}{4.563445in}}%
\pgfpathlineto{\pgfqpoint{5.168595in}{4.567567in}}%
\pgfpathlineto{\pgfqpoint{5.169669in}{4.564218in}}%
\pgfpathlineto{\pgfqpoint{5.170743in}{4.566923in}}%
\pgfpathlineto{\pgfqpoint{5.173964in}{4.567482in}}%
\pgfpathlineto{\pgfqpoint{5.175038in}{4.564862in}}%
\pgfpathlineto{\pgfqpoint{5.176112in}{4.564090in}}%
\pgfpathlineto{\pgfqpoint{5.178260in}{4.560612in}}%
\pgfpathlineto{\pgfqpoint{5.181481in}{4.558422in}}%
\pgfpathlineto{\pgfqpoint{5.182555in}{4.560225in}}%
\pgfpathlineto{\pgfqpoint{5.183629in}{4.559925in}}%
\pgfpathlineto{\pgfqpoint{5.184703in}{4.560397in}}%
\pgfpathlineto{\pgfqpoint{5.185777in}{4.559495in}}%
\pgfpathlineto{\pgfqpoint{5.191146in}{4.563703in}}%
\pgfpathlineto{\pgfqpoint{5.193293in}{4.567567in}}%
\pgfpathlineto{\pgfqpoint{5.196515in}{4.566451in}}%
\pgfpathlineto{\pgfqpoint{5.197589in}{4.569886in}}%
\pgfpathlineto{\pgfqpoint{5.198663in}{4.562887in}}%
\pgfpathlineto{\pgfqpoint{5.200810in}{4.568469in}}%
\pgfpathlineto{\pgfqpoint{5.204032in}{4.572119in}}%
\pgfpathlineto{\pgfqpoint{5.205106in}{4.571303in}}%
\pgfpathlineto{\pgfqpoint{5.206180in}{4.568898in}}%
\pgfpathlineto{\pgfqpoint{5.207253in}{4.570401in}}%
\pgfpathlineto{\pgfqpoint{5.208327in}{4.561900in}}%
\pgfpathlineto{\pgfqpoint{5.211549in}{4.558078in}}%
\pgfpathlineto{\pgfqpoint{5.212623in}{4.550908in}}%
\pgfpathlineto{\pgfqpoint{5.214770in}{4.570573in}}%
\pgfpathlineto{\pgfqpoint{5.215844in}{4.569500in}}%
\pgfpathlineto{\pgfqpoint{5.221213in}{4.574180in}}%
\pgfpathlineto{\pgfqpoint{5.223361in}{4.575038in}}%
\pgfpathlineto{\pgfqpoint{5.227656in}{4.574953in}}%
\pgfpathlineto{\pgfqpoint{5.228730in}{4.570058in}}%
\pgfpathlineto{\pgfqpoint{5.230878in}{4.569972in}}%
\pgfpathlineto{\pgfqpoint{5.234099in}{4.560139in}}%
\pgfpathlineto{\pgfqpoint{5.235173in}{4.552454in}}%
\pgfpathlineto{\pgfqpoint{5.237321in}{4.565292in}}%
\pgfpathlineto{\pgfqpoint{5.238395in}{4.560655in}}%
\pgfpathlineto{\pgfqpoint{5.242690in}{4.555889in}}%
\pgfpathlineto{\pgfqpoint{5.244838in}{4.542492in}}%
\pgfpathlineto{\pgfqpoint{5.245912in}{4.543136in}}%
\pgfpathlineto{\pgfqpoint{5.251281in}{4.549663in}}%
\pgfpathlineto{\pgfqpoint{5.252355in}{4.536481in}}%
\pgfpathlineto{\pgfqpoint{5.256650in}{4.532187in}}%
\pgfpathlineto{\pgfqpoint{5.258798in}{4.525918in}}%
\pgfpathlineto{\pgfqpoint{5.259871in}{4.526992in}}%
\pgfpathlineto{\pgfqpoint{5.260945in}{4.521753in}}%
\pgfpathlineto{\pgfqpoint{5.264167in}{4.527507in}}%
\pgfpathlineto{\pgfqpoint{5.265241in}{4.533905in}}%
\pgfpathlineto{\pgfqpoint{5.266315in}{4.533432in}}%
\pgfpathlineto{\pgfqpoint{5.267388in}{4.537898in}}%
\pgfpathlineto{\pgfqpoint{5.268462in}{4.539014in}}%
\pgfpathlineto{\pgfqpoint{5.271684in}{4.538928in}}%
\pgfpathlineto{\pgfqpoint{5.272758in}{4.542363in}}%
\pgfpathlineto{\pgfqpoint{5.273831in}{4.543007in}}%
\pgfpathlineto{\pgfqpoint{5.274905in}{4.520895in}}%
\pgfpathlineto{\pgfqpoint{5.275979in}{4.511320in}}%
\pgfpathlineto{\pgfqpoint{5.280274in}{4.515313in}}%
\pgfpathlineto{\pgfqpoint{5.281348in}{4.518104in}}%
\pgfpathlineto{\pgfqpoint{5.282422in}{4.512608in}}%
\pgfpathlineto{\pgfqpoint{5.283496in}{4.518318in}}%
\pgfpathlineto{\pgfqpoint{5.286717in}{4.520208in}}%
\pgfpathlineto{\pgfqpoint{5.287791in}{4.522440in}}%
\pgfpathlineto{\pgfqpoint{5.289939in}{4.531972in}}%
\pgfpathlineto{\pgfqpoint{5.291013in}{4.525360in}}%
\pgfpathlineto{\pgfqpoint{5.294234in}{4.527121in}}%
\pgfpathlineto{\pgfqpoint{5.295308in}{4.526648in}}%
\pgfpathlineto{\pgfqpoint{5.296382in}{4.521496in}}%
\pgfpathlineto{\pgfqpoint{5.297456in}{4.523600in}}%
\pgfpathlineto{\pgfqpoint{5.298530in}{4.520251in}}%
\pgfpathlineto{\pgfqpoint{5.301751in}{4.521023in}}%
\pgfpathlineto{\pgfqpoint{5.302825in}{4.515442in}}%
\pgfpathlineto{\pgfqpoint{5.303899in}{4.516816in}}%
\pgfpathlineto{\pgfqpoint{5.304973in}{4.525231in}}%
\pgfpathlineto{\pgfqpoint{5.306047in}{4.521410in}}%
\pgfpathlineto{\pgfqpoint{5.309268in}{4.524974in}}%
\pgfpathlineto{\pgfqpoint{5.310342in}{4.523213in}}%
\pgfpathlineto{\pgfqpoint{5.311416in}{4.526434in}}%
\pgfpathlineto{\pgfqpoint{5.312490in}{4.525145in}}%
\pgfpathlineto{\pgfqpoint{5.313563in}{4.529783in}}%
\pgfpathlineto{\pgfqpoint{5.316785in}{4.527808in}}%
\pgfpathlineto{\pgfqpoint{5.318933in}{4.519435in}}%
\pgfpathlineto{\pgfqpoint{5.320006in}{4.512908in}}%
\pgfpathlineto{\pgfqpoint{5.321080in}{4.510890in}}%
\pgfpathlineto{\pgfqpoint{5.324302in}{4.511191in}}%
\pgfpathlineto{\pgfqpoint{5.325376in}{4.512522in}}%
\pgfpathlineto{\pgfqpoint{5.327523in}{4.518877in}}%
\pgfpathlineto{\pgfqpoint{5.331819in}{4.518576in}}%
\pgfpathlineto{\pgfqpoint{5.332893in}{4.513381in}}%
\pgfpathlineto{\pgfqpoint{5.336114in}{4.518404in}}%
\pgfpathlineto{\pgfqpoint{5.339336in}{4.517116in}}%
\pgfpathlineto{\pgfqpoint{5.341483in}{4.519048in}}%
\pgfpathlineto{\pgfqpoint{5.342557in}{4.523729in}}%
\pgfpathlineto{\pgfqpoint{5.343631in}{4.509302in}}%
\pgfpathlineto{\pgfqpoint{5.347926in}{4.509173in}}%
\pgfpathlineto{\pgfqpoint{5.349000in}{4.513810in}}%
\pgfpathlineto{\pgfqpoint{5.351148in}{4.512007in}}%
\pgfpathlineto{\pgfqpoint{5.354369in}{4.509989in}}%
\pgfpathlineto{\pgfqpoint{5.355443in}{4.509989in}}%
\pgfpathlineto{\pgfqpoint{5.356517in}{4.508658in}}%
\pgfpathlineto{\pgfqpoint{5.358665in}{4.510762in}}%
\pgfpathlineto{\pgfqpoint{5.361886in}{4.513166in}}%
\pgfpathlineto{\pgfqpoint{5.362960in}{4.511406in}}%
\pgfpathlineto{\pgfqpoint{5.364034in}{4.511449in}}%
\pgfpathlineto{\pgfqpoint{5.366181in}{4.515957in}}%
\pgfpathlineto{\pgfqpoint{5.369403in}{4.518877in}}%
\pgfpathlineto{\pgfqpoint{5.370477in}{4.516386in}}%
\pgfpathlineto{\pgfqpoint{5.372625in}{4.523084in}}%
\pgfpathlineto{\pgfqpoint{5.373698in}{4.520938in}}%
\pgfpathlineto{\pgfqpoint{5.376920in}{4.520766in}}%
\pgfpathlineto{\pgfqpoint{5.377994in}{4.525532in}}%
\pgfpathlineto{\pgfqpoint{5.380141in}{4.523127in}}%
\pgfpathlineto{\pgfqpoint{5.381215in}{4.525060in}}%
\pgfpathlineto{\pgfqpoint{5.386584in}{4.520508in}}%
\pgfpathlineto{\pgfqpoint{5.387658in}{4.520422in}}%
\pgfpathlineto{\pgfqpoint{5.388732in}{4.518963in}}%
\pgfpathlineto{\pgfqpoint{5.391954in}{4.517932in}}%
\pgfpathlineto{\pgfqpoint{5.394101in}{4.522483in}}%
\pgfpathlineto{\pgfqpoint{5.395175in}{4.517503in}}%
\pgfpathlineto{\pgfqpoint{5.396249in}{4.517631in}}%
\pgfpathlineto{\pgfqpoint{5.399470in}{4.515270in}}%
\pgfpathlineto{\pgfqpoint{5.400544in}{4.516730in}}%
\pgfpathlineto{\pgfqpoint{5.401618in}{4.520809in}}%
\pgfpathlineto{\pgfqpoint{5.402692in}{4.521281in}}%
\pgfpathlineto{\pgfqpoint{5.403766in}{4.518190in}}%
\pgfpathlineto{\pgfqpoint{5.406987in}{4.517073in}}%
\pgfpathlineto{\pgfqpoint{5.408061in}{4.517503in}}%
\pgfpathlineto{\pgfqpoint{5.410209in}{4.523170in}}%
\pgfpathlineto{\pgfqpoint{5.411283in}{4.520809in}}%
\pgfpathlineto{\pgfqpoint{5.414504in}{4.525060in}}%
\pgfpathlineto{\pgfqpoint{5.415578in}{4.525532in}}%
\pgfpathlineto{\pgfqpoint{5.417726in}{4.519564in}}%
\pgfpathlineto{\pgfqpoint{5.418800in}{4.519564in}}%
\pgfpathlineto{\pgfqpoint{5.422021in}{4.511234in}}%
\pgfpathlineto{\pgfqpoint{5.423095in}{4.512093in}}%
\pgfpathlineto{\pgfqpoint{5.424169in}{4.514841in}}%
\pgfpathlineto{\pgfqpoint{5.429538in}{4.511577in}}%
\pgfpathlineto{\pgfqpoint{5.430612in}{4.511406in}}%
\pgfpathlineto{\pgfqpoint{5.431686in}{4.504879in}}%
\pgfpathlineto{\pgfqpoint{5.432759in}{4.506511in}}%
\pgfpathlineto{\pgfqpoint{5.433833in}{4.510418in}}%
\pgfpathlineto{\pgfqpoint{5.438129in}{4.517116in}}%
\pgfpathlineto{\pgfqpoint{5.439203in}{4.515399in}}%
\pgfpathlineto{\pgfqpoint{5.441350in}{4.518104in}}%
\pgfpathlineto{\pgfqpoint{5.444572in}{4.518447in}}%
\pgfpathlineto{\pgfqpoint{5.445646in}{4.517030in}}%
\pgfpathlineto{\pgfqpoint{5.446719in}{4.517202in}}%
\pgfpathlineto{\pgfqpoint{5.448867in}{4.504750in}}%
\pgfpathlineto{\pgfqpoint{5.452089in}{4.500800in}}%
\pgfpathlineto{\pgfqpoint{5.453162in}{4.501573in}}%
\pgfpathlineto{\pgfqpoint{5.455310in}{4.505652in}}%
\pgfpathlineto{\pgfqpoint{5.460679in}{4.504020in}}%
\pgfpathlineto{\pgfqpoint{5.461753in}{4.503505in}}%
\pgfpathlineto{\pgfqpoint{5.462827in}{4.501144in}}%
\pgfpathlineto{\pgfqpoint{5.463901in}{4.520122in}}%
\pgfpathlineto{\pgfqpoint{5.467122in}{4.526348in}}%
\pgfpathlineto{\pgfqpoint{5.468196in}{4.526648in}}%
\pgfpathlineto{\pgfqpoint{5.470344in}{4.524330in}}%
\pgfpathlineto{\pgfqpoint{5.471418in}{4.524931in}}%
\pgfpathlineto{\pgfqpoint{5.474639in}{4.525274in}}%
\pgfpathlineto{\pgfqpoint{5.475713in}{4.526262in}}%
\pgfpathlineto{\pgfqpoint{5.476787in}{4.525103in}}%
\pgfpathlineto{\pgfqpoint{5.478935in}{4.509302in}}%
\pgfpathlineto{\pgfqpoint{5.483230in}{4.495175in}}%
\pgfpathlineto{\pgfqpoint{5.485378in}{4.508701in}}%
\pgfpathlineto{\pgfqpoint{5.486451in}{4.507756in}}%
\pgfpathlineto{\pgfqpoint{5.489673in}{4.508056in}}%
\pgfpathlineto{\pgfqpoint{5.490747in}{4.497107in}}%
\pgfpathlineto{\pgfqpoint{5.491821in}{4.500886in}}%
\pgfpathlineto{\pgfqpoint{5.492894in}{4.502174in}}%
\pgfpathlineto{\pgfqpoint{5.493968in}{4.497408in}}%
\pgfpathlineto{\pgfqpoint{5.498264in}{4.503119in}}%
\pgfpathlineto{\pgfqpoint{5.499337in}{4.501616in}}%
\pgfpathlineto{\pgfqpoint{5.501485in}{4.503162in}}%
\pgfpathlineto{\pgfqpoint{5.504707in}{4.501702in}}%
\pgfpathlineto{\pgfqpoint{5.506854in}{4.510461in}}%
\pgfpathlineto{\pgfqpoint{5.507928in}{4.509430in}}%
\pgfpathlineto{\pgfqpoint{5.509002in}{4.504965in}}%
\pgfpathlineto{\pgfqpoint{5.512224in}{4.508142in}}%
\pgfpathlineto{\pgfqpoint{5.513297in}{4.503977in}}%
\pgfpathlineto{\pgfqpoint{5.514371in}{4.503677in}}%
\pgfpathlineto{\pgfqpoint{5.515445in}{4.499898in}}%
\pgfpathlineto{\pgfqpoint{5.516519in}{4.501530in}}%
\pgfpathlineto{\pgfqpoint{5.519740in}{4.494445in}}%
\pgfpathlineto{\pgfqpoint{5.520814in}{4.493501in}}%
\pgfpathlineto{\pgfqpoint{5.521888in}{4.497623in}}%
\pgfpathlineto{\pgfqpoint{5.522962in}{4.496678in}}%
\pgfpathlineto{\pgfqpoint{5.524036in}{4.498739in}}%
\pgfpathlineto{\pgfqpoint{5.527257in}{4.510375in}}%
\pgfpathlineto{\pgfqpoint{5.528331in}{4.508829in}}%
\pgfpathlineto{\pgfqpoint{5.529405in}{4.511105in}}%
\pgfpathlineto{\pgfqpoint{5.530479in}{4.511105in}}%
\pgfpathlineto{\pgfqpoint{5.531553in}{4.511706in}}%
\pgfpathlineto{\pgfqpoint{5.534774in}{4.511620in}}%
\pgfpathlineto{\pgfqpoint{5.536922in}{4.506983in}}%
\pgfpathlineto{\pgfqpoint{5.539069in}{4.511191in}}%
\pgfpathlineto{\pgfqpoint{5.543365in}{4.510160in}}%
\pgfpathlineto{\pgfqpoint{5.544439in}{4.508400in}}%
\pgfpathlineto{\pgfqpoint{5.545513in}{4.492170in}}%
\pgfpathlineto{\pgfqpoint{5.546586in}{4.500628in}}%
\pgfpathlineto{\pgfqpoint{5.550882in}{4.498267in}}%
\pgfpathlineto{\pgfqpoint{5.551956in}{4.500070in}}%
\pgfpathlineto{\pgfqpoint{5.553029in}{4.499126in}}%
\pgfpathlineto{\pgfqpoint{5.554103in}{4.495261in}}%
\pgfpathlineto{\pgfqpoint{5.558399in}{4.498353in}}%
\pgfpathlineto{\pgfqpoint{5.559472in}{4.498524in}}%
\pgfpathlineto{\pgfqpoint{5.560546in}{4.497966in}}%
\pgfpathlineto{\pgfqpoint{5.561620in}{4.499426in}}%
\pgfpathlineto{\pgfqpoint{5.564842in}{4.495905in}}%
\pgfpathlineto{\pgfqpoint{5.565915in}{4.495691in}}%
\pgfpathlineto{\pgfqpoint{5.566989in}{4.493844in}}%
\pgfpathlineto{\pgfqpoint{5.569137in}{4.486931in}}%
\pgfpathlineto{\pgfqpoint{5.572358in}{4.488778in}}%
\pgfpathlineto{\pgfqpoint{5.573432in}{4.486545in}}%
\pgfpathlineto{\pgfqpoint{5.575580in}{4.493157in}}%
\pgfpathlineto{\pgfqpoint{5.576654in}{4.491826in}}%
\pgfpathlineto{\pgfqpoint{5.579875in}{4.491139in}}%
\pgfpathlineto{\pgfqpoint{5.580949in}{4.488649in}}%
\pgfpathlineto{\pgfqpoint{5.584171in}{4.489551in}}%
\pgfpathlineto{\pgfqpoint{5.587392in}{4.488692in}}%
\pgfpathlineto{\pgfqpoint{5.588466in}{4.490967in}}%
\pgfpathlineto{\pgfqpoint{5.590614in}{4.483754in}}%
\pgfpathlineto{\pgfqpoint{5.591688in}{4.486502in}}%
\pgfpathlineto{\pgfqpoint{5.594909in}{4.484527in}}%
\pgfpathlineto{\pgfqpoint{5.595983in}{4.481693in}}%
\pgfpathlineto{\pgfqpoint{5.597057in}{4.481478in}}%
\pgfpathlineto{\pgfqpoint{5.598131in}{4.482509in}}%
\pgfpathlineto{\pgfqpoint{5.599204in}{4.477442in}}%
\pgfpathlineto{\pgfqpoint{5.602426in}{4.477356in}}%
\pgfpathlineto{\pgfqpoint{5.603500in}{4.482681in}}%
\pgfpathlineto{\pgfqpoint{5.604574in}{4.484956in}}%
\pgfpathlineto{\pgfqpoint{5.606721in}{4.473449in}}%
\pgfpathlineto{\pgfqpoint{5.609943in}{4.475596in}}%
\pgfpathlineto{\pgfqpoint{5.611017in}{4.477399in}}%
\pgfpathlineto{\pgfqpoint{5.612091in}{4.481951in}}%
\pgfpathlineto{\pgfqpoint{5.613164in}{4.482724in}}%
\pgfpathlineto{\pgfqpoint{5.617460in}{4.481092in}}%
\pgfpathlineto{\pgfqpoint{5.618534in}{4.484269in}}%
\pgfpathlineto{\pgfqpoint{5.620681in}{4.480233in}}%
\pgfpathlineto{\pgfqpoint{5.624977in}{4.472333in}}%
\pgfpathlineto{\pgfqpoint{5.626050in}{4.468125in}}%
\pgfpathlineto{\pgfqpoint{5.627124in}{4.460654in}}%
\pgfpathlineto{\pgfqpoint{5.628198in}{4.458292in}}%
\pgfpathlineto{\pgfqpoint{5.629272in}{4.457434in}}%
\pgfpathlineto{\pgfqpoint{5.632493in}{4.459151in}}%
\pgfpathlineto{\pgfqpoint{5.633567in}{4.460568in}}%
\pgfpathlineto{\pgfqpoint{5.634641in}{4.454256in}}%
\pgfpathlineto{\pgfqpoint{5.635715in}{4.456060in}}%
\pgfpathlineto{\pgfqpoint{5.636789in}{4.454514in}}%
\pgfpathlineto{\pgfqpoint{5.641084in}{4.453440in}}%
\pgfpathlineto{\pgfqpoint{5.642158in}{4.454986in}}%
\pgfpathlineto{\pgfqpoint{5.643232in}{4.453440in}}%
\pgfpathlineto{\pgfqpoint{5.644306in}{4.422654in}}%
\pgfpathlineto{\pgfqpoint{5.648601in}{4.422783in}}%
\pgfpathlineto{\pgfqpoint{5.649675in}{4.420465in}}%
\pgfpathlineto{\pgfqpoint{5.650749in}{4.413809in}}%
\pgfpathlineto{\pgfqpoint{5.651823in}{4.416300in}}%
\pgfpathlineto{\pgfqpoint{5.655044in}{4.421195in}}%
\pgfpathlineto{\pgfqpoint{5.656118in}{4.416987in}}%
\pgfpathlineto{\pgfqpoint{5.658266in}{4.419906in}}%
\pgfpathlineto{\pgfqpoint{5.659339in}{4.418275in}}%
\pgfpathlineto{\pgfqpoint{5.662561in}{4.411834in}}%
\pgfpathlineto{\pgfqpoint{5.663635in}{4.412779in}}%
\pgfpathlineto{\pgfqpoint{5.664709in}{4.411405in}}%
\pgfpathlineto{\pgfqpoint{5.665782in}{4.406596in}}%
\pgfpathlineto{\pgfqpoint{5.666856in}{4.412908in}}%
\pgfpathlineto{\pgfqpoint{5.671152in}{4.415012in}}%
\pgfpathlineto{\pgfqpoint{5.677595in}{4.424973in}}%
\pgfpathlineto{\pgfqpoint{5.679742in}{4.420937in}}%
\pgfpathlineto{\pgfqpoint{5.680816in}{4.423985in}}%
\pgfpathlineto{\pgfqpoint{5.681890in}{4.423942in}}%
\pgfpathlineto{\pgfqpoint{5.685112in}{4.424758in}}%
\pgfpathlineto{\pgfqpoint{5.686185in}{4.429696in}}%
\pgfpathlineto{\pgfqpoint{5.687259in}{4.431027in}}%
\pgfpathlineto{\pgfqpoint{5.688333in}{4.434934in}}%
\pgfpathlineto{\pgfqpoint{5.692628in}{4.438627in}}%
\pgfpathlineto{\pgfqpoint{5.693702in}{4.440387in}}%
\pgfpathlineto{\pgfqpoint{5.695850in}{4.437639in}}%
\pgfpathlineto{\pgfqpoint{5.696924in}{4.440516in}}%
\pgfpathlineto{\pgfqpoint{5.700145in}{4.440946in}}%
\pgfpathlineto{\pgfqpoint{5.701219in}{4.439572in}}%
\pgfpathlineto{\pgfqpoint{5.703367in}{4.443050in}}%
\pgfpathlineto{\pgfqpoint{5.704441in}{4.447644in}}%
\pgfpathlineto{\pgfqpoint{5.707662in}{4.447601in}}%
\pgfpathlineto{\pgfqpoint{5.708736in}{4.445196in}}%
\pgfpathlineto{\pgfqpoint{5.709810in}{4.445282in}}%
\pgfpathlineto{\pgfqpoint{5.710884in}{4.444595in}}%
\pgfpathlineto{\pgfqpoint{5.715179in}{4.443822in}}%
\pgfpathlineto{\pgfqpoint{5.716253in}{4.445153in}}%
\pgfpathlineto{\pgfqpoint{5.717327in}{4.443865in}}%
\pgfpathlineto{\pgfqpoint{5.718401in}{4.448374in}}%
\pgfpathlineto{\pgfqpoint{5.719474in}{4.447172in}}%
\pgfpathlineto{\pgfqpoint{5.722696in}{4.445239in}}%
\pgfpathlineto{\pgfqpoint{5.723770in}{4.443479in}}%
\pgfpathlineto{\pgfqpoint{5.724844in}{4.443393in}}%
\pgfpathlineto{\pgfqpoint{5.725917in}{4.439228in}}%
\pgfpathlineto{\pgfqpoint{5.726991in}{4.441847in}}%
\pgfpathlineto{\pgfqpoint{5.730213in}{4.443050in}}%
\pgfpathlineto{\pgfqpoint{5.731287in}{4.446656in}}%
\pgfpathlineto{\pgfqpoint{5.732360in}{4.452710in}}%
\pgfpathlineto{\pgfqpoint{5.733434in}{4.454127in}}%
\pgfpathlineto{\pgfqpoint{5.734508in}{4.452625in}}%
\pgfpathlineto{\pgfqpoint{5.737730in}{4.454471in}}%
\pgfpathlineto{\pgfqpoint{5.740951in}{4.466751in}}%
\pgfpathlineto{\pgfqpoint{5.742025in}{4.468082in}}%
\pgfpathlineto{\pgfqpoint{5.745246in}{4.467137in}}%
\pgfpathlineto{\pgfqpoint{5.746320in}{4.469198in}}%
\pgfpathlineto{\pgfqpoint{5.747394in}{4.469284in}}%
\pgfpathlineto{\pgfqpoint{5.749542in}{4.466021in}}%
\pgfpathlineto{\pgfqpoint{5.752763in}{4.467051in}}%
\pgfpathlineto{\pgfqpoint{5.754911in}{4.461255in}}%
\pgfpathlineto{\pgfqpoint{5.755985in}{4.459881in}}%
\pgfpathlineto{\pgfqpoint{5.757059in}{4.462285in}}%
\pgfpathlineto{\pgfqpoint{5.760280in}{4.460139in}}%
\pgfpathlineto{\pgfqpoint{5.761354in}{4.463616in}}%
\pgfpathlineto{\pgfqpoint{5.764576in}{4.460697in}}%
\pgfpathlineto{\pgfqpoint{5.767797in}{4.460482in}}%
\pgfpathlineto{\pgfqpoint{5.768871in}{4.455158in}}%
\pgfpathlineto{\pgfqpoint{5.769945in}{4.458249in}}%
\pgfpathlineto{\pgfqpoint{5.771019in}{4.455072in}}%
\pgfpathlineto{\pgfqpoint{5.772092in}{4.459881in}}%
\pgfpathlineto{\pgfqpoint{5.775314in}{4.458550in}}%
\pgfpathlineto{\pgfqpoint{5.776388in}{4.463745in}}%
\pgfpathlineto{\pgfqpoint{5.777462in}{4.465549in}}%
\pgfpathlineto{\pgfqpoint{5.778535in}{4.465205in}}%
\pgfpathlineto{\pgfqpoint{5.779609in}{4.466407in}}%
\pgfpathlineto{\pgfqpoint{5.784979in}{4.467996in}}%
\pgfpathlineto{\pgfqpoint{5.786052in}{4.470014in}}%
\pgfpathlineto{\pgfqpoint{5.787126in}{4.466279in}}%
\pgfpathlineto{\pgfqpoint{5.790348in}{4.468125in}}%
\pgfpathlineto{\pgfqpoint{5.791422in}{4.467910in}}%
\pgfpathlineto{\pgfqpoint{5.792495in}{4.469413in}}%
\pgfpathlineto{\pgfqpoint{5.797865in}{4.458850in}}%
\pgfpathlineto{\pgfqpoint{5.798938in}{4.448245in}}%
\pgfpathlineto{\pgfqpoint{5.801086in}{4.451766in}}%
\pgfpathlineto{\pgfqpoint{5.802160in}{4.451465in}}%
\pgfpathlineto{\pgfqpoint{5.805381in}{4.453269in}}%
\pgfpathlineto{\pgfqpoint{5.806455in}{4.453226in}}%
\pgfpathlineto{\pgfqpoint{5.807529in}{4.451852in}}%
\pgfpathlineto{\pgfqpoint{5.808603in}{4.457133in}}%
\pgfpathlineto{\pgfqpoint{5.809677in}{4.444123in}}%
\pgfpathlineto{\pgfqpoint{5.812898in}{4.434376in}}%
\pgfpathlineto{\pgfqpoint{5.813972in}{4.435321in}}%
\pgfpathlineto{\pgfqpoint{5.816120in}{4.448245in}}%
\pgfpathlineto{\pgfqpoint{5.817194in}{4.447944in}}%
\pgfpathlineto{\pgfqpoint{5.821489in}{4.441633in}}%
\pgfpathlineto{\pgfqpoint{5.823637in}{4.444338in}}%
\pgfpathlineto{\pgfqpoint{5.824711in}{4.451208in}}%
\pgfpathlineto{\pgfqpoint{5.827932in}{4.453956in}}%
\pgfpathlineto{\pgfqpoint{5.829006in}{4.457434in}}%
\pgfpathlineto{\pgfqpoint{5.830080in}{4.457820in}}%
\pgfpathlineto{\pgfqpoint{5.831154in}{4.459924in}}%
\pgfpathlineto{\pgfqpoint{5.832227in}{4.460611in}}%
\pgfpathlineto{\pgfqpoint{5.836523in}{4.462157in}}%
\pgfpathlineto{\pgfqpoint{5.837597in}{4.463488in}}%
\pgfpathlineto{\pgfqpoint{5.838670in}{4.459194in}}%
\pgfpathlineto{\pgfqpoint{5.839744in}{4.462672in}}%
\pgfpathlineto{\pgfqpoint{5.844040in}{4.463015in}}%
\pgfpathlineto{\pgfqpoint{5.846187in}{4.464776in}}%
\pgfpathlineto{\pgfqpoint{5.847261in}{4.463402in}}%
\pgfpathlineto{\pgfqpoint{5.850483in}{4.462071in}}%
\pgfpathlineto{\pgfqpoint{5.851557in}{4.459366in}}%
\pgfpathlineto{\pgfqpoint{5.852630in}{4.460826in}}%
\pgfpathlineto{\pgfqpoint{5.853704in}{4.461255in}}%
\pgfpathlineto{\pgfqpoint{5.854778in}{4.467738in}}%
\pgfpathlineto{\pgfqpoint{5.858000in}{4.469027in}}%
\pgfpathlineto{\pgfqpoint{5.860147in}{4.464561in}}%
\pgfpathlineto{\pgfqpoint{5.861221in}{4.467567in}}%
\pgfpathlineto{\pgfqpoint{5.862295in}{4.467180in}}%
\pgfpathlineto{\pgfqpoint{5.865516in}{4.468211in}}%
\pgfpathlineto{\pgfqpoint{5.866590in}{4.466837in}}%
\pgfpathlineto{\pgfqpoint{5.867664in}{4.468425in}}%
\pgfpathlineto{\pgfqpoint{5.873033in}{4.467094in}}%
\pgfpathlineto{\pgfqpoint{5.874107in}{4.468425in}}%
\pgfpathlineto{\pgfqpoint{5.875181in}{4.466279in}}%
\pgfpathlineto{\pgfqpoint{5.877329in}{4.464776in}}%
\pgfpathlineto{\pgfqpoint{5.880550in}{4.467738in}}%
\pgfpathlineto{\pgfqpoint{5.881624in}{4.467481in}}%
\pgfpathlineto{\pgfqpoint{5.882698in}{4.467996in}}%
\pgfpathlineto{\pgfqpoint{5.883772in}{4.465033in}}%
\pgfpathlineto{\pgfqpoint{5.884846in}{4.466407in}}%
\pgfpathlineto{\pgfqpoint{5.889141in}{4.468726in}}%
\pgfpathlineto{\pgfqpoint{5.890215in}{4.470486in}}%
\pgfpathlineto{\pgfqpoint{5.891289in}{4.470701in}}%
\pgfpathlineto{\pgfqpoint{5.892362in}{4.466107in}}%
\pgfpathlineto{\pgfqpoint{5.895584in}{4.469327in}}%
\pgfpathlineto{\pgfqpoint{5.897732in}{4.459366in}}%
\pgfpathlineto{\pgfqpoint{5.898805in}{4.460826in}}%
\pgfpathlineto{\pgfqpoint{5.899879in}{4.460139in}}%
\pgfpathlineto{\pgfqpoint{5.903101in}{4.461770in}}%
\pgfpathlineto{\pgfqpoint{5.904175in}{4.460353in}}%
\pgfpathlineto{\pgfqpoint{5.906322in}{4.464089in}}%
\pgfpathlineto{\pgfqpoint{5.907396in}{4.460911in}}%
\pgfpathlineto{\pgfqpoint{5.910618in}{4.459151in}}%
\pgfpathlineto{\pgfqpoint{5.911691in}{4.462672in}}%
\pgfpathlineto{\pgfqpoint{5.912765in}{4.462414in}}%
\pgfpathlineto{\pgfqpoint{5.913839in}{4.458936in}}%
\pgfpathlineto{\pgfqpoint{5.914913in}{4.461684in}}%
\pgfpathlineto{\pgfqpoint{5.918134in}{4.460740in}}%
\pgfpathlineto{\pgfqpoint{5.919208in}{4.461169in}}%
\pgfpathlineto{\pgfqpoint{5.920282in}{4.464303in}}%
\pgfpathlineto{\pgfqpoint{5.921356in}{4.454342in}}%
\pgfpathlineto{\pgfqpoint{5.922430in}{4.453612in}}%
\pgfpathlineto{\pgfqpoint{5.925651in}{4.454170in}}%
\pgfpathlineto{\pgfqpoint{5.926725in}{4.449919in}}%
\pgfpathlineto{\pgfqpoint{5.928873in}{4.448030in}}%
\pgfpathlineto{\pgfqpoint{5.929947in}{4.446957in}}%
\pgfpathlineto{\pgfqpoint{5.933168in}{4.445926in}}%
\pgfpathlineto{\pgfqpoint{5.934242in}{4.446699in}}%
\pgfpathlineto{\pgfqpoint{5.935316in}{4.451508in}}%
\pgfpathlineto{\pgfqpoint{5.936390in}{4.474308in}}%
\pgfpathlineto{\pgfqpoint{5.937464in}{4.476669in}}%
\pgfpathlineto{\pgfqpoint{5.940685in}{4.475553in}}%
\pgfpathlineto{\pgfqpoint{5.941759in}{4.474136in}}%
\pgfpathlineto{\pgfqpoint{5.943907in}{4.474909in}}%
\pgfpathlineto{\pgfqpoint{5.944980in}{4.472934in}}%
\pgfpathlineto{\pgfqpoint{5.948202in}{4.472805in}}%
\pgfpathlineto{\pgfqpoint{5.949276in}{4.472118in}}%
\pgfpathlineto{\pgfqpoint{5.950350in}{4.468812in}}%
\pgfpathlineto{\pgfqpoint{5.951423in}{4.468340in}}%
\pgfpathlineto{\pgfqpoint{5.952497in}{4.469069in}}%
\pgfpathlineto{\pgfqpoint{5.955719in}{4.475210in}}%
\pgfpathlineto{\pgfqpoint{5.956793in}{4.475467in}}%
\pgfpathlineto{\pgfqpoint{5.958940in}{4.488005in}}%
\pgfpathlineto{\pgfqpoint{5.960014in}{4.489636in}}%
\pgfpathlineto{\pgfqpoint{5.963236in}{4.497537in}}%
\pgfpathlineto{\pgfqpoint{5.964310in}{4.497752in}}%
\pgfpathlineto{\pgfqpoint{5.965383in}{4.494488in}}%
\pgfpathlineto{\pgfqpoint{5.966457in}{4.494875in}}%
\pgfpathlineto{\pgfqpoint{5.967531in}{4.491697in}}%
\pgfpathlineto{\pgfqpoint{5.971826in}{4.494660in}}%
\pgfpathlineto{\pgfqpoint{5.972900in}{4.499426in}}%
\pgfpathlineto{\pgfqpoint{5.975048in}{4.499340in}}%
\pgfpathlineto{\pgfqpoint{5.978269in}{4.496335in}}%
\pgfpathlineto{\pgfqpoint{5.979343in}{4.493715in}}%
\pgfpathlineto{\pgfqpoint{5.981491in}{4.497966in}}%
\pgfpathlineto{\pgfqpoint{5.982565in}{4.495218in}}%
\pgfpathlineto{\pgfqpoint{5.985786in}{4.495905in}}%
\pgfpathlineto{\pgfqpoint{5.986860in}{4.497065in}}%
\pgfpathlineto{\pgfqpoint{5.987934in}{4.505223in}}%
\pgfpathlineto{\pgfqpoint{5.989008in}{4.507799in}}%
\pgfpathlineto{\pgfqpoint{5.990082in}{4.507198in}}%
\pgfpathlineto{\pgfqpoint{5.993303in}{4.502303in}}%
\pgfpathlineto{\pgfqpoint{5.995451in}{4.504321in}}%
\pgfpathlineto{\pgfqpoint{5.996525in}{4.507928in}}%
\pgfpathlineto{\pgfqpoint{5.997599in}{4.508142in}}%
\pgfpathlineto{\pgfqpoint{6.000820in}{4.506296in}}%
\pgfpathlineto{\pgfqpoint{6.002968in}{4.509473in}}%
\pgfpathlineto{\pgfqpoint{6.004042in}{4.506425in}}%
\pgfpathlineto{\pgfqpoint{6.005115in}{4.508056in}}%
\pgfpathlineto{\pgfqpoint{6.009411in}{4.508099in}}%
\pgfpathlineto{\pgfqpoint{6.011558in}{4.503720in}}%
\pgfpathlineto{\pgfqpoint{6.012632in}{4.504364in}}%
\pgfpathlineto{\pgfqpoint{6.016928in}{4.509602in}}%
\pgfpathlineto{\pgfqpoint{6.018001in}{4.514712in}}%
\pgfpathlineto{\pgfqpoint{6.019075in}{4.510804in}}%
\pgfpathlineto{\pgfqpoint{6.023371in}{4.513037in}}%
\pgfpathlineto{\pgfqpoint{6.024445in}{4.516300in}}%
\pgfpathlineto{\pgfqpoint{6.025518in}{4.517374in}}%
\pgfpathlineto{\pgfqpoint{6.026592in}{4.517245in}}%
\pgfpathlineto{\pgfqpoint{6.027666in}{4.516172in}}%
\pgfpathlineto{\pgfqpoint{6.031961in}{4.516086in}}%
\pgfpathlineto{\pgfqpoint{6.033035in}{4.519778in}}%
\pgfpathlineto{\pgfqpoint{6.035183in}{4.514454in}}%
\pgfpathlineto{\pgfqpoint{6.038404in}{4.513509in}}%
\pgfpathlineto{\pgfqpoint{6.039478in}{4.519521in}}%
\pgfpathlineto{\pgfqpoint{6.040552in}{4.517288in}}%
\pgfpathlineto{\pgfqpoint{6.042700in}{4.517116in}}%
\pgfpathlineto{\pgfqpoint{6.045921in}{4.518920in}}%
\pgfpathlineto{\pgfqpoint{6.046995in}{4.515184in}}%
\pgfpathlineto{\pgfqpoint{6.048069in}{4.516773in}}%
\pgfpathlineto{\pgfqpoint{6.049143in}{4.515742in}}%
\pgfpathlineto{\pgfqpoint{6.050217in}{4.522054in}}%
\pgfpathlineto{\pgfqpoint{6.055586in}{4.521066in}}%
\pgfpathlineto{\pgfqpoint{6.057734in}{4.523900in}}%
\pgfpathlineto{\pgfqpoint{6.060955in}{4.525661in}}%
\pgfpathlineto{\pgfqpoint{6.062029in}{4.527722in}}%
\pgfpathlineto{\pgfqpoint{6.063103in}{4.528538in}}%
\pgfpathlineto{\pgfqpoint{6.064177in}{4.528151in}}%
\pgfpathlineto{\pgfqpoint{6.065250in}{4.528967in}}%
\pgfpathlineto{\pgfqpoint{6.069546in}{4.530040in}}%
\pgfpathlineto{\pgfqpoint{6.070620in}{4.529611in}}%
\pgfpathlineto{\pgfqpoint{6.071693in}{4.530384in}}%
\pgfpathlineto{\pgfqpoint{6.072767in}{4.529182in}}%
\pgfpathlineto{\pgfqpoint{6.075989in}{4.530856in}}%
\pgfpathlineto{\pgfqpoint{6.077063in}{4.530427in}}%
\pgfpathlineto{\pgfqpoint{6.078136in}{4.538113in}}%
\pgfpathlineto{\pgfqpoint{6.079210in}{4.530598in}}%
\pgfpathlineto{\pgfqpoint{6.080284in}{4.529697in}}%
\pgfpathlineto{\pgfqpoint{6.083506in}{4.528108in}}%
\pgfpathlineto{\pgfqpoint{6.084579in}{4.528452in}}%
\pgfpathlineto{\pgfqpoint{6.085653in}{4.526219in}}%
\pgfpathlineto{\pgfqpoint{6.087801in}{4.527593in}}%
\pgfpathlineto{\pgfqpoint{6.091022in}{4.526906in}}%
\pgfpathlineto{\pgfqpoint{6.092096in}{4.528881in}}%
\pgfpathlineto{\pgfqpoint{6.093170in}{4.526992in}}%
\pgfpathlineto{\pgfqpoint{6.094244in}{4.529225in}}%
\pgfpathlineto{\pgfqpoint{6.095318in}{4.527078in}}%
\pgfpathlineto{\pgfqpoint{6.098539in}{4.525403in}}%
\pgfpathlineto{\pgfqpoint{6.099613in}{4.519950in}}%
\pgfpathlineto{\pgfqpoint{6.101761in}{4.521238in}}%
\pgfpathlineto{\pgfqpoint{6.102835in}{4.522741in}}%
\pgfpathlineto{\pgfqpoint{6.106056in}{4.520251in}}%
\pgfpathlineto{\pgfqpoint{6.107130in}{4.524544in}}%
\pgfpathlineto{\pgfqpoint{6.108204in}{4.522913in}}%
\pgfpathlineto{\pgfqpoint{6.109278in}{4.526906in}}%
\pgfpathlineto{\pgfqpoint{6.110352in}{4.526477in}}%
\pgfpathlineto{\pgfqpoint{6.114647in}{4.522956in}}%
\pgfpathlineto{\pgfqpoint{6.115721in}{4.522226in}}%
\pgfpathlineto{\pgfqpoint{6.116795in}{4.522870in}}%
\pgfpathlineto{\pgfqpoint{6.117868in}{4.522269in}}%
\pgfpathlineto{\pgfqpoint{6.122164in}{4.520165in}}%
\pgfpathlineto{\pgfqpoint{6.124311in}{4.514068in}}%
\pgfpathlineto{\pgfqpoint{6.128607in}{4.517674in}}%
\pgfpathlineto{\pgfqpoint{6.129681in}{4.514025in}}%
\pgfpathlineto{\pgfqpoint{6.130755in}{4.513037in}}%
\pgfpathlineto{\pgfqpoint{6.131828in}{4.531629in}}%
\pgfpathlineto{\pgfqpoint{6.132902in}{4.529826in}}%
\pgfpathlineto{\pgfqpoint{6.137198in}{4.534162in}}%
\pgfpathlineto{\pgfqpoint{6.139345in}{4.532917in}}%
\pgfpathlineto{\pgfqpoint{6.140419in}{4.528409in}}%
\pgfpathlineto{\pgfqpoint{6.143641in}{4.528323in}}%
\pgfpathlineto{\pgfqpoint{6.144714in}{4.529611in}}%
\pgfpathlineto{\pgfqpoint{6.146862in}{4.524587in}}%
\pgfpathlineto{\pgfqpoint{6.151157in}{4.523900in}}%
\pgfpathlineto{\pgfqpoint{6.153305in}{4.525918in}}%
\pgfpathlineto{\pgfqpoint{6.155453in}{4.521109in}}%
\pgfpathlineto{\pgfqpoint{6.158674in}{4.524587in}}%
\pgfpathlineto{\pgfqpoint{6.159748in}{4.523771in}}%
\pgfpathlineto{\pgfqpoint{6.160822in}{4.516429in}}%
\pgfpathlineto{\pgfqpoint{6.161896in}{4.516472in}}%
\pgfpathlineto{\pgfqpoint{6.162970in}{4.518233in}}%
\pgfpathlineto{\pgfqpoint{6.166191in}{4.518963in}}%
\pgfpathlineto{\pgfqpoint{6.167265in}{4.519907in}}%
\pgfpathlineto{\pgfqpoint{6.168339in}{4.519564in}}%
\pgfpathlineto{\pgfqpoint{6.169413in}{4.520895in}}%
\pgfpathlineto{\pgfqpoint{6.170487in}{4.520981in}}%
\pgfpathlineto{\pgfqpoint{6.175856in}{4.518834in}}%
\pgfpathlineto{\pgfqpoint{6.176930in}{4.524373in}}%
\pgfpathlineto{\pgfqpoint{6.178003in}{4.525274in}}%
\pgfpathlineto{\pgfqpoint{6.181225in}{4.527249in}}%
\pgfpathlineto{\pgfqpoint{6.182299in}{4.526734in}}%
\pgfpathlineto{\pgfqpoint{6.183373in}{4.530727in}}%
\pgfpathlineto{\pgfqpoint{6.184446in}{4.531328in}}%
\pgfpathlineto{\pgfqpoint{6.185520in}{4.532831in}}%
\pgfpathlineto{\pgfqpoint{6.188742in}{4.532230in}}%
\pgfpathlineto{\pgfqpoint{6.190889in}{4.535021in}}%
\pgfpathlineto{\pgfqpoint{6.191963in}{4.534420in}}%
\pgfpathlineto{\pgfqpoint{6.193037in}{4.537554in}}%
\pgfpathlineto{\pgfqpoint{6.196259in}{4.539358in}}%
\pgfpathlineto{\pgfqpoint{6.197333in}{4.541977in}}%
\pgfpathlineto{\pgfqpoint{6.198406in}{4.540732in}}%
\pgfpathlineto{\pgfqpoint{6.200554in}{4.540775in}}%
\pgfpathlineto{\pgfqpoint{6.204849in}{4.544338in}}%
\pgfpathlineto{\pgfqpoint{6.205923in}{4.548031in}}%
\pgfpathlineto{\pgfqpoint{6.206997in}{4.546614in}}%
\pgfpathlineto{\pgfqpoint{6.208071in}{4.549147in}}%
\pgfpathlineto{\pgfqpoint{6.211292in}{4.552711in}}%
\pgfpathlineto{\pgfqpoint{6.213440in}{4.553183in}}%
\pgfpathlineto{\pgfqpoint{6.214514in}{4.548374in}}%
\pgfpathlineto{\pgfqpoint{6.215588in}{4.550865in}}%
\pgfpathlineto{\pgfqpoint{6.218809in}{4.550736in}}%
\pgfpathlineto{\pgfqpoint{6.219883in}{4.550135in}}%
\pgfpathlineto{\pgfqpoint{6.222031in}{4.555202in}}%
\pgfpathlineto{\pgfqpoint{6.223105in}{4.554815in}}%
\pgfpathlineto{\pgfqpoint{6.226326in}{4.554557in}}%
\pgfpathlineto{\pgfqpoint{6.228474in}{4.557520in}}%
\pgfpathlineto{\pgfqpoint{6.229548in}{4.555116in}}%
\pgfpathlineto{\pgfqpoint{6.230622in}{4.556103in}}%
\pgfpathlineto{\pgfqpoint{6.233843in}{4.553656in}}%
\pgfpathlineto{\pgfqpoint{6.234917in}{4.555330in}}%
\pgfpathlineto{\pgfqpoint{6.235991in}{4.554901in}}%
\pgfpathlineto{\pgfqpoint{6.237065in}{4.548847in}}%
\pgfpathlineto{\pgfqpoint{6.238138in}{4.552797in}}%
\pgfpathlineto{\pgfqpoint{6.242434in}{4.554643in}}%
\pgfpathlineto{\pgfqpoint{6.243508in}{4.554901in}}%
\pgfpathlineto{\pgfqpoint{6.244581in}{4.555889in}}%
\pgfpathlineto{\pgfqpoint{6.245655in}{4.557692in}}%
\pgfpathlineto{\pgfqpoint{6.251024in}{4.556533in}}%
\pgfpathlineto{\pgfqpoint{6.252098in}{4.551809in}}%
\pgfpathlineto{\pgfqpoint{6.253172in}{4.550693in}}%
\pgfpathlineto{\pgfqpoint{6.256394in}{4.555588in}}%
\pgfpathlineto{\pgfqpoint{6.257467in}{4.561127in}}%
\pgfpathlineto{\pgfqpoint{6.258541in}{4.563617in}}%
\pgfpathlineto{\pgfqpoint{6.260689in}{4.555244in}}%
\pgfpathlineto{\pgfqpoint{6.266058in}{4.554815in}}%
\pgfpathlineto{\pgfqpoint{6.268206in}{4.555588in}}%
\pgfpathlineto{\pgfqpoint{6.272501in}{4.555373in}}%
\pgfpathlineto{\pgfqpoint{6.275723in}{4.558379in}}%
\pgfpathlineto{\pgfqpoint{6.281092in}{4.554600in}}%
\pgfpathlineto{\pgfqpoint{6.282166in}{4.551165in}}%
\pgfpathlineto{\pgfqpoint{6.283240in}{4.550521in}}%
\pgfpathlineto{\pgfqpoint{6.286461in}{4.556533in}}%
\pgfpathlineto{\pgfqpoint{6.287535in}{4.560096in}}%
\pgfpathlineto{\pgfqpoint{6.288609in}{4.560483in}}%
\pgfpathlineto{\pgfqpoint{6.289683in}{4.558594in}}%
\pgfpathlineto{\pgfqpoint{6.290756in}{4.561943in}}%
\pgfpathlineto{\pgfqpoint{6.293978in}{4.565592in}}%
\pgfpathlineto{\pgfqpoint{6.295052in}{4.570315in}}%
\pgfpathlineto{\pgfqpoint{6.296126in}{4.567954in}}%
\pgfpathlineto{\pgfqpoint{6.298273in}{4.567825in}}%
\pgfpathlineto{\pgfqpoint{6.301495in}{4.567009in}}%
\pgfpathlineto{\pgfqpoint{6.302569in}{4.569027in}}%
\pgfpathlineto{\pgfqpoint{6.304716in}{4.575081in}}%
\pgfpathlineto{\pgfqpoint{6.305790in}{4.576412in}}%
\pgfpathlineto{\pgfqpoint{6.309012in}{4.576756in}}%
\pgfpathlineto{\pgfqpoint{6.310086in}{4.580449in}}%
\pgfpathlineto{\pgfqpoint{6.311159in}{4.578688in}}%
\pgfpathlineto{\pgfqpoint{6.313307in}{4.582424in}}%
\pgfpathlineto{\pgfqpoint{6.316529in}{4.583025in}}%
\pgfpathlineto{\pgfqpoint{6.318676in}{4.584141in}}%
\pgfpathlineto{\pgfqpoint{6.319750in}{4.582681in}}%
\pgfpathlineto{\pgfqpoint{6.320824in}{4.587920in}}%
\pgfpathlineto{\pgfqpoint{6.325119in}{4.583025in}}%
\pgfpathlineto{\pgfqpoint{6.326193in}{4.584656in}}%
\pgfpathlineto{\pgfqpoint{6.327267in}{4.583884in}}%
\pgfpathlineto{\pgfqpoint{6.328341in}{4.584699in}}%
\pgfpathlineto{\pgfqpoint{6.331562in}{4.585902in}}%
\pgfpathlineto{\pgfqpoint{6.332636in}{4.592085in}}%
\pgfpathlineto{\pgfqpoint{6.333710in}{4.590711in}}%
\pgfpathlineto{\pgfqpoint{6.334784in}{4.599770in}}%
\pgfpathlineto{\pgfqpoint{6.335858in}{4.600200in}}%
\pgfpathlineto{\pgfqpoint{6.339079in}{4.597151in}}%
\pgfpathlineto{\pgfqpoint{6.341227in}{4.600200in}}%
\pgfpathlineto{\pgfqpoint{6.342301in}{4.600973in}}%
\pgfpathlineto{\pgfqpoint{6.343375in}{4.602862in}}%
\pgfpathlineto{\pgfqpoint{6.346596in}{4.602261in}}%
\pgfpathlineto{\pgfqpoint{6.347670in}{4.598439in}}%
\pgfpathlineto{\pgfqpoint{6.348744in}{4.597409in}}%
\pgfpathlineto{\pgfqpoint{6.349818in}{4.591655in}}%
\pgfpathlineto{\pgfqpoint{6.350891in}{4.590668in}}%
\pgfpathlineto{\pgfqpoint{6.354113in}{4.592256in}}%
\pgfpathlineto{\pgfqpoint{6.355187in}{4.591698in}}%
\pgfpathlineto{\pgfqpoint{6.356261in}{4.589594in}}%
\pgfpathlineto{\pgfqpoint{6.358408in}{4.591398in}}%
\pgfpathlineto{\pgfqpoint{6.361630in}{4.592471in}}%
\pgfpathlineto{\pgfqpoint{6.362704in}{4.594403in}}%
\pgfpathlineto{\pgfqpoint{6.363777in}{4.591956in}}%
\pgfpathlineto{\pgfqpoint{6.365925in}{4.590496in}}%
\pgfpathlineto{\pgfqpoint{6.369147in}{4.590453in}}%
\pgfpathlineto{\pgfqpoint{6.371294in}{4.603592in}}%
\pgfpathlineto{\pgfqpoint{6.372368in}{4.608229in}}%
\pgfpathlineto{\pgfqpoint{6.373442in}{4.608873in}}%
\pgfpathlineto{\pgfqpoint{6.377737in}{4.612394in}}%
\pgfpathlineto{\pgfqpoint{6.378811in}{4.610333in}}%
\pgfpathlineto{\pgfqpoint{6.379885in}{4.611879in}}%
\pgfpathlineto{\pgfqpoint{6.380959in}{4.611750in}}%
\pgfpathlineto{\pgfqpoint{6.384180in}{4.613639in}}%
\pgfpathlineto{\pgfqpoint{6.385254in}{4.615185in}}%
\pgfpathlineto{\pgfqpoint{6.386328in}{4.608530in}}%
\pgfpathlineto{\pgfqpoint{6.387402in}{4.605867in}}%
\pgfpathlineto{\pgfqpoint{6.388476in}{4.611621in}}%
\pgfpathlineto{\pgfqpoint{6.391697in}{4.616473in}}%
\pgfpathlineto{\pgfqpoint{6.393845in}{4.611578in}}%
\pgfpathlineto{\pgfqpoint{6.394919in}{4.611535in}}%
\pgfpathlineto{\pgfqpoint{6.395993in}{4.612523in}}%
\pgfpathlineto{\pgfqpoint{6.400288in}{4.611836in}}%
\pgfpathlineto{\pgfqpoint{6.402436in}{4.616559in}}%
\pgfpathlineto{\pgfqpoint{6.403510in}{4.614927in}}%
\pgfpathlineto{\pgfqpoint{6.403510in}{4.614927in}}%
\pgfusepath{stroke}%
\end{pgfscope}%
\begin{pgfscope}%
\pgfpathrectangle{\pgfqpoint{3.937600in}{4.233896in}}{\pgfqpoint{2.583333in}{0.400885in}}%
\pgfusepath{clip}%
\pgfsetroundcap%
\pgfsetroundjoin%
\pgfsetlinewidth{1.505625pt}%
\definecolor{currentstroke}{rgb}{1.000000,0.498039,0.054902}%
\pgfsetstrokecolor{currentstroke}%
\pgfsetdash{}{0pt}%
\pgfpathmoveto{\pgfqpoint{4.055025in}{4.384956in}}%
\pgfpathlineto{\pgfqpoint{4.056098in}{4.384956in}}%
\pgfpathlineto{\pgfqpoint{4.057172in}{4.382810in}}%
\pgfpathlineto{\pgfqpoint{4.061468in}{4.382810in}}%
\pgfpathlineto{\pgfqpoint{4.063615in}{4.380708in}}%
\pgfpathlineto{\pgfqpoint{4.064689in}{4.378104in}}%
\pgfpathlineto{\pgfqpoint{4.070058in}{4.376064in}}%
\pgfpathlineto{\pgfqpoint{4.072206in}{4.373492in}}%
\pgfpathlineto{\pgfqpoint{4.085092in}{4.373492in}}%
\pgfpathlineto{\pgfqpoint{4.087240in}{4.369964in}}%
\pgfpathlineto{\pgfqpoint{4.088314in}{4.366319in}}%
\pgfpathlineto{\pgfqpoint{4.091535in}{4.367744in}}%
\pgfpathlineto{\pgfqpoint{4.092609in}{4.366693in}}%
\pgfpathlineto{\pgfqpoint{4.093683in}{4.368268in}}%
\pgfpathlineto{\pgfqpoint{4.094757in}{4.366079in}}%
\pgfpathlineto{\pgfqpoint{4.095831in}{4.367679in}}%
\pgfpathlineto{\pgfqpoint{4.099052in}{4.366846in}}%
\pgfpathlineto{\pgfqpoint{4.101200in}{4.367169in}}%
\pgfpathlineto{\pgfqpoint{4.102274in}{4.362772in}}%
\pgfpathlineto{\pgfqpoint{4.109790in}{4.363394in}}%
\pgfpathlineto{\pgfqpoint{4.110864in}{4.361309in}}%
\pgfpathlineto{\pgfqpoint{4.114086in}{4.358735in}}%
\pgfpathlineto{\pgfqpoint{4.115160in}{4.359982in}}%
\pgfpathlineto{\pgfqpoint{4.116233in}{4.362582in}}%
\pgfpathlineto{\pgfqpoint{4.117307in}{4.360482in}}%
\pgfpathlineto{\pgfqpoint{4.118381in}{4.362227in}}%
\pgfpathlineto{\pgfqpoint{4.123750in}{4.362295in}}%
\pgfpathlineto{\pgfqpoint{4.124824in}{4.360174in}}%
\pgfpathlineto{\pgfqpoint{4.125898in}{4.359395in}}%
\pgfpathlineto{\pgfqpoint{4.129120in}{4.359395in}}%
\pgfpathlineto{\pgfqpoint{4.132341in}{4.347904in}}%
\pgfpathlineto{\pgfqpoint{4.133415in}{4.348343in}}%
\pgfpathlineto{\pgfqpoint{4.136636in}{4.346462in}}%
\pgfpathlineto{\pgfqpoint{4.137710in}{4.347379in}}%
\pgfpathlineto{\pgfqpoint{4.140932in}{4.346510in}}%
\pgfpathlineto{\pgfqpoint{4.144153in}{4.342867in}}%
\pgfpathlineto{\pgfqpoint{4.145227in}{4.343932in}}%
\pgfpathlineto{\pgfqpoint{4.146301in}{4.341854in}}%
\pgfpathlineto{\pgfqpoint{4.147375in}{4.344681in}}%
\pgfpathlineto{\pgfqpoint{4.148449in}{4.344765in}}%
\pgfpathlineto{\pgfqpoint{4.151670in}{4.344372in}}%
\pgfpathlineto{\pgfqpoint{4.152744in}{4.342922in}}%
\pgfpathlineto{\pgfqpoint{4.153818in}{4.345437in}}%
\pgfpathlineto{\pgfqpoint{4.154892in}{4.343568in}}%
\pgfpathlineto{\pgfqpoint{4.159187in}{4.345914in}}%
\pgfpathlineto{\pgfqpoint{4.160261in}{4.348423in}}%
\pgfpathlineto{\pgfqpoint{4.162409in}{4.343574in}}%
\pgfpathlineto{\pgfqpoint{4.163482in}{4.345453in}}%
\pgfpathlineto{\pgfqpoint{4.168852in}{4.343530in}}%
\pgfpathlineto{\pgfqpoint{4.169925in}{4.344691in}}%
\pgfpathlineto{\pgfqpoint{4.170999in}{4.344999in}}%
\pgfpathlineto{\pgfqpoint{4.174221in}{4.345309in}}%
\pgfpathlineto{\pgfqpoint{4.175295in}{4.344545in}}%
\pgfpathlineto{\pgfqpoint{4.177442in}{4.339661in}}%
\pgfpathlineto{\pgfqpoint{4.178516in}{4.338300in}}%
\pgfpathlineto{\pgfqpoint{4.181738in}{4.338197in}}%
\pgfpathlineto{\pgfqpoint{4.182811in}{4.336454in}}%
\pgfpathlineto{\pgfqpoint{4.183885in}{4.336254in}}%
\pgfpathlineto{\pgfqpoint{4.184959in}{4.336702in}}%
\pgfpathlineto{\pgfqpoint{4.186033in}{4.338409in}}%
\pgfpathlineto{\pgfqpoint{4.190328in}{4.338898in}}%
\pgfpathlineto{\pgfqpoint{4.191402in}{4.339908in}}%
\pgfpathlineto{\pgfqpoint{4.192476in}{4.339987in}}%
\pgfpathlineto{\pgfqpoint{4.193550in}{4.339461in}}%
\pgfpathlineto{\pgfqpoint{4.196771in}{4.342334in}}%
\pgfpathlineto{\pgfqpoint{4.246168in}{4.342334in}}%
\pgfpathlineto{\pgfqpoint{4.249389in}{4.340286in}}%
\pgfpathlineto{\pgfqpoint{4.250463in}{4.338932in}}%
\pgfpathlineto{\pgfqpoint{4.252611in}{4.339297in}}%
\pgfpathlineto{\pgfqpoint{4.253685in}{4.340788in}}%
\pgfpathlineto{\pgfqpoint{4.260128in}{4.341350in}}%
\pgfpathlineto{\pgfqpoint{4.261202in}{4.341350in}}%
\pgfpathlineto{\pgfqpoint{4.265497in}{4.339506in}}%
\pgfpathlineto{\pgfqpoint{4.275162in}{4.339506in}}%
\pgfpathlineto{\pgfqpoint{4.276235in}{4.337650in}}%
\pgfpathlineto{\pgfqpoint{4.326706in}{4.337650in}}%
\pgfpathlineto{\pgfqpoint{4.327780in}{4.333369in}}%
\pgfpathlineto{\pgfqpoint{4.328853in}{4.332866in}}%
\pgfpathlineto{\pgfqpoint{4.340666in}{4.333066in}}%
\pgfpathlineto{\pgfqpoint{4.351404in}{4.333066in}}%
\pgfpathlineto{\pgfqpoint{4.356773in}{4.332486in}}%
\pgfpathlineto{\pgfqpoint{4.363216in}{4.332486in}}%
\pgfpathlineto{\pgfqpoint{4.364290in}{4.330806in}}%
\pgfpathlineto{\pgfqpoint{4.444828in}{4.330806in}}%
\pgfpathlineto{\pgfqpoint{4.446976in}{4.327421in}}%
\pgfpathlineto{\pgfqpoint{4.448050in}{4.326894in}}%
\pgfpathlineto{\pgfqpoint{4.449123in}{4.325561in}}%
\pgfpathlineto{\pgfqpoint{4.454493in}{4.324195in}}%
\pgfpathlineto{\pgfqpoint{4.456640in}{4.322182in}}%
\pgfpathlineto{\pgfqpoint{4.459862in}{4.322248in}}%
\pgfpathlineto{\pgfqpoint{4.462009in}{4.323423in}}%
\pgfpathlineto{\pgfqpoint{4.463083in}{4.323175in}}%
\pgfpathlineto{\pgfqpoint{4.464157in}{4.325150in}}%
\pgfpathlineto{\pgfqpoint{4.468452in}{4.325937in}}%
\pgfpathlineto{\pgfqpoint{4.469526in}{4.326946in}}%
\pgfpathlineto{\pgfqpoint{4.471674in}{4.325712in}}%
\pgfpathlineto{\pgfqpoint{4.474896in}{4.326624in}}%
\pgfpathlineto{\pgfqpoint{4.475969in}{4.325818in}}%
\pgfpathlineto{\pgfqpoint{4.478117in}{4.326075in}}%
\pgfpathlineto{\pgfqpoint{4.479191in}{4.323703in}}%
\pgfpathlineto{\pgfqpoint{4.482412in}{4.324732in}}%
\pgfpathlineto{\pgfqpoint{4.483486in}{4.322107in}}%
\pgfpathlineto{\pgfqpoint{4.484560in}{4.322307in}}%
\pgfpathlineto{\pgfqpoint{4.485634in}{4.319089in}}%
\pgfpathlineto{\pgfqpoint{4.486708in}{4.319747in}}%
\pgfpathlineto{\pgfqpoint{4.493151in}{4.318700in}}%
\pgfpathlineto{\pgfqpoint{4.494225in}{4.319944in}}%
\pgfpathlineto{\pgfqpoint{4.498520in}{4.318782in}}%
\pgfpathlineto{\pgfqpoint{4.500668in}{4.320204in}}%
\pgfpathlineto{\pgfqpoint{4.501741in}{4.318194in}}%
\pgfpathlineto{\pgfqpoint{4.506037in}{4.319431in}}%
\pgfpathlineto{\pgfqpoint{4.507111in}{4.318300in}}%
\pgfpathlineto{\pgfqpoint{4.508185in}{4.318970in}}%
\pgfpathlineto{\pgfqpoint{4.512480in}{4.317558in}}%
\pgfpathlineto{\pgfqpoint{4.513554in}{4.315240in}}%
\pgfpathlineto{\pgfqpoint{4.514628in}{4.314265in}}%
\pgfpathlineto{\pgfqpoint{4.516775in}{4.314186in}}%
\pgfpathlineto{\pgfqpoint{4.519997in}{4.312760in}}%
\pgfpathlineto{\pgfqpoint{4.522144in}{4.313103in}}%
\pgfpathlineto{\pgfqpoint{4.523218in}{4.312969in}}%
\pgfpathlineto{\pgfqpoint{4.524292in}{4.311725in}}%
\pgfpathlineto{\pgfqpoint{4.527514in}{4.312268in}}%
\pgfpathlineto{\pgfqpoint{4.528587in}{4.313383in}}%
\pgfpathlineto{\pgfqpoint{4.529661in}{4.311860in}}%
\pgfpathlineto{\pgfqpoint{4.530735in}{4.312855in}}%
\pgfpathlineto{\pgfqpoint{4.531809in}{4.311461in}}%
\pgfpathlineto{\pgfqpoint{4.535030in}{4.311647in}}%
\pgfpathlineto{\pgfqpoint{4.536104in}{4.309833in}}%
\pgfpathlineto{\pgfqpoint{4.542547in}{4.309685in}}%
\pgfpathlineto{\pgfqpoint{4.543621in}{4.309053in}}%
\pgfpathlineto{\pgfqpoint{4.544695in}{4.310999in}}%
\pgfpathlineto{\pgfqpoint{4.545769in}{4.310185in}}%
\pgfpathlineto{\pgfqpoint{4.546843in}{4.312574in}}%
\pgfpathlineto{\pgfqpoint{4.550064in}{4.312041in}}%
\pgfpathlineto{\pgfqpoint{4.551138in}{4.312701in}}%
\pgfpathlineto{\pgfqpoint{4.554360in}{4.311901in}}%
\pgfpathlineto{\pgfqpoint{4.557581in}{4.314609in}}%
\pgfpathlineto{\pgfqpoint{4.558655in}{4.313721in}}%
\pgfpathlineto{\pgfqpoint{4.559729in}{4.314538in}}%
\pgfpathlineto{\pgfqpoint{4.560803in}{4.312901in}}%
\pgfpathlineto{\pgfqpoint{4.561876in}{4.309069in}}%
\pgfpathlineto{\pgfqpoint{4.565098in}{4.310001in}}%
\pgfpathlineto{\pgfqpoint{4.566172in}{4.308689in}}%
\pgfpathlineto{\pgfqpoint{4.567246in}{4.308742in}}%
\pgfpathlineto{\pgfqpoint{4.568319in}{4.307585in}}%
\pgfpathlineto{\pgfqpoint{4.569393in}{4.308195in}}%
\pgfpathlineto{\pgfqpoint{4.572615in}{4.308389in}}%
\pgfpathlineto{\pgfqpoint{4.573689in}{4.307132in}}%
\pgfpathlineto{\pgfqpoint{4.574763in}{4.307340in}}%
\pgfpathlineto{\pgfqpoint{4.576910in}{4.304292in}}%
\pgfpathlineto{\pgfqpoint{4.587649in}{4.304949in}}%
\pgfpathlineto{\pgfqpoint{4.589796in}{4.300507in}}%
\pgfpathlineto{\pgfqpoint{4.590870in}{4.301270in}}%
\pgfpathlineto{\pgfqpoint{4.591944in}{4.299721in}}%
\pgfpathlineto{\pgfqpoint{4.596239in}{4.297299in}}%
\pgfpathlineto{\pgfqpoint{4.597313in}{4.298173in}}%
\pgfpathlineto{\pgfqpoint{4.605904in}{4.295940in}}%
\pgfpathlineto{\pgfqpoint{4.606978in}{4.296483in}}%
\pgfpathlineto{\pgfqpoint{4.610199in}{4.295511in}}%
\pgfpathlineto{\pgfqpoint{4.611273in}{4.296022in}}%
\pgfpathlineto{\pgfqpoint{4.612347in}{4.297683in}}%
\pgfpathlineto{\pgfqpoint{4.614495in}{4.293457in}}%
\pgfpathlineto{\pgfqpoint{4.617716in}{4.293166in}}%
\pgfpathlineto{\pgfqpoint{4.619864in}{4.297542in}}%
\pgfpathlineto{\pgfqpoint{4.620938in}{4.296893in}}%
\pgfpathlineto{\pgfqpoint{4.622011in}{4.299798in}}%
\pgfpathlineto{\pgfqpoint{4.625233in}{4.298604in}}%
\pgfpathlineto{\pgfqpoint{4.626307in}{4.297069in}}%
\pgfpathlineto{\pgfqpoint{4.627381in}{4.298042in}}%
\pgfpathlineto{\pgfqpoint{4.628454in}{4.299794in}}%
\pgfpathlineto{\pgfqpoint{4.629528in}{4.299249in}}%
\pgfpathlineto{\pgfqpoint{4.632750in}{4.301029in}}%
\pgfpathlineto{\pgfqpoint{4.635971in}{4.296746in}}%
\pgfpathlineto{\pgfqpoint{4.637045in}{4.297216in}}%
\pgfpathlineto{\pgfqpoint{4.640267in}{4.296066in}}%
\pgfpathlineto{\pgfqpoint{4.641340in}{4.297077in}}%
\pgfpathlineto{\pgfqpoint{4.642414in}{4.297135in}}%
\pgfpathlineto{\pgfqpoint{4.644562in}{4.294853in}}%
\pgfpathlineto{\pgfqpoint{4.648857in}{4.293236in}}%
\pgfpathlineto{\pgfqpoint{4.649931in}{4.294785in}}%
\pgfpathlineto{\pgfqpoint{4.651005in}{4.294033in}}%
\pgfpathlineto{\pgfqpoint{4.652079in}{4.292308in}}%
\pgfpathlineto{\pgfqpoint{4.655300in}{4.292880in}}%
\pgfpathlineto{\pgfqpoint{4.656374in}{4.292380in}}%
\pgfpathlineto{\pgfqpoint{4.657448in}{4.294124in}}%
\pgfpathlineto{\pgfqpoint{4.658522in}{4.297582in}}%
\pgfpathlineto{\pgfqpoint{4.659596in}{4.297509in}}%
\pgfpathlineto{\pgfqpoint{4.662817in}{4.296639in}}%
\pgfpathlineto{\pgfqpoint{4.663891in}{4.297038in}}%
\pgfpathlineto{\pgfqpoint{4.666039in}{4.295269in}}%
\pgfpathlineto{\pgfqpoint{4.667113in}{4.295810in}}%
\pgfpathlineto{\pgfqpoint{4.671408in}{4.296005in}}%
\pgfpathlineto{\pgfqpoint{4.672482in}{4.297834in}}%
\pgfpathlineto{\pgfqpoint{4.673556in}{4.295350in}}%
\pgfpathlineto{\pgfqpoint{4.678925in}{4.295113in}}%
\pgfpathlineto{\pgfqpoint{4.679999in}{4.295956in}}%
\pgfpathlineto{\pgfqpoint{4.681073in}{4.294605in}}%
\pgfpathlineto{\pgfqpoint{4.682146in}{4.295467in}}%
\pgfpathlineto{\pgfqpoint{4.685368in}{4.295481in}}%
\pgfpathlineto{\pgfqpoint{4.686442in}{4.294547in}}%
\pgfpathlineto{\pgfqpoint{4.687516in}{4.294985in}}%
\pgfpathlineto{\pgfqpoint{4.688589in}{4.296240in}}%
\pgfpathlineto{\pgfqpoint{4.689663in}{4.295872in}}%
\pgfpathlineto{\pgfqpoint{4.697180in}{4.297838in}}%
\pgfpathlineto{\pgfqpoint{4.700402in}{4.297882in}}%
\pgfpathlineto{\pgfqpoint{4.701475in}{4.300167in}}%
\pgfpathlineto{\pgfqpoint{4.708992in}{4.299443in}}%
\pgfpathlineto{\pgfqpoint{4.711140in}{4.297555in}}%
\pgfpathlineto{\pgfqpoint{4.712214in}{4.298152in}}%
\pgfpathlineto{\pgfqpoint{4.715435in}{4.297740in}}%
\pgfpathlineto{\pgfqpoint{4.717583in}{4.295727in}}%
\pgfpathlineto{\pgfqpoint{4.722952in}{4.295181in}}%
\pgfpathlineto{\pgfqpoint{4.725100in}{4.292618in}}%
\pgfpathlineto{\pgfqpoint{4.727248in}{4.293013in}}%
\pgfpathlineto{\pgfqpoint{4.732617in}{4.294644in}}%
\pgfpathlineto{\pgfqpoint{4.733691in}{4.294219in}}%
\pgfpathlineto{\pgfqpoint{4.734764in}{4.294763in}}%
\pgfpathlineto{\pgfqpoint{4.737986in}{4.295231in}}%
\pgfpathlineto{\pgfqpoint{4.739060in}{4.294703in}}%
\pgfpathlineto{\pgfqpoint{4.741207in}{4.296841in}}%
\pgfpathlineto{\pgfqpoint{4.742281in}{4.296467in}}%
\pgfpathlineto{\pgfqpoint{4.747651in}{4.299241in}}%
\pgfpathlineto{\pgfqpoint{4.748724in}{4.295860in}}%
\pgfpathlineto{\pgfqpoint{4.749798in}{4.294901in}}%
\pgfpathlineto{\pgfqpoint{4.753020in}{4.294043in}}%
\pgfpathlineto{\pgfqpoint{4.754094in}{4.295062in}}%
\pgfpathlineto{\pgfqpoint{4.755167in}{4.293702in}}%
\pgfpathlineto{\pgfqpoint{4.756241in}{4.288858in}}%
\pgfpathlineto{\pgfqpoint{4.757315in}{4.288529in}}%
\pgfpathlineto{\pgfqpoint{4.760537in}{4.288662in}}%
\pgfpathlineto{\pgfqpoint{4.761610in}{4.288141in}}%
\pgfpathlineto{\pgfqpoint{4.763758in}{4.288103in}}%
\pgfpathlineto{\pgfqpoint{4.764832in}{4.286223in}}%
\pgfpathlineto{\pgfqpoint{4.768053in}{4.286074in}}%
\pgfpathlineto{\pgfqpoint{4.769127in}{4.285125in}}%
\pgfpathlineto{\pgfqpoint{4.770201in}{4.285696in}}%
\pgfpathlineto{\pgfqpoint{4.771275in}{4.287057in}}%
\pgfpathlineto{\pgfqpoint{4.772349in}{4.286670in}}%
\pgfpathlineto{\pgfqpoint{4.777718in}{4.286704in}}%
\pgfpathlineto{\pgfqpoint{4.778792in}{4.287993in}}%
\pgfpathlineto{\pgfqpoint{4.779866in}{4.287048in}}%
\pgfpathlineto{\pgfqpoint{4.785235in}{4.287407in}}%
\pgfpathlineto{\pgfqpoint{4.787383in}{4.285942in}}%
\pgfpathlineto{\pgfqpoint{4.792752in}{4.286513in}}%
\pgfpathlineto{\pgfqpoint{4.793826in}{4.284709in}}%
\pgfpathlineto{\pgfqpoint{4.794899in}{4.284809in}}%
\pgfpathlineto{\pgfqpoint{4.802416in}{4.282850in}}%
\pgfpathlineto{\pgfqpoint{4.808859in}{4.284092in}}%
\pgfpathlineto{\pgfqpoint{4.809933in}{4.282691in}}%
\pgfpathlineto{\pgfqpoint{4.814228in}{4.283318in}}%
\pgfpathlineto{\pgfqpoint{4.816376in}{4.285304in}}%
\pgfpathlineto{\pgfqpoint{4.820672in}{4.284051in}}%
\pgfpathlineto{\pgfqpoint{4.821745in}{4.284466in}}%
\pgfpathlineto{\pgfqpoint{4.822819in}{4.282555in}}%
\pgfpathlineto{\pgfqpoint{4.823893in}{4.282143in}}%
\pgfpathlineto{\pgfqpoint{4.824967in}{4.281034in}}%
\pgfpathlineto{\pgfqpoint{4.836779in}{4.278161in}}%
\pgfpathlineto{\pgfqpoint{4.838927in}{4.279271in}}%
\pgfpathlineto{\pgfqpoint{4.840001in}{4.279007in}}%
\pgfpathlineto{\pgfqpoint{4.846444in}{4.279569in}}%
\pgfpathlineto{\pgfqpoint{4.851813in}{4.281167in}}%
\pgfpathlineto{\pgfqpoint{4.852887in}{4.280103in}}%
\pgfpathlineto{\pgfqpoint{4.853961in}{4.280541in}}%
\pgfpathlineto{\pgfqpoint{4.855034in}{4.277580in}}%
\pgfpathlineto{\pgfqpoint{4.860404in}{4.277436in}}%
\pgfpathlineto{\pgfqpoint{4.862551in}{4.281112in}}%
\pgfpathlineto{\pgfqpoint{4.865773in}{4.282269in}}%
\pgfpathlineto{\pgfqpoint{4.866847in}{4.281385in}}%
\pgfpathlineto{\pgfqpoint{4.867920in}{4.282416in}}%
\pgfpathlineto{\pgfqpoint{4.868994in}{4.281380in}}%
\pgfpathlineto{\pgfqpoint{4.870068in}{4.282896in}}%
\pgfpathlineto{\pgfqpoint{4.873290in}{4.285025in}}%
\pgfpathlineto{\pgfqpoint{4.874363in}{4.283843in}}%
\pgfpathlineto{\pgfqpoint{4.875437in}{4.284128in}}%
\pgfpathlineto{\pgfqpoint{4.877585in}{4.280859in}}%
\pgfpathlineto{\pgfqpoint{4.882954in}{4.278982in}}%
\pgfpathlineto{\pgfqpoint{4.885102in}{4.278992in}}%
\pgfpathlineto{\pgfqpoint{4.892619in}{4.279215in}}%
\pgfpathlineto{\pgfqpoint{4.896914in}{4.278176in}}%
\pgfpathlineto{\pgfqpoint{4.897988in}{4.278329in}}%
\pgfpathlineto{\pgfqpoint{4.900136in}{4.276969in}}%
\pgfpathlineto{\pgfqpoint{4.903357in}{4.278090in}}%
\pgfpathlineto{\pgfqpoint{4.904431in}{4.275774in}}%
\pgfpathlineto{\pgfqpoint{4.905505in}{4.276160in}}%
\pgfpathlineto{\pgfqpoint{4.906579in}{4.275007in}}%
\pgfpathlineto{\pgfqpoint{4.907652in}{4.274734in}}%
\pgfpathlineto{\pgfqpoint{4.913022in}{4.275508in}}%
\pgfpathlineto{\pgfqpoint{4.914095in}{4.277330in}}%
\pgfpathlineto{\pgfqpoint{4.915169in}{4.277779in}}%
\pgfpathlineto{\pgfqpoint{4.919465in}{4.276545in}}%
\pgfpathlineto{\pgfqpoint{4.920539in}{4.277270in}}%
\pgfpathlineto{\pgfqpoint{4.921612in}{4.276447in}}%
\pgfpathlineto{\pgfqpoint{4.926982in}{4.277165in}}%
\pgfpathlineto{\pgfqpoint{4.928055in}{4.278182in}}%
\pgfpathlineto{\pgfqpoint{4.930203in}{4.277477in}}%
\pgfpathlineto{\pgfqpoint{4.933425in}{4.277853in}}%
\pgfpathlineto{\pgfqpoint{4.934498in}{4.276856in}}%
\pgfpathlineto{\pgfqpoint{4.935572in}{4.277310in}}%
\pgfpathlineto{\pgfqpoint{4.936646in}{4.276813in}}%
\pgfpathlineto{\pgfqpoint{4.937720in}{4.278358in}}%
\pgfpathlineto{\pgfqpoint{4.942015in}{4.280741in}}%
\pgfpathlineto{\pgfqpoint{4.943089in}{4.278635in}}%
\pgfpathlineto{\pgfqpoint{4.944163in}{4.281654in}}%
\pgfpathlineto{\pgfqpoint{4.945237in}{4.282440in}}%
\pgfpathlineto{\pgfqpoint{4.949532in}{4.280979in}}%
\pgfpathlineto{\pgfqpoint{4.950606in}{4.279680in}}%
\pgfpathlineto{\pgfqpoint{4.951680in}{4.280768in}}%
\pgfpathlineto{\pgfqpoint{4.965640in}{4.279632in}}%
\pgfpathlineto{\pgfqpoint{4.966714in}{4.280191in}}%
\pgfpathlineto{\pgfqpoint{4.971009in}{4.279686in}}%
\pgfpathlineto{\pgfqpoint{4.972083in}{4.280755in}}%
\pgfpathlineto{\pgfqpoint{4.973157in}{4.279072in}}%
\pgfpathlineto{\pgfqpoint{4.975304in}{4.278288in}}%
\pgfpathlineto{\pgfqpoint{4.978526in}{4.277564in}}%
\pgfpathlineto{\pgfqpoint{4.982821in}{4.279481in}}%
\pgfpathlineto{\pgfqpoint{4.986043in}{4.278914in}}%
\pgfpathlineto{\pgfqpoint{4.987116in}{4.279846in}}%
\pgfpathlineto{\pgfqpoint{4.990338in}{4.278299in}}%
\pgfpathlineto{\pgfqpoint{4.994633in}{4.275971in}}%
\pgfpathlineto{\pgfqpoint{4.996781in}{4.276060in}}%
\pgfpathlineto{\pgfqpoint{5.002150in}{4.275686in}}%
\pgfpathlineto{\pgfqpoint{5.003224in}{4.275613in}}%
\pgfpathlineto{\pgfqpoint{5.004298in}{4.274792in}}%
\pgfpathlineto{\pgfqpoint{5.005372in}{4.273070in}}%
\pgfpathlineto{\pgfqpoint{5.009667in}{4.272780in}}%
\pgfpathlineto{\pgfqpoint{5.016110in}{4.273487in}}%
\pgfpathlineto{\pgfqpoint{5.020405in}{4.272565in}}%
\pgfpathlineto{\pgfqpoint{5.023627in}{4.272590in}}%
\pgfpathlineto{\pgfqpoint{5.024701in}{4.273415in}}%
\pgfpathlineto{\pgfqpoint{5.027922in}{4.273021in}}%
\pgfpathlineto{\pgfqpoint{5.033292in}{4.272800in}}%
\pgfpathlineto{\pgfqpoint{5.034365in}{4.272314in}}%
\pgfpathlineto{\pgfqpoint{5.048325in}{4.273166in}}%
\pgfpathlineto{\pgfqpoint{5.049399in}{4.274326in}}%
\pgfpathlineto{\pgfqpoint{5.050473in}{4.273890in}}%
\pgfpathlineto{\pgfqpoint{5.053694in}{4.274408in}}%
\pgfpathlineto{\pgfqpoint{5.055842in}{4.274083in}}%
\pgfpathlineto{\pgfqpoint{5.056916in}{4.274187in}}%
\pgfpathlineto{\pgfqpoint{5.057990in}{4.275171in}}%
\pgfpathlineto{\pgfqpoint{5.062285in}{4.275350in}}%
\pgfpathlineto{\pgfqpoint{5.063359in}{4.276014in}}%
\pgfpathlineto{\pgfqpoint{5.065507in}{4.279854in}}%
\pgfpathlineto{\pgfqpoint{5.068728in}{4.279431in}}%
\pgfpathlineto{\pgfqpoint{5.070876in}{4.279920in}}%
\pgfpathlineto{\pgfqpoint{5.071950in}{4.280273in}}%
\pgfpathlineto{\pgfqpoint{5.073024in}{4.278908in}}%
\pgfpathlineto{\pgfqpoint{5.077319in}{4.279261in}}%
\pgfpathlineto{\pgfqpoint{5.079467in}{4.279081in}}%
\pgfpathlineto{\pgfqpoint{5.080540in}{4.279695in}}%
\pgfpathlineto{\pgfqpoint{5.083762in}{4.279172in}}%
\pgfpathlineto{\pgfqpoint{5.085910in}{4.277959in}}%
\pgfpathlineto{\pgfqpoint{5.086983in}{4.277382in}}%
\pgfpathlineto{\pgfqpoint{5.088057in}{4.277609in}}%
\pgfpathlineto{\pgfqpoint{5.091279in}{4.276980in}}%
\pgfpathlineto{\pgfqpoint{5.093427in}{4.277289in}}%
\pgfpathlineto{\pgfqpoint{5.094500in}{4.277478in}}%
\pgfpathlineto{\pgfqpoint{5.095574in}{4.277012in}}%
\pgfpathlineto{\pgfqpoint{5.103091in}{4.276961in}}%
\pgfpathlineto{\pgfqpoint{5.106313in}{4.277543in}}%
\pgfpathlineto{\pgfqpoint{5.107386in}{4.278457in}}%
\pgfpathlineto{\pgfqpoint{5.109534in}{4.277989in}}%
\pgfpathlineto{\pgfqpoint{5.110608in}{4.278683in}}%
\pgfpathlineto{\pgfqpoint{5.113829in}{4.278918in}}%
\pgfpathlineto{\pgfqpoint{5.117051in}{4.276482in}}%
\pgfpathlineto{\pgfqpoint{5.122420in}{4.278042in}}%
\pgfpathlineto{\pgfqpoint{5.123494in}{4.277849in}}%
\pgfpathlineto{\pgfqpoint{5.124568in}{4.279415in}}%
\pgfpathlineto{\pgfqpoint{5.125642in}{4.277925in}}%
\pgfpathlineto{\pgfqpoint{5.129937in}{4.278671in}}%
\pgfpathlineto{\pgfqpoint{5.131011in}{4.280015in}}%
\pgfpathlineto{\pgfqpoint{5.132085in}{4.279954in}}%
\pgfpathlineto{\pgfqpoint{5.133159in}{4.278738in}}%
\pgfpathlineto{\pgfqpoint{5.136380in}{4.278856in}}%
\pgfpathlineto{\pgfqpoint{5.137454in}{4.280395in}}%
\pgfpathlineto{\pgfqpoint{5.138528in}{4.278436in}}%
\pgfpathlineto{\pgfqpoint{5.139602in}{4.279814in}}%
\pgfpathlineto{\pgfqpoint{5.151414in}{4.279814in}}%
\pgfpathlineto{\pgfqpoint{5.152488in}{4.278242in}}%
\pgfpathlineto{\pgfqpoint{5.153561in}{4.278242in}}%
\pgfpathlineto{\pgfqpoint{5.155709in}{4.275977in}}%
\pgfpathlineto{\pgfqpoint{5.158931in}{4.275766in}}%
\pgfpathlineto{\pgfqpoint{5.160004in}{4.274546in}}%
\pgfpathlineto{\pgfqpoint{5.162152in}{4.273931in}}%
\pgfpathlineto{\pgfqpoint{5.163226in}{4.272905in}}%
\pgfpathlineto{\pgfqpoint{5.167521in}{4.271738in}}%
\pgfpathlineto{\pgfqpoint{5.168595in}{4.270917in}}%
\pgfpathlineto{\pgfqpoint{5.169669in}{4.271569in}}%
\pgfpathlineto{\pgfqpoint{5.170743in}{4.271033in}}%
\pgfpathlineto{\pgfqpoint{5.175038in}{4.271434in}}%
\pgfpathlineto{\pgfqpoint{5.185777in}{4.272502in}}%
\pgfpathlineto{\pgfqpoint{5.197589in}{4.270426in}}%
\pgfpathlineto{\pgfqpoint{5.198663in}{4.271772in}}%
\pgfpathlineto{\pgfqpoint{5.204032in}{4.269958in}}%
\pgfpathlineto{\pgfqpoint{5.207253in}{4.270282in}}%
\pgfpathlineto{\pgfqpoint{5.208327in}{4.271911in}}%
\pgfpathlineto{\pgfqpoint{5.211549in}{4.272678in}}%
\pgfpathlineto{\pgfqpoint{5.212623in}{4.274147in}}%
\pgfpathlineto{\pgfqpoint{5.214770in}{4.270066in}}%
\pgfpathlineto{\pgfqpoint{5.215844in}{4.270271in}}%
\pgfpathlineto{\pgfqpoint{5.223361in}{4.269217in}}%
\pgfpathlineto{\pgfqpoint{5.227656in}{4.269233in}}%
\pgfpathlineto{\pgfqpoint{5.228730in}{4.270147in}}%
\pgfpathlineto{\pgfqpoint{5.230878in}{4.270163in}}%
\pgfpathlineto{\pgfqpoint{5.234099in}{4.272046in}}%
\pgfpathlineto{\pgfqpoint{5.235173in}{4.273599in}}%
\pgfpathlineto{\pgfqpoint{5.237321in}{4.270939in}}%
\pgfpathlineto{\pgfqpoint{5.238395in}{4.271849in}}%
\pgfpathlineto{\pgfqpoint{5.244838in}{4.272812in}}%
\pgfpathlineto{\pgfqpoint{5.245912in}{4.272812in}}%
\pgfpathlineto{\pgfqpoint{5.253428in}{4.271424in}}%
\pgfpathlineto{\pgfqpoint{5.643232in}{4.271424in}}%
\pgfpathlineto{\pgfqpoint{5.644306in}{4.262671in}}%
\pgfpathlineto{\pgfqpoint{5.649675in}{4.262049in}}%
\pgfpathlineto{\pgfqpoint{5.650749in}{4.260157in}}%
\pgfpathlineto{\pgfqpoint{5.651823in}{4.260865in}}%
\pgfpathlineto{\pgfqpoint{5.655044in}{4.262256in}}%
\pgfpathlineto{\pgfqpoint{5.656118in}{4.261060in}}%
\pgfpathlineto{\pgfqpoint{5.658266in}{4.261890in}}%
\pgfpathlineto{\pgfqpoint{5.665782in}{4.258106in}}%
\pgfpathlineto{\pgfqpoint{5.666856in}{4.259900in}}%
\pgfpathlineto{\pgfqpoint{5.672225in}{4.260999in}}%
\pgfpathlineto{\pgfqpoint{5.674373in}{4.262268in}}%
\pgfpathlineto{\pgfqpoint{5.677595in}{4.263330in}}%
\pgfpathlineto{\pgfqpoint{5.679742in}{4.262183in}}%
\pgfpathlineto{\pgfqpoint{5.680816in}{4.263050in}}%
\pgfpathlineto{\pgfqpoint{5.685112in}{4.263269in}}%
\pgfpathlineto{\pgfqpoint{5.686185in}{4.264673in}}%
\pgfpathlineto{\pgfqpoint{5.687259in}{4.265052in}}%
\pgfpathlineto{\pgfqpoint{5.688333in}{4.266162in}}%
\pgfpathlineto{\pgfqpoint{6.077063in}{4.266162in}}%
\pgfpathlineto{\pgfqpoint{6.078136in}{4.264607in}}%
\pgfpathlineto{\pgfqpoint{6.200554in}{4.264331in}}%
\pgfpathlineto{\pgfqpoint{6.213440in}{4.262022in}}%
\pgfpathlineto{\pgfqpoint{6.214514in}{4.262877in}}%
\pgfpathlineto{\pgfqpoint{6.215588in}{4.262422in}}%
\pgfpathlineto{\pgfqpoint{6.220957in}{4.262057in}}%
\pgfpathlineto{\pgfqpoint{6.223105in}{4.261712in}}%
\pgfpathlineto{\pgfqpoint{6.235991in}{4.261687in}}%
\pgfpathlineto{\pgfqpoint{6.237065in}{4.262752in}}%
\pgfpathlineto{\pgfqpoint{6.238138in}{4.262033in}}%
\pgfpathlineto{\pgfqpoint{6.249951in}{4.261193in}}%
\pgfpathlineto{\pgfqpoint{6.251024in}{4.261371in}}%
\pgfpathlineto{\pgfqpoint{6.253172in}{4.262393in}}%
\pgfpathlineto{\pgfqpoint{6.256394in}{4.261512in}}%
\pgfpathlineto{\pgfqpoint{6.258541in}{4.260120in}}%
\pgfpathlineto{\pgfqpoint{6.260689in}{4.261537in}}%
\pgfpathlineto{\pgfqpoint{6.273575in}{4.261311in}}%
\pgfpathlineto{\pgfqpoint{6.275723in}{4.260990in}}%
\pgfpathlineto{\pgfqpoint{6.283240in}{4.262361in}}%
\pgfpathlineto{\pgfqpoint{6.288609in}{4.260594in}}%
\pgfpathlineto{\pgfqpoint{6.289683in}{4.260915in}}%
\pgfpathlineto{\pgfqpoint{6.290756in}{4.260339in}}%
\pgfpathlineto{\pgfqpoint{6.293978in}{4.259724in}}%
\pgfpathlineto{\pgfqpoint{6.295052in}{4.258943in}}%
\pgfpathlineto{\pgfqpoint{6.297199in}{4.259330in}}%
\pgfpathlineto{\pgfqpoint{6.302569in}{4.259146in}}%
\pgfpathlineto{\pgfqpoint{6.305790in}{4.257963in}}%
\pgfpathlineto{\pgfqpoint{6.309012in}{4.257909in}}%
\pgfpathlineto{\pgfqpoint{6.310086in}{4.257334in}}%
\pgfpathlineto{\pgfqpoint{6.311159in}{4.257603in}}%
\pgfpathlineto{\pgfqpoint{6.313307in}{4.257030in}}%
\pgfpathlineto{\pgfqpoint{6.319750in}{4.256990in}}%
\pgfpathlineto{\pgfqpoint{6.320824in}{4.256200in}}%
\pgfpathlineto{\pgfqpoint{6.327267in}{4.256791in}}%
\pgfpathlineto{\pgfqpoint{6.331562in}{4.256490in}}%
\pgfpathlineto{\pgfqpoint{6.332636in}{4.255573in}}%
\pgfpathlineto{\pgfqpoint{6.333710in}{4.255770in}}%
\pgfpathlineto{\pgfqpoint{6.334784in}{4.254459in}}%
\pgfpathlineto{\pgfqpoint{6.335858in}{4.254400in}}%
\pgfpathlineto{\pgfqpoint{6.340153in}{4.254556in}}%
\pgfpathlineto{\pgfqpoint{6.346596in}{4.254111in}}%
\pgfpathlineto{\pgfqpoint{6.350891in}{4.255722in}}%
\pgfpathlineto{\pgfqpoint{6.362704in}{4.255182in}}%
\pgfpathlineto{\pgfqpoint{6.365925in}{4.255738in}}%
\pgfpathlineto{\pgfqpoint{6.369147in}{4.255745in}}%
\pgfpathlineto{\pgfqpoint{6.371294in}{4.253874in}}%
\pgfpathlineto{\pgfqpoint{6.373442in}{4.253161in}}%
\pgfpathlineto{\pgfqpoint{6.380959in}{4.252779in}}%
\pgfpathlineto{\pgfqpoint{6.385254in}{4.252334in}}%
\pgfpathlineto{\pgfqpoint{6.387402in}{4.253537in}}%
\pgfpathlineto{\pgfqpoint{6.388476in}{4.252767in}}%
\pgfpathlineto{\pgfqpoint{6.392771in}{4.252420in}}%
\pgfpathlineto{\pgfqpoint{6.394919in}{4.252767in}}%
\pgfpathlineto{\pgfqpoint{6.395993in}{4.252639in}}%
\pgfpathlineto{\pgfqpoint{6.401362in}{4.252427in}}%
\pgfpathlineto{\pgfqpoint{6.403510in}{4.252325in}}%
\pgfpathlineto{\pgfqpoint{6.403510in}{4.252325in}}%
\pgfusepath{stroke}%
\end{pgfscope}%
\begin{pgfscope}%
\pgfsetrectcap%
\pgfsetmiterjoin%
\pgfsetlinewidth{0.803000pt}%
\definecolor{currentstroke}{rgb}{1.000000,1.000000,1.000000}%
\pgfsetstrokecolor{currentstroke}%
\pgfsetdash{}{0pt}%
\pgfpathmoveto{\pgfqpoint{3.937600in}{4.233896in}}%
\pgfpathlineto{\pgfqpoint{3.937600in}{4.634781in}}%
\pgfusepath{stroke}%
\end{pgfscope}%
\begin{pgfscope}%
\pgfsetrectcap%
\pgfsetmiterjoin%
\pgfsetlinewidth{0.803000pt}%
\definecolor{currentstroke}{rgb}{1.000000,1.000000,1.000000}%
\pgfsetstrokecolor{currentstroke}%
\pgfsetdash{}{0pt}%
\pgfpathmoveto{\pgfqpoint{6.520934in}{4.233896in}}%
\pgfpathlineto{\pgfqpoint{6.520934in}{4.634781in}}%
\pgfusepath{stroke}%
\end{pgfscope}%
\begin{pgfscope}%
\pgfsetrectcap%
\pgfsetmiterjoin%
\pgfsetlinewidth{0.803000pt}%
\definecolor{currentstroke}{rgb}{1.000000,1.000000,1.000000}%
\pgfsetstrokecolor{currentstroke}%
\pgfsetdash{}{0pt}%
\pgfpathmoveto{\pgfqpoint{3.937600in}{4.233896in}}%
\pgfpathlineto{\pgfqpoint{6.520934in}{4.233896in}}%
\pgfusepath{stroke}%
\end{pgfscope}%
\begin{pgfscope}%
\pgfsetrectcap%
\pgfsetmiterjoin%
\pgfsetlinewidth{0.803000pt}%
\definecolor{currentstroke}{rgb}{1.000000,1.000000,1.000000}%
\pgfsetstrokecolor{currentstroke}%
\pgfsetdash{}{0pt}%
\pgfpathmoveto{\pgfqpoint{3.937600in}{4.634781in}}%
\pgfpathlineto{\pgfqpoint{6.520934in}{4.634781in}}%
\pgfusepath{stroke}%
\end{pgfscope}%
\begin{pgfscope}%
\definecolor{textcolor}{rgb}{0.150000,0.150000,0.150000}%
\pgfsetstrokecolor{textcolor}%
\pgfsetfillcolor{textcolor}%
\pgftext[x=5.229267in,y=4.718114in,,base]{\color{textcolor}\rmfamily\fontsize{16.800000}{20.160000}\selectfont AXP}%
\end{pgfscope}%
\begin{pgfscope}%
\pgfsetbuttcap%
\pgfsetmiterjoin%
\definecolor{currentfill}{rgb}{0.917647,0.917647,0.949020}%
\pgfsetfillcolor{currentfill}%
\pgfsetlinewidth{0.000000pt}%
\definecolor{currentstroke}{rgb}{0.000000,0.000000,0.000000}%
\pgfsetstrokecolor{currentstroke}%
\pgfsetstrokeopacity{0.000000}%
\pgfsetdash{}{0pt}%
\pgfpathmoveto{\pgfqpoint{0.320934in}{3.271772in}}%
\pgfpathlineto{\pgfqpoint{2.904267in}{3.271772in}}%
\pgfpathlineto{\pgfqpoint{2.904267in}{3.672657in}}%
\pgfpathlineto{\pgfqpoint{0.320934in}{3.672657in}}%
\pgfpathclose%
\pgfusepath{fill}%
\end{pgfscope}%
\begin{pgfscope}%
\pgfpathrectangle{\pgfqpoint{0.320934in}{3.271772in}}{\pgfqpoint{2.583333in}{0.400885in}}%
\pgfusepath{clip}%
\pgfsetroundcap%
\pgfsetroundjoin%
\pgfsetlinewidth{0.803000pt}%
\definecolor{currentstroke}{rgb}{1.000000,1.000000,1.000000}%
\pgfsetstrokecolor{currentstroke}%
\pgfsetdash{}{0pt}%
\pgfpathmoveto{\pgfqpoint{0.436210in}{3.271772in}}%
\pgfpathlineto{\pgfqpoint{0.436210in}{3.672657in}}%
\pgfusepath{stroke}%
\end{pgfscope}%
\begin{pgfscope}%
\definecolor{textcolor}{rgb}{0.150000,0.150000,0.150000}%
\pgfsetstrokecolor{textcolor}%
\pgfsetfillcolor{textcolor}%
\pgftext[x=0.436210in,y=3.174550in,,top]{\color{textcolor}\rmfamily\fontsize{14.000000}{16.800000}\selectfont 2012}%
\end{pgfscope}%
\begin{pgfscope}%
\pgfpathrectangle{\pgfqpoint{0.320934in}{3.271772in}}{\pgfqpoint{2.583333in}{0.400885in}}%
\pgfusepath{clip}%
\pgfsetroundcap%
\pgfsetroundjoin%
\pgfsetlinewidth{0.803000pt}%
\definecolor{currentstroke}{rgb}{1.000000,1.000000,1.000000}%
\pgfsetstrokecolor{currentstroke}%
\pgfsetdash{}{0pt}%
\pgfpathmoveto{\pgfqpoint{0.829235in}{3.271772in}}%
\pgfpathlineto{\pgfqpoint{0.829235in}{3.672657in}}%
\pgfusepath{stroke}%
\end{pgfscope}%
\begin{pgfscope}%
\definecolor{textcolor}{rgb}{0.150000,0.150000,0.150000}%
\pgfsetstrokecolor{textcolor}%
\pgfsetfillcolor{textcolor}%
\pgftext[x=0.829235in,y=3.174550in,,top]{\color{textcolor}\rmfamily\fontsize{14.000000}{16.800000}\selectfont 2013}%
\end{pgfscope}%
\begin{pgfscope}%
\pgfpathrectangle{\pgfqpoint{0.320934in}{3.271772in}}{\pgfqpoint{2.583333in}{0.400885in}}%
\pgfusepath{clip}%
\pgfsetroundcap%
\pgfsetroundjoin%
\pgfsetlinewidth{0.803000pt}%
\definecolor{currentstroke}{rgb}{1.000000,1.000000,1.000000}%
\pgfsetstrokecolor{currentstroke}%
\pgfsetdash{}{0pt}%
\pgfpathmoveto{\pgfqpoint{1.221186in}{3.271772in}}%
\pgfpathlineto{\pgfqpoint{1.221186in}{3.672657in}}%
\pgfusepath{stroke}%
\end{pgfscope}%
\begin{pgfscope}%
\definecolor{textcolor}{rgb}{0.150000,0.150000,0.150000}%
\pgfsetstrokecolor{textcolor}%
\pgfsetfillcolor{textcolor}%
\pgftext[x=1.221186in,y=3.174550in,,top]{\color{textcolor}\rmfamily\fontsize{14.000000}{16.800000}\selectfont 2014}%
\end{pgfscope}%
\begin{pgfscope}%
\pgfpathrectangle{\pgfqpoint{0.320934in}{3.271772in}}{\pgfqpoint{2.583333in}{0.400885in}}%
\pgfusepath{clip}%
\pgfsetroundcap%
\pgfsetroundjoin%
\pgfsetlinewidth{0.803000pt}%
\definecolor{currentstroke}{rgb}{1.000000,1.000000,1.000000}%
\pgfsetstrokecolor{currentstroke}%
\pgfsetdash{}{0pt}%
\pgfpathmoveto{\pgfqpoint{1.613137in}{3.271772in}}%
\pgfpathlineto{\pgfqpoint{1.613137in}{3.672657in}}%
\pgfusepath{stroke}%
\end{pgfscope}%
\begin{pgfscope}%
\definecolor{textcolor}{rgb}{0.150000,0.150000,0.150000}%
\pgfsetstrokecolor{textcolor}%
\pgfsetfillcolor{textcolor}%
\pgftext[x=1.613137in,y=3.174550in,,top]{\color{textcolor}\rmfamily\fontsize{14.000000}{16.800000}\selectfont 2015}%
\end{pgfscope}%
\begin{pgfscope}%
\pgfpathrectangle{\pgfqpoint{0.320934in}{3.271772in}}{\pgfqpoint{2.583333in}{0.400885in}}%
\pgfusepath{clip}%
\pgfsetroundcap%
\pgfsetroundjoin%
\pgfsetlinewidth{0.803000pt}%
\definecolor{currentstroke}{rgb}{1.000000,1.000000,1.000000}%
\pgfsetstrokecolor{currentstroke}%
\pgfsetdash{}{0pt}%
\pgfpathmoveto{\pgfqpoint{2.005088in}{3.271772in}}%
\pgfpathlineto{\pgfqpoint{2.005088in}{3.672657in}}%
\pgfusepath{stroke}%
\end{pgfscope}%
\begin{pgfscope}%
\definecolor{textcolor}{rgb}{0.150000,0.150000,0.150000}%
\pgfsetstrokecolor{textcolor}%
\pgfsetfillcolor{textcolor}%
\pgftext[x=2.005088in,y=3.174550in,,top]{\color{textcolor}\rmfamily\fontsize{14.000000}{16.800000}\selectfont 2016}%
\end{pgfscope}%
\begin{pgfscope}%
\pgfpathrectangle{\pgfqpoint{0.320934in}{3.271772in}}{\pgfqpoint{2.583333in}{0.400885in}}%
\pgfusepath{clip}%
\pgfsetroundcap%
\pgfsetroundjoin%
\pgfsetlinewidth{0.803000pt}%
\definecolor{currentstroke}{rgb}{1.000000,1.000000,1.000000}%
\pgfsetstrokecolor{currentstroke}%
\pgfsetdash{}{0pt}%
\pgfpathmoveto{\pgfqpoint{2.398113in}{3.271772in}}%
\pgfpathlineto{\pgfqpoint{2.398113in}{3.672657in}}%
\pgfusepath{stroke}%
\end{pgfscope}%
\begin{pgfscope}%
\definecolor{textcolor}{rgb}{0.150000,0.150000,0.150000}%
\pgfsetstrokecolor{textcolor}%
\pgfsetfillcolor{textcolor}%
\pgftext[x=2.398113in,y=3.174550in,,top]{\color{textcolor}\rmfamily\fontsize{14.000000}{16.800000}\selectfont 2017}%
\end{pgfscope}%
\begin{pgfscope}%
\pgfpathrectangle{\pgfqpoint{0.320934in}{3.271772in}}{\pgfqpoint{2.583333in}{0.400885in}}%
\pgfusepath{clip}%
\pgfsetroundcap%
\pgfsetroundjoin%
\pgfsetlinewidth{0.803000pt}%
\definecolor{currentstroke}{rgb}{1.000000,1.000000,1.000000}%
\pgfsetstrokecolor{currentstroke}%
\pgfsetdash{}{0pt}%
\pgfpathmoveto{\pgfqpoint{2.790064in}{3.271772in}}%
\pgfpathlineto{\pgfqpoint{2.790064in}{3.672657in}}%
\pgfusepath{stroke}%
\end{pgfscope}%
\begin{pgfscope}%
\definecolor{textcolor}{rgb}{0.150000,0.150000,0.150000}%
\pgfsetstrokecolor{textcolor}%
\pgfsetfillcolor{textcolor}%
\pgftext[x=2.790064in,y=3.174550in,,top]{\color{textcolor}\rmfamily\fontsize{14.000000}{16.800000}\selectfont 2018}%
\end{pgfscope}%
\begin{pgfscope}%
\pgfpathrectangle{\pgfqpoint{0.320934in}{3.271772in}}{\pgfqpoint{2.583333in}{0.400885in}}%
\pgfusepath{clip}%
\pgfsetroundcap%
\pgfsetroundjoin%
\pgfsetlinewidth{0.803000pt}%
\definecolor{currentstroke}{rgb}{1.000000,1.000000,1.000000}%
\pgfsetstrokecolor{currentstroke}%
\pgfsetdash{}{0pt}%
\pgfpathmoveto{\pgfqpoint{0.320934in}{3.442023in}}%
\pgfpathlineto{\pgfqpoint{2.904267in}{3.442023in}}%
\pgfusepath{stroke}%
\end{pgfscope}%
\begin{pgfscope}%
\definecolor{textcolor}{rgb}{0.150000,0.150000,0.150000}%
\pgfsetstrokecolor{textcolor}%
\pgfsetfillcolor{textcolor}%
\pgftext[x=0.100000in,y=3.368157in,left,base]{\color{textcolor}\rmfamily\fontsize{14.000000}{16.800000}\selectfont 1}%
\end{pgfscope}%
\begin{pgfscope}%
\pgfpathrectangle{\pgfqpoint{0.320934in}{3.271772in}}{\pgfqpoint{2.583333in}{0.400885in}}%
\pgfusepath{clip}%
\pgfsetroundcap%
\pgfsetroundjoin%
\pgfsetlinewidth{0.803000pt}%
\definecolor{currentstroke}{rgb}{1.000000,1.000000,1.000000}%
\pgfsetstrokecolor{currentstroke}%
\pgfsetdash{}{0pt}%
\pgfpathmoveto{\pgfqpoint{0.320934in}{3.636911in}}%
\pgfpathlineto{\pgfqpoint{2.904267in}{3.636911in}}%
\pgfusepath{stroke}%
\end{pgfscope}%
\begin{pgfscope}%
\definecolor{textcolor}{rgb}{0.150000,0.150000,0.150000}%
\pgfsetstrokecolor{textcolor}%
\pgfsetfillcolor{textcolor}%
\pgftext[x=0.100000in,y=3.563044in,left,base]{\color{textcolor}\rmfamily\fontsize{14.000000}{16.800000}\selectfont 2}%
\end{pgfscope}%
\begin{pgfscope}%
\pgfpathrectangle{\pgfqpoint{0.320934in}{3.271772in}}{\pgfqpoint{2.583333in}{0.400885in}}%
\pgfusepath{clip}%
\pgfsetroundcap%
\pgfsetroundjoin%
\pgfsetlinewidth{1.505625pt}%
\definecolor{currentstroke}{rgb}{0.000000,0.000000,0.000000}%
\pgfsetstrokecolor{currentstroke}%
\pgfsetdash{}{0pt}%
\pgfpathmoveto{\pgfqpoint{0.438358in}{3.442023in}}%
\pgfpathlineto{\pgfqpoint{0.439432in}{3.444143in}}%
\pgfpathlineto{\pgfqpoint{0.440506in}{3.444002in}}%
\pgfpathlineto{\pgfqpoint{0.441580in}{3.445132in}}%
\pgfpathlineto{\pgfqpoint{0.444801in}{3.447394in}}%
\pgfpathlineto{\pgfqpoint{0.445875in}{3.445839in}}%
\pgfpathlineto{\pgfqpoint{0.446949in}{3.447535in}}%
\pgfpathlineto{\pgfqpoint{0.448023in}{3.448100in}}%
\pgfpathlineto{\pgfqpoint{0.449096in}{3.447111in}}%
\pgfpathlineto{\pgfqpoint{0.453392in}{3.446122in}}%
\pgfpathlineto{\pgfqpoint{0.454466in}{3.449089in}}%
\pgfpathlineto{\pgfqpoint{0.455539in}{3.450361in}}%
\pgfpathlineto{\pgfqpoint{0.456613in}{3.450361in}}%
\pgfpathlineto{\pgfqpoint{0.460909in}{3.447111in}}%
\pgfpathlineto{\pgfqpoint{0.461982in}{3.450220in}}%
\pgfpathlineto{\pgfqpoint{0.464130in}{3.449089in}}%
\pgfpathlineto{\pgfqpoint{0.467352in}{3.447818in}}%
\pgfpathlineto{\pgfqpoint{0.468425in}{3.445698in}}%
\pgfpathlineto{\pgfqpoint{0.469499in}{3.446404in}}%
\pgfpathlineto{\pgfqpoint{0.470573in}{3.446122in}}%
\pgfpathlineto{\pgfqpoint{0.471647in}{3.449089in}}%
\pgfpathlineto{\pgfqpoint{0.474869in}{3.449372in}}%
\pgfpathlineto{\pgfqpoint{0.477016in}{3.451351in}}%
\pgfpathlineto{\pgfqpoint{0.478090in}{3.450220in}}%
\pgfpathlineto{\pgfqpoint{0.479164in}{3.447535in}}%
\pgfpathlineto{\pgfqpoint{0.482385in}{3.449513in}}%
\pgfpathlineto{\pgfqpoint{0.484533in}{3.446263in}}%
\pgfpathlineto{\pgfqpoint{0.486681in}{3.451775in}}%
\pgfpathlineto{\pgfqpoint{0.490976in}{3.453188in}}%
\pgfpathlineto{\pgfqpoint{0.492050in}{3.452905in}}%
\pgfpathlineto{\pgfqpoint{0.493124in}{3.453895in}}%
\pgfpathlineto{\pgfqpoint{0.497419in}{3.451351in}}%
\pgfpathlineto{\pgfqpoint{0.498493in}{3.452340in}}%
\pgfpathlineto{\pgfqpoint{0.499567in}{3.451209in}}%
\pgfpathlineto{\pgfqpoint{0.500641in}{3.451916in}}%
\pgfpathlineto{\pgfqpoint{0.501714in}{3.450361in}}%
\pgfpathlineto{\pgfqpoint{0.504936in}{3.448948in}}%
\pgfpathlineto{\pgfqpoint{0.506010in}{3.444426in}}%
\pgfpathlineto{\pgfqpoint{0.508157in}{3.450927in}}%
\pgfpathlineto{\pgfqpoint{0.509231in}{3.451068in}}%
\pgfpathlineto{\pgfqpoint{0.512453in}{3.452057in}}%
\pgfpathlineto{\pgfqpoint{0.513527in}{3.457004in}}%
\pgfpathlineto{\pgfqpoint{0.514601in}{3.459124in}}%
\pgfpathlineto{\pgfqpoint{0.515674in}{3.463081in}}%
\pgfpathlineto{\pgfqpoint{0.516748in}{3.463505in}}%
\pgfpathlineto{\pgfqpoint{0.519970in}{3.463646in}}%
\pgfpathlineto{\pgfqpoint{0.521044in}{3.462091in}}%
\pgfpathlineto{\pgfqpoint{0.522117in}{3.462091in}}%
\pgfpathlineto{\pgfqpoint{0.523191in}{3.459689in}}%
\pgfpathlineto{\pgfqpoint{0.524265in}{3.458982in}}%
\pgfpathlineto{\pgfqpoint{0.527487in}{3.461809in}}%
\pgfpathlineto{\pgfqpoint{0.529634in}{3.461385in}}%
\pgfpathlineto{\pgfqpoint{0.530708in}{3.460819in}}%
\pgfpathlineto{\pgfqpoint{0.531782in}{3.462091in}}%
\pgfpathlineto{\pgfqpoint{0.536077in}{3.460961in}}%
\pgfpathlineto{\pgfqpoint{0.538225in}{3.455873in}}%
\pgfpathlineto{\pgfqpoint{0.542520in}{3.452764in}}%
\pgfpathlineto{\pgfqpoint{0.543594in}{3.447818in}}%
\pgfpathlineto{\pgfqpoint{0.545742in}{3.453895in}}%
\pgfpathlineto{\pgfqpoint{0.546816in}{3.449372in}}%
\pgfpathlineto{\pgfqpoint{0.550037in}{3.449513in}}%
\pgfpathlineto{\pgfqpoint{0.551111in}{3.454319in}}%
\pgfpathlineto{\pgfqpoint{0.552185in}{3.451633in}}%
\pgfpathlineto{\pgfqpoint{0.553259in}{3.452057in}}%
\pgfpathlineto{\pgfqpoint{0.554333in}{3.454460in}}%
\pgfpathlineto{\pgfqpoint{0.557554in}{3.451351in}}%
\pgfpathlineto{\pgfqpoint{0.558628in}{3.456438in}}%
\pgfpathlineto{\pgfqpoint{0.559702in}{3.455449in}}%
\pgfpathlineto{\pgfqpoint{0.561849in}{3.458982in}}%
\pgfpathlineto{\pgfqpoint{0.565071in}{3.456862in}}%
\pgfpathlineto{\pgfqpoint{0.566145in}{3.459124in}}%
\pgfpathlineto{\pgfqpoint{0.567219in}{3.458841in}}%
\pgfpathlineto{\pgfqpoint{0.568292in}{3.457145in}}%
\pgfpathlineto{\pgfqpoint{0.569366in}{3.454319in}}%
\pgfpathlineto{\pgfqpoint{0.572588in}{3.454036in}}%
\pgfpathlineto{\pgfqpoint{0.573662in}{3.453329in}}%
\pgfpathlineto{\pgfqpoint{0.574735in}{3.449655in}}%
\pgfpathlineto{\pgfqpoint{0.575809in}{3.451633in}}%
\pgfpathlineto{\pgfqpoint{0.576883in}{3.450785in}}%
\pgfpathlineto{\pgfqpoint{0.581179in}{3.444143in}}%
\pgfpathlineto{\pgfqpoint{0.582252in}{3.450644in}}%
\pgfpathlineto{\pgfqpoint{0.583326in}{3.449372in}}%
\pgfpathlineto{\pgfqpoint{0.588695in}{3.452481in}}%
\pgfpathlineto{\pgfqpoint{0.589769in}{3.452481in}}%
\pgfpathlineto{\pgfqpoint{0.590843in}{3.453329in}}%
\pgfpathlineto{\pgfqpoint{0.591917in}{3.452764in}}%
\pgfpathlineto{\pgfqpoint{0.596212in}{3.454319in}}%
\pgfpathlineto{\pgfqpoint{0.597286in}{3.451068in}}%
\pgfpathlineto{\pgfqpoint{0.598360in}{3.451633in}}%
\pgfpathlineto{\pgfqpoint{0.599434in}{3.445698in}}%
\pgfpathlineto{\pgfqpoint{0.602655in}{3.441458in}}%
\pgfpathlineto{\pgfqpoint{0.603729in}{3.442447in}}%
\pgfpathlineto{\pgfqpoint{0.604803in}{3.449372in}}%
\pgfpathlineto{\pgfqpoint{0.606951in}{3.452764in}}%
\pgfpathlineto{\pgfqpoint{0.610172in}{3.451775in}}%
\pgfpathlineto{\pgfqpoint{0.611246in}{3.455732in}}%
\pgfpathlineto{\pgfqpoint{0.612320in}{3.454601in}}%
\pgfpathlineto{\pgfqpoint{0.614468in}{3.461950in}}%
\pgfpathlineto{\pgfqpoint{0.617689in}{3.459265in}}%
\pgfpathlineto{\pgfqpoint{0.618763in}{3.461950in}}%
\pgfpathlineto{\pgfqpoint{0.619837in}{3.462939in}}%
\pgfpathlineto{\pgfqpoint{0.620911in}{3.458700in}}%
\pgfpathlineto{\pgfqpoint{0.621984in}{3.461667in}}%
\pgfpathlineto{\pgfqpoint{0.625206in}{3.458558in}}%
\pgfpathlineto{\pgfqpoint{0.628427in}{3.465907in}}%
\pgfpathlineto{\pgfqpoint{0.629501in}{3.472832in}}%
\pgfpathlineto{\pgfqpoint{0.633797in}{3.468451in}}%
\pgfpathlineto{\pgfqpoint{0.635944in}{3.467320in}}%
\pgfpathlineto{\pgfqpoint{0.637018in}{3.463787in}}%
\pgfpathlineto{\pgfqpoint{0.640240in}{3.464211in}}%
\pgfpathlineto{\pgfqpoint{0.641313in}{3.459689in}}%
\pgfpathlineto{\pgfqpoint{0.642387in}{3.460254in}}%
\pgfpathlineto{\pgfqpoint{0.643461in}{3.457710in}}%
\pgfpathlineto{\pgfqpoint{0.644535in}{3.461243in}}%
\pgfpathlineto{\pgfqpoint{0.647757in}{3.459265in}}%
\pgfpathlineto{\pgfqpoint{0.649904in}{3.462091in}}%
\pgfpathlineto{\pgfqpoint{0.650978in}{3.461526in}}%
\pgfpathlineto{\pgfqpoint{0.655273in}{3.464777in}}%
\pgfpathlineto{\pgfqpoint{0.656347in}{3.463363in}}%
\pgfpathlineto{\pgfqpoint{0.657421in}{3.463787in}}%
\pgfpathlineto{\pgfqpoint{0.659569in}{3.473680in}}%
\pgfpathlineto{\pgfqpoint{0.664938in}{3.471702in}}%
\pgfpathlineto{\pgfqpoint{0.666012in}{3.469440in}}%
\pgfpathlineto{\pgfqpoint{0.667086in}{3.474104in}}%
\pgfpathlineto{\pgfqpoint{0.670307in}{3.474245in}}%
\pgfpathlineto{\pgfqpoint{0.671381in}{3.475941in}}%
\pgfpathlineto{\pgfqpoint{0.672455in}{3.474669in}}%
\pgfpathlineto{\pgfqpoint{0.674602in}{3.475659in}}%
\pgfpathlineto{\pgfqpoint{0.679972in}{3.474104in}}%
\pgfpathlineto{\pgfqpoint{0.681045in}{3.475093in}}%
\pgfpathlineto{\pgfqpoint{0.682119in}{3.474528in}}%
\pgfpathlineto{\pgfqpoint{0.685341in}{3.473821in}}%
\pgfpathlineto{\pgfqpoint{0.688562in}{3.470712in}}%
\pgfpathlineto{\pgfqpoint{0.689636in}{3.472408in}}%
\pgfpathlineto{\pgfqpoint{0.692858in}{3.472973in}}%
\pgfpathlineto{\pgfqpoint{0.695005in}{3.472691in}}%
\pgfpathlineto{\pgfqpoint{0.696079in}{3.470712in}}%
\pgfpathlineto{\pgfqpoint{0.697153in}{3.471419in}}%
\pgfpathlineto{\pgfqpoint{0.701448in}{3.469299in}}%
\pgfpathlineto{\pgfqpoint{0.702522in}{3.470854in}}%
\pgfpathlineto{\pgfqpoint{0.703596in}{3.477920in}}%
\pgfpathlineto{\pgfqpoint{0.704670in}{3.481029in}}%
\pgfpathlineto{\pgfqpoint{0.707891in}{3.479757in}}%
\pgfpathlineto{\pgfqpoint{0.708965in}{3.481029in}}%
\pgfpathlineto{\pgfqpoint{0.710039in}{3.484279in}}%
\pgfpathlineto{\pgfqpoint{0.712187in}{3.486541in}}%
\pgfpathlineto{\pgfqpoint{0.715408in}{3.485975in}}%
\pgfpathlineto{\pgfqpoint{0.719704in}{3.493042in}}%
\pgfpathlineto{\pgfqpoint{0.723999in}{3.490639in}}%
\pgfpathlineto{\pgfqpoint{0.725073in}{3.488378in}}%
\pgfpathlineto{\pgfqpoint{0.726147in}{3.495161in}}%
\pgfpathlineto{\pgfqpoint{0.727221in}{3.495020in}}%
\pgfpathlineto{\pgfqpoint{0.733664in}{3.497564in}}%
\pgfpathlineto{\pgfqpoint{0.734737in}{3.499401in}}%
\pgfpathlineto{\pgfqpoint{0.737959in}{3.497281in}}%
\pgfpathlineto{\pgfqpoint{0.740107in}{3.491911in}}%
\pgfpathlineto{\pgfqpoint{0.741180in}{3.492759in}}%
\pgfpathlineto{\pgfqpoint{0.742254in}{3.492476in}}%
\pgfpathlineto{\pgfqpoint{0.746550in}{3.494172in}}%
\pgfpathlineto{\pgfqpoint{0.747623in}{3.497140in}}%
\pgfpathlineto{\pgfqpoint{0.748697in}{3.496009in}}%
\pgfpathlineto{\pgfqpoint{0.749771in}{3.487530in}}%
\pgfpathlineto{\pgfqpoint{0.752993in}{3.483997in}}%
\pgfpathlineto{\pgfqpoint{0.754067in}{3.479333in}}%
\pgfpathlineto{\pgfqpoint{0.756214in}{3.479192in}}%
\pgfpathlineto{\pgfqpoint{0.757288in}{3.477496in}}%
\pgfpathlineto{\pgfqpoint{0.762657in}{3.476931in}}%
\pgfpathlineto{\pgfqpoint{0.763731in}{3.480040in}}%
\pgfpathlineto{\pgfqpoint{0.764805in}{3.479757in}}%
\pgfpathlineto{\pgfqpoint{0.768026in}{3.480746in}}%
\pgfpathlineto{\pgfqpoint{0.769100in}{3.482725in}}%
\pgfpathlineto{\pgfqpoint{0.771248in}{3.475093in}}%
\pgfpathlineto{\pgfqpoint{0.772322in}{3.476365in}}%
\pgfpathlineto{\pgfqpoint{0.775543in}{3.475093in}}%
\pgfpathlineto{\pgfqpoint{0.776617in}{3.472832in}}%
\pgfpathlineto{\pgfqpoint{0.777691in}{3.465483in}}%
\pgfpathlineto{\pgfqpoint{0.779839in}{3.467038in}}%
\pgfpathlineto{\pgfqpoint{0.783060in}{3.472549in}}%
\pgfpathlineto{\pgfqpoint{0.784134in}{3.472125in}}%
\pgfpathlineto{\pgfqpoint{0.785208in}{3.472832in}}%
\pgfpathlineto{\pgfqpoint{0.787356in}{3.476789in}}%
\pgfpathlineto{\pgfqpoint{0.790577in}{3.476931in}}%
\pgfpathlineto{\pgfqpoint{0.791651in}{3.474952in}}%
\pgfpathlineto{\pgfqpoint{0.792725in}{3.477920in}}%
\pgfpathlineto{\pgfqpoint{0.794872in}{3.477778in}}%
\pgfpathlineto{\pgfqpoint{0.798094in}{3.474387in}}%
\pgfpathlineto{\pgfqpoint{0.799168in}{3.474811in}}%
\pgfpathlineto{\pgfqpoint{0.800242in}{3.478768in}}%
\pgfpathlineto{\pgfqpoint{0.802389in}{3.481312in}}%
\pgfpathlineto{\pgfqpoint{0.805611in}{3.480605in}}%
\pgfpathlineto{\pgfqpoint{0.806685in}{3.481877in}}%
\pgfpathlineto{\pgfqpoint{0.807758in}{3.484845in}}%
\pgfpathlineto{\pgfqpoint{0.808832in}{3.483149in}}%
\pgfpathlineto{\pgfqpoint{0.809906in}{3.483149in}}%
\pgfpathlineto{\pgfqpoint{0.813128in}{3.486541in}}%
\pgfpathlineto{\pgfqpoint{0.814201in}{3.483855in}}%
\pgfpathlineto{\pgfqpoint{0.815275in}{3.476365in}}%
\pgfpathlineto{\pgfqpoint{0.816349in}{3.478909in}}%
\pgfpathlineto{\pgfqpoint{0.817423in}{3.477072in}}%
\pgfpathlineto{\pgfqpoint{0.822792in}{3.475941in}}%
\pgfpathlineto{\pgfqpoint{0.823866in}{3.474952in}}%
\pgfpathlineto{\pgfqpoint{0.824940in}{3.472267in}}%
\pgfpathlineto{\pgfqpoint{0.830309in}{3.482160in}}%
\pgfpathlineto{\pgfqpoint{0.831383in}{3.479474in}}%
\pgfpathlineto{\pgfqpoint{0.832457in}{3.480605in}}%
\pgfpathlineto{\pgfqpoint{0.835678in}{3.479898in}}%
\pgfpathlineto{\pgfqpoint{0.836752in}{3.477355in}}%
\pgfpathlineto{\pgfqpoint{0.837826in}{3.477920in}}%
\pgfpathlineto{\pgfqpoint{0.838900in}{3.480322in}}%
\pgfpathlineto{\pgfqpoint{0.839974in}{3.479898in}}%
\pgfpathlineto{\pgfqpoint{0.843195in}{3.479757in}}%
\pgfpathlineto{\pgfqpoint{0.844269in}{3.480605in}}%
\pgfpathlineto{\pgfqpoint{0.845343in}{3.479757in}}%
\pgfpathlineto{\pgfqpoint{0.846417in}{3.481736in}}%
\pgfpathlineto{\pgfqpoint{0.847490in}{3.489932in}}%
\pgfpathlineto{\pgfqpoint{0.851786in}{3.489508in}}%
\pgfpathlineto{\pgfqpoint{0.852860in}{3.488802in}}%
\pgfpathlineto{\pgfqpoint{0.853934in}{3.489932in}}%
\pgfpathlineto{\pgfqpoint{0.855007in}{3.492618in}}%
\pgfpathlineto{\pgfqpoint{0.858229in}{3.494879in}}%
\pgfpathlineto{\pgfqpoint{0.859303in}{3.494879in}}%
\pgfpathlineto{\pgfqpoint{0.860377in}{3.491911in}}%
\pgfpathlineto{\pgfqpoint{0.861450in}{3.492476in}}%
\pgfpathlineto{\pgfqpoint{0.862524in}{3.496292in}}%
\pgfpathlineto{\pgfqpoint{0.865746in}{3.492900in}}%
\pgfpathlineto{\pgfqpoint{0.866820in}{3.495444in}}%
\pgfpathlineto{\pgfqpoint{0.867893in}{3.494314in}}%
\pgfpathlineto{\pgfqpoint{0.870041in}{3.494879in}}%
\pgfpathlineto{\pgfqpoint{0.873263in}{3.494455in}}%
\pgfpathlineto{\pgfqpoint{0.874336in}{3.495868in}}%
\pgfpathlineto{\pgfqpoint{0.875410in}{3.504772in}}%
\pgfpathlineto{\pgfqpoint{0.876484in}{3.504913in}}%
\pgfpathlineto{\pgfqpoint{0.877558in}{3.503641in}}%
\pgfpathlineto{\pgfqpoint{0.881853in}{3.508729in}}%
\pgfpathlineto{\pgfqpoint{0.882927in}{3.504913in}}%
\pgfpathlineto{\pgfqpoint{0.884001in}{3.505337in}}%
\pgfpathlineto{\pgfqpoint{0.885075in}{3.506891in}}%
\pgfpathlineto{\pgfqpoint{0.888296in}{3.500391in}}%
\pgfpathlineto{\pgfqpoint{0.890444in}{3.506609in}}%
\pgfpathlineto{\pgfqpoint{0.891518in}{3.504913in}}%
\pgfpathlineto{\pgfqpoint{0.892592in}{3.504630in}}%
\pgfpathlineto{\pgfqpoint{0.895813in}{3.505478in}}%
\pgfpathlineto{\pgfqpoint{0.896887in}{3.509011in}}%
\pgfpathlineto{\pgfqpoint{0.897961in}{3.510001in}}%
\pgfpathlineto{\pgfqpoint{0.899035in}{3.510001in}}%
\pgfpathlineto{\pgfqpoint{0.900109in}{3.511131in}}%
\pgfpathlineto{\pgfqpoint{0.903330in}{3.509435in}}%
\pgfpathlineto{\pgfqpoint{0.904404in}{3.507033in}}%
\pgfpathlineto{\pgfqpoint{0.905478in}{3.508022in}}%
\pgfpathlineto{\pgfqpoint{0.906552in}{3.510142in}}%
\pgfpathlineto{\pgfqpoint{0.907625in}{3.507457in}}%
\pgfpathlineto{\pgfqpoint{0.910847in}{3.505337in}}%
\pgfpathlineto{\pgfqpoint{0.911921in}{3.506044in}}%
\pgfpathlineto{\pgfqpoint{0.912995in}{3.507598in}}%
\pgfpathlineto{\pgfqpoint{0.914068in}{3.505761in}}%
\pgfpathlineto{\pgfqpoint{0.915142in}{3.506609in}}%
\pgfpathlineto{\pgfqpoint{0.918364in}{3.505196in}}%
\pgfpathlineto{\pgfqpoint{0.919438in}{3.503782in}}%
\pgfpathlineto{\pgfqpoint{0.925881in}{3.503358in}}%
\pgfpathlineto{\pgfqpoint{0.926955in}{3.506326in}}%
\pgfpathlineto{\pgfqpoint{0.928028in}{3.502510in}}%
\pgfpathlineto{\pgfqpoint{0.929102in}{3.503358in}}%
\pgfpathlineto{\pgfqpoint{0.930176in}{3.501804in}}%
\pgfpathlineto{\pgfqpoint{0.933398in}{3.503782in}}%
\pgfpathlineto{\pgfqpoint{0.934471in}{3.503217in}}%
\pgfpathlineto{\pgfqpoint{0.935545in}{3.509011in}}%
\pgfpathlineto{\pgfqpoint{0.936619in}{3.509011in}}%
\pgfpathlineto{\pgfqpoint{0.937693in}{3.507598in}}%
\pgfpathlineto{\pgfqpoint{0.940914in}{3.500391in}}%
\pgfpathlineto{\pgfqpoint{0.941988in}{3.503641in}}%
\pgfpathlineto{\pgfqpoint{0.943062in}{3.499825in}}%
\pgfpathlineto{\pgfqpoint{0.944136in}{3.498836in}}%
\pgfpathlineto{\pgfqpoint{0.945210in}{3.488661in}}%
\pgfpathlineto{\pgfqpoint{0.948431in}{3.484138in}}%
\pgfpathlineto{\pgfqpoint{0.949505in}{3.485834in}}%
\pgfpathlineto{\pgfqpoint{0.950579in}{3.490922in}}%
\pgfpathlineto{\pgfqpoint{0.951653in}{3.490922in}}%
\pgfpathlineto{\pgfqpoint{0.952727in}{3.493748in}}%
\pgfpathlineto{\pgfqpoint{0.957022in}{3.494596in}}%
\pgfpathlineto{\pgfqpoint{0.958096in}{3.493042in}}%
\pgfpathlineto{\pgfqpoint{0.960244in}{3.497705in}}%
\pgfpathlineto{\pgfqpoint{0.963465in}{3.497847in}}%
\pgfpathlineto{\pgfqpoint{0.964539in}{3.498977in}}%
\pgfpathlineto{\pgfqpoint{0.965613in}{3.502652in}}%
\pgfpathlineto{\pgfqpoint{0.966687in}{3.500108in}}%
\pgfpathlineto{\pgfqpoint{0.967760in}{3.501380in}}%
\pgfpathlineto{\pgfqpoint{0.970982in}{3.500814in}}%
\pgfpathlineto{\pgfqpoint{0.975277in}{3.507598in}}%
\pgfpathlineto{\pgfqpoint{0.979573in}{3.509859in}}%
\pgfpathlineto{\pgfqpoint{0.980646in}{3.512120in}}%
\pgfpathlineto{\pgfqpoint{0.982794in}{3.508446in}}%
\pgfpathlineto{\pgfqpoint{0.989237in}{3.509153in}}%
\pgfpathlineto{\pgfqpoint{0.990311in}{3.506044in}}%
\pgfpathlineto{\pgfqpoint{0.993533in}{3.509577in}}%
\pgfpathlineto{\pgfqpoint{0.994606in}{3.509859in}}%
\pgfpathlineto{\pgfqpoint{0.995680in}{3.506044in}}%
\pgfpathlineto{\pgfqpoint{0.996754in}{3.506750in}}%
\pgfpathlineto{\pgfqpoint{0.997828in}{3.512120in}}%
\pgfpathlineto{\pgfqpoint{1.001049in}{3.511131in}}%
\pgfpathlineto{\pgfqpoint{1.003197in}{3.508022in}}%
\pgfpathlineto{\pgfqpoint{1.004271in}{3.510001in}}%
\pgfpathlineto{\pgfqpoint{1.005345in}{3.508305in}}%
\pgfpathlineto{\pgfqpoint{1.008566in}{3.511131in}}%
\pgfpathlineto{\pgfqpoint{1.009640in}{3.517350in}}%
\pgfpathlineto{\pgfqpoint{1.011788in}{3.507315in}}%
\pgfpathlineto{\pgfqpoint{1.012862in}{3.508587in}}%
\pgfpathlineto{\pgfqpoint{1.016083in}{3.503782in}}%
\pgfpathlineto{\pgfqpoint{1.018231in}{3.507315in}}%
\pgfpathlineto{\pgfqpoint{1.019305in}{3.508163in}}%
\pgfpathlineto{\pgfqpoint{1.020378in}{3.506750in}}%
\pgfpathlineto{\pgfqpoint{1.023600in}{3.508305in}}%
\pgfpathlineto{\pgfqpoint{1.024674in}{3.503500in}}%
\pgfpathlineto{\pgfqpoint{1.025748in}{3.503500in}}%
\pgfpathlineto{\pgfqpoint{1.027895in}{3.507174in}}%
\pgfpathlineto{\pgfqpoint{1.031117in}{3.508163in}}%
\pgfpathlineto{\pgfqpoint{1.032191in}{3.511555in}}%
\pgfpathlineto{\pgfqpoint{1.033265in}{3.510566in}}%
\pgfpathlineto{\pgfqpoint{1.034338in}{3.515088in}}%
\pgfpathlineto{\pgfqpoint{1.035412in}{3.513110in}}%
\pgfpathlineto{\pgfqpoint{1.038634in}{3.511555in}}%
\pgfpathlineto{\pgfqpoint{1.039708in}{3.509435in}}%
\pgfpathlineto{\pgfqpoint{1.041855in}{3.511555in}}%
\pgfpathlineto{\pgfqpoint{1.042929in}{3.523850in}}%
\pgfpathlineto{\pgfqpoint{1.046151in}{3.525405in}}%
\pgfpathlineto{\pgfqpoint{1.048298in}{3.522720in}}%
\pgfpathlineto{\pgfqpoint{1.049372in}{3.523426in}}%
\pgfpathlineto{\pgfqpoint{1.054741in}{3.521165in}}%
\pgfpathlineto{\pgfqpoint{1.055815in}{3.519893in}}%
\pgfpathlineto{\pgfqpoint{1.056889in}{3.522720in}}%
\pgfpathlineto{\pgfqpoint{1.057963in}{3.523568in}}%
\pgfpathlineto{\pgfqpoint{1.061184in}{3.521589in}}%
\pgfpathlineto{\pgfqpoint{1.062258in}{3.519187in}}%
\pgfpathlineto{\pgfqpoint{1.064406in}{3.519469in}}%
\pgfpathlineto{\pgfqpoint{1.065480in}{3.518480in}}%
\pgfpathlineto{\pgfqpoint{1.068701in}{3.518763in}}%
\pgfpathlineto{\pgfqpoint{1.069775in}{3.518056in}}%
\pgfpathlineto{\pgfqpoint{1.071923in}{3.515795in}}%
\pgfpathlineto{\pgfqpoint{1.077292in}{3.512544in}}%
\pgfpathlineto{\pgfqpoint{1.078366in}{3.511414in}}%
\pgfpathlineto{\pgfqpoint{1.079440in}{3.513251in}}%
\pgfpathlineto{\pgfqpoint{1.080513in}{3.513251in}}%
\pgfpathlineto{\pgfqpoint{1.083735in}{3.511414in}}%
\pgfpathlineto{\pgfqpoint{1.084809in}{3.506609in}}%
\pgfpathlineto{\pgfqpoint{1.085883in}{3.506750in}}%
\pgfpathlineto{\pgfqpoint{1.086956in}{3.505761in}}%
\pgfpathlineto{\pgfqpoint{1.088030in}{3.506185in}}%
\pgfpathlineto{\pgfqpoint{1.092326in}{3.505196in}}%
\pgfpathlineto{\pgfqpoint{1.093399in}{3.506467in}}%
\pgfpathlineto{\pgfqpoint{1.095547in}{3.506326in}}%
\pgfpathlineto{\pgfqpoint{1.098769in}{3.508870in}}%
\pgfpathlineto{\pgfqpoint{1.100916in}{3.516784in}}%
\pgfpathlineto{\pgfqpoint{1.101990in}{3.514099in}}%
\pgfpathlineto{\pgfqpoint{1.103064in}{3.513251in}}%
\pgfpathlineto{\pgfqpoint{1.106286in}{3.517350in}}%
\pgfpathlineto{\pgfqpoint{1.108433in}{3.525405in}}%
\pgfpathlineto{\pgfqpoint{1.109507in}{3.523003in}}%
\pgfpathlineto{\pgfqpoint{1.110581in}{3.517915in}}%
\pgfpathlineto{\pgfqpoint{1.114876in}{3.521448in}}%
\pgfpathlineto{\pgfqpoint{1.115950in}{3.520459in}}%
\pgfpathlineto{\pgfqpoint{1.117024in}{3.520600in}}%
\pgfpathlineto{\pgfqpoint{1.118098in}{3.518339in}}%
\pgfpathlineto{\pgfqpoint{1.121319in}{3.516643in}}%
\pgfpathlineto{\pgfqpoint{1.123467in}{3.521589in}}%
\pgfpathlineto{\pgfqpoint{1.124541in}{3.518904in}}%
\pgfpathlineto{\pgfqpoint{1.128836in}{3.517208in}}%
\pgfpathlineto{\pgfqpoint{1.129910in}{3.514099in}}%
\pgfpathlineto{\pgfqpoint{1.130984in}{3.512968in}}%
\pgfpathlineto{\pgfqpoint{1.132058in}{3.520600in}}%
\pgfpathlineto{\pgfqpoint{1.133132in}{3.522296in}}%
\pgfpathlineto{\pgfqpoint{1.136353in}{3.522155in}}%
\pgfpathlineto{\pgfqpoint{1.137427in}{3.519893in}}%
\pgfpathlineto{\pgfqpoint{1.138501in}{3.521872in}}%
\pgfpathlineto{\pgfqpoint{1.139575in}{3.525546in}}%
\pgfpathlineto{\pgfqpoint{1.140648in}{3.535298in}}%
\pgfpathlineto{\pgfqpoint{1.143870in}{3.541940in}}%
\pgfpathlineto{\pgfqpoint{1.144944in}{3.540668in}}%
\pgfpathlineto{\pgfqpoint{1.146018in}{3.536994in}}%
\pgfpathlineto{\pgfqpoint{1.147091in}{3.539679in}}%
\pgfpathlineto{\pgfqpoint{1.148165in}{3.538972in}}%
\pgfpathlineto{\pgfqpoint{1.152461in}{3.542788in}}%
\pgfpathlineto{\pgfqpoint{1.153534in}{3.544484in}}%
\pgfpathlineto{\pgfqpoint{1.154608in}{3.541940in}}%
\pgfpathlineto{\pgfqpoint{1.155682in}{3.546462in}}%
\pgfpathlineto{\pgfqpoint{1.159977in}{3.545049in}}%
\pgfpathlineto{\pgfqpoint{1.161051in}{3.550561in}}%
\pgfpathlineto{\pgfqpoint{1.162125in}{3.547169in}}%
\pgfpathlineto{\pgfqpoint{1.163199in}{3.552257in}}%
\pgfpathlineto{\pgfqpoint{1.166421in}{3.551833in}}%
\pgfpathlineto{\pgfqpoint{1.167494in}{3.552257in}}%
\pgfpathlineto{\pgfqpoint{1.168568in}{3.553387in}}%
\pgfpathlineto{\pgfqpoint{1.169642in}{3.551550in}}%
\pgfpathlineto{\pgfqpoint{1.170716in}{3.553953in}}%
\pgfpathlineto{\pgfqpoint{1.173937in}{3.554094in}}%
\pgfpathlineto{\pgfqpoint{1.175011in}{3.551974in}}%
\pgfpathlineto{\pgfqpoint{1.177159in}{3.550702in}}%
\pgfpathlineto{\pgfqpoint{1.178233in}{3.552539in}}%
\pgfpathlineto{\pgfqpoint{1.181454in}{3.548582in}}%
\pgfpathlineto{\pgfqpoint{1.183602in}{3.549713in}}%
\pgfpathlineto{\pgfqpoint{1.185750in}{3.547876in}}%
\pgfpathlineto{\pgfqpoint{1.188971in}{3.547876in}}%
\pgfpathlineto{\pgfqpoint{1.190045in}{3.546745in}}%
\pgfpathlineto{\pgfqpoint{1.191119in}{3.547593in}}%
\pgfpathlineto{\pgfqpoint{1.192193in}{3.545473in}}%
\pgfpathlineto{\pgfqpoint{1.193266in}{3.550985in}}%
\pgfpathlineto{\pgfqpoint{1.196488in}{3.553811in}}%
\pgfpathlineto{\pgfqpoint{1.197562in}{3.553246in}}%
\pgfpathlineto{\pgfqpoint{1.198636in}{3.546886in}}%
\pgfpathlineto{\pgfqpoint{1.199710in}{3.546462in}}%
\pgfpathlineto{\pgfqpoint{1.200783in}{3.549854in}}%
\pgfpathlineto{\pgfqpoint{1.205079in}{3.551974in}}%
\pgfpathlineto{\pgfqpoint{1.206153in}{3.556214in}}%
\pgfpathlineto{\pgfqpoint{1.207226in}{3.557768in}}%
\pgfpathlineto{\pgfqpoint{1.208300in}{3.558192in}}%
\pgfpathlineto{\pgfqpoint{1.211522in}{3.558616in}}%
\pgfpathlineto{\pgfqpoint{1.212596in}{3.561019in}}%
\pgfpathlineto{\pgfqpoint{1.214743in}{3.563563in}}%
\pgfpathlineto{\pgfqpoint{1.215817in}{3.563563in}}%
\pgfpathlineto{\pgfqpoint{1.219039in}{3.564269in}}%
\pgfpathlineto{\pgfqpoint{1.220112in}{3.565824in}}%
\pgfpathlineto{\pgfqpoint{1.222260in}{3.559747in}}%
\pgfpathlineto{\pgfqpoint{1.223334in}{3.559606in}}%
\pgfpathlineto{\pgfqpoint{1.226555in}{3.557062in}}%
\pgfpathlineto{\pgfqpoint{1.227629in}{3.557345in}}%
\pgfpathlineto{\pgfqpoint{1.228703in}{3.556497in}}%
\pgfpathlineto{\pgfqpoint{1.229777in}{3.556638in}}%
\pgfpathlineto{\pgfqpoint{1.230851in}{3.553670in}}%
\pgfpathlineto{\pgfqpoint{1.234072in}{3.550985in}}%
\pgfpathlineto{\pgfqpoint{1.236220in}{3.558051in}}%
\pgfpathlineto{\pgfqpoint{1.237294in}{3.556355in}}%
\pgfpathlineto{\pgfqpoint{1.238368in}{3.549289in}}%
\pgfpathlineto{\pgfqpoint{1.242663in}{3.546039in}}%
\pgfpathlineto{\pgfqpoint{1.244811in}{3.540668in}}%
\pgfpathlineto{\pgfqpoint{1.245885in}{3.530775in}}%
\pgfpathlineto{\pgfqpoint{1.249106in}{3.532189in}}%
\pgfpathlineto{\pgfqpoint{1.250180in}{3.536570in}}%
\pgfpathlineto{\pgfqpoint{1.251254in}{3.534733in}}%
\pgfpathlineto{\pgfqpoint{1.252328in}{3.536994in}}%
\pgfpathlineto{\pgfqpoint{1.253401in}{3.532895in}}%
\pgfpathlineto{\pgfqpoint{1.256623in}{3.523992in}}%
\pgfpathlineto{\pgfqpoint{1.257697in}{3.526536in}}%
\pgfpathlineto{\pgfqpoint{1.258771in}{3.525970in}}%
\pgfpathlineto{\pgfqpoint{1.260918in}{3.533602in}}%
\pgfpathlineto{\pgfqpoint{1.264140in}{3.531906in}}%
\pgfpathlineto{\pgfqpoint{1.265214in}{3.536287in}}%
\pgfpathlineto{\pgfqpoint{1.266287in}{3.535863in}}%
\pgfpathlineto{\pgfqpoint{1.267361in}{3.536428in}}%
\pgfpathlineto{\pgfqpoint{1.268435in}{3.539820in}}%
\pgfpathlineto{\pgfqpoint{1.272731in}{3.538831in}}%
\pgfpathlineto{\pgfqpoint{1.273804in}{3.535863in}}%
\pgfpathlineto{\pgfqpoint{1.274878in}{3.535298in}}%
\pgfpathlineto{\pgfqpoint{1.275952in}{3.533178in}}%
\pgfpathlineto{\pgfqpoint{1.279174in}{3.537135in}}%
\pgfpathlineto{\pgfqpoint{1.281321in}{3.537276in}}%
\pgfpathlineto{\pgfqpoint{1.282395in}{3.539538in}}%
\pgfpathlineto{\pgfqpoint{1.283469in}{3.539255in}}%
\pgfpathlineto{\pgfqpoint{1.286690in}{3.535298in}}%
\pgfpathlineto{\pgfqpoint{1.288838in}{3.544484in}}%
\pgfpathlineto{\pgfqpoint{1.289912in}{3.547876in}}%
\pgfpathlineto{\pgfqpoint{1.290986in}{3.546886in}}%
\pgfpathlineto{\pgfqpoint{1.294207in}{3.545756in}}%
\pgfpathlineto{\pgfqpoint{1.296355in}{3.542647in}}%
\pgfpathlineto{\pgfqpoint{1.298503in}{3.535156in}}%
\pgfpathlineto{\pgfqpoint{1.301724in}{3.538831in}}%
\pgfpathlineto{\pgfqpoint{1.302798in}{3.541375in}}%
\pgfpathlineto{\pgfqpoint{1.303872in}{3.537135in}}%
\pgfpathlineto{\pgfqpoint{1.304946in}{3.536994in}}%
\pgfpathlineto{\pgfqpoint{1.306020in}{3.538407in}}%
\pgfpathlineto{\pgfqpoint{1.309241in}{3.538548in}}%
\pgfpathlineto{\pgfqpoint{1.310315in}{3.541940in}}%
\pgfpathlineto{\pgfqpoint{1.311389in}{3.540951in}}%
\pgfpathlineto{\pgfqpoint{1.312463in}{3.543212in}}%
\pgfpathlineto{\pgfqpoint{1.313536in}{3.543919in}}%
\pgfpathlineto{\pgfqpoint{1.317832in}{3.543777in}}%
\pgfpathlineto{\pgfqpoint{1.319979in}{3.548017in}}%
\pgfpathlineto{\pgfqpoint{1.321053in}{3.545615in}}%
\pgfpathlineto{\pgfqpoint{1.325349in}{3.542505in}}%
\pgfpathlineto{\pgfqpoint{1.326422in}{3.544767in}}%
\pgfpathlineto{\pgfqpoint{1.327496in}{3.540527in}}%
\pgfpathlineto{\pgfqpoint{1.328570in}{3.538831in}}%
\pgfpathlineto{\pgfqpoint{1.332865in}{3.543212in}}%
\pgfpathlineto{\pgfqpoint{1.335013in}{3.551692in}}%
\pgfpathlineto{\pgfqpoint{1.340382in}{3.551974in}}%
\pgfpathlineto{\pgfqpoint{1.341456in}{3.550137in}}%
\pgfpathlineto{\pgfqpoint{1.342530in}{3.550561in}}%
\pgfpathlineto{\pgfqpoint{1.343604in}{3.552257in}}%
\pgfpathlineto{\pgfqpoint{1.346825in}{3.554235in}}%
\pgfpathlineto{\pgfqpoint{1.347899in}{3.554094in}}%
\pgfpathlineto{\pgfqpoint{1.348973in}{3.555507in}}%
\pgfpathlineto{\pgfqpoint{1.351121in}{3.553105in}}%
\pgfpathlineto{\pgfqpoint{1.354342in}{3.551974in}}%
\pgfpathlineto{\pgfqpoint{1.355416in}{3.547452in}}%
\pgfpathlineto{\pgfqpoint{1.356490in}{3.551409in}}%
\pgfpathlineto{\pgfqpoint{1.358638in}{3.550137in}}%
\pgfpathlineto{\pgfqpoint{1.362933in}{3.555931in}}%
\pgfpathlineto{\pgfqpoint{1.365081in}{3.552257in}}%
\pgfpathlineto{\pgfqpoint{1.366154in}{3.552963in}}%
\pgfpathlineto{\pgfqpoint{1.369376in}{3.552257in}}%
\pgfpathlineto{\pgfqpoint{1.370450in}{3.548724in}}%
\pgfpathlineto{\pgfqpoint{1.371524in}{3.550985in}}%
\pgfpathlineto{\pgfqpoint{1.377967in}{3.551833in}}%
\pgfpathlineto{\pgfqpoint{1.380114in}{3.553811in}}%
\pgfpathlineto{\pgfqpoint{1.381188in}{3.554377in}}%
\pgfpathlineto{\pgfqpoint{1.384410in}{3.554801in}}%
\pgfpathlineto{\pgfqpoint{1.385484in}{3.554377in}}%
\pgfpathlineto{\pgfqpoint{1.386557in}{3.551692in}}%
\pgfpathlineto{\pgfqpoint{1.387631in}{3.554094in}}%
\pgfpathlineto{\pgfqpoint{1.388705in}{3.558899in}}%
\pgfpathlineto{\pgfqpoint{1.391927in}{3.561867in}}%
\pgfpathlineto{\pgfqpoint{1.393000in}{3.561443in}}%
\pgfpathlineto{\pgfqpoint{1.395148in}{3.556355in}}%
\pgfpathlineto{\pgfqpoint{1.396222in}{3.557203in}}%
\pgfpathlineto{\pgfqpoint{1.399443in}{3.554801in}}%
\pgfpathlineto{\pgfqpoint{1.401591in}{3.555507in}}%
\pgfpathlineto{\pgfqpoint{1.402665in}{3.558475in}}%
\pgfpathlineto{\pgfqpoint{1.403739in}{3.559040in}}%
\pgfpathlineto{\pgfqpoint{1.409108in}{3.552681in}}%
\pgfpathlineto{\pgfqpoint{1.410182in}{3.551126in}}%
\pgfpathlineto{\pgfqpoint{1.411256in}{3.552822in}}%
\pgfpathlineto{\pgfqpoint{1.414477in}{3.550985in}}%
\pgfpathlineto{\pgfqpoint{1.415551in}{3.552398in}}%
\pgfpathlineto{\pgfqpoint{1.417699in}{3.557768in}}%
\pgfpathlineto{\pgfqpoint{1.421994in}{3.556497in}}%
\pgfpathlineto{\pgfqpoint{1.423068in}{3.552115in}}%
\pgfpathlineto{\pgfqpoint{1.424142in}{3.551550in}}%
\pgfpathlineto{\pgfqpoint{1.425216in}{3.550137in}}%
\pgfpathlineto{\pgfqpoint{1.426289in}{3.554094in}}%
\pgfpathlineto{\pgfqpoint{1.429511in}{3.555366in}}%
\pgfpathlineto{\pgfqpoint{1.430585in}{3.554801in}}%
\pgfpathlineto{\pgfqpoint{1.431659in}{3.559606in}}%
\pgfpathlineto{\pgfqpoint{1.432732in}{3.554801in}}%
\pgfpathlineto{\pgfqpoint{1.437028in}{3.547593in}}%
\pgfpathlineto{\pgfqpoint{1.438102in}{3.548017in}}%
\pgfpathlineto{\pgfqpoint{1.439175in}{3.546745in}}%
\pgfpathlineto{\pgfqpoint{1.440249in}{3.547028in}}%
\pgfpathlineto{\pgfqpoint{1.441323in}{3.545332in}}%
\pgfpathlineto{\pgfqpoint{1.444545in}{3.543071in}}%
\pgfpathlineto{\pgfqpoint{1.445619in}{3.541375in}}%
\pgfpathlineto{\pgfqpoint{1.446692in}{3.543636in}}%
\pgfpathlineto{\pgfqpoint{1.447766in}{3.537983in}}%
\pgfpathlineto{\pgfqpoint{1.448840in}{3.540244in}}%
\pgfpathlineto{\pgfqpoint{1.452062in}{3.539396in}}%
\pgfpathlineto{\pgfqpoint{1.453135in}{3.536428in}}%
\pgfpathlineto{\pgfqpoint{1.454209in}{3.541375in}}%
\pgfpathlineto{\pgfqpoint{1.455283in}{3.541940in}}%
\pgfpathlineto{\pgfqpoint{1.456357in}{3.543919in}}%
\pgfpathlineto{\pgfqpoint{1.459578in}{3.545332in}}%
\pgfpathlineto{\pgfqpoint{1.460652in}{3.543212in}}%
\pgfpathlineto{\pgfqpoint{1.461726in}{3.545756in}}%
\pgfpathlineto{\pgfqpoint{1.462800in}{3.546462in}}%
\pgfpathlineto{\pgfqpoint{1.463874in}{3.543636in}}%
\pgfpathlineto{\pgfqpoint{1.467095in}{3.548582in}}%
\pgfpathlineto{\pgfqpoint{1.468169in}{3.548300in}}%
\pgfpathlineto{\pgfqpoint{1.469243in}{3.551974in}}%
\pgfpathlineto{\pgfqpoint{1.470317in}{3.552822in}}%
\pgfpathlineto{\pgfqpoint{1.471391in}{3.549572in}}%
\pgfpathlineto{\pgfqpoint{1.474612in}{3.550137in}}%
\pgfpathlineto{\pgfqpoint{1.475686in}{3.547876in}}%
\pgfpathlineto{\pgfqpoint{1.476760in}{3.549289in}}%
\pgfpathlineto{\pgfqpoint{1.477834in}{3.547876in}}%
\pgfpathlineto{\pgfqpoint{1.483203in}{3.546039in}}%
\pgfpathlineto{\pgfqpoint{1.484277in}{3.547169in}}%
\pgfpathlineto{\pgfqpoint{1.485351in}{3.547310in}}%
\pgfpathlineto{\pgfqpoint{1.486424in}{3.549006in}}%
\pgfpathlineto{\pgfqpoint{1.489646in}{3.548724in}}%
\pgfpathlineto{\pgfqpoint{1.490720in}{3.546604in}}%
\pgfpathlineto{\pgfqpoint{1.492867in}{3.548017in}}%
\pgfpathlineto{\pgfqpoint{1.493941in}{3.546321in}}%
\pgfpathlineto{\pgfqpoint{1.497163in}{3.546886in}}%
\pgfpathlineto{\pgfqpoint{1.498237in}{3.550278in}}%
\pgfpathlineto{\pgfqpoint{1.499310in}{3.550844in}}%
\pgfpathlineto{\pgfqpoint{1.501458in}{3.553670in}}%
\pgfpathlineto{\pgfqpoint{1.506827in}{3.549572in}}%
\pgfpathlineto{\pgfqpoint{1.507901in}{3.545049in}}%
\pgfpathlineto{\pgfqpoint{1.508975in}{3.546039in}}%
\pgfpathlineto{\pgfqpoint{1.512197in}{3.543495in}}%
\pgfpathlineto{\pgfqpoint{1.513270in}{3.545897in}}%
\pgfpathlineto{\pgfqpoint{1.514344in}{3.540527in}}%
\pgfpathlineto{\pgfqpoint{1.515418in}{3.540103in}}%
\pgfpathlineto{\pgfqpoint{1.516492in}{3.543353in}}%
\pgfpathlineto{\pgfqpoint{1.519713in}{3.541233in}}%
\pgfpathlineto{\pgfqpoint{1.520787in}{3.536428in}}%
\pgfpathlineto{\pgfqpoint{1.521861in}{3.541516in}}%
\pgfpathlineto{\pgfqpoint{1.524009in}{3.530210in}}%
\pgfpathlineto{\pgfqpoint{1.527230in}{3.526394in}}%
\pgfpathlineto{\pgfqpoint{1.529378in}{3.530210in}}%
\pgfpathlineto{\pgfqpoint{1.530452in}{3.529927in}}%
\pgfpathlineto{\pgfqpoint{1.531526in}{3.536570in}}%
\pgfpathlineto{\pgfqpoint{1.534747in}{3.538972in}}%
\pgfpathlineto{\pgfqpoint{1.535821in}{3.543919in}}%
\pgfpathlineto{\pgfqpoint{1.536895in}{3.540951in}}%
\pgfpathlineto{\pgfqpoint{1.539042in}{3.546180in}}%
\pgfpathlineto{\pgfqpoint{1.542264in}{3.544767in}}%
\pgfpathlineto{\pgfqpoint{1.543338in}{3.548865in}}%
\pgfpathlineto{\pgfqpoint{1.544412in}{3.546321in}}%
\pgfpathlineto{\pgfqpoint{1.545486in}{3.546462in}}%
\pgfpathlineto{\pgfqpoint{1.546559in}{3.548158in}}%
\pgfpathlineto{\pgfqpoint{1.550855in}{3.546886in}}%
\pgfpathlineto{\pgfqpoint{1.551929in}{3.548300in}}%
\pgfpathlineto{\pgfqpoint{1.553002in}{3.554518in}}%
\pgfpathlineto{\pgfqpoint{1.554076in}{3.555083in}}%
\pgfpathlineto{\pgfqpoint{1.557298in}{3.555790in}}%
\pgfpathlineto{\pgfqpoint{1.558372in}{3.554801in}}%
\pgfpathlineto{\pgfqpoint{1.559445in}{3.556355in}}%
\pgfpathlineto{\pgfqpoint{1.560519in}{3.555225in}}%
\pgfpathlineto{\pgfqpoint{1.564815in}{3.557486in}}%
\pgfpathlineto{\pgfqpoint{1.565888in}{3.562150in}}%
\pgfpathlineto{\pgfqpoint{1.568036in}{3.560171in}}%
\pgfpathlineto{\pgfqpoint{1.569110in}{3.561867in}}%
\pgfpathlineto{\pgfqpoint{1.572331in}{3.562008in}}%
\pgfpathlineto{\pgfqpoint{1.573405in}{3.560312in}}%
\pgfpathlineto{\pgfqpoint{1.574479in}{3.560454in}}%
\pgfpathlineto{\pgfqpoint{1.579848in}{3.550561in}}%
\pgfpathlineto{\pgfqpoint{1.580922in}{3.550844in}}%
\pgfpathlineto{\pgfqpoint{1.581996in}{3.554801in}}%
\pgfpathlineto{\pgfqpoint{1.583070in}{3.551409in}}%
\pgfpathlineto{\pgfqpoint{1.587365in}{3.546745in}}%
\pgfpathlineto{\pgfqpoint{1.588439in}{3.545473in}}%
\pgfpathlineto{\pgfqpoint{1.589513in}{3.541799in}}%
\pgfpathlineto{\pgfqpoint{1.590587in}{3.543495in}}%
\pgfpathlineto{\pgfqpoint{1.591661in}{3.537418in}}%
\pgfpathlineto{\pgfqpoint{1.595956in}{3.532754in}}%
\pgfpathlineto{\pgfqpoint{1.597030in}{3.534733in}}%
\pgfpathlineto{\pgfqpoint{1.599177in}{3.548724in}}%
\pgfpathlineto{\pgfqpoint{1.602399in}{3.549713in}}%
\pgfpathlineto{\pgfqpoint{1.603473in}{3.551833in}}%
\pgfpathlineto{\pgfqpoint{1.604547in}{3.551126in}}%
\pgfpathlineto{\pgfqpoint{1.609916in}{3.549713in}}%
\pgfpathlineto{\pgfqpoint{1.610990in}{3.548158in}}%
\pgfpathlineto{\pgfqpoint{1.612063in}{3.544625in}}%
\pgfpathlineto{\pgfqpoint{1.614211in}{3.542081in}}%
\pgfpathlineto{\pgfqpoint{1.617433in}{3.536711in}}%
\pgfpathlineto{\pgfqpoint{1.618507in}{3.530493in}}%
\pgfpathlineto{\pgfqpoint{1.619580in}{3.530634in}}%
\pgfpathlineto{\pgfqpoint{1.620654in}{3.534026in}}%
\pgfpathlineto{\pgfqpoint{1.621728in}{3.529927in}}%
\pgfpathlineto{\pgfqpoint{1.624950in}{3.529362in}}%
\pgfpathlineto{\pgfqpoint{1.628171in}{3.524698in}}%
\pgfpathlineto{\pgfqpoint{1.629245in}{3.524840in}}%
\pgfpathlineto{\pgfqpoint{1.633540in}{3.527808in}}%
\pgfpathlineto{\pgfqpoint{1.636762in}{3.535298in}}%
\pgfpathlineto{\pgfqpoint{1.639983in}{3.536570in}}%
\pgfpathlineto{\pgfqpoint{1.641057in}{3.534167in}}%
\pgfpathlineto{\pgfqpoint{1.642131in}{3.527808in}}%
\pgfpathlineto{\pgfqpoint{1.643205in}{3.530634in}}%
\pgfpathlineto{\pgfqpoint{1.644279in}{3.528373in}}%
\pgfpathlineto{\pgfqpoint{1.647500in}{3.532047in}}%
\pgfpathlineto{\pgfqpoint{1.648574in}{3.535156in}}%
\pgfpathlineto{\pgfqpoint{1.649648in}{3.531482in}}%
\pgfpathlineto{\pgfqpoint{1.650722in}{3.535580in}}%
\pgfpathlineto{\pgfqpoint{1.651796in}{3.535722in}}%
\pgfpathlineto{\pgfqpoint{1.656091in}{3.538124in}}%
\pgfpathlineto{\pgfqpoint{1.657165in}{3.538690in}}%
\pgfpathlineto{\pgfqpoint{1.658239in}{3.540103in}}%
\pgfpathlineto{\pgfqpoint{1.659312in}{3.543212in}}%
\pgfpathlineto{\pgfqpoint{1.663608in}{3.543353in}}%
\pgfpathlineto{\pgfqpoint{1.664682in}{3.544343in}}%
\pgfpathlineto{\pgfqpoint{1.665755in}{3.544201in}}%
\pgfpathlineto{\pgfqpoint{1.666829in}{3.546604in}}%
\pgfpathlineto{\pgfqpoint{1.670051in}{3.546180in}}%
\pgfpathlineto{\pgfqpoint{1.671125in}{3.548724in}}%
\pgfpathlineto{\pgfqpoint{1.672198in}{3.554942in}}%
\pgfpathlineto{\pgfqpoint{1.673272in}{3.554659in}}%
\pgfpathlineto{\pgfqpoint{1.674346in}{3.555931in}}%
\pgfpathlineto{\pgfqpoint{1.677568in}{3.557345in}}%
\pgfpathlineto{\pgfqpoint{1.679715in}{3.551974in}}%
\pgfpathlineto{\pgfqpoint{1.680789in}{3.553811in}}%
\pgfpathlineto{\pgfqpoint{1.681863in}{3.549148in}}%
\pgfpathlineto{\pgfqpoint{1.685085in}{3.551692in}}%
\pgfpathlineto{\pgfqpoint{1.686158in}{3.546180in}}%
\pgfpathlineto{\pgfqpoint{1.687232in}{3.546321in}}%
\pgfpathlineto{\pgfqpoint{1.688306in}{3.548865in}}%
\pgfpathlineto{\pgfqpoint{1.689380in}{3.544625in}}%
\pgfpathlineto{\pgfqpoint{1.692601in}{3.549430in}}%
\pgfpathlineto{\pgfqpoint{1.693675in}{3.547734in}}%
\pgfpathlineto{\pgfqpoint{1.694749in}{3.551692in}}%
\pgfpathlineto{\pgfqpoint{1.695823in}{3.548017in}}%
\pgfpathlineto{\pgfqpoint{1.696897in}{3.548865in}}%
\pgfpathlineto{\pgfqpoint{1.700118in}{3.549713in}}%
\pgfpathlineto{\pgfqpoint{1.701192in}{3.547310in}}%
\pgfpathlineto{\pgfqpoint{1.702266in}{3.543071in}}%
\pgfpathlineto{\pgfqpoint{1.703340in}{3.541799in}}%
\pgfpathlineto{\pgfqpoint{1.704414in}{3.542505in}}%
\pgfpathlineto{\pgfqpoint{1.707635in}{3.545473in}}%
\pgfpathlineto{\pgfqpoint{1.708709in}{3.541799in}}%
\pgfpathlineto{\pgfqpoint{1.709783in}{3.542223in}}%
\pgfpathlineto{\pgfqpoint{1.710857in}{3.543353in}}%
\pgfpathlineto{\pgfqpoint{1.715152in}{3.546321in}}%
\pgfpathlineto{\pgfqpoint{1.716226in}{3.544343in}}%
\pgfpathlineto{\pgfqpoint{1.717300in}{3.544201in}}%
\pgfpathlineto{\pgfqpoint{1.718374in}{3.552822in}}%
\pgfpathlineto{\pgfqpoint{1.719447in}{3.585751in}}%
\pgfpathlineto{\pgfqpoint{1.722669in}{3.575293in}}%
\pgfpathlineto{\pgfqpoint{1.723743in}{3.576565in}}%
\pgfpathlineto{\pgfqpoint{1.725890in}{3.571194in}}%
\pgfpathlineto{\pgfqpoint{1.726964in}{3.570912in}}%
\pgfpathlineto{\pgfqpoint{1.730186in}{3.568085in}}%
\pgfpathlineto{\pgfqpoint{1.731260in}{3.563422in}}%
\pgfpathlineto{\pgfqpoint{1.732333in}{3.566813in}}%
\pgfpathlineto{\pgfqpoint{1.734481in}{3.565541in}}%
\pgfpathlineto{\pgfqpoint{1.737703in}{3.566389in}}%
\pgfpathlineto{\pgfqpoint{1.738776in}{3.569357in}}%
\pgfpathlineto{\pgfqpoint{1.740924in}{3.568792in}}%
\pgfpathlineto{\pgfqpoint{1.741998in}{3.571618in}}%
\pgfpathlineto{\pgfqpoint{1.745219in}{3.571053in}}%
\pgfpathlineto{\pgfqpoint{1.746293in}{3.566955in}}%
\pgfpathlineto{\pgfqpoint{1.747367in}{3.565683in}}%
\pgfpathlineto{\pgfqpoint{1.749515in}{3.572184in}}%
\pgfpathlineto{\pgfqpoint{1.752736in}{3.566955in}}%
\pgfpathlineto{\pgfqpoint{1.754884in}{3.570346in}}%
\pgfpathlineto{\pgfqpoint{1.755958in}{3.572749in}}%
\pgfpathlineto{\pgfqpoint{1.757032in}{3.571053in}}%
\pgfpathlineto{\pgfqpoint{1.761327in}{3.572042in}}%
\pgfpathlineto{\pgfqpoint{1.762401in}{3.575434in}}%
\pgfpathlineto{\pgfqpoint{1.763475in}{3.576423in}}%
\pgfpathlineto{\pgfqpoint{1.769918in}{3.574021in}}%
\pgfpathlineto{\pgfqpoint{1.770992in}{3.575293in}}%
\pgfpathlineto{\pgfqpoint{1.772065in}{3.571053in}}%
\pgfpathlineto{\pgfqpoint{1.776361in}{3.571760in}}%
\pgfpathlineto{\pgfqpoint{1.777435in}{3.574162in}}%
\pgfpathlineto{\pgfqpoint{1.778508in}{3.570912in}}%
\pgfpathlineto{\pgfqpoint{1.779582in}{3.571336in}}%
\pgfpathlineto{\pgfqpoint{1.782804in}{3.570770in}}%
\pgfpathlineto{\pgfqpoint{1.783878in}{3.571760in}}%
\pgfpathlineto{\pgfqpoint{1.784951in}{3.575293in}}%
\pgfpathlineto{\pgfqpoint{1.787099in}{3.572466in}}%
\pgfpathlineto{\pgfqpoint{1.790321in}{3.570346in}}%
\pgfpathlineto{\pgfqpoint{1.792468in}{3.571053in}}%
\pgfpathlineto{\pgfqpoint{1.793542in}{3.575010in}}%
\pgfpathlineto{\pgfqpoint{1.794616in}{3.573456in}}%
\pgfpathlineto{\pgfqpoint{1.797838in}{3.575575in}}%
\pgfpathlineto{\pgfqpoint{1.798911in}{3.577130in}}%
\pgfpathlineto{\pgfqpoint{1.801059in}{3.571053in}}%
\pgfpathlineto{\pgfqpoint{1.802133in}{3.571618in}}%
\pgfpathlineto{\pgfqpoint{1.806428in}{3.565400in}}%
\pgfpathlineto{\pgfqpoint{1.808576in}{3.567944in}}%
\pgfpathlineto{\pgfqpoint{1.812871in}{3.562291in}}%
\pgfpathlineto{\pgfqpoint{1.813945in}{3.564269in}}%
\pgfpathlineto{\pgfqpoint{1.815019in}{3.557345in}}%
\pgfpathlineto{\pgfqpoint{1.816093in}{3.558899in}}%
\pgfpathlineto{\pgfqpoint{1.817167in}{3.561867in}}%
\pgfpathlineto{\pgfqpoint{1.820388in}{3.564269in}}%
\pgfpathlineto{\pgfqpoint{1.824684in}{3.573456in}}%
\pgfpathlineto{\pgfqpoint{1.827905in}{3.572325in}}%
\pgfpathlineto{\pgfqpoint{1.831127in}{3.561726in}}%
\pgfpathlineto{\pgfqpoint{1.832200in}{3.555649in}}%
\pgfpathlineto{\pgfqpoint{1.835422in}{3.558051in}}%
\pgfpathlineto{\pgfqpoint{1.837570in}{3.561726in}}%
\pgfpathlineto{\pgfqpoint{1.838643in}{3.560030in}}%
\pgfpathlineto{\pgfqpoint{1.839717in}{3.559888in}}%
\pgfpathlineto{\pgfqpoint{1.842939in}{3.557062in}}%
\pgfpathlineto{\pgfqpoint{1.844013in}{3.557486in}}%
\pgfpathlineto{\pgfqpoint{1.845086in}{3.559888in}}%
\pgfpathlineto{\pgfqpoint{1.846160in}{3.559040in}}%
\pgfpathlineto{\pgfqpoint{1.847234in}{3.556073in}}%
\pgfpathlineto{\pgfqpoint{1.850456in}{3.561443in}}%
\pgfpathlineto{\pgfqpoint{1.851529in}{3.555083in}}%
\pgfpathlineto{\pgfqpoint{1.852603in}{3.556921in}}%
\pgfpathlineto{\pgfqpoint{1.853677in}{3.556073in}}%
\pgfpathlineto{\pgfqpoint{1.854751in}{3.559606in}}%
\pgfpathlineto{\pgfqpoint{1.857973in}{3.561160in}}%
\pgfpathlineto{\pgfqpoint{1.859046in}{3.559464in}}%
\pgfpathlineto{\pgfqpoint{1.860120in}{3.555366in}}%
\pgfpathlineto{\pgfqpoint{1.862268in}{3.541799in}}%
\pgfpathlineto{\pgfqpoint{1.865489in}{3.533037in}}%
\pgfpathlineto{\pgfqpoint{1.866563in}{3.525970in}}%
\pgfpathlineto{\pgfqpoint{1.869785in}{3.548582in}}%
\pgfpathlineto{\pgfqpoint{1.873006in}{3.544484in}}%
\pgfpathlineto{\pgfqpoint{1.874080in}{3.533178in}}%
\pgfpathlineto{\pgfqpoint{1.875154in}{3.541516in}}%
\pgfpathlineto{\pgfqpoint{1.876228in}{3.540809in}}%
\pgfpathlineto{\pgfqpoint{1.877302in}{3.534733in}}%
\pgfpathlineto{\pgfqpoint{1.881597in}{3.546180in}}%
\pgfpathlineto{\pgfqpoint{1.882671in}{3.541233in}}%
\pgfpathlineto{\pgfqpoint{1.883745in}{3.542788in}}%
\pgfpathlineto{\pgfqpoint{1.884818in}{3.546039in}}%
\pgfpathlineto{\pgfqpoint{1.888040in}{3.543919in}}%
\pgfpathlineto{\pgfqpoint{1.890188in}{3.557768in}}%
\pgfpathlineto{\pgfqpoint{1.891262in}{3.553529in}}%
\pgfpathlineto{\pgfqpoint{1.892335in}{3.546886in}}%
\pgfpathlineto{\pgfqpoint{1.895557in}{3.550420in}}%
\pgfpathlineto{\pgfqpoint{1.897705in}{3.550985in}}%
\pgfpathlineto{\pgfqpoint{1.898778in}{3.548300in}}%
\pgfpathlineto{\pgfqpoint{1.899852in}{3.548300in}}%
\pgfpathlineto{\pgfqpoint{1.903074in}{3.540951in}}%
\pgfpathlineto{\pgfqpoint{1.904148in}{3.544060in}}%
\pgfpathlineto{\pgfqpoint{1.905221in}{3.551974in}}%
\pgfpathlineto{\pgfqpoint{1.906295in}{3.551550in}}%
\pgfpathlineto{\pgfqpoint{1.907369in}{3.554942in}}%
\pgfpathlineto{\pgfqpoint{1.913812in}{3.585892in}}%
\pgfpathlineto{\pgfqpoint{1.914886in}{3.586457in}}%
\pgfpathlineto{\pgfqpoint{1.918107in}{3.586740in}}%
\pgfpathlineto{\pgfqpoint{1.920255in}{3.580804in}}%
\pgfpathlineto{\pgfqpoint{1.921329in}{3.585892in}}%
\pgfpathlineto{\pgfqpoint{1.922403in}{3.597481in}}%
\pgfpathlineto{\pgfqpoint{1.925624in}{3.597622in}}%
\pgfpathlineto{\pgfqpoint{1.926698in}{3.595078in}}%
\pgfpathlineto{\pgfqpoint{1.927772in}{3.595926in}}%
\pgfpathlineto{\pgfqpoint{1.928846in}{3.604688in}}%
\pgfpathlineto{\pgfqpoint{1.929920in}{3.603840in}}%
\pgfpathlineto{\pgfqpoint{1.933141in}{3.604264in}}%
\pgfpathlineto{\pgfqpoint{1.936363in}{3.601721in}}%
\pgfpathlineto{\pgfqpoint{1.937437in}{3.596774in}}%
\pgfpathlineto{\pgfqpoint{1.941732in}{3.604830in}}%
\pgfpathlineto{\pgfqpoint{1.942806in}{3.604264in}}%
\pgfpathlineto{\pgfqpoint{1.943880in}{3.605395in}}%
\pgfpathlineto{\pgfqpoint{1.944953in}{3.608787in}}%
\pgfpathlineto{\pgfqpoint{1.948175in}{3.606808in}}%
\pgfpathlineto{\pgfqpoint{1.950323in}{3.617832in}}%
\pgfpathlineto{\pgfqpoint{1.951396in}{3.611755in}}%
\pgfpathlineto{\pgfqpoint{1.952470in}{3.613168in}}%
\pgfpathlineto{\pgfqpoint{1.955692in}{3.614157in}}%
\pgfpathlineto{\pgfqpoint{1.956766in}{3.613592in}}%
\pgfpathlineto{\pgfqpoint{1.957840in}{3.615994in}}%
\pgfpathlineto{\pgfqpoint{1.958913in}{3.613027in}}%
\pgfpathlineto{\pgfqpoint{1.959987in}{3.617690in}}%
\pgfpathlineto{\pgfqpoint{1.963209in}{3.616842in}}%
\pgfpathlineto{\pgfqpoint{1.964283in}{3.617690in}}%
\pgfpathlineto{\pgfqpoint{1.965356in}{3.614157in}}%
\pgfpathlineto{\pgfqpoint{1.967504in}{3.614157in}}%
\pgfpathlineto{\pgfqpoint{1.970726in}{3.609070in}}%
\pgfpathlineto{\pgfqpoint{1.971799in}{3.611755in}}%
\pgfpathlineto{\pgfqpoint{1.972873in}{3.609352in}}%
\pgfpathlineto{\pgfqpoint{1.973947in}{3.610059in}}%
\pgfpathlineto{\pgfqpoint{1.975021in}{3.615712in}}%
\pgfpathlineto{\pgfqpoint{1.978242in}{3.614299in}}%
\pgfpathlineto{\pgfqpoint{1.979316in}{3.612037in}}%
\pgfpathlineto{\pgfqpoint{1.981464in}{3.617549in}}%
\pgfpathlineto{\pgfqpoint{1.982538in}{3.612885in}}%
\pgfpathlineto{\pgfqpoint{1.985759in}{3.612885in}}%
\pgfpathlineto{\pgfqpoint{1.986833in}{3.613592in}}%
\pgfpathlineto{\pgfqpoint{1.987907in}{3.621647in}}%
\pgfpathlineto{\pgfqpoint{1.990055in}{3.615853in}}%
\pgfpathlineto{\pgfqpoint{1.994350in}{3.618397in}}%
\pgfpathlineto{\pgfqpoint{1.995424in}{3.624050in}}%
\pgfpathlineto{\pgfqpoint{1.996498in}{3.622637in}}%
\pgfpathlineto{\pgfqpoint{2.000793in}{3.623485in}}%
\pgfpathlineto{\pgfqpoint{2.001867in}{3.628007in}}%
\pgfpathlineto{\pgfqpoint{2.002941in}{3.625322in}}%
\pgfpathlineto{\pgfqpoint{2.004015in}{3.626452in}}%
\pgfpathlineto{\pgfqpoint{2.008310in}{3.621082in}}%
\pgfpathlineto{\pgfqpoint{2.009384in}{3.621506in}}%
\pgfpathlineto{\pgfqpoint{2.010458in}{3.615570in}}%
\pgfpathlineto{\pgfqpoint{2.011531in}{3.599883in}}%
\pgfpathlineto{\pgfqpoint{2.012605in}{3.593665in}}%
\pgfpathlineto{\pgfqpoint{2.016901in}{3.595926in}}%
\pgfpathlineto{\pgfqpoint{2.017974in}{3.590980in}}%
\pgfpathlineto{\pgfqpoint{2.019048in}{3.601014in}}%
\pgfpathlineto{\pgfqpoint{2.020122in}{3.594089in}}%
\pgfpathlineto{\pgfqpoint{2.024417in}{3.594089in}}%
\pgfpathlineto{\pgfqpoint{2.025491in}{3.588153in}}%
\pgfpathlineto{\pgfqpoint{2.026565in}{3.595361in}}%
\pgfpathlineto{\pgfqpoint{2.027639in}{3.590980in}}%
\pgfpathlineto{\pgfqpoint{2.030861in}{3.588577in}}%
\pgfpathlineto{\pgfqpoint{2.031934in}{3.591828in}}%
\pgfpathlineto{\pgfqpoint{2.033008in}{3.588153in}}%
\pgfpathlineto{\pgfqpoint{2.034082in}{3.590697in}}%
\pgfpathlineto{\pgfqpoint{2.035156in}{3.601579in}}%
\pgfpathlineto{\pgfqpoint{2.038377in}{3.595926in}}%
\pgfpathlineto{\pgfqpoint{2.039451in}{3.590980in}}%
\pgfpathlineto{\pgfqpoint{2.041599in}{3.602427in}}%
\pgfpathlineto{\pgfqpoint{2.042673in}{3.594654in}}%
\pgfpathlineto{\pgfqpoint{2.045894in}{3.590132in}}%
\pgfpathlineto{\pgfqpoint{2.046968in}{3.591545in}}%
\pgfpathlineto{\pgfqpoint{2.048042in}{3.591828in}}%
\pgfpathlineto{\pgfqpoint{2.049116in}{3.581370in}}%
\pgfpathlineto{\pgfqpoint{2.050190in}{3.591263in}}%
\pgfpathlineto{\pgfqpoint{2.054485in}{3.598611in}}%
\pgfpathlineto{\pgfqpoint{2.055559in}{3.604406in}}%
\pgfpathlineto{\pgfqpoint{2.056633in}{3.601297in}}%
\pgfpathlineto{\pgfqpoint{2.057706in}{3.600590in}}%
\pgfpathlineto{\pgfqpoint{2.060928in}{3.605254in}}%
\pgfpathlineto{\pgfqpoint{2.063076in}{3.599742in}}%
\pgfpathlineto{\pgfqpoint{2.064150in}{3.605960in}}%
\pgfpathlineto{\pgfqpoint{2.065223in}{3.608080in}}%
\pgfpathlineto{\pgfqpoint{2.068445in}{3.604830in}}%
\pgfpathlineto{\pgfqpoint{2.069519in}{3.613875in}}%
\pgfpathlineto{\pgfqpoint{2.070593in}{3.617549in}}%
\pgfpathlineto{\pgfqpoint{2.071666in}{3.618114in}}%
\pgfpathlineto{\pgfqpoint{2.072740in}{3.621082in}}%
\pgfpathlineto{\pgfqpoint{2.075962in}{3.618962in}}%
\pgfpathlineto{\pgfqpoint{2.077036in}{3.616136in}}%
\pgfpathlineto{\pgfqpoint{2.078109in}{3.615994in}}%
\pgfpathlineto{\pgfqpoint{2.079183in}{3.614723in}}%
\pgfpathlineto{\pgfqpoint{2.080257in}{3.619528in}}%
\pgfpathlineto{\pgfqpoint{2.085626in}{3.617549in}}%
\pgfpathlineto{\pgfqpoint{2.086700in}{3.627159in}}%
\pgfpathlineto{\pgfqpoint{2.087774in}{3.626735in}}%
\pgfpathlineto{\pgfqpoint{2.090995in}{3.628714in}}%
\pgfpathlineto{\pgfqpoint{2.093143in}{3.628572in}}%
\pgfpathlineto{\pgfqpoint{2.094217in}{3.628996in}}%
\pgfpathlineto{\pgfqpoint{2.098512in}{3.633660in}}%
\pgfpathlineto{\pgfqpoint{2.099586in}{3.633519in}}%
\pgfpathlineto{\pgfqpoint{2.100660in}{3.637900in}}%
\pgfpathlineto{\pgfqpoint{2.101734in}{3.637335in}}%
\pgfpathlineto{\pgfqpoint{2.102808in}{3.639030in}}%
\pgfpathlineto{\pgfqpoint{2.109251in}{3.623202in}}%
\pgfpathlineto{\pgfqpoint{2.110325in}{3.625039in}}%
\pgfpathlineto{\pgfqpoint{2.113546in}{3.624050in}}%
\pgfpathlineto{\pgfqpoint{2.116768in}{3.627866in}}%
\pgfpathlineto{\pgfqpoint{2.121063in}{3.628431in}}%
\pgfpathlineto{\pgfqpoint{2.122137in}{3.629562in}}%
\pgfpathlineto{\pgfqpoint{2.123211in}{3.629562in}}%
\pgfpathlineto{\pgfqpoint{2.125358in}{3.624757in}}%
\pgfpathlineto{\pgfqpoint{2.128580in}{3.623767in}}%
\pgfpathlineto{\pgfqpoint{2.129654in}{3.626452in}}%
\pgfpathlineto{\pgfqpoint{2.130728in}{3.626876in}}%
\pgfpathlineto{\pgfqpoint{2.131801in}{3.626452in}}%
\pgfpathlineto{\pgfqpoint{2.132875in}{3.624615in}}%
\pgfpathlineto{\pgfqpoint{2.136097in}{3.626311in}}%
\pgfpathlineto{\pgfqpoint{2.137171in}{3.623202in}}%
\pgfpathlineto{\pgfqpoint{2.138244in}{3.616277in}}%
\pgfpathlineto{\pgfqpoint{2.139318in}{3.614016in}}%
\pgfpathlineto{\pgfqpoint{2.140392in}{3.616842in}}%
\pgfpathlineto{\pgfqpoint{2.143614in}{3.613733in}}%
\pgfpathlineto{\pgfqpoint{2.144687in}{3.621223in}}%
\pgfpathlineto{\pgfqpoint{2.145761in}{3.619528in}}%
\pgfpathlineto{\pgfqpoint{2.146835in}{3.616560in}}%
\pgfpathlineto{\pgfqpoint{2.147909in}{3.611048in}}%
\pgfpathlineto{\pgfqpoint{2.151130in}{3.614864in}}%
\pgfpathlineto{\pgfqpoint{2.152204in}{3.611896in}}%
\pgfpathlineto{\pgfqpoint{2.153278in}{3.610624in}}%
\pgfpathlineto{\pgfqpoint{2.154352in}{3.607515in}}%
\pgfpathlineto{\pgfqpoint{2.155426in}{3.610059in}}%
\pgfpathlineto{\pgfqpoint{2.158647in}{3.609070in}}%
\pgfpathlineto{\pgfqpoint{2.160795in}{3.616560in}}%
\pgfpathlineto{\pgfqpoint{2.161869in}{3.615712in}}%
\pgfpathlineto{\pgfqpoint{2.162943in}{3.616842in}}%
\pgfpathlineto{\pgfqpoint{2.167238in}{3.618256in}}%
\pgfpathlineto{\pgfqpoint{2.169386in}{3.615994in}}%
\pgfpathlineto{\pgfqpoint{2.170460in}{3.614723in}}%
\pgfpathlineto{\pgfqpoint{2.174755in}{3.617125in}}%
\pgfpathlineto{\pgfqpoint{2.175829in}{3.619245in}}%
\pgfpathlineto{\pgfqpoint{2.176903in}{3.618397in}}%
\pgfpathlineto{\pgfqpoint{2.177976in}{3.615853in}}%
\pgfpathlineto{\pgfqpoint{2.181198in}{3.613309in}}%
\pgfpathlineto{\pgfqpoint{2.182272in}{3.620799in}}%
\pgfpathlineto{\pgfqpoint{2.183346in}{3.622637in}}%
\pgfpathlineto{\pgfqpoint{2.184419in}{3.626170in}}%
\pgfpathlineto{\pgfqpoint{2.185493in}{3.625605in}}%
\pgfpathlineto{\pgfqpoint{2.189789in}{3.629844in}}%
\pgfpathlineto{\pgfqpoint{2.190862in}{3.627866in}}%
\pgfpathlineto{\pgfqpoint{2.191936in}{3.632953in}}%
\pgfpathlineto{\pgfqpoint{2.193010in}{3.615994in}}%
\pgfpathlineto{\pgfqpoint{2.196232in}{3.609776in}}%
\pgfpathlineto{\pgfqpoint{2.200527in}{3.636628in}}%
\pgfpathlineto{\pgfqpoint{2.204822in}{3.636063in}}%
\pgfpathlineto{\pgfqpoint{2.205896in}{3.639737in}}%
\pgfpathlineto{\pgfqpoint{2.206970in}{3.640726in}}%
\pgfpathlineto{\pgfqpoint{2.208044in}{3.645390in}}%
\pgfpathlineto{\pgfqpoint{2.212339in}{3.646097in}}%
\pgfpathlineto{\pgfqpoint{2.213413in}{3.647369in}}%
\pgfpathlineto{\pgfqpoint{2.215561in}{3.653870in}}%
\pgfpathlineto{\pgfqpoint{2.219856in}{3.654435in}}%
\pgfpathlineto{\pgfqpoint{2.222004in}{3.650195in}}%
\pgfpathlineto{\pgfqpoint{2.223078in}{3.643694in}}%
\pgfpathlineto{\pgfqpoint{2.230594in}{3.632247in}}%
\pgfpathlineto{\pgfqpoint{2.233816in}{3.632388in}}%
\pgfpathlineto{\pgfqpoint{2.234890in}{3.631116in}}%
\pgfpathlineto{\pgfqpoint{2.238111in}{3.634084in}}%
\pgfpathlineto{\pgfqpoint{2.248850in}{3.633519in}}%
\pgfpathlineto{\pgfqpoint{2.249924in}{3.632953in}}%
\pgfpathlineto{\pgfqpoint{2.252071in}{3.635921in}}%
\pgfpathlineto{\pgfqpoint{2.253145in}{3.633660in}}%
\pgfpathlineto{\pgfqpoint{2.256367in}{3.634508in}}%
\pgfpathlineto{\pgfqpoint{2.257440in}{3.633377in}}%
\pgfpathlineto{\pgfqpoint{2.260662in}{3.633377in}}%
\pgfpathlineto{\pgfqpoint{2.264957in}{3.635073in}}%
\pgfpathlineto{\pgfqpoint{2.266031in}{3.633519in}}%
\pgfpathlineto{\pgfqpoint{2.267105in}{3.633095in}}%
\pgfpathlineto{\pgfqpoint{2.268179in}{3.634084in}}%
\pgfpathlineto{\pgfqpoint{2.272474in}{3.631116in}}%
\pgfpathlineto{\pgfqpoint{2.274622in}{3.631116in}}%
\pgfpathlineto{\pgfqpoint{2.275696in}{3.619528in}}%
\pgfpathlineto{\pgfqpoint{2.278917in}{3.624191in}}%
\pgfpathlineto{\pgfqpoint{2.279991in}{3.616277in}}%
\pgfpathlineto{\pgfqpoint{2.281065in}{3.614440in}}%
\pgfpathlineto{\pgfqpoint{2.282139in}{3.617973in}}%
\pgfpathlineto{\pgfqpoint{2.286434in}{3.614016in}}%
\pgfpathlineto{\pgfqpoint{2.289656in}{3.621647in}}%
\pgfpathlineto{\pgfqpoint{2.290729in}{3.619669in}}%
\pgfpathlineto{\pgfqpoint{2.293951in}{3.615288in}}%
\pgfpathlineto{\pgfqpoint{2.295025in}{3.619528in}}%
\pgfpathlineto{\pgfqpoint{2.296099in}{3.619810in}}%
\pgfpathlineto{\pgfqpoint{2.297172in}{3.615288in}}%
\pgfpathlineto{\pgfqpoint{2.298246in}{3.616277in}}%
\pgfpathlineto{\pgfqpoint{2.301468in}{3.616560in}}%
\pgfpathlineto{\pgfqpoint{2.302542in}{3.614864in}}%
\pgfpathlineto{\pgfqpoint{2.303616in}{3.614864in}}%
\pgfpathlineto{\pgfqpoint{2.305763in}{3.609635in}}%
\pgfpathlineto{\pgfqpoint{2.308985in}{3.606808in}}%
\pgfpathlineto{\pgfqpoint{2.310059in}{3.607656in}}%
\pgfpathlineto{\pgfqpoint{2.311132in}{3.607374in}}%
\pgfpathlineto{\pgfqpoint{2.312206in}{3.605819in}}%
\pgfpathlineto{\pgfqpoint{2.313280in}{3.607232in}}%
\pgfpathlineto{\pgfqpoint{2.316502in}{3.606808in}}%
\pgfpathlineto{\pgfqpoint{2.318649in}{3.609352in}}%
\pgfpathlineto{\pgfqpoint{2.319723in}{3.609493in}}%
\pgfpathlineto{\pgfqpoint{2.320797in}{3.608363in}}%
\pgfpathlineto{\pgfqpoint{2.324018in}{3.607656in}}%
\pgfpathlineto{\pgfqpoint{2.325092in}{3.604264in}}%
\pgfpathlineto{\pgfqpoint{2.326166in}{3.606950in}}%
\pgfpathlineto{\pgfqpoint{2.327240in}{3.603982in}}%
\pgfpathlineto{\pgfqpoint{2.328314in}{3.611331in}}%
\pgfpathlineto{\pgfqpoint{2.331535in}{3.609917in}}%
\pgfpathlineto{\pgfqpoint{2.332609in}{3.607091in}}%
\pgfpathlineto{\pgfqpoint{2.334757in}{3.599601in}}%
\pgfpathlineto{\pgfqpoint{2.335831in}{3.601579in}}%
\pgfpathlineto{\pgfqpoint{2.339052in}{3.612461in}}%
\pgfpathlineto{\pgfqpoint{2.340126in}{3.613875in}}%
\pgfpathlineto{\pgfqpoint{2.341200in}{3.616418in}}%
\pgfpathlineto{\pgfqpoint{2.342274in}{3.626170in}}%
\pgfpathlineto{\pgfqpoint{2.343348in}{3.629986in}}%
\pgfpathlineto{\pgfqpoint{2.346569in}{3.627442in}}%
\pgfpathlineto{\pgfqpoint{2.347643in}{3.630410in}}%
\pgfpathlineto{\pgfqpoint{2.348717in}{3.630268in}}%
\pgfpathlineto{\pgfqpoint{2.349791in}{3.630975in}}%
\pgfpathlineto{\pgfqpoint{2.350864in}{3.629420in}}%
\pgfpathlineto{\pgfqpoint{2.354086in}{3.631964in}}%
\pgfpathlineto{\pgfqpoint{2.356234in}{3.637759in}}%
\pgfpathlineto{\pgfqpoint{2.358381in}{3.639030in}}%
\pgfpathlineto{\pgfqpoint{2.361603in}{3.636628in}}%
\pgfpathlineto{\pgfqpoint{2.363750in}{3.630551in}}%
\pgfpathlineto{\pgfqpoint{2.364824in}{3.638465in}}%
\pgfpathlineto{\pgfqpoint{2.365898in}{3.637759in}}%
\pgfpathlineto{\pgfqpoint{2.369120in}{3.634932in}}%
\pgfpathlineto{\pgfqpoint{2.370193in}{3.635639in}}%
\pgfpathlineto{\pgfqpoint{2.371267in}{3.641009in}}%
\pgfpathlineto{\pgfqpoint{2.372341in}{3.640161in}}%
\pgfpathlineto{\pgfqpoint{2.373415in}{3.643270in}}%
\pgfpathlineto{\pgfqpoint{2.376637in}{3.644259in}}%
\pgfpathlineto{\pgfqpoint{2.377710in}{3.642705in}}%
\pgfpathlineto{\pgfqpoint{2.379858in}{3.636769in}}%
\pgfpathlineto{\pgfqpoint{2.380932in}{3.642846in}}%
\pgfpathlineto{\pgfqpoint{2.384153in}{3.644966in}}%
\pgfpathlineto{\pgfqpoint{2.385227in}{3.649065in}}%
\pgfpathlineto{\pgfqpoint{2.387375in}{3.646803in}}%
\pgfpathlineto{\pgfqpoint{2.388449in}{3.647510in}}%
\pgfpathlineto{\pgfqpoint{2.392744in}{3.647793in}}%
\pgfpathlineto{\pgfqpoint{2.393818in}{3.645249in}}%
\pgfpathlineto{\pgfqpoint{2.394892in}{3.645390in}}%
\pgfpathlineto{\pgfqpoint{2.395966in}{3.643977in}}%
\pgfpathlineto{\pgfqpoint{2.401335in}{3.645249in}}%
\pgfpathlineto{\pgfqpoint{2.402409in}{3.642988in}}%
\pgfpathlineto{\pgfqpoint{2.403482in}{3.644118in}}%
\pgfpathlineto{\pgfqpoint{2.407778in}{3.641150in}}%
\pgfpathlineto{\pgfqpoint{2.408852in}{3.642422in}}%
\pgfpathlineto{\pgfqpoint{2.409926in}{3.641292in}}%
\pgfpathlineto{\pgfqpoint{2.417442in}{3.639030in}}%
\pgfpathlineto{\pgfqpoint{2.418516in}{3.630551in}}%
\pgfpathlineto{\pgfqpoint{2.421738in}{3.620799in}}%
\pgfpathlineto{\pgfqpoint{2.423885in}{3.628572in}}%
\pgfpathlineto{\pgfqpoint{2.424959in}{3.627866in}}%
\pgfpathlineto{\pgfqpoint{2.426033in}{3.624050in}}%
\pgfpathlineto{\pgfqpoint{2.429255in}{3.623343in}}%
\pgfpathlineto{\pgfqpoint{2.430328in}{3.620093in}}%
\pgfpathlineto{\pgfqpoint{2.436771in}{3.619669in}}%
\pgfpathlineto{\pgfqpoint{2.438919in}{3.616701in}}%
\pgfpathlineto{\pgfqpoint{2.441067in}{3.620376in}}%
\pgfpathlineto{\pgfqpoint{2.444288in}{3.624333in}}%
\pgfpathlineto{\pgfqpoint{2.445362in}{3.627442in}}%
\pgfpathlineto{\pgfqpoint{2.447510in}{3.629562in}}%
\pgfpathlineto{\pgfqpoint{2.448584in}{3.628572in}}%
\pgfpathlineto{\pgfqpoint{2.452879in}{3.630410in}}%
\pgfpathlineto{\pgfqpoint{2.453953in}{3.628007in}}%
\pgfpathlineto{\pgfqpoint{2.455027in}{3.627159in}}%
\pgfpathlineto{\pgfqpoint{2.456101in}{3.629279in}}%
\pgfpathlineto{\pgfqpoint{2.460396in}{3.624474in}}%
\pgfpathlineto{\pgfqpoint{2.461470in}{3.629279in}}%
\pgfpathlineto{\pgfqpoint{2.462544in}{3.629279in}}%
\pgfpathlineto{\pgfqpoint{2.463617in}{3.628431in}}%
\pgfpathlineto{\pgfqpoint{2.466839in}{3.626876in}}%
\pgfpathlineto{\pgfqpoint{2.470060in}{3.622637in}}%
\pgfpathlineto{\pgfqpoint{2.471134in}{3.630410in}}%
\pgfpathlineto{\pgfqpoint{2.474356in}{3.625181in}}%
\pgfpathlineto{\pgfqpoint{2.475430in}{3.621082in}}%
\pgfpathlineto{\pgfqpoint{2.476504in}{3.623909in}}%
\pgfpathlineto{\pgfqpoint{2.477577in}{3.623767in}}%
\pgfpathlineto{\pgfqpoint{2.478651in}{3.625322in}}%
\pgfpathlineto{\pgfqpoint{2.481873in}{3.623626in}}%
\pgfpathlineto{\pgfqpoint{2.482947in}{3.619245in}}%
\pgfpathlineto{\pgfqpoint{2.486168in}{3.623343in}}%
\pgfpathlineto{\pgfqpoint{2.489390in}{3.619810in}}%
\pgfpathlineto{\pgfqpoint{2.490463in}{3.622071in}}%
\pgfpathlineto{\pgfqpoint{2.491537in}{3.622919in}}%
\pgfpathlineto{\pgfqpoint{2.492611in}{3.625322in}}%
\pgfpathlineto{\pgfqpoint{2.493685in}{3.624333in}}%
\pgfpathlineto{\pgfqpoint{2.496906in}{3.625322in}}%
\pgfpathlineto{\pgfqpoint{2.497980in}{3.627159in}}%
\pgfpathlineto{\pgfqpoint{2.500128in}{3.626029in}}%
\pgfpathlineto{\pgfqpoint{2.501202in}{3.626735in}}%
\pgfpathlineto{\pgfqpoint{2.505497in}{3.627442in}}%
\pgfpathlineto{\pgfqpoint{2.507645in}{3.621365in}}%
\pgfpathlineto{\pgfqpoint{2.511940in}{3.622354in}}%
\pgfpathlineto{\pgfqpoint{2.515162in}{3.630268in}}%
\pgfpathlineto{\pgfqpoint{2.516236in}{3.621223in}}%
\pgfpathlineto{\pgfqpoint{2.519457in}{3.621223in}}%
\pgfpathlineto{\pgfqpoint{2.520531in}{3.619952in}}%
\pgfpathlineto{\pgfqpoint{2.522679in}{3.615288in}}%
\pgfpathlineto{\pgfqpoint{2.523752in}{3.614157in}}%
\pgfpathlineto{\pgfqpoint{2.526974in}{3.613451in}}%
\pgfpathlineto{\pgfqpoint{2.528048in}{3.614157in}}%
\pgfpathlineto{\pgfqpoint{2.529122in}{3.617125in}}%
\pgfpathlineto{\pgfqpoint{2.534491in}{3.615146in}}%
\pgfpathlineto{\pgfqpoint{2.536638in}{3.610483in}}%
\pgfpathlineto{\pgfqpoint{2.537712in}{3.612603in}}%
\pgfpathlineto{\pgfqpoint{2.538786in}{3.604971in}}%
\pgfpathlineto{\pgfqpoint{2.542008in}{3.603840in}}%
\pgfpathlineto{\pgfqpoint{2.543081in}{3.602145in}}%
\pgfpathlineto{\pgfqpoint{2.544155in}{3.594089in}}%
\pgfpathlineto{\pgfqpoint{2.545229in}{3.594937in}}%
\pgfpathlineto{\pgfqpoint{2.546303in}{3.602286in}}%
\pgfpathlineto{\pgfqpoint{2.549525in}{3.603840in}}%
\pgfpathlineto{\pgfqpoint{2.550598in}{3.605112in}}%
\pgfpathlineto{\pgfqpoint{2.552746in}{3.595078in}}%
\pgfpathlineto{\pgfqpoint{2.558115in}{3.593524in}}%
\pgfpathlineto{\pgfqpoint{2.559189in}{3.593806in}}%
\pgfpathlineto{\pgfqpoint{2.561337in}{3.600025in}}%
\pgfpathlineto{\pgfqpoint{2.564558in}{3.601297in}}%
\pgfpathlineto{\pgfqpoint{2.565632in}{3.600731in}}%
\pgfpathlineto{\pgfqpoint{2.566706in}{3.597481in}}%
\pgfpathlineto{\pgfqpoint{2.567780in}{3.596350in}}%
\pgfpathlineto{\pgfqpoint{2.572075in}{3.613451in}}%
\pgfpathlineto{\pgfqpoint{2.573149in}{3.607232in}}%
\pgfpathlineto{\pgfqpoint{2.574223in}{3.610341in}}%
\pgfpathlineto{\pgfqpoint{2.576370in}{3.620517in}}%
\pgfpathlineto{\pgfqpoint{2.579592in}{3.617973in}}%
\pgfpathlineto{\pgfqpoint{2.581740in}{3.604830in}}%
\pgfpathlineto{\pgfqpoint{2.582814in}{3.601862in}}%
\pgfpathlineto{\pgfqpoint{2.587109in}{3.602710in}}%
\pgfpathlineto{\pgfqpoint{2.588183in}{3.597481in}}%
\pgfpathlineto{\pgfqpoint{2.590330in}{3.595078in}}%
\pgfpathlineto{\pgfqpoint{2.591404in}{3.594937in}}%
\pgfpathlineto{\pgfqpoint{2.594626in}{3.600590in}}%
\pgfpathlineto{\pgfqpoint{2.596773in}{3.599318in}}%
\pgfpathlineto{\pgfqpoint{2.597847in}{3.585892in}}%
\pgfpathlineto{\pgfqpoint{2.598921in}{3.583914in}}%
\pgfpathlineto{\pgfqpoint{2.602143in}{3.582500in}}%
\pgfpathlineto{\pgfqpoint{2.604290in}{3.589425in}}%
\pgfpathlineto{\pgfqpoint{2.605364in}{3.592110in}}%
\pgfpathlineto{\pgfqpoint{2.606438in}{3.591969in}}%
\pgfpathlineto{\pgfqpoint{2.609659in}{3.592534in}}%
\pgfpathlineto{\pgfqpoint{2.611807in}{3.594089in}}%
\pgfpathlineto{\pgfqpoint{2.612881in}{3.590839in}}%
\pgfpathlineto{\pgfqpoint{2.613955in}{3.580804in}}%
\pgfpathlineto{\pgfqpoint{2.617176in}{3.574586in}}%
\pgfpathlineto{\pgfqpoint{2.618250in}{3.574728in}}%
\pgfpathlineto{\pgfqpoint{2.620398in}{3.579250in}}%
\pgfpathlineto{\pgfqpoint{2.621472in}{3.575858in}}%
\pgfpathlineto{\pgfqpoint{2.624693in}{3.576847in}}%
\pgfpathlineto{\pgfqpoint{2.625767in}{3.574728in}}%
\pgfpathlineto{\pgfqpoint{2.626841in}{3.575717in}}%
\pgfpathlineto{\pgfqpoint{2.627915in}{3.578826in}}%
\pgfpathlineto{\pgfqpoint{2.628989in}{3.579109in}}%
\pgfpathlineto{\pgfqpoint{2.633284in}{3.576282in}}%
\pgfpathlineto{\pgfqpoint{2.634358in}{3.578261in}}%
\pgfpathlineto{\pgfqpoint{2.635432in}{3.572890in}}%
\pgfpathlineto{\pgfqpoint{2.636505in}{3.571618in}}%
\pgfpathlineto{\pgfqpoint{2.639727in}{3.573738in}}%
\pgfpathlineto{\pgfqpoint{2.640801in}{3.570912in}}%
\pgfpathlineto{\pgfqpoint{2.641875in}{3.570346in}}%
\pgfpathlineto{\pgfqpoint{2.644022in}{3.563280in}}%
\pgfpathlineto{\pgfqpoint{2.647244in}{3.562432in}}%
\pgfpathlineto{\pgfqpoint{2.648318in}{3.563845in}}%
\pgfpathlineto{\pgfqpoint{2.649392in}{3.561160in}}%
\pgfpathlineto{\pgfqpoint{2.650465in}{3.560030in}}%
\pgfpathlineto{\pgfqpoint{2.651539in}{3.562432in}}%
\pgfpathlineto{\pgfqpoint{2.655835in}{3.561867in}}%
\pgfpathlineto{\pgfqpoint{2.656908in}{3.559747in}}%
\pgfpathlineto{\pgfqpoint{2.657982in}{3.563280in}}%
\pgfpathlineto{\pgfqpoint{2.659056in}{3.570912in}}%
\pgfpathlineto{\pgfqpoint{2.663351in}{3.565824in}}%
\pgfpathlineto{\pgfqpoint{2.664425in}{3.568085in}}%
\pgfpathlineto{\pgfqpoint{2.665499in}{3.556497in}}%
\pgfpathlineto{\pgfqpoint{2.666573in}{3.553811in}}%
\pgfpathlineto{\pgfqpoint{2.669794in}{3.552539in}}%
\pgfpathlineto{\pgfqpoint{2.673016in}{3.559464in}}%
\pgfpathlineto{\pgfqpoint{2.674090in}{3.558334in}}%
\pgfpathlineto{\pgfqpoint{2.677311in}{3.565259in}}%
\pgfpathlineto{\pgfqpoint{2.678385in}{3.561867in}}%
\pgfpathlineto{\pgfqpoint{2.679459in}{3.563422in}}%
\pgfpathlineto{\pgfqpoint{2.680533in}{3.569075in}}%
\pgfpathlineto{\pgfqpoint{2.681607in}{3.570629in}}%
\pgfpathlineto{\pgfqpoint{2.684828in}{3.573738in}}%
\pgfpathlineto{\pgfqpoint{2.685902in}{3.571336in}}%
\pgfpathlineto{\pgfqpoint{2.686976in}{3.564128in}}%
\pgfpathlineto{\pgfqpoint{2.689124in}{3.561584in}}%
\pgfpathlineto{\pgfqpoint{2.692345in}{3.566672in}}%
\pgfpathlineto{\pgfqpoint{2.693419in}{3.569640in}}%
\pgfpathlineto{\pgfqpoint{2.694493in}{3.565541in}}%
\pgfpathlineto{\pgfqpoint{2.695567in}{3.566248in}}%
\pgfpathlineto{\pgfqpoint{2.696640in}{3.564269in}}%
\pgfpathlineto{\pgfqpoint{2.699862in}{3.551833in}}%
\pgfpathlineto{\pgfqpoint{2.700936in}{3.550985in}}%
\pgfpathlineto{\pgfqpoint{2.702010in}{3.547169in}}%
\pgfpathlineto{\pgfqpoint{2.704157in}{3.546039in}}%
\pgfpathlineto{\pgfqpoint{2.707379in}{3.550985in}}%
\pgfpathlineto{\pgfqpoint{2.708453in}{3.548724in}}%
\pgfpathlineto{\pgfqpoint{2.709526in}{3.547876in}}%
\pgfpathlineto{\pgfqpoint{2.711674in}{3.557062in}}%
\pgfpathlineto{\pgfqpoint{2.717043in}{3.526818in}}%
\pgfpathlineto{\pgfqpoint{2.718117in}{3.524416in}}%
\pgfpathlineto{\pgfqpoint{2.719191in}{3.517491in}}%
\pgfpathlineto{\pgfqpoint{2.722413in}{3.512544in}}%
\pgfpathlineto{\pgfqpoint{2.724560in}{3.507457in}}%
\pgfpathlineto{\pgfqpoint{2.725634in}{3.506467in}}%
\pgfpathlineto{\pgfqpoint{2.726708in}{3.509011in}}%
\pgfpathlineto{\pgfqpoint{2.729929in}{3.508870in}}%
\pgfpathlineto{\pgfqpoint{2.731003in}{3.510001in}}%
\pgfpathlineto{\pgfqpoint{2.733151in}{3.507174in}}%
\pgfpathlineto{\pgfqpoint{2.734225in}{3.513675in}}%
\pgfpathlineto{\pgfqpoint{2.737446in}{3.494455in}}%
\pgfpathlineto{\pgfqpoint{2.738520in}{3.479898in}}%
\pgfpathlineto{\pgfqpoint{2.739594in}{3.484562in}}%
\pgfpathlineto{\pgfqpoint{2.741742in}{3.483997in}}%
\pgfpathlineto{\pgfqpoint{2.744963in}{3.481029in}}%
\pgfpathlineto{\pgfqpoint{2.746037in}{3.479050in}}%
\pgfpathlineto{\pgfqpoint{2.747111in}{3.483149in}}%
\pgfpathlineto{\pgfqpoint{2.749258in}{3.483714in}}%
\pgfpathlineto{\pgfqpoint{2.752480in}{3.482725in}}%
\pgfpathlineto{\pgfqpoint{2.753554in}{3.486541in}}%
\pgfpathlineto{\pgfqpoint{2.754628in}{3.487530in}}%
\pgfpathlineto{\pgfqpoint{2.755702in}{3.484986in}}%
\pgfpathlineto{\pgfqpoint{2.756775in}{3.479616in}}%
\pgfpathlineto{\pgfqpoint{2.759997in}{3.480605in}}%
\pgfpathlineto{\pgfqpoint{2.761071in}{3.478061in}}%
\pgfpathlineto{\pgfqpoint{2.764292in}{3.479050in}}%
\pgfpathlineto{\pgfqpoint{2.767514in}{3.478344in}}%
\pgfpathlineto{\pgfqpoint{2.768588in}{3.481736in}}%
\pgfpathlineto{\pgfqpoint{2.770735in}{3.478202in}}%
\pgfpathlineto{\pgfqpoint{2.771809in}{3.480464in}}%
\pgfpathlineto{\pgfqpoint{2.775031in}{3.479757in}}%
\pgfpathlineto{\pgfqpoint{2.777178in}{3.475659in}}%
\pgfpathlineto{\pgfqpoint{2.783621in}{3.476931in}}%
\pgfpathlineto{\pgfqpoint{2.785769in}{3.476083in}}%
\pgfpathlineto{\pgfqpoint{2.786843in}{3.477213in}}%
\pgfpathlineto{\pgfqpoint{2.786843in}{3.477213in}}%
\pgfusepath{stroke}%
\end{pgfscope}%
\begin{pgfscope}%
\pgfpathrectangle{\pgfqpoint{0.320934in}{3.271772in}}{\pgfqpoint{2.583333in}{0.400885in}}%
\pgfusepath{clip}%
\pgfsetroundcap%
\pgfsetroundjoin%
\pgfsetlinewidth{1.505625pt}%
\definecolor{currentstroke}{rgb}{0.172549,0.627451,0.172549}%
\pgfsetstrokecolor{currentstroke}%
\pgfsetdash{}{0pt}%
\pgfpathmoveto{\pgfqpoint{0.438358in}{3.442023in}}%
\pgfpathlineto{\pgfqpoint{0.439432in}{3.439903in}}%
\pgfpathlineto{\pgfqpoint{0.440506in}{3.439903in}}%
\pgfpathlineto{\pgfqpoint{0.441580in}{3.438796in}}%
\pgfpathlineto{\pgfqpoint{0.444801in}{3.436607in}}%
\pgfpathlineto{\pgfqpoint{0.445875in}{3.436607in}}%
\pgfpathlineto{\pgfqpoint{0.446949in}{3.434990in}}%
\pgfpathlineto{\pgfqpoint{0.449096in}{3.434460in}}%
\pgfpathlineto{\pgfqpoint{0.453392in}{3.434460in}}%
\pgfpathlineto{\pgfqpoint{0.454466in}{3.431667in}}%
\pgfpathlineto{\pgfqpoint{0.455539in}{3.430504in}}%
\pgfpathlineto{\pgfqpoint{0.460909in}{3.430504in}}%
\pgfpathlineto{\pgfqpoint{0.461982in}{3.427653in}}%
\pgfpathlineto{\pgfqpoint{0.474869in}{3.427653in}}%
\pgfpathlineto{\pgfqpoint{0.477016in}{3.425894in}}%
\pgfpathlineto{\pgfqpoint{0.478090in}{3.426884in}}%
\pgfpathlineto{\pgfqpoint{0.485607in}{3.426884in}}%
\pgfpathlineto{\pgfqpoint{0.486681in}{3.424366in}}%
\pgfpathlineto{\pgfqpoint{0.493124in}{3.422537in}}%
\pgfpathlineto{\pgfqpoint{0.497419in}{3.424705in}}%
\pgfpathlineto{\pgfqpoint{0.498493in}{3.423845in}}%
\pgfpathlineto{\pgfqpoint{0.499567in}{3.423845in}}%
\pgfpathlineto{\pgfqpoint{0.501714in}{3.423233in}}%
\pgfpathlineto{\pgfqpoint{0.504936in}{3.423233in}}%
\pgfpathlineto{\pgfqpoint{0.506010in}{3.419287in}}%
\pgfpathlineto{\pgfqpoint{0.509231in}{3.419287in}}%
\pgfpathlineto{\pgfqpoint{0.512453in}{3.418452in}}%
\pgfpathlineto{\pgfqpoint{0.513527in}{3.414317in}}%
\pgfpathlineto{\pgfqpoint{0.514601in}{3.412628in}}%
\pgfpathlineto{\pgfqpoint{0.515674in}{3.409539in}}%
\pgfpathlineto{\pgfqpoint{0.516748in}{3.409220in}}%
\pgfpathlineto{\pgfqpoint{0.519970in}{3.409114in}}%
\pgfpathlineto{\pgfqpoint{0.521044in}{3.410277in}}%
\pgfpathlineto{\pgfqpoint{0.522117in}{3.410277in}}%
\pgfpathlineto{\pgfqpoint{0.523191in}{3.412101in}}%
\pgfpathlineto{\pgfqpoint{0.524265in}{3.412649in}}%
\pgfpathlineto{\pgfqpoint{0.527487in}{3.410441in}}%
\pgfpathlineto{\pgfqpoint{0.529634in}{3.410763in}}%
\pgfpathlineto{\pgfqpoint{0.530708in}{3.411195in}}%
\pgfpathlineto{\pgfqpoint{0.531782in}{3.410218in}}%
\pgfpathlineto{\pgfqpoint{0.536077in}{3.411078in}}%
\pgfpathlineto{\pgfqpoint{0.537151in}{3.412921in}}%
\pgfpathlineto{\pgfqpoint{0.560776in}{3.412921in}}%
\pgfpathlineto{\pgfqpoint{0.561849in}{3.411583in}}%
\pgfpathlineto{\pgfqpoint{0.565071in}{3.411583in}}%
\pgfpathlineto{\pgfqpoint{0.566145in}{3.409810in}}%
\pgfpathlineto{\pgfqpoint{0.569366in}{3.410027in}}%
\pgfpathlineto{\pgfqpoint{0.576883in}{3.410027in}}%
\pgfpathlineto{\pgfqpoint{0.582252in}{3.404714in}}%
\pgfpathlineto{\pgfqpoint{0.598360in}{3.404714in}}%
\pgfpathlineto{\pgfqpoint{0.599434in}{3.400140in}}%
\pgfpathlineto{\pgfqpoint{0.602655in}{3.396873in}}%
\pgfpathlineto{\pgfqpoint{0.603729in}{3.397635in}}%
\pgfpathlineto{\pgfqpoint{0.612320in}{3.397635in}}%
\pgfpathlineto{\pgfqpoint{0.614468in}{3.392397in}}%
\pgfpathlineto{\pgfqpoint{0.617689in}{3.394213in}}%
\pgfpathlineto{\pgfqpoint{0.618763in}{3.392351in}}%
\pgfpathlineto{\pgfqpoint{0.619837in}{3.391682in}}%
\pgfpathlineto{\pgfqpoint{0.620911in}{3.394522in}}%
\pgfpathlineto{\pgfqpoint{0.621984in}{3.392454in}}%
\pgfpathlineto{\pgfqpoint{0.625206in}{3.394561in}}%
\pgfpathlineto{\pgfqpoint{0.628427in}{3.389537in}}%
\pgfpathlineto{\pgfqpoint{0.629501in}{3.385029in}}%
\pgfpathlineto{\pgfqpoint{0.633797in}{3.387718in}}%
\pgfpathlineto{\pgfqpoint{0.635944in}{3.388436in}}%
\pgfpathlineto{\pgfqpoint{0.637018in}{3.390703in}}%
\pgfpathlineto{\pgfqpoint{0.643461in}{3.390041in}}%
\pgfpathlineto{\pgfqpoint{0.644535in}{3.387644in}}%
\pgfpathlineto{\pgfqpoint{0.647757in}{3.387644in}}%
\pgfpathlineto{\pgfqpoint{0.649904in}{3.385784in}}%
\pgfpathlineto{\pgfqpoint{0.650978in}{3.386148in}}%
\pgfpathlineto{\pgfqpoint{0.655273in}{3.384053in}}%
\pgfpathlineto{\pgfqpoint{0.656347in}{3.384942in}}%
\pgfpathlineto{\pgfqpoint{0.657421in}{3.384672in}}%
\pgfpathlineto{\pgfqpoint{0.659569in}{3.378524in}}%
\pgfpathlineto{\pgfqpoint{0.664938in}{3.379677in}}%
\pgfpathlineto{\pgfqpoint{0.666012in}{3.381011in}}%
\pgfpathlineto{\pgfqpoint{0.667086in}{3.378203in}}%
\pgfpathlineto{\pgfqpoint{0.670307in}{3.378121in}}%
\pgfpathlineto{\pgfqpoint{0.671381in}{3.377143in}}%
\pgfpathlineto{\pgfqpoint{0.672455in}{3.377866in}}%
\pgfpathlineto{\pgfqpoint{0.674602in}{3.377298in}}%
\pgfpathlineto{\pgfqpoint{0.679972in}{3.378186in}}%
\pgfpathlineto{\pgfqpoint{0.681045in}{3.377615in}}%
\pgfpathlineto{\pgfqpoint{0.688562in}{3.380160in}}%
\pgfpathlineto{\pgfqpoint{0.689636in}{3.379151in}}%
\pgfpathlineto{\pgfqpoint{0.695005in}{3.378984in}}%
\pgfpathlineto{\pgfqpoint{0.696079in}{3.380141in}}%
\pgfpathlineto{\pgfqpoint{0.697153in}{3.379720in}}%
\pgfpathlineto{\pgfqpoint{0.701448in}{3.380973in}}%
\pgfpathlineto{\pgfqpoint{0.702522in}{3.380037in}}%
\pgfpathlineto{\pgfqpoint{0.703596in}{3.375839in}}%
\pgfpathlineto{\pgfqpoint{0.704670in}{3.374105in}}%
\pgfpathlineto{\pgfqpoint{0.707891in}{3.374796in}}%
\pgfpathlineto{\pgfqpoint{0.708965in}{3.374098in}}%
\pgfpathlineto{\pgfqpoint{0.711113in}{3.371587in}}%
\pgfpathlineto{\pgfqpoint{0.712187in}{3.371145in}}%
\pgfpathlineto{\pgfqpoint{0.715408in}{3.371438in}}%
\pgfpathlineto{\pgfqpoint{0.719704in}{3.367839in}}%
\pgfpathlineto{\pgfqpoint{0.723999in}{3.369022in}}%
\pgfpathlineto{\pgfqpoint{0.725073in}{3.370154in}}%
\pgfpathlineto{\pgfqpoint{0.726147in}{3.366695in}}%
\pgfpathlineto{\pgfqpoint{0.727221in}{3.366763in}}%
\pgfpathlineto{\pgfqpoint{0.733664in}{3.365542in}}%
\pgfpathlineto{\pgfqpoint{0.734737in}{3.364674in}}%
\pgfpathlineto{\pgfqpoint{0.737959in}{3.365661in}}%
\pgfpathlineto{\pgfqpoint{0.740107in}{3.368232in}}%
\pgfpathlineto{\pgfqpoint{0.742254in}{3.367952in}}%
\pgfpathlineto{\pgfqpoint{0.746550in}{3.367117in}}%
\pgfpathlineto{\pgfqpoint{0.747623in}{3.365675in}}%
\pgfpathlineto{\pgfqpoint{0.748697in}{3.366211in}}%
\pgfpathlineto{\pgfqpoint{0.749771in}{3.370268in}}%
\pgfpathlineto{\pgfqpoint{0.752993in}{3.372078in}}%
\pgfpathlineto{\pgfqpoint{0.754067in}{3.374538in}}%
\pgfpathlineto{\pgfqpoint{0.762657in}{3.374616in}}%
\pgfpathlineto{\pgfqpoint{0.763731in}{3.372891in}}%
\pgfpathlineto{\pgfqpoint{0.764805in}{3.373043in}}%
\pgfpathlineto{\pgfqpoint{0.768026in}{3.372508in}}%
\pgfpathlineto{\pgfqpoint{0.769100in}{3.371446in}}%
\pgfpathlineto{\pgfqpoint{0.800242in}{3.371446in}}%
\pgfpathlineto{\pgfqpoint{0.802389in}{3.370088in}}%
\pgfpathlineto{\pgfqpoint{0.805611in}{3.370459in}}%
\pgfpathlineto{\pgfqpoint{0.806685in}{3.369787in}}%
\pgfpathlineto{\pgfqpoint{0.807758in}{3.368236in}}%
\pgfpathlineto{\pgfqpoint{0.808832in}{3.369100in}}%
\pgfpathlineto{\pgfqpoint{0.809906in}{3.369100in}}%
\pgfpathlineto{\pgfqpoint{0.813128in}{3.367348in}}%
\pgfpathlineto{\pgfqpoint{0.814201in}{3.368696in}}%
\pgfpathlineto{\pgfqpoint{0.828161in}{3.368696in}}%
\pgfpathlineto{\pgfqpoint{0.830309in}{3.366690in}}%
\pgfpathlineto{\pgfqpoint{0.831383in}{3.366690in}}%
\pgfpathlineto{\pgfqpoint{0.832457in}{3.366108in}}%
\pgfpathlineto{\pgfqpoint{0.845343in}{3.366108in}}%
\pgfpathlineto{\pgfqpoint{0.846417in}{3.365096in}}%
\pgfpathlineto{\pgfqpoint{0.847490in}{3.360975in}}%
\pgfpathlineto{\pgfqpoint{0.853934in}{3.360971in}}%
\pgfpathlineto{\pgfqpoint{0.855007in}{3.359712in}}%
\pgfpathlineto{\pgfqpoint{0.859303in}{3.358675in}}%
\pgfpathlineto{\pgfqpoint{0.860377in}{3.360011in}}%
\pgfpathlineto{\pgfqpoint{0.861450in}{3.359750in}}%
\pgfpathlineto{\pgfqpoint{0.862524in}{3.357999in}}%
\pgfpathlineto{\pgfqpoint{0.865746in}{3.359508in}}%
\pgfpathlineto{\pgfqpoint{0.866820in}{3.358345in}}%
\pgfpathlineto{\pgfqpoint{0.867893in}{3.358851in}}%
\pgfpathlineto{\pgfqpoint{0.870041in}{3.358596in}}%
\pgfpathlineto{\pgfqpoint{0.873263in}{3.358787in}}%
\pgfpathlineto{\pgfqpoint{0.874336in}{3.358149in}}%
\pgfpathlineto{\pgfqpoint{0.875410in}{3.354175in}}%
\pgfpathlineto{\pgfqpoint{0.876484in}{3.354116in}}%
\pgfpathlineto{\pgfqpoint{0.877558in}{3.354644in}}%
\pgfpathlineto{\pgfqpoint{0.881853in}{3.352512in}}%
\pgfpathlineto{\pgfqpoint{0.882927in}{3.354049in}}%
\pgfpathlineto{\pgfqpoint{0.885075in}{3.353230in}}%
\pgfpathlineto{\pgfqpoint{0.888296in}{3.355886in}}%
\pgfpathlineto{\pgfqpoint{0.890444in}{3.353247in}}%
\pgfpathlineto{\pgfqpoint{0.892592in}{3.354058in}}%
\pgfpathlineto{\pgfqpoint{0.895813in}{3.353706in}}%
\pgfpathlineto{\pgfqpoint{0.896887in}{3.352248in}}%
\pgfpathlineto{\pgfqpoint{0.900109in}{3.351401in}}%
\pgfpathlineto{\pgfqpoint{0.903330in}{3.352071in}}%
\pgfpathlineto{\pgfqpoint{0.904404in}{3.353032in}}%
\pgfpathlineto{\pgfqpoint{0.906552in}{3.351771in}}%
\pgfpathlineto{\pgfqpoint{0.907625in}{3.352840in}}%
\pgfpathlineto{\pgfqpoint{0.910847in}{3.353701in}}%
\pgfpathlineto{\pgfqpoint{0.912995in}{3.352771in}}%
\pgfpathlineto{\pgfqpoint{0.914068in}{3.353516in}}%
\pgfpathlineto{\pgfqpoint{0.915142in}{3.353167in}}%
\pgfpathlineto{\pgfqpoint{0.921585in}{3.354328in}}%
\pgfpathlineto{\pgfqpoint{0.925881in}{3.354506in}}%
\pgfpathlineto{\pgfqpoint{0.926955in}{3.353262in}}%
\pgfpathlineto{\pgfqpoint{0.928028in}{3.354824in}}%
\pgfpathlineto{\pgfqpoint{0.929102in}{3.354467in}}%
\pgfpathlineto{\pgfqpoint{0.930176in}{3.355118in}}%
\pgfpathlineto{\pgfqpoint{0.934471in}{3.354515in}}%
\pgfpathlineto{\pgfqpoint{0.935545in}{3.352085in}}%
\pgfpathlineto{\pgfqpoint{0.936619in}{3.352085in}}%
\pgfpathlineto{\pgfqpoint{0.937693in}{3.352652in}}%
\pgfpathlineto{\pgfqpoint{0.940914in}{3.355572in}}%
\pgfpathlineto{\pgfqpoint{0.941988in}{3.354180in}}%
\pgfpathlineto{\pgfqpoint{0.943062in}{3.355772in}}%
\pgfpathlineto{\pgfqpoint{0.945210in}{3.356197in}}%
\pgfpathlineto{\pgfqpoint{0.951653in}{3.356197in}}%
\pgfpathlineto{\pgfqpoint{0.952727in}{3.354933in}}%
\pgfpathlineto{\pgfqpoint{0.958096in}{3.354563in}}%
\pgfpathlineto{\pgfqpoint{0.960244in}{3.352544in}}%
\pgfpathlineto{\pgfqpoint{0.964539in}{3.352009in}}%
\pgfpathlineto{\pgfqpoint{0.965613in}{3.350479in}}%
\pgfpathlineto{\pgfqpoint{0.966687in}{3.351508in}}%
\pgfpathlineto{\pgfqpoint{0.967760in}{3.350983in}}%
\pgfpathlineto{\pgfqpoint{0.970982in}{3.351214in}}%
\pgfpathlineto{\pgfqpoint{0.975277in}{3.348481in}}%
\pgfpathlineto{\pgfqpoint{0.979573in}{3.347605in}}%
\pgfpathlineto{\pgfqpoint{0.980646in}{3.346741in}}%
\pgfpathlineto{\pgfqpoint{0.982794in}{3.348131in}}%
\pgfpathlineto{\pgfqpoint{0.989237in}{3.347857in}}%
\pgfpathlineto{\pgfqpoint{0.990311in}{3.349052in}}%
\pgfpathlineto{\pgfqpoint{0.994606in}{3.347553in}}%
\pgfpathlineto{\pgfqpoint{0.995680in}{3.349012in}}%
\pgfpathlineto{\pgfqpoint{0.996754in}{3.348734in}}%
\pgfpathlineto{\pgfqpoint{0.997828in}{3.346632in}}%
\pgfpathlineto{\pgfqpoint{1.001049in}{3.347004in}}%
\pgfpathlineto{\pgfqpoint{1.003197in}{3.348186in}}%
\pgfpathlineto{\pgfqpoint{1.004271in}{3.347419in}}%
\pgfpathlineto{\pgfqpoint{1.005345in}{3.348066in}}%
\pgfpathlineto{\pgfqpoint{1.008566in}{3.346974in}}%
\pgfpathlineto{\pgfqpoint{1.009640in}{3.344622in}}%
\pgfpathlineto{\pgfqpoint{1.011788in}{3.348308in}}%
\pgfpathlineto{\pgfqpoint{1.012862in}{3.347813in}}%
\pgfpathlineto{\pgfqpoint{1.016083in}{3.349663in}}%
\pgfpathlineto{\pgfqpoint{1.019305in}{3.347932in}}%
\pgfpathlineto{\pgfqpoint{1.020378in}{3.348478in}}%
\pgfpathlineto{\pgfqpoint{1.023600in}{3.347871in}}%
\pgfpathlineto{\pgfqpoint{1.024674in}{3.349724in}}%
\pgfpathlineto{\pgfqpoint{1.025748in}{3.349724in}}%
\pgfpathlineto{\pgfqpoint{1.027895in}{3.348254in}}%
\pgfpathlineto{\pgfqpoint{1.031117in}{3.347869in}}%
\pgfpathlineto{\pgfqpoint{1.032191in}{3.346560in}}%
\pgfpathlineto{\pgfqpoint{1.033265in}{3.346932in}}%
\pgfpathlineto{\pgfqpoint{1.034338in}{3.345219in}}%
\pgfpathlineto{\pgfqpoint{1.035412in}{3.345943in}}%
\pgfpathlineto{\pgfqpoint{1.038634in}{3.346521in}}%
\pgfpathlineto{\pgfqpoint{1.039708in}{3.347317in}}%
\pgfpathlineto{\pgfqpoint{1.041855in}{3.346511in}}%
\pgfpathlineto{\pgfqpoint{1.042929in}{3.341890in}}%
\pgfpathlineto{\pgfqpoint{1.046151in}{3.341358in}}%
\pgfpathlineto{\pgfqpoint{1.048298in}{3.342271in}}%
\pgfpathlineto{\pgfqpoint{1.050446in}{3.342173in}}%
\pgfpathlineto{\pgfqpoint{1.055815in}{3.343250in}}%
\pgfpathlineto{\pgfqpoint{1.057963in}{3.341961in}}%
\pgfpathlineto{\pgfqpoint{1.061184in}{3.342640in}}%
\pgfpathlineto{\pgfqpoint{1.062258in}{3.343476in}}%
\pgfpathlineto{\pgfqpoint{1.064406in}{3.343376in}}%
\pgfpathlineto{\pgfqpoint{1.065480in}{3.343725in}}%
\pgfpathlineto{\pgfqpoint{1.069775in}{3.343876in}}%
\pgfpathlineto{\pgfqpoint{1.071923in}{3.344686in}}%
\pgfpathlineto{\pgfqpoint{1.079440in}{3.345606in}}%
\pgfpathlineto{\pgfqpoint{1.080513in}{3.345606in}}%
\pgfpathlineto{\pgfqpoint{1.083735in}{3.346286in}}%
\pgfpathlineto{\pgfqpoint{1.084809in}{3.348089in}}%
\pgfpathlineto{\pgfqpoint{1.088030in}{3.348252in}}%
\pgfpathlineto{\pgfqpoint{1.092326in}{3.348638in}}%
\pgfpathlineto{\pgfqpoint{1.094473in}{3.348193in}}%
\pgfpathlineto{\pgfqpoint{1.095547in}{3.348193in}}%
\pgfpathlineto{\pgfqpoint{1.098769in}{3.347201in}}%
\pgfpathlineto{\pgfqpoint{1.100916in}{3.344215in}}%
\pgfpathlineto{\pgfqpoint{1.103064in}{3.345493in}}%
\pgfpathlineto{\pgfqpoint{1.106286in}{3.343978in}}%
\pgfpathlineto{\pgfqpoint{1.108433in}{3.341133in}}%
\pgfpathlineto{\pgfqpoint{1.109507in}{3.341944in}}%
\pgfpathlineto{\pgfqpoint{1.110581in}{3.343693in}}%
\pgfpathlineto{\pgfqpoint{1.114876in}{3.342436in}}%
\pgfpathlineto{\pgfqpoint{1.121319in}{3.344124in}}%
\pgfpathlineto{\pgfqpoint{1.123467in}{3.342359in}}%
\pgfpathlineto{\pgfqpoint{1.125615in}{3.343490in}}%
\pgfpathlineto{\pgfqpoint{1.128836in}{3.343892in}}%
\pgfpathlineto{\pgfqpoint{1.130984in}{3.345420in}}%
\pgfpathlineto{\pgfqpoint{1.132058in}{3.342599in}}%
\pgfpathlineto{\pgfqpoint{1.133132in}{3.342007in}}%
\pgfpathlineto{\pgfqpoint{1.136353in}{3.342055in}}%
\pgfpathlineto{\pgfqpoint{1.137427in}{3.342836in}}%
\pgfpathlineto{\pgfqpoint{1.139575in}{3.340871in}}%
\pgfpathlineto{\pgfqpoint{1.140648in}{3.337588in}}%
\pgfpathlineto{\pgfqpoint{1.143870in}{3.335503in}}%
\pgfpathlineto{\pgfqpoint{1.144944in}{3.335884in}}%
\pgfpathlineto{\pgfqpoint{1.146018in}{3.336995in}}%
\pgfpathlineto{\pgfqpoint{1.147091in}{3.336163in}}%
\pgfpathlineto{\pgfqpoint{1.148165in}{3.336378in}}%
\pgfpathlineto{\pgfqpoint{1.158904in}{3.334477in}}%
\pgfpathlineto{\pgfqpoint{1.159977in}{3.334518in}}%
\pgfpathlineto{\pgfqpoint{1.161051in}{3.332901in}}%
\pgfpathlineto{\pgfqpoint{1.162125in}{3.333860in}}%
\pgfpathlineto{\pgfqpoint{1.163199in}{3.332389in}}%
\pgfpathlineto{\pgfqpoint{1.169642in}{3.332583in}}%
\pgfpathlineto{\pgfqpoint{1.170716in}{3.331908in}}%
\pgfpathlineto{\pgfqpoint{1.173937in}{3.331869in}}%
\pgfpathlineto{\pgfqpoint{1.176085in}{3.332652in}}%
\pgfpathlineto{\pgfqpoint{1.177159in}{3.332811in}}%
\pgfpathlineto{\pgfqpoint{1.178233in}{3.332293in}}%
\pgfpathlineto{\pgfqpoint{1.182528in}{3.333234in}}%
\pgfpathlineto{\pgfqpoint{1.183602in}{3.333073in}}%
\pgfpathlineto{\pgfqpoint{1.185750in}{3.333595in}}%
\pgfpathlineto{\pgfqpoint{1.191119in}{3.333674in}}%
\pgfpathlineto{\pgfqpoint{1.192193in}{3.334285in}}%
\pgfpathlineto{\pgfqpoint{1.193266in}{3.332675in}}%
\pgfpathlineto{\pgfqpoint{1.197562in}{3.332035in}}%
\pgfpathlineto{\pgfqpoint{1.198636in}{3.333799in}}%
\pgfpathlineto{\pgfqpoint{1.199710in}{3.333922in}}%
\pgfpathlineto{\pgfqpoint{1.200783in}{3.332938in}}%
\pgfpathlineto{\pgfqpoint{1.205079in}{3.332339in}}%
\pgfpathlineto{\pgfqpoint{1.207226in}{3.330732in}}%
\pgfpathlineto{\pgfqpoint{1.212596in}{3.329861in}}%
\pgfpathlineto{\pgfqpoint{1.215817in}{3.329190in}}%
\pgfpathlineto{\pgfqpoint{1.220112in}{3.328606in}}%
\pgfpathlineto{\pgfqpoint{1.222260in}{3.330159in}}%
\pgfpathlineto{\pgfqpoint{1.223334in}{3.330197in}}%
\pgfpathlineto{\pgfqpoint{1.234072in}{3.332533in}}%
\pgfpathlineto{\pgfqpoint{1.236220in}{3.330569in}}%
\pgfpathlineto{\pgfqpoint{1.237294in}{3.331024in}}%
\pgfpathlineto{\pgfqpoint{1.238368in}{3.332941in}}%
\pgfpathlineto{\pgfqpoint{1.242663in}{3.333864in}}%
\pgfpathlineto{\pgfqpoint{1.244811in}{3.335435in}}%
\pgfpathlineto{\pgfqpoint{1.245885in}{3.338411in}}%
\pgfpathlineto{\pgfqpoint{1.249106in}{3.337956in}}%
\pgfpathlineto{\pgfqpoint{1.250180in}{3.336560in}}%
\pgfpathlineto{\pgfqpoint{1.251254in}{3.337128in}}%
\pgfpathlineto{\pgfqpoint{1.252328in}{3.336420in}}%
\pgfpathlineto{\pgfqpoint{1.253401in}{3.337683in}}%
\pgfpathlineto{\pgfqpoint{1.256623in}{3.337683in}}%
\pgfpathlineto{\pgfqpoint{1.257697in}{3.336851in}}%
\pgfpathlineto{\pgfqpoint{1.258771in}{3.337032in}}%
\pgfpathlineto{\pgfqpoint{1.260918in}{3.334603in}}%
\pgfpathlineto{\pgfqpoint{1.264140in}{3.335121in}}%
\pgfpathlineto{\pgfqpoint{1.265214in}{3.333767in}}%
\pgfpathlineto{\pgfqpoint{1.267361in}{3.333724in}}%
\pgfpathlineto{\pgfqpoint{1.268435in}{3.332709in}}%
\pgfpathlineto{\pgfqpoint{1.272731in}{3.332998in}}%
\pgfpathlineto{\pgfqpoint{1.274878in}{3.334042in}}%
\pgfpathlineto{\pgfqpoint{1.275952in}{3.334681in}}%
\pgfpathlineto{\pgfqpoint{1.280247in}{3.333512in}}%
\pgfpathlineto{\pgfqpoint{1.281321in}{3.333428in}}%
\pgfpathlineto{\pgfqpoint{1.282395in}{3.332755in}}%
\pgfpathlineto{\pgfqpoint{1.283469in}{3.332838in}}%
\pgfpathlineto{\pgfqpoint{1.286690in}{3.333999in}}%
\pgfpathlineto{\pgfqpoint{1.288838in}{3.331269in}}%
\pgfpathlineto{\pgfqpoint{1.289912in}{3.330309in}}%
\pgfpathlineto{\pgfqpoint{1.296355in}{3.331774in}}%
\pgfpathlineto{\pgfqpoint{1.298503in}{3.333944in}}%
\pgfpathlineto{\pgfqpoint{1.302798in}{3.332089in}}%
\pgfpathlineto{\pgfqpoint{1.303872in}{3.333313in}}%
\pgfpathlineto{\pgfqpoint{1.306020in}{3.332935in}}%
\pgfpathlineto{\pgfqpoint{1.309241in}{3.332893in}}%
\pgfpathlineto{\pgfqpoint{1.310315in}{3.331895in}}%
\pgfpathlineto{\pgfqpoint{1.311389in}{3.332180in}}%
\pgfpathlineto{\pgfqpoint{1.313536in}{3.331324in}}%
\pgfpathlineto{\pgfqpoint{1.318906in}{3.330802in}}%
\pgfpathlineto{\pgfqpoint{1.319979in}{3.330168in}}%
\pgfpathlineto{\pgfqpoint{1.321053in}{3.330831in}}%
\pgfpathlineto{\pgfqpoint{1.325349in}{3.331707in}}%
\pgfpathlineto{\pgfqpoint{1.326422in}{3.331060in}}%
\pgfpathlineto{\pgfqpoint{1.328570in}{3.332747in}}%
\pgfpathlineto{\pgfqpoint{1.332865in}{3.331470in}}%
\pgfpathlineto{\pgfqpoint{1.335013in}{3.329087in}}%
\pgfpathlineto{\pgfqpoint{1.340382in}{3.329011in}}%
\pgfpathlineto{\pgfqpoint{1.342530in}{3.329390in}}%
\pgfpathlineto{\pgfqpoint{1.343604in}{3.328930in}}%
\pgfpathlineto{\pgfqpoint{1.348973in}{3.328062in}}%
\pgfpathlineto{\pgfqpoint{1.351121in}{3.328695in}}%
\pgfpathlineto{\pgfqpoint{1.354342in}{3.328997in}}%
\pgfpathlineto{\pgfqpoint{1.355416in}{3.330211in}}%
\pgfpathlineto{\pgfqpoint{1.356490in}{3.329117in}}%
\pgfpathlineto{\pgfqpoint{1.358638in}{3.329460in}}%
\pgfpathlineto{\pgfqpoint{1.362933in}{3.327893in}}%
\pgfpathlineto{\pgfqpoint{1.365081in}{3.328860in}}%
\pgfpathlineto{\pgfqpoint{1.366154in}{3.328670in}}%
\pgfpathlineto{\pgfqpoint{1.369376in}{3.328859in}}%
\pgfpathlineto{\pgfqpoint{1.370450in}{3.329805in}}%
\pgfpathlineto{\pgfqpoint{1.372598in}{3.329147in}}%
\pgfpathlineto{\pgfqpoint{1.379041in}{3.328653in}}%
\pgfpathlineto{\pgfqpoint{1.381188in}{3.328277in}}%
\pgfpathlineto{\pgfqpoint{1.385484in}{3.328277in}}%
\pgfpathlineto{\pgfqpoint{1.386557in}{3.328986in}}%
\pgfpathlineto{\pgfqpoint{1.391927in}{3.326308in}}%
\pgfpathlineto{\pgfqpoint{1.393000in}{3.326415in}}%
\pgfpathlineto{\pgfqpoint{1.395148in}{3.327708in}}%
\pgfpathlineto{\pgfqpoint{1.396222in}{3.327487in}}%
\pgfpathlineto{\pgfqpoint{1.400517in}{3.327961in}}%
\pgfpathlineto{\pgfqpoint{1.401591in}{3.327924in}}%
\pgfpathlineto{\pgfqpoint{1.403739in}{3.327001in}}%
\pgfpathlineto{\pgfqpoint{1.411256in}{3.328607in}}%
\pgfpathlineto{\pgfqpoint{1.415551in}{3.328716in}}%
\pgfpathlineto{\pgfqpoint{1.417699in}{3.327293in}}%
\pgfpathlineto{\pgfqpoint{1.421994in}{3.327621in}}%
\pgfpathlineto{\pgfqpoint{1.423068in}{3.328761in}}%
\pgfpathlineto{\pgfqpoint{1.425216in}{3.329292in}}%
\pgfpathlineto{\pgfqpoint{1.426289in}{3.328219in}}%
\pgfpathlineto{\pgfqpoint{1.430585in}{3.328031in}}%
\pgfpathlineto{\pgfqpoint{1.431659in}{3.326768in}}%
\pgfpathlineto{\pgfqpoint{1.433806in}{3.328438in}}%
\pgfpathlineto{\pgfqpoint{1.438102in}{3.329786in}}%
\pgfpathlineto{\pgfqpoint{1.441323in}{3.330526in}}%
\pgfpathlineto{\pgfqpoint{1.445619in}{3.331640in}}%
\pgfpathlineto{\pgfqpoint{1.446692in}{3.330990in}}%
\pgfpathlineto{\pgfqpoint{1.447766in}{3.330990in}}%
\pgfpathlineto{\pgfqpoint{1.448840in}{3.330338in}}%
\pgfpathlineto{\pgfqpoint{1.453135in}{3.330338in}}%
\pgfpathlineto{\pgfqpoint{1.454209in}{3.328916in}}%
\pgfpathlineto{\pgfqpoint{1.462800in}{3.327510in}}%
\pgfpathlineto{\pgfqpoint{1.463874in}{3.328268in}}%
\pgfpathlineto{\pgfqpoint{1.470317in}{3.325796in}}%
\pgfpathlineto{\pgfqpoint{1.471391in}{3.326633in}}%
\pgfpathlineto{\pgfqpoint{1.474612in}{3.326484in}}%
\pgfpathlineto{\pgfqpoint{1.475686in}{3.327076in}}%
\pgfpathlineto{\pgfqpoint{1.476760in}{3.326700in}}%
\pgfpathlineto{\pgfqpoint{1.478908in}{3.327148in}}%
\pgfpathlineto{\pgfqpoint{1.485351in}{3.327220in}}%
\pgfpathlineto{\pgfqpoint{1.486424in}{3.326767in}}%
\pgfpathlineto{\pgfqpoint{1.497163in}{3.327323in}}%
\pgfpathlineto{\pgfqpoint{1.498237in}{3.326415in}}%
\pgfpathlineto{\pgfqpoint{1.501458in}{3.325534in}}%
\pgfpathlineto{\pgfqpoint{1.506827in}{3.326590in}}%
\pgfpathlineto{\pgfqpoint{1.507901in}{3.327778in}}%
\pgfpathlineto{\pgfqpoint{1.508975in}{3.327510in}}%
\pgfpathlineto{\pgfqpoint{1.512197in}{3.328194in}}%
\pgfpathlineto{\pgfqpoint{1.513270in}{3.327537in}}%
\pgfpathlineto{\pgfqpoint{1.515418in}{3.327537in}}%
\pgfpathlineto{\pgfqpoint{1.516492in}{3.326645in}}%
\pgfpathlineto{\pgfqpoint{1.534747in}{3.326645in}}%
\pgfpathlineto{\pgfqpoint{1.535821in}{3.325298in}}%
\pgfpathlineto{\pgfqpoint{1.536895in}{3.325298in}}%
\pgfpathlineto{\pgfqpoint{1.539042in}{3.323919in}}%
\pgfpathlineto{\pgfqpoint{1.542264in}{3.324282in}}%
\pgfpathlineto{\pgfqpoint{1.543338in}{3.323219in}}%
\pgfpathlineto{\pgfqpoint{1.544412in}{3.323861in}}%
\pgfpathlineto{\pgfqpoint{1.551929in}{3.323351in}}%
\pgfpathlineto{\pgfqpoint{1.553002in}{3.321778in}}%
\pgfpathlineto{\pgfqpoint{1.557298in}{3.321469in}}%
\pgfpathlineto{\pgfqpoint{1.558372in}{3.321708in}}%
\pgfpathlineto{\pgfqpoint{1.559445in}{3.321331in}}%
\pgfpathlineto{\pgfqpoint{1.561593in}{3.321500in}}%
\pgfpathlineto{\pgfqpoint{1.564815in}{3.321057in}}%
\pgfpathlineto{\pgfqpoint{1.565888in}{3.319946in}}%
\pgfpathlineto{\pgfqpoint{1.568036in}{3.320405in}}%
\pgfpathlineto{\pgfqpoint{1.569110in}{3.320008in}}%
\pgfpathlineto{\pgfqpoint{1.574479in}{3.320334in}}%
\pgfpathlineto{\pgfqpoint{1.579848in}{3.322682in}}%
\pgfpathlineto{\pgfqpoint{1.580922in}{3.322612in}}%
\pgfpathlineto{\pgfqpoint{1.581996in}{3.321628in}}%
\pgfpathlineto{\pgfqpoint{1.584144in}{3.322694in}}%
\pgfpathlineto{\pgfqpoint{1.588439in}{3.323610in}}%
\pgfpathlineto{\pgfqpoint{1.598104in}{3.323610in}}%
\pgfpathlineto{\pgfqpoint{1.599177in}{3.322149in}}%
\pgfpathlineto{\pgfqpoint{1.606694in}{3.321690in}}%
\pgfpathlineto{\pgfqpoint{1.621728in}{3.322282in}}%
\pgfpathlineto{\pgfqpoint{1.670051in}{3.322282in}}%
\pgfpathlineto{\pgfqpoint{1.671125in}{3.321643in}}%
\pgfpathlineto{\pgfqpoint{1.672198in}{3.320107in}}%
\pgfpathlineto{\pgfqpoint{1.674346in}{3.319871in}}%
\pgfpathlineto{\pgfqpoint{1.677568in}{3.319539in}}%
\pgfpathlineto{\pgfqpoint{1.679715in}{3.320803in}}%
\pgfpathlineto{\pgfqpoint{1.680789in}{3.320359in}}%
\pgfpathlineto{\pgfqpoint{1.681863in}{3.321472in}}%
\pgfpathlineto{\pgfqpoint{1.693675in}{3.319039in}}%
\pgfpathlineto{\pgfqpoint{1.694749in}{3.318093in}}%
\pgfpathlineto{\pgfqpoint{1.695823in}{3.318949in}}%
\pgfpathlineto{\pgfqpoint{1.700118in}{3.318545in}}%
\pgfpathlineto{\pgfqpoint{1.717300in}{3.318545in}}%
\pgfpathlineto{\pgfqpoint{1.718374in}{3.316473in}}%
\pgfpathlineto{\pgfqpoint{1.719447in}{3.309004in}}%
\pgfpathlineto{\pgfqpoint{1.722669in}{3.310914in}}%
\pgfpathlineto{\pgfqpoint{1.723743in}{3.310667in}}%
\pgfpathlineto{\pgfqpoint{1.725890in}{3.311711in}}%
\pgfpathlineto{\pgfqpoint{1.730186in}{3.312332in}}%
\pgfpathlineto{\pgfqpoint{1.731260in}{3.313279in}}%
\pgfpathlineto{\pgfqpoint{1.732333in}{3.312570in}}%
\pgfpathlineto{\pgfqpoint{1.734481in}{3.312831in}}%
\pgfpathlineto{\pgfqpoint{1.740924in}{3.312160in}}%
\pgfpathlineto{\pgfqpoint{1.741998in}{3.311589in}}%
\pgfpathlineto{\pgfqpoint{1.745219in}{3.311701in}}%
\pgfpathlineto{\pgfqpoint{1.747367in}{3.312778in}}%
\pgfpathlineto{\pgfqpoint{1.749515in}{3.311452in}}%
\pgfpathlineto{\pgfqpoint{1.752736in}{3.312486in}}%
\pgfpathlineto{\pgfqpoint{1.757032in}{3.311650in}}%
\pgfpathlineto{\pgfqpoint{1.761327in}{3.311454in}}%
\pgfpathlineto{\pgfqpoint{1.763475in}{3.310591in}}%
\pgfpathlineto{\pgfqpoint{1.776361in}{3.311488in}}%
\pgfpathlineto{\pgfqpoint{1.777435in}{3.311012in}}%
\pgfpathlineto{\pgfqpoint{1.778508in}{3.311646in}}%
\pgfpathlineto{\pgfqpoint{1.787099in}{3.311327in}}%
\pgfpathlineto{\pgfqpoint{1.792468in}{3.311604in}}%
\pgfpathlineto{\pgfqpoint{1.793542in}{3.310817in}}%
\pgfpathlineto{\pgfqpoint{1.794616in}{3.311119in}}%
\pgfpathlineto{\pgfqpoint{1.798911in}{3.310402in}}%
\pgfpathlineto{\pgfqpoint{1.801059in}{3.311578in}}%
\pgfpathlineto{\pgfqpoint{1.802133in}{3.311466in}}%
\pgfpathlineto{\pgfqpoint{1.806428in}{3.312704in}}%
\pgfpathlineto{\pgfqpoint{1.808576in}{3.312182in}}%
\pgfpathlineto{\pgfqpoint{1.812871in}{3.313328in}}%
\pgfpathlineto{\pgfqpoint{1.813945in}{3.312913in}}%
\pgfpathlineto{\pgfqpoint{1.815019in}{3.314349in}}%
\pgfpathlineto{\pgfqpoint{1.824684in}{3.311005in}}%
\pgfpathlineto{\pgfqpoint{1.827905in}{3.311226in}}%
\pgfpathlineto{\pgfqpoint{1.832200in}{3.314640in}}%
\pgfpathlineto{\pgfqpoint{1.839717in}{3.313714in}}%
\pgfpathlineto{\pgfqpoint{1.844013in}{3.314224in}}%
\pgfpathlineto{\pgfqpoint{1.845086in}{3.313705in}}%
\pgfpathlineto{\pgfqpoint{1.847234in}{3.314520in}}%
\pgfpathlineto{\pgfqpoint{1.852603in}{3.312954in}}%
\pgfpathlineto{\pgfqpoint{1.853677in}{3.313134in}}%
\pgfpathlineto{\pgfqpoint{1.854751in}{3.312379in}}%
\pgfpathlineto{\pgfqpoint{1.859046in}{3.312405in}}%
\pgfpathlineto{\pgfqpoint{1.889114in}{3.312405in}}%
\pgfpathlineto{\pgfqpoint{1.890188in}{3.310793in}}%
\pgfpathlineto{\pgfqpoint{1.907369in}{3.310793in}}%
\pgfpathlineto{\pgfqpoint{1.913812in}{3.304772in}}%
\pgfpathlineto{\pgfqpoint{1.918107in}{3.304628in}}%
\pgfpathlineto{\pgfqpoint{1.920255in}{3.305642in}}%
\pgfpathlineto{\pgfqpoint{1.921329in}{3.304750in}}%
\pgfpathlineto{\pgfqpoint{1.922403in}{3.302779in}}%
\pgfpathlineto{\pgfqpoint{1.927772in}{3.303024in}}%
\pgfpathlineto{\pgfqpoint{1.928846in}{3.301620in}}%
\pgfpathlineto{\pgfqpoint{1.937437in}{3.302841in}}%
\pgfpathlineto{\pgfqpoint{1.944953in}{3.300968in}}%
\pgfpathlineto{\pgfqpoint{1.948175in}{3.301262in}}%
\pgfpathlineto{\pgfqpoint{1.950323in}{3.299628in}}%
\pgfpathlineto{\pgfqpoint{1.951396in}{3.300488in}}%
\pgfpathlineto{\pgfqpoint{1.955692in}{3.300138in}}%
\pgfpathlineto{\pgfqpoint{1.958913in}{3.300296in}}%
\pgfpathlineto{\pgfqpoint{1.959987in}{3.299618in}}%
\pgfpathlineto{\pgfqpoint{1.967504in}{3.300118in}}%
\pgfpathlineto{\pgfqpoint{1.970726in}{3.300852in}}%
\pgfpathlineto{\pgfqpoint{1.971799in}{3.300454in}}%
\pgfpathlineto{\pgfqpoint{1.973947in}{3.300700in}}%
\pgfpathlineto{\pgfqpoint{1.975021in}{3.299866in}}%
\pgfpathlineto{\pgfqpoint{1.980390in}{3.299899in}}%
\pgfpathlineto{\pgfqpoint{1.981464in}{3.299595in}}%
\pgfpathlineto{\pgfqpoint{1.982538in}{3.300256in}}%
\pgfpathlineto{\pgfqpoint{1.986833in}{3.300153in}}%
\pgfpathlineto{\pgfqpoint{1.987907in}{3.298988in}}%
\pgfpathlineto{\pgfqpoint{1.990055in}{3.299796in}}%
\pgfpathlineto{\pgfqpoint{1.994350in}{3.299434in}}%
\pgfpathlineto{\pgfqpoint{1.995424in}{3.298638in}}%
\pgfpathlineto{\pgfqpoint{1.996498in}{3.298831in}}%
\pgfpathlineto{\pgfqpoint{2.000793in}{3.298714in}}%
\pgfpathlineto{\pgfqpoint{2.001867in}{3.298094in}}%
\pgfpathlineto{\pgfqpoint{2.004015in}{3.298300in}}%
\pgfpathlineto{\pgfqpoint{2.010458in}{3.299788in}}%
\pgfpathlineto{\pgfqpoint{2.012605in}{3.302997in}}%
\pgfpathlineto{\pgfqpoint{2.016901in}{3.302634in}}%
\pgfpathlineto{\pgfqpoint{2.017974in}{3.303421in}}%
\pgfpathlineto{\pgfqpoint{2.019048in}{3.301778in}}%
\pgfpathlineto{\pgfqpoint{2.020122in}{3.302847in}}%
\pgfpathlineto{\pgfqpoint{2.024417in}{3.302847in}}%
\pgfpathlineto{\pgfqpoint{2.025491in}{3.303800in}}%
\pgfpathlineto{\pgfqpoint{2.026565in}{3.302603in}}%
\pgfpathlineto{\pgfqpoint{2.027639in}{3.303301in}}%
\pgfpathlineto{\pgfqpoint{2.030861in}{3.303693in}}%
\pgfpathlineto{\pgfqpoint{2.031934in}{3.303155in}}%
\pgfpathlineto{\pgfqpoint{2.033008in}{3.303752in}}%
\pgfpathlineto{\pgfqpoint{2.034082in}{3.303329in}}%
\pgfpathlineto{\pgfqpoint{2.035156in}{3.301550in}}%
\pgfpathlineto{\pgfqpoint{2.039451in}{3.303201in}}%
\pgfpathlineto{\pgfqpoint{2.041599in}{3.301365in}}%
\pgfpathlineto{\pgfqpoint{2.042673in}{3.302552in}}%
\pgfpathlineto{\pgfqpoint{2.046968in}{3.303042in}}%
\pgfpathlineto{\pgfqpoint{2.048042in}{3.302996in}}%
\pgfpathlineto{\pgfqpoint{2.049116in}{3.304690in}}%
\pgfpathlineto{\pgfqpoint{2.050190in}{3.302987in}}%
\pgfpathlineto{\pgfqpoint{2.054485in}{3.301794in}}%
\pgfpathlineto{\pgfqpoint{2.055559in}{3.300893in}}%
\pgfpathlineto{\pgfqpoint{2.057706in}{3.301469in}}%
\pgfpathlineto{\pgfqpoint{2.060928in}{3.300752in}}%
\pgfpathlineto{\pgfqpoint{2.063076in}{3.301584in}}%
\pgfpathlineto{\pgfqpoint{2.065223in}{3.300307in}}%
\pgfpathlineto{\pgfqpoint{2.068445in}{3.300786in}}%
\pgfpathlineto{\pgfqpoint{2.070593in}{3.298906in}}%
\pgfpathlineto{\pgfqpoint{2.075962in}{3.298704in}}%
\pgfpathlineto{\pgfqpoint{2.079183in}{3.299295in}}%
\pgfpathlineto{\pgfqpoint{2.080257in}{3.298613in}}%
\pgfpathlineto{\pgfqpoint{2.085626in}{3.298887in}}%
\pgfpathlineto{\pgfqpoint{2.086700in}{3.297545in}}%
\pgfpathlineto{\pgfqpoint{2.093143in}{3.297356in}}%
\pgfpathlineto{\pgfqpoint{2.099586in}{3.296706in}}%
\pgfpathlineto{\pgfqpoint{2.100660in}{3.296144in}}%
\pgfpathlineto{\pgfqpoint{2.102808in}{3.296001in}}%
\pgfpathlineto{\pgfqpoint{2.109251in}{3.298027in}}%
\pgfpathlineto{\pgfqpoint{2.110325in}{3.297778in}}%
\pgfpathlineto{\pgfqpoint{2.115694in}{3.297456in}}%
\pgfpathlineto{\pgfqpoint{2.124284in}{3.297454in}}%
\pgfpathlineto{\pgfqpoint{2.125358in}{3.297809in}}%
\pgfpathlineto{\pgfqpoint{2.132875in}{3.297824in}}%
\pgfpathlineto{\pgfqpoint{2.137171in}{3.298010in}}%
\pgfpathlineto{\pgfqpoint{2.139318in}{3.299264in}}%
\pgfpathlineto{\pgfqpoint{2.140392in}{3.298862in}}%
\pgfpathlineto{\pgfqpoint{2.143614in}{3.299297in}}%
\pgfpathlineto{\pgfqpoint{2.144687in}{3.298232in}}%
\pgfpathlineto{\pgfqpoint{2.146835in}{3.298872in}}%
\pgfpathlineto{\pgfqpoint{2.147909in}{3.299644in}}%
\pgfpathlineto{\pgfqpoint{2.151130in}{3.299094in}}%
\pgfpathlineto{\pgfqpoint{2.154352in}{3.300145in}}%
\pgfpathlineto{\pgfqpoint{2.155426in}{3.299771in}}%
\pgfpathlineto{\pgfqpoint{2.158647in}{3.299915in}}%
\pgfpathlineto{\pgfqpoint{2.160795in}{3.298833in}}%
\pgfpathlineto{\pgfqpoint{2.162943in}{3.298793in}}%
\pgfpathlineto{\pgfqpoint{2.169386in}{3.298910in}}%
\pgfpathlineto{\pgfqpoint{2.170460in}{3.299088in}}%
\pgfpathlineto{\pgfqpoint{2.177976in}{3.298923in}}%
\pgfpathlineto{\pgfqpoint{2.181198in}{3.299280in}}%
\pgfpathlineto{\pgfqpoint{2.182272in}{3.298213in}}%
\pgfpathlineto{\pgfqpoint{2.189789in}{3.296997in}}%
\pgfpathlineto{\pgfqpoint{2.190862in}{3.297255in}}%
\pgfpathlineto{\pgfqpoint{2.191936in}{3.296585in}}%
\pgfpathlineto{\pgfqpoint{2.193010in}{3.298759in}}%
\pgfpathlineto{\pgfqpoint{2.196232in}{3.299629in}}%
\pgfpathlineto{\pgfqpoint{2.200527in}{3.295923in}}%
\pgfpathlineto{\pgfqpoint{2.205896in}{3.295532in}}%
\pgfpathlineto{\pgfqpoint{2.214487in}{3.294199in}}%
\pgfpathlineto{\pgfqpoint{2.215561in}{3.293837in}}%
\pgfpathlineto{\pgfqpoint{2.222004in}{3.294260in}}%
\pgfpathlineto{\pgfqpoint{2.223078in}{3.295020in}}%
\pgfpathlineto{\pgfqpoint{2.230594in}{3.296431in}}%
\pgfpathlineto{\pgfqpoint{2.274622in}{3.296565in}}%
\pgfpathlineto{\pgfqpoint{2.275696in}{3.298057in}}%
\pgfpathlineto{\pgfqpoint{2.278917in}{3.297419in}}%
\pgfpathlineto{\pgfqpoint{2.279991in}{3.298475in}}%
\pgfpathlineto{\pgfqpoint{2.281065in}{3.298730in}}%
\pgfpathlineto{\pgfqpoint{2.282139in}{3.298234in}}%
\pgfpathlineto{\pgfqpoint{2.287508in}{3.298363in}}%
\pgfpathlineto{\pgfqpoint{2.290729in}{3.297989in}}%
\pgfpathlineto{\pgfqpoint{2.293951in}{3.298587in}}%
\pgfpathlineto{\pgfqpoint{2.296099in}{3.297956in}}%
\pgfpathlineto{\pgfqpoint{2.297172in}{3.298572in}}%
\pgfpathlineto{\pgfqpoint{2.301468in}{3.298395in}}%
\pgfpathlineto{\pgfqpoint{2.308985in}{3.299775in}}%
\pgfpathlineto{\pgfqpoint{2.313280in}{3.299711in}}%
\pgfpathlineto{\pgfqpoint{2.327240in}{3.300176in}}%
\pgfpathlineto{\pgfqpoint{2.328314in}{3.299084in}}%
\pgfpathlineto{\pgfqpoint{2.332609in}{3.299692in}}%
\pgfpathlineto{\pgfqpoint{2.334757in}{3.300796in}}%
\pgfpathlineto{\pgfqpoint{2.335831in}{3.300495in}}%
\pgfpathlineto{\pgfqpoint{2.343348in}{3.296447in}}%
\pgfpathlineto{\pgfqpoint{2.347643in}{3.296387in}}%
\pgfpathlineto{\pgfqpoint{2.354086in}{3.296185in}}%
\pgfpathlineto{\pgfqpoint{2.356234in}{3.295451in}}%
\pgfpathlineto{\pgfqpoint{2.361603in}{3.295589in}}%
\pgfpathlineto{\pgfqpoint{2.363750in}{3.296351in}}%
\pgfpathlineto{\pgfqpoint{2.364824in}{3.295335in}}%
\pgfpathlineto{\pgfqpoint{2.370193in}{3.295683in}}%
\pgfpathlineto{\pgfqpoint{2.371267in}{3.295012in}}%
\pgfpathlineto{\pgfqpoint{2.373415in}{3.294735in}}%
\pgfpathlineto{\pgfqpoint{2.377710in}{3.294802in}}%
\pgfpathlineto{\pgfqpoint{2.379858in}{3.295523in}}%
\pgfpathlineto{\pgfqpoint{2.380932in}{3.294768in}}%
\pgfpathlineto{\pgfqpoint{2.392744in}{3.294173in}}%
\pgfpathlineto{\pgfqpoint{2.395966in}{3.294622in}}%
\pgfpathlineto{\pgfqpoint{2.406704in}{3.294823in}}%
\pgfpathlineto{\pgfqpoint{2.410999in}{3.294976in}}%
\pgfpathlineto{\pgfqpoint{2.417442in}{3.295217in}}%
\pgfpathlineto{\pgfqpoint{2.418516in}{3.296257in}}%
\pgfpathlineto{\pgfqpoint{2.421738in}{3.297506in}}%
\pgfpathlineto{\pgfqpoint{2.424959in}{3.296560in}}%
\pgfpathlineto{\pgfqpoint{2.426033in}{3.297056in}}%
\pgfpathlineto{\pgfqpoint{2.437845in}{3.297811in}}%
\pgfpathlineto{\pgfqpoint{2.439993in}{3.297770in}}%
\pgfpathlineto{\pgfqpoint{2.448584in}{3.296446in}}%
\pgfpathlineto{\pgfqpoint{2.456101in}{3.296350in}}%
\pgfpathlineto{\pgfqpoint{2.460396in}{3.296972in}}%
\pgfpathlineto{\pgfqpoint{2.462544in}{3.296337in}}%
\pgfpathlineto{\pgfqpoint{2.470060in}{3.297204in}}%
\pgfpathlineto{\pgfqpoint{2.471134in}{3.296168in}}%
\pgfpathlineto{\pgfqpoint{2.477577in}{3.297015in}}%
\pgfpathlineto{\pgfqpoint{2.478651in}{3.296809in}}%
\pgfpathlineto{\pgfqpoint{2.490463in}{3.297223in}}%
\pgfpathlineto{\pgfqpoint{2.501202in}{3.296603in}}%
\pgfpathlineto{\pgfqpoint{2.506571in}{3.296952in}}%
\pgfpathlineto{\pgfqpoint{2.507645in}{3.297307in}}%
\pgfpathlineto{\pgfqpoint{2.514088in}{3.296575in}}%
\pgfpathlineto{\pgfqpoint{2.515162in}{3.296133in}}%
\pgfpathlineto{\pgfqpoint{2.516236in}{3.297290in}}%
\pgfpathlineto{\pgfqpoint{2.521605in}{3.297784in}}%
\pgfpathlineto{\pgfqpoint{2.523752in}{3.298250in}}%
\pgfpathlineto{\pgfqpoint{2.528048in}{3.298250in}}%
\pgfpathlineto{\pgfqpoint{2.530195in}{3.297875in}}%
\pgfpathlineto{\pgfqpoint{2.534491in}{3.298108in}}%
\pgfpathlineto{\pgfqpoint{2.537712in}{3.298456in}}%
\pgfpathlineto{\pgfqpoint{2.568854in}{3.298456in}}%
\pgfpathlineto{\pgfqpoint{2.572075in}{3.296632in}}%
\pgfpathlineto{\pgfqpoint{2.573149in}{3.297472in}}%
\pgfpathlineto{\pgfqpoint{2.576370in}{3.295658in}}%
\pgfpathlineto{\pgfqpoint{2.579592in}{3.295988in}}%
\pgfpathlineto{\pgfqpoint{2.580666in}{3.297124in}}%
\pgfpathlineto{\pgfqpoint{2.734225in}{3.297124in}}%
\pgfpathlineto{\pgfqpoint{2.737446in}{3.293519in}}%
\pgfpathlineto{\pgfqpoint{2.738520in}{3.290789in}}%
\pgfpathlineto{\pgfqpoint{2.739594in}{3.291664in}}%
\pgfpathlineto{\pgfqpoint{2.744963in}{3.291001in}}%
\pgfpathlineto{\pgfqpoint{2.746037in}{3.290630in}}%
\pgfpathlineto{\pgfqpoint{2.747111in}{3.291399in}}%
\pgfpathlineto{\pgfqpoint{2.755702in}{3.291743in}}%
\pgfpathlineto{\pgfqpoint{2.756775in}{3.290736in}}%
\pgfpathlineto{\pgfqpoint{2.764292in}{3.290630in}}%
\pgfpathlineto{\pgfqpoint{2.767514in}{3.290498in}}%
\pgfpathlineto{\pgfqpoint{2.768588in}{3.291134in}}%
\pgfpathlineto{\pgfqpoint{2.770735in}{3.290471in}}%
\pgfpathlineto{\pgfqpoint{2.771809in}{3.290895in}}%
\pgfpathlineto{\pgfqpoint{2.776104in}{3.290339in}}%
\pgfpathlineto{\pgfqpoint{2.778252in}{3.290047in}}%
\pgfpathlineto{\pgfqpoint{2.786843in}{3.290286in}}%
\pgfpathlineto{\pgfqpoint{2.786843in}{3.290286in}}%
\pgfusepath{stroke}%
\end{pgfscope}%
\begin{pgfscope}%
\pgfsetrectcap%
\pgfsetmiterjoin%
\pgfsetlinewidth{0.803000pt}%
\definecolor{currentstroke}{rgb}{1.000000,1.000000,1.000000}%
\pgfsetstrokecolor{currentstroke}%
\pgfsetdash{}{0pt}%
\pgfpathmoveto{\pgfqpoint{0.320934in}{3.271772in}}%
\pgfpathlineto{\pgfqpoint{0.320934in}{3.672657in}}%
\pgfusepath{stroke}%
\end{pgfscope}%
\begin{pgfscope}%
\pgfsetrectcap%
\pgfsetmiterjoin%
\pgfsetlinewidth{0.803000pt}%
\definecolor{currentstroke}{rgb}{1.000000,1.000000,1.000000}%
\pgfsetstrokecolor{currentstroke}%
\pgfsetdash{}{0pt}%
\pgfpathmoveto{\pgfqpoint{2.904267in}{3.271772in}}%
\pgfpathlineto{\pgfqpoint{2.904267in}{3.672657in}}%
\pgfusepath{stroke}%
\end{pgfscope}%
\begin{pgfscope}%
\pgfsetrectcap%
\pgfsetmiterjoin%
\pgfsetlinewidth{0.803000pt}%
\definecolor{currentstroke}{rgb}{1.000000,1.000000,1.000000}%
\pgfsetstrokecolor{currentstroke}%
\pgfsetdash{}{0pt}%
\pgfpathmoveto{\pgfqpoint{0.320934in}{3.271772in}}%
\pgfpathlineto{\pgfqpoint{2.904267in}{3.271772in}}%
\pgfusepath{stroke}%
\end{pgfscope}%
\begin{pgfscope}%
\pgfsetrectcap%
\pgfsetmiterjoin%
\pgfsetlinewidth{0.803000pt}%
\definecolor{currentstroke}{rgb}{1.000000,1.000000,1.000000}%
\pgfsetstrokecolor{currentstroke}%
\pgfsetdash{}{0pt}%
\pgfpathmoveto{\pgfqpoint{0.320934in}{3.672657in}}%
\pgfpathlineto{\pgfqpoint{2.904267in}{3.672657in}}%
\pgfusepath{stroke}%
\end{pgfscope}%
\begin{pgfscope}%
\definecolor{textcolor}{rgb}{0.150000,0.150000,0.150000}%
\pgfsetstrokecolor{textcolor}%
\pgfsetfillcolor{textcolor}%
\pgftext[x=1.612600in,y=3.755990in,,base]{\color{textcolor}\rmfamily\fontsize{16.800000}{20.160000}\selectfont GE}%
\end{pgfscope}%
\begin{pgfscope}%
\pgfsetbuttcap%
\pgfsetmiterjoin%
\definecolor{currentfill}{rgb}{0.917647,0.917647,0.949020}%
\pgfsetfillcolor{currentfill}%
\pgfsetlinewidth{0.000000pt}%
\definecolor{currentstroke}{rgb}{0.000000,0.000000,0.000000}%
\pgfsetstrokecolor{currentstroke}%
\pgfsetstrokeopacity{0.000000}%
\pgfsetdash{}{0pt}%
\pgfpathmoveto{\pgfqpoint{3.937600in}{3.271772in}}%
\pgfpathlineto{\pgfqpoint{6.520934in}{3.271772in}}%
\pgfpathlineto{\pgfqpoint{6.520934in}{3.672657in}}%
\pgfpathlineto{\pgfqpoint{3.937600in}{3.672657in}}%
\pgfpathclose%
\pgfusepath{fill}%
\end{pgfscope}%
\begin{pgfscope}%
\pgfpathrectangle{\pgfqpoint{3.937600in}{3.271772in}}{\pgfqpoint{2.583333in}{0.400885in}}%
\pgfusepath{clip}%
\pgfsetroundcap%
\pgfsetroundjoin%
\pgfsetlinewidth{0.803000pt}%
\definecolor{currentstroke}{rgb}{1.000000,1.000000,1.000000}%
\pgfsetstrokecolor{currentstroke}%
\pgfsetdash{}{0pt}%
\pgfpathmoveto{\pgfqpoint{4.052877in}{3.271772in}}%
\pgfpathlineto{\pgfqpoint{4.052877in}{3.672657in}}%
\pgfusepath{stroke}%
\end{pgfscope}%
\begin{pgfscope}%
\definecolor{textcolor}{rgb}{0.150000,0.150000,0.150000}%
\pgfsetstrokecolor{textcolor}%
\pgfsetfillcolor{textcolor}%
\pgftext[x=4.052877in,y=3.174550in,,top]{\color{textcolor}\rmfamily\fontsize{14.000000}{16.800000}\selectfont 2012}%
\end{pgfscope}%
\begin{pgfscope}%
\pgfpathrectangle{\pgfqpoint{3.937600in}{3.271772in}}{\pgfqpoint{2.583333in}{0.400885in}}%
\pgfusepath{clip}%
\pgfsetroundcap%
\pgfsetroundjoin%
\pgfsetlinewidth{0.803000pt}%
\definecolor{currentstroke}{rgb}{1.000000,1.000000,1.000000}%
\pgfsetstrokecolor{currentstroke}%
\pgfsetdash{}{0pt}%
\pgfpathmoveto{\pgfqpoint{4.445902in}{3.271772in}}%
\pgfpathlineto{\pgfqpoint{4.445902in}{3.672657in}}%
\pgfusepath{stroke}%
\end{pgfscope}%
\begin{pgfscope}%
\definecolor{textcolor}{rgb}{0.150000,0.150000,0.150000}%
\pgfsetstrokecolor{textcolor}%
\pgfsetfillcolor{textcolor}%
\pgftext[x=4.445902in,y=3.174550in,,top]{\color{textcolor}\rmfamily\fontsize{14.000000}{16.800000}\selectfont 2013}%
\end{pgfscope}%
\begin{pgfscope}%
\pgfpathrectangle{\pgfqpoint{3.937600in}{3.271772in}}{\pgfqpoint{2.583333in}{0.400885in}}%
\pgfusepath{clip}%
\pgfsetroundcap%
\pgfsetroundjoin%
\pgfsetlinewidth{0.803000pt}%
\definecolor{currentstroke}{rgb}{1.000000,1.000000,1.000000}%
\pgfsetstrokecolor{currentstroke}%
\pgfsetdash{}{0pt}%
\pgfpathmoveto{\pgfqpoint{4.837853in}{3.271772in}}%
\pgfpathlineto{\pgfqpoint{4.837853in}{3.672657in}}%
\pgfusepath{stroke}%
\end{pgfscope}%
\begin{pgfscope}%
\definecolor{textcolor}{rgb}{0.150000,0.150000,0.150000}%
\pgfsetstrokecolor{textcolor}%
\pgfsetfillcolor{textcolor}%
\pgftext[x=4.837853in,y=3.174550in,,top]{\color{textcolor}\rmfamily\fontsize{14.000000}{16.800000}\selectfont 2014}%
\end{pgfscope}%
\begin{pgfscope}%
\pgfpathrectangle{\pgfqpoint{3.937600in}{3.271772in}}{\pgfqpoint{2.583333in}{0.400885in}}%
\pgfusepath{clip}%
\pgfsetroundcap%
\pgfsetroundjoin%
\pgfsetlinewidth{0.803000pt}%
\definecolor{currentstroke}{rgb}{1.000000,1.000000,1.000000}%
\pgfsetstrokecolor{currentstroke}%
\pgfsetdash{}{0pt}%
\pgfpathmoveto{\pgfqpoint{5.229804in}{3.271772in}}%
\pgfpathlineto{\pgfqpoint{5.229804in}{3.672657in}}%
\pgfusepath{stroke}%
\end{pgfscope}%
\begin{pgfscope}%
\definecolor{textcolor}{rgb}{0.150000,0.150000,0.150000}%
\pgfsetstrokecolor{textcolor}%
\pgfsetfillcolor{textcolor}%
\pgftext[x=5.229804in,y=3.174550in,,top]{\color{textcolor}\rmfamily\fontsize{14.000000}{16.800000}\selectfont 2015}%
\end{pgfscope}%
\begin{pgfscope}%
\pgfpathrectangle{\pgfqpoint{3.937600in}{3.271772in}}{\pgfqpoint{2.583333in}{0.400885in}}%
\pgfusepath{clip}%
\pgfsetroundcap%
\pgfsetroundjoin%
\pgfsetlinewidth{0.803000pt}%
\definecolor{currentstroke}{rgb}{1.000000,1.000000,1.000000}%
\pgfsetstrokecolor{currentstroke}%
\pgfsetdash{}{0pt}%
\pgfpathmoveto{\pgfqpoint{5.621755in}{3.271772in}}%
\pgfpathlineto{\pgfqpoint{5.621755in}{3.672657in}}%
\pgfusepath{stroke}%
\end{pgfscope}%
\begin{pgfscope}%
\definecolor{textcolor}{rgb}{0.150000,0.150000,0.150000}%
\pgfsetstrokecolor{textcolor}%
\pgfsetfillcolor{textcolor}%
\pgftext[x=5.621755in,y=3.174550in,,top]{\color{textcolor}\rmfamily\fontsize{14.000000}{16.800000}\selectfont 2016}%
\end{pgfscope}%
\begin{pgfscope}%
\pgfpathrectangle{\pgfqpoint{3.937600in}{3.271772in}}{\pgfqpoint{2.583333in}{0.400885in}}%
\pgfusepath{clip}%
\pgfsetroundcap%
\pgfsetroundjoin%
\pgfsetlinewidth{0.803000pt}%
\definecolor{currentstroke}{rgb}{1.000000,1.000000,1.000000}%
\pgfsetstrokecolor{currentstroke}%
\pgfsetdash{}{0pt}%
\pgfpathmoveto{\pgfqpoint{6.014780in}{3.271772in}}%
\pgfpathlineto{\pgfqpoint{6.014780in}{3.672657in}}%
\pgfusepath{stroke}%
\end{pgfscope}%
\begin{pgfscope}%
\definecolor{textcolor}{rgb}{0.150000,0.150000,0.150000}%
\pgfsetstrokecolor{textcolor}%
\pgfsetfillcolor{textcolor}%
\pgftext[x=6.014780in,y=3.174550in,,top]{\color{textcolor}\rmfamily\fontsize{14.000000}{16.800000}\selectfont 2017}%
\end{pgfscope}%
\begin{pgfscope}%
\pgfpathrectangle{\pgfqpoint{3.937600in}{3.271772in}}{\pgfqpoint{2.583333in}{0.400885in}}%
\pgfusepath{clip}%
\pgfsetroundcap%
\pgfsetroundjoin%
\pgfsetlinewidth{0.803000pt}%
\definecolor{currentstroke}{rgb}{1.000000,1.000000,1.000000}%
\pgfsetstrokecolor{currentstroke}%
\pgfsetdash{}{0pt}%
\pgfpathmoveto{\pgfqpoint{6.406731in}{3.271772in}}%
\pgfpathlineto{\pgfqpoint{6.406731in}{3.672657in}}%
\pgfusepath{stroke}%
\end{pgfscope}%
\begin{pgfscope}%
\definecolor{textcolor}{rgb}{0.150000,0.150000,0.150000}%
\pgfsetstrokecolor{textcolor}%
\pgfsetfillcolor{textcolor}%
\pgftext[x=6.406731in,y=3.174550in,,top]{\color{textcolor}\rmfamily\fontsize{14.000000}{16.800000}\selectfont 2018}%
\end{pgfscope}%
\begin{pgfscope}%
\pgfpathrectangle{\pgfqpoint{3.937600in}{3.271772in}}{\pgfqpoint{2.583333in}{0.400885in}}%
\pgfusepath{clip}%
\pgfsetroundcap%
\pgfsetroundjoin%
\pgfsetlinewidth{0.803000pt}%
\definecolor{currentstroke}{rgb}{1.000000,1.000000,1.000000}%
\pgfsetstrokecolor{currentstroke}%
\pgfsetdash{}{0pt}%
\pgfpathmoveto{\pgfqpoint{3.937600in}{3.425433in}}%
\pgfpathlineto{\pgfqpoint{6.520934in}{3.425433in}}%
\pgfusepath{stroke}%
\end{pgfscope}%
\begin{pgfscope}%
\definecolor{textcolor}{rgb}{0.150000,0.150000,0.150000}%
\pgfsetstrokecolor{textcolor}%
\pgfsetfillcolor{textcolor}%
\pgftext[x=3.716667in,y=3.351566in,left,base]{\color{textcolor}\rmfamily\fontsize{14.000000}{16.800000}\selectfont 1}%
\end{pgfscope}%
\begin{pgfscope}%
\pgfpathrectangle{\pgfqpoint{3.937600in}{3.271772in}}{\pgfqpoint{2.583333in}{0.400885in}}%
\pgfusepath{clip}%
\pgfsetroundcap%
\pgfsetroundjoin%
\pgfsetlinewidth{0.803000pt}%
\definecolor{currentstroke}{rgb}{1.000000,1.000000,1.000000}%
\pgfsetstrokecolor{currentstroke}%
\pgfsetdash{}{0pt}%
\pgfpathmoveto{\pgfqpoint{3.937600in}{3.594516in}}%
\pgfpathlineto{\pgfqpoint{6.520934in}{3.594516in}}%
\pgfusepath{stroke}%
\end{pgfscope}%
\begin{pgfscope}%
\definecolor{textcolor}{rgb}{0.150000,0.150000,0.150000}%
\pgfsetstrokecolor{textcolor}%
\pgfsetfillcolor{textcolor}%
\pgftext[x=3.716667in,y=3.520650in,left,base]{\color{textcolor}\rmfamily\fontsize{14.000000}{16.800000}\selectfont 2}%
\end{pgfscope}%
\begin{pgfscope}%
\pgfpathrectangle{\pgfqpoint{3.937600in}{3.271772in}}{\pgfqpoint{2.583333in}{0.400885in}}%
\pgfusepath{clip}%
\pgfsetroundcap%
\pgfsetroundjoin%
\pgfsetlinewidth{1.505625pt}%
\definecolor{currentstroke}{rgb}{0.000000,0.000000,0.000000}%
\pgfsetstrokecolor{currentstroke}%
\pgfsetdash{}{0pt}%
\pgfpathmoveto{\pgfqpoint{4.055025in}{3.425433in}}%
\pgfpathlineto{\pgfqpoint{4.057172in}{3.431381in}}%
\pgfpathlineto{\pgfqpoint{4.058246in}{3.430331in}}%
\pgfpathlineto{\pgfqpoint{4.062542in}{3.432693in}}%
\pgfpathlineto{\pgfqpoint{4.063615in}{3.434180in}}%
\pgfpathlineto{\pgfqpoint{4.064689in}{3.433830in}}%
\pgfpathlineto{\pgfqpoint{4.065763in}{3.429631in}}%
\pgfpathlineto{\pgfqpoint{4.070058in}{3.428931in}}%
\pgfpathlineto{\pgfqpoint{4.072206in}{3.432955in}}%
\pgfpathlineto{\pgfqpoint{4.073280in}{3.438116in}}%
\pgfpathlineto{\pgfqpoint{4.078649in}{3.441702in}}%
\pgfpathlineto{\pgfqpoint{4.079723in}{3.440653in}}%
\pgfpathlineto{\pgfqpoint{4.084018in}{3.440653in}}%
\pgfpathlineto{\pgfqpoint{4.085092in}{3.438378in}}%
\pgfpathlineto{\pgfqpoint{4.086166in}{3.439341in}}%
\pgfpathlineto{\pgfqpoint{4.087240in}{3.438903in}}%
\pgfpathlineto{\pgfqpoint{4.088314in}{3.442052in}}%
\pgfpathlineto{\pgfqpoint{4.091535in}{3.441965in}}%
\pgfpathlineto{\pgfqpoint{4.092609in}{3.441352in}}%
\pgfpathlineto{\pgfqpoint{4.093683in}{3.442839in}}%
\pgfpathlineto{\pgfqpoint{4.094757in}{3.442927in}}%
\pgfpathlineto{\pgfqpoint{4.095831in}{3.441790in}}%
\pgfpathlineto{\pgfqpoint{4.099052in}{3.441790in}}%
\pgfpathlineto{\pgfqpoint{4.100126in}{3.442402in}}%
\pgfpathlineto{\pgfqpoint{4.101200in}{3.441003in}}%
\pgfpathlineto{\pgfqpoint{4.102274in}{3.442752in}}%
\pgfpathlineto{\pgfqpoint{4.103347in}{3.446426in}}%
\pgfpathlineto{\pgfqpoint{4.107643in}{3.445026in}}%
\pgfpathlineto{\pgfqpoint{4.108717in}{3.442052in}}%
\pgfpathlineto{\pgfqpoint{4.109790in}{3.441527in}}%
\pgfpathlineto{\pgfqpoint{4.114086in}{3.443102in}}%
\pgfpathlineto{\pgfqpoint{4.115160in}{3.445551in}}%
\pgfpathlineto{\pgfqpoint{4.116233in}{3.443102in}}%
\pgfpathlineto{\pgfqpoint{4.117307in}{3.442927in}}%
\pgfpathlineto{\pgfqpoint{4.118381in}{3.443364in}}%
\pgfpathlineto{\pgfqpoint{4.121603in}{3.440740in}}%
\pgfpathlineto{\pgfqpoint{4.122676in}{3.441178in}}%
\pgfpathlineto{\pgfqpoint{4.123750in}{3.443277in}}%
\pgfpathlineto{\pgfqpoint{4.124824in}{3.442752in}}%
\pgfpathlineto{\pgfqpoint{4.125898in}{3.444414in}}%
\pgfpathlineto{\pgfqpoint{4.129120in}{3.443802in}}%
\pgfpathlineto{\pgfqpoint{4.130193in}{3.447301in}}%
\pgfpathlineto{\pgfqpoint{4.131267in}{3.447126in}}%
\pgfpathlineto{\pgfqpoint{4.132341in}{3.449137in}}%
\pgfpathlineto{\pgfqpoint{4.133415in}{3.448963in}}%
\pgfpathlineto{\pgfqpoint{4.138784in}{3.449312in}}%
\pgfpathlineto{\pgfqpoint{4.139858in}{3.450187in}}%
\pgfpathlineto{\pgfqpoint{4.140932in}{3.450012in}}%
\pgfpathlineto{\pgfqpoint{4.144153in}{3.452199in}}%
\pgfpathlineto{\pgfqpoint{4.145227in}{3.452199in}}%
\pgfpathlineto{\pgfqpoint{4.146301in}{3.449487in}}%
\pgfpathlineto{\pgfqpoint{4.147375in}{3.451937in}}%
\pgfpathlineto{\pgfqpoint{4.148449in}{3.451674in}}%
\pgfpathlineto{\pgfqpoint{4.151670in}{3.453511in}}%
\pgfpathlineto{\pgfqpoint{4.153818in}{3.450362in}}%
\pgfpathlineto{\pgfqpoint{4.154892in}{3.451324in}}%
\pgfpathlineto{\pgfqpoint{4.159187in}{3.449137in}}%
\pgfpathlineto{\pgfqpoint{4.160261in}{3.447038in}}%
\pgfpathlineto{\pgfqpoint{4.161335in}{3.449837in}}%
\pgfpathlineto{\pgfqpoint{4.162409in}{3.454211in}}%
\pgfpathlineto{\pgfqpoint{4.163482in}{3.451499in}}%
\pgfpathlineto{\pgfqpoint{4.167778in}{3.454123in}}%
\pgfpathlineto{\pgfqpoint{4.169925in}{3.448700in}}%
\pgfpathlineto{\pgfqpoint{4.170999in}{3.448088in}}%
\pgfpathlineto{\pgfqpoint{4.174221in}{3.447038in}}%
\pgfpathlineto{\pgfqpoint{4.175295in}{3.446076in}}%
\pgfpathlineto{\pgfqpoint{4.177442in}{3.452374in}}%
\pgfpathlineto{\pgfqpoint{4.178516in}{3.453511in}}%
\pgfpathlineto{\pgfqpoint{4.181738in}{3.453599in}}%
\pgfpathlineto{\pgfqpoint{4.182811in}{3.457447in}}%
\pgfpathlineto{\pgfqpoint{4.183885in}{3.459022in}}%
\pgfpathlineto{\pgfqpoint{4.184959in}{3.456135in}}%
\pgfpathlineto{\pgfqpoint{4.186033in}{3.451587in}}%
\pgfpathlineto{\pgfqpoint{4.189254in}{3.450537in}}%
\pgfpathlineto{\pgfqpoint{4.191402in}{3.446601in}}%
\pgfpathlineto{\pgfqpoint{4.192476in}{3.446951in}}%
\pgfpathlineto{\pgfqpoint{4.193550in}{3.449662in}}%
\pgfpathlineto{\pgfqpoint{4.198919in}{3.441790in}}%
\pgfpathlineto{\pgfqpoint{4.199993in}{3.439603in}}%
\pgfpathlineto{\pgfqpoint{4.201067in}{3.438728in}}%
\pgfpathlineto{\pgfqpoint{4.204288in}{3.439341in}}%
\pgfpathlineto{\pgfqpoint{4.205362in}{3.438466in}}%
\pgfpathlineto{\pgfqpoint{4.206436in}{3.434355in}}%
\pgfpathlineto{\pgfqpoint{4.208584in}{3.436454in}}%
\pgfpathlineto{\pgfqpoint{4.213953in}{3.439166in}}%
\pgfpathlineto{\pgfqpoint{4.215027in}{3.437154in}}%
\pgfpathlineto{\pgfqpoint{4.216100in}{3.432255in}}%
\pgfpathlineto{\pgfqpoint{4.219322in}{3.431556in}}%
\pgfpathlineto{\pgfqpoint{4.220396in}{3.434267in}}%
\pgfpathlineto{\pgfqpoint{4.221470in}{3.438728in}}%
\pgfpathlineto{\pgfqpoint{4.222543in}{3.437854in}}%
\pgfpathlineto{\pgfqpoint{4.223617in}{3.441090in}}%
\pgfpathlineto{\pgfqpoint{4.226839in}{3.438203in}}%
\pgfpathlineto{\pgfqpoint{4.227913in}{3.441877in}}%
\pgfpathlineto{\pgfqpoint{4.228986in}{3.442052in}}%
\pgfpathlineto{\pgfqpoint{4.231134in}{3.447650in}}%
\pgfpathlineto{\pgfqpoint{4.235430in}{3.448788in}}%
\pgfpathlineto{\pgfqpoint{4.236503in}{3.449750in}}%
\pgfpathlineto{\pgfqpoint{4.237577in}{3.443189in}}%
\pgfpathlineto{\pgfqpoint{4.238651in}{3.444851in}}%
\pgfpathlineto{\pgfqpoint{4.241873in}{3.438641in}}%
\pgfpathlineto{\pgfqpoint{4.242946in}{3.438291in}}%
\pgfpathlineto{\pgfqpoint{4.244020in}{3.439778in}}%
\pgfpathlineto{\pgfqpoint{4.245094in}{3.437066in}}%
\pgfpathlineto{\pgfqpoint{4.246168in}{3.442839in}}%
\pgfpathlineto{\pgfqpoint{4.249389in}{3.442927in}}%
\pgfpathlineto{\pgfqpoint{4.250463in}{3.444239in}}%
\pgfpathlineto{\pgfqpoint{4.252611in}{3.442140in}}%
\pgfpathlineto{\pgfqpoint{4.253685in}{3.439341in}}%
\pgfpathlineto{\pgfqpoint{4.256906in}{3.439428in}}%
\pgfpathlineto{\pgfqpoint{4.257980in}{3.435142in}}%
\pgfpathlineto{\pgfqpoint{4.259054in}{3.434005in}}%
\pgfpathlineto{\pgfqpoint{4.260128in}{3.429456in}}%
\pgfpathlineto{\pgfqpoint{4.261202in}{3.433043in}}%
\pgfpathlineto{\pgfqpoint{4.264423in}{3.432168in}}%
\pgfpathlineto{\pgfqpoint{4.265497in}{3.433917in}}%
\pgfpathlineto{\pgfqpoint{4.266571in}{3.439690in}}%
\pgfpathlineto{\pgfqpoint{4.267645in}{3.438641in}}%
\pgfpathlineto{\pgfqpoint{4.268719in}{3.434880in}}%
\pgfpathlineto{\pgfqpoint{4.271940in}{3.433043in}}%
\pgfpathlineto{\pgfqpoint{4.273014in}{3.431293in}}%
\pgfpathlineto{\pgfqpoint{4.274088in}{3.432168in}}%
\pgfpathlineto{\pgfqpoint{4.276235in}{3.438378in}}%
\pgfpathlineto{\pgfqpoint{4.280531in}{3.436192in}}%
\pgfpathlineto{\pgfqpoint{4.281605in}{3.437766in}}%
\pgfpathlineto{\pgfqpoint{4.282678in}{3.437591in}}%
\pgfpathlineto{\pgfqpoint{4.283752in}{3.441440in}}%
\pgfpathlineto{\pgfqpoint{4.286974in}{3.442052in}}%
\pgfpathlineto{\pgfqpoint{4.289121in}{3.444064in}}%
\pgfpathlineto{\pgfqpoint{4.291269in}{3.446076in}}%
\pgfpathlineto{\pgfqpoint{4.294491in}{3.444764in}}%
\pgfpathlineto{\pgfqpoint{4.296638in}{3.441790in}}%
\pgfpathlineto{\pgfqpoint{4.297712in}{3.443977in}}%
\pgfpathlineto{\pgfqpoint{4.298786in}{3.442140in}}%
\pgfpathlineto{\pgfqpoint{4.303081in}{3.440653in}}%
\pgfpathlineto{\pgfqpoint{4.304155in}{3.437941in}}%
\pgfpathlineto{\pgfqpoint{4.305229in}{3.433043in}}%
\pgfpathlineto{\pgfqpoint{4.306303in}{3.432168in}}%
\pgfpathlineto{\pgfqpoint{4.309524in}{3.431643in}}%
\pgfpathlineto{\pgfqpoint{4.310598in}{3.432780in}}%
\pgfpathlineto{\pgfqpoint{4.312746in}{3.427619in}}%
\pgfpathlineto{\pgfqpoint{4.313820in}{3.431556in}}%
\pgfpathlineto{\pgfqpoint{4.319189in}{3.428494in}}%
\pgfpathlineto{\pgfqpoint{4.320263in}{3.433480in}}%
\pgfpathlineto{\pgfqpoint{4.321337in}{3.427094in}}%
\pgfpathlineto{\pgfqpoint{4.324558in}{3.420534in}}%
\pgfpathlineto{\pgfqpoint{4.325632in}{3.421059in}}%
\pgfpathlineto{\pgfqpoint{4.326706in}{3.420009in}}%
\pgfpathlineto{\pgfqpoint{4.327780in}{3.421234in}}%
\pgfpathlineto{\pgfqpoint{4.328853in}{3.421321in}}%
\pgfpathlineto{\pgfqpoint{4.332075in}{3.420884in}}%
\pgfpathlineto{\pgfqpoint{4.333149in}{3.421321in}}%
\pgfpathlineto{\pgfqpoint{4.334223in}{3.419747in}}%
\pgfpathlineto{\pgfqpoint{4.336370in}{3.419572in}}%
\pgfpathlineto{\pgfqpoint{4.339592in}{3.417298in}}%
\pgfpathlineto{\pgfqpoint{4.340666in}{3.415461in}}%
\pgfpathlineto{\pgfqpoint{4.341740in}{3.416248in}}%
\pgfpathlineto{\pgfqpoint{4.342813in}{3.419309in}}%
\pgfpathlineto{\pgfqpoint{4.343887in}{3.416248in}}%
\pgfpathlineto{\pgfqpoint{4.348183in}{3.417560in}}%
\pgfpathlineto{\pgfqpoint{4.349256in}{3.415461in}}%
\pgfpathlineto{\pgfqpoint{4.350330in}{3.414936in}}%
\pgfpathlineto{\pgfqpoint{4.351404in}{3.416423in}}%
\pgfpathlineto{\pgfqpoint{4.354626in}{3.415198in}}%
\pgfpathlineto{\pgfqpoint{4.355699in}{3.410912in}}%
\pgfpathlineto{\pgfqpoint{4.358921in}{3.407938in}}%
\pgfpathlineto{\pgfqpoint{4.362142in}{3.409688in}}%
\pgfpathlineto{\pgfqpoint{4.363216in}{3.414061in}}%
\pgfpathlineto{\pgfqpoint{4.364290in}{3.410125in}}%
\pgfpathlineto{\pgfqpoint{4.365364in}{3.409250in}}%
\pgfpathlineto{\pgfqpoint{4.366438in}{3.406451in}}%
\pgfpathlineto{\pgfqpoint{4.370733in}{3.408725in}}%
\pgfpathlineto{\pgfqpoint{4.371807in}{3.407851in}}%
\pgfpathlineto{\pgfqpoint{4.373955in}{3.411262in}}%
\pgfpathlineto{\pgfqpoint{4.379324in}{3.408988in}}%
\pgfpathlineto{\pgfqpoint{4.380398in}{3.413449in}}%
\pgfpathlineto{\pgfqpoint{4.381472in}{3.412049in}}%
\pgfpathlineto{\pgfqpoint{4.384693in}{3.412049in}}%
\pgfpathlineto{\pgfqpoint{4.385767in}{3.411262in}}%
\pgfpathlineto{\pgfqpoint{4.386841in}{3.405489in}}%
\pgfpathlineto{\pgfqpoint{4.388988in}{3.404702in}}%
\pgfpathlineto{\pgfqpoint{4.392210in}{3.404439in}}%
\pgfpathlineto{\pgfqpoint{4.394358in}{3.398666in}}%
\pgfpathlineto{\pgfqpoint{4.399727in}{3.400765in}}%
\pgfpathlineto{\pgfqpoint{4.400801in}{3.395430in}}%
\pgfpathlineto{\pgfqpoint{4.401875in}{3.394380in}}%
\pgfpathlineto{\pgfqpoint{4.404022in}{3.397004in}}%
\pgfpathlineto{\pgfqpoint{4.408318in}{3.398491in}}%
\pgfpathlineto{\pgfqpoint{4.409391in}{3.399628in}}%
\pgfpathlineto{\pgfqpoint{4.410465in}{3.395604in}}%
\pgfpathlineto{\pgfqpoint{4.411539in}{3.395867in}}%
\pgfpathlineto{\pgfqpoint{4.414761in}{3.395692in}}%
\pgfpathlineto{\pgfqpoint{4.415834in}{3.398753in}}%
\pgfpathlineto{\pgfqpoint{4.416908in}{3.397879in}}%
\pgfpathlineto{\pgfqpoint{4.417982in}{3.400066in}}%
\pgfpathlineto{\pgfqpoint{4.422277in}{3.399541in}}%
\pgfpathlineto{\pgfqpoint{4.423351in}{3.403564in}}%
\pgfpathlineto{\pgfqpoint{4.424425in}{3.403739in}}%
\pgfpathlineto{\pgfqpoint{4.425499in}{3.402427in}}%
\pgfpathlineto{\pgfqpoint{4.426573in}{3.402777in}}%
\pgfpathlineto{\pgfqpoint{4.429794in}{3.403040in}}%
\pgfpathlineto{\pgfqpoint{4.430868in}{3.405839in}}%
\pgfpathlineto{\pgfqpoint{4.431942in}{3.406801in}}%
\pgfpathlineto{\pgfqpoint{4.433016in}{3.406276in}}%
\pgfpathlineto{\pgfqpoint{4.434090in}{3.404439in}}%
\pgfpathlineto{\pgfqpoint{4.437311in}{3.403564in}}%
\pgfpathlineto{\pgfqpoint{4.439459in}{3.403564in}}%
\pgfpathlineto{\pgfqpoint{4.440533in}{3.402602in}}%
\pgfpathlineto{\pgfqpoint{4.441607in}{3.400590in}}%
\pgfpathlineto{\pgfqpoint{4.444828in}{3.403390in}}%
\pgfpathlineto{\pgfqpoint{4.446976in}{3.408813in}}%
\pgfpathlineto{\pgfqpoint{4.448050in}{3.408375in}}%
\pgfpathlineto{\pgfqpoint{4.449123in}{3.407238in}}%
\pgfpathlineto{\pgfqpoint{4.452345in}{3.407851in}}%
\pgfpathlineto{\pgfqpoint{4.453419in}{3.406713in}}%
\pgfpathlineto{\pgfqpoint{4.456640in}{3.413186in}}%
\pgfpathlineto{\pgfqpoint{4.459862in}{3.413186in}}%
\pgfpathlineto{\pgfqpoint{4.460936in}{3.412399in}}%
\pgfpathlineto{\pgfqpoint{4.462009in}{3.413974in}}%
\pgfpathlineto{\pgfqpoint{4.463083in}{3.418085in}}%
\pgfpathlineto{\pgfqpoint{4.464157in}{3.407851in}}%
\pgfpathlineto{\pgfqpoint{4.469526in}{3.406888in}}%
\pgfpathlineto{\pgfqpoint{4.470600in}{3.405751in}}%
\pgfpathlineto{\pgfqpoint{4.474896in}{3.406451in}}%
\pgfpathlineto{\pgfqpoint{4.475969in}{3.408113in}}%
\pgfpathlineto{\pgfqpoint{4.477043in}{3.408725in}}%
\pgfpathlineto{\pgfqpoint{4.478117in}{3.406364in}}%
\pgfpathlineto{\pgfqpoint{4.479191in}{3.408638in}}%
\pgfpathlineto{\pgfqpoint{4.482412in}{3.407238in}}%
\pgfpathlineto{\pgfqpoint{4.483486in}{3.408988in}}%
\pgfpathlineto{\pgfqpoint{4.485634in}{3.406364in}}%
\pgfpathlineto{\pgfqpoint{4.486708in}{3.407676in}}%
\pgfpathlineto{\pgfqpoint{4.489929in}{3.407938in}}%
\pgfpathlineto{\pgfqpoint{4.492077in}{3.409513in}}%
\pgfpathlineto{\pgfqpoint{4.493151in}{3.409338in}}%
\pgfpathlineto{\pgfqpoint{4.494225in}{3.408550in}}%
\pgfpathlineto{\pgfqpoint{4.498520in}{3.408375in}}%
\pgfpathlineto{\pgfqpoint{4.500668in}{3.402252in}}%
\pgfpathlineto{\pgfqpoint{4.501741in}{3.403477in}}%
\pgfpathlineto{\pgfqpoint{4.504963in}{3.402165in}}%
\pgfpathlineto{\pgfqpoint{4.507111in}{3.407151in}}%
\pgfpathlineto{\pgfqpoint{4.508185in}{3.406801in}}%
\pgfpathlineto{\pgfqpoint{4.509258in}{3.407938in}}%
\pgfpathlineto{\pgfqpoint{4.512480in}{3.409688in}}%
\pgfpathlineto{\pgfqpoint{4.515701in}{3.414149in}}%
\pgfpathlineto{\pgfqpoint{4.516775in}{3.411874in}}%
\pgfpathlineto{\pgfqpoint{4.519997in}{3.412662in}}%
\pgfpathlineto{\pgfqpoint{4.522144in}{3.412487in}}%
\pgfpathlineto{\pgfqpoint{4.523218in}{3.412399in}}%
\pgfpathlineto{\pgfqpoint{4.524292in}{3.410475in}}%
\pgfpathlineto{\pgfqpoint{4.527514in}{3.409600in}}%
\pgfpathlineto{\pgfqpoint{4.528587in}{3.408725in}}%
\pgfpathlineto{\pgfqpoint{4.529661in}{3.408988in}}%
\pgfpathlineto{\pgfqpoint{4.530735in}{3.408026in}}%
\pgfpathlineto{\pgfqpoint{4.531809in}{3.410037in}}%
\pgfpathlineto{\pgfqpoint{4.535030in}{3.408813in}}%
\pgfpathlineto{\pgfqpoint{4.536104in}{3.413274in}}%
\pgfpathlineto{\pgfqpoint{4.538252in}{3.413711in}}%
\pgfpathlineto{\pgfqpoint{4.542547in}{3.410825in}}%
\pgfpathlineto{\pgfqpoint{4.543621in}{3.411000in}}%
\pgfpathlineto{\pgfqpoint{4.544695in}{3.408026in}}%
\pgfpathlineto{\pgfqpoint{4.545769in}{3.408725in}}%
\pgfpathlineto{\pgfqpoint{4.546843in}{3.407238in}}%
\pgfpathlineto{\pgfqpoint{4.550064in}{3.408375in}}%
\pgfpathlineto{\pgfqpoint{4.552212in}{3.416773in}}%
\pgfpathlineto{\pgfqpoint{4.553286in}{3.413711in}}%
\pgfpathlineto{\pgfqpoint{4.554360in}{3.412574in}}%
\pgfpathlineto{\pgfqpoint{4.557581in}{3.410475in}}%
\pgfpathlineto{\pgfqpoint{4.558655in}{3.414324in}}%
\pgfpathlineto{\pgfqpoint{4.559729in}{3.414411in}}%
\pgfpathlineto{\pgfqpoint{4.561876in}{3.418085in}}%
\pgfpathlineto{\pgfqpoint{4.565098in}{3.421234in}}%
\pgfpathlineto{\pgfqpoint{4.567246in}{3.426832in}}%
\pgfpathlineto{\pgfqpoint{4.568319in}{3.424820in}}%
\pgfpathlineto{\pgfqpoint{4.569393in}{3.424995in}}%
\pgfpathlineto{\pgfqpoint{4.576910in}{3.430681in}}%
\pgfpathlineto{\pgfqpoint{4.580132in}{3.430331in}}%
\pgfpathlineto{\pgfqpoint{4.582279in}{3.432780in}}%
\pgfpathlineto{\pgfqpoint{4.584427in}{3.434617in}}%
\pgfpathlineto{\pgfqpoint{4.588722in}{3.429806in}}%
\pgfpathlineto{\pgfqpoint{4.589796in}{3.432430in}}%
\pgfpathlineto{\pgfqpoint{4.590870in}{3.430506in}}%
\pgfpathlineto{\pgfqpoint{4.591944in}{3.431206in}}%
\pgfpathlineto{\pgfqpoint{4.597313in}{3.431468in}}%
\pgfpathlineto{\pgfqpoint{4.598387in}{3.431293in}}%
\pgfpathlineto{\pgfqpoint{4.599461in}{3.430331in}}%
\pgfpathlineto{\pgfqpoint{4.603756in}{3.431556in}}%
\pgfpathlineto{\pgfqpoint{4.604830in}{3.432955in}}%
\pgfpathlineto{\pgfqpoint{4.605904in}{3.432518in}}%
\pgfpathlineto{\pgfqpoint{4.606978in}{3.432955in}}%
\pgfpathlineto{\pgfqpoint{4.610199in}{3.439953in}}%
\pgfpathlineto{\pgfqpoint{4.611273in}{3.440828in}}%
\pgfpathlineto{\pgfqpoint{4.612347in}{3.436017in}}%
\pgfpathlineto{\pgfqpoint{4.614495in}{3.435229in}}%
\pgfpathlineto{\pgfqpoint{4.617716in}{3.438291in}}%
\pgfpathlineto{\pgfqpoint{4.619864in}{3.434267in}}%
\pgfpathlineto{\pgfqpoint{4.620938in}{3.438116in}}%
\pgfpathlineto{\pgfqpoint{4.622011in}{3.437679in}}%
\pgfpathlineto{\pgfqpoint{4.625233in}{3.438991in}}%
\pgfpathlineto{\pgfqpoint{4.626307in}{3.441615in}}%
\pgfpathlineto{\pgfqpoint{4.627381in}{3.438203in}}%
\pgfpathlineto{\pgfqpoint{4.628454in}{3.432343in}}%
\pgfpathlineto{\pgfqpoint{4.629528in}{3.432430in}}%
\pgfpathlineto{\pgfqpoint{4.632750in}{3.427882in}}%
\pgfpathlineto{\pgfqpoint{4.634897in}{3.431031in}}%
\pgfpathlineto{\pgfqpoint{4.635971in}{3.431293in}}%
\pgfpathlineto{\pgfqpoint{4.637045in}{3.432605in}}%
\pgfpathlineto{\pgfqpoint{4.642414in}{3.429194in}}%
\pgfpathlineto{\pgfqpoint{4.644562in}{3.431381in}}%
\pgfpathlineto{\pgfqpoint{4.647784in}{3.425083in}}%
\pgfpathlineto{\pgfqpoint{4.648857in}{3.424733in}}%
\pgfpathlineto{\pgfqpoint{4.649931in}{3.425520in}}%
\pgfpathlineto{\pgfqpoint{4.651005in}{3.430856in}}%
\pgfpathlineto{\pgfqpoint{4.652079in}{3.430243in}}%
\pgfpathlineto{\pgfqpoint{4.655300in}{3.430506in}}%
\pgfpathlineto{\pgfqpoint{4.656374in}{3.432780in}}%
\pgfpathlineto{\pgfqpoint{4.657448in}{3.432080in}}%
\pgfpathlineto{\pgfqpoint{4.658522in}{3.425433in}}%
\pgfpathlineto{\pgfqpoint{4.659596in}{3.423945in}}%
\pgfpathlineto{\pgfqpoint{4.663891in}{3.421846in}}%
\pgfpathlineto{\pgfqpoint{4.667113in}{3.425607in}}%
\pgfpathlineto{\pgfqpoint{4.670334in}{3.425433in}}%
\pgfpathlineto{\pgfqpoint{4.671408in}{3.426482in}}%
\pgfpathlineto{\pgfqpoint{4.672482in}{3.426132in}}%
\pgfpathlineto{\pgfqpoint{4.673556in}{3.425170in}}%
\pgfpathlineto{\pgfqpoint{4.674629in}{3.425258in}}%
\pgfpathlineto{\pgfqpoint{4.677851in}{3.424733in}}%
\pgfpathlineto{\pgfqpoint{4.679999in}{3.423071in}}%
\pgfpathlineto{\pgfqpoint{4.681073in}{3.421234in}}%
\pgfpathlineto{\pgfqpoint{4.685368in}{3.422633in}}%
\pgfpathlineto{\pgfqpoint{4.686442in}{3.421759in}}%
\pgfpathlineto{\pgfqpoint{4.687516in}{3.422196in}}%
\pgfpathlineto{\pgfqpoint{4.688589in}{3.418172in}}%
\pgfpathlineto{\pgfqpoint{4.689663in}{3.417298in}}%
\pgfpathlineto{\pgfqpoint{4.692885in}{3.420009in}}%
\pgfpathlineto{\pgfqpoint{4.693959in}{3.421759in}}%
\pgfpathlineto{\pgfqpoint{4.695032in}{3.419222in}}%
\pgfpathlineto{\pgfqpoint{4.696106in}{3.419834in}}%
\pgfpathlineto{\pgfqpoint{4.697180in}{3.421234in}}%
\pgfpathlineto{\pgfqpoint{4.701475in}{3.419397in}}%
\pgfpathlineto{\pgfqpoint{4.702549in}{3.420097in}}%
\pgfpathlineto{\pgfqpoint{4.703623in}{3.418435in}}%
\pgfpathlineto{\pgfqpoint{4.704697in}{3.417822in}}%
\pgfpathlineto{\pgfqpoint{4.708992in}{3.418522in}}%
\pgfpathlineto{\pgfqpoint{4.710066in}{3.422633in}}%
\pgfpathlineto{\pgfqpoint{4.711140in}{3.422371in}}%
\pgfpathlineto{\pgfqpoint{4.716509in}{3.425170in}}%
\pgfpathlineto{\pgfqpoint{4.718657in}{3.422633in}}%
\pgfpathlineto{\pgfqpoint{4.719731in}{3.428582in}}%
\pgfpathlineto{\pgfqpoint{4.722952in}{3.428144in}}%
\pgfpathlineto{\pgfqpoint{4.724026in}{3.430768in}}%
\pgfpathlineto{\pgfqpoint{4.725100in}{3.431905in}}%
\pgfpathlineto{\pgfqpoint{4.726174in}{3.431993in}}%
\pgfpathlineto{\pgfqpoint{4.727248in}{3.430943in}}%
\pgfpathlineto{\pgfqpoint{4.730469in}{3.429894in}}%
\pgfpathlineto{\pgfqpoint{4.732617in}{3.430418in}}%
\pgfpathlineto{\pgfqpoint{4.734764in}{3.425170in}}%
\pgfpathlineto{\pgfqpoint{4.740134in}{3.424470in}}%
\pgfpathlineto{\pgfqpoint{4.741207in}{3.422371in}}%
\pgfpathlineto{\pgfqpoint{4.742281in}{3.423945in}}%
\pgfpathlineto{\pgfqpoint{4.745503in}{3.424033in}}%
\pgfpathlineto{\pgfqpoint{4.746577in}{3.421496in}}%
\pgfpathlineto{\pgfqpoint{4.747651in}{3.422284in}}%
\pgfpathlineto{\pgfqpoint{4.748724in}{3.426045in}}%
\pgfpathlineto{\pgfqpoint{4.749798in}{3.427182in}}%
\pgfpathlineto{\pgfqpoint{4.753020in}{3.428582in}}%
\pgfpathlineto{\pgfqpoint{4.754094in}{3.428144in}}%
\pgfpathlineto{\pgfqpoint{4.756241in}{3.432080in}}%
\pgfpathlineto{\pgfqpoint{4.757315in}{3.431818in}}%
\pgfpathlineto{\pgfqpoint{4.760537in}{3.433655in}}%
\pgfpathlineto{\pgfqpoint{4.761610in}{3.433130in}}%
\pgfpathlineto{\pgfqpoint{4.762684in}{3.430768in}}%
\pgfpathlineto{\pgfqpoint{4.763758in}{3.431031in}}%
\pgfpathlineto{\pgfqpoint{4.764832in}{3.434442in}}%
\pgfpathlineto{\pgfqpoint{4.768053in}{3.435317in}}%
\pgfpathlineto{\pgfqpoint{4.769127in}{3.436454in}}%
\pgfpathlineto{\pgfqpoint{4.771275in}{3.436104in}}%
\pgfpathlineto{\pgfqpoint{4.772349in}{3.435054in}}%
\pgfpathlineto{\pgfqpoint{4.776644in}{3.434530in}}%
\pgfpathlineto{\pgfqpoint{4.777718in}{3.436192in}}%
\pgfpathlineto{\pgfqpoint{4.778792in}{3.434792in}}%
\pgfpathlineto{\pgfqpoint{4.783087in}{3.435579in}}%
\pgfpathlineto{\pgfqpoint{4.785235in}{3.438728in}}%
\pgfpathlineto{\pgfqpoint{4.786309in}{3.437241in}}%
\pgfpathlineto{\pgfqpoint{4.787383in}{3.438203in}}%
\pgfpathlineto{\pgfqpoint{4.790604in}{3.438728in}}%
\pgfpathlineto{\pgfqpoint{4.791678in}{3.439516in}}%
\pgfpathlineto{\pgfqpoint{4.792752in}{3.438466in}}%
\pgfpathlineto{\pgfqpoint{4.793826in}{3.443452in}}%
\pgfpathlineto{\pgfqpoint{4.794899in}{3.433305in}}%
\pgfpathlineto{\pgfqpoint{4.799195in}{3.431731in}}%
\pgfpathlineto{\pgfqpoint{4.800269in}{3.433567in}}%
\pgfpathlineto{\pgfqpoint{4.805638in}{3.432080in}}%
\pgfpathlineto{\pgfqpoint{4.806712in}{3.430943in}}%
\pgfpathlineto{\pgfqpoint{4.807785in}{3.432343in}}%
\pgfpathlineto{\pgfqpoint{4.809933in}{3.440390in}}%
\pgfpathlineto{\pgfqpoint{4.813155in}{3.441178in}}%
\pgfpathlineto{\pgfqpoint{4.814228in}{3.440390in}}%
\pgfpathlineto{\pgfqpoint{4.815302in}{3.437416in}}%
\pgfpathlineto{\pgfqpoint{4.816376in}{3.437766in}}%
\pgfpathlineto{\pgfqpoint{4.817450in}{3.436454in}}%
\pgfpathlineto{\pgfqpoint{4.820672in}{3.437679in}}%
\pgfpathlineto{\pgfqpoint{4.821745in}{3.439166in}}%
\pgfpathlineto{\pgfqpoint{4.822819in}{3.442839in}}%
\pgfpathlineto{\pgfqpoint{4.824967in}{3.442140in}}%
\pgfpathlineto{\pgfqpoint{4.829262in}{3.444939in}}%
\pgfpathlineto{\pgfqpoint{4.831410in}{3.446951in}}%
\pgfpathlineto{\pgfqpoint{4.832484in}{3.446163in}}%
\pgfpathlineto{\pgfqpoint{4.836779in}{3.448875in}}%
\pgfpathlineto{\pgfqpoint{4.843222in}{3.445114in}}%
\pgfpathlineto{\pgfqpoint{4.844296in}{3.446076in}}%
\pgfpathlineto{\pgfqpoint{4.846444in}{3.444064in}}%
\pgfpathlineto{\pgfqpoint{4.847517in}{3.445639in}}%
\pgfpathlineto{\pgfqpoint{4.850739in}{3.445464in}}%
\pgfpathlineto{\pgfqpoint{4.851813in}{3.452899in}}%
\pgfpathlineto{\pgfqpoint{4.852887in}{3.454123in}}%
\pgfpathlineto{\pgfqpoint{4.853961in}{3.453161in}}%
\pgfpathlineto{\pgfqpoint{4.855034in}{3.448000in}}%
\pgfpathlineto{\pgfqpoint{4.859330in}{3.446076in}}%
\pgfpathlineto{\pgfqpoint{4.862551in}{3.440303in}}%
\pgfpathlineto{\pgfqpoint{4.865773in}{3.439690in}}%
\pgfpathlineto{\pgfqpoint{4.866847in}{3.441003in}}%
\pgfpathlineto{\pgfqpoint{4.867920in}{3.439341in}}%
\pgfpathlineto{\pgfqpoint{4.868994in}{3.439778in}}%
\pgfpathlineto{\pgfqpoint{4.874363in}{3.432955in}}%
\pgfpathlineto{\pgfqpoint{4.875437in}{3.432430in}}%
\pgfpathlineto{\pgfqpoint{4.877585in}{3.437591in}}%
\pgfpathlineto{\pgfqpoint{4.880806in}{3.438203in}}%
\pgfpathlineto{\pgfqpoint{4.884028in}{3.441265in}}%
\pgfpathlineto{\pgfqpoint{4.885102in}{3.441702in}}%
\pgfpathlineto{\pgfqpoint{4.889397in}{3.441702in}}%
\pgfpathlineto{\pgfqpoint{4.890471in}{3.439778in}}%
\pgfpathlineto{\pgfqpoint{4.891545in}{3.441527in}}%
\pgfpathlineto{\pgfqpoint{4.892619in}{3.439166in}}%
\pgfpathlineto{\pgfqpoint{4.899062in}{3.441702in}}%
\pgfpathlineto{\pgfqpoint{4.900136in}{3.441702in}}%
\pgfpathlineto{\pgfqpoint{4.903357in}{3.439778in}}%
\pgfpathlineto{\pgfqpoint{4.904431in}{3.440565in}}%
\pgfpathlineto{\pgfqpoint{4.905505in}{3.439778in}}%
\pgfpathlineto{\pgfqpoint{4.906579in}{3.440740in}}%
\pgfpathlineto{\pgfqpoint{4.907652in}{3.440828in}}%
\pgfpathlineto{\pgfqpoint{4.910874in}{3.442315in}}%
\pgfpathlineto{\pgfqpoint{4.911948in}{3.441440in}}%
\pgfpathlineto{\pgfqpoint{4.913022in}{3.441702in}}%
\pgfpathlineto{\pgfqpoint{4.915169in}{3.439778in}}%
\pgfpathlineto{\pgfqpoint{4.919465in}{3.442140in}}%
\pgfpathlineto{\pgfqpoint{4.920539in}{3.443627in}}%
\pgfpathlineto{\pgfqpoint{4.921612in}{3.446688in}}%
\pgfpathlineto{\pgfqpoint{4.922686in}{3.444764in}}%
\pgfpathlineto{\pgfqpoint{4.925908in}{3.444414in}}%
\pgfpathlineto{\pgfqpoint{4.926982in}{3.446951in}}%
\pgfpathlineto{\pgfqpoint{4.929129in}{3.445814in}}%
\pgfpathlineto{\pgfqpoint{4.930203in}{3.448175in}}%
\pgfpathlineto{\pgfqpoint{4.933425in}{3.449575in}}%
\pgfpathlineto{\pgfqpoint{4.934498in}{3.450887in}}%
\pgfpathlineto{\pgfqpoint{4.935572in}{3.450187in}}%
\pgfpathlineto{\pgfqpoint{4.936646in}{3.454036in}}%
\pgfpathlineto{\pgfqpoint{4.937720in}{3.452199in}}%
\pgfpathlineto{\pgfqpoint{4.940941in}{3.454648in}}%
\pgfpathlineto{\pgfqpoint{4.942015in}{3.457797in}}%
\pgfpathlineto{\pgfqpoint{4.943089in}{3.458322in}}%
\pgfpathlineto{\pgfqpoint{4.945237in}{3.452286in}}%
\pgfpathlineto{\pgfqpoint{4.949532in}{3.456748in}}%
\pgfpathlineto{\pgfqpoint{4.951680in}{3.458759in}}%
\pgfpathlineto{\pgfqpoint{4.955975in}{3.458060in}}%
\pgfpathlineto{\pgfqpoint{4.958123in}{3.456573in}}%
\pgfpathlineto{\pgfqpoint{4.959197in}{3.456573in}}%
\pgfpathlineto{\pgfqpoint{4.960271in}{3.452899in}}%
\pgfpathlineto{\pgfqpoint{4.963492in}{3.453424in}}%
\pgfpathlineto{\pgfqpoint{4.965640in}{3.456135in}}%
\pgfpathlineto{\pgfqpoint{4.966714in}{3.454386in}}%
\pgfpathlineto{\pgfqpoint{4.967787in}{3.454036in}}%
\pgfpathlineto{\pgfqpoint{4.972083in}{3.454123in}}%
\pgfpathlineto{\pgfqpoint{4.973157in}{3.455435in}}%
\pgfpathlineto{\pgfqpoint{4.975304in}{3.454911in}}%
\pgfpathlineto{\pgfqpoint{4.979600in}{3.456048in}}%
\pgfpathlineto{\pgfqpoint{4.980673in}{3.455173in}}%
\pgfpathlineto{\pgfqpoint{4.982821in}{3.451324in}}%
\pgfpathlineto{\pgfqpoint{4.988190in}{3.454123in}}%
\pgfpathlineto{\pgfqpoint{4.989264in}{3.453774in}}%
\pgfpathlineto{\pgfqpoint{4.990338in}{3.454823in}}%
\pgfpathlineto{\pgfqpoint{4.996781in}{3.459897in}}%
\pgfpathlineto{\pgfqpoint{4.997855in}{3.462608in}}%
\pgfpathlineto{\pgfqpoint{5.001076in}{3.462171in}}%
\pgfpathlineto{\pgfqpoint{5.002150in}{3.465145in}}%
\pgfpathlineto{\pgfqpoint{5.003224in}{3.464708in}}%
\pgfpathlineto{\pgfqpoint{5.004298in}{3.465145in}}%
\pgfpathlineto{\pgfqpoint{5.005372in}{3.468994in}}%
\pgfpathlineto{\pgfqpoint{5.008593in}{3.467069in}}%
\pgfpathlineto{\pgfqpoint{5.009667in}{3.469519in}}%
\pgfpathlineto{\pgfqpoint{5.010741in}{3.467244in}}%
\pgfpathlineto{\pgfqpoint{5.011815in}{3.467419in}}%
\pgfpathlineto{\pgfqpoint{5.012889in}{3.481852in}}%
\pgfpathlineto{\pgfqpoint{5.016110in}{3.482902in}}%
\pgfpathlineto{\pgfqpoint{5.018258in}{3.482289in}}%
\pgfpathlineto{\pgfqpoint{5.020405in}{3.484389in}}%
\pgfpathlineto{\pgfqpoint{5.023627in}{3.484564in}}%
\pgfpathlineto{\pgfqpoint{5.025775in}{3.489462in}}%
\pgfpathlineto{\pgfqpoint{5.026849in}{3.488762in}}%
\pgfpathlineto{\pgfqpoint{5.027922in}{3.489900in}}%
\pgfpathlineto{\pgfqpoint{5.031144in}{3.489637in}}%
\pgfpathlineto{\pgfqpoint{5.034365in}{3.491474in}}%
\pgfpathlineto{\pgfqpoint{5.038661in}{3.490599in}}%
\pgfpathlineto{\pgfqpoint{5.039735in}{3.488850in}}%
\pgfpathlineto{\pgfqpoint{5.040808in}{3.489550in}}%
\pgfpathlineto{\pgfqpoint{5.041882in}{3.492349in}}%
\pgfpathlineto{\pgfqpoint{5.042956in}{3.492261in}}%
\pgfpathlineto{\pgfqpoint{5.046178in}{3.494098in}}%
\pgfpathlineto{\pgfqpoint{5.047251in}{3.495760in}}%
\pgfpathlineto{\pgfqpoint{5.048325in}{3.517978in}}%
\pgfpathlineto{\pgfqpoint{5.049399in}{3.510805in}}%
\pgfpathlineto{\pgfqpoint{5.050473in}{3.510805in}}%
\pgfpathlineto{\pgfqpoint{5.053694in}{3.513517in}}%
\pgfpathlineto{\pgfqpoint{5.054768in}{3.519028in}}%
\pgfpathlineto{\pgfqpoint{5.056916in}{3.514917in}}%
\pgfpathlineto{\pgfqpoint{5.062285in}{3.514479in}}%
\pgfpathlineto{\pgfqpoint{5.063359in}{3.515704in}}%
\pgfpathlineto{\pgfqpoint{5.064433in}{3.512205in}}%
\pgfpathlineto{\pgfqpoint{5.065507in}{3.511068in}}%
\pgfpathlineto{\pgfqpoint{5.068728in}{3.513430in}}%
\pgfpathlineto{\pgfqpoint{5.069802in}{3.505820in}}%
\pgfpathlineto{\pgfqpoint{5.070876in}{3.505994in}}%
\pgfpathlineto{\pgfqpoint{5.073024in}{3.504158in}}%
\pgfpathlineto{\pgfqpoint{5.077319in}{3.508181in}}%
\pgfpathlineto{\pgfqpoint{5.078393in}{3.515529in}}%
\pgfpathlineto{\pgfqpoint{5.079467in}{3.514304in}}%
\pgfpathlineto{\pgfqpoint{5.080540in}{3.516054in}}%
\pgfpathlineto{\pgfqpoint{5.083762in}{3.517891in}}%
\pgfpathlineto{\pgfqpoint{5.084836in}{3.517366in}}%
\pgfpathlineto{\pgfqpoint{5.085910in}{3.518590in}}%
\pgfpathlineto{\pgfqpoint{5.086983in}{3.523489in}}%
\pgfpathlineto{\pgfqpoint{5.088057in}{3.521914in}}%
\pgfpathlineto{\pgfqpoint{5.094500in}{3.519728in}}%
\pgfpathlineto{\pgfqpoint{5.095574in}{3.521739in}}%
\pgfpathlineto{\pgfqpoint{5.099870in}{3.519115in}}%
\pgfpathlineto{\pgfqpoint{5.100943in}{3.519115in}}%
\pgfpathlineto{\pgfqpoint{5.102017in}{3.521652in}}%
\pgfpathlineto{\pgfqpoint{5.106313in}{3.524888in}}%
\pgfpathlineto{\pgfqpoint{5.107386in}{3.521652in}}%
\pgfpathlineto{\pgfqpoint{5.108460in}{3.522527in}}%
\pgfpathlineto{\pgfqpoint{5.109534in}{3.522527in}}%
\pgfpathlineto{\pgfqpoint{5.110608in}{3.519465in}}%
\pgfpathlineto{\pgfqpoint{5.113829in}{3.518853in}}%
\pgfpathlineto{\pgfqpoint{5.114903in}{3.521827in}}%
\pgfpathlineto{\pgfqpoint{5.115977in}{3.522177in}}%
\pgfpathlineto{\pgfqpoint{5.117051in}{3.523664in}}%
\pgfpathlineto{\pgfqpoint{5.118125in}{3.520952in}}%
\pgfpathlineto{\pgfqpoint{5.121346in}{3.520165in}}%
\pgfpathlineto{\pgfqpoint{5.122420in}{3.517978in}}%
\pgfpathlineto{\pgfqpoint{5.123494in}{3.520427in}}%
\pgfpathlineto{\pgfqpoint{5.124568in}{3.515791in}}%
\pgfpathlineto{\pgfqpoint{5.125642in}{3.516754in}}%
\pgfpathlineto{\pgfqpoint{5.128863in}{3.521565in}}%
\pgfpathlineto{\pgfqpoint{5.129937in}{3.520952in}}%
\pgfpathlineto{\pgfqpoint{5.132085in}{3.511068in}}%
\pgfpathlineto{\pgfqpoint{5.133159in}{3.515004in}}%
\pgfpathlineto{\pgfqpoint{5.136380in}{3.515616in}}%
\pgfpathlineto{\pgfqpoint{5.137454in}{3.510718in}}%
\pgfpathlineto{\pgfqpoint{5.138528in}{3.516841in}}%
\pgfpathlineto{\pgfqpoint{5.139602in}{3.511855in}}%
\pgfpathlineto{\pgfqpoint{5.140675in}{3.498909in}}%
\pgfpathlineto{\pgfqpoint{5.143897in}{3.495498in}}%
\pgfpathlineto{\pgfqpoint{5.144971in}{3.500659in}}%
\pgfpathlineto{\pgfqpoint{5.147118in}{3.490774in}}%
\pgfpathlineto{\pgfqpoint{5.148192in}{3.494885in}}%
\pgfpathlineto{\pgfqpoint{5.151414in}{3.496373in}}%
\pgfpathlineto{\pgfqpoint{5.152488in}{3.504158in}}%
\pgfpathlineto{\pgfqpoint{5.153561in}{3.501621in}}%
\pgfpathlineto{\pgfqpoint{5.155709in}{3.508531in}}%
\pgfpathlineto{\pgfqpoint{5.158931in}{3.508706in}}%
\pgfpathlineto{\pgfqpoint{5.160004in}{3.512817in}}%
\pgfpathlineto{\pgfqpoint{5.161078in}{3.514129in}}%
\pgfpathlineto{\pgfqpoint{5.162152in}{3.503983in}}%
\pgfpathlineto{\pgfqpoint{5.163226in}{3.514829in}}%
\pgfpathlineto{\pgfqpoint{5.166448in}{3.517103in}}%
\pgfpathlineto{\pgfqpoint{5.167521in}{3.518853in}}%
\pgfpathlineto{\pgfqpoint{5.168595in}{3.514654in}}%
\pgfpathlineto{\pgfqpoint{5.169669in}{3.515092in}}%
\pgfpathlineto{\pgfqpoint{5.170743in}{3.513255in}}%
\pgfpathlineto{\pgfqpoint{5.173964in}{3.510805in}}%
\pgfpathlineto{\pgfqpoint{5.176112in}{3.511680in}}%
\pgfpathlineto{\pgfqpoint{5.178260in}{3.516054in}}%
\pgfpathlineto{\pgfqpoint{5.181481in}{3.518328in}}%
\pgfpathlineto{\pgfqpoint{5.182555in}{3.521914in}}%
\pgfpathlineto{\pgfqpoint{5.183629in}{3.519115in}}%
\pgfpathlineto{\pgfqpoint{5.184703in}{3.531361in}}%
\pgfpathlineto{\pgfqpoint{5.185777in}{3.528650in}}%
\pgfpathlineto{\pgfqpoint{5.188998in}{3.533636in}}%
\pgfpathlineto{\pgfqpoint{5.190072in}{3.534161in}}%
\pgfpathlineto{\pgfqpoint{5.191146in}{3.538622in}}%
\pgfpathlineto{\pgfqpoint{5.193293in}{3.541333in}}%
\pgfpathlineto{\pgfqpoint{5.196515in}{3.540721in}}%
\pgfpathlineto{\pgfqpoint{5.197589in}{3.543957in}}%
\pgfpathlineto{\pgfqpoint{5.198663in}{3.542733in}}%
\pgfpathlineto{\pgfqpoint{5.199737in}{3.542908in}}%
\pgfpathlineto{\pgfqpoint{5.200810in}{3.544570in}}%
\pgfpathlineto{\pgfqpoint{5.204032in}{3.540896in}}%
\pgfpathlineto{\pgfqpoint{5.206180in}{3.534948in}}%
\pgfpathlineto{\pgfqpoint{5.207253in}{3.537135in}}%
\pgfpathlineto{\pgfqpoint{5.208327in}{3.533548in}}%
\pgfpathlineto{\pgfqpoint{5.211549in}{3.531099in}}%
\pgfpathlineto{\pgfqpoint{5.212623in}{3.528387in}}%
\pgfpathlineto{\pgfqpoint{5.214770in}{3.539584in}}%
\pgfpathlineto{\pgfqpoint{5.215844in}{3.534598in}}%
\pgfpathlineto{\pgfqpoint{5.220139in}{3.542733in}}%
\pgfpathlineto{\pgfqpoint{5.221213in}{3.542733in}}%
\pgfpathlineto{\pgfqpoint{5.223361in}{3.543608in}}%
\pgfpathlineto{\pgfqpoint{5.226582in}{3.540808in}}%
\pgfpathlineto{\pgfqpoint{5.228730in}{3.533986in}}%
\pgfpathlineto{\pgfqpoint{5.230878in}{3.534510in}}%
\pgfpathlineto{\pgfqpoint{5.234099in}{3.531361in}}%
\pgfpathlineto{\pgfqpoint{5.235173in}{3.526201in}}%
\pgfpathlineto{\pgfqpoint{5.237321in}{3.537047in}}%
\pgfpathlineto{\pgfqpoint{5.238395in}{3.537572in}}%
\pgfpathlineto{\pgfqpoint{5.242690in}{3.535560in}}%
\pgfpathlineto{\pgfqpoint{5.244838in}{3.533198in}}%
\pgfpathlineto{\pgfqpoint{5.245912in}{3.535210in}}%
\pgfpathlineto{\pgfqpoint{5.250207in}{3.532411in}}%
\pgfpathlineto{\pgfqpoint{5.252355in}{3.538709in}}%
\pgfpathlineto{\pgfqpoint{5.253428in}{3.535210in}}%
\pgfpathlineto{\pgfqpoint{5.256650in}{3.530312in}}%
\pgfpathlineto{\pgfqpoint{5.257724in}{3.517891in}}%
\pgfpathlineto{\pgfqpoint{5.258798in}{3.514742in}}%
\pgfpathlineto{\pgfqpoint{5.259871in}{3.518066in}}%
\pgfpathlineto{\pgfqpoint{5.260945in}{3.509143in}}%
\pgfpathlineto{\pgfqpoint{5.264167in}{3.513779in}}%
\pgfpathlineto{\pgfqpoint{5.265241in}{3.514129in}}%
\pgfpathlineto{\pgfqpoint{5.266315in}{3.515267in}}%
\pgfpathlineto{\pgfqpoint{5.267388in}{3.517891in}}%
\pgfpathlineto{\pgfqpoint{5.268462in}{3.512905in}}%
\pgfpathlineto{\pgfqpoint{5.271684in}{3.510106in}}%
\pgfpathlineto{\pgfqpoint{5.272758in}{3.515966in}}%
\pgfpathlineto{\pgfqpoint{5.273831in}{3.514829in}}%
\pgfpathlineto{\pgfqpoint{5.274905in}{3.519290in}}%
\pgfpathlineto{\pgfqpoint{5.275979in}{3.521127in}}%
\pgfpathlineto{\pgfqpoint{5.280274in}{3.524014in}}%
\pgfpathlineto{\pgfqpoint{5.281348in}{3.520427in}}%
\pgfpathlineto{\pgfqpoint{5.282422in}{3.519903in}}%
\pgfpathlineto{\pgfqpoint{5.283496in}{3.521477in}}%
\pgfpathlineto{\pgfqpoint{5.286717in}{3.516491in}}%
\pgfpathlineto{\pgfqpoint{5.287791in}{3.521477in}}%
\pgfpathlineto{\pgfqpoint{5.291013in}{3.512555in}}%
\pgfpathlineto{\pgfqpoint{5.294234in}{3.518765in}}%
\pgfpathlineto{\pgfqpoint{5.296382in}{3.519203in}}%
\pgfpathlineto{\pgfqpoint{5.298530in}{3.512118in}}%
\pgfpathlineto{\pgfqpoint{5.301751in}{3.508444in}}%
\pgfpathlineto{\pgfqpoint{5.302825in}{3.500571in}}%
\pgfpathlineto{\pgfqpoint{5.303899in}{3.505470in}}%
\pgfpathlineto{\pgfqpoint{5.304973in}{3.493661in}}%
\pgfpathlineto{\pgfqpoint{5.306047in}{3.494623in}}%
\pgfpathlineto{\pgfqpoint{5.309268in}{3.493923in}}%
\pgfpathlineto{\pgfqpoint{5.310342in}{3.492086in}}%
\pgfpathlineto{\pgfqpoint{5.311416in}{3.494361in}}%
\pgfpathlineto{\pgfqpoint{5.312490in}{3.493224in}}%
\pgfpathlineto{\pgfqpoint{5.313563in}{3.497597in}}%
\pgfpathlineto{\pgfqpoint{5.316785in}{3.496722in}}%
\pgfpathlineto{\pgfqpoint{5.317859in}{3.493573in}}%
\pgfpathlineto{\pgfqpoint{5.318933in}{3.486663in}}%
\pgfpathlineto{\pgfqpoint{5.320006in}{3.488150in}}%
\pgfpathlineto{\pgfqpoint{5.321080in}{3.502933in}}%
\pgfpathlineto{\pgfqpoint{5.325376in}{3.497247in}}%
\pgfpathlineto{\pgfqpoint{5.326449in}{3.493748in}}%
\pgfpathlineto{\pgfqpoint{5.327523in}{3.493748in}}%
\pgfpathlineto{\pgfqpoint{5.331819in}{3.495498in}}%
\pgfpathlineto{\pgfqpoint{5.332893in}{3.497247in}}%
\pgfpathlineto{\pgfqpoint{5.333966in}{3.497597in}}%
\pgfpathlineto{\pgfqpoint{5.335040in}{3.497072in}}%
\pgfpathlineto{\pgfqpoint{5.336114in}{3.502408in}}%
\pgfpathlineto{\pgfqpoint{5.339336in}{3.500834in}}%
\pgfpathlineto{\pgfqpoint{5.340409in}{3.498997in}}%
\pgfpathlineto{\pgfqpoint{5.341483in}{3.509318in}}%
\pgfpathlineto{\pgfqpoint{5.342557in}{3.509581in}}%
\pgfpathlineto{\pgfqpoint{5.343631in}{3.506519in}}%
\pgfpathlineto{\pgfqpoint{5.346852in}{3.508531in}}%
\pgfpathlineto{\pgfqpoint{5.347926in}{3.506257in}}%
\pgfpathlineto{\pgfqpoint{5.349000in}{3.508269in}}%
\pgfpathlineto{\pgfqpoint{5.351148in}{3.503545in}}%
\pgfpathlineto{\pgfqpoint{5.354369in}{3.506782in}}%
\pgfpathlineto{\pgfqpoint{5.355443in}{3.510805in}}%
\pgfpathlineto{\pgfqpoint{5.356517in}{3.509756in}}%
\pgfpathlineto{\pgfqpoint{5.357591in}{3.507132in}}%
\pgfpathlineto{\pgfqpoint{5.358665in}{3.513867in}}%
\pgfpathlineto{\pgfqpoint{5.361886in}{3.513954in}}%
\pgfpathlineto{\pgfqpoint{5.364034in}{3.506432in}}%
\pgfpathlineto{\pgfqpoint{5.365108in}{3.506519in}}%
\pgfpathlineto{\pgfqpoint{5.366181in}{3.510893in}}%
\pgfpathlineto{\pgfqpoint{5.369403in}{3.510018in}}%
\pgfpathlineto{\pgfqpoint{5.370477in}{3.506607in}}%
\pgfpathlineto{\pgfqpoint{5.372625in}{3.512205in}}%
\pgfpathlineto{\pgfqpoint{5.373698in}{3.512380in}}%
\pgfpathlineto{\pgfqpoint{5.376920in}{3.515616in}}%
\pgfpathlineto{\pgfqpoint{5.377994in}{3.513605in}}%
\pgfpathlineto{\pgfqpoint{5.380141in}{3.516754in}}%
\pgfpathlineto{\pgfqpoint{5.385511in}{3.513255in}}%
\pgfpathlineto{\pgfqpoint{5.387658in}{3.520252in}}%
\pgfpathlineto{\pgfqpoint{5.388732in}{3.523751in}}%
\pgfpathlineto{\pgfqpoint{5.391954in}{3.519553in}}%
\pgfpathlineto{\pgfqpoint{5.395175in}{3.507132in}}%
\pgfpathlineto{\pgfqpoint{5.396249in}{3.503458in}}%
\pgfpathlineto{\pgfqpoint{5.399470in}{3.499259in}}%
\pgfpathlineto{\pgfqpoint{5.400544in}{3.498909in}}%
\pgfpathlineto{\pgfqpoint{5.401618in}{3.503283in}}%
\pgfpathlineto{\pgfqpoint{5.402692in}{3.503545in}}%
\pgfpathlineto{\pgfqpoint{5.403766in}{3.499434in}}%
\pgfpathlineto{\pgfqpoint{5.406987in}{3.499959in}}%
\pgfpathlineto{\pgfqpoint{5.409135in}{3.504332in}}%
\pgfpathlineto{\pgfqpoint{5.410209in}{3.507656in}}%
\pgfpathlineto{\pgfqpoint{5.411283in}{3.505207in}}%
\pgfpathlineto{\pgfqpoint{5.414504in}{3.506694in}}%
\pgfpathlineto{\pgfqpoint{5.416652in}{3.503983in}}%
\pgfpathlineto{\pgfqpoint{5.417726in}{3.504595in}}%
\pgfpathlineto{\pgfqpoint{5.418800in}{3.497072in}}%
\pgfpathlineto{\pgfqpoint{5.422021in}{3.492174in}}%
\pgfpathlineto{\pgfqpoint{5.423095in}{3.492436in}}%
\pgfpathlineto{\pgfqpoint{5.424169in}{3.490599in}}%
\pgfpathlineto{\pgfqpoint{5.425243in}{3.493486in}}%
\pgfpathlineto{\pgfqpoint{5.430612in}{3.488413in}}%
\pgfpathlineto{\pgfqpoint{5.432759in}{3.480890in}}%
\pgfpathlineto{\pgfqpoint{5.433833in}{3.482727in}}%
\pgfpathlineto{\pgfqpoint{5.437055in}{3.487100in}}%
\pgfpathlineto{\pgfqpoint{5.438129in}{3.486488in}}%
\pgfpathlineto{\pgfqpoint{5.439203in}{3.486751in}}%
\pgfpathlineto{\pgfqpoint{5.440276in}{3.488413in}}%
\pgfpathlineto{\pgfqpoint{5.441350in}{3.485089in}}%
\pgfpathlineto{\pgfqpoint{5.444572in}{3.482202in}}%
\pgfpathlineto{\pgfqpoint{5.445646in}{3.479228in}}%
\pgfpathlineto{\pgfqpoint{5.446719in}{3.478353in}}%
\pgfpathlineto{\pgfqpoint{5.447793in}{3.478266in}}%
\pgfpathlineto{\pgfqpoint{5.448867in}{3.474155in}}%
\pgfpathlineto{\pgfqpoint{5.452089in}{3.476341in}}%
\pgfpathlineto{\pgfqpoint{5.453162in}{3.481065in}}%
\pgfpathlineto{\pgfqpoint{5.454236in}{3.481502in}}%
\pgfpathlineto{\pgfqpoint{5.455310in}{3.480715in}}%
\pgfpathlineto{\pgfqpoint{5.460679in}{3.482377in}}%
\pgfpathlineto{\pgfqpoint{5.461753in}{3.484214in}}%
\pgfpathlineto{\pgfqpoint{5.463901in}{3.482377in}}%
\pgfpathlineto{\pgfqpoint{5.467122in}{3.488325in}}%
\pgfpathlineto{\pgfqpoint{5.468196in}{3.483077in}}%
\pgfpathlineto{\pgfqpoint{5.469270in}{3.486838in}}%
\pgfpathlineto{\pgfqpoint{5.470344in}{3.482289in}}%
\pgfpathlineto{\pgfqpoint{5.471418in}{3.483427in}}%
\pgfpathlineto{\pgfqpoint{5.474639in}{3.483864in}}%
\pgfpathlineto{\pgfqpoint{5.475713in}{3.482552in}}%
\pgfpathlineto{\pgfqpoint{5.477861in}{3.471793in}}%
\pgfpathlineto{\pgfqpoint{5.478935in}{3.464183in}}%
\pgfpathlineto{\pgfqpoint{5.482156in}{3.461733in}}%
\pgfpathlineto{\pgfqpoint{5.483230in}{3.458759in}}%
\pgfpathlineto{\pgfqpoint{5.484304in}{3.469956in}}%
\pgfpathlineto{\pgfqpoint{5.485378in}{3.473280in}}%
\pgfpathlineto{\pgfqpoint{5.486451in}{3.478703in}}%
\pgfpathlineto{\pgfqpoint{5.489673in}{3.479665in}}%
\pgfpathlineto{\pgfqpoint{5.490747in}{3.474067in}}%
\pgfpathlineto{\pgfqpoint{5.492894in}{3.483864in}}%
\pgfpathlineto{\pgfqpoint{5.493968in}{3.479490in}}%
\pgfpathlineto{\pgfqpoint{5.498264in}{3.487188in}}%
\pgfpathlineto{\pgfqpoint{5.499337in}{3.485176in}}%
\pgfpathlineto{\pgfqpoint{5.500411in}{3.485351in}}%
\pgfpathlineto{\pgfqpoint{5.501485in}{3.486926in}}%
\pgfpathlineto{\pgfqpoint{5.504707in}{3.486313in}}%
\pgfpathlineto{\pgfqpoint{5.505781in}{3.489025in}}%
\pgfpathlineto{\pgfqpoint{5.506854in}{3.489287in}}%
\pgfpathlineto{\pgfqpoint{5.507928in}{3.488850in}}%
\pgfpathlineto{\pgfqpoint{5.509002in}{3.483427in}}%
\pgfpathlineto{\pgfqpoint{5.512224in}{3.484564in}}%
\pgfpathlineto{\pgfqpoint{5.513297in}{3.480715in}}%
\pgfpathlineto{\pgfqpoint{5.514371in}{3.481240in}}%
\pgfpathlineto{\pgfqpoint{5.515445in}{3.479228in}}%
\pgfpathlineto{\pgfqpoint{5.516519in}{3.481765in}}%
\pgfpathlineto{\pgfqpoint{5.519740in}{3.481415in}}%
\pgfpathlineto{\pgfqpoint{5.520814in}{3.485176in}}%
\pgfpathlineto{\pgfqpoint{5.521888in}{3.492174in}}%
\pgfpathlineto{\pgfqpoint{5.522962in}{3.491124in}}%
\pgfpathlineto{\pgfqpoint{5.524036in}{3.495060in}}%
\pgfpathlineto{\pgfqpoint{5.527257in}{3.500571in}}%
\pgfpathlineto{\pgfqpoint{5.530479in}{3.510805in}}%
\pgfpathlineto{\pgfqpoint{5.531553in}{3.507831in}}%
\pgfpathlineto{\pgfqpoint{5.534774in}{3.508356in}}%
\pgfpathlineto{\pgfqpoint{5.535848in}{3.507044in}}%
\pgfpathlineto{\pgfqpoint{5.536922in}{3.512992in}}%
\pgfpathlineto{\pgfqpoint{5.537996in}{3.512642in}}%
\pgfpathlineto{\pgfqpoint{5.539069in}{3.514917in}}%
\pgfpathlineto{\pgfqpoint{5.542291in}{3.519203in}}%
\pgfpathlineto{\pgfqpoint{5.544439in}{3.517453in}}%
\pgfpathlineto{\pgfqpoint{5.546586in}{3.529437in}}%
\pgfpathlineto{\pgfqpoint{5.550882in}{3.526113in}}%
\pgfpathlineto{\pgfqpoint{5.551956in}{3.527950in}}%
\pgfpathlineto{\pgfqpoint{5.553029in}{3.522614in}}%
\pgfpathlineto{\pgfqpoint{5.554103in}{3.521302in}}%
\pgfpathlineto{\pgfqpoint{5.557325in}{3.523226in}}%
\pgfpathlineto{\pgfqpoint{5.559472in}{3.525501in}}%
\pgfpathlineto{\pgfqpoint{5.565915in}{3.518066in}}%
\pgfpathlineto{\pgfqpoint{5.569137in}{3.509406in}}%
\pgfpathlineto{\pgfqpoint{5.572358in}{3.509318in}}%
\pgfpathlineto{\pgfqpoint{5.574506in}{3.517628in}}%
\pgfpathlineto{\pgfqpoint{5.575580in}{3.526638in}}%
\pgfpathlineto{\pgfqpoint{5.576654in}{3.529437in}}%
\pgfpathlineto{\pgfqpoint{5.580949in}{3.527075in}}%
\pgfpathlineto{\pgfqpoint{5.582023in}{3.527863in}}%
\pgfpathlineto{\pgfqpoint{5.584171in}{3.527863in}}%
\pgfpathlineto{\pgfqpoint{5.587392in}{3.530312in}}%
\pgfpathlineto{\pgfqpoint{5.588466in}{3.532848in}}%
\pgfpathlineto{\pgfqpoint{5.589540in}{3.530837in}}%
\pgfpathlineto{\pgfqpoint{5.590614in}{3.524626in}}%
\pgfpathlineto{\pgfqpoint{5.591688in}{3.531711in}}%
\pgfpathlineto{\pgfqpoint{5.594909in}{3.532061in}}%
\pgfpathlineto{\pgfqpoint{5.595983in}{3.530224in}}%
\pgfpathlineto{\pgfqpoint{5.597057in}{3.530662in}}%
\pgfpathlineto{\pgfqpoint{5.598131in}{3.530312in}}%
\pgfpathlineto{\pgfqpoint{5.599204in}{3.526375in}}%
\pgfpathlineto{\pgfqpoint{5.602426in}{3.527950in}}%
\pgfpathlineto{\pgfqpoint{5.603500in}{3.533548in}}%
\pgfpathlineto{\pgfqpoint{5.604574in}{3.534510in}}%
\pgfpathlineto{\pgfqpoint{5.605647in}{3.531449in}}%
\pgfpathlineto{\pgfqpoint{5.606721in}{3.523226in}}%
\pgfpathlineto{\pgfqpoint{5.609943in}{3.526201in}}%
\pgfpathlineto{\pgfqpoint{5.612091in}{3.532149in}}%
\pgfpathlineto{\pgfqpoint{5.617460in}{3.531624in}}%
\pgfpathlineto{\pgfqpoint{5.618534in}{3.535648in}}%
\pgfpathlineto{\pgfqpoint{5.620681in}{3.527863in}}%
\pgfpathlineto{\pgfqpoint{5.626050in}{3.522964in}}%
\pgfpathlineto{\pgfqpoint{5.627124in}{3.517016in}}%
\pgfpathlineto{\pgfqpoint{5.628198in}{3.507219in}}%
\pgfpathlineto{\pgfqpoint{5.629272in}{3.504682in}}%
\pgfpathlineto{\pgfqpoint{5.632493in}{3.508969in}}%
\pgfpathlineto{\pgfqpoint{5.633567in}{3.513867in}}%
\pgfpathlineto{\pgfqpoint{5.634641in}{3.507831in}}%
\pgfpathlineto{\pgfqpoint{5.635715in}{3.514304in}}%
\pgfpathlineto{\pgfqpoint{5.636789in}{3.490862in}}%
\pgfpathlineto{\pgfqpoint{5.641084in}{3.491212in}}%
\pgfpathlineto{\pgfqpoint{5.642158in}{3.489550in}}%
\pgfpathlineto{\pgfqpoint{5.643232in}{3.490075in}}%
\pgfpathlineto{\pgfqpoint{5.644306in}{3.492174in}}%
\pgfpathlineto{\pgfqpoint{5.647527in}{3.489637in}}%
\pgfpathlineto{\pgfqpoint{5.648601in}{3.492261in}}%
\pgfpathlineto{\pgfqpoint{5.649675in}{3.491299in}}%
\pgfpathlineto{\pgfqpoint{5.650749in}{3.492524in}}%
\pgfpathlineto{\pgfqpoint{5.651823in}{3.500834in}}%
\pgfpathlineto{\pgfqpoint{5.655044in}{3.499259in}}%
\pgfpathlineto{\pgfqpoint{5.656118in}{3.491212in}}%
\pgfpathlineto{\pgfqpoint{5.657192in}{3.489550in}}%
\pgfpathlineto{\pgfqpoint{5.658266in}{3.493049in}}%
\pgfpathlineto{\pgfqpoint{5.659339in}{3.487188in}}%
\pgfpathlineto{\pgfqpoint{5.663635in}{3.485351in}}%
\pgfpathlineto{\pgfqpoint{5.664709in}{3.480802in}}%
\pgfpathlineto{\pgfqpoint{5.665782in}{3.480715in}}%
\pgfpathlineto{\pgfqpoint{5.666856in}{3.484039in}}%
\pgfpathlineto{\pgfqpoint{5.671152in}{3.485176in}}%
\pgfpathlineto{\pgfqpoint{5.672225in}{3.490599in}}%
\pgfpathlineto{\pgfqpoint{5.673299in}{3.490249in}}%
\pgfpathlineto{\pgfqpoint{5.674373in}{3.484564in}}%
\pgfpathlineto{\pgfqpoint{5.677595in}{3.489637in}}%
\pgfpathlineto{\pgfqpoint{5.678669in}{3.485264in}}%
\pgfpathlineto{\pgfqpoint{5.681890in}{3.493224in}}%
\pgfpathlineto{\pgfqpoint{5.685112in}{3.491562in}}%
\pgfpathlineto{\pgfqpoint{5.686185in}{3.497772in}}%
\pgfpathlineto{\pgfqpoint{5.687259in}{3.499172in}}%
\pgfpathlineto{\pgfqpoint{5.689407in}{3.499871in}}%
\pgfpathlineto{\pgfqpoint{5.692628in}{3.502321in}}%
\pgfpathlineto{\pgfqpoint{5.693702in}{3.499259in}}%
\pgfpathlineto{\pgfqpoint{5.695850in}{3.504770in}}%
\pgfpathlineto{\pgfqpoint{5.696924in}{3.508794in}}%
\pgfpathlineto{\pgfqpoint{5.700145in}{3.506169in}}%
\pgfpathlineto{\pgfqpoint{5.701219in}{3.507919in}}%
\pgfpathlineto{\pgfqpoint{5.702293in}{3.508269in}}%
\pgfpathlineto{\pgfqpoint{5.703367in}{3.510543in}}%
\pgfpathlineto{\pgfqpoint{5.704441in}{3.516141in}}%
\pgfpathlineto{\pgfqpoint{5.707662in}{3.513430in}}%
\pgfpathlineto{\pgfqpoint{5.708736in}{3.513255in}}%
\pgfpathlineto{\pgfqpoint{5.709810in}{3.510718in}}%
\pgfpathlineto{\pgfqpoint{5.710884in}{3.509756in}}%
\pgfpathlineto{\pgfqpoint{5.715179in}{3.509931in}}%
\pgfpathlineto{\pgfqpoint{5.717327in}{3.516404in}}%
\pgfpathlineto{\pgfqpoint{5.718401in}{3.513517in}}%
\pgfpathlineto{\pgfqpoint{5.719474in}{3.514304in}}%
\pgfpathlineto{\pgfqpoint{5.723770in}{3.509931in}}%
\pgfpathlineto{\pgfqpoint{5.724844in}{3.511330in}}%
\pgfpathlineto{\pgfqpoint{5.725917in}{3.507132in}}%
\pgfpathlineto{\pgfqpoint{5.726991in}{3.507831in}}%
\pgfpathlineto{\pgfqpoint{5.730213in}{3.508094in}}%
\pgfpathlineto{\pgfqpoint{5.732360in}{3.511768in}}%
\pgfpathlineto{\pgfqpoint{5.734508in}{3.506432in}}%
\pgfpathlineto{\pgfqpoint{5.737730in}{3.507919in}}%
\pgfpathlineto{\pgfqpoint{5.738803in}{3.507569in}}%
\pgfpathlineto{\pgfqpoint{5.739877in}{3.510718in}}%
\pgfpathlineto{\pgfqpoint{5.740951in}{3.510456in}}%
\pgfpathlineto{\pgfqpoint{5.742025in}{3.507831in}}%
\pgfpathlineto{\pgfqpoint{5.745246in}{3.505907in}}%
\pgfpathlineto{\pgfqpoint{5.746320in}{3.505994in}}%
\pgfpathlineto{\pgfqpoint{5.747394in}{3.508706in}}%
\pgfpathlineto{\pgfqpoint{5.749542in}{3.497072in}}%
\pgfpathlineto{\pgfqpoint{5.752763in}{3.499696in}}%
\pgfpathlineto{\pgfqpoint{5.754911in}{3.495673in}}%
\pgfpathlineto{\pgfqpoint{5.755985in}{3.496110in}}%
\pgfpathlineto{\pgfqpoint{5.757059in}{3.497247in}}%
\pgfpathlineto{\pgfqpoint{5.760280in}{3.495323in}}%
\pgfpathlineto{\pgfqpoint{5.761354in}{3.498034in}}%
\pgfpathlineto{\pgfqpoint{5.762428in}{3.497335in}}%
\pgfpathlineto{\pgfqpoint{5.763502in}{3.494973in}}%
\pgfpathlineto{\pgfqpoint{5.767797in}{3.500046in}}%
\pgfpathlineto{\pgfqpoint{5.768871in}{3.496722in}}%
\pgfpathlineto{\pgfqpoint{5.769945in}{3.496810in}}%
\pgfpathlineto{\pgfqpoint{5.771019in}{3.493923in}}%
\pgfpathlineto{\pgfqpoint{5.772092in}{3.498122in}}%
\pgfpathlineto{\pgfqpoint{5.775314in}{3.498734in}}%
\pgfpathlineto{\pgfqpoint{5.776388in}{3.505382in}}%
\pgfpathlineto{\pgfqpoint{5.777462in}{3.508006in}}%
\pgfpathlineto{\pgfqpoint{5.779609in}{3.509493in}}%
\pgfpathlineto{\pgfqpoint{5.784979in}{3.510193in}}%
\pgfpathlineto{\pgfqpoint{5.786052in}{3.510980in}}%
\pgfpathlineto{\pgfqpoint{5.787126in}{3.509843in}}%
\pgfpathlineto{\pgfqpoint{5.790348in}{3.510368in}}%
\pgfpathlineto{\pgfqpoint{5.791422in}{3.511943in}}%
\pgfpathlineto{\pgfqpoint{5.793569in}{3.512467in}}%
\pgfpathlineto{\pgfqpoint{5.794643in}{3.513255in}}%
\pgfpathlineto{\pgfqpoint{5.797865in}{3.514304in}}%
\pgfpathlineto{\pgfqpoint{5.798938in}{3.514042in}}%
\pgfpathlineto{\pgfqpoint{5.800012in}{3.509843in}}%
\pgfpathlineto{\pgfqpoint{5.802160in}{3.510980in}}%
\pgfpathlineto{\pgfqpoint{5.806455in}{3.515529in}}%
\pgfpathlineto{\pgfqpoint{5.807529in}{3.515267in}}%
\pgfpathlineto{\pgfqpoint{5.808603in}{3.520865in}}%
\pgfpathlineto{\pgfqpoint{5.809677in}{3.509318in}}%
\pgfpathlineto{\pgfqpoint{5.812898in}{3.502671in}}%
\pgfpathlineto{\pgfqpoint{5.813972in}{3.506432in}}%
\pgfpathlineto{\pgfqpoint{5.816120in}{3.519378in}}%
\pgfpathlineto{\pgfqpoint{5.817194in}{3.518940in}}%
\pgfpathlineto{\pgfqpoint{5.821489in}{3.518416in}}%
\pgfpathlineto{\pgfqpoint{5.823637in}{3.522527in}}%
\pgfpathlineto{\pgfqpoint{5.824711in}{3.529000in}}%
\pgfpathlineto{\pgfqpoint{5.827932in}{3.531974in}}%
\pgfpathlineto{\pgfqpoint{5.829006in}{3.536522in}}%
\pgfpathlineto{\pgfqpoint{5.830080in}{3.537047in}}%
\pgfpathlineto{\pgfqpoint{5.831154in}{3.538622in}}%
\pgfpathlineto{\pgfqpoint{5.832227in}{3.537572in}}%
\pgfpathlineto{\pgfqpoint{5.835449in}{3.537397in}}%
\pgfpathlineto{\pgfqpoint{5.836523in}{3.538184in}}%
\pgfpathlineto{\pgfqpoint{5.837597in}{3.542558in}}%
\pgfpathlineto{\pgfqpoint{5.838670in}{3.531099in}}%
\pgfpathlineto{\pgfqpoint{5.839744in}{3.534248in}}%
\pgfpathlineto{\pgfqpoint{5.842966in}{3.534510in}}%
\pgfpathlineto{\pgfqpoint{5.844040in}{3.537747in}}%
\pgfpathlineto{\pgfqpoint{5.845113in}{3.535648in}}%
\pgfpathlineto{\pgfqpoint{5.846187in}{3.535123in}}%
\pgfpathlineto{\pgfqpoint{5.847261in}{3.535822in}}%
\pgfpathlineto{\pgfqpoint{5.850483in}{3.535822in}}%
\pgfpathlineto{\pgfqpoint{5.851557in}{3.533461in}}%
\pgfpathlineto{\pgfqpoint{5.852630in}{3.533023in}}%
\pgfpathlineto{\pgfqpoint{5.854778in}{3.538971in}}%
\pgfpathlineto{\pgfqpoint{5.858000in}{3.539409in}}%
\pgfpathlineto{\pgfqpoint{5.859073in}{3.538447in}}%
\pgfpathlineto{\pgfqpoint{5.860147in}{3.535298in}}%
\pgfpathlineto{\pgfqpoint{5.861221in}{3.536522in}}%
\pgfpathlineto{\pgfqpoint{5.862295in}{3.535648in}}%
\pgfpathlineto{\pgfqpoint{5.865516in}{3.538359in}}%
\pgfpathlineto{\pgfqpoint{5.866590in}{3.540808in}}%
\pgfpathlineto{\pgfqpoint{5.867664in}{3.539234in}}%
\pgfpathlineto{\pgfqpoint{5.868738in}{3.538884in}}%
\pgfpathlineto{\pgfqpoint{5.869812in}{3.541071in}}%
\pgfpathlineto{\pgfqpoint{5.874107in}{3.542383in}}%
\pgfpathlineto{\pgfqpoint{5.875181in}{3.540371in}}%
\pgfpathlineto{\pgfqpoint{5.876255in}{3.539846in}}%
\pgfpathlineto{\pgfqpoint{5.877329in}{3.541246in}}%
\pgfpathlineto{\pgfqpoint{5.881624in}{3.544745in}}%
\pgfpathlineto{\pgfqpoint{5.883772in}{3.547369in}}%
\pgfpathlineto{\pgfqpoint{5.884846in}{3.547806in}}%
\pgfpathlineto{\pgfqpoint{5.889141in}{3.551830in}}%
\pgfpathlineto{\pgfqpoint{5.890215in}{3.550868in}}%
\pgfpathlineto{\pgfqpoint{5.891289in}{3.550780in}}%
\pgfpathlineto{\pgfqpoint{5.892362in}{3.542645in}}%
\pgfpathlineto{\pgfqpoint{5.895584in}{3.547806in}}%
\pgfpathlineto{\pgfqpoint{5.896658in}{3.544045in}}%
\pgfpathlineto{\pgfqpoint{5.897732in}{3.544132in}}%
\pgfpathlineto{\pgfqpoint{5.899879in}{3.560665in}}%
\pgfpathlineto{\pgfqpoint{5.903101in}{3.556553in}}%
\pgfpathlineto{\pgfqpoint{5.904175in}{3.556378in}}%
\pgfpathlineto{\pgfqpoint{5.905248in}{3.558915in}}%
\pgfpathlineto{\pgfqpoint{5.906322in}{3.559702in}}%
\pgfpathlineto{\pgfqpoint{5.907396in}{3.556816in}}%
\pgfpathlineto{\pgfqpoint{5.910618in}{3.552442in}}%
\pgfpathlineto{\pgfqpoint{5.912765in}{3.558828in}}%
\pgfpathlineto{\pgfqpoint{5.913839in}{3.557866in}}%
\pgfpathlineto{\pgfqpoint{5.914913in}{3.561364in}}%
\pgfpathlineto{\pgfqpoint{5.918134in}{3.560577in}}%
\pgfpathlineto{\pgfqpoint{5.919208in}{3.559615in}}%
\pgfpathlineto{\pgfqpoint{5.920282in}{3.563289in}}%
\pgfpathlineto{\pgfqpoint{5.922430in}{3.564164in}}%
\pgfpathlineto{\pgfqpoint{5.925651in}{3.563551in}}%
\pgfpathlineto{\pgfqpoint{5.926725in}{3.557428in}}%
\pgfpathlineto{\pgfqpoint{5.928873in}{3.555066in}}%
\pgfpathlineto{\pgfqpoint{5.929947in}{3.558915in}}%
\pgfpathlineto{\pgfqpoint{5.933168in}{3.557603in}}%
\pgfpathlineto{\pgfqpoint{5.934242in}{3.561364in}}%
\pgfpathlineto{\pgfqpoint{5.935316in}{3.543258in}}%
\pgfpathlineto{\pgfqpoint{5.936390in}{3.542558in}}%
\pgfpathlineto{\pgfqpoint{5.937464in}{3.540371in}}%
\pgfpathlineto{\pgfqpoint{5.940685in}{3.541246in}}%
\pgfpathlineto{\pgfqpoint{5.943907in}{3.537572in}}%
\pgfpathlineto{\pgfqpoint{5.944980in}{3.537047in}}%
\pgfpathlineto{\pgfqpoint{5.948202in}{3.538097in}}%
\pgfpathlineto{\pgfqpoint{5.949276in}{3.535210in}}%
\pgfpathlineto{\pgfqpoint{5.950350in}{3.535910in}}%
\pgfpathlineto{\pgfqpoint{5.952497in}{3.529962in}}%
\pgfpathlineto{\pgfqpoint{5.955719in}{3.538709in}}%
\pgfpathlineto{\pgfqpoint{5.957867in}{3.539234in}}%
\pgfpathlineto{\pgfqpoint{5.958940in}{3.537222in}}%
\pgfpathlineto{\pgfqpoint{5.960014in}{3.538097in}}%
\pgfpathlineto{\pgfqpoint{5.963236in}{3.537047in}}%
\pgfpathlineto{\pgfqpoint{5.964310in}{3.540546in}}%
\pgfpathlineto{\pgfqpoint{5.965383in}{3.539934in}}%
\pgfpathlineto{\pgfqpoint{5.966457in}{3.541421in}}%
\pgfpathlineto{\pgfqpoint{5.967531in}{3.540808in}}%
\pgfpathlineto{\pgfqpoint{5.970753in}{3.541071in}}%
\pgfpathlineto{\pgfqpoint{5.971826in}{3.545182in}}%
\pgfpathlineto{\pgfqpoint{5.972900in}{3.542908in}}%
\pgfpathlineto{\pgfqpoint{5.975048in}{3.544832in}}%
\pgfpathlineto{\pgfqpoint{5.978269in}{3.545444in}}%
\pgfpathlineto{\pgfqpoint{5.979343in}{3.543782in}}%
\pgfpathlineto{\pgfqpoint{5.981491in}{3.531186in}}%
\pgfpathlineto{\pgfqpoint{5.982565in}{3.534423in}}%
\pgfpathlineto{\pgfqpoint{5.985786in}{3.536260in}}%
\pgfpathlineto{\pgfqpoint{5.986860in}{3.538971in}}%
\pgfpathlineto{\pgfqpoint{5.987934in}{3.545357in}}%
\pgfpathlineto{\pgfqpoint{5.989008in}{3.546931in}}%
\pgfpathlineto{\pgfqpoint{5.993303in}{3.549118in}}%
\pgfpathlineto{\pgfqpoint{5.994377in}{3.555941in}}%
\pgfpathlineto{\pgfqpoint{5.995451in}{3.553842in}}%
\pgfpathlineto{\pgfqpoint{5.996525in}{3.555854in}}%
\pgfpathlineto{\pgfqpoint{5.997599in}{3.551917in}}%
\pgfpathlineto{\pgfqpoint{6.000820in}{3.556641in}}%
\pgfpathlineto{\pgfqpoint{6.001894in}{3.559265in}}%
\pgfpathlineto{\pgfqpoint{6.002968in}{3.557341in}}%
\pgfpathlineto{\pgfqpoint{6.004042in}{3.556991in}}%
\pgfpathlineto{\pgfqpoint{6.009411in}{3.558128in}}%
\pgfpathlineto{\pgfqpoint{6.010485in}{3.554542in}}%
\pgfpathlineto{\pgfqpoint{6.011558in}{3.554804in}}%
\pgfpathlineto{\pgfqpoint{6.012632in}{3.551568in}}%
\pgfpathlineto{\pgfqpoint{6.016928in}{3.554279in}}%
\pgfpathlineto{\pgfqpoint{6.018001in}{3.552705in}}%
\pgfpathlineto{\pgfqpoint{6.019075in}{3.552267in}}%
\pgfpathlineto{\pgfqpoint{6.020149in}{3.553317in}}%
\pgfpathlineto{\pgfqpoint{6.023371in}{3.554367in}}%
\pgfpathlineto{\pgfqpoint{6.024445in}{3.553754in}}%
\pgfpathlineto{\pgfqpoint{6.025518in}{3.557166in}}%
\pgfpathlineto{\pgfqpoint{6.026592in}{3.555154in}}%
\pgfpathlineto{\pgfqpoint{6.027666in}{3.555854in}}%
\pgfpathlineto{\pgfqpoint{6.033035in}{3.555591in}}%
\pgfpathlineto{\pgfqpoint{6.034109in}{3.554017in}}%
\pgfpathlineto{\pgfqpoint{6.035183in}{3.557078in}}%
\pgfpathlineto{\pgfqpoint{6.038404in}{3.555679in}}%
\pgfpathlineto{\pgfqpoint{6.039478in}{3.562589in}}%
\pgfpathlineto{\pgfqpoint{6.040552in}{3.564076in}}%
\pgfpathlineto{\pgfqpoint{6.041626in}{3.562064in}}%
\pgfpathlineto{\pgfqpoint{6.042700in}{3.565476in}}%
\pgfpathlineto{\pgfqpoint{6.045921in}{3.560927in}}%
\pgfpathlineto{\pgfqpoint{6.048069in}{3.553667in}}%
\pgfpathlineto{\pgfqpoint{6.050217in}{3.555766in}}%
\pgfpathlineto{\pgfqpoint{6.053438in}{3.553667in}}%
\pgfpathlineto{\pgfqpoint{6.055586in}{3.554629in}}%
\pgfpathlineto{\pgfqpoint{6.056660in}{3.547019in}}%
\pgfpathlineto{\pgfqpoint{6.057734in}{3.546057in}}%
\pgfpathlineto{\pgfqpoint{6.063103in}{3.551917in}}%
\pgfpathlineto{\pgfqpoint{6.064177in}{3.554891in}}%
\pgfpathlineto{\pgfqpoint{6.065250in}{3.555416in}}%
\pgfpathlineto{\pgfqpoint{6.069546in}{3.555766in}}%
\pgfpathlineto{\pgfqpoint{6.070620in}{3.552092in}}%
\pgfpathlineto{\pgfqpoint{6.071693in}{3.552967in}}%
\pgfpathlineto{\pgfqpoint{6.072767in}{3.555854in}}%
\pgfpathlineto{\pgfqpoint{6.075989in}{3.555679in}}%
\pgfpathlineto{\pgfqpoint{6.078136in}{3.550955in}}%
\pgfpathlineto{\pgfqpoint{6.080284in}{3.550693in}}%
\pgfpathlineto{\pgfqpoint{6.083506in}{3.547981in}}%
\pgfpathlineto{\pgfqpoint{6.084579in}{3.549818in}}%
\pgfpathlineto{\pgfqpoint{6.085653in}{3.548331in}}%
\pgfpathlineto{\pgfqpoint{6.087801in}{3.550780in}}%
\pgfpathlineto{\pgfqpoint{6.091022in}{3.544570in}}%
\pgfpathlineto{\pgfqpoint{6.092096in}{3.544745in}}%
\pgfpathlineto{\pgfqpoint{6.093170in}{3.544132in}}%
\pgfpathlineto{\pgfqpoint{6.094244in}{3.544395in}}%
\pgfpathlineto{\pgfqpoint{6.095318in}{3.545532in}}%
\pgfpathlineto{\pgfqpoint{6.098539in}{3.546844in}}%
\pgfpathlineto{\pgfqpoint{6.099613in}{3.543608in}}%
\pgfpathlineto{\pgfqpoint{6.100687in}{3.546319in}}%
\pgfpathlineto{\pgfqpoint{6.102835in}{3.544570in}}%
\pgfpathlineto{\pgfqpoint{6.106056in}{3.546494in}}%
\pgfpathlineto{\pgfqpoint{6.107130in}{3.548244in}}%
\pgfpathlineto{\pgfqpoint{6.108204in}{3.547981in}}%
\pgfpathlineto{\pgfqpoint{6.109278in}{3.549468in}}%
\pgfpathlineto{\pgfqpoint{6.110352in}{3.552092in}}%
\pgfpathlineto{\pgfqpoint{6.113573in}{3.552792in}}%
\pgfpathlineto{\pgfqpoint{6.114647in}{3.553754in}}%
\pgfpathlineto{\pgfqpoint{6.115721in}{3.553317in}}%
\pgfpathlineto{\pgfqpoint{6.116795in}{3.551742in}}%
\pgfpathlineto{\pgfqpoint{6.117868in}{3.551742in}}%
\pgfpathlineto{\pgfqpoint{6.123238in}{3.548419in}}%
\pgfpathlineto{\pgfqpoint{6.124311in}{3.545357in}}%
\pgfpathlineto{\pgfqpoint{6.128607in}{3.547194in}}%
\pgfpathlineto{\pgfqpoint{6.130755in}{3.550780in}}%
\pgfpathlineto{\pgfqpoint{6.132902in}{3.554104in}}%
\pgfpathlineto{\pgfqpoint{6.137198in}{3.558653in}}%
\pgfpathlineto{\pgfqpoint{6.138271in}{3.559090in}}%
\pgfpathlineto{\pgfqpoint{6.139345in}{3.563201in}}%
\pgfpathlineto{\pgfqpoint{6.140419in}{3.552705in}}%
\pgfpathlineto{\pgfqpoint{6.143641in}{3.554017in}}%
\pgfpathlineto{\pgfqpoint{6.144714in}{3.559440in}}%
\pgfpathlineto{\pgfqpoint{6.145788in}{3.561802in}}%
\pgfpathlineto{\pgfqpoint{6.146862in}{3.560665in}}%
\pgfpathlineto{\pgfqpoint{6.147936in}{3.560490in}}%
\pgfpathlineto{\pgfqpoint{6.152231in}{3.556728in}}%
\pgfpathlineto{\pgfqpoint{6.154379in}{3.551130in}}%
\pgfpathlineto{\pgfqpoint{6.155453in}{3.549818in}}%
\pgfpathlineto{\pgfqpoint{6.158674in}{3.550605in}}%
\pgfpathlineto{\pgfqpoint{6.159748in}{3.552180in}}%
\pgfpathlineto{\pgfqpoint{6.160822in}{3.545794in}}%
\pgfpathlineto{\pgfqpoint{6.162970in}{3.548768in}}%
\pgfpathlineto{\pgfqpoint{6.167265in}{3.552530in}}%
\pgfpathlineto{\pgfqpoint{6.169413in}{3.555854in}}%
\pgfpathlineto{\pgfqpoint{6.170487in}{3.555854in}}%
\pgfpathlineto{\pgfqpoint{6.176930in}{3.554717in}}%
\pgfpathlineto{\pgfqpoint{6.178003in}{3.556291in}}%
\pgfpathlineto{\pgfqpoint{6.181225in}{3.556466in}}%
\pgfpathlineto{\pgfqpoint{6.182299in}{3.554717in}}%
\pgfpathlineto{\pgfqpoint{6.184446in}{3.557691in}}%
\pgfpathlineto{\pgfqpoint{6.185520in}{3.551305in}}%
\pgfpathlineto{\pgfqpoint{6.188742in}{3.551480in}}%
\pgfpathlineto{\pgfqpoint{6.189816in}{3.552705in}}%
\pgfpathlineto{\pgfqpoint{6.191963in}{3.547981in}}%
\pgfpathlineto{\pgfqpoint{6.193037in}{3.547194in}}%
\pgfpathlineto{\pgfqpoint{6.196259in}{3.549643in}}%
\pgfpathlineto{\pgfqpoint{6.197333in}{3.544307in}}%
\pgfpathlineto{\pgfqpoint{6.199480in}{3.540109in}}%
\pgfpathlineto{\pgfqpoint{6.200554in}{3.538709in}}%
\pgfpathlineto{\pgfqpoint{6.203776in}{3.537747in}}%
\pgfpathlineto{\pgfqpoint{6.204849in}{3.534248in}}%
\pgfpathlineto{\pgfqpoint{6.205923in}{3.538797in}}%
\pgfpathlineto{\pgfqpoint{6.206997in}{3.533373in}}%
\pgfpathlineto{\pgfqpoint{6.208071in}{3.535035in}}%
\pgfpathlineto{\pgfqpoint{6.211292in}{3.532673in}}%
\pgfpathlineto{\pgfqpoint{6.213440in}{3.539934in}}%
\pgfpathlineto{\pgfqpoint{6.214514in}{3.534073in}}%
\pgfpathlineto{\pgfqpoint{6.215588in}{3.536172in}}%
\pgfpathlineto{\pgfqpoint{6.218809in}{3.534248in}}%
\pgfpathlineto{\pgfqpoint{6.220957in}{3.539234in}}%
\pgfpathlineto{\pgfqpoint{6.222031in}{3.539146in}}%
\pgfpathlineto{\pgfqpoint{6.223105in}{3.542820in}}%
\pgfpathlineto{\pgfqpoint{6.226326in}{3.541071in}}%
\pgfpathlineto{\pgfqpoint{6.228474in}{3.541771in}}%
\pgfpathlineto{\pgfqpoint{6.229548in}{3.543345in}}%
\pgfpathlineto{\pgfqpoint{6.230622in}{3.543170in}}%
\pgfpathlineto{\pgfqpoint{6.233843in}{3.541333in}}%
\pgfpathlineto{\pgfqpoint{6.237065in}{3.545182in}}%
\pgfpathlineto{\pgfqpoint{6.238138in}{3.547981in}}%
\pgfpathlineto{\pgfqpoint{6.241360in}{3.549293in}}%
\pgfpathlineto{\pgfqpoint{6.242434in}{3.556553in}}%
\pgfpathlineto{\pgfqpoint{6.243508in}{3.559003in}}%
\pgfpathlineto{\pgfqpoint{6.244581in}{3.559965in}}%
\pgfpathlineto{\pgfqpoint{6.245655in}{3.558390in}}%
\pgfpathlineto{\pgfqpoint{6.251024in}{3.560840in}}%
\pgfpathlineto{\pgfqpoint{6.253172in}{3.554804in}}%
\pgfpathlineto{\pgfqpoint{6.256394in}{3.558740in}}%
\pgfpathlineto{\pgfqpoint{6.260689in}{3.547719in}}%
\pgfpathlineto{\pgfqpoint{6.263910in}{3.546931in}}%
\pgfpathlineto{\pgfqpoint{6.264984in}{3.544657in}}%
\pgfpathlineto{\pgfqpoint{6.272501in}{3.545357in}}%
\pgfpathlineto{\pgfqpoint{6.274649in}{3.548156in}}%
\pgfpathlineto{\pgfqpoint{6.275723in}{3.548331in}}%
\pgfpathlineto{\pgfqpoint{6.280018in}{3.547719in}}%
\pgfpathlineto{\pgfqpoint{6.281092in}{3.553929in}}%
\pgfpathlineto{\pgfqpoint{6.283240in}{3.549206in}}%
\pgfpathlineto{\pgfqpoint{6.286461in}{3.554017in}}%
\pgfpathlineto{\pgfqpoint{6.288609in}{3.558653in}}%
\pgfpathlineto{\pgfqpoint{6.289683in}{3.559877in}}%
\pgfpathlineto{\pgfqpoint{6.290756in}{3.564251in}}%
\pgfpathlineto{\pgfqpoint{6.293978in}{3.564251in}}%
\pgfpathlineto{\pgfqpoint{6.295052in}{3.566175in}}%
\pgfpathlineto{\pgfqpoint{6.296126in}{3.564863in}}%
\pgfpathlineto{\pgfqpoint{6.297199in}{3.565913in}}%
\pgfpathlineto{\pgfqpoint{6.301495in}{3.565563in}}%
\pgfpathlineto{\pgfqpoint{6.302569in}{3.568187in}}%
\pgfpathlineto{\pgfqpoint{6.303643in}{3.568712in}}%
\pgfpathlineto{\pgfqpoint{6.310086in}{3.584020in}}%
\pgfpathlineto{\pgfqpoint{6.311159in}{3.583670in}}%
\pgfpathlineto{\pgfqpoint{6.313307in}{3.586119in}}%
\pgfpathlineto{\pgfqpoint{6.316529in}{3.588043in}}%
\pgfpathlineto{\pgfqpoint{6.317602in}{3.586294in}}%
\pgfpathlineto{\pgfqpoint{6.318676in}{3.583407in}}%
\pgfpathlineto{\pgfqpoint{6.319750in}{3.582445in}}%
\pgfpathlineto{\pgfqpoint{6.320824in}{3.586469in}}%
\pgfpathlineto{\pgfqpoint{6.325119in}{3.587431in}}%
\pgfpathlineto{\pgfqpoint{6.326193in}{3.591280in}}%
\pgfpathlineto{\pgfqpoint{6.327267in}{3.589968in}}%
\pgfpathlineto{\pgfqpoint{6.328341in}{3.592767in}}%
\pgfpathlineto{\pgfqpoint{6.332636in}{3.597141in}}%
\pgfpathlineto{\pgfqpoint{6.333710in}{3.595654in}}%
\pgfpathlineto{\pgfqpoint{6.334784in}{3.600464in}}%
\pgfpathlineto{\pgfqpoint{6.335858in}{3.625831in}}%
\pgfpathlineto{\pgfqpoint{6.339079in}{3.625569in}}%
\pgfpathlineto{\pgfqpoint{6.341227in}{3.645075in}}%
\pgfpathlineto{\pgfqpoint{6.342301in}{3.648312in}}%
\pgfpathlineto{\pgfqpoint{6.343375in}{3.641926in}}%
\pgfpathlineto{\pgfqpoint{6.346596in}{3.647262in}}%
\pgfpathlineto{\pgfqpoint{6.347670in}{3.647962in}}%
\pgfpathlineto{\pgfqpoint{6.348744in}{3.647262in}}%
\pgfpathlineto{\pgfqpoint{6.349818in}{3.643938in}}%
\pgfpathlineto{\pgfqpoint{6.350891in}{3.637903in}}%
\pgfpathlineto{\pgfqpoint{6.355187in}{3.640264in}}%
\pgfpathlineto{\pgfqpoint{6.356261in}{3.636853in}}%
\pgfpathlineto{\pgfqpoint{6.357334in}{3.638427in}}%
\pgfpathlineto{\pgfqpoint{6.358408in}{3.629943in}}%
\pgfpathlineto{\pgfqpoint{6.361630in}{3.629855in}}%
\pgfpathlineto{\pgfqpoint{6.362704in}{3.632567in}}%
\pgfpathlineto{\pgfqpoint{6.363777in}{3.630118in}}%
\pgfpathlineto{\pgfqpoint{6.365925in}{3.630905in}}%
\pgfpathlineto{\pgfqpoint{6.369147in}{3.628718in}}%
\pgfpathlineto{\pgfqpoint{6.370221in}{3.630730in}}%
\pgfpathlineto{\pgfqpoint{6.371294in}{3.624257in}}%
\pgfpathlineto{\pgfqpoint{6.372368in}{3.631692in}}%
\pgfpathlineto{\pgfqpoint{6.373442in}{3.630380in}}%
\pgfpathlineto{\pgfqpoint{6.376664in}{3.628718in}}%
\pgfpathlineto{\pgfqpoint{6.377737in}{3.619971in}}%
\pgfpathlineto{\pgfqpoint{6.378811in}{3.620058in}}%
\pgfpathlineto{\pgfqpoint{6.379885in}{3.616997in}}%
\pgfpathlineto{\pgfqpoint{6.380959in}{3.619184in}}%
\pgfpathlineto{\pgfqpoint{6.384180in}{3.621808in}}%
\pgfpathlineto{\pgfqpoint{6.385254in}{3.619009in}}%
\pgfpathlineto{\pgfqpoint{6.386328in}{3.619096in}}%
\pgfpathlineto{\pgfqpoint{6.387402in}{3.618484in}}%
\pgfpathlineto{\pgfqpoint{6.388476in}{3.629330in}}%
\pgfpathlineto{\pgfqpoint{6.393845in}{3.654435in}}%
\pgfpathlineto{\pgfqpoint{6.394919in}{3.647787in}}%
\pgfpathlineto{\pgfqpoint{6.395993in}{3.647262in}}%
\pgfpathlineto{\pgfqpoint{6.400288in}{3.642101in}}%
\pgfpathlineto{\pgfqpoint{6.401362in}{3.642276in}}%
\pgfpathlineto{\pgfqpoint{6.402436in}{3.643238in}}%
\pgfpathlineto{\pgfqpoint{6.403510in}{3.642714in}}%
\pgfpathlineto{\pgfqpoint{6.403510in}{3.642714in}}%
\pgfusepath{stroke}%
\end{pgfscope}%
\begin{pgfscope}%
\pgfpathrectangle{\pgfqpoint{3.937600in}{3.271772in}}{\pgfqpoint{2.583333in}{0.400885in}}%
\pgfusepath{clip}%
\pgfsetroundcap%
\pgfsetroundjoin%
\pgfsetlinewidth{1.505625pt}%
\definecolor{currentstroke}{rgb}{0.839216,0.152941,0.156863}%
\pgfsetstrokecolor{currentstroke}%
\pgfsetdash{}{0pt}%
\pgfpathmoveto{\pgfqpoint{4.055025in}{3.425433in}}%
\pgfpathlineto{\pgfqpoint{4.056098in}{3.425433in}}%
\pgfpathlineto{\pgfqpoint{4.057172in}{3.423466in}}%
\pgfpathlineto{\pgfqpoint{4.061468in}{3.423466in}}%
\pgfpathlineto{\pgfqpoint{4.064689in}{3.421639in}}%
\pgfpathlineto{\pgfqpoint{4.072206in}{3.421639in}}%
\pgfpathlineto{\pgfqpoint{4.073280in}{3.416809in}}%
\pgfpathlineto{\pgfqpoint{4.078649in}{3.413671in}}%
\pgfpathlineto{\pgfqpoint{4.079723in}{3.414562in}}%
\pgfpathlineto{\pgfqpoint{4.087240in}{3.414562in}}%
\pgfpathlineto{\pgfqpoint{4.088314in}{3.411833in}}%
\pgfpathlineto{\pgfqpoint{4.092609in}{3.412419in}}%
\pgfpathlineto{\pgfqpoint{4.093683in}{3.411165in}}%
\pgfpathlineto{\pgfqpoint{4.101200in}{3.410581in}}%
\pgfpathlineto{\pgfqpoint{4.102274in}{3.409120in}}%
\pgfpathlineto{\pgfqpoint{4.103347in}{3.406109in}}%
\pgfpathlineto{\pgfqpoint{4.109790in}{3.407212in}}%
\pgfpathlineto{\pgfqpoint{4.114086in}{3.407212in}}%
\pgfpathlineto{\pgfqpoint{4.115160in}{3.405233in}}%
\pgfpathlineto{\pgfqpoint{4.124824in}{3.405233in}}%
\pgfpathlineto{\pgfqpoint{4.125898in}{3.403906in}}%
\pgfpathlineto{\pgfqpoint{4.129120in}{3.403906in}}%
\pgfpathlineto{\pgfqpoint{4.130193in}{3.401152in}}%
\pgfpathlineto{\pgfqpoint{4.131267in}{3.401284in}}%
\pgfpathlineto{\pgfqpoint{4.132341in}{3.399756in}}%
\pgfpathlineto{\pgfqpoint{4.136636in}{3.399821in}}%
\pgfpathlineto{\pgfqpoint{4.138784in}{3.399625in}}%
\pgfpathlineto{\pgfqpoint{4.139858in}{3.398976in}}%
\pgfpathlineto{\pgfqpoint{4.140932in}{3.399105in}}%
\pgfpathlineto{\pgfqpoint{4.145227in}{3.397493in}}%
\pgfpathlineto{\pgfqpoint{4.146301in}{3.399447in}}%
\pgfpathlineto{\pgfqpoint{4.147375in}{3.397632in}}%
\pgfpathlineto{\pgfqpoint{4.148449in}{3.397822in}}%
\pgfpathlineto{\pgfqpoint{4.151670in}{3.396491in}}%
\pgfpathlineto{\pgfqpoint{4.153818in}{3.398747in}}%
\pgfpathlineto{\pgfqpoint{4.154892in}{3.398041in}}%
\pgfpathlineto{\pgfqpoint{4.160261in}{3.399630in}}%
\pgfpathlineto{\pgfqpoint{4.162409in}{3.394335in}}%
\pgfpathlineto{\pgfqpoint{4.163482in}{3.396226in}}%
\pgfpathlineto{\pgfqpoint{4.167778in}{3.394353in}}%
\pgfpathlineto{\pgfqpoint{4.168852in}{3.396916in}}%
\pgfpathlineto{\pgfqpoint{4.176368in}{3.396916in}}%
\pgfpathlineto{\pgfqpoint{4.178516in}{3.394268in}}%
\pgfpathlineto{\pgfqpoint{4.181738in}{3.394207in}}%
\pgfpathlineto{\pgfqpoint{4.182811in}{3.391517in}}%
\pgfpathlineto{\pgfqpoint{4.183885in}{3.390459in}}%
\pgfpathlineto{\pgfqpoint{4.184959in}{3.392369in}}%
\pgfpathlineto{\pgfqpoint{4.205362in}{3.392369in}}%
\pgfpathlineto{\pgfqpoint{4.206436in}{3.389299in}}%
\pgfpathlineto{\pgfqpoint{4.208584in}{3.390866in}}%
\pgfpathlineto{\pgfqpoint{4.215027in}{3.390866in}}%
\pgfpathlineto{\pgfqpoint{4.216100in}{3.387222in}}%
\pgfpathlineto{\pgfqpoint{4.219322in}{3.386701in}}%
\pgfpathlineto{\pgfqpoint{4.220396in}{3.388719in}}%
\pgfpathlineto{\pgfqpoint{4.256906in}{3.388719in}}%
\pgfpathlineto{\pgfqpoint{4.257980in}{3.385620in}}%
\pgfpathlineto{\pgfqpoint{4.259054in}{3.384798in}}%
\pgfpathlineto{\pgfqpoint{4.260128in}{3.381509in}}%
\pgfpathlineto{\pgfqpoint{4.261202in}{3.384102in}}%
\pgfpathlineto{\pgfqpoint{4.264423in}{3.383470in}}%
\pgfpathlineto{\pgfqpoint{4.265497in}{3.384734in}}%
\pgfpathlineto{\pgfqpoint{4.267645in}{3.384734in}}%
\pgfpathlineto{\pgfqpoint{4.268719in}{3.382085in}}%
\pgfpathlineto{\pgfqpoint{4.271940in}{3.380792in}}%
\pgfpathlineto{\pgfqpoint{4.273014in}{3.379560in}}%
\pgfpathlineto{\pgfqpoint{4.274088in}{3.380176in}}%
\pgfpathlineto{\pgfqpoint{4.275162in}{3.382024in}}%
\pgfpathlineto{\pgfqpoint{4.304155in}{3.382024in}}%
\pgfpathlineto{\pgfqpoint{4.305229in}{3.378634in}}%
\pgfpathlineto{\pgfqpoint{4.306303in}{3.378028in}}%
\pgfpathlineto{\pgfqpoint{4.309524in}{3.377665in}}%
\pgfpathlineto{\pgfqpoint{4.310598in}{3.378452in}}%
\pgfpathlineto{\pgfqpoint{4.312746in}{3.374880in}}%
\pgfpathlineto{\pgfqpoint{4.313820in}{3.377605in}}%
\pgfpathlineto{\pgfqpoint{4.319189in}{3.375486in}}%
\pgfpathlineto{\pgfqpoint{4.320263in}{3.378936in}}%
\pgfpathlineto{\pgfqpoint{4.321337in}{3.374517in}}%
\pgfpathlineto{\pgfqpoint{4.324558in}{3.369977in}}%
\pgfpathlineto{\pgfqpoint{4.325632in}{3.370340in}}%
\pgfpathlineto{\pgfqpoint{4.326706in}{3.369614in}}%
\pgfpathlineto{\pgfqpoint{4.327780in}{3.370461in}}%
\pgfpathlineto{\pgfqpoint{4.333149in}{3.370522in}}%
\pgfpathlineto{\pgfqpoint{4.334223in}{3.369432in}}%
\pgfpathlineto{\pgfqpoint{4.336370in}{3.369311in}}%
\pgfpathlineto{\pgfqpoint{4.339592in}{3.367737in}}%
\pgfpathlineto{\pgfqpoint{4.340666in}{3.366466in}}%
\pgfpathlineto{\pgfqpoint{4.341740in}{3.367011in}}%
\pgfpathlineto{\pgfqpoint{4.342813in}{3.369129in}}%
\pgfpathlineto{\pgfqpoint{4.343887in}{3.367011in}}%
\pgfpathlineto{\pgfqpoint{4.348183in}{3.367919in}}%
\pgfpathlineto{\pgfqpoint{4.349256in}{3.366466in}}%
\pgfpathlineto{\pgfqpoint{4.350330in}{3.366103in}}%
\pgfpathlineto{\pgfqpoint{4.351404in}{3.367132in}}%
\pgfpathlineto{\pgfqpoint{4.354626in}{3.366284in}}%
\pgfpathlineto{\pgfqpoint{4.355699in}{3.363318in}}%
\pgfpathlineto{\pgfqpoint{4.358921in}{3.361260in}}%
\pgfpathlineto{\pgfqpoint{4.362142in}{3.362470in}}%
\pgfpathlineto{\pgfqpoint{4.363216in}{3.365497in}}%
\pgfpathlineto{\pgfqpoint{4.364290in}{3.362773in}}%
\pgfpathlineto{\pgfqpoint{4.365364in}{3.362168in}}%
\pgfpathlineto{\pgfqpoint{4.366438in}{3.360230in}}%
\pgfpathlineto{\pgfqpoint{4.370733in}{3.361804in}}%
\pgfpathlineto{\pgfqpoint{4.371807in}{3.361199in}}%
\pgfpathlineto{\pgfqpoint{4.373955in}{3.363560in}}%
\pgfpathlineto{\pgfqpoint{4.379324in}{3.361986in}}%
\pgfpathlineto{\pgfqpoint{4.380398in}{3.365073in}}%
\pgfpathlineto{\pgfqpoint{4.381472in}{3.364105in}}%
\pgfpathlineto{\pgfqpoint{4.385767in}{3.363560in}}%
\pgfpathlineto{\pgfqpoint{4.386841in}{3.359565in}}%
\pgfpathlineto{\pgfqpoint{4.388988in}{3.359020in}}%
\pgfpathlineto{\pgfqpoint{4.392210in}{3.358838in}}%
\pgfpathlineto{\pgfqpoint{4.394358in}{3.354843in}}%
\pgfpathlineto{\pgfqpoint{4.399727in}{3.356296in}}%
\pgfpathlineto{\pgfqpoint{4.400801in}{3.352603in}}%
\pgfpathlineto{\pgfqpoint{4.401875in}{3.351876in}}%
\pgfpathlineto{\pgfqpoint{4.404022in}{3.353692in}}%
\pgfpathlineto{\pgfqpoint{4.409391in}{3.355509in}}%
\pgfpathlineto{\pgfqpoint{4.410465in}{3.352724in}}%
\pgfpathlineto{\pgfqpoint{4.411539in}{3.352905in}}%
\pgfpathlineto{\pgfqpoint{4.414761in}{3.352784in}}%
\pgfpathlineto{\pgfqpoint{4.415834in}{3.354903in}}%
\pgfpathlineto{\pgfqpoint{4.416908in}{3.354298in}}%
\pgfpathlineto{\pgfqpoint{4.417982in}{3.355811in}}%
\pgfpathlineto{\pgfqpoint{4.422277in}{3.355448in}}%
\pgfpathlineto{\pgfqpoint{4.423351in}{3.358233in}}%
\pgfpathlineto{\pgfqpoint{4.424425in}{3.358354in}}%
\pgfpathlineto{\pgfqpoint{4.425499in}{3.357446in}}%
\pgfpathlineto{\pgfqpoint{4.429794in}{3.357870in}}%
\pgfpathlineto{\pgfqpoint{4.430868in}{3.359807in}}%
\pgfpathlineto{\pgfqpoint{4.431942in}{3.360473in}}%
\pgfpathlineto{\pgfqpoint{4.433016in}{3.360109in}}%
\pgfpathlineto{\pgfqpoint{4.434090in}{3.358838in}}%
\pgfpathlineto{\pgfqpoint{4.440533in}{3.357567in}}%
\pgfpathlineto{\pgfqpoint{4.441607in}{3.356174in}}%
\pgfpathlineto{\pgfqpoint{4.444828in}{3.358112in}}%
\pgfpathlineto{\pgfqpoint{4.446976in}{3.361865in}}%
\pgfpathlineto{\pgfqpoint{4.449123in}{3.360775in}}%
\pgfpathlineto{\pgfqpoint{4.452345in}{3.361199in}}%
\pgfpathlineto{\pgfqpoint{4.453419in}{3.360412in}}%
\pgfpathlineto{\pgfqpoint{4.456640in}{3.364892in}}%
\pgfpathlineto{\pgfqpoint{4.463083in}{3.364347in}}%
\pgfpathlineto{\pgfqpoint{4.464157in}{3.357513in}}%
\pgfpathlineto{\pgfqpoint{4.469526in}{3.356871in}}%
\pgfpathlineto{\pgfqpoint{4.470600in}{3.356111in}}%
\pgfpathlineto{\pgfqpoint{4.474896in}{3.356579in}}%
\pgfpathlineto{\pgfqpoint{4.477043in}{3.358097in}}%
\pgfpathlineto{\pgfqpoint{4.478117in}{3.356520in}}%
\pgfpathlineto{\pgfqpoint{4.479191in}{3.358039in}}%
\pgfpathlineto{\pgfqpoint{4.482412in}{3.357104in}}%
\pgfpathlineto{\pgfqpoint{4.483486in}{3.358272in}}%
\pgfpathlineto{\pgfqpoint{4.485634in}{3.356520in}}%
\pgfpathlineto{\pgfqpoint{4.486708in}{3.357396in}}%
\pgfpathlineto{\pgfqpoint{4.489929in}{3.357572in}}%
\pgfpathlineto{\pgfqpoint{4.492077in}{3.358623in}}%
\pgfpathlineto{\pgfqpoint{4.498520in}{3.357864in}}%
\pgfpathlineto{\pgfqpoint{4.500668in}{3.353775in}}%
\pgfpathlineto{\pgfqpoint{4.501741in}{3.354593in}}%
\pgfpathlineto{\pgfqpoint{4.504963in}{3.353717in}}%
\pgfpathlineto{\pgfqpoint{4.507111in}{3.357046in}}%
\pgfpathlineto{\pgfqpoint{4.508185in}{3.356812in}}%
\pgfpathlineto{\pgfqpoint{4.509258in}{3.357572in}}%
\pgfpathlineto{\pgfqpoint{4.513554in}{3.358740in}}%
\pgfpathlineto{\pgfqpoint{4.524292in}{3.358740in}}%
\pgfpathlineto{\pgfqpoint{4.538252in}{3.356310in}}%
\pgfpathlineto{\pgfqpoint{4.543621in}{3.356310in}}%
\pgfpathlineto{\pgfqpoint{4.544695in}{3.354387in}}%
\pgfpathlineto{\pgfqpoint{4.545769in}{3.354840in}}%
\pgfpathlineto{\pgfqpoint{4.546843in}{3.353879in}}%
\pgfpathlineto{\pgfqpoint{4.552212in}{3.354614in}}%
\pgfpathlineto{\pgfqpoint{4.792752in}{3.354614in}}%
\pgfpathlineto{\pgfqpoint{4.793826in}{3.351923in}}%
\pgfpathlineto{\pgfqpoint{4.821745in}{3.351923in}}%
\pgfpathlineto{\pgfqpoint{4.822819in}{3.350003in}}%
\pgfpathlineto{\pgfqpoint{4.824967in}{3.350047in}}%
\pgfpathlineto{\pgfqpoint{4.836779in}{3.346735in}}%
\pgfpathlineto{\pgfqpoint{4.843222in}{3.348517in}}%
\pgfpathlineto{\pgfqpoint{4.844296in}{3.348048in}}%
\pgfpathlineto{\pgfqpoint{4.846444in}{3.349025in}}%
\pgfpathlineto{\pgfqpoint{4.847517in}{3.348248in}}%
\pgfpathlineto{\pgfqpoint{4.850739in}{3.348333in}}%
\pgfpathlineto{\pgfqpoint{4.851813in}{3.344716in}}%
\pgfpathlineto{\pgfqpoint{4.852887in}{3.344166in}}%
\pgfpathlineto{\pgfqpoint{4.853961in}{3.344593in}}%
\pgfpathlineto{\pgfqpoint{4.855034in}{3.346907in}}%
\pgfpathlineto{\pgfqpoint{4.859330in}{3.347816in}}%
\pgfpathlineto{\pgfqpoint{4.860404in}{3.348786in}}%
\pgfpathlineto{\pgfqpoint{4.920539in}{3.348786in}}%
\pgfpathlineto{\pgfqpoint{4.921612in}{3.347275in}}%
\pgfpathlineto{\pgfqpoint{4.922686in}{3.348194in}}%
\pgfpathlineto{\pgfqpoint{4.925908in}{3.348365in}}%
\pgfpathlineto{\pgfqpoint{4.926982in}{3.347124in}}%
\pgfpathlineto{\pgfqpoint{4.929129in}{3.347667in}}%
\pgfpathlineto{\pgfqpoint{4.930203in}{3.346529in}}%
\pgfpathlineto{\pgfqpoint{4.935572in}{3.345583in}}%
\pgfpathlineto{\pgfqpoint{4.936646in}{3.343811in}}%
\pgfpathlineto{\pgfqpoint{4.937720in}{3.344623in}}%
\pgfpathlineto{\pgfqpoint{4.940941in}{3.343520in}}%
\pgfpathlineto{\pgfqpoint{4.942015in}{3.342135in}}%
\pgfpathlineto{\pgfqpoint{4.943089in}{3.341912in}}%
\pgfpathlineto{\pgfqpoint{4.945237in}{3.344503in}}%
\pgfpathlineto{\pgfqpoint{4.951680in}{3.341655in}}%
\pgfpathlineto{\pgfqpoint{4.959197in}{3.342583in}}%
\pgfpathlineto{\pgfqpoint{4.960271in}{3.344166in}}%
\pgfpathlineto{\pgfqpoint{4.964566in}{3.343426in}}%
\pgfpathlineto{\pgfqpoint{4.965640in}{3.342734in}}%
\pgfpathlineto{\pgfqpoint{4.967787in}{3.343645in}}%
\pgfpathlineto{\pgfqpoint{4.972083in}{3.343606in}}%
\pgfpathlineto{\pgfqpoint{4.973157in}{3.343027in}}%
\pgfpathlineto{\pgfqpoint{4.975304in}{3.343256in}}%
\pgfpathlineto{\pgfqpoint{4.980673in}{3.343138in}}%
\pgfpathlineto{\pgfqpoint{4.982821in}{3.344833in}}%
\pgfpathlineto{\pgfqpoint{4.996781in}{3.341069in}}%
\pgfpathlineto{\pgfqpoint{4.997855in}{3.339941in}}%
\pgfpathlineto{\pgfqpoint{5.001076in}{3.340118in}}%
\pgfpathlineto{\pgfqpoint{5.002150in}{3.338908in}}%
\pgfpathlineto{\pgfqpoint{5.004298in}{3.338907in}}%
\pgfpathlineto{\pgfqpoint{5.005372in}{3.337385in}}%
\pgfpathlineto{\pgfqpoint{5.008593in}{3.338118in}}%
\pgfpathlineto{\pgfqpoint{5.009667in}{3.337168in}}%
\pgfpathlineto{\pgfqpoint{5.010741in}{3.338030in}}%
\pgfpathlineto{\pgfqpoint{5.011815in}{3.337963in}}%
\pgfpathlineto{\pgfqpoint{5.012889in}{3.332382in}}%
\pgfpathlineto{\pgfqpoint{5.024701in}{3.330782in}}%
\pgfpathlineto{\pgfqpoint{5.025775in}{3.329877in}}%
\pgfpathlineto{\pgfqpoint{5.026849in}{3.330098in}}%
\pgfpathlineto{\pgfqpoint{5.027922in}{3.329737in}}%
\pgfpathlineto{\pgfqpoint{5.033292in}{3.329627in}}%
\pgfpathlineto{\pgfqpoint{5.034365in}{3.329243in}}%
\pgfpathlineto{\pgfqpoint{5.040808in}{3.329839in}}%
\pgfpathlineto{\pgfqpoint{5.041882in}{3.328957in}}%
\pgfpathlineto{\pgfqpoint{5.042956in}{3.328984in}}%
\pgfpathlineto{\pgfqpoint{5.047251in}{3.327914in}}%
\pgfpathlineto{\pgfqpoint{5.048325in}{3.321273in}}%
\pgfpathlineto{\pgfqpoint{5.049399in}{3.323053in}}%
\pgfpathlineto{\pgfqpoint{5.050473in}{3.323053in}}%
\pgfpathlineto{\pgfqpoint{5.053694in}{3.322342in}}%
\pgfpathlineto{\pgfqpoint{5.054768in}{3.320928in}}%
\pgfpathlineto{\pgfqpoint{5.057990in}{3.321947in}}%
\pgfpathlineto{\pgfqpoint{5.063359in}{3.321746in}}%
\pgfpathlineto{\pgfqpoint{5.065507in}{3.322923in}}%
\pgfpathlineto{\pgfqpoint{5.068728in}{3.322305in}}%
\pgfpathlineto{\pgfqpoint{5.069802in}{3.324258in}}%
\pgfpathlineto{\pgfqpoint{5.071950in}{3.324567in}}%
\pgfpathlineto{\pgfqpoint{5.073024in}{3.324711in}}%
\pgfpathlineto{\pgfqpoint{5.077319in}{3.323607in}}%
\pgfpathlineto{\pgfqpoint{5.078393in}{3.321645in}}%
\pgfpathlineto{\pgfqpoint{5.079467in}{3.321953in}}%
\pgfpathlineto{\pgfqpoint{5.080540in}{3.321508in}}%
\pgfpathlineto{\pgfqpoint{5.085910in}{3.320873in}}%
\pgfpathlineto{\pgfqpoint{5.086983in}{3.319668in}}%
\pgfpathlineto{\pgfqpoint{5.088057in}{3.320041in}}%
\pgfpathlineto{\pgfqpoint{5.100943in}{3.320708in}}%
\pgfpathlineto{\pgfqpoint{5.103091in}{3.319918in}}%
\pgfpathlineto{\pgfqpoint{5.106313in}{3.319312in}}%
\pgfpathlineto{\pgfqpoint{5.107386in}{3.320071in}}%
\pgfpathlineto{\pgfqpoint{5.109534in}{3.319861in}}%
\pgfpathlineto{\pgfqpoint{5.110608in}{3.320591in}}%
\pgfpathlineto{\pgfqpoint{5.113829in}{3.320741in}}%
\pgfpathlineto{\pgfqpoint{5.115977in}{3.319927in}}%
\pgfpathlineto{\pgfqpoint{5.117051in}{3.319572in}}%
\pgfpathlineto{\pgfqpoint{5.118125in}{3.320213in}}%
\pgfpathlineto{\pgfqpoint{5.123494in}{3.320329in}}%
\pgfpathlineto{\pgfqpoint{5.124568in}{3.321453in}}%
\pgfpathlineto{\pgfqpoint{5.129937in}{3.320160in}}%
\pgfpathlineto{\pgfqpoint{5.132085in}{3.322586in}}%
\pgfpathlineto{\pgfqpoint{5.133159in}{3.321562in}}%
\pgfpathlineto{\pgfqpoint{5.136380in}{3.321408in}}%
\pgfpathlineto{\pgfqpoint{5.137454in}{3.322637in}}%
\pgfpathlineto{\pgfqpoint{5.138528in}{3.321041in}}%
\pgfpathlineto{\pgfqpoint{5.139602in}{3.322279in}}%
\pgfpathlineto{\pgfqpoint{5.140675in}{3.325620in}}%
\pgfpathlineto{\pgfqpoint{5.143897in}{3.326594in}}%
\pgfpathlineto{\pgfqpoint{5.144971in}{3.325078in}}%
\pgfpathlineto{\pgfqpoint{5.147118in}{3.327911in}}%
\pgfpathlineto{\pgfqpoint{5.148192in}{3.326656in}}%
\pgfpathlineto{\pgfqpoint{5.151414in}{3.326217in}}%
\pgfpathlineto{\pgfqpoint{5.152488in}{3.323951in}}%
\pgfpathlineto{\pgfqpoint{5.153561in}{3.324643in}}%
\pgfpathlineto{\pgfqpoint{5.155709in}{3.322746in}}%
\pgfpathlineto{\pgfqpoint{5.158931in}{3.322700in}}%
\pgfpathlineto{\pgfqpoint{5.161078in}{3.321285in}}%
\pgfpathlineto{\pgfqpoint{5.162152in}{3.323841in}}%
\pgfpathlineto{\pgfqpoint{5.163226in}{3.320884in}}%
\pgfpathlineto{\pgfqpoint{5.167521in}{3.319887in}}%
\pgfpathlineto{\pgfqpoint{5.168595in}{3.320904in}}%
\pgfpathlineto{\pgfqpoint{5.169669in}{3.320794in}}%
\pgfpathlineto{\pgfqpoint{5.170743in}{3.321252in}}%
\pgfpathlineto{\pgfqpoint{5.175038in}{3.321781in}}%
\pgfpathlineto{\pgfqpoint{5.177186in}{3.321042in}}%
\pgfpathlineto{\pgfqpoint{5.178260in}{3.320537in}}%
\pgfpathlineto{\pgfqpoint{5.181481in}{3.319975in}}%
\pgfpathlineto{\pgfqpoint{5.182555in}{3.319104in}}%
\pgfpathlineto{\pgfqpoint{5.183629in}{3.319765in}}%
\pgfpathlineto{\pgfqpoint{5.184703in}{3.316810in}}%
\pgfpathlineto{\pgfqpoint{5.185777in}{3.317406in}}%
\pgfpathlineto{\pgfqpoint{5.193293in}{3.314648in}}%
\pgfpathlineto{\pgfqpoint{5.196515in}{3.314773in}}%
\pgfpathlineto{\pgfqpoint{5.197589in}{3.314109in}}%
\pgfpathlineto{\pgfqpoint{5.199737in}{3.314319in}}%
\pgfpathlineto{\pgfqpoint{5.200810in}{3.313983in}}%
\pgfpathlineto{\pgfqpoint{5.205106in}{3.315202in}}%
\pgfpathlineto{\pgfqpoint{5.206180in}{3.315950in}}%
\pgfpathlineto{\pgfqpoint{5.207253in}{3.315482in}}%
\pgfpathlineto{\pgfqpoint{5.208327in}{3.316237in}}%
\pgfpathlineto{\pgfqpoint{5.212623in}{3.317363in}}%
\pgfpathlineto{\pgfqpoint{5.214770in}{3.314902in}}%
\pgfpathlineto{\pgfqpoint{5.215844in}{3.315933in}}%
\pgfpathlineto{\pgfqpoint{5.221213in}{3.314208in}}%
\pgfpathlineto{\pgfqpoint{5.223361in}{3.314031in}}%
\pgfpathlineto{\pgfqpoint{5.227656in}{3.315256in}}%
\pgfpathlineto{\pgfqpoint{5.228730in}{3.316007in}}%
\pgfpathlineto{\pgfqpoint{5.230878in}{3.315894in}}%
\pgfpathlineto{\pgfqpoint{5.234099in}{3.316568in}}%
\pgfpathlineto{\pgfqpoint{5.235173in}{3.317698in}}%
\pgfpathlineto{\pgfqpoint{5.237321in}{3.315281in}}%
\pgfpathlineto{\pgfqpoint{5.238395in}{3.315171in}}%
\pgfpathlineto{\pgfqpoint{5.250207in}{3.316257in}}%
\pgfpathlineto{\pgfqpoint{5.252355in}{3.314906in}}%
\pgfpathlineto{\pgfqpoint{5.253428in}{3.315631in}}%
\pgfpathlineto{\pgfqpoint{5.256650in}{3.316673in}}%
\pgfpathlineto{\pgfqpoint{5.257724in}{3.319408in}}%
\pgfpathlineto{\pgfqpoint{5.258798in}{3.320167in}}%
\pgfpathlineto{\pgfqpoint{5.259871in}{3.319346in}}%
\pgfpathlineto{\pgfqpoint{5.260945in}{3.321494in}}%
\pgfpathlineto{\pgfqpoint{5.267388in}{3.319286in}}%
\pgfpathlineto{\pgfqpoint{5.268462in}{3.320486in}}%
\pgfpathlineto{\pgfqpoint{5.271684in}{3.321186in}}%
\pgfpathlineto{\pgfqpoint{5.272758in}{3.319688in}}%
\pgfpathlineto{\pgfqpoint{5.273831in}{3.319966in}}%
\pgfpathlineto{\pgfqpoint{5.275979in}{3.318431in}}%
\pgfpathlineto{\pgfqpoint{5.280274in}{3.317754in}}%
\pgfpathlineto{\pgfqpoint{5.282422in}{3.318700in}}%
\pgfpathlineto{\pgfqpoint{5.283496in}{3.318328in}}%
\pgfpathlineto{\pgfqpoint{5.286717in}{3.319494in}}%
\pgfpathlineto{\pgfqpoint{5.287791in}{3.318283in}}%
\pgfpathlineto{\pgfqpoint{5.291013in}{3.320415in}}%
\pgfpathlineto{\pgfqpoint{5.295308in}{3.318778in}}%
\pgfpathlineto{\pgfqpoint{5.296382in}{3.318757in}}%
\pgfpathlineto{\pgfqpoint{5.298530in}{3.320462in}}%
\pgfpathlineto{\pgfqpoint{5.301751in}{3.321383in}}%
\pgfpathlineto{\pgfqpoint{5.302825in}{3.323414in}}%
\pgfpathlineto{\pgfqpoint{5.303899in}{3.322069in}}%
\pgfpathlineto{\pgfqpoint{5.304973in}{3.325184in}}%
\pgfpathlineto{\pgfqpoint{5.306047in}{3.324905in}}%
\pgfpathlineto{\pgfqpoint{5.312490in}{3.325297in}}%
\pgfpathlineto{\pgfqpoint{5.313563in}{3.324024in}}%
\pgfpathlineto{\pgfqpoint{5.316785in}{3.324270in}}%
\pgfpathlineto{\pgfqpoint{5.317859in}{3.325159in}}%
\pgfpathlineto{\pgfqpoint{5.320006in}{3.325159in}}%
\pgfpathlineto{\pgfqpoint{5.321080in}{3.320771in}}%
\pgfpathlineto{\pgfqpoint{5.325376in}{3.322270in}}%
\pgfpathlineto{\pgfqpoint{5.326449in}{3.323227in}}%
\pgfpathlineto{\pgfqpoint{5.331819in}{3.322735in}}%
\pgfpathlineto{\pgfqpoint{5.333966in}{3.322153in}}%
\pgfpathlineto{\pgfqpoint{5.335040in}{3.322296in}}%
\pgfpathlineto{\pgfqpoint{5.336114in}{3.320835in}}%
\pgfpathlineto{\pgfqpoint{5.340409in}{3.321735in}}%
\pgfpathlineto{\pgfqpoint{5.341483in}{3.318953in}}%
\pgfpathlineto{\pgfqpoint{5.342557in}{3.318889in}}%
\pgfpathlineto{\pgfqpoint{5.343631in}{3.319645in}}%
\pgfpathlineto{\pgfqpoint{5.346852in}{3.319136in}}%
\pgfpathlineto{\pgfqpoint{5.347926in}{3.319702in}}%
\pgfpathlineto{\pgfqpoint{5.349000in}{3.319192in}}%
\pgfpathlineto{\pgfqpoint{5.351148in}{3.320381in}}%
\pgfpathlineto{\pgfqpoint{5.354369in}{3.319543in}}%
\pgfpathlineto{\pgfqpoint{5.355443in}{3.318527in}}%
\pgfpathlineto{\pgfqpoint{5.357591in}{3.319430in}}%
\pgfpathlineto{\pgfqpoint{5.358665in}{3.317736in}}%
\pgfpathlineto{\pgfqpoint{5.361886in}{3.317715in}}%
\pgfpathlineto{\pgfqpoint{5.364034in}{3.319534in}}%
\pgfpathlineto{\pgfqpoint{5.365108in}{3.319511in}}%
\pgfpathlineto{\pgfqpoint{5.366181in}{3.318407in}}%
\pgfpathlineto{\pgfqpoint{5.369403in}{3.318620in}}%
\pgfpathlineto{\pgfqpoint{5.370477in}{3.319458in}}%
\pgfpathlineto{\pgfqpoint{5.372625in}{3.318062in}}%
\pgfpathlineto{\pgfqpoint{5.373698in}{3.318019in}}%
\pgfpathlineto{\pgfqpoint{5.376920in}{3.317240in}}%
\pgfpathlineto{\pgfqpoint{5.377994in}{3.317712in}}%
\pgfpathlineto{\pgfqpoint{5.380141in}{3.316966in}}%
\pgfpathlineto{\pgfqpoint{5.385511in}{3.317784in}}%
\pgfpathlineto{\pgfqpoint{5.387658in}{3.316130in}}%
\pgfpathlineto{\pgfqpoint{5.388732in}{3.315338in}}%
\pgfpathlineto{\pgfqpoint{5.391954in}{3.316264in}}%
\pgfpathlineto{\pgfqpoint{5.396249in}{3.320103in}}%
\pgfpathlineto{\pgfqpoint{5.400544in}{3.321279in}}%
\pgfpathlineto{\pgfqpoint{5.401618in}{3.320108in}}%
\pgfpathlineto{\pgfqpoint{5.402692in}{3.320041in}}%
\pgfpathlineto{\pgfqpoint{5.403766in}{3.321100in}}%
\pgfpathlineto{\pgfqpoint{5.408061in}{3.320450in}}%
\pgfpathlineto{\pgfqpoint{5.410209in}{3.318960in}}%
\pgfpathlineto{\pgfqpoint{5.411283in}{3.319570in}}%
\pgfpathlineto{\pgfqpoint{5.415578in}{3.319521in}}%
\pgfpathlineto{\pgfqpoint{5.417726in}{3.319719in}}%
\pgfpathlineto{\pgfqpoint{5.418800in}{3.321640in}}%
\pgfpathlineto{\pgfqpoint{5.524036in}{3.321640in}}%
\pgfpathlineto{\pgfqpoint{5.528331in}{3.319059in}}%
\pgfpathlineto{\pgfqpoint{5.530479in}{3.317527in}}%
\pgfpathlineto{\pgfqpoint{5.531553in}{3.318242in}}%
\pgfpathlineto{\pgfqpoint{5.535848in}{3.318434in}}%
\pgfpathlineto{\pgfqpoint{5.536922in}{3.316961in}}%
\pgfpathlineto{\pgfqpoint{5.537996in}{3.317044in}}%
\pgfpathlineto{\pgfqpoint{5.542291in}{3.315508in}}%
\pgfpathlineto{\pgfqpoint{5.544439in}{3.315903in}}%
\pgfpathlineto{\pgfqpoint{5.546586in}{3.313223in}}%
\pgfpathlineto{\pgfqpoint{5.557325in}{3.314512in}}%
\pgfpathlineto{\pgfqpoint{5.559472in}{3.314018in}}%
\pgfpathlineto{\pgfqpoint{5.566989in}{3.316274in}}%
\pgfpathlineto{\pgfqpoint{5.569137in}{3.317645in}}%
\pgfpathlineto{\pgfqpoint{5.572358in}{3.317667in}}%
\pgfpathlineto{\pgfqpoint{5.574506in}{3.315685in}}%
\pgfpathlineto{\pgfqpoint{5.575580in}{3.313639in}}%
\pgfpathlineto{\pgfqpoint{5.576654in}{3.313045in}}%
\pgfpathlineto{\pgfqpoint{5.584171in}{3.313372in}}%
\pgfpathlineto{\pgfqpoint{5.589540in}{3.312741in}}%
\pgfpathlineto{\pgfqpoint{5.590614in}{3.314017in}}%
\pgfpathlineto{\pgfqpoint{5.591688in}{3.312494in}}%
\pgfpathlineto{\pgfqpoint{5.598131in}{3.312778in}}%
\pgfpathlineto{\pgfqpoint{5.599204in}{3.313589in}}%
\pgfpathlineto{\pgfqpoint{5.602426in}{3.313255in}}%
\pgfpathlineto{\pgfqpoint{5.603500in}{3.312082in}}%
\pgfpathlineto{\pgfqpoint{5.604574in}{3.311889in}}%
\pgfpathlineto{\pgfqpoint{5.605647in}{3.312500in}}%
\pgfpathlineto{\pgfqpoint{5.606721in}{3.314178in}}%
\pgfpathlineto{\pgfqpoint{5.609943in}{3.313534in}}%
\pgfpathlineto{\pgfqpoint{5.612091in}{3.312286in}}%
\pgfpathlineto{\pgfqpoint{5.617460in}{3.312393in}}%
\pgfpathlineto{\pgfqpoint{5.618534in}{3.311573in}}%
\pgfpathlineto{\pgfqpoint{5.620681in}{3.313134in}}%
\pgfpathlineto{\pgfqpoint{5.626050in}{3.314166in}}%
\pgfpathlineto{\pgfqpoint{5.627124in}{3.315456in}}%
\pgfpathlineto{\pgfqpoint{5.628198in}{3.317677in}}%
\pgfpathlineto{\pgfqpoint{5.629272in}{3.318297in}}%
\pgfpathlineto{\pgfqpoint{5.632493in}{3.317228in}}%
\pgfpathlineto{\pgfqpoint{5.633567in}{3.316048in}}%
\pgfpathlineto{\pgfqpoint{5.634641in}{3.317447in}}%
\pgfpathlineto{\pgfqpoint{5.635715in}{3.315874in}}%
\pgfpathlineto{\pgfqpoint{5.694776in}{3.315874in}}%
\pgfpathlineto{\pgfqpoint{5.696924in}{3.314200in}}%
\pgfpathlineto{\pgfqpoint{5.700145in}{3.314802in}}%
\pgfpathlineto{\pgfqpoint{5.703367in}{3.313788in}}%
\pgfpathlineto{\pgfqpoint{5.704441in}{3.312523in}}%
\pgfpathlineto{\pgfqpoint{5.710884in}{3.313926in}}%
\pgfpathlineto{\pgfqpoint{5.715179in}{3.313886in}}%
\pgfpathlineto{\pgfqpoint{5.717327in}{3.312435in}}%
\pgfpathlineto{\pgfqpoint{5.718401in}{3.313058in}}%
\pgfpathlineto{\pgfqpoint{5.719474in}{3.312884in}}%
\pgfpathlineto{\pgfqpoint{5.723770in}{3.313848in}}%
\pgfpathlineto{\pgfqpoint{5.724844in}{3.313531in}}%
\pgfpathlineto{\pgfqpoint{5.725917in}{3.314472in}}%
\pgfpathlineto{\pgfqpoint{5.732360in}{3.313410in}}%
\pgfpathlineto{\pgfqpoint{5.734508in}{3.314615in}}%
\pgfpathlineto{\pgfqpoint{5.742025in}{3.314273in}}%
\pgfpathlineto{\pgfqpoint{5.746320in}{3.314273in}}%
\pgfpathlineto{\pgfqpoint{5.748468in}{3.313644in}}%
\pgfpathlineto{\pgfqpoint{5.776388in}{3.313644in}}%
\pgfpathlineto{\pgfqpoint{5.778535in}{3.312843in}}%
\pgfpathlineto{\pgfqpoint{5.786052in}{3.312376in}}%
\pgfpathlineto{\pgfqpoint{5.787126in}{3.312627in}}%
\pgfpathlineto{\pgfqpoint{5.794643in}{3.311876in}}%
\pgfpathlineto{\pgfqpoint{5.798938in}{3.311706in}}%
\pgfpathlineto{\pgfqpoint{5.800012in}{3.312608in}}%
\pgfpathlineto{\pgfqpoint{5.807529in}{3.311418in}}%
\pgfpathlineto{\pgfqpoint{5.808603in}{3.310227in}}%
\pgfpathlineto{\pgfqpoint{5.809677in}{3.312579in}}%
\pgfpathlineto{\pgfqpoint{5.813972in}{3.312579in}}%
\pgfpathlineto{\pgfqpoint{5.816120in}{3.309741in}}%
\pgfpathlineto{\pgfqpoint{5.821489in}{3.309937in}}%
\pgfpathlineto{\pgfqpoint{5.823637in}{3.309103in}}%
\pgfpathlineto{\pgfqpoint{5.824711in}{3.307820in}}%
\pgfpathlineto{\pgfqpoint{5.827932in}{3.307258in}}%
\pgfpathlineto{\pgfqpoint{5.829006in}{3.306418in}}%
\pgfpathlineto{\pgfqpoint{5.837597in}{3.305348in}}%
\pgfpathlineto{\pgfqpoint{5.838670in}{3.307310in}}%
\pgfpathlineto{\pgfqpoint{5.839744in}{3.306726in}}%
\pgfpathlineto{\pgfqpoint{5.842966in}{3.306678in}}%
\pgfpathlineto{\pgfqpoint{5.844040in}{3.306092in}}%
\pgfpathlineto{\pgfqpoint{5.846187in}{3.306558in}}%
\pgfpathlineto{\pgfqpoint{5.850483in}{3.306432in}}%
\pgfpathlineto{\pgfqpoint{5.852630in}{3.306935in}}%
\pgfpathlineto{\pgfqpoint{5.854778in}{3.305859in}}%
\pgfpathlineto{\pgfqpoint{5.859073in}{3.305950in}}%
\pgfpathlineto{\pgfqpoint{5.860147in}{3.306504in}}%
\pgfpathlineto{\pgfqpoint{5.866590in}{3.305522in}}%
\pgfpathlineto{\pgfqpoint{5.868738in}{3.305856in}}%
\pgfpathlineto{\pgfqpoint{5.869812in}{3.305472in}}%
\pgfpathlineto{\pgfqpoint{5.884846in}{3.304327in}}%
\pgfpathlineto{\pgfqpoint{5.891289in}{3.303833in}}%
\pgfpathlineto{\pgfqpoint{5.892362in}{3.305144in}}%
\pgfpathlineto{\pgfqpoint{5.895584in}{3.304265in}}%
\pgfpathlineto{\pgfqpoint{5.897732in}{3.304868in}}%
\pgfpathlineto{\pgfqpoint{5.899879in}{3.302159in}}%
\pgfpathlineto{\pgfqpoint{5.904175in}{3.302805in}}%
\pgfpathlineto{\pgfqpoint{5.906322in}{3.302292in}}%
\pgfpathlineto{\pgfqpoint{5.907396in}{3.302729in}}%
\pgfpathlineto{\pgfqpoint{5.910618in}{3.303404in}}%
\pgfpathlineto{\pgfqpoint{5.912765in}{3.302399in}}%
\pgfpathlineto{\pgfqpoint{5.913839in}{3.302545in}}%
\pgfpathlineto{\pgfqpoint{5.914913in}{3.302009in}}%
\pgfpathlineto{\pgfqpoint{5.920282in}{3.301716in}}%
\pgfpathlineto{\pgfqpoint{5.925651in}{3.301677in}}%
\pgfpathlineto{\pgfqpoint{5.926725in}{3.302580in}}%
\pgfpathlineto{\pgfqpoint{5.928873in}{3.302944in}}%
\pgfpathlineto{\pgfqpoint{5.929947in}{3.302344in}}%
\pgfpathlineto{\pgfqpoint{5.933168in}{3.302543in}}%
\pgfpathlineto{\pgfqpoint{5.934242in}{3.301966in}}%
\pgfpathlineto{\pgfqpoint{5.935316in}{3.304674in}}%
\pgfpathlineto{\pgfqpoint{5.952497in}{3.306998in}}%
\pgfpathlineto{\pgfqpoint{5.956793in}{3.305303in}}%
\pgfpathlineto{\pgfqpoint{5.970753in}{3.304959in}}%
\pgfpathlineto{\pgfqpoint{5.971826in}{3.304257in}}%
\pgfpathlineto{\pgfqpoint{5.972900in}{3.304634in}}%
\pgfpathlineto{\pgfqpoint{5.978269in}{3.304208in}}%
\pgfpathlineto{\pgfqpoint{5.980417in}{3.305318in}}%
\pgfpathlineto{\pgfqpoint{5.981491in}{3.306638in}}%
\pgfpathlineto{\pgfqpoint{5.982565in}{3.306045in}}%
\pgfpathlineto{\pgfqpoint{5.986860in}{3.305239in}}%
\pgfpathlineto{\pgfqpoint{5.987934in}{3.304134in}}%
\pgfpathlineto{\pgfqpoint{5.990082in}{3.303788in}}%
\pgfpathlineto{\pgfqpoint{5.993303in}{3.303517in}}%
\pgfpathlineto{\pgfqpoint{5.994377in}{3.302418in}}%
\pgfpathlineto{\pgfqpoint{5.995451in}{3.302741in}}%
\pgfpathlineto{\pgfqpoint{5.996525in}{3.302427in}}%
\pgfpathlineto{\pgfqpoint{5.997599in}{3.303033in}}%
\pgfpathlineto{\pgfqpoint{6.002968in}{3.302174in}}%
\pgfpathlineto{\pgfqpoint{6.020149in}{3.302777in}}%
\pgfpathlineto{\pgfqpoint{6.033035in}{3.302415in}}%
\pgfpathlineto{\pgfqpoint{6.034109in}{3.302657in}}%
\pgfpathlineto{\pgfqpoint{6.035183in}{3.302181in}}%
\pgfpathlineto{\pgfqpoint{6.038404in}{3.302394in}}%
\pgfpathlineto{\pgfqpoint{6.039478in}{3.301331in}}%
\pgfpathlineto{\pgfqpoint{6.042700in}{3.300903in}}%
\pgfpathlineto{\pgfqpoint{6.046995in}{3.302285in}}%
\pgfpathlineto{\pgfqpoint{6.049143in}{3.302457in}}%
\pgfpathlineto{\pgfqpoint{6.050217in}{3.302322in}}%
\pgfpathlineto{\pgfqpoint{6.055586in}{3.302494in}}%
\pgfpathlineto{\pgfqpoint{6.056660in}{3.303672in}}%
\pgfpathlineto{\pgfqpoint{6.057734in}{3.303828in}}%
\pgfpathlineto{\pgfqpoint{6.065250in}{3.302329in}}%
\pgfpathlineto{\pgfqpoint{6.069546in}{3.302275in}}%
\pgfpathlineto{\pgfqpoint{6.070620in}{3.302838in}}%
\pgfpathlineto{\pgfqpoint{6.075989in}{3.302277in}}%
\pgfpathlineto{\pgfqpoint{6.079210in}{3.303035in}}%
\pgfpathlineto{\pgfqpoint{6.087801in}{3.303026in}}%
\pgfpathlineto{\pgfqpoint{6.092096in}{3.303982in}}%
\pgfpathlineto{\pgfqpoint{6.095318in}{3.303851in}}%
\pgfpathlineto{\pgfqpoint{6.098539in}{3.303635in}}%
\pgfpathlineto{\pgfqpoint{6.099613in}{3.304162in}}%
\pgfpathlineto{\pgfqpoint{6.101761in}{3.303839in}}%
\pgfpathlineto{\pgfqpoint{6.102835in}{3.303997in}}%
\pgfpathlineto{\pgfqpoint{6.117868in}{3.302830in}}%
\pgfpathlineto{\pgfqpoint{6.128607in}{3.303547in}}%
\pgfpathlineto{\pgfqpoint{6.132902in}{3.302444in}}%
\pgfpathlineto{\pgfqpoint{6.139345in}{3.301063in}}%
\pgfpathlineto{\pgfqpoint{6.140419in}{3.302592in}}%
\pgfpathlineto{\pgfqpoint{6.143641in}{3.302387in}}%
\pgfpathlineto{\pgfqpoint{6.145788in}{3.301196in}}%
\pgfpathlineto{\pgfqpoint{6.158674in}{3.302889in}}%
\pgfpathlineto{\pgfqpoint{6.159748in}{3.302640in}}%
\pgfpathlineto{\pgfqpoint{6.160822in}{3.303639in}}%
\pgfpathlineto{\pgfqpoint{6.169413in}{3.302040in}}%
\pgfpathlineto{\pgfqpoint{6.184446in}{3.301756in}}%
\pgfpathlineto{\pgfqpoint{6.185520in}{3.302718in}}%
\pgfpathlineto{\pgfqpoint{6.190889in}{3.302948in}}%
\pgfpathlineto{\pgfqpoint{6.193037in}{3.303366in}}%
\pgfpathlineto{\pgfqpoint{6.196259in}{3.302970in}}%
\pgfpathlineto{\pgfqpoint{6.198406in}{3.304208in}}%
\pgfpathlineto{\pgfqpoint{6.200554in}{3.304753in}}%
\pgfpathlineto{\pgfqpoint{6.205923in}{3.304717in}}%
\pgfpathlineto{\pgfqpoint{6.206997in}{3.305646in}}%
\pgfpathlineto{\pgfqpoint{6.208071in}{3.305350in}}%
\pgfpathlineto{\pgfqpoint{6.211292in}{3.305765in}}%
\pgfpathlineto{\pgfqpoint{6.213440in}{3.304467in}}%
\pgfpathlineto{\pgfqpoint{6.214514in}{3.305461in}}%
\pgfpathlineto{\pgfqpoint{6.215588in}{3.305090in}}%
\pgfpathlineto{\pgfqpoint{6.219883in}{3.305024in}}%
\pgfpathlineto{\pgfqpoint{6.223105in}{3.303941in}}%
\pgfpathlineto{\pgfqpoint{6.228474in}{3.304114in}}%
\pgfpathlineto{\pgfqpoint{6.230622in}{3.303880in}}%
\pgfpathlineto{\pgfqpoint{6.235991in}{3.303847in}}%
\pgfpathlineto{\pgfqpoint{6.245655in}{3.301442in}}%
\pgfpathlineto{\pgfqpoint{6.252098in}{3.301630in}}%
\pgfpathlineto{\pgfqpoint{6.253172in}{3.301973in}}%
\pgfpathlineto{\pgfqpoint{6.256394in}{3.301371in}}%
\pgfpathlineto{\pgfqpoint{6.260689in}{3.303053in}}%
\pgfpathlineto{\pgfqpoint{6.273575in}{3.303217in}}%
\pgfpathlineto{\pgfqpoint{6.275723in}{3.302949in}}%
\pgfpathlineto{\pgfqpoint{6.280018in}{3.303047in}}%
\pgfpathlineto{\pgfqpoint{6.281092in}{3.302051in}}%
\pgfpathlineto{\pgfqpoint{6.283240in}{3.302782in}}%
\pgfpathlineto{\pgfqpoint{6.295052in}{3.300210in}}%
\pgfpathlineto{\pgfqpoint{6.298273in}{3.300271in}}%
\pgfpathlineto{\pgfqpoint{6.303643in}{3.299849in}}%
\pgfpathlineto{\pgfqpoint{6.312233in}{3.297617in}}%
\pgfpathlineto{\pgfqpoint{6.317602in}{3.297494in}}%
\pgfpathlineto{\pgfqpoint{6.319750in}{3.297976in}}%
\pgfpathlineto{\pgfqpoint{6.320824in}{3.297463in}}%
\pgfpathlineto{\pgfqpoint{6.327267in}{3.297025in}}%
\pgfpathlineto{\pgfqpoint{6.328341in}{3.296684in}}%
\pgfpathlineto{\pgfqpoint{6.334784in}{3.295769in}}%
\pgfpathlineto{\pgfqpoint{6.335858in}{3.292863in}}%
\pgfpathlineto{\pgfqpoint{6.339079in}{3.292889in}}%
\pgfpathlineto{\pgfqpoint{6.342301in}{3.290719in}}%
\pgfpathlineto{\pgfqpoint{6.343375in}{3.291279in}}%
\pgfpathlineto{\pgfqpoint{6.348744in}{3.290796in}}%
\pgfpathlineto{\pgfqpoint{6.350891in}{3.291629in}}%
\pgfpathlineto{\pgfqpoint{6.357334in}{3.291577in}}%
\pgfpathlineto{\pgfqpoint{6.358408in}{3.292359in}}%
\pgfpathlineto{\pgfqpoint{6.376664in}{3.292445in}}%
\pgfpathlineto{\pgfqpoint{6.377737in}{3.293293in}}%
\pgfpathlineto{\pgfqpoint{6.387402in}{3.293436in}}%
\pgfpathlineto{\pgfqpoint{6.388476in}{3.292326in}}%
\pgfpathlineto{\pgfqpoint{6.393845in}{3.289994in}}%
\pgfpathlineto{\pgfqpoint{6.395993in}{3.290602in}}%
\pgfpathlineto{\pgfqpoint{6.403510in}{3.290999in}}%
\pgfpathlineto{\pgfqpoint{6.403510in}{3.290999in}}%
\pgfusepath{stroke}%
\end{pgfscope}%
\begin{pgfscope}%
\pgfsetrectcap%
\pgfsetmiterjoin%
\pgfsetlinewidth{0.803000pt}%
\definecolor{currentstroke}{rgb}{1.000000,1.000000,1.000000}%
\pgfsetstrokecolor{currentstroke}%
\pgfsetdash{}{0pt}%
\pgfpathmoveto{\pgfqpoint{3.937600in}{3.271772in}}%
\pgfpathlineto{\pgfqpoint{3.937600in}{3.672657in}}%
\pgfusepath{stroke}%
\end{pgfscope}%
\begin{pgfscope}%
\pgfsetrectcap%
\pgfsetmiterjoin%
\pgfsetlinewidth{0.803000pt}%
\definecolor{currentstroke}{rgb}{1.000000,1.000000,1.000000}%
\pgfsetstrokecolor{currentstroke}%
\pgfsetdash{}{0pt}%
\pgfpathmoveto{\pgfqpoint{6.520934in}{3.271772in}}%
\pgfpathlineto{\pgfqpoint{6.520934in}{3.672657in}}%
\pgfusepath{stroke}%
\end{pgfscope}%
\begin{pgfscope}%
\pgfsetrectcap%
\pgfsetmiterjoin%
\pgfsetlinewidth{0.803000pt}%
\definecolor{currentstroke}{rgb}{1.000000,1.000000,1.000000}%
\pgfsetstrokecolor{currentstroke}%
\pgfsetdash{}{0pt}%
\pgfpathmoveto{\pgfqpoint{3.937600in}{3.271772in}}%
\pgfpathlineto{\pgfqpoint{6.520934in}{3.271772in}}%
\pgfusepath{stroke}%
\end{pgfscope}%
\begin{pgfscope}%
\pgfsetrectcap%
\pgfsetmiterjoin%
\pgfsetlinewidth{0.803000pt}%
\definecolor{currentstroke}{rgb}{1.000000,1.000000,1.000000}%
\pgfsetstrokecolor{currentstroke}%
\pgfsetdash{}{0pt}%
\pgfpathmoveto{\pgfqpoint{3.937600in}{3.672657in}}%
\pgfpathlineto{\pgfqpoint{6.520934in}{3.672657in}}%
\pgfusepath{stroke}%
\end{pgfscope}%
\begin{pgfscope}%
\definecolor{textcolor}{rgb}{0.150000,0.150000,0.150000}%
\pgfsetstrokecolor{textcolor}%
\pgfsetfillcolor{textcolor}%
\pgftext[x=5.229267in,y=3.755990in,,base]{\color{textcolor}\rmfamily\fontsize{16.800000}{20.160000}\selectfont INTC}%
\end{pgfscope}%
\begin{pgfscope}%
\pgfsetbuttcap%
\pgfsetmiterjoin%
\definecolor{currentfill}{rgb}{0.917647,0.917647,0.949020}%
\pgfsetfillcolor{currentfill}%
\pgfsetlinewidth{0.000000pt}%
\definecolor{currentstroke}{rgb}{0.000000,0.000000,0.000000}%
\pgfsetstrokecolor{currentstroke}%
\pgfsetstrokeopacity{0.000000}%
\pgfsetdash{}{0pt}%
\pgfpathmoveto{\pgfqpoint{0.320934in}{2.309648in}}%
\pgfpathlineto{\pgfqpoint{2.904267in}{2.309648in}}%
\pgfpathlineto{\pgfqpoint{2.904267in}{2.710533in}}%
\pgfpathlineto{\pgfqpoint{0.320934in}{2.710533in}}%
\pgfpathclose%
\pgfusepath{fill}%
\end{pgfscope}%
\begin{pgfscope}%
\pgfpathrectangle{\pgfqpoint{0.320934in}{2.309648in}}{\pgfqpoint{2.583333in}{0.400885in}}%
\pgfusepath{clip}%
\pgfsetroundcap%
\pgfsetroundjoin%
\pgfsetlinewidth{0.803000pt}%
\definecolor{currentstroke}{rgb}{1.000000,1.000000,1.000000}%
\pgfsetstrokecolor{currentstroke}%
\pgfsetdash{}{0pt}%
\pgfpathmoveto{\pgfqpoint{0.436210in}{2.309648in}}%
\pgfpathlineto{\pgfqpoint{0.436210in}{2.710533in}}%
\pgfusepath{stroke}%
\end{pgfscope}%
\begin{pgfscope}%
\definecolor{textcolor}{rgb}{0.150000,0.150000,0.150000}%
\pgfsetstrokecolor{textcolor}%
\pgfsetfillcolor{textcolor}%
\pgftext[x=0.436210in,y=2.212426in,,top]{\color{textcolor}\rmfamily\fontsize{14.000000}{16.800000}\selectfont 2012}%
\end{pgfscope}%
\begin{pgfscope}%
\pgfpathrectangle{\pgfqpoint{0.320934in}{2.309648in}}{\pgfqpoint{2.583333in}{0.400885in}}%
\pgfusepath{clip}%
\pgfsetroundcap%
\pgfsetroundjoin%
\pgfsetlinewidth{0.803000pt}%
\definecolor{currentstroke}{rgb}{1.000000,1.000000,1.000000}%
\pgfsetstrokecolor{currentstroke}%
\pgfsetdash{}{0pt}%
\pgfpathmoveto{\pgfqpoint{0.829235in}{2.309648in}}%
\pgfpathlineto{\pgfqpoint{0.829235in}{2.710533in}}%
\pgfusepath{stroke}%
\end{pgfscope}%
\begin{pgfscope}%
\definecolor{textcolor}{rgb}{0.150000,0.150000,0.150000}%
\pgfsetstrokecolor{textcolor}%
\pgfsetfillcolor{textcolor}%
\pgftext[x=0.829235in,y=2.212426in,,top]{\color{textcolor}\rmfamily\fontsize{14.000000}{16.800000}\selectfont 2013}%
\end{pgfscope}%
\begin{pgfscope}%
\pgfpathrectangle{\pgfqpoint{0.320934in}{2.309648in}}{\pgfqpoint{2.583333in}{0.400885in}}%
\pgfusepath{clip}%
\pgfsetroundcap%
\pgfsetroundjoin%
\pgfsetlinewidth{0.803000pt}%
\definecolor{currentstroke}{rgb}{1.000000,1.000000,1.000000}%
\pgfsetstrokecolor{currentstroke}%
\pgfsetdash{}{0pt}%
\pgfpathmoveto{\pgfqpoint{1.221186in}{2.309648in}}%
\pgfpathlineto{\pgfqpoint{1.221186in}{2.710533in}}%
\pgfusepath{stroke}%
\end{pgfscope}%
\begin{pgfscope}%
\definecolor{textcolor}{rgb}{0.150000,0.150000,0.150000}%
\pgfsetstrokecolor{textcolor}%
\pgfsetfillcolor{textcolor}%
\pgftext[x=1.221186in,y=2.212426in,,top]{\color{textcolor}\rmfamily\fontsize{14.000000}{16.800000}\selectfont 2014}%
\end{pgfscope}%
\begin{pgfscope}%
\pgfpathrectangle{\pgfqpoint{0.320934in}{2.309648in}}{\pgfqpoint{2.583333in}{0.400885in}}%
\pgfusepath{clip}%
\pgfsetroundcap%
\pgfsetroundjoin%
\pgfsetlinewidth{0.803000pt}%
\definecolor{currentstroke}{rgb}{1.000000,1.000000,1.000000}%
\pgfsetstrokecolor{currentstroke}%
\pgfsetdash{}{0pt}%
\pgfpathmoveto{\pgfqpoint{1.613137in}{2.309648in}}%
\pgfpathlineto{\pgfqpoint{1.613137in}{2.710533in}}%
\pgfusepath{stroke}%
\end{pgfscope}%
\begin{pgfscope}%
\definecolor{textcolor}{rgb}{0.150000,0.150000,0.150000}%
\pgfsetstrokecolor{textcolor}%
\pgfsetfillcolor{textcolor}%
\pgftext[x=1.613137in,y=2.212426in,,top]{\color{textcolor}\rmfamily\fontsize{14.000000}{16.800000}\selectfont 2015}%
\end{pgfscope}%
\begin{pgfscope}%
\pgfpathrectangle{\pgfqpoint{0.320934in}{2.309648in}}{\pgfqpoint{2.583333in}{0.400885in}}%
\pgfusepath{clip}%
\pgfsetroundcap%
\pgfsetroundjoin%
\pgfsetlinewidth{0.803000pt}%
\definecolor{currentstroke}{rgb}{1.000000,1.000000,1.000000}%
\pgfsetstrokecolor{currentstroke}%
\pgfsetdash{}{0pt}%
\pgfpathmoveto{\pgfqpoint{2.005088in}{2.309648in}}%
\pgfpathlineto{\pgfqpoint{2.005088in}{2.710533in}}%
\pgfusepath{stroke}%
\end{pgfscope}%
\begin{pgfscope}%
\definecolor{textcolor}{rgb}{0.150000,0.150000,0.150000}%
\pgfsetstrokecolor{textcolor}%
\pgfsetfillcolor{textcolor}%
\pgftext[x=2.005088in,y=2.212426in,,top]{\color{textcolor}\rmfamily\fontsize{14.000000}{16.800000}\selectfont 2016}%
\end{pgfscope}%
\begin{pgfscope}%
\pgfpathrectangle{\pgfqpoint{0.320934in}{2.309648in}}{\pgfqpoint{2.583333in}{0.400885in}}%
\pgfusepath{clip}%
\pgfsetroundcap%
\pgfsetroundjoin%
\pgfsetlinewidth{0.803000pt}%
\definecolor{currentstroke}{rgb}{1.000000,1.000000,1.000000}%
\pgfsetstrokecolor{currentstroke}%
\pgfsetdash{}{0pt}%
\pgfpathmoveto{\pgfqpoint{2.398113in}{2.309648in}}%
\pgfpathlineto{\pgfqpoint{2.398113in}{2.710533in}}%
\pgfusepath{stroke}%
\end{pgfscope}%
\begin{pgfscope}%
\definecolor{textcolor}{rgb}{0.150000,0.150000,0.150000}%
\pgfsetstrokecolor{textcolor}%
\pgfsetfillcolor{textcolor}%
\pgftext[x=2.398113in,y=2.212426in,,top]{\color{textcolor}\rmfamily\fontsize{14.000000}{16.800000}\selectfont 2017}%
\end{pgfscope}%
\begin{pgfscope}%
\pgfpathrectangle{\pgfqpoint{0.320934in}{2.309648in}}{\pgfqpoint{2.583333in}{0.400885in}}%
\pgfusepath{clip}%
\pgfsetroundcap%
\pgfsetroundjoin%
\pgfsetlinewidth{0.803000pt}%
\definecolor{currentstroke}{rgb}{1.000000,1.000000,1.000000}%
\pgfsetstrokecolor{currentstroke}%
\pgfsetdash{}{0pt}%
\pgfpathmoveto{\pgfqpoint{2.790064in}{2.309648in}}%
\pgfpathlineto{\pgfqpoint{2.790064in}{2.710533in}}%
\pgfusepath{stroke}%
\end{pgfscope}%
\begin{pgfscope}%
\definecolor{textcolor}{rgb}{0.150000,0.150000,0.150000}%
\pgfsetstrokecolor{textcolor}%
\pgfsetfillcolor{textcolor}%
\pgftext[x=2.790064in,y=2.212426in,,top]{\color{textcolor}\rmfamily\fontsize{14.000000}{16.800000}\selectfont 2018}%
\end{pgfscope}%
\begin{pgfscope}%
\pgfpathrectangle{\pgfqpoint{0.320934in}{2.309648in}}{\pgfqpoint{2.583333in}{0.400885in}}%
\pgfusepath{clip}%
\pgfsetroundcap%
\pgfsetroundjoin%
\pgfsetlinewidth{0.803000pt}%
\definecolor{currentstroke}{rgb}{1.000000,1.000000,1.000000}%
\pgfsetstrokecolor{currentstroke}%
\pgfsetdash{}{0pt}%
\pgfpathmoveto{\pgfqpoint{0.320934in}{2.441950in}}%
\pgfpathlineto{\pgfqpoint{2.904267in}{2.441950in}}%
\pgfusepath{stroke}%
\end{pgfscope}%
\begin{pgfscope}%
\definecolor{textcolor}{rgb}{0.150000,0.150000,0.150000}%
\pgfsetstrokecolor{textcolor}%
\pgfsetfillcolor{textcolor}%
\pgftext[x=0.100000in,y=2.368084in,left,base]{\color{textcolor}\rmfamily\fontsize{14.000000}{16.800000}\selectfont 1}%
\end{pgfscope}%
\begin{pgfscope}%
\pgfpathrectangle{\pgfqpoint{0.320934in}{2.309648in}}{\pgfqpoint{2.583333in}{0.400885in}}%
\pgfusepath{clip}%
\pgfsetroundcap%
\pgfsetroundjoin%
\pgfsetlinewidth{0.803000pt}%
\definecolor{currentstroke}{rgb}{1.000000,1.000000,1.000000}%
\pgfsetstrokecolor{currentstroke}%
\pgfsetdash{}{0pt}%
\pgfpathmoveto{\pgfqpoint{0.320934in}{2.599671in}}%
\pgfpathlineto{\pgfqpoint{2.904267in}{2.599671in}}%
\pgfusepath{stroke}%
\end{pgfscope}%
\begin{pgfscope}%
\definecolor{textcolor}{rgb}{0.150000,0.150000,0.150000}%
\pgfsetstrokecolor{textcolor}%
\pgfsetfillcolor{textcolor}%
\pgftext[x=0.100000in,y=2.525804in,left,base]{\color{textcolor}\rmfamily\fontsize{14.000000}{16.800000}\selectfont 2}%
\end{pgfscope}%
\begin{pgfscope}%
\pgfpathrectangle{\pgfqpoint{0.320934in}{2.309648in}}{\pgfqpoint{2.583333in}{0.400885in}}%
\pgfusepath{clip}%
\pgfsetroundcap%
\pgfsetroundjoin%
\pgfsetlinewidth{1.505625pt}%
\definecolor{currentstroke}{rgb}{0.000000,0.000000,0.000000}%
\pgfsetstrokecolor{currentstroke}%
\pgfsetdash{}{0pt}%
\pgfpathmoveto{\pgfqpoint{0.438358in}{2.441950in}}%
\pgfpathlineto{\pgfqpoint{0.439432in}{2.440999in}}%
\pgfpathlineto{\pgfqpoint{0.440506in}{2.440820in}}%
\pgfpathlineto{\pgfqpoint{0.441580in}{2.439453in}}%
\pgfpathlineto{\pgfqpoint{0.444801in}{2.439690in}}%
\pgfpathlineto{\pgfqpoint{0.445875in}{2.440344in}}%
\pgfpathlineto{\pgfqpoint{0.448023in}{2.440404in}}%
\pgfpathlineto{\pgfqpoint{0.449096in}{2.440463in}}%
\pgfpathlineto{\pgfqpoint{0.455539in}{2.440315in}}%
\pgfpathlineto{\pgfqpoint{0.456613in}{2.440493in}}%
\pgfpathlineto{\pgfqpoint{0.460909in}{2.439869in}}%
\pgfpathlineto{\pgfqpoint{0.461982in}{2.440374in}}%
\pgfpathlineto{\pgfqpoint{0.463056in}{2.441534in}}%
\pgfpathlineto{\pgfqpoint{0.464130in}{2.441207in}}%
\pgfpathlineto{\pgfqpoint{0.469499in}{2.441504in}}%
\pgfpathlineto{\pgfqpoint{0.475942in}{2.440463in}}%
\pgfpathlineto{\pgfqpoint{0.477016in}{2.440434in}}%
\pgfpathlineto{\pgfqpoint{0.479164in}{2.438888in}}%
\pgfpathlineto{\pgfqpoint{0.497419in}{2.439898in}}%
\pgfpathlineto{\pgfqpoint{0.498493in}{2.441653in}}%
\pgfpathlineto{\pgfqpoint{0.501714in}{2.440671in}}%
\pgfpathlineto{\pgfqpoint{0.504936in}{2.440999in}}%
\pgfpathlineto{\pgfqpoint{0.506010in}{2.439661in}}%
\pgfpathlineto{\pgfqpoint{0.507084in}{2.439542in}}%
\pgfpathlineto{\pgfqpoint{0.508157in}{2.440880in}}%
\pgfpathlineto{\pgfqpoint{0.509231in}{2.440612in}}%
\pgfpathlineto{\pgfqpoint{0.513527in}{2.442039in}}%
\pgfpathlineto{\pgfqpoint{0.515674in}{2.441385in}}%
\pgfpathlineto{\pgfqpoint{0.519970in}{2.441742in}}%
\pgfpathlineto{\pgfqpoint{0.523191in}{2.439928in}}%
\pgfpathlineto{\pgfqpoint{0.524265in}{2.440136in}}%
\pgfpathlineto{\pgfqpoint{0.529634in}{2.442723in}}%
\pgfpathlineto{\pgfqpoint{0.530708in}{2.442545in}}%
\pgfpathlineto{\pgfqpoint{0.531782in}{2.443555in}}%
\pgfpathlineto{\pgfqpoint{0.535003in}{2.444150in}}%
\pgfpathlineto{\pgfqpoint{0.538225in}{2.442039in}}%
\pgfpathlineto{\pgfqpoint{0.542520in}{2.441028in}}%
\pgfpathlineto{\pgfqpoint{0.543594in}{2.439304in}}%
\pgfpathlineto{\pgfqpoint{0.545742in}{2.439185in}}%
\pgfpathlineto{\pgfqpoint{0.546816in}{2.437698in}}%
\pgfpathlineto{\pgfqpoint{0.551111in}{2.439334in}}%
\pgfpathlineto{\pgfqpoint{0.552185in}{2.437015in}}%
\pgfpathlineto{\pgfqpoint{0.553259in}{2.436479in}}%
\pgfpathlineto{\pgfqpoint{0.554333in}{2.438115in}}%
\pgfpathlineto{\pgfqpoint{0.557554in}{2.437282in}}%
\pgfpathlineto{\pgfqpoint{0.561849in}{2.440850in}}%
\pgfpathlineto{\pgfqpoint{0.568292in}{2.442039in}}%
\pgfpathlineto{\pgfqpoint{0.569366in}{2.440612in}}%
\pgfpathlineto{\pgfqpoint{0.573662in}{2.441177in}}%
\pgfpathlineto{\pgfqpoint{0.574735in}{2.439482in}}%
\pgfpathlineto{\pgfqpoint{0.575809in}{2.440196in}}%
\pgfpathlineto{\pgfqpoint{0.576883in}{2.439631in}}%
\pgfpathlineto{\pgfqpoint{0.584400in}{2.437252in}}%
\pgfpathlineto{\pgfqpoint{0.588695in}{2.437669in}}%
\pgfpathlineto{\pgfqpoint{0.589769in}{2.437044in}}%
\pgfpathlineto{\pgfqpoint{0.590843in}{2.438115in}}%
\pgfpathlineto{\pgfqpoint{0.591917in}{2.436688in}}%
\pgfpathlineto{\pgfqpoint{0.596212in}{2.436896in}}%
\pgfpathlineto{\pgfqpoint{0.597286in}{2.435944in}}%
\pgfpathlineto{\pgfqpoint{0.598360in}{2.436479in}}%
\pgfpathlineto{\pgfqpoint{0.599434in}{2.434904in}}%
\pgfpathlineto{\pgfqpoint{0.602655in}{2.436271in}}%
\pgfpathlineto{\pgfqpoint{0.603729in}{2.435944in}}%
\pgfpathlineto{\pgfqpoint{0.604803in}{2.437401in}}%
\pgfpathlineto{\pgfqpoint{0.606951in}{2.437817in}}%
\pgfpathlineto{\pgfqpoint{0.610172in}{2.435736in}}%
\pgfpathlineto{\pgfqpoint{0.614468in}{2.445220in}}%
\pgfpathlineto{\pgfqpoint{0.617689in}{2.445934in}}%
\pgfpathlineto{\pgfqpoint{0.619837in}{2.447628in}}%
\pgfpathlineto{\pgfqpoint{0.620911in}{2.446142in}}%
\pgfpathlineto{\pgfqpoint{0.621984in}{2.446737in}}%
\pgfpathlineto{\pgfqpoint{0.626280in}{2.446231in}}%
\pgfpathlineto{\pgfqpoint{0.627354in}{2.447242in}}%
\pgfpathlineto{\pgfqpoint{0.628427in}{2.447480in}}%
\pgfpathlineto{\pgfqpoint{0.629501in}{2.448996in}}%
\pgfpathlineto{\pgfqpoint{0.633797in}{2.450185in}}%
\pgfpathlineto{\pgfqpoint{0.637018in}{2.449204in}}%
\pgfpathlineto{\pgfqpoint{0.643461in}{2.449383in}}%
\pgfpathlineto{\pgfqpoint{0.644535in}{2.451553in}}%
\pgfpathlineto{\pgfqpoint{0.647757in}{2.451166in}}%
\pgfpathlineto{\pgfqpoint{0.649904in}{2.453426in}}%
\pgfpathlineto{\pgfqpoint{0.650978in}{2.453812in}}%
\pgfpathlineto{\pgfqpoint{0.652052in}{2.451612in}}%
\pgfpathlineto{\pgfqpoint{0.655273in}{2.450334in}}%
\pgfpathlineto{\pgfqpoint{0.656347in}{2.448491in}}%
\pgfpathlineto{\pgfqpoint{0.657421in}{2.448937in}}%
\pgfpathlineto{\pgfqpoint{0.659569in}{2.453783in}}%
\pgfpathlineto{\pgfqpoint{0.664938in}{2.453456in}}%
\pgfpathlineto{\pgfqpoint{0.666012in}{2.451166in}}%
\pgfpathlineto{\pgfqpoint{0.667086in}{2.452802in}}%
\pgfpathlineto{\pgfqpoint{0.670307in}{2.452118in}}%
\pgfpathlineto{\pgfqpoint{0.671381in}{2.450780in}}%
\pgfpathlineto{\pgfqpoint{0.678898in}{2.451642in}}%
\pgfpathlineto{\pgfqpoint{0.682119in}{2.449591in}}%
\pgfpathlineto{\pgfqpoint{0.688562in}{2.449442in}}%
\pgfpathlineto{\pgfqpoint{0.689636in}{2.450601in}}%
\pgfpathlineto{\pgfqpoint{0.702522in}{2.449799in}}%
\pgfpathlineto{\pgfqpoint{0.703596in}{2.451196in}}%
\pgfpathlineto{\pgfqpoint{0.710039in}{2.451939in}}%
\pgfpathlineto{\pgfqpoint{0.711113in}{2.454021in}}%
\pgfpathlineto{\pgfqpoint{0.712187in}{2.452742in}}%
\pgfpathlineto{\pgfqpoint{0.715408in}{2.452207in}}%
\pgfpathlineto{\pgfqpoint{0.718630in}{2.453783in}}%
\pgfpathlineto{\pgfqpoint{0.719704in}{2.454199in}}%
\pgfpathlineto{\pgfqpoint{0.722925in}{2.454050in}}%
\pgfpathlineto{\pgfqpoint{0.723999in}{2.454823in}}%
\pgfpathlineto{\pgfqpoint{0.725073in}{2.454050in}}%
\pgfpathlineto{\pgfqpoint{0.733664in}{2.454704in}}%
\pgfpathlineto{\pgfqpoint{0.734737in}{2.455656in}}%
\pgfpathlineto{\pgfqpoint{0.737959in}{2.455121in}}%
\pgfpathlineto{\pgfqpoint{0.739033in}{2.452593in}}%
\pgfpathlineto{\pgfqpoint{0.742254in}{2.451523in}}%
\pgfpathlineto{\pgfqpoint{0.745476in}{2.453069in}}%
\pgfpathlineto{\pgfqpoint{0.747623in}{2.458956in}}%
\pgfpathlineto{\pgfqpoint{0.748697in}{2.462702in}}%
\pgfpathlineto{\pgfqpoint{0.749771in}{2.461096in}}%
\pgfpathlineto{\pgfqpoint{0.752993in}{2.460888in}}%
\pgfpathlineto{\pgfqpoint{0.754067in}{2.458688in}}%
\pgfpathlineto{\pgfqpoint{0.755140in}{2.458331in}}%
\pgfpathlineto{\pgfqpoint{0.756214in}{2.459313in}}%
\pgfpathlineto{\pgfqpoint{0.757288in}{2.458718in}}%
\pgfpathlineto{\pgfqpoint{0.762657in}{2.458510in}}%
\pgfpathlineto{\pgfqpoint{0.763731in}{2.460204in}}%
\pgfpathlineto{\pgfqpoint{0.764805in}{2.458718in}}%
\pgfpathlineto{\pgfqpoint{0.768026in}{2.458450in}}%
\pgfpathlineto{\pgfqpoint{0.769100in}{2.458986in}}%
\pgfpathlineto{\pgfqpoint{0.771248in}{2.455656in}}%
\pgfpathlineto{\pgfqpoint{0.772322in}{2.456191in}}%
\pgfpathlineto{\pgfqpoint{0.776617in}{2.455299in}}%
\pgfpathlineto{\pgfqpoint{0.778765in}{2.454229in}}%
\pgfpathlineto{\pgfqpoint{0.779839in}{2.454526in}}%
\pgfpathlineto{\pgfqpoint{0.783060in}{2.454645in}}%
\pgfpathlineto{\pgfqpoint{0.784134in}{2.455685in}}%
\pgfpathlineto{\pgfqpoint{0.785208in}{2.455507in}}%
\pgfpathlineto{\pgfqpoint{0.787356in}{2.456934in}}%
\pgfpathlineto{\pgfqpoint{0.791651in}{2.455061in}}%
\pgfpathlineto{\pgfqpoint{0.792725in}{2.456280in}}%
\pgfpathlineto{\pgfqpoint{0.793799in}{2.456102in}}%
\pgfpathlineto{\pgfqpoint{0.794872in}{2.457350in}}%
\pgfpathlineto{\pgfqpoint{0.798094in}{2.457172in}}%
\pgfpathlineto{\pgfqpoint{0.802389in}{2.459134in}}%
\pgfpathlineto{\pgfqpoint{0.805611in}{2.459521in}}%
\pgfpathlineto{\pgfqpoint{0.806685in}{2.460769in}}%
\pgfpathlineto{\pgfqpoint{0.809906in}{2.459729in}}%
\pgfpathlineto{\pgfqpoint{0.814201in}{2.460383in}}%
\pgfpathlineto{\pgfqpoint{0.815275in}{2.459580in}}%
\pgfpathlineto{\pgfqpoint{0.816349in}{2.459877in}}%
\pgfpathlineto{\pgfqpoint{0.817423in}{2.458688in}}%
\pgfpathlineto{\pgfqpoint{0.820645in}{2.458064in}}%
\pgfpathlineto{\pgfqpoint{0.823866in}{2.458242in}}%
\pgfpathlineto{\pgfqpoint{0.824940in}{2.456726in}}%
\pgfpathlineto{\pgfqpoint{0.828161in}{2.458272in}}%
\pgfpathlineto{\pgfqpoint{0.830309in}{2.460115in}}%
\pgfpathlineto{\pgfqpoint{0.831383in}{2.459877in}}%
\pgfpathlineto{\pgfqpoint{0.832457in}{2.461869in}}%
\pgfpathlineto{\pgfqpoint{0.836752in}{2.461542in}}%
\pgfpathlineto{\pgfqpoint{0.839974in}{2.463861in}}%
\pgfpathlineto{\pgfqpoint{0.843195in}{2.464396in}}%
\pgfpathlineto{\pgfqpoint{0.844269in}{2.463921in}}%
\pgfpathlineto{\pgfqpoint{0.846417in}{2.465229in}}%
\pgfpathlineto{\pgfqpoint{0.847490in}{2.466061in}}%
\pgfpathlineto{\pgfqpoint{0.851786in}{2.464694in}}%
\pgfpathlineto{\pgfqpoint{0.853934in}{2.465734in}}%
\pgfpathlineto{\pgfqpoint{0.855007in}{2.467756in}}%
\pgfpathlineto{\pgfqpoint{0.858229in}{2.467013in}}%
\pgfpathlineto{\pgfqpoint{0.859303in}{2.468975in}}%
\pgfpathlineto{\pgfqpoint{0.861450in}{2.467756in}}%
\pgfpathlineto{\pgfqpoint{0.862524in}{2.468410in}}%
\pgfpathlineto{\pgfqpoint{0.865746in}{2.468232in}}%
\pgfpathlineto{\pgfqpoint{0.867893in}{2.471413in}}%
\pgfpathlineto{\pgfqpoint{0.868967in}{2.470580in}}%
\pgfpathlineto{\pgfqpoint{0.870041in}{2.471621in}}%
\pgfpathlineto{\pgfqpoint{0.873263in}{2.471472in}}%
\pgfpathlineto{\pgfqpoint{0.874336in}{2.472424in}}%
\pgfpathlineto{\pgfqpoint{0.875410in}{2.472067in}}%
\pgfpathlineto{\pgfqpoint{0.877558in}{2.473316in}}%
\pgfpathlineto{\pgfqpoint{0.881853in}{2.475308in}}%
\pgfpathlineto{\pgfqpoint{0.882927in}{2.474535in}}%
\pgfpathlineto{\pgfqpoint{0.885075in}{2.475070in}}%
\pgfpathlineto{\pgfqpoint{0.888296in}{2.473345in}}%
\pgfpathlineto{\pgfqpoint{0.889370in}{2.473821in}}%
\pgfpathlineto{\pgfqpoint{0.890444in}{2.475248in}}%
\pgfpathlineto{\pgfqpoint{0.891518in}{2.474713in}}%
\pgfpathlineto{\pgfqpoint{0.892592in}{2.476200in}}%
\pgfpathlineto{\pgfqpoint{0.895813in}{2.477448in}}%
\pgfpathlineto{\pgfqpoint{0.896887in}{2.478578in}}%
\pgfpathlineto{\pgfqpoint{0.897961in}{2.477924in}}%
\pgfpathlineto{\pgfqpoint{0.900109in}{2.479916in}}%
\pgfpathlineto{\pgfqpoint{0.905478in}{2.480808in}}%
\pgfpathlineto{\pgfqpoint{0.906552in}{2.482205in}}%
\pgfpathlineto{\pgfqpoint{0.907625in}{2.482413in}}%
\pgfpathlineto{\pgfqpoint{0.911921in}{2.481581in}}%
\pgfpathlineto{\pgfqpoint{0.912995in}{2.483067in}}%
\pgfpathlineto{\pgfqpoint{0.914068in}{2.481967in}}%
\pgfpathlineto{\pgfqpoint{0.915142in}{2.483781in}}%
\pgfpathlineto{\pgfqpoint{0.918364in}{2.483632in}}%
\pgfpathlineto{\pgfqpoint{0.919438in}{2.486575in}}%
\pgfpathlineto{\pgfqpoint{0.921585in}{2.488270in}}%
\pgfpathlineto{\pgfqpoint{0.925881in}{2.489281in}}%
\pgfpathlineto{\pgfqpoint{0.926955in}{2.491184in}}%
\pgfpathlineto{\pgfqpoint{0.928028in}{2.489638in}}%
\pgfpathlineto{\pgfqpoint{0.929102in}{2.490470in}}%
\pgfpathlineto{\pgfqpoint{0.933398in}{2.487230in}}%
\pgfpathlineto{\pgfqpoint{0.937693in}{2.491303in}}%
\pgfpathlineto{\pgfqpoint{0.940914in}{2.488716in}}%
\pgfpathlineto{\pgfqpoint{0.941988in}{2.493057in}}%
\pgfpathlineto{\pgfqpoint{0.943062in}{2.494216in}}%
\pgfpathlineto{\pgfqpoint{0.944136in}{2.492403in}}%
\pgfpathlineto{\pgfqpoint{0.945210in}{2.495673in}}%
\pgfpathlineto{\pgfqpoint{0.948431in}{2.496535in}}%
\pgfpathlineto{\pgfqpoint{0.949505in}{2.498081in}}%
\pgfpathlineto{\pgfqpoint{0.950579in}{2.495435in}}%
\pgfpathlineto{\pgfqpoint{0.951653in}{2.497516in}}%
\pgfpathlineto{\pgfqpoint{0.952727in}{2.497249in}}%
\pgfpathlineto{\pgfqpoint{0.955948in}{2.498408in}}%
\pgfpathlineto{\pgfqpoint{0.957022in}{2.497546in}}%
\pgfpathlineto{\pgfqpoint{0.958096in}{2.495019in}}%
\pgfpathlineto{\pgfqpoint{0.960244in}{2.498824in}}%
\pgfpathlineto{\pgfqpoint{0.963465in}{2.496149in}}%
\pgfpathlineto{\pgfqpoint{0.964539in}{2.498289in}}%
\pgfpathlineto{\pgfqpoint{0.965613in}{2.498111in}}%
\pgfpathlineto{\pgfqpoint{0.966687in}{2.497338in}}%
\pgfpathlineto{\pgfqpoint{0.967760in}{2.498854in}}%
\pgfpathlineto{\pgfqpoint{0.970982in}{2.499092in}}%
\pgfpathlineto{\pgfqpoint{0.973130in}{2.503552in}}%
\pgfpathlineto{\pgfqpoint{0.974203in}{2.503076in}}%
\pgfpathlineto{\pgfqpoint{0.975277in}{2.504681in}}%
\pgfpathlineto{\pgfqpoint{0.978499in}{2.504503in}}%
\pgfpathlineto{\pgfqpoint{0.979573in}{2.505930in}}%
\pgfpathlineto{\pgfqpoint{0.980646in}{2.505633in}}%
\pgfpathlineto{\pgfqpoint{0.982794in}{2.503135in}}%
\pgfpathlineto{\pgfqpoint{0.987089in}{2.505127in}}%
\pgfpathlineto{\pgfqpoint{0.988163in}{2.500192in}}%
\pgfpathlineto{\pgfqpoint{0.989237in}{2.501054in}}%
\pgfpathlineto{\pgfqpoint{0.990311in}{2.496476in}}%
\pgfpathlineto{\pgfqpoint{0.993533in}{2.497814in}}%
\pgfpathlineto{\pgfqpoint{0.995680in}{2.495227in}}%
\pgfpathlineto{\pgfqpoint{0.997828in}{2.498319in}}%
\pgfpathlineto{\pgfqpoint{1.001049in}{2.498914in}}%
\pgfpathlineto{\pgfqpoint{1.002123in}{2.497784in}}%
\pgfpathlineto{\pgfqpoint{1.003197in}{2.495376in}}%
\pgfpathlineto{\pgfqpoint{1.004271in}{2.498319in}}%
\pgfpathlineto{\pgfqpoint{1.005345in}{2.498319in}}%
\pgfpathlineto{\pgfqpoint{1.008566in}{2.500133in}}%
\pgfpathlineto{\pgfqpoint{1.009640in}{2.501976in}}%
\pgfpathlineto{\pgfqpoint{1.011788in}{2.492551in}}%
\pgfpathlineto{\pgfqpoint{1.017157in}{2.499449in}}%
\pgfpathlineto{\pgfqpoint{1.018231in}{2.503581in}}%
\pgfpathlineto{\pgfqpoint{1.019305in}{2.502868in}}%
\pgfpathlineto{\pgfqpoint{1.020378in}{2.500727in}}%
\pgfpathlineto{\pgfqpoint{1.023600in}{2.502660in}}%
\pgfpathlineto{\pgfqpoint{1.024674in}{2.502511in}}%
\pgfpathlineto{\pgfqpoint{1.025748in}{2.503046in}}%
\pgfpathlineto{\pgfqpoint{1.027895in}{2.505781in}}%
\pgfpathlineto{\pgfqpoint{1.033265in}{2.509260in}}%
\pgfpathlineto{\pgfqpoint{1.035412in}{2.511133in}}%
\pgfpathlineto{\pgfqpoint{1.039708in}{2.512173in}}%
\pgfpathlineto{\pgfqpoint{1.040781in}{2.511490in}}%
\pgfpathlineto{\pgfqpoint{1.041855in}{2.511609in}}%
\pgfpathlineto{\pgfqpoint{1.042929in}{2.516782in}}%
\pgfpathlineto{\pgfqpoint{1.048298in}{2.517109in}}%
\pgfpathlineto{\pgfqpoint{1.053667in}{2.519249in}}%
\pgfpathlineto{\pgfqpoint{1.054741in}{2.519160in}}%
\pgfpathlineto{\pgfqpoint{1.057963in}{2.522222in}}%
\pgfpathlineto{\pgfqpoint{1.062258in}{2.520766in}}%
\pgfpathlineto{\pgfqpoint{1.064406in}{2.519576in}}%
\pgfpathlineto{\pgfqpoint{1.065480in}{2.517109in}}%
\pgfpathlineto{\pgfqpoint{1.068701in}{2.516217in}}%
\pgfpathlineto{\pgfqpoint{1.069775in}{2.518744in}}%
\pgfpathlineto{\pgfqpoint{1.071923in}{2.510033in}}%
\pgfpathlineto{\pgfqpoint{1.072997in}{2.509587in}}%
\pgfpathlineto{\pgfqpoint{1.076218in}{2.512292in}}%
\pgfpathlineto{\pgfqpoint{1.079440in}{2.506792in}}%
\pgfpathlineto{\pgfqpoint{1.080513in}{2.508844in}}%
\pgfpathlineto{\pgfqpoint{1.083735in}{2.506614in}}%
\pgfpathlineto{\pgfqpoint{1.084809in}{2.503135in}}%
\pgfpathlineto{\pgfqpoint{1.085883in}{2.504057in}}%
\pgfpathlineto{\pgfqpoint{1.088030in}{2.503760in}}%
\pgfpathlineto{\pgfqpoint{1.092326in}{2.503789in}}%
\pgfpathlineto{\pgfqpoint{1.093399in}{2.505008in}}%
\pgfpathlineto{\pgfqpoint{1.098769in}{2.506673in}}%
\pgfpathlineto{\pgfqpoint{1.100916in}{2.510925in}}%
\pgfpathlineto{\pgfqpoint{1.103064in}{2.509230in}}%
\pgfpathlineto{\pgfqpoint{1.107359in}{2.510479in}}%
\pgfpathlineto{\pgfqpoint{1.108433in}{2.512649in}}%
\pgfpathlineto{\pgfqpoint{1.109507in}{2.513065in}}%
\pgfpathlineto{\pgfqpoint{1.110581in}{2.512055in}}%
\pgfpathlineto{\pgfqpoint{1.113802in}{2.510568in}}%
\pgfpathlineto{\pgfqpoint{1.115950in}{2.505454in}}%
\pgfpathlineto{\pgfqpoint{1.117024in}{2.505425in}}%
\pgfpathlineto{\pgfqpoint{1.118098in}{2.504562in}}%
\pgfpathlineto{\pgfqpoint{1.121319in}{2.504473in}}%
\pgfpathlineto{\pgfqpoint{1.122393in}{2.506435in}}%
\pgfpathlineto{\pgfqpoint{1.123467in}{2.505990in}}%
\pgfpathlineto{\pgfqpoint{1.124541in}{2.504176in}}%
\pgfpathlineto{\pgfqpoint{1.125615in}{2.506049in}}%
\pgfpathlineto{\pgfqpoint{1.128836in}{2.504206in}}%
\pgfpathlineto{\pgfqpoint{1.129910in}{2.501708in}}%
\pgfpathlineto{\pgfqpoint{1.130984in}{2.502600in}}%
\pgfpathlineto{\pgfqpoint{1.133132in}{2.511490in}}%
\pgfpathlineto{\pgfqpoint{1.137427in}{2.512709in}}%
\pgfpathlineto{\pgfqpoint{1.139575in}{2.517882in}}%
\pgfpathlineto{\pgfqpoint{1.140648in}{2.517020in}}%
\pgfpathlineto{\pgfqpoint{1.143870in}{2.515920in}}%
\pgfpathlineto{\pgfqpoint{1.144944in}{2.518863in}}%
\pgfpathlineto{\pgfqpoint{1.146018in}{2.518209in}}%
\pgfpathlineto{\pgfqpoint{1.147091in}{2.518833in}}%
\pgfpathlineto{\pgfqpoint{1.148165in}{2.518179in}}%
\pgfpathlineto{\pgfqpoint{1.151387in}{2.518952in}}%
\pgfpathlineto{\pgfqpoint{1.152461in}{2.520855in}}%
\pgfpathlineto{\pgfqpoint{1.154608in}{2.519517in}}%
\pgfpathlineto{\pgfqpoint{1.155682in}{2.521449in}}%
\pgfpathlineto{\pgfqpoint{1.162125in}{2.519695in}}%
\pgfpathlineto{\pgfqpoint{1.163199in}{2.523174in}}%
\pgfpathlineto{\pgfqpoint{1.166421in}{2.523768in}}%
\pgfpathlineto{\pgfqpoint{1.167494in}{2.521925in}}%
\pgfpathlineto{\pgfqpoint{1.168568in}{2.521360in}}%
\pgfpathlineto{\pgfqpoint{1.170716in}{2.524036in}}%
\pgfpathlineto{\pgfqpoint{1.173937in}{2.523798in}}%
\pgfpathlineto{\pgfqpoint{1.176085in}{2.525968in}}%
\pgfpathlineto{\pgfqpoint{1.177159in}{2.526087in}}%
\pgfpathlineto{\pgfqpoint{1.178233in}{2.527901in}}%
\pgfpathlineto{\pgfqpoint{1.181454in}{2.528882in}}%
\pgfpathlineto{\pgfqpoint{1.182528in}{2.527425in}}%
\pgfpathlineto{\pgfqpoint{1.185750in}{2.526385in}}%
\pgfpathlineto{\pgfqpoint{1.190045in}{2.524631in}}%
\pgfpathlineto{\pgfqpoint{1.191119in}{2.523768in}}%
\pgfpathlineto{\pgfqpoint{1.192193in}{2.522074in}}%
\pgfpathlineto{\pgfqpoint{1.193266in}{2.525820in}}%
\pgfpathlineto{\pgfqpoint{1.196488in}{2.525820in}}%
\pgfpathlineto{\pgfqpoint{1.197562in}{2.525077in}}%
\pgfpathlineto{\pgfqpoint{1.198636in}{2.522431in}}%
\pgfpathlineto{\pgfqpoint{1.199710in}{2.517436in}}%
\pgfpathlineto{\pgfqpoint{1.200783in}{2.517911in}}%
\pgfpathlineto{\pgfqpoint{1.204005in}{2.517971in}}%
\pgfpathlineto{\pgfqpoint{1.205079in}{2.516157in}}%
\pgfpathlineto{\pgfqpoint{1.206153in}{2.521212in}}%
\pgfpathlineto{\pgfqpoint{1.207226in}{2.519547in}}%
\pgfpathlineto{\pgfqpoint{1.208300in}{2.519814in}}%
\pgfpathlineto{\pgfqpoint{1.212596in}{2.519755in}}%
\pgfpathlineto{\pgfqpoint{1.214743in}{2.520974in}}%
\pgfpathlineto{\pgfqpoint{1.215817in}{2.520498in}}%
\pgfpathlineto{\pgfqpoint{1.219039in}{2.520349in}}%
\pgfpathlineto{\pgfqpoint{1.220112in}{2.518536in}}%
\pgfpathlineto{\pgfqpoint{1.222260in}{2.517109in}}%
\pgfpathlineto{\pgfqpoint{1.223334in}{2.519190in}}%
\pgfpathlineto{\pgfqpoint{1.226555in}{2.520439in}}%
\pgfpathlineto{\pgfqpoint{1.227629in}{2.525433in}}%
\pgfpathlineto{\pgfqpoint{1.228703in}{2.525106in}}%
\pgfpathlineto{\pgfqpoint{1.229777in}{2.526563in}}%
\pgfpathlineto{\pgfqpoint{1.230851in}{2.526593in}}%
\pgfpathlineto{\pgfqpoint{1.234072in}{2.525998in}}%
\pgfpathlineto{\pgfqpoint{1.236220in}{2.526741in}}%
\pgfpathlineto{\pgfqpoint{1.237294in}{2.526355in}}%
\pgfpathlineto{\pgfqpoint{1.238368in}{2.527425in}}%
\pgfpathlineto{\pgfqpoint{1.242663in}{2.524779in}}%
\pgfpathlineto{\pgfqpoint{1.243737in}{2.525523in}}%
\pgfpathlineto{\pgfqpoint{1.245885in}{2.516038in}}%
\pgfpathlineto{\pgfqpoint{1.249106in}{2.514314in}}%
\pgfpathlineto{\pgfqpoint{1.250180in}{2.514730in}}%
\pgfpathlineto{\pgfqpoint{1.251254in}{2.511668in}}%
\pgfpathlineto{\pgfqpoint{1.252328in}{2.513184in}}%
\pgfpathlineto{\pgfqpoint{1.253401in}{2.510568in}}%
\pgfpathlineto{\pgfqpoint{1.256623in}{2.506227in}}%
\pgfpathlineto{\pgfqpoint{1.257697in}{2.505811in}}%
\pgfpathlineto{\pgfqpoint{1.258771in}{2.507506in}}%
\pgfpathlineto{\pgfqpoint{1.260918in}{2.514582in}}%
\pgfpathlineto{\pgfqpoint{1.264140in}{2.517198in}}%
\pgfpathlineto{\pgfqpoint{1.265214in}{2.522074in}}%
\pgfpathlineto{\pgfqpoint{1.266287in}{2.520676in}}%
\pgfpathlineto{\pgfqpoint{1.268435in}{2.521539in}}%
\pgfpathlineto{\pgfqpoint{1.272731in}{2.520022in}}%
\pgfpathlineto{\pgfqpoint{1.273804in}{2.518655in}}%
\pgfpathlineto{\pgfqpoint{1.274878in}{2.520558in}}%
\pgfpathlineto{\pgfqpoint{1.279174in}{2.518982in}}%
\pgfpathlineto{\pgfqpoint{1.281321in}{2.518982in}}%
\pgfpathlineto{\pgfqpoint{1.282395in}{2.519636in}}%
\pgfpathlineto{\pgfqpoint{1.283469in}{2.521598in}}%
\pgfpathlineto{\pgfqpoint{1.286690in}{2.520141in}}%
\pgfpathlineto{\pgfqpoint{1.287764in}{2.524720in}}%
\pgfpathlineto{\pgfqpoint{1.288838in}{2.522787in}}%
\pgfpathlineto{\pgfqpoint{1.290986in}{2.524690in}}%
\pgfpathlineto{\pgfqpoint{1.296355in}{2.525404in}}%
\pgfpathlineto{\pgfqpoint{1.297429in}{2.523858in}}%
\pgfpathlineto{\pgfqpoint{1.298503in}{2.523382in}}%
\pgfpathlineto{\pgfqpoint{1.301724in}{2.526266in}}%
\pgfpathlineto{\pgfqpoint{1.302798in}{2.526266in}}%
\pgfpathlineto{\pgfqpoint{1.303872in}{2.525374in}}%
\pgfpathlineto{\pgfqpoint{1.304946in}{2.526741in}}%
\pgfpathlineto{\pgfqpoint{1.306020in}{2.531409in}}%
\pgfpathlineto{\pgfqpoint{1.309241in}{2.529536in}}%
\pgfpathlineto{\pgfqpoint{1.310315in}{2.535155in}}%
\pgfpathlineto{\pgfqpoint{1.311389in}{2.534293in}}%
\pgfpathlineto{\pgfqpoint{1.316758in}{2.537326in}}%
\pgfpathlineto{\pgfqpoint{1.317832in}{2.536582in}}%
\pgfpathlineto{\pgfqpoint{1.319979in}{2.537415in}}%
\pgfpathlineto{\pgfqpoint{1.321053in}{2.537831in}}%
\pgfpathlineto{\pgfqpoint{1.324275in}{2.536463in}}%
\pgfpathlineto{\pgfqpoint{1.325349in}{2.536909in}}%
\pgfpathlineto{\pgfqpoint{1.326422in}{2.539199in}}%
\pgfpathlineto{\pgfqpoint{1.327496in}{2.532985in}}%
\pgfpathlineto{\pgfqpoint{1.328570in}{2.533817in}}%
\pgfpathlineto{\pgfqpoint{1.331792in}{2.534531in}}%
\pgfpathlineto{\pgfqpoint{1.332865in}{2.539823in}}%
\pgfpathlineto{\pgfqpoint{1.333939in}{2.538663in}}%
\pgfpathlineto{\pgfqpoint{1.341456in}{2.542469in}}%
\pgfpathlineto{\pgfqpoint{1.343604in}{2.541339in}}%
\pgfpathlineto{\pgfqpoint{1.346825in}{2.545353in}}%
\pgfpathlineto{\pgfqpoint{1.347899in}{2.544550in}}%
\pgfpathlineto{\pgfqpoint{1.348973in}{2.545204in}}%
\pgfpathlineto{\pgfqpoint{1.350047in}{2.543272in}}%
\pgfpathlineto{\pgfqpoint{1.351121in}{2.540120in}}%
\pgfpathlineto{\pgfqpoint{1.354342in}{2.541904in}}%
\pgfpathlineto{\pgfqpoint{1.355416in}{2.540626in}}%
\pgfpathlineto{\pgfqpoint{1.356490in}{2.544253in}}%
\pgfpathlineto{\pgfqpoint{1.357564in}{2.543182in}}%
\pgfpathlineto{\pgfqpoint{1.358638in}{2.544253in}}%
\pgfpathlineto{\pgfqpoint{1.361859in}{2.543242in}}%
\pgfpathlineto{\pgfqpoint{1.362933in}{2.544520in}}%
\pgfpathlineto{\pgfqpoint{1.366154in}{2.543391in}}%
\pgfpathlineto{\pgfqpoint{1.369376in}{2.543539in}}%
\pgfpathlineto{\pgfqpoint{1.370450in}{2.542528in}}%
\pgfpathlineto{\pgfqpoint{1.372598in}{2.546185in}}%
\pgfpathlineto{\pgfqpoint{1.373671in}{2.546245in}}%
\pgfpathlineto{\pgfqpoint{1.377967in}{2.545799in}}%
\pgfpathlineto{\pgfqpoint{1.379041in}{2.544461in}}%
\pgfpathlineto{\pgfqpoint{1.381188in}{2.547493in}}%
\pgfpathlineto{\pgfqpoint{1.386557in}{2.550645in}}%
\pgfpathlineto{\pgfqpoint{1.387631in}{2.552042in}}%
\pgfpathlineto{\pgfqpoint{1.391927in}{2.552042in}}%
\pgfpathlineto{\pgfqpoint{1.393000in}{2.554331in}}%
\pgfpathlineto{\pgfqpoint{1.395148in}{2.550258in}}%
\pgfpathlineto{\pgfqpoint{1.399443in}{2.550050in}}%
\pgfpathlineto{\pgfqpoint{1.400517in}{2.548712in}}%
\pgfpathlineto{\pgfqpoint{1.402665in}{2.553588in}}%
\pgfpathlineto{\pgfqpoint{1.403739in}{2.557364in}}%
\pgfpathlineto{\pgfqpoint{1.408034in}{2.555610in}}%
\pgfpathlineto{\pgfqpoint{1.409108in}{2.558642in}}%
\pgfpathlineto{\pgfqpoint{1.410182in}{2.558345in}}%
\pgfpathlineto{\pgfqpoint{1.411256in}{2.556650in}}%
\pgfpathlineto{\pgfqpoint{1.414477in}{2.555669in}}%
\pgfpathlineto{\pgfqpoint{1.415551in}{2.558910in}}%
\pgfpathlineto{\pgfqpoint{1.416625in}{2.558880in}}%
\pgfpathlineto{\pgfqpoint{1.417699in}{2.557750in}}%
\pgfpathlineto{\pgfqpoint{1.421994in}{2.560486in}}%
\pgfpathlineto{\pgfqpoint{1.423068in}{2.558523in}}%
\pgfpathlineto{\pgfqpoint{1.424142in}{2.559356in}}%
\pgfpathlineto{\pgfqpoint{1.425216in}{2.558732in}}%
\pgfpathlineto{\pgfqpoint{1.426289in}{2.556918in}}%
\pgfpathlineto{\pgfqpoint{1.429511in}{2.557661in}}%
\pgfpathlineto{\pgfqpoint{1.431659in}{2.549456in}}%
\pgfpathlineto{\pgfqpoint{1.432732in}{2.544639in}}%
\pgfpathlineto{\pgfqpoint{1.433806in}{2.548356in}}%
\pgfpathlineto{\pgfqpoint{1.437028in}{2.546988in}}%
\pgfpathlineto{\pgfqpoint{1.438102in}{2.550110in}}%
\pgfpathlineto{\pgfqpoint{1.439175in}{2.549366in}}%
\pgfpathlineto{\pgfqpoint{1.444545in}{2.549158in}}%
\pgfpathlineto{\pgfqpoint{1.445619in}{2.548772in}}%
\pgfpathlineto{\pgfqpoint{1.446692in}{2.549664in}}%
\pgfpathlineto{\pgfqpoint{1.447766in}{2.543926in}}%
\pgfpathlineto{\pgfqpoint{1.448840in}{2.543420in}}%
\pgfpathlineto{\pgfqpoint{1.452062in}{2.544104in}}%
\pgfpathlineto{\pgfqpoint{1.453135in}{2.543212in}}%
\pgfpathlineto{\pgfqpoint{1.454209in}{2.545531in}}%
\pgfpathlineto{\pgfqpoint{1.455283in}{2.543510in}}%
\pgfpathlineto{\pgfqpoint{1.456357in}{2.546483in}}%
\pgfpathlineto{\pgfqpoint{1.459578in}{2.546691in}}%
\pgfpathlineto{\pgfqpoint{1.460652in}{2.545323in}}%
\pgfpathlineto{\pgfqpoint{1.461726in}{2.548207in}}%
\pgfpathlineto{\pgfqpoint{1.462800in}{2.548920in}}%
\pgfpathlineto{\pgfqpoint{1.463874in}{2.546720in}}%
\pgfpathlineto{\pgfqpoint{1.468169in}{2.551358in}}%
\pgfpathlineto{\pgfqpoint{1.469243in}{2.552012in}}%
\pgfpathlineto{\pgfqpoint{1.470317in}{2.554569in}}%
\pgfpathlineto{\pgfqpoint{1.471391in}{2.553529in}}%
\pgfpathlineto{\pgfqpoint{1.476760in}{2.553856in}}%
\pgfpathlineto{\pgfqpoint{1.477834in}{2.553142in}}%
\pgfpathlineto{\pgfqpoint{1.478908in}{2.555194in}}%
\pgfpathlineto{\pgfqpoint{1.483203in}{2.554213in}}%
\pgfpathlineto{\pgfqpoint{1.484277in}{2.555253in}}%
\pgfpathlineto{\pgfqpoint{1.485351in}{2.555491in}}%
\pgfpathlineto{\pgfqpoint{1.486424in}{2.556977in}}%
\pgfpathlineto{\pgfqpoint{1.490720in}{2.555372in}}%
\pgfpathlineto{\pgfqpoint{1.491794in}{2.558494in}}%
\pgfpathlineto{\pgfqpoint{1.492867in}{2.557334in}}%
\pgfpathlineto{\pgfqpoint{1.497163in}{2.557780in}}%
\pgfpathlineto{\pgfqpoint{1.498237in}{2.560813in}}%
\pgfpathlineto{\pgfqpoint{1.499310in}{2.561615in}}%
\pgfpathlineto{\pgfqpoint{1.501458in}{2.566313in}}%
\pgfpathlineto{\pgfqpoint{1.504680in}{2.566016in}}%
\pgfpathlineto{\pgfqpoint{1.505753in}{2.564945in}}%
\pgfpathlineto{\pgfqpoint{1.506827in}{2.568007in}}%
\pgfpathlineto{\pgfqpoint{1.507901in}{2.563994in}}%
\pgfpathlineto{\pgfqpoint{1.508975in}{2.563994in}}%
\pgfpathlineto{\pgfqpoint{1.513270in}{2.562656in}}%
\pgfpathlineto{\pgfqpoint{1.514344in}{2.556680in}}%
\pgfpathlineto{\pgfqpoint{1.515418in}{2.555491in}}%
\pgfpathlineto{\pgfqpoint{1.516492in}{2.558850in}}%
\pgfpathlineto{\pgfqpoint{1.519713in}{2.558137in}}%
\pgfpathlineto{\pgfqpoint{1.520787in}{2.551685in}}%
\pgfpathlineto{\pgfqpoint{1.521861in}{2.558286in}}%
\pgfpathlineto{\pgfqpoint{1.522935in}{2.550883in}}%
\pgfpathlineto{\pgfqpoint{1.527230in}{2.543153in}}%
\pgfpathlineto{\pgfqpoint{1.528304in}{2.537623in}}%
\pgfpathlineto{\pgfqpoint{1.529378in}{2.540774in}}%
\pgfpathlineto{\pgfqpoint{1.530452in}{2.537028in}}%
\pgfpathlineto{\pgfqpoint{1.531526in}{2.542053in}}%
\pgfpathlineto{\pgfqpoint{1.534747in}{2.543361in}}%
\pgfpathlineto{\pgfqpoint{1.537969in}{2.552310in}}%
\pgfpathlineto{\pgfqpoint{1.539042in}{2.553618in}}%
\pgfpathlineto{\pgfqpoint{1.542264in}{2.556086in}}%
\pgfpathlineto{\pgfqpoint{1.544412in}{2.559980in}}%
\pgfpathlineto{\pgfqpoint{1.546559in}{2.565778in}}%
\pgfpathlineto{\pgfqpoint{1.549781in}{2.564945in}}%
\pgfpathlineto{\pgfqpoint{1.550855in}{2.567948in}}%
\pgfpathlineto{\pgfqpoint{1.553002in}{2.568989in}}%
\pgfpathlineto{\pgfqpoint{1.554076in}{2.566878in}}%
\pgfpathlineto{\pgfqpoint{1.558372in}{2.568721in}}%
\pgfpathlineto{\pgfqpoint{1.559445in}{2.568305in}}%
\pgfpathlineto{\pgfqpoint{1.560519in}{2.569137in}}%
\pgfpathlineto{\pgfqpoint{1.561593in}{2.566759in}}%
\pgfpathlineto{\pgfqpoint{1.564815in}{2.567116in}}%
\pgfpathlineto{\pgfqpoint{1.565888in}{2.568513in}}%
\pgfpathlineto{\pgfqpoint{1.566962in}{2.568335in}}%
\pgfpathlineto{\pgfqpoint{1.568036in}{2.566818in}}%
\pgfpathlineto{\pgfqpoint{1.569110in}{2.567799in}}%
\pgfpathlineto{\pgfqpoint{1.573405in}{2.564767in}}%
\pgfpathlineto{\pgfqpoint{1.576627in}{2.568840in}}%
\pgfpathlineto{\pgfqpoint{1.579848in}{2.568245in}}%
\pgfpathlineto{\pgfqpoint{1.580922in}{2.569524in}}%
\pgfpathlineto{\pgfqpoint{1.581996in}{2.567443in}}%
\pgfpathlineto{\pgfqpoint{1.583070in}{2.567026in}}%
\pgfpathlineto{\pgfqpoint{1.584144in}{2.569524in}}%
\pgfpathlineto{\pgfqpoint{1.587365in}{2.569553in}}%
\pgfpathlineto{\pgfqpoint{1.588439in}{2.568305in}}%
\pgfpathlineto{\pgfqpoint{1.589513in}{2.563548in}}%
\pgfpathlineto{\pgfqpoint{1.590587in}{2.564826in}}%
\pgfpathlineto{\pgfqpoint{1.591661in}{2.558791in}}%
\pgfpathlineto{\pgfqpoint{1.594882in}{2.557572in}}%
\pgfpathlineto{\pgfqpoint{1.595956in}{2.554391in}}%
\pgfpathlineto{\pgfqpoint{1.597030in}{2.557840in}}%
\pgfpathlineto{\pgfqpoint{1.598104in}{2.565064in}}%
\pgfpathlineto{\pgfqpoint{1.599177in}{2.561734in}}%
\pgfpathlineto{\pgfqpoint{1.602399in}{2.564856in}}%
\pgfpathlineto{\pgfqpoint{1.603473in}{2.558405in}}%
\pgfpathlineto{\pgfqpoint{1.606694in}{2.560456in}}%
\pgfpathlineto{\pgfqpoint{1.610990in}{2.561229in}}%
\pgfpathlineto{\pgfqpoint{1.612063in}{2.559148in}}%
\pgfpathlineto{\pgfqpoint{1.614211in}{2.559029in}}%
\pgfpathlineto{\pgfqpoint{1.617433in}{2.557126in}}%
\pgfpathlineto{\pgfqpoint{1.618507in}{2.555758in}}%
\pgfpathlineto{\pgfqpoint{1.619580in}{2.561764in}}%
\pgfpathlineto{\pgfqpoint{1.620654in}{2.563934in}}%
\pgfpathlineto{\pgfqpoint{1.621728in}{2.560129in}}%
\pgfpathlineto{\pgfqpoint{1.624950in}{2.559178in}}%
\pgfpathlineto{\pgfqpoint{1.626023in}{2.559653in}}%
\pgfpathlineto{\pgfqpoint{1.627097in}{2.557661in}}%
\pgfpathlineto{\pgfqpoint{1.628171in}{2.553707in}}%
\pgfpathlineto{\pgfqpoint{1.629245in}{2.557780in}}%
\pgfpathlineto{\pgfqpoint{1.633540in}{2.550526in}}%
\pgfpathlineto{\pgfqpoint{1.634614in}{2.552131in}}%
\pgfpathlineto{\pgfqpoint{1.635688in}{2.557037in}}%
\pgfpathlineto{\pgfqpoint{1.636762in}{2.552934in}}%
\pgfpathlineto{\pgfqpoint{1.641057in}{2.552637in}}%
\pgfpathlineto{\pgfqpoint{1.642131in}{2.551031in}}%
\pgfpathlineto{\pgfqpoint{1.643205in}{2.553410in}}%
\pgfpathlineto{\pgfqpoint{1.644279in}{2.547523in}}%
\pgfpathlineto{\pgfqpoint{1.647500in}{2.549337in}}%
\pgfpathlineto{\pgfqpoint{1.648574in}{2.553618in}}%
\pgfpathlineto{\pgfqpoint{1.649648in}{2.550734in}}%
\pgfpathlineto{\pgfqpoint{1.650722in}{2.553618in}}%
\pgfpathlineto{\pgfqpoint{1.651796in}{2.550050in}}%
\pgfpathlineto{\pgfqpoint{1.655017in}{2.546572in}}%
\pgfpathlineto{\pgfqpoint{1.656091in}{2.548058in}}%
\pgfpathlineto{\pgfqpoint{1.657165in}{2.548147in}}%
\pgfpathlineto{\pgfqpoint{1.658239in}{2.543034in}}%
\pgfpathlineto{\pgfqpoint{1.659312in}{2.546156in}}%
\pgfpathlineto{\pgfqpoint{1.663608in}{2.548296in}}%
\pgfpathlineto{\pgfqpoint{1.664682in}{2.547047in}}%
\pgfpathlineto{\pgfqpoint{1.665755in}{2.548980in}}%
\pgfpathlineto{\pgfqpoint{1.666829in}{2.549664in}}%
\pgfpathlineto{\pgfqpoint{1.670051in}{2.549456in}}%
\pgfpathlineto{\pgfqpoint{1.672198in}{2.552191in}}%
\pgfpathlineto{\pgfqpoint{1.673272in}{2.556413in}}%
\pgfpathlineto{\pgfqpoint{1.674346in}{2.555640in}}%
\pgfpathlineto{\pgfqpoint{1.677568in}{2.557513in}}%
\pgfpathlineto{\pgfqpoint{1.679715in}{2.553350in}}%
\pgfpathlineto{\pgfqpoint{1.680789in}{2.555669in}}%
\pgfpathlineto{\pgfqpoint{1.681863in}{2.549277in}}%
\pgfpathlineto{\pgfqpoint{1.685085in}{2.550734in}}%
\pgfpathlineto{\pgfqpoint{1.687232in}{2.544550in}}%
\pgfpathlineto{\pgfqpoint{1.688306in}{2.548534in}}%
\pgfpathlineto{\pgfqpoint{1.689380in}{2.546899in}}%
\pgfpathlineto{\pgfqpoint{1.692601in}{2.551804in}}%
\pgfpathlineto{\pgfqpoint{1.693675in}{2.548683in}}%
\pgfpathlineto{\pgfqpoint{1.694749in}{2.552845in}}%
\pgfpathlineto{\pgfqpoint{1.695823in}{2.553469in}}%
\pgfpathlineto{\pgfqpoint{1.696897in}{2.555342in}}%
\pgfpathlineto{\pgfqpoint{1.700118in}{2.556888in}}%
\pgfpathlineto{\pgfqpoint{1.701192in}{2.554183in}}%
\pgfpathlineto{\pgfqpoint{1.702266in}{2.549902in}}%
\pgfpathlineto{\pgfqpoint{1.703340in}{2.549366in}}%
\pgfpathlineto{\pgfqpoint{1.704414in}{2.549902in}}%
\pgfpathlineto{\pgfqpoint{1.707635in}{2.553083in}}%
\pgfpathlineto{\pgfqpoint{1.709783in}{2.546750in}}%
\pgfpathlineto{\pgfqpoint{1.710857in}{2.548029in}}%
\pgfpathlineto{\pgfqpoint{1.715152in}{2.546750in}}%
\pgfpathlineto{\pgfqpoint{1.716226in}{2.549247in}}%
\pgfpathlineto{\pgfqpoint{1.717300in}{2.549396in}}%
\pgfpathlineto{\pgfqpoint{1.719447in}{2.554450in}}%
\pgfpathlineto{\pgfqpoint{1.722669in}{2.550437in}}%
\pgfpathlineto{\pgfqpoint{1.724817in}{2.550585in}}%
\pgfpathlineto{\pgfqpoint{1.725890in}{2.548445in}}%
\pgfpathlineto{\pgfqpoint{1.726964in}{2.547880in}}%
\pgfpathlineto{\pgfqpoint{1.732333in}{2.550139in}}%
\pgfpathlineto{\pgfqpoint{1.733407in}{2.550199in}}%
\pgfpathlineto{\pgfqpoint{1.734481in}{2.551834in}}%
\pgfpathlineto{\pgfqpoint{1.737703in}{2.550526in}}%
\pgfpathlineto{\pgfqpoint{1.738776in}{2.550942in}}%
\pgfpathlineto{\pgfqpoint{1.739850in}{2.550020in}}%
\pgfpathlineto{\pgfqpoint{1.740924in}{2.546869in}}%
\pgfpathlineto{\pgfqpoint{1.741998in}{2.549337in}}%
\pgfpathlineto{\pgfqpoint{1.745219in}{2.549902in}}%
\pgfpathlineto{\pgfqpoint{1.746293in}{2.547672in}}%
\pgfpathlineto{\pgfqpoint{1.747367in}{2.546780in}}%
\pgfpathlineto{\pgfqpoint{1.748441in}{2.548088in}}%
\pgfpathlineto{\pgfqpoint{1.749515in}{2.552875in}}%
\pgfpathlineto{\pgfqpoint{1.752736in}{2.551715in}}%
\pgfpathlineto{\pgfqpoint{1.753810in}{2.550229in}}%
\pgfpathlineto{\pgfqpoint{1.754884in}{2.550437in}}%
\pgfpathlineto{\pgfqpoint{1.755958in}{2.553826in}}%
\pgfpathlineto{\pgfqpoint{1.761327in}{2.559475in}}%
\pgfpathlineto{\pgfqpoint{1.763475in}{2.557364in}}%
\pgfpathlineto{\pgfqpoint{1.764549in}{2.554510in}}%
\pgfpathlineto{\pgfqpoint{1.768844in}{2.553112in}}%
\pgfpathlineto{\pgfqpoint{1.769918in}{2.553945in}}%
\pgfpathlineto{\pgfqpoint{1.770992in}{2.553975in}}%
\pgfpathlineto{\pgfqpoint{1.772065in}{2.551299in}}%
\pgfpathlineto{\pgfqpoint{1.777435in}{2.551150in}}%
\pgfpathlineto{\pgfqpoint{1.779582in}{2.547166in}}%
\pgfpathlineto{\pgfqpoint{1.782804in}{2.545472in}}%
\pgfpathlineto{\pgfqpoint{1.783878in}{2.546156in}}%
\pgfpathlineto{\pgfqpoint{1.786025in}{2.548891in}}%
\pgfpathlineto{\pgfqpoint{1.787099in}{2.546512in}}%
\pgfpathlineto{\pgfqpoint{1.790321in}{2.544223in}}%
\pgfpathlineto{\pgfqpoint{1.791395in}{2.546542in}}%
\pgfpathlineto{\pgfqpoint{1.792468in}{2.547523in}}%
\pgfpathlineto{\pgfqpoint{1.793542in}{2.551864in}}%
\pgfpathlineto{\pgfqpoint{1.794616in}{2.550556in}}%
\pgfpathlineto{\pgfqpoint{1.797838in}{2.551150in}}%
\pgfpathlineto{\pgfqpoint{1.801059in}{2.548564in}}%
\pgfpathlineto{\pgfqpoint{1.802133in}{2.549961in}}%
\pgfpathlineto{\pgfqpoint{1.805354in}{2.544728in}}%
\pgfpathlineto{\pgfqpoint{1.806428in}{2.544134in}}%
\pgfpathlineto{\pgfqpoint{1.807502in}{2.546839in}}%
\pgfpathlineto{\pgfqpoint{1.812871in}{2.546126in}}%
\pgfpathlineto{\pgfqpoint{1.813945in}{2.548029in}}%
\pgfpathlineto{\pgfqpoint{1.815019in}{2.544966in}}%
\pgfpathlineto{\pgfqpoint{1.816093in}{2.546661in}}%
\pgfpathlineto{\pgfqpoint{1.817167in}{2.549664in}}%
\pgfpathlineto{\pgfqpoint{1.820388in}{2.551626in}}%
\pgfpathlineto{\pgfqpoint{1.821462in}{2.550318in}}%
\pgfpathlineto{\pgfqpoint{1.823610in}{2.553885in}}%
\pgfpathlineto{\pgfqpoint{1.824684in}{2.551121in}}%
\pgfpathlineto{\pgfqpoint{1.828979in}{2.551834in}}%
\pgfpathlineto{\pgfqpoint{1.831127in}{2.551448in}}%
\pgfpathlineto{\pgfqpoint{1.832200in}{2.548653in}}%
\pgfpathlineto{\pgfqpoint{1.835422in}{2.546334in}}%
\pgfpathlineto{\pgfqpoint{1.837570in}{2.550199in}}%
\pgfpathlineto{\pgfqpoint{1.838643in}{2.550496in}}%
\pgfpathlineto{\pgfqpoint{1.839717in}{2.551477in}}%
\pgfpathlineto{\pgfqpoint{1.844013in}{2.550377in}}%
\pgfpathlineto{\pgfqpoint{1.845086in}{2.552310in}}%
\pgfpathlineto{\pgfqpoint{1.846160in}{2.548445in}}%
\pgfpathlineto{\pgfqpoint{1.847234in}{2.547850in}}%
\pgfpathlineto{\pgfqpoint{1.850456in}{2.550348in}}%
\pgfpathlineto{\pgfqpoint{1.851529in}{2.548266in}}%
\pgfpathlineto{\pgfqpoint{1.853677in}{2.546869in}}%
\pgfpathlineto{\pgfqpoint{1.857973in}{2.550585in}}%
\pgfpathlineto{\pgfqpoint{1.859046in}{2.549247in}}%
\pgfpathlineto{\pgfqpoint{1.860120in}{2.549069in}}%
\pgfpathlineto{\pgfqpoint{1.861194in}{2.547702in}}%
\pgfpathlineto{\pgfqpoint{1.862268in}{2.541012in}}%
\pgfpathlineto{\pgfqpoint{1.865489in}{2.533669in}}%
\pgfpathlineto{\pgfqpoint{1.866563in}{2.528050in}}%
\pgfpathlineto{\pgfqpoint{1.867637in}{2.539823in}}%
\pgfpathlineto{\pgfqpoint{1.868711in}{2.542796in}}%
\pgfpathlineto{\pgfqpoint{1.869785in}{2.539972in}}%
\pgfpathlineto{\pgfqpoint{1.873006in}{2.536790in}}%
\pgfpathlineto{\pgfqpoint{1.874080in}{2.531706in}}%
\pgfpathlineto{\pgfqpoint{1.875154in}{2.535096in}}%
\pgfpathlineto{\pgfqpoint{1.876228in}{2.533193in}}%
\pgfpathlineto{\pgfqpoint{1.877302in}{2.529596in}}%
\pgfpathlineto{\pgfqpoint{1.881597in}{2.536671in}}%
\pgfpathlineto{\pgfqpoint{1.882671in}{2.532063in}}%
\pgfpathlineto{\pgfqpoint{1.884818in}{2.533966in}}%
\pgfpathlineto{\pgfqpoint{1.888040in}{2.535007in}}%
\pgfpathlineto{\pgfqpoint{1.889114in}{2.537920in}}%
\pgfpathlineto{\pgfqpoint{1.891262in}{2.539080in}}%
\pgfpathlineto{\pgfqpoint{1.892335in}{2.535185in}}%
\pgfpathlineto{\pgfqpoint{1.897705in}{2.534115in}}%
\pgfpathlineto{\pgfqpoint{1.898778in}{2.532747in}}%
\pgfpathlineto{\pgfqpoint{1.899852in}{2.528763in}}%
\pgfpathlineto{\pgfqpoint{1.903074in}{2.529774in}}%
\pgfpathlineto{\pgfqpoint{1.904148in}{2.534263in}}%
\pgfpathlineto{\pgfqpoint{1.905221in}{2.535096in}}%
\pgfpathlineto{\pgfqpoint{1.906295in}{2.534590in}}%
\pgfpathlineto{\pgfqpoint{1.907369in}{2.536642in}}%
\pgfpathlineto{\pgfqpoint{1.910591in}{2.538872in}}%
\pgfpathlineto{\pgfqpoint{1.911664in}{2.535274in}}%
\pgfpathlineto{\pgfqpoint{1.912738in}{2.539466in}}%
\pgfpathlineto{\pgfqpoint{1.914886in}{2.540507in}}%
\pgfpathlineto{\pgfqpoint{1.918107in}{2.542172in}}%
\pgfpathlineto{\pgfqpoint{1.919181in}{2.540745in}}%
\pgfpathlineto{\pgfqpoint{1.920255in}{2.538247in}}%
\pgfpathlineto{\pgfqpoint{1.921329in}{2.545293in}}%
\pgfpathlineto{\pgfqpoint{1.922403in}{2.548237in}}%
\pgfpathlineto{\pgfqpoint{1.925624in}{2.547434in}}%
\pgfpathlineto{\pgfqpoint{1.926698in}{2.546483in}}%
\pgfpathlineto{\pgfqpoint{1.927772in}{2.546601in}}%
\pgfpathlineto{\pgfqpoint{1.928846in}{2.551685in}}%
\pgfpathlineto{\pgfqpoint{1.929920in}{2.553796in}}%
\pgfpathlineto{\pgfqpoint{1.933141in}{2.552785in}}%
\pgfpathlineto{\pgfqpoint{1.935289in}{2.554242in}}%
\pgfpathlineto{\pgfqpoint{1.936363in}{2.556650in}}%
\pgfpathlineto{\pgfqpoint{1.937437in}{2.555729in}}%
\pgfpathlineto{\pgfqpoint{1.940658in}{2.558791in}}%
\pgfpathlineto{\pgfqpoint{1.942806in}{2.558167in}}%
\pgfpathlineto{\pgfqpoint{1.943880in}{2.559207in}}%
\pgfpathlineto{\pgfqpoint{1.948175in}{2.555223in}}%
\pgfpathlineto{\pgfqpoint{1.950323in}{2.557959in}}%
\pgfpathlineto{\pgfqpoint{1.951396in}{2.553677in}}%
\pgfpathlineto{\pgfqpoint{1.952470in}{2.552637in}}%
\pgfpathlineto{\pgfqpoint{1.956766in}{2.556977in}}%
\pgfpathlineto{\pgfqpoint{1.957840in}{2.560129in}}%
\pgfpathlineto{\pgfqpoint{1.958913in}{2.559653in}}%
\pgfpathlineto{\pgfqpoint{1.959987in}{2.561645in}}%
\pgfpathlineto{\pgfqpoint{1.963209in}{2.562448in}}%
\pgfpathlineto{\pgfqpoint{1.964283in}{2.560456in}}%
\pgfpathlineto{\pgfqpoint{1.965356in}{2.560248in}}%
\pgfpathlineto{\pgfqpoint{1.967504in}{2.561348in}}%
\pgfpathlineto{\pgfqpoint{1.970726in}{2.558286in}}%
\pgfpathlineto{\pgfqpoint{1.971799in}{2.561318in}}%
\pgfpathlineto{\pgfqpoint{1.972873in}{2.560486in}}%
\pgfpathlineto{\pgfqpoint{1.973947in}{2.557067in}}%
\pgfpathlineto{\pgfqpoint{1.975021in}{2.562924in}}%
\pgfpathlineto{\pgfqpoint{1.978242in}{2.563934in}}%
\pgfpathlineto{\pgfqpoint{1.979316in}{2.561496in}}%
\pgfpathlineto{\pgfqpoint{1.980390in}{2.560753in}}%
\pgfpathlineto{\pgfqpoint{1.981464in}{2.562091in}}%
\pgfpathlineto{\pgfqpoint{1.982538in}{2.559475in}}%
\pgfpathlineto{\pgfqpoint{1.985759in}{2.560753in}}%
\pgfpathlineto{\pgfqpoint{1.987907in}{2.569137in}}%
\pgfpathlineto{\pgfqpoint{1.990055in}{2.560218in}}%
\pgfpathlineto{\pgfqpoint{1.993276in}{2.559207in}}%
\pgfpathlineto{\pgfqpoint{1.995424in}{2.564380in}}%
\pgfpathlineto{\pgfqpoint{1.996498in}{2.565005in}}%
\pgfpathlineto{\pgfqpoint{2.000793in}{2.563667in}}%
\pgfpathlineto{\pgfqpoint{2.001867in}{2.565837in}}%
\pgfpathlineto{\pgfqpoint{2.002941in}{2.565183in}}%
\pgfpathlineto{\pgfqpoint{2.004015in}{2.562299in}}%
\pgfpathlineto{\pgfqpoint{2.008310in}{2.556234in}}%
\pgfpathlineto{\pgfqpoint{2.009384in}{2.557364in}}%
\pgfpathlineto{\pgfqpoint{2.010458in}{2.555996in}}%
\pgfpathlineto{\pgfqpoint{2.012605in}{2.549961in}}%
\pgfpathlineto{\pgfqpoint{2.015827in}{2.548356in}}%
\pgfpathlineto{\pgfqpoint{2.016901in}{2.550169in}}%
\pgfpathlineto{\pgfqpoint{2.017974in}{2.546869in}}%
\pgfpathlineto{\pgfqpoint{2.019048in}{2.551923in}}%
\pgfpathlineto{\pgfqpoint{2.020122in}{2.546810in}}%
\pgfpathlineto{\pgfqpoint{2.024417in}{2.548177in}}%
\pgfpathlineto{\pgfqpoint{2.025491in}{2.543420in}}%
\pgfpathlineto{\pgfqpoint{2.026565in}{2.543955in}}%
\pgfpathlineto{\pgfqpoint{2.027639in}{2.546156in}}%
\pgfpathlineto{\pgfqpoint{2.030861in}{2.545204in}}%
\pgfpathlineto{\pgfqpoint{2.031934in}{2.558137in}}%
\pgfpathlineto{\pgfqpoint{2.033008in}{2.560783in}}%
\pgfpathlineto{\pgfqpoint{2.034082in}{2.561080in}}%
\pgfpathlineto{\pgfqpoint{2.035156in}{2.566967in}}%
\pgfpathlineto{\pgfqpoint{2.038377in}{2.566759in}}%
\pgfpathlineto{\pgfqpoint{2.039451in}{2.564172in}}%
\pgfpathlineto{\pgfqpoint{2.040525in}{2.566134in}}%
\pgfpathlineto{\pgfqpoint{2.041599in}{2.565510in}}%
\pgfpathlineto{\pgfqpoint{2.042673in}{2.556413in}}%
\pgfpathlineto{\pgfqpoint{2.045894in}{2.560367in}}%
\pgfpathlineto{\pgfqpoint{2.046968in}{2.560278in}}%
\pgfpathlineto{\pgfqpoint{2.049116in}{2.559534in}}%
\pgfpathlineto{\pgfqpoint{2.055559in}{2.561705in}}%
\pgfpathlineto{\pgfqpoint{2.056633in}{2.566432in}}%
\pgfpathlineto{\pgfqpoint{2.057706in}{2.568245in}}%
\pgfpathlineto{\pgfqpoint{2.060928in}{2.569851in}}%
\pgfpathlineto{\pgfqpoint{2.062002in}{2.568037in}}%
\pgfpathlineto{\pgfqpoint{2.063076in}{2.570416in}}%
\pgfpathlineto{\pgfqpoint{2.064150in}{2.574310in}}%
\pgfpathlineto{\pgfqpoint{2.065223in}{2.572675in}}%
\pgfpathlineto{\pgfqpoint{2.068445in}{2.571099in}}%
\pgfpathlineto{\pgfqpoint{2.069519in}{2.576600in}}%
\pgfpathlineto{\pgfqpoint{2.072740in}{2.574637in}}%
\pgfpathlineto{\pgfqpoint{2.075962in}{2.575291in}}%
\pgfpathlineto{\pgfqpoint{2.077036in}{2.573864in}}%
\pgfpathlineto{\pgfqpoint{2.080257in}{2.577937in}}%
\pgfpathlineto{\pgfqpoint{2.084552in}{2.578056in}}%
\pgfpathlineto{\pgfqpoint{2.085626in}{2.577105in}}%
\pgfpathlineto{\pgfqpoint{2.086700in}{2.575291in}}%
\pgfpathlineto{\pgfqpoint{2.087774in}{2.577343in}}%
\pgfpathlineto{\pgfqpoint{2.092069in}{2.577016in}}%
\pgfpathlineto{\pgfqpoint{2.093143in}{2.580078in}}%
\pgfpathlineto{\pgfqpoint{2.094217in}{2.579573in}}%
\pgfpathlineto{\pgfqpoint{2.098512in}{2.579335in}}%
\pgfpathlineto{\pgfqpoint{2.099586in}{2.581832in}}%
\pgfpathlineto{\pgfqpoint{2.100660in}{2.581386in}}%
\pgfpathlineto{\pgfqpoint{2.101734in}{2.579246in}}%
\pgfpathlineto{\pgfqpoint{2.102808in}{2.581951in}}%
\pgfpathlineto{\pgfqpoint{2.106029in}{2.580316in}}%
\pgfpathlineto{\pgfqpoint{2.108177in}{2.582575in}}%
\pgfpathlineto{\pgfqpoint{2.110325in}{2.581713in}}%
\pgfpathlineto{\pgfqpoint{2.113546in}{2.581357in}}%
\pgfpathlineto{\pgfqpoint{2.115694in}{2.583824in}}%
\pgfpathlineto{\pgfqpoint{2.116768in}{2.583735in}}%
\pgfpathlineto{\pgfqpoint{2.121063in}{2.586708in}}%
\pgfpathlineto{\pgfqpoint{2.123211in}{2.593933in}}%
\pgfpathlineto{\pgfqpoint{2.124284in}{2.593903in}}%
\pgfpathlineto{\pgfqpoint{2.125358in}{2.593219in}}%
\pgfpathlineto{\pgfqpoint{2.128580in}{2.593724in}}%
\pgfpathlineto{\pgfqpoint{2.129654in}{2.592089in}}%
\pgfpathlineto{\pgfqpoint{2.131801in}{2.590930in}}%
\pgfpathlineto{\pgfqpoint{2.132875in}{2.589830in}}%
\pgfpathlineto{\pgfqpoint{2.136097in}{2.591673in}}%
\pgfpathlineto{\pgfqpoint{2.137171in}{2.591495in}}%
\pgfpathlineto{\pgfqpoint{2.138244in}{2.590216in}}%
\pgfpathlineto{\pgfqpoint{2.139318in}{2.592030in}}%
\pgfpathlineto{\pgfqpoint{2.140392in}{2.591643in}}%
\pgfpathlineto{\pgfqpoint{2.143614in}{2.594319in}}%
\pgfpathlineto{\pgfqpoint{2.144687in}{2.596906in}}%
\pgfpathlineto{\pgfqpoint{2.147909in}{2.593873in}}%
\pgfpathlineto{\pgfqpoint{2.151130in}{2.596281in}}%
\pgfpathlineto{\pgfqpoint{2.154352in}{2.591941in}}%
\pgfpathlineto{\pgfqpoint{2.155426in}{2.593546in}}%
\pgfpathlineto{\pgfqpoint{2.158647in}{2.592238in}}%
\pgfpathlineto{\pgfqpoint{2.160795in}{2.595508in}}%
\pgfpathlineto{\pgfqpoint{2.161869in}{2.594260in}}%
\pgfpathlineto{\pgfqpoint{2.162943in}{2.594706in}}%
\pgfpathlineto{\pgfqpoint{2.168312in}{2.593933in}}%
\pgfpathlineto{\pgfqpoint{2.169386in}{2.598630in}}%
\pgfpathlineto{\pgfqpoint{2.174755in}{2.602049in}}%
\pgfpathlineto{\pgfqpoint{2.175829in}{2.602227in}}%
\pgfpathlineto{\pgfqpoint{2.176903in}{2.605527in}}%
\pgfpathlineto{\pgfqpoint{2.177976in}{2.605557in}}%
\pgfpathlineto{\pgfqpoint{2.181198in}{2.604963in}}%
\pgfpathlineto{\pgfqpoint{2.182272in}{2.605854in}}%
\pgfpathlineto{\pgfqpoint{2.183346in}{2.603922in}}%
\pgfpathlineto{\pgfqpoint{2.184419in}{2.604546in}}%
\pgfpathlineto{\pgfqpoint{2.185493in}{2.601365in}}%
\pgfpathlineto{\pgfqpoint{2.188715in}{2.604279in}}%
\pgfpathlineto{\pgfqpoint{2.189789in}{2.603268in}}%
\pgfpathlineto{\pgfqpoint{2.190862in}{2.604041in}}%
\pgfpathlineto{\pgfqpoint{2.191936in}{2.606568in}}%
\pgfpathlineto{\pgfqpoint{2.193010in}{2.601781in}}%
\pgfpathlineto{\pgfqpoint{2.196232in}{2.604309in}}%
\pgfpathlineto{\pgfqpoint{2.199453in}{2.617330in}}%
\pgfpathlineto{\pgfqpoint{2.200527in}{2.617301in}}%
\pgfpathlineto{\pgfqpoint{2.205896in}{2.621017in}}%
\pgfpathlineto{\pgfqpoint{2.206970in}{2.620660in}}%
\pgfpathlineto{\pgfqpoint{2.208044in}{2.621612in}}%
\pgfpathlineto{\pgfqpoint{2.218782in}{2.622385in}}%
\pgfpathlineto{\pgfqpoint{2.219856in}{2.628182in}}%
\pgfpathlineto{\pgfqpoint{2.223078in}{2.627588in}}%
\pgfpathlineto{\pgfqpoint{2.226299in}{2.627201in}}%
\pgfpathlineto{\pgfqpoint{2.227373in}{2.627915in}}%
\pgfpathlineto{\pgfqpoint{2.229521in}{2.626012in}}%
\pgfpathlineto{\pgfqpoint{2.230594in}{2.628123in}}%
\pgfpathlineto{\pgfqpoint{2.233816in}{2.628598in}}%
\pgfpathlineto{\pgfqpoint{2.234890in}{2.627082in}}%
\pgfpathlineto{\pgfqpoint{2.235964in}{2.624496in}}%
\pgfpathlineto{\pgfqpoint{2.237038in}{2.624377in}}%
\pgfpathlineto{\pgfqpoint{2.238111in}{2.625417in}}%
\pgfpathlineto{\pgfqpoint{2.243481in}{2.623009in}}%
\pgfpathlineto{\pgfqpoint{2.244554in}{2.624139in}}%
\pgfpathlineto{\pgfqpoint{2.245628in}{2.622623in}}%
\pgfpathlineto{\pgfqpoint{2.248850in}{2.620125in}}%
\pgfpathlineto{\pgfqpoint{2.249924in}{2.614684in}}%
\pgfpathlineto{\pgfqpoint{2.250997in}{2.617360in}}%
\pgfpathlineto{\pgfqpoint{2.252071in}{2.615755in}}%
\pgfpathlineto{\pgfqpoint{2.253145in}{2.615755in}}%
\pgfpathlineto{\pgfqpoint{2.256367in}{2.613555in}}%
\pgfpathlineto{\pgfqpoint{2.257440in}{2.614417in}}%
\pgfpathlineto{\pgfqpoint{2.258514in}{2.612395in}}%
\pgfpathlineto{\pgfqpoint{2.259588in}{2.612009in}}%
\pgfpathlineto{\pgfqpoint{2.260662in}{2.613317in}}%
\pgfpathlineto{\pgfqpoint{2.263883in}{2.615755in}}%
\pgfpathlineto{\pgfqpoint{2.264957in}{2.614506in}}%
\pgfpathlineto{\pgfqpoint{2.267105in}{2.613436in}}%
\pgfpathlineto{\pgfqpoint{2.268179in}{2.614090in}}%
\pgfpathlineto{\pgfqpoint{2.272474in}{2.615279in}}%
\pgfpathlineto{\pgfqpoint{2.274622in}{2.614506in}}%
\pgfpathlineto{\pgfqpoint{2.275696in}{2.611087in}}%
\pgfpathlineto{\pgfqpoint{2.278917in}{2.613644in}}%
\pgfpathlineto{\pgfqpoint{2.279991in}{2.609363in}}%
\pgfpathlineto{\pgfqpoint{2.281065in}{2.610047in}}%
\pgfpathlineto{\pgfqpoint{2.282139in}{2.612187in}}%
\pgfpathlineto{\pgfqpoint{2.283213in}{2.611147in}}%
\pgfpathlineto{\pgfqpoint{2.286434in}{2.609511in}}%
\pgfpathlineto{\pgfqpoint{2.287508in}{2.610314in}}%
\pgfpathlineto{\pgfqpoint{2.289656in}{2.614476in}}%
\pgfpathlineto{\pgfqpoint{2.290729in}{2.612693in}}%
\pgfpathlineto{\pgfqpoint{2.293951in}{2.609838in}}%
\pgfpathlineto{\pgfqpoint{2.295025in}{2.613822in}}%
\pgfpathlineto{\pgfqpoint{2.296099in}{2.614298in}}%
\pgfpathlineto{\pgfqpoint{2.297172in}{2.608411in}}%
\pgfpathlineto{\pgfqpoint{2.298246in}{2.610790in}}%
\pgfpathlineto{\pgfqpoint{2.303616in}{2.613703in}}%
\pgfpathlineto{\pgfqpoint{2.304689in}{2.612514in}}%
\pgfpathlineto{\pgfqpoint{2.305763in}{2.613882in}}%
\pgfpathlineto{\pgfqpoint{2.308985in}{2.615428in}}%
\pgfpathlineto{\pgfqpoint{2.310059in}{2.609452in}}%
\pgfpathlineto{\pgfqpoint{2.312206in}{2.611147in}}%
\pgfpathlineto{\pgfqpoint{2.313280in}{2.609214in}}%
\pgfpathlineto{\pgfqpoint{2.316502in}{2.611801in}}%
\pgfpathlineto{\pgfqpoint{2.317575in}{2.603268in}}%
\pgfpathlineto{\pgfqpoint{2.318649in}{2.601008in}}%
\pgfpathlineto{\pgfqpoint{2.319723in}{2.601781in}}%
\pgfpathlineto{\pgfqpoint{2.320797in}{2.597827in}}%
\pgfpathlineto{\pgfqpoint{2.324018in}{2.598303in}}%
\pgfpathlineto{\pgfqpoint{2.326166in}{2.600919in}}%
\pgfpathlineto{\pgfqpoint{2.327240in}{2.604071in}}%
\pgfpathlineto{\pgfqpoint{2.328314in}{2.603060in}}%
\pgfpathlineto{\pgfqpoint{2.331535in}{2.604873in}}%
\pgfpathlineto{\pgfqpoint{2.333683in}{2.601752in}}%
\pgfpathlineto{\pgfqpoint{2.335831in}{2.602465in}}%
\pgfpathlineto{\pgfqpoint{2.340126in}{2.607817in}}%
\pgfpathlineto{\pgfqpoint{2.341200in}{2.616825in}}%
\pgfpathlineto{\pgfqpoint{2.346569in}{2.606568in}}%
\pgfpathlineto{\pgfqpoint{2.347643in}{2.605795in}}%
\pgfpathlineto{\pgfqpoint{2.349791in}{2.606479in}}%
\pgfpathlineto{\pgfqpoint{2.350864in}{2.605349in}}%
\pgfpathlineto{\pgfqpoint{2.354086in}{2.604338in}}%
\pgfpathlineto{\pgfqpoint{2.355160in}{2.598065in}}%
\pgfpathlineto{\pgfqpoint{2.356234in}{2.598957in}}%
\pgfpathlineto{\pgfqpoint{2.358381in}{2.601930in}}%
\pgfpathlineto{\pgfqpoint{2.361603in}{2.599135in}}%
\pgfpathlineto{\pgfqpoint{2.362677in}{2.597322in}}%
\pgfpathlineto{\pgfqpoint{2.363750in}{2.594051in}}%
\pgfpathlineto{\pgfqpoint{2.364824in}{2.594260in}}%
\pgfpathlineto{\pgfqpoint{2.365898in}{2.595895in}}%
\pgfpathlineto{\pgfqpoint{2.370193in}{2.596162in}}%
\pgfpathlineto{\pgfqpoint{2.371267in}{2.593487in}}%
\pgfpathlineto{\pgfqpoint{2.372341in}{2.593189in}}%
\pgfpathlineto{\pgfqpoint{2.373415in}{2.596727in}}%
\pgfpathlineto{\pgfqpoint{2.377710in}{2.606836in}}%
\pgfpathlineto{\pgfqpoint{2.378784in}{2.604309in}}%
\pgfpathlineto{\pgfqpoint{2.379858in}{2.606836in}}%
\pgfpathlineto{\pgfqpoint{2.385227in}{2.606182in}}%
\pgfpathlineto{\pgfqpoint{2.386301in}{2.605200in}}%
\pgfpathlineto{\pgfqpoint{2.387375in}{2.605557in}}%
\pgfpathlineto{\pgfqpoint{2.388449in}{2.607014in}}%
\pgfpathlineto{\pgfqpoint{2.392744in}{2.606865in}}%
\pgfpathlineto{\pgfqpoint{2.393818in}{2.604636in}}%
\pgfpathlineto{\pgfqpoint{2.394892in}{2.605706in}}%
\pgfpathlineto{\pgfqpoint{2.395966in}{2.604933in}}%
\pgfpathlineto{\pgfqpoint{2.400261in}{2.606687in}}%
\pgfpathlineto{\pgfqpoint{2.401335in}{2.606152in}}%
\pgfpathlineto{\pgfqpoint{2.402409in}{2.609511in}}%
\pgfpathlineto{\pgfqpoint{2.403482in}{2.607965in}}%
\pgfpathlineto{\pgfqpoint{2.407778in}{2.607579in}}%
\pgfpathlineto{\pgfqpoint{2.408852in}{2.603595in}}%
\pgfpathlineto{\pgfqpoint{2.410999in}{2.603238in}}%
\pgfpathlineto{\pgfqpoint{2.415295in}{2.603981in}}%
\pgfpathlineto{\pgfqpoint{2.416369in}{2.603506in}}%
\pgfpathlineto{\pgfqpoint{2.417442in}{2.602108in}}%
\pgfpathlineto{\pgfqpoint{2.421738in}{2.601306in}}%
\pgfpathlineto{\pgfqpoint{2.422812in}{2.595330in}}%
\pgfpathlineto{\pgfqpoint{2.423885in}{2.598214in}}%
\pgfpathlineto{\pgfqpoint{2.424959in}{2.595538in}}%
\pgfpathlineto{\pgfqpoint{2.426033in}{2.599849in}}%
\pgfpathlineto{\pgfqpoint{2.430328in}{2.599462in}}%
\pgfpathlineto{\pgfqpoint{2.431402in}{2.599403in}}%
\pgfpathlineto{\pgfqpoint{2.432476in}{2.600354in}}%
\pgfpathlineto{\pgfqpoint{2.433550in}{2.600562in}}%
\pgfpathlineto{\pgfqpoint{2.438919in}{2.599879in}}%
\pgfpathlineto{\pgfqpoint{2.439993in}{2.601781in}}%
\pgfpathlineto{\pgfqpoint{2.441067in}{2.605022in}}%
\pgfpathlineto{\pgfqpoint{2.444288in}{2.606806in}}%
\pgfpathlineto{\pgfqpoint{2.445362in}{2.608144in}}%
\pgfpathlineto{\pgfqpoint{2.448584in}{2.615101in}}%
\pgfpathlineto{\pgfqpoint{2.452879in}{2.617330in}}%
\pgfpathlineto{\pgfqpoint{2.453953in}{2.616914in}}%
\pgfpathlineto{\pgfqpoint{2.456101in}{2.628123in}}%
\pgfpathlineto{\pgfqpoint{2.460396in}{2.626666in}}%
\pgfpathlineto{\pgfqpoint{2.461470in}{2.631274in}}%
\pgfpathlineto{\pgfqpoint{2.462544in}{2.630650in}}%
\pgfpathlineto{\pgfqpoint{2.463617in}{2.631096in}}%
\pgfpathlineto{\pgfqpoint{2.467913in}{2.631215in}}%
\pgfpathlineto{\pgfqpoint{2.468987in}{2.631958in}}%
\pgfpathlineto{\pgfqpoint{2.470060in}{2.637131in}}%
\pgfpathlineto{\pgfqpoint{2.471134in}{2.637874in}}%
\pgfpathlineto{\pgfqpoint{2.474356in}{2.639182in}}%
\pgfpathlineto{\pgfqpoint{2.475430in}{2.640223in}}%
\pgfpathlineto{\pgfqpoint{2.476504in}{2.645575in}}%
\pgfpathlineto{\pgfqpoint{2.478651in}{2.643047in}}%
\pgfpathlineto{\pgfqpoint{2.481873in}{2.643077in}}%
\pgfpathlineto{\pgfqpoint{2.485094in}{2.637012in}}%
\pgfpathlineto{\pgfqpoint{2.486168in}{2.635823in}}%
\pgfpathlineto{\pgfqpoint{2.489390in}{2.636715in}}%
\pgfpathlineto{\pgfqpoint{2.490463in}{2.636328in}}%
\pgfpathlineto{\pgfqpoint{2.491537in}{2.634247in}}%
\pgfpathlineto{\pgfqpoint{2.493685in}{2.633236in}}%
\pgfpathlineto{\pgfqpoint{2.499054in}{2.633920in}}%
\pgfpathlineto{\pgfqpoint{2.500128in}{2.634634in}}%
\pgfpathlineto{\pgfqpoint{2.505497in}{2.632285in}}%
\pgfpathlineto{\pgfqpoint{2.506571in}{2.635615in}}%
\pgfpathlineto{\pgfqpoint{2.507645in}{2.634455in}}%
\pgfpathlineto{\pgfqpoint{2.511940in}{2.636507in}}%
\pgfpathlineto{\pgfqpoint{2.513014in}{2.625566in}}%
\pgfpathlineto{\pgfqpoint{2.514088in}{2.624317in}}%
\pgfpathlineto{\pgfqpoint{2.515162in}{2.625715in}}%
\pgfpathlineto{\pgfqpoint{2.516236in}{2.625417in}}%
\pgfpathlineto{\pgfqpoint{2.522679in}{2.630947in}}%
\pgfpathlineto{\pgfqpoint{2.523752in}{2.630204in}}%
\pgfpathlineto{\pgfqpoint{2.526974in}{2.629817in}}%
\pgfpathlineto{\pgfqpoint{2.528048in}{2.630828in}}%
\pgfpathlineto{\pgfqpoint{2.529122in}{2.629817in}}%
\pgfpathlineto{\pgfqpoint{2.530195in}{2.631542in}}%
\pgfpathlineto{\pgfqpoint{2.531269in}{2.630293in}}%
\pgfpathlineto{\pgfqpoint{2.535565in}{2.629461in}}%
\pgfpathlineto{\pgfqpoint{2.536638in}{2.628361in}}%
\pgfpathlineto{\pgfqpoint{2.538786in}{2.630680in}}%
\pgfpathlineto{\pgfqpoint{2.543081in}{2.642245in}}%
\pgfpathlineto{\pgfqpoint{2.544155in}{2.639153in}}%
\pgfpathlineto{\pgfqpoint{2.545229in}{2.640015in}}%
\pgfpathlineto{\pgfqpoint{2.546303in}{2.640074in}}%
\pgfpathlineto{\pgfqpoint{2.551672in}{2.641501in}}%
\pgfpathlineto{\pgfqpoint{2.552746in}{2.643939in}}%
\pgfpathlineto{\pgfqpoint{2.553820in}{2.642215in}}%
\pgfpathlineto{\pgfqpoint{2.558115in}{2.642750in}}%
\pgfpathlineto{\pgfqpoint{2.561337in}{2.651134in}}%
\pgfpathlineto{\pgfqpoint{2.564558in}{2.651937in}}%
\pgfpathlineto{\pgfqpoint{2.565632in}{2.653245in}}%
\pgfpathlineto{\pgfqpoint{2.567780in}{2.652472in}}%
\pgfpathlineto{\pgfqpoint{2.568854in}{2.655237in}}%
\pgfpathlineto{\pgfqpoint{2.573149in}{2.656605in}}%
\pgfpathlineto{\pgfqpoint{2.574223in}{2.658715in}}%
\pgfpathlineto{\pgfqpoint{2.575297in}{2.659488in}}%
\pgfpathlineto{\pgfqpoint{2.576370in}{2.663175in}}%
\pgfpathlineto{\pgfqpoint{2.579592in}{2.662402in}}%
\pgfpathlineto{\pgfqpoint{2.580666in}{2.662818in}}%
\pgfpathlineto{\pgfqpoint{2.581740in}{2.664751in}}%
\pgfpathlineto{\pgfqpoint{2.582814in}{2.667962in}}%
\pgfpathlineto{\pgfqpoint{2.583887in}{2.669032in}}%
\pgfpathlineto{\pgfqpoint{2.587109in}{2.668794in}}%
\pgfpathlineto{\pgfqpoint{2.590330in}{2.658359in}}%
\pgfpathlineto{\pgfqpoint{2.591404in}{2.657378in}}%
\pgfpathlineto{\pgfqpoint{2.594626in}{2.659102in}}%
\pgfpathlineto{\pgfqpoint{2.596773in}{2.661183in}}%
\pgfpathlineto{\pgfqpoint{2.597847in}{2.658002in}}%
\pgfpathlineto{\pgfqpoint{2.598921in}{2.658061in}}%
\pgfpathlineto{\pgfqpoint{2.603216in}{2.654375in}}%
\pgfpathlineto{\pgfqpoint{2.604290in}{2.657199in}}%
\pgfpathlineto{\pgfqpoint{2.605364in}{2.656159in}}%
\pgfpathlineto{\pgfqpoint{2.606438in}{2.658240in}}%
\pgfpathlineto{\pgfqpoint{2.609659in}{2.656961in}}%
\pgfpathlineto{\pgfqpoint{2.610733in}{2.663502in}}%
\pgfpathlineto{\pgfqpoint{2.611807in}{2.665613in}}%
\pgfpathlineto{\pgfqpoint{2.612881in}{2.669448in}}%
\pgfpathlineto{\pgfqpoint{2.613955in}{2.665881in}}%
\pgfpathlineto{\pgfqpoint{2.618250in}{2.656218in}}%
\pgfpathlineto{\pgfqpoint{2.619324in}{2.653602in}}%
\pgfpathlineto{\pgfqpoint{2.620398in}{2.653245in}}%
\pgfpathlineto{\pgfqpoint{2.621472in}{2.656129in}}%
\pgfpathlineto{\pgfqpoint{2.624693in}{2.658567in}}%
\pgfpathlineto{\pgfqpoint{2.626841in}{2.656991in}}%
\pgfpathlineto{\pgfqpoint{2.627915in}{2.660351in}}%
\pgfpathlineto{\pgfqpoint{2.632210in}{2.658983in}}%
\pgfpathlineto{\pgfqpoint{2.633284in}{2.657318in}}%
\pgfpathlineto{\pgfqpoint{2.634358in}{2.660024in}}%
\pgfpathlineto{\pgfqpoint{2.636505in}{2.659667in}}%
\pgfpathlineto{\pgfqpoint{2.639727in}{2.660737in}}%
\pgfpathlineto{\pgfqpoint{2.640801in}{2.660440in}}%
\pgfpathlineto{\pgfqpoint{2.641875in}{2.662670in}}%
\pgfpathlineto{\pgfqpoint{2.642948in}{2.659488in}}%
\pgfpathlineto{\pgfqpoint{2.644022in}{2.658329in}}%
\pgfpathlineto{\pgfqpoint{2.647244in}{2.660648in}}%
\pgfpathlineto{\pgfqpoint{2.648318in}{2.664156in}}%
\pgfpathlineto{\pgfqpoint{2.649392in}{2.658805in}}%
\pgfpathlineto{\pgfqpoint{2.650465in}{2.659072in}}%
\pgfpathlineto{\pgfqpoint{2.651539in}{2.658002in}}%
\pgfpathlineto{\pgfqpoint{2.654761in}{2.658180in}}%
\pgfpathlineto{\pgfqpoint{2.655835in}{2.659518in}}%
\pgfpathlineto{\pgfqpoint{2.656908in}{2.656278in}}%
\pgfpathlineto{\pgfqpoint{2.657982in}{2.659964in}}%
\pgfpathlineto{\pgfqpoint{2.659056in}{2.656159in}}%
\pgfpathlineto{\pgfqpoint{2.663351in}{2.652948in}}%
\pgfpathlineto{\pgfqpoint{2.664425in}{2.655148in}}%
\pgfpathlineto{\pgfqpoint{2.665499in}{2.659459in}}%
\pgfpathlineto{\pgfqpoint{2.666573in}{2.656010in}}%
\pgfpathlineto{\pgfqpoint{2.669794in}{2.662343in}}%
\pgfpathlineto{\pgfqpoint{2.670868in}{2.660707in}}%
\pgfpathlineto{\pgfqpoint{2.671942in}{2.660202in}}%
\pgfpathlineto{\pgfqpoint{2.673016in}{2.665108in}}%
\pgfpathlineto{\pgfqpoint{2.677311in}{2.668527in}}%
\pgfpathlineto{\pgfqpoint{2.678385in}{2.668051in}}%
\pgfpathlineto{\pgfqpoint{2.680533in}{2.658210in}}%
\pgfpathlineto{\pgfqpoint{2.681607in}{2.657199in}}%
\pgfpathlineto{\pgfqpoint{2.685902in}{2.655921in}}%
\pgfpathlineto{\pgfqpoint{2.686976in}{2.652531in}}%
\pgfpathlineto{\pgfqpoint{2.688050in}{2.651729in}}%
\pgfpathlineto{\pgfqpoint{2.689124in}{2.653275in}}%
\pgfpathlineto{\pgfqpoint{2.692345in}{2.656694in}}%
\pgfpathlineto{\pgfqpoint{2.694493in}{2.661451in}}%
\pgfpathlineto{\pgfqpoint{2.695567in}{2.662283in}}%
\pgfpathlineto{\pgfqpoint{2.699862in}{2.663026in}}%
\pgfpathlineto{\pgfqpoint{2.700936in}{2.664305in}}%
\pgfpathlineto{\pgfqpoint{2.702010in}{2.672124in}}%
\pgfpathlineto{\pgfqpoint{2.703083in}{2.672629in}}%
\pgfpathlineto{\pgfqpoint{2.704157in}{2.671500in}}%
\pgfpathlineto{\pgfqpoint{2.707379in}{2.670608in}}%
\pgfpathlineto{\pgfqpoint{2.708453in}{2.683867in}}%
\pgfpathlineto{\pgfqpoint{2.709526in}{2.683570in}}%
\pgfpathlineto{\pgfqpoint{2.710600in}{2.687405in}}%
\pgfpathlineto{\pgfqpoint{2.714896in}{2.691895in}}%
\pgfpathlineto{\pgfqpoint{2.715969in}{2.686276in}}%
\pgfpathlineto{\pgfqpoint{2.717043in}{2.688327in}}%
\pgfpathlineto{\pgfqpoint{2.718117in}{2.686751in}}%
\pgfpathlineto{\pgfqpoint{2.719191in}{2.686692in}}%
\pgfpathlineto{\pgfqpoint{2.723486in}{2.679943in}}%
\pgfpathlineto{\pgfqpoint{2.724560in}{2.681578in}}%
\pgfpathlineto{\pgfqpoint{2.725634in}{2.681430in}}%
\pgfpathlineto{\pgfqpoint{2.726708in}{2.681846in}}%
\pgfpathlineto{\pgfqpoint{2.731003in}{2.680984in}}%
\pgfpathlineto{\pgfqpoint{2.732077in}{2.685384in}}%
\pgfpathlineto{\pgfqpoint{2.734225in}{2.680389in}}%
\pgfpathlineto{\pgfqpoint{2.737446in}{2.680954in}}%
\pgfpathlineto{\pgfqpoint{2.740668in}{2.678427in}}%
\pgfpathlineto{\pgfqpoint{2.741742in}{2.675959in}}%
\pgfpathlineto{\pgfqpoint{2.744963in}{2.675751in}}%
\pgfpathlineto{\pgfqpoint{2.746037in}{2.676911in}}%
\pgfpathlineto{\pgfqpoint{2.747111in}{2.673937in}}%
\pgfpathlineto{\pgfqpoint{2.752480in}{2.678635in}}%
\pgfpathlineto{\pgfqpoint{2.753554in}{2.684105in}}%
\pgfpathlineto{\pgfqpoint{2.754628in}{2.683511in}}%
\pgfpathlineto{\pgfqpoint{2.755702in}{2.682143in}}%
\pgfpathlineto{\pgfqpoint{2.756775in}{2.684016in}}%
\pgfpathlineto{\pgfqpoint{2.759997in}{2.681221in}}%
\pgfpathlineto{\pgfqpoint{2.761071in}{2.683124in}}%
\pgfpathlineto{\pgfqpoint{2.762145in}{2.687078in}}%
\pgfpathlineto{\pgfqpoint{2.763218in}{2.684076in}}%
\pgfpathlineto{\pgfqpoint{2.764292in}{2.685741in}}%
\pgfpathlineto{\pgfqpoint{2.767514in}{2.687316in}}%
\pgfpathlineto{\pgfqpoint{2.768588in}{2.691479in}}%
\pgfpathlineto{\pgfqpoint{2.769661in}{2.692311in}}%
\pgfpathlineto{\pgfqpoint{2.770735in}{2.688773in}}%
\pgfpathlineto{\pgfqpoint{2.771809in}{2.691092in}}%
\pgfpathlineto{\pgfqpoint{2.778252in}{2.687078in}}%
\pgfpathlineto{\pgfqpoint{2.779326in}{2.684403in}}%
\pgfpathlineto{\pgfqpoint{2.783621in}{2.684313in}}%
\pgfpathlineto{\pgfqpoint{2.784695in}{2.685681in}}%
\pgfpathlineto{\pgfqpoint{2.785769in}{2.685651in}}%
\pgfpathlineto{\pgfqpoint{2.786843in}{2.683273in}}%
\pgfpathlineto{\pgfqpoint{2.786843in}{2.683273in}}%
\pgfusepath{stroke}%
\end{pgfscope}%
\begin{pgfscope}%
\pgfpathrectangle{\pgfqpoint{0.320934in}{2.309648in}}{\pgfqpoint{2.583333in}{0.400885in}}%
\pgfusepath{clip}%
\pgfsetroundcap%
\pgfsetroundjoin%
\pgfsetlinewidth{1.505625pt}%
\definecolor{currentstroke}{rgb}{0.580392,0.403922,0.741176}%
\pgfsetstrokecolor{currentstroke}%
\pgfsetdash{}{0pt}%
\pgfpathmoveto{\pgfqpoint{0.438358in}{2.441950in}}%
\pgfpathlineto{\pgfqpoint{0.439432in}{2.440999in}}%
\pgfpathlineto{\pgfqpoint{0.440506in}{2.440999in}}%
\pgfpathlineto{\pgfqpoint{0.441580in}{2.439629in}}%
\pgfpathlineto{\pgfqpoint{0.461982in}{2.439629in}}%
\pgfpathlineto{\pgfqpoint{0.463056in}{2.438475in}}%
\pgfpathlineto{\pgfqpoint{0.464130in}{2.438796in}}%
\pgfpathlineto{\pgfqpoint{0.471647in}{2.438500in}}%
\pgfpathlineto{\pgfqpoint{0.477016in}{2.438500in}}%
\pgfpathlineto{\pgfqpoint{0.479164in}{2.436973in}}%
\pgfpathlineto{\pgfqpoint{0.486681in}{2.437090in}}%
\pgfpathlineto{\pgfqpoint{0.497419in}{2.437090in}}%
\pgfpathlineto{\pgfqpoint{0.498493in}{2.435368in}}%
\pgfpathlineto{\pgfqpoint{0.501714in}{2.435596in}}%
\pgfpathlineto{\pgfqpoint{0.509231in}{2.435481in}}%
\pgfpathlineto{\pgfqpoint{0.513527in}{2.434106in}}%
\pgfpathlineto{\pgfqpoint{0.515674in}{2.434728in}}%
\pgfpathlineto{\pgfqpoint{0.522117in}{2.434386in}}%
\pgfpathlineto{\pgfqpoint{0.524265in}{2.434386in}}%
\pgfpathlineto{\pgfqpoint{0.529634in}{2.431919in}}%
\pgfpathlineto{\pgfqpoint{0.530708in}{2.432085in}}%
\pgfpathlineto{\pgfqpoint{0.531782in}{2.431141in}}%
\pgfpathlineto{\pgfqpoint{0.535003in}{2.430592in}}%
\pgfpathlineto{\pgfqpoint{0.538225in}{2.432538in}}%
\pgfpathlineto{\pgfqpoint{0.542520in}{2.432538in}}%
\pgfpathlineto{\pgfqpoint{0.543594in}{2.430907in}}%
\pgfpathlineto{\pgfqpoint{0.545742in}{2.430795in}}%
\pgfpathlineto{\pgfqpoint{0.546816in}{2.429389in}}%
\pgfpathlineto{\pgfqpoint{0.551111in}{2.430935in}}%
\pgfpathlineto{\pgfqpoint{0.552185in}{2.428742in}}%
\pgfpathlineto{\pgfqpoint{0.553259in}{2.428236in}}%
\pgfpathlineto{\pgfqpoint{0.554333in}{2.429782in}}%
\pgfpathlineto{\pgfqpoint{0.557554in}{2.428995in}}%
\pgfpathlineto{\pgfqpoint{0.558628in}{2.429923in}}%
\pgfpathlineto{\pgfqpoint{0.576883in}{2.429675in}}%
\pgfpathlineto{\pgfqpoint{0.584400in}{2.427449in}}%
\pgfpathlineto{\pgfqpoint{0.588695in}{2.427839in}}%
\pgfpathlineto{\pgfqpoint{0.589769in}{2.427255in}}%
\pgfpathlineto{\pgfqpoint{0.590843in}{2.428256in}}%
\pgfpathlineto{\pgfqpoint{0.591917in}{2.426921in}}%
\pgfpathlineto{\pgfqpoint{0.596212in}{2.427115in}}%
\pgfpathlineto{\pgfqpoint{0.597286in}{2.426225in}}%
\pgfpathlineto{\pgfqpoint{0.598360in}{2.426726in}}%
\pgfpathlineto{\pgfqpoint{0.599434in}{2.425251in}}%
\pgfpathlineto{\pgfqpoint{0.602655in}{2.426531in}}%
\pgfpathlineto{\pgfqpoint{0.603729in}{2.426225in}}%
\pgfpathlineto{\pgfqpoint{0.604803in}{2.427588in}}%
\pgfpathlineto{\pgfqpoint{0.606951in}{2.427978in}}%
\pgfpathlineto{\pgfqpoint{0.610172in}{2.426030in}}%
\pgfpathlineto{\pgfqpoint{0.612320in}{2.426030in}}%
\pgfpathlineto{\pgfqpoint{0.614468in}{2.422635in}}%
\pgfpathlineto{\pgfqpoint{0.617689in}{2.422021in}}%
\pgfpathlineto{\pgfqpoint{0.619837in}{2.420585in}}%
\pgfpathlineto{\pgfqpoint{0.620911in}{2.421825in}}%
\pgfpathlineto{\pgfqpoint{0.621984in}{2.421320in}}%
\pgfpathlineto{\pgfqpoint{0.626280in}{2.421747in}}%
\pgfpathlineto{\pgfqpoint{0.629501in}{2.419422in}}%
\pgfpathlineto{\pgfqpoint{0.633797in}{2.418447in}}%
\pgfpathlineto{\pgfqpoint{0.637018in}{2.419243in}}%
\pgfpathlineto{\pgfqpoint{0.643461in}{2.419095in}}%
\pgfpathlineto{\pgfqpoint{0.644535in}{2.417323in}}%
\pgfpathlineto{\pgfqpoint{0.647757in}{2.417630in}}%
\pgfpathlineto{\pgfqpoint{0.649904in}{2.415837in}}%
\pgfpathlineto{\pgfqpoint{0.650978in}{2.415536in}}%
\pgfpathlineto{\pgfqpoint{0.652052in}{2.417239in}}%
\pgfpathlineto{\pgfqpoint{0.655273in}{2.418255in}}%
\pgfpathlineto{\pgfqpoint{0.656347in}{2.419743in}}%
\pgfpathlineto{\pgfqpoint{0.657421in}{2.419375in}}%
\pgfpathlineto{\pgfqpoint{0.659569in}{2.415453in}}%
\pgfpathlineto{\pgfqpoint{0.664938in}{2.415704in}}%
\pgfpathlineto{\pgfqpoint{0.666012in}{2.417483in}}%
\pgfpathlineto{\pgfqpoint{0.667086in}{2.416178in}}%
\pgfpathlineto{\pgfqpoint{0.670307in}{2.416713in}}%
\pgfpathlineto{\pgfqpoint{0.671381in}{2.417769in}}%
\pgfpathlineto{\pgfqpoint{0.678898in}{2.417076in}}%
\pgfpathlineto{\pgfqpoint{0.682119in}{2.418716in}}%
\pgfpathlineto{\pgfqpoint{0.688562in}{2.418837in}}%
\pgfpathlineto{\pgfqpoint{0.689636in}{2.417892in}}%
\pgfpathlineto{\pgfqpoint{0.702522in}{2.418536in}}%
\pgfpathlineto{\pgfqpoint{0.703596in}{2.417403in}}%
\pgfpathlineto{\pgfqpoint{0.710039in}{2.416810in}}%
\pgfpathlineto{\pgfqpoint{0.711113in}{2.415165in}}%
\pgfpathlineto{\pgfqpoint{0.712187in}{2.416151in}}%
\pgfpathlineto{\pgfqpoint{0.715408in}{2.416570in}}%
\pgfpathlineto{\pgfqpoint{0.718630in}{2.415335in}}%
\pgfpathlineto{\pgfqpoint{0.719704in}{2.415013in}}%
\pgfpathlineto{\pgfqpoint{0.722925in}{2.415127in}}%
\pgfpathlineto{\pgfqpoint{0.723999in}{2.414531in}}%
\pgfpathlineto{\pgfqpoint{0.726147in}{2.415099in}}%
\pgfpathlineto{\pgfqpoint{0.727221in}{2.415305in}}%
\pgfpathlineto{\pgfqpoint{0.737959in}{2.414292in}}%
\pgfpathlineto{\pgfqpoint{0.739033in}{2.416215in}}%
\pgfpathlineto{\pgfqpoint{0.742254in}{2.417057in}}%
\pgfpathlineto{\pgfqpoint{0.745476in}{2.415830in}}%
\pgfpathlineto{\pgfqpoint{0.747623in}{2.411317in}}%
\pgfpathlineto{\pgfqpoint{0.748697in}{2.408592in}}%
\pgfpathlineto{\pgfqpoint{0.749771in}{2.409711in}}%
\pgfpathlineto{\pgfqpoint{0.752993in}{2.409858in}}%
\pgfpathlineto{\pgfqpoint{0.754067in}{2.411423in}}%
\pgfpathlineto{\pgfqpoint{0.755140in}{2.411683in}}%
\pgfpathlineto{\pgfqpoint{0.756214in}{2.410965in}}%
\pgfpathlineto{\pgfqpoint{0.757288in}{2.411395in}}%
\pgfpathlineto{\pgfqpoint{0.762657in}{2.411547in}}%
\pgfpathlineto{\pgfqpoint{0.763731in}{2.410309in}}%
\pgfpathlineto{\pgfqpoint{0.764805in}{2.411374in}}%
\pgfpathlineto{\pgfqpoint{0.769100in}{2.411178in}}%
\pgfpathlineto{\pgfqpoint{0.771248in}{2.413620in}}%
\pgfpathlineto{\pgfqpoint{0.772322in}{2.413216in}}%
\pgfpathlineto{\pgfqpoint{0.777691in}{2.414337in}}%
\pgfpathlineto{\pgfqpoint{0.778765in}{2.414700in}}%
\pgfpathlineto{\pgfqpoint{0.779839in}{2.414472in}}%
\pgfpathlineto{\pgfqpoint{0.783060in}{2.414381in}}%
\pgfpathlineto{\pgfqpoint{0.784134in}{2.413587in}}%
\pgfpathlineto{\pgfqpoint{0.785208in}{2.413721in}}%
\pgfpathlineto{\pgfqpoint{0.787356in}{2.412642in}}%
\pgfpathlineto{\pgfqpoint{0.791651in}{2.414042in}}%
\pgfpathlineto{\pgfqpoint{0.792725in}{2.413116in}}%
\pgfpathlineto{\pgfqpoint{0.793799in}{2.413249in}}%
\pgfpathlineto{\pgfqpoint{0.794872in}{2.412312in}}%
\pgfpathlineto{\pgfqpoint{0.799168in}{2.412069in}}%
\pgfpathlineto{\pgfqpoint{0.805611in}{2.410720in}}%
\pgfpathlineto{\pgfqpoint{0.806685in}{2.409819in}}%
\pgfpathlineto{\pgfqpoint{0.809906in}{2.410562in}}%
\pgfpathlineto{\pgfqpoint{0.814201in}{2.410091in}}%
\pgfpathlineto{\pgfqpoint{0.815275in}{2.410665in}}%
\pgfpathlineto{\pgfqpoint{0.816349in}{2.410450in}}%
\pgfpathlineto{\pgfqpoint{0.817423in}{2.411305in}}%
\pgfpathlineto{\pgfqpoint{0.820645in}{2.411760in}}%
\pgfpathlineto{\pgfqpoint{0.823866in}{2.411628in}}%
\pgfpathlineto{\pgfqpoint{0.824940in}{2.412738in}}%
\pgfpathlineto{\pgfqpoint{0.831383in}{2.410408in}}%
\pgfpathlineto{\pgfqpoint{0.832457in}{2.408977in}}%
\pgfpathlineto{\pgfqpoint{0.836752in}{2.409207in}}%
\pgfpathlineto{\pgfqpoint{0.839974in}{2.407585in}}%
\pgfpathlineto{\pgfqpoint{0.846417in}{2.406648in}}%
\pgfpathlineto{\pgfqpoint{0.847490in}{2.406085in}}%
\pgfpathlineto{\pgfqpoint{0.852860in}{2.406718in}}%
\pgfpathlineto{\pgfqpoint{0.853934in}{2.406295in}}%
\pgfpathlineto{\pgfqpoint{0.855007in}{2.404936in}}%
\pgfpathlineto{\pgfqpoint{0.858229in}{2.405424in}}%
\pgfpathlineto{\pgfqpoint{0.859303in}{2.404123in}}%
\pgfpathlineto{\pgfqpoint{0.861450in}{2.404917in}}%
\pgfpathlineto{\pgfqpoint{0.862524in}{2.404487in}}%
\pgfpathlineto{\pgfqpoint{0.865746in}{2.404603in}}%
\pgfpathlineto{\pgfqpoint{0.867893in}{2.402539in}}%
\pgfpathlineto{\pgfqpoint{0.868967in}{2.403066in}}%
\pgfpathlineto{\pgfqpoint{0.870041in}{2.402402in}}%
\pgfpathlineto{\pgfqpoint{0.873263in}{2.402496in}}%
\pgfpathlineto{\pgfqpoint{0.874336in}{2.401895in}}%
\pgfpathlineto{\pgfqpoint{0.876484in}{2.401875in}}%
\pgfpathlineto{\pgfqpoint{0.877558in}{2.401336in}}%
\pgfpathlineto{\pgfqpoint{0.881853in}{2.400103in}}%
\pgfpathlineto{\pgfqpoint{0.882927in}{2.400572in}}%
\pgfpathlineto{\pgfqpoint{0.885075in}{2.400244in}}%
\pgfpathlineto{\pgfqpoint{0.888296in}{2.401293in}}%
\pgfpathlineto{\pgfqpoint{0.889370in}{2.400998in}}%
\pgfpathlineto{\pgfqpoint{0.890444in}{2.400119in}}%
\pgfpathlineto{\pgfqpoint{0.891518in}{2.400444in}}%
\pgfpathlineto{\pgfqpoint{0.892592in}{2.399537in}}%
\pgfpathlineto{\pgfqpoint{0.907625in}{2.395899in}}%
\pgfpathlineto{\pgfqpoint{0.911921in}{2.396367in}}%
\pgfpathlineto{\pgfqpoint{0.912995in}{2.395523in}}%
\pgfpathlineto{\pgfqpoint{0.914068in}{2.396138in}}%
\pgfpathlineto{\pgfqpoint{0.915142in}{2.395112in}}%
\pgfpathlineto{\pgfqpoint{0.918364in}{2.395194in}}%
\pgfpathlineto{\pgfqpoint{0.919438in}{2.393557in}}%
\pgfpathlineto{\pgfqpoint{0.921585in}{2.392645in}}%
\pgfpathlineto{\pgfqpoint{0.925881in}{2.392107in}}%
\pgfpathlineto{\pgfqpoint{0.926955in}{2.391106in}}%
\pgfpathlineto{\pgfqpoint{0.928028in}{2.391905in}}%
\pgfpathlineto{\pgfqpoint{0.929102in}{2.391468in}}%
\pgfpathlineto{\pgfqpoint{0.933398in}{2.393164in}}%
\pgfpathlineto{\pgfqpoint{0.937693in}{2.391011in}}%
\pgfpathlineto{\pgfqpoint{0.940914in}{2.392344in}}%
\pgfpathlineto{\pgfqpoint{0.941988in}{2.390049in}}%
\pgfpathlineto{\pgfqpoint{0.943062in}{2.389462in}}%
\pgfpathlineto{\pgfqpoint{0.944136in}{2.390371in}}%
\pgfpathlineto{\pgfqpoint{0.945210in}{2.388703in}}%
\pgfpathlineto{\pgfqpoint{0.948431in}{2.388277in}}%
\pgfpathlineto{\pgfqpoint{0.949505in}{2.387520in}}%
\pgfpathlineto{\pgfqpoint{0.950579in}{2.388798in}}%
\pgfpathlineto{\pgfqpoint{0.951653in}{2.387767in}}%
\pgfpathlineto{\pgfqpoint{0.952727in}{2.387897in}}%
\pgfpathlineto{\pgfqpoint{0.955948in}{2.387333in}}%
\pgfpathlineto{\pgfqpoint{0.957022in}{2.387748in}}%
\pgfpathlineto{\pgfqpoint{0.958096in}{2.388974in}}%
\pgfpathlineto{\pgfqpoint{0.960244in}{2.387099in}}%
\pgfpathlineto{\pgfqpoint{0.963465in}{2.388382in}}%
\pgfpathlineto{\pgfqpoint{0.964539in}{2.387330in}}%
\pgfpathlineto{\pgfqpoint{0.967760in}{2.387052in}}%
\pgfpathlineto{\pgfqpoint{0.970982in}{2.386938in}}%
\pgfpathlineto{\pgfqpoint{0.973130in}{2.384827in}}%
\pgfpathlineto{\pgfqpoint{0.974203in}{2.385045in}}%
\pgfpathlineto{\pgfqpoint{0.975277in}{2.384305in}}%
\pgfpathlineto{\pgfqpoint{0.978499in}{2.384386in}}%
\pgfpathlineto{\pgfqpoint{0.979573in}{2.383737in}}%
\pgfpathlineto{\pgfqpoint{0.981720in}{2.384553in}}%
\pgfpathlineto{\pgfqpoint{0.982794in}{2.385001in}}%
\pgfpathlineto{\pgfqpoint{0.987089in}{2.384084in}}%
\pgfpathlineto{\pgfqpoint{0.988163in}{2.386315in}}%
\pgfpathlineto{\pgfqpoint{0.989237in}{2.385907in}}%
\pgfpathlineto{\pgfqpoint{0.990311in}{2.388054in}}%
\pgfpathlineto{\pgfqpoint{0.993533in}{2.387400in}}%
\pgfpathlineto{\pgfqpoint{0.995680in}{2.388657in}}%
\pgfpathlineto{\pgfqpoint{0.997828in}{2.387137in}}%
\pgfpathlineto{\pgfqpoint{1.001049in}{2.386851in}}%
\pgfpathlineto{\pgfqpoint{1.002123in}{2.387391in}}%
\pgfpathlineto{\pgfqpoint{1.003197in}{2.388554in}}%
\pgfpathlineto{\pgfqpoint{1.004271in}{2.387100in}}%
\pgfpathlineto{\pgfqpoint{1.005345in}{2.387100in}}%
\pgfpathlineto{\pgfqpoint{1.008566in}{2.386228in}}%
\pgfpathlineto{\pgfqpoint{1.009640in}{2.385358in}}%
\pgfpathlineto{\pgfqpoint{1.011788in}{2.389827in}}%
\pgfpathlineto{\pgfqpoint{1.017157in}{2.386401in}}%
\pgfpathlineto{\pgfqpoint{1.018231in}{2.384439in}}%
\pgfpathlineto{\pgfqpoint{1.019305in}{2.384765in}}%
\pgfpathlineto{\pgfqpoint{1.020378in}{2.385749in}}%
\pgfpathlineto{\pgfqpoint{1.039708in}{2.380613in}}%
\pgfpathlineto{\pgfqpoint{1.041855in}{2.380852in}}%
\pgfpathlineto{\pgfqpoint{1.042929in}{2.378653in}}%
\pgfpathlineto{\pgfqpoint{1.050446in}{2.378040in}}%
\pgfpathlineto{\pgfqpoint{1.056889in}{2.377093in}}%
\pgfpathlineto{\pgfqpoint{1.057963in}{2.376486in}}%
\pgfpathlineto{\pgfqpoint{1.064406in}{2.377518in}}%
\pgfpathlineto{\pgfqpoint{1.065480in}{2.378496in}}%
\pgfpathlineto{\pgfqpoint{1.068701in}{2.378857in}}%
\pgfpathlineto{\pgfqpoint{1.069775in}{2.377826in}}%
\pgfpathlineto{\pgfqpoint{1.071923in}{2.381362in}}%
\pgfpathlineto{\pgfqpoint{1.072997in}{2.381554in}}%
\pgfpathlineto{\pgfqpoint{1.076218in}{2.380385in}}%
\pgfpathlineto{\pgfqpoint{1.079440in}{2.382741in}}%
\pgfpathlineto{\pgfqpoint{1.080513in}{2.381833in}}%
\pgfpathlineto{\pgfqpoint{1.083735in}{2.382802in}}%
\pgfpathlineto{\pgfqpoint{1.084809in}{2.384344in}}%
\pgfpathlineto{\pgfqpoint{1.086956in}{2.383868in}}%
\pgfpathlineto{\pgfqpoint{1.088030in}{2.384057in}}%
\pgfpathlineto{\pgfqpoint{1.093399in}{2.383489in}}%
\pgfpathlineto{\pgfqpoint{1.098769in}{2.382744in}}%
\pgfpathlineto{\pgfqpoint{1.100916in}{2.380878in}}%
\pgfpathlineto{\pgfqpoint{1.103064in}{2.381603in}}%
\pgfpathlineto{\pgfqpoint{1.107359in}{2.381063in}}%
\pgfpathlineto{\pgfqpoint{1.108433in}{2.380134in}}%
\pgfpathlineto{\pgfqpoint{1.109507in}{2.379960in}}%
\pgfpathlineto{\pgfqpoint{1.110581in}{2.380382in}}%
\pgfpathlineto{\pgfqpoint{1.113802in}{2.381010in}}%
\pgfpathlineto{\pgfqpoint{1.117024in}{2.383234in}}%
\pgfpathlineto{\pgfqpoint{1.118098in}{2.383620in}}%
\pgfpathlineto{\pgfqpoint{1.121319in}{2.383660in}}%
\pgfpathlineto{\pgfqpoint{1.122393in}{2.382774in}}%
\pgfpathlineto{\pgfqpoint{1.123467in}{2.382972in}}%
\pgfpathlineto{\pgfqpoint{1.124541in}{2.383780in}}%
\pgfpathlineto{\pgfqpoint{1.125615in}{2.382932in}}%
\pgfpathlineto{\pgfqpoint{1.128836in}{2.383752in}}%
\pgfpathlineto{\pgfqpoint{1.129910in}{2.384882in}}%
\pgfpathlineto{\pgfqpoint{1.130984in}{2.384469in}}%
\pgfpathlineto{\pgfqpoint{1.133132in}{2.380470in}}%
\pgfpathlineto{\pgfqpoint{1.137427in}{2.379955in}}%
\pgfpathlineto{\pgfqpoint{1.139575in}{2.377811in}}%
\pgfpathlineto{\pgfqpoint{1.140648in}{2.378156in}}%
\pgfpathlineto{\pgfqpoint{1.143870in}{2.378600in}}%
\pgfpathlineto{\pgfqpoint{1.144944in}{2.377401in}}%
\pgfpathlineto{\pgfqpoint{1.148165in}{2.377672in}}%
\pgfpathlineto{\pgfqpoint{1.151387in}{2.377363in}}%
\pgfpathlineto{\pgfqpoint{1.152461in}{2.376608in}}%
\pgfpathlineto{\pgfqpoint{1.154608in}{2.377131in}}%
\pgfpathlineto{\pgfqpoint{1.155682in}{2.376368in}}%
\pgfpathlineto{\pgfqpoint{1.162125in}{2.377051in}}%
\pgfpathlineto{\pgfqpoint{1.163199in}{2.375680in}}%
\pgfpathlineto{\pgfqpoint{1.166421in}{2.375452in}}%
\pgfpathlineto{\pgfqpoint{1.168568in}{2.376373in}}%
\pgfpathlineto{\pgfqpoint{1.170716in}{2.375339in}}%
\pgfpathlineto{\pgfqpoint{1.173937in}{2.375429in}}%
\pgfpathlineto{\pgfqpoint{1.176085in}{2.374606in}}%
\pgfpathlineto{\pgfqpoint{1.177159in}{2.374562in}}%
\pgfpathlineto{\pgfqpoint{1.178233in}{2.373885in}}%
\pgfpathlineto{\pgfqpoint{1.181454in}{2.373524in}}%
\pgfpathlineto{\pgfqpoint{1.183602in}{2.374132in}}%
\pgfpathlineto{\pgfqpoint{1.191119in}{2.375422in}}%
\pgfpathlineto{\pgfqpoint{1.192193in}{2.376067in}}%
\pgfpathlineto{\pgfqpoint{1.193266in}{2.374621in}}%
\pgfpathlineto{\pgfqpoint{1.197562in}{2.374899in}}%
\pgfpathlineto{\pgfqpoint{1.198636in}{2.375895in}}%
\pgfpathlineto{\pgfqpoint{1.199710in}{2.377817in}}%
\pgfpathlineto{\pgfqpoint{1.204005in}{2.377602in}}%
\pgfpathlineto{\pgfqpoint{1.205079in}{2.378327in}}%
\pgfpathlineto{\pgfqpoint{1.206153in}{2.376276in}}%
\pgfpathlineto{\pgfqpoint{1.207226in}{2.376923in}}%
\pgfpathlineto{\pgfqpoint{1.215817in}{2.376546in}}%
\pgfpathlineto{\pgfqpoint{1.219039in}{2.376605in}}%
\pgfpathlineto{\pgfqpoint{1.220112in}{2.377314in}}%
\pgfpathlineto{\pgfqpoint{1.222260in}{2.377881in}}%
\pgfpathlineto{\pgfqpoint{1.223334in}{2.377044in}}%
\pgfpathlineto{\pgfqpoint{1.226555in}{2.376551in}}%
\pgfpathlineto{\pgfqpoint{1.227629in}{2.374599in}}%
\pgfpathlineto{\pgfqpoint{1.228703in}{2.374721in}}%
\pgfpathlineto{\pgfqpoint{1.230851in}{2.374163in}}%
\pgfpathlineto{\pgfqpoint{1.237294in}{2.374250in}}%
\pgfpathlineto{\pgfqpoint{1.238368in}{2.373852in}}%
\pgfpathlineto{\pgfqpoint{1.243737in}{2.374547in}}%
\pgfpathlineto{\pgfqpoint{1.245885in}{2.378167in}}%
\pgfpathlineto{\pgfqpoint{1.250180in}{2.378694in}}%
\pgfpathlineto{\pgfqpoint{1.251254in}{2.379949in}}%
\pgfpathlineto{\pgfqpoint{1.252328in}{2.379311in}}%
\pgfpathlineto{\pgfqpoint{1.253401in}{2.380397in}}%
\pgfpathlineto{\pgfqpoint{1.257697in}{2.380397in}}%
\pgfpathlineto{\pgfqpoint{1.258771in}{2.379662in}}%
\pgfpathlineto{\pgfqpoint{1.260918in}{2.376685in}}%
\pgfpathlineto{\pgfqpoint{1.264140in}{2.375634in}}%
\pgfpathlineto{\pgfqpoint{1.265214in}{2.373721in}}%
\pgfpathlineto{\pgfqpoint{1.266287in}{2.374247in}}%
\pgfpathlineto{\pgfqpoint{1.268435in}{2.373920in}}%
\pgfpathlineto{\pgfqpoint{1.279174in}{2.374881in}}%
\pgfpathlineto{\pgfqpoint{1.282395in}{2.374629in}}%
\pgfpathlineto{\pgfqpoint{1.283469in}{2.373875in}}%
\pgfpathlineto{\pgfqpoint{1.286690in}{2.374425in}}%
\pgfpathlineto{\pgfqpoint{1.287764in}{2.372675in}}%
\pgfpathlineto{\pgfqpoint{1.288838in}{2.373386in}}%
\pgfpathlineto{\pgfqpoint{1.290986in}{2.372677in}}%
\pgfpathlineto{\pgfqpoint{1.297429in}{2.372981in}}%
\pgfpathlineto{\pgfqpoint{1.298503in}{2.373157in}}%
\pgfpathlineto{\pgfqpoint{1.302798in}{2.372084in}}%
\pgfpathlineto{\pgfqpoint{1.303872in}{2.372408in}}%
\pgfpathlineto{\pgfqpoint{1.304946in}{2.371908in}}%
\pgfpathlineto{\pgfqpoint{1.306020in}{2.370221in}}%
\pgfpathlineto{\pgfqpoint{1.309241in}{2.370872in}}%
\pgfpathlineto{\pgfqpoint{1.310315in}{2.368887in}}%
\pgfpathlineto{\pgfqpoint{1.312463in}{2.368976in}}%
\pgfpathlineto{\pgfqpoint{1.321053in}{2.367985in}}%
\pgfpathlineto{\pgfqpoint{1.325349in}{2.368288in}}%
\pgfpathlineto{\pgfqpoint{1.326422in}{2.367527in}}%
\pgfpathlineto{\pgfqpoint{1.327496in}{2.369557in}}%
\pgfpathlineto{\pgfqpoint{1.328570in}{2.369271in}}%
\pgfpathlineto{\pgfqpoint{1.331792in}{2.369028in}}%
\pgfpathlineto{\pgfqpoint{1.332865in}{2.367235in}}%
\pgfpathlineto{\pgfqpoint{1.335013in}{2.367427in}}%
\pgfpathlineto{\pgfqpoint{1.341456in}{2.366373in}}%
\pgfpathlineto{\pgfqpoint{1.343604in}{2.366733in}}%
\pgfpathlineto{\pgfqpoint{1.346825in}{2.365445in}}%
\pgfpathlineto{\pgfqpoint{1.350047in}{2.366092in}}%
\pgfpathlineto{\pgfqpoint{1.351121in}{2.367088in}}%
\pgfpathlineto{\pgfqpoint{1.354342in}{2.366510in}}%
\pgfpathlineto{\pgfqpoint{1.355416in}{2.366919in}}%
\pgfpathlineto{\pgfqpoint{1.356490in}{2.365749in}}%
\pgfpathlineto{\pgfqpoint{1.357564in}{2.366084in}}%
\pgfpathlineto{\pgfqpoint{1.358638in}{2.365746in}}%
\pgfpathlineto{\pgfqpoint{1.366154in}{2.366014in}}%
\pgfpathlineto{\pgfqpoint{1.371524in}{2.365558in}}%
\pgfpathlineto{\pgfqpoint{1.373671in}{2.365113in}}%
\pgfpathlineto{\pgfqpoint{1.380114in}{2.365293in}}%
\pgfpathlineto{\pgfqpoint{1.381188in}{2.364721in}}%
\pgfpathlineto{\pgfqpoint{1.393000in}{2.362671in}}%
\pgfpathlineto{\pgfqpoint{1.396222in}{2.363863in}}%
\pgfpathlineto{\pgfqpoint{1.401591in}{2.363642in}}%
\pgfpathlineto{\pgfqpoint{1.403739in}{2.361761in}}%
\pgfpathlineto{\pgfqpoint{1.408034in}{2.362260in}}%
\pgfpathlineto{\pgfqpoint{1.409108in}{2.361388in}}%
\pgfpathlineto{\pgfqpoint{1.414477in}{2.362229in}}%
\pgfpathlineto{\pgfqpoint{1.415551in}{2.361297in}}%
\pgfpathlineto{\pgfqpoint{1.429511in}{2.361631in}}%
\pgfpathlineto{\pgfqpoint{1.432732in}{2.365434in}}%
\pgfpathlineto{\pgfqpoint{1.433806in}{2.364275in}}%
\pgfpathlineto{\pgfqpoint{1.437028in}{2.364690in}}%
\pgfpathlineto{\pgfqpoint{1.438102in}{2.363734in}}%
\pgfpathlineto{\pgfqpoint{1.441323in}{2.364019in}}%
\pgfpathlineto{\pgfqpoint{1.446692in}{2.363866in}}%
\pgfpathlineto{\pgfqpoint{1.447766in}{2.365587in}}%
\pgfpathlineto{\pgfqpoint{1.448840in}{2.365746in}}%
\pgfpathlineto{\pgfqpoint{1.460652in}{2.365126in}}%
\pgfpathlineto{\pgfqpoint{1.462800in}{2.364016in}}%
\pgfpathlineto{\pgfqpoint{1.463874in}{2.364679in}}%
\pgfpathlineto{\pgfqpoint{1.470317in}{2.362317in}}%
\pgfpathlineto{\pgfqpoint{1.471391in}{2.362618in}}%
\pgfpathlineto{\pgfqpoint{1.490720in}{2.362073in}}%
\pgfpathlineto{\pgfqpoint{1.491794in}{2.361177in}}%
\pgfpathlineto{\pgfqpoint{1.493941in}{2.361477in}}%
\pgfpathlineto{\pgfqpoint{1.497163in}{2.361376in}}%
\pgfpathlineto{\pgfqpoint{1.500384in}{2.359468in}}%
\pgfpathlineto{\pgfqpoint{1.501458in}{2.359021in}}%
\pgfpathlineto{\pgfqpoint{1.505753in}{2.359384in}}%
\pgfpathlineto{\pgfqpoint{1.506827in}{2.358565in}}%
\pgfpathlineto{\pgfqpoint{1.507901in}{2.359616in}}%
\pgfpathlineto{\pgfqpoint{1.513270in}{2.359976in}}%
\pgfpathlineto{\pgfqpoint{1.514344in}{2.361602in}}%
\pgfpathlineto{\pgfqpoint{1.515418in}{2.361940in}}%
\pgfpathlineto{\pgfqpoint{1.516492in}{2.360977in}}%
\pgfpathlineto{\pgfqpoint{1.519713in}{2.361177in}}%
\pgfpathlineto{\pgfqpoint{1.520787in}{2.362989in}}%
\pgfpathlineto{\pgfqpoint{1.521861in}{2.361045in}}%
\pgfpathlineto{\pgfqpoint{1.522935in}{2.363120in}}%
\pgfpathlineto{\pgfqpoint{1.527230in}{2.365435in}}%
\pgfpathlineto{\pgfqpoint{1.528304in}{2.367169in}}%
\pgfpathlineto{\pgfqpoint{1.529378in}{2.366137in}}%
\pgfpathlineto{\pgfqpoint{1.530452in}{2.367333in}}%
\pgfpathlineto{\pgfqpoint{1.531526in}{2.365682in}}%
\pgfpathlineto{\pgfqpoint{1.534747in}{2.365268in}}%
\pgfpathlineto{\pgfqpoint{1.539042in}{2.362150in}}%
\pgfpathlineto{\pgfqpoint{1.543338in}{2.360904in}}%
\pgfpathlineto{\pgfqpoint{1.546559in}{2.358752in}}%
\pgfpathlineto{\pgfqpoint{1.549781in}{2.358973in}}%
\pgfpathlineto{\pgfqpoint{1.551929in}{2.358034in}}%
\pgfpathlineto{\pgfqpoint{1.553002in}{2.357902in}}%
\pgfpathlineto{\pgfqpoint{1.554076in}{2.358448in}}%
\pgfpathlineto{\pgfqpoint{1.560519in}{2.357857in}}%
\pgfpathlineto{\pgfqpoint{1.561593in}{2.358471in}}%
\pgfpathlineto{\pgfqpoint{1.569110in}{2.358194in}}%
\pgfpathlineto{\pgfqpoint{1.573405in}{2.358987in}}%
\pgfpathlineto{\pgfqpoint{1.576627in}{2.357909in}}%
\pgfpathlineto{\pgfqpoint{1.584144in}{2.357720in}}%
\pgfpathlineto{\pgfqpoint{1.588439in}{2.358034in}}%
\pgfpathlineto{\pgfqpoint{1.589513in}{2.359270in}}%
\pgfpathlineto{\pgfqpoint{1.590587in}{2.358927in}}%
\pgfpathlineto{\pgfqpoint{1.591661in}{2.360533in}}%
\pgfpathlineto{\pgfqpoint{1.594882in}{2.360872in}}%
\pgfpathlineto{\pgfqpoint{1.595956in}{2.361764in}}%
\pgfpathlineto{\pgfqpoint{1.597030in}{2.360774in}}%
\pgfpathlineto{\pgfqpoint{1.598104in}{2.358753in}}%
\pgfpathlineto{\pgfqpoint{1.599177in}{2.359637in}}%
\pgfpathlineto{\pgfqpoint{1.602399in}{2.358789in}}%
\pgfpathlineto{\pgfqpoint{1.603473in}{2.360503in}}%
\pgfpathlineto{\pgfqpoint{1.606694in}{2.359934in}}%
\pgfpathlineto{\pgfqpoint{1.610990in}{2.359722in}}%
\pgfpathlineto{\pgfqpoint{1.612063in}{2.360290in}}%
\pgfpathlineto{\pgfqpoint{1.617433in}{2.360849in}}%
\pgfpathlineto{\pgfqpoint{1.618507in}{2.361233in}}%
\pgfpathlineto{\pgfqpoint{1.619580in}{2.359530in}}%
\pgfpathlineto{\pgfqpoint{1.620654in}{2.358941in}}%
\pgfpathlineto{\pgfqpoint{1.621728in}{2.359958in}}%
\pgfpathlineto{\pgfqpoint{1.627097in}{2.360636in}}%
\pgfpathlineto{\pgfqpoint{1.628171in}{2.361741in}}%
\pgfpathlineto{\pgfqpoint{1.629245in}{2.360569in}}%
\pgfpathlineto{\pgfqpoint{1.633540in}{2.362594in}}%
\pgfpathlineto{\pgfqpoint{1.634614in}{2.362121in}}%
\pgfpathlineto{\pgfqpoint{1.635688in}{2.360695in}}%
\pgfpathlineto{\pgfqpoint{1.636762in}{2.361845in}}%
\pgfpathlineto{\pgfqpoint{1.641057in}{2.361931in}}%
\pgfpathlineto{\pgfqpoint{1.642131in}{2.362396in}}%
\pgfpathlineto{\pgfqpoint{1.643205in}{2.361699in}}%
\pgfpathlineto{\pgfqpoint{1.644279in}{2.363393in}}%
\pgfpathlineto{\pgfqpoint{1.647500in}{2.362848in}}%
\pgfpathlineto{\pgfqpoint{1.648574in}{2.361578in}}%
\pgfpathlineto{\pgfqpoint{1.649648in}{2.362406in}}%
\pgfpathlineto{\pgfqpoint{1.650722in}{2.361560in}}%
\pgfpathlineto{\pgfqpoint{1.651796in}{2.362584in}}%
\pgfpathlineto{\pgfqpoint{1.655017in}{2.363610in}}%
\pgfpathlineto{\pgfqpoint{1.657165in}{2.363133in}}%
\pgfpathlineto{\pgfqpoint{1.658239in}{2.363133in}}%
\pgfpathlineto{\pgfqpoint{1.659312in}{2.362181in}}%
\pgfpathlineto{\pgfqpoint{1.666829in}{2.361140in}}%
\pgfpathlineto{\pgfqpoint{1.671125in}{2.360794in}}%
\pgfpathlineto{\pgfqpoint{1.672198in}{2.360410in}}%
\pgfpathlineto{\pgfqpoint{1.673272in}{2.359210in}}%
\pgfpathlineto{\pgfqpoint{1.674346in}{2.359423in}}%
\pgfpathlineto{\pgfqpoint{1.677568in}{2.358904in}}%
\pgfpathlineto{\pgfqpoint{1.679715in}{2.360050in}}%
\pgfpathlineto{\pgfqpoint{1.680789in}{2.359397in}}%
\pgfpathlineto{\pgfqpoint{1.681863in}{2.361167in}}%
\pgfpathlineto{\pgfqpoint{1.685085in}{2.360744in}}%
\pgfpathlineto{\pgfqpoint{1.686158in}{2.361606in}}%
\pgfpathlineto{\pgfqpoint{1.687232in}{2.361606in}}%
\pgfpathlineto{\pgfqpoint{1.688306in}{2.360422in}}%
\pgfpathlineto{\pgfqpoint{1.689380in}{2.360893in}}%
\pgfpathlineto{\pgfqpoint{1.692601in}{2.359461in}}%
\pgfpathlineto{\pgfqpoint{1.693675in}{2.360339in}}%
\pgfpathlineto{\pgfqpoint{1.694749in}{2.359141in}}%
\pgfpathlineto{\pgfqpoint{1.700118in}{2.358024in}}%
\pgfpathlineto{\pgfqpoint{1.703340in}{2.360091in}}%
\pgfpathlineto{\pgfqpoint{1.708709in}{2.359726in}}%
\pgfpathlineto{\pgfqpoint{1.709783in}{2.360813in}}%
\pgfpathlineto{\pgfqpoint{1.710857in}{2.360440in}}%
\pgfpathlineto{\pgfqpoint{1.715152in}{2.360809in}}%
\pgfpathlineto{\pgfqpoint{1.716226in}{2.360081in}}%
\pgfpathlineto{\pgfqpoint{1.717300in}{2.360038in}}%
\pgfpathlineto{\pgfqpoint{1.719447in}{2.358606in}}%
\pgfpathlineto{\pgfqpoint{1.723743in}{2.359728in}}%
\pgfpathlineto{\pgfqpoint{1.724817in}{2.359669in}}%
\pgfpathlineto{\pgfqpoint{1.726964in}{2.360438in}}%
\pgfpathlineto{\pgfqpoint{1.734481in}{2.359305in}}%
\pgfpathlineto{\pgfqpoint{1.740924in}{2.359815in}}%
\pgfpathlineto{\pgfqpoint{1.741998in}{2.359105in}}%
\pgfpathlineto{\pgfqpoint{1.745219in}{2.358945in}}%
\pgfpathlineto{\pgfqpoint{1.747367in}{2.359572in}}%
\pgfpathlineto{\pgfqpoint{1.748441in}{2.359197in}}%
\pgfpathlineto{\pgfqpoint{1.749515in}{2.357837in}}%
\pgfpathlineto{\pgfqpoint{1.754884in}{2.358507in}}%
\pgfpathlineto{\pgfqpoint{1.757032in}{2.357222in}}%
\pgfpathlineto{\pgfqpoint{1.762401in}{2.356293in}}%
\pgfpathlineto{\pgfqpoint{1.768844in}{2.357731in}}%
\pgfpathlineto{\pgfqpoint{1.770992in}{2.357496in}}%
\pgfpathlineto{\pgfqpoint{1.772065in}{2.358223in}}%
\pgfpathlineto{\pgfqpoint{1.792468in}{2.358629in}}%
\pgfpathlineto{\pgfqpoint{1.793542in}{2.357403in}}%
\pgfpathlineto{\pgfqpoint{1.794616in}{2.357760in}}%
\pgfpathlineto{\pgfqpoint{1.808576in}{2.357436in}}%
\pgfpathlineto{\pgfqpoint{1.821462in}{2.357250in}}%
\pgfpathlineto{\pgfqpoint{1.823610in}{2.356278in}}%
\pgfpathlineto{\pgfqpoint{1.824684in}{2.357016in}}%
\pgfpathlineto{\pgfqpoint{1.832200in}{2.356927in}}%
\pgfpathlineto{\pgfqpoint{1.844013in}{2.356715in}}%
\pgfpathlineto{\pgfqpoint{1.846160in}{2.356188in}}%
\pgfpathlineto{\pgfqpoint{1.928846in}{2.356188in}}%
\pgfpathlineto{\pgfqpoint{1.929920in}{2.355620in}}%
\pgfpathlineto{\pgfqpoint{1.934215in}{2.355706in}}%
\pgfpathlineto{\pgfqpoint{1.943880in}{2.354197in}}%
\pgfpathlineto{\pgfqpoint{1.948175in}{2.355216in}}%
\pgfpathlineto{\pgfqpoint{1.950323in}{2.354503in}}%
\pgfpathlineto{\pgfqpoint{1.951396in}{2.355603in}}%
\pgfpathlineto{\pgfqpoint{1.952470in}{2.355603in}}%
\pgfpathlineto{\pgfqpoint{1.959987in}{2.353259in}}%
\pgfpathlineto{\pgfqpoint{1.963209in}{2.353059in}}%
\pgfpathlineto{\pgfqpoint{1.965356in}{2.353604in}}%
\pgfpathlineto{\pgfqpoint{1.967504in}{2.353328in}}%
\pgfpathlineto{\pgfqpoint{1.970726in}{2.354091in}}%
\pgfpathlineto{\pgfqpoint{1.971799in}{2.353318in}}%
\pgfpathlineto{\pgfqpoint{1.972873in}{2.353526in}}%
\pgfpathlineto{\pgfqpoint{1.973947in}{2.354383in}}%
\pgfpathlineto{\pgfqpoint{1.975021in}{2.352877in}}%
\pgfpathlineto{\pgfqpoint{1.978242in}{2.352628in}}%
\pgfpathlineto{\pgfqpoint{1.980390in}{2.353410in}}%
\pgfpathlineto{\pgfqpoint{1.981464in}{2.353075in}}%
\pgfpathlineto{\pgfqpoint{1.982538in}{2.353723in}}%
\pgfpathlineto{\pgfqpoint{1.985759in}{2.353400in}}%
\pgfpathlineto{\pgfqpoint{1.987907in}{2.351332in}}%
\pgfpathlineto{\pgfqpoint{1.990055in}{2.353466in}}%
\pgfpathlineto{\pgfqpoint{1.993276in}{2.353720in}}%
\pgfpathlineto{\pgfqpoint{1.995424in}{2.352424in}}%
\pgfpathlineto{\pgfqpoint{1.996498in}{2.352272in}}%
\pgfpathlineto{\pgfqpoint{2.000793in}{2.352596in}}%
\pgfpathlineto{\pgfqpoint{2.001867in}{2.352065in}}%
\pgfpathlineto{\pgfqpoint{2.004015in}{2.352921in}}%
\pgfpathlineto{\pgfqpoint{2.008310in}{2.354419in}}%
\pgfpathlineto{\pgfqpoint{2.009384in}{2.354127in}}%
\pgfpathlineto{\pgfqpoint{2.011531in}{2.354477in}}%
\pgfpathlineto{\pgfqpoint{2.030861in}{2.354477in}}%
\pgfpathlineto{\pgfqpoint{2.031934in}{2.350996in}}%
\pgfpathlineto{\pgfqpoint{2.035156in}{2.348876in}}%
\pgfpathlineto{\pgfqpoint{2.038377in}{2.348923in}}%
\pgfpathlineto{\pgfqpoint{2.039451in}{2.349516in}}%
\pgfpathlineto{\pgfqpoint{2.041599in}{2.349202in}}%
\pgfpathlineto{\pgfqpoint{2.042673in}{2.351303in}}%
\pgfpathlineto{\pgfqpoint{2.046968in}{2.350350in}}%
\pgfpathlineto{\pgfqpoint{2.050190in}{2.350449in}}%
\pgfpathlineto{\pgfqpoint{2.055559in}{2.350008in}}%
\pgfpathlineto{\pgfqpoint{2.057706in}{2.348471in}}%
\pgfpathlineto{\pgfqpoint{2.063076in}{2.347975in}}%
\pgfpathlineto{\pgfqpoint{2.064150in}{2.347108in}}%
\pgfpathlineto{\pgfqpoint{2.065223in}{2.347462in}}%
\pgfpathlineto{\pgfqpoint{2.068445in}{2.347807in}}%
\pgfpathlineto{\pgfqpoint{2.069519in}{2.346588in}}%
\pgfpathlineto{\pgfqpoint{2.077036in}{2.347174in}}%
\pgfpathlineto{\pgfqpoint{2.080257in}{2.346297in}}%
\pgfpathlineto{\pgfqpoint{2.092069in}{2.346486in}}%
\pgfpathlineto{\pgfqpoint{2.093143in}{2.345835in}}%
\pgfpathlineto{\pgfqpoint{2.098512in}{2.345990in}}%
\pgfpathlineto{\pgfqpoint{2.100660in}{2.345559in}}%
\pgfpathlineto{\pgfqpoint{2.101734in}{2.346001in}}%
\pgfpathlineto{\pgfqpoint{2.102808in}{2.345434in}}%
\pgfpathlineto{\pgfqpoint{2.107103in}{2.345566in}}%
\pgfpathlineto{\pgfqpoint{2.109251in}{2.345382in}}%
\pgfpathlineto{\pgfqpoint{2.113546in}{2.345553in}}%
\pgfpathlineto{\pgfqpoint{2.116768in}{2.345063in}}%
\pgfpathlineto{\pgfqpoint{2.121063in}{2.344462in}}%
\pgfpathlineto{\pgfqpoint{2.123211in}{2.343039in}}%
\pgfpathlineto{\pgfqpoint{2.131801in}{2.343612in}}%
\pgfpathlineto{\pgfqpoint{2.132875in}{2.343825in}}%
\pgfpathlineto{\pgfqpoint{2.146835in}{2.342684in}}%
\pgfpathlineto{\pgfqpoint{2.147909in}{2.343030in}}%
\pgfpathlineto{\pgfqpoint{2.152204in}{2.342884in}}%
\pgfpathlineto{\pgfqpoint{2.158647in}{2.343332in}}%
\pgfpathlineto{\pgfqpoint{2.161869in}{2.342942in}}%
\pgfpathlineto{\pgfqpoint{2.167238in}{2.343048in}}%
\pgfpathlineto{\pgfqpoint{2.168312in}{2.343003in}}%
\pgfpathlineto{\pgfqpoint{2.169386in}{2.342112in}}%
\pgfpathlineto{\pgfqpoint{2.182272in}{2.340800in}}%
\pgfpathlineto{\pgfqpoint{2.185493in}{2.341593in}}%
\pgfpathlineto{\pgfqpoint{2.191936in}{2.340658in}}%
\pgfpathlineto{\pgfqpoint{2.193010in}{2.341496in}}%
\pgfpathlineto{\pgfqpoint{2.196232in}{2.341040in}}%
\pgfpathlineto{\pgfqpoint{2.199453in}{2.338788in}}%
\pgfpathlineto{\pgfqpoint{2.204822in}{2.338350in}}%
\pgfpathlineto{\pgfqpoint{2.208044in}{2.338091in}}%
\pgfpathlineto{\pgfqpoint{2.218782in}{2.337968in}}%
\pgfpathlineto{\pgfqpoint{2.219856in}{2.337046in}}%
\pgfpathlineto{\pgfqpoint{2.230594in}{2.337052in}}%
\pgfpathlineto{\pgfqpoint{2.234890in}{2.337212in}}%
\pgfpathlineto{\pgfqpoint{2.237038in}{2.337630in}}%
\pgfpathlineto{\pgfqpoint{2.238111in}{2.337467in}}%
\pgfpathlineto{\pgfqpoint{2.248850in}{2.338299in}}%
\pgfpathlineto{\pgfqpoint{2.249924in}{2.339175in}}%
\pgfpathlineto{\pgfqpoint{2.250997in}{2.338730in}}%
\pgfpathlineto{\pgfqpoint{2.253145in}{2.338993in}}%
\pgfpathlineto{\pgfqpoint{2.260662in}{2.339393in}}%
\pgfpathlineto{\pgfqpoint{2.266031in}{2.339250in}}%
\pgfpathlineto{\pgfqpoint{2.268179in}{2.339259in}}%
\pgfpathlineto{\pgfqpoint{2.274622in}{2.339189in}}%
\pgfpathlineto{\pgfqpoint{2.275696in}{2.339758in}}%
\pgfpathlineto{\pgfqpoint{2.278917in}{2.339323in}}%
\pgfpathlineto{\pgfqpoint{2.279991in}{2.340039in}}%
\pgfpathlineto{\pgfqpoint{2.283213in}{2.339732in}}%
\pgfpathlineto{\pgfqpoint{2.287508in}{2.339872in}}%
\pgfpathlineto{\pgfqpoint{2.289656in}{2.339166in}}%
\pgfpathlineto{\pgfqpoint{2.293951in}{2.339942in}}%
\pgfpathlineto{\pgfqpoint{2.296099in}{2.339181in}}%
\pgfpathlineto{\pgfqpoint{2.297172in}{2.340161in}}%
\pgfpathlineto{\pgfqpoint{2.298246in}{2.339751in}}%
\pgfpathlineto{\pgfqpoint{2.308985in}{2.338968in}}%
\pgfpathlineto{\pgfqpoint{2.310059in}{2.339956in}}%
\pgfpathlineto{\pgfqpoint{2.316502in}{2.339550in}}%
\pgfpathlineto{\pgfqpoint{2.317575in}{2.340991in}}%
\pgfpathlineto{\pgfqpoint{2.320797in}{2.341964in}}%
\pgfpathlineto{\pgfqpoint{2.326166in}{2.341398in}}%
\pgfpathlineto{\pgfqpoint{2.327240in}{2.340829in}}%
\pgfpathlineto{\pgfqpoint{2.328314in}{2.341008in}}%
\pgfpathlineto{\pgfqpoint{2.332609in}{2.340999in}}%
\pgfpathlineto{\pgfqpoint{2.334757in}{2.341152in}}%
\pgfpathlineto{\pgfqpoint{2.335831in}{2.341109in}}%
\pgfpathlineto{\pgfqpoint{2.340126in}{2.340158in}}%
\pgfpathlineto{\pgfqpoint{2.341200in}{2.338601in}}%
\pgfpathlineto{\pgfqpoint{2.347643in}{2.340445in}}%
\pgfpathlineto{\pgfqpoint{2.354086in}{2.340699in}}%
\pgfpathlineto{\pgfqpoint{2.355160in}{2.341806in}}%
\pgfpathlineto{\pgfqpoint{2.358381in}{2.341100in}}%
\pgfpathlineto{\pgfqpoint{2.362677in}{2.341930in}}%
\pgfpathlineto{\pgfqpoint{2.363750in}{2.342533in}}%
\pgfpathlineto{\pgfqpoint{2.373415in}{2.342020in}}%
\pgfpathlineto{\pgfqpoint{2.377710in}{2.340166in}}%
\pgfpathlineto{\pgfqpoint{2.378784in}{2.340604in}}%
\pgfpathlineto{\pgfqpoint{2.380932in}{2.340164in}}%
\pgfpathlineto{\pgfqpoint{2.392744in}{2.340151in}}%
\pgfpathlineto{\pgfqpoint{2.394892in}{2.340350in}}%
\pgfpathlineto{\pgfqpoint{2.395966in}{2.340484in}}%
\pgfpathlineto{\pgfqpoint{2.407778in}{2.340015in}}%
\pgfpathlineto{\pgfqpoint{2.409926in}{2.340755in}}%
\pgfpathlineto{\pgfqpoint{2.421738in}{2.341108in}}%
\pgfpathlineto{\pgfqpoint{2.422812in}{2.342180in}}%
\pgfpathlineto{\pgfqpoint{2.423885in}{2.341643in}}%
\pgfpathlineto{\pgfqpoint{2.424959in}{2.342132in}}%
\pgfpathlineto{\pgfqpoint{2.426033in}{2.341330in}}%
\pgfpathlineto{\pgfqpoint{2.438919in}{2.341324in}}%
\pgfpathlineto{\pgfqpoint{2.444288in}{2.340088in}}%
\pgfpathlineto{\pgfqpoint{2.455027in}{2.337387in}}%
\pgfpathlineto{\pgfqpoint{2.456101in}{2.336585in}}%
\pgfpathlineto{\pgfqpoint{2.460396in}{2.336807in}}%
\pgfpathlineto{\pgfqpoint{2.461470in}{2.336099in}}%
\pgfpathlineto{\pgfqpoint{2.467913in}{2.336108in}}%
\pgfpathlineto{\pgfqpoint{2.468987in}{2.335997in}}%
\pgfpathlineto{\pgfqpoint{2.471134in}{2.335119in}}%
\pgfpathlineto{\pgfqpoint{2.475430in}{2.334782in}}%
\pgfpathlineto{\pgfqpoint{2.476504in}{2.334022in}}%
\pgfpathlineto{\pgfqpoint{2.478651in}{2.334372in}}%
\pgfpathlineto{\pgfqpoint{2.482947in}{2.334688in}}%
\pgfpathlineto{\pgfqpoint{2.486168in}{2.335396in}}%
\pgfpathlineto{\pgfqpoint{2.491537in}{2.335624in}}%
\pgfpathlineto{\pgfqpoint{2.497980in}{2.335720in}}%
\pgfpathlineto{\pgfqpoint{2.511940in}{2.335286in}}%
\pgfpathlineto{\pgfqpoint{2.513014in}{2.336871in}}%
\pgfpathlineto{\pgfqpoint{2.516236in}{2.336893in}}%
\pgfpathlineto{\pgfqpoint{2.523752in}{2.336158in}}%
\pgfpathlineto{\pgfqpoint{2.538786in}{2.336082in}}%
\pgfpathlineto{\pgfqpoint{2.543081in}{2.334369in}}%
\pgfpathlineto{\pgfqpoint{2.545229in}{2.334679in}}%
\pgfpathlineto{\pgfqpoint{2.558115in}{2.334290in}}%
\pgfpathlineto{\pgfqpoint{2.561337in}{2.333136in}}%
\pgfpathlineto{\pgfqpoint{2.573149in}{2.332414in}}%
\pgfpathlineto{\pgfqpoint{2.576370in}{2.331572in}}%
\pgfpathlineto{\pgfqpoint{2.581740in}{2.331374in}}%
\pgfpathlineto{\pgfqpoint{2.583887in}{2.330846in}}%
\pgfpathlineto{\pgfqpoint{2.588183in}{2.331329in}}%
\pgfpathlineto{\pgfqpoint{2.591404in}{2.332290in}}%
\pgfpathlineto{\pgfqpoint{2.598921in}{2.332196in}}%
\pgfpathlineto{\pgfqpoint{2.605364in}{2.332435in}}%
\pgfpathlineto{\pgfqpoint{2.606438in}{2.332165in}}%
\pgfpathlineto{\pgfqpoint{2.609659in}{2.332329in}}%
\pgfpathlineto{\pgfqpoint{2.611807in}{2.331222in}}%
\pgfpathlineto{\pgfqpoint{2.612881in}{2.330750in}}%
\pgfpathlineto{\pgfqpoint{2.618250in}{2.332383in}}%
\pgfpathlineto{\pgfqpoint{2.620398in}{2.332768in}}%
\pgfpathlineto{\pgfqpoint{2.621472in}{2.332389in}}%
\pgfpathlineto{\pgfqpoint{2.628989in}{2.331902in}}%
\pgfpathlineto{\pgfqpoint{2.640801in}{2.331826in}}%
\pgfpathlineto{\pgfqpoint{2.641875in}{2.331544in}}%
\pgfpathlineto{\pgfqpoint{2.644022in}{2.332089in}}%
\pgfpathlineto{\pgfqpoint{2.650465in}{2.331979in}}%
\pgfpathlineto{\pgfqpoint{2.654761in}{2.332093in}}%
\pgfpathlineto{\pgfqpoint{2.655835in}{2.331921in}}%
\pgfpathlineto{\pgfqpoint{2.656908in}{2.332333in}}%
\pgfpathlineto{\pgfqpoint{2.657982in}{2.331856in}}%
\pgfpathlineto{\pgfqpoint{2.659056in}{2.332339in}}%
\pgfpathlineto{\pgfqpoint{2.664425in}{2.332465in}}%
\pgfpathlineto{\pgfqpoint{2.665499in}{2.331904in}}%
\pgfpathlineto{\pgfqpoint{2.666573in}{2.332342in}}%
\pgfpathlineto{\pgfqpoint{2.670868in}{2.331727in}}%
\pgfpathlineto{\pgfqpoint{2.671942in}{2.331791in}}%
\pgfpathlineto{\pgfqpoint{2.674090in}{2.331075in}}%
\pgfpathlineto{\pgfqpoint{2.678385in}{2.330808in}}%
\pgfpathlineto{\pgfqpoint{2.680533in}{2.332017in}}%
\pgfpathlineto{\pgfqpoint{2.689124in}{2.332651in}}%
\pgfpathlineto{\pgfqpoint{2.700936in}{2.331237in}}%
\pgfpathlineto{\pgfqpoint{2.702010in}{2.330270in}}%
\pgfpathlineto{\pgfqpoint{2.707379in}{2.330450in}}%
\pgfpathlineto{\pgfqpoint{2.708453in}{2.328863in}}%
\pgfpathlineto{\pgfqpoint{2.710600in}{2.328468in}}%
\pgfpathlineto{\pgfqpoint{2.714896in}{2.327977in}}%
\pgfpathlineto{\pgfqpoint{2.715969in}{2.328580in}}%
\pgfpathlineto{\pgfqpoint{2.718117in}{2.328526in}}%
\pgfpathlineto{\pgfqpoint{2.722413in}{2.329089in}}%
\pgfpathlineto{\pgfqpoint{2.725634in}{2.329111in}}%
\pgfpathlineto{\pgfqpoint{2.734225in}{2.329219in}}%
\pgfpathlineto{\pgfqpoint{2.740668in}{2.329442in}}%
\pgfpathlineto{\pgfqpoint{2.741742in}{2.329725in}}%
\pgfpathlineto{\pgfqpoint{2.752480in}{2.329410in}}%
\pgfpathlineto{\pgfqpoint{2.753554in}{2.328784in}}%
\pgfpathlineto{\pgfqpoint{2.761071in}{2.328888in}}%
\pgfpathlineto{\pgfqpoint{2.762145in}{2.328446in}}%
\pgfpathlineto{\pgfqpoint{2.764292in}{2.328590in}}%
\pgfpathlineto{\pgfqpoint{2.777178in}{2.328398in}}%
\pgfpathlineto{\pgfqpoint{2.786843in}{2.328848in}}%
\pgfpathlineto{\pgfqpoint{2.786843in}{2.328848in}}%
\pgfusepath{stroke}%
\end{pgfscope}%
\begin{pgfscope}%
\pgfsetrectcap%
\pgfsetmiterjoin%
\pgfsetlinewidth{0.803000pt}%
\definecolor{currentstroke}{rgb}{1.000000,1.000000,1.000000}%
\pgfsetstrokecolor{currentstroke}%
\pgfsetdash{}{0pt}%
\pgfpathmoveto{\pgfqpoint{0.320934in}{2.309648in}}%
\pgfpathlineto{\pgfqpoint{0.320934in}{2.710533in}}%
\pgfusepath{stroke}%
\end{pgfscope}%
\begin{pgfscope}%
\pgfsetrectcap%
\pgfsetmiterjoin%
\pgfsetlinewidth{0.803000pt}%
\definecolor{currentstroke}{rgb}{1.000000,1.000000,1.000000}%
\pgfsetstrokecolor{currentstroke}%
\pgfsetdash{}{0pt}%
\pgfpathmoveto{\pgfqpoint{2.904267in}{2.309648in}}%
\pgfpathlineto{\pgfqpoint{2.904267in}{2.710533in}}%
\pgfusepath{stroke}%
\end{pgfscope}%
\begin{pgfscope}%
\pgfsetrectcap%
\pgfsetmiterjoin%
\pgfsetlinewidth{0.803000pt}%
\definecolor{currentstroke}{rgb}{1.000000,1.000000,1.000000}%
\pgfsetstrokecolor{currentstroke}%
\pgfsetdash{}{0pt}%
\pgfpathmoveto{\pgfqpoint{0.320934in}{2.309648in}}%
\pgfpathlineto{\pgfqpoint{2.904267in}{2.309648in}}%
\pgfusepath{stroke}%
\end{pgfscope}%
\begin{pgfscope}%
\pgfsetrectcap%
\pgfsetmiterjoin%
\pgfsetlinewidth{0.803000pt}%
\definecolor{currentstroke}{rgb}{1.000000,1.000000,1.000000}%
\pgfsetstrokecolor{currentstroke}%
\pgfsetdash{}{0pt}%
\pgfpathmoveto{\pgfqpoint{0.320934in}{2.710533in}}%
\pgfpathlineto{\pgfqpoint{2.904267in}{2.710533in}}%
\pgfusepath{stroke}%
\end{pgfscope}%
\begin{pgfscope}%
\definecolor{textcolor}{rgb}{0.150000,0.150000,0.150000}%
\pgfsetstrokecolor{textcolor}%
\pgfsetfillcolor{textcolor}%
\pgftext[x=1.612600in,y=2.793866in,,base]{\color{textcolor}\rmfamily\fontsize{16.800000}{20.160000}\selectfont JNJ}%
\end{pgfscope}%
\begin{pgfscope}%
\pgfsetbuttcap%
\pgfsetmiterjoin%
\definecolor{currentfill}{rgb}{0.917647,0.917647,0.949020}%
\pgfsetfillcolor{currentfill}%
\pgfsetlinewidth{0.000000pt}%
\definecolor{currentstroke}{rgb}{0.000000,0.000000,0.000000}%
\pgfsetstrokecolor{currentstroke}%
\pgfsetstrokeopacity{0.000000}%
\pgfsetdash{}{0pt}%
\pgfpathmoveto{\pgfqpoint{3.937600in}{2.309648in}}%
\pgfpathlineto{\pgfqpoint{6.520934in}{2.309648in}}%
\pgfpathlineto{\pgfqpoint{6.520934in}{2.710533in}}%
\pgfpathlineto{\pgfqpoint{3.937600in}{2.710533in}}%
\pgfpathclose%
\pgfusepath{fill}%
\end{pgfscope}%
\begin{pgfscope}%
\pgfpathrectangle{\pgfqpoint{3.937600in}{2.309648in}}{\pgfqpoint{2.583333in}{0.400885in}}%
\pgfusepath{clip}%
\pgfsetroundcap%
\pgfsetroundjoin%
\pgfsetlinewidth{0.803000pt}%
\definecolor{currentstroke}{rgb}{1.000000,1.000000,1.000000}%
\pgfsetstrokecolor{currentstroke}%
\pgfsetdash{}{0pt}%
\pgfpathmoveto{\pgfqpoint{4.052877in}{2.309648in}}%
\pgfpathlineto{\pgfqpoint{4.052877in}{2.710533in}}%
\pgfusepath{stroke}%
\end{pgfscope}%
\begin{pgfscope}%
\definecolor{textcolor}{rgb}{0.150000,0.150000,0.150000}%
\pgfsetstrokecolor{textcolor}%
\pgfsetfillcolor{textcolor}%
\pgftext[x=4.052877in,y=2.212426in,,top]{\color{textcolor}\rmfamily\fontsize{14.000000}{16.800000}\selectfont 2012}%
\end{pgfscope}%
\begin{pgfscope}%
\pgfpathrectangle{\pgfqpoint{3.937600in}{2.309648in}}{\pgfqpoint{2.583333in}{0.400885in}}%
\pgfusepath{clip}%
\pgfsetroundcap%
\pgfsetroundjoin%
\pgfsetlinewidth{0.803000pt}%
\definecolor{currentstroke}{rgb}{1.000000,1.000000,1.000000}%
\pgfsetstrokecolor{currentstroke}%
\pgfsetdash{}{0pt}%
\pgfpathmoveto{\pgfqpoint{4.445902in}{2.309648in}}%
\pgfpathlineto{\pgfqpoint{4.445902in}{2.710533in}}%
\pgfusepath{stroke}%
\end{pgfscope}%
\begin{pgfscope}%
\definecolor{textcolor}{rgb}{0.150000,0.150000,0.150000}%
\pgfsetstrokecolor{textcolor}%
\pgfsetfillcolor{textcolor}%
\pgftext[x=4.445902in,y=2.212426in,,top]{\color{textcolor}\rmfamily\fontsize{14.000000}{16.800000}\selectfont 2013}%
\end{pgfscope}%
\begin{pgfscope}%
\pgfpathrectangle{\pgfqpoint{3.937600in}{2.309648in}}{\pgfqpoint{2.583333in}{0.400885in}}%
\pgfusepath{clip}%
\pgfsetroundcap%
\pgfsetroundjoin%
\pgfsetlinewidth{0.803000pt}%
\definecolor{currentstroke}{rgb}{1.000000,1.000000,1.000000}%
\pgfsetstrokecolor{currentstroke}%
\pgfsetdash{}{0pt}%
\pgfpathmoveto{\pgfqpoint{4.837853in}{2.309648in}}%
\pgfpathlineto{\pgfqpoint{4.837853in}{2.710533in}}%
\pgfusepath{stroke}%
\end{pgfscope}%
\begin{pgfscope}%
\definecolor{textcolor}{rgb}{0.150000,0.150000,0.150000}%
\pgfsetstrokecolor{textcolor}%
\pgfsetfillcolor{textcolor}%
\pgftext[x=4.837853in,y=2.212426in,,top]{\color{textcolor}\rmfamily\fontsize{14.000000}{16.800000}\selectfont 2014}%
\end{pgfscope}%
\begin{pgfscope}%
\pgfpathrectangle{\pgfqpoint{3.937600in}{2.309648in}}{\pgfqpoint{2.583333in}{0.400885in}}%
\pgfusepath{clip}%
\pgfsetroundcap%
\pgfsetroundjoin%
\pgfsetlinewidth{0.803000pt}%
\definecolor{currentstroke}{rgb}{1.000000,1.000000,1.000000}%
\pgfsetstrokecolor{currentstroke}%
\pgfsetdash{}{0pt}%
\pgfpathmoveto{\pgfqpoint{5.229804in}{2.309648in}}%
\pgfpathlineto{\pgfqpoint{5.229804in}{2.710533in}}%
\pgfusepath{stroke}%
\end{pgfscope}%
\begin{pgfscope}%
\definecolor{textcolor}{rgb}{0.150000,0.150000,0.150000}%
\pgfsetstrokecolor{textcolor}%
\pgfsetfillcolor{textcolor}%
\pgftext[x=5.229804in,y=2.212426in,,top]{\color{textcolor}\rmfamily\fontsize{14.000000}{16.800000}\selectfont 2015}%
\end{pgfscope}%
\begin{pgfscope}%
\pgfpathrectangle{\pgfqpoint{3.937600in}{2.309648in}}{\pgfqpoint{2.583333in}{0.400885in}}%
\pgfusepath{clip}%
\pgfsetroundcap%
\pgfsetroundjoin%
\pgfsetlinewidth{0.803000pt}%
\definecolor{currentstroke}{rgb}{1.000000,1.000000,1.000000}%
\pgfsetstrokecolor{currentstroke}%
\pgfsetdash{}{0pt}%
\pgfpathmoveto{\pgfqpoint{5.621755in}{2.309648in}}%
\pgfpathlineto{\pgfqpoint{5.621755in}{2.710533in}}%
\pgfusepath{stroke}%
\end{pgfscope}%
\begin{pgfscope}%
\definecolor{textcolor}{rgb}{0.150000,0.150000,0.150000}%
\pgfsetstrokecolor{textcolor}%
\pgfsetfillcolor{textcolor}%
\pgftext[x=5.621755in,y=2.212426in,,top]{\color{textcolor}\rmfamily\fontsize{14.000000}{16.800000}\selectfont 2016}%
\end{pgfscope}%
\begin{pgfscope}%
\pgfpathrectangle{\pgfqpoint{3.937600in}{2.309648in}}{\pgfqpoint{2.583333in}{0.400885in}}%
\pgfusepath{clip}%
\pgfsetroundcap%
\pgfsetroundjoin%
\pgfsetlinewidth{0.803000pt}%
\definecolor{currentstroke}{rgb}{1.000000,1.000000,1.000000}%
\pgfsetstrokecolor{currentstroke}%
\pgfsetdash{}{0pt}%
\pgfpathmoveto{\pgfqpoint{6.014780in}{2.309648in}}%
\pgfpathlineto{\pgfqpoint{6.014780in}{2.710533in}}%
\pgfusepath{stroke}%
\end{pgfscope}%
\begin{pgfscope}%
\definecolor{textcolor}{rgb}{0.150000,0.150000,0.150000}%
\pgfsetstrokecolor{textcolor}%
\pgfsetfillcolor{textcolor}%
\pgftext[x=6.014780in,y=2.212426in,,top]{\color{textcolor}\rmfamily\fontsize{14.000000}{16.800000}\selectfont 2017}%
\end{pgfscope}%
\begin{pgfscope}%
\pgfpathrectangle{\pgfqpoint{3.937600in}{2.309648in}}{\pgfqpoint{2.583333in}{0.400885in}}%
\pgfusepath{clip}%
\pgfsetroundcap%
\pgfsetroundjoin%
\pgfsetlinewidth{0.803000pt}%
\definecolor{currentstroke}{rgb}{1.000000,1.000000,1.000000}%
\pgfsetstrokecolor{currentstroke}%
\pgfsetdash{}{0pt}%
\pgfpathmoveto{\pgfqpoint{6.406731in}{2.309648in}}%
\pgfpathlineto{\pgfqpoint{6.406731in}{2.710533in}}%
\pgfusepath{stroke}%
\end{pgfscope}%
\begin{pgfscope}%
\definecolor{textcolor}{rgb}{0.150000,0.150000,0.150000}%
\pgfsetstrokecolor{textcolor}%
\pgfsetfillcolor{textcolor}%
\pgftext[x=6.406731in,y=2.212426in,,top]{\color{textcolor}\rmfamily\fontsize{14.000000}{16.800000}\selectfont 2018}%
\end{pgfscope}%
\begin{pgfscope}%
\pgfpathrectangle{\pgfqpoint{3.937600in}{2.309648in}}{\pgfqpoint{2.583333in}{0.400885in}}%
\pgfusepath{clip}%
\pgfsetroundcap%
\pgfsetroundjoin%
\pgfsetlinewidth{0.803000pt}%
\definecolor{currentstroke}{rgb}{1.000000,1.000000,1.000000}%
\pgfsetstrokecolor{currentstroke}%
\pgfsetdash{}{0pt}%
\pgfpathmoveto{\pgfqpoint{3.937600in}{2.367612in}}%
\pgfpathlineto{\pgfqpoint{6.520934in}{2.367612in}}%
\pgfusepath{stroke}%
\end{pgfscope}%
\begin{pgfscope}%
\definecolor{textcolor}{rgb}{0.150000,0.150000,0.150000}%
\pgfsetstrokecolor{textcolor}%
\pgfsetfillcolor{textcolor}%
\pgftext[x=3.531147in,y=2.293746in,left,base]{\color{textcolor}\rmfamily\fontsize{14.000000}{16.800000}\selectfont 0.5}%
\end{pgfscope}%
\begin{pgfscope}%
\pgfpathrectangle{\pgfqpoint{3.937600in}{2.309648in}}{\pgfqpoint{2.583333in}{0.400885in}}%
\pgfusepath{clip}%
\pgfsetroundcap%
\pgfsetroundjoin%
\pgfsetlinewidth{0.803000pt}%
\definecolor{currentstroke}{rgb}{1.000000,1.000000,1.000000}%
\pgfsetstrokecolor{currentstroke}%
\pgfsetdash{}{0pt}%
\pgfpathmoveto{\pgfqpoint{3.937600in}{2.503412in}}%
\pgfpathlineto{\pgfqpoint{6.520934in}{2.503412in}}%
\pgfusepath{stroke}%
\end{pgfscope}%
\begin{pgfscope}%
\definecolor{textcolor}{rgb}{0.150000,0.150000,0.150000}%
\pgfsetstrokecolor{textcolor}%
\pgfsetfillcolor{textcolor}%
\pgftext[x=3.531147in,y=2.429546in,left,base]{\color{textcolor}\rmfamily\fontsize{14.000000}{16.800000}\selectfont 1.0}%
\end{pgfscope}%
\begin{pgfscope}%
\pgfpathrectangle{\pgfqpoint{3.937600in}{2.309648in}}{\pgfqpoint{2.583333in}{0.400885in}}%
\pgfusepath{clip}%
\pgfsetroundcap%
\pgfsetroundjoin%
\pgfsetlinewidth{0.803000pt}%
\definecolor{currentstroke}{rgb}{1.000000,1.000000,1.000000}%
\pgfsetstrokecolor{currentstroke}%
\pgfsetdash{}{0pt}%
\pgfpathmoveto{\pgfqpoint{3.937600in}{2.639212in}}%
\pgfpathlineto{\pgfqpoint{6.520934in}{2.639212in}}%
\pgfusepath{stroke}%
\end{pgfscope}%
\begin{pgfscope}%
\definecolor{textcolor}{rgb}{0.150000,0.150000,0.150000}%
\pgfsetstrokecolor{textcolor}%
\pgfsetfillcolor{textcolor}%
\pgftext[x=3.531147in,y=2.565346in,left,base]{\color{textcolor}\rmfamily\fontsize{14.000000}{16.800000}\selectfont 1.5}%
\end{pgfscope}%
\begin{pgfscope}%
\pgfpathrectangle{\pgfqpoint{3.937600in}{2.309648in}}{\pgfqpoint{2.583333in}{0.400885in}}%
\pgfusepath{clip}%
\pgfsetroundcap%
\pgfsetroundjoin%
\pgfsetlinewidth{1.505625pt}%
\definecolor{currentstroke}{rgb}{0.000000,0.000000,0.000000}%
\pgfsetstrokecolor{currentstroke}%
\pgfsetdash{}{0pt}%
\pgfpathmoveto{\pgfqpoint{4.055025in}{2.503412in}}%
\pgfpathlineto{\pgfqpoint{4.056098in}{2.503256in}}%
\pgfpathlineto{\pgfqpoint{4.058246in}{2.501489in}}%
\pgfpathlineto{\pgfqpoint{4.061468in}{2.502633in}}%
\pgfpathlineto{\pgfqpoint{4.062542in}{2.501333in}}%
\pgfpathlineto{\pgfqpoint{4.063615in}{2.498734in}}%
\pgfpathlineto{\pgfqpoint{4.065763in}{2.499254in}}%
\pgfpathlineto{\pgfqpoint{4.070058in}{2.501073in}}%
\pgfpathlineto{\pgfqpoint{4.071132in}{2.502269in}}%
\pgfpathlineto{\pgfqpoint{4.073280in}{2.503101in}}%
\pgfpathlineto{\pgfqpoint{4.077575in}{2.496031in}}%
\pgfpathlineto{\pgfqpoint{4.078649in}{2.498006in}}%
\pgfpathlineto{\pgfqpoint{4.079723in}{2.497227in}}%
\pgfpathlineto{\pgfqpoint{4.080797in}{2.495199in}}%
\pgfpathlineto{\pgfqpoint{4.085092in}{2.490053in}}%
\pgfpathlineto{\pgfqpoint{4.087240in}{2.491197in}}%
\pgfpathlineto{\pgfqpoint{4.088314in}{2.488910in}}%
\pgfpathlineto{\pgfqpoint{4.092609in}{2.492756in}}%
\pgfpathlineto{\pgfqpoint{4.093683in}{2.492496in}}%
\pgfpathlineto{\pgfqpoint{4.094757in}{2.494108in}}%
\pgfpathlineto{\pgfqpoint{4.095831in}{2.493484in}}%
\pgfpathlineto{\pgfqpoint{4.101200in}{2.496239in}}%
\pgfpathlineto{\pgfqpoint{4.102274in}{2.498890in}}%
\pgfpathlineto{\pgfqpoint{4.103347in}{2.497695in}}%
\pgfpathlineto{\pgfqpoint{4.108717in}{2.495771in}}%
\pgfpathlineto{\pgfqpoint{4.109790in}{2.503880in}}%
\pgfpathlineto{\pgfqpoint{4.110864in}{2.505076in}}%
\pgfpathlineto{\pgfqpoint{4.114086in}{2.505024in}}%
\pgfpathlineto{\pgfqpoint{4.115160in}{2.507831in}}%
\pgfpathlineto{\pgfqpoint{4.116233in}{2.508818in}}%
\pgfpathlineto{\pgfqpoint{4.117307in}{2.504868in}}%
\pgfpathlineto{\pgfqpoint{4.118381in}{2.504920in}}%
\pgfpathlineto{\pgfqpoint{4.121603in}{2.506063in}}%
\pgfpathlineto{\pgfqpoint{4.123750in}{2.504556in}}%
\pgfpathlineto{\pgfqpoint{4.124824in}{2.505856in}}%
\pgfpathlineto{\pgfqpoint{4.125898in}{2.505959in}}%
\pgfpathlineto{\pgfqpoint{4.130193in}{2.509962in}}%
\pgfpathlineto{\pgfqpoint{4.132341in}{2.509026in}}%
\pgfpathlineto{\pgfqpoint{4.133415in}{2.507259in}}%
\pgfpathlineto{\pgfqpoint{4.138784in}{2.507051in}}%
\pgfpathlineto{\pgfqpoint{4.139858in}{2.508403in}}%
\pgfpathlineto{\pgfqpoint{4.140932in}{2.508039in}}%
\pgfpathlineto{\pgfqpoint{4.144153in}{2.508143in}}%
\pgfpathlineto{\pgfqpoint{4.145227in}{2.506895in}}%
\pgfpathlineto{\pgfqpoint{4.146301in}{2.507051in}}%
\pgfpathlineto{\pgfqpoint{4.147375in}{2.506271in}}%
\pgfpathlineto{\pgfqpoint{4.148449in}{2.507103in}}%
\pgfpathlineto{\pgfqpoint{4.151670in}{2.508559in}}%
\pgfpathlineto{\pgfqpoint{4.152744in}{2.506635in}}%
\pgfpathlineto{\pgfqpoint{4.154892in}{2.507519in}}%
\pgfpathlineto{\pgfqpoint{4.159187in}{2.505492in}}%
\pgfpathlineto{\pgfqpoint{4.160261in}{2.503568in}}%
\pgfpathlineto{\pgfqpoint{4.161335in}{2.503984in}}%
\pgfpathlineto{\pgfqpoint{4.163482in}{2.501385in}}%
\pgfpathlineto{\pgfqpoint{4.167778in}{2.506323in}}%
\pgfpathlineto{\pgfqpoint{4.169925in}{2.504504in}}%
\pgfpathlineto{\pgfqpoint{4.170999in}{2.508351in}}%
\pgfpathlineto{\pgfqpoint{4.174221in}{2.504816in}}%
\pgfpathlineto{\pgfqpoint{4.176368in}{2.508143in}}%
\pgfpathlineto{\pgfqpoint{4.177442in}{2.508039in}}%
\pgfpathlineto{\pgfqpoint{4.178516in}{2.498006in}}%
\pgfpathlineto{\pgfqpoint{4.181738in}{2.494680in}}%
\pgfpathlineto{\pgfqpoint{4.182811in}{2.494420in}}%
\pgfpathlineto{\pgfqpoint{4.184959in}{2.498266in}}%
\pgfpathlineto{\pgfqpoint{4.186033in}{2.497331in}}%
\pgfpathlineto{\pgfqpoint{4.190328in}{2.496863in}}%
\pgfpathlineto{\pgfqpoint{4.191402in}{2.494836in}}%
\pgfpathlineto{\pgfqpoint{4.192476in}{2.496759in}}%
\pgfpathlineto{\pgfqpoint{4.193550in}{2.494888in}}%
\pgfpathlineto{\pgfqpoint{4.196771in}{2.494472in}}%
\pgfpathlineto{\pgfqpoint{4.197845in}{2.495043in}}%
\pgfpathlineto{\pgfqpoint{4.198919in}{2.497383in}}%
\pgfpathlineto{\pgfqpoint{4.201067in}{2.494212in}}%
\pgfpathlineto{\pgfqpoint{4.204288in}{2.493640in}}%
\pgfpathlineto{\pgfqpoint{4.205362in}{2.492652in}}%
\pgfpathlineto{\pgfqpoint{4.206436in}{2.489534in}}%
\pgfpathlineto{\pgfqpoint{4.207510in}{2.490261in}}%
\pgfpathlineto{\pgfqpoint{4.208584in}{2.489949in}}%
\pgfpathlineto{\pgfqpoint{4.212879in}{2.491873in}}%
\pgfpathlineto{\pgfqpoint{4.213953in}{2.489222in}}%
\pgfpathlineto{\pgfqpoint{4.215027in}{2.489118in}}%
\pgfpathlineto{\pgfqpoint{4.216100in}{2.486051in}}%
\pgfpathlineto{\pgfqpoint{4.219322in}{2.485427in}}%
\pgfpathlineto{\pgfqpoint{4.220396in}{2.484491in}}%
\pgfpathlineto{\pgfqpoint{4.223617in}{2.491041in}}%
\pgfpathlineto{\pgfqpoint{4.226839in}{2.490157in}}%
\pgfpathlineto{\pgfqpoint{4.227913in}{2.491041in}}%
\pgfpathlineto{\pgfqpoint{4.228986in}{2.490261in}}%
\pgfpathlineto{\pgfqpoint{4.230060in}{2.492808in}}%
\pgfpathlineto{\pgfqpoint{4.231134in}{2.491561in}}%
\pgfpathlineto{\pgfqpoint{4.235430in}{2.488806in}}%
\pgfpathlineto{\pgfqpoint{4.236503in}{2.481269in}}%
\pgfpathlineto{\pgfqpoint{4.237577in}{2.478618in}}%
\pgfpathlineto{\pgfqpoint{4.238651in}{2.478981in}}%
\pgfpathlineto{\pgfqpoint{4.242946in}{2.476642in}}%
\pgfpathlineto{\pgfqpoint{4.244020in}{2.479605in}}%
\pgfpathlineto{\pgfqpoint{4.245094in}{2.480853in}}%
\pgfpathlineto{\pgfqpoint{4.246168in}{2.484803in}}%
\pgfpathlineto{\pgfqpoint{4.249389in}{2.484595in}}%
\pgfpathlineto{\pgfqpoint{4.250463in}{2.485271in}}%
\pgfpathlineto{\pgfqpoint{4.253685in}{2.484959in}}%
\pgfpathlineto{\pgfqpoint{4.257980in}{2.486831in}}%
\pgfpathlineto{\pgfqpoint{4.259054in}{2.485427in}}%
\pgfpathlineto{\pgfqpoint{4.261202in}{2.500709in}}%
\pgfpathlineto{\pgfqpoint{4.264423in}{2.499514in}}%
\pgfpathlineto{\pgfqpoint{4.265497in}{2.501749in}}%
\pgfpathlineto{\pgfqpoint{4.267645in}{2.502321in}}%
\pgfpathlineto{\pgfqpoint{4.268719in}{2.501541in}}%
\pgfpathlineto{\pgfqpoint{4.271940in}{2.500086in}}%
\pgfpathlineto{\pgfqpoint{4.273014in}{2.498578in}}%
\pgfpathlineto{\pgfqpoint{4.274088in}{2.498578in}}%
\pgfpathlineto{\pgfqpoint{4.276235in}{2.502997in}}%
\pgfpathlineto{\pgfqpoint{4.279457in}{2.503049in}}%
\pgfpathlineto{\pgfqpoint{4.282678in}{2.496447in}}%
\pgfpathlineto{\pgfqpoint{4.283752in}{2.504712in}}%
\pgfpathlineto{\pgfqpoint{4.286974in}{2.506011in}}%
\pgfpathlineto{\pgfqpoint{4.289121in}{2.509806in}}%
\pgfpathlineto{\pgfqpoint{4.291269in}{2.510014in}}%
\pgfpathlineto{\pgfqpoint{4.294491in}{2.508818in}}%
\pgfpathlineto{\pgfqpoint{4.295564in}{2.509858in}}%
\pgfpathlineto{\pgfqpoint{4.296638in}{2.509494in}}%
\pgfpathlineto{\pgfqpoint{4.297712in}{2.511002in}}%
\pgfpathlineto{\pgfqpoint{4.298786in}{2.511002in}}%
\pgfpathlineto{\pgfqpoint{4.303081in}{2.510014in}}%
\pgfpathlineto{\pgfqpoint{4.304155in}{2.510378in}}%
\pgfpathlineto{\pgfqpoint{4.305229in}{2.509650in}}%
\pgfpathlineto{\pgfqpoint{4.306303in}{2.511054in}}%
\pgfpathlineto{\pgfqpoint{4.309524in}{2.511417in}}%
\pgfpathlineto{\pgfqpoint{4.312746in}{2.510482in}}%
\pgfpathlineto{\pgfqpoint{4.313820in}{2.511781in}}%
\pgfpathlineto{\pgfqpoint{4.318115in}{2.512665in}}%
\pgfpathlineto{\pgfqpoint{4.319189in}{2.512197in}}%
\pgfpathlineto{\pgfqpoint{4.320263in}{2.516148in}}%
\pgfpathlineto{\pgfqpoint{4.321337in}{2.517291in}}%
\pgfpathlineto{\pgfqpoint{4.324558in}{2.517291in}}%
\pgfpathlineto{\pgfqpoint{4.326706in}{2.515576in}}%
\pgfpathlineto{\pgfqpoint{4.327780in}{2.518955in}}%
\pgfpathlineto{\pgfqpoint{4.328853in}{2.519994in}}%
\pgfpathlineto{\pgfqpoint{4.334223in}{2.520410in}}%
\pgfpathlineto{\pgfqpoint{4.335297in}{2.521658in}}%
\pgfpathlineto{\pgfqpoint{4.336370in}{2.521086in}}%
\pgfpathlineto{\pgfqpoint{4.339592in}{2.522489in}}%
\pgfpathlineto{\pgfqpoint{4.342813in}{2.520566in}}%
\pgfpathlineto{\pgfqpoint{4.347109in}{2.521190in}}%
\pgfpathlineto{\pgfqpoint{4.348183in}{2.518435in}}%
\pgfpathlineto{\pgfqpoint{4.350330in}{2.520878in}}%
\pgfpathlineto{\pgfqpoint{4.351404in}{2.521918in}}%
\pgfpathlineto{\pgfqpoint{4.354626in}{2.519734in}}%
\pgfpathlineto{\pgfqpoint{4.357847in}{2.515160in}}%
\pgfpathlineto{\pgfqpoint{4.358921in}{2.514900in}}%
\pgfpathlineto{\pgfqpoint{4.363216in}{2.519319in}}%
\pgfpathlineto{\pgfqpoint{4.364290in}{2.523633in}}%
\pgfpathlineto{\pgfqpoint{4.365364in}{2.523633in}}%
\pgfpathlineto{\pgfqpoint{4.366438in}{2.519890in}}%
\pgfpathlineto{\pgfqpoint{4.369659in}{2.519578in}}%
\pgfpathlineto{\pgfqpoint{4.370733in}{2.515108in}}%
\pgfpathlineto{\pgfqpoint{4.371807in}{2.517811in}}%
\pgfpathlineto{\pgfqpoint{4.372881in}{2.526180in}}%
\pgfpathlineto{\pgfqpoint{4.373955in}{2.523529in}}%
\pgfpathlineto{\pgfqpoint{4.381472in}{2.522489in}}%
\pgfpathlineto{\pgfqpoint{4.384693in}{2.520306in}}%
\pgfpathlineto{\pgfqpoint{4.385767in}{2.521398in}}%
\pgfpathlineto{\pgfqpoint{4.387915in}{2.512873in}}%
\pgfpathlineto{\pgfqpoint{4.388988in}{2.513289in}}%
\pgfpathlineto{\pgfqpoint{4.392210in}{2.513601in}}%
\pgfpathlineto{\pgfqpoint{4.395431in}{2.510430in}}%
\pgfpathlineto{\pgfqpoint{4.396505in}{2.512509in}}%
\pgfpathlineto{\pgfqpoint{4.404022in}{2.524153in}}%
\pgfpathlineto{\pgfqpoint{4.407244in}{2.523685in}}%
\pgfpathlineto{\pgfqpoint{4.408318in}{2.521606in}}%
\pgfpathlineto{\pgfqpoint{4.409391in}{2.523529in}}%
\pgfpathlineto{\pgfqpoint{4.410465in}{2.523789in}}%
\pgfpathlineto{\pgfqpoint{4.411539in}{2.525140in}}%
\pgfpathlineto{\pgfqpoint{4.414761in}{2.524101in}}%
\pgfpathlineto{\pgfqpoint{4.415834in}{2.522957in}}%
\pgfpathlineto{\pgfqpoint{4.416908in}{2.523373in}}%
\pgfpathlineto{\pgfqpoint{4.419056in}{2.527116in}}%
\pgfpathlineto{\pgfqpoint{4.422277in}{2.526804in}}%
\pgfpathlineto{\pgfqpoint{4.423351in}{2.528623in}}%
\pgfpathlineto{\pgfqpoint{4.424425in}{2.529091in}}%
\pgfpathlineto{\pgfqpoint{4.425499in}{2.526440in}}%
\pgfpathlineto{\pgfqpoint{4.426573in}{2.525556in}}%
\pgfpathlineto{\pgfqpoint{4.430868in}{2.525764in}}%
\pgfpathlineto{\pgfqpoint{4.431942in}{2.523113in}}%
\pgfpathlineto{\pgfqpoint{4.433016in}{2.525140in}}%
\pgfpathlineto{\pgfqpoint{4.434090in}{2.520514in}}%
\pgfpathlineto{\pgfqpoint{4.437311in}{2.519630in}}%
\pgfpathlineto{\pgfqpoint{4.439459in}{2.517447in}}%
\pgfpathlineto{\pgfqpoint{4.440533in}{2.517343in}}%
\pgfpathlineto{\pgfqpoint{4.441607in}{2.513913in}}%
\pgfpathlineto{\pgfqpoint{4.444828in}{2.517031in}}%
\pgfpathlineto{\pgfqpoint{4.446976in}{2.523321in}}%
\pgfpathlineto{\pgfqpoint{4.448050in}{2.521450in}}%
\pgfpathlineto{\pgfqpoint{4.449123in}{2.522074in}}%
\pgfpathlineto{\pgfqpoint{4.453419in}{2.519630in}}%
\pgfpathlineto{\pgfqpoint{4.455566in}{2.522801in}}%
\pgfpathlineto{\pgfqpoint{4.456640in}{2.522593in}}%
\pgfpathlineto{\pgfqpoint{4.462009in}{2.525452in}}%
\pgfpathlineto{\pgfqpoint{4.464157in}{2.527999in}}%
\pgfpathlineto{\pgfqpoint{4.468452in}{2.528051in}}%
\pgfpathlineto{\pgfqpoint{4.469526in}{2.531170in}}%
\pgfpathlineto{\pgfqpoint{4.470600in}{2.530027in}}%
\pgfpathlineto{\pgfqpoint{4.471674in}{2.542034in}}%
\pgfpathlineto{\pgfqpoint{4.474896in}{2.544217in}}%
\pgfpathlineto{\pgfqpoint{4.475969in}{2.549416in}}%
\pgfpathlineto{\pgfqpoint{4.478117in}{2.550091in}}%
\pgfpathlineto{\pgfqpoint{4.479191in}{2.553314in}}%
\pgfpathlineto{\pgfqpoint{4.482412in}{2.550507in}}%
\pgfpathlineto{\pgfqpoint{4.484560in}{2.554302in}}%
\pgfpathlineto{\pgfqpoint{4.485634in}{2.554302in}}%
\pgfpathlineto{\pgfqpoint{4.486708in}{2.552586in}}%
\pgfpathlineto{\pgfqpoint{4.489929in}{2.552846in}}%
\pgfpathlineto{\pgfqpoint{4.491003in}{2.553574in}}%
\pgfpathlineto{\pgfqpoint{4.492077in}{2.556017in}}%
\pgfpathlineto{\pgfqpoint{4.493151in}{2.556953in}}%
\pgfpathlineto{\pgfqpoint{4.494225in}{2.555965in}}%
\pgfpathlineto{\pgfqpoint{4.498520in}{2.559500in}}%
\pgfpathlineto{\pgfqpoint{4.499594in}{2.558252in}}%
\pgfpathlineto{\pgfqpoint{4.501741in}{2.557836in}}%
\pgfpathlineto{\pgfqpoint{4.504963in}{2.553314in}}%
\pgfpathlineto{\pgfqpoint{4.506037in}{2.553990in}}%
\pgfpathlineto{\pgfqpoint{4.507111in}{2.556849in}}%
\pgfpathlineto{\pgfqpoint{4.508185in}{2.554406in}}%
\pgfpathlineto{\pgfqpoint{4.509258in}{2.555757in}}%
\pgfpathlineto{\pgfqpoint{4.512480in}{2.556537in}}%
\pgfpathlineto{\pgfqpoint{4.514628in}{2.558772in}}%
\pgfpathlineto{\pgfqpoint{4.515701in}{2.557473in}}%
\pgfpathlineto{\pgfqpoint{4.516775in}{2.558668in}}%
\pgfpathlineto{\pgfqpoint{4.519997in}{2.559396in}}%
\pgfpathlineto{\pgfqpoint{4.521071in}{2.558616in}}%
\pgfpathlineto{\pgfqpoint{4.522144in}{2.557057in}}%
\pgfpathlineto{\pgfqpoint{4.523218in}{2.559552in}}%
\pgfpathlineto{\pgfqpoint{4.524292in}{2.555081in}}%
\pgfpathlineto{\pgfqpoint{4.527514in}{2.554354in}}%
\pgfpathlineto{\pgfqpoint{4.529661in}{2.560383in}}%
\pgfpathlineto{\pgfqpoint{4.530735in}{2.558772in}}%
\pgfpathlineto{\pgfqpoint{4.531809in}{2.559032in}}%
\pgfpathlineto{\pgfqpoint{4.535030in}{2.556537in}}%
\pgfpathlineto{\pgfqpoint{4.536104in}{2.559604in}}%
\pgfpathlineto{\pgfqpoint{4.537178in}{2.558148in}}%
\pgfpathlineto{\pgfqpoint{4.538252in}{2.558148in}}%
\pgfpathlineto{\pgfqpoint{4.542547in}{2.560851in}}%
\pgfpathlineto{\pgfqpoint{4.543621in}{2.566205in}}%
\pgfpathlineto{\pgfqpoint{4.544695in}{2.562671in}}%
\pgfpathlineto{\pgfqpoint{4.545769in}{2.564438in}}%
\pgfpathlineto{\pgfqpoint{4.546843in}{2.563086in}}%
\pgfpathlineto{\pgfqpoint{4.550064in}{2.565478in}}%
\pgfpathlineto{\pgfqpoint{4.551138in}{2.563242in}}%
\pgfpathlineto{\pgfqpoint{4.552212in}{2.567401in}}%
\pgfpathlineto{\pgfqpoint{4.554360in}{2.570936in}}%
\pgfpathlineto{\pgfqpoint{4.557581in}{2.569116in}}%
\pgfpathlineto{\pgfqpoint{4.558655in}{2.571040in}}%
\pgfpathlineto{\pgfqpoint{4.559729in}{2.566621in}}%
\pgfpathlineto{\pgfqpoint{4.560803in}{2.570052in}}%
\pgfpathlineto{\pgfqpoint{4.561876in}{2.576653in}}%
\pgfpathlineto{\pgfqpoint{4.565098in}{2.576550in}}%
\pgfpathlineto{\pgfqpoint{4.566172in}{2.581384in}}%
\pgfpathlineto{\pgfqpoint{4.567246in}{2.560799in}}%
\pgfpathlineto{\pgfqpoint{4.568319in}{2.558512in}}%
\pgfpathlineto{\pgfqpoint{4.569393in}{2.560747in}}%
\pgfpathlineto{\pgfqpoint{4.572615in}{2.563190in}}%
\pgfpathlineto{\pgfqpoint{4.573689in}{2.559292in}}%
\pgfpathlineto{\pgfqpoint{4.574763in}{2.560228in}}%
\pgfpathlineto{\pgfqpoint{4.576910in}{2.565374in}}%
\pgfpathlineto{\pgfqpoint{4.580132in}{2.563502in}}%
\pgfpathlineto{\pgfqpoint{4.581206in}{2.564282in}}%
\pgfpathlineto{\pgfqpoint{4.582279in}{2.566413in}}%
\pgfpathlineto{\pgfqpoint{4.583353in}{2.565478in}}%
\pgfpathlineto{\pgfqpoint{4.584427in}{2.567817in}}%
\pgfpathlineto{\pgfqpoint{4.587649in}{2.567089in}}%
\pgfpathlineto{\pgfqpoint{4.589796in}{2.575978in}}%
\pgfpathlineto{\pgfqpoint{4.590870in}{2.573950in}}%
\pgfpathlineto{\pgfqpoint{4.591944in}{2.573171in}}%
\pgfpathlineto{\pgfqpoint{4.596239in}{2.567973in}}%
\pgfpathlineto{\pgfqpoint{4.597313in}{2.568077in}}%
\pgfpathlineto{\pgfqpoint{4.598387in}{2.567557in}}%
\pgfpathlineto{\pgfqpoint{4.599461in}{2.581124in}}%
\pgfpathlineto{\pgfqpoint{4.603756in}{2.576757in}}%
\pgfpathlineto{\pgfqpoint{4.604830in}{2.568389in}}%
\pgfpathlineto{\pgfqpoint{4.605904in}{2.569220in}}%
\pgfpathlineto{\pgfqpoint{4.606978in}{2.559292in}}%
\pgfpathlineto{\pgfqpoint{4.610199in}{2.563086in}}%
\pgfpathlineto{\pgfqpoint{4.611273in}{2.561891in}}%
\pgfpathlineto{\pgfqpoint{4.612347in}{2.558824in}}%
\pgfpathlineto{\pgfqpoint{4.613421in}{2.559552in}}%
\pgfpathlineto{\pgfqpoint{4.614495in}{2.563502in}}%
\pgfpathlineto{\pgfqpoint{4.618790in}{2.565062in}}%
\pgfpathlineto{\pgfqpoint{4.619864in}{2.562879in}}%
\pgfpathlineto{\pgfqpoint{4.620938in}{2.566413in}}%
\pgfpathlineto{\pgfqpoint{4.622011in}{2.564698in}}%
\pgfpathlineto{\pgfqpoint{4.625233in}{2.568648in}}%
\pgfpathlineto{\pgfqpoint{4.626307in}{2.569012in}}%
\pgfpathlineto{\pgfqpoint{4.627381in}{2.562879in}}%
\pgfpathlineto{\pgfqpoint{4.628454in}{2.552846in}}%
\pgfpathlineto{\pgfqpoint{4.629528in}{2.562151in}}%
\pgfpathlineto{\pgfqpoint{4.632750in}{2.558512in}}%
\pgfpathlineto{\pgfqpoint{4.633824in}{2.558980in}}%
\pgfpathlineto{\pgfqpoint{4.634897in}{2.561995in}}%
\pgfpathlineto{\pgfqpoint{4.635971in}{2.563138in}}%
\pgfpathlineto{\pgfqpoint{4.637045in}{2.560228in}}%
\pgfpathlineto{\pgfqpoint{4.642414in}{2.566985in}}%
\pgfpathlineto{\pgfqpoint{4.644562in}{2.565997in}}%
\pgfpathlineto{\pgfqpoint{4.647784in}{2.567817in}}%
\pgfpathlineto{\pgfqpoint{4.648857in}{2.571247in}}%
\pgfpathlineto{\pgfqpoint{4.649931in}{2.572235in}}%
\pgfpathlineto{\pgfqpoint{4.652079in}{2.579720in}}%
\pgfpathlineto{\pgfqpoint{4.655300in}{2.579512in}}%
\pgfpathlineto{\pgfqpoint{4.657448in}{2.576030in}}%
\pgfpathlineto{\pgfqpoint{4.658522in}{2.576913in}}%
\pgfpathlineto{\pgfqpoint{4.659596in}{2.581540in}}%
\pgfpathlineto{\pgfqpoint{4.662817in}{2.580916in}}%
\pgfpathlineto{\pgfqpoint{4.663891in}{2.579720in}}%
\pgfpathlineto{\pgfqpoint{4.664965in}{2.576809in}}%
\pgfpathlineto{\pgfqpoint{4.666039in}{2.577433in}}%
\pgfpathlineto{\pgfqpoint{4.667113in}{2.577329in}}%
\pgfpathlineto{\pgfqpoint{4.670334in}{2.576030in}}%
\pgfpathlineto{\pgfqpoint{4.671408in}{2.577537in}}%
\pgfpathlineto{\pgfqpoint{4.672482in}{2.576913in}}%
\pgfpathlineto{\pgfqpoint{4.673556in}{2.582683in}}%
\pgfpathlineto{\pgfqpoint{4.674629in}{2.581176in}}%
\pgfpathlineto{\pgfqpoint{4.677851in}{2.581644in}}%
\pgfpathlineto{\pgfqpoint{4.679999in}{2.584087in}}%
\pgfpathlineto{\pgfqpoint{4.681073in}{2.584970in}}%
\pgfpathlineto{\pgfqpoint{4.682146in}{2.582683in}}%
\pgfpathlineto{\pgfqpoint{4.686442in}{2.582787in}}%
\pgfpathlineto{\pgfqpoint{4.687516in}{2.581020in}}%
\pgfpathlineto{\pgfqpoint{4.689663in}{2.575198in}}%
\pgfpathlineto{\pgfqpoint{4.695032in}{2.572963in}}%
\pgfpathlineto{\pgfqpoint{4.697180in}{2.575666in}}%
\pgfpathlineto{\pgfqpoint{4.701475in}{2.566933in}}%
\pgfpathlineto{\pgfqpoint{4.702549in}{2.562099in}}%
\pgfpathlineto{\pgfqpoint{4.704697in}{2.566569in}}%
\pgfpathlineto{\pgfqpoint{4.708992in}{2.565997in}}%
\pgfpathlineto{\pgfqpoint{4.711140in}{2.563346in}}%
\pgfpathlineto{\pgfqpoint{4.712214in}{2.563398in}}%
\pgfpathlineto{\pgfqpoint{4.715435in}{2.567765in}}%
\pgfpathlineto{\pgfqpoint{4.716509in}{2.566829in}}%
\pgfpathlineto{\pgfqpoint{4.717583in}{2.568233in}}%
\pgfpathlineto{\pgfqpoint{4.718657in}{2.568181in}}%
\pgfpathlineto{\pgfqpoint{4.719731in}{2.571559in}}%
\pgfpathlineto{\pgfqpoint{4.722952in}{2.576342in}}%
\pgfpathlineto{\pgfqpoint{4.724026in}{2.574938in}}%
\pgfpathlineto{\pgfqpoint{4.725100in}{2.576913in}}%
\pgfpathlineto{\pgfqpoint{4.726174in}{2.576186in}}%
\pgfpathlineto{\pgfqpoint{4.727248in}{2.573015in}}%
\pgfpathlineto{\pgfqpoint{4.730469in}{2.572547in}}%
\pgfpathlineto{\pgfqpoint{4.732617in}{2.565841in}}%
\pgfpathlineto{\pgfqpoint{4.733691in}{2.567245in}}%
\pgfpathlineto{\pgfqpoint{4.734764in}{2.563658in}}%
\pgfpathlineto{\pgfqpoint{4.737986in}{2.556693in}}%
\pgfpathlineto{\pgfqpoint{4.739060in}{2.559136in}}%
\pgfpathlineto{\pgfqpoint{4.741207in}{2.557784in}}%
\pgfpathlineto{\pgfqpoint{4.742281in}{2.558564in}}%
\pgfpathlineto{\pgfqpoint{4.745503in}{2.556953in}}%
\pgfpathlineto{\pgfqpoint{4.749798in}{2.569116in}}%
\pgfpathlineto{\pgfqpoint{4.753020in}{2.570208in}}%
\pgfpathlineto{\pgfqpoint{4.754094in}{2.565322in}}%
\pgfpathlineto{\pgfqpoint{4.756241in}{2.575822in}}%
\pgfpathlineto{\pgfqpoint{4.757315in}{2.575770in}}%
\pgfpathlineto{\pgfqpoint{4.760537in}{2.573899in}}%
\pgfpathlineto{\pgfqpoint{4.761610in}{2.579980in}}%
\pgfpathlineto{\pgfqpoint{4.762684in}{2.582267in}}%
\pgfpathlineto{\pgfqpoint{4.763758in}{2.580968in}}%
\pgfpathlineto{\pgfqpoint{4.764832in}{2.578317in}}%
\pgfpathlineto{\pgfqpoint{4.768053in}{2.583983in}}%
\pgfpathlineto{\pgfqpoint{4.769127in}{2.588973in}}%
\pgfpathlineto{\pgfqpoint{4.771275in}{2.581592in}}%
\pgfpathlineto{\pgfqpoint{4.772349in}{2.583307in}}%
\pgfpathlineto{\pgfqpoint{4.776644in}{2.584503in}}%
\pgfpathlineto{\pgfqpoint{4.777718in}{2.590480in}}%
\pgfpathlineto{\pgfqpoint{4.778792in}{2.588401in}}%
\pgfpathlineto{\pgfqpoint{4.779866in}{2.589181in}}%
\pgfpathlineto{\pgfqpoint{4.783087in}{2.588141in}}%
\pgfpathlineto{\pgfqpoint{4.785235in}{2.593495in}}%
\pgfpathlineto{\pgfqpoint{4.787383in}{2.599317in}}%
\pgfpathlineto{\pgfqpoint{4.791678in}{2.597134in}}%
\pgfpathlineto{\pgfqpoint{4.792752in}{2.598641in}}%
\pgfpathlineto{\pgfqpoint{4.793826in}{2.598537in}}%
\pgfpathlineto{\pgfqpoint{4.794899in}{2.599785in}}%
\pgfpathlineto{\pgfqpoint{4.798121in}{2.601760in}}%
\pgfpathlineto{\pgfqpoint{4.800269in}{2.596874in}}%
\pgfpathlineto{\pgfqpoint{4.802416in}{2.596614in}}%
\pgfpathlineto{\pgfqpoint{4.805638in}{2.592820in}}%
\pgfpathlineto{\pgfqpoint{4.806712in}{2.594899in}}%
\pgfpathlineto{\pgfqpoint{4.808859in}{2.589961in}}%
\pgfpathlineto{\pgfqpoint{4.809933in}{2.597914in}}%
\pgfpathlineto{\pgfqpoint{4.813155in}{2.599057in}}%
\pgfpathlineto{\pgfqpoint{4.814228in}{2.594119in}}%
\pgfpathlineto{\pgfqpoint{4.815302in}{2.595730in}}%
\pgfpathlineto{\pgfqpoint{4.816376in}{2.588297in}}%
\pgfpathlineto{\pgfqpoint{4.817450in}{2.588609in}}%
\pgfpathlineto{\pgfqpoint{4.820672in}{2.585646in}}%
\pgfpathlineto{\pgfqpoint{4.821745in}{2.582267in}}%
\pgfpathlineto{\pgfqpoint{4.822819in}{2.588661in}}%
\pgfpathlineto{\pgfqpoint{4.823893in}{2.586582in}}%
\pgfpathlineto{\pgfqpoint{4.824967in}{2.586322in}}%
\pgfpathlineto{\pgfqpoint{4.828188in}{2.583983in}}%
\pgfpathlineto{\pgfqpoint{4.829262in}{2.583983in}}%
\pgfpathlineto{\pgfqpoint{4.832484in}{2.587050in}}%
\pgfpathlineto{\pgfqpoint{4.835705in}{2.586998in}}%
\pgfpathlineto{\pgfqpoint{4.838927in}{2.580656in}}%
\pgfpathlineto{\pgfqpoint{4.840001in}{2.580292in}}%
\pgfpathlineto{\pgfqpoint{4.843222in}{2.581124in}}%
\pgfpathlineto{\pgfqpoint{4.844296in}{2.584503in}}%
\pgfpathlineto{\pgfqpoint{4.845370in}{2.579356in}}%
\pgfpathlineto{\pgfqpoint{4.846444in}{2.580136in}}%
\pgfpathlineto{\pgfqpoint{4.850739in}{2.578369in}}%
\pgfpathlineto{\pgfqpoint{4.851813in}{2.582111in}}%
\pgfpathlineto{\pgfqpoint{4.852887in}{2.581748in}}%
\pgfpathlineto{\pgfqpoint{4.853961in}{2.580760in}}%
\pgfpathlineto{\pgfqpoint{4.855034in}{2.577797in}}%
\pgfpathlineto{\pgfqpoint{4.859330in}{2.579097in}}%
\pgfpathlineto{\pgfqpoint{4.860404in}{2.577589in}}%
\pgfpathlineto{\pgfqpoint{4.861477in}{2.573275in}}%
\pgfpathlineto{\pgfqpoint{4.862551in}{2.577381in}}%
\pgfpathlineto{\pgfqpoint{4.865773in}{2.574262in}}%
\pgfpathlineto{\pgfqpoint{4.866847in}{2.577069in}}%
\pgfpathlineto{\pgfqpoint{4.868994in}{2.567297in}}%
\pgfpathlineto{\pgfqpoint{4.873290in}{2.562203in}}%
\pgfpathlineto{\pgfqpoint{4.877585in}{2.569220in}}%
\pgfpathlineto{\pgfqpoint{4.880806in}{2.572339in}}%
\pgfpathlineto{\pgfqpoint{4.881880in}{2.575874in}}%
\pgfpathlineto{\pgfqpoint{4.882954in}{2.570000in}}%
\pgfpathlineto{\pgfqpoint{4.884028in}{2.571351in}}%
\pgfpathlineto{\pgfqpoint{4.885102in}{2.578317in}}%
\pgfpathlineto{\pgfqpoint{4.889397in}{2.572079in}}%
\pgfpathlineto{\pgfqpoint{4.890471in}{2.572859in}}%
\pgfpathlineto{\pgfqpoint{4.891545in}{2.571871in}}%
\pgfpathlineto{\pgfqpoint{4.892619in}{2.572079in}}%
\pgfpathlineto{\pgfqpoint{4.895840in}{2.571611in}}%
\pgfpathlineto{\pgfqpoint{4.896914in}{2.572755in}}%
\pgfpathlineto{\pgfqpoint{4.897988in}{2.571611in}}%
\pgfpathlineto{\pgfqpoint{4.900136in}{2.575094in}}%
\pgfpathlineto{\pgfqpoint{4.903357in}{2.569948in}}%
\pgfpathlineto{\pgfqpoint{4.904431in}{2.574210in}}%
\pgfpathlineto{\pgfqpoint{4.905505in}{2.571455in}}%
\pgfpathlineto{\pgfqpoint{4.907652in}{2.573899in}}%
\pgfpathlineto{\pgfqpoint{4.910874in}{2.574522in}}%
\pgfpathlineto{\pgfqpoint{4.913022in}{2.577641in}}%
\pgfpathlineto{\pgfqpoint{4.914095in}{2.577433in}}%
\pgfpathlineto{\pgfqpoint{4.915169in}{2.576498in}}%
\pgfpathlineto{\pgfqpoint{4.918391in}{2.580240in}}%
\pgfpathlineto{\pgfqpoint{4.919465in}{2.579928in}}%
\pgfpathlineto{\pgfqpoint{4.920539in}{2.575614in}}%
\pgfpathlineto{\pgfqpoint{4.922686in}{2.571715in}}%
\pgfpathlineto{\pgfqpoint{4.926982in}{2.580136in}}%
\pgfpathlineto{\pgfqpoint{4.928055in}{2.578785in}}%
\pgfpathlineto{\pgfqpoint{4.930203in}{2.579928in}}%
\pgfpathlineto{\pgfqpoint{4.933425in}{2.583567in}}%
\pgfpathlineto{\pgfqpoint{4.935572in}{2.581540in}}%
\pgfpathlineto{\pgfqpoint{4.936646in}{2.581384in}}%
\pgfpathlineto{\pgfqpoint{4.937720in}{2.579928in}}%
\pgfpathlineto{\pgfqpoint{4.940941in}{2.583099in}}%
\pgfpathlineto{\pgfqpoint{4.942015in}{2.586842in}}%
\pgfpathlineto{\pgfqpoint{4.943089in}{2.587466in}}%
\pgfpathlineto{\pgfqpoint{4.945237in}{2.584295in}}%
\pgfpathlineto{\pgfqpoint{4.949532in}{2.584607in}}%
\pgfpathlineto{\pgfqpoint{4.950606in}{2.588141in}}%
\pgfpathlineto{\pgfqpoint{4.951680in}{2.588609in}}%
\pgfpathlineto{\pgfqpoint{4.955975in}{2.587777in}}%
\pgfpathlineto{\pgfqpoint{4.958123in}{2.585334in}}%
\pgfpathlineto{\pgfqpoint{4.959197in}{2.588817in}}%
\pgfpathlineto{\pgfqpoint{4.960271in}{2.589961in}}%
\pgfpathlineto{\pgfqpoint{4.963492in}{2.596666in}}%
\pgfpathlineto{\pgfqpoint{4.964566in}{2.594483in}}%
\pgfpathlineto{\pgfqpoint{4.965640in}{2.594951in}}%
\pgfpathlineto{\pgfqpoint{4.966714in}{2.594015in}}%
\pgfpathlineto{\pgfqpoint{4.967787in}{2.592196in}}%
\pgfpathlineto{\pgfqpoint{4.971009in}{2.591208in}}%
\pgfpathlineto{\pgfqpoint{4.972083in}{2.588713in}}%
\pgfpathlineto{\pgfqpoint{4.973157in}{2.592924in}}%
\pgfpathlineto{\pgfqpoint{4.974230in}{2.593235in}}%
\pgfpathlineto{\pgfqpoint{4.975304in}{2.594275in}}%
\pgfpathlineto{\pgfqpoint{4.980673in}{2.588869in}}%
\pgfpathlineto{\pgfqpoint{4.981747in}{2.586062in}}%
\pgfpathlineto{\pgfqpoint{4.982821in}{2.585178in}}%
\pgfpathlineto{\pgfqpoint{4.986043in}{2.583411in}}%
\pgfpathlineto{\pgfqpoint{4.989264in}{2.586582in}}%
\pgfpathlineto{\pgfqpoint{4.994633in}{2.584087in}}%
\pgfpathlineto{\pgfqpoint{4.995707in}{2.584191in}}%
\pgfpathlineto{\pgfqpoint{4.997855in}{2.587206in}}%
\pgfpathlineto{\pgfqpoint{5.001076in}{2.585334in}}%
\pgfpathlineto{\pgfqpoint{5.002150in}{2.583411in}}%
\pgfpathlineto{\pgfqpoint{5.003224in}{2.583151in}}%
\pgfpathlineto{\pgfqpoint{5.004298in}{2.584243in}}%
\pgfpathlineto{\pgfqpoint{5.005372in}{2.583879in}}%
\pgfpathlineto{\pgfqpoint{5.010741in}{2.583983in}}%
\pgfpathlineto{\pgfqpoint{5.012889in}{2.582163in}}%
\pgfpathlineto{\pgfqpoint{5.018258in}{2.582839in}}%
\pgfpathlineto{\pgfqpoint{5.019332in}{2.584814in}}%
\pgfpathlineto{\pgfqpoint{5.020405in}{2.583411in}}%
\pgfpathlineto{\pgfqpoint{5.023627in}{2.581644in}}%
\pgfpathlineto{\pgfqpoint{5.024701in}{2.579408in}}%
\pgfpathlineto{\pgfqpoint{5.025775in}{2.580760in}}%
\pgfpathlineto{\pgfqpoint{5.026849in}{2.577693in}}%
\pgfpathlineto{\pgfqpoint{5.027922in}{2.579408in}}%
\pgfpathlineto{\pgfqpoint{5.031144in}{2.577537in}}%
\pgfpathlineto{\pgfqpoint{5.032218in}{2.580552in}}%
\pgfpathlineto{\pgfqpoint{5.034365in}{2.583671in}}%
\pgfpathlineto{\pgfqpoint{5.038661in}{2.584555in}}%
\pgfpathlineto{\pgfqpoint{5.039735in}{2.586218in}}%
\pgfpathlineto{\pgfqpoint{5.040808in}{2.591104in}}%
\pgfpathlineto{\pgfqpoint{5.041882in}{2.590844in}}%
\pgfpathlineto{\pgfqpoint{5.042956in}{2.588869in}}%
\pgfpathlineto{\pgfqpoint{5.047251in}{2.589285in}}%
\pgfpathlineto{\pgfqpoint{5.048325in}{2.590740in}}%
\pgfpathlineto{\pgfqpoint{5.049399in}{2.588349in}}%
\pgfpathlineto{\pgfqpoint{5.050473in}{2.588973in}}%
\pgfpathlineto{\pgfqpoint{5.055842in}{2.586530in}}%
\pgfpathlineto{\pgfqpoint{5.056916in}{2.587725in}}%
\pgfpathlineto{\pgfqpoint{5.057990in}{2.584607in}}%
\pgfpathlineto{\pgfqpoint{5.061211in}{2.583255in}}%
\pgfpathlineto{\pgfqpoint{5.064433in}{2.574678in}}%
\pgfpathlineto{\pgfqpoint{5.065507in}{2.585022in}}%
\pgfpathlineto{\pgfqpoint{5.068728in}{2.583099in}}%
\pgfpathlineto{\pgfqpoint{5.069802in}{2.583931in}}%
\pgfpathlineto{\pgfqpoint{5.070876in}{2.591364in}}%
\pgfpathlineto{\pgfqpoint{5.071950in}{2.587154in}}%
\pgfpathlineto{\pgfqpoint{5.073024in}{2.590740in}}%
\pgfpathlineto{\pgfqpoint{5.076245in}{2.593131in}}%
\pgfpathlineto{\pgfqpoint{5.078393in}{2.593131in}}%
\pgfpathlineto{\pgfqpoint{5.079467in}{2.595211in}}%
\pgfpathlineto{\pgfqpoint{5.080540in}{2.594431in}}%
\pgfpathlineto{\pgfqpoint{5.088057in}{2.601604in}}%
\pgfpathlineto{\pgfqpoint{5.091279in}{2.602228in}}%
\pgfpathlineto{\pgfqpoint{5.094500in}{2.599993in}}%
\pgfpathlineto{\pgfqpoint{5.095574in}{2.600357in}}%
\pgfpathlineto{\pgfqpoint{5.100943in}{2.599421in}}%
\pgfpathlineto{\pgfqpoint{5.102017in}{2.602956in}}%
\pgfpathlineto{\pgfqpoint{5.103091in}{2.603268in}}%
\pgfpathlineto{\pgfqpoint{5.106313in}{2.601292in}}%
\pgfpathlineto{\pgfqpoint{5.107386in}{2.599785in}}%
\pgfpathlineto{\pgfqpoint{5.108460in}{2.602696in}}%
\pgfpathlineto{\pgfqpoint{5.110608in}{2.600981in}}%
\pgfpathlineto{\pgfqpoint{5.115977in}{2.604983in}}%
\pgfpathlineto{\pgfqpoint{5.117051in}{2.605139in}}%
\pgfpathlineto{\pgfqpoint{5.118125in}{2.606387in}}%
\pgfpathlineto{\pgfqpoint{5.121346in}{2.607894in}}%
\pgfpathlineto{\pgfqpoint{5.122420in}{2.606231in}}%
\pgfpathlineto{\pgfqpoint{5.123494in}{2.609765in}}%
\pgfpathlineto{\pgfqpoint{5.124568in}{2.605763in}}%
\pgfpathlineto{\pgfqpoint{5.125642in}{2.606854in}}%
\pgfpathlineto{\pgfqpoint{5.128863in}{2.606231in}}%
\pgfpathlineto{\pgfqpoint{5.131011in}{2.600461in}}%
\pgfpathlineto{\pgfqpoint{5.132085in}{2.600097in}}%
\pgfpathlineto{\pgfqpoint{5.133159in}{2.603372in}}%
\pgfpathlineto{\pgfqpoint{5.136380in}{2.602384in}}%
\pgfpathlineto{\pgfqpoint{5.137454in}{2.600565in}}%
\pgfpathlineto{\pgfqpoint{5.138528in}{2.605087in}}%
\pgfpathlineto{\pgfqpoint{5.139602in}{2.602800in}}%
\pgfpathlineto{\pgfqpoint{5.140675in}{2.607322in}}%
\pgfpathlineto{\pgfqpoint{5.143897in}{2.601500in}}%
\pgfpathlineto{\pgfqpoint{5.144971in}{2.602280in}}%
\pgfpathlineto{\pgfqpoint{5.147118in}{2.596458in}}%
\pgfpathlineto{\pgfqpoint{5.148192in}{2.601033in}}%
\pgfpathlineto{\pgfqpoint{5.153561in}{2.608154in}}%
\pgfpathlineto{\pgfqpoint{5.154635in}{2.603684in}}%
\pgfpathlineto{\pgfqpoint{5.155709in}{2.612312in}}%
\pgfpathlineto{\pgfqpoint{5.158931in}{2.615847in}}%
\pgfpathlineto{\pgfqpoint{5.160004in}{2.618186in}}%
\pgfpathlineto{\pgfqpoint{5.161078in}{2.618498in}}%
\pgfpathlineto{\pgfqpoint{5.163226in}{2.621773in}}%
\pgfpathlineto{\pgfqpoint{5.166448in}{2.622241in}}%
\pgfpathlineto{\pgfqpoint{5.167521in}{2.627855in}}%
\pgfpathlineto{\pgfqpoint{5.168595in}{2.629466in}}%
\pgfpathlineto{\pgfqpoint{5.169669in}{2.629050in}}%
\pgfpathlineto{\pgfqpoint{5.170743in}{2.630090in}}%
\pgfpathlineto{\pgfqpoint{5.175038in}{2.632481in}}%
\pgfpathlineto{\pgfqpoint{5.176112in}{2.631649in}}%
\pgfpathlineto{\pgfqpoint{5.178260in}{2.625516in}}%
\pgfpathlineto{\pgfqpoint{5.181481in}{2.624320in}}%
\pgfpathlineto{\pgfqpoint{5.182555in}{2.624788in}}%
\pgfpathlineto{\pgfqpoint{5.183629in}{2.628271in}}%
\pgfpathlineto{\pgfqpoint{5.184703in}{2.627127in}}%
\pgfpathlineto{\pgfqpoint{5.185777in}{2.627699in}}%
\pgfpathlineto{\pgfqpoint{5.188998in}{2.625516in}}%
\pgfpathlineto{\pgfqpoint{5.190072in}{2.628582in}}%
\pgfpathlineto{\pgfqpoint{5.191146in}{2.628946in}}%
\pgfpathlineto{\pgfqpoint{5.193293in}{2.635860in}}%
\pgfpathlineto{\pgfqpoint{5.196515in}{2.634300in}}%
\pgfpathlineto{\pgfqpoint{5.197589in}{2.638719in}}%
\pgfpathlineto{\pgfqpoint{5.198663in}{2.633936in}}%
\pgfpathlineto{\pgfqpoint{5.199737in}{2.636535in}}%
\pgfpathlineto{\pgfqpoint{5.200810in}{2.635652in}}%
\pgfpathlineto{\pgfqpoint{5.204032in}{2.637367in}}%
\pgfpathlineto{\pgfqpoint{5.205106in}{2.637107in}}%
\pgfpathlineto{\pgfqpoint{5.206180in}{2.633936in}}%
\pgfpathlineto{\pgfqpoint{5.207253in}{2.635808in}}%
\pgfpathlineto{\pgfqpoint{5.208327in}{2.631961in}}%
\pgfpathlineto{\pgfqpoint{5.211549in}{2.630402in}}%
\pgfpathlineto{\pgfqpoint{5.212623in}{2.631077in}}%
\pgfpathlineto{\pgfqpoint{5.214770in}{2.642877in}}%
\pgfpathlineto{\pgfqpoint{5.215844in}{2.643137in}}%
\pgfpathlineto{\pgfqpoint{5.219066in}{2.645580in}}%
\pgfpathlineto{\pgfqpoint{5.220139in}{2.648647in}}%
\pgfpathlineto{\pgfqpoint{5.221213in}{2.647971in}}%
\pgfpathlineto{\pgfqpoint{5.223361in}{2.649427in}}%
\pgfpathlineto{\pgfqpoint{5.227656in}{2.644696in}}%
\pgfpathlineto{\pgfqpoint{5.228730in}{2.638823in}}%
\pgfpathlineto{\pgfqpoint{5.230878in}{2.635912in}}%
\pgfpathlineto{\pgfqpoint{5.234099in}{2.633988in}}%
\pgfpathlineto{\pgfqpoint{5.235173in}{2.632169in}}%
\pgfpathlineto{\pgfqpoint{5.236247in}{2.634248in}}%
\pgfpathlineto{\pgfqpoint{5.237321in}{2.638875in}}%
\pgfpathlineto{\pgfqpoint{5.238395in}{2.635080in}}%
\pgfpathlineto{\pgfqpoint{5.241616in}{2.633573in}}%
\pgfpathlineto{\pgfqpoint{5.242690in}{2.635288in}}%
\pgfpathlineto{\pgfqpoint{5.244838in}{2.633313in}}%
\pgfpathlineto{\pgfqpoint{5.245912in}{2.639550in}}%
\pgfpathlineto{\pgfqpoint{5.250207in}{2.639290in}}%
\pgfpathlineto{\pgfqpoint{5.251281in}{2.640070in}}%
\pgfpathlineto{\pgfqpoint{5.252355in}{2.644125in}}%
\pgfpathlineto{\pgfqpoint{5.253428in}{2.637159in}}%
\pgfpathlineto{\pgfqpoint{5.256650in}{2.634924in}}%
\pgfpathlineto{\pgfqpoint{5.257724in}{2.620993in}}%
\pgfpathlineto{\pgfqpoint{5.258798in}{2.614963in}}%
\pgfpathlineto{\pgfqpoint{5.259871in}{2.617303in}}%
\pgfpathlineto{\pgfqpoint{5.260945in}{2.611117in}}%
\pgfpathlineto{\pgfqpoint{5.264167in}{2.614859in}}%
\pgfpathlineto{\pgfqpoint{5.265241in}{2.618602in}}%
\pgfpathlineto{\pgfqpoint{5.266315in}{2.617874in}}%
\pgfpathlineto{\pgfqpoint{5.267388in}{2.621981in}}%
\pgfpathlineto{\pgfqpoint{5.268462in}{2.617043in}}%
\pgfpathlineto{\pgfqpoint{5.271684in}{2.614548in}}%
\pgfpathlineto{\pgfqpoint{5.274905in}{2.618966in}}%
\pgfpathlineto{\pgfqpoint{5.275979in}{2.618342in}}%
\pgfpathlineto{\pgfqpoint{5.280274in}{2.616523in}}%
\pgfpathlineto{\pgfqpoint{5.281348in}{2.620006in}}%
\pgfpathlineto{\pgfqpoint{5.282422in}{2.615275in}}%
\pgfpathlineto{\pgfqpoint{5.283496in}{2.613716in}}%
\pgfpathlineto{\pgfqpoint{5.287791in}{2.616575in}}%
\pgfpathlineto{\pgfqpoint{5.288865in}{2.616367in}}%
\pgfpathlineto{\pgfqpoint{5.289939in}{2.615067in}}%
\pgfpathlineto{\pgfqpoint{5.291013in}{2.614911in}}%
\pgfpathlineto{\pgfqpoint{5.294234in}{2.616159in}}%
\pgfpathlineto{\pgfqpoint{5.295308in}{2.615015in}}%
\pgfpathlineto{\pgfqpoint{5.296382in}{2.611377in}}%
\pgfpathlineto{\pgfqpoint{5.297456in}{2.612624in}}%
\pgfpathlineto{\pgfqpoint{5.298530in}{2.603788in}}%
\pgfpathlineto{\pgfqpoint{5.301751in}{2.605711in}}%
\pgfpathlineto{\pgfqpoint{5.302825in}{2.598745in}}%
\pgfpathlineto{\pgfqpoint{5.303899in}{2.598070in}}%
\pgfpathlineto{\pgfqpoint{5.304973in}{2.601240in}}%
\pgfpathlineto{\pgfqpoint{5.306047in}{2.600045in}}%
\pgfpathlineto{\pgfqpoint{5.309268in}{2.607842in}}%
\pgfpathlineto{\pgfqpoint{5.310342in}{2.604619in}}%
\pgfpathlineto{\pgfqpoint{5.311416in}{2.608674in}}%
\pgfpathlineto{\pgfqpoint{5.312490in}{2.607010in}}%
\pgfpathlineto{\pgfqpoint{5.313563in}{2.613144in}}%
\pgfpathlineto{\pgfqpoint{5.316785in}{2.613664in}}%
\pgfpathlineto{\pgfqpoint{5.320006in}{2.601500in}}%
\pgfpathlineto{\pgfqpoint{5.324302in}{2.604047in}}%
\pgfpathlineto{\pgfqpoint{5.325376in}{2.600565in}}%
\pgfpathlineto{\pgfqpoint{5.326449in}{2.602228in}}%
\pgfpathlineto{\pgfqpoint{5.327523in}{2.602748in}}%
\pgfpathlineto{\pgfqpoint{5.331819in}{2.605503in}}%
\pgfpathlineto{\pgfqpoint{5.332893in}{2.602540in}}%
\pgfpathlineto{\pgfqpoint{5.333966in}{2.604255in}}%
\pgfpathlineto{\pgfqpoint{5.335040in}{2.604827in}}%
\pgfpathlineto{\pgfqpoint{5.336114in}{2.606906in}}%
\pgfpathlineto{\pgfqpoint{5.339336in}{2.607270in}}%
\pgfpathlineto{\pgfqpoint{5.340409in}{2.607998in}}%
\pgfpathlineto{\pgfqpoint{5.342557in}{2.607582in}}%
\pgfpathlineto{\pgfqpoint{5.343631in}{2.603216in}}%
\pgfpathlineto{\pgfqpoint{5.349000in}{2.605711in}}%
\pgfpathlineto{\pgfqpoint{5.350074in}{2.599005in}}%
\pgfpathlineto{\pgfqpoint{5.351148in}{2.599265in}}%
\pgfpathlineto{\pgfqpoint{5.355443in}{2.596614in}}%
\pgfpathlineto{\pgfqpoint{5.357591in}{2.592508in}}%
\pgfpathlineto{\pgfqpoint{5.358665in}{2.596042in}}%
\pgfpathlineto{\pgfqpoint{5.361886in}{2.596302in}}%
\pgfpathlineto{\pgfqpoint{5.362960in}{2.595003in}}%
\pgfpathlineto{\pgfqpoint{5.364034in}{2.596510in}}%
\pgfpathlineto{\pgfqpoint{5.365108in}{2.595627in}}%
\pgfpathlineto{\pgfqpoint{5.366181in}{2.599057in}}%
\pgfpathlineto{\pgfqpoint{5.371551in}{2.593339in}}%
\pgfpathlineto{\pgfqpoint{5.373698in}{2.599473in}}%
\pgfpathlineto{\pgfqpoint{5.376920in}{2.598070in}}%
\pgfpathlineto{\pgfqpoint{5.377994in}{2.598485in}}%
\pgfpathlineto{\pgfqpoint{5.379068in}{2.596874in}}%
\pgfpathlineto{\pgfqpoint{5.380141in}{2.596562in}}%
\pgfpathlineto{\pgfqpoint{5.381215in}{2.594483in}}%
\pgfpathlineto{\pgfqpoint{5.385511in}{2.590792in}}%
\pgfpathlineto{\pgfqpoint{5.386584in}{2.591936in}}%
\pgfpathlineto{\pgfqpoint{5.387658in}{2.591676in}}%
\pgfpathlineto{\pgfqpoint{5.388732in}{2.587414in}}%
\pgfpathlineto{\pgfqpoint{5.391954in}{2.589493in}}%
\pgfpathlineto{\pgfqpoint{5.393027in}{2.588089in}}%
\pgfpathlineto{\pgfqpoint{5.394101in}{2.588193in}}%
\pgfpathlineto{\pgfqpoint{5.395175in}{2.586322in}}%
\pgfpathlineto{\pgfqpoint{5.396249in}{2.583047in}}%
\pgfpathlineto{\pgfqpoint{5.399470in}{2.584295in}}%
\pgfpathlineto{\pgfqpoint{5.401618in}{2.592612in}}%
\pgfpathlineto{\pgfqpoint{5.402692in}{2.592040in}}%
\pgfpathlineto{\pgfqpoint{5.403766in}{2.589597in}}%
\pgfpathlineto{\pgfqpoint{5.406987in}{2.586166in}}%
\pgfpathlineto{\pgfqpoint{5.410209in}{2.598433in}}%
\pgfpathlineto{\pgfqpoint{5.411283in}{2.597134in}}%
\pgfpathlineto{\pgfqpoint{5.414504in}{2.596770in}}%
\pgfpathlineto{\pgfqpoint{5.415578in}{2.593755in}}%
\pgfpathlineto{\pgfqpoint{5.417726in}{2.591936in}}%
\pgfpathlineto{\pgfqpoint{5.418800in}{2.591728in}}%
\pgfpathlineto{\pgfqpoint{5.422021in}{2.587102in}}%
\pgfpathlineto{\pgfqpoint{5.423095in}{2.586738in}}%
\pgfpathlineto{\pgfqpoint{5.424169in}{2.593443in}}%
\pgfpathlineto{\pgfqpoint{5.425243in}{2.594379in}}%
\pgfpathlineto{\pgfqpoint{5.429538in}{2.594951in}}%
\pgfpathlineto{\pgfqpoint{5.430612in}{2.602488in}}%
\pgfpathlineto{\pgfqpoint{5.432759in}{2.597706in}}%
\pgfpathlineto{\pgfqpoint{5.438129in}{2.603943in}}%
\pgfpathlineto{\pgfqpoint{5.440276in}{2.605139in}}%
\pgfpathlineto{\pgfqpoint{5.444572in}{2.604619in}}%
\pgfpathlineto{\pgfqpoint{5.445646in}{2.602124in}}%
\pgfpathlineto{\pgfqpoint{5.447793in}{2.600877in}}%
\pgfpathlineto{\pgfqpoint{5.448867in}{2.599005in}}%
\pgfpathlineto{\pgfqpoint{5.452089in}{2.597550in}}%
\pgfpathlineto{\pgfqpoint{5.454236in}{2.600513in}}%
\pgfpathlineto{\pgfqpoint{5.455310in}{2.585750in}}%
\pgfpathlineto{\pgfqpoint{5.456384in}{2.582579in}}%
\pgfpathlineto{\pgfqpoint{5.459605in}{2.581228in}}%
\pgfpathlineto{\pgfqpoint{5.460679in}{2.578993in}}%
\pgfpathlineto{\pgfqpoint{5.463901in}{2.577017in}}%
\pgfpathlineto{\pgfqpoint{5.467122in}{2.581124in}}%
\pgfpathlineto{\pgfqpoint{5.468196in}{2.580396in}}%
\pgfpathlineto{\pgfqpoint{5.469270in}{2.581176in}}%
\pgfpathlineto{\pgfqpoint{5.470344in}{2.578369in}}%
\pgfpathlineto{\pgfqpoint{5.471418in}{2.577641in}}%
\pgfpathlineto{\pgfqpoint{5.474639in}{2.577225in}}%
\pgfpathlineto{\pgfqpoint{5.475713in}{2.575406in}}%
\pgfpathlineto{\pgfqpoint{5.476787in}{2.570780in}}%
\pgfpathlineto{\pgfqpoint{5.477861in}{2.569844in}}%
\pgfpathlineto{\pgfqpoint{5.478935in}{2.560383in}}%
\pgfpathlineto{\pgfqpoint{5.483230in}{2.544737in}}%
\pgfpathlineto{\pgfqpoint{5.484304in}{2.556069in}}%
\pgfpathlineto{\pgfqpoint{5.485378in}{2.558720in}}%
\pgfpathlineto{\pgfqpoint{5.486451in}{2.557473in}}%
\pgfpathlineto{\pgfqpoint{5.489673in}{2.555029in}}%
\pgfpathlineto{\pgfqpoint{5.490747in}{2.546920in}}%
\pgfpathlineto{\pgfqpoint{5.491821in}{2.551079in}}%
\pgfpathlineto{\pgfqpoint{5.492894in}{2.551599in}}%
\pgfpathlineto{\pgfqpoint{5.493968in}{2.546297in}}%
\pgfpathlineto{\pgfqpoint{5.498264in}{2.551859in}}%
\pgfpathlineto{\pgfqpoint{5.499337in}{2.544997in}}%
\pgfpathlineto{\pgfqpoint{5.500411in}{2.544269in}}%
\pgfpathlineto{\pgfqpoint{5.501485in}{2.544737in}}%
\pgfpathlineto{\pgfqpoint{5.504707in}{2.543074in}}%
\pgfpathlineto{\pgfqpoint{5.505781in}{2.549416in}}%
\pgfpathlineto{\pgfqpoint{5.506854in}{2.552378in}}%
\pgfpathlineto{\pgfqpoint{5.507928in}{2.553054in}}%
\pgfpathlineto{\pgfqpoint{5.509002in}{2.551651in}}%
\pgfpathlineto{\pgfqpoint{5.512224in}{2.554925in}}%
\pgfpathlineto{\pgfqpoint{5.513297in}{2.552794in}}%
\pgfpathlineto{\pgfqpoint{5.514371in}{2.553158in}}%
\pgfpathlineto{\pgfqpoint{5.516519in}{2.564178in}}%
\pgfpathlineto{\pgfqpoint{5.519740in}{2.560020in}}%
\pgfpathlineto{\pgfqpoint{5.520814in}{2.562359in}}%
\pgfpathlineto{\pgfqpoint{5.521888in}{2.560799in}}%
\pgfpathlineto{\pgfqpoint{5.522962in}{2.560851in}}%
\pgfpathlineto{\pgfqpoint{5.524036in}{2.563034in}}%
\pgfpathlineto{\pgfqpoint{5.529405in}{2.568960in}}%
\pgfpathlineto{\pgfqpoint{5.530479in}{2.572079in}}%
\pgfpathlineto{\pgfqpoint{5.531553in}{2.572443in}}%
\pgfpathlineto{\pgfqpoint{5.534774in}{2.571767in}}%
\pgfpathlineto{\pgfqpoint{5.535848in}{2.570728in}}%
\pgfpathlineto{\pgfqpoint{5.537996in}{2.571455in}}%
\pgfpathlineto{\pgfqpoint{5.539069in}{2.574366in}}%
\pgfpathlineto{\pgfqpoint{5.542291in}{2.575562in}}%
\pgfpathlineto{\pgfqpoint{5.543365in}{2.572183in}}%
\pgfpathlineto{\pgfqpoint{5.544439in}{2.571403in}}%
\pgfpathlineto{\pgfqpoint{5.545513in}{2.577225in}}%
\pgfpathlineto{\pgfqpoint{5.546586in}{2.587258in}}%
\pgfpathlineto{\pgfqpoint{5.549808in}{2.589389in}}%
\pgfpathlineto{\pgfqpoint{5.550882in}{2.588453in}}%
\pgfpathlineto{\pgfqpoint{5.551956in}{2.584866in}}%
\pgfpathlineto{\pgfqpoint{5.553029in}{2.587206in}}%
\pgfpathlineto{\pgfqpoint{5.554103in}{2.584243in}}%
\pgfpathlineto{\pgfqpoint{5.557325in}{2.585282in}}%
\pgfpathlineto{\pgfqpoint{5.558399in}{2.587362in}}%
\pgfpathlineto{\pgfqpoint{5.559472in}{2.587414in}}%
\pgfpathlineto{\pgfqpoint{5.561620in}{2.580500in}}%
\pgfpathlineto{\pgfqpoint{5.564842in}{2.579720in}}%
\pgfpathlineto{\pgfqpoint{5.566989in}{2.582319in}}%
\pgfpathlineto{\pgfqpoint{5.569137in}{2.573119in}}%
\pgfpathlineto{\pgfqpoint{5.572358in}{2.579253in}}%
\pgfpathlineto{\pgfqpoint{5.573432in}{2.578213in}}%
\pgfpathlineto{\pgfqpoint{5.574506in}{2.582060in}}%
\pgfpathlineto{\pgfqpoint{5.575580in}{2.583515in}}%
\pgfpathlineto{\pgfqpoint{5.576654in}{2.581696in}}%
\pgfpathlineto{\pgfqpoint{5.579875in}{2.582371in}}%
\pgfpathlineto{\pgfqpoint{5.580949in}{2.584607in}}%
\pgfpathlineto{\pgfqpoint{5.582023in}{2.582060in}}%
\pgfpathlineto{\pgfqpoint{5.584171in}{2.581124in}}%
\pgfpathlineto{\pgfqpoint{5.587392in}{2.577173in}}%
\pgfpathlineto{\pgfqpoint{5.588466in}{2.582215in}}%
\pgfpathlineto{\pgfqpoint{5.590614in}{2.581332in}}%
\pgfpathlineto{\pgfqpoint{5.591688in}{2.590948in}}%
\pgfpathlineto{\pgfqpoint{5.594909in}{2.593443in}}%
\pgfpathlineto{\pgfqpoint{5.595983in}{2.590584in}}%
\pgfpathlineto{\pgfqpoint{5.597057in}{2.590376in}}%
\pgfpathlineto{\pgfqpoint{5.599204in}{2.590740in}}%
\pgfpathlineto{\pgfqpoint{5.602426in}{2.593079in}}%
\pgfpathlineto{\pgfqpoint{5.604574in}{2.605555in}}%
\pgfpathlineto{\pgfqpoint{5.605647in}{2.602280in}}%
\pgfpathlineto{\pgfqpoint{5.606721in}{2.592352in}}%
\pgfpathlineto{\pgfqpoint{5.609943in}{2.596094in}}%
\pgfpathlineto{\pgfqpoint{5.612091in}{2.600617in}}%
\pgfpathlineto{\pgfqpoint{5.613164in}{2.599993in}}%
\pgfpathlineto{\pgfqpoint{5.617460in}{2.600617in}}%
\pgfpathlineto{\pgfqpoint{5.618534in}{2.602644in}}%
\pgfpathlineto{\pgfqpoint{5.619607in}{2.601292in}}%
\pgfpathlineto{\pgfqpoint{5.620681in}{2.598226in}}%
\pgfpathlineto{\pgfqpoint{5.624977in}{2.593443in}}%
\pgfpathlineto{\pgfqpoint{5.626050in}{2.594587in}}%
\pgfpathlineto{\pgfqpoint{5.629272in}{2.582371in}}%
\pgfpathlineto{\pgfqpoint{5.632493in}{2.585594in}}%
\pgfpathlineto{\pgfqpoint{5.633567in}{2.584866in}}%
\pgfpathlineto{\pgfqpoint{5.634641in}{2.581800in}}%
\pgfpathlineto{\pgfqpoint{5.635715in}{2.583203in}}%
\pgfpathlineto{\pgfqpoint{5.636789in}{2.577797in}}%
\pgfpathlineto{\pgfqpoint{5.641084in}{2.585854in}}%
\pgfpathlineto{\pgfqpoint{5.642158in}{2.584763in}}%
\pgfpathlineto{\pgfqpoint{5.644306in}{2.591884in}}%
\pgfpathlineto{\pgfqpoint{5.647527in}{2.589545in}}%
\pgfpathlineto{\pgfqpoint{5.648601in}{2.598641in}}%
\pgfpathlineto{\pgfqpoint{5.649675in}{2.598589in}}%
\pgfpathlineto{\pgfqpoint{5.650749in}{2.603320in}}%
\pgfpathlineto{\pgfqpoint{5.651823in}{2.612052in}}%
\pgfpathlineto{\pgfqpoint{5.655044in}{2.609401in}}%
\pgfpathlineto{\pgfqpoint{5.656118in}{2.605191in}}%
\pgfpathlineto{\pgfqpoint{5.657192in}{2.609297in}}%
\pgfpathlineto{\pgfqpoint{5.658266in}{2.607426in}}%
\pgfpathlineto{\pgfqpoint{5.662561in}{2.616367in}}%
\pgfpathlineto{\pgfqpoint{5.663635in}{2.616471in}}%
\pgfpathlineto{\pgfqpoint{5.664709in}{2.611741in}}%
\pgfpathlineto{\pgfqpoint{5.665782in}{2.603736in}}%
\pgfpathlineto{\pgfqpoint{5.666856in}{2.608778in}}%
\pgfpathlineto{\pgfqpoint{5.671152in}{2.611013in}}%
\pgfpathlineto{\pgfqpoint{5.672225in}{2.615587in}}%
\pgfpathlineto{\pgfqpoint{5.673299in}{2.613404in}}%
\pgfpathlineto{\pgfqpoint{5.674373in}{2.612520in}}%
\pgfpathlineto{\pgfqpoint{5.677595in}{2.614080in}}%
\pgfpathlineto{\pgfqpoint{5.679742in}{2.611429in}}%
\pgfpathlineto{\pgfqpoint{5.680816in}{2.615067in}}%
\pgfpathlineto{\pgfqpoint{5.681890in}{2.609297in}}%
\pgfpathlineto{\pgfqpoint{5.685112in}{2.605555in}}%
\pgfpathlineto{\pgfqpoint{5.688333in}{2.617407in}}%
\pgfpathlineto{\pgfqpoint{5.689407in}{2.620421in}}%
\pgfpathlineto{\pgfqpoint{5.694776in}{2.617978in}}%
\pgfpathlineto{\pgfqpoint{5.696924in}{2.612312in}}%
\pgfpathlineto{\pgfqpoint{5.700145in}{2.609609in}}%
\pgfpathlineto{\pgfqpoint{5.702293in}{2.610441in}}%
\pgfpathlineto{\pgfqpoint{5.703367in}{2.616991in}}%
\pgfpathlineto{\pgfqpoint{5.704441in}{2.618862in}}%
\pgfpathlineto{\pgfqpoint{5.707662in}{2.619642in}}%
\pgfpathlineto{\pgfqpoint{5.708736in}{2.616939in}}%
\pgfpathlineto{\pgfqpoint{5.710884in}{2.617614in}}%
\pgfpathlineto{\pgfqpoint{5.715179in}{2.616367in}}%
\pgfpathlineto{\pgfqpoint{5.716253in}{2.617199in}}%
\pgfpathlineto{\pgfqpoint{5.717327in}{2.616679in}}%
\pgfpathlineto{\pgfqpoint{5.718401in}{2.614911in}}%
\pgfpathlineto{\pgfqpoint{5.719474in}{2.620629in}}%
\pgfpathlineto{\pgfqpoint{5.723770in}{2.618914in}}%
\pgfpathlineto{\pgfqpoint{5.724844in}{2.621929in}}%
\pgfpathlineto{\pgfqpoint{5.725917in}{2.619278in}}%
\pgfpathlineto{\pgfqpoint{5.726991in}{2.619070in}}%
\pgfpathlineto{\pgfqpoint{5.730213in}{2.616887in}}%
\pgfpathlineto{\pgfqpoint{5.731287in}{2.617355in}}%
\pgfpathlineto{\pgfqpoint{5.732360in}{2.615639in}}%
\pgfpathlineto{\pgfqpoint{5.738803in}{2.622605in}}%
\pgfpathlineto{\pgfqpoint{5.739877in}{2.614496in}}%
\pgfpathlineto{\pgfqpoint{5.740951in}{2.610961in}}%
\pgfpathlineto{\pgfqpoint{5.745246in}{2.613872in}}%
\pgfpathlineto{\pgfqpoint{5.746320in}{2.605139in}}%
\pgfpathlineto{\pgfqpoint{5.747394in}{2.606698in}}%
\pgfpathlineto{\pgfqpoint{5.748468in}{2.606127in}}%
\pgfpathlineto{\pgfqpoint{5.752763in}{2.611793in}}%
\pgfpathlineto{\pgfqpoint{5.753837in}{2.612416in}}%
\pgfpathlineto{\pgfqpoint{5.754911in}{2.614755in}}%
\pgfpathlineto{\pgfqpoint{5.755985in}{2.613352in}}%
\pgfpathlineto{\pgfqpoint{5.757059in}{2.617251in}}%
\pgfpathlineto{\pgfqpoint{5.760280in}{2.617199in}}%
\pgfpathlineto{\pgfqpoint{5.761354in}{2.618862in}}%
\pgfpathlineto{\pgfqpoint{5.762428in}{2.617303in}}%
\pgfpathlineto{\pgfqpoint{5.763502in}{2.618550in}}%
\pgfpathlineto{\pgfqpoint{5.764576in}{2.612988in}}%
\pgfpathlineto{\pgfqpoint{5.767797in}{2.614859in}}%
\pgfpathlineto{\pgfqpoint{5.769945in}{2.606543in}}%
\pgfpathlineto{\pgfqpoint{5.771019in}{2.608102in}}%
\pgfpathlineto{\pgfqpoint{5.772092in}{2.607322in}}%
\pgfpathlineto{\pgfqpoint{5.775314in}{2.608154in}}%
\pgfpathlineto{\pgfqpoint{5.777462in}{2.614184in}}%
\pgfpathlineto{\pgfqpoint{5.778535in}{2.612936in}}%
\pgfpathlineto{\pgfqpoint{5.779609in}{2.613924in}}%
\pgfpathlineto{\pgfqpoint{5.783905in}{2.612104in}}%
\pgfpathlineto{\pgfqpoint{5.784979in}{2.615639in}}%
\pgfpathlineto{\pgfqpoint{5.786052in}{2.616367in}}%
\pgfpathlineto{\pgfqpoint{5.787126in}{2.618810in}}%
\pgfpathlineto{\pgfqpoint{5.790348in}{2.620213in}}%
\pgfpathlineto{\pgfqpoint{5.791422in}{2.618134in}}%
\pgfpathlineto{\pgfqpoint{5.794643in}{2.622241in}}%
\pgfpathlineto{\pgfqpoint{5.797865in}{2.619278in}}%
\pgfpathlineto{\pgfqpoint{5.798938in}{2.622968in}}%
\pgfpathlineto{\pgfqpoint{5.800012in}{2.621097in}}%
\pgfpathlineto{\pgfqpoint{5.801086in}{2.623228in}}%
\pgfpathlineto{\pgfqpoint{5.802160in}{2.621929in}}%
\pgfpathlineto{\pgfqpoint{5.805381in}{2.621513in}}%
\pgfpathlineto{\pgfqpoint{5.808603in}{2.626971in}}%
\pgfpathlineto{\pgfqpoint{5.809677in}{2.617822in}}%
\pgfpathlineto{\pgfqpoint{5.812898in}{2.612988in}}%
\pgfpathlineto{\pgfqpoint{5.816120in}{2.629154in}}%
\pgfpathlineto{\pgfqpoint{5.817194in}{2.629674in}}%
\pgfpathlineto{\pgfqpoint{5.821489in}{2.632741in}}%
\pgfpathlineto{\pgfqpoint{5.823637in}{2.629882in}}%
\pgfpathlineto{\pgfqpoint{5.824711in}{2.634300in}}%
\pgfpathlineto{\pgfqpoint{5.829006in}{2.634196in}}%
\pgfpathlineto{\pgfqpoint{5.830080in}{2.634872in}}%
\pgfpathlineto{\pgfqpoint{5.831154in}{2.634768in}}%
\pgfpathlineto{\pgfqpoint{5.832227in}{2.635444in}}%
\pgfpathlineto{\pgfqpoint{5.835449in}{2.635080in}}%
\pgfpathlineto{\pgfqpoint{5.836523in}{2.636172in}}%
\pgfpathlineto{\pgfqpoint{5.838670in}{2.635028in}}%
\pgfpathlineto{\pgfqpoint{5.839744in}{2.637211in}}%
\pgfpathlineto{\pgfqpoint{5.842966in}{2.637575in}}%
\pgfpathlineto{\pgfqpoint{5.845113in}{2.631285in}}%
\pgfpathlineto{\pgfqpoint{5.846187in}{2.632897in}}%
\pgfpathlineto{\pgfqpoint{5.847261in}{2.636587in}}%
\pgfpathlineto{\pgfqpoint{5.851557in}{2.642149in}}%
\pgfpathlineto{\pgfqpoint{5.852630in}{2.638407in}}%
\pgfpathlineto{\pgfqpoint{5.853704in}{2.638771in}}%
\pgfpathlineto{\pgfqpoint{5.854778in}{2.637523in}}%
\pgfpathlineto{\pgfqpoint{5.858000in}{2.637419in}}%
\pgfpathlineto{\pgfqpoint{5.860147in}{2.640018in}}%
\pgfpathlineto{\pgfqpoint{5.862295in}{2.643449in}}%
\pgfpathlineto{\pgfqpoint{5.865516in}{2.643345in}}%
\pgfpathlineto{\pgfqpoint{5.866590in}{2.641266in}}%
\pgfpathlineto{\pgfqpoint{5.868738in}{2.645372in}}%
\pgfpathlineto{\pgfqpoint{5.873033in}{2.642565in}}%
\pgfpathlineto{\pgfqpoint{5.874107in}{2.645164in}}%
\pgfpathlineto{\pgfqpoint{5.875181in}{2.644748in}}%
\pgfpathlineto{\pgfqpoint{5.876255in}{2.647555in}}%
\pgfpathlineto{\pgfqpoint{5.877329in}{2.645996in}}%
\pgfpathlineto{\pgfqpoint{5.880550in}{2.649427in}}%
\pgfpathlineto{\pgfqpoint{5.881624in}{2.645840in}}%
\pgfpathlineto{\pgfqpoint{5.882698in}{2.644748in}}%
\pgfpathlineto{\pgfqpoint{5.883772in}{2.649479in}}%
\pgfpathlineto{\pgfqpoint{5.884846in}{2.648959in}}%
\pgfpathlineto{\pgfqpoint{5.889141in}{2.651038in}}%
\pgfpathlineto{\pgfqpoint{5.890215in}{2.647763in}}%
\pgfpathlineto{\pgfqpoint{5.891289in}{2.646984in}}%
\pgfpathlineto{\pgfqpoint{5.892362in}{2.639706in}}%
\pgfpathlineto{\pgfqpoint{5.895584in}{2.649167in}}%
\pgfpathlineto{\pgfqpoint{5.896658in}{2.643501in}}%
\pgfpathlineto{\pgfqpoint{5.897732in}{2.643345in}}%
\pgfpathlineto{\pgfqpoint{5.898805in}{2.648283in}}%
\pgfpathlineto{\pgfqpoint{5.899879in}{2.648231in}}%
\pgfpathlineto{\pgfqpoint{5.904175in}{2.650726in}}%
\pgfpathlineto{\pgfqpoint{5.905248in}{2.647036in}}%
\pgfpathlineto{\pgfqpoint{5.906322in}{2.652702in}}%
\pgfpathlineto{\pgfqpoint{5.907396in}{2.646880in}}%
\pgfpathlineto{\pgfqpoint{5.910618in}{2.647296in}}%
\pgfpathlineto{\pgfqpoint{5.911691in}{2.649687in}}%
\pgfpathlineto{\pgfqpoint{5.912765in}{2.654885in}}%
\pgfpathlineto{\pgfqpoint{5.913839in}{2.649115in}}%
\pgfpathlineto{\pgfqpoint{5.914913in}{2.656288in}}%
\pgfpathlineto{\pgfqpoint{5.919208in}{2.649687in}}%
\pgfpathlineto{\pgfqpoint{5.922430in}{2.657484in}}%
\pgfpathlineto{\pgfqpoint{5.925651in}{2.653013in}}%
\pgfpathlineto{\pgfqpoint{5.926725in}{2.650570in}}%
\pgfpathlineto{\pgfqpoint{5.927799in}{2.650726in}}%
\pgfpathlineto{\pgfqpoint{5.928873in}{2.649115in}}%
\pgfpathlineto{\pgfqpoint{5.929947in}{2.650051in}}%
\pgfpathlineto{\pgfqpoint{5.933168in}{2.647192in}}%
\pgfpathlineto{\pgfqpoint{5.934242in}{2.645424in}}%
\pgfpathlineto{\pgfqpoint{5.936390in}{2.636587in}}%
\pgfpathlineto{\pgfqpoint{5.937464in}{2.633729in}}%
\pgfpathlineto{\pgfqpoint{5.940685in}{2.632637in}}%
\pgfpathlineto{\pgfqpoint{5.941759in}{2.646308in}}%
\pgfpathlineto{\pgfqpoint{5.942833in}{2.648335in}}%
\pgfpathlineto{\pgfqpoint{5.943907in}{2.644437in}}%
\pgfpathlineto{\pgfqpoint{5.944980in}{2.645684in}}%
\pgfpathlineto{\pgfqpoint{5.950350in}{2.645216in}}%
\pgfpathlineto{\pgfqpoint{5.951423in}{2.644541in}}%
\pgfpathlineto{\pgfqpoint{5.952497in}{2.637315in}}%
\pgfpathlineto{\pgfqpoint{5.955719in}{2.644333in}}%
\pgfpathlineto{\pgfqpoint{5.956793in}{2.648647in}}%
\pgfpathlineto{\pgfqpoint{5.957867in}{2.641370in}}%
\pgfpathlineto{\pgfqpoint{5.958940in}{2.627179in}}%
\pgfpathlineto{\pgfqpoint{5.960014in}{2.630142in}}%
\pgfpathlineto{\pgfqpoint{5.963236in}{2.627387in}}%
\pgfpathlineto{\pgfqpoint{5.964310in}{2.630350in}}%
\pgfpathlineto{\pgfqpoint{5.965383in}{2.628271in}}%
\pgfpathlineto{\pgfqpoint{5.966457in}{2.627699in}}%
\pgfpathlineto{\pgfqpoint{5.967531in}{2.622605in}}%
\pgfpathlineto{\pgfqpoint{5.971826in}{2.626243in}}%
\pgfpathlineto{\pgfqpoint{5.972900in}{2.625879in}}%
\pgfpathlineto{\pgfqpoint{5.975048in}{2.629570in}}%
\pgfpathlineto{\pgfqpoint{5.979343in}{2.626867in}}%
\pgfpathlineto{\pgfqpoint{5.981491in}{2.621929in}}%
\pgfpathlineto{\pgfqpoint{5.982565in}{2.624528in}}%
\pgfpathlineto{\pgfqpoint{5.985786in}{2.627335in}}%
\pgfpathlineto{\pgfqpoint{5.986860in}{2.626971in}}%
\pgfpathlineto{\pgfqpoint{5.987934in}{2.633001in}}%
\pgfpathlineto{\pgfqpoint{5.989008in}{2.629778in}}%
\pgfpathlineto{\pgfqpoint{5.990082in}{2.633936in}}%
\pgfpathlineto{\pgfqpoint{5.993303in}{2.637523in}}%
\pgfpathlineto{\pgfqpoint{5.994377in}{2.637783in}}%
\pgfpathlineto{\pgfqpoint{5.995451in}{2.633936in}}%
\pgfpathlineto{\pgfqpoint{5.996525in}{2.635392in}}%
\pgfpathlineto{\pgfqpoint{6.001894in}{2.634872in}}%
\pgfpathlineto{\pgfqpoint{6.002968in}{2.633469in}}%
\pgfpathlineto{\pgfqpoint{6.004042in}{2.634404in}}%
\pgfpathlineto{\pgfqpoint{6.005115in}{2.636743in}}%
\pgfpathlineto{\pgfqpoint{6.009411in}{2.635028in}}%
\pgfpathlineto{\pgfqpoint{6.010485in}{2.632481in}}%
\pgfpathlineto{\pgfqpoint{6.011558in}{2.633832in}}%
\pgfpathlineto{\pgfqpoint{6.012632in}{2.632533in}}%
\pgfpathlineto{\pgfqpoint{6.016928in}{2.633105in}}%
\pgfpathlineto{\pgfqpoint{6.018001in}{2.634560in}}%
\pgfpathlineto{\pgfqpoint{6.019075in}{2.637211in}}%
\pgfpathlineto{\pgfqpoint{6.020149in}{2.637055in}}%
\pgfpathlineto{\pgfqpoint{6.023371in}{2.634040in}}%
\pgfpathlineto{\pgfqpoint{6.024445in}{2.629726in}}%
\pgfpathlineto{\pgfqpoint{6.026592in}{2.631389in}}%
\pgfpathlineto{\pgfqpoint{6.027666in}{2.632221in}}%
\pgfpathlineto{\pgfqpoint{6.033035in}{2.639810in}}%
\pgfpathlineto{\pgfqpoint{6.034109in}{2.638667in}}%
\pgfpathlineto{\pgfqpoint{6.035183in}{2.651870in}}%
\pgfpathlineto{\pgfqpoint{6.038404in}{2.649531in}}%
\pgfpathlineto{\pgfqpoint{6.039478in}{2.653845in}}%
\pgfpathlineto{\pgfqpoint{6.041626in}{2.647815in}}%
\pgfpathlineto{\pgfqpoint{6.042700in}{2.648387in}}%
\pgfpathlineto{\pgfqpoint{6.045921in}{2.648543in}}%
\pgfpathlineto{\pgfqpoint{6.046995in}{2.652598in}}%
\pgfpathlineto{\pgfqpoint{6.048069in}{2.651298in}}%
\pgfpathlineto{\pgfqpoint{6.049143in}{2.653377in}}%
\pgfpathlineto{\pgfqpoint{6.050217in}{2.651714in}}%
\pgfpathlineto{\pgfqpoint{6.053438in}{2.651662in}}%
\pgfpathlineto{\pgfqpoint{6.055586in}{2.656132in}}%
\pgfpathlineto{\pgfqpoint{6.056660in}{2.657744in}}%
\pgfpathlineto{\pgfqpoint{6.057734in}{2.654417in}}%
\pgfpathlineto{\pgfqpoint{6.060955in}{2.656028in}}%
\pgfpathlineto{\pgfqpoint{6.062029in}{2.653845in}}%
\pgfpathlineto{\pgfqpoint{6.063103in}{2.669543in}}%
\pgfpathlineto{\pgfqpoint{6.064177in}{2.667932in}}%
\pgfpathlineto{\pgfqpoint{6.065250in}{2.669387in}}%
\pgfpathlineto{\pgfqpoint{6.069546in}{2.672142in}}%
\pgfpathlineto{\pgfqpoint{6.072767in}{2.669179in}}%
\pgfpathlineto{\pgfqpoint{6.075989in}{2.668400in}}%
\pgfpathlineto{\pgfqpoint{6.077063in}{2.669283in}}%
\pgfpathlineto{\pgfqpoint{6.078136in}{2.672142in}}%
\pgfpathlineto{\pgfqpoint{6.080284in}{2.666528in}}%
\pgfpathlineto{\pgfqpoint{6.084579in}{2.665541in}}%
\pgfpathlineto{\pgfqpoint{6.085653in}{2.664813in}}%
\pgfpathlineto{\pgfqpoint{6.086727in}{2.665801in}}%
\pgfpathlineto{\pgfqpoint{6.087801in}{2.669283in}}%
\pgfpathlineto{\pgfqpoint{6.091022in}{2.670427in}}%
\pgfpathlineto{\pgfqpoint{6.092096in}{2.668972in}}%
\pgfpathlineto{\pgfqpoint{6.093170in}{2.670895in}}%
\pgfpathlineto{\pgfqpoint{6.094244in}{2.671051in}}%
\pgfpathlineto{\pgfqpoint{6.095318in}{2.668972in}}%
\pgfpathlineto{\pgfqpoint{6.099613in}{2.669855in}}%
\pgfpathlineto{\pgfqpoint{6.102835in}{2.666892in}}%
\pgfpathlineto{\pgfqpoint{6.106056in}{2.666476in}}%
\pgfpathlineto{\pgfqpoint{6.107130in}{2.667776in}}%
\pgfpathlineto{\pgfqpoint{6.108204in}{2.667048in}}%
\pgfpathlineto{\pgfqpoint{6.110352in}{2.663410in}}%
\pgfpathlineto{\pgfqpoint{6.113573in}{2.662630in}}%
\pgfpathlineto{\pgfqpoint{6.114647in}{2.663721in}}%
\pgfpathlineto{\pgfqpoint{6.115721in}{2.663981in}}%
\pgfpathlineto{\pgfqpoint{6.116795in}{2.661278in}}%
\pgfpathlineto{\pgfqpoint{6.117868in}{2.660447in}}%
\pgfpathlineto{\pgfqpoint{6.121090in}{2.661694in}}%
\pgfpathlineto{\pgfqpoint{6.123238in}{2.665645in}}%
\pgfpathlineto{\pgfqpoint{6.124311in}{2.664293in}}%
\pgfpathlineto{\pgfqpoint{6.128607in}{2.666009in}}%
\pgfpathlineto{\pgfqpoint{6.129681in}{2.667984in}}%
\pgfpathlineto{\pgfqpoint{6.132902in}{2.660759in}}%
\pgfpathlineto{\pgfqpoint{6.137198in}{2.667464in}}%
\pgfpathlineto{\pgfqpoint{6.138271in}{2.656496in}}%
\pgfpathlineto{\pgfqpoint{6.139345in}{2.656288in}}%
\pgfpathlineto{\pgfqpoint{6.140419in}{2.654521in}}%
\pgfpathlineto{\pgfqpoint{6.143641in}{2.653377in}}%
\pgfpathlineto{\pgfqpoint{6.144714in}{2.649167in}}%
\pgfpathlineto{\pgfqpoint{6.145788in}{2.650154in}}%
\pgfpathlineto{\pgfqpoint{6.151157in}{2.650778in}}%
\pgfpathlineto{\pgfqpoint{6.152231in}{2.650051in}}%
\pgfpathlineto{\pgfqpoint{6.153305in}{2.650466in}}%
\pgfpathlineto{\pgfqpoint{6.154379in}{2.648907in}}%
\pgfpathlineto{\pgfqpoint{6.160822in}{2.649375in}}%
\pgfpathlineto{\pgfqpoint{6.161896in}{2.647451in}}%
\pgfpathlineto{\pgfqpoint{6.162970in}{2.649271in}}%
\pgfpathlineto{\pgfqpoint{6.166191in}{2.649115in}}%
\pgfpathlineto{\pgfqpoint{6.167265in}{2.648491in}}%
\pgfpathlineto{\pgfqpoint{6.170487in}{2.654157in}}%
\pgfpathlineto{\pgfqpoint{6.174782in}{2.654885in}}%
\pgfpathlineto{\pgfqpoint{6.175856in}{2.658212in}}%
\pgfpathlineto{\pgfqpoint{6.176930in}{2.658419in}}%
\pgfpathlineto{\pgfqpoint{6.178003in}{2.660603in}}%
\pgfpathlineto{\pgfqpoint{6.183373in}{2.661486in}}%
\pgfpathlineto{\pgfqpoint{6.184446in}{2.657068in}}%
\pgfpathlineto{\pgfqpoint{6.185520in}{2.658523in}}%
\pgfpathlineto{\pgfqpoint{6.188742in}{2.658939in}}%
\pgfpathlineto{\pgfqpoint{6.189816in}{2.658056in}}%
\pgfpathlineto{\pgfqpoint{6.190889in}{2.659927in}}%
\pgfpathlineto{\pgfqpoint{6.191963in}{2.664449in}}%
\pgfpathlineto{\pgfqpoint{6.193037in}{2.665801in}}%
\pgfpathlineto{\pgfqpoint{6.196259in}{2.666788in}}%
\pgfpathlineto{\pgfqpoint{6.199480in}{2.662734in}}%
\pgfpathlineto{\pgfqpoint{6.200554in}{2.664657in}}%
\pgfpathlineto{\pgfqpoint{6.203776in}{2.664345in}}%
\pgfpathlineto{\pgfqpoint{6.204849in}{2.660707in}}%
\pgfpathlineto{\pgfqpoint{6.205923in}{2.659563in}}%
\pgfpathlineto{\pgfqpoint{6.206997in}{2.652909in}}%
\pgfpathlineto{\pgfqpoint{6.208071in}{2.653637in}}%
\pgfpathlineto{\pgfqpoint{6.211292in}{2.656496in}}%
\pgfpathlineto{\pgfqpoint{6.213440in}{2.656184in}}%
\pgfpathlineto{\pgfqpoint{6.214514in}{2.654729in}}%
\pgfpathlineto{\pgfqpoint{6.215588in}{2.656080in}}%
\pgfpathlineto{\pgfqpoint{6.219883in}{2.651818in}}%
\pgfpathlineto{\pgfqpoint{6.220957in}{2.652754in}}%
\pgfpathlineto{\pgfqpoint{6.222031in}{2.651506in}}%
\pgfpathlineto{\pgfqpoint{6.223105in}{2.653429in}}%
\pgfpathlineto{\pgfqpoint{6.226326in}{2.655612in}}%
\pgfpathlineto{\pgfqpoint{6.227400in}{2.660863in}}%
\pgfpathlineto{\pgfqpoint{6.229548in}{2.664033in}}%
\pgfpathlineto{\pgfqpoint{6.230622in}{2.664085in}}%
\pgfpathlineto{\pgfqpoint{6.233843in}{2.662006in}}%
\pgfpathlineto{\pgfqpoint{6.234917in}{2.666684in}}%
\pgfpathlineto{\pgfqpoint{6.235991in}{2.667464in}}%
\pgfpathlineto{\pgfqpoint{6.237065in}{2.674170in}}%
\pgfpathlineto{\pgfqpoint{6.238138in}{2.671882in}}%
\pgfpathlineto{\pgfqpoint{6.242434in}{2.676249in}}%
\pgfpathlineto{\pgfqpoint{6.244581in}{2.675053in}}%
\pgfpathlineto{\pgfqpoint{6.245655in}{2.674118in}}%
\pgfpathlineto{\pgfqpoint{6.251024in}{2.680251in}}%
\pgfpathlineto{\pgfqpoint{6.252098in}{2.679368in}}%
\pgfpathlineto{\pgfqpoint{6.253172in}{2.677392in}}%
\pgfpathlineto{\pgfqpoint{6.256394in}{2.678900in}}%
\pgfpathlineto{\pgfqpoint{6.257467in}{2.681603in}}%
\pgfpathlineto{\pgfqpoint{6.258541in}{2.682798in}}%
\pgfpathlineto{\pgfqpoint{6.259615in}{2.680979in}}%
\pgfpathlineto{\pgfqpoint{6.260689in}{2.682902in}}%
\pgfpathlineto{\pgfqpoint{6.263910in}{2.684826in}}%
\pgfpathlineto{\pgfqpoint{6.264984in}{2.684462in}}%
\pgfpathlineto{\pgfqpoint{6.267132in}{2.681967in}}%
\pgfpathlineto{\pgfqpoint{6.268206in}{2.683110in}}%
\pgfpathlineto{\pgfqpoint{6.271427in}{2.682902in}}%
\pgfpathlineto{\pgfqpoint{6.272501in}{2.682175in}}%
\pgfpathlineto{\pgfqpoint{6.273575in}{2.679992in}}%
\pgfpathlineto{\pgfqpoint{6.275723in}{2.683214in}}%
\pgfpathlineto{\pgfqpoint{6.281092in}{2.684150in}}%
\pgfpathlineto{\pgfqpoint{6.282166in}{2.685346in}}%
\pgfpathlineto{\pgfqpoint{6.283240in}{2.684722in}}%
\pgfpathlineto{\pgfqpoint{6.286461in}{2.690336in}}%
\pgfpathlineto{\pgfqpoint{6.287535in}{2.687997in}}%
\pgfpathlineto{\pgfqpoint{6.289683in}{2.688204in}}%
\pgfpathlineto{\pgfqpoint{6.290756in}{2.686801in}}%
\pgfpathlineto{\pgfqpoint{6.293978in}{2.686229in}}%
\pgfpathlineto{\pgfqpoint{6.295052in}{2.691219in}}%
\pgfpathlineto{\pgfqpoint{6.296126in}{2.692311in}}%
\pgfpathlineto{\pgfqpoint{6.297199in}{2.683734in}}%
\pgfpathlineto{\pgfqpoint{6.298273in}{2.681811in}}%
\pgfpathlineto{\pgfqpoint{6.301495in}{2.684150in}}%
\pgfpathlineto{\pgfqpoint{6.302569in}{2.683786in}}%
\pgfpathlineto{\pgfqpoint{6.303643in}{2.675105in}}%
\pgfpathlineto{\pgfqpoint{6.305790in}{2.675625in}}%
\pgfpathlineto{\pgfqpoint{6.311159in}{2.682695in}}%
\pgfpathlineto{\pgfqpoint{6.312233in}{2.680771in}}%
\pgfpathlineto{\pgfqpoint{6.313307in}{2.682227in}}%
\pgfpathlineto{\pgfqpoint{6.316529in}{2.681187in}}%
\pgfpathlineto{\pgfqpoint{6.317602in}{2.678796in}}%
\pgfpathlineto{\pgfqpoint{6.318676in}{2.678016in}}%
\pgfpathlineto{\pgfqpoint{6.320824in}{2.685709in}}%
\pgfpathlineto{\pgfqpoint{6.324045in}{2.686177in}}%
\pgfpathlineto{\pgfqpoint{6.325119in}{2.684514in}}%
\pgfpathlineto{\pgfqpoint{6.326193in}{2.684358in}}%
\pgfpathlineto{\pgfqpoint{6.327267in}{2.681967in}}%
\pgfpathlineto{\pgfqpoint{6.328341in}{2.665541in}}%
\pgfpathlineto{\pgfqpoint{6.332636in}{2.659303in}}%
\pgfpathlineto{\pgfqpoint{6.333710in}{2.658731in}}%
\pgfpathlineto{\pgfqpoint{6.334784in}{2.661850in}}%
\pgfpathlineto{\pgfqpoint{6.335858in}{2.659615in}}%
\pgfpathlineto{\pgfqpoint{6.339079in}{2.655820in}}%
\pgfpathlineto{\pgfqpoint{6.340153in}{2.656184in}}%
\pgfpathlineto{\pgfqpoint{6.341227in}{2.658887in}}%
\pgfpathlineto{\pgfqpoint{6.342301in}{2.657016in}}%
\pgfpathlineto{\pgfqpoint{6.343375in}{2.657328in}}%
\pgfpathlineto{\pgfqpoint{6.346596in}{2.654729in}}%
\pgfpathlineto{\pgfqpoint{6.348744in}{2.662266in}}%
\pgfpathlineto{\pgfqpoint{6.349818in}{2.663254in}}%
\pgfpathlineto{\pgfqpoint{6.350891in}{2.665125in}}%
\pgfpathlineto{\pgfqpoint{6.354113in}{2.669231in}}%
\pgfpathlineto{\pgfqpoint{6.355187in}{2.668608in}}%
\pgfpathlineto{\pgfqpoint{6.356261in}{2.665437in}}%
\pgfpathlineto{\pgfqpoint{6.357334in}{2.670479in}}%
\pgfpathlineto{\pgfqpoint{6.358408in}{2.666424in}}%
\pgfpathlineto{\pgfqpoint{6.361630in}{2.665645in}}%
\pgfpathlineto{\pgfqpoint{6.362704in}{2.667880in}}%
\pgfpathlineto{\pgfqpoint{6.363777in}{2.665957in}}%
\pgfpathlineto{\pgfqpoint{6.365925in}{2.666528in}}%
\pgfpathlineto{\pgfqpoint{6.369147in}{2.669024in}}%
\pgfpathlineto{\pgfqpoint{6.370221in}{2.671207in}}%
\pgfpathlineto{\pgfqpoint{6.371294in}{2.671103in}}%
\pgfpathlineto{\pgfqpoint{6.373442in}{2.675937in}}%
\pgfpathlineto{\pgfqpoint{6.376664in}{2.681083in}}%
\pgfpathlineto{\pgfqpoint{6.377737in}{2.681031in}}%
\pgfpathlineto{\pgfqpoint{6.378811in}{2.680303in}}%
\pgfpathlineto{\pgfqpoint{6.379885in}{2.674637in}}%
\pgfpathlineto{\pgfqpoint{6.380959in}{2.675989in}}%
\pgfpathlineto{\pgfqpoint{6.384180in}{2.675313in}}%
\pgfpathlineto{\pgfqpoint{6.385254in}{2.673442in}}%
\pgfpathlineto{\pgfqpoint{6.386328in}{2.678484in}}%
\pgfpathlineto{\pgfqpoint{6.387402in}{2.679056in}}%
\pgfpathlineto{\pgfqpoint{6.388476in}{2.683422in}}%
\pgfpathlineto{\pgfqpoint{6.391697in}{2.683422in}}%
\pgfpathlineto{\pgfqpoint{6.393845in}{2.681655in}}%
\pgfpathlineto{\pgfqpoint{6.394919in}{2.682383in}}%
\pgfpathlineto{\pgfqpoint{6.395993in}{2.684618in}}%
\pgfpathlineto{\pgfqpoint{6.400288in}{2.686333in}}%
\pgfpathlineto{\pgfqpoint{6.401362in}{2.684462in}}%
\pgfpathlineto{\pgfqpoint{6.402436in}{2.684306in}}%
\pgfpathlineto{\pgfqpoint{6.403510in}{2.683422in}}%
\pgfpathlineto{\pgfqpoint{6.403510in}{2.683422in}}%
\pgfusepath{stroke}%
\end{pgfscope}%
\begin{pgfscope}%
\pgfpathrectangle{\pgfqpoint{3.937600in}{2.309648in}}{\pgfqpoint{2.583333in}{0.400885in}}%
\pgfusepath{clip}%
\pgfsetroundcap%
\pgfsetroundjoin%
\pgfsetlinewidth{1.505625pt}%
\definecolor{currentstroke}{rgb}{0.549020,0.337255,0.294118}%
\pgfsetstrokecolor{currentstroke}%
\pgfsetdash{}{0pt}%
\pgfpathmoveto{\pgfqpoint{4.055025in}{2.503412in}}%
\pgfpathlineto{\pgfqpoint{4.056098in}{2.503412in}}%
\pgfpathlineto{\pgfqpoint{4.058246in}{2.501644in}}%
\pgfpathlineto{\pgfqpoint{4.061468in}{2.501644in}}%
\pgfpathlineto{\pgfqpoint{4.062542in}{2.500349in}}%
\pgfpathlineto{\pgfqpoint{4.063615in}{2.497760in}}%
\pgfpathlineto{\pgfqpoint{4.065763in}{2.498278in}}%
\pgfpathlineto{\pgfqpoint{4.072206in}{2.498278in}}%
\pgfpathlineto{\pgfqpoint{4.073280in}{2.497664in}}%
\pgfpathlineto{\pgfqpoint{4.077575in}{2.490736in}}%
\pgfpathlineto{\pgfqpoint{4.078649in}{2.492671in}}%
\pgfpathlineto{\pgfqpoint{4.079723in}{2.491907in}}%
\pgfpathlineto{\pgfqpoint{4.080797in}{2.489921in}}%
\pgfpathlineto{\pgfqpoint{4.085092in}{2.484878in}}%
\pgfpathlineto{\pgfqpoint{4.087240in}{2.485998in}}%
\pgfpathlineto{\pgfqpoint{4.088314in}{2.483757in}}%
\pgfpathlineto{\pgfqpoint{4.092609in}{2.487527in}}%
\pgfpathlineto{\pgfqpoint{4.095831in}{2.487272in}}%
\pgfpathlineto{\pgfqpoint{4.108717in}{2.487272in}}%
\pgfpathlineto{\pgfqpoint{4.109790in}{2.479424in}}%
\pgfpathlineto{\pgfqpoint{4.110864in}{2.478336in}}%
\pgfpathlineto{\pgfqpoint{4.114086in}{2.478383in}}%
\pgfpathlineto{\pgfqpoint{4.115160in}{2.475850in}}%
\pgfpathlineto{\pgfqpoint{4.116233in}{2.474976in}}%
\pgfpathlineto{\pgfqpoint{4.117307in}{2.478444in}}%
\pgfpathlineto{\pgfqpoint{4.118381in}{2.478397in}}%
\pgfpathlineto{\pgfqpoint{4.121603in}{2.477365in}}%
\pgfpathlineto{\pgfqpoint{4.123750in}{2.478718in}}%
\pgfpathlineto{\pgfqpoint{4.124824in}{2.477541in}}%
\pgfpathlineto{\pgfqpoint{4.125898in}{2.477448in}}%
\pgfpathlineto{\pgfqpoint{4.130193in}{2.473878in}}%
\pgfpathlineto{\pgfqpoint{4.132341in}{2.474693in}}%
\pgfpathlineto{\pgfqpoint{4.133415in}{2.476242in}}%
\pgfpathlineto{\pgfqpoint{4.138784in}{2.476426in}}%
\pgfpathlineto{\pgfqpoint{4.139858in}{2.475225in}}%
\pgfpathlineto{\pgfqpoint{4.140932in}{2.475545in}}%
\pgfpathlineto{\pgfqpoint{4.148449in}{2.475454in}}%
\pgfpathlineto{\pgfqpoint{4.152744in}{2.474165in}}%
\pgfpathlineto{\pgfqpoint{4.169925in}{2.474165in}}%
\pgfpathlineto{\pgfqpoint{4.170999in}{2.470747in}}%
\pgfpathlineto{\pgfqpoint{4.175295in}{2.470747in}}%
\pgfpathlineto{\pgfqpoint{4.176368in}{2.469118in}}%
\pgfpathlineto{\pgfqpoint{4.178516in}{2.469207in}}%
\pgfpathlineto{\pgfqpoint{4.181738in}{2.466240in}}%
\pgfpathlineto{\pgfqpoint{4.182811in}{2.466008in}}%
\pgfpathlineto{\pgfqpoint{4.183885in}{2.467492in}}%
\pgfpathlineto{\pgfqpoint{4.190328in}{2.467492in}}%
\pgfpathlineto{\pgfqpoint{4.191402in}{2.465689in}}%
\pgfpathlineto{\pgfqpoint{4.192476in}{2.465689in}}%
\pgfpathlineto{\pgfqpoint{4.193550in}{2.464037in}}%
\pgfpathlineto{\pgfqpoint{4.196771in}{2.463670in}}%
\pgfpathlineto{\pgfqpoint{4.198919in}{2.464175in}}%
\pgfpathlineto{\pgfqpoint{4.199993in}{2.464175in}}%
\pgfpathlineto{\pgfqpoint{4.201067in}{2.462575in}}%
\pgfpathlineto{\pgfqpoint{4.204288in}{2.462072in}}%
\pgfpathlineto{\pgfqpoint{4.205362in}{2.461204in}}%
\pgfpathlineto{\pgfqpoint{4.206436in}{2.458461in}}%
\pgfpathlineto{\pgfqpoint{4.207510in}{2.459101in}}%
\pgfpathlineto{\pgfqpoint{4.208584in}{2.458827in}}%
\pgfpathlineto{\pgfqpoint{4.212879in}{2.460518in}}%
\pgfpathlineto{\pgfqpoint{4.213953in}{2.458187in}}%
\pgfpathlineto{\pgfqpoint{4.215027in}{2.458095in}}%
\pgfpathlineto{\pgfqpoint{4.216100in}{2.455398in}}%
\pgfpathlineto{\pgfqpoint{4.219322in}{2.454849in}}%
\pgfpathlineto{\pgfqpoint{4.220396in}{2.454027in}}%
\pgfpathlineto{\pgfqpoint{4.223617in}{2.459786in}}%
\pgfpathlineto{\pgfqpoint{4.226839in}{2.459009in}}%
\pgfpathlineto{\pgfqpoint{4.227913in}{2.459786in}}%
\pgfpathlineto{\pgfqpoint{4.228986in}{2.459101in}}%
\pgfpathlineto{\pgfqpoint{4.230060in}{2.459101in}}%
\pgfpathlineto{\pgfqpoint{4.231134in}{2.458014in}}%
\pgfpathlineto{\pgfqpoint{4.235430in}{2.455615in}}%
\pgfpathlineto{\pgfqpoint{4.236503in}{2.449051in}}%
\pgfpathlineto{\pgfqpoint{4.237577in}{2.446743in}}%
\pgfpathlineto{\pgfqpoint{4.238651in}{2.447060in}}%
\pgfpathlineto{\pgfqpoint{4.242946in}{2.445023in}}%
\pgfpathlineto{\pgfqpoint{4.244020in}{2.447603in}}%
\pgfpathlineto{\pgfqpoint{4.245094in}{2.448689in}}%
\pgfpathlineto{\pgfqpoint{4.246168in}{2.452130in}}%
\pgfpathlineto{\pgfqpoint{4.249389in}{2.451949in}}%
\pgfpathlineto{\pgfqpoint{4.250463in}{2.452537in}}%
\pgfpathlineto{\pgfqpoint{4.253685in}{2.452265in}}%
\pgfpathlineto{\pgfqpoint{4.257980in}{2.453895in}}%
\pgfpathlineto{\pgfqpoint{4.259054in}{2.452673in}}%
\pgfpathlineto{\pgfqpoint{4.283752in}{2.452673in}}%
\pgfpathlineto{\pgfqpoint{4.286974in}{2.451621in}}%
\pgfpathlineto{\pgfqpoint{4.289121in}{2.448600in}}%
\pgfpathlineto{\pgfqpoint{4.291269in}{2.448438in}}%
\pgfpathlineto{\pgfqpoint{4.294491in}{2.449369in}}%
\pgfpathlineto{\pgfqpoint{4.295564in}{2.448552in}}%
\pgfpathlineto{\pgfqpoint{4.296638in}{2.448836in}}%
\pgfpathlineto{\pgfqpoint{4.297712in}{2.447658in}}%
\pgfpathlineto{\pgfqpoint{4.298786in}{2.447658in}}%
\pgfpathlineto{\pgfqpoint{4.303081in}{2.448421in}}%
\pgfpathlineto{\pgfqpoint{4.304155in}{2.448138in}}%
\pgfpathlineto{\pgfqpoint{4.305229in}{2.448703in}}%
\pgfpathlineto{\pgfqpoint{4.306303in}{2.447607in}}%
\pgfpathlineto{\pgfqpoint{4.310598in}{2.447687in}}%
\pgfpathlineto{\pgfqpoint{4.312746in}{2.448048in}}%
\pgfpathlineto{\pgfqpoint{4.313820in}{2.447040in}}%
\pgfpathlineto{\pgfqpoint{4.319189in}{2.446718in}}%
\pgfpathlineto{\pgfqpoint{4.320263in}{2.443690in}}%
\pgfpathlineto{\pgfqpoint{4.321337in}{2.442838in}}%
\pgfpathlineto{\pgfqpoint{4.324558in}{2.442838in}}%
\pgfpathlineto{\pgfqpoint{4.326706in}{2.444109in}}%
\pgfpathlineto{\pgfqpoint{4.327780in}{2.441582in}}%
\pgfpathlineto{\pgfqpoint{4.328853in}{2.440822in}}%
\pgfpathlineto{\pgfqpoint{4.334223in}{2.440521in}}%
\pgfpathlineto{\pgfqpoint{4.335297in}{2.439618in}}%
\pgfpathlineto{\pgfqpoint{4.336370in}{2.440028in}}%
\pgfpathlineto{\pgfqpoint{4.339592in}{2.439018in}}%
\pgfpathlineto{\pgfqpoint{4.342813in}{2.440393in}}%
\pgfpathlineto{\pgfqpoint{4.347109in}{2.439943in}}%
\pgfpathlineto{\pgfqpoint{4.348183in}{2.441925in}}%
\pgfpathlineto{\pgfqpoint{4.350330in}{2.440141in}}%
\pgfpathlineto{\pgfqpoint{4.351404in}{2.439392in}}%
\pgfpathlineto{\pgfqpoint{4.354626in}{2.440954in}}%
\pgfpathlineto{\pgfqpoint{4.357847in}{2.444308in}}%
\pgfpathlineto{\pgfqpoint{4.358921in}{2.444503in}}%
\pgfpathlineto{\pgfqpoint{4.363216in}{2.441203in}}%
\pgfpathlineto{\pgfqpoint{4.364290in}{2.438061in}}%
\pgfpathlineto{\pgfqpoint{4.365364in}{2.438061in}}%
\pgfpathlineto{\pgfqpoint{4.366438in}{2.440706in}}%
\pgfpathlineto{\pgfqpoint{4.369659in}{2.440933in}}%
\pgfpathlineto{\pgfqpoint{4.370733in}{2.444181in}}%
\pgfpathlineto{\pgfqpoint{4.371807in}{2.442155in}}%
\pgfpathlineto{\pgfqpoint{4.372881in}{2.436000in}}%
\pgfpathlineto{\pgfqpoint{4.373955in}{2.437839in}}%
\pgfpathlineto{\pgfqpoint{4.381472in}{2.438574in}}%
\pgfpathlineto{\pgfqpoint{4.384693in}{2.440127in}}%
\pgfpathlineto{\pgfqpoint{4.385767in}{2.439338in}}%
\pgfpathlineto{\pgfqpoint{4.386841in}{2.441983in}}%
\pgfpathlineto{\pgfqpoint{4.396505in}{2.441983in}}%
\pgfpathlineto{\pgfqpoint{4.404022in}{2.433495in}}%
\pgfpathlineto{\pgfqpoint{4.407244in}{2.433818in}}%
\pgfpathlineto{\pgfqpoint{4.408318in}{2.435257in}}%
\pgfpathlineto{\pgfqpoint{4.409391in}{2.433907in}}%
\pgfpathlineto{\pgfqpoint{4.410465in}{2.433727in}}%
\pgfpathlineto{\pgfqpoint{4.411539in}{2.432792in}}%
\pgfpathlineto{\pgfqpoint{4.416908in}{2.434005in}}%
\pgfpathlineto{\pgfqpoint{4.419056in}{2.431425in}}%
\pgfpathlineto{\pgfqpoint{4.422277in}{2.431636in}}%
\pgfpathlineto{\pgfqpoint{4.423351in}{2.430403in}}%
\pgfpathlineto{\pgfqpoint{4.424425in}{2.430090in}}%
\pgfpathlineto{\pgfqpoint{4.425499in}{2.431858in}}%
\pgfpathlineto{\pgfqpoint{4.426573in}{2.432458in}}%
\pgfpathlineto{\pgfqpoint{4.430868in}{2.432316in}}%
\pgfpathlineto{\pgfqpoint{4.431942in}{2.434125in}}%
\pgfpathlineto{\pgfqpoint{4.433016in}{2.432717in}}%
\pgfpathlineto{\pgfqpoint{4.434090in}{2.435885in}}%
\pgfpathlineto{\pgfqpoint{4.439459in}{2.436510in}}%
\pgfpathlineto{\pgfqpoint{4.444828in}{2.436510in}}%
\pgfpathlineto{\pgfqpoint{4.446976in}{2.431996in}}%
\pgfpathlineto{\pgfqpoint{4.448050in}{2.433281in}}%
\pgfpathlineto{\pgfqpoint{4.449123in}{2.432847in}}%
\pgfpathlineto{\pgfqpoint{4.453419in}{2.434544in}}%
\pgfpathlineto{\pgfqpoint{4.455566in}{2.432322in}}%
\pgfpathlineto{\pgfqpoint{4.456640in}{2.432466in}}%
\pgfpathlineto{\pgfqpoint{4.463083in}{2.429553in}}%
\pgfpathlineto{\pgfqpoint{4.464157in}{2.428787in}}%
\pgfpathlineto{\pgfqpoint{4.468452in}{2.428752in}}%
\pgfpathlineto{\pgfqpoint{4.469526in}{2.426679in}}%
\pgfpathlineto{\pgfqpoint{4.470600in}{2.427423in}}%
\pgfpathlineto{\pgfqpoint{4.471674in}{2.419547in}}%
\pgfpathlineto{\pgfqpoint{4.474896in}{2.418226in}}%
\pgfpathlineto{\pgfqpoint{4.475969in}{2.415124in}}%
\pgfpathlineto{\pgfqpoint{4.478117in}{2.414734in}}%
\pgfpathlineto{\pgfqpoint{4.479191in}{2.412882in}}%
\pgfpathlineto{\pgfqpoint{4.482412in}{2.414463in}}%
\pgfpathlineto{\pgfqpoint{4.484560in}{2.412301in}}%
\pgfpathlineto{\pgfqpoint{4.485634in}{2.412301in}}%
\pgfpathlineto{\pgfqpoint{4.486708in}{2.413261in}}%
\pgfpathlineto{\pgfqpoint{4.491003in}{2.412703in}}%
\pgfpathlineto{\pgfqpoint{4.493151in}{2.410811in}}%
\pgfpathlineto{\pgfqpoint{4.494225in}{2.411355in}}%
\pgfpathlineto{\pgfqpoint{4.498520in}{2.409397in}}%
\pgfpathlineto{\pgfqpoint{4.500668in}{2.410159in}}%
\pgfpathlineto{\pgfqpoint{4.501741in}{2.410301in}}%
\pgfpathlineto{\pgfqpoint{4.504963in}{2.412776in}}%
\pgfpathlineto{\pgfqpoint{4.506037in}{2.412396in}}%
\pgfpathlineto{\pgfqpoint{4.507111in}{2.410794in}}%
\pgfpathlineto{\pgfqpoint{4.508185in}{2.412139in}}%
\pgfpathlineto{\pgfqpoint{4.509258in}{2.411383in}}%
\pgfpathlineto{\pgfqpoint{4.512480in}{2.410951in}}%
\pgfpathlineto{\pgfqpoint{4.514628in}{2.409722in}}%
\pgfpathlineto{\pgfqpoint{4.515701in}{2.410429in}}%
\pgfpathlineto{\pgfqpoint{4.516775in}{2.409773in}}%
\pgfpathlineto{\pgfqpoint{4.521071in}{2.409800in}}%
\pgfpathlineto{\pgfqpoint{4.522144in}{2.410649in}}%
\pgfpathlineto{\pgfqpoint{4.523218in}{2.409277in}}%
\pgfpathlineto{\pgfqpoint{4.524292in}{2.411698in}}%
\pgfpathlineto{\pgfqpoint{4.527514in}{2.412102in}}%
\pgfpathlineto{\pgfqpoint{4.529661in}{2.408760in}}%
\pgfpathlineto{\pgfqpoint{4.530735in}{2.409628in}}%
\pgfpathlineto{\pgfqpoint{4.531809in}{2.409486in}}%
\pgfpathlineto{\pgfqpoint{4.535030in}{2.410841in}}%
\pgfpathlineto{\pgfqpoint{4.536104in}{2.409150in}}%
\pgfpathlineto{\pgfqpoint{4.537178in}{2.409938in}}%
\pgfpathlineto{\pgfqpoint{4.538252in}{2.409938in}}%
\pgfpathlineto{\pgfqpoint{4.542547in}{2.408462in}}%
\pgfpathlineto{\pgfqpoint{4.543621in}{2.405588in}}%
\pgfpathlineto{\pgfqpoint{4.544695in}{2.407425in}}%
\pgfpathlineto{\pgfqpoint{4.545769in}{2.406487in}}%
\pgfpathlineto{\pgfqpoint{4.546843in}{2.407196in}}%
\pgfpathlineto{\pgfqpoint{4.550064in}{2.405930in}}%
\pgfpathlineto{\pgfqpoint{4.551138in}{2.407097in}}%
\pgfpathlineto{\pgfqpoint{4.553286in}{2.403959in}}%
\pgfpathlineto{\pgfqpoint{4.554360in}{2.403084in}}%
\pgfpathlineto{\pgfqpoint{4.557581in}{2.404003in}}%
\pgfpathlineto{\pgfqpoint{4.558655in}{2.403021in}}%
\pgfpathlineto{\pgfqpoint{4.559729in}{2.405251in}}%
\pgfpathlineto{\pgfqpoint{4.560803in}{2.403474in}}%
\pgfpathlineto{\pgfqpoint{4.561876in}{2.400123in}}%
\pgfpathlineto{\pgfqpoint{4.565098in}{2.400174in}}%
\pgfpathlineto{\pgfqpoint{4.566172in}{2.397813in}}%
\pgfpathlineto{\pgfqpoint{4.567246in}{2.407588in}}%
\pgfpathlineto{\pgfqpoint{4.568319in}{2.408810in}}%
\pgfpathlineto{\pgfqpoint{4.569393in}{2.407599in}}%
\pgfpathlineto{\pgfqpoint{4.572615in}{2.406294in}}%
\pgfpathlineto{\pgfqpoint{4.573689in}{2.408346in}}%
\pgfpathlineto{\pgfqpoint{4.574763in}{2.407842in}}%
\pgfpathlineto{\pgfqpoint{4.576910in}{2.405103in}}%
\pgfpathlineto{\pgfqpoint{4.580132in}{2.406075in}}%
\pgfpathlineto{\pgfqpoint{4.581206in}{2.405666in}}%
\pgfpathlineto{\pgfqpoint{4.582279in}{2.404551in}}%
\pgfpathlineto{\pgfqpoint{4.583353in}{2.405034in}}%
\pgfpathlineto{\pgfqpoint{4.584427in}{2.403820in}}%
\pgfpathlineto{\pgfqpoint{4.587649in}{2.404192in}}%
\pgfpathlineto{\pgfqpoint{4.589796in}{2.399681in}}%
\pgfpathlineto{\pgfqpoint{4.591944in}{2.401054in}}%
\pgfpathlineto{\pgfqpoint{4.596239in}{2.403646in}}%
\pgfpathlineto{\pgfqpoint{4.598387in}{2.403858in}}%
\pgfpathlineto{\pgfqpoint{4.599461in}{2.396906in}}%
\pgfpathlineto{\pgfqpoint{4.603756in}{2.398970in}}%
\pgfpathlineto{\pgfqpoint{4.604830in}{2.403025in}}%
\pgfpathlineto{\pgfqpoint{4.605904in}{2.402602in}}%
\pgfpathlineto{\pgfqpoint{4.606978in}{2.407628in}}%
\pgfpathlineto{\pgfqpoint{4.610199in}{2.405591in}}%
\pgfpathlineto{\pgfqpoint{4.611273in}{2.406218in}}%
\pgfpathlineto{\pgfqpoint{4.612347in}{2.407838in}}%
\pgfpathlineto{\pgfqpoint{4.613421in}{2.407447in}}%
\pgfpathlineto{\pgfqpoint{4.614495in}{2.405330in}}%
\pgfpathlineto{\pgfqpoint{4.618790in}{2.404515in}}%
\pgfpathlineto{\pgfqpoint{4.619864in}{2.405646in}}%
\pgfpathlineto{\pgfqpoint{4.620938in}{2.403790in}}%
\pgfpathlineto{\pgfqpoint{4.622011in}{2.404672in}}%
\pgfpathlineto{\pgfqpoint{4.626307in}{2.402436in}}%
\pgfpathlineto{\pgfqpoint{4.627381in}{2.405540in}}%
\pgfpathlineto{\pgfqpoint{4.628454in}{2.410804in}}%
\pgfpathlineto{\pgfqpoint{4.629528in}{2.405616in}}%
\pgfpathlineto{\pgfqpoint{4.632750in}{2.407531in}}%
\pgfpathlineto{\pgfqpoint{4.633824in}{2.407279in}}%
\pgfpathlineto{\pgfqpoint{4.634897in}{2.405662in}}%
\pgfpathlineto{\pgfqpoint{4.635971in}{2.405060in}}%
\pgfpathlineto{\pgfqpoint{4.637045in}{2.406582in}}%
\pgfpathlineto{\pgfqpoint{4.642414in}{2.403022in}}%
\pgfpathlineto{\pgfqpoint{4.644562in}{2.403527in}}%
\pgfpathlineto{\pgfqpoint{4.647784in}{2.402592in}}%
\pgfpathlineto{\pgfqpoint{4.648857in}{2.400848in}}%
\pgfpathlineto{\pgfqpoint{4.649931in}{2.400356in}}%
\pgfpathlineto{\pgfqpoint{4.652079in}{2.396690in}}%
\pgfpathlineto{\pgfqpoint{4.655300in}{2.396789in}}%
\pgfpathlineto{\pgfqpoint{4.657448in}{2.398449in}}%
\pgfpathlineto{\pgfqpoint{4.658522in}{2.398021in}}%
\pgfpathlineto{\pgfqpoint{4.659596in}{2.395793in}}%
\pgfpathlineto{\pgfqpoint{4.663891in}{2.396648in}}%
\pgfpathlineto{\pgfqpoint{4.664965in}{2.398027in}}%
\pgfpathlineto{\pgfqpoint{4.667113in}{2.397776in}}%
\pgfpathlineto{\pgfqpoint{4.670334in}{2.398400in}}%
\pgfpathlineto{\pgfqpoint{4.671408in}{2.397671in}}%
\pgfpathlineto{\pgfqpoint{4.672482in}{2.397970in}}%
\pgfpathlineto{\pgfqpoint{4.673556in}{2.395192in}}%
\pgfpathlineto{\pgfqpoint{4.674629in}{2.395894in}}%
\pgfpathlineto{\pgfqpoint{4.677851in}{2.395674in}}%
\pgfpathlineto{\pgfqpoint{4.681073in}{2.394126in}}%
\pgfpathlineto{\pgfqpoint{4.682146in}{2.395177in}}%
\pgfpathlineto{\pgfqpoint{4.686442in}{2.395128in}}%
\pgfpathlineto{\pgfqpoint{4.687516in}{2.395951in}}%
\pgfpathlineto{\pgfqpoint{4.689663in}{2.398710in}}%
\pgfpathlineto{\pgfqpoint{4.695032in}{2.399800in}}%
\pgfpathlineto{\pgfqpoint{4.697180in}{2.398474in}}%
\pgfpathlineto{\pgfqpoint{4.701475in}{2.402751in}}%
\pgfpathlineto{\pgfqpoint{4.702549in}{2.405217in}}%
\pgfpathlineto{\pgfqpoint{4.704697in}{2.402885in}}%
\pgfpathlineto{\pgfqpoint{4.710066in}{2.403764in}}%
\pgfpathlineto{\pgfqpoint{4.711140in}{2.404542in}}%
\pgfpathlineto{\pgfqpoint{4.712214in}{2.404515in}}%
\pgfpathlineto{\pgfqpoint{4.715435in}{2.402241in}}%
\pgfpathlineto{\pgfqpoint{4.716509in}{2.402715in}}%
\pgfpathlineto{\pgfqpoint{4.717583in}{2.401999in}}%
\pgfpathlineto{\pgfqpoint{4.718657in}{2.402026in}}%
\pgfpathlineto{\pgfqpoint{4.719731in}{2.400316in}}%
\pgfpathlineto{\pgfqpoint{4.722952in}{2.397944in}}%
\pgfpathlineto{\pgfqpoint{4.724026in}{2.398621in}}%
\pgfpathlineto{\pgfqpoint{4.725100in}{2.397661in}}%
\pgfpathlineto{\pgfqpoint{4.726174in}{2.398010in}}%
\pgfpathlineto{\pgfqpoint{4.727248in}{2.399541in}}%
\pgfpathlineto{\pgfqpoint{4.730469in}{2.399771in}}%
\pgfpathlineto{\pgfqpoint{4.732617in}{2.403108in}}%
\pgfpathlineto{\pgfqpoint{4.733691in}{2.402388in}}%
\pgfpathlineto{\pgfqpoint{4.734764in}{2.404212in}}%
\pgfpathlineto{\pgfqpoint{4.746577in}{2.404212in}}%
\pgfpathlineto{\pgfqpoint{4.749798in}{2.399437in}}%
\pgfpathlineto{\pgfqpoint{4.753020in}{2.398895in}}%
\pgfpathlineto{\pgfqpoint{4.754094in}{2.401307in}}%
\pgfpathlineto{\pgfqpoint{4.756241in}{2.396052in}}%
\pgfpathlineto{\pgfqpoint{4.757315in}{2.396077in}}%
\pgfpathlineto{\pgfqpoint{4.760537in}{2.396971in}}%
\pgfpathlineto{\pgfqpoint{4.761610in}{2.394035in}}%
\pgfpathlineto{\pgfqpoint{4.762684in}{2.392969in}}%
\pgfpathlineto{\pgfqpoint{4.764832in}{2.394795in}}%
\pgfpathlineto{\pgfqpoint{4.768053in}{2.392130in}}%
\pgfpathlineto{\pgfqpoint{4.769127in}{2.389858in}}%
\pgfpathlineto{\pgfqpoint{4.771275in}{2.393158in}}%
\pgfpathlineto{\pgfqpoint{4.772349in}{2.392367in}}%
\pgfpathlineto{\pgfqpoint{4.776644in}{2.391822in}}%
\pgfpathlineto{\pgfqpoint{4.777718in}{2.389110in}}%
\pgfpathlineto{\pgfqpoint{4.778792in}{2.390021in}}%
\pgfpathlineto{\pgfqpoint{4.779866in}{2.389676in}}%
\pgfpathlineto{\pgfqpoint{4.783087in}{2.390135in}}%
\pgfpathlineto{\pgfqpoint{4.787383in}{2.385282in}}%
\pgfpathlineto{\pgfqpoint{4.792752in}{2.385559in}}%
\pgfpathlineto{\pgfqpoint{4.793826in}{2.385603in}}%
\pgfpathlineto{\pgfqpoint{4.794899in}{2.385080in}}%
\pgfpathlineto{\pgfqpoint{4.798121in}{2.384257in}}%
\pgfpathlineto{\pgfqpoint{4.800269in}{2.386282in}}%
\pgfpathlineto{\pgfqpoint{4.802416in}{2.386392in}}%
\pgfpathlineto{\pgfqpoint{4.805638in}{2.388000in}}%
\pgfpathlineto{\pgfqpoint{4.806712in}{2.387100in}}%
\pgfpathlineto{\pgfqpoint{4.808859in}{2.389227in}}%
\pgfpathlineto{\pgfqpoint{4.809933in}{2.385731in}}%
\pgfpathlineto{\pgfqpoint{4.813155in}{2.385250in}}%
\pgfpathlineto{\pgfqpoint{4.814228in}{2.387313in}}%
\pgfpathlineto{\pgfqpoint{4.815302in}{2.386622in}}%
\pgfpathlineto{\pgfqpoint{4.816376in}{2.389784in}}%
\pgfpathlineto{\pgfqpoint{4.817450in}{2.389646in}}%
\pgfpathlineto{\pgfqpoint{4.820672in}{2.390956in}}%
\pgfpathlineto{\pgfqpoint{4.821745in}{2.392476in}}%
\pgfpathlineto{\pgfqpoint{4.822819in}{2.389545in}}%
\pgfpathlineto{\pgfqpoint{4.823893in}{2.390464in}}%
\pgfpathlineto{\pgfqpoint{4.824967in}{2.390580in}}%
\pgfpathlineto{\pgfqpoint{4.829262in}{2.391628in}}%
\pgfpathlineto{\pgfqpoint{4.832484in}{2.390242in}}%
\pgfpathlineto{\pgfqpoint{4.835705in}{2.390265in}}%
\pgfpathlineto{\pgfqpoint{4.838927in}{2.393118in}}%
\pgfpathlineto{\pgfqpoint{4.840001in}{2.393287in}}%
\pgfpathlineto{\pgfqpoint{4.843222in}{2.392901in}}%
\pgfpathlineto{\pgfqpoint{4.844296in}{2.391343in}}%
\pgfpathlineto{\pgfqpoint{4.845370in}{2.393671in}}%
\pgfpathlineto{\pgfqpoint{4.846444in}{2.393308in}}%
\pgfpathlineto{\pgfqpoint{4.850739in}{2.394129in}}%
\pgfpathlineto{\pgfqpoint{4.851813in}{2.392376in}}%
\pgfpathlineto{\pgfqpoint{4.853961in}{2.392996in}}%
\pgfpathlineto{\pgfqpoint{4.855034in}{2.394365in}}%
\pgfpathlineto{\pgfqpoint{4.859330in}{2.393754in}}%
\pgfpathlineto{\pgfqpoint{4.860404in}{2.394457in}}%
\pgfpathlineto{\pgfqpoint{4.861477in}{2.396487in}}%
\pgfpathlineto{\pgfqpoint{4.862551in}{2.394506in}}%
\pgfpathlineto{\pgfqpoint{4.865773in}{2.395975in}}%
\pgfpathlineto{\pgfqpoint{4.866847in}{2.394629in}}%
\pgfpathlineto{\pgfqpoint{4.880806in}{2.394629in}}%
\pgfpathlineto{\pgfqpoint{4.881880in}{2.392939in}}%
\pgfpathlineto{\pgfqpoint{4.884028in}{2.392939in}}%
\pgfpathlineto{\pgfqpoint{4.885102in}{2.389634in}}%
\pgfpathlineto{\pgfqpoint{4.899062in}{2.389634in}}%
\pgfpathlineto{\pgfqpoint{4.900136in}{2.388696in}}%
\pgfpathlineto{\pgfqpoint{4.910874in}{2.388696in}}%
\pgfpathlineto{\pgfqpoint{4.913022in}{2.387274in}}%
\pgfpathlineto{\pgfqpoint{4.915169in}{2.387789in}}%
\pgfpathlineto{\pgfqpoint{4.918391in}{2.386095in}}%
\pgfpathlineto{\pgfqpoint{4.919465in}{2.386234in}}%
\pgfpathlineto{\pgfqpoint{4.920539in}{2.388147in}}%
\pgfpathlineto{\pgfqpoint{4.922686in}{2.388147in}}%
\pgfpathlineto{\pgfqpoint{4.926982in}{2.384311in}}%
\pgfpathlineto{\pgfqpoint{4.928055in}{2.384903in}}%
\pgfpathlineto{\pgfqpoint{4.930203in}{2.384399in}}%
\pgfpathlineto{\pgfqpoint{4.933425in}{2.382804in}}%
\pgfpathlineto{\pgfqpoint{4.937720in}{2.384377in}}%
\pgfpathlineto{\pgfqpoint{4.940941in}{2.382987in}}%
\pgfpathlineto{\pgfqpoint{4.942015in}{2.381377in}}%
\pgfpathlineto{\pgfqpoint{4.943089in}{2.381114in}}%
\pgfpathlineto{\pgfqpoint{4.945237in}{2.382451in}}%
\pgfpathlineto{\pgfqpoint{4.949532in}{2.382318in}}%
\pgfpathlineto{\pgfqpoint{4.950606in}{2.380810in}}%
\pgfpathlineto{\pgfqpoint{4.951680in}{2.380614in}}%
\pgfpathlineto{\pgfqpoint{4.957049in}{2.381527in}}%
\pgfpathlineto{\pgfqpoint{4.958123in}{2.381988in}}%
\pgfpathlineto{\pgfqpoint{4.959197in}{2.380509in}}%
\pgfpathlineto{\pgfqpoint{4.960271in}{2.380032in}}%
\pgfpathlineto{\pgfqpoint{4.963492in}{2.377257in}}%
\pgfpathlineto{\pgfqpoint{4.964566in}{2.378128in}}%
\pgfpathlineto{\pgfqpoint{4.965640in}{2.377939in}}%
\pgfpathlineto{\pgfqpoint{4.972083in}{2.380480in}}%
\pgfpathlineto{\pgfqpoint{4.973157in}{2.378726in}}%
\pgfpathlineto{\pgfqpoint{4.975304in}{2.378177in}}%
\pgfpathlineto{\pgfqpoint{4.980673in}{2.380378in}}%
\pgfpathlineto{\pgfqpoint{4.982821in}{2.381920in}}%
\pgfpathlineto{\pgfqpoint{4.986043in}{2.382671in}}%
\pgfpathlineto{\pgfqpoint{4.989264in}{2.381318in}}%
\pgfpathlineto{\pgfqpoint{4.995707in}{2.382328in}}%
\pgfpathlineto{\pgfqpoint{4.997855in}{2.381045in}}%
\pgfpathlineto{\pgfqpoint{5.005372in}{2.382445in}}%
\pgfpathlineto{\pgfqpoint{5.010741in}{2.382400in}}%
\pgfpathlineto{\pgfqpoint{5.012889in}{2.383180in}}%
\pgfpathlineto{\pgfqpoint{5.018258in}{2.382887in}}%
\pgfpathlineto{\pgfqpoint{5.019332in}{2.382037in}}%
\pgfpathlineto{\pgfqpoint{5.020405in}{2.382634in}}%
\pgfpathlineto{\pgfqpoint{5.025775in}{2.383392in}}%
\pgfpathlineto{\pgfqpoint{5.033292in}{2.382850in}}%
\pgfpathlineto{\pgfqpoint{5.034365in}{2.382042in}}%
\pgfpathlineto{\pgfqpoint{5.038661in}{2.381665in}}%
\pgfpathlineto{\pgfqpoint{5.039735in}{2.380959in}}%
\pgfpathlineto{\pgfqpoint{5.040808in}{2.378902in}}%
\pgfpathlineto{\pgfqpoint{5.041882in}{2.379009in}}%
\pgfpathlineto{\pgfqpoint{5.042956in}{2.379818in}}%
\pgfpathlineto{\pgfqpoint{5.048325in}{2.379044in}}%
\pgfpathlineto{\pgfqpoint{5.049399in}{2.380025in}}%
\pgfpathlineto{\pgfqpoint{5.050473in}{2.379765in}}%
\pgfpathlineto{\pgfqpoint{5.056916in}{2.380280in}}%
\pgfpathlineto{\pgfqpoint{5.057990in}{2.381581in}}%
\pgfpathlineto{\pgfqpoint{5.063359in}{2.382154in}}%
\pgfpathlineto{\pgfqpoint{5.064433in}{2.382154in}}%
\pgfpathlineto{\pgfqpoint{5.065507in}{2.377619in}}%
\pgfpathlineto{\pgfqpoint{5.068728in}{2.378412in}}%
\pgfpathlineto{\pgfqpoint{5.069802in}{2.378065in}}%
\pgfpathlineto{\pgfqpoint{5.070876in}{2.374978in}}%
\pgfpathlineto{\pgfqpoint{5.071950in}{2.376655in}}%
\pgfpathlineto{\pgfqpoint{5.073024in}{2.375193in}}%
\pgfpathlineto{\pgfqpoint{5.077319in}{2.374340in}}%
\pgfpathlineto{\pgfqpoint{5.078393in}{2.374237in}}%
\pgfpathlineto{\pgfqpoint{5.079467in}{2.373418in}}%
\pgfpathlineto{\pgfqpoint{5.080540in}{2.373721in}}%
\pgfpathlineto{\pgfqpoint{5.088057in}{2.370953in}}%
\pgfpathlineto{\pgfqpoint{5.093427in}{2.371089in}}%
\pgfpathlineto{\pgfqpoint{5.094500in}{2.371559in}}%
\pgfpathlineto{\pgfqpoint{5.095574in}{2.371421in}}%
\pgfpathlineto{\pgfqpoint{5.100943in}{2.371776in}}%
\pgfpathlineto{\pgfqpoint{5.102017in}{2.370430in}}%
\pgfpathlineto{\pgfqpoint{5.103091in}{2.370314in}}%
\pgfpathlineto{\pgfqpoint{5.107386in}{2.371619in}}%
\pgfpathlineto{\pgfqpoint{5.108460in}{2.370513in}}%
\pgfpathlineto{\pgfqpoint{5.110608in}{2.371155in}}%
\pgfpathlineto{\pgfqpoint{5.118125in}{2.369134in}}%
\pgfpathlineto{\pgfqpoint{5.121346in}{2.368582in}}%
\pgfpathlineto{\pgfqpoint{5.122420in}{2.369187in}}%
\pgfpathlineto{\pgfqpoint{5.123494in}{2.367890in}}%
\pgfpathlineto{\pgfqpoint{5.124568in}{2.369331in}}%
\pgfpathlineto{\pgfqpoint{5.125642in}{2.368929in}}%
\pgfpathlineto{\pgfqpoint{5.128863in}{2.369157in}}%
\pgfpathlineto{\pgfqpoint{5.131011in}{2.371290in}}%
\pgfpathlineto{\pgfqpoint{5.132085in}{2.371428in}}%
\pgfpathlineto{\pgfqpoint{5.133159in}{2.370186in}}%
\pgfpathlineto{\pgfqpoint{5.137454in}{2.371235in}}%
\pgfpathlineto{\pgfqpoint{5.138528in}{2.369526in}}%
\pgfpathlineto{\pgfqpoint{5.139602in}{2.370369in}}%
\pgfpathlineto{\pgfqpoint{5.140675in}{2.368680in}}%
\pgfpathlineto{\pgfqpoint{5.143897in}{2.370802in}}%
\pgfpathlineto{\pgfqpoint{5.144971in}{2.370509in}}%
\pgfpathlineto{\pgfqpoint{5.147118in}{2.372706in}}%
\pgfpathlineto{\pgfqpoint{5.148192in}{2.370938in}}%
\pgfpathlineto{\pgfqpoint{5.153561in}{2.368284in}}%
\pgfpathlineto{\pgfqpoint{5.154635in}{2.369906in}}%
\pgfpathlineto{\pgfqpoint{5.155709in}{2.366701in}}%
\pgfpathlineto{\pgfqpoint{5.163226in}{2.363408in}}%
\pgfpathlineto{\pgfqpoint{5.166448in}{2.363250in}}%
\pgfpathlineto{\pgfqpoint{5.167521in}{2.361360in}}%
\pgfpathlineto{\pgfqpoint{5.168595in}{2.360833in}}%
\pgfpathlineto{\pgfqpoint{5.170743in}{2.360630in}}%
\pgfpathlineto{\pgfqpoint{5.176112in}{2.360124in}}%
\pgfpathlineto{\pgfqpoint{5.178260in}{2.362107in}}%
\pgfpathlineto{\pgfqpoint{5.182555in}{2.362347in}}%
\pgfpathlineto{\pgfqpoint{5.183629in}{2.361190in}}%
\pgfpathlineto{\pgfqpoint{5.185777in}{2.361375in}}%
\pgfpathlineto{\pgfqpoint{5.188998in}{2.362090in}}%
\pgfpathlineto{\pgfqpoint{5.190072in}{2.361075in}}%
\pgfpathlineto{\pgfqpoint{5.191146in}{2.360957in}}%
\pgfpathlineto{\pgfqpoint{5.193293in}{2.358708in}}%
\pgfpathlineto{\pgfqpoint{5.196515in}{2.359198in}}%
\pgfpathlineto{\pgfqpoint{5.197589in}{2.357800in}}%
\pgfpathlineto{\pgfqpoint{5.198663in}{2.359280in}}%
\pgfpathlineto{\pgfqpoint{5.199737in}{2.358456in}}%
\pgfpathlineto{\pgfqpoint{5.200810in}{2.358733in}}%
\pgfpathlineto{\pgfqpoint{5.205106in}{2.358275in}}%
\pgfpathlineto{\pgfqpoint{5.206180in}{2.359264in}}%
\pgfpathlineto{\pgfqpoint{5.207253in}{2.358671in}}%
\pgfpathlineto{\pgfqpoint{5.208327in}{2.359879in}}%
\pgfpathlineto{\pgfqpoint{5.212623in}{2.360160in}}%
\pgfpathlineto{\pgfqpoint{5.214770in}{2.356422in}}%
\pgfpathlineto{\pgfqpoint{5.219066in}{2.355604in}}%
\pgfpathlineto{\pgfqpoint{5.220139in}{2.354686in}}%
\pgfpathlineto{\pgfqpoint{5.221213in}{2.354885in}}%
\pgfpathlineto{\pgfqpoint{5.223361in}{2.354455in}}%
\pgfpathlineto{\pgfqpoint{5.227656in}{2.355850in}}%
\pgfpathlineto{\pgfqpoint{5.228730in}{2.357614in}}%
\pgfpathlineto{\pgfqpoint{5.230878in}{2.358514in}}%
\pgfpathlineto{\pgfqpoint{5.236247in}{2.359029in}}%
\pgfpathlineto{\pgfqpoint{5.237321in}{2.357566in}}%
\pgfpathlineto{\pgfqpoint{5.238395in}{2.358739in}}%
\pgfpathlineto{\pgfqpoint{5.241616in}{2.359213in}}%
\pgfpathlineto{\pgfqpoint{5.242690in}{2.358669in}}%
\pgfpathlineto{\pgfqpoint{5.244838in}{2.359291in}}%
\pgfpathlineto{\pgfqpoint{5.245912in}{2.357311in}}%
\pgfpathlineto{\pgfqpoint{5.251281in}{2.357151in}}%
\pgfpathlineto{\pgfqpoint{5.252355in}{2.355906in}}%
\pgfpathlineto{\pgfqpoint{5.253428in}{2.358002in}}%
\pgfpathlineto{\pgfqpoint{5.256650in}{2.358698in}}%
\pgfpathlineto{\pgfqpoint{5.257724in}{2.363083in}}%
\pgfpathlineto{\pgfqpoint{5.258798in}{2.365117in}}%
\pgfpathlineto{\pgfqpoint{5.259871in}{2.364303in}}%
\pgfpathlineto{\pgfqpoint{5.260945in}{2.366429in}}%
\pgfpathlineto{\pgfqpoint{5.264167in}{2.365101in}}%
\pgfpathlineto{\pgfqpoint{5.265241in}{2.363798in}}%
\pgfpathlineto{\pgfqpoint{5.266315in}{2.364047in}}%
\pgfpathlineto{\pgfqpoint{5.267388in}{2.362640in}}%
\pgfpathlineto{\pgfqpoint{5.268462in}{2.364296in}}%
\pgfpathlineto{\pgfqpoint{5.271684in}{2.365154in}}%
\pgfpathlineto{\pgfqpoint{5.274905in}{2.363626in}}%
\pgfpathlineto{\pgfqpoint{5.280274in}{2.364460in}}%
\pgfpathlineto{\pgfqpoint{5.281348in}{2.363259in}}%
\pgfpathlineto{\pgfqpoint{5.282422in}{2.364861in}}%
\pgfpathlineto{\pgfqpoint{5.283496in}{2.365402in}}%
\pgfpathlineto{\pgfqpoint{5.288865in}{2.364476in}}%
\pgfpathlineto{\pgfqpoint{5.291013in}{2.364978in}}%
\pgfpathlineto{\pgfqpoint{5.295308in}{2.364940in}}%
\pgfpathlineto{\pgfqpoint{5.296382in}{2.366204in}}%
\pgfpathlineto{\pgfqpoint{5.297456in}{2.365762in}}%
\pgfpathlineto{\pgfqpoint{5.298530in}{2.368870in}}%
\pgfpathlineto{\pgfqpoint{5.303899in}{2.368162in}}%
\pgfpathlineto{\pgfqpoint{5.306047in}{2.368162in}}%
\pgfpathlineto{\pgfqpoint{5.309268in}{2.365275in}}%
\pgfpathlineto{\pgfqpoint{5.310342in}{2.366418in}}%
\pgfpathlineto{\pgfqpoint{5.311416in}{2.364954in}}%
\pgfpathlineto{\pgfqpoint{5.312490in}{2.365542in}}%
\pgfpathlineto{\pgfqpoint{5.313563in}{2.363356in}}%
\pgfpathlineto{\pgfqpoint{5.316785in}{2.363177in}}%
\pgfpathlineto{\pgfqpoint{5.318933in}{2.366069in}}%
\pgfpathlineto{\pgfqpoint{5.321080in}{2.366069in}}%
\pgfpathlineto{\pgfqpoint{5.326449in}{2.365410in}}%
\pgfpathlineto{\pgfqpoint{5.327523in}{2.365410in}}%
\pgfpathlineto{\pgfqpoint{5.341483in}{2.362630in}}%
\pgfpathlineto{\pgfqpoint{5.346852in}{2.362099in}}%
\pgfpathlineto{\pgfqpoint{5.349000in}{2.361754in}}%
\pgfpathlineto{\pgfqpoint{5.650749in}{2.361754in}}%
\pgfpathlineto{\pgfqpoint{5.651823in}{2.358700in}}%
\pgfpathlineto{\pgfqpoint{5.656118in}{2.359585in}}%
\pgfpathlineto{\pgfqpoint{5.657192in}{2.358179in}}%
\pgfpathlineto{\pgfqpoint{5.658266in}{2.358806in}}%
\pgfpathlineto{\pgfqpoint{5.662561in}{2.355811in}}%
\pgfpathlineto{\pgfqpoint{5.663635in}{2.355777in}}%
\pgfpathlineto{\pgfqpoint{5.664709in}{2.357302in}}%
\pgfpathlineto{\pgfqpoint{5.665782in}{2.357302in}}%
\pgfpathlineto{\pgfqpoint{5.666856in}{2.355600in}}%
\pgfpathlineto{\pgfqpoint{5.671152in}{2.354866in}}%
\pgfpathlineto{\pgfqpoint{5.672225in}{2.353382in}}%
\pgfpathlineto{\pgfqpoint{5.674373in}{2.354357in}}%
\pgfpathlineto{\pgfqpoint{5.677595in}{2.353855in}}%
\pgfpathlineto{\pgfqpoint{5.679742in}{2.354704in}}%
\pgfpathlineto{\pgfqpoint{5.680816in}{2.353526in}}%
\pgfpathlineto{\pgfqpoint{5.681890in}{2.355358in}}%
\pgfpathlineto{\pgfqpoint{5.685112in}{2.355358in}}%
\pgfpathlineto{\pgfqpoint{5.688333in}{2.351512in}}%
\pgfpathlineto{\pgfqpoint{5.689407in}{2.350576in}}%
\pgfpathlineto{\pgfqpoint{5.694776in}{2.351324in}}%
\pgfpathlineto{\pgfqpoint{5.696924in}{2.353091in}}%
\pgfpathlineto{\pgfqpoint{5.701219in}{2.353734in}}%
\pgfpathlineto{\pgfqpoint{5.702293in}{2.353683in}}%
\pgfpathlineto{\pgfqpoint{5.703367in}{2.351575in}}%
\pgfpathlineto{\pgfqpoint{5.704441in}{2.350993in}}%
\pgfpathlineto{\pgfqpoint{5.707662in}{2.350753in}}%
\pgfpathlineto{\pgfqpoint{5.708736in}{2.351582in}}%
\pgfpathlineto{\pgfqpoint{5.710884in}{2.351372in}}%
\pgfpathlineto{\pgfqpoint{5.718401in}{2.352211in}}%
\pgfpathlineto{\pgfqpoint{5.719474in}{2.350414in}}%
\pgfpathlineto{\pgfqpoint{5.723770in}{2.350938in}}%
\pgfpathlineto{\pgfqpoint{5.724844in}{2.350010in}}%
\pgfpathlineto{\pgfqpoint{5.725917in}{2.350814in}}%
\pgfpathlineto{\pgfqpoint{5.731287in}{2.351403in}}%
\pgfpathlineto{\pgfqpoint{5.732360in}{2.351935in}}%
\pgfpathlineto{\pgfqpoint{5.738803in}{2.349784in}}%
\pgfpathlineto{\pgfqpoint{5.740951in}{2.353344in}}%
\pgfpathlineto{\pgfqpoint{5.747394in}{2.352413in}}%
\pgfpathlineto{\pgfqpoint{5.749542in}{2.352413in}}%
\pgfpathlineto{\pgfqpoint{5.754911in}{2.350202in}}%
\pgfpathlineto{\pgfqpoint{5.755985in}{2.350635in}}%
\pgfpathlineto{\pgfqpoint{5.757059in}{2.349421in}}%
\pgfpathlineto{\pgfqpoint{5.763502in}{2.349021in}}%
\pgfpathlineto{\pgfqpoint{5.764576in}{2.350706in}}%
\pgfpathlineto{\pgfqpoint{5.767797in}{2.350123in}}%
\pgfpathlineto{\pgfqpoint{5.768871in}{2.351584in}}%
\pgfpathlineto{\pgfqpoint{5.775314in}{2.351584in}}%
\pgfpathlineto{\pgfqpoint{5.777462in}{2.349679in}}%
\pgfpathlineto{\pgfqpoint{5.778535in}{2.350064in}}%
\pgfpathlineto{\pgfqpoint{5.779609in}{2.349757in}}%
\pgfpathlineto{\pgfqpoint{5.783905in}{2.350319in}}%
\pgfpathlineto{\pgfqpoint{5.784979in}{2.349218in}}%
\pgfpathlineto{\pgfqpoint{5.794643in}{2.347216in}}%
\pgfpathlineto{\pgfqpoint{5.797865in}{2.348092in}}%
\pgfpathlineto{\pgfqpoint{5.798938in}{2.346984in}}%
\pgfpathlineto{\pgfqpoint{5.800012in}{2.347535in}}%
\pgfpathlineto{\pgfqpoint{5.801086in}{2.346902in}}%
\pgfpathlineto{\pgfqpoint{5.802160in}{2.347284in}}%
\pgfpathlineto{\pgfqpoint{5.806455in}{2.346898in}}%
\pgfpathlineto{\pgfqpoint{5.807529in}{2.346669in}}%
\pgfpathlineto{\pgfqpoint{5.808603in}{2.345801in}}%
\pgfpathlineto{\pgfqpoint{5.809677in}{2.348440in}}%
\pgfpathlineto{\pgfqpoint{5.812898in}{2.349901in}}%
\pgfpathlineto{\pgfqpoint{5.816120in}{2.345026in}}%
\pgfpathlineto{\pgfqpoint{5.822563in}{2.344545in}}%
\pgfpathlineto{\pgfqpoint{5.823637in}{2.344809in}}%
\pgfpathlineto{\pgfqpoint{5.824711in}{2.343555in}}%
\pgfpathlineto{\pgfqpoint{5.835449in}{2.343338in}}%
\pgfpathlineto{\pgfqpoint{5.837597in}{2.343265in}}%
\pgfpathlineto{\pgfqpoint{5.838670in}{2.343351in}}%
\pgfpathlineto{\pgfqpoint{5.839744in}{2.342747in}}%
\pgfpathlineto{\pgfqpoint{5.842966in}{2.342647in}}%
\pgfpathlineto{\pgfqpoint{5.845113in}{2.344378in}}%
\pgfpathlineto{\pgfqpoint{5.851557in}{2.341375in}}%
\pgfpathlineto{\pgfqpoint{5.852630in}{2.342374in}}%
\pgfpathlineto{\pgfqpoint{5.854778in}{2.342614in}}%
\pgfpathlineto{\pgfqpoint{5.859073in}{2.342344in}}%
\pgfpathlineto{\pgfqpoint{5.862295in}{2.341013in}}%
\pgfpathlineto{\pgfqpoint{5.867664in}{2.341104in}}%
\pgfpathlineto{\pgfqpoint{5.868738in}{2.340497in}}%
\pgfpathlineto{\pgfqpoint{5.873033in}{2.341236in}}%
\pgfpathlineto{\pgfqpoint{5.874107in}{2.340544in}}%
\pgfpathlineto{\pgfqpoint{5.875181in}{2.340653in}}%
\pgfpathlineto{\pgfqpoint{5.876255in}{2.339913in}}%
\pgfpathlineto{\pgfqpoint{5.877329in}{2.340319in}}%
\pgfpathlineto{\pgfqpoint{5.880550in}{2.339420in}}%
\pgfpathlineto{\pgfqpoint{5.882698in}{2.340630in}}%
\pgfpathlineto{\pgfqpoint{5.883772in}{2.339384in}}%
\pgfpathlineto{\pgfqpoint{5.884846in}{2.339517in}}%
\pgfpathlineto{\pgfqpoint{5.889141in}{2.338981in}}%
\pgfpathlineto{\pgfqpoint{5.892362in}{2.341917in}}%
\pgfpathlineto{\pgfqpoint{5.895584in}{2.339363in}}%
\pgfpathlineto{\pgfqpoint{5.896658in}{2.340823in}}%
\pgfpathlineto{\pgfqpoint{5.897732in}{2.340865in}}%
\pgfpathlineto{\pgfqpoint{5.898805in}{2.339556in}}%
\pgfpathlineto{\pgfqpoint{5.903101in}{2.339179in}}%
\pgfpathlineto{\pgfqpoint{5.904175in}{2.338926in}}%
\pgfpathlineto{\pgfqpoint{5.905248in}{2.339869in}}%
\pgfpathlineto{\pgfqpoint{5.906322in}{2.338395in}}%
\pgfpathlineto{\pgfqpoint{5.907396in}{2.339869in}}%
\pgfpathlineto{\pgfqpoint{5.910618in}{2.339761in}}%
\pgfpathlineto{\pgfqpoint{5.911691in}{2.339140in}}%
\pgfpathlineto{\pgfqpoint{5.912765in}{2.337805in}}%
\pgfpathlineto{\pgfqpoint{5.913839in}{2.339250in}}%
\pgfpathlineto{\pgfqpoint{5.914913in}{2.337403in}}%
\pgfpathlineto{\pgfqpoint{5.919208in}{2.339054in}}%
\pgfpathlineto{\pgfqpoint{5.922430in}{2.337077in}}%
\pgfpathlineto{\pgfqpoint{5.928873in}{2.339171in}}%
\pgfpathlineto{\pgfqpoint{5.929947in}{2.338930in}}%
\pgfpathlineto{\pgfqpoint{5.934242in}{2.340121in}}%
\pgfpathlineto{\pgfqpoint{5.936390in}{2.342457in}}%
\pgfpathlineto{\pgfqpoint{5.937464in}{2.343239in}}%
\pgfpathlineto{\pgfqpoint{5.940685in}{2.343541in}}%
\pgfpathlineto{\pgfqpoint{5.941759in}{2.339731in}}%
\pgfpathlineto{\pgfqpoint{5.942833in}{2.339203in}}%
\pgfpathlineto{\pgfqpoint{5.943907in}{2.340208in}}%
\pgfpathlineto{\pgfqpoint{5.944980in}{2.339880in}}%
\pgfpathlineto{\pgfqpoint{5.951423in}{2.340179in}}%
\pgfpathlineto{\pgfqpoint{5.952497in}{2.342076in}}%
\pgfpathlineto{\pgfqpoint{5.956793in}{2.339035in}}%
\pgfpathlineto{\pgfqpoint{5.957867in}{2.340907in}}%
\pgfpathlineto{\pgfqpoint{5.958940in}{2.344687in}}%
\pgfpathlineto{\pgfqpoint{5.960014in}{2.343841in}}%
\pgfpathlineto{\pgfqpoint{5.963236in}{2.344616in}}%
\pgfpathlineto{\pgfqpoint{5.964310in}{2.343771in}}%
\pgfpathlineto{\pgfqpoint{5.967531in}{2.345967in}}%
\pgfpathlineto{\pgfqpoint{5.975048in}{2.343951in}}%
\pgfpathlineto{\pgfqpoint{5.980417in}{2.345310in}}%
\pgfpathlineto{\pgfqpoint{5.981491in}{2.346136in}}%
\pgfpathlineto{\pgfqpoint{5.982565in}{2.345374in}}%
\pgfpathlineto{\pgfqpoint{5.986860in}{2.344666in}}%
\pgfpathlineto{\pgfqpoint{5.987934in}{2.342944in}}%
\pgfpathlineto{\pgfqpoint{5.989008in}{2.343837in}}%
\pgfpathlineto{\pgfqpoint{5.990082in}{2.342666in}}%
\pgfpathlineto{\pgfqpoint{5.994377in}{2.341607in}}%
\pgfpathlineto{\pgfqpoint{5.995451in}{2.342647in}}%
\pgfpathlineto{\pgfqpoint{5.997599in}{2.342246in}}%
\pgfpathlineto{\pgfqpoint{6.001894in}{2.342388in}}%
\pgfpathlineto{\pgfqpoint{6.002968in}{2.342773in}}%
\pgfpathlineto{\pgfqpoint{6.005115in}{2.341871in}}%
\pgfpathlineto{\pgfqpoint{6.009411in}{2.342338in}}%
\pgfpathlineto{\pgfqpoint{6.010485in}{2.343036in}}%
\pgfpathlineto{\pgfqpoint{6.011558in}{2.342661in}}%
\pgfpathlineto{\pgfqpoint{6.012632in}{2.343019in}}%
\pgfpathlineto{\pgfqpoint{6.018001in}{2.342458in}}%
\pgfpathlineto{\pgfqpoint{6.019075in}{2.341729in}}%
\pgfpathlineto{\pgfqpoint{6.020149in}{2.341772in}}%
\pgfpathlineto{\pgfqpoint{6.023371in}{2.342590in}}%
\pgfpathlineto{\pgfqpoint{6.024445in}{2.343778in}}%
\pgfpathlineto{\pgfqpoint{6.033035in}{2.340984in}}%
\pgfpathlineto{\pgfqpoint{6.034109in}{2.341290in}}%
\pgfpathlineto{\pgfqpoint{6.035183in}{2.337737in}}%
\pgfpathlineto{\pgfqpoint{6.038404in}{2.338327in}}%
\pgfpathlineto{\pgfqpoint{6.039478in}{2.337227in}}%
\pgfpathlineto{\pgfqpoint{6.041626in}{2.338744in}}%
\pgfpathlineto{\pgfqpoint{6.050217in}{2.337736in}}%
\pgfpathlineto{\pgfqpoint{6.053438in}{2.337749in}}%
\pgfpathlineto{\pgfqpoint{6.055586in}{2.336626in}}%
\pgfpathlineto{\pgfqpoint{6.056660in}{2.336228in}}%
\pgfpathlineto{\pgfqpoint{6.057734in}{2.337044in}}%
\pgfpathlineto{\pgfqpoint{6.060955in}{2.336643in}}%
\pgfpathlineto{\pgfqpoint{6.062029in}{2.337182in}}%
\pgfpathlineto{\pgfqpoint{6.063103in}{2.333263in}}%
\pgfpathlineto{\pgfqpoint{6.064177in}{2.333636in}}%
\pgfpathlineto{\pgfqpoint{6.065250in}{2.333296in}}%
\pgfpathlineto{\pgfqpoint{6.070620in}{2.332907in}}%
\pgfpathlineto{\pgfqpoint{6.072767in}{2.333339in}}%
\pgfpathlineto{\pgfqpoint{6.077063in}{2.333314in}}%
\pgfpathlineto{\pgfqpoint{6.078136in}{2.332650in}}%
\pgfpathlineto{\pgfqpoint{6.080284in}{2.333944in}}%
\pgfpathlineto{\pgfqpoint{6.086727in}{2.334114in}}%
\pgfpathlineto{\pgfqpoint{6.087801in}{2.333293in}}%
\pgfpathlineto{\pgfqpoint{6.094244in}{2.332881in}}%
\pgfpathlineto{\pgfqpoint{6.095318in}{2.333359in}}%
\pgfpathlineto{\pgfqpoint{6.100687in}{2.333370in}}%
\pgfpathlineto{\pgfqpoint{6.102835in}{2.333842in}}%
\pgfpathlineto{\pgfqpoint{6.109278in}{2.334255in}}%
\pgfpathlineto{\pgfqpoint{6.110352in}{2.334661in}}%
\pgfpathlineto{\pgfqpoint{6.121090in}{2.335065in}}%
\pgfpathlineto{\pgfqpoint{6.124311in}{2.334439in}}%
\pgfpathlineto{\pgfqpoint{6.130755in}{2.334137in}}%
\pgfpathlineto{\pgfqpoint{6.132902in}{2.335270in}}%
\pgfpathlineto{\pgfqpoint{6.137198in}{2.333664in}}%
\pgfpathlineto{\pgfqpoint{6.138271in}{2.336228in}}%
\pgfpathlineto{\pgfqpoint{6.143641in}{2.336998in}}%
\pgfpathlineto{\pgfqpoint{6.144714in}{2.338048in}}%
\pgfpathlineto{\pgfqpoint{6.147936in}{2.337705in}}%
\pgfpathlineto{\pgfqpoint{6.160822in}{2.337992in}}%
\pgfpathlineto{\pgfqpoint{6.161896in}{2.338481in}}%
\pgfpathlineto{\pgfqpoint{6.162970in}{2.338014in}}%
\pgfpathlineto{\pgfqpoint{6.168339in}{2.337695in}}%
\pgfpathlineto{\pgfqpoint{6.170487in}{2.336779in}}%
\pgfpathlineto{\pgfqpoint{6.174782in}{2.336598in}}%
\pgfpathlineto{\pgfqpoint{6.175856in}{2.335774in}}%
\pgfpathlineto{\pgfqpoint{6.183373in}{2.334979in}}%
\pgfpathlineto{\pgfqpoint{6.184446in}{2.336040in}}%
\pgfpathlineto{\pgfqpoint{6.185520in}{2.335683in}}%
\pgfpathlineto{\pgfqpoint{6.190889in}{2.335340in}}%
\pgfpathlineto{\pgfqpoint{6.193037in}{2.333926in}}%
\pgfpathlineto{\pgfqpoint{6.197333in}{2.333962in}}%
\pgfpathlineto{\pgfqpoint{6.200554in}{2.334190in}}%
\pgfpathlineto{\pgfqpoint{6.203776in}{2.334264in}}%
\pgfpathlineto{\pgfqpoint{6.206997in}{2.337013in}}%
\pgfpathlineto{\pgfqpoint{6.213440in}{2.336196in}}%
\pgfpathlineto{\pgfqpoint{6.214514in}{2.336554in}}%
\pgfpathlineto{\pgfqpoint{6.215588in}{2.336219in}}%
\pgfpathlineto{\pgfqpoint{6.222031in}{2.337350in}}%
\pgfpathlineto{\pgfqpoint{6.223105in}{2.336866in}}%
\pgfpathlineto{\pgfqpoint{6.226326in}{2.336322in}}%
\pgfpathlineto{\pgfqpoint{6.228474in}{2.334552in}}%
\pgfpathlineto{\pgfqpoint{6.230622in}{2.334255in}}%
\pgfpathlineto{\pgfqpoint{6.233843in}{2.334748in}}%
\pgfpathlineto{\pgfqpoint{6.234917in}{2.333628in}}%
\pgfpathlineto{\pgfqpoint{6.235991in}{2.333446in}}%
\pgfpathlineto{\pgfqpoint{6.237065in}{2.331881in}}%
\pgfpathlineto{\pgfqpoint{6.238138in}{2.332399in}}%
\pgfpathlineto{\pgfqpoint{6.243508in}{2.331475in}}%
\pgfpathlineto{\pgfqpoint{6.245655in}{2.331884in}}%
\pgfpathlineto{\pgfqpoint{6.252098in}{2.330701in}}%
\pgfpathlineto{\pgfqpoint{6.253172in}{2.331137in}}%
\pgfpathlineto{\pgfqpoint{6.257467in}{2.330203in}}%
\pgfpathlineto{\pgfqpoint{6.258541in}{2.329941in}}%
\pgfpathlineto{\pgfqpoint{6.259615in}{2.330337in}}%
\pgfpathlineto{\pgfqpoint{6.260689in}{2.329915in}}%
\pgfpathlineto{\pgfqpoint{6.264984in}{2.329576in}}%
\pgfpathlineto{\pgfqpoint{6.267132in}{2.330116in}}%
\pgfpathlineto{\pgfqpoint{6.268206in}{2.329866in}}%
\pgfpathlineto{\pgfqpoint{6.272501in}{2.330069in}}%
\pgfpathlineto{\pgfqpoint{6.273575in}{2.330546in}}%
\pgfpathlineto{\pgfqpoint{6.275723in}{2.329838in}}%
\pgfpathlineto{\pgfqpoint{6.283240in}{2.329511in}}%
\pgfpathlineto{\pgfqpoint{6.286461in}{2.328300in}}%
\pgfpathlineto{\pgfqpoint{6.288609in}{2.328748in}}%
\pgfpathlineto{\pgfqpoint{6.293978in}{2.329168in}}%
\pgfpathlineto{\pgfqpoint{6.295052in}{2.328099in}}%
\pgfpathlineto{\pgfqpoint{6.296126in}{2.327870in}}%
\pgfpathlineto{\pgfqpoint{6.297199in}{2.329659in}}%
\pgfpathlineto{\pgfqpoint{6.298273in}{2.330076in}}%
\pgfpathlineto{\pgfqpoint{6.302569in}{2.329643in}}%
\pgfpathlineto{\pgfqpoint{6.303643in}{2.331522in}}%
\pgfpathlineto{\pgfqpoint{6.305790in}{2.331406in}}%
\pgfpathlineto{\pgfqpoint{6.311159in}{2.329834in}}%
\pgfpathlineto{\pgfqpoint{6.312233in}{2.330252in}}%
\pgfpathlineto{\pgfqpoint{6.313307in}{2.329933in}}%
\pgfpathlineto{\pgfqpoint{6.319750in}{2.330117in}}%
\pgfpathlineto{\pgfqpoint{6.320824in}{2.329162in}}%
\pgfpathlineto{\pgfqpoint{6.326193in}{2.329451in}}%
\pgfpathlineto{\pgfqpoint{6.327267in}{2.329967in}}%
\pgfpathlineto{\pgfqpoint{6.328341in}{2.333549in}}%
\pgfpathlineto{\pgfqpoint{6.333710in}{2.335158in}}%
\pgfpathlineto{\pgfqpoint{6.334784in}{2.334403in}}%
\pgfpathlineto{\pgfqpoint{6.335858in}{2.334936in}}%
\pgfpathlineto{\pgfqpoint{6.340153in}{2.335762in}}%
\pgfpathlineto{\pgfqpoint{6.341227in}{2.335100in}}%
\pgfpathlineto{\pgfqpoint{6.343375in}{2.335476in}}%
\pgfpathlineto{\pgfqpoint{6.346596in}{2.336109in}}%
\pgfpathlineto{\pgfqpoint{6.348744in}{2.334266in}}%
\pgfpathlineto{\pgfqpoint{6.355187in}{2.332767in}}%
\pgfpathlineto{\pgfqpoint{6.356261in}{2.333500in}}%
\pgfpathlineto{\pgfqpoint{6.357334in}{2.332317in}}%
\pgfpathlineto{\pgfqpoint{6.358408in}{2.333246in}}%
\pgfpathlineto{\pgfqpoint{6.361630in}{2.333428in}}%
\pgfpathlineto{\pgfqpoint{6.362704in}{2.332905in}}%
\pgfpathlineto{\pgfqpoint{6.363777in}{2.333351in}}%
\pgfpathlineto{\pgfqpoint{6.369147in}{2.332635in}}%
\pgfpathlineto{\pgfqpoint{6.373442in}{2.331057in}}%
\pgfpathlineto{\pgfqpoint{6.377737in}{2.329918in}}%
\pgfpathlineto{\pgfqpoint{6.378811in}{2.330077in}}%
\pgfpathlineto{\pgfqpoint{6.379885in}{2.331318in}}%
\pgfpathlineto{\pgfqpoint{6.380959in}{2.331015in}}%
\pgfpathlineto{\pgfqpoint{6.385254in}{2.331585in}}%
\pgfpathlineto{\pgfqpoint{6.386328in}{2.330446in}}%
\pgfpathlineto{\pgfqpoint{6.387402in}{2.330319in}}%
\pgfpathlineto{\pgfqpoint{6.388476in}{2.329358in}}%
\pgfpathlineto{\pgfqpoint{6.395993in}{2.329097in}}%
\pgfpathlineto{\pgfqpoint{6.401362in}{2.329127in}}%
\pgfpathlineto{\pgfqpoint{6.403510in}{2.329351in}}%
\pgfpathlineto{\pgfqpoint{6.403510in}{2.329351in}}%
\pgfusepath{stroke}%
\end{pgfscope}%
\begin{pgfscope}%
\pgfsetrectcap%
\pgfsetmiterjoin%
\pgfsetlinewidth{0.803000pt}%
\definecolor{currentstroke}{rgb}{1.000000,1.000000,1.000000}%
\pgfsetstrokecolor{currentstroke}%
\pgfsetdash{}{0pt}%
\pgfpathmoveto{\pgfqpoint{3.937600in}{2.309648in}}%
\pgfpathlineto{\pgfqpoint{3.937600in}{2.710533in}}%
\pgfusepath{stroke}%
\end{pgfscope}%
\begin{pgfscope}%
\pgfsetrectcap%
\pgfsetmiterjoin%
\pgfsetlinewidth{0.803000pt}%
\definecolor{currentstroke}{rgb}{1.000000,1.000000,1.000000}%
\pgfsetstrokecolor{currentstroke}%
\pgfsetdash{}{0pt}%
\pgfpathmoveto{\pgfqpoint{6.520934in}{2.309648in}}%
\pgfpathlineto{\pgfqpoint{6.520934in}{2.710533in}}%
\pgfusepath{stroke}%
\end{pgfscope}%
\begin{pgfscope}%
\pgfsetrectcap%
\pgfsetmiterjoin%
\pgfsetlinewidth{0.803000pt}%
\definecolor{currentstroke}{rgb}{1.000000,1.000000,1.000000}%
\pgfsetstrokecolor{currentstroke}%
\pgfsetdash{}{0pt}%
\pgfpathmoveto{\pgfqpoint{3.937600in}{2.309648in}}%
\pgfpathlineto{\pgfqpoint{6.520934in}{2.309648in}}%
\pgfusepath{stroke}%
\end{pgfscope}%
\begin{pgfscope}%
\pgfsetrectcap%
\pgfsetmiterjoin%
\pgfsetlinewidth{0.803000pt}%
\definecolor{currentstroke}{rgb}{1.000000,1.000000,1.000000}%
\pgfsetstrokecolor{currentstroke}%
\pgfsetdash{}{0pt}%
\pgfpathmoveto{\pgfqpoint{3.937600in}{2.710533in}}%
\pgfpathlineto{\pgfqpoint{6.520934in}{2.710533in}}%
\pgfusepath{stroke}%
\end{pgfscope}%
\begin{pgfscope}%
\definecolor{textcolor}{rgb}{0.150000,0.150000,0.150000}%
\pgfsetstrokecolor{textcolor}%
\pgfsetfillcolor{textcolor}%
\pgftext[x=5.229267in,y=2.793866in,,base]{\color{textcolor}\rmfamily\fontsize{16.800000}{20.160000}\selectfont PG}%
\end{pgfscope}%
\begin{pgfscope}%
\pgfsetbuttcap%
\pgfsetmiterjoin%
\definecolor{currentfill}{rgb}{0.917647,0.917647,0.949020}%
\pgfsetfillcolor{currentfill}%
\pgfsetlinewidth{0.000000pt}%
\definecolor{currentstroke}{rgb}{0.000000,0.000000,0.000000}%
\pgfsetstrokecolor{currentstroke}%
\pgfsetstrokeopacity{0.000000}%
\pgfsetdash{}{0pt}%
\pgfpathmoveto{\pgfqpoint{0.320934in}{1.347524in}}%
\pgfpathlineto{\pgfqpoint{2.904267in}{1.347524in}}%
\pgfpathlineto{\pgfqpoint{2.904267in}{1.748409in}}%
\pgfpathlineto{\pgfqpoint{0.320934in}{1.748409in}}%
\pgfpathclose%
\pgfusepath{fill}%
\end{pgfscope}%
\begin{pgfscope}%
\pgfpathrectangle{\pgfqpoint{0.320934in}{1.347524in}}{\pgfqpoint{2.583333in}{0.400885in}}%
\pgfusepath{clip}%
\pgfsetroundcap%
\pgfsetroundjoin%
\pgfsetlinewidth{0.803000pt}%
\definecolor{currentstroke}{rgb}{1.000000,1.000000,1.000000}%
\pgfsetstrokecolor{currentstroke}%
\pgfsetdash{}{0pt}%
\pgfpathmoveto{\pgfqpoint{0.436210in}{1.347524in}}%
\pgfpathlineto{\pgfqpoint{0.436210in}{1.748409in}}%
\pgfusepath{stroke}%
\end{pgfscope}%
\begin{pgfscope}%
\definecolor{textcolor}{rgb}{0.150000,0.150000,0.150000}%
\pgfsetstrokecolor{textcolor}%
\pgfsetfillcolor{textcolor}%
\pgftext[x=0.436210in,y=1.250302in,,top]{\color{textcolor}\rmfamily\fontsize{14.000000}{16.800000}\selectfont 2012}%
\end{pgfscope}%
\begin{pgfscope}%
\pgfpathrectangle{\pgfqpoint{0.320934in}{1.347524in}}{\pgfqpoint{2.583333in}{0.400885in}}%
\pgfusepath{clip}%
\pgfsetroundcap%
\pgfsetroundjoin%
\pgfsetlinewidth{0.803000pt}%
\definecolor{currentstroke}{rgb}{1.000000,1.000000,1.000000}%
\pgfsetstrokecolor{currentstroke}%
\pgfsetdash{}{0pt}%
\pgfpathmoveto{\pgfqpoint{0.829235in}{1.347524in}}%
\pgfpathlineto{\pgfqpoint{0.829235in}{1.748409in}}%
\pgfusepath{stroke}%
\end{pgfscope}%
\begin{pgfscope}%
\definecolor{textcolor}{rgb}{0.150000,0.150000,0.150000}%
\pgfsetstrokecolor{textcolor}%
\pgfsetfillcolor{textcolor}%
\pgftext[x=0.829235in,y=1.250302in,,top]{\color{textcolor}\rmfamily\fontsize{14.000000}{16.800000}\selectfont 2013}%
\end{pgfscope}%
\begin{pgfscope}%
\pgfpathrectangle{\pgfqpoint{0.320934in}{1.347524in}}{\pgfqpoint{2.583333in}{0.400885in}}%
\pgfusepath{clip}%
\pgfsetroundcap%
\pgfsetroundjoin%
\pgfsetlinewidth{0.803000pt}%
\definecolor{currentstroke}{rgb}{1.000000,1.000000,1.000000}%
\pgfsetstrokecolor{currentstroke}%
\pgfsetdash{}{0pt}%
\pgfpathmoveto{\pgfqpoint{1.221186in}{1.347524in}}%
\pgfpathlineto{\pgfqpoint{1.221186in}{1.748409in}}%
\pgfusepath{stroke}%
\end{pgfscope}%
\begin{pgfscope}%
\definecolor{textcolor}{rgb}{0.150000,0.150000,0.150000}%
\pgfsetstrokecolor{textcolor}%
\pgfsetfillcolor{textcolor}%
\pgftext[x=1.221186in,y=1.250302in,,top]{\color{textcolor}\rmfamily\fontsize{14.000000}{16.800000}\selectfont 2014}%
\end{pgfscope}%
\begin{pgfscope}%
\pgfpathrectangle{\pgfqpoint{0.320934in}{1.347524in}}{\pgfqpoint{2.583333in}{0.400885in}}%
\pgfusepath{clip}%
\pgfsetroundcap%
\pgfsetroundjoin%
\pgfsetlinewidth{0.803000pt}%
\definecolor{currentstroke}{rgb}{1.000000,1.000000,1.000000}%
\pgfsetstrokecolor{currentstroke}%
\pgfsetdash{}{0pt}%
\pgfpathmoveto{\pgfqpoint{1.613137in}{1.347524in}}%
\pgfpathlineto{\pgfqpoint{1.613137in}{1.748409in}}%
\pgfusepath{stroke}%
\end{pgfscope}%
\begin{pgfscope}%
\definecolor{textcolor}{rgb}{0.150000,0.150000,0.150000}%
\pgfsetstrokecolor{textcolor}%
\pgfsetfillcolor{textcolor}%
\pgftext[x=1.613137in,y=1.250302in,,top]{\color{textcolor}\rmfamily\fontsize{14.000000}{16.800000}\selectfont 2015}%
\end{pgfscope}%
\begin{pgfscope}%
\pgfpathrectangle{\pgfqpoint{0.320934in}{1.347524in}}{\pgfqpoint{2.583333in}{0.400885in}}%
\pgfusepath{clip}%
\pgfsetroundcap%
\pgfsetroundjoin%
\pgfsetlinewidth{0.803000pt}%
\definecolor{currentstroke}{rgb}{1.000000,1.000000,1.000000}%
\pgfsetstrokecolor{currentstroke}%
\pgfsetdash{}{0pt}%
\pgfpathmoveto{\pgfqpoint{2.005088in}{1.347524in}}%
\pgfpathlineto{\pgfqpoint{2.005088in}{1.748409in}}%
\pgfusepath{stroke}%
\end{pgfscope}%
\begin{pgfscope}%
\definecolor{textcolor}{rgb}{0.150000,0.150000,0.150000}%
\pgfsetstrokecolor{textcolor}%
\pgfsetfillcolor{textcolor}%
\pgftext[x=2.005088in,y=1.250302in,,top]{\color{textcolor}\rmfamily\fontsize{14.000000}{16.800000}\selectfont 2016}%
\end{pgfscope}%
\begin{pgfscope}%
\pgfpathrectangle{\pgfqpoint{0.320934in}{1.347524in}}{\pgfqpoint{2.583333in}{0.400885in}}%
\pgfusepath{clip}%
\pgfsetroundcap%
\pgfsetroundjoin%
\pgfsetlinewidth{0.803000pt}%
\definecolor{currentstroke}{rgb}{1.000000,1.000000,1.000000}%
\pgfsetstrokecolor{currentstroke}%
\pgfsetdash{}{0pt}%
\pgfpathmoveto{\pgfqpoint{2.398113in}{1.347524in}}%
\pgfpathlineto{\pgfqpoint{2.398113in}{1.748409in}}%
\pgfusepath{stroke}%
\end{pgfscope}%
\begin{pgfscope}%
\definecolor{textcolor}{rgb}{0.150000,0.150000,0.150000}%
\pgfsetstrokecolor{textcolor}%
\pgfsetfillcolor{textcolor}%
\pgftext[x=2.398113in,y=1.250302in,,top]{\color{textcolor}\rmfamily\fontsize{14.000000}{16.800000}\selectfont 2017}%
\end{pgfscope}%
\begin{pgfscope}%
\pgfpathrectangle{\pgfqpoint{0.320934in}{1.347524in}}{\pgfqpoint{2.583333in}{0.400885in}}%
\pgfusepath{clip}%
\pgfsetroundcap%
\pgfsetroundjoin%
\pgfsetlinewidth{0.803000pt}%
\definecolor{currentstroke}{rgb}{1.000000,1.000000,1.000000}%
\pgfsetstrokecolor{currentstroke}%
\pgfsetdash{}{0pt}%
\pgfpathmoveto{\pgfqpoint{2.790064in}{1.347524in}}%
\pgfpathlineto{\pgfqpoint{2.790064in}{1.748409in}}%
\pgfusepath{stroke}%
\end{pgfscope}%
\begin{pgfscope}%
\definecolor{textcolor}{rgb}{0.150000,0.150000,0.150000}%
\pgfsetstrokecolor{textcolor}%
\pgfsetfillcolor{textcolor}%
\pgftext[x=2.790064in,y=1.250302in,,top]{\color{textcolor}\rmfamily\fontsize{14.000000}{16.800000}\selectfont 2018}%
\end{pgfscope}%
\begin{pgfscope}%
\pgfpathrectangle{\pgfqpoint{0.320934in}{1.347524in}}{\pgfqpoint{2.583333in}{0.400885in}}%
\pgfusepath{clip}%
\pgfsetroundcap%
\pgfsetroundjoin%
\pgfsetlinewidth{0.803000pt}%
\definecolor{currentstroke}{rgb}{1.000000,1.000000,1.000000}%
\pgfsetstrokecolor{currentstroke}%
\pgfsetdash{}{0pt}%
\pgfpathmoveto{\pgfqpoint{0.320934in}{1.516269in}}%
\pgfpathlineto{\pgfqpoint{2.904267in}{1.516269in}}%
\pgfusepath{stroke}%
\end{pgfscope}%
\begin{pgfscope}%
\definecolor{textcolor}{rgb}{0.150000,0.150000,0.150000}%
\pgfsetstrokecolor{textcolor}%
\pgfsetfillcolor{textcolor}%
\pgftext[x=0.100000in,y=1.442402in,left,base]{\color{textcolor}\rmfamily\fontsize{14.000000}{16.800000}\selectfont 1}%
\end{pgfscope}%
\begin{pgfscope}%
\pgfpathrectangle{\pgfqpoint{0.320934in}{1.347524in}}{\pgfqpoint{2.583333in}{0.400885in}}%
\pgfusepath{clip}%
\pgfsetroundcap%
\pgfsetroundjoin%
\pgfsetlinewidth{0.803000pt}%
\definecolor{currentstroke}{rgb}{1.000000,1.000000,1.000000}%
\pgfsetstrokecolor{currentstroke}%
\pgfsetdash{}{0pt}%
\pgfpathmoveto{\pgfqpoint{0.320934in}{1.734716in}}%
\pgfpathlineto{\pgfqpoint{2.904267in}{1.734716in}}%
\pgfusepath{stroke}%
\end{pgfscope}%
\begin{pgfscope}%
\definecolor{textcolor}{rgb}{0.150000,0.150000,0.150000}%
\pgfsetstrokecolor{textcolor}%
\pgfsetfillcolor{textcolor}%
\pgftext[x=0.100000in,y=1.660850in,left,base]{\color{textcolor}\rmfamily\fontsize{14.000000}{16.800000}\selectfont 2}%
\end{pgfscope}%
\begin{pgfscope}%
\pgfpathrectangle{\pgfqpoint{0.320934in}{1.347524in}}{\pgfqpoint{2.583333in}{0.400885in}}%
\pgfusepath{clip}%
\pgfsetroundcap%
\pgfsetroundjoin%
\pgfsetlinewidth{1.505625pt}%
\definecolor{currentstroke}{rgb}{0.000000,0.000000,0.000000}%
\pgfsetstrokecolor{currentstroke}%
\pgfsetdash{}{0pt}%
\pgfpathmoveto{\pgfqpoint{0.438358in}{1.516269in}}%
\pgfpathlineto{\pgfqpoint{0.439432in}{1.517392in}}%
\pgfpathlineto{\pgfqpoint{0.441580in}{1.514022in}}%
\pgfpathlineto{\pgfqpoint{0.444801in}{1.514618in}}%
\pgfpathlineto{\pgfqpoint{0.445875in}{1.520271in}}%
\pgfpathlineto{\pgfqpoint{0.448023in}{1.523817in}}%
\pgfpathlineto{\pgfqpoint{0.449096in}{1.520411in}}%
\pgfpathlineto{\pgfqpoint{0.453392in}{1.523220in}}%
\pgfpathlineto{\pgfqpoint{0.454466in}{1.524870in}}%
\pgfpathlineto{\pgfqpoint{0.456613in}{1.522202in}}%
\pgfpathlineto{\pgfqpoint{0.459835in}{1.522694in}}%
\pgfpathlineto{\pgfqpoint{0.460909in}{1.525397in}}%
\pgfpathlineto{\pgfqpoint{0.463056in}{1.524309in}}%
\pgfpathlineto{\pgfqpoint{0.464130in}{1.524905in}}%
\pgfpathlineto{\pgfqpoint{0.467352in}{1.524870in}}%
\pgfpathlineto{\pgfqpoint{0.468425in}{1.527047in}}%
\pgfpathlineto{\pgfqpoint{0.469499in}{1.532524in}}%
\pgfpathlineto{\pgfqpoint{0.470573in}{1.531927in}}%
\pgfpathlineto{\pgfqpoint{0.471647in}{1.534947in}}%
\pgfpathlineto{\pgfqpoint{0.475942in}{1.532700in}}%
\pgfpathlineto{\pgfqpoint{0.478090in}{1.542951in}}%
\pgfpathlineto{\pgfqpoint{0.479164in}{1.542109in}}%
\pgfpathlineto{\pgfqpoint{0.482385in}{1.546146in}}%
\pgfpathlineto{\pgfqpoint{0.483459in}{1.545444in}}%
\pgfpathlineto{\pgfqpoint{0.484533in}{1.540880in}}%
\pgfpathlineto{\pgfqpoint{0.486681in}{1.544321in}}%
\pgfpathlineto{\pgfqpoint{0.492050in}{1.544847in}}%
\pgfpathlineto{\pgfqpoint{0.493124in}{1.543478in}}%
\pgfpathlineto{\pgfqpoint{0.494198in}{1.544882in}}%
\pgfpathlineto{\pgfqpoint{0.498493in}{1.543408in}}%
\pgfpathlineto{\pgfqpoint{0.501714in}{1.546568in}}%
\pgfpathlineto{\pgfqpoint{0.504936in}{1.542881in}}%
\pgfpathlineto{\pgfqpoint{0.506010in}{1.537299in}}%
\pgfpathlineto{\pgfqpoint{0.508157in}{1.543618in}}%
\pgfpathlineto{\pgfqpoint{0.509231in}{1.543654in}}%
\pgfpathlineto{\pgfqpoint{0.512453in}{1.544953in}}%
\pgfpathlineto{\pgfqpoint{0.513527in}{1.552536in}}%
\pgfpathlineto{\pgfqpoint{0.515674in}{1.553484in}}%
\pgfpathlineto{\pgfqpoint{0.516748in}{1.549341in}}%
\pgfpathlineto{\pgfqpoint{0.519970in}{1.546813in}}%
\pgfpathlineto{\pgfqpoint{0.521044in}{1.542811in}}%
\pgfpathlineto{\pgfqpoint{0.524265in}{1.538528in}}%
\pgfpathlineto{\pgfqpoint{0.527487in}{1.543513in}}%
\pgfpathlineto{\pgfqpoint{0.528560in}{1.542354in}}%
\pgfpathlineto{\pgfqpoint{0.529634in}{1.537685in}}%
\pgfpathlineto{\pgfqpoint{0.531782in}{1.541863in}}%
\pgfpathlineto{\pgfqpoint{0.535003in}{1.541231in}}%
\pgfpathlineto{\pgfqpoint{0.537151in}{1.538879in}}%
\pgfpathlineto{\pgfqpoint{0.542520in}{1.533893in}}%
\pgfpathlineto{\pgfqpoint{0.543594in}{1.528732in}}%
\pgfpathlineto{\pgfqpoint{0.545742in}{1.536667in}}%
\pgfpathlineto{\pgfqpoint{0.546816in}{1.532629in}}%
\pgfpathlineto{\pgfqpoint{0.550037in}{1.532875in}}%
\pgfpathlineto{\pgfqpoint{0.551111in}{1.536878in}}%
\pgfpathlineto{\pgfqpoint{0.552185in}{1.536737in}}%
\pgfpathlineto{\pgfqpoint{0.553259in}{1.534385in}}%
\pgfpathlineto{\pgfqpoint{0.554333in}{1.536140in}}%
\pgfpathlineto{\pgfqpoint{0.557554in}{1.532489in}}%
\pgfpathlineto{\pgfqpoint{0.559702in}{1.532664in}}%
\pgfpathlineto{\pgfqpoint{0.561849in}{1.539089in}}%
\pgfpathlineto{\pgfqpoint{0.567219in}{1.537790in}}%
\pgfpathlineto{\pgfqpoint{0.568292in}{1.535894in}}%
\pgfpathlineto{\pgfqpoint{0.569366in}{1.531681in}}%
\pgfpathlineto{\pgfqpoint{0.573662in}{1.529750in}}%
\pgfpathlineto{\pgfqpoint{0.574735in}{1.524414in}}%
\pgfpathlineto{\pgfqpoint{0.575809in}{1.525256in}}%
\pgfpathlineto{\pgfqpoint{0.576883in}{1.524905in}}%
\pgfpathlineto{\pgfqpoint{0.580105in}{1.521289in}}%
\pgfpathlineto{\pgfqpoint{0.581179in}{1.521956in}}%
\pgfpathlineto{\pgfqpoint{0.584400in}{1.512161in}}%
\pgfpathlineto{\pgfqpoint{0.588695in}{1.516760in}}%
\pgfpathlineto{\pgfqpoint{0.589769in}{1.517111in}}%
\pgfpathlineto{\pgfqpoint{0.591917in}{1.514022in}}%
\pgfpathlineto{\pgfqpoint{0.596212in}{1.520166in}}%
\pgfpathlineto{\pgfqpoint{0.597286in}{1.516163in}}%
\pgfpathlineto{\pgfqpoint{0.598360in}{1.517252in}}%
\pgfpathlineto{\pgfqpoint{0.599434in}{1.511072in}}%
\pgfpathlineto{\pgfqpoint{0.602655in}{1.509387in}}%
\pgfpathlineto{\pgfqpoint{0.603729in}{1.507702in}}%
\pgfpathlineto{\pgfqpoint{0.605877in}{1.521078in}}%
\pgfpathlineto{\pgfqpoint{0.606951in}{1.521394in}}%
\pgfpathlineto{\pgfqpoint{0.611246in}{1.517989in}}%
\pgfpathlineto{\pgfqpoint{0.612320in}{1.515566in}}%
\pgfpathlineto{\pgfqpoint{0.614468in}{1.518375in}}%
\pgfpathlineto{\pgfqpoint{0.617689in}{1.519885in}}%
\pgfpathlineto{\pgfqpoint{0.618763in}{1.524063in}}%
\pgfpathlineto{\pgfqpoint{0.619837in}{1.522588in}}%
\pgfpathlineto{\pgfqpoint{0.620911in}{1.519358in}}%
\pgfpathlineto{\pgfqpoint{0.621984in}{1.520482in}}%
\pgfpathlineto{\pgfqpoint{0.625206in}{1.516198in}}%
\pgfpathlineto{\pgfqpoint{0.626280in}{1.515847in}}%
\pgfpathlineto{\pgfqpoint{0.627354in}{1.517427in}}%
\pgfpathlineto{\pgfqpoint{0.628427in}{1.512793in}}%
\pgfpathlineto{\pgfqpoint{0.629501in}{1.521465in}}%
\pgfpathlineto{\pgfqpoint{0.632723in}{1.520025in}}%
\pgfpathlineto{\pgfqpoint{0.633797in}{1.522132in}}%
\pgfpathlineto{\pgfqpoint{0.635944in}{1.521043in}}%
\pgfpathlineto{\pgfqpoint{0.637018in}{1.517217in}}%
\pgfpathlineto{\pgfqpoint{0.640240in}{1.517919in}}%
\pgfpathlineto{\pgfqpoint{0.641313in}{1.517392in}}%
\pgfpathlineto{\pgfqpoint{0.642387in}{1.512512in}}%
\pgfpathlineto{\pgfqpoint{0.643461in}{1.510546in}}%
\pgfpathlineto{\pgfqpoint{0.644535in}{1.515742in}}%
\pgfpathlineto{\pgfqpoint{0.647757in}{1.514618in}}%
\pgfpathlineto{\pgfqpoint{0.648830in}{1.515742in}}%
\pgfpathlineto{\pgfqpoint{0.650978in}{1.522342in}}%
\pgfpathlineto{\pgfqpoint{0.652052in}{1.517638in}}%
\pgfpathlineto{\pgfqpoint{0.655273in}{1.514794in}}%
\pgfpathlineto{\pgfqpoint{0.656347in}{1.510827in}}%
\pgfpathlineto{\pgfqpoint{0.659569in}{1.517778in}}%
\pgfpathlineto{\pgfqpoint{0.662790in}{1.519815in}}%
\pgfpathlineto{\pgfqpoint{0.663864in}{1.518235in}}%
\pgfpathlineto{\pgfqpoint{0.664938in}{1.519288in}}%
\pgfpathlineto{\pgfqpoint{0.666012in}{1.518235in}}%
\pgfpathlineto{\pgfqpoint{0.667086in}{1.525256in}}%
\pgfpathlineto{\pgfqpoint{0.670307in}{1.524800in}}%
\pgfpathlineto{\pgfqpoint{0.671381in}{1.529154in}}%
\pgfpathlineto{\pgfqpoint{0.673529in}{1.525924in}}%
\pgfpathlineto{\pgfqpoint{0.674602in}{1.528451in}}%
\pgfpathlineto{\pgfqpoint{0.677824in}{1.527223in}}%
\pgfpathlineto{\pgfqpoint{0.678898in}{1.528135in}}%
\pgfpathlineto{\pgfqpoint{0.681045in}{1.532664in}}%
\pgfpathlineto{\pgfqpoint{0.682119in}{1.537439in}}%
\pgfpathlineto{\pgfqpoint{0.685341in}{1.536597in}}%
\pgfpathlineto{\pgfqpoint{0.686415in}{1.534174in}}%
\pgfpathlineto{\pgfqpoint{0.687489in}{1.535333in}}%
\pgfpathlineto{\pgfqpoint{0.688562in}{1.533963in}}%
\pgfpathlineto{\pgfqpoint{0.689636in}{1.536597in}}%
\pgfpathlineto{\pgfqpoint{0.693932in}{1.538247in}}%
\pgfpathlineto{\pgfqpoint{0.695005in}{1.536842in}}%
\pgfpathlineto{\pgfqpoint{0.696079in}{1.533542in}}%
\pgfpathlineto{\pgfqpoint{0.697153in}{1.535894in}}%
\pgfpathlineto{\pgfqpoint{0.702522in}{1.530488in}}%
\pgfpathlineto{\pgfqpoint{0.703596in}{1.534560in}}%
\pgfpathlineto{\pgfqpoint{0.704670in}{1.534595in}}%
\pgfpathlineto{\pgfqpoint{0.707891in}{1.532208in}}%
\pgfpathlineto{\pgfqpoint{0.710039in}{1.532875in}}%
\pgfpathlineto{\pgfqpoint{0.712187in}{1.543654in}}%
\pgfpathlineto{\pgfqpoint{0.715408in}{1.542776in}}%
\pgfpathlineto{\pgfqpoint{0.716482in}{1.540775in}}%
\pgfpathlineto{\pgfqpoint{0.717556in}{1.541477in}}%
\pgfpathlineto{\pgfqpoint{0.718630in}{1.539089in}}%
\pgfpathlineto{\pgfqpoint{0.719704in}{1.538598in}}%
\pgfpathlineto{\pgfqpoint{0.722925in}{1.536421in}}%
\pgfpathlineto{\pgfqpoint{0.723999in}{1.532664in}}%
\pgfpathlineto{\pgfqpoint{0.726147in}{1.531436in}}%
\pgfpathlineto{\pgfqpoint{0.727221in}{1.531260in}}%
\pgfpathlineto{\pgfqpoint{0.737959in}{1.531471in}}%
\pgfpathlineto{\pgfqpoint{0.740107in}{1.524484in}}%
\pgfpathlineto{\pgfqpoint{0.745476in}{1.524870in}}%
\pgfpathlineto{\pgfqpoint{0.747623in}{1.531997in}}%
\pgfpathlineto{\pgfqpoint{0.748697in}{1.534069in}}%
\pgfpathlineto{\pgfqpoint{0.749771in}{1.530347in}}%
\pgfpathlineto{\pgfqpoint{0.752993in}{1.529891in}}%
\pgfpathlineto{\pgfqpoint{0.754067in}{1.527609in}}%
\pgfpathlineto{\pgfqpoint{0.755140in}{1.530101in}}%
\pgfpathlineto{\pgfqpoint{0.756214in}{1.528276in}}%
\pgfpathlineto{\pgfqpoint{0.757288in}{1.530979in}}%
\pgfpathlineto{\pgfqpoint{0.762657in}{1.530874in}}%
\pgfpathlineto{\pgfqpoint{0.763731in}{1.533542in}}%
\pgfpathlineto{\pgfqpoint{0.764805in}{1.530593in}}%
\pgfpathlineto{\pgfqpoint{0.768026in}{1.530101in}}%
\pgfpathlineto{\pgfqpoint{0.769100in}{1.536246in}}%
\pgfpathlineto{\pgfqpoint{0.771248in}{1.524905in}}%
\pgfpathlineto{\pgfqpoint{0.772322in}{1.523957in}}%
\pgfpathlineto{\pgfqpoint{0.775543in}{1.527258in}}%
\pgfpathlineto{\pgfqpoint{0.776617in}{1.527398in}}%
\pgfpathlineto{\pgfqpoint{0.777691in}{1.521956in}}%
\pgfpathlineto{\pgfqpoint{0.778765in}{1.522518in}}%
\pgfpathlineto{\pgfqpoint{0.785208in}{1.529996in}}%
\pgfpathlineto{\pgfqpoint{0.787356in}{1.533858in}}%
\pgfpathlineto{\pgfqpoint{0.791651in}{1.534490in}}%
\pgfpathlineto{\pgfqpoint{0.792725in}{1.537404in}}%
\pgfpathlineto{\pgfqpoint{0.793799in}{1.537299in}}%
\pgfpathlineto{\pgfqpoint{0.794872in}{1.538352in}}%
\pgfpathlineto{\pgfqpoint{0.798094in}{1.537404in}}%
\pgfpathlineto{\pgfqpoint{0.802389in}{1.540985in}}%
\pgfpathlineto{\pgfqpoint{0.807758in}{1.541020in}}%
\pgfpathlineto{\pgfqpoint{0.809906in}{1.537966in}}%
\pgfpathlineto{\pgfqpoint{0.813128in}{1.538001in}}%
\pgfpathlineto{\pgfqpoint{0.814201in}{1.544707in}}%
\pgfpathlineto{\pgfqpoint{0.815275in}{1.547129in}}%
\pgfpathlineto{\pgfqpoint{0.816349in}{1.547551in}}%
\pgfpathlineto{\pgfqpoint{0.817423in}{1.545655in}}%
\pgfpathlineto{\pgfqpoint{0.823866in}{1.544215in}}%
\pgfpathlineto{\pgfqpoint{0.824940in}{1.540459in}}%
\pgfpathlineto{\pgfqpoint{0.828161in}{1.544040in}}%
\pgfpathlineto{\pgfqpoint{0.830309in}{1.550043in}}%
\pgfpathlineto{\pgfqpoint{0.831383in}{1.550956in}}%
\pgfpathlineto{\pgfqpoint{0.832457in}{1.552957in}}%
\pgfpathlineto{\pgfqpoint{0.835678in}{1.551729in}}%
\pgfpathlineto{\pgfqpoint{0.836752in}{1.548674in}}%
\pgfpathlineto{\pgfqpoint{0.837826in}{1.551693in}}%
\pgfpathlineto{\pgfqpoint{0.839974in}{1.553554in}}%
\pgfpathlineto{\pgfqpoint{0.844269in}{1.555907in}}%
\pgfpathlineto{\pgfqpoint{0.845343in}{1.554748in}}%
\pgfpathlineto{\pgfqpoint{0.847490in}{1.558891in}}%
\pgfpathlineto{\pgfqpoint{0.851786in}{1.560436in}}%
\pgfpathlineto{\pgfqpoint{0.853934in}{1.564789in}}%
\pgfpathlineto{\pgfqpoint{0.855007in}{1.567422in}}%
\pgfpathlineto{\pgfqpoint{0.859303in}{1.567317in}}%
\pgfpathlineto{\pgfqpoint{0.860377in}{1.565351in}}%
\pgfpathlineto{\pgfqpoint{0.861450in}{1.560752in}}%
\pgfpathlineto{\pgfqpoint{0.862524in}{1.567563in}}%
\pgfpathlineto{\pgfqpoint{0.867893in}{1.566158in}}%
\pgfpathlineto{\pgfqpoint{0.868967in}{1.567879in}}%
\pgfpathlineto{\pgfqpoint{0.870041in}{1.568300in}}%
\pgfpathlineto{\pgfqpoint{0.873263in}{1.566861in}}%
\pgfpathlineto{\pgfqpoint{0.874336in}{1.567984in}}%
\pgfpathlineto{\pgfqpoint{0.876484in}{1.568721in}}%
\pgfpathlineto{\pgfqpoint{0.877558in}{1.572022in}}%
\pgfpathlineto{\pgfqpoint{0.881853in}{1.572724in}}%
\pgfpathlineto{\pgfqpoint{0.884001in}{1.567879in}}%
\pgfpathlineto{\pgfqpoint{0.885075in}{1.571144in}}%
\pgfpathlineto{\pgfqpoint{0.888296in}{1.564719in}}%
\pgfpathlineto{\pgfqpoint{0.889370in}{1.567247in}}%
\pgfpathlineto{\pgfqpoint{0.890444in}{1.571390in}}%
\pgfpathlineto{\pgfqpoint{0.891518in}{1.571319in}}%
\pgfpathlineto{\pgfqpoint{0.895813in}{1.567036in}}%
\pgfpathlineto{\pgfqpoint{0.896887in}{1.572724in}}%
\pgfpathlineto{\pgfqpoint{0.897961in}{1.572899in}}%
\pgfpathlineto{\pgfqpoint{0.900109in}{1.575919in}}%
\pgfpathlineto{\pgfqpoint{0.904404in}{1.579149in}}%
\pgfpathlineto{\pgfqpoint{0.905478in}{1.578973in}}%
\pgfpathlineto{\pgfqpoint{0.906552in}{1.580061in}}%
\pgfpathlineto{\pgfqpoint{0.910847in}{1.578025in}}%
\pgfpathlineto{\pgfqpoint{0.912995in}{1.580061in}}%
\pgfpathlineto{\pgfqpoint{0.914068in}{1.577007in}}%
\pgfpathlineto{\pgfqpoint{0.915142in}{1.580518in}}%
\pgfpathlineto{\pgfqpoint{0.919438in}{1.577674in}}%
\pgfpathlineto{\pgfqpoint{0.920511in}{1.577534in}}%
\pgfpathlineto{\pgfqpoint{0.921585in}{1.580026in}}%
\pgfpathlineto{\pgfqpoint{0.925881in}{1.578482in}}%
\pgfpathlineto{\pgfqpoint{0.929102in}{1.579219in}}%
\pgfpathlineto{\pgfqpoint{0.930176in}{1.577639in}}%
\pgfpathlineto{\pgfqpoint{0.933398in}{1.580869in}}%
\pgfpathlineto{\pgfqpoint{0.936619in}{1.587224in}}%
\pgfpathlineto{\pgfqpoint{0.937693in}{1.586837in}}%
\pgfpathlineto{\pgfqpoint{0.940914in}{1.580272in}}%
\pgfpathlineto{\pgfqpoint{0.941988in}{1.583291in}}%
\pgfpathlineto{\pgfqpoint{0.944136in}{1.574339in}}%
\pgfpathlineto{\pgfqpoint{0.945210in}{1.579394in}}%
\pgfpathlineto{\pgfqpoint{0.948431in}{1.580623in}}%
\pgfpathlineto{\pgfqpoint{0.950579in}{1.575708in}}%
\pgfpathlineto{\pgfqpoint{0.951653in}{1.576024in}}%
\pgfpathlineto{\pgfqpoint{0.952727in}{1.573145in}}%
\pgfpathlineto{\pgfqpoint{0.955948in}{1.574549in}}%
\pgfpathlineto{\pgfqpoint{0.958096in}{1.572829in}}%
\pgfpathlineto{\pgfqpoint{0.959170in}{1.574936in}}%
\pgfpathlineto{\pgfqpoint{0.960244in}{1.579043in}}%
\pgfpathlineto{\pgfqpoint{0.963465in}{1.580202in}}%
\pgfpathlineto{\pgfqpoint{0.967760in}{1.585503in}}%
\pgfpathlineto{\pgfqpoint{0.970982in}{1.584626in}}%
\pgfpathlineto{\pgfqpoint{0.972056in}{1.587575in}}%
\pgfpathlineto{\pgfqpoint{0.973130in}{1.588839in}}%
\pgfpathlineto{\pgfqpoint{0.974203in}{1.586873in}}%
\pgfpathlineto{\pgfqpoint{0.975277in}{1.593508in}}%
\pgfpathlineto{\pgfqpoint{0.978499in}{1.593157in}}%
\pgfpathlineto{\pgfqpoint{0.979573in}{1.594105in}}%
\pgfpathlineto{\pgfqpoint{0.981720in}{1.587856in}}%
\pgfpathlineto{\pgfqpoint{0.982794in}{1.586486in}}%
\pgfpathlineto{\pgfqpoint{0.987089in}{1.589330in}}%
\pgfpathlineto{\pgfqpoint{0.988163in}{1.586732in}}%
\pgfpathlineto{\pgfqpoint{0.989237in}{1.589295in}}%
\pgfpathlineto{\pgfqpoint{0.990311in}{1.586065in}}%
\pgfpathlineto{\pgfqpoint{0.993533in}{1.587083in}}%
\pgfpathlineto{\pgfqpoint{0.996754in}{1.579535in}}%
\pgfpathlineto{\pgfqpoint{0.997828in}{1.584836in}}%
\pgfpathlineto{\pgfqpoint{1.001049in}{1.583748in}}%
\pgfpathlineto{\pgfqpoint{1.002123in}{1.582344in}}%
\pgfpathlineto{\pgfqpoint{1.003197in}{1.579324in}}%
\pgfpathlineto{\pgfqpoint{1.004271in}{1.584204in}}%
\pgfpathlineto{\pgfqpoint{1.005345in}{1.583397in}}%
\pgfpathlineto{\pgfqpoint{1.008566in}{1.586311in}}%
\pgfpathlineto{\pgfqpoint{1.009640in}{1.589927in}}%
\pgfpathlineto{\pgfqpoint{1.011788in}{1.578306in}}%
\pgfpathlineto{\pgfqpoint{1.012862in}{1.577814in}}%
\pgfpathlineto{\pgfqpoint{1.016083in}{1.575708in}}%
\pgfpathlineto{\pgfqpoint{1.017157in}{1.576586in}}%
\pgfpathlineto{\pgfqpoint{1.018231in}{1.580237in}}%
\pgfpathlineto{\pgfqpoint{1.019305in}{1.581852in}}%
\pgfpathlineto{\pgfqpoint{1.020378in}{1.580097in}}%
\pgfpathlineto{\pgfqpoint{1.023600in}{1.585574in}}%
\pgfpathlineto{\pgfqpoint{1.024674in}{1.582730in}}%
\pgfpathlineto{\pgfqpoint{1.027895in}{1.591015in}}%
\pgfpathlineto{\pgfqpoint{1.031117in}{1.592385in}}%
\pgfpathlineto{\pgfqpoint{1.032191in}{1.595544in}}%
\pgfpathlineto{\pgfqpoint{1.033265in}{1.594842in}}%
\pgfpathlineto{\pgfqpoint{1.034338in}{1.600600in}}%
\pgfpathlineto{\pgfqpoint{1.038634in}{1.602285in}}%
\pgfpathlineto{\pgfqpoint{1.039708in}{1.601513in}}%
\pgfpathlineto{\pgfqpoint{1.042929in}{1.609097in}}%
\pgfpathlineto{\pgfqpoint{1.046151in}{1.607973in}}%
\pgfpathlineto{\pgfqpoint{1.047224in}{1.617101in}}%
\pgfpathlineto{\pgfqpoint{1.049372in}{1.616118in}}%
\pgfpathlineto{\pgfqpoint{1.050446in}{1.616645in}}%
\pgfpathlineto{\pgfqpoint{1.053667in}{1.616961in}}%
\pgfpathlineto{\pgfqpoint{1.054741in}{1.618471in}}%
\pgfpathlineto{\pgfqpoint{1.055815in}{1.618471in}}%
\pgfpathlineto{\pgfqpoint{1.056889in}{1.623351in}}%
\pgfpathlineto{\pgfqpoint{1.057963in}{1.625141in}}%
\pgfpathlineto{\pgfqpoint{1.061184in}{1.621736in}}%
\pgfpathlineto{\pgfqpoint{1.062258in}{1.617347in}}%
\pgfpathlineto{\pgfqpoint{1.063332in}{1.619875in}}%
\pgfpathlineto{\pgfqpoint{1.064406in}{1.620577in}}%
\pgfpathlineto{\pgfqpoint{1.065480in}{1.618611in}}%
\pgfpathlineto{\pgfqpoint{1.068701in}{1.618365in}}%
\pgfpathlineto{\pgfqpoint{1.069775in}{1.622192in}}%
\pgfpathlineto{\pgfqpoint{1.070849in}{1.618541in}}%
\pgfpathlineto{\pgfqpoint{1.071923in}{1.612221in}}%
\pgfpathlineto{\pgfqpoint{1.072997in}{1.612502in}}%
\pgfpathlineto{\pgfqpoint{1.077292in}{1.610396in}}%
\pgfpathlineto{\pgfqpoint{1.078366in}{1.608289in}}%
\pgfpathlineto{\pgfqpoint{1.079440in}{1.612116in}}%
\pgfpathlineto{\pgfqpoint{1.083735in}{1.610115in}}%
\pgfpathlineto{\pgfqpoint{1.084809in}{1.602917in}}%
\pgfpathlineto{\pgfqpoint{1.085883in}{1.603023in}}%
\pgfpathlineto{\pgfqpoint{1.086956in}{1.604497in}}%
\pgfpathlineto{\pgfqpoint{1.088030in}{1.603374in}}%
\pgfpathlineto{\pgfqpoint{1.094473in}{1.614328in}}%
\pgfpathlineto{\pgfqpoint{1.095547in}{1.612958in}}%
\pgfpathlineto{\pgfqpoint{1.098769in}{1.616750in}}%
\pgfpathlineto{\pgfqpoint{1.100916in}{1.627002in}}%
\pgfpathlineto{\pgfqpoint{1.101990in}{1.627002in}}%
\pgfpathlineto{\pgfqpoint{1.103064in}{1.628687in}}%
\pgfpathlineto{\pgfqpoint{1.107359in}{1.634796in}}%
\pgfpathlineto{\pgfqpoint{1.109507in}{1.639711in}}%
\pgfpathlineto{\pgfqpoint{1.110581in}{1.632339in}}%
\pgfpathlineto{\pgfqpoint{1.113802in}{1.631847in}}%
\pgfpathlineto{\pgfqpoint{1.114876in}{1.633603in}}%
\pgfpathlineto{\pgfqpoint{1.115950in}{1.631356in}}%
\pgfpathlineto{\pgfqpoint{1.117024in}{1.632584in}}%
\pgfpathlineto{\pgfqpoint{1.118098in}{1.631672in}}%
\pgfpathlineto{\pgfqpoint{1.122393in}{1.625598in}}%
\pgfpathlineto{\pgfqpoint{1.124541in}{1.614363in}}%
\pgfpathlineto{\pgfqpoint{1.125615in}{1.616118in}}%
\pgfpathlineto{\pgfqpoint{1.128836in}{1.615381in}}%
\pgfpathlineto{\pgfqpoint{1.129910in}{1.611519in}}%
\pgfpathlineto{\pgfqpoint{1.130984in}{1.611765in}}%
\pgfpathlineto{\pgfqpoint{1.132058in}{1.621139in}}%
\pgfpathlineto{\pgfqpoint{1.133132in}{1.624474in}}%
\pgfpathlineto{\pgfqpoint{1.136353in}{1.624158in}}%
\pgfpathlineto{\pgfqpoint{1.137427in}{1.620788in}}%
\pgfpathlineto{\pgfqpoint{1.138501in}{1.622719in}}%
\pgfpathlineto{\pgfqpoint{1.139575in}{1.627564in}}%
\pgfpathlineto{\pgfqpoint{1.140648in}{1.626721in}}%
\pgfpathlineto{\pgfqpoint{1.143870in}{1.626335in}}%
\pgfpathlineto{\pgfqpoint{1.144944in}{1.621806in}}%
\pgfpathlineto{\pgfqpoint{1.146018in}{1.622578in}}%
\pgfpathlineto{\pgfqpoint{1.148165in}{1.626054in}}%
\pgfpathlineto{\pgfqpoint{1.151387in}{1.621385in}}%
\pgfpathlineto{\pgfqpoint{1.152461in}{1.622684in}}%
\pgfpathlineto{\pgfqpoint{1.153534in}{1.621244in}}%
\pgfpathlineto{\pgfqpoint{1.154608in}{1.622157in}}%
\pgfpathlineto{\pgfqpoint{1.155682in}{1.626019in}}%
\pgfpathlineto{\pgfqpoint{1.158904in}{1.627458in}}%
\pgfpathlineto{\pgfqpoint{1.159977in}{1.626546in}}%
\pgfpathlineto{\pgfqpoint{1.161051in}{1.629565in}}%
\pgfpathlineto{\pgfqpoint{1.162125in}{1.625282in}}%
\pgfpathlineto{\pgfqpoint{1.163199in}{1.629179in}}%
\pgfpathlineto{\pgfqpoint{1.166421in}{1.627739in}}%
\pgfpathlineto{\pgfqpoint{1.167494in}{1.625843in}}%
\pgfpathlineto{\pgfqpoint{1.168568in}{1.627669in}}%
\pgfpathlineto{\pgfqpoint{1.169642in}{1.631461in}}%
\pgfpathlineto{\pgfqpoint{1.170716in}{1.631145in}}%
\pgfpathlineto{\pgfqpoint{1.173937in}{1.632795in}}%
\pgfpathlineto{\pgfqpoint{1.175011in}{1.632690in}}%
\pgfpathlineto{\pgfqpoint{1.176085in}{1.631882in}}%
\pgfpathlineto{\pgfqpoint{1.178233in}{1.636095in}}%
\pgfpathlineto{\pgfqpoint{1.181454in}{1.636411in}}%
\pgfpathlineto{\pgfqpoint{1.183602in}{1.640098in}}%
\pgfpathlineto{\pgfqpoint{1.185750in}{1.638096in}}%
\pgfpathlineto{\pgfqpoint{1.188971in}{1.636411in}}%
\pgfpathlineto{\pgfqpoint{1.191119in}{1.632198in}}%
\pgfpathlineto{\pgfqpoint{1.192193in}{1.632549in}}%
\pgfpathlineto{\pgfqpoint{1.193266in}{1.638904in}}%
\pgfpathlineto{\pgfqpoint{1.196488in}{1.639079in}}%
\pgfpathlineto{\pgfqpoint{1.197562in}{1.638483in}}%
\pgfpathlineto{\pgfqpoint{1.198636in}{1.631356in}}%
\pgfpathlineto{\pgfqpoint{1.200783in}{1.627318in}}%
\pgfpathlineto{\pgfqpoint{1.204005in}{1.630969in}}%
\pgfpathlineto{\pgfqpoint{1.205079in}{1.628161in}}%
\pgfpathlineto{\pgfqpoint{1.206153in}{1.634866in}}%
\pgfpathlineto{\pgfqpoint{1.207226in}{1.633918in}}%
\pgfpathlineto{\pgfqpoint{1.208300in}{1.637500in}}%
\pgfpathlineto{\pgfqpoint{1.211522in}{1.637991in}}%
\pgfpathlineto{\pgfqpoint{1.214743in}{1.643714in}}%
\pgfpathlineto{\pgfqpoint{1.215817in}{1.644065in}}%
\pgfpathlineto{\pgfqpoint{1.219039in}{1.643854in}}%
\pgfpathlineto{\pgfqpoint{1.220112in}{1.647119in}}%
\pgfpathlineto{\pgfqpoint{1.222260in}{1.643117in}}%
\pgfpathlineto{\pgfqpoint{1.223334in}{1.644381in}}%
\pgfpathlineto{\pgfqpoint{1.226555in}{1.644030in}}%
\pgfpathlineto{\pgfqpoint{1.227629in}{1.646242in}}%
\pgfpathlineto{\pgfqpoint{1.230851in}{1.647225in}}%
\pgfpathlineto{\pgfqpoint{1.235146in}{1.643538in}}%
\pgfpathlineto{\pgfqpoint{1.236220in}{1.647962in}}%
\pgfpathlineto{\pgfqpoint{1.242663in}{1.650771in}}%
\pgfpathlineto{\pgfqpoint{1.243737in}{1.654247in}}%
\pgfpathlineto{\pgfqpoint{1.244811in}{1.650314in}}%
\pgfpathlineto{\pgfqpoint{1.245885in}{1.640975in}}%
\pgfpathlineto{\pgfqpoint{1.249106in}{1.647084in}}%
\pgfpathlineto{\pgfqpoint{1.250180in}{1.647330in}}%
\pgfpathlineto{\pgfqpoint{1.251254in}{1.645504in}}%
\pgfpathlineto{\pgfqpoint{1.252328in}{1.649823in}}%
\pgfpathlineto{\pgfqpoint{1.253401in}{1.647822in}}%
\pgfpathlineto{\pgfqpoint{1.258771in}{1.629038in}}%
\pgfpathlineto{\pgfqpoint{1.260918in}{1.637710in}}%
\pgfpathlineto{\pgfqpoint{1.264140in}{1.640379in}}%
\pgfpathlineto{\pgfqpoint{1.265214in}{1.644556in}}%
\pgfpathlineto{\pgfqpoint{1.268435in}{1.649191in}}%
\pgfpathlineto{\pgfqpoint{1.272731in}{1.648770in}}%
\pgfpathlineto{\pgfqpoint{1.273804in}{1.649717in}}%
\pgfpathlineto{\pgfqpoint{1.274878in}{1.652807in}}%
\pgfpathlineto{\pgfqpoint{1.279174in}{1.657266in}}%
\pgfpathlineto{\pgfqpoint{1.280247in}{1.655405in}}%
\pgfpathlineto{\pgfqpoint{1.281321in}{1.656107in}}%
\pgfpathlineto{\pgfqpoint{1.283469in}{1.658916in}}%
\pgfpathlineto{\pgfqpoint{1.286690in}{1.658109in}}%
\pgfpathlineto{\pgfqpoint{1.287764in}{1.660671in}}%
\pgfpathlineto{\pgfqpoint{1.288838in}{1.660110in}}%
\pgfpathlineto{\pgfqpoint{1.290986in}{1.662883in}}%
\pgfpathlineto{\pgfqpoint{1.294207in}{1.661198in}}%
\pgfpathlineto{\pgfqpoint{1.295281in}{1.654633in}}%
\pgfpathlineto{\pgfqpoint{1.296355in}{1.655159in}}%
\pgfpathlineto{\pgfqpoint{1.297429in}{1.646171in}}%
\pgfpathlineto{\pgfqpoint{1.298503in}{1.645259in}}%
\pgfpathlineto{\pgfqpoint{1.302798in}{1.651684in}}%
\pgfpathlineto{\pgfqpoint{1.303872in}{1.649612in}}%
\pgfpathlineto{\pgfqpoint{1.304946in}{1.648875in}}%
\pgfpathlineto{\pgfqpoint{1.306020in}{1.651333in}}%
\pgfpathlineto{\pgfqpoint{1.309241in}{1.648805in}}%
\pgfpathlineto{\pgfqpoint{1.310315in}{1.653299in}}%
\pgfpathlineto{\pgfqpoint{1.312463in}{1.648980in}}%
\pgfpathlineto{\pgfqpoint{1.313536in}{1.652070in}}%
\pgfpathlineto{\pgfqpoint{1.316758in}{1.658354in}}%
\pgfpathlineto{\pgfqpoint{1.317832in}{1.661690in}}%
\pgfpathlineto{\pgfqpoint{1.318906in}{1.667693in}}%
\pgfpathlineto{\pgfqpoint{1.319979in}{1.667377in}}%
\pgfpathlineto{\pgfqpoint{1.321053in}{1.662462in}}%
\pgfpathlineto{\pgfqpoint{1.325349in}{1.655054in}}%
\pgfpathlineto{\pgfqpoint{1.326422in}{1.658951in}}%
\pgfpathlineto{\pgfqpoint{1.327496in}{1.651297in}}%
\pgfpathlineto{\pgfqpoint{1.328570in}{1.649366in}}%
\pgfpathlineto{\pgfqpoint{1.331792in}{1.652491in}}%
\pgfpathlineto{\pgfqpoint{1.332865in}{1.655265in}}%
\pgfpathlineto{\pgfqpoint{1.333939in}{1.662146in}}%
\pgfpathlineto{\pgfqpoint{1.335013in}{1.663691in}}%
\pgfpathlineto{\pgfqpoint{1.339309in}{1.662848in}}%
\pgfpathlineto{\pgfqpoint{1.341456in}{1.667026in}}%
\pgfpathlineto{\pgfqpoint{1.342530in}{1.664955in}}%
\pgfpathlineto{\pgfqpoint{1.343604in}{1.659478in}}%
\pgfpathlineto{\pgfqpoint{1.346825in}{1.660987in}}%
\pgfpathlineto{\pgfqpoint{1.347899in}{1.660566in}}%
\pgfpathlineto{\pgfqpoint{1.348973in}{1.662954in}}%
\pgfpathlineto{\pgfqpoint{1.350047in}{1.658214in}}%
\pgfpathlineto{\pgfqpoint{1.351121in}{1.657371in}}%
\pgfpathlineto{\pgfqpoint{1.354342in}{1.658284in}}%
\pgfpathlineto{\pgfqpoint{1.355416in}{1.655791in}}%
\pgfpathlineto{\pgfqpoint{1.356490in}{1.658565in}}%
\pgfpathlineto{\pgfqpoint{1.358638in}{1.658811in}}%
\pgfpathlineto{\pgfqpoint{1.361859in}{1.663866in}}%
\pgfpathlineto{\pgfqpoint{1.362933in}{1.664217in}}%
\pgfpathlineto{\pgfqpoint{1.364007in}{1.661619in}}%
\pgfpathlineto{\pgfqpoint{1.366154in}{1.653123in}}%
\pgfpathlineto{\pgfqpoint{1.369376in}{1.654492in}}%
\pgfpathlineto{\pgfqpoint{1.370450in}{1.648559in}}%
\pgfpathlineto{\pgfqpoint{1.371524in}{1.653931in}}%
\pgfpathlineto{\pgfqpoint{1.372598in}{1.654563in}}%
\pgfpathlineto{\pgfqpoint{1.373671in}{1.656107in}}%
\pgfpathlineto{\pgfqpoint{1.379041in}{1.657371in}}%
\pgfpathlineto{\pgfqpoint{1.380114in}{1.658635in}}%
\pgfpathlineto{\pgfqpoint{1.381188in}{1.658214in}}%
\pgfpathlineto{\pgfqpoint{1.385484in}{1.663199in}}%
\pgfpathlineto{\pgfqpoint{1.386557in}{1.661058in}}%
\pgfpathlineto{\pgfqpoint{1.388705in}{1.666535in}}%
\pgfpathlineto{\pgfqpoint{1.391927in}{1.670221in}}%
\pgfpathlineto{\pgfqpoint{1.395148in}{1.660180in}}%
\pgfpathlineto{\pgfqpoint{1.396222in}{1.660004in}}%
\pgfpathlineto{\pgfqpoint{1.400517in}{1.660671in}}%
\pgfpathlineto{\pgfqpoint{1.402665in}{1.662427in}}%
\pgfpathlineto{\pgfqpoint{1.403739in}{1.663726in}}%
\pgfpathlineto{\pgfqpoint{1.406960in}{1.661058in}}%
\pgfpathlineto{\pgfqpoint{1.408034in}{1.656564in}}%
\pgfpathlineto{\pgfqpoint{1.409108in}{1.657898in}}%
\pgfpathlineto{\pgfqpoint{1.410182in}{1.656774in}}%
\pgfpathlineto{\pgfqpoint{1.411256in}{1.659337in}}%
\pgfpathlineto{\pgfqpoint{1.414477in}{1.655826in}}%
\pgfpathlineto{\pgfqpoint{1.415551in}{1.657336in}}%
\pgfpathlineto{\pgfqpoint{1.416625in}{1.654879in}}%
\pgfpathlineto{\pgfqpoint{1.417699in}{1.656072in}}%
\pgfpathlineto{\pgfqpoint{1.421994in}{1.654738in}}%
\pgfpathlineto{\pgfqpoint{1.423068in}{1.651543in}}%
\pgfpathlineto{\pgfqpoint{1.425216in}{1.649823in}}%
\pgfpathlineto{\pgfqpoint{1.426289in}{1.651754in}}%
\pgfpathlineto{\pgfqpoint{1.429511in}{1.654071in}}%
\pgfpathlineto{\pgfqpoint{1.430585in}{1.653931in}}%
\pgfpathlineto{\pgfqpoint{1.431659in}{1.652456in}}%
\pgfpathlineto{\pgfqpoint{1.432732in}{1.647506in}}%
\pgfpathlineto{\pgfqpoint{1.433806in}{1.649998in}}%
\pgfpathlineto{\pgfqpoint{1.437028in}{1.648173in}}%
\pgfpathlineto{\pgfqpoint{1.439175in}{1.638096in}}%
\pgfpathlineto{\pgfqpoint{1.440249in}{1.635674in}}%
\pgfpathlineto{\pgfqpoint{1.441323in}{1.635463in}}%
\pgfpathlineto{\pgfqpoint{1.444545in}{1.635814in}}%
\pgfpathlineto{\pgfqpoint{1.446692in}{1.627704in}}%
\pgfpathlineto{\pgfqpoint{1.447766in}{1.623877in}}%
\pgfpathlineto{\pgfqpoint{1.448840in}{1.622649in}}%
\pgfpathlineto{\pgfqpoint{1.453135in}{1.623491in}}%
\pgfpathlineto{\pgfqpoint{1.454209in}{1.619664in}}%
\pgfpathlineto{\pgfqpoint{1.455283in}{1.620998in}}%
\pgfpathlineto{\pgfqpoint{1.456357in}{1.626405in}}%
\pgfpathlineto{\pgfqpoint{1.459578in}{1.625668in}}%
\pgfpathlineto{\pgfqpoint{1.460652in}{1.623140in}}%
\pgfpathlineto{\pgfqpoint{1.461726in}{1.627002in}}%
\pgfpathlineto{\pgfqpoint{1.462800in}{1.627669in}}%
\pgfpathlineto{\pgfqpoint{1.463874in}{1.627248in}}%
\pgfpathlineto{\pgfqpoint{1.469243in}{1.639957in}}%
\pgfpathlineto{\pgfqpoint{1.470317in}{1.641046in}}%
\pgfpathlineto{\pgfqpoint{1.471391in}{1.638974in}}%
\pgfpathlineto{\pgfqpoint{1.474612in}{1.640308in}}%
\pgfpathlineto{\pgfqpoint{1.475686in}{1.639852in}}%
\pgfpathlineto{\pgfqpoint{1.476760in}{1.638026in}}%
\pgfpathlineto{\pgfqpoint{1.477834in}{1.638096in}}%
\pgfpathlineto{\pgfqpoint{1.478908in}{1.634550in}}%
\pgfpathlineto{\pgfqpoint{1.484277in}{1.638377in}}%
\pgfpathlineto{\pgfqpoint{1.485351in}{1.638412in}}%
\pgfpathlineto{\pgfqpoint{1.486424in}{1.636797in}}%
\pgfpathlineto{\pgfqpoint{1.497163in}{1.635639in}}%
\pgfpathlineto{\pgfqpoint{1.498237in}{1.636306in}}%
\pgfpathlineto{\pgfqpoint{1.499310in}{1.634726in}}%
\pgfpathlineto{\pgfqpoint{1.500384in}{1.636236in}}%
\pgfpathlineto{\pgfqpoint{1.501458in}{1.636025in}}%
\pgfpathlineto{\pgfqpoint{1.504680in}{1.629846in}}%
\pgfpathlineto{\pgfqpoint{1.505753in}{1.626581in}}%
\pgfpathlineto{\pgfqpoint{1.506827in}{1.628617in}}%
\pgfpathlineto{\pgfqpoint{1.507901in}{1.623667in}}%
\pgfpathlineto{\pgfqpoint{1.508975in}{1.626019in}}%
\pgfpathlineto{\pgfqpoint{1.512197in}{1.625527in}}%
\pgfpathlineto{\pgfqpoint{1.513270in}{1.627142in}}%
\pgfpathlineto{\pgfqpoint{1.514344in}{1.621665in}}%
\pgfpathlineto{\pgfqpoint{1.515418in}{1.619770in}}%
\pgfpathlineto{\pgfqpoint{1.516492in}{1.623456in}}%
\pgfpathlineto{\pgfqpoint{1.519713in}{1.622824in}}%
\pgfpathlineto{\pgfqpoint{1.520787in}{1.614117in}}%
\pgfpathlineto{\pgfqpoint{1.521861in}{1.617874in}}%
\pgfpathlineto{\pgfqpoint{1.522935in}{1.609483in}}%
\pgfpathlineto{\pgfqpoint{1.524009in}{1.609483in}}%
\pgfpathlineto{\pgfqpoint{1.527230in}{1.607517in}}%
\pgfpathlineto{\pgfqpoint{1.528304in}{1.610044in}}%
\pgfpathlineto{\pgfqpoint{1.529378in}{1.607095in}}%
\pgfpathlineto{\pgfqpoint{1.530452in}{1.607271in}}%
\pgfpathlineto{\pgfqpoint{1.531526in}{1.614433in}}%
\pgfpathlineto{\pgfqpoint{1.534747in}{1.614293in}}%
\pgfpathlineto{\pgfqpoint{1.535821in}{1.615837in}}%
\pgfpathlineto{\pgfqpoint{1.536895in}{1.613345in}}%
\pgfpathlineto{\pgfqpoint{1.537969in}{1.619594in}}%
\pgfpathlineto{\pgfqpoint{1.539042in}{1.621595in}}%
\pgfpathlineto{\pgfqpoint{1.542264in}{1.622754in}}%
\pgfpathlineto{\pgfqpoint{1.543338in}{1.629214in}}%
\pgfpathlineto{\pgfqpoint{1.544412in}{1.627915in}}%
\pgfpathlineto{\pgfqpoint{1.546559in}{1.631496in}}%
\pgfpathlineto{\pgfqpoint{1.549781in}{1.629319in}}%
\pgfpathlineto{\pgfqpoint{1.550855in}{1.631145in}}%
\pgfpathlineto{\pgfqpoint{1.553002in}{1.636446in}}%
\pgfpathlineto{\pgfqpoint{1.554076in}{1.637991in}}%
\pgfpathlineto{\pgfqpoint{1.557298in}{1.637745in}}%
\pgfpathlineto{\pgfqpoint{1.558372in}{1.635428in}}%
\pgfpathlineto{\pgfqpoint{1.559445in}{1.636903in}}%
\pgfpathlineto{\pgfqpoint{1.560519in}{1.636903in}}%
\pgfpathlineto{\pgfqpoint{1.561593in}{1.634761in}}%
\pgfpathlineto{\pgfqpoint{1.564815in}{1.634480in}}%
\pgfpathlineto{\pgfqpoint{1.565888in}{1.638974in}}%
\pgfpathlineto{\pgfqpoint{1.566962in}{1.638553in}}%
\pgfpathlineto{\pgfqpoint{1.568036in}{1.639044in}}%
\pgfpathlineto{\pgfqpoint{1.569110in}{1.643679in}}%
\pgfpathlineto{\pgfqpoint{1.572331in}{1.638939in}}%
\pgfpathlineto{\pgfqpoint{1.573405in}{1.648102in}}%
\pgfpathlineto{\pgfqpoint{1.574479in}{1.643257in}}%
\pgfpathlineto{\pgfqpoint{1.576627in}{1.642977in}}%
\pgfpathlineto{\pgfqpoint{1.580922in}{1.641748in}}%
\pgfpathlineto{\pgfqpoint{1.581996in}{1.645856in}}%
\pgfpathlineto{\pgfqpoint{1.584144in}{1.646803in}}%
\pgfpathlineto{\pgfqpoint{1.587365in}{1.652316in}}%
\pgfpathlineto{\pgfqpoint{1.588439in}{1.658389in}}%
\pgfpathlineto{\pgfqpoint{1.589513in}{1.653720in}}%
\pgfpathlineto{\pgfqpoint{1.590587in}{1.655405in}}%
\pgfpathlineto{\pgfqpoint{1.591661in}{1.649472in}}%
\pgfpathlineto{\pgfqpoint{1.594882in}{1.648734in}}%
\pgfpathlineto{\pgfqpoint{1.597030in}{1.654984in}}%
\pgfpathlineto{\pgfqpoint{1.598104in}{1.664569in}}%
\pgfpathlineto{\pgfqpoint{1.599177in}{1.660285in}}%
\pgfpathlineto{\pgfqpoint{1.602399in}{1.665130in}}%
\pgfpathlineto{\pgfqpoint{1.603473in}{1.665341in}}%
\pgfpathlineto{\pgfqpoint{1.604547in}{1.664393in}}%
\pgfpathlineto{\pgfqpoint{1.606694in}{1.665446in}}%
\pgfpathlineto{\pgfqpoint{1.609916in}{1.664253in}}%
\pgfpathlineto{\pgfqpoint{1.610990in}{1.662181in}}%
\pgfpathlineto{\pgfqpoint{1.612063in}{1.658425in}}%
\pgfpathlineto{\pgfqpoint{1.614211in}{1.658530in}}%
\pgfpathlineto{\pgfqpoint{1.617433in}{1.652526in}}%
\pgfpathlineto{\pgfqpoint{1.618507in}{1.647506in}}%
\pgfpathlineto{\pgfqpoint{1.619580in}{1.651297in}}%
\pgfpathlineto{\pgfqpoint{1.620654in}{1.657336in}}%
\pgfpathlineto{\pgfqpoint{1.621728in}{1.655335in}}%
\pgfpathlineto{\pgfqpoint{1.624950in}{1.656704in}}%
\pgfpathlineto{\pgfqpoint{1.626023in}{1.656353in}}%
\pgfpathlineto{\pgfqpoint{1.627097in}{1.653790in}}%
\pgfpathlineto{\pgfqpoint{1.628171in}{1.653790in}}%
\pgfpathlineto{\pgfqpoint{1.629245in}{1.662041in}}%
\pgfpathlineto{\pgfqpoint{1.633540in}{1.666359in}}%
\pgfpathlineto{\pgfqpoint{1.635688in}{1.675523in}}%
\pgfpathlineto{\pgfqpoint{1.639983in}{1.670186in}}%
\pgfpathlineto{\pgfqpoint{1.641057in}{1.671450in}}%
\pgfpathlineto{\pgfqpoint{1.642131in}{1.664428in}}%
\pgfpathlineto{\pgfqpoint{1.643205in}{1.662954in}}%
\pgfpathlineto{\pgfqpoint{1.644279in}{1.657722in}}%
\pgfpathlineto{\pgfqpoint{1.647500in}{1.663375in}}%
\pgfpathlineto{\pgfqpoint{1.648574in}{1.670607in}}%
\pgfpathlineto{\pgfqpoint{1.649648in}{1.667167in}}%
\pgfpathlineto{\pgfqpoint{1.650722in}{1.674434in}}%
\pgfpathlineto{\pgfqpoint{1.651796in}{1.673486in}}%
\pgfpathlineto{\pgfqpoint{1.655017in}{1.671977in}}%
\pgfpathlineto{\pgfqpoint{1.657165in}{1.672012in}}%
\pgfpathlineto{\pgfqpoint{1.658239in}{1.674820in}}%
\pgfpathlineto{\pgfqpoint{1.659312in}{1.680052in}}%
\pgfpathlineto{\pgfqpoint{1.663608in}{1.680297in}}%
\pgfpathlineto{\pgfqpoint{1.665755in}{1.685283in}}%
\pgfpathlineto{\pgfqpoint{1.666829in}{1.689075in}}%
\pgfpathlineto{\pgfqpoint{1.670051in}{1.687846in}}%
\pgfpathlineto{\pgfqpoint{1.671125in}{1.688092in}}%
\pgfpathlineto{\pgfqpoint{1.673272in}{1.684546in}}%
\pgfpathlineto{\pgfqpoint{1.674346in}{1.682158in}}%
\pgfpathlineto{\pgfqpoint{1.677568in}{1.686196in}}%
\pgfpathlineto{\pgfqpoint{1.678641in}{1.681491in}}%
\pgfpathlineto{\pgfqpoint{1.679715in}{1.679420in}}%
\pgfpathlineto{\pgfqpoint{1.680789in}{1.678612in}}%
\pgfpathlineto{\pgfqpoint{1.681863in}{1.674329in}}%
\pgfpathlineto{\pgfqpoint{1.685085in}{1.681631in}}%
\pgfpathlineto{\pgfqpoint{1.686158in}{1.667939in}}%
\pgfpathlineto{\pgfqpoint{1.687232in}{1.670888in}}%
\pgfpathlineto{\pgfqpoint{1.688306in}{1.680052in}}%
\pgfpathlineto{\pgfqpoint{1.689380in}{1.672152in}}%
\pgfpathlineto{\pgfqpoint{1.692601in}{1.676330in}}%
\pgfpathlineto{\pgfqpoint{1.693675in}{1.675698in}}%
\pgfpathlineto{\pgfqpoint{1.694749in}{1.677102in}}%
\pgfpathlineto{\pgfqpoint{1.695823in}{1.674188in}}%
\pgfpathlineto{\pgfqpoint{1.696897in}{1.674434in}}%
\pgfpathlineto{\pgfqpoint{1.700118in}{1.671977in}}%
\pgfpathlineto{\pgfqpoint{1.701192in}{1.672714in}}%
\pgfpathlineto{\pgfqpoint{1.702266in}{1.664955in}}%
\pgfpathlineto{\pgfqpoint{1.703340in}{1.663656in}}%
\pgfpathlineto{\pgfqpoint{1.704414in}{1.666359in}}%
\pgfpathlineto{\pgfqpoint{1.707635in}{1.672433in}}%
\pgfpathlineto{\pgfqpoint{1.709783in}{1.663270in}}%
\pgfpathlineto{\pgfqpoint{1.710857in}{1.667061in}}%
\pgfpathlineto{\pgfqpoint{1.715152in}{1.669378in}}%
\pgfpathlineto{\pgfqpoint{1.716226in}{1.668220in}}%
\pgfpathlineto{\pgfqpoint{1.717300in}{1.669308in}}%
\pgfpathlineto{\pgfqpoint{1.718374in}{1.669449in}}%
\pgfpathlineto{\pgfqpoint{1.719447in}{1.671274in}}%
\pgfpathlineto{\pgfqpoint{1.722669in}{1.667974in}}%
\pgfpathlineto{\pgfqpoint{1.724817in}{1.669414in}}%
\pgfpathlineto{\pgfqpoint{1.725890in}{1.668185in}}%
\pgfpathlineto{\pgfqpoint{1.726964in}{1.660707in}}%
\pgfpathlineto{\pgfqpoint{1.731260in}{1.666500in}}%
\pgfpathlineto{\pgfqpoint{1.732333in}{1.666535in}}%
\pgfpathlineto{\pgfqpoint{1.733407in}{1.667483in}}%
\pgfpathlineto{\pgfqpoint{1.734481in}{1.664007in}}%
\pgfpathlineto{\pgfqpoint{1.737703in}{1.662708in}}%
\pgfpathlineto{\pgfqpoint{1.738776in}{1.663761in}}%
\pgfpathlineto{\pgfqpoint{1.739850in}{1.661549in}}%
\pgfpathlineto{\pgfqpoint{1.740924in}{1.656423in}}%
\pgfpathlineto{\pgfqpoint{1.741998in}{1.661760in}}%
\pgfpathlineto{\pgfqpoint{1.745219in}{1.664920in}}%
\pgfpathlineto{\pgfqpoint{1.746293in}{1.660601in}}%
\pgfpathlineto{\pgfqpoint{1.747367in}{1.660601in}}%
\pgfpathlineto{\pgfqpoint{1.748441in}{1.663656in}}%
\pgfpathlineto{\pgfqpoint{1.749515in}{1.671169in}}%
\pgfpathlineto{\pgfqpoint{1.753810in}{1.667799in}}%
\pgfpathlineto{\pgfqpoint{1.754884in}{1.669870in}}%
\pgfpathlineto{\pgfqpoint{1.755958in}{1.675487in}}%
\pgfpathlineto{\pgfqpoint{1.757032in}{1.673416in}}%
\pgfpathlineto{\pgfqpoint{1.760253in}{1.673486in}}%
\pgfpathlineto{\pgfqpoint{1.761327in}{1.675242in}}%
\pgfpathlineto{\pgfqpoint{1.762401in}{1.674610in}}%
\pgfpathlineto{\pgfqpoint{1.763475in}{1.675417in}}%
\pgfpathlineto{\pgfqpoint{1.768844in}{1.668290in}}%
\pgfpathlineto{\pgfqpoint{1.769918in}{1.670713in}}%
\pgfpathlineto{\pgfqpoint{1.770992in}{1.670888in}}%
\pgfpathlineto{\pgfqpoint{1.772065in}{1.669238in}}%
\pgfpathlineto{\pgfqpoint{1.775287in}{1.668676in}}%
\pgfpathlineto{\pgfqpoint{1.776361in}{1.669765in}}%
\pgfpathlineto{\pgfqpoint{1.777435in}{1.673486in}}%
\pgfpathlineto{\pgfqpoint{1.778508in}{1.669133in}}%
\pgfpathlineto{\pgfqpoint{1.779582in}{1.668676in}}%
\pgfpathlineto{\pgfqpoint{1.782804in}{1.666078in}}%
\pgfpathlineto{\pgfqpoint{1.783878in}{1.666464in}}%
\pgfpathlineto{\pgfqpoint{1.786025in}{1.672889in}}%
\pgfpathlineto{\pgfqpoint{1.787099in}{1.670607in}}%
\pgfpathlineto{\pgfqpoint{1.790321in}{1.661128in}}%
\pgfpathlineto{\pgfqpoint{1.792468in}{1.662567in}}%
\pgfpathlineto{\pgfqpoint{1.793542in}{1.665201in}}%
\pgfpathlineto{\pgfqpoint{1.794616in}{1.662006in}}%
\pgfpathlineto{\pgfqpoint{1.798911in}{1.663094in}}%
\pgfpathlineto{\pgfqpoint{1.801059in}{1.656950in}}%
\pgfpathlineto{\pgfqpoint{1.802133in}{1.657652in}}%
\pgfpathlineto{\pgfqpoint{1.806428in}{1.649437in}}%
\pgfpathlineto{\pgfqpoint{1.807502in}{1.649086in}}%
\pgfpathlineto{\pgfqpoint{1.808576in}{1.645118in}}%
\pgfpathlineto{\pgfqpoint{1.812871in}{1.644486in}}%
\pgfpathlineto{\pgfqpoint{1.813945in}{1.646803in}}%
\pgfpathlineto{\pgfqpoint{1.815019in}{1.642274in}}%
\pgfpathlineto{\pgfqpoint{1.816093in}{1.643082in}}%
\pgfpathlineto{\pgfqpoint{1.817167in}{1.647084in}}%
\pgfpathlineto{\pgfqpoint{1.820388in}{1.651508in}}%
\pgfpathlineto{\pgfqpoint{1.821462in}{1.651297in}}%
\pgfpathlineto{\pgfqpoint{1.822536in}{1.650455in}}%
\pgfpathlineto{\pgfqpoint{1.823610in}{1.650490in}}%
\pgfpathlineto{\pgfqpoint{1.824684in}{1.648840in}}%
\pgfpathlineto{\pgfqpoint{1.827905in}{1.648032in}}%
\pgfpathlineto{\pgfqpoint{1.828979in}{1.623386in}}%
\pgfpathlineto{\pgfqpoint{1.830053in}{1.619734in}}%
\pgfpathlineto{\pgfqpoint{1.831127in}{1.618400in}}%
\pgfpathlineto{\pgfqpoint{1.832200in}{1.612607in}}%
\pgfpathlineto{\pgfqpoint{1.835422in}{1.611238in}}%
\pgfpathlineto{\pgfqpoint{1.836496in}{1.611554in}}%
\pgfpathlineto{\pgfqpoint{1.837570in}{1.612783in}}%
\pgfpathlineto{\pgfqpoint{1.838643in}{1.617101in}}%
\pgfpathlineto{\pgfqpoint{1.839717in}{1.615802in}}%
\pgfpathlineto{\pgfqpoint{1.845086in}{1.611238in}}%
\pgfpathlineto{\pgfqpoint{1.846160in}{1.611554in}}%
\pgfpathlineto{\pgfqpoint{1.847234in}{1.609377in}}%
\pgfpathlineto{\pgfqpoint{1.850456in}{1.613380in}}%
\pgfpathlineto{\pgfqpoint{1.851529in}{1.610396in}}%
\pgfpathlineto{\pgfqpoint{1.852603in}{1.612678in}}%
\pgfpathlineto{\pgfqpoint{1.853677in}{1.611695in}}%
\pgfpathlineto{\pgfqpoint{1.854751in}{1.612713in}}%
\pgfpathlineto{\pgfqpoint{1.859046in}{1.614995in}}%
\pgfpathlineto{\pgfqpoint{1.860120in}{1.611168in}}%
\pgfpathlineto{\pgfqpoint{1.862268in}{1.594561in}}%
\pgfpathlineto{\pgfqpoint{1.865489in}{1.587750in}}%
\pgfpathlineto{\pgfqpoint{1.866563in}{1.581150in}}%
\pgfpathlineto{\pgfqpoint{1.868711in}{1.595404in}}%
\pgfpathlineto{\pgfqpoint{1.869785in}{1.595299in}}%
\pgfpathlineto{\pgfqpoint{1.873006in}{1.590103in}}%
\pgfpathlineto{\pgfqpoint{1.874080in}{1.584064in}}%
\pgfpathlineto{\pgfqpoint{1.875154in}{1.588909in}}%
\pgfpathlineto{\pgfqpoint{1.876228in}{1.591051in}}%
\pgfpathlineto{\pgfqpoint{1.877302in}{1.587153in}}%
\pgfpathlineto{\pgfqpoint{1.881597in}{1.593965in}}%
\pgfpathlineto{\pgfqpoint{1.883745in}{1.589436in}}%
\pgfpathlineto{\pgfqpoint{1.884818in}{1.592420in}}%
\pgfpathlineto{\pgfqpoint{1.888040in}{1.590805in}}%
\pgfpathlineto{\pgfqpoint{1.890188in}{1.597546in}}%
\pgfpathlineto{\pgfqpoint{1.891262in}{1.595544in}}%
\pgfpathlineto{\pgfqpoint{1.892335in}{1.588382in}}%
\pgfpathlineto{\pgfqpoint{1.895557in}{1.589927in}}%
\pgfpathlineto{\pgfqpoint{1.896631in}{1.579219in}}%
\pgfpathlineto{\pgfqpoint{1.897705in}{1.575287in}}%
\pgfpathlineto{\pgfqpoint{1.898778in}{1.574830in}}%
\pgfpathlineto{\pgfqpoint{1.899852in}{1.576340in}}%
\pgfpathlineto{\pgfqpoint{1.903074in}{1.574865in}}%
\pgfpathlineto{\pgfqpoint{1.905221in}{1.581747in}}%
\pgfpathlineto{\pgfqpoint{1.906295in}{1.579745in}}%
\pgfpathlineto{\pgfqpoint{1.907369in}{1.584239in}}%
\pgfpathlineto{\pgfqpoint{1.910591in}{1.592174in}}%
\pgfpathlineto{\pgfqpoint{1.911664in}{1.593122in}}%
\pgfpathlineto{\pgfqpoint{1.914886in}{1.602110in}}%
\pgfpathlineto{\pgfqpoint{1.918107in}{1.602285in}}%
\pgfpathlineto{\pgfqpoint{1.919181in}{1.598634in}}%
\pgfpathlineto{\pgfqpoint{1.920255in}{1.591893in}}%
\pgfpathlineto{\pgfqpoint{1.921329in}{1.595123in}}%
\pgfpathlineto{\pgfqpoint{1.922403in}{1.594526in}}%
\pgfpathlineto{\pgfqpoint{1.925624in}{1.591507in}}%
\pgfpathlineto{\pgfqpoint{1.927772in}{1.610396in}}%
\pgfpathlineto{\pgfqpoint{1.928846in}{1.616188in}}%
\pgfpathlineto{\pgfqpoint{1.929920in}{1.618857in}}%
\pgfpathlineto{\pgfqpoint{1.933141in}{1.617593in}}%
\pgfpathlineto{\pgfqpoint{1.934215in}{1.613310in}}%
\pgfpathlineto{\pgfqpoint{1.935289in}{1.614714in}}%
\pgfpathlineto{\pgfqpoint{1.936363in}{1.613836in}}%
\pgfpathlineto{\pgfqpoint{1.937437in}{1.611800in}}%
\pgfpathlineto{\pgfqpoint{1.940658in}{1.614925in}}%
\pgfpathlineto{\pgfqpoint{1.942806in}{1.617944in}}%
\pgfpathlineto{\pgfqpoint{1.943880in}{1.619419in}}%
\pgfpathlineto{\pgfqpoint{1.944953in}{1.619419in}}%
\pgfpathlineto{\pgfqpoint{1.950323in}{1.613345in}}%
\pgfpathlineto{\pgfqpoint{1.951396in}{1.616434in}}%
\pgfpathlineto{\pgfqpoint{1.952470in}{1.607622in}}%
\pgfpathlineto{\pgfqpoint{1.955692in}{1.611975in}}%
\pgfpathlineto{\pgfqpoint{1.956766in}{1.611063in}}%
\pgfpathlineto{\pgfqpoint{1.957840in}{1.611589in}}%
\pgfpathlineto{\pgfqpoint{1.958913in}{1.613485in}}%
\pgfpathlineto{\pgfqpoint{1.963209in}{1.612818in}}%
\pgfpathlineto{\pgfqpoint{1.964283in}{1.610466in}}%
\pgfpathlineto{\pgfqpoint{1.965356in}{1.610185in}}%
\pgfpathlineto{\pgfqpoint{1.970726in}{1.606253in}}%
\pgfpathlineto{\pgfqpoint{1.971799in}{1.608324in}}%
\pgfpathlineto{\pgfqpoint{1.972873in}{1.603690in}}%
\pgfpathlineto{\pgfqpoint{1.973947in}{1.601934in}}%
\pgfpathlineto{\pgfqpoint{1.975021in}{1.605235in}}%
\pgfpathlineto{\pgfqpoint{1.978242in}{1.605796in}}%
\pgfpathlineto{\pgfqpoint{1.979316in}{1.600425in}}%
\pgfpathlineto{\pgfqpoint{1.980390in}{1.600179in}}%
\pgfpathlineto{\pgfqpoint{1.981464in}{1.599266in}}%
\pgfpathlineto{\pgfqpoint{1.982538in}{1.597405in}}%
\pgfpathlineto{\pgfqpoint{1.985759in}{1.596563in}}%
\pgfpathlineto{\pgfqpoint{1.986833in}{1.597300in}}%
\pgfpathlineto{\pgfqpoint{1.987907in}{1.603304in}}%
\pgfpathlineto{\pgfqpoint{1.990055in}{1.594351in}}%
\pgfpathlineto{\pgfqpoint{1.993276in}{1.598423in}}%
\pgfpathlineto{\pgfqpoint{1.995424in}{1.607060in}}%
\pgfpathlineto{\pgfqpoint{1.996498in}{1.607060in}}%
\pgfpathlineto{\pgfqpoint{2.000793in}{1.606393in}}%
\pgfpathlineto{\pgfqpoint{2.001867in}{1.610185in}}%
\pgfpathlineto{\pgfqpoint{2.002941in}{1.608991in}}%
\pgfpathlineto{\pgfqpoint{2.004015in}{1.606323in}}%
\pgfpathlineto{\pgfqpoint{2.008310in}{1.604708in}}%
\pgfpathlineto{\pgfqpoint{2.009384in}{1.605199in}}%
\pgfpathlineto{\pgfqpoint{2.010458in}{1.596843in}}%
\pgfpathlineto{\pgfqpoint{2.012605in}{1.588101in}}%
\pgfpathlineto{\pgfqpoint{2.016901in}{1.588417in}}%
\pgfpathlineto{\pgfqpoint{2.017974in}{1.583081in}}%
\pgfpathlineto{\pgfqpoint{2.019048in}{1.583643in}}%
\pgfpathlineto{\pgfqpoint{2.020122in}{1.572794in}}%
\pgfpathlineto{\pgfqpoint{2.024417in}{1.571530in}}%
\pgfpathlineto{\pgfqpoint{2.025491in}{1.570231in}}%
\pgfpathlineto{\pgfqpoint{2.027639in}{1.575111in}}%
\pgfpathlineto{\pgfqpoint{2.030861in}{1.570442in}}%
\pgfpathlineto{\pgfqpoint{2.031934in}{1.572864in}}%
\pgfpathlineto{\pgfqpoint{2.033008in}{1.573356in}}%
\pgfpathlineto{\pgfqpoint{2.034082in}{1.575427in}}%
\pgfpathlineto{\pgfqpoint{2.035156in}{1.579394in}}%
\pgfpathlineto{\pgfqpoint{2.038377in}{1.578973in}}%
\pgfpathlineto{\pgfqpoint{2.039451in}{1.572127in}}%
\pgfpathlineto{\pgfqpoint{2.040525in}{1.573847in}}%
\pgfpathlineto{\pgfqpoint{2.041599in}{1.580799in}}%
\pgfpathlineto{\pgfqpoint{2.045894in}{1.576480in}}%
\pgfpathlineto{\pgfqpoint{2.046968in}{1.577955in}}%
\pgfpathlineto{\pgfqpoint{2.048042in}{1.577077in}}%
\pgfpathlineto{\pgfqpoint{2.049116in}{1.569669in}}%
\pgfpathlineto{\pgfqpoint{2.050190in}{1.573812in}}%
\pgfpathlineto{\pgfqpoint{2.054485in}{1.575532in}}%
\pgfpathlineto{\pgfqpoint{2.055559in}{1.582905in}}%
\pgfpathlineto{\pgfqpoint{2.056633in}{1.583678in}}%
\pgfpathlineto{\pgfqpoint{2.057706in}{1.583256in}}%
\pgfpathlineto{\pgfqpoint{2.060928in}{1.596633in}}%
\pgfpathlineto{\pgfqpoint{2.062002in}{1.594140in}}%
\pgfpathlineto{\pgfqpoint{2.063076in}{1.600670in}}%
\pgfpathlineto{\pgfqpoint{2.064150in}{1.615100in}}%
\pgfpathlineto{\pgfqpoint{2.068445in}{1.610396in}}%
\pgfpathlineto{\pgfqpoint{2.069519in}{1.605305in}}%
\pgfpathlineto{\pgfqpoint{2.071666in}{1.608745in}}%
\pgfpathlineto{\pgfqpoint{2.072740in}{1.611624in}}%
\pgfpathlineto{\pgfqpoint{2.077036in}{1.611308in}}%
\pgfpathlineto{\pgfqpoint{2.079183in}{1.608956in}}%
\pgfpathlineto{\pgfqpoint{2.080257in}{1.610817in}}%
\pgfpathlineto{\pgfqpoint{2.083479in}{1.611063in}}%
\pgfpathlineto{\pgfqpoint{2.084552in}{1.609342in}}%
\pgfpathlineto{\pgfqpoint{2.086700in}{1.617593in}}%
\pgfpathlineto{\pgfqpoint{2.087774in}{1.618260in}}%
\pgfpathlineto{\pgfqpoint{2.090995in}{1.618646in}}%
\pgfpathlineto{\pgfqpoint{2.092069in}{1.616996in}}%
\pgfpathlineto{\pgfqpoint{2.093143in}{1.618435in}}%
\pgfpathlineto{\pgfqpoint{2.098512in}{1.617804in}}%
\pgfpathlineto{\pgfqpoint{2.099586in}{1.621736in}}%
\pgfpathlineto{\pgfqpoint{2.100660in}{1.622157in}}%
\pgfpathlineto{\pgfqpoint{2.102808in}{1.621244in}}%
\pgfpathlineto{\pgfqpoint{2.106029in}{1.622192in}}%
\pgfpathlineto{\pgfqpoint{2.107103in}{1.620893in}}%
\pgfpathlineto{\pgfqpoint{2.110325in}{1.625598in}}%
\pgfpathlineto{\pgfqpoint{2.113546in}{1.628266in}}%
\pgfpathlineto{\pgfqpoint{2.114620in}{1.631285in}}%
\pgfpathlineto{\pgfqpoint{2.115694in}{1.636236in}}%
\pgfpathlineto{\pgfqpoint{2.116768in}{1.636481in}}%
\pgfpathlineto{\pgfqpoint{2.117841in}{1.636130in}}%
\pgfpathlineto{\pgfqpoint{2.121063in}{1.637921in}}%
\pgfpathlineto{\pgfqpoint{2.122137in}{1.637500in}}%
\pgfpathlineto{\pgfqpoint{2.123211in}{1.639220in}}%
\pgfpathlineto{\pgfqpoint{2.124284in}{1.638904in}}%
\pgfpathlineto{\pgfqpoint{2.125358in}{1.639782in}}%
\pgfpathlineto{\pgfqpoint{2.128580in}{1.638167in}}%
\pgfpathlineto{\pgfqpoint{2.129654in}{1.636868in}}%
\pgfpathlineto{\pgfqpoint{2.130728in}{1.640379in}}%
\pgfpathlineto{\pgfqpoint{2.131801in}{1.635007in}}%
\pgfpathlineto{\pgfqpoint{2.132875in}{1.635463in}}%
\pgfpathlineto{\pgfqpoint{2.136097in}{1.635463in}}%
\pgfpathlineto{\pgfqpoint{2.138244in}{1.623491in}}%
\pgfpathlineto{\pgfqpoint{2.139318in}{1.622262in}}%
\pgfpathlineto{\pgfqpoint{2.140392in}{1.624966in}}%
\pgfpathlineto{\pgfqpoint{2.143614in}{1.621630in}}%
\pgfpathlineto{\pgfqpoint{2.144687in}{1.628336in}}%
\pgfpathlineto{\pgfqpoint{2.145761in}{1.626511in}}%
\pgfpathlineto{\pgfqpoint{2.146835in}{1.626054in}}%
\pgfpathlineto{\pgfqpoint{2.147909in}{1.622192in}}%
\pgfpathlineto{\pgfqpoint{2.151130in}{1.627213in}}%
\pgfpathlineto{\pgfqpoint{2.152204in}{1.621525in}}%
\pgfpathlineto{\pgfqpoint{2.153278in}{1.621174in}}%
\pgfpathlineto{\pgfqpoint{2.154352in}{1.618611in}}%
\pgfpathlineto{\pgfqpoint{2.155426in}{1.620542in}}%
\pgfpathlineto{\pgfqpoint{2.158647in}{1.619910in}}%
\pgfpathlineto{\pgfqpoint{2.159721in}{1.623175in}}%
\pgfpathlineto{\pgfqpoint{2.160795in}{1.624509in}}%
\pgfpathlineto{\pgfqpoint{2.161869in}{1.624825in}}%
\pgfpathlineto{\pgfqpoint{2.162943in}{1.625949in}}%
\pgfpathlineto{\pgfqpoint{2.167238in}{1.625352in}}%
\pgfpathlineto{\pgfqpoint{2.168312in}{1.624615in}}%
\pgfpathlineto{\pgfqpoint{2.169386in}{1.626054in}}%
\pgfpathlineto{\pgfqpoint{2.170460in}{1.624896in}}%
\pgfpathlineto{\pgfqpoint{2.176903in}{1.631040in}}%
\pgfpathlineto{\pgfqpoint{2.181198in}{1.627423in}}%
\pgfpathlineto{\pgfqpoint{2.182272in}{1.627353in}}%
\pgfpathlineto{\pgfqpoint{2.183346in}{1.625247in}}%
\pgfpathlineto{\pgfqpoint{2.184419in}{1.627142in}}%
\pgfpathlineto{\pgfqpoint{2.185493in}{1.627388in}}%
\pgfpathlineto{\pgfqpoint{2.188715in}{1.629003in}}%
\pgfpathlineto{\pgfqpoint{2.190862in}{1.627880in}}%
\pgfpathlineto{\pgfqpoint{2.191936in}{1.631075in}}%
\pgfpathlineto{\pgfqpoint{2.193010in}{1.619875in}}%
\pgfpathlineto{\pgfqpoint{2.196232in}{1.614398in}}%
\pgfpathlineto{\pgfqpoint{2.199453in}{1.631777in}}%
\pgfpathlineto{\pgfqpoint{2.200527in}{1.632374in}}%
\pgfpathlineto{\pgfqpoint{2.204822in}{1.625492in}}%
\pgfpathlineto{\pgfqpoint{2.206970in}{1.629881in}}%
\pgfpathlineto{\pgfqpoint{2.208044in}{1.635393in}}%
\pgfpathlineto{\pgfqpoint{2.211265in}{1.636376in}}%
\pgfpathlineto{\pgfqpoint{2.213413in}{1.640063in}}%
\pgfpathlineto{\pgfqpoint{2.214487in}{1.640203in}}%
\pgfpathlineto{\pgfqpoint{2.215561in}{1.641397in}}%
\pgfpathlineto{\pgfqpoint{2.219856in}{1.641783in}}%
\pgfpathlineto{\pgfqpoint{2.220930in}{1.643152in}}%
\pgfpathlineto{\pgfqpoint{2.222004in}{1.642520in}}%
\pgfpathlineto{\pgfqpoint{2.223078in}{1.640168in}}%
\pgfpathlineto{\pgfqpoint{2.226299in}{1.638623in}}%
\pgfpathlineto{\pgfqpoint{2.227373in}{1.649156in}}%
\pgfpathlineto{\pgfqpoint{2.229521in}{1.648173in}}%
\pgfpathlineto{\pgfqpoint{2.230594in}{1.648383in}}%
\pgfpathlineto{\pgfqpoint{2.233816in}{1.646066in}}%
\pgfpathlineto{\pgfqpoint{2.234890in}{1.643889in}}%
\pgfpathlineto{\pgfqpoint{2.237038in}{1.644205in}}%
\pgfpathlineto{\pgfqpoint{2.238111in}{1.648699in}}%
\pgfpathlineto{\pgfqpoint{2.241333in}{1.648840in}}%
\pgfpathlineto{\pgfqpoint{2.242407in}{1.650630in}}%
\pgfpathlineto{\pgfqpoint{2.243481in}{1.649928in}}%
\pgfpathlineto{\pgfqpoint{2.244554in}{1.653369in}}%
\pgfpathlineto{\pgfqpoint{2.245628in}{1.652351in}}%
\pgfpathlineto{\pgfqpoint{2.248850in}{1.655019in}}%
\pgfpathlineto{\pgfqpoint{2.249924in}{1.653299in}}%
\pgfpathlineto{\pgfqpoint{2.252071in}{1.656037in}}%
\pgfpathlineto{\pgfqpoint{2.256367in}{1.653299in}}%
\pgfpathlineto{\pgfqpoint{2.257440in}{1.651684in}}%
\pgfpathlineto{\pgfqpoint{2.258514in}{1.651543in}}%
\pgfpathlineto{\pgfqpoint{2.260662in}{1.649402in}}%
\pgfpathlineto{\pgfqpoint{2.263883in}{1.651578in}}%
\pgfpathlineto{\pgfqpoint{2.266031in}{1.646523in}}%
\pgfpathlineto{\pgfqpoint{2.268179in}{1.648173in}}%
\pgfpathlineto{\pgfqpoint{2.274622in}{1.644311in}}%
\pgfpathlineto{\pgfqpoint{2.275696in}{1.634234in}}%
\pgfpathlineto{\pgfqpoint{2.278917in}{1.638658in}}%
\pgfpathlineto{\pgfqpoint{2.279991in}{1.633076in}}%
\pgfpathlineto{\pgfqpoint{2.281065in}{1.630899in}}%
\pgfpathlineto{\pgfqpoint{2.282139in}{1.634340in}}%
\pgfpathlineto{\pgfqpoint{2.283213in}{1.625773in}}%
\pgfpathlineto{\pgfqpoint{2.286434in}{1.626897in}}%
\pgfpathlineto{\pgfqpoint{2.287508in}{1.626230in}}%
\pgfpathlineto{\pgfqpoint{2.289656in}{1.635288in}}%
\pgfpathlineto{\pgfqpoint{2.290729in}{1.633848in}}%
\pgfpathlineto{\pgfqpoint{2.293951in}{1.632760in}}%
\pgfpathlineto{\pgfqpoint{2.296099in}{1.633146in}}%
\pgfpathlineto{\pgfqpoint{2.297172in}{1.628933in}}%
\pgfpathlineto{\pgfqpoint{2.298246in}{1.630688in}}%
\pgfpathlineto{\pgfqpoint{2.301468in}{1.633427in}}%
\pgfpathlineto{\pgfqpoint{2.302542in}{1.630162in}}%
\pgfpathlineto{\pgfqpoint{2.303616in}{1.632830in}}%
\pgfpathlineto{\pgfqpoint{2.304689in}{1.632268in}}%
\pgfpathlineto{\pgfqpoint{2.305763in}{1.627353in}}%
\pgfpathlineto{\pgfqpoint{2.308985in}{1.625387in}}%
\pgfpathlineto{\pgfqpoint{2.310059in}{1.621420in}}%
\pgfpathlineto{\pgfqpoint{2.311132in}{1.621946in}}%
\pgfpathlineto{\pgfqpoint{2.312206in}{1.624966in}}%
\pgfpathlineto{\pgfqpoint{2.313280in}{1.625984in}}%
\pgfpathlineto{\pgfqpoint{2.316502in}{1.624509in}}%
\pgfpathlineto{\pgfqpoint{2.317575in}{1.625422in}}%
\pgfpathlineto{\pgfqpoint{2.318649in}{1.624685in}}%
\pgfpathlineto{\pgfqpoint{2.320797in}{1.621104in}}%
\pgfpathlineto{\pgfqpoint{2.324018in}{1.623877in}}%
\pgfpathlineto{\pgfqpoint{2.325092in}{1.629916in}}%
\pgfpathlineto{\pgfqpoint{2.326166in}{1.628757in}}%
\pgfpathlineto{\pgfqpoint{2.327240in}{1.625703in}}%
\pgfpathlineto{\pgfqpoint{2.328314in}{1.631496in}}%
\pgfpathlineto{\pgfqpoint{2.331535in}{1.632655in}}%
\pgfpathlineto{\pgfqpoint{2.332609in}{1.632058in}}%
\pgfpathlineto{\pgfqpoint{2.334757in}{1.628898in}}%
\pgfpathlineto{\pgfqpoint{2.335831in}{1.629811in}}%
\pgfpathlineto{\pgfqpoint{2.340126in}{1.636517in}}%
\pgfpathlineto{\pgfqpoint{2.341200in}{1.641221in}}%
\pgfpathlineto{\pgfqpoint{2.342274in}{1.653018in}}%
\pgfpathlineto{\pgfqpoint{2.343348in}{1.654492in}}%
\pgfpathlineto{\pgfqpoint{2.347643in}{1.650630in}}%
\pgfpathlineto{\pgfqpoint{2.348717in}{1.650420in}}%
\pgfpathlineto{\pgfqpoint{2.350864in}{1.648910in}}%
\pgfpathlineto{\pgfqpoint{2.355160in}{1.650349in}}%
\pgfpathlineto{\pgfqpoint{2.356234in}{1.654211in}}%
\pgfpathlineto{\pgfqpoint{2.358381in}{1.656423in}}%
\pgfpathlineto{\pgfqpoint{2.361603in}{1.655335in}}%
\pgfpathlineto{\pgfqpoint{2.362677in}{1.656634in}}%
\pgfpathlineto{\pgfqpoint{2.363750in}{1.652948in}}%
\pgfpathlineto{\pgfqpoint{2.364824in}{1.652140in}}%
\pgfpathlineto{\pgfqpoint{2.365898in}{1.654563in}}%
\pgfpathlineto{\pgfqpoint{2.369120in}{1.651929in}}%
\pgfpathlineto{\pgfqpoint{2.370193in}{1.651964in}}%
\pgfpathlineto{\pgfqpoint{2.371267in}{1.659408in}}%
\pgfpathlineto{\pgfqpoint{2.372341in}{1.655265in}}%
\pgfpathlineto{\pgfqpoint{2.373415in}{1.659759in}}%
\pgfpathlineto{\pgfqpoint{2.376637in}{1.661725in}}%
\pgfpathlineto{\pgfqpoint{2.377710in}{1.661479in}}%
\pgfpathlineto{\pgfqpoint{2.379858in}{1.654282in}}%
\pgfpathlineto{\pgfqpoint{2.380932in}{1.655581in}}%
\pgfpathlineto{\pgfqpoint{2.384153in}{1.663164in}}%
\pgfpathlineto{\pgfqpoint{2.387375in}{1.661970in}}%
\pgfpathlineto{\pgfqpoint{2.388449in}{1.662602in}}%
\pgfpathlineto{\pgfqpoint{2.392744in}{1.663691in}}%
\pgfpathlineto{\pgfqpoint{2.393818in}{1.661058in}}%
\pgfpathlineto{\pgfqpoint{2.394892in}{1.662251in}}%
\pgfpathlineto{\pgfqpoint{2.395966in}{1.659197in}}%
\pgfpathlineto{\pgfqpoint{2.400261in}{1.663199in}}%
\pgfpathlineto{\pgfqpoint{2.401335in}{1.663410in}}%
\pgfpathlineto{\pgfqpoint{2.402409in}{1.664885in}}%
\pgfpathlineto{\pgfqpoint{2.403482in}{1.668852in}}%
\pgfpathlineto{\pgfqpoint{2.408852in}{1.663726in}}%
\pgfpathlineto{\pgfqpoint{2.409926in}{1.663164in}}%
\pgfpathlineto{\pgfqpoint{2.410999in}{1.661163in}}%
\pgfpathlineto{\pgfqpoint{2.415295in}{1.660355in}}%
\pgfpathlineto{\pgfqpoint{2.417442in}{1.662848in}}%
\pgfpathlineto{\pgfqpoint{2.418516in}{1.663059in}}%
\pgfpathlineto{\pgfqpoint{2.421738in}{1.661584in}}%
\pgfpathlineto{\pgfqpoint{2.422812in}{1.665762in}}%
\pgfpathlineto{\pgfqpoint{2.426033in}{1.659443in}}%
\pgfpathlineto{\pgfqpoint{2.429255in}{1.657722in}}%
\pgfpathlineto{\pgfqpoint{2.430328in}{1.659372in}}%
\pgfpathlineto{\pgfqpoint{2.431402in}{1.654457in}}%
\pgfpathlineto{\pgfqpoint{2.432476in}{1.655054in}}%
\pgfpathlineto{\pgfqpoint{2.433550in}{1.659302in}}%
\pgfpathlineto{\pgfqpoint{2.437845in}{1.664569in}}%
\pgfpathlineto{\pgfqpoint{2.438919in}{1.661760in}}%
\pgfpathlineto{\pgfqpoint{2.439993in}{1.660742in}}%
\pgfpathlineto{\pgfqpoint{2.441067in}{1.663901in}}%
\pgfpathlineto{\pgfqpoint{2.444288in}{1.667096in}}%
\pgfpathlineto{\pgfqpoint{2.445362in}{1.665622in}}%
\pgfpathlineto{\pgfqpoint{2.446436in}{1.668922in}}%
\pgfpathlineto{\pgfqpoint{2.448584in}{1.669730in}}%
\pgfpathlineto{\pgfqpoint{2.453953in}{1.671415in}}%
\pgfpathlineto{\pgfqpoint{2.455027in}{1.669519in}}%
\pgfpathlineto{\pgfqpoint{2.456101in}{1.670748in}}%
\pgfpathlineto{\pgfqpoint{2.459322in}{1.671906in}}%
\pgfpathlineto{\pgfqpoint{2.460396in}{1.671064in}}%
\pgfpathlineto{\pgfqpoint{2.461470in}{1.674820in}}%
\pgfpathlineto{\pgfqpoint{2.462544in}{1.671520in}}%
\pgfpathlineto{\pgfqpoint{2.463617in}{1.670362in}}%
\pgfpathlineto{\pgfqpoint{2.466839in}{1.668115in}}%
\pgfpathlineto{\pgfqpoint{2.467913in}{1.670151in}}%
\pgfpathlineto{\pgfqpoint{2.468987in}{1.668395in}}%
\pgfpathlineto{\pgfqpoint{2.471134in}{1.669694in}}%
\pgfpathlineto{\pgfqpoint{2.474356in}{1.670256in}}%
\pgfpathlineto{\pgfqpoint{2.475430in}{1.668676in}}%
\pgfpathlineto{\pgfqpoint{2.476504in}{1.672819in}}%
\pgfpathlineto{\pgfqpoint{2.477577in}{1.670362in}}%
\pgfpathlineto{\pgfqpoint{2.478651in}{1.674048in}}%
\pgfpathlineto{\pgfqpoint{2.481873in}{1.674259in}}%
\pgfpathlineto{\pgfqpoint{2.482947in}{1.669765in}}%
\pgfpathlineto{\pgfqpoint{2.485094in}{1.668711in}}%
\pgfpathlineto{\pgfqpoint{2.489390in}{1.668676in}}%
\pgfpathlineto{\pgfqpoint{2.490463in}{1.671696in}}%
\pgfpathlineto{\pgfqpoint{2.491537in}{1.669449in}}%
\pgfpathlineto{\pgfqpoint{2.492611in}{1.670713in}}%
\pgfpathlineto{\pgfqpoint{2.493685in}{1.669940in}}%
\pgfpathlineto{\pgfqpoint{2.496906in}{1.668992in}}%
\pgfpathlineto{\pgfqpoint{2.497980in}{1.672503in}}%
\pgfpathlineto{\pgfqpoint{2.499054in}{1.670888in}}%
\pgfpathlineto{\pgfqpoint{2.500128in}{1.670607in}}%
\pgfpathlineto{\pgfqpoint{2.501202in}{1.672503in}}%
\pgfpathlineto{\pgfqpoint{2.504423in}{1.672152in}}%
\pgfpathlineto{\pgfqpoint{2.505497in}{1.673030in}}%
\pgfpathlineto{\pgfqpoint{2.506571in}{1.670046in}}%
\pgfpathlineto{\pgfqpoint{2.507645in}{1.669519in}}%
\pgfpathlineto{\pgfqpoint{2.513014in}{1.674399in}}%
\pgfpathlineto{\pgfqpoint{2.514088in}{1.672293in}}%
\pgfpathlineto{\pgfqpoint{2.516236in}{1.679139in}}%
\pgfpathlineto{\pgfqpoint{2.520531in}{1.685388in}}%
\pgfpathlineto{\pgfqpoint{2.521605in}{1.689812in}}%
\pgfpathlineto{\pgfqpoint{2.522679in}{1.691708in}}%
\pgfpathlineto{\pgfqpoint{2.523752in}{1.692410in}}%
\pgfpathlineto{\pgfqpoint{2.526974in}{1.692199in}}%
\pgfpathlineto{\pgfqpoint{2.528048in}{1.693498in}}%
\pgfpathlineto{\pgfqpoint{2.530195in}{1.699081in}}%
\pgfpathlineto{\pgfqpoint{2.531269in}{1.700239in}}%
\pgfpathlineto{\pgfqpoint{2.534491in}{1.699361in}}%
\pgfpathlineto{\pgfqpoint{2.535565in}{1.700625in}}%
\pgfpathlineto{\pgfqpoint{2.536638in}{1.699151in}}%
\pgfpathlineto{\pgfqpoint{2.537712in}{1.700029in}}%
\pgfpathlineto{\pgfqpoint{2.538786in}{1.698519in}}%
\pgfpathlineto{\pgfqpoint{2.542008in}{1.698800in}}%
\pgfpathlineto{\pgfqpoint{2.543081in}{1.700415in}}%
\pgfpathlineto{\pgfqpoint{2.544155in}{1.696553in}}%
\pgfpathlineto{\pgfqpoint{2.545229in}{1.695851in}}%
\pgfpathlineto{\pgfqpoint{2.546303in}{1.701819in}}%
\pgfpathlineto{\pgfqpoint{2.549525in}{1.703469in}}%
\pgfpathlineto{\pgfqpoint{2.550598in}{1.704838in}}%
\pgfpathlineto{\pgfqpoint{2.552746in}{1.705435in}}%
\pgfpathlineto{\pgfqpoint{2.553820in}{1.704101in}}%
\pgfpathlineto{\pgfqpoint{2.559189in}{1.702205in}}%
\pgfpathlineto{\pgfqpoint{2.561337in}{1.705014in}}%
\pgfpathlineto{\pgfqpoint{2.564558in}{1.701222in}}%
\pgfpathlineto{\pgfqpoint{2.565632in}{1.698343in}}%
\pgfpathlineto{\pgfqpoint{2.566706in}{1.697220in}}%
\pgfpathlineto{\pgfqpoint{2.567780in}{1.697746in}}%
\pgfpathlineto{\pgfqpoint{2.568854in}{1.700064in}}%
\pgfpathlineto{\pgfqpoint{2.572075in}{1.697536in}}%
\pgfpathlineto{\pgfqpoint{2.573149in}{1.698133in}}%
\pgfpathlineto{\pgfqpoint{2.574223in}{1.697992in}}%
\pgfpathlineto{\pgfqpoint{2.575297in}{1.700450in}}%
\pgfpathlineto{\pgfqpoint{2.576370in}{1.699361in}}%
\pgfpathlineto{\pgfqpoint{2.579592in}{1.704101in}}%
\pgfpathlineto{\pgfqpoint{2.580666in}{1.702978in}}%
\pgfpathlineto{\pgfqpoint{2.583887in}{1.705330in}}%
\pgfpathlineto{\pgfqpoint{2.588183in}{1.703259in}}%
\pgfpathlineto{\pgfqpoint{2.589257in}{1.706278in}}%
\pgfpathlineto{\pgfqpoint{2.590330in}{1.703434in}}%
\pgfpathlineto{\pgfqpoint{2.591404in}{1.704979in}}%
\pgfpathlineto{\pgfqpoint{2.594626in}{1.704874in}}%
\pgfpathlineto{\pgfqpoint{2.596773in}{1.707261in}}%
\pgfpathlineto{\pgfqpoint{2.597847in}{1.704242in}}%
\pgfpathlineto{\pgfqpoint{2.598921in}{1.706629in}}%
\pgfpathlineto{\pgfqpoint{2.602143in}{1.708139in}}%
\pgfpathlineto{\pgfqpoint{2.603216in}{1.709754in}}%
\pgfpathlineto{\pgfqpoint{2.604290in}{1.710315in}}%
\pgfpathlineto{\pgfqpoint{2.605364in}{1.708174in}}%
\pgfpathlineto{\pgfqpoint{2.606438in}{1.709297in}}%
\pgfpathlineto{\pgfqpoint{2.609659in}{1.708314in}}%
\pgfpathlineto{\pgfqpoint{2.610733in}{1.706699in}}%
\pgfpathlineto{\pgfqpoint{2.611807in}{1.708209in}}%
\pgfpathlineto{\pgfqpoint{2.612881in}{1.706102in}}%
\pgfpathlineto{\pgfqpoint{2.613955in}{1.709578in}}%
\pgfpathlineto{\pgfqpoint{2.617176in}{1.708384in}}%
\pgfpathlineto{\pgfqpoint{2.618250in}{1.699326in}}%
\pgfpathlineto{\pgfqpoint{2.620398in}{1.693569in}}%
\pgfpathlineto{\pgfqpoint{2.621472in}{1.694165in}}%
\pgfpathlineto{\pgfqpoint{2.624693in}{1.693182in}}%
\pgfpathlineto{\pgfqpoint{2.625767in}{1.694200in}}%
\pgfpathlineto{\pgfqpoint{2.627915in}{1.701608in}}%
\pgfpathlineto{\pgfqpoint{2.628989in}{1.702907in}}%
\pgfpathlineto{\pgfqpoint{2.632210in}{1.693007in}}%
\pgfpathlineto{\pgfqpoint{2.633284in}{1.691918in}}%
\pgfpathlineto{\pgfqpoint{2.634358in}{1.688618in}}%
\pgfpathlineto{\pgfqpoint{2.635432in}{1.687144in}}%
\pgfpathlineto{\pgfqpoint{2.639727in}{1.688443in}}%
\pgfpathlineto{\pgfqpoint{2.640801in}{1.684475in}}%
\pgfpathlineto{\pgfqpoint{2.641875in}{1.693674in}}%
\pgfpathlineto{\pgfqpoint{2.642948in}{1.687284in}}%
\pgfpathlineto{\pgfqpoint{2.644022in}{1.685213in}}%
\pgfpathlineto{\pgfqpoint{2.647244in}{1.684546in}}%
\pgfpathlineto{\pgfqpoint{2.648318in}{1.685915in}}%
\pgfpathlineto{\pgfqpoint{2.649392in}{1.690409in}}%
\pgfpathlineto{\pgfqpoint{2.650465in}{1.684616in}}%
\pgfpathlineto{\pgfqpoint{2.651539in}{1.683843in}}%
\pgfpathlineto{\pgfqpoint{2.654761in}{1.684721in}}%
\pgfpathlineto{\pgfqpoint{2.655835in}{1.696026in}}%
\pgfpathlineto{\pgfqpoint{2.656908in}{1.699045in}}%
\pgfpathlineto{\pgfqpoint{2.657982in}{1.699432in}}%
\pgfpathlineto{\pgfqpoint{2.664425in}{1.665516in}}%
\pgfpathlineto{\pgfqpoint{2.665499in}{1.666745in}}%
\pgfpathlineto{\pgfqpoint{2.666573in}{1.665306in}}%
\pgfpathlineto{\pgfqpoint{2.669794in}{1.665622in}}%
\pgfpathlineto{\pgfqpoint{2.671942in}{1.667763in}}%
\pgfpathlineto{\pgfqpoint{2.673016in}{1.677348in}}%
\pgfpathlineto{\pgfqpoint{2.677311in}{1.676119in}}%
\pgfpathlineto{\pgfqpoint{2.679459in}{1.680578in}}%
\pgfpathlineto{\pgfqpoint{2.681607in}{1.683633in}}%
\pgfpathlineto{\pgfqpoint{2.684828in}{1.681210in}}%
\pgfpathlineto{\pgfqpoint{2.685902in}{1.682650in}}%
\pgfpathlineto{\pgfqpoint{2.686976in}{1.690900in}}%
\pgfpathlineto{\pgfqpoint{2.688050in}{1.686406in}}%
\pgfpathlineto{\pgfqpoint{2.689124in}{1.687214in}}%
\pgfpathlineto{\pgfqpoint{2.692345in}{1.692269in}}%
\pgfpathlineto{\pgfqpoint{2.693419in}{1.692691in}}%
\pgfpathlineto{\pgfqpoint{2.694493in}{1.692410in}}%
\pgfpathlineto{\pgfqpoint{2.695567in}{1.694236in}}%
\pgfpathlineto{\pgfqpoint{2.696640in}{1.694446in}}%
\pgfpathlineto{\pgfqpoint{2.699862in}{1.695780in}}%
\pgfpathlineto{\pgfqpoint{2.702010in}{1.692831in}}%
\pgfpathlineto{\pgfqpoint{2.703083in}{1.696412in}}%
\pgfpathlineto{\pgfqpoint{2.704157in}{1.696061in}}%
\pgfpathlineto{\pgfqpoint{2.707379in}{1.697044in}}%
\pgfpathlineto{\pgfqpoint{2.708453in}{1.698238in}}%
\pgfpathlineto{\pgfqpoint{2.709526in}{1.697606in}}%
\pgfpathlineto{\pgfqpoint{2.710600in}{1.698659in}}%
\pgfpathlineto{\pgfqpoint{2.711674in}{1.703504in}}%
\pgfpathlineto{\pgfqpoint{2.714896in}{1.703364in}}%
\pgfpathlineto{\pgfqpoint{2.717043in}{1.696974in}}%
\pgfpathlineto{\pgfqpoint{2.718117in}{1.700134in}}%
\pgfpathlineto{\pgfqpoint{2.719191in}{1.697255in}}%
\pgfpathlineto{\pgfqpoint{2.722413in}{1.699818in}}%
\pgfpathlineto{\pgfqpoint{2.723486in}{1.699572in}}%
\pgfpathlineto{\pgfqpoint{2.724560in}{1.700766in}}%
\pgfpathlineto{\pgfqpoint{2.725634in}{1.705260in}}%
\pgfpathlineto{\pgfqpoint{2.726708in}{1.703961in}}%
\pgfpathlineto{\pgfqpoint{2.729929in}{1.701679in}}%
\pgfpathlineto{\pgfqpoint{2.731003in}{1.702872in}}%
\pgfpathlineto{\pgfqpoint{2.732077in}{1.701328in}}%
\pgfpathlineto{\pgfqpoint{2.733151in}{1.694903in}}%
\pgfpathlineto{\pgfqpoint{2.734225in}{1.694025in}}%
\pgfpathlineto{\pgfqpoint{2.737446in}{1.690549in}}%
\pgfpathlineto{\pgfqpoint{2.738520in}{1.696342in}}%
\pgfpathlineto{\pgfqpoint{2.739594in}{1.692234in}}%
\pgfpathlineto{\pgfqpoint{2.740668in}{1.695675in}}%
\pgfpathlineto{\pgfqpoint{2.741742in}{1.691076in}}%
\pgfpathlineto{\pgfqpoint{2.744963in}{1.690549in}}%
\pgfpathlineto{\pgfqpoint{2.746037in}{1.692796in}}%
\pgfpathlineto{\pgfqpoint{2.747111in}{1.691743in}}%
\pgfpathlineto{\pgfqpoint{2.752480in}{1.693077in}}%
\pgfpathlineto{\pgfqpoint{2.754628in}{1.696588in}}%
\pgfpathlineto{\pgfqpoint{2.755702in}{1.707682in}}%
\pgfpathlineto{\pgfqpoint{2.756775in}{1.703188in}}%
\pgfpathlineto{\pgfqpoint{2.759997in}{1.702907in}}%
\pgfpathlineto{\pgfqpoint{2.761071in}{1.703750in}}%
\pgfpathlineto{\pgfqpoint{2.764292in}{1.712282in}}%
\pgfpathlineto{\pgfqpoint{2.768588in}{1.714529in}}%
\pgfpathlineto{\pgfqpoint{2.769661in}{1.717302in}}%
\pgfpathlineto{\pgfqpoint{2.770735in}{1.715476in}}%
\pgfpathlineto{\pgfqpoint{2.771809in}{1.723622in}}%
\pgfpathlineto{\pgfqpoint{2.779326in}{1.727168in}}%
\pgfpathlineto{\pgfqpoint{2.783621in}{1.726887in}}%
\pgfpathlineto{\pgfqpoint{2.785769in}{1.730187in}}%
\pgfpathlineto{\pgfqpoint{2.786843in}{1.728326in}}%
\pgfpathlineto{\pgfqpoint{2.786843in}{1.728326in}}%
\pgfusepath{stroke}%
\end{pgfscope}%
\begin{pgfscope}%
\pgfpathrectangle{\pgfqpoint{0.320934in}{1.347524in}}{\pgfqpoint{2.583333in}{0.400885in}}%
\pgfusepath{clip}%
\pgfsetroundcap%
\pgfsetroundjoin%
\pgfsetlinewidth{1.505625pt}%
\definecolor{currentstroke}{rgb}{0.890196,0.466667,0.760784}%
\pgfsetstrokecolor{currentstroke}%
\pgfsetdash{}{0pt}%
\pgfpathmoveto{\pgfqpoint{0.438358in}{1.516269in}}%
\pgfpathlineto{\pgfqpoint{0.439432in}{1.515145in}}%
\pgfpathlineto{\pgfqpoint{0.441580in}{1.511809in}}%
\pgfpathlineto{\pgfqpoint{0.444801in}{1.511809in}}%
\pgfpathlineto{\pgfqpoint{0.445875in}{1.506230in}}%
\pgfpathlineto{\pgfqpoint{0.448023in}{1.502934in}}%
\pgfpathlineto{\pgfqpoint{0.449096in}{1.502934in}}%
\pgfpathlineto{\pgfqpoint{0.453392in}{1.500345in}}%
\pgfpathlineto{\pgfqpoint{0.454466in}{1.498863in}}%
\pgfpathlineto{\pgfqpoint{0.455539in}{1.499951in}}%
\pgfpathlineto{\pgfqpoint{0.459835in}{1.499951in}}%
\pgfpathlineto{\pgfqpoint{0.460909in}{1.497521in}}%
\pgfpathlineto{\pgfqpoint{0.463056in}{1.497860in}}%
\pgfpathlineto{\pgfqpoint{0.467352in}{1.497860in}}%
\pgfpathlineto{\pgfqpoint{0.468425in}{1.495942in}}%
\pgfpathlineto{\pgfqpoint{0.469499in}{1.491208in}}%
\pgfpathlineto{\pgfqpoint{0.470573in}{1.491700in}}%
\pgfpathlineto{\pgfqpoint{0.471647in}{1.489199in}}%
\pgfpathlineto{\pgfqpoint{0.475942in}{1.491021in}}%
\pgfpathlineto{\pgfqpoint{0.478090in}{1.482764in}}%
\pgfpathlineto{\pgfqpoint{0.479164in}{1.483400in}}%
\pgfpathlineto{\pgfqpoint{0.482385in}{1.480333in}}%
\pgfpathlineto{\pgfqpoint{0.483459in}{1.480849in}}%
\pgfpathlineto{\pgfqpoint{0.484533in}{1.484222in}}%
\pgfpathlineto{\pgfqpoint{0.486681in}{1.481602in}}%
\pgfpathlineto{\pgfqpoint{0.492050in}{1.481209in}}%
\pgfpathlineto{\pgfqpoint{0.493124in}{1.482226in}}%
\pgfpathlineto{\pgfqpoint{0.494198in}{1.481172in}}%
\pgfpathlineto{\pgfqpoint{0.498493in}{1.482269in}}%
\pgfpathlineto{\pgfqpoint{0.501714in}{1.479916in}}%
\pgfpathlineto{\pgfqpoint{0.507084in}{1.479916in}}%
\pgfpathlineto{\pgfqpoint{0.508157in}{1.477784in}}%
\pgfpathlineto{\pgfqpoint{0.509231in}{1.477759in}}%
\pgfpathlineto{\pgfqpoint{0.512453in}{1.476808in}}%
\pgfpathlineto{\pgfqpoint{0.513527in}{1.471315in}}%
\pgfpathlineto{\pgfqpoint{0.515674in}{1.470671in}}%
\pgfpathlineto{\pgfqpoint{0.516748in}{1.473472in}}%
\pgfpathlineto{\pgfqpoint{0.521044in}{1.475237in}}%
\pgfpathlineto{\pgfqpoint{0.573662in}{1.475237in}}%
\pgfpathlineto{\pgfqpoint{0.574735in}{1.471155in}}%
\pgfpathlineto{\pgfqpoint{0.575809in}{1.471799in}}%
\pgfpathlineto{\pgfqpoint{0.576883in}{1.471531in}}%
\pgfpathlineto{\pgfqpoint{0.580105in}{1.468764in}}%
\pgfpathlineto{\pgfqpoint{0.581179in}{1.469275in}}%
\pgfpathlineto{\pgfqpoint{0.584400in}{1.461782in}}%
\pgfpathlineto{\pgfqpoint{0.588695in}{1.465300in}}%
\pgfpathlineto{\pgfqpoint{0.589769in}{1.465568in}}%
\pgfpathlineto{\pgfqpoint{0.591917in}{1.463205in}}%
\pgfpathlineto{\pgfqpoint{0.596212in}{1.467905in}}%
\pgfpathlineto{\pgfqpoint{0.597286in}{1.464843in}}%
\pgfpathlineto{\pgfqpoint{0.598360in}{1.465676in}}%
\pgfpathlineto{\pgfqpoint{0.599434in}{1.460949in}}%
\pgfpathlineto{\pgfqpoint{0.602655in}{1.459660in}}%
\pgfpathlineto{\pgfqpoint{0.603729in}{1.458371in}}%
\pgfpathlineto{\pgfqpoint{0.604803in}{1.464628in}}%
\pgfpathlineto{\pgfqpoint{0.606951in}{1.464628in}}%
\pgfpathlineto{\pgfqpoint{0.611246in}{1.462088in}}%
\pgfpathlineto{\pgfqpoint{0.612320in}{1.460280in}}%
\pgfpathlineto{\pgfqpoint{0.614468in}{1.462376in}}%
\pgfpathlineto{\pgfqpoint{0.619837in}{1.462376in}}%
\pgfpathlineto{\pgfqpoint{0.620911in}{1.460011in}}%
\pgfpathlineto{\pgfqpoint{0.621984in}{1.460011in}}%
\pgfpathlineto{\pgfqpoint{0.625206in}{1.456891in}}%
\pgfpathlineto{\pgfqpoint{0.626280in}{1.456635in}}%
\pgfpathlineto{\pgfqpoint{0.627354in}{1.457786in}}%
\pgfpathlineto{\pgfqpoint{0.628427in}{1.454410in}}%
\pgfpathlineto{\pgfqpoint{0.635944in}{1.454410in}}%
\pgfpathlineto{\pgfqpoint{0.637018in}{1.451726in}}%
\pgfpathlineto{\pgfqpoint{0.641313in}{1.451849in}}%
\pgfpathlineto{\pgfqpoint{0.642387in}{1.448426in}}%
\pgfpathlineto{\pgfqpoint{0.643461in}{1.447046in}}%
\pgfpathlineto{\pgfqpoint{0.644535in}{1.450691in}}%
\pgfpathlineto{\pgfqpoint{0.647757in}{1.449903in}}%
\pgfpathlineto{\pgfqpoint{0.648830in}{1.450691in}}%
\pgfpathlineto{\pgfqpoint{0.652052in}{1.450691in}}%
\pgfpathlineto{\pgfqpoint{0.655273in}{1.448714in}}%
\pgfpathlineto{\pgfqpoint{0.656347in}{1.445955in}}%
\pgfpathlineto{\pgfqpoint{0.658495in}{1.448006in}}%
\pgfpathlineto{\pgfqpoint{0.711113in}{1.448006in}}%
\pgfpathlineto{\pgfqpoint{0.712187in}{1.444426in}}%
\pgfpathlineto{\pgfqpoint{0.715408in}{1.444950in}}%
\pgfpathlineto{\pgfqpoint{0.716482in}{1.446152in}}%
\pgfpathlineto{\pgfqpoint{0.718630in}{1.445723in}}%
\pgfpathlineto{\pgfqpoint{0.800242in}{1.445723in}}%
\pgfpathlineto{\pgfqpoint{0.801315in}{1.444537in}}%
\pgfpathlineto{\pgfqpoint{0.802389in}{1.444325in}}%
\pgfpathlineto{\pgfqpoint{0.813128in}{1.444302in}}%
\pgfpathlineto{\pgfqpoint{0.814201in}{1.440213in}}%
\pgfpathlineto{\pgfqpoint{0.815275in}{1.438815in}}%
\pgfpathlineto{\pgfqpoint{0.816349in}{1.438577in}}%
\pgfpathlineto{\pgfqpoint{0.817423in}{1.439646in}}%
\pgfpathlineto{\pgfqpoint{0.824940in}{1.440472in}}%
\pgfpathlineto{\pgfqpoint{0.828161in}{1.438367in}}%
\pgfpathlineto{\pgfqpoint{0.830309in}{1.434940in}}%
\pgfpathlineto{\pgfqpoint{0.832457in}{1.433363in}}%
\pgfpathlineto{\pgfqpoint{0.835678in}{1.434016in}}%
\pgfpathlineto{\pgfqpoint{0.836752in}{1.435655in}}%
\pgfpathlineto{\pgfqpoint{0.838900in}{1.433336in}}%
\pgfpathlineto{\pgfqpoint{0.844269in}{1.431761in}}%
\pgfpathlineto{\pgfqpoint{0.845343in}{1.432362in}}%
\pgfpathlineto{\pgfqpoint{0.847490in}{1.430209in}}%
\pgfpathlineto{\pgfqpoint{0.851786in}{1.429426in}}%
\pgfpathlineto{\pgfqpoint{0.855007in}{1.425985in}}%
\pgfpathlineto{\pgfqpoint{0.859303in}{1.426035in}}%
\pgfpathlineto{\pgfqpoint{0.860377in}{1.426971in}}%
\pgfpathlineto{\pgfqpoint{0.861450in}{1.429191in}}%
\pgfpathlineto{\pgfqpoint{0.862524in}{1.425788in}}%
\pgfpathlineto{\pgfqpoint{0.867893in}{1.426455in}}%
\pgfpathlineto{\pgfqpoint{0.870041in}{1.425431in}}%
\pgfpathlineto{\pgfqpoint{0.873263in}{1.426110in}}%
\pgfpathlineto{\pgfqpoint{0.875410in}{1.425491in}}%
\pgfpathlineto{\pgfqpoint{0.876484in}{1.425226in}}%
\pgfpathlineto{\pgfqpoint{0.877558in}{1.423674in}}%
\pgfpathlineto{\pgfqpoint{0.881853in}{1.423351in}}%
\pgfpathlineto{\pgfqpoint{0.884001in}{1.425583in}}%
\pgfpathlineto{\pgfqpoint{0.885075in}{1.424039in}}%
\pgfpathlineto{\pgfqpoint{0.888296in}{1.427005in}}%
\pgfpathlineto{\pgfqpoint{0.891518in}{1.423847in}}%
\pgfpathlineto{\pgfqpoint{0.895813in}{1.425833in}}%
\pgfpathlineto{\pgfqpoint{0.896887in}{1.423129in}}%
\pgfpathlineto{\pgfqpoint{0.897961in}{1.423049in}}%
\pgfpathlineto{\pgfqpoint{0.900109in}{1.421681in}}%
\pgfpathlineto{\pgfqpoint{0.906552in}{1.419850in}}%
\pgfpathlineto{\pgfqpoint{0.910847in}{1.420733in}}%
\pgfpathlineto{\pgfqpoint{0.912995in}{1.419843in}}%
\pgfpathlineto{\pgfqpoint{0.914068in}{1.421164in}}%
\pgfpathlineto{\pgfqpoint{0.915142in}{1.419613in}}%
\pgfpathlineto{\pgfqpoint{0.920511in}{1.420903in}}%
\pgfpathlineto{\pgfqpoint{0.921585in}{1.419806in}}%
\pgfpathlineto{\pgfqpoint{0.926955in}{1.420382in}}%
\pgfpathlineto{\pgfqpoint{0.929102in}{1.420152in}}%
\pgfpathlineto{\pgfqpoint{0.930176in}{1.420838in}}%
\pgfpathlineto{\pgfqpoint{0.934471in}{1.418453in}}%
\pgfpathlineto{\pgfqpoint{0.936619in}{1.416729in}}%
\pgfpathlineto{\pgfqpoint{0.937693in}{1.416887in}}%
\pgfpathlineto{\pgfqpoint{0.940914in}{1.419592in}}%
\pgfpathlineto{\pgfqpoint{0.941988in}{1.418290in}}%
\pgfpathlineto{\pgfqpoint{0.944136in}{1.422128in}}%
\pgfpathlineto{\pgfqpoint{0.945210in}{1.419855in}}%
\pgfpathlineto{\pgfqpoint{0.948431in}{1.419323in}}%
\pgfpathlineto{\pgfqpoint{0.950579in}{1.421453in}}%
\pgfpathlineto{\pgfqpoint{0.951653in}{1.421312in}}%
\pgfpathlineto{\pgfqpoint{0.952727in}{1.422590in}}%
\pgfpathlineto{\pgfqpoint{0.955948in}{1.421954in}}%
\pgfpathlineto{\pgfqpoint{0.958096in}{1.422728in}}%
\pgfpathlineto{\pgfqpoint{0.959170in}{1.421771in}}%
\pgfpathlineto{\pgfqpoint{0.960244in}{1.419934in}}%
\pgfpathlineto{\pgfqpoint{0.964539in}{1.418931in}}%
\pgfpathlineto{\pgfqpoint{0.967760in}{1.417179in}}%
\pgfpathlineto{\pgfqpoint{0.970982in}{1.417543in}}%
\pgfpathlineto{\pgfqpoint{0.973130in}{1.415795in}}%
\pgfpathlineto{\pgfqpoint{0.974203in}{1.416592in}}%
\pgfpathlineto{\pgfqpoint{0.975277in}{1.413866in}}%
\pgfpathlineto{\pgfqpoint{0.979573in}{1.413630in}}%
\pgfpathlineto{\pgfqpoint{0.981720in}{1.416098in}}%
\pgfpathlineto{\pgfqpoint{0.982794in}{1.416656in}}%
\pgfpathlineto{\pgfqpoint{0.987089in}{1.415485in}}%
\pgfpathlineto{\pgfqpoint{0.988163in}{1.416534in}}%
\pgfpathlineto{\pgfqpoint{0.989237in}{1.415481in}}%
\pgfpathlineto{\pgfqpoint{0.990311in}{1.416785in}}%
\pgfpathlineto{\pgfqpoint{0.993533in}{1.416364in}}%
\pgfpathlineto{\pgfqpoint{0.996754in}{1.419505in}}%
\pgfpathlineto{\pgfqpoint{0.997828in}{1.417215in}}%
\pgfpathlineto{\pgfqpoint{1.002123in}{1.418257in}}%
\pgfpathlineto{\pgfqpoint{1.003197in}{1.419535in}}%
\pgfpathlineto{\pgfqpoint{1.004271in}{1.417425in}}%
\pgfpathlineto{\pgfqpoint{1.005345in}{1.417762in}}%
\pgfpathlineto{\pgfqpoint{1.008566in}{1.416538in}}%
\pgfpathlineto{\pgfqpoint{1.009640in}{1.415050in}}%
\pgfpathlineto{\pgfqpoint{1.011788in}{1.419808in}}%
\pgfpathlineto{\pgfqpoint{1.017157in}{1.420553in}}%
\pgfpathlineto{\pgfqpoint{1.019305in}{1.418252in}}%
\pgfpathlineto{\pgfqpoint{1.020378in}{1.418997in}}%
\pgfpathlineto{\pgfqpoint{1.023600in}{1.416646in}}%
\pgfpathlineto{\pgfqpoint{1.024674in}{1.417820in}}%
\pgfpathlineto{\pgfqpoint{1.027895in}{1.414375in}}%
\pgfpathlineto{\pgfqpoint{1.031117in}{1.413831in}}%
\pgfpathlineto{\pgfqpoint{1.032191in}{1.412586in}}%
\pgfpathlineto{\pgfqpoint{1.033265in}{1.412857in}}%
\pgfpathlineto{\pgfqpoint{1.034338in}{1.410627in}}%
\pgfpathlineto{\pgfqpoint{1.040781in}{1.409297in}}%
\pgfpathlineto{\pgfqpoint{1.041855in}{1.408773in}}%
\pgfpathlineto{\pgfqpoint{1.042929in}{1.407520in}}%
\pgfpathlineto{\pgfqpoint{1.046151in}{1.407916in}}%
\pgfpathlineto{\pgfqpoint{1.047224in}{1.404676in}}%
\pgfpathlineto{\pgfqpoint{1.050446in}{1.404828in}}%
\pgfpathlineto{\pgfqpoint{1.055815in}{1.404216in}}%
\pgfpathlineto{\pgfqpoint{1.056889in}{1.402597in}}%
\pgfpathlineto{\pgfqpoint{1.057963in}{1.402021in}}%
\pgfpathlineto{\pgfqpoint{1.061184in}{1.403105in}}%
\pgfpathlineto{\pgfqpoint{1.062258in}{1.404531in}}%
\pgfpathlineto{\pgfqpoint{1.064406in}{1.403456in}}%
\pgfpathlineto{\pgfqpoint{1.065480in}{1.404100in}}%
\pgfpathlineto{\pgfqpoint{1.068701in}{1.404181in}}%
\pgfpathlineto{\pgfqpoint{1.069775in}{1.402911in}}%
\pgfpathlineto{\pgfqpoint{1.070849in}{1.404094in}}%
\pgfpathlineto{\pgfqpoint{1.071923in}{1.406189in}}%
\pgfpathlineto{\pgfqpoint{1.072997in}{1.406092in}}%
\pgfpathlineto{\pgfqpoint{1.077292in}{1.406819in}}%
\pgfpathlineto{\pgfqpoint{1.078366in}{1.407553in}}%
\pgfpathlineto{\pgfqpoint{1.079440in}{1.406201in}}%
\pgfpathlineto{\pgfqpoint{1.083735in}{1.406892in}}%
\pgfpathlineto{\pgfqpoint{1.084809in}{1.409406in}}%
\pgfpathlineto{\pgfqpoint{1.088030in}{1.409235in}}%
\pgfpathlineto{\pgfqpoint{1.094473in}{1.405303in}}%
\pgfpathlineto{\pgfqpoint{1.095547in}{1.405768in}}%
\pgfpathlineto{\pgfqpoint{1.098769in}{1.404469in}}%
\pgfpathlineto{\pgfqpoint{1.100916in}{1.401095in}}%
\pgfpathlineto{\pgfqpoint{1.103064in}{1.400566in}}%
\pgfpathlineto{\pgfqpoint{1.108433in}{1.398117in}}%
\pgfpathlineto{\pgfqpoint{1.109507in}{1.397223in}}%
\pgfpathlineto{\pgfqpoint{1.110581in}{1.399367in}}%
\pgfpathlineto{\pgfqpoint{1.113802in}{1.399516in}}%
\pgfpathlineto{\pgfqpoint{1.114876in}{1.398982in}}%
\pgfpathlineto{\pgfqpoint{1.115950in}{1.399659in}}%
\pgfpathlineto{\pgfqpoint{1.117024in}{1.399284in}}%
\pgfpathlineto{\pgfqpoint{1.118098in}{1.399560in}}%
\pgfpathlineto{\pgfqpoint{1.122393in}{1.401423in}}%
\pgfpathlineto{\pgfqpoint{1.124541in}{1.405031in}}%
\pgfpathlineto{\pgfqpoint{1.125615in}{1.404436in}}%
\pgfpathlineto{\pgfqpoint{1.128836in}{1.404683in}}%
\pgfpathlineto{\pgfqpoint{1.129910in}{1.405983in}}%
\pgfpathlineto{\pgfqpoint{1.130984in}{1.405898in}}%
\pgfpathlineto{\pgfqpoint{1.132058in}{1.402671in}}%
\pgfpathlineto{\pgfqpoint{1.133132in}{1.401589in}}%
\pgfpathlineto{\pgfqpoint{1.136353in}{1.401690in}}%
\pgfpathlineto{\pgfqpoint{1.137427in}{1.402763in}}%
\pgfpathlineto{\pgfqpoint{1.138501in}{1.402135in}}%
\pgfpathlineto{\pgfqpoint{1.139575in}{1.400580in}}%
\pgfpathlineto{\pgfqpoint{1.140648in}{1.400842in}}%
\pgfpathlineto{\pgfqpoint{1.143870in}{1.400963in}}%
\pgfpathlineto{\pgfqpoint{1.144944in}{1.402385in}}%
\pgfpathlineto{\pgfqpoint{1.147091in}{1.401470in}}%
\pgfpathlineto{\pgfqpoint{1.148165in}{1.401025in}}%
\pgfpathlineto{\pgfqpoint{1.151387in}{1.402493in}}%
\pgfpathlineto{\pgfqpoint{1.152461in}{1.402073in}}%
\pgfpathlineto{\pgfqpoint{1.153534in}{1.402535in}}%
\pgfpathlineto{\pgfqpoint{1.154608in}{1.402239in}}%
\pgfpathlineto{\pgfqpoint{1.155682in}{1.400996in}}%
\pgfpathlineto{\pgfqpoint{1.161051in}{1.399882in}}%
\pgfpathlineto{\pgfqpoint{1.162125in}{1.401200in}}%
\pgfpathlineto{\pgfqpoint{1.163199in}{1.399969in}}%
\pgfpathlineto{\pgfqpoint{1.168568in}{1.400428in}}%
\pgfpathlineto{\pgfqpoint{1.169642in}{1.399249in}}%
\pgfpathlineto{\pgfqpoint{1.170716in}{1.399345in}}%
\pgfpathlineto{\pgfqpoint{1.175011in}{1.398874in}}%
\pgfpathlineto{\pgfqpoint{1.176085in}{1.399118in}}%
\pgfpathlineto{\pgfqpoint{1.178233in}{1.397847in}}%
\pgfpathlineto{\pgfqpoint{1.182528in}{1.397184in}}%
\pgfpathlineto{\pgfqpoint{1.183602in}{1.396672in}}%
\pgfpathlineto{\pgfqpoint{1.192193in}{1.398887in}}%
\pgfpathlineto{\pgfqpoint{1.193266in}{1.396968in}}%
\pgfpathlineto{\pgfqpoint{1.197562in}{1.397090in}}%
\pgfpathlineto{\pgfqpoint{1.198636in}{1.399167in}}%
\pgfpathlineto{\pgfqpoint{1.200783in}{1.400401in}}%
\pgfpathlineto{\pgfqpoint{1.204005in}{1.399265in}}%
\pgfpathlineto{\pgfqpoint{1.205079in}{1.400120in}}%
\pgfpathlineto{\pgfqpoint{1.206153in}{1.398043in}}%
\pgfpathlineto{\pgfqpoint{1.207226in}{1.398325in}}%
\pgfpathlineto{\pgfqpoint{1.208300in}{1.397254in}}%
\pgfpathlineto{\pgfqpoint{1.212596in}{1.396526in}}%
\pgfpathlineto{\pgfqpoint{1.215817in}{1.395354in}}%
\pgfpathlineto{\pgfqpoint{1.219039in}{1.395413in}}%
\pgfpathlineto{\pgfqpoint{1.220112in}{1.394492in}}%
\pgfpathlineto{\pgfqpoint{1.222260in}{1.395600in}}%
\pgfpathlineto{\pgfqpoint{1.223334in}{1.395242in}}%
\pgfpathlineto{\pgfqpoint{1.226555in}{1.395340in}}%
\pgfpathlineto{\pgfqpoint{1.228703in}{1.394600in}}%
\pgfpathlineto{\pgfqpoint{1.230851in}{1.394445in}}%
\pgfpathlineto{\pgfqpoint{1.235146in}{1.395467in}}%
\pgfpathlineto{\pgfqpoint{1.236220in}{1.394217in}}%
\pgfpathlineto{\pgfqpoint{1.242663in}{1.393445in}}%
\pgfpathlineto{\pgfqpoint{1.243737in}{1.392504in}}%
\pgfpathlineto{\pgfqpoint{1.244811in}{1.393548in}}%
\pgfpathlineto{\pgfqpoint{1.245885in}{1.396085in}}%
\pgfpathlineto{\pgfqpoint{1.250180in}{1.394267in}}%
\pgfpathlineto{\pgfqpoint{1.251254in}{1.394771in}}%
\pgfpathlineto{\pgfqpoint{1.252328in}{1.393567in}}%
\pgfpathlineto{\pgfqpoint{1.253401in}{1.394111in}}%
\pgfpathlineto{\pgfqpoint{1.258771in}{1.399433in}}%
\pgfpathlineto{\pgfqpoint{1.260918in}{1.396807in}}%
\pgfpathlineto{\pgfqpoint{1.264140in}{1.396030in}}%
\pgfpathlineto{\pgfqpoint{1.265214in}{1.394832in}}%
\pgfpathlineto{\pgfqpoint{1.268435in}{1.393547in}}%
\pgfpathlineto{\pgfqpoint{1.273804in}{1.393403in}}%
\pgfpathlineto{\pgfqpoint{1.275952in}{1.392292in}}%
\pgfpathlineto{\pgfqpoint{1.279174in}{1.391379in}}%
\pgfpathlineto{\pgfqpoint{1.280247in}{1.391863in}}%
\pgfpathlineto{\pgfqpoint{1.290986in}{1.389925in}}%
\pgfpathlineto{\pgfqpoint{1.294207in}{1.390351in}}%
\pgfpathlineto{\pgfqpoint{1.295281in}{1.392022in}}%
\pgfpathlineto{\pgfqpoint{1.296355in}{1.391883in}}%
\pgfpathlineto{\pgfqpoint{1.297429in}{1.394249in}}%
\pgfpathlineto{\pgfqpoint{1.298503in}{1.394502in}}%
\pgfpathlineto{\pgfqpoint{1.302798in}{1.392721in}}%
\pgfpathlineto{\pgfqpoint{1.304946in}{1.393477in}}%
\pgfpathlineto{\pgfqpoint{1.306020in}{1.392807in}}%
\pgfpathlineto{\pgfqpoint{1.309241in}{1.393487in}}%
\pgfpathlineto{\pgfqpoint{1.310315in}{1.392262in}}%
\pgfpathlineto{\pgfqpoint{1.312463in}{1.393416in}}%
\pgfpathlineto{\pgfqpoint{1.313536in}{1.392575in}}%
\pgfpathlineto{\pgfqpoint{1.317832in}{1.390033in}}%
\pgfpathlineto{\pgfqpoint{1.318906in}{1.388511in}}%
\pgfpathlineto{\pgfqpoint{1.319979in}{1.388589in}}%
\pgfpathlineto{\pgfqpoint{1.321053in}{1.389796in}}%
\pgfpathlineto{\pgfqpoint{1.325349in}{1.391676in}}%
\pgfpathlineto{\pgfqpoint{1.326422in}{1.390652in}}%
\pgfpathlineto{\pgfqpoint{1.327496in}{1.392619in}}%
\pgfpathlineto{\pgfqpoint{1.328570in}{1.393137in}}%
\pgfpathlineto{\pgfqpoint{1.332865in}{1.391551in}}%
\pgfpathlineto{\pgfqpoint{1.333939in}{1.389747in}}%
\pgfpathlineto{\pgfqpoint{1.335013in}{1.389357in}}%
\pgfpathlineto{\pgfqpoint{1.339309in}{1.389568in}}%
\pgfpathlineto{\pgfqpoint{1.341456in}{1.388523in}}%
\pgfpathlineto{\pgfqpoint{1.342530in}{1.389032in}}%
\pgfpathlineto{\pgfqpoint{1.343604in}{1.390392in}}%
\pgfpathlineto{\pgfqpoint{1.348973in}{1.389506in}}%
\pgfpathlineto{\pgfqpoint{1.350047in}{1.390696in}}%
\pgfpathlineto{\pgfqpoint{1.351121in}{1.390913in}}%
\pgfpathlineto{\pgfqpoint{1.354342in}{1.390677in}}%
\pgfpathlineto{\pgfqpoint{1.355416in}{1.391319in}}%
\pgfpathlineto{\pgfqpoint{1.357564in}{1.390504in}}%
\pgfpathlineto{\pgfqpoint{1.358638in}{1.390531in}}%
\pgfpathlineto{\pgfqpoint{1.362933in}{1.389145in}}%
\pgfpathlineto{\pgfqpoint{1.365081in}{1.391062in}}%
\pgfpathlineto{\pgfqpoint{1.366154in}{1.391965in}}%
\pgfpathlineto{\pgfqpoint{1.369376in}{1.391602in}}%
\pgfpathlineto{\pgfqpoint{1.370450in}{1.393162in}}%
\pgfpathlineto{\pgfqpoint{1.371524in}{1.391702in}}%
\pgfpathlineto{\pgfqpoint{1.387631in}{1.389033in}}%
\pgfpathlineto{\pgfqpoint{1.388705in}{1.388465in}}%
\pgfpathlineto{\pgfqpoint{1.391927in}{1.387558in}}%
\pgfpathlineto{\pgfqpoint{1.396222in}{1.390066in}}%
\pgfpathlineto{\pgfqpoint{1.402665in}{1.389451in}}%
\pgfpathlineto{\pgfqpoint{1.403739in}{1.389125in}}%
\pgfpathlineto{\pgfqpoint{1.406960in}{1.389791in}}%
\pgfpathlineto{\pgfqpoint{1.408034in}{1.390929in}}%
\pgfpathlineto{\pgfqpoint{1.409108in}{1.390582in}}%
\pgfpathlineto{\pgfqpoint{1.410182in}{1.390872in}}%
\pgfpathlineto{\pgfqpoint{1.411256in}{1.390207in}}%
\pgfpathlineto{\pgfqpoint{1.414477in}{1.391105in}}%
\pgfpathlineto{\pgfqpoint{1.415551in}{1.390711in}}%
\pgfpathlineto{\pgfqpoint{1.416625in}{1.391346in}}%
\pgfpathlineto{\pgfqpoint{1.417699in}{1.391034in}}%
\pgfpathlineto{\pgfqpoint{1.421994in}{1.391381in}}%
\pgfpathlineto{\pgfqpoint{1.424142in}{1.392406in}}%
\pgfpathlineto{\pgfqpoint{1.425216in}{1.392678in}}%
\pgfpathlineto{\pgfqpoint{1.426289in}{1.392158in}}%
\pgfpathlineto{\pgfqpoint{1.430585in}{1.391577in}}%
\pgfpathlineto{\pgfqpoint{1.431659in}{1.391966in}}%
\pgfpathlineto{\pgfqpoint{1.432732in}{1.393280in}}%
\pgfpathlineto{\pgfqpoint{1.433806in}{1.392599in}}%
\pgfpathlineto{\pgfqpoint{1.437028in}{1.393091in}}%
\pgfpathlineto{\pgfqpoint{1.439175in}{1.395867in}}%
\pgfpathlineto{\pgfqpoint{1.441323in}{1.396627in}}%
\pgfpathlineto{\pgfqpoint{1.461726in}{1.396524in}}%
\pgfpathlineto{\pgfqpoint{1.468169in}{1.396092in}}%
\pgfpathlineto{\pgfqpoint{1.470317in}{1.394743in}}%
\pgfpathlineto{\pgfqpoint{1.471391in}{1.395328in}}%
\pgfpathlineto{\pgfqpoint{1.475686in}{1.395076in}}%
\pgfpathlineto{\pgfqpoint{1.477834in}{1.395575in}}%
\pgfpathlineto{\pgfqpoint{1.478908in}{1.395575in}}%
\pgfpathlineto{\pgfqpoint{1.486424in}{1.394457in}}%
\pgfpathlineto{\pgfqpoint{1.568036in}{1.394457in}}%
\pgfpathlineto{\pgfqpoint{1.569110in}{1.393145in}}%
\pgfpathlineto{\pgfqpoint{1.572331in}{1.393145in}}%
\pgfpathlineto{\pgfqpoint{1.573405in}{1.390584in}}%
\pgfpathlineto{\pgfqpoint{1.574479in}{1.391867in}}%
\pgfpathlineto{\pgfqpoint{1.580922in}{1.392279in}}%
\pgfpathlineto{\pgfqpoint{1.581996in}{1.391151in}}%
\pgfpathlineto{\pgfqpoint{1.584144in}{1.390897in}}%
\pgfpathlineto{\pgfqpoint{1.587365in}{1.389427in}}%
\pgfpathlineto{\pgfqpoint{1.588439in}{1.387857in}}%
\pgfpathlineto{\pgfqpoint{1.589513in}{1.389023in}}%
\pgfpathlineto{\pgfqpoint{1.590587in}{1.388591in}}%
\pgfpathlineto{\pgfqpoint{1.591661in}{1.390097in}}%
\pgfpathlineto{\pgfqpoint{1.594882in}{1.390291in}}%
\pgfpathlineto{\pgfqpoint{1.597030in}{1.388657in}}%
\pgfpathlineto{\pgfqpoint{1.598104in}{1.386219in}}%
\pgfpathlineto{\pgfqpoint{1.599177in}{1.387252in}}%
\pgfpathlineto{\pgfqpoint{1.603473in}{1.386006in}}%
\pgfpathlineto{\pgfqpoint{1.604547in}{1.386233in}}%
\pgfpathlineto{\pgfqpoint{1.606694in}{1.385979in}}%
\pgfpathlineto{\pgfqpoint{1.610990in}{1.386765in}}%
\pgfpathlineto{\pgfqpoint{1.612063in}{1.387683in}}%
\pgfpathlineto{\pgfqpoint{1.614211in}{1.387656in}}%
\pgfpathlineto{\pgfqpoint{1.617433in}{1.389151in}}%
\pgfpathlineto{\pgfqpoint{1.618507in}{1.390444in}}%
\pgfpathlineto{\pgfqpoint{1.620654in}{1.387875in}}%
\pgfpathlineto{\pgfqpoint{1.621728in}{1.388376in}}%
\pgfpathlineto{\pgfqpoint{1.626023in}{1.388117in}}%
\pgfpathlineto{\pgfqpoint{1.628171in}{1.388763in}}%
\pgfpathlineto{\pgfqpoint{1.629245in}{1.386655in}}%
\pgfpathlineto{\pgfqpoint{1.633540in}{1.385602in}}%
\pgfpathlineto{\pgfqpoint{1.635688in}{1.383446in}}%
\pgfpathlineto{\pgfqpoint{1.639983in}{1.384663in}}%
\pgfpathlineto{\pgfqpoint{1.641057in}{1.384368in}}%
\pgfpathlineto{\pgfqpoint{1.642131in}{1.385994in}}%
\pgfpathlineto{\pgfqpoint{1.643205in}{1.386349in}}%
\pgfpathlineto{\pgfqpoint{1.644279in}{1.387617in}}%
\pgfpathlineto{\pgfqpoint{1.647500in}{1.386207in}}%
\pgfpathlineto{\pgfqpoint{1.648574in}{1.384458in}}%
\pgfpathlineto{\pgfqpoint{1.649648in}{1.385258in}}%
\pgfpathlineto{\pgfqpoint{1.650722in}{1.383537in}}%
\pgfpathlineto{\pgfqpoint{1.655017in}{1.384099in}}%
\pgfpathlineto{\pgfqpoint{1.658239in}{1.383443in}}%
\pgfpathlineto{\pgfqpoint{1.659312in}{1.382255in}}%
\pgfpathlineto{\pgfqpoint{1.664682in}{1.381736in}}%
\pgfpathlineto{\pgfqpoint{1.666829in}{1.380292in}}%
\pgfpathlineto{\pgfqpoint{1.673272in}{1.381254in}}%
\pgfpathlineto{\pgfqpoint{1.674346in}{1.381769in}}%
\pgfpathlineto{\pgfqpoint{1.677568in}{1.380887in}}%
\pgfpathlineto{\pgfqpoint{1.679715in}{1.382347in}}%
\pgfpathlineto{\pgfqpoint{1.680789in}{1.382526in}}%
\pgfpathlineto{\pgfqpoint{1.681863in}{1.383479in}}%
\pgfpathlineto{\pgfqpoint{1.685085in}{1.381817in}}%
\pgfpathlineto{\pgfqpoint{1.686158in}{1.384814in}}%
\pgfpathlineto{\pgfqpoint{1.687232in}{1.384121in}}%
\pgfpathlineto{\pgfqpoint{1.688306in}{1.382001in}}%
\pgfpathlineto{\pgfqpoint{1.689380in}{1.383741in}}%
\pgfpathlineto{\pgfqpoint{1.694749in}{1.382607in}}%
\pgfpathlineto{\pgfqpoint{1.695823in}{1.383259in}}%
\pgfpathlineto{\pgfqpoint{1.696897in}{1.383203in}}%
\pgfpathlineto{\pgfqpoint{1.701192in}{1.383591in}}%
\pgfpathlineto{\pgfqpoint{1.702266in}{1.385366in}}%
\pgfpathlineto{\pgfqpoint{1.703340in}{1.385676in}}%
\pgfpathlineto{\pgfqpoint{1.707635in}{1.383589in}}%
\pgfpathlineto{\pgfqpoint{1.709783in}{1.385713in}}%
\pgfpathlineto{\pgfqpoint{1.710857in}{1.384801in}}%
\pgfpathlineto{\pgfqpoint{1.719447in}{1.383813in}}%
\pgfpathlineto{\pgfqpoint{1.723743in}{1.384367in}}%
\pgfpathlineto{\pgfqpoint{1.725890in}{1.384522in}}%
\pgfpathlineto{\pgfqpoint{1.726964in}{1.386272in}}%
\pgfpathlineto{\pgfqpoint{1.733407in}{1.384636in}}%
\pgfpathlineto{\pgfqpoint{1.734481in}{1.385453in}}%
\pgfpathlineto{\pgfqpoint{1.739850in}{1.386040in}}%
\pgfpathlineto{\pgfqpoint{1.740924in}{1.387283in}}%
\pgfpathlineto{\pgfqpoint{1.741998in}{1.385951in}}%
\pgfpathlineto{\pgfqpoint{1.745219in}{1.385186in}}%
\pgfpathlineto{\pgfqpoint{1.746293in}{1.386214in}}%
\pgfpathlineto{\pgfqpoint{1.747367in}{1.386214in}}%
\pgfpathlineto{\pgfqpoint{1.748441in}{1.385470in}}%
\pgfpathlineto{\pgfqpoint{1.749515in}{1.383670in}}%
\pgfpathlineto{\pgfqpoint{1.753810in}{1.384448in}}%
\pgfpathlineto{\pgfqpoint{1.754884in}{1.383963in}}%
\pgfpathlineto{\pgfqpoint{1.755958in}{1.382662in}}%
\pgfpathlineto{\pgfqpoint{1.757032in}{1.383128in}}%
\pgfpathlineto{\pgfqpoint{1.764549in}{1.383028in}}%
\pgfpathlineto{\pgfqpoint{1.768844in}{1.384285in}}%
\pgfpathlineto{\pgfqpoint{1.770992in}{1.383679in}}%
\pgfpathlineto{\pgfqpoint{1.772065in}{1.384059in}}%
\pgfpathlineto{\pgfqpoint{1.776361in}{1.383936in}}%
\pgfpathlineto{\pgfqpoint{1.777435in}{1.383074in}}%
\pgfpathlineto{\pgfqpoint{1.778508in}{1.384062in}}%
\pgfpathlineto{\pgfqpoint{1.783878in}{1.384682in}}%
\pgfpathlineto{\pgfqpoint{1.786025in}{1.383181in}}%
\pgfpathlineto{\pgfqpoint{1.787099in}{1.383700in}}%
\pgfpathlineto{\pgfqpoint{1.790321in}{1.385884in}}%
\pgfpathlineto{\pgfqpoint{1.793542in}{1.384902in}}%
\pgfpathlineto{\pgfqpoint{1.794616in}{1.385660in}}%
\pgfpathlineto{\pgfqpoint{1.798911in}{1.385397in}}%
\pgfpathlineto{\pgfqpoint{1.801059in}{1.386881in}}%
\pgfpathlineto{\pgfqpoint{1.805354in}{1.386707in}}%
\pgfpathlineto{\pgfqpoint{2.245628in}{1.386707in}}%
\pgfpathlineto{\pgfqpoint{2.253145in}{1.385516in}}%
\pgfpathlineto{\pgfqpoint{2.361603in}{1.385236in}}%
\pgfpathlineto{\pgfqpoint{2.363750in}{1.384918in}}%
\pgfpathlineto{\pgfqpoint{2.370193in}{1.384918in}}%
\pgfpathlineto{\pgfqpoint{2.371267in}{1.383088in}}%
\pgfpathlineto{\pgfqpoint{2.372341in}{1.384065in}}%
\pgfpathlineto{\pgfqpoint{2.373415in}{1.382980in}}%
\pgfpathlineto{\pgfqpoint{2.377710in}{1.382575in}}%
\pgfpathlineto{\pgfqpoint{2.378784in}{1.383377in}}%
\pgfpathlineto{\pgfqpoint{2.380932in}{1.383065in}}%
\pgfpathlineto{\pgfqpoint{2.384153in}{1.381258in}}%
\pgfpathlineto{\pgfqpoint{2.388449in}{1.381386in}}%
\pgfpathlineto{\pgfqpoint{2.392744in}{1.381137in}}%
\pgfpathlineto{\pgfqpoint{2.393818in}{1.381736in}}%
\pgfpathlineto{\pgfqpoint{2.394892in}{1.381460in}}%
\pgfpathlineto{\pgfqpoint{2.395966in}{1.382161in}}%
\pgfpathlineto{\pgfqpoint{2.402409in}{1.380843in}}%
\pgfpathlineto{\pgfqpoint{2.403482in}{1.379946in}}%
\pgfpathlineto{\pgfqpoint{2.410999in}{1.381673in}}%
\pgfpathlineto{\pgfqpoint{2.416369in}{1.381517in}}%
\pgfpathlineto{\pgfqpoint{2.418516in}{1.381235in}}%
\pgfpathlineto{\pgfqpoint{2.421738in}{1.381572in}}%
\pgfpathlineto{\pgfqpoint{2.422812in}{1.380610in}}%
\pgfpathlineto{\pgfqpoint{2.426033in}{1.382048in}}%
\pgfpathlineto{\pgfqpoint{2.430328in}{1.382061in}}%
\pgfpathlineto{\pgfqpoint{2.432476in}{1.382061in}}%
\pgfpathlineto{\pgfqpoint{2.433550in}{1.381059in}}%
\pgfpathlineto{\pgfqpoint{2.437845in}{1.379855in}}%
\pgfpathlineto{\pgfqpoint{2.439993in}{1.380714in}}%
\pgfpathlineto{\pgfqpoint{2.441067in}{1.379992in}}%
\pgfpathlineto{\pgfqpoint{2.448584in}{1.378690in}}%
\pgfpathlineto{\pgfqpoint{2.462544in}{1.378284in}}%
\pgfpathlineto{\pgfqpoint{2.466839in}{1.379020in}}%
\pgfpathlineto{\pgfqpoint{2.467913in}{1.378573in}}%
\pgfpathlineto{\pgfqpoint{2.470060in}{1.378823in}}%
\pgfpathlineto{\pgfqpoint{2.474356in}{1.378548in}}%
\pgfpathlineto{\pgfqpoint{2.475430in}{1.378890in}}%
\pgfpathlineto{\pgfqpoint{2.476504in}{1.377985in}}%
\pgfpathlineto{\pgfqpoint{2.477577in}{1.378510in}}%
\pgfpathlineto{\pgfqpoint{2.478651in}{1.377712in}}%
\pgfpathlineto{\pgfqpoint{2.481873in}{1.377667in}}%
\pgfpathlineto{\pgfqpoint{2.482947in}{1.378620in}}%
\pgfpathlineto{\pgfqpoint{2.486168in}{1.378880in}}%
\pgfpathlineto{\pgfqpoint{2.489390in}{1.378857in}}%
\pgfpathlineto{\pgfqpoint{2.490463in}{1.378197in}}%
\pgfpathlineto{\pgfqpoint{2.491537in}{1.378680in}}%
\pgfpathlineto{\pgfqpoint{2.493685in}{1.378572in}}%
\pgfpathlineto{\pgfqpoint{2.496906in}{1.378778in}}%
\pgfpathlineto{\pgfqpoint{2.497980in}{1.378012in}}%
\pgfpathlineto{\pgfqpoint{2.500128in}{1.378418in}}%
\pgfpathlineto{\pgfqpoint{2.501202in}{1.378008in}}%
\pgfpathlineto{\pgfqpoint{2.511940in}{1.377791in}}%
\pgfpathlineto{\pgfqpoint{2.513014in}{1.377590in}}%
\pgfpathlineto{\pgfqpoint{2.514088in}{1.378036in}}%
\pgfpathlineto{\pgfqpoint{2.516236in}{1.376583in}}%
\pgfpathlineto{\pgfqpoint{2.520531in}{1.375300in}}%
\pgfpathlineto{\pgfqpoint{2.522679in}{1.374046in}}%
\pgfpathlineto{\pgfqpoint{2.537712in}{1.372457in}}%
\pgfpathlineto{\pgfqpoint{2.538786in}{1.372737in}}%
\pgfpathlineto{\pgfqpoint{2.544155in}{1.373098in}}%
\pgfpathlineto{\pgfqpoint{2.545229in}{1.373231in}}%
\pgfpathlineto{\pgfqpoint{2.546303in}{1.372100in}}%
\pgfpathlineto{\pgfqpoint{2.553820in}{1.371680in}}%
\pgfpathlineto{\pgfqpoint{2.560263in}{1.371709in}}%
\pgfpathlineto{\pgfqpoint{2.561337in}{1.371511in}}%
\pgfpathlineto{\pgfqpoint{2.567780in}{1.372839in}}%
\pgfpathlineto{\pgfqpoint{2.568854in}{1.372404in}}%
\pgfpathlineto{\pgfqpoint{2.574223in}{1.372787in}}%
\pgfpathlineto{\pgfqpoint{2.575297in}{1.372327in}}%
\pgfpathlineto{\pgfqpoint{2.576370in}{1.372528in}}%
\pgfpathlineto{\pgfqpoint{2.580666in}{1.371850in}}%
\pgfpathlineto{\pgfqpoint{2.583887in}{1.371422in}}%
\pgfpathlineto{\pgfqpoint{2.591404in}{1.371476in}}%
\pgfpathlineto{\pgfqpoint{2.598921in}{1.371169in}}%
\pgfpathlineto{\pgfqpoint{2.606438in}{1.370689in}}%
\pgfpathlineto{\pgfqpoint{2.617176in}{1.370841in}}%
\pgfpathlineto{\pgfqpoint{2.618250in}{1.372452in}}%
\pgfpathlineto{\pgfqpoint{2.620398in}{1.373530in}}%
\pgfpathlineto{\pgfqpoint{2.626841in}{1.372470in}}%
\pgfpathlineto{\pgfqpoint{2.628989in}{1.371768in}}%
\pgfpathlineto{\pgfqpoint{2.634358in}{1.374420in}}%
\pgfpathlineto{\pgfqpoint{2.636505in}{1.374626in}}%
\pgfpathlineto{\pgfqpoint{2.639727in}{1.374453in}}%
\pgfpathlineto{\pgfqpoint{2.640801in}{1.375231in}}%
\pgfpathlineto{\pgfqpoint{2.641875in}{1.373390in}}%
\pgfpathlineto{\pgfqpoint{2.644022in}{1.375018in}}%
\pgfpathlineto{\pgfqpoint{2.648318in}{1.374877in}}%
\pgfpathlineto{\pgfqpoint{2.649392in}{1.373985in}}%
\pgfpathlineto{\pgfqpoint{2.650465in}{1.375109in}}%
\pgfpathlineto{\pgfqpoint{2.651539in}{1.375263in}}%
\pgfpathlineto{\pgfqpoint{2.654761in}{1.375087in}}%
\pgfpathlineto{\pgfqpoint{2.655835in}{1.372829in}}%
\pgfpathlineto{\pgfqpoint{2.657982in}{1.372189in}}%
\pgfpathlineto{\pgfqpoint{2.663351in}{1.377602in}}%
\pgfpathlineto{\pgfqpoint{2.671942in}{1.377602in}}%
\pgfpathlineto{\pgfqpoint{2.673016in}{1.375535in}}%
\pgfpathlineto{\pgfqpoint{2.678385in}{1.375179in}}%
\pgfpathlineto{\pgfqpoint{2.681607in}{1.374260in}}%
\pgfpathlineto{\pgfqpoint{2.685902in}{1.374451in}}%
\pgfpathlineto{\pgfqpoint{2.686976in}{1.372808in}}%
\pgfpathlineto{\pgfqpoint{2.688050in}{1.373665in}}%
\pgfpathlineto{\pgfqpoint{2.700936in}{1.372183in}}%
\pgfpathlineto{\pgfqpoint{2.702010in}{1.372414in}}%
\pgfpathlineto{\pgfqpoint{2.703083in}{1.371738in}}%
\pgfpathlineto{\pgfqpoint{2.707379in}{1.371620in}}%
\pgfpathlineto{\pgfqpoint{2.714896in}{1.370458in}}%
\pgfpathlineto{\pgfqpoint{2.717043in}{1.371612in}}%
\pgfpathlineto{\pgfqpoint{2.718117in}{1.371027in}}%
\pgfpathlineto{\pgfqpoint{2.719191in}{1.371551in}}%
\pgfpathlineto{\pgfqpoint{2.726708in}{1.370320in}}%
\pgfpathlineto{\pgfqpoint{2.732077in}{1.370790in}}%
\pgfpathlineto{\pgfqpoint{2.733151in}{1.371951in}}%
\pgfpathlineto{\pgfqpoint{2.737446in}{1.372767in}}%
\pgfpathlineto{\pgfqpoint{2.738520in}{1.371662in}}%
\pgfpathlineto{\pgfqpoint{2.739594in}{1.372423in}}%
\pgfpathlineto{\pgfqpoint{2.740668in}{1.371772in}}%
\pgfpathlineto{\pgfqpoint{2.741742in}{1.372627in}}%
\pgfpathlineto{\pgfqpoint{2.749258in}{1.372384in}}%
\pgfpathlineto{\pgfqpoint{2.753554in}{1.371874in}}%
\pgfpathlineto{\pgfqpoint{2.754628in}{1.371586in}}%
\pgfpathlineto{\pgfqpoint{2.755702in}{1.369534in}}%
\pgfpathlineto{\pgfqpoint{2.756775in}{1.370320in}}%
\pgfpathlineto{\pgfqpoint{2.761071in}{1.370220in}}%
\pgfpathlineto{\pgfqpoint{2.764292in}{1.368718in}}%
\pgfpathlineto{\pgfqpoint{2.770735in}{1.368169in}}%
\pgfpathlineto{\pgfqpoint{2.771809in}{1.366798in}}%
\pgfpathlineto{\pgfqpoint{2.785769in}{1.365746in}}%
\pgfpathlineto{\pgfqpoint{2.786843in}{1.366039in}}%
\pgfpathlineto{\pgfqpoint{2.786843in}{1.366039in}}%
\pgfusepath{stroke}%
\end{pgfscope}%
\begin{pgfscope}%
\pgfsetrectcap%
\pgfsetmiterjoin%
\pgfsetlinewidth{0.803000pt}%
\definecolor{currentstroke}{rgb}{1.000000,1.000000,1.000000}%
\pgfsetstrokecolor{currentstroke}%
\pgfsetdash{}{0pt}%
\pgfpathmoveto{\pgfqpoint{0.320934in}{1.347524in}}%
\pgfpathlineto{\pgfqpoint{0.320934in}{1.748409in}}%
\pgfusepath{stroke}%
\end{pgfscope}%
\begin{pgfscope}%
\pgfsetrectcap%
\pgfsetmiterjoin%
\pgfsetlinewidth{0.803000pt}%
\definecolor{currentstroke}{rgb}{1.000000,1.000000,1.000000}%
\pgfsetstrokecolor{currentstroke}%
\pgfsetdash{}{0pt}%
\pgfpathmoveto{\pgfqpoint{2.904267in}{1.347524in}}%
\pgfpathlineto{\pgfqpoint{2.904267in}{1.748409in}}%
\pgfusepath{stroke}%
\end{pgfscope}%
\begin{pgfscope}%
\pgfsetrectcap%
\pgfsetmiterjoin%
\pgfsetlinewidth{0.803000pt}%
\definecolor{currentstroke}{rgb}{1.000000,1.000000,1.000000}%
\pgfsetstrokecolor{currentstroke}%
\pgfsetdash{}{0pt}%
\pgfpathmoveto{\pgfqpoint{0.320934in}{1.347524in}}%
\pgfpathlineto{\pgfqpoint{2.904267in}{1.347524in}}%
\pgfusepath{stroke}%
\end{pgfscope}%
\begin{pgfscope}%
\pgfsetrectcap%
\pgfsetmiterjoin%
\pgfsetlinewidth{0.803000pt}%
\definecolor{currentstroke}{rgb}{1.000000,1.000000,1.000000}%
\pgfsetstrokecolor{currentstroke}%
\pgfsetdash{}{0pt}%
\pgfpathmoveto{\pgfqpoint{0.320934in}{1.748409in}}%
\pgfpathlineto{\pgfqpoint{2.904267in}{1.748409in}}%
\pgfusepath{stroke}%
\end{pgfscope}%
\begin{pgfscope}%
\definecolor{textcolor}{rgb}{0.150000,0.150000,0.150000}%
\pgfsetstrokecolor{textcolor}%
\pgfsetfillcolor{textcolor}%
\pgftext[x=1.612600in,y=1.831742in,,base]{\color{textcolor}\rmfamily\fontsize{16.800000}{20.160000}\selectfont UTX}%
\end{pgfscope}%
\begin{pgfscope}%
\pgfsetbuttcap%
\pgfsetmiterjoin%
\definecolor{currentfill}{rgb}{0.917647,0.917647,0.949020}%
\pgfsetfillcolor{currentfill}%
\pgfsetlinewidth{0.000000pt}%
\definecolor{currentstroke}{rgb}{0.000000,0.000000,0.000000}%
\pgfsetstrokecolor{currentstroke}%
\pgfsetstrokeopacity{0.000000}%
\pgfsetdash{}{0pt}%
\pgfpathmoveto{\pgfqpoint{3.937600in}{1.347524in}}%
\pgfpathlineto{\pgfqpoint{6.520934in}{1.347524in}}%
\pgfpathlineto{\pgfqpoint{6.520934in}{1.748409in}}%
\pgfpathlineto{\pgfqpoint{3.937600in}{1.748409in}}%
\pgfpathclose%
\pgfusepath{fill}%
\end{pgfscope}%
\begin{pgfscope}%
\pgfpathrectangle{\pgfqpoint{3.937600in}{1.347524in}}{\pgfqpoint{2.583333in}{0.400885in}}%
\pgfusepath{clip}%
\pgfsetroundcap%
\pgfsetroundjoin%
\pgfsetlinewidth{0.803000pt}%
\definecolor{currentstroke}{rgb}{1.000000,1.000000,1.000000}%
\pgfsetstrokecolor{currentstroke}%
\pgfsetdash{}{0pt}%
\pgfpathmoveto{\pgfqpoint{4.052877in}{1.347524in}}%
\pgfpathlineto{\pgfqpoint{4.052877in}{1.748409in}}%
\pgfusepath{stroke}%
\end{pgfscope}%
\begin{pgfscope}%
\definecolor{textcolor}{rgb}{0.150000,0.150000,0.150000}%
\pgfsetstrokecolor{textcolor}%
\pgfsetfillcolor{textcolor}%
\pgftext[x=4.052877in,y=1.250302in,,top]{\color{textcolor}\rmfamily\fontsize{14.000000}{16.800000}\selectfont 2012}%
\end{pgfscope}%
\begin{pgfscope}%
\pgfpathrectangle{\pgfqpoint{3.937600in}{1.347524in}}{\pgfqpoint{2.583333in}{0.400885in}}%
\pgfusepath{clip}%
\pgfsetroundcap%
\pgfsetroundjoin%
\pgfsetlinewidth{0.803000pt}%
\definecolor{currentstroke}{rgb}{1.000000,1.000000,1.000000}%
\pgfsetstrokecolor{currentstroke}%
\pgfsetdash{}{0pt}%
\pgfpathmoveto{\pgfqpoint{4.445902in}{1.347524in}}%
\pgfpathlineto{\pgfqpoint{4.445902in}{1.748409in}}%
\pgfusepath{stroke}%
\end{pgfscope}%
\begin{pgfscope}%
\definecolor{textcolor}{rgb}{0.150000,0.150000,0.150000}%
\pgfsetstrokecolor{textcolor}%
\pgfsetfillcolor{textcolor}%
\pgftext[x=4.445902in,y=1.250302in,,top]{\color{textcolor}\rmfamily\fontsize{14.000000}{16.800000}\selectfont 2013}%
\end{pgfscope}%
\begin{pgfscope}%
\pgfpathrectangle{\pgfqpoint{3.937600in}{1.347524in}}{\pgfqpoint{2.583333in}{0.400885in}}%
\pgfusepath{clip}%
\pgfsetroundcap%
\pgfsetroundjoin%
\pgfsetlinewidth{0.803000pt}%
\definecolor{currentstroke}{rgb}{1.000000,1.000000,1.000000}%
\pgfsetstrokecolor{currentstroke}%
\pgfsetdash{}{0pt}%
\pgfpathmoveto{\pgfqpoint{4.837853in}{1.347524in}}%
\pgfpathlineto{\pgfqpoint{4.837853in}{1.748409in}}%
\pgfusepath{stroke}%
\end{pgfscope}%
\begin{pgfscope}%
\definecolor{textcolor}{rgb}{0.150000,0.150000,0.150000}%
\pgfsetstrokecolor{textcolor}%
\pgfsetfillcolor{textcolor}%
\pgftext[x=4.837853in,y=1.250302in,,top]{\color{textcolor}\rmfamily\fontsize{14.000000}{16.800000}\selectfont 2014}%
\end{pgfscope}%
\begin{pgfscope}%
\pgfpathrectangle{\pgfqpoint{3.937600in}{1.347524in}}{\pgfqpoint{2.583333in}{0.400885in}}%
\pgfusepath{clip}%
\pgfsetroundcap%
\pgfsetroundjoin%
\pgfsetlinewidth{0.803000pt}%
\definecolor{currentstroke}{rgb}{1.000000,1.000000,1.000000}%
\pgfsetstrokecolor{currentstroke}%
\pgfsetdash{}{0pt}%
\pgfpathmoveto{\pgfqpoint{5.229804in}{1.347524in}}%
\pgfpathlineto{\pgfqpoint{5.229804in}{1.748409in}}%
\pgfusepath{stroke}%
\end{pgfscope}%
\begin{pgfscope}%
\definecolor{textcolor}{rgb}{0.150000,0.150000,0.150000}%
\pgfsetstrokecolor{textcolor}%
\pgfsetfillcolor{textcolor}%
\pgftext[x=5.229804in,y=1.250302in,,top]{\color{textcolor}\rmfamily\fontsize{14.000000}{16.800000}\selectfont 2015}%
\end{pgfscope}%
\begin{pgfscope}%
\pgfpathrectangle{\pgfqpoint{3.937600in}{1.347524in}}{\pgfqpoint{2.583333in}{0.400885in}}%
\pgfusepath{clip}%
\pgfsetroundcap%
\pgfsetroundjoin%
\pgfsetlinewidth{0.803000pt}%
\definecolor{currentstroke}{rgb}{1.000000,1.000000,1.000000}%
\pgfsetstrokecolor{currentstroke}%
\pgfsetdash{}{0pt}%
\pgfpathmoveto{\pgfqpoint{5.621755in}{1.347524in}}%
\pgfpathlineto{\pgfqpoint{5.621755in}{1.748409in}}%
\pgfusepath{stroke}%
\end{pgfscope}%
\begin{pgfscope}%
\definecolor{textcolor}{rgb}{0.150000,0.150000,0.150000}%
\pgfsetstrokecolor{textcolor}%
\pgfsetfillcolor{textcolor}%
\pgftext[x=5.621755in,y=1.250302in,,top]{\color{textcolor}\rmfamily\fontsize{14.000000}{16.800000}\selectfont 2016}%
\end{pgfscope}%
\begin{pgfscope}%
\pgfpathrectangle{\pgfqpoint{3.937600in}{1.347524in}}{\pgfqpoint{2.583333in}{0.400885in}}%
\pgfusepath{clip}%
\pgfsetroundcap%
\pgfsetroundjoin%
\pgfsetlinewidth{0.803000pt}%
\definecolor{currentstroke}{rgb}{1.000000,1.000000,1.000000}%
\pgfsetstrokecolor{currentstroke}%
\pgfsetdash{}{0pt}%
\pgfpathmoveto{\pgfqpoint{6.014780in}{1.347524in}}%
\pgfpathlineto{\pgfqpoint{6.014780in}{1.748409in}}%
\pgfusepath{stroke}%
\end{pgfscope}%
\begin{pgfscope}%
\definecolor{textcolor}{rgb}{0.150000,0.150000,0.150000}%
\pgfsetstrokecolor{textcolor}%
\pgfsetfillcolor{textcolor}%
\pgftext[x=6.014780in,y=1.250302in,,top]{\color{textcolor}\rmfamily\fontsize{14.000000}{16.800000}\selectfont 2017}%
\end{pgfscope}%
\begin{pgfscope}%
\pgfpathrectangle{\pgfqpoint{3.937600in}{1.347524in}}{\pgfqpoint{2.583333in}{0.400885in}}%
\pgfusepath{clip}%
\pgfsetroundcap%
\pgfsetroundjoin%
\pgfsetlinewidth{0.803000pt}%
\definecolor{currentstroke}{rgb}{1.000000,1.000000,1.000000}%
\pgfsetstrokecolor{currentstroke}%
\pgfsetdash{}{0pt}%
\pgfpathmoveto{\pgfqpoint{6.406731in}{1.347524in}}%
\pgfpathlineto{\pgfqpoint{6.406731in}{1.748409in}}%
\pgfusepath{stroke}%
\end{pgfscope}%
\begin{pgfscope}%
\definecolor{textcolor}{rgb}{0.150000,0.150000,0.150000}%
\pgfsetstrokecolor{textcolor}%
\pgfsetfillcolor{textcolor}%
\pgftext[x=6.406731in,y=1.250302in,,top]{\color{textcolor}\rmfamily\fontsize{14.000000}{16.800000}\selectfont 2018}%
\end{pgfscope}%
\begin{pgfscope}%
\pgfpathrectangle{\pgfqpoint{3.937600in}{1.347524in}}{\pgfqpoint{2.583333in}{0.400885in}}%
\pgfusepath{clip}%
\pgfsetroundcap%
\pgfsetroundjoin%
\pgfsetlinewidth{0.803000pt}%
\definecolor{currentstroke}{rgb}{1.000000,1.000000,1.000000}%
\pgfsetstrokecolor{currentstroke}%
\pgfsetdash{}{0pt}%
\pgfpathmoveto{\pgfqpoint{3.937600in}{1.412929in}}%
\pgfpathlineto{\pgfqpoint{6.520934in}{1.412929in}}%
\pgfusepath{stroke}%
\end{pgfscope}%
\begin{pgfscope}%
\definecolor{textcolor}{rgb}{0.150000,0.150000,0.150000}%
\pgfsetstrokecolor{textcolor}%
\pgfsetfillcolor{textcolor}%
\pgftext[x=3.531147in,y=1.339063in,left,base]{\color{textcolor}\rmfamily\fontsize{14.000000}{16.800000}\selectfont 0.5}%
\end{pgfscope}%
\begin{pgfscope}%
\pgfpathrectangle{\pgfqpoint{3.937600in}{1.347524in}}{\pgfqpoint{2.583333in}{0.400885in}}%
\pgfusepath{clip}%
\pgfsetroundcap%
\pgfsetroundjoin%
\pgfsetlinewidth{0.803000pt}%
\definecolor{currentstroke}{rgb}{1.000000,1.000000,1.000000}%
\pgfsetstrokecolor{currentstroke}%
\pgfsetdash{}{0pt}%
\pgfpathmoveto{\pgfqpoint{3.937600in}{1.537847in}}%
\pgfpathlineto{\pgfqpoint{6.520934in}{1.537847in}}%
\pgfusepath{stroke}%
\end{pgfscope}%
\begin{pgfscope}%
\definecolor{textcolor}{rgb}{0.150000,0.150000,0.150000}%
\pgfsetstrokecolor{textcolor}%
\pgfsetfillcolor{textcolor}%
\pgftext[x=3.531147in,y=1.463981in,left,base]{\color{textcolor}\rmfamily\fontsize{14.000000}{16.800000}\selectfont 1.0}%
\end{pgfscope}%
\begin{pgfscope}%
\pgfpathrectangle{\pgfqpoint{3.937600in}{1.347524in}}{\pgfqpoint{2.583333in}{0.400885in}}%
\pgfusepath{clip}%
\pgfsetroundcap%
\pgfsetroundjoin%
\pgfsetlinewidth{0.803000pt}%
\definecolor{currentstroke}{rgb}{1.000000,1.000000,1.000000}%
\pgfsetstrokecolor{currentstroke}%
\pgfsetdash{}{0pt}%
\pgfpathmoveto{\pgfqpoint{3.937600in}{1.662765in}}%
\pgfpathlineto{\pgfqpoint{6.520934in}{1.662765in}}%
\pgfusepath{stroke}%
\end{pgfscope}%
\begin{pgfscope}%
\definecolor{textcolor}{rgb}{0.150000,0.150000,0.150000}%
\pgfsetstrokecolor{textcolor}%
\pgfsetfillcolor{textcolor}%
\pgftext[x=3.531147in,y=1.588899in,left,base]{\color{textcolor}\rmfamily\fontsize{14.000000}{16.800000}\selectfont 1.5}%
\end{pgfscope}%
\begin{pgfscope}%
\pgfpathrectangle{\pgfqpoint{3.937600in}{1.347524in}}{\pgfqpoint{2.583333in}{0.400885in}}%
\pgfusepath{clip}%
\pgfsetroundcap%
\pgfsetroundjoin%
\pgfsetlinewidth{1.505625pt}%
\definecolor{currentstroke}{rgb}{0.000000,0.000000,0.000000}%
\pgfsetstrokecolor{currentstroke}%
\pgfsetdash{}{0pt}%
\pgfpathmoveto{\pgfqpoint{4.055025in}{1.537847in}}%
\pgfpathlineto{\pgfqpoint{4.057172in}{1.532929in}}%
\pgfpathlineto{\pgfqpoint{4.058246in}{1.532214in}}%
\pgfpathlineto{\pgfqpoint{4.061468in}{1.532482in}}%
\pgfpathlineto{\pgfqpoint{4.062542in}{1.533734in}}%
\pgfpathlineto{\pgfqpoint{4.063615in}{1.535880in}}%
\pgfpathlineto{\pgfqpoint{4.073280in}{1.536327in}}%
\pgfpathlineto{\pgfqpoint{4.076501in}{1.532661in}}%
\pgfpathlineto{\pgfqpoint{4.077575in}{1.528816in}}%
\pgfpathlineto{\pgfqpoint{4.078649in}{1.528100in}}%
\pgfpathlineto{\pgfqpoint{4.079723in}{1.525865in}}%
\pgfpathlineto{\pgfqpoint{4.080797in}{1.525060in}}%
\pgfpathlineto{\pgfqpoint{4.086166in}{1.528816in}}%
\pgfpathlineto{\pgfqpoint{4.087240in}{1.527296in}}%
\pgfpathlineto{\pgfqpoint{4.088314in}{1.529084in}}%
\pgfpathlineto{\pgfqpoint{4.091535in}{1.530962in}}%
\pgfpathlineto{\pgfqpoint{4.092609in}{1.529620in}}%
\pgfpathlineto{\pgfqpoint{4.094757in}{1.529620in}}%
\pgfpathlineto{\pgfqpoint{4.095831in}{1.528100in}}%
\pgfpathlineto{\pgfqpoint{4.099052in}{1.530962in}}%
\pgfpathlineto{\pgfqpoint{4.100126in}{1.530336in}}%
\pgfpathlineto{\pgfqpoint{4.101200in}{1.528995in}}%
\pgfpathlineto{\pgfqpoint{4.102274in}{1.530425in}}%
\pgfpathlineto{\pgfqpoint{4.103347in}{1.533018in}}%
\pgfpathlineto{\pgfqpoint{4.107643in}{1.533197in}}%
\pgfpathlineto{\pgfqpoint{4.108717in}{1.531409in}}%
\pgfpathlineto{\pgfqpoint{4.110864in}{1.530962in}}%
\pgfpathlineto{\pgfqpoint{4.116233in}{1.530783in}}%
\pgfpathlineto{\pgfqpoint{4.118381in}{1.534360in}}%
\pgfpathlineto{\pgfqpoint{4.121603in}{1.536506in}}%
\pgfpathlineto{\pgfqpoint{4.122676in}{1.534539in}}%
\pgfpathlineto{\pgfqpoint{4.123750in}{1.535701in}}%
\pgfpathlineto{\pgfqpoint{4.124824in}{1.537847in}}%
\pgfpathlineto{\pgfqpoint{4.125898in}{1.537132in}}%
\pgfpathlineto{\pgfqpoint{4.133415in}{1.540083in}}%
\pgfpathlineto{\pgfqpoint{4.139858in}{1.540708in}}%
\pgfpathlineto{\pgfqpoint{4.140932in}{1.539188in}}%
\pgfpathlineto{\pgfqpoint{4.144153in}{1.538562in}}%
\pgfpathlineto{\pgfqpoint{4.146301in}{1.531767in}}%
\pgfpathlineto{\pgfqpoint{4.147375in}{1.530604in}}%
\pgfpathlineto{\pgfqpoint{4.148449in}{1.531588in}}%
\pgfpathlineto{\pgfqpoint{4.151670in}{1.533466in}}%
\pgfpathlineto{\pgfqpoint{4.153818in}{1.532571in}}%
\pgfpathlineto{\pgfqpoint{4.154892in}{1.531141in}}%
\pgfpathlineto{\pgfqpoint{4.159187in}{1.529799in}}%
\pgfpathlineto{\pgfqpoint{4.160261in}{1.525597in}}%
\pgfpathlineto{\pgfqpoint{4.161335in}{1.529263in}}%
\pgfpathlineto{\pgfqpoint{4.162409in}{1.530425in}}%
\pgfpathlineto{\pgfqpoint{4.163482in}{1.528547in}}%
\pgfpathlineto{\pgfqpoint{4.166704in}{1.529620in}}%
\pgfpathlineto{\pgfqpoint{4.167778in}{1.531588in}}%
\pgfpathlineto{\pgfqpoint{4.168852in}{1.531141in}}%
\pgfpathlineto{\pgfqpoint{4.170999in}{1.538026in}}%
\pgfpathlineto{\pgfqpoint{4.174221in}{1.536953in}}%
\pgfpathlineto{\pgfqpoint{4.175295in}{1.543033in}}%
\pgfpathlineto{\pgfqpoint{4.176368in}{1.542854in}}%
\pgfpathlineto{\pgfqpoint{4.177442in}{1.547147in}}%
\pgfpathlineto{\pgfqpoint{4.178516in}{1.547683in}}%
\pgfpathlineto{\pgfqpoint{4.181738in}{1.548667in}}%
\pgfpathlineto{\pgfqpoint{4.182811in}{1.549829in}}%
\pgfpathlineto{\pgfqpoint{4.184959in}{1.550366in}}%
\pgfpathlineto{\pgfqpoint{4.186033in}{1.547862in}}%
\pgfpathlineto{\pgfqpoint{4.190328in}{1.549740in}}%
\pgfpathlineto{\pgfqpoint{4.191402in}{1.547862in}}%
\pgfpathlineto{\pgfqpoint{4.192476in}{1.549740in}}%
\pgfpathlineto{\pgfqpoint{4.193550in}{1.553674in}}%
\pgfpathlineto{\pgfqpoint{4.196771in}{1.551975in}}%
\pgfpathlineto{\pgfqpoint{4.197845in}{1.552959in}}%
\pgfpathlineto{\pgfqpoint{4.198919in}{1.551886in}}%
\pgfpathlineto{\pgfqpoint{4.199993in}{1.555105in}}%
\pgfpathlineto{\pgfqpoint{4.201067in}{1.556088in}}%
\pgfpathlineto{\pgfqpoint{4.204288in}{1.554837in}}%
\pgfpathlineto{\pgfqpoint{4.205362in}{1.555194in}}%
\pgfpathlineto{\pgfqpoint{4.206436in}{1.554479in}}%
\pgfpathlineto{\pgfqpoint{4.208584in}{1.555552in}}%
\pgfpathlineto{\pgfqpoint{4.212879in}{1.557519in}}%
\pgfpathlineto{\pgfqpoint{4.213953in}{1.555373in}}%
\pgfpathlineto{\pgfqpoint{4.215027in}{1.556804in}}%
\pgfpathlineto{\pgfqpoint{4.216100in}{1.552869in}}%
\pgfpathlineto{\pgfqpoint{4.219322in}{1.554837in}}%
\pgfpathlineto{\pgfqpoint{4.220396in}{1.553942in}}%
\pgfpathlineto{\pgfqpoint{4.221470in}{1.557609in}}%
\pgfpathlineto{\pgfqpoint{4.222543in}{1.556804in}}%
\pgfpathlineto{\pgfqpoint{4.223617in}{1.561990in}}%
\pgfpathlineto{\pgfqpoint{4.226839in}{1.562705in}}%
\pgfpathlineto{\pgfqpoint{4.227913in}{1.565209in}}%
\pgfpathlineto{\pgfqpoint{4.228986in}{1.565477in}}%
\pgfpathlineto{\pgfqpoint{4.230060in}{1.570396in}}%
\pgfpathlineto{\pgfqpoint{4.231134in}{1.569144in}}%
\pgfpathlineto{\pgfqpoint{4.234356in}{1.570843in}}%
\pgfpathlineto{\pgfqpoint{4.235430in}{1.570306in}}%
\pgfpathlineto{\pgfqpoint{4.236503in}{1.567534in}}%
\pgfpathlineto{\pgfqpoint{4.237577in}{1.567713in}}%
\pgfpathlineto{\pgfqpoint{4.238651in}{1.571737in}}%
\pgfpathlineto{\pgfqpoint{4.241873in}{1.569770in}}%
\pgfpathlineto{\pgfqpoint{4.242946in}{1.571021in}}%
\pgfpathlineto{\pgfqpoint{4.244020in}{1.570664in}}%
\pgfpathlineto{\pgfqpoint{4.245094in}{1.571826in}}%
\pgfpathlineto{\pgfqpoint{4.246168in}{1.574866in}}%
\pgfpathlineto{\pgfqpoint{4.249389in}{1.578086in}}%
\pgfpathlineto{\pgfqpoint{4.253685in}{1.577996in}}%
\pgfpathlineto{\pgfqpoint{4.256906in}{1.580053in}}%
\pgfpathlineto{\pgfqpoint{4.257980in}{1.579784in}}%
\pgfpathlineto{\pgfqpoint{4.259054in}{1.581126in}}%
\pgfpathlineto{\pgfqpoint{4.260128in}{1.579606in}}%
\pgfpathlineto{\pgfqpoint{4.261202in}{1.583182in}}%
\pgfpathlineto{\pgfqpoint{4.264423in}{1.583540in}}%
\pgfpathlineto{\pgfqpoint{4.266571in}{1.587564in}}%
\pgfpathlineto{\pgfqpoint{4.267645in}{1.578801in}}%
\pgfpathlineto{\pgfqpoint{4.271940in}{1.577549in}}%
\pgfpathlineto{\pgfqpoint{4.273014in}{1.573704in}}%
\pgfpathlineto{\pgfqpoint{4.274088in}{1.573436in}}%
\pgfpathlineto{\pgfqpoint{4.276235in}{1.581126in}}%
\pgfpathlineto{\pgfqpoint{4.279457in}{1.581483in}}%
\pgfpathlineto{\pgfqpoint{4.281605in}{1.583182in}}%
\pgfpathlineto{\pgfqpoint{4.282678in}{1.579248in}}%
\pgfpathlineto{\pgfqpoint{4.283752in}{1.578264in}}%
\pgfpathlineto{\pgfqpoint{4.286974in}{1.579784in}}%
\pgfpathlineto{\pgfqpoint{4.289121in}{1.576387in}}%
\pgfpathlineto{\pgfqpoint{4.291269in}{1.579159in}}%
\pgfpathlineto{\pgfqpoint{4.294491in}{1.577013in}}%
\pgfpathlineto{\pgfqpoint{4.295564in}{1.577370in}}%
\pgfpathlineto{\pgfqpoint{4.297712in}{1.576029in}}%
\pgfpathlineto{\pgfqpoint{4.298786in}{1.575671in}}%
\pgfpathlineto{\pgfqpoint{4.302008in}{1.573257in}}%
\pgfpathlineto{\pgfqpoint{4.303081in}{1.567981in}}%
\pgfpathlineto{\pgfqpoint{4.305229in}{1.563779in}}%
\pgfpathlineto{\pgfqpoint{4.306303in}{1.569859in}}%
\pgfpathlineto{\pgfqpoint{4.310598in}{1.566550in}}%
\pgfpathlineto{\pgfqpoint{4.311672in}{1.569322in}}%
\pgfpathlineto{\pgfqpoint{4.312746in}{1.567176in}}%
\pgfpathlineto{\pgfqpoint{4.319189in}{1.573972in}}%
\pgfpathlineto{\pgfqpoint{4.320263in}{1.576208in}}%
\pgfpathlineto{\pgfqpoint{4.321337in}{1.573436in}}%
\pgfpathlineto{\pgfqpoint{4.325632in}{1.576834in}}%
\pgfpathlineto{\pgfqpoint{4.327780in}{1.585597in}}%
\pgfpathlineto{\pgfqpoint{4.328853in}{1.578711in}}%
\pgfpathlineto{\pgfqpoint{4.332075in}{1.579069in}}%
\pgfpathlineto{\pgfqpoint{4.335297in}{1.584971in}}%
\pgfpathlineto{\pgfqpoint{4.336370in}{1.585954in}}%
\pgfpathlineto{\pgfqpoint{4.340666in}{1.585776in}}%
\pgfpathlineto{\pgfqpoint{4.341740in}{1.585597in}}%
\pgfpathlineto{\pgfqpoint{4.342813in}{1.586759in}}%
\pgfpathlineto{\pgfqpoint{4.343887in}{1.585507in}}%
\pgfpathlineto{\pgfqpoint{4.348183in}{1.587385in}}%
\pgfpathlineto{\pgfqpoint{4.349256in}{1.590157in}}%
\pgfpathlineto{\pgfqpoint{4.350330in}{1.596506in}}%
\pgfpathlineto{\pgfqpoint{4.351404in}{1.598562in}}%
\pgfpathlineto{\pgfqpoint{4.354626in}{1.595343in}}%
\pgfpathlineto{\pgfqpoint{4.358921in}{1.582467in}}%
\pgfpathlineto{\pgfqpoint{4.362142in}{1.581662in}}%
\pgfpathlineto{\pgfqpoint{4.363216in}{1.578890in}}%
\pgfpathlineto{\pgfqpoint{4.364290in}{1.583182in}}%
\pgfpathlineto{\pgfqpoint{4.365364in}{1.590157in}}%
\pgfpathlineto{\pgfqpoint{4.366438in}{1.586044in}}%
\pgfpathlineto{\pgfqpoint{4.369659in}{1.583451in}}%
\pgfpathlineto{\pgfqpoint{4.370733in}{1.578890in}}%
\pgfpathlineto{\pgfqpoint{4.371807in}{1.579963in}}%
\pgfpathlineto{\pgfqpoint{4.372881in}{1.580053in}}%
\pgfpathlineto{\pgfqpoint{4.373955in}{1.583182in}}%
\pgfpathlineto{\pgfqpoint{4.379324in}{1.582646in}}%
\pgfpathlineto{\pgfqpoint{4.380398in}{1.585954in}}%
\pgfpathlineto{\pgfqpoint{4.381472in}{1.581841in}}%
\pgfpathlineto{\pgfqpoint{4.384693in}{1.579695in}}%
\pgfpathlineto{\pgfqpoint{4.385767in}{1.580410in}}%
\pgfpathlineto{\pgfqpoint{4.387915in}{1.569233in}}%
\pgfpathlineto{\pgfqpoint{4.388988in}{1.569412in}}%
\pgfpathlineto{\pgfqpoint{4.393284in}{1.568786in}}%
\pgfpathlineto{\pgfqpoint{4.394358in}{1.566819in}}%
\pgfpathlineto{\pgfqpoint{4.396505in}{1.561275in}}%
\pgfpathlineto{\pgfqpoint{4.399727in}{1.570574in}}%
\pgfpathlineto{\pgfqpoint{4.400801in}{1.570574in}}%
\pgfpathlineto{\pgfqpoint{4.404022in}{1.576834in}}%
\pgfpathlineto{\pgfqpoint{4.407244in}{1.573793in}}%
\pgfpathlineto{\pgfqpoint{4.408318in}{1.571647in}}%
\pgfpathlineto{\pgfqpoint{4.410465in}{1.578443in}}%
\pgfpathlineto{\pgfqpoint{4.411539in}{1.579159in}}%
\pgfpathlineto{\pgfqpoint{4.414761in}{1.579069in}}%
\pgfpathlineto{\pgfqpoint{4.415834in}{1.576208in}}%
\pgfpathlineto{\pgfqpoint{4.417982in}{1.581394in}}%
\pgfpathlineto{\pgfqpoint{4.419056in}{1.581126in}}%
\pgfpathlineto{\pgfqpoint{4.422277in}{1.578622in}}%
\pgfpathlineto{\pgfqpoint{4.424425in}{1.583630in}}%
\pgfpathlineto{\pgfqpoint{4.426573in}{1.579784in}}%
\pgfpathlineto{\pgfqpoint{4.429794in}{1.579159in}}%
\pgfpathlineto{\pgfqpoint{4.430868in}{1.577549in}}%
\pgfpathlineto{\pgfqpoint{4.431942in}{1.574330in}}%
\pgfpathlineto{\pgfqpoint{4.433016in}{1.577191in}}%
\pgfpathlineto{\pgfqpoint{4.434090in}{1.575582in}}%
\pgfpathlineto{\pgfqpoint{4.440533in}{1.574956in}}%
\pgfpathlineto{\pgfqpoint{4.441607in}{1.571111in}}%
\pgfpathlineto{\pgfqpoint{4.444828in}{1.573615in}}%
\pgfpathlineto{\pgfqpoint{4.446976in}{1.580142in}}%
\pgfpathlineto{\pgfqpoint{4.448050in}{1.578801in}}%
\pgfpathlineto{\pgfqpoint{4.449123in}{1.580410in}}%
\pgfpathlineto{\pgfqpoint{4.452345in}{1.582914in}}%
\pgfpathlineto{\pgfqpoint{4.453419in}{1.575761in}}%
\pgfpathlineto{\pgfqpoint{4.454493in}{1.575135in}}%
\pgfpathlineto{\pgfqpoint{4.455566in}{1.579069in}}%
\pgfpathlineto{\pgfqpoint{4.456640in}{1.577102in}}%
\pgfpathlineto{\pgfqpoint{4.459862in}{1.572363in}}%
\pgfpathlineto{\pgfqpoint{4.462009in}{1.565120in}}%
\pgfpathlineto{\pgfqpoint{4.464157in}{1.572005in}}%
\pgfpathlineto{\pgfqpoint{4.468452in}{1.574688in}}%
\pgfpathlineto{\pgfqpoint{4.470600in}{1.572363in}}%
\pgfpathlineto{\pgfqpoint{4.471674in}{1.572899in}}%
\pgfpathlineto{\pgfqpoint{4.474896in}{1.573615in}}%
\pgfpathlineto{\pgfqpoint{4.475969in}{1.578443in}}%
\pgfpathlineto{\pgfqpoint{4.477043in}{1.579159in}}%
\pgfpathlineto{\pgfqpoint{4.478117in}{1.579159in}}%
\pgfpathlineto{\pgfqpoint{4.479191in}{1.585507in}}%
\pgfpathlineto{\pgfqpoint{4.483486in}{1.585507in}}%
\pgfpathlineto{\pgfqpoint{4.484560in}{1.586759in}}%
\pgfpathlineto{\pgfqpoint{4.485634in}{1.584881in}}%
\pgfpathlineto{\pgfqpoint{4.486708in}{1.584166in}}%
\pgfpathlineto{\pgfqpoint{4.489929in}{1.583898in}}%
\pgfpathlineto{\pgfqpoint{4.492077in}{1.585239in}}%
\pgfpathlineto{\pgfqpoint{4.493151in}{1.583987in}}%
\pgfpathlineto{\pgfqpoint{4.494225in}{1.584434in}}%
\pgfpathlineto{\pgfqpoint{4.498520in}{1.585150in}}%
\pgfpathlineto{\pgfqpoint{4.500668in}{1.589263in}}%
\pgfpathlineto{\pgfqpoint{4.501741in}{1.591141in}}%
\pgfpathlineto{\pgfqpoint{4.504963in}{1.593287in}}%
\pgfpathlineto{\pgfqpoint{4.507111in}{1.597489in}}%
\pgfpathlineto{\pgfqpoint{4.512480in}{1.602586in}}%
\pgfpathlineto{\pgfqpoint{4.513554in}{1.606431in}}%
\pgfpathlineto{\pgfqpoint{4.514628in}{1.603659in}}%
\pgfpathlineto{\pgfqpoint{4.515701in}{1.605001in}}%
\pgfpathlineto{\pgfqpoint{4.516775in}{1.608220in}}%
\pgfpathlineto{\pgfqpoint{4.519997in}{1.607236in}}%
\pgfpathlineto{\pgfqpoint{4.521071in}{1.610902in}}%
\pgfpathlineto{\pgfqpoint{4.522144in}{1.608130in}}%
\pgfpathlineto{\pgfqpoint{4.523218in}{1.611707in}}%
\pgfpathlineto{\pgfqpoint{4.524292in}{1.608667in}}%
\pgfpathlineto{\pgfqpoint{4.528587in}{1.614837in}}%
\pgfpathlineto{\pgfqpoint{4.529661in}{1.612512in}}%
\pgfpathlineto{\pgfqpoint{4.531809in}{1.615284in}}%
\pgfpathlineto{\pgfqpoint{4.535030in}{1.616267in}}%
\pgfpathlineto{\pgfqpoint{4.536104in}{1.618413in}}%
\pgfpathlineto{\pgfqpoint{4.537178in}{1.614747in}}%
\pgfpathlineto{\pgfqpoint{4.538252in}{1.616178in}}%
\pgfpathlineto{\pgfqpoint{4.542547in}{1.616625in}}%
\pgfpathlineto{\pgfqpoint{4.543621in}{1.618503in}}%
\pgfpathlineto{\pgfqpoint{4.544695in}{1.615105in}}%
\pgfpathlineto{\pgfqpoint{4.546843in}{1.618950in}}%
\pgfpathlineto{\pgfqpoint{4.550064in}{1.621543in}}%
\pgfpathlineto{\pgfqpoint{4.551138in}{1.621007in}}%
\pgfpathlineto{\pgfqpoint{4.554360in}{1.631200in}}%
\pgfpathlineto{\pgfqpoint{4.557581in}{1.629680in}}%
\pgfpathlineto{\pgfqpoint{4.558655in}{1.628428in}}%
\pgfpathlineto{\pgfqpoint{4.559729in}{1.622258in}}%
\pgfpathlineto{\pgfqpoint{4.561876in}{1.640500in}}%
\pgfpathlineto{\pgfqpoint{4.565098in}{1.640142in}}%
\pgfpathlineto{\pgfqpoint{4.566172in}{1.641036in}}%
\pgfpathlineto{\pgfqpoint{4.567246in}{1.637460in}}%
\pgfpathlineto{\pgfqpoint{4.568319in}{1.647117in}}%
\pgfpathlineto{\pgfqpoint{4.569393in}{1.649889in}}%
\pgfpathlineto{\pgfqpoint{4.572615in}{1.648726in}}%
\pgfpathlineto{\pgfqpoint{4.573689in}{1.651767in}}%
\pgfpathlineto{\pgfqpoint{4.574763in}{1.641573in}}%
\pgfpathlineto{\pgfqpoint{4.576910in}{1.643451in}}%
\pgfpathlineto{\pgfqpoint{4.580132in}{1.639159in}}%
\pgfpathlineto{\pgfqpoint{4.581206in}{1.645060in}}%
\pgfpathlineto{\pgfqpoint{4.582279in}{1.646312in}}%
\pgfpathlineto{\pgfqpoint{4.583353in}{1.643630in}}%
\pgfpathlineto{\pgfqpoint{4.584427in}{1.644881in}}%
\pgfpathlineto{\pgfqpoint{4.587649in}{1.642557in}}%
\pgfpathlineto{\pgfqpoint{4.589796in}{1.649621in}}%
\pgfpathlineto{\pgfqpoint{4.590870in}{1.646938in}}%
\pgfpathlineto{\pgfqpoint{4.591944in}{1.647922in}}%
\pgfpathlineto{\pgfqpoint{4.595165in}{1.643808in}}%
\pgfpathlineto{\pgfqpoint{4.597313in}{1.635314in}}%
\pgfpathlineto{\pgfqpoint{4.598387in}{1.638086in}}%
\pgfpathlineto{\pgfqpoint{4.599461in}{1.634777in}}%
\pgfpathlineto{\pgfqpoint{4.603756in}{1.630932in}}%
\pgfpathlineto{\pgfqpoint{4.604830in}{1.622437in}}%
\pgfpathlineto{\pgfqpoint{4.606978in}{1.615105in}}%
\pgfpathlineto{\pgfqpoint{4.610199in}{1.616357in}}%
\pgfpathlineto{\pgfqpoint{4.611273in}{1.617519in}}%
\pgfpathlineto{\pgfqpoint{4.612347in}{1.613853in}}%
\pgfpathlineto{\pgfqpoint{4.613421in}{1.625120in}}%
\pgfpathlineto{\pgfqpoint{4.614495in}{1.626998in}}%
\pgfpathlineto{\pgfqpoint{4.617716in}{1.628965in}}%
\pgfpathlineto{\pgfqpoint{4.619864in}{1.624673in}}%
\pgfpathlineto{\pgfqpoint{4.622011in}{1.632542in}}%
\pgfpathlineto{\pgfqpoint{4.625233in}{1.630127in}}%
\pgfpathlineto{\pgfqpoint{4.626307in}{1.635850in}}%
\pgfpathlineto{\pgfqpoint{4.628454in}{1.618324in}}%
\pgfpathlineto{\pgfqpoint{4.629528in}{1.622080in}}%
\pgfpathlineto{\pgfqpoint{4.632750in}{1.619397in}}%
\pgfpathlineto{\pgfqpoint{4.633824in}{1.628339in}}%
\pgfpathlineto{\pgfqpoint{4.635971in}{1.632095in}}%
\pgfpathlineto{\pgfqpoint{4.637045in}{1.627624in}}%
\pgfpathlineto{\pgfqpoint{4.640267in}{1.627802in}}%
\pgfpathlineto{\pgfqpoint{4.644562in}{1.634151in}}%
\pgfpathlineto{\pgfqpoint{4.647784in}{1.636744in}}%
\pgfpathlineto{\pgfqpoint{4.648857in}{1.635314in}}%
\pgfpathlineto{\pgfqpoint{4.649931in}{1.632452in}}%
\pgfpathlineto{\pgfqpoint{4.651005in}{1.637013in}}%
\pgfpathlineto{\pgfqpoint{4.652079in}{1.631558in}}%
\pgfpathlineto{\pgfqpoint{4.655300in}{1.628518in}}%
\pgfpathlineto{\pgfqpoint{4.657448in}{1.633794in}}%
\pgfpathlineto{\pgfqpoint{4.658522in}{1.628607in}}%
\pgfpathlineto{\pgfqpoint{4.659596in}{1.628428in}}%
\pgfpathlineto{\pgfqpoint{4.664965in}{1.631379in}}%
\pgfpathlineto{\pgfqpoint{4.667113in}{1.635761in}}%
\pgfpathlineto{\pgfqpoint{4.670334in}{1.638890in}}%
\pgfpathlineto{\pgfqpoint{4.672482in}{1.625209in}}%
\pgfpathlineto{\pgfqpoint{4.673556in}{1.628875in}}%
\pgfpathlineto{\pgfqpoint{4.674629in}{1.630485in}}%
\pgfpathlineto{\pgfqpoint{4.677851in}{1.630217in}}%
\pgfpathlineto{\pgfqpoint{4.679999in}{1.628339in}}%
\pgfpathlineto{\pgfqpoint{4.682146in}{1.624136in}}%
\pgfpathlineto{\pgfqpoint{4.685368in}{1.626372in}}%
\pgfpathlineto{\pgfqpoint{4.687516in}{1.621185in}}%
\pgfpathlineto{\pgfqpoint{4.688589in}{1.618861in}}%
\pgfpathlineto{\pgfqpoint{4.689663in}{1.613138in}}%
\pgfpathlineto{\pgfqpoint{4.692885in}{1.611707in}}%
\pgfpathlineto{\pgfqpoint{4.693959in}{1.614568in}}%
\pgfpathlineto{\pgfqpoint{4.695032in}{1.610187in}}%
\pgfpathlineto{\pgfqpoint{4.696106in}{1.608488in}}%
\pgfpathlineto{\pgfqpoint{4.697180in}{1.612512in}}%
\pgfpathlineto{\pgfqpoint{4.700402in}{1.607951in}}%
\pgfpathlineto{\pgfqpoint{4.701475in}{1.607951in}}%
\pgfpathlineto{\pgfqpoint{4.702549in}{1.605358in}}%
\pgfpathlineto{\pgfqpoint{4.703623in}{1.613943in}}%
\pgfpathlineto{\pgfqpoint{4.704697in}{1.610902in}}%
\pgfpathlineto{\pgfqpoint{4.708992in}{1.601603in}}%
\pgfpathlineto{\pgfqpoint{4.710066in}{1.606789in}}%
\pgfpathlineto{\pgfqpoint{4.711140in}{1.605895in}}%
\pgfpathlineto{\pgfqpoint{4.712214in}{1.603838in}}%
\pgfpathlineto{\pgfqpoint{4.715435in}{1.600887in}}%
\pgfpathlineto{\pgfqpoint{4.716509in}{1.604732in}}%
\pgfpathlineto{\pgfqpoint{4.717583in}{1.605090in}}%
\pgfpathlineto{\pgfqpoint{4.719731in}{1.613495in}}%
\pgfpathlineto{\pgfqpoint{4.725100in}{1.620023in}}%
\pgfpathlineto{\pgfqpoint{4.726174in}{1.618592in}}%
\pgfpathlineto{\pgfqpoint{4.727248in}{1.613674in}}%
\pgfpathlineto{\pgfqpoint{4.730469in}{1.615016in}}%
\pgfpathlineto{\pgfqpoint{4.731543in}{1.610187in}}%
\pgfpathlineto{\pgfqpoint{4.732617in}{1.607951in}}%
\pgfpathlineto{\pgfqpoint{4.733691in}{1.612869in}}%
\pgfpathlineto{\pgfqpoint{4.734764in}{1.608309in}}%
\pgfpathlineto{\pgfqpoint{4.737986in}{1.606074in}}%
\pgfpathlineto{\pgfqpoint{4.739060in}{1.608220in}}%
\pgfpathlineto{\pgfqpoint{4.740134in}{1.606878in}}%
\pgfpathlineto{\pgfqpoint{4.742281in}{1.609024in}}%
\pgfpathlineto{\pgfqpoint{4.745503in}{1.610276in}}%
\pgfpathlineto{\pgfqpoint{4.746577in}{1.605448in}}%
\pgfpathlineto{\pgfqpoint{4.747651in}{1.606431in}}%
\pgfpathlineto{\pgfqpoint{4.748724in}{1.610992in}}%
\pgfpathlineto{\pgfqpoint{4.749798in}{1.612601in}}%
\pgfpathlineto{\pgfqpoint{4.753020in}{1.610634in}}%
\pgfpathlineto{\pgfqpoint{4.754094in}{1.607236in}}%
\pgfpathlineto{\pgfqpoint{4.755167in}{1.613674in}}%
\pgfpathlineto{\pgfqpoint{4.757315in}{1.632720in}}%
\pgfpathlineto{\pgfqpoint{4.760537in}{1.636655in}}%
\pgfpathlineto{\pgfqpoint{4.761610in}{1.640500in}}%
\pgfpathlineto{\pgfqpoint{4.763758in}{1.635492in}}%
\pgfpathlineto{\pgfqpoint{4.764832in}{1.637549in}}%
\pgfpathlineto{\pgfqpoint{4.768053in}{1.636565in}}%
\pgfpathlineto{\pgfqpoint{4.769127in}{1.640142in}}%
\pgfpathlineto{\pgfqpoint{4.770201in}{1.636297in}}%
\pgfpathlineto{\pgfqpoint{4.772349in}{1.636029in}}%
\pgfpathlineto{\pgfqpoint{4.775570in}{1.640053in}}%
\pgfpathlineto{\pgfqpoint{4.776644in}{1.633346in}}%
\pgfpathlineto{\pgfqpoint{4.777718in}{1.636834in}}%
\pgfpathlineto{\pgfqpoint{4.778792in}{1.633794in}}%
\pgfpathlineto{\pgfqpoint{4.779866in}{1.634062in}}%
\pgfpathlineto{\pgfqpoint{4.783087in}{1.632363in}}%
\pgfpathlineto{\pgfqpoint{4.784161in}{1.633704in}}%
\pgfpathlineto{\pgfqpoint{4.785235in}{1.632542in}}%
\pgfpathlineto{\pgfqpoint{4.786309in}{1.634509in}}%
\pgfpathlineto{\pgfqpoint{4.787383in}{1.634777in}}%
\pgfpathlineto{\pgfqpoint{4.790604in}{1.637907in}}%
\pgfpathlineto{\pgfqpoint{4.791678in}{1.637996in}}%
\pgfpathlineto{\pgfqpoint{4.792752in}{1.635403in}}%
\pgfpathlineto{\pgfqpoint{4.793826in}{1.635224in}}%
\pgfpathlineto{\pgfqpoint{4.794899in}{1.634151in}}%
\pgfpathlineto{\pgfqpoint{4.798121in}{1.632720in}}%
\pgfpathlineto{\pgfqpoint{4.799195in}{1.632989in}}%
\pgfpathlineto{\pgfqpoint{4.805638in}{1.627534in}}%
\pgfpathlineto{\pgfqpoint{4.806712in}{1.629859in}}%
\pgfpathlineto{\pgfqpoint{4.807785in}{1.628339in}}%
\pgfpathlineto{\pgfqpoint{4.808859in}{1.625120in}}%
\pgfpathlineto{\pgfqpoint{4.809933in}{1.629054in}}%
\pgfpathlineto{\pgfqpoint{4.813155in}{1.629680in}}%
\pgfpathlineto{\pgfqpoint{4.816376in}{1.619755in}}%
\pgfpathlineto{\pgfqpoint{4.817450in}{1.617788in}}%
\pgfpathlineto{\pgfqpoint{4.820672in}{1.620649in}}%
\pgfpathlineto{\pgfqpoint{4.821745in}{1.615820in}}%
\pgfpathlineto{\pgfqpoint{4.822819in}{1.622080in}}%
\pgfpathlineto{\pgfqpoint{4.823893in}{1.621811in}}%
\pgfpathlineto{\pgfqpoint{4.824967in}{1.619397in}}%
\pgfpathlineto{\pgfqpoint{4.828188in}{1.622884in}}%
\pgfpathlineto{\pgfqpoint{4.829262in}{1.625567in}}%
\pgfpathlineto{\pgfqpoint{4.831410in}{1.626998in}}%
\pgfpathlineto{\pgfqpoint{4.836779in}{1.626729in}}%
\pgfpathlineto{\pgfqpoint{4.838927in}{1.625746in}}%
\pgfpathlineto{\pgfqpoint{4.840001in}{1.621722in}}%
\pgfpathlineto{\pgfqpoint{4.843222in}{1.623600in}}%
\pgfpathlineto{\pgfqpoint{4.844296in}{1.627802in}}%
\pgfpathlineto{\pgfqpoint{4.845370in}{1.625925in}}%
\pgfpathlineto{\pgfqpoint{4.846444in}{1.618950in}}%
\pgfpathlineto{\pgfqpoint{4.847517in}{1.620738in}}%
\pgfpathlineto{\pgfqpoint{4.850739in}{1.615731in}}%
\pgfpathlineto{\pgfqpoint{4.851813in}{1.615999in}}%
\pgfpathlineto{\pgfqpoint{4.852887in}{1.624315in}}%
\pgfpathlineto{\pgfqpoint{4.853961in}{1.626103in}}%
\pgfpathlineto{\pgfqpoint{4.860404in}{1.617788in}}%
\pgfpathlineto{\pgfqpoint{4.861477in}{1.621454in}}%
\pgfpathlineto{\pgfqpoint{4.862551in}{1.619844in}}%
\pgfpathlineto{\pgfqpoint{4.865773in}{1.620291in}}%
\pgfpathlineto{\pgfqpoint{4.866847in}{1.617966in}}%
\pgfpathlineto{\pgfqpoint{4.867920in}{1.620291in}}%
\pgfpathlineto{\pgfqpoint{4.868994in}{1.619844in}}%
\pgfpathlineto{\pgfqpoint{4.870068in}{1.622616in}}%
\pgfpathlineto{\pgfqpoint{4.873290in}{1.611349in}}%
\pgfpathlineto{\pgfqpoint{4.874363in}{1.614211in}}%
\pgfpathlineto{\pgfqpoint{4.875437in}{1.613317in}}%
\pgfpathlineto{\pgfqpoint{4.876511in}{1.613227in}}%
\pgfpathlineto{\pgfqpoint{4.877585in}{1.614121in}}%
\pgfpathlineto{\pgfqpoint{4.880806in}{1.614837in}}%
\pgfpathlineto{\pgfqpoint{4.882954in}{1.618056in}}%
\pgfpathlineto{\pgfqpoint{4.884028in}{1.617609in}}%
\pgfpathlineto{\pgfqpoint{4.885102in}{1.612065in}}%
\pgfpathlineto{\pgfqpoint{4.889397in}{1.608399in}}%
\pgfpathlineto{\pgfqpoint{4.890471in}{1.612244in}}%
\pgfpathlineto{\pgfqpoint{4.891545in}{1.623242in}}%
\pgfpathlineto{\pgfqpoint{4.892619in}{1.617340in}}%
\pgfpathlineto{\pgfqpoint{4.895840in}{1.610098in}}%
\pgfpathlineto{\pgfqpoint{4.897988in}{1.610992in}}%
\pgfpathlineto{\pgfqpoint{4.899062in}{1.618950in}}%
\pgfpathlineto{\pgfqpoint{4.900136in}{1.619486in}}%
\pgfpathlineto{\pgfqpoint{4.903357in}{1.617609in}}%
\pgfpathlineto{\pgfqpoint{4.904431in}{1.621722in}}%
\pgfpathlineto{\pgfqpoint{4.905505in}{1.618145in}}%
\pgfpathlineto{\pgfqpoint{4.906579in}{1.618682in}}%
\pgfpathlineto{\pgfqpoint{4.907652in}{1.616536in}}%
\pgfpathlineto{\pgfqpoint{4.910874in}{1.615731in}}%
\pgfpathlineto{\pgfqpoint{4.914095in}{1.608756in}}%
\pgfpathlineto{\pgfqpoint{4.918391in}{1.610634in}}%
\pgfpathlineto{\pgfqpoint{4.919465in}{1.613406in}}%
\pgfpathlineto{\pgfqpoint{4.920539in}{1.610992in}}%
\pgfpathlineto{\pgfqpoint{4.921612in}{1.616983in}}%
\pgfpathlineto{\pgfqpoint{4.922686in}{1.614837in}}%
\pgfpathlineto{\pgfqpoint{4.925908in}{1.615552in}}%
\pgfpathlineto{\pgfqpoint{4.926982in}{1.616983in}}%
\pgfpathlineto{\pgfqpoint{4.928055in}{1.615552in}}%
\pgfpathlineto{\pgfqpoint{4.929129in}{1.620291in}}%
\pgfpathlineto{\pgfqpoint{4.930203in}{1.618413in}}%
\pgfpathlineto{\pgfqpoint{4.933425in}{1.619486in}}%
\pgfpathlineto{\pgfqpoint{4.936646in}{1.623242in}}%
\pgfpathlineto{\pgfqpoint{4.937720in}{1.622706in}}%
\pgfpathlineto{\pgfqpoint{4.940941in}{1.623242in}}%
\pgfpathlineto{\pgfqpoint{4.942015in}{1.627624in}}%
\pgfpathlineto{\pgfqpoint{4.943089in}{1.626014in}}%
\pgfpathlineto{\pgfqpoint{4.945237in}{1.619665in}}%
\pgfpathlineto{\pgfqpoint{4.948458in}{1.621007in}}%
\pgfpathlineto{\pgfqpoint{4.949532in}{1.618592in}}%
\pgfpathlineto{\pgfqpoint{4.950606in}{1.619844in}}%
\pgfpathlineto{\pgfqpoint{4.951680in}{1.623331in}}%
\pgfpathlineto{\pgfqpoint{4.955975in}{1.626014in}}%
\pgfpathlineto{\pgfqpoint{4.957049in}{1.625567in}}%
\pgfpathlineto{\pgfqpoint{4.958123in}{1.622169in}}%
\pgfpathlineto{\pgfqpoint{4.959197in}{1.614032in}}%
\pgfpathlineto{\pgfqpoint{4.960271in}{1.611618in}}%
\pgfpathlineto{\pgfqpoint{4.964566in}{1.617519in}}%
\pgfpathlineto{\pgfqpoint{4.965640in}{1.617251in}}%
\pgfpathlineto{\pgfqpoint{4.966714in}{1.620649in}}%
\pgfpathlineto{\pgfqpoint{4.967787in}{1.619934in}}%
\pgfpathlineto{\pgfqpoint{4.972083in}{1.622437in}}%
\pgfpathlineto{\pgfqpoint{4.974230in}{1.629591in}}%
\pgfpathlineto{\pgfqpoint{4.975304in}{1.629591in}}%
\pgfpathlineto{\pgfqpoint{4.978526in}{1.627266in}}%
\pgfpathlineto{\pgfqpoint{4.979600in}{1.625120in}}%
\pgfpathlineto{\pgfqpoint{4.980673in}{1.626282in}}%
\pgfpathlineto{\pgfqpoint{4.981747in}{1.625925in}}%
\pgfpathlineto{\pgfqpoint{4.982821in}{1.633704in}}%
\pgfpathlineto{\pgfqpoint{4.986043in}{1.634241in}}%
\pgfpathlineto{\pgfqpoint{4.987116in}{1.630753in}}%
\pgfpathlineto{\pgfqpoint{4.989264in}{1.636387in}}%
\pgfpathlineto{\pgfqpoint{4.990338in}{1.638443in}}%
\pgfpathlineto{\pgfqpoint{4.994633in}{1.637549in}}%
\pgfpathlineto{\pgfqpoint{4.995707in}{1.638443in}}%
\pgfpathlineto{\pgfqpoint{4.996781in}{1.638264in}}%
\pgfpathlineto{\pgfqpoint{4.997855in}{1.639963in}}%
\pgfpathlineto{\pgfqpoint{5.001076in}{1.640589in}}%
\pgfpathlineto{\pgfqpoint{5.002150in}{1.635224in}}%
\pgfpathlineto{\pgfqpoint{5.003224in}{1.634241in}}%
\pgfpathlineto{\pgfqpoint{5.005372in}{1.636208in}}%
\pgfpathlineto{\pgfqpoint{5.008593in}{1.637191in}}%
\pgfpathlineto{\pgfqpoint{5.009667in}{1.636923in}}%
\pgfpathlineto{\pgfqpoint{5.010741in}{1.635850in}}%
\pgfpathlineto{\pgfqpoint{5.011815in}{1.633525in}}%
\pgfpathlineto{\pgfqpoint{5.012889in}{1.634509in}}%
\pgfpathlineto{\pgfqpoint{5.016110in}{1.635403in}}%
\pgfpathlineto{\pgfqpoint{5.017184in}{1.634688in}}%
\pgfpathlineto{\pgfqpoint{5.018258in}{1.636208in}}%
\pgfpathlineto{\pgfqpoint{5.019332in}{1.636565in}}%
\pgfpathlineto{\pgfqpoint{5.020405in}{1.635940in}}%
\pgfpathlineto{\pgfqpoint{5.023627in}{1.638354in}}%
\pgfpathlineto{\pgfqpoint{5.024701in}{1.635224in}}%
\pgfpathlineto{\pgfqpoint{5.025775in}{1.636118in}}%
\pgfpathlineto{\pgfqpoint{5.026849in}{1.634688in}}%
\pgfpathlineto{\pgfqpoint{5.027922in}{1.635492in}}%
\pgfpathlineto{\pgfqpoint{5.031144in}{1.632720in}}%
\pgfpathlineto{\pgfqpoint{5.033292in}{1.637907in}}%
\pgfpathlineto{\pgfqpoint{5.034365in}{1.638264in}}%
\pgfpathlineto{\pgfqpoint{5.038661in}{1.638533in}}%
\pgfpathlineto{\pgfqpoint{5.039735in}{1.635224in}}%
\pgfpathlineto{\pgfqpoint{5.040808in}{1.636208in}}%
\pgfpathlineto{\pgfqpoint{5.042956in}{1.646312in}}%
\pgfpathlineto{\pgfqpoint{5.046178in}{1.647832in}}%
\pgfpathlineto{\pgfqpoint{5.048325in}{1.650873in}}%
\pgfpathlineto{\pgfqpoint{5.049399in}{1.646312in}}%
\pgfpathlineto{\pgfqpoint{5.050473in}{1.649442in}}%
\pgfpathlineto{\pgfqpoint{5.053694in}{1.649084in}}%
\pgfpathlineto{\pgfqpoint{5.054768in}{1.651051in}}%
\pgfpathlineto{\pgfqpoint{5.055842in}{1.650515in}}%
\pgfpathlineto{\pgfqpoint{5.061211in}{1.655343in}}%
\pgfpathlineto{\pgfqpoint{5.062285in}{1.658115in}}%
\pgfpathlineto{\pgfqpoint{5.063359in}{1.656595in}}%
\pgfpathlineto{\pgfqpoint{5.064433in}{1.647028in}}%
\pgfpathlineto{\pgfqpoint{5.065507in}{1.642825in}}%
\pgfpathlineto{\pgfqpoint{5.068728in}{1.645597in}}%
\pgfpathlineto{\pgfqpoint{5.071950in}{1.634419in}}%
\pgfpathlineto{\pgfqpoint{5.073024in}{1.634777in}}%
\pgfpathlineto{\pgfqpoint{5.076245in}{1.634598in}}%
\pgfpathlineto{\pgfqpoint{5.078393in}{1.636387in}}%
\pgfpathlineto{\pgfqpoint{5.079467in}{1.636923in}}%
\pgfpathlineto{\pgfqpoint{5.080540in}{1.635492in}}%
\pgfpathlineto{\pgfqpoint{5.083762in}{1.635403in}}%
\pgfpathlineto{\pgfqpoint{5.084836in}{1.634777in}}%
\pgfpathlineto{\pgfqpoint{5.086983in}{1.636029in}}%
\pgfpathlineto{\pgfqpoint{5.088057in}{1.634330in}}%
\pgfpathlineto{\pgfqpoint{5.093427in}{1.639963in}}%
\pgfpathlineto{\pgfqpoint{5.094500in}{1.639874in}}%
\pgfpathlineto{\pgfqpoint{5.095574in}{1.642735in}}%
\pgfpathlineto{\pgfqpoint{5.099870in}{1.642378in}}%
\pgfpathlineto{\pgfqpoint{5.100943in}{1.643182in}}%
\pgfpathlineto{\pgfqpoint{5.102017in}{1.642020in}}%
\pgfpathlineto{\pgfqpoint{5.103091in}{1.643630in}}%
\pgfpathlineto{\pgfqpoint{5.106313in}{1.640679in}}%
\pgfpathlineto{\pgfqpoint{5.107386in}{1.636208in}}%
\pgfpathlineto{\pgfqpoint{5.108460in}{1.635135in}}%
\pgfpathlineto{\pgfqpoint{5.109534in}{1.637013in}}%
\pgfpathlineto{\pgfqpoint{5.110608in}{1.632631in}}%
\pgfpathlineto{\pgfqpoint{5.113829in}{1.633794in}}%
\pgfpathlineto{\pgfqpoint{5.118125in}{1.646580in}}%
\pgfpathlineto{\pgfqpoint{5.121346in}{1.645329in}}%
\pgfpathlineto{\pgfqpoint{5.122420in}{1.643451in}}%
\pgfpathlineto{\pgfqpoint{5.123494in}{1.644524in}}%
\pgfpathlineto{\pgfqpoint{5.124568in}{1.641305in}}%
\pgfpathlineto{\pgfqpoint{5.125642in}{1.642378in}}%
\pgfpathlineto{\pgfqpoint{5.128863in}{1.642288in}}%
\pgfpathlineto{\pgfqpoint{5.129937in}{1.643987in}}%
\pgfpathlineto{\pgfqpoint{5.131011in}{1.639963in}}%
\pgfpathlineto{\pgfqpoint{5.132085in}{1.638980in}}%
\pgfpathlineto{\pgfqpoint{5.133159in}{1.642020in}}%
\pgfpathlineto{\pgfqpoint{5.136380in}{1.644613in}}%
\pgfpathlineto{\pgfqpoint{5.137454in}{1.641752in}}%
\pgfpathlineto{\pgfqpoint{5.138528in}{1.647117in}}%
\pgfpathlineto{\pgfqpoint{5.139602in}{1.640321in}}%
\pgfpathlineto{\pgfqpoint{5.140675in}{1.640411in}}%
\pgfpathlineto{\pgfqpoint{5.147118in}{1.631290in}}%
\pgfpathlineto{\pgfqpoint{5.148192in}{1.634151in}}%
\pgfpathlineto{\pgfqpoint{5.152488in}{1.638622in}}%
\pgfpathlineto{\pgfqpoint{5.153561in}{1.635850in}}%
\pgfpathlineto{\pgfqpoint{5.154635in}{1.635224in}}%
\pgfpathlineto{\pgfqpoint{5.155709in}{1.639159in}}%
\pgfpathlineto{\pgfqpoint{5.158931in}{1.643898in}}%
\pgfpathlineto{\pgfqpoint{5.160004in}{1.647743in}}%
\pgfpathlineto{\pgfqpoint{5.161078in}{1.646849in}}%
\pgfpathlineto{\pgfqpoint{5.162152in}{1.647296in}}%
\pgfpathlineto{\pgfqpoint{5.163226in}{1.649799in}}%
\pgfpathlineto{\pgfqpoint{5.166448in}{1.650873in}}%
\pgfpathlineto{\pgfqpoint{5.169669in}{1.650068in}}%
\pgfpathlineto{\pgfqpoint{5.170743in}{1.654270in}}%
\pgfpathlineto{\pgfqpoint{5.175038in}{1.652393in}}%
\pgfpathlineto{\pgfqpoint{5.176112in}{1.653913in}}%
\pgfpathlineto{\pgfqpoint{5.178260in}{1.658831in}}%
\pgfpathlineto{\pgfqpoint{5.181481in}{1.658115in}}%
\pgfpathlineto{\pgfqpoint{5.182555in}{1.656864in}}%
\pgfpathlineto{\pgfqpoint{5.183629in}{1.651588in}}%
\pgfpathlineto{\pgfqpoint{5.184703in}{1.649442in}}%
\pgfpathlineto{\pgfqpoint{5.185777in}{1.649531in}}%
\pgfpathlineto{\pgfqpoint{5.190072in}{1.643272in}}%
\pgfpathlineto{\pgfqpoint{5.191146in}{1.648369in}}%
\pgfpathlineto{\pgfqpoint{5.193293in}{1.652303in}}%
\pgfpathlineto{\pgfqpoint{5.196515in}{1.648279in}}%
\pgfpathlineto{\pgfqpoint{5.197589in}{1.641662in}}%
\pgfpathlineto{\pgfqpoint{5.198663in}{1.639337in}}%
\pgfpathlineto{\pgfqpoint{5.199737in}{1.639248in}}%
\pgfpathlineto{\pgfqpoint{5.200810in}{1.637996in}}%
\pgfpathlineto{\pgfqpoint{5.204032in}{1.640142in}}%
\pgfpathlineto{\pgfqpoint{5.205106in}{1.625835in}}%
\pgfpathlineto{\pgfqpoint{5.206180in}{1.620560in}}%
\pgfpathlineto{\pgfqpoint{5.207253in}{1.621811in}}%
\pgfpathlineto{\pgfqpoint{5.208327in}{1.616178in}}%
\pgfpathlineto{\pgfqpoint{5.211549in}{1.615105in}}%
\pgfpathlineto{\pgfqpoint{5.212623in}{1.615820in}}%
\pgfpathlineto{\pgfqpoint{5.214770in}{1.626819in}}%
\pgfpathlineto{\pgfqpoint{5.215844in}{1.626551in}}%
\pgfpathlineto{\pgfqpoint{5.220139in}{1.631290in}}%
\pgfpathlineto{\pgfqpoint{5.221213in}{1.631290in}}%
\pgfpathlineto{\pgfqpoint{5.223361in}{1.632631in}}%
\pgfpathlineto{\pgfqpoint{5.226582in}{1.630396in}}%
\pgfpathlineto{\pgfqpoint{5.227656in}{1.628786in}}%
\pgfpathlineto{\pgfqpoint{5.228730in}{1.624852in}}%
\pgfpathlineto{\pgfqpoint{5.230878in}{1.626193in}}%
\pgfpathlineto{\pgfqpoint{5.234099in}{1.623331in}}%
\pgfpathlineto{\pgfqpoint{5.235173in}{1.626729in}}%
\pgfpathlineto{\pgfqpoint{5.236247in}{1.624583in}}%
\pgfpathlineto{\pgfqpoint{5.237321in}{1.631737in}}%
\pgfpathlineto{\pgfqpoint{5.238395in}{1.628697in}}%
\pgfpathlineto{\pgfqpoint{5.242690in}{1.631737in}}%
\pgfpathlineto{\pgfqpoint{5.243764in}{1.630217in}}%
\pgfpathlineto{\pgfqpoint{5.244838in}{1.631200in}}%
\pgfpathlineto{\pgfqpoint{5.245912in}{1.637639in}}%
\pgfpathlineto{\pgfqpoint{5.251281in}{1.639516in}}%
\pgfpathlineto{\pgfqpoint{5.253428in}{1.631558in}}%
\pgfpathlineto{\pgfqpoint{5.256650in}{1.630127in}}%
\pgfpathlineto{\pgfqpoint{5.258798in}{1.623510in}}%
\pgfpathlineto{\pgfqpoint{5.259871in}{1.623957in}}%
\pgfpathlineto{\pgfqpoint{5.260945in}{1.621007in}}%
\pgfpathlineto{\pgfqpoint{5.264167in}{1.630306in}}%
\pgfpathlineto{\pgfqpoint{5.265241in}{1.636476in}}%
\pgfpathlineto{\pgfqpoint{5.266315in}{1.636297in}}%
\pgfpathlineto{\pgfqpoint{5.267388in}{1.636744in}}%
\pgfpathlineto{\pgfqpoint{5.268462in}{1.647385in}}%
\pgfpathlineto{\pgfqpoint{5.271684in}{1.645686in}}%
\pgfpathlineto{\pgfqpoint{5.273831in}{1.650962in}}%
\pgfpathlineto{\pgfqpoint{5.275979in}{1.647296in}}%
\pgfpathlineto{\pgfqpoint{5.280274in}{1.646312in}}%
\pgfpathlineto{\pgfqpoint{5.281348in}{1.644613in}}%
\pgfpathlineto{\pgfqpoint{5.282422in}{1.644345in}}%
\pgfpathlineto{\pgfqpoint{5.283496in}{1.644792in}}%
\pgfpathlineto{\pgfqpoint{5.286717in}{1.643361in}}%
\pgfpathlineto{\pgfqpoint{5.287791in}{1.646580in}}%
\pgfpathlineto{\pgfqpoint{5.288865in}{1.646491in}}%
\pgfpathlineto{\pgfqpoint{5.291013in}{1.648279in}}%
\pgfpathlineto{\pgfqpoint{5.295308in}{1.648995in}}%
\pgfpathlineto{\pgfqpoint{5.296382in}{1.645507in}}%
\pgfpathlineto{\pgfqpoint{5.297456in}{1.644434in}}%
\pgfpathlineto{\pgfqpoint{5.298530in}{1.639874in}}%
\pgfpathlineto{\pgfqpoint{5.301751in}{1.639427in}}%
\pgfpathlineto{\pgfqpoint{5.302825in}{1.634151in}}%
\pgfpathlineto{\pgfqpoint{5.303899in}{1.635403in}}%
\pgfpathlineto{\pgfqpoint{5.304973in}{1.643093in}}%
\pgfpathlineto{\pgfqpoint{5.306047in}{1.643808in}}%
\pgfpathlineto{\pgfqpoint{5.309268in}{1.647028in}}%
\pgfpathlineto{\pgfqpoint{5.310342in}{1.644613in}}%
\pgfpathlineto{\pgfqpoint{5.311416in}{1.648995in}}%
\pgfpathlineto{\pgfqpoint{5.312490in}{1.647206in}}%
\pgfpathlineto{\pgfqpoint{5.313563in}{1.649084in}}%
\pgfpathlineto{\pgfqpoint{5.316785in}{1.649710in}}%
\pgfpathlineto{\pgfqpoint{5.317859in}{1.648011in}}%
\pgfpathlineto{\pgfqpoint{5.320006in}{1.640768in}}%
\pgfpathlineto{\pgfqpoint{5.321080in}{1.641841in}}%
\pgfpathlineto{\pgfqpoint{5.324302in}{1.645865in}}%
\pgfpathlineto{\pgfqpoint{5.325376in}{1.642288in}}%
\pgfpathlineto{\pgfqpoint{5.326449in}{1.644434in}}%
\pgfpathlineto{\pgfqpoint{5.327523in}{1.648458in}}%
\pgfpathlineto{\pgfqpoint{5.331819in}{1.649621in}}%
\pgfpathlineto{\pgfqpoint{5.332893in}{1.647028in}}%
\pgfpathlineto{\pgfqpoint{5.333966in}{1.649978in}}%
\pgfpathlineto{\pgfqpoint{5.335040in}{1.649084in}}%
\pgfpathlineto{\pgfqpoint{5.336114in}{1.650694in}}%
\pgfpathlineto{\pgfqpoint{5.339336in}{1.649352in}}%
\pgfpathlineto{\pgfqpoint{5.341483in}{1.651946in}}%
\pgfpathlineto{\pgfqpoint{5.342557in}{1.651051in}}%
\pgfpathlineto{\pgfqpoint{5.343631in}{1.648279in}}%
\pgfpathlineto{\pgfqpoint{5.346852in}{1.651856in}}%
\pgfpathlineto{\pgfqpoint{5.347926in}{1.650336in}}%
\pgfpathlineto{\pgfqpoint{5.350074in}{1.656774in}}%
\pgfpathlineto{\pgfqpoint{5.351148in}{1.656595in}}%
\pgfpathlineto{\pgfqpoint{5.354369in}{1.657042in}}%
\pgfpathlineto{\pgfqpoint{5.355443in}{1.660440in}}%
\pgfpathlineto{\pgfqpoint{5.357591in}{1.659636in}}%
\pgfpathlineto{\pgfqpoint{5.358665in}{1.659457in}}%
\pgfpathlineto{\pgfqpoint{5.361886in}{1.660262in}}%
\pgfpathlineto{\pgfqpoint{5.364034in}{1.653734in}}%
\pgfpathlineto{\pgfqpoint{5.365108in}{1.654449in}}%
\pgfpathlineto{\pgfqpoint{5.366181in}{1.657490in}}%
\pgfpathlineto{\pgfqpoint{5.370477in}{1.653645in}}%
\pgfpathlineto{\pgfqpoint{5.371551in}{1.654449in}}%
\pgfpathlineto{\pgfqpoint{5.372625in}{1.656238in}}%
\pgfpathlineto{\pgfqpoint{5.373698in}{1.654896in}}%
\pgfpathlineto{\pgfqpoint{5.377994in}{1.653108in}}%
\pgfpathlineto{\pgfqpoint{5.380141in}{1.655701in}}%
\pgfpathlineto{\pgfqpoint{5.381215in}{1.653555in}}%
\pgfpathlineto{\pgfqpoint{5.385511in}{1.652124in}}%
\pgfpathlineto{\pgfqpoint{5.386584in}{1.653287in}}%
\pgfpathlineto{\pgfqpoint{5.388732in}{1.652303in}}%
\pgfpathlineto{\pgfqpoint{5.394101in}{1.649621in}}%
\pgfpathlineto{\pgfqpoint{5.396249in}{1.636029in}}%
\pgfpathlineto{\pgfqpoint{5.399470in}{1.637549in}}%
\pgfpathlineto{\pgfqpoint{5.400544in}{1.636834in}}%
\pgfpathlineto{\pgfqpoint{5.401618in}{1.637817in}}%
\pgfpathlineto{\pgfqpoint{5.402692in}{1.639874in}}%
\pgfpathlineto{\pgfqpoint{5.403766in}{1.636118in}}%
\pgfpathlineto{\pgfqpoint{5.406987in}{1.634330in}}%
\pgfpathlineto{\pgfqpoint{5.408061in}{1.637370in}}%
\pgfpathlineto{\pgfqpoint{5.409135in}{1.636297in}}%
\pgfpathlineto{\pgfqpoint{5.410209in}{1.639963in}}%
\pgfpathlineto{\pgfqpoint{5.411283in}{1.637728in}}%
\pgfpathlineto{\pgfqpoint{5.414504in}{1.638175in}}%
\pgfpathlineto{\pgfqpoint{5.415578in}{1.639963in}}%
\pgfpathlineto{\pgfqpoint{5.416652in}{1.636476in}}%
\pgfpathlineto{\pgfqpoint{5.418800in}{1.638890in}}%
\pgfpathlineto{\pgfqpoint{5.423095in}{1.631469in}}%
\pgfpathlineto{\pgfqpoint{5.425243in}{1.635761in}}%
\pgfpathlineto{\pgfqpoint{5.429538in}{1.634598in}}%
\pgfpathlineto{\pgfqpoint{5.430612in}{1.636118in}}%
\pgfpathlineto{\pgfqpoint{5.431686in}{1.635224in}}%
\pgfpathlineto{\pgfqpoint{5.432759in}{1.632899in}}%
\pgfpathlineto{\pgfqpoint{5.433833in}{1.638354in}}%
\pgfpathlineto{\pgfqpoint{5.437055in}{1.639874in}}%
\pgfpathlineto{\pgfqpoint{5.438129in}{1.641484in}}%
\pgfpathlineto{\pgfqpoint{5.439203in}{1.640858in}}%
\pgfpathlineto{\pgfqpoint{5.440276in}{1.644613in}}%
\pgfpathlineto{\pgfqpoint{5.441350in}{1.642825in}}%
\pgfpathlineto{\pgfqpoint{5.444572in}{1.646580in}}%
\pgfpathlineto{\pgfqpoint{5.445646in}{1.638175in}}%
\pgfpathlineto{\pgfqpoint{5.446719in}{1.634330in}}%
\pgfpathlineto{\pgfqpoint{5.447793in}{1.633525in}}%
\pgfpathlineto{\pgfqpoint{5.448867in}{1.631200in}}%
\pgfpathlineto{\pgfqpoint{5.452089in}{1.629680in}}%
\pgfpathlineto{\pgfqpoint{5.453162in}{1.630127in}}%
\pgfpathlineto{\pgfqpoint{5.454236in}{1.635135in}}%
\pgfpathlineto{\pgfqpoint{5.456384in}{1.636834in}}%
\pgfpathlineto{\pgfqpoint{5.459605in}{1.638175in}}%
\pgfpathlineto{\pgfqpoint{5.460679in}{1.635940in}}%
\pgfpathlineto{\pgfqpoint{5.462827in}{1.635671in}}%
\pgfpathlineto{\pgfqpoint{5.463901in}{1.633615in}}%
\pgfpathlineto{\pgfqpoint{5.469270in}{1.644703in}}%
\pgfpathlineto{\pgfqpoint{5.471418in}{1.642020in}}%
\pgfpathlineto{\pgfqpoint{5.475713in}{1.641841in}}%
\pgfpathlineto{\pgfqpoint{5.476787in}{1.641662in}}%
\pgfpathlineto{\pgfqpoint{5.482156in}{1.621543in}}%
\pgfpathlineto{\pgfqpoint{5.483230in}{1.612333in}}%
\pgfpathlineto{\pgfqpoint{5.485378in}{1.632363in}}%
\pgfpathlineto{\pgfqpoint{5.486451in}{1.631469in}}%
\pgfpathlineto{\pgfqpoint{5.489673in}{1.631022in}}%
\pgfpathlineto{\pgfqpoint{5.490747in}{1.622706in}}%
\pgfpathlineto{\pgfqpoint{5.492894in}{1.628875in}}%
\pgfpathlineto{\pgfqpoint{5.493968in}{1.622169in}}%
\pgfpathlineto{\pgfqpoint{5.498264in}{1.630038in}}%
\pgfpathlineto{\pgfqpoint{5.499337in}{1.626372in}}%
\pgfpathlineto{\pgfqpoint{5.500411in}{1.626908in}}%
\pgfpathlineto{\pgfqpoint{5.501485in}{1.628965in}}%
\pgfpathlineto{\pgfqpoint{5.504707in}{1.628339in}}%
\pgfpathlineto{\pgfqpoint{5.505781in}{1.633704in}}%
\pgfpathlineto{\pgfqpoint{5.506854in}{1.632363in}}%
\pgfpathlineto{\pgfqpoint{5.509002in}{1.620291in}}%
\pgfpathlineto{\pgfqpoint{5.512224in}{1.621990in}}%
\pgfpathlineto{\pgfqpoint{5.514371in}{1.615910in}}%
\pgfpathlineto{\pgfqpoint{5.516519in}{1.617698in}}%
\pgfpathlineto{\pgfqpoint{5.521888in}{1.612333in}}%
\pgfpathlineto{\pgfqpoint{5.522962in}{1.608309in}}%
\pgfpathlineto{\pgfqpoint{5.524036in}{1.607415in}}%
\pgfpathlineto{\pgfqpoint{5.527257in}{1.615999in}}%
\pgfpathlineto{\pgfqpoint{5.528331in}{1.616446in}}%
\pgfpathlineto{\pgfqpoint{5.530479in}{1.621990in}}%
\pgfpathlineto{\pgfqpoint{5.531553in}{1.621454in}}%
\pgfpathlineto{\pgfqpoint{5.535848in}{1.622974in}}%
\pgfpathlineto{\pgfqpoint{5.536922in}{1.620202in}}%
\pgfpathlineto{\pgfqpoint{5.537996in}{1.625388in}}%
\pgfpathlineto{\pgfqpoint{5.542291in}{1.625567in}}%
\pgfpathlineto{\pgfqpoint{5.543365in}{1.629680in}}%
\pgfpathlineto{\pgfqpoint{5.544439in}{1.626819in}}%
\pgfpathlineto{\pgfqpoint{5.545513in}{1.634598in}}%
\pgfpathlineto{\pgfqpoint{5.546586in}{1.636565in}}%
\pgfpathlineto{\pgfqpoint{5.549808in}{1.638086in}}%
\pgfpathlineto{\pgfqpoint{5.550882in}{1.636655in}}%
\pgfpathlineto{\pgfqpoint{5.551956in}{1.638980in}}%
\pgfpathlineto{\pgfqpoint{5.553029in}{1.638443in}}%
\pgfpathlineto{\pgfqpoint{5.554103in}{1.642020in}}%
\pgfpathlineto{\pgfqpoint{5.557325in}{1.641305in}}%
\pgfpathlineto{\pgfqpoint{5.559472in}{1.636476in}}%
\pgfpathlineto{\pgfqpoint{5.560546in}{1.636923in}}%
\pgfpathlineto{\pgfqpoint{5.561620in}{1.633704in}}%
\pgfpathlineto{\pgfqpoint{5.565915in}{1.628607in}}%
\pgfpathlineto{\pgfqpoint{5.566989in}{1.630217in}}%
\pgfpathlineto{\pgfqpoint{5.569137in}{1.621990in}}%
\pgfpathlineto{\pgfqpoint{5.572358in}{1.628160in}}%
\pgfpathlineto{\pgfqpoint{5.573432in}{1.628428in}}%
\pgfpathlineto{\pgfqpoint{5.575580in}{1.633615in}}%
\pgfpathlineto{\pgfqpoint{5.576654in}{1.630753in}}%
\pgfpathlineto{\pgfqpoint{5.579875in}{1.627802in}}%
\pgfpathlineto{\pgfqpoint{5.580949in}{1.629233in}}%
\pgfpathlineto{\pgfqpoint{5.582023in}{1.627266in}}%
\pgfpathlineto{\pgfqpoint{5.584171in}{1.629591in}}%
\pgfpathlineto{\pgfqpoint{5.588466in}{1.632184in}}%
\pgfpathlineto{\pgfqpoint{5.590614in}{1.624494in}}%
\pgfpathlineto{\pgfqpoint{5.591688in}{1.633168in}}%
\pgfpathlineto{\pgfqpoint{5.594909in}{1.635850in}}%
\pgfpathlineto{\pgfqpoint{5.597057in}{1.630574in}}%
\pgfpathlineto{\pgfqpoint{5.598131in}{1.630217in}}%
\pgfpathlineto{\pgfqpoint{5.599204in}{1.626461in}}%
\pgfpathlineto{\pgfqpoint{5.603500in}{1.632005in}}%
\pgfpathlineto{\pgfqpoint{5.604574in}{1.639248in}}%
\pgfpathlineto{\pgfqpoint{5.606721in}{1.632095in}}%
\pgfpathlineto{\pgfqpoint{5.609943in}{1.634598in}}%
\pgfpathlineto{\pgfqpoint{5.612091in}{1.642557in}}%
\pgfpathlineto{\pgfqpoint{5.613164in}{1.640768in}}%
\pgfpathlineto{\pgfqpoint{5.617460in}{1.641036in}}%
\pgfpathlineto{\pgfqpoint{5.618534in}{1.644524in}}%
\pgfpathlineto{\pgfqpoint{5.620681in}{1.637013in}}%
\pgfpathlineto{\pgfqpoint{5.624977in}{1.634419in}}%
\pgfpathlineto{\pgfqpoint{5.626050in}{1.639159in}}%
\pgfpathlineto{\pgfqpoint{5.629272in}{1.630753in}}%
\pgfpathlineto{\pgfqpoint{5.632493in}{1.632720in}}%
\pgfpathlineto{\pgfqpoint{5.633567in}{1.631469in}}%
\pgfpathlineto{\pgfqpoint{5.634641in}{1.625478in}}%
\pgfpathlineto{\pgfqpoint{5.635715in}{1.631022in}}%
\pgfpathlineto{\pgfqpoint{5.636789in}{1.627624in}}%
\pgfpathlineto{\pgfqpoint{5.641084in}{1.631022in}}%
\pgfpathlineto{\pgfqpoint{5.642158in}{1.627624in}}%
\pgfpathlineto{\pgfqpoint{5.644306in}{1.647653in}}%
\pgfpathlineto{\pgfqpoint{5.647527in}{1.647564in}}%
\pgfpathlineto{\pgfqpoint{5.649675in}{1.662855in}}%
\pgfpathlineto{\pgfqpoint{5.650749in}{1.662676in}}%
\pgfpathlineto{\pgfqpoint{5.651823in}{1.670008in}}%
\pgfpathlineto{\pgfqpoint{5.655044in}{1.676089in}}%
\pgfpathlineto{\pgfqpoint{5.656118in}{1.669561in}}%
\pgfpathlineto{\pgfqpoint{5.657192in}{1.675016in}}%
\pgfpathlineto{\pgfqpoint{5.658266in}{1.673496in}}%
\pgfpathlineto{\pgfqpoint{5.659339in}{1.677698in}}%
\pgfpathlineto{\pgfqpoint{5.662561in}{1.675910in}}%
\pgfpathlineto{\pgfqpoint{5.663635in}{1.671349in}}%
\pgfpathlineto{\pgfqpoint{5.664709in}{1.670098in}}%
\pgfpathlineto{\pgfqpoint{5.665782in}{1.665537in}}%
\pgfpathlineto{\pgfqpoint{5.666856in}{1.671081in}}%
\pgfpathlineto{\pgfqpoint{5.672225in}{1.672691in}}%
\pgfpathlineto{\pgfqpoint{5.673299in}{1.677430in}}%
\pgfpathlineto{\pgfqpoint{5.674373in}{1.676804in}}%
\pgfpathlineto{\pgfqpoint{5.677595in}{1.678414in}}%
\pgfpathlineto{\pgfqpoint{5.678669in}{1.675016in}}%
\pgfpathlineto{\pgfqpoint{5.680816in}{1.678682in}}%
\pgfpathlineto{\pgfqpoint{5.685112in}{1.675820in}}%
\pgfpathlineto{\pgfqpoint{5.687259in}{1.686461in}}%
\pgfpathlineto{\pgfqpoint{5.688333in}{1.684673in}}%
\pgfpathlineto{\pgfqpoint{5.689407in}{1.684047in}}%
\pgfpathlineto{\pgfqpoint{5.692628in}{1.687177in}}%
\pgfpathlineto{\pgfqpoint{5.693702in}{1.689054in}}%
\pgfpathlineto{\pgfqpoint{5.695850in}{1.687981in}}%
\pgfpathlineto{\pgfqpoint{5.696924in}{1.689591in}}%
\pgfpathlineto{\pgfqpoint{5.700145in}{1.689680in}}%
\pgfpathlineto{\pgfqpoint{5.701219in}{1.690664in}}%
\pgfpathlineto{\pgfqpoint{5.703367in}{1.697996in}}%
\pgfpathlineto{\pgfqpoint{5.704441in}{1.695045in}}%
\pgfpathlineto{\pgfqpoint{5.707662in}{1.696566in}}%
\pgfpathlineto{\pgfqpoint{5.709810in}{1.692452in}}%
\pgfpathlineto{\pgfqpoint{5.710884in}{1.697460in}}%
\pgfpathlineto{\pgfqpoint{5.715179in}{1.696208in}}%
\pgfpathlineto{\pgfqpoint{5.716253in}{1.701215in}}%
\pgfpathlineto{\pgfqpoint{5.719474in}{1.700858in}}%
\pgfpathlineto{\pgfqpoint{5.722696in}{1.703987in}}%
\pgfpathlineto{\pgfqpoint{5.723770in}{1.701484in}}%
\pgfpathlineto{\pgfqpoint{5.724844in}{1.701484in}}%
\pgfpathlineto{\pgfqpoint{5.725917in}{1.689770in}}%
\pgfpathlineto{\pgfqpoint{5.726991in}{1.691111in}}%
\pgfpathlineto{\pgfqpoint{5.730213in}{1.686729in}}%
\pgfpathlineto{\pgfqpoint{5.731287in}{1.689323in}}%
\pgfpathlineto{\pgfqpoint{5.732360in}{1.684226in}}%
\pgfpathlineto{\pgfqpoint{5.733434in}{1.684762in}}%
\pgfpathlineto{\pgfqpoint{5.734508in}{1.684673in}}%
\pgfpathlineto{\pgfqpoint{5.737730in}{1.687624in}}%
\pgfpathlineto{\pgfqpoint{5.738803in}{1.690306in}}%
\pgfpathlineto{\pgfqpoint{5.739877in}{1.687803in}}%
\pgfpathlineto{\pgfqpoint{5.740951in}{1.674479in}}%
\pgfpathlineto{\pgfqpoint{5.742025in}{1.678503in}}%
\pgfpathlineto{\pgfqpoint{5.745246in}{1.680112in}}%
\pgfpathlineto{\pgfqpoint{5.746320in}{1.677698in}}%
\pgfpathlineto{\pgfqpoint{5.747394in}{1.687355in}}%
\pgfpathlineto{\pgfqpoint{5.748468in}{1.682169in}}%
\pgfpathlineto{\pgfqpoint{5.749542in}{1.681543in}}%
\pgfpathlineto{\pgfqpoint{5.752763in}{1.684494in}}%
\pgfpathlineto{\pgfqpoint{5.753837in}{1.679487in}}%
\pgfpathlineto{\pgfqpoint{5.754911in}{1.680738in}}%
\pgfpathlineto{\pgfqpoint{5.755985in}{1.680738in}}%
\pgfpathlineto{\pgfqpoint{5.757059in}{1.682884in}}%
\pgfpathlineto{\pgfqpoint{5.760280in}{1.682616in}}%
\pgfpathlineto{\pgfqpoint{5.761354in}{1.686193in}}%
\pgfpathlineto{\pgfqpoint{5.762428in}{1.683153in}}%
\pgfpathlineto{\pgfqpoint{5.763502in}{1.685656in}}%
\pgfpathlineto{\pgfqpoint{5.764576in}{1.681543in}}%
\pgfpathlineto{\pgfqpoint{5.767797in}{1.683779in}}%
\pgfpathlineto{\pgfqpoint{5.769945in}{1.677251in}}%
\pgfpathlineto{\pgfqpoint{5.771019in}{1.671439in}}%
\pgfpathlineto{\pgfqpoint{5.772092in}{1.671618in}}%
\pgfpathlineto{\pgfqpoint{5.775314in}{1.667594in}}%
\pgfpathlineto{\pgfqpoint{5.777462in}{1.673138in}}%
\pgfpathlineto{\pgfqpoint{5.779609in}{1.679039in}}%
\pgfpathlineto{\pgfqpoint{5.783905in}{1.681186in}}%
\pgfpathlineto{\pgfqpoint{5.784979in}{1.677609in}}%
\pgfpathlineto{\pgfqpoint{5.787126in}{1.681364in}}%
\pgfpathlineto{\pgfqpoint{5.790348in}{1.679755in}}%
\pgfpathlineto{\pgfqpoint{5.791422in}{1.687803in}}%
\pgfpathlineto{\pgfqpoint{5.792495in}{1.686014in}}%
\pgfpathlineto{\pgfqpoint{5.793569in}{1.689323in}}%
\pgfpathlineto{\pgfqpoint{5.794643in}{1.694867in}}%
\pgfpathlineto{\pgfqpoint{5.797865in}{1.694151in}}%
\pgfpathlineto{\pgfqpoint{5.798938in}{1.697370in}}%
\pgfpathlineto{\pgfqpoint{5.800012in}{1.696208in}}%
\pgfpathlineto{\pgfqpoint{5.802160in}{1.703451in}}%
\pgfpathlineto{\pgfqpoint{5.805381in}{1.703361in}}%
\pgfpathlineto{\pgfqpoint{5.806455in}{1.705955in}}%
\pgfpathlineto{\pgfqpoint{5.807529in}{1.705418in}}%
\pgfpathlineto{\pgfqpoint{5.808603in}{1.710336in}}%
\pgfpathlineto{\pgfqpoint{5.809677in}{1.708458in}}%
\pgfpathlineto{\pgfqpoint{5.813972in}{1.711499in}}%
\pgfpathlineto{\pgfqpoint{5.815046in}{1.713376in}}%
\pgfpathlineto{\pgfqpoint{5.817194in}{1.722408in}}%
\pgfpathlineto{\pgfqpoint{5.821489in}{1.724733in}}%
\pgfpathlineto{\pgfqpoint{5.822563in}{1.727057in}}%
\pgfpathlineto{\pgfqpoint{5.823637in}{1.720172in}}%
\pgfpathlineto{\pgfqpoint{5.824711in}{1.724196in}}%
\pgfpathlineto{\pgfqpoint{5.827932in}{1.724464in}}%
\pgfpathlineto{\pgfqpoint{5.829006in}{1.720888in}}%
\pgfpathlineto{\pgfqpoint{5.830080in}{1.725001in}}%
\pgfpathlineto{\pgfqpoint{5.831154in}{1.723749in}}%
\pgfpathlineto{\pgfqpoint{5.832227in}{1.723749in}}%
\pgfpathlineto{\pgfqpoint{5.835449in}{1.724464in}}%
\pgfpathlineto{\pgfqpoint{5.836523in}{1.722676in}}%
\pgfpathlineto{\pgfqpoint{5.837597in}{1.722050in}}%
\pgfpathlineto{\pgfqpoint{5.838670in}{1.720083in}}%
\pgfpathlineto{\pgfqpoint{5.839744in}{1.725806in}}%
\pgfpathlineto{\pgfqpoint{5.842966in}{1.724017in}}%
\pgfpathlineto{\pgfqpoint{5.844040in}{1.715701in}}%
\pgfpathlineto{\pgfqpoint{5.845113in}{1.719725in}}%
\pgfpathlineto{\pgfqpoint{5.846187in}{1.716059in}}%
\pgfpathlineto{\pgfqpoint{5.847261in}{1.720351in}}%
\pgfpathlineto{\pgfqpoint{5.850483in}{1.713287in}}%
\pgfpathlineto{\pgfqpoint{5.851557in}{1.709352in}}%
\pgfpathlineto{\pgfqpoint{5.852630in}{1.708637in}}%
\pgfpathlineto{\pgfqpoint{5.853704in}{1.708816in}}%
\pgfpathlineto{\pgfqpoint{5.854778in}{1.706580in}}%
\pgfpathlineto{\pgfqpoint{5.858000in}{1.706223in}}%
\pgfpathlineto{\pgfqpoint{5.859073in}{1.706759in}}%
\pgfpathlineto{\pgfqpoint{5.861221in}{1.708279in}}%
\pgfpathlineto{\pgfqpoint{5.862295in}{1.706670in}}%
\pgfpathlineto{\pgfqpoint{5.865516in}{1.706312in}}%
\pgfpathlineto{\pgfqpoint{5.866590in}{1.699695in}}%
\pgfpathlineto{\pgfqpoint{5.867664in}{1.702914in}}%
\pgfpathlineto{\pgfqpoint{5.869812in}{1.697281in}}%
\pgfpathlineto{\pgfqpoint{5.874107in}{1.698712in}}%
\pgfpathlineto{\pgfqpoint{5.875181in}{1.697907in}}%
\pgfpathlineto{\pgfqpoint{5.876255in}{1.699606in}}%
\pgfpathlineto{\pgfqpoint{5.877329in}{1.694330in}}%
\pgfpathlineto{\pgfqpoint{5.880550in}{1.697639in}}%
\pgfpathlineto{\pgfqpoint{5.881624in}{1.695850in}}%
\pgfpathlineto{\pgfqpoint{5.882698in}{1.696387in}}%
\pgfpathlineto{\pgfqpoint{5.884846in}{1.700679in}}%
\pgfpathlineto{\pgfqpoint{5.890215in}{1.707117in}}%
\pgfpathlineto{\pgfqpoint{5.891289in}{1.706223in}}%
\pgfpathlineto{\pgfqpoint{5.892362in}{1.692363in}}%
\pgfpathlineto{\pgfqpoint{5.895584in}{1.698265in}}%
\pgfpathlineto{\pgfqpoint{5.896658in}{1.689501in}}%
\pgfpathlineto{\pgfqpoint{5.897732in}{1.689770in}}%
\pgfpathlineto{\pgfqpoint{5.898805in}{1.693615in}}%
\pgfpathlineto{\pgfqpoint{5.899879in}{1.692810in}}%
\pgfpathlineto{\pgfqpoint{5.903101in}{1.687534in}}%
\pgfpathlineto{\pgfqpoint{5.904175in}{1.688071in}}%
\pgfpathlineto{\pgfqpoint{5.906322in}{1.696476in}}%
\pgfpathlineto{\pgfqpoint{5.907396in}{1.698175in}}%
\pgfpathlineto{\pgfqpoint{5.910618in}{1.694956in}}%
\pgfpathlineto{\pgfqpoint{5.911691in}{1.697639in}}%
\pgfpathlineto{\pgfqpoint{5.912765in}{1.694241in}}%
\pgfpathlineto{\pgfqpoint{5.913839in}{1.694688in}}%
\pgfpathlineto{\pgfqpoint{5.914913in}{1.693615in}}%
\pgfpathlineto{\pgfqpoint{5.918134in}{1.692810in}}%
\pgfpathlineto{\pgfqpoint{5.920282in}{1.684762in}}%
\pgfpathlineto{\pgfqpoint{5.921356in}{1.684673in}}%
\pgfpathlineto{\pgfqpoint{5.922430in}{1.681990in}}%
\pgfpathlineto{\pgfqpoint{5.925651in}{1.684136in}}%
\pgfpathlineto{\pgfqpoint{5.926725in}{1.681811in}}%
\pgfpathlineto{\pgfqpoint{5.927799in}{1.685031in}}%
\pgfpathlineto{\pgfqpoint{5.929947in}{1.684852in}}%
\pgfpathlineto{\pgfqpoint{5.933168in}{1.686014in}}%
\pgfpathlineto{\pgfqpoint{5.934242in}{1.684762in}}%
\pgfpathlineto{\pgfqpoint{5.935316in}{1.685656in}}%
\pgfpathlineto{\pgfqpoint{5.937464in}{1.668399in}}%
\pgfpathlineto{\pgfqpoint{5.940685in}{1.668488in}}%
\pgfpathlineto{\pgfqpoint{5.942833in}{1.663928in}}%
\pgfpathlineto{\pgfqpoint{5.943907in}{1.671081in}}%
\pgfpathlineto{\pgfqpoint{5.944980in}{1.668488in}}%
\pgfpathlineto{\pgfqpoint{5.948202in}{1.667594in}}%
\pgfpathlineto{\pgfqpoint{5.949276in}{1.664196in}}%
\pgfpathlineto{\pgfqpoint{5.950350in}{1.658473in}}%
\pgfpathlineto{\pgfqpoint{5.951423in}{1.657937in}}%
\pgfpathlineto{\pgfqpoint{5.952497in}{1.659546in}}%
\pgfpathlineto{\pgfqpoint{5.956793in}{1.664107in}}%
\pgfpathlineto{\pgfqpoint{5.957867in}{1.665716in}}%
\pgfpathlineto{\pgfqpoint{5.958940in}{1.656506in}}%
\pgfpathlineto{\pgfqpoint{5.960014in}{1.656506in}}%
\pgfpathlineto{\pgfqpoint{5.963236in}{1.652482in}}%
\pgfpathlineto{\pgfqpoint{5.964310in}{1.661871in}}%
\pgfpathlineto{\pgfqpoint{5.965383in}{1.666253in}}%
\pgfpathlineto{\pgfqpoint{5.966457in}{1.665537in}}%
\pgfpathlineto{\pgfqpoint{5.967531in}{1.667415in}}%
\pgfpathlineto{\pgfqpoint{5.970753in}{1.669293in}}%
\pgfpathlineto{\pgfqpoint{5.972900in}{1.684405in}}%
\pgfpathlineto{\pgfqpoint{5.975048in}{1.687892in}}%
\pgfpathlineto{\pgfqpoint{5.978269in}{1.691469in}}%
\pgfpathlineto{\pgfqpoint{5.979343in}{1.690217in}}%
\pgfpathlineto{\pgfqpoint{5.980417in}{1.681811in}}%
\pgfpathlineto{\pgfqpoint{5.985786in}{1.680649in}}%
\pgfpathlineto{\pgfqpoint{5.986860in}{1.685478in}}%
\pgfpathlineto{\pgfqpoint{5.987934in}{1.693525in}}%
\pgfpathlineto{\pgfqpoint{5.989008in}{1.691558in}}%
\pgfpathlineto{\pgfqpoint{5.990082in}{1.694420in}}%
\pgfpathlineto{\pgfqpoint{5.993303in}{1.696476in}}%
\pgfpathlineto{\pgfqpoint{5.994377in}{1.701215in}}%
\pgfpathlineto{\pgfqpoint{5.995451in}{1.695493in}}%
\pgfpathlineto{\pgfqpoint{5.996525in}{1.696923in}}%
\pgfpathlineto{\pgfqpoint{5.997599in}{1.700500in}}%
\pgfpathlineto{\pgfqpoint{6.001894in}{1.707206in}}%
\pgfpathlineto{\pgfqpoint{6.002968in}{1.706044in}}%
\pgfpathlineto{\pgfqpoint{6.004042in}{1.711409in}}%
\pgfpathlineto{\pgfqpoint{6.005115in}{1.711677in}}%
\pgfpathlineto{\pgfqpoint{6.009411in}{1.711320in}}%
\pgfpathlineto{\pgfqpoint{6.010485in}{1.709800in}}%
\pgfpathlineto{\pgfqpoint{6.011558in}{1.712124in}}%
\pgfpathlineto{\pgfqpoint{6.012632in}{1.709263in}}%
\pgfpathlineto{\pgfqpoint{6.016928in}{1.718741in}}%
\pgfpathlineto{\pgfqpoint{6.018001in}{1.718294in}}%
\pgfpathlineto{\pgfqpoint{6.019075in}{1.719278in}}%
\pgfpathlineto{\pgfqpoint{6.020149in}{1.712840in}}%
\pgfpathlineto{\pgfqpoint{6.023371in}{1.708190in}}%
\pgfpathlineto{\pgfqpoint{6.024445in}{1.708816in}}%
\pgfpathlineto{\pgfqpoint{6.025518in}{1.706491in}}%
\pgfpathlineto{\pgfqpoint{6.026592in}{1.708190in}}%
\pgfpathlineto{\pgfqpoint{6.027666in}{1.707206in}}%
\pgfpathlineto{\pgfqpoint{6.031961in}{1.708727in}}%
\pgfpathlineto{\pgfqpoint{6.033035in}{1.704792in}}%
\pgfpathlineto{\pgfqpoint{6.034109in}{1.705686in}}%
\pgfpathlineto{\pgfqpoint{6.035183in}{1.708548in}}%
\pgfpathlineto{\pgfqpoint{6.038404in}{1.706044in}}%
\pgfpathlineto{\pgfqpoint{6.039478in}{1.687803in}}%
\pgfpathlineto{\pgfqpoint{6.040552in}{1.685031in}}%
\pgfpathlineto{\pgfqpoint{6.041626in}{1.679844in}}%
\pgfpathlineto{\pgfqpoint{6.042700in}{1.683600in}}%
\pgfpathlineto{\pgfqpoint{6.045921in}{1.681811in}}%
\pgfpathlineto{\pgfqpoint{6.046995in}{1.678950in}}%
\pgfpathlineto{\pgfqpoint{6.048069in}{1.674032in}}%
\pgfpathlineto{\pgfqpoint{6.049143in}{1.673138in}}%
\pgfpathlineto{\pgfqpoint{6.050217in}{1.675552in}}%
\pgfpathlineto{\pgfqpoint{6.053438in}{1.671081in}}%
\pgfpathlineto{\pgfqpoint{6.054512in}{1.671171in}}%
\pgfpathlineto{\pgfqpoint{6.057734in}{1.678682in}}%
\pgfpathlineto{\pgfqpoint{6.060955in}{1.675284in}}%
\pgfpathlineto{\pgfqpoint{6.063103in}{1.671528in}}%
\pgfpathlineto{\pgfqpoint{6.064177in}{1.674569in}}%
\pgfpathlineto{\pgfqpoint{6.065250in}{1.680381in}}%
\pgfpathlineto{\pgfqpoint{6.069546in}{1.682259in}}%
\pgfpathlineto{\pgfqpoint{6.070620in}{1.684226in}}%
\pgfpathlineto{\pgfqpoint{6.071693in}{1.689323in}}%
\pgfpathlineto{\pgfqpoint{6.072767in}{1.691648in}}%
\pgfpathlineto{\pgfqpoint{6.077063in}{1.683868in}}%
\pgfpathlineto{\pgfqpoint{6.080284in}{1.687534in}}%
\pgfpathlineto{\pgfqpoint{6.083506in}{1.687087in}}%
\pgfpathlineto{\pgfqpoint{6.085653in}{1.680112in}}%
\pgfpathlineto{\pgfqpoint{6.087801in}{1.681633in}}%
\pgfpathlineto{\pgfqpoint{6.091022in}{1.682616in}}%
\pgfpathlineto{\pgfqpoint{6.092096in}{1.681722in}}%
\pgfpathlineto{\pgfqpoint{6.093170in}{1.687981in}}%
\pgfpathlineto{\pgfqpoint{6.094244in}{1.687177in}}%
\pgfpathlineto{\pgfqpoint{6.095318in}{1.689949in}}%
\pgfpathlineto{\pgfqpoint{6.099613in}{1.688071in}}%
\pgfpathlineto{\pgfqpoint{6.100687in}{1.684494in}}%
\pgfpathlineto{\pgfqpoint{6.101761in}{1.683958in}}%
\pgfpathlineto{\pgfqpoint{6.102835in}{1.684315in}}%
\pgfpathlineto{\pgfqpoint{6.106056in}{1.679934in}}%
\pgfpathlineto{\pgfqpoint{6.107130in}{1.681275in}}%
\pgfpathlineto{\pgfqpoint{6.110352in}{1.676893in}}%
\pgfpathlineto{\pgfqpoint{6.114647in}{1.681364in}}%
\pgfpathlineto{\pgfqpoint{6.115721in}{1.679039in}}%
\pgfpathlineto{\pgfqpoint{6.116795in}{1.678950in}}%
\pgfpathlineto{\pgfqpoint{6.117868in}{1.680738in}}%
\pgfpathlineto{\pgfqpoint{6.121090in}{1.679844in}}%
\pgfpathlineto{\pgfqpoint{6.123238in}{1.682884in}}%
\pgfpathlineto{\pgfqpoint{6.124311in}{1.680470in}}%
\pgfpathlineto{\pgfqpoint{6.128607in}{1.681990in}}%
\pgfpathlineto{\pgfqpoint{6.129681in}{1.685299in}}%
\pgfpathlineto{\pgfqpoint{6.130755in}{1.683063in}}%
\pgfpathlineto{\pgfqpoint{6.131828in}{1.678771in}}%
\pgfpathlineto{\pgfqpoint{6.132902in}{1.669382in}}%
\pgfpathlineto{\pgfqpoint{6.136124in}{1.667773in}}%
\pgfpathlineto{\pgfqpoint{6.137198in}{1.664911in}}%
\pgfpathlineto{\pgfqpoint{6.138271in}{1.670276in}}%
\pgfpathlineto{\pgfqpoint{6.140419in}{1.658563in}}%
\pgfpathlineto{\pgfqpoint{6.144714in}{1.658563in}}%
\pgfpathlineto{\pgfqpoint{6.145788in}{1.660530in}}%
\pgfpathlineto{\pgfqpoint{6.146862in}{1.658294in}}%
\pgfpathlineto{\pgfqpoint{6.147936in}{1.664911in}}%
\pgfpathlineto{\pgfqpoint{6.151157in}{1.664375in}}%
\pgfpathlineto{\pgfqpoint{6.152231in}{1.662676in}}%
\pgfpathlineto{\pgfqpoint{6.153305in}{1.662408in}}%
\pgfpathlineto{\pgfqpoint{6.155453in}{1.658026in}}%
\pgfpathlineto{\pgfqpoint{6.159748in}{1.653734in}}%
\pgfpathlineto{\pgfqpoint{6.160822in}{1.647028in}}%
\pgfpathlineto{\pgfqpoint{6.162970in}{1.654628in}}%
\pgfpathlineto{\pgfqpoint{6.167265in}{1.655075in}}%
\pgfpathlineto{\pgfqpoint{6.168339in}{1.651588in}}%
\pgfpathlineto{\pgfqpoint{6.169413in}{1.653734in}}%
\pgfpathlineto{\pgfqpoint{6.170487in}{1.653823in}}%
\pgfpathlineto{\pgfqpoint{6.174782in}{1.660887in}}%
\pgfpathlineto{\pgfqpoint{6.175856in}{1.664464in}}%
\pgfpathlineto{\pgfqpoint{6.178003in}{1.662855in}}%
\pgfpathlineto{\pgfqpoint{6.181225in}{1.662318in}}%
\pgfpathlineto{\pgfqpoint{6.183373in}{1.663302in}}%
\pgfpathlineto{\pgfqpoint{6.184446in}{1.660798in}}%
\pgfpathlineto{\pgfqpoint{6.185520in}{1.665090in}}%
\pgfpathlineto{\pgfqpoint{6.188742in}{1.668935in}}%
\pgfpathlineto{\pgfqpoint{6.189816in}{1.663033in}}%
\pgfpathlineto{\pgfqpoint{6.190889in}{1.664911in}}%
\pgfpathlineto{\pgfqpoint{6.193037in}{1.664375in}}%
\pgfpathlineto{\pgfqpoint{6.196259in}{1.663928in}}%
\pgfpathlineto{\pgfqpoint{6.198406in}{1.654539in}}%
\pgfpathlineto{\pgfqpoint{6.200554in}{1.654360in}}%
\pgfpathlineto{\pgfqpoint{6.203776in}{1.657311in}}%
\pgfpathlineto{\pgfqpoint{6.204849in}{1.649978in}}%
\pgfpathlineto{\pgfqpoint{6.205923in}{1.649978in}}%
\pgfpathlineto{\pgfqpoint{6.206997in}{1.646491in}}%
\pgfpathlineto{\pgfqpoint{6.208071in}{1.648458in}}%
\pgfpathlineto{\pgfqpoint{6.211292in}{1.651230in}}%
\pgfpathlineto{\pgfqpoint{6.213440in}{1.648369in}}%
\pgfpathlineto{\pgfqpoint{6.214514in}{1.643898in}}%
\pgfpathlineto{\pgfqpoint{6.215588in}{1.643540in}}%
\pgfpathlineto{\pgfqpoint{6.218809in}{1.641305in}}%
\pgfpathlineto{\pgfqpoint{6.219883in}{1.638712in}}%
\pgfpathlineto{\pgfqpoint{6.222031in}{1.643630in}}%
\pgfpathlineto{\pgfqpoint{6.223105in}{1.644256in}}%
\pgfpathlineto{\pgfqpoint{6.226326in}{1.645060in}}%
\pgfpathlineto{\pgfqpoint{6.227400in}{1.642557in}}%
\pgfpathlineto{\pgfqpoint{6.228474in}{1.643272in}}%
\pgfpathlineto{\pgfqpoint{6.229548in}{1.649621in}}%
\pgfpathlineto{\pgfqpoint{6.230622in}{1.649710in}}%
\pgfpathlineto{\pgfqpoint{6.233843in}{1.645418in}}%
\pgfpathlineto{\pgfqpoint{6.235991in}{1.651051in}}%
\pgfpathlineto{\pgfqpoint{6.237065in}{1.678950in}}%
\pgfpathlineto{\pgfqpoint{6.241360in}{1.683779in}}%
\pgfpathlineto{\pgfqpoint{6.242434in}{1.687803in}}%
\pgfpathlineto{\pgfqpoint{6.243508in}{1.682259in}}%
\pgfpathlineto{\pgfqpoint{6.245655in}{1.687981in}}%
\pgfpathlineto{\pgfqpoint{6.248877in}{1.687534in}}%
\pgfpathlineto{\pgfqpoint{6.252098in}{1.680649in}}%
\pgfpathlineto{\pgfqpoint{6.253172in}{1.681007in}}%
\pgfpathlineto{\pgfqpoint{6.256394in}{1.686908in}}%
\pgfpathlineto{\pgfqpoint{6.257467in}{1.684405in}}%
\pgfpathlineto{\pgfqpoint{6.258541in}{1.683868in}}%
\pgfpathlineto{\pgfqpoint{6.259615in}{1.679665in}}%
\pgfpathlineto{\pgfqpoint{6.260689in}{1.677966in}}%
\pgfpathlineto{\pgfqpoint{6.264984in}{1.683421in}}%
\pgfpathlineto{\pgfqpoint{6.266058in}{1.682795in}}%
\pgfpathlineto{\pgfqpoint{6.267132in}{1.683063in}}%
\pgfpathlineto{\pgfqpoint{6.268206in}{1.686104in}}%
\pgfpathlineto{\pgfqpoint{6.271427in}{1.685478in}}%
\pgfpathlineto{\pgfqpoint{6.272501in}{1.684673in}}%
\pgfpathlineto{\pgfqpoint{6.273575in}{1.681543in}}%
\pgfpathlineto{\pgfqpoint{6.275723in}{1.679844in}}%
\pgfpathlineto{\pgfqpoint{6.280018in}{1.675284in}}%
\pgfpathlineto{\pgfqpoint{6.281092in}{1.671618in}}%
\pgfpathlineto{\pgfqpoint{6.282166in}{1.665805in}}%
\pgfpathlineto{\pgfqpoint{6.283240in}{1.665090in}}%
\pgfpathlineto{\pgfqpoint{6.286461in}{1.666610in}}%
\pgfpathlineto{\pgfqpoint{6.288609in}{1.674390in}}%
\pgfpathlineto{\pgfqpoint{6.289683in}{1.673853in}}%
\pgfpathlineto{\pgfqpoint{6.290756in}{1.679397in}}%
\pgfpathlineto{\pgfqpoint{6.293978in}{1.681275in}}%
\pgfpathlineto{\pgfqpoint{6.295052in}{1.691469in}}%
\pgfpathlineto{\pgfqpoint{6.296126in}{1.692631in}}%
\pgfpathlineto{\pgfqpoint{6.297199in}{1.688250in}}%
\pgfpathlineto{\pgfqpoint{6.298273in}{1.696029in}}%
\pgfpathlineto{\pgfqpoint{6.301495in}{1.696029in}}%
\pgfpathlineto{\pgfqpoint{6.302569in}{1.692899in}}%
\pgfpathlineto{\pgfqpoint{6.303643in}{1.692899in}}%
\pgfpathlineto{\pgfqpoint{6.304716in}{1.692095in}}%
\pgfpathlineto{\pgfqpoint{6.305790in}{1.692721in}}%
\pgfpathlineto{\pgfqpoint{6.309012in}{1.691648in}}%
\pgfpathlineto{\pgfqpoint{6.310086in}{1.695671in}}%
\pgfpathlineto{\pgfqpoint{6.311159in}{1.696029in}}%
\pgfpathlineto{\pgfqpoint{6.312233in}{1.694956in}}%
\pgfpathlineto{\pgfqpoint{6.313307in}{1.691916in}}%
\pgfpathlineto{\pgfqpoint{6.317602in}{1.694688in}}%
\pgfpathlineto{\pgfqpoint{6.318676in}{1.692363in}}%
\pgfpathlineto{\pgfqpoint{6.320824in}{1.684047in}}%
\pgfpathlineto{\pgfqpoint{6.324045in}{1.686014in}}%
\pgfpathlineto{\pgfqpoint{6.326193in}{1.690575in}}%
\pgfpathlineto{\pgfqpoint{6.328341in}{1.697907in}}%
\pgfpathlineto{\pgfqpoint{6.331562in}{1.693436in}}%
\pgfpathlineto{\pgfqpoint{6.332636in}{1.692989in}}%
\pgfpathlineto{\pgfqpoint{6.333710in}{1.690485in}}%
\pgfpathlineto{\pgfqpoint{6.334784in}{1.692631in}}%
\pgfpathlineto{\pgfqpoint{6.335858in}{1.692452in}}%
\pgfpathlineto{\pgfqpoint{6.339079in}{1.683868in}}%
\pgfpathlineto{\pgfqpoint{6.341227in}{1.683868in}}%
\pgfpathlineto{\pgfqpoint{6.342301in}{1.680738in}}%
\pgfpathlineto{\pgfqpoint{6.343375in}{1.680470in}}%
\pgfpathlineto{\pgfqpoint{6.346596in}{1.664822in}}%
\pgfpathlineto{\pgfqpoint{6.347670in}{1.665090in}}%
\pgfpathlineto{\pgfqpoint{6.348744in}{1.664196in}}%
\pgfpathlineto{\pgfqpoint{6.350891in}{1.659457in}}%
\pgfpathlineto{\pgfqpoint{6.354113in}{1.658294in}}%
\pgfpathlineto{\pgfqpoint{6.355187in}{1.653913in}}%
\pgfpathlineto{\pgfqpoint{6.356261in}{1.653019in}}%
\pgfpathlineto{\pgfqpoint{6.358408in}{1.663838in}}%
\pgfpathlineto{\pgfqpoint{6.361630in}{1.670366in}}%
\pgfpathlineto{\pgfqpoint{6.362704in}{1.670187in}}%
\pgfpathlineto{\pgfqpoint{6.363777in}{1.677788in}}%
\pgfpathlineto{\pgfqpoint{6.365925in}{1.677072in}}%
\pgfpathlineto{\pgfqpoint{6.369147in}{1.682437in}}%
\pgfpathlineto{\pgfqpoint{6.372368in}{1.709174in}}%
\pgfpathlineto{\pgfqpoint{6.373442in}{1.712124in}}%
\pgfpathlineto{\pgfqpoint{6.376664in}{1.716059in}}%
\pgfpathlineto{\pgfqpoint{6.377737in}{1.709352in}}%
\pgfpathlineto{\pgfqpoint{6.379885in}{1.705239in}}%
\pgfpathlineto{\pgfqpoint{6.380959in}{1.710783in}}%
\pgfpathlineto{\pgfqpoint{6.384180in}{1.717043in}}%
\pgfpathlineto{\pgfqpoint{6.385254in}{1.728220in}}%
\pgfpathlineto{\pgfqpoint{6.386328in}{1.725716in}}%
\pgfpathlineto{\pgfqpoint{6.387402in}{1.721156in}}%
\pgfpathlineto{\pgfqpoint{6.388476in}{1.723838in}}%
\pgfpathlineto{\pgfqpoint{6.391697in}{1.728667in}}%
\pgfpathlineto{\pgfqpoint{6.392771in}{1.725180in}}%
\pgfpathlineto{\pgfqpoint{6.393845in}{1.724733in}}%
\pgfpathlineto{\pgfqpoint{6.395993in}{1.728220in}}%
\pgfpathlineto{\pgfqpoint{6.401362in}{1.728935in}}%
\pgfpathlineto{\pgfqpoint{6.402436in}{1.730187in}}%
\pgfpathlineto{\pgfqpoint{6.403510in}{1.726074in}}%
\pgfpathlineto{\pgfqpoint{6.403510in}{1.726074in}}%
\pgfusepath{stroke}%
\end{pgfscope}%
\begin{pgfscope}%
\pgfpathrectangle{\pgfqpoint{3.937600in}{1.347524in}}{\pgfqpoint{2.583333in}{0.400885in}}%
\pgfusepath{clip}%
\pgfsetroundcap%
\pgfsetroundjoin%
\pgfsetlinewidth{1.505625pt}%
\definecolor{currentstroke}{rgb}{0.498039,0.498039,0.498039}%
\pgfsetstrokecolor{currentstroke}%
\pgfsetdash{}{0pt}%
\pgfpathmoveto{\pgfqpoint{4.055025in}{1.537847in}}%
\pgfpathlineto{\pgfqpoint{4.056098in}{1.537847in}}%
\pgfpathlineto{\pgfqpoint{4.058246in}{1.535401in}}%
\pgfpathlineto{\pgfqpoint{4.073280in}{1.535401in}}%
\pgfpathlineto{\pgfqpoint{4.076501in}{1.531749in}}%
\pgfpathlineto{\pgfqpoint{4.077575in}{1.527918in}}%
\pgfpathlineto{\pgfqpoint{4.078649in}{1.527205in}}%
\pgfpathlineto{\pgfqpoint{4.079723in}{1.524978in}}%
\pgfpathlineto{\pgfqpoint{4.080797in}{1.524176in}}%
\pgfpathlineto{\pgfqpoint{4.085092in}{1.527027in}}%
\pgfpathlineto{\pgfqpoint{4.086166in}{1.527027in}}%
\pgfpathlineto{\pgfqpoint{4.087240in}{1.525518in}}%
\pgfpathlineto{\pgfqpoint{4.118381in}{1.525518in}}%
\pgfpathlineto{\pgfqpoint{4.121603in}{1.523449in}}%
\pgfpathlineto{\pgfqpoint{4.122676in}{1.523449in}}%
\pgfpathlineto{\pgfqpoint{4.123750in}{1.522339in}}%
\pgfpathlineto{\pgfqpoint{4.124824in}{1.520309in}}%
\pgfpathlineto{\pgfqpoint{4.125898in}{1.520974in}}%
\pgfpathlineto{\pgfqpoint{4.133415in}{1.518235in}}%
\pgfpathlineto{\pgfqpoint{4.139858in}{1.517659in}}%
\pgfpathlineto{\pgfqpoint{4.140932in}{1.519040in}}%
\pgfpathlineto{\pgfqpoint{4.146301in}{1.519616in}}%
\pgfpathlineto{\pgfqpoint{4.159187in}{1.519616in}}%
\pgfpathlineto{\pgfqpoint{4.160261in}{1.515590in}}%
\pgfpathlineto{\pgfqpoint{4.162409in}{1.515590in}}%
\pgfpathlineto{\pgfqpoint{4.163482in}{1.513827in}}%
\pgfpathlineto{\pgfqpoint{4.169925in}{1.513827in}}%
\pgfpathlineto{\pgfqpoint{4.170999in}{1.510384in}}%
\pgfpathlineto{\pgfqpoint{4.174221in}{1.510384in}}%
\pgfpathlineto{\pgfqpoint{4.175295in}{1.504952in}}%
\pgfpathlineto{\pgfqpoint{4.176368in}{1.505104in}}%
\pgfpathlineto{\pgfqpoint{4.177442in}{1.501448in}}%
\pgfpathlineto{\pgfqpoint{4.184959in}{1.498820in}}%
\pgfpathlineto{\pgfqpoint{4.186033in}{1.500832in}}%
\pgfpathlineto{\pgfqpoint{4.190328in}{1.499295in}}%
\pgfpathlineto{\pgfqpoint{4.191402in}{1.500811in}}%
\pgfpathlineto{\pgfqpoint{4.192476in}{1.499273in}}%
\pgfpathlineto{\pgfqpoint{4.193550in}{1.496098in}}%
\pgfpathlineto{\pgfqpoint{4.196771in}{1.497428in}}%
\pgfpathlineto{\pgfqpoint{4.197845in}{1.496648in}}%
\pgfpathlineto{\pgfqpoint{4.198919in}{1.497493in}}%
\pgfpathlineto{\pgfqpoint{4.199993in}{1.494937in}}%
\pgfpathlineto{\pgfqpoint{4.201067in}{1.494175in}}%
\pgfpathlineto{\pgfqpoint{4.204288in}{1.495138in}}%
\pgfpathlineto{\pgfqpoint{4.205362in}{1.494861in}}%
\pgfpathlineto{\pgfqpoint{4.206436in}{1.495414in}}%
\pgfpathlineto{\pgfqpoint{4.208584in}{1.494581in}}%
\pgfpathlineto{\pgfqpoint{4.212879in}{1.493062in}}%
\pgfpathlineto{\pgfqpoint{4.213953in}{1.494695in}}%
\pgfpathlineto{\pgfqpoint{4.215027in}{1.493589in}}%
\pgfpathlineto{\pgfqpoint{4.216100in}{1.496598in}}%
\pgfpathlineto{\pgfqpoint{4.219322in}{1.495048in}}%
\pgfpathlineto{\pgfqpoint{4.220396in}{1.495742in}}%
\pgfpathlineto{\pgfqpoint{4.221470in}{1.492878in}}%
\pgfpathlineto{\pgfqpoint{4.222543in}{1.493490in}}%
\pgfpathlineto{\pgfqpoint{4.223617in}{1.489525in}}%
\pgfpathlineto{\pgfqpoint{4.226839in}{1.488999in}}%
\pgfpathlineto{\pgfqpoint{4.227913in}{1.487167in}}%
\pgfpathlineto{\pgfqpoint{4.228986in}{1.486974in}}%
\pgfpathlineto{\pgfqpoint{4.230060in}{1.483448in}}%
\pgfpathlineto{\pgfqpoint{4.231134in}{1.484314in}}%
\pgfpathlineto{\pgfqpoint{4.234356in}{1.483128in}}%
\pgfpathlineto{\pgfqpoint{4.235430in}{1.483498in}}%
\pgfpathlineto{\pgfqpoint{4.236503in}{1.485418in}}%
\pgfpathlineto{\pgfqpoint{4.237577in}{1.485291in}}%
\pgfpathlineto{\pgfqpoint{4.238651in}{1.482453in}}%
\pgfpathlineto{\pgfqpoint{4.241873in}{1.483801in}}%
\pgfpathlineto{\pgfqpoint{4.242946in}{1.482932in}}%
\pgfpathlineto{\pgfqpoint{4.244020in}{1.483178in}}%
\pgfpathlineto{\pgfqpoint{4.245094in}{1.482375in}}%
\pgfpathlineto{\pgfqpoint{4.246168in}{1.480293in}}%
\pgfpathlineto{\pgfqpoint{4.250463in}{1.478077in}}%
\pgfpathlineto{\pgfqpoint{4.253685in}{1.478194in}}%
\pgfpathlineto{\pgfqpoint{4.256906in}{1.476845in}}%
\pgfpathlineto{\pgfqpoint{4.257980in}{1.477018in}}%
\pgfpathlineto{\pgfqpoint{4.259054in}{1.476149in}}%
\pgfpathlineto{\pgfqpoint{4.260128in}{1.477125in}}%
\pgfpathlineto{\pgfqpoint{4.261202in}{1.474805in}}%
\pgfpathlineto{\pgfqpoint{4.264423in}{1.474579in}}%
\pgfpathlineto{\pgfqpoint{4.266571in}{1.472055in}}%
\pgfpathlineto{\pgfqpoint{4.267645in}{1.477439in}}%
\pgfpathlineto{\pgfqpoint{4.271940in}{1.478256in}}%
\pgfpathlineto{\pgfqpoint{4.273014in}{1.480782in}}%
\pgfpathlineto{\pgfqpoint{4.274088in}{1.480963in}}%
\pgfpathlineto{\pgfqpoint{4.276235in}{1.475829in}}%
\pgfpathlineto{\pgfqpoint{4.279457in}{1.475600in}}%
\pgfpathlineto{\pgfqpoint{4.281605in}{1.474517in}}%
\pgfpathlineto{\pgfqpoint{4.282678in}{1.477003in}}%
\pgfpathlineto{\pgfqpoint{4.283752in}{1.477641in}}%
\pgfpathlineto{\pgfqpoint{4.286974in}{1.476648in}}%
\pgfpathlineto{\pgfqpoint{4.289121in}{1.478857in}}%
\pgfpathlineto{\pgfqpoint{4.291269in}{1.477031in}}%
\pgfpathlineto{\pgfqpoint{4.294491in}{1.478424in}}%
\pgfpathlineto{\pgfqpoint{4.295564in}{1.478189in}}%
\pgfpathlineto{\pgfqpoint{4.297712in}{1.479072in}}%
\pgfpathlineto{\pgfqpoint{4.298786in}{1.479309in}}%
\pgfpathlineto{\pgfqpoint{4.303081in}{1.480915in}}%
\pgfpathlineto{\pgfqpoint{4.313820in}{1.480915in}}%
\pgfpathlineto{\pgfqpoint{4.320263in}{1.475577in}}%
\pgfpathlineto{\pgfqpoint{4.321337in}{1.477382in}}%
\pgfpathlineto{\pgfqpoint{4.325632in}{1.475139in}}%
\pgfpathlineto{\pgfqpoint{4.327780in}{1.469546in}}%
\pgfpathlineto{\pgfqpoint{4.328853in}{1.473747in}}%
\pgfpathlineto{\pgfqpoint{4.332075in}{1.473518in}}%
\pgfpathlineto{\pgfqpoint{4.335297in}{1.469806in}}%
\pgfpathlineto{\pgfqpoint{4.336370in}{1.469204in}}%
\pgfpathlineto{\pgfqpoint{4.342813in}{1.468712in}}%
\pgfpathlineto{\pgfqpoint{4.343887in}{1.469469in}}%
\pgfpathlineto{\pgfqpoint{4.348183in}{1.468326in}}%
\pgfpathlineto{\pgfqpoint{4.349256in}{1.466657in}}%
\pgfpathlineto{\pgfqpoint{4.350330in}{1.462903in}}%
\pgfpathlineto{\pgfqpoint{4.351404in}{1.461737in}}%
\pgfpathlineto{\pgfqpoint{4.354626in}{1.463538in}}%
\pgfpathlineto{\pgfqpoint{4.358921in}{1.471126in}}%
\pgfpathlineto{\pgfqpoint{4.362142in}{1.471627in}}%
\pgfpathlineto{\pgfqpoint{4.363216in}{1.473360in}}%
\pgfpathlineto{\pgfqpoint{4.364290in}{1.470625in}}%
\pgfpathlineto{\pgfqpoint{4.365364in}{1.466310in}}%
\pgfpathlineto{\pgfqpoint{4.366438in}{1.468737in}}%
\pgfpathlineto{\pgfqpoint{4.369659in}{1.470310in}}%
\pgfpathlineto{\pgfqpoint{4.370733in}{1.473124in}}%
\pgfpathlineto{\pgfqpoint{4.372881in}{1.472384in}}%
\pgfpathlineto{\pgfqpoint{4.373955in}{1.470408in}}%
\pgfpathlineto{\pgfqpoint{4.379324in}{1.470740in}}%
\pgfpathlineto{\pgfqpoint{4.380398in}{1.468688in}}%
\pgfpathlineto{\pgfqpoint{4.381472in}{1.471182in}}%
\pgfpathlineto{\pgfqpoint{4.384693in}{1.472520in}}%
\pgfpathlineto{\pgfqpoint{4.386841in}{1.472068in}}%
\pgfpathlineto{\pgfqpoint{4.416908in}{1.472068in}}%
\pgfpathlineto{\pgfqpoint{4.417982in}{1.470598in}}%
\pgfpathlineto{\pgfqpoint{4.422277in}{1.470765in}}%
\pgfpathlineto{\pgfqpoint{4.424425in}{1.467642in}}%
\pgfpathlineto{\pgfqpoint{4.425499in}{1.468729in}}%
\pgfpathlineto{\pgfqpoint{4.449123in}{1.468729in}}%
\pgfpathlineto{\pgfqpoint{4.453419in}{1.467182in}}%
\pgfpathlineto{\pgfqpoint{4.478117in}{1.467182in}}%
\pgfpathlineto{\pgfqpoint{4.479191in}{1.463275in}}%
\pgfpathlineto{\pgfqpoint{4.483486in}{1.463274in}}%
\pgfpathlineto{\pgfqpoint{4.484560in}{1.462537in}}%
\pgfpathlineto{\pgfqpoint{4.486708in}{1.464057in}}%
\pgfpathlineto{\pgfqpoint{4.491003in}{1.463737in}}%
\pgfpathlineto{\pgfqpoint{4.492077in}{1.463419in}}%
\pgfpathlineto{\pgfqpoint{4.493151in}{1.464158in}}%
\pgfpathlineto{\pgfqpoint{4.494225in}{1.463892in}}%
\pgfpathlineto{\pgfqpoint{4.498520in}{1.463468in}}%
\pgfpathlineto{\pgfqpoint{4.500668in}{1.461053in}}%
\pgfpathlineto{\pgfqpoint{4.501741in}{1.459975in}}%
\pgfpathlineto{\pgfqpoint{4.504963in}{1.458757in}}%
\pgfpathlineto{\pgfqpoint{4.507111in}{1.456422in}}%
\pgfpathlineto{\pgfqpoint{4.512480in}{1.453676in}}%
\pgfpathlineto{\pgfqpoint{4.513554in}{1.451651in}}%
\pgfpathlineto{\pgfqpoint{4.514628in}{1.453076in}}%
\pgfpathlineto{\pgfqpoint{4.515701in}{1.452374in}}%
\pgfpathlineto{\pgfqpoint{4.516775in}{1.450705in}}%
\pgfpathlineto{\pgfqpoint{4.519997in}{1.451205in}}%
\pgfpathlineto{\pgfqpoint{4.521071in}{1.449331in}}%
\pgfpathlineto{\pgfqpoint{4.522144in}{1.450716in}}%
\pgfpathlineto{\pgfqpoint{4.523218in}{1.448898in}}%
\pgfpathlineto{\pgfqpoint{4.524292in}{1.450409in}}%
\pgfpathlineto{\pgfqpoint{4.528587in}{1.447304in}}%
\pgfpathlineto{\pgfqpoint{4.529661in}{1.448437in}}%
\pgfpathlineto{\pgfqpoint{4.531809in}{1.447073in}}%
\pgfpathlineto{\pgfqpoint{4.535030in}{1.446595in}}%
\pgfpathlineto{\pgfqpoint{4.536104in}{1.445558in}}%
\pgfpathlineto{\pgfqpoint{4.537178in}{1.447306in}}%
\pgfpathlineto{\pgfqpoint{4.538252in}{1.446608in}}%
\pgfpathlineto{\pgfqpoint{4.542547in}{1.446392in}}%
\pgfpathlineto{\pgfqpoint{4.543621in}{1.445487in}}%
\pgfpathlineto{\pgfqpoint{4.544695in}{1.447106in}}%
\pgfpathlineto{\pgfqpoint{4.546843in}{1.445247in}}%
\pgfpathlineto{\pgfqpoint{4.550064in}{1.444015in}}%
\pgfpathlineto{\pgfqpoint{4.551138in}{1.444266in}}%
\pgfpathlineto{\pgfqpoint{4.554360in}{1.439576in}}%
\pgfpathlineto{\pgfqpoint{4.558655in}{1.440805in}}%
\pgfpathlineto{\pgfqpoint{4.559729in}{1.443574in}}%
\pgfpathlineto{\pgfqpoint{4.561876in}{1.435310in}}%
\pgfpathlineto{\pgfqpoint{4.566172in}{1.435085in}}%
\pgfpathlineto{\pgfqpoint{4.567246in}{1.436575in}}%
\pgfpathlineto{\pgfqpoint{4.568319in}{1.432469in}}%
\pgfpathlineto{\pgfqpoint{4.569393in}{1.431354in}}%
\pgfpathlineto{\pgfqpoint{4.572615in}{1.431815in}}%
\pgfpathlineto{\pgfqpoint{4.573689in}{1.430603in}}%
\pgfpathlineto{\pgfqpoint{4.574763in}{1.434598in}}%
\pgfpathlineto{\pgfqpoint{4.576910in}{1.433822in}}%
\pgfpathlineto{\pgfqpoint{4.580132in}{1.435583in}}%
\pgfpathlineto{\pgfqpoint{4.581206in}{1.433103in}}%
\pgfpathlineto{\pgfqpoint{4.582279in}{1.432594in}}%
\pgfpathlineto{\pgfqpoint{4.583353in}{1.433676in}}%
\pgfpathlineto{\pgfqpoint{4.584427in}{1.433164in}}%
\pgfpathlineto{\pgfqpoint{4.587649in}{1.434109in}}%
\pgfpathlineto{\pgfqpoint{4.589796in}{1.431226in}}%
\pgfpathlineto{\pgfqpoint{4.590870in}{1.432288in}}%
\pgfpathlineto{\pgfqpoint{4.591944in}{1.431893in}}%
\pgfpathlineto{\pgfqpoint{4.595165in}{1.433537in}}%
\pgfpathlineto{\pgfqpoint{4.597313in}{1.437054in}}%
\pgfpathlineto{\pgfqpoint{4.598387in}{1.435864in}}%
\pgfpathlineto{\pgfqpoint{4.599461in}{1.437261in}}%
\pgfpathlineto{\pgfqpoint{4.603756in}{1.438916in}}%
\pgfpathlineto{\pgfqpoint{4.604830in}{1.442655in}}%
\pgfpathlineto{\pgfqpoint{4.606978in}{1.446082in}}%
\pgfpathlineto{\pgfqpoint{4.611273in}{1.444920in}}%
\pgfpathlineto{\pgfqpoint{4.612347in}{1.446666in}}%
\pgfpathlineto{\pgfqpoint{4.613421in}{1.441180in}}%
\pgfpathlineto{\pgfqpoint{4.614495in}{1.440327in}}%
\pgfpathlineto{\pgfqpoint{4.617716in}{1.439443in}}%
\pgfpathlineto{\pgfqpoint{4.619864in}{1.441361in}}%
\pgfpathlineto{\pgfqpoint{4.622011in}{1.437815in}}%
\pgfpathlineto{\pgfqpoint{4.625233in}{1.438865in}}%
\pgfpathlineto{\pgfqpoint{4.626307in}{1.436341in}}%
\pgfpathlineto{\pgfqpoint{4.628454in}{1.444004in}}%
\pgfpathlineto{\pgfqpoint{4.629528in}{1.442230in}}%
\pgfpathlineto{\pgfqpoint{4.632750in}{1.443468in}}%
\pgfpathlineto{\pgfqpoint{4.633824in}{1.439274in}}%
\pgfpathlineto{\pgfqpoint{4.635971in}{1.437613in}}%
\pgfpathlineto{\pgfqpoint{4.637045in}{1.439557in}}%
\pgfpathlineto{\pgfqpoint{4.640267in}{1.439477in}}%
\pgfpathlineto{\pgfqpoint{4.644562in}{1.436681in}}%
\pgfpathlineto{\pgfqpoint{4.647784in}{1.435568in}}%
\pgfpathlineto{\pgfqpoint{4.649931in}{1.437394in}}%
\pgfpathlineto{\pgfqpoint{4.651005in}{1.435416in}}%
\pgfpathlineto{\pgfqpoint{4.652079in}{1.437720in}}%
\pgfpathlineto{\pgfqpoint{4.655300in}{1.439045in}}%
\pgfpathlineto{\pgfqpoint{4.657448in}{1.436722in}}%
\pgfpathlineto{\pgfqpoint{4.658522in}{1.438952in}}%
\pgfpathlineto{\pgfqpoint{4.659596in}{1.439032in}}%
\pgfpathlineto{\pgfqpoint{4.666039in}{1.436753in}}%
\pgfpathlineto{\pgfqpoint{4.667113in}{1.435829in}}%
\pgfpathlineto{\pgfqpoint{4.670334in}{1.434499in}}%
\pgfpathlineto{\pgfqpoint{4.672482in}{1.440324in}}%
\pgfpathlineto{\pgfqpoint{4.674629in}{1.437956in}}%
\pgfpathlineto{\pgfqpoint{4.678925in}{1.438427in}}%
\pgfpathlineto{\pgfqpoint{4.681073in}{1.439851in}}%
\pgfpathlineto{\pgfqpoint{4.682146in}{1.440774in}}%
\pgfpathlineto{\pgfqpoint{4.685368in}{1.439758in}}%
\pgfpathlineto{\pgfqpoint{4.687516in}{1.442100in}}%
\pgfpathlineto{\pgfqpoint{4.688589in}{1.443176in}}%
\pgfpathlineto{\pgfqpoint{4.722952in}{1.443176in}}%
\pgfpathlineto{\pgfqpoint{4.725100in}{1.441832in}}%
\pgfpathlineto{\pgfqpoint{4.726174in}{1.442495in}}%
\pgfpathlineto{\pgfqpoint{4.755167in}{1.442495in}}%
\pgfpathlineto{\pgfqpoint{4.757315in}{1.433706in}}%
\pgfpathlineto{\pgfqpoint{4.760537in}{1.432043in}}%
\pgfpathlineto{\pgfqpoint{4.761610in}{1.430454in}}%
\pgfpathlineto{\pgfqpoint{4.763758in}{1.432492in}}%
\pgfpathlineto{\pgfqpoint{4.764832in}{1.431637in}}%
\pgfpathlineto{\pgfqpoint{4.768053in}{1.432041in}}%
\pgfpathlineto{\pgfqpoint{4.769127in}{1.430563in}}%
\pgfpathlineto{\pgfqpoint{4.770201in}{1.432120in}}%
\pgfpathlineto{\pgfqpoint{4.772349in}{1.432231in}}%
\pgfpathlineto{\pgfqpoint{4.775570in}{1.430563in}}%
\pgfpathlineto{\pgfqpoint{4.776644in}{1.433279in}}%
\pgfpathlineto{\pgfqpoint{4.777718in}{1.431812in}}%
\pgfpathlineto{\pgfqpoint{4.778792in}{1.433065in}}%
\pgfpathlineto{\pgfqpoint{4.779866in}{1.432953in}}%
\pgfpathlineto{\pgfqpoint{4.783087in}{1.433665in}}%
\pgfpathlineto{\pgfqpoint{4.784161in}{1.433097in}}%
\pgfpathlineto{\pgfqpoint{4.785235in}{1.433585in}}%
\pgfpathlineto{\pgfqpoint{4.786309in}{1.432754in}}%
\pgfpathlineto{\pgfqpoint{4.787383in}{1.432642in}}%
\pgfpathlineto{\pgfqpoint{4.791678in}{1.431300in}}%
\pgfpathlineto{\pgfqpoint{4.792752in}{1.432361in}}%
\pgfpathlineto{\pgfqpoint{4.794899in}{1.432882in}}%
\pgfpathlineto{\pgfqpoint{4.802416in}{1.434615in}}%
\pgfpathlineto{\pgfqpoint{4.805638in}{1.435688in}}%
\pgfpathlineto{\pgfqpoint{4.806712in}{1.434677in}}%
\pgfpathlineto{\pgfqpoint{4.807785in}{1.435329in}}%
\pgfpathlineto{\pgfqpoint{4.808859in}{1.436723in}}%
\pgfpathlineto{\pgfqpoint{4.809933in}{1.434987in}}%
\pgfpathlineto{\pgfqpoint{4.813155in}{1.434718in}}%
\pgfpathlineto{\pgfqpoint{4.814228in}{1.436253in}}%
\pgfpathlineto{\pgfqpoint{4.828188in}{1.436253in}}%
\pgfpathlineto{\pgfqpoint{4.829262in}{1.435066in}}%
\pgfpathlineto{\pgfqpoint{4.832484in}{1.434481in}}%
\pgfpathlineto{\pgfqpoint{4.843222in}{1.434984in}}%
\pgfpathlineto{\pgfqpoint{4.844296in}{1.433143in}}%
\pgfpathlineto{\pgfqpoint{4.845370in}{1.433945in}}%
\pgfpathlineto{\pgfqpoint{4.852887in}{1.433945in}}%
\pgfpathlineto{\pgfqpoint{4.853961in}{1.433169in}}%
\pgfpathlineto{\pgfqpoint{4.940941in}{1.433169in}}%
\pgfpathlineto{\pgfqpoint{4.942015in}{1.431272in}}%
\pgfpathlineto{\pgfqpoint{4.973157in}{1.431272in}}%
\pgfpathlineto{\pgfqpoint{4.974230in}{1.430138in}}%
\pgfpathlineto{\pgfqpoint{4.981747in}{1.430138in}}%
\pgfpathlineto{\pgfqpoint{4.982821in}{1.426866in}}%
\pgfpathlineto{\pgfqpoint{4.986043in}{1.426651in}}%
\pgfpathlineto{\pgfqpoint{4.987116in}{1.428047in}}%
\pgfpathlineto{\pgfqpoint{4.989264in}{1.425762in}}%
\pgfpathlineto{\pgfqpoint{4.990338in}{1.424949in}}%
\pgfpathlineto{\pgfqpoint{4.996781in}{1.425017in}}%
\pgfpathlineto{\pgfqpoint{4.997855in}{1.424353in}}%
\pgfpathlineto{\pgfqpoint{5.001076in}{1.424110in}}%
\pgfpathlineto{\pgfqpoint{5.002150in}{1.426181in}}%
\pgfpathlineto{\pgfqpoint{5.003224in}{1.426573in}}%
\pgfpathlineto{\pgfqpoint{5.005372in}{1.425788in}}%
\pgfpathlineto{\pgfqpoint{5.009667in}{1.425504in}}%
\pgfpathlineto{\pgfqpoint{5.012889in}{1.426453in}}%
\pgfpathlineto{\pgfqpoint{5.023627in}{1.424923in}}%
\pgfpathlineto{\pgfqpoint{5.024701in}{1.426146in}}%
\pgfpathlineto{\pgfqpoint{5.025775in}{1.425790in}}%
\pgfpathlineto{\pgfqpoint{5.026849in}{1.426356in}}%
\pgfpathlineto{\pgfqpoint{5.027922in}{1.426035in}}%
\pgfpathlineto{\pgfqpoint{5.031144in}{1.427136in}}%
\pgfpathlineto{\pgfqpoint{5.033292in}{1.425058in}}%
\pgfpathlineto{\pgfqpoint{5.038661in}{1.424813in}}%
\pgfpathlineto{\pgfqpoint{5.039735in}{1.426104in}}%
\pgfpathlineto{\pgfqpoint{5.040808in}{1.425713in}}%
\pgfpathlineto{\pgfqpoint{5.042956in}{1.421774in}}%
\pgfpathlineto{\pgfqpoint{5.047251in}{1.420743in}}%
\pgfpathlineto{\pgfqpoint{5.048325in}{1.420086in}}%
\pgfpathlineto{\pgfqpoint{5.049399in}{1.421746in}}%
\pgfpathlineto{\pgfqpoint{5.050473in}{1.420577in}}%
\pgfpathlineto{\pgfqpoint{5.053694in}{1.420709in}}%
\pgfpathlineto{\pgfqpoint{5.054768in}{1.419986in}}%
\pgfpathlineto{\pgfqpoint{5.056916in}{1.419822in}}%
\pgfpathlineto{\pgfqpoint{5.057990in}{1.419206in}}%
\pgfpathlineto{\pgfqpoint{5.061211in}{1.418435in}}%
\pgfpathlineto{\pgfqpoint{5.062285in}{1.417451in}}%
\pgfpathlineto{\pgfqpoint{5.063359in}{1.417982in}}%
\pgfpathlineto{\pgfqpoint{5.065507in}{1.422917in}}%
\pgfpathlineto{\pgfqpoint{5.068728in}{1.421863in}}%
\pgfpathlineto{\pgfqpoint{5.071950in}{1.426136in}}%
\pgfpathlineto{\pgfqpoint{5.079467in}{1.425142in}}%
\pgfpathlineto{\pgfqpoint{5.080540in}{1.425705in}}%
\pgfpathlineto{\pgfqpoint{5.088057in}{1.426162in}}%
\pgfpathlineto{\pgfqpoint{5.093427in}{1.423933in}}%
\pgfpathlineto{\pgfqpoint{5.094500in}{1.423968in}}%
\pgfpathlineto{\pgfqpoint{5.095574in}{1.422862in}}%
\pgfpathlineto{\pgfqpoint{5.102017in}{1.423132in}}%
\pgfpathlineto{\pgfqpoint{5.103091in}{1.422518in}}%
\pgfpathlineto{\pgfqpoint{5.106313in}{1.423634in}}%
\pgfpathlineto{\pgfqpoint{5.107386in}{1.425354in}}%
\pgfpathlineto{\pgfqpoint{5.108460in}{1.425777in}}%
\pgfpathlineto{\pgfqpoint{5.109534in}{1.425032in}}%
\pgfpathlineto{\pgfqpoint{5.113829in}{1.425032in}}%
\pgfpathlineto{\pgfqpoint{5.118125in}{1.420097in}}%
\pgfpathlineto{\pgfqpoint{5.123494in}{1.420852in}}%
\pgfpathlineto{\pgfqpoint{5.124568in}{1.422052in}}%
\pgfpathlineto{\pgfqpoint{5.125642in}{1.421644in}}%
\pgfpathlineto{\pgfqpoint{5.128863in}{1.421678in}}%
\pgfpathlineto{\pgfqpoint{5.129937in}{1.421037in}}%
\pgfpathlineto{\pgfqpoint{5.131011in}{1.422541in}}%
\pgfpathlineto{\pgfqpoint{5.132085in}{1.422917in}}%
\pgfpathlineto{\pgfqpoint{5.133159in}{1.421748in}}%
\pgfpathlineto{\pgfqpoint{5.136380in}{1.420769in}}%
\pgfpathlineto{\pgfqpoint{5.137454in}{1.421834in}}%
\pgfpathlineto{\pgfqpoint{5.138528in}{1.419804in}}%
\pgfpathlineto{\pgfqpoint{5.139602in}{1.422298in}}%
\pgfpathlineto{\pgfqpoint{5.148192in}{1.422264in}}%
\pgfpathlineto{\pgfqpoint{5.155709in}{1.419038in}}%
\pgfpathlineto{\pgfqpoint{5.158931in}{1.417270in}}%
\pgfpathlineto{\pgfqpoint{5.160004in}{1.415873in}}%
\pgfpathlineto{\pgfqpoint{5.162152in}{1.416031in}}%
\pgfpathlineto{\pgfqpoint{5.163226in}{1.415139in}}%
\pgfpathlineto{\pgfqpoint{5.168595in}{1.414887in}}%
\pgfpathlineto{\pgfqpoint{5.169669in}{1.415044in}}%
\pgfpathlineto{\pgfqpoint{5.170743in}{1.413569in}}%
\pgfpathlineto{\pgfqpoint{5.176112in}{1.413688in}}%
\pgfpathlineto{\pgfqpoint{5.178260in}{1.412010in}}%
\pgfpathlineto{\pgfqpoint{5.182555in}{1.412669in}}%
\pgfpathlineto{\pgfqpoint{5.184703in}{1.415199in}}%
\pgfpathlineto{\pgfqpoint{5.185777in}{1.415167in}}%
\pgfpathlineto{\pgfqpoint{5.190072in}{1.417380in}}%
\pgfpathlineto{\pgfqpoint{5.191146in}{1.415524in}}%
\pgfpathlineto{\pgfqpoint{5.193293in}{1.414132in}}%
\pgfpathlineto{\pgfqpoint{5.196515in}{1.415525in}}%
\pgfpathlineto{\pgfqpoint{5.197589in}{1.417867in}}%
\pgfpathlineto{\pgfqpoint{5.198663in}{1.418721in}}%
\pgfpathlineto{\pgfqpoint{5.200810in}{1.418754in}}%
\pgfpathlineto{\pgfqpoint{5.206180in}{1.417952in}}%
\pgfpathlineto{\pgfqpoint{5.267388in}{1.417952in}}%
\pgfpathlineto{\pgfqpoint{5.268462in}{1.413988in}}%
\pgfpathlineto{\pgfqpoint{5.271684in}{1.414583in}}%
\pgfpathlineto{\pgfqpoint{5.273831in}{1.412730in}}%
\pgfpathlineto{\pgfqpoint{5.275979in}{1.413996in}}%
\pgfpathlineto{\pgfqpoint{5.286717in}{1.415384in}}%
\pgfpathlineto{\pgfqpoint{5.287791in}{1.414230in}}%
\pgfpathlineto{\pgfqpoint{5.289939in}{1.413821in}}%
\pgfpathlineto{\pgfqpoint{5.295308in}{1.413384in}}%
\pgfpathlineto{\pgfqpoint{5.297456in}{1.414975in}}%
\pgfpathlineto{\pgfqpoint{5.303899in}{1.414975in}}%
\pgfpathlineto{\pgfqpoint{5.304973in}{1.412165in}}%
\pgfpathlineto{\pgfqpoint{5.309268in}{1.410794in}}%
\pgfpathlineto{\pgfqpoint{5.310342in}{1.411619in}}%
\pgfpathlineto{\pgfqpoint{5.311416in}{1.410101in}}%
\pgfpathlineto{\pgfqpoint{5.312490in}{1.410705in}}%
\pgfpathlineto{\pgfqpoint{5.313563in}{1.410064in}}%
\pgfpathlineto{\pgfqpoint{5.316785in}{1.409852in}}%
\pgfpathlineto{\pgfqpoint{5.317859in}{1.410425in}}%
\pgfpathlineto{\pgfqpoint{5.318933in}{1.412097in}}%
\pgfpathlineto{\pgfqpoint{5.321080in}{1.412097in}}%
\pgfpathlineto{\pgfqpoint{5.324302in}{1.410686in}}%
\pgfpathlineto{\pgfqpoint{5.325376in}{1.411912in}}%
\pgfpathlineto{\pgfqpoint{5.327523in}{1.409771in}}%
\pgfpathlineto{\pgfqpoint{5.331819in}{1.409378in}}%
\pgfpathlineto{\pgfqpoint{5.332893in}{1.410249in}}%
\pgfpathlineto{\pgfqpoint{5.333966in}{1.409244in}}%
\pgfpathlineto{\pgfqpoint{5.335040in}{1.409544in}}%
\pgfpathlineto{\pgfqpoint{5.336114in}{1.409002in}}%
\pgfpathlineto{\pgfqpoint{5.340409in}{1.409119in}}%
\pgfpathlineto{\pgfqpoint{5.341483in}{1.408581in}}%
\pgfpathlineto{\pgfqpoint{5.342557in}{1.408877in}}%
\pgfpathlineto{\pgfqpoint{5.343631in}{1.409800in}}%
\pgfpathlineto{\pgfqpoint{5.346852in}{1.408591in}}%
\pgfpathlineto{\pgfqpoint{5.347926in}{1.409095in}}%
\pgfpathlineto{\pgfqpoint{5.350074in}{1.406962in}}%
\pgfpathlineto{\pgfqpoint{5.354369in}{1.406875in}}%
\pgfpathlineto{\pgfqpoint{5.355443in}{1.405781in}}%
\pgfpathlineto{\pgfqpoint{5.358665in}{1.406092in}}%
\pgfpathlineto{\pgfqpoint{5.361886in}{1.405836in}}%
\pgfpathlineto{\pgfqpoint{5.364034in}{1.407921in}}%
\pgfpathlineto{\pgfqpoint{5.365108in}{1.407686in}}%
\pgfpathlineto{\pgfqpoint{5.366181in}{1.406693in}}%
\pgfpathlineto{\pgfqpoint{5.371551in}{1.407670in}}%
\pgfpathlineto{\pgfqpoint{5.372625in}{1.407086in}}%
\pgfpathlineto{\pgfqpoint{5.373698in}{1.407520in}}%
\pgfpathlineto{\pgfqpoint{5.379068in}{1.407780in}}%
\pgfpathlineto{\pgfqpoint{5.380141in}{1.407253in}}%
\pgfpathlineto{\pgfqpoint{5.381215in}{1.407949in}}%
\pgfpathlineto{\pgfqpoint{5.396249in}{1.409248in}}%
\pgfpathlineto{\pgfqpoint{5.441350in}{1.409248in}}%
\pgfpathlineto{\pgfqpoint{5.445646in}{1.407965in}}%
\pgfpathlineto{\pgfqpoint{5.643232in}{1.407965in}}%
\pgfpathlineto{\pgfqpoint{5.644306in}{1.404906in}}%
\pgfpathlineto{\pgfqpoint{5.647527in}{1.404935in}}%
\pgfpathlineto{\pgfqpoint{5.649675in}{1.400061in}}%
\pgfpathlineto{\pgfqpoint{5.650749in}{1.400114in}}%
\pgfpathlineto{\pgfqpoint{5.651823in}{1.397920in}}%
\pgfpathlineto{\pgfqpoint{5.655044in}{1.396171in}}%
\pgfpathlineto{\pgfqpoint{5.656118in}{1.397990in}}%
\pgfpathlineto{\pgfqpoint{5.657192in}{1.396418in}}%
\pgfpathlineto{\pgfqpoint{5.658266in}{1.396844in}}%
\pgfpathlineto{\pgfqpoint{5.659339in}{1.395657in}}%
\pgfpathlineto{\pgfqpoint{5.662561in}{1.396151in}}%
\pgfpathlineto{\pgfqpoint{5.663635in}{1.397423in}}%
\pgfpathlineto{\pgfqpoint{5.664709in}{1.397780in}}%
\pgfpathlineto{\pgfqpoint{5.665782in}{1.399090in}}%
\pgfpathlineto{\pgfqpoint{5.666856in}{1.397459in}}%
\pgfpathlineto{\pgfqpoint{5.672225in}{1.397000in}}%
\pgfpathlineto{\pgfqpoint{5.673299in}{1.395657in}}%
\pgfpathlineto{\pgfqpoint{5.674373in}{1.395830in}}%
\pgfpathlineto{\pgfqpoint{5.677595in}{1.395384in}}%
\pgfpathlineto{\pgfqpoint{5.678669in}{1.396318in}}%
\pgfpathlineto{\pgfqpoint{5.681890in}{1.395469in}}%
\pgfpathlineto{\pgfqpoint{5.685112in}{1.396085in}}%
\pgfpathlineto{\pgfqpoint{5.687259in}{1.393160in}}%
\pgfpathlineto{\pgfqpoint{5.689407in}{1.393798in}}%
\pgfpathlineto{\pgfqpoint{5.694776in}{1.392701in}}%
\pgfpathlineto{\pgfqpoint{5.696924in}{1.392327in}}%
\pgfpathlineto{\pgfqpoint{5.701219in}{1.392048in}}%
\pgfpathlineto{\pgfqpoint{5.703367in}{1.390170in}}%
\pgfpathlineto{\pgfqpoint{5.704441in}{1.390906in}}%
\pgfpathlineto{\pgfqpoint{5.707662in}{1.390521in}}%
\pgfpathlineto{\pgfqpoint{5.709810in}{1.391559in}}%
\pgfpathlineto{\pgfqpoint{5.710884in}{1.390277in}}%
\pgfpathlineto{\pgfqpoint{5.715179in}{1.390589in}}%
\pgfpathlineto{\pgfqpoint{5.716253in}{1.389331in}}%
\pgfpathlineto{\pgfqpoint{5.719474in}{1.389418in}}%
\pgfpathlineto{\pgfqpoint{5.722696in}{1.388650in}}%
\pgfpathlineto{\pgfqpoint{5.724844in}{1.389255in}}%
\pgfpathlineto{\pgfqpoint{5.725917in}{1.392124in}}%
\pgfpathlineto{\pgfqpoint{5.726991in}{1.391776in}}%
\pgfpathlineto{\pgfqpoint{5.730213in}{1.392904in}}%
\pgfpathlineto{\pgfqpoint{5.731287in}{1.392222in}}%
\pgfpathlineto{\pgfqpoint{5.732360in}{1.393545in}}%
\pgfpathlineto{\pgfqpoint{5.737730in}{1.392642in}}%
\pgfpathlineto{\pgfqpoint{5.738803in}{1.391940in}}%
\pgfpathlineto{\pgfqpoint{5.739877in}{1.392586in}}%
\pgfpathlineto{\pgfqpoint{5.740951in}{1.396071in}}%
\pgfpathlineto{\pgfqpoint{5.742025in}{1.394946in}}%
\pgfpathlineto{\pgfqpoint{5.745246in}{1.394506in}}%
\pgfpathlineto{\pgfqpoint{5.746320in}{1.395161in}}%
\pgfpathlineto{\pgfqpoint{5.747394in}{1.392506in}}%
\pgfpathlineto{\pgfqpoint{5.748468in}{1.393863in}}%
\pgfpathlineto{\pgfqpoint{5.749542in}{1.394031in}}%
\pgfpathlineto{\pgfqpoint{5.752763in}{1.393236in}}%
\pgfpathlineto{\pgfqpoint{5.753837in}{1.394565in}}%
\pgfpathlineto{\pgfqpoint{5.757059in}{1.393644in}}%
\pgfpathlineto{\pgfqpoint{5.760280in}{1.393716in}}%
\pgfpathlineto{\pgfqpoint{5.761354in}{1.392758in}}%
\pgfpathlineto{\pgfqpoint{5.762428in}{1.393557in}}%
\pgfpathlineto{\pgfqpoint{5.763502in}{1.392889in}}%
\pgfpathlineto{\pgfqpoint{5.764576in}{1.393973in}}%
\pgfpathlineto{\pgfqpoint{5.767797in}{1.393372in}}%
\pgfpathlineto{\pgfqpoint{5.769945in}{1.395123in}}%
\pgfpathlineto{\pgfqpoint{5.771019in}{1.396723in}}%
\pgfpathlineto{\pgfqpoint{5.772092in}{1.396672in}}%
\pgfpathlineto{\pgfqpoint{5.775314in}{1.397812in}}%
\pgfpathlineto{\pgfqpoint{5.779609in}{1.394573in}}%
\pgfpathlineto{\pgfqpoint{5.783905in}{1.393988in}}%
\pgfpathlineto{\pgfqpoint{5.784979in}{1.394952in}}%
\pgfpathlineto{\pgfqpoint{5.787126in}{1.393926in}}%
\pgfpathlineto{\pgfqpoint{5.790348in}{1.394359in}}%
\pgfpathlineto{\pgfqpoint{5.791422in}{1.392174in}}%
\pgfpathlineto{\pgfqpoint{5.792495in}{1.392640in}}%
\pgfpathlineto{\pgfqpoint{5.794643in}{1.390337in}}%
\pgfpathlineto{\pgfqpoint{5.797865in}{1.390517in}}%
\pgfpathlineto{\pgfqpoint{5.798938in}{1.389704in}}%
\pgfpathlineto{\pgfqpoint{5.800012in}{1.389993in}}%
\pgfpathlineto{\pgfqpoint{5.802160in}{1.388198in}}%
\pgfpathlineto{\pgfqpoint{5.805381in}{1.388219in}}%
\pgfpathlineto{\pgfqpoint{5.806455in}{1.387594in}}%
\pgfpathlineto{\pgfqpoint{5.807529in}{1.387722in}}%
\pgfpathlineto{\pgfqpoint{5.808603in}{1.386547in}}%
\pgfpathlineto{\pgfqpoint{5.809677in}{1.386985in}}%
\pgfpathlineto{\pgfqpoint{5.815046in}{1.385835in}}%
\pgfpathlineto{\pgfqpoint{5.817194in}{1.383778in}}%
\pgfpathlineto{\pgfqpoint{5.822563in}{1.382758in}}%
\pgfpathlineto{\pgfqpoint{5.823637in}{1.384244in}}%
\pgfpathlineto{\pgfqpoint{5.824711in}{1.383348in}}%
\pgfpathlineto{\pgfqpoint{5.827932in}{1.383289in}}%
\pgfpathlineto{\pgfqpoint{5.829006in}{1.384070in}}%
\pgfpathlineto{\pgfqpoint{5.830080in}{1.383157in}}%
\pgfpathlineto{\pgfqpoint{5.832227in}{1.383430in}}%
\pgfpathlineto{\pgfqpoint{5.836523in}{1.383663in}}%
\pgfpathlineto{\pgfqpoint{5.838670in}{1.384235in}}%
\pgfpathlineto{\pgfqpoint{5.839744in}{1.382961in}}%
\pgfpathlineto{\pgfqpoint{5.842966in}{1.383349in}}%
\pgfpathlineto{\pgfqpoint{5.844040in}{1.385167in}}%
\pgfpathlineto{\pgfqpoint{5.845113in}{1.384253in}}%
\pgfpathlineto{\pgfqpoint{5.846187in}{1.385070in}}%
\pgfpathlineto{\pgfqpoint{5.847261in}{1.384097in}}%
\pgfpathlineto{\pgfqpoint{5.854778in}{1.387220in}}%
\pgfpathlineto{\pgfqpoint{5.860147in}{1.386902in}}%
\pgfpathlineto{\pgfqpoint{5.862295in}{1.387196in}}%
\pgfpathlineto{\pgfqpoint{5.865516in}{1.387281in}}%
\pgfpathlineto{\pgfqpoint{5.866590in}{1.388852in}}%
\pgfpathlineto{\pgfqpoint{5.867664in}{1.388063in}}%
\pgfpathlineto{\pgfqpoint{5.869812in}{1.389430in}}%
\pgfpathlineto{\pgfqpoint{5.876255in}{1.388855in}}%
\pgfpathlineto{\pgfqpoint{5.877329in}{1.390147in}}%
\pgfpathlineto{\pgfqpoint{5.880550in}{1.389316in}}%
\pgfpathlineto{\pgfqpoint{5.882698in}{1.389624in}}%
\pgfpathlineto{\pgfqpoint{5.890215in}{1.387002in}}%
\pgfpathlineto{\pgfqpoint{5.891289in}{1.387213in}}%
\pgfpathlineto{\pgfqpoint{5.892362in}{1.390501in}}%
\pgfpathlineto{\pgfqpoint{5.895584in}{1.389005in}}%
\pgfpathlineto{\pgfqpoint{5.896658in}{1.391162in}}%
\pgfpathlineto{\pgfqpoint{5.897732in}{1.391093in}}%
\pgfpathlineto{\pgfqpoint{5.898805in}{1.390106in}}%
\pgfpathlineto{\pgfqpoint{5.899879in}{1.390309in}}%
\pgfpathlineto{\pgfqpoint{5.904175in}{1.391503in}}%
\pgfpathlineto{\pgfqpoint{5.906322in}{1.389351in}}%
\pgfpathlineto{\pgfqpoint{5.907396in}{1.388929in}}%
\pgfpathlineto{\pgfqpoint{5.910618in}{1.389721in}}%
\pgfpathlineto{\pgfqpoint{5.911691in}{1.389051in}}%
\pgfpathlineto{\pgfqpoint{5.912765in}{1.389889in}}%
\pgfpathlineto{\pgfqpoint{5.914913in}{1.390045in}}%
\pgfpathlineto{\pgfqpoint{5.918134in}{1.390248in}}%
\pgfpathlineto{\pgfqpoint{5.920282in}{1.392300in}}%
\pgfpathlineto{\pgfqpoint{5.921356in}{1.392324in}}%
\pgfpathlineto{\pgfqpoint{5.922430in}{1.393029in}}%
\pgfpathlineto{\pgfqpoint{5.925651in}{1.392457in}}%
\pgfpathlineto{\pgfqpoint{5.926725in}{1.393070in}}%
\pgfpathlineto{\pgfqpoint{5.927799in}{1.392211in}}%
\pgfpathlineto{\pgfqpoint{5.935316in}{1.392045in}}%
\pgfpathlineto{\pgfqpoint{5.936390in}{1.394618in}}%
\pgfpathlineto{\pgfqpoint{5.942833in}{1.394618in}}%
\pgfpathlineto{\pgfqpoint{5.943907in}{1.392589in}}%
\pgfpathlineto{\pgfqpoint{5.970753in}{1.392072in}}%
\pgfpathlineto{\pgfqpoint{5.972900in}{1.388023in}}%
\pgfpathlineto{\pgfqpoint{5.978269in}{1.386256in}}%
\pgfpathlineto{\pgfqpoint{5.979343in}{1.386561in}}%
\pgfpathlineto{\pgfqpoint{5.980417in}{1.388621in}}%
\pgfpathlineto{\pgfqpoint{5.985786in}{1.388918in}}%
\pgfpathlineto{\pgfqpoint{5.987934in}{1.385659in}}%
\pgfpathlineto{\pgfqpoint{5.989008in}{1.386133in}}%
\pgfpathlineto{\pgfqpoint{5.990082in}{1.385437in}}%
\pgfpathlineto{\pgfqpoint{5.993303in}{1.384944in}}%
\pgfpathlineto{\pgfqpoint{5.994377in}{1.383820in}}%
\pgfpathlineto{\pgfqpoint{5.995451in}{1.385147in}}%
\pgfpathlineto{\pgfqpoint{5.997599in}{1.383959in}}%
\pgfpathlineto{\pgfqpoint{6.001894in}{1.382408in}}%
\pgfpathlineto{\pgfqpoint{6.002968in}{1.382670in}}%
\pgfpathlineto{\pgfqpoint{6.004042in}{1.381455in}}%
\pgfpathlineto{\pgfqpoint{6.009411in}{1.381475in}}%
\pgfpathlineto{\pgfqpoint{6.010485in}{1.381810in}}%
\pgfpathlineto{\pgfqpoint{6.011558in}{1.381293in}}%
\pgfpathlineto{\pgfqpoint{6.012632in}{1.381923in}}%
\pgfpathlineto{\pgfqpoint{6.016928in}{1.379809in}}%
\pgfpathlineto{\pgfqpoint{6.019075in}{1.379695in}}%
\pgfpathlineto{\pgfqpoint{6.020149in}{1.381063in}}%
\pgfpathlineto{\pgfqpoint{6.025518in}{1.382461in}}%
\pgfpathlineto{\pgfqpoint{6.026592in}{1.382077in}}%
\pgfpathlineto{\pgfqpoint{6.027666in}{1.382297in}}%
\pgfpathlineto{\pgfqpoint{6.031961in}{1.381955in}}%
\pgfpathlineto{\pgfqpoint{6.033035in}{1.382834in}}%
\pgfpathlineto{\pgfqpoint{6.035183in}{1.381982in}}%
\pgfpathlineto{\pgfqpoint{6.038404in}{1.382542in}}%
\pgfpathlineto{\pgfqpoint{6.039478in}{1.386667in}}%
\pgfpathlineto{\pgfqpoint{6.041626in}{1.388649in}}%
\pgfpathlineto{\pgfqpoint{6.042700in}{1.387684in}}%
\pgfpathlineto{\pgfqpoint{6.045921in}{1.388135in}}%
\pgfpathlineto{\pgfqpoint{6.049143in}{1.390131in}}%
\pgfpathlineto{\pgfqpoint{6.050217in}{1.389491in}}%
\pgfpathlineto{\pgfqpoint{6.054512in}{1.389491in}}%
\pgfpathlineto{\pgfqpoint{6.057734in}{1.387525in}}%
\pgfpathlineto{\pgfqpoint{6.062029in}{1.388391in}}%
\pgfpathlineto{\pgfqpoint{6.063103in}{1.388391in}}%
\pgfpathlineto{\pgfqpoint{6.064177in}{1.387595in}}%
\pgfpathlineto{\pgfqpoint{6.065250in}{1.386098in}}%
\pgfpathlineto{\pgfqpoint{6.070620in}{1.385141in}}%
\pgfpathlineto{\pgfqpoint{6.072767in}{1.383336in}}%
\pgfpathlineto{\pgfqpoint{6.078136in}{1.384838in}}%
\pgfpathlineto{\pgfqpoint{6.080284in}{1.384295in}}%
\pgfpathlineto{\pgfqpoint{6.083506in}{1.384402in}}%
\pgfpathlineto{\pgfqpoint{6.085653in}{1.386100in}}%
\pgfpathlineto{\pgfqpoint{6.087801in}{1.385720in}}%
\pgfpathlineto{\pgfqpoint{6.092096in}{1.385697in}}%
\pgfpathlineto{\pgfqpoint{6.093170in}{1.384144in}}%
\pgfpathlineto{\pgfqpoint{6.094244in}{1.384338in}}%
\pgfpathlineto{\pgfqpoint{6.095318in}{1.383669in}}%
\pgfpathlineto{\pgfqpoint{6.099613in}{1.384116in}}%
\pgfpathlineto{\pgfqpoint{6.100687in}{1.384976in}}%
\pgfpathlineto{\pgfqpoint{6.109278in}{1.386246in}}%
\pgfpathlineto{\pgfqpoint{6.110352in}{1.386852in}}%
\pgfpathlineto{\pgfqpoint{6.114647in}{1.385721in}}%
\pgfpathlineto{\pgfqpoint{6.116795in}{1.386321in}}%
\pgfpathlineto{\pgfqpoint{6.117868in}{1.385871in}}%
\pgfpathlineto{\pgfqpoint{6.122164in}{1.385780in}}%
\pgfpathlineto{\pgfqpoint{6.123238in}{1.385335in}}%
\pgfpathlineto{\pgfqpoint{6.124311in}{1.385931in}}%
\pgfpathlineto{\pgfqpoint{6.128607in}{1.385551in}}%
\pgfpathlineto{\pgfqpoint{6.129681in}{1.384732in}}%
\pgfpathlineto{\pgfqpoint{6.132902in}{1.386333in}}%
\pgfpathlineto{\pgfqpoint{6.235991in}{1.386333in}}%
\pgfpathlineto{\pgfqpoint{6.237065in}{1.378777in}}%
\pgfpathlineto{\pgfqpoint{6.241360in}{1.377661in}}%
\pgfpathlineto{\pgfqpoint{6.242434in}{1.376750in}}%
\pgfpathlineto{\pgfqpoint{6.243508in}{1.377980in}}%
\pgfpathlineto{\pgfqpoint{6.245655in}{1.376683in}}%
\pgfpathlineto{\pgfqpoint{6.249951in}{1.377259in}}%
\pgfpathlineto{\pgfqpoint{6.252098in}{1.378330in}}%
\pgfpathlineto{\pgfqpoint{6.253172in}{1.378247in}}%
\pgfpathlineto{\pgfqpoint{6.256394in}{1.376892in}}%
\pgfpathlineto{\pgfqpoint{6.260689in}{1.378915in}}%
\pgfpathlineto{\pgfqpoint{6.268206in}{1.377041in}}%
\pgfpathlineto{\pgfqpoint{6.272501in}{1.377362in}}%
\pgfpathlineto{\pgfqpoint{6.274649in}{1.378353in}}%
\pgfpathlineto{\pgfqpoint{6.281092in}{1.378456in}}%
\pgfpathlineto{\pgfqpoint{6.289683in}{1.378456in}}%
\pgfpathlineto{\pgfqpoint{6.290756in}{1.377157in}}%
\pgfpathlineto{\pgfqpoint{6.293978in}{1.376729in}}%
\pgfpathlineto{\pgfqpoint{6.295052in}{1.374429in}}%
\pgfpathlineto{\pgfqpoint{6.296126in}{1.374180in}}%
\pgfpathlineto{\pgfqpoint{6.297199in}{1.375113in}}%
\pgfpathlineto{\pgfqpoint{6.298273in}{1.373420in}}%
\pgfpathlineto{\pgfqpoint{6.301495in}{1.373420in}}%
\pgfpathlineto{\pgfqpoint{6.302569in}{1.374076in}}%
\pgfpathlineto{\pgfqpoint{6.309012in}{1.374341in}}%
\pgfpathlineto{\pgfqpoint{6.310086in}{1.373481in}}%
\pgfpathlineto{\pgfqpoint{6.312233in}{1.373630in}}%
\pgfpathlineto{\pgfqpoint{6.313307in}{1.374270in}}%
\pgfpathlineto{\pgfqpoint{6.317602in}{1.373680in}}%
\pgfpathlineto{\pgfqpoint{6.319750in}{1.375065in}}%
\pgfpathlineto{\pgfqpoint{6.320824in}{1.375960in}}%
\pgfpathlineto{\pgfqpoint{6.325119in}{1.374972in}}%
\pgfpathlineto{\pgfqpoint{6.327267in}{1.373526in}}%
\pgfpathlineto{\pgfqpoint{6.328341in}{1.372963in}}%
\pgfpathlineto{\pgfqpoint{6.333710in}{1.374516in}}%
\pgfpathlineto{\pgfqpoint{6.335858in}{1.374093in}}%
\pgfpathlineto{\pgfqpoint{6.339079in}{1.375920in}}%
\pgfpathlineto{\pgfqpoint{6.341227in}{1.375920in}}%
\pgfpathlineto{\pgfqpoint{6.343375in}{1.376675in}}%
\pgfpathlineto{\pgfqpoint{6.365925in}{1.376675in}}%
\pgfpathlineto{\pgfqpoint{6.369147in}{1.375453in}}%
\pgfpathlineto{\pgfqpoint{6.372368in}{1.369781in}}%
\pgfpathlineto{\pgfqpoint{6.373442in}{1.369208in}}%
\pgfpathlineto{\pgfqpoint{6.376664in}{1.368455in}}%
\pgfpathlineto{\pgfqpoint{6.377737in}{1.369715in}}%
\pgfpathlineto{\pgfqpoint{6.379885in}{1.370517in}}%
\pgfpathlineto{\pgfqpoint{6.380959in}{1.369420in}}%
\pgfpathlineto{\pgfqpoint{6.384180in}{1.368215in}}%
\pgfpathlineto{\pgfqpoint{6.385254in}{1.366125in}}%
\pgfpathlineto{\pgfqpoint{6.388476in}{1.366897in}}%
\pgfpathlineto{\pgfqpoint{6.391697in}{1.366023in}}%
\pgfpathlineto{\pgfqpoint{6.393845in}{1.366720in}}%
\pgfpathlineto{\pgfqpoint{6.395993in}{1.366094in}}%
\pgfpathlineto{\pgfqpoint{6.403510in}{1.366469in}}%
\pgfpathlineto{\pgfqpoint{6.403510in}{1.366469in}}%
\pgfusepath{stroke}%
\end{pgfscope}%
\begin{pgfscope}%
\pgfsetrectcap%
\pgfsetmiterjoin%
\pgfsetlinewidth{0.803000pt}%
\definecolor{currentstroke}{rgb}{1.000000,1.000000,1.000000}%
\pgfsetstrokecolor{currentstroke}%
\pgfsetdash{}{0pt}%
\pgfpathmoveto{\pgfqpoint{3.937600in}{1.347524in}}%
\pgfpathlineto{\pgfqpoint{3.937600in}{1.748409in}}%
\pgfusepath{stroke}%
\end{pgfscope}%
\begin{pgfscope}%
\pgfsetrectcap%
\pgfsetmiterjoin%
\pgfsetlinewidth{0.803000pt}%
\definecolor{currentstroke}{rgb}{1.000000,1.000000,1.000000}%
\pgfsetstrokecolor{currentstroke}%
\pgfsetdash{}{0pt}%
\pgfpathmoveto{\pgfqpoint{6.520934in}{1.347524in}}%
\pgfpathlineto{\pgfqpoint{6.520934in}{1.748409in}}%
\pgfusepath{stroke}%
\end{pgfscope}%
\begin{pgfscope}%
\pgfsetrectcap%
\pgfsetmiterjoin%
\pgfsetlinewidth{0.803000pt}%
\definecolor{currentstroke}{rgb}{1.000000,1.000000,1.000000}%
\pgfsetstrokecolor{currentstroke}%
\pgfsetdash{}{0pt}%
\pgfpathmoveto{\pgfqpoint{3.937600in}{1.347524in}}%
\pgfpathlineto{\pgfqpoint{6.520934in}{1.347524in}}%
\pgfusepath{stroke}%
\end{pgfscope}%
\begin{pgfscope}%
\pgfsetrectcap%
\pgfsetmiterjoin%
\pgfsetlinewidth{0.803000pt}%
\definecolor{currentstroke}{rgb}{1.000000,1.000000,1.000000}%
\pgfsetstrokecolor{currentstroke}%
\pgfsetdash{}{0pt}%
\pgfpathmoveto{\pgfqpoint{3.937600in}{1.748409in}}%
\pgfpathlineto{\pgfqpoint{6.520934in}{1.748409in}}%
\pgfusepath{stroke}%
\end{pgfscope}%
\begin{pgfscope}%
\definecolor{textcolor}{rgb}{0.150000,0.150000,0.150000}%
\pgfsetstrokecolor{textcolor}%
\pgfsetfillcolor{textcolor}%
\pgftext[x=5.229267in,y=1.831742in,,base]{\color{textcolor}\rmfamily\fontsize{16.800000}{20.160000}\selectfont VZ}%
\end{pgfscope}%
\begin{pgfscope}%
\pgfsetbuttcap%
\pgfsetmiterjoin%
\definecolor{currentfill}{rgb}{0.917647,0.917647,0.949020}%
\pgfsetfillcolor{currentfill}%
\pgfsetlinewidth{0.000000pt}%
\definecolor{currentstroke}{rgb}{0.000000,0.000000,0.000000}%
\pgfsetstrokecolor{currentstroke}%
\pgfsetstrokeopacity{0.000000}%
\pgfsetdash{}{0pt}%
\pgfpathmoveto{\pgfqpoint{0.320934in}{0.385400in}}%
\pgfpathlineto{\pgfqpoint{2.904267in}{0.385400in}}%
\pgfpathlineto{\pgfqpoint{2.904267in}{0.786285in}}%
\pgfpathlineto{\pgfqpoint{0.320934in}{0.786285in}}%
\pgfpathclose%
\pgfusepath{fill}%
\end{pgfscope}%
\begin{pgfscope}%
\pgfpathrectangle{\pgfqpoint{0.320934in}{0.385400in}}{\pgfqpoint{2.583333in}{0.400885in}}%
\pgfusepath{clip}%
\pgfsetroundcap%
\pgfsetroundjoin%
\pgfsetlinewidth{0.803000pt}%
\definecolor{currentstroke}{rgb}{1.000000,1.000000,1.000000}%
\pgfsetstrokecolor{currentstroke}%
\pgfsetdash{}{0pt}%
\pgfpathmoveto{\pgfqpoint{0.436210in}{0.385400in}}%
\pgfpathlineto{\pgfqpoint{0.436210in}{0.786285in}}%
\pgfusepath{stroke}%
\end{pgfscope}%
\begin{pgfscope}%
\definecolor{textcolor}{rgb}{0.150000,0.150000,0.150000}%
\pgfsetstrokecolor{textcolor}%
\pgfsetfillcolor{textcolor}%
\pgftext[x=0.436210in,y=0.288178in,,top]{\color{textcolor}\rmfamily\fontsize{14.000000}{16.800000}\selectfont 2012}%
\end{pgfscope}%
\begin{pgfscope}%
\pgfpathrectangle{\pgfqpoint{0.320934in}{0.385400in}}{\pgfqpoint{2.583333in}{0.400885in}}%
\pgfusepath{clip}%
\pgfsetroundcap%
\pgfsetroundjoin%
\pgfsetlinewidth{0.803000pt}%
\definecolor{currentstroke}{rgb}{1.000000,1.000000,1.000000}%
\pgfsetstrokecolor{currentstroke}%
\pgfsetdash{}{0pt}%
\pgfpathmoveto{\pgfqpoint{0.829235in}{0.385400in}}%
\pgfpathlineto{\pgfqpoint{0.829235in}{0.786285in}}%
\pgfusepath{stroke}%
\end{pgfscope}%
\begin{pgfscope}%
\definecolor{textcolor}{rgb}{0.150000,0.150000,0.150000}%
\pgfsetstrokecolor{textcolor}%
\pgfsetfillcolor{textcolor}%
\pgftext[x=0.829235in,y=0.288178in,,top]{\color{textcolor}\rmfamily\fontsize{14.000000}{16.800000}\selectfont 2013}%
\end{pgfscope}%
\begin{pgfscope}%
\pgfpathrectangle{\pgfqpoint{0.320934in}{0.385400in}}{\pgfqpoint{2.583333in}{0.400885in}}%
\pgfusepath{clip}%
\pgfsetroundcap%
\pgfsetroundjoin%
\pgfsetlinewidth{0.803000pt}%
\definecolor{currentstroke}{rgb}{1.000000,1.000000,1.000000}%
\pgfsetstrokecolor{currentstroke}%
\pgfsetdash{}{0pt}%
\pgfpathmoveto{\pgfqpoint{1.221186in}{0.385400in}}%
\pgfpathlineto{\pgfqpoint{1.221186in}{0.786285in}}%
\pgfusepath{stroke}%
\end{pgfscope}%
\begin{pgfscope}%
\definecolor{textcolor}{rgb}{0.150000,0.150000,0.150000}%
\pgfsetstrokecolor{textcolor}%
\pgfsetfillcolor{textcolor}%
\pgftext[x=1.221186in,y=0.288178in,,top]{\color{textcolor}\rmfamily\fontsize{14.000000}{16.800000}\selectfont 2014}%
\end{pgfscope}%
\begin{pgfscope}%
\pgfpathrectangle{\pgfqpoint{0.320934in}{0.385400in}}{\pgfqpoint{2.583333in}{0.400885in}}%
\pgfusepath{clip}%
\pgfsetroundcap%
\pgfsetroundjoin%
\pgfsetlinewidth{0.803000pt}%
\definecolor{currentstroke}{rgb}{1.000000,1.000000,1.000000}%
\pgfsetstrokecolor{currentstroke}%
\pgfsetdash{}{0pt}%
\pgfpathmoveto{\pgfqpoint{1.613137in}{0.385400in}}%
\pgfpathlineto{\pgfqpoint{1.613137in}{0.786285in}}%
\pgfusepath{stroke}%
\end{pgfscope}%
\begin{pgfscope}%
\definecolor{textcolor}{rgb}{0.150000,0.150000,0.150000}%
\pgfsetstrokecolor{textcolor}%
\pgfsetfillcolor{textcolor}%
\pgftext[x=1.613137in,y=0.288178in,,top]{\color{textcolor}\rmfamily\fontsize{14.000000}{16.800000}\selectfont 2015}%
\end{pgfscope}%
\begin{pgfscope}%
\pgfpathrectangle{\pgfqpoint{0.320934in}{0.385400in}}{\pgfqpoint{2.583333in}{0.400885in}}%
\pgfusepath{clip}%
\pgfsetroundcap%
\pgfsetroundjoin%
\pgfsetlinewidth{0.803000pt}%
\definecolor{currentstroke}{rgb}{1.000000,1.000000,1.000000}%
\pgfsetstrokecolor{currentstroke}%
\pgfsetdash{}{0pt}%
\pgfpathmoveto{\pgfqpoint{2.005088in}{0.385400in}}%
\pgfpathlineto{\pgfqpoint{2.005088in}{0.786285in}}%
\pgfusepath{stroke}%
\end{pgfscope}%
\begin{pgfscope}%
\definecolor{textcolor}{rgb}{0.150000,0.150000,0.150000}%
\pgfsetstrokecolor{textcolor}%
\pgfsetfillcolor{textcolor}%
\pgftext[x=2.005088in,y=0.288178in,,top]{\color{textcolor}\rmfamily\fontsize{14.000000}{16.800000}\selectfont 2016}%
\end{pgfscope}%
\begin{pgfscope}%
\pgfpathrectangle{\pgfqpoint{0.320934in}{0.385400in}}{\pgfqpoint{2.583333in}{0.400885in}}%
\pgfusepath{clip}%
\pgfsetroundcap%
\pgfsetroundjoin%
\pgfsetlinewidth{0.803000pt}%
\definecolor{currentstroke}{rgb}{1.000000,1.000000,1.000000}%
\pgfsetstrokecolor{currentstroke}%
\pgfsetdash{}{0pt}%
\pgfpathmoveto{\pgfqpoint{2.398113in}{0.385400in}}%
\pgfpathlineto{\pgfqpoint{2.398113in}{0.786285in}}%
\pgfusepath{stroke}%
\end{pgfscope}%
\begin{pgfscope}%
\definecolor{textcolor}{rgb}{0.150000,0.150000,0.150000}%
\pgfsetstrokecolor{textcolor}%
\pgfsetfillcolor{textcolor}%
\pgftext[x=2.398113in,y=0.288178in,,top]{\color{textcolor}\rmfamily\fontsize{14.000000}{16.800000}\selectfont 2017}%
\end{pgfscope}%
\begin{pgfscope}%
\pgfpathrectangle{\pgfqpoint{0.320934in}{0.385400in}}{\pgfqpoint{2.583333in}{0.400885in}}%
\pgfusepath{clip}%
\pgfsetroundcap%
\pgfsetroundjoin%
\pgfsetlinewidth{0.803000pt}%
\definecolor{currentstroke}{rgb}{1.000000,1.000000,1.000000}%
\pgfsetstrokecolor{currentstroke}%
\pgfsetdash{}{0pt}%
\pgfpathmoveto{\pgfqpoint{2.790064in}{0.385400in}}%
\pgfpathlineto{\pgfqpoint{2.790064in}{0.786285in}}%
\pgfusepath{stroke}%
\end{pgfscope}%
\begin{pgfscope}%
\definecolor{textcolor}{rgb}{0.150000,0.150000,0.150000}%
\pgfsetstrokecolor{textcolor}%
\pgfsetfillcolor{textcolor}%
\pgftext[x=2.790064in,y=0.288178in,,top]{\color{textcolor}\rmfamily\fontsize{14.000000}{16.800000}\selectfont 2018}%
\end{pgfscope}%
\begin{pgfscope}%
\pgfpathrectangle{\pgfqpoint{0.320934in}{0.385400in}}{\pgfqpoint{2.583333in}{0.400885in}}%
\pgfusepath{clip}%
\pgfsetroundcap%
\pgfsetroundjoin%
\pgfsetlinewidth{0.803000pt}%
\definecolor{currentstroke}{rgb}{1.000000,1.000000,1.000000}%
\pgfsetstrokecolor{currentstroke}%
\pgfsetdash{}{0pt}%
\pgfpathmoveto{\pgfqpoint{0.320934in}{0.394341in}}%
\pgfpathlineto{\pgfqpoint{2.904267in}{0.394341in}}%
\pgfusepath{stroke}%
\end{pgfscope}%
\begin{pgfscope}%
\definecolor{textcolor}{rgb}{0.150000,0.150000,0.150000}%
\pgfsetstrokecolor{textcolor}%
\pgfsetfillcolor{textcolor}%
\pgftext[x=0.100000in,y=0.320475in,left,base]{\color{textcolor}\rmfamily\fontsize{14.000000}{16.800000}\selectfont 0}%
\end{pgfscope}%
\begin{pgfscope}%
\pgfpathrectangle{\pgfqpoint{0.320934in}{0.385400in}}{\pgfqpoint{2.583333in}{0.400885in}}%
\pgfusepath{clip}%
\pgfsetroundcap%
\pgfsetroundjoin%
\pgfsetlinewidth{0.803000pt}%
\definecolor{currentstroke}{rgb}{1.000000,1.000000,1.000000}%
\pgfsetstrokecolor{currentstroke}%
\pgfsetdash{}{0pt}%
\pgfpathmoveto{\pgfqpoint{0.320934in}{0.766708in}}%
\pgfpathlineto{\pgfqpoint{2.904267in}{0.766708in}}%
\pgfusepath{stroke}%
\end{pgfscope}%
\begin{pgfscope}%
\definecolor{textcolor}{rgb}{0.150000,0.150000,0.150000}%
\pgfsetstrokecolor{textcolor}%
\pgfsetfillcolor{textcolor}%
\pgftext[x=0.100000in,y=0.692842in,left,base]{\color{textcolor}\rmfamily\fontsize{14.000000}{16.800000}\selectfont 5}%
\end{pgfscope}%
\begin{pgfscope}%
\pgfpathrectangle{\pgfqpoint{0.320934in}{0.385400in}}{\pgfqpoint{2.583333in}{0.400885in}}%
\pgfusepath{clip}%
\pgfsetroundcap%
\pgfsetroundjoin%
\pgfsetlinewidth{1.505625pt}%
\definecolor{currentstroke}{rgb}{0.000000,0.000000,0.000000}%
\pgfsetstrokecolor{currentstroke}%
\pgfsetdash{}{0pt}%
\pgfpathmoveto{\pgfqpoint{0.438358in}{0.468814in}}%
\pgfpathlineto{\pgfqpoint{0.439432in}{0.467492in}}%
\pgfpathlineto{\pgfqpoint{0.440506in}{0.468054in}}%
\pgfpathlineto{\pgfqpoint{0.441580in}{0.467194in}}%
\pgfpathlineto{\pgfqpoint{0.446949in}{0.466037in}}%
\pgfpathlineto{\pgfqpoint{0.448023in}{0.467624in}}%
\pgfpathlineto{\pgfqpoint{0.449096in}{0.467194in}}%
\pgfpathlineto{\pgfqpoint{0.453392in}{0.468484in}}%
\pgfpathlineto{\pgfqpoint{0.454466in}{0.469409in}}%
\pgfpathlineto{\pgfqpoint{0.456613in}{0.467095in}}%
\pgfpathlineto{\pgfqpoint{0.459835in}{0.466368in}}%
\pgfpathlineto{\pgfqpoint{0.460909in}{0.467393in}}%
\pgfpathlineto{\pgfqpoint{0.463056in}{0.467294in}}%
\pgfpathlineto{\pgfqpoint{0.464130in}{0.467426in}}%
\pgfpathlineto{\pgfqpoint{0.467352in}{0.466566in}}%
\pgfpathlineto{\pgfqpoint{0.468425in}{0.467128in}}%
\pgfpathlineto{\pgfqpoint{0.469499in}{0.468484in}}%
\pgfpathlineto{\pgfqpoint{0.470573in}{0.471062in}}%
\pgfpathlineto{\pgfqpoint{0.471647in}{0.471756in}}%
\pgfpathlineto{\pgfqpoint{0.475942in}{0.471723in}}%
\pgfpathlineto{\pgfqpoint{0.477016in}{0.472715in}}%
\pgfpathlineto{\pgfqpoint{0.478090in}{0.475657in}}%
\pgfpathlineto{\pgfqpoint{0.479164in}{0.476714in}}%
\pgfpathlineto{\pgfqpoint{0.482385in}{0.475855in}}%
\pgfpathlineto{\pgfqpoint{0.483459in}{0.477673in}}%
\pgfpathlineto{\pgfqpoint{0.484533in}{0.478235in}}%
\pgfpathlineto{\pgfqpoint{0.485607in}{0.477276in}}%
\pgfpathlineto{\pgfqpoint{0.486681in}{0.478169in}}%
\pgfpathlineto{\pgfqpoint{0.490976in}{0.477574in}}%
\pgfpathlineto{\pgfqpoint{0.492050in}{0.479160in}}%
\pgfpathlineto{\pgfqpoint{0.493124in}{0.479226in}}%
\pgfpathlineto{\pgfqpoint{0.494198in}{0.480020in}}%
\pgfpathlineto{\pgfqpoint{0.497419in}{0.479524in}}%
\pgfpathlineto{\pgfqpoint{0.498493in}{0.481045in}}%
\pgfpathlineto{\pgfqpoint{0.499567in}{0.479160in}}%
\pgfpathlineto{\pgfqpoint{0.500641in}{0.479755in}}%
\pgfpathlineto{\pgfqpoint{0.501714in}{0.478995in}}%
\pgfpathlineto{\pgfqpoint{0.504936in}{0.479061in}}%
\pgfpathlineto{\pgfqpoint{0.506010in}{0.478037in}}%
\pgfpathlineto{\pgfqpoint{0.507084in}{0.478632in}}%
\pgfpathlineto{\pgfqpoint{0.508157in}{0.480317in}}%
\pgfpathlineto{\pgfqpoint{0.509231in}{0.479722in}}%
\pgfpathlineto{\pgfqpoint{0.512453in}{0.479260in}}%
\pgfpathlineto{\pgfqpoint{0.513527in}{0.479821in}}%
\pgfpathlineto{\pgfqpoint{0.514601in}{0.479425in}}%
\pgfpathlineto{\pgfqpoint{0.519970in}{0.480945in}}%
\pgfpathlineto{\pgfqpoint{0.521044in}{0.479260in}}%
\pgfpathlineto{\pgfqpoint{0.523191in}{0.479821in}}%
\pgfpathlineto{\pgfqpoint{0.524265in}{0.480912in}}%
\pgfpathlineto{\pgfqpoint{0.528560in}{0.481673in}}%
\pgfpathlineto{\pgfqpoint{0.531782in}{0.480350in}}%
\pgfpathlineto{\pgfqpoint{0.535003in}{0.481045in}}%
\pgfpathlineto{\pgfqpoint{0.536077in}{0.482069in}}%
\pgfpathlineto{\pgfqpoint{0.537151in}{0.481045in}}%
\pgfpathlineto{\pgfqpoint{0.538225in}{0.482532in}}%
\pgfpathlineto{\pgfqpoint{0.542520in}{0.481441in}}%
\pgfpathlineto{\pgfqpoint{0.543594in}{0.479425in}}%
\pgfpathlineto{\pgfqpoint{0.544668in}{0.479888in}}%
\pgfpathlineto{\pgfqpoint{0.546816in}{0.484086in}}%
\pgfpathlineto{\pgfqpoint{0.550037in}{0.482400in}}%
\pgfpathlineto{\pgfqpoint{0.551111in}{0.483325in}}%
\pgfpathlineto{\pgfqpoint{0.554333in}{0.482532in}}%
\pgfpathlineto{\pgfqpoint{0.557554in}{0.480416in}}%
\pgfpathlineto{\pgfqpoint{0.558628in}{0.481011in}}%
\pgfpathlineto{\pgfqpoint{0.560776in}{0.484119in}}%
\pgfpathlineto{\pgfqpoint{0.561849in}{0.484350in}}%
\pgfpathlineto{\pgfqpoint{0.567219in}{0.483391in}}%
\pgfpathlineto{\pgfqpoint{0.568292in}{0.479193in}}%
\pgfpathlineto{\pgfqpoint{0.569366in}{0.480185in}}%
\pgfpathlineto{\pgfqpoint{0.573662in}{0.480450in}}%
\pgfpathlineto{\pgfqpoint{0.574735in}{0.479921in}}%
\pgfpathlineto{\pgfqpoint{0.575809in}{0.480450in}}%
\pgfpathlineto{\pgfqpoint{0.581179in}{0.479326in}}%
\pgfpathlineto{\pgfqpoint{0.582252in}{0.480648in}}%
\pgfpathlineto{\pgfqpoint{0.584400in}{0.477045in}}%
\pgfpathlineto{\pgfqpoint{0.590843in}{0.482301in}}%
\pgfpathlineto{\pgfqpoint{0.591917in}{0.482003in}}%
\pgfpathlineto{\pgfqpoint{0.596212in}{0.482664in}}%
\pgfpathlineto{\pgfqpoint{0.599434in}{0.476780in}}%
\pgfpathlineto{\pgfqpoint{0.602655in}{0.478169in}}%
\pgfpathlineto{\pgfqpoint{0.603729in}{0.477904in}}%
\pgfpathlineto{\pgfqpoint{0.604803in}{0.479788in}}%
\pgfpathlineto{\pgfqpoint{0.606951in}{0.479954in}}%
\pgfpathlineto{\pgfqpoint{0.610172in}{0.479788in}}%
\pgfpathlineto{\pgfqpoint{0.611246in}{0.480383in}}%
\pgfpathlineto{\pgfqpoint{0.612320in}{0.478731in}}%
\pgfpathlineto{\pgfqpoint{0.614468in}{0.481276in}}%
\pgfpathlineto{\pgfqpoint{0.618763in}{0.483458in}}%
\pgfpathlineto{\pgfqpoint{0.619837in}{0.484350in}}%
\pgfpathlineto{\pgfqpoint{0.620911in}{0.482069in}}%
\pgfpathlineto{\pgfqpoint{0.621984in}{0.486102in}}%
\pgfpathlineto{\pgfqpoint{0.625206in}{0.483391in}}%
\pgfpathlineto{\pgfqpoint{0.626280in}{0.484780in}}%
\pgfpathlineto{\pgfqpoint{0.627354in}{0.484978in}}%
\pgfpathlineto{\pgfqpoint{0.628427in}{0.483590in}}%
\pgfpathlineto{\pgfqpoint{0.629501in}{0.485143in}}%
\pgfpathlineto{\pgfqpoint{0.632723in}{0.487325in}}%
\pgfpathlineto{\pgfqpoint{0.633797in}{0.487127in}}%
\pgfpathlineto{\pgfqpoint{0.635944in}{0.487623in}}%
\pgfpathlineto{\pgfqpoint{0.637018in}{0.486333in}}%
\pgfpathlineto{\pgfqpoint{0.640240in}{0.485143in}}%
\pgfpathlineto{\pgfqpoint{0.642387in}{0.482995in}}%
\pgfpathlineto{\pgfqpoint{0.644535in}{0.485474in}}%
\pgfpathlineto{\pgfqpoint{0.648830in}{0.488515in}}%
\pgfpathlineto{\pgfqpoint{0.649904in}{0.488218in}}%
\pgfpathlineto{\pgfqpoint{0.650978in}{0.486433in}}%
\pgfpathlineto{\pgfqpoint{0.652052in}{0.486697in}}%
\pgfpathlineto{\pgfqpoint{0.655273in}{0.485738in}}%
\pgfpathlineto{\pgfqpoint{0.656347in}{0.484416in}}%
\pgfpathlineto{\pgfqpoint{0.657421in}{0.484086in}}%
\pgfpathlineto{\pgfqpoint{0.659569in}{0.489176in}}%
\pgfpathlineto{\pgfqpoint{0.662790in}{0.490432in}}%
\pgfpathlineto{\pgfqpoint{0.664938in}{0.487953in}}%
\pgfpathlineto{\pgfqpoint{0.667086in}{0.490531in}}%
\pgfpathlineto{\pgfqpoint{0.672455in}{0.490664in}}%
\pgfpathlineto{\pgfqpoint{0.673529in}{0.488713in}}%
\pgfpathlineto{\pgfqpoint{0.674602in}{0.489143in}}%
\pgfpathlineto{\pgfqpoint{0.677824in}{0.488581in}}%
\pgfpathlineto{\pgfqpoint{0.678898in}{0.489738in}}%
\pgfpathlineto{\pgfqpoint{0.682119in}{0.489474in}}%
\pgfpathlineto{\pgfqpoint{0.688562in}{0.488350in}}%
\pgfpathlineto{\pgfqpoint{0.689636in}{0.487986in}}%
\pgfpathlineto{\pgfqpoint{0.695005in}{0.488978in}}%
\pgfpathlineto{\pgfqpoint{0.696079in}{0.488019in}}%
\pgfpathlineto{\pgfqpoint{0.697153in}{0.489176in}}%
\pgfpathlineto{\pgfqpoint{0.701448in}{0.489374in}}%
\pgfpathlineto{\pgfqpoint{0.702522in}{0.488680in}}%
\pgfpathlineto{\pgfqpoint{0.703596in}{0.490102in}}%
\pgfpathlineto{\pgfqpoint{0.704670in}{0.490234in}}%
\pgfpathlineto{\pgfqpoint{0.707891in}{0.489474in}}%
\pgfpathlineto{\pgfqpoint{0.710039in}{0.493011in}}%
\pgfpathlineto{\pgfqpoint{0.711113in}{0.494134in}}%
\pgfpathlineto{\pgfqpoint{0.712187in}{0.493606in}}%
\pgfpathlineto{\pgfqpoint{0.716482in}{0.493176in}}%
\pgfpathlineto{\pgfqpoint{0.717556in}{0.494002in}}%
\pgfpathlineto{\pgfqpoint{0.722925in}{0.493209in}}%
\pgfpathlineto{\pgfqpoint{0.723999in}{0.493837in}}%
\pgfpathlineto{\pgfqpoint{0.725073in}{0.492316in}}%
\pgfpathlineto{\pgfqpoint{0.727221in}{0.493639in}}%
\pgfpathlineto{\pgfqpoint{0.730442in}{0.495424in}}%
\pgfpathlineto{\pgfqpoint{0.731516in}{0.494895in}}%
\pgfpathlineto{\pgfqpoint{0.733664in}{0.497605in}}%
\pgfpathlineto{\pgfqpoint{0.734737in}{0.498035in}}%
\pgfpathlineto{\pgfqpoint{0.737959in}{0.496911in}}%
\pgfpathlineto{\pgfqpoint{0.739033in}{0.495457in}}%
\pgfpathlineto{\pgfqpoint{0.740107in}{0.495952in}}%
\pgfpathlineto{\pgfqpoint{0.741180in}{0.497142in}}%
\pgfpathlineto{\pgfqpoint{0.745476in}{0.497770in}}%
\pgfpathlineto{\pgfqpoint{0.747623in}{0.499754in}}%
\pgfpathlineto{\pgfqpoint{0.748697in}{0.499258in}}%
\pgfpathlineto{\pgfqpoint{0.749771in}{0.497837in}}%
\pgfpathlineto{\pgfqpoint{0.752993in}{0.497109in}}%
\pgfpathlineto{\pgfqpoint{0.754067in}{0.495357in}}%
\pgfpathlineto{\pgfqpoint{0.755140in}{0.495258in}}%
\pgfpathlineto{\pgfqpoint{0.757288in}{0.496614in}}%
\pgfpathlineto{\pgfqpoint{0.762657in}{0.496944in}}%
\pgfpathlineto{\pgfqpoint{0.763731in}{0.500712in}}%
\pgfpathlineto{\pgfqpoint{0.768026in}{0.498960in}}%
\pgfpathlineto{\pgfqpoint{0.769100in}{0.500712in}}%
\pgfpathlineto{\pgfqpoint{0.771248in}{0.499390in}}%
\pgfpathlineto{\pgfqpoint{0.772322in}{0.500018in}}%
\pgfpathlineto{\pgfqpoint{0.775543in}{0.500250in}}%
\pgfpathlineto{\pgfqpoint{0.778765in}{0.498828in}}%
\pgfpathlineto{\pgfqpoint{0.779839in}{0.500845in}}%
\pgfpathlineto{\pgfqpoint{0.784134in}{0.503555in}}%
\pgfpathlineto{\pgfqpoint{0.785208in}{0.503787in}}%
\pgfpathlineto{\pgfqpoint{0.787356in}{0.504877in}}%
\pgfpathlineto{\pgfqpoint{0.792725in}{0.504249in}}%
\pgfpathlineto{\pgfqpoint{0.794872in}{0.506067in}}%
\pgfpathlineto{\pgfqpoint{0.800242in}{0.504844in}}%
\pgfpathlineto{\pgfqpoint{0.802389in}{0.505208in}}%
\pgfpathlineto{\pgfqpoint{0.806685in}{0.505472in}}%
\pgfpathlineto{\pgfqpoint{0.808832in}{0.504183in}}%
\pgfpathlineto{\pgfqpoint{0.809906in}{0.503919in}}%
\pgfpathlineto{\pgfqpoint{0.814201in}{0.506828in}}%
\pgfpathlineto{\pgfqpoint{0.815275in}{0.505671in}}%
\pgfpathlineto{\pgfqpoint{0.816349in}{0.508117in}}%
\pgfpathlineto{\pgfqpoint{0.817423in}{0.506861in}}%
\pgfpathlineto{\pgfqpoint{0.820645in}{0.507158in}}%
\pgfpathlineto{\pgfqpoint{0.824940in}{0.505274in}}%
\pgfpathlineto{\pgfqpoint{0.828161in}{0.507456in}}%
\pgfpathlineto{\pgfqpoint{0.830309in}{0.510298in}}%
\pgfpathlineto{\pgfqpoint{0.831383in}{0.510398in}}%
\pgfpathlineto{\pgfqpoint{0.832457in}{0.511323in}}%
\pgfpathlineto{\pgfqpoint{0.835678in}{0.512183in}}%
\pgfpathlineto{\pgfqpoint{0.837826in}{0.515091in}}%
\pgfpathlineto{\pgfqpoint{0.838900in}{0.514133in}}%
\pgfpathlineto{\pgfqpoint{0.839974in}{0.514596in}}%
\pgfpathlineto{\pgfqpoint{0.846417in}{0.513670in}}%
\pgfpathlineto{\pgfqpoint{0.847490in}{0.512447in}}%
\pgfpathlineto{\pgfqpoint{0.855007in}{0.513637in}}%
\pgfpathlineto{\pgfqpoint{0.858229in}{0.511026in}}%
\pgfpathlineto{\pgfqpoint{0.859303in}{0.511257in}}%
\pgfpathlineto{\pgfqpoint{0.860377in}{0.510034in}}%
\pgfpathlineto{\pgfqpoint{0.861450in}{0.512183in}}%
\pgfpathlineto{\pgfqpoint{0.862524in}{0.512678in}}%
\pgfpathlineto{\pgfqpoint{0.865746in}{0.511257in}}%
\pgfpathlineto{\pgfqpoint{0.867893in}{0.514364in}}%
\pgfpathlineto{\pgfqpoint{0.868967in}{0.511555in}}%
\pgfpathlineto{\pgfqpoint{0.870041in}{0.512116in}}%
\pgfpathlineto{\pgfqpoint{0.873263in}{0.510993in}}%
\pgfpathlineto{\pgfqpoint{0.874336in}{0.511356in}}%
\pgfpathlineto{\pgfqpoint{0.875410in}{0.510827in}}%
\pgfpathlineto{\pgfqpoint{0.877558in}{0.513240in}}%
\pgfpathlineto{\pgfqpoint{0.881853in}{0.512976in}}%
\pgfpathlineto{\pgfqpoint{0.882927in}{0.511323in}}%
\pgfpathlineto{\pgfqpoint{0.885075in}{0.514331in}}%
\pgfpathlineto{\pgfqpoint{0.888296in}{0.511654in}}%
\pgfpathlineto{\pgfqpoint{0.890444in}{0.514662in}}%
\pgfpathlineto{\pgfqpoint{0.892592in}{0.513306in}}%
\pgfpathlineto{\pgfqpoint{0.899035in}{0.515158in}}%
\pgfpathlineto{\pgfqpoint{0.903330in}{0.515653in}}%
\pgfpathlineto{\pgfqpoint{0.905478in}{0.514298in}}%
\pgfpathlineto{\pgfqpoint{0.906552in}{0.515257in}}%
\pgfpathlineto{\pgfqpoint{0.907625in}{0.513670in}}%
\pgfpathlineto{\pgfqpoint{0.910847in}{0.513306in}}%
\pgfpathlineto{\pgfqpoint{0.911921in}{0.511753in}}%
\pgfpathlineto{\pgfqpoint{0.912995in}{0.514265in}}%
\pgfpathlineto{\pgfqpoint{0.914068in}{0.513108in}}%
\pgfpathlineto{\pgfqpoint{0.915142in}{0.514827in}}%
\pgfpathlineto{\pgfqpoint{0.918364in}{0.517769in}}%
\pgfpathlineto{\pgfqpoint{0.919438in}{0.520380in}}%
\pgfpathlineto{\pgfqpoint{0.921585in}{0.522165in}}%
\pgfpathlineto{\pgfqpoint{0.925881in}{0.520281in}}%
\pgfpathlineto{\pgfqpoint{0.926955in}{0.520744in}}%
\pgfpathlineto{\pgfqpoint{0.928028in}{0.518331in}}%
\pgfpathlineto{\pgfqpoint{0.929102in}{0.519620in}}%
\pgfpathlineto{\pgfqpoint{0.930176in}{0.518628in}}%
\pgfpathlineto{\pgfqpoint{0.933398in}{0.519488in}}%
\pgfpathlineto{\pgfqpoint{0.934471in}{0.518331in}}%
\pgfpathlineto{\pgfqpoint{0.935545in}{0.520017in}}%
\pgfpathlineto{\pgfqpoint{0.936619in}{0.520479in}}%
\pgfpathlineto{\pgfqpoint{0.940914in}{0.515719in}}%
\pgfpathlineto{\pgfqpoint{0.941988in}{0.518298in}}%
\pgfpathlineto{\pgfqpoint{0.943062in}{0.516447in}}%
\pgfpathlineto{\pgfqpoint{0.944136in}{0.515819in}}%
\pgfpathlineto{\pgfqpoint{0.945210in}{0.517736in}}%
\pgfpathlineto{\pgfqpoint{0.948431in}{0.517405in}}%
\pgfpathlineto{\pgfqpoint{0.951653in}{0.521339in}}%
\pgfpathlineto{\pgfqpoint{0.952727in}{0.520215in}}%
\pgfpathlineto{\pgfqpoint{0.957022in}{0.521141in}}%
\pgfpathlineto{\pgfqpoint{0.958096in}{0.519289in}}%
\pgfpathlineto{\pgfqpoint{0.959170in}{0.526363in}}%
\pgfpathlineto{\pgfqpoint{0.960244in}{0.529470in}}%
\pgfpathlineto{\pgfqpoint{0.963465in}{0.528975in}}%
\pgfpathlineto{\pgfqpoint{0.964539in}{0.529669in}}%
\pgfpathlineto{\pgfqpoint{0.967760in}{0.528875in}}%
\pgfpathlineto{\pgfqpoint{0.970982in}{0.529008in}}%
\pgfpathlineto{\pgfqpoint{0.972056in}{0.530065in}}%
\pgfpathlineto{\pgfqpoint{0.973130in}{0.532148in}}%
\pgfpathlineto{\pgfqpoint{0.974203in}{0.530660in}}%
\pgfpathlineto{\pgfqpoint{0.975277in}{0.534263in}}%
\pgfpathlineto{\pgfqpoint{0.978499in}{0.532016in}}%
\pgfpathlineto{\pgfqpoint{0.979573in}{0.531983in}}%
\pgfpathlineto{\pgfqpoint{0.981720in}{0.529371in}}%
\pgfpathlineto{\pgfqpoint{0.982794in}{0.531156in}}%
\pgfpathlineto{\pgfqpoint{0.987089in}{0.530826in}}%
\pgfpathlineto{\pgfqpoint{0.988163in}{0.529404in}}%
\pgfpathlineto{\pgfqpoint{0.989237in}{0.531586in}}%
\pgfpathlineto{\pgfqpoint{0.990311in}{0.529404in}}%
\pgfpathlineto{\pgfqpoint{0.994606in}{0.530958in}}%
\pgfpathlineto{\pgfqpoint{0.995680in}{0.528710in}}%
\pgfpathlineto{\pgfqpoint{0.996754in}{0.530330in}}%
\pgfpathlineto{\pgfqpoint{1.001049in}{0.532578in}}%
\pgfpathlineto{\pgfqpoint{1.002123in}{0.530561in}}%
\pgfpathlineto{\pgfqpoint{1.003197in}{0.530165in}}%
\pgfpathlineto{\pgfqpoint{1.004271in}{0.532512in}}%
\pgfpathlineto{\pgfqpoint{1.005345in}{0.531520in}}%
\pgfpathlineto{\pgfqpoint{1.008566in}{0.532578in}}%
\pgfpathlineto{\pgfqpoint{1.009640in}{0.533867in}}%
\pgfpathlineto{\pgfqpoint{1.010714in}{0.532809in}}%
\pgfpathlineto{\pgfqpoint{1.011788in}{0.529636in}}%
\pgfpathlineto{\pgfqpoint{1.012862in}{0.530429in}}%
\pgfpathlineto{\pgfqpoint{1.016083in}{0.529504in}}%
\pgfpathlineto{\pgfqpoint{1.019305in}{0.534032in}}%
\pgfpathlineto{\pgfqpoint{1.020378in}{0.532908in}}%
\pgfpathlineto{\pgfqpoint{1.025748in}{0.536148in}}%
\pgfpathlineto{\pgfqpoint{1.027895in}{0.538990in}}%
\pgfpathlineto{\pgfqpoint{1.033265in}{0.535883in}}%
\pgfpathlineto{\pgfqpoint{1.034338in}{0.538362in}}%
\pgfpathlineto{\pgfqpoint{1.035412in}{0.538924in}}%
\pgfpathlineto{\pgfqpoint{1.040781in}{0.538131in}}%
\pgfpathlineto{\pgfqpoint{1.041855in}{0.539156in}}%
\pgfpathlineto{\pgfqpoint{1.042929in}{0.538329in}}%
\pgfpathlineto{\pgfqpoint{1.046151in}{0.539321in}}%
\pgfpathlineto{\pgfqpoint{1.048298in}{0.535916in}}%
\pgfpathlineto{\pgfqpoint{1.049372in}{0.541899in}}%
\pgfpathlineto{\pgfqpoint{1.050446in}{0.540808in}}%
\pgfpathlineto{\pgfqpoint{1.054741in}{0.539486in}}%
\pgfpathlineto{\pgfqpoint{1.055815in}{0.528545in}}%
\pgfpathlineto{\pgfqpoint{1.056889in}{0.530165in}}%
\pgfpathlineto{\pgfqpoint{1.057963in}{0.533834in}}%
\pgfpathlineto{\pgfqpoint{1.061184in}{0.534263in}}%
\pgfpathlineto{\pgfqpoint{1.063332in}{0.531850in}}%
\pgfpathlineto{\pgfqpoint{1.065480in}{0.530660in}}%
\pgfpathlineto{\pgfqpoint{1.070849in}{0.530561in}}%
\pgfpathlineto{\pgfqpoint{1.071923in}{0.527190in}}%
\pgfpathlineto{\pgfqpoint{1.072997in}{0.526562in}}%
\pgfpathlineto{\pgfqpoint{1.076218in}{0.527983in}}%
\pgfpathlineto{\pgfqpoint{1.077292in}{0.526628in}}%
\pgfpathlineto{\pgfqpoint{1.078366in}{0.530594in}}%
\pgfpathlineto{\pgfqpoint{1.080513in}{0.531123in}}%
\pgfpathlineto{\pgfqpoint{1.084809in}{0.527355in}}%
\pgfpathlineto{\pgfqpoint{1.086956in}{0.528280in}}%
\pgfpathlineto{\pgfqpoint{1.088030in}{0.527553in}}%
\pgfpathlineto{\pgfqpoint{1.092326in}{0.529537in}}%
\pgfpathlineto{\pgfqpoint{1.093399in}{0.528909in}}%
\pgfpathlineto{\pgfqpoint{1.095547in}{0.529272in}}%
\pgfpathlineto{\pgfqpoint{1.098769in}{0.530727in}}%
\pgfpathlineto{\pgfqpoint{1.099843in}{0.535321in}}%
\pgfpathlineto{\pgfqpoint{1.100916in}{0.536643in}}%
\pgfpathlineto{\pgfqpoint{1.101990in}{0.535685in}}%
\pgfpathlineto{\pgfqpoint{1.103064in}{0.538693in}}%
\pgfpathlineto{\pgfqpoint{1.106286in}{0.538990in}}%
\pgfpathlineto{\pgfqpoint{1.110581in}{0.546196in}}%
\pgfpathlineto{\pgfqpoint{1.113802in}{0.544213in}}%
\pgfpathlineto{\pgfqpoint{1.115950in}{0.540643in}}%
\pgfpathlineto{\pgfqpoint{1.117024in}{0.542164in}}%
\pgfpathlineto{\pgfqpoint{1.121319in}{0.540313in}}%
\pgfpathlineto{\pgfqpoint{1.122393in}{0.541932in}}%
\pgfpathlineto{\pgfqpoint{1.123467in}{0.540841in}}%
\pgfpathlineto{\pgfqpoint{1.124541in}{0.538428in}}%
\pgfpathlineto{\pgfqpoint{1.125615in}{0.539817in}}%
\pgfpathlineto{\pgfqpoint{1.128836in}{0.536643in}}%
\pgfpathlineto{\pgfqpoint{1.129910in}{0.533768in}}%
\pgfpathlineto{\pgfqpoint{1.130984in}{0.534759in}}%
\pgfpathlineto{\pgfqpoint{1.133132in}{0.541139in}}%
\pgfpathlineto{\pgfqpoint{1.136353in}{0.542098in}}%
\pgfpathlineto{\pgfqpoint{1.137427in}{0.540511in}}%
\pgfpathlineto{\pgfqpoint{1.139575in}{0.545800in}}%
\pgfpathlineto{\pgfqpoint{1.140648in}{0.547453in}}%
\pgfpathlineto{\pgfqpoint{1.144944in}{0.547089in}}%
\pgfpathlineto{\pgfqpoint{1.146018in}{0.546263in}}%
\pgfpathlineto{\pgfqpoint{1.147091in}{0.549337in}}%
\pgfpathlineto{\pgfqpoint{1.153534in}{0.550031in}}%
\pgfpathlineto{\pgfqpoint{1.154608in}{0.544544in}}%
\pgfpathlineto{\pgfqpoint{1.155682in}{0.546461in}}%
\pgfpathlineto{\pgfqpoint{1.158904in}{0.544378in}}%
\pgfpathlineto{\pgfqpoint{1.161051in}{0.546362in}}%
\pgfpathlineto{\pgfqpoint{1.162125in}{0.544114in}}%
\pgfpathlineto{\pgfqpoint{1.163199in}{0.545800in}}%
\pgfpathlineto{\pgfqpoint{1.166421in}{0.546527in}}%
\pgfpathlineto{\pgfqpoint{1.167494in}{0.545767in}}%
\pgfpathlineto{\pgfqpoint{1.168568in}{0.548180in}}%
\pgfpathlineto{\pgfqpoint{1.169642in}{0.548444in}}%
\pgfpathlineto{\pgfqpoint{1.170716in}{0.549866in}}%
\pgfpathlineto{\pgfqpoint{1.173937in}{0.548279in}}%
\pgfpathlineto{\pgfqpoint{1.175011in}{0.546329in}}%
\pgfpathlineto{\pgfqpoint{1.176085in}{0.546891in}}%
\pgfpathlineto{\pgfqpoint{1.177159in}{0.549568in}}%
\pgfpathlineto{\pgfqpoint{1.178233in}{0.549965in}}%
\pgfpathlineto{\pgfqpoint{1.181454in}{0.549832in}}%
\pgfpathlineto{\pgfqpoint{1.182528in}{0.551056in}}%
\pgfpathlineto{\pgfqpoint{1.183602in}{0.551419in}}%
\pgfpathlineto{\pgfqpoint{1.185750in}{0.550989in}}%
\pgfpathlineto{\pgfqpoint{1.188971in}{0.552279in}}%
\pgfpathlineto{\pgfqpoint{1.190045in}{0.549700in}}%
\pgfpathlineto{\pgfqpoint{1.191119in}{0.550461in}}%
\pgfpathlineto{\pgfqpoint{1.192193in}{0.549667in}}%
\pgfpathlineto{\pgfqpoint{1.196488in}{0.549568in}}%
\pgfpathlineto{\pgfqpoint{1.197562in}{0.547915in}}%
\pgfpathlineto{\pgfqpoint{1.198636in}{0.552708in}}%
\pgfpathlineto{\pgfqpoint{1.199710in}{0.550989in}}%
\pgfpathlineto{\pgfqpoint{1.200783in}{0.553997in}}%
\pgfpathlineto{\pgfqpoint{1.204005in}{0.554295in}}%
\pgfpathlineto{\pgfqpoint{1.205079in}{0.558559in}}%
\pgfpathlineto{\pgfqpoint{1.206153in}{0.560146in}}%
\pgfpathlineto{\pgfqpoint{1.207226in}{0.560708in}}%
\pgfpathlineto{\pgfqpoint{1.208300in}{0.560642in}}%
\pgfpathlineto{\pgfqpoint{1.214743in}{0.563881in}}%
\pgfpathlineto{\pgfqpoint{1.215817in}{0.563484in}}%
\pgfpathlineto{\pgfqpoint{1.219039in}{0.564443in}}%
\pgfpathlineto{\pgfqpoint{1.220112in}{0.565798in}}%
\pgfpathlineto{\pgfqpoint{1.222260in}{0.564509in}}%
\pgfpathlineto{\pgfqpoint{1.223334in}{0.564641in}}%
\pgfpathlineto{\pgfqpoint{1.226555in}{0.563617in}}%
\pgfpathlineto{\pgfqpoint{1.228703in}{0.565435in}}%
\pgfpathlineto{\pgfqpoint{1.230851in}{0.564608in}}%
\pgfpathlineto{\pgfqpoint{1.234072in}{0.562889in}}%
\pgfpathlineto{\pgfqpoint{1.235146in}{0.565798in}}%
\pgfpathlineto{\pgfqpoint{1.236220in}{0.566625in}}%
\pgfpathlineto{\pgfqpoint{1.237294in}{0.565104in}}%
\pgfpathlineto{\pgfqpoint{1.238368in}{0.573136in}}%
\pgfpathlineto{\pgfqpoint{1.242663in}{0.572938in}}%
\pgfpathlineto{\pgfqpoint{1.243737in}{0.573731in}}%
\pgfpathlineto{\pgfqpoint{1.245885in}{0.564707in}}%
\pgfpathlineto{\pgfqpoint{1.249106in}{0.560840in}}%
\pgfpathlineto{\pgfqpoint{1.250180in}{0.564476in}}%
\pgfpathlineto{\pgfqpoint{1.251254in}{0.561534in}}%
\pgfpathlineto{\pgfqpoint{1.252328in}{0.564410in}}%
\pgfpathlineto{\pgfqpoint{1.253401in}{0.560212in}}%
\pgfpathlineto{\pgfqpoint{1.256623in}{0.558724in}}%
\pgfpathlineto{\pgfqpoint{1.258771in}{0.560377in}}%
\pgfpathlineto{\pgfqpoint{1.260918in}{0.565104in}}%
\pgfpathlineto{\pgfqpoint{1.264140in}{0.564178in}}%
\pgfpathlineto{\pgfqpoint{1.265214in}{0.565534in}}%
\pgfpathlineto{\pgfqpoint{1.266287in}{0.568211in}}%
\pgfpathlineto{\pgfqpoint{1.267361in}{0.568112in}}%
\pgfpathlineto{\pgfqpoint{1.268435in}{0.569633in}}%
\pgfpathlineto{\pgfqpoint{1.272731in}{0.569666in}}%
\pgfpathlineto{\pgfqpoint{1.273804in}{0.567980in}}%
\pgfpathlineto{\pgfqpoint{1.275952in}{0.567583in}}%
\pgfpathlineto{\pgfqpoint{1.280247in}{0.570492in}}%
\pgfpathlineto{\pgfqpoint{1.281321in}{0.569699in}}%
\pgfpathlineto{\pgfqpoint{1.283469in}{0.569566in}}%
\pgfpathlineto{\pgfqpoint{1.286690in}{0.566096in}}%
\pgfpathlineto{\pgfqpoint{1.287764in}{0.569236in}}%
\pgfpathlineto{\pgfqpoint{1.288838in}{0.567153in}}%
\pgfpathlineto{\pgfqpoint{1.290986in}{0.569269in}}%
\pgfpathlineto{\pgfqpoint{1.294207in}{0.569236in}}%
\pgfpathlineto{\pgfqpoint{1.295281in}{0.570525in}}%
\pgfpathlineto{\pgfqpoint{1.296355in}{0.569699in}}%
\pgfpathlineto{\pgfqpoint{1.297429in}{0.565567in}}%
\pgfpathlineto{\pgfqpoint{1.298503in}{0.565567in}}%
\pgfpathlineto{\pgfqpoint{1.301724in}{0.567914in}}%
\pgfpathlineto{\pgfqpoint{1.302798in}{0.569930in}}%
\pgfpathlineto{\pgfqpoint{1.304946in}{0.566393in}}%
\pgfpathlineto{\pgfqpoint{1.306020in}{0.567583in}}%
\pgfpathlineto{\pgfqpoint{1.309241in}{0.565567in}}%
\pgfpathlineto{\pgfqpoint{1.311389in}{0.561633in}}%
\pgfpathlineto{\pgfqpoint{1.312463in}{0.561699in}}%
\pgfpathlineto{\pgfqpoint{1.313536in}{0.558824in}}%
\pgfpathlineto{\pgfqpoint{1.316758in}{0.561765in}}%
\pgfpathlineto{\pgfqpoint{1.317832in}{0.560873in}}%
\pgfpathlineto{\pgfqpoint{1.319979in}{0.561104in}}%
\pgfpathlineto{\pgfqpoint{1.321053in}{0.555419in}}%
\pgfpathlineto{\pgfqpoint{1.325349in}{0.551452in}}%
\pgfpathlineto{\pgfqpoint{1.326422in}{0.555320in}}%
\pgfpathlineto{\pgfqpoint{1.328570in}{0.546858in}}%
\pgfpathlineto{\pgfqpoint{1.331792in}{0.550229in}}%
\pgfpathlineto{\pgfqpoint{1.332865in}{0.552609in}}%
\pgfpathlineto{\pgfqpoint{1.333939in}{0.556708in}}%
\pgfpathlineto{\pgfqpoint{1.335013in}{0.555617in}}%
\pgfpathlineto{\pgfqpoint{1.340382in}{0.557204in}}%
\pgfpathlineto{\pgfqpoint{1.341456in}{0.556311in}}%
\pgfpathlineto{\pgfqpoint{1.342530in}{0.556741in}}%
\pgfpathlineto{\pgfqpoint{1.343604in}{0.548643in}}%
\pgfpathlineto{\pgfqpoint{1.348973in}{0.551485in}}%
\pgfpathlineto{\pgfqpoint{1.350047in}{0.554196in}}%
\pgfpathlineto{\pgfqpoint{1.351121in}{0.552874in}}%
\pgfpathlineto{\pgfqpoint{1.354342in}{0.554989in}}%
\pgfpathlineto{\pgfqpoint{1.355416in}{0.553601in}}%
\pgfpathlineto{\pgfqpoint{1.357564in}{0.557898in}}%
\pgfpathlineto{\pgfqpoint{1.358638in}{0.557832in}}%
\pgfpathlineto{\pgfqpoint{1.362933in}{0.558824in}}%
\pgfpathlineto{\pgfqpoint{1.364007in}{0.558328in}}%
\pgfpathlineto{\pgfqpoint{1.365081in}{0.556444in}}%
\pgfpathlineto{\pgfqpoint{1.366154in}{0.558295in}}%
\pgfpathlineto{\pgfqpoint{1.369376in}{0.558724in}}%
\pgfpathlineto{\pgfqpoint{1.370450in}{0.556840in}}%
\pgfpathlineto{\pgfqpoint{1.371524in}{0.558526in}}%
\pgfpathlineto{\pgfqpoint{1.372598in}{0.557964in}}%
\pgfpathlineto{\pgfqpoint{1.373671in}{0.560080in}}%
\pgfpathlineto{\pgfqpoint{1.377967in}{0.561964in}}%
\pgfpathlineto{\pgfqpoint{1.379041in}{0.561435in}}%
\pgfpathlineto{\pgfqpoint{1.381188in}{0.562228in}}%
\pgfpathlineto{\pgfqpoint{1.384410in}{0.561203in}}%
\pgfpathlineto{\pgfqpoint{1.385484in}{0.559485in}}%
\pgfpathlineto{\pgfqpoint{1.387631in}{0.560179in}}%
\pgfpathlineto{\pgfqpoint{1.388705in}{0.560807in}}%
\pgfpathlineto{\pgfqpoint{1.391927in}{0.560476in}}%
\pgfpathlineto{\pgfqpoint{1.393000in}{0.561765in}}%
\pgfpathlineto{\pgfqpoint{1.395148in}{0.559848in}}%
\pgfpathlineto{\pgfqpoint{1.399443in}{0.558625in}}%
\pgfpathlineto{\pgfqpoint{1.401591in}{0.559419in}}%
\pgfpathlineto{\pgfqpoint{1.403739in}{0.558063in}}%
\pgfpathlineto{\pgfqpoint{1.406960in}{0.558030in}}%
\pgfpathlineto{\pgfqpoint{1.408034in}{0.556741in}}%
\pgfpathlineto{\pgfqpoint{1.409108in}{0.557733in}}%
\pgfpathlineto{\pgfqpoint{1.411256in}{0.557898in}}%
\pgfpathlineto{\pgfqpoint{1.414477in}{0.559022in}}%
\pgfpathlineto{\pgfqpoint{1.415551in}{0.561765in}}%
\pgfpathlineto{\pgfqpoint{1.416625in}{0.562228in}}%
\pgfpathlineto{\pgfqpoint{1.417699in}{0.563550in}}%
\pgfpathlineto{\pgfqpoint{1.421994in}{0.563716in}}%
\pgfpathlineto{\pgfqpoint{1.423068in}{0.562559in}}%
\pgfpathlineto{\pgfqpoint{1.424142in}{0.563253in}}%
\pgfpathlineto{\pgfqpoint{1.425216in}{0.562790in}}%
\pgfpathlineto{\pgfqpoint{1.430585in}{0.567649in}}%
\pgfpathlineto{\pgfqpoint{1.431659in}{0.568343in}}%
\pgfpathlineto{\pgfqpoint{1.432732in}{0.564608in}}%
\pgfpathlineto{\pgfqpoint{1.433806in}{0.566426in}}%
\pgfpathlineto{\pgfqpoint{1.437028in}{0.565666in}}%
\pgfpathlineto{\pgfqpoint{1.438102in}{0.567253in}}%
\pgfpathlineto{\pgfqpoint{1.439175in}{0.567220in}}%
\pgfpathlineto{\pgfqpoint{1.440249in}{0.568410in}}%
\pgfpathlineto{\pgfqpoint{1.441323in}{0.562195in}}%
\pgfpathlineto{\pgfqpoint{1.446692in}{0.561633in}}%
\pgfpathlineto{\pgfqpoint{1.447766in}{0.559253in}}%
\pgfpathlineto{\pgfqpoint{1.448840in}{0.559881in}}%
\pgfpathlineto{\pgfqpoint{1.452062in}{0.560146in}}%
\pgfpathlineto{\pgfqpoint{1.453135in}{0.558890in}}%
\pgfpathlineto{\pgfqpoint{1.454209in}{0.558989in}}%
\pgfpathlineto{\pgfqpoint{1.455283in}{0.557700in}}%
\pgfpathlineto{\pgfqpoint{1.456357in}{0.558824in}}%
\pgfpathlineto{\pgfqpoint{1.460652in}{0.558691in}}%
\pgfpathlineto{\pgfqpoint{1.461726in}{0.560840in}}%
\pgfpathlineto{\pgfqpoint{1.462800in}{0.561699in}}%
\pgfpathlineto{\pgfqpoint{1.463874in}{0.559848in}}%
\pgfpathlineto{\pgfqpoint{1.469243in}{0.564608in}}%
\pgfpathlineto{\pgfqpoint{1.470317in}{0.564245in}}%
\pgfpathlineto{\pgfqpoint{1.471391in}{0.564509in}}%
\pgfpathlineto{\pgfqpoint{1.474612in}{0.564443in}}%
\pgfpathlineto{\pgfqpoint{1.476760in}{0.565335in}}%
\pgfpathlineto{\pgfqpoint{1.478908in}{0.561699in}}%
\pgfpathlineto{\pgfqpoint{1.484277in}{0.563583in}}%
\pgfpathlineto{\pgfqpoint{1.486424in}{0.563022in}}%
\pgfpathlineto{\pgfqpoint{1.489646in}{0.564245in}}%
\pgfpathlineto{\pgfqpoint{1.490720in}{0.563088in}}%
\pgfpathlineto{\pgfqpoint{1.491794in}{0.565137in}}%
\pgfpathlineto{\pgfqpoint{1.493941in}{0.562889in}}%
\pgfpathlineto{\pgfqpoint{1.497163in}{0.563352in}}%
\pgfpathlineto{\pgfqpoint{1.498237in}{0.565468in}}%
\pgfpathlineto{\pgfqpoint{1.499310in}{0.564079in}}%
\pgfpathlineto{\pgfqpoint{1.500384in}{0.564773in}}%
\pgfpathlineto{\pgfqpoint{1.501458in}{0.564641in}}%
\pgfpathlineto{\pgfqpoint{1.505753in}{0.561633in}}%
\pgfpathlineto{\pgfqpoint{1.506827in}{0.563187in}}%
\pgfpathlineto{\pgfqpoint{1.507901in}{0.560278in}}%
\pgfpathlineto{\pgfqpoint{1.508975in}{0.561237in}}%
\pgfpathlineto{\pgfqpoint{1.512197in}{0.560443in}}%
\pgfpathlineto{\pgfqpoint{1.513270in}{0.562360in}}%
\pgfpathlineto{\pgfqpoint{1.514344in}{0.559716in}}%
\pgfpathlineto{\pgfqpoint{1.515418in}{0.559485in}}%
\pgfpathlineto{\pgfqpoint{1.516492in}{0.561270in}}%
\pgfpathlineto{\pgfqpoint{1.519713in}{0.561104in}}%
\pgfpathlineto{\pgfqpoint{1.520787in}{0.558162in}}%
\pgfpathlineto{\pgfqpoint{1.521861in}{0.561567in}}%
\pgfpathlineto{\pgfqpoint{1.524009in}{0.555749in}}%
\pgfpathlineto{\pgfqpoint{1.527230in}{0.555220in}}%
\pgfpathlineto{\pgfqpoint{1.529378in}{0.552047in}}%
\pgfpathlineto{\pgfqpoint{1.531526in}{0.556576in}}%
\pgfpathlineto{\pgfqpoint{1.534747in}{0.558030in}}%
\pgfpathlineto{\pgfqpoint{1.535821in}{0.562327in}}%
\pgfpathlineto{\pgfqpoint{1.536895in}{0.560443in}}%
\pgfpathlineto{\pgfqpoint{1.537969in}{0.563088in}}%
\pgfpathlineto{\pgfqpoint{1.539042in}{0.562460in}}%
\pgfpathlineto{\pgfqpoint{1.542264in}{0.562393in}}%
\pgfpathlineto{\pgfqpoint{1.543338in}{0.565005in}}%
\pgfpathlineto{\pgfqpoint{1.544412in}{0.563385in}}%
\pgfpathlineto{\pgfqpoint{1.545486in}{0.580706in}}%
\pgfpathlineto{\pgfqpoint{1.546559in}{0.584474in}}%
\pgfpathlineto{\pgfqpoint{1.549781in}{0.584540in}}%
\pgfpathlineto{\pgfqpoint{1.550855in}{0.585664in}}%
\pgfpathlineto{\pgfqpoint{1.551929in}{0.590854in}}%
\pgfpathlineto{\pgfqpoint{1.553002in}{0.591284in}}%
\pgfpathlineto{\pgfqpoint{1.554076in}{0.593135in}}%
\pgfpathlineto{\pgfqpoint{1.558372in}{0.590986in}}%
\pgfpathlineto{\pgfqpoint{1.559445in}{0.594292in}}%
\pgfpathlineto{\pgfqpoint{1.560519in}{0.593465in}}%
\pgfpathlineto{\pgfqpoint{1.561593in}{0.591813in}}%
\pgfpathlineto{\pgfqpoint{1.566962in}{0.592639in}}%
\pgfpathlineto{\pgfqpoint{1.569110in}{0.596077in}}%
\pgfpathlineto{\pgfqpoint{1.572331in}{0.596407in}}%
\pgfpathlineto{\pgfqpoint{1.573405in}{0.598490in}}%
\pgfpathlineto{\pgfqpoint{1.574479in}{0.598490in}}%
\pgfpathlineto{\pgfqpoint{1.576627in}{0.599217in}}%
\pgfpathlineto{\pgfqpoint{1.579848in}{0.599217in}}%
\pgfpathlineto{\pgfqpoint{1.581996in}{0.601928in}}%
\pgfpathlineto{\pgfqpoint{1.583070in}{0.601564in}}%
\pgfpathlineto{\pgfqpoint{1.584144in}{0.603316in}}%
\pgfpathlineto{\pgfqpoint{1.587365in}{0.603118in}}%
\pgfpathlineto{\pgfqpoint{1.588439in}{0.603944in}}%
\pgfpathlineto{\pgfqpoint{1.589513in}{0.602027in}}%
\pgfpathlineto{\pgfqpoint{1.590587in}{0.603151in}}%
\pgfpathlineto{\pgfqpoint{1.591661in}{0.598126in}}%
\pgfpathlineto{\pgfqpoint{1.594882in}{0.598060in}}%
\pgfpathlineto{\pgfqpoint{1.595956in}{0.595449in}}%
\pgfpathlineto{\pgfqpoint{1.598104in}{0.603977in}}%
\pgfpathlineto{\pgfqpoint{1.599177in}{0.601994in}}%
\pgfpathlineto{\pgfqpoint{1.603473in}{0.604836in}}%
\pgfpathlineto{\pgfqpoint{1.604547in}{0.606721in}}%
\pgfpathlineto{\pgfqpoint{1.610990in}{0.604274in}}%
\pgfpathlineto{\pgfqpoint{1.612063in}{0.602423in}}%
\pgfpathlineto{\pgfqpoint{1.614211in}{0.604638in}}%
\pgfpathlineto{\pgfqpoint{1.618507in}{0.598688in}}%
\pgfpathlineto{\pgfqpoint{1.620654in}{0.604208in}}%
\pgfpathlineto{\pgfqpoint{1.621728in}{0.601101in}}%
\pgfpathlineto{\pgfqpoint{1.624950in}{0.600671in}}%
\pgfpathlineto{\pgfqpoint{1.626023in}{0.601300in}}%
\pgfpathlineto{\pgfqpoint{1.627097in}{0.597135in}}%
\pgfpathlineto{\pgfqpoint{1.628171in}{0.595217in}}%
\pgfpathlineto{\pgfqpoint{1.629245in}{0.596672in}}%
\pgfpathlineto{\pgfqpoint{1.636762in}{0.599316in}}%
\pgfpathlineto{\pgfqpoint{1.639983in}{0.597928in}}%
\pgfpathlineto{\pgfqpoint{1.642131in}{0.589829in}}%
\pgfpathlineto{\pgfqpoint{1.643205in}{0.591152in}}%
\pgfpathlineto{\pgfqpoint{1.644279in}{0.596639in}}%
\pgfpathlineto{\pgfqpoint{1.647500in}{0.596969in}}%
\pgfpathlineto{\pgfqpoint{1.649648in}{0.604539in}}%
\pgfpathlineto{\pgfqpoint{1.650722in}{0.610026in}}%
\pgfpathlineto{\pgfqpoint{1.651796in}{0.606555in}}%
\pgfpathlineto{\pgfqpoint{1.656091in}{0.604274in}}%
\pgfpathlineto{\pgfqpoint{1.658239in}{0.610886in}}%
\pgfpathlineto{\pgfqpoint{1.659312in}{0.609861in}}%
\pgfpathlineto{\pgfqpoint{1.663608in}{0.610852in}}%
\pgfpathlineto{\pgfqpoint{1.664682in}{0.609464in}}%
\pgfpathlineto{\pgfqpoint{1.665755in}{0.609464in}}%
\pgfpathlineto{\pgfqpoint{1.666829in}{0.612571in}}%
\pgfpathlineto{\pgfqpoint{1.672198in}{0.612571in}}%
\pgfpathlineto{\pgfqpoint{1.673272in}{0.613166in}}%
\pgfpathlineto{\pgfqpoint{1.674346in}{0.611216in}}%
\pgfpathlineto{\pgfqpoint{1.677568in}{0.616802in}}%
\pgfpathlineto{\pgfqpoint{1.679715in}{0.613166in}}%
\pgfpathlineto{\pgfqpoint{1.680789in}{0.613464in}}%
\pgfpathlineto{\pgfqpoint{1.681863in}{0.609629in}}%
\pgfpathlineto{\pgfqpoint{1.685085in}{0.611315in}}%
\pgfpathlineto{\pgfqpoint{1.686158in}{0.606324in}}%
\pgfpathlineto{\pgfqpoint{1.687232in}{0.605960in}}%
\pgfpathlineto{\pgfqpoint{1.688306in}{0.609828in}}%
\pgfpathlineto{\pgfqpoint{1.689380in}{0.606192in}}%
\pgfpathlineto{\pgfqpoint{1.692601in}{0.609398in}}%
\pgfpathlineto{\pgfqpoint{1.693675in}{0.605762in}}%
\pgfpathlineto{\pgfqpoint{1.694749in}{0.608307in}}%
\pgfpathlineto{\pgfqpoint{1.695823in}{0.607977in}}%
\pgfpathlineto{\pgfqpoint{1.696897in}{0.609894in}}%
\pgfpathlineto{\pgfqpoint{1.701192in}{0.608902in}}%
\pgfpathlineto{\pgfqpoint{1.702266in}{0.604506in}}%
\pgfpathlineto{\pgfqpoint{1.704414in}{0.603911in}}%
\pgfpathlineto{\pgfqpoint{1.707635in}{0.604274in}}%
\pgfpathlineto{\pgfqpoint{1.709783in}{0.602754in}}%
\pgfpathlineto{\pgfqpoint{1.710857in}{0.603118in}}%
\pgfpathlineto{\pgfqpoint{1.715152in}{0.602820in}}%
\pgfpathlineto{\pgfqpoint{1.717300in}{0.606985in}}%
\pgfpathlineto{\pgfqpoint{1.719447in}{0.606456in}}%
\pgfpathlineto{\pgfqpoint{1.723743in}{0.603878in}}%
\pgfpathlineto{\pgfqpoint{1.725890in}{0.604241in}}%
\pgfpathlineto{\pgfqpoint{1.726964in}{0.600638in}}%
\pgfpathlineto{\pgfqpoint{1.730186in}{0.601266in}}%
\pgfpathlineto{\pgfqpoint{1.731260in}{0.603283in}}%
\pgfpathlineto{\pgfqpoint{1.732333in}{0.611811in}}%
\pgfpathlineto{\pgfqpoint{1.734481in}{0.610092in}}%
\pgfpathlineto{\pgfqpoint{1.737703in}{0.608902in}}%
\pgfpathlineto{\pgfqpoint{1.738776in}{0.607844in}}%
\pgfpathlineto{\pgfqpoint{1.739850in}{0.609662in}}%
\pgfpathlineto{\pgfqpoint{1.740924in}{0.605531in}}%
\pgfpathlineto{\pgfqpoint{1.741998in}{0.604638in}}%
\pgfpathlineto{\pgfqpoint{1.745219in}{0.604043in}}%
\pgfpathlineto{\pgfqpoint{1.746293in}{0.605233in}}%
\pgfpathlineto{\pgfqpoint{1.747367in}{0.604308in}}%
\pgfpathlineto{\pgfqpoint{1.748441in}{0.607216in}}%
\pgfpathlineto{\pgfqpoint{1.749515in}{0.616472in}}%
\pgfpathlineto{\pgfqpoint{1.754884in}{0.614224in}}%
\pgfpathlineto{\pgfqpoint{1.755958in}{0.618554in}}%
\pgfpathlineto{\pgfqpoint{1.757032in}{0.617166in}}%
\pgfpathlineto{\pgfqpoint{1.761327in}{0.619083in}}%
\pgfpathlineto{\pgfqpoint{1.763475in}{0.616538in}}%
\pgfpathlineto{\pgfqpoint{1.764549in}{0.617331in}}%
\pgfpathlineto{\pgfqpoint{1.768844in}{0.613894in}}%
\pgfpathlineto{\pgfqpoint{1.769918in}{0.616935in}}%
\pgfpathlineto{\pgfqpoint{1.770992in}{0.617133in}}%
\pgfpathlineto{\pgfqpoint{1.772065in}{0.614323in}}%
\pgfpathlineto{\pgfqpoint{1.775287in}{0.615712in}}%
\pgfpathlineto{\pgfqpoint{1.777435in}{0.615183in}}%
\pgfpathlineto{\pgfqpoint{1.778508in}{0.612836in}}%
\pgfpathlineto{\pgfqpoint{1.779582in}{0.613332in}}%
\pgfpathlineto{\pgfqpoint{1.782804in}{0.611117in}}%
\pgfpathlineto{\pgfqpoint{1.783878in}{0.611976in}}%
\pgfpathlineto{\pgfqpoint{1.784951in}{0.617364in}}%
\pgfpathlineto{\pgfqpoint{1.786025in}{0.617397in}}%
\pgfpathlineto{\pgfqpoint{1.790321in}{0.613993in}}%
\pgfpathlineto{\pgfqpoint{1.791395in}{0.615381in}}%
\pgfpathlineto{\pgfqpoint{1.792468in}{0.614555in}}%
\pgfpathlineto{\pgfqpoint{1.793542in}{0.616968in}}%
\pgfpathlineto{\pgfqpoint{1.794616in}{0.614356in}}%
\pgfpathlineto{\pgfqpoint{1.797838in}{0.615645in}}%
\pgfpathlineto{\pgfqpoint{1.798911in}{0.616703in}}%
\pgfpathlineto{\pgfqpoint{1.801059in}{0.614125in}}%
\pgfpathlineto{\pgfqpoint{1.802133in}{0.614555in}}%
\pgfpathlineto{\pgfqpoint{1.805354in}{0.608043in}}%
\pgfpathlineto{\pgfqpoint{1.808576in}{0.612935in}}%
\pgfpathlineto{\pgfqpoint{1.812871in}{0.612538in}}%
\pgfpathlineto{\pgfqpoint{1.813945in}{0.611414in}}%
\pgfpathlineto{\pgfqpoint{1.815019in}{0.608076in}}%
\pgfpathlineto{\pgfqpoint{1.816093in}{0.609101in}}%
\pgfpathlineto{\pgfqpoint{1.817167in}{0.613497in}}%
\pgfpathlineto{\pgfqpoint{1.824684in}{0.621364in}}%
\pgfpathlineto{\pgfqpoint{1.827905in}{0.627215in}}%
\pgfpathlineto{\pgfqpoint{1.828979in}{0.625033in}}%
\pgfpathlineto{\pgfqpoint{1.831127in}{0.624174in}}%
\pgfpathlineto{\pgfqpoint{1.832200in}{0.633925in}}%
\pgfpathlineto{\pgfqpoint{1.835422in}{0.630917in}}%
\pgfpathlineto{\pgfqpoint{1.838643in}{0.638982in}}%
\pgfpathlineto{\pgfqpoint{1.839717in}{0.635677in}}%
\pgfpathlineto{\pgfqpoint{1.842939in}{0.637032in}}%
\pgfpathlineto{\pgfqpoint{1.845086in}{0.634520in}}%
\pgfpathlineto{\pgfqpoint{1.846160in}{0.630025in}}%
\pgfpathlineto{\pgfqpoint{1.847234in}{0.632041in}}%
\pgfpathlineto{\pgfqpoint{1.850456in}{0.632603in}}%
\pgfpathlineto{\pgfqpoint{1.851529in}{0.629330in}}%
\pgfpathlineto{\pgfqpoint{1.854751in}{0.632471in}}%
\pgfpathlineto{\pgfqpoint{1.860120in}{0.633033in}}%
\pgfpathlineto{\pgfqpoint{1.861194in}{0.631611in}}%
\pgfpathlineto{\pgfqpoint{1.862268in}{0.622752in}}%
\pgfpathlineto{\pgfqpoint{1.866563in}{0.609200in}}%
\pgfpathlineto{\pgfqpoint{1.867637in}{0.621133in}}%
\pgfpathlineto{\pgfqpoint{1.868711in}{0.626620in}}%
\pgfpathlineto{\pgfqpoint{1.869785in}{0.626818in}}%
\pgfpathlineto{\pgfqpoint{1.873006in}{0.623083in}}%
\pgfpathlineto{\pgfqpoint{1.874080in}{0.615579in}}%
\pgfpathlineto{\pgfqpoint{1.876228in}{0.620207in}}%
\pgfpathlineto{\pgfqpoint{1.877302in}{0.616240in}}%
\pgfpathlineto{\pgfqpoint{1.881597in}{0.620637in}}%
\pgfpathlineto{\pgfqpoint{1.882671in}{0.617629in}}%
\pgfpathlineto{\pgfqpoint{1.884818in}{0.621364in}}%
\pgfpathlineto{\pgfqpoint{1.888040in}{0.618951in}}%
\pgfpathlineto{\pgfqpoint{1.890188in}{0.622091in}}%
\pgfpathlineto{\pgfqpoint{1.891262in}{0.621992in}}%
\pgfpathlineto{\pgfqpoint{1.892335in}{0.618257in}}%
\pgfpathlineto{\pgfqpoint{1.895557in}{0.621430in}}%
\pgfpathlineto{\pgfqpoint{1.896631in}{0.619711in}}%
\pgfpathlineto{\pgfqpoint{1.897705in}{0.621959in}}%
\pgfpathlineto{\pgfqpoint{1.898778in}{0.619678in}}%
\pgfpathlineto{\pgfqpoint{1.899852in}{0.621133in}}%
\pgfpathlineto{\pgfqpoint{1.903074in}{0.609927in}}%
\pgfpathlineto{\pgfqpoint{1.905221in}{0.617827in}}%
\pgfpathlineto{\pgfqpoint{1.906295in}{0.618885in}}%
\pgfpathlineto{\pgfqpoint{1.907369in}{0.621067in}}%
\pgfpathlineto{\pgfqpoint{1.910591in}{0.626091in}}%
\pgfpathlineto{\pgfqpoint{1.911664in}{0.625595in}}%
\pgfpathlineto{\pgfqpoint{1.913812in}{0.631446in}}%
\pgfpathlineto{\pgfqpoint{1.914886in}{0.631710in}}%
\pgfpathlineto{\pgfqpoint{1.918107in}{0.634950in}}%
\pgfpathlineto{\pgfqpoint{1.919181in}{0.634983in}}%
\pgfpathlineto{\pgfqpoint{1.920255in}{0.632404in}}%
\pgfpathlineto{\pgfqpoint{1.922403in}{0.638189in}}%
\pgfpathlineto{\pgfqpoint{1.925624in}{0.641362in}}%
\pgfpathlineto{\pgfqpoint{1.927772in}{0.636437in}}%
\pgfpathlineto{\pgfqpoint{1.929920in}{0.641594in}}%
\pgfpathlineto{\pgfqpoint{1.933141in}{0.645164in}}%
\pgfpathlineto{\pgfqpoint{1.934215in}{0.643048in}}%
\pgfpathlineto{\pgfqpoint{1.935289in}{0.647379in}}%
\pgfpathlineto{\pgfqpoint{1.936363in}{0.646222in}}%
\pgfpathlineto{\pgfqpoint{1.940658in}{0.635677in}}%
\pgfpathlineto{\pgfqpoint{1.941732in}{0.644271in}}%
\pgfpathlineto{\pgfqpoint{1.942806in}{0.645726in}}%
\pgfpathlineto{\pgfqpoint{1.943880in}{0.648635in}}%
\pgfpathlineto{\pgfqpoint{1.944953in}{0.646982in}}%
\pgfpathlineto{\pgfqpoint{1.948175in}{0.644734in}}%
\pgfpathlineto{\pgfqpoint{1.949249in}{0.649792in}}%
\pgfpathlineto{\pgfqpoint{1.950323in}{0.648833in}}%
\pgfpathlineto{\pgfqpoint{1.951396in}{0.646023in}}%
\pgfpathlineto{\pgfqpoint{1.952470in}{0.645395in}}%
\pgfpathlineto{\pgfqpoint{1.955692in}{0.647940in}}%
\pgfpathlineto{\pgfqpoint{1.956766in}{0.647709in}}%
\pgfpathlineto{\pgfqpoint{1.957840in}{0.652932in}}%
\pgfpathlineto{\pgfqpoint{1.958913in}{0.651940in}}%
\pgfpathlineto{\pgfqpoint{1.959987in}{0.652072in}}%
\pgfpathlineto{\pgfqpoint{1.963209in}{0.651808in}}%
\pgfpathlineto{\pgfqpoint{1.965356in}{0.650056in}}%
\pgfpathlineto{\pgfqpoint{1.967504in}{0.650948in}}%
\pgfpathlineto{\pgfqpoint{1.970726in}{0.648271in}}%
\pgfpathlineto{\pgfqpoint{1.971799in}{0.651180in}}%
\pgfpathlineto{\pgfqpoint{1.973947in}{0.645957in}}%
\pgfpathlineto{\pgfqpoint{1.975021in}{0.652767in}}%
\pgfpathlineto{\pgfqpoint{1.979316in}{0.648469in}}%
\pgfpathlineto{\pgfqpoint{1.980390in}{0.644668in}}%
\pgfpathlineto{\pgfqpoint{1.981464in}{0.645428in}}%
\pgfpathlineto{\pgfqpoint{1.982538in}{0.638982in}}%
\pgfpathlineto{\pgfqpoint{1.985759in}{0.641396in}}%
\pgfpathlineto{\pgfqpoint{1.987907in}{0.651015in}}%
\pgfpathlineto{\pgfqpoint{1.988981in}{0.647246in}}%
\pgfpathlineto{\pgfqpoint{1.990055in}{0.639644in}}%
\pgfpathlineto{\pgfqpoint{1.994350in}{0.643048in}}%
\pgfpathlineto{\pgfqpoint{1.995424in}{0.646850in}}%
\pgfpathlineto{\pgfqpoint{1.996498in}{0.645891in}}%
\pgfpathlineto{\pgfqpoint{2.000793in}{0.646750in}}%
\pgfpathlineto{\pgfqpoint{2.001867in}{0.648932in}}%
\pgfpathlineto{\pgfqpoint{2.004015in}{0.643610in}}%
\pgfpathlineto{\pgfqpoint{2.008310in}{0.637660in}}%
\pgfpathlineto{\pgfqpoint{2.009384in}{0.639478in}}%
\pgfpathlineto{\pgfqpoint{2.012605in}{0.628570in}}%
\pgfpathlineto{\pgfqpoint{2.015827in}{0.631942in}}%
\pgfpathlineto{\pgfqpoint{2.016901in}{0.634619in}}%
\pgfpathlineto{\pgfqpoint{2.017974in}{0.629297in}}%
\pgfpathlineto{\pgfqpoint{2.019048in}{0.631545in}}%
\pgfpathlineto{\pgfqpoint{2.020122in}{0.625198in}}%
\pgfpathlineto{\pgfqpoint{2.024417in}{0.623810in}}%
\pgfpathlineto{\pgfqpoint{2.025491in}{0.621529in}}%
\pgfpathlineto{\pgfqpoint{2.027639in}{0.628008in}}%
\pgfpathlineto{\pgfqpoint{2.030861in}{0.624934in}}%
\pgfpathlineto{\pgfqpoint{2.031934in}{0.625364in}}%
\pgfpathlineto{\pgfqpoint{2.033008in}{0.622257in}}%
\pgfpathlineto{\pgfqpoint{2.034082in}{0.617166in}}%
\pgfpathlineto{\pgfqpoint{2.035156in}{0.633760in}}%
\pgfpathlineto{\pgfqpoint{2.038377in}{0.633396in}}%
\pgfpathlineto{\pgfqpoint{2.039451in}{0.630256in}}%
\pgfpathlineto{\pgfqpoint{2.040525in}{0.633396in}}%
\pgfpathlineto{\pgfqpoint{2.041599in}{0.631148in}}%
\pgfpathlineto{\pgfqpoint{2.042673in}{0.624273in}}%
\pgfpathlineto{\pgfqpoint{2.045894in}{0.612175in}}%
\pgfpathlineto{\pgfqpoint{2.046968in}{0.613960in}}%
\pgfpathlineto{\pgfqpoint{2.048042in}{0.620273in}}%
\pgfpathlineto{\pgfqpoint{2.049116in}{0.614852in}}%
\pgfpathlineto{\pgfqpoint{2.050190in}{0.621133in}}%
\pgfpathlineto{\pgfqpoint{2.054485in}{0.623347in}}%
\pgfpathlineto{\pgfqpoint{2.055559in}{0.626785in}}%
\pgfpathlineto{\pgfqpoint{2.056633in}{0.624240in}}%
\pgfpathlineto{\pgfqpoint{2.057706in}{0.625165in}}%
\pgfpathlineto{\pgfqpoint{2.060928in}{0.630124in}}%
\pgfpathlineto{\pgfqpoint{2.062002in}{0.627182in}}%
\pgfpathlineto{\pgfqpoint{2.063076in}{0.626190in}}%
\pgfpathlineto{\pgfqpoint{2.064150in}{0.630785in}}%
\pgfpathlineto{\pgfqpoint{2.065223in}{0.629033in}}%
\pgfpathlineto{\pgfqpoint{2.068445in}{0.627942in}}%
\pgfpathlineto{\pgfqpoint{2.069519in}{0.635214in}}%
\pgfpathlineto{\pgfqpoint{2.071666in}{0.632867in}}%
\pgfpathlineto{\pgfqpoint{2.072740in}{0.632834in}}%
\pgfpathlineto{\pgfqpoint{2.075962in}{0.626488in}}%
\pgfpathlineto{\pgfqpoint{2.077036in}{0.622223in}}%
\pgfpathlineto{\pgfqpoint{2.078109in}{0.622422in}}%
\pgfpathlineto{\pgfqpoint{2.079183in}{0.620934in}}%
\pgfpathlineto{\pgfqpoint{2.080257in}{0.625496in}}%
\pgfpathlineto{\pgfqpoint{2.083479in}{0.625033in}}%
\pgfpathlineto{\pgfqpoint{2.085626in}{0.627876in}}%
\pgfpathlineto{\pgfqpoint{2.087774in}{0.632438in}}%
\pgfpathlineto{\pgfqpoint{2.090995in}{0.632404in}}%
\pgfpathlineto{\pgfqpoint{2.092069in}{0.629793in}}%
\pgfpathlineto{\pgfqpoint{2.093143in}{0.632867in}}%
\pgfpathlineto{\pgfqpoint{2.094217in}{0.633594in}}%
\pgfpathlineto{\pgfqpoint{2.098512in}{0.633396in}}%
\pgfpathlineto{\pgfqpoint{2.100660in}{0.642123in}}%
\pgfpathlineto{\pgfqpoint{2.101734in}{0.641131in}}%
\pgfpathlineto{\pgfqpoint{2.102808in}{0.644734in}}%
\pgfpathlineto{\pgfqpoint{2.106029in}{0.645494in}}%
\pgfpathlineto{\pgfqpoint{2.107103in}{0.642718in}}%
\pgfpathlineto{\pgfqpoint{2.108177in}{0.646717in}}%
\pgfpathlineto{\pgfqpoint{2.109251in}{0.644701in}}%
\pgfpathlineto{\pgfqpoint{2.110325in}{0.646155in}}%
\pgfpathlineto{\pgfqpoint{2.113546in}{0.645461in}}%
\pgfpathlineto{\pgfqpoint{2.114620in}{0.647742in}}%
\pgfpathlineto{\pgfqpoint{2.116768in}{0.653560in}}%
\pgfpathlineto{\pgfqpoint{2.117841in}{0.652767in}}%
\pgfpathlineto{\pgfqpoint{2.121063in}{0.657196in}}%
\pgfpathlineto{\pgfqpoint{2.122137in}{0.654915in}}%
\pgfpathlineto{\pgfqpoint{2.123211in}{0.656204in}}%
\pgfpathlineto{\pgfqpoint{2.124284in}{0.655047in}}%
\pgfpathlineto{\pgfqpoint{2.125358in}{0.649626in}}%
\pgfpathlineto{\pgfqpoint{2.128580in}{0.646519in}}%
\pgfpathlineto{\pgfqpoint{2.130728in}{0.648502in}}%
\pgfpathlineto{\pgfqpoint{2.131801in}{0.645032in}}%
\pgfpathlineto{\pgfqpoint{2.132875in}{0.643610in}}%
\pgfpathlineto{\pgfqpoint{2.136097in}{0.647544in}}%
\pgfpathlineto{\pgfqpoint{2.137171in}{0.643445in}}%
\pgfpathlineto{\pgfqpoint{2.138244in}{0.643048in}}%
\pgfpathlineto{\pgfqpoint{2.140392in}{0.645164in}}%
\pgfpathlineto{\pgfqpoint{2.143614in}{0.646750in}}%
\pgfpathlineto{\pgfqpoint{2.144687in}{0.649924in}}%
\pgfpathlineto{\pgfqpoint{2.145761in}{0.644238in}}%
\pgfpathlineto{\pgfqpoint{2.146835in}{0.646155in}}%
\pgfpathlineto{\pgfqpoint{2.147909in}{0.642718in}}%
\pgfpathlineto{\pgfqpoint{2.151130in}{0.645858in}}%
\pgfpathlineto{\pgfqpoint{2.152204in}{0.642585in}}%
\pgfpathlineto{\pgfqpoint{2.153278in}{0.644668in}}%
\pgfpathlineto{\pgfqpoint{2.154352in}{0.642883in}}%
\pgfpathlineto{\pgfqpoint{2.155426in}{0.645428in}}%
\pgfpathlineto{\pgfqpoint{2.158647in}{0.643941in}}%
\pgfpathlineto{\pgfqpoint{2.159721in}{0.650948in}}%
\pgfpathlineto{\pgfqpoint{2.160795in}{0.649924in}}%
\pgfpathlineto{\pgfqpoint{2.161869in}{0.649725in}}%
\pgfpathlineto{\pgfqpoint{2.162943in}{0.651874in}}%
\pgfpathlineto{\pgfqpoint{2.167238in}{0.649527in}}%
\pgfpathlineto{\pgfqpoint{2.168312in}{0.650453in}}%
\pgfpathlineto{\pgfqpoint{2.169386in}{0.652800in}}%
\pgfpathlineto{\pgfqpoint{2.170460in}{0.652767in}}%
\pgfpathlineto{\pgfqpoint{2.174755in}{0.654915in}}%
\pgfpathlineto{\pgfqpoint{2.175829in}{0.658287in}}%
\pgfpathlineto{\pgfqpoint{2.176903in}{0.657031in}}%
\pgfpathlineto{\pgfqpoint{2.177976in}{0.653560in}}%
\pgfpathlineto{\pgfqpoint{2.181198in}{0.647775in}}%
\pgfpathlineto{\pgfqpoint{2.182272in}{0.648337in}}%
\pgfpathlineto{\pgfqpoint{2.183346in}{0.647048in}}%
\pgfpathlineto{\pgfqpoint{2.184419in}{0.647643in}}%
\pgfpathlineto{\pgfqpoint{2.185493in}{0.643247in}}%
\pgfpathlineto{\pgfqpoint{2.189789in}{0.644337in}}%
\pgfpathlineto{\pgfqpoint{2.190862in}{0.641759in}}%
\pgfpathlineto{\pgfqpoint{2.191936in}{0.647246in}}%
\pgfpathlineto{\pgfqpoint{2.193010in}{0.636966in}}%
\pgfpathlineto{\pgfqpoint{2.196232in}{0.631446in}}%
\pgfpathlineto{\pgfqpoint{2.198379in}{0.642420in}}%
\pgfpathlineto{\pgfqpoint{2.199453in}{0.634123in}}%
\pgfpathlineto{\pgfqpoint{2.200527in}{0.635115in}}%
\pgfpathlineto{\pgfqpoint{2.204822in}{0.635743in}}%
\pgfpathlineto{\pgfqpoint{2.205896in}{0.633760in}}%
\pgfpathlineto{\pgfqpoint{2.206970in}{0.635214in}}%
\pgfpathlineto{\pgfqpoint{2.208044in}{0.641396in}}%
\pgfpathlineto{\pgfqpoint{2.211265in}{0.641726in}}%
\pgfpathlineto{\pgfqpoint{2.212339in}{0.644800in}}%
\pgfpathlineto{\pgfqpoint{2.213413in}{0.644767in}}%
\pgfpathlineto{\pgfqpoint{2.214487in}{0.646949in}}%
\pgfpathlineto{\pgfqpoint{2.215561in}{0.647478in}}%
\pgfpathlineto{\pgfqpoint{2.218782in}{0.647511in}}%
\pgfpathlineto{\pgfqpoint{2.220930in}{0.650882in}}%
\pgfpathlineto{\pgfqpoint{2.222004in}{0.649064in}}%
\pgfpathlineto{\pgfqpoint{2.223078in}{0.652667in}}%
\pgfpathlineto{\pgfqpoint{2.228447in}{0.648205in}}%
\pgfpathlineto{\pgfqpoint{2.229521in}{0.650353in}}%
\pgfpathlineto{\pgfqpoint{2.230594in}{0.646651in}}%
\pgfpathlineto{\pgfqpoint{2.234890in}{0.647412in}}%
\pgfpathlineto{\pgfqpoint{2.235964in}{0.648800in}}%
\pgfpathlineto{\pgfqpoint{2.238111in}{0.653428in}}%
\pgfpathlineto{\pgfqpoint{2.241333in}{0.652800in}}%
\pgfpathlineto{\pgfqpoint{2.242407in}{0.653163in}}%
\pgfpathlineto{\pgfqpoint{2.243481in}{0.651940in}}%
\pgfpathlineto{\pgfqpoint{2.244554in}{0.653362in}}%
\pgfpathlineto{\pgfqpoint{2.245628in}{0.653031in}}%
\pgfpathlineto{\pgfqpoint{2.248850in}{0.655907in}}%
\pgfpathlineto{\pgfqpoint{2.249924in}{0.655576in}}%
\pgfpathlineto{\pgfqpoint{2.250997in}{0.656138in}}%
\pgfpathlineto{\pgfqpoint{2.252071in}{0.654287in}}%
\pgfpathlineto{\pgfqpoint{2.256367in}{0.656832in}}%
\pgfpathlineto{\pgfqpoint{2.257440in}{0.656006in}}%
\pgfpathlineto{\pgfqpoint{2.258514in}{0.654353in}}%
\pgfpathlineto{\pgfqpoint{2.259588in}{0.654452in}}%
\pgfpathlineto{\pgfqpoint{2.260662in}{0.655246in}}%
\pgfpathlineto{\pgfqpoint{2.263883in}{0.656237in}}%
\pgfpathlineto{\pgfqpoint{2.264957in}{0.657196in}}%
\pgfpathlineto{\pgfqpoint{2.266031in}{0.656336in}}%
\pgfpathlineto{\pgfqpoint{2.267105in}{0.657626in}}%
\pgfpathlineto{\pgfqpoint{2.268179in}{0.659873in}}%
\pgfpathlineto{\pgfqpoint{2.272474in}{0.661592in}}%
\pgfpathlineto{\pgfqpoint{2.273548in}{0.663906in}}%
\pgfpathlineto{\pgfqpoint{2.274622in}{0.663014in}}%
\pgfpathlineto{\pgfqpoint{2.275696in}{0.657725in}}%
\pgfpathlineto{\pgfqpoint{2.278917in}{0.663014in}}%
\pgfpathlineto{\pgfqpoint{2.279991in}{0.659543in}}%
\pgfpathlineto{\pgfqpoint{2.281065in}{0.658221in}}%
\pgfpathlineto{\pgfqpoint{2.282139in}{0.659906in}}%
\pgfpathlineto{\pgfqpoint{2.283213in}{0.660105in}}%
\pgfpathlineto{\pgfqpoint{2.287508in}{0.661427in}}%
\pgfpathlineto{\pgfqpoint{2.288582in}{0.663840in}}%
\pgfpathlineto{\pgfqpoint{2.289656in}{0.664303in}}%
\pgfpathlineto{\pgfqpoint{2.290729in}{0.661625in}}%
\pgfpathlineto{\pgfqpoint{2.293951in}{0.659278in}}%
\pgfpathlineto{\pgfqpoint{2.295025in}{0.660402in}}%
\pgfpathlineto{\pgfqpoint{2.296099in}{0.663014in}}%
\pgfpathlineto{\pgfqpoint{2.297172in}{0.659642in}}%
\pgfpathlineto{\pgfqpoint{2.298246in}{0.662154in}}%
\pgfpathlineto{\pgfqpoint{2.301468in}{0.662683in}}%
\pgfpathlineto{\pgfqpoint{2.302542in}{0.662253in}}%
\pgfpathlineto{\pgfqpoint{2.303616in}{0.664270in}}%
\pgfpathlineto{\pgfqpoint{2.304689in}{0.664303in}}%
\pgfpathlineto{\pgfqpoint{2.305763in}{0.662749in}}%
\pgfpathlineto{\pgfqpoint{2.308985in}{0.663509in}}%
\pgfpathlineto{\pgfqpoint{2.310059in}{0.660006in}}%
\pgfpathlineto{\pgfqpoint{2.311132in}{0.660700in}}%
\pgfpathlineto{\pgfqpoint{2.312206in}{0.659477in}}%
\pgfpathlineto{\pgfqpoint{2.313280in}{0.661361in}}%
\pgfpathlineto{\pgfqpoint{2.316502in}{0.660369in}}%
\pgfpathlineto{\pgfqpoint{2.317575in}{0.658518in}}%
\pgfpathlineto{\pgfqpoint{2.318649in}{0.662518in}}%
\pgfpathlineto{\pgfqpoint{2.320797in}{0.661030in}}%
\pgfpathlineto{\pgfqpoint{2.324018in}{0.663675in}}%
\pgfpathlineto{\pgfqpoint{2.325092in}{0.659973in}}%
\pgfpathlineto{\pgfqpoint{2.326166in}{0.659080in}}%
\pgfpathlineto{\pgfqpoint{2.331535in}{0.661526in}}%
\pgfpathlineto{\pgfqpoint{2.333683in}{0.655279in}}%
\pgfpathlineto{\pgfqpoint{2.334757in}{0.655510in}}%
\pgfpathlineto{\pgfqpoint{2.335831in}{0.654585in}}%
\pgfpathlineto{\pgfqpoint{2.340126in}{0.662749in}}%
\pgfpathlineto{\pgfqpoint{2.341200in}{0.663906in}}%
\pgfpathlineto{\pgfqpoint{2.342274in}{0.659477in}}%
\pgfpathlineto{\pgfqpoint{2.343348in}{0.659510in}}%
\pgfpathlineto{\pgfqpoint{2.346569in}{0.648172in}}%
\pgfpathlineto{\pgfqpoint{2.347643in}{0.648767in}}%
\pgfpathlineto{\pgfqpoint{2.349791in}{0.657427in}}%
\pgfpathlineto{\pgfqpoint{2.350864in}{0.656634in}}%
\pgfpathlineto{\pgfqpoint{2.354086in}{0.659444in}}%
\pgfpathlineto{\pgfqpoint{2.355160in}{0.653725in}}%
\pgfpathlineto{\pgfqpoint{2.356234in}{0.652568in}}%
\pgfpathlineto{\pgfqpoint{2.358381in}{0.654386in}}%
\pgfpathlineto{\pgfqpoint{2.361603in}{0.650982in}}%
\pgfpathlineto{\pgfqpoint{2.362677in}{0.651213in}}%
\pgfpathlineto{\pgfqpoint{2.364824in}{0.639115in}}%
\pgfpathlineto{\pgfqpoint{2.365898in}{0.640073in}}%
\pgfpathlineto{\pgfqpoint{2.369120in}{0.645230in}}%
\pgfpathlineto{\pgfqpoint{2.370193in}{0.644569in}}%
\pgfpathlineto{\pgfqpoint{2.371267in}{0.651709in}}%
\pgfpathlineto{\pgfqpoint{2.373415in}{0.651180in}}%
\pgfpathlineto{\pgfqpoint{2.376637in}{0.649097in}}%
\pgfpathlineto{\pgfqpoint{2.377710in}{0.651345in}}%
\pgfpathlineto{\pgfqpoint{2.378784in}{0.651147in}}%
\pgfpathlineto{\pgfqpoint{2.379858in}{0.652337in}}%
\pgfpathlineto{\pgfqpoint{2.380932in}{0.648602in}}%
\pgfpathlineto{\pgfqpoint{2.384153in}{0.647775in}}%
\pgfpathlineto{\pgfqpoint{2.385227in}{0.648635in}}%
\pgfpathlineto{\pgfqpoint{2.387375in}{0.647147in}}%
\pgfpathlineto{\pgfqpoint{2.388449in}{0.647940in}}%
\pgfpathlineto{\pgfqpoint{2.394892in}{0.648535in}}%
\pgfpathlineto{\pgfqpoint{2.395966in}{0.647544in}}%
\pgfpathlineto{\pgfqpoint{2.400261in}{0.652337in}}%
\pgfpathlineto{\pgfqpoint{2.402409in}{0.657493in}}%
\pgfpathlineto{\pgfqpoint{2.403482in}{0.661129in}}%
\pgfpathlineto{\pgfqpoint{2.406704in}{0.659642in}}%
\pgfpathlineto{\pgfqpoint{2.407778in}{0.658221in}}%
\pgfpathlineto{\pgfqpoint{2.408852in}{0.659807in}}%
\pgfpathlineto{\pgfqpoint{2.410999in}{0.657758in}}%
\pgfpathlineto{\pgfqpoint{2.415295in}{0.658055in}}%
\pgfpathlineto{\pgfqpoint{2.417442in}{0.659576in}}%
\pgfpathlineto{\pgfqpoint{2.421738in}{0.660931in}}%
\pgfpathlineto{\pgfqpoint{2.423885in}{0.666617in}}%
\pgfpathlineto{\pgfqpoint{2.424959in}{0.664468in}}%
\pgfpathlineto{\pgfqpoint{2.426033in}{0.666187in}}%
\pgfpathlineto{\pgfqpoint{2.429255in}{0.665956in}}%
\pgfpathlineto{\pgfqpoint{2.430328in}{0.662749in}}%
\pgfpathlineto{\pgfqpoint{2.432476in}{0.661427in}}%
\pgfpathlineto{\pgfqpoint{2.433550in}{0.673690in}}%
\pgfpathlineto{\pgfqpoint{2.437845in}{0.672699in}}%
\pgfpathlineto{\pgfqpoint{2.438919in}{0.670484in}}%
\pgfpathlineto{\pgfqpoint{2.441067in}{0.673095in}}%
\pgfpathlineto{\pgfqpoint{2.444288in}{0.674847in}}%
\pgfpathlineto{\pgfqpoint{2.445362in}{0.676203in}}%
\pgfpathlineto{\pgfqpoint{2.446436in}{0.678979in}}%
\pgfpathlineto{\pgfqpoint{2.448584in}{0.678715in}}%
\pgfpathlineto{\pgfqpoint{2.452879in}{0.680202in}}%
\pgfpathlineto{\pgfqpoint{2.453953in}{0.679806in}}%
\pgfpathlineto{\pgfqpoint{2.456101in}{0.681855in}}%
\pgfpathlineto{\pgfqpoint{2.460396in}{0.680268in}}%
\pgfpathlineto{\pgfqpoint{2.461470in}{0.683673in}}%
\pgfpathlineto{\pgfqpoint{2.462544in}{0.682186in}}%
\pgfpathlineto{\pgfqpoint{2.463617in}{0.683045in}}%
\pgfpathlineto{\pgfqpoint{2.470060in}{0.684070in}}%
\pgfpathlineto{\pgfqpoint{2.471134in}{0.686086in}}%
\pgfpathlineto{\pgfqpoint{2.474356in}{0.687309in}}%
\pgfpathlineto{\pgfqpoint{2.475430in}{0.685491in}}%
\pgfpathlineto{\pgfqpoint{2.477577in}{0.687111in}}%
\pgfpathlineto{\pgfqpoint{2.478651in}{0.687739in}}%
\pgfpathlineto{\pgfqpoint{2.481873in}{0.684301in}}%
\pgfpathlineto{\pgfqpoint{2.482947in}{0.680665in}}%
\pgfpathlineto{\pgfqpoint{2.486168in}{0.684334in}}%
\pgfpathlineto{\pgfqpoint{2.489390in}{0.683541in}}%
\pgfpathlineto{\pgfqpoint{2.491537in}{0.684301in}}%
\pgfpathlineto{\pgfqpoint{2.493685in}{0.683276in}}%
\pgfpathlineto{\pgfqpoint{2.496906in}{0.684896in}}%
\pgfpathlineto{\pgfqpoint{2.497980in}{0.683012in}}%
\pgfpathlineto{\pgfqpoint{2.500128in}{0.684004in}}%
\pgfpathlineto{\pgfqpoint{2.501202in}{0.682880in}}%
\pgfpathlineto{\pgfqpoint{2.505497in}{0.683210in}}%
\pgfpathlineto{\pgfqpoint{2.506571in}{0.682682in}}%
\pgfpathlineto{\pgfqpoint{2.507645in}{0.683276in}}%
\pgfpathlineto{\pgfqpoint{2.511940in}{0.686351in}}%
\pgfpathlineto{\pgfqpoint{2.514088in}{0.686152in}}%
\pgfpathlineto{\pgfqpoint{2.515162in}{0.690714in}}%
\pgfpathlineto{\pgfqpoint{2.516236in}{0.690714in}}%
\pgfpathlineto{\pgfqpoint{2.520531in}{0.693821in}}%
\pgfpathlineto{\pgfqpoint{2.523752in}{0.690945in}}%
\pgfpathlineto{\pgfqpoint{2.526974in}{0.691078in}}%
\pgfpathlineto{\pgfqpoint{2.528048in}{0.695209in}}%
\pgfpathlineto{\pgfqpoint{2.529122in}{0.694912in}}%
\pgfpathlineto{\pgfqpoint{2.530195in}{0.695474in}}%
\pgfpathlineto{\pgfqpoint{2.531269in}{0.693755in}}%
\pgfpathlineto{\pgfqpoint{2.535565in}{0.693458in}}%
\pgfpathlineto{\pgfqpoint{2.536638in}{0.694284in}}%
\pgfpathlineto{\pgfqpoint{2.537712in}{0.693788in}}%
\pgfpathlineto{\pgfqpoint{2.538786in}{0.695837in}}%
\pgfpathlineto{\pgfqpoint{2.542008in}{0.697457in}}%
\pgfpathlineto{\pgfqpoint{2.543081in}{0.697193in}}%
\pgfpathlineto{\pgfqpoint{2.544155in}{0.693226in}}%
\pgfpathlineto{\pgfqpoint{2.545229in}{0.693061in}}%
\pgfpathlineto{\pgfqpoint{2.546303in}{0.695573in}}%
\pgfpathlineto{\pgfqpoint{2.549525in}{0.698251in}}%
\pgfpathlineto{\pgfqpoint{2.550598in}{0.700069in}}%
\pgfpathlineto{\pgfqpoint{2.551672in}{0.703143in}}%
\pgfpathlineto{\pgfqpoint{2.552746in}{0.703903in}}%
\pgfpathlineto{\pgfqpoint{2.553820in}{0.702680in}}%
\pgfpathlineto{\pgfqpoint{2.558115in}{0.702845in}}%
\pgfpathlineto{\pgfqpoint{2.559189in}{0.704531in}}%
\pgfpathlineto{\pgfqpoint{2.560263in}{0.705060in}}%
\pgfpathlineto{\pgfqpoint{2.561337in}{0.707506in}}%
\pgfpathlineto{\pgfqpoint{2.564558in}{0.708828in}}%
\pgfpathlineto{\pgfqpoint{2.565632in}{0.706349in}}%
\pgfpathlineto{\pgfqpoint{2.566706in}{0.707308in}}%
\pgfpathlineto{\pgfqpoint{2.567780in}{0.707308in}}%
\pgfpathlineto{\pgfqpoint{2.568854in}{0.702349in}}%
\pgfpathlineto{\pgfqpoint{2.572075in}{0.698879in}}%
\pgfpathlineto{\pgfqpoint{2.573149in}{0.704035in}}%
\pgfpathlineto{\pgfqpoint{2.574223in}{0.704829in}}%
\pgfpathlineto{\pgfqpoint{2.575297in}{0.701060in}}%
\pgfpathlineto{\pgfqpoint{2.576370in}{0.701060in}}%
\pgfpathlineto{\pgfqpoint{2.579592in}{0.703077in}}%
\pgfpathlineto{\pgfqpoint{2.580666in}{0.701754in}}%
\pgfpathlineto{\pgfqpoint{2.581740in}{0.702283in}}%
\pgfpathlineto{\pgfqpoint{2.582814in}{0.700366in}}%
\pgfpathlineto{\pgfqpoint{2.583887in}{0.705655in}}%
\pgfpathlineto{\pgfqpoint{2.587109in}{0.704498in}}%
\pgfpathlineto{\pgfqpoint{2.588183in}{0.703440in}}%
\pgfpathlineto{\pgfqpoint{2.589257in}{0.707770in}}%
\pgfpathlineto{\pgfqpoint{2.590330in}{0.701887in}}%
\pgfpathlineto{\pgfqpoint{2.591404in}{0.699804in}}%
\pgfpathlineto{\pgfqpoint{2.594626in}{0.698383in}}%
\pgfpathlineto{\pgfqpoint{2.596773in}{0.700531in}}%
\pgfpathlineto{\pgfqpoint{2.597847in}{0.698085in}}%
\pgfpathlineto{\pgfqpoint{2.598921in}{0.700267in}}%
\pgfpathlineto{\pgfqpoint{2.603216in}{0.705093in}}%
\pgfpathlineto{\pgfqpoint{2.604290in}{0.707605in}}%
\pgfpathlineto{\pgfqpoint{2.605364in}{0.706845in}}%
\pgfpathlineto{\pgfqpoint{2.606438in}{0.710051in}}%
\pgfpathlineto{\pgfqpoint{2.609659in}{0.709721in}}%
\pgfpathlineto{\pgfqpoint{2.611807in}{0.714348in}}%
\pgfpathlineto{\pgfqpoint{2.612881in}{0.713886in}}%
\pgfpathlineto{\pgfqpoint{2.613955in}{0.718745in}}%
\pgfpathlineto{\pgfqpoint{2.617176in}{0.721257in}}%
\pgfpathlineto{\pgfqpoint{2.618250in}{0.720034in}}%
\pgfpathlineto{\pgfqpoint{2.619324in}{0.722811in}}%
\pgfpathlineto{\pgfqpoint{2.620398in}{0.718646in}}%
\pgfpathlineto{\pgfqpoint{2.621472in}{0.717290in}}%
\pgfpathlineto{\pgfqpoint{2.624693in}{0.718613in}}%
\pgfpathlineto{\pgfqpoint{2.625767in}{0.722877in}}%
\pgfpathlineto{\pgfqpoint{2.626841in}{0.724232in}}%
\pgfpathlineto{\pgfqpoint{2.627915in}{0.721984in}}%
\pgfpathlineto{\pgfqpoint{2.628989in}{0.722943in}}%
\pgfpathlineto{\pgfqpoint{2.632210in}{0.724893in}}%
\pgfpathlineto{\pgfqpoint{2.634358in}{0.723207in}}%
\pgfpathlineto{\pgfqpoint{2.635432in}{0.718381in}}%
\pgfpathlineto{\pgfqpoint{2.641875in}{0.731405in}}%
\pgfpathlineto{\pgfqpoint{2.642948in}{0.727637in}}%
\pgfpathlineto{\pgfqpoint{2.644022in}{0.728761in}}%
\pgfpathlineto{\pgfqpoint{2.647244in}{0.731074in}}%
\pgfpathlineto{\pgfqpoint{2.648318in}{0.733851in}}%
\pgfpathlineto{\pgfqpoint{2.649392in}{0.730942in}}%
\pgfpathlineto{\pgfqpoint{2.651539in}{0.731504in}}%
\pgfpathlineto{\pgfqpoint{2.655835in}{0.732892in}}%
\pgfpathlineto{\pgfqpoint{2.656908in}{0.732760in}}%
\pgfpathlineto{\pgfqpoint{2.657982in}{0.732066in}}%
\pgfpathlineto{\pgfqpoint{2.659056in}{0.733289in}}%
\pgfpathlineto{\pgfqpoint{2.663351in}{0.730413in}}%
\pgfpathlineto{\pgfqpoint{2.664425in}{0.730942in}}%
\pgfpathlineto{\pgfqpoint{2.665499in}{0.735471in}}%
\pgfpathlineto{\pgfqpoint{2.666573in}{0.735041in}}%
\pgfpathlineto{\pgfqpoint{2.669794in}{0.740627in}}%
\pgfpathlineto{\pgfqpoint{2.670868in}{0.740826in}}%
\pgfpathlineto{\pgfqpoint{2.671942in}{0.739603in}}%
\pgfpathlineto{\pgfqpoint{2.673016in}{0.740429in}}%
\pgfpathlineto{\pgfqpoint{2.674090in}{0.737884in}}%
\pgfpathlineto{\pgfqpoint{2.677311in}{0.736330in}}%
\pgfpathlineto{\pgfqpoint{2.678385in}{0.738214in}}%
\pgfpathlineto{\pgfqpoint{2.679459in}{0.736793in}}%
\pgfpathlineto{\pgfqpoint{2.681607in}{0.738710in}}%
\pgfpathlineto{\pgfqpoint{2.684828in}{0.730446in}}%
\pgfpathlineto{\pgfqpoint{2.685902in}{0.730182in}}%
\pgfpathlineto{\pgfqpoint{2.688050in}{0.735537in}}%
\pgfpathlineto{\pgfqpoint{2.689124in}{0.737685in}}%
\pgfpathlineto{\pgfqpoint{2.693419in}{0.738809in}}%
\pgfpathlineto{\pgfqpoint{2.694493in}{0.737917in}}%
\pgfpathlineto{\pgfqpoint{2.696640in}{0.742545in}}%
\pgfpathlineto{\pgfqpoint{2.699862in}{0.743437in}}%
\pgfpathlineto{\pgfqpoint{2.700936in}{0.744429in}}%
\pgfpathlineto{\pgfqpoint{2.702010in}{0.748131in}}%
\pgfpathlineto{\pgfqpoint{2.703083in}{0.747040in}}%
\pgfpathlineto{\pgfqpoint{2.704157in}{0.748825in}}%
\pgfpathlineto{\pgfqpoint{2.707379in}{0.747668in}}%
\pgfpathlineto{\pgfqpoint{2.708453in}{0.745189in}}%
\pgfpathlineto{\pgfqpoint{2.709526in}{0.746015in}}%
\pgfpathlineto{\pgfqpoint{2.710600in}{0.743470in}}%
\pgfpathlineto{\pgfqpoint{2.711674in}{0.745222in}}%
\pgfpathlineto{\pgfqpoint{2.714896in}{0.745156in}}%
\pgfpathlineto{\pgfqpoint{2.718117in}{0.752560in}}%
\pgfpathlineto{\pgfqpoint{2.719191in}{0.752263in}}%
\pgfpathlineto{\pgfqpoint{2.723486in}{0.753122in}}%
\pgfpathlineto{\pgfqpoint{2.724560in}{0.756692in}}%
\pgfpathlineto{\pgfqpoint{2.725634in}{0.756395in}}%
\pgfpathlineto{\pgfqpoint{2.726708in}{0.757651in}}%
\pgfpathlineto{\pgfqpoint{2.732077in}{0.761254in}}%
\pgfpathlineto{\pgfqpoint{2.734225in}{0.759337in}}%
\pgfpathlineto{\pgfqpoint{2.737446in}{0.757783in}}%
\pgfpathlineto{\pgfqpoint{2.738520in}{0.759667in}}%
\pgfpathlineto{\pgfqpoint{2.739594in}{0.754015in}}%
\pgfpathlineto{\pgfqpoint{2.740668in}{0.757188in}}%
\pgfpathlineto{\pgfqpoint{2.741742in}{0.753254in}}%
\pgfpathlineto{\pgfqpoint{2.744963in}{0.753684in}}%
\pgfpathlineto{\pgfqpoint{2.746037in}{0.758576in}}%
\pgfpathlineto{\pgfqpoint{2.747111in}{0.756527in}}%
\pgfpathlineto{\pgfqpoint{2.749258in}{0.760262in}}%
\pgfpathlineto{\pgfqpoint{2.752480in}{0.761617in}}%
\pgfpathlineto{\pgfqpoint{2.753554in}{0.764824in}}%
\pgfpathlineto{\pgfqpoint{2.754628in}{0.753387in}}%
\pgfpathlineto{\pgfqpoint{2.755702in}{0.762312in}}%
\pgfpathlineto{\pgfqpoint{2.756775in}{0.756229in}}%
\pgfpathlineto{\pgfqpoint{2.759997in}{0.745453in}}%
\pgfpathlineto{\pgfqpoint{2.764292in}{0.762345in}}%
\pgfpathlineto{\pgfqpoint{2.767514in}{0.761584in}}%
\pgfpathlineto{\pgfqpoint{2.768588in}{0.765154in}}%
\pgfpathlineto{\pgfqpoint{2.769661in}{0.764659in}}%
\pgfpathlineto{\pgfqpoint{2.770735in}{0.763369in}}%
\pgfpathlineto{\pgfqpoint{2.771809in}{0.766311in}}%
\pgfpathlineto{\pgfqpoint{2.775031in}{0.765452in}}%
\pgfpathlineto{\pgfqpoint{2.776104in}{0.760824in}}%
\pgfpathlineto{\pgfqpoint{2.777178in}{0.760758in}}%
\pgfpathlineto{\pgfqpoint{2.779326in}{0.762642in}}%
\pgfpathlineto{\pgfqpoint{2.783621in}{0.763601in}}%
\pgfpathlineto{\pgfqpoint{2.784695in}{0.766972in}}%
\pgfpathlineto{\pgfqpoint{2.785769in}{0.768063in}}%
\pgfpathlineto{\pgfqpoint{2.786843in}{0.766972in}}%
\pgfpathlineto{\pgfqpoint{2.786843in}{0.766972in}}%
\pgfusepath{stroke}%
\end{pgfscope}%
\begin{pgfscope}%
\pgfpathrectangle{\pgfqpoint{0.320934in}{0.385400in}}{\pgfqpoint{2.583333in}{0.400885in}}%
\pgfusepath{clip}%
\pgfsetroundcap%
\pgfsetroundjoin%
\pgfsetlinewidth{1.505625pt}%
\definecolor{currentstroke}{rgb}{0.737255,0.741176,0.133333}%
\pgfsetstrokecolor{currentstroke}%
\pgfsetdash{}{0pt}%
\pgfpathmoveto{\pgfqpoint{0.438358in}{0.468814in}}%
\pgfpathlineto{\pgfqpoint{0.440506in}{0.468814in}}%
\pgfpathlineto{\pgfqpoint{0.441580in}{0.467946in}}%
\pgfpathlineto{\pgfqpoint{0.453392in}{0.465495in}}%
\pgfpathlineto{\pgfqpoint{0.454466in}{0.464607in}}%
\pgfpathlineto{\pgfqpoint{0.456613in}{0.464607in}}%
\pgfpathlineto{\pgfqpoint{0.461982in}{0.463905in}}%
\pgfpathlineto{\pgfqpoint{0.468425in}{0.463905in}}%
\pgfpathlineto{\pgfqpoint{0.469499in}{0.462609in}}%
\pgfpathlineto{\pgfqpoint{0.470573in}{0.460235in}}%
\pgfpathlineto{\pgfqpoint{0.471647in}{0.459639in}}%
\pgfpathlineto{\pgfqpoint{0.477016in}{0.458829in}}%
\pgfpathlineto{\pgfqpoint{0.478090in}{0.456408in}}%
\pgfpathlineto{\pgfqpoint{0.479164in}{0.455600in}}%
\pgfpathlineto{\pgfqpoint{0.482385in}{0.456240in}}%
\pgfpathlineto{\pgfqpoint{0.483459in}{0.454859in}}%
\pgfpathlineto{\pgfqpoint{0.484533in}{0.454451in}}%
\pgfpathlineto{\pgfqpoint{0.485607in}{0.455138in}}%
\pgfpathlineto{\pgfqpoint{0.486681in}{0.454484in}}%
\pgfpathlineto{\pgfqpoint{0.490976in}{0.454910in}}%
\pgfpathlineto{\pgfqpoint{0.492050in}{0.453756in}}%
\pgfpathlineto{\pgfqpoint{0.497419in}{0.453495in}}%
\pgfpathlineto{\pgfqpoint{0.498493in}{0.452439in}}%
\pgfpathlineto{\pgfqpoint{0.499567in}{0.453702in}}%
\pgfpathlineto{\pgfqpoint{0.500641in}{0.453285in}}%
\pgfpathlineto{\pgfqpoint{0.501714in}{0.453810in}}%
\pgfpathlineto{\pgfqpoint{0.504936in}{0.453763in}}%
\pgfpathlineto{\pgfqpoint{0.506010in}{0.454482in}}%
\pgfpathlineto{\pgfqpoint{0.507084in}{0.454055in}}%
\pgfpathlineto{\pgfqpoint{0.508157in}{0.452860in}}%
\pgfpathlineto{\pgfqpoint{0.509231in}{0.453265in}}%
\pgfpathlineto{\pgfqpoint{0.514601in}{0.453466in}}%
\pgfpathlineto{\pgfqpoint{0.523191in}{0.452024in}}%
\pgfpathlineto{\pgfqpoint{0.524265in}{0.451288in}}%
\pgfpathlineto{\pgfqpoint{0.528560in}{0.450786in}}%
\pgfpathlineto{\pgfqpoint{0.531782in}{0.451648in}}%
\pgfpathlineto{\pgfqpoint{0.538225in}{0.450195in}}%
\pgfpathlineto{\pgfqpoint{0.544668in}{0.450886in}}%
\pgfpathlineto{\pgfqpoint{0.546816in}{0.448173in}}%
\pgfpathlineto{\pgfqpoint{0.550037in}{0.449184in}}%
\pgfpathlineto{\pgfqpoint{0.551111in}{0.448607in}}%
\pgfpathlineto{\pgfqpoint{0.558628in}{0.449094in}}%
\pgfpathlineto{\pgfqpoint{0.560776in}{0.447161in}}%
\pgfpathlineto{\pgfqpoint{0.561849in}{0.447025in}}%
\pgfpathlineto{\pgfqpoint{0.576883in}{0.447588in}}%
\pgfpathlineto{\pgfqpoint{0.617689in}{0.447588in}}%
\pgfpathlineto{\pgfqpoint{0.620911in}{0.446502in}}%
\pgfpathlineto{\pgfqpoint{0.621984in}{0.444104in}}%
\pgfpathlineto{\pgfqpoint{0.625206in}{0.445574in}}%
\pgfpathlineto{\pgfqpoint{0.627354in}{0.444665in}}%
\pgfpathlineto{\pgfqpoint{0.628427in}{0.445435in}}%
\pgfpathlineto{\pgfqpoint{0.629501in}{0.444546in}}%
\pgfpathlineto{\pgfqpoint{0.633797in}{0.443444in}}%
\pgfpathlineto{\pgfqpoint{0.635944in}{0.443182in}}%
\pgfpathlineto{\pgfqpoint{0.637018in}{0.443857in}}%
\pgfpathlineto{\pgfqpoint{0.642387in}{0.445045in}}%
\pgfpathlineto{\pgfqpoint{0.644535in}{0.443646in}}%
\pgfpathlineto{\pgfqpoint{0.649904in}{0.442172in}}%
\pgfpathlineto{\pgfqpoint{0.650978in}{0.443082in}}%
\pgfpathlineto{\pgfqpoint{0.652052in}{0.442942in}}%
\pgfpathlineto{\pgfqpoint{0.657421in}{0.443446in}}%
\pgfpathlineto{\pgfqpoint{0.659569in}{0.440730in}}%
\pgfpathlineto{\pgfqpoint{0.662790in}{0.440115in}}%
\pgfpathlineto{\pgfqpoint{0.664938in}{0.441312in}}%
\pgfpathlineto{\pgfqpoint{0.667086in}{0.440035in}}%
\pgfpathlineto{\pgfqpoint{0.672455in}{0.439970in}}%
\pgfpathlineto{\pgfqpoint{0.673529in}{0.440894in}}%
\pgfpathlineto{\pgfqpoint{0.674602in}{0.440682in}}%
\pgfpathlineto{\pgfqpoint{0.677824in}{0.440957in}}%
\pgfpathlineto{\pgfqpoint{0.679972in}{0.440368in}}%
\pgfpathlineto{\pgfqpoint{0.681045in}{0.440145in}}%
\pgfpathlineto{\pgfqpoint{0.682119in}{0.440508in}}%
\pgfpathlineto{\pgfqpoint{0.696079in}{0.441210in}}%
\pgfpathlineto{\pgfqpoint{0.697153in}{0.440631in}}%
\pgfpathlineto{\pgfqpoint{0.707891in}{0.440470in}}%
\pgfpathlineto{\pgfqpoint{0.710039in}{0.438786in}}%
\pgfpathlineto{\pgfqpoint{0.711113in}{0.438279in}}%
\pgfpathlineto{\pgfqpoint{0.715408in}{0.438645in}}%
\pgfpathlineto{\pgfqpoint{0.718630in}{0.438391in}}%
\pgfpathlineto{\pgfqpoint{0.719704in}{0.438260in}}%
\pgfpathlineto{\pgfqpoint{0.726147in}{0.438605in}}%
\pgfpathlineto{\pgfqpoint{0.731516in}{0.437906in}}%
\pgfpathlineto{\pgfqpoint{0.733664in}{0.436745in}}%
\pgfpathlineto{\pgfqpoint{0.734737in}{0.436568in}}%
\pgfpathlineto{\pgfqpoint{0.740107in}{0.437419in}}%
\pgfpathlineto{\pgfqpoint{0.742254in}{0.436887in}}%
\pgfpathlineto{\pgfqpoint{0.746550in}{0.436276in}}%
\pgfpathlineto{\pgfqpoint{0.747623in}{0.435851in}}%
\pgfpathlineto{\pgfqpoint{0.752993in}{0.436908in}}%
\pgfpathlineto{\pgfqpoint{0.754067in}{0.437634in}}%
\pgfpathlineto{\pgfqpoint{0.756214in}{0.437336in}}%
\pgfpathlineto{\pgfqpoint{0.757288in}{0.437098in}}%
\pgfpathlineto{\pgfqpoint{0.762657in}{0.436960in}}%
\pgfpathlineto{\pgfqpoint{0.763731in}{0.435395in}}%
\pgfpathlineto{\pgfqpoint{0.768026in}{0.436075in}}%
\pgfpathlineto{\pgfqpoint{0.769100in}{0.435376in}}%
\pgfpathlineto{\pgfqpoint{0.771248in}{0.435888in}}%
\pgfpathlineto{\pgfqpoint{0.772322in}{0.435640in}}%
\pgfpathlineto{\pgfqpoint{0.776617in}{0.435781in}}%
\pgfpathlineto{\pgfqpoint{0.778765in}{0.436106in}}%
\pgfpathlineto{\pgfqpoint{0.779839in}{0.435300in}}%
\pgfpathlineto{\pgfqpoint{0.787356in}{0.433784in}}%
\pgfpathlineto{\pgfqpoint{0.793799in}{0.433769in}}%
\pgfpathlineto{\pgfqpoint{0.794872in}{0.433356in}}%
\pgfpathlineto{\pgfqpoint{0.802389in}{0.433654in}}%
\pgfpathlineto{\pgfqpoint{0.815275in}{0.433469in}}%
\pgfpathlineto{\pgfqpoint{0.816349in}{0.432609in}}%
\pgfpathlineto{\pgfqpoint{0.817423in}{0.433032in}}%
\pgfpathlineto{\pgfqpoint{0.822792in}{0.433291in}}%
\pgfpathlineto{\pgfqpoint{0.824940in}{0.433580in}}%
\pgfpathlineto{\pgfqpoint{0.839974in}{0.430483in}}%
\pgfpathlineto{\pgfqpoint{0.859303in}{0.431491in}}%
\pgfpathlineto{\pgfqpoint{0.860377in}{0.431880in}}%
\pgfpathlineto{\pgfqpoint{0.862524in}{0.431028in}}%
\pgfpathlineto{\pgfqpoint{0.865746in}{0.431468in}}%
\pgfpathlineto{\pgfqpoint{0.867893in}{0.430493in}}%
\pgfpathlineto{\pgfqpoint{0.868967in}{0.431339in}}%
\pgfpathlineto{\pgfqpoint{0.870041in}{0.431162in}}%
\pgfpathlineto{\pgfqpoint{0.875410in}{0.431565in}}%
\pgfpathlineto{\pgfqpoint{0.877558in}{0.430801in}}%
\pgfpathlineto{\pgfqpoint{0.889370in}{0.430788in}}%
\pgfpathlineto{\pgfqpoint{0.890444in}{0.430322in}}%
\pgfpathlineto{\pgfqpoint{0.892592in}{0.430729in}}%
\pgfpathlineto{\pgfqpoint{0.903330in}{0.430019in}}%
\pgfpathlineto{\pgfqpoint{0.905478in}{0.430420in}}%
\pgfpathlineto{\pgfqpoint{0.906552in}{0.430131in}}%
\pgfpathlineto{\pgfqpoint{0.907625in}{0.430601in}}%
\pgfpathlineto{\pgfqpoint{0.914068in}{0.430746in}}%
\pgfpathlineto{\pgfqpoint{0.915142in}{0.430219in}}%
\pgfpathlineto{\pgfqpoint{0.921585in}{0.428121in}}%
\pgfpathlineto{\pgfqpoint{0.930176in}{0.429055in}}%
\pgfpathlineto{\pgfqpoint{0.937693in}{0.428911in}}%
\pgfpathlineto{\pgfqpoint{0.940914in}{0.429845in}}%
\pgfpathlineto{\pgfqpoint{0.941988in}{0.429091in}}%
\pgfpathlineto{\pgfqpoint{0.944136in}{0.429791in}}%
\pgfpathlineto{\pgfqpoint{0.945210in}{0.429232in}}%
\pgfpathlineto{\pgfqpoint{0.949505in}{0.428865in}}%
\pgfpathlineto{\pgfqpoint{0.952727in}{0.428530in}}%
\pgfpathlineto{\pgfqpoint{0.958096in}{0.428775in}}%
\pgfpathlineto{\pgfqpoint{0.959170in}{0.426825in}}%
\pgfpathlineto{\pgfqpoint{0.960244in}{0.426061in}}%
\pgfpathlineto{\pgfqpoint{0.970982in}{0.426168in}}%
\pgfpathlineto{\pgfqpoint{0.975277in}{0.424938in}}%
\pgfpathlineto{\pgfqpoint{0.982794in}{0.425614in}}%
\pgfpathlineto{\pgfqpoint{0.997828in}{0.425667in}}%
\pgfpathlineto{\pgfqpoint{1.001049in}{0.425250in}}%
\pgfpathlineto{\pgfqpoint{1.003197in}{0.425792in}}%
\pgfpathlineto{\pgfqpoint{1.004271in}{0.425249in}}%
\pgfpathlineto{\pgfqpoint{1.005345in}{0.425471in}}%
\pgfpathlineto{\pgfqpoint{1.010714in}{0.425174in}}%
\pgfpathlineto{\pgfqpoint{1.011788in}{0.425881in}}%
\pgfpathlineto{\pgfqpoint{1.012862in}{0.425696in}}%
\pgfpathlineto{\pgfqpoint{1.017157in}{0.425523in}}%
\pgfpathlineto{\pgfqpoint{1.019305in}{0.424875in}}%
\pgfpathlineto{\pgfqpoint{1.020378in}{0.425120in}}%
\pgfpathlineto{\pgfqpoint{1.031117in}{0.424224in}}%
\pgfpathlineto{\pgfqpoint{1.033265in}{0.424446in}}%
\pgfpathlineto{\pgfqpoint{1.035412in}{0.423803in}}%
\pgfpathlineto{\pgfqpoint{1.048298in}{0.424416in}}%
\pgfpathlineto{\pgfqpoint{1.049372in}{0.423145in}}%
\pgfpathlineto{\pgfqpoint{1.053667in}{0.423541in}}%
\pgfpathlineto{\pgfqpoint{1.054741in}{0.423621in}}%
\pgfpathlineto{\pgfqpoint{1.055815in}{0.425828in}}%
\pgfpathlineto{\pgfqpoint{1.061184in}{0.424514in}}%
\pgfpathlineto{\pgfqpoint{1.065480in}{0.425306in}}%
\pgfpathlineto{\pgfqpoint{1.070849in}{0.425328in}}%
\pgfpathlineto{\pgfqpoint{1.072997in}{0.426245in}}%
\pgfpathlineto{\pgfqpoint{1.080513in}{0.425146in}}%
\pgfpathlineto{\pgfqpoint{1.085883in}{0.425813in}}%
\pgfpathlineto{\pgfqpoint{1.098769in}{0.425206in}}%
\pgfpathlineto{\pgfqpoint{1.100916in}{0.423886in}}%
\pgfpathlineto{\pgfqpoint{1.101990in}{0.424086in}}%
\pgfpathlineto{\pgfqpoint{1.103064in}{0.423453in}}%
\pgfpathlineto{\pgfqpoint{1.107359in}{0.423054in}}%
\pgfpathlineto{\pgfqpoint{1.110581in}{0.421994in}}%
\pgfpathlineto{\pgfqpoint{1.130984in}{0.424181in}}%
\pgfpathlineto{\pgfqpoint{1.133132in}{0.422854in}}%
\pgfpathlineto{\pgfqpoint{1.139575in}{0.421953in}}%
\pgfpathlineto{\pgfqpoint{1.140648in}{0.421652in}}%
\pgfpathlineto{\pgfqpoint{1.148165in}{0.421291in}}%
\pgfpathlineto{\pgfqpoint{1.153534in}{0.421187in}}%
\pgfpathlineto{\pgfqpoint{1.154608in}{0.422133in}}%
\pgfpathlineto{\pgfqpoint{1.155682in}{0.421778in}}%
\pgfpathlineto{\pgfqpoint{1.159977in}{0.422019in}}%
\pgfpathlineto{\pgfqpoint{1.161051in}{0.421788in}}%
\pgfpathlineto{\pgfqpoint{1.162125in}{0.422194in}}%
\pgfpathlineto{\pgfqpoint{1.163199in}{0.421881in}}%
\pgfpathlineto{\pgfqpoint{1.193266in}{0.421139in}}%
\pgfpathlineto{\pgfqpoint{1.197562in}{0.421459in}}%
\pgfpathlineto{\pgfqpoint{1.198636in}{0.420613in}}%
\pgfpathlineto{\pgfqpoint{1.199710in}{0.420898in}}%
\pgfpathlineto{\pgfqpoint{1.200783in}{0.420388in}}%
\pgfpathlineto{\pgfqpoint{1.204005in}{0.420339in}}%
\pgfpathlineto{\pgfqpoint{1.206153in}{0.419402in}}%
\pgfpathlineto{\pgfqpoint{1.220112in}{0.418569in}}%
\pgfpathlineto{\pgfqpoint{1.227629in}{0.418692in}}%
\pgfpathlineto{\pgfqpoint{1.230851in}{0.418734in}}%
\pgfpathlineto{\pgfqpoint{1.235146in}{0.418555in}}%
\pgfpathlineto{\pgfqpoint{1.237294in}{0.418651in}}%
\pgfpathlineto{\pgfqpoint{1.238368in}{0.417508in}}%
\pgfpathlineto{\pgfqpoint{1.244811in}{0.417899in}}%
\pgfpathlineto{\pgfqpoint{1.245885in}{0.418621in}}%
\pgfpathlineto{\pgfqpoint{1.249106in}{0.419172in}}%
\pgfpathlineto{\pgfqpoint{1.250180in}{0.418630in}}%
\pgfpathlineto{\pgfqpoint{1.251254in}{0.419050in}}%
\pgfpathlineto{\pgfqpoint{1.252328in}{0.418625in}}%
\pgfpathlineto{\pgfqpoint{1.253401in}{0.419224in}}%
\pgfpathlineto{\pgfqpoint{1.258771in}{0.419196in}}%
\pgfpathlineto{\pgfqpoint{1.260918in}{0.418498in}}%
\pgfpathlineto{\pgfqpoint{1.265214in}{0.418435in}}%
\pgfpathlineto{\pgfqpoint{1.268435in}{0.417864in}}%
\pgfpathlineto{\pgfqpoint{1.309241in}{0.418361in}}%
\pgfpathlineto{\pgfqpoint{1.313536in}{0.419331in}}%
\pgfpathlineto{\pgfqpoint{1.319979in}{0.418981in}}%
\pgfpathlineto{\pgfqpoint{1.321053in}{0.419821in}}%
\pgfpathlineto{\pgfqpoint{1.325349in}{0.420453in}}%
\pgfpathlineto{\pgfqpoint{1.326422in}{0.419810in}}%
\pgfpathlineto{\pgfqpoint{1.328570in}{0.421185in}}%
\pgfpathlineto{\pgfqpoint{1.332865in}{0.420191in}}%
\pgfpathlineto{\pgfqpoint{1.333939in}{0.419521in}}%
\pgfpathlineto{\pgfqpoint{1.335013in}{0.419690in}}%
\pgfpathlineto{\pgfqpoint{1.342530in}{0.419513in}}%
\pgfpathlineto{\pgfqpoint{1.343604in}{0.420768in}}%
\pgfpathlineto{\pgfqpoint{1.364007in}{0.419177in}}%
\pgfpathlineto{\pgfqpoint{1.366154in}{0.419175in}}%
\pgfpathlineto{\pgfqpoint{1.396222in}{0.418979in}}%
\pgfpathlineto{\pgfqpoint{1.411256in}{0.419209in}}%
\pgfpathlineto{\pgfqpoint{1.440249in}{0.417670in}}%
\pgfpathlineto{\pgfqpoint{1.441323in}{0.418502in}}%
\pgfpathlineto{\pgfqpoint{1.463874in}{0.418827in}}%
\pgfpathlineto{\pgfqpoint{1.471391in}{0.418146in}}%
\pgfpathlineto{\pgfqpoint{1.519713in}{0.418582in}}%
\pgfpathlineto{\pgfqpoint{1.520787in}{0.419010in}}%
\pgfpathlineto{\pgfqpoint{1.521861in}{0.418497in}}%
\pgfpathlineto{\pgfqpoint{1.524009in}{0.419027in}}%
\pgfpathlineto{\pgfqpoint{1.531526in}{0.418659in}}%
\pgfpathlineto{\pgfqpoint{1.534747in}{0.418441in}}%
\pgfpathlineto{\pgfqpoint{1.535821in}{0.417809in}}%
\pgfpathlineto{\pgfqpoint{1.536895in}{0.418072in}}%
\pgfpathlineto{\pgfqpoint{1.539042in}{0.417781in}}%
\pgfpathlineto{\pgfqpoint{1.544412in}{0.417645in}}%
\pgfpathlineto{\pgfqpoint{1.545486in}{0.415257in}}%
\pgfpathlineto{\pgfqpoint{1.546559in}{0.414834in}}%
\pgfpathlineto{\pgfqpoint{1.550855in}{0.414706in}}%
\pgfpathlineto{\pgfqpoint{1.553002in}{0.414110in}}%
\pgfpathlineto{\pgfqpoint{1.554076in}{0.413924in}}%
\pgfpathlineto{\pgfqpoint{1.564815in}{0.413971in}}%
\pgfpathlineto{\pgfqpoint{1.572331in}{0.413595in}}%
\pgfpathlineto{\pgfqpoint{1.576627in}{0.413329in}}%
\pgfpathlineto{\pgfqpoint{1.603473in}{0.412783in}}%
\pgfpathlineto{\pgfqpoint{1.606694in}{0.412686in}}%
\pgfpathlineto{\pgfqpoint{1.647500in}{0.413428in}}%
\pgfpathlineto{\pgfqpoint{1.651796in}{0.412536in}}%
\pgfpathlineto{\pgfqpoint{1.657165in}{0.412498in}}%
\pgfpathlineto{\pgfqpoint{1.659312in}{0.412246in}}%
\pgfpathlineto{\pgfqpoint{1.685085in}{0.412099in}}%
\pgfpathlineto{\pgfqpoint{1.687232in}{0.412538in}}%
\pgfpathlineto{\pgfqpoint{1.688306in}{0.412206in}}%
\pgfpathlineto{\pgfqpoint{1.689380in}{0.412507in}}%
\pgfpathlineto{\pgfqpoint{1.696897in}{0.412182in}}%
\pgfpathlineto{\pgfqpoint{1.731260in}{0.412724in}}%
\pgfpathlineto{\pgfqpoint{1.732333in}{0.411974in}}%
\pgfpathlineto{\pgfqpoint{1.747367in}{0.412593in}}%
\pgfpathlineto{\pgfqpoint{1.752736in}{0.411670in}}%
\pgfpathlineto{\pgfqpoint{1.757032in}{0.411495in}}%
\pgfpathlineto{\pgfqpoint{1.764549in}{0.411481in}}%
\pgfpathlineto{\pgfqpoint{1.775287in}{0.411595in}}%
\pgfpathlineto{\pgfqpoint{1.816093in}{0.412077in}}%
\pgfpathlineto{\pgfqpoint{1.817167in}{0.411714in}}%
\pgfpathlineto{\pgfqpoint{1.842939in}{0.409969in}}%
\pgfpathlineto{\pgfqpoint{1.852603in}{0.410405in}}%
\pgfpathlineto{\pgfqpoint{1.860120in}{0.410219in}}%
\pgfpathlineto{\pgfqpoint{1.862268in}{0.410910in}}%
\pgfpathlineto{\pgfqpoint{1.866563in}{0.411920in}}%
\pgfpathlineto{\pgfqpoint{1.868711in}{0.410542in}}%
\pgfpathlineto{\pgfqpoint{1.873006in}{0.410789in}}%
\pgfpathlineto{\pgfqpoint{1.874080in}{0.411328in}}%
\pgfpathlineto{\pgfqpoint{1.877302in}{0.411269in}}%
\pgfpathlineto{\pgfqpoint{1.884818in}{0.410875in}}%
\pgfpathlineto{\pgfqpoint{1.892335in}{0.411096in}}%
\pgfpathlineto{\pgfqpoint{1.899852in}{0.410875in}}%
\pgfpathlineto{\pgfqpoint{1.903074in}{0.411692in}}%
\pgfpathlineto{\pgfqpoint{1.907369in}{0.410827in}}%
\pgfpathlineto{\pgfqpoint{1.919181in}{0.409858in}}%
\pgfpathlineto{\pgfqpoint{1.921329in}{0.409782in}}%
\pgfpathlineto{\pgfqpoint{1.926698in}{0.409587in}}%
\pgfpathlineto{\pgfqpoint{1.928846in}{0.409557in}}%
\pgfpathlineto{\pgfqpoint{1.936363in}{0.409141in}}%
\pgfpathlineto{\pgfqpoint{1.940658in}{0.409771in}}%
\pgfpathlineto{\pgfqpoint{1.942806in}{0.409135in}}%
\pgfpathlineto{\pgfqpoint{1.944953in}{0.409059in}}%
\pgfpathlineto{\pgfqpoint{1.956766in}{0.409006in}}%
\pgfpathlineto{\pgfqpoint{1.958913in}{0.408759in}}%
\pgfpathlineto{\pgfqpoint{1.981464in}{0.409109in}}%
\pgfpathlineto{\pgfqpoint{1.982538in}{0.409488in}}%
\pgfpathlineto{\pgfqpoint{2.002941in}{0.409019in}}%
\pgfpathlineto{\pgfqpoint{2.012605in}{0.410098in}}%
\pgfpathlineto{\pgfqpoint{2.019048in}{0.409886in}}%
\pgfpathlineto{\pgfqpoint{2.020122in}{0.410302in}}%
\pgfpathlineto{\pgfqpoint{2.034082in}{0.410855in}}%
\pgfpathlineto{\pgfqpoint{2.035156in}{0.409625in}}%
\pgfpathlineto{\pgfqpoint{2.041599in}{0.409787in}}%
\pgfpathlineto{\pgfqpoint{2.045894in}{0.411071in}}%
\pgfpathlineto{\pgfqpoint{2.057706in}{0.410087in}}%
\pgfpathlineto{\pgfqpoint{2.065223in}{0.409810in}}%
\pgfpathlineto{\pgfqpoint{2.085626in}{0.409846in}}%
\pgfpathlineto{\pgfqpoint{2.090995in}{0.409549in}}%
\pgfpathlineto{\pgfqpoint{2.093143in}{0.409515in}}%
\pgfpathlineto{\pgfqpoint{2.125358in}{0.408481in}}%
\pgfpathlineto{\pgfqpoint{2.139318in}{0.408788in}}%
\pgfpathlineto{\pgfqpoint{2.146835in}{0.408652in}}%
\pgfpathlineto{\pgfqpoint{2.147909in}{0.408847in}}%
\pgfpathlineto{\pgfqpoint{2.162943in}{0.408311in}}%
\pgfpathlineto{\pgfqpoint{2.174755in}{0.408145in}}%
\pgfpathlineto{\pgfqpoint{2.176903in}{0.408031in}}%
\pgfpathlineto{\pgfqpoint{2.183346in}{0.408562in}}%
\pgfpathlineto{\pgfqpoint{2.191936in}{0.408537in}}%
\pgfpathlineto{\pgfqpoint{2.193010in}{0.409115in}}%
\pgfpathlineto{\pgfqpoint{2.197305in}{0.409076in}}%
\pgfpathlineto{\pgfqpoint{2.198379in}{0.408767in}}%
\pgfpathlineto{\pgfqpoint{2.200527in}{0.409188in}}%
\pgfpathlineto{\pgfqpoint{2.213413in}{0.408602in}}%
\pgfpathlineto{\pgfqpoint{2.220930in}{0.408260in}}%
\pgfpathlineto{\pgfqpoint{2.228447in}{0.408401in}}%
\pgfpathlineto{\pgfqpoint{2.230594in}{0.408483in}}%
\pgfpathlineto{\pgfqpoint{2.332609in}{0.407793in}}%
\pgfpathlineto{\pgfqpoint{2.335831in}{0.408000in}}%
\pgfpathlineto{\pgfqpoint{2.342274in}{0.407735in}}%
\pgfpathlineto{\pgfqpoint{2.343348in}{0.407733in}}%
\pgfpathlineto{\pgfqpoint{2.347643in}{0.408273in}}%
\pgfpathlineto{\pgfqpoint{2.350864in}{0.407847in}}%
\pgfpathlineto{\pgfqpoint{2.358381in}{0.407955in}}%
\pgfpathlineto{\pgfqpoint{2.370193in}{0.408464in}}%
\pgfpathlineto{\pgfqpoint{2.372341in}{0.408066in}}%
\pgfpathlineto{\pgfqpoint{2.395966in}{0.408280in}}%
\pgfpathlineto{\pgfqpoint{2.410999in}{0.407728in}}%
\pgfpathlineto{\pgfqpoint{2.432476in}{0.407537in}}%
\pgfpathlineto{\pgfqpoint{2.433550in}{0.406931in}}%
\pgfpathlineto{\pgfqpoint{2.453953in}{0.406657in}}%
\pgfpathlineto{\pgfqpoint{2.467913in}{0.406480in}}%
\pgfpathlineto{\pgfqpoint{2.507645in}{0.406499in}}%
\pgfpathlineto{\pgfqpoint{2.543081in}{0.405932in}}%
\pgfpathlineto{\pgfqpoint{2.546303in}{0.405991in}}%
\pgfpathlineto{\pgfqpoint{2.568854in}{0.405727in}}%
\pgfpathlineto{\pgfqpoint{2.582814in}{0.405793in}}%
\pgfpathlineto{\pgfqpoint{2.583887in}{0.405595in}}%
\pgfpathlineto{\pgfqpoint{2.609659in}{0.405437in}}%
\pgfpathlineto{\pgfqpoint{2.620398in}{0.405125in}}%
\pgfpathlineto{\pgfqpoint{2.628989in}{0.404981in}}%
\pgfpathlineto{\pgfqpoint{2.681607in}{0.404476in}}%
\pgfpathlineto{\pgfqpoint{2.689124in}{0.404498in}}%
\pgfpathlineto{\pgfqpoint{2.786843in}{0.403649in}}%
\pgfpathlineto{\pgfqpoint{2.786843in}{0.403649in}}%
\pgfusepath{stroke}%
\end{pgfscope}%
\begin{pgfscope}%
\pgfsetrectcap%
\pgfsetmiterjoin%
\pgfsetlinewidth{0.803000pt}%
\definecolor{currentstroke}{rgb}{1.000000,1.000000,1.000000}%
\pgfsetstrokecolor{currentstroke}%
\pgfsetdash{}{0pt}%
\pgfpathmoveto{\pgfqpoint{0.320934in}{0.385400in}}%
\pgfpathlineto{\pgfqpoint{0.320934in}{0.786285in}}%
\pgfusepath{stroke}%
\end{pgfscope}%
\begin{pgfscope}%
\pgfsetrectcap%
\pgfsetmiterjoin%
\pgfsetlinewidth{0.803000pt}%
\definecolor{currentstroke}{rgb}{1.000000,1.000000,1.000000}%
\pgfsetstrokecolor{currentstroke}%
\pgfsetdash{}{0pt}%
\pgfpathmoveto{\pgfqpoint{2.904267in}{0.385400in}}%
\pgfpathlineto{\pgfqpoint{2.904267in}{0.786285in}}%
\pgfusepath{stroke}%
\end{pgfscope}%
\begin{pgfscope}%
\pgfsetrectcap%
\pgfsetmiterjoin%
\pgfsetlinewidth{0.803000pt}%
\definecolor{currentstroke}{rgb}{1.000000,1.000000,1.000000}%
\pgfsetstrokecolor{currentstroke}%
\pgfsetdash{}{0pt}%
\pgfpathmoveto{\pgfqpoint{0.320934in}{0.385400in}}%
\pgfpathlineto{\pgfqpoint{2.904267in}{0.385400in}}%
\pgfusepath{stroke}%
\end{pgfscope}%
\begin{pgfscope}%
\pgfsetrectcap%
\pgfsetmiterjoin%
\pgfsetlinewidth{0.803000pt}%
\definecolor{currentstroke}{rgb}{1.000000,1.000000,1.000000}%
\pgfsetstrokecolor{currentstroke}%
\pgfsetdash{}{0pt}%
\pgfpathmoveto{\pgfqpoint{0.320934in}{0.786285in}}%
\pgfpathlineto{\pgfqpoint{2.904267in}{0.786285in}}%
\pgfusepath{stroke}%
\end{pgfscope}%
\begin{pgfscope}%
\definecolor{textcolor}{rgb}{0.150000,0.150000,0.150000}%
\pgfsetstrokecolor{textcolor}%
\pgfsetfillcolor{textcolor}%
\pgftext[x=1.612600in,y=0.869619in,,base]{\color{textcolor}\rmfamily\fontsize{16.800000}{20.160000}\selectfont V}%
\end{pgfscope}%
\begin{pgfscope}%
\pgfsetbuttcap%
\pgfsetmiterjoin%
\definecolor{currentfill}{rgb}{0.917647,0.917647,0.949020}%
\pgfsetfillcolor{currentfill}%
\pgfsetlinewidth{0.000000pt}%
\definecolor{currentstroke}{rgb}{0.000000,0.000000,0.000000}%
\pgfsetstrokecolor{currentstroke}%
\pgfsetstrokeopacity{0.000000}%
\pgfsetdash{}{0pt}%
\pgfpathmoveto{\pgfqpoint{3.937600in}{0.385400in}}%
\pgfpathlineto{\pgfqpoint{6.520934in}{0.385400in}}%
\pgfpathlineto{\pgfqpoint{6.520934in}{0.786285in}}%
\pgfpathlineto{\pgfqpoint{3.937600in}{0.786285in}}%
\pgfpathclose%
\pgfusepath{fill}%
\end{pgfscope}%
\begin{pgfscope}%
\pgfpathrectangle{\pgfqpoint{3.937600in}{0.385400in}}{\pgfqpoint{2.583333in}{0.400885in}}%
\pgfusepath{clip}%
\pgfsetroundcap%
\pgfsetroundjoin%
\pgfsetlinewidth{0.803000pt}%
\definecolor{currentstroke}{rgb}{1.000000,1.000000,1.000000}%
\pgfsetstrokecolor{currentstroke}%
\pgfsetdash{}{0pt}%
\pgfpathmoveto{\pgfqpoint{4.052877in}{0.385400in}}%
\pgfpathlineto{\pgfqpoint{4.052877in}{0.786285in}}%
\pgfusepath{stroke}%
\end{pgfscope}%
\begin{pgfscope}%
\definecolor{textcolor}{rgb}{0.150000,0.150000,0.150000}%
\pgfsetstrokecolor{textcolor}%
\pgfsetfillcolor{textcolor}%
\pgftext[x=4.052877in,y=0.288178in,,top]{\color{textcolor}\rmfamily\fontsize{14.000000}{16.800000}\selectfont 2012}%
\end{pgfscope}%
\begin{pgfscope}%
\pgfpathrectangle{\pgfqpoint{3.937600in}{0.385400in}}{\pgfqpoint{2.583333in}{0.400885in}}%
\pgfusepath{clip}%
\pgfsetroundcap%
\pgfsetroundjoin%
\pgfsetlinewidth{0.803000pt}%
\definecolor{currentstroke}{rgb}{1.000000,1.000000,1.000000}%
\pgfsetstrokecolor{currentstroke}%
\pgfsetdash{}{0pt}%
\pgfpathmoveto{\pgfqpoint{4.445902in}{0.385400in}}%
\pgfpathlineto{\pgfqpoint{4.445902in}{0.786285in}}%
\pgfusepath{stroke}%
\end{pgfscope}%
\begin{pgfscope}%
\definecolor{textcolor}{rgb}{0.150000,0.150000,0.150000}%
\pgfsetstrokecolor{textcolor}%
\pgfsetfillcolor{textcolor}%
\pgftext[x=4.445902in,y=0.288178in,,top]{\color{textcolor}\rmfamily\fontsize{14.000000}{16.800000}\selectfont 2013}%
\end{pgfscope}%
\begin{pgfscope}%
\pgfpathrectangle{\pgfqpoint{3.937600in}{0.385400in}}{\pgfqpoint{2.583333in}{0.400885in}}%
\pgfusepath{clip}%
\pgfsetroundcap%
\pgfsetroundjoin%
\pgfsetlinewidth{0.803000pt}%
\definecolor{currentstroke}{rgb}{1.000000,1.000000,1.000000}%
\pgfsetstrokecolor{currentstroke}%
\pgfsetdash{}{0pt}%
\pgfpathmoveto{\pgfqpoint{4.837853in}{0.385400in}}%
\pgfpathlineto{\pgfqpoint{4.837853in}{0.786285in}}%
\pgfusepath{stroke}%
\end{pgfscope}%
\begin{pgfscope}%
\definecolor{textcolor}{rgb}{0.150000,0.150000,0.150000}%
\pgfsetstrokecolor{textcolor}%
\pgfsetfillcolor{textcolor}%
\pgftext[x=4.837853in,y=0.288178in,,top]{\color{textcolor}\rmfamily\fontsize{14.000000}{16.800000}\selectfont 2014}%
\end{pgfscope}%
\begin{pgfscope}%
\pgfpathrectangle{\pgfqpoint{3.937600in}{0.385400in}}{\pgfqpoint{2.583333in}{0.400885in}}%
\pgfusepath{clip}%
\pgfsetroundcap%
\pgfsetroundjoin%
\pgfsetlinewidth{0.803000pt}%
\definecolor{currentstroke}{rgb}{1.000000,1.000000,1.000000}%
\pgfsetstrokecolor{currentstroke}%
\pgfsetdash{}{0pt}%
\pgfpathmoveto{\pgfqpoint{5.229804in}{0.385400in}}%
\pgfpathlineto{\pgfqpoint{5.229804in}{0.786285in}}%
\pgfusepath{stroke}%
\end{pgfscope}%
\begin{pgfscope}%
\definecolor{textcolor}{rgb}{0.150000,0.150000,0.150000}%
\pgfsetstrokecolor{textcolor}%
\pgfsetfillcolor{textcolor}%
\pgftext[x=5.229804in,y=0.288178in,,top]{\color{textcolor}\rmfamily\fontsize{14.000000}{16.800000}\selectfont 2015}%
\end{pgfscope}%
\begin{pgfscope}%
\pgfpathrectangle{\pgfqpoint{3.937600in}{0.385400in}}{\pgfqpoint{2.583333in}{0.400885in}}%
\pgfusepath{clip}%
\pgfsetroundcap%
\pgfsetroundjoin%
\pgfsetlinewidth{0.803000pt}%
\definecolor{currentstroke}{rgb}{1.000000,1.000000,1.000000}%
\pgfsetstrokecolor{currentstroke}%
\pgfsetdash{}{0pt}%
\pgfpathmoveto{\pgfqpoint{5.621755in}{0.385400in}}%
\pgfpathlineto{\pgfqpoint{5.621755in}{0.786285in}}%
\pgfusepath{stroke}%
\end{pgfscope}%
\begin{pgfscope}%
\definecolor{textcolor}{rgb}{0.150000,0.150000,0.150000}%
\pgfsetstrokecolor{textcolor}%
\pgfsetfillcolor{textcolor}%
\pgftext[x=5.621755in,y=0.288178in,,top]{\color{textcolor}\rmfamily\fontsize{14.000000}{16.800000}\selectfont 2016}%
\end{pgfscope}%
\begin{pgfscope}%
\pgfpathrectangle{\pgfqpoint{3.937600in}{0.385400in}}{\pgfqpoint{2.583333in}{0.400885in}}%
\pgfusepath{clip}%
\pgfsetroundcap%
\pgfsetroundjoin%
\pgfsetlinewidth{0.803000pt}%
\definecolor{currentstroke}{rgb}{1.000000,1.000000,1.000000}%
\pgfsetstrokecolor{currentstroke}%
\pgfsetdash{}{0pt}%
\pgfpathmoveto{\pgfqpoint{6.014780in}{0.385400in}}%
\pgfpathlineto{\pgfqpoint{6.014780in}{0.786285in}}%
\pgfusepath{stroke}%
\end{pgfscope}%
\begin{pgfscope}%
\definecolor{textcolor}{rgb}{0.150000,0.150000,0.150000}%
\pgfsetstrokecolor{textcolor}%
\pgfsetfillcolor{textcolor}%
\pgftext[x=6.014780in,y=0.288178in,,top]{\color{textcolor}\rmfamily\fontsize{14.000000}{16.800000}\selectfont 2017}%
\end{pgfscope}%
\begin{pgfscope}%
\pgfpathrectangle{\pgfqpoint{3.937600in}{0.385400in}}{\pgfqpoint{2.583333in}{0.400885in}}%
\pgfusepath{clip}%
\pgfsetroundcap%
\pgfsetroundjoin%
\pgfsetlinewidth{0.803000pt}%
\definecolor{currentstroke}{rgb}{1.000000,1.000000,1.000000}%
\pgfsetstrokecolor{currentstroke}%
\pgfsetdash{}{0pt}%
\pgfpathmoveto{\pgfqpoint{6.406731in}{0.385400in}}%
\pgfpathlineto{\pgfqpoint{6.406731in}{0.786285in}}%
\pgfusepath{stroke}%
\end{pgfscope}%
\begin{pgfscope}%
\definecolor{textcolor}{rgb}{0.150000,0.150000,0.150000}%
\pgfsetstrokecolor{textcolor}%
\pgfsetfillcolor{textcolor}%
\pgftext[x=6.406731in,y=0.288178in,,top]{\color{textcolor}\rmfamily\fontsize{14.000000}{16.800000}\selectfont 2018}%
\end{pgfscope}%
\begin{pgfscope}%
\pgfpathrectangle{\pgfqpoint{3.937600in}{0.385400in}}{\pgfqpoint{2.583333in}{0.400885in}}%
\pgfusepath{clip}%
\pgfsetroundcap%
\pgfsetroundjoin%
\pgfsetlinewidth{0.803000pt}%
\definecolor{currentstroke}{rgb}{1.000000,1.000000,1.000000}%
\pgfsetstrokecolor{currentstroke}%
\pgfsetdash{}{0pt}%
\pgfpathmoveto{\pgfqpoint{3.937600in}{0.495511in}}%
\pgfpathlineto{\pgfqpoint{6.520934in}{0.495511in}}%
\pgfusepath{stroke}%
\end{pgfscope}%
\begin{pgfscope}%
\definecolor{textcolor}{rgb}{0.150000,0.150000,0.150000}%
\pgfsetstrokecolor{textcolor}%
\pgfsetfillcolor{textcolor}%
\pgftext[x=3.716667in,y=0.421645in,left,base]{\color{textcolor}\rmfamily\fontsize{14.000000}{16.800000}\selectfont 1}%
\end{pgfscope}%
\begin{pgfscope}%
\pgfpathrectangle{\pgfqpoint{3.937600in}{0.385400in}}{\pgfqpoint{2.583333in}{0.400885in}}%
\pgfusepath{clip}%
\pgfsetroundcap%
\pgfsetroundjoin%
\pgfsetlinewidth{0.803000pt}%
\definecolor{currentstroke}{rgb}{1.000000,1.000000,1.000000}%
\pgfsetstrokecolor{currentstroke}%
\pgfsetdash{}{0pt}%
\pgfpathmoveto{\pgfqpoint{3.937600in}{0.612717in}}%
\pgfpathlineto{\pgfqpoint{6.520934in}{0.612717in}}%
\pgfusepath{stroke}%
\end{pgfscope}%
\begin{pgfscope}%
\definecolor{textcolor}{rgb}{0.150000,0.150000,0.150000}%
\pgfsetstrokecolor{textcolor}%
\pgfsetfillcolor{textcolor}%
\pgftext[x=3.716667in,y=0.538851in,left,base]{\color{textcolor}\rmfamily\fontsize{14.000000}{16.800000}\selectfont 2}%
\end{pgfscope}%
\begin{pgfscope}%
\pgfpathrectangle{\pgfqpoint{3.937600in}{0.385400in}}{\pgfqpoint{2.583333in}{0.400885in}}%
\pgfusepath{clip}%
\pgfsetroundcap%
\pgfsetroundjoin%
\pgfsetlinewidth{0.803000pt}%
\definecolor{currentstroke}{rgb}{1.000000,1.000000,1.000000}%
\pgfsetstrokecolor{currentstroke}%
\pgfsetdash{}{0pt}%
\pgfpathmoveto{\pgfqpoint{3.937600in}{0.729923in}}%
\pgfpathlineto{\pgfqpoint{6.520934in}{0.729923in}}%
\pgfusepath{stroke}%
\end{pgfscope}%
\begin{pgfscope}%
\definecolor{textcolor}{rgb}{0.150000,0.150000,0.150000}%
\pgfsetstrokecolor{textcolor}%
\pgfsetfillcolor{textcolor}%
\pgftext[x=3.716667in,y=0.656057in,left,base]{\color{textcolor}\rmfamily\fontsize{14.000000}{16.800000}\selectfont 3}%
\end{pgfscope}%
\begin{pgfscope}%
\pgfpathrectangle{\pgfqpoint{3.937600in}{0.385400in}}{\pgfqpoint{2.583333in}{0.400885in}}%
\pgfusepath{clip}%
\pgfsetroundcap%
\pgfsetroundjoin%
\pgfsetlinewidth{1.505625pt}%
\definecolor{currentstroke}{rgb}{0.000000,0.000000,0.000000}%
\pgfsetstrokecolor{currentstroke}%
\pgfsetdash{}{0pt}%
\pgfpathmoveto{\pgfqpoint{4.055025in}{0.495511in}}%
\pgfpathlineto{\pgfqpoint{4.058246in}{0.500402in}}%
\pgfpathlineto{\pgfqpoint{4.062542in}{0.499552in}}%
\pgfpathlineto{\pgfqpoint{4.063615in}{0.496700in}}%
\pgfpathlineto{\pgfqpoint{4.064689in}{0.496768in}}%
\pgfpathlineto{\pgfqpoint{4.065763in}{0.495783in}}%
\pgfpathlineto{\pgfqpoint{4.070058in}{0.496020in}}%
\pgfpathlineto{\pgfqpoint{4.072206in}{0.498941in}}%
\pgfpathlineto{\pgfqpoint{4.073280in}{0.498568in}}%
\pgfpathlineto{\pgfqpoint{4.077575in}{0.498364in}}%
\pgfpathlineto{\pgfqpoint{4.078649in}{0.499315in}}%
\pgfpathlineto{\pgfqpoint{4.080797in}{0.498364in}}%
\pgfpathlineto{\pgfqpoint{4.085092in}{0.497311in}}%
\pgfpathlineto{\pgfqpoint{4.086166in}{0.498602in}}%
\pgfpathlineto{\pgfqpoint{4.087240in}{0.497345in}}%
\pgfpathlineto{\pgfqpoint{4.088314in}{0.500673in}}%
\pgfpathlineto{\pgfqpoint{4.091535in}{0.502066in}}%
\pgfpathlineto{\pgfqpoint{4.093683in}{0.504545in}}%
\pgfpathlineto{\pgfqpoint{4.094757in}{0.505360in}}%
\pgfpathlineto{\pgfqpoint{4.095831in}{0.505088in}}%
\pgfpathlineto{\pgfqpoint{4.099052in}{0.506141in}}%
\pgfpathlineto{\pgfqpoint{4.101200in}{0.504477in}}%
\pgfpathlineto{\pgfqpoint{4.103347in}{0.506005in}}%
\pgfpathlineto{\pgfqpoint{4.107643in}{0.505462in}}%
\pgfpathlineto{\pgfqpoint{4.108717in}{0.504545in}}%
\pgfpathlineto{\pgfqpoint{4.109790in}{0.505190in}}%
\pgfpathlineto{\pgfqpoint{4.110864in}{0.504681in}}%
\pgfpathlineto{\pgfqpoint{4.116233in}{0.506753in}}%
\pgfpathlineto{\pgfqpoint{4.117307in}{0.507975in}}%
\pgfpathlineto{\pgfqpoint{4.118381in}{0.507873in}}%
\pgfpathlineto{\pgfqpoint{4.121603in}{0.508926in}}%
\pgfpathlineto{\pgfqpoint{4.122676in}{0.506787in}}%
\pgfpathlineto{\pgfqpoint{4.123750in}{0.506005in}}%
\pgfpathlineto{\pgfqpoint{4.125898in}{0.507534in}}%
\pgfpathlineto{\pgfqpoint{4.129120in}{0.507839in}}%
\pgfpathlineto{\pgfqpoint{4.130193in}{0.512934in}}%
\pgfpathlineto{\pgfqpoint{4.131267in}{0.511304in}}%
\pgfpathlineto{\pgfqpoint{4.132341in}{0.511270in}}%
\pgfpathlineto{\pgfqpoint{4.133415in}{0.510421in}}%
\pgfpathlineto{\pgfqpoint{4.136636in}{0.511202in}}%
\pgfpathlineto{\pgfqpoint{4.137710in}{0.510590in}}%
\pgfpathlineto{\pgfqpoint{4.139858in}{0.510726in}}%
\pgfpathlineto{\pgfqpoint{4.140932in}{0.511847in}}%
\pgfpathlineto{\pgfqpoint{4.144153in}{0.514055in}}%
\pgfpathlineto{\pgfqpoint{4.145227in}{0.513375in}}%
\pgfpathlineto{\pgfqpoint{4.147375in}{0.509877in}}%
\pgfpathlineto{\pgfqpoint{4.148449in}{0.512221in}}%
\pgfpathlineto{\pgfqpoint{4.151670in}{0.512424in}}%
\pgfpathlineto{\pgfqpoint{4.153818in}{0.509639in}}%
\pgfpathlineto{\pgfqpoint{4.154892in}{0.510081in}}%
\pgfpathlineto{\pgfqpoint{4.159187in}{0.507126in}}%
\pgfpathlineto{\pgfqpoint{4.160261in}{0.503696in}}%
\pgfpathlineto{\pgfqpoint{4.161335in}{0.504851in}}%
\pgfpathlineto{\pgfqpoint{4.162409in}{0.507228in}}%
\pgfpathlineto{\pgfqpoint{4.163482in}{0.506311in}}%
\pgfpathlineto{\pgfqpoint{4.166704in}{0.505734in}}%
\pgfpathlineto{\pgfqpoint{4.167778in}{0.508858in}}%
\pgfpathlineto{\pgfqpoint{4.168852in}{0.508281in}}%
\pgfpathlineto{\pgfqpoint{4.169925in}{0.507024in}}%
\pgfpathlineto{\pgfqpoint{4.170999in}{0.507839in}}%
\pgfpathlineto{\pgfqpoint{4.174221in}{0.506821in}}%
\pgfpathlineto{\pgfqpoint{4.175295in}{0.507330in}}%
\pgfpathlineto{\pgfqpoint{4.177442in}{0.510930in}}%
\pgfpathlineto{\pgfqpoint{4.178516in}{0.510930in}}%
\pgfpathlineto{\pgfqpoint{4.181738in}{0.510183in}}%
\pgfpathlineto{\pgfqpoint{4.182811in}{0.512255in}}%
\pgfpathlineto{\pgfqpoint{4.183885in}{0.511507in}}%
\pgfpathlineto{\pgfqpoint{4.184959in}{0.512323in}}%
\pgfpathlineto{\pgfqpoint{4.186033in}{0.509639in}}%
\pgfpathlineto{\pgfqpoint{4.190328in}{0.513817in}}%
\pgfpathlineto{\pgfqpoint{4.191402in}{0.516024in}}%
\pgfpathlineto{\pgfqpoint{4.193550in}{0.517689in}}%
\pgfpathlineto{\pgfqpoint{4.198919in}{0.516194in}}%
\pgfpathlineto{\pgfqpoint{4.201067in}{0.512323in}}%
\pgfpathlineto{\pgfqpoint{4.205362in}{0.514089in}}%
\pgfpathlineto{\pgfqpoint{4.206436in}{0.513545in}}%
\pgfpathlineto{\pgfqpoint{4.212879in}{0.517417in}}%
\pgfpathlineto{\pgfqpoint{4.213953in}{0.516568in}}%
\pgfpathlineto{\pgfqpoint{4.215027in}{0.518130in}}%
\pgfpathlineto{\pgfqpoint{4.216100in}{0.514123in}}%
\pgfpathlineto{\pgfqpoint{4.219322in}{0.514157in}}%
\pgfpathlineto{\pgfqpoint{4.223617in}{0.519760in}}%
\pgfpathlineto{\pgfqpoint{4.226839in}{0.518402in}}%
\pgfpathlineto{\pgfqpoint{4.227913in}{0.520168in}}%
\pgfpathlineto{\pgfqpoint{4.228986in}{0.519726in}}%
\pgfpathlineto{\pgfqpoint{4.230060in}{0.522647in}}%
\pgfpathlineto{\pgfqpoint{4.231134in}{0.522342in}}%
\pgfpathlineto{\pgfqpoint{4.234356in}{0.522376in}}%
\pgfpathlineto{\pgfqpoint{4.236503in}{0.524311in}}%
\pgfpathlineto{\pgfqpoint{4.237577in}{0.523293in}}%
\pgfpathlineto{\pgfqpoint{4.238651in}{0.523530in}}%
\pgfpathlineto{\pgfqpoint{4.241873in}{0.521153in}}%
\pgfpathlineto{\pgfqpoint{4.244020in}{0.524753in}}%
\pgfpathlineto{\pgfqpoint{4.245094in}{0.524583in}}%
\pgfpathlineto{\pgfqpoint{4.246168in}{0.526655in}}%
\pgfpathlineto{\pgfqpoint{4.249389in}{0.527334in}}%
\pgfpathlineto{\pgfqpoint{4.250463in}{0.526961in}}%
\pgfpathlineto{\pgfqpoint{4.253685in}{0.525262in}}%
\pgfpathlineto{\pgfqpoint{4.256906in}{0.525127in}}%
\pgfpathlineto{\pgfqpoint{4.257980in}{0.523191in}}%
\pgfpathlineto{\pgfqpoint{4.259054in}{0.522919in}}%
\pgfpathlineto{\pgfqpoint{4.260128in}{0.523327in}}%
\pgfpathlineto{\pgfqpoint{4.261202in}{0.525738in}}%
\pgfpathlineto{\pgfqpoint{4.264423in}{0.524719in}}%
\pgfpathlineto{\pgfqpoint{4.265497in}{0.529270in}}%
\pgfpathlineto{\pgfqpoint{4.266571in}{0.529270in}}%
\pgfpathlineto{\pgfqpoint{4.268719in}{0.526961in}}%
\pgfpathlineto{\pgfqpoint{4.271940in}{0.525093in}}%
\pgfpathlineto{\pgfqpoint{4.274088in}{0.526044in}}%
\pgfpathlineto{\pgfqpoint{4.275162in}{0.530357in}}%
\pgfpathlineto{\pgfqpoint{4.276235in}{0.531070in}}%
\pgfpathlineto{\pgfqpoint{4.279457in}{0.530663in}}%
\pgfpathlineto{\pgfqpoint{4.281605in}{0.527640in}}%
\pgfpathlineto{\pgfqpoint{4.282678in}{0.528149in}}%
\pgfpathlineto{\pgfqpoint{4.283752in}{0.530561in}}%
\pgfpathlineto{\pgfqpoint{4.286974in}{0.530187in}}%
\pgfpathlineto{\pgfqpoint{4.288048in}{0.530663in}}%
\pgfpathlineto{\pgfqpoint{4.289121in}{0.532768in}}%
\pgfpathlineto{\pgfqpoint{4.291269in}{0.530187in}}%
\pgfpathlineto{\pgfqpoint{4.294491in}{0.530832in}}%
\pgfpathlineto{\pgfqpoint{4.295564in}{0.530323in}}%
\pgfpathlineto{\pgfqpoint{4.298786in}{0.532666in}}%
\pgfpathlineto{\pgfqpoint{4.302008in}{0.532632in}}%
\pgfpathlineto{\pgfqpoint{4.303081in}{0.530153in}}%
\pgfpathlineto{\pgfqpoint{4.304155in}{0.530221in}}%
\pgfpathlineto{\pgfqpoint{4.305229in}{0.528727in}}%
\pgfpathlineto{\pgfqpoint{4.306303in}{0.529915in}}%
\pgfpathlineto{\pgfqpoint{4.310598in}{0.530119in}}%
\pgfpathlineto{\pgfqpoint{4.311672in}{0.531376in}}%
\pgfpathlineto{\pgfqpoint{4.312746in}{0.529474in}}%
\pgfpathlineto{\pgfqpoint{4.318115in}{0.530221in}}%
\pgfpathlineto{\pgfqpoint{4.320263in}{0.536946in}}%
\pgfpathlineto{\pgfqpoint{4.324558in}{0.535893in}}%
\pgfpathlineto{\pgfqpoint{4.326706in}{0.536470in}}%
\pgfpathlineto{\pgfqpoint{4.327780in}{0.539221in}}%
\pgfpathlineto{\pgfqpoint{4.328853in}{0.538440in}}%
\pgfpathlineto{\pgfqpoint{4.333149in}{0.537081in}}%
\pgfpathlineto{\pgfqpoint{4.334223in}{0.539527in}}%
\pgfpathlineto{\pgfqpoint{4.336370in}{0.539629in}}%
\pgfpathlineto{\pgfqpoint{4.339592in}{0.540206in}}%
\pgfpathlineto{\pgfqpoint{4.341740in}{0.537183in}}%
\pgfpathlineto{\pgfqpoint{4.342813in}{0.539153in}}%
\pgfpathlineto{\pgfqpoint{4.343887in}{0.538236in}}%
\pgfpathlineto{\pgfqpoint{4.347109in}{0.537591in}}%
\pgfpathlineto{\pgfqpoint{4.348183in}{0.536266in}}%
\pgfpathlineto{\pgfqpoint{4.349256in}{0.538814in}}%
\pgfpathlineto{\pgfqpoint{4.351404in}{0.540342in}}%
\pgfpathlineto{\pgfqpoint{4.354626in}{0.538372in}}%
\pgfpathlineto{\pgfqpoint{4.355699in}{0.535893in}}%
\pgfpathlineto{\pgfqpoint{4.356773in}{0.534976in}}%
\pgfpathlineto{\pgfqpoint{4.357847in}{0.532293in}}%
\pgfpathlineto{\pgfqpoint{4.358921in}{0.533074in}}%
\pgfpathlineto{\pgfqpoint{4.362142in}{0.533685in}}%
\pgfpathlineto{\pgfqpoint{4.363216in}{0.535044in}}%
\pgfpathlineto{\pgfqpoint{4.364290in}{0.538270in}}%
\pgfpathlineto{\pgfqpoint{4.365364in}{0.538678in}}%
\pgfpathlineto{\pgfqpoint{4.366438in}{0.537081in}}%
\pgfpathlineto{\pgfqpoint{4.369659in}{0.536742in}}%
\pgfpathlineto{\pgfqpoint{4.370733in}{0.533583in}}%
\pgfpathlineto{\pgfqpoint{4.371807in}{0.533244in}}%
\pgfpathlineto{\pgfqpoint{4.373955in}{0.531512in}}%
\pgfpathlineto{\pgfqpoint{4.379324in}{0.528557in}}%
\pgfpathlineto{\pgfqpoint{4.380398in}{0.530595in}}%
\pgfpathlineto{\pgfqpoint{4.381472in}{0.530832in}}%
\pgfpathlineto{\pgfqpoint{4.385767in}{0.532700in}}%
\pgfpathlineto{\pgfqpoint{4.386841in}{0.531512in}}%
\pgfpathlineto{\pgfqpoint{4.387915in}{0.531376in}}%
\pgfpathlineto{\pgfqpoint{4.388988in}{0.522274in}}%
\pgfpathlineto{\pgfqpoint{4.392210in}{0.523462in}}%
\pgfpathlineto{\pgfqpoint{4.393284in}{0.525025in}}%
\pgfpathlineto{\pgfqpoint{4.394358in}{0.522613in}}%
\pgfpathlineto{\pgfqpoint{4.395431in}{0.523530in}}%
\pgfpathlineto{\pgfqpoint{4.396505in}{0.523360in}}%
\pgfpathlineto{\pgfqpoint{4.399727in}{0.524855in}}%
\pgfpathlineto{\pgfqpoint{4.401875in}{0.527232in}}%
\pgfpathlineto{\pgfqpoint{4.404022in}{0.528998in}}%
\pgfpathlineto{\pgfqpoint{4.407244in}{0.528285in}}%
\pgfpathlineto{\pgfqpoint{4.408318in}{0.526961in}}%
\pgfpathlineto{\pgfqpoint{4.410465in}{0.530391in}}%
\pgfpathlineto{\pgfqpoint{4.411539in}{0.530221in}}%
\pgfpathlineto{\pgfqpoint{4.415834in}{0.529100in}}%
\pgfpathlineto{\pgfqpoint{4.419056in}{0.531240in}}%
\pgfpathlineto{\pgfqpoint{4.423351in}{0.531987in}}%
\pgfpathlineto{\pgfqpoint{4.424425in}{0.532497in}}%
\pgfpathlineto{\pgfqpoint{4.426573in}{0.529474in}}%
\pgfpathlineto{\pgfqpoint{4.429794in}{0.531376in}}%
\pgfpathlineto{\pgfqpoint{4.430868in}{0.534297in}}%
\pgfpathlineto{\pgfqpoint{4.431942in}{0.533414in}}%
\pgfpathlineto{\pgfqpoint{4.433016in}{0.536504in}}%
\pgfpathlineto{\pgfqpoint{4.434090in}{0.533617in}}%
\pgfpathlineto{\pgfqpoint{4.439459in}{0.533142in}}%
\pgfpathlineto{\pgfqpoint{4.441607in}{0.530968in}}%
\pgfpathlineto{\pgfqpoint{4.444828in}{0.532938in}}%
\pgfpathlineto{\pgfqpoint{4.446976in}{0.537014in}}%
\pgfpathlineto{\pgfqpoint{4.448050in}{0.537353in}}%
\pgfpathlineto{\pgfqpoint{4.449123in}{0.540410in}}%
\pgfpathlineto{\pgfqpoint{4.453419in}{0.535961in}}%
\pgfpathlineto{\pgfqpoint{4.455566in}{0.536063in}}%
\pgfpathlineto{\pgfqpoint{4.456640in}{0.535417in}}%
\pgfpathlineto{\pgfqpoint{4.459862in}{0.535451in}}%
\pgfpathlineto{\pgfqpoint{4.462009in}{0.538372in}}%
\pgfpathlineto{\pgfqpoint{4.463083in}{0.541089in}}%
\pgfpathlineto{\pgfqpoint{4.464157in}{0.540885in}}%
\pgfpathlineto{\pgfqpoint{4.468452in}{0.542074in}}%
\pgfpathlineto{\pgfqpoint{4.469526in}{0.545878in}}%
\pgfpathlineto{\pgfqpoint{4.470600in}{0.545878in}}%
\pgfpathlineto{\pgfqpoint{4.471674in}{0.547202in}}%
\pgfpathlineto{\pgfqpoint{4.474896in}{0.547135in}}%
\pgfpathlineto{\pgfqpoint{4.477043in}{0.545368in}}%
\pgfpathlineto{\pgfqpoint{4.478117in}{0.545674in}}%
\pgfpathlineto{\pgfqpoint{4.479191in}{0.547848in}}%
\pgfpathlineto{\pgfqpoint{4.482412in}{0.545708in}}%
\pgfpathlineto{\pgfqpoint{4.484560in}{0.547644in}}%
\pgfpathlineto{\pgfqpoint{4.485634in}{0.547135in}}%
\pgfpathlineto{\pgfqpoint{4.486708in}{0.548085in}}%
\pgfpathlineto{\pgfqpoint{4.493151in}{0.548765in}}%
\pgfpathlineto{\pgfqpoint{4.494225in}{0.551040in}}%
\pgfpathlineto{\pgfqpoint{4.498520in}{0.551414in}}%
\pgfpathlineto{\pgfqpoint{4.499594in}{0.547882in}}%
\pgfpathlineto{\pgfqpoint{4.500668in}{0.546557in}}%
\pgfpathlineto{\pgfqpoint{4.501741in}{0.546795in}}%
\pgfpathlineto{\pgfqpoint{4.504963in}{0.544757in}}%
\pgfpathlineto{\pgfqpoint{4.506037in}{0.545708in}}%
\pgfpathlineto{\pgfqpoint{4.507111in}{0.547508in}}%
\pgfpathlineto{\pgfqpoint{4.508185in}{0.547848in}}%
\pgfpathlineto{\pgfqpoint{4.509258in}{0.550157in}}%
\pgfpathlineto{\pgfqpoint{4.512480in}{0.551618in}}%
\pgfpathlineto{\pgfqpoint{4.513554in}{0.553723in}}%
\pgfpathlineto{\pgfqpoint{4.515701in}{0.553248in}}%
\pgfpathlineto{\pgfqpoint{4.516775in}{0.556576in}}%
\pgfpathlineto{\pgfqpoint{4.519997in}{0.557391in}}%
\pgfpathlineto{\pgfqpoint{4.521071in}{0.555693in}}%
\pgfpathlineto{\pgfqpoint{4.523218in}{0.557663in}}%
\pgfpathlineto{\pgfqpoint{4.524292in}{0.557154in}}%
\pgfpathlineto{\pgfqpoint{4.527514in}{0.554810in}}%
\pgfpathlineto{\pgfqpoint{4.528587in}{0.553214in}}%
\pgfpathlineto{\pgfqpoint{4.529661in}{0.555150in}}%
\pgfpathlineto{\pgfqpoint{4.530735in}{0.553214in}}%
\pgfpathlineto{\pgfqpoint{4.531809in}{0.554674in}}%
\pgfpathlineto{\pgfqpoint{4.535030in}{0.552908in}}%
\pgfpathlineto{\pgfqpoint{4.536104in}{0.554199in}}%
\pgfpathlineto{\pgfqpoint{4.537178in}{0.553689in}}%
\pgfpathlineto{\pgfqpoint{4.538252in}{0.554742in}}%
\pgfpathlineto{\pgfqpoint{4.542547in}{0.554403in}}%
\pgfpathlineto{\pgfqpoint{4.543621in}{0.556780in}}%
\pgfpathlineto{\pgfqpoint{4.544695in}{0.556135in}}%
\pgfpathlineto{\pgfqpoint{4.551138in}{0.562010in}}%
\pgfpathlineto{\pgfqpoint{4.553286in}{0.566391in}}%
\pgfpathlineto{\pgfqpoint{4.554360in}{0.566391in}}%
\pgfpathlineto{\pgfqpoint{4.557581in}{0.561195in}}%
\pgfpathlineto{\pgfqpoint{4.558655in}{0.567003in}}%
\pgfpathlineto{\pgfqpoint{4.559729in}{0.566765in}}%
\pgfpathlineto{\pgfqpoint{4.560803in}{0.564625in}}%
\pgfpathlineto{\pgfqpoint{4.561876in}{0.569516in}}%
\pgfpathlineto{\pgfqpoint{4.565098in}{0.570909in}}%
\pgfpathlineto{\pgfqpoint{4.566172in}{0.572709in}}%
\pgfpathlineto{\pgfqpoint{4.567246in}{0.570705in}}%
\pgfpathlineto{\pgfqpoint{4.568319in}{0.570875in}}%
\pgfpathlineto{\pgfqpoint{4.569393in}{0.570467in}}%
\pgfpathlineto{\pgfqpoint{4.572615in}{0.573999in}}%
\pgfpathlineto{\pgfqpoint{4.573689in}{0.573490in}}%
\pgfpathlineto{\pgfqpoint{4.574763in}{0.574644in}}%
\pgfpathlineto{\pgfqpoint{4.576910in}{0.579569in}}%
\pgfpathlineto{\pgfqpoint{4.580132in}{0.580384in}}%
\pgfpathlineto{\pgfqpoint{4.581206in}{0.583509in}}%
\pgfpathlineto{\pgfqpoint{4.582279in}{0.583271in}}%
\pgfpathlineto{\pgfqpoint{4.584427in}{0.587041in}}%
\pgfpathlineto{\pgfqpoint{4.588722in}{0.587856in}}%
\pgfpathlineto{\pgfqpoint{4.589796in}{0.588501in}}%
\pgfpathlineto{\pgfqpoint{4.590870in}{0.584765in}}%
\pgfpathlineto{\pgfqpoint{4.591944in}{0.585105in}}%
\pgfpathlineto{\pgfqpoint{4.596239in}{0.582762in}}%
\pgfpathlineto{\pgfqpoint{4.598387in}{0.580928in}}%
\pgfpathlineto{\pgfqpoint{4.603756in}{0.585445in}}%
\pgfpathlineto{\pgfqpoint{4.604830in}{0.584120in}}%
\pgfpathlineto{\pgfqpoint{4.606978in}{0.574237in}}%
\pgfpathlineto{\pgfqpoint{4.610199in}{0.576478in}}%
\pgfpathlineto{\pgfqpoint{4.611273in}{0.578177in}}%
\pgfpathlineto{\pgfqpoint{4.612347in}{0.574373in}}%
\pgfpathlineto{\pgfqpoint{4.613421in}{0.574407in}}%
\pgfpathlineto{\pgfqpoint{4.614495in}{0.579739in}}%
\pgfpathlineto{\pgfqpoint{4.617716in}{0.576580in}}%
\pgfpathlineto{\pgfqpoint{4.618790in}{0.576478in}}%
\pgfpathlineto{\pgfqpoint{4.619864in}{0.573965in}}%
\pgfpathlineto{\pgfqpoint{4.620938in}{0.578007in}}%
\pgfpathlineto{\pgfqpoint{4.622011in}{0.576478in}}%
\pgfpathlineto{\pgfqpoint{4.625233in}{0.578618in}}%
\pgfpathlineto{\pgfqpoint{4.626307in}{0.581097in}}%
\pgfpathlineto{\pgfqpoint{4.627381in}{0.578109in}}%
\pgfpathlineto{\pgfqpoint{4.628454in}{0.570807in}}%
\pgfpathlineto{\pgfqpoint{4.629528in}{0.573150in}}%
\pgfpathlineto{\pgfqpoint{4.632750in}{0.572233in}}%
\pgfpathlineto{\pgfqpoint{4.633824in}{0.572641in}}%
\pgfpathlineto{\pgfqpoint{4.635971in}{0.576207in}}%
\pgfpathlineto{\pgfqpoint{4.637045in}{0.574441in}}%
\pgfpathlineto{\pgfqpoint{4.640267in}{0.576886in}}%
\pgfpathlineto{\pgfqpoint{4.641340in}{0.574780in}}%
\pgfpathlineto{\pgfqpoint{4.642414in}{0.575867in}}%
\pgfpathlineto{\pgfqpoint{4.644562in}{0.576546in}}%
\pgfpathlineto{\pgfqpoint{4.648857in}{0.580011in}}%
\pgfpathlineto{\pgfqpoint{4.649931in}{0.579909in}}%
\pgfpathlineto{\pgfqpoint{4.651005in}{0.585105in}}%
\pgfpathlineto{\pgfqpoint{4.652079in}{0.586362in}}%
\pgfpathlineto{\pgfqpoint{4.655300in}{0.583101in}}%
\pgfpathlineto{\pgfqpoint{4.656374in}{0.580282in}}%
\pgfpathlineto{\pgfqpoint{4.658522in}{0.582728in}}%
\pgfpathlineto{\pgfqpoint{4.659596in}{0.580690in}}%
\pgfpathlineto{\pgfqpoint{4.662817in}{0.578346in}}%
\pgfpathlineto{\pgfqpoint{4.666039in}{0.578788in}}%
\pgfpathlineto{\pgfqpoint{4.667113in}{0.580146in}}%
\pgfpathlineto{\pgfqpoint{4.670334in}{0.578992in}}%
\pgfpathlineto{\pgfqpoint{4.671408in}{0.577667in}}%
\pgfpathlineto{\pgfqpoint{4.673556in}{0.581301in}}%
\pgfpathlineto{\pgfqpoint{4.674629in}{0.584901in}}%
\pgfpathlineto{\pgfqpoint{4.677851in}{0.583373in}}%
\pgfpathlineto{\pgfqpoint{4.678925in}{0.586565in}}%
\pgfpathlineto{\pgfqpoint{4.679999in}{0.583033in}}%
\pgfpathlineto{\pgfqpoint{4.681073in}{0.582524in}}%
\pgfpathlineto{\pgfqpoint{4.682146in}{0.579365in}}%
\pgfpathlineto{\pgfqpoint{4.685368in}{0.576886in}}%
\pgfpathlineto{\pgfqpoint{4.687516in}{0.576954in}}%
\pgfpathlineto{\pgfqpoint{4.688589in}{0.572063in}}%
\pgfpathlineto{\pgfqpoint{4.689663in}{0.571418in}}%
\pgfpathlineto{\pgfqpoint{4.693959in}{0.570501in}}%
\pgfpathlineto{\pgfqpoint{4.695032in}{0.568192in}}%
\pgfpathlineto{\pgfqpoint{4.696106in}{0.569754in}}%
\pgfpathlineto{\pgfqpoint{4.697180in}{0.570026in}}%
\pgfpathlineto{\pgfqpoint{4.700402in}{0.568871in}}%
\pgfpathlineto{\pgfqpoint{4.701475in}{0.566799in}}%
\pgfpathlineto{\pgfqpoint{4.703623in}{0.568022in}}%
\pgfpathlineto{\pgfqpoint{4.704697in}{0.567241in}}%
\pgfpathlineto{\pgfqpoint{4.710066in}{0.568124in}}%
\pgfpathlineto{\pgfqpoint{4.712214in}{0.568973in}}%
\pgfpathlineto{\pgfqpoint{4.715435in}{0.569618in}}%
\pgfpathlineto{\pgfqpoint{4.719731in}{0.585445in}}%
\pgfpathlineto{\pgfqpoint{4.725100in}{0.586735in}}%
\pgfpathlineto{\pgfqpoint{4.727248in}{0.580214in}}%
\pgfpathlineto{\pgfqpoint{4.730469in}{0.579433in}}%
\pgfpathlineto{\pgfqpoint{4.731543in}{0.578075in}}%
\pgfpathlineto{\pgfqpoint{4.732617in}{0.578482in}}%
\pgfpathlineto{\pgfqpoint{4.733691in}{0.580928in}}%
\pgfpathlineto{\pgfqpoint{4.734764in}{0.580792in}}%
\pgfpathlineto{\pgfqpoint{4.737986in}{0.578618in}}%
\pgfpathlineto{\pgfqpoint{4.739060in}{0.579671in}}%
\pgfpathlineto{\pgfqpoint{4.740134in}{0.579841in}}%
\pgfpathlineto{\pgfqpoint{4.741207in}{0.577158in}}%
\pgfpathlineto{\pgfqpoint{4.742281in}{0.581131in}}%
\pgfpathlineto{\pgfqpoint{4.745503in}{0.578924in}}%
\pgfpathlineto{\pgfqpoint{4.747651in}{0.575833in}}%
\pgfpathlineto{\pgfqpoint{4.748724in}{0.582014in}}%
\pgfpathlineto{\pgfqpoint{4.749798in}{0.583950in}}%
\pgfpathlineto{\pgfqpoint{4.753020in}{0.585886in}}%
\pgfpathlineto{\pgfqpoint{4.754094in}{0.584664in}}%
\pgfpathlineto{\pgfqpoint{4.756241in}{0.584596in}}%
\pgfpathlineto{\pgfqpoint{4.757315in}{0.586871in}}%
\pgfpathlineto{\pgfqpoint{4.760537in}{0.588298in}}%
\pgfpathlineto{\pgfqpoint{4.761610in}{0.592611in}}%
\pgfpathlineto{\pgfqpoint{4.762684in}{0.589894in}}%
\pgfpathlineto{\pgfqpoint{4.763758in}{0.592781in}}%
\pgfpathlineto{\pgfqpoint{4.764832in}{0.593426in}}%
\pgfpathlineto{\pgfqpoint{4.769127in}{0.592373in}}%
\pgfpathlineto{\pgfqpoint{4.770201in}{0.590981in}}%
\pgfpathlineto{\pgfqpoint{4.771275in}{0.591354in}}%
\pgfpathlineto{\pgfqpoint{4.772349in}{0.592645in}}%
\pgfpathlineto{\pgfqpoint{4.776644in}{0.592169in}}%
\pgfpathlineto{\pgfqpoint{4.777718in}{0.592611in}}%
\pgfpathlineto{\pgfqpoint{4.778792in}{0.586871in}}%
\pgfpathlineto{\pgfqpoint{4.779866in}{0.591320in}}%
\pgfpathlineto{\pgfqpoint{4.783087in}{0.590573in}}%
\pgfpathlineto{\pgfqpoint{4.784161in}{0.588807in}}%
\pgfpathlineto{\pgfqpoint{4.786309in}{0.595803in}}%
\pgfpathlineto{\pgfqpoint{4.787383in}{0.595735in}}%
\pgfpathlineto{\pgfqpoint{4.790604in}{0.594173in}}%
\pgfpathlineto{\pgfqpoint{4.791678in}{0.592984in}}%
\pgfpathlineto{\pgfqpoint{4.792752in}{0.593324in}}%
\pgfpathlineto{\pgfqpoint{4.793826in}{0.595532in}}%
\pgfpathlineto{\pgfqpoint{4.794899in}{0.596347in}}%
\pgfpathlineto{\pgfqpoint{4.798121in}{0.594920in}}%
\pgfpathlineto{\pgfqpoint{4.799195in}{0.599403in}}%
\pgfpathlineto{\pgfqpoint{4.800269in}{0.598113in}}%
\pgfpathlineto{\pgfqpoint{4.802416in}{0.597400in}}%
\pgfpathlineto{\pgfqpoint{4.805638in}{0.598554in}}%
\pgfpathlineto{\pgfqpoint{4.806712in}{0.595430in}}%
\pgfpathlineto{\pgfqpoint{4.807785in}{0.595634in}}%
\pgfpathlineto{\pgfqpoint{4.808859in}{0.596449in}}%
\pgfpathlineto{\pgfqpoint{4.809933in}{0.600252in}}%
\pgfpathlineto{\pgfqpoint{4.813155in}{0.599166in}}%
\pgfpathlineto{\pgfqpoint{4.814228in}{0.600592in}}%
\pgfpathlineto{\pgfqpoint{4.815302in}{0.597332in}}%
\pgfpathlineto{\pgfqpoint{4.817450in}{0.597230in}}%
\pgfpathlineto{\pgfqpoint{4.821745in}{0.600490in}}%
\pgfpathlineto{\pgfqpoint{4.823893in}{0.607758in}}%
\pgfpathlineto{\pgfqpoint{4.824967in}{0.605958in}}%
\pgfpathlineto{\pgfqpoint{4.828188in}{0.608709in}}%
\pgfpathlineto{\pgfqpoint{4.829262in}{0.610509in}}%
\pgfpathlineto{\pgfqpoint{4.831410in}{0.612955in}}%
\pgfpathlineto{\pgfqpoint{4.832484in}{0.612106in}}%
\pgfpathlineto{\pgfqpoint{4.835705in}{0.618015in}}%
\pgfpathlineto{\pgfqpoint{4.836779in}{0.618525in}}%
\pgfpathlineto{\pgfqpoint{4.840001in}{0.617608in}}%
\pgfpathlineto{\pgfqpoint{4.843222in}{0.616725in}}%
\pgfpathlineto{\pgfqpoint{4.844296in}{0.618355in}}%
\pgfpathlineto{\pgfqpoint{4.845370in}{0.614823in}}%
\pgfpathlineto{\pgfqpoint{4.846444in}{0.613804in}}%
\pgfpathlineto{\pgfqpoint{4.847517in}{0.615366in}}%
\pgfpathlineto{\pgfqpoint{4.850739in}{0.608709in}}%
\pgfpathlineto{\pgfqpoint{4.851813in}{0.612411in}}%
\pgfpathlineto{\pgfqpoint{4.855034in}{0.610917in}}%
\pgfpathlineto{\pgfqpoint{4.859330in}{0.611630in}}%
\pgfpathlineto{\pgfqpoint{4.860404in}{0.615094in}}%
\pgfpathlineto{\pgfqpoint{4.861477in}{0.613464in}}%
\pgfpathlineto{\pgfqpoint{4.862551in}{0.606977in}}%
\pgfpathlineto{\pgfqpoint{4.865773in}{0.605483in}}%
\pgfpathlineto{\pgfqpoint{4.866847in}{0.607453in}}%
\pgfpathlineto{\pgfqpoint{4.867920in}{0.602596in}}%
\pgfpathlineto{\pgfqpoint{4.868994in}{0.608539in}}%
\pgfpathlineto{\pgfqpoint{4.870068in}{0.606604in}}%
\pgfpathlineto{\pgfqpoint{4.873290in}{0.598385in}}%
\pgfpathlineto{\pgfqpoint{4.875437in}{0.603954in}}%
\pgfpathlineto{\pgfqpoint{4.876511in}{0.615909in}}%
\pgfpathlineto{\pgfqpoint{4.877585in}{0.616249in}}%
\pgfpathlineto{\pgfqpoint{4.882954in}{0.623279in}}%
\pgfpathlineto{\pgfqpoint{4.884028in}{0.623245in}}%
\pgfpathlineto{\pgfqpoint{4.885102in}{0.627423in}}%
\pgfpathlineto{\pgfqpoint{4.889397in}{0.628544in}}%
\pgfpathlineto{\pgfqpoint{4.890471in}{0.626302in}}%
\pgfpathlineto{\pgfqpoint{4.891545in}{0.627321in}}%
\pgfpathlineto{\pgfqpoint{4.892619in}{0.630276in}}%
\pgfpathlineto{\pgfqpoint{4.895840in}{0.632144in}}%
\pgfpathlineto{\pgfqpoint{4.896914in}{0.630513in}}%
\pgfpathlineto{\pgfqpoint{4.897988in}{0.630106in}}%
\pgfpathlineto{\pgfqpoint{4.900136in}{0.632415in}}%
\pgfpathlineto{\pgfqpoint{4.903357in}{0.628170in}}%
\pgfpathlineto{\pgfqpoint{4.904431in}{0.635234in}}%
\pgfpathlineto{\pgfqpoint{4.906579in}{0.640363in}}%
\pgfpathlineto{\pgfqpoint{4.907652in}{0.636797in}}%
\pgfpathlineto{\pgfqpoint{4.910874in}{0.636049in}}%
\pgfpathlineto{\pgfqpoint{4.911948in}{0.633163in}}%
\pgfpathlineto{\pgfqpoint{4.913022in}{0.634181in}}%
\pgfpathlineto{\pgfqpoint{4.914095in}{0.629630in}}%
\pgfpathlineto{\pgfqpoint{4.915169in}{0.630072in}}%
\pgfpathlineto{\pgfqpoint{4.919465in}{0.636117in}}%
\pgfpathlineto{\pgfqpoint{4.920539in}{0.631498in}}%
\pgfpathlineto{\pgfqpoint{4.921612in}{0.632415in}}%
\pgfpathlineto{\pgfqpoint{4.922686in}{0.630955in}}%
\pgfpathlineto{\pgfqpoint{4.925908in}{0.628238in}}%
\pgfpathlineto{\pgfqpoint{4.926982in}{0.628442in}}%
\pgfpathlineto{\pgfqpoint{4.928055in}{0.625521in}}%
\pgfpathlineto{\pgfqpoint{4.929129in}{0.625079in}}%
\pgfpathlineto{\pgfqpoint{4.930203in}{0.626676in}}%
\pgfpathlineto{\pgfqpoint{4.933425in}{0.630072in}}%
\pgfpathlineto{\pgfqpoint{4.934498in}{0.634793in}}%
\pgfpathlineto{\pgfqpoint{4.936646in}{0.635166in}}%
\pgfpathlineto{\pgfqpoint{4.937720in}{0.631193in}}%
\pgfpathlineto{\pgfqpoint{4.940941in}{0.627117in}}%
\pgfpathlineto{\pgfqpoint{4.942015in}{0.628510in}}%
\pgfpathlineto{\pgfqpoint{4.943089in}{0.631329in}}%
\pgfpathlineto{\pgfqpoint{4.944163in}{0.622023in}}%
\pgfpathlineto{\pgfqpoint{4.945237in}{0.620460in}}%
\pgfpathlineto{\pgfqpoint{4.949532in}{0.622498in}}%
\pgfpathlineto{\pgfqpoint{4.951680in}{0.629834in}}%
\pgfpathlineto{\pgfqpoint{4.955975in}{0.627049in}}%
\pgfpathlineto{\pgfqpoint{4.957049in}{0.628136in}}%
\pgfpathlineto{\pgfqpoint{4.959197in}{0.628578in}}%
\pgfpathlineto{\pgfqpoint{4.960271in}{0.624298in}}%
\pgfpathlineto{\pgfqpoint{4.963492in}{0.622872in}}%
\pgfpathlineto{\pgfqpoint{4.965640in}{0.627796in}}%
\pgfpathlineto{\pgfqpoint{4.966714in}{0.628476in}}%
\pgfpathlineto{\pgfqpoint{4.967787in}{0.630819in}}%
\pgfpathlineto{\pgfqpoint{4.971009in}{0.633706in}}%
\pgfpathlineto{\pgfqpoint{4.972083in}{0.633095in}}%
\pgfpathlineto{\pgfqpoint{4.973157in}{0.630751in}}%
\pgfpathlineto{\pgfqpoint{4.974230in}{0.634895in}}%
\pgfpathlineto{\pgfqpoint{4.975304in}{0.635981in}}%
\pgfpathlineto{\pgfqpoint{4.978526in}{0.637476in}}%
\pgfpathlineto{\pgfqpoint{4.979600in}{0.636389in}}%
\pgfpathlineto{\pgfqpoint{4.981747in}{0.630344in}}%
\pgfpathlineto{\pgfqpoint{4.986043in}{0.633163in}}%
\pgfpathlineto{\pgfqpoint{4.987116in}{0.633298in}}%
\pgfpathlineto{\pgfqpoint{4.988190in}{0.636729in}}%
\pgfpathlineto{\pgfqpoint{4.989264in}{0.637238in}}%
\pgfpathlineto{\pgfqpoint{4.990338in}{0.640295in}}%
\pgfpathlineto{\pgfqpoint{4.994633in}{0.641619in}}%
\pgfpathlineto{\pgfqpoint{4.995707in}{0.641178in}}%
\pgfpathlineto{\pgfqpoint{4.996781in}{0.642536in}}%
\pgfpathlineto{\pgfqpoint{4.997855in}{0.642468in}}%
\pgfpathlineto{\pgfqpoint{5.001076in}{0.643283in}}%
\pgfpathlineto{\pgfqpoint{5.002150in}{0.642061in}}%
\pgfpathlineto{\pgfqpoint{5.004298in}{0.644880in}}%
\pgfpathlineto{\pgfqpoint{5.005372in}{0.644336in}}%
\pgfpathlineto{\pgfqpoint{5.008593in}{0.647087in}}%
\pgfpathlineto{\pgfqpoint{5.010741in}{0.643419in}}%
\pgfpathlineto{\pgfqpoint{5.011815in}{0.638665in}}%
\pgfpathlineto{\pgfqpoint{5.012889in}{0.638665in}}%
\pgfpathlineto{\pgfqpoint{5.016110in}{0.640227in}}%
\pgfpathlineto{\pgfqpoint{5.017184in}{0.639785in}}%
\pgfpathlineto{\pgfqpoint{5.019332in}{0.641721in}}%
\pgfpathlineto{\pgfqpoint{5.020405in}{0.638732in}}%
\pgfpathlineto{\pgfqpoint{5.024701in}{0.638291in}}%
\pgfpathlineto{\pgfqpoint{5.025775in}{0.642129in}}%
\pgfpathlineto{\pgfqpoint{5.027922in}{0.646510in}}%
\pgfpathlineto{\pgfqpoint{5.031144in}{0.647902in}}%
\pgfpathlineto{\pgfqpoint{5.032218in}{0.650212in}}%
\pgfpathlineto{\pgfqpoint{5.033292in}{0.650110in}}%
\pgfpathlineto{\pgfqpoint{5.034365in}{0.651367in}}%
\pgfpathlineto{\pgfqpoint{5.038661in}{0.650586in}}%
\pgfpathlineto{\pgfqpoint{5.039735in}{0.648276in}}%
\pgfpathlineto{\pgfqpoint{5.040808in}{0.652555in}}%
\pgfpathlineto{\pgfqpoint{5.041882in}{0.651435in}}%
\pgfpathlineto{\pgfqpoint{5.042956in}{0.651536in}}%
\pgfpathlineto{\pgfqpoint{5.046178in}{0.651129in}}%
\pgfpathlineto{\pgfqpoint{5.049399in}{0.645661in}}%
\pgfpathlineto{\pgfqpoint{5.050473in}{0.648140in}}%
\pgfpathlineto{\pgfqpoint{5.053694in}{0.647902in}}%
\pgfpathlineto{\pgfqpoint{5.054768in}{0.649397in}}%
\pgfpathlineto{\pgfqpoint{5.055842in}{0.648853in}}%
\pgfpathlineto{\pgfqpoint{5.056916in}{0.651231in}}%
\pgfpathlineto{\pgfqpoint{5.057990in}{0.649431in}}%
\pgfpathlineto{\pgfqpoint{5.061211in}{0.652352in}}%
\pgfpathlineto{\pgfqpoint{5.062285in}{0.649363in}}%
\pgfpathlineto{\pgfqpoint{5.063359in}{0.652521in}}%
\pgfpathlineto{\pgfqpoint{5.064433in}{0.648344in}}%
\pgfpathlineto{\pgfqpoint{5.065507in}{0.646782in}}%
\pgfpathlineto{\pgfqpoint{5.068728in}{0.652623in}}%
\pgfpathlineto{\pgfqpoint{5.069802in}{0.651095in}}%
\pgfpathlineto{\pgfqpoint{5.070876in}{0.650586in}}%
\pgfpathlineto{\pgfqpoint{5.071950in}{0.647189in}}%
\pgfpathlineto{\pgfqpoint{5.073024in}{0.651401in}}%
\pgfpathlineto{\pgfqpoint{5.076245in}{0.653438in}}%
\pgfpathlineto{\pgfqpoint{5.077319in}{0.652521in}}%
\pgfpathlineto{\pgfqpoint{5.078393in}{0.653744in}}%
\pgfpathlineto{\pgfqpoint{5.080540in}{0.659042in}}%
\pgfpathlineto{\pgfqpoint{5.084836in}{0.661589in}}%
\pgfpathlineto{\pgfqpoint{5.085910in}{0.660808in}}%
\pgfpathlineto{\pgfqpoint{5.086983in}{0.662506in}}%
\pgfpathlineto{\pgfqpoint{5.088057in}{0.662846in}}%
\pgfpathlineto{\pgfqpoint{5.091279in}{0.662506in}}%
\pgfpathlineto{\pgfqpoint{5.092353in}{0.661352in}}%
\pgfpathlineto{\pgfqpoint{5.093427in}{0.662473in}}%
\pgfpathlineto{\pgfqpoint{5.094500in}{0.662031in}}%
\pgfpathlineto{\pgfqpoint{5.095574in}{0.660910in}}%
\pgfpathlineto{\pgfqpoint{5.100943in}{0.664239in}}%
\pgfpathlineto{\pgfqpoint{5.102017in}{0.661725in}}%
\pgfpathlineto{\pgfqpoint{5.103091in}{0.664239in}}%
\pgfpathlineto{\pgfqpoint{5.106313in}{0.663050in}}%
\pgfpathlineto{\pgfqpoint{5.107386in}{0.660027in}}%
\pgfpathlineto{\pgfqpoint{5.108460in}{0.659789in}}%
\pgfpathlineto{\pgfqpoint{5.109534in}{0.661216in}}%
\pgfpathlineto{\pgfqpoint{5.110608in}{0.660265in}}%
\pgfpathlineto{\pgfqpoint{5.118125in}{0.662846in}}%
\pgfpathlineto{\pgfqpoint{5.121346in}{0.659076in}}%
\pgfpathlineto{\pgfqpoint{5.122420in}{0.655986in}}%
\pgfpathlineto{\pgfqpoint{5.123494in}{0.659586in}}%
\pgfpathlineto{\pgfqpoint{5.124568in}{0.655238in}}%
\pgfpathlineto{\pgfqpoint{5.125642in}{0.657344in}}%
\pgfpathlineto{\pgfqpoint{5.129937in}{0.658261in}}%
\pgfpathlineto{\pgfqpoint{5.131011in}{0.653404in}}%
\pgfpathlineto{\pgfqpoint{5.132085in}{0.651197in}}%
\pgfpathlineto{\pgfqpoint{5.133159in}{0.656427in}}%
\pgfpathlineto{\pgfqpoint{5.136380in}{0.656767in}}%
\pgfpathlineto{\pgfqpoint{5.137454in}{0.652216in}}%
\pgfpathlineto{\pgfqpoint{5.138528in}{0.655340in}}%
\pgfpathlineto{\pgfqpoint{5.139602in}{0.647801in}}%
\pgfpathlineto{\pgfqpoint{5.140675in}{0.649567in}}%
\pgfpathlineto{\pgfqpoint{5.143897in}{0.642163in}}%
\pgfpathlineto{\pgfqpoint{5.144971in}{0.642876in}}%
\pgfpathlineto{\pgfqpoint{5.146045in}{0.636389in}}%
\pgfpathlineto{\pgfqpoint{5.147118in}{0.635336in}}%
\pgfpathlineto{\pgfqpoint{5.148192in}{0.641891in}}%
\pgfpathlineto{\pgfqpoint{5.151414in}{0.647223in}}%
\pgfpathlineto{\pgfqpoint{5.152488in}{0.653574in}}%
\pgfpathlineto{\pgfqpoint{5.153561in}{0.652182in}}%
\pgfpathlineto{\pgfqpoint{5.155709in}{0.656937in}}%
\pgfpathlineto{\pgfqpoint{5.158931in}{0.656427in}}%
\pgfpathlineto{\pgfqpoint{5.160004in}{0.661080in}}%
\pgfpathlineto{\pgfqpoint{5.161078in}{0.659823in}}%
\pgfpathlineto{\pgfqpoint{5.162152in}{0.661997in}}%
\pgfpathlineto{\pgfqpoint{5.163226in}{0.665631in}}%
\pgfpathlineto{\pgfqpoint{5.166448in}{0.666684in}}%
\pgfpathlineto{\pgfqpoint{5.167521in}{0.662405in}}%
\pgfpathlineto{\pgfqpoint{5.169669in}{0.667601in}}%
\pgfpathlineto{\pgfqpoint{5.170743in}{0.661284in}}%
\pgfpathlineto{\pgfqpoint{5.173964in}{0.660605in}}%
\pgfpathlineto{\pgfqpoint{5.175038in}{0.661250in}}%
\pgfpathlineto{\pgfqpoint{5.176112in}{0.660978in}}%
\pgfpathlineto{\pgfqpoint{5.178260in}{0.663831in}}%
\pgfpathlineto{\pgfqpoint{5.182555in}{0.662167in}}%
\pgfpathlineto{\pgfqpoint{5.183629in}{0.660740in}}%
\pgfpathlineto{\pgfqpoint{5.184703in}{0.657854in}}%
\pgfpathlineto{\pgfqpoint{5.185777in}{0.658023in}}%
\pgfpathlineto{\pgfqpoint{5.188998in}{0.663118in}}%
\pgfpathlineto{\pgfqpoint{5.190072in}{0.666480in}}%
\pgfpathlineto{\pgfqpoint{5.193293in}{0.669197in}}%
\pgfpathlineto{\pgfqpoint{5.196515in}{0.669775in}}%
\pgfpathlineto{\pgfqpoint{5.197589in}{0.672220in}}%
\pgfpathlineto{\pgfqpoint{5.198663in}{0.671065in}}%
\pgfpathlineto{\pgfqpoint{5.199737in}{0.671473in}}%
\pgfpathlineto{\pgfqpoint{5.200810in}{0.673137in}}%
\pgfpathlineto{\pgfqpoint{5.204032in}{0.673239in}}%
\pgfpathlineto{\pgfqpoint{5.206180in}{0.666412in}}%
\pgfpathlineto{\pgfqpoint{5.207253in}{0.670454in}}%
\pgfpathlineto{\pgfqpoint{5.211549in}{0.667771in}}%
\pgfpathlineto{\pgfqpoint{5.212623in}{0.665427in}}%
\pgfpathlineto{\pgfqpoint{5.214770in}{0.673205in}}%
\pgfpathlineto{\pgfqpoint{5.215844in}{0.674088in}}%
\pgfpathlineto{\pgfqpoint{5.220139in}{0.679828in}}%
\pgfpathlineto{\pgfqpoint{5.221213in}{0.678978in}}%
\pgfpathlineto{\pgfqpoint{5.223361in}{0.680914in}}%
\pgfpathlineto{\pgfqpoint{5.226582in}{0.682409in}}%
\pgfpathlineto{\pgfqpoint{5.228730in}{0.678231in}}%
\pgfpathlineto{\pgfqpoint{5.230878in}{0.676839in}}%
\pgfpathlineto{\pgfqpoint{5.235173in}{0.670929in}}%
\pgfpathlineto{\pgfqpoint{5.238395in}{0.678435in}}%
\pgfpathlineto{\pgfqpoint{5.241616in}{0.679114in}}%
\pgfpathlineto{\pgfqpoint{5.242690in}{0.681424in}}%
\pgfpathlineto{\pgfqpoint{5.243764in}{0.678367in}}%
\pgfpathlineto{\pgfqpoint{5.244838in}{0.678741in}}%
\pgfpathlineto{\pgfqpoint{5.245912in}{0.681390in}}%
\pgfpathlineto{\pgfqpoint{5.250207in}{0.679997in}}%
\pgfpathlineto{\pgfqpoint{5.251281in}{0.678163in}}%
\pgfpathlineto{\pgfqpoint{5.252355in}{0.681288in}}%
\pgfpathlineto{\pgfqpoint{5.253428in}{0.679929in}}%
\pgfpathlineto{\pgfqpoint{5.256650in}{0.680711in}}%
\pgfpathlineto{\pgfqpoint{5.258798in}{0.673409in}}%
\pgfpathlineto{\pgfqpoint{5.259871in}{0.675141in}}%
\pgfpathlineto{\pgfqpoint{5.260945in}{0.667941in}}%
\pgfpathlineto{\pgfqpoint{5.264167in}{0.671031in}}%
\pgfpathlineto{\pgfqpoint{5.265241in}{0.677960in}}%
\pgfpathlineto{\pgfqpoint{5.266315in}{0.700817in}}%
\pgfpathlineto{\pgfqpoint{5.267388in}{0.705130in}}%
\pgfpathlineto{\pgfqpoint{5.268462in}{0.703160in}}%
\pgfpathlineto{\pgfqpoint{5.271684in}{0.702243in}}%
\pgfpathlineto{\pgfqpoint{5.272758in}{0.702854in}}%
\pgfpathlineto{\pgfqpoint{5.273831in}{0.702685in}}%
\pgfpathlineto{\pgfqpoint{5.274905in}{0.708153in}}%
\pgfpathlineto{\pgfqpoint{5.275979in}{0.710021in}}%
\pgfpathlineto{\pgfqpoint{5.280274in}{0.709851in}}%
\pgfpathlineto{\pgfqpoint{5.281348in}{0.709036in}}%
\pgfpathlineto{\pgfqpoint{5.282422in}{0.709138in}}%
\pgfpathlineto{\pgfqpoint{5.283496in}{0.711243in}}%
\pgfpathlineto{\pgfqpoint{5.286717in}{0.712636in}}%
\pgfpathlineto{\pgfqpoint{5.287791in}{0.711617in}}%
\pgfpathlineto{\pgfqpoint{5.288865in}{0.714470in}}%
\pgfpathlineto{\pgfqpoint{5.291013in}{0.709749in}}%
\pgfpathlineto{\pgfqpoint{5.295308in}{0.716949in}}%
\pgfpathlineto{\pgfqpoint{5.298530in}{0.708900in}}%
\pgfpathlineto{\pgfqpoint{5.301751in}{0.713485in}}%
\pgfpathlineto{\pgfqpoint{5.302825in}{0.706590in}}%
\pgfpathlineto{\pgfqpoint{5.303899in}{0.705945in}}%
\pgfpathlineto{\pgfqpoint{5.304973in}{0.719564in}}%
\pgfpathlineto{\pgfqpoint{5.306047in}{0.717255in}}%
\pgfpathlineto{\pgfqpoint{5.309268in}{0.720209in}}%
\pgfpathlineto{\pgfqpoint{5.310342in}{0.718919in}}%
\pgfpathlineto{\pgfqpoint{5.311416in}{0.722111in}}%
\pgfpathlineto{\pgfqpoint{5.312490in}{0.720209in}}%
\pgfpathlineto{\pgfqpoint{5.313563in}{0.723572in}}%
\pgfpathlineto{\pgfqpoint{5.316785in}{0.722926in}}%
\pgfpathlineto{\pgfqpoint{5.317859in}{0.719394in}}%
\pgfpathlineto{\pgfqpoint{5.318933in}{0.712670in}}%
\pgfpathlineto{\pgfqpoint{5.324302in}{0.716236in}}%
\pgfpathlineto{\pgfqpoint{5.325376in}{0.712296in}}%
\pgfpathlineto{\pgfqpoint{5.327523in}{0.715862in}}%
\pgfpathlineto{\pgfqpoint{5.332893in}{0.714028in}}%
\pgfpathlineto{\pgfqpoint{5.333966in}{0.716983in}}%
\pgfpathlineto{\pgfqpoint{5.336114in}{0.718885in}}%
\pgfpathlineto{\pgfqpoint{5.339336in}{0.717458in}}%
\pgfpathlineto{\pgfqpoint{5.341483in}{0.718953in}}%
\pgfpathlineto{\pgfqpoint{5.342557in}{0.722519in}}%
\pgfpathlineto{\pgfqpoint{5.343631in}{0.718036in}}%
\pgfpathlineto{\pgfqpoint{5.346852in}{0.722926in}}%
\pgfpathlineto{\pgfqpoint{5.347926in}{0.721194in}}%
\pgfpathlineto{\pgfqpoint{5.349000in}{0.722009in}}%
\pgfpathlineto{\pgfqpoint{5.351148in}{0.727104in}}%
\pgfpathlineto{\pgfqpoint{5.354369in}{0.729108in}}%
\pgfpathlineto{\pgfqpoint{5.356517in}{0.727987in}}%
\pgfpathlineto{\pgfqpoint{5.357591in}{0.724523in}}%
\pgfpathlineto{\pgfqpoint{5.358665in}{0.730229in}}%
\pgfpathlineto{\pgfqpoint{5.361886in}{0.731859in}}%
\pgfpathlineto{\pgfqpoint{5.362960in}{0.731179in}}%
\pgfpathlineto{\pgfqpoint{5.364034in}{0.727681in}}%
\pgfpathlineto{\pgfqpoint{5.365108in}{0.726221in}}%
\pgfpathlineto{\pgfqpoint{5.366181in}{0.728938in}}%
\pgfpathlineto{\pgfqpoint{5.369403in}{0.724115in}}%
\pgfpathlineto{\pgfqpoint{5.370477in}{0.726153in}}%
\pgfpathlineto{\pgfqpoint{5.371551in}{0.726017in}}%
\pgfpathlineto{\pgfqpoint{5.373698in}{0.729549in}}%
\pgfpathlineto{\pgfqpoint{5.376920in}{0.729651in}}%
\pgfpathlineto{\pgfqpoint{5.377994in}{0.730364in}}%
\pgfpathlineto{\pgfqpoint{5.379068in}{0.729210in}}%
\pgfpathlineto{\pgfqpoint{5.380141in}{0.729821in}}%
\pgfpathlineto{\pgfqpoint{5.381215in}{0.729413in}}%
\pgfpathlineto{\pgfqpoint{5.385511in}{0.726798in}}%
\pgfpathlineto{\pgfqpoint{5.386584in}{0.729753in}}%
\pgfpathlineto{\pgfqpoint{5.387658in}{0.730262in}}%
\pgfpathlineto{\pgfqpoint{5.388732in}{0.729753in}}%
\pgfpathlineto{\pgfqpoint{5.391954in}{0.731655in}}%
\pgfpathlineto{\pgfqpoint{5.393027in}{0.730976in}}%
\pgfpathlineto{\pgfqpoint{5.394101in}{0.732300in}}%
\pgfpathlineto{\pgfqpoint{5.395175in}{0.729549in}}%
\pgfpathlineto{\pgfqpoint{5.396249in}{0.729549in}}%
\pgfpathlineto{\pgfqpoint{5.399470in}{0.726323in}}%
\pgfpathlineto{\pgfqpoint{5.400544in}{0.723877in}}%
\pgfpathlineto{\pgfqpoint{5.401618in}{0.728598in}}%
\pgfpathlineto{\pgfqpoint{5.402692in}{0.730568in}}%
\pgfpathlineto{\pgfqpoint{5.403766in}{0.728428in}}%
\pgfpathlineto{\pgfqpoint{5.406987in}{0.729142in}}%
\pgfpathlineto{\pgfqpoint{5.410209in}{0.738821in}}%
\pgfpathlineto{\pgfqpoint{5.411283in}{0.736919in}}%
\pgfpathlineto{\pgfqpoint{5.414504in}{0.739840in}}%
\pgfpathlineto{\pgfqpoint{5.415578in}{0.742625in}}%
\pgfpathlineto{\pgfqpoint{5.416652in}{0.740587in}}%
\pgfpathlineto{\pgfqpoint{5.418800in}{0.744459in}}%
\pgfpathlineto{\pgfqpoint{5.422021in}{0.738312in}}%
\pgfpathlineto{\pgfqpoint{5.424169in}{0.747040in}}%
\pgfpathlineto{\pgfqpoint{5.425243in}{0.746531in}}%
\pgfpathlineto{\pgfqpoint{5.429538in}{0.748874in}}%
\pgfpathlineto{\pgfqpoint{5.430612in}{0.753357in}}%
\pgfpathlineto{\pgfqpoint{5.431686in}{0.747244in}}%
\pgfpathlineto{\pgfqpoint{5.432759in}{0.748568in}}%
\pgfpathlineto{\pgfqpoint{5.433833in}{0.751252in}}%
\pgfpathlineto{\pgfqpoint{5.437055in}{0.756414in}}%
\pgfpathlineto{\pgfqpoint{5.438129in}{0.755769in}}%
\pgfpathlineto{\pgfqpoint{5.439203in}{0.757195in}}%
\pgfpathlineto{\pgfqpoint{5.440276in}{0.759674in}}%
\pgfpathlineto{\pgfqpoint{5.441350in}{0.758995in}}%
\pgfpathlineto{\pgfqpoint{5.444572in}{0.761305in}}%
\pgfpathlineto{\pgfqpoint{5.445646in}{0.760456in}}%
\pgfpathlineto{\pgfqpoint{5.446719in}{0.760489in}}%
\pgfpathlineto{\pgfqpoint{5.447793in}{0.758791in}}%
\pgfpathlineto{\pgfqpoint{5.448867in}{0.759165in}}%
\pgfpathlineto{\pgfqpoint{5.452089in}{0.757059in}}%
\pgfpathlineto{\pgfqpoint{5.453162in}{0.757705in}}%
\pgfpathlineto{\pgfqpoint{5.454236in}{0.762154in}}%
\pgfpathlineto{\pgfqpoint{5.455310in}{0.762731in}}%
\pgfpathlineto{\pgfqpoint{5.456384in}{0.762663in}}%
\pgfpathlineto{\pgfqpoint{5.460679in}{0.768063in}}%
\pgfpathlineto{\pgfqpoint{5.461753in}{0.732334in}}%
\pgfpathlineto{\pgfqpoint{5.462827in}{0.725983in}}%
\pgfpathlineto{\pgfqpoint{5.463901in}{0.728530in}}%
\pgfpathlineto{\pgfqpoint{5.467122in}{0.733829in}}%
\pgfpathlineto{\pgfqpoint{5.468196in}{0.724217in}}%
\pgfpathlineto{\pgfqpoint{5.469270in}{0.720991in}}%
\pgfpathlineto{\pgfqpoint{5.470344in}{0.722689in}}%
\pgfpathlineto{\pgfqpoint{5.471418in}{0.721534in}}%
\pgfpathlineto{\pgfqpoint{5.474639in}{0.727579in}}%
\pgfpathlineto{\pgfqpoint{5.475713in}{0.720821in}}%
\pgfpathlineto{\pgfqpoint{5.476787in}{0.719259in}}%
\pgfpathlineto{\pgfqpoint{5.477861in}{0.698643in}}%
\pgfpathlineto{\pgfqpoint{5.482156in}{0.683733in}}%
\pgfpathlineto{\pgfqpoint{5.483230in}{0.685431in}}%
\pgfpathlineto{\pgfqpoint{5.485378in}{0.705537in}}%
\pgfpathlineto{\pgfqpoint{5.486451in}{0.706522in}}%
\pgfpathlineto{\pgfqpoint{5.489673in}{0.704620in}}%
\pgfpathlineto{\pgfqpoint{5.490747in}{0.697013in}}%
\pgfpathlineto{\pgfqpoint{5.491821in}{0.704654in}}%
\pgfpathlineto{\pgfqpoint{5.492894in}{0.704960in}}%
\pgfpathlineto{\pgfqpoint{5.493968in}{0.701700in}}%
\pgfpathlineto{\pgfqpoint{5.498264in}{0.711447in}}%
\pgfpathlineto{\pgfqpoint{5.499337in}{0.704722in}}%
\pgfpathlineto{\pgfqpoint{5.500411in}{0.706930in}}%
\pgfpathlineto{\pgfqpoint{5.501485in}{0.712941in}}%
\pgfpathlineto{\pgfqpoint{5.505781in}{0.709579in}}%
\pgfpathlineto{\pgfqpoint{5.507928in}{0.712058in}}%
\pgfpathlineto{\pgfqpoint{5.509002in}{0.707677in}}%
\pgfpathlineto{\pgfqpoint{5.512224in}{0.709511in}}%
\pgfpathlineto{\pgfqpoint{5.515445in}{0.700579in}}%
\pgfpathlineto{\pgfqpoint{5.516519in}{0.699560in}}%
\pgfpathlineto{\pgfqpoint{5.519740in}{0.693752in}}%
\pgfpathlineto{\pgfqpoint{5.520814in}{0.696741in}}%
\pgfpathlineto{\pgfqpoint{5.521888in}{0.705639in}}%
\pgfpathlineto{\pgfqpoint{5.524036in}{0.708187in}}%
\pgfpathlineto{\pgfqpoint{5.527257in}{0.710938in}}%
\pgfpathlineto{\pgfqpoint{5.528331in}{0.710666in}}%
\pgfpathlineto{\pgfqpoint{5.529405in}{0.709443in}}%
\pgfpathlineto{\pgfqpoint{5.531553in}{0.716406in}}%
\pgfpathlineto{\pgfqpoint{5.535848in}{0.719700in}}%
\pgfpathlineto{\pgfqpoint{5.536922in}{0.716949in}}%
\pgfpathlineto{\pgfqpoint{5.537996in}{0.723877in}}%
\pgfpathlineto{\pgfqpoint{5.544439in}{0.730908in}}%
\pgfpathlineto{\pgfqpoint{5.545513in}{0.741029in}}%
\pgfpathlineto{\pgfqpoint{5.546586in}{0.740519in}}%
\pgfpathlineto{\pgfqpoint{5.550882in}{0.742693in}}%
\pgfpathlineto{\pgfqpoint{5.553029in}{0.746768in}}%
\pgfpathlineto{\pgfqpoint{5.554103in}{0.742591in}}%
\pgfpathlineto{\pgfqpoint{5.558399in}{0.748365in}}%
\pgfpathlineto{\pgfqpoint{5.559472in}{0.741029in}}%
\pgfpathlineto{\pgfqpoint{5.560546in}{0.740214in}}%
\pgfpathlineto{\pgfqpoint{5.561620in}{0.748772in}}%
\pgfpathlineto{\pgfqpoint{5.564842in}{0.751184in}}%
\pgfpathlineto{\pgfqpoint{5.565915in}{0.754376in}}%
\pgfpathlineto{\pgfqpoint{5.566989in}{0.751489in}}%
\pgfpathlineto{\pgfqpoint{5.568063in}{0.750504in}}%
\pgfpathlineto{\pgfqpoint{5.569137in}{0.746123in}}%
\pgfpathlineto{\pgfqpoint{5.573432in}{0.750267in}}%
\pgfpathlineto{\pgfqpoint{5.574506in}{0.756686in}}%
\pgfpathlineto{\pgfqpoint{5.575580in}{0.758520in}}%
\pgfpathlineto{\pgfqpoint{5.576654in}{0.762867in}}%
\pgfpathlineto{\pgfqpoint{5.579875in}{0.760795in}}%
\pgfpathlineto{\pgfqpoint{5.580949in}{0.756074in}}%
\pgfpathlineto{\pgfqpoint{5.582023in}{0.758384in}}%
\pgfpathlineto{\pgfqpoint{5.584171in}{0.747040in}}%
\pgfpathlineto{\pgfqpoint{5.587392in}{0.741742in}}%
\pgfpathlineto{\pgfqpoint{5.588466in}{0.747889in}}%
\pgfpathlineto{\pgfqpoint{5.590614in}{0.736681in}}%
\pgfpathlineto{\pgfqpoint{5.591688in}{0.744187in}}%
\pgfpathlineto{\pgfqpoint{5.594909in}{0.742897in}}%
\pgfpathlineto{\pgfqpoint{5.597057in}{0.735323in}}%
\pgfpathlineto{\pgfqpoint{5.598131in}{0.735323in}}%
\pgfpathlineto{\pgfqpoint{5.599204in}{0.726561in}}%
\pgfpathlineto{\pgfqpoint{5.602426in}{0.730772in}}%
\pgfpathlineto{\pgfqpoint{5.604574in}{0.745104in}}%
\pgfpathlineto{\pgfqpoint{5.605647in}{0.739365in}}%
\pgfpathlineto{\pgfqpoint{5.606721in}{0.725542in}}%
\pgfpathlineto{\pgfqpoint{5.609943in}{0.721874in}}%
\pgfpathlineto{\pgfqpoint{5.611017in}{0.722383in}}%
\pgfpathlineto{\pgfqpoint{5.612091in}{0.718579in}}%
\pgfpathlineto{\pgfqpoint{5.617460in}{0.724013in}}%
\pgfpathlineto{\pgfqpoint{5.618534in}{0.723470in}}%
\pgfpathlineto{\pgfqpoint{5.619607in}{0.721092in}}%
\pgfpathlineto{\pgfqpoint{5.620681in}{0.717017in}}%
\pgfpathlineto{\pgfqpoint{5.624977in}{0.710258in}}%
\pgfpathlineto{\pgfqpoint{5.626050in}{0.703534in}}%
\pgfpathlineto{\pgfqpoint{5.629272in}{0.698235in}}%
\pgfpathlineto{\pgfqpoint{5.632493in}{0.700375in}}%
\pgfpathlineto{\pgfqpoint{5.633567in}{0.705368in}}%
\pgfpathlineto{\pgfqpoint{5.634641in}{0.695756in}}%
\pgfpathlineto{\pgfqpoint{5.635715in}{0.697760in}}%
\pgfpathlineto{\pgfqpoint{5.636789in}{0.680982in}}%
\pgfpathlineto{\pgfqpoint{5.641084in}{0.681220in}}%
\pgfpathlineto{\pgfqpoint{5.642158in}{0.676601in}}%
\pgfpathlineto{\pgfqpoint{5.643232in}{0.681356in}}%
\pgfpathlineto{\pgfqpoint{5.644306in}{0.690662in}}%
\pgfpathlineto{\pgfqpoint{5.647527in}{0.685465in}}%
\pgfpathlineto{\pgfqpoint{5.648601in}{0.688624in}}%
\pgfpathlineto{\pgfqpoint{5.649675in}{0.682341in}}%
\pgfpathlineto{\pgfqpoint{5.650749in}{0.679794in}}%
\pgfpathlineto{\pgfqpoint{5.651823in}{0.687164in}}%
\pgfpathlineto{\pgfqpoint{5.655044in}{0.685024in}}%
\pgfpathlineto{\pgfqpoint{5.656118in}{0.678469in}}%
\pgfpathlineto{\pgfqpoint{5.657192in}{0.684990in}}%
\pgfpathlineto{\pgfqpoint{5.658266in}{0.685907in}}%
\pgfpathlineto{\pgfqpoint{5.659339in}{0.680982in}}%
\pgfpathlineto{\pgfqpoint{5.662561in}{0.675243in}}%
\pgfpathlineto{\pgfqpoint{5.663635in}{0.675888in}}%
\pgfpathlineto{\pgfqpoint{5.664709in}{0.664714in}}%
\pgfpathlineto{\pgfqpoint{5.666856in}{0.672118in}}%
\pgfpathlineto{\pgfqpoint{5.671152in}{0.677790in}}%
\pgfpathlineto{\pgfqpoint{5.672225in}{0.686145in}}%
\pgfpathlineto{\pgfqpoint{5.674373in}{0.684548in}}%
\pgfpathlineto{\pgfqpoint{5.677595in}{0.688930in}}%
\pgfpathlineto{\pgfqpoint{5.678669in}{0.685737in}}%
\pgfpathlineto{\pgfqpoint{5.680816in}{0.686620in}}%
\pgfpathlineto{\pgfqpoint{5.681890in}{0.685533in}}%
\pgfpathlineto{\pgfqpoint{5.685112in}{0.686213in}}%
\pgfpathlineto{\pgfqpoint{5.686185in}{0.693073in}}%
\pgfpathlineto{\pgfqpoint{5.687259in}{0.690967in}}%
\pgfpathlineto{\pgfqpoint{5.688333in}{0.696843in}}%
\pgfpathlineto{\pgfqpoint{5.689407in}{0.695756in}}%
\pgfpathlineto{\pgfqpoint{5.692628in}{0.698677in}}%
\pgfpathlineto{\pgfqpoint{5.693702in}{0.693617in}}%
\pgfpathlineto{\pgfqpoint{5.694776in}{0.693107in}}%
\pgfpathlineto{\pgfqpoint{5.695850in}{0.691103in}}%
\pgfpathlineto{\pgfqpoint{5.696924in}{0.693990in}}%
\pgfpathlineto{\pgfqpoint{5.700145in}{0.696809in}}%
\pgfpathlineto{\pgfqpoint{5.701219in}{0.694975in}}%
\pgfpathlineto{\pgfqpoint{5.702293in}{0.695620in}}%
\pgfpathlineto{\pgfqpoint{5.703367in}{0.699356in}}%
\pgfpathlineto{\pgfqpoint{5.704441in}{0.698066in}}%
\pgfpathlineto{\pgfqpoint{5.707662in}{0.695688in}}%
\pgfpathlineto{\pgfqpoint{5.709810in}{0.690424in}}%
\pgfpathlineto{\pgfqpoint{5.710884in}{0.691681in}}%
\pgfpathlineto{\pgfqpoint{5.716253in}{0.694703in}}%
\pgfpathlineto{\pgfqpoint{5.718401in}{0.698405in}}%
\pgfpathlineto{\pgfqpoint{5.719474in}{0.697658in}}%
\pgfpathlineto{\pgfqpoint{5.722696in}{0.696401in}}%
\pgfpathlineto{\pgfqpoint{5.723770in}{0.690967in}}%
\pgfpathlineto{\pgfqpoint{5.724844in}{0.692530in}}%
\pgfpathlineto{\pgfqpoint{5.725917in}{0.688284in}}%
\pgfpathlineto{\pgfqpoint{5.726991in}{0.689099in}}%
\pgfpathlineto{\pgfqpoint{5.730213in}{0.688624in}}%
\pgfpathlineto{\pgfqpoint{5.731287in}{0.692088in}}%
\pgfpathlineto{\pgfqpoint{5.732360in}{0.698983in}}%
\pgfpathlineto{\pgfqpoint{5.733434in}{0.696232in}}%
\pgfpathlineto{\pgfqpoint{5.734508in}{0.696096in}}%
\pgfpathlineto{\pgfqpoint{5.739877in}{0.711175in}}%
\pgfpathlineto{\pgfqpoint{5.740951in}{0.710021in}}%
\pgfpathlineto{\pgfqpoint{5.742025in}{0.712806in}}%
\pgfpathlineto{\pgfqpoint{5.747394in}{0.717662in}}%
\pgfpathlineto{\pgfqpoint{5.749542in}{0.711141in}}%
\pgfpathlineto{\pgfqpoint{5.752763in}{0.714707in}}%
\pgfpathlineto{\pgfqpoint{5.753837in}{0.712840in}}%
\pgfpathlineto{\pgfqpoint{5.754911in}{0.712466in}}%
\pgfpathlineto{\pgfqpoint{5.757059in}{0.718511in}}%
\pgfpathlineto{\pgfqpoint{5.760280in}{0.717866in}}%
\pgfpathlineto{\pgfqpoint{5.761354in}{0.721908in}}%
\pgfpathlineto{\pgfqpoint{5.762428in}{0.708017in}}%
\pgfpathlineto{\pgfqpoint{5.763502in}{0.706149in}}%
\pgfpathlineto{\pgfqpoint{5.764576in}{0.702311in}}%
\pgfpathlineto{\pgfqpoint{5.767797in}{0.701802in}}%
\pgfpathlineto{\pgfqpoint{5.768871in}{0.700443in}}%
\pgfpathlineto{\pgfqpoint{5.771019in}{0.695518in}}%
\pgfpathlineto{\pgfqpoint{5.772092in}{0.699934in}}%
\pgfpathlineto{\pgfqpoint{5.775314in}{0.697998in}}%
\pgfpathlineto{\pgfqpoint{5.777462in}{0.700205in}}%
\pgfpathlineto{\pgfqpoint{5.778535in}{0.700035in}}%
\pgfpathlineto{\pgfqpoint{5.779609in}{0.701564in}}%
\pgfpathlineto{\pgfqpoint{5.783905in}{0.698134in}}%
\pgfpathlineto{\pgfqpoint{5.784979in}{0.695858in}}%
\pgfpathlineto{\pgfqpoint{5.787126in}{0.696605in}}%
\pgfpathlineto{\pgfqpoint{5.790348in}{0.696707in}}%
\pgfpathlineto{\pgfqpoint{5.792495in}{0.694330in}}%
\pgfpathlineto{\pgfqpoint{5.793569in}{0.693684in}}%
\pgfpathlineto{\pgfqpoint{5.794643in}{0.692054in}}%
\pgfpathlineto{\pgfqpoint{5.797865in}{0.692801in}}%
\pgfpathlineto{\pgfqpoint{5.798938in}{0.695484in}}%
\pgfpathlineto{\pgfqpoint{5.800012in}{0.695077in}}%
\pgfpathlineto{\pgfqpoint{5.801086in}{0.695417in}}%
\pgfpathlineto{\pgfqpoint{5.802160in}{0.697420in}}%
\pgfpathlineto{\pgfqpoint{5.805381in}{0.699254in}}%
\pgfpathlineto{\pgfqpoint{5.806455in}{0.696843in}}%
\pgfpathlineto{\pgfqpoint{5.807529in}{0.696741in}}%
\pgfpathlineto{\pgfqpoint{5.808603in}{0.697488in}}%
\pgfpathlineto{\pgfqpoint{5.809677in}{0.686858in}}%
\pgfpathlineto{\pgfqpoint{5.812898in}{0.682545in}}%
\pgfpathlineto{\pgfqpoint{5.815046in}{0.690899in}}%
\pgfpathlineto{\pgfqpoint{5.816120in}{0.693617in}}%
\pgfpathlineto{\pgfqpoint{5.817194in}{0.694296in}}%
\pgfpathlineto{\pgfqpoint{5.821489in}{0.693107in}}%
\pgfpathlineto{\pgfqpoint{5.824711in}{0.701768in}}%
\pgfpathlineto{\pgfqpoint{5.829006in}{0.703636in}}%
\pgfpathlineto{\pgfqpoint{5.830080in}{0.702583in}}%
\pgfpathlineto{\pgfqpoint{5.831154in}{0.702888in}}%
\pgfpathlineto{\pgfqpoint{5.832227in}{0.702345in}}%
\pgfpathlineto{\pgfqpoint{5.835449in}{0.703466in}}%
\pgfpathlineto{\pgfqpoint{5.836523in}{0.701258in}}%
\pgfpathlineto{\pgfqpoint{5.837597in}{0.697217in}}%
\pgfpathlineto{\pgfqpoint{5.839744in}{0.695552in}}%
\pgfpathlineto{\pgfqpoint{5.842966in}{0.694500in}}%
\pgfpathlineto{\pgfqpoint{5.845113in}{0.691069in}}%
\pgfpathlineto{\pgfqpoint{5.846187in}{0.689711in}}%
\pgfpathlineto{\pgfqpoint{5.847261in}{0.689847in}}%
\pgfpathlineto{\pgfqpoint{5.850483in}{0.688522in}}%
\pgfpathlineto{\pgfqpoint{5.851557in}{0.686790in}}%
\pgfpathlineto{\pgfqpoint{5.852630in}{0.690288in}}%
\pgfpathlineto{\pgfqpoint{5.853704in}{0.687265in}}%
\pgfpathlineto{\pgfqpoint{5.854778in}{0.689439in}}%
\pgfpathlineto{\pgfqpoint{5.858000in}{0.689201in}}%
\pgfpathlineto{\pgfqpoint{5.860147in}{0.696028in}}%
\pgfpathlineto{\pgfqpoint{5.861221in}{0.695756in}}%
\pgfpathlineto{\pgfqpoint{5.862295in}{0.692733in}}%
\pgfpathlineto{\pgfqpoint{5.865516in}{0.693583in}}%
\pgfpathlineto{\pgfqpoint{5.866590in}{0.692869in}}%
\pgfpathlineto{\pgfqpoint{5.867664in}{0.692835in}}%
\pgfpathlineto{\pgfqpoint{5.873033in}{0.689575in}}%
\pgfpathlineto{\pgfqpoint{5.874107in}{0.689915in}}%
\pgfpathlineto{\pgfqpoint{5.876255in}{0.688522in}}%
\pgfpathlineto{\pgfqpoint{5.877329in}{0.687435in}}%
\pgfpathlineto{\pgfqpoint{5.882698in}{0.684990in}}%
\pgfpathlineto{\pgfqpoint{5.883772in}{0.684345in}}%
\pgfpathlineto{\pgfqpoint{5.884846in}{0.684854in}}%
\pgfpathlineto{\pgfqpoint{5.890215in}{0.682579in}}%
\pgfpathlineto{\pgfqpoint{5.891289in}{0.683801in}}%
\pgfpathlineto{\pgfqpoint{5.892362in}{0.678367in}}%
\pgfpathlineto{\pgfqpoint{5.895584in}{0.682341in}}%
\pgfpathlineto{\pgfqpoint{5.897732in}{0.677858in}}%
\pgfpathlineto{\pgfqpoint{5.899879in}{0.678843in}}%
\pgfpathlineto{\pgfqpoint{5.903101in}{0.679046in}}%
\pgfpathlineto{\pgfqpoint{5.904175in}{0.680099in}}%
\pgfpathlineto{\pgfqpoint{5.905248in}{0.678265in}}%
\pgfpathlineto{\pgfqpoint{5.906322in}{0.681594in}}%
\pgfpathlineto{\pgfqpoint{5.907396in}{0.681152in}}%
\pgfpathlineto{\pgfqpoint{5.911691in}{0.676092in}}%
\pgfpathlineto{\pgfqpoint{5.912765in}{0.677654in}}%
\pgfpathlineto{\pgfqpoint{5.913839in}{0.676363in}}%
\pgfpathlineto{\pgfqpoint{5.914913in}{0.679794in}}%
\pgfpathlineto{\pgfqpoint{5.918134in}{0.678605in}}%
\pgfpathlineto{\pgfqpoint{5.919208in}{0.678945in}}%
\pgfpathlineto{\pgfqpoint{5.920282in}{0.678469in}}%
\pgfpathlineto{\pgfqpoint{5.921356in}{0.679692in}}%
\pgfpathlineto{\pgfqpoint{5.922430in}{0.678605in}}%
\pgfpathlineto{\pgfqpoint{5.925651in}{0.678605in}}%
\pgfpathlineto{\pgfqpoint{5.927799in}{0.675107in}}%
\pgfpathlineto{\pgfqpoint{5.928873in}{0.674156in}}%
\pgfpathlineto{\pgfqpoint{5.929947in}{0.674733in}}%
\pgfpathlineto{\pgfqpoint{5.933168in}{0.673205in}}%
\pgfpathlineto{\pgfqpoint{5.934242in}{0.674326in}}%
\pgfpathlineto{\pgfqpoint{5.935316in}{0.676771in}}%
\pgfpathlineto{\pgfqpoint{5.936390in}{0.677111in}}%
\pgfpathlineto{\pgfqpoint{5.937464in}{0.680371in}}%
\pgfpathlineto{\pgfqpoint{5.940685in}{0.681458in}}%
\pgfpathlineto{\pgfqpoint{5.941759in}{0.679454in}}%
\pgfpathlineto{\pgfqpoint{5.943907in}{0.683563in}}%
\pgfpathlineto{\pgfqpoint{5.944980in}{0.683020in}}%
\pgfpathlineto{\pgfqpoint{5.950350in}{0.676737in}}%
\pgfpathlineto{\pgfqpoint{5.951423in}{0.681458in}}%
\pgfpathlineto{\pgfqpoint{5.952497in}{0.678469in}}%
\pgfpathlineto{\pgfqpoint{5.955719in}{0.684888in}}%
\pgfpathlineto{\pgfqpoint{5.956793in}{0.684752in}}%
\pgfpathlineto{\pgfqpoint{5.958940in}{0.686620in}}%
\pgfpathlineto{\pgfqpoint{5.960014in}{0.695451in}}%
\pgfpathlineto{\pgfqpoint{5.963236in}{0.696232in}}%
\pgfpathlineto{\pgfqpoint{5.964310in}{0.695518in}}%
\pgfpathlineto{\pgfqpoint{5.965383in}{0.700137in}}%
\pgfpathlineto{\pgfqpoint{5.966457in}{0.700952in}}%
\pgfpathlineto{\pgfqpoint{5.967531in}{0.697285in}}%
\pgfpathlineto{\pgfqpoint{5.970753in}{0.695281in}}%
\pgfpathlineto{\pgfqpoint{5.971826in}{0.695552in}}%
\pgfpathlineto{\pgfqpoint{5.972900in}{0.697352in}}%
\pgfpathlineto{\pgfqpoint{5.975048in}{0.699152in}}%
\pgfpathlineto{\pgfqpoint{5.978269in}{0.699628in}}%
\pgfpathlineto{\pgfqpoint{5.979343in}{0.701903in}}%
\pgfpathlineto{\pgfqpoint{5.980417in}{0.700137in}}%
\pgfpathlineto{\pgfqpoint{5.981491in}{0.699560in}}%
\pgfpathlineto{\pgfqpoint{5.982565in}{0.698134in}}%
\pgfpathlineto{\pgfqpoint{5.986860in}{0.705130in}}%
\pgfpathlineto{\pgfqpoint{5.987934in}{0.709443in}}%
\pgfpathlineto{\pgfqpoint{5.990082in}{0.721398in}}%
\pgfpathlineto{\pgfqpoint{5.994377in}{0.718070in}}%
\pgfpathlineto{\pgfqpoint{5.996525in}{0.719836in}}%
\pgfpathlineto{\pgfqpoint{5.997599in}{0.718274in}}%
\pgfpathlineto{\pgfqpoint{6.000820in}{0.722825in}}%
\pgfpathlineto{\pgfqpoint{6.002968in}{0.723674in}}%
\pgfpathlineto{\pgfqpoint{6.005115in}{0.722349in}}%
\pgfpathlineto{\pgfqpoint{6.009411in}{0.722417in}}%
\pgfpathlineto{\pgfqpoint{6.010485in}{0.719564in}}%
\pgfpathlineto{\pgfqpoint{6.011558in}{0.720413in}}%
\pgfpathlineto{\pgfqpoint{6.012632in}{0.719292in}}%
\pgfpathlineto{\pgfqpoint{6.016928in}{0.725372in}}%
\pgfpathlineto{\pgfqpoint{6.018001in}{0.729821in}}%
\pgfpathlineto{\pgfqpoint{6.019075in}{0.729651in}}%
\pgfpathlineto{\pgfqpoint{6.020149in}{0.734881in}}%
\pgfpathlineto{\pgfqpoint{6.023371in}{0.732844in}}%
\pgfpathlineto{\pgfqpoint{6.024445in}{0.732912in}}%
\pgfpathlineto{\pgfqpoint{6.025518in}{0.736376in}}%
\pgfpathlineto{\pgfqpoint{6.026592in}{0.730127in}}%
\pgfpathlineto{\pgfqpoint{6.027666in}{0.731859in}}%
\pgfpathlineto{\pgfqpoint{6.031961in}{0.731553in}}%
\pgfpathlineto{\pgfqpoint{6.033035in}{0.732198in}}%
\pgfpathlineto{\pgfqpoint{6.034109in}{0.729312in}}%
\pgfpathlineto{\pgfqpoint{6.035183in}{0.730568in}}%
\pgfpathlineto{\pgfqpoint{6.038404in}{0.728768in}}%
\pgfpathlineto{\pgfqpoint{6.039478in}{0.731349in}}%
\pgfpathlineto{\pgfqpoint{6.041626in}{0.731859in}}%
\pgfpathlineto{\pgfqpoint{6.042700in}{0.735900in}}%
\pgfpathlineto{\pgfqpoint{6.045921in}{0.741266in}}%
\pgfpathlineto{\pgfqpoint{6.046995in}{0.740349in}}%
\pgfpathlineto{\pgfqpoint{6.048069in}{0.742455in}}%
\pgfpathlineto{\pgfqpoint{6.050217in}{0.739195in}}%
\pgfpathlineto{\pgfqpoint{6.053438in}{0.736817in}}%
\pgfpathlineto{\pgfqpoint{6.054512in}{0.734949in}}%
\pgfpathlineto{\pgfqpoint{6.055586in}{0.734949in}}%
\pgfpathlineto{\pgfqpoint{6.056660in}{0.736546in}}%
\pgfpathlineto{\pgfqpoint{6.057734in}{0.735798in}}%
\pgfpathlineto{\pgfqpoint{6.060955in}{0.737055in}}%
\pgfpathlineto{\pgfqpoint{6.062029in}{0.739297in}}%
\pgfpathlineto{\pgfqpoint{6.063103in}{0.738787in}}%
\pgfpathlineto{\pgfqpoint{6.064177in}{0.740519in}}%
\pgfpathlineto{\pgfqpoint{6.065250in}{0.738414in}}%
\pgfpathlineto{\pgfqpoint{6.070620in}{0.738583in}}%
\pgfpathlineto{\pgfqpoint{6.071693in}{0.737327in}}%
\pgfpathlineto{\pgfqpoint{6.072767in}{0.739263in}}%
\pgfpathlineto{\pgfqpoint{6.077063in}{0.738515in}}%
\pgfpathlineto{\pgfqpoint{6.078136in}{0.741606in}}%
\pgfpathlineto{\pgfqpoint{6.079210in}{0.740146in}}%
\pgfpathlineto{\pgfqpoint{6.080284in}{0.742251in}}%
\pgfpathlineto{\pgfqpoint{6.083506in}{0.740383in}}%
\pgfpathlineto{\pgfqpoint{6.084579in}{0.741029in}}%
\pgfpathlineto{\pgfqpoint{6.085653in}{0.740961in}}%
\pgfpathlineto{\pgfqpoint{6.086727in}{0.741572in}}%
\pgfpathlineto{\pgfqpoint{6.087801in}{0.741232in}}%
\pgfpathlineto{\pgfqpoint{6.091022in}{0.743168in}}%
\pgfpathlineto{\pgfqpoint{6.092096in}{0.745750in}}%
\pgfpathlineto{\pgfqpoint{6.094244in}{0.743814in}}%
\pgfpathlineto{\pgfqpoint{6.095318in}{0.743950in}}%
\pgfpathlineto{\pgfqpoint{6.098539in}{0.747074in}}%
\pgfpathlineto{\pgfqpoint{6.099613in}{0.743950in}}%
\pgfpathlineto{\pgfqpoint{6.101761in}{0.745546in}}%
\pgfpathlineto{\pgfqpoint{6.102835in}{0.745206in}}%
\pgfpathlineto{\pgfqpoint{6.106056in}{0.745987in}}%
\pgfpathlineto{\pgfqpoint{6.107130in}{0.748195in}}%
\pgfpathlineto{\pgfqpoint{6.108204in}{0.746497in}}%
\pgfpathlineto{\pgfqpoint{6.109278in}{0.748602in}}%
\pgfpathlineto{\pgfqpoint{6.110352in}{0.749316in}}%
\pgfpathlineto{\pgfqpoint{6.116795in}{0.748195in}}%
\pgfpathlineto{\pgfqpoint{6.117868in}{0.746633in}}%
\pgfpathlineto{\pgfqpoint{6.121090in}{0.746157in}}%
\pgfpathlineto{\pgfqpoint{6.122164in}{0.748263in}}%
\pgfpathlineto{\pgfqpoint{6.123238in}{0.748161in}}%
\pgfpathlineto{\pgfqpoint{6.128607in}{0.750572in}}%
\pgfpathlineto{\pgfqpoint{6.129681in}{0.751931in}}%
\pgfpathlineto{\pgfqpoint{6.130755in}{0.750402in}}%
\pgfpathlineto{\pgfqpoint{6.131828in}{0.753867in}}%
\pgfpathlineto{\pgfqpoint{6.132902in}{0.752746in}}%
\pgfpathlineto{\pgfqpoint{6.136124in}{0.750301in}}%
\pgfpathlineto{\pgfqpoint{6.137198in}{0.755157in}}%
\pgfpathlineto{\pgfqpoint{6.139345in}{0.757331in}}%
\pgfpathlineto{\pgfqpoint{6.143641in}{0.753697in}}%
\pgfpathlineto{\pgfqpoint{6.144714in}{0.752508in}}%
\pgfpathlineto{\pgfqpoint{6.145788in}{0.743508in}}%
\pgfpathlineto{\pgfqpoint{6.146862in}{0.742048in}}%
\pgfpathlineto{\pgfqpoint{6.147936in}{0.744731in}}%
\pgfpathlineto{\pgfqpoint{6.151157in}{0.742863in}}%
\pgfpathlineto{\pgfqpoint{6.152231in}{0.744968in}}%
\pgfpathlineto{\pgfqpoint{6.153305in}{0.737089in}}%
\pgfpathlineto{\pgfqpoint{6.154379in}{0.736817in}}%
\pgfpathlineto{\pgfqpoint{6.155453in}{0.737191in}}%
\pgfpathlineto{\pgfqpoint{6.158674in}{0.735357in}}%
\pgfpathlineto{\pgfqpoint{6.160822in}{0.726119in}}%
\pgfpathlineto{\pgfqpoint{6.161896in}{0.727342in}}%
\pgfpathlineto{\pgfqpoint{6.162970in}{0.730093in}}%
\pgfpathlineto{\pgfqpoint{6.166191in}{0.730466in}}%
\pgfpathlineto{\pgfqpoint{6.167265in}{0.728462in}}%
\pgfpathlineto{\pgfqpoint{6.168339in}{0.730704in}}%
\pgfpathlineto{\pgfqpoint{6.169413in}{0.729278in}}%
\pgfpathlineto{\pgfqpoint{6.170487in}{0.733013in}}%
\pgfpathlineto{\pgfqpoint{6.174782in}{0.732776in}}%
\pgfpathlineto{\pgfqpoint{6.175856in}{0.731451in}}%
\pgfpathlineto{\pgfqpoint{6.176930in}{0.732334in}}%
\pgfpathlineto{\pgfqpoint{6.178003in}{0.728972in}}%
\pgfpathlineto{\pgfqpoint{6.181225in}{0.726832in}}%
\pgfpathlineto{\pgfqpoint{6.182299in}{0.723470in}}%
\pgfpathlineto{\pgfqpoint{6.183373in}{0.724862in}}%
\pgfpathlineto{\pgfqpoint{6.184446in}{0.719632in}}%
\pgfpathlineto{\pgfqpoint{6.185520in}{0.723877in}}%
\pgfpathlineto{\pgfqpoint{6.188742in}{0.728530in}}%
\pgfpathlineto{\pgfqpoint{6.190889in}{0.725576in}}%
\pgfpathlineto{\pgfqpoint{6.191963in}{0.725066in}}%
\pgfpathlineto{\pgfqpoint{6.193037in}{0.723504in}}%
\pgfpathlineto{\pgfqpoint{6.196259in}{0.723062in}}%
\pgfpathlineto{\pgfqpoint{6.197333in}{0.718375in}}%
\pgfpathlineto{\pgfqpoint{6.198406in}{0.721194in}}%
\pgfpathlineto{\pgfqpoint{6.199480in}{0.719292in}}%
\pgfpathlineto{\pgfqpoint{6.200554in}{0.719768in}}%
\pgfpathlineto{\pgfqpoint{6.203776in}{0.723708in}}%
\pgfpathlineto{\pgfqpoint{6.204849in}{0.723198in}}%
\pgfpathlineto{\pgfqpoint{6.205923in}{0.727919in}}%
\pgfpathlineto{\pgfqpoint{6.206997in}{0.724183in}}%
\pgfpathlineto{\pgfqpoint{6.208071in}{0.725949in}}%
\pgfpathlineto{\pgfqpoint{6.211292in}{0.729889in}}%
\pgfpathlineto{\pgfqpoint{6.213440in}{0.723911in}}%
\pgfpathlineto{\pgfqpoint{6.214514in}{0.718953in}}%
\pgfpathlineto{\pgfqpoint{6.215588in}{0.718851in}}%
\pgfpathlineto{\pgfqpoint{6.219883in}{0.720583in}}%
\pgfpathlineto{\pgfqpoint{6.220957in}{0.722349in}}%
\pgfpathlineto{\pgfqpoint{6.222031in}{0.722077in}}%
\pgfpathlineto{\pgfqpoint{6.223105in}{0.724693in}}%
\pgfpathlineto{\pgfqpoint{6.226326in}{0.723708in}}%
\pgfpathlineto{\pgfqpoint{6.229548in}{0.732198in}}%
\pgfpathlineto{\pgfqpoint{6.230622in}{0.731281in}}%
\pgfpathlineto{\pgfqpoint{6.233843in}{0.731010in}}%
\pgfpathlineto{\pgfqpoint{6.234917in}{0.729074in}}%
\pgfpathlineto{\pgfqpoint{6.235991in}{0.730806in}}%
\pgfpathlineto{\pgfqpoint{6.237065in}{0.740893in}}%
\pgfpathlineto{\pgfqpoint{6.241360in}{0.740655in}}%
\pgfpathlineto{\pgfqpoint{6.242434in}{0.742897in}}%
\pgfpathlineto{\pgfqpoint{6.243508in}{0.736512in}}%
\pgfpathlineto{\pgfqpoint{6.244581in}{0.737972in}}%
\pgfpathlineto{\pgfqpoint{6.245655in}{0.733285in}}%
\pgfpathlineto{\pgfqpoint{6.248877in}{0.728870in}}%
\pgfpathlineto{\pgfqpoint{6.249951in}{0.730942in}}%
\pgfpathlineto{\pgfqpoint{6.251024in}{0.717255in}}%
\pgfpathlineto{\pgfqpoint{6.252098in}{0.712364in}}%
\pgfpathlineto{\pgfqpoint{6.253172in}{0.714470in}}%
\pgfpathlineto{\pgfqpoint{6.256394in}{0.712534in}}%
\pgfpathlineto{\pgfqpoint{6.257467in}{0.712907in}}%
\pgfpathlineto{\pgfqpoint{6.258541in}{0.715183in}}%
\pgfpathlineto{\pgfqpoint{6.260689in}{0.710224in}}%
\pgfpathlineto{\pgfqpoint{6.263910in}{0.711787in}}%
\pgfpathlineto{\pgfqpoint{6.264984in}{0.717051in}}%
\pgfpathlineto{\pgfqpoint{6.266058in}{0.712873in}}%
\pgfpathlineto{\pgfqpoint{6.267132in}{0.712941in}}%
\pgfpathlineto{\pgfqpoint{6.268206in}{0.715862in}}%
\pgfpathlineto{\pgfqpoint{6.272501in}{0.716406in}}%
\pgfpathlineto{\pgfqpoint{6.273575in}{0.717391in}}%
\pgfpathlineto{\pgfqpoint{6.274649in}{0.711889in}}%
\pgfpathlineto{\pgfqpoint{6.275723in}{0.712873in}}%
\pgfpathlineto{\pgfqpoint{6.281092in}{0.712873in}}%
\pgfpathlineto{\pgfqpoint{6.282166in}{0.698235in}}%
\pgfpathlineto{\pgfqpoint{6.286461in}{0.698337in}}%
\pgfpathlineto{\pgfqpoint{6.288609in}{0.704043in}}%
\pgfpathlineto{\pgfqpoint{6.289683in}{0.700986in}}%
\pgfpathlineto{\pgfqpoint{6.290756in}{0.703058in}}%
\pgfpathlineto{\pgfqpoint{6.293978in}{0.701666in}}%
\pgfpathlineto{\pgfqpoint{6.295052in}{0.702753in}}%
\pgfpathlineto{\pgfqpoint{6.296126in}{0.705334in}}%
\pgfpathlineto{\pgfqpoint{6.298273in}{0.703296in}}%
\pgfpathlineto{\pgfqpoint{6.301495in}{0.706522in}}%
\pgfpathlineto{\pgfqpoint{6.302569in}{0.703398in}}%
\pgfpathlineto{\pgfqpoint{6.303643in}{0.705402in}}%
\pgfpathlineto{\pgfqpoint{6.304716in}{0.701496in}}%
\pgfpathlineto{\pgfqpoint{6.310086in}{0.710530in}}%
\pgfpathlineto{\pgfqpoint{6.316529in}{0.706522in}}%
\pgfpathlineto{\pgfqpoint{6.317602in}{0.706522in}}%
\pgfpathlineto{\pgfqpoint{6.318676in}{0.703160in}}%
\pgfpathlineto{\pgfqpoint{6.319750in}{0.697794in}}%
\pgfpathlineto{\pgfqpoint{6.320824in}{0.699288in}}%
\pgfpathlineto{\pgfqpoint{6.325119in}{0.702515in}}%
\pgfpathlineto{\pgfqpoint{6.326193in}{0.702141in}}%
\pgfpathlineto{\pgfqpoint{6.328341in}{0.705945in}}%
\pgfpathlineto{\pgfqpoint{6.332636in}{0.702277in}}%
\pgfpathlineto{\pgfqpoint{6.333710in}{0.700681in}}%
\pgfpathlineto{\pgfqpoint{6.334784in}{0.703160in}}%
\pgfpathlineto{\pgfqpoint{6.335858in}{0.702345in}}%
\pgfpathlineto{\pgfqpoint{6.340153in}{0.700715in}}%
\pgfpathlineto{\pgfqpoint{6.341227in}{0.704722in}}%
\pgfpathlineto{\pgfqpoint{6.342301in}{0.702481in}}%
\pgfpathlineto{\pgfqpoint{6.343375in}{0.703432in}}%
\pgfpathlineto{\pgfqpoint{6.347670in}{0.713213in}}%
\pgfpathlineto{\pgfqpoint{6.348744in}{0.711821in}}%
\pgfpathlineto{\pgfqpoint{6.350891in}{0.723674in}}%
\pgfpathlineto{\pgfqpoint{6.354113in}{0.723538in}}%
\pgfpathlineto{\pgfqpoint{6.355187in}{0.718375in}}%
\pgfpathlineto{\pgfqpoint{6.356261in}{0.720074in}}%
\pgfpathlineto{\pgfqpoint{6.358408in}{0.719259in}}%
\pgfpathlineto{\pgfqpoint{6.361630in}{0.716983in}}%
\pgfpathlineto{\pgfqpoint{6.362704in}{0.717798in}}%
\pgfpathlineto{\pgfqpoint{6.363777in}{0.716949in}}%
\pgfpathlineto{\pgfqpoint{6.365925in}{0.716609in}}%
\pgfpathlineto{\pgfqpoint{6.369147in}{0.717119in}}%
\pgfpathlineto{\pgfqpoint{6.370221in}{0.719157in}}%
\pgfpathlineto{\pgfqpoint{6.371294in}{0.725202in}}%
\pgfpathlineto{\pgfqpoint{6.372368in}{0.723810in}}%
\pgfpathlineto{\pgfqpoint{6.373442in}{0.725236in}}%
\pgfpathlineto{\pgfqpoint{6.376664in}{0.741606in}}%
\pgfpathlineto{\pgfqpoint{6.378811in}{0.725915in}}%
\pgfpathlineto{\pgfqpoint{6.380959in}{0.724625in}}%
\pgfpathlineto{\pgfqpoint{6.385254in}{0.735255in}}%
\pgfpathlineto{\pgfqpoint{6.386328in}{0.735866in}}%
\pgfpathlineto{\pgfqpoint{6.387402in}{0.745682in}}%
\pgfpathlineto{\pgfqpoint{6.388476in}{0.748025in}}%
\pgfpathlineto{\pgfqpoint{6.391697in}{0.747210in}}%
\pgfpathlineto{\pgfqpoint{6.392771in}{0.749825in}}%
\pgfpathlineto{\pgfqpoint{6.393845in}{0.742761in}}%
\pgfpathlineto{\pgfqpoint{6.394919in}{0.742387in}}%
\pgfpathlineto{\pgfqpoint{6.395993in}{0.739399in}}%
\pgfpathlineto{\pgfqpoint{6.400288in}{0.737565in}}%
\pgfpathlineto{\pgfqpoint{6.401362in}{0.735968in}}%
\pgfpathlineto{\pgfqpoint{6.402436in}{0.736376in}}%
\pgfpathlineto{\pgfqpoint{6.403510in}{0.735527in}}%
\pgfpathlineto{\pgfqpoint{6.403510in}{0.735527in}}%
\pgfusepath{stroke}%
\end{pgfscope}%
\begin{pgfscope}%
\pgfpathrectangle{\pgfqpoint{3.937600in}{0.385400in}}{\pgfqpoint{2.583333in}{0.400885in}}%
\pgfusepath{clip}%
\pgfsetroundcap%
\pgfsetroundjoin%
\pgfsetlinewidth{1.505625pt}%
\definecolor{currentstroke}{rgb}{0.090196,0.745098,0.811765}%
\pgfsetstrokecolor{currentstroke}%
\pgfsetdash{}{0pt}%
\pgfpathmoveto{\pgfqpoint{4.055025in}{0.495511in}}%
\pgfpathlineto{\pgfqpoint{4.056098in}{0.495511in}}%
\pgfpathlineto{\pgfqpoint{4.058246in}{0.492336in}}%
\pgfpathlineto{\pgfqpoint{4.064689in}{0.492336in}}%
\pgfpathlineto{\pgfqpoint{4.065763in}{0.491388in}}%
\pgfpathlineto{\pgfqpoint{4.087240in}{0.491388in}}%
\pgfpathlineto{\pgfqpoint{4.088314in}{0.488226in}}%
\pgfpathlineto{\pgfqpoint{4.091535in}{0.486976in}}%
\pgfpathlineto{\pgfqpoint{4.093683in}{0.484818in}}%
\pgfpathlineto{\pgfqpoint{4.094757in}{0.484131in}}%
\pgfpathlineto{\pgfqpoint{4.095831in}{0.484357in}}%
\pgfpathlineto{\pgfqpoint{4.099052in}{0.483476in}}%
\pgfpathlineto{\pgfqpoint{4.101200in}{0.484853in}}%
\pgfpathlineto{\pgfqpoint{4.103347in}{0.483570in}}%
\pgfpathlineto{\pgfqpoint{4.107643in}{0.484018in}}%
\pgfpathlineto{\pgfqpoint{4.108717in}{0.484781in}}%
\pgfpathlineto{\pgfqpoint{4.109790in}{0.484236in}}%
\pgfpathlineto{\pgfqpoint{4.110864in}{0.484662in}}%
\pgfpathlineto{\pgfqpoint{4.116233in}{0.482934in}}%
\pgfpathlineto{\pgfqpoint{4.117307in}{0.481938in}}%
\pgfpathlineto{\pgfqpoint{4.118381in}{0.482020in}}%
\pgfpathlineto{\pgfqpoint{4.121603in}{0.481177in}}%
\pgfpathlineto{\pgfqpoint{4.122676in}{0.482862in}}%
\pgfpathlineto{\pgfqpoint{4.123750in}{0.483498in}}%
\pgfpathlineto{\pgfqpoint{4.125898in}{0.482246in}}%
\pgfpathlineto{\pgfqpoint{4.129120in}{0.482000in}}%
\pgfpathlineto{\pgfqpoint{4.130193in}{0.477922in}}%
\pgfpathlineto{\pgfqpoint{4.131267in}{0.479128in}}%
\pgfpathlineto{\pgfqpoint{4.132341in}{0.479154in}}%
\pgfpathlineto{\pgfqpoint{4.133415in}{0.479798in}}%
\pgfpathlineto{\pgfqpoint{4.136636in}{0.479198in}}%
\pgfpathlineto{\pgfqpoint{4.138784in}{0.479610in}}%
\pgfpathlineto{\pgfqpoint{4.139858in}{0.479558in}}%
\pgfpathlineto{\pgfqpoint{4.140932in}{0.478701in}}%
\pgfpathlineto{\pgfqpoint{4.144153in}{0.477041in}}%
\pgfpathlineto{\pgfqpoint{4.145227in}{0.477535in}}%
\pgfpathlineto{\pgfqpoint{4.147375in}{0.480139in}}%
\pgfpathlineto{\pgfqpoint{4.148449in}{0.478325in}}%
\pgfpathlineto{\pgfqpoint{4.151670in}{0.478173in}}%
\pgfpathlineto{\pgfqpoint{4.152744in}{0.479437in}}%
\pgfpathlineto{\pgfqpoint{4.176368in}{0.479437in}}%
\pgfpathlineto{\pgfqpoint{4.177442in}{0.477886in}}%
\pgfpathlineto{\pgfqpoint{4.181738in}{0.477886in}}%
\pgfpathlineto{\pgfqpoint{4.182811in}{0.476321in}}%
\pgfpathlineto{\pgfqpoint{4.183885in}{0.476868in}}%
\pgfpathlineto{\pgfqpoint{4.190328in}{0.473193in}}%
\pgfpathlineto{\pgfqpoint{4.191402in}{0.471647in}}%
\pgfpathlineto{\pgfqpoint{4.193550in}{0.470526in}}%
\pgfpathlineto{\pgfqpoint{4.198919in}{0.471517in}}%
\pgfpathlineto{\pgfqpoint{4.201067in}{0.474171in}}%
\pgfpathlineto{\pgfqpoint{4.205362in}{0.472907in}}%
\pgfpathlineto{\pgfqpoint{4.206436in}{0.473286in}}%
\pgfpathlineto{\pgfqpoint{4.212879in}{0.470595in}}%
\pgfpathlineto{\pgfqpoint{4.213953in}{0.471159in}}%
\pgfpathlineto{\pgfqpoint{4.215027in}{0.470109in}}%
\pgfpathlineto{\pgfqpoint{4.216100in}{0.472741in}}%
\pgfpathlineto{\pgfqpoint{4.219322in}{0.472717in}}%
\pgfpathlineto{\pgfqpoint{4.223617in}{0.468930in}}%
\pgfpathlineto{\pgfqpoint{4.226839in}{0.469801in}}%
\pgfpathlineto{\pgfqpoint{4.227913in}{0.468647in}}%
\pgfpathlineto{\pgfqpoint{4.228986in}{0.468928in}}%
\pgfpathlineto{\pgfqpoint{4.230060in}{0.467057in}}%
\pgfpathlineto{\pgfqpoint{4.231134in}{0.467245in}}%
\pgfpathlineto{\pgfqpoint{4.234356in}{0.467224in}}%
\pgfpathlineto{\pgfqpoint{4.236503in}{0.466036in}}%
\pgfpathlineto{\pgfqpoint{4.237577in}{0.466648in}}%
\pgfpathlineto{\pgfqpoint{4.238651in}{0.466503in}}%
\pgfpathlineto{\pgfqpoint{4.241873in}{0.467947in}}%
\pgfpathlineto{\pgfqpoint{4.244020in}{0.465716in}}%
\pgfpathlineto{\pgfqpoint{4.245094in}{0.465818in}}%
\pgfpathlineto{\pgfqpoint{4.246168in}{0.464578in}}%
\pgfpathlineto{\pgfqpoint{4.250463in}{0.464398in}}%
\pgfpathlineto{\pgfqpoint{4.253685in}{0.465384in}}%
\pgfpathlineto{\pgfqpoint{4.256906in}{0.465465in}}%
\pgfpathlineto{\pgfqpoint{4.257980in}{0.466614in}}%
\pgfpathlineto{\pgfqpoint{4.260128in}{0.466530in}}%
\pgfpathlineto{\pgfqpoint{4.261202in}{0.465063in}}%
\pgfpathlineto{\pgfqpoint{4.264423in}{0.465663in}}%
\pgfpathlineto{\pgfqpoint{4.265497in}{0.462947in}}%
\pgfpathlineto{\pgfqpoint{4.266571in}{0.462947in}}%
\pgfpathlineto{\pgfqpoint{4.268719in}{0.464252in}}%
\pgfpathlineto{\pgfqpoint{4.271940in}{0.465332in}}%
\pgfpathlineto{\pgfqpoint{4.274088in}{0.464770in}}%
\pgfpathlineto{\pgfqpoint{4.275162in}{0.462246in}}%
\pgfpathlineto{\pgfqpoint{4.276235in}{0.461852in}}%
\pgfpathlineto{\pgfqpoint{4.279457in}{0.462075in}}%
\pgfpathlineto{\pgfqpoint{4.281605in}{0.463752in}}%
\pgfpathlineto{\pgfqpoint{4.282678in}{0.463460in}}%
\pgfpathlineto{\pgfqpoint{4.283752in}{0.462090in}}%
\pgfpathlineto{\pgfqpoint{4.288048in}{0.462032in}}%
\pgfpathlineto{\pgfqpoint{4.289121in}{0.460875in}}%
\pgfpathlineto{\pgfqpoint{4.291269in}{0.462266in}}%
\pgfpathlineto{\pgfqpoint{4.302008in}{0.460926in}}%
\pgfpathlineto{\pgfqpoint{4.303081in}{0.462253in}}%
\pgfpathlineto{\pgfqpoint{4.304155in}{0.462216in}}%
\pgfpathlineto{\pgfqpoint{4.305229in}{0.463041in}}%
\pgfpathlineto{\pgfqpoint{4.306303in}{0.462372in}}%
\pgfpathlineto{\pgfqpoint{4.310598in}{0.462259in}}%
\pgfpathlineto{\pgfqpoint{4.311672in}{0.461564in}}%
\pgfpathlineto{\pgfqpoint{4.312746in}{0.462598in}}%
\pgfpathlineto{\pgfqpoint{4.318115in}{0.462182in}}%
\pgfpathlineto{\pgfqpoint{4.320263in}{0.458550in}}%
\pgfpathlineto{\pgfqpoint{4.325632in}{0.459014in}}%
\pgfpathlineto{\pgfqpoint{4.326706in}{0.458788in}}%
\pgfpathlineto{\pgfqpoint{4.327780in}{0.457388in}}%
\pgfpathlineto{\pgfqpoint{4.328853in}{0.457772in}}%
\pgfpathlineto{\pgfqpoint{4.333149in}{0.458449in}}%
\pgfpathlineto{\pgfqpoint{4.334223in}{0.457215in}}%
\pgfpathlineto{\pgfqpoint{4.336370in}{0.457165in}}%
\pgfpathlineto{\pgfqpoint{4.339592in}{0.456883in}}%
\pgfpathlineto{\pgfqpoint{4.341740in}{0.458363in}}%
\pgfpathlineto{\pgfqpoint{4.342813in}{0.457370in}}%
\pgfpathlineto{\pgfqpoint{4.343887in}{0.457821in}}%
\pgfpathlineto{\pgfqpoint{4.348183in}{0.458806in}}%
\pgfpathlineto{\pgfqpoint{4.349256in}{0.457508in}}%
\pgfpathlineto{\pgfqpoint{4.351404in}{0.456757in}}%
\pgfpathlineto{\pgfqpoint{4.354626in}{0.457710in}}%
\pgfpathlineto{\pgfqpoint{4.357847in}{0.460798in}}%
\pgfpathlineto{\pgfqpoint{4.358921in}{0.460380in}}%
\pgfpathlineto{\pgfqpoint{4.362142in}{0.460056in}}%
\pgfpathlineto{\pgfqpoint{4.363216in}{0.459341in}}%
\pgfpathlineto{\pgfqpoint{4.364290in}{0.457673in}}%
\pgfpathlineto{\pgfqpoint{4.365364in}{0.457471in}}%
\pgfpathlineto{\pgfqpoint{4.366438in}{0.458259in}}%
\pgfpathlineto{\pgfqpoint{4.369659in}{0.458430in}}%
\pgfpathlineto{\pgfqpoint{4.370733in}{0.460027in}}%
\pgfpathlineto{\pgfqpoint{4.373955in}{0.460834in}}%
\pgfpathlineto{\pgfqpoint{4.381472in}{0.460834in}}%
\pgfpathlineto{\pgfqpoint{4.387915in}{0.459828in}}%
\pgfpathlineto{\pgfqpoint{4.429794in}{0.459828in}}%
\pgfpathlineto{\pgfqpoint{4.430868in}{0.458273in}}%
\pgfpathlineto{\pgfqpoint{4.431942in}{0.458725in}}%
\pgfpathlineto{\pgfqpoint{4.433016in}{0.457123in}}%
\pgfpathlineto{\pgfqpoint{4.434090in}{0.458561in}}%
\pgfpathlineto{\pgfqpoint{4.444828in}{0.458807in}}%
\pgfpathlineto{\pgfqpoint{4.446976in}{0.456685in}}%
\pgfpathlineto{\pgfqpoint{4.448050in}{0.456518in}}%
\pgfpathlineto{\pgfqpoint{4.449123in}{0.455014in}}%
\pgfpathlineto{\pgfqpoint{4.453419in}{0.457134in}}%
\pgfpathlineto{\pgfqpoint{4.460936in}{0.456620in}}%
\pgfpathlineto{\pgfqpoint{4.464157in}{0.454710in}}%
\pgfpathlineto{\pgfqpoint{4.468452in}{0.454151in}}%
\pgfpathlineto{\pgfqpoint{4.469526in}{0.452390in}}%
\pgfpathlineto{\pgfqpoint{4.470600in}{0.452390in}}%
\pgfpathlineto{\pgfqpoint{4.471674in}{0.451804in}}%
\pgfpathlineto{\pgfqpoint{4.475969in}{0.452322in}}%
\pgfpathlineto{\pgfqpoint{4.478117in}{0.452471in}}%
\pgfpathlineto{\pgfqpoint{4.479191in}{0.451508in}}%
\pgfpathlineto{\pgfqpoint{4.482412in}{0.452431in}}%
\pgfpathlineto{\pgfqpoint{4.484560in}{0.451579in}}%
\pgfpathlineto{\pgfqpoint{4.485634in}{0.451799in}}%
\pgfpathlineto{\pgfqpoint{4.486708in}{0.451385in}}%
\pgfpathlineto{\pgfqpoint{4.493151in}{0.451093in}}%
\pgfpathlineto{\pgfqpoint{4.494225in}{0.450121in}}%
\pgfpathlineto{\pgfqpoint{4.498520in}{0.449966in}}%
\pgfpathlineto{\pgfqpoint{4.500668in}{0.451999in}}%
\pgfpathlineto{\pgfqpoint{4.501741in}{0.451895in}}%
\pgfpathlineto{\pgfqpoint{4.504963in}{0.452785in}}%
\pgfpathlineto{\pgfqpoint{4.508185in}{0.451416in}}%
\pgfpathlineto{\pgfqpoint{4.509258in}{0.450420in}}%
\pgfpathlineto{\pgfqpoint{4.512480in}{0.449807in}}%
\pgfpathlineto{\pgfqpoint{4.513554in}{0.448939in}}%
\pgfpathlineto{\pgfqpoint{4.515701in}{0.449130in}}%
\pgfpathlineto{\pgfqpoint{4.516775in}{0.447783in}}%
\pgfpathlineto{\pgfqpoint{4.519997in}{0.447465in}}%
\pgfpathlineto{\pgfqpoint{4.521071in}{0.448121in}}%
\pgfpathlineto{\pgfqpoint{4.524292in}{0.447546in}}%
\pgfpathlineto{\pgfqpoint{4.528587in}{0.449087in}}%
\pgfpathlineto{\pgfqpoint{4.529661in}{0.448304in}}%
\pgfpathlineto{\pgfqpoint{4.530735in}{0.449070in}}%
\pgfpathlineto{\pgfqpoint{4.531809in}{0.448479in}}%
\pgfpathlineto{\pgfqpoint{4.535030in}{0.449182in}}%
\pgfpathlineto{\pgfqpoint{4.536104in}{0.448658in}}%
\pgfpathlineto{\pgfqpoint{4.537178in}{0.448862in}}%
\pgfpathlineto{\pgfqpoint{4.538252in}{0.448438in}}%
\pgfpathlineto{\pgfqpoint{4.542547in}{0.448573in}}%
\pgfpathlineto{\pgfqpoint{4.543621in}{0.447625in}}%
\pgfpathlineto{\pgfqpoint{4.544695in}{0.447875in}}%
\pgfpathlineto{\pgfqpoint{4.554360in}{0.444031in}}%
\pgfpathlineto{\pgfqpoint{4.557581in}{0.445847in}}%
\pgfpathlineto{\pgfqpoint{4.558655in}{0.443702in}}%
\pgfpathlineto{\pgfqpoint{4.559729in}{0.443785in}}%
\pgfpathlineto{\pgfqpoint{4.560803in}{0.444528in}}%
\pgfpathlineto{\pgfqpoint{4.561876in}{0.442790in}}%
\pgfpathlineto{\pgfqpoint{4.568319in}{0.442319in}}%
\pgfpathlineto{\pgfqpoint{4.569393in}{0.442455in}}%
\pgfpathlineto{\pgfqpoint{4.576910in}{0.439511in}}%
\pgfpathlineto{\pgfqpoint{4.580132in}{0.439263in}}%
\pgfpathlineto{\pgfqpoint{4.581206in}{0.438320in}}%
\pgfpathlineto{\pgfqpoint{4.582279in}{0.438390in}}%
\pgfpathlineto{\pgfqpoint{4.584427in}{0.437295in}}%
\pgfpathlineto{\pgfqpoint{4.589796in}{0.436884in}}%
\pgfpathlineto{\pgfqpoint{4.590870in}{0.437925in}}%
\pgfpathlineto{\pgfqpoint{4.591944in}{0.437827in}}%
\pgfpathlineto{\pgfqpoint{4.599461in}{0.438813in}}%
\pgfpathlineto{\pgfqpoint{4.603756in}{0.437702in}}%
\pgfpathlineto{\pgfqpoint{4.604830in}{0.438082in}}%
\pgfpathlineto{\pgfqpoint{4.606978in}{0.441023in}}%
\pgfpathlineto{\pgfqpoint{4.611273in}{0.439774in}}%
\pgfpathlineto{\pgfqpoint{4.612347in}{0.440944in}}%
\pgfpathlineto{\pgfqpoint{4.613421in}{0.440933in}}%
\pgfpathlineto{\pgfqpoint{4.614495in}{0.439230in}}%
\pgfpathlineto{\pgfqpoint{4.619864in}{0.441002in}}%
\pgfpathlineto{\pgfqpoint{4.620938in}{0.439707in}}%
\pgfpathlineto{\pgfqpoint{4.622011in}{0.440177in}}%
\pgfpathlineto{\pgfqpoint{4.626307in}{0.438752in}}%
\pgfpathlineto{\pgfqpoint{4.627381in}{0.439642in}}%
\pgfpathlineto{\pgfqpoint{4.628454in}{0.441884in}}%
\pgfpathlineto{\pgfqpoint{4.629528in}{0.441110in}}%
\pgfpathlineto{\pgfqpoint{4.633824in}{0.441273in}}%
\pgfpathlineto{\pgfqpoint{4.635971in}{0.440128in}}%
\pgfpathlineto{\pgfqpoint{4.637045in}{0.440680in}}%
\pgfpathlineto{\pgfqpoint{4.640267in}{0.439902in}}%
\pgfpathlineto{\pgfqpoint{4.641340in}{0.440555in}}%
\pgfpathlineto{\pgfqpoint{4.644562in}{0.439998in}}%
\pgfpathlineto{\pgfqpoint{4.649931in}{0.438957in}}%
\pgfpathlineto{\pgfqpoint{4.651005in}{0.437393in}}%
\pgfpathlineto{\pgfqpoint{4.652079in}{0.437034in}}%
\pgfpathlineto{\pgfqpoint{4.657448in}{0.438471in}}%
\pgfpathlineto{\pgfqpoint{4.658522in}{0.438048in}}%
\pgfpathlineto{\pgfqpoint{4.659596in}{0.438643in}}%
\pgfpathlineto{\pgfqpoint{4.663891in}{0.439301in}}%
\pgfpathlineto{\pgfqpoint{4.667113in}{0.438794in}}%
\pgfpathlineto{\pgfqpoint{4.672482in}{0.439104in}}%
\pgfpathlineto{\pgfqpoint{4.674629in}{0.437369in}}%
\pgfpathlineto{\pgfqpoint{4.677851in}{0.437806in}}%
\pgfpathlineto{\pgfqpoint{4.678925in}{0.436879in}}%
\pgfpathlineto{\pgfqpoint{4.679999in}{0.437873in}}%
\pgfpathlineto{\pgfqpoint{4.681073in}{0.438021in}}%
\pgfpathlineto{\pgfqpoint{4.682146in}{0.438945in}}%
\pgfpathlineto{\pgfqpoint{4.687516in}{0.439671in}}%
\pgfpathlineto{\pgfqpoint{4.688589in}{0.441182in}}%
\pgfpathlineto{\pgfqpoint{4.695032in}{0.442452in}}%
\pgfpathlineto{\pgfqpoint{4.697180in}{0.441834in}}%
\pgfpathlineto{\pgfqpoint{4.700402in}{0.442217in}}%
\pgfpathlineto{\pgfqpoint{4.701475in}{0.442912in}}%
\pgfpathlineto{\pgfqpoint{4.704697in}{0.442758in}}%
\pgfpathlineto{\pgfqpoint{4.715435in}{0.441955in}}%
\pgfpathlineto{\pgfqpoint{4.719731in}{0.436996in}}%
\pgfpathlineto{\pgfqpoint{4.725100in}{0.436631in}}%
\pgfpathlineto{\pgfqpoint{4.727248in}{0.438482in}}%
\pgfpathlineto{\pgfqpoint{4.732617in}{0.438999in}}%
\pgfpathlineto{\pgfqpoint{4.733691in}{0.438257in}}%
\pgfpathlineto{\pgfqpoint{4.740134in}{0.438572in}}%
\pgfpathlineto{\pgfqpoint{4.741207in}{0.439374in}}%
\pgfpathlineto{\pgfqpoint{4.742281in}{0.438154in}}%
\pgfpathlineto{\pgfqpoint{4.747651in}{0.439744in}}%
\pgfpathlineto{\pgfqpoint{4.748724in}{0.437822in}}%
\pgfpathlineto{\pgfqpoint{4.749798in}{0.437256in}}%
\pgfpathlineto{\pgfqpoint{4.754094in}{0.437045in}}%
\pgfpathlineto{\pgfqpoint{4.756241in}{0.437064in}}%
\pgfpathlineto{\pgfqpoint{4.757315in}{0.436416in}}%
\pgfpathlineto{\pgfqpoint{4.760537in}{0.436019in}}%
\pgfpathlineto{\pgfqpoint{4.761610in}{0.434833in}}%
\pgfpathlineto{\pgfqpoint{4.762684in}{0.435550in}}%
\pgfpathlineto{\pgfqpoint{4.764832in}{0.434599in}}%
\pgfpathlineto{\pgfqpoint{4.776644in}{0.434924in}}%
\pgfpathlineto{\pgfqpoint{4.777718in}{0.434807in}}%
\pgfpathlineto{\pgfqpoint{4.778792in}{0.436320in}}%
\pgfpathlineto{\pgfqpoint{4.779866in}{0.435083in}}%
\pgfpathlineto{\pgfqpoint{4.784161in}{0.435756in}}%
\pgfpathlineto{\pgfqpoint{4.786309in}{0.433877in}}%
\pgfpathlineto{\pgfqpoint{4.790604in}{0.434294in}}%
\pgfpathlineto{\pgfqpoint{4.792752in}{0.434514in}}%
\pgfpathlineto{\pgfqpoint{4.794899in}{0.433728in}}%
\pgfpathlineto{\pgfqpoint{4.798121in}{0.434090in}}%
\pgfpathlineto{\pgfqpoint{4.799195in}{0.432936in}}%
\pgfpathlineto{\pgfqpoint{4.802416in}{0.433433in}}%
\pgfpathlineto{\pgfqpoint{4.805638in}{0.433142in}}%
\pgfpathlineto{\pgfqpoint{4.806712in}{0.433920in}}%
\pgfpathlineto{\pgfqpoint{4.808859in}{0.433660in}}%
\pgfpathlineto{\pgfqpoint{4.809933in}{0.432694in}}%
\pgfpathlineto{\pgfqpoint{4.817450in}{0.433430in}}%
\pgfpathlineto{\pgfqpoint{4.821745in}{0.432612in}}%
\pgfpathlineto{\pgfqpoint{4.823893in}{0.430861in}}%
\pgfpathlineto{\pgfqpoint{4.824967in}{0.431273in}}%
\pgfpathlineto{\pgfqpoint{4.832484in}{0.429863in}}%
\pgfpathlineto{\pgfqpoint{4.836779in}{0.428453in}}%
\pgfpathlineto{\pgfqpoint{4.845370in}{0.429224in}}%
\pgfpathlineto{\pgfqpoint{4.850739in}{0.430530in}}%
\pgfpathlineto{\pgfqpoint{4.851813in}{0.429691in}}%
\pgfpathlineto{\pgfqpoint{4.855034in}{0.430021in}}%
\pgfpathlineto{\pgfqpoint{4.859330in}{0.429862in}}%
\pgfpathlineto{\pgfqpoint{4.860404in}{0.429097in}}%
\pgfpathlineto{\pgfqpoint{4.861477in}{0.429446in}}%
\pgfpathlineto{\pgfqpoint{4.862551in}{0.430857in}}%
\pgfpathlineto{\pgfqpoint{4.865773in}{0.431200in}}%
\pgfpathlineto{\pgfqpoint{4.866847in}{0.430742in}}%
\pgfpathlineto{\pgfqpoint{4.867920in}{0.431853in}}%
\pgfpathlineto{\pgfqpoint{4.868994in}{0.430434in}}%
\pgfpathlineto{\pgfqpoint{4.873290in}{0.432765in}}%
\pgfpathlineto{\pgfqpoint{4.875437in}{0.431403in}}%
\pgfpathlineto{\pgfqpoint{4.876511in}{0.428590in}}%
\pgfpathlineto{\pgfqpoint{4.877585in}{0.428518in}}%
\pgfpathlineto{\pgfqpoint{4.885102in}{0.426231in}}%
\pgfpathlineto{\pgfqpoint{4.900136in}{0.425271in}}%
\pgfpathlineto{\pgfqpoint{4.903357in}{0.426056in}}%
\pgfpathlineto{\pgfqpoint{4.905505in}{0.424154in}}%
\pgfpathlineto{\pgfqpoint{4.906579in}{0.423788in}}%
\pgfpathlineto{\pgfqpoint{4.907652in}{0.424407in}}%
\pgfpathlineto{\pgfqpoint{4.913022in}{0.424871in}}%
\pgfpathlineto{\pgfqpoint{4.914095in}{0.425700in}}%
\pgfpathlineto{\pgfqpoint{4.918391in}{0.424838in}}%
\pgfpathlineto{\pgfqpoint{4.919465in}{0.424492in}}%
\pgfpathlineto{\pgfqpoint{4.920539in}{0.425319in}}%
\pgfpathlineto{\pgfqpoint{4.922686in}{0.425418in}}%
\pgfpathlineto{\pgfqpoint{4.930203in}{0.426216in}}%
\pgfpathlineto{\pgfqpoint{4.933425in}{0.425561in}}%
\pgfpathlineto{\pgfqpoint{4.934498in}{0.424675in}}%
\pgfpathlineto{\pgfqpoint{4.936646in}{0.424607in}}%
\pgfpathlineto{\pgfqpoint{4.937720in}{0.425323in}}%
\pgfpathlineto{\pgfqpoint{4.942015in}{0.425814in}}%
\pgfpathlineto{\pgfqpoint{4.943089in}{0.425279in}}%
\pgfpathlineto{\pgfqpoint{4.944163in}{0.427006in}}%
\pgfpathlineto{\pgfqpoint{4.945237in}{0.427318in}}%
\pgfpathlineto{\pgfqpoint{4.949532in}{0.426906in}}%
\pgfpathlineto{\pgfqpoint{4.951680in}{0.425468in}}%
\pgfpathlineto{\pgfqpoint{4.958123in}{0.425743in}}%
\pgfpathlineto{\pgfqpoint{4.959197in}{0.425698in}}%
\pgfpathlineto{\pgfqpoint{4.960271in}{0.426508in}}%
\pgfpathlineto{\pgfqpoint{4.963492in}{0.426788in}}%
\pgfpathlineto{\pgfqpoint{4.967787in}{0.425248in}}%
\pgfpathlineto{\pgfqpoint{4.972083in}{0.424822in}}%
\pgfpathlineto{\pgfqpoint{4.973157in}{0.425250in}}%
\pgfpathlineto{\pgfqpoint{4.975304in}{0.424284in}}%
\pgfpathlineto{\pgfqpoint{4.979600in}{0.424209in}}%
\pgfpathlineto{\pgfqpoint{4.981747in}{0.425296in}}%
\pgfpathlineto{\pgfqpoint{4.989264in}{0.424033in}}%
\pgfpathlineto{\pgfqpoint{4.990338in}{0.423493in}}%
\pgfpathlineto{\pgfqpoint{5.010741in}{0.422945in}}%
\pgfpathlineto{\pgfqpoint{5.011815in}{0.423746in}}%
\pgfpathlineto{\pgfqpoint{5.024701in}{0.423802in}}%
\pgfpathlineto{\pgfqpoint{5.027922in}{0.422392in}}%
\pgfpathlineto{\pgfqpoint{5.046178in}{0.421625in}}%
\pgfpathlineto{\pgfqpoint{5.049399in}{0.422504in}}%
\pgfpathlineto{\pgfqpoint{5.050473in}{0.422094in}}%
\pgfpathlineto{\pgfqpoint{5.055842in}{0.421977in}}%
\pgfpathlineto{\pgfqpoint{5.056916in}{0.421593in}}%
\pgfpathlineto{\pgfqpoint{5.057990in}{0.421879in}}%
\pgfpathlineto{\pgfqpoint{5.061211in}{0.421409in}}%
\pgfpathlineto{\pgfqpoint{5.062285in}{0.421880in}}%
\pgfpathlineto{\pgfqpoint{5.063359in}{0.421372in}}%
\pgfpathlineto{\pgfqpoint{5.065507in}{0.422281in}}%
\pgfpathlineto{\pgfqpoint{5.069802in}{0.421564in}}%
\pgfpathlineto{\pgfqpoint{5.073024in}{0.421498in}}%
\pgfpathlineto{\pgfqpoint{5.094500in}{0.419862in}}%
\pgfpathlineto{\pgfqpoint{5.095574in}{0.420026in}}%
\pgfpathlineto{\pgfqpoint{5.103091in}{0.419530in}}%
\pgfpathlineto{\pgfqpoint{5.117051in}{0.419793in}}%
\pgfpathlineto{\pgfqpoint{5.118125in}{0.419723in}}%
\pgfpathlineto{\pgfqpoint{5.125642in}{0.420508in}}%
\pgfpathlineto{\pgfqpoint{5.129937in}{0.420369in}}%
\pgfpathlineto{\pgfqpoint{5.132085in}{0.421442in}}%
\pgfpathlineto{\pgfqpoint{5.133159in}{0.420616in}}%
\pgfpathlineto{\pgfqpoint{5.136380in}{0.420564in}}%
\pgfpathlineto{\pgfqpoint{5.137454in}{0.421255in}}%
\pgfpathlineto{\pgfqpoint{5.138528in}{0.420765in}}%
\pgfpathlineto{\pgfqpoint{5.139602in}{0.421920in}}%
\pgfpathlineto{\pgfqpoint{5.140675in}{0.421634in}}%
\pgfpathlineto{\pgfqpoint{5.147118in}{0.423971in}}%
\pgfpathlineto{\pgfqpoint{5.148192in}{0.422806in}}%
\pgfpathlineto{\pgfqpoint{5.151414in}{0.421906in}}%
\pgfpathlineto{\pgfqpoint{5.152488in}{0.420876in}}%
\pgfpathlineto{\pgfqpoint{5.153561in}{0.421092in}}%
\pgfpathlineto{\pgfqpoint{5.155709in}{0.420355in}}%
\pgfpathlineto{\pgfqpoint{5.158931in}{0.420432in}}%
\pgfpathlineto{\pgfqpoint{5.160004in}{0.419727in}}%
\pgfpathlineto{\pgfqpoint{5.162152in}{0.419590in}}%
\pgfpathlineto{\pgfqpoint{5.163226in}{0.419061in}}%
\pgfpathlineto{\pgfqpoint{5.166448in}{0.418912in}}%
\pgfpathlineto{\pgfqpoint{5.167521in}{0.419514in}}%
\pgfpathlineto{\pgfqpoint{5.169669in}{0.418767in}}%
\pgfpathlineto{\pgfqpoint{5.170743in}{0.419651in}}%
\pgfpathlineto{\pgfqpoint{5.177186in}{0.419421in}}%
\pgfpathlineto{\pgfqpoint{5.178260in}{0.419279in}}%
\pgfpathlineto{\pgfqpoint{5.188998in}{0.419362in}}%
\pgfpathlineto{\pgfqpoint{5.191146in}{0.418757in}}%
\pgfpathlineto{\pgfqpoint{5.200810in}{0.417954in}}%
\pgfpathlineto{\pgfqpoint{5.205106in}{0.418301in}}%
\pgfpathlineto{\pgfqpoint{5.206180in}{0.418868in}}%
\pgfpathlineto{\pgfqpoint{5.207253in}{0.418299in}}%
\pgfpathlineto{\pgfqpoint{5.213696in}{0.418446in}}%
\pgfpathlineto{\pgfqpoint{5.215844in}{0.417788in}}%
\pgfpathlineto{\pgfqpoint{5.223361in}{0.416887in}}%
\pgfpathlineto{\pgfqpoint{5.227656in}{0.417018in}}%
\pgfpathlineto{\pgfqpoint{5.235173in}{0.418188in}}%
\pgfpathlineto{\pgfqpoint{5.238395in}{0.417182in}}%
\pgfpathlineto{\pgfqpoint{5.256650in}{0.416872in}}%
\pgfpathlineto{\pgfqpoint{5.260945in}{0.418535in}}%
\pgfpathlineto{\pgfqpoint{5.264167in}{0.418106in}}%
\pgfpathlineto{\pgfqpoint{5.265241in}{0.417164in}}%
\pgfpathlineto{\pgfqpoint{5.266315in}{0.414200in}}%
\pgfpathlineto{\pgfqpoint{5.267388in}{0.413720in}}%
\pgfpathlineto{\pgfqpoint{5.271684in}{0.414034in}}%
\pgfpathlineto{\pgfqpoint{5.274905in}{0.413384in}}%
\pgfpathlineto{\pgfqpoint{5.275979in}{0.413185in}}%
\pgfpathlineto{\pgfqpoint{5.291013in}{0.413205in}}%
\pgfpathlineto{\pgfqpoint{5.296382in}{0.412702in}}%
\pgfpathlineto{\pgfqpoint{5.298530in}{0.413276in}}%
\pgfpathlineto{\pgfqpoint{5.301751in}{0.412791in}}%
\pgfpathlineto{\pgfqpoint{5.303899in}{0.413569in}}%
\pgfpathlineto{\pgfqpoint{5.304973in}{0.412103in}}%
\pgfpathlineto{\pgfqpoint{5.306047in}{0.412332in}}%
\pgfpathlineto{\pgfqpoint{5.316785in}{0.411762in}}%
\pgfpathlineto{\pgfqpoint{5.321080in}{0.412614in}}%
\pgfpathlineto{\pgfqpoint{5.327523in}{0.412437in}}%
\pgfpathlineto{\pgfqpoint{5.339336in}{0.412272in}}%
\pgfpathlineto{\pgfqpoint{5.366181in}{0.411118in}}%
\pgfpathlineto{\pgfqpoint{5.371551in}{0.411387in}}%
\pgfpathlineto{\pgfqpoint{5.373698in}{0.411052in}}%
\pgfpathlineto{\pgfqpoint{5.399470in}{0.411344in}}%
\pgfpathlineto{\pgfqpoint{5.400544in}{0.411576in}}%
\pgfpathlineto{\pgfqpoint{5.402692in}{0.410937in}}%
\pgfpathlineto{\pgfqpoint{5.403766in}{0.411136in}}%
\pgfpathlineto{\pgfqpoint{5.409135in}{0.410681in}}%
\pgfpathlineto{\pgfqpoint{5.410209in}{0.410179in}}%
\pgfpathlineto{\pgfqpoint{5.411283in}{0.410347in}}%
\pgfpathlineto{\pgfqpoint{5.418800in}{0.409681in}}%
\pgfpathlineto{\pgfqpoint{5.423095in}{0.409900in}}%
\pgfpathlineto{\pgfqpoint{5.425243in}{0.409486in}}%
\pgfpathlineto{\pgfqpoint{5.460679in}{0.407730in}}%
\pgfpathlineto{\pgfqpoint{5.461753in}{0.410428in}}%
\pgfpathlineto{\pgfqpoint{5.462827in}{0.411004in}}%
\pgfpathlineto{\pgfqpoint{5.467122in}{0.410273in}}%
\pgfpathlineto{\pgfqpoint{5.469270in}{0.411444in}}%
\pgfpathlineto{\pgfqpoint{5.474639in}{0.410807in}}%
\pgfpathlineto{\pgfqpoint{5.476787in}{0.411587in}}%
\pgfpathlineto{\pgfqpoint{5.477861in}{0.413600in}}%
\pgfpathlineto{\pgfqpoint{5.482156in}{0.415272in}}%
\pgfpathlineto{\pgfqpoint{5.483230in}{0.415066in}}%
\pgfpathlineto{\pgfqpoint{5.485378in}{0.412736in}}%
\pgfpathlineto{\pgfqpoint{5.489673in}{0.412831in}}%
\pgfpathlineto{\pgfqpoint{5.490747in}{0.413636in}}%
\pgfpathlineto{\pgfqpoint{5.491821in}{0.412789in}}%
\pgfpathlineto{\pgfqpoint{5.500411in}{0.412500in}}%
\pgfpathlineto{\pgfqpoint{5.501485in}{0.411874in}}%
\pgfpathlineto{\pgfqpoint{5.516519in}{0.413253in}}%
\pgfpathlineto{\pgfqpoint{5.520814in}{0.413548in}}%
\pgfpathlineto{\pgfqpoint{5.521888in}{0.412563in}}%
\pgfpathlineto{\pgfqpoint{5.528331in}{0.412042in}}%
\pgfpathlineto{\pgfqpoint{5.530479in}{0.411767in}}%
\pgfpathlineto{\pgfqpoint{5.531553in}{0.411461in}}%
\pgfpathlineto{\pgfqpoint{5.544439in}{0.410075in}}%
\pgfpathlineto{\pgfqpoint{5.545513in}{0.409163in}}%
\pgfpathlineto{\pgfqpoint{5.550882in}{0.409022in}}%
\pgfpathlineto{\pgfqpoint{5.554103in}{0.409025in}}%
\pgfpathlineto{\pgfqpoint{5.558399in}{0.408541in}}%
\pgfpathlineto{\pgfqpoint{5.560546in}{0.409209in}}%
\pgfpathlineto{\pgfqpoint{5.561620in}{0.408479in}}%
\pgfpathlineto{\pgfqpoint{5.568063in}{0.408333in}}%
\pgfpathlineto{\pgfqpoint{5.569137in}{0.408686in}}%
\pgfpathlineto{\pgfqpoint{5.575580in}{0.407683in}}%
\pgfpathlineto{\pgfqpoint{5.576654in}{0.407347in}}%
\pgfpathlineto{\pgfqpoint{5.582023in}{0.407684in}}%
\pgfpathlineto{\pgfqpoint{5.584171in}{0.408560in}}%
\pgfpathlineto{\pgfqpoint{5.587392in}{0.408995in}}%
\pgfpathlineto{\pgfqpoint{5.588466in}{0.408476in}}%
\pgfpathlineto{\pgfqpoint{5.590614in}{0.409404in}}%
\pgfpathlineto{\pgfqpoint{5.591688in}{0.408753in}}%
\pgfpathlineto{\pgfqpoint{5.595983in}{0.409225in}}%
\pgfpathlineto{\pgfqpoint{5.599204in}{0.410267in}}%
\pgfpathlineto{\pgfqpoint{5.602426in}{0.409881in}}%
\pgfpathlineto{\pgfqpoint{5.604574in}{0.408621in}}%
\pgfpathlineto{\pgfqpoint{5.605647in}{0.409095in}}%
\pgfpathlineto{\pgfqpoint{5.606721in}{0.410274in}}%
\pgfpathlineto{\pgfqpoint{5.613164in}{0.410829in}}%
\pgfpathlineto{\pgfqpoint{5.619607in}{0.410674in}}%
\pgfpathlineto{\pgfqpoint{5.620681in}{0.411058in}}%
\pgfpathlineto{\pgfqpoint{5.627124in}{0.412570in}}%
\pgfpathlineto{\pgfqpoint{5.629272in}{0.412950in}}%
\pgfpathlineto{\pgfqpoint{5.633567in}{0.412184in}}%
\pgfpathlineto{\pgfqpoint{5.634641in}{0.413180in}}%
\pgfpathlineto{\pgfqpoint{5.636789in}{0.412960in}}%
\pgfpathlineto{\pgfqpoint{5.687259in}{0.412960in}}%
\pgfpathlineto{\pgfqpoint{5.688333in}{0.412309in}}%
\pgfpathlineto{\pgfqpoint{5.696924in}{0.411996in}}%
\pgfpathlineto{\pgfqpoint{5.708736in}{0.411434in}}%
\pgfpathlineto{\pgfqpoint{5.731287in}{0.411381in}}%
\pgfpathlineto{\pgfqpoint{5.732360in}{0.410654in}}%
\pgfpathlineto{\pgfqpoint{5.734508in}{0.410654in}}%
\pgfpathlineto{\pgfqpoint{5.742025in}{0.409004in}}%
\pgfpathlineto{\pgfqpoint{5.748468in}{0.408923in}}%
\pgfpathlineto{\pgfqpoint{5.749542in}{0.409149in}}%
\pgfpathlineto{\pgfqpoint{5.761354in}{0.408168in}}%
\pgfpathlineto{\pgfqpoint{5.762428in}{0.409376in}}%
\pgfpathlineto{\pgfqpoint{5.769945in}{0.410100in}}%
\pgfpathlineto{\pgfqpoint{5.986860in}{0.409628in}}%
\pgfpathlineto{\pgfqpoint{5.990082in}{0.408118in}}%
\pgfpathlineto{\pgfqpoint{5.997599in}{0.408389in}}%
\pgfpathlineto{\pgfqpoint{6.004042in}{0.407951in}}%
\pgfpathlineto{\pgfqpoint{6.012632in}{0.408291in}}%
\pgfpathlineto{\pgfqpoint{6.025518in}{0.406837in}}%
\pgfpathlineto{\pgfqpoint{6.026592in}{0.407335in}}%
\pgfpathlineto{\pgfqpoint{6.027666in}{0.407192in}}%
\pgfpathlineto{\pgfqpoint{6.042700in}{0.406857in}}%
\pgfpathlineto{\pgfqpoint{6.049143in}{0.406508in}}%
\pgfpathlineto{\pgfqpoint{6.057734in}{0.406853in}}%
\pgfpathlineto{\pgfqpoint{6.098539in}{0.405967in}}%
\pgfpathlineto{\pgfqpoint{6.100687in}{0.406121in}}%
\pgfpathlineto{\pgfqpoint{6.117868in}{0.405995in}}%
\pgfpathlineto{\pgfqpoint{6.130755in}{0.405712in}}%
\pgfpathlineto{\pgfqpoint{6.132902in}{0.405537in}}%
\pgfpathlineto{\pgfqpoint{6.137198in}{0.405357in}}%
\pgfpathlineto{\pgfqpoint{6.140419in}{0.405260in}}%
\pgfpathlineto{\pgfqpoint{6.144714in}{0.405546in}}%
\pgfpathlineto{\pgfqpoint{6.146862in}{0.406313in}}%
\pgfpathlineto{\pgfqpoint{6.147936in}{0.406107in}}%
\pgfpathlineto{\pgfqpoint{6.159748in}{0.407122in}}%
\pgfpathlineto{\pgfqpoint{6.161896in}{0.407465in}}%
\pgfpathlineto{\pgfqpoint{6.162970in}{0.407235in}}%
\pgfpathlineto{\pgfqpoint{6.181225in}{0.407496in}}%
\pgfpathlineto{\pgfqpoint{6.184446in}{0.408101in}}%
\pgfpathlineto{\pgfqpoint{6.185520in}{0.407731in}}%
\pgfpathlineto{\pgfqpoint{6.190889in}{0.407581in}}%
\pgfpathlineto{\pgfqpoint{6.200554in}{0.407549in}}%
\pgfpathlineto{\pgfqpoint{6.211292in}{0.406690in}}%
\pgfpathlineto{\pgfqpoint{6.215588in}{0.407173in}}%
\pgfpathlineto{\pgfqpoint{6.226326in}{0.406841in}}%
\pgfpathlineto{\pgfqpoint{6.230622in}{0.406222in}}%
\pgfpathlineto{\pgfqpoint{6.235991in}{0.406258in}}%
\pgfpathlineto{\pgfqpoint{6.237065in}{0.405458in}}%
\pgfpathlineto{\pgfqpoint{6.245655in}{0.406026in}}%
\pgfpathlineto{\pgfqpoint{6.252098in}{0.406205in}}%
\pgfpathlineto{\pgfqpoint{6.372368in}{0.405710in}}%
\pgfpathlineto{\pgfqpoint{6.373442in}{0.405597in}}%
\pgfpathlineto{\pgfqpoint{6.376664in}{0.404309in}}%
\pgfpathlineto{\pgfqpoint{6.378811in}{0.405455in}}%
\pgfpathlineto{\pgfqpoint{6.380959in}{0.405506in}}%
\pgfpathlineto{\pgfqpoint{6.391697in}{0.403803in}}%
\pgfpathlineto{\pgfqpoint{6.392771in}{0.403622in}}%
\pgfpathlineto{\pgfqpoint{6.394919in}{0.404130in}}%
\pgfpathlineto{\pgfqpoint{6.395993in}{0.404342in}}%
\pgfpathlineto{\pgfqpoint{6.403510in}{0.404623in}}%
\pgfpathlineto{\pgfqpoint{6.403510in}{0.404623in}}%
\pgfusepath{stroke}%
\end{pgfscope}%
\begin{pgfscope}%
\pgfsetrectcap%
\pgfsetmiterjoin%
\pgfsetlinewidth{0.803000pt}%
\definecolor{currentstroke}{rgb}{1.000000,1.000000,1.000000}%
\pgfsetstrokecolor{currentstroke}%
\pgfsetdash{}{0pt}%
\pgfpathmoveto{\pgfqpoint{3.937600in}{0.385400in}}%
\pgfpathlineto{\pgfqpoint{3.937600in}{0.786285in}}%
\pgfusepath{stroke}%
\end{pgfscope}%
\begin{pgfscope}%
\pgfsetrectcap%
\pgfsetmiterjoin%
\pgfsetlinewidth{0.803000pt}%
\definecolor{currentstroke}{rgb}{1.000000,1.000000,1.000000}%
\pgfsetstrokecolor{currentstroke}%
\pgfsetdash{}{0pt}%
\pgfpathmoveto{\pgfqpoint{6.520934in}{0.385400in}}%
\pgfpathlineto{\pgfqpoint{6.520934in}{0.786285in}}%
\pgfusepath{stroke}%
\end{pgfscope}%
\begin{pgfscope}%
\pgfsetrectcap%
\pgfsetmiterjoin%
\pgfsetlinewidth{0.803000pt}%
\definecolor{currentstroke}{rgb}{1.000000,1.000000,1.000000}%
\pgfsetstrokecolor{currentstroke}%
\pgfsetdash{}{0pt}%
\pgfpathmoveto{\pgfqpoint{3.937600in}{0.385400in}}%
\pgfpathlineto{\pgfqpoint{6.520934in}{0.385400in}}%
\pgfusepath{stroke}%
\end{pgfscope}%
\begin{pgfscope}%
\pgfsetrectcap%
\pgfsetmiterjoin%
\pgfsetlinewidth{0.803000pt}%
\definecolor{currentstroke}{rgb}{1.000000,1.000000,1.000000}%
\pgfsetstrokecolor{currentstroke}%
\pgfsetdash{}{0pt}%
\pgfpathmoveto{\pgfqpoint{3.937600in}{0.786285in}}%
\pgfpathlineto{\pgfqpoint{6.520934in}{0.786285in}}%
\pgfusepath{stroke}%
\end{pgfscope}%
\begin{pgfscope}%
\definecolor{textcolor}{rgb}{0.150000,0.150000,0.150000}%
\pgfsetstrokecolor{textcolor}%
\pgfsetfillcolor{textcolor}%
\pgftext[x=5.229267in,y=0.869619in,,base]{\color{textcolor}\rmfamily\fontsize{16.800000}{20.160000}\selectfont DIS}%
\end{pgfscope}%
\begin{pgfscope}%
\definecolor{textcolor}{rgb}{0.150000,0.150000,0.150000}%
\pgfsetstrokecolor{textcolor}%
\pgfsetfillcolor{textcolor}%
\pgftext[x=3.320934in,y=5.834781in,,top]{\color{textcolor}\rmfamily\fontsize{16.800000}{20.160000}\selectfont Return Trading Strategy vs. Buy and Hold}%
\end{pgfscope}%
\end{pgfpicture}%
\makeatother%
\endgroup%

        \end{adjustbox}
        %\caption{Flower two.}
    \end{minipage}
    \caption{Caption}
    \label{fig:mean_reversion_cum_mean}
\end{figure}{}

\subsubsection{Cumulative Mean - Returns}

\begin{figure}
    \centering
    \begin{adjustbox}{width=.7\textwidth,center}
        %% Creator: Matplotlib, PGF backend
%%
%% To include the figure in your LaTeX document, write
%%   \input{<filename>.pgf}
%%
%% Make sure the required packages are loaded in your preamble
%%   \usepackage{pgf}
%%
%% Figures using additional raster images can only be included by \input if
%% they are in the same directory as the main LaTeX file. For loading figures
%% from other directories you can use the `import` package
%%   \usepackage{import}
%% and then include the figures with
%%   \import{<path to file>}{<filename>.pgf}
%%
%% Matplotlib used the following preamble
%%   \usepackage{fontspec}
%%   \setmainfont{DejaVuSerif.ttf}[Path=/opt/tljh/user/lib/python3.6/site-packages/matplotlib/mpl-data/fonts/ttf/]
%%   \setsansfont{DejaVuSans.ttf}[Path=/opt/tljh/user/lib/python3.6/site-packages/matplotlib/mpl-data/fonts/ttf/]
%%   \setmonofont{DejaVuSansMono.ttf}[Path=/opt/tljh/user/lib/python3.6/site-packages/matplotlib/mpl-data/fonts/ttf/]
%%
\begingroup%
\makeatletter%
\begin{pgfpicture}%
\pgfpathrectangle{\pgfpointorigin}{\pgfqpoint{6.939674in}{5.934781in}}%
\pgfusepath{use as bounding box, clip}%
\begin{pgfscope}%
\pgfsetbuttcap%
\pgfsetmiterjoin%
\definecolor{currentfill}{rgb}{1.000000,1.000000,1.000000}%
\pgfsetfillcolor{currentfill}%
\pgfsetlinewidth{0.000000pt}%
\definecolor{currentstroke}{rgb}{1.000000,1.000000,1.000000}%
\pgfsetstrokecolor{currentstroke}%
\pgfsetdash{}{0pt}%
\pgfpathmoveto{\pgfqpoint{0.000000in}{0.000000in}}%
\pgfpathlineto{\pgfqpoint{6.939674in}{0.000000in}}%
\pgfpathlineto{\pgfqpoint{6.939674in}{5.934781in}}%
\pgfpathlineto{\pgfqpoint{0.000000in}{5.934781in}}%
\pgfpathclose%
\pgfusepath{fill}%
\end{pgfscope}%
\begin{pgfscope}%
\pgfsetbuttcap%
\pgfsetmiterjoin%
\definecolor{currentfill}{rgb}{0.917647,0.917647,0.949020}%
\pgfsetfillcolor{currentfill}%
\pgfsetlinewidth{0.000000pt}%
\definecolor{currentstroke}{rgb}{0.000000,0.000000,0.000000}%
\pgfsetstrokecolor{currentstroke}%
\pgfsetstrokeopacity{0.000000}%
\pgfsetdash{}{0pt}%
\pgfpathmoveto{\pgfqpoint{0.506453in}{4.233896in}}%
\pgfpathlineto{\pgfqpoint{3.089787in}{4.233896in}}%
\pgfpathlineto{\pgfqpoint{3.089787in}{4.634781in}}%
\pgfpathlineto{\pgfqpoint{0.506453in}{4.634781in}}%
\pgfpathclose%
\pgfusepath{fill}%
\end{pgfscope}%
\begin{pgfscope}%
\pgfpathrectangle{\pgfqpoint{0.506453in}{4.233896in}}{\pgfqpoint{2.583333in}{0.400885in}}%
\pgfusepath{clip}%
\pgfsetroundcap%
\pgfsetroundjoin%
\pgfsetlinewidth{0.803000pt}%
\definecolor{currentstroke}{rgb}{1.000000,1.000000,1.000000}%
\pgfsetstrokecolor{currentstroke}%
\pgfsetdash{}{0pt}%
\pgfpathmoveto{\pgfqpoint{0.621730in}{4.233896in}}%
\pgfpathlineto{\pgfqpoint{0.621730in}{4.634781in}}%
\pgfusepath{stroke}%
\end{pgfscope}%
\begin{pgfscope}%
\definecolor{textcolor}{rgb}{0.150000,0.150000,0.150000}%
\pgfsetstrokecolor{textcolor}%
\pgfsetfillcolor{textcolor}%
\pgftext[x=0.621730in,y=4.136674in,,top]{\color{textcolor}\rmfamily\fontsize{14.000000}{16.800000}\selectfont 2012}%
\end{pgfscope}%
\begin{pgfscope}%
\pgfpathrectangle{\pgfqpoint{0.506453in}{4.233896in}}{\pgfqpoint{2.583333in}{0.400885in}}%
\pgfusepath{clip}%
\pgfsetroundcap%
\pgfsetroundjoin%
\pgfsetlinewidth{0.803000pt}%
\definecolor{currentstroke}{rgb}{1.000000,1.000000,1.000000}%
\pgfsetstrokecolor{currentstroke}%
\pgfsetdash{}{0pt}%
\pgfpathmoveto{\pgfqpoint{1.014755in}{4.233896in}}%
\pgfpathlineto{\pgfqpoint{1.014755in}{4.634781in}}%
\pgfusepath{stroke}%
\end{pgfscope}%
\begin{pgfscope}%
\definecolor{textcolor}{rgb}{0.150000,0.150000,0.150000}%
\pgfsetstrokecolor{textcolor}%
\pgfsetfillcolor{textcolor}%
\pgftext[x=1.014755in,y=4.136674in,,top]{\color{textcolor}\rmfamily\fontsize{14.000000}{16.800000}\selectfont 2013}%
\end{pgfscope}%
\begin{pgfscope}%
\pgfpathrectangle{\pgfqpoint{0.506453in}{4.233896in}}{\pgfqpoint{2.583333in}{0.400885in}}%
\pgfusepath{clip}%
\pgfsetroundcap%
\pgfsetroundjoin%
\pgfsetlinewidth{0.803000pt}%
\definecolor{currentstroke}{rgb}{1.000000,1.000000,1.000000}%
\pgfsetstrokecolor{currentstroke}%
\pgfsetdash{}{0pt}%
\pgfpathmoveto{\pgfqpoint{1.406706in}{4.233896in}}%
\pgfpathlineto{\pgfqpoint{1.406706in}{4.634781in}}%
\pgfusepath{stroke}%
\end{pgfscope}%
\begin{pgfscope}%
\definecolor{textcolor}{rgb}{0.150000,0.150000,0.150000}%
\pgfsetstrokecolor{textcolor}%
\pgfsetfillcolor{textcolor}%
\pgftext[x=1.406706in,y=4.136674in,,top]{\color{textcolor}\rmfamily\fontsize{14.000000}{16.800000}\selectfont 2014}%
\end{pgfscope}%
\begin{pgfscope}%
\pgfpathrectangle{\pgfqpoint{0.506453in}{4.233896in}}{\pgfqpoint{2.583333in}{0.400885in}}%
\pgfusepath{clip}%
\pgfsetroundcap%
\pgfsetroundjoin%
\pgfsetlinewidth{0.803000pt}%
\definecolor{currentstroke}{rgb}{1.000000,1.000000,1.000000}%
\pgfsetstrokecolor{currentstroke}%
\pgfsetdash{}{0pt}%
\pgfpathmoveto{\pgfqpoint{1.798657in}{4.233896in}}%
\pgfpathlineto{\pgfqpoint{1.798657in}{4.634781in}}%
\pgfusepath{stroke}%
\end{pgfscope}%
\begin{pgfscope}%
\definecolor{textcolor}{rgb}{0.150000,0.150000,0.150000}%
\pgfsetstrokecolor{textcolor}%
\pgfsetfillcolor{textcolor}%
\pgftext[x=1.798657in,y=4.136674in,,top]{\color{textcolor}\rmfamily\fontsize{14.000000}{16.800000}\selectfont 2015}%
\end{pgfscope}%
\begin{pgfscope}%
\pgfpathrectangle{\pgfqpoint{0.506453in}{4.233896in}}{\pgfqpoint{2.583333in}{0.400885in}}%
\pgfusepath{clip}%
\pgfsetroundcap%
\pgfsetroundjoin%
\pgfsetlinewidth{0.803000pt}%
\definecolor{currentstroke}{rgb}{1.000000,1.000000,1.000000}%
\pgfsetstrokecolor{currentstroke}%
\pgfsetdash{}{0pt}%
\pgfpathmoveto{\pgfqpoint{2.190608in}{4.233896in}}%
\pgfpathlineto{\pgfqpoint{2.190608in}{4.634781in}}%
\pgfusepath{stroke}%
\end{pgfscope}%
\begin{pgfscope}%
\definecolor{textcolor}{rgb}{0.150000,0.150000,0.150000}%
\pgfsetstrokecolor{textcolor}%
\pgfsetfillcolor{textcolor}%
\pgftext[x=2.190608in,y=4.136674in,,top]{\color{textcolor}\rmfamily\fontsize{14.000000}{16.800000}\selectfont 2016}%
\end{pgfscope}%
\begin{pgfscope}%
\pgfpathrectangle{\pgfqpoint{0.506453in}{4.233896in}}{\pgfqpoint{2.583333in}{0.400885in}}%
\pgfusepath{clip}%
\pgfsetroundcap%
\pgfsetroundjoin%
\pgfsetlinewidth{0.803000pt}%
\definecolor{currentstroke}{rgb}{1.000000,1.000000,1.000000}%
\pgfsetstrokecolor{currentstroke}%
\pgfsetdash{}{0pt}%
\pgfpathmoveto{\pgfqpoint{2.583633in}{4.233896in}}%
\pgfpathlineto{\pgfqpoint{2.583633in}{4.634781in}}%
\pgfusepath{stroke}%
\end{pgfscope}%
\begin{pgfscope}%
\definecolor{textcolor}{rgb}{0.150000,0.150000,0.150000}%
\pgfsetstrokecolor{textcolor}%
\pgfsetfillcolor{textcolor}%
\pgftext[x=2.583633in,y=4.136674in,,top]{\color{textcolor}\rmfamily\fontsize{14.000000}{16.800000}\selectfont 2017}%
\end{pgfscope}%
\begin{pgfscope}%
\pgfpathrectangle{\pgfqpoint{0.506453in}{4.233896in}}{\pgfqpoint{2.583333in}{0.400885in}}%
\pgfusepath{clip}%
\pgfsetroundcap%
\pgfsetroundjoin%
\pgfsetlinewidth{0.803000pt}%
\definecolor{currentstroke}{rgb}{1.000000,1.000000,1.000000}%
\pgfsetstrokecolor{currentstroke}%
\pgfsetdash{}{0pt}%
\pgfpathmoveto{\pgfqpoint{2.975584in}{4.233896in}}%
\pgfpathlineto{\pgfqpoint{2.975584in}{4.634781in}}%
\pgfusepath{stroke}%
\end{pgfscope}%
\begin{pgfscope}%
\definecolor{textcolor}{rgb}{0.150000,0.150000,0.150000}%
\pgfsetstrokecolor{textcolor}%
\pgfsetfillcolor{textcolor}%
\pgftext[x=2.975584in,y=4.136674in,,top]{\color{textcolor}\rmfamily\fontsize{14.000000}{16.800000}\selectfont 2018}%
\end{pgfscope}%
\begin{pgfscope}%
\pgfpathrectangle{\pgfqpoint{0.506453in}{4.233896in}}{\pgfqpoint{2.583333in}{0.400885in}}%
\pgfusepath{clip}%
\pgfsetroundcap%
\pgfsetroundjoin%
\pgfsetlinewidth{0.803000pt}%
\definecolor{currentstroke}{rgb}{1.000000,1.000000,1.000000}%
\pgfsetstrokecolor{currentstroke}%
\pgfsetdash{}{0pt}%
\pgfpathmoveto{\pgfqpoint{0.506453in}{4.251280in}}%
\pgfpathlineto{\pgfqpoint{3.089787in}{4.251280in}}%
\pgfusepath{stroke}%
\end{pgfscope}%
\begin{pgfscope}%
\definecolor{textcolor}{rgb}{0.150000,0.150000,0.150000}%
\pgfsetstrokecolor{textcolor}%
\pgfsetfillcolor{textcolor}%
\pgftext[x=0.100000in,y=4.177414in,left,base]{\color{textcolor}\rmfamily\fontsize{14.000000}{16.800000}\selectfont 0.0}%
\end{pgfscope}%
\begin{pgfscope}%
\pgfpathrectangle{\pgfqpoint{0.506453in}{4.233896in}}{\pgfqpoint{2.583333in}{0.400885in}}%
\pgfusepath{clip}%
\pgfsetroundcap%
\pgfsetroundjoin%
\pgfsetlinewidth{0.803000pt}%
\definecolor{currentstroke}{rgb}{1.000000,1.000000,1.000000}%
\pgfsetstrokecolor{currentstroke}%
\pgfsetdash{}{0pt}%
\pgfpathmoveto{\pgfqpoint{0.506453in}{4.521580in}}%
\pgfpathlineto{\pgfqpoint{3.089787in}{4.521580in}}%
\pgfusepath{stroke}%
\end{pgfscope}%
\begin{pgfscope}%
\definecolor{textcolor}{rgb}{0.150000,0.150000,0.150000}%
\pgfsetstrokecolor{textcolor}%
\pgfsetfillcolor{textcolor}%
\pgftext[x=0.100000in,y=4.447714in,left,base]{\color{textcolor}\rmfamily\fontsize{14.000000}{16.800000}\selectfont 2.5}%
\end{pgfscope}%
\begin{pgfscope}%
\pgfpathrectangle{\pgfqpoint{0.506453in}{4.233896in}}{\pgfqpoint{2.583333in}{0.400885in}}%
\pgfusepath{clip}%
\pgfsetroundcap%
\pgfsetroundjoin%
\pgfsetlinewidth{1.505625pt}%
\definecolor{currentstroke}{rgb}{0.000000,0.000000,0.000000}%
\pgfsetstrokecolor{currentstroke}%
\pgfsetdash{}{0pt}%
\pgfpathmoveto{\pgfqpoint{0.623878in}{4.359400in}}%
\pgfpathlineto{\pgfqpoint{0.624952in}{4.360301in}}%
\pgfpathlineto{\pgfqpoint{0.627099in}{4.359258in}}%
\pgfpathlineto{\pgfqpoint{0.631395in}{4.360459in}}%
\pgfpathlineto{\pgfqpoint{0.632468in}{4.359764in}}%
\pgfpathlineto{\pgfqpoint{0.633542in}{4.360428in}}%
\pgfpathlineto{\pgfqpoint{0.634616in}{4.359543in}}%
\pgfpathlineto{\pgfqpoint{0.638911in}{4.360364in}}%
\pgfpathlineto{\pgfqpoint{0.641059in}{4.362403in}}%
\pgfpathlineto{\pgfqpoint{0.642133in}{4.362198in}}%
\pgfpathlineto{\pgfqpoint{0.646428in}{4.362561in}}%
\pgfpathlineto{\pgfqpoint{0.647502in}{4.363273in}}%
\pgfpathlineto{\pgfqpoint{0.648576in}{4.364711in}}%
\pgfpathlineto{\pgfqpoint{0.653945in}{4.363573in}}%
\pgfpathlineto{\pgfqpoint{0.655019in}{4.364410in}}%
\pgfpathlineto{\pgfqpoint{0.663610in}{4.365280in}}%
\pgfpathlineto{\pgfqpoint{0.664684in}{4.364142in}}%
\pgfpathlineto{\pgfqpoint{0.668979in}{4.365232in}}%
\pgfpathlineto{\pgfqpoint{0.670053in}{4.364727in}}%
\pgfpathlineto{\pgfqpoint{0.671127in}{4.365564in}}%
\pgfpathlineto{\pgfqpoint{0.672200in}{4.365438in}}%
\pgfpathlineto{\pgfqpoint{0.678643in}{4.365880in}}%
\pgfpathlineto{\pgfqpoint{0.679717in}{4.366275in}}%
\pgfpathlineto{\pgfqpoint{0.685087in}{4.365501in}}%
\pgfpathlineto{\pgfqpoint{0.690456in}{4.364790in}}%
\pgfpathlineto{\pgfqpoint{0.691530in}{4.362040in}}%
\pgfpathlineto{\pgfqpoint{0.692603in}{4.362719in}}%
\pgfpathlineto{\pgfqpoint{0.693677in}{4.364331in}}%
\pgfpathlineto{\pgfqpoint{0.694751in}{4.364458in}}%
\pgfpathlineto{\pgfqpoint{0.697973in}{4.365422in}}%
\pgfpathlineto{\pgfqpoint{0.699046in}{4.367002in}}%
\pgfpathlineto{\pgfqpoint{0.700120in}{4.367145in}}%
\pgfpathlineto{\pgfqpoint{0.701194in}{4.368630in}}%
\pgfpathlineto{\pgfqpoint{0.702268in}{4.368046in}}%
\pgfpathlineto{\pgfqpoint{0.705489in}{4.368283in}}%
\pgfpathlineto{\pgfqpoint{0.709785in}{4.366623in}}%
\pgfpathlineto{\pgfqpoint{0.714080in}{4.367508in}}%
\pgfpathlineto{\pgfqpoint{0.715154in}{4.366607in}}%
\pgfpathlineto{\pgfqpoint{0.717302in}{4.367587in}}%
\pgfpathlineto{\pgfqpoint{0.720523in}{4.367619in}}%
\pgfpathlineto{\pgfqpoint{0.721597in}{4.367050in}}%
\pgfpathlineto{\pgfqpoint{0.723745in}{4.365058in}}%
\pgfpathlineto{\pgfqpoint{0.728040in}{4.363841in}}%
\pgfpathlineto{\pgfqpoint{0.729114in}{4.361566in}}%
\pgfpathlineto{\pgfqpoint{0.730188in}{4.362561in}}%
\pgfpathlineto{\pgfqpoint{0.731262in}{4.364537in}}%
\pgfpathlineto{\pgfqpoint{0.732335in}{4.363004in}}%
\pgfpathlineto{\pgfqpoint{0.735557in}{4.363936in}}%
\pgfpathlineto{\pgfqpoint{0.736631in}{4.365296in}}%
\pgfpathlineto{\pgfqpoint{0.738778in}{4.364458in}}%
\pgfpathlineto{\pgfqpoint{0.739852in}{4.365343in}}%
\pgfpathlineto{\pgfqpoint{0.743074in}{4.364885in}}%
\pgfpathlineto{\pgfqpoint{0.744148in}{4.366655in}}%
\pgfpathlineto{\pgfqpoint{0.747369in}{4.367793in}}%
\pgfpathlineto{\pgfqpoint{0.753812in}{4.367824in}}%
\pgfpathlineto{\pgfqpoint{0.754886in}{4.366892in}}%
\pgfpathlineto{\pgfqpoint{0.762403in}{4.364363in}}%
\pgfpathlineto{\pgfqpoint{0.766698in}{4.363130in}}%
\pgfpathlineto{\pgfqpoint{0.767772in}{4.363336in}}%
\pgfpathlineto{\pgfqpoint{0.769920in}{4.360918in}}%
\pgfpathlineto{\pgfqpoint{0.773141in}{4.362166in}}%
\pgfpathlineto{\pgfqpoint{0.774215in}{4.361724in}}%
\pgfpathlineto{\pgfqpoint{0.776363in}{4.362846in}}%
\pgfpathlineto{\pgfqpoint{0.777437in}{4.362577in}}%
\pgfpathlineto{\pgfqpoint{0.781732in}{4.363826in}}%
\pgfpathlineto{\pgfqpoint{0.782806in}{4.362150in}}%
\pgfpathlineto{\pgfqpoint{0.783880in}{4.362103in}}%
\pgfpathlineto{\pgfqpoint{0.784953in}{4.360048in}}%
\pgfpathlineto{\pgfqpoint{0.789249in}{4.359606in}}%
\pgfpathlineto{\pgfqpoint{0.790323in}{4.362387in}}%
\pgfpathlineto{\pgfqpoint{0.792470in}{4.364189in}}%
\pgfpathlineto{\pgfqpoint{0.795692in}{4.363114in}}%
\pgfpathlineto{\pgfqpoint{0.796766in}{4.365137in}}%
\pgfpathlineto{\pgfqpoint{0.797840in}{4.364363in}}%
\pgfpathlineto{\pgfqpoint{0.799987in}{4.366070in}}%
\pgfpathlineto{\pgfqpoint{0.803209in}{4.365912in}}%
\pgfpathlineto{\pgfqpoint{0.804283in}{4.366576in}}%
\pgfpathlineto{\pgfqpoint{0.805356in}{4.366212in}}%
\pgfpathlineto{\pgfqpoint{0.806430in}{4.365137in}}%
\pgfpathlineto{\pgfqpoint{0.807504in}{4.365280in}}%
\pgfpathlineto{\pgfqpoint{0.810726in}{4.363984in}}%
\pgfpathlineto{\pgfqpoint{0.811799in}{4.364426in}}%
\pgfpathlineto{\pgfqpoint{0.812873in}{4.365706in}}%
\pgfpathlineto{\pgfqpoint{0.813947in}{4.365706in}}%
\pgfpathlineto{\pgfqpoint{0.815021in}{4.368915in}}%
\pgfpathlineto{\pgfqpoint{0.818242in}{4.368488in}}%
\pgfpathlineto{\pgfqpoint{0.819316in}{4.369041in}}%
\pgfpathlineto{\pgfqpoint{0.821464in}{4.368867in}}%
\pgfpathlineto{\pgfqpoint{0.822538in}{4.368109in}}%
\pgfpathlineto{\pgfqpoint{0.825759in}{4.368077in}}%
\pgfpathlineto{\pgfqpoint{0.828981in}{4.364727in}}%
\pgfpathlineto{\pgfqpoint{0.830055in}{4.366275in}}%
\pgfpathlineto{\pgfqpoint{0.833276in}{4.366939in}}%
\pgfpathlineto{\pgfqpoint{0.834350in}{4.368093in}}%
\pgfpathlineto{\pgfqpoint{0.835424in}{4.370574in}}%
\pgfpathlineto{\pgfqpoint{0.836498in}{4.370527in}}%
\pgfpathlineto{\pgfqpoint{0.837572in}{4.369421in}}%
\pgfpathlineto{\pgfqpoint{0.840793in}{4.368599in}}%
\pgfpathlineto{\pgfqpoint{0.841867in}{4.367129in}}%
\pgfpathlineto{\pgfqpoint{0.842941in}{4.367793in}}%
\pgfpathlineto{\pgfqpoint{0.845088in}{4.371681in}}%
\pgfpathlineto{\pgfqpoint{0.850458in}{4.371017in}}%
\pgfpathlineto{\pgfqpoint{0.851531in}{4.369547in}}%
\pgfpathlineto{\pgfqpoint{0.852605in}{4.371665in}}%
\pgfpathlineto{\pgfqpoint{0.857975in}{4.371570in}}%
\pgfpathlineto{\pgfqpoint{0.859048in}{4.371523in}}%
\pgfpathlineto{\pgfqpoint{0.860122in}{4.372439in}}%
\pgfpathlineto{\pgfqpoint{0.865491in}{4.372771in}}%
\pgfpathlineto{\pgfqpoint{0.867639in}{4.375000in}}%
\pgfpathlineto{\pgfqpoint{0.870861in}{4.374541in}}%
\pgfpathlineto{\pgfqpoint{0.871934in}{4.373656in}}%
\pgfpathlineto{\pgfqpoint{0.873008in}{4.373735in}}%
\pgfpathlineto{\pgfqpoint{0.874082in}{4.372803in}}%
\pgfpathlineto{\pgfqpoint{0.875156in}{4.373925in}}%
\pgfpathlineto{\pgfqpoint{0.880525in}{4.373403in}}%
\pgfpathlineto{\pgfqpoint{0.881599in}{4.372518in}}%
\pgfpathlineto{\pgfqpoint{0.882673in}{4.373625in}}%
\pgfpathlineto{\pgfqpoint{0.888042in}{4.372502in}}%
\pgfpathlineto{\pgfqpoint{0.889116in}{4.374525in}}%
\pgfpathlineto{\pgfqpoint{0.890190in}{4.373909in}}%
\pgfpathlineto{\pgfqpoint{0.893411in}{4.371080in}}%
\pgfpathlineto{\pgfqpoint{0.894485in}{4.371744in}}%
\pgfpathlineto{\pgfqpoint{0.895559in}{4.371254in}}%
\pgfpathlineto{\pgfqpoint{0.897707in}{4.375442in}}%
\pgfpathlineto{\pgfqpoint{0.908445in}{4.375126in}}%
\pgfpathlineto{\pgfqpoint{0.909519in}{4.373925in}}%
\pgfpathlineto{\pgfqpoint{0.910593in}{4.373609in}}%
\pgfpathlineto{\pgfqpoint{0.911666in}{4.373909in}}%
\pgfpathlineto{\pgfqpoint{0.912740in}{4.373388in}}%
\pgfpathlineto{\pgfqpoint{0.919183in}{4.376043in}}%
\pgfpathlineto{\pgfqpoint{0.920257in}{4.376738in}}%
\pgfpathlineto{\pgfqpoint{0.923479in}{4.377291in}}%
\pgfpathlineto{\pgfqpoint{0.925626in}{4.374525in}}%
\pgfpathlineto{\pgfqpoint{0.927774in}{4.373830in}}%
\pgfpathlineto{\pgfqpoint{0.930996in}{4.373877in}}%
\pgfpathlineto{\pgfqpoint{0.933143in}{4.376548in}}%
\pgfpathlineto{\pgfqpoint{0.934217in}{4.376454in}}%
\pgfpathlineto{\pgfqpoint{0.935291in}{4.374067in}}%
\pgfpathlineto{\pgfqpoint{0.938512in}{4.373530in}}%
\pgfpathlineto{\pgfqpoint{0.939586in}{4.368520in}}%
\pgfpathlineto{\pgfqpoint{0.942808in}{4.367587in}}%
\pgfpathlineto{\pgfqpoint{0.948177in}{4.367018in}}%
\pgfpathlineto{\pgfqpoint{0.949251in}{4.369199in}}%
\pgfpathlineto{\pgfqpoint{0.950325in}{4.368836in}}%
\pgfpathlineto{\pgfqpoint{0.953546in}{4.369658in}}%
\pgfpathlineto{\pgfqpoint{0.954620in}{4.371333in}}%
\pgfpathlineto{\pgfqpoint{0.956768in}{4.368267in}}%
\pgfpathlineto{\pgfqpoint{0.961063in}{4.369104in}}%
\pgfpathlineto{\pgfqpoint{0.962137in}{4.368978in}}%
\pgfpathlineto{\pgfqpoint{0.963211in}{4.366639in}}%
\pgfpathlineto{\pgfqpoint{0.965358in}{4.368204in}}%
\pgfpathlineto{\pgfqpoint{0.969654in}{4.369673in}}%
\pgfpathlineto{\pgfqpoint{0.970728in}{4.369563in}}%
\pgfpathlineto{\pgfqpoint{0.972875in}{4.371349in}}%
\pgfpathlineto{\pgfqpoint{0.977171in}{4.371396in}}%
\pgfpathlineto{\pgfqpoint{0.978244in}{4.372218in}}%
\pgfpathlineto{\pgfqpoint{0.979318in}{4.371839in}}%
\pgfpathlineto{\pgfqpoint{0.980392in}{4.372250in}}%
\pgfpathlineto{\pgfqpoint{0.984687in}{4.371143in}}%
\pgfpathlineto{\pgfqpoint{0.986835in}{4.372250in}}%
\pgfpathlineto{\pgfqpoint{0.987909in}{4.372992in}}%
\pgfpathlineto{\pgfqpoint{0.991130in}{4.373467in}}%
\pgfpathlineto{\pgfqpoint{0.992204in}{4.375869in}}%
\pgfpathlineto{\pgfqpoint{0.995426in}{4.374004in}}%
\pgfpathlineto{\pgfqpoint{0.998647in}{4.375015in}}%
\pgfpathlineto{\pgfqpoint{0.999721in}{4.376106in}}%
\pgfpathlineto{\pgfqpoint{1.000795in}{4.375000in}}%
\pgfpathlineto{\pgfqpoint{1.001869in}{4.376469in}}%
\pgfpathlineto{\pgfqpoint{1.002943in}{4.375094in}}%
\pgfpathlineto{\pgfqpoint{1.008312in}{4.375063in}}%
\pgfpathlineto{\pgfqpoint{1.010460in}{4.373340in}}%
\pgfpathlineto{\pgfqpoint{1.013681in}{4.374763in}}%
\pgfpathlineto{\pgfqpoint{1.015829in}{4.377339in}}%
\pgfpathlineto{\pgfqpoint{1.016903in}{4.377196in}}%
\pgfpathlineto{\pgfqpoint{1.017976in}{4.378113in}}%
\pgfpathlineto{\pgfqpoint{1.022272in}{4.378287in}}%
\pgfpathlineto{\pgfqpoint{1.024419in}{4.380136in}}%
\pgfpathlineto{\pgfqpoint{1.025493in}{4.379330in}}%
\pgfpathlineto{\pgfqpoint{1.031936in}{4.381732in}}%
\pgfpathlineto{\pgfqpoint{1.033010in}{4.382602in}}%
\pgfpathlineto{\pgfqpoint{1.039453in}{4.383834in}}%
\pgfpathlineto{\pgfqpoint{1.040527in}{4.385067in}}%
\pgfpathlineto{\pgfqpoint{1.043749in}{4.385146in}}%
\pgfpathlineto{\pgfqpoint{1.044822in}{4.386679in}}%
\pgfpathlineto{\pgfqpoint{1.045896in}{4.385336in}}%
\pgfpathlineto{\pgfqpoint{1.046970in}{4.385004in}}%
\pgfpathlineto{\pgfqpoint{1.048044in}{4.386347in}}%
\pgfpathlineto{\pgfqpoint{1.051265in}{4.385304in}}%
\pgfpathlineto{\pgfqpoint{1.053413in}{4.387849in}}%
\pgfpathlineto{\pgfqpoint{1.054487in}{4.387232in}}%
\pgfpathlineto{\pgfqpoint{1.055561in}{4.387817in}}%
\pgfpathlineto{\pgfqpoint{1.058782in}{4.387770in}}%
\pgfpathlineto{\pgfqpoint{1.059856in}{4.388876in}}%
\pgfpathlineto{\pgfqpoint{1.063078in}{4.389429in}}%
\pgfpathlineto{\pgfqpoint{1.067373in}{4.390694in}}%
\pgfpathlineto{\pgfqpoint{1.069521in}{4.388734in}}%
\pgfpathlineto{\pgfqpoint{1.070595in}{4.389840in}}%
\pgfpathlineto{\pgfqpoint{1.073816in}{4.387438in}}%
\pgfpathlineto{\pgfqpoint{1.074890in}{4.388197in}}%
\pgfpathlineto{\pgfqpoint{1.075964in}{4.389872in}}%
\pgfpathlineto{\pgfqpoint{1.077038in}{4.390457in}}%
\pgfpathlineto{\pgfqpoint{1.081333in}{4.389493in}}%
\pgfpathlineto{\pgfqpoint{1.082407in}{4.391057in}}%
\pgfpathlineto{\pgfqpoint{1.084554in}{4.391168in}}%
\pgfpathlineto{\pgfqpoint{1.085628in}{4.392748in}}%
\pgfpathlineto{\pgfqpoint{1.088850in}{4.392875in}}%
\pgfpathlineto{\pgfqpoint{1.089924in}{4.391958in}}%
\pgfpathlineto{\pgfqpoint{1.090997in}{4.391911in}}%
\pgfpathlineto{\pgfqpoint{1.093145in}{4.393665in}}%
\pgfpathlineto{\pgfqpoint{1.097441in}{4.392037in}}%
\pgfpathlineto{\pgfqpoint{1.098514in}{4.392669in}}%
\pgfpathlineto{\pgfqpoint{1.099588in}{4.391705in}}%
\pgfpathlineto{\pgfqpoint{1.100662in}{4.393697in}}%
\pgfpathlineto{\pgfqpoint{1.103884in}{4.392021in}}%
\pgfpathlineto{\pgfqpoint{1.104957in}{4.393222in}}%
\pgfpathlineto{\pgfqpoint{1.106031in}{4.392179in}}%
\pgfpathlineto{\pgfqpoint{1.107105in}{4.393539in}}%
\pgfpathlineto{\pgfqpoint{1.111400in}{4.392653in}}%
\pgfpathlineto{\pgfqpoint{1.112474in}{4.393823in}}%
\pgfpathlineto{\pgfqpoint{1.113548in}{4.392701in}}%
\pgfpathlineto{\pgfqpoint{1.115696in}{4.392827in}}%
\pgfpathlineto{\pgfqpoint{1.119991in}{4.393064in}}%
\pgfpathlineto{\pgfqpoint{1.121065in}{4.395388in}}%
\pgfpathlineto{\pgfqpoint{1.122139in}{4.396146in}}%
\pgfpathlineto{\pgfqpoint{1.126434in}{4.392812in}}%
\pgfpathlineto{\pgfqpoint{1.127508in}{4.393333in}}%
\pgfpathlineto{\pgfqpoint{1.129656in}{4.391768in}}%
\pgfpathlineto{\pgfqpoint{1.130729in}{4.392748in}}%
\pgfpathlineto{\pgfqpoint{1.133951in}{4.392859in}}%
\pgfpathlineto{\pgfqpoint{1.135025in}{4.394993in}}%
\pgfpathlineto{\pgfqpoint{1.136099in}{4.395625in}}%
\pgfpathlineto{\pgfqpoint{1.137173in}{4.391626in}}%
\pgfpathlineto{\pgfqpoint{1.138246in}{4.390172in}}%
\pgfpathlineto{\pgfqpoint{1.141468in}{4.390220in}}%
\pgfpathlineto{\pgfqpoint{1.142542in}{4.391405in}}%
\pgfpathlineto{\pgfqpoint{1.143616in}{4.391184in}}%
\pgfpathlineto{\pgfqpoint{1.145763in}{4.395577in}}%
\pgfpathlineto{\pgfqpoint{1.151132in}{4.396067in}}%
\pgfpathlineto{\pgfqpoint{1.152206in}{4.398375in}}%
\pgfpathlineto{\pgfqpoint{1.153280in}{4.399118in}}%
\pgfpathlineto{\pgfqpoint{1.157575in}{4.399276in}}%
\pgfpathlineto{\pgfqpoint{1.158649in}{4.400493in}}%
\pgfpathlineto{\pgfqpoint{1.159723in}{4.399908in}}%
\pgfpathlineto{\pgfqpoint{1.160797in}{4.400335in}}%
\pgfpathlineto{\pgfqpoint{1.165092in}{4.401172in}}%
\pgfpathlineto{\pgfqpoint{1.167240in}{4.399876in}}%
\pgfpathlineto{\pgfqpoint{1.168314in}{4.399687in}}%
\pgfpathlineto{\pgfqpoint{1.172609in}{4.401457in}}%
\pgfpathlineto{\pgfqpoint{1.173683in}{4.400840in}}%
\pgfpathlineto{\pgfqpoint{1.174757in}{4.401235in}}%
\pgfpathlineto{\pgfqpoint{1.175831in}{4.399687in}}%
\pgfpathlineto{\pgfqpoint{1.179052in}{4.400145in}}%
\pgfpathlineto{\pgfqpoint{1.180126in}{4.399355in}}%
\pgfpathlineto{\pgfqpoint{1.181200in}{4.397426in}}%
\pgfpathlineto{\pgfqpoint{1.182274in}{4.397537in}}%
\pgfpathlineto{\pgfqpoint{1.183348in}{4.400809in}}%
\pgfpathlineto{\pgfqpoint{1.186569in}{4.400414in}}%
\pgfpathlineto{\pgfqpoint{1.187643in}{4.399623in}}%
\pgfpathlineto{\pgfqpoint{1.188717in}{4.398011in}}%
\pgfpathlineto{\pgfqpoint{1.189791in}{4.400935in}}%
\pgfpathlineto{\pgfqpoint{1.190864in}{4.400714in}}%
\pgfpathlineto{\pgfqpoint{1.194086in}{4.401899in}}%
\pgfpathlineto{\pgfqpoint{1.195160in}{4.403322in}}%
\pgfpathlineto{\pgfqpoint{1.196234in}{4.401441in}}%
\pgfpathlineto{\pgfqpoint{1.197307in}{4.397679in}}%
\pgfpathlineto{\pgfqpoint{1.198381in}{4.398770in}}%
\pgfpathlineto{\pgfqpoint{1.201603in}{4.395957in}}%
\pgfpathlineto{\pgfqpoint{1.202677in}{4.396952in}}%
\pgfpathlineto{\pgfqpoint{1.203751in}{4.398896in}}%
\pgfpathlineto{\pgfqpoint{1.204824in}{4.399639in}}%
\pgfpathlineto{\pgfqpoint{1.205898in}{4.398454in}}%
\pgfpathlineto{\pgfqpoint{1.209120in}{4.398391in}}%
\pgfpathlineto{\pgfqpoint{1.210194in}{4.397616in}}%
\pgfpathlineto{\pgfqpoint{1.211267in}{4.398580in}}%
\pgfpathlineto{\pgfqpoint{1.213415in}{4.401393in}}%
\pgfpathlineto{\pgfqpoint{1.216637in}{4.402184in}}%
\pgfpathlineto{\pgfqpoint{1.217710in}{4.403827in}}%
\pgfpathlineto{\pgfqpoint{1.218784in}{4.403938in}}%
\pgfpathlineto{\pgfqpoint{1.220932in}{4.406246in}}%
\pgfpathlineto{\pgfqpoint{1.224153in}{4.405850in}}%
\pgfpathlineto{\pgfqpoint{1.225227in}{4.405155in}}%
\pgfpathlineto{\pgfqpoint{1.226301in}{4.405566in}}%
\pgfpathlineto{\pgfqpoint{1.228449in}{4.407652in}}%
\pgfpathlineto{\pgfqpoint{1.231670in}{4.407779in}}%
\pgfpathlineto{\pgfqpoint{1.232744in}{4.408411in}}%
\pgfpathlineto{\pgfqpoint{1.233818in}{4.407842in}}%
\pgfpathlineto{\pgfqpoint{1.235966in}{4.408616in}}%
\pgfpathlineto{\pgfqpoint{1.240261in}{4.408521in}}%
\pgfpathlineto{\pgfqpoint{1.243483in}{4.410434in}}%
\pgfpathlineto{\pgfqpoint{1.248852in}{4.409865in}}%
\pgfpathlineto{\pgfqpoint{1.249926in}{4.411003in}}%
\pgfpathlineto{\pgfqpoint{1.250999in}{4.410513in}}%
\pgfpathlineto{\pgfqpoint{1.255295in}{4.410987in}}%
\pgfpathlineto{\pgfqpoint{1.256369in}{4.409580in}}%
\pgfpathlineto{\pgfqpoint{1.257442in}{4.407194in}}%
\pgfpathlineto{\pgfqpoint{1.258516in}{4.407257in}}%
\pgfpathlineto{\pgfqpoint{1.262812in}{4.406609in}}%
\pgfpathlineto{\pgfqpoint{1.263885in}{4.404776in}}%
\pgfpathlineto{\pgfqpoint{1.264959in}{4.406467in}}%
\pgfpathlineto{\pgfqpoint{1.266033in}{4.406087in}}%
\pgfpathlineto{\pgfqpoint{1.269255in}{4.405993in}}%
\pgfpathlineto{\pgfqpoint{1.270329in}{4.403827in}}%
\pgfpathlineto{\pgfqpoint{1.273550in}{4.404981in}}%
\pgfpathlineto{\pgfqpoint{1.277845in}{4.404491in}}%
\pgfpathlineto{\pgfqpoint{1.278919in}{4.406293in}}%
\pgfpathlineto{\pgfqpoint{1.281067in}{4.406957in}}%
\pgfpathlineto{\pgfqpoint{1.285362in}{4.410418in}}%
\pgfpathlineto{\pgfqpoint{1.286436in}{4.411919in}}%
\pgfpathlineto{\pgfqpoint{1.287510in}{4.411240in}}%
\pgfpathlineto{\pgfqpoint{1.288584in}{4.411777in}}%
\pgfpathlineto{\pgfqpoint{1.291805in}{4.412631in}}%
\pgfpathlineto{\pgfqpoint{1.292879in}{4.413579in}}%
\pgfpathlineto{\pgfqpoint{1.293953in}{4.415412in}}%
\pgfpathlineto{\pgfqpoint{1.295027in}{4.415792in}}%
\pgfpathlineto{\pgfqpoint{1.296101in}{4.413690in}}%
\pgfpathlineto{\pgfqpoint{1.299322in}{4.415175in}}%
\pgfpathlineto{\pgfqpoint{1.301470in}{4.413942in}}%
\pgfpathlineto{\pgfqpoint{1.302544in}{4.414559in}}%
\pgfpathlineto{\pgfqpoint{1.303617in}{4.413974in}}%
\pgfpathlineto{\pgfqpoint{1.306839in}{4.412868in}}%
\pgfpathlineto{\pgfqpoint{1.307913in}{4.413152in}}%
\pgfpathlineto{\pgfqpoint{1.310061in}{4.411714in}}%
\pgfpathlineto{\pgfqpoint{1.311134in}{4.412868in}}%
\pgfpathlineto{\pgfqpoint{1.314356in}{4.411904in}}%
\pgfpathlineto{\pgfqpoint{1.315430in}{4.409833in}}%
\pgfpathlineto{\pgfqpoint{1.316504in}{4.410386in}}%
\pgfpathlineto{\pgfqpoint{1.318651in}{4.414654in}}%
\pgfpathlineto{\pgfqpoint{1.321873in}{4.415570in}}%
\pgfpathlineto{\pgfqpoint{1.322947in}{4.413437in}}%
\pgfpathlineto{\pgfqpoint{1.326168in}{4.417514in}}%
\pgfpathlineto{\pgfqpoint{1.332611in}{4.418399in}}%
\pgfpathlineto{\pgfqpoint{1.333685in}{4.419648in}}%
\pgfpathlineto{\pgfqpoint{1.337980in}{4.420817in}}%
\pgfpathlineto{\pgfqpoint{1.339054in}{4.420154in}}%
\pgfpathlineto{\pgfqpoint{1.340128in}{4.421592in}}%
\pgfpathlineto{\pgfqpoint{1.345497in}{4.421955in}}%
\pgfpathlineto{\pgfqpoint{1.346571in}{4.423315in}}%
\pgfpathlineto{\pgfqpoint{1.347645in}{4.422351in}}%
\pgfpathlineto{\pgfqpoint{1.348719in}{4.424484in}}%
\pgfpathlineto{\pgfqpoint{1.351940in}{4.424437in}}%
\pgfpathlineto{\pgfqpoint{1.354088in}{4.425290in}}%
\pgfpathlineto{\pgfqpoint{1.355162in}{4.426918in}}%
\pgfpathlineto{\pgfqpoint{1.361605in}{4.426950in}}%
\pgfpathlineto{\pgfqpoint{1.363752in}{4.429463in}}%
\pgfpathlineto{\pgfqpoint{1.366974in}{4.429747in}}%
\pgfpathlineto{\pgfqpoint{1.369122in}{4.432829in}}%
\pgfpathlineto{\pgfqpoint{1.371269in}{4.432845in}}%
\pgfpathlineto{\pgfqpoint{1.375565in}{4.423441in}}%
\pgfpathlineto{\pgfqpoint{1.376639in}{4.423251in}}%
\pgfpathlineto{\pgfqpoint{1.377712in}{4.423757in}}%
\pgfpathlineto{\pgfqpoint{1.378786in}{4.426175in}}%
\pgfpathlineto{\pgfqpoint{1.382008in}{4.426128in}}%
\pgfpathlineto{\pgfqpoint{1.384155in}{4.423678in}}%
\pgfpathlineto{\pgfqpoint{1.386303in}{4.423220in}}%
\pgfpathlineto{\pgfqpoint{1.389525in}{4.424879in}}%
\pgfpathlineto{\pgfqpoint{1.391672in}{4.435958in}}%
\pgfpathlineto{\pgfqpoint{1.393820in}{4.437207in}}%
\pgfpathlineto{\pgfqpoint{1.398115in}{4.437570in}}%
\pgfpathlineto{\pgfqpoint{1.401337in}{4.440779in}}%
\pgfpathlineto{\pgfqpoint{1.404558in}{4.440874in}}%
\pgfpathlineto{\pgfqpoint{1.405632in}{4.442012in}}%
\pgfpathlineto{\pgfqpoint{1.407780in}{4.439119in}}%
\pgfpathlineto{\pgfqpoint{1.408854in}{4.439562in}}%
\pgfpathlineto{\pgfqpoint{1.413149in}{4.438471in}}%
\pgfpathlineto{\pgfqpoint{1.414223in}{4.437080in}}%
\pgfpathlineto{\pgfqpoint{1.416371in}{4.436480in}}%
\pgfpathlineto{\pgfqpoint{1.419592in}{4.434441in}}%
\pgfpathlineto{\pgfqpoint{1.420666in}{4.438139in}}%
\pgfpathlineto{\pgfqpoint{1.421740in}{4.439546in}}%
\pgfpathlineto{\pgfqpoint{1.422814in}{4.439167in}}%
\pgfpathlineto{\pgfqpoint{1.423887in}{4.438013in}}%
\pgfpathlineto{\pgfqpoint{1.428183in}{4.437570in}}%
\pgfpathlineto{\pgfqpoint{1.429257in}{4.436875in}}%
\pgfpathlineto{\pgfqpoint{1.430330in}{4.434473in}}%
\pgfpathlineto{\pgfqpoint{1.431404in}{4.428372in}}%
\pgfpathlineto{\pgfqpoint{1.434626in}{4.426649in}}%
\pgfpathlineto{\pgfqpoint{1.436773in}{4.428404in}}%
\pgfpathlineto{\pgfqpoint{1.437847in}{4.425417in}}%
\pgfpathlineto{\pgfqpoint{1.438921in}{4.425606in}}%
\pgfpathlineto{\pgfqpoint{1.442143in}{4.419774in}}%
\pgfpathlineto{\pgfqpoint{1.443217in}{4.423615in}}%
\pgfpathlineto{\pgfqpoint{1.444290in}{4.424484in}}%
\pgfpathlineto{\pgfqpoint{1.446438in}{4.428514in}}%
\pgfpathlineto{\pgfqpoint{1.449660in}{4.427661in}}%
\pgfpathlineto{\pgfqpoint{1.450733in}{4.429399in}}%
\pgfpathlineto{\pgfqpoint{1.451807in}{4.429826in}}%
\pgfpathlineto{\pgfqpoint{1.452881in}{4.429415in}}%
\pgfpathlineto{\pgfqpoint{1.453955in}{4.432134in}}%
\pgfpathlineto{\pgfqpoint{1.458250in}{4.431691in}}%
\pgfpathlineto{\pgfqpoint{1.459324in}{4.430000in}}%
\pgfpathlineto{\pgfqpoint{1.460398in}{4.431359in}}%
\pgfpathlineto{\pgfqpoint{1.461472in}{4.431375in}}%
\pgfpathlineto{\pgfqpoint{1.464693in}{4.432244in}}%
\pgfpathlineto{\pgfqpoint{1.465767in}{4.433240in}}%
\pgfpathlineto{\pgfqpoint{1.466841in}{4.433145in}}%
\pgfpathlineto{\pgfqpoint{1.467915in}{4.435168in}}%
\pgfpathlineto{\pgfqpoint{1.468989in}{4.435705in}}%
\pgfpathlineto{\pgfqpoint{1.472210in}{4.432260in}}%
\pgfpathlineto{\pgfqpoint{1.473284in}{4.432892in}}%
\pgfpathlineto{\pgfqpoint{1.474358in}{4.434504in}}%
\pgfpathlineto{\pgfqpoint{1.476506in}{4.434852in}}%
\pgfpathlineto{\pgfqpoint{1.479727in}{4.434109in}}%
\pgfpathlineto{\pgfqpoint{1.480801in}{4.432671in}}%
\pgfpathlineto{\pgfqpoint{1.481875in}{4.432750in}}%
\pgfpathlineto{\pgfqpoint{1.484022in}{4.429004in}}%
\pgfpathlineto{\pgfqpoint{1.488318in}{4.432987in}}%
\pgfpathlineto{\pgfqpoint{1.489392in}{4.430917in}}%
\pgfpathlineto{\pgfqpoint{1.491539in}{4.433509in}}%
\pgfpathlineto{\pgfqpoint{1.494761in}{4.432545in}}%
\pgfpathlineto{\pgfqpoint{1.495835in}{4.434789in}}%
\pgfpathlineto{\pgfqpoint{1.496908in}{4.433461in}}%
\pgfpathlineto{\pgfqpoint{1.497982in}{4.433114in}}%
\pgfpathlineto{\pgfqpoint{1.499056in}{4.434978in}}%
\pgfpathlineto{\pgfqpoint{1.503351in}{4.438171in}}%
\pgfpathlineto{\pgfqpoint{1.504425in}{4.437444in}}%
\pgfpathlineto{\pgfqpoint{1.505499in}{4.437634in}}%
\pgfpathlineto{\pgfqpoint{1.506573in}{4.437254in}}%
\pgfpathlineto{\pgfqpoint{1.509794in}{4.435247in}}%
\pgfpathlineto{\pgfqpoint{1.510868in}{4.435864in}}%
\pgfpathlineto{\pgfqpoint{1.511942in}{4.437223in}}%
\pgfpathlineto{\pgfqpoint{1.514090in}{4.432497in}}%
\pgfpathlineto{\pgfqpoint{1.517311in}{4.433540in}}%
\pgfpathlineto{\pgfqpoint{1.518385in}{4.434836in}}%
\pgfpathlineto{\pgfqpoint{1.519459in}{4.438503in}}%
\pgfpathlineto{\pgfqpoint{1.520533in}{4.439815in}}%
\pgfpathlineto{\pgfqpoint{1.525902in}{4.441395in}}%
\pgfpathlineto{\pgfqpoint{1.529124in}{4.438203in}}%
\pgfpathlineto{\pgfqpoint{1.533419in}{4.439562in}}%
\pgfpathlineto{\pgfqpoint{1.535567in}{4.444035in}}%
\pgfpathlineto{\pgfqpoint{1.536640in}{4.443086in}}%
\pgfpathlineto{\pgfqpoint{1.539862in}{4.443750in}}%
\pgfpathlineto{\pgfqpoint{1.540936in}{4.441948in}}%
\pgfpathlineto{\pgfqpoint{1.542010in}{4.444477in}}%
\pgfpathlineto{\pgfqpoint{1.543083in}{4.444050in}}%
\pgfpathlineto{\pgfqpoint{1.547379in}{4.446785in}}%
\pgfpathlineto{\pgfqpoint{1.549527in}{4.445109in}}%
\pgfpathlineto{\pgfqpoint{1.550600in}{4.444256in}}%
\pgfpathlineto{\pgfqpoint{1.554896in}{4.445204in}}%
\pgfpathlineto{\pgfqpoint{1.555970in}{4.443229in}}%
\pgfpathlineto{\pgfqpoint{1.557043in}{4.444999in}}%
\pgfpathlineto{\pgfqpoint{1.558117in}{4.444525in}}%
\pgfpathlineto{\pgfqpoint{1.559191in}{4.445662in}}%
\pgfpathlineto{\pgfqpoint{1.564560in}{4.446058in}}%
\pgfpathlineto{\pgfqpoint{1.565634in}{4.447385in}}%
\pgfpathlineto{\pgfqpoint{1.566708in}{4.447606in}}%
\pgfpathlineto{\pgfqpoint{1.569929in}{4.447290in}}%
\pgfpathlineto{\pgfqpoint{1.571003in}{4.448081in}}%
\pgfpathlineto{\pgfqpoint{1.572077in}{4.447211in}}%
\pgfpathlineto{\pgfqpoint{1.574225in}{4.450483in}}%
\pgfpathlineto{\pgfqpoint{1.577446in}{4.451415in}}%
\pgfpathlineto{\pgfqpoint{1.579594in}{4.450167in}}%
\pgfpathlineto{\pgfqpoint{1.580668in}{4.448318in}}%
\pgfpathlineto{\pgfqpoint{1.581742in}{4.448729in}}%
\pgfpathlineto{\pgfqpoint{1.586037in}{4.449171in}}%
\pgfpathlineto{\pgfqpoint{1.589259in}{4.451194in}}%
\pgfpathlineto{\pgfqpoint{1.592480in}{4.449724in}}%
\pgfpathlineto{\pgfqpoint{1.593554in}{4.448349in}}%
\pgfpathlineto{\pgfqpoint{1.595702in}{4.449234in}}%
\pgfpathlineto{\pgfqpoint{1.599997in}{4.448555in}}%
\pgfpathlineto{\pgfqpoint{1.602145in}{4.451305in}}%
\pgfpathlineto{\pgfqpoint{1.603218in}{4.451542in}}%
\pgfpathlineto{\pgfqpoint{1.609661in}{4.450546in}}%
\pgfpathlineto{\pgfqpoint{1.610735in}{4.449456in}}%
\pgfpathlineto{\pgfqpoint{1.611809in}{4.450025in}}%
\pgfpathlineto{\pgfqpoint{1.616105in}{4.451068in}}%
\pgfpathlineto{\pgfqpoint{1.617178in}{4.452585in}}%
\pgfpathlineto{\pgfqpoint{1.618252in}{4.448966in}}%
\pgfpathlineto{\pgfqpoint{1.619326in}{4.450752in}}%
\pgfpathlineto{\pgfqpoint{1.622548in}{4.450025in}}%
\pgfpathlineto{\pgfqpoint{1.623621in}{4.451147in}}%
\pgfpathlineto{\pgfqpoint{1.624695in}{4.450546in}}%
\pgfpathlineto{\pgfqpoint{1.626843in}{4.451147in}}%
\pgfpathlineto{\pgfqpoint{1.630064in}{4.451716in}}%
\pgfpathlineto{\pgfqpoint{1.631138in}{4.449629in}}%
\pgfpathlineto{\pgfqpoint{1.632212in}{4.449250in}}%
\pgfpathlineto{\pgfqpoint{1.633286in}{4.445315in}}%
\pgfpathlineto{\pgfqpoint{1.634360in}{4.444240in}}%
\pgfpathlineto{\pgfqpoint{1.637581in}{4.445125in}}%
\pgfpathlineto{\pgfqpoint{1.638655in}{4.443797in}}%
\pgfpathlineto{\pgfqpoint{1.640803in}{4.442897in}}%
\pgfpathlineto{\pgfqpoint{1.641877in}{4.445267in}}%
\pgfpathlineto{\pgfqpoint{1.646172in}{4.445299in}}%
\pgfpathlineto{\pgfqpoint{1.648320in}{4.447322in}}%
\pgfpathlineto{\pgfqpoint{1.649394in}{4.446737in}}%
\pgfpathlineto{\pgfqpoint{1.652615in}{4.449772in}}%
\pgfpathlineto{\pgfqpoint{1.653689in}{4.449946in}}%
\pgfpathlineto{\pgfqpoint{1.654763in}{4.451621in}}%
\pgfpathlineto{\pgfqpoint{1.656910in}{4.450973in}}%
\pgfpathlineto{\pgfqpoint{1.661206in}{4.451621in}}%
\pgfpathlineto{\pgfqpoint{1.662280in}{4.450704in}}%
\pgfpathlineto{\pgfqpoint{1.669796in}{4.450530in}}%
\pgfpathlineto{\pgfqpoint{1.670870in}{4.450246in}}%
\pgfpathlineto{\pgfqpoint{1.671944in}{4.451068in}}%
\pgfpathlineto{\pgfqpoint{1.675166in}{4.452016in}}%
\pgfpathlineto{\pgfqpoint{1.676239in}{4.451431in}}%
\pgfpathlineto{\pgfqpoint{1.677313in}{4.451668in}}%
\pgfpathlineto{\pgfqpoint{1.679461in}{4.450704in}}%
\pgfpathlineto{\pgfqpoint{1.683756in}{4.451969in}}%
\pgfpathlineto{\pgfqpoint{1.684830in}{4.452680in}}%
\pgfpathlineto{\pgfqpoint{1.685904in}{4.454719in}}%
\pgfpathlineto{\pgfqpoint{1.686978in}{4.454513in}}%
\pgfpathlineto{\pgfqpoint{1.690199in}{4.453154in}}%
\pgfpathlineto{\pgfqpoint{1.691273in}{4.451305in}}%
\pgfpathlineto{\pgfqpoint{1.692347in}{4.452000in}}%
\pgfpathlineto{\pgfqpoint{1.693421in}{4.448681in}}%
\pgfpathlineto{\pgfqpoint{1.697716in}{4.448286in}}%
\pgfpathlineto{\pgfqpoint{1.698790in}{4.447575in}}%
\pgfpathlineto{\pgfqpoint{1.699864in}{4.444114in}}%
\pgfpathlineto{\pgfqpoint{1.700938in}{4.443402in}}%
\pgfpathlineto{\pgfqpoint{1.702012in}{4.445473in}}%
\pgfpathlineto{\pgfqpoint{1.705233in}{4.445710in}}%
\pgfpathlineto{\pgfqpoint{1.706307in}{4.442027in}}%
\pgfpathlineto{\pgfqpoint{1.707381in}{4.447196in}}%
\pgfpathlineto{\pgfqpoint{1.708455in}{4.443355in}}%
\pgfpathlineto{\pgfqpoint{1.709528in}{4.436701in}}%
\pgfpathlineto{\pgfqpoint{1.712750in}{4.435405in}}%
\pgfpathlineto{\pgfqpoint{1.713824in}{4.437175in}}%
\pgfpathlineto{\pgfqpoint{1.714898in}{4.437239in}}%
\pgfpathlineto{\pgfqpoint{1.715971in}{4.438392in}}%
\pgfpathlineto{\pgfqpoint{1.717045in}{4.441648in}}%
\pgfpathlineto{\pgfqpoint{1.720267in}{4.441917in}}%
\pgfpathlineto{\pgfqpoint{1.721341in}{4.446532in}}%
\pgfpathlineto{\pgfqpoint{1.722415in}{4.443782in}}%
\pgfpathlineto{\pgfqpoint{1.724562in}{4.457152in}}%
\pgfpathlineto{\pgfqpoint{1.727784in}{4.458496in}}%
\pgfpathlineto{\pgfqpoint{1.728858in}{4.460566in}}%
\pgfpathlineto{\pgfqpoint{1.729931in}{4.460503in}}%
\pgfpathlineto{\pgfqpoint{1.732079in}{4.464328in}}%
\pgfpathlineto{\pgfqpoint{1.735301in}{4.463664in}}%
\pgfpathlineto{\pgfqpoint{1.736374in}{4.466098in}}%
\pgfpathlineto{\pgfqpoint{1.739596in}{4.468168in}}%
\pgfpathlineto{\pgfqpoint{1.742817in}{4.469717in}}%
\pgfpathlineto{\pgfqpoint{1.743891in}{4.468943in}}%
\pgfpathlineto{\pgfqpoint{1.747113in}{4.471361in}}%
\pgfpathlineto{\pgfqpoint{1.750334in}{4.471140in}}%
\pgfpathlineto{\pgfqpoint{1.751408in}{4.473163in}}%
\pgfpathlineto{\pgfqpoint{1.752482in}{4.472515in}}%
\pgfpathlineto{\pgfqpoint{1.754630in}{4.474364in}}%
\pgfpathlineto{\pgfqpoint{1.757851in}{4.473969in}}%
\pgfpathlineto{\pgfqpoint{1.758925in}{4.471440in}}%
\pgfpathlineto{\pgfqpoint{1.759999in}{4.471788in}}%
\pgfpathlineto{\pgfqpoint{1.762147in}{4.474269in}}%
\pgfpathlineto{\pgfqpoint{1.765368in}{4.471582in}}%
\pgfpathlineto{\pgfqpoint{1.767516in}{4.477288in}}%
\pgfpathlineto{\pgfqpoint{1.769663in}{4.477303in}}%
\pgfpathlineto{\pgfqpoint{1.773959in}{4.475296in}}%
\pgfpathlineto{\pgfqpoint{1.775033in}{4.471693in}}%
\pgfpathlineto{\pgfqpoint{1.776106in}{4.472957in}}%
\pgfpathlineto{\pgfqpoint{1.777180in}{4.470128in}}%
\pgfpathlineto{\pgfqpoint{1.780402in}{4.469749in}}%
\pgfpathlineto{\pgfqpoint{1.782549in}{4.474980in}}%
\pgfpathlineto{\pgfqpoint{1.783623in}{4.481523in}}%
\pgfpathlineto{\pgfqpoint{1.784697in}{4.481776in}}%
\pgfpathlineto{\pgfqpoint{1.787919in}{4.484273in}}%
\pgfpathlineto{\pgfqpoint{1.788993in}{4.483720in}}%
\pgfpathlineto{\pgfqpoint{1.790066in}{4.483831in}}%
\pgfpathlineto{\pgfqpoint{1.792214in}{4.482867in}}%
\pgfpathlineto{\pgfqpoint{1.795436in}{4.483483in}}%
\pgfpathlineto{\pgfqpoint{1.796509in}{4.482282in}}%
\pgfpathlineto{\pgfqpoint{1.797583in}{4.480164in}}%
\pgfpathlineto{\pgfqpoint{1.799731in}{4.479801in}}%
\pgfpathlineto{\pgfqpoint{1.804026in}{4.472262in}}%
\pgfpathlineto{\pgfqpoint{1.805100in}{4.473858in}}%
\pgfpathlineto{\pgfqpoint{1.806174in}{4.479200in}}%
\pgfpathlineto{\pgfqpoint{1.807248in}{4.476403in}}%
\pgfpathlineto{\pgfqpoint{1.813691in}{4.473668in}}%
\pgfpathlineto{\pgfqpoint{1.814765in}{4.476924in}}%
\pgfpathlineto{\pgfqpoint{1.819060in}{4.476845in}}%
\pgfpathlineto{\pgfqpoint{1.820134in}{4.477683in}}%
\pgfpathlineto{\pgfqpoint{1.821208in}{4.482345in}}%
\pgfpathlineto{\pgfqpoint{1.822282in}{4.479737in}}%
\pgfpathlineto{\pgfqpoint{1.825503in}{4.480053in}}%
\pgfpathlineto{\pgfqpoint{1.826577in}{4.479200in}}%
\pgfpathlineto{\pgfqpoint{1.827651in}{4.479627in}}%
\pgfpathlineto{\pgfqpoint{1.828725in}{4.482677in}}%
\pgfpathlineto{\pgfqpoint{1.829798in}{4.477351in}}%
\pgfpathlineto{\pgfqpoint{1.833020in}{4.480306in}}%
\pgfpathlineto{\pgfqpoint{1.834094in}{4.482424in}}%
\pgfpathlineto{\pgfqpoint{1.835168in}{4.480844in}}%
\pgfpathlineto{\pgfqpoint{1.836241in}{4.483262in}}%
\pgfpathlineto{\pgfqpoint{1.840537in}{4.480891in}}%
\pgfpathlineto{\pgfqpoint{1.841611in}{4.482250in}}%
\pgfpathlineto{\pgfqpoint{1.842684in}{4.481808in}}%
\pgfpathlineto{\pgfqpoint{1.843758in}{4.483815in}}%
\pgfpathlineto{\pgfqpoint{1.844832in}{4.483862in}}%
\pgfpathlineto{\pgfqpoint{1.851275in}{4.485459in}}%
\pgfpathlineto{\pgfqpoint{1.852349in}{4.486913in}}%
\pgfpathlineto{\pgfqpoint{1.858792in}{4.488999in}}%
\pgfpathlineto{\pgfqpoint{1.859866in}{4.487656in}}%
\pgfpathlineto{\pgfqpoint{1.863087in}{4.490248in}}%
\pgfpathlineto{\pgfqpoint{1.865235in}{4.485569in}}%
\pgfpathlineto{\pgfqpoint{1.866309in}{4.486138in}}%
\pgfpathlineto{\pgfqpoint{1.867383in}{4.481634in}}%
\pgfpathlineto{\pgfqpoint{1.870604in}{4.484463in}}%
\pgfpathlineto{\pgfqpoint{1.871678in}{4.478694in}}%
\pgfpathlineto{\pgfqpoint{1.872752in}{4.477951in}}%
\pgfpathlineto{\pgfqpoint{1.873826in}{4.481792in}}%
\pgfpathlineto{\pgfqpoint{1.874900in}{4.479374in}}%
\pgfpathlineto{\pgfqpoint{1.878121in}{4.484226in}}%
\pgfpathlineto{\pgfqpoint{1.879195in}{4.481460in}}%
\pgfpathlineto{\pgfqpoint{1.880269in}{4.484605in}}%
\pgfpathlineto{\pgfqpoint{1.881343in}{4.483467in}}%
\pgfpathlineto{\pgfqpoint{1.882416in}{4.484621in}}%
\pgfpathlineto{\pgfqpoint{1.886712in}{4.484384in}}%
\pgfpathlineto{\pgfqpoint{1.887786in}{4.479311in}}%
\pgfpathlineto{\pgfqpoint{1.888859in}{4.479153in}}%
\pgfpathlineto{\pgfqpoint{1.893155in}{4.484005in}}%
\pgfpathlineto{\pgfqpoint{1.894229in}{4.482472in}}%
\pgfpathlineto{\pgfqpoint{1.895303in}{4.479058in}}%
\pgfpathlineto{\pgfqpoint{1.896376in}{4.479453in}}%
\pgfpathlineto{\pgfqpoint{1.901746in}{4.484226in}}%
\pgfpathlineto{\pgfqpoint{1.902819in}{4.484321in}}%
\pgfpathlineto{\pgfqpoint{1.904967in}{4.485443in}}%
\pgfpathlineto{\pgfqpoint{1.908189in}{4.483720in}}%
\pgfpathlineto{\pgfqpoint{1.911410in}{4.483783in}}%
\pgfpathlineto{\pgfqpoint{1.912484in}{4.477920in}}%
\pgfpathlineto{\pgfqpoint{1.915705in}{4.481855in}}%
\pgfpathlineto{\pgfqpoint{1.916779in}{4.481128in}}%
\pgfpathlineto{\pgfqpoint{1.917853in}{4.482076in}}%
\pgfpathlineto{\pgfqpoint{1.918927in}{4.475059in}}%
\pgfpathlineto{\pgfqpoint{1.920001in}{4.474111in}}%
\pgfpathlineto{\pgfqpoint{1.923222in}{4.472704in}}%
\pgfpathlineto{\pgfqpoint{1.924296in}{4.473194in}}%
\pgfpathlineto{\pgfqpoint{1.926444in}{4.470476in}}%
\pgfpathlineto{\pgfqpoint{1.927518in}{4.472278in}}%
\pgfpathlineto{\pgfqpoint{1.930739in}{4.474158in}}%
\pgfpathlineto{\pgfqpoint{1.931813in}{4.472594in}}%
\pgfpathlineto{\pgfqpoint{1.932887in}{4.472214in}}%
\pgfpathlineto{\pgfqpoint{1.933961in}{4.473574in}}%
\pgfpathlineto{\pgfqpoint{1.935035in}{4.476371in}}%
\pgfpathlineto{\pgfqpoint{1.939330in}{4.475628in}}%
\pgfpathlineto{\pgfqpoint{1.942551in}{4.480148in}}%
\pgfpathlineto{\pgfqpoint{1.947921in}{4.479627in}}%
\pgfpathlineto{\pgfqpoint{1.948994in}{4.479848in}}%
\pgfpathlineto{\pgfqpoint{1.950068in}{4.478331in}}%
\pgfpathlineto{\pgfqpoint{1.954364in}{4.476371in}}%
\pgfpathlineto{\pgfqpoint{1.955437in}{4.478378in}}%
\pgfpathlineto{\pgfqpoint{1.956511in}{4.478173in}}%
\pgfpathlineto{\pgfqpoint{1.957585in}{4.475644in}}%
\pgfpathlineto{\pgfqpoint{1.961881in}{4.475644in}}%
\pgfpathlineto{\pgfqpoint{1.962954in}{4.477177in}}%
\pgfpathlineto{\pgfqpoint{1.965102in}{4.472831in}}%
\pgfpathlineto{\pgfqpoint{1.968324in}{4.472135in}}%
\pgfpathlineto{\pgfqpoint{1.969397in}{4.472689in}}%
\pgfpathlineto{\pgfqpoint{1.970471in}{4.475597in}}%
\pgfpathlineto{\pgfqpoint{1.971545in}{4.476750in}}%
\pgfpathlineto{\pgfqpoint{1.972619in}{4.474190in}}%
\pgfpathlineto{\pgfqpoint{1.975840in}{4.471124in}}%
\pgfpathlineto{\pgfqpoint{1.977988in}{4.472641in}}%
\pgfpathlineto{\pgfqpoint{1.979062in}{4.476418in}}%
\pgfpathlineto{\pgfqpoint{1.980136in}{4.475470in}}%
\pgfpathlineto{\pgfqpoint{1.984431in}{4.476719in}}%
\pgfpathlineto{\pgfqpoint{1.986579in}{4.471187in}}%
\pgfpathlineto{\pgfqpoint{1.987653in}{4.472847in}}%
\pgfpathlineto{\pgfqpoint{1.990874in}{4.468421in}}%
\pgfpathlineto{\pgfqpoint{1.991948in}{4.468911in}}%
\pgfpathlineto{\pgfqpoint{1.993022in}{4.470808in}}%
\pgfpathlineto{\pgfqpoint{1.994096in}{4.470428in}}%
\pgfpathlineto{\pgfqpoint{1.998391in}{4.470002in}}%
\pgfpathlineto{\pgfqpoint{1.999465in}{4.470507in}}%
\pgfpathlineto{\pgfqpoint{2.000539in}{4.466667in}}%
\pgfpathlineto{\pgfqpoint{2.002686in}{4.469828in}}%
\pgfpathlineto{\pgfqpoint{2.006982in}{4.472404in}}%
\pgfpathlineto{\pgfqpoint{2.008056in}{4.471313in}}%
\pgfpathlineto{\pgfqpoint{2.009129in}{4.472957in}}%
\pgfpathlineto{\pgfqpoint{2.010203in}{4.472278in}}%
\pgfpathlineto{\pgfqpoint{2.013425in}{4.472878in}}%
\pgfpathlineto{\pgfqpoint{2.014499in}{4.470966in}}%
\pgfpathlineto{\pgfqpoint{2.015572in}{4.470476in}}%
\pgfpathlineto{\pgfqpoint{2.016646in}{4.462131in}}%
\pgfpathlineto{\pgfqpoint{2.020942in}{4.461025in}}%
\pgfpathlineto{\pgfqpoint{2.022015in}{4.464407in}}%
\pgfpathlineto{\pgfqpoint{2.024163in}{4.465055in}}%
\pgfpathlineto{\pgfqpoint{2.025237in}{4.464739in}}%
\pgfpathlineto{\pgfqpoint{2.028459in}{4.463016in}}%
\pgfpathlineto{\pgfqpoint{2.030606in}{4.464265in}}%
\pgfpathlineto{\pgfqpoint{2.031680in}{4.461799in}}%
\pgfpathlineto{\pgfqpoint{2.032754in}{4.461278in}}%
\pgfpathlineto{\pgfqpoint{2.035975in}{4.464596in}}%
\pgfpathlineto{\pgfqpoint{2.037049in}{4.460693in}}%
\pgfpathlineto{\pgfqpoint{2.038123in}{4.460756in}}%
\pgfpathlineto{\pgfqpoint{2.039197in}{4.459175in}}%
\pgfpathlineto{\pgfqpoint{2.040271in}{4.460424in}}%
\pgfpathlineto{\pgfqpoint{2.043492in}{4.461767in}}%
\pgfpathlineto{\pgfqpoint{2.045640in}{4.458448in}}%
\pgfpathlineto{\pgfqpoint{2.047788in}{4.453075in}}%
\pgfpathlineto{\pgfqpoint{2.052083in}{4.446769in}}%
\pgfpathlineto{\pgfqpoint{2.053157in}{4.453976in}}%
\pgfpathlineto{\pgfqpoint{2.054231in}{4.455651in}}%
\pgfpathlineto{\pgfqpoint{2.055304in}{4.456094in}}%
\pgfpathlineto{\pgfqpoint{2.058526in}{4.453154in}}%
\pgfpathlineto{\pgfqpoint{2.059600in}{4.447986in}}%
\pgfpathlineto{\pgfqpoint{2.060674in}{4.451874in}}%
\pgfpathlineto{\pgfqpoint{2.061747in}{4.452585in}}%
\pgfpathlineto{\pgfqpoint{2.062821in}{4.449882in}}%
\pgfpathlineto{\pgfqpoint{2.067117in}{4.454987in}}%
\pgfpathlineto{\pgfqpoint{2.068191in}{4.451336in}}%
\pgfpathlineto{\pgfqpoint{2.069264in}{4.451226in}}%
\pgfpathlineto{\pgfqpoint{2.070338in}{4.451905in}}%
\pgfpathlineto{\pgfqpoint{2.073560in}{4.451257in}}%
\pgfpathlineto{\pgfqpoint{2.074634in}{4.455224in}}%
\pgfpathlineto{\pgfqpoint{2.075707in}{4.456078in}}%
\pgfpathlineto{\pgfqpoint{2.076781in}{4.454308in}}%
\pgfpathlineto{\pgfqpoint{2.077855in}{4.449566in}}%
\pgfpathlineto{\pgfqpoint{2.081077in}{4.450167in}}%
\pgfpathlineto{\pgfqpoint{2.082150in}{4.447306in}}%
\pgfpathlineto{\pgfqpoint{2.084298in}{4.446674in}}%
\pgfpathlineto{\pgfqpoint{2.085372in}{4.449487in}}%
\pgfpathlineto{\pgfqpoint{2.088593in}{4.447812in}}%
\pgfpathlineto{\pgfqpoint{2.089667in}{4.452316in}}%
\pgfpathlineto{\pgfqpoint{2.090741in}{4.452632in}}%
\pgfpathlineto{\pgfqpoint{2.091815in}{4.451257in}}%
\pgfpathlineto{\pgfqpoint{2.092889in}{4.454655in}}%
\pgfpathlineto{\pgfqpoint{2.096110in}{4.459096in}}%
\pgfpathlineto{\pgfqpoint{2.097184in}{4.458338in}}%
\pgfpathlineto{\pgfqpoint{2.099332in}{4.463569in}}%
\pgfpathlineto{\pgfqpoint{2.100406in}{4.464170in}}%
\pgfpathlineto{\pgfqpoint{2.103627in}{4.464407in}}%
\pgfpathlineto{\pgfqpoint{2.105775in}{4.461925in}}%
\pgfpathlineto{\pgfqpoint{2.106849in}{4.463158in}}%
\pgfpathlineto{\pgfqpoint{2.107923in}{4.462463in}}%
\pgfpathlineto{\pgfqpoint{2.111144in}{4.461483in}}%
\pgfpathlineto{\pgfqpoint{2.113292in}{4.464059in}}%
\pgfpathlineto{\pgfqpoint{2.114366in}{4.472831in}}%
\pgfpathlineto{\pgfqpoint{2.115439in}{4.472546in}}%
\pgfpathlineto{\pgfqpoint{2.119735in}{4.473874in}}%
\pgfpathlineto{\pgfqpoint{2.120809in}{4.475786in}}%
\pgfpathlineto{\pgfqpoint{2.122956in}{4.474553in}}%
\pgfpathlineto{\pgfqpoint{2.126178in}{4.478410in}}%
\pgfpathlineto{\pgfqpoint{2.127252in}{4.476750in}}%
\pgfpathlineto{\pgfqpoint{2.130473in}{4.477462in}}%
\pgfpathlineto{\pgfqpoint{2.133695in}{4.474917in}}%
\pgfpathlineto{\pgfqpoint{2.134769in}{4.475091in}}%
\pgfpathlineto{\pgfqpoint{2.135842in}{4.477161in}}%
\pgfpathlineto{\pgfqpoint{2.136916in}{4.473257in}}%
\pgfpathlineto{\pgfqpoint{2.137990in}{4.472341in}}%
\pgfpathlineto{\pgfqpoint{2.141212in}{4.475976in}}%
\pgfpathlineto{\pgfqpoint{2.142285in}{4.474301in}}%
\pgfpathlineto{\pgfqpoint{2.144433in}{4.477699in}}%
\pgfpathlineto{\pgfqpoint{2.145507in}{4.478505in}}%
\pgfpathlineto{\pgfqpoint{2.148728in}{4.478078in}}%
\pgfpathlineto{\pgfqpoint{2.149802in}{4.476798in}}%
\pgfpathlineto{\pgfqpoint{2.150876in}{4.476624in}}%
\pgfpathlineto{\pgfqpoint{2.153024in}{4.477130in}}%
\pgfpathlineto{\pgfqpoint{2.156245in}{4.475122in}}%
\pgfpathlineto{\pgfqpoint{2.157319in}{4.475581in}}%
\pgfpathlineto{\pgfqpoint{2.159467in}{4.471977in}}%
\pgfpathlineto{\pgfqpoint{2.160541in}{4.477462in}}%
\pgfpathlineto{\pgfqpoint{2.163762in}{4.476798in}}%
\pgfpathlineto{\pgfqpoint{2.165910in}{4.474364in}}%
\pgfpathlineto{\pgfqpoint{2.166984in}{4.476466in}}%
\pgfpathlineto{\pgfqpoint{2.168058in}{4.472625in}}%
\pgfpathlineto{\pgfqpoint{2.171279in}{4.476624in}}%
\pgfpathlineto{\pgfqpoint{2.172353in}{4.463048in}}%
\pgfpathlineto{\pgfqpoint{2.173427in}{4.465640in}}%
\pgfpathlineto{\pgfqpoint{2.174501in}{4.464075in}}%
\pgfpathlineto{\pgfqpoint{2.175574in}{4.461309in}}%
\pgfpathlineto{\pgfqpoint{2.178796in}{4.462115in}}%
\pgfpathlineto{\pgfqpoint{2.182017in}{4.467410in}}%
\pgfpathlineto{\pgfqpoint{2.186313in}{4.467457in}}%
\pgfpathlineto{\pgfqpoint{2.187387in}{4.469844in}}%
\pgfpathlineto{\pgfqpoint{2.193830in}{4.461167in}}%
\pgfpathlineto{\pgfqpoint{2.194903in}{4.462084in}}%
\pgfpathlineto{\pgfqpoint{2.197051in}{4.452806in}}%
\pgfpathlineto{\pgfqpoint{2.198125in}{4.452127in}}%
\pgfpathlineto{\pgfqpoint{2.201347in}{4.452079in}}%
\pgfpathlineto{\pgfqpoint{2.202420in}{4.452648in}}%
\pgfpathlineto{\pgfqpoint{2.203494in}{4.449582in}}%
\pgfpathlineto{\pgfqpoint{2.204568in}{4.453106in}}%
\pgfpathlineto{\pgfqpoint{2.205642in}{4.449550in}}%
\pgfpathlineto{\pgfqpoint{2.209937in}{4.449124in}}%
\pgfpathlineto{\pgfqpoint{2.211011in}{4.447069in}}%
\pgfpathlineto{\pgfqpoint{2.212085in}{4.448223in}}%
\pgfpathlineto{\pgfqpoint{2.213159in}{4.450736in}}%
\pgfpathlineto{\pgfqpoint{2.216380in}{4.447938in}}%
\pgfpathlineto{\pgfqpoint{2.217454in}{4.458259in}}%
\pgfpathlineto{\pgfqpoint{2.218528in}{4.459349in}}%
\pgfpathlineto{\pgfqpoint{2.219602in}{4.461878in}}%
\pgfpathlineto{\pgfqpoint{2.220676in}{4.467141in}}%
\pgfpathlineto{\pgfqpoint{2.224971in}{4.462668in}}%
\pgfpathlineto{\pgfqpoint{2.226045in}{4.469322in}}%
\pgfpathlineto{\pgfqpoint{2.227119in}{4.470634in}}%
\pgfpathlineto{\pgfqpoint{2.231414in}{4.471298in}}%
\pgfpathlineto{\pgfqpoint{2.232488in}{4.472499in}}%
\pgfpathlineto{\pgfqpoint{2.234635in}{4.468232in}}%
\pgfpathlineto{\pgfqpoint{2.235709in}{4.472973in}}%
\pgfpathlineto{\pgfqpoint{2.240005in}{4.475233in}}%
\pgfpathlineto{\pgfqpoint{2.241079in}{4.476798in}}%
\pgfpathlineto{\pgfqpoint{2.242152in}{4.477035in}}%
\pgfpathlineto{\pgfqpoint{2.243226in}{4.476545in}}%
\pgfpathlineto{\pgfqpoint{2.246448in}{4.478347in}}%
\pgfpathlineto{\pgfqpoint{2.247522in}{4.476150in}}%
\pgfpathlineto{\pgfqpoint{2.248595in}{4.477667in}}%
\pgfpathlineto{\pgfqpoint{2.249669in}{4.480212in}}%
\pgfpathlineto{\pgfqpoint{2.250743in}{4.479168in}}%
\pgfpathlineto{\pgfqpoint{2.253965in}{4.477161in}}%
\pgfpathlineto{\pgfqpoint{2.255038in}{4.481081in}}%
\pgfpathlineto{\pgfqpoint{2.257186in}{4.480765in}}%
\pgfpathlineto{\pgfqpoint{2.258260in}{4.481760in}}%
\pgfpathlineto{\pgfqpoint{2.261481in}{4.482487in}}%
\pgfpathlineto{\pgfqpoint{2.263629in}{4.481792in}}%
\pgfpathlineto{\pgfqpoint{2.264703in}{4.481508in}}%
\pgfpathlineto{\pgfqpoint{2.265777in}{4.484368in}}%
\pgfpathlineto{\pgfqpoint{2.268998in}{4.484273in}}%
\pgfpathlineto{\pgfqpoint{2.272220in}{4.487071in}}%
\pgfpathlineto{\pgfqpoint{2.273294in}{4.489331in}}%
\pgfpathlineto{\pgfqpoint{2.279737in}{4.488082in}}%
\pgfpathlineto{\pgfqpoint{2.284032in}{4.490706in}}%
\pgfpathlineto{\pgfqpoint{2.285106in}{4.487987in}}%
\pgfpathlineto{\pgfqpoint{2.286180in}{4.491385in}}%
\pgfpathlineto{\pgfqpoint{2.287254in}{4.491212in}}%
\pgfpathlineto{\pgfqpoint{2.288327in}{4.492508in}}%
\pgfpathlineto{\pgfqpoint{2.292623in}{4.490026in}}%
\pgfpathlineto{\pgfqpoint{2.293697in}{4.491464in}}%
\pgfpathlineto{\pgfqpoint{2.294770in}{4.491970in}}%
\pgfpathlineto{\pgfqpoint{2.295844in}{4.491212in}}%
\pgfpathlineto{\pgfqpoint{2.299066in}{4.491148in}}%
\pgfpathlineto{\pgfqpoint{2.300140in}{4.493219in}}%
\pgfpathlineto{\pgfqpoint{2.301213in}{4.494025in}}%
\pgfpathlineto{\pgfqpoint{2.302287in}{4.493408in}}%
\pgfpathlineto{\pgfqpoint{2.303361in}{4.494309in}}%
\pgfpathlineto{\pgfqpoint{2.307657in}{4.495558in}}%
\pgfpathlineto{\pgfqpoint{2.309804in}{4.494135in}}%
\pgfpathlineto{\pgfqpoint{2.314100in}{4.493725in}}%
\pgfpathlineto{\pgfqpoint{2.315173in}{4.490564in}}%
\pgfpathlineto{\pgfqpoint{2.316247in}{4.492840in}}%
\pgfpathlineto{\pgfqpoint{2.317321in}{4.491749in}}%
\pgfpathlineto{\pgfqpoint{2.321616in}{4.493677in}}%
\pgfpathlineto{\pgfqpoint{2.322690in}{4.493140in}}%
\pgfpathlineto{\pgfqpoint{2.323764in}{4.491970in}}%
\pgfpathlineto{\pgfqpoint{2.324838in}{4.492855in}}%
\pgfpathlineto{\pgfqpoint{2.325912in}{4.494467in}}%
\pgfpathlineto{\pgfqpoint{2.329133in}{4.493930in}}%
\pgfpathlineto{\pgfqpoint{2.330207in}{4.496443in}}%
\pgfpathlineto{\pgfqpoint{2.331281in}{4.495732in}}%
\pgfpathlineto{\pgfqpoint{2.332355in}{4.496285in}}%
\pgfpathlineto{\pgfqpoint{2.333429in}{4.493646in}}%
\pgfpathlineto{\pgfqpoint{2.336650in}{4.495416in}}%
\pgfpathlineto{\pgfqpoint{2.337724in}{4.492903in}}%
\pgfpathlineto{\pgfqpoint{2.338798in}{4.493077in}}%
\pgfpathlineto{\pgfqpoint{2.339872in}{4.490611in}}%
\pgfpathlineto{\pgfqpoint{2.340946in}{4.490453in}}%
\pgfpathlineto{\pgfqpoint{2.344167in}{4.492049in}}%
\pgfpathlineto{\pgfqpoint{2.346315in}{4.497676in}}%
\pgfpathlineto{\pgfqpoint{2.347389in}{4.496111in}}%
\pgfpathlineto{\pgfqpoint{2.348462in}{4.496079in}}%
\pgfpathlineto{\pgfqpoint{2.352758in}{4.495258in}}%
\pgfpathlineto{\pgfqpoint{2.353832in}{4.495795in}}%
\pgfpathlineto{\pgfqpoint{2.354905in}{4.494847in}}%
\pgfpathlineto{\pgfqpoint{2.355979in}{4.495337in}}%
\pgfpathlineto{\pgfqpoint{2.361348in}{4.499746in}}%
\pgfpathlineto{\pgfqpoint{2.363496in}{4.495605in}}%
\pgfpathlineto{\pgfqpoint{2.366718in}{4.493535in}}%
\pgfpathlineto{\pgfqpoint{2.368865in}{4.494499in}}%
\pgfpathlineto{\pgfqpoint{2.369939in}{4.497597in}}%
\pgfpathlineto{\pgfqpoint{2.371013in}{4.496158in}}%
\pgfpathlineto{\pgfqpoint{2.374235in}{4.499477in}}%
\pgfpathlineto{\pgfqpoint{2.376382in}{4.499477in}}%
\pgfpathlineto{\pgfqpoint{2.377456in}{4.503650in}}%
\pgfpathlineto{\pgfqpoint{2.378530in}{4.496411in}}%
\pgfpathlineto{\pgfqpoint{2.381751in}{4.493614in}}%
\pgfpathlineto{\pgfqpoint{2.386047in}{4.505720in}}%
\pgfpathlineto{\pgfqpoint{2.390342in}{4.505847in}}%
\pgfpathlineto{\pgfqpoint{2.392490in}{4.504756in}}%
\pgfpathlineto{\pgfqpoint{2.393564in}{4.508012in}}%
\pgfpathlineto{\pgfqpoint{2.396785in}{4.509308in}}%
\pgfpathlineto{\pgfqpoint{2.397859in}{4.510968in}}%
\pgfpathlineto{\pgfqpoint{2.398933in}{4.511047in}}%
\pgfpathlineto{\pgfqpoint{2.400007in}{4.513496in}}%
\pgfpathlineto{\pgfqpoint{2.401080in}{4.514223in}}%
\pgfpathlineto{\pgfqpoint{2.405376in}{4.513986in}}%
\pgfpathlineto{\pgfqpoint{2.406450in}{4.514239in}}%
\pgfpathlineto{\pgfqpoint{2.407524in}{4.512564in}}%
\pgfpathlineto{\pgfqpoint{2.408597in}{4.512832in}}%
\pgfpathlineto{\pgfqpoint{2.411819in}{4.511647in}}%
\pgfpathlineto{\pgfqpoint{2.412893in}{4.508802in}}%
\pgfpathlineto{\pgfqpoint{2.413967in}{4.509687in}}%
\pgfpathlineto{\pgfqpoint{2.415040in}{4.509261in}}%
\pgfpathlineto{\pgfqpoint{2.416114in}{4.509814in}}%
\pgfpathlineto{\pgfqpoint{2.421483in}{4.509845in}}%
\pgfpathlineto{\pgfqpoint{2.422557in}{4.509008in}}%
\pgfpathlineto{\pgfqpoint{2.423631in}{4.510114in}}%
\pgfpathlineto{\pgfqpoint{2.429000in}{4.510478in}}%
\pgfpathlineto{\pgfqpoint{2.430074in}{4.513639in}}%
\pgfpathlineto{\pgfqpoint{2.431148in}{4.512580in}}%
\pgfpathlineto{\pgfqpoint{2.434369in}{4.513006in}}%
\pgfpathlineto{\pgfqpoint{2.435443in}{4.511094in}}%
\pgfpathlineto{\pgfqpoint{2.436517in}{4.513623in}}%
\pgfpathlineto{\pgfqpoint{2.437591in}{4.512611in}}%
\pgfpathlineto{\pgfqpoint{2.438665in}{4.513243in}}%
\pgfpathlineto{\pgfqpoint{2.441886in}{4.512453in}}%
\pgfpathlineto{\pgfqpoint{2.442960in}{4.513480in}}%
\pgfpathlineto{\pgfqpoint{2.444034in}{4.512959in}}%
\pgfpathlineto{\pgfqpoint{2.450477in}{4.514034in}}%
\pgfpathlineto{\pgfqpoint{2.451551in}{4.512706in}}%
\pgfpathlineto{\pgfqpoint{2.453699in}{4.514998in}}%
\pgfpathlineto{\pgfqpoint{2.457994in}{4.514476in}}%
\pgfpathlineto{\pgfqpoint{2.459068in}{4.513449in}}%
\pgfpathlineto{\pgfqpoint{2.460142in}{4.513955in}}%
\pgfpathlineto{\pgfqpoint{2.461215in}{4.507459in}}%
\pgfpathlineto{\pgfqpoint{2.464437in}{4.511047in}}%
\pgfpathlineto{\pgfqpoint{2.465511in}{4.508075in}}%
\pgfpathlineto{\pgfqpoint{2.466585in}{4.507427in}}%
\pgfpathlineto{\pgfqpoint{2.467658in}{4.508834in}}%
\pgfpathlineto{\pgfqpoint{2.468732in}{4.506605in}}%
\pgfpathlineto{\pgfqpoint{2.473028in}{4.510320in}}%
\pgfpathlineto{\pgfqpoint{2.474101in}{4.513212in}}%
\pgfpathlineto{\pgfqpoint{2.475175in}{4.513591in}}%
\pgfpathlineto{\pgfqpoint{2.476249in}{4.510003in}}%
\pgfpathlineto{\pgfqpoint{2.479471in}{4.507901in}}%
\pgfpathlineto{\pgfqpoint{2.480545in}{4.508423in}}%
\pgfpathlineto{\pgfqpoint{2.481618in}{4.510304in}}%
\pgfpathlineto{\pgfqpoint{2.482692in}{4.507064in}}%
\pgfpathlineto{\pgfqpoint{2.483766in}{4.508312in}}%
\pgfpathlineto{\pgfqpoint{2.486988in}{4.506574in}}%
\pgfpathlineto{\pgfqpoint{2.488061in}{4.501674in}}%
\pgfpathlineto{\pgfqpoint{2.489135in}{4.502717in}}%
\pgfpathlineto{\pgfqpoint{2.491283in}{4.501169in}}%
\pgfpathlineto{\pgfqpoint{2.494504in}{4.500884in}}%
\pgfpathlineto{\pgfqpoint{2.495578in}{4.498766in}}%
\pgfpathlineto{\pgfqpoint{2.497726in}{4.499256in}}%
\pgfpathlineto{\pgfqpoint{2.498800in}{4.499715in}}%
\pgfpathlineto{\pgfqpoint{2.505243in}{4.499019in}}%
\pgfpathlineto{\pgfqpoint{2.506317in}{4.498498in}}%
\pgfpathlineto{\pgfqpoint{2.509538in}{4.501074in}}%
\pgfpathlineto{\pgfqpoint{2.510612in}{4.493725in}}%
\pgfpathlineto{\pgfqpoint{2.511686in}{4.494135in}}%
\pgfpathlineto{\pgfqpoint{2.512760in}{4.493045in}}%
\pgfpathlineto{\pgfqpoint{2.513834in}{4.493061in}}%
\pgfpathlineto{\pgfqpoint{2.517055in}{4.492365in}}%
\pgfpathlineto{\pgfqpoint{2.518129in}{4.490848in}}%
\pgfpathlineto{\pgfqpoint{2.520277in}{4.494610in}}%
\pgfpathlineto{\pgfqpoint{2.521350in}{4.494088in}}%
\pgfpathlineto{\pgfqpoint{2.525646in}{4.500726in}}%
\pgfpathlineto{\pgfqpoint{2.526720in}{4.499794in}}%
\pgfpathlineto{\pgfqpoint{2.527793in}{4.505467in}}%
\pgfpathlineto{\pgfqpoint{2.528867in}{4.506637in}}%
\pgfpathlineto{\pgfqpoint{2.532089in}{4.503476in}}%
\pgfpathlineto{\pgfqpoint{2.533163in}{4.505404in}}%
\pgfpathlineto{\pgfqpoint{2.534236in}{4.503761in}}%
\pgfpathlineto{\pgfqpoint{2.535310in}{4.504883in}}%
\pgfpathlineto{\pgfqpoint{2.536384in}{4.505167in}}%
\pgfpathlineto{\pgfqpoint{2.539606in}{4.503049in}}%
\pgfpathlineto{\pgfqpoint{2.543901in}{4.505942in}}%
\pgfpathlineto{\pgfqpoint{2.547123in}{4.504488in}}%
\pgfpathlineto{\pgfqpoint{2.548196in}{4.504883in}}%
\pgfpathlineto{\pgfqpoint{2.549270in}{4.503365in}}%
\pgfpathlineto{\pgfqpoint{2.550344in}{4.504677in}}%
\pgfpathlineto{\pgfqpoint{2.555713in}{4.503460in}}%
\pgfpathlineto{\pgfqpoint{2.556787in}{4.509703in}}%
\pgfpathlineto{\pgfqpoint{2.557861in}{4.509450in}}%
\pgfpathlineto{\pgfqpoint{2.558935in}{4.513275in}}%
\pgfpathlineto{\pgfqpoint{2.562156in}{4.514966in}}%
\pgfpathlineto{\pgfqpoint{2.563230in}{4.513781in}}%
\pgfpathlineto{\pgfqpoint{2.564304in}{4.510509in}}%
\pgfpathlineto{\pgfqpoint{2.565378in}{4.509656in}}%
\pgfpathlineto{\pgfqpoint{2.566452in}{4.511758in}}%
\pgfpathlineto{\pgfqpoint{2.573968in}{4.513654in}}%
\pgfpathlineto{\pgfqpoint{2.578264in}{4.513907in}}%
\pgfpathlineto{\pgfqpoint{2.579338in}{4.512674in}}%
\pgfpathlineto{\pgfqpoint{2.581485in}{4.513401in}}%
\pgfpathlineto{\pgfqpoint{2.585781in}{4.512627in}}%
\pgfpathlineto{\pgfqpoint{2.586855in}{4.513038in}}%
\pgfpathlineto{\pgfqpoint{2.587928in}{4.512137in}}%
\pgfpathlineto{\pgfqpoint{2.589002in}{4.512896in}}%
\pgfpathlineto{\pgfqpoint{2.593298in}{4.510478in}}%
\pgfpathlineto{\pgfqpoint{2.594371in}{4.512406in}}%
\pgfpathlineto{\pgfqpoint{2.596519in}{4.511663in}}%
\pgfpathlineto{\pgfqpoint{2.600814in}{4.511473in}}%
\pgfpathlineto{\pgfqpoint{2.601888in}{4.513275in}}%
\pgfpathlineto{\pgfqpoint{2.604036in}{4.513275in}}%
\pgfpathlineto{\pgfqpoint{2.607257in}{4.513307in}}%
\pgfpathlineto{\pgfqpoint{2.608331in}{4.509577in}}%
\pgfpathlineto{\pgfqpoint{2.609405in}{4.510699in}}%
\pgfpathlineto{\pgfqpoint{2.610479in}{4.510825in}}%
\pgfpathlineto{\pgfqpoint{2.611553in}{4.511789in}}%
\pgfpathlineto{\pgfqpoint{2.615848in}{4.507886in}}%
\pgfpathlineto{\pgfqpoint{2.616922in}{4.508407in}}%
\pgfpathlineto{\pgfqpoint{2.617996in}{4.506953in}}%
\pgfpathlineto{\pgfqpoint{2.619070in}{4.508217in}}%
\pgfpathlineto{\pgfqpoint{2.622291in}{4.508297in}}%
\pgfpathlineto{\pgfqpoint{2.623365in}{4.509276in}}%
\pgfpathlineto{\pgfqpoint{2.626587in}{4.514034in}}%
\pgfpathlineto{\pgfqpoint{2.630882in}{4.517843in}}%
\pgfpathlineto{\pgfqpoint{2.633030in}{4.522252in}}%
\pgfpathlineto{\pgfqpoint{2.634103in}{4.521572in}}%
\pgfpathlineto{\pgfqpoint{2.638399in}{4.522189in}}%
\pgfpathlineto{\pgfqpoint{2.639473in}{4.526140in}}%
\pgfpathlineto{\pgfqpoint{2.640546in}{4.527847in}}%
\pgfpathlineto{\pgfqpoint{2.641620in}{4.528163in}}%
\pgfpathlineto{\pgfqpoint{2.644842in}{4.527420in}}%
\pgfpathlineto{\pgfqpoint{2.645916in}{4.526598in}}%
\pgfpathlineto{\pgfqpoint{2.646989in}{4.531782in}}%
\pgfpathlineto{\pgfqpoint{2.648063in}{4.531830in}}%
\pgfpathlineto{\pgfqpoint{2.649137in}{4.530976in}}%
\pgfpathlineto{\pgfqpoint{2.653433in}{4.530644in}}%
\pgfpathlineto{\pgfqpoint{2.655580in}{4.531846in}}%
\pgfpathlineto{\pgfqpoint{2.656654in}{4.533774in}}%
\pgfpathlineto{\pgfqpoint{2.659876in}{4.534232in}}%
\pgfpathlineto{\pgfqpoint{2.660949in}{4.532478in}}%
\pgfpathlineto{\pgfqpoint{2.662023in}{4.533758in}}%
\pgfpathlineto{\pgfqpoint{2.663097in}{4.532446in}}%
\pgfpathlineto{\pgfqpoint{2.664171in}{4.535481in}}%
\pgfpathlineto{\pgfqpoint{2.667392in}{4.536445in}}%
\pgfpathlineto{\pgfqpoint{2.668466in}{4.535133in}}%
\pgfpathlineto{\pgfqpoint{2.670614in}{4.535149in}}%
\pgfpathlineto{\pgfqpoint{2.671688in}{4.534216in}}%
\pgfpathlineto{\pgfqpoint{2.674909in}{4.532620in}}%
\pgfpathlineto{\pgfqpoint{2.675983in}{4.533458in}}%
\pgfpathlineto{\pgfqpoint{2.677057in}{4.532999in}}%
\pgfpathlineto{\pgfqpoint{2.678131in}{4.533884in}}%
\pgfpathlineto{\pgfqpoint{2.679205in}{4.533963in}}%
\pgfpathlineto{\pgfqpoint{2.686722in}{4.531972in}}%
\pgfpathlineto{\pgfqpoint{2.689943in}{4.531561in}}%
\pgfpathlineto{\pgfqpoint{2.691017in}{4.532098in}}%
\pgfpathlineto{\pgfqpoint{2.692091in}{4.531545in}}%
\pgfpathlineto{\pgfqpoint{2.693165in}{4.529996in}}%
\pgfpathlineto{\pgfqpoint{2.697460in}{4.532525in}}%
\pgfpathlineto{\pgfqpoint{2.699608in}{4.531735in}}%
\pgfpathlineto{\pgfqpoint{2.700681in}{4.533710in}}%
\pgfpathlineto{\pgfqpoint{2.701755in}{4.534200in}}%
\pgfpathlineto{\pgfqpoint{2.706051in}{4.539574in}}%
\pgfpathlineto{\pgfqpoint{2.707124in}{4.539384in}}%
\pgfpathlineto{\pgfqpoint{2.708198in}{4.541012in}}%
\pgfpathlineto{\pgfqpoint{2.712494in}{4.539100in}}%
\pgfpathlineto{\pgfqpoint{2.715715in}{4.546038in}}%
\pgfpathlineto{\pgfqpoint{2.716789in}{4.545880in}}%
\pgfpathlineto{\pgfqpoint{2.720011in}{4.544774in}}%
\pgfpathlineto{\pgfqpoint{2.721084in}{4.543746in}}%
\pgfpathlineto{\pgfqpoint{2.722158in}{4.541802in}}%
\pgfpathlineto{\pgfqpoint{2.724306in}{4.541692in}}%
\pgfpathlineto{\pgfqpoint{2.728601in}{4.543683in}}%
\pgfpathlineto{\pgfqpoint{2.729675in}{4.540854in}}%
\pgfpathlineto{\pgfqpoint{2.731823in}{4.542292in}}%
\pgfpathlineto{\pgfqpoint{2.735044in}{4.546323in}}%
\pgfpathlineto{\pgfqpoint{2.736118in}{4.545153in}}%
\pgfpathlineto{\pgfqpoint{2.737192in}{4.544805in}}%
\pgfpathlineto{\pgfqpoint{2.739340in}{4.549531in}}%
\pgfpathlineto{\pgfqpoint{2.743635in}{4.552155in}}%
\pgfpathlineto{\pgfqpoint{2.744709in}{4.555173in}}%
\pgfpathlineto{\pgfqpoint{2.745783in}{4.554999in}}%
\pgfpathlineto{\pgfqpoint{2.746856in}{4.558492in}}%
\pgfpathlineto{\pgfqpoint{2.750078in}{4.557781in}}%
\pgfpathlineto{\pgfqpoint{2.752226in}{4.555979in}}%
\pgfpathlineto{\pgfqpoint{2.754373in}{4.558824in}}%
\pgfpathlineto{\pgfqpoint{2.757595in}{4.559520in}}%
\pgfpathlineto{\pgfqpoint{2.759743in}{4.563408in}}%
\pgfpathlineto{\pgfqpoint{2.761890in}{4.568212in}}%
\pgfpathlineto{\pgfqpoint{2.766186in}{4.568386in}}%
\pgfpathlineto{\pgfqpoint{2.768333in}{4.566695in}}%
\pgfpathlineto{\pgfqpoint{2.769407in}{4.567706in}}%
\pgfpathlineto{\pgfqpoint{2.772629in}{4.567264in}}%
\pgfpathlineto{\pgfqpoint{2.773702in}{4.563060in}}%
\pgfpathlineto{\pgfqpoint{2.774776in}{4.564308in}}%
\pgfpathlineto{\pgfqpoint{2.775850in}{4.560199in}}%
\pgfpathlineto{\pgfqpoint{2.776924in}{4.560705in}}%
\pgfpathlineto{\pgfqpoint{2.780145in}{4.563139in}}%
\pgfpathlineto{\pgfqpoint{2.782293in}{4.563028in}}%
\pgfpathlineto{\pgfqpoint{2.783367in}{4.560452in}}%
\pgfpathlineto{\pgfqpoint{2.784441in}{4.562775in}}%
\pgfpathlineto{\pgfqpoint{2.787662in}{4.564119in}}%
\pgfpathlineto{\pgfqpoint{2.788736in}{4.562886in}}%
\pgfpathlineto{\pgfqpoint{2.789810in}{4.565320in}}%
\pgfpathlineto{\pgfqpoint{2.790884in}{4.565020in}}%
\pgfpathlineto{\pgfqpoint{2.791958in}{4.566031in}}%
\pgfpathlineto{\pgfqpoint{2.796253in}{4.565336in}}%
\pgfpathlineto{\pgfqpoint{2.798401in}{4.567027in}}%
\pgfpathlineto{\pgfqpoint{2.799475in}{4.565114in}}%
\pgfpathlineto{\pgfqpoint{2.802696in}{4.563392in}}%
\pgfpathlineto{\pgfqpoint{2.803770in}{4.547619in}}%
\pgfpathlineto{\pgfqpoint{2.804844in}{4.547081in}}%
\pgfpathlineto{\pgfqpoint{2.805918in}{4.548599in}}%
\pgfpathlineto{\pgfqpoint{2.806991in}{4.548109in}}%
\pgfpathlineto{\pgfqpoint{2.810213in}{4.550274in}}%
\pgfpathlineto{\pgfqpoint{2.813434in}{4.559851in}}%
\pgfpathlineto{\pgfqpoint{2.817730in}{4.559583in}}%
\pgfpathlineto{\pgfqpoint{2.818804in}{4.558081in}}%
\pgfpathlineto{\pgfqpoint{2.820951in}{4.557797in}}%
\pgfpathlineto{\pgfqpoint{2.822025in}{4.557418in}}%
\pgfpathlineto{\pgfqpoint{2.825247in}{4.559488in}}%
\pgfpathlineto{\pgfqpoint{2.826321in}{4.559203in}}%
\pgfpathlineto{\pgfqpoint{2.827394in}{4.560294in}}%
\pgfpathlineto{\pgfqpoint{2.829542in}{4.553782in}}%
\pgfpathlineto{\pgfqpoint{2.832764in}{4.555300in}}%
\pgfpathlineto{\pgfqpoint{2.833837in}{4.556643in}}%
\pgfpathlineto{\pgfqpoint{2.834911in}{4.554225in}}%
\pgfpathlineto{\pgfqpoint{2.835985in}{4.553403in}}%
\pgfpathlineto{\pgfqpoint{2.840280in}{4.553925in}}%
\pgfpathlineto{\pgfqpoint{2.842428in}{4.555758in}}%
\pgfpathlineto{\pgfqpoint{2.843502in}{4.556691in}}%
\pgfpathlineto{\pgfqpoint{2.848871in}{4.551712in}}%
\pgfpathlineto{\pgfqpoint{2.849945in}{4.553308in}}%
\pgfpathlineto{\pgfqpoint{2.852093in}{4.558745in}}%
\pgfpathlineto{\pgfqpoint{2.855314in}{4.564530in}}%
\pgfpathlineto{\pgfqpoint{2.857462in}{4.564419in}}%
\pgfpathlineto{\pgfqpoint{2.859610in}{4.570188in}}%
\pgfpathlineto{\pgfqpoint{2.863905in}{4.570504in}}%
\pgfpathlineto{\pgfqpoint{2.864979in}{4.565952in}}%
\pgfpathlineto{\pgfqpoint{2.866053in}{4.565841in}}%
\pgfpathlineto{\pgfqpoint{2.867126in}{4.566331in}}%
\pgfpathlineto{\pgfqpoint{2.871422in}{4.566727in}}%
\pgfpathlineto{\pgfqpoint{2.872496in}{4.564403in}}%
\pgfpathlineto{\pgfqpoint{2.874643in}{4.565035in}}%
\pgfpathlineto{\pgfqpoint{2.877865in}{4.569318in}}%
\pgfpathlineto{\pgfqpoint{2.880012in}{4.574929in}}%
\pgfpathlineto{\pgfqpoint{2.882160in}{4.574929in}}%
\pgfpathlineto{\pgfqpoint{2.887529in}{4.574913in}}%
\pgfpathlineto{\pgfqpoint{2.888603in}{4.576525in}}%
\pgfpathlineto{\pgfqpoint{2.889677in}{4.576731in}}%
\pgfpathlineto{\pgfqpoint{2.892899in}{4.578217in}}%
\pgfpathlineto{\pgfqpoint{2.893972in}{4.576778in}}%
\pgfpathlineto{\pgfqpoint{2.896120in}{4.579007in}}%
\pgfpathlineto{\pgfqpoint{2.897194in}{4.582105in}}%
\pgfpathlineto{\pgfqpoint{2.900415in}{4.582452in}}%
\pgfpathlineto{\pgfqpoint{2.901489in}{4.602034in}}%
\pgfpathlineto{\pgfqpoint{2.902563in}{4.606570in}}%
\pgfpathlineto{\pgfqpoint{2.903637in}{4.599474in}}%
\pgfpathlineto{\pgfqpoint{2.904711in}{4.602176in}}%
\pgfpathlineto{\pgfqpoint{2.909006in}{4.595365in}}%
\pgfpathlineto{\pgfqpoint{2.910080in}{4.595349in}}%
\pgfpathlineto{\pgfqpoint{2.911154in}{4.598415in}}%
\pgfpathlineto{\pgfqpoint{2.912228in}{4.598399in}}%
\pgfpathlineto{\pgfqpoint{2.916523in}{4.595159in}}%
\pgfpathlineto{\pgfqpoint{2.917597in}{4.594827in}}%
\pgfpathlineto{\pgfqpoint{2.919744in}{4.591271in}}%
\pgfpathlineto{\pgfqpoint{2.922966in}{4.592425in}}%
\pgfpathlineto{\pgfqpoint{2.924040in}{4.594084in}}%
\pgfpathlineto{\pgfqpoint{2.925114in}{4.591192in}}%
\pgfpathlineto{\pgfqpoint{2.926188in}{4.594211in}}%
\pgfpathlineto{\pgfqpoint{2.927261in}{4.594132in}}%
\pgfpathlineto{\pgfqpoint{2.930483in}{4.597309in}}%
\pgfpathlineto{\pgfqpoint{2.931557in}{4.601197in}}%
\pgfpathlineto{\pgfqpoint{2.932631in}{4.599189in}}%
\pgfpathlineto{\pgfqpoint{2.934778in}{4.598889in}}%
\pgfpathlineto{\pgfqpoint{2.938000in}{4.602824in}}%
\pgfpathlineto{\pgfqpoint{2.940147in}{4.608846in}}%
\pgfpathlineto{\pgfqpoint{2.941221in}{4.616559in}}%
\pgfpathlineto{\pgfqpoint{2.942295in}{4.613572in}}%
\pgfpathlineto{\pgfqpoint{2.946590in}{4.609225in}}%
\pgfpathlineto{\pgfqpoint{2.947664in}{4.609842in}}%
\pgfpathlineto{\pgfqpoint{2.948738in}{4.612291in}}%
\pgfpathlineto{\pgfqpoint{2.949812in}{4.609036in}}%
\pgfpathlineto{\pgfqpoint{2.953033in}{4.610790in}}%
\pgfpathlineto{\pgfqpoint{2.954107in}{4.606697in}}%
\pgfpathlineto{\pgfqpoint{2.955181in}{4.610506in}}%
\pgfpathlineto{\pgfqpoint{2.956255in}{4.608957in}}%
\pgfpathlineto{\pgfqpoint{2.957329in}{4.608830in}}%
\pgfpathlineto{\pgfqpoint{2.961624in}{4.609415in}}%
\pgfpathlineto{\pgfqpoint{2.964846in}{4.603931in}}%
\pgfpathlineto{\pgfqpoint{2.969141in}{4.605006in}}%
\pgfpathlineto{\pgfqpoint{2.970215in}{4.606128in}}%
\pgfpathlineto{\pgfqpoint{2.972363in}{4.604879in}}%
\pgfpathlineto{\pgfqpoint{2.972363in}{4.604879in}}%
\pgfusepath{stroke}%
\end{pgfscope}%
\begin{pgfscope}%
\pgfpathrectangle{\pgfqpoint{0.506453in}{4.233896in}}{\pgfqpoint{2.583333in}{0.400885in}}%
\pgfusepath{clip}%
\pgfsetroundcap%
\pgfsetroundjoin%
\pgfsetlinewidth{1.505625pt}%
\definecolor{currentstroke}{rgb}{0.121569,0.466667,0.705882}%
\pgfsetstrokecolor{currentstroke}%
\pgfsetdash{}{0pt}%
\pgfpathmoveto{\pgfqpoint{0.623878in}{4.359400in}}%
\pgfpathlineto{\pgfqpoint{0.626025in}{4.358914in}}%
\pgfpathlineto{\pgfqpoint{0.627099in}{4.358366in}}%
\pgfpathlineto{\pgfqpoint{0.631395in}{4.359557in}}%
\pgfpathlineto{\pgfqpoint{0.632468in}{4.358867in}}%
\pgfpathlineto{\pgfqpoint{0.633542in}{4.359526in}}%
\pgfpathlineto{\pgfqpoint{0.634616in}{4.358648in}}%
\pgfpathlineto{\pgfqpoint{0.638911in}{4.357833in}}%
\pgfpathlineto{\pgfqpoint{0.639985in}{4.356768in}}%
\pgfpathlineto{\pgfqpoint{0.641059in}{4.357676in}}%
\pgfpathlineto{\pgfqpoint{0.645354in}{4.357434in}}%
\pgfpathlineto{\pgfqpoint{0.647502in}{4.358508in}}%
\pgfpathlineto{\pgfqpoint{0.648576in}{4.357131in}}%
\pgfpathlineto{\pgfqpoint{0.653945in}{4.356069in}}%
\pgfpathlineto{\pgfqpoint{0.655019in}{4.356851in}}%
\pgfpathlineto{\pgfqpoint{0.663610in}{4.357662in}}%
\pgfpathlineto{\pgfqpoint{0.664684in}{4.356600in}}%
\pgfpathlineto{\pgfqpoint{0.670053in}{4.355032in}}%
\pgfpathlineto{\pgfqpoint{0.671127in}{4.355798in}}%
\pgfpathlineto{\pgfqpoint{0.676496in}{4.355740in}}%
\pgfpathlineto{\pgfqpoint{0.679717in}{4.356449in}}%
\pgfpathlineto{\pgfqpoint{0.687234in}{4.355639in}}%
\pgfpathlineto{\pgfqpoint{0.690456in}{4.355090in}}%
\pgfpathlineto{\pgfqpoint{0.691530in}{4.352575in}}%
\pgfpathlineto{\pgfqpoint{0.692603in}{4.353196in}}%
\pgfpathlineto{\pgfqpoint{0.693677in}{4.351722in}}%
\pgfpathlineto{\pgfqpoint{0.694751in}{4.351834in}}%
\pgfpathlineto{\pgfqpoint{0.697973in}{4.350978in}}%
\pgfpathlineto{\pgfqpoint{0.699046in}{4.349597in}}%
\pgfpathlineto{\pgfqpoint{0.700120in}{4.349718in}}%
\pgfpathlineto{\pgfqpoint{0.702268in}{4.347972in}}%
\pgfpathlineto{\pgfqpoint{0.706563in}{4.347749in}}%
\pgfpathlineto{\pgfqpoint{0.709785in}{4.346794in}}%
\pgfpathlineto{\pgfqpoint{0.714080in}{4.347527in}}%
\pgfpathlineto{\pgfqpoint{0.715154in}{4.346781in}}%
\pgfpathlineto{\pgfqpoint{0.717302in}{4.347592in}}%
\pgfpathlineto{\pgfqpoint{0.721597in}{4.347147in}}%
\pgfpathlineto{\pgfqpoint{0.723745in}{4.345498in}}%
\pgfpathlineto{\pgfqpoint{0.728040in}{4.344491in}}%
\pgfpathlineto{\pgfqpoint{0.730188in}{4.341781in}}%
\pgfpathlineto{\pgfqpoint{0.732335in}{4.338971in}}%
\pgfpathlineto{\pgfqpoint{0.735557in}{4.338240in}}%
\pgfpathlineto{\pgfqpoint{0.737705in}{4.336881in}}%
\pgfpathlineto{\pgfqpoint{0.738778in}{4.336559in}}%
\pgfpathlineto{\pgfqpoint{0.739852in}{4.337226in}}%
\pgfpathlineto{\pgfqpoint{0.743074in}{4.336881in}}%
\pgfpathlineto{\pgfqpoint{0.744148in}{4.335547in}}%
\pgfpathlineto{\pgfqpoint{0.747369in}{4.336378in}}%
\pgfpathlineto{\pgfqpoint{0.753812in}{4.336401in}}%
\pgfpathlineto{\pgfqpoint{0.754886in}{4.335720in}}%
\pgfpathlineto{\pgfqpoint{0.762403in}{4.333873in}}%
\pgfpathlineto{\pgfqpoint{0.766698in}{4.332973in}}%
\pgfpathlineto{\pgfqpoint{0.767772in}{4.333123in}}%
\pgfpathlineto{\pgfqpoint{0.769920in}{4.331357in}}%
\pgfpathlineto{\pgfqpoint{0.777437in}{4.329475in}}%
\pgfpathlineto{\pgfqpoint{0.781732in}{4.328598in}}%
\pgfpathlineto{\pgfqpoint{0.782806in}{4.327447in}}%
\pgfpathlineto{\pgfqpoint{0.783880in}{4.327414in}}%
\pgfpathlineto{\pgfqpoint{0.784953in}{4.326003in}}%
\pgfpathlineto{\pgfqpoint{0.789249in}{4.325699in}}%
\pgfpathlineto{\pgfqpoint{0.790323in}{4.323788in}}%
\pgfpathlineto{\pgfqpoint{0.791397in}{4.323045in}}%
\pgfpathlineto{\pgfqpoint{0.792470in}{4.323469in}}%
\pgfpathlineto{\pgfqpoint{0.795692in}{4.322782in}}%
\pgfpathlineto{\pgfqpoint{0.797840in}{4.321011in}}%
\pgfpathlineto{\pgfqpoint{0.799987in}{4.322064in}}%
\pgfpathlineto{\pgfqpoint{0.807504in}{4.321577in}}%
\pgfpathlineto{\pgfqpoint{0.813947in}{4.320261in}}%
\pgfpathlineto{\pgfqpoint{0.815021in}{4.318327in}}%
\pgfpathlineto{\pgfqpoint{0.827907in}{4.317012in}}%
\pgfpathlineto{\pgfqpoint{0.830055in}{4.315057in}}%
\pgfpathlineto{\pgfqpoint{0.833276in}{4.315425in}}%
\pgfpathlineto{\pgfqpoint{0.834350in}{4.314785in}}%
\pgfpathlineto{\pgfqpoint{0.835424in}{4.313436in}}%
\pgfpathlineto{\pgfqpoint{0.837572in}{4.312835in}}%
\pgfpathlineto{\pgfqpoint{0.840793in}{4.312407in}}%
\pgfpathlineto{\pgfqpoint{0.841867in}{4.311641in}}%
\pgfpathlineto{\pgfqpoint{0.842941in}{4.311987in}}%
\pgfpathlineto{\pgfqpoint{0.845088in}{4.309992in}}%
\pgfpathlineto{\pgfqpoint{0.851531in}{4.308952in}}%
\pgfpathlineto{\pgfqpoint{0.852605in}{4.307919in}}%
\pgfpathlineto{\pgfqpoint{0.859048in}{4.307852in}}%
\pgfpathlineto{\pgfqpoint{0.860122in}{4.308283in}}%
\pgfpathlineto{\pgfqpoint{0.865491in}{4.308440in}}%
\pgfpathlineto{\pgfqpoint{0.866565in}{4.307696in}}%
\pgfpathlineto{\pgfqpoint{0.867639in}{4.307993in}}%
\pgfpathlineto{\pgfqpoint{0.873008in}{4.307413in}}%
\pgfpathlineto{\pgfqpoint{0.874082in}{4.306986in}}%
\pgfpathlineto{\pgfqpoint{0.875156in}{4.307500in}}%
\pgfpathlineto{\pgfqpoint{0.888042in}{4.306848in}}%
\pgfpathlineto{\pgfqpoint{0.890190in}{4.305648in}}%
\pgfpathlineto{\pgfqpoint{0.893411in}{4.304393in}}%
\pgfpathlineto{\pgfqpoint{0.895559in}{4.304471in}}%
\pgfpathlineto{\pgfqpoint{0.897707in}{4.302644in}}%
\pgfpathlineto{\pgfqpoint{0.912740in}{4.301794in}}%
\pgfpathlineto{\pgfqpoint{0.917036in}{4.301452in}}%
\pgfpathlineto{\pgfqpoint{0.923479in}{4.302433in}}%
\pgfpathlineto{\pgfqpoint{0.925626in}{4.301311in}}%
\pgfpathlineto{\pgfqpoint{0.927774in}{4.301028in}}%
\pgfpathlineto{\pgfqpoint{0.930996in}{4.301048in}}%
\pgfpathlineto{\pgfqpoint{0.932069in}{4.300278in}}%
\pgfpathlineto{\pgfqpoint{0.934217in}{4.300545in}}%
\pgfpathlineto{\pgfqpoint{0.935291in}{4.299606in}}%
\pgfpathlineto{\pgfqpoint{0.938512in}{4.299394in}}%
\pgfpathlineto{\pgfqpoint{0.939586in}{4.297423in}}%
\pgfpathlineto{\pgfqpoint{0.942808in}{4.297056in}}%
\pgfpathlineto{\pgfqpoint{0.948177in}{4.296832in}}%
\pgfpathlineto{\pgfqpoint{0.949251in}{4.295973in}}%
\pgfpathlineto{\pgfqpoint{0.950325in}{4.295835in}}%
\pgfpathlineto{\pgfqpoint{0.953546in}{4.296147in}}%
\pgfpathlineto{\pgfqpoint{0.956768in}{4.294382in}}%
\pgfpathlineto{\pgfqpoint{0.962137in}{4.294644in}}%
\pgfpathlineto{\pgfqpoint{0.963211in}{4.293783in}}%
\pgfpathlineto{\pgfqpoint{0.965358in}{4.294359in}}%
\pgfpathlineto{\pgfqpoint{0.976097in}{4.293082in}}%
\pgfpathlineto{\pgfqpoint{0.980392in}{4.293484in}}%
\pgfpathlineto{\pgfqpoint{0.985761in}{4.293396in}}%
\pgfpathlineto{\pgfqpoint{0.991130in}{4.293909in}}%
\pgfpathlineto{\pgfqpoint{0.993278in}{4.292822in}}%
\pgfpathlineto{\pgfqpoint{0.995426in}{4.292445in}}%
\pgfpathlineto{\pgfqpoint{1.000795in}{4.292779in}}%
\pgfpathlineto{\pgfqpoint{1.002943in}{4.291836in}}%
\pgfpathlineto{\pgfqpoint{1.009386in}{4.291639in}}%
\pgfpathlineto{\pgfqpoint{1.010460in}{4.291261in}}%
\pgfpathlineto{\pgfqpoint{1.021198in}{4.290262in}}%
\pgfpathlineto{\pgfqpoint{1.023346in}{4.290635in}}%
\pgfpathlineto{\pgfqpoint{1.025493in}{4.290582in}}%
\pgfpathlineto{\pgfqpoint{1.043749in}{4.292367in}}%
\pgfpathlineto{\pgfqpoint{1.048044in}{4.290991in}}%
\pgfpathlineto{\pgfqpoint{1.058782in}{4.290471in}}%
\pgfpathlineto{\pgfqpoint{1.060930in}{4.290803in}}%
\pgfpathlineto{\pgfqpoint{1.063078in}{4.290948in}}%
\pgfpathlineto{\pgfqpoint{1.068447in}{4.290916in}}%
\pgfpathlineto{\pgfqpoint{1.073816in}{4.290376in}}%
\pgfpathlineto{\pgfqpoint{1.074890in}{4.290594in}}%
\pgfpathlineto{\pgfqpoint{1.075964in}{4.290113in}}%
\pgfpathlineto{\pgfqpoint{1.078111in}{4.290188in}}%
\pgfpathlineto{\pgfqpoint{1.084554in}{4.289598in}}%
\pgfpathlineto{\pgfqpoint{1.085628in}{4.289165in}}%
\pgfpathlineto{\pgfqpoint{1.099588in}{4.288886in}}%
\pgfpathlineto{\pgfqpoint{1.100662in}{4.288353in}}%
\pgfpathlineto{\pgfqpoint{1.111400in}{4.287378in}}%
\pgfpathlineto{\pgfqpoint{1.112474in}{4.287677in}}%
\pgfpathlineto{\pgfqpoint{1.114622in}{4.287439in}}%
\pgfpathlineto{\pgfqpoint{1.123213in}{4.286886in}}%
\pgfpathlineto{\pgfqpoint{1.129656in}{4.285995in}}%
\pgfpathlineto{\pgfqpoint{1.130729in}{4.286238in}}%
\pgfpathlineto{\pgfqpoint{1.136099in}{4.285889in}}%
\pgfpathlineto{\pgfqpoint{1.138246in}{4.284582in}}%
\pgfpathlineto{\pgfqpoint{1.144689in}{4.284347in}}%
\pgfpathlineto{\pgfqpoint{1.145763in}{4.283787in}}%
\pgfpathlineto{\pgfqpoint{1.172609in}{4.283278in}}%
\pgfpathlineto{\pgfqpoint{1.188717in}{4.281176in}}%
\pgfpathlineto{\pgfqpoint{1.190864in}{4.280536in}}%
\pgfpathlineto{\pgfqpoint{1.196234in}{4.280679in}}%
\pgfpathlineto{\pgfqpoint{1.197307in}{4.279942in}}%
\pgfpathlineto{\pgfqpoint{1.198381in}{4.280156in}}%
\pgfpathlineto{\pgfqpoint{1.205898in}{4.279335in}}%
\pgfpathlineto{\pgfqpoint{1.216637in}{4.278968in}}%
\pgfpathlineto{\pgfqpoint{1.218784in}{4.278686in}}%
\pgfpathlineto{\pgfqpoint{1.220932in}{4.279100in}}%
\pgfpathlineto{\pgfqpoint{1.231670in}{4.279376in}}%
\pgfpathlineto{\pgfqpoint{1.235966in}{4.279526in}}%
\pgfpathlineto{\pgfqpoint{1.248852in}{4.279750in}}%
\pgfpathlineto{\pgfqpoint{1.250999in}{4.279867in}}%
\pgfpathlineto{\pgfqpoint{1.256369in}{4.279699in}}%
\pgfpathlineto{\pgfqpoint{1.258516in}{4.279282in}}%
\pgfpathlineto{\pgfqpoint{1.263885in}{4.278837in}}%
\pgfpathlineto{\pgfqpoint{1.266033in}{4.278466in}}%
\pgfpathlineto{\pgfqpoint{1.281067in}{4.277983in}}%
\pgfpathlineto{\pgfqpoint{1.285362in}{4.277780in}}%
\pgfpathlineto{\pgfqpoint{1.288584in}{4.278006in}}%
\pgfpathlineto{\pgfqpoint{1.295027in}{4.278063in}}%
\pgfpathlineto{\pgfqpoint{1.296101in}{4.277720in}}%
\pgfpathlineto{\pgfqpoint{1.303617in}{4.277767in}}%
\pgfpathlineto{\pgfqpoint{1.333685in}{4.276555in}}%
\pgfpathlineto{\pgfqpoint{1.371269in}{4.277328in}}%
\pgfpathlineto{\pgfqpoint{1.376639in}{4.275951in}}%
\pgfpathlineto{\pgfqpoint{1.378786in}{4.275677in}}%
\pgfpathlineto{\pgfqpoint{1.386303in}{4.275265in}}%
\pgfpathlineto{\pgfqpoint{1.389525in}{4.275033in}}%
\pgfpathlineto{\pgfqpoint{1.391672in}{4.273564in}}%
\pgfpathlineto{\pgfqpoint{1.401337in}{4.274146in}}%
\pgfpathlineto{\pgfqpoint{1.415297in}{4.273671in}}%
\pgfpathlineto{\pgfqpoint{1.430330in}{4.272510in}}%
\pgfpathlineto{\pgfqpoint{1.431404in}{4.271803in}}%
\pgfpathlineto{\pgfqpoint{1.442143in}{4.270806in}}%
\pgfpathlineto{\pgfqpoint{1.444290in}{4.270458in}}%
\pgfpathlineto{\pgfqpoint{1.446438in}{4.270017in}}%
\pgfpathlineto{\pgfqpoint{1.549527in}{4.266578in}}%
\pgfpathlineto{\pgfqpoint{1.558117in}{4.266532in}}%
\pgfpathlineto{\pgfqpoint{1.566708in}{4.266776in}}%
\pgfpathlineto{\pgfqpoint{1.698790in}{4.265350in}}%
\pgfpathlineto{\pgfqpoint{1.702012in}{4.264902in}}%
\pgfpathlineto{\pgfqpoint{1.706307in}{4.264661in}}%
\pgfpathlineto{\pgfqpoint{1.709528in}{4.263601in}}%
\pgfpathlineto{\pgfqpoint{1.722415in}{4.263050in}}%
\pgfpathlineto{\pgfqpoint{1.724562in}{4.262257in}}%
\pgfpathlineto{\pgfqpoint{1.759999in}{4.262521in}}%
\pgfpathlineto{\pgfqpoint{1.792214in}{4.261119in}}%
\pgfpathlineto{\pgfqpoint{1.805100in}{4.260736in}}%
\pgfpathlineto{\pgfqpoint{1.807248in}{4.260396in}}%
\pgfpathlineto{\pgfqpoint{1.872752in}{4.258863in}}%
\pgfpathlineto{\pgfqpoint{1.881343in}{4.258282in}}%
\pgfpathlineto{\pgfqpoint{1.889933in}{4.258191in}}%
\pgfpathlineto{\pgfqpoint{1.931813in}{4.257326in}}%
\pgfpathlineto{\pgfqpoint{2.058526in}{4.255592in}}%
\pgfpathlineto{\pgfqpoint{2.067117in}{4.255253in}}%
\pgfpathlineto{\pgfqpoint{2.077855in}{4.254997in}}%
\pgfpathlineto{\pgfqpoint{2.103627in}{4.254543in}}%
\pgfpathlineto{\pgfqpoint{2.145507in}{4.254274in}}%
\pgfpathlineto{\pgfqpoint{2.189534in}{4.253694in}}%
\pgfpathlineto{\pgfqpoint{2.273294in}{4.253017in}}%
\pgfpathlineto{\pgfqpoint{2.972363in}{4.252120in}}%
\pgfpathlineto{\pgfqpoint{2.972363in}{4.252120in}}%
\pgfusepath{stroke}%
\end{pgfscope}%
\begin{pgfscope}%
\pgfsetrectcap%
\pgfsetmiterjoin%
\pgfsetlinewidth{0.803000pt}%
\definecolor{currentstroke}{rgb}{1.000000,1.000000,1.000000}%
\pgfsetstrokecolor{currentstroke}%
\pgfsetdash{}{0pt}%
\pgfpathmoveto{\pgfqpoint{0.506453in}{4.233896in}}%
\pgfpathlineto{\pgfqpoint{0.506453in}{4.634781in}}%
\pgfusepath{stroke}%
\end{pgfscope}%
\begin{pgfscope}%
\pgfsetrectcap%
\pgfsetmiterjoin%
\pgfsetlinewidth{0.803000pt}%
\definecolor{currentstroke}{rgb}{1.000000,1.000000,1.000000}%
\pgfsetstrokecolor{currentstroke}%
\pgfsetdash{}{0pt}%
\pgfpathmoveto{\pgfqpoint{3.089787in}{4.233896in}}%
\pgfpathlineto{\pgfqpoint{3.089787in}{4.634781in}}%
\pgfusepath{stroke}%
\end{pgfscope}%
\begin{pgfscope}%
\pgfsetrectcap%
\pgfsetmiterjoin%
\pgfsetlinewidth{0.803000pt}%
\definecolor{currentstroke}{rgb}{1.000000,1.000000,1.000000}%
\pgfsetstrokecolor{currentstroke}%
\pgfsetdash{}{0pt}%
\pgfpathmoveto{\pgfqpoint{0.506453in}{4.233896in}}%
\pgfpathlineto{\pgfqpoint{3.089787in}{4.233896in}}%
\pgfusepath{stroke}%
\end{pgfscope}%
\begin{pgfscope}%
\pgfsetrectcap%
\pgfsetmiterjoin%
\pgfsetlinewidth{0.803000pt}%
\definecolor{currentstroke}{rgb}{1.000000,1.000000,1.000000}%
\pgfsetstrokecolor{currentstroke}%
\pgfsetdash{}{0pt}%
\pgfpathmoveto{\pgfqpoint{0.506453in}{4.634781in}}%
\pgfpathlineto{\pgfqpoint{3.089787in}{4.634781in}}%
\pgfusepath{stroke}%
\end{pgfscope}%
\begin{pgfscope}%
\definecolor{textcolor}{rgb}{0.150000,0.150000,0.150000}%
\pgfsetstrokecolor{textcolor}%
\pgfsetfillcolor{textcolor}%
\pgftext[x=1.798120in,y=4.718114in,,base]{\color{textcolor}\rmfamily\fontsize{16.800000}{20.160000}\selectfont MMM}%
\end{pgfscope}%
\begin{pgfscope}%
\pgfsetbuttcap%
\pgfsetmiterjoin%
\definecolor{currentfill}{rgb}{0.917647,0.917647,0.949020}%
\pgfsetfillcolor{currentfill}%
\pgfsetlinewidth{0.000000pt}%
\definecolor{currentstroke}{rgb}{0.000000,0.000000,0.000000}%
\pgfsetstrokecolor{currentstroke}%
\pgfsetstrokeopacity{0.000000}%
\pgfsetdash{}{0pt}%
\pgfpathmoveto{\pgfqpoint{4.123120in}{4.233896in}}%
\pgfpathlineto{\pgfqpoint{6.706453in}{4.233896in}}%
\pgfpathlineto{\pgfqpoint{6.706453in}{4.634781in}}%
\pgfpathlineto{\pgfqpoint{4.123120in}{4.634781in}}%
\pgfpathclose%
\pgfusepath{fill}%
\end{pgfscope}%
\begin{pgfscope}%
\pgfpathrectangle{\pgfqpoint{4.123120in}{4.233896in}}{\pgfqpoint{2.583333in}{0.400885in}}%
\pgfusepath{clip}%
\pgfsetroundcap%
\pgfsetroundjoin%
\pgfsetlinewidth{0.803000pt}%
\definecolor{currentstroke}{rgb}{1.000000,1.000000,1.000000}%
\pgfsetstrokecolor{currentstroke}%
\pgfsetdash{}{0pt}%
\pgfpathmoveto{\pgfqpoint{4.238397in}{4.233896in}}%
\pgfpathlineto{\pgfqpoint{4.238397in}{4.634781in}}%
\pgfusepath{stroke}%
\end{pgfscope}%
\begin{pgfscope}%
\definecolor{textcolor}{rgb}{0.150000,0.150000,0.150000}%
\pgfsetstrokecolor{textcolor}%
\pgfsetfillcolor{textcolor}%
\pgftext[x=4.238397in,y=4.136674in,,top]{\color{textcolor}\rmfamily\fontsize{14.000000}{16.800000}\selectfont 2012}%
\end{pgfscope}%
\begin{pgfscope}%
\pgfpathrectangle{\pgfqpoint{4.123120in}{4.233896in}}{\pgfqpoint{2.583333in}{0.400885in}}%
\pgfusepath{clip}%
\pgfsetroundcap%
\pgfsetroundjoin%
\pgfsetlinewidth{0.803000pt}%
\definecolor{currentstroke}{rgb}{1.000000,1.000000,1.000000}%
\pgfsetstrokecolor{currentstroke}%
\pgfsetdash{}{0pt}%
\pgfpathmoveto{\pgfqpoint{4.631422in}{4.233896in}}%
\pgfpathlineto{\pgfqpoint{4.631422in}{4.634781in}}%
\pgfusepath{stroke}%
\end{pgfscope}%
\begin{pgfscope}%
\definecolor{textcolor}{rgb}{0.150000,0.150000,0.150000}%
\pgfsetstrokecolor{textcolor}%
\pgfsetfillcolor{textcolor}%
\pgftext[x=4.631422in,y=4.136674in,,top]{\color{textcolor}\rmfamily\fontsize{14.000000}{16.800000}\selectfont 2013}%
\end{pgfscope}%
\begin{pgfscope}%
\pgfpathrectangle{\pgfqpoint{4.123120in}{4.233896in}}{\pgfqpoint{2.583333in}{0.400885in}}%
\pgfusepath{clip}%
\pgfsetroundcap%
\pgfsetroundjoin%
\pgfsetlinewidth{0.803000pt}%
\definecolor{currentstroke}{rgb}{1.000000,1.000000,1.000000}%
\pgfsetstrokecolor{currentstroke}%
\pgfsetdash{}{0pt}%
\pgfpathmoveto{\pgfqpoint{5.023373in}{4.233896in}}%
\pgfpathlineto{\pgfqpoint{5.023373in}{4.634781in}}%
\pgfusepath{stroke}%
\end{pgfscope}%
\begin{pgfscope}%
\definecolor{textcolor}{rgb}{0.150000,0.150000,0.150000}%
\pgfsetstrokecolor{textcolor}%
\pgfsetfillcolor{textcolor}%
\pgftext[x=5.023373in,y=4.136674in,,top]{\color{textcolor}\rmfamily\fontsize{14.000000}{16.800000}\selectfont 2014}%
\end{pgfscope}%
\begin{pgfscope}%
\pgfpathrectangle{\pgfqpoint{4.123120in}{4.233896in}}{\pgfqpoint{2.583333in}{0.400885in}}%
\pgfusepath{clip}%
\pgfsetroundcap%
\pgfsetroundjoin%
\pgfsetlinewidth{0.803000pt}%
\definecolor{currentstroke}{rgb}{1.000000,1.000000,1.000000}%
\pgfsetstrokecolor{currentstroke}%
\pgfsetdash{}{0pt}%
\pgfpathmoveto{\pgfqpoint{5.415324in}{4.233896in}}%
\pgfpathlineto{\pgfqpoint{5.415324in}{4.634781in}}%
\pgfusepath{stroke}%
\end{pgfscope}%
\begin{pgfscope}%
\definecolor{textcolor}{rgb}{0.150000,0.150000,0.150000}%
\pgfsetstrokecolor{textcolor}%
\pgfsetfillcolor{textcolor}%
\pgftext[x=5.415324in,y=4.136674in,,top]{\color{textcolor}\rmfamily\fontsize{14.000000}{16.800000}\selectfont 2015}%
\end{pgfscope}%
\begin{pgfscope}%
\pgfpathrectangle{\pgfqpoint{4.123120in}{4.233896in}}{\pgfqpoint{2.583333in}{0.400885in}}%
\pgfusepath{clip}%
\pgfsetroundcap%
\pgfsetroundjoin%
\pgfsetlinewidth{0.803000pt}%
\definecolor{currentstroke}{rgb}{1.000000,1.000000,1.000000}%
\pgfsetstrokecolor{currentstroke}%
\pgfsetdash{}{0pt}%
\pgfpathmoveto{\pgfqpoint{5.807275in}{4.233896in}}%
\pgfpathlineto{\pgfqpoint{5.807275in}{4.634781in}}%
\pgfusepath{stroke}%
\end{pgfscope}%
\begin{pgfscope}%
\definecolor{textcolor}{rgb}{0.150000,0.150000,0.150000}%
\pgfsetstrokecolor{textcolor}%
\pgfsetfillcolor{textcolor}%
\pgftext[x=5.807275in,y=4.136674in,,top]{\color{textcolor}\rmfamily\fontsize{14.000000}{16.800000}\selectfont 2016}%
\end{pgfscope}%
\begin{pgfscope}%
\pgfpathrectangle{\pgfqpoint{4.123120in}{4.233896in}}{\pgfqpoint{2.583333in}{0.400885in}}%
\pgfusepath{clip}%
\pgfsetroundcap%
\pgfsetroundjoin%
\pgfsetlinewidth{0.803000pt}%
\definecolor{currentstroke}{rgb}{1.000000,1.000000,1.000000}%
\pgfsetstrokecolor{currentstroke}%
\pgfsetdash{}{0pt}%
\pgfpathmoveto{\pgfqpoint{6.200300in}{4.233896in}}%
\pgfpathlineto{\pgfqpoint{6.200300in}{4.634781in}}%
\pgfusepath{stroke}%
\end{pgfscope}%
\begin{pgfscope}%
\definecolor{textcolor}{rgb}{0.150000,0.150000,0.150000}%
\pgfsetstrokecolor{textcolor}%
\pgfsetfillcolor{textcolor}%
\pgftext[x=6.200300in,y=4.136674in,,top]{\color{textcolor}\rmfamily\fontsize{14.000000}{16.800000}\selectfont 2017}%
\end{pgfscope}%
\begin{pgfscope}%
\pgfpathrectangle{\pgfqpoint{4.123120in}{4.233896in}}{\pgfqpoint{2.583333in}{0.400885in}}%
\pgfusepath{clip}%
\pgfsetroundcap%
\pgfsetroundjoin%
\pgfsetlinewidth{0.803000pt}%
\definecolor{currentstroke}{rgb}{1.000000,1.000000,1.000000}%
\pgfsetstrokecolor{currentstroke}%
\pgfsetdash{}{0pt}%
\pgfpathmoveto{\pgfqpoint{6.592251in}{4.233896in}}%
\pgfpathlineto{\pgfqpoint{6.592251in}{4.634781in}}%
\pgfusepath{stroke}%
\end{pgfscope}%
\begin{pgfscope}%
\definecolor{textcolor}{rgb}{0.150000,0.150000,0.150000}%
\pgfsetstrokecolor{textcolor}%
\pgfsetfillcolor{textcolor}%
\pgftext[x=6.592251in,y=4.136674in,,top]{\color{textcolor}\rmfamily\fontsize{14.000000}{16.800000}\selectfont 2018}%
\end{pgfscope}%
\begin{pgfscope}%
\pgfpathrectangle{\pgfqpoint{4.123120in}{4.233896in}}{\pgfqpoint{2.583333in}{0.400885in}}%
\pgfusepath{clip}%
\pgfsetroundcap%
\pgfsetroundjoin%
\pgfsetlinewidth{0.803000pt}%
\definecolor{currentstroke}{rgb}{1.000000,1.000000,1.000000}%
\pgfsetstrokecolor{currentstroke}%
\pgfsetdash{}{0pt}%
\pgfpathmoveto{\pgfqpoint{4.123120in}{4.251771in}}%
\pgfpathlineto{\pgfqpoint{6.706453in}{4.251771in}}%
\pgfusepath{stroke}%
\end{pgfscope}%
\begin{pgfscope}%
\definecolor{textcolor}{rgb}{0.150000,0.150000,0.150000}%
\pgfsetstrokecolor{textcolor}%
\pgfsetfillcolor{textcolor}%
\pgftext[x=3.902186in,y=4.177905in,left,base]{\color{textcolor}\rmfamily\fontsize{14.000000}{16.800000}\selectfont 0}%
\end{pgfscope}%
\begin{pgfscope}%
\pgfpathrectangle{\pgfqpoint{4.123120in}{4.233896in}}{\pgfqpoint{2.583333in}{0.400885in}}%
\pgfusepath{clip}%
\pgfsetroundcap%
\pgfsetroundjoin%
\pgfsetlinewidth{0.803000pt}%
\definecolor{currentstroke}{rgb}{1.000000,1.000000,1.000000}%
\pgfsetstrokecolor{currentstroke}%
\pgfsetdash{}{0pt}%
\pgfpathmoveto{\pgfqpoint{4.123120in}{4.576560in}}%
\pgfpathlineto{\pgfqpoint{6.706453in}{4.576560in}}%
\pgfusepath{stroke}%
\end{pgfscope}%
\begin{pgfscope}%
\definecolor{textcolor}{rgb}{0.150000,0.150000,0.150000}%
\pgfsetstrokecolor{textcolor}%
\pgfsetfillcolor{textcolor}%
\pgftext[x=3.902186in,y=4.502694in,left,base]{\color{textcolor}\rmfamily\fontsize{14.000000}{16.800000}\selectfont 2}%
\end{pgfscope}%
\begin{pgfscope}%
\pgfpathrectangle{\pgfqpoint{4.123120in}{4.233896in}}{\pgfqpoint{2.583333in}{0.400885in}}%
\pgfusepath{clip}%
\pgfsetroundcap%
\pgfsetroundjoin%
\pgfsetlinewidth{1.505625pt}%
\definecolor{currentstroke}{rgb}{0.000000,0.000000,0.000000}%
\pgfsetstrokecolor{currentstroke}%
\pgfsetdash{}{0pt}%
\pgfpathmoveto{\pgfqpoint{4.240544in}{4.414166in}}%
\pgfpathlineto{\pgfqpoint{4.241618in}{4.414278in}}%
\pgfpathlineto{\pgfqpoint{4.242692in}{4.416154in}}%
\pgfpathlineto{\pgfqpoint{4.243766in}{4.414353in}}%
\pgfpathlineto{\pgfqpoint{4.246987in}{4.414766in}}%
\pgfpathlineto{\pgfqpoint{4.249135in}{4.416642in}}%
\pgfpathlineto{\pgfqpoint{4.250209in}{4.419006in}}%
\pgfpathlineto{\pgfqpoint{4.255578in}{4.420920in}}%
\pgfpathlineto{\pgfqpoint{4.257726in}{4.423396in}}%
\pgfpathlineto{\pgfqpoint{4.258800in}{4.420319in}}%
\pgfpathlineto{\pgfqpoint{4.263095in}{4.417580in}}%
\pgfpathlineto{\pgfqpoint{4.264169in}{4.420770in}}%
\pgfpathlineto{\pgfqpoint{4.269538in}{4.417243in}}%
\pgfpathlineto{\pgfqpoint{4.271686in}{4.422270in}}%
\pgfpathlineto{\pgfqpoint{4.272760in}{4.424109in}}%
\pgfpathlineto{\pgfqpoint{4.273833in}{4.427786in}}%
\pgfpathlineto{\pgfqpoint{4.277055in}{4.426285in}}%
\pgfpathlineto{\pgfqpoint{4.278129in}{4.427373in}}%
\pgfpathlineto{\pgfqpoint{4.279203in}{4.425722in}}%
\pgfpathlineto{\pgfqpoint{4.280276in}{4.427974in}}%
\pgfpathlineto{\pgfqpoint{4.281350in}{4.426285in}}%
\pgfpathlineto{\pgfqpoint{4.284572in}{4.427148in}}%
\pgfpathlineto{\pgfqpoint{4.285646in}{4.426811in}}%
\pgfpathlineto{\pgfqpoint{4.286719in}{4.425310in}}%
\pgfpathlineto{\pgfqpoint{4.287793in}{4.429850in}}%
\pgfpathlineto{\pgfqpoint{4.295310in}{4.429175in}}%
\pgfpathlineto{\pgfqpoint{4.296384in}{4.431426in}}%
\pgfpathlineto{\pgfqpoint{4.299606in}{4.434277in}}%
\pgfpathlineto{\pgfqpoint{4.300679in}{4.432852in}}%
\pgfpathlineto{\pgfqpoint{4.301753in}{4.429925in}}%
\pgfpathlineto{\pgfqpoint{4.302827in}{4.432214in}}%
\pgfpathlineto{\pgfqpoint{4.303901in}{4.430263in}}%
\pgfpathlineto{\pgfqpoint{4.307122in}{4.430188in}}%
\pgfpathlineto{\pgfqpoint{4.308196in}{4.426060in}}%
\pgfpathlineto{\pgfqpoint{4.311418in}{4.430976in}}%
\pgfpathlineto{\pgfqpoint{4.314639in}{4.429512in}}%
\pgfpathlineto{\pgfqpoint{4.317861in}{4.442832in}}%
\pgfpathlineto{\pgfqpoint{4.318935in}{4.442270in}}%
\pgfpathlineto{\pgfqpoint{4.322156in}{4.444671in}}%
\pgfpathlineto{\pgfqpoint{4.323230in}{4.443470in}}%
\pgfpathlineto{\pgfqpoint{4.326451in}{4.444596in}}%
\pgfpathlineto{\pgfqpoint{4.329673in}{4.449361in}}%
\pgfpathlineto{\pgfqpoint{4.330747in}{4.447898in}}%
\pgfpathlineto{\pgfqpoint{4.331821in}{4.450712in}}%
\pgfpathlineto{\pgfqpoint{4.332894in}{4.446772in}}%
\pgfpathlineto{\pgfqpoint{4.333968in}{4.446660in}}%
\pgfpathlineto{\pgfqpoint{4.337190in}{4.447185in}}%
\pgfpathlineto{\pgfqpoint{4.338264in}{4.449136in}}%
\pgfpathlineto{\pgfqpoint{4.339338in}{4.445684in}}%
\pgfpathlineto{\pgfqpoint{4.340411in}{4.448161in}}%
\pgfpathlineto{\pgfqpoint{4.344707in}{4.444971in}}%
\pgfpathlineto{\pgfqpoint{4.345781in}{4.441669in}}%
\pgfpathlineto{\pgfqpoint{4.347928in}{4.447935in}}%
\pgfpathlineto{\pgfqpoint{4.349002in}{4.445384in}}%
\pgfpathlineto{\pgfqpoint{4.354371in}{4.447935in}}%
\pgfpathlineto{\pgfqpoint{4.355445in}{4.446360in}}%
\pgfpathlineto{\pgfqpoint{4.356519in}{4.445947in}}%
\pgfpathlineto{\pgfqpoint{4.359740in}{4.445534in}}%
\pgfpathlineto{\pgfqpoint{4.360814in}{4.446547in}}%
\pgfpathlineto{\pgfqpoint{4.362962in}{4.453189in}}%
\pgfpathlineto{\pgfqpoint{4.364036in}{4.455140in}}%
\pgfpathlineto{\pgfqpoint{4.367257in}{4.455290in}}%
\pgfpathlineto{\pgfqpoint{4.368331in}{4.457841in}}%
\pgfpathlineto{\pgfqpoint{4.369405in}{4.458141in}}%
\pgfpathlineto{\pgfqpoint{4.370479in}{4.457466in}}%
\pgfpathlineto{\pgfqpoint{4.371553in}{4.454915in}}%
\pgfpathlineto{\pgfqpoint{4.374774in}{4.454915in}}%
\pgfpathlineto{\pgfqpoint{4.375848in}{4.454202in}}%
\pgfpathlineto{\pgfqpoint{4.376922in}{4.452738in}}%
\pgfpathlineto{\pgfqpoint{4.377996in}{4.452626in}}%
\pgfpathlineto{\pgfqpoint{4.379070in}{4.453376in}}%
\pgfpathlineto{\pgfqpoint{4.384439in}{4.445759in}}%
\pgfpathlineto{\pgfqpoint{4.385513in}{4.439868in}}%
\pgfpathlineto{\pgfqpoint{4.386586in}{4.439043in}}%
\pgfpathlineto{\pgfqpoint{4.389808in}{4.442157in}}%
\pgfpathlineto{\pgfqpoint{4.390882in}{4.442232in}}%
\pgfpathlineto{\pgfqpoint{4.391956in}{4.440956in}}%
\pgfpathlineto{\pgfqpoint{4.393029in}{4.442195in}}%
\pgfpathlineto{\pgfqpoint{4.394103in}{4.440394in}}%
\pgfpathlineto{\pgfqpoint{4.398399in}{4.442945in}}%
\pgfpathlineto{\pgfqpoint{4.399472in}{4.439230in}}%
\pgfpathlineto{\pgfqpoint{4.400546in}{4.440469in}}%
\pgfpathlineto{\pgfqpoint{4.401620in}{4.432364in}}%
\pgfpathlineto{\pgfqpoint{4.404842in}{4.432627in}}%
\pgfpathlineto{\pgfqpoint{4.405916in}{4.434240in}}%
\pgfpathlineto{\pgfqpoint{4.406989in}{4.438968in}}%
\pgfpathlineto{\pgfqpoint{4.408063in}{4.438480in}}%
\pgfpathlineto{\pgfqpoint{4.409137in}{4.440581in}}%
\pgfpathlineto{\pgfqpoint{4.412359in}{4.438030in}}%
\pgfpathlineto{\pgfqpoint{4.413432in}{4.442682in}}%
\pgfpathlineto{\pgfqpoint{4.414506in}{4.438030in}}%
\pgfpathlineto{\pgfqpoint{4.415580in}{4.437842in}}%
\pgfpathlineto{\pgfqpoint{4.416654in}{4.442007in}}%
\pgfpathlineto{\pgfqpoint{4.419875in}{4.440544in}}%
\pgfpathlineto{\pgfqpoint{4.420949in}{4.444221in}}%
\pgfpathlineto{\pgfqpoint{4.422023in}{4.445909in}}%
\pgfpathlineto{\pgfqpoint{4.423097in}{4.442120in}}%
\pgfpathlineto{\pgfqpoint{4.424171in}{4.443733in}}%
\pgfpathlineto{\pgfqpoint{4.427392in}{4.441219in}}%
\pgfpathlineto{\pgfqpoint{4.428466in}{4.441444in}}%
\pgfpathlineto{\pgfqpoint{4.429540in}{4.444033in}}%
\pgfpathlineto{\pgfqpoint{4.430614in}{4.443395in}}%
\pgfpathlineto{\pgfqpoint{4.431688in}{4.448536in}}%
\pgfpathlineto{\pgfqpoint{4.434909in}{4.451350in}}%
\pgfpathlineto{\pgfqpoint{4.435983in}{4.453264in}}%
\pgfpathlineto{\pgfqpoint{4.438131in}{4.452738in}}%
\pgfpathlineto{\pgfqpoint{4.439205in}{4.450599in}}%
\pgfpathlineto{\pgfqpoint{4.443500in}{4.449812in}}%
\pgfpathlineto{\pgfqpoint{4.444574in}{4.448761in}}%
\pgfpathlineto{\pgfqpoint{4.445648in}{4.444859in}}%
\pgfpathlineto{\pgfqpoint{4.446721in}{4.448236in}}%
\pgfpathlineto{\pgfqpoint{4.451017in}{4.450787in}}%
\pgfpathlineto{\pgfqpoint{4.452091in}{4.449474in}}%
\pgfpathlineto{\pgfqpoint{4.453164in}{4.442495in}}%
\pgfpathlineto{\pgfqpoint{4.454238in}{4.441069in}}%
\pgfpathlineto{\pgfqpoint{4.458534in}{4.440431in}}%
\pgfpathlineto{\pgfqpoint{4.459607in}{4.441857in}}%
\pgfpathlineto{\pgfqpoint{4.460681in}{4.447673in}}%
\pgfpathlineto{\pgfqpoint{4.461755in}{4.450262in}}%
\pgfpathlineto{\pgfqpoint{4.464977in}{4.449474in}}%
\pgfpathlineto{\pgfqpoint{4.468198in}{4.443358in}}%
\pgfpathlineto{\pgfqpoint{4.469272in}{4.447147in}}%
\pgfpathlineto{\pgfqpoint{4.472494in}{4.445421in}}%
\pgfpathlineto{\pgfqpoint{4.474641in}{4.448161in}}%
\pgfpathlineto{\pgfqpoint{4.475715in}{4.443283in}}%
\pgfpathlineto{\pgfqpoint{4.476789in}{4.441182in}}%
\pgfpathlineto{\pgfqpoint{4.481084in}{4.441969in}}%
\pgfpathlineto{\pgfqpoint{4.484306in}{4.447072in}}%
\pgfpathlineto{\pgfqpoint{4.488601in}{4.443733in}}%
\pgfpathlineto{\pgfqpoint{4.489675in}{4.444483in}}%
\pgfpathlineto{\pgfqpoint{4.490749in}{4.443133in}}%
\pgfpathlineto{\pgfqpoint{4.491823in}{4.446735in}}%
\pgfpathlineto{\pgfqpoint{4.495044in}{4.446510in}}%
\pgfpathlineto{\pgfqpoint{4.496118in}{4.446997in}}%
\pgfpathlineto{\pgfqpoint{4.498266in}{4.445684in}}%
\pgfpathlineto{\pgfqpoint{4.499339in}{4.449511in}}%
\pgfpathlineto{\pgfqpoint{4.503635in}{4.450562in}}%
\pgfpathlineto{\pgfqpoint{4.504709in}{4.445722in}}%
\pgfpathlineto{\pgfqpoint{4.506856in}{4.447560in}}%
\pgfpathlineto{\pgfqpoint{4.510078in}{4.446847in}}%
\pgfpathlineto{\pgfqpoint{4.511152in}{4.445872in}}%
\pgfpathlineto{\pgfqpoint{4.512226in}{4.446022in}}%
\pgfpathlineto{\pgfqpoint{4.513299in}{4.452025in}}%
\pgfpathlineto{\pgfqpoint{4.514373in}{4.452776in}}%
\pgfpathlineto{\pgfqpoint{4.517595in}{4.452476in}}%
\pgfpathlineto{\pgfqpoint{4.518669in}{4.450750in}}%
\pgfpathlineto{\pgfqpoint{4.519742in}{4.450750in}}%
\pgfpathlineto{\pgfqpoint{4.521890in}{4.448010in}}%
\pgfpathlineto{\pgfqpoint{4.525112in}{4.447373in}}%
\pgfpathlineto{\pgfqpoint{4.526185in}{4.445534in}}%
\pgfpathlineto{\pgfqpoint{4.527259in}{4.442157in}}%
\pgfpathlineto{\pgfqpoint{4.529407in}{4.444596in}}%
\pgfpathlineto{\pgfqpoint{4.532628in}{4.447560in}}%
\pgfpathlineto{\pgfqpoint{4.533702in}{4.445684in}}%
\pgfpathlineto{\pgfqpoint{4.534776in}{4.447147in}}%
\pgfpathlineto{\pgfqpoint{4.535850in}{4.450412in}}%
\pgfpathlineto{\pgfqpoint{4.536924in}{4.451087in}}%
\pgfpathlineto{\pgfqpoint{4.540145in}{4.451950in}}%
\pgfpathlineto{\pgfqpoint{4.542293in}{4.449061in}}%
\pgfpathlineto{\pgfqpoint{4.543367in}{4.450787in}}%
\pgfpathlineto{\pgfqpoint{4.544441in}{4.448798in}}%
\pgfpathlineto{\pgfqpoint{4.547662in}{4.447785in}}%
\pgfpathlineto{\pgfqpoint{4.549810in}{4.453826in}}%
\pgfpathlineto{\pgfqpoint{4.550884in}{4.447860in}}%
\pgfpathlineto{\pgfqpoint{4.551958in}{4.445309in}}%
\pgfpathlineto{\pgfqpoint{4.555179in}{4.444896in}}%
\pgfpathlineto{\pgfqpoint{4.556253in}{4.440243in}}%
\pgfpathlineto{\pgfqpoint{4.557327in}{4.439756in}}%
\pgfpathlineto{\pgfqpoint{4.559474in}{4.441519in}}%
\pgfpathlineto{\pgfqpoint{4.564844in}{4.442270in}}%
\pgfpathlineto{\pgfqpoint{4.565917in}{4.445271in}}%
\pgfpathlineto{\pgfqpoint{4.570213in}{4.443583in}}%
\pgfpathlineto{\pgfqpoint{4.571287in}{4.446397in}}%
\pgfpathlineto{\pgfqpoint{4.572360in}{4.440919in}}%
\pgfpathlineto{\pgfqpoint{4.573434in}{4.440881in}}%
\pgfpathlineto{\pgfqpoint{4.574508in}{4.441782in}}%
\pgfpathlineto{\pgfqpoint{4.577730in}{4.440731in}}%
\pgfpathlineto{\pgfqpoint{4.579877in}{4.434353in}}%
\pgfpathlineto{\pgfqpoint{4.580951in}{4.434315in}}%
\pgfpathlineto{\pgfqpoint{4.582025in}{4.436566in}}%
\pgfpathlineto{\pgfqpoint{4.585247in}{4.439756in}}%
\pgfpathlineto{\pgfqpoint{4.586320in}{4.441894in}}%
\pgfpathlineto{\pgfqpoint{4.587394in}{4.442307in}}%
\pgfpathlineto{\pgfqpoint{4.589542in}{4.444108in}}%
\pgfpathlineto{\pgfqpoint{4.592763in}{4.441294in}}%
\pgfpathlineto{\pgfqpoint{4.593837in}{4.437054in}}%
\pgfpathlineto{\pgfqpoint{4.594911in}{4.440694in}}%
\pgfpathlineto{\pgfqpoint{4.595985in}{4.442082in}}%
\pgfpathlineto{\pgfqpoint{4.597059in}{4.442007in}}%
\pgfpathlineto{\pgfqpoint{4.600280in}{4.442345in}}%
\pgfpathlineto{\pgfqpoint{4.601354in}{4.441819in}}%
\pgfpathlineto{\pgfqpoint{4.602428in}{4.443733in}}%
\pgfpathlineto{\pgfqpoint{4.603502in}{4.442757in}}%
\pgfpathlineto{\pgfqpoint{4.604576in}{4.444446in}}%
\pgfpathlineto{\pgfqpoint{4.607797in}{4.444934in}}%
\pgfpathlineto{\pgfqpoint{4.608871in}{4.445984in}}%
\pgfpathlineto{\pgfqpoint{4.609945in}{4.448048in}}%
\pgfpathlineto{\pgfqpoint{4.611019in}{4.448236in}}%
\pgfpathlineto{\pgfqpoint{4.612093in}{4.444558in}}%
\pgfpathlineto{\pgfqpoint{4.615314in}{4.446697in}}%
\pgfpathlineto{\pgfqpoint{4.616388in}{4.448573in}}%
\pgfpathlineto{\pgfqpoint{4.617462in}{4.445046in}}%
\pgfpathlineto{\pgfqpoint{4.618536in}{4.447147in}}%
\pgfpathlineto{\pgfqpoint{4.619609in}{4.447973in}}%
\pgfpathlineto{\pgfqpoint{4.622831in}{4.447560in}}%
\pgfpathlineto{\pgfqpoint{4.624979in}{4.446209in}}%
\pgfpathlineto{\pgfqpoint{4.626052in}{4.444709in}}%
\pgfpathlineto{\pgfqpoint{4.627126in}{4.444634in}}%
\pgfpathlineto{\pgfqpoint{4.630348in}{4.447410in}}%
\pgfpathlineto{\pgfqpoint{4.632495in}{4.452438in}}%
\pgfpathlineto{\pgfqpoint{4.633569in}{4.453264in}}%
\pgfpathlineto{\pgfqpoint{4.634643in}{4.455365in}}%
\pgfpathlineto{\pgfqpoint{4.637865in}{4.456228in}}%
\pgfpathlineto{\pgfqpoint{4.638938in}{4.457391in}}%
\pgfpathlineto{\pgfqpoint{4.640012in}{4.457579in}}%
\pgfpathlineto{\pgfqpoint{4.642160in}{4.460918in}}%
\pgfpathlineto{\pgfqpoint{4.645382in}{4.460805in}}%
\pgfpathlineto{\pgfqpoint{4.646455in}{4.459305in}}%
\pgfpathlineto{\pgfqpoint{4.647529in}{4.458817in}}%
\pgfpathlineto{\pgfqpoint{4.648603in}{4.459230in}}%
\pgfpathlineto{\pgfqpoint{4.649677in}{4.455928in}}%
\pgfpathlineto{\pgfqpoint{4.653972in}{4.454652in}}%
\pgfpathlineto{\pgfqpoint{4.655046in}{4.453038in}}%
\pgfpathlineto{\pgfqpoint{4.657194in}{4.454990in}}%
\pgfpathlineto{\pgfqpoint{4.660415in}{4.453526in}}%
\pgfpathlineto{\pgfqpoint{4.661489in}{4.454802in}}%
\pgfpathlineto{\pgfqpoint{4.662563in}{4.454389in}}%
\pgfpathlineto{\pgfqpoint{4.663637in}{4.452626in}}%
\pgfpathlineto{\pgfqpoint{4.664711in}{4.456378in}}%
\pgfpathlineto{\pgfqpoint{4.667932in}{4.454689in}}%
\pgfpathlineto{\pgfqpoint{4.669006in}{4.458929in}}%
\pgfpathlineto{\pgfqpoint{4.670080in}{4.458592in}}%
\pgfpathlineto{\pgfqpoint{4.671154in}{4.463995in}}%
\pgfpathlineto{\pgfqpoint{4.672227in}{4.462832in}}%
\pgfpathlineto{\pgfqpoint{4.678670in}{4.464670in}}%
\pgfpathlineto{\pgfqpoint{4.679744in}{4.462456in}}%
\pgfpathlineto{\pgfqpoint{4.684040in}{4.464483in}}%
\pgfpathlineto{\pgfqpoint{4.686187in}{4.461969in}}%
\pgfpathlineto{\pgfqpoint{4.687261in}{4.465458in}}%
\pgfpathlineto{\pgfqpoint{4.691557in}{4.463244in}}%
\pgfpathlineto{\pgfqpoint{4.692630in}{4.465233in}}%
\pgfpathlineto{\pgfqpoint{4.693704in}{4.464032in}}%
\pgfpathlineto{\pgfqpoint{4.698000in}{4.466546in}}%
\pgfpathlineto{\pgfqpoint{4.699073in}{4.470749in}}%
\pgfpathlineto{\pgfqpoint{4.700147in}{4.472587in}}%
\pgfpathlineto{\pgfqpoint{4.701221in}{4.472250in}}%
\pgfpathlineto{\pgfqpoint{4.702295in}{4.472737in}}%
\pgfpathlineto{\pgfqpoint{4.705516in}{4.475477in}}%
\pgfpathlineto{\pgfqpoint{4.706590in}{4.474764in}}%
\pgfpathlineto{\pgfqpoint{4.708738in}{4.475064in}}%
\pgfpathlineto{\pgfqpoint{4.709812in}{4.477503in}}%
\pgfpathlineto{\pgfqpoint{4.713033in}{4.476415in}}%
\pgfpathlineto{\pgfqpoint{4.714107in}{4.474201in}}%
\pgfpathlineto{\pgfqpoint{4.715181in}{4.477165in}}%
\pgfpathlineto{\pgfqpoint{4.716255in}{4.475176in}}%
\pgfpathlineto{\pgfqpoint{4.717329in}{4.477915in}}%
\pgfpathlineto{\pgfqpoint{4.720550in}{4.477540in}}%
\pgfpathlineto{\pgfqpoint{4.721624in}{4.481180in}}%
\pgfpathlineto{\pgfqpoint{4.722698in}{4.481105in}}%
\pgfpathlineto{\pgfqpoint{4.723772in}{4.482155in}}%
\pgfpathlineto{\pgfqpoint{4.728067in}{4.481480in}}%
\pgfpathlineto{\pgfqpoint{4.729141in}{4.482793in}}%
\pgfpathlineto{\pgfqpoint{4.730215in}{4.478703in}}%
\pgfpathlineto{\pgfqpoint{4.731289in}{4.480354in}}%
\pgfpathlineto{\pgfqpoint{4.732362in}{4.475439in}}%
\pgfpathlineto{\pgfqpoint{4.735584in}{4.476490in}}%
\pgfpathlineto{\pgfqpoint{4.736658in}{4.475176in}}%
\pgfpathlineto{\pgfqpoint{4.739879in}{4.476752in}}%
\pgfpathlineto{\pgfqpoint{4.743101in}{4.471349in}}%
\pgfpathlineto{\pgfqpoint{4.744175in}{4.473038in}}%
\pgfpathlineto{\pgfqpoint{4.745248in}{4.471462in}}%
\pgfpathlineto{\pgfqpoint{4.746322in}{4.474576in}}%
\pgfpathlineto{\pgfqpoint{4.747396in}{4.482080in}}%
\pgfpathlineto{\pgfqpoint{4.750618in}{4.480129in}}%
\pgfpathlineto{\pgfqpoint{4.751692in}{4.482831in}}%
\pgfpathlineto{\pgfqpoint{4.752765in}{4.482718in}}%
\pgfpathlineto{\pgfqpoint{4.753839in}{4.485157in}}%
\pgfpathlineto{\pgfqpoint{4.754913in}{4.483844in}}%
\pgfpathlineto{\pgfqpoint{4.758135in}{4.483431in}}%
\pgfpathlineto{\pgfqpoint{4.759208in}{4.486095in}}%
\pgfpathlineto{\pgfqpoint{4.760282in}{4.485645in}}%
\pgfpathlineto{\pgfqpoint{4.762430in}{4.492324in}}%
\pgfpathlineto{\pgfqpoint{4.765651in}{4.491761in}}%
\pgfpathlineto{\pgfqpoint{4.767799in}{4.492624in}}%
\pgfpathlineto{\pgfqpoint{4.773168in}{4.490823in}}%
\pgfpathlineto{\pgfqpoint{4.775316in}{4.501066in}}%
\pgfpathlineto{\pgfqpoint{4.776390in}{4.499190in}}%
\pgfpathlineto{\pgfqpoint{4.777464in}{4.502943in}}%
\pgfpathlineto{\pgfqpoint{4.780685in}{4.506620in}}%
\pgfpathlineto{\pgfqpoint{4.781759in}{4.509059in}}%
\pgfpathlineto{\pgfqpoint{4.782833in}{4.506770in}}%
\pgfpathlineto{\pgfqpoint{4.783907in}{4.507633in}}%
\pgfpathlineto{\pgfqpoint{4.784981in}{4.509621in}}%
\pgfpathlineto{\pgfqpoint{4.789276in}{4.512661in}}%
\pgfpathlineto{\pgfqpoint{4.790350in}{4.511535in}}%
\pgfpathlineto{\pgfqpoint{4.791424in}{4.512586in}}%
\pgfpathlineto{\pgfqpoint{4.792497in}{4.511122in}}%
\pgfpathlineto{\pgfqpoint{4.795719in}{4.513711in}}%
\pgfpathlineto{\pgfqpoint{4.796793in}{4.512323in}}%
\pgfpathlineto{\pgfqpoint{4.797867in}{4.507858in}}%
\pgfpathlineto{\pgfqpoint{4.800014in}{4.519077in}}%
\pgfpathlineto{\pgfqpoint{4.803236in}{4.519902in}}%
\pgfpathlineto{\pgfqpoint{4.805383in}{4.507708in}}%
\pgfpathlineto{\pgfqpoint{4.806457in}{4.509396in}}%
\pgfpathlineto{\pgfqpoint{4.807531in}{4.501742in}}%
\pgfpathlineto{\pgfqpoint{4.810753in}{4.504706in}}%
\pgfpathlineto{\pgfqpoint{4.811826in}{4.508608in}}%
\pgfpathlineto{\pgfqpoint{4.812900in}{4.506057in}}%
\pgfpathlineto{\pgfqpoint{4.813974in}{4.501554in}}%
\pgfpathlineto{\pgfqpoint{4.815048in}{4.502905in}}%
\pgfpathlineto{\pgfqpoint{4.818270in}{4.498440in}}%
\pgfpathlineto{\pgfqpoint{4.821491in}{4.509096in}}%
\pgfpathlineto{\pgfqpoint{4.822565in}{4.507858in}}%
\pgfpathlineto{\pgfqpoint{4.825786in}{4.510860in}}%
\pgfpathlineto{\pgfqpoint{4.826860in}{4.508158in}}%
\pgfpathlineto{\pgfqpoint{4.827934in}{4.508008in}}%
\pgfpathlineto{\pgfqpoint{4.830082in}{4.513974in}}%
\pgfpathlineto{\pgfqpoint{4.833303in}{4.516488in}}%
\pgfpathlineto{\pgfqpoint{4.834377in}{4.518439in}}%
\pgfpathlineto{\pgfqpoint{4.835451in}{4.514049in}}%
\pgfpathlineto{\pgfqpoint{4.836525in}{4.516113in}}%
\pgfpathlineto{\pgfqpoint{4.837599in}{4.520915in}}%
\pgfpathlineto{\pgfqpoint{4.840820in}{4.519265in}}%
\pgfpathlineto{\pgfqpoint{4.841894in}{4.520690in}}%
\pgfpathlineto{\pgfqpoint{4.842968in}{4.515662in}}%
\pgfpathlineto{\pgfqpoint{4.844042in}{4.506057in}}%
\pgfpathlineto{\pgfqpoint{4.845115in}{4.506244in}}%
\pgfpathlineto{\pgfqpoint{4.848337in}{4.508496in}}%
\pgfpathlineto{\pgfqpoint{4.849411in}{4.507445in}}%
\pgfpathlineto{\pgfqpoint{4.850485in}{4.510710in}}%
\pgfpathlineto{\pgfqpoint{4.851559in}{4.512098in}}%
\pgfpathlineto{\pgfqpoint{4.852632in}{4.510635in}}%
\pgfpathlineto{\pgfqpoint{4.855854in}{4.509659in}}%
\pgfpathlineto{\pgfqpoint{4.856928in}{4.510109in}}%
\pgfpathlineto{\pgfqpoint{4.858002in}{4.505231in}}%
\pgfpathlineto{\pgfqpoint{4.859075in}{4.511610in}}%
\pgfpathlineto{\pgfqpoint{4.864445in}{4.512248in}}%
\pgfpathlineto{\pgfqpoint{4.865518in}{4.509959in}}%
\pgfpathlineto{\pgfqpoint{4.866592in}{4.513561in}}%
\pgfpathlineto{\pgfqpoint{4.867666in}{4.511197in}}%
\pgfpathlineto{\pgfqpoint{4.870888in}{4.511160in}}%
\pgfpathlineto{\pgfqpoint{4.871961in}{4.513674in}}%
\pgfpathlineto{\pgfqpoint{4.873035in}{4.512473in}}%
\pgfpathlineto{\pgfqpoint{4.874109in}{4.509059in}}%
\pgfpathlineto{\pgfqpoint{4.875183in}{4.510034in}}%
\pgfpathlineto{\pgfqpoint{4.879478in}{4.506957in}}%
\pgfpathlineto{\pgfqpoint{4.880552in}{4.504106in}}%
\pgfpathlineto{\pgfqpoint{4.881626in}{4.505532in}}%
\pgfpathlineto{\pgfqpoint{4.882700in}{4.504819in}}%
\pgfpathlineto{\pgfqpoint{4.885921in}{4.504706in}}%
\pgfpathlineto{\pgfqpoint{4.886995in}{4.498853in}}%
\pgfpathlineto{\pgfqpoint{4.889143in}{4.499903in}}%
\pgfpathlineto{\pgfqpoint{4.890217in}{4.498853in}}%
\pgfpathlineto{\pgfqpoint{4.894512in}{4.500616in}}%
\pgfpathlineto{\pgfqpoint{4.896660in}{4.505344in}}%
\pgfpathlineto{\pgfqpoint{4.897734in}{4.503806in}}%
\pgfpathlineto{\pgfqpoint{4.900955in}{4.504856in}}%
\pgfpathlineto{\pgfqpoint{4.903103in}{4.510072in}}%
\pgfpathlineto{\pgfqpoint{4.908472in}{4.511535in}}%
\pgfpathlineto{\pgfqpoint{4.910620in}{4.518552in}}%
\pgfpathlineto{\pgfqpoint{4.911693in}{4.518289in}}%
\pgfpathlineto{\pgfqpoint{4.915989in}{4.514424in}}%
\pgfpathlineto{\pgfqpoint{4.917063in}{4.513148in}}%
\pgfpathlineto{\pgfqpoint{4.918136in}{4.512848in}}%
\pgfpathlineto{\pgfqpoint{4.919210in}{4.514011in}}%
\pgfpathlineto{\pgfqpoint{4.920284in}{4.512511in}}%
\pgfpathlineto{\pgfqpoint{4.923506in}{4.511235in}}%
\pgfpathlineto{\pgfqpoint{4.924580in}{4.512661in}}%
\pgfpathlineto{\pgfqpoint{4.926727in}{4.506882in}}%
\pgfpathlineto{\pgfqpoint{4.927801in}{4.507858in}}%
\pgfpathlineto{\pgfqpoint{4.933170in}{4.500654in}}%
\pgfpathlineto{\pgfqpoint{4.934244in}{4.509059in}}%
\pgfpathlineto{\pgfqpoint{4.935318in}{4.511610in}}%
\pgfpathlineto{\pgfqpoint{4.938539in}{4.513936in}}%
\pgfpathlineto{\pgfqpoint{4.939613in}{4.511122in}}%
\pgfpathlineto{\pgfqpoint{4.940687in}{4.514799in}}%
\pgfpathlineto{\pgfqpoint{4.941761in}{4.528270in}}%
\pgfpathlineto{\pgfqpoint{4.942835in}{4.529283in}}%
\pgfpathlineto{\pgfqpoint{4.946056in}{4.528870in}}%
\pgfpathlineto{\pgfqpoint{4.947130in}{4.530484in}}%
\pgfpathlineto{\pgfqpoint{4.948204in}{4.529658in}}%
\pgfpathlineto{\pgfqpoint{4.949278in}{4.530596in}}%
\pgfpathlineto{\pgfqpoint{4.950352in}{4.536487in}}%
\pgfpathlineto{\pgfqpoint{4.953573in}{4.536975in}}%
\pgfpathlineto{\pgfqpoint{4.954647in}{4.540089in}}%
\pgfpathlineto{\pgfqpoint{4.955721in}{4.538176in}}%
\pgfpathlineto{\pgfqpoint{4.956795in}{4.533673in}}%
\pgfpathlineto{\pgfqpoint{4.957869in}{4.534911in}}%
\pgfpathlineto{\pgfqpoint{4.962164in}{4.534011in}}%
\pgfpathlineto{\pgfqpoint{4.963238in}{4.534799in}}%
\pgfpathlineto{\pgfqpoint{4.964312in}{4.530634in}}%
\pgfpathlineto{\pgfqpoint{4.965385in}{4.533598in}}%
\pgfpathlineto{\pgfqpoint{4.968607in}{4.532322in}}%
\pgfpathlineto{\pgfqpoint{4.969681in}{4.531046in}}%
\pgfpathlineto{\pgfqpoint{4.971828in}{4.534011in}}%
\pgfpathlineto{\pgfqpoint{4.972902in}{4.537125in}}%
\pgfpathlineto{\pgfqpoint{4.976124in}{4.535549in}}%
\pgfpathlineto{\pgfqpoint{4.977198in}{4.535962in}}%
\pgfpathlineto{\pgfqpoint{4.978271in}{4.535249in}}%
\pgfpathlineto{\pgfqpoint{4.979345in}{4.541102in}}%
\pgfpathlineto{\pgfqpoint{4.980419in}{4.540765in}}%
\pgfpathlineto{\pgfqpoint{4.983641in}{4.543316in}}%
\pgfpathlineto{\pgfqpoint{4.985788in}{4.546731in}}%
\pgfpathlineto{\pgfqpoint{4.987936in}{4.547481in}}%
\pgfpathlineto{\pgfqpoint{4.991158in}{4.545717in}}%
\pgfpathlineto{\pgfqpoint{4.992231in}{4.543354in}}%
\pgfpathlineto{\pgfqpoint{4.994379in}{4.543128in}}%
\pgfpathlineto{\pgfqpoint{4.995453in}{4.547931in}}%
\pgfpathlineto{\pgfqpoint{4.998674in}{4.547293in}}%
\pgfpathlineto{\pgfqpoint{4.999748in}{4.545717in}}%
\pgfpathlineto{\pgfqpoint{5.000822in}{4.540915in}}%
\pgfpathlineto{\pgfqpoint{5.001896in}{4.538851in}}%
\pgfpathlineto{\pgfqpoint{5.002970in}{4.540164in}}%
\pgfpathlineto{\pgfqpoint{5.006191in}{4.543053in}}%
\pgfpathlineto{\pgfqpoint{5.007265in}{4.541628in}}%
\pgfpathlineto{\pgfqpoint{5.008339in}{4.548119in}}%
\pgfpathlineto{\pgfqpoint{5.009413in}{4.549582in}}%
\pgfpathlineto{\pgfqpoint{5.010487in}{4.553560in}}%
\pgfpathlineto{\pgfqpoint{5.018003in}{4.559150in}}%
\pgfpathlineto{\pgfqpoint{5.021225in}{4.560539in}}%
\pgfpathlineto{\pgfqpoint{5.022299in}{4.564441in}}%
\pgfpathlineto{\pgfqpoint{5.024447in}{4.560051in}}%
\pgfpathlineto{\pgfqpoint{5.025520in}{4.561064in}}%
\pgfpathlineto{\pgfqpoint{5.028742in}{4.560914in}}%
\pgfpathlineto{\pgfqpoint{5.029816in}{4.559751in}}%
\pgfpathlineto{\pgfqpoint{5.030890in}{4.560689in}}%
\pgfpathlineto{\pgfqpoint{5.036259in}{4.552359in}}%
\pgfpathlineto{\pgfqpoint{5.037333in}{4.552809in}}%
\pgfpathlineto{\pgfqpoint{5.038406in}{4.556711in}}%
\pgfpathlineto{\pgfqpoint{5.039480in}{4.555060in}}%
\pgfpathlineto{\pgfqpoint{5.040554in}{4.566092in}}%
\pgfpathlineto{\pgfqpoint{5.044849in}{4.564779in}}%
\pgfpathlineto{\pgfqpoint{5.045923in}{4.566655in}}%
\pgfpathlineto{\pgfqpoint{5.048071in}{4.552209in}}%
\pgfpathlineto{\pgfqpoint{5.051292in}{4.547969in}}%
\pgfpathlineto{\pgfqpoint{5.052366in}{4.551121in}}%
\pgfpathlineto{\pgfqpoint{5.053440in}{4.547368in}}%
\pgfpathlineto{\pgfqpoint{5.054514in}{4.551046in}}%
\pgfpathlineto{\pgfqpoint{5.055588in}{4.545530in}}%
\pgfpathlineto{\pgfqpoint{5.058809in}{4.538063in}}%
\pgfpathlineto{\pgfqpoint{5.059883in}{4.542003in}}%
\pgfpathlineto{\pgfqpoint{5.060957in}{4.541027in}}%
\pgfpathlineto{\pgfqpoint{5.063105in}{4.552359in}}%
\pgfpathlineto{\pgfqpoint{5.068474in}{4.559338in}}%
\pgfpathlineto{\pgfqpoint{5.069548in}{4.558925in}}%
\pgfpathlineto{\pgfqpoint{5.070622in}{4.559300in}}%
\pgfpathlineto{\pgfqpoint{5.074917in}{4.559338in}}%
\pgfpathlineto{\pgfqpoint{5.075991in}{4.558738in}}%
\pgfpathlineto{\pgfqpoint{5.077065in}{4.559338in}}%
\pgfpathlineto{\pgfqpoint{5.078138in}{4.558437in}}%
\pgfpathlineto{\pgfqpoint{5.081360in}{4.562452in}}%
\pgfpathlineto{\pgfqpoint{5.082434in}{4.562415in}}%
\pgfpathlineto{\pgfqpoint{5.083508in}{4.561814in}}%
\pgfpathlineto{\pgfqpoint{5.084581in}{4.563728in}}%
\pgfpathlineto{\pgfqpoint{5.085655in}{4.567180in}}%
\pgfpathlineto{\pgfqpoint{5.088877in}{4.562640in}}%
\pgfpathlineto{\pgfqpoint{5.089951in}{4.571758in}}%
\pgfpathlineto{\pgfqpoint{5.091024in}{4.570144in}}%
\pgfpathlineto{\pgfqpoint{5.092098in}{4.574909in}}%
\pgfpathlineto{\pgfqpoint{5.093172in}{4.576073in}}%
\pgfpathlineto{\pgfqpoint{5.096394in}{4.575510in}}%
\pgfpathlineto{\pgfqpoint{5.098541in}{4.572771in}}%
\pgfpathlineto{\pgfqpoint{5.099615in}{4.565116in}}%
\pgfpathlineto{\pgfqpoint{5.100689in}{4.563315in}}%
\pgfpathlineto{\pgfqpoint{5.104984in}{4.568231in}}%
\pgfpathlineto{\pgfqpoint{5.106058in}{4.565266in}}%
\pgfpathlineto{\pgfqpoint{5.107132in}{4.568568in}}%
\pgfpathlineto{\pgfqpoint{5.112501in}{4.565642in}}%
\pgfpathlineto{\pgfqpoint{5.113575in}{4.561552in}}%
\pgfpathlineto{\pgfqpoint{5.115723in}{4.564328in}}%
\pgfpathlineto{\pgfqpoint{5.118944in}{4.562827in}}%
\pgfpathlineto{\pgfqpoint{5.120018in}{4.566767in}}%
\pgfpathlineto{\pgfqpoint{5.121092in}{4.564929in}}%
\pgfpathlineto{\pgfqpoint{5.122166in}{4.566917in}}%
\pgfpathlineto{\pgfqpoint{5.123240in}{4.560651in}}%
\pgfpathlineto{\pgfqpoint{5.126461in}{4.551758in}}%
\pgfpathlineto{\pgfqpoint{5.127535in}{4.551383in}}%
\pgfpathlineto{\pgfqpoint{5.128609in}{4.559113in}}%
\pgfpathlineto{\pgfqpoint{5.129683in}{4.547443in}}%
\pgfpathlineto{\pgfqpoint{5.130757in}{4.544629in}}%
\pgfpathlineto{\pgfqpoint{5.135052in}{4.549807in}}%
\pgfpathlineto{\pgfqpoint{5.136126in}{4.554535in}}%
\pgfpathlineto{\pgfqpoint{5.137200in}{4.550445in}}%
\pgfpathlineto{\pgfqpoint{5.141495in}{4.551984in}}%
\pgfpathlineto{\pgfqpoint{5.142569in}{4.553372in}}%
\pgfpathlineto{\pgfqpoint{5.143643in}{4.553597in}}%
\pgfpathlineto{\pgfqpoint{5.144716in}{4.554573in}}%
\pgfpathlineto{\pgfqpoint{5.145790in}{4.553259in}}%
\pgfpathlineto{\pgfqpoint{5.149012in}{4.553334in}}%
\pgfpathlineto{\pgfqpoint{5.150086in}{4.555886in}}%
\pgfpathlineto{\pgfqpoint{5.153307in}{4.552884in}}%
\pgfpathlineto{\pgfqpoint{5.156529in}{4.554385in}}%
\pgfpathlineto{\pgfqpoint{5.157602in}{4.550370in}}%
\pgfpathlineto{\pgfqpoint{5.158676in}{4.556524in}}%
\pgfpathlineto{\pgfqpoint{5.159750in}{4.558738in}}%
\pgfpathlineto{\pgfqpoint{5.164046in}{4.562340in}}%
\pgfpathlineto{\pgfqpoint{5.166193in}{4.558212in}}%
\pgfpathlineto{\pgfqpoint{5.167267in}{4.555210in}}%
\pgfpathlineto{\pgfqpoint{5.168341in}{4.554873in}}%
\pgfpathlineto{\pgfqpoint{5.171562in}{4.557012in}}%
\pgfpathlineto{\pgfqpoint{5.172636in}{4.553447in}}%
\pgfpathlineto{\pgfqpoint{5.173710in}{4.556111in}}%
\pgfpathlineto{\pgfqpoint{5.174784in}{4.557124in}}%
\pgfpathlineto{\pgfqpoint{5.180153in}{4.568343in}}%
\pgfpathlineto{\pgfqpoint{5.181227in}{4.567217in}}%
\pgfpathlineto{\pgfqpoint{5.183375in}{4.568718in}}%
\pgfpathlineto{\pgfqpoint{5.186596in}{4.570069in}}%
\pgfpathlineto{\pgfqpoint{5.187670in}{4.569506in}}%
\pgfpathlineto{\pgfqpoint{5.188744in}{4.569807in}}%
\pgfpathlineto{\pgfqpoint{5.189818in}{4.573221in}}%
\pgfpathlineto{\pgfqpoint{5.190891in}{4.580538in}}%
\pgfpathlineto{\pgfqpoint{5.194113in}{4.582827in}}%
\pgfpathlineto{\pgfqpoint{5.197335in}{4.580012in}}%
\pgfpathlineto{\pgfqpoint{5.198408in}{4.580313in}}%
\pgfpathlineto{\pgfqpoint{5.201630in}{4.578662in}}%
\pgfpathlineto{\pgfqpoint{5.202704in}{4.579675in}}%
\pgfpathlineto{\pgfqpoint{5.203778in}{4.582789in}}%
\pgfpathlineto{\pgfqpoint{5.204851in}{4.581101in}}%
\pgfpathlineto{\pgfqpoint{5.205925in}{4.582714in}}%
\pgfpathlineto{\pgfqpoint{5.209147in}{4.582601in}}%
\pgfpathlineto{\pgfqpoint{5.210221in}{4.578887in}}%
\pgfpathlineto{\pgfqpoint{5.211294in}{4.579600in}}%
\pgfpathlineto{\pgfqpoint{5.212368in}{4.578437in}}%
\pgfpathlineto{\pgfqpoint{5.213442in}{4.580613in}}%
\pgfpathlineto{\pgfqpoint{5.216664in}{4.580388in}}%
\pgfpathlineto{\pgfqpoint{5.217737in}{4.582076in}}%
\pgfpathlineto{\pgfqpoint{5.218811in}{4.581588in}}%
\pgfpathlineto{\pgfqpoint{5.219885in}{4.583765in}}%
\pgfpathlineto{\pgfqpoint{5.224180in}{4.582226in}}%
\pgfpathlineto{\pgfqpoint{5.225254in}{4.579375in}}%
\pgfpathlineto{\pgfqpoint{5.226328in}{4.580838in}}%
\pgfpathlineto{\pgfqpoint{5.227402in}{4.579862in}}%
\pgfpathlineto{\pgfqpoint{5.233845in}{4.579900in}}%
\pgfpathlineto{\pgfqpoint{5.234919in}{4.574759in}}%
\pgfpathlineto{\pgfqpoint{5.235993in}{4.576635in}}%
\pgfpathlineto{\pgfqpoint{5.239214in}{4.574384in}}%
\pgfpathlineto{\pgfqpoint{5.240288in}{4.576110in}}%
\pgfpathlineto{\pgfqpoint{5.242436in}{4.575322in}}%
\pgfpathlineto{\pgfqpoint{5.243510in}{4.571082in}}%
\pgfpathlineto{\pgfqpoint{5.247805in}{4.570332in}}%
\pgfpathlineto{\pgfqpoint{5.248879in}{4.567555in}}%
\pgfpathlineto{\pgfqpoint{5.251026in}{4.552134in}}%
\pgfpathlineto{\pgfqpoint{5.254248in}{4.553710in}}%
\pgfpathlineto{\pgfqpoint{5.255322in}{4.551758in}}%
\pgfpathlineto{\pgfqpoint{5.256396in}{4.551871in}}%
\pgfpathlineto{\pgfqpoint{5.257469in}{4.550558in}}%
\pgfpathlineto{\pgfqpoint{5.258543in}{4.555586in}}%
\pgfpathlineto{\pgfqpoint{5.261765in}{4.553785in}}%
\pgfpathlineto{\pgfqpoint{5.262839in}{4.554235in}}%
\pgfpathlineto{\pgfqpoint{5.263912in}{4.555361in}}%
\pgfpathlineto{\pgfqpoint{5.264986in}{4.554910in}}%
\pgfpathlineto{\pgfqpoint{5.266060in}{4.552584in}}%
\pgfpathlineto{\pgfqpoint{5.269282in}{4.554535in}}%
\pgfpathlineto{\pgfqpoint{5.270356in}{4.557875in}}%
\pgfpathlineto{\pgfqpoint{5.271429in}{4.559150in}}%
\pgfpathlineto{\pgfqpoint{5.272503in}{4.561402in}}%
\pgfpathlineto{\pgfqpoint{5.273577in}{4.560501in}}%
\pgfpathlineto{\pgfqpoint{5.276799in}{4.562978in}}%
\pgfpathlineto{\pgfqpoint{5.277872in}{4.561439in}}%
\pgfpathlineto{\pgfqpoint{5.278946in}{4.561739in}}%
\pgfpathlineto{\pgfqpoint{5.280020in}{4.560989in}}%
\pgfpathlineto{\pgfqpoint{5.281094in}{4.562827in}}%
\pgfpathlineto{\pgfqpoint{5.285389in}{4.563428in}}%
\pgfpathlineto{\pgfqpoint{5.286463in}{4.564891in}}%
\pgfpathlineto{\pgfqpoint{5.287537in}{4.563165in}}%
\pgfpathlineto{\pgfqpoint{5.288611in}{4.563015in}}%
\pgfpathlineto{\pgfqpoint{5.291832in}{4.560689in}}%
\pgfpathlineto{\pgfqpoint{5.292906in}{4.557087in}}%
\pgfpathlineto{\pgfqpoint{5.293980in}{4.558850in}}%
\pgfpathlineto{\pgfqpoint{5.295054in}{4.558888in}}%
\pgfpathlineto{\pgfqpoint{5.296128in}{4.556186in}}%
\pgfpathlineto{\pgfqpoint{5.299349in}{4.555286in}}%
\pgfpathlineto{\pgfqpoint{5.302571in}{4.564741in}}%
\pgfpathlineto{\pgfqpoint{5.303645in}{4.563353in}}%
\pgfpathlineto{\pgfqpoint{5.306866in}{4.560876in}}%
\pgfpathlineto{\pgfqpoint{5.307940in}{4.558512in}}%
\pgfpathlineto{\pgfqpoint{5.309014in}{4.559263in}}%
\pgfpathlineto{\pgfqpoint{5.310088in}{4.553147in}}%
\pgfpathlineto{\pgfqpoint{5.311161in}{4.558738in}}%
\pgfpathlineto{\pgfqpoint{5.314383in}{4.557274in}}%
\pgfpathlineto{\pgfqpoint{5.315457in}{4.555848in}}%
\pgfpathlineto{\pgfqpoint{5.316531in}{4.550708in}}%
\pgfpathlineto{\pgfqpoint{5.317604in}{4.550933in}}%
\pgfpathlineto{\pgfqpoint{5.318678in}{4.555436in}}%
\pgfpathlineto{\pgfqpoint{5.321900in}{4.554985in}}%
\pgfpathlineto{\pgfqpoint{5.322974in}{4.549132in}}%
\pgfpathlineto{\pgfqpoint{5.324047in}{4.556299in}}%
\pgfpathlineto{\pgfqpoint{5.326195in}{4.547856in}}%
\pgfpathlineto{\pgfqpoint{5.329417in}{4.540164in}}%
\pgfpathlineto{\pgfqpoint{5.330490in}{4.540014in}}%
\pgfpathlineto{\pgfqpoint{5.331564in}{4.533710in}}%
\pgfpathlineto{\pgfqpoint{5.332638in}{4.531309in}}%
\pgfpathlineto{\pgfqpoint{5.333712in}{4.539451in}}%
\pgfpathlineto{\pgfqpoint{5.336934in}{4.544442in}}%
\pgfpathlineto{\pgfqpoint{5.338007in}{4.550145in}}%
\pgfpathlineto{\pgfqpoint{5.339081in}{4.544292in}}%
\pgfpathlineto{\pgfqpoint{5.341229in}{4.552772in}}%
\pgfpathlineto{\pgfqpoint{5.344450in}{4.553597in}}%
\pgfpathlineto{\pgfqpoint{5.345524in}{4.558400in}}%
\pgfpathlineto{\pgfqpoint{5.347672in}{4.560914in}}%
\pgfpathlineto{\pgfqpoint{5.348746in}{4.565154in}}%
\pgfpathlineto{\pgfqpoint{5.351967in}{4.568268in}}%
\pgfpathlineto{\pgfqpoint{5.353041in}{4.570144in}}%
\pgfpathlineto{\pgfqpoint{5.354115in}{4.573746in}}%
\pgfpathlineto{\pgfqpoint{5.355189in}{4.570820in}}%
\pgfpathlineto{\pgfqpoint{5.356263in}{4.573183in}}%
\pgfpathlineto{\pgfqpoint{5.359484in}{4.573671in}}%
\pgfpathlineto{\pgfqpoint{5.360558in}{4.571382in}}%
\pgfpathlineto{\pgfqpoint{5.361632in}{4.570707in}}%
\pgfpathlineto{\pgfqpoint{5.363779in}{4.567668in}}%
\pgfpathlineto{\pgfqpoint{5.367001in}{4.565754in}}%
\pgfpathlineto{\pgfqpoint{5.368075in}{4.567330in}}%
\pgfpathlineto{\pgfqpoint{5.369149in}{4.567067in}}%
\pgfpathlineto{\pgfqpoint{5.370223in}{4.567480in}}%
\pgfpathlineto{\pgfqpoint{5.371296in}{4.566692in}}%
\pgfpathlineto{\pgfqpoint{5.376666in}{4.570369in}}%
\pgfpathlineto{\pgfqpoint{5.378813in}{4.573746in}}%
\pgfpathlineto{\pgfqpoint{5.382035in}{4.572771in}}%
\pgfpathlineto{\pgfqpoint{5.383109in}{4.575772in}}%
\pgfpathlineto{\pgfqpoint{5.384182in}{4.569656in}}%
\pgfpathlineto{\pgfqpoint{5.386330in}{4.574534in}}%
\pgfpathlineto{\pgfqpoint{5.389552in}{4.577724in}}%
\pgfpathlineto{\pgfqpoint{5.390625in}{4.577011in}}%
\pgfpathlineto{\pgfqpoint{5.391699in}{4.574909in}}%
\pgfpathlineto{\pgfqpoint{5.392773in}{4.576223in}}%
\pgfpathlineto{\pgfqpoint{5.393847in}{4.568793in}}%
\pgfpathlineto{\pgfqpoint{5.397068in}{4.565454in}}%
\pgfpathlineto{\pgfqpoint{5.398142in}{4.559188in}}%
\pgfpathlineto{\pgfqpoint{5.400290in}{4.576373in}}%
\pgfpathlineto{\pgfqpoint{5.401364in}{4.575435in}}%
\pgfpathlineto{\pgfqpoint{5.406733in}{4.579525in}}%
\pgfpathlineto{\pgfqpoint{5.408881in}{4.580275in}}%
\pgfpathlineto{\pgfqpoint{5.413176in}{4.580200in}}%
\pgfpathlineto{\pgfqpoint{5.414250in}{4.575923in}}%
\pgfpathlineto{\pgfqpoint{5.416398in}{4.575848in}}%
\pgfpathlineto{\pgfqpoint{5.419619in}{4.567255in}}%
\pgfpathlineto{\pgfqpoint{5.420693in}{4.560539in}}%
\pgfpathlineto{\pgfqpoint{5.422841in}{4.571758in}}%
\pgfpathlineto{\pgfqpoint{5.423914in}{4.567705in}}%
\pgfpathlineto{\pgfqpoint{5.428210in}{4.563540in}}%
\pgfpathlineto{\pgfqpoint{5.430357in}{4.551834in}}%
\pgfpathlineto{\pgfqpoint{5.431431in}{4.552396in}}%
\pgfpathlineto{\pgfqpoint{5.436800in}{4.558100in}}%
\pgfpathlineto{\pgfqpoint{5.437874in}{4.546580in}}%
\pgfpathlineto{\pgfqpoint{5.442170in}{4.542828in}}%
\pgfpathlineto{\pgfqpoint{5.444317in}{4.537350in}}%
\pgfpathlineto{\pgfqpoint{5.445391in}{4.538288in}}%
\pgfpathlineto{\pgfqpoint{5.446465in}{4.533710in}}%
\pgfpathlineto{\pgfqpoint{5.449687in}{4.538738in}}%
\pgfpathlineto{\pgfqpoint{5.450760in}{4.544329in}}%
\pgfpathlineto{\pgfqpoint{5.451834in}{4.543916in}}%
\pgfpathlineto{\pgfqpoint{5.452908in}{4.547819in}}%
\pgfpathlineto{\pgfqpoint{5.453982in}{4.548794in}}%
\pgfpathlineto{\pgfqpoint{5.457203in}{4.548719in}}%
\pgfpathlineto{\pgfqpoint{5.458277in}{4.551721in}}%
\pgfpathlineto{\pgfqpoint{5.459351in}{4.552284in}}%
\pgfpathlineto{\pgfqpoint{5.460425in}{4.532960in}}%
\pgfpathlineto{\pgfqpoint{5.461499in}{4.524593in}}%
\pgfpathlineto{\pgfqpoint{5.465794in}{4.528082in}}%
\pgfpathlineto{\pgfqpoint{5.466868in}{4.530521in}}%
\pgfpathlineto{\pgfqpoint{5.467942in}{4.525718in}}%
\pgfpathlineto{\pgfqpoint{5.469016in}{4.530709in}}%
\pgfpathlineto{\pgfqpoint{5.472237in}{4.532360in}}%
\pgfpathlineto{\pgfqpoint{5.473311in}{4.534311in}}%
\pgfpathlineto{\pgfqpoint{5.475459in}{4.542641in}}%
\pgfpathlineto{\pgfqpoint{5.476533in}{4.536862in}}%
\pgfpathlineto{\pgfqpoint{5.479754in}{4.538401in}}%
\pgfpathlineto{\pgfqpoint{5.480828in}{4.537988in}}%
\pgfpathlineto{\pgfqpoint{5.481902in}{4.533485in}}%
\pgfpathlineto{\pgfqpoint{5.482976in}{4.535324in}}%
\pgfpathlineto{\pgfqpoint{5.484049in}{4.532397in}}%
\pgfpathlineto{\pgfqpoint{5.487271in}{4.533073in}}%
\pgfpathlineto{\pgfqpoint{5.488345in}{4.528195in}}%
\pgfpathlineto{\pgfqpoint{5.489419in}{4.529395in}}%
\pgfpathlineto{\pgfqpoint{5.490492in}{4.536750in}}%
\pgfpathlineto{\pgfqpoint{5.491566in}{4.533410in}}%
\pgfpathlineto{\pgfqpoint{5.494788in}{4.536525in}}%
\pgfpathlineto{\pgfqpoint{5.495862in}{4.534986in}}%
\pgfpathlineto{\pgfqpoint{5.496935in}{4.537800in}}%
\pgfpathlineto{\pgfqpoint{5.498009in}{4.536675in}}%
\pgfpathlineto{\pgfqpoint{5.499083in}{4.540727in}}%
\pgfpathlineto{\pgfqpoint{5.502305in}{4.539001in}}%
\pgfpathlineto{\pgfqpoint{5.504452in}{4.531684in}}%
\pgfpathlineto{\pgfqpoint{5.505526in}{4.525981in}}%
\pgfpathlineto{\pgfqpoint{5.506600in}{4.524217in}}%
\pgfpathlineto{\pgfqpoint{5.509822in}{4.524480in}}%
\pgfpathlineto{\pgfqpoint{5.510895in}{4.525643in}}%
\pgfpathlineto{\pgfqpoint{5.513043in}{4.531196in}}%
\pgfpathlineto{\pgfqpoint{5.517338in}{4.530934in}}%
\pgfpathlineto{\pgfqpoint{5.518412in}{4.526394in}}%
\pgfpathlineto{\pgfqpoint{5.521634in}{4.530784in}}%
\pgfpathlineto{\pgfqpoint{5.524855in}{4.529658in}}%
\pgfpathlineto{\pgfqpoint{5.527003in}{4.531347in}}%
\pgfpathlineto{\pgfqpoint{5.528077in}{4.535436in}}%
\pgfpathlineto{\pgfqpoint{5.529151in}{4.522829in}}%
\pgfpathlineto{\pgfqpoint{5.533446in}{4.522717in}}%
\pgfpathlineto{\pgfqpoint{5.534520in}{4.526769in}}%
\pgfpathlineto{\pgfqpoint{5.536667in}{4.525193in}}%
\pgfpathlineto{\pgfqpoint{5.542037in}{4.522266in}}%
\pgfpathlineto{\pgfqpoint{5.544184in}{4.524105in}}%
\pgfpathlineto{\pgfqpoint{5.547406in}{4.526206in}}%
\pgfpathlineto{\pgfqpoint{5.548480in}{4.524668in}}%
\pgfpathlineto{\pgfqpoint{5.549554in}{4.524705in}}%
\pgfpathlineto{\pgfqpoint{5.551701in}{4.528645in}}%
\pgfpathlineto{\pgfqpoint{5.554923in}{4.531196in}}%
\pgfpathlineto{\pgfqpoint{5.555997in}{4.529020in}}%
\pgfpathlineto{\pgfqpoint{5.558144in}{4.534874in}}%
\pgfpathlineto{\pgfqpoint{5.559218in}{4.532998in}}%
\pgfpathlineto{\pgfqpoint{5.562440in}{4.532847in}}%
\pgfpathlineto{\pgfqpoint{5.563513in}{4.537012in}}%
\pgfpathlineto{\pgfqpoint{5.565661in}{4.534911in}}%
\pgfpathlineto{\pgfqpoint{5.566735in}{4.536600in}}%
\pgfpathlineto{\pgfqpoint{5.572104in}{4.532622in}}%
\pgfpathlineto{\pgfqpoint{5.573178in}{4.532547in}}%
\pgfpathlineto{\pgfqpoint{5.574252in}{4.531272in}}%
\pgfpathlineto{\pgfqpoint{5.577473in}{4.530371in}}%
\pgfpathlineto{\pgfqpoint{5.579621in}{4.534348in}}%
\pgfpathlineto{\pgfqpoint{5.580695in}{4.529996in}}%
\pgfpathlineto{\pgfqpoint{5.581769in}{4.530108in}}%
\pgfpathlineto{\pgfqpoint{5.584990in}{4.528045in}}%
\pgfpathlineto{\pgfqpoint{5.586064in}{4.529320in}}%
\pgfpathlineto{\pgfqpoint{5.587138in}{4.532885in}}%
\pgfpathlineto{\pgfqpoint{5.588212in}{4.533298in}}%
\pgfpathlineto{\pgfqpoint{5.589286in}{4.530596in}}%
\pgfpathlineto{\pgfqpoint{5.592507in}{4.529621in}}%
\pgfpathlineto{\pgfqpoint{5.593581in}{4.529996in}}%
\pgfpathlineto{\pgfqpoint{5.595729in}{4.534949in}}%
\pgfpathlineto{\pgfqpoint{5.596802in}{4.532885in}}%
\pgfpathlineto{\pgfqpoint{5.601098in}{4.537012in}}%
\pgfpathlineto{\pgfqpoint{5.603245in}{4.531797in}}%
\pgfpathlineto{\pgfqpoint{5.604319in}{4.531797in}}%
\pgfpathlineto{\pgfqpoint{5.607541in}{4.524518in}}%
\pgfpathlineto{\pgfqpoint{5.608615in}{4.525268in}}%
\pgfpathlineto{\pgfqpoint{5.609688in}{4.527669in}}%
\pgfpathlineto{\pgfqpoint{5.615058in}{4.524818in}}%
\pgfpathlineto{\pgfqpoint{5.616132in}{4.524668in}}%
\pgfpathlineto{\pgfqpoint{5.617205in}{4.518964in}}%
\pgfpathlineto{\pgfqpoint{5.618279in}{4.520390in}}%
\pgfpathlineto{\pgfqpoint{5.619353in}{4.523805in}}%
\pgfpathlineto{\pgfqpoint{5.623648in}{4.529658in}}%
\pgfpathlineto{\pgfqpoint{5.624722in}{4.528157in}}%
\pgfpathlineto{\pgfqpoint{5.626870in}{4.530521in}}%
\pgfpathlineto{\pgfqpoint{5.630091in}{4.530821in}}%
\pgfpathlineto{\pgfqpoint{5.631165in}{4.529583in}}%
\pgfpathlineto{\pgfqpoint{5.632239in}{4.529733in}}%
\pgfpathlineto{\pgfqpoint{5.634387in}{4.518852in}}%
\pgfpathlineto{\pgfqpoint{5.637608in}{4.515400in}}%
\pgfpathlineto{\pgfqpoint{5.638682in}{4.516075in}}%
\pgfpathlineto{\pgfqpoint{5.640830in}{4.519640in}}%
\pgfpathlineto{\pgfqpoint{5.647273in}{4.517764in}}%
\pgfpathlineto{\pgfqpoint{5.648347in}{4.515700in}}%
\pgfpathlineto{\pgfqpoint{5.649421in}{4.532285in}}%
\pgfpathlineto{\pgfqpoint{5.652642in}{4.537725in}}%
\pgfpathlineto{\pgfqpoint{5.653716in}{4.537988in}}%
\pgfpathlineto{\pgfqpoint{5.655864in}{4.535962in}}%
\pgfpathlineto{\pgfqpoint{5.656937in}{4.536487in}}%
\pgfpathlineto{\pgfqpoint{5.660159in}{4.536787in}}%
\pgfpathlineto{\pgfqpoint{5.661233in}{4.537650in}}%
\pgfpathlineto{\pgfqpoint{5.662307in}{4.536637in}}%
\pgfpathlineto{\pgfqpoint{5.664454in}{4.522829in}}%
\pgfpathlineto{\pgfqpoint{5.668750in}{4.510484in}}%
\pgfpathlineto{\pgfqpoint{5.670897in}{4.522304in}}%
\pgfpathlineto{\pgfqpoint{5.671971in}{4.521478in}}%
\pgfpathlineto{\pgfqpoint{5.675193in}{4.521741in}}%
\pgfpathlineto{\pgfqpoint{5.676266in}{4.512173in}}%
\pgfpathlineto{\pgfqpoint{5.677340in}{4.515475in}}%
\pgfpathlineto{\pgfqpoint{5.678414in}{4.516600in}}%
\pgfpathlineto{\pgfqpoint{5.679488in}{4.512436in}}%
\pgfpathlineto{\pgfqpoint{5.683783in}{4.517426in}}%
\pgfpathlineto{\pgfqpoint{5.684857in}{4.516113in}}%
\pgfpathlineto{\pgfqpoint{5.687005in}{4.517463in}}%
\pgfpathlineto{\pgfqpoint{5.690226in}{4.516188in}}%
\pgfpathlineto{\pgfqpoint{5.692374in}{4.523842in}}%
\pgfpathlineto{\pgfqpoint{5.693448in}{4.522942in}}%
\pgfpathlineto{\pgfqpoint{5.694522in}{4.519039in}}%
\pgfpathlineto{\pgfqpoint{5.697743in}{4.521816in}}%
\pgfpathlineto{\pgfqpoint{5.698817in}{4.518176in}}%
\pgfpathlineto{\pgfqpoint{5.699891in}{4.517914in}}%
\pgfpathlineto{\pgfqpoint{5.700965in}{4.514612in}}%
\pgfpathlineto{\pgfqpoint{5.702039in}{4.516038in}}%
\pgfpathlineto{\pgfqpoint{5.705260in}{4.509847in}}%
\pgfpathlineto{\pgfqpoint{5.706334in}{4.509021in}}%
\pgfpathlineto{\pgfqpoint{5.707408in}{4.512623in}}%
\pgfpathlineto{\pgfqpoint{5.708482in}{4.511798in}}%
\pgfpathlineto{\pgfqpoint{5.709555in}{4.513599in}}%
\pgfpathlineto{\pgfqpoint{5.712777in}{4.523767in}}%
\pgfpathlineto{\pgfqpoint{5.713851in}{4.522416in}}%
\pgfpathlineto{\pgfqpoint{5.714925in}{4.524405in}}%
\pgfpathlineto{\pgfqpoint{5.720294in}{4.524855in}}%
\pgfpathlineto{\pgfqpoint{5.722442in}{4.520803in}}%
\pgfpathlineto{\pgfqpoint{5.724589in}{4.524480in}}%
\pgfpathlineto{\pgfqpoint{5.728885in}{4.523580in}}%
\pgfpathlineto{\pgfqpoint{5.729958in}{4.522041in}}%
\pgfpathlineto{\pgfqpoint{5.731032in}{4.507858in}}%
\pgfpathlineto{\pgfqpoint{5.732106in}{4.515250in}}%
\pgfpathlineto{\pgfqpoint{5.736401in}{4.513186in}}%
\pgfpathlineto{\pgfqpoint{5.737475in}{4.514762in}}%
\pgfpathlineto{\pgfqpoint{5.738549in}{4.513936in}}%
\pgfpathlineto{\pgfqpoint{5.739623in}{4.510559in}}%
\pgfpathlineto{\pgfqpoint{5.743918in}{4.513261in}}%
\pgfpathlineto{\pgfqpoint{5.744992in}{4.513411in}}%
\pgfpathlineto{\pgfqpoint{5.746066in}{4.512923in}}%
\pgfpathlineto{\pgfqpoint{5.747140in}{4.514199in}}%
\pgfpathlineto{\pgfqpoint{5.750361in}{4.511122in}}%
\pgfpathlineto{\pgfqpoint{5.751435in}{4.510935in}}%
\pgfpathlineto{\pgfqpoint{5.752509in}{4.509321in}}%
\pgfpathlineto{\pgfqpoint{5.754657in}{4.503280in}}%
\pgfpathlineto{\pgfqpoint{5.757878in}{4.504894in}}%
\pgfpathlineto{\pgfqpoint{5.758952in}{4.502943in}}%
\pgfpathlineto{\pgfqpoint{5.761100in}{4.508721in}}%
\pgfpathlineto{\pgfqpoint{5.762174in}{4.507558in}}%
\pgfpathlineto{\pgfqpoint{5.765395in}{4.506957in}}%
\pgfpathlineto{\pgfqpoint{5.766469in}{4.504781in}}%
\pgfpathlineto{\pgfqpoint{5.769690in}{4.505569in}}%
\pgfpathlineto{\pgfqpoint{5.772912in}{4.504819in}}%
\pgfpathlineto{\pgfqpoint{5.773986in}{4.506807in}}%
\pgfpathlineto{\pgfqpoint{5.776133in}{4.500504in}}%
\pgfpathlineto{\pgfqpoint{5.777207in}{4.502905in}}%
\pgfpathlineto{\pgfqpoint{5.780429in}{4.501179in}}%
\pgfpathlineto{\pgfqpoint{5.781503in}{4.498703in}}%
\pgfpathlineto{\pgfqpoint{5.782576in}{4.498515in}}%
\pgfpathlineto{\pgfqpoint{5.783650in}{4.499415in}}%
\pgfpathlineto{\pgfqpoint{5.784724in}{4.494988in}}%
\pgfpathlineto{\pgfqpoint{5.787946in}{4.494913in}}%
\pgfpathlineto{\pgfqpoint{5.789020in}{4.499566in}}%
\pgfpathlineto{\pgfqpoint{5.790093in}{4.501554in}}%
\pgfpathlineto{\pgfqpoint{5.792241in}{4.491498in}}%
\pgfpathlineto{\pgfqpoint{5.795463in}{4.493374in}}%
\pgfpathlineto{\pgfqpoint{5.796536in}{4.494950in}}%
\pgfpathlineto{\pgfqpoint{5.797610in}{4.498928in}}%
\pgfpathlineto{\pgfqpoint{5.798684in}{4.499603in}}%
\pgfpathlineto{\pgfqpoint{5.802979in}{4.498177in}}%
\pgfpathlineto{\pgfqpoint{5.804053in}{4.500954in}}%
\pgfpathlineto{\pgfqpoint{5.806201in}{4.497427in}}%
\pgfpathlineto{\pgfqpoint{5.810496in}{4.490523in}}%
\pgfpathlineto{\pgfqpoint{5.811570in}{4.486846in}}%
\pgfpathlineto{\pgfqpoint{5.812644in}{4.480317in}}%
\pgfpathlineto{\pgfqpoint{5.813718in}{4.478253in}}%
\pgfpathlineto{\pgfqpoint{5.814792in}{4.477503in}}%
\pgfpathlineto{\pgfqpoint{5.818013in}{4.479004in}}%
\pgfpathlineto{\pgfqpoint{5.819087in}{4.480242in}}%
\pgfpathlineto{\pgfqpoint{5.820161in}{4.474726in}}%
\pgfpathlineto{\pgfqpoint{5.821235in}{4.476302in}}%
\pgfpathlineto{\pgfqpoint{5.822309in}{4.474951in}}%
\pgfpathlineto{\pgfqpoint{5.826604in}{4.474013in}}%
\pgfpathlineto{\pgfqpoint{5.827678in}{4.475364in}}%
\pgfpathlineto{\pgfqpoint{5.828752in}{4.474013in}}%
\pgfpathlineto{\pgfqpoint{5.829825in}{4.447110in}}%
\pgfpathlineto{\pgfqpoint{5.834121in}{4.447223in}}%
\pgfpathlineto{\pgfqpoint{5.835195in}{4.445196in}}%
\pgfpathlineto{\pgfqpoint{5.836268in}{4.439380in}}%
\pgfpathlineto{\pgfqpoint{5.837342in}{4.441557in}}%
\pgfpathlineto{\pgfqpoint{5.840564in}{4.445834in}}%
\pgfpathlineto{\pgfqpoint{5.841638in}{4.442157in}}%
\pgfpathlineto{\pgfqpoint{5.843785in}{4.444709in}}%
\pgfpathlineto{\pgfqpoint{5.844859in}{4.443283in}}%
\pgfpathlineto{\pgfqpoint{5.848081in}{4.437654in}}%
\pgfpathlineto{\pgfqpoint{5.849154in}{4.438480in}}%
\pgfpathlineto{\pgfqpoint{5.850228in}{4.437279in}}%
\pgfpathlineto{\pgfqpoint{5.851302in}{4.433077in}}%
\pgfpathlineto{\pgfqpoint{5.852376in}{4.438592in}}%
\pgfpathlineto{\pgfqpoint{5.856671in}{4.440431in}}%
\pgfpathlineto{\pgfqpoint{5.863114in}{4.449136in}}%
\pgfpathlineto{\pgfqpoint{5.865262in}{4.445609in}}%
\pgfpathlineto{\pgfqpoint{5.866336in}{4.448273in}}%
\pgfpathlineto{\pgfqpoint{5.867410in}{4.448236in}}%
\pgfpathlineto{\pgfqpoint{5.870631in}{4.448949in}}%
\pgfpathlineto{\pgfqpoint{5.871705in}{4.453264in}}%
\pgfpathlineto{\pgfqpoint{5.872779in}{4.454427in}}%
\pgfpathlineto{\pgfqpoint{5.873853in}{4.457841in}}%
\pgfpathlineto{\pgfqpoint{5.878148in}{4.461068in}}%
\pgfpathlineto{\pgfqpoint{5.879222in}{4.462607in}}%
\pgfpathlineto{\pgfqpoint{5.881370in}{4.460205in}}%
\pgfpathlineto{\pgfqpoint{5.882443in}{4.462719in}}%
\pgfpathlineto{\pgfqpoint{5.885665in}{4.463094in}}%
\pgfpathlineto{\pgfqpoint{5.886739in}{4.461894in}}%
\pgfpathlineto{\pgfqpoint{5.888887in}{4.464933in}}%
\pgfpathlineto{\pgfqpoint{5.889960in}{4.468948in}}%
\pgfpathlineto{\pgfqpoint{5.893182in}{4.468910in}}%
\pgfpathlineto{\pgfqpoint{5.894256in}{4.466809in}}%
\pgfpathlineto{\pgfqpoint{5.895330in}{4.466884in}}%
\pgfpathlineto{\pgfqpoint{5.896403in}{4.466284in}}%
\pgfpathlineto{\pgfqpoint{5.900699in}{4.465608in}}%
\pgfpathlineto{\pgfqpoint{5.901773in}{4.466771in}}%
\pgfpathlineto{\pgfqpoint{5.902846in}{4.465646in}}%
\pgfpathlineto{\pgfqpoint{5.903920in}{4.469586in}}%
\pgfpathlineto{\pgfqpoint{5.904994in}{4.468535in}}%
\pgfpathlineto{\pgfqpoint{5.908216in}{4.466846in}}%
\pgfpathlineto{\pgfqpoint{5.909289in}{4.465308in}}%
\pgfpathlineto{\pgfqpoint{5.910363in}{4.465233in}}%
\pgfpathlineto{\pgfqpoint{5.911437in}{4.461593in}}%
\pgfpathlineto{\pgfqpoint{5.912511in}{4.463882in}}%
\pgfpathlineto{\pgfqpoint{5.915732in}{4.464933in}}%
\pgfpathlineto{\pgfqpoint{5.916806in}{4.468085in}}%
\pgfpathlineto{\pgfqpoint{5.917880in}{4.473375in}}%
\pgfpathlineto{\pgfqpoint{5.918954in}{4.474614in}}%
\pgfpathlineto{\pgfqpoint{5.920028in}{4.473300in}}%
\pgfpathlineto{\pgfqpoint{5.923249in}{4.474914in}}%
\pgfpathlineto{\pgfqpoint{5.926471in}{4.485645in}}%
\pgfpathlineto{\pgfqpoint{5.927545in}{4.486808in}}%
\pgfpathlineto{\pgfqpoint{5.930766in}{4.485983in}}%
\pgfpathlineto{\pgfqpoint{5.931840in}{4.487784in}}%
\pgfpathlineto{\pgfqpoint{5.932914in}{4.487859in}}%
\pgfpathlineto{\pgfqpoint{5.935062in}{4.485007in}}%
\pgfpathlineto{\pgfqpoint{5.938283in}{4.485908in}}%
\pgfpathlineto{\pgfqpoint{5.940431in}{4.480842in}}%
\pgfpathlineto{\pgfqpoint{5.941505in}{4.479641in}}%
\pgfpathlineto{\pgfqpoint{5.942578in}{4.481743in}}%
\pgfpathlineto{\pgfqpoint{5.945800in}{4.479867in}}%
\pgfpathlineto{\pgfqpoint{5.946874in}{4.482906in}}%
\pgfpathlineto{\pgfqpoint{5.950095in}{4.480354in}}%
\pgfpathlineto{\pgfqpoint{5.953317in}{4.480167in}}%
\pgfpathlineto{\pgfqpoint{5.954391in}{4.475514in}}%
\pgfpathlineto{\pgfqpoint{5.955465in}{4.478216in}}%
\pgfpathlineto{\pgfqpoint{5.956538in}{4.475439in}}%
\pgfpathlineto{\pgfqpoint{5.957612in}{4.479641in}}%
\pgfpathlineto{\pgfqpoint{5.960834in}{4.478478in}}%
\pgfpathlineto{\pgfqpoint{5.961908in}{4.483018in}}%
\pgfpathlineto{\pgfqpoint{5.962981in}{4.484594in}}%
\pgfpathlineto{\pgfqpoint{5.964055in}{4.484294in}}%
\pgfpathlineto{\pgfqpoint{5.965129in}{4.485345in}}%
\pgfpathlineto{\pgfqpoint{5.970498in}{4.486733in}}%
\pgfpathlineto{\pgfqpoint{5.971572in}{4.488497in}}%
\pgfpathlineto{\pgfqpoint{5.972646in}{4.485232in}}%
\pgfpathlineto{\pgfqpoint{5.975867in}{4.486846in}}%
\pgfpathlineto{\pgfqpoint{5.976941in}{4.486658in}}%
\pgfpathlineto{\pgfqpoint{5.978015in}{4.487971in}}%
\pgfpathlineto{\pgfqpoint{5.983384in}{4.478741in}}%
\pgfpathlineto{\pgfqpoint{5.984458in}{4.469473in}}%
\pgfpathlineto{\pgfqpoint{5.986606in}{4.472550in}}%
\pgfpathlineto{\pgfqpoint{5.987680in}{4.472287in}}%
\pgfpathlineto{\pgfqpoint{5.991975in}{4.473826in}}%
\pgfpathlineto{\pgfqpoint{5.993049in}{4.472625in}}%
\pgfpathlineto{\pgfqpoint{5.994123in}{4.477240in}}%
\pgfpathlineto{\pgfqpoint{5.995197in}{4.465871in}}%
\pgfpathlineto{\pgfqpoint{5.998418in}{4.457353in}}%
\pgfpathlineto{\pgfqpoint{5.999492in}{4.458179in}}%
\pgfpathlineto{\pgfqpoint{6.001640in}{4.469473in}}%
\pgfpathlineto{\pgfqpoint{6.002713in}{4.469210in}}%
\pgfpathlineto{\pgfqpoint{6.007009in}{4.463695in}}%
\pgfpathlineto{\pgfqpoint{6.009156in}{4.466059in}}%
\pgfpathlineto{\pgfqpoint{6.010230in}{4.472062in}}%
\pgfpathlineto{\pgfqpoint{6.013452in}{4.474463in}}%
\pgfpathlineto{\pgfqpoint{6.014526in}{4.477503in}}%
\pgfpathlineto{\pgfqpoint{6.015599in}{4.477840in}}%
\pgfpathlineto{\pgfqpoint{6.016673in}{4.479679in}}%
\pgfpathlineto{\pgfqpoint{6.017747in}{4.480279in}}%
\pgfpathlineto{\pgfqpoint{6.022042in}{4.481630in}}%
\pgfpathlineto{\pgfqpoint{6.023116in}{4.482793in}}%
\pgfpathlineto{\pgfqpoint{6.024190in}{4.479041in}}%
\pgfpathlineto{\pgfqpoint{6.025264in}{4.482080in}}%
\pgfpathlineto{\pgfqpoint{6.029559in}{4.482381in}}%
\pgfpathlineto{\pgfqpoint{6.031707in}{4.483919in}}%
\pgfpathlineto{\pgfqpoint{6.032781in}{4.482718in}}%
\pgfpathlineto{\pgfqpoint{6.036002in}{4.481555in}}%
\pgfpathlineto{\pgfqpoint{6.037076in}{4.479191in}}%
\pgfpathlineto{\pgfqpoint{6.038150in}{4.480467in}}%
\pgfpathlineto{\pgfqpoint{6.039224in}{4.480842in}}%
\pgfpathlineto{\pgfqpoint{6.040298in}{4.486508in}}%
\pgfpathlineto{\pgfqpoint{6.043519in}{4.487634in}}%
\pgfpathlineto{\pgfqpoint{6.045667in}{4.483731in}}%
\pgfpathlineto{\pgfqpoint{6.046741in}{4.486358in}}%
\pgfpathlineto{\pgfqpoint{6.047815in}{4.486020in}}%
\pgfpathlineto{\pgfqpoint{6.051036in}{4.486921in}}%
\pgfpathlineto{\pgfqpoint{6.052110in}{4.485720in}}%
\pgfpathlineto{\pgfqpoint{6.053184in}{4.487108in}}%
\pgfpathlineto{\pgfqpoint{6.058553in}{4.485945in}}%
\pgfpathlineto{\pgfqpoint{6.059627in}{4.487108in}}%
\pgfpathlineto{\pgfqpoint{6.060701in}{4.485232in}}%
\pgfpathlineto{\pgfqpoint{6.062848in}{4.483919in}}%
\pgfpathlineto{\pgfqpoint{6.066070in}{4.486508in}}%
\pgfpathlineto{\pgfqpoint{6.067144in}{4.486283in}}%
\pgfpathlineto{\pgfqpoint{6.068218in}{4.486733in}}%
\pgfpathlineto{\pgfqpoint{6.069291in}{4.484144in}}%
\pgfpathlineto{\pgfqpoint{6.070365in}{4.485345in}}%
\pgfpathlineto{\pgfqpoint{6.074661in}{4.487371in}}%
\pgfpathlineto{\pgfqpoint{6.075734in}{4.488909in}}%
\pgfpathlineto{\pgfqpoint{6.076808in}{4.489097in}}%
\pgfpathlineto{\pgfqpoint{6.077882in}{4.485082in}}%
\pgfpathlineto{\pgfqpoint{6.081104in}{4.487896in}}%
\pgfpathlineto{\pgfqpoint{6.083251in}{4.479191in}}%
\pgfpathlineto{\pgfqpoint{6.084325in}{4.480467in}}%
\pgfpathlineto{\pgfqpoint{6.085399in}{4.479867in}}%
\pgfpathlineto{\pgfqpoint{6.088620in}{4.481292in}}%
\pgfpathlineto{\pgfqpoint{6.089694in}{4.480054in}}%
\pgfpathlineto{\pgfqpoint{6.091842in}{4.483319in}}%
\pgfpathlineto{\pgfqpoint{6.092916in}{4.480542in}}%
\pgfpathlineto{\pgfqpoint{6.096137in}{4.479004in}}%
\pgfpathlineto{\pgfqpoint{6.097211in}{4.482080in}}%
\pgfpathlineto{\pgfqpoint{6.098285in}{4.481855in}}%
\pgfpathlineto{\pgfqpoint{6.099359in}{4.478816in}}%
\pgfpathlineto{\pgfqpoint{6.100433in}{4.481217in}}%
\pgfpathlineto{\pgfqpoint{6.103654in}{4.480392in}}%
\pgfpathlineto{\pgfqpoint{6.104728in}{4.480767in}}%
\pgfpathlineto{\pgfqpoint{6.105802in}{4.483506in}}%
\pgfpathlineto{\pgfqpoint{6.106876in}{4.474801in}}%
\pgfpathlineto{\pgfqpoint{6.107950in}{4.474163in}}%
\pgfpathlineto{\pgfqpoint{6.111171in}{4.474651in}}%
\pgfpathlineto{\pgfqpoint{6.112245in}{4.470936in}}%
\pgfpathlineto{\pgfqpoint{6.114393in}{4.469285in}}%
\pgfpathlineto{\pgfqpoint{6.115466in}{4.468347in}}%
\pgfpathlineto{\pgfqpoint{6.118688in}{4.467447in}}%
\pgfpathlineto{\pgfqpoint{6.119762in}{4.468122in}}%
\pgfpathlineto{\pgfqpoint{6.120836in}{4.472325in}}%
\pgfpathlineto{\pgfqpoint{6.121909in}{4.492249in}}%
\pgfpathlineto{\pgfqpoint{6.122983in}{4.494312in}}%
\pgfpathlineto{\pgfqpoint{6.126205in}{4.493337in}}%
\pgfpathlineto{\pgfqpoint{6.127279in}{4.492099in}}%
\pgfpathlineto{\pgfqpoint{6.129426in}{4.492774in}}%
\pgfpathlineto{\pgfqpoint{6.130500in}{4.491048in}}%
\pgfpathlineto{\pgfqpoint{6.133722in}{4.490936in}}%
\pgfpathlineto{\pgfqpoint{6.134796in}{4.490335in}}%
\pgfpathlineto{\pgfqpoint{6.135869in}{4.487446in}}%
\pgfpathlineto{\pgfqpoint{6.136943in}{4.487033in}}%
\pgfpathlineto{\pgfqpoint{6.138017in}{4.487671in}}%
\pgfpathlineto{\pgfqpoint{6.141239in}{4.493037in}}%
\pgfpathlineto{\pgfqpoint{6.142312in}{4.493262in}}%
\pgfpathlineto{\pgfqpoint{6.144460in}{4.504218in}}%
\pgfpathlineto{\pgfqpoint{6.145534in}{4.505644in}}%
\pgfpathlineto{\pgfqpoint{6.148755in}{4.512548in}}%
\pgfpathlineto{\pgfqpoint{6.149829in}{4.512736in}}%
\pgfpathlineto{\pgfqpoint{6.150903in}{4.509884in}}%
\pgfpathlineto{\pgfqpoint{6.151977in}{4.510222in}}%
\pgfpathlineto{\pgfqpoint{6.153051in}{4.507445in}}%
\pgfpathlineto{\pgfqpoint{6.157346in}{4.510034in}}%
\pgfpathlineto{\pgfqpoint{6.158420in}{4.514199in}}%
\pgfpathlineto{\pgfqpoint{6.160568in}{4.514124in}}%
\pgfpathlineto{\pgfqpoint{6.163789in}{4.511498in}}%
\pgfpathlineto{\pgfqpoint{6.164863in}{4.509209in}}%
\pgfpathlineto{\pgfqpoint{6.167011in}{4.512923in}}%
\pgfpathlineto{\pgfqpoint{6.168085in}{4.510522in}}%
\pgfpathlineto{\pgfqpoint{6.171306in}{4.511122in}}%
\pgfpathlineto{\pgfqpoint{6.172380in}{4.512135in}}%
\pgfpathlineto{\pgfqpoint{6.173454in}{4.519265in}}%
\pgfpathlineto{\pgfqpoint{6.174528in}{4.521516in}}%
\pgfpathlineto{\pgfqpoint{6.175601in}{4.520991in}}%
\pgfpathlineto{\pgfqpoint{6.178823in}{4.516713in}}%
\pgfpathlineto{\pgfqpoint{6.180971in}{4.518477in}}%
\pgfpathlineto{\pgfqpoint{6.182044in}{4.521628in}}%
\pgfpathlineto{\pgfqpoint{6.183118in}{4.521816in}}%
\pgfpathlineto{\pgfqpoint{6.186340in}{4.520203in}}%
\pgfpathlineto{\pgfqpoint{6.188487in}{4.522979in}}%
\pgfpathlineto{\pgfqpoint{6.189561in}{4.520315in}}%
\pgfpathlineto{\pgfqpoint{6.190635in}{4.521741in}}%
\pgfpathlineto{\pgfqpoint{6.194930in}{4.521779in}}%
\pgfpathlineto{\pgfqpoint{6.197078in}{4.517951in}}%
\pgfpathlineto{\pgfqpoint{6.198152in}{4.518514in}}%
\pgfpathlineto{\pgfqpoint{6.202447in}{4.523092in}}%
\pgfpathlineto{\pgfqpoint{6.203521in}{4.527557in}}%
\pgfpathlineto{\pgfqpoint{6.204595in}{4.524142in}}%
\pgfpathlineto{\pgfqpoint{6.208890in}{4.526094in}}%
\pgfpathlineto{\pgfqpoint{6.209964in}{4.528945in}}%
\pgfpathlineto{\pgfqpoint{6.211038in}{4.529883in}}%
\pgfpathlineto{\pgfqpoint{6.212112in}{4.529771in}}%
\pgfpathlineto{\pgfqpoint{6.213186in}{4.528833in}}%
\pgfpathlineto{\pgfqpoint{6.217481in}{4.528758in}}%
\pgfpathlineto{\pgfqpoint{6.218555in}{4.531984in}}%
\pgfpathlineto{\pgfqpoint{6.220703in}{4.527332in}}%
\pgfpathlineto{\pgfqpoint{6.223924in}{4.526506in}}%
\pgfpathlineto{\pgfqpoint{6.224998in}{4.531759in}}%
\pgfpathlineto{\pgfqpoint{6.226072in}{4.529808in}}%
\pgfpathlineto{\pgfqpoint{6.228219in}{4.529658in}}%
\pgfpathlineto{\pgfqpoint{6.231441in}{4.531234in}}%
\pgfpathlineto{\pgfqpoint{6.232515in}{4.527970in}}%
\pgfpathlineto{\pgfqpoint{6.233589in}{4.529358in}}%
\pgfpathlineto{\pgfqpoint{6.234663in}{4.528457in}}%
\pgfpathlineto{\pgfqpoint{6.235736in}{4.533973in}}%
\pgfpathlineto{\pgfqpoint{6.241106in}{4.533110in}}%
\pgfpathlineto{\pgfqpoint{6.243253in}{4.535587in}}%
\pgfpathlineto{\pgfqpoint{6.246475in}{4.537125in}}%
\pgfpathlineto{\pgfqpoint{6.247549in}{4.538926in}}%
\pgfpathlineto{\pgfqpoint{6.248622in}{4.539639in}}%
\pgfpathlineto{\pgfqpoint{6.249696in}{4.539301in}}%
\pgfpathlineto{\pgfqpoint{6.250770in}{4.540014in}}%
\pgfpathlineto{\pgfqpoint{6.255065in}{4.540952in}}%
\pgfpathlineto{\pgfqpoint{6.256139in}{4.540577in}}%
\pgfpathlineto{\pgfqpoint{6.257213in}{4.541252in}}%
\pgfpathlineto{\pgfqpoint{6.258287in}{4.540202in}}%
\pgfpathlineto{\pgfqpoint{6.261508in}{4.541665in}}%
\pgfpathlineto{\pgfqpoint{6.262582in}{4.541290in}}%
\pgfpathlineto{\pgfqpoint{6.263656in}{4.548006in}}%
\pgfpathlineto{\pgfqpoint{6.264730in}{4.541440in}}%
\pgfpathlineto{\pgfqpoint{6.265804in}{4.540652in}}%
\pgfpathlineto{\pgfqpoint{6.269025in}{4.539264in}}%
\pgfpathlineto{\pgfqpoint{6.270099in}{4.539564in}}%
\pgfpathlineto{\pgfqpoint{6.271173in}{4.537613in}}%
\pgfpathlineto{\pgfqpoint{6.273321in}{4.538813in}}%
\pgfpathlineto{\pgfqpoint{6.276542in}{4.538213in}}%
\pgfpathlineto{\pgfqpoint{6.277616in}{4.539939in}}%
\pgfpathlineto{\pgfqpoint{6.278690in}{4.538288in}}%
\pgfpathlineto{\pgfqpoint{6.279764in}{4.540239in}}%
\pgfpathlineto{\pgfqpoint{6.280838in}{4.538363in}}%
\pgfpathlineto{\pgfqpoint{6.284059in}{4.536900in}}%
\pgfpathlineto{\pgfqpoint{6.285133in}{4.532135in}}%
\pgfpathlineto{\pgfqpoint{6.287281in}{4.533260in}}%
\pgfpathlineto{\pgfqpoint{6.288354in}{4.534573in}}%
\pgfpathlineto{\pgfqpoint{6.291576in}{4.532397in}}%
\pgfpathlineto{\pgfqpoint{6.292650in}{4.536149in}}%
\pgfpathlineto{\pgfqpoint{6.293724in}{4.534724in}}%
\pgfpathlineto{\pgfqpoint{6.294797in}{4.538213in}}%
\pgfpathlineto{\pgfqpoint{6.295871in}{4.537838in}}%
\pgfpathlineto{\pgfqpoint{6.302314in}{4.534686in}}%
\pgfpathlineto{\pgfqpoint{6.307684in}{4.532322in}}%
\pgfpathlineto{\pgfqpoint{6.309831in}{4.526994in}}%
\pgfpathlineto{\pgfqpoint{6.314127in}{4.530146in}}%
\pgfpathlineto{\pgfqpoint{6.315200in}{4.526957in}}%
\pgfpathlineto{\pgfqpoint{6.316274in}{4.526094in}}%
\pgfpathlineto{\pgfqpoint{6.317348in}{4.542340in}}%
\pgfpathlineto{\pgfqpoint{6.318422in}{4.540765in}}%
\pgfpathlineto{\pgfqpoint{6.322717in}{4.544554in}}%
\pgfpathlineto{\pgfqpoint{6.324865in}{4.543466in}}%
\pgfpathlineto{\pgfqpoint{6.325939in}{4.539526in}}%
\pgfpathlineto{\pgfqpoint{6.329160in}{4.539451in}}%
\pgfpathlineto{\pgfqpoint{6.330234in}{4.540577in}}%
\pgfpathlineto{\pgfqpoint{6.332382in}{4.536187in}}%
\pgfpathlineto{\pgfqpoint{6.336677in}{4.535587in}}%
\pgfpathlineto{\pgfqpoint{6.338825in}{4.537350in}}%
\pgfpathlineto{\pgfqpoint{6.340973in}{4.533148in}}%
\pgfpathlineto{\pgfqpoint{6.344194in}{4.536187in}}%
\pgfpathlineto{\pgfqpoint{6.345268in}{4.535474in}}%
\pgfpathlineto{\pgfqpoint{6.346342in}{4.529058in}}%
\pgfpathlineto{\pgfqpoint{6.347416in}{4.529095in}}%
\pgfpathlineto{\pgfqpoint{6.348489in}{4.530634in}}%
\pgfpathlineto{\pgfqpoint{6.351711in}{4.531272in}}%
\pgfpathlineto{\pgfqpoint{6.352785in}{4.532097in}}%
\pgfpathlineto{\pgfqpoint{6.353859in}{4.531797in}}%
\pgfpathlineto{\pgfqpoint{6.354932in}{4.532960in}}%
\pgfpathlineto{\pgfqpoint{6.356006in}{4.533035in}}%
\pgfpathlineto{\pgfqpoint{6.361375in}{4.531159in}}%
\pgfpathlineto{\pgfqpoint{6.362449in}{4.535999in}}%
\pgfpathlineto{\pgfqpoint{6.366745in}{4.538513in}}%
\pgfpathlineto{\pgfqpoint{6.367818in}{4.538063in}}%
\pgfpathlineto{\pgfqpoint{6.368892in}{4.541553in}}%
\pgfpathlineto{\pgfqpoint{6.369966in}{4.542078in}}%
\pgfpathlineto{\pgfqpoint{6.371040in}{4.543391in}}%
\pgfpathlineto{\pgfqpoint{6.374262in}{4.542866in}}%
\pgfpathlineto{\pgfqpoint{6.376409in}{4.545305in}}%
\pgfpathlineto{\pgfqpoint{6.377483in}{4.544779in}}%
\pgfpathlineto{\pgfqpoint{6.378557in}{4.547519in}}%
\pgfpathlineto{\pgfqpoint{6.381778in}{4.549094in}}%
\pgfpathlineto{\pgfqpoint{6.382852in}{4.551383in}}%
\pgfpathlineto{\pgfqpoint{6.383926in}{4.550295in}}%
\pgfpathlineto{\pgfqpoint{6.386074in}{4.550333in}}%
\pgfpathlineto{\pgfqpoint{6.390369in}{4.553447in}}%
\pgfpathlineto{\pgfqpoint{6.391443in}{4.556674in}}%
\pgfpathlineto{\pgfqpoint{6.392517in}{4.555436in}}%
\pgfpathlineto{\pgfqpoint{6.393591in}{4.557649in}}%
\pgfpathlineto{\pgfqpoint{6.396812in}{4.560764in}}%
\pgfpathlineto{\pgfqpoint{6.398960in}{4.561176in}}%
\pgfpathlineto{\pgfqpoint{6.400034in}{4.556974in}}%
\pgfpathlineto{\pgfqpoint{6.401107in}{4.559150in}}%
\pgfpathlineto{\pgfqpoint{6.405403in}{4.558512in}}%
\pgfpathlineto{\pgfqpoint{6.407551in}{4.562940in}}%
\pgfpathlineto{\pgfqpoint{6.408624in}{4.562602in}}%
\pgfpathlineto{\pgfqpoint{6.411846in}{4.562377in}}%
\pgfpathlineto{\pgfqpoint{6.413994in}{4.564966in}}%
\pgfpathlineto{\pgfqpoint{6.415067in}{4.562865in}}%
\pgfpathlineto{\pgfqpoint{6.416141in}{4.563728in}}%
\pgfpathlineto{\pgfqpoint{6.419363in}{4.561589in}}%
\pgfpathlineto{\pgfqpoint{6.420437in}{4.563053in}}%
\pgfpathlineto{\pgfqpoint{6.421510in}{4.562677in}}%
\pgfpathlineto{\pgfqpoint{6.422584in}{4.557387in}}%
\pgfpathlineto{\pgfqpoint{6.423658in}{4.560839in}}%
\pgfpathlineto{\pgfqpoint{6.427953in}{4.562452in}}%
\pgfpathlineto{\pgfqpoint{6.429027in}{4.562677in}}%
\pgfpathlineto{\pgfqpoint{6.430101in}{4.563540in}}%
\pgfpathlineto{\pgfqpoint{6.431175in}{4.565116in}}%
\pgfpathlineto{\pgfqpoint{6.436544in}{4.564103in}}%
\pgfpathlineto{\pgfqpoint{6.437618in}{4.559976in}}%
\pgfpathlineto{\pgfqpoint{6.438692in}{4.559000in}}%
\pgfpathlineto{\pgfqpoint{6.441913in}{4.563278in}}%
\pgfpathlineto{\pgfqpoint{6.442987in}{4.568118in}}%
\pgfpathlineto{\pgfqpoint{6.444061in}{4.570294in}}%
\pgfpathlineto{\pgfqpoint{6.446209in}{4.562978in}}%
\pgfpathlineto{\pgfqpoint{6.452652in}{4.562902in}}%
\pgfpathlineto{\pgfqpoint{6.453726in}{4.563278in}}%
\pgfpathlineto{\pgfqpoint{6.458021in}{4.563090in}}%
\pgfpathlineto{\pgfqpoint{6.461242in}{4.565717in}}%
\pgfpathlineto{\pgfqpoint{6.466612in}{4.562415in}}%
\pgfpathlineto{\pgfqpoint{6.467685in}{4.559413in}}%
\pgfpathlineto{\pgfqpoint{6.468759in}{4.558850in}}%
\pgfpathlineto{\pgfqpoint{6.471981in}{4.564103in}}%
\pgfpathlineto{\pgfqpoint{6.473055in}{4.567217in}}%
\pgfpathlineto{\pgfqpoint{6.474129in}{4.567555in}}%
\pgfpathlineto{\pgfqpoint{6.475202in}{4.565904in}}%
\pgfpathlineto{\pgfqpoint{6.476276in}{4.568831in}}%
\pgfpathlineto{\pgfqpoint{6.479498in}{4.572020in}}%
\pgfpathlineto{\pgfqpoint{6.480572in}{4.576148in}}%
\pgfpathlineto{\pgfqpoint{6.481645in}{4.574084in}}%
\pgfpathlineto{\pgfqpoint{6.483793in}{4.573971in}}%
\pgfpathlineto{\pgfqpoint{6.487015in}{4.573259in}}%
\pgfpathlineto{\pgfqpoint{6.488088in}{4.575022in}}%
\pgfpathlineto{\pgfqpoint{6.490236in}{4.580313in}}%
\pgfpathlineto{\pgfqpoint{6.491310in}{4.581476in}}%
\pgfpathlineto{\pgfqpoint{6.494531in}{4.581776in}}%
\pgfpathlineto{\pgfqpoint{6.495605in}{4.585003in}}%
\pgfpathlineto{\pgfqpoint{6.496679in}{4.583464in}}%
\pgfpathlineto{\pgfqpoint{6.498827in}{4.586729in}}%
\pgfpathlineto{\pgfqpoint{6.503122in}{4.587967in}}%
\pgfpathlineto{\pgfqpoint{6.504196in}{4.588230in}}%
\pgfpathlineto{\pgfqpoint{6.505270in}{4.586954in}}%
\pgfpathlineto{\pgfqpoint{6.506344in}{4.591532in}}%
\pgfpathlineto{\pgfqpoint{6.510639in}{4.587254in}}%
\pgfpathlineto{\pgfqpoint{6.511713in}{4.588680in}}%
\pgfpathlineto{\pgfqpoint{6.512787in}{4.588005in}}%
\pgfpathlineto{\pgfqpoint{6.513861in}{4.588718in}}%
\pgfpathlineto{\pgfqpoint{6.517082in}{4.589768in}}%
\pgfpathlineto{\pgfqpoint{6.518156in}{4.595171in}}%
\pgfpathlineto{\pgfqpoint{6.519230in}{4.593971in}}%
\pgfpathlineto{\pgfqpoint{6.520304in}{4.601888in}}%
\pgfpathlineto{\pgfqpoint{6.521377in}{4.602263in}}%
\pgfpathlineto{\pgfqpoint{6.524599in}{4.599599in}}%
\pgfpathlineto{\pgfqpoint{6.526747in}{4.602263in}}%
\pgfpathlineto{\pgfqpoint{6.527820in}{4.602938in}}%
\pgfpathlineto{\pgfqpoint{6.528894in}{4.604589in}}%
\pgfpathlineto{\pgfqpoint{6.532116in}{4.604064in}}%
\pgfpathlineto{\pgfqpoint{6.533190in}{4.600725in}}%
\pgfpathlineto{\pgfqpoint{6.534263in}{4.599824in}}%
\pgfpathlineto{\pgfqpoint{6.535337in}{4.594796in}}%
\pgfpathlineto{\pgfqpoint{6.536411in}{4.593933in}}%
\pgfpathlineto{\pgfqpoint{6.539633in}{4.595321in}}%
\pgfpathlineto{\pgfqpoint{6.540706in}{4.594834in}}%
\pgfpathlineto{\pgfqpoint{6.541780in}{4.592995in}}%
\pgfpathlineto{\pgfqpoint{6.543928in}{4.594571in}}%
\pgfpathlineto{\pgfqpoint{6.547150in}{4.595509in}}%
\pgfpathlineto{\pgfqpoint{6.548223in}{4.597197in}}%
\pgfpathlineto{\pgfqpoint{6.549297in}{4.595059in}}%
\pgfpathlineto{\pgfqpoint{6.551445in}{4.593783in}}%
\pgfpathlineto{\pgfqpoint{6.554666in}{4.593745in}}%
\pgfpathlineto{\pgfqpoint{6.556814in}{4.605227in}}%
\pgfpathlineto{\pgfqpoint{6.557888in}{4.609280in}}%
\pgfpathlineto{\pgfqpoint{6.558962in}{4.609842in}}%
\pgfpathlineto{\pgfqpoint{6.563257in}{4.612919in}}%
\pgfpathlineto{\pgfqpoint{6.564331in}{4.611118in}}%
\pgfpathlineto{\pgfqpoint{6.565405in}{4.612469in}}%
\pgfpathlineto{\pgfqpoint{6.566479in}{4.612356in}}%
\pgfpathlineto{\pgfqpoint{6.569700in}{4.614007in}}%
\pgfpathlineto{\pgfqpoint{6.570774in}{4.615358in}}%
\pgfpathlineto{\pgfqpoint{6.571848in}{4.609542in}}%
\pgfpathlineto{\pgfqpoint{6.572922in}{4.607216in}}%
\pgfpathlineto{\pgfqpoint{6.573995in}{4.612244in}}%
\pgfpathlineto{\pgfqpoint{6.577217in}{4.616484in}}%
\pgfpathlineto{\pgfqpoint{6.579365in}{4.612206in}}%
\pgfpathlineto{\pgfqpoint{6.580439in}{4.612169in}}%
\pgfpathlineto{\pgfqpoint{6.581512in}{4.613032in}}%
\pgfpathlineto{\pgfqpoint{6.585808in}{4.612431in}}%
\pgfpathlineto{\pgfqpoint{6.587955in}{4.616559in}}%
\pgfpathlineto{\pgfqpoint{6.589029in}{4.615133in}}%
\pgfpathlineto{\pgfqpoint{6.589029in}{4.615133in}}%
\pgfusepath{stroke}%
\end{pgfscope}%
\begin{pgfscope}%
\pgfpathrectangle{\pgfqpoint{4.123120in}{4.233896in}}{\pgfqpoint{2.583333in}{0.400885in}}%
\pgfusepath{clip}%
\pgfsetroundcap%
\pgfsetroundjoin%
\pgfsetlinewidth{1.505625pt}%
\definecolor{currentstroke}{rgb}{1.000000,0.498039,0.054902}%
\pgfsetstrokecolor{currentstroke}%
\pgfsetdash{}{0pt}%
\pgfpathmoveto{\pgfqpoint{4.240544in}{4.414166in}}%
\pgfpathlineto{\pgfqpoint{4.241618in}{4.414053in}}%
\pgfpathlineto{\pgfqpoint{4.243766in}{4.410422in}}%
\pgfpathlineto{\pgfqpoint{4.246987in}{4.410825in}}%
\pgfpathlineto{\pgfqpoint{4.249135in}{4.412656in}}%
\pgfpathlineto{\pgfqpoint{4.250209in}{4.410349in}}%
\pgfpathlineto{\pgfqpoint{4.255578in}{4.412164in}}%
\pgfpathlineto{\pgfqpoint{4.257726in}{4.414512in}}%
\pgfpathlineto{\pgfqpoint{4.258800in}{4.411594in}}%
\pgfpathlineto{\pgfqpoint{4.263095in}{4.408997in}}%
\pgfpathlineto{\pgfqpoint{4.264169in}{4.405973in}}%
\pgfpathlineto{\pgfqpoint{4.269538in}{4.402754in}}%
\pgfpathlineto{\pgfqpoint{4.270612in}{4.399639in}}%
\pgfpathlineto{\pgfqpoint{4.272760in}{4.402661in}}%
\pgfpathlineto{\pgfqpoint{4.273833in}{4.399442in}}%
\pgfpathlineto{\pgfqpoint{4.277055in}{4.398183in}}%
\pgfpathlineto{\pgfqpoint{4.278129in}{4.399096in}}%
\pgfpathlineto{\pgfqpoint{4.279203in}{4.397710in}}%
\pgfpathlineto{\pgfqpoint{4.280276in}{4.399599in}}%
\pgfpathlineto{\pgfqpoint{4.281350in}{4.398183in}}%
\pgfpathlineto{\pgfqpoint{4.285646in}{4.398623in}}%
\pgfpathlineto{\pgfqpoint{4.286719in}{4.397364in}}%
\pgfpathlineto{\pgfqpoint{4.287793in}{4.393555in}}%
\pgfpathlineto{\pgfqpoint{4.295310in}{4.393017in}}%
\pgfpathlineto{\pgfqpoint{4.296384in}{4.394810in}}%
\pgfpathlineto{\pgfqpoint{4.300679in}{4.391440in}}%
\pgfpathlineto{\pgfqpoint{4.301753in}{4.389182in}}%
\pgfpathlineto{\pgfqpoint{4.302827in}{4.390948in}}%
\pgfpathlineto{\pgfqpoint{4.303901in}{4.389443in}}%
\pgfpathlineto{\pgfqpoint{4.307122in}{4.389385in}}%
\pgfpathlineto{\pgfqpoint{4.308196in}{4.386201in}}%
\pgfpathlineto{\pgfqpoint{4.311418in}{4.389993in}}%
\pgfpathlineto{\pgfqpoint{4.314639in}{4.388864in}}%
\pgfpathlineto{\pgfqpoint{4.316787in}{4.380336in}}%
\pgfpathlineto{\pgfqpoint{4.317861in}{4.381637in}}%
\pgfpathlineto{\pgfqpoint{4.318935in}{4.381254in}}%
\pgfpathlineto{\pgfqpoint{4.322156in}{4.382887in}}%
\pgfpathlineto{\pgfqpoint{4.323230in}{4.382070in}}%
\pgfpathlineto{\pgfqpoint{4.326451in}{4.382836in}}%
\pgfpathlineto{\pgfqpoint{4.330747in}{4.378650in}}%
\pgfpathlineto{\pgfqpoint{4.331821in}{4.380470in}}%
\pgfpathlineto{\pgfqpoint{4.332894in}{4.377922in}}%
\pgfpathlineto{\pgfqpoint{4.333968in}{4.377849in}}%
\pgfpathlineto{\pgfqpoint{4.337190in}{4.378189in}}%
\pgfpathlineto{\pgfqpoint{4.338264in}{4.379451in}}%
\pgfpathlineto{\pgfqpoint{4.339338in}{4.377218in}}%
\pgfpathlineto{\pgfqpoint{4.340411in}{4.378820in}}%
\pgfpathlineto{\pgfqpoint{4.344707in}{4.376757in}}%
\pgfpathlineto{\pgfqpoint{4.345781in}{4.374620in}}%
\pgfpathlineto{\pgfqpoint{4.346854in}{4.376295in}}%
\pgfpathlineto{\pgfqpoint{4.349002in}{4.372328in}}%
\pgfpathlineto{\pgfqpoint{4.354371in}{4.373917in}}%
\pgfpathlineto{\pgfqpoint{4.356519in}{4.372678in}}%
\pgfpathlineto{\pgfqpoint{4.359740in}{4.372421in}}%
\pgfpathlineto{\pgfqpoint{4.360814in}{4.373052in}}%
\pgfpathlineto{\pgfqpoint{4.361888in}{4.370342in}}%
\pgfpathlineto{\pgfqpoint{4.364036in}{4.372867in}}%
\pgfpathlineto{\pgfqpoint{4.367257in}{4.372956in}}%
\pgfpathlineto{\pgfqpoint{4.368331in}{4.374475in}}%
\pgfpathlineto{\pgfqpoint{4.369405in}{4.374654in}}%
\pgfpathlineto{\pgfqpoint{4.370479in}{4.374252in}}%
\pgfpathlineto{\pgfqpoint{4.371553in}{4.372733in}}%
\pgfpathlineto{\pgfqpoint{4.375848in}{4.372308in}}%
\pgfpathlineto{\pgfqpoint{4.376922in}{4.371437in}}%
\pgfpathlineto{\pgfqpoint{4.379070in}{4.371816in}}%
\pgfpathlineto{\pgfqpoint{4.384439in}{4.367281in}}%
\pgfpathlineto{\pgfqpoint{4.385513in}{4.363773in}}%
\pgfpathlineto{\pgfqpoint{4.391956in}{4.360736in}}%
\pgfpathlineto{\pgfqpoint{4.393029in}{4.361449in}}%
\pgfpathlineto{\pgfqpoint{4.394103in}{4.360412in}}%
\pgfpathlineto{\pgfqpoint{4.398399in}{4.361881in}}%
\pgfpathlineto{\pgfqpoint{4.399472in}{4.359742in}}%
\pgfpathlineto{\pgfqpoint{4.400546in}{4.360455in}}%
\pgfpathlineto{\pgfqpoint{4.401620in}{4.355787in}}%
\pgfpathlineto{\pgfqpoint{4.404842in}{4.355938in}}%
\pgfpathlineto{\pgfqpoint{4.405916in}{4.356867in}}%
\pgfpathlineto{\pgfqpoint{4.406989in}{4.354144in}}%
\pgfpathlineto{\pgfqpoint{4.408063in}{4.353877in}}%
\pgfpathlineto{\pgfqpoint{4.409137in}{4.355027in}}%
\pgfpathlineto{\pgfqpoint{4.412359in}{4.353631in}}%
\pgfpathlineto{\pgfqpoint{4.414506in}{4.348666in}}%
\pgfpathlineto{\pgfqpoint{4.415580in}{4.348569in}}%
\pgfpathlineto{\pgfqpoint{4.416654in}{4.346402in}}%
\pgfpathlineto{\pgfqpoint{4.419875in}{4.345674in}}%
\pgfpathlineto{\pgfqpoint{4.420949in}{4.343845in}}%
\pgfpathlineto{\pgfqpoint{4.422023in}{4.344653in}}%
\pgfpathlineto{\pgfqpoint{4.423097in}{4.342840in}}%
\pgfpathlineto{\pgfqpoint{4.424171in}{4.343612in}}%
\pgfpathlineto{\pgfqpoint{4.428466in}{4.342517in}}%
\pgfpathlineto{\pgfqpoint{4.429540in}{4.343755in}}%
\pgfpathlineto{\pgfqpoint{4.430614in}{4.343450in}}%
\pgfpathlineto{\pgfqpoint{4.431688in}{4.340991in}}%
\pgfpathlineto{\pgfqpoint{4.435983in}{4.343134in}}%
\pgfpathlineto{\pgfqpoint{4.438131in}{4.342896in}}%
\pgfpathlineto{\pgfqpoint{4.439205in}{4.341926in}}%
\pgfpathlineto{\pgfqpoint{4.444574in}{4.341093in}}%
\pgfpathlineto{\pgfqpoint{4.446721in}{4.337792in}}%
\pgfpathlineto{\pgfqpoint{4.451017in}{4.338909in}}%
\pgfpathlineto{\pgfqpoint{4.452091in}{4.338334in}}%
\pgfpathlineto{\pgfqpoint{4.453164in}{4.335279in}}%
\pgfpathlineto{\pgfqpoint{4.454238in}{4.334654in}}%
\pgfpathlineto{\pgfqpoint{4.458534in}{4.334375in}}%
\pgfpathlineto{\pgfqpoint{4.459607in}{4.334999in}}%
\pgfpathlineto{\pgfqpoint{4.460681in}{4.332453in}}%
\pgfpathlineto{\pgfqpoint{4.461755in}{4.333519in}}%
\pgfpathlineto{\pgfqpoint{4.464977in}{4.333195in}}%
\pgfpathlineto{\pgfqpoint{4.468198in}{4.330676in}}%
\pgfpathlineto{\pgfqpoint{4.469272in}{4.329115in}}%
\pgfpathlineto{\pgfqpoint{4.472494in}{4.328432in}}%
\pgfpathlineto{\pgfqpoint{4.474641in}{4.329516in}}%
\pgfpathlineto{\pgfqpoint{4.476789in}{4.326753in}}%
\pgfpathlineto{\pgfqpoint{4.482158in}{4.327837in}}%
\pgfpathlineto{\pgfqpoint{4.484306in}{4.329085in}}%
\pgfpathlineto{\pgfqpoint{4.490749in}{4.327525in}}%
\pgfpathlineto{\pgfqpoint{4.491823in}{4.326100in}}%
\pgfpathlineto{\pgfqpoint{4.498266in}{4.325699in}}%
\pgfpathlineto{\pgfqpoint{4.499339in}{4.324240in}}%
\pgfpathlineto{\pgfqpoint{4.503635in}{4.324625in}}%
\pgfpathlineto{\pgfqpoint{4.504709in}{4.322851in}}%
\pgfpathlineto{\pgfqpoint{4.506856in}{4.323525in}}%
\pgfpathlineto{\pgfqpoint{4.512226in}{4.322961in}}%
\pgfpathlineto{\pgfqpoint{4.513299in}{4.320761in}}%
\pgfpathlineto{\pgfqpoint{4.514373in}{4.321019in}}%
\pgfpathlineto{\pgfqpoint{4.517595in}{4.320916in}}%
\pgfpathlineto{\pgfqpoint{4.519742in}{4.320321in}}%
\pgfpathlineto{\pgfqpoint{4.521890in}{4.319378in}}%
\pgfpathlineto{\pgfqpoint{4.525112in}{4.319158in}}%
\pgfpathlineto{\pgfqpoint{4.528333in}{4.317865in}}%
\pgfpathlineto{\pgfqpoint{4.529407in}{4.318201in}}%
\pgfpathlineto{\pgfqpoint{4.536924in}{4.316170in}}%
\pgfpathlineto{\pgfqpoint{4.540145in}{4.316449in}}%
\pgfpathlineto{\pgfqpoint{4.542293in}{4.315515in}}%
\pgfpathlineto{\pgfqpoint{4.543367in}{4.316073in}}%
\pgfpathlineto{\pgfqpoint{4.544441in}{4.315430in}}%
\pgfpathlineto{\pgfqpoint{4.547662in}{4.315103in}}%
\pgfpathlineto{\pgfqpoint{4.548736in}{4.313963in}}%
\pgfpathlineto{\pgfqpoint{4.549810in}{4.314747in}}%
\pgfpathlineto{\pgfqpoint{4.551958in}{4.312092in}}%
\pgfpathlineto{\pgfqpoint{4.555179in}{4.311964in}}%
\pgfpathlineto{\pgfqpoint{4.556253in}{4.310514in}}%
\pgfpathlineto{\pgfqpoint{4.558401in}{4.310584in}}%
\pgfpathlineto{\pgfqpoint{4.559474in}{4.310911in}}%
\pgfpathlineto{\pgfqpoint{4.564844in}{4.311145in}}%
\pgfpathlineto{\pgfqpoint{4.565917in}{4.310209in}}%
\pgfpathlineto{\pgfqpoint{4.570213in}{4.309700in}}%
\pgfpathlineto{\pgfqpoint{4.571287in}{4.308850in}}%
\pgfpathlineto{\pgfqpoint{4.572360in}{4.307243in}}%
\pgfpathlineto{\pgfqpoint{4.577730in}{4.307188in}}%
\pgfpathlineto{\pgfqpoint{4.580951in}{4.305306in}}%
\pgfpathlineto{\pgfqpoint{4.582025in}{4.305967in}}%
\pgfpathlineto{\pgfqpoint{4.585247in}{4.305031in}}%
\pgfpathlineto{\pgfqpoint{4.587394in}{4.305754in}}%
\pgfpathlineto{\pgfqpoint{4.589542in}{4.306264in}}%
\pgfpathlineto{\pgfqpoint{4.592763in}{4.305467in}}%
\pgfpathlineto{\pgfqpoint{4.594911in}{4.303235in}}%
\pgfpathlineto{\pgfqpoint{4.597059in}{4.303592in}}%
\pgfpathlineto{\pgfqpoint{4.607797in}{4.304390in}}%
\pgfpathlineto{\pgfqpoint{4.611019in}{4.305289in}}%
\pgfpathlineto{\pgfqpoint{4.612093in}{4.304287in}}%
\pgfpathlineto{\pgfqpoint{4.616388in}{4.305381in}}%
\pgfpathlineto{\pgfqpoint{4.617462in}{4.304420in}}%
\pgfpathlineto{\pgfqpoint{4.619609in}{4.305218in}}%
\pgfpathlineto{\pgfqpoint{4.624979in}{4.304737in}}%
\pgfpathlineto{\pgfqpoint{4.627126in}{4.304308in}}%
\pgfpathlineto{\pgfqpoint{4.630348in}{4.303551in}}%
\pgfpathlineto{\pgfqpoint{4.632495in}{4.302221in}}%
\pgfpathlineto{\pgfqpoint{4.645382in}{4.304324in}}%
\pgfpathlineto{\pgfqpoint{4.647529in}{4.303824in}}%
\pgfpathlineto{\pgfqpoint{4.648603in}{4.303928in}}%
\pgfpathlineto{\pgfqpoint{4.649677in}{4.303098in}}%
\pgfpathlineto{\pgfqpoint{4.660415in}{4.302494in}}%
\pgfpathlineto{\pgfqpoint{4.662563in}{4.302711in}}%
\pgfpathlineto{\pgfqpoint{4.670080in}{4.299810in}}%
\pgfpathlineto{\pgfqpoint{4.671154in}{4.298555in}}%
\pgfpathlineto{\pgfqpoint{4.672227in}{4.298299in}}%
\pgfpathlineto{\pgfqpoint{4.685114in}{4.298382in}}%
\pgfpathlineto{\pgfqpoint{4.691557in}{4.296867in}}%
\pgfpathlineto{\pgfqpoint{4.692630in}{4.297291in}}%
\pgfpathlineto{\pgfqpoint{4.694778in}{4.297195in}}%
\pgfpathlineto{\pgfqpoint{4.698000in}{4.297571in}}%
\pgfpathlineto{\pgfqpoint{4.699073in}{4.296675in}}%
\pgfpathlineto{\pgfqpoint{4.701221in}{4.296983in}}%
\pgfpathlineto{\pgfqpoint{4.713033in}{4.297837in}}%
\pgfpathlineto{\pgfqpoint{4.716255in}{4.296378in}}%
\pgfpathlineto{\pgfqpoint{4.717329in}{4.296925in}}%
\pgfpathlineto{\pgfqpoint{4.720550in}{4.296850in}}%
\pgfpathlineto{\pgfqpoint{4.721624in}{4.296123in}}%
\pgfpathlineto{\pgfqpoint{4.731289in}{4.295964in}}%
\pgfpathlineto{\pgfqpoint{4.732362in}{4.295014in}}%
\pgfpathlineto{\pgfqpoint{4.739879in}{4.295267in}}%
\pgfpathlineto{\pgfqpoint{4.743101in}{4.294223in}}%
\pgfpathlineto{\pgfqpoint{4.744175in}{4.294549in}}%
\pgfpathlineto{\pgfqpoint{4.746322in}{4.293643in}}%
\pgfpathlineto{\pgfqpoint{4.747396in}{4.292232in}}%
\pgfpathlineto{\pgfqpoint{4.750618in}{4.291889in}}%
\pgfpathlineto{\pgfqpoint{4.753839in}{4.292773in}}%
\pgfpathlineto{\pgfqpoint{4.754913in}{4.292542in}}%
\pgfpathlineto{\pgfqpoint{4.773168in}{4.292429in}}%
\pgfpathlineto{\pgfqpoint{4.775316in}{4.290722in}}%
\pgfpathlineto{\pgfqpoint{4.780685in}{4.289285in}}%
\pgfpathlineto{\pgfqpoint{4.781759in}{4.289644in}}%
\pgfpathlineto{\pgfqpoint{4.783907in}{4.289435in}}%
\pgfpathlineto{\pgfqpoint{4.784981in}{4.289727in}}%
\pgfpathlineto{\pgfqpoint{4.791424in}{4.290164in}}%
\pgfpathlineto{\pgfqpoint{4.792497in}{4.289948in}}%
\pgfpathlineto{\pgfqpoint{4.796793in}{4.290125in}}%
\pgfpathlineto{\pgfqpoint{4.800014in}{4.287851in}}%
\pgfpathlineto{\pgfqpoint{4.803236in}{4.287963in}}%
\pgfpathlineto{\pgfqpoint{4.805383in}{4.286317in}}%
\pgfpathlineto{\pgfqpoint{4.806457in}{4.286545in}}%
\pgfpathlineto{\pgfqpoint{4.807531in}{4.285512in}}%
\pgfpathlineto{\pgfqpoint{4.810753in}{4.285912in}}%
\pgfpathlineto{\pgfqpoint{4.813974in}{4.284462in}}%
\pgfpathlineto{\pgfqpoint{4.815048in}{4.284638in}}%
\pgfpathlineto{\pgfqpoint{4.822565in}{4.283115in}}%
\pgfpathlineto{\pgfqpoint{4.826860in}{4.283152in}}%
\pgfpathlineto{\pgfqpoint{4.827934in}{4.283134in}}%
\pgfpathlineto{\pgfqpoint{4.830082in}{4.282403in}}%
\pgfpathlineto{\pgfqpoint{4.836525in}{4.282653in}}%
\pgfpathlineto{\pgfqpoint{4.837599in}{4.282092in}}%
\pgfpathlineto{\pgfqpoint{4.842968in}{4.281500in}}%
\pgfpathlineto{\pgfqpoint{4.844042in}{4.280418in}}%
\pgfpathlineto{\pgfqpoint{4.858002in}{4.280325in}}%
\pgfpathlineto{\pgfqpoint{4.859075in}{4.279607in}}%
\pgfpathlineto{\pgfqpoint{4.865518in}{4.279430in}}%
\pgfpathlineto{\pgfqpoint{4.867666in}{4.278798in}}%
\pgfpathlineto{\pgfqpoint{4.879478in}{4.278356in}}%
\pgfpathlineto{\pgfqpoint{4.881626in}{4.278207in}}%
\pgfpathlineto{\pgfqpoint{4.894512in}{4.277695in}}%
\pgfpathlineto{\pgfqpoint{4.897734in}{4.278028in}}%
\pgfpathlineto{\pgfqpoint{4.915989in}{4.278194in}}%
\pgfpathlineto{\pgfqpoint{4.920284in}{4.278001in}}%
\pgfpathlineto{\pgfqpoint{4.931023in}{4.277058in}}%
\pgfpathlineto{\pgfqpoint{4.935318in}{4.276203in}}%
\pgfpathlineto{\pgfqpoint{4.939613in}{4.276157in}}%
\pgfpathlineto{\pgfqpoint{4.940687in}{4.275811in}}%
\pgfpathlineto{\pgfqpoint{4.941761in}{4.274580in}}%
\pgfpathlineto{\pgfqpoint{4.949278in}{4.274772in}}%
\pgfpathlineto{\pgfqpoint{4.950352in}{4.274286in}}%
\pgfpathlineto{\pgfqpoint{4.963238in}{4.274153in}}%
\pgfpathlineto{\pgfqpoint{4.965385in}{4.274058in}}%
\pgfpathlineto{\pgfqpoint{4.976124in}{4.274212in}}%
\pgfpathlineto{\pgfqpoint{4.978271in}{4.274188in}}%
\pgfpathlineto{\pgfqpoint{4.980419in}{4.273700in}}%
\pgfpathlineto{\pgfqpoint{4.992231in}{4.273896in}}%
\pgfpathlineto{\pgfqpoint{4.999748in}{4.273352in}}%
\pgfpathlineto{\pgfqpoint{5.001896in}{4.272848in}}%
\pgfpathlineto{\pgfqpoint{5.007265in}{4.273052in}}%
\pgfpathlineto{\pgfqpoint{5.009413in}{4.272678in}}%
\pgfpathlineto{\pgfqpoint{5.010487in}{4.272399in}}%
\pgfpathlineto{\pgfqpoint{5.028742in}{4.272901in}}%
\pgfpathlineto{\pgfqpoint{5.046997in}{4.271351in}}%
\pgfpathlineto{\pgfqpoint{5.048071in}{4.270865in}}%
\pgfpathlineto{\pgfqpoint{5.062031in}{4.269233in}}%
\pgfpathlineto{\pgfqpoint{5.066326in}{4.268701in}}%
\pgfpathlineto{\pgfqpoint{5.085655in}{4.269265in}}%
\pgfpathlineto{\pgfqpoint{5.091024in}{4.268423in}}%
\pgfpathlineto{\pgfqpoint{5.093172in}{4.268233in}}%
\pgfpathlineto{\pgfqpoint{5.099615in}{4.267677in}}%
\pgfpathlineto{\pgfqpoint{5.103911in}{4.267778in}}%
\pgfpathlineto{\pgfqpoint{5.112501in}{4.267704in}}%
\pgfpathlineto{\pgfqpoint{5.115723in}{4.267637in}}%
\pgfpathlineto{\pgfqpoint{5.127535in}{4.266980in}}%
\pgfpathlineto{\pgfqpoint{5.130757in}{4.265889in}}%
\pgfpathlineto{\pgfqpoint{5.137200in}{4.265720in}}%
\pgfpathlineto{\pgfqpoint{5.157602in}{4.265716in}}%
\pgfpathlineto{\pgfqpoint{5.159750in}{4.265528in}}%
\pgfpathlineto{\pgfqpoint{5.171562in}{4.265451in}}%
\pgfpathlineto{\pgfqpoint{5.174784in}{4.265456in}}%
\pgfpathlineto{\pgfqpoint{5.175858in}{4.265554in}}%
\pgfpathlineto{\pgfqpoint{5.183375in}{4.265164in}}%
\pgfpathlineto{\pgfqpoint{5.209147in}{4.265129in}}%
\pgfpathlineto{\pgfqpoint{5.213442in}{4.265048in}}%
\pgfpathlineto{\pgfqpoint{5.248879in}{4.264521in}}%
\pgfpathlineto{\pgfqpoint{5.251026in}{4.263898in}}%
\pgfpathlineto{\pgfqpoint{5.263912in}{4.263623in}}%
\pgfpathlineto{\pgfqpoint{5.280020in}{4.263843in}}%
\pgfpathlineto{\pgfqpoint{5.288611in}{4.263922in}}%
\pgfpathlineto{\pgfqpoint{5.322974in}{4.262625in}}%
\pgfpathlineto{\pgfqpoint{5.326195in}{4.262070in}}%
\pgfpathlineto{\pgfqpoint{5.348746in}{4.260428in}}%
\pgfpathlineto{\pgfqpoint{5.370223in}{4.260492in}}%
\pgfpathlineto{\pgfqpoint{5.386330in}{4.260687in}}%
\pgfpathlineto{\pgfqpoint{5.397068in}{4.260436in}}%
\pgfpathlineto{\pgfqpoint{5.401364in}{4.259778in}}%
\pgfpathlineto{\pgfqpoint{5.419619in}{4.259576in}}%
\pgfpathlineto{\pgfqpoint{5.427136in}{4.258969in}}%
\pgfpathlineto{\pgfqpoint{5.449687in}{4.257966in}}%
\pgfpathlineto{\pgfqpoint{5.453982in}{4.257776in}}%
\pgfpathlineto{\pgfqpoint{5.467942in}{4.257309in}}%
\pgfpathlineto{\pgfqpoint{5.476533in}{4.257012in}}%
\pgfpathlineto{\pgfqpoint{5.499083in}{4.256668in}}%
\pgfpathlineto{\pgfqpoint{6.014526in}{4.252541in}}%
\pgfpathlineto{\pgfqpoint{6.238958in}{4.252227in}}%
\pgfpathlineto{\pgfqpoint{6.589029in}{4.252121in}}%
\pgfpathlineto{\pgfqpoint{6.589029in}{4.252121in}}%
\pgfusepath{stroke}%
\end{pgfscope}%
\begin{pgfscope}%
\pgfsetrectcap%
\pgfsetmiterjoin%
\pgfsetlinewidth{0.803000pt}%
\definecolor{currentstroke}{rgb}{1.000000,1.000000,1.000000}%
\pgfsetstrokecolor{currentstroke}%
\pgfsetdash{}{0pt}%
\pgfpathmoveto{\pgfqpoint{4.123120in}{4.233896in}}%
\pgfpathlineto{\pgfqpoint{4.123120in}{4.634781in}}%
\pgfusepath{stroke}%
\end{pgfscope}%
\begin{pgfscope}%
\pgfsetrectcap%
\pgfsetmiterjoin%
\pgfsetlinewidth{0.803000pt}%
\definecolor{currentstroke}{rgb}{1.000000,1.000000,1.000000}%
\pgfsetstrokecolor{currentstroke}%
\pgfsetdash{}{0pt}%
\pgfpathmoveto{\pgfqpoint{6.706453in}{4.233896in}}%
\pgfpathlineto{\pgfqpoint{6.706453in}{4.634781in}}%
\pgfusepath{stroke}%
\end{pgfscope}%
\begin{pgfscope}%
\pgfsetrectcap%
\pgfsetmiterjoin%
\pgfsetlinewidth{0.803000pt}%
\definecolor{currentstroke}{rgb}{1.000000,1.000000,1.000000}%
\pgfsetstrokecolor{currentstroke}%
\pgfsetdash{}{0pt}%
\pgfpathmoveto{\pgfqpoint{4.123120in}{4.233896in}}%
\pgfpathlineto{\pgfqpoint{6.706453in}{4.233896in}}%
\pgfusepath{stroke}%
\end{pgfscope}%
\begin{pgfscope}%
\pgfsetrectcap%
\pgfsetmiterjoin%
\pgfsetlinewidth{0.803000pt}%
\definecolor{currentstroke}{rgb}{1.000000,1.000000,1.000000}%
\pgfsetstrokecolor{currentstroke}%
\pgfsetdash{}{0pt}%
\pgfpathmoveto{\pgfqpoint{4.123120in}{4.634781in}}%
\pgfpathlineto{\pgfqpoint{6.706453in}{4.634781in}}%
\pgfusepath{stroke}%
\end{pgfscope}%
\begin{pgfscope}%
\definecolor{textcolor}{rgb}{0.150000,0.150000,0.150000}%
\pgfsetstrokecolor{textcolor}%
\pgfsetfillcolor{textcolor}%
\pgftext[x=5.414787in,y=4.718114in,,base]{\color{textcolor}\rmfamily\fontsize{16.800000}{20.160000}\selectfont AXP}%
\end{pgfscope}%
\begin{pgfscope}%
\pgfsetbuttcap%
\pgfsetmiterjoin%
\definecolor{currentfill}{rgb}{0.917647,0.917647,0.949020}%
\pgfsetfillcolor{currentfill}%
\pgfsetlinewidth{0.000000pt}%
\definecolor{currentstroke}{rgb}{0.000000,0.000000,0.000000}%
\pgfsetstrokecolor{currentstroke}%
\pgfsetstrokeopacity{0.000000}%
\pgfsetdash{}{0pt}%
\pgfpathmoveto{\pgfqpoint{0.506453in}{3.271772in}}%
\pgfpathlineto{\pgfqpoint{3.089787in}{3.271772in}}%
\pgfpathlineto{\pgfqpoint{3.089787in}{3.672657in}}%
\pgfpathlineto{\pgfqpoint{0.506453in}{3.672657in}}%
\pgfpathclose%
\pgfusepath{fill}%
\end{pgfscope}%
\begin{pgfscope}%
\pgfpathrectangle{\pgfqpoint{0.506453in}{3.271772in}}{\pgfqpoint{2.583333in}{0.400885in}}%
\pgfusepath{clip}%
\pgfsetroundcap%
\pgfsetroundjoin%
\pgfsetlinewidth{0.803000pt}%
\definecolor{currentstroke}{rgb}{1.000000,1.000000,1.000000}%
\pgfsetstrokecolor{currentstroke}%
\pgfsetdash{}{0pt}%
\pgfpathmoveto{\pgfqpoint{0.621730in}{3.271772in}}%
\pgfpathlineto{\pgfqpoint{0.621730in}{3.672657in}}%
\pgfusepath{stroke}%
\end{pgfscope}%
\begin{pgfscope}%
\definecolor{textcolor}{rgb}{0.150000,0.150000,0.150000}%
\pgfsetstrokecolor{textcolor}%
\pgfsetfillcolor{textcolor}%
\pgftext[x=0.621730in,y=3.174550in,,top]{\color{textcolor}\rmfamily\fontsize{14.000000}{16.800000}\selectfont 2012}%
\end{pgfscope}%
\begin{pgfscope}%
\pgfpathrectangle{\pgfqpoint{0.506453in}{3.271772in}}{\pgfqpoint{2.583333in}{0.400885in}}%
\pgfusepath{clip}%
\pgfsetroundcap%
\pgfsetroundjoin%
\pgfsetlinewidth{0.803000pt}%
\definecolor{currentstroke}{rgb}{1.000000,1.000000,1.000000}%
\pgfsetstrokecolor{currentstroke}%
\pgfsetdash{}{0pt}%
\pgfpathmoveto{\pgfqpoint{1.014755in}{3.271772in}}%
\pgfpathlineto{\pgfqpoint{1.014755in}{3.672657in}}%
\pgfusepath{stroke}%
\end{pgfscope}%
\begin{pgfscope}%
\definecolor{textcolor}{rgb}{0.150000,0.150000,0.150000}%
\pgfsetstrokecolor{textcolor}%
\pgfsetfillcolor{textcolor}%
\pgftext[x=1.014755in,y=3.174550in,,top]{\color{textcolor}\rmfamily\fontsize{14.000000}{16.800000}\selectfont 2013}%
\end{pgfscope}%
\begin{pgfscope}%
\pgfpathrectangle{\pgfqpoint{0.506453in}{3.271772in}}{\pgfqpoint{2.583333in}{0.400885in}}%
\pgfusepath{clip}%
\pgfsetroundcap%
\pgfsetroundjoin%
\pgfsetlinewidth{0.803000pt}%
\definecolor{currentstroke}{rgb}{1.000000,1.000000,1.000000}%
\pgfsetstrokecolor{currentstroke}%
\pgfsetdash{}{0pt}%
\pgfpathmoveto{\pgfqpoint{1.406706in}{3.271772in}}%
\pgfpathlineto{\pgfqpoint{1.406706in}{3.672657in}}%
\pgfusepath{stroke}%
\end{pgfscope}%
\begin{pgfscope}%
\definecolor{textcolor}{rgb}{0.150000,0.150000,0.150000}%
\pgfsetstrokecolor{textcolor}%
\pgfsetfillcolor{textcolor}%
\pgftext[x=1.406706in,y=3.174550in,,top]{\color{textcolor}\rmfamily\fontsize{14.000000}{16.800000}\selectfont 2014}%
\end{pgfscope}%
\begin{pgfscope}%
\pgfpathrectangle{\pgfqpoint{0.506453in}{3.271772in}}{\pgfqpoint{2.583333in}{0.400885in}}%
\pgfusepath{clip}%
\pgfsetroundcap%
\pgfsetroundjoin%
\pgfsetlinewidth{0.803000pt}%
\definecolor{currentstroke}{rgb}{1.000000,1.000000,1.000000}%
\pgfsetstrokecolor{currentstroke}%
\pgfsetdash{}{0pt}%
\pgfpathmoveto{\pgfqpoint{1.798657in}{3.271772in}}%
\pgfpathlineto{\pgfqpoint{1.798657in}{3.672657in}}%
\pgfusepath{stroke}%
\end{pgfscope}%
\begin{pgfscope}%
\definecolor{textcolor}{rgb}{0.150000,0.150000,0.150000}%
\pgfsetstrokecolor{textcolor}%
\pgfsetfillcolor{textcolor}%
\pgftext[x=1.798657in,y=3.174550in,,top]{\color{textcolor}\rmfamily\fontsize{14.000000}{16.800000}\selectfont 2015}%
\end{pgfscope}%
\begin{pgfscope}%
\pgfpathrectangle{\pgfqpoint{0.506453in}{3.271772in}}{\pgfqpoint{2.583333in}{0.400885in}}%
\pgfusepath{clip}%
\pgfsetroundcap%
\pgfsetroundjoin%
\pgfsetlinewidth{0.803000pt}%
\definecolor{currentstroke}{rgb}{1.000000,1.000000,1.000000}%
\pgfsetstrokecolor{currentstroke}%
\pgfsetdash{}{0pt}%
\pgfpathmoveto{\pgfqpoint{2.190608in}{3.271772in}}%
\pgfpathlineto{\pgfqpoint{2.190608in}{3.672657in}}%
\pgfusepath{stroke}%
\end{pgfscope}%
\begin{pgfscope}%
\definecolor{textcolor}{rgb}{0.150000,0.150000,0.150000}%
\pgfsetstrokecolor{textcolor}%
\pgfsetfillcolor{textcolor}%
\pgftext[x=2.190608in,y=3.174550in,,top]{\color{textcolor}\rmfamily\fontsize{14.000000}{16.800000}\selectfont 2016}%
\end{pgfscope}%
\begin{pgfscope}%
\pgfpathrectangle{\pgfqpoint{0.506453in}{3.271772in}}{\pgfqpoint{2.583333in}{0.400885in}}%
\pgfusepath{clip}%
\pgfsetroundcap%
\pgfsetroundjoin%
\pgfsetlinewidth{0.803000pt}%
\definecolor{currentstroke}{rgb}{1.000000,1.000000,1.000000}%
\pgfsetstrokecolor{currentstroke}%
\pgfsetdash{}{0pt}%
\pgfpathmoveto{\pgfqpoint{2.583633in}{3.271772in}}%
\pgfpathlineto{\pgfqpoint{2.583633in}{3.672657in}}%
\pgfusepath{stroke}%
\end{pgfscope}%
\begin{pgfscope}%
\definecolor{textcolor}{rgb}{0.150000,0.150000,0.150000}%
\pgfsetstrokecolor{textcolor}%
\pgfsetfillcolor{textcolor}%
\pgftext[x=2.583633in,y=3.174550in,,top]{\color{textcolor}\rmfamily\fontsize{14.000000}{16.800000}\selectfont 2017}%
\end{pgfscope}%
\begin{pgfscope}%
\pgfpathrectangle{\pgfqpoint{0.506453in}{3.271772in}}{\pgfqpoint{2.583333in}{0.400885in}}%
\pgfusepath{clip}%
\pgfsetroundcap%
\pgfsetroundjoin%
\pgfsetlinewidth{0.803000pt}%
\definecolor{currentstroke}{rgb}{1.000000,1.000000,1.000000}%
\pgfsetstrokecolor{currentstroke}%
\pgfsetdash{}{0pt}%
\pgfpathmoveto{\pgfqpoint{2.975584in}{3.271772in}}%
\pgfpathlineto{\pgfqpoint{2.975584in}{3.672657in}}%
\pgfusepath{stroke}%
\end{pgfscope}%
\begin{pgfscope}%
\definecolor{textcolor}{rgb}{0.150000,0.150000,0.150000}%
\pgfsetstrokecolor{textcolor}%
\pgfsetfillcolor{textcolor}%
\pgftext[x=2.975584in,y=3.174550in,,top]{\color{textcolor}\rmfamily\fontsize{14.000000}{16.800000}\selectfont 2018}%
\end{pgfscope}%
\begin{pgfscope}%
\pgfpathrectangle{\pgfqpoint{0.506453in}{3.271772in}}{\pgfqpoint{2.583333in}{0.400885in}}%
\pgfusepath{clip}%
\pgfsetroundcap%
\pgfsetroundjoin%
\pgfsetlinewidth{0.803000pt}%
\definecolor{currentstroke}{rgb}{1.000000,1.000000,1.000000}%
\pgfsetstrokecolor{currentstroke}%
\pgfsetdash{}{0pt}%
\pgfpathmoveto{\pgfqpoint{0.506453in}{3.289809in}}%
\pgfpathlineto{\pgfqpoint{3.089787in}{3.289809in}}%
\pgfusepath{stroke}%
\end{pgfscope}%
\begin{pgfscope}%
\definecolor{textcolor}{rgb}{0.150000,0.150000,0.150000}%
\pgfsetstrokecolor{textcolor}%
\pgfsetfillcolor{textcolor}%
\pgftext[x=0.285520in,y=3.215943in,left,base]{\color{textcolor}\rmfamily\fontsize{14.000000}{16.800000}\selectfont 0}%
\end{pgfscope}%
\begin{pgfscope}%
\pgfpathrectangle{\pgfqpoint{0.506453in}{3.271772in}}{\pgfqpoint{2.583333in}{0.400885in}}%
\pgfusepath{clip}%
\pgfsetroundcap%
\pgfsetroundjoin%
\pgfsetlinewidth{0.803000pt}%
\definecolor{currentstroke}{rgb}{1.000000,1.000000,1.000000}%
\pgfsetstrokecolor{currentstroke}%
\pgfsetdash{}{0pt}%
\pgfpathmoveto{\pgfqpoint{0.506453in}{3.638747in}}%
\pgfpathlineto{\pgfqpoint{3.089787in}{3.638747in}}%
\pgfusepath{stroke}%
\end{pgfscope}%
\begin{pgfscope}%
\definecolor{textcolor}{rgb}{0.150000,0.150000,0.150000}%
\pgfsetstrokecolor{textcolor}%
\pgfsetfillcolor{textcolor}%
\pgftext[x=0.285520in,y=3.564880in,left,base]{\color{textcolor}\rmfamily\fontsize{14.000000}{16.800000}\selectfont 2}%
\end{pgfscope}%
\begin{pgfscope}%
\pgfpathrectangle{\pgfqpoint{0.506453in}{3.271772in}}{\pgfqpoint{2.583333in}{0.400885in}}%
\pgfusepath{clip}%
\pgfsetroundcap%
\pgfsetroundjoin%
\pgfsetlinewidth{1.505625pt}%
\definecolor{currentstroke}{rgb}{0.000000,0.000000,0.000000}%
\pgfsetstrokecolor{currentstroke}%
\pgfsetdash{}{0pt}%
\pgfpathmoveto{\pgfqpoint{0.623878in}{3.464278in}}%
\pgfpathlineto{\pgfqpoint{0.624952in}{3.466176in}}%
\pgfpathlineto{\pgfqpoint{0.626025in}{3.466049in}}%
\pgfpathlineto{\pgfqpoint{0.627099in}{3.467061in}}%
\pgfpathlineto{\pgfqpoint{0.630321in}{3.469085in}}%
\pgfpathlineto{\pgfqpoint{0.631395in}{3.467694in}}%
\pgfpathlineto{\pgfqpoint{0.632468in}{3.469212in}}%
\pgfpathlineto{\pgfqpoint{0.633542in}{3.469718in}}%
\pgfpathlineto{\pgfqpoint{0.634616in}{3.468832in}}%
\pgfpathlineto{\pgfqpoint{0.638911in}{3.467947in}}%
\pgfpathlineto{\pgfqpoint{0.639985in}{3.470604in}}%
\pgfpathlineto{\pgfqpoint{0.641059in}{3.471742in}}%
\pgfpathlineto{\pgfqpoint{0.642133in}{3.471742in}}%
\pgfpathlineto{\pgfqpoint{0.646428in}{3.468832in}}%
\pgfpathlineto{\pgfqpoint{0.647502in}{3.471616in}}%
\pgfpathlineto{\pgfqpoint{0.649650in}{3.470604in}}%
\pgfpathlineto{\pgfqpoint{0.652871in}{3.469465in}}%
\pgfpathlineto{\pgfqpoint{0.653945in}{3.467567in}}%
\pgfpathlineto{\pgfqpoint{0.655019in}{3.468200in}}%
\pgfpathlineto{\pgfqpoint{0.656093in}{3.467947in}}%
\pgfpathlineto{\pgfqpoint{0.657167in}{3.470604in}}%
\pgfpathlineto{\pgfqpoint{0.660388in}{3.470857in}}%
\pgfpathlineto{\pgfqpoint{0.662536in}{3.472628in}}%
\pgfpathlineto{\pgfqpoint{0.663610in}{3.471616in}}%
\pgfpathlineto{\pgfqpoint{0.664684in}{3.469212in}}%
\pgfpathlineto{\pgfqpoint{0.667905in}{3.470983in}}%
\pgfpathlineto{\pgfqpoint{0.670053in}{3.468073in}}%
\pgfpathlineto{\pgfqpoint{0.672200in}{3.473008in}}%
\pgfpathlineto{\pgfqpoint{0.678643in}{3.474905in}}%
\pgfpathlineto{\pgfqpoint{0.682939in}{3.472628in}}%
\pgfpathlineto{\pgfqpoint{0.684013in}{3.473514in}}%
\pgfpathlineto{\pgfqpoint{0.685087in}{3.472501in}}%
\pgfpathlineto{\pgfqpoint{0.686160in}{3.473134in}}%
\pgfpathlineto{\pgfqpoint{0.687234in}{3.471742in}}%
\pgfpathlineto{\pgfqpoint{0.690456in}{3.470477in}}%
\pgfpathlineto{\pgfqpoint{0.691530in}{3.466429in}}%
\pgfpathlineto{\pgfqpoint{0.693677in}{3.472248in}}%
\pgfpathlineto{\pgfqpoint{0.697973in}{3.473261in}}%
\pgfpathlineto{\pgfqpoint{0.699046in}{3.477689in}}%
\pgfpathlineto{\pgfqpoint{0.700120in}{3.479587in}}%
\pgfpathlineto{\pgfqpoint{0.701194in}{3.483129in}}%
\pgfpathlineto{\pgfqpoint{0.702268in}{3.483509in}}%
\pgfpathlineto{\pgfqpoint{0.705489in}{3.483635in}}%
\pgfpathlineto{\pgfqpoint{0.706563in}{3.482243in}}%
\pgfpathlineto{\pgfqpoint{0.707637in}{3.482243in}}%
\pgfpathlineto{\pgfqpoint{0.708711in}{3.480093in}}%
\pgfpathlineto{\pgfqpoint{0.709785in}{3.479460in}}%
\pgfpathlineto{\pgfqpoint{0.713006in}{3.481990in}}%
\pgfpathlineto{\pgfqpoint{0.715154in}{3.481611in}}%
\pgfpathlineto{\pgfqpoint{0.716228in}{3.481105in}}%
\pgfpathlineto{\pgfqpoint{0.717302in}{3.482243in}}%
\pgfpathlineto{\pgfqpoint{0.721597in}{3.481231in}}%
\pgfpathlineto{\pgfqpoint{0.723745in}{3.476677in}}%
\pgfpathlineto{\pgfqpoint{0.728040in}{3.473893in}}%
\pgfpathlineto{\pgfqpoint{0.729114in}{3.469465in}}%
\pgfpathlineto{\pgfqpoint{0.731262in}{3.474905in}}%
\pgfpathlineto{\pgfqpoint{0.732335in}{3.470857in}}%
\pgfpathlineto{\pgfqpoint{0.735557in}{3.470983in}}%
\pgfpathlineto{\pgfqpoint{0.736631in}{3.475285in}}%
\pgfpathlineto{\pgfqpoint{0.737705in}{3.472881in}}%
\pgfpathlineto{\pgfqpoint{0.738778in}{3.473261in}}%
\pgfpathlineto{\pgfqpoint{0.739852in}{3.475411in}}%
\pgfpathlineto{\pgfqpoint{0.743074in}{3.472628in}}%
\pgfpathlineto{\pgfqpoint{0.744148in}{3.477183in}}%
\pgfpathlineto{\pgfqpoint{0.745221in}{3.476297in}}%
\pgfpathlineto{\pgfqpoint{0.747369in}{3.479460in}}%
\pgfpathlineto{\pgfqpoint{0.750591in}{3.477562in}}%
\pgfpathlineto{\pgfqpoint{0.751664in}{3.479587in}}%
\pgfpathlineto{\pgfqpoint{0.752738in}{3.479333in}}%
\pgfpathlineto{\pgfqpoint{0.753812in}{3.477815in}}%
\pgfpathlineto{\pgfqpoint{0.754886in}{3.475285in}}%
\pgfpathlineto{\pgfqpoint{0.759181in}{3.474399in}}%
\pgfpathlineto{\pgfqpoint{0.760255in}{3.471110in}}%
\pgfpathlineto{\pgfqpoint{0.761329in}{3.472881in}}%
\pgfpathlineto{\pgfqpoint{0.762403in}{3.472122in}}%
\pgfpathlineto{\pgfqpoint{0.766698in}{3.466176in}}%
\pgfpathlineto{\pgfqpoint{0.767772in}{3.471995in}}%
\pgfpathlineto{\pgfqpoint{0.768846in}{3.470857in}}%
\pgfpathlineto{\pgfqpoint{0.774215in}{3.473640in}}%
\pgfpathlineto{\pgfqpoint{0.775289in}{3.473640in}}%
\pgfpathlineto{\pgfqpoint{0.776363in}{3.474399in}}%
\pgfpathlineto{\pgfqpoint{0.777437in}{3.473893in}}%
\pgfpathlineto{\pgfqpoint{0.781732in}{3.475285in}}%
\pgfpathlineto{\pgfqpoint{0.782806in}{3.472375in}}%
\pgfpathlineto{\pgfqpoint{0.783880in}{3.472881in}}%
\pgfpathlineto{\pgfqpoint{0.784953in}{3.467567in}}%
\pgfpathlineto{\pgfqpoint{0.788175in}{3.463772in}}%
\pgfpathlineto{\pgfqpoint{0.789249in}{3.464657in}}%
\pgfpathlineto{\pgfqpoint{0.790323in}{3.470857in}}%
\pgfpathlineto{\pgfqpoint{0.792470in}{3.473893in}}%
\pgfpathlineto{\pgfqpoint{0.795692in}{3.473008in}}%
\pgfpathlineto{\pgfqpoint{0.796766in}{3.476550in}}%
\pgfpathlineto{\pgfqpoint{0.797840in}{3.475538in}}%
\pgfpathlineto{\pgfqpoint{0.799987in}{3.482117in}}%
\pgfpathlineto{\pgfqpoint{0.803209in}{3.479713in}}%
\pgfpathlineto{\pgfqpoint{0.804283in}{3.482117in}}%
\pgfpathlineto{\pgfqpoint{0.805356in}{3.483003in}}%
\pgfpathlineto{\pgfqpoint{0.806430in}{3.479207in}}%
\pgfpathlineto{\pgfqpoint{0.807504in}{3.481864in}}%
\pgfpathlineto{\pgfqpoint{0.810726in}{3.479080in}}%
\pgfpathlineto{\pgfqpoint{0.813947in}{3.485659in}}%
\pgfpathlineto{\pgfqpoint{0.815021in}{3.491859in}}%
\pgfpathlineto{\pgfqpoint{0.819316in}{3.487937in}}%
\pgfpathlineto{\pgfqpoint{0.821464in}{3.486925in}}%
\pgfpathlineto{\pgfqpoint{0.822538in}{3.483762in}}%
\pgfpathlineto{\pgfqpoint{0.825759in}{3.484141in}}%
\pgfpathlineto{\pgfqpoint{0.826833in}{3.480093in}}%
\pgfpathlineto{\pgfqpoint{0.827907in}{3.480599in}}%
\pgfpathlineto{\pgfqpoint{0.828981in}{3.478321in}}%
\pgfpathlineto{\pgfqpoint{0.830055in}{3.481484in}}%
\pgfpathlineto{\pgfqpoint{0.833276in}{3.479713in}}%
\pgfpathlineto{\pgfqpoint{0.835424in}{3.482243in}}%
\pgfpathlineto{\pgfqpoint{0.836498in}{3.481737in}}%
\pgfpathlineto{\pgfqpoint{0.840793in}{3.484647in}}%
\pgfpathlineto{\pgfqpoint{0.841867in}{3.483382in}}%
\pgfpathlineto{\pgfqpoint{0.842941in}{3.483762in}}%
\pgfpathlineto{\pgfqpoint{0.845088in}{3.492618in}}%
\pgfpathlineto{\pgfqpoint{0.850458in}{3.490847in}}%
\pgfpathlineto{\pgfqpoint{0.851531in}{3.488822in}}%
\pgfpathlineto{\pgfqpoint{0.852605in}{3.492997in}}%
\pgfpathlineto{\pgfqpoint{0.855827in}{3.493124in}}%
\pgfpathlineto{\pgfqpoint{0.856901in}{3.494642in}}%
\pgfpathlineto{\pgfqpoint{0.857975in}{3.493504in}}%
\pgfpathlineto{\pgfqpoint{0.860122in}{3.494389in}}%
\pgfpathlineto{\pgfqpoint{0.865491in}{3.492997in}}%
\pgfpathlineto{\pgfqpoint{0.866565in}{3.493883in}}%
\pgfpathlineto{\pgfqpoint{0.867639in}{3.493377in}}%
\pgfpathlineto{\pgfqpoint{0.871934in}{3.491985in}}%
\pgfpathlineto{\pgfqpoint{0.873008in}{3.491353in}}%
\pgfpathlineto{\pgfqpoint{0.874082in}{3.489961in}}%
\pgfpathlineto{\pgfqpoint{0.875156in}{3.491479in}}%
\pgfpathlineto{\pgfqpoint{0.879451in}{3.491606in}}%
\pgfpathlineto{\pgfqpoint{0.880525in}{3.491732in}}%
\pgfpathlineto{\pgfqpoint{0.881599in}{3.489961in}}%
\pgfpathlineto{\pgfqpoint{0.882673in}{3.490594in}}%
\pgfpathlineto{\pgfqpoint{0.886968in}{3.488696in}}%
\pgfpathlineto{\pgfqpoint{0.888042in}{3.490088in}}%
\pgfpathlineto{\pgfqpoint{0.889116in}{3.496413in}}%
\pgfpathlineto{\pgfqpoint{0.890190in}{3.499197in}}%
\pgfpathlineto{\pgfqpoint{0.893411in}{3.498058in}}%
\pgfpathlineto{\pgfqpoint{0.894485in}{3.499197in}}%
\pgfpathlineto{\pgfqpoint{0.895559in}{3.502107in}}%
\pgfpathlineto{\pgfqpoint{0.897707in}{3.504131in}}%
\pgfpathlineto{\pgfqpoint{0.900928in}{3.503625in}}%
\pgfpathlineto{\pgfqpoint{0.905223in}{3.509951in}}%
\pgfpathlineto{\pgfqpoint{0.909519in}{3.507800in}}%
\pgfpathlineto{\pgfqpoint{0.910593in}{3.505776in}}%
\pgfpathlineto{\pgfqpoint{0.911666in}{3.511849in}}%
\pgfpathlineto{\pgfqpoint{0.912740in}{3.511722in}}%
\pgfpathlineto{\pgfqpoint{0.919183in}{3.514000in}}%
\pgfpathlineto{\pgfqpoint{0.920257in}{3.515644in}}%
\pgfpathlineto{\pgfqpoint{0.923479in}{3.513746in}}%
\pgfpathlineto{\pgfqpoint{0.925626in}{3.508939in}}%
\pgfpathlineto{\pgfqpoint{0.926700in}{3.509698in}}%
\pgfpathlineto{\pgfqpoint{0.927774in}{3.509445in}}%
\pgfpathlineto{\pgfqpoint{0.932069in}{3.510963in}}%
\pgfpathlineto{\pgfqpoint{0.933143in}{3.513620in}}%
\pgfpathlineto{\pgfqpoint{0.934217in}{3.512608in}}%
\pgfpathlineto{\pgfqpoint{0.935291in}{3.505017in}}%
\pgfpathlineto{\pgfqpoint{0.938512in}{3.501854in}}%
\pgfpathlineto{\pgfqpoint{0.939586in}{3.497679in}}%
\pgfpathlineto{\pgfqpoint{0.941734in}{3.497552in}}%
\pgfpathlineto{\pgfqpoint{0.942808in}{3.496034in}}%
\pgfpathlineto{\pgfqpoint{0.948177in}{3.495528in}}%
\pgfpathlineto{\pgfqpoint{0.949251in}{3.498311in}}%
\pgfpathlineto{\pgfqpoint{0.950325in}{3.498058in}}%
\pgfpathlineto{\pgfqpoint{0.953546in}{3.498944in}}%
\pgfpathlineto{\pgfqpoint{0.954620in}{3.500715in}}%
\pgfpathlineto{\pgfqpoint{0.956768in}{3.493883in}}%
\pgfpathlineto{\pgfqpoint{0.957841in}{3.495022in}}%
\pgfpathlineto{\pgfqpoint{0.961063in}{3.493883in}}%
\pgfpathlineto{\pgfqpoint{0.962137in}{3.491859in}}%
\pgfpathlineto{\pgfqpoint{0.963211in}{3.485280in}}%
\pgfpathlineto{\pgfqpoint{0.965358in}{3.486672in}}%
\pgfpathlineto{\pgfqpoint{0.968580in}{3.491606in}}%
\pgfpathlineto{\pgfqpoint{0.969654in}{3.491226in}}%
\pgfpathlineto{\pgfqpoint{0.970728in}{3.491859in}}%
\pgfpathlineto{\pgfqpoint{0.972875in}{3.495401in}}%
\pgfpathlineto{\pgfqpoint{0.976097in}{3.495528in}}%
\pgfpathlineto{\pgfqpoint{0.977171in}{3.493757in}}%
\pgfpathlineto{\pgfqpoint{0.978244in}{3.496413in}}%
\pgfpathlineto{\pgfqpoint{0.980392in}{3.496287in}}%
\pgfpathlineto{\pgfqpoint{0.983614in}{3.493250in}}%
\pgfpathlineto{\pgfqpoint{0.984687in}{3.493630in}}%
\pgfpathlineto{\pgfqpoint{0.985761in}{3.497173in}}%
\pgfpathlineto{\pgfqpoint{0.987909in}{3.499450in}}%
\pgfpathlineto{\pgfqpoint{0.991130in}{3.498817in}}%
\pgfpathlineto{\pgfqpoint{0.992204in}{3.499956in}}%
\pgfpathlineto{\pgfqpoint{0.993278in}{3.502613in}}%
\pgfpathlineto{\pgfqpoint{0.994352in}{3.501095in}}%
\pgfpathlineto{\pgfqpoint{0.995426in}{3.501095in}}%
\pgfpathlineto{\pgfqpoint{0.998647in}{3.504131in}}%
\pgfpathlineto{\pgfqpoint{0.999721in}{3.501727in}}%
\pgfpathlineto{\pgfqpoint{1.000795in}{3.495022in}}%
\pgfpathlineto{\pgfqpoint{1.001869in}{3.497299in}}%
\pgfpathlineto{\pgfqpoint{1.002943in}{3.495654in}}%
\pgfpathlineto{\pgfqpoint{1.008312in}{3.494642in}}%
\pgfpathlineto{\pgfqpoint{1.009386in}{3.493757in}}%
\pgfpathlineto{\pgfqpoint{1.010460in}{3.491353in}}%
\pgfpathlineto{\pgfqpoint{1.015829in}{3.500209in}}%
\pgfpathlineto{\pgfqpoint{1.016903in}{3.497805in}}%
\pgfpathlineto{\pgfqpoint{1.017976in}{3.498817in}}%
\pgfpathlineto{\pgfqpoint{1.021198in}{3.498185in}}%
\pgfpathlineto{\pgfqpoint{1.022272in}{3.495907in}}%
\pgfpathlineto{\pgfqpoint{1.023346in}{3.496413in}}%
\pgfpathlineto{\pgfqpoint{1.024419in}{3.498564in}}%
\pgfpathlineto{\pgfqpoint{1.025493in}{3.498185in}}%
\pgfpathlineto{\pgfqpoint{1.028715in}{3.498058in}}%
\pgfpathlineto{\pgfqpoint{1.029789in}{3.498817in}}%
\pgfpathlineto{\pgfqpoint{1.030863in}{3.498058in}}%
\pgfpathlineto{\pgfqpoint{1.031936in}{3.499829in}}%
\pgfpathlineto{\pgfqpoint{1.033010in}{3.507168in}}%
\pgfpathlineto{\pgfqpoint{1.038379in}{3.506155in}}%
\pgfpathlineto{\pgfqpoint{1.039453in}{3.507168in}}%
\pgfpathlineto{\pgfqpoint{1.040527in}{3.509571in}}%
\pgfpathlineto{\pgfqpoint{1.043749in}{3.511596in}}%
\pgfpathlineto{\pgfqpoint{1.044822in}{3.511596in}}%
\pgfpathlineto{\pgfqpoint{1.045896in}{3.508939in}}%
\pgfpathlineto{\pgfqpoint{1.046970in}{3.509445in}}%
\pgfpathlineto{\pgfqpoint{1.048044in}{3.512861in}}%
\pgfpathlineto{\pgfqpoint{1.051265in}{3.509824in}}%
\pgfpathlineto{\pgfqpoint{1.052339in}{3.512102in}}%
\pgfpathlineto{\pgfqpoint{1.053413in}{3.511090in}}%
\pgfpathlineto{\pgfqpoint{1.055561in}{3.511596in}}%
\pgfpathlineto{\pgfqpoint{1.058782in}{3.511216in}}%
\pgfpathlineto{\pgfqpoint{1.059856in}{3.512481in}}%
\pgfpathlineto{\pgfqpoint{1.060930in}{3.520452in}}%
\pgfpathlineto{\pgfqpoint{1.062004in}{3.520578in}}%
\pgfpathlineto{\pgfqpoint{1.063078in}{3.519440in}}%
\pgfpathlineto{\pgfqpoint{1.067373in}{3.523994in}}%
\pgfpathlineto{\pgfqpoint{1.068447in}{3.520578in}}%
\pgfpathlineto{\pgfqpoint{1.069521in}{3.520958in}}%
\pgfpathlineto{\pgfqpoint{1.070595in}{3.522350in}}%
\pgfpathlineto{\pgfqpoint{1.073816in}{3.516530in}}%
\pgfpathlineto{\pgfqpoint{1.075964in}{3.522097in}}%
\pgfpathlineto{\pgfqpoint{1.077038in}{3.520578in}}%
\pgfpathlineto{\pgfqpoint{1.078111in}{3.520325in}}%
\pgfpathlineto{\pgfqpoint{1.081333in}{3.521085in}}%
\pgfpathlineto{\pgfqpoint{1.082407in}{3.524247in}}%
\pgfpathlineto{\pgfqpoint{1.083481in}{3.525133in}}%
\pgfpathlineto{\pgfqpoint{1.084554in}{3.525133in}}%
\pgfpathlineto{\pgfqpoint{1.085628in}{3.526145in}}%
\pgfpathlineto{\pgfqpoint{1.088850in}{3.524627in}}%
\pgfpathlineto{\pgfqpoint{1.089924in}{3.522476in}}%
\pgfpathlineto{\pgfqpoint{1.090997in}{3.523362in}}%
\pgfpathlineto{\pgfqpoint{1.092071in}{3.525260in}}%
\pgfpathlineto{\pgfqpoint{1.093145in}{3.522856in}}%
\pgfpathlineto{\pgfqpoint{1.096367in}{3.520958in}}%
\pgfpathlineto{\pgfqpoint{1.097441in}{3.521591in}}%
\pgfpathlineto{\pgfqpoint{1.098514in}{3.522982in}}%
\pgfpathlineto{\pgfqpoint{1.099588in}{3.521338in}}%
\pgfpathlineto{\pgfqpoint{1.100662in}{3.522097in}}%
\pgfpathlineto{\pgfqpoint{1.103884in}{3.520831in}}%
\pgfpathlineto{\pgfqpoint{1.104957in}{3.519566in}}%
\pgfpathlineto{\pgfqpoint{1.111400in}{3.519187in}}%
\pgfpathlineto{\pgfqpoint{1.112474in}{3.521844in}}%
\pgfpathlineto{\pgfqpoint{1.113548in}{3.518428in}}%
\pgfpathlineto{\pgfqpoint{1.114622in}{3.519187in}}%
\pgfpathlineto{\pgfqpoint{1.115696in}{3.517795in}}%
\pgfpathlineto{\pgfqpoint{1.118917in}{3.519566in}}%
\pgfpathlineto{\pgfqpoint{1.119991in}{3.519060in}}%
\pgfpathlineto{\pgfqpoint{1.121065in}{3.524247in}}%
\pgfpathlineto{\pgfqpoint{1.122139in}{3.524247in}}%
\pgfpathlineto{\pgfqpoint{1.123213in}{3.522982in}}%
\pgfpathlineto{\pgfqpoint{1.126434in}{3.516530in}}%
\pgfpathlineto{\pgfqpoint{1.127508in}{3.519440in}}%
\pgfpathlineto{\pgfqpoint{1.128582in}{3.516024in}}%
\pgfpathlineto{\pgfqpoint{1.129656in}{3.515138in}}%
\pgfpathlineto{\pgfqpoint{1.130729in}{3.506029in}}%
\pgfpathlineto{\pgfqpoint{1.133951in}{3.501980in}}%
\pgfpathlineto{\pgfqpoint{1.135025in}{3.503498in}}%
\pgfpathlineto{\pgfqpoint{1.136099in}{3.508053in}}%
\pgfpathlineto{\pgfqpoint{1.137173in}{3.508053in}}%
\pgfpathlineto{\pgfqpoint{1.138246in}{3.510584in}}%
\pgfpathlineto{\pgfqpoint{1.142542in}{3.511343in}}%
\pgfpathlineto{\pgfqpoint{1.143616in}{3.509951in}}%
\pgfpathlineto{\pgfqpoint{1.145763in}{3.514126in}}%
\pgfpathlineto{\pgfqpoint{1.148985in}{3.514253in}}%
\pgfpathlineto{\pgfqpoint{1.150059in}{3.515265in}}%
\pgfpathlineto{\pgfqpoint{1.151132in}{3.518554in}}%
\pgfpathlineto{\pgfqpoint{1.152206in}{3.516277in}}%
\pgfpathlineto{\pgfqpoint{1.153280in}{3.517415in}}%
\pgfpathlineto{\pgfqpoint{1.156502in}{3.516909in}}%
\pgfpathlineto{\pgfqpoint{1.160797in}{3.522982in}}%
\pgfpathlineto{\pgfqpoint{1.165092in}{3.525007in}}%
\pgfpathlineto{\pgfqpoint{1.166166in}{3.527031in}}%
\pgfpathlineto{\pgfqpoint{1.168314in}{3.523741in}}%
\pgfpathlineto{\pgfqpoint{1.174757in}{3.524374in}}%
\pgfpathlineto{\pgfqpoint{1.175831in}{3.521591in}}%
\pgfpathlineto{\pgfqpoint{1.179052in}{3.524754in}}%
\pgfpathlineto{\pgfqpoint{1.180126in}{3.525007in}}%
\pgfpathlineto{\pgfqpoint{1.181200in}{3.521591in}}%
\pgfpathlineto{\pgfqpoint{1.182274in}{3.522223in}}%
\pgfpathlineto{\pgfqpoint{1.183348in}{3.527031in}}%
\pgfpathlineto{\pgfqpoint{1.186569in}{3.526145in}}%
\pgfpathlineto{\pgfqpoint{1.188717in}{3.523362in}}%
\pgfpathlineto{\pgfqpoint{1.189791in}{3.525133in}}%
\pgfpathlineto{\pgfqpoint{1.190864in}{3.523615in}}%
\pgfpathlineto{\pgfqpoint{1.194086in}{3.526145in}}%
\pgfpathlineto{\pgfqpoint{1.195160in}{3.531712in}}%
\pgfpathlineto{\pgfqpoint{1.197307in}{3.522729in}}%
\pgfpathlineto{\pgfqpoint{1.198381in}{3.523868in}}%
\pgfpathlineto{\pgfqpoint{1.201603in}{3.519566in}}%
\pgfpathlineto{\pgfqpoint{1.203751in}{3.522729in}}%
\pgfpathlineto{\pgfqpoint{1.204824in}{3.523488in}}%
\pgfpathlineto{\pgfqpoint{1.205898in}{3.522223in}}%
\pgfpathlineto{\pgfqpoint{1.209120in}{3.523615in}}%
\pgfpathlineto{\pgfqpoint{1.210194in}{3.519313in}}%
\pgfpathlineto{\pgfqpoint{1.211267in}{3.519313in}}%
\pgfpathlineto{\pgfqpoint{1.213415in}{3.522603in}}%
\pgfpathlineto{\pgfqpoint{1.216637in}{3.523488in}}%
\pgfpathlineto{\pgfqpoint{1.217710in}{3.526525in}}%
\pgfpathlineto{\pgfqpoint{1.218784in}{3.525639in}}%
\pgfpathlineto{\pgfqpoint{1.219858in}{3.529688in}}%
\pgfpathlineto{\pgfqpoint{1.220932in}{3.527917in}}%
\pgfpathlineto{\pgfqpoint{1.224153in}{3.526525in}}%
\pgfpathlineto{\pgfqpoint{1.225227in}{3.524627in}}%
\pgfpathlineto{\pgfqpoint{1.227375in}{3.526525in}}%
\pgfpathlineto{\pgfqpoint{1.228449in}{3.537532in}}%
\pgfpathlineto{\pgfqpoint{1.231670in}{3.538924in}}%
\pgfpathlineto{\pgfqpoint{1.233818in}{3.536520in}}%
\pgfpathlineto{\pgfqpoint{1.234892in}{3.537152in}}%
\pgfpathlineto{\pgfqpoint{1.240261in}{3.535128in}}%
\pgfpathlineto{\pgfqpoint{1.241335in}{3.533989in}}%
\pgfpathlineto{\pgfqpoint{1.242409in}{3.536520in}}%
\pgfpathlineto{\pgfqpoint{1.243483in}{3.537279in}}%
\pgfpathlineto{\pgfqpoint{1.246704in}{3.535508in}}%
\pgfpathlineto{\pgfqpoint{1.247778in}{3.533357in}}%
\pgfpathlineto{\pgfqpoint{1.249926in}{3.533610in}}%
\pgfpathlineto{\pgfqpoint{1.250999in}{3.532724in}}%
\pgfpathlineto{\pgfqpoint{1.254221in}{3.532977in}}%
\pgfpathlineto{\pgfqpoint{1.255295in}{3.532345in}}%
\pgfpathlineto{\pgfqpoint{1.257442in}{3.530320in}}%
\pgfpathlineto{\pgfqpoint{1.262812in}{3.527410in}}%
\pgfpathlineto{\pgfqpoint{1.263885in}{3.526398in}}%
\pgfpathlineto{\pgfqpoint{1.264959in}{3.528043in}}%
\pgfpathlineto{\pgfqpoint{1.266033in}{3.528043in}}%
\pgfpathlineto{\pgfqpoint{1.269255in}{3.526398in}}%
\pgfpathlineto{\pgfqpoint{1.270329in}{3.522097in}}%
\pgfpathlineto{\pgfqpoint{1.271402in}{3.522223in}}%
\pgfpathlineto{\pgfqpoint{1.272476in}{3.521338in}}%
\pgfpathlineto{\pgfqpoint{1.273550in}{3.521717in}}%
\pgfpathlineto{\pgfqpoint{1.277845in}{3.520831in}}%
\pgfpathlineto{\pgfqpoint{1.278919in}{3.521970in}}%
\pgfpathlineto{\pgfqpoint{1.281067in}{3.521844in}}%
\pgfpathlineto{\pgfqpoint{1.284288in}{3.524121in}}%
\pgfpathlineto{\pgfqpoint{1.286436in}{3.531206in}}%
\pgfpathlineto{\pgfqpoint{1.287510in}{3.528802in}}%
\pgfpathlineto{\pgfqpoint{1.288584in}{3.528043in}}%
\pgfpathlineto{\pgfqpoint{1.291805in}{3.531712in}}%
\pgfpathlineto{\pgfqpoint{1.293953in}{3.538924in}}%
\pgfpathlineto{\pgfqpoint{1.295027in}{3.536773in}}%
\pgfpathlineto{\pgfqpoint{1.296101in}{3.532218in}}%
\pgfpathlineto{\pgfqpoint{1.300396in}{3.535381in}}%
\pgfpathlineto{\pgfqpoint{1.301470in}{3.534495in}}%
\pgfpathlineto{\pgfqpoint{1.302544in}{3.534622in}}%
\pgfpathlineto{\pgfqpoint{1.303617in}{3.532598in}}%
\pgfpathlineto{\pgfqpoint{1.306839in}{3.531079in}}%
\pgfpathlineto{\pgfqpoint{1.308987in}{3.535508in}}%
\pgfpathlineto{\pgfqpoint{1.310061in}{3.533104in}}%
\pgfpathlineto{\pgfqpoint{1.314356in}{3.531586in}}%
\pgfpathlineto{\pgfqpoint{1.315430in}{3.528802in}}%
\pgfpathlineto{\pgfqpoint{1.316504in}{3.527790in}}%
\pgfpathlineto{\pgfqpoint{1.317577in}{3.534622in}}%
\pgfpathlineto{\pgfqpoint{1.318651in}{3.536140in}}%
\pgfpathlineto{\pgfqpoint{1.321873in}{3.536014in}}%
\pgfpathlineto{\pgfqpoint{1.322947in}{3.533989in}}%
\pgfpathlineto{\pgfqpoint{1.324020in}{3.535761in}}%
\pgfpathlineto{\pgfqpoint{1.325094in}{3.539050in}}%
\pgfpathlineto{\pgfqpoint{1.326168in}{3.547780in}}%
\pgfpathlineto{\pgfqpoint{1.329390in}{3.553726in}}%
\pgfpathlineto{\pgfqpoint{1.330463in}{3.552588in}}%
\pgfpathlineto{\pgfqpoint{1.331537in}{3.549298in}}%
\pgfpathlineto{\pgfqpoint{1.332611in}{3.551702in}}%
\pgfpathlineto{\pgfqpoint{1.333685in}{3.551069in}}%
\pgfpathlineto{\pgfqpoint{1.337980in}{3.554485in}}%
\pgfpathlineto{\pgfqpoint{1.339054in}{3.556004in}}%
\pgfpathlineto{\pgfqpoint{1.340128in}{3.553726in}}%
\pgfpathlineto{\pgfqpoint{1.341202in}{3.557775in}}%
\pgfpathlineto{\pgfqpoint{1.345497in}{3.556510in}}%
\pgfpathlineto{\pgfqpoint{1.346571in}{3.561444in}}%
\pgfpathlineto{\pgfqpoint{1.347645in}{3.558407in}}%
\pgfpathlineto{\pgfqpoint{1.348719in}{3.562962in}}%
\pgfpathlineto{\pgfqpoint{1.353014in}{3.562962in}}%
\pgfpathlineto{\pgfqpoint{1.354088in}{3.563974in}}%
\pgfpathlineto{\pgfqpoint{1.355162in}{3.562330in}}%
\pgfpathlineto{\pgfqpoint{1.356236in}{3.564480in}}%
\pgfpathlineto{\pgfqpoint{1.359457in}{3.564607in}}%
\pgfpathlineto{\pgfqpoint{1.360531in}{3.562709in}}%
\pgfpathlineto{\pgfqpoint{1.362679in}{3.561570in}}%
\pgfpathlineto{\pgfqpoint{1.363752in}{3.563215in}}%
\pgfpathlineto{\pgfqpoint{1.366974in}{3.559673in}}%
\pgfpathlineto{\pgfqpoint{1.369122in}{3.560685in}}%
\pgfpathlineto{\pgfqpoint{1.371269in}{3.559040in}}%
\pgfpathlineto{\pgfqpoint{1.374491in}{3.559040in}}%
\pgfpathlineto{\pgfqpoint{1.375565in}{3.558028in}}%
\pgfpathlineto{\pgfqpoint{1.376639in}{3.558787in}}%
\pgfpathlineto{\pgfqpoint{1.377712in}{3.556889in}}%
\pgfpathlineto{\pgfqpoint{1.378786in}{3.561823in}}%
\pgfpathlineto{\pgfqpoint{1.382008in}{3.564354in}}%
\pgfpathlineto{\pgfqpoint{1.383082in}{3.563848in}}%
\pgfpathlineto{\pgfqpoint{1.384155in}{3.558154in}}%
\pgfpathlineto{\pgfqpoint{1.385229in}{3.557775in}}%
\pgfpathlineto{\pgfqpoint{1.386303in}{3.560811in}}%
\pgfpathlineto{\pgfqpoint{1.390598in}{3.562709in}}%
\pgfpathlineto{\pgfqpoint{1.391672in}{3.566505in}}%
\pgfpathlineto{\pgfqpoint{1.392746in}{3.567896in}}%
\pgfpathlineto{\pgfqpoint{1.393820in}{3.568276in}}%
\pgfpathlineto{\pgfqpoint{1.397041in}{3.568655in}}%
\pgfpathlineto{\pgfqpoint{1.398115in}{3.570806in}}%
\pgfpathlineto{\pgfqpoint{1.400263in}{3.573084in}}%
\pgfpathlineto{\pgfqpoint{1.401337in}{3.573084in}}%
\pgfpathlineto{\pgfqpoint{1.404558in}{3.573716in}}%
\pgfpathlineto{\pgfqpoint{1.405632in}{3.575108in}}%
\pgfpathlineto{\pgfqpoint{1.407780in}{3.569668in}}%
\pgfpathlineto{\pgfqpoint{1.408854in}{3.569541in}}%
\pgfpathlineto{\pgfqpoint{1.412075in}{3.567264in}}%
\pgfpathlineto{\pgfqpoint{1.413149in}{3.567517in}}%
\pgfpathlineto{\pgfqpoint{1.414223in}{3.566758in}}%
\pgfpathlineto{\pgfqpoint{1.415297in}{3.566884in}}%
\pgfpathlineto{\pgfqpoint{1.416371in}{3.564227in}}%
\pgfpathlineto{\pgfqpoint{1.419592in}{3.561823in}}%
\pgfpathlineto{\pgfqpoint{1.421740in}{3.568149in}}%
\pgfpathlineto{\pgfqpoint{1.422814in}{3.566631in}}%
\pgfpathlineto{\pgfqpoint{1.423887in}{3.560305in}}%
\pgfpathlineto{\pgfqpoint{1.428183in}{3.557395in}}%
\pgfpathlineto{\pgfqpoint{1.430330in}{3.552588in}}%
\pgfpathlineto{\pgfqpoint{1.431404in}{3.543731in}}%
\pgfpathlineto{\pgfqpoint{1.434626in}{3.544996in}}%
\pgfpathlineto{\pgfqpoint{1.435700in}{3.548919in}}%
\pgfpathlineto{\pgfqpoint{1.436773in}{3.547274in}}%
\pgfpathlineto{\pgfqpoint{1.437847in}{3.549298in}}%
\pgfpathlineto{\pgfqpoint{1.438921in}{3.545629in}}%
\pgfpathlineto{\pgfqpoint{1.442143in}{3.537658in}}%
\pgfpathlineto{\pgfqpoint{1.443217in}{3.539936in}}%
\pgfpathlineto{\pgfqpoint{1.444290in}{3.539430in}}%
\pgfpathlineto{\pgfqpoint{1.446438in}{3.546262in}}%
\pgfpathlineto{\pgfqpoint{1.449660in}{3.544743in}}%
\pgfpathlineto{\pgfqpoint{1.450733in}{3.548666in}}%
\pgfpathlineto{\pgfqpoint{1.451807in}{3.548286in}}%
\pgfpathlineto{\pgfqpoint{1.452881in}{3.548792in}}%
\pgfpathlineto{\pgfqpoint{1.453955in}{3.551828in}}%
\pgfpathlineto{\pgfqpoint{1.458250in}{3.550943in}}%
\pgfpathlineto{\pgfqpoint{1.459324in}{3.548286in}}%
\pgfpathlineto{\pgfqpoint{1.460398in}{3.547780in}}%
\pgfpathlineto{\pgfqpoint{1.461472in}{3.545882in}}%
\pgfpathlineto{\pgfqpoint{1.464693in}{3.549425in}}%
\pgfpathlineto{\pgfqpoint{1.466841in}{3.549551in}}%
\pgfpathlineto{\pgfqpoint{1.467915in}{3.551575in}}%
\pgfpathlineto{\pgfqpoint{1.468989in}{3.551322in}}%
\pgfpathlineto{\pgfqpoint{1.472210in}{3.547780in}}%
\pgfpathlineto{\pgfqpoint{1.474358in}{3.556004in}}%
\pgfpathlineto{\pgfqpoint{1.475432in}{3.559040in}}%
\pgfpathlineto{\pgfqpoint{1.476506in}{3.558154in}}%
\pgfpathlineto{\pgfqpoint{1.479727in}{3.557142in}}%
\pgfpathlineto{\pgfqpoint{1.481875in}{3.554359in}}%
\pgfpathlineto{\pgfqpoint{1.484022in}{3.547653in}}%
\pgfpathlineto{\pgfqpoint{1.487244in}{3.550943in}}%
\pgfpathlineto{\pgfqpoint{1.488318in}{3.553220in}}%
\pgfpathlineto{\pgfqpoint{1.489392in}{3.549425in}}%
\pgfpathlineto{\pgfqpoint{1.490465in}{3.549298in}}%
\pgfpathlineto{\pgfqpoint{1.491539in}{3.550563in}}%
\pgfpathlineto{\pgfqpoint{1.494761in}{3.550690in}}%
\pgfpathlineto{\pgfqpoint{1.495835in}{3.553726in}}%
\pgfpathlineto{\pgfqpoint{1.496908in}{3.552841in}}%
\pgfpathlineto{\pgfqpoint{1.497982in}{3.554865in}}%
\pgfpathlineto{\pgfqpoint{1.499056in}{3.555498in}}%
\pgfpathlineto{\pgfqpoint{1.503351in}{3.555371in}}%
\pgfpathlineto{\pgfqpoint{1.505499in}{3.559167in}}%
\pgfpathlineto{\pgfqpoint{1.506573in}{3.557016in}}%
\pgfpathlineto{\pgfqpoint{1.510868in}{3.554232in}}%
\pgfpathlineto{\pgfqpoint{1.511942in}{3.556257in}}%
\pgfpathlineto{\pgfqpoint{1.513016in}{3.552461in}}%
\pgfpathlineto{\pgfqpoint{1.514090in}{3.550943in}}%
\pgfpathlineto{\pgfqpoint{1.518385in}{3.554865in}}%
\pgfpathlineto{\pgfqpoint{1.520533in}{3.562456in}}%
\pgfpathlineto{\pgfqpoint{1.525902in}{3.562709in}}%
\pgfpathlineto{\pgfqpoint{1.526976in}{3.561064in}}%
\pgfpathlineto{\pgfqpoint{1.528050in}{3.561444in}}%
\pgfpathlineto{\pgfqpoint{1.529124in}{3.562962in}}%
\pgfpathlineto{\pgfqpoint{1.532345in}{3.564733in}}%
\pgfpathlineto{\pgfqpoint{1.533419in}{3.564607in}}%
\pgfpathlineto{\pgfqpoint{1.534493in}{3.565872in}}%
\pgfpathlineto{\pgfqpoint{1.536640in}{3.563721in}}%
\pgfpathlineto{\pgfqpoint{1.539862in}{3.562709in}}%
\pgfpathlineto{\pgfqpoint{1.540936in}{3.558660in}}%
\pgfpathlineto{\pgfqpoint{1.542010in}{3.562203in}}%
\pgfpathlineto{\pgfqpoint{1.544157in}{3.561064in}}%
\pgfpathlineto{\pgfqpoint{1.548453in}{3.566252in}}%
\pgfpathlineto{\pgfqpoint{1.550600in}{3.562962in}}%
\pgfpathlineto{\pgfqpoint{1.551674in}{3.563595in}}%
\pgfpathlineto{\pgfqpoint{1.554896in}{3.562962in}}%
\pgfpathlineto{\pgfqpoint{1.555970in}{3.559799in}}%
\pgfpathlineto{\pgfqpoint{1.557043in}{3.561823in}}%
\pgfpathlineto{\pgfqpoint{1.563486in}{3.562583in}}%
\pgfpathlineto{\pgfqpoint{1.566708in}{3.564860in}}%
\pgfpathlineto{\pgfqpoint{1.571003in}{3.564860in}}%
\pgfpathlineto{\pgfqpoint{1.572077in}{3.562456in}}%
\pgfpathlineto{\pgfqpoint{1.573151in}{3.564607in}}%
\pgfpathlineto{\pgfqpoint{1.574225in}{3.568908in}}%
\pgfpathlineto{\pgfqpoint{1.577446in}{3.571565in}}%
\pgfpathlineto{\pgfqpoint{1.578520in}{3.571186in}}%
\pgfpathlineto{\pgfqpoint{1.580668in}{3.566631in}}%
\pgfpathlineto{\pgfqpoint{1.581742in}{3.567390in}}%
\pgfpathlineto{\pgfqpoint{1.584963in}{3.565239in}}%
\pgfpathlineto{\pgfqpoint{1.587111in}{3.565872in}}%
\pgfpathlineto{\pgfqpoint{1.588185in}{3.568529in}}%
\pgfpathlineto{\pgfqpoint{1.589259in}{3.569035in}}%
\pgfpathlineto{\pgfqpoint{1.594628in}{3.563342in}}%
\pgfpathlineto{\pgfqpoint{1.595702in}{3.561950in}}%
\pgfpathlineto{\pgfqpoint{1.596775in}{3.563468in}}%
\pgfpathlineto{\pgfqpoint{1.599997in}{3.561823in}}%
\pgfpathlineto{\pgfqpoint{1.601071in}{3.563089in}}%
\pgfpathlineto{\pgfqpoint{1.603218in}{3.567896in}}%
\pgfpathlineto{\pgfqpoint{1.607514in}{3.566758in}}%
\pgfpathlineto{\pgfqpoint{1.608588in}{3.562836in}}%
\pgfpathlineto{\pgfqpoint{1.609661in}{3.562330in}}%
\pgfpathlineto{\pgfqpoint{1.610735in}{3.561064in}}%
\pgfpathlineto{\pgfqpoint{1.611809in}{3.564607in}}%
\pgfpathlineto{\pgfqpoint{1.615031in}{3.565746in}}%
\pgfpathlineto{\pgfqpoint{1.616105in}{3.565239in}}%
\pgfpathlineto{\pgfqpoint{1.617178in}{3.569541in}}%
\pgfpathlineto{\pgfqpoint{1.618252in}{3.565239in}}%
\pgfpathlineto{\pgfqpoint{1.622548in}{3.558787in}}%
\pgfpathlineto{\pgfqpoint{1.623621in}{3.559167in}}%
\pgfpathlineto{\pgfqpoint{1.624695in}{3.558028in}}%
\pgfpathlineto{\pgfqpoint{1.625769in}{3.558281in}}%
\pgfpathlineto{\pgfqpoint{1.626843in}{3.556763in}}%
\pgfpathlineto{\pgfqpoint{1.630064in}{3.554738in}}%
\pgfpathlineto{\pgfqpoint{1.631138in}{3.553220in}}%
\pgfpathlineto{\pgfqpoint{1.632212in}{3.555244in}}%
\pgfpathlineto{\pgfqpoint{1.633286in}{3.550184in}}%
\pgfpathlineto{\pgfqpoint{1.634360in}{3.552208in}}%
\pgfpathlineto{\pgfqpoint{1.637581in}{3.551449in}}%
\pgfpathlineto{\pgfqpoint{1.638655in}{3.548792in}}%
\pgfpathlineto{\pgfqpoint{1.639729in}{3.553220in}}%
\pgfpathlineto{\pgfqpoint{1.640803in}{3.553726in}}%
\pgfpathlineto{\pgfqpoint{1.641877in}{3.555498in}}%
\pgfpathlineto{\pgfqpoint{1.645098in}{3.556763in}}%
\pgfpathlineto{\pgfqpoint{1.646172in}{3.554865in}}%
\pgfpathlineto{\pgfqpoint{1.647246in}{3.557142in}}%
\pgfpathlineto{\pgfqpoint{1.648320in}{3.557775in}}%
\pgfpathlineto{\pgfqpoint{1.649394in}{3.555244in}}%
\pgfpathlineto{\pgfqpoint{1.652615in}{3.559673in}}%
\pgfpathlineto{\pgfqpoint{1.653689in}{3.559420in}}%
\pgfpathlineto{\pgfqpoint{1.654763in}{3.562709in}}%
\pgfpathlineto{\pgfqpoint{1.655837in}{3.563468in}}%
\pgfpathlineto{\pgfqpoint{1.656910in}{3.560558in}}%
\pgfpathlineto{\pgfqpoint{1.660132in}{3.561064in}}%
\pgfpathlineto{\pgfqpoint{1.661206in}{3.559040in}}%
\pgfpathlineto{\pgfqpoint{1.662280in}{3.560305in}}%
\pgfpathlineto{\pgfqpoint{1.663353in}{3.559040in}}%
\pgfpathlineto{\pgfqpoint{1.668723in}{3.557395in}}%
\pgfpathlineto{\pgfqpoint{1.669796in}{3.558407in}}%
\pgfpathlineto{\pgfqpoint{1.670870in}{3.558534in}}%
\pgfpathlineto{\pgfqpoint{1.671944in}{3.560052in}}%
\pgfpathlineto{\pgfqpoint{1.675166in}{3.559799in}}%
\pgfpathlineto{\pgfqpoint{1.676239in}{3.557901in}}%
\pgfpathlineto{\pgfqpoint{1.678387in}{3.559167in}}%
\pgfpathlineto{\pgfqpoint{1.679461in}{3.557648in}}%
\pgfpathlineto{\pgfqpoint{1.682682in}{3.558154in}}%
\pgfpathlineto{\pgfqpoint{1.683756in}{3.561191in}}%
\pgfpathlineto{\pgfqpoint{1.684830in}{3.561697in}}%
\pgfpathlineto{\pgfqpoint{1.686978in}{3.564227in}}%
\pgfpathlineto{\pgfqpoint{1.692347in}{3.560558in}}%
\pgfpathlineto{\pgfqpoint{1.693421in}{3.556510in}}%
\pgfpathlineto{\pgfqpoint{1.694495in}{3.557395in}}%
\pgfpathlineto{\pgfqpoint{1.697716in}{3.555118in}}%
\pgfpathlineto{\pgfqpoint{1.698790in}{3.557269in}}%
\pgfpathlineto{\pgfqpoint{1.699864in}{3.552461in}}%
\pgfpathlineto{\pgfqpoint{1.700938in}{3.552082in}}%
\pgfpathlineto{\pgfqpoint{1.702012in}{3.554991in}}%
\pgfpathlineto{\pgfqpoint{1.705233in}{3.553094in}}%
\pgfpathlineto{\pgfqpoint{1.706307in}{3.548792in}}%
\pgfpathlineto{\pgfqpoint{1.707381in}{3.553347in}}%
\pgfpathlineto{\pgfqpoint{1.709528in}{3.543225in}}%
\pgfpathlineto{\pgfqpoint{1.712750in}{3.539809in}}%
\pgfpathlineto{\pgfqpoint{1.714898in}{3.543225in}}%
\pgfpathlineto{\pgfqpoint{1.715971in}{3.542972in}}%
\pgfpathlineto{\pgfqpoint{1.717045in}{3.548919in}}%
\pgfpathlineto{\pgfqpoint{1.720267in}{3.551069in}}%
\pgfpathlineto{\pgfqpoint{1.721341in}{3.555498in}}%
\pgfpathlineto{\pgfqpoint{1.722415in}{3.552841in}}%
\pgfpathlineto{\pgfqpoint{1.724562in}{3.557522in}}%
\pgfpathlineto{\pgfqpoint{1.727784in}{3.556257in}}%
\pgfpathlineto{\pgfqpoint{1.728858in}{3.559926in}}%
\pgfpathlineto{\pgfqpoint{1.729931in}{3.557648in}}%
\pgfpathlineto{\pgfqpoint{1.731005in}{3.557775in}}%
\pgfpathlineto{\pgfqpoint{1.732079in}{3.559293in}}%
\pgfpathlineto{\pgfqpoint{1.736374in}{3.558154in}}%
\pgfpathlineto{\pgfqpoint{1.737448in}{3.559420in}}%
\pgfpathlineto{\pgfqpoint{1.738522in}{3.564986in}}%
\pgfpathlineto{\pgfqpoint{1.739596in}{3.565492in}}%
\pgfpathlineto{\pgfqpoint{1.742817in}{3.566125in}}%
\pgfpathlineto{\pgfqpoint{1.743891in}{3.565239in}}%
\pgfpathlineto{\pgfqpoint{1.744965in}{3.566631in}}%
\pgfpathlineto{\pgfqpoint{1.746039in}{3.565619in}}%
\pgfpathlineto{\pgfqpoint{1.750334in}{3.567643in}}%
\pgfpathlineto{\pgfqpoint{1.751408in}{3.571818in}}%
\pgfpathlineto{\pgfqpoint{1.753556in}{3.570047in}}%
\pgfpathlineto{\pgfqpoint{1.754630in}{3.571565in}}%
\pgfpathlineto{\pgfqpoint{1.757851in}{3.571692in}}%
\pgfpathlineto{\pgfqpoint{1.758925in}{3.570174in}}%
\pgfpathlineto{\pgfqpoint{1.759999in}{3.570300in}}%
\pgfpathlineto{\pgfqpoint{1.765368in}{3.561444in}}%
\pgfpathlineto{\pgfqpoint{1.766442in}{3.561697in}}%
\pgfpathlineto{\pgfqpoint{1.767516in}{3.565239in}}%
\pgfpathlineto{\pgfqpoint{1.768590in}{3.562203in}}%
\pgfpathlineto{\pgfqpoint{1.772885in}{3.558028in}}%
\pgfpathlineto{\pgfqpoint{1.773959in}{3.556889in}}%
\pgfpathlineto{\pgfqpoint{1.775033in}{3.553600in}}%
\pgfpathlineto{\pgfqpoint{1.776106in}{3.555118in}}%
\pgfpathlineto{\pgfqpoint{1.777180in}{3.549678in}}%
\pgfpathlineto{\pgfqpoint{1.781476in}{3.545503in}}%
\pgfpathlineto{\pgfqpoint{1.782549in}{3.547274in}}%
\pgfpathlineto{\pgfqpoint{1.784697in}{3.559799in}}%
\pgfpathlineto{\pgfqpoint{1.787919in}{3.560685in}}%
\pgfpathlineto{\pgfqpoint{1.788993in}{3.562583in}}%
\pgfpathlineto{\pgfqpoint{1.790066in}{3.561950in}}%
\pgfpathlineto{\pgfqpoint{1.795436in}{3.560685in}}%
\pgfpathlineto{\pgfqpoint{1.796509in}{3.559293in}}%
\pgfpathlineto{\pgfqpoint{1.797583in}{3.556130in}}%
\pgfpathlineto{\pgfqpoint{1.799731in}{3.553853in}}%
\pgfpathlineto{\pgfqpoint{1.802952in}{3.549045in}}%
\pgfpathlineto{\pgfqpoint{1.804026in}{3.543478in}}%
\pgfpathlineto{\pgfqpoint{1.805100in}{3.543605in}}%
\pgfpathlineto{\pgfqpoint{1.806174in}{3.546641in}}%
\pgfpathlineto{\pgfqpoint{1.807248in}{3.542972in}}%
\pgfpathlineto{\pgfqpoint{1.810469in}{3.542466in}}%
\pgfpathlineto{\pgfqpoint{1.813691in}{3.538291in}}%
\pgfpathlineto{\pgfqpoint{1.814765in}{3.538418in}}%
\pgfpathlineto{\pgfqpoint{1.819060in}{3.541074in}}%
\pgfpathlineto{\pgfqpoint{1.822282in}{3.547780in}}%
\pgfpathlineto{\pgfqpoint{1.825503in}{3.548919in}}%
\pgfpathlineto{\pgfqpoint{1.826577in}{3.546768in}}%
\pgfpathlineto{\pgfqpoint{1.827651in}{3.541074in}}%
\pgfpathlineto{\pgfqpoint{1.828725in}{3.543605in}}%
\pgfpathlineto{\pgfqpoint{1.829798in}{3.541581in}}%
\pgfpathlineto{\pgfqpoint{1.833020in}{3.544870in}}%
\pgfpathlineto{\pgfqpoint{1.834094in}{3.547653in}}%
\pgfpathlineto{\pgfqpoint{1.835168in}{3.544364in}}%
\pgfpathlineto{\pgfqpoint{1.836241in}{3.548033in}}%
\pgfpathlineto{\pgfqpoint{1.837315in}{3.548159in}}%
\pgfpathlineto{\pgfqpoint{1.842684in}{3.550816in}}%
\pgfpathlineto{\pgfqpoint{1.843758in}{3.552082in}}%
\pgfpathlineto{\pgfqpoint{1.844832in}{3.554865in}}%
\pgfpathlineto{\pgfqpoint{1.849127in}{3.554991in}}%
\pgfpathlineto{\pgfqpoint{1.850201in}{3.555877in}}%
\pgfpathlineto{\pgfqpoint{1.851275in}{3.555751in}}%
\pgfpathlineto{\pgfqpoint{1.852349in}{3.557901in}}%
\pgfpathlineto{\pgfqpoint{1.855570in}{3.557522in}}%
\pgfpathlineto{\pgfqpoint{1.856644in}{3.559799in}}%
\pgfpathlineto{\pgfqpoint{1.857718in}{3.565366in}}%
\pgfpathlineto{\pgfqpoint{1.858792in}{3.565113in}}%
\pgfpathlineto{\pgfqpoint{1.859866in}{3.566252in}}%
\pgfpathlineto{\pgfqpoint{1.863087in}{3.567517in}}%
\pgfpathlineto{\pgfqpoint{1.865235in}{3.562709in}}%
\pgfpathlineto{\pgfqpoint{1.866309in}{3.564354in}}%
\pgfpathlineto{\pgfqpoint{1.867383in}{3.560179in}}%
\pgfpathlineto{\pgfqpoint{1.870604in}{3.562456in}}%
\pgfpathlineto{\pgfqpoint{1.871678in}{3.557522in}}%
\pgfpathlineto{\pgfqpoint{1.872752in}{3.557648in}}%
\pgfpathlineto{\pgfqpoint{1.873826in}{3.559926in}}%
\pgfpathlineto{\pgfqpoint{1.874900in}{3.556130in}}%
\pgfpathlineto{\pgfqpoint{1.878121in}{3.560432in}}%
\pgfpathlineto{\pgfqpoint{1.879195in}{3.558914in}}%
\pgfpathlineto{\pgfqpoint{1.880269in}{3.562456in}}%
\pgfpathlineto{\pgfqpoint{1.881343in}{3.559167in}}%
\pgfpathlineto{\pgfqpoint{1.882416in}{3.559926in}}%
\pgfpathlineto{\pgfqpoint{1.885638in}{3.560685in}}%
\pgfpathlineto{\pgfqpoint{1.886712in}{3.558534in}}%
\pgfpathlineto{\pgfqpoint{1.887786in}{3.554738in}}%
\pgfpathlineto{\pgfqpoint{1.888859in}{3.553600in}}%
\pgfpathlineto{\pgfqpoint{1.889933in}{3.554232in}}%
\pgfpathlineto{\pgfqpoint{1.893155in}{3.556889in}}%
\pgfpathlineto{\pgfqpoint{1.894229in}{3.553600in}}%
\pgfpathlineto{\pgfqpoint{1.895303in}{3.553979in}}%
\pgfpathlineto{\pgfqpoint{1.896376in}{3.554991in}}%
\pgfpathlineto{\pgfqpoint{1.900672in}{3.557648in}}%
\pgfpathlineto{\pgfqpoint{1.901746in}{3.555877in}}%
\pgfpathlineto{\pgfqpoint{1.902819in}{3.555751in}}%
\pgfpathlineto{\pgfqpoint{1.903893in}{3.563468in}}%
\pgfpathlineto{\pgfqpoint{1.904967in}{3.592947in}}%
\pgfpathlineto{\pgfqpoint{1.908189in}{3.583585in}}%
\pgfpathlineto{\pgfqpoint{1.909262in}{3.584723in}}%
\pgfpathlineto{\pgfqpoint{1.911410in}{3.579916in}}%
\pgfpathlineto{\pgfqpoint{1.912484in}{3.579663in}}%
\pgfpathlineto{\pgfqpoint{1.915705in}{3.577132in}}%
\pgfpathlineto{\pgfqpoint{1.916779in}{3.572957in}}%
\pgfpathlineto{\pgfqpoint{1.917853in}{3.575993in}}%
\pgfpathlineto{\pgfqpoint{1.920001in}{3.574855in}}%
\pgfpathlineto{\pgfqpoint{1.923222in}{3.575614in}}%
\pgfpathlineto{\pgfqpoint{1.924296in}{3.578271in}}%
\pgfpathlineto{\pgfqpoint{1.926444in}{3.577765in}}%
\pgfpathlineto{\pgfqpoint{1.927518in}{3.580295in}}%
\pgfpathlineto{\pgfqpoint{1.930739in}{3.579789in}}%
\pgfpathlineto{\pgfqpoint{1.931813in}{3.576120in}}%
\pgfpathlineto{\pgfqpoint{1.932887in}{3.574981in}}%
\pgfpathlineto{\pgfqpoint{1.935035in}{3.580801in}}%
\pgfpathlineto{\pgfqpoint{1.938256in}{3.576120in}}%
\pgfpathlineto{\pgfqpoint{1.940404in}{3.579156in}}%
\pgfpathlineto{\pgfqpoint{1.941478in}{3.581307in}}%
\pgfpathlineto{\pgfqpoint{1.942551in}{3.579789in}}%
\pgfpathlineto{\pgfqpoint{1.946847in}{3.580675in}}%
\pgfpathlineto{\pgfqpoint{1.947921in}{3.583711in}}%
\pgfpathlineto{\pgfqpoint{1.948994in}{3.584597in}}%
\pgfpathlineto{\pgfqpoint{1.955437in}{3.582446in}}%
\pgfpathlineto{\pgfqpoint{1.956511in}{3.583585in}}%
\pgfpathlineto{\pgfqpoint{1.957585in}{3.579789in}}%
\pgfpathlineto{\pgfqpoint{1.961881in}{3.580422in}}%
\pgfpathlineto{\pgfqpoint{1.962954in}{3.582572in}}%
\pgfpathlineto{\pgfqpoint{1.964028in}{3.579663in}}%
\pgfpathlineto{\pgfqpoint{1.965102in}{3.580042in}}%
\pgfpathlineto{\pgfqpoint{1.968324in}{3.579536in}}%
\pgfpathlineto{\pgfqpoint{1.969397in}{3.580422in}}%
\pgfpathlineto{\pgfqpoint{1.970471in}{3.583585in}}%
\pgfpathlineto{\pgfqpoint{1.972619in}{3.581054in}}%
\pgfpathlineto{\pgfqpoint{1.975840in}{3.579156in}}%
\pgfpathlineto{\pgfqpoint{1.977988in}{3.579789in}}%
\pgfpathlineto{\pgfqpoint{1.979062in}{3.583332in}}%
\pgfpathlineto{\pgfqpoint{1.980136in}{3.581940in}}%
\pgfpathlineto{\pgfqpoint{1.983357in}{3.583838in}}%
\pgfpathlineto{\pgfqpoint{1.984431in}{3.585229in}}%
\pgfpathlineto{\pgfqpoint{1.986579in}{3.579789in}}%
\pgfpathlineto{\pgfqpoint{1.987653in}{3.580295in}}%
\pgfpathlineto{\pgfqpoint{1.991948in}{3.574728in}}%
\pgfpathlineto{\pgfqpoint{1.994096in}{3.577006in}}%
\pgfpathlineto{\pgfqpoint{1.998391in}{3.571945in}}%
\pgfpathlineto{\pgfqpoint{1.999465in}{3.573716in}}%
\pgfpathlineto{\pgfqpoint{2.000539in}{3.567517in}}%
\pgfpathlineto{\pgfqpoint{2.001613in}{3.568908in}}%
\pgfpathlineto{\pgfqpoint{2.002686in}{3.571565in}}%
\pgfpathlineto{\pgfqpoint{2.005908in}{3.573716in}}%
\pgfpathlineto{\pgfqpoint{2.010203in}{3.581940in}}%
\pgfpathlineto{\pgfqpoint{2.013425in}{3.580928in}}%
\pgfpathlineto{\pgfqpoint{2.016646in}{3.571439in}}%
\pgfpathlineto{\pgfqpoint{2.017720in}{3.565999in}}%
\pgfpathlineto{\pgfqpoint{2.020942in}{3.568149in}}%
\pgfpathlineto{\pgfqpoint{2.023089in}{3.571439in}}%
\pgfpathlineto{\pgfqpoint{2.024163in}{3.569921in}}%
\pgfpathlineto{\pgfqpoint{2.025237in}{3.569794in}}%
\pgfpathlineto{\pgfqpoint{2.028459in}{3.567264in}}%
\pgfpathlineto{\pgfqpoint{2.029532in}{3.567643in}}%
\pgfpathlineto{\pgfqpoint{2.030606in}{3.569794in}}%
\pgfpathlineto{\pgfqpoint{2.031680in}{3.569035in}}%
\pgfpathlineto{\pgfqpoint{2.032754in}{3.566378in}}%
\pgfpathlineto{\pgfqpoint{2.035975in}{3.571186in}}%
\pgfpathlineto{\pgfqpoint{2.037049in}{3.565492in}}%
\pgfpathlineto{\pgfqpoint{2.038123in}{3.567137in}}%
\pgfpathlineto{\pgfqpoint{2.039197in}{3.566378in}}%
\pgfpathlineto{\pgfqpoint{2.040271in}{3.569541in}}%
\pgfpathlineto{\pgfqpoint{2.043492in}{3.570933in}}%
\pgfpathlineto{\pgfqpoint{2.044566in}{3.569415in}}%
\pgfpathlineto{\pgfqpoint{2.045640in}{3.565746in}}%
\pgfpathlineto{\pgfqpoint{2.047788in}{3.553600in}}%
\pgfpathlineto{\pgfqpoint{2.051009in}{3.545756in}}%
\pgfpathlineto{\pgfqpoint{2.052083in}{3.539430in}}%
\pgfpathlineto{\pgfqpoint{2.055304in}{3.559673in}}%
\pgfpathlineto{\pgfqpoint{2.058526in}{3.556004in}}%
\pgfpathlineto{\pgfqpoint{2.059600in}{3.545882in}}%
\pgfpathlineto{\pgfqpoint{2.060674in}{3.553347in}}%
\pgfpathlineto{\pgfqpoint{2.061747in}{3.552714in}}%
\pgfpathlineto{\pgfqpoint{2.062821in}{3.547274in}}%
\pgfpathlineto{\pgfqpoint{2.067117in}{3.557522in}}%
\pgfpathlineto{\pgfqpoint{2.068191in}{3.553094in}}%
\pgfpathlineto{\pgfqpoint{2.069264in}{3.554485in}}%
\pgfpathlineto{\pgfqpoint{2.070338in}{3.557395in}}%
\pgfpathlineto{\pgfqpoint{2.073560in}{3.555498in}}%
\pgfpathlineto{\pgfqpoint{2.075707in}{3.567896in}}%
\pgfpathlineto{\pgfqpoint{2.076781in}{3.564101in}}%
\pgfpathlineto{\pgfqpoint{2.077855in}{3.558154in}}%
\pgfpathlineto{\pgfqpoint{2.081077in}{3.561317in}}%
\pgfpathlineto{\pgfqpoint{2.083224in}{3.561823in}}%
\pgfpathlineto{\pgfqpoint{2.084298in}{3.559420in}}%
\pgfpathlineto{\pgfqpoint{2.085372in}{3.559420in}}%
\pgfpathlineto{\pgfqpoint{2.088593in}{3.552841in}}%
\pgfpathlineto{\pgfqpoint{2.089667in}{3.555624in}}%
\pgfpathlineto{\pgfqpoint{2.090741in}{3.562709in}}%
\pgfpathlineto{\pgfqpoint{2.091815in}{3.562330in}}%
\pgfpathlineto{\pgfqpoint{2.092889in}{3.565366in}}%
\pgfpathlineto{\pgfqpoint{2.099332in}{3.593073in}}%
\pgfpathlineto{\pgfqpoint{2.100406in}{3.593580in}}%
\pgfpathlineto{\pgfqpoint{2.103627in}{3.593833in}}%
\pgfpathlineto{\pgfqpoint{2.105775in}{3.588519in}}%
\pgfpathlineto{\pgfqpoint{2.106849in}{3.593073in}}%
\pgfpathlineto{\pgfqpoint{2.107923in}{3.603448in}}%
\pgfpathlineto{\pgfqpoint{2.111144in}{3.603574in}}%
\pgfpathlineto{\pgfqpoint{2.112218in}{3.601297in}}%
\pgfpathlineto{\pgfqpoint{2.113292in}{3.602056in}}%
\pgfpathlineto{\pgfqpoint{2.114366in}{3.609900in}}%
\pgfpathlineto{\pgfqpoint{2.115439in}{3.609141in}}%
\pgfpathlineto{\pgfqpoint{2.118661in}{3.609521in}}%
\pgfpathlineto{\pgfqpoint{2.121882in}{3.607244in}}%
\pgfpathlineto{\pgfqpoint{2.122956in}{3.602815in}}%
\pgfpathlineto{\pgfqpoint{2.127252in}{3.610027in}}%
\pgfpathlineto{\pgfqpoint{2.128325in}{3.609521in}}%
\pgfpathlineto{\pgfqpoint{2.129399in}{3.610533in}}%
\pgfpathlineto{\pgfqpoint{2.130473in}{3.613569in}}%
\pgfpathlineto{\pgfqpoint{2.133695in}{3.611798in}}%
\pgfpathlineto{\pgfqpoint{2.135842in}{3.621667in}}%
\pgfpathlineto{\pgfqpoint{2.136916in}{3.616226in}}%
\pgfpathlineto{\pgfqpoint{2.137990in}{3.617492in}}%
\pgfpathlineto{\pgfqpoint{2.141212in}{3.618377in}}%
\pgfpathlineto{\pgfqpoint{2.142285in}{3.617871in}}%
\pgfpathlineto{\pgfqpoint{2.143359in}{3.620022in}}%
\pgfpathlineto{\pgfqpoint{2.144433in}{3.617365in}}%
\pgfpathlineto{\pgfqpoint{2.145507in}{3.621540in}}%
\pgfpathlineto{\pgfqpoint{2.148728in}{3.620781in}}%
\pgfpathlineto{\pgfqpoint{2.149802in}{3.621540in}}%
\pgfpathlineto{\pgfqpoint{2.150876in}{3.618377in}}%
\pgfpathlineto{\pgfqpoint{2.153024in}{3.618377in}}%
\pgfpathlineto{\pgfqpoint{2.156245in}{3.613822in}}%
\pgfpathlineto{\pgfqpoint{2.157319in}{3.616226in}}%
\pgfpathlineto{\pgfqpoint{2.158393in}{3.614076in}}%
\pgfpathlineto{\pgfqpoint{2.159467in}{3.614708in}}%
\pgfpathlineto{\pgfqpoint{2.160541in}{3.619769in}}%
\pgfpathlineto{\pgfqpoint{2.163762in}{3.618504in}}%
\pgfpathlineto{\pgfqpoint{2.164836in}{3.616479in}}%
\pgfpathlineto{\pgfqpoint{2.166984in}{3.621414in}}%
\pgfpathlineto{\pgfqpoint{2.168058in}{3.617238in}}%
\pgfpathlineto{\pgfqpoint{2.171279in}{3.617238in}}%
\pgfpathlineto{\pgfqpoint{2.172353in}{3.617871in}}%
\pgfpathlineto{\pgfqpoint{2.173427in}{3.625083in}}%
\pgfpathlineto{\pgfqpoint{2.175574in}{3.619895in}}%
\pgfpathlineto{\pgfqpoint{2.179870in}{3.622173in}}%
\pgfpathlineto{\pgfqpoint{2.180944in}{3.627233in}}%
\pgfpathlineto{\pgfqpoint{2.182017in}{3.625968in}}%
\pgfpathlineto{\pgfqpoint{2.186313in}{3.626727in}}%
\pgfpathlineto{\pgfqpoint{2.187387in}{3.630776in}}%
\pgfpathlineto{\pgfqpoint{2.188460in}{3.628372in}}%
\pgfpathlineto{\pgfqpoint{2.189534in}{3.629384in}}%
\pgfpathlineto{\pgfqpoint{2.193830in}{3.624577in}}%
\pgfpathlineto{\pgfqpoint{2.194903in}{3.624956in}}%
\pgfpathlineto{\pgfqpoint{2.195977in}{3.619642in}}%
\pgfpathlineto{\pgfqpoint{2.197051in}{3.605599in}}%
\pgfpathlineto{\pgfqpoint{2.198125in}{3.600032in}}%
\pgfpathlineto{\pgfqpoint{2.202420in}{3.602056in}}%
\pgfpathlineto{\pgfqpoint{2.203494in}{3.597628in}}%
\pgfpathlineto{\pgfqpoint{2.204568in}{3.606611in}}%
\pgfpathlineto{\pgfqpoint{2.205642in}{3.600412in}}%
\pgfpathlineto{\pgfqpoint{2.209937in}{3.600412in}}%
\pgfpathlineto{\pgfqpoint{2.211011in}{3.595098in}}%
\pgfpathlineto{\pgfqpoint{2.212085in}{3.601550in}}%
\pgfpathlineto{\pgfqpoint{2.213159in}{3.597628in}}%
\pgfpathlineto{\pgfqpoint{2.216380in}{3.595477in}}%
\pgfpathlineto{\pgfqpoint{2.217454in}{3.598387in}}%
\pgfpathlineto{\pgfqpoint{2.218528in}{3.595098in}}%
\pgfpathlineto{\pgfqpoint{2.219602in}{3.597375in}}%
\pgfpathlineto{\pgfqpoint{2.220676in}{3.607117in}}%
\pgfpathlineto{\pgfqpoint{2.223897in}{3.602056in}}%
\pgfpathlineto{\pgfqpoint{2.224971in}{3.597628in}}%
\pgfpathlineto{\pgfqpoint{2.227119in}{3.607876in}}%
\pgfpathlineto{\pgfqpoint{2.228192in}{3.600918in}}%
\pgfpathlineto{\pgfqpoint{2.231414in}{3.596869in}}%
\pgfpathlineto{\pgfqpoint{2.232488in}{3.598134in}}%
\pgfpathlineto{\pgfqpoint{2.233562in}{3.598387in}}%
\pgfpathlineto{\pgfqpoint{2.234635in}{3.589025in}}%
\pgfpathlineto{\pgfqpoint{2.235709in}{3.597881in}}%
\pgfpathlineto{\pgfqpoint{2.240005in}{3.604460in}}%
\pgfpathlineto{\pgfqpoint{2.241079in}{3.609647in}}%
\pgfpathlineto{\pgfqpoint{2.242152in}{3.606864in}}%
\pgfpathlineto{\pgfqpoint{2.243226in}{3.606231in}}%
\pgfpathlineto{\pgfqpoint{2.246448in}{3.610406in}}%
\pgfpathlineto{\pgfqpoint{2.248595in}{3.605472in}}%
\pgfpathlineto{\pgfqpoint{2.249669in}{3.611039in}}%
\pgfpathlineto{\pgfqpoint{2.250743in}{3.612937in}}%
\pgfpathlineto{\pgfqpoint{2.253965in}{3.610027in}}%
\pgfpathlineto{\pgfqpoint{2.255038in}{3.618124in}}%
\pgfpathlineto{\pgfqpoint{2.256112in}{3.621414in}}%
\pgfpathlineto{\pgfqpoint{2.257186in}{3.621920in}}%
\pgfpathlineto{\pgfqpoint{2.258260in}{3.624577in}}%
\pgfpathlineto{\pgfqpoint{2.261481in}{3.622679in}}%
\pgfpathlineto{\pgfqpoint{2.262555in}{3.620148in}}%
\pgfpathlineto{\pgfqpoint{2.263629in}{3.620022in}}%
\pgfpathlineto{\pgfqpoint{2.264703in}{3.618883in}}%
\pgfpathlineto{\pgfqpoint{2.265777in}{3.623185in}}%
\pgfpathlineto{\pgfqpoint{2.271146in}{3.621414in}}%
\pgfpathlineto{\pgfqpoint{2.272220in}{3.630017in}}%
\pgfpathlineto{\pgfqpoint{2.273294in}{3.629637in}}%
\pgfpathlineto{\pgfqpoint{2.276515in}{3.631409in}}%
\pgfpathlineto{\pgfqpoint{2.278663in}{3.631282in}}%
\pgfpathlineto{\pgfqpoint{2.279737in}{3.631662in}}%
\pgfpathlineto{\pgfqpoint{2.284032in}{3.635837in}}%
\pgfpathlineto{\pgfqpoint{2.285106in}{3.635710in}}%
\pgfpathlineto{\pgfqpoint{2.286180in}{3.639632in}}%
\pgfpathlineto{\pgfqpoint{2.287254in}{3.639126in}}%
\pgfpathlineto{\pgfqpoint{2.288327in}{3.640644in}}%
\pgfpathlineto{\pgfqpoint{2.294770in}{3.626474in}}%
\pgfpathlineto{\pgfqpoint{2.295844in}{3.628119in}}%
\pgfpathlineto{\pgfqpoint{2.299066in}{3.627233in}}%
\pgfpathlineto{\pgfqpoint{2.302287in}{3.630649in}}%
\pgfpathlineto{\pgfqpoint{2.306583in}{3.631155in}}%
\pgfpathlineto{\pgfqpoint{2.307657in}{3.632168in}}%
\pgfpathlineto{\pgfqpoint{2.308730in}{3.632168in}}%
\pgfpathlineto{\pgfqpoint{2.310878in}{3.627866in}}%
\pgfpathlineto{\pgfqpoint{2.314100in}{3.626980in}}%
\pgfpathlineto{\pgfqpoint{2.315173in}{3.629384in}}%
\pgfpathlineto{\pgfqpoint{2.316247in}{3.629764in}}%
\pgfpathlineto{\pgfqpoint{2.317321in}{3.629384in}}%
\pgfpathlineto{\pgfqpoint{2.318395in}{3.627739in}}%
\pgfpathlineto{\pgfqpoint{2.321616in}{3.629258in}}%
\pgfpathlineto{\pgfqpoint{2.322690in}{3.626474in}}%
\pgfpathlineto{\pgfqpoint{2.323764in}{3.620275in}}%
\pgfpathlineto{\pgfqpoint{2.324838in}{3.618251in}}%
\pgfpathlineto{\pgfqpoint{2.325912in}{3.620781in}}%
\pgfpathlineto{\pgfqpoint{2.329133in}{3.617998in}}%
\pgfpathlineto{\pgfqpoint{2.330207in}{3.624703in}}%
\pgfpathlineto{\pgfqpoint{2.331281in}{3.623185in}}%
\pgfpathlineto{\pgfqpoint{2.332355in}{3.620528in}}%
\pgfpathlineto{\pgfqpoint{2.333429in}{3.615594in}}%
\pgfpathlineto{\pgfqpoint{2.336650in}{3.619010in}}%
\pgfpathlineto{\pgfqpoint{2.337724in}{3.616353in}}%
\pgfpathlineto{\pgfqpoint{2.338798in}{3.615214in}}%
\pgfpathlineto{\pgfqpoint{2.339872in}{3.612431in}}%
\pgfpathlineto{\pgfqpoint{2.340946in}{3.614708in}}%
\pgfpathlineto{\pgfqpoint{2.344167in}{3.613822in}}%
\pgfpathlineto{\pgfqpoint{2.346315in}{3.620528in}}%
\pgfpathlineto{\pgfqpoint{2.347389in}{3.619769in}}%
\pgfpathlineto{\pgfqpoint{2.348462in}{3.620781in}}%
\pgfpathlineto{\pgfqpoint{2.352758in}{3.622046in}}%
\pgfpathlineto{\pgfqpoint{2.354905in}{3.620022in}}%
\pgfpathlineto{\pgfqpoint{2.355979in}{3.618883in}}%
\pgfpathlineto{\pgfqpoint{2.360275in}{3.621034in}}%
\pgfpathlineto{\pgfqpoint{2.361348in}{3.622932in}}%
\pgfpathlineto{\pgfqpoint{2.362422in}{3.622173in}}%
\pgfpathlineto{\pgfqpoint{2.363496in}{3.619895in}}%
\pgfpathlineto{\pgfqpoint{2.366718in}{3.617618in}}%
\pgfpathlineto{\pgfqpoint{2.367791in}{3.624324in}}%
\pgfpathlineto{\pgfqpoint{2.368865in}{3.625968in}}%
\pgfpathlineto{\pgfqpoint{2.369939in}{3.629131in}}%
\pgfpathlineto{\pgfqpoint{2.371013in}{3.628625in}}%
\pgfpathlineto{\pgfqpoint{2.375308in}{3.632421in}}%
\pgfpathlineto{\pgfqpoint{2.376382in}{3.630649in}}%
\pgfpathlineto{\pgfqpoint{2.377456in}{3.635204in}}%
\pgfpathlineto{\pgfqpoint{2.378530in}{3.620022in}}%
\pgfpathlineto{\pgfqpoint{2.381751in}{3.614455in}}%
\pgfpathlineto{\pgfqpoint{2.386047in}{3.638494in}}%
\pgfpathlineto{\pgfqpoint{2.390342in}{3.637987in}}%
\pgfpathlineto{\pgfqpoint{2.391416in}{3.641277in}}%
\pgfpathlineto{\pgfqpoint{2.392490in}{3.642163in}}%
\pgfpathlineto{\pgfqpoint{2.393564in}{3.646338in}}%
\pgfpathlineto{\pgfqpoint{2.397859in}{3.646970in}}%
\pgfpathlineto{\pgfqpoint{2.398933in}{3.648109in}}%
\pgfpathlineto{\pgfqpoint{2.401080in}{3.653929in}}%
\pgfpathlineto{\pgfqpoint{2.405376in}{3.654435in}}%
\pgfpathlineto{\pgfqpoint{2.407524in}{3.650639in}}%
\pgfpathlineto{\pgfqpoint{2.408597in}{3.644819in}}%
\pgfpathlineto{\pgfqpoint{2.416114in}{3.634571in}}%
\pgfpathlineto{\pgfqpoint{2.419336in}{3.634698in}}%
\pgfpathlineto{\pgfqpoint{2.420410in}{3.633559in}}%
\pgfpathlineto{\pgfqpoint{2.423631in}{3.636216in}}%
\pgfpathlineto{\pgfqpoint{2.436517in}{3.636216in}}%
\pgfpathlineto{\pgfqpoint{2.437591in}{3.637861in}}%
\pgfpathlineto{\pgfqpoint{2.438665in}{3.635837in}}%
\pgfpathlineto{\pgfqpoint{2.441886in}{3.636596in}}%
\pgfpathlineto{\pgfqpoint{2.442960in}{3.635584in}}%
\pgfpathlineto{\pgfqpoint{2.446182in}{3.635584in}}%
\pgfpathlineto{\pgfqpoint{2.450477in}{3.637102in}}%
\pgfpathlineto{\pgfqpoint{2.451551in}{3.635710in}}%
\pgfpathlineto{\pgfqpoint{2.452625in}{3.635331in}}%
\pgfpathlineto{\pgfqpoint{2.453699in}{3.636216in}}%
\pgfpathlineto{\pgfqpoint{2.457994in}{3.633559in}}%
\pgfpathlineto{\pgfqpoint{2.460142in}{3.633559in}}%
\pgfpathlineto{\pgfqpoint{2.461215in}{3.623185in}}%
\pgfpathlineto{\pgfqpoint{2.464437in}{3.627360in}}%
\pgfpathlineto{\pgfqpoint{2.465511in}{3.620275in}}%
\pgfpathlineto{\pgfqpoint{2.466585in}{3.618630in}}%
\pgfpathlineto{\pgfqpoint{2.467658in}{3.621793in}}%
\pgfpathlineto{\pgfqpoint{2.471954in}{3.618251in}}%
\pgfpathlineto{\pgfqpoint{2.475175in}{3.625083in}}%
\pgfpathlineto{\pgfqpoint{2.476249in}{3.623311in}}%
\pgfpathlineto{\pgfqpoint{2.479471in}{3.619389in}}%
\pgfpathlineto{\pgfqpoint{2.480545in}{3.623185in}}%
\pgfpathlineto{\pgfqpoint{2.481618in}{3.623438in}}%
\pgfpathlineto{\pgfqpoint{2.482692in}{3.619389in}}%
\pgfpathlineto{\pgfqpoint{2.483766in}{3.620275in}}%
\pgfpathlineto{\pgfqpoint{2.486988in}{3.620528in}}%
\pgfpathlineto{\pgfqpoint{2.488061in}{3.619010in}}%
\pgfpathlineto{\pgfqpoint{2.489135in}{3.619010in}}%
\pgfpathlineto{\pgfqpoint{2.491283in}{3.614329in}}%
\pgfpathlineto{\pgfqpoint{2.494504in}{3.611798in}}%
\pgfpathlineto{\pgfqpoint{2.495578in}{3.612557in}}%
\pgfpathlineto{\pgfqpoint{2.496652in}{3.612304in}}%
\pgfpathlineto{\pgfqpoint{2.497726in}{3.610913in}}%
\pgfpathlineto{\pgfqpoint{2.498800in}{3.612178in}}%
\pgfpathlineto{\pgfqpoint{2.502021in}{3.611798in}}%
\pgfpathlineto{\pgfqpoint{2.504169in}{3.614076in}}%
\pgfpathlineto{\pgfqpoint{2.505243in}{3.614202in}}%
\pgfpathlineto{\pgfqpoint{2.506317in}{3.613190in}}%
\pgfpathlineto{\pgfqpoint{2.509538in}{3.612557in}}%
\pgfpathlineto{\pgfqpoint{2.510612in}{3.609521in}}%
\pgfpathlineto{\pgfqpoint{2.511686in}{3.611925in}}%
\pgfpathlineto{\pgfqpoint{2.512760in}{3.609268in}}%
\pgfpathlineto{\pgfqpoint{2.513834in}{3.615847in}}%
\pgfpathlineto{\pgfqpoint{2.517055in}{3.614582in}}%
\pgfpathlineto{\pgfqpoint{2.518129in}{3.612051in}}%
\pgfpathlineto{\pgfqpoint{2.520277in}{3.605346in}}%
\pgfpathlineto{\pgfqpoint{2.521350in}{3.607117in}}%
\pgfpathlineto{\pgfqpoint{2.524572in}{3.616859in}}%
\pgfpathlineto{\pgfqpoint{2.525646in}{3.618124in}}%
\pgfpathlineto{\pgfqpoint{2.526720in}{3.620401in}}%
\pgfpathlineto{\pgfqpoint{2.527793in}{3.629131in}}%
\pgfpathlineto{\pgfqpoint{2.528867in}{3.632547in}}%
\pgfpathlineto{\pgfqpoint{2.532089in}{3.630270in}}%
\pgfpathlineto{\pgfqpoint{2.533163in}{3.632927in}}%
\pgfpathlineto{\pgfqpoint{2.534236in}{3.632800in}}%
\pgfpathlineto{\pgfqpoint{2.535310in}{3.633433in}}%
\pgfpathlineto{\pgfqpoint{2.536384in}{3.632041in}}%
\pgfpathlineto{\pgfqpoint{2.539606in}{3.634318in}}%
\pgfpathlineto{\pgfqpoint{2.541753in}{3.639506in}}%
\pgfpathlineto{\pgfqpoint{2.543901in}{3.640644in}}%
\pgfpathlineto{\pgfqpoint{2.547123in}{3.638494in}}%
\pgfpathlineto{\pgfqpoint{2.549270in}{3.633053in}}%
\pgfpathlineto{\pgfqpoint{2.550344in}{3.640138in}}%
\pgfpathlineto{\pgfqpoint{2.555713in}{3.637608in}}%
\pgfpathlineto{\pgfqpoint{2.556787in}{3.642416in}}%
\pgfpathlineto{\pgfqpoint{2.557861in}{3.641657in}}%
\pgfpathlineto{\pgfqpoint{2.558935in}{3.644440in}}%
\pgfpathlineto{\pgfqpoint{2.562156in}{3.645326in}}%
\pgfpathlineto{\pgfqpoint{2.563230in}{3.643934in}}%
\pgfpathlineto{\pgfqpoint{2.565378in}{3.638620in}}%
\pgfpathlineto{\pgfqpoint{2.566452in}{3.644060in}}%
\pgfpathlineto{\pgfqpoint{2.569673in}{3.645958in}}%
\pgfpathlineto{\pgfqpoint{2.570747in}{3.649627in}}%
\pgfpathlineto{\pgfqpoint{2.572895in}{3.647603in}}%
\pgfpathlineto{\pgfqpoint{2.573968in}{3.648235in}}%
\pgfpathlineto{\pgfqpoint{2.578264in}{3.648489in}}%
\pgfpathlineto{\pgfqpoint{2.579338in}{3.646211in}}%
\pgfpathlineto{\pgfqpoint{2.580412in}{3.646338in}}%
\pgfpathlineto{\pgfqpoint{2.581485in}{3.645073in}}%
\pgfpathlineto{\pgfqpoint{2.586855in}{3.646211in}}%
\pgfpathlineto{\pgfqpoint{2.587928in}{3.644187in}}%
\pgfpathlineto{\pgfqpoint{2.589002in}{3.645199in}}%
\pgfpathlineto{\pgfqpoint{2.593298in}{3.642542in}}%
\pgfpathlineto{\pgfqpoint{2.594371in}{3.643681in}}%
\pgfpathlineto{\pgfqpoint{2.596519in}{3.642416in}}%
\pgfpathlineto{\pgfqpoint{2.602962in}{3.640644in}}%
\pgfpathlineto{\pgfqpoint{2.604036in}{3.633053in}}%
\pgfpathlineto{\pgfqpoint{2.607257in}{3.624324in}}%
\pgfpathlineto{\pgfqpoint{2.609405in}{3.631282in}}%
\pgfpathlineto{\pgfqpoint{2.610479in}{3.630649in}}%
\pgfpathlineto{\pgfqpoint{2.611553in}{3.627233in}}%
\pgfpathlineto{\pgfqpoint{2.614774in}{3.626601in}}%
\pgfpathlineto{\pgfqpoint{2.615848in}{3.623691in}}%
\pgfpathlineto{\pgfqpoint{2.622291in}{3.623311in}}%
\pgfpathlineto{\pgfqpoint{2.624439in}{3.620654in}}%
\pgfpathlineto{\pgfqpoint{2.626587in}{3.623944in}}%
\pgfpathlineto{\pgfqpoint{2.629808in}{3.627486in}}%
\pgfpathlineto{\pgfqpoint{2.630882in}{3.630270in}}%
\pgfpathlineto{\pgfqpoint{2.633030in}{3.632168in}}%
\pgfpathlineto{\pgfqpoint{2.634103in}{3.631282in}}%
\pgfpathlineto{\pgfqpoint{2.638399in}{3.632927in}}%
\pgfpathlineto{\pgfqpoint{2.639473in}{3.630776in}}%
\pgfpathlineto{\pgfqpoint{2.640546in}{3.630017in}}%
\pgfpathlineto{\pgfqpoint{2.641620in}{3.631915in}}%
\pgfpathlineto{\pgfqpoint{2.645916in}{3.627613in}}%
\pgfpathlineto{\pgfqpoint{2.646989in}{3.631915in}}%
\pgfpathlineto{\pgfqpoint{2.648063in}{3.631915in}}%
\pgfpathlineto{\pgfqpoint{2.649137in}{3.631155in}}%
\pgfpathlineto{\pgfqpoint{2.652359in}{3.629764in}}%
\pgfpathlineto{\pgfqpoint{2.655580in}{3.625968in}}%
\pgfpathlineto{\pgfqpoint{2.656654in}{3.632927in}}%
\pgfpathlineto{\pgfqpoint{2.659876in}{3.628246in}}%
\pgfpathlineto{\pgfqpoint{2.660949in}{3.624577in}}%
\pgfpathlineto{\pgfqpoint{2.662023in}{3.627107in}}%
\pgfpathlineto{\pgfqpoint{2.663097in}{3.626980in}}%
\pgfpathlineto{\pgfqpoint{2.664171in}{3.628372in}}%
\pgfpathlineto{\pgfqpoint{2.667392in}{3.626854in}}%
\pgfpathlineto{\pgfqpoint{2.668466in}{3.622932in}}%
\pgfpathlineto{\pgfqpoint{2.671688in}{3.626601in}}%
\pgfpathlineto{\pgfqpoint{2.674909in}{3.623438in}}%
\pgfpathlineto{\pgfqpoint{2.675983in}{3.625462in}}%
\pgfpathlineto{\pgfqpoint{2.677057in}{3.626221in}}%
\pgfpathlineto{\pgfqpoint{2.678131in}{3.628372in}}%
\pgfpathlineto{\pgfqpoint{2.679205in}{3.627486in}}%
\pgfpathlineto{\pgfqpoint{2.682426in}{3.628372in}}%
\pgfpathlineto{\pgfqpoint{2.683500in}{3.630017in}}%
\pgfpathlineto{\pgfqpoint{2.685648in}{3.629005in}}%
\pgfpathlineto{\pgfqpoint{2.686722in}{3.629637in}}%
\pgfpathlineto{\pgfqpoint{2.691017in}{3.630270in}}%
\pgfpathlineto{\pgfqpoint{2.693165in}{3.624830in}}%
\pgfpathlineto{\pgfqpoint{2.697460in}{3.625715in}}%
\pgfpathlineto{\pgfqpoint{2.700681in}{3.632800in}}%
\pgfpathlineto{\pgfqpoint{2.701755in}{3.624703in}}%
\pgfpathlineto{\pgfqpoint{2.704977in}{3.624703in}}%
\pgfpathlineto{\pgfqpoint{2.706051in}{3.623564in}}%
\pgfpathlineto{\pgfqpoint{2.708198in}{3.619389in}}%
\pgfpathlineto{\pgfqpoint{2.709272in}{3.618377in}}%
\pgfpathlineto{\pgfqpoint{2.712494in}{3.617745in}}%
\pgfpathlineto{\pgfqpoint{2.713567in}{3.618377in}}%
\pgfpathlineto{\pgfqpoint{2.714641in}{3.621034in}}%
\pgfpathlineto{\pgfqpoint{2.720011in}{3.619263in}}%
\pgfpathlineto{\pgfqpoint{2.722158in}{3.615088in}}%
\pgfpathlineto{\pgfqpoint{2.723232in}{3.616985in}}%
\pgfpathlineto{\pgfqpoint{2.724306in}{3.610153in}}%
\pgfpathlineto{\pgfqpoint{2.727527in}{3.609141in}}%
\pgfpathlineto{\pgfqpoint{2.728601in}{3.607623in}}%
\pgfpathlineto{\pgfqpoint{2.729675in}{3.600412in}}%
\pgfpathlineto{\pgfqpoint{2.730749in}{3.601171in}}%
\pgfpathlineto{\pgfqpoint{2.731823in}{3.607750in}}%
\pgfpathlineto{\pgfqpoint{2.735044in}{3.609141in}}%
\pgfpathlineto{\pgfqpoint{2.736118in}{3.610280in}}%
\pgfpathlineto{\pgfqpoint{2.738266in}{3.601297in}}%
\pgfpathlineto{\pgfqpoint{2.743635in}{3.599905in}}%
\pgfpathlineto{\pgfqpoint{2.744709in}{3.600158in}}%
\pgfpathlineto{\pgfqpoint{2.746856in}{3.605725in}}%
\pgfpathlineto{\pgfqpoint{2.750078in}{3.606864in}}%
\pgfpathlineto{\pgfqpoint{2.751152in}{3.606358in}}%
\pgfpathlineto{\pgfqpoint{2.752226in}{3.603448in}}%
\pgfpathlineto{\pgfqpoint{2.753300in}{3.602436in}}%
\pgfpathlineto{\pgfqpoint{2.757595in}{3.617745in}}%
\pgfpathlineto{\pgfqpoint{2.758669in}{3.612178in}}%
\pgfpathlineto{\pgfqpoint{2.759743in}{3.614961in}}%
\pgfpathlineto{\pgfqpoint{2.761890in}{3.624070in}}%
\pgfpathlineto{\pgfqpoint{2.765112in}{3.621793in}}%
\pgfpathlineto{\pgfqpoint{2.767259in}{3.610027in}}%
\pgfpathlineto{\pgfqpoint{2.768333in}{3.607370in}}%
\pgfpathlineto{\pgfqpoint{2.772629in}{3.608129in}}%
\pgfpathlineto{\pgfqpoint{2.773702in}{3.603448in}}%
\pgfpathlineto{\pgfqpoint{2.775850in}{3.601297in}}%
\pgfpathlineto{\pgfqpoint{2.776924in}{3.601171in}}%
\pgfpathlineto{\pgfqpoint{2.780145in}{3.606231in}}%
\pgfpathlineto{\pgfqpoint{2.782293in}{3.605093in}}%
\pgfpathlineto{\pgfqpoint{2.783367in}{3.593073in}}%
\pgfpathlineto{\pgfqpoint{2.784441in}{3.591302in}}%
\pgfpathlineto{\pgfqpoint{2.787662in}{3.590037in}}%
\pgfpathlineto{\pgfqpoint{2.789810in}{3.596236in}}%
\pgfpathlineto{\pgfqpoint{2.790884in}{3.598640in}}%
\pgfpathlineto{\pgfqpoint{2.791958in}{3.598514in}}%
\pgfpathlineto{\pgfqpoint{2.795179in}{3.599020in}}%
\pgfpathlineto{\pgfqpoint{2.797327in}{3.600412in}}%
\pgfpathlineto{\pgfqpoint{2.798401in}{3.597502in}}%
\pgfpathlineto{\pgfqpoint{2.799475in}{3.588519in}}%
\pgfpathlineto{\pgfqpoint{2.802696in}{3.582952in}}%
\pgfpathlineto{\pgfqpoint{2.803770in}{3.583079in}}%
\pgfpathlineto{\pgfqpoint{2.805918in}{3.587127in}}%
\pgfpathlineto{\pgfqpoint{2.806991in}{3.584091in}}%
\pgfpathlineto{\pgfqpoint{2.810213in}{3.584976in}}%
\pgfpathlineto{\pgfqpoint{2.811287in}{3.583079in}}%
\pgfpathlineto{\pgfqpoint{2.812361in}{3.583964in}}%
\pgfpathlineto{\pgfqpoint{2.813434in}{3.586748in}}%
\pgfpathlineto{\pgfqpoint{2.814508in}{3.587001in}}%
\pgfpathlineto{\pgfqpoint{2.818804in}{3.584470in}}%
\pgfpathlineto{\pgfqpoint{2.819877in}{3.586241in}}%
\pgfpathlineto{\pgfqpoint{2.820951in}{3.581434in}}%
\pgfpathlineto{\pgfqpoint{2.822025in}{3.580295in}}%
\pgfpathlineto{\pgfqpoint{2.825247in}{3.582193in}}%
\pgfpathlineto{\pgfqpoint{2.826321in}{3.579663in}}%
\pgfpathlineto{\pgfqpoint{2.827394in}{3.579156in}}%
\pgfpathlineto{\pgfqpoint{2.829542in}{3.572831in}}%
\pgfpathlineto{\pgfqpoint{2.832764in}{3.572071in}}%
\pgfpathlineto{\pgfqpoint{2.833837in}{3.573337in}}%
\pgfpathlineto{\pgfqpoint{2.834911in}{3.570933in}}%
\pgfpathlineto{\pgfqpoint{2.835985in}{3.569921in}}%
\pgfpathlineto{\pgfqpoint{2.837059in}{3.572071in}}%
\pgfpathlineto{\pgfqpoint{2.841354in}{3.571565in}}%
\pgfpathlineto{\pgfqpoint{2.842428in}{3.569668in}}%
\pgfpathlineto{\pgfqpoint{2.843502in}{3.572831in}}%
\pgfpathlineto{\pgfqpoint{2.844576in}{3.579663in}}%
\pgfpathlineto{\pgfqpoint{2.848871in}{3.575108in}}%
\pgfpathlineto{\pgfqpoint{2.849945in}{3.577132in}}%
\pgfpathlineto{\pgfqpoint{2.851019in}{3.566758in}}%
\pgfpathlineto{\pgfqpoint{2.852093in}{3.564354in}}%
\pgfpathlineto{\pgfqpoint{2.855314in}{3.563215in}}%
\pgfpathlineto{\pgfqpoint{2.858536in}{3.569415in}}%
\pgfpathlineto{\pgfqpoint{2.859610in}{3.568402in}}%
\pgfpathlineto{\pgfqpoint{2.862831in}{3.574602in}}%
\pgfpathlineto{\pgfqpoint{2.863905in}{3.571565in}}%
\pgfpathlineto{\pgfqpoint{2.864979in}{3.572957in}}%
\pgfpathlineto{\pgfqpoint{2.866053in}{3.578018in}}%
\pgfpathlineto{\pgfqpoint{2.867126in}{3.579409in}}%
\pgfpathlineto{\pgfqpoint{2.870348in}{3.582193in}}%
\pgfpathlineto{\pgfqpoint{2.871422in}{3.580042in}}%
\pgfpathlineto{\pgfqpoint{2.872496in}{3.573590in}}%
\pgfpathlineto{\pgfqpoint{2.874643in}{3.571312in}}%
\pgfpathlineto{\pgfqpoint{2.877865in}{3.575867in}}%
\pgfpathlineto{\pgfqpoint{2.878939in}{3.578524in}}%
\pgfpathlineto{\pgfqpoint{2.880012in}{3.574855in}}%
\pgfpathlineto{\pgfqpoint{2.881086in}{3.575487in}}%
\pgfpathlineto{\pgfqpoint{2.882160in}{3.573716in}}%
\pgfpathlineto{\pgfqpoint{2.885382in}{3.562583in}}%
\pgfpathlineto{\pgfqpoint{2.886455in}{3.561823in}}%
\pgfpathlineto{\pgfqpoint{2.887529in}{3.558407in}}%
\pgfpathlineto{\pgfqpoint{2.889677in}{3.557395in}}%
\pgfpathlineto{\pgfqpoint{2.892899in}{3.561823in}}%
\pgfpathlineto{\pgfqpoint{2.893972in}{3.559799in}}%
\pgfpathlineto{\pgfqpoint{2.895046in}{3.559040in}}%
\pgfpathlineto{\pgfqpoint{2.897194in}{3.567264in}}%
\pgfpathlineto{\pgfqpoint{2.902563in}{3.540189in}}%
\pgfpathlineto{\pgfqpoint{2.903637in}{3.538038in}}%
\pgfpathlineto{\pgfqpoint{2.904711in}{3.531839in}}%
\pgfpathlineto{\pgfqpoint{2.907932in}{3.527410in}}%
\pgfpathlineto{\pgfqpoint{2.910080in}{3.522856in}}%
\pgfpathlineto{\pgfqpoint{2.911154in}{3.521970in}}%
\pgfpathlineto{\pgfqpoint{2.912228in}{3.524247in}}%
\pgfpathlineto{\pgfqpoint{2.915449in}{3.524121in}}%
\pgfpathlineto{\pgfqpoint{2.916523in}{3.525133in}}%
\pgfpathlineto{\pgfqpoint{2.918671in}{3.522603in}}%
\pgfpathlineto{\pgfqpoint{2.919744in}{3.528423in}}%
\pgfpathlineto{\pgfqpoint{2.922966in}{3.511216in}}%
\pgfpathlineto{\pgfqpoint{2.924040in}{3.498185in}}%
\pgfpathlineto{\pgfqpoint{2.925114in}{3.502360in}}%
\pgfpathlineto{\pgfqpoint{2.927261in}{3.501854in}}%
\pgfpathlineto{\pgfqpoint{2.930483in}{3.499197in}}%
\pgfpathlineto{\pgfqpoint{2.931557in}{3.497426in}}%
\pgfpathlineto{\pgfqpoint{2.932631in}{3.501095in}}%
\pgfpathlineto{\pgfqpoint{2.934778in}{3.501601in}}%
\pgfpathlineto{\pgfqpoint{2.938000in}{3.500715in}}%
\pgfpathlineto{\pgfqpoint{2.939074in}{3.504131in}}%
\pgfpathlineto{\pgfqpoint{2.940147in}{3.505017in}}%
\pgfpathlineto{\pgfqpoint{2.941221in}{3.502739in}}%
\pgfpathlineto{\pgfqpoint{2.942295in}{3.497932in}}%
\pgfpathlineto{\pgfqpoint{2.945517in}{3.498817in}}%
\pgfpathlineto{\pgfqpoint{2.946590in}{3.496540in}}%
\pgfpathlineto{\pgfqpoint{2.949812in}{3.497426in}}%
\pgfpathlineto{\pgfqpoint{2.953033in}{3.496793in}}%
\pgfpathlineto{\pgfqpoint{2.954107in}{3.499829in}}%
\pgfpathlineto{\pgfqpoint{2.956255in}{3.496666in}}%
\pgfpathlineto{\pgfqpoint{2.957329in}{3.498691in}}%
\pgfpathlineto{\pgfqpoint{2.960550in}{3.498058in}}%
\pgfpathlineto{\pgfqpoint{2.962698in}{3.494389in}}%
\pgfpathlineto{\pgfqpoint{2.969141in}{3.495528in}}%
\pgfpathlineto{\pgfqpoint{2.971289in}{3.494769in}}%
\pgfpathlineto{\pgfqpoint{2.972363in}{3.495781in}}%
\pgfpathlineto{\pgfqpoint{2.972363in}{3.495781in}}%
\pgfusepath{stroke}%
\end{pgfscope}%
\begin{pgfscope}%
\pgfpathrectangle{\pgfqpoint{0.506453in}{3.271772in}}{\pgfqpoint{2.583333in}{0.400885in}}%
\pgfusepath{clip}%
\pgfsetroundcap%
\pgfsetroundjoin%
\pgfsetlinewidth{1.505625pt}%
\definecolor{currentstroke}{rgb}{0.172549,0.627451,0.172549}%
\pgfsetstrokecolor{currentstroke}%
\pgfsetdash{}{0pt}%
\pgfpathmoveto{\pgfqpoint{0.623878in}{3.464278in}}%
\pgfpathlineto{\pgfqpoint{0.624952in}{3.462380in}}%
\pgfpathlineto{\pgfqpoint{0.626025in}{3.462256in}}%
\pgfpathlineto{\pgfqpoint{0.627099in}{3.463247in}}%
\pgfpathlineto{\pgfqpoint{0.630321in}{3.461266in}}%
\pgfpathlineto{\pgfqpoint{0.631395in}{3.459935in}}%
\pgfpathlineto{\pgfqpoint{0.632468in}{3.461387in}}%
\pgfpathlineto{\pgfqpoint{0.633542in}{3.461871in}}%
\pgfpathlineto{\pgfqpoint{0.634616in}{3.461024in}}%
\pgfpathlineto{\pgfqpoint{0.638911in}{3.460177in}}%
\pgfpathlineto{\pgfqpoint{0.639985in}{3.457636in}}%
\pgfpathlineto{\pgfqpoint{0.641059in}{3.458693in}}%
\pgfpathlineto{\pgfqpoint{0.642133in}{3.458693in}}%
\pgfpathlineto{\pgfqpoint{0.646428in}{3.455992in}}%
\pgfpathlineto{\pgfqpoint{0.647502in}{3.453408in}}%
\pgfpathlineto{\pgfqpoint{0.649650in}{3.452497in}}%
\pgfpathlineto{\pgfqpoint{0.652871in}{3.451472in}}%
\pgfpathlineto{\pgfqpoint{0.653945in}{3.449765in}}%
\pgfpathlineto{\pgfqpoint{0.655019in}{3.450334in}}%
\pgfpathlineto{\pgfqpoint{0.656093in}{3.450106in}}%
\pgfpathlineto{\pgfqpoint{0.657167in}{3.447716in}}%
\pgfpathlineto{\pgfqpoint{0.660388in}{3.447937in}}%
\pgfpathlineto{\pgfqpoint{0.662536in}{3.449484in}}%
\pgfpathlineto{\pgfqpoint{0.663610in}{3.448600in}}%
\pgfpathlineto{\pgfqpoint{0.664684in}{3.446500in}}%
\pgfpathlineto{\pgfqpoint{0.668979in}{3.443978in}}%
\pgfpathlineto{\pgfqpoint{0.671127in}{3.440403in}}%
\pgfpathlineto{\pgfqpoint{0.672200in}{3.438294in}}%
\pgfpathlineto{\pgfqpoint{0.678643in}{3.439832in}}%
\pgfpathlineto{\pgfqpoint{0.682939in}{3.437986in}}%
\pgfpathlineto{\pgfqpoint{0.684013in}{3.438704in}}%
\pgfpathlineto{\pgfqpoint{0.685087in}{3.437883in}}%
\pgfpathlineto{\pgfqpoint{0.686160in}{3.438396in}}%
\pgfpathlineto{\pgfqpoint{0.687234in}{3.437268in}}%
\pgfpathlineto{\pgfqpoint{0.690456in}{3.436243in}}%
\pgfpathlineto{\pgfqpoint{0.692603in}{3.430295in}}%
\pgfpathlineto{\pgfqpoint{0.693677in}{3.428319in}}%
\pgfpathlineto{\pgfqpoint{0.697973in}{3.429088in}}%
\pgfpathlineto{\pgfqpoint{0.699046in}{3.425726in}}%
\pgfpathlineto{\pgfqpoint{0.700120in}{3.427099in}}%
\pgfpathlineto{\pgfqpoint{0.701194in}{3.424536in}}%
\pgfpathlineto{\pgfqpoint{0.702268in}{3.424800in}}%
\pgfpathlineto{\pgfqpoint{0.705489in}{3.424889in}}%
\pgfpathlineto{\pgfqpoint{0.706563in}{3.423919in}}%
\pgfpathlineto{\pgfqpoint{0.707637in}{3.423919in}}%
\pgfpathlineto{\pgfqpoint{0.708711in}{3.422420in}}%
\pgfpathlineto{\pgfqpoint{0.716228in}{3.419614in}}%
\pgfpathlineto{\pgfqpoint{0.717302in}{3.420387in}}%
\pgfpathlineto{\pgfqpoint{0.721597in}{3.419700in}}%
\pgfpathlineto{\pgfqpoint{0.723745in}{3.416610in}}%
\pgfpathlineto{\pgfqpoint{0.728040in}{3.414721in}}%
\pgfpathlineto{\pgfqpoint{0.730188in}{3.409913in}}%
\pgfpathlineto{\pgfqpoint{0.732335in}{3.405493in}}%
\pgfpathlineto{\pgfqpoint{0.735557in}{3.405574in}}%
\pgfpathlineto{\pgfqpoint{0.737705in}{3.401360in}}%
\pgfpathlineto{\pgfqpoint{0.738778in}{3.401592in}}%
\pgfpathlineto{\pgfqpoint{0.739852in}{3.400281in}}%
\pgfpathlineto{\pgfqpoint{0.743074in}{3.398624in}}%
\pgfpathlineto{\pgfqpoint{0.744148in}{3.395913in}}%
\pgfpathlineto{\pgfqpoint{0.745221in}{3.395412in}}%
\pgfpathlineto{\pgfqpoint{0.747369in}{3.397203in}}%
\pgfpathlineto{\pgfqpoint{0.750591in}{3.396128in}}%
\pgfpathlineto{\pgfqpoint{0.751664in}{3.397275in}}%
\pgfpathlineto{\pgfqpoint{0.752738in}{3.397131in}}%
\pgfpathlineto{\pgfqpoint{0.754886in}{3.394839in}}%
\pgfpathlineto{\pgfqpoint{0.759181in}{3.394337in}}%
\pgfpathlineto{\pgfqpoint{0.760255in}{3.392474in}}%
\pgfpathlineto{\pgfqpoint{0.761329in}{3.393477in}}%
\pgfpathlineto{\pgfqpoint{0.762403in}{3.393048in}}%
\pgfpathlineto{\pgfqpoint{0.766698in}{3.389680in}}%
\pgfpathlineto{\pgfqpoint{0.767772in}{3.386385in}}%
\pgfpathlineto{\pgfqpoint{0.768846in}{3.385781in}}%
\pgfpathlineto{\pgfqpoint{0.776363in}{3.387659in}}%
\pgfpathlineto{\pgfqpoint{0.777437in}{3.387391in}}%
\pgfpathlineto{\pgfqpoint{0.781732in}{3.388128in}}%
\pgfpathlineto{\pgfqpoint{0.782806in}{3.386586in}}%
\pgfpathlineto{\pgfqpoint{0.783880in}{3.386854in}}%
\pgfpathlineto{\pgfqpoint{0.784953in}{3.384037in}}%
\pgfpathlineto{\pgfqpoint{0.788175in}{3.382025in}}%
\pgfpathlineto{\pgfqpoint{0.789249in}{3.382495in}}%
\pgfpathlineto{\pgfqpoint{0.790323in}{3.379209in}}%
\pgfpathlineto{\pgfqpoint{0.792470in}{3.380708in}}%
\pgfpathlineto{\pgfqpoint{0.795692in}{3.380271in}}%
\pgfpathlineto{\pgfqpoint{0.796766in}{3.378521in}}%
\pgfpathlineto{\pgfqpoint{0.797840in}{3.378041in}}%
\pgfpathlineto{\pgfqpoint{0.799987in}{3.374970in}}%
\pgfpathlineto{\pgfqpoint{0.803209in}{3.373905in}}%
\pgfpathlineto{\pgfqpoint{0.805356in}{3.375362in}}%
\pgfpathlineto{\pgfqpoint{0.807504in}{3.372504in}}%
\pgfpathlineto{\pgfqpoint{0.810726in}{3.371306in}}%
\pgfpathlineto{\pgfqpoint{0.812873in}{3.368785in}}%
\pgfpathlineto{\pgfqpoint{0.813947in}{3.369041in}}%
\pgfpathlineto{\pgfqpoint{0.815021in}{3.366533in}}%
\pgfpathlineto{\pgfqpoint{0.819316in}{3.365043in}}%
\pgfpathlineto{\pgfqpoint{0.821464in}{3.364659in}}%
\pgfpathlineto{\pgfqpoint{0.822538in}{3.363458in}}%
\pgfpathlineto{\pgfqpoint{0.825759in}{3.363602in}}%
\pgfpathlineto{\pgfqpoint{0.826833in}{3.362065in}}%
\pgfpathlineto{\pgfqpoint{0.827907in}{3.362257in}}%
\pgfpathlineto{\pgfqpoint{0.830055in}{3.360191in}}%
\pgfpathlineto{\pgfqpoint{0.833276in}{3.359541in}}%
\pgfpathlineto{\pgfqpoint{0.835424in}{3.360470in}}%
\pgfpathlineto{\pgfqpoint{0.837572in}{3.360563in}}%
\pgfpathlineto{\pgfqpoint{0.840793in}{3.361353in}}%
\pgfpathlineto{\pgfqpoint{0.842941in}{3.361027in}}%
\pgfpathlineto{\pgfqpoint{0.845088in}{3.357844in}}%
\pgfpathlineto{\pgfqpoint{0.851531in}{3.356571in}}%
\pgfpathlineto{\pgfqpoint{0.852605in}{3.355170in}}%
\pgfpathlineto{\pgfqpoint{0.870861in}{3.355089in}}%
\pgfpathlineto{\pgfqpoint{0.874082in}{3.354193in}}%
\pgfpathlineto{\pgfqpoint{0.875156in}{3.354682in}}%
\pgfpathlineto{\pgfqpoint{0.880525in}{3.354763in}}%
\pgfpathlineto{\pgfqpoint{0.881599in}{3.354193in}}%
\pgfpathlineto{\pgfqpoint{0.882673in}{3.354397in}}%
\pgfpathlineto{\pgfqpoint{0.886968in}{3.353786in}}%
\pgfpathlineto{\pgfqpoint{0.888042in}{3.354234in}}%
\pgfpathlineto{\pgfqpoint{0.889116in}{3.352199in}}%
\pgfpathlineto{\pgfqpoint{0.890190in}{3.351358in}}%
\pgfpathlineto{\pgfqpoint{0.897707in}{3.351082in}}%
\pgfpathlineto{\pgfqpoint{0.902002in}{3.351444in}}%
\pgfpathlineto{\pgfqpoint{0.905223in}{3.352746in}}%
\pgfpathlineto{\pgfqpoint{0.910593in}{3.351552in}}%
\pgfpathlineto{\pgfqpoint{0.911666in}{3.349816in}}%
\pgfpathlineto{\pgfqpoint{0.917036in}{3.349987in}}%
\pgfpathlineto{\pgfqpoint{0.920257in}{3.350842in}}%
\pgfpathlineto{\pgfqpoint{0.923479in}{3.350329in}}%
\pgfpathlineto{\pgfqpoint{0.925626in}{3.349029in}}%
\pgfpathlineto{\pgfqpoint{0.930996in}{3.349577in}}%
\pgfpathlineto{\pgfqpoint{0.932069in}{3.349577in}}%
\pgfpathlineto{\pgfqpoint{0.933143in}{3.350295in}}%
\pgfpathlineto{\pgfqpoint{0.934217in}{3.350021in}}%
\pgfpathlineto{\pgfqpoint{0.935291in}{3.347969in}}%
\pgfpathlineto{\pgfqpoint{0.938512in}{3.347115in}}%
\pgfpathlineto{\pgfqpoint{0.939586in}{3.345986in}}%
\pgfpathlineto{\pgfqpoint{0.948177in}{3.345405in}}%
\pgfpathlineto{\pgfqpoint{0.949251in}{3.344653in}}%
\pgfpathlineto{\pgfqpoint{0.953546in}{3.344819in}}%
\pgfpathlineto{\pgfqpoint{0.954620in}{3.345285in}}%
\pgfpathlineto{\pgfqpoint{0.956768in}{3.343488in}}%
\pgfpathlineto{\pgfqpoint{0.957841in}{3.343788in}}%
\pgfpathlineto{\pgfqpoint{0.962137in}{3.342956in}}%
\pgfpathlineto{\pgfqpoint{0.963211in}{3.341225in}}%
\pgfpathlineto{\pgfqpoint{0.965358in}{3.341591in}}%
\pgfpathlineto{\pgfqpoint{0.969654in}{3.340198in}}%
\pgfpathlineto{\pgfqpoint{0.970728in}{3.340357in}}%
\pgfpathlineto{\pgfqpoint{0.972875in}{3.339470in}}%
\pgfpathlineto{\pgfqpoint{0.977171in}{3.339073in}}%
\pgfpathlineto{\pgfqpoint{0.979318in}{3.338402in}}%
\pgfpathlineto{\pgfqpoint{0.980392in}{3.338402in}}%
\pgfpathlineto{\pgfqpoint{0.986835in}{3.337259in}}%
\pgfpathlineto{\pgfqpoint{0.987909in}{3.337460in}}%
\pgfpathlineto{\pgfqpoint{0.995426in}{3.336635in}}%
\pgfpathlineto{\pgfqpoint{0.999721in}{3.335444in}}%
\pgfpathlineto{\pgfqpoint{1.000795in}{3.334000in}}%
\pgfpathlineto{\pgfqpoint{1.001869in}{3.334491in}}%
\pgfpathlineto{\pgfqpoint{1.002943in}{3.334136in}}%
\pgfpathlineto{\pgfqpoint{1.009386in}{3.333728in}}%
\pgfpathlineto{\pgfqpoint{1.013681in}{3.332039in}}%
\pgfpathlineto{\pgfqpoint{1.017976in}{3.331067in}}%
\pgfpathlineto{\pgfqpoint{1.030863in}{3.330917in}}%
\pgfpathlineto{\pgfqpoint{1.031936in}{3.331267in}}%
\pgfpathlineto{\pgfqpoint{1.033010in}{3.329818in}}%
\pgfpathlineto{\pgfqpoint{1.039453in}{3.329818in}}%
\pgfpathlineto{\pgfqpoint{1.040527in}{3.330261in}}%
\pgfpathlineto{\pgfqpoint{1.044822in}{3.330633in}}%
\pgfpathlineto{\pgfqpoint{1.048044in}{3.329609in}}%
\pgfpathlineto{\pgfqpoint{1.051265in}{3.329067in}}%
\pgfpathlineto{\pgfqpoint{1.053413in}{3.329293in}}%
\pgfpathlineto{\pgfqpoint{1.059856in}{3.329541in}}%
\pgfpathlineto{\pgfqpoint{1.060930in}{3.328119in}}%
\pgfpathlineto{\pgfqpoint{1.063078in}{3.327951in}}%
\pgfpathlineto{\pgfqpoint{1.078111in}{3.325606in}}%
\pgfpathlineto{\pgfqpoint{1.085628in}{3.325520in}}%
\pgfpathlineto{\pgfqpoint{1.107105in}{3.324526in}}%
\pgfpathlineto{\pgfqpoint{1.119991in}{3.324449in}}%
\pgfpathlineto{\pgfqpoint{1.121065in}{3.323665in}}%
\pgfpathlineto{\pgfqpoint{1.123213in}{3.323483in}}%
\pgfpathlineto{\pgfqpoint{1.129656in}{3.321525in}}%
\pgfpathlineto{\pgfqpoint{1.130729in}{3.320243in}}%
\pgfpathlineto{\pgfqpoint{1.137173in}{3.319246in}}%
\pgfpathlineto{\pgfqpoint{1.138246in}{3.319587in}}%
\pgfpathlineto{\pgfqpoint{1.165092in}{3.320620in}}%
\pgfpathlineto{\pgfqpoint{1.167240in}{3.320620in}}%
\pgfpathlineto{\pgfqpoint{1.168314in}{3.320454in}}%
\pgfpathlineto{\pgfqpoint{1.175831in}{3.320172in}}%
\pgfpathlineto{\pgfqpoint{1.190864in}{3.318405in}}%
\pgfpathlineto{\pgfqpoint{1.194086in}{3.318714in}}%
\pgfpathlineto{\pgfqpoint{1.197307in}{3.316985in}}%
\pgfpathlineto{\pgfqpoint{1.198381in}{3.317118in}}%
\pgfpathlineto{\pgfqpoint{1.202677in}{3.316823in}}%
\pgfpathlineto{\pgfqpoint{1.205898in}{3.316926in}}%
\pgfpathlineto{\pgfqpoint{1.210194in}{3.316587in}}%
\pgfpathlineto{\pgfqpoint{1.216637in}{3.316303in}}%
\pgfpathlineto{\pgfqpoint{1.220932in}{3.315225in}}%
\pgfpathlineto{\pgfqpoint{1.227375in}{3.315076in}}%
\pgfpathlineto{\pgfqpoint{1.228449in}{3.313902in}}%
\pgfpathlineto{\pgfqpoint{1.235966in}{3.313828in}}%
\pgfpathlineto{\pgfqpoint{1.263885in}{3.312819in}}%
\pgfpathlineto{\pgfqpoint{1.266033in}{3.312979in}}%
\pgfpathlineto{\pgfqpoint{1.293953in}{3.311061in}}%
\pgfpathlineto{\pgfqpoint{1.296101in}{3.310489in}}%
\pgfpathlineto{\pgfqpoint{1.303617in}{3.310521in}}%
\pgfpathlineto{\pgfqpoint{1.325094in}{3.309355in}}%
\pgfpathlineto{\pgfqpoint{1.326168in}{3.308671in}}%
\pgfpathlineto{\pgfqpoint{1.333685in}{3.308050in}}%
\pgfpathlineto{\pgfqpoint{1.341202in}{3.307953in}}%
\pgfpathlineto{\pgfqpoint{1.347645in}{3.307335in}}%
\pgfpathlineto{\pgfqpoint{1.348719in}{3.307038in}}%
\pgfpathlineto{\pgfqpoint{1.383082in}{3.306467in}}%
\pgfpathlineto{\pgfqpoint{1.385229in}{3.306098in}}%
\pgfpathlineto{\pgfqpoint{1.389525in}{3.306367in}}%
\pgfpathlineto{\pgfqpoint{1.407780in}{3.306354in}}%
\pgfpathlineto{\pgfqpoint{1.422814in}{3.305728in}}%
\pgfpathlineto{\pgfqpoint{1.423887in}{3.305364in}}%
\pgfpathlineto{\pgfqpoint{1.430330in}{3.304920in}}%
\pgfpathlineto{\pgfqpoint{1.431404in}{3.304411in}}%
\pgfpathlineto{\pgfqpoint{1.438921in}{3.304075in}}%
\pgfpathlineto{\pgfqpoint{1.445364in}{3.303489in}}%
\pgfpathlineto{\pgfqpoint{1.449660in}{3.303544in}}%
\pgfpathlineto{\pgfqpoint{1.452881in}{3.303339in}}%
\pgfpathlineto{\pgfqpoint{1.453955in}{3.303498in}}%
\pgfpathlineto{\pgfqpoint{1.472210in}{3.302919in}}%
\pgfpathlineto{\pgfqpoint{1.474358in}{3.302778in}}%
\pgfpathlineto{\pgfqpoint{1.476506in}{3.302882in}}%
\pgfpathlineto{\pgfqpoint{1.603218in}{3.301483in}}%
\pgfpathlineto{\pgfqpoint{1.617178in}{3.300898in}}%
\pgfpathlineto{\pgfqpoint{1.622548in}{3.300471in}}%
\pgfpathlineto{\pgfqpoint{1.637581in}{3.300181in}}%
\pgfpathlineto{\pgfqpoint{1.641877in}{3.299987in}}%
\pgfpathlineto{\pgfqpoint{1.655837in}{3.299704in}}%
\pgfpathlineto{\pgfqpoint{1.662280in}{3.299589in}}%
\pgfpathlineto{\pgfqpoint{1.676239in}{3.299502in}}%
\pgfpathlineto{\pgfqpoint{1.698790in}{3.299480in}}%
\pgfpathlineto{\pgfqpoint{1.702012in}{3.299397in}}%
\pgfpathlineto{\pgfqpoint{1.708455in}{3.298841in}}%
\pgfpathlineto{\pgfqpoint{1.712750in}{3.298536in}}%
\pgfpathlineto{\pgfqpoint{1.720267in}{3.298510in}}%
\pgfpathlineto{\pgfqpoint{1.724562in}{3.298428in}}%
\pgfpathlineto{\pgfqpoint{1.776106in}{3.297347in}}%
\pgfpathlineto{\pgfqpoint{1.781476in}{3.297074in}}%
\pgfpathlineto{\pgfqpoint{1.784697in}{3.296777in}}%
\pgfpathlineto{\pgfqpoint{1.802952in}{3.296499in}}%
\pgfpathlineto{\pgfqpoint{1.807248in}{3.296188in}}%
\pgfpathlineto{\pgfqpoint{1.903893in}{3.295382in}}%
\pgfpathlineto{\pgfqpoint{1.904967in}{3.294782in}}%
\pgfpathlineto{\pgfqpoint{1.920001in}{3.294485in}}%
\pgfpathlineto{\pgfqpoint{2.016646in}{3.294317in}}%
\pgfpathlineto{\pgfqpoint{2.023089in}{3.294317in}}%
\pgfpathlineto{\pgfqpoint{2.150876in}{3.291978in}}%
\pgfpathlineto{\pgfqpoint{2.219602in}{3.291461in}}%
\pgfpathlineto{\pgfqpoint{2.235709in}{3.291179in}}%
\pgfpathlineto{\pgfqpoint{2.363496in}{3.290863in}}%
\pgfpathlineto{\pgfqpoint{2.972363in}{3.289995in}}%
\pgfpathlineto{\pgfqpoint{2.972363in}{3.289995in}}%
\pgfusepath{stroke}%
\end{pgfscope}%
\begin{pgfscope}%
\pgfsetrectcap%
\pgfsetmiterjoin%
\pgfsetlinewidth{0.803000pt}%
\definecolor{currentstroke}{rgb}{1.000000,1.000000,1.000000}%
\pgfsetstrokecolor{currentstroke}%
\pgfsetdash{}{0pt}%
\pgfpathmoveto{\pgfqpoint{0.506453in}{3.271772in}}%
\pgfpathlineto{\pgfqpoint{0.506453in}{3.672657in}}%
\pgfusepath{stroke}%
\end{pgfscope}%
\begin{pgfscope}%
\pgfsetrectcap%
\pgfsetmiterjoin%
\pgfsetlinewidth{0.803000pt}%
\definecolor{currentstroke}{rgb}{1.000000,1.000000,1.000000}%
\pgfsetstrokecolor{currentstroke}%
\pgfsetdash{}{0pt}%
\pgfpathmoveto{\pgfqpoint{3.089787in}{3.271772in}}%
\pgfpathlineto{\pgfqpoint{3.089787in}{3.672657in}}%
\pgfusepath{stroke}%
\end{pgfscope}%
\begin{pgfscope}%
\pgfsetrectcap%
\pgfsetmiterjoin%
\pgfsetlinewidth{0.803000pt}%
\definecolor{currentstroke}{rgb}{1.000000,1.000000,1.000000}%
\pgfsetstrokecolor{currentstroke}%
\pgfsetdash{}{0pt}%
\pgfpathmoveto{\pgfqpoint{0.506453in}{3.271772in}}%
\pgfpathlineto{\pgfqpoint{3.089787in}{3.271772in}}%
\pgfusepath{stroke}%
\end{pgfscope}%
\begin{pgfscope}%
\pgfsetrectcap%
\pgfsetmiterjoin%
\pgfsetlinewidth{0.803000pt}%
\definecolor{currentstroke}{rgb}{1.000000,1.000000,1.000000}%
\pgfsetstrokecolor{currentstroke}%
\pgfsetdash{}{0pt}%
\pgfpathmoveto{\pgfqpoint{0.506453in}{3.672657in}}%
\pgfpathlineto{\pgfqpoint{3.089787in}{3.672657in}}%
\pgfusepath{stroke}%
\end{pgfscope}%
\begin{pgfscope}%
\definecolor{textcolor}{rgb}{0.150000,0.150000,0.150000}%
\pgfsetstrokecolor{textcolor}%
\pgfsetfillcolor{textcolor}%
\pgftext[x=1.798120in,y=3.755990in,,base]{\color{textcolor}\rmfamily\fontsize{16.800000}{20.160000}\selectfont GE}%
\end{pgfscope}%
\begin{pgfscope}%
\pgfsetbuttcap%
\pgfsetmiterjoin%
\definecolor{currentfill}{rgb}{0.917647,0.917647,0.949020}%
\pgfsetfillcolor{currentfill}%
\pgfsetlinewidth{0.000000pt}%
\definecolor{currentstroke}{rgb}{0.000000,0.000000,0.000000}%
\pgfsetstrokecolor{currentstroke}%
\pgfsetstrokeopacity{0.000000}%
\pgfsetdash{}{0pt}%
\pgfpathmoveto{\pgfqpoint{4.123120in}{3.271772in}}%
\pgfpathlineto{\pgfqpoint{6.706453in}{3.271772in}}%
\pgfpathlineto{\pgfqpoint{6.706453in}{3.672657in}}%
\pgfpathlineto{\pgfqpoint{4.123120in}{3.672657in}}%
\pgfpathclose%
\pgfusepath{fill}%
\end{pgfscope}%
\begin{pgfscope}%
\pgfpathrectangle{\pgfqpoint{4.123120in}{3.271772in}}{\pgfqpoint{2.583333in}{0.400885in}}%
\pgfusepath{clip}%
\pgfsetroundcap%
\pgfsetroundjoin%
\pgfsetlinewidth{0.803000pt}%
\definecolor{currentstroke}{rgb}{1.000000,1.000000,1.000000}%
\pgfsetstrokecolor{currentstroke}%
\pgfsetdash{}{0pt}%
\pgfpathmoveto{\pgfqpoint{4.238397in}{3.271772in}}%
\pgfpathlineto{\pgfqpoint{4.238397in}{3.672657in}}%
\pgfusepath{stroke}%
\end{pgfscope}%
\begin{pgfscope}%
\definecolor{textcolor}{rgb}{0.150000,0.150000,0.150000}%
\pgfsetstrokecolor{textcolor}%
\pgfsetfillcolor{textcolor}%
\pgftext[x=4.238397in,y=3.174550in,,top]{\color{textcolor}\rmfamily\fontsize{14.000000}{16.800000}\selectfont 2012}%
\end{pgfscope}%
\begin{pgfscope}%
\pgfpathrectangle{\pgfqpoint{4.123120in}{3.271772in}}{\pgfqpoint{2.583333in}{0.400885in}}%
\pgfusepath{clip}%
\pgfsetroundcap%
\pgfsetroundjoin%
\pgfsetlinewidth{0.803000pt}%
\definecolor{currentstroke}{rgb}{1.000000,1.000000,1.000000}%
\pgfsetstrokecolor{currentstroke}%
\pgfsetdash{}{0pt}%
\pgfpathmoveto{\pgfqpoint{4.631422in}{3.271772in}}%
\pgfpathlineto{\pgfqpoint{4.631422in}{3.672657in}}%
\pgfusepath{stroke}%
\end{pgfscope}%
\begin{pgfscope}%
\definecolor{textcolor}{rgb}{0.150000,0.150000,0.150000}%
\pgfsetstrokecolor{textcolor}%
\pgfsetfillcolor{textcolor}%
\pgftext[x=4.631422in,y=3.174550in,,top]{\color{textcolor}\rmfamily\fontsize{14.000000}{16.800000}\selectfont 2013}%
\end{pgfscope}%
\begin{pgfscope}%
\pgfpathrectangle{\pgfqpoint{4.123120in}{3.271772in}}{\pgfqpoint{2.583333in}{0.400885in}}%
\pgfusepath{clip}%
\pgfsetroundcap%
\pgfsetroundjoin%
\pgfsetlinewidth{0.803000pt}%
\definecolor{currentstroke}{rgb}{1.000000,1.000000,1.000000}%
\pgfsetstrokecolor{currentstroke}%
\pgfsetdash{}{0pt}%
\pgfpathmoveto{\pgfqpoint{5.023373in}{3.271772in}}%
\pgfpathlineto{\pgfqpoint{5.023373in}{3.672657in}}%
\pgfusepath{stroke}%
\end{pgfscope}%
\begin{pgfscope}%
\definecolor{textcolor}{rgb}{0.150000,0.150000,0.150000}%
\pgfsetstrokecolor{textcolor}%
\pgfsetfillcolor{textcolor}%
\pgftext[x=5.023373in,y=3.174550in,,top]{\color{textcolor}\rmfamily\fontsize{14.000000}{16.800000}\selectfont 2014}%
\end{pgfscope}%
\begin{pgfscope}%
\pgfpathrectangle{\pgfqpoint{4.123120in}{3.271772in}}{\pgfqpoint{2.583333in}{0.400885in}}%
\pgfusepath{clip}%
\pgfsetroundcap%
\pgfsetroundjoin%
\pgfsetlinewidth{0.803000pt}%
\definecolor{currentstroke}{rgb}{1.000000,1.000000,1.000000}%
\pgfsetstrokecolor{currentstroke}%
\pgfsetdash{}{0pt}%
\pgfpathmoveto{\pgfqpoint{5.415324in}{3.271772in}}%
\pgfpathlineto{\pgfqpoint{5.415324in}{3.672657in}}%
\pgfusepath{stroke}%
\end{pgfscope}%
\begin{pgfscope}%
\definecolor{textcolor}{rgb}{0.150000,0.150000,0.150000}%
\pgfsetstrokecolor{textcolor}%
\pgfsetfillcolor{textcolor}%
\pgftext[x=5.415324in,y=3.174550in,,top]{\color{textcolor}\rmfamily\fontsize{14.000000}{16.800000}\selectfont 2015}%
\end{pgfscope}%
\begin{pgfscope}%
\pgfpathrectangle{\pgfqpoint{4.123120in}{3.271772in}}{\pgfqpoint{2.583333in}{0.400885in}}%
\pgfusepath{clip}%
\pgfsetroundcap%
\pgfsetroundjoin%
\pgfsetlinewidth{0.803000pt}%
\definecolor{currentstroke}{rgb}{1.000000,1.000000,1.000000}%
\pgfsetstrokecolor{currentstroke}%
\pgfsetdash{}{0pt}%
\pgfpathmoveto{\pgfqpoint{5.807275in}{3.271772in}}%
\pgfpathlineto{\pgfqpoint{5.807275in}{3.672657in}}%
\pgfusepath{stroke}%
\end{pgfscope}%
\begin{pgfscope}%
\definecolor{textcolor}{rgb}{0.150000,0.150000,0.150000}%
\pgfsetstrokecolor{textcolor}%
\pgfsetfillcolor{textcolor}%
\pgftext[x=5.807275in,y=3.174550in,,top]{\color{textcolor}\rmfamily\fontsize{14.000000}{16.800000}\selectfont 2016}%
\end{pgfscope}%
\begin{pgfscope}%
\pgfpathrectangle{\pgfqpoint{4.123120in}{3.271772in}}{\pgfqpoint{2.583333in}{0.400885in}}%
\pgfusepath{clip}%
\pgfsetroundcap%
\pgfsetroundjoin%
\pgfsetlinewidth{0.803000pt}%
\definecolor{currentstroke}{rgb}{1.000000,1.000000,1.000000}%
\pgfsetstrokecolor{currentstroke}%
\pgfsetdash{}{0pt}%
\pgfpathmoveto{\pgfqpoint{6.200300in}{3.271772in}}%
\pgfpathlineto{\pgfqpoint{6.200300in}{3.672657in}}%
\pgfusepath{stroke}%
\end{pgfscope}%
\begin{pgfscope}%
\definecolor{textcolor}{rgb}{0.150000,0.150000,0.150000}%
\pgfsetstrokecolor{textcolor}%
\pgfsetfillcolor{textcolor}%
\pgftext[x=6.200300in,y=3.174550in,,top]{\color{textcolor}\rmfamily\fontsize{14.000000}{16.800000}\selectfont 2017}%
\end{pgfscope}%
\begin{pgfscope}%
\pgfpathrectangle{\pgfqpoint{4.123120in}{3.271772in}}{\pgfqpoint{2.583333in}{0.400885in}}%
\pgfusepath{clip}%
\pgfsetroundcap%
\pgfsetroundjoin%
\pgfsetlinewidth{0.803000pt}%
\definecolor{currentstroke}{rgb}{1.000000,1.000000,1.000000}%
\pgfsetstrokecolor{currentstroke}%
\pgfsetdash{}{0pt}%
\pgfpathmoveto{\pgfqpoint{6.592251in}{3.271772in}}%
\pgfpathlineto{\pgfqpoint{6.592251in}{3.672657in}}%
\pgfusepath{stroke}%
\end{pgfscope}%
\begin{pgfscope}%
\definecolor{textcolor}{rgb}{0.150000,0.150000,0.150000}%
\pgfsetstrokecolor{textcolor}%
\pgfsetfillcolor{textcolor}%
\pgftext[x=6.592251in,y=3.174550in,,top]{\color{textcolor}\rmfamily\fontsize{14.000000}{16.800000}\selectfont 2018}%
\end{pgfscope}%
\begin{pgfscope}%
\pgfpathrectangle{\pgfqpoint{4.123120in}{3.271772in}}{\pgfqpoint{2.583333in}{0.400885in}}%
\pgfusepath{clip}%
\pgfsetroundcap%
\pgfsetroundjoin%
\pgfsetlinewidth{0.803000pt}%
\definecolor{currentstroke}{rgb}{1.000000,1.000000,1.000000}%
\pgfsetstrokecolor{currentstroke}%
\pgfsetdash{}{0pt}%
\pgfpathmoveto{\pgfqpoint{4.123120in}{3.289926in}}%
\pgfpathlineto{\pgfqpoint{6.706453in}{3.289926in}}%
\pgfusepath{stroke}%
\end{pgfscope}%
\begin{pgfscope}%
\definecolor{textcolor}{rgb}{0.150000,0.150000,0.150000}%
\pgfsetstrokecolor{textcolor}%
\pgfsetfillcolor{textcolor}%
\pgftext[x=3.902186in,y=3.216060in,left,base]{\color{textcolor}\rmfamily\fontsize{14.000000}{16.800000}\selectfont 0}%
\end{pgfscope}%
\begin{pgfscope}%
\pgfpathrectangle{\pgfqpoint{4.123120in}{3.271772in}}{\pgfqpoint{2.583333in}{0.400885in}}%
\pgfusepath{clip}%
\pgfsetroundcap%
\pgfsetroundjoin%
\pgfsetlinewidth{0.803000pt}%
\definecolor{currentstroke}{rgb}{1.000000,1.000000,1.000000}%
\pgfsetstrokecolor{currentstroke}%
\pgfsetdash{}{0pt}%
\pgfpathmoveto{\pgfqpoint{4.123120in}{3.599570in}}%
\pgfpathlineto{\pgfqpoint{6.706453in}{3.599570in}}%
\pgfusepath{stroke}%
\end{pgfscope}%
\begin{pgfscope}%
\definecolor{textcolor}{rgb}{0.150000,0.150000,0.150000}%
\pgfsetstrokecolor{textcolor}%
\pgfsetfillcolor{textcolor}%
\pgftext[x=3.902186in,y=3.525704in,left,base]{\color{textcolor}\rmfamily\fontsize{14.000000}{16.800000}\selectfont 2}%
\end{pgfscope}%
\begin{pgfscope}%
\pgfpathrectangle{\pgfqpoint{4.123120in}{3.271772in}}{\pgfqpoint{2.583333in}{0.400885in}}%
\pgfusepath{clip}%
\pgfsetroundcap%
\pgfsetroundjoin%
\pgfsetlinewidth{1.505625pt}%
\definecolor{currentstroke}{rgb}{0.000000,0.000000,0.000000}%
\pgfsetstrokecolor{currentstroke}%
\pgfsetdash{}{0pt}%
\pgfpathmoveto{\pgfqpoint{4.240544in}{3.444748in}}%
\pgfpathlineto{\pgfqpoint{4.242692in}{3.450194in}}%
\pgfpathlineto{\pgfqpoint{4.243766in}{3.449233in}}%
\pgfpathlineto{\pgfqpoint{4.248061in}{3.451396in}}%
\pgfpathlineto{\pgfqpoint{4.249135in}{3.452757in}}%
\pgfpathlineto{\pgfqpoint{4.250209in}{3.452437in}}%
\pgfpathlineto{\pgfqpoint{4.251283in}{3.448593in}}%
\pgfpathlineto{\pgfqpoint{4.255578in}{3.447952in}}%
\pgfpathlineto{\pgfqpoint{4.257726in}{3.451636in}}%
\pgfpathlineto{\pgfqpoint{4.258800in}{3.456362in}}%
\pgfpathlineto{\pgfqpoint{4.264169in}{3.459646in}}%
\pgfpathlineto{\pgfqpoint{4.265243in}{3.458684in}}%
\pgfpathlineto{\pgfqpoint{4.269538in}{3.458684in}}%
\pgfpathlineto{\pgfqpoint{4.270612in}{3.456602in}}%
\pgfpathlineto{\pgfqpoint{4.271686in}{3.457483in}}%
\pgfpathlineto{\pgfqpoint{4.272760in}{3.457083in}}%
\pgfpathlineto{\pgfqpoint{4.273833in}{3.459966in}}%
\pgfpathlineto{\pgfqpoint{4.278129in}{3.459325in}}%
\pgfpathlineto{\pgfqpoint{4.279203in}{3.460687in}}%
\pgfpathlineto{\pgfqpoint{4.280276in}{3.460767in}}%
\pgfpathlineto{\pgfqpoint{4.281350in}{3.459726in}}%
\pgfpathlineto{\pgfqpoint{4.284572in}{3.459726in}}%
\pgfpathlineto{\pgfqpoint{4.285646in}{3.460286in}}%
\pgfpathlineto{\pgfqpoint{4.286719in}{3.459005in}}%
\pgfpathlineto{\pgfqpoint{4.287793in}{3.460607in}}%
\pgfpathlineto{\pgfqpoint{4.288867in}{3.463971in}}%
\pgfpathlineto{\pgfqpoint{4.293162in}{3.462689in}}%
\pgfpathlineto{\pgfqpoint{4.294236in}{3.459966in}}%
\pgfpathlineto{\pgfqpoint{4.295310in}{3.459485in}}%
\pgfpathlineto{\pgfqpoint{4.299606in}{3.460927in}}%
\pgfpathlineto{\pgfqpoint{4.300679in}{3.463170in}}%
\pgfpathlineto{\pgfqpoint{4.301753in}{3.460927in}}%
\pgfpathlineto{\pgfqpoint{4.303901in}{3.461167in}}%
\pgfpathlineto{\pgfqpoint{4.307122in}{3.458765in}}%
\pgfpathlineto{\pgfqpoint{4.308196in}{3.459165in}}%
\pgfpathlineto{\pgfqpoint{4.309270in}{3.461087in}}%
\pgfpathlineto{\pgfqpoint{4.310344in}{3.460607in}}%
\pgfpathlineto{\pgfqpoint{4.311418in}{3.462129in}}%
\pgfpathlineto{\pgfqpoint{4.314639in}{3.461568in}}%
\pgfpathlineto{\pgfqpoint{4.315713in}{3.464772in}}%
\pgfpathlineto{\pgfqpoint{4.316787in}{3.464611in}}%
\pgfpathlineto{\pgfqpoint{4.317861in}{3.466454in}}%
\pgfpathlineto{\pgfqpoint{4.322156in}{3.466374in}}%
\pgfpathlineto{\pgfqpoint{4.324304in}{3.466614in}}%
\pgfpathlineto{\pgfqpoint{4.325378in}{3.467415in}}%
\pgfpathlineto{\pgfqpoint{4.326451in}{3.467255in}}%
\pgfpathlineto{\pgfqpoint{4.329673in}{3.469257in}}%
\pgfpathlineto{\pgfqpoint{4.330747in}{3.469257in}}%
\pgfpathlineto{\pgfqpoint{4.331821in}{3.466774in}}%
\pgfpathlineto{\pgfqpoint{4.332894in}{3.469017in}}%
\pgfpathlineto{\pgfqpoint{4.333968in}{3.468776in}}%
\pgfpathlineto{\pgfqpoint{4.337190in}{3.470458in}}%
\pgfpathlineto{\pgfqpoint{4.339338in}{3.467575in}}%
\pgfpathlineto{\pgfqpoint{4.340411in}{3.468456in}}%
\pgfpathlineto{\pgfqpoint{4.344707in}{3.466454in}}%
\pgfpathlineto{\pgfqpoint{4.345781in}{3.464531in}}%
\pgfpathlineto{\pgfqpoint{4.346854in}{3.467094in}}%
\pgfpathlineto{\pgfqpoint{4.347928in}{3.471099in}}%
\pgfpathlineto{\pgfqpoint{4.349002in}{3.468616in}}%
\pgfpathlineto{\pgfqpoint{4.353297in}{3.471019in}}%
\pgfpathlineto{\pgfqpoint{4.355445in}{3.466053in}}%
\pgfpathlineto{\pgfqpoint{4.356519in}{3.465492in}}%
\pgfpathlineto{\pgfqpoint{4.360814in}{3.463650in}}%
\pgfpathlineto{\pgfqpoint{4.362962in}{3.469417in}}%
\pgfpathlineto{\pgfqpoint{4.364036in}{3.470458in}}%
\pgfpathlineto{\pgfqpoint{4.367257in}{3.470538in}}%
\pgfpathlineto{\pgfqpoint{4.368331in}{3.474063in}}%
\pgfpathlineto{\pgfqpoint{4.369405in}{3.475504in}}%
\pgfpathlineto{\pgfqpoint{4.370479in}{3.472861in}}%
\pgfpathlineto{\pgfqpoint{4.371553in}{3.468696in}}%
\pgfpathlineto{\pgfqpoint{4.374774in}{3.467735in}}%
\pgfpathlineto{\pgfqpoint{4.376922in}{3.464131in}}%
\pgfpathlineto{\pgfqpoint{4.377996in}{3.464451in}}%
\pgfpathlineto{\pgfqpoint{4.379070in}{3.466934in}}%
\pgfpathlineto{\pgfqpoint{4.384439in}{3.459726in}}%
\pgfpathlineto{\pgfqpoint{4.385513in}{3.457723in}}%
\pgfpathlineto{\pgfqpoint{4.386586in}{3.456922in}}%
\pgfpathlineto{\pgfqpoint{4.389808in}{3.457483in}}%
\pgfpathlineto{\pgfqpoint{4.390882in}{3.456682in}}%
\pgfpathlineto{\pgfqpoint{4.391956in}{3.452918in}}%
\pgfpathlineto{\pgfqpoint{4.394103in}{3.454840in}}%
\pgfpathlineto{\pgfqpoint{4.399472in}{3.457323in}}%
\pgfpathlineto{\pgfqpoint{4.400546in}{3.455481in}}%
\pgfpathlineto{\pgfqpoint{4.401620in}{3.450995in}}%
\pgfpathlineto{\pgfqpoint{4.404842in}{3.450355in}}%
\pgfpathlineto{\pgfqpoint{4.405916in}{3.452838in}}%
\pgfpathlineto{\pgfqpoint{4.406989in}{3.456922in}}%
\pgfpathlineto{\pgfqpoint{4.408063in}{3.456121in}}%
\pgfpathlineto{\pgfqpoint{4.409137in}{3.459085in}}%
\pgfpathlineto{\pgfqpoint{4.412359in}{3.456442in}}%
\pgfpathlineto{\pgfqpoint{4.413432in}{3.459806in}}%
\pgfpathlineto{\pgfqpoint{4.414506in}{3.459966in}}%
\pgfpathlineto{\pgfqpoint{4.416654in}{3.465092in}}%
\pgfpathlineto{\pgfqpoint{4.420949in}{3.466133in}}%
\pgfpathlineto{\pgfqpoint{4.422023in}{3.467014in}}%
\pgfpathlineto{\pgfqpoint{4.423097in}{3.461007in}}%
\pgfpathlineto{\pgfqpoint{4.424171in}{3.462529in}}%
\pgfpathlineto{\pgfqpoint{4.427392in}{3.456842in}}%
\pgfpathlineto{\pgfqpoint{4.428466in}{3.456522in}}%
\pgfpathlineto{\pgfqpoint{4.429540in}{3.457884in}}%
\pgfpathlineto{\pgfqpoint{4.430614in}{3.455401in}}%
\pgfpathlineto{\pgfqpoint{4.431688in}{3.460687in}}%
\pgfpathlineto{\pgfqpoint{4.434909in}{3.460767in}}%
\pgfpathlineto{\pgfqpoint{4.435983in}{3.461968in}}%
\pgfpathlineto{\pgfqpoint{4.438131in}{3.460046in}}%
\pgfpathlineto{\pgfqpoint{4.439205in}{3.457483in}}%
\pgfpathlineto{\pgfqpoint{4.442426in}{3.457563in}}%
\pgfpathlineto{\pgfqpoint{4.443500in}{3.453639in}}%
\pgfpathlineto{\pgfqpoint{4.444574in}{3.452597in}}%
\pgfpathlineto{\pgfqpoint{4.445648in}{3.448432in}}%
\pgfpathlineto{\pgfqpoint{4.446721in}{3.451716in}}%
\pgfpathlineto{\pgfqpoint{4.449943in}{3.450915in}}%
\pgfpathlineto{\pgfqpoint{4.451017in}{3.452517in}}%
\pgfpathlineto{\pgfqpoint{4.452091in}{3.457803in}}%
\pgfpathlineto{\pgfqpoint{4.453164in}{3.456842in}}%
\pgfpathlineto{\pgfqpoint{4.454238in}{3.453398in}}%
\pgfpathlineto{\pgfqpoint{4.457460in}{3.451716in}}%
\pgfpathlineto{\pgfqpoint{4.458534in}{3.450114in}}%
\pgfpathlineto{\pgfqpoint{4.459607in}{3.450915in}}%
\pgfpathlineto{\pgfqpoint{4.461755in}{3.456602in}}%
\pgfpathlineto{\pgfqpoint{4.466050in}{3.454600in}}%
\pgfpathlineto{\pgfqpoint{4.467124in}{3.456041in}}%
\pgfpathlineto{\pgfqpoint{4.468198in}{3.455881in}}%
\pgfpathlineto{\pgfqpoint{4.469272in}{3.459405in}}%
\pgfpathlineto{\pgfqpoint{4.472494in}{3.459966in}}%
\pgfpathlineto{\pgfqpoint{4.474641in}{3.461808in}}%
\pgfpathlineto{\pgfqpoint{4.476789in}{3.463650in}}%
\pgfpathlineto{\pgfqpoint{4.480010in}{3.462449in}}%
\pgfpathlineto{\pgfqpoint{4.482158in}{3.459726in}}%
\pgfpathlineto{\pgfqpoint{4.483232in}{3.461728in}}%
\pgfpathlineto{\pgfqpoint{4.484306in}{3.460046in}}%
\pgfpathlineto{\pgfqpoint{4.488601in}{3.458684in}}%
\pgfpathlineto{\pgfqpoint{4.489675in}{3.456202in}}%
\pgfpathlineto{\pgfqpoint{4.490749in}{3.451716in}}%
\pgfpathlineto{\pgfqpoint{4.491823in}{3.450915in}}%
\pgfpathlineto{\pgfqpoint{4.495044in}{3.450435in}}%
\pgfpathlineto{\pgfqpoint{4.496118in}{3.451476in}}%
\pgfpathlineto{\pgfqpoint{4.498266in}{3.446750in}}%
\pgfpathlineto{\pgfqpoint{4.499339in}{3.450355in}}%
\pgfpathlineto{\pgfqpoint{4.504709in}{3.447551in}}%
\pgfpathlineto{\pgfqpoint{4.505782in}{3.452117in}}%
\pgfpathlineto{\pgfqpoint{4.506856in}{3.446270in}}%
\pgfpathlineto{\pgfqpoint{4.510078in}{3.440263in}}%
\pgfpathlineto{\pgfqpoint{4.511152in}{3.440743in}}%
\pgfpathlineto{\pgfqpoint{4.512226in}{3.439782in}}%
\pgfpathlineto{\pgfqpoint{4.513299in}{3.440904in}}%
\pgfpathlineto{\pgfqpoint{4.514373in}{3.440984in}}%
\pgfpathlineto{\pgfqpoint{4.518669in}{3.440984in}}%
\pgfpathlineto{\pgfqpoint{4.519742in}{3.439542in}}%
\pgfpathlineto{\pgfqpoint{4.521890in}{3.439382in}}%
\pgfpathlineto{\pgfqpoint{4.525112in}{3.437299in}}%
\pgfpathlineto{\pgfqpoint{4.526185in}{3.435617in}}%
\pgfpathlineto{\pgfqpoint{4.527259in}{3.436338in}}%
\pgfpathlineto{\pgfqpoint{4.528333in}{3.439141in}}%
\pgfpathlineto{\pgfqpoint{4.529407in}{3.436338in}}%
\pgfpathlineto{\pgfqpoint{4.533702in}{3.437540in}}%
\pgfpathlineto{\pgfqpoint{4.534776in}{3.435617in}}%
\pgfpathlineto{\pgfqpoint{4.535850in}{3.435137in}}%
\pgfpathlineto{\pgfqpoint{4.536924in}{3.436498in}}%
\pgfpathlineto{\pgfqpoint{4.540145in}{3.435377in}}%
\pgfpathlineto{\pgfqpoint{4.541219in}{3.431452in}}%
\pgfpathlineto{\pgfqpoint{4.544441in}{3.428729in}}%
\pgfpathlineto{\pgfqpoint{4.547662in}{3.430331in}}%
\pgfpathlineto{\pgfqpoint{4.548736in}{3.434336in}}%
\pgfpathlineto{\pgfqpoint{4.549810in}{3.430732in}}%
\pgfpathlineto{\pgfqpoint{4.550884in}{3.429931in}}%
\pgfpathlineto{\pgfqpoint{4.551958in}{3.427368in}}%
\pgfpathlineto{\pgfqpoint{4.556253in}{3.429450in}}%
\pgfpathlineto{\pgfqpoint{4.557327in}{3.428649in}}%
\pgfpathlineto{\pgfqpoint{4.559474in}{3.431773in}}%
\pgfpathlineto{\pgfqpoint{4.564844in}{3.429690in}}%
\pgfpathlineto{\pgfqpoint{4.565917in}{3.433775in}}%
\pgfpathlineto{\pgfqpoint{4.566991in}{3.432494in}}%
\pgfpathlineto{\pgfqpoint{4.570213in}{3.432494in}}%
\pgfpathlineto{\pgfqpoint{4.571287in}{3.431773in}}%
\pgfpathlineto{\pgfqpoint{4.572360in}{3.426487in}}%
\pgfpathlineto{\pgfqpoint{4.574508in}{3.425766in}}%
\pgfpathlineto{\pgfqpoint{4.577730in}{3.425525in}}%
\pgfpathlineto{\pgfqpoint{4.579877in}{3.420239in}}%
\pgfpathlineto{\pgfqpoint{4.585247in}{3.422161in}}%
\pgfpathlineto{\pgfqpoint{4.586320in}{3.417276in}}%
\pgfpathlineto{\pgfqpoint{4.587394in}{3.416315in}}%
\pgfpathlineto{\pgfqpoint{4.589542in}{3.418717in}}%
\pgfpathlineto{\pgfqpoint{4.593837in}{3.420079in}}%
\pgfpathlineto{\pgfqpoint{4.594911in}{3.421120in}}%
\pgfpathlineto{\pgfqpoint{4.595985in}{3.417436in}}%
\pgfpathlineto{\pgfqpoint{4.597059in}{3.417676in}}%
\pgfpathlineto{\pgfqpoint{4.600280in}{3.417516in}}%
\pgfpathlineto{\pgfqpoint{4.601354in}{3.420319in}}%
\pgfpathlineto{\pgfqpoint{4.602428in}{3.419518in}}%
\pgfpathlineto{\pgfqpoint{4.603502in}{3.421521in}}%
\pgfpathlineto{\pgfqpoint{4.607797in}{3.421040in}}%
\pgfpathlineto{\pgfqpoint{4.608871in}{3.424724in}}%
\pgfpathlineto{\pgfqpoint{4.609945in}{3.424885in}}%
\pgfpathlineto{\pgfqpoint{4.611019in}{3.423683in}}%
\pgfpathlineto{\pgfqpoint{4.612093in}{3.424004in}}%
\pgfpathlineto{\pgfqpoint{4.615314in}{3.424244in}}%
\pgfpathlineto{\pgfqpoint{4.616388in}{3.426807in}}%
\pgfpathlineto{\pgfqpoint{4.617462in}{3.427688in}}%
\pgfpathlineto{\pgfqpoint{4.618536in}{3.427207in}}%
\pgfpathlineto{\pgfqpoint{4.619609in}{3.425525in}}%
\pgfpathlineto{\pgfqpoint{4.626052in}{3.423843in}}%
\pgfpathlineto{\pgfqpoint{4.627126in}{3.422001in}}%
\pgfpathlineto{\pgfqpoint{4.630348in}{3.424564in}}%
\pgfpathlineto{\pgfqpoint{4.632495in}{3.429530in}}%
\pgfpathlineto{\pgfqpoint{4.633569in}{3.429130in}}%
\pgfpathlineto{\pgfqpoint{4.634643in}{3.428088in}}%
\pgfpathlineto{\pgfqpoint{4.637865in}{3.428649in}}%
\pgfpathlineto{\pgfqpoint{4.638938in}{3.427608in}}%
\pgfpathlineto{\pgfqpoint{4.642160in}{3.433535in}}%
\pgfpathlineto{\pgfqpoint{4.645382in}{3.433535in}}%
\pgfpathlineto{\pgfqpoint{4.646455in}{3.432814in}}%
\pgfpathlineto{\pgfqpoint{4.647529in}{3.434256in}}%
\pgfpathlineto{\pgfqpoint{4.648603in}{3.438020in}}%
\pgfpathlineto{\pgfqpoint{4.649677in}{3.428649in}}%
\pgfpathlineto{\pgfqpoint{4.655046in}{3.427768in}}%
\pgfpathlineto{\pgfqpoint{4.656120in}{3.426727in}}%
\pgfpathlineto{\pgfqpoint{4.660415in}{3.427368in}}%
\pgfpathlineto{\pgfqpoint{4.662563in}{3.429450in}}%
\pgfpathlineto{\pgfqpoint{4.663637in}{3.427288in}}%
\pgfpathlineto{\pgfqpoint{4.664711in}{3.429370in}}%
\pgfpathlineto{\pgfqpoint{4.667932in}{3.428088in}}%
\pgfpathlineto{\pgfqpoint{4.669006in}{3.429690in}}%
\pgfpathlineto{\pgfqpoint{4.671154in}{3.427288in}}%
\pgfpathlineto{\pgfqpoint{4.672227in}{3.428489in}}%
\pgfpathlineto{\pgfqpoint{4.675449in}{3.428729in}}%
\pgfpathlineto{\pgfqpoint{4.677597in}{3.430171in}}%
\pgfpathlineto{\pgfqpoint{4.684040in}{3.429130in}}%
\pgfpathlineto{\pgfqpoint{4.686187in}{3.423523in}}%
\pgfpathlineto{\pgfqpoint{4.687261in}{3.424644in}}%
\pgfpathlineto{\pgfqpoint{4.690483in}{3.423443in}}%
\pgfpathlineto{\pgfqpoint{4.692630in}{3.428008in}}%
\pgfpathlineto{\pgfqpoint{4.693704in}{3.427688in}}%
\pgfpathlineto{\pgfqpoint{4.694778in}{3.428729in}}%
\pgfpathlineto{\pgfqpoint{4.698000in}{3.430331in}}%
\pgfpathlineto{\pgfqpoint{4.701221in}{3.434416in}}%
\pgfpathlineto{\pgfqpoint{4.702295in}{3.432333in}}%
\pgfpathlineto{\pgfqpoint{4.706590in}{3.432734in}}%
\pgfpathlineto{\pgfqpoint{4.708738in}{3.432814in}}%
\pgfpathlineto{\pgfqpoint{4.709812in}{3.431052in}}%
\pgfpathlineto{\pgfqpoint{4.715181in}{3.429690in}}%
\pgfpathlineto{\pgfqpoint{4.716255in}{3.428809in}}%
\pgfpathlineto{\pgfqpoint{4.717329in}{3.430651in}}%
\pgfpathlineto{\pgfqpoint{4.720550in}{3.429530in}}%
\pgfpathlineto{\pgfqpoint{4.721624in}{3.433615in}}%
\pgfpathlineto{\pgfqpoint{4.723772in}{3.434015in}}%
\pgfpathlineto{\pgfqpoint{4.728067in}{3.431372in}}%
\pgfpathlineto{\pgfqpoint{4.729141in}{3.431533in}}%
\pgfpathlineto{\pgfqpoint{4.730215in}{3.428809in}}%
\pgfpathlineto{\pgfqpoint{4.731289in}{3.429450in}}%
\pgfpathlineto{\pgfqpoint{4.732362in}{3.428088in}}%
\pgfpathlineto{\pgfqpoint{4.735584in}{3.429130in}}%
\pgfpathlineto{\pgfqpoint{4.737732in}{3.436819in}}%
\pgfpathlineto{\pgfqpoint{4.738805in}{3.434015in}}%
\pgfpathlineto{\pgfqpoint{4.739879in}{3.432974in}}%
\pgfpathlineto{\pgfqpoint{4.743101in}{3.431052in}}%
\pgfpathlineto{\pgfqpoint{4.744175in}{3.434576in}}%
\pgfpathlineto{\pgfqpoint{4.745248in}{3.434656in}}%
\pgfpathlineto{\pgfqpoint{4.747396in}{3.438020in}}%
\pgfpathlineto{\pgfqpoint{4.750618in}{3.440904in}}%
\pgfpathlineto{\pgfqpoint{4.752765in}{3.446030in}}%
\pgfpathlineto{\pgfqpoint{4.753839in}{3.444187in}}%
\pgfpathlineto{\pgfqpoint{4.754913in}{3.444348in}}%
\pgfpathlineto{\pgfqpoint{4.762430in}{3.449554in}}%
\pgfpathlineto{\pgfqpoint{4.765651in}{3.449233in}}%
\pgfpathlineto{\pgfqpoint{4.767799in}{3.451476in}}%
\pgfpathlineto{\pgfqpoint{4.769947in}{3.453158in}}%
\pgfpathlineto{\pgfqpoint{4.774242in}{3.448753in}}%
\pgfpathlineto{\pgfqpoint{4.775316in}{3.451156in}}%
\pgfpathlineto{\pgfqpoint{4.776390in}{3.449394in}}%
\pgfpathlineto{\pgfqpoint{4.777464in}{3.450034in}}%
\pgfpathlineto{\pgfqpoint{4.782833in}{3.450275in}}%
\pgfpathlineto{\pgfqpoint{4.783907in}{3.450114in}}%
\pgfpathlineto{\pgfqpoint{4.784981in}{3.449233in}}%
\pgfpathlineto{\pgfqpoint{4.789276in}{3.450355in}}%
\pgfpathlineto{\pgfqpoint{4.790350in}{3.451636in}}%
\pgfpathlineto{\pgfqpoint{4.791424in}{3.451236in}}%
\pgfpathlineto{\pgfqpoint{4.792497in}{3.451636in}}%
\pgfpathlineto{\pgfqpoint{4.795719in}{3.458044in}}%
\pgfpathlineto{\pgfqpoint{4.796793in}{3.458845in}}%
\pgfpathlineto{\pgfqpoint{4.797867in}{3.454439in}}%
\pgfpathlineto{\pgfqpoint{4.800014in}{3.453719in}}%
\pgfpathlineto{\pgfqpoint{4.803236in}{3.456522in}}%
\pgfpathlineto{\pgfqpoint{4.805383in}{3.452838in}}%
\pgfpathlineto{\pgfqpoint{4.806457in}{3.456362in}}%
\pgfpathlineto{\pgfqpoint{4.807531in}{3.455961in}}%
\pgfpathlineto{\pgfqpoint{4.810753in}{3.457163in}}%
\pgfpathlineto{\pgfqpoint{4.811826in}{3.459565in}}%
\pgfpathlineto{\pgfqpoint{4.812900in}{3.456442in}}%
\pgfpathlineto{\pgfqpoint{4.813974in}{3.451076in}}%
\pgfpathlineto{\pgfqpoint{4.815048in}{3.451156in}}%
\pgfpathlineto{\pgfqpoint{4.818270in}{3.446991in}}%
\pgfpathlineto{\pgfqpoint{4.820417in}{3.449874in}}%
\pgfpathlineto{\pgfqpoint{4.821491in}{3.450114in}}%
\pgfpathlineto{\pgfqpoint{4.822565in}{3.451316in}}%
\pgfpathlineto{\pgfqpoint{4.827934in}{3.448192in}}%
\pgfpathlineto{\pgfqpoint{4.830082in}{3.450194in}}%
\pgfpathlineto{\pgfqpoint{4.833303in}{3.444428in}}%
\pgfpathlineto{\pgfqpoint{4.834377in}{3.444107in}}%
\pgfpathlineto{\pgfqpoint{4.835451in}{3.444828in}}%
\pgfpathlineto{\pgfqpoint{4.836525in}{3.449714in}}%
\pgfpathlineto{\pgfqpoint{4.837599in}{3.449153in}}%
\pgfpathlineto{\pgfqpoint{4.840820in}{3.449394in}}%
\pgfpathlineto{\pgfqpoint{4.841894in}{3.451476in}}%
\pgfpathlineto{\pgfqpoint{4.842968in}{3.450835in}}%
\pgfpathlineto{\pgfqpoint{4.844042in}{3.444748in}}%
\pgfpathlineto{\pgfqpoint{4.845115in}{3.443386in}}%
\pgfpathlineto{\pgfqpoint{4.849411in}{3.441464in}}%
\pgfpathlineto{\pgfqpoint{4.852632in}{3.444908in}}%
\pgfpathlineto{\pgfqpoint{4.855854in}{3.444748in}}%
\pgfpathlineto{\pgfqpoint{4.856928in}{3.445709in}}%
\pgfpathlineto{\pgfqpoint{4.865518in}{3.442586in}}%
\pgfpathlineto{\pgfqpoint{4.866592in}{3.440904in}}%
\pgfpathlineto{\pgfqpoint{4.870888in}{3.442185in}}%
\pgfpathlineto{\pgfqpoint{4.871961in}{3.441384in}}%
\pgfpathlineto{\pgfqpoint{4.873035in}{3.441785in}}%
\pgfpathlineto{\pgfqpoint{4.874109in}{3.438100in}}%
\pgfpathlineto{\pgfqpoint{4.875183in}{3.437299in}}%
\pgfpathlineto{\pgfqpoint{4.878404in}{3.439782in}}%
\pgfpathlineto{\pgfqpoint{4.879478in}{3.441384in}}%
\pgfpathlineto{\pgfqpoint{4.880552in}{3.439061in}}%
\pgfpathlineto{\pgfqpoint{4.881626in}{3.439622in}}%
\pgfpathlineto{\pgfqpoint{4.882700in}{3.440904in}}%
\pgfpathlineto{\pgfqpoint{4.886995in}{3.439222in}}%
\pgfpathlineto{\pgfqpoint{4.888069in}{3.439862in}}%
\pgfpathlineto{\pgfqpoint{4.890217in}{3.437780in}}%
\pgfpathlineto{\pgfqpoint{4.894512in}{3.438421in}}%
\pgfpathlineto{\pgfqpoint{4.895586in}{3.442185in}}%
\pgfpathlineto{\pgfqpoint{4.896660in}{3.441945in}}%
\pgfpathlineto{\pgfqpoint{4.902029in}{3.444508in}}%
\pgfpathlineto{\pgfqpoint{4.904177in}{3.442185in}}%
\pgfpathlineto{\pgfqpoint{4.905250in}{3.447631in}}%
\pgfpathlineto{\pgfqpoint{4.908472in}{3.447231in}}%
\pgfpathlineto{\pgfqpoint{4.910620in}{3.450675in}}%
\pgfpathlineto{\pgfqpoint{4.911693in}{3.450755in}}%
\pgfpathlineto{\pgfqpoint{4.912767in}{3.449794in}}%
\pgfpathlineto{\pgfqpoint{4.915989in}{3.448833in}}%
\pgfpathlineto{\pgfqpoint{4.918136in}{3.449313in}}%
\pgfpathlineto{\pgfqpoint{4.920284in}{3.444508in}}%
\pgfpathlineto{\pgfqpoint{4.925653in}{3.443867in}}%
\pgfpathlineto{\pgfqpoint{4.926727in}{3.441945in}}%
\pgfpathlineto{\pgfqpoint{4.927801in}{3.443386in}}%
\pgfpathlineto{\pgfqpoint{4.931023in}{3.443467in}}%
\pgfpathlineto{\pgfqpoint{4.932096in}{3.441144in}}%
\pgfpathlineto{\pgfqpoint{4.933170in}{3.441865in}}%
\pgfpathlineto{\pgfqpoint{4.934244in}{3.445309in}}%
\pgfpathlineto{\pgfqpoint{4.935318in}{3.446350in}}%
\pgfpathlineto{\pgfqpoint{4.938539in}{3.447631in}}%
\pgfpathlineto{\pgfqpoint{4.939613in}{3.447231in}}%
\pgfpathlineto{\pgfqpoint{4.941761in}{3.450835in}}%
\pgfpathlineto{\pgfqpoint{4.942835in}{3.450595in}}%
\pgfpathlineto{\pgfqpoint{4.946056in}{3.452277in}}%
\pgfpathlineto{\pgfqpoint{4.947130in}{3.451796in}}%
\pgfpathlineto{\pgfqpoint{4.948204in}{3.449634in}}%
\pgfpathlineto{\pgfqpoint{4.949278in}{3.449874in}}%
\pgfpathlineto{\pgfqpoint{4.950352in}{3.452998in}}%
\pgfpathlineto{\pgfqpoint{4.953573in}{3.453799in}}%
\pgfpathlineto{\pgfqpoint{4.954647in}{3.454840in}}%
\pgfpathlineto{\pgfqpoint{4.956795in}{3.454520in}}%
\pgfpathlineto{\pgfqpoint{4.957869in}{3.453558in}}%
\pgfpathlineto{\pgfqpoint{4.962164in}{3.453078in}}%
\pgfpathlineto{\pgfqpoint{4.963238in}{3.454600in}}%
\pgfpathlineto{\pgfqpoint{4.964312in}{3.453318in}}%
\pgfpathlineto{\pgfqpoint{4.968607in}{3.454039in}}%
\pgfpathlineto{\pgfqpoint{4.970755in}{3.456922in}}%
\pgfpathlineto{\pgfqpoint{4.971828in}{3.455561in}}%
\pgfpathlineto{\pgfqpoint{4.972902in}{3.456442in}}%
\pgfpathlineto{\pgfqpoint{4.978271in}{3.456682in}}%
\pgfpathlineto{\pgfqpoint{4.979345in}{3.461247in}}%
\pgfpathlineto{\pgfqpoint{4.980419in}{3.451957in}}%
\pgfpathlineto{\pgfqpoint{4.984714in}{3.450515in}}%
\pgfpathlineto{\pgfqpoint{4.985788in}{3.452197in}}%
\pgfpathlineto{\pgfqpoint{4.991158in}{3.450835in}}%
\pgfpathlineto{\pgfqpoint{4.992231in}{3.449794in}}%
\pgfpathlineto{\pgfqpoint{4.993305in}{3.451076in}}%
\pgfpathlineto{\pgfqpoint{4.995453in}{3.458444in}}%
\pgfpathlineto{\pgfqpoint{4.998674in}{3.459165in}}%
\pgfpathlineto{\pgfqpoint{4.999748in}{3.458444in}}%
\pgfpathlineto{\pgfqpoint{5.000822in}{3.455721in}}%
\pgfpathlineto{\pgfqpoint{5.001896in}{3.456041in}}%
\pgfpathlineto{\pgfqpoint{5.002970in}{3.454840in}}%
\pgfpathlineto{\pgfqpoint{5.006191in}{3.455961in}}%
\pgfpathlineto{\pgfqpoint{5.007265in}{3.457323in}}%
\pgfpathlineto{\pgfqpoint{5.008339in}{3.460687in}}%
\pgfpathlineto{\pgfqpoint{5.010487in}{3.460046in}}%
\pgfpathlineto{\pgfqpoint{5.014782in}{3.462609in}}%
\pgfpathlineto{\pgfqpoint{5.016930in}{3.464451in}}%
\pgfpathlineto{\pgfqpoint{5.018003in}{3.463730in}}%
\pgfpathlineto{\pgfqpoint{5.022299in}{3.466213in}}%
\pgfpathlineto{\pgfqpoint{5.028742in}{3.462769in}}%
\pgfpathlineto{\pgfqpoint{5.029816in}{3.463650in}}%
\pgfpathlineto{\pgfqpoint{5.031963in}{3.461808in}}%
\pgfpathlineto{\pgfqpoint{5.033037in}{3.463250in}}%
\pgfpathlineto{\pgfqpoint{5.036259in}{3.463090in}}%
\pgfpathlineto{\pgfqpoint{5.037333in}{3.469898in}}%
\pgfpathlineto{\pgfqpoint{5.038406in}{3.471019in}}%
\pgfpathlineto{\pgfqpoint{5.039480in}{3.470138in}}%
\pgfpathlineto{\pgfqpoint{5.040554in}{3.465412in}}%
\pgfpathlineto{\pgfqpoint{5.044849in}{3.463650in}}%
\pgfpathlineto{\pgfqpoint{5.048071in}{3.458364in}}%
\pgfpathlineto{\pgfqpoint{5.051292in}{3.457803in}}%
\pgfpathlineto{\pgfqpoint{5.052366in}{3.459005in}}%
\pgfpathlineto{\pgfqpoint{5.053440in}{3.457483in}}%
\pgfpathlineto{\pgfqpoint{5.054514in}{3.457884in}}%
\pgfpathlineto{\pgfqpoint{5.059883in}{3.451636in}}%
\pgfpathlineto{\pgfqpoint{5.060957in}{3.451156in}}%
\pgfpathlineto{\pgfqpoint{5.063105in}{3.455881in}}%
\pgfpathlineto{\pgfqpoint{5.066326in}{3.456442in}}%
\pgfpathlineto{\pgfqpoint{5.069548in}{3.459245in}}%
\pgfpathlineto{\pgfqpoint{5.070622in}{3.459646in}}%
\pgfpathlineto{\pgfqpoint{5.074917in}{3.459646in}}%
\pgfpathlineto{\pgfqpoint{5.075991in}{3.457884in}}%
\pgfpathlineto{\pgfqpoint{5.077065in}{3.459485in}}%
\pgfpathlineto{\pgfqpoint{5.078138in}{3.457323in}}%
\pgfpathlineto{\pgfqpoint{5.084581in}{3.459646in}}%
\pgfpathlineto{\pgfqpoint{5.085655in}{3.459646in}}%
\pgfpathlineto{\pgfqpoint{5.088877in}{3.457884in}}%
\pgfpathlineto{\pgfqpoint{5.089951in}{3.458604in}}%
\pgfpathlineto{\pgfqpoint{5.091024in}{3.457884in}}%
\pgfpathlineto{\pgfqpoint{5.092098in}{3.458765in}}%
\pgfpathlineto{\pgfqpoint{5.093172in}{3.458845in}}%
\pgfpathlineto{\pgfqpoint{5.096394in}{3.460206in}}%
\pgfpathlineto{\pgfqpoint{5.097468in}{3.459405in}}%
\pgfpathlineto{\pgfqpoint{5.098541in}{3.459646in}}%
\pgfpathlineto{\pgfqpoint{5.100689in}{3.457884in}}%
\pgfpathlineto{\pgfqpoint{5.104984in}{3.460046in}}%
\pgfpathlineto{\pgfqpoint{5.106058in}{3.461408in}}%
\pgfpathlineto{\pgfqpoint{5.107132in}{3.464211in}}%
\pgfpathlineto{\pgfqpoint{5.108206in}{3.462449in}}%
\pgfpathlineto{\pgfqpoint{5.111427in}{3.462129in}}%
\pgfpathlineto{\pgfqpoint{5.112501in}{3.464451in}}%
\pgfpathlineto{\pgfqpoint{5.114649in}{3.463410in}}%
\pgfpathlineto{\pgfqpoint{5.115723in}{3.465573in}}%
\pgfpathlineto{\pgfqpoint{5.118944in}{3.466854in}}%
\pgfpathlineto{\pgfqpoint{5.120018in}{3.468055in}}%
\pgfpathlineto{\pgfqpoint{5.121092in}{3.467415in}}%
\pgfpathlineto{\pgfqpoint{5.122166in}{3.470939in}}%
\pgfpathlineto{\pgfqpoint{5.123240in}{3.469257in}}%
\pgfpathlineto{\pgfqpoint{5.126461in}{3.471500in}}%
\pgfpathlineto{\pgfqpoint{5.127535in}{3.474383in}}%
\pgfpathlineto{\pgfqpoint{5.128609in}{3.474864in}}%
\pgfpathlineto{\pgfqpoint{5.130757in}{3.469337in}}%
\pgfpathlineto{\pgfqpoint{5.135052in}{3.473422in}}%
\pgfpathlineto{\pgfqpoint{5.137200in}{3.475264in}}%
\pgfpathlineto{\pgfqpoint{5.142569in}{3.473902in}}%
\pgfpathlineto{\pgfqpoint{5.144716in}{3.473262in}}%
\pgfpathlineto{\pgfqpoint{5.145790in}{3.469898in}}%
\pgfpathlineto{\pgfqpoint{5.149012in}{3.470378in}}%
\pgfpathlineto{\pgfqpoint{5.151159in}{3.472861in}}%
\pgfpathlineto{\pgfqpoint{5.152233in}{3.471259in}}%
\pgfpathlineto{\pgfqpoint{5.153307in}{3.470939in}}%
\pgfpathlineto{\pgfqpoint{5.157602in}{3.471019in}}%
\pgfpathlineto{\pgfqpoint{5.158676in}{3.472220in}}%
\pgfpathlineto{\pgfqpoint{5.160824in}{3.471740in}}%
\pgfpathlineto{\pgfqpoint{5.166193in}{3.471980in}}%
\pgfpathlineto{\pgfqpoint{5.168341in}{3.468456in}}%
\pgfpathlineto{\pgfqpoint{5.173710in}{3.471019in}}%
\pgfpathlineto{\pgfqpoint{5.174784in}{3.470699in}}%
\pgfpathlineto{\pgfqpoint{5.175858in}{3.471660in}}%
\pgfpathlineto{\pgfqpoint{5.182301in}{3.476305in}}%
\pgfpathlineto{\pgfqpoint{5.183375in}{3.478788in}}%
\pgfpathlineto{\pgfqpoint{5.186596in}{3.478388in}}%
\pgfpathlineto{\pgfqpoint{5.187670in}{3.481111in}}%
\pgfpathlineto{\pgfqpoint{5.188744in}{3.480710in}}%
\pgfpathlineto{\pgfqpoint{5.189818in}{3.481111in}}%
\pgfpathlineto{\pgfqpoint{5.190891in}{3.484635in}}%
\pgfpathlineto{\pgfqpoint{5.194113in}{3.482873in}}%
\pgfpathlineto{\pgfqpoint{5.195187in}{3.485116in}}%
\pgfpathlineto{\pgfqpoint{5.196261in}{3.483033in}}%
\pgfpathlineto{\pgfqpoint{5.197335in}{3.483193in}}%
\pgfpathlineto{\pgfqpoint{5.198408in}{3.496409in}}%
\pgfpathlineto{\pgfqpoint{5.201630in}{3.497370in}}%
\pgfpathlineto{\pgfqpoint{5.203778in}{3.496809in}}%
\pgfpathlineto{\pgfqpoint{5.205925in}{3.498732in}}%
\pgfpathlineto{\pgfqpoint{5.209147in}{3.498892in}}%
\pgfpathlineto{\pgfqpoint{5.211294in}{3.503377in}}%
\pgfpathlineto{\pgfqpoint{5.212368in}{3.502736in}}%
\pgfpathlineto{\pgfqpoint{5.213442in}{3.503778in}}%
\pgfpathlineto{\pgfqpoint{5.216664in}{3.503537in}}%
\pgfpathlineto{\pgfqpoint{5.219885in}{3.505219in}}%
\pgfpathlineto{\pgfqpoint{5.224180in}{3.504418in}}%
\pgfpathlineto{\pgfqpoint{5.225254in}{3.502816in}}%
\pgfpathlineto{\pgfqpoint{5.226328in}{3.503457in}}%
\pgfpathlineto{\pgfqpoint{5.227402in}{3.506020in}}%
\pgfpathlineto{\pgfqpoint{5.228476in}{3.505940in}}%
\pgfpathlineto{\pgfqpoint{5.231697in}{3.507622in}}%
\pgfpathlineto{\pgfqpoint{5.232771in}{3.509144in}}%
\pgfpathlineto{\pgfqpoint{5.233845in}{3.529488in}}%
\pgfpathlineto{\pgfqpoint{5.234919in}{3.522920in}}%
\pgfpathlineto{\pgfqpoint{5.235993in}{3.522920in}}%
\pgfpathlineto{\pgfqpoint{5.239214in}{3.525403in}}%
\pgfpathlineto{\pgfqpoint{5.240288in}{3.530449in}}%
\pgfpathlineto{\pgfqpoint{5.242436in}{3.526685in}}%
\pgfpathlineto{\pgfqpoint{5.247805in}{3.526284in}}%
\pgfpathlineto{\pgfqpoint{5.248879in}{3.527405in}}%
\pgfpathlineto{\pgfqpoint{5.249953in}{3.524202in}}%
\pgfpathlineto{\pgfqpoint{5.251026in}{3.523160in}}%
\pgfpathlineto{\pgfqpoint{5.254248in}{3.525323in}}%
\pgfpathlineto{\pgfqpoint{5.255322in}{3.518355in}}%
\pgfpathlineto{\pgfqpoint{5.256396in}{3.518515in}}%
\pgfpathlineto{\pgfqpoint{5.258543in}{3.516833in}}%
\pgfpathlineto{\pgfqpoint{5.262839in}{3.520517in}}%
\pgfpathlineto{\pgfqpoint{5.263912in}{3.527245in}}%
\pgfpathlineto{\pgfqpoint{5.264986in}{3.526124in}}%
\pgfpathlineto{\pgfqpoint{5.266060in}{3.527726in}}%
\pgfpathlineto{\pgfqpoint{5.269282in}{3.529408in}}%
\pgfpathlineto{\pgfqpoint{5.270356in}{3.528927in}}%
\pgfpathlineto{\pgfqpoint{5.271429in}{3.530048in}}%
\pgfpathlineto{\pgfqpoint{5.272503in}{3.534534in}}%
\pgfpathlineto{\pgfqpoint{5.273577in}{3.533092in}}%
\pgfpathlineto{\pgfqpoint{5.280020in}{3.531090in}}%
\pgfpathlineto{\pgfqpoint{5.281094in}{3.532932in}}%
\pgfpathlineto{\pgfqpoint{5.286463in}{3.530529in}}%
\pgfpathlineto{\pgfqpoint{5.287537in}{3.532852in}}%
\pgfpathlineto{\pgfqpoint{5.291832in}{3.535815in}}%
\pgfpathlineto{\pgfqpoint{5.292906in}{3.532852in}}%
\pgfpathlineto{\pgfqpoint{5.293980in}{3.533653in}}%
\pgfpathlineto{\pgfqpoint{5.295054in}{3.533653in}}%
\pgfpathlineto{\pgfqpoint{5.296128in}{3.530849in}}%
\pgfpathlineto{\pgfqpoint{5.299349in}{3.530289in}}%
\pgfpathlineto{\pgfqpoint{5.300423in}{3.533012in}}%
\pgfpathlineto{\pgfqpoint{5.301497in}{3.533332in}}%
\pgfpathlineto{\pgfqpoint{5.302571in}{3.534694in}}%
\pgfpathlineto{\pgfqpoint{5.303645in}{3.532211in}}%
\pgfpathlineto{\pgfqpoint{5.306866in}{3.531490in}}%
\pgfpathlineto{\pgfqpoint{5.307940in}{3.529488in}}%
\pgfpathlineto{\pgfqpoint{5.309014in}{3.531730in}}%
\pgfpathlineto{\pgfqpoint{5.310088in}{3.527485in}}%
\pgfpathlineto{\pgfqpoint{5.311161in}{3.528366in}}%
\pgfpathlineto{\pgfqpoint{5.314383in}{3.532772in}}%
\pgfpathlineto{\pgfqpoint{5.315457in}{3.532211in}}%
\pgfpathlineto{\pgfqpoint{5.317604in}{3.523160in}}%
\pgfpathlineto{\pgfqpoint{5.318678in}{3.526765in}}%
\pgfpathlineto{\pgfqpoint{5.321900in}{3.527325in}}%
\pgfpathlineto{\pgfqpoint{5.322974in}{3.522840in}}%
\pgfpathlineto{\pgfqpoint{5.324047in}{3.528447in}}%
\pgfpathlineto{\pgfqpoint{5.325121in}{3.523881in}}%
\pgfpathlineto{\pgfqpoint{5.326195in}{3.512027in}}%
\pgfpathlineto{\pgfqpoint{5.329417in}{3.508904in}}%
\pgfpathlineto{\pgfqpoint{5.330490in}{3.513629in}}%
\pgfpathlineto{\pgfqpoint{5.332638in}{3.504578in}}%
\pgfpathlineto{\pgfqpoint{5.333712in}{3.508343in}}%
\pgfpathlineto{\pgfqpoint{5.336934in}{3.509705in}}%
\pgfpathlineto{\pgfqpoint{5.338007in}{3.516833in}}%
\pgfpathlineto{\pgfqpoint{5.339081in}{3.514510in}}%
\pgfpathlineto{\pgfqpoint{5.341229in}{3.520838in}}%
\pgfpathlineto{\pgfqpoint{5.344450in}{3.520998in}}%
\pgfpathlineto{\pgfqpoint{5.345524in}{3.524762in}}%
\pgfpathlineto{\pgfqpoint{5.346598in}{3.525964in}}%
\pgfpathlineto{\pgfqpoint{5.347672in}{3.516673in}}%
\pgfpathlineto{\pgfqpoint{5.348746in}{3.526604in}}%
\pgfpathlineto{\pgfqpoint{5.351967in}{3.528687in}}%
\pgfpathlineto{\pgfqpoint{5.353041in}{3.530289in}}%
\pgfpathlineto{\pgfqpoint{5.354115in}{3.526444in}}%
\pgfpathlineto{\pgfqpoint{5.355189in}{3.526845in}}%
\pgfpathlineto{\pgfqpoint{5.356263in}{3.525163in}}%
\pgfpathlineto{\pgfqpoint{5.359484in}{3.522920in}}%
\pgfpathlineto{\pgfqpoint{5.361632in}{3.523721in}}%
\pgfpathlineto{\pgfqpoint{5.363779in}{3.527726in}}%
\pgfpathlineto{\pgfqpoint{5.367001in}{3.529808in}}%
\pgfpathlineto{\pgfqpoint{5.368075in}{3.533092in}}%
\pgfpathlineto{\pgfqpoint{5.369149in}{3.530529in}}%
\pgfpathlineto{\pgfqpoint{5.370223in}{3.541742in}}%
\pgfpathlineto{\pgfqpoint{5.371296in}{3.539259in}}%
\pgfpathlineto{\pgfqpoint{5.374518in}{3.543825in}}%
\pgfpathlineto{\pgfqpoint{5.375592in}{3.544305in}}%
\pgfpathlineto{\pgfqpoint{5.376666in}{3.548390in}}%
\pgfpathlineto{\pgfqpoint{5.378813in}{3.550873in}}%
\pgfpathlineto{\pgfqpoint{5.382035in}{3.550312in}}%
\pgfpathlineto{\pgfqpoint{5.383109in}{3.553276in}}%
\pgfpathlineto{\pgfqpoint{5.384182in}{3.552154in}}%
\pgfpathlineto{\pgfqpoint{5.385256in}{3.552315in}}%
\pgfpathlineto{\pgfqpoint{5.386330in}{3.553836in}}%
\pgfpathlineto{\pgfqpoint{5.389552in}{3.550473in}}%
\pgfpathlineto{\pgfqpoint{5.391699in}{3.545026in}}%
\pgfpathlineto{\pgfqpoint{5.392773in}{3.547028in}}%
\pgfpathlineto{\pgfqpoint{5.393847in}{3.543745in}}%
\pgfpathlineto{\pgfqpoint{5.397068in}{3.541502in}}%
\pgfpathlineto{\pgfqpoint{5.398142in}{3.539019in}}%
\pgfpathlineto{\pgfqpoint{5.400290in}{3.549271in}}%
\pgfpathlineto{\pgfqpoint{5.401364in}{3.544706in}}%
\pgfpathlineto{\pgfqpoint{5.405659in}{3.552154in}}%
\pgfpathlineto{\pgfqpoint{5.406733in}{3.552154in}}%
\pgfpathlineto{\pgfqpoint{5.408881in}{3.552955in}}%
\pgfpathlineto{\pgfqpoint{5.412102in}{3.550392in}}%
\pgfpathlineto{\pgfqpoint{5.414250in}{3.544145in}}%
\pgfpathlineto{\pgfqpoint{5.416398in}{3.544626in}}%
\pgfpathlineto{\pgfqpoint{5.419619in}{3.541742in}}%
\pgfpathlineto{\pgfqpoint{5.420693in}{3.537017in}}%
\pgfpathlineto{\pgfqpoint{5.422841in}{3.546948in}}%
\pgfpathlineto{\pgfqpoint{5.423914in}{3.547429in}}%
\pgfpathlineto{\pgfqpoint{5.428210in}{3.545587in}}%
\pgfpathlineto{\pgfqpoint{5.430357in}{3.543424in}}%
\pgfpathlineto{\pgfqpoint{5.431431in}{3.545266in}}%
\pgfpathlineto{\pgfqpoint{5.435727in}{3.542703in}}%
\pgfpathlineto{\pgfqpoint{5.437874in}{3.548470in}}%
\pgfpathlineto{\pgfqpoint{5.438948in}{3.545266in}}%
\pgfpathlineto{\pgfqpoint{5.442170in}{3.540781in}}%
\pgfpathlineto{\pgfqpoint{5.443244in}{3.529408in}}%
\pgfpathlineto{\pgfqpoint{5.444317in}{3.526524in}}%
\pgfpathlineto{\pgfqpoint{5.445391in}{3.529568in}}%
\pgfpathlineto{\pgfqpoint{5.446465in}{3.521398in}}%
\pgfpathlineto{\pgfqpoint{5.449687in}{3.525643in}}%
\pgfpathlineto{\pgfqpoint{5.450760in}{3.525964in}}%
\pgfpathlineto{\pgfqpoint{5.451834in}{3.527005in}}%
\pgfpathlineto{\pgfqpoint{5.452908in}{3.529408in}}%
\pgfpathlineto{\pgfqpoint{5.453982in}{3.524842in}}%
\pgfpathlineto{\pgfqpoint{5.457203in}{3.522279in}}%
\pgfpathlineto{\pgfqpoint{5.458277in}{3.527646in}}%
\pgfpathlineto{\pgfqpoint{5.459351in}{3.526604in}}%
\pgfpathlineto{\pgfqpoint{5.460425in}{3.530689in}}%
\pgfpathlineto{\pgfqpoint{5.461499in}{3.532371in}}%
\pgfpathlineto{\pgfqpoint{5.465794in}{3.535014in}}%
\pgfpathlineto{\pgfqpoint{5.466868in}{3.531730in}}%
\pgfpathlineto{\pgfqpoint{5.467942in}{3.531250in}}%
\pgfpathlineto{\pgfqpoint{5.469016in}{3.532692in}}%
\pgfpathlineto{\pgfqpoint{5.472237in}{3.528126in}}%
\pgfpathlineto{\pgfqpoint{5.473311in}{3.532692in}}%
\pgfpathlineto{\pgfqpoint{5.476533in}{3.524522in}}%
\pgfpathlineto{\pgfqpoint{5.479754in}{3.530209in}}%
\pgfpathlineto{\pgfqpoint{5.481902in}{3.530609in}}%
\pgfpathlineto{\pgfqpoint{5.484049in}{3.524121in}}%
\pgfpathlineto{\pgfqpoint{5.487271in}{3.520758in}}%
\pgfpathlineto{\pgfqpoint{5.488345in}{3.513549in}}%
\pgfpathlineto{\pgfqpoint{5.489419in}{3.518034in}}%
\pgfpathlineto{\pgfqpoint{5.490492in}{3.507222in}}%
\pgfpathlineto{\pgfqpoint{5.491566in}{3.508103in}}%
\pgfpathlineto{\pgfqpoint{5.494788in}{3.507462in}}%
\pgfpathlineto{\pgfqpoint{5.495862in}{3.505780in}}%
\pgfpathlineto{\pgfqpoint{5.496935in}{3.507862in}}%
\pgfpathlineto{\pgfqpoint{5.498009in}{3.506821in}}%
\pgfpathlineto{\pgfqpoint{5.499083in}{3.510826in}}%
\pgfpathlineto{\pgfqpoint{5.502305in}{3.510025in}}%
\pgfpathlineto{\pgfqpoint{5.503378in}{3.507141in}}%
\pgfpathlineto{\pgfqpoint{5.504452in}{3.500814in}}%
\pgfpathlineto{\pgfqpoint{5.505526in}{3.502176in}}%
\pgfpathlineto{\pgfqpoint{5.506600in}{3.515712in}}%
\pgfpathlineto{\pgfqpoint{5.510895in}{3.510505in}}%
\pgfpathlineto{\pgfqpoint{5.511969in}{3.507302in}}%
\pgfpathlineto{\pgfqpoint{5.513043in}{3.507302in}}%
\pgfpathlineto{\pgfqpoint{5.517338in}{3.508904in}}%
\pgfpathlineto{\pgfqpoint{5.518412in}{3.510505in}}%
\pgfpathlineto{\pgfqpoint{5.519486in}{3.510826in}}%
\pgfpathlineto{\pgfqpoint{5.520560in}{3.510345in}}%
\pgfpathlineto{\pgfqpoint{5.521634in}{3.515231in}}%
\pgfpathlineto{\pgfqpoint{5.524855in}{3.513789in}}%
\pgfpathlineto{\pgfqpoint{5.525929in}{3.512107in}}%
\pgfpathlineto{\pgfqpoint{5.527003in}{3.521558in}}%
\pgfpathlineto{\pgfqpoint{5.528077in}{3.521799in}}%
\pgfpathlineto{\pgfqpoint{5.529151in}{3.518995in}}%
\pgfpathlineto{\pgfqpoint{5.532372in}{3.520838in}}%
\pgfpathlineto{\pgfqpoint{5.533446in}{3.518755in}}%
\pgfpathlineto{\pgfqpoint{5.534520in}{3.520597in}}%
\pgfpathlineto{\pgfqpoint{5.536667in}{3.516272in}}%
\pgfpathlineto{\pgfqpoint{5.539889in}{3.519236in}}%
\pgfpathlineto{\pgfqpoint{5.540963in}{3.522920in}}%
\pgfpathlineto{\pgfqpoint{5.542037in}{3.521959in}}%
\pgfpathlineto{\pgfqpoint{5.543111in}{3.519556in}}%
\pgfpathlineto{\pgfqpoint{5.544184in}{3.525723in}}%
\pgfpathlineto{\pgfqpoint{5.547406in}{3.525803in}}%
\pgfpathlineto{\pgfqpoint{5.549554in}{3.518915in}}%
\pgfpathlineto{\pgfqpoint{5.550627in}{3.518995in}}%
\pgfpathlineto{\pgfqpoint{5.551701in}{3.523000in}}%
\pgfpathlineto{\pgfqpoint{5.554923in}{3.522199in}}%
\pgfpathlineto{\pgfqpoint{5.555997in}{3.519076in}}%
\pgfpathlineto{\pgfqpoint{5.558144in}{3.524202in}}%
\pgfpathlineto{\pgfqpoint{5.559218in}{3.524362in}}%
\pgfpathlineto{\pgfqpoint{5.562440in}{3.527325in}}%
\pgfpathlineto{\pgfqpoint{5.563513in}{3.525483in}}%
\pgfpathlineto{\pgfqpoint{5.565661in}{3.528366in}}%
\pgfpathlineto{\pgfqpoint{5.571030in}{3.525163in}}%
\pgfpathlineto{\pgfqpoint{5.573178in}{3.531570in}}%
\pgfpathlineto{\pgfqpoint{5.574252in}{3.534774in}}%
\pgfpathlineto{\pgfqpoint{5.577473in}{3.530929in}}%
\pgfpathlineto{\pgfqpoint{5.580695in}{3.519556in}}%
\pgfpathlineto{\pgfqpoint{5.581769in}{3.516192in}}%
\pgfpathlineto{\pgfqpoint{5.584990in}{3.512348in}}%
\pgfpathlineto{\pgfqpoint{5.586064in}{3.512027in}}%
\pgfpathlineto{\pgfqpoint{5.587138in}{3.516032in}}%
\pgfpathlineto{\pgfqpoint{5.588212in}{3.516272in}}%
\pgfpathlineto{\pgfqpoint{5.589286in}{3.512508in}}%
\pgfpathlineto{\pgfqpoint{5.592507in}{3.512988in}}%
\pgfpathlineto{\pgfqpoint{5.594655in}{3.516993in}}%
\pgfpathlineto{\pgfqpoint{5.595729in}{3.520037in}}%
\pgfpathlineto{\pgfqpoint{5.596802in}{3.517794in}}%
\pgfpathlineto{\pgfqpoint{5.600024in}{3.519156in}}%
\pgfpathlineto{\pgfqpoint{5.602172in}{3.516673in}}%
\pgfpathlineto{\pgfqpoint{5.603245in}{3.517233in}}%
\pgfpathlineto{\pgfqpoint{5.604319in}{3.510345in}}%
\pgfpathlineto{\pgfqpoint{5.607541in}{3.505860in}}%
\pgfpathlineto{\pgfqpoint{5.608615in}{3.506100in}}%
\pgfpathlineto{\pgfqpoint{5.609688in}{3.504418in}}%
\pgfpathlineto{\pgfqpoint{5.610762in}{3.507061in}}%
\pgfpathlineto{\pgfqpoint{5.616132in}{3.502416in}}%
\pgfpathlineto{\pgfqpoint{5.618279in}{3.495528in}}%
\pgfpathlineto{\pgfqpoint{5.619353in}{3.497210in}}%
\pgfpathlineto{\pgfqpoint{5.622575in}{3.501215in}}%
\pgfpathlineto{\pgfqpoint{5.623648in}{3.500654in}}%
\pgfpathlineto{\pgfqpoint{5.624722in}{3.500894in}}%
\pgfpathlineto{\pgfqpoint{5.625796in}{3.502416in}}%
\pgfpathlineto{\pgfqpoint{5.626870in}{3.499372in}}%
\pgfpathlineto{\pgfqpoint{5.630091in}{3.496729in}}%
\pgfpathlineto{\pgfqpoint{5.631165in}{3.494006in}}%
\pgfpathlineto{\pgfqpoint{5.632239in}{3.493205in}}%
\pgfpathlineto{\pgfqpoint{5.633313in}{3.493125in}}%
\pgfpathlineto{\pgfqpoint{5.634387in}{3.489361in}}%
\pgfpathlineto{\pgfqpoint{5.637608in}{3.491363in}}%
\pgfpathlineto{\pgfqpoint{5.638682in}{3.495688in}}%
\pgfpathlineto{\pgfqpoint{5.639756in}{3.496088in}}%
\pgfpathlineto{\pgfqpoint{5.640830in}{3.495368in}}%
\pgfpathlineto{\pgfqpoint{5.646199in}{3.496889in}}%
\pgfpathlineto{\pgfqpoint{5.647273in}{3.498571in}}%
\pgfpathlineto{\pgfqpoint{5.649421in}{3.496889in}}%
\pgfpathlineto{\pgfqpoint{5.652642in}{3.502336in}}%
\pgfpathlineto{\pgfqpoint{5.653716in}{3.497530in}}%
\pgfpathlineto{\pgfqpoint{5.654790in}{3.500974in}}%
\pgfpathlineto{\pgfqpoint{5.655864in}{3.496809in}}%
\pgfpathlineto{\pgfqpoint{5.656937in}{3.497851in}}%
\pgfpathlineto{\pgfqpoint{5.660159in}{3.498251in}}%
\pgfpathlineto{\pgfqpoint{5.661233in}{3.497050in}}%
\pgfpathlineto{\pgfqpoint{5.663380in}{3.487198in}}%
\pgfpathlineto{\pgfqpoint{5.664454in}{3.480230in}}%
\pgfpathlineto{\pgfqpoint{5.667676in}{3.477987in}}%
\pgfpathlineto{\pgfqpoint{5.668750in}{3.475264in}}%
\pgfpathlineto{\pgfqpoint{5.669823in}{3.485516in}}%
\pgfpathlineto{\pgfqpoint{5.670897in}{3.488560in}}%
\pgfpathlineto{\pgfqpoint{5.671971in}{3.493525in}}%
\pgfpathlineto{\pgfqpoint{5.675193in}{3.494407in}}%
\pgfpathlineto{\pgfqpoint{5.676266in}{3.489280in}}%
\pgfpathlineto{\pgfqpoint{5.678414in}{3.498251in}}%
\pgfpathlineto{\pgfqpoint{5.679488in}{3.494246in}}%
\pgfpathlineto{\pgfqpoint{5.683783in}{3.501295in}}%
\pgfpathlineto{\pgfqpoint{5.684857in}{3.499452in}}%
\pgfpathlineto{\pgfqpoint{5.685931in}{3.499613in}}%
\pgfpathlineto{\pgfqpoint{5.687005in}{3.501054in}}%
\pgfpathlineto{\pgfqpoint{5.690226in}{3.500494in}}%
\pgfpathlineto{\pgfqpoint{5.691300in}{3.502977in}}%
\pgfpathlineto{\pgfqpoint{5.692374in}{3.503217in}}%
\pgfpathlineto{\pgfqpoint{5.693448in}{3.502816in}}%
\pgfpathlineto{\pgfqpoint{5.694522in}{3.497851in}}%
\pgfpathlineto{\pgfqpoint{5.697743in}{3.498892in}}%
\pgfpathlineto{\pgfqpoint{5.698817in}{3.495368in}}%
\pgfpathlineto{\pgfqpoint{5.699891in}{3.495848in}}%
\pgfpathlineto{\pgfqpoint{5.700965in}{3.494006in}}%
\pgfpathlineto{\pgfqpoint{5.702039in}{3.496329in}}%
\pgfpathlineto{\pgfqpoint{5.705260in}{3.496008in}}%
\pgfpathlineto{\pgfqpoint{5.706334in}{3.499452in}}%
\pgfpathlineto{\pgfqpoint{5.707408in}{3.505860in}}%
\pgfpathlineto{\pgfqpoint{5.708482in}{3.504899in}}%
\pgfpathlineto{\pgfqpoint{5.709555in}{3.508503in}}%
\pgfpathlineto{\pgfqpoint{5.712777in}{3.513549in}}%
\pgfpathlineto{\pgfqpoint{5.715999in}{3.522920in}}%
\pgfpathlineto{\pgfqpoint{5.717072in}{3.520197in}}%
\pgfpathlineto{\pgfqpoint{5.720294in}{3.520677in}}%
\pgfpathlineto{\pgfqpoint{5.721368in}{3.519476in}}%
\pgfpathlineto{\pgfqpoint{5.722442in}{3.524922in}}%
\pgfpathlineto{\pgfqpoint{5.723515in}{3.524602in}}%
\pgfpathlineto{\pgfqpoint{5.724589in}{3.526685in}}%
\pgfpathlineto{\pgfqpoint{5.727811in}{3.530609in}}%
\pgfpathlineto{\pgfqpoint{5.729958in}{3.529007in}}%
\pgfpathlineto{\pgfqpoint{5.732106in}{3.539980in}}%
\pgfpathlineto{\pgfqpoint{5.736401in}{3.536937in}}%
\pgfpathlineto{\pgfqpoint{5.737475in}{3.538619in}}%
\pgfpathlineto{\pgfqpoint{5.738549in}{3.533733in}}%
\pgfpathlineto{\pgfqpoint{5.739623in}{3.532531in}}%
\pgfpathlineto{\pgfqpoint{5.742844in}{3.534293in}}%
\pgfpathlineto{\pgfqpoint{5.744992in}{3.536376in}}%
\pgfpathlineto{\pgfqpoint{5.751435in}{3.529568in}}%
\pgfpathlineto{\pgfqpoint{5.754657in}{3.521639in}}%
\pgfpathlineto{\pgfqpoint{5.757878in}{3.521558in}}%
\pgfpathlineto{\pgfqpoint{5.760026in}{3.529167in}}%
\pgfpathlineto{\pgfqpoint{5.761100in}{3.537417in}}%
\pgfpathlineto{\pgfqpoint{5.762174in}{3.539980in}}%
\pgfpathlineto{\pgfqpoint{5.766469in}{3.537818in}}%
\pgfpathlineto{\pgfqpoint{5.767543in}{3.538538in}}%
\pgfpathlineto{\pgfqpoint{5.769690in}{3.538538in}}%
\pgfpathlineto{\pgfqpoint{5.772912in}{3.540781in}}%
\pgfpathlineto{\pgfqpoint{5.773986in}{3.543104in}}%
\pgfpathlineto{\pgfqpoint{5.775060in}{3.541262in}}%
\pgfpathlineto{\pgfqpoint{5.776133in}{3.535575in}}%
\pgfpathlineto{\pgfqpoint{5.777207in}{3.542063in}}%
\pgfpathlineto{\pgfqpoint{5.780429in}{3.542383in}}%
\pgfpathlineto{\pgfqpoint{5.781503in}{3.540701in}}%
\pgfpathlineto{\pgfqpoint{5.782576in}{3.541101in}}%
\pgfpathlineto{\pgfqpoint{5.783650in}{3.540781in}}%
\pgfpathlineto{\pgfqpoint{5.784724in}{3.537177in}}%
\pgfpathlineto{\pgfqpoint{5.787946in}{3.538619in}}%
\pgfpathlineto{\pgfqpoint{5.789020in}{3.543745in}}%
\pgfpathlineto{\pgfqpoint{5.790093in}{3.544626in}}%
\pgfpathlineto{\pgfqpoint{5.791167in}{3.541822in}}%
\pgfpathlineto{\pgfqpoint{5.792241in}{3.534293in}}%
\pgfpathlineto{\pgfqpoint{5.795463in}{3.537017in}}%
\pgfpathlineto{\pgfqpoint{5.797610in}{3.542463in}}%
\pgfpathlineto{\pgfqpoint{5.802979in}{3.541983in}}%
\pgfpathlineto{\pgfqpoint{5.804053in}{3.545667in}}%
\pgfpathlineto{\pgfqpoint{5.806201in}{3.538538in}}%
\pgfpathlineto{\pgfqpoint{5.811570in}{3.534053in}}%
\pgfpathlineto{\pgfqpoint{5.812644in}{3.528607in}}%
\pgfpathlineto{\pgfqpoint{5.813718in}{3.519636in}}%
\pgfpathlineto{\pgfqpoint{5.814792in}{3.517313in}}%
\pgfpathlineto{\pgfqpoint{5.818013in}{3.521238in}}%
\pgfpathlineto{\pgfqpoint{5.819087in}{3.525723in}}%
\pgfpathlineto{\pgfqpoint{5.820161in}{3.520197in}}%
\pgfpathlineto{\pgfqpoint{5.821235in}{3.526124in}}%
\pgfpathlineto{\pgfqpoint{5.822309in}{3.504659in}}%
\pgfpathlineto{\pgfqpoint{5.826604in}{3.504979in}}%
\pgfpathlineto{\pgfqpoint{5.827678in}{3.503457in}}%
\pgfpathlineto{\pgfqpoint{5.828752in}{3.503938in}}%
\pgfpathlineto{\pgfqpoint{5.829825in}{3.505860in}}%
\pgfpathlineto{\pgfqpoint{5.833047in}{3.503537in}}%
\pgfpathlineto{\pgfqpoint{5.834121in}{3.505940in}}%
\pgfpathlineto{\pgfqpoint{5.835195in}{3.505059in}}%
\pgfpathlineto{\pgfqpoint{5.836268in}{3.506180in}}%
\pgfpathlineto{\pgfqpoint{5.837342in}{3.513789in}}%
\pgfpathlineto{\pgfqpoint{5.840564in}{3.512348in}}%
\pgfpathlineto{\pgfqpoint{5.841638in}{3.504979in}}%
\pgfpathlineto{\pgfqpoint{5.842711in}{3.503457in}}%
\pgfpathlineto{\pgfqpoint{5.843785in}{3.506661in}}%
\pgfpathlineto{\pgfqpoint{5.844859in}{3.501295in}}%
\pgfpathlineto{\pgfqpoint{5.849154in}{3.499613in}}%
\pgfpathlineto{\pgfqpoint{5.850228in}{3.495448in}}%
\pgfpathlineto{\pgfqpoint{5.851302in}{3.495368in}}%
\pgfpathlineto{\pgfqpoint{5.852376in}{3.498411in}}%
\pgfpathlineto{\pgfqpoint{5.856671in}{3.499452in}}%
\pgfpathlineto{\pgfqpoint{5.857745in}{3.504418in}}%
\pgfpathlineto{\pgfqpoint{5.858819in}{3.504098in}}%
\pgfpathlineto{\pgfqpoint{5.859893in}{3.498892in}}%
\pgfpathlineto{\pgfqpoint{5.863114in}{3.503537in}}%
\pgfpathlineto{\pgfqpoint{5.864188in}{3.499533in}}%
\pgfpathlineto{\pgfqpoint{5.867410in}{3.506821in}}%
\pgfpathlineto{\pgfqpoint{5.870631in}{3.505299in}}%
\pgfpathlineto{\pgfqpoint{5.871705in}{3.510986in}}%
\pgfpathlineto{\pgfqpoint{5.872779in}{3.512268in}}%
\pgfpathlineto{\pgfqpoint{5.874927in}{3.512908in}}%
\pgfpathlineto{\pgfqpoint{5.878148in}{3.515151in}}%
\pgfpathlineto{\pgfqpoint{5.879222in}{3.512348in}}%
\pgfpathlineto{\pgfqpoint{5.881370in}{3.517394in}}%
\pgfpathlineto{\pgfqpoint{5.882443in}{3.521078in}}%
\pgfpathlineto{\pgfqpoint{5.885665in}{3.518675in}}%
\pgfpathlineto{\pgfqpoint{5.886739in}{3.520277in}}%
\pgfpathlineto{\pgfqpoint{5.887813in}{3.520597in}}%
\pgfpathlineto{\pgfqpoint{5.888887in}{3.522680in}}%
\pgfpathlineto{\pgfqpoint{5.889960in}{3.527806in}}%
\pgfpathlineto{\pgfqpoint{5.894256in}{3.525163in}}%
\pgfpathlineto{\pgfqpoint{5.895330in}{3.522840in}}%
\pgfpathlineto{\pgfqpoint{5.896403in}{3.521959in}}%
\pgfpathlineto{\pgfqpoint{5.900699in}{3.522119in}}%
\pgfpathlineto{\pgfqpoint{5.902846in}{3.528046in}}%
\pgfpathlineto{\pgfqpoint{5.903920in}{3.525403in}}%
\pgfpathlineto{\pgfqpoint{5.904994in}{3.526124in}}%
\pgfpathlineto{\pgfqpoint{5.909289in}{3.522119in}}%
\pgfpathlineto{\pgfqpoint{5.910363in}{3.523401in}}%
\pgfpathlineto{\pgfqpoint{5.911437in}{3.519556in}}%
\pgfpathlineto{\pgfqpoint{5.912511in}{3.520197in}}%
\pgfpathlineto{\pgfqpoint{5.915732in}{3.520437in}}%
\pgfpathlineto{\pgfqpoint{5.917880in}{3.523801in}}%
\pgfpathlineto{\pgfqpoint{5.920028in}{3.518915in}}%
\pgfpathlineto{\pgfqpoint{5.923249in}{3.520277in}}%
\pgfpathlineto{\pgfqpoint{5.924323in}{3.519957in}}%
\pgfpathlineto{\pgfqpoint{5.925397in}{3.522840in}}%
\pgfpathlineto{\pgfqpoint{5.926471in}{3.522600in}}%
\pgfpathlineto{\pgfqpoint{5.927545in}{3.520197in}}%
\pgfpathlineto{\pgfqpoint{5.930766in}{3.518435in}}%
\pgfpathlineto{\pgfqpoint{5.931840in}{3.518515in}}%
\pgfpathlineto{\pgfqpoint{5.932914in}{3.520998in}}%
\pgfpathlineto{\pgfqpoint{5.935062in}{3.510345in}}%
\pgfpathlineto{\pgfqpoint{5.938283in}{3.512748in}}%
\pgfpathlineto{\pgfqpoint{5.940431in}{3.509064in}}%
\pgfpathlineto{\pgfqpoint{5.941505in}{3.509464in}}%
\pgfpathlineto{\pgfqpoint{5.942578in}{3.510505in}}%
\pgfpathlineto{\pgfqpoint{5.945800in}{3.508743in}}%
\pgfpathlineto{\pgfqpoint{5.946874in}{3.511226in}}%
\pgfpathlineto{\pgfqpoint{5.947948in}{3.510586in}}%
\pgfpathlineto{\pgfqpoint{5.949021in}{3.508423in}}%
\pgfpathlineto{\pgfqpoint{5.953317in}{3.513068in}}%
\pgfpathlineto{\pgfqpoint{5.954391in}{3.510025in}}%
\pgfpathlineto{\pgfqpoint{5.955465in}{3.510105in}}%
\pgfpathlineto{\pgfqpoint{5.956538in}{3.507462in}}%
\pgfpathlineto{\pgfqpoint{5.957612in}{3.511306in}}%
\pgfpathlineto{\pgfqpoint{5.960834in}{3.511867in}}%
\pgfpathlineto{\pgfqpoint{5.961908in}{3.517954in}}%
\pgfpathlineto{\pgfqpoint{5.962981in}{3.520357in}}%
\pgfpathlineto{\pgfqpoint{5.965129in}{3.521719in}}%
\pgfpathlineto{\pgfqpoint{5.970498in}{3.522359in}}%
\pgfpathlineto{\pgfqpoint{5.971572in}{3.523080in}}%
\pgfpathlineto{\pgfqpoint{5.972646in}{3.522039in}}%
\pgfpathlineto{\pgfqpoint{5.975867in}{3.522520in}}%
\pgfpathlineto{\pgfqpoint{5.976941in}{3.523961in}}%
\pgfpathlineto{\pgfqpoint{5.979089in}{3.524442in}}%
\pgfpathlineto{\pgfqpoint{5.980163in}{3.525163in}}%
\pgfpathlineto{\pgfqpoint{5.984458in}{3.525884in}}%
\pgfpathlineto{\pgfqpoint{5.985532in}{3.522039in}}%
\pgfpathlineto{\pgfqpoint{5.987680in}{3.523080in}}%
\pgfpathlineto{\pgfqpoint{5.991975in}{3.527245in}}%
\pgfpathlineto{\pgfqpoint{5.993049in}{3.527005in}}%
\pgfpathlineto{\pgfqpoint{5.994123in}{3.532131in}}%
\pgfpathlineto{\pgfqpoint{5.995197in}{3.521558in}}%
\pgfpathlineto{\pgfqpoint{5.998418in}{3.515471in}}%
\pgfpathlineto{\pgfqpoint{5.999492in}{3.518915in}}%
\pgfpathlineto{\pgfqpoint{6.001640in}{3.530769in}}%
\pgfpathlineto{\pgfqpoint{6.002713in}{3.530369in}}%
\pgfpathlineto{\pgfqpoint{6.007009in}{3.529888in}}%
\pgfpathlineto{\pgfqpoint{6.009156in}{3.533653in}}%
\pgfpathlineto{\pgfqpoint{6.010230in}{3.539580in}}%
\pgfpathlineto{\pgfqpoint{6.013452in}{3.542303in}}%
\pgfpathlineto{\pgfqpoint{6.014526in}{3.546468in}}%
\pgfpathlineto{\pgfqpoint{6.015599in}{3.546948in}}%
\pgfpathlineto{\pgfqpoint{6.016673in}{3.548390in}}%
\pgfpathlineto{\pgfqpoint{6.017747in}{3.547429in}}%
\pgfpathlineto{\pgfqpoint{6.020969in}{3.547269in}}%
\pgfpathlineto{\pgfqpoint{6.022042in}{3.547990in}}%
\pgfpathlineto{\pgfqpoint{6.023116in}{3.551994in}}%
\pgfpathlineto{\pgfqpoint{6.024190in}{3.541502in}}%
\pgfpathlineto{\pgfqpoint{6.025264in}{3.544385in}}%
\pgfpathlineto{\pgfqpoint{6.028486in}{3.544626in}}%
\pgfpathlineto{\pgfqpoint{6.029559in}{3.547589in}}%
\pgfpathlineto{\pgfqpoint{6.030633in}{3.545667in}}%
\pgfpathlineto{\pgfqpoint{6.031707in}{3.545186in}}%
\pgfpathlineto{\pgfqpoint{6.032781in}{3.545827in}}%
\pgfpathlineto{\pgfqpoint{6.036002in}{3.545827in}}%
\pgfpathlineto{\pgfqpoint{6.037076in}{3.543664in}}%
\pgfpathlineto{\pgfqpoint{6.038150in}{3.543264in}}%
\pgfpathlineto{\pgfqpoint{6.040298in}{3.548710in}}%
\pgfpathlineto{\pgfqpoint{6.043519in}{3.549111in}}%
\pgfpathlineto{\pgfqpoint{6.044593in}{3.548230in}}%
\pgfpathlineto{\pgfqpoint{6.045667in}{3.545346in}}%
\pgfpathlineto{\pgfqpoint{6.046741in}{3.546468in}}%
\pgfpathlineto{\pgfqpoint{6.047815in}{3.545667in}}%
\pgfpathlineto{\pgfqpoint{6.051036in}{3.548150in}}%
\pgfpathlineto{\pgfqpoint{6.052110in}{3.550392in}}%
\pgfpathlineto{\pgfqpoint{6.053184in}{3.548951in}}%
\pgfpathlineto{\pgfqpoint{6.054258in}{3.548630in}}%
\pgfpathlineto{\pgfqpoint{6.055331in}{3.550633in}}%
\pgfpathlineto{\pgfqpoint{6.059627in}{3.551834in}}%
\pgfpathlineto{\pgfqpoint{6.060701in}{3.549992in}}%
\pgfpathlineto{\pgfqpoint{6.061775in}{3.549511in}}%
\pgfpathlineto{\pgfqpoint{6.062848in}{3.550793in}}%
\pgfpathlineto{\pgfqpoint{6.067144in}{3.553997in}}%
\pgfpathlineto{\pgfqpoint{6.069291in}{3.556399in}}%
\pgfpathlineto{\pgfqpoint{6.070365in}{3.556800in}}%
\pgfpathlineto{\pgfqpoint{6.074661in}{3.560484in}}%
\pgfpathlineto{\pgfqpoint{6.075734in}{3.559603in}}%
\pgfpathlineto{\pgfqpoint{6.076808in}{3.559523in}}%
\pgfpathlineto{\pgfqpoint{6.077882in}{3.552074in}}%
\pgfpathlineto{\pgfqpoint{6.081104in}{3.556800in}}%
\pgfpathlineto{\pgfqpoint{6.082177in}{3.553356in}}%
\pgfpathlineto{\pgfqpoint{6.083251in}{3.553436in}}%
\pgfpathlineto{\pgfqpoint{6.085399in}{3.568574in}}%
\pgfpathlineto{\pgfqpoint{6.088620in}{3.564809in}}%
\pgfpathlineto{\pgfqpoint{6.089694in}{3.564649in}}%
\pgfpathlineto{\pgfqpoint{6.090768in}{3.566972in}}%
\pgfpathlineto{\pgfqpoint{6.091842in}{3.567693in}}%
\pgfpathlineto{\pgfqpoint{6.092916in}{3.565050in}}%
\pgfpathlineto{\pgfqpoint{6.096137in}{3.561045in}}%
\pgfpathlineto{\pgfqpoint{6.098285in}{3.566892in}}%
\pgfpathlineto{\pgfqpoint{6.099359in}{3.566011in}}%
\pgfpathlineto{\pgfqpoint{6.100433in}{3.569215in}}%
\pgfpathlineto{\pgfqpoint{6.103654in}{3.568494in}}%
\pgfpathlineto{\pgfqpoint{6.104728in}{3.567613in}}%
\pgfpathlineto{\pgfqpoint{6.105802in}{3.570977in}}%
\pgfpathlineto{\pgfqpoint{6.107950in}{3.571778in}}%
\pgfpathlineto{\pgfqpoint{6.111171in}{3.571217in}}%
\pgfpathlineto{\pgfqpoint{6.112245in}{3.565610in}}%
\pgfpathlineto{\pgfqpoint{6.114393in}{3.563448in}}%
\pgfpathlineto{\pgfqpoint{6.115466in}{3.566972in}}%
\pgfpathlineto{\pgfqpoint{6.118688in}{3.565771in}}%
\pgfpathlineto{\pgfqpoint{6.119762in}{3.569215in}}%
\pgfpathlineto{\pgfqpoint{6.120836in}{3.552635in}}%
\pgfpathlineto{\pgfqpoint{6.121909in}{3.551994in}}%
\pgfpathlineto{\pgfqpoint{6.122983in}{3.549992in}}%
\pgfpathlineto{\pgfqpoint{6.126205in}{3.550793in}}%
\pgfpathlineto{\pgfqpoint{6.129426in}{3.547429in}}%
\pgfpathlineto{\pgfqpoint{6.130500in}{3.546948in}}%
\pgfpathlineto{\pgfqpoint{6.133722in}{3.547909in}}%
\pgfpathlineto{\pgfqpoint{6.134796in}{3.545266in}}%
\pgfpathlineto{\pgfqpoint{6.135869in}{3.545907in}}%
\pgfpathlineto{\pgfqpoint{6.138017in}{3.540461in}}%
\pgfpathlineto{\pgfqpoint{6.141239in}{3.548470in}}%
\pgfpathlineto{\pgfqpoint{6.143386in}{3.548951in}}%
\pgfpathlineto{\pgfqpoint{6.144460in}{3.547109in}}%
\pgfpathlineto{\pgfqpoint{6.145534in}{3.547909in}}%
\pgfpathlineto{\pgfqpoint{6.148755in}{3.546948in}}%
\pgfpathlineto{\pgfqpoint{6.149829in}{3.550152in}}%
\pgfpathlineto{\pgfqpoint{6.150903in}{3.549591in}}%
\pgfpathlineto{\pgfqpoint{6.151977in}{3.550953in}}%
\pgfpathlineto{\pgfqpoint{6.153051in}{3.550392in}}%
\pgfpathlineto{\pgfqpoint{6.156272in}{3.550633in}}%
\pgfpathlineto{\pgfqpoint{6.157346in}{3.554397in}}%
\pgfpathlineto{\pgfqpoint{6.158420in}{3.552315in}}%
\pgfpathlineto{\pgfqpoint{6.160568in}{3.554077in}}%
\pgfpathlineto{\pgfqpoint{6.163789in}{3.554637in}}%
\pgfpathlineto{\pgfqpoint{6.164863in}{3.553116in}}%
\pgfpathlineto{\pgfqpoint{6.167011in}{3.541582in}}%
\pgfpathlineto{\pgfqpoint{6.168085in}{3.544546in}}%
\pgfpathlineto{\pgfqpoint{6.171306in}{3.546228in}}%
\pgfpathlineto{\pgfqpoint{6.172380in}{3.548710in}}%
\pgfpathlineto{\pgfqpoint{6.173454in}{3.554557in}}%
\pgfpathlineto{\pgfqpoint{6.174528in}{3.555999in}}%
\pgfpathlineto{\pgfqpoint{6.178823in}{3.558001in}}%
\pgfpathlineto{\pgfqpoint{6.179897in}{3.564249in}}%
\pgfpathlineto{\pgfqpoint{6.180971in}{3.562326in}}%
\pgfpathlineto{\pgfqpoint{6.182044in}{3.564169in}}%
\pgfpathlineto{\pgfqpoint{6.183118in}{3.560564in}}%
\pgfpathlineto{\pgfqpoint{6.186340in}{3.564889in}}%
\pgfpathlineto{\pgfqpoint{6.187414in}{3.567292in}}%
\pgfpathlineto{\pgfqpoint{6.188487in}{3.565530in}}%
\pgfpathlineto{\pgfqpoint{6.190635in}{3.565450in}}%
\pgfpathlineto{\pgfqpoint{6.194930in}{3.566251in}}%
\pgfpathlineto{\pgfqpoint{6.196004in}{3.562967in}}%
\pgfpathlineto{\pgfqpoint{6.197078in}{3.563207in}}%
\pgfpathlineto{\pgfqpoint{6.198152in}{3.560244in}}%
\pgfpathlineto{\pgfqpoint{6.202447in}{3.562727in}}%
\pgfpathlineto{\pgfqpoint{6.203521in}{3.561285in}}%
\pgfpathlineto{\pgfqpoint{6.204595in}{3.560885in}}%
\pgfpathlineto{\pgfqpoint{6.205669in}{3.561846in}}%
\pgfpathlineto{\pgfqpoint{6.208890in}{3.562807in}}%
\pgfpathlineto{\pgfqpoint{6.209964in}{3.562246in}}%
\pgfpathlineto{\pgfqpoint{6.211038in}{3.565370in}}%
\pgfpathlineto{\pgfqpoint{6.212112in}{3.563528in}}%
\pgfpathlineto{\pgfqpoint{6.213186in}{3.564169in}}%
\pgfpathlineto{\pgfqpoint{6.218555in}{3.563928in}}%
\pgfpathlineto{\pgfqpoint{6.219629in}{3.562487in}}%
\pgfpathlineto{\pgfqpoint{6.220703in}{3.565290in}}%
\pgfpathlineto{\pgfqpoint{6.223924in}{3.564008in}}%
\pgfpathlineto{\pgfqpoint{6.224998in}{3.570336in}}%
\pgfpathlineto{\pgfqpoint{6.226072in}{3.571697in}}%
\pgfpathlineto{\pgfqpoint{6.227146in}{3.569855in}}%
\pgfpathlineto{\pgfqpoint{6.228219in}{3.572979in}}%
\pgfpathlineto{\pgfqpoint{6.231441in}{3.568814in}}%
\pgfpathlineto{\pgfqpoint{6.233589in}{3.562166in}}%
\pgfpathlineto{\pgfqpoint{6.235736in}{3.564089in}}%
\pgfpathlineto{\pgfqpoint{6.238958in}{3.562166in}}%
\pgfpathlineto{\pgfqpoint{6.241106in}{3.563047in}}%
\pgfpathlineto{\pgfqpoint{6.242179in}{3.556079in}}%
\pgfpathlineto{\pgfqpoint{6.243253in}{3.555198in}}%
\pgfpathlineto{\pgfqpoint{6.248622in}{3.560564in}}%
\pgfpathlineto{\pgfqpoint{6.249696in}{3.563288in}}%
\pgfpathlineto{\pgfqpoint{6.250770in}{3.563768in}}%
\pgfpathlineto{\pgfqpoint{6.255065in}{3.564089in}}%
\pgfpathlineto{\pgfqpoint{6.256139in}{3.560725in}}%
\pgfpathlineto{\pgfqpoint{6.257213in}{3.561526in}}%
\pgfpathlineto{\pgfqpoint{6.258287in}{3.564169in}}%
\pgfpathlineto{\pgfqpoint{6.261508in}{3.564008in}}%
\pgfpathlineto{\pgfqpoint{6.263656in}{3.559683in}}%
\pgfpathlineto{\pgfqpoint{6.265804in}{3.559443in}}%
\pgfpathlineto{\pgfqpoint{6.269025in}{3.556960in}}%
\pgfpathlineto{\pgfqpoint{6.270099in}{3.558642in}}%
\pgfpathlineto{\pgfqpoint{6.271173in}{3.557281in}}%
\pgfpathlineto{\pgfqpoint{6.273321in}{3.559523in}}%
\pgfpathlineto{\pgfqpoint{6.276542in}{3.553836in}}%
\pgfpathlineto{\pgfqpoint{6.277616in}{3.553997in}}%
\pgfpathlineto{\pgfqpoint{6.278690in}{3.553436in}}%
\pgfpathlineto{\pgfqpoint{6.279764in}{3.553676in}}%
\pgfpathlineto{\pgfqpoint{6.280838in}{3.554717in}}%
\pgfpathlineto{\pgfqpoint{6.284059in}{3.555919in}}%
\pgfpathlineto{\pgfqpoint{6.285133in}{3.552955in}}%
\pgfpathlineto{\pgfqpoint{6.286207in}{3.555438in}}%
\pgfpathlineto{\pgfqpoint{6.288354in}{3.553836in}}%
\pgfpathlineto{\pgfqpoint{6.291576in}{3.555599in}}%
\pgfpathlineto{\pgfqpoint{6.292650in}{3.557200in}}%
\pgfpathlineto{\pgfqpoint{6.293724in}{3.556960in}}%
\pgfpathlineto{\pgfqpoint{6.294797in}{3.558322in}}%
\pgfpathlineto{\pgfqpoint{6.295871in}{3.560725in}}%
\pgfpathlineto{\pgfqpoint{6.299093in}{3.561365in}}%
\pgfpathlineto{\pgfqpoint{6.300167in}{3.562246in}}%
\pgfpathlineto{\pgfqpoint{6.301241in}{3.561846in}}%
\pgfpathlineto{\pgfqpoint{6.302314in}{3.560404in}}%
\pgfpathlineto{\pgfqpoint{6.303388in}{3.560404in}}%
\pgfpathlineto{\pgfqpoint{6.308757in}{3.557361in}}%
\pgfpathlineto{\pgfqpoint{6.309831in}{3.554557in}}%
\pgfpathlineto{\pgfqpoint{6.314127in}{3.556239in}}%
\pgfpathlineto{\pgfqpoint{6.316274in}{3.559523in}}%
\pgfpathlineto{\pgfqpoint{6.318422in}{3.562567in}}%
\pgfpathlineto{\pgfqpoint{6.323791in}{3.567132in}}%
\pgfpathlineto{\pgfqpoint{6.324865in}{3.570897in}}%
\pgfpathlineto{\pgfqpoint{6.325939in}{3.561285in}}%
\pgfpathlineto{\pgfqpoint{6.329160in}{3.562487in}}%
\pgfpathlineto{\pgfqpoint{6.330234in}{3.567452in}}%
\pgfpathlineto{\pgfqpoint{6.331308in}{3.569615in}}%
\pgfpathlineto{\pgfqpoint{6.332382in}{3.568574in}}%
\pgfpathlineto{\pgfqpoint{6.333456in}{3.568414in}}%
\pgfpathlineto{\pgfqpoint{6.337751in}{3.564970in}}%
\pgfpathlineto{\pgfqpoint{6.339899in}{3.559844in}}%
\pgfpathlineto{\pgfqpoint{6.340973in}{3.558642in}}%
\pgfpathlineto{\pgfqpoint{6.344194in}{3.559363in}}%
\pgfpathlineto{\pgfqpoint{6.345268in}{3.560805in}}%
\pgfpathlineto{\pgfqpoint{6.346342in}{3.554958in}}%
\pgfpathlineto{\pgfqpoint{6.348489in}{3.557681in}}%
\pgfpathlineto{\pgfqpoint{6.352785in}{3.561125in}}%
\pgfpathlineto{\pgfqpoint{6.354932in}{3.564169in}}%
\pgfpathlineto{\pgfqpoint{6.356006in}{3.564169in}}%
\pgfpathlineto{\pgfqpoint{6.362449in}{3.563127in}}%
\pgfpathlineto{\pgfqpoint{6.363523in}{3.564569in}}%
\pgfpathlineto{\pgfqpoint{6.366745in}{3.564729in}}%
\pgfpathlineto{\pgfqpoint{6.367818in}{3.563127in}}%
\pgfpathlineto{\pgfqpoint{6.369966in}{3.565851in}}%
\pgfpathlineto{\pgfqpoint{6.371040in}{3.560004in}}%
\pgfpathlineto{\pgfqpoint{6.374262in}{3.560164in}}%
\pgfpathlineto{\pgfqpoint{6.375335in}{3.561285in}}%
\pgfpathlineto{\pgfqpoint{6.377483in}{3.556960in}}%
\pgfpathlineto{\pgfqpoint{6.378557in}{3.556239in}}%
\pgfpathlineto{\pgfqpoint{6.381778in}{3.558482in}}%
\pgfpathlineto{\pgfqpoint{6.382852in}{3.553596in}}%
\pgfpathlineto{\pgfqpoint{6.385000in}{3.549752in}}%
\pgfpathlineto{\pgfqpoint{6.386074in}{3.548470in}}%
\pgfpathlineto{\pgfqpoint{6.389295in}{3.547589in}}%
\pgfpathlineto{\pgfqpoint{6.390369in}{3.544385in}}%
\pgfpathlineto{\pgfqpoint{6.391443in}{3.548550in}}%
\pgfpathlineto{\pgfqpoint{6.392517in}{3.543584in}}%
\pgfpathlineto{\pgfqpoint{6.393591in}{3.545106in}}%
\pgfpathlineto{\pgfqpoint{6.396812in}{3.542944in}}%
\pgfpathlineto{\pgfqpoint{6.398960in}{3.549591in}}%
\pgfpathlineto{\pgfqpoint{6.400034in}{3.544225in}}%
\pgfpathlineto{\pgfqpoint{6.401107in}{3.546147in}}%
\pgfpathlineto{\pgfqpoint{6.404329in}{3.544385in}}%
\pgfpathlineto{\pgfqpoint{6.406477in}{3.548951in}}%
\pgfpathlineto{\pgfqpoint{6.407551in}{3.548871in}}%
\pgfpathlineto{\pgfqpoint{6.408624in}{3.552235in}}%
\pgfpathlineto{\pgfqpoint{6.411846in}{3.550633in}}%
\pgfpathlineto{\pgfqpoint{6.413994in}{3.551273in}}%
\pgfpathlineto{\pgfqpoint{6.415067in}{3.552715in}}%
\pgfpathlineto{\pgfqpoint{6.416141in}{3.552555in}}%
\pgfpathlineto{\pgfqpoint{6.419363in}{3.550873in}}%
\pgfpathlineto{\pgfqpoint{6.422584in}{3.554397in}}%
\pgfpathlineto{\pgfqpoint{6.423658in}{3.556960in}}%
\pgfpathlineto{\pgfqpoint{6.426880in}{3.558162in}}%
\pgfpathlineto{\pgfqpoint{6.427953in}{3.564809in}}%
\pgfpathlineto{\pgfqpoint{6.429027in}{3.567052in}}%
\pgfpathlineto{\pgfqpoint{6.430101in}{3.567933in}}%
\pgfpathlineto{\pgfqpoint{6.431175in}{3.566491in}}%
\pgfpathlineto{\pgfqpoint{6.436544in}{3.568734in}}%
\pgfpathlineto{\pgfqpoint{6.438692in}{3.563207in}}%
\pgfpathlineto{\pgfqpoint{6.441913in}{3.566812in}}%
\pgfpathlineto{\pgfqpoint{6.446209in}{3.556720in}}%
\pgfpathlineto{\pgfqpoint{6.449430in}{3.555999in}}%
\pgfpathlineto{\pgfqpoint{6.450504in}{3.553917in}}%
\pgfpathlineto{\pgfqpoint{6.458021in}{3.554557in}}%
\pgfpathlineto{\pgfqpoint{6.460169in}{3.557120in}}%
\pgfpathlineto{\pgfqpoint{6.461242in}{3.557281in}}%
\pgfpathlineto{\pgfqpoint{6.465538in}{3.556720in}}%
\pgfpathlineto{\pgfqpoint{6.466612in}{3.562407in}}%
\pgfpathlineto{\pgfqpoint{6.468759in}{3.558081in}}%
\pgfpathlineto{\pgfqpoint{6.473055in}{3.564889in}}%
\pgfpathlineto{\pgfqpoint{6.475202in}{3.567853in}}%
\pgfpathlineto{\pgfqpoint{6.476276in}{3.571858in}}%
\pgfpathlineto{\pgfqpoint{6.479498in}{3.571858in}}%
\pgfpathlineto{\pgfqpoint{6.480572in}{3.573620in}}%
\pgfpathlineto{\pgfqpoint{6.481645in}{3.572418in}}%
\pgfpathlineto{\pgfqpoint{6.482719in}{3.573379in}}%
\pgfpathlineto{\pgfqpoint{6.487015in}{3.573059in}}%
\pgfpathlineto{\pgfqpoint{6.488088in}{3.575462in}}%
\pgfpathlineto{\pgfqpoint{6.489162in}{3.575942in}}%
\pgfpathlineto{\pgfqpoint{6.495605in}{3.589959in}}%
\pgfpathlineto{\pgfqpoint{6.496679in}{3.589639in}}%
\pgfpathlineto{\pgfqpoint{6.498827in}{3.591881in}}%
\pgfpathlineto{\pgfqpoint{6.502048in}{3.593643in}}%
\pgfpathlineto{\pgfqpoint{6.503122in}{3.592041in}}%
\pgfpathlineto{\pgfqpoint{6.504196in}{3.589398in}}%
\pgfpathlineto{\pgfqpoint{6.505270in}{3.588517in}}%
\pgfpathlineto{\pgfqpoint{6.506344in}{3.592202in}}%
\pgfpathlineto{\pgfqpoint{6.510639in}{3.593083in}}%
\pgfpathlineto{\pgfqpoint{6.511713in}{3.596607in}}%
\pgfpathlineto{\pgfqpoint{6.512787in}{3.595405in}}%
\pgfpathlineto{\pgfqpoint{6.513861in}{3.597968in}}%
\pgfpathlineto{\pgfqpoint{6.518156in}{3.601973in}}%
\pgfpathlineto{\pgfqpoint{6.519230in}{3.600612in}}%
\pgfpathlineto{\pgfqpoint{6.520304in}{3.605017in}}%
\pgfpathlineto{\pgfqpoint{6.521377in}{3.628244in}}%
\pgfpathlineto{\pgfqpoint{6.524599in}{3.628004in}}%
\pgfpathlineto{\pgfqpoint{6.526747in}{3.645865in}}%
\pgfpathlineto{\pgfqpoint{6.527820in}{3.648828in}}%
\pgfpathlineto{\pgfqpoint{6.528894in}{3.642981in}}%
\pgfpathlineto{\pgfqpoint{6.533190in}{3.648508in}}%
\pgfpathlineto{\pgfqpoint{6.534263in}{3.647867in}}%
\pgfpathlineto{\pgfqpoint{6.535337in}{3.644824in}}%
\pgfpathlineto{\pgfqpoint{6.536411in}{3.639297in}}%
\pgfpathlineto{\pgfqpoint{6.540706in}{3.641460in}}%
\pgfpathlineto{\pgfqpoint{6.541780in}{3.638336in}}%
\pgfpathlineto{\pgfqpoint{6.542854in}{3.639778in}}%
\pgfpathlineto{\pgfqpoint{6.543928in}{3.632008in}}%
\pgfpathlineto{\pgfqpoint{6.547150in}{3.631928in}}%
\pgfpathlineto{\pgfqpoint{6.548223in}{3.634411in}}%
\pgfpathlineto{\pgfqpoint{6.549297in}{3.632169in}}%
\pgfpathlineto{\pgfqpoint{6.551445in}{3.632890in}}%
\pgfpathlineto{\pgfqpoint{6.554666in}{3.630887in}}%
\pgfpathlineto{\pgfqpoint{6.555740in}{3.632729in}}%
\pgfpathlineto{\pgfqpoint{6.556814in}{3.626802in}}%
\pgfpathlineto{\pgfqpoint{6.557888in}{3.633610in}}%
\pgfpathlineto{\pgfqpoint{6.558962in}{3.632409in}}%
\pgfpathlineto{\pgfqpoint{6.562183in}{3.630887in}}%
\pgfpathlineto{\pgfqpoint{6.563257in}{3.622878in}}%
\pgfpathlineto{\pgfqpoint{6.564331in}{3.622958in}}%
\pgfpathlineto{\pgfqpoint{6.565405in}{3.620155in}}%
\pgfpathlineto{\pgfqpoint{6.566479in}{3.622157in}}%
\pgfpathlineto{\pgfqpoint{6.569700in}{3.624560in}}%
\pgfpathlineto{\pgfqpoint{6.570774in}{3.621997in}}%
\pgfpathlineto{\pgfqpoint{6.571848in}{3.622077in}}%
\pgfpathlineto{\pgfqpoint{6.572922in}{3.621516in}}%
\pgfpathlineto{\pgfqpoint{6.573995in}{3.631448in}}%
\pgfpathlineto{\pgfqpoint{6.579365in}{3.654435in}}%
\pgfpathlineto{\pgfqpoint{6.580439in}{3.648348in}}%
\pgfpathlineto{\pgfqpoint{6.581512in}{3.647867in}}%
\pgfpathlineto{\pgfqpoint{6.585808in}{3.643142in}}%
\pgfpathlineto{\pgfqpoint{6.586882in}{3.643302in}}%
\pgfpathlineto{\pgfqpoint{6.587955in}{3.644183in}}%
\pgfpathlineto{\pgfqpoint{6.589029in}{3.643702in}}%
\pgfpathlineto{\pgfqpoint{6.589029in}{3.643702in}}%
\pgfusepath{stroke}%
\end{pgfscope}%
\begin{pgfscope}%
\pgfpathrectangle{\pgfqpoint{4.123120in}{3.271772in}}{\pgfqpoint{2.583333in}{0.400885in}}%
\pgfusepath{clip}%
\pgfsetroundcap%
\pgfsetroundjoin%
\pgfsetlinewidth{1.505625pt}%
\definecolor{currentstroke}{rgb}{0.839216,0.152941,0.156863}%
\pgfsetstrokecolor{currentstroke}%
\pgfsetdash{}{0pt}%
\pgfpathmoveto{\pgfqpoint{4.240544in}{3.444748in}}%
\pgfpathlineto{\pgfqpoint{4.241618in}{3.444748in}}%
\pgfpathlineto{\pgfqpoint{4.242692in}{3.446548in}}%
\pgfpathlineto{\pgfqpoint{4.243766in}{3.445609in}}%
\pgfpathlineto{\pgfqpoint{4.248061in}{3.447722in}}%
\pgfpathlineto{\pgfqpoint{4.249135in}{3.449053in}}%
\pgfpathlineto{\pgfqpoint{4.250209in}{3.448740in}}%
\pgfpathlineto{\pgfqpoint{4.251283in}{3.444983in}}%
\pgfpathlineto{\pgfqpoint{4.255578in}{3.444357in}}%
\pgfpathlineto{\pgfqpoint{4.257726in}{3.447957in}}%
\pgfpathlineto{\pgfqpoint{4.258800in}{3.443339in}}%
\pgfpathlineto{\pgfqpoint{4.264169in}{3.446366in}}%
\pgfpathlineto{\pgfqpoint{4.265243in}{3.445480in}}%
\pgfpathlineto{\pgfqpoint{4.269538in}{3.445480in}}%
\pgfpathlineto{\pgfqpoint{4.270612in}{3.443561in}}%
\pgfpathlineto{\pgfqpoint{4.271686in}{3.444373in}}%
\pgfpathlineto{\pgfqpoint{4.272760in}{3.444004in}}%
\pgfpathlineto{\pgfqpoint{4.273833in}{3.441346in}}%
\pgfpathlineto{\pgfqpoint{4.278129in}{3.440775in}}%
\pgfpathlineto{\pgfqpoint{4.279203in}{3.441988in}}%
\pgfpathlineto{\pgfqpoint{4.280276in}{3.442059in}}%
\pgfpathlineto{\pgfqpoint{4.281350in}{3.441132in}}%
\pgfpathlineto{\pgfqpoint{4.285646in}{3.441631in}}%
\pgfpathlineto{\pgfqpoint{4.286719in}{3.440490in}}%
\pgfpathlineto{\pgfqpoint{4.287793in}{3.441916in}}%
\pgfpathlineto{\pgfqpoint{4.288867in}{3.438921in}}%
\pgfpathlineto{\pgfqpoint{4.293162in}{3.437824in}}%
\pgfpathlineto{\pgfqpoint{4.294236in}{3.435492in}}%
\pgfpathlineto{\pgfqpoint{4.295310in}{3.435081in}}%
\pgfpathlineto{\pgfqpoint{4.299606in}{3.436315in}}%
\pgfpathlineto{\pgfqpoint{4.300679in}{3.438235in}}%
\pgfpathlineto{\pgfqpoint{4.301753in}{3.436315in}}%
\pgfpathlineto{\pgfqpoint{4.303901in}{3.436521in}}%
\pgfpathlineto{\pgfqpoint{4.307122in}{3.434464in}}%
\pgfpathlineto{\pgfqpoint{4.308196in}{3.434807in}}%
\pgfpathlineto{\pgfqpoint{4.309270in}{3.436452in}}%
\pgfpathlineto{\pgfqpoint{4.310344in}{3.436041in}}%
\pgfpathlineto{\pgfqpoint{4.311418in}{3.437344in}}%
\pgfpathlineto{\pgfqpoint{4.314639in}{3.436864in}}%
\pgfpathlineto{\pgfqpoint{4.315713in}{3.434121in}}%
\pgfpathlineto{\pgfqpoint{4.316787in}{3.433989in}}%
\pgfpathlineto{\pgfqpoint{4.317861in}{3.435508in}}%
\pgfpathlineto{\pgfqpoint{4.322156in}{3.435442in}}%
\pgfpathlineto{\pgfqpoint{4.324304in}{3.435640in}}%
\pgfpathlineto{\pgfqpoint{4.325378in}{3.436301in}}%
\pgfpathlineto{\pgfqpoint{4.326451in}{3.436169in}}%
\pgfpathlineto{\pgfqpoint{4.330747in}{3.437820in}}%
\pgfpathlineto{\pgfqpoint{4.332894in}{3.433923in}}%
\pgfpathlineto{\pgfqpoint{4.333968in}{3.433730in}}%
\pgfpathlineto{\pgfqpoint{4.337190in}{3.435082in}}%
\pgfpathlineto{\pgfqpoint{4.339338in}{3.432764in}}%
\pgfpathlineto{\pgfqpoint{4.340411in}{3.433472in}}%
\pgfpathlineto{\pgfqpoint{4.344707in}{3.431862in}}%
\pgfpathlineto{\pgfqpoint{4.346854in}{3.428256in}}%
\pgfpathlineto{\pgfqpoint{4.349002in}{3.423276in}}%
\pgfpathlineto{\pgfqpoint{4.353297in}{3.425069in}}%
\pgfpathlineto{\pgfqpoint{4.355445in}{3.421363in}}%
\pgfpathlineto{\pgfqpoint{4.360814in}{3.419570in}}%
\pgfpathlineto{\pgfqpoint{4.362962in}{3.415334in}}%
\pgfpathlineto{\pgfqpoint{4.364036in}{3.416062in}}%
\pgfpathlineto{\pgfqpoint{4.367257in}{3.416118in}}%
\pgfpathlineto{\pgfqpoint{4.368331in}{3.413655in}}%
\pgfpathlineto{\pgfqpoint{4.369405in}{3.414624in}}%
\pgfpathlineto{\pgfqpoint{4.371553in}{3.410049in}}%
\pgfpathlineto{\pgfqpoint{4.374774in}{3.409404in}}%
\pgfpathlineto{\pgfqpoint{4.376922in}{3.406982in}}%
\pgfpathlineto{\pgfqpoint{4.377996in}{3.407197in}}%
\pgfpathlineto{\pgfqpoint{4.379070in}{3.405529in}}%
\pgfpathlineto{\pgfqpoint{4.384439in}{3.400821in}}%
\pgfpathlineto{\pgfqpoint{4.386586in}{3.398990in}}%
\pgfpathlineto{\pgfqpoint{4.389808in}{3.399356in}}%
\pgfpathlineto{\pgfqpoint{4.390882in}{3.398833in}}%
\pgfpathlineto{\pgfqpoint{4.391956in}{3.396375in}}%
\pgfpathlineto{\pgfqpoint{4.394103in}{3.397630in}}%
\pgfpathlineto{\pgfqpoint{4.400546in}{3.395147in}}%
\pgfpathlineto{\pgfqpoint{4.401620in}{3.392297in}}%
\pgfpathlineto{\pgfqpoint{4.404842in}{3.391889in}}%
\pgfpathlineto{\pgfqpoint{4.409137in}{3.385588in}}%
\pgfpathlineto{\pgfqpoint{4.412359in}{3.384093in}}%
\pgfpathlineto{\pgfqpoint{4.413432in}{3.382191in}}%
\pgfpathlineto{\pgfqpoint{4.414506in}{3.382278in}}%
\pgfpathlineto{\pgfqpoint{4.416654in}{3.379535in}}%
\pgfpathlineto{\pgfqpoint{4.420949in}{3.380067in}}%
\pgfpathlineto{\pgfqpoint{4.422023in}{3.380518in}}%
\pgfpathlineto{\pgfqpoint{4.423097in}{3.377445in}}%
\pgfpathlineto{\pgfqpoint{4.424171in}{3.378224in}}%
\pgfpathlineto{\pgfqpoint{4.427392in}{3.375315in}}%
\pgfpathlineto{\pgfqpoint{4.428466in}{3.375151in}}%
\pgfpathlineto{\pgfqpoint{4.429540in}{3.375847in}}%
\pgfpathlineto{\pgfqpoint{4.430614in}{3.374577in}}%
\pgfpathlineto{\pgfqpoint{4.431688in}{3.371873in}}%
\pgfpathlineto{\pgfqpoint{4.434909in}{3.371911in}}%
\pgfpathlineto{\pgfqpoint{4.435983in}{3.372488in}}%
\pgfpathlineto{\pgfqpoint{4.438131in}{3.371565in}}%
\pgfpathlineto{\pgfqpoint{4.439205in}{3.370335in}}%
\pgfpathlineto{\pgfqpoint{4.442426in}{3.370374in}}%
\pgfpathlineto{\pgfqpoint{4.443500in}{3.368490in}}%
\pgfpathlineto{\pgfqpoint{4.444574in}{3.367991in}}%
\pgfpathlineto{\pgfqpoint{4.446721in}{3.364416in}}%
\pgfpathlineto{\pgfqpoint{4.449943in}{3.364047in}}%
\pgfpathlineto{\pgfqpoint{4.451017in}{3.364785in}}%
\pgfpathlineto{\pgfqpoint{4.452091in}{3.362351in}}%
\pgfpathlineto{\pgfqpoint{4.453164in}{3.361936in}}%
\pgfpathlineto{\pgfqpoint{4.454238in}{3.360451in}}%
\pgfpathlineto{\pgfqpoint{4.459607in}{3.359379in}}%
\pgfpathlineto{\pgfqpoint{4.461755in}{3.356968in}}%
\pgfpathlineto{\pgfqpoint{4.466050in}{3.356162in}}%
\pgfpathlineto{\pgfqpoint{4.467124in}{3.356742in}}%
\pgfpathlineto{\pgfqpoint{4.468198in}{3.356678in}}%
\pgfpathlineto{\pgfqpoint{4.469272in}{3.355260in}}%
\pgfpathlineto{\pgfqpoint{4.473567in}{3.355940in}}%
\pgfpathlineto{\pgfqpoint{4.476789in}{3.356897in}}%
\pgfpathlineto{\pgfqpoint{4.481084in}{3.355909in}}%
\pgfpathlineto{\pgfqpoint{4.482158in}{3.355384in}}%
\pgfpathlineto{\pgfqpoint{4.483232in}{3.356156in}}%
\pgfpathlineto{\pgfqpoint{4.484306in}{3.355507in}}%
\pgfpathlineto{\pgfqpoint{4.488601in}{3.354982in}}%
\pgfpathlineto{\pgfqpoint{4.489675in}{3.354025in}}%
\pgfpathlineto{\pgfqpoint{4.490749in}{3.352296in}}%
\pgfpathlineto{\pgfqpoint{4.491823in}{3.351987in}}%
\pgfpathlineto{\pgfqpoint{4.497192in}{3.351401in}}%
\pgfpathlineto{\pgfqpoint{4.499339in}{3.348992in}}%
\pgfpathlineto{\pgfqpoint{4.504709in}{3.347960in}}%
\pgfpathlineto{\pgfqpoint{4.506856in}{3.344248in}}%
\pgfpathlineto{\pgfqpoint{4.510078in}{3.342161in}}%
\pgfpathlineto{\pgfqpoint{4.514373in}{3.342411in}}%
\pgfpathlineto{\pgfqpoint{4.521890in}{3.341855in}}%
\pgfpathlineto{\pgfqpoint{4.527259in}{3.340797in}}%
\pgfpathlineto{\pgfqpoint{4.529407in}{3.338886in}}%
\pgfpathlineto{\pgfqpoint{4.533702in}{3.339287in}}%
\pgfpathlineto{\pgfqpoint{4.535850in}{3.338484in}}%
\pgfpathlineto{\pgfqpoint{4.536924in}{3.338939in}}%
\pgfpathlineto{\pgfqpoint{4.540145in}{3.338564in}}%
\pgfpathlineto{\pgfqpoint{4.541219in}{3.337252in}}%
\pgfpathlineto{\pgfqpoint{4.544441in}{3.336341in}}%
\pgfpathlineto{\pgfqpoint{4.547662in}{3.336877in}}%
\pgfpathlineto{\pgfqpoint{4.549810in}{3.334399in}}%
\pgfpathlineto{\pgfqpoint{4.551958in}{3.333337in}}%
\pgfpathlineto{\pgfqpoint{4.559474in}{3.334728in}}%
\pgfpathlineto{\pgfqpoint{4.564844in}{3.334070in}}%
\pgfpathlineto{\pgfqpoint{4.565917in}{3.332780in}}%
\pgfpathlineto{\pgfqpoint{4.566991in}{3.332398in}}%
\pgfpathlineto{\pgfqpoint{4.571287in}{3.332184in}}%
\pgfpathlineto{\pgfqpoint{4.572360in}{3.330609in}}%
\pgfpathlineto{\pgfqpoint{4.578804in}{3.329368in}}%
\pgfpathlineto{\pgfqpoint{4.579877in}{3.328748in}}%
\pgfpathlineto{\pgfqpoint{4.585247in}{3.329320in}}%
\pgfpathlineto{\pgfqpoint{4.586320in}{3.327865in}}%
\pgfpathlineto{\pgfqpoint{4.589542in}{3.326863in}}%
\pgfpathlineto{\pgfqpoint{4.594911in}{3.327552in}}%
\pgfpathlineto{\pgfqpoint{4.595985in}{3.326495in}}%
\pgfpathlineto{\pgfqpoint{4.600280in}{3.326518in}}%
\pgfpathlineto{\pgfqpoint{4.602428in}{3.325494in}}%
\pgfpathlineto{\pgfqpoint{4.604576in}{3.324945in}}%
\pgfpathlineto{\pgfqpoint{4.607797in}{3.324817in}}%
\pgfpathlineto{\pgfqpoint{4.608871in}{3.323836in}}%
\pgfpathlineto{\pgfqpoint{4.615314in}{3.323716in}}%
\pgfpathlineto{\pgfqpoint{4.616388in}{3.323071in}}%
\pgfpathlineto{\pgfqpoint{4.618536in}{3.323168in}}%
\pgfpathlineto{\pgfqpoint{4.619609in}{3.322760in}}%
\pgfpathlineto{\pgfqpoint{4.627126in}{3.321907in}}%
\pgfpathlineto{\pgfqpoint{4.630348in}{3.321287in}}%
\pgfpathlineto{\pgfqpoint{4.633569in}{3.320043in}}%
\pgfpathlineto{\pgfqpoint{4.634643in}{3.319818in}}%
\pgfpathlineto{\pgfqpoint{4.638938in}{3.319714in}}%
\pgfpathlineto{\pgfqpoint{4.641086in}{3.318726in}}%
\pgfpathlineto{\pgfqpoint{4.642160in}{3.318985in}}%
\pgfpathlineto{\pgfqpoint{4.647529in}{3.319131in}}%
\pgfpathlineto{\pgfqpoint{4.648603in}{3.318369in}}%
\pgfpathlineto{\pgfqpoint{4.649677in}{3.316569in}}%
\pgfpathlineto{\pgfqpoint{4.670080in}{3.315729in}}%
\pgfpathlineto{\pgfqpoint{4.672227in}{3.315744in}}%
\pgfpathlineto{\pgfqpoint{4.685114in}{3.315415in}}%
\pgfpathlineto{\pgfqpoint{4.686187in}{3.314818in}}%
\pgfpathlineto{\pgfqpoint{4.687261in}{3.315027in}}%
\pgfpathlineto{\pgfqpoint{4.691557in}{3.314371in}}%
\pgfpathlineto{\pgfqpoint{4.693704in}{3.313911in}}%
\pgfpathlineto{\pgfqpoint{4.701221in}{3.315083in}}%
\pgfpathlineto{\pgfqpoint{4.702295in}{3.314720in}}%
\pgfpathlineto{\pgfqpoint{4.709812in}{3.314497in}}%
\pgfpathlineto{\pgfqpoint{4.720550in}{3.313596in}}%
\pgfpathlineto{\pgfqpoint{4.721624in}{3.312903in}}%
\pgfpathlineto{\pgfqpoint{4.723772in}{3.312967in}}%
\pgfpathlineto{\pgfqpoint{4.736658in}{3.311494in}}%
\pgfpathlineto{\pgfqpoint{4.739879in}{3.310437in}}%
\pgfpathlineto{\pgfqpoint{4.750618in}{3.309192in}}%
\pgfpathlineto{\pgfqpoint{4.751692in}{3.308773in}}%
\pgfpathlineto{\pgfqpoint{4.753839in}{3.308773in}}%
\pgfpathlineto{\pgfqpoint{4.774242in}{3.308736in}}%
\pgfpathlineto{\pgfqpoint{4.777464in}{3.308323in}}%
\pgfpathlineto{\pgfqpoint{4.792497in}{3.308507in}}%
\pgfpathlineto{\pgfqpoint{4.800014in}{3.307312in}}%
\pgfpathlineto{\pgfqpoint{4.812900in}{3.305819in}}%
\pgfpathlineto{\pgfqpoint{4.815048in}{3.305315in}}%
\pgfpathlineto{\pgfqpoint{4.820417in}{3.305192in}}%
\pgfpathlineto{\pgfqpoint{4.830082in}{3.305223in}}%
\pgfpathlineto{\pgfqpoint{4.837599in}{3.304194in}}%
\pgfpathlineto{\pgfqpoint{4.844042in}{3.303799in}}%
\pgfpathlineto{\pgfqpoint{4.850485in}{3.303613in}}%
\pgfpathlineto{\pgfqpoint{4.860149in}{3.303785in}}%
\pgfpathlineto{\pgfqpoint{4.875183in}{3.303132in}}%
\pgfpathlineto{\pgfqpoint{4.886995in}{3.302861in}}%
\pgfpathlineto{\pgfqpoint{4.894512in}{3.302791in}}%
\pgfpathlineto{\pgfqpoint{4.896660in}{3.302446in}}%
\pgfpathlineto{\pgfqpoint{4.904177in}{3.302465in}}%
\pgfpathlineto{\pgfqpoint{4.905250in}{3.302017in}}%
\pgfpathlineto{\pgfqpoint{4.925653in}{3.301373in}}%
\pgfpathlineto{\pgfqpoint{4.927801in}{3.301337in}}%
\pgfpathlineto{\pgfqpoint{4.979345in}{3.301010in}}%
\pgfpathlineto{\pgfqpoint{4.980419in}{3.300409in}}%
\pgfpathlineto{\pgfqpoint{5.055588in}{3.298669in}}%
\pgfpathlineto{\pgfqpoint{5.067400in}{3.298385in}}%
\pgfpathlineto{\pgfqpoint{5.089951in}{3.298433in}}%
\pgfpathlineto{\pgfqpoint{5.103911in}{3.298466in}}%
\pgfpathlineto{\pgfqpoint{5.111427in}{3.298332in}}%
\pgfpathlineto{\pgfqpoint{5.115723in}{3.298271in}}%
\pgfpathlineto{\pgfqpoint{5.128609in}{3.298107in}}%
\pgfpathlineto{\pgfqpoint{5.133978in}{3.297745in}}%
\pgfpathlineto{\pgfqpoint{5.143643in}{3.297800in}}%
\pgfpathlineto{\pgfqpoint{5.197335in}{3.297258in}}%
\pgfpathlineto{\pgfqpoint{5.198408in}{3.296756in}}%
\pgfpathlineto{\pgfqpoint{5.232771in}{3.297178in}}%
\pgfpathlineto{\pgfqpoint{5.234919in}{3.296324in}}%
\pgfpathlineto{\pgfqpoint{5.272503in}{3.295774in}}%
\pgfpathlineto{\pgfqpoint{5.288611in}{3.295749in}}%
\pgfpathlineto{\pgfqpoint{5.311161in}{3.295626in}}%
\pgfpathlineto{\pgfqpoint{5.330490in}{3.294487in}}%
\pgfpathlineto{\pgfqpoint{5.338007in}{3.294112in}}%
\pgfpathlineto{\pgfqpoint{5.362706in}{3.293671in}}%
\pgfpathlineto{\pgfqpoint{5.375592in}{3.293471in}}%
\pgfpathlineto{\pgfqpoint{5.385256in}{3.293467in}}%
\pgfpathlineto{\pgfqpoint{5.399216in}{3.293224in}}%
\pgfpathlineto{\pgfqpoint{5.408881in}{3.293051in}}%
\pgfpathlineto{\pgfqpoint{5.505526in}{3.291684in}}%
\pgfpathlineto{\pgfqpoint{5.511969in}{3.291510in}}%
\pgfpathlineto{\pgfqpoint{5.633313in}{3.290957in}}%
\pgfpathlineto{\pgfqpoint{5.684857in}{3.290585in}}%
\pgfpathlineto{\pgfqpoint{5.932914in}{3.290149in}}%
\pgfpathlineto{\pgfqpoint{6.077882in}{3.290094in}}%
\pgfpathlineto{\pgfqpoint{6.589029in}{3.289994in}}%
\pgfpathlineto{\pgfqpoint{6.589029in}{3.289994in}}%
\pgfusepath{stroke}%
\end{pgfscope}%
\begin{pgfscope}%
\pgfsetrectcap%
\pgfsetmiterjoin%
\pgfsetlinewidth{0.803000pt}%
\definecolor{currentstroke}{rgb}{1.000000,1.000000,1.000000}%
\pgfsetstrokecolor{currentstroke}%
\pgfsetdash{}{0pt}%
\pgfpathmoveto{\pgfqpoint{4.123120in}{3.271772in}}%
\pgfpathlineto{\pgfqpoint{4.123120in}{3.672657in}}%
\pgfusepath{stroke}%
\end{pgfscope}%
\begin{pgfscope}%
\pgfsetrectcap%
\pgfsetmiterjoin%
\pgfsetlinewidth{0.803000pt}%
\definecolor{currentstroke}{rgb}{1.000000,1.000000,1.000000}%
\pgfsetstrokecolor{currentstroke}%
\pgfsetdash{}{0pt}%
\pgfpathmoveto{\pgfqpoint{6.706453in}{3.271772in}}%
\pgfpathlineto{\pgfqpoint{6.706453in}{3.672657in}}%
\pgfusepath{stroke}%
\end{pgfscope}%
\begin{pgfscope}%
\pgfsetrectcap%
\pgfsetmiterjoin%
\pgfsetlinewidth{0.803000pt}%
\definecolor{currentstroke}{rgb}{1.000000,1.000000,1.000000}%
\pgfsetstrokecolor{currentstroke}%
\pgfsetdash{}{0pt}%
\pgfpathmoveto{\pgfqpoint{4.123120in}{3.271772in}}%
\pgfpathlineto{\pgfqpoint{6.706453in}{3.271772in}}%
\pgfusepath{stroke}%
\end{pgfscope}%
\begin{pgfscope}%
\pgfsetrectcap%
\pgfsetmiterjoin%
\pgfsetlinewidth{0.803000pt}%
\definecolor{currentstroke}{rgb}{1.000000,1.000000,1.000000}%
\pgfsetstrokecolor{currentstroke}%
\pgfsetdash{}{0pt}%
\pgfpathmoveto{\pgfqpoint{4.123120in}{3.672657in}}%
\pgfpathlineto{\pgfqpoint{6.706453in}{3.672657in}}%
\pgfusepath{stroke}%
\end{pgfscope}%
\begin{pgfscope}%
\definecolor{textcolor}{rgb}{0.150000,0.150000,0.150000}%
\pgfsetstrokecolor{textcolor}%
\pgfsetfillcolor{textcolor}%
\pgftext[x=5.414787in,y=3.755990in,,base]{\color{textcolor}\rmfamily\fontsize{16.800000}{20.160000}\selectfont INTC}%
\end{pgfscope}%
\begin{pgfscope}%
\pgfsetbuttcap%
\pgfsetmiterjoin%
\definecolor{currentfill}{rgb}{0.917647,0.917647,0.949020}%
\pgfsetfillcolor{currentfill}%
\pgfsetlinewidth{0.000000pt}%
\definecolor{currentstroke}{rgb}{0.000000,0.000000,0.000000}%
\pgfsetstrokecolor{currentstroke}%
\pgfsetstrokeopacity{0.000000}%
\pgfsetdash{}{0pt}%
\pgfpathmoveto{\pgfqpoint{0.506453in}{2.309648in}}%
\pgfpathlineto{\pgfqpoint{3.089787in}{2.309648in}}%
\pgfpathlineto{\pgfqpoint{3.089787in}{2.710533in}}%
\pgfpathlineto{\pgfqpoint{0.506453in}{2.710533in}}%
\pgfpathclose%
\pgfusepath{fill}%
\end{pgfscope}%
\begin{pgfscope}%
\pgfpathrectangle{\pgfqpoint{0.506453in}{2.309648in}}{\pgfqpoint{2.583333in}{0.400885in}}%
\pgfusepath{clip}%
\pgfsetroundcap%
\pgfsetroundjoin%
\pgfsetlinewidth{0.803000pt}%
\definecolor{currentstroke}{rgb}{1.000000,1.000000,1.000000}%
\pgfsetstrokecolor{currentstroke}%
\pgfsetdash{}{0pt}%
\pgfpathmoveto{\pgfqpoint{0.621730in}{2.309648in}}%
\pgfpathlineto{\pgfqpoint{0.621730in}{2.710533in}}%
\pgfusepath{stroke}%
\end{pgfscope}%
\begin{pgfscope}%
\definecolor{textcolor}{rgb}{0.150000,0.150000,0.150000}%
\pgfsetstrokecolor{textcolor}%
\pgfsetfillcolor{textcolor}%
\pgftext[x=0.621730in,y=2.212426in,,top]{\color{textcolor}\rmfamily\fontsize{14.000000}{16.800000}\selectfont 2012}%
\end{pgfscope}%
\begin{pgfscope}%
\pgfpathrectangle{\pgfqpoint{0.506453in}{2.309648in}}{\pgfqpoint{2.583333in}{0.400885in}}%
\pgfusepath{clip}%
\pgfsetroundcap%
\pgfsetroundjoin%
\pgfsetlinewidth{0.803000pt}%
\definecolor{currentstroke}{rgb}{1.000000,1.000000,1.000000}%
\pgfsetstrokecolor{currentstroke}%
\pgfsetdash{}{0pt}%
\pgfpathmoveto{\pgfqpoint{1.014755in}{2.309648in}}%
\pgfpathlineto{\pgfqpoint{1.014755in}{2.710533in}}%
\pgfusepath{stroke}%
\end{pgfscope}%
\begin{pgfscope}%
\definecolor{textcolor}{rgb}{0.150000,0.150000,0.150000}%
\pgfsetstrokecolor{textcolor}%
\pgfsetfillcolor{textcolor}%
\pgftext[x=1.014755in,y=2.212426in,,top]{\color{textcolor}\rmfamily\fontsize{14.000000}{16.800000}\selectfont 2013}%
\end{pgfscope}%
\begin{pgfscope}%
\pgfpathrectangle{\pgfqpoint{0.506453in}{2.309648in}}{\pgfqpoint{2.583333in}{0.400885in}}%
\pgfusepath{clip}%
\pgfsetroundcap%
\pgfsetroundjoin%
\pgfsetlinewidth{0.803000pt}%
\definecolor{currentstroke}{rgb}{1.000000,1.000000,1.000000}%
\pgfsetstrokecolor{currentstroke}%
\pgfsetdash{}{0pt}%
\pgfpathmoveto{\pgfqpoint{1.406706in}{2.309648in}}%
\pgfpathlineto{\pgfqpoint{1.406706in}{2.710533in}}%
\pgfusepath{stroke}%
\end{pgfscope}%
\begin{pgfscope}%
\definecolor{textcolor}{rgb}{0.150000,0.150000,0.150000}%
\pgfsetstrokecolor{textcolor}%
\pgfsetfillcolor{textcolor}%
\pgftext[x=1.406706in,y=2.212426in,,top]{\color{textcolor}\rmfamily\fontsize{14.000000}{16.800000}\selectfont 2014}%
\end{pgfscope}%
\begin{pgfscope}%
\pgfpathrectangle{\pgfqpoint{0.506453in}{2.309648in}}{\pgfqpoint{2.583333in}{0.400885in}}%
\pgfusepath{clip}%
\pgfsetroundcap%
\pgfsetroundjoin%
\pgfsetlinewidth{0.803000pt}%
\definecolor{currentstroke}{rgb}{1.000000,1.000000,1.000000}%
\pgfsetstrokecolor{currentstroke}%
\pgfsetdash{}{0pt}%
\pgfpathmoveto{\pgfqpoint{1.798657in}{2.309648in}}%
\pgfpathlineto{\pgfqpoint{1.798657in}{2.710533in}}%
\pgfusepath{stroke}%
\end{pgfscope}%
\begin{pgfscope}%
\definecolor{textcolor}{rgb}{0.150000,0.150000,0.150000}%
\pgfsetstrokecolor{textcolor}%
\pgfsetfillcolor{textcolor}%
\pgftext[x=1.798657in,y=2.212426in,,top]{\color{textcolor}\rmfamily\fontsize{14.000000}{16.800000}\selectfont 2015}%
\end{pgfscope}%
\begin{pgfscope}%
\pgfpathrectangle{\pgfqpoint{0.506453in}{2.309648in}}{\pgfqpoint{2.583333in}{0.400885in}}%
\pgfusepath{clip}%
\pgfsetroundcap%
\pgfsetroundjoin%
\pgfsetlinewidth{0.803000pt}%
\definecolor{currentstroke}{rgb}{1.000000,1.000000,1.000000}%
\pgfsetstrokecolor{currentstroke}%
\pgfsetdash{}{0pt}%
\pgfpathmoveto{\pgfqpoint{2.190608in}{2.309648in}}%
\pgfpathlineto{\pgfqpoint{2.190608in}{2.710533in}}%
\pgfusepath{stroke}%
\end{pgfscope}%
\begin{pgfscope}%
\definecolor{textcolor}{rgb}{0.150000,0.150000,0.150000}%
\pgfsetstrokecolor{textcolor}%
\pgfsetfillcolor{textcolor}%
\pgftext[x=2.190608in,y=2.212426in,,top]{\color{textcolor}\rmfamily\fontsize{14.000000}{16.800000}\selectfont 2016}%
\end{pgfscope}%
\begin{pgfscope}%
\pgfpathrectangle{\pgfqpoint{0.506453in}{2.309648in}}{\pgfqpoint{2.583333in}{0.400885in}}%
\pgfusepath{clip}%
\pgfsetroundcap%
\pgfsetroundjoin%
\pgfsetlinewidth{0.803000pt}%
\definecolor{currentstroke}{rgb}{1.000000,1.000000,1.000000}%
\pgfsetstrokecolor{currentstroke}%
\pgfsetdash{}{0pt}%
\pgfpathmoveto{\pgfqpoint{2.583633in}{2.309648in}}%
\pgfpathlineto{\pgfqpoint{2.583633in}{2.710533in}}%
\pgfusepath{stroke}%
\end{pgfscope}%
\begin{pgfscope}%
\definecolor{textcolor}{rgb}{0.150000,0.150000,0.150000}%
\pgfsetstrokecolor{textcolor}%
\pgfsetfillcolor{textcolor}%
\pgftext[x=2.583633in,y=2.212426in,,top]{\color{textcolor}\rmfamily\fontsize{14.000000}{16.800000}\selectfont 2017}%
\end{pgfscope}%
\begin{pgfscope}%
\pgfpathrectangle{\pgfqpoint{0.506453in}{2.309648in}}{\pgfqpoint{2.583333in}{0.400885in}}%
\pgfusepath{clip}%
\pgfsetroundcap%
\pgfsetroundjoin%
\pgfsetlinewidth{0.803000pt}%
\definecolor{currentstroke}{rgb}{1.000000,1.000000,1.000000}%
\pgfsetstrokecolor{currentstroke}%
\pgfsetdash{}{0pt}%
\pgfpathmoveto{\pgfqpoint{2.975584in}{2.309648in}}%
\pgfpathlineto{\pgfqpoint{2.975584in}{2.710533in}}%
\pgfusepath{stroke}%
\end{pgfscope}%
\begin{pgfscope}%
\definecolor{textcolor}{rgb}{0.150000,0.150000,0.150000}%
\pgfsetstrokecolor{textcolor}%
\pgfsetfillcolor{textcolor}%
\pgftext[x=2.975584in,y=2.212426in,,top]{\color{textcolor}\rmfamily\fontsize{14.000000}{16.800000}\selectfont 2018}%
\end{pgfscope}%
\begin{pgfscope}%
\pgfpathrectangle{\pgfqpoint{0.506453in}{2.309648in}}{\pgfqpoint{2.583333in}{0.400885in}}%
\pgfusepath{clip}%
\pgfsetroundcap%
\pgfsetroundjoin%
\pgfsetlinewidth{0.803000pt}%
\definecolor{currentstroke}{rgb}{1.000000,1.000000,1.000000}%
\pgfsetstrokecolor{currentstroke}%
\pgfsetdash{}{0pt}%
\pgfpathmoveto{\pgfqpoint{0.506453in}{2.327152in}}%
\pgfpathlineto{\pgfqpoint{3.089787in}{2.327152in}}%
\pgfusepath{stroke}%
\end{pgfscope}%
\begin{pgfscope}%
\definecolor{textcolor}{rgb}{0.150000,0.150000,0.150000}%
\pgfsetstrokecolor{textcolor}%
\pgfsetfillcolor{textcolor}%
\pgftext[x=0.100000in,y=2.253286in,left,base]{\color{textcolor}\rmfamily\fontsize{14.000000}{16.800000}\selectfont 0.0}%
\end{pgfscope}%
\begin{pgfscope}%
\pgfpathrectangle{\pgfqpoint{0.506453in}{2.309648in}}{\pgfqpoint{2.583333in}{0.400885in}}%
\pgfusepath{clip}%
\pgfsetroundcap%
\pgfsetroundjoin%
\pgfsetlinewidth{0.803000pt}%
\definecolor{currentstroke}{rgb}{1.000000,1.000000,1.000000}%
\pgfsetstrokecolor{currentstroke}%
\pgfsetdash{}{0pt}%
\pgfpathmoveto{\pgfqpoint{0.506453in}{2.679980in}}%
\pgfpathlineto{\pgfqpoint{3.089787in}{2.679980in}}%
\pgfusepath{stroke}%
\end{pgfscope}%
\begin{pgfscope}%
\definecolor{textcolor}{rgb}{0.150000,0.150000,0.150000}%
\pgfsetstrokecolor{textcolor}%
\pgfsetfillcolor{textcolor}%
\pgftext[x=0.100000in,y=2.606114in,left,base]{\color{textcolor}\rmfamily\fontsize{14.000000}{16.800000}\selectfont 2.5}%
\end{pgfscope}%
\begin{pgfscope}%
\pgfpathrectangle{\pgfqpoint{0.506453in}{2.309648in}}{\pgfqpoint{2.583333in}{0.400885in}}%
\pgfusepath{clip}%
\pgfsetroundcap%
\pgfsetroundjoin%
\pgfsetlinewidth{1.505625pt}%
\definecolor{currentstroke}{rgb}{0.000000,0.000000,0.000000}%
\pgfsetstrokecolor{currentstroke}%
\pgfsetdash{}{0pt}%
\pgfpathmoveto{\pgfqpoint{0.623878in}{2.468283in}}%
\pgfpathlineto{\pgfqpoint{0.627099in}{2.466049in}}%
\pgfpathlineto{\pgfqpoint{0.642133in}{2.466980in}}%
\pgfpathlineto{\pgfqpoint{0.646428in}{2.466421in}}%
\pgfpathlineto{\pgfqpoint{0.649650in}{2.467618in}}%
\pgfpathlineto{\pgfqpoint{0.655019in}{2.467884in}}%
\pgfpathlineto{\pgfqpoint{0.668979in}{2.465570in}}%
\pgfpathlineto{\pgfqpoint{0.671127in}{2.466235in}}%
\pgfpathlineto{\pgfqpoint{0.676496in}{2.466501in}}%
\pgfpathlineto{\pgfqpoint{0.682939in}{2.466448in}}%
\pgfpathlineto{\pgfqpoint{0.684013in}{2.468017in}}%
\pgfpathlineto{\pgfqpoint{0.687234in}{2.467139in}}%
\pgfpathlineto{\pgfqpoint{0.690456in}{2.467432in}}%
\pgfpathlineto{\pgfqpoint{0.691530in}{2.466235in}}%
\pgfpathlineto{\pgfqpoint{0.692603in}{2.466128in}}%
\pgfpathlineto{\pgfqpoint{0.693677in}{2.467326in}}%
\pgfpathlineto{\pgfqpoint{0.694751in}{2.467086in}}%
\pgfpathlineto{\pgfqpoint{0.700120in}{2.467804in}}%
\pgfpathlineto{\pgfqpoint{0.707637in}{2.467113in}}%
\pgfpathlineto{\pgfqpoint{0.708711in}{2.466474in}}%
\pgfpathlineto{\pgfqpoint{0.709785in}{2.466660in}}%
\pgfpathlineto{\pgfqpoint{0.717302in}{2.469720in}}%
\pgfpathlineto{\pgfqpoint{0.720523in}{2.470252in}}%
\pgfpathlineto{\pgfqpoint{0.723745in}{2.468363in}}%
\pgfpathlineto{\pgfqpoint{0.728040in}{2.467459in}}%
\pgfpathlineto{\pgfqpoint{0.729114in}{2.465916in}}%
\pgfpathlineto{\pgfqpoint{0.731262in}{2.465809in}}%
\pgfpathlineto{\pgfqpoint{0.732335in}{2.464479in}}%
\pgfpathlineto{\pgfqpoint{0.736631in}{2.465942in}}%
\pgfpathlineto{\pgfqpoint{0.737705in}{2.463867in}}%
\pgfpathlineto{\pgfqpoint{0.738778in}{2.463388in}}%
\pgfpathlineto{\pgfqpoint{0.739852in}{2.464851in}}%
\pgfpathlineto{\pgfqpoint{0.743074in}{2.464106in}}%
\pgfpathlineto{\pgfqpoint{0.747369in}{2.467299in}}%
\pgfpathlineto{\pgfqpoint{0.753812in}{2.468363in}}%
\pgfpathlineto{\pgfqpoint{0.754886in}{2.467086in}}%
\pgfpathlineto{\pgfqpoint{0.759181in}{2.467592in}}%
\pgfpathlineto{\pgfqpoint{0.760255in}{2.466075in}}%
\pgfpathlineto{\pgfqpoint{0.761329in}{2.466714in}}%
\pgfpathlineto{\pgfqpoint{0.762403in}{2.466208in}}%
\pgfpathlineto{\pgfqpoint{0.769920in}{2.464080in}}%
\pgfpathlineto{\pgfqpoint{0.774215in}{2.464452in}}%
\pgfpathlineto{\pgfqpoint{0.775289in}{2.463894in}}%
\pgfpathlineto{\pgfqpoint{0.776363in}{2.464851in}}%
\pgfpathlineto{\pgfqpoint{0.777437in}{2.463574in}}%
\pgfpathlineto{\pgfqpoint{0.781732in}{2.463761in}}%
\pgfpathlineto{\pgfqpoint{0.782806in}{2.462909in}}%
\pgfpathlineto{\pgfqpoint{0.783880in}{2.463388in}}%
\pgfpathlineto{\pgfqpoint{0.784953in}{2.461978in}}%
\pgfpathlineto{\pgfqpoint{0.788175in}{2.463202in}}%
\pgfpathlineto{\pgfqpoint{0.789249in}{2.462909in}}%
\pgfpathlineto{\pgfqpoint{0.790323in}{2.464213in}}%
\pgfpathlineto{\pgfqpoint{0.792470in}{2.464585in}}%
\pgfpathlineto{\pgfqpoint{0.795692in}{2.462723in}}%
\pgfpathlineto{\pgfqpoint{0.799987in}{2.471210in}}%
\pgfpathlineto{\pgfqpoint{0.803209in}{2.471848in}}%
\pgfpathlineto{\pgfqpoint{0.805356in}{2.473364in}}%
\pgfpathlineto{\pgfqpoint{0.806430in}{2.472034in}}%
\pgfpathlineto{\pgfqpoint{0.807504in}{2.472566in}}%
\pgfpathlineto{\pgfqpoint{0.811799in}{2.472114in}}%
\pgfpathlineto{\pgfqpoint{0.815021in}{2.474588in}}%
\pgfpathlineto{\pgfqpoint{0.819316in}{2.475652in}}%
\pgfpathlineto{\pgfqpoint{0.822538in}{2.474774in}}%
\pgfpathlineto{\pgfqpoint{0.828981in}{2.474934in}}%
\pgfpathlineto{\pgfqpoint{0.830055in}{2.476876in}}%
\pgfpathlineto{\pgfqpoint{0.833276in}{2.476530in}}%
\pgfpathlineto{\pgfqpoint{0.835424in}{2.478552in}}%
\pgfpathlineto{\pgfqpoint{0.836498in}{2.478898in}}%
\pgfpathlineto{\pgfqpoint{0.837572in}{2.476929in}}%
\pgfpathlineto{\pgfqpoint{0.840793in}{2.475785in}}%
\pgfpathlineto{\pgfqpoint{0.841867in}{2.474136in}}%
\pgfpathlineto{\pgfqpoint{0.842941in}{2.474535in}}%
\pgfpathlineto{\pgfqpoint{0.845088in}{2.478871in}}%
\pgfpathlineto{\pgfqpoint{0.850458in}{2.478579in}}%
\pgfpathlineto{\pgfqpoint{0.851531in}{2.476530in}}%
\pgfpathlineto{\pgfqpoint{0.852605in}{2.477993in}}%
\pgfpathlineto{\pgfqpoint{0.855827in}{2.477382in}}%
\pgfpathlineto{\pgfqpoint{0.856901in}{2.476184in}}%
\pgfpathlineto{\pgfqpoint{0.864418in}{2.476956in}}%
\pgfpathlineto{\pgfqpoint{0.867639in}{2.475120in}}%
\pgfpathlineto{\pgfqpoint{0.874082in}{2.474987in}}%
\pgfpathlineto{\pgfqpoint{0.875156in}{2.476025in}}%
\pgfpathlineto{\pgfqpoint{0.888042in}{2.475307in}}%
\pgfpathlineto{\pgfqpoint{0.889116in}{2.476557in}}%
\pgfpathlineto{\pgfqpoint{0.895559in}{2.477222in}}%
\pgfpathlineto{\pgfqpoint{0.896633in}{2.479084in}}%
\pgfpathlineto{\pgfqpoint{0.897707in}{2.477940in}}%
\pgfpathlineto{\pgfqpoint{0.900928in}{2.477461in}}%
\pgfpathlineto{\pgfqpoint{0.904150in}{2.478871in}}%
\pgfpathlineto{\pgfqpoint{0.905223in}{2.479244in}}%
\pgfpathlineto{\pgfqpoint{0.908445in}{2.479111in}}%
\pgfpathlineto{\pgfqpoint{0.909519in}{2.479803in}}%
\pgfpathlineto{\pgfqpoint{0.910593in}{2.479111in}}%
\pgfpathlineto{\pgfqpoint{0.919183in}{2.479696in}}%
\pgfpathlineto{\pgfqpoint{0.920257in}{2.480547in}}%
\pgfpathlineto{\pgfqpoint{0.923479in}{2.480069in}}%
\pgfpathlineto{\pgfqpoint{0.924553in}{2.477807in}}%
\pgfpathlineto{\pgfqpoint{0.927774in}{2.476850in}}%
\pgfpathlineto{\pgfqpoint{0.930996in}{2.478233in}}%
\pgfpathlineto{\pgfqpoint{0.933143in}{2.483500in}}%
\pgfpathlineto{\pgfqpoint{0.934217in}{2.486852in}}%
\pgfpathlineto{\pgfqpoint{0.935291in}{2.485416in}}%
\pgfpathlineto{\pgfqpoint{0.938512in}{2.485230in}}%
\pgfpathlineto{\pgfqpoint{0.939586in}{2.483261in}}%
\pgfpathlineto{\pgfqpoint{0.940660in}{2.482942in}}%
\pgfpathlineto{\pgfqpoint{0.941734in}{2.483820in}}%
\pgfpathlineto{\pgfqpoint{0.942808in}{2.483288in}}%
\pgfpathlineto{\pgfqpoint{0.948177in}{2.483101in}}%
\pgfpathlineto{\pgfqpoint{0.949251in}{2.484618in}}%
\pgfpathlineto{\pgfqpoint{0.950325in}{2.483288in}}%
\pgfpathlineto{\pgfqpoint{0.953546in}{2.483048in}}%
\pgfpathlineto{\pgfqpoint{0.954620in}{2.483527in}}%
\pgfpathlineto{\pgfqpoint{0.956768in}{2.480547in}}%
\pgfpathlineto{\pgfqpoint{0.957841in}{2.481026in}}%
\pgfpathlineto{\pgfqpoint{0.962137in}{2.480228in}}%
\pgfpathlineto{\pgfqpoint{0.964285in}{2.479270in}}%
\pgfpathlineto{\pgfqpoint{0.965358in}{2.479536in}}%
\pgfpathlineto{\pgfqpoint{0.968580in}{2.479643in}}%
\pgfpathlineto{\pgfqpoint{0.969654in}{2.480574in}}%
\pgfpathlineto{\pgfqpoint{0.970728in}{2.480414in}}%
\pgfpathlineto{\pgfqpoint{0.972875in}{2.481691in}}%
\pgfpathlineto{\pgfqpoint{0.977171in}{2.480015in}}%
\pgfpathlineto{\pgfqpoint{0.978244in}{2.481106in}}%
\pgfpathlineto{\pgfqpoint{0.979318in}{2.480946in}}%
\pgfpathlineto{\pgfqpoint{0.980392in}{2.482064in}}%
\pgfpathlineto{\pgfqpoint{0.984687in}{2.482356in}}%
\pgfpathlineto{\pgfqpoint{0.986835in}{2.482782in}}%
\pgfpathlineto{\pgfqpoint{0.987909in}{2.483660in}}%
\pgfpathlineto{\pgfqpoint{0.991130in}{2.484006in}}%
\pgfpathlineto{\pgfqpoint{0.992204in}{2.485123in}}%
\pgfpathlineto{\pgfqpoint{0.995426in}{2.484192in}}%
\pgfpathlineto{\pgfqpoint{0.999721in}{2.484777in}}%
\pgfpathlineto{\pgfqpoint{1.000795in}{2.484059in}}%
\pgfpathlineto{\pgfqpoint{1.001869in}{2.484325in}}%
\pgfpathlineto{\pgfqpoint{1.002943in}{2.483261in}}%
\pgfpathlineto{\pgfqpoint{1.006164in}{2.482702in}}%
\pgfpathlineto{\pgfqpoint{1.009386in}{2.482862in}}%
\pgfpathlineto{\pgfqpoint{1.010460in}{2.481505in}}%
\pgfpathlineto{\pgfqpoint{1.013681in}{2.482889in}}%
\pgfpathlineto{\pgfqpoint{1.015829in}{2.484538in}}%
\pgfpathlineto{\pgfqpoint{1.016903in}{2.484325in}}%
\pgfpathlineto{\pgfqpoint{1.017976in}{2.486108in}}%
\pgfpathlineto{\pgfqpoint{1.022272in}{2.485815in}}%
\pgfpathlineto{\pgfqpoint{1.025493in}{2.487890in}}%
\pgfpathlineto{\pgfqpoint{1.028715in}{2.488369in}}%
\pgfpathlineto{\pgfqpoint{1.029789in}{2.487943in}}%
\pgfpathlineto{\pgfqpoint{1.033010in}{2.489859in}}%
\pgfpathlineto{\pgfqpoint{1.037306in}{2.488635in}}%
\pgfpathlineto{\pgfqpoint{1.039453in}{2.489566in}}%
\pgfpathlineto{\pgfqpoint{1.040527in}{2.491375in}}%
\pgfpathlineto{\pgfqpoint{1.043749in}{2.490710in}}%
\pgfpathlineto{\pgfqpoint{1.044822in}{2.492466in}}%
\pgfpathlineto{\pgfqpoint{1.046970in}{2.491375in}}%
\pgfpathlineto{\pgfqpoint{1.048044in}{2.491960in}}%
\pgfpathlineto{\pgfqpoint{1.051265in}{2.491801in}}%
\pgfpathlineto{\pgfqpoint{1.053413in}{2.494647in}}%
\pgfpathlineto{\pgfqpoint{1.054487in}{2.493902in}}%
\pgfpathlineto{\pgfqpoint{1.055561in}{2.494833in}}%
\pgfpathlineto{\pgfqpoint{1.058782in}{2.494700in}}%
\pgfpathlineto{\pgfqpoint{1.059856in}{2.495552in}}%
\pgfpathlineto{\pgfqpoint{1.060930in}{2.495233in}}%
\pgfpathlineto{\pgfqpoint{1.063078in}{2.496350in}}%
\pgfpathlineto{\pgfqpoint{1.067373in}{2.498132in}}%
\pgfpathlineto{\pgfqpoint{1.068447in}{2.497441in}}%
\pgfpathlineto{\pgfqpoint{1.070595in}{2.497919in}}%
\pgfpathlineto{\pgfqpoint{1.073816in}{2.496376in}}%
\pgfpathlineto{\pgfqpoint{1.074890in}{2.496802in}}%
\pgfpathlineto{\pgfqpoint{1.075964in}{2.498079in}}%
\pgfpathlineto{\pgfqpoint{1.077038in}{2.497600in}}%
\pgfpathlineto{\pgfqpoint{1.078111in}{2.498930in}}%
\pgfpathlineto{\pgfqpoint{1.081333in}{2.500048in}}%
\pgfpathlineto{\pgfqpoint{1.082407in}{2.501059in}}%
\pgfpathlineto{\pgfqpoint{1.083481in}{2.500473in}}%
\pgfpathlineto{\pgfqpoint{1.085628in}{2.502256in}}%
\pgfpathlineto{\pgfqpoint{1.090997in}{2.503054in}}%
\pgfpathlineto{\pgfqpoint{1.092071in}{2.504304in}}%
\pgfpathlineto{\pgfqpoint{1.093145in}{2.504491in}}%
\pgfpathlineto{\pgfqpoint{1.097441in}{2.503746in}}%
\pgfpathlineto{\pgfqpoint{1.098514in}{2.505076in}}%
\pgfpathlineto{\pgfqpoint{1.099588in}{2.504091in}}%
\pgfpathlineto{\pgfqpoint{1.100662in}{2.505714in}}%
\pgfpathlineto{\pgfqpoint{1.103884in}{2.505581in}}%
\pgfpathlineto{\pgfqpoint{1.104957in}{2.508215in}}%
\pgfpathlineto{\pgfqpoint{1.107105in}{2.509731in}}%
\pgfpathlineto{\pgfqpoint{1.111400in}{2.510636in}}%
\pgfpathlineto{\pgfqpoint{1.112474in}{2.512339in}}%
\pgfpathlineto{\pgfqpoint{1.113548in}{2.510955in}}%
\pgfpathlineto{\pgfqpoint{1.114622in}{2.511700in}}%
\pgfpathlineto{\pgfqpoint{1.118917in}{2.508800in}}%
\pgfpathlineto{\pgfqpoint{1.123213in}{2.512445in}}%
\pgfpathlineto{\pgfqpoint{1.126434in}{2.510130in}}%
\pgfpathlineto{\pgfqpoint{1.127508in}{2.514015in}}%
\pgfpathlineto{\pgfqpoint{1.128582in}{2.515052in}}%
\pgfpathlineto{\pgfqpoint{1.129656in}{2.513429in}}%
\pgfpathlineto{\pgfqpoint{1.130729in}{2.516356in}}%
\pgfpathlineto{\pgfqpoint{1.133951in}{2.517127in}}%
\pgfpathlineto{\pgfqpoint{1.135025in}{2.518511in}}%
\pgfpathlineto{\pgfqpoint{1.136099in}{2.516143in}}%
\pgfpathlineto{\pgfqpoint{1.137173in}{2.518005in}}%
\pgfpathlineto{\pgfqpoint{1.138246in}{2.517766in}}%
\pgfpathlineto{\pgfqpoint{1.141468in}{2.518803in}}%
\pgfpathlineto{\pgfqpoint{1.142542in}{2.518032in}}%
\pgfpathlineto{\pgfqpoint{1.143616in}{2.515770in}}%
\pgfpathlineto{\pgfqpoint{1.145763in}{2.519176in}}%
\pgfpathlineto{\pgfqpoint{1.148985in}{2.516781in}}%
\pgfpathlineto{\pgfqpoint{1.150059in}{2.518697in}}%
\pgfpathlineto{\pgfqpoint{1.152206in}{2.517845in}}%
\pgfpathlineto{\pgfqpoint{1.153280in}{2.519202in}}%
\pgfpathlineto{\pgfqpoint{1.156502in}{2.519415in}}%
\pgfpathlineto{\pgfqpoint{1.158649in}{2.523406in}}%
\pgfpathlineto{\pgfqpoint{1.159723in}{2.522980in}}%
\pgfpathlineto{\pgfqpoint{1.160797in}{2.524417in}}%
\pgfpathlineto{\pgfqpoint{1.164018in}{2.524257in}}%
\pgfpathlineto{\pgfqpoint{1.165092in}{2.525534in}}%
\pgfpathlineto{\pgfqpoint{1.166166in}{2.525268in}}%
\pgfpathlineto{\pgfqpoint{1.168314in}{2.523033in}}%
\pgfpathlineto{\pgfqpoint{1.172609in}{2.524816in}}%
\pgfpathlineto{\pgfqpoint{1.173683in}{2.520399in}}%
\pgfpathlineto{\pgfqpoint{1.174757in}{2.521171in}}%
\pgfpathlineto{\pgfqpoint{1.175831in}{2.517074in}}%
\pgfpathlineto{\pgfqpoint{1.179052in}{2.518271in}}%
\pgfpathlineto{\pgfqpoint{1.181200in}{2.515957in}}%
\pgfpathlineto{\pgfqpoint{1.183348in}{2.518723in}}%
\pgfpathlineto{\pgfqpoint{1.186569in}{2.519255in}}%
\pgfpathlineto{\pgfqpoint{1.187643in}{2.518245in}}%
\pgfpathlineto{\pgfqpoint{1.188717in}{2.516090in}}%
\pgfpathlineto{\pgfqpoint{1.189791in}{2.518723in}}%
\pgfpathlineto{\pgfqpoint{1.190864in}{2.518723in}}%
\pgfpathlineto{\pgfqpoint{1.194086in}{2.520346in}}%
\pgfpathlineto{\pgfqpoint{1.195160in}{2.521996in}}%
\pgfpathlineto{\pgfqpoint{1.197307in}{2.513562in}}%
\pgfpathlineto{\pgfqpoint{1.202677in}{2.519734in}}%
\pgfpathlineto{\pgfqpoint{1.203751in}{2.523432in}}%
\pgfpathlineto{\pgfqpoint{1.204824in}{2.522794in}}%
\pgfpathlineto{\pgfqpoint{1.205898in}{2.520878in}}%
\pgfpathlineto{\pgfqpoint{1.209120in}{2.522607in}}%
\pgfpathlineto{\pgfqpoint{1.210194in}{2.522474in}}%
\pgfpathlineto{\pgfqpoint{1.211267in}{2.522953in}}%
\pgfpathlineto{\pgfqpoint{1.213415in}{2.525401in}}%
\pgfpathlineto{\pgfqpoint{1.218784in}{2.528513in}}%
\pgfpathlineto{\pgfqpoint{1.220932in}{2.530189in}}%
\pgfpathlineto{\pgfqpoint{1.225227in}{2.531121in}}%
\pgfpathlineto{\pgfqpoint{1.226301in}{2.530509in}}%
\pgfpathlineto{\pgfqpoint{1.227375in}{2.530615in}}%
\pgfpathlineto{\pgfqpoint{1.228449in}{2.535244in}}%
\pgfpathlineto{\pgfqpoint{1.233818in}{2.535537in}}%
\pgfpathlineto{\pgfqpoint{1.239187in}{2.537452in}}%
\pgfpathlineto{\pgfqpoint{1.240261in}{2.537372in}}%
\pgfpathlineto{\pgfqpoint{1.243483in}{2.540113in}}%
\pgfpathlineto{\pgfqpoint{1.249926in}{2.537745in}}%
\pgfpathlineto{\pgfqpoint{1.250999in}{2.535537in}}%
\pgfpathlineto{\pgfqpoint{1.254221in}{2.534739in}}%
\pgfpathlineto{\pgfqpoint{1.255295in}{2.537000in}}%
\pgfpathlineto{\pgfqpoint{1.257442in}{2.529205in}}%
\pgfpathlineto{\pgfqpoint{1.258516in}{2.528806in}}%
\pgfpathlineto{\pgfqpoint{1.261738in}{2.531227in}}%
\pgfpathlineto{\pgfqpoint{1.264959in}{2.526305in}}%
\pgfpathlineto{\pgfqpoint{1.266033in}{2.528141in}}%
\pgfpathlineto{\pgfqpoint{1.269255in}{2.526146in}}%
\pgfpathlineto{\pgfqpoint{1.270329in}{2.523033in}}%
\pgfpathlineto{\pgfqpoint{1.271402in}{2.523858in}}%
\pgfpathlineto{\pgfqpoint{1.273550in}{2.523592in}}%
\pgfpathlineto{\pgfqpoint{1.277845in}{2.523618in}}%
\pgfpathlineto{\pgfqpoint{1.279993in}{2.525028in}}%
\pgfpathlineto{\pgfqpoint{1.284288in}{2.526199in}}%
\pgfpathlineto{\pgfqpoint{1.286436in}{2.530003in}}%
\pgfpathlineto{\pgfqpoint{1.288584in}{2.528487in}}%
\pgfpathlineto{\pgfqpoint{1.292879in}{2.529604in}}%
\pgfpathlineto{\pgfqpoint{1.293953in}{2.531546in}}%
\pgfpathlineto{\pgfqpoint{1.295027in}{2.531919in}}%
\pgfpathlineto{\pgfqpoint{1.296101in}{2.531014in}}%
\pgfpathlineto{\pgfqpoint{1.299322in}{2.529684in}}%
\pgfpathlineto{\pgfqpoint{1.301470in}{2.525108in}}%
\pgfpathlineto{\pgfqpoint{1.302544in}{2.525082in}}%
\pgfpathlineto{\pgfqpoint{1.303617in}{2.524310in}}%
\pgfpathlineto{\pgfqpoint{1.306839in}{2.524230in}}%
\pgfpathlineto{\pgfqpoint{1.307913in}{2.525986in}}%
\pgfpathlineto{\pgfqpoint{1.308987in}{2.525587in}}%
\pgfpathlineto{\pgfqpoint{1.310061in}{2.523964in}}%
\pgfpathlineto{\pgfqpoint{1.311134in}{2.525640in}}%
\pgfpathlineto{\pgfqpoint{1.314356in}{2.523991in}}%
\pgfpathlineto{\pgfqpoint{1.315430in}{2.521756in}}%
\pgfpathlineto{\pgfqpoint{1.316504in}{2.522554in}}%
\pgfpathlineto{\pgfqpoint{1.318651in}{2.530509in}}%
\pgfpathlineto{\pgfqpoint{1.322947in}{2.531599in}}%
\pgfpathlineto{\pgfqpoint{1.325094in}{2.536228in}}%
\pgfpathlineto{\pgfqpoint{1.326168in}{2.535457in}}%
\pgfpathlineto{\pgfqpoint{1.329390in}{2.534473in}}%
\pgfpathlineto{\pgfqpoint{1.330463in}{2.537106in}}%
\pgfpathlineto{\pgfqpoint{1.331537in}{2.536521in}}%
\pgfpathlineto{\pgfqpoint{1.332611in}{2.537080in}}%
\pgfpathlineto{\pgfqpoint{1.333685in}{2.536495in}}%
\pgfpathlineto{\pgfqpoint{1.336906in}{2.537186in}}%
\pgfpathlineto{\pgfqpoint{1.337980in}{2.538889in}}%
\pgfpathlineto{\pgfqpoint{1.340128in}{2.537692in}}%
\pgfpathlineto{\pgfqpoint{1.341202in}{2.539421in}}%
\pgfpathlineto{\pgfqpoint{1.347645in}{2.537851in}}%
\pgfpathlineto{\pgfqpoint{1.348719in}{2.540964in}}%
\pgfpathlineto{\pgfqpoint{1.351940in}{2.541496in}}%
\pgfpathlineto{\pgfqpoint{1.353014in}{2.539847in}}%
\pgfpathlineto{\pgfqpoint{1.354088in}{2.539341in}}%
\pgfpathlineto{\pgfqpoint{1.356236in}{2.541735in}}%
\pgfpathlineto{\pgfqpoint{1.359457in}{2.541523in}}%
\pgfpathlineto{\pgfqpoint{1.361605in}{2.543465in}}%
\pgfpathlineto{\pgfqpoint{1.362679in}{2.543571in}}%
\pgfpathlineto{\pgfqpoint{1.363752in}{2.545194in}}%
\pgfpathlineto{\pgfqpoint{1.366974in}{2.546072in}}%
\pgfpathlineto{\pgfqpoint{1.368048in}{2.544768in}}%
\pgfpathlineto{\pgfqpoint{1.371269in}{2.543837in}}%
\pgfpathlineto{\pgfqpoint{1.375565in}{2.542267in}}%
\pgfpathlineto{\pgfqpoint{1.376639in}{2.541496in}}%
\pgfpathlineto{\pgfqpoint{1.377712in}{2.539980in}}%
\pgfpathlineto{\pgfqpoint{1.378786in}{2.543332in}}%
\pgfpathlineto{\pgfqpoint{1.382008in}{2.543332in}}%
\pgfpathlineto{\pgfqpoint{1.383082in}{2.542667in}}%
\pgfpathlineto{\pgfqpoint{1.384155in}{2.540299in}}%
\pgfpathlineto{\pgfqpoint{1.385229in}{2.535829in}}%
\pgfpathlineto{\pgfqpoint{1.386303in}{2.536255in}}%
\pgfpathlineto{\pgfqpoint{1.389525in}{2.536308in}}%
\pgfpathlineto{\pgfqpoint{1.390598in}{2.534685in}}%
\pgfpathlineto{\pgfqpoint{1.391672in}{2.539208in}}%
\pgfpathlineto{\pgfqpoint{1.392746in}{2.537718in}}%
\pgfpathlineto{\pgfqpoint{1.393820in}{2.537958in}}%
\pgfpathlineto{\pgfqpoint{1.398115in}{2.537904in}}%
\pgfpathlineto{\pgfqpoint{1.400263in}{2.538995in}}%
\pgfpathlineto{\pgfqpoint{1.401337in}{2.538570in}}%
\pgfpathlineto{\pgfqpoint{1.404558in}{2.538437in}}%
\pgfpathlineto{\pgfqpoint{1.405632in}{2.536814in}}%
\pgfpathlineto{\pgfqpoint{1.407780in}{2.535537in}}%
\pgfpathlineto{\pgfqpoint{1.408854in}{2.537399in}}%
\pgfpathlineto{\pgfqpoint{1.412075in}{2.538516in}}%
\pgfpathlineto{\pgfqpoint{1.413149in}{2.542986in}}%
\pgfpathlineto{\pgfqpoint{1.414223in}{2.542693in}}%
\pgfpathlineto{\pgfqpoint{1.415297in}{2.543997in}}%
\pgfpathlineto{\pgfqpoint{1.416371in}{2.544023in}}%
\pgfpathlineto{\pgfqpoint{1.419592in}{2.543491in}}%
\pgfpathlineto{\pgfqpoint{1.421740in}{2.544156in}}%
\pgfpathlineto{\pgfqpoint{1.422814in}{2.543810in}}%
\pgfpathlineto{\pgfqpoint{1.423887in}{2.544768in}}%
\pgfpathlineto{\pgfqpoint{1.428183in}{2.542400in}}%
\pgfpathlineto{\pgfqpoint{1.429257in}{2.543066in}}%
\pgfpathlineto{\pgfqpoint{1.431404in}{2.534579in}}%
\pgfpathlineto{\pgfqpoint{1.434626in}{2.533036in}}%
\pgfpathlineto{\pgfqpoint{1.435700in}{2.533409in}}%
\pgfpathlineto{\pgfqpoint{1.436773in}{2.530668in}}%
\pgfpathlineto{\pgfqpoint{1.437847in}{2.532025in}}%
\pgfpathlineto{\pgfqpoint{1.438921in}{2.529684in}}%
\pgfpathlineto{\pgfqpoint{1.442143in}{2.525800in}}%
\pgfpathlineto{\pgfqpoint{1.443217in}{2.525427in}}%
\pgfpathlineto{\pgfqpoint{1.444290in}{2.526944in}}%
\pgfpathlineto{\pgfqpoint{1.446438in}{2.533275in}}%
\pgfpathlineto{\pgfqpoint{1.449660in}{2.535617in}}%
\pgfpathlineto{\pgfqpoint{1.450733in}{2.539980in}}%
\pgfpathlineto{\pgfqpoint{1.451807in}{2.538729in}}%
\pgfpathlineto{\pgfqpoint{1.453955in}{2.539501in}}%
\pgfpathlineto{\pgfqpoint{1.458250in}{2.538144in}}%
\pgfpathlineto{\pgfqpoint{1.459324in}{2.536920in}}%
\pgfpathlineto{\pgfqpoint{1.460398in}{2.538623in}}%
\pgfpathlineto{\pgfqpoint{1.464693in}{2.537213in}}%
\pgfpathlineto{\pgfqpoint{1.466841in}{2.537213in}}%
\pgfpathlineto{\pgfqpoint{1.467915in}{2.537798in}}%
\pgfpathlineto{\pgfqpoint{1.468989in}{2.539554in}}%
\pgfpathlineto{\pgfqpoint{1.472210in}{2.538250in}}%
\pgfpathlineto{\pgfqpoint{1.473284in}{2.542347in}}%
\pgfpathlineto{\pgfqpoint{1.474358in}{2.540618in}}%
\pgfpathlineto{\pgfqpoint{1.476506in}{2.542321in}}%
\pgfpathlineto{\pgfqpoint{1.481875in}{2.542959in}}%
\pgfpathlineto{\pgfqpoint{1.482949in}{2.541576in}}%
\pgfpathlineto{\pgfqpoint{1.484022in}{2.541150in}}%
\pgfpathlineto{\pgfqpoint{1.487244in}{2.543731in}}%
\pgfpathlineto{\pgfqpoint{1.488318in}{2.543731in}}%
\pgfpathlineto{\pgfqpoint{1.489392in}{2.542933in}}%
\pgfpathlineto{\pgfqpoint{1.490465in}{2.544156in}}%
\pgfpathlineto{\pgfqpoint{1.491539in}{2.548333in}}%
\pgfpathlineto{\pgfqpoint{1.494761in}{2.546657in}}%
\pgfpathlineto{\pgfqpoint{1.495835in}{2.551685in}}%
\pgfpathlineto{\pgfqpoint{1.496908in}{2.550914in}}%
\pgfpathlineto{\pgfqpoint{1.502278in}{2.553627in}}%
\pgfpathlineto{\pgfqpoint{1.503351in}{2.552962in}}%
\pgfpathlineto{\pgfqpoint{1.505499in}{2.553707in}}%
\pgfpathlineto{\pgfqpoint{1.506573in}{2.554079in}}%
\pgfpathlineto{\pgfqpoint{1.509794in}{2.552856in}}%
\pgfpathlineto{\pgfqpoint{1.510868in}{2.553255in}}%
\pgfpathlineto{\pgfqpoint{1.511942in}{2.555303in}}%
\pgfpathlineto{\pgfqpoint{1.513016in}{2.549743in}}%
\pgfpathlineto{\pgfqpoint{1.514090in}{2.550488in}}%
\pgfpathlineto{\pgfqpoint{1.517311in}{2.551126in}}%
\pgfpathlineto{\pgfqpoint{1.518385in}{2.555862in}}%
\pgfpathlineto{\pgfqpoint{1.519459in}{2.554824in}}%
\pgfpathlineto{\pgfqpoint{1.526976in}{2.558230in}}%
\pgfpathlineto{\pgfqpoint{1.529124in}{2.557219in}}%
\pgfpathlineto{\pgfqpoint{1.532345in}{2.560810in}}%
\pgfpathlineto{\pgfqpoint{1.533419in}{2.560092in}}%
\pgfpathlineto{\pgfqpoint{1.534493in}{2.560677in}}%
\pgfpathlineto{\pgfqpoint{1.535567in}{2.558948in}}%
\pgfpathlineto{\pgfqpoint{1.536640in}{2.556128in}}%
\pgfpathlineto{\pgfqpoint{1.539862in}{2.557724in}}%
\pgfpathlineto{\pgfqpoint{1.540936in}{2.556580in}}%
\pgfpathlineto{\pgfqpoint{1.542010in}{2.559826in}}%
\pgfpathlineto{\pgfqpoint{1.543083in}{2.558868in}}%
\pgfpathlineto{\pgfqpoint{1.544157in}{2.559826in}}%
\pgfpathlineto{\pgfqpoint{1.547379in}{2.558921in}}%
\pgfpathlineto{\pgfqpoint{1.548453in}{2.560065in}}%
\pgfpathlineto{\pgfqpoint{1.551674in}{2.559054in}}%
\pgfpathlineto{\pgfqpoint{1.554896in}{2.559187in}}%
\pgfpathlineto{\pgfqpoint{1.555970in}{2.558283in}}%
\pgfpathlineto{\pgfqpoint{1.558117in}{2.561555in}}%
\pgfpathlineto{\pgfqpoint{1.559191in}{2.561608in}}%
\pgfpathlineto{\pgfqpoint{1.563486in}{2.561209in}}%
\pgfpathlineto{\pgfqpoint{1.564560in}{2.560012in}}%
\pgfpathlineto{\pgfqpoint{1.566708in}{2.562726in}}%
\pgfpathlineto{\pgfqpoint{1.572077in}{2.565545in}}%
\pgfpathlineto{\pgfqpoint{1.573151in}{2.566796in}}%
\pgfpathlineto{\pgfqpoint{1.577446in}{2.566796in}}%
\pgfpathlineto{\pgfqpoint{1.578520in}{2.568844in}}%
\pgfpathlineto{\pgfqpoint{1.580668in}{2.565200in}}%
\pgfpathlineto{\pgfqpoint{1.584963in}{2.565013in}}%
\pgfpathlineto{\pgfqpoint{1.586037in}{2.563816in}}%
\pgfpathlineto{\pgfqpoint{1.588185in}{2.568179in}}%
\pgfpathlineto{\pgfqpoint{1.589259in}{2.571558in}}%
\pgfpathlineto{\pgfqpoint{1.593554in}{2.569988in}}%
\pgfpathlineto{\pgfqpoint{1.594628in}{2.572702in}}%
\pgfpathlineto{\pgfqpoint{1.595702in}{2.572436in}}%
\pgfpathlineto{\pgfqpoint{1.596775in}{2.570919in}}%
\pgfpathlineto{\pgfqpoint{1.599997in}{2.570041in}}%
\pgfpathlineto{\pgfqpoint{1.601071in}{2.572941in}}%
\pgfpathlineto{\pgfqpoint{1.602145in}{2.572915in}}%
\pgfpathlineto{\pgfqpoint{1.603218in}{2.571904in}}%
\pgfpathlineto{\pgfqpoint{1.607514in}{2.574351in}}%
\pgfpathlineto{\pgfqpoint{1.608588in}{2.572595in}}%
\pgfpathlineto{\pgfqpoint{1.609661in}{2.573340in}}%
\pgfpathlineto{\pgfqpoint{1.610735in}{2.572782in}}%
\pgfpathlineto{\pgfqpoint{1.611809in}{2.571159in}}%
\pgfpathlineto{\pgfqpoint{1.615031in}{2.571824in}}%
\pgfpathlineto{\pgfqpoint{1.617178in}{2.564481in}}%
\pgfpathlineto{\pgfqpoint{1.618252in}{2.560172in}}%
\pgfpathlineto{\pgfqpoint{1.619326in}{2.563497in}}%
\pgfpathlineto{\pgfqpoint{1.622548in}{2.562273in}}%
\pgfpathlineto{\pgfqpoint{1.623621in}{2.565067in}}%
\pgfpathlineto{\pgfqpoint{1.624695in}{2.564402in}}%
\pgfpathlineto{\pgfqpoint{1.630064in}{2.564215in}}%
\pgfpathlineto{\pgfqpoint{1.631138in}{2.563869in}}%
\pgfpathlineto{\pgfqpoint{1.632212in}{2.564668in}}%
\pgfpathlineto{\pgfqpoint{1.633286in}{2.559533in}}%
\pgfpathlineto{\pgfqpoint{1.634360in}{2.559081in}}%
\pgfpathlineto{\pgfqpoint{1.637581in}{2.559693in}}%
\pgfpathlineto{\pgfqpoint{1.638655in}{2.558895in}}%
\pgfpathlineto{\pgfqpoint{1.639729in}{2.560970in}}%
\pgfpathlineto{\pgfqpoint{1.640803in}{2.559161in}}%
\pgfpathlineto{\pgfqpoint{1.641877in}{2.561821in}}%
\pgfpathlineto{\pgfqpoint{1.645098in}{2.562007in}}%
\pgfpathlineto{\pgfqpoint{1.646172in}{2.560783in}}%
\pgfpathlineto{\pgfqpoint{1.647246in}{2.563364in}}%
\pgfpathlineto{\pgfqpoint{1.648320in}{2.564002in}}%
\pgfpathlineto{\pgfqpoint{1.649394in}{2.562034in}}%
\pgfpathlineto{\pgfqpoint{1.653689in}{2.566184in}}%
\pgfpathlineto{\pgfqpoint{1.654763in}{2.566769in}}%
\pgfpathlineto{\pgfqpoint{1.655837in}{2.569057in}}%
\pgfpathlineto{\pgfqpoint{1.656910in}{2.568126in}}%
\pgfpathlineto{\pgfqpoint{1.662280in}{2.568419in}}%
\pgfpathlineto{\pgfqpoint{1.663353in}{2.567780in}}%
\pgfpathlineto{\pgfqpoint{1.664427in}{2.569616in}}%
\pgfpathlineto{\pgfqpoint{1.668723in}{2.568738in}}%
\pgfpathlineto{\pgfqpoint{1.671944in}{2.571212in}}%
\pgfpathlineto{\pgfqpoint{1.676239in}{2.569775in}}%
\pgfpathlineto{\pgfqpoint{1.677313in}{2.572569in}}%
\pgfpathlineto{\pgfqpoint{1.678387in}{2.571531in}}%
\pgfpathlineto{\pgfqpoint{1.682682in}{2.571930in}}%
\pgfpathlineto{\pgfqpoint{1.683756in}{2.574644in}}%
\pgfpathlineto{\pgfqpoint{1.684830in}{2.575362in}}%
\pgfpathlineto{\pgfqpoint{1.686978in}{2.579566in}}%
\pgfpathlineto{\pgfqpoint{1.690199in}{2.579299in}}%
\pgfpathlineto{\pgfqpoint{1.691273in}{2.578342in}}%
\pgfpathlineto{\pgfqpoint{1.692347in}{2.581082in}}%
\pgfpathlineto{\pgfqpoint{1.693421in}{2.577490in}}%
\pgfpathlineto{\pgfqpoint{1.694495in}{2.577490in}}%
\pgfpathlineto{\pgfqpoint{1.698790in}{2.576293in}}%
\pgfpathlineto{\pgfqpoint{1.699864in}{2.570946in}}%
\pgfpathlineto{\pgfqpoint{1.700938in}{2.569882in}}%
\pgfpathlineto{\pgfqpoint{1.702012in}{2.572888in}}%
\pgfpathlineto{\pgfqpoint{1.705233in}{2.572250in}}%
\pgfpathlineto{\pgfqpoint{1.706307in}{2.566477in}}%
\pgfpathlineto{\pgfqpoint{1.707381in}{2.572383in}}%
\pgfpathlineto{\pgfqpoint{1.708455in}{2.565758in}}%
\pgfpathlineto{\pgfqpoint{1.712750in}{2.558841in}}%
\pgfpathlineto{\pgfqpoint{1.713824in}{2.553893in}}%
\pgfpathlineto{\pgfqpoint{1.714898in}{2.556713in}}%
\pgfpathlineto{\pgfqpoint{1.715971in}{2.553361in}}%
\pgfpathlineto{\pgfqpoint{1.717045in}{2.557857in}}%
\pgfpathlineto{\pgfqpoint{1.720267in}{2.559028in}}%
\pgfpathlineto{\pgfqpoint{1.723488in}{2.567035in}}%
\pgfpathlineto{\pgfqpoint{1.724562in}{2.568206in}}%
\pgfpathlineto{\pgfqpoint{1.727784in}{2.570414in}}%
\pgfpathlineto{\pgfqpoint{1.729931in}{2.573899in}}%
\pgfpathlineto{\pgfqpoint{1.732079in}{2.579087in}}%
\pgfpathlineto{\pgfqpoint{1.735301in}{2.578342in}}%
\pgfpathlineto{\pgfqpoint{1.736374in}{2.581029in}}%
\pgfpathlineto{\pgfqpoint{1.738522in}{2.581960in}}%
\pgfpathlineto{\pgfqpoint{1.739596in}{2.580071in}}%
\pgfpathlineto{\pgfqpoint{1.743891in}{2.581720in}}%
\pgfpathlineto{\pgfqpoint{1.744965in}{2.581348in}}%
\pgfpathlineto{\pgfqpoint{1.746039in}{2.582093in}}%
\pgfpathlineto{\pgfqpoint{1.747113in}{2.579965in}}%
\pgfpathlineto{\pgfqpoint{1.750334in}{2.580284in}}%
\pgfpathlineto{\pgfqpoint{1.751408in}{2.581534in}}%
\pgfpathlineto{\pgfqpoint{1.752482in}{2.581375in}}%
\pgfpathlineto{\pgfqpoint{1.753556in}{2.580018in}}%
\pgfpathlineto{\pgfqpoint{1.754630in}{2.580896in}}%
\pgfpathlineto{\pgfqpoint{1.758925in}{2.578182in}}%
\pgfpathlineto{\pgfqpoint{1.762147in}{2.581827in}}%
\pgfpathlineto{\pgfqpoint{1.765368in}{2.581295in}}%
\pgfpathlineto{\pgfqpoint{1.766442in}{2.582439in}}%
\pgfpathlineto{\pgfqpoint{1.767516in}{2.580576in}}%
\pgfpathlineto{\pgfqpoint{1.768590in}{2.580204in}}%
\pgfpathlineto{\pgfqpoint{1.769663in}{2.582439in}}%
\pgfpathlineto{\pgfqpoint{1.772885in}{2.582465in}}%
\pgfpathlineto{\pgfqpoint{1.773959in}{2.581348in}}%
\pgfpathlineto{\pgfqpoint{1.775033in}{2.577091in}}%
\pgfpathlineto{\pgfqpoint{1.776106in}{2.578235in}}%
\pgfpathlineto{\pgfqpoint{1.777180in}{2.572835in}}%
\pgfpathlineto{\pgfqpoint{1.780402in}{2.571744in}}%
\pgfpathlineto{\pgfqpoint{1.781476in}{2.568898in}}%
\pgfpathlineto{\pgfqpoint{1.782549in}{2.571984in}}%
\pgfpathlineto{\pgfqpoint{1.783623in}{2.578448in}}%
\pgfpathlineto{\pgfqpoint{1.784697in}{2.575469in}}%
\pgfpathlineto{\pgfqpoint{1.787919in}{2.578262in}}%
\pgfpathlineto{\pgfqpoint{1.788993in}{2.572489in}}%
\pgfpathlineto{\pgfqpoint{1.792214in}{2.574325in}}%
\pgfpathlineto{\pgfqpoint{1.796509in}{2.575016in}}%
\pgfpathlineto{\pgfqpoint{1.797583in}{2.573154in}}%
\pgfpathlineto{\pgfqpoint{1.799731in}{2.573048in}}%
\pgfpathlineto{\pgfqpoint{1.802952in}{2.571345in}}%
\pgfpathlineto{\pgfqpoint{1.804026in}{2.570121in}}%
\pgfpathlineto{\pgfqpoint{1.805100in}{2.575495in}}%
\pgfpathlineto{\pgfqpoint{1.806174in}{2.577437in}}%
\pgfpathlineto{\pgfqpoint{1.807248in}{2.574032in}}%
\pgfpathlineto{\pgfqpoint{1.810469in}{2.573181in}}%
\pgfpathlineto{\pgfqpoint{1.811543in}{2.573606in}}%
\pgfpathlineto{\pgfqpoint{1.812617in}{2.571824in}}%
\pgfpathlineto{\pgfqpoint{1.813691in}{2.568286in}}%
\pgfpathlineto{\pgfqpoint{1.814765in}{2.571930in}}%
\pgfpathlineto{\pgfqpoint{1.819060in}{2.565439in}}%
\pgfpathlineto{\pgfqpoint{1.820134in}{2.566876in}}%
\pgfpathlineto{\pgfqpoint{1.821208in}{2.571265in}}%
\pgfpathlineto{\pgfqpoint{1.822282in}{2.567594in}}%
\pgfpathlineto{\pgfqpoint{1.826577in}{2.567328in}}%
\pgfpathlineto{\pgfqpoint{1.827651in}{2.565891in}}%
\pgfpathlineto{\pgfqpoint{1.828725in}{2.568020in}}%
\pgfpathlineto{\pgfqpoint{1.829798in}{2.562752in}}%
\pgfpathlineto{\pgfqpoint{1.833020in}{2.564375in}}%
\pgfpathlineto{\pgfqpoint{1.834094in}{2.568206in}}%
\pgfpathlineto{\pgfqpoint{1.835168in}{2.565625in}}%
\pgfpathlineto{\pgfqpoint{1.836241in}{2.568206in}}%
\pgfpathlineto{\pgfqpoint{1.837315in}{2.565013in}}%
\pgfpathlineto{\pgfqpoint{1.840537in}{2.561901in}}%
\pgfpathlineto{\pgfqpoint{1.841611in}{2.563231in}}%
\pgfpathlineto{\pgfqpoint{1.842684in}{2.563311in}}%
\pgfpathlineto{\pgfqpoint{1.843758in}{2.558735in}}%
\pgfpathlineto{\pgfqpoint{1.844832in}{2.561528in}}%
\pgfpathlineto{\pgfqpoint{1.849127in}{2.563444in}}%
\pgfpathlineto{\pgfqpoint{1.850201in}{2.562326in}}%
\pgfpathlineto{\pgfqpoint{1.851275in}{2.564056in}}%
\pgfpathlineto{\pgfqpoint{1.852349in}{2.564668in}}%
\pgfpathlineto{\pgfqpoint{1.855570in}{2.564481in}}%
\pgfpathlineto{\pgfqpoint{1.857718in}{2.566929in}}%
\pgfpathlineto{\pgfqpoint{1.858792in}{2.570707in}}%
\pgfpathlineto{\pgfqpoint{1.859866in}{2.570015in}}%
\pgfpathlineto{\pgfqpoint{1.863087in}{2.571691in}}%
\pgfpathlineto{\pgfqpoint{1.865235in}{2.567966in}}%
\pgfpathlineto{\pgfqpoint{1.866309in}{2.570041in}}%
\pgfpathlineto{\pgfqpoint{1.867383in}{2.564322in}}%
\pgfpathlineto{\pgfqpoint{1.870604in}{2.565625in}}%
\pgfpathlineto{\pgfqpoint{1.872752in}{2.560092in}}%
\pgfpathlineto{\pgfqpoint{1.873826in}{2.563657in}}%
\pgfpathlineto{\pgfqpoint{1.874900in}{2.562193in}}%
\pgfpathlineto{\pgfqpoint{1.878121in}{2.566583in}}%
\pgfpathlineto{\pgfqpoint{1.879195in}{2.563790in}}%
\pgfpathlineto{\pgfqpoint{1.880269in}{2.567514in}}%
\pgfpathlineto{\pgfqpoint{1.881343in}{2.568073in}}%
\pgfpathlineto{\pgfqpoint{1.882416in}{2.569749in}}%
\pgfpathlineto{\pgfqpoint{1.885638in}{2.571132in}}%
\pgfpathlineto{\pgfqpoint{1.886712in}{2.568711in}}%
\pgfpathlineto{\pgfqpoint{1.887786in}{2.564880in}}%
\pgfpathlineto{\pgfqpoint{1.888859in}{2.564402in}}%
\pgfpathlineto{\pgfqpoint{1.889933in}{2.564880in}}%
\pgfpathlineto{\pgfqpoint{1.893155in}{2.567727in}}%
\pgfpathlineto{\pgfqpoint{1.895303in}{2.562060in}}%
\pgfpathlineto{\pgfqpoint{1.896376in}{2.563204in}}%
\pgfpathlineto{\pgfqpoint{1.900672in}{2.562060in}}%
\pgfpathlineto{\pgfqpoint{1.901746in}{2.564295in}}%
\pgfpathlineto{\pgfqpoint{1.902819in}{2.564428in}}%
\pgfpathlineto{\pgfqpoint{1.904967in}{2.568951in}}%
\pgfpathlineto{\pgfqpoint{1.908189in}{2.565359in}}%
\pgfpathlineto{\pgfqpoint{1.910336in}{2.565492in}}%
\pgfpathlineto{\pgfqpoint{1.911410in}{2.563577in}}%
\pgfpathlineto{\pgfqpoint{1.912484in}{2.563071in}}%
\pgfpathlineto{\pgfqpoint{1.917853in}{2.565093in}}%
\pgfpathlineto{\pgfqpoint{1.918927in}{2.565146in}}%
\pgfpathlineto{\pgfqpoint{1.920001in}{2.566610in}}%
\pgfpathlineto{\pgfqpoint{1.923222in}{2.565439in}}%
\pgfpathlineto{\pgfqpoint{1.924296in}{2.565812in}}%
\pgfpathlineto{\pgfqpoint{1.925370in}{2.564987in}}%
\pgfpathlineto{\pgfqpoint{1.926444in}{2.562167in}}%
\pgfpathlineto{\pgfqpoint{1.927518in}{2.564375in}}%
\pgfpathlineto{\pgfqpoint{1.930739in}{2.564880in}}%
\pgfpathlineto{\pgfqpoint{1.931813in}{2.562885in}}%
\pgfpathlineto{\pgfqpoint{1.932887in}{2.562087in}}%
\pgfpathlineto{\pgfqpoint{1.933961in}{2.563258in}}%
\pgfpathlineto{\pgfqpoint{1.935035in}{2.567541in}}%
\pgfpathlineto{\pgfqpoint{1.938256in}{2.566503in}}%
\pgfpathlineto{\pgfqpoint{1.939330in}{2.565173in}}%
\pgfpathlineto{\pgfqpoint{1.940404in}{2.565359in}}%
\pgfpathlineto{\pgfqpoint{1.941478in}{2.568392in}}%
\pgfpathlineto{\pgfqpoint{1.946847in}{2.573447in}}%
\pgfpathlineto{\pgfqpoint{1.948994in}{2.571558in}}%
\pgfpathlineto{\pgfqpoint{1.950068in}{2.569004in}}%
\pgfpathlineto{\pgfqpoint{1.954364in}{2.567754in}}%
\pgfpathlineto{\pgfqpoint{1.955437in}{2.568498in}}%
\pgfpathlineto{\pgfqpoint{1.956511in}{2.568525in}}%
\pgfpathlineto{\pgfqpoint{1.957585in}{2.566131in}}%
\pgfpathlineto{\pgfqpoint{1.962954in}{2.565998in}}%
\pgfpathlineto{\pgfqpoint{1.965102in}{2.562433in}}%
\pgfpathlineto{\pgfqpoint{1.968324in}{2.560916in}}%
\pgfpathlineto{\pgfqpoint{1.969397in}{2.561528in}}%
\pgfpathlineto{\pgfqpoint{1.971545in}{2.563976in}}%
\pgfpathlineto{\pgfqpoint{1.972619in}{2.561848in}}%
\pgfpathlineto{\pgfqpoint{1.975840in}{2.559799in}}%
\pgfpathlineto{\pgfqpoint{1.979062in}{2.566636in}}%
\pgfpathlineto{\pgfqpoint{1.980136in}{2.565466in}}%
\pgfpathlineto{\pgfqpoint{1.983357in}{2.565998in}}%
\pgfpathlineto{\pgfqpoint{1.986579in}{2.563683in}}%
\pgfpathlineto{\pgfqpoint{1.987653in}{2.564934in}}%
\pgfpathlineto{\pgfqpoint{1.990874in}{2.560251in}}%
\pgfpathlineto{\pgfqpoint{1.991948in}{2.559719in}}%
\pgfpathlineto{\pgfqpoint{1.993022in}{2.562140in}}%
\pgfpathlineto{\pgfqpoint{1.998391in}{2.561502in}}%
\pgfpathlineto{\pgfqpoint{1.999465in}{2.563204in}}%
\pgfpathlineto{\pgfqpoint{2.000539in}{2.560464in}}%
\pgfpathlineto{\pgfqpoint{2.001613in}{2.561981in}}%
\pgfpathlineto{\pgfqpoint{2.002686in}{2.564668in}}%
\pgfpathlineto{\pgfqpoint{2.005908in}{2.566423in}}%
\pgfpathlineto{\pgfqpoint{2.006982in}{2.565253in}}%
\pgfpathlineto{\pgfqpoint{2.009129in}{2.568445in}}%
\pgfpathlineto{\pgfqpoint{2.010203in}{2.565971in}}%
\pgfpathlineto{\pgfqpoint{2.014499in}{2.566610in}}%
\pgfpathlineto{\pgfqpoint{2.016646in}{2.566264in}}%
\pgfpathlineto{\pgfqpoint{2.017720in}{2.563763in}}%
\pgfpathlineto{\pgfqpoint{2.020942in}{2.561688in}}%
\pgfpathlineto{\pgfqpoint{2.023089in}{2.565146in}}%
\pgfpathlineto{\pgfqpoint{2.024163in}{2.565412in}}%
\pgfpathlineto{\pgfqpoint{2.025237in}{2.566290in}}%
\pgfpathlineto{\pgfqpoint{2.029532in}{2.565306in}}%
\pgfpathlineto{\pgfqpoint{2.030606in}{2.567035in}}%
\pgfpathlineto{\pgfqpoint{2.031680in}{2.563577in}}%
\pgfpathlineto{\pgfqpoint{2.032754in}{2.563045in}}%
\pgfpathlineto{\pgfqpoint{2.035975in}{2.565279in}}%
\pgfpathlineto{\pgfqpoint{2.037049in}{2.563417in}}%
\pgfpathlineto{\pgfqpoint{2.039197in}{2.562167in}}%
\pgfpathlineto{\pgfqpoint{2.043492in}{2.565492in}}%
\pgfpathlineto{\pgfqpoint{2.044566in}{2.564295in}}%
\pgfpathlineto{\pgfqpoint{2.045640in}{2.564136in}}%
\pgfpathlineto{\pgfqpoint{2.046714in}{2.562912in}}%
\pgfpathlineto{\pgfqpoint{2.047788in}{2.556926in}}%
\pgfpathlineto{\pgfqpoint{2.051009in}{2.550355in}}%
\pgfpathlineto{\pgfqpoint{2.052083in}{2.545327in}}%
\pgfpathlineto{\pgfqpoint{2.053157in}{2.555862in}}%
\pgfpathlineto{\pgfqpoint{2.054231in}{2.558522in}}%
\pgfpathlineto{\pgfqpoint{2.055304in}{2.555995in}}%
\pgfpathlineto{\pgfqpoint{2.058526in}{2.553148in}}%
\pgfpathlineto{\pgfqpoint{2.059600in}{2.548599in}}%
\pgfpathlineto{\pgfqpoint{2.060674in}{2.551632in}}%
\pgfpathlineto{\pgfqpoint{2.061747in}{2.549929in}}%
\pgfpathlineto{\pgfqpoint{2.062821in}{2.546710in}}%
\pgfpathlineto{\pgfqpoint{2.067117in}{2.553042in}}%
\pgfpathlineto{\pgfqpoint{2.068191in}{2.548918in}}%
\pgfpathlineto{\pgfqpoint{2.070338in}{2.550621in}}%
\pgfpathlineto{\pgfqpoint{2.073560in}{2.551552in}}%
\pgfpathlineto{\pgfqpoint{2.074634in}{2.554159in}}%
\pgfpathlineto{\pgfqpoint{2.076781in}{2.555197in}}%
\pgfpathlineto{\pgfqpoint{2.077855in}{2.551712in}}%
\pgfpathlineto{\pgfqpoint{2.083224in}{2.550754in}}%
\pgfpathlineto{\pgfqpoint{2.084298in}{2.549530in}}%
\pgfpathlineto{\pgfqpoint{2.085372in}{2.545965in}}%
\pgfpathlineto{\pgfqpoint{2.088593in}{2.546870in}}%
\pgfpathlineto{\pgfqpoint{2.089667in}{2.550887in}}%
\pgfpathlineto{\pgfqpoint{2.090741in}{2.551632in}}%
\pgfpathlineto{\pgfqpoint{2.091815in}{2.551180in}}%
\pgfpathlineto{\pgfqpoint{2.092889in}{2.553015in}}%
\pgfpathlineto{\pgfqpoint{2.096110in}{2.555011in}}%
\pgfpathlineto{\pgfqpoint{2.097184in}{2.551791in}}%
\pgfpathlineto{\pgfqpoint{2.098258in}{2.555543in}}%
\pgfpathlineto{\pgfqpoint{2.100406in}{2.556474in}}%
\pgfpathlineto{\pgfqpoint{2.103627in}{2.557963in}}%
\pgfpathlineto{\pgfqpoint{2.104701in}{2.556687in}}%
\pgfpathlineto{\pgfqpoint{2.105775in}{2.554452in}}%
\pgfpathlineto{\pgfqpoint{2.106849in}{2.560757in}}%
\pgfpathlineto{\pgfqpoint{2.107923in}{2.563391in}}%
\pgfpathlineto{\pgfqpoint{2.111144in}{2.562672in}}%
\pgfpathlineto{\pgfqpoint{2.112218in}{2.561821in}}%
\pgfpathlineto{\pgfqpoint{2.113292in}{2.561927in}}%
\pgfpathlineto{\pgfqpoint{2.114366in}{2.566477in}}%
\pgfpathlineto{\pgfqpoint{2.115439in}{2.568365in}}%
\pgfpathlineto{\pgfqpoint{2.118661in}{2.567461in}}%
\pgfpathlineto{\pgfqpoint{2.120809in}{2.568765in}}%
\pgfpathlineto{\pgfqpoint{2.121882in}{2.570919in}}%
\pgfpathlineto{\pgfqpoint{2.122956in}{2.570095in}}%
\pgfpathlineto{\pgfqpoint{2.126178in}{2.572835in}}%
\pgfpathlineto{\pgfqpoint{2.128325in}{2.572276in}}%
\pgfpathlineto{\pgfqpoint{2.129399in}{2.573207in}}%
\pgfpathlineto{\pgfqpoint{2.133695in}{2.569642in}}%
\pgfpathlineto{\pgfqpoint{2.135842in}{2.572090in}}%
\pgfpathlineto{\pgfqpoint{2.136916in}{2.568259in}}%
\pgfpathlineto{\pgfqpoint{2.137990in}{2.567328in}}%
\pgfpathlineto{\pgfqpoint{2.142285in}{2.571212in}}%
\pgfpathlineto{\pgfqpoint{2.143359in}{2.574032in}}%
\pgfpathlineto{\pgfqpoint{2.144433in}{2.573606in}}%
\pgfpathlineto{\pgfqpoint{2.145507in}{2.575389in}}%
\pgfpathlineto{\pgfqpoint{2.148728in}{2.576107in}}%
\pgfpathlineto{\pgfqpoint{2.149802in}{2.574325in}}%
\pgfpathlineto{\pgfqpoint{2.150876in}{2.574138in}}%
\pgfpathlineto{\pgfqpoint{2.153024in}{2.575123in}}%
\pgfpathlineto{\pgfqpoint{2.156245in}{2.572383in}}%
\pgfpathlineto{\pgfqpoint{2.157319in}{2.575096in}}%
\pgfpathlineto{\pgfqpoint{2.158393in}{2.574351in}}%
\pgfpathlineto{\pgfqpoint{2.159467in}{2.571292in}}%
\pgfpathlineto{\pgfqpoint{2.160541in}{2.576533in}}%
\pgfpathlineto{\pgfqpoint{2.163762in}{2.577437in}}%
\pgfpathlineto{\pgfqpoint{2.164836in}{2.575256in}}%
\pgfpathlineto{\pgfqpoint{2.165910in}{2.574591in}}%
\pgfpathlineto{\pgfqpoint{2.166984in}{2.575788in}}%
\pgfpathlineto{\pgfqpoint{2.168058in}{2.573447in}}%
\pgfpathlineto{\pgfqpoint{2.171279in}{2.574591in}}%
\pgfpathlineto{\pgfqpoint{2.173427in}{2.582093in}}%
\pgfpathlineto{\pgfqpoint{2.175574in}{2.574112in}}%
\pgfpathlineto{\pgfqpoint{2.178796in}{2.573207in}}%
\pgfpathlineto{\pgfqpoint{2.180944in}{2.577836in}}%
\pgfpathlineto{\pgfqpoint{2.182017in}{2.578395in}}%
\pgfpathlineto{\pgfqpoint{2.186313in}{2.577198in}}%
\pgfpathlineto{\pgfqpoint{2.187387in}{2.579140in}}%
\pgfpathlineto{\pgfqpoint{2.188460in}{2.578555in}}%
\pgfpathlineto{\pgfqpoint{2.189534in}{2.575974in}}%
\pgfpathlineto{\pgfqpoint{2.193830in}{2.570547in}}%
\pgfpathlineto{\pgfqpoint{2.194903in}{2.571558in}}%
\pgfpathlineto{\pgfqpoint{2.195977in}{2.570334in}}%
\pgfpathlineto{\pgfqpoint{2.198125in}{2.564934in}}%
\pgfpathlineto{\pgfqpoint{2.201347in}{2.563497in}}%
\pgfpathlineto{\pgfqpoint{2.202420in}{2.565120in}}%
\pgfpathlineto{\pgfqpoint{2.203494in}{2.562167in}}%
\pgfpathlineto{\pgfqpoint{2.204568in}{2.566689in}}%
\pgfpathlineto{\pgfqpoint{2.205642in}{2.562114in}}%
\pgfpathlineto{\pgfqpoint{2.209937in}{2.563337in}}%
\pgfpathlineto{\pgfqpoint{2.211011in}{2.559081in}}%
\pgfpathlineto{\pgfqpoint{2.212085in}{2.559560in}}%
\pgfpathlineto{\pgfqpoint{2.213159in}{2.561528in}}%
\pgfpathlineto{\pgfqpoint{2.216380in}{2.560677in}}%
\pgfpathlineto{\pgfqpoint{2.217454in}{2.572250in}}%
\pgfpathlineto{\pgfqpoint{2.218528in}{2.574617in}}%
\pgfpathlineto{\pgfqpoint{2.219602in}{2.574883in}}%
\pgfpathlineto{\pgfqpoint{2.220676in}{2.580151in}}%
\pgfpathlineto{\pgfqpoint{2.223897in}{2.579965in}}%
\pgfpathlineto{\pgfqpoint{2.224971in}{2.577650in}}%
\pgfpathlineto{\pgfqpoint{2.226045in}{2.579406in}}%
\pgfpathlineto{\pgfqpoint{2.227119in}{2.578847in}}%
\pgfpathlineto{\pgfqpoint{2.228192in}{2.570707in}}%
\pgfpathlineto{\pgfqpoint{2.231414in}{2.574245in}}%
\pgfpathlineto{\pgfqpoint{2.235709in}{2.573793in}}%
\pgfpathlineto{\pgfqpoint{2.241079in}{2.575442in}}%
\pgfpathlineto{\pgfqpoint{2.242152in}{2.579672in}}%
\pgfpathlineto{\pgfqpoint{2.243226in}{2.581295in}}%
\pgfpathlineto{\pgfqpoint{2.246448in}{2.582731in}}%
\pgfpathlineto{\pgfqpoint{2.247522in}{2.581109in}}%
\pgfpathlineto{\pgfqpoint{2.248595in}{2.583237in}}%
\pgfpathlineto{\pgfqpoint{2.249669in}{2.586722in}}%
\pgfpathlineto{\pgfqpoint{2.250743in}{2.585259in}}%
\pgfpathlineto{\pgfqpoint{2.253965in}{2.583849in}}%
\pgfpathlineto{\pgfqpoint{2.255038in}{2.588770in}}%
\pgfpathlineto{\pgfqpoint{2.258260in}{2.587014in}}%
\pgfpathlineto{\pgfqpoint{2.261481in}{2.587600in}}%
\pgfpathlineto{\pgfqpoint{2.262555in}{2.586323in}}%
\pgfpathlineto{\pgfqpoint{2.265777in}{2.589967in}}%
\pgfpathlineto{\pgfqpoint{2.271146in}{2.589223in}}%
\pgfpathlineto{\pgfqpoint{2.272220in}{2.587600in}}%
\pgfpathlineto{\pgfqpoint{2.273294in}{2.589435in}}%
\pgfpathlineto{\pgfqpoint{2.277589in}{2.589143in}}%
\pgfpathlineto{\pgfqpoint{2.278663in}{2.591883in}}%
\pgfpathlineto{\pgfqpoint{2.279737in}{2.591431in}}%
\pgfpathlineto{\pgfqpoint{2.284032in}{2.591218in}}%
\pgfpathlineto{\pgfqpoint{2.285106in}{2.593453in}}%
\pgfpathlineto{\pgfqpoint{2.286180in}{2.593053in}}%
\pgfpathlineto{\pgfqpoint{2.287254in}{2.591138in}}%
\pgfpathlineto{\pgfqpoint{2.288327in}{2.593559in}}%
\pgfpathlineto{\pgfqpoint{2.291549in}{2.592096in}}%
\pgfpathlineto{\pgfqpoint{2.293697in}{2.594118in}}%
\pgfpathlineto{\pgfqpoint{2.295844in}{2.593346in}}%
\pgfpathlineto{\pgfqpoint{2.299066in}{2.593027in}}%
\pgfpathlineto{\pgfqpoint{2.301213in}{2.595235in}}%
\pgfpathlineto{\pgfqpoint{2.302287in}{2.595155in}}%
\pgfpathlineto{\pgfqpoint{2.306583in}{2.597815in}}%
\pgfpathlineto{\pgfqpoint{2.308730in}{2.604280in}}%
\pgfpathlineto{\pgfqpoint{2.309804in}{2.604254in}}%
\pgfpathlineto{\pgfqpoint{2.310878in}{2.603642in}}%
\pgfpathlineto{\pgfqpoint{2.314100in}{2.604094in}}%
\pgfpathlineto{\pgfqpoint{2.315173in}{2.602631in}}%
\pgfpathlineto{\pgfqpoint{2.317321in}{2.601593in}}%
\pgfpathlineto{\pgfqpoint{2.318395in}{2.600609in}}%
\pgfpathlineto{\pgfqpoint{2.321616in}{2.602258in}}%
\pgfpathlineto{\pgfqpoint{2.322690in}{2.602099in}}%
\pgfpathlineto{\pgfqpoint{2.323764in}{2.600955in}}%
\pgfpathlineto{\pgfqpoint{2.324838in}{2.602578in}}%
\pgfpathlineto{\pgfqpoint{2.325912in}{2.602232in}}%
\pgfpathlineto{\pgfqpoint{2.329133in}{2.604626in}}%
\pgfpathlineto{\pgfqpoint{2.330207in}{2.606940in}}%
\pgfpathlineto{\pgfqpoint{2.333429in}{2.604227in}}%
\pgfpathlineto{\pgfqpoint{2.336650in}{2.606382in}}%
\pgfpathlineto{\pgfqpoint{2.339872in}{2.602498in}}%
\pgfpathlineto{\pgfqpoint{2.340946in}{2.603934in}}%
\pgfpathlineto{\pgfqpoint{2.344167in}{2.602764in}}%
\pgfpathlineto{\pgfqpoint{2.346315in}{2.605690in}}%
\pgfpathlineto{\pgfqpoint{2.347389in}{2.604573in}}%
\pgfpathlineto{\pgfqpoint{2.348462in}{2.604972in}}%
\pgfpathlineto{\pgfqpoint{2.353832in}{2.604280in}}%
\pgfpathlineto{\pgfqpoint{2.354905in}{2.608483in}}%
\pgfpathlineto{\pgfqpoint{2.360275in}{2.611543in}}%
\pgfpathlineto{\pgfqpoint{2.361348in}{2.611702in}}%
\pgfpathlineto{\pgfqpoint{2.362422in}{2.614655in}}%
\pgfpathlineto{\pgfqpoint{2.363496in}{2.614682in}}%
\pgfpathlineto{\pgfqpoint{2.366718in}{2.614150in}}%
\pgfpathlineto{\pgfqpoint{2.367791in}{2.614948in}}%
\pgfpathlineto{\pgfqpoint{2.368865in}{2.613219in}}%
\pgfpathlineto{\pgfqpoint{2.369939in}{2.613778in}}%
\pgfpathlineto{\pgfqpoint{2.371013in}{2.610931in}}%
\pgfpathlineto{\pgfqpoint{2.374235in}{2.613538in}}%
\pgfpathlineto{\pgfqpoint{2.375308in}{2.612634in}}%
\pgfpathlineto{\pgfqpoint{2.376382in}{2.613325in}}%
\pgfpathlineto{\pgfqpoint{2.377456in}{2.615587in}}%
\pgfpathlineto{\pgfqpoint{2.378530in}{2.611303in}}%
\pgfpathlineto{\pgfqpoint{2.381751in}{2.613565in}}%
\pgfpathlineto{\pgfqpoint{2.384973in}{2.625217in}}%
\pgfpathlineto{\pgfqpoint{2.386047in}{2.625190in}}%
\pgfpathlineto{\pgfqpoint{2.393564in}{2.629048in}}%
\pgfpathlineto{\pgfqpoint{2.404302in}{2.629740in}}%
\pgfpathlineto{\pgfqpoint{2.405376in}{2.634927in}}%
\pgfpathlineto{\pgfqpoint{2.408597in}{2.634395in}}%
\pgfpathlineto{\pgfqpoint{2.411819in}{2.634049in}}%
\pgfpathlineto{\pgfqpoint{2.412893in}{2.634688in}}%
\pgfpathlineto{\pgfqpoint{2.415040in}{2.632985in}}%
\pgfpathlineto{\pgfqpoint{2.416114in}{2.634874in}}%
\pgfpathlineto{\pgfqpoint{2.419336in}{2.635300in}}%
\pgfpathlineto{\pgfqpoint{2.420410in}{2.633943in}}%
\pgfpathlineto{\pgfqpoint{2.421483in}{2.631628in}}%
\pgfpathlineto{\pgfqpoint{2.422557in}{2.631522in}}%
\pgfpathlineto{\pgfqpoint{2.423631in}{2.632453in}}%
\pgfpathlineto{\pgfqpoint{2.429000in}{2.630298in}}%
\pgfpathlineto{\pgfqpoint{2.430074in}{2.631309in}}%
\pgfpathlineto{\pgfqpoint{2.431148in}{2.629952in}}%
\pgfpathlineto{\pgfqpoint{2.434369in}{2.627718in}}%
\pgfpathlineto{\pgfqpoint{2.435443in}{2.622849in}}%
\pgfpathlineto{\pgfqpoint{2.436517in}{2.625244in}}%
\pgfpathlineto{\pgfqpoint{2.437591in}{2.623807in}}%
\pgfpathlineto{\pgfqpoint{2.438665in}{2.623807in}}%
\pgfpathlineto{\pgfqpoint{2.441886in}{2.621838in}}%
\pgfpathlineto{\pgfqpoint{2.442960in}{2.622610in}}%
\pgfpathlineto{\pgfqpoint{2.444034in}{2.620801in}}%
\pgfpathlineto{\pgfqpoint{2.445108in}{2.620455in}}%
\pgfpathlineto{\pgfqpoint{2.446182in}{2.621626in}}%
\pgfpathlineto{\pgfqpoint{2.449403in}{2.623807in}}%
\pgfpathlineto{\pgfqpoint{2.451551in}{2.622370in}}%
\pgfpathlineto{\pgfqpoint{2.452625in}{2.621732in}}%
\pgfpathlineto{\pgfqpoint{2.453699in}{2.622317in}}%
\pgfpathlineto{\pgfqpoint{2.457994in}{2.623381in}}%
\pgfpathlineto{\pgfqpoint{2.460142in}{2.622690in}}%
\pgfpathlineto{\pgfqpoint{2.461215in}{2.619630in}}%
\pgfpathlineto{\pgfqpoint{2.464437in}{2.621918in}}%
\pgfpathlineto{\pgfqpoint{2.465511in}{2.618087in}}%
\pgfpathlineto{\pgfqpoint{2.466585in}{2.618699in}}%
\pgfpathlineto{\pgfqpoint{2.467658in}{2.620615in}}%
\pgfpathlineto{\pgfqpoint{2.468732in}{2.619684in}}%
\pgfpathlineto{\pgfqpoint{2.471954in}{2.618220in}}%
\pgfpathlineto{\pgfqpoint{2.473028in}{2.618939in}}%
\pgfpathlineto{\pgfqpoint{2.475175in}{2.622663in}}%
\pgfpathlineto{\pgfqpoint{2.476249in}{2.621067in}}%
\pgfpathlineto{\pgfqpoint{2.479471in}{2.618513in}}%
\pgfpathlineto{\pgfqpoint{2.480545in}{2.622078in}}%
\pgfpathlineto{\pgfqpoint{2.481618in}{2.622503in}}%
\pgfpathlineto{\pgfqpoint{2.482692in}{2.617236in}}%
\pgfpathlineto{\pgfqpoint{2.483766in}{2.619364in}}%
\pgfpathlineto{\pgfqpoint{2.489135in}{2.621971in}}%
\pgfpathlineto{\pgfqpoint{2.490209in}{2.620907in}}%
\pgfpathlineto{\pgfqpoint{2.491283in}{2.622131in}}%
\pgfpathlineto{\pgfqpoint{2.494504in}{2.623514in}}%
\pgfpathlineto{\pgfqpoint{2.495578in}{2.618167in}}%
\pgfpathlineto{\pgfqpoint{2.497726in}{2.619684in}}%
\pgfpathlineto{\pgfqpoint{2.498800in}{2.617954in}}%
\pgfpathlineto{\pgfqpoint{2.502021in}{2.620269in}}%
\pgfpathlineto{\pgfqpoint{2.503095in}{2.612634in}}%
\pgfpathlineto{\pgfqpoint{2.504169in}{2.610612in}}%
\pgfpathlineto{\pgfqpoint{2.505243in}{2.611303in}}%
\pgfpathlineto{\pgfqpoint{2.506317in}{2.607765in}}%
\pgfpathlineto{\pgfqpoint{2.509538in}{2.608191in}}%
\pgfpathlineto{\pgfqpoint{2.511686in}{2.610532in}}%
\pgfpathlineto{\pgfqpoint{2.512760in}{2.613352in}}%
\pgfpathlineto{\pgfqpoint{2.513834in}{2.612447in}}%
\pgfpathlineto{\pgfqpoint{2.517055in}{2.614070in}}%
\pgfpathlineto{\pgfqpoint{2.519203in}{2.611277in}}%
\pgfpathlineto{\pgfqpoint{2.521350in}{2.611915in}}%
\pgfpathlineto{\pgfqpoint{2.525646in}{2.616704in}}%
\pgfpathlineto{\pgfqpoint{2.526720in}{2.624765in}}%
\pgfpathlineto{\pgfqpoint{2.532089in}{2.615587in}}%
\pgfpathlineto{\pgfqpoint{2.533163in}{2.614895in}}%
\pgfpathlineto{\pgfqpoint{2.535310in}{2.615507in}}%
\pgfpathlineto{\pgfqpoint{2.536384in}{2.614496in}}%
\pgfpathlineto{\pgfqpoint{2.539606in}{2.613591in}}%
\pgfpathlineto{\pgfqpoint{2.540679in}{2.607978in}}%
\pgfpathlineto{\pgfqpoint{2.541753in}{2.608776in}}%
\pgfpathlineto{\pgfqpoint{2.543901in}{2.611436in}}%
\pgfpathlineto{\pgfqpoint{2.547123in}{2.608936in}}%
\pgfpathlineto{\pgfqpoint{2.548196in}{2.607313in}}%
\pgfpathlineto{\pgfqpoint{2.549270in}{2.604387in}}%
\pgfpathlineto{\pgfqpoint{2.550344in}{2.604573in}}%
\pgfpathlineto{\pgfqpoint{2.551418in}{2.606036in}}%
\pgfpathlineto{\pgfqpoint{2.555713in}{2.606275in}}%
\pgfpathlineto{\pgfqpoint{2.556787in}{2.603881in}}%
\pgfpathlineto{\pgfqpoint{2.557861in}{2.603615in}}%
\pgfpathlineto{\pgfqpoint{2.558935in}{2.606781in}}%
\pgfpathlineto{\pgfqpoint{2.563230in}{2.615826in}}%
\pgfpathlineto{\pgfqpoint{2.564304in}{2.613565in}}%
\pgfpathlineto{\pgfqpoint{2.565378in}{2.615826in}}%
\pgfpathlineto{\pgfqpoint{2.570747in}{2.615241in}}%
\pgfpathlineto{\pgfqpoint{2.571821in}{2.614363in}}%
\pgfpathlineto{\pgfqpoint{2.572895in}{2.614682in}}%
\pgfpathlineto{\pgfqpoint{2.573968in}{2.615986in}}%
\pgfpathlineto{\pgfqpoint{2.578264in}{2.615853in}}%
\pgfpathlineto{\pgfqpoint{2.579338in}{2.613857in}}%
\pgfpathlineto{\pgfqpoint{2.580412in}{2.614815in}}%
\pgfpathlineto{\pgfqpoint{2.581485in}{2.614123in}}%
\pgfpathlineto{\pgfqpoint{2.585781in}{2.615693in}}%
\pgfpathlineto{\pgfqpoint{2.586855in}{2.615214in}}%
\pgfpathlineto{\pgfqpoint{2.587928in}{2.618220in}}%
\pgfpathlineto{\pgfqpoint{2.589002in}{2.616837in}}%
\pgfpathlineto{\pgfqpoint{2.593298in}{2.616491in}}%
\pgfpathlineto{\pgfqpoint{2.594371in}{2.612926in}}%
\pgfpathlineto{\pgfqpoint{2.596519in}{2.612607in}}%
\pgfpathlineto{\pgfqpoint{2.600814in}{2.613272in}}%
\pgfpathlineto{\pgfqpoint{2.601888in}{2.612846in}}%
\pgfpathlineto{\pgfqpoint{2.602962in}{2.611596in}}%
\pgfpathlineto{\pgfqpoint{2.607257in}{2.610878in}}%
\pgfpathlineto{\pgfqpoint{2.608331in}{2.605530in}}%
\pgfpathlineto{\pgfqpoint{2.609405in}{2.608111in}}%
\pgfpathlineto{\pgfqpoint{2.610479in}{2.605717in}}%
\pgfpathlineto{\pgfqpoint{2.611553in}{2.609574in}}%
\pgfpathlineto{\pgfqpoint{2.615848in}{2.609228in}}%
\pgfpathlineto{\pgfqpoint{2.616922in}{2.609175in}}%
\pgfpathlineto{\pgfqpoint{2.619070in}{2.610213in}}%
\pgfpathlineto{\pgfqpoint{2.624439in}{2.609601in}}%
\pgfpathlineto{\pgfqpoint{2.625513in}{2.611303in}}%
\pgfpathlineto{\pgfqpoint{2.626587in}{2.614203in}}%
\pgfpathlineto{\pgfqpoint{2.629808in}{2.615799in}}%
\pgfpathlineto{\pgfqpoint{2.630882in}{2.616997in}}%
\pgfpathlineto{\pgfqpoint{2.634103in}{2.623222in}}%
\pgfpathlineto{\pgfqpoint{2.638399in}{2.625217in}}%
\pgfpathlineto{\pgfqpoint{2.639473in}{2.624845in}}%
\pgfpathlineto{\pgfqpoint{2.641620in}{2.634874in}}%
\pgfpathlineto{\pgfqpoint{2.645916in}{2.633571in}}%
\pgfpathlineto{\pgfqpoint{2.646989in}{2.637694in}}%
\pgfpathlineto{\pgfqpoint{2.648063in}{2.637135in}}%
\pgfpathlineto{\pgfqpoint{2.649137in}{2.637534in}}%
\pgfpathlineto{\pgfqpoint{2.653433in}{2.637641in}}%
\pgfpathlineto{\pgfqpoint{2.654506in}{2.638306in}}%
\pgfpathlineto{\pgfqpoint{2.655580in}{2.642935in}}%
\pgfpathlineto{\pgfqpoint{2.656654in}{2.643600in}}%
\pgfpathlineto{\pgfqpoint{2.660949in}{2.645702in}}%
\pgfpathlineto{\pgfqpoint{2.662023in}{2.650490in}}%
\pgfpathlineto{\pgfqpoint{2.664171in}{2.648229in}}%
\pgfpathlineto{\pgfqpoint{2.667392in}{2.648256in}}%
\pgfpathlineto{\pgfqpoint{2.670614in}{2.642829in}}%
\pgfpathlineto{\pgfqpoint{2.671688in}{2.641764in}}%
\pgfpathlineto{\pgfqpoint{2.674909in}{2.642563in}}%
\pgfpathlineto{\pgfqpoint{2.675983in}{2.642217in}}%
\pgfpathlineto{\pgfqpoint{2.677057in}{2.640354in}}%
\pgfpathlineto{\pgfqpoint{2.679205in}{2.639450in}}%
\pgfpathlineto{\pgfqpoint{2.686722in}{2.640354in}}%
\pgfpathlineto{\pgfqpoint{2.691017in}{2.638599in}}%
\pgfpathlineto{\pgfqpoint{2.692091in}{2.641578in}}%
\pgfpathlineto{\pgfqpoint{2.693165in}{2.640541in}}%
\pgfpathlineto{\pgfqpoint{2.697460in}{2.642376in}}%
\pgfpathlineto{\pgfqpoint{2.698534in}{2.632586in}}%
\pgfpathlineto{\pgfqpoint{2.699608in}{2.631469in}}%
\pgfpathlineto{\pgfqpoint{2.700681in}{2.632719in}}%
\pgfpathlineto{\pgfqpoint{2.701755in}{2.632453in}}%
\pgfpathlineto{\pgfqpoint{2.708198in}{2.637401in}}%
\pgfpathlineto{\pgfqpoint{2.709272in}{2.636736in}}%
\pgfpathlineto{\pgfqpoint{2.712494in}{2.636391in}}%
\pgfpathlineto{\pgfqpoint{2.713567in}{2.637295in}}%
\pgfpathlineto{\pgfqpoint{2.714641in}{2.636391in}}%
\pgfpathlineto{\pgfqpoint{2.715715in}{2.637934in}}%
\pgfpathlineto{\pgfqpoint{2.716789in}{2.636816in}}%
\pgfpathlineto{\pgfqpoint{2.721084in}{2.636071in}}%
\pgfpathlineto{\pgfqpoint{2.722158in}{2.635087in}}%
\pgfpathlineto{\pgfqpoint{2.724306in}{2.637162in}}%
\pgfpathlineto{\pgfqpoint{2.728601in}{2.647511in}}%
\pgfpathlineto{\pgfqpoint{2.729675in}{2.644744in}}%
\pgfpathlineto{\pgfqpoint{2.730749in}{2.645515in}}%
\pgfpathlineto{\pgfqpoint{2.735044in}{2.646234in}}%
\pgfpathlineto{\pgfqpoint{2.736118in}{2.646872in}}%
\pgfpathlineto{\pgfqpoint{2.737192in}{2.646846in}}%
\pgfpathlineto{\pgfqpoint{2.738266in}{2.649027in}}%
\pgfpathlineto{\pgfqpoint{2.739340in}{2.647484in}}%
\pgfpathlineto{\pgfqpoint{2.743635in}{2.647963in}}%
\pgfpathlineto{\pgfqpoint{2.746856in}{2.655465in}}%
\pgfpathlineto{\pgfqpoint{2.750078in}{2.656183in}}%
\pgfpathlineto{\pgfqpoint{2.751152in}{2.657354in}}%
\pgfpathlineto{\pgfqpoint{2.753300in}{2.656662in}}%
\pgfpathlineto{\pgfqpoint{2.754373in}{2.659136in}}%
\pgfpathlineto{\pgfqpoint{2.758669in}{2.660360in}}%
\pgfpathlineto{\pgfqpoint{2.759743in}{2.662249in}}%
\pgfpathlineto{\pgfqpoint{2.760816in}{2.662941in}}%
\pgfpathlineto{\pgfqpoint{2.761890in}{2.666240in}}%
\pgfpathlineto{\pgfqpoint{2.765112in}{2.665548in}}%
\pgfpathlineto{\pgfqpoint{2.766186in}{2.665920in}}%
\pgfpathlineto{\pgfqpoint{2.767259in}{2.667650in}}%
\pgfpathlineto{\pgfqpoint{2.768333in}{2.670523in}}%
\pgfpathlineto{\pgfqpoint{2.769407in}{2.671480in}}%
\pgfpathlineto{\pgfqpoint{2.772629in}{2.671268in}}%
\pgfpathlineto{\pgfqpoint{2.775850in}{2.661930in}}%
\pgfpathlineto{\pgfqpoint{2.776924in}{2.661052in}}%
\pgfpathlineto{\pgfqpoint{2.780145in}{2.662595in}}%
\pgfpathlineto{\pgfqpoint{2.782293in}{2.664457in}}%
\pgfpathlineto{\pgfqpoint{2.783367in}{2.661611in}}%
\pgfpathlineto{\pgfqpoint{2.784441in}{2.661664in}}%
\pgfpathlineto{\pgfqpoint{2.788736in}{2.658365in}}%
\pgfpathlineto{\pgfqpoint{2.789810in}{2.660892in}}%
\pgfpathlineto{\pgfqpoint{2.790884in}{2.659961in}}%
\pgfpathlineto{\pgfqpoint{2.791958in}{2.661823in}}%
\pgfpathlineto{\pgfqpoint{2.795179in}{2.660679in}}%
\pgfpathlineto{\pgfqpoint{2.796253in}{2.666532in}}%
\pgfpathlineto{\pgfqpoint{2.797327in}{2.668421in}}%
\pgfpathlineto{\pgfqpoint{2.798401in}{2.671853in}}%
\pgfpathlineto{\pgfqpoint{2.799475in}{2.668661in}}%
\pgfpathlineto{\pgfqpoint{2.803770in}{2.660014in}}%
\pgfpathlineto{\pgfqpoint{2.804844in}{2.657673in}}%
\pgfpathlineto{\pgfqpoint{2.805918in}{2.657354in}}%
\pgfpathlineto{\pgfqpoint{2.806991in}{2.659935in}}%
\pgfpathlineto{\pgfqpoint{2.810213in}{2.662116in}}%
\pgfpathlineto{\pgfqpoint{2.812361in}{2.660706in}}%
\pgfpathlineto{\pgfqpoint{2.813434in}{2.663712in}}%
\pgfpathlineto{\pgfqpoint{2.818804in}{2.660999in}}%
\pgfpathlineto{\pgfqpoint{2.819877in}{2.663420in}}%
\pgfpathlineto{\pgfqpoint{2.822025in}{2.663100in}}%
\pgfpathlineto{\pgfqpoint{2.825247in}{2.664058in}}%
\pgfpathlineto{\pgfqpoint{2.826321in}{2.663792in}}%
\pgfpathlineto{\pgfqpoint{2.827394in}{2.665787in}}%
\pgfpathlineto{\pgfqpoint{2.828468in}{2.662941in}}%
\pgfpathlineto{\pgfqpoint{2.829542in}{2.661903in}}%
\pgfpathlineto{\pgfqpoint{2.832764in}{2.663978in}}%
\pgfpathlineto{\pgfqpoint{2.833837in}{2.667117in}}%
\pgfpathlineto{\pgfqpoint{2.834911in}{2.662329in}}%
\pgfpathlineto{\pgfqpoint{2.835985in}{2.662568in}}%
\pgfpathlineto{\pgfqpoint{2.837059in}{2.661611in}}%
\pgfpathlineto{\pgfqpoint{2.840280in}{2.661770in}}%
\pgfpathlineto{\pgfqpoint{2.841354in}{2.662967in}}%
\pgfpathlineto{\pgfqpoint{2.842428in}{2.660068in}}%
\pgfpathlineto{\pgfqpoint{2.843502in}{2.663366in}}%
\pgfpathlineto{\pgfqpoint{2.844576in}{2.659961in}}%
\pgfpathlineto{\pgfqpoint{2.848871in}{2.657088in}}%
\pgfpathlineto{\pgfqpoint{2.849945in}{2.659057in}}%
\pgfpathlineto{\pgfqpoint{2.851019in}{2.662914in}}%
\pgfpathlineto{\pgfqpoint{2.852093in}{2.659828in}}%
\pgfpathlineto{\pgfqpoint{2.855314in}{2.665495in}}%
\pgfpathlineto{\pgfqpoint{2.856388in}{2.664031in}}%
\pgfpathlineto{\pgfqpoint{2.857462in}{2.663579in}}%
\pgfpathlineto{\pgfqpoint{2.858536in}{2.667969in}}%
\pgfpathlineto{\pgfqpoint{2.862831in}{2.671028in}}%
\pgfpathlineto{\pgfqpoint{2.863905in}{2.670603in}}%
\pgfpathlineto{\pgfqpoint{2.866053in}{2.661797in}}%
\pgfpathlineto{\pgfqpoint{2.867126in}{2.660892in}}%
\pgfpathlineto{\pgfqpoint{2.871422in}{2.659748in}}%
\pgfpathlineto{\pgfqpoint{2.872496in}{2.656716in}}%
\pgfpathlineto{\pgfqpoint{2.873569in}{2.655997in}}%
\pgfpathlineto{\pgfqpoint{2.874643in}{2.657381in}}%
\pgfpathlineto{\pgfqpoint{2.877865in}{2.660440in}}%
\pgfpathlineto{\pgfqpoint{2.880012in}{2.664697in}}%
\pgfpathlineto{\pgfqpoint{2.882160in}{2.665521in}}%
\pgfpathlineto{\pgfqpoint{2.885382in}{2.666107in}}%
\pgfpathlineto{\pgfqpoint{2.886455in}{2.667251in}}%
\pgfpathlineto{\pgfqpoint{2.887529in}{2.674247in}}%
\pgfpathlineto{\pgfqpoint{2.888603in}{2.674699in}}%
\pgfpathlineto{\pgfqpoint{2.889677in}{2.673689in}}%
\pgfpathlineto{\pgfqpoint{2.892899in}{2.672890in}}%
\pgfpathlineto{\pgfqpoint{2.893972in}{2.684756in}}%
\pgfpathlineto{\pgfqpoint{2.895046in}{2.684490in}}%
\pgfpathlineto{\pgfqpoint{2.896120in}{2.687921in}}%
\pgfpathlineto{\pgfqpoint{2.900415in}{2.691939in}}%
\pgfpathlineto{\pgfqpoint{2.901489in}{2.686910in}}%
\pgfpathlineto{\pgfqpoint{2.902563in}{2.688746in}}%
\pgfpathlineto{\pgfqpoint{2.903637in}{2.687336in}}%
\pgfpathlineto{\pgfqpoint{2.904711in}{2.687283in}}%
\pgfpathlineto{\pgfqpoint{2.909006in}{2.681244in}}%
\pgfpathlineto{\pgfqpoint{2.910080in}{2.682707in}}%
\pgfpathlineto{\pgfqpoint{2.916523in}{2.682175in}}%
\pgfpathlineto{\pgfqpoint{2.917597in}{2.686112in}}%
\pgfpathlineto{\pgfqpoint{2.919744in}{2.681643in}}%
\pgfpathlineto{\pgfqpoint{2.922966in}{2.682148in}}%
\pgfpathlineto{\pgfqpoint{2.926188in}{2.679887in}}%
\pgfpathlineto{\pgfqpoint{2.927261in}{2.677679in}}%
\pgfpathlineto{\pgfqpoint{2.930483in}{2.677493in}}%
\pgfpathlineto{\pgfqpoint{2.931557in}{2.678530in}}%
\pgfpathlineto{\pgfqpoint{2.932631in}{2.675870in}}%
\pgfpathlineto{\pgfqpoint{2.938000in}{2.680073in}}%
\pgfpathlineto{\pgfqpoint{2.939074in}{2.684968in}}%
\pgfpathlineto{\pgfqpoint{2.940147in}{2.684436in}}%
\pgfpathlineto{\pgfqpoint{2.941221in}{2.683213in}}%
\pgfpathlineto{\pgfqpoint{2.942295in}{2.684889in}}%
\pgfpathlineto{\pgfqpoint{2.945517in}{2.682388in}}%
\pgfpathlineto{\pgfqpoint{2.946590in}{2.684091in}}%
\pgfpathlineto{\pgfqpoint{2.947664in}{2.687629in}}%
\pgfpathlineto{\pgfqpoint{2.948738in}{2.684942in}}%
\pgfpathlineto{\pgfqpoint{2.949812in}{2.686432in}}%
\pgfpathlineto{\pgfqpoint{2.953033in}{2.687842in}}%
\pgfpathlineto{\pgfqpoint{2.954107in}{2.691566in}}%
\pgfpathlineto{\pgfqpoint{2.955181in}{2.692311in}}%
\pgfpathlineto{\pgfqpoint{2.956255in}{2.689145in}}%
\pgfpathlineto{\pgfqpoint{2.957329in}{2.691220in}}%
\pgfpathlineto{\pgfqpoint{2.963772in}{2.687629in}}%
\pgfpathlineto{\pgfqpoint{2.964846in}{2.685234in}}%
\pgfpathlineto{\pgfqpoint{2.969141in}{2.685155in}}%
\pgfpathlineto{\pgfqpoint{2.970215in}{2.686378in}}%
\pgfpathlineto{\pgfqpoint{2.971289in}{2.686352in}}%
\pgfpathlineto{\pgfqpoint{2.972363in}{2.684224in}}%
\pgfpathlineto{\pgfqpoint{2.972363in}{2.684224in}}%
\pgfusepath{stroke}%
\end{pgfscope}%
\begin{pgfscope}%
\pgfpathrectangle{\pgfqpoint{0.506453in}{2.309648in}}{\pgfqpoint{2.583333in}{0.400885in}}%
\pgfusepath{clip}%
\pgfsetroundcap%
\pgfsetroundjoin%
\pgfsetlinewidth{1.505625pt}%
\definecolor{currentstroke}{rgb}{0.580392,0.403922,0.741176}%
\pgfsetstrokecolor{currentstroke}%
\pgfsetdash{}{0pt}%
\pgfpathmoveto{\pgfqpoint{0.623878in}{2.468283in}}%
\pgfpathlineto{\pgfqpoint{0.627099in}{2.466049in}}%
\pgfpathlineto{\pgfqpoint{0.631395in}{2.465252in}}%
\pgfpathlineto{\pgfqpoint{0.633542in}{2.465305in}}%
\pgfpathlineto{\pgfqpoint{0.642133in}{2.465384in}}%
\pgfpathlineto{\pgfqpoint{0.647502in}{2.464384in}}%
\pgfpathlineto{\pgfqpoint{0.649650in}{2.463082in}}%
\pgfpathlineto{\pgfqpoint{0.655019in}{2.463340in}}%
\pgfpathlineto{\pgfqpoint{0.668979in}{2.461100in}}%
\pgfpathlineto{\pgfqpoint{0.670053in}{2.461177in}}%
\pgfpathlineto{\pgfqpoint{0.671127in}{2.460611in}}%
\pgfpathlineto{\pgfqpoint{0.676496in}{2.460866in}}%
\pgfpathlineto{\pgfqpoint{0.682939in}{2.460815in}}%
\pgfpathlineto{\pgfqpoint{0.684013in}{2.459309in}}%
\pgfpathlineto{\pgfqpoint{0.687234in}{2.458485in}}%
\pgfpathlineto{\pgfqpoint{0.690456in}{2.458760in}}%
\pgfpathlineto{\pgfqpoint{0.691530in}{2.457636in}}%
\pgfpathlineto{\pgfqpoint{0.692603in}{2.457537in}}%
\pgfpathlineto{\pgfqpoint{0.693677in}{2.456413in}}%
\pgfpathlineto{\pgfqpoint{0.697973in}{2.455481in}}%
\pgfpathlineto{\pgfqpoint{0.699046in}{2.455942in}}%
\pgfpathlineto{\pgfqpoint{0.701194in}{2.455408in}}%
\pgfpathlineto{\pgfqpoint{0.705489in}{2.455700in}}%
\pgfpathlineto{\pgfqpoint{0.708711in}{2.454220in}}%
\pgfpathlineto{\pgfqpoint{0.709785in}{2.454389in}}%
\pgfpathlineto{\pgfqpoint{0.713006in}{2.453152in}}%
\pgfpathlineto{\pgfqpoint{0.716228in}{2.453866in}}%
\pgfpathlineto{\pgfqpoint{0.717302in}{2.453057in}}%
\pgfpathlineto{\pgfqpoint{0.720523in}{2.453527in}}%
\pgfpathlineto{\pgfqpoint{0.723745in}{2.451859in}}%
\pgfpathlineto{\pgfqpoint{0.728040in}{2.451060in}}%
\pgfpathlineto{\pgfqpoint{0.729114in}{2.449697in}}%
\pgfpathlineto{\pgfqpoint{0.731262in}{2.449603in}}%
\pgfpathlineto{\pgfqpoint{0.732335in}{2.448428in}}%
\pgfpathlineto{\pgfqpoint{0.735557in}{2.447583in}}%
\pgfpathlineto{\pgfqpoint{0.736631in}{2.448023in}}%
\pgfpathlineto{\pgfqpoint{0.737705in}{2.446216in}}%
\pgfpathlineto{\pgfqpoint{0.738778in}{2.445799in}}%
\pgfpathlineto{\pgfqpoint{0.739852in}{2.444524in}}%
\pgfpathlineto{\pgfqpoint{0.744148in}{2.443141in}}%
\pgfpathlineto{\pgfqpoint{0.746295in}{2.441384in}}%
\pgfpathlineto{\pgfqpoint{0.753812in}{2.442427in}}%
\pgfpathlineto{\pgfqpoint{0.754886in}{2.441384in}}%
\pgfpathlineto{\pgfqpoint{0.759181in}{2.441797in}}%
\pgfpathlineto{\pgfqpoint{0.760255in}{2.440559in}}%
\pgfpathlineto{\pgfqpoint{0.761329in}{2.441080in}}%
\pgfpathlineto{\pgfqpoint{0.766698in}{2.439386in}}%
\pgfpathlineto{\pgfqpoint{0.768846in}{2.439278in}}%
\pgfpathlineto{\pgfqpoint{0.769920in}{2.438930in}}%
\pgfpathlineto{\pgfqpoint{0.775289in}{2.438778in}}%
\pgfpathlineto{\pgfqpoint{0.777437in}{2.436969in}}%
\pgfpathlineto{\pgfqpoint{0.781732in}{2.437118in}}%
\pgfpathlineto{\pgfqpoint{0.782806in}{2.436433in}}%
\pgfpathlineto{\pgfqpoint{0.783880in}{2.436819in}}%
\pgfpathlineto{\pgfqpoint{0.784953in}{2.435684in}}%
\pgfpathlineto{\pgfqpoint{0.789249in}{2.434467in}}%
\pgfpathlineto{\pgfqpoint{0.790323in}{2.433437in}}%
\pgfpathlineto{\pgfqpoint{0.792470in}{2.433726in}}%
\pgfpathlineto{\pgfqpoint{0.795692in}{2.432281in}}%
\pgfpathlineto{\pgfqpoint{0.799987in}{2.425987in}}%
\pgfpathlineto{\pgfqpoint{0.803209in}{2.426425in}}%
\pgfpathlineto{\pgfqpoint{0.804283in}{2.425805in}}%
\pgfpathlineto{\pgfqpoint{0.805356in}{2.426219in}}%
\pgfpathlineto{\pgfqpoint{0.806430in}{2.425318in}}%
\pgfpathlineto{\pgfqpoint{0.807504in}{2.425678in}}%
\pgfpathlineto{\pgfqpoint{0.811799in}{2.425372in}}%
\pgfpathlineto{\pgfqpoint{0.812873in}{2.424759in}}%
\pgfpathlineto{\pgfqpoint{0.813947in}{2.424902in}}%
\pgfpathlineto{\pgfqpoint{0.815021in}{2.423994in}}%
\pgfpathlineto{\pgfqpoint{0.822538in}{2.422864in}}%
\pgfpathlineto{\pgfqpoint{0.828981in}{2.422968in}}%
\pgfpathlineto{\pgfqpoint{0.830055in}{2.421709in}}%
\pgfpathlineto{\pgfqpoint{0.833276in}{2.421490in}}%
\pgfpathlineto{\pgfqpoint{0.834350in}{2.420734in}}%
\pgfpathlineto{\pgfqpoint{0.836498in}{2.421462in}}%
\pgfpathlineto{\pgfqpoint{0.837572in}{2.420238in}}%
\pgfpathlineto{\pgfqpoint{0.840793in}{2.419527in}}%
\pgfpathlineto{\pgfqpoint{0.841867in}{2.418502in}}%
\pgfpathlineto{\pgfqpoint{0.842941in}{2.418750in}}%
\pgfpathlineto{\pgfqpoint{0.845088in}{2.416092in}}%
\pgfpathlineto{\pgfqpoint{0.850458in}{2.415921in}}%
\pgfpathlineto{\pgfqpoint{0.852605in}{2.413862in}}%
\pgfpathlineto{\pgfqpoint{0.855827in}{2.413510in}}%
\pgfpathlineto{\pgfqpoint{0.856901in}{2.412822in}}%
\pgfpathlineto{\pgfqpoint{0.865491in}{2.412899in}}%
\pgfpathlineto{\pgfqpoint{0.873008in}{2.412149in}}%
\pgfpathlineto{\pgfqpoint{0.874082in}{2.412134in}}%
\pgfpathlineto{\pgfqpoint{0.875156in}{2.411538in}}%
\pgfpathlineto{\pgfqpoint{0.895559in}{2.410792in}}%
\pgfpathlineto{\pgfqpoint{0.897707in}{2.409133in}}%
\pgfpathlineto{\pgfqpoint{0.902002in}{2.409234in}}%
\pgfpathlineto{\pgfqpoint{0.923479in}{2.410290in}}%
\pgfpathlineto{\pgfqpoint{0.924553in}{2.409060in}}%
\pgfpathlineto{\pgfqpoint{0.930996in}{2.407787in}}%
\pgfpathlineto{\pgfqpoint{0.933143in}{2.405022in}}%
\pgfpathlineto{\pgfqpoint{0.935291in}{2.402667in}}%
\pgfpathlineto{\pgfqpoint{0.938512in}{2.402578in}}%
\pgfpathlineto{\pgfqpoint{0.939586in}{2.401639in}}%
\pgfpathlineto{\pgfqpoint{0.940660in}{2.401487in}}%
\pgfpathlineto{\pgfqpoint{0.941734in}{2.401906in}}%
\pgfpathlineto{\pgfqpoint{0.942808in}{2.401652in}}%
\pgfpathlineto{\pgfqpoint{0.948177in}{2.401563in}}%
\pgfpathlineto{\pgfqpoint{0.950325in}{2.400217in}}%
\pgfpathlineto{\pgfqpoint{0.955694in}{2.399644in}}%
\pgfpathlineto{\pgfqpoint{0.956768in}{2.398935in}}%
\pgfpathlineto{\pgfqpoint{0.957841in}{2.399159in}}%
\pgfpathlineto{\pgfqpoint{0.977171in}{2.397503in}}%
\pgfpathlineto{\pgfqpoint{0.980392in}{2.396422in}}%
\pgfpathlineto{\pgfqpoint{0.986835in}{2.396743in}}%
\pgfpathlineto{\pgfqpoint{0.987909in}{2.397136in}}%
\pgfpathlineto{\pgfqpoint{0.992204in}{2.396791in}}%
\pgfpathlineto{\pgfqpoint{0.995426in}{2.396380in}}%
\pgfpathlineto{\pgfqpoint{1.001869in}{2.396439in}}%
\pgfpathlineto{\pgfqpoint{1.002943in}{2.395970in}}%
\pgfpathlineto{\pgfqpoint{1.010460in}{2.395196in}}%
\pgfpathlineto{\pgfqpoint{1.016903in}{2.393782in}}%
\pgfpathlineto{\pgfqpoint{1.017976in}{2.393026in}}%
\pgfpathlineto{\pgfqpoint{1.023346in}{2.393191in}}%
\pgfpathlineto{\pgfqpoint{1.025493in}{2.393765in}}%
\pgfpathlineto{\pgfqpoint{1.039453in}{2.394459in}}%
\pgfpathlineto{\pgfqpoint{1.040527in}{2.393710in}}%
\pgfpathlineto{\pgfqpoint{1.043749in}{2.393440in}}%
\pgfpathlineto{\pgfqpoint{1.045896in}{2.392443in}}%
\pgfpathlineto{\pgfqpoint{1.048044in}{2.392528in}}%
\pgfpathlineto{\pgfqpoint{1.052339in}{2.391979in}}%
\pgfpathlineto{\pgfqpoint{1.054487in}{2.391059in}}%
\pgfpathlineto{\pgfqpoint{1.055561in}{2.391416in}}%
\pgfpathlineto{\pgfqpoint{1.062004in}{2.391702in}}%
\pgfpathlineto{\pgfqpoint{1.063078in}{2.391997in}}%
\pgfpathlineto{\pgfqpoint{1.073816in}{2.390655in}}%
\pgfpathlineto{\pgfqpoint{1.074890in}{2.390815in}}%
\pgfpathlineto{\pgfqpoint{1.078111in}{2.389667in}}%
\pgfpathlineto{\pgfqpoint{1.090997in}{2.391168in}}%
\pgfpathlineto{\pgfqpoint{1.093145in}{2.390780in}}%
\pgfpathlineto{\pgfqpoint{1.099588in}{2.389687in}}%
\pgfpathlineto{\pgfqpoint{1.100662in}{2.389114in}}%
\pgfpathlineto{\pgfqpoint{1.103884in}{2.389068in}}%
\pgfpathlineto{\pgfqpoint{1.104957in}{2.388154in}}%
\pgfpathlineto{\pgfqpoint{1.107105in}{2.388665in}}%
\pgfpathlineto{\pgfqpoint{1.111400in}{2.388969in}}%
\pgfpathlineto{\pgfqpoint{1.113548in}{2.387938in}}%
\pgfpathlineto{\pgfqpoint{1.115696in}{2.387912in}}%
\pgfpathlineto{\pgfqpoint{1.119991in}{2.387525in}}%
\pgfpathlineto{\pgfqpoint{1.123213in}{2.388431in}}%
\pgfpathlineto{\pgfqpoint{1.126434in}{2.387665in}}%
\pgfpathlineto{\pgfqpoint{1.127508in}{2.386381in}}%
\pgfpathlineto{\pgfqpoint{1.128582in}{2.386710in}}%
\pgfpathlineto{\pgfqpoint{1.130729in}{2.385268in}}%
\pgfpathlineto{\pgfqpoint{1.135025in}{2.385080in}}%
\pgfpathlineto{\pgfqpoint{1.137173in}{2.383799in}}%
\pgfpathlineto{\pgfqpoint{1.143616in}{2.383136in}}%
\pgfpathlineto{\pgfqpoint{1.144689in}{2.382512in}}%
\pgfpathlineto{\pgfqpoint{1.145763in}{2.382891in}}%
\pgfpathlineto{\pgfqpoint{1.153280in}{2.381783in}}%
\pgfpathlineto{\pgfqpoint{1.156502in}{2.381844in}}%
\pgfpathlineto{\pgfqpoint{1.158649in}{2.380720in}}%
\pgfpathlineto{\pgfqpoint{1.172609in}{2.379360in}}%
\pgfpathlineto{\pgfqpoint{1.173683in}{2.378193in}}%
\pgfpathlineto{\pgfqpoint{1.174757in}{2.378397in}}%
\pgfpathlineto{\pgfqpoint{1.175831in}{2.377315in}}%
\pgfpathlineto{\pgfqpoint{1.180126in}{2.377273in}}%
\pgfpathlineto{\pgfqpoint{1.183348in}{2.376818in}}%
\pgfpathlineto{\pgfqpoint{1.187643in}{2.376694in}}%
\pgfpathlineto{\pgfqpoint{1.190864in}{2.375453in}}%
\pgfpathlineto{\pgfqpoint{1.195160in}{2.374635in}}%
\pgfpathlineto{\pgfqpoint{1.197307in}{2.372579in}}%
\pgfpathlineto{\pgfqpoint{1.198381in}{2.372897in}}%
\pgfpathlineto{\pgfqpoint{1.202677in}{2.371724in}}%
\pgfpathlineto{\pgfqpoint{1.203751in}{2.370869in}}%
\pgfpathlineto{\pgfqpoint{1.218784in}{2.369410in}}%
\pgfpathlineto{\pgfqpoint{1.224153in}{2.369957in}}%
\pgfpathlineto{\pgfqpoint{1.227375in}{2.369851in}}%
\pgfpathlineto{\pgfqpoint{1.228449in}{2.368879in}}%
\pgfpathlineto{\pgfqpoint{1.241335in}{2.369456in}}%
\pgfpathlineto{\pgfqpoint{1.246704in}{2.369589in}}%
\pgfpathlineto{\pgfqpoint{1.249926in}{2.369381in}}%
\pgfpathlineto{\pgfqpoint{1.250999in}{2.368938in}}%
\pgfpathlineto{\pgfqpoint{1.255295in}{2.368325in}}%
\pgfpathlineto{\pgfqpoint{1.257442in}{2.366795in}}%
\pgfpathlineto{\pgfqpoint{1.262812in}{2.365931in}}%
\pgfpathlineto{\pgfqpoint{1.266033in}{2.364948in}}%
\pgfpathlineto{\pgfqpoint{1.277845in}{2.364097in}}%
\pgfpathlineto{\pgfqpoint{1.281067in}{2.364412in}}%
\pgfpathlineto{\pgfqpoint{1.285362in}{2.364167in}}%
\pgfpathlineto{\pgfqpoint{1.288584in}{2.363599in}}%
\pgfpathlineto{\pgfqpoint{1.295027in}{2.363516in}}%
\pgfpathlineto{\pgfqpoint{1.317577in}{2.359953in}}%
\pgfpathlineto{\pgfqpoint{1.318651in}{2.359327in}}%
\pgfpathlineto{\pgfqpoint{1.324020in}{2.359079in}}%
\pgfpathlineto{\pgfqpoint{1.326168in}{2.358659in}}%
\pgfpathlineto{\pgfqpoint{1.347645in}{2.357224in}}%
\pgfpathlineto{\pgfqpoint{1.348719in}{2.356779in}}%
\pgfpathlineto{\pgfqpoint{1.375565in}{2.356960in}}%
\pgfpathlineto{\pgfqpoint{1.384155in}{2.355771in}}%
\pgfpathlineto{\pgfqpoint{1.385229in}{2.355171in}}%
\pgfpathlineto{\pgfqpoint{1.390598in}{2.355018in}}%
\pgfpathlineto{\pgfqpoint{1.392746in}{2.354219in}}%
\pgfpathlineto{\pgfqpoint{1.405632in}{2.354103in}}%
\pgfpathlineto{\pgfqpoint{1.421740in}{2.353418in}}%
\pgfpathlineto{\pgfqpoint{1.430330in}{2.352848in}}%
\pgfpathlineto{\pgfqpoint{1.431404in}{2.352258in}}%
\pgfpathlineto{\pgfqpoint{1.445364in}{2.350925in}}%
\pgfpathlineto{\pgfqpoint{1.446438in}{2.350580in}}%
\pgfpathlineto{\pgfqpoint{1.476506in}{2.348491in}}%
\pgfpathlineto{\pgfqpoint{1.488318in}{2.348119in}}%
\pgfpathlineto{\pgfqpoint{1.506573in}{2.347345in}}%
\pgfpathlineto{\pgfqpoint{1.517311in}{2.346724in}}%
\pgfpathlineto{\pgfqpoint{1.519459in}{2.346223in}}%
\pgfpathlineto{\pgfqpoint{1.528050in}{2.346057in}}%
\pgfpathlineto{\pgfqpoint{1.572077in}{2.345253in}}%
\pgfpathlineto{\pgfqpoint{1.577446in}{2.345348in}}%
\pgfpathlineto{\pgfqpoint{1.581742in}{2.344921in}}%
\pgfpathlineto{\pgfqpoint{1.588185in}{2.344495in}}%
\pgfpathlineto{\pgfqpoint{1.589259in}{2.344251in}}%
\pgfpathlineto{\pgfqpoint{1.615031in}{2.343173in}}%
\pgfpathlineto{\pgfqpoint{1.619326in}{2.342192in}}%
\pgfpathlineto{\pgfqpoint{1.646172in}{2.341089in}}%
\pgfpathlineto{\pgfqpoint{1.652615in}{2.340650in}}%
\pgfpathlineto{\pgfqpoint{1.664427in}{2.340618in}}%
\pgfpathlineto{\pgfqpoint{1.698790in}{2.339812in}}%
\pgfpathlineto{\pgfqpoint{1.702012in}{2.339333in}}%
\pgfpathlineto{\pgfqpoint{1.706307in}{2.339015in}}%
\pgfpathlineto{\pgfqpoint{1.709528in}{2.338316in}}%
\pgfpathlineto{\pgfqpoint{1.732079in}{2.337088in}}%
\pgfpathlineto{\pgfqpoint{1.781476in}{2.336145in}}%
\pgfpathlineto{\pgfqpoint{1.792214in}{2.335464in}}%
\pgfpathlineto{\pgfqpoint{1.811543in}{2.335081in}}%
\pgfpathlineto{\pgfqpoint{1.849127in}{2.333613in}}%
\pgfpathlineto{\pgfqpoint{1.904967in}{2.332578in}}%
\pgfpathlineto{\pgfqpoint{1.942551in}{2.332170in}}%
\pgfpathlineto{\pgfqpoint{2.008056in}{2.331568in}}%
\pgfpathlineto{\pgfqpoint{2.025237in}{2.331559in}}%
\pgfpathlineto{\pgfqpoint{2.052083in}{2.331013in}}%
\pgfpathlineto{\pgfqpoint{2.055304in}{2.330744in}}%
\pgfpathlineto{\pgfqpoint{2.182017in}{2.329508in}}%
\pgfpathlineto{\pgfqpoint{2.972363in}{2.327870in}}%
\pgfpathlineto{\pgfqpoint{2.972363in}{2.327870in}}%
\pgfusepath{stroke}%
\end{pgfscope}%
\begin{pgfscope}%
\pgfsetrectcap%
\pgfsetmiterjoin%
\pgfsetlinewidth{0.803000pt}%
\definecolor{currentstroke}{rgb}{1.000000,1.000000,1.000000}%
\pgfsetstrokecolor{currentstroke}%
\pgfsetdash{}{0pt}%
\pgfpathmoveto{\pgfqpoint{0.506453in}{2.309648in}}%
\pgfpathlineto{\pgfqpoint{0.506453in}{2.710533in}}%
\pgfusepath{stroke}%
\end{pgfscope}%
\begin{pgfscope}%
\pgfsetrectcap%
\pgfsetmiterjoin%
\pgfsetlinewidth{0.803000pt}%
\definecolor{currentstroke}{rgb}{1.000000,1.000000,1.000000}%
\pgfsetstrokecolor{currentstroke}%
\pgfsetdash{}{0pt}%
\pgfpathmoveto{\pgfqpoint{3.089787in}{2.309648in}}%
\pgfpathlineto{\pgfqpoint{3.089787in}{2.710533in}}%
\pgfusepath{stroke}%
\end{pgfscope}%
\begin{pgfscope}%
\pgfsetrectcap%
\pgfsetmiterjoin%
\pgfsetlinewidth{0.803000pt}%
\definecolor{currentstroke}{rgb}{1.000000,1.000000,1.000000}%
\pgfsetstrokecolor{currentstroke}%
\pgfsetdash{}{0pt}%
\pgfpathmoveto{\pgfqpoint{0.506453in}{2.309648in}}%
\pgfpathlineto{\pgfqpoint{3.089787in}{2.309648in}}%
\pgfusepath{stroke}%
\end{pgfscope}%
\begin{pgfscope}%
\pgfsetrectcap%
\pgfsetmiterjoin%
\pgfsetlinewidth{0.803000pt}%
\definecolor{currentstroke}{rgb}{1.000000,1.000000,1.000000}%
\pgfsetstrokecolor{currentstroke}%
\pgfsetdash{}{0pt}%
\pgfpathmoveto{\pgfqpoint{0.506453in}{2.710533in}}%
\pgfpathlineto{\pgfqpoint{3.089787in}{2.710533in}}%
\pgfusepath{stroke}%
\end{pgfscope}%
\begin{pgfscope}%
\definecolor{textcolor}{rgb}{0.150000,0.150000,0.150000}%
\pgfsetstrokecolor{textcolor}%
\pgfsetfillcolor{textcolor}%
\pgftext[x=1.798120in,y=2.793866in,,base]{\color{textcolor}\rmfamily\fontsize{16.800000}{20.160000}\selectfont JNJ}%
\end{pgfscope}%
\begin{pgfscope}%
\pgfsetbuttcap%
\pgfsetmiterjoin%
\definecolor{currentfill}{rgb}{0.917647,0.917647,0.949020}%
\pgfsetfillcolor{currentfill}%
\pgfsetlinewidth{0.000000pt}%
\definecolor{currentstroke}{rgb}{0.000000,0.000000,0.000000}%
\pgfsetstrokecolor{currentstroke}%
\pgfsetstrokeopacity{0.000000}%
\pgfsetdash{}{0pt}%
\pgfpathmoveto{\pgfqpoint{4.123120in}{2.309648in}}%
\pgfpathlineto{\pgfqpoint{6.706453in}{2.309648in}}%
\pgfpathlineto{\pgfqpoint{6.706453in}{2.710533in}}%
\pgfpathlineto{\pgfqpoint{4.123120in}{2.710533in}}%
\pgfpathclose%
\pgfusepath{fill}%
\end{pgfscope}%
\begin{pgfscope}%
\pgfpathrectangle{\pgfqpoint{4.123120in}{2.309648in}}{\pgfqpoint{2.583333in}{0.400885in}}%
\pgfusepath{clip}%
\pgfsetroundcap%
\pgfsetroundjoin%
\pgfsetlinewidth{0.803000pt}%
\definecolor{currentstroke}{rgb}{1.000000,1.000000,1.000000}%
\pgfsetstrokecolor{currentstroke}%
\pgfsetdash{}{0pt}%
\pgfpathmoveto{\pgfqpoint{4.238397in}{2.309648in}}%
\pgfpathlineto{\pgfqpoint{4.238397in}{2.710533in}}%
\pgfusepath{stroke}%
\end{pgfscope}%
\begin{pgfscope}%
\definecolor{textcolor}{rgb}{0.150000,0.150000,0.150000}%
\pgfsetstrokecolor{textcolor}%
\pgfsetfillcolor{textcolor}%
\pgftext[x=4.238397in,y=2.212426in,,top]{\color{textcolor}\rmfamily\fontsize{14.000000}{16.800000}\selectfont 2012}%
\end{pgfscope}%
\begin{pgfscope}%
\pgfpathrectangle{\pgfqpoint{4.123120in}{2.309648in}}{\pgfqpoint{2.583333in}{0.400885in}}%
\pgfusepath{clip}%
\pgfsetroundcap%
\pgfsetroundjoin%
\pgfsetlinewidth{0.803000pt}%
\definecolor{currentstroke}{rgb}{1.000000,1.000000,1.000000}%
\pgfsetstrokecolor{currentstroke}%
\pgfsetdash{}{0pt}%
\pgfpathmoveto{\pgfqpoint{4.631422in}{2.309648in}}%
\pgfpathlineto{\pgfqpoint{4.631422in}{2.710533in}}%
\pgfusepath{stroke}%
\end{pgfscope}%
\begin{pgfscope}%
\definecolor{textcolor}{rgb}{0.150000,0.150000,0.150000}%
\pgfsetstrokecolor{textcolor}%
\pgfsetfillcolor{textcolor}%
\pgftext[x=4.631422in,y=2.212426in,,top]{\color{textcolor}\rmfamily\fontsize{14.000000}{16.800000}\selectfont 2013}%
\end{pgfscope}%
\begin{pgfscope}%
\pgfpathrectangle{\pgfqpoint{4.123120in}{2.309648in}}{\pgfqpoint{2.583333in}{0.400885in}}%
\pgfusepath{clip}%
\pgfsetroundcap%
\pgfsetroundjoin%
\pgfsetlinewidth{0.803000pt}%
\definecolor{currentstroke}{rgb}{1.000000,1.000000,1.000000}%
\pgfsetstrokecolor{currentstroke}%
\pgfsetdash{}{0pt}%
\pgfpathmoveto{\pgfqpoint{5.023373in}{2.309648in}}%
\pgfpathlineto{\pgfqpoint{5.023373in}{2.710533in}}%
\pgfusepath{stroke}%
\end{pgfscope}%
\begin{pgfscope}%
\definecolor{textcolor}{rgb}{0.150000,0.150000,0.150000}%
\pgfsetstrokecolor{textcolor}%
\pgfsetfillcolor{textcolor}%
\pgftext[x=5.023373in,y=2.212426in,,top]{\color{textcolor}\rmfamily\fontsize{14.000000}{16.800000}\selectfont 2014}%
\end{pgfscope}%
\begin{pgfscope}%
\pgfpathrectangle{\pgfqpoint{4.123120in}{2.309648in}}{\pgfqpoint{2.583333in}{0.400885in}}%
\pgfusepath{clip}%
\pgfsetroundcap%
\pgfsetroundjoin%
\pgfsetlinewidth{0.803000pt}%
\definecolor{currentstroke}{rgb}{1.000000,1.000000,1.000000}%
\pgfsetstrokecolor{currentstroke}%
\pgfsetdash{}{0pt}%
\pgfpathmoveto{\pgfqpoint{5.415324in}{2.309648in}}%
\pgfpathlineto{\pgfqpoint{5.415324in}{2.710533in}}%
\pgfusepath{stroke}%
\end{pgfscope}%
\begin{pgfscope}%
\definecolor{textcolor}{rgb}{0.150000,0.150000,0.150000}%
\pgfsetstrokecolor{textcolor}%
\pgfsetfillcolor{textcolor}%
\pgftext[x=5.415324in,y=2.212426in,,top]{\color{textcolor}\rmfamily\fontsize{14.000000}{16.800000}\selectfont 2015}%
\end{pgfscope}%
\begin{pgfscope}%
\pgfpathrectangle{\pgfqpoint{4.123120in}{2.309648in}}{\pgfqpoint{2.583333in}{0.400885in}}%
\pgfusepath{clip}%
\pgfsetroundcap%
\pgfsetroundjoin%
\pgfsetlinewidth{0.803000pt}%
\definecolor{currentstroke}{rgb}{1.000000,1.000000,1.000000}%
\pgfsetstrokecolor{currentstroke}%
\pgfsetdash{}{0pt}%
\pgfpathmoveto{\pgfqpoint{5.807275in}{2.309648in}}%
\pgfpathlineto{\pgfqpoint{5.807275in}{2.710533in}}%
\pgfusepath{stroke}%
\end{pgfscope}%
\begin{pgfscope}%
\definecolor{textcolor}{rgb}{0.150000,0.150000,0.150000}%
\pgfsetstrokecolor{textcolor}%
\pgfsetfillcolor{textcolor}%
\pgftext[x=5.807275in,y=2.212426in,,top]{\color{textcolor}\rmfamily\fontsize{14.000000}{16.800000}\selectfont 2016}%
\end{pgfscope}%
\begin{pgfscope}%
\pgfpathrectangle{\pgfqpoint{4.123120in}{2.309648in}}{\pgfqpoint{2.583333in}{0.400885in}}%
\pgfusepath{clip}%
\pgfsetroundcap%
\pgfsetroundjoin%
\pgfsetlinewidth{0.803000pt}%
\definecolor{currentstroke}{rgb}{1.000000,1.000000,1.000000}%
\pgfsetstrokecolor{currentstroke}%
\pgfsetdash{}{0pt}%
\pgfpathmoveto{\pgfqpoint{6.200300in}{2.309648in}}%
\pgfpathlineto{\pgfqpoint{6.200300in}{2.710533in}}%
\pgfusepath{stroke}%
\end{pgfscope}%
\begin{pgfscope}%
\definecolor{textcolor}{rgb}{0.150000,0.150000,0.150000}%
\pgfsetstrokecolor{textcolor}%
\pgfsetfillcolor{textcolor}%
\pgftext[x=6.200300in,y=2.212426in,,top]{\color{textcolor}\rmfamily\fontsize{14.000000}{16.800000}\selectfont 2017}%
\end{pgfscope}%
\begin{pgfscope}%
\pgfpathrectangle{\pgfqpoint{4.123120in}{2.309648in}}{\pgfqpoint{2.583333in}{0.400885in}}%
\pgfusepath{clip}%
\pgfsetroundcap%
\pgfsetroundjoin%
\pgfsetlinewidth{0.803000pt}%
\definecolor{currentstroke}{rgb}{1.000000,1.000000,1.000000}%
\pgfsetstrokecolor{currentstroke}%
\pgfsetdash{}{0pt}%
\pgfpathmoveto{\pgfqpoint{6.592251in}{2.309648in}}%
\pgfpathlineto{\pgfqpoint{6.592251in}{2.710533in}}%
\pgfusepath{stroke}%
\end{pgfscope}%
\begin{pgfscope}%
\definecolor{textcolor}{rgb}{0.150000,0.150000,0.150000}%
\pgfsetstrokecolor{textcolor}%
\pgfsetfillcolor{textcolor}%
\pgftext[x=6.592251in,y=2.212426in,,top]{\color{textcolor}\rmfamily\fontsize{14.000000}{16.800000}\selectfont 2018}%
\end{pgfscope}%
\begin{pgfscope}%
\pgfpathrectangle{\pgfqpoint{4.123120in}{2.309648in}}{\pgfqpoint{2.583333in}{0.400885in}}%
\pgfusepath{clip}%
\pgfsetroundcap%
\pgfsetroundjoin%
\pgfsetlinewidth{0.803000pt}%
\definecolor{currentstroke}{rgb}{1.000000,1.000000,1.000000}%
\pgfsetstrokecolor{currentstroke}%
\pgfsetdash{}{0pt}%
\pgfpathmoveto{\pgfqpoint{4.123120in}{2.326539in}}%
\pgfpathlineto{\pgfqpoint{6.706453in}{2.326539in}}%
\pgfusepath{stroke}%
\end{pgfscope}%
\begin{pgfscope}%
\definecolor{textcolor}{rgb}{0.150000,0.150000,0.150000}%
\pgfsetstrokecolor{textcolor}%
\pgfsetfillcolor{textcolor}%
\pgftext[x=3.902186in,y=2.252673in,left,base]{\color{textcolor}\rmfamily\fontsize{14.000000}{16.800000}\selectfont 0}%
\end{pgfscope}%
\begin{pgfscope}%
\pgfpathrectangle{\pgfqpoint{4.123120in}{2.309648in}}{\pgfqpoint{2.583333in}{0.400885in}}%
\pgfusepath{clip}%
\pgfsetroundcap%
\pgfsetroundjoin%
\pgfsetlinewidth{0.803000pt}%
\definecolor{currentstroke}{rgb}{1.000000,1.000000,1.000000}%
\pgfsetstrokecolor{currentstroke}%
\pgfsetdash{}{0pt}%
\pgfpathmoveto{\pgfqpoint{4.123120in}{2.542270in}}%
\pgfpathlineto{\pgfqpoint{6.706453in}{2.542270in}}%
\pgfusepath{stroke}%
\end{pgfscope}%
\begin{pgfscope}%
\definecolor{textcolor}{rgb}{0.150000,0.150000,0.150000}%
\pgfsetstrokecolor{textcolor}%
\pgfsetfillcolor{textcolor}%
\pgftext[x=3.902186in,y=2.468404in,left,base]{\color{textcolor}\rmfamily\fontsize{14.000000}{16.800000}\selectfont 1}%
\end{pgfscope}%
\begin{pgfscope}%
\pgfpathrectangle{\pgfqpoint{4.123120in}{2.309648in}}{\pgfqpoint{2.583333in}{0.400885in}}%
\pgfusepath{clip}%
\pgfsetroundcap%
\pgfsetroundjoin%
\pgfsetlinewidth{1.505625pt}%
\definecolor{currentstroke}{rgb}{0.000000,0.000000,0.000000}%
\pgfsetstrokecolor{currentstroke}%
\pgfsetdash{}{0pt}%
\pgfpathmoveto{\pgfqpoint{4.240544in}{2.542270in}}%
\pgfpathlineto{\pgfqpoint{4.241618in}{2.542146in}}%
\pgfpathlineto{\pgfqpoint{4.243766in}{2.540742in}}%
\pgfpathlineto{\pgfqpoint{4.246987in}{2.541650in}}%
\pgfpathlineto{\pgfqpoint{4.248061in}{2.540618in}}%
\pgfpathlineto{\pgfqpoint{4.249135in}{2.538554in}}%
\pgfpathlineto{\pgfqpoint{4.251283in}{2.538967in}}%
\pgfpathlineto{\pgfqpoint{4.255578in}{2.540412in}}%
\pgfpathlineto{\pgfqpoint{4.256652in}{2.541361in}}%
\pgfpathlineto{\pgfqpoint{4.258800in}{2.542022in}}%
\pgfpathlineto{\pgfqpoint{4.263095in}{2.536407in}}%
\pgfpathlineto{\pgfqpoint{4.264169in}{2.537976in}}%
\pgfpathlineto{\pgfqpoint{4.265243in}{2.537356in}}%
\pgfpathlineto{\pgfqpoint{4.266317in}{2.535746in}}%
\pgfpathlineto{\pgfqpoint{4.270612in}{2.531659in}}%
\pgfpathlineto{\pgfqpoint{4.272760in}{2.532567in}}%
\pgfpathlineto{\pgfqpoint{4.273833in}{2.530750in}}%
\pgfpathlineto{\pgfqpoint{4.278129in}{2.533806in}}%
\pgfpathlineto{\pgfqpoint{4.279203in}{2.533599in}}%
\pgfpathlineto{\pgfqpoint{4.280276in}{2.534879in}}%
\pgfpathlineto{\pgfqpoint{4.281350in}{2.534384in}}%
\pgfpathlineto{\pgfqpoint{4.286719in}{2.536572in}}%
\pgfpathlineto{\pgfqpoint{4.287793in}{2.538678in}}%
\pgfpathlineto{\pgfqpoint{4.288867in}{2.537728in}}%
\pgfpathlineto{\pgfqpoint{4.294236in}{2.536200in}}%
\pgfpathlineto{\pgfqpoint{4.295310in}{2.542641in}}%
\pgfpathlineto{\pgfqpoint{4.296384in}{2.543591in}}%
\pgfpathlineto{\pgfqpoint{4.299606in}{2.543550in}}%
\pgfpathlineto{\pgfqpoint{4.300679in}{2.545779in}}%
\pgfpathlineto{\pgfqpoint{4.301753in}{2.546564in}}%
\pgfpathlineto{\pgfqpoint{4.302827in}{2.543426in}}%
\pgfpathlineto{\pgfqpoint{4.303901in}{2.543467in}}%
\pgfpathlineto{\pgfqpoint{4.307122in}{2.544375in}}%
\pgfpathlineto{\pgfqpoint{4.309270in}{2.543178in}}%
\pgfpathlineto{\pgfqpoint{4.310344in}{2.544210in}}%
\pgfpathlineto{\pgfqpoint{4.311418in}{2.544293in}}%
\pgfpathlineto{\pgfqpoint{4.315713in}{2.547472in}}%
\pgfpathlineto{\pgfqpoint{4.317861in}{2.546729in}}%
\pgfpathlineto{\pgfqpoint{4.318935in}{2.545325in}}%
\pgfpathlineto{\pgfqpoint{4.324304in}{2.545160in}}%
\pgfpathlineto{\pgfqpoint{4.325378in}{2.546233in}}%
\pgfpathlineto{\pgfqpoint{4.326451in}{2.545944in}}%
\pgfpathlineto{\pgfqpoint{4.329673in}{2.546027in}}%
\pgfpathlineto{\pgfqpoint{4.330747in}{2.545036in}}%
\pgfpathlineto{\pgfqpoint{4.331821in}{2.545160in}}%
\pgfpathlineto{\pgfqpoint{4.332894in}{2.544541in}}%
\pgfpathlineto{\pgfqpoint{4.333968in}{2.545201in}}%
\pgfpathlineto{\pgfqpoint{4.337190in}{2.546357in}}%
\pgfpathlineto{\pgfqpoint{4.338264in}{2.544830in}}%
\pgfpathlineto{\pgfqpoint{4.340411in}{2.545531in}}%
\pgfpathlineto{\pgfqpoint{4.344707in}{2.543921in}}%
\pgfpathlineto{\pgfqpoint{4.345781in}{2.542394in}}%
\pgfpathlineto{\pgfqpoint{4.346854in}{2.542724in}}%
\pgfpathlineto{\pgfqpoint{4.349002in}{2.540659in}}%
\pgfpathlineto{\pgfqpoint{4.353297in}{2.544582in}}%
\pgfpathlineto{\pgfqpoint{4.355445in}{2.543137in}}%
\pgfpathlineto{\pgfqpoint{4.356519in}{2.546192in}}%
\pgfpathlineto{\pgfqpoint{4.359740in}{2.543384in}}%
\pgfpathlineto{\pgfqpoint{4.361888in}{2.546027in}}%
\pgfpathlineto{\pgfqpoint{4.362962in}{2.545944in}}%
\pgfpathlineto{\pgfqpoint{4.364036in}{2.537976in}}%
\pgfpathlineto{\pgfqpoint{4.368331in}{2.535127in}}%
\pgfpathlineto{\pgfqpoint{4.370479in}{2.538182in}}%
\pgfpathlineto{\pgfqpoint{4.371553in}{2.537439in}}%
\pgfpathlineto{\pgfqpoint{4.375848in}{2.537067in}}%
\pgfpathlineto{\pgfqpoint{4.376922in}{2.535457in}}%
\pgfpathlineto{\pgfqpoint{4.377996in}{2.536985in}}%
\pgfpathlineto{\pgfqpoint{4.379070in}{2.535498in}}%
\pgfpathlineto{\pgfqpoint{4.382291in}{2.535168in}}%
\pgfpathlineto{\pgfqpoint{4.383365in}{2.535622in}}%
\pgfpathlineto{\pgfqpoint{4.384439in}{2.537480in}}%
\pgfpathlineto{\pgfqpoint{4.386586in}{2.534962in}}%
\pgfpathlineto{\pgfqpoint{4.389808in}{2.534508in}}%
\pgfpathlineto{\pgfqpoint{4.390882in}{2.533723in}}%
\pgfpathlineto{\pgfqpoint{4.391956in}{2.531246in}}%
\pgfpathlineto{\pgfqpoint{4.393029in}{2.531824in}}%
\pgfpathlineto{\pgfqpoint{4.394103in}{2.531576in}}%
\pgfpathlineto{\pgfqpoint{4.398399in}{2.533104in}}%
\pgfpathlineto{\pgfqpoint{4.399472in}{2.530998in}}%
\pgfpathlineto{\pgfqpoint{4.400546in}{2.530915in}}%
\pgfpathlineto{\pgfqpoint{4.401620in}{2.528479in}}%
\pgfpathlineto{\pgfqpoint{4.404842in}{2.527984in}}%
\pgfpathlineto{\pgfqpoint{4.405916in}{2.527241in}}%
\pgfpathlineto{\pgfqpoint{4.409137in}{2.532443in}}%
\pgfpathlineto{\pgfqpoint{4.412359in}{2.531741in}}%
\pgfpathlineto{\pgfqpoint{4.413432in}{2.532443in}}%
\pgfpathlineto{\pgfqpoint{4.414506in}{2.531824in}}%
\pgfpathlineto{\pgfqpoint{4.415580in}{2.533847in}}%
\pgfpathlineto{\pgfqpoint{4.416654in}{2.532856in}}%
\pgfpathlineto{\pgfqpoint{4.420949in}{2.530668in}}%
\pgfpathlineto{\pgfqpoint{4.422023in}{2.524681in}}%
\pgfpathlineto{\pgfqpoint{4.423097in}{2.522575in}}%
\pgfpathlineto{\pgfqpoint{4.424171in}{2.522864in}}%
\pgfpathlineto{\pgfqpoint{4.428466in}{2.521006in}}%
\pgfpathlineto{\pgfqpoint{4.429540in}{2.523360in}}%
\pgfpathlineto{\pgfqpoint{4.430614in}{2.524351in}}%
\pgfpathlineto{\pgfqpoint{4.431688in}{2.527489in}}%
\pgfpathlineto{\pgfqpoint{4.434909in}{2.527323in}}%
\pgfpathlineto{\pgfqpoint{4.435983in}{2.527860in}}%
\pgfpathlineto{\pgfqpoint{4.439205in}{2.527612in}}%
\pgfpathlineto{\pgfqpoint{4.443500in}{2.529099in}}%
\pgfpathlineto{\pgfqpoint{4.444574in}{2.527984in}}%
\pgfpathlineto{\pgfqpoint{4.446721in}{2.540123in}}%
\pgfpathlineto{\pgfqpoint{4.449943in}{2.539173in}}%
\pgfpathlineto{\pgfqpoint{4.451017in}{2.540948in}}%
\pgfpathlineto{\pgfqpoint{4.454238in}{2.540783in}}%
\pgfpathlineto{\pgfqpoint{4.457460in}{2.539627in}}%
\pgfpathlineto{\pgfqpoint{4.458534in}{2.538430in}}%
\pgfpathlineto{\pgfqpoint{4.459607in}{2.538430in}}%
\pgfpathlineto{\pgfqpoint{4.461755in}{2.541939in}}%
\pgfpathlineto{\pgfqpoint{4.464977in}{2.541981in}}%
\pgfpathlineto{\pgfqpoint{4.468198in}{2.536737in}}%
\pgfpathlineto{\pgfqpoint{4.469272in}{2.543302in}}%
\pgfpathlineto{\pgfqpoint{4.472494in}{2.544334in}}%
\pgfpathlineto{\pgfqpoint{4.474641in}{2.547348in}}%
\pgfpathlineto{\pgfqpoint{4.476789in}{2.547513in}}%
\pgfpathlineto{\pgfqpoint{4.480010in}{2.546564in}}%
\pgfpathlineto{\pgfqpoint{4.481084in}{2.547389in}}%
\pgfpathlineto{\pgfqpoint{4.482158in}{2.547100in}}%
\pgfpathlineto{\pgfqpoint{4.483232in}{2.548298in}}%
\pgfpathlineto{\pgfqpoint{4.484306in}{2.548298in}}%
\pgfpathlineto{\pgfqpoint{4.488601in}{2.547513in}}%
\pgfpathlineto{\pgfqpoint{4.489675in}{2.547802in}}%
\pgfpathlineto{\pgfqpoint{4.490749in}{2.547224in}}%
\pgfpathlineto{\pgfqpoint{4.491823in}{2.548339in}}%
\pgfpathlineto{\pgfqpoint{4.496118in}{2.548256in}}%
\pgfpathlineto{\pgfqpoint{4.498266in}{2.547885in}}%
\pgfpathlineto{\pgfqpoint{4.499339in}{2.548917in}}%
\pgfpathlineto{\pgfqpoint{4.504709in}{2.549247in}}%
\pgfpathlineto{\pgfqpoint{4.505782in}{2.552385in}}%
\pgfpathlineto{\pgfqpoint{4.506856in}{2.553294in}}%
\pgfpathlineto{\pgfqpoint{4.510078in}{2.553294in}}%
\pgfpathlineto{\pgfqpoint{4.512226in}{2.551931in}}%
\pgfpathlineto{\pgfqpoint{4.513299in}{2.554615in}}%
\pgfpathlineto{\pgfqpoint{4.514373in}{2.555441in}}%
\pgfpathlineto{\pgfqpoint{4.519742in}{2.555771in}}%
\pgfpathlineto{\pgfqpoint{4.520816in}{2.556762in}}%
\pgfpathlineto{\pgfqpoint{4.521890in}{2.556308in}}%
\pgfpathlineto{\pgfqpoint{4.525112in}{2.557422in}}%
\pgfpathlineto{\pgfqpoint{4.528333in}{2.555895in}}%
\pgfpathlineto{\pgfqpoint{4.532628in}{2.556390in}}%
\pgfpathlineto{\pgfqpoint{4.533702in}{2.554202in}}%
\pgfpathlineto{\pgfqpoint{4.535850in}{2.556143in}}%
\pgfpathlineto{\pgfqpoint{4.536924in}{2.556968in}}%
\pgfpathlineto{\pgfqpoint{4.540145in}{2.555234in}}%
\pgfpathlineto{\pgfqpoint{4.543367in}{2.551601in}}%
\pgfpathlineto{\pgfqpoint{4.544441in}{2.551394in}}%
\pgfpathlineto{\pgfqpoint{4.548736in}{2.554904in}}%
\pgfpathlineto{\pgfqpoint{4.549810in}{2.558331in}}%
\pgfpathlineto{\pgfqpoint{4.550884in}{2.558331in}}%
\pgfpathlineto{\pgfqpoint{4.551958in}{2.555358in}}%
\pgfpathlineto{\pgfqpoint{4.555179in}{2.555110in}}%
\pgfpathlineto{\pgfqpoint{4.556253in}{2.551560in}}%
\pgfpathlineto{\pgfqpoint{4.557327in}{2.553707in}}%
\pgfpathlineto{\pgfqpoint{4.558401in}{2.560354in}}%
\pgfpathlineto{\pgfqpoint{4.559474in}{2.558248in}}%
\pgfpathlineto{\pgfqpoint{4.566991in}{2.557422in}}%
\pgfpathlineto{\pgfqpoint{4.570213in}{2.555688in}}%
\pgfpathlineto{\pgfqpoint{4.571287in}{2.556555in}}%
\pgfpathlineto{\pgfqpoint{4.573434in}{2.549784in}}%
\pgfpathlineto{\pgfqpoint{4.574508in}{2.550114in}}%
\pgfpathlineto{\pgfqpoint{4.577730in}{2.550362in}}%
\pgfpathlineto{\pgfqpoint{4.580951in}{2.547844in}}%
\pgfpathlineto{\pgfqpoint{4.582025in}{2.549495in}}%
\pgfpathlineto{\pgfqpoint{4.589542in}{2.558744in}}%
\pgfpathlineto{\pgfqpoint{4.592763in}{2.558372in}}%
\pgfpathlineto{\pgfqpoint{4.593837in}{2.556721in}}%
\pgfpathlineto{\pgfqpoint{4.594911in}{2.558248in}}%
\pgfpathlineto{\pgfqpoint{4.595985in}{2.558455in}}%
\pgfpathlineto{\pgfqpoint{4.597059in}{2.559528in}}%
\pgfpathlineto{\pgfqpoint{4.600280in}{2.558702in}}%
\pgfpathlineto{\pgfqpoint{4.601354in}{2.557794in}}%
\pgfpathlineto{\pgfqpoint{4.602428in}{2.558124in}}%
\pgfpathlineto{\pgfqpoint{4.604576in}{2.561097in}}%
\pgfpathlineto{\pgfqpoint{4.607797in}{2.560849in}}%
\pgfpathlineto{\pgfqpoint{4.608871in}{2.562294in}}%
\pgfpathlineto{\pgfqpoint{4.609945in}{2.562666in}}%
\pgfpathlineto{\pgfqpoint{4.611019in}{2.560560in}}%
\pgfpathlineto{\pgfqpoint{4.612093in}{2.559858in}}%
\pgfpathlineto{\pgfqpoint{4.616388in}{2.560024in}}%
\pgfpathlineto{\pgfqpoint{4.617462in}{2.557918in}}%
\pgfpathlineto{\pgfqpoint{4.618536in}{2.559528in}}%
\pgfpathlineto{\pgfqpoint{4.619609in}{2.555853in}}%
\pgfpathlineto{\pgfqpoint{4.622831in}{2.555152in}}%
\pgfpathlineto{\pgfqpoint{4.624979in}{2.553417in}}%
\pgfpathlineto{\pgfqpoint{4.626052in}{2.553335in}}%
\pgfpathlineto{\pgfqpoint{4.627126in}{2.550610in}}%
\pgfpathlineto{\pgfqpoint{4.630348in}{2.553087in}}%
\pgfpathlineto{\pgfqpoint{4.632495in}{2.558083in}}%
\pgfpathlineto{\pgfqpoint{4.633569in}{2.556597in}}%
\pgfpathlineto{\pgfqpoint{4.634643in}{2.557092in}}%
\pgfpathlineto{\pgfqpoint{4.638938in}{2.555152in}}%
\pgfpathlineto{\pgfqpoint{4.641086in}{2.557670in}}%
\pgfpathlineto{\pgfqpoint{4.642160in}{2.557505in}}%
\pgfpathlineto{\pgfqpoint{4.647529in}{2.559776in}}%
\pgfpathlineto{\pgfqpoint{4.649677in}{2.561799in}}%
\pgfpathlineto{\pgfqpoint{4.653972in}{2.561840in}}%
\pgfpathlineto{\pgfqpoint{4.655046in}{2.564318in}}%
\pgfpathlineto{\pgfqpoint{4.656120in}{2.563409in}}%
\pgfpathlineto{\pgfqpoint{4.657194in}{2.572947in}}%
\pgfpathlineto{\pgfqpoint{4.660415in}{2.574681in}}%
\pgfpathlineto{\pgfqpoint{4.661489in}{2.578810in}}%
\pgfpathlineto{\pgfqpoint{4.663637in}{2.579346in}}%
\pgfpathlineto{\pgfqpoint{4.664711in}{2.581906in}}%
\pgfpathlineto{\pgfqpoint{4.667932in}{2.579677in}}%
\pgfpathlineto{\pgfqpoint{4.670080in}{2.582691in}}%
\pgfpathlineto{\pgfqpoint{4.671154in}{2.582691in}}%
\pgfpathlineto{\pgfqpoint{4.672227in}{2.581328in}}%
\pgfpathlineto{\pgfqpoint{4.676523in}{2.582113in}}%
\pgfpathlineto{\pgfqpoint{4.677597in}{2.584053in}}%
\pgfpathlineto{\pgfqpoint{4.678670in}{2.584797in}}%
\pgfpathlineto{\pgfqpoint{4.679744in}{2.584012in}}%
\pgfpathlineto{\pgfqpoint{4.684040in}{2.586820in}}%
\pgfpathlineto{\pgfqpoint{4.685114in}{2.585829in}}%
\pgfpathlineto{\pgfqpoint{4.687261in}{2.585498in}}%
\pgfpathlineto{\pgfqpoint{4.690483in}{2.581906in}}%
\pgfpathlineto{\pgfqpoint{4.691557in}{2.582443in}}%
\pgfpathlineto{\pgfqpoint{4.692630in}{2.584714in}}%
\pgfpathlineto{\pgfqpoint{4.693704in}{2.582773in}}%
\pgfpathlineto{\pgfqpoint{4.694778in}{2.583847in}}%
\pgfpathlineto{\pgfqpoint{4.698000in}{2.584466in}}%
\pgfpathlineto{\pgfqpoint{4.700147in}{2.586242in}}%
\pgfpathlineto{\pgfqpoint{4.701221in}{2.585209in}}%
\pgfpathlineto{\pgfqpoint{4.702295in}{2.586159in}}%
\pgfpathlineto{\pgfqpoint{4.705516in}{2.586737in}}%
\pgfpathlineto{\pgfqpoint{4.707664in}{2.584879in}}%
\pgfpathlineto{\pgfqpoint{4.708738in}{2.586861in}}%
\pgfpathlineto{\pgfqpoint{4.709812in}{2.583310in}}%
\pgfpathlineto{\pgfqpoint{4.713033in}{2.582732in}}%
\pgfpathlineto{\pgfqpoint{4.715181in}{2.587522in}}%
\pgfpathlineto{\pgfqpoint{4.716255in}{2.586242in}}%
\pgfpathlineto{\pgfqpoint{4.717329in}{2.586448in}}%
\pgfpathlineto{\pgfqpoint{4.720550in}{2.584466in}}%
\pgfpathlineto{\pgfqpoint{4.721624in}{2.586902in}}%
\pgfpathlineto{\pgfqpoint{4.722698in}{2.585746in}}%
\pgfpathlineto{\pgfqpoint{4.723772in}{2.585746in}}%
\pgfpathlineto{\pgfqpoint{4.728067in}{2.587893in}}%
\pgfpathlineto{\pgfqpoint{4.729141in}{2.592146in}}%
\pgfpathlineto{\pgfqpoint{4.730215in}{2.589338in}}%
\pgfpathlineto{\pgfqpoint{4.731289in}{2.590742in}}%
\pgfpathlineto{\pgfqpoint{4.732362in}{2.589669in}}%
\pgfpathlineto{\pgfqpoint{4.735584in}{2.591568in}}%
\pgfpathlineto{\pgfqpoint{4.736658in}{2.589792in}}%
\pgfpathlineto{\pgfqpoint{4.737732in}{2.593095in}}%
\pgfpathlineto{\pgfqpoint{4.739879in}{2.595903in}}%
\pgfpathlineto{\pgfqpoint{4.743101in}{2.594458in}}%
\pgfpathlineto{\pgfqpoint{4.744175in}{2.595986in}}%
\pgfpathlineto{\pgfqpoint{4.745248in}{2.592476in}}%
\pgfpathlineto{\pgfqpoint{4.746322in}{2.595201in}}%
\pgfpathlineto{\pgfqpoint{4.747396in}{2.600445in}}%
\pgfpathlineto{\pgfqpoint{4.750618in}{2.600362in}}%
\pgfpathlineto{\pgfqpoint{4.751692in}{2.604202in}}%
\pgfpathlineto{\pgfqpoint{4.752765in}{2.587852in}}%
\pgfpathlineto{\pgfqpoint{4.753839in}{2.586035in}}%
\pgfpathlineto{\pgfqpoint{4.754913in}{2.587811in}}%
\pgfpathlineto{\pgfqpoint{4.758135in}{2.589751in}}%
\pgfpathlineto{\pgfqpoint{4.759208in}{2.586654in}}%
\pgfpathlineto{\pgfqpoint{4.760282in}{2.587398in}}%
\pgfpathlineto{\pgfqpoint{4.762430in}{2.591485in}}%
\pgfpathlineto{\pgfqpoint{4.765651in}{2.589999in}}%
\pgfpathlineto{\pgfqpoint{4.766725in}{2.590618in}}%
\pgfpathlineto{\pgfqpoint{4.767799in}{2.592311in}}%
\pgfpathlineto{\pgfqpoint{4.768873in}{2.591568in}}%
\pgfpathlineto{\pgfqpoint{4.769947in}{2.593426in}}%
\pgfpathlineto{\pgfqpoint{4.773168in}{2.592848in}}%
\pgfpathlineto{\pgfqpoint{4.775316in}{2.599908in}}%
\pgfpathlineto{\pgfqpoint{4.777464in}{2.597678in}}%
\pgfpathlineto{\pgfqpoint{4.781759in}{2.593550in}}%
\pgfpathlineto{\pgfqpoint{4.783907in}{2.593219in}}%
\pgfpathlineto{\pgfqpoint{4.784981in}{2.603996in}}%
\pgfpathlineto{\pgfqpoint{4.789276in}{2.600527in}}%
\pgfpathlineto{\pgfqpoint{4.790350in}{2.593880in}}%
\pgfpathlineto{\pgfqpoint{4.791424in}{2.594541in}}%
\pgfpathlineto{\pgfqpoint{4.792497in}{2.586654in}}%
\pgfpathlineto{\pgfqpoint{4.795719in}{2.589669in}}%
\pgfpathlineto{\pgfqpoint{4.796793in}{2.588719in}}%
\pgfpathlineto{\pgfqpoint{4.797867in}{2.586283in}}%
\pgfpathlineto{\pgfqpoint{4.798940in}{2.586861in}}%
\pgfpathlineto{\pgfqpoint{4.800014in}{2.589999in}}%
\pgfpathlineto{\pgfqpoint{4.804310in}{2.591237in}}%
\pgfpathlineto{\pgfqpoint{4.805383in}{2.589503in}}%
\pgfpathlineto{\pgfqpoint{4.806457in}{2.592311in}}%
\pgfpathlineto{\pgfqpoint{4.807531in}{2.590948in}}%
\pgfpathlineto{\pgfqpoint{4.811826in}{2.594375in}}%
\pgfpathlineto{\pgfqpoint{4.812900in}{2.589503in}}%
\pgfpathlineto{\pgfqpoint{4.813974in}{2.581535in}}%
\pgfpathlineto{\pgfqpoint{4.815048in}{2.588925in}}%
\pgfpathlineto{\pgfqpoint{4.818270in}{2.586035in}}%
\pgfpathlineto{\pgfqpoint{4.819343in}{2.586407in}}%
\pgfpathlineto{\pgfqpoint{4.820417in}{2.588801in}}%
\pgfpathlineto{\pgfqpoint{4.821491in}{2.589710in}}%
\pgfpathlineto{\pgfqpoint{4.822565in}{2.587398in}}%
\pgfpathlineto{\pgfqpoint{4.827934in}{2.592765in}}%
\pgfpathlineto{\pgfqpoint{4.830082in}{2.591981in}}%
\pgfpathlineto{\pgfqpoint{4.833303in}{2.593426in}}%
\pgfpathlineto{\pgfqpoint{4.834377in}{2.596151in}}%
\pgfpathlineto{\pgfqpoint{4.835451in}{2.596935in}}%
\pgfpathlineto{\pgfqpoint{4.837599in}{2.602881in}}%
\pgfpathlineto{\pgfqpoint{4.840820in}{2.602716in}}%
\pgfpathlineto{\pgfqpoint{4.842968in}{2.599949in}}%
\pgfpathlineto{\pgfqpoint{4.844042in}{2.600651in}}%
\pgfpathlineto{\pgfqpoint{4.845115in}{2.604326in}}%
\pgfpathlineto{\pgfqpoint{4.848337in}{2.603830in}}%
\pgfpathlineto{\pgfqpoint{4.849411in}{2.602881in}}%
\pgfpathlineto{\pgfqpoint{4.850485in}{2.600569in}}%
\pgfpathlineto{\pgfqpoint{4.852632in}{2.600981in}}%
\pgfpathlineto{\pgfqpoint{4.855854in}{2.599949in}}%
\pgfpathlineto{\pgfqpoint{4.856928in}{2.601147in}}%
\pgfpathlineto{\pgfqpoint{4.858002in}{2.600651in}}%
\pgfpathlineto{\pgfqpoint{4.859075in}{2.605234in}}%
\pgfpathlineto{\pgfqpoint{4.860149in}{2.604037in}}%
\pgfpathlineto{\pgfqpoint{4.863371in}{2.604408in}}%
\pgfpathlineto{\pgfqpoint{4.865518in}{2.606349in}}%
\pgfpathlineto{\pgfqpoint{4.866592in}{2.607051in}}%
\pgfpathlineto{\pgfqpoint{4.867666in}{2.605234in}}%
\pgfpathlineto{\pgfqpoint{4.871961in}{2.605317in}}%
\pgfpathlineto{\pgfqpoint{4.873035in}{2.603913in}}%
\pgfpathlineto{\pgfqpoint{4.875183in}{2.599289in}}%
\pgfpathlineto{\pgfqpoint{4.880552in}{2.597513in}}%
\pgfpathlineto{\pgfqpoint{4.882700in}{2.599660in}}%
\pgfpathlineto{\pgfqpoint{4.886995in}{2.592724in}}%
\pgfpathlineto{\pgfqpoint{4.888069in}{2.588884in}}%
\pgfpathlineto{\pgfqpoint{4.890217in}{2.592435in}}%
\pgfpathlineto{\pgfqpoint{4.894512in}{2.591981in}}%
\pgfpathlineto{\pgfqpoint{4.897734in}{2.589916in}}%
\pgfpathlineto{\pgfqpoint{4.900955in}{2.593384in}}%
\pgfpathlineto{\pgfqpoint{4.902029in}{2.592641in}}%
\pgfpathlineto{\pgfqpoint{4.903103in}{2.593756in}}%
\pgfpathlineto{\pgfqpoint{4.904177in}{2.593715in}}%
\pgfpathlineto{\pgfqpoint{4.905250in}{2.596399in}}%
\pgfpathlineto{\pgfqpoint{4.908472in}{2.600197in}}%
\pgfpathlineto{\pgfqpoint{4.909546in}{2.599082in}}%
\pgfpathlineto{\pgfqpoint{4.910620in}{2.600651in}}%
\pgfpathlineto{\pgfqpoint{4.911693in}{2.600073in}}%
\pgfpathlineto{\pgfqpoint{4.912767in}{2.597555in}}%
\pgfpathlineto{\pgfqpoint{4.915989in}{2.597183in}}%
\pgfpathlineto{\pgfqpoint{4.918136in}{2.591857in}}%
\pgfpathlineto{\pgfqpoint{4.919210in}{2.592972in}}%
\pgfpathlineto{\pgfqpoint{4.920284in}{2.590123in}}%
\pgfpathlineto{\pgfqpoint{4.923506in}{2.584590in}}%
\pgfpathlineto{\pgfqpoint{4.924580in}{2.586531in}}%
\pgfpathlineto{\pgfqpoint{4.926727in}{2.585457in}}%
\pgfpathlineto{\pgfqpoint{4.927801in}{2.586076in}}%
\pgfpathlineto{\pgfqpoint{4.931023in}{2.584797in}}%
\pgfpathlineto{\pgfqpoint{4.935318in}{2.594458in}}%
\pgfpathlineto{\pgfqpoint{4.938539in}{2.595325in}}%
\pgfpathlineto{\pgfqpoint{4.939613in}{2.591444in}}%
\pgfpathlineto{\pgfqpoint{4.941761in}{2.599784in}}%
\pgfpathlineto{\pgfqpoint{4.942835in}{2.599743in}}%
\pgfpathlineto{\pgfqpoint{4.946056in}{2.598256in}}%
\pgfpathlineto{\pgfqpoint{4.947130in}{2.603087in}}%
\pgfpathlineto{\pgfqpoint{4.948204in}{2.604904in}}%
\pgfpathlineto{\pgfqpoint{4.949278in}{2.603872in}}%
\pgfpathlineto{\pgfqpoint{4.950352in}{2.601766in}}%
\pgfpathlineto{\pgfqpoint{4.953573in}{2.606266in}}%
\pgfpathlineto{\pgfqpoint{4.954647in}{2.610230in}}%
\pgfpathlineto{\pgfqpoint{4.956795in}{2.604367in}}%
\pgfpathlineto{\pgfqpoint{4.957869in}{2.605730in}}%
\pgfpathlineto{\pgfqpoint{4.962164in}{2.606679in}}%
\pgfpathlineto{\pgfqpoint{4.963238in}{2.611427in}}%
\pgfpathlineto{\pgfqpoint{4.964312in}{2.609776in}}%
\pgfpathlineto{\pgfqpoint{4.965385in}{2.610395in}}%
\pgfpathlineto{\pgfqpoint{4.968607in}{2.609569in}}%
\pgfpathlineto{\pgfqpoint{4.970755in}{2.613822in}}%
\pgfpathlineto{\pgfqpoint{4.972902in}{2.618446in}}%
\pgfpathlineto{\pgfqpoint{4.977198in}{2.616712in}}%
\pgfpathlineto{\pgfqpoint{4.978271in}{2.617910in}}%
\pgfpathlineto{\pgfqpoint{4.979345in}{2.617827in}}%
\pgfpathlineto{\pgfqpoint{4.980419in}{2.618818in}}%
\pgfpathlineto{\pgfqpoint{4.983641in}{2.620387in}}%
\pgfpathlineto{\pgfqpoint{4.985788in}{2.616506in}}%
\pgfpathlineto{\pgfqpoint{4.987936in}{2.616299in}}%
\pgfpathlineto{\pgfqpoint{4.991158in}{2.613285in}}%
\pgfpathlineto{\pgfqpoint{4.992231in}{2.614937in}}%
\pgfpathlineto{\pgfqpoint{4.994379in}{2.611015in}}%
\pgfpathlineto{\pgfqpoint{4.995453in}{2.617332in}}%
\pgfpathlineto{\pgfqpoint{4.998674in}{2.618240in}}%
\pgfpathlineto{\pgfqpoint{4.999748in}{2.614318in}}%
\pgfpathlineto{\pgfqpoint{5.000822in}{2.615598in}}%
\pgfpathlineto{\pgfqpoint{5.001896in}{2.609693in}}%
\pgfpathlineto{\pgfqpoint{5.002970in}{2.609941in}}%
\pgfpathlineto{\pgfqpoint{5.006191in}{2.607588in}}%
\pgfpathlineto{\pgfqpoint{5.007265in}{2.604904in}}%
\pgfpathlineto{\pgfqpoint{5.008339in}{2.609982in}}%
\pgfpathlineto{\pgfqpoint{5.009413in}{2.608331in}}%
\pgfpathlineto{\pgfqpoint{5.010487in}{2.608124in}}%
\pgfpathlineto{\pgfqpoint{5.014782in}{2.606266in}}%
\pgfpathlineto{\pgfqpoint{5.018003in}{2.608702in}}%
\pgfpathlineto{\pgfqpoint{5.021225in}{2.608661in}}%
\pgfpathlineto{\pgfqpoint{5.024447in}{2.603624in}}%
\pgfpathlineto{\pgfqpoint{5.025520in}{2.603335in}}%
\pgfpathlineto{\pgfqpoint{5.028742in}{2.603996in}}%
\pgfpathlineto{\pgfqpoint{5.029816in}{2.606679in}}%
\pgfpathlineto{\pgfqpoint{5.030890in}{2.602592in}}%
\pgfpathlineto{\pgfqpoint{5.031963in}{2.603211in}}%
\pgfpathlineto{\pgfqpoint{5.036259in}{2.601807in}}%
\pgfpathlineto{\pgfqpoint{5.037333in}{2.604780in}}%
\pgfpathlineto{\pgfqpoint{5.039480in}{2.603707in}}%
\pgfpathlineto{\pgfqpoint{5.040554in}{2.601353in}}%
\pgfpathlineto{\pgfqpoint{5.044849in}{2.602385in}}%
\pgfpathlineto{\pgfqpoint{5.045923in}{2.601188in}}%
\pgfpathlineto{\pgfqpoint{5.046997in}{2.597761in}}%
\pgfpathlineto{\pgfqpoint{5.048071in}{2.601023in}}%
\pgfpathlineto{\pgfqpoint{5.051292in}{2.598545in}}%
\pgfpathlineto{\pgfqpoint{5.052366in}{2.600775in}}%
\pgfpathlineto{\pgfqpoint{5.054514in}{2.593013in}}%
\pgfpathlineto{\pgfqpoint{5.058809in}{2.588967in}}%
\pgfpathlineto{\pgfqpoint{5.063105in}{2.594541in}}%
\pgfpathlineto{\pgfqpoint{5.066326in}{2.597018in}}%
\pgfpathlineto{\pgfqpoint{5.067400in}{2.599825in}}%
\pgfpathlineto{\pgfqpoint{5.068474in}{2.595160in}}%
\pgfpathlineto{\pgfqpoint{5.069548in}{2.596233in}}%
\pgfpathlineto{\pgfqpoint{5.070622in}{2.601766in}}%
\pgfpathlineto{\pgfqpoint{5.074917in}{2.596811in}}%
\pgfpathlineto{\pgfqpoint{5.075991in}{2.597431in}}%
\pgfpathlineto{\pgfqpoint{5.077065in}{2.596646in}}%
\pgfpathlineto{\pgfqpoint{5.078138in}{2.596811in}}%
\pgfpathlineto{\pgfqpoint{5.081360in}{2.596440in}}%
\pgfpathlineto{\pgfqpoint{5.082434in}{2.597348in}}%
\pgfpathlineto{\pgfqpoint{5.083508in}{2.596440in}}%
\pgfpathlineto{\pgfqpoint{5.085655in}{2.599206in}}%
\pgfpathlineto{\pgfqpoint{5.088877in}{2.595119in}}%
\pgfpathlineto{\pgfqpoint{5.089951in}{2.598504in}}%
\pgfpathlineto{\pgfqpoint{5.091024in}{2.596316in}}%
\pgfpathlineto{\pgfqpoint{5.093172in}{2.598256in}}%
\pgfpathlineto{\pgfqpoint{5.096394in}{2.598752in}}%
\pgfpathlineto{\pgfqpoint{5.098541in}{2.601229in}}%
\pgfpathlineto{\pgfqpoint{5.099615in}{2.601064in}}%
\pgfpathlineto{\pgfqpoint{5.100689in}{2.600321in}}%
\pgfpathlineto{\pgfqpoint{5.103911in}{2.603294in}}%
\pgfpathlineto{\pgfqpoint{5.104984in}{2.603046in}}%
\pgfpathlineto{\pgfqpoint{5.106058in}{2.599619in}}%
\pgfpathlineto{\pgfqpoint{5.108206in}{2.596522in}}%
\pgfpathlineto{\pgfqpoint{5.112501in}{2.603211in}}%
\pgfpathlineto{\pgfqpoint{5.113575in}{2.602138in}}%
\pgfpathlineto{\pgfqpoint{5.115723in}{2.603046in}}%
\pgfpathlineto{\pgfqpoint{5.118944in}{2.605936in}}%
\pgfpathlineto{\pgfqpoint{5.121092in}{2.604326in}}%
\pgfpathlineto{\pgfqpoint{5.122166in}{2.604202in}}%
\pgfpathlineto{\pgfqpoint{5.123240in}{2.603046in}}%
\pgfpathlineto{\pgfqpoint{5.126461in}{2.605564in}}%
\pgfpathlineto{\pgfqpoint{5.127535in}{2.608537in}}%
\pgfpathlineto{\pgfqpoint{5.128609in}{2.609033in}}%
\pgfpathlineto{\pgfqpoint{5.130757in}{2.606514in}}%
\pgfpathlineto{\pgfqpoint{5.135052in}{2.606762in}}%
\pgfpathlineto{\pgfqpoint{5.136126in}{2.609569in}}%
\pgfpathlineto{\pgfqpoint{5.137200in}{2.609941in}}%
\pgfpathlineto{\pgfqpoint{5.141495in}{2.609280in}}%
\pgfpathlineto{\pgfqpoint{5.143643in}{2.607340in}}%
\pgfpathlineto{\pgfqpoint{5.144716in}{2.610106in}}%
\pgfpathlineto{\pgfqpoint{5.145790in}{2.611015in}}%
\pgfpathlineto{\pgfqpoint{5.149012in}{2.616341in}}%
\pgfpathlineto{\pgfqpoint{5.150086in}{2.614607in}}%
\pgfpathlineto{\pgfqpoint{5.151159in}{2.614978in}}%
\pgfpathlineto{\pgfqpoint{5.152233in}{2.614235in}}%
\pgfpathlineto{\pgfqpoint{5.153307in}{2.612790in}}%
\pgfpathlineto{\pgfqpoint{5.156529in}{2.612005in}}%
\pgfpathlineto{\pgfqpoint{5.157602in}{2.610024in}}%
\pgfpathlineto{\pgfqpoint{5.158676in}{2.613368in}}%
\pgfpathlineto{\pgfqpoint{5.159750in}{2.613616in}}%
\pgfpathlineto{\pgfqpoint{5.160824in}{2.614441in}}%
\pgfpathlineto{\pgfqpoint{5.166193in}{2.610147in}}%
\pgfpathlineto{\pgfqpoint{5.167267in}{2.607918in}}%
\pgfpathlineto{\pgfqpoint{5.168341in}{2.607216in}}%
\pgfpathlineto{\pgfqpoint{5.171562in}{2.605812in}}%
\pgfpathlineto{\pgfqpoint{5.174784in}{2.608331in}}%
\pgfpathlineto{\pgfqpoint{5.180153in}{2.606349in}}%
\pgfpathlineto{\pgfqpoint{5.181227in}{2.606432in}}%
\pgfpathlineto{\pgfqpoint{5.183375in}{2.608826in}}%
\pgfpathlineto{\pgfqpoint{5.186596in}{2.607340in}}%
\pgfpathlineto{\pgfqpoint{5.187670in}{2.605812in}}%
\pgfpathlineto{\pgfqpoint{5.188744in}{2.605606in}}%
\pgfpathlineto{\pgfqpoint{5.189818in}{2.606473in}}%
\pgfpathlineto{\pgfqpoint{5.190891in}{2.606184in}}%
\pgfpathlineto{\pgfqpoint{5.196261in}{2.606266in}}%
\pgfpathlineto{\pgfqpoint{5.198408in}{2.604821in}}%
\pgfpathlineto{\pgfqpoint{5.203778in}{2.605358in}}%
\pgfpathlineto{\pgfqpoint{5.204851in}{2.606927in}}%
\pgfpathlineto{\pgfqpoint{5.205925in}{2.605812in}}%
\pgfpathlineto{\pgfqpoint{5.209147in}{2.604408in}}%
\pgfpathlineto{\pgfqpoint{5.210221in}{2.602633in}}%
\pgfpathlineto{\pgfqpoint{5.211294in}{2.603707in}}%
\pgfpathlineto{\pgfqpoint{5.212368in}{2.601271in}}%
\pgfpathlineto{\pgfqpoint{5.213442in}{2.602633in}}%
\pgfpathlineto{\pgfqpoint{5.216664in}{2.601147in}}%
\pgfpathlineto{\pgfqpoint{5.217737in}{2.603541in}}%
\pgfpathlineto{\pgfqpoint{5.219885in}{2.606019in}}%
\pgfpathlineto{\pgfqpoint{5.224180in}{2.606721in}}%
\pgfpathlineto{\pgfqpoint{5.225254in}{2.608042in}}%
\pgfpathlineto{\pgfqpoint{5.226328in}{2.611923in}}%
\pgfpathlineto{\pgfqpoint{5.227402in}{2.611716in}}%
\pgfpathlineto{\pgfqpoint{5.228476in}{2.610147in}}%
\pgfpathlineto{\pgfqpoint{5.232771in}{2.610478in}}%
\pgfpathlineto{\pgfqpoint{5.233845in}{2.611634in}}%
\pgfpathlineto{\pgfqpoint{5.234919in}{2.609735in}}%
\pgfpathlineto{\pgfqpoint{5.235993in}{2.610230in}}%
\pgfpathlineto{\pgfqpoint{5.241362in}{2.608290in}}%
\pgfpathlineto{\pgfqpoint{5.242436in}{2.609239in}}%
\pgfpathlineto{\pgfqpoint{5.243510in}{2.606762in}}%
\pgfpathlineto{\pgfqpoint{5.246731in}{2.605688in}}%
\pgfpathlineto{\pgfqpoint{5.249953in}{2.598876in}}%
\pgfpathlineto{\pgfqpoint{5.251026in}{2.607092in}}%
\pgfpathlineto{\pgfqpoint{5.254248in}{2.605564in}}%
\pgfpathlineto{\pgfqpoint{5.255322in}{2.606225in}}%
\pgfpathlineto{\pgfqpoint{5.256396in}{2.612129in}}%
\pgfpathlineto{\pgfqpoint{5.257469in}{2.608785in}}%
\pgfpathlineto{\pgfqpoint{5.258543in}{2.611634in}}%
\pgfpathlineto{\pgfqpoint{5.261765in}{2.613533in}}%
\pgfpathlineto{\pgfqpoint{5.263912in}{2.613533in}}%
\pgfpathlineto{\pgfqpoint{5.264986in}{2.615185in}}%
\pgfpathlineto{\pgfqpoint{5.266060in}{2.614565in}}%
\pgfpathlineto{\pgfqpoint{5.273577in}{2.620263in}}%
\pgfpathlineto{\pgfqpoint{5.276799in}{2.620759in}}%
\pgfpathlineto{\pgfqpoint{5.280020in}{2.618983in}}%
\pgfpathlineto{\pgfqpoint{5.281094in}{2.619272in}}%
\pgfpathlineto{\pgfqpoint{5.286463in}{2.618529in}}%
\pgfpathlineto{\pgfqpoint{5.287537in}{2.621337in}}%
\pgfpathlineto{\pgfqpoint{5.288611in}{2.621584in}}%
\pgfpathlineto{\pgfqpoint{5.291832in}{2.620015in}}%
\pgfpathlineto{\pgfqpoint{5.292906in}{2.618818in}}%
\pgfpathlineto{\pgfqpoint{5.293980in}{2.621130in}}%
\pgfpathlineto{\pgfqpoint{5.296128in}{2.619768in}}%
\pgfpathlineto{\pgfqpoint{5.301497in}{2.622947in}}%
\pgfpathlineto{\pgfqpoint{5.302571in}{2.623071in}}%
\pgfpathlineto{\pgfqpoint{5.303645in}{2.624062in}}%
\pgfpathlineto{\pgfqpoint{5.306866in}{2.625259in}}%
\pgfpathlineto{\pgfqpoint{5.307940in}{2.623938in}}%
\pgfpathlineto{\pgfqpoint{5.309014in}{2.626745in}}%
\pgfpathlineto{\pgfqpoint{5.310088in}{2.623566in}}%
\pgfpathlineto{\pgfqpoint{5.311161in}{2.624433in}}%
\pgfpathlineto{\pgfqpoint{5.314383in}{2.623938in}}%
\pgfpathlineto{\pgfqpoint{5.316531in}{2.619355in}}%
\pgfpathlineto{\pgfqpoint{5.317604in}{2.619066in}}%
\pgfpathlineto{\pgfqpoint{5.318678in}{2.621667in}}%
\pgfpathlineto{\pgfqpoint{5.321900in}{2.620882in}}%
\pgfpathlineto{\pgfqpoint{5.322974in}{2.619437in}}%
\pgfpathlineto{\pgfqpoint{5.324047in}{2.623029in}}%
\pgfpathlineto{\pgfqpoint{5.325121in}{2.621213in}}%
\pgfpathlineto{\pgfqpoint{5.326195in}{2.624805in}}%
\pgfpathlineto{\pgfqpoint{5.329417in}{2.620181in}}%
\pgfpathlineto{\pgfqpoint{5.330490in}{2.620800in}}%
\pgfpathlineto{\pgfqpoint{5.332638in}{2.616176in}}%
\pgfpathlineto{\pgfqpoint{5.333712in}{2.619809in}}%
\pgfpathlineto{\pgfqpoint{5.339081in}{2.625465in}}%
\pgfpathlineto{\pgfqpoint{5.340155in}{2.621915in}}%
\pgfpathlineto{\pgfqpoint{5.341229in}{2.628768in}}%
\pgfpathlineto{\pgfqpoint{5.344450in}{2.631576in}}%
\pgfpathlineto{\pgfqpoint{5.345524in}{2.633434in}}%
\pgfpathlineto{\pgfqpoint{5.346598in}{2.633682in}}%
\pgfpathlineto{\pgfqpoint{5.348746in}{2.636283in}}%
\pgfpathlineto{\pgfqpoint{5.351967in}{2.636654in}}%
\pgfpathlineto{\pgfqpoint{5.353041in}{2.641114in}}%
\pgfpathlineto{\pgfqpoint{5.354115in}{2.642394in}}%
\pgfpathlineto{\pgfqpoint{5.355189in}{2.642063in}}%
\pgfpathlineto{\pgfqpoint{5.356263in}{2.642889in}}%
\pgfpathlineto{\pgfqpoint{5.360558in}{2.644788in}}%
\pgfpathlineto{\pgfqpoint{5.361632in}{2.644128in}}%
\pgfpathlineto{\pgfqpoint{5.363779in}{2.639256in}}%
\pgfpathlineto{\pgfqpoint{5.367001in}{2.638306in}}%
\pgfpathlineto{\pgfqpoint{5.368075in}{2.638678in}}%
\pgfpathlineto{\pgfqpoint{5.369149in}{2.641444in}}%
\pgfpathlineto{\pgfqpoint{5.370223in}{2.640536in}}%
\pgfpathlineto{\pgfqpoint{5.371296in}{2.640990in}}%
\pgfpathlineto{\pgfqpoint{5.374518in}{2.639256in}}%
\pgfpathlineto{\pgfqpoint{5.375592in}{2.641692in}}%
\pgfpathlineto{\pgfqpoint{5.376666in}{2.641981in}}%
\pgfpathlineto{\pgfqpoint{5.378813in}{2.647472in}}%
\pgfpathlineto{\pgfqpoint{5.382035in}{2.646233in}}%
\pgfpathlineto{\pgfqpoint{5.383109in}{2.649743in}}%
\pgfpathlineto{\pgfqpoint{5.384182in}{2.645944in}}%
\pgfpathlineto{\pgfqpoint{5.385256in}{2.648009in}}%
\pgfpathlineto{\pgfqpoint{5.386330in}{2.647307in}}%
\pgfpathlineto{\pgfqpoint{5.389552in}{2.648669in}}%
\pgfpathlineto{\pgfqpoint{5.390625in}{2.648463in}}%
\pgfpathlineto{\pgfqpoint{5.391699in}{2.645944in}}%
\pgfpathlineto{\pgfqpoint{5.392773in}{2.647431in}}%
\pgfpathlineto{\pgfqpoint{5.393847in}{2.644375in}}%
\pgfpathlineto{\pgfqpoint{5.397068in}{2.643137in}}%
\pgfpathlineto{\pgfqpoint{5.398142in}{2.643673in}}%
\pgfpathlineto{\pgfqpoint{5.400290in}{2.653046in}}%
\pgfpathlineto{\pgfqpoint{5.401364in}{2.653252in}}%
\pgfpathlineto{\pgfqpoint{5.404585in}{2.655193in}}%
\pgfpathlineto{\pgfqpoint{5.405659in}{2.657629in}}%
\pgfpathlineto{\pgfqpoint{5.406733in}{2.657092in}}%
\pgfpathlineto{\pgfqpoint{5.408881in}{2.658248in}}%
\pgfpathlineto{\pgfqpoint{5.413176in}{2.654491in}}%
\pgfpathlineto{\pgfqpoint{5.414250in}{2.649825in}}%
\pgfpathlineto{\pgfqpoint{5.416398in}{2.647513in}}%
\pgfpathlineto{\pgfqpoint{5.419619in}{2.645986in}}%
\pgfpathlineto{\pgfqpoint{5.420693in}{2.644541in}}%
\pgfpathlineto{\pgfqpoint{5.421767in}{2.646192in}}%
\pgfpathlineto{\pgfqpoint{5.422841in}{2.649867in}}%
\pgfpathlineto{\pgfqpoint{5.423914in}{2.646853in}}%
\pgfpathlineto{\pgfqpoint{5.427136in}{2.645655in}}%
\pgfpathlineto{\pgfqpoint{5.428210in}{2.647018in}}%
\pgfpathlineto{\pgfqpoint{5.430357in}{2.645449in}}%
\pgfpathlineto{\pgfqpoint{5.431431in}{2.650403in}}%
\pgfpathlineto{\pgfqpoint{5.435727in}{2.650197in}}%
\pgfpathlineto{\pgfqpoint{5.436800in}{2.650816in}}%
\pgfpathlineto{\pgfqpoint{5.437874in}{2.654037in}}%
\pgfpathlineto{\pgfqpoint{5.438948in}{2.648504in}}%
\pgfpathlineto{\pgfqpoint{5.442170in}{2.646729in}}%
\pgfpathlineto{\pgfqpoint{5.443244in}{2.635664in}}%
\pgfpathlineto{\pgfqpoint{5.444317in}{2.630874in}}%
\pgfpathlineto{\pgfqpoint{5.445391in}{2.632732in}}%
\pgfpathlineto{\pgfqpoint{5.446465in}{2.627819in}}%
\pgfpathlineto{\pgfqpoint{5.449687in}{2.630792in}}%
\pgfpathlineto{\pgfqpoint{5.450760in}{2.633764in}}%
\pgfpathlineto{\pgfqpoint{5.451834in}{2.633186in}}%
\pgfpathlineto{\pgfqpoint{5.452908in}{2.636448in}}%
\pgfpathlineto{\pgfqpoint{5.453982in}{2.632526in}}%
\pgfpathlineto{\pgfqpoint{5.457203in}{2.630544in}}%
\pgfpathlineto{\pgfqpoint{5.460425in}{2.634053in}}%
\pgfpathlineto{\pgfqpoint{5.465794in}{2.632113in}}%
\pgfpathlineto{\pgfqpoint{5.466868in}{2.634879in}}%
\pgfpathlineto{\pgfqpoint{5.467942in}{2.631122in}}%
\pgfpathlineto{\pgfqpoint{5.469016in}{2.629883in}}%
\pgfpathlineto{\pgfqpoint{5.473311in}{2.632154in}}%
\pgfpathlineto{\pgfqpoint{5.474385in}{2.631989in}}%
\pgfpathlineto{\pgfqpoint{5.475459in}{2.630957in}}%
\pgfpathlineto{\pgfqpoint{5.476533in}{2.630833in}}%
\pgfpathlineto{\pgfqpoint{5.479754in}{2.631824in}}%
\pgfpathlineto{\pgfqpoint{5.480828in}{2.630915in}}%
\pgfpathlineto{\pgfqpoint{5.481902in}{2.628025in}}%
\pgfpathlineto{\pgfqpoint{5.482976in}{2.629016in}}%
\pgfpathlineto{\pgfqpoint{5.484049in}{2.621997in}}%
\pgfpathlineto{\pgfqpoint{5.487271in}{2.623525in}}%
\pgfpathlineto{\pgfqpoint{5.488345in}{2.617992in}}%
\pgfpathlineto{\pgfqpoint{5.489419in}{2.617455in}}%
\pgfpathlineto{\pgfqpoint{5.490492in}{2.619974in}}%
\pgfpathlineto{\pgfqpoint{5.491566in}{2.619024in}}%
\pgfpathlineto{\pgfqpoint{5.494788in}{2.625218in}}%
\pgfpathlineto{\pgfqpoint{5.495862in}{2.622658in}}%
\pgfpathlineto{\pgfqpoint{5.496935in}{2.625878in}}%
\pgfpathlineto{\pgfqpoint{5.498009in}{2.624557in}}%
\pgfpathlineto{\pgfqpoint{5.499083in}{2.629429in}}%
\pgfpathlineto{\pgfqpoint{5.502305in}{2.629842in}}%
\pgfpathlineto{\pgfqpoint{5.505526in}{2.620181in}}%
\pgfpathlineto{\pgfqpoint{5.509822in}{2.622204in}}%
\pgfpathlineto{\pgfqpoint{5.510895in}{2.619437in}}%
\pgfpathlineto{\pgfqpoint{5.511969in}{2.620759in}}%
\pgfpathlineto{\pgfqpoint{5.517338in}{2.623360in}}%
\pgfpathlineto{\pgfqpoint{5.518412in}{2.621006in}}%
\pgfpathlineto{\pgfqpoint{5.519486in}{2.622369in}}%
\pgfpathlineto{\pgfqpoint{5.520560in}{2.622823in}}%
\pgfpathlineto{\pgfqpoint{5.521634in}{2.624474in}}%
\pgfpathlineto{\pgfqpoint{5.528077in}{2.625011in}}%
\pgfpathlineto{\pgfqpoint{5.529151in}{2.621543in}}%
\pgfpathlineto{\pgfqpoint{5.534520in}{2.623525in}}%
\pgfpathlineto{\pgfqpoint{5.535594in}{2.618199in}}%
\pgfpathlineto{\pgfqpoint{5.536667in}{2.618405in}}%
\pgfpathlineto{\pgfqpoint{5.540963in}{2.616299in}}%
\pgfpathlineto{\pgfqpoint{5.543111in}{2.613038in}}%
\pgfpathlineto{\pgfqpoint{5.544184in}{2.615845in}}%
\pgfpathlineto{\pgfqpoint{5.547406in}{2.616052in}}%
\pgfpathlineto{\pgfqpoint{5.548480in}{2.615019in}}%
\pgfpathlineto{\pgfqpoint{5.549554in}{2.616217in}}%
\pgfpathlineto{\pgfqpoint{5.550627in}{2.615515in}}%
\pgfpathlineto{\pgfqpoint{5.551701in}{2.618240in}}%
\pgfpathlineto{\pgfqpoint{5.557070in}{2.613698in}}%
\pgfpathlineto{\pgfqpoint{5.559218in}{2.618570in}}%
\pgfpathlineto{\pgfqpoint{5.562440in}{2.617455in}}%
\pgfpathlineto{\pgfqpoint{5.563513in}{2.617786in}}%
\pgfpathlineto{\pgfqpoint{5.564587in}{2.616506in}}%
\pgfpathlineto{\pgfqpoint{5.565661in}{2.616258in}}%
\pgfpathlineto{\pgfqpoint{5.566735in}{2.614607in}}%
\pgfpathlineto{\pgfqpoint{5.571030in}{2.611675in}}%
\pgfpathlineto{\pgfqpoint{5.572104in}{2.612583in}}%
\pgfpathlineto{\pgfqpoint{5.573178in}{2.612377in}}%
\pgfpathlineto{\pgfqpoint{5.574252in}{2.608991in}}%
\pgfpathlineto{\pgfqpoint{5.577473in}{2.610643in}}%
\pgfpathlineto{\pgfqpoint{5.578547in}{2.609528in}}%
\pgfpathlineto{\pgfqpoint{5.579621in}{2.609611in}}%
\pgfpathlineto{\pgfqpoint{5.580695in}{2.608124in}}%
\pgfpathlineto{\pgfqpoint{5.581769in}{2.605523in}}%
\pgfpathlineto{\pgfqpoint{5.584990in}{2.606514in}}%
\pgfpathlineto{\pgfqpoint{5.587138in}{2.613120in}}%
\pgfpathlineto{\pgfqpoint{5.588212in}{2.612666in}}%
\pgfpathlineto{\pgfqpoint{5.589286in}{2.610726in}}%
\pgfpathlineto{\pgfqpoint{5.592507in}{2.608000in}}%
\pgfpathlineto{\pgfqpoint{5.595729in}{2.617745in}}%
\pgfpathlineto{\pgfqpoint{5.596802in}{2.616712in}}%
\pgfpathlineto{\pgfqpoint{5.600024in}{2.616423in}}%
\pgfpathlineto{\pgfqpoint{5.601098in}{2.614029in}}%
\pgfpathlineto{\pgfqpoint{5.603245in}{2.612583in}}%
\pgfpathlineto{\pgfqpoint{5.604319in}{2.612418in}}%
\pgfpathlineto{\pgfqpoint{5.607541in}{2.608744in}}%
\pgfpathlineto{\pgfqpoint{5.608615in}{2.608455in}}%
\pgfpathlineto{\pgfqpoint{5.609688in}{2.613781in}}%
\pgfpathlineto{\pgfqpoint{5.610762in}{2.614524in}}%
\pgfpathlineto{\pgfqpoint{5.615058in}{2.614978in}}%
\pgfpathlineto{\pgfqpoint{5.616132in}{2.620965in}}%
\pgfpathlineto{\pgfqpoint{5.618279in}{2.617166in}}%
\pgfpathlineto{\pgfqpoint{5.623648in}{2.622121in}}%
\pgfpathlineto{\pgfqpoint{5.626870in}{2.622864in}}%
\pgfpathlineto{\pgfqpoint{5.630091in}{2.622658in}}%
\pgfpathlineto{\pgfqpoint{5.631165in}{2.620676in}}%
\pgfpathlineto{\pgfqpoint{5.633313in}{2.619685in}}%
\pgfpathlineto{\pgfqpoint{5.634387in}{2.618199in}}%
\pgfpathlineto{\pgfqpoint{5.637608in}{2.617043in}}%
\pgfpathlineto{\pgfqpoint{5.639756in}{2.619396in}}%
\pgfpathlineto{\pgfqpoint{5.640830in}{2.607670in}}%
\pgfpathlineto{\pgfqpoint{5.641904in}{2.605152in}}%
\pgfpathlineto{\pgfqpoint{5.645125in}{2.604078in}}%
\pgfpathlineto{\pgfqpoint{5.646199in}{2.602303in}}%
\pgfpathlineto{\pgfqpoint{5.649421in}{2.600734in}}%
\pgfpathlineto{\pgfqpoint{5.652642in}{2.603996in}}%
\pgfpathlineto{\pgfqpoint{5.653716in}{2.603417in}}%
\pgfpathlineto{\pgfqpoint{5.654790in}{2.604037in}}%
\pgfpathlineto{\pgfqpoint{5.655864in}{2.601807in}}%
\pgfpathlineto{\pgfqpoint{5.656937in}{2.601229in}}%
\pgfpathlineto{\pgfqpoint{5.660159in}{2.600899in}}%
\pgfpathlineto{\pgfqpoint{5.661233in}{2.599454in}}%
\pgfpathlineto{\pgfqpoint{5.662307in}{2.595779in}}%
\pgfpathlineto{\pgfqpoint{5.663380in}{2.595036in}}%
\pgfpathlineto{\pgfqpoint{5.664454in}{2.587522in}}%
\pgfpathlineto{\pgfqpoint{5.668750in}{2.575094in}}%
\pgfpathlineto{\pgfqpoint{5.669823in}{2.584095in}}%
\pgfpathlineto{\pgfqpoint{5.670897in}{2.586200in}}%
\pgfpathlineto{\pgfqpoint{5.671971in}{2.585209in}}%
\pgfpathlineto{\pgfqpoint{5.675193in}{2.583269in}}%
\pgfpathlineto{\pgfqpoint{5.676266in}{2.576828in}}%
\pgfpathlineto{\pgfqpoint{5.677340in}{2.580131in}}%
\pgfpathlineto{\pgfqpoint{5.678414in}{2.580544in}}%
\pgfpathlineto{\pgfqpoint{5.679488in}{2.576332in}}%
\pgfpathlineto{\pgfqpoint{5.683783in}{2.580750in}}%
\pgfpathlineto{\pgfqpoint{5.684857in}{2.575300in}}%
\pgfpathlineto{\pgfqpoint{5.685931in}{2.574722in}}%
\pgfpathlineto{\pgfqpoint{5.687005in}{2.575094in}}%
\pgfpathlineto{\pgfqpoint{5.690226in}{2.573773in}}%
\pgfpathlineto{\pgfqpoint{5.691300in}{2.578810in}}%
\pgfpathlineto{\pgfqpoint{5.692374in}{2.581163in}}%
\pgfpathlineto{\pgfqpoint{5.693448in}{2.581700in}}%
\pgfpathlineto{\pgfqpoint{5.694522in}{2.580585in}}%
\pgfpathlineto{\pgfqpoint{5.697743in}{2.583186in}}%
\pgfpathlineto{\pgfqpoint{5.698817in}{2.581493in}}%
\pgfpathlineto{\pgfqpoint{5.699891in}{2.581782in}}%
\pgfpathlineto{\pgfqpoint{5.702039in}{2.590536in}}%
\pgfpathlineto{\pgfqpoint{5.705260in}{2.587233in}}%
\pgfpathlineto{\pgfqpoint{5.706334in}{2.589090in}}%
\pgfpathlineto{\pgfqpoint{5.707408in}{2.587852in}}%
\pgfpathlineto{\pgfqpoint{5.708482in}{2.587893in}}%
\pgfpathlineto{\pgfqpoint{5.709555in}{2.589627in}}%
\pgfpathlineto{\pgfqpoint{5.714925in}{2.594334in}}%
\pgfpathlineto{\pgfqpoint{5.715999in}{2.596811in}}%
\pgfpathlineto{\pgfqpoint{5.717072in}{2.597100in}}%
\pgfpathlineto{\pgfqpoint{5.720294in}{2.596564in}}%
\pgfpathlineto{\pgfqpoint{5.721368in}{2.595738in}}%
\pgfpathlineto{\pgfqpoint{5.723515in}{2.596316in}}%
\pgfpathlineto{\pgfqpoint{5.724589in}{2.598628in}}%
\pgfpathlineto{\pgfqpoint{5.727811in}{2.599578in}}%
\pgfpathlineto{\pgfqpoint{5.728885in}{2.596894in}}%
\pgfpathlineto{\pgfqpoint{5.729958in}{2.596275in}}%
\pgfpathlineto{\pgfqpoint{5.731032in}{2.600899in}}%
\pgfpathlineto{\pgfqpoint{5.732106in}{2.608868in}}%
\pgfpathlineto{\pgfqpoint{5.735328in}{2.610560in}}%
\pgfpathlineto{\pgfqpoint{5.736401in}{2.609817in}}%
\pgfpathlineto{\pgfqpoint{5.737475in}{2.606968in}}%
\pgfpathlineto{\pgfqpoint{5.738549in}{2.608826in}}%
\pgfpathlineto{\pgfqpoint{5.739623in}{2.606473in}}%
\pgfpathlineto{\pgfqpoint{5.742844in}{2.607299in}}%
\pgfpathlineto{\pgfqpoint{5.743918in}{2.608950in}}%
\pgfpathlineto{\pgfqpoint{5.744992in}{2.608991in}}%
\pgfpathlineto{\pgfqpoint{5.747140in}{2.603500in}}%
\pgfpathlineto{\pgfqpoint{5.750361in}{2.602881in}}%
\pgfpathlineto{\pgfqpoint{5.752509in}{2.604945in}}%
\pgfpathlineto{\pgfqpoint{5.754657in}{2.597637in}}%
\pgfpathlineto{\pgfqpoint{5.757878in}{2.602509in}}%
\pgfpathlineto{\pgfqpoint{5.758952in}{2.601683in}}%
\pgfpathlineto{\pgfqpoint{5.760026in}{2.604739in}}%
\pgfpathlineto{\pgfqpoint{5.761100in}{2.605895in}}%
\pgfpathlineto{\pgfqpoint{5.762174in}{2.604450in}}%
\pgfpathlineto{\pgfqpoint{5.765395in}{2.604986in}}%
\pgfpathlineto{\pgfqpoint{5.766469in}{2.606762in}}%
\pgfpathlineto{\pgfqpoint{5.767543in}{2.604739in}}%
\pgfpathlineto{\pgfqpoint{5.769690in}{2.603996in}}%
\pgfpathlineto{\pgfqpoint{5.772912in}{2.600858in}}%
\pgfpathlineto{\pgfqpoint{5.773986in}{2.604863in}}%
\pgfpathlineto{\pgfqpoint{5.776133in}{2.604161in}}%
\pgfpathlineto{\pgfqpoint{5.777207in}{2.611799in}}%
\pgfpathlineto{\pgfqpoint{5.780429in}{2.613781in}}%
\pgfpathlineto{\pgfqpoint{5.781503in}{2.611510in}}%
\pgfpathlineto{\pgfqpoint{5.783650in}{2.611675in}}%
\pgfpathlineto{\pgfqpoint{5.784724in}{2.611634in}}%
\pgfpathlineto{\pgfqpoint{5.787946in}{2.613492in}}%
\pgfpathlineto{\pgfqpoint{5.790093in}{2.623401in}}%
\pgfpathlineto{\pgfqpoint{5.791167in}{2.620800in}}%
\pgfpathlineto{\pgfqpoint{5.792241in}{2.612914in}}%
\pgfpathlineto{\pgfqpoint{5.795463in}{2.615887in}}%
\pgfpathlineto{\pgfqpoint{5.797610in}{2.619479in}}%
\pgfpathlineto{\pgfqpoint{5.798684in}{2.618983in}}%
\pgfpathlineto{\pgfqpoint{5.802979in}{2.619479in}}%
\pgfpathlineto{\pgfqpoint{5.804053in}{2.621089in}}%
\pgfpathlineto{\pgfqpoint{5.805127in}{2.620015in}}%
\pgfpathlineto{\pgfqpoint{5.806201in}{2.617579in}}%
\pgfpathlineto{\pgfqpoint{5.810496in}{2.613781in}}%
\pgfpathlineto{\pgfqpoint{5.811570in}{2.614689in}}%
\pgfpathlineto{\pgfqpoint{5.814792in}{2.604986in}}%
\pgfpathlineto{\pgfqpoint{5.818013in}{2.607546in}}%
\pgfpathlineto{\pgfqpoint{5.819087in}{2.606968in}}%
\pgfpathlineto{\pgfqpoint{5.820161in}{2.604532in}}%
\pgfpathlineto{\pgfqpoint{5.821235in}{2.605647in}}%
\pgfpathlineto{\pgfqpoint{5.822309in}{2.601353in}}%
\pgfpathlineto{\pgfqpoint{5.826604in}{2.607753in}}%
\pgfpathlineto{\pgfqpoint{5.827678in}{2.606886in}}%
\pgfpathlineto{\pgfqpoint{5.829825in}{2.612542in}}%
\pgfpathlineto{\pgfqpoint{5.833047in}{2.610684in}}%
\pgfpathlineto{\pgfqpoint{5.834121in}{2.617910in}}%
\pgfpathlineto{\pgfqpoint{5.835195in}{2.617868in}}%
\pgfpathlineto{\pgfqpoint{5.836268in}{2.621626in}}%
\pgfpathlineto{\pgfqpoint{5.837342in}{2.628562in}}%
\pgfpathlineto{\pgfqpoint{5.840564in}{2.626456in}}%
\pgfpathlineto{\pgfqpoint{5.841638in}{2.623112in}}%
\pgfpathlineto{\pgfqpoint{5.842711in}{2.626374in}}%
\pgfpathlineto{\pgfqpoint{5.843785in}{2.624887in}}%
\pgfpathlineto{\pgfqpoint{5.848081in}{2.631989in}}%
\pgfpathlineto{\pgfqpoint{5.849154in}{2.632072in}}%
\pgfpathlineto{\pgfqpoint{5.850228in}{2.628314in}}%
\pgfpathlineto{\pgfqpoint{5.851302in}{2.621956in}}%
\pgfpathlineto{\pgfqpoint{5.852376in}{2.625961in}}%
\pgfpathlineto{\pgfqpoint{5.856671in}{2.627736in}}%
\pgfpathlineto{\pgfqpoint{5.857745in}{2.631370in}}%
\pgfpathlineto{\pgfqpoint{5.859893in}{2.628934in}}%
\pgfpathlineto{\pgfqpoint{5.863114in}{2.630172in}}%
\pgfpathlineto{\pgfqpoint{5.865262in}{2.628067in}}%
\pgfpathlineto{\pgfqpoint{5.866336in}{2.630957in}}%
\pgfpathlineto{\pgfqpoint{5.867410in}{2.626374in}}%
\pgfpathlineto{\pgfqpoint{5.870631in}{2.623401in}}%
\pgfpathlineto{\pgfqpoint{5.873853in}{2.632815in}}%
\pgfpathlineto{\pgfqpoint{5.874927in}{2.635209in}}%
\pgfpathlineto{\pgfqpoint{5.880296in}{2.633269in}}%
\pgfpathlineto{\pgfqpoint{5.882443in}{2.628768in}}%
\pgfpathlineto{\pgfqpoint{5.885665in}{2.626621in}}%
\pgfpathlineto{\pgfqpoint{5.887813in}{2.627282in}}%
\pgfpathlineto{\pgfqpoint{5.888887in}{2.632484in}}%
\pgfpathlineto{\pgfqpoint{5.889960in}{2.633971in}}%
\pgfpathlineto{\pgfqpoint{5.893182in}{2.634590in}}%
\pgfpathlineto{\pgfqpoint{5.894256in}{2.632443in}}%
\pgfpathlineto{\pgfqpoint{5.896403in}{2.632980in}}%
\pgfpathlineto{\pgfqpoint{5.900699in}{2.631989in}}%
\pgfpathlineto{\pgfqpoint{5.901773in}{2.632650in}}%
\pgfpathlineto{\pgfqpoint{5.902846in}{2.632237in}}%
\pgfpathlineto{\pgfqpoint{5.903920in}{2.630833in}}%
\pgfpathlineto{\pgfqpoint{5.904994in}{2.635375in}}%
\pgfpathlineto{\pgfqpoint{5.909289in}{2.634012in}}%
\pgfpathlineto{\pgfqpoint{5.910363in}{2.636407in}}%
\pgfpathlineto{\pgfqpoint{5.911437in}{2.634301in}}%
\pgfpathlineto{\pgfqpoint{5.912511in}{2.634136in}}%
\pgfpathlineto{\pgfqpoint{5.915732in}{2.632402in}}%
\pgfpathlineto{\pgfqpoint{5.916806in}{2.632773in}}%
\pgfpathlineto{\pgfqpoint{5.917880in}{2.631411in}}%
\pgfpathlineto{\pgfqpoint{5.924323in}{2.636944in}}%
\pgfpathlineto{\pgfqpoint{5.925397in}{2.630503in}}%
\pgfpathlineto{\pgfqpoint{5.926471in}{2.627695in}}%
\pgfpathlineto{\pgfqpoint{5.930766in}{2.630007in}}%
\pgfpathlineto{\pgfqpoint{5.931840in}{2.623071in}}%
\pgfpathlineto{\pgfqpoint{5.932914in}{2.624309in}}%
\pgfpathlineto{\pgfqpoint{5.933988in}{2.623855in}}%
\pgfpathlineto{\pgfqpoint{5.938283in}{2.628356in}}%
\pgfpathlineto{\pgfqpoint{5.939357in}{2.628851in}}%
\pgfpathlineto{\pgfqpoint{5.940431in}{2.630709in}}%
\pgfpathlineto{\pgfqpoint{5.941505in}{2.629594in}}%
\pgfpathlineto{\pgfqpoint{5.942578in}{2.632691in}}%
\pgfpathlineto{\pgfqpoint{5.945800in}{2.632650in}}%
\pgfpathlineto{\pgfqpoint{5.946874in}{2.633971in}}%
\pgfpathlineto{\pgfqpoint{5.947948in}{2.632732in}}%
\pgfpathlineto{\pgfqpoint{5.949021in}{2.633723in}}%
\pgfpathlineto{\pgfqpoint{5.950095in}{2.629305in}}%
\pgfpathlineto{\pgfqpoint{5.953317in}{2.630792in}}%
\pgfpathlineto{\pgfqpoint{5.955465in}{2.624185in}}%
\pgfpathlineto{\pgfqpoint{5.956538in}{2.625424in}}%
\pgfpathlineto{\pgfqpoint{5.957612in}{2.624805in}}%
\pgfpathlineto{\pgfqpoint{5.960834in}{2.625465in}}%
\pgfpathlineto{\pgfqpoint{5.962981in}{2.630255in}}%
\pgfpathlineto{\pgfqpoint{5.964055in}{2.629264in}}%
\pgfpathlineto{\pgfqpoint{5.965129in}{2.630048in}}%
\pgfpathlineto{\pgfqpoint{5.969424in}{2.628603in}}%
\pgfpathlineto{\pgfqpoint{5.970498in}{2.631411in}}%
\pgfpathlineto{\pgfqpoint{5.971572in}{2.631989in}}%
\pgfpathlineto{\pgfqpoint{5.972646in}{2.633929in}}%
\pgfpathlineto{\pgfqpoint{5.975867in}{2.635044in}}%
\pgfpathlineto{\pgfqpoint{5.976941in}{2.633393in}}%
\pgfpathlineto{\pgfqpoint{5.980163in}{2.636654in}}%
\pgfpathlineto{\pgfqpoint{5.983384in}{2.634301in}}%
\pgfpathlineto{\pgfqpoint{5.984458in}{2.637233in}}%
\pgfpathlineto{\pgfqpoint{5.985532in}{2.635746in}}%
\pgfpathlineto{\pgfqpoint{5.986606in}{2.637439in}}%
\pgfpathlineto{\pgfqpoint{5.987680in}{2.636407in}}%
\pgfpathlineto{\pgfqpoint{5.990901in}{2.636076in}}%
\pgfpathlineto{\pgfqpoint{5.994123in}{2.640412in}}%
\pgfpathlineto{\pgfqpoint{5.995197in}{2.633145in}}%
\pgfpathlineto{\pgfqpoint{5.998418in}{2.629305in}}%
\pgfpathlineto{\pgfqpoint{6.001640in}{2.642146in}}%
\pgfpathlineto{\pgfqpoint{6.002713in}{2.642559in}}%
\pgfpathlineto{\pgfqpoint{6.007009in}{2.644995in}}%
\pgfpathlineto{\pgfqpoint{6.009156in}{2.642724in}}%
\pgfpathlineto{\pgfqpoint{6.010230in}{2.646233in}}%
\pgfpathlineto{\pgfqpoint{6.014526in}{2.646151in}}%
\pgfpathlineto{\pgfqpoint{6.017747in}{2.647142in}}%
\pgfpathlineto{\pgfqpoint{6.020969in}{2.646853in}}%
\pgfpathlineto{\pgfqpoint{6.022042in}{2.647720in}}%
\pgfpathlineto{\pgfqpoint{6.024190in}{2.646811in}}%
\pgfpathlineto{\pgfqpoint{6.025264in}{2.648545in}}%
\pgfpathlineto{\pgfqpoint{6.028486in}{2.648835in}}%
\pgfpathlineto{\pgfqpoint{6.030633in}{2.643839in}}%
\pgfpathlineto{\pgfqpoint{6.031707in}{2.645119in}}%
\pgfpathlineto{\pgfqpoint{6.032781in}{2.648050in}}%
\pgfpathlineto{\pgfqpoint{6.037076in}{2.652468in}}%
\pgfpathlineto{\pgfqpoint{6.038150in}{2.649495in}}%
\pgfpathlineto{\pgfqpoint{6.039224in}{2.649784in}}%
\pgfpathlineto{\pgfqpoint{6.040298in}{2.648793in}}%
\pgfpathlineto{\pgfqpoint{6.043519in}{2.648711in}}%
\pgfpathlineto{\pgfqpoint{6.045667in}{2.650775in}}%
\pgfpathlineto{\pgfqpoint{6.047815in}{2.653500in}}%
\pgfpathlineto{\pgfqpoint{6.051036in}{2.653417in}}%
\pgfpathlineto{\pgfqpoint{6.052110in}{2.651766in}}%
\pgfpathlineto{\pgfqpoint{6.054258in}{2.655028in}}%
\pgfpathlineto{\pgfqpoint{6.058553in}{2.652798in}}%
\pgfpathlineto{\pgfqpoint{6.059627in}{2.654863in}}%
\pgfpathlineto{\pgfqpoint{6.060701in}{2.654532in}}%
\pgfpathlineto{\pgfqpoint{6.061775in}{2.656762in}}%
\pgfpathlineto{\pgfqpoint{6.062848in}{2.655523in}}%
\pgfpathlineto{\pgfqpoint{6.066070in}{2.658248in}}%
\pgfpathlineto{\pgfqpoint{6.067144in}{2.655399in}}%
\pgfpathlineto{\pgfqpoint{6.068218in}{2.654532in}}%
\pgfpathlineto{\pgfqpoint{6.069291in}{2.658290in}}%
\pgfpathlineto{\pgfqpoint{6.070365in}{2.657877in}}%
\pgfpathlineto{\pgfqpoint{6.074661in}{2.659528in}}%
\pgfpathlineto{\pgfqpoint{6.075734in}{2.656927in}}%
\pgfpathlineto{\pgfqpoint{6.076808in}{2.656308in}}%
\pgfpathlineto{\pgfqpoint{6.077882in}{2.650527in}}%
\pgfpathlineto{\pgfqpoint{6.081104in}{2.658042in}}%
\pgfpathlineto{\pgfqpoint{6.082177in}{2.653541in}}%
\pgfpathlineto{\pgfqpoint{6.083251in}{2.653417in}}%
\pgfpathlineto{\pgfqpoint{6.084325in}{2.657340in}}%
\pgfpathlineto{\pgfqpoint{6.085399in}{2.657299in}}%
\pgfpathlineto{\pgfqpoint{6.089694in}{2.659280in}}%
\pgfpathlineto{\pgfqpoint{6.090768in}{2.656349in}}%
\pgfpathlineto{\pgfqpoint{6.091842in}{2.660849in}}%
\pgfpathlineto{\pgfqpoint{6.092916in}{2.656225in}}%
\pgfpathlineto{\pgfqpoint{6.096137in}{2.656555in}}%
\pgfpathlineto{\pgfqpoint{6.097211in}{2.658455in}}%
\pgfpathlineto{\pgfqpoint{6.098285in}{2.662583in}}%
\pgfpathlineto{\pgfqpoint{6.099359in}{2.658000in}}%
\pgfpathlineto{\pgfqpoint{6.100433in}{2.663698in}}%
\pgfpathlineto{\pgfqpoint{6.104728in}{2.658455in}}%
\pgfpathlineto{\pgfqpoint{6.107950in}{2.664648in}}%
\pgfpathlineto{\pgfqpoint{6.114393in}{2.658000in}}%
\pgfpathlineto{\pgfqpoint{6.115466in}{2.658744in}}%
\pgfpathlineto{\pgfqpoint{6.119762in}{2.655069in}}%
\pgfpathlineto{\pgfqpoint{6.121909in}{2.648050in}}%
\pgfpathlineto{\pgfqpoint{6.122983in}{2.645779in}}%
\pgfpathlineto{\pgfqpoint{6.126205in}{2.644912in}}%
\pgfpathlineto{\pgfqpoint{6.127279in}{2.655771in}}%
\pgfpathlineto{\pgfqpoint{6.128353in}{2.657381in}}%
\pgfpathlineto{\pgfqpoint{6.129426in}{2.654285in}}%
\pgfpathlineto{\pgfqpoint{6.130500in}{2.655275in}}%
\pgfpathlineto{\pgfqpoint{6.135869in}{2.654904in}}%
\pgfpathlineto{\pgfqpoint{6.136943in}{2.654367in}}%
\pgfpathlineto{\pgfqpoint{6.138017in}{2.648628in}}%
\pgfpathlineto{\pgfqpoint{6.141239in}{2.654202in}}%
\pgfpathlineto{\pgfqpoint{6.142312in}{2.657629in}}%
\pgfpathlineto{\pgfqpoint{6.143386in}{2.651849in}}%
\pgfpathlineto{\pgfqpoint{6.144460in}{2.640577in}}%
\pgfpathlineto{\pgfqpoint{6.145534in}{2.642930in}}%
\pgfpathlineto{\pgfqpoint{6.148755in}{2.640742in}}%
\pgfpathlineto{\pgfqpoint{6.149829in}{2.643095in}}%
\pgfpathlineto{\pgfqpoint{6.150903in}{2.641444in}}%
\pgfpathlineto{\pgfqpoint{6.151977in}{2.640990in}}%
\pgfpathlineto{\pgfqpoint{6.153051in}{2.636944in}}%
\pgfpathlineto{\pgfqpoint{6.157346in}{2.639834in}}%
\pgfpathlineto{\pgfqpoint{6.158420in}{2.639545in}}%
\pgfpathlineto{\pgfqpoint{6.160568in}{2.642476in}}%
\pgfpathlineto{\pgfqpoint{6.164863in}{2.640329in}}%
\pgfpathlineto{\pgfqpoint{6.167011in}{2.636407in}}%
\pgfpathlineto{\pgfqpoint{6.168085in}{2.638471in}}%
\pgfpathlineto{\pgfqpoint{6.171306in}{2.640701in}}%
\pgfpathlineto{\pgfqpoint{6.172380in}{2.640412in}}%
\pgfpathlineto{\pgfqpoint{6.173454in}{2.645201in}}%
\pgfpathlineto{\pgfqpoint{6.174528in}{2.642641in}}%
\pgfpathlineto{\pgfqpoint{6.175601in}{2.645944in}}%
\pgfpathlineto{\pgfqpoint{6.178823in}{2.648793in}}%
\pgfpathlineto{\pgfqpoint{6.179897in}{2.649000in}}%
\pgfpathlineto{\pgfqpoint{6.180971in}{2.645944in}}%
\pgfpathlineto{\pgfqpoint{6.182044in}{2.647100in}}%
\pgfpathlineto{\pgfqpoint{6.187414in}{2.646688in}}%
\pgfpathlineto{\pgfqpoint{6.188487in}{2.645573in}}%
\pgfpathlineto{\pgfqpoint{6.189561in}{2.646316in}}%
\pgfpathlineto{\pgfqpoint{6.190635in}{2.648174in}}%
\pgfpathlineto{\pgfqpoint{6.194930in}{2.646811in}}%
\pgfpathlineto{\pgfqpoint{6.196004in}{2.644788in}}%
\pgfpathlineto{\pgfqpoint{6.197078in}{2.645862in}}%
\pgfpathlineto{\pgfqpoint{6.198152in}{2.644830in}}%
\pgfpathlineto{\pgfqpoint{6.202447in}{2.645284in}}%
\pgfpathlineto{\pgfqpoint{6.203521in}{2.646440in}}%
\pgfpathlineto{\pgfqpoint{6.204595in}{2.648545in}}%
\pgfpathlineto{\pgfqpoint{6.205669in}{2.648422in}}%
\pgfpathlineto{\pgfqpoint{6.208890in}{2.646027in}}%
\pgfpathlineto{\pgfqpoint{6.209964in}{2.642600in}}%
\pgfpathlineto{\pgfqpoint{6.212112in}{2.643921in}}%
\pgfpathlineto{\pgfqpoint{6.213186in}{2.644582in}}%
\pgfpathlineto{\pgfqpoint{6.218555in}{2.650610in}}%
\pgfpathlineto{\pgfqpoint{6.219629in}{2.649702in}}%
\pgfpathlineto{\pgfqpoint{6.220703in}{2.660189in}}%
\pgfpathlineto{\pgfqpoint{6.223924in}{2.658331in}}%
\pgfpathlineto{\pgfqpoint{6.224998in}{2.661758in}}%
\pgfpathlineto{\pgfqpoint{6.227146in}{2.656968in}}%
\pgfpathlineto{\pgfqpoint{6.228219in}{2.657422in}}%
\pgfpathlineto{\pgfqpoint{6.231441in}{2.657546in}}%
\pgfpathlineto{\pgfqpoint{6.232515in}{2.660767in}}%
\pgfpathlineto{\pgfqpoint{6.233589in}{2.659735in}}%
\pgfpathlineto{\pgfqpoint{6.234663in}{2.661386in}}%
\pgfpathlineto{\pgfqpoint{6.235736in}{2.660065in}}%
\pgfpathlineto{\pgfqpoint{6.238958in}{2.660024in}}%
\pgfpathlineto{\pgfqpoint{6.241106in}{2.663574in}}%
\pgfpathlineto{\pgfqpoint{6.242179in}{2.664854in}}%
\pgfpathlineto{\pgfqpoint{6.243253in}{2.662212in}}%
\pgfpathlineto{\pgfqpoint{6.246475in}{2.663492in}}%
\pgfpathlineto{\pgfqpoint{6.247549in}{2.661758in}}%
\pgfpathlineto{\pgfqpoint{6.248622in}{2.674227in}}%
\pgfpathlineto{\pgfqpoint{6.249696in}{2.672947in}}%
\pgfpathlineto{\pgfqpoint{6.250770in}{2.674103in}}%
\pgfpathlineto{\pgfqpoint{6.255065in}{2.676291in}}%
\pgfpathlineto{\pgfqpoint{6.258287in}{2.673938in}}%
\pgfpathlineto{\pgfqpoint{6.261508in}{2.673318in}}%
\pgfpathlineto{\pgfqpoint{6.262582in}{2.674020in}}%
\pgfpathlineto{\pgfqpoint{6.263656in}{2.676291in}}%
\pgfpathlineto{\pgfqpoint{6.265804in}{2.671832in}}%
\pgfpathlineto{\pgfqpoint{6.272247in}{2.671254in}}%
\pgfpathlineto{\pgfqpoint{6.273321in}{2.674020in}}%
\pgfpathlineto{\pgfqpoint{6.276542in}{2.674929in}}%
\pgfpathlineto{\pgfqpoint{6.277616in}{2.673773in}}%
\pgfpathlineto{\pgfqpoint{6.278690in}{2.675300in}}%
\pgfpathlineto{\pgfqpoint{6.279764in}{2.675424in}}%
\pgfpathlineto{\pgfqpoint{6.280838in}{2.673773in}}%
\pgfpathlineto{\pgfqpoint{6.285133in}{2.674474in}}%
\pgfpathlineto{\pgfqpoint{6.288354in}{2.672121in}}%
\pgfpathlineto{\pgfqpoint{6.291576in}{2.671791in}}%
\pgfpathlineto{\pgfqpoint{6.292650in}{2.672823in}}%
\pgfpathlineto{\pgfqpoint{6.293724in}{2.672245in}}%
\pgfpathlineto{\pgfqpoint{6.295871in}{2.669355in}}%
\pgfpathlineto{\pgfqpoint{6.299093in}{2.668735in}}%
\pgfpathlineto{\pgfqpoint{6.301241in}{2.669809in}}%
\pgfpathlineto{\pgfqpoint{6.302314in}{2.667662in}}%
\pgfpathlineto{\pgfqpoint{6.303388in}{2.667001in}}%
\pgfpathlineto{\pgfqpoint{6.306610in}{2.667992in}}%
\pgfpathlineto{\pgfqpoint{6.308757in}{2.671130in}}%
\pgfpathlineto{\pgfqpoint{6.309831in}{2.670057in}}%
\pgfpathlineto{\pgfqpoint{6.314127in}{2.671419in}}%
\pgfpathlineto{\pgfqpoint{6.315200in}{2.672988in}}%
\pgfpathlineto{\pgfqpoint{6.318422in}{2.667249in}}%
\pgfpathlineto{\pgfqpoint{6.322717in}{2.672575in}}%
\pgfpathlineto{\pgfqpoint{6.323791in}{2.663863in}}%
\pgfpathlineto{\pgfqpoint{6.324865in}{2.663698in}}%
\pgfpathlineto{\pgfqpoint{6.325939in}{2.662294in}}%
\pgfpathlineto{\pgfqpoint{6.329160in}{2.661386in}}%
\pgfpathlineto{\pgfqpoint{6.330234in}{2.658042in}}%
\pgfpathlineto{\pgfqpoint{6.331308in}{2.658826in}}%
\pgfpathlineto{\pgfqpoint{6.336677in}{2.659322in}}%
\pgfpathlineto{\pgfqpoint{6.337751in}{2.658744in}}%
\pgfpathlineto{\pgfqpoint{6.338825in}{2.659074in}}%
\pgfpathlineto{\pgfqpoint{6.339899in}{2.657835in}}%
\pgfpathlineto{\pgfqpoint{6.346342in}{2.658207in}}%
\pgfpathlineto{\pgfqpoint{6.347416in}{2.656679in}}%
\pgfpathlineto{\pgfqpoint{6.348489in}{2.658124in}}%
\pgfpathlineto{\pgfqpoint{6.352785in}{2.657505in}}%
\pgfpathlineto{\pgfqpoint{6.356006in}{2.662005in}}%
\pgfpathlineto{\pgfqpoint{6.360302in}{2.662583in}}%
\pgfpathlineto{\pgfqpoint{6.361375in}{2.665226in}}%
\pgfpathlineto{\pgfqpoint{6.362449in}{2.665391in}}%
\pgfpathlineto{\pgfqpoint{6.363523in}{2.667125in}}%
\pgfpathlineto{\pgfqpoint{6.368892in}{2.667827in}}%
\pgfpathlineto{\pgfqpoint{6.369966in}{2.664318in}}%
\pgfpathlineto{\pgfqpoint{6.371040in}{2.665474in}}%
\pgfpathlineto{\pgfqpoint{6.374262in}{2.665804in}}%
\pgfpathlineto{\pgfqpoint{6.375335in}{2.665102in}}%
\pgfpathlineto{\pgfqpoint{6.376409in}{2.666588in}}%
\pgfpathlineto{\pgfqpoint{6.377483in}{2.670181in}}%
\pgfpathlineto{\pgfqpoint{6.378557in}{2.671254in}}%
\pgfpathlineto{\pgfqpoint{6.381778in}{2.672038in}}%
\pgfpathlineto{\pgfqpoint{6.385000in}{2.668818in}}%
\pgfpathlineto{\pgfqpoint{6.386074in}{2.670346in}}%
\pgfpathlineto{\pgfqpoint{6.389295in}{2.670098in}}%
\pgfpathlineto{\pgfqpoint{6.390369in}{2.667208in}}%
\pgfpathlineto{\pgfqpoint{6.391443in}{2.666299in}}%
\pgfpathlineto{\pgfqpoint{6.392517in}{2.661015in}}%
\pgfpathlineto{\pgfqpoint{6.396812in}{2.663863in}}%
\pgfpathlineto{\pgfqpoint{6.398960in}{2.663616in}}%
\pgfpathlineto{\pgfqpoint{6.400034in}{2.662460in}}%
\pgfpathlineto{\pgfqpoint{6.401107in}{2.663533in}}%
\pgfpathlineto{\pgfqpoint{6.405403in}{2.660147in}}%
\pgfpathlineto{\pgfqpoint{6.406477in}{2.660891in}}%
\pgfpathlineto{\pgfqpoint{6.407551in}{2.659900in}}%
\pgfpathlineto{\pgfqpoint{6.408624in}{2.661427in}}%
\pgfpathlineto{\pgfqpoint{6.411846in}{2.663162in}}%
\pgfpathlineto{\pgfqpoint{6.412920in}{2.667332in}}%
\pgfpathlineto{\pgfqpoint{6.415067in}{2.669850in}}%
\pgfpathlineto{\pgfqpoint{6.416141in}{2.669891in}}%
\pgfpathlineto{\pgfqpoint{6.419363in}{2.668240in}}%
\pgfpathlineto{\pgfqpoint{6.420437in}{2.671956in}}%
\pgfpathlineto{\pgfqpoint{6.421510in}{2.672575in}}%
\pgfpathlineto{\pgfqpoint{6.422584in}{2.677901in}}%
\pgfpathlineto{\pgfqpoint{6.423658in}{2.676085in}}%
\pgfpathlineto{\pgfqpoint{6.427953in}{2.679553in}}%
\pgfpathlineto{\pgfqpoint{6.430101in}{2.678603in}}%
\pgfpathlineto{\pgfqpoint{6.431175in}{2.677860in}}%
\pgfpathlineto{\pgfqpoint{6.436544in}{2.682732in}}%
\pgfpathlineto{\pgfqpoint{6.437618in}{2.682030in}}%
\pgfpathlineto{\pgfqpoint{6.438692in}{2.680461in}}%
\pgfpathlineto{\pgfqpoint{6.441913in}{2.681659in}}%
\pgfpathlineto{\pgfqpoint{6.444061in}{2.684755in}}%
\pgfpathlineto{\pgfqpoint{6.445135in}{2.683310in}}%
\pgfpathlineto{\pgfqpoint{6.446209in}{2.684838in}}%
\pgfpathlineto{\pgfqpoint{6.449430in}{2.686365in}}%
\pgfpathlineto{\pgfqpoint{6.450504in}{2.686076in}}%
\pgfpathlineto{\pgfqpoint{6.452652in}{2.684095in}}%
\pgfpathlineto{\pgfqpoint{6.453726in}{2.685003in}}%
\pgfpathlineto{\pgfqpoint{6.458021in}{2.684260in}}%
\pgfpathlineto{\pgfqpoint{6.459095in}{2.682526in}}%
\pgfpathlineto{\pgfqpoint{6.461242in}{2.685086in}}%
\pgfpathlineto{\pgfqpoint{6.466612in}{2.685829in}}%
\pgfpathlineto{\pgfqpoint{6.467685in}{2.686778in}}%
\pgfpathlineto{\pgfqpoint{6.468759in}{2.686283in}}%
\pgfpathlineto{\pgfqpoint{6.471981in}{2.690742in}}%
\pgfpathlineto{\pgfqpoint{6.473055in}{2.688884in}}%
\pgfpathlineto{\pgfqpoint{6.475202in}{2.689049in}}%
\pgfpathlineto{\pgfqpoint{6.476276in}{2.687934in}}%
\pgfpathlineto{\pgfqpoint{6.479498in}{2.687480in}}%
\pgfpathlineto{\pgfqpoint{6.480572in}{2.691444in}}%
\pgfpathlineto{\pgfqpoint{6.481645in}{2.692311in}}%
\pgfpathlineto{\pgfqpoint{6.482719in}{2.685498in}}%
\pgfpathlineto{\pgfqpoint{6.483793in}{2.683971in}}%
\pgfpathlineto{\pgfqpoint{6.487015in}{2.685829in}}%
\pgfpathlineto{\pgfqpoint{6.488088in}{2.685540in}}%
\pgfpathlineto{\pgfqpoint{6.489162in}{2.678645in}}%
\pgfpathlineto{\pgfqpoint{6.491310in}{2.679057in}}%
\pgfpathlineto{\pgfqpoint{6.496679in}{2.684673in}}%
\pgfpathlineto{\pgfqpoint{6.497753in}{2.683145in}}%
\pgfpathlineto{\pgfqpoint{6.498827in}{2.684301in}}%
\pgfpathlineto{\pgfqpoint{6.502048in}{2.683475in}}%
\pgfpathlineto{\pgfqpoint{6.503122in}{2.681576in}}%
\pgfpathlineto{\pgfqpoint{6.504196in}{2.680957in}}%
\pgfpathlineto{\pgfqpoint{6.506344in}{2.687067in}}%
\pgfpathlineto{\pgfqpoint{6.509565in}{2.687439in}}%
\pgfpathlineto{\pgfqpoint{6.510639in}{2.686118in}}%
\pgfpathlineto{\pgfqpoint{6.511713in}{2.685994in}}%
\pgfpathlineto{\pgfqpoint{6.512787in}{2.684095in}}%
\pgfpathlineto{\pgfqpoint{6.513861in}{2.671048in}}%
\pgfpathlineto{\pgfqpoint{6.519230in}{2.665639in}}%
\pgfpathlineto{\pgfqpoint{6.520304in}{2.668116in}}%
\pgfpathlineto{\pgfqpoint{6.521377in}{2.666341in}}%
\pgfpathlineto{\pgfqpoint{6.524599in}{2.663327in}}%
\pgfpathlineto{\pgfqpoint{6.525673in}{2.663616in}}%
\pgfpathlineto{\pgfqpoint{6.526747in}{2.665763in}}%
\pgfpathlineto{\pgfqpoint{6.527820in}{2.664276in}}%
\pgfpathlineto{\pgfqpoint{6.528894in}{2.664524in}}%
\pgfpathlineto{\pgfqpoint{6.532116in}{2.662460in}}%
\pgfpathlineto{\pgfqpoint{6.534263in}{2.668446in}}%
\pgfpathlineto{\pgfqpoint{6.535337in}{2.669231in}}%
\pgfpathlineto{\pgfqpoint{6.536411in}{2.670717in}}%
\pgfpathlineto{\pgfqpoint{6.539633in}{2.673979in}}%
\pgfpathlineto{\pgfqpoint{6.540706in}{2.673484in}}%
\pgfpathlineto{\pgfqpoint{6.541780in}{2.670965in}}%
\pgfpathlineto{\pgfqpoint{6.542854in}{2.674970in}}%
\pgfpathlineto{\pgfqpoint{6.543928in}{2.671749in}}%
\pgfpathlineto{\pgfqpoint{6.547150in}{2.671130in}}%
\pgfpathlineto{\pgfqpoint{6.548223in}{2.672906in}}%
\pgfpathlineto{\pgfqpoint{6.549297in}{2.671378in}}%
\pgfpathlineto{\pgfqpoint{6.551445in}{2.671832in}}%
\pgfpathlineto{\pgfqpoint{6.554666in}{2.673814in}}%
\pgfpathlineto{\pgfqpoint{6.555740in}{2.675548in}}%
\pgfpathlineto{\pgfqpoint{6.556814in}{2.675465in}}%
\pgfpathlineto{\pgfqpoint{6.558962in}{2.679305in}}%
\pgfpathlineto{\pgfqpoint{6.562183in}{2.683393in}}%
\pgfpathlineto{\pgfqpoint{6.564331in}{2.682773in}}%
\pgfpathlineto{\pgfqpoint{6.565405in}{2.678273in}}%
\pgfpathlineto{\pgfqpoint{6.566479in}{2.679346in}}%
\pgfpathlineto{\pgfqpoint{6.569700in}{2.678810in}}%
\pgfpathlineto{\pgfqpoint{6.570774in}{2.677323in}}%
\pgfpathlineto{\pgfqpoint{6.571848in}{2.681328in}}%
\pgfpathlineto{\pgfqpoint{6.572922in}{2.681782in}}%
\pgfpathlineto{\pgfqpoint{6.573995in}{2.685251in}}%
\pgfpathlineto{\pgfqpoint{6.577217in}{2.685251in}}%
\pgfpathlineto{\pgfqpoint{6.579365in}{2.683847in}}%
\pgfpathlineto{\pgfqpoint{6.580439in}{2.684425in}}%
\pgfpathlineto{\pgfqpoint{6.581512in}{2.686200in}}%
\pgfpathlineto{\pgfqpoint{6.585808in}{2.687563in}}%
\pgfpathlineto{\pgfqpoint{6.586882in}{2.686076in}}%
\pgfpathlineto{\pgfqpoint{6.587955in}{2.685953in}}%
\pgfpathlineto{\pgfqpoint{6.589029in}{2.685251in}}%
\pgfpathlineto{\pgfqpoint{6.589029in}{2.685251in}}%
\pgfusepath{stroke}%
\end{pgfscope}%
\begin{pgfscope}%
\pgfpathrectangle{\pgfqpoint{4.123120in}{2.309648in}}{\pgfqpoint{2.583333in}{0.400885in}}%
\pgfusepath{clip}%
\pgfsetroundcap%
\pgfsetroundjoin%
\pgfsetlinewidth{1.505625pt}%
\definecolor{currentstroke}{rgb}{0.549020,0.337255,0.294118}%
\pgfsetstrokecolor{currentstroke}%
\pgfsetdash{}{0pt}%
\pgfpathmoveto{\pgfqpoint{4.240544in}{2.542270in}}%
\pgfpathlineto{\pgfqpoint{4.241618in}{2.542146in}}%
\pgfpathlineto{\pgfqpoint{4.243766in}{2.540742in}}%
\pgfpathlineto{\pgfqpoint{4.246987in}{2.539834in}}%
\pgfpathlineto{\pgfqpoint{4.248061in}{2.538810in}}%
\pgfpathlineto{\pgfqpoint{4.249135in}{2.536763in}}%
\pgfpathlineto{\pgfqpoint{4.251283in}{2.537173in}}%
\pgfpathlineto{\pgfqpoint{4.257726in}{2.534971in}}%
\pgfpathlineto{\pgfqpoint{4.258800in}{2.535451in}}%
\pgfpathlineto{\pgfqpoint{4.269538in}{2.522965in}}%
\pgfpathlineto{\pgfqpoint{4.270612in}{2.522453in}}%
\pgfpathlineto{\pgfqpoint{4.272760in}{2.523320in}}%
\pgfpathlineto{\pgfqpoint{4.273833in}{2.521585in}}%
\pgfpathlineto{\pgfqpoint{4.277055in}{2.519258in}}%
\pgfpathlineto{\pgfqpoint{4.278129in}{2.519836in}}%
\pgfpathlineto{\pgfqpoint{4.279203in}{2.519643in}}%
\pgfpathlineto{\pgfqpoint{4.281350in}{2.517993in}}%
\pgfpathlineto{\pgfqpoint{4.286719in}{2.520009in}}%
\pgfpathlineto{\pgfqpoint{4.288867in}{2.517212in}}%
\pgfpathlineto{\pgfqpoint{4.294236in}{2.515833in}}%
\pgfpathlineto{\pgfqpoint{4.295310in}{2.510018in}}%
\pgfpathlineto{\pgfqpoint{4.296384in}{2.510824in}}%
\pgfpathlineto{\pgfqpoint{4.299606in}{2.510789in}}%
\pgfpathlineto{\pgfqpoint{4.300679in}{2.508896in}}%
\pgfpathlineto{\pgfqpoint{4.301753in}{2.509548in}}%
\pgfpathlineto{\pgfqpoint{4.302827in}{2.506938in}}%
\pgfpathlineto{\pgfqpoint{4.303901in}{2.506973in}}%
\pgfpathlineto{\pgfqpoint{4.308196in}{2.507419in}}%
\pgfpathlineto{\pgfqpoint{4.309270in}{2.506732in}}%
\pgfpathlineto{\pgfqpoint{4.310344in}{2.507591in}}%
\pgfpathlineto{\pgfqpoint{4.311418in}{2.507659in}}%
\pgfpathlineto{\pgfqpoint{4.314639in}{2.505530in}}%
\pgfpathlineto{\pgfqpoint{4.315713in}{2.506033in}}%
\pgfpathlineto{\pgfqpoint{4.317861in}{2.505430in}}%
\pgfpathlineto{\pgfqpoint{4.318935in}{2.504289in}}%
\pgfpathlineto{\pgfqpoint{4.324304in}{2.504155in}}%
\pgfpathlineto{\pgfqpoint{4.325378in}{2.505027in}}%
\pgfpathlineto{\pgfqpoint{4.326451in}{2.504792in}}%
\pgfpathlineto{\pgfqpoint{4.329673in}{2.504859in}}%
\pgfpathlineto{\pgfqpoint{4.330747in}{2.504054in}}%
\pgfpathlineto{\pgfqpoint{4.331821in}{2.504155in}}%
\pgfpathlineto{\pgfqpoint{4.332894in}{2.503652in}}%
\pgfpathlineto{\pgfqpoint{4.333968in}{2.504189in}}%
\pgfpathlineto{\pgfqpoint{4.337190in}{2.505128in}}%
\pgfpathlineto{\pgfqpoint{4.338264in}{2.503887in}}%
\pgfpathlineto{\pgfqpoint{4.340411in}{2.504457in}}%
\pgfpathlineto{\pgfqpoint{4.344707in}{2.503149in}}%
\pgfpathlineto{\pgfqpoint{4.345781in}{2.501908in}}%
\pgfpathlineto{\pgfqpoint{4.346854in}{2.502176in}}%
\pgfpathlineto{\pgfqpoint{4.352224in}{2.497949in}}%
\pgfpathlineto{\pgfqpoint{4.353297in}{2.498568in}}%
\pgfpathlineto{\pgfqpoint{4.355445in}{2.497428in}}%
\pgfpathlineto{\pgfqpoint{4.356519in}{2.495018in}}%
\pgfpathlineto{\pgfqpoint{4.359740in}{2.492864in}}%
\pgfpathlineto{\pgfqpoint{4.362962in}{2.494828in}}%
\pgfpathlineto{\pgfqpoint{4.364036in}{2.488715in}}%
\pgfpathlineto{\pgfqpoint{4.368331in}{2.486530in}}%
\pgfpathlineto{\pgfqpoint{4.369405in}{2.487544in}}%
\pgfpathlineto{\pgfqpoint{4.371553in}{2.485653in}}%
\pgfpathlineto{\pgfqpoint{4.375848in}{2.485373in}}%
\pgfpathlineto{\pgfqpoint{4.376922in}{2.484158in}}%
\pgfpathlineto{\pgfqpoint{4.377996in}{2.485310in}}%
\pgfpathlineto{\pgfqpoint{4.379070in}{2.484189in}}%
\pgfpathlineto{\pgfqpoint{4.383365in}{2.484282in}}%
\pgfpathlineto{\pgfqpoint{4.385513in}{2.482085in}}%
\pgfpathlineto{\pgfqpoint{4.386586in}{2.481014in}}%
\pgfpathlineto{\pgfqpoint{4.390882in}{2.480096in}}%
\pgfpathlineto{\pgfqpoint{4.391956in}{2.478260in}}%
\pgfpathlineto{\pgfqpoint{4.393029in}{2.478688in}}%
\pgfpathlineto{\pgfqpoint{4.398399in}{2.477372in}}%
\pgfpathlineto{\pgfqpoint{4.399472in}{2.475835in}}%
\pgfpathlineto{\pgfqpoint{4.400546in}{2.475774in}}%
\pgfpathlineto{\pgfqpoint{4.401620in}{2.473996in}}%
\pgfpathlineto{\pgfqpoint{4.405916in}{2.473091in}}%
\pgfpathlineto{\pgfqpoint{4.409137in}{2.469339in}}%
\pgfpathlineto{\pgfqpoint{4.412359in}{2.468852in}}%
\pgfpathlineto{\pgfqpoint{4.413432in}{2.469339in}}%
\pgfpathlineto{\pgfqpoint{4.414506in}{2.468910in}}%
\pgfpathlineto{\pgfqpoint{4.416654in}{2.466833in}}%
\pgfpathlineto{\pgfqpoint{4.420949in}{2.465345in}}%
\pgfpathlineto{\pgfqpoint{4.422023in}{2.461274in}}%
\pgfpathlineto{\pgfqpoint{4.423097in}{2.459842in}}%
\pgfpathlineto{\pgfqpoint{4.424171in}{2.460038in}}%
\pgfpathlineto{\pgfqpoint{4.428466in}{2.458775in}}%
\pgfpathlineto{\pgfqpoint{4.429540in}{2.457175in}}%
\pgfpathlineto{\pgfqpoint{4.430614in}{2.457832in}}%
\pgfpathlineto{\pgfqpoint{4.431688in}{2.455750in}}%
\pgfpathlineto{\pgfqpoint{4.439205in}{2.455829in}}%
\pgfpathlineto{\pgfqpoint{4.443500in}{2.456785in}}%
\pgfpathlineto{\pgfqpoint{4.444574in}{2.456068in}}%
\pgfpathlineto{\pgfqpoint{4.446721in}{2.448476in}}%
\pgfpathlineto{\pgfqpoint{4.449943in}{2.447934in}}%
\pgfpathlineto{\pgfqpoint{4.451017in}{2.448948in}}%
\pgfpathlineto{\pgfqpoint{4.454238in}{2.448853in}}%
\pgfpathlineto{\pgfqpoint{4.459607in}{2.447510in}}%
\pgfpathlineto{\pgfqpoint{4.461755in}{2.449513in}}%
\pgfpathlineto{\pgfqpoint{4.464977in}{2.449537in}}%
\pgfpathlineto{\pgfqpoint{4.468198in}{2.446543in}}%
\pgfpathlineto{\pgfqpoint{4.469272in}{2.442795in}}%
\pgfpathlineto{\pgfqpoint{4.473567in}{2.444080in}}%
\pgfpathlineto{\pgfqpoint{4.474641in}{2.444965in}}%
\pgfpathlineto{\pgfqpoint{4.499339in}{2.445807in}}%
\pgfpathlineto{\pgfqpoint{4.504709in}{2.445984in}}%
\pgfpathlineto{\pgfqpoint{4.505782in}{2.444301in}}%
\pgfpathlineto{\pgfqpoint{4.506856in}{2.444775in}}%
\pgfpathlineto{\pgfqpoint{4.511152in}{2.444344in}}%
\pgfpathlineto{\pgfqpoint{4.512226in}{2.444064in}}%
\pgfpathlineto{\pgfqpoint{4.513299in}{2.442665in}}%
\pgfpathlineto{\pgfqpoint{4.514373in}{2.443085in}}%
\pgfpathlineto{\pgfqpoint{4.532628in}{2.443569in}}%
\pgfpathlineto{\pgfqpoint{4.533702in}{2.442455in}}%
\pgfpathlineto{\pgfqpoint{4.536924in}{2.443863in}}%
\pgfpathlineto{\pgfqpoint{4.541219in}{2.442308in}}%
\pgfpathlineto{\pgfqpoint{4.543367in}{2.441130in}}%
\pgfpathlineto{\pgfqpoint{4.544441in}{2.441025in}}%
\pgfpathlineto{\pgfqpoint{4.547662in}{2.439722in}}%
\pgfpathlineto{\pgfqpoint{4.548736in}{2.440194in}}%
\pgfpathlineto{\pgfqpoint{4.549810in}{2.438489in}}%
\pgfpathlineto{\pgfqpoint{4.550884in}{2.438489in}}%
\pgfpathlineto{\pgfqpoint{4.551958in}{2.437053in}}%
\pgfpathlineto{\pgfqpoint{4.555179in}{2.436933in}}%
\pgfpathlineto{\pgfqpoint{4.559474in}{2.430091in}}%
\pgfpathlineto{\pgfqpoint{4.571287in}{2.429334in}}%
\pgfpathlineto{\pgfqpoint{4.573434in}{2.426308in}}%
\pgfpathlineto{\pgfqpoint{4.577730in}{2.426566in}}%
\pgfpathlineto{\pgfqpoint{4.580951in}{2.425441in}}%
\pgfpathlineto{\pgfqpoint{4.582025in}{2.426179in}}%
\pgfpathlineto{\pgfqpoint{4.585247in}{2.424537in}}%
\pgfpathlineto{\pgfqpoint{4.587394in}{2.425322in}}%
\pgfpathlineto{\pgfqpoint{4.589542in}{2.423698in}}%
\pgfpathlineto{\pgfqpoint{4.592763in}{2.423542in}}%
\pgfpathlineto{\pgfqpoint{4.593837in}{2.422851in}}%
\pgfpathlineto{\pgfqpoint{4.595985in}{2.423577in}}%
\pgfpathlineto{\pgfqpoint{4.597059in}{2.424026in}}%
\pgfpathlineto{\pgfqpoint{4.602428in}{2.423438in}}%
\pgfpathlineto{\pgfqpoint{4.604576in}{2.424682in}}%
\pgfpathlineto{\pgfqpoint{4.607797in}{2.424579in}}%
\pgfpathlineto{\pgfqpoint{4.609945in}{2.425339in}}%
\pgfpathlineto{\pgfqpoint{4.612093in}{2.424164in}}%
\pgfpathlineto{\pgfqpoint{4.616388in}{2.424233in}}%
\pgfpathlineto{\pgfqpoint{4.617462in}{2.423352in}}%
\pgfpathlineto{\pgfqpoint{4.618536in}{2.424026in}}%
\pgfpathlineto{\pgfqpoint{4.619609in}{2.422488in}}%
\pgfpathlineto{\pgfqpoint{4.626052in}{2.421434in}}%
\pgfpathlineto{\pgfqpoint{4.627126in}{2.420294in}}%
\pgfpathlineto{\pgfqpoint{4.630348in}{2.419258in}}%
\pgfpathlineto{\pgfqpoint{4.633569in}{2.416631in}}%
\pgfpathlineto{\pgfqpoint{4.634643in}{2.416825in}}%
\pgfpathlineto{\pgfqpoint{4.638938in}{2.416065in}}%
\pgfpathlineto{\pgfqpoint{4.642160in}{2.416987in}}%
\pgfpathlineto{\pgfqpoint{4.653972in}{2.418684in}}%
\pgfpathlineto{\pgfqpoint{4.656120in}{2.417366in}}%
\pgfpathlineto{\pgfqpoint{4.657194in}{2.413709in}}%
\pgfpathlineto{\pgfqpoint{4.660415in}{2.414322in}}%
\pgfpathlineto{\pgfqpoint{4.661489in}{2.412862in}}%
\pgfpathlineto{\pgfqpoint{4.663637in}{2.413045in}}%
\pgfpathlineto{\pgfqpoint{4.664711in}{2.412169in}}%
\pgfpathlineto{\pgfqpoint{4.667932in}{2.411422in}}%
\pgfpathlineto{\pgfqpoint{4.671154in}{2.412432in}}%
\pgfpathlineto{\pgfqpoint{4.672227in}{2.411976in}}%
\pgfpathlineto{\pgfqpoint{4.676523in}{2.412239in}}%
\pgfpathlineto{\pgfqpoint{4.678670in}{2.413139in}}%
\pgfpathlineto{\pgfqpoint{4.691557in}{2.410498in}}%
\pgfpathlineto{\pgfqpoint{4.692630in}{2.411243in}}%
\pgfpathlineto{\pgfqpoint{4.693704in}{2.410607in}}%
\pgfpathlineto{\pgfqpoint{4.694778in}{2.410959in}}%
\pgfpathlineto{\pgfqpoint{4.708738in}{2.411948in}}%
\pgfpathlineto{\pgfqpoint{4.709812in}{2.410783in}}%
\pgfpathlineto{\pgfqpoint{4.713033in}{2.410593in}}%
\pgfpathlineto{\pgfqpoint{4.714107in}{2.409550in}}%
\pgfpathlineto{\pgfqpoint{4.715181in}{2.410065in}}%
\pgfpathlineto{\pgfqpoint{4.717329in}{2.409722in}}%
\pgfpathlineto{\pgfqpoint{4.720550in}{2.409087in}}%
\pgfpathlineto{\pgfqpoint{4.722698in}{2.407945in}}%
\pgfpathlineto{\pgfqpoint{4.723772in}{2.407945in}}%
\pgfpathlineto{\pgfqpoint{4.728067in}{2.408619in}}%
\pgfpathlineto{\pgfqpoint{4.730215in}{2.406430in}}%
\pgfpathlineto{\pgfqpoint{4.731289in}{2.406857in}}%
\pgfpathlineto{\pgfqpoint{4.732362in}{2.406530in}}%
\pgfpathlineto{\pgfqpoint{4.735584in}{2.407108in}}%
\pgfpathlineto{\pgfqpoint{4.738805in}{2.405992in}}%
\pgfpathlineto{\pgfqpoint{4.739879in}{2.406396in}}%
\pgfpathlineto{\pgfqpoint{4.743101in}{2.405968in}}%
\pgfpathlineto{\pgfqpoint{4.744175in}{2.406421in}}%
\pgfpathlineto{\pgfqpoint{4.747396in}{2.403049in}}%
\pgfpathlineto{\pgfqpoint{4.750618in}{2.403026in}}%
\pgfpathlineto{\pgfqpoint{4.751692in}{2.401954in}}%
\pgfpathlineto{\pgfqpoint{4.752765in}{2.397513in}}%
\pgfpathlineto{\pgfqpoint{4.753839in}{2.397019in}}%
\pgfpathlineto{\pgfqpoint{4.754913in}{2.397502in}}%
\pgfpathlineto{\pgfqpoint{4.758135in}{2.398029in}}%
\pgfpathlineto{\pgfqpoint{4.759208in}{2.397188in}}%
\pgfpathlineto{\pgfqpoint{4.760282in}{2.397390in}}%
\pgfpathlineto{\pgfqpoint{4.761356in}{2.396672in}}%
\pgfpathlineto{\pgfqpoint{4.762430in}{2.397056in}}%
\pgfpathlineto{\pgfqpoint{4.766725in}{2.396826in}}%
\pgfpathlineto{\pgfqpoint{4.767799in}{2.397276in}}%
\pgfpathlineto{\pgfqpoint{4.768873in}{2.397078in}}%
\pgfpathlineto{\pgfqpoint{4.769947in}{2.397573in}}%
\pgfpathlineto{\pgfqpoint{4.773168in}{2.397419in}}%
\pgfpathlineto{\pgfqpoint{4.776390in}{2.395157in}}%
\pgfpathlineto{\pgfqpoint{4.783907in}{2.393875in}}%
\pgfpathlineto{\pgfqpoint{4.784981in}{2.391154in}}%
\pgfpathlineto{\pgfqpoint{4.789276in}{2.390346in}}%
\pgfpathlineto{\pgfqpoint{4.790350in}{2.388798in}}%
\pgfpathlineto{\pgfqpoint{4.791424in}{2.388952in}}%
\pgfpathlineto{\pgfqpoint{4.792497in}{2.387116in}}%
\pgfpathlineto{\pgfqpoint{4.803236in}{2.385299in}}%
\pgfpathlineto{\pgfqpoint{4.805383in}{2.384951in}}%
\pgfpathlineto{\pgfqpoint{4.807531in}{2.384031in}}%
\pgfpathlineto{\pgfqpoint{4.812900in}{2.382375in}}%
\pgfpathlineto{\pgfqpoint{4.815048in}{2.379114in}}%
\pgfpathlineto{\pgfqpoint{4.819343in}{2.378610in}}%
\pgfpathlineto{\pgfqpoint{4.821491in}{2.379271in}}%
\pgfpathlineto{\pgfqpoint{4.822565in}{2.378808in}}%
\pgfpathlineto{\pgfqpoint{4.826860in}{2.378379in}}%
\pgfpathlineto{\pgfqpoint{4.833303in}{2.378596in}}%
\pgfpathlineto{\pgfqpoint{4.834377in}{2.378065in}}%
\pgfpathlineto{\pgfqpoint{4.835451in}{2.378215in}}%
\pgfpathlineto{\pgfqpoint{4.836525in}{2.377583in}}%
\pgfpathlineto{\pgfqpoint{4.837599in}{2.378076in}}%
\pgfpathlineto{\pgfqpoint{4.844042in}{2.377660in}}%
\pgfpathlineto{\pgfqpoint{4.845115in}{2.376975in}}%
\pgfpathlineto{\pgfqpoint{4.858002in}{2.376308in}}%
\pgfpathlineto{\pgfqpoint{4.860149in}{2.375265in}}%
\pgfpathlineto{\pgfqpoint{4.873035in}{2.375244in}}%
\pgfpathlineto{\pgfqpoint{4.875183in}{2.374432in}}%
\pgfpathlineto{\pgfqpoint{4.886995in}{2.373279in}}%
\pgfpathlineto{\pgfqpoint{4.888069in}{2.372605in}}%
\pgfpathlineto{\pgfqpoint{4.890217in}{2.373228in}}%
\pgfpathlineto{\pgfqpoint{4.897734in}{2.372786in}}%
\pgfpathlineto{\pgfqpoint{4.903103in}{2.372241in}}%
\pgfpathlineto{\pgfqpoint{4.905250in}{2.371774in}}%
\pgfpathlineto{\pgfqpoint{4.910620in}{2.371212in}}%
\pgfpathlineto{\pgfqpoint{4.917063in}{2.370283in}}%
\pgfpathlineto{\pgfqpoint{4.918136in}{2.369779in}}%
\pgfpathlineto{\pgfqpoint{4.919210in}{2.369960in}}%
\pgfpathlineto{\pgfqpoint{4.923506in}{2.368594in}}%
\pgfpathlineto{\pgfqpoint{4.925653in}{2.368783in}}%
\pgfpathlineto{\pgfqpoint{4.932096in}{2.369005in}}%
\pgfpathlineto{\pgfqpoint{4.933170in}{2.369348in}}%
\pgfpathlineto{\pgfqpoint{4.934244in}{2.368823in}}%
\pgfpathlineto{\pgfqpoint{4.935318in}{2.369145in}}%
\pgfpathlineto{\pgfqpoint{4.938539in}{2.369283in}}%
\pgfpathlineto{\pgfqpoint{4.942835in}{2.367353in}}%
\pgfpathlineto{\pgfqpoint{4.946056in}{2.367131in}}%
\pgfpathlineto{\pgfqpoint{4.947130in}{2.366410in}}%
\pgfpathlineto{\pgfqpoint{4.949278in}{2.366523in}}%
\pgfpathlineto{\pgfqpoint{4.957869in}{2.364407in}}%
\pgfpathlineto{\pgfqpoint{4.962164in}{2.364536in}}%
\pgfpathlineto{\pgfqpoint{4.964312in}{2.363675in}}%
\pgfpathlineto{\pgfqpoint{4.984714in}{2.364720in}}%
\pgfpathlineto{\pgfqpoint{4.991158in}{2.364135in}}%
\pgfpathlineto{\pgfqpoint{4.993305in}{2.364135in}}%
\pgfpathlineto{\pgfqpoint{4.999748in}{2.362631in}}%
\pgfpathlineto{\pgfqpoint{5.000822in}{2.362792in}}%
\pgfpathlineto{\pgfqpoint{5.001896in}{2.362051in}}%
\pgfpathlineto{\pgfqpoint{5.006191in}{2.361787in}}%
\pgfpathlineto{\pgfqpoint{5.010487in}{2.360589in}}%
\pgfpathlineto{\pgfqpoint{5.016930in}{2.360554in}}%
\pgfpathlineto{\pgfqpoint{5.021225in}{2.360654in}}%
\pgfpathlineto{\pgfqpoint{5.025520in}{2.360010in}}%
\pgfpathlineto{\pgfqpoint{5.036259in}{2.359825in}}%
\pgfpathlineto{\pgfqpoint{5.038406in}{2.359431in}}%
\pgfpathlineto{\pgfqpoint{5.053440in}{2.357634in}}%
\pgfpathlineto{\pgfqpoint{5.055588in}{2.357223in}}%
\pgfpathlineto{\pgfqpoint{5.060957in}{2.357157in}}%
\pgfpathlineto{\pgfqpoint{5.066326in}{2.357786in}}%
\pgfpathlineto{\pgfqpoint{5.070622in}{2.356429in}}%
\pgfpathlineto{\pgfqpoint{5.078138in}{2.355891in}}%
\pgfpathlineto{\pgfqpoint{5.089951in}{2.355340in}}%
\pgfpathlineto{\pgfqpoint{5.092098in}{2.355196in}}%
\pgfpathlineto{\pgfqpoint{5.099615in}{2.355611in}}%
\pgfpathlineto{\pgfqpoint{5.108206in}{2.354516in}}%
\pgfpathlineto{\pgfqpoint{5.113575in}{2.354077in}}%
\pgfpathlineto{\pgfqpoint{5.118944in}{2.353879in}}%
\pgfpathlineto{\pgfqpoint{5.123240in}{2.353596in}}%
\pgfpathlineto{\pgfqpoint{5.128609in}{2.353599in}}%
\pgfpathlineto{\pgfqpoint{5.133978in}{2.353373in}}%
\pgfpathlineto{\pgfqpoint{5.143643in}{2.352903in}}%
\pgfpathlineto{\pgfqpoint{5.145790in}{2.352727in}}%
\pgfpathlineto{\pgfqpoint{5.153307in}{2.351922in}}%
\pgfpathlineto{\pgfqpoint{5.166193in}{2.351101in}}%
\pgfpathlineto{\pgfqpoint{5.168341in}{2.350847in}}%
\pgfpathlineto{\pgfqpoint{5.182301in}{2.350869in}}%
\pgfpathlineto{\pgfqpoint{5.186596in}{2.350858in}}%
\pgfpathlineto{\pgfqpoint{5.190891in}{2.350758in}}%
\pgfpathlineto{\pgfqpoint{5.213442in}{2.350450in}}%
\pgfpathlineto{\pgfqpoint{5.219885in}{2.350743in}}%
\pgfpathlineto{\pgfqpoint{5.233845in}{2.350558in}}%
\pgfpathlineto{\pgfqpoint{5.235993in}{2.350440in}}%
\pgfpathlineto{\pgfqpoint{5.248879in}{2.349730in}}%
\pgfpathlineto{\pgfqpoint{5.258543in}{2.347783in}}%
\pgfpathlineto{\pgfqpoint{5.281094in}{2.348352in}}%
\pgfpathlineto{\pgfqpoint{5.322974in}{2.347548in}}%
\pgfpathlineto{\pgfqpoint{5.326195in}{2.346912in}}%
\pgfpathlineto{\pgfqpoint{5.346598in}{2.345295in}}%
\pgfpathlineto{\pgfqpoint{5.351967in}{2.345476in}}%
\pgfpathlineto{\pgfqpoint{5.354115in}{2.345280in}}%
\pgfpathlineto{\pgfqpoint{5.399216in}{2.344051in}}%
\pgfpathlineto{\pgfqpoint{5.401364in}{2.343810in}}%
\pgfpathlineto{\pgfqpoint{5.413176in}{2.343876in}}%
\pgfpathlineto{\pgfqpoint{5.416398in}{2.343507in}}%
\pgfpathlineto{\pgfqpoint{5.437874in}{2.342618in}}%
\pgfpathlineto{\pgfqpoint{5.438948in}{2.342346in}}%
\pgfpathlineto{\pgfqpoint{5.443244in}{2.341716in}}%
\pgfpathlineto{\pgfqpoint{5.446465in}{2.341331in}}%
\pgfpathlineto{\pgfqpoint{5.473311in}{2.340662in}}%
\pgfpathlineto{\pgfqpoint{5.488345in}{2.340008in}}%
\pgfpathlineto{\pgfqpoint{5.503378in}{2.339122in}}%
\pgfpathlineto{\pgfqpoint{5.506600in}{2.338881in}}%
\pgfpathlineto{\pgfqpoint{5.534520in}{2.338997in}}%
\pgfpathlineto{\pgfqpoint{5.536667in}{2.338782in}}%
\pgfpathlineto{\pgfqpoint{5.593581in}{2.337183in}}%
\pgfpathlineto{\pgfqpoint{5.601098in}{2.336821in}}%
\pgfpathlineto{\pgfqpoint{5.678414in}{2.333788in}}%
\pgfpathlineto{\pgfqpoint{5.685931in}{2.333375in}}%
\pgfpathlineto{\pgfqpoint{5.699891in}{2.333030in}}%
\pgfpathlineto{\pgfqpoint{5.708482in}{2.332748in}}%
\pgfpathlineto{\pgfqpoint{5.724589in}{2.332863in}}%
\pgfpathlineto{\pgfqpoint{5.746066in}{2.332472in}}%
\pgfpathlineto{\pgfqpoint{5.761100in}{2.332124in}}%
\pgfpathlineto{\pgfqpoint{5.806201in}{2.331431in}}%
\pgfpathlineto{\pgfqpoint{5.964055in}{2.329618in}}%
\pgfpathlineto{\pgfqpoint{6.017747in}{2.329469in}}%
\pgfpathlineto{\pgfqpoint{6.136943in}{2.328832in}}%
\pgfpathlineto{\pgfqpoint{6.157346in}{2.328612in}}%
\pgfpathlineto{\pgfqpoint{6.589029in}{2.327875in}}%
\pgfpathlineto{\pgfqpoint{6.589029in}{2.327875in}}%
\pgfusepath{stroke}%
\end{pgfscope}%
\begin{pgfscope}%
\pgfsetrectcap%
\pgfsetmiterjoin%
\pgfsetlinewidth{0.803000pt}%
\definecolor{currentstroke}{rgb}{1.000000,1.000000,1.000000}%
\pgfsetstrokecolor{currentstroke}%
\pgfsetdash{}{0pt}%
\pgfpathmoveto{\pgfqpoint{4.123120in}{2.309648in}}%
\pgfpathlineto{\pgfqpoint{4.123120in}{2.710533in}}%
\pgfusepath{stroke}%
\end{pgfscope}%
\begin{pgfscope}%
\pgfsetrectcap%
\pgfsetmiterjoin%
\pgfsetlinewidth{0.803000pt}%
\definecolor{currentstroke}{rgb}{1.000000,1.000000,1.000000}%
\pgfsetstrokecolor{currentstroke}%
\pgfsetdash{}{0pt}%
\pgfpathmoveto{\pgfqpoint{6.706453in}{2.309648in}}%
\pgfpathlineto{\pgfqpoint{6.706453in}{2.710533in}}%
\pgfusepath{stroke}%
\end{pgfscope}%
\begin{pgfscope}%
\pgfsetrectcap%
\pgfsetmiterjoin%
\pgfsetlinewidth{0.803000pt}%
\definecolor{currentstroke}{rgb}{1.000000,1.000000,1.000000}%
\pgfsetstrokecolor{currentstroke}%
\pgfsetdash{}{0pt}%
\pgfpathmoveto{\pgfqpoint{4.123120in}{2.309648in}}%
\pgfpathlineto{\pgfqpoint{6.706453in}{2.309648in}}%
\pgfusepath{stroke}%
\end{pgfscope}%
\begin{pgfscope}%
\pgfsetrectcap%
\pgfsetmiterjoin%
\pgfsetlinewidth{0.803000pt}%
\definecolor{currentstroke}{rgb}{1.000000,1.000000,1.000000}%
\pgfsetstrokecolor{currentstroke}%
\pgfsetdash{}{0pt}%
\pgfpathmoveto{\pgfqpoint{4.123120in}{2.710533in}}%
\pgfpathlineto{\pgfqpoint{6.706453in}{2.710533in}}%
\pgfusepath{stroke}%
\end{pgfscope}%
\begin{pgfscope}%
\definecolor{textcolor}{rgb}{0.150000,0.150000,0.150000}%
\pgfsetstrokecolor{textcolor}%
\pgfsetfillcolor{textcolor}%
\pgftext[x=5.414787in,y=2.793866in,,base]{\color{textcolor}\rmfamily\fontsize{16.800000}{20.160000}\selectfont PG}%
\end{pgfscope}%
\begin{pgfscope}%
\pgfsetbuttcap%
\pgfsetmiterjoin%
\definecolor{currentfill}{rgb}{0.917647,0.917647,0.949020}%
\pgfsetfillcolor{currentfill}%
\pgfsetlinewidth{0.000000pt}%
\definecolor{currentstroke}{rgb}{0.000000,0.000000,0.000000}%
\pgfsetstrokecolor{currentstroke}%
\pgfsetstrokeopacity{0.000000}%
\pgfsetdash{}{0pt}%
\pgfpathmoveto{\pgfqpoint{0.506453in}{1.347524in}}%
\pgfpathlineto{\pgfqpoint{3.089787in}{1.347524in}}%
\pgfpathlineto{\pgfqpoint{3.089787in}{1.748409in}}%
\pgfpathlineto{\pgfqpoint{0.506453in}{1.748409in}}%
\pgfpathclose%
\pgfusepath{fill}%
\end{pgfscope}%
\begin{pgfscope}%
\pgfpathrectangle{\pgfqpoint{0.506453in}{1.347524in}}{\pgfqpoint{2.583333in}{0.400885in}}%
\pgfusepath{clip}%
\pgfsetroundcap%
\pgfsetroundjoin%
\pgfsetlinewidth{0.803000pt}%
\definecolor{currentstroke}{rgb}{1.000000,1.000000,1.000000}%
\pgfsetstrokecolor{currentstroke}%
\pgfsetdash{}{0pt}%
\pgfpathmoveto{\pgfqpoint{0.621730in}{1.347524in}}%
\pgfpathlineto{\pgfqpoint{0.621730in}{1.748409in}}%
\pgfusepath{stroke}%
\end{pgfscope}%
\begin{pgfscope}%
\definecolor{textcolor}{rgb}{0.150000,0.150000,0.150000}%
\pgfsetstrokecolor{textcolor}%
\pgfsetfillcolor{textcolor}%
\pgftext[x=0.621730in,y=1.250302in,,top]{\color{textcolor}\rmfamily\fontsize{14.000000}{16.800000}\selectfont 2012}%
\end{pgfscope}%
\begin{pgfscope}%
\pgfpathrectangle{\pgfqpoint{0.506453in}{1.347524in}}{\pgfqpoint{2.583333in}{0.400885in}}%
\pgfusepath{clip}%
\pgfsetroundcap%
\pgfsetroundjoin%
\pgfsetlinewidth{0.803000pt}%
\definecolor{currentstroke}{rgb}{1.000000,1.000000,1.000000}%
\pgfsetstrokecolor{currentstroke}%
\pgfsetdash{}{0pt}%
\pgfpathmoveto{\pgfqpoint{1.014755in}{1.347524in}}%
\pgfpathlineto{\pgfqpoint{1.014755in}{1.748409in}}%
\pgfusepath{stroke}%
\end{pgfscope}%
\begin{pgfscope}%
\definecolor{textcolor}{rgb}{0.150000,0.150000,0.150000}%
\pgfsetstrokecolor{textcolor}%
\pgfsetfillcolor{textcolor}%
\pgftext[x=1.014755in,y=1.250302in,,top]{\color{textcolor}\rmfamily\fontsize{14.000000}{16.800000}\selectfont 2013}%
\end{pgfscope}%
\begin{pgfscope}%
\pgfpathrectangle{\pgfqpoint{0.506453in}{1.347524in}}{\pgfqpoint{2.583333in}{0.400885in}}%
\pgfusepath{clip}%
\pgfsetroundcap%
\pgfsetroundjoin%
\pgfsetlinewidth{0.803000pt}%
\definecolor{currentstroke}{rgb}{1.000000,1.000000,1.000000}%
\pgfsetstrokecolor{currentstroke}%
\pgfsetdash{}{0pt}%
\pgfpathmoveto{\pgfqpoint{1.406706in}{1.347524in}}%
\pgfpathlineto{\pgfqpoint{1.406706in}{1.748409in}}%
\pgfusepath{stroke}%
\end{pgfscope}%
\begin{pgfscope}%
\definecolor{textcolor}{rgb}{0.150000,0.150000,0.150000}%
\pgfsetstrokecolor{textcolor}%
\pgfsetfillcolor{textcolor}%
\pgftext[x=1.406706in,y=1.250302in,,top]{\color{textcolor}\rmfamily\fontsize{14.000000}{16.800000}\selectfont 2014}%
\end{pgfscope}%
\begin{pgfscope}%
\pgfpathrectangle{\pgfqpoint{0.506453in}{1.347524in}}{\pgfqpoint{2.583333in}{0.400885in}}%
\pgfusepath{clip}%
\pgfsetroundcap%
\pgfsetroundjoin%
\pgfsetlinewidth{0.803000pt}%
\definecolor{currentstroke}{rgb}{1.000000,1.000000,1.000000}%
\pgfsetstrokecolor{currentstroke}%
\pgfsetdash{}{0pt}%
\pgfpathmoveto{\pgfqpoint{1.798657in}{1.347524in}}%
\pgfpathlineto{\pgfqpoint{1.798657in}{1.748409in}}%
\pgfusepath{stroke}%
\end{pgfscope}%
\begin{pgfscope}%
\definecolor{textcolor}{rgb}{0.150000,0.150000,0.150000}%
\pgfsetstrokecolor{textcolor}%
\pgfsetfillcolor{textcolor}%
\pgftext[x=1.798657in,y=1.250302in,,top]{\color{textcolor}\rmfamily\fontsize{14.000000}{16.800000}\selectfont 2015}%
\end{pgfscope}%
\begin{pgfscope}%
\pgfpathrectangle{\pgfqpoint{0.506453in}{1.347524in}}{\pgfqpoint{2.583333in}{0.400885in}}%
\pgfusepath{clip}%
\pgfsetroundcap%
\pgfsetroundjoin%
\pgfsetlinewidth{0.803000pt}%
\definecolor{currentstroke}{rgb}{1.000000,1.000000,1.000000}%
\pgfsetstrokecolor{currentstroke}%
\pgfsetdash{}{0pt}%
\pgfpathmoveto{\pgfqpoint{2.190608in}{1.347524in}}%
\pgfpathlineto{\pgfqpoint{2.190608in}{1.748409in}}%
\pgfusepath{stroke}%
\end{pgfscope}%
\begin{pgfscope}%
\definecolor{textcolor}{rgb}{0.150000,0.150000,0.150000}%
\pgfsetstrokecolor{textcolor}%
\pgfsetfillcolor{textcolor}%
\pgftext[x=2.190608in,y=1.250302in,,top]{\color{textcolor}\rmfamily\fontsize{14.000000}{16.800000}\selectfont 2016}%
\end{pgfscope}%
\begin{pgfscope}%
\pgfpathrectangle{\pgfqpoint{0.506453in}{1.347524in}}{\pgfqpoint{2.583333in}{0.400885in}}%
\pgfusepath{clip}%
\pgfsetroundcap%
\pgfsetroundjoin%
\pgfsetlinewidth{0.803000pt}%
\definecolor{currentstroke}{rgb}{1.000000,1.000000,1.000000}%
\pgfsetstrokecolor{currentstroke}%
\pgfsetdash{}{0pt}%
\pgfpathmoveto{\pgfqpoint{2.583633in}{1.347524in}}%
\pgfpathlineto{\pgfqpoint{2.583633in}{1.748409in}}%
\pgfusepath{stroke}%
\end{pgfscope}%
\begin{pgfscope}%
\definecolor{textcolor}{rgb}{0.150000,0.150000,0.150000}%
\pgfsetstrokecolor{textcolor}%
\pgfsetfillcolor{textcolor}%
\pgftext[x=2.583633in,y=1.250302in,,top]{\color{textcolor}\rmfamily\fontsize{14.000000}{16.800000}\selectfont 2017}%
\end{pgfscope}%
\begin{pgfscope}%
\pgfpathrectangle{\pgfqpoint{0.506453in}{1.347524in}}{\pgfqpoint{2.583333in}{0.400885in}}%
\pgfusepath{clip}%
\pgfsetroundcap%
\pgfsetroundjoin%
\pgfsetlinewidth{0.803000pt}%
\definecolor{currentstroke}{rgb}{1.000000,1.000000,1.000000}%
\pgfsetstrokecolor{currentstroke}%
\pgfsetdash{}{0pt}%
\pgfpathmoveto{\pgfqpoint{2.975584in}{1.347524in}}%
\pgfpathlineto{\pgfqpoint{2.975584in}{1.748409in}}%
\pgfusepath{stroke}%
\end{pgfscope}%
\begin{pgfscope}%
\definecolor{textcolor}{rgb}{0.150000,0.150000,0.150000}%
\pgfsetstrokecolor{textcolor}%
\pgfsetfillcolor{textcolor}%
\pgftext[x=2.975584in,y=1.250302in,,top]{\color{textcolor}\rmfamily\fontsize{14.000000}{16.800000}\selectfont 2018}%
\end{pgfscope}%
\begin{pgfscope}%
\pgfpathrectangle{\pgfqpoint{0.506453in}{1.347524in}}{\pgfqpoint{2.583333in}{0.400885in}}%
\pgfusepath{clip}%
\pgfsetroundcap%
\pgfsetroundjoin%
\pgfsetlinewidth{0.803000pt}%
\definecolor{currentstroke}{rgb}{1.000000,1.000000,1.000000}%
\pgfsetstrokecolor{currentstroke}%
\pgfsetdash{}{0pt}%
\pgfpathmoveto{\pgfqpoint{0.506453in}{1.365182in}}%
\pgfpathlineto{\pgfqpoint{3.089787in}{1.365182in}}%
\pgfusepath{stroke}%
\end{pgfscope}%
\begin{pgfscope}%
\definecolor{textcolor}{rgb}{0.150000,0.150000,0.150000}%
\pgfsetstrokecolor{textcolor}%
\pgfsetfillcolor{textcolor}%
\pgftext[x=0.285520in,y=1.291316in,left,base]{\color{textcolor}\rmfamily\fontsize{14.000000}{16.800000}\selectfont 0}%
\end{pgfscope}%
\begin{pgfscope}%
\pgfpathrectangle{\pgfqpoint{0.506453in}{1.347524in}}{\pgfqpoint{2.583333in}{0.400885in}}%
\pgfusepath{clip}%
\pgfsetroundcap%
\pgfsetroundjoin%
\pgfsetlinewidth{0.803000pt}%
\definecolor{currentstroke}{rgb}{1.000000,1.000000,1.000000}%
\pgfsetstrokecolor{currentstroke}%
\pgfsetdash{}{0pt}%
\pgfpathmoveto{\pgfqpoint{0.506453in}{1.734011in}}%
\pgfpathlineto{\pgfqpoint{3.089787in}{1.734011in}}%
\pgfusepath{stroke}%
\end{pgfscope}%
\begin{pgfscope}%
\definecolor{textcolor}{rgb}{0.150000,0.150000,0.150000}%
\pgfsetstrokecolor{textcolor}%
\pgfsetfillcolor{textcolor}%
\pgftext[x=0.285520in,y=1.660144in,left,base]{\color{textcolor}\rmfamily\fontsize{14.000000}{16.800000}\selectfont 2}%
\end{pgfscope}%
\begin{pgfscope}%
\pgfpathrectangle{\pgfqpoint{0.506453in}{1.347524in}}{\pgfqpoint{2.583333in}{0.400885in}}%
\pgfusepath{clip}%
\pgfsetroundcap%
\pgfsetroundjoin%
\pgfsetlinewidth{1.505625pt}%
\definecolor{currentstroke}{rgb}{0.000000,0.000000,0.000000}%
\pgfsetstrokecolor{currentstroke}%
\pgfsetdash{}{0pt}%
\pgfpathmoveto{\pgfqpoint{0.623878in}{1.549596in}}%
\pgfpathlineto{\pgfqpoint{0.624952in}{1.550545in}}%
\pgfpathlineto{\pgfqpoint{0.627099in}{1.547699in}}%
\pgfpathlineto{\pgfqpoint{0.630321in}{1.548203in}}%
\pgfpathlineto{\pgfqpoint{0.631395in}{1.552975in}}%
\pgfpathlineto{\pgfqpoint{0.633542in}{1.555969in}}%
\pgfpathlineto{\pgfqpoint{0.634616in}{1.553094in}}%
\pgfpathlineto{\pgfqpoint{0.638911in}{1.555465in}}%
\pgfpathlineto{\pgfqpoint{0.639985in}{1.556858in}}%
\pgfpathlineto{\pgfqpoint{0.642133in}{1.554605in}}%
\pgfpathlineto{\pgfqpoint{0.645354in}{1.555020in}}%
\pgfpathlineto{\pgfqpoint{0.646428in}{1.557302in}}%
\pgfpathlineto{\pgfqpoint{0.649650in}{1.556887in}}%
\pgfpathlineto{\pgfqpoint{0.652871in}{1.556858in}}%
\pgfpathlineto{\pgfqpoint{0.653945in}{1.558695in}}%
\pgfpathlineto{\pgfqpoint{0.655019in}{1.563319in}}%
\pgfpathlineto{\pgfqpoint{0.656093in}{1.562815in}}%
\pgfpathlineto{\pgfqpoint{0.657167in}{1.565364in}}%
\pgfpathlineto{\pgfqpoint{0.661462in}{1.563467in}}%
\pgfpathlineto{\pgfqpoint{0.663610in}{1.572122in}}%
\pgfpathlineto{\pgfqpoint{0.664684in}{1.571411in}}%
\pgfpathlineto{\pgfqpoint{0.667905in}{1.574819in}}%
\pgfpathlineto{\pgfqpoint{0.668979in}{1.574226in}}%
\pgfpathlineto{\pgfqpoint{0.670053in}{1.570373in}}%
\pgfpathlineto{\pgfqpoint{0.672200in}{1.573278in}}%
\pgfpathlineto{\pgfqpoint{0.677570in}{1.573722in}}%
\pgfpathlineto{\pgfqpoint{0.678643in}{1.572566in}}%
\pgfpathlineto{\pgfqpoint{0.679717in}{1.573752in}}%
\pgfpathlineto{\pgfqpoint{0.684013in}{1.572507in}}%
\pgfpathlineto{\pgfqpoint{0.687234in}{1.575175in}}%
\pgfpathlineto{\pgfqpoint{0.690456in}{1.572063in}}%
\pgfpathlineto{\pgfqpoint{0.691530in}{1.567350in}}%
\pgfpathlineto{\pgfqpoint{0.693677in}{1.572685in}}%
\pgfpathlineto{\pgfqpoint{0.694751in}{1.572715in}}%
\pgfpathlineto{\pgfqpoint{0.697973in}{1.573811in}}%
\pgfpathlineto{\pgfqpoint{0.699046in}{1.580213in}}%
\pgfpathlineto{\pgfqpoint{0.701194in}{1.581014in}}%
\pgfpathlineto{\pgfqpoint{0.702268in}{1.577516in}}%
\pgfpathlineto{\pgfqpoint{0.705489in}{1.575382in}}%
\pgfpathlineto{\pgfqpoint{0.706563in}{1.572003in}}%
\pgfpathlineto{\pgfqpoint{0.709785in}{1.568387in}}%
\pgfpathlineto{\pgfqpoint{0.713006in}{1.572596in}}%
\pgfpathlineto{\pgfqpoint{0.714080in}{1.571618in}}%
\pgfpathlineto{\pgfqpoint{0.715154in}{1.567676in}}%
\pgfpathlineto{\pgfqpoint{0.717302in}{1.571203in}}%
\pgfpathlineto{\pgfqpoint{0.720523in}{1.570670in}}%
\pgfpathlineto{\pgfqpoint{0.722671in}{1.568684in}}%
\pgfpathlineto{\pgfqpoint{0.728040in}{1.564475in}}%
\pgfpathlineto{\pgfqpoint{0.729114in}{1.560118in}}%
\pgfpathlineto{\pgfqpoint{0.731262in}{1.566816in}}%
\pgfpathlineto{\pgfqpoint{0.732335in}{1.563408in}}%
\pgfpathlineto{\pgfqpoint{0.735557in}{1.563615in}}%
\pgfpathlineto{\pgfqpoint{0.736631in}{1.566994in}}%
\pgfpathlineto{\pgfqpoint{0.737705in}{1.566876in}}%
\pgfpathlineto{\pgfqpoint{0.738778in}{1.564890in}}%
\pgfpathlineto{\pgfqpoint{0.739852in}{1.566372in}}%
\pgfpathlineto{\pgfqpoint{0.743074in}{1.563289in}}%
\pgfpathlineto{\pgfqpoint{0.745221in}{1.563438in}}%
\pgfpathlineto{\pgfqpoint{0.747369in}{1.568862in}}%
\pgfpathlineto{\pgfqpoint{0.752738in}{1.567765in}}%
\pgfpathlineto{\pgfqpoint{0.753812in}{1.566164in}}%
\pgfpathlineto{\pgfqpoint{0.754886in}{1.562608in}}%
\pgfpathlineto{\pgfqpoint{0.759181in}{1.560978in}}%
\pgfpathlineto{\pgfqpoint{0.760255in}{1.556472in}}%
\pgfpathlineto{\pgfqpoint{0.761329in}{1.557184in}}%
\pgfpathlineto{\pgfqpoint{0.762403in}{1.556887in}}%
\pgfpathlineto{\pgfqpoint{0.765624in}{1.553835in}}%
\pgfpathlineto{\pgfqpoint{0.766698in}{1.554398in}}%
\pgfpathlineto{\pgfqpoint{0.769920in}{1.546128in}}%
\pgfpathlineto{\pgfqpoint{0.774215in}{1.550011in}}%
\pgfpathlineto{\pgfqpoint{0.775289in}{1.550307in}}%
\pgfpathlineto{\pgfqpoint{0.777437in}{1.547699in}}%
\pgfpathlineto{\pgfqpoint{0.781732in}{1.552886in}}%
\pgfpathlineto{\pgfqpoint{0.782806in}{1.549507in}}%
\pgfpathlineto{\pgfqpoint{0.783880in}{1.550426in}}%
\pgfpathlineto{\pgfqpoint{0.784953in}{1.545210in}}%
\pgfpathlineto{\pgfqpoint{0.788175in}{1.543787in}}%
\pgfpathlineto{\pgfqpoint{0.789249in}{1.542364in}}%
\pgfpathlineto{\pgfqpoint{0.791397in}{1.553657in}}%
\pgfpathlineto{\pgfqpoint{0.792470in}{1.553923in}}%
\pgfpathlineto{\pgfqpoint{0.796766in}{1.551048in}}%
\pgfpathlineto{\pgfqpoint{0.797840in}{1.549003in}}%
\pgfpathlineto{\pgfqpoint{0.799987in}{1.551374in}}%
\pgfpathlineto{\pgfqpoint{0.803209in}{1.552649in}}%
\pgfpathlineto{\pgfqpoint{0.804283in}{1.556176in}}%
\pgfpathlineto{\pgfqpoint{0.805356in}{1.554931in}}%
\pgfpathlineto{\pgfqpoint{0.806430in}{1.552204in}}%
\pgfpathlineto{\pgfqpoint{0.807504in}{1.553153in}}%
\pgfpathlineto{\pgfqpoint{0.810726in}{1.549537in}}%
\pgfpathlineto{\pgfqpoint{0.811799in}{1.549240in}}%
\pgfpathlineto{\pgfqpoint{0.812873in}{1.550574in}}%
\pgfpathlineto{\pgfqpoint{0.813947in}{1.546662in}}%
\pgfpathlineto{\pgfqpoint{0.815021in}{1.553983in}}%
\pgfpathlineto{\pgfqpoint{0.818242in}{1.552768in}}%
\pgfpathlineto{\pgfqpoint{0.819316in}{1.554546in}}%
\pgfpathlineto{\pgfqpoint{0.821464in}{1.553627in}}%
\pgfpathlineto{\pgfqpoint{0.822538in}{1.550396in}}%
\pgfpathlineto{\pgfqpoint{0.825759in}{1.550989in}}%
\pgfpathlineto{\pgfqpoint{0.826833in}{1.550545in}}%
\pgfpathlineto{\pgfqpoint{0.827907in}{1.546425in}}%
\pgfpathlineto{\pgfqpoint{0.828981in}{1.544765in}}%
\pgfpathlineto{\pgfqpoint{0.830055in}{1.549152in}}%
\pgfpathlineto{\pgfqpoint{0.833276in}{1.548203in}}%
\pgfpathlineto{\pgfqpoint{0.834350in}{1.549152in}}%
\pgfpathlineto{\pgfqpoint{0.836498in}{1.554724in}}%
\pgfpathlineto{\pgfqpoint{0.837572in}{1.550752in}}%
\pgfpathlineto{\pgfqpoint{0.840793in}{1.548351in}}%
\pgfpathlineto{\pgfqpoint{0.841867in}{1.545002in}}%
\pgfpathlineto{\pgfqpoint{0.845088in}{1.550871in}}%
\pgfpathlineto{\pgfqpoint{0.848310in}{1.552590in}}%
\pgfpathlineto{\pgfqpoint{0.849384in}{1.551256in}}%
\pgfpathlineto{\pgfqpoint{0.850458in}{1.552145in}}%
\pgfpathlineto{\pgfqpoint{0.851531in}{1.551256in}}%
\pgfpathlineto{\pgfqpoint{0.852605in}{1.557184in}}%
\pgfpathlineto{\pgfqpoint{0.855827in}{1.556798in}}%
\pgfpathlineto{\pgfqpoint{0.856901in}{1.560474in}}%
\pgfpathlineto{\pgfqpoint{0.859048in}{1.557747in}}%
\pgfpathlineto{\pgfqpoint{0.860122in}{1.559881in}}%
\pgfpathlineto{\pgfqpoint{0.863344in}{1.558844in}}%
\pgfpathlineto{\pgfqpoint{0.864418in}{1.559614in}}%
\pgfpathlineto{\pgfqpoint{0.866565in}{1.563438in}}%
\pgfpathlineto{\pgfqpoint{0.867639in}{1.567469in}}%
\pgfpathlineto{\pgfqpoint{0.870861in}{1.566757in}}%
\pgfpathlineto{\pgfqpoint{0.871934in}{1.564712in}}%
\pgfpathlineto{\pgfqpoint{0.873008in}{1.565690in}}%
\pgfpathlineto{\pgfqpoint{0.874082in}{1.564534in}}%
\pgfpathlineto{\pgfqpoint{0.875156in}{1.566757in}}%
\pgfpathlineto{\pgfqpoint{0.879451in}{1.568150in}}%
\pgfpathlineto{\pgfqpoint{0.880525in}{1.566965in}}%
\pgfpathlineto{\pgfqpoint{0.881599in}{1.564179in}}%
\pgfpathlineto{\pgfqpoint{0.882673in}{1.566164in}}%
\pgfpathlineto{\pgfqpoint{0.888042in}{1.561600in}}%
\pgfpathlineto{\pgfqpoint{0.889116in}{1.565038in}}%
\pgfpathlineto{\pgfqpoint{0.890190in}{1.565068in}}%
\pgfpathlineto{\pgfqpoint{0.893411in}{1.563052in}}%
\pgfpathlineto{\pgfqpoint{0.895559in}{1.563615in}}%
\pgfpathlineto{\pgfqpoint{0.897707in}{1.572715in}}%
\pgfpathlineto{\pgfqpoint{0.900928in}{1.571974in}}%
\pgfpathlineto{\pgfqpoint{0.902002in}{1.570284in}}%
\pgfpathlineto{\pgfqpoint{0.903076in}{1.570877in}}%
\pgfpathlineto{\pgfqpoint{0.904150in}{1.568862in}}%
\pgfpathlineto{\pgfqpoint{0.908445in}{1.566609in}}%
\pgfpathlineto{\pgfqpoint{0.909519in}{1.563438in}}%
\pgfpathlineto{\pgfqpoint{0.912740in}{1.562252in}}%
\pgfpathlineto{\pgfqpoint{0.923479in}{1.562430in}}%
\pgfpathlineto{\pgfqpoint{0.925626in}{1.556532in}}%
\pgfpathlineto{\pgfqpoint{0.930996in}{1.556858in}}%
\pgfpathlineto{\pgfqpoint{0.933143in}{1.562874in}}%
\pgfpathlineto{\pgfqpoint{0.934217in}{1.564623in}}%
\pgfpathlineto{\pgfqpoint{0.935291in}{1.561481in}}%
\pgfpathlineto{\pgfqpoint{0.938512in}{1.561096in}}%
\pgfpathlineto{\pgfqpoint{0.939586in}{1.559170in}}%
\pgfpathlineto{\pgfqpoint{0.940660in}{1.561274in}}%
\pgfpathlineto{\pgfqpoint{0.941734in}{1.559733in}}%
\pgfpathlineto{\pgfqpoint{0.942808in}{1.562015in}}%
\pgfpathlineto{\pgfqpoint{0.948177in}{1.561926in}}%
\pgfpathlineto{\pgfqpoint{0.949251in}{1.564179in}}%
\pgfpathlineto{\pgfqpoint{0.950325in}{1.561689in}}%
\pgfpathlineto{\pgfqpoint{0.953546in}{1.561274in}}%
\pgfpathlineto{\pgfqpoint{0.954620in}{1.566461in}}%
\pgfpathlineto{\pgfqpoint{0.956768in}{1.556887in}}%
\pgfpathlineto{\pgfqpoint{0.957841in}{1.556087in}}%
\pgfpathlineto{\pgfqpoint{0.961063in}{1.558873in}}%
\pgfpathlineto{\pgfqpoint{0.962137in}{1.558992in}}%
\pgfpathlineto{\pgfqpoint{0.963211in}{1.554398in}}%
\pgfpathlineto{\pgfqpoint{0.965358in}{1.555909in}}%
\pgfpathlineto{\pgfqpoint{0.968580in}{1.559288in}}%
\pgfpathlineto{\pgfqpoint{0.969654in}{1.559555in}}%
\pgfpathlineto{\pgfqpoint{0.972875in}{1.564445in}}%
\pgfpathlineto{\pgfqpoint{0.977171in}{1.564979in}}%
\pgfpathlineto{\pgfqpoint{0.978244in}{1.567439in}}%
\pgfpathlineto{\pgfqpoint{0.979318in}{1.567350in}}%
\pgfpathlineto{\pgfqpoint{0.980392in}{1.568239in}}%
\pgfpathlineto{\pgfqpoint{0.983614in}{1.567439in}}%
\pgfpathlineto{\pgfqpoint{0.987909in}{1.570462in}}%
\pgfpathlineto{\pgfqpoint{0.993278in}{1.570492in}}%
\pgfpathlineto{\pgfqpoint{0.995426in}{1.567913in}}%
\pgfpathlineto{\pgfqpoint{0.998647in}{1.567943in}}%
\pgfpathlineto{\pgfqpoint{0.999721in}{1.573604in}}%
\pgfpathlineto{\pgfqpoint{1.000795in}{1.575649in}}%
\pgfpathlineto{\pgfqpoint{1.001869in}{1.576005in}}%
\pgfpathlineto{\pgfqpoint{1.002943in}{1.574404in}}%
\pgfpathlineto{\pgfqpoint{1.009386in}{1.573189in}}%
\pgfpathlineto{\pgfqpoint{1.010460in}{1.570017in}}%
\pgfpathlineto{\pgfqpoint{1.013681in}{1.573041in}}%
\pgfpathlineto{\pgfqpoint{1.015829in}{1.578109in}}%
\pgfpathlineto{\pgfqpoint{1.016903in}{1.578880in}}%
\pgfpathlineto{\pgfqpoint{1.017976in}{1.580569in}}%
\pgfpathlineto{\pgfqpoint{1.021198in}{1.579532in}}%
\pgfpathlineto{\pgfqpoint{1.022272in}{1.576953in}}%
\pgfpathlineto{\pgfqpoint{1.023346in}{1.579502in}}%
\pgfpathlineto{\pgfqpoint{1.025493in}{1.581073in}}%
\pgfpathlineto{\pgfqpoint{1.029789in}{1.583059in}}%
\pgfpathlineto{\pgfqpoint{1.030863in}{1.582081in}}%
\pgfpathlineto{\pgfqpoint{1.033010in}{1.585578in}}%
\pgfpathlineto{\pgfqpoint{1.037306in}{1.586882in}}%
\pgfpathlineto{\pgfqpoint{1.039453in}{1.590557in}}%
\pgfpathlineto{\pgfqpoint{1.040527in}{1.592780in}}%
\pgfpathlineto{\pgfqpoint{1.044822in}{1.592691in}}%
\pgfpathlineto{\pgfqpoint{1.045896in}{1.591032in}}%
\pgfpathlineto{\pgfqpoint{1.046970in}{1.587149in}}%
\pgfpathlineto{\pgfqpoint{1.048044in}{1.592899in}}%
\pgfpathlineto{\pgfqpoint{1.053413in}{1.591713in}}%
\pgfpathlineto{\pgfqpoint{1.054487in}{1.593166in}}%
\pgfpathlineto{\pgfqpoint{1.055561in}{1.593521in}}%
\pgfpathlineto{\pgfqpoint{1.058782in}{1.592306in}}%
\pgfpathlineto{\pgfqpoint{1.059856in}{1.593255in}}%
\pgfpathlineto{\pgfqpoint{1.062004in}{1.593877in}}%
\pgfpathlineto{\pgfqpoint{1.063078in}{1.596663in}}%
\pgfpathlineto{\pgfqpoint{1.067373in}{1.597256in}}%
\pgfpathlineto{\pgfqpoint{1.069521in}{1.593166in}}%
\pgfpathlineto{\pgfqpoint{1.070595in}{1.595922in}}%
\pgfpathlineto{\pgfqpoint{1.073816in}{1.590498in}}%
\pgfpathlineto{\pgfqpoint{1.074890in}{1.592632in}}%
\pgfpathlineto{\pgfqpoint{1.075964in}{1.596130in}}%
\pgfpathlineto{\pgfqpoint{1.077038in}{1.596070in}}%
\pgfpathlineto{\pgfqpoint{1.081333in}{1.592454in}}%
\pgfpathlineto{\pgfqpoint{1.082407in}{1.597256in}}%
\pgfpathlineto{\pgfqpoint{1.083481in}{1.597404in}}%
\pgfpathlineto{\pgfqpoint{1.085628in}{1.599953in}}%
\pgfpathlineto{\pgfqpoint{1.089924in}{1.602680in}}%
\pgfpathlineto{\pgfqpoint{1.090997in}{1.602532in}}%
\pgfpathlineto{\pgfqpoint{1.092071in}{1.603450in}}%
\pgfpathlineto{\pgfqpoint{1.096367in}{1.601731in}}%
\pgfpathlineto{\pgfqpoint{1.098514in}{1.603450in}}%
\pgfpathlineto{\pgfqpoint{1.099588in}{1.600872in}}%
\pgfpathlineto{\pgfqpoint{1.100662in}{1.603836in}}%
\pgfpathlineto{\pgfqpoint{1.104957in}{1.601435in}}%
\pgfpathlineto{\pgfqpoint{1.106031in}{1.601316in}}%
\pgfpathlineto{\pgfqpoint{1.107105in}{1.603421in}}%
\pgfpathlineto{\pgfqpoint{1.112474in}{1.602294in}}%
\pgfpathlineto{\pgfqpoint{1.113548in}{1.602828in}}%
\pgfpathlineto{\pgfqpoint{1.114622in}{1.602739in}}%
\pgfpathlineto{\pgfqpoint{1.115696in}{1.601405in}}%
\pgfpathlineto{\pgfqpoint{1.118917in}{1.604132in}}%
\pgfpathlineto{\pgfqpoint{1.122139in}{1.609497in}}%
\pgfpathlineto{\pgfqpoint{1.123213in}{1.609171in}}%
\pgfpathlineto{\pgfqpoint{1.126434in}{1.603628in}}%
\pgfpathlineto{\pgfqpoint{1.127508in}{1.606177in}}%
\pgfpathlineto{\pgfqpoint{1.129656in}{1.598619in}}%
\pgfpathlineto{\pgfqpoint{1.130729in}{1.602887in}}%
\pgfpathlineto{\pgfqpoint{1.133951in}{1.603925in}}%
\pgfpathlineto{\pgfqpoint{1.136099in}{1.599775in}}%
\pgfpathlineto{\pgfqpoint{1.137173in}{1.600042in}}%
\pgfpathlineto{\pgfqpoint{1.138246in}{1.597611in}}%
\pgfpathlineto{\pgfqpoint{1.141468in}{1.598797in}}%
\pgfpathlineto{\pgfqpoint{1.143616in}{1.597345in}}%
\pgfpathlineto{\pgfqpoint{1.144689in}{1.599123in}}%
\pgfpathlineto{\pgfqpoint{1.145763in}{1.602591in}}%
\pgfpathlineto{\pgfqpoint{1.148985in}{1.603569in}}%
\pgfpathlineto{\pgfqpoint{1.153280in}{1.608044in}}%
\pgfpathlineto{\pgfqpoint{1.156502in}{1.607303in}}%
\pgfpathlineto{\pgfqpoint{1.157575in}{1.609793in}}%
\pgfpathlineto{\pgfqpoint{1.158649in}{1.610860in}}%
\pgfpathlineto{\pgfqpoint{1.159723in}{1.609200in}}%
\pgfpathlineto{\pgfqpoint{1.160797in}{1.614802in}}%
\pgfpathlineto{\pgfqpoint{1.164018in}{1.614506in}}%
\pgfpathlineto{\pgfqpoint{1.165092in}{1.615306in}}%
\pgfpathlineto{\pgfqpoint{1.167240in}{1.610030in}}%
\pgfpathlineto{\pgfqpoint{1.168314in}{1.608874in}}%
\pgfpathlineto{\pgfqpoint{1.172609in}{1.611275in}}%
\pgfpathlineto{\pgfqpoint{1.173683in}{1.609082in}}%
\pgfpathlineto{\pgfqpoint{1.174757in}{1.611245in}}%
\pgfpathlineto{\pgfqpoint{1.175831in}{1.608519in}}%
\pgfpathlineto{\pgfqpoint{1.179052in}{1.609378in}}%
\pgfpathlineto{\pgfqpoint{1.182274in}{1.603006in}}%
\pgfpathlineto{\pgfqpoint{1.183348in}{1.607481in}}%
\pgfpathlineto{\pgfqpoint{1.186569in}{1.606562in}}%
\pgfpathlineto{\pgfqpoint{1.187643in}{1.605377in}}%
\pgfpathlineto{\pgfqpoint{1.188717in}{1.602828in}}%
\pgfpathlineto{\pgfqpoint{1.189791in}{1.606948in}}%
\pgfpathlineto{\pgfqpoint{1.190864in}{1.606266in}}%
\pgfpathlineto{\pgfqpoint{1.194086in}{1.608726in}}%
\pgfpathlineto{\pgfqpoint{1.195160in}{1.611779in}}%
\pgfpathlineto{\pgfqpoint{1.197307in}{1.601968in}}%
\pgfpathlineto{\pgfqpoint{1.201603in}{1.599775in}}%
\pgfpathlineto{\pgfqpoint{1.202677in}{1.600516in}}%
\pgfpathlineto{\pgfqpoint{1.203751in}{1.603599in}}%
\pgfpathlineto{\pgfqpoint{1.204824in}{1.604962in}}%
\pgfpathlineto{\pgfqpoint{1.205898in}{1.603480in}}%
\pgfpathlineto{\pgfqpoint{1.209120in}{1.608104in}}%
\pgfpathlineto{\pgfqpoint{1.210194in}{1.605703in}}%
\pgfpathlineto{\pgfqpoint{1.213415in}{1.612698in}}%
\pgfpathlineto{\pgfqpoint{1.216637in}{1.613854in}}%
\pgfpathlineto{\pgfqpoint{1.217710in}{1.616521in}}%
\pgfpathlineto{\pgfqpoint{1.218784in}{1.615928in}}%
\pgfpathlineto{\pgfqpoint{1.219858in}{1.620789in}}%
\pgfpathlineto{\pgfqpoint{1.224153in}{1.622212in}}%
\pgfpathlineto{\pgfqpoint{1.225227in}{1.621560in}}%
\pgfpathlineto{\pgfqpoint{1.228449in}{1.627962in}}%
\pgfpathlineto{\pgfqpoint{1.231670in}{1.627013in}}%
\pgfpathlineto{\pgfqpoint{1.232744in}{1.634720in}}%
\pgfpathlineto{\pgfqpoint{1.234892in}{1.633890in}}%
\pgfpathlineto{\pgfqpoint{1.235966in}{1.634334in}}%
\pgfpathlineto{\pgfqpoint{1.239187in}{1.634601in}}%
\pgfpathlineto{\pgfqpoint{1.240261in}{1.635876in}}%
\pgfpathlineto{\pgfqpoint{1.241335in}{1.635876in}}%
\pgfpathlineto{\pgfqpoint{1.242409in}{1.639995in}}%
\pgfpathlineto{\pgfqpoint{1.243483in}{1.641507in}}%
\pgfpathlineto{\pgfqpoint{1.246704in}{1.638632in}}%
\pgfpathlineto{\pgfqpoint{1.247778in}{1.634927in}}%
\pgfpathlineto{\pgfqpoint{1.248852in}{1.637061in}}%
\pgfpathlineto{\pgfqpoint{1.249926in}{1.637654in}}%
\pgfpathlineto{\pgfqpoint{1.250999in}{1.635994in}}%
\pgfpathlineto{\pgfqpoint{1.254221in}{1.635787in}}%
\pgfpathlineto{\pgfqpoint{1.255295in}{1.639017in}}%
\pgfpathlineto{\pgfqpoint{1.256369in}{1.635935in}}%
\pgfpathlineto{\pgfqpoint{1.257442in}{1.630600in}}%
\pgfpathlineto{\pgfqpoint{1.258516in}{1.630837in}}%
\pgfpathlineto{\pgfqpoint{1.262812in}{1.629059in}}%
\pgfpathlineto{\pgfqpoint{1.263885in}{1.627280in}}%
\pgfpathlineto{\pgfqpoint{1.264959in}{1.630511in}}%
\pgfpathlineto{\pgfqpoint{1.269255in}{1.628821in}}%
\pgfpathlineto{\pgfqpoint{1.270329in}{1.622745in}}%
\pgfpathlineto{\pgfqpoint{1.271402in}{1.622834in}}%
\pgfpathlineto{\pgfqpoint{1.272476in}{1.624079in}}%
\pgfpathlineto{\pgfqpoint{1.273550in}{1.623131in}}%
\pgfpathlineto{\pgfqpoint{1.279993in}{1.632378in}}%
\pgfpathlineto{\pgfqpoint{1.281067in}{1.631222in}}%
\pgfpathlineto{\pgfqpoint{1.284288in}{1.634423in}}%
\pgfpathlineto{\pgfqpoint{1.286436in}{1.643078in}}%
\pgfpathlineto{\pgfqpoint{1.287510in}{1.643078in}}%
\pgfpathlineto{\pgfqpoint{1.295027in}{1.653807in}}%
\pgfpathlineto{\pgfqpoint{1.296101in}{1.647583in}}%
\pgfpathlineto{\pgfqpoint{1.299322in}{1.647168in}}%
\pgfpathlineto{\pgfqpoint{1.300396in}{1.648650in}}%
\pgfpathlineto{\pgfqpoint{1.301470in}{1.646753in}}%
\pgfpathlineto{\pgfqpoint{1.302544in}{1.647790in}}%
\pgfpathlineto{\pgfqpoint{1.303617in}{1.647020in}}%
\pgfpathlineto{\pgfqpoint{1.307913in}{1.641892in}}%
\pgfpathlineto{\pgfqpoint{1.310061in}{1.632408in}}%
\pgfpathlineto{\pgfqpoint{1.311134in}{1.633890in}}%
\pgfpathlineto{\pgfqpoint{1.314356in}{1.633267in}}%
\pgfpathlineto{\pgfqpoint{1.315430in}{1.630007in}}%
\pgfpathlineto{\pgfqpoint{1.316504in}{1.630214in}}%
\pgfpathlineto{\pgfqpoint{1.317577in}{1.638128in}}%
\pgfpathlineto{\pgfqpoint{1.318651in}{1.640944in}}%
\pgfpathlineto{\pgfqpoint{1.321873in}{1.640677in}}%
\pgfpathlineto{\pgfqpoint{1.322947in}{1.637832in}}%
\pgfpathlineto{\pgfqpoint{1.324020in}{1.639462in}}%
\pgfpathlineto{\pgfqpoint{1.325094in}{1.643552in}}%
\pgfpathlineto{\pgfqpoint{1.326168in}{1.642841in}}%
\pgfpathlineto{\pgfqpoint{1.329390in}{1.642515in}}%
\pgfpathlineto{\pgfqpoint{1.330463in}{1.638691in}}%
\pgfpathlineto{\pgfqpoint{1.331537in}{1.639343in}}%
\pgfpathlineto{\pgfqpoint{1.333685in}{1.642278in}}%
\pgfpathlineto{\pgfqpoint{1.336906in}{1.638336in}}%
\pgfpathlineto{\pgfqpoint{1.337980in}{1.639432in}}%
\pgfpathlineto{\pgfqpoint{1.339054in}{1.638217in}}%
\pgfpathlineto{\pgfqpoint{1.340128in}{1.638988in}}%
\pgfpathlineto{\pgfqpoint{1.341202in}{1.642248in}}%
\pgfpathlineto{\pgfqpoint{1.344423in}{1.643463in}}%
\pgfpathlineto{\pgfqpoint{1.345497in}{1.642693in}}%
\pgfpathlineto{\pgfqpoint{1.346571in}{1.645241in}}%
\pgfpathlineto{\pgfqpoint{1.347645in}{1.641626in}}%
\pgfpathlineto{\pgfqpoint{1.348719in}{1.644915in}}%
\pgfpathlineto{\pgfqpoint{1.351940in}{1.643700in}}%
\pgfpathlineto{\pgfqpoint{1.353014in}{1.642100in}}%
\pgfpathlineto{\pgfqpoint{1.354088in}{1.643641in}}%
\pgfpathlineto{\pgfqpoint{1.355162in}{1.646842in}}%
\pgfpathlineto{\pgfqpoint{1.356236in}{1.646575in}}%
\pgfpathlineto{\pgfqpoint{1.360531in}{1.647879in}}%
\pgfpathlineto{\pgfqpoint{1.361605in}{1.647198in}}%
\pgfpathlineto{\pgfqpoint{1.363752in}{1.650754in}}%
\pgfpathlineto{\pgfqpoint{1.366974in}{1.651021in}}%
\pgfpathlineto{\pgfqpoint{1.369122in}{1.654133in}}%
\pgfpathlineto{\pgfqpoint{1.371269in}{1.652444in}}%
\pgfpathlineto{\pgfqpoint{1.374491in}{1.651021in}}%
\pgfpathlineto{\pgfqpoint{1.376639in}{1.647464in}}%
\pgfpathlineto{\pgfqpoint{1.377712in}{1.647761in}}%
\pgfpathlineto{\pgfqpoint{1.378786in}{1.653125in}}%
\pgfpathlineto{\pgfqpoint{1.383082in}{1.652770in}}%
\pgfpathlineto{\pgfqpoint{1.384155in}{1.646753in}}%
\pgfpathlineto{\pgfqpoint{1.386303in}{1.643345in}}%
\pgfpathlineto{\pgfqpoint{1.389525in}{1.646427in}}%
\pgfpathlineto{\pgfqpoint{1.390598in}{1.644056in}}%
\pgfpathlineto{\pgfqpoint{1.391672in}{1.649717in}}%
\pgfpathlineto{\pgfqpoint{1.392746in}{1.648917in}}%
\pgfpathlineto{\pgfqpoint{1.393820in}{1.651940in}}%
\pgfpathlineto{\pgfqpoint{1.397041in}{1.652355in}}%
\pgfpathlineto{\pgfqpoint{1.400263in}{1.657186in}}%
\pgfpathlineto{\pgfqpoint{1.401337in}{1.657482in}}%
\pgfpathlineto{\pgfqpoint{1.404558in}{1.657305in}}%
\pgfpathlineto{\pgfqpoint{1.405632in}{1.660061in}}%
\pgfpathlineto{\pgfqpoint{1.407780in}{1.656682in}}%
\pgfpathlineto{\pgfqpoint{1.408854in}{1.657749in}}%
\pgfpathlineto{\pgfqpoint{1.412075in}{1.657453in}}%
\pgfpathlineto{\pgfqpoint{1.413149in}{1.659320in}}%
\pgfpathlineto{\pgfqpoint{1.416371in}{1.660150in}}%
\pgfpathlineto{\pgfqpoint{1.420666in}{1.657038in}}%
\pgfpathlineto{\pgfqpoint{1.421740in}{1.660772in}}%
\pgfpathlineto{\pgfqpoint{1.428183in}{1.663143in}}%
\pgfpathlineto{\pgfqpoint{1.429257in}{1.666078in}}%
\pgfpathlineto{\pgfqpoint{1.430330in}{1.662758in}}%
\pgfpathlineto{\pgfqpoint{1.431404in}{1.654874in}}%
\pgfpathlineto{\pgfqpoint{1.434626in}{1.660031in}}%
\pgfpathlineto{\pgfqpoint{1.435700in}{1.660239in}}%
\pgfpathlineto{\pgfqpoint{1.436773in}{1.658698in}}%
\pgfpathlineto{\pgfqpoint{1.437847in}{1.662343in}}%
\pgfpathlineto{\pgfqpoint{1.438921in}{1.660654in}}%
\pgfpathlineto{\pgfqpoint{1.444290in}{1.644797in}}%
\pgfpathlineto{\pgfqpoint{1.446438in}{1.652118in}}%
\pgfpathlineto{\pgfqpoint{1.449660in}{1.654370in}}%
\pgfpathlineto{\pgfqpoint{1.450733in}{1.657897in}}%
\pgfpathlineto{\pgfqpoint{1.453955in}{1.661810in}}%
\pgfpathlineto{\pgfqpoint{1.458250in}{1.661454in}}%
\pgfpathlineto{\pgfqpoint{1.459324in}{1.662254in}}%
\pgfpathlineto{\pgfqpoint{1.460398in}{1.664863in}}%
\pgfpathlineto{\pgfqpoint{1.464693in}{1.668627in}}%
\pgfpathlineto{\pgfqpoint{1.465767in}{1.667056in}}%
\pgfpathlineto{\pgfqpoint{1.466841in}{1.667649in}}%
\pgfpathlineto{\pgfqpoint{1.468989in}{1.670020in}}%
\pgfpathlineto{\pgfqpoint{1.472210in}{1.669338in}}%
\pgfpathlineto{\pgfqpoint{1.473284in}{1.671502in}}%
\pgfpathlineto{\pgfqpoint{1.474358in}{1.671027in}}%
\pgfpathlineto{\pgfqpoint{1.476506in}{1.673369in}}%
\pgfpathlineto{\pgfqpoint{1.479727in}{1.671946in}}%
\pgfpathlineto{\pgfqpoint{1.480801in}{1.666404in}}%
\pgfpathlineto{\pgfqpoint{1.481875in}{1.666848in}}%
\pgfpathlineto{\pgfqpoint{1.482949in}{1.659261in}}%
\pgfpathlineto{\pgfqpoint{1.484022in}{1.658490in}}%
\pgfpathlineto{\pgfqpoint{1.488318in}{1.663914in}}%
\pgfpathlineto{\pgfqpoint{1.489392in}{1.662165in}}%
\pgfpathlineto{\pgfqpoint{1.490465in}{1.661543in}}%
\pgfpathlineto{\pgfqpoint{1.491539in}{1.663618in}}%
\pgfpathlineto{\pgfqpoint{1.494761in}{1.661484in}}%
\pgfpathlineto{\pgfqpoint{1.495835in}{1.665277in}}%
\pgfpathlineto{\pgfqpoint{1.497982in}{1.661632in}}%
\pgfpathlineto{\pgfqpoint{1.499056in}{1.664240in}}%
\pgfpathlineto{\pgfqpoint{1.503351in}{1.672361in}}%
\pgfpathlineto{\pgfqpoint{1.504425in}{1.677430in}}%
\pgfpathlineto{\pgfqpoint{1.505499in}{1.677163in}}%
\pgfpathlineto{\pgfqpoint{1.506573in}{1.673013in}}%
\pgfpathlineto{\pgfqpoint{1.510868in}{1.666759in}}%
\pgfpathlineto{\pgfqpoint{1.511942in}{1.670049in}}%
\pgfpathlineto{\pgfqpoint{1.513016in}{1.663588in}}%
\pgfpathlineto{\pgfqpoint{1.514090in}{1.661958in}}%
\pgfpathlineto{\pgfqpoint{1.517311in}{1.664596in}}%
\pgfpathlineto{\pgfqpoint{1.518385in}{1.666937in}}%
\pgfpathlineto{\pgfqpoint{1.519459in}{1.672747in}}%
\pgfpathlineto{\pgfqpoint{1.520533in}{1.674051in}}%
\pgfpathlineto{\pgfqpoint{1.524828in}{1.673339in}}%
\pgfpathlineto{\pgfqpoint{1.526976in}{1.676866in}}%
\pgfpathlineto{\pgfqpoint{1.528050in}{1.675118in}}%
\pgfpathlineto{\pgfqpoint{1.529124in}{1.670494in}}%
\pgfpathlineto{\pgfqpoint{1.532345in}{1.671768in}}%
\pgfpathlineto{\pgfqpoint{1.533419in}{1.671413in}}%
\pgfpathlineto{\pgfqpoint{1.534493in}{1.673428in}}%
\pgfpathlineto{\pgfqpoint{1.535567in}{1.669427in}}%
\pgfpathlineto{\pgfqpoint{1.536640in}{1.668716in}}%
\pgfpathlineto{\pgfqpoint{1.539862in}{1.669486in}}%
\pgfpathlineto{\pgfqpoint{1.540936in}{1.667382in}}%
\pgfpathlineto{\pgfqpoint{1.542010in}{1.669723in}}%
\pgfpathlineto{\pgfqpoint{1.544157in}{1.669931in}}%
\pgfpathlineto{\pgfqpoint{1.547379in}{1.674199in}}%
\pgfpathlineto{\pgfqpoint{1.548453in}{1.674495in}}%
\pgfpathlineto{\pgfqpoint{1.549527in}{1.672302in}}%
\pgfpathlineto{\pgfqpoint{1.551674in}{1.665129in}}%
\pgfpathlineto{\pgfqpoint{1.554896in}{1.666285in}}%
\pgfpathlineto{\pgfqpoint{1.555970in}{1.661276in}}%
\pgfpathlineto{\pgfqpoint{1.557043in}{1.665811in}}%
\pgfpathlineto{\pgfqpoint{1.558117in}{1.666345in}}%
\pgfpathlineto{\pgfqpoint{1.559191in}{1.667649in}}%
\pgfpathlineto{\pgfqpoint{1.564560in}{1.668716in}}%
\pgfpathlineto{\pgfqpoint{1.565634in}{1.669783in}}%
\pgfpathlineto{\pgfqpoint{1.566708in}{1.669427in}}%
\pgfpathlineto{\pgfqpoint{1.571003in}{1.673636in}}%
\pgfpathlineto{\pgfqpoint{1.572077in}{1.671828in}}%
\pgfpathlineto{\pgfqpoint{1.574225in}{1.676451in}}%
\pgfpathlineto{\pgfqpoint{1.577446in}{1.679564in}}%
\pgfpathlineto{\pgfqpoint{1.580668in}{1.671087in}}%
\pgfpathlineto{\pgfqpoint{1.581742in}{1.670939in}}%
\pgfpathlineto{\pgfqpoint{1.586037in}{1.671502in}}%
\pgfpathlineto{\pgfqpoint{1.588185in}{1.672984in}}%
\pgfpathlineto{\pgfqpoint{1.589259in}{1.674080in}}%
\pgfpathlineto{\pgfqpoint{1.592480in}{1.671828in}}%
\pgfpathlineto{\pgfqpoint{1.593554in}{1.668034in}}%
\pgfpathlineto{\pgfqpoint{1.594628in}{1.669160in}}%
\pgfpathlineto{\pgfqpoint{1.595702in}{1.668212in}}%
\pgfpathlineto{\pgfqpoint{1.596775in}{1.670375in}}%
\pgfpathlineto{\pgfqpoint{1.599997in}{1.667412in}}%
\pgfpathlineto{\pgfqpoint{1.601071in}{1.668686in}}%
\pgfpathlineto{\pgfqpoint{1.602145in}{1.666611in}}%
\pgfpathlineto{\pgfqpoint{1.603218in}{1.667619in}}%
\pgfpathlineto{\pgfqpoint{1.607514in}{1.666493in}}%
\pgfpathlineto{\pgfqpoint{1.608588in}{1.663796in}}%
\pgfpathlineto{\pgfqpoint{1.610735in}{1.662343in}}%
\pgfpathlineto{\pgfqpoint{1.611809in}{1.663973in}}%
\pgfpathlineto{\pgfqpoint{1.615031in}{1.665930in}}%
\pgfpathlineto{\pgfqpoint{1.616105in}{1.665811in}}%
\pgfpathlineto{\pgfqpoint{1.617178in}{1.664566in}}%
\pgfpathlineto{\pgfqpoint{1.618252in}{1.660387in}}%
\pgfpathlineto{\pgfqpoint{1.619326in}{1.662491in}}%
\pgfpathlineto{\pgfqpoint{1.622548in}{1.660950in}}%
\pgfpathlineto{\pgfqpoint{1.624695in}{1.652444in}}%
\pgfpathlineto{\pgfqpoint{1.625769in}{1.650399in}}%
\pgfpathlineto{\pgfqpoint{1.626843in}{1.650221in}}%
\pgfpathlineto{\pgfqpoint{1.630064in}{1.650517in}}%
\pgfpathlineto{\pgfqpoint{1.632212in}{1.643671in}}%
\pgfpathlineto{\pgfqpoint{1.633286in}{1.640440in}}%
\pgfpathlineto{\pgfqpoint{1.634360in}{1.639403in}}%
\pgfpathlineto{\pgfqpoint{1.638655in}{1.640114in}}%
\pgfpathlineto{\pgfqpoint{1.639729in}{1.636883in}}%
\pgfpathlineto{\pgfqpoint{1.640803in}{1.638010in}}%
\pgfpathlineto{\pgfqpoint{1.641877in}{1.642574in}}%
\pgfpathlineto{\pgfqpoint{1.645098in}{1.641952in}}%
\pgfpathlineto{\pgfqpoint{1.646172in}{1.639818in}}%
\pgfpathlineto{\pgfqpoint{1.647246in}{1.643078in}}%
\pgfpathlineto{\pgfqpoint{1.648320in}{1.643641in}}%
\pgfpathlineto{\pgfqpoint{1.649394in}{1.643285in}}%
\pgfpathlineto{\pgfqpoint{1.654763in}{1.654015in}}%
\pgfpathlineto{\pgfqpoint{1.655837in}{1.654933in}}%
\pgfpathlineto{\pgfqpoint{1.656910in}{1.653185in}}%
\pgfpathlineto{\pgfqpoint{1.660132in}{1.654311in}}%
\pgfpathlineto{\pgfqpoint{1.661206in}{1.653926in}}%
\pgfpathlineto{\pgfqpoint{1.662280in}{1.652384in}}%
\pgfpathlineto{\pgfqpoint{1.663353in}{1.652444in}}%
\pgfpathlineto{\pgfqpoint{1.664427in}{1.649450in}}%
\pgfpathlineto{\pgfqpoint{1.669796in}{1.652681in}}%
\pgfpathlineto{\pgfqpoint{1.670870in}{1.652711in}}%
\pgfpathlineto{\pgfqpoint{1.671944in}{1.651347in}}%
\pgfpathlineto{\pgfqpoint{1.682682in}{1.650369in}}%
\pgfpathlineto{\pgfqpoint{1.683756in}{1.650932in}}%
\pgfpathlineto{\pgfqpoint{1.684830in}{1.649598in}}%
\pgfpathlineto{\pgfqpoint{1.685904in}{1.650873in}}%
\pgfpathlineto{\pgfqpoint{1.686978in}{1.650695in}}%
\pgfpathlineto{\pgfqpoint{1.691273in}{1.642722in}}%
\pgfpathlineto{\pgfqpoint{1.692347in}{1.644441in}}%
\pgfpathlineto{\pgfqpoint{1.693421in}{1.640262in}}%
\pgfpathlineto{\pgfqpoint{1.694495in}{1.642248in}}%
\pgfpathlineto{\pgfqpoint{1.697716in}{1.641833in}}%
\pgfpathlineto{\pgfqpoint{1.698790in}{1.643196in}}%
\pgfpathlineto{\pgfqpoint{1.699864in}{1.638573in}}%
\pgfpathlineto{\pgfqpoint{1.700938in}{1.636972in}}%
\pgfpathlineto{\pgfqpoint{1.702012in}{1.640084in}}%
\pgfpathlineto{\pgfqpoint{1.705233in}{1.639551in}}%
\pgfpathlineto{\pgfqpoint{1.706307in}{1.632200in}}%
\pgfpathlineto{\pgfqpoint{1.707381in}{1.635372in}}%
\pgfpathlineto{\pgfqpoint{1.708455in}{1.628288in}}%
\pgfpathlineto{\pgfqpoint{1.709528in}{1.628288in}}%
\pgfpathlineto{\pgfqpoint{1.712750in}{1.626628in}}%
\pgfpathlineto{\pgfqpoint{1.713824in}{1.628762in}}%
\pgfpathlineto{\pgfqpoint{1.714898in}{1.626272in}}%
\pgfpathlineto{\pgfqpoint{1.715971in}{1.626421in}}%
\pgfpathlineto{\pgfqpoint{1.717045in}{1.632467in}}%
\pgfpathlineto{\pgfqpoint{1.720267in}{1.632348in}}%
\pgfpathlineto{\pgfqpoint{1.721341in}{1.633653in}}%
\pgfpathlineto{\pgfqpoint{1.722415in}{1.631548in}}%
\pgfpathlineto{\pgfqpoint{1.723488in}{1.636824in}}%
\pgfpathlineto{\pgfqpoint{1.724562in}{1.638513in}}%
\pgfpathlineto{\pgfqpoint{1.727784in}{1.639491in}}%
\pgfpathlineto{\pgfqpoint{1.728858in}{1.644945in}}%
\pgfpathlineto{\pgfqpoint{1.729931in}{1.643848in}}%
\pgfpathlineto{\pgfqpoint{1.732079in}{1.646872in}}%
\pgfpathlineto{\pgfqpoint{1.735301in}{1.645034in}}%
\pgfpathlineto{\pgfqpoint{1.736374in}{1.646575in}}%
\pgfpathlineto{\pgfqpoint{1.738522in}{1.651051in}}%
\pgfpathlineto{\pgfqpoint{1.739596in}{1.652355in}}%
\pgfpathlineto{\pgfqpoint{1.742817in}{1.652147in}}%
\pgfpathlineto{\pgfqpoint{1.743891in}{1.650191in}}%
\pgfpathlineto{\pgfqpoint{1.744965in}{1.651436in}}%
\pgfpathlineto{\pgfqpoint{1.746039in}{1.651436in}}%
\pgfpathlineto{\pgfqpoint{1.747113in}{1.649628in}}%
\pgfpathlineto{\pgfqpoint{1.750334in}{1.649391in}}%
\pgfpathlineto{\pgfqpoint{1.751408in}{1.653185in}}%
\pgfpathlineto{\pgfqpoint{1.752482in}{1.652829in}}%
\pgfpathlineto{\pgfqpoint{1.753556in}{1.653244in}}%
\pgfpathlineto{\pgfqpoint{1.754630in}{1.657156in}}%
\pgfpathlineto{\pgfqpoint{1.757851in}{1.653155in}}%
\pgfpathlineto{\pgfqpoint{1.758925in}{1.660891in}}%
\pgfpathlineto{\pgfqpoint{1.759999in}{1.656801in}}%
\pgfpathlineto{\pgfqpoint{1.762147in}{1.656564in}}%
\pgfpathlineto{\pgfqpoint{1.766442in}{1.655526in}}%
\pgfpathlineto{\pgfqpoint{1.767516in}{1.658994in}}%
\pgfpathlineto{\pgfqpoint{1.769663in}{1.659794in}}%
\pgfpathlineto{\pgfqpoint{1.772885in}{1.664448in}}%
\pgfpathlineto{\pgfqpoint{1.773959in}{1.669575in}}%
\pgfpathlineto{\pgfqpoint{1.775033in}{1.665633in}}%
\pgfpathlineto{\pgfqpoint{1.776106in}{1.667056in}}%
\pgfpathlineto{\pgfqpoint{1.777180in}{1.662047in}}%
\pgfpathlineto{\pgfqpoint{1.780402in}{1.661424in}}%
\pgfpathlineto{\pgfqpoint{1.782549in}{1.666700in}}%
\pgfpathlineto{\pgfqpoint{1.783623in}{1.674792in}}%
\pgfpathlineto{\pgfqpoint{1.784697in}{1.671176in}}%
\pgfpathlineto{\pgfqpoint{1.787919in}{1.675266in}}%
\pgfpathlineto{\pgfqpoint{1.788993in}{1.675444in}}%
\pgfpathlineto{\pgfqpoint{1.790066in}{1.674643in}}%
\pgfpathlineto{\pgfqpoint{1.792214in}{1.675533in}}%
\pgfpathlineto{\pgfqpoint{1.795436in}{1.674525in}}%
\pgfpathlineto{\pgfqpoint{1.796509in}{1.672776in}}%
\pgfpathlineto{\pgfqpoint{1.797583in}{1.669605in}}%
\pgfpathlineto{\pgfqpoint{1.799731in}{1.669694in}}%
\pgfpathlineto{\pgfqpoint{1.802952in}{1.664625in}}%
\pgfpathlineto{\pgfqpoint{1.804026in}{1.660387in}}%
\pgfpathlineto{\pgfqpoint{1.805100in}{1.663588in}}%
\pgfpathlineto{\pgfqpoint{1.806174in}{1.668686in}}%
\pgfpathlineto{\pgfqpoint{1.807248in}{1.666997in}}%
\pgfpathlineto{\pgfqpoint{1.810469in}{1.668152in}}%
\pgfpathlineto{\pgfqpoint{1.811543in}{1.667856in}}%
\pgfpathlineto{\pgfqpoint{1.812617in}{1.665692in}}%
\pgfpathlineto{\pgfqpoint{1.813691in}{1.665692in}}%
\pgfpathlineto{\pgfqpoint{1.814765in}{1.672658in}}%
\pgfpathlineto{\pgfqpoint{1.819060in}{1.676303in}}%
\pgfpathlineto{\pgfqpoint{1.821208in}{1.684039in}}%
\pgfpathlineto{\pgfqpoint{1.825503in}{1.679534in}}%
\pgfpathlineto{\pgfqpoint{1.826577in}{1.680601in}}%
\pgfpathlineto{\pgfqpoint{1.827651in}{1.674673in}}%
\pgfpathlineto{\pgfqpoint{1.828725in}{1.673428in}}%
\pgfpathlineto{\pgfqpoint{1.829798in}{1.669012in}}%
\pgfpathlineto{\pgfqpoint{1.833020in}{1.673784in}}%
\pgfpathlineto{\pgfqpoint{1.834094in}{1.679890in}}%
\pgfpathlineto{\pgfqpoint{1.835168in}{1.676985in}}%
\pgfpathlineto{\pgfqpoint{1.836241in}{1.683120in}}%
\pgfpathlineto{\pgfqpoint{1.837315in}{1.682320in}}%
\pgfpathlineto{\pgfqpoint{1.840537in}{1.681045in}}%
\pgfpathlineto{\pgfqpoint{1.842684in}{1.681075in}}%
\pgfpathlineto{\pgfqpoint{1.843758in}{1.683446in}}%
\pgfpathlineto{\pgfqpoint{1.844832in}{1.687862in}}%
\pgfpathlineto{\pgfqpoint{1.849127in}{1.688070in}}%
\pgfpathlineto{\pgfqpoint{1.851275in}{1.692279in}}%
\pgfpathlineto{\pgfqpoint{1.852349in}{1.695480in}}%
\pgfpathlineto{\pgfqpoint{1.857718in}{1.693524in}}%
\pgfpathlineto{\pgfqpoint{1.859866in}{1.689641in}}%
\pgfpathlineto{\pgfqpoint{1.863087in}{1.693049in}}%
\pgfpathlineto{\pgfqpoint{1.864161in}{1.689078in}}%
\pgfpathlineto{\pgfqpoint{1.865235in}{1.687329in}}%
\pgfpathlineto{\pgfqpoint{1.866309in}{1.686647in}}%
\pgfpathlineto{\pgfqpoint{1.867383in}{1.683031in}}%
\pgfpathlineto{\pgfqpoint{1.870604in}{1.689196in}}%
\pgfpathlineto{\pgfqpoint{1.871678in}{1.677637in}}%
\pgfpathlineto{\pgfqpoint{1.872752in}{1.680127in}}%
\pgfpathlineto{\pgfqpoint{1.873826in}{1.687862in}}%
\pgfpathlineto{\pgfqpoint{1.874900in}{1.681194in}}%
\pgfpathlineto{\pgfqpoint{1.878121in}{1.684721in}}%
\pgfpathlineto{\pgfqpoint{1.879195in}{1.684187in}}%
\pgfpathlineto{\pgfqpoint{1.880269in}{1.685373in}}%
\pgfpathlineto{\pgfqpoint{1.881343in}{1.682913in}}%
\pgfpathlineto{\pgfqpoint{1.882416in}{1.683120in}}%
\pgfpathlineto{\pgfqpoint{1.885638in}{1.681045in}}%
\pgfpathlineto{\pgfqpoint{1.886712in}{1.681668in}}%
\pgfpathlineto{\pgfqpoint{1.887786in}{1.675118in}}%
\pgfpathlineto{\pgfqpoint{1.888859in}{1.674021in}}%
\pgfpathlineto{\pgfqpoint{1.889933in}{1.676303in}}%
\pgfpathlineto{\pgfqpoint{1.893155in}{1.681431in}}%
\pgfpathlineto{\pgfqpoint{1.895303in}{1.673695in}}%
\pgfpathlineto{\pgfqpoint{1.896376in}{1.676896in}}%
\pgfpathlineto{\pgfqpoint{1.900672in}{1.678852in}}%
\pgfpathlineto{\pgfqpoint{1.901746in}{1.677874in}}%
\pgfpathlineto{\pgfqpoint{1.902819in}{1.678793in}}%
\pgfpathlineto{\pgfqpoint{1.903893in}{1.678911in}}%
\pgfpathlineto{\pgfqpoint{1.904967in}{1.680453in}}%
\pgfpathlineto{\pgfqpoint{1.908189in}{1.677667in}}%
\pgfpathlineto{\pgfqpoint{1.910336in}{1.678882in}}%
\pgfpathlineto{\pgfqpoint{1.911410in}{1.677844in}}%
\pgfpathlineto{\pgfqpoint{1.912484in}{1.671531in}}%
\pgfpathlineto{\pgfqpoint{1.916779in}{1.676422in}}%
\pgfpathlineto{\pgfqpoint{1.917853in}{1.676451in}}%
\pgfpathlineto{\pgfqpoint{1.918927in}{1.677252in}}%
\pgfpathlineto{\pgfqpoint{1.920001in}{1.674317in}}%
\pgfpathlineto{\pgfqpoint{1.923222in}{1.673221in}}%
\pgfpathlineto{\pgfqpoint{1.924296in}{1.674110in}}%
\pgfpathlineto{\pgfqpoint{1.925370in}{1.672243in}}%
\pgfpathlineto{\pgfqpoint{1.926444in}{1.667915in}}%
\pgfpathlineto{\pgfqpoint{1.927518in}{1.672421in}}%
\pgfpathlineto{\pgfqpoint{1.930739in}{1.675088in}}%
\pgfpathlineto{\pgfqpoint{1.931813in}{1.671442in}}%
\pgfpathlineto{\pgfqpoint{1.932887in}{1.671442in}}%
\pgfpathlineto{\pgfqpoint{1.933961in}{1.674021in}}%
\pgfpathlineto{\pgfqpoint{1.935035in}{1.680364in}}%
\pgfpathlineto{\pgfqpoint{1.939330in}{1.677518in}}%
\pgfpathlineto{\pgfqpoint{1.940404in}{1.679267in}}%
\pgfpathlineto{\pgfqpoint{1.941478in}{1.684009in}}%
\pgfpathlineto{\pgfqpoint{1.942551in}{1.682261in}}%
\pgfpathlineto{\pgfqpoint{1.945773in}{1.682320in}}%
\pgfpathlineto{\pgfqpoint{1.946847in}{1.683802in}}%
\pgfpathlineto{\pgfqpoint{1.947921in}{1.683268in}}%
\pgfpathlineto{\pgfqpoint{1.948994in}{1.683950in}}%
\pgfpathlineto{\pgfqpoint{1.954364in}{1.677933in}}%
\pgfpathlineto{\pgfqpoint{1.955437in}{1.679978in}}%
\pgfpathlineto{\pgfqpoint{1.956511in}{1.680127in}}%
\pgfpathlineto{\pgfqpoint{1.957585in}{1.678734in}}%
\pgfpathlineto{\pgfqpoint{1.960807in}{1.678259in}}%
\pgfpathlineto{\pgfqpoint{1.961881in}{1.679178in}}%
\pgfpathlineto{\pgfqpoint{1.962954in}{1.682320in}}%
\pgfpathlineto{\pgfqpoint{1.964028in}{1.678645in}}%
\pgfpathlineto{\pgfqpoint{1.965102in}{1.678259in}}%
\pgfpathlineto{\pgfqpoint{1.968324in}{1.676066in}}%
\pgfpathlineto{\pgfqpoint{1.969397in}{1.676392in}}%
\pgfpathlineto{\pgfqpoint{1.971545in}{1.681816in}}%
\pgfpathlineto{\pgfqpoint{1.972619in}{1.679890in}}%
\pgfpathlineto{\pgfqpoint{1.975840in}{1.671887in}}%
\pgfpathlineto{\pgfqpoint{1.977988in}{1.673102in}}%
\pgfpathlineto{\pgfqpoint{1.979062in}{1.675325in}}%
\pgfpathlineto{\pgfqpoint{1.980136in}{1.672628in}}%
\pgfpathlineto{\pgfqpoint{1.984431in}{1.673547in}}%
\pgfpathlineto{\pgfqpoint{1.986579in}{1.668360in}}%
\pgfpathlineto{\pgfqpoint{1.987653in}{1.668953in}}%
\pgfpathlineto{\pgfqpoint{1.991948in}{1.662017in}}%
\pgfpathlineto{\pgfqpoint{1.993022in}{1.661721in}}%
\pgfpathlineto{\pgfqpoint{1.994096in}{1.658372in}}%
\pgfpathlineto{\pgfqpoint{1.998391in}{1.657838in}}%
\pgfpathlineto{\pgfqpoint{1.999465in}{1.659794in}}%
\pgfpathlineto{\pgfqpoint{2.000539in}{1.655971in}}%
\pgfpathlineto{\pgfqpoint{2.001613in}{1.656653in}}%
\pgfpathlineto{\pgfqpoint{2.002686in}{1.660031in}}%
\pgfpathlineto{\pgfqpoint{2.005908in}{1.663766in}}%
\pgfpathlineto{\pgfqpoint{2.009129in}{1.662906in}}%
\pgfpathlineto{\pgfqpoint{2.010203in}{1.661513in}}%
\pgfpathlineto{\pgfqpoint{2.013425in}{1.660832in}}%
\pgfpathlineto{\pgfqpoint{2.014499in}{1.640025in}}%
\pgfpathlineto{\pgfqpoint{2.015572in}{1.636943in}}%
\pgfpathlineto{\pgfqpoint{2.016646in}{1.635816in}}%
\pgfpathlineto{\pgfqpoint{2.017720in}{1.630926in}}%
\pgfpathlineto{\pgfqpoint{2.020942in}{1.629770in}}%
\pgfpathlineto{\pgfqpoint{2.022015in}{1.630037in}}%
\pgfpathlineto{\pgfqpoint{2.023089in}{1.631074in}}%
\pgfpathlineto{\pgfqpoint{2.024163in}{1.634720in}}%
\pgfpathlineto{\pgfqpoint{2.025237in}{1.633623in}}%
\pgfpathlineto{\pgfqpoint{2.030606in}{1.629770in}}%
\pgfpathlineto{\pgfqpoint{2.031680in}{1.630037in}}%
\pgfpathlineto{\pgfqpoint{2.032754in}{1.628199in}}%
\pgfpathlineto{\pgfqpoint{2.035975in}{1.631578in}}%
\pgfpathlineto{\pgfqpoint{2.037049in}{1.629059in}}%
\pgfpathlineto{\pgfqpoint{2.038123in}{1.630985in}}%
\pgfpathlineto{\pgfqpoint{2.039197in}{1.630155in}}%
\pgfpathlineto{\pgfqpoint{2.040271in}{1.631015in}}%
\pgfpathlineto{\pgfqpoint{2.044566in}{1.632941in}}%
\pgfpathlineto{\pgfqpoint{2.045640in}{1.629711in}}%
\pgfpathlineto{\pgfqpoint{2.047788in}{1.615691in}}%
\pgfpathlineto{\pgfqpoint{2.051009in}{1.609941in}}%
\pgfpathlineto{\pgfqpoint{2.052083in}{1.604369in}}%
\pgfpathlineto{\pgfqpoint{2.054231in}{1.616403in}}%
\pgfpathlineto{\pgfqpoint{2.055304in}{1.616314in}}%
\pgfpathlineto{\pgfqpoint{2.058526in}{1.611927in}}%
\pgfpathlineto{\pgfqpoint{2.059600in}{1.606829in}}%
\pgfpathlineto{\pgfqpoint{2.060674in}{1.610919in}}%
\pgfpathlineto{\pgfqpoint{2.061747in}{1.612727in}}%
\pgfpathlineto{\pgfqpoint{2.062821in}{1.609437in}}%
\pgfpathlineto{\pgfqpoint{2.067117in}{1.615187in}}%
\pgfpathlineto{\pgfqpoint{2.069264in}{1.611364in}}%
\pgfpathlineto{\pgfqpoint{2.070338in}{1.613883in}}%
\pgfpathlineto{\pgfqpoint{2.073560in}{1.612520in}}%
\pgfpathlineto{\pgfqpoint{2.075707in}{1.618211in}}%
\pgfpathlineto{\pgfqpoint{2.076781in}{1.616521in}}%
\pgfpathlineto{\pgfqpoint{2.077855in}{1.610475in}}%
\pgfpathlineto{\pgfqpoint{2.081077in}{1.611779in}}%
\pgfpathlineto{\pgfqpoint{2.082150in}{1.602739in}}%
\pgfpathlineto{\pgfqpoint{2.083224in}{1.599419in}}%
\pgfpathlineto{\pgfqpoint{2.084298in}{1.599034in}}%
\pgfpathlineto{\pgfqpoint{2.085372in}{1.600309in}}%
\pgfpathlineto{\pgfqpoint{2.088593in}{1.599064in}}%
\pgfpathlineto{\pgfqpoint{2.090741in}{1.604873in}}%
\pgfpathlineto{\pgfqpoint{2.091815in}{1.603184in}}%
\pgfpathlineto{\pgfqpoint{2.092889in}{1.606977in}}%
\pgfpathlineto{\pgfqpoint{2.096110in}{1.613676in}}%
\pgfpathlineto{\pgfqpoint{2.097184in}{1.614476in}}%
\pgfpathlineto{\pgfqpoint{2.100406in}{1.622064in}}%
\pgfpathlineto{\pgfqpoint{2.103627in}{1.622212in}}%
\pgfpathlineto{\pgfqpoint{2.104701in}{1.619129in}}%
\pgfpathlineto{\pgfqpoint{2.105775in}{1.613439in}}%
\pgfpathlineto{\pgfqpoint{2.106849in}{1.616166in}}%
\pgfpathlineto{\pgfqpoint{2.107923in}{1.615662in}}%
\pgfpathlineto{\pgfqpoint{2.111144in}{1.613113in}}%
\pgfpathlineto{\pgfqpoint{2.113292in}{1.629059in}}%
\pgfpathlineto{\pgfqpoint{2.114366in}{1.633949in}}%
\pgfpathlineto{\pgfqpoint{2.115439in}{1.636202in}}%
\pgfpathlineto{\pgfqpoint{2.118661in}{1.635135in}}%
\pgfpathlineto{\pgfqpoint{2.119735in}{1.631519in}}%
\pgfpathlineto{\pgfqpoint{2.120809in}{1.632704in}}%
\pgfpathlineto{\pgfqpoint{2.121882in}{1.631963in}}%
\pgfpathlineto{\pgfqpoint{2.122956in}{1.630244in}}%
\pgfpathlineto{\pgfqpoint{2.126178in}{1.632882in}}%
\pgfpathlineto{\pgfqpoint{2.128325in}{1.635431in}}%
\pgfpathlineto{\pgfqpoint{2.129399in}{1.636676in}}%
\pgfpathlineto{\pgfqpoint{2.130473in}{1.636676in}}%
\pgfpathlineto{\pgfqpoint{2.135842in}{1.631548in}}%
\pgfpathlineto{\pgfqpoint{2.136916in}{1.634156in}}%
\pgfpathlineto{\pgfqpoint{2.137990in}{1.626717in}}%
\pgfpathlineto{\pgfqpoint{2.141212in}{1.630392in}}%
\pgfpathlineto{\pgfqpoint{2.142285in}{1.629622in}}%
\pgfpathlineto{\pgfqpoint{2.143359in}{1.630066in}}%
\pgfpathlineto{\pgfqpoint{2.144433in}{1.631667in}}%
\pgfpathlineto{\pgfqpoint{2.148728in}{1.631104in}}%
\pgfpathlineto{\pgfqpoint{2.149802in}{1.629118in}}%
\pgfpathlineto{\pgfqpoint{2.150876in}{1.628881in}}%
\pgfpathlineto{\pgfqpoint{2.156245in}{1.625561in}}%
\pgfpathlineto{\pgfqpoint{2.157319in}{1.627310in}}%
\pgfpathlineto{\pgfqpoint{2.158393in}{1.623397in}}%
\pgfpathlineto{\pgfqpoint{2.159467in}{1.621916in}}%
\pgfpathlineto{\pgfqpoint{2.160541in}{1.624702in}}%
\pgfpathlineto{\pgfqpoint{2.163762in}{1.625176in}}%
\pgfpathlineto{\pgfqpoint{2.164836in}{1.620641in}}%
\pgfpathlineto{\pgfqpoint{2.166984in}{1.619663in}}%
\pgfpathlineto{\pgfqpoint{2.168058in}{1.618092in}}%
\pgfpathlineto{\pgfqpoint{2.171279in}{1.617381in}}%
\pgfpathlineto{\pgfqpoint{2.172353in}{1.618003in}}%
\pgfpathlineto{\pgfqpoint{2.173427in}{1.623071in}}%
\pgfpathlineto{\pgfqpoint{2.175574in}{1.615513in}}%
\pgfpathlineto{\pgfqpoint{2.178796in}{1.618952in}}%
\pgfpathlineto{\pgfqpoint{2.180944in}{1.626243in}}%
\pgfpathlineto{\pgfqpoint{2.182017in}{1.626243in}}%
\pgfpathlineto{\pgfqpoint{2.186313in}{1.625680in}}%
\pgfpathlineto{\pgfqpoint{2.187387in}{1.628881in}}%
\pgfpathlineto{\pgfqpoint{2.188460in}{1.627873in}}%
\pgfpathlineto{\pgfqpoint{2.189534in}{1.625620in}}%
\pgfpathlineto{\pgfqpoint{2.193830in}{1.624257in}}%
\pgfpathlineto{\pgfqpoint{2.194903in}{1.624672in}}%
\pgfpathlineto{\pgfqpoint{2.195977in}{1.617618in}}%
\pgfpathlineto{\pgfqpoint{2.198125in}{1.610238in}}%
\pgfpathlineto{\pgfqpoint{2.202420in}{1.610504in}}%
\pgfpathlineto{\pgfqpoint{2.203494in}{1.605999in}}%
\pgfpathlineto{\pgfqpoint{2.204568in}{1.606474in}}%
\pgfpathlineto{\pgfqpoint{2.205642in}{1.597315in}}%
\pgfpathlineto{\pgfqpoint{2.209937in}{1.596248in}}%
\pgfpathlineto{\pgfqpoint{2.211011in}{1.595151in}}%
\pgfpathlineto{\pgfqpoint{2.213159in}{1.599271in}}%
\pgfpathlineto{\pgfqpoint{2.216380in}{1.595329in}}%
\pgfpathlineto{\pgfqpoint{2.217454in}{1.597374in}}%
\pgfpathlineto{\pgfqpoint{2.218528in}{1.597789in}}%
\pgfpathlineto{\pgfqpoint{2.219602in}{1.599538in}}%
\pgfpathlineto{\pgfqpoint{2.220676in}{1.602887in}}%
\pgfpathlineto{\pgfqpoint{2.223897in}{1.602532in}}%
\pgfpathlineto{\pgfqpoint{2.224971in}{1.596752in}}%
\pgfpathlineto{\pgfqpoint{2.226045in}{1.598204in}}%
\pgfpathlineto{\pgfqpoint{2.227119in}{1.604073in}}%
\pgfpathlineto{\pgfqpoint{2.231414in}{1.600427in}}%
\pgfpathlineto{\pgfqpoint{2.232488in}{1.601672in}}%
\pgfpathlineto{\pgfqpoint{2.233562in}{1.600931in}}%
\pgfpathlineto{\pgfqpoint{2.234635in}{1.594677in}}%
\pgfpathlineto{\pgfqpoint{2.235709in}{1.598175in}}%
\pgfpathlineto{\pgfqpoint{2.240005in}{1.599627in}}%
\pgfpathlineto{\pgfqpoint{2.241079in}{1.605851in}}%
\pgfpathlineto{\pgfqpoint{2.242152in}{1.606503in}}%
\pgfpathlineto{\pgfqpoint{2.243226in}{1.606148in}}%
\pgfpathlineto{\pgfqpoint{2.246448in}{1.617440in}}%
\pgfpathlineto{\pgfqpoint{2.247522in}{1.615336in}}%
\pgfpathlineto{\pgfqpoint{2.248595in}{1.620849in}}%
\pgfpathlineto{\pgfqpoint{2.249669in}{1.633030in}}%
\pgfpathlineto{\pgfqpoint{2.253965in}{1.629059in}}%
\pgfpathlineto{\pgfqpoint{2.255038in}{1.624761in}}%
\pgfpathlineto{\pgfqpoint{2.257186in}{1.627665in}}%
\pgfpathlineto{\pgfqpoint{2.258260in}{1.630096in}}%
\pgfpathlineto{\pgfqpoint{2.262555in}{1.629829in}}%
\pgfpathlineto{\pgfqpoint{2.264703in}{1.627843in}}%
\pgfpathlineto{\pgfqpoint{2.265777in}{1.629414in}}%
\pgfpathlineto{\pgfqpoint{2.268998in}{1.629622in}}%
\pgfpathlineto{\pgfqpoint{2.270072in}{1.628169in}}%
\pgfpathlineto{\pgfqpoint{2.272220in}{1.635135in}}%
\pgfpathlineto{\pgfqpoint{2.273294in}{1.635698in}}%
\pgfpathlineto{\pgfqpoint{2.276515in}{1.636024in}}%
\pgfpathlineto{\pgfqpoint{2.277589in}{1.634631in}}%
\pgfpathlineto{\pgfqpoint{2.278663in}{1.635846in}}%
\pgfpathlineto{\pgfqpoint{2.284032in}{1.635312in}}%
\pgfpathlineto{\pgfqpoint{2.285106in}{1.638632in}}%
\pgfpathlineto{\pgfqpoint{2.286180in}{1.638988in}}%
\pgfpathlineto{\pgfqpoint{2.288327in}{1.638217in}}%
\pgfpathlineto{\pgfqpoint{2.291549in}{1.639017in}}%
\pgfpathlineto{\pgfqpoint{2.292623in}{1.637921in}}%
\pgfpathlineto{\pgfqpoint{2.295844in}{1.641892in}}%
\pgfpathlineto{\pgfqpoint{2.299066in}{1.644145in}}%
\pgfpathlineto{\pgfqpoint{2.300140in}{1.646694in}}%
\pgfpathlineto{\pgfqpoint{2.301213in}{1.650873in}}%
\pgfpathlineto{\pgfqpoint{2.307657in}{1.651940in}}%
\pgfpathlineto{\pgfqpoint{2.308730in}{1.653392in}}%
\pgfpathlineto{\pgfqpoint{2.309804in}{1.653125in}}%
\pgfpathlineto{\pgfqpoint{2.310878in}{1.653866in}}%
\pgfpathlineto{\pgfqpoint{2.314100in}{1.652503in}}%
\pgfpathlineto{\pgfqpoint{2.315173in}{1.651406in}}%
\pgfpathlineto{\pgfqpoint{2.316247in}{1.654370in}}%
\pgfpathlineto{\pgfqpoint{2.317321in}{1.649836in}}%
\pgfpathlineto{\pgfqpoint{2.318395in}{1.650221in}}%
\pgfpathlineto{\pgfqpoint{2.321616in}{1.650221in}}%
\pgfpathlineto{\pgfqpoint{2.323764in}{1.640114in}}%
\pgfpathlineto{\pgfqpoint{2.324838in}{1.639077in}}%
\pgfpathlineto{\pgfqpoint{2.325912in}{1.641359in}}%
\pgfpathlineto{\pgfqpoint{2.329133in}{1.638543in}}%
\pgfpathlineto{\pgfqpoint{2.330207in}{1.644204in}}%
\pgfpathlineto{\pgfqpoint{2.331281in}{1.642663in}}%
\pgfpathlineto{\pgfqpoint{2.332355in}{1.642278in}}%
\pgfpathlineto{\pgfqpoint{2.333429in}{1.639017in}}%
\pgfpathlineto{\pgfqpoint{2.336650in}{1.643256in}}%
\pgfpathlineto{\pgfqpoint{2.337724in}{1.638454in}}%
\pgfpathlineto{\pgfqpoint{2.338798in}{1.638158in}}%
\pgfpathlineto{\pgfqpoint{2.339872in}{1.635994in}}%
\pgfpathlineto{\pgfqpoint{2.340946in}{1.637624in}}%
\pgfpathlineto{\pgfqpoint{2.344167in}{1.637091in}}%
\pgfpathlineto{\pgfqpoint{2.345241in}{1.639847in}}%
\pgfpathlineto{\pgfqpoint{2.346315in}{1.640973in}}%
\pgfpathlineto{\pgfqpoint{2.347389in}{1.641240in}}%
\pgfpathlineto{\pgfqpoint{2.348462in}{1.642189in}}%
\pgfpathlineto{\pgfqpoint{2.353832in}{1.641062in}}%
\pgfpathlineto{\pgfqpoint{2.354905in}{1.642278in}}%
\pgfpathlineto{\pgfqpoint{2.355979in}{1.641299in}}%
\pgfpathlineto{\pgfqpoint{2.362422in}{1.646486in}}%
\pgfpathlineto{\pgfqpoint{2.366718in}{1.643433in}}%
\pgfpathlineto{\pgfqpoint{2.367791in}{1.643374in}}%
\pgfpathlineto{\pgfqpoint{2.368865in}{1.641596in}}%
\pgfpathlineto{\pgfqpoint{2.369939in}{1.643196in}}%
\pgfpathlineto{\pgfqpoint{2.371013in}{1.643404in}}%
\pgfpathlineto{\pgfqpoint{2.374235in}{1.644767in}}%
\pgfpathlineto{\pgfqpoint{2.376382in}{1.643819in}}%
\pgfpathlineto{\pgfqpoint{2.377456in}{1.646516in}}%
\pgfpathlineto{\pgfqpoint{2.378530in}{1.637061in}}%
\pgfpathlineto{\pgfqpoint{2.381751in}{1.632437in}}%
\pgfpathlineto{\pgfqpoint{2.384973in}{1.647109in}}%
\pgfpathlineto{\pgfqpoint{2.386047in}{1.647613in}}%
\pgfpathlineto{\pgfqpoint{2.390342in}{1.641803in}}%
\pgfpathlineto{\pgfqpoint{2.392490in}{1.645508in}}%
\pgfpathlineto{\pgfqpoint{2.393564in}{1.650162in}}%
\pgfpathlineto{\pgfqpoint{2.396785in}{1.650991in}}%
\pgfpathlineto{\pgfqpoint{2.398933in}{1.654104in}}%
\pgfpathlineto{\pgfqpoint{2.400007in}{1.654222in}}%
\pgfpathlineto{\pgfqpoint{2.401080in}{1.655230in}}%
\pgfpathlineto{\pgfqpoint{2.405376in}{1.655556in}}%
\pgfpathlineto{\pgfqpoint{2.406450in}{1.656712in}}%
\pgfpathlineto{\pgfqpoint{2.407524in}{1.656178in}}%
\pgfpathlineto{\pgfqpoint{2.408597in}{1.654192in}}%
\pgfpathlineto{\pgfqpoint{2.411819in}{1.652888in}}%
\pgfpathlineto{\pgfqpoint{2.412893in}{1.661780in}}%
\pgfpathlineto{\pgfqpoint{2.415040in}{1.660950in}}%
\pgfpathlineto{\pgfqpoint{2.416114in}{1.661128in}}%
\pgfpathlineto{\pgfqpoint{2.419336in}{1.659172in}}%
\pgfpathlineto{\pgfqpoint{2.420410in}{1.657334in}}%
\pgfpathlineto{\pgfqpoint{2.422557in}{1.657601in}}%
\pgfpathlineto{\pgfqpoint{2.423631in}{1.661395in}}%
\pgfpathlineto{\pgfqpoint{2.426853in}{1.661513in}}%
\pgfpathlineto{\pgfqpoint{2.427926in}{1.663025in}}%
\pgfpathlineto{\pgfqpoint{2.429000in}{1.662432in}}%
\pgfpathlineto{\pgfqpoint{2.430074in}{1.665337in}}%
\pgfpathlineto{\pgfqpoint{2.431148in}{1.664477in}}%
\pgfpathlineto{\pgfqpoint{2.434369in}{1.666730in}}%
\pgfpathlineto{\pgfqpoint{2.435443in}{1.665277in}}%
\pgfpathlineto{\pgfqpoint{2.437591in}{1.667589in}}%
\pgfpathlineto{\pgfqpoint{2.441886in}{1.665277in}}%
\pgfpathlineto{\pgfqpoint{2.442960in}{1.663914in}}%
\pgfpathlineto{\pgfqpoint{2.445108in}{1.663143in}}%
\pgfpathlineto{\pgfqpoint{2.446182in}{1.661988in}}%
\pgfpathlineto{\pgfqpoint{2.449403in}{1.663825in}}%
\pgfpathlineto{\pgfqpoint{2.451551in}{1.659557in}}%
\pgfpathlineto{\pgfqpoint{2.453699in}{1.660950in}}%
\pgfpathlineto{\pgfqpoint{2.460142in}{1.657690in}}%
\pgfpathlineto{\pgfqpoint{2.461215in}{1.649183in}}%
\pgfpathlineto{\pgfqpoint{2.464437in}{1.652918in}}%
\pgfpathlineto{\pgfqpoint{2.465511in}{1.648205in}}%
\pgfpathlineto{\pgfqpoint{2.466585in}{1.646368in}}%
\pgfpathlineto{\pgfqpoint{2.467658in}{1.649272in}}%
\pgfpathlineto{\pgfqpoint{2.468732in}{1.642040in}}%
\pgfpathlineto{\pgfqpoint{2.471954in}{1.642989in}}%
\pgfpathlineto{\pgfqpoint{2.473028in}{1.642426in}}%
\pgfpathlineto{\pgfqpoint{2.475175in}{1.650073in}}%
\pgfpathlineto{\pgfqpoint{2.476249in}{1.648857in}}%
\pgfpathlineto{\pgfqpoint{2.479471in}{1.647939in}}%
\pgfpathlineto{\pgfqpoint{2.481618in}{1.648265in}}%
\pgfpathlineto{\pgfqpoint{2.482692in}{1.644708in}}%
\pgfpathlineto{\pgfqpoint{2.483766in}{1.646190in}}%
\pgfpathlineto{\pgfqpoint{2.486988in}{1.648502in}}%
\pgfpathlineto{\pgfqpoint{2.488061in}{1.645745in}}%
\pgfpathlineto{\pgfqpoint{2.489135in}{1.647998in}}%
\pgfpathlineto{\pgfqpoint{2.490209in}{1.647524in}}%
\pgfpathlineto{\pgfqpoint{2.491283in}{1.643374in}}%
\pgfpathlineto{\pgfqpoint{2.494504in}{1.641714in}}%
\pgfpathlineto{\pgfqpoint{2.495578in}{1.638365in}}%
\pgfpathlineto{\pgfqpoint{2.496652in}{1.638810in}}%
\pgfpathlineto{\pgfqpoint{2.497726in}{1.641359in}}%
\pgfpathlineto{\pgfqpoint{2.498800in}{1.642218in}}%
\pgfpathlineto{\pgfqpoint{2.502021in}{1.640973in}}%
\pgfpathlineto{\pgfqpoint{2.503095in}{1.641744in}}%
\pgfpathlineto{\pgfqpoint{2.504169in}{1.641122in}}%
\pgfpathlineto{\pgfqpoint{2.506317in}{1.638098in}}%
\pgfpathlineto{\pgfqpoint{2.509538in}{1.640440in}}%
\pgfpathlineto{\pgfqpoint{2.510612in}{1.645538in}}%
\pgfpathlineto{\pgfqpoint{2.511686in}{1.644560in}}%
\pgfpathlineto{\pgfqpoint{2.512760in}{1.641981in}}%
\pgfpathlineto{\pgfqpoint{2.513834in}{1.646872in}}%
\pgfpathlineto{\pgfqpoint{2.517055in}{1.647850in}}%
\pgfpathlineto{\pgfqpoint{2.518129in}{1.647346in}}%
\pgfpathlineto{\pgfqpoint{2.520277in}{1.644678in}}%
\pgfpathlineto{\pgfqpoint{2.521350in}{1.645449in}}%
\pgfpathlineto{\pgfqpoint{2.525646in}{1.651110in}}%
\pgfpathlineto{\pgfqpoint{2.526720in}{1.655082in}}%
\pgfpathlineto{\pgfqpoint{2.527793in}{1.665040in}}%
\pgfpathlineto{\pgfqpoint{2.528867in}{1.666285in}}%
\pgfpathlineto{\pgfqpoint{2.533163in}{1.663025in}}%
\pgfpathlineto{\pgfqpoint{2.535310in}{1.662136in}}%
\pgfpathlineto{\pgfqpoint{2.536384in}{1.661573in}}%
\pgfpathlineto{\pgfqpoint{2.540679in}{1.662788in}}%
\pgfpathlineto{\pgfqpoint{2.541753in}{1.666048in}}%
\pgfpathlineto{\pgfqpoint{2.543901in}{1.667915in}}%
\pgfpathlineto{\pgfqpoint{2.547123in}{1.666997in}}%
\pgfpathlineto{\pgfqpoint{2.548196in}{1.668093in}}%
\pgfpathlineto{\pgfqpoint{2.549270in}{1.664981in}}%
\pgfpathlineto{\pgfqpoint{2.550344in}{1.664299in}}%
\pgfpathlineto{\pgfqpoint{2.551418in}{1.666345in}}%
\pgfpathlineto{\pgfqpoint{2.554639in}{1.664122in}}%
\pgfpathlineto{\pgfqpoint{2.555713in}{1.664151in}}%
\pgfpathlineto{\pgfqpoint{2.556787in}{1.670435in}}%
\pgfpathlineto{\pgfqpoint{2.557861in}{1.666937in}}%
\pgfpathlineto{\pgfqpoint{2.558935in}{1.670731in}}%
\pgfpathlineto{\pgfqpoint{2.562156in}{1.672391in}}%
\pgfpathlineto{\pgfqpoint{2.563230in}{1.672183in}}%
\pgfpathlineto{\pgfqpoint{2.565378in}{1.666107in}}%
\pgfpathlineto{\pgfqpoint{2.566452in}{1.667204in}}%
\pgfpathlineto{\pgfqpoint{2.569673in}{1.673606in}}%
\pgfpathlineto{\pgfqpoint{2.572895in}{1.672598in}}%
\pgfpathlineto{\pgfqpoint{2.573968in}{1.673132in}}%
\pgfpathlineto{\pgfqpoint{2.578264in}{1.674051in}}%
\pgfpathlineto{\pgfqpoint{2.579338in}{1.671828in}}%
\pgfpathlineto{\pgfqpoint{2.580412in}{1.672835in}}%
\pgfpathlineto{\pgfqpoint{2.581485in}{1.670257in}}%
\pgfpathlineto{\pgfqpoint{2.587928in}{1.675058in}}%
\pgfpathlineto{\pgfqpoint{2.589002in}{1.678408in}}%
\pgfpathlineto{\pgfqpoint{2.594371in}{1.674080in}}%
\pgfpathlineto{\pgfqpoint{2.595445in}{1.673606in}}%
\pgfpathlineto{\pgfqpoint{2.596519in}{1.671917in}}%
\pgfpathlineto{\pgfqpoint{2.600814in}{1.671235in}}%
\pgfpathlineto{\pgfqpoint{2.602962in}{1.673339in}}%
\pgfpathlineto{\pgfqpoint{2.604036in}{1.673517in}}%
\pgfpathlineto{\pgfqpoint{2.607257in}{1.672272in}}%
\pgfpathlineto{\pgfqpoint{2.608331in}{1.675799in}}%
\pgfpathlineto{\pgfqpoint{2.611553in}{1.670464in}}%
\pgfpathlineto{\pgfqpoint{2.614774in}{1.669012in}}%
\pgfpathlineto{\pgfqpoint{2.615848in}{1.670405in}}%
\pgfpathlineto{\pgfqpoint{2.616922in}{1.666256in}}%
\pgfpathlineto{\pgfqpoint{2.617996in}{1.666759in}}%
\pgfpathlineto{\pgfqpoint{2.619070in}{1.670346in}}%
\pgfpathlineto{\pgfqpoint{2.623365in}{1.674792in}}%
\pgfpathlineto{\pgfqpoint{2.624439in}{1.672421in}}%
\pgfpathlineto{\pgfqpoint{2.625513in}{1.671561in}}%
\pgfpathlineto{\pgfqpoint{2.626587in}{1.674228in}}%
\pgfpathlineto{\pgfqpoint{2.629808in}{1.676926in}}%
\pgfpathlineto{\pgfqpoint{2.630882in}{1.675681in}}%
\pgfpathlineto{\pgfqpoint{2.631956in}{1.678467in}}%
\pgfpathlineto{\pgfqpoint{2.634103in}{1.679149in}}%
\pgfpathlineto{\pgfqpoint{2.639473in}{1.680571in}}%
\pgfpathlineto{\pgfqpoint{2.640546in}{1.678971in}}%
\pgfpathlineto{\pgfqpoint{2.641620in}{1.680008in}}%
\pgfpathlineto{\pgfqpoint{2.644842in}{1.680986in}}%
\pgfpathlineto{\pgfqpoint{2.645916in}{1.680275in}}%
\pgfpathlineto{\pgfqpoint{2.646989in}{1.683446in}}%
\pgfpathlineto{\pgfqpoint{2.648063in}{1.680660in}}%
\pgfpathlineto{\pgfqpoint{2.649137in}{1.679682in}}%
\pgfpathlineto{\pgfqpoint{2.652359in}{1.677785in}}%
\pgfpathlineto{\pgfqpoint{2.653433in}{1.679504in}}%
\pgfpathlineto{\pgfqpoint{2.654506in}{1.678022in}}%
\pgfpathlineto{\pgfqpoint{2.656654in}{1.679119in}}%
\pgfpathlineto{\pgfqpoint{2.659876in}{1.679593in}}%
\pgfpathlineto{\pgfqpoint{2.660949in}{1.678259in}}%
\pgfpathlineto{\pgfqpoint{2.662023in}{1.681757in}}%
\pgfpathlineto{\pgfqpoint{2.663097in}{1.679682in}}%
\pgfpathlineto{\pgfqpoint{2.664171in}{1.682794in}}%
\pgfpathlineto{\pgfqpoint{2.667392in}{1.682972in}}%
\pgfpathlineto{\pgfqpoint{2.668466in}{1.679178in}}%
\pgfpathlineto{\pgfqpoint{2.670614in}{1.678289in}}%
\pgfpathlineto{\pgfqpoint{2.674909in}{1.678259in}}%
\pgfpathlineto{\pgfqpoint{2.675983in}{1.680808in}}%
\pgfpathlineto{\pgfqpoint{2.677057in}{1.678911in}}%
\pgfpathlineto{\pgfqpoint{2.678131in}{1.679978in}}%
\pgfpathlineto{\pgfqpoint{2.679205in}{1.679326in}}%
\pgfpathlineto{\pgfqpoint{2.682426in}{1.678526in}}%
\pgfpathlineto{\pgfqpoint{2.683500in}{1.681490in}}%
\pgfpathlineto{\pgfqpoint{2.684574in}{1.680127in}}%
\pgfpathlineto{\pgfqpoint{2.685648in}{1.679890in}}%
\pgfpathlineto{\pgfqpoint{2.686722in}{1.681490in}}%
\pgfpathlineto{\pgfqpoint{2.689943in}{1.681194in}}%
\pgfpathlineto{\pgfqpoint{2.691017in}{1.681935in}}%
\pgfpathlineto{\pgfqpoint{2.692091in}{1.679415in}}%
\pgfpathlineto{\pgfqpoint{2.693165in}{1.678971in}}%
\pgfpathlineto{\pgfqpoint{2.698534in}{1.683091in}}%
\pgfpathlineto{\pgfqpoint{2.699608in}{1.681312in}}%
\pgfpathlineto{\pgfqpoint{2.701755in}{1.687092in}}%
\pgfpathlineto{\pgfqpoint{2.706051in}{1.692368in}}%
\pgfpathlineto{\pgfqpoint{2.707124in}{1.696102in}}%
\pgfpathlineto{\pgfqpoint{2.708198in}{1.697703in}}%
\pgfpathlineto{\pgfqpoint{2.709272in}{1.698295in}}%
\pgfpathlineto{\pgfqpoint{2.712494in}{1.698118in}}%
\pgfpathlineto{\pgfqpoint{2.713567in}{1.699214in}}%
\pgfpathlineto{\pgfqpoint{2.715715in}{1.703927in}}%
\pgfpathlineto{\pgfqpoint{2.716789in}{1.704905in}}%
\pgfpathlineto{\pgfqpoint{2.720011in}{1.704164in}}%
\pgfpathlineto{\pgfqpoint{2.721084in}{1.705231in}}%
\pgfpathlineto{\pgfqpoint{2.722158in}{1.703986in}}%
\pgfpathlineto{\pgfqpoint{2.723232in}{1.704727in}}%
\pgfpathlineto{\pgfqpoint{2.724306in}{1.703453in}}%
\pgfpathlineto{\pgfqpoint{2.727527in}{1.703690in}}%
\pgfpathlineto{\pgfqpoint{2.728601in}{1.705053in}}%
\pgfpathlineto{\pgfqpoint{2.729675in}{1.701793in}}%
\pgfpathlineto{\pgfqpoint{2.730749in}{1.701200in}}%
\pgfpathlineto{\pgfqpoint{2.731823in}{1.706239in}}%
\pgfpathlineto{\pgfqpoint{2.735044in}{1.707632in}}%
\pgfpathlineto{\pgfqpoint{2.736118in}{1.708788in}}%
\pgfpathlineto{\pgfqpoint{2.738266in}{1.709292in}}%
\pgfpathlineto{\pgfqpoint{2.739340in}{1.708165in}}%
\pgfpathlineto{\pgfqpoint{2.744709in}{1.706565in}}%
\pgfpathlineto{\pgfqpoint{2.746856in}{1.708936in}}%
\pgfpathlineto{\pgfqpoint{2.750078in}{1.705735in}}%
\pgfpathlineto{\pgfqpoint{2.751152in}{1.703304in}}%
\pgfpathlineto{\pgfqpoint{2.752226in}{1.702356in}}%
\pgfpathlineto{\pgfqpoint{2.753300in}{1.702801in}}%
\pgfpathlineto{\pgfqpoint{2.754373in}{1.704757in}}%
\pgfpathlineto{\pgfqpoint{2.757595in}{1.702623in}}%
\pgfpathlineto{\pgfqpoint{2.758669in}{1.703127in}}%
\pgfpathlineto{\pgfqpoint{2.759743in}{1.703008in}}%
\pgfpathlineto{\pgfqpoint{2.760816in}{1.705083in}}%
\pgfpathlineto{\pgfqpoint{2.761890in}{1.704164in}}%
\pgfpathlineto{\pgfqpoint{2.765112in}{1.708165in}}%
\pgfpathlineto{\pgfqpoint{2.766186in}{1.707217in}}%
\pgfpathlineto{\pgfqpoint{2.769407in}{1.709203in}}%
\pgfpathlineto{\pgfqpoint{2.773702in}{1.707454in}}%
\pgfpathlineto{\pgfqpoint{2.774776in}{1.710003in}}%
\pgfpathlineto{\pgfqpoint{2.775850in}{1.707602in}}%
\pgfpathlineto{\pgfqpoint{2.776924in}{1.708906in}}%
\pgfpathlineto{\pgfqpoint{2.780145in}{1.708817in}}%
\pgfpathlineto{\pgfqpoint{2.782293in}{1.710833in}}%
\pgfpathlineto{\pgfqpoint{2.783367in}{1.708284in}}%
\pgfpathlineto{\pgfqpoint{2.784441in}{1.710299in}}%
\pgfpathlineto{\pgfqpoint{2.787662in}{1.711574in}}%
\pgfpathlineto{\pgfqpoint{2.789810in}{1.713411in}}%
\pgfpathlineto{\pgfqpoint{2.790884in}{1.711603in}}%
\pgfpathlineto{\pgfqpoint{2.791958in}{1.712552in}}%
\pgfpathlineto{\pgfqpoint{2.795179in}{1.711722in}}%
\pgfpathlineto{\pgfqpoint{2.796253in}{1.710359in}}%
\pgfpathlineto{\pgfqpoint{2.797327in}{1.711633in}}%
\pgfpathlineto{\pgfqpoint{2.798401in}{1.709855in}}%
\pgfpathlineto{\pgfqpoint{2.799475in}{1.712789in}}%
\pgfpathlineto{\pgfqpoint{2.802696in}{1.711781in}}%
\pgfpathlineto{\pgfqpoint{2.803770in}{1.704134in}}%
\pgfpathlineto{\pgfqpoint{2.805918in}{1.699274in}}%
\pgfpathlineto{\pgfqpoint{2.806991in}{1.699777in}}%
\pgfpathlineto{\pgfqpoint{2.810213in}{1.698947in}}%
\pgfpathlineto{\pgfqpoint{2.811287in}{1.699807in}}%
\pgfpathlineto{\pgfqpoint{2.813434in}{1.706061in}}%
\pgfpathlineto{\pgfqpoint{2.814508in}{1.707158in}}%
\pgfpathlineto{\pgfqpoint{2.817730in}{1.698799in}}%
\pgfpathlineto{\pgfqpoint{2.818804in}{1.697880in}}%
\pgfpathlineto{\pgfqpoint{2.820951in}{1.693850in}}%
\pgfpathlineto{\pgfqpoint{2.825247in}{1.694946in}}%
\pgfpathlineto{\pgfqpoint{2.826321in}{1.691597in}}%
\pgfpathlineto{\pgfqpoint{2.827394in}{1.699362in}}%
\pgfpathlineto{\pgfqpoint{2.828468in}{1.693968in}}%
\pgfpathlineto{\pgfqpoint{2.829542in}{1.692219in}}%
\pgfpathlineto{\pgfqpoint{2.832764in}{1.691656in}}%
\pgfpathlineto{\pgfqpoint{2.833837in}{1.692812in}}%
\pgfpathlineto{\pgfqpoint{2.834911in}{1.696606in}}%
\pgfpathlineto{\pgfqpoint{2.835985in}{1.691716in}}%
\pgfpathlineto{\pgfqpoint{2.837059in}{1.691063in}}%
\pgfpathlineto{\pgfqpoint{2.840280in}{1.691804in}}%
\pgfpathlineto{\pgfqpoint{2.841354in}{1.701348in}}%
\pgfpathlineto{\pgfqpoint{2.842428in}{1.703897in}}%
\pgfpathlineto{\pgfqpoint{2.843502in}{1.704223in}}%
\pgfpathlineto{\pgfqpoint{2.849945in}{1.675592in}}%
\pgfpathlineto{\pgfqpoint{2.851019in}{1.676629in}}%
\pgfpathlineto{\pgfqpoint{2.852093in}{1.675414in}}%
\pgfpathlineto{\pgfqpoint{2.856388in}{1.676303in}}%
\pgfpathlineto{\pgfqpoint{2.857462in}{1.677489in}}%
\pgfpathlineto{\pgfqpoint{2.858536in}{1.685580in}}%
\pgfpathlineto{\pgfqpoint{2.862831in}{1.684543in}}%
\pgfpathlineto{\pgfqpoint{2.864979in}{1.688307in}}%
\pgfpathlineto{\pgfqpoint{2.867126in}{1.690886in}}%
\pgfpathlineto{\pgfqpoint{2.870348in}{1.688841in}}%
\pgfpathlineto{\pgfqpoint{2.871422in}{1.690056in}}%
\pgfpathlineto{\pgfqpoint{2.872496in}{1.697021in}}%
\pgfpathlineto{\pgfqpoint{2.873569in}{1.693227in}}%
\pgfpathlineto{\pgfqpoint{2.874643in}{1.693909in}}%
\pgfpathlineto{\pgfqpoint{2.877865in}{1.698177in}}%
\pgfpathlineto{\pgfqpoint{2.878939in}{1.698533in}}%
\pgfpathlineto{\pgfqpoint{2.880012in}{1.698295in}}%
\pgfpathlineto{\pgfqpoint{2.881086in}{1.699837in}}%
\pgfpathlineto{\pgfqpoint{2.882160in}{1.700014in}}%
\pgfpathlineto{\pgfqpoint{2.885382in}{1.701141in}}%
\pgfpathlineto{\pgfqpoint{2.887529in}{1.698651in}}%
\pgfpathlineto{\pgfqpoint{2.888603in}{1.701674in}}%
\pgfpathlineto{\pgfqpoint{2.889677in}{1.701378in}}%
\pgfpathlineto{\pgfqpoint{2.892899in}{1.702208in}}%
\pgfpathlineto{\pgfqpoint{2.893972in}{1.703216in}}%
\pgfpathlineto{\pgfqpoint{2.895046in}{1.702682in}}%
\pgfpathlineto{\pgfqpoint{2.896120in}{1.703571in}}%
\pgfpathlineto{\pgfqpoint{2.897194in}{1.707661in}}%
\pgfpathlineto{\pgfqpoint{2.900415in}{1.707543in}}%
\pgfpathlineto{\pgfqpoint{2.902563in}{1.702149in}}%
\pgfpathlineto{\pgfqpoint{2.903637in}{1.704816in}}%
\pgfpathlineto{\pgfqpoint{2.904711in}{1.702386in}}%
\pgfpathlineto{\pgfqpoint{2.907932in}{1.704549in}}%
\pgfpathlineto{\pgfqpoint{2.909006in}{1.704342in}}%
\pgfpathlineto{\pgfqpoint{2.910080in}{1.705350in}}%
\pgfpathlineto{\pgfqpoint{2.911154in}{1.709143in}}%
\pgfpathlineto{\pgfqpoint{2.912228in}{1.708047in}}%
\pgfpathlineto{\pgfqpoint{2.915449in}{1.706120in}}%
\pgfpathlineto{\pgfqpoint{2.916523in}{1.707128in}}%
\pgfpathlineto{\pgfqpoint{2.917597in}{1.705824in}}%
\pgfpathlineto{\pgfqpoint{2.918671in}{1.700400in}}%
\pgfpathlineto{\pgfqpoint{2.919744in}{1.699659in}}%
\pgfpathlineto{\pgfqpoint{2.922966in}{1.696725in}}%
\pgfpathlineto{\pgfqpoint{2.924040in}{1.701615in}}%
\pgfpathlineto{\pgfqpoint{2.925114in}{1.698147in}}%
\pgfpathlineto{\pgfqpoint{2.926188in}{1.701052in}}%
\pgfpathlineto{\pgfqpoint{2.927261in}{1.697169in}}%
\pgfpathlineto{\pgfqpoint{2.930483in}{1.696725in}}%
\pgfpathlineto{\pgfqpoint{2.931557in}{1.698621in}}%
\pgfpathlineto{\pgfqpoint{2.932631in}{1.697732in}}%
\pgfpathlineto{\pgfqpoint{2.938000in}{1.698859in}}%
\pgfpathlineto{\pgfqpoint{2.940147in}{1.701822in}}%
\pgfpathlineto{\pgfqpoint{2.941221in}{1.711188in}}%
\pgfpathlineto{\pgfqpoint{2.942295in}{1.707395in}}%
\pgfpathlineto{\pgfqpoint{2.945517in}{1.707158in}}%
\pgfpathlineto{\pgfqpoint{2.946590in}{1.707869in}}%
\pgfpathlineto{\pgfqpoint{2.949812in}{1.715071in}}%
\pgfpathlineto{\pgfqpoint{2.954107in}{1.716968in}}%
\pgfpathlineto{\pgfqpoint{2.955181in}{1.719310in}}%
\pgfpathlineto{\pgfqpoint{2.956255in}{1.717768in}}%
\pgfpathlineto{\pgfqpoint{2.957329in}{1.724645in}}%
\pgfpathlineto{\pgfqpoint{2.964846in}{1.727638in}}%
\pgfpathlineto{\pgfqpoint{2.969141in}{1.727401in}}%
\pgfpathlineto{\pgfqpoint{2.971289in}{1.730187in}}%
\pgfpathlineto{\pgfqpoint{2.972363in}{1.728616in}}%
\pgfpathlineto{\pgfqpoint{2.972363in}{1.728616in}}%
\pgfusepath{stroke}%
\end{pgfscope}%
\begin{pgfscope}%
\pgfpathrectangle{\pgfqpoint{0.506453in}{1.347524in}}{\pgfqpoint{2.583333in}{0.400885in}}%
\pgfusepath{clip}%
\pgfsetroundcap%
\pgfsetroundjoin%
\pgfsetlinewidth{1.505625pt}%
\definecolor{currentstroke}{rgb}{0.890196,0.466667,0.760784}%
\pgfsetstrokecolor{currentstroke}%
\pgfsetdash{}{0pt}%
\pgfpathmoveto{\pgfqpoint{0.623878in}{1.549596in}}%
\pgfpathlineto{\pgfqpoint{0.624952in}{1.550545in}}%
\pgfpathlineto{\pgfqpoint{0.627099in}{1.547699in}}%
\pgfpathlineto{\pgfqpoint{0.630321in}{1.548203in}}%
\pgfpathlineto{\pgfqpoint{0.631395in}{1.543431in}}%
\pgfpathlineto{\pgfqpoint{0.633542in}{1.546273in}}%
\pgfpathlineto{\pgfqpoint{0.634616in}{1.543544in}}%
\pgfpathlineto{\pgfqpoint{0.638911in}{1.545794in}}%
\pgfpathlineto{\pgfqpoint{0.639985in}{1.547117in}}%
\pgfpathlineto{\pgfqpoint{0.642133in}{1.544979in}}%
\pgfpathlineto{\pgfqpoint{0.645354in}{1.545372in}}%
\pgfpathlineto{\pgfqpoint{0.646428in}{1.547539in}}%
\pgfpathlineto{\pgfqpoint{0.649650in}{1.547145in}}%
\pgfpathlineto{\pgfqpoint{0.652871in}{1.547117in}}%
\pgfpathlineto{\pgfqpoint{0.653945in}{1.548861in}}%
\pgfpathlineto{\pgfqpoint{0.655019in}{1.544472in}}%
\pgfpathlineto{\pgfqpoint{0.656093in}{1.544016in}}%
\pgfpathlineto{\pgfqpoint{0.657167in}{1.546323in}}%
\pgfpathlineto{\pgfqpoint{0.661462in}{1.544606in}}%
\pgfpathlineto{\pgfqpoint{0.664684in}{1.536348in}}%
\pgfpathlineto{\pgfqpoint{0.668979in}{1.533043in}}%
\pgfpathlineto{\pgfqpoint{0.670053in}{1.529949in}}%
\pgfpathlineto{\pgfqpoint{0.672200in}{1.532281in}}%
\pgfpathlineto{\pgfqpoint{0.677570in}{1.532638in}}%
\pgfpathlineto{\pgfqpoint{0.678643in}{1.531710in}}%
\pgfpathlineto{\pgfqpoint{0.679717in}{1.532662in}}%
\pgfpathlineto{\pgfqpoint{0.684013in}{1.531662in}}%
\pgfpathlineto{\pgfqpoint{0.687234in}{1.533804in}}%
\pgfpathlineto{\pgfqpoint{0.690456in}{1.531305in}}%
\pgfpathlineto{\pgfqpoint{0.692603in}{1.525165in}}%
\pgfpathlineto{\pgfqpoint{0.693677in}{1.527038in}}%
\pgfpathlineto{\pgfqpoint{0.694751in}{1.527061in}}%
\pgfpathlineto{\pgfqpoint{0.697973in}{1.527916in}}%
\pgfpathlineto{\pgfqpoint{0.699046in}{1.522922in}}%
\pgfpathlineto{\pgfqpoint{0.701194in}{1.523510in}}%
\pgfpathlineto{\pgfqpoint{0.702268in}{1.520944in}}%
\pgfpathlineto{\pgfqpoint{0.705489in}{1.519378in}}%
\pgfpathlineto{\pgfqpoint{0.706563in}{1.516900in}}%
\pgfpathlineto{\pgfqpoint{0.714080in}{1.510471in}}%
\pgfpathlineto{\pgfqpoint{0.715154in}{1.507697in}}%
\pgfpathlineto{\pgfqpoint{0.717302in}{1.510179in}}%
\pgfpathlineto{\pgfqpoint{0.720523in}{1.509804in}}%
\pgfpathlineto{\pgfqpoint{0.722671in}{1.508406in}}%
\pgfpathlineto{\pgfqpoint{0.728040in}{1.505444in}}%
\pgfpathlineto{\pgfqpoint{0.731262in}{1.497742in}}%
\pgfpathlineto{\pgfqpoint{0.732335in}{1.495501in}}%
\pgfpathlineto{\pgfqpoint{0.735557in}{1.495637in}}%
\pgfpathlineto{\pgfqpoint{0.736631in}{1.493416in}}%
\pgfpathlineto{\pgfqpoint{0.737705in}{1.493341in}}%
\pgfpathlineto{\pgfqpoint{0.738778in}{1.492079in}}%
\pgfpathlineto{\pgfqpoint{0.739852in}{1.493020in}}%
\pgfpathlineto{\pgfqpoint{0.743074in}{1.491062in}}%
\pgfpathlineto{\pgfqpoint{0.745221in}{1.491156in}}%
\pgfpathlineto{\pgfqpoint{0.746295in}{1.488896in}}%
\pgfpathlineto{\pgfqpoint{0.747369in}{1.490041in}}%
\pgfpathlineto{\pgfqpoint{0.753812in}{1.488387in}}%
\pgfpathlineto{\pgfqpoint{0.754886in}{1.486207in}}%
\pgfpathlineto{\pgfqpoint{0.759181in}{1.485208in}}%
\pgfpathlineto{\pgfqpoint{0.760255in}{1.482446in}}%
\pgfpathlineto{\pgfqpoint{0.761329in}{1.482882in}}%
\pgfpathlineto{\pgfqpoint{0.762403in}{1.482700in}}%
\pgfpathlineto{\pgfqpoint{0.765624in}{1.480829in}}%
\pgfpathlineto{\pgfqpoint{0.766698in}{1.481174in}}%
\pgfpathlineto{\pgfqpoint{0.769920in}{1.476105in}}%
\pgfpathlineto{\pgfqpoint{0.773141in}{1.474088in}}%
\pgfpathlineto{\pgfqpoint{0.775289in}{1.474614in}}%
\pgfpathlineto{\pgfqpoint{0.781732in}{1.470006in}}%
\pgfpathlineto{\pgfqpoint{0.782806in}{1.468119in}}%
\pgfpathlineto{\pgfqpoint{0.783880in}{1.468632in}}%
\pgfpathlineto{\pgfqpoint{0.784953in}{1.465719in}}%
\pgfpathlineto{\pgfqpoint{0.789249in}{1.464130in}}%
\pgfpathlineto{\pgfqpoint{0.791397in}{1.458007in}}%
\pgfpathlineto{\pgfqpoint{0.792470in}{1.458139in}}%
\pgfpathlineto{\pgfqpoint{0.796766in}{1.456723in}}%
\pgfpathlineto{\pgfqpoint{0.797840in}{1.455716in}}%
\pgfpathlineto{\pgfqpoint{0.799987in}{1.456883in}}%
\pgfpathlineto{\pgfqpoint{0.803209in}{1.457511in}}%
\pgfpathlineto{\pgfqpoint{0.804283in}{1.455774in}}%
\pgfpathlineto{\pgfqpoint{0.805356in}{1.455183in}}%
\pgfpathlineto{\pgfqpoint{0.806430in}{1.453890in}}%
\pgfpathlineto{\pgfqpoint{0.807504in}{1.454340in}}%
\pgfpathlineto{\pgfqpoint{0.811799in}{1.452484in}}%
\pgfpathlineto{\pgfqpoint{0.812873in}{1.453117in}}%
\pgfpathlineto{\pgfqpoint{0.813947in}{1.451261in}}%
\pgfpathlineto{\pgfqpoint{0.815021in}{1.447789in}}%
\pgfpathlineto{\pgfqpoint{0.818242in}{1.447257in}}%
\pgfpathlineto{\pgfqpoint{0.819316in}{1.448035in}}%
\pgfpathlineto{\pgfqpoint{0.821464in}{1.447633in}}%
\pgfpathlineto{\pgfqpoint{0.822538in}{1.446220in}}%
\pgfpathlineto{\pgfqpoint{0.826833in}{1.446284in}}%
\pgfpathlineto{\pgfqpoint{0.830055in}{1.441836in}}%
\pgfpathlineto{\pgfqpoint{0.834350in}{1.441836in}}%
\pgfpathlineto{\pgfqpoint{0.835424in}{1.440527in}}%
\pgfpathlineto{\pgfqpoint{0.836498in}{1.441506in}}%
\pgfpathlineto{\pgfqpoint{0.837572in}{1.439907in}}%
\pgfpathlineto{\pgfqpoint{0.840793in}{1.438940in}}%
\pgfpathlineto{\pgfqpoint{0.841867in}{1.437591in}}%
\pgfpathlineto{\pgfqpoint{0.844015in}{1.438594in}}%
\pgfpathlineto{\pgfqpoint{0.845088in}{1.437233in}}%
\pgfpathlineto{\pgfqpoint{0.848310in}{1.437900in}}%
\pgfpathlineto{\pgfqpoint{0.849384in}{1.437383in}}%
\pgfpathlineto{\pgfqpoint{0.850458in}{1.437728in}}%
\pgfpathlineto{\pgfqpoint{0.851531in}{1.437383in}}%
\pgfpathlineto{\pgfqpoint{0.852605in}{1.435083in}}%
\pgfpathlineto{\pgfqpoint{0.855827in}{1.434942in}}%
\pgfpathlineto{\pgfqpoint{0.857975in}{1.433044in}}%
\pgfpathlineto{\pgfqpoint{0.859048in}{1.432649in}}%
\pgfpathlineto{\pgfqpoint{0.860122in}{1.433397in}}%
\pgfpathlineto{\pgfqpoint{0.864418in}{1.433303in}}%
\pgfpathlineto{\pgfqpoint{0.866565in}{1.434643in}}%
\pgfpathlineto{\pgfqpoint{0.867639in}{1.433231in}}%
\pgfpathlineto{\pgfqpoint{0.870861in}{1.432991in}}%
\pgfpathlineto{\pgfqpoint{0.871934in}{1.432303in}}%
\pgfpathlineto{\pgfqpoint{0.873008in}{1.432632in}}%
\pgfpathlineto{\pgfqpoint{0.874082in}{1.432243in}}%
\pgfpathlineto{\pgfqpoint{0.875156in}{1.432991in}}%
\pgfpathlineto{\pgfqpoint{0.879451in}{1.433460in}}%
\pgfpathlineto{\pgfqpoint{0.882673in}{1.432792in}}%
\pgfpathlineto{\pgfqpoint{0.888042in}{1.431256in}}%
\pgfpathlineto{\pgfqpoint{0.889116in}{1.430100in}}%
\pgfpathlineto{\pgfqpoint{0.890190in}{1.430109in}}%
\pgfpathlineto{\pgfqpoint{0.894485in}{1.429580in}}%
\pgfpathlineto{\pgfqpoint{0.895559in}{1.429638in}}%
\pgfpathlineto{\pgfqpoint{0.897707in}{1.426748in}}%
\pgfpathlineto{\pgfqpoint{0.903076in}{1.426203in}}%
\pgfpathlineto{\pgfqpoint{0.905223in}{1.425482in}}%
\pgfpathlineto{\pgfqpoint{0.908445in}{1.424937in}}%
\pgfpathlineto{\pgfqpoint{0.909519in}{1.423996in}}%
\pgfpathlineto{\pgfqpoint{0.912740in}{1.423644in}}%
\pgfpathlineto{\pgfqpoint{0.923479in}{1.423697in}}%
\pgfpathlineto{\pgfqpoint{0.926700in}{1.421965in}}%
\pgfpathlineto{\pgfqpoint{0.930996in}{1.422044in}}%
\pgfpathlineto{\pgfqpoint{0.932069in}{1.420963in}}%
\pgfpathlineto{\pgfqpoint{0.934217in}{1.422139in}}%
\pgfpathlineto{\pgfqpoint{0.935291in}{1.421242in}}%
\pgfpathlineto{\pgfqpoint{0.942808in}{1.421394in}}%
\pgfpathlineto{\pgfqpoint{0.948177in}{1.421369in}}%
\pgfpathlineto{\pgfqpoint{0.949251in}{1.422012in}}%
\pgfpathlineto{\pgfqpoint{0.950325in}{1.421301in}}%
\pgfpathlineto{\pgfqpoint{0.953546in}{1.421183in}}%
\pgfpathlineto{\pgfqpoint{0.956768in}{1.417108in}}%
\pgfpathlineto{\pgfqpoint{0.962137in}{1.416168in}}%
\pgfpathlineto{\pgfqpoint{0.963211in}{1.414959in}}%
\pgfpathlineto{\pgfqpoint{0.965358in}{1.415357in}}%
\pgfpathlineto{\pgfqpoint{0.969654in}{1.414536in}}%
\pgfpathlineto{\pgfqpoint{0.970728in}{1.414950in}}%
\pgfpathlineto{\pgfqpoint{0.972875in}{1.414122in}}%
\pgfpathlineto{\pgfqpoint{0.977171in}{1.414253in}}%
\pgfpathlineto{\pgfqpoint{0.979318in}{1.414835in}}%
\pgfpathlineto{\pgfqpoint{0.980392in}{1.415054in}}%
\pgfpathlineto{\pgfqpoint{0.985761in}{1.415127in}}%
\pgfpathlineto{\pgfqpoint{0.987909in}{1.415600in}}%
\pgfpathlineto{\pgfqpoint{0.994352in}{1.415214in}}%
\pgfpathlineto{\pgfqpoint{0.995426in}{1.414974in}}%
\pgfpathlineto{\pgfqpoint{0.998647in}{1.414981in}}%
\pgfpathlineto{\pgfqpoint{0.999721in}{1.413591in}}%
\pgfpathlineto{\pgfqpoint{1.001869in}{1.414148in}}%
\pgfpathlineto{\pgfqpoint{1.002943in}{1.413776in}}%
\pgfpathlineto{\pgfqpoint{1.009386in}{1.413494in}}%
\pgfpathlineto{\pgfqpoint{1.010460in}{1.412758in}}%
\pgfpathlineto{\pgfqpoint{1.013681in}{1.412055in}}%
\pgfpathlineto{\pgfqpoint{1.015829in}{1.410913in}}%
\pgfpathlineto{\pgfqpoint{1.021198in}{1.411218in}}%
\pgfpathlineto{\pgfqpoint{1.022272in}{1.410664in}}%
\pgfpathlineto{\pgfqpoint{1.024419in}{1.411434in}}%
\pgfpathlineto{\pgfqpoint{1.039453in}{1.413586in}}%
\pgfpathlineto{\pgfqpoint{1.040527in}{1.414063in}}%
\pgfpathlineto{\pgfqpoint{1.045896in}{1.413688in}}%
\pgfpathlineto{\pgfqpoint{1.048044in}{1.411619in}}%
\pgfpathlineto{\pgfqpoint{1.058782in}{1.411498in}}%
\pgfpathlineto{\pgfqpoint{1.067373in}{1.412508in}}%
\pgfpathlineto{\pgfqpoint{1.069521in}{1.411673in}}%
\pgfpathlineto{\pgfqpoint{1.070595in}{1.412236in}}%
\pgfpathlineto{\pgfqpoint{1.073816in}{1.411130in}}%
\pgfpathlineto{\pgfqpoint{1.074890in}{1.411565in}}%
\pgfpathlineto{\pgfqpoint{1.075964in}{1.410851in}}%
\pgfpathlineto{\pgfqpoint{1.078111in}{1.410629in}}%
\pgfpathlineto{\pgfqpoint{1.081333in}{1.410125in}}%
\pgfpathlineto{\pgfqpoint{1.082407in}{1.409175in}}%
\pgfpathlineto{\pgfqpoint{1.085628in}{1.409687in}}%
\pgfpathlineto{\pgfqpoint{1.093145in}{1.410271in}}%
\pgfpathlineto{\pgfqpoint{1.099588in}{1.409861in}}%
\pgfpathlineto{\pgfqpoint{1.100662in}{1.410423in}}%
\pgfpathlineto{\pgfqpoint{1.106031in}{1.409945in}}%
\pgfpathlineto{\pgfqpoint{1.107105in}{1.410344in}}%
\pgfpathlineto{\pgfqpoint{1.115696in}{1.409962in}}%
\pgfpathlineto{\pgfqpoint{1.123213in}{1.411434in}}%
\pgfpathlineto{\pgfqpoint{1.126434in}{1.410383in}}%
\pgfpathlineto{\pgfqpoint{1.127508in}{1.410866in}}%
\pgfpathlineto{\pgfqpoint{1.130729in}{1.408625in}}%
\pgfpathlineto{\pgfqpoint{1.135025in}{1.408468in}}%
\pgfpathlineto{\pgfqpoint{1.138246in}{1.407660in}}%
\pgfpathlineto{\pgfqpoint{1.144689in}{1.407937in}}%
\pgfpathlineto{\pgfqpoint{1.145763in}{1.407303in}}%
\pgfpathlineto{\pgfqpoint{1.152206in}{1.408050in}}%
\pgfpathlineto{\pgfqpoint{1.153280in}{1.408270in}}%
\pgfpathlineto{\pgfqpoint{1.157575in}{1.408581in}}%
\pgfpathlineto{\pgfqpoint{1.159723in}{1.408476in}}%
\pgfpathlineto{\pgfqpoint{1.160797in}{1.407482in}}%
\pgfpathlineto{\pgfqpoint{1.166166in}{1.407005in}}%
\pgfpathlineto{\pgfqpoint{1.168314in}{1.406477in}}%
\pgfpathlineto{\pgfqpoint{1.174757in}{1.406879in}}%
\pgfpathlineto{\pgfqpoint{1.175831in}{1.406417in}}%
\pgfpathlineto{\pgfqpoint{1.180126in}{1.406146in}}%
\pgfpathlineto{\pgfqpoint{1.183348in}{1.404724in}}%
\pgfpathlineto{\pgfqpoint{1.188717in}{1.403965in}}%
\pgfpathlineto{\pgfqpoint{1.190864in}{1.403185in}}%
\pgfpathlineto{\pgfqpoint{1.194086in}{1.403573in}}%
\pgfpathlineto{\pgfqpoint{1.198381in}{1.401520in}}%
\pgfpathlineto{\pgfqpoint{1.209120in}{1.400176in}}%
\pgfpathlineto{\pgfqpoint{1.210194in}{1.399830in}}%
\pgfpathlineto{\pgfqpoint{1.211267in}{1.400172in}}%
\pgfpathlineto{\pgfqpoint{1.213415in}{1.399506in}}%
\pgfpathlineto{\pgfqpoint{1.220932in}{1.399331in}}%
\pgfpathlineto{\pgfqpoint{1.227375in}{1.399845in}}%
\pgfpathlineto{\pgfqpoint{1.228449in}{1.400237in}}%
\pgfpathlineto{\pgfqpoint{1.231670in}{1.400110in}}%
\pgfpathlineto{\pgfqpoint{1.232744in}{1.399082in}}%
\pgfpathlineto{\pgfqpoint{1.256369in}{1.398214in}}%
\pgfpathlineto{\pgfqpoint{1.257442in}{1.397563in}}%
\pgfpathlineto{\pgfqpoint{1.262812in}{1.397375in}}%
\pgfpathlineto{\pgfqpoint{1.271402in}{1.395850in}}%
\pgfpathlineto{\pgfqpoint{1.273550in}{1.395885in}}%
\pgfpathlineto{\pgfqpoint{1.278919in}{1.395303in}}%
\pgfpathlineto{\pgfqpoint{1.284288in}{1.395628in}}%
\pgfpathlineto{\pgfqpoint{1.287510in}{1.394665in}}%
\pgfpathlineto{\pgfqpoint{1.295027in}{1.395803in}}%
\pgfpathlineto{\pgfqpoint{1.296101in}{1.395143in}}%
\pgfpathlineto{\pgfqpoint{1.307913in}{1.394539in}}%
\pgfpathlineto{\pgfqpoint{1.310061in}{1.393533in}}%
\pgfpathlineto{\pgfqpoint{1.314356in}{1.393624in}}%
\pgfpathlineto{\pgfqpoint{1.318651in}{1.392742in}}%
\pgfpathlineto{\pgfqpoint{1.325094in}{1.392185in}}%
\pgfpathlineto{\pgfqpoint{1.340128in}{1.391742in}}%
\pgfpathlineto{\pgfqpoint{1.341202in}{1.392059in}}%
\pgfpathlineto{\pgfqpoint{1.374491in}{1.392910in}}%
\pgfpathlineto{\pgfqpoint{1.386303in}{1.391160in}}%
\pgfpathlineto{\pgfqpoint{1.390598in}{1.391226in}}%
\pgfpathlineto{\pgfqpoint{1.392746in}{1.390626in}}%
\pgfpathlineto{\pgfqpoint{1.393820in}{1.390897in}}%
\pgfpathlineto{\pgfqpoint{1.404558in}{1.391378in}}%
\pgfpathlineto{\pgfqpoint{1.405632in}{1.391625in}}%
\pgfpathlineto{\pgfqpoint{1.408854in}{1.391418in}}%
\pgfpathlineto{\pgfqpoint{1.430330in}{1.391192in}}%
\pgfpathlineto{\pgfqpoint{1.431404in}{1.390503in}}%
\pgfpathlineto{\pgfqpoint{1.444290in}{1.388189in}}%
\pgfpathlineto{\pgfqpoint{1.445364in}{1.387860in}}%
\pgfpathlineto{\pgfqpoint{1.446438in}{1.388125in}}%
\pgfpathlineto{\pgfqpoint{1.468989in}{1.388969in}}%
\pgfpathlineto{\pgfqpoint{1.481875in}{1.388722in}}%
\pgfpathlineto{\pgfqpoint{1.484022in}{1.388069in}}%
\pgfpathlineto{\pgfqpoint{1.491539in}{1.387733in}}%
\pgfpathlineto{\pgfqpoint{1.505499in}{1.386639in}}%
\pgfpathlineto{\pgfqpoint{1.509794in}{1.385997in}}%
\pgfpathlineto{\pgfqpoint{1.520533in}{1.385623in}}%
\pgfpathlineto{\pgfqpoint{1.549527in}{1.384946in}}%
\pgfpathlineto{\pgfqpoint{1.551674in}{1.384484in}}%
\pgfpathlineto{\pgfqpoint{1.555970in}{1.384236in}}%
\pgfpathlineto{\pgfqpoint{1.558117in}{1.383978in}}%
\pgfpathlineto{\pgfqpoint{1.571003in}{1.384433in}}%
\pgfpathlineto{\pgfqpoint{1.574225in}{1.384608in}}%
\pgfpathlineto{\pgfqpoint{1.579594in}{1.384510in}}%
\pgfpathlineto{\pgfqpoint{1.581742in}{1.384264in}}%
\pgfpathlineto{\pgfqpoint{1.608588in}{1.383819in}}%
\pgfpathlineto{\pgfqpoint{1.632212in}{1.382563in}}%
\pgfpathlineto{\pgfqpoint{1.634360in}{1.382296in}}%
\pgfpathlineto{\pgfqpoint{1.641877in}{1.381924in}}%
\pgfpathlineto{\pgfqpoint{1.649394in}{1.381573in}}%
\pgfpathlineto{\pgfqpoint{1.654763in}{1.381433in}}%
\pgfpathlineto{\pgfqpoint{1.663353in}{1.381344in}}%
\pgfpathlineto{\pgfqpoint{1.664427in}{1.381176in}}%
\pgfpathlineto{\pgfqpoint{1.679461in}{1.381231in}}%
\pgfpathlineto{\pgfqpoint{1.692347in}{1.380894in}}%
\pgfpathlineto{\pgfqpoint{1.694495in}{1.380771in}}%
\pgfpathlineto{\pgfqpoint{1.705233in}{1.380269in}}%
\pgfpathlineto{\pgfqpoint{1.708455in}{1.379310in}}%
\pgfpathlineto{\pgfqpoint{1.722415in}{1.378838in}}%
\pgfpathlineto{\pgfqpoint{1.724562in}{1.378651in}}%
\pgfpathlineto{\pgfqpoint{1.750334in}{1.378641in}}%
\pgfpathlineto{\pgfqpoint{1.754630in}{1.378284in}}%
\pgfpathlineto{\pgfqpoint{1.844832in}{1.373626in}}%
\pgfpathlineto{\pgfqpoint{1.870604in}{1.373338in}}%
\pgfpathlineto{\pgfqpoint{1.873826in}{1.372915in}}%
\pgfpathlineto{\pgfqpoint{1.874900in}{1.372755in}}%
\pgfpathlineto{\pgfqpoint{1.889933in}{1.372638in}}%
\pgfpathlineto{\pgfqpoint{1.909262in}{1.372445in}}%
\pgfpathlineto{\pgfqpoint{1.925370in}{1.372131in}}%
\pgfpathlineto{\pgfqpoint{1.932887in}{1.371909in}}%
\pgfpathlineto{\pgfqpoint{1.940404in}{1.371803in}}%
\pgfpathlineto{\pgfqpoint{1.946847in}{1.371699in}}%
\pgfpathlineto{\pgfqpoint{2.016646in}{1.370455in}}%
\pgfpathlineto{\pgfqpoint{2.024163in}{1.370292in}}%
\pgfpathlineto{\pgfqpoint{2.189534in}{1.367606in}}%
\pgfpathlineto{\pgfqpoint{2.248595in}{1.366944in}}%
\pgfpathlineto{\pgfqpoint{2.257186in}{1.366827in}}%
\pgfpathlineto{\pgfqpoint{2.291549in}{1.366770in}}%
\pgfpathlineto{\pgfqpoint{2.442960in}{1.366426in}}%
\pgfpathlineto{\pgfqpoint{2.972363in}{1.365759in}}%
\pgfpathlineto{\pgfqpoint{2.972363in}{1.365759in}}%
\pgfusepath{stroke}%
\end{pgfscope}%
\begin{pgfscope}%
\pgfsetrectcap%
\pgfsetmiterjoin%
\pgfsetlinewidth{0.803000pt}%
\definecolor{currentstroke}{rgb}{1.000000,1.000000,1.000000}%
\pgfsetstrokecolor{currentstroke}%
\pgfsetdash{}{0pt}%
\pgfpathmoveto{\pgfqpoint{0.506453in}{1.347524in}}%
\pgfpathlineto{\pgfqpoint{0.506453in}{1.748409in}}%
\pgfusepath{stroke}%
\end{pgfscope}%
\begin{pgfscope}%
\pgfsetrectcap%
\pgfsetmiterjoin%
\pgfsetlinewidth{0.803000pt}%
\definecolor{currentstroke}{rgb}{1.000000,1.000000,1.000000}%
\pgfsetstrokecolor{currentstroke}%
\pgfsetdash{}{0pt}%
\pgfpathmoveto{\pgfqpoint{3.089787in}{1.347524in}}%
\pgfpathlineto{\pgfqpoint{3.089787in}{1.748409in}}%
\pgfusepath{stroke}%
\end{pgfscope}%
\begin{pgfscope}%
\pgfsetrectcap%
\pgfsetmiterjoin%
\pgfsetlinewidth{0.803000pt}%
\definecolor{currentstroke}{rgb}{1.000000,1.000000,1.000000}%
\pgfsetstrokecolor{currentstroke}%
\pgfsetdash{}{0pt}%
\pgfpathmoveto{\pgfqpoint{0.506453in}{1.347524in}}%
\pgfpathlineto{\pgfqpoint{3.089787in}{1.347524in}}%
\pgfusepath{stroke}%
\end{pgfscope}%
\begin{pgfscope}%
\pgfsetrectcap%
\pgfsetmiterjoin%
\pgfsetlinewidth{0.803000pt}%
\definecolor{currentstroke}{rgb}{1.000000,1.000000,1.000000}%
\pgfsetstrokecolor{currentstroke}%
\pgfsetdash{}{0pt}%
\pgfpathmoveto{\pgfqpoint{0.506453in}{1.748409in}}%
\pgfpathlineto{\pgfqpoint{3.089787in}{1.748409in}}%
\pgfusepath{stroke}%
\end{pgfscope}%
\begin{pgfscope}%
\definecolor{textcolor}{rgb}{0.150000,0.150000,0.150000}%
\pgfsetstrokecolor{textcolor}%
\pgfsetfillcolor{textcolor}%
\pgftext[x=1.798120in,y=1.831742in,,base]{\color{textcolor}\rmfamily\fontsize{16.800000}{20.160000}\selectfont UTX}%
\end{pgfscope}%
\begin{pgfscope}%
\pgfsetbuttcap%
\pgfsetmiterjoin%
\definecolor{currentfill}{rgb}{0.917647,0.917647,0.949020}%
\pgfsetfillcolor{currentfill}%
\pgfsetlinewidth{0.000000pt}%
\definecolor{currentstroke}{rgb}{0.000000,0.000000,0.000000}%
\pgfsetstrokecolor{currentstroke}%
\pgfsetstrokeopacity{0.000000}%
\pgfsetdash{}{0pt}%
\pgfpathmoveto{\pgfqpoint{4.123120in}{1.347524in}}%
\pgfpathlineto{\pgfqpoint{6.706453in}{1.347524in}}%
\pgfpathlineto{\pgfqpoint{6.706453in}{1.748409in}}%
\pgfpathlineto{\pgfqpoint{4.123120in}{1.748409in}}%
\pgfpathclose%
\pgfusepath{fill}%
\end{pgfscope}%
\begin{pgfscope}%
\pgfpathrectangle{\pgfqpoint{4.123120in}{1.347524in}}{\pgfqpoint{2.583333in}{0.400885in}}%
\pgfusepath{clip}%
\pgfsetroundcap%
\pgfsetroundjoin%
\pgfsetlinewidth{0.803000pt}%
\definecolor{currentstroke}{rgb}{1.000000,1.000000,1.000000}%
\pgfsetstrokecolor{currentstroke}%
\pgfsetdash{}{0pt}%
\pgfpathmoveto{\pgfqpoint{4.238397in}{1.347524in}}%
\pgfpathlineto{\pgfqpoint{4.238397in}{1.748409in}}%
\pgfusepath{stroke}%
\end{pgfscope}%
\begin{pgfscope}%
\definecolor{textcolor}{rgb}{0.150000,0.150000,0.150000}%
\pgfsetstrokecolor{textcolor}%
\pgfsetfillcolor{textcolor}%
\pgftext[x=4.238397in,y=1.250302in,,top]{\color{textcolor}\rmfamily\fontsize{14.000000}{16.800000}\selectfont 2012}%
\end{pgfscope}%
\begin{pgfscope}%
\pgfpathrectangle{\pgfqpoint{4.123120in}{1.347524in}}{\pgfqpoint{2.583333in}{0.400885in}}%
\pgfusepath{clip}%
\pgfsetroundcap%
\pgfsetroundjoin%
\pgfsetlinewidth{0.803000pt}%
\definecolor{currentstroke}{rgb}{1.000000,1.000000,1.000000}%
\pgfsetstrokecolor{currentstroke}%
\pgfsetdash{}{0pt}%
\pgfpathmoveto{\pgfqpoint{4.631422in}{1.347524in}}%
\pgfpathlineto{\pgfqpoint{4.631422in}{1.748409in}}%
\pgfusepath{stroke}%
\end{pgfscope}%
\begin{pgfscope}%
\definecolor{textcolor}{rgb}{0.150000,0.150000,0.150000}%
\pgfsetstrokecolor{textcolor}%
\pgfsetfillcolor{textcolor}%
\pgftext[x=4.631422in,y=1.250302in,,top]{\color{textcolor}\rmfamily\fontsize{14.000000}{16.800000}\selectfont 2013}%
\end{pgfscope}%
\begin{pgfscope}%
\pgfpathrectangle{\pgfqpoint{4.123120in}{1.347524in}}{\pgfqpoint{2.583333in}{0.400885in}}%
\pgfusepath{clip}%
\pgfsetroundcap%
\pgfsetroundjoin%
\pgfsetlinewidth{0.803000pt}%
\definecolor{currentstroke}{rgb}{1.000000,1.000000,1.000000}%
\pgfsetstrokecolor{currentstroke}%
\pgfsetdash{}{0pt}%
\pgfpathmoveto{\pgfqpoint{5.023373in}{1.347524in}}%
\pgfpathlineto{\pgfqpoint{5.023373in}{1.748409in}}%
\pgfusepath{stroke}%
\end{pgfscope}%
\begin{pgfscope}%
\definecolor{textcolor}{rgb}{0.150000,0.150000,0.150000}%
\pgfsetstrokecolor{textcolor}%
\pgfsetfillcolor{textcolor}%
\pgftext[x=5.023373in,y=1.250302in,,top]{\color{textcolor}\rmfamily\fontsize{14.000000}{16.800000}\selectfont 2014}%
\end{pgfscope}%
\begin{pgfscope}%
\pgfpathrectangle{\pgfqpoint{4.123120in}{1.347524in}}{\pgfqpoint{2.583333in}{0.400885in}}%
\pgfusepath{clip}%
\pgfsetroundcap%
\pgfsetroundjoin%
\pgfsetlinewidth{0.803000pt}%
\definecolor{currentstroke}{rgb}{1.000000,1.000000,1.000000}%
\pgfsetstrokecolor{currentstroke}%
\pgfsetdash{}{0pt}%
\pgfpathmoveto{\pgfqpoint{5.415324in}{1.347524in}}%
\pgfpathlineto{\pgfqpoint{5.415324in}{1.748409in}}%
\pgfusepath{stroke}%
\end{pgfscope}%
\begin{pgfscope}%
\definecolor{textcolor}{rgb}{0.150000,0.150000,0.150000}%
\pgfsetstrokecolor{textcolor}%
\pgfsetfillcolor{textcolor}%
\pgftext[x=5.415324in,y=1.250302in,,top]{\color{textcolor}\rmfamily\fontsize{14.000000}{16.800000}\selectfont 2015}%
\end{pgfscope}%
\begin{pgfscope}%
\pgfpathrectangle{\pgfqpoint{4.123120in}{1.347524in}}{\pgfqpoint{2.583333in}{0.400885in}}%
\pgfusepath{clip}%
\pgfsetroundcap%
\pgfsetroundjoin%
\pgfsetlinewidth{0.803000pt}%
\definecolor{currentstroke}{rgb}{1.000000,1.000000,1.000000}%
\pgfsetstrokecolor{currentstroke}%
\pgfsetdash{}{0pt}%
\pgfpathmoveto{\pgfqpoint{5.807275in}{1.347524in}}%
\pgfpathlineto{\pgfqpoint{5.807275in}{1.748409in}}%
\pgfusepath{stroke}%
\end{pgfscope}%
\begin{pgfscope}%
\definecolor{textcolor}{rgb}{0.150000,0.150000,0.150000}%
\pgfsetstrokecolor{textcolor}%
\pgfsetfillcolor{textcolor}%
\pgftext[x=5.807275in,y=1.250302in,,top]{\color{textcolor}\rmfamily\fontsize{14.000000}{16.800000}\selectfont 2016}%
\end{pgfscope}%
\begin{pgfscope}%
\pgfpathrectangle{\pgfqpoint{4.123120in}{1.347524in}}{\pgfqpoint{2.583333in}{0.400885in}}%
\pgfusepath{clip}%
\pgfsetroundcap%
\pgfsetroundjoin%
\pgfsetlinewidth{0.803000pt}%
\definecolor{currentstroke}{rgb}{1.000000,1.000000,1.000000}%
\pgfsetstrokecolor{currentstroke}%
\pgfsetdash{}{0pt}%
\pgfpathmoveto{\pgfqpoint{6.200300in}{1.347524in}}%
\pgfpathlineto{\pgfqpoint{6.200300in}{1.748409in}}%
\pgfusepath{stroke}%
\end{pgfscope}%
\begin{pgfscope}%
\definecolor{textcolor}{rgb}{0.150000,0.150000,0.150000}%
\pgfsetstrokecolor{textcolor}%
\pgfsetfillcolor{textcolor}%
\pgftext[x=6.200300in,y=1.250302in,,top]{\color{textcolor}\rmfamily\fontsize{14.000000}{16.800000}\selectfont 2017}%
\end{pgfscope}%
\begin{pgfscope}%
\pgfpathrectangle{\pgfqpoint{4.123120in}{1.347524in}}{\pgfqpoint{2.583333in}{0.400885in}}%
\pgfusepath{clip}%
\pgfsetroundcap%
\pgfsetroundjoin%
\pgfsetlinewidth{0.803000pt}%
\definecolor{currentstroke}{rgb}{1.000000,1.000000,1.000000}%
\pgfsetstrokecolor{currentstroke}%
\pgfsetdash{}{0pt}%
\pgfpathmoveto{\pgfqpoint{6.592251in}{1.347524in}}%
\pgfpathlineto{\pgfqpoint{6.592251in}{1.748409in}}%
\pgfusepath{stroke}%
\end{pgfscope}%
\begin{pgfscope}%
\definecolor{textcolor}{rgb}{0.150000,0.150000,0.150000}%
\pgfsetstrokecolor{textcolor}%
\pgfsetfillcolor{textcolor}%
\pgftext[x=6.592251in,y=1.250302in,,top]{\color{textcolor}\rmfamily\fontsize{14.000000}{16.800000}\selectfont 2018}%
\end{pgfscope}%
\begin{pgfscope}%
\pgfpathrectangle{\pgfqpoint{4.123120in}{1.347524in}}{\pgfqpoint{2.583333in}{0.400885in}}%
\pgfusepath{clip}%
\pgfsetroundcap%
\pgfsetroundjoin%
\pgfsetlinewidth{0.803000pt}%
\definecolor{currentstroke}{rgb}{1.000000,1.000000,1.000000}%
\pgfsetstrokecolor{currentstroke}%
\pgfsetdash{}{0pt}%
\pgfpathmoveto{\pgfqpoint{4.123120in}{1.365367in}}%
\pgfpathlineto{\pgfqpoint{6.706453in}{1.365367in}}%
\pgfusepath{stroke}%
\end{pgfscope}%
\begin{pgfscope}%
\definecolor{textcolor}{rgb}{0.150000,0.150000,0.150000}%
\pgfsetstrokecolor{textcolor}%
\pgfsetfillcolor{textcolor}%
\pgftext[x=3.902186in,y=1.291501in,left,base]{\color{textcolor}\rmfamily\fontsize{14.000000}{16.800000}\selectfont 0}%
\end{pgfscope}%
\begin{pgfscope}%
\pgfpathrectangle{\pgfqpoint{4.123120in}{1.347524in}}{\pgfqpoint{2.583333in}{0.400885in}}%
\pgfusepath{clip}%
\pgfsetroundcap%
\pgfsetroundjoin%
\pgfsetlinewidth{0.803000pt}%
\definecolor{currentstroke}{rgb}{1.000000,1.000000,1.000000}%
\pgfsetstrokecolor{currentstroke}%
\pgfsetdash{}{0pt}%
\pgfpathmoveto{\pgfqpoint{4.123120in}{1.571496in}}%
\pgfpathlineto{\pgfqpoint{6.706453in}{1.571496in}}%
\pgfusepath{stroke}%
\end{pgfscope}%
\begin{pgfscope}%
\definecolor{textcolor}{rgb}{0.150000,0.150000,0.150000}%
\pgfsetstrokecolor{textcolor}%
\pgfsetfillcolor{textcolor}%
\pgftext[x=3.902186in,y=1.497630in,left,base]{\color{textcolor}\rmfamily\fontsize{14.000000}{16.800000}\selectfont 1}%
\end{pgfscope}%
\begin{pgfscope}%
\pgfpathrectangle{\pgfqpoint{4.123120in}{1.347524in}}{\pgfqpoint{2.583333in}{0.400885in}}%
\pgfusepath{clip}%
\pgfsetroundcap%
\pgfsetroundjoin%
\pgfsetlinewidth{1.505625pt}%
\definecolor{currentstroke}{rgb}{0.000000,0.000000,0.000000}%
\pgfsetstrokecolor{currentstroke}%
\pgfsetdash{}{0pt}%
\pgfpathmoveto{\pgfqpoint{4.240544in}{1.571496in}}%
\pgfpathlineto{\pgfqpoint{4.242692in}{1.567438in}}%
\pgfpathlineto{\pgfqpoint{4.243766in}{1.566848in}}%
\pgfpathlineto{\pgfqpoint{4.246987in}{1.567069in}}%
\pgfpathlineto{\pgfqpoint{4.250209in}{1.569947in}}%
\pgfpathlineto{\pgfqpoint{4.258800in}{1.570242in}}%
\pgfpathlineto{\pgfqpoint{4.262021in}{1.567217in}}%
\pgfpathlineto{\pgfqpoint{4.263095in}{1.564045in}}%
\pgfpathlineto{\pgfqpoint{4.264169in}{1.563454in}}%
\pgfpathlineto{\pgfqpoint{4.265243in}{1.561610in}}%
\pgfpathlineto{\pgfqpoint{4.266317in}{1.560946in}}%
\pgfpathlineto{\pgfqpoint{4.271686in}{1.564045in}}%
\pgfpathlineto{\pgfqpoint{4.272760in}{1.562790in}}%
\pgfpathlineto{\pgfqpoint{4.273833in}{1.564266in}}%
\pgfpathlineto{\pgfqpoint{4.277055in}{1.565815in}}%
\pgfpathlineto{\pgfqpoint{4.278129in}{1.564709in}}%
\pgfpathlineto{\pgfqpoint{4.280276in}{1.564709in}}%
\pgfpathlineto{\pgfqpoint{4.281350in}{1.563454in}}%
\pgfpathlineto{\pgfqpoint{4.284572in}{1.565815in}}%
\pgfpathlineto{\pgfqpoint{4.286719in}{1.564192in}}%
\pgfpathlineto{\pgfqpoint{4.287793in}{1.565373in}}%
\pgfpathlineto{\pgfqpoint{4.288867in}{1.567512in}}%
\pgfpathlineto{\pgfqpoint{4.293162in}{1.567660in}}%
\pgfpathlineto{\pgfqpoint{4.294236in}{1.566184in}}%
\pgfpathlineto{\pgfqpoint{4.296384in}{1.565815in}}%
\pgfpathlineto{\pgfqpoint{4.301753in}{1.565668in}}%
\pgfpathlineto{\pgfqpoint{4.303901in}{1.568619in}}%
\pgfpathlineto{\pgfqpoint{4.307122in}{1.570389in}}%
\pgfpathlineto{\pgfqpoint{4.308196in}{1.568766in}}%
\pgfpathlineto{\pgfqpoint{4.309270in}{1.569725in}}%
\pgfpathlineto{\pgfqpoint{4.310344in}{1.571496in}}%
\pgfpathlineto{\pgfqpoint{4.311418in}{1.570906in}}%
\pgfpathlineto{\pgfqpoint{4.318935in}{1.573340in}}%
\pgfpathlineto{\pgfqpoint{4.325378in}{1.573857in}}%
\pgfpathlineto{\pgfqpoint{4.326451in}{1.572603in}}%
\pgfpathlineto{\pgfqpoint{4.329673in}{1.572086in}}%
\pgfpathlineto{\pgfqpoint{4.331821in}{1.566479in}}%
\pgfpathlineto{\pgfqpoint{4.332894in}{1.565520in}}%
\pgfpathlineto{\pgfqpoint{4.333968in}{1.566332in}}%
\pgfpathlineto{\pgfqpoint{4.337190in}{1.567881in}}%
\pgfpathlineto{\pgfqpoint{4.339338in}{1.567143in}}%
\pgfpathlineto{\pgfqpoint{4.340411in}{1.565963in}}%
\pgfpathlineto{\pgfqpoint{4.344707in}{1.564856in}}%
\pgfpathlineto{\pgfqpoint{4.345781in}{1.561389in}}%
\pgfpathlineto{\pgfqpoint{4.346854in}{1.564413in}}%
\pgfpathlineto{\pgfqpoint{4.347928in}{1.565373in}}%
\pgfpathlineto{\pgfqpoint{4.349002in}{1.563823in}}%
\pgfpathlineto{\pgfqpoint{4.352224in}{1.564709in}}%
\pgfpathlineto{\pgfqpoint{4.353297in}{1.566332in}}%
\pgfpathlineto{\pgfqpoint{4.354371in}{1.565963in}}%
\pgfpathlineto{\pgfqpoint{4.356519in}{1.571643in}}%
\pgfpathlineto{\pgfqpoint{4.359740in}{1.570758in}}%
\pgfpathlineto{\pgfqpoint{4.360814in}{1.575775in}}%
\pgfpathlineto{\pgfqpoint{4.361888in}{1.575627in}}%
\pgfpathlineto{\pgfqpoint{4.362962in}{1.579169in}}%
\pgfpathlineto{\pgfqpoint{4.370479in}{1.581825in}}%
\pgfpathlineto{\pgfqpoint{4.371553in}{1.579759in}}%
\pgfpathlineto{\pgfqpoint{4.375848in}{1.581308in}}%
\pgfpathlineto{\pgfqpoint{4.376922in}{1.579759in}}%
\pgfpathlineto{\pgfqpoint{4.377996in}{1.581308in}}%
\pgfpathlineto{\pgfqpoint{4.379070in}{1.584554in}}%
\pgfpathlineto{\pgfqpoint{4.382291in}{1.583152in}}%
\pgfpathlineto{\pgfqpoint{4.383365in}{1.583964in}}%
\pgfpathlineto{\pgfqpoint{4.384439in}{1.583079in}}%
\pgfpathlineto{\pgfqpoint{4.385513in}{1.585735in}}%
\pgfpathlineto{\pgfqpoint{4.386586in}{1.586546in}}%
\pgfpathlineto{\pgfqpoint{4.389808in}{1.585513in}}%
\pgfpathlineto{\pgfqpoint{4.390882in}{1.585808in}}%
\pgfpathlineto{\pgfqpoint{4.391956in}{1.585218in}}%
\pgfpathlineto{\pgfqpoint{4.394103in}{1.586103in}}%
\pgfpathlineto{\pgfqpoint{4.398399in}{1.587727in}}%
\pgfpathlineto{\pgfqpoint{4.399472in}{1.585956in}}%
\pgfpathlineto{\pgfqpoint{4.400546in}{1.587136in}}%
\pgfpathlineto{\pgfqpoint{4.401620in}{1.583890in}}%
\pgfpathlineto{\pgfqpoint{4.404842in}{1.585513in}}%
\pgfpathlineto{\pgfqpoint{4.405916in}{1.584776in}}%
\pgfpathlineto{\pgfqpoint{4.406989in}{1.587800in}}%
\pgfpathlineto{\pgfqpoint{4.408063in}{1.587136in}}%
\pgfpathlineto{\pgfqpoint{4.409137in}{1.591415in}}%
\pgfpathlineto{\pgfqpoint{4.412359in}{1.592006in}}%
\pgfpathlineto{\pgfqpoint{4.413432in}{1.594071in}}%
\pgfpathlineto{\pgfqpoint{4.414506in}{1.594293in}}%
\pgfpathlineto{\pgfqpoint{4.415580in}{1.598350in}}%
\pgfpathlineto{\pgfqpoint{4.416654in}{1.597317in}}%
\pgfpathlineto{\pgfqpoint{4.419875in}{1.598719in}}%
\pgfpathlineto{\pgfqpoint{4.420949in}{1.598276in}}%
\pgfpathlineto{\pgfqpoint{4.422023in}{1.595989in}}%
\pgfpathlineto{\pgfqpoint{4.423097in}{1.596137in}}%
\pgfpathlineto{\pgfqpoint{4.424171in}{1.599457in}}%
\pgfpathlineto{\pgfqpoint{4.427392in}{1.597834in}}%
\pgfpathlineto{\pgfqpoint{4.428466in}{1.598867in}}%
\pgfpathlineto{\pgfqpoint{4.429540in}{1.598572in}}%
\pgfpathlineto{\pgfqpoint{4.430614in}{1.599531in}}%
\pgfpathlineto{\pgfqpoint{4.431688in}{1.602039in}}%
\pgfpathlineto{\pgfqpoint{4.434909in}{1.604695in}}%
\pgfpathlineto{\pgfqpoint{4.439205in}{1.604621in}}%
\pgfpathlineto{\pgfqpoint{4.442426in}{1.606318in}}%
\pgfpathlineto{\pgfqpoint{4.443500in}{1.606097in}}%
\pgfpathlineto{\pgfqpoint{4.444574in}{1.607203in}}%
\pgfpathlineto{\pgfqpoint{4.445648in}{1.605949in}}%
\pgfpathlineto{\pgfqpoint{4.446721in}{1.608900in}}%
\pgfpathlineto{\pgfqpoint{4.449943in}{1.609195in}}%
\pgfpathlineto{\pgfqpoint{4.452091in}{1.612515in}}%
\pgfpathlineto{\pgfqpoint{4.453164in}{1.605285in}}%
\pgfpathlineto{\pgfqpoint{4.457460in}{1.604252in}}%
\pgfpathlineto{\pgfqpoint{4.458534in}{1.601080in}}%
\pgfpathlineto{\pgfqpoint{4.459607in}{1.600859in}}%
\pgfpathlineto{\pgfqpoint{4.461755in}{1.607203in}}%
\pgfpathlineto{\pgfqpoint{4.464977in}{1.607498in}}%
\pgfpathlineto{\pgfqpoint{4.467124in}{1.608900in}}%
\pgfpathlineto{\pgfqpoint{4.468198in}{1.605654in}}%
\pgfpathlineto{\pgfqpoint{4.469272in}{1.604842in}}%
\pgfpathlineto{\pgfqpoint{4.472494in}{1.606097in}}%
\pgfpathlineto{\pgfqpoint{4.474641in}{1.603293in}}%
\pgfpathlineto{\pgfqpoint{4.476789in}{1.605580in}}%
\pgfpathlineto{\pgfqpoint{4.480010in}{1.603810in}}%
\pgfpathlineto{\pgfqpoint{4.481084in}{1.604105in}}%
\pgfpathlineto{\pgfqpoint{4.483232in}{1.602998in}}%
\pgfpathlineto{\pgfqpoint{4.484306in}{1.602703in}}%
\pgfpathlineto{\pgfqpoint{4.487527in}{1.600711in}}%
\pgfpathlineto{\pgfqpoint{4.488601in}{1.596358in}}%
\pgfpathlineto{\pgfqpoint{4.490749in}{1.592891in}}%
\pgfpathlineto{\pgfqpoint{4.491823in}{1.597908in}}%
\pgfpathlineto{\pgfqpoint{4.496118in}{1.595178in}}%
\pgfpathlineto{\pgfqpoint{4.497192in}{1.597465in}}%
\pgfpathlineto{\pgfqpoint{4.498266in}{1.595694in}}%
\pgfpathlineto{\pgfqpoint{4.505782in}{1.603146in}}%
\pgfpathlineto{\pgfqpoint{4.506856in}{1.600859in}}%
\pgfpathlineto{\pgfqpoint{4.511152in}{1.603662in}}%
\pgfpathlineto{\pgfqpoint{4.513299in}{1.610892in}}%
\pgfpathlineto{\pgfqpoint{4.514373in}{1.605211in}}%
\pgfpathlineto{\pgfqpoint{4.517595in}{1.605506in}}%
\pgfpathlineto{\pgfqpoint{4.520816in}{1.610376in}}%
\pgfpathlineto{\pgfqpoint{4.521890in}{1.611187in}}%
\pgfpathlineto{\pgfqpoint{4.526185in}{1.611040in}}%
\pgfpathlineto{\pgfqpoint{4.527259in}{1.610892in}}%
\pgfpathlineto{\pgfqpoint{4.528333in}{1.611851in}}%
\pgfpathlineto{\pgfqpoint{4.529407in}{1.610818in}}%
\pgfpathlineto{\pgfqpoint{4.533702in}{1.612368in}}%
\pgfpathlineto{\pgfqpoint{4.534776in}{1.614655in}}%
\pgfpathlineto{\pgfqpoint{4.535850in}{1.619893in}}%
\pgfpathlineto{\pgfqpoint{4.536924in}{1.621590in}}%
\pgfpathlineto{\pgfqpoint{4.540145in}{1.618934in}}%
\pgfpathlineto{\pgfqpoint{4.544441in}{1.608310in}}%
\pgfpathlineto{\pgfqpoint{4.547662in}{1.607646in}}%
\pgfpathlineto{\pgfqpoint{4.548736in}{1.605359in}}%
\pgfpathlineto{\pgfqpoint{4.549810in}{1.608900in}}%
\pgfpathlineto{\pgfqpoint{4.550884in}{1.614655in}}%
\pgfpathlineto{\pgfqpoint{4.551958in}{1.611261in}}%
\pgfpathlineto{\pgfqpoint{4.555179in}{1.609121in}}%
\pgfpathlineto{\pgfqpoint{4.556253in}{1.605359in}}%
\pgfpathlineto{\pgfqpoint{4.557327in}{1.606244in}}%
\pgfpathlineto{\pgfqpoint{4.558401in}{1.606318in}}%
\pgfpathlineto{\pgfqpoint{4.559474in}{1.608900in}}%
\pgfpathlineto{\pgfqpoint{4.564844in}{1.608457in}}%
\pgfpathlineto{\pgfqpoint{4.565917in}{1.611187in}}%
\pgfpathlineto{\pgfqpoint{4.566991in}{1.607793in}}%
\pgfpathlineto{\pgfqpoint{4.570213in}{1.606023in}}%
\pgfpathlineto{\pgfqpoint{4.571287in}{1.606613in}}%
\pgfpathlineto{\pgfqpoint{4.573434in}{1.597391in}}%
\pgfpathlineto{\pgfqpoint{4.574508in}{1.597539in}}%
\pgfpathlineto{\pgfqpoint{4.578804in}{1.597022in}}%
\pgfpathlineto{\pgfqpoint{4.579877in}{1.595399in}}%
\pgfpathlineto{\pgfqpoint{4.582025in}{1.590825in}}%
\pgfpathlineto{\pgfqpoint{4.585247in}{1.598498in}}%
\pgfpathlineto{\pgfqpoint{4.586320in}{1.598498in}}%
\pgfpathlineto{\pgfqpoint{4.589542in}{1.603662in}}%
\pgfpathlineto{\pgfqpoint{4.592763in}{1.601154in}}%
\pgfpathlineto{\pgfqpoint{4.593837in}{1.599383in}}%
\pgfpathlineto{\pgfqpoint{4.595985in}{1.604990in}}%
\pgfpathlineto{\pgfqpoint{4.597059in}{1.605580in}}%
\pgfpathlineto{\pgfqpoint{4.600280in}{1.605506in}}%
\pgfpathlineto{\pgfqpoint{4.601354in}{1.603146in}}%
\pgfpathlineto{\pgfqpoint{4.603502in}{1.607425in}}%
\pgfpathlineto{\pgfqpoint{4.604576in}{1.607203in}}%
\pgfpathlineto{\pgfqpoint{4.607797in}{1.605138in}}%
\pgfpathlineto{\pgfqpoint{4.609945in}{1.609269in}}%
\pgfpathlineto{\pgfqpoint{4.612093in}{1.606097in}}%
\pgfpathlineto{\pgfqpoint{4.615314in}{1.605580in}}%
\pgfpathlineto{\pgfqpoint{4.616388in}{1.604252in}}%
\pgfpathlineto{\pgfqpoint{4.617462in}{1.601596in}}%
\pgfpathlineto{\pgfqpoint{4.618536in}{1.603957in}}%
\pgfpathlineto{\pgfqpoint{4.619609in}{1.602629in}}%
\pgfpathlineto{\pgfqpoint{4.626052in}{1.602113in}}%
\pgfpathlineto{\pgfqpoint{4.627126in}{1.598940in}}%
\pgfpathlineto{\pgfqpoint{4.630348in}{1.601006in}}%
\pgfpathlineto{\pgfqpoint{4.632495in}{1.606392in}}%
\pgfpathlineto{\pgfqpoint{4.633569in}{1.605285in}}%
\pgfpathlineto{\pgfqpoint{4.634643in}{1.606613in}}%
\pgfpathlineto{\pgfqpoint{4.637865in}{1.608679in}}%
\pgfpathlineto{\pgfqpoint{4.638938in}{1.602777in}}%
\pgfpathlineto{\pgfqpoint{4.640012in}{1.602260in}}%
\pgfpathlineto{\pgfqpoint{4.641086in}{1.605506in}}%
\pgfpathlineto{\pgfqpoint{4.646455in}{1.596580in}}%
\pgfpathlineto{\pgfqpoint{4.647529in}{1.593997in}}%
\pgfpathlineto{\pgfqpoint{4.649677in}{1.599678in}}%
\pgfpathlineto{\pgfqpoint{4.653972in}{1.601891in}}%
\pgfpathlineto{\pgfqpoint{4.656120in}{1.599973in}}%
\pgfpathlineto{\pgfqpoint{4.657194in}{1.600416in}}%
\pgfpathlineto{\pgfqpoint{4.660415in}{1.601006in}}%
\pgfpathlineto{\pgfqpoint{4.661489in}{1.604990in}}%
\pgfpathlineto{\pgfqpoint{4.663637in}{1.605580in}}%
\pgfpathlineto{\pgfqpoint{4.664711in}{1.610818in}}%
\pgfpathlineto{\pgfqpoint{4.669006in}{1.610818in}}%
\pgfpathlineto{\pgfqpoint{4.670080in}{1.611851in}}%
\pgfpathlineto{\pgfqpoint{4.672227in}{1.609712in}}%
\pgfpathlineto{\pgfqpoint{4.675449in}{1.609490in}}%
\pgfpathlineto{\pgfqpoint{4.677597in}{1.610597in}}%
\pgfpathlineto{\pgfqpoint{4.678670in}{1.609564in}}%
\pgfpathlineto{\pgfqpoint{4.679744in}{1.609933in}}%
\pgfpathlineto{\pgfqpoint{4.684040in}{1.610523in}}%
\pgfpathlineto{\pgfqpoint{4.686187in}{1.613917in}}%
\pgfpathlineto{\pgfqpoint{4.687261in}{1.615466in}}%
\pgfpathlineto{\pgfqpoint{4.690483in}{1.617237in}}%
\pgfpathlineto{\pgfqpoint{4.692630in}{1.620704in}}%
\pgfpathlineto{\pgfqpoint{4.698000in}{1.624909in}}%
\pgfpathlineto{\pgfqpoint{4.699073in}{1.628082in}}%
\pgfpathlineto{\pgfqpoint{4.700147in}{1.625795in}}%
\pgfpathlineto{\pgfqpoint{4.701221in}{1.626901in}}%
\pgfpathlineto{\pgfqpoint{4.702295in}{1.629557in}}%
\pgfpathlineto{\pgfqpoint{4.705516in}{1.628746in}}%
\pgfpathlineto{\pgfqpoint{4.706590in}{1.631771in}}%
\pgfpathlineto{\pgfqpoint{4.707664in}{1.629483in}}%
\pgfpathlineto{\pgfqpoint{4.708738in}{1.632435in}}%
\pgfpathlineto{\pgfqpoint{4.709812in}{1.629926in}}%
\pgfpathlineto{\pgfqpoint{4.714107in}{1.635017in}}%
\pgfpathlineto{\pgfqpoint{4.715181in}{1.633098in}}%
\pgfpathlineto{\pgfqpoint{4.717329in}{1.635386in}}%
\pgfpathlineto{\pgfqpoint{4.720550in}{1.636197in}}%
\pgfpathlineto{\pgfqpoint{4.721624in}{1.637968in}}%
\pgfpathlineto{\pgfqpoint{4.722698in}{1.634943in}}%
\pgfpathlineto{\pgfqpoint{4.723772in}{1.636123in}}%
\pgfpathlineto{\pgfqpoint{4.728067in}{1.636492in}}%
\pgfpathlineto{\pgfqpoint{4.729141in}{1.638041in}}%
\pgfpathlineto{\pgfqpoint{4.730215in}{1.635238in}}%
\pgfpathlineto{\pgfqpoint{4.732362in}{1.638410in}}%
\pgfpathlineto{\pgfqpoint{4.735584in}{1.640550in}}%
\pgfpathlineto{\pgfqpoint{4.736658in}{1.640107in}}%
\pgfpathlineto{\pgfqpoint{4.739879in}{1.648518in}}%
\pgfpathlineto{\pgfqpoint{4.743101in}{1.647263in}}%
\pgfpathlineto{\pgfqpoint{4.744175in}{1.646231in}}%
\pgfpathlineto{\pgfqpoint{4.745248in}{1.641140in}}%
\pgfpathlineto{\pgfqpoint{4.747396in}{1.656190in}}%
\pgfpathlineto{\pgfqpoint{4.750618in}{1.655895in}}%
\pgfpathlineto{\pgfqpoint{4.751692in}{1.656633in}}%
\pgfpathlineto{\pgfqpoint{4.752765in}{1.653682in}}%
\pgfpathlineto{\pgfqpoint{4.753839in}{1.661650in}}%
\pgfpathlineto{\pgfqpoint{4.754913in}{1.663937in}}%
\pgfpathlineto{\pgfqpoint{4.758135in}{1.662978in}}%
\pgfpathlineto{\pgfqpoint{4.759208in}{1.665486in}}%
\pgfpathlineto{\pgfqpoint{4.760282in}{1.657076in}}%
\pgfpathlineto{\pgfqpoint{4.762430in}{1.658625in}}%
\pgfpathlineto{\pgfqpoint{4.765651in}{1.655084in}}%
\pgfpathlineto{\pgfqpoint{4.766725in}{1.659953in}}%
\pgfpathlineto{\pgfqpoint{4.767799in}{1.660986in}}%
\pgfpathlineto{\pgfqpoint{4.768873in}{1.658772in}}%
\pgfpathlineto{\pgfqpoint{4.769947in}{1.659805in}}%
\pgfpathlineto{\pgfqpoint{4.773168in}{1.657887in}}%
\pgfpathlineto{\pgfqpoint{4.775316in}{1.663715in}}%
\pgfpathlineto{\pgfqpoint{4.776390in}{1.661502in}}%
\pgfpathlineto{\pgfqpoint{4.777464in}{1.662314in}}%
\pgfpathlineto{\pgfqpoint{4.780685in}{1.658920in}}%
\pgfpathlineto{\pgfqpoint{4.782833in}{1.651911in}}%
\pgfpathlineto{\pgfqpoint{4.783907in}{1.654198in}}%
\pgfpathlineto{\pgfqpoint{4.784981in}{1.651469in}}%
\pgfpathlineto{\pgfqpoint{4.789276in}{1.648296in}}%
\pgfpathlineto{\pgfqpoint{4.790350in}{1.641288in}}%
\pgfpathlineto{\pgfqpoint{4.792497in}{1.635238in}}%
\pgfpathlineto{\pgfqpoint{4.795719in}{1.636271in}}%
\pgfpathlineto{\pgfqpoint{4.796793in}{1.637230in}}%
\pgfpathlineto{\pgfqpoint{4.797867in}{1.634205in}}%
\pgfpathlineto{\pgfqpoint{4.798940in}{1.643501in}}%
\pgfpathlineto{\pgfqpoint{4.800014in}{1.645050in}}%
\pgfpathlineto{\pgfqpoint{4.803236in}{1.646673in}}%
\pgfpathlineto{\pgfqpoint{4.805383in}{1.643132in}}%
\pgfpathlineto{\pgfqpoint{4.807531in}{1.649624in}}%
\pgfpathlineto{\pgfqpoint{4.810753in}{1.647632in}}%
\pgfpathlineto{\pgfqpoint{4.811826in}{1.652354in}}%
\pgfpathlineto{\pgfqpoint{4.813974in}{1.637894in}}%
\pgfpathlineto{\pgfqpoint{4.815048in}{1.640992in}}%
\pgfpathlineto{\pgfqpoint{4.818270in}{1.638779in}}%
\pgfpathlineto{\pgfqpoint{4.819343in}{1.646157in}}%
\pgfpathlineto{\pgfqpoint{4.821491in}{1.649255in}}%
\pgfpathlineto{\pgfqpoint{4.822565in}{1.645567in}}%
\pgfpathlineto{\pgfqpoint{4.825786in}{1.645714in}}%
\pgfpathlineto{\pgfqpoint{4.830082in}{1.650952in}}%
\pgfpathlineto{\pgfqpoint{4.833303in}{1.653092in}}%
\pgfpathlineto{\pgfqpoint{4.834377in}{1.651911in}}%
\pgfpathlineto{\pgfqpoint{4.835451in}{1.649550in}}%
\pgfpathlineto{\pgfqpoint{4.836525in}{1.653313in}}%
\pgfpathlineto{\pgfqpoint{4.837599in}{1.648813in}}%
\pgfpathlineto{\pgfqpoint{4.840820in}{1.646304in}}%
\pgfpathlineto{\pgfqpoint{4.842968in}{1.650657in}}%
\pgfpathlineto{\pgfqpoint{4.844042in}{1.646378in}}%
\pgfpathlineto{\pgfqpoint{4.845115in}{1.646231in}}%
\pgfpathlineto{\pgfqpoint{4.850485in}{1.648665in}}%
\pgfpathlineto{\pgfqpoint{4.852632in}{1.652280in}}%
\pgfpathlineto{\pgfqpoint{4.855854in}{1.654862in}}%
\pgfpathlineto{\pgfqpoint{4.858002in}{1.643575in}}%
\pgfpathlineto{\pgfqpoint{4.859075in}{1.646599in}}%
\pgfpathlineto{\pgfqpoint{4.860149in}{1.647927in}}%
\pgfpathlineto{\pgfqpoint{4.863371in}{1.647706in}}%
\pgfpathlineto{\pgfqpoint{4.865518in}{1.646157in}}%
\pgfpathlineto{\pgfqpoint{4.867666in}{1.642689in}}%
\pgfpathlineto{\pgfqpoint{4.870888in}{1.644534in}}%
\pgfpathlineto{\pgfqpoint{4.873035in}{1.640255in}}%
\pgfpathlineto{\pgfqpoint{4.874109in}{1.638337in}}%
\pgfpathlineto{\pgfqpoint{4.875183in}{1.633615in}}%
\pgfpathlineto{\pgfqpoint{4.878404in}{1.632435in}}%
\pgfpathlineto{\pgfqpoint{4.879478in}{1.634795in}}%
\pgfpathlineto{\pgfqpoint{4.880552in}{1.631180in}}%
\pgfpathlineto{\pgfqpoint{4.881626in}{1.629779in}}%
\pgfpathlineto{\pgfqpoint{4.882700in}{1.633098in}}%
\pgfpathlineto{\pgfqpoint{4.885921in}{1.629336in}}%
\pgfpathlineto{\pgfqpoint{4.886995in}{1.629336in}}%
\pgfpathlineto{\pgfqpoint{4.888069in}{1.627196in}}%
\pgfpathlineto{\pgfqpoint{4.889143in}{1.634279in}}%
\pgfpathlineto{\pgfqpoint{4.890217in}{1.631771in}}%
\pgfpathlineto{\pgfqpoint{4.894512in}{1.624098in}}%
\pgfpathlineto{\pgfqpoint{4.895586in}{1.628377in}}%
\pgfpathlineto{\pgfqpoint{4.896660in}{1.627639in}}%
\pgfpathlineto{\pgfqpoint{4.897734in}{1.625942in}}%
\pgfpathlineto{\pgfqpoint{4.900955in}{1.623508in}}%
\pgfpathlineto{\pgfqpoint{4.902029in}{1.626680in}}%
\pgfpathlineto{\pgfqpoint{4.903103in}{1.626975in}}%
\pgfpathlineto{\pgfqpoint{4.905250in}{1.633910in}}%
\pgfpathlineto{\pgfqpoint{4.910620in}{1.639296in}}%
\pgfpathlineto{\pgfqpoint{4.911693in}{1.638115in}}%
\pgfpathlineto{\pgfqpoint{4.912767in}{1.634058in}}%
\pgfpathlineto{\pgfqpoint{4.915989in}{1.635164in}}%
\pgfpathlineto{\pgfqpoint{4.917063in}{1.631180in}}%
\pgfpathlineto{\pgfqpoint{4.918136in}{1.629336in}}%
\pgfpathlineto{\pgfqpoint{4.919210in}{1.633394in}}%
\pgfpathlineto{\pgfqpoint{4.920284in}{1.629631in}}%
\pgfpathlineto{\pgfqpoint{4.923506in}{1.627787in}}%
\pgfpathlineto{\pgfqpoint{4.924580in}{1.629557in}}%
\pgfpathlineto{\pgfqpoint{4.925653in}{1.628451in}}%
\pgfpathlineto{\pgfqpoint{4.927801in}{1.630221in}}%
\pgfpathlineto{\pgfqpoint{4.931023in}{1.631254in}}%
\pgfpathlineto{\pgfqpoint{4.932096in}{1.627270in}}%
\pgfpathlineto{\pgfqpoint{4.933170in}{1.628082in}}%
\pgfpathlineto{\pgfqpoint{4.934244in}{1.631844in}}%
\pgfpathlineto{\pgfqpoint{4.935318in}{1.633172in}}%
\pgfpathlineto{\pgfqpoint{4.938539in}{1.631549in}}%
\pgfpathlineto{\pgfqpoint{4.939613in}{1.628746in}}%
\pgfpathlineto{\pgfqpoint{4.940687in}{1.634058in}}%
\pgfpathlineto{\pgfqpoint{4.942835in}{1.649772in}}%
\pgfpathlineto{\pgfqpoint{4.946056in}{1.653018in}}%
\pgfpathlineto{\pgfqpoint{4.947130in}{1.656190in}}%
\pgfpathlineto{\pgfqpoint{4.949278in}{1.652059in}}%
\pgfpathlineto{\pgfqpoint{4.950352in}{1.653756in}}%
\pgfpathlineto{\pgfqpoint{4.953573in}{1.652944in}}%
\pgfpathlineto{\pgfqpoint{4.954647in}{1.655895in}}%
\pgfpathlineto{\pgfqpoint{4.955721in}{1.652723in}}%
\pgfpathlineto{\pgfqpoint{4.957869in}{1.652501in}}%
\pgfpathlineto{\pgfqpoint{4.961090in}{1.655821in}}%
\pgfpathlineto{\pgfqpoint{4.962164in}{1.650288in}}%
\pgfpathlineto{\pgfqpoint{4.963238in}{1.653165in}}%
\pgfpathlineto{\pgfqpoint{4.964312in}{1.650657in}}%
\pgfpathlineto{\pgfqpoint{4.965385in}{1.650878in}}%
\pgfpathlineto{\pgfqpoint{4.968607in}{1.649477in}}%
\pgfpathlineto{\pgfqpoint{4.969681in}{1.650583in}}%
\pgfpathlineto{\pgfqpoint{4.970755in}{1.649624in}}%
\pgfpathlineto{\pgfqpoint{4.971828in}{1.651247in}}%
\pgfpathlineto{\pgfqpoint{4.972902in}{1.651469in}}%
\pgfpathlineto{\pgfqpoint{4.976124in}{1.654051in}}%
\pgfpathlineto{\pgfqpoint{4.977198in}{1.654125in}}%
\pgfpathlineto{\pgfqpoint{4.978271in}{1.651985in}}%
\pgfpathlineto{\pgfqpoint{4.979345in}{1.651837in}}%
\pgfpathlineto{\pgfqpoint{4.980419in}{1.650952in}}%
\pgfpathlineto{\pgfqpoint{4.983641in}{1.649772in}}%
\pgfpathlineto{\pgfqpoint{4.984714in}{1.649993in}}%
\pgfpathlineto{\pgfqpoint{4.991158in}{1.645493in}}%
\pgfpathlineto{\pgfqpoint{4.992231in}{1.647411in}}%
\pgfpathlineto{\pgfqpoint{4.993305in}{1.646157in}}%
\pgfpathlineto{\pgfqpoint{4.994379in}{1.643501in}}%
\pgfpathlineto{\pgfqpoint{4.995453in}{1.646747in}}%
\pgfpathlineto{\pgfqpoint{4.998674in}{1.647263in}}%
\pgfpathlineto{\pgfqpoint{5.001896in}{1.639074in}}%
\pgfpathlineto{\pgfqpoint{5.002970in}{1.637451in}}%
\pgfpathlineto{\pgfqpoint{5.006191in}{1.639812in}}%
\pgfpathlineto{\pgfqpoint{5.007265in}{1.635828in}}%
\pgfpathlineto{\pgfqpoint{5.008339in}{1.640992in}}%
\pgfpathlineto{\pgfqpoint{5.009413in}{1.640771in}}%
\pgfpathlineto{\pgfqpoint{5.010487in}{1.638779in}}%
\pgfpathlineto{\pgfqpoint{5.013708in}{1.641656in}}%
\pgfpathlineto{\pgfqpoint{5.014782in}{1.643870in}}%
\pgfpathlineto{\pgfqpoint{5.016930in}{1.645050in}}%
\pgfpathlineto{\pgfqpoint{5.024447in}{1.644017in}}%
\pgfpathlineto{\pgfqpoint{5.025520in}{1.640697in}}%
\pgfpathlineto{\pgfqpoint{5.028742in}{1.642247in}}%
\pgfpathlineto{\pgfqpoint{5.029816in}{1.645714in}}%
\pgfpathlineto{\pgfqpoint{5.030890in}{1.644165in}}%
\pgfpathlineto{\pgfqpoint{5.031963in}{1.638410in}}%
\pgfpathlineto{\pgfqpoint{5.033037in}{1.639886in}}%
\pgfpathlineto{\pgfqpoint{5.036259in}{1.635754in}}%
\pgfpathlineto{\pgfqpoint{5.037333in}{1.635976in}}%
\pgfpathlineto{\pgfqpoint{5.038406in}{1.642837in}}%
\pgfpathlineto{\pgfqpoint{5.039480in}{1.644312in}}%
\pgfpathlineto{\pgfqpoint{5.045923in}{1.637451in}}%
\pgfpathlineto{\pgfqpoint{5.046997in}{1.640476in}}%
\pgfpathlineto{\pgfqpoint{5.048071in}{1.639148in}}%
\pgfpathlineto{\pgfqpoint{5.051292in}{1.639517in}}%
\pgfpathlineto{\pgfqpoint{5.052366in}{1.637599in}}%
\pgfpathlineto{\pgfqpoint{5.053440in}{1.639517in}}%
\pgfpathlineto{\pgfqpoint{5.054514in}{1.639148in}}%
\pgfpathlineto{\pgfqpoint{5.055588in}{1.641435in}}%
\pgfpathlineto{\pgfqpoint{5.058809in}{1.632139in}}%
\pgfpathlineto{\pgfqpoint{5.059883in}{1.634500in}}%
\pgfpathlineto{\pgfqpoint{5.062031in}{1.633689in}}%
\pgfpathlineto{\pgfqpoint{5.063105in}{1.634426in}}%
\pgfpathlineto{\pgfqpoint{5.066326in}{1.635017in}}%
\pgfpathlineto{\pgfqpoint{5.068474in}{1.637673in}}%
\pgfpathlineto{\pgfqpoint{5.069548in}{1.637304in}}%
\pgfpathlineto{\pgfqpoint{5.070622in}{1.632730in}}%
\pgfpathlineto{\pgfqpoint{5.074917in}{1.629705in}}%
\pgfpathlineto{\pgfqpoint{5.075991in}{1.632877in}}%
\pgfpathlineto{\pgfqpoint{5.077065in}{1.641952in}}%
\pgfpathlineto{\pgfqpoint{5.078138in}{1.637082in}}%
\pgfpathlineto{\pgfqpoint{5.081360in}{1.631107in}}%
\pgfpathlineto{\pgfqpoint{5.083508in}{1.631844in}}%
\pgfpathlineto{\pgfqpoint{5.084581in}{1.638410in}}%
\pgfpathlineto{\pgfqpoint{5.085655in}{1.638853in}}%
\pgfpathlineto{\pgfqpoint{5.088877in}{1.637304in}}%
\pgfpathlineto{\pgfqpoint{5.089951in}{1.640697in}}%
\pgfpathlineto{\pgfqpoint{5.091024in}{1.637746in}}%
\pgfpathlineto{\pgfqpoint{5.092098in}{1.638189in}}%
\pgfpathlineto{\pgfqpoint{5.093172in}{1.636418in}}%
\pgfpathlineto{\pgfqpoint{5.096394in}{1.635754in}}%
\pgfpathlineto{\pgfqpoint{5.099615in}{1.630000in}}%
\pgfpathlineto{\pgfqpoint{5.103911in}{1.631549in}}%
\pgfpathlineto{\pgfqpoint{5.104984in}{1.633836in}}%
\pgfpathlineto{\pgfqpoint{5.106058in}{1.631844in}}%
\pgfpathlineto{\pgfqpoint{5.107132in}{1.636787in}}%
\pgfpathlineto{\pgfqpoint{5.108206in}{1.635017in}}%
\pgfpathlineto{\pgfqpoint{5.111427in}{1.635607in}}%
\pgfpathlineto{\pgfqpoint{5.112501in}{1.636787in}}%
\pgfpathlineto{\pgfqpoint{5.113575in}{1.635607in}}%
\pgfpathlineto{\pgfqpoint{5.114649in}{1.639517in}}%
\pgfpathlineto{\pgfqpoint{5.115723in}{1.637968in}}%
\pgfpathlineto{\pgfqpoint{5.118944in}{1.638853in}}%
\pgfpathlineto{\pgfqpoint{5.122166in}{1.641952in}}%
\pgfpathlineto{\pgfqpoint{5.123240in}{1.641509in}}%
\pgfpathlineto{\pgfqpoint{5.126461in}{1.641952in}}%
\pgfpathlineto{\pgfqpoint{5.127535in}{1.645567in}}%
\pgfpathlineto{\pgfqpoint{5.128609in}{1.644239in}}%
\pgfpathlineto{\pgfqpoint{5.130757in}{1.639001in}}%
\pgfpathlineto{\pgfqpoint{5.133978in}{1.640107in}}%
\pgfpathlineto{\pgfqpoint{5.135052in}{1.638115in}}%
\pgfpathlineto{\pgfqpoint{5.136126in}{1.639148in}}%
\pgfpathlineto{\pgfqpoint{5.137200in}{1.642025in}}%
\pgfpathlineto{\pgfqpoint{5.141495in}{1.644239in}}%
\pgfpathlineto{\pgfqpoint{5.142569in}{1.643870in}}%
\pgfpathlineto{\pgfqpoint{5.143643in}{1.641066in}}%
\pgfpathlineto{\pgfqpoint{5.144716in}{1.634353in}}%
\pgfpathlineto{\pgfqpoint{5.145790in}{1.632361in}}%
\pgfpathlineto{\pgfqpoint{5.150086in}{1.637230in}}%
\pgfpathlineto{\pgfqpoint{5.151159in}{1.637009in}}%
\pgfpathlineto{\pgfqpoint{5.152233in}{1.639812in}}%
\pgfpathlineto{\pgfqpoint{5.153307in}{1.639222in}}%
\pgfpathlineto{\pgfqpoint{5.157602in}{1.641288in}}%
\pgfpathlineto{\pgfqpoint{5.159750in}{1.647190in}}%
\pgfpathlineto{\pgfqpoint{5.160824in}{1.647190in}}%
\pgfpathlineto{\pgfqpoint{5.164046in}{1.645271in}}%
\pgfpathlineto{\pgfqpoint{5.165119in}{1.643501in}}%
\pgfpathlineto{\pgfqpoint{5.166193in}{1.644460in}}%
\pgfpathlineto{\pgfqpoint{5.167267in}{1.644165in}}%
\pgfpathlineto{\pgfqpoint{5.168341in}{1.650583in}}%
\pgfpathlineto{\pgfqpoint{5.171562in}{1.651026in}}%
\pgfpathlineto{\pgfqpoint{5.172636in}{1.648149in}}%
\pgfpathlineto{\pgfqpoint{5.174784in}{1.652797in}}%
\pgfpathlineto{\pgfqpoint{5.175858in}{1.654493in}}%
\pgfpathlineto{\pgfqpoint{5.180153in}{1.653756in}}%
\pgfpathlineto{\pgfqpoint{5.181227in}{1.654493in}}%
\pgfpathlineto{\pgfqpoint{5.182301in}{1.654346in}}%
\pgfpathlineto{\pgfqpoint{5.183375in}{1.655748in}}%
\pgfpathlineto{\pgfqpoint{5.186596in}{1.656264in}}%
\pgfpathlineto{\pgfqpoint{5.187670in}{1.651837in}}%
\pgfpathlineto{\pgfqpoint{5.188744in}{1.651026in}}%
\pgfpathlineto{\pgfqpoint{5.190891in}{1.652649in}}%
\pgfpathlineto{\pgfqpoint{5.195187in}{1.653239in}}%
\pgfpathlineto{\pgfqpoint{5.196261in}{1.652354in}}%
\pgfpathlineto{\pgfqpoint{5.197335in}{1.650436in}}%
\pgfpathlineto{\pgfqpoint{5.198408in}{1.651247in}}%
\pgfpathlineto{\pgfqpoint{5.201630in}{1.651985in}}%
\pgfpathlineto{\pgfqpoint{5.202704in}{1.651395in}}%
\pgfpathlineto{\pgfqpoint{5.203778in}{1.652649in}}%
\pgfpathlineto{\pgfqpoint{5.204851in}{1.652944in}}%
\pgfpathlineto{\pgfqpoint{5.205925in}{1.652428in}}%
\pgfpathlineto{\pgfqpoint{5.209147in}{1.654420in}}%
\pgfpathlineto{\pgfqpoint{5.210221in}{1.651837in}}%
\pgfpathlineto{\pgfqpoint{5.211294in}{1.652575in}}%
\pgfpathlineto{\pgfqpoint{5.212368in}{1.651395in}}%
\pgfpathlineto{\pgfqpoint{5.213442in}{1.652059in}}%
\pgfpathlineto{\pgfqpoint{5.216664in}{1.649772in}}%
\pgfpathlineto{\pgfqpoint{5.218811in}{1.654051in}}%
\pgfpathlineto{\pgfqpoint{5.219885in}{1.654346in}}%
\pgfpathlineto{\pgfqpoint{5.224180in}{1.654567in}}%
\pgfpathlineto{\pgfqpoint{5.225254in}{1.651837in}}%
\pgfpathlineto{\pgfqpoint{5.226328in}{1.652649in}}%
\pgfpathlineto{\pgfqpoint{5.228476in}{1.660986in}}%
\pgfpathlineto{\pgfqpoint{5.231697in}{1.662240in}}%
\pgfpathlineto{\pgfqpoint{5.233845in}{1.664748in}}%
\pgfpathlineto{\pgfqpoint{5.234919in}{1.660986in}}%
\pgfpathlineto{\pgfqpoint{5.235993in}{1.663568in}}%
\pgfpathlineto{\pgfqpoint{5.239214in}{1.663273in}}%
\pgfpathlineto{\pgfqpoint{5.240288in}{1.664896in}}%
\pgfpathlineto{\pgfqpoint{5.241362in}{1.664453in}}%
\pgfpathlineto{\pgfqpoint{5.246731in}{1.668437in}}%
\pgfpathlineto{\pgfqpoint{5.247805in}{1.670724in}}%
\pgfpathlineto{\pgfqpoint{5.248879in}{1.669470in}}%
\pgfpathlineto{\pgfqpoint{5.249953in}{1.661576in}}%
\pgfpathlineto{\pgfqpoint{5.251026in}{1.658108in}}%
\pgfpathlineto{\pgfqpoint{5.254248in}{1.660395in}}%
\pgfpathlineto{\pgfqpoint{5.257469in}{1.651174in}}%
\pgfpathlineto{\pgfqpoint{5.258543in}{1.651469in}}%
\pgfpathlineto{\pgfqpoint{5.261765in}{1.651321in}}%
\pgfpathlineto{\pgfqpoint{5.263912in}{1.652797in}}%
\pgfpathlineto{\pgfqpoint{5.264986in}{1.653239in}}%
\pgfpathlineto{\pgfqpoint{5.266060in}{1.652059in}}%
\pgfpathlineto{\pgfqpoint{5.271429in}{1.652206in}}%
\pgfpathlineto{\pgfqpoint{5.272503in}{1.652501in}}%
\pgfpathlineto{\pgfqpoint{5.273577in}{1.651100in}}%
\pgfpathlineto{\pgfqpoint{5.278946in}{1.655748in}}%
\pgfpathlineto{\pgfqpoint{5.280020in}{1.655674in}}%
\pgfpathlineto{\pgfqpoint{5.281094in}{1.658035in}}%
\pgfpathlineto{\pgfqpoint{5.285389in}{1.657740in}}%
\pgfpathlineto{\pgfqpoint{5.286463in}{1.658404in}}%
\pgfpathlineto{\pgfqpoint{5.287537in}{1.657444in}}%
\pgfpathlineto{\pgfqpoint{5.288611in}{1.658772in}}%
\pgfpathlineto{\pgfqpoint{5.291832in}{1.656338in}}%
\pgfpathlineto{\pgfqpoint{5.292906in}{1.652649in}}%
\pgfpathlineto{\pgfqpoint{5.293980in}{1.651764in}}%
\pgfpathlineto{\pgfqpoint{5.295054in}{1.653313in}}%
\pgfpathlineto{\pgfqpoint{5.296128in}{1.649698in}}%
\pgfpathlineto{\pgfqpoint{5.299349in}{1.650657in}}%
\pgfpathlineto{\pgfqpoint{5.303645in}{1.661207in}}%
\pgfpathlineto{\pgfqpoint{5.306866in}{1.660174in}}%
\pgfpathlineto{\pgfqpoint{5.307940in}{1.658625in}}%
\pgfpathlineto{\pgfqpoint{5.309014in}{1.659510in}}%
\pgfpathlineto{\pgfqpoint{5.310088in}{1.656854in}}%
\pgfpathlineto{\pgfqpoint{5.311161in}{1.657740in}}%
\pgfpathlineto{\pgfqpoint{5.314383in}{1.657666in}}%
\pgfpathlineto{\pgfqpoint{5.315457in}{1.659067in}}%
\pgfpathlineto{\pgfqpoint{5.316531in}{1.655748in}}%
\pgfpathlineto{\pgfqpoint{5.317604in}{1.654936in}}%
\pgfpathlineto{\pgfqpoint{5.318678in}{1.657444in}}%
\pgfpathlineto{\pgfqpoint{5.321900in}{1.659584in}}%
\pgfpathlineto{\pgfqpoint{5.322974in}{1.657223in}}%
\pgfpathlineto{\pgfqpoint{5.324047in}{1.661650in}}%
\pgfpathlineto{\pgfqpoint{5.325121in}{1.656043in}}%
\pgfpathlineto{\pgfqpoint{5.326195in}{1.656116in}}%
\pgfpathlineto{\pgfqpoint{5.332638in}{1.648591in}}%
\pgfpathlineto{\pgfqpoint{5.333712in}{1.650952in}}%
\pgfpathlineto{\pgfqpoint{5.338007in}{1.654641in}}%
\pgfpathlineto{\pgfqpoint{5.339081in}{1.652354in}}%
\pgfpathlineto{\pgfqpoint{5.340155in}{1.651837in}}%
\pgfpathlineto{\pgfqpoint{5.341229in}{1.655084in}}%
\pgfpathlineto{\pgfqpoint{5.344450in}{1.658994in}}%
\pgfpathlineto{\pgfqpoint{5.345524in}{1.662166in}}%
\pgfpathlineto{\pgfqpoint{5.346598in}{1.661428in}}%
\pgfpathlineto{\pgfqpoint{5.347672in}{1.661797in}}%
\pgfpathlineto{\pgfqpoint{5.348746in}{1.663863in}}%
\pgfpathlineto{\pgfqpoint{5.351967in}{1.664748in}}%
\pgfpathlineto{\pgfqpoint{5.355189in}{1.664084in}}%
\pgfpathlineto{\pgfqpoint{5.356263in}{1.667552in}}%
\pgfpathlineto{\pgfqpoint{5.360558in}{1.666002in}}%
\pgfpathlineto{\pgfqpoint{5.361632in}{1.667257in}}%
\pgfpathlineto{\pgfqpoint{5.363779in}{1.671314in}}%
\pgfpathlineto{\pgfqpoint{5.367001in}{1.670724in}}%
\pgfpathlineto{\pgfqpoint{5.368075in}{1.669691in}}%
\pgfpathlineto{\pgfqpoint{5.369149in}{1.665338in}}%
\pgfpathlineto{\pgfqpoint{5.370223in}{1.663568in}}%
\pgfpathlineto{\pgfqpoint{5.371296in}{1.663642in}}%
\pgfpathlineto{\pgfqpoint{5.375592in}{1.658477in}}%
\pgfpathlineto{\pgfqpoint{5.376666in}{1.662682in}}%
\pgfpathlineto{\pgfqpoint{5.378813in}{1.665929in}}%
\pgfpathlineto{\pgfqpoint{5.382035in}{1.662609in}}%
\pgfpathlineto{\pgfqpoint{5.383109in}{1.657149in}}%
\pgfpathlineto{\pgfqpoint{5.384182in}{1.655231in}}%
\pgfpathlineto{\pgfqpoint{5.385256in}{1.655157in}}%
\pgfpathlineto{\pgfqpoint{5.386330in}{1.654125in}}%
\pgfpathlineto{\pgfqpoint{5.389552in}{1.655895in}}%
\pgfpathlineto{\pgfqpoint{5.390625in}{1.644091in}}%
\pgfpathlineto{\pgfqpoint{5.391699in}{1.639738in}}%
\pgfpathlineto{\pgfqpoint{5.392773in}{1.640771in}}%
\pgfpathlineto{\pgfqpoint{5.393847in}{1.636123in}}%
\pgfpathlineto{\pgfqpoint{5.397068in}{1.635238in}}%
\pgfpathlineto{\pgfqpoint{5.398142in}{1.635828in}}%
\pgfpathlineto{\pgfqpoint{5.400290in}{1.644903in}}%
\pgfpathlineto{\pgfqpoint{5.401364in}{1.644681in}}%
\pgfpathlineto{\pgfqpoint{5.405659in}{1.648591in}}%
\pgfpathlineto{\pgfqpoint{5.406733in}{1.648591in}}%
\pgfpathlineto{\pgfqpoint{5.408881in}{1.649698in}}%
\pgfpathlineto{\pgfqpoint{5.412102in}{1.647854in}}%
\pgfpathlineto{\pgfqpoint{5.413176in}{1.646526in}}%
\pgfpathlineto{\pgfqpoint{5.414250in}{1.643280in}}%
\pgfpathlineto{\pgfqpoint{5.416398in}{1.644386in}}%
\pgfpathlineto{\pgfqpoint{5.419619in}{1.642025in}}%
\pgfpathlineto{\pgfqpoint{5.420693in}{1.644829in}}%
\pgfpathlineto{\pgfqpoint{5.421767in}{1.643058in}}%
\pgfpathlineto{\pgfqpoint{5.422841in}{1.648960in}}%
\pgfpathlineto{\pgfqpoint{5.423914in}{1.646452in}}%
\pgfpathlineto{\pgfqpoint{5.428210in}{1.648960in}}%
\pgfpathlineto{\pgfqpoint{5.429284in}{1.647706in}}%
\pgfpathlineto{\pgfqpoint{5.430357in}{1.648518in}}%
\pgfpathlineto{\pgfqpoint{5.431431in}{1.653829in}}%
\pgfpathlineto{\pgfqpoint{5.436800in}{1.655379in}}%
\pgfpathlineto{\pgfqpoint{5.438948in}{1.648813in}}%
\pgfpathlineto{\pgfqpoint{5.442170in}{1.647632in}}%
\pgfpathlineto{\pgfqpoint{5.444317in}{1.642173in}}%
\pgfpathlineto{\pgfqpoint{5.445391in}{1.642542in}}%
\pgfpathlineto{\pgfqpoint{5.446465in}{1.640107in}}%
\pgfpathlineto{\pgfqpoint{5.449687in}{1.647780in}}%
\pgfpathlineto{\pgfqpoint{5.450760in}{1.652870in}}%
\pgfpathlineto{\pgfqpoint{5.452908in}{1.653092in}}%
\pgfpathlineto{\pgfqpoint{5.453982in}{1.661871in}}%
\pgfpathlineto{\pgfqpoint{5.457203in}{1.660469in}}%
\pgfpathlineto{\pgfqpoint{5.459351in}{1.664822in}}%
\pgfpathlineto{\pgfqpoint{5.461499in}{1.661797in}}%
\pgfpathlineto{\pgfqpoint{5.465794in}{1.660986in}}%
\pgfpathlineto{\pgfqpoint{5.466868in}{1.659584in}}%
\pgfpathlineto{\pgfqpoint{5.467942in}{1.659363in}}%
\pgfpathlineto{\pgfqpoint{5.469016in}{1.659731in}}%
\pgfpathlineto{\pgfqpoint{5.472237in}{1.658551in}}%
\pgfpathlineto{\pgfqpoint{5.473311in}{1.661207in}}%
\pgfpathlineto{\pgfqpoint{5.474385in}{1.661133in}}%
\pgfpathlineto{\pgfqpoint{5.476533in}{1.662609in}}%
\pgfpathlineto{\pgfqpoint{5.480828in}{1.663199in}}%
\pgfpathlineto{\pgfqpoint{5.481902in}{1.660322in}}%
\pgfpathlineto{\pgfqpoint{5.482976in}{1.659436in}}%
\pgfpathlineto{\pgfqpoint{5.484049in}{1.655674in}}%
\pgfpathlineto{\pgfqpoint{5.487271in}{1.655305in}}%
\pgfpathlineto{\pgfqpoint{5.488345in}{1.650952in}}%
\pgfpathlineto{\pgfqpoint{5.489419in}{1.651985in}}%
\pgfpathlineto{\pgfqpoint{5.490492in}{1.658330in}}%
\pgfpathlineto{\pgfqpoint{5.491566in}{1.658920in}}%
\pgfpathlineto{\pgfqpoint{5.494788in}{1.661576in}}%
\pgfpathlineto{\pgfqpoint{5.495862in}{1.659584in}}%
\pgfpathlineto{\pgfqpoint{5.496935in}{1.663199in}}%
\pgfpathlineto{\pgfqpoint{5.498009in}{1.661723in}}%
\pgfpathlineto{\pgfqpoint{5.499083in}{1.663273in}}%
\pgfpathlineto{\pgfqpoint{5.502305in}{1.663789in}}%
\pgfpathlineto{\pgfqpoint{5.503378in}{1.662387in}}%
\pgfpathlineto{\pgfqpoint{5.505526in}{1.656412in}}%
\pgfpathlineto{\pgfqpoint{5.509822in}{1.660617in}}%
\pgfpathlineto{\pgfqpoint{5.510895in}{1.657666in}}%
\pgfpathlineto{\pgfqpoint{5.511969in}{1.659436in}}%
\pgfpathlineto{\pgfqpoint{5.513043in}{1.662756in}}%
\pgfpathlineto{\pgfqpoint{5.517338in}{1.663715in}}%
\pgfpathlineto{\pgfqpoint{5.518412in}{1.661576in}}%
\pgfpathlineto{\pgfqpoint{5.519486in}{1.664010in}}%
\pgfpathlineto{\pgfqpoint{5.520560in}{1.663273in}}%
\pgfpathlineto{\pgfqpoint{5.521634in}{1.664601in}}%
\pgfpathlineto{\pgfqpoint{5.524855in}{1.663494in}}%
\pgfpathlineto{\pgfqpoint{5.527003in}{1.665634in}}%
\pgfpathlineto{\pgfqpoint{5.528077in}{1.664896in}}%
\pgfpathlineto{\pgfqpoint{5.529151in}{1.662609in}}%
\pgfpathlineto{\pgfqpoint{5.532372in}{1.665560in}}%
\pgfpathlineto{\pgfqpoint{5.533446in}{1.664306in}}%
\pgfpathlineto{\pgfqpoint{5.535594in}{1.669617in}}%
\pgfpathlineto{\pgfqpoint{5.536667in}{1.669470in}}%
\pgfpathlineto{\pgfqpoint{5.539889in}{1.669839in}}%
\pgfpathlineto{\pgfqpoint{5.540963in}{1.672642in}}%
\pgfpathlineto{\pgfqpoint{5.544184in}{1.671831in}}%
\pgfpathlineto{\pgfqpoint{5.547406in}{1.672495in}}%
\pgfpathlineto{\pgfqpoint{5.549554in}{1.667109in}}%
\pgfpathlineto{\pgfqpoint{5.550627in}{1.667699in}}%
\pgfpathlineto{\pgfqpoint{5.551701in}{1.670208in}}%
\pgfpathlineto{\pgfqpoint{5.555997in}{1.667035in}}%
\pgfpathlineto{\pgfqpoint{5.557070in}{1.667699in}}%
\pgfpathlineto{\pgfqpoint{5.558144in}{1.669175in}}%
\pgfpathlineto{\pgfqpoint{5.559218in}{1.668068in}}%
\pgfpathlineto{\pgfqpoint{5.563513in}{1.666593in}}%
\pgfpathlineto{\pgfqpoint{5.565661in}{1.668732in}}%
\pgfpathlineto{\pgfqpoint{5.566735in}{1.666961in}}%
\pgfpathlineto{\pgfqpoint{5.571030in}{1.665781in}}%
\pgfpathlineto{\pgfqpoint{5.572104in}{1.666740in}}%
\pgfpathlineto{\pgfqpoint{5.574252in}{1.665929in}}%
\pgfpathlineto{\pgfqpoint{5.579621in}{1.663715in}}%
\pgfpathlineto{\pgfqpoint{5.581769in}{1.652501in}}%
\pgfpathlineto{\pgfqpoint{5.584990in}{1.653756in}}%
\pgfpathlineto{\pgfqpoint{5.586064in}{1.653165in}}%
\pgfpathlineto{\pgfqpoint{5.587138in}{1.653977in}}%
\pgfpathlineto{\pgfqpoint{5.588212in}{1.655674in}}%
\pgfpathlineto{\pgfqpoint{5.589286in}{1.652575in}}%
\pgfpathlineto{\pgfqpoint{5.592507in}{1.651100in}}%
\pgfpathlineto{\pgfqpoint{5.593581in}{1.653608in}}%
\pgfpathlineto{\pgfqpoint{5.594655in}{1.652723in}}%
\pgfpathlineto{\pgfqpoint{5.595729in}{1.655748in}}%
\pgfpathlineto{\pgfqpoint{5.596802in}{1.653903in}}%
\pgfpathlineto{\pgfqpoint{5.600024in}{1.654272in}}%
\pgfpathlineto{\pgfqpoint{5.601098in}{1.655748in}}%
\pgfpathlineto{\pgfqpoint{5.602172in}{1.652870in}}%
\pgfpathlineto{\pgfqpoint{5.604319in}{1.654862in}}%
\pgfpathlineto{\pgfqpoint{5.608615in}{1.648739in}}%
\pgfpathlineto{\pgfqpoint{5.610762in}{1.652280in}}%
\pgfpathlineto{\pgfqpoint{5.615058in}{1.651321in}}%
\pgfpathlineto{\pgfqpoint{5.616132in}{1.652575in}}%
\pgfpathlineto{\pgfqpoint{5.617205in}{1.651837in}}%
\pgfpathlineto{\pgfqpoint{5.618279in}{1.649919in}}%
\pgfpathlineto{\pgfqpoint{5.619353in}{1.654420in}}%
\pgfpathlineto{\pgfqpoint{5.622575in}{1.655674in}}%
\pgfpathlineto{\pgfqpoint{5.623648in}{1.657002in}}%
\pgfpathlineto{\pgfqpoint{5.624722in}{1.656485in}}%
\pgfpathlineto{\pgfqpoint{5.625796in}{1.659584in}}%
\pgfpathlineto{\pgfqpoint{5.626870in}{1.658108in}}%
\pgfpathlineto{\pgfqpoint{5.630091in}{1.661207in}}%
\pgfpathlineto{\pgfqpoint{5.631165in}{1.654272in}}%
\pgfpathlineto{\pgfqpoint{5.632239in}{1.651100in}}%
\pgfpathlineto{\pgfqpoint{5.633313in}{1.650436in}}%
\pgfpathlineto{\pgfqpoint{5.634387in}{1.648518in}}%
\pgfpathlineto{\pgfqpoint{5.637608in}{1.647263in}}%
\pgfpathlineto{\pgfqpoint{5.638682in}{1.647632in}}%
\pgfpathlineto{\pgfqpoint{5.639756in}{1.651764in}}%
\pgfpathlineto{\pgfqpoint{5.641904in}{1.653165in}}%
\pgfpathlineto{\pgfqpoint{5.645125in}{1.654272in}}%
\pgfpathlineto{\pgfqpoint{5.646199in}{1.652428in}}%
\pgfpathlineto{\pgfqpoint{5.648347in}{1.652206in}}%
\pgfpathlineto{\pgfqpoint{5.649421in}{1.650510in}}%
\pgfpathlineto{\pgfqpoint{5.654790in}{1.659658in}}%
\pgfpathlineto{\pgfqpoint{5.656937in}{1.657444in}}%
\pgfpathlineto{\pgfqpoint{5.661233in}{1.657297in}}%
\pgfpathlineto{\pgfqpoint{5.662307in}{1.657149in}}%
\pgfpathlineto{\pgfqpoint{5.667676in}{1.640550in}}%
\pgfpathlineto{\pgfqpoint{5.668750in}{1.632951in}}%
\pgfpathlineto{\pgfqpoint{5.670897in}{1.649477in}}%
\pgfpathlineto{\pgfqpoint{5.671971in}{1.648739in}}%
\pgfpathlineto{\pgfqpoint{5.675193in}{1.648370in}}%
\pgfpathlineto{\pgfqpoint{5.676266in}{1.641509in}}%
\pgfpathlineto{\pgfqpoint{5.678414in}{1.646599in}}%
\pgfpathlineto{\pgfqpoint{5.679488in}{1.641066in}}%
\pgfpathlineto{\pgfqpoint{5.683783in}{1.647559in}}%
\pgfpathlineto{\pgfqpoint{5.684857in}{1.644534in}}%
\pgfpathlineto{\pgfqpoint{5.685931in}{1.644976in}}%
\pgfpathlineto{\pgfqpoint{5.687005in}{1.646673in}}%
\pgfpathlineto{\pgfqpoint{5.690226in}{1.646157in}}%
\pgfpathlineto{\pgfqpoint{5.691300in}{1.650583in}}%
\pgfpathlineto{\pgfqpoint{5.692374in}{1.649477in}}%
\pgfpathlineto{\pgfqpoint{5.694522in}{1.639517in}}%
\pgfpathlineto{\pgfqpoint{5.697743in}{1.640919in}}%
\pgfpathlineto{\pgfqpoint{5.699891in}{1.635902in}}%
\pgfpathlineto{\pgfqpoint{5.702039in}{1.637377in}}%
\pgfpathlineto{\pgfqpoint{5.707408in}{1.632951in}}%
\pgfpathlineto{\pgfqpoint{5.708482in}{1.629631in}}%
\pgfpathlineto{\pgfqpoint{5.709555in}{1.628893in}}%
\pgfpathlineto{\pgfqpoint{5.712777in}{1.635976in}}%
\pgfpathlineto{\pgfqpoint{5.713851in}{1.636345in}}%
\pgfpathlineto{\pgfqpoint{5.715999in}{1.640919in}}%
\pgfpathlineto{\pgfqpoint{5.717072in}{1.640476in}}%
\pgfpathlineto{\pgfqpoint{5.721368in}{1.641730in}}%
\pgfpathlineto{\pgfqpoint{5.722442in}{1.639443in}}%
\pgfpathlineto{\pgfqpoint{5.723515in}{1.643722in}}%
\pgfpathlineto{\pgfqpoint{5.727811in}{1.643870in}}%
\pgfpathlineto{\pgfqpoint{5.728885in}{1.647263in}}%
\pgfpathlineto{\pgfqpoint{5.729958in}{1.644903in}}%
\pgfpathlineto{\pgfqpoint{5.731032in}{1.651321in}}%
\pgfpathlineto{\pgfqpoint{5.732106in}{1.652944in}}%
\pgfpathlineto{\pgfqpoint{5.735328in}{1.654198in}}%
\pgfpathlineto{\pgfqpoint{5.736401in}{1.653018in}}%
\pgfpathlineto{\pgfqpoint{5.737475in}{1.654936in}}%
\pgfpathlineto{\pgfqpoint{5.738549in}{1.654493in}}%
\pgfpathlineto{\pgfqpoint{5.739623in}{1.657444in}}%
\pgfpathlineto{\pgfqpoint{5.742844in}{1.656854in}}%
\pgfpathlineto{\pgfqpoint{5.744992in}{1.652870in}}%
\pgfpathlineto{\pgfqpoint{5.746066in}{1.653239in}}%
\pgfpathlineto{\pgfqpoint{5.747140in}{1.650583in}}%
\pgfpathlineto{\pgfqpoint{5.751435in}{1.646378in}}%
\pgfpathlineto{\pgfqpoint{5.752509in}{1.647706in}}%
\pgfpathlineto{\pgfqpoint{5.754657in}{1.640919in}}%
\pgfpathlineto{\pgfqpoint{5.757878in}{1.646009in}}%
\pgfpathlineto{\pgfqpoint{5.758952in}{1.646231in}}%
\pgfpathlineto{\pgfqpoint{5.761100in}{1.650510in}}%
\pgfpathlineto{\pgfqpoint{5.762174in}{1.648149in}}%
\pgfpathlineto{\pgfqpoint{5.765395in}{1.645714in}}%
\pgfpathlineto{\pgfqpoint{5.766469in}{1.646895in}}%
\pgfpathlineto{\pgfqpoint{5.767543in}{1.645271in}}%
\pgfpathlineto{\pgfqpoint{5.769690in}{1.647190in}}%
\pgfpathlineto{\pgfqpoint{5.773986in}{1.649329in}}%
\pgfpathlineto{\pgfqpoint{5.776133in}{1.642984in}}%
\pgfpathlineto{\pgfqpoint{5.777207in}{1.650141in}}%
\pgfpathlineto{\pgfqpoint{5.780429in}{1.652354in}}%
\pgfpathlineto{\pgfqpoint{5.782576in}{1.648001in}}%
\pgfpathlineto{\pgfqpoint{5.783650in}{1.647706in}}%
\pgfpathlineto{\pgfqpoint{5.784724in}{1.644607in}}%
\pgfpathlineto{\pgfqpoint{5.789020in}{1.649182in}}%
\pgfpathlineto{\pgfqpoint{5.790093in}{1.655157in}}%
\pgfpathlineto{\pgfqpoint{5.792241in}{1.649255in}}%
\pgfpathlineto{\pgfqpoint{5.795463in}{1.651321in}}%
\pgfpathlineto{\pgfqpoint{5.797610in}{1.657887in}}%
\pgfpathlineto{\pgfqpoint{5.798684in}{1.656412in}}%
\pgfpathlineto{\pgfqpoint{5.802979in}{1.656633in}}%
\pgfpathlineto{\pgfqpoint{5.804053in}{1.659510in}}%
\pgfpathlineto{\pgfqpoint{5.806201in}{1.653313in}}%
\pgfpathlineto{\pgfqpoint{5.810496in}{1.651174in}}%
\pgfpathlineto{\pgfqpoint{5.811570in}{1.655084in}}%
\pgfpathlineto{\pgfqpoint{5.814792in}{1.648149in}}%
\pgfpathlineto{\pgfqpoint{5.818013in}{1.649772in}}%
\pgfpathlineto{\pgfqpoint{5.819087in}{1.648739in}}%
\pgfpathlineto{\pgfqpoint{5.820161in}{1.643796in}}%
\pgfpathlineto{\pgfqpoint{5.821235in}{1.648370in}}%
\pgfpathlineto{\pgfqpoint{5.822309in}{1.645567in}}%
\pgfpathlineto{\pgfqpoint{5.826604in}{1.648370in}}%
\pgfpathlineto{\pgfqpoint{5.827678in}{1.645567in}}%
\pgfpathlineto{\pgfqpoint{5.829825in}{1.662092in}}%
\pgfpathlineto{\pgfqpoint{5.833047in}{1.662019in}}%
\pgfpathlineto{\pgfqpoint{5.835195in}{1.674634in}}%
\pgfpathlineto{\pgfqpoint{5.836268in}{1.674487in}}%
\pgfpathlineto{\pgfqpoint{5.837342in}{1.680536in}}%
\pgfpathlineto{\pgfqpoint{5.840564in}{1.685553in}}%
\pgfpathlineto{\pgfqpoint{5.841638in}{1.680167in}}%
\pgfpathlineto{\pgfqpoint{5.842711in}{1.684668in}}%
\pgfpathlineto{\pgfqpoint{5.843785in}{1.683413in}}%
\pgfpathlineto{\pgfqpoint{5.844859in}{1.686881in}}%
\pgfpathlineto{\pgfqpoint{5.848081in}{1.685405in}}%
\pgfpathlineto{\pgfqpoint{5.849154in}{1.681643in}}%
\pgfpathlineto{\pgfqpoint{5.850228in}{1.680610in}}%
\pgfpathlineto{\pgfqpoint{5.851302in}{1.676847in}}%
\pgfpathlineto{\pgfqpoint{5.852376in}{1.681421in}}%
\pgfpathlineto{\pgfqpoint{5.857745in}{1.682749in}}%
\pgfpathlineto{\pgfqpoint{5.858819in}{1.686660in}}%
\pgfpathlineto{\pgfqpoint{5.859893in}{1.686143in}}%
\pgfpathlineto{\pgfqpoint{5.863114in}{1.687471in}}%
\pgfpathlineto{\pgfqpoint{5.864188in}{1.684668in}}%
\pgfpathlineto{\pgfqpoint{5.866336in}{1.687692in}}%
\pgfpathlineto{\pgfqpoint{5.870631in}{1.685332in}}%
\pgfpathlineto{\pgfqpoint{5.872779in}{1.694111in}}%
\pgfpathlineto{\pgfqpoint{5.873853in}{1.692635in}}%
\pgfpathlineto{\pgfqpoint{5.874927in}{1.692119in}}%
\pgfpathlineto{\pgfqpoint{5.879222in}{1.696250in}}%
\pgfpathlineto{\pgfqpoint{5.881370in}{1.695365in}}%
\pgfpathlineto{\pgfqpoint{5.882443in}{1.696693in}}%
\pgfpathlineto{\pgfqpoint{5.885665in}{1.696767in}}%
\pgfpathlineto{\pgfqpoint{5.886739in}{1.697578in}}%
\pgfpathlineto{\pgfqpoint{5.888887in}{1.703628in}}%
\pgfpathlineto{\pgfqpoint{5.889960in}{1.701193in}}%
\pgfpathlineto{\pgfqpoint{5.893182in}{1.702447in}}%
\pgfpathlineto{\pgfqpoint{5.895330in}{1.699054in}}%
\pgfpathlineto{\pgfqpoint{5.896403in}{1.703185in}}%
\pgfpathlineto{\pgfqpoint{5.900699in}{1.702152in}}%
\pgfpathlineto{\pgfqpoint{5.901773in}{1.706284in}}%
\pgfpathlineto{\pgfqpoint{5.904994in}{1.705989in}}%
\pgfpathlineto{\pgfqpoint{5.908216in}{1.708571in}}%
\pgfpathlineto{\pgfqpoint{5.909289in}{1.706505in}}%
\pgfpathlineto{\pgfqpoint{5.910363in}{1.706505in}}%
\pgfpathlineto{\pgfqpoint{5.911437in}{1.696841in}}%
\pgfpathlineto{\pgfqpoint{5.912511in}{1.697947in}}%
\pgfpathlineto{\pgfqpoint{5.915732in}{1.694332in}}%
\pgfpathlineto{\pgfqpoint{5.916806in}{1.696472in}}%
\pgfpathlineto{\pgfqpoint{5.917880in}{1.692266in}}%
\pgfpathlineto{\pgfqpoint{5.923249in}{1.695070in}}%
\pgfpathlineto{\pgfqpoint{5.924323in}{1.697283in}}%
\pgfpathlineto{\pgfqpoint{5.925397in}{1.695217in}}%
\pgfpathlineto{\pgfqpoint{5.926471in}{1.684225in}}%
\pgfpathlineto{\pgfqpoint{5.927545in}{1.687545in}}%
\pgfpathlineto{\pgfqpoint{5.930766in}{1.688873in}}%
\pgfpathlineto{\pgfqpoint{5.931840in}{1.686881in}}%
\pgfpathlineto{\pgfqpoint{5.932914in}{1.694849in}}%
\pgfpathlineto{\pgfqpoint{5.933988in}{1.690570in}}%
\pgfpathlineto{\pgfqpoint{5.935062in}{1.690053in}}%
\pgfpathlineto{\pgfqpoint{5.938283in}{1.692488in}}%
\pgfpathlineto{\pgfqpoint{5.939357in}{1.688356in}}%
\pgfpathlineto{\pgfqpoint{5.940431in}{1.689389in}}%
\pgfpathlineto{\pgfqpoint{5.941505in}{1.689389in}}%
\pgfpathlineto{\pgfqpoint{5.942578in}{1.691160in}}%
\pgfpathlineto{\pgfqpoint{5.945800in}{1.690939in}}%
\pgfpathlineto{\pgfqpoint{5.946874in}{1.693890in}}%
\pgfpathlineto{\pgfqpoint{5.947948in}{1.691381in}}%
\pgfpathlineto{\pgfqpoint{5.949021in}{1.693447in}}%
\pgfpathlineto{\pgfqpoint{5.950095in}{1.690053in}}%
\pgfpathlineto{\pgfqpoint{5.953317in}{1.691898in}}%
\pgfpathlineto{\pgfqpoint{5.955465in}{1.686512in}}%
\pgfpathlineto{\pgfqpoint{5.956538in}{1.681717in}}%
\pgfpathlineto{\pgfqpoint{5.957612in}{1.681864in}}%
\pgfpathlineto{\pgfqpoint{5.960834in}{1.678544in}}%
\pgfpathlineto{\pgfqpoint{5.962981in}{1.683118in}}%
\pgfpathlineto{\pgfqpoint{5.965129in}{1.687987in}}%
\pgfpathlineto{\pgfqpoint{5.969424in}{1.689758in}}%
\pgfpathlineto{\pgfqpoint{5.970498in}{1.686807in}}%
\pgfpathlineto{\pgfqpoint{5.972646in}{1.689906in}}%
\pgfpathlineto{\pgfqpoint{5.975867in}{1.688578in}}%
\pgfpathlineto{\pgfqpoint{5.976941in}{1.695217in}}%
\pgfpathlineto{\pgfqpoint{5.978015in}{1.693742in}}%
\pgfpathlineto{\pgfqpoint{5.979089in}{1.696472in}}%
\pgfpathlineto{\pgfqpoint{5.980163in}{1.701046in}}%
\pgfpathlineto{\pgfqpoint{5.983384in}{1.700456in}}%
\pgfpathlineto{\pgfqpoint{5.984458in}{1.703111in}}%
\pgfpathlineto{\pgfqpoint{5.985532in}{1.702152in}}%
\pgfpathlineto{\pgfqpoint{5.987680in}{1.708128in}}%
\pgfpathlineto{\pgfqpoint{5.990901in}{1.708054in}}%
\pgfpathlineto{\pgfqpoint{5.991975in}{1.710194in}}%
\pgfpathlineto{\pgfqpoint{5.993049in}{1.709751in}}%
\pgfpathlineto{\pgfqpoint{5.994123in}{1.713809in}}%
\pgfpathlineto{\pgfqpoint{5.995197in}{1.712260in}}%
\pgfpathlineto{\pgfqpoint{5.999492in}{1.714768in}}%
\pgfpathlineto{\pgfqpoint{6.000566in}{1.716317in}}%
\pgfpathlineto{\pgfqpoint{6.002713in}{1.723769in}}%
\pgfpathlineto{\pgfqpoint{6.007009in}{1.725687in}}%
\pgfpathlineto{\pgfqpoint{6.008083in}{1.727605in}}%
\pgfpathlineto{\pgfqpoint{6.009156in}{1.721924in}}%
\pgfpathlineto{\pgfqpoint{6.010230in}{1.725244in}}%
\pgfpathlineto{\pgfqpoint{6.013452in}{1.725465in}}%
\pgfpathlineto{\pgfqpoint{6.014526in}{1.722514in}}%
\pgfpathlineto{\pgfqpoint{6.015599in}{1.725908in}}%
\pgfpathlineto{\pgfqpoint{6.016673in}{1.724875in}}%
\pgfpathlineto{\pgfqpoint{6.017747in}{1.724875in}}%
\pgfpathlineto{\pgfqpoint{6.020969in}{1.725465in}}%
\pgfpathlineto{\pgfqpoint{6.022042in}{1.723990in}}%
\pgfpathlineto{\pgfqpoint{6.023116in}{1.723474in}}%
\pgfpathlineto{\pgfqpoint{6.024190in}{1.721850in}}%
\pgfpathlineto{\pgfqpoint{6.025264in}{1.726572in}}%
\pgfpathlineto{\pgfqpoint{6.028486in}{1.725097in}}%
\pgfpathlineto{\pgfqpoint{6.029559in}{1.718235in}}%
\pgfpathlineto{\pgfqpoint{6.030633in}{1.721555in}}%
\pgfpathlineto{\pgfqpoint{6.031707in}{1.718531in}}%
\pgfpathlineto{\pgfqpoint{6.032781in}{1.722072in}}%
\pgfpathlineto{\pgfqpoint{6.036002in}{1.716244in}}%
\pgfpathlineto{\pgfqpoint{6.037076in}{1.712997in}}%
\pgfpathlineto{\pgfqpoint{6.038150in}{1.712407in}}%
\pgfpathlineto{\pgfqpoint{6.039224in}{1.712555in}}%
\pgfpathlineto{\pgfqpoint{6.040298in}{1.710710in}}%
\pgfpathlineto{\pgfqpoint{6.044593in}{1.710858in}}%
\pgfpathlineto{\pgfqpoint{6.046741in}{1.712112in}}%
\pgfpathlineto{\pgfqpoint{6.047815in}{1.710784in}}%
\pgfpathlineto{\pgfqpoint{6.051036in}{1.710489in}}%
\pgfpathlineto{\pgfqpoint{6.052110in}{1.705030in}}%
\pgfpathlineto{\pgfqpoint{6.053184in}{1.707686in}}%
\pgfpathlineto{\pgfqpoint{6.055331in}{1.703038in}}%
\pgfpathlineto{\pgfqpoint{6.059627in}{1.704218in}}%
\pgfpathlineto{\pgfqpoint{6.060701in}{1.703554in}}%
\pgfpathlineto{\pgfqpoint{6.061775in}{1.704956in}}%
\pgfpathlineto{\pgfqpoint{6.062848in}{1.700603in}}%
\pgfpathlineto{\pgfqpoint{6.066070in}{1.703333in}}%
\pgfpathlineto{\pgfqpoint{6.067144in}{1.701857in}}%
\pgfpathlineto{\pgfqpoint{6.068218in}{1.702300in}}%
\pgfpathlineto{\pgfqpoint{6.070365in}{1.705841in}}%
\pgfpathlineto{\pgfqpoint{6.075734in}{1.711153in}}%
\pgfpathlineto{\pgfqpoint{6.076808in}{1.710415in}}%
\pgfpathlineto{\pgfqpoint{6.077882in}{1.698980in}}%
\pgfpathlineto{\pgfqpoint{6.081104in}{1.703849in}}%
\pgfpathlineto{\pgfqpoint{6.082177in}{1.696619in}}%
\pgfpathlineto{\pgfqpoint{6.083251in}{1.696841in}}%
\pgfpathlineto{\pgfqpoint{6.084325in}{1.700013in}}%
\pgfpathlineto{\pgfqpoint{6.085399in}{1.699349in}}%
\pgfpathlineto{\pgfqpoint{6.088620in}{1.694996in}}%
\pgfpathlineto{\pgfqpoint{6.089694in}{1.695439in}}%
\pgfpathlineto{\pgfqpoint{6.091842in}{1.702374in}}%
\pgfpathlineto{\pgfqpoint{6.092916in}{1.703775in}}%
\pgfpathlineto{\pgfqpoint{6.096137in}{1.701120in}}%
\pgfpathlineto{\pgfqpoint{6.097211in}{1.703333in}}%
\pgfpathlineto{\pgfqpoint{6.098285in}{1.700529in}}%
\pgfpathlineto{\pgfqpoint{6.099359in}{1.700898in}}%
\pgfpathlineto{\pgfqpoint{6.100433in}{1.700013in}}%
\pgfpathlineto{\pgfqpoint{6.103654in}{1.699349in}}%
\pgfpathlineto{\pgfqpoint{6.105802in}{1.692709in}}%
\pgfpathlineto{\pgfqpoint{6.106876in}{1.692635in}}%
\pgfpathlineto{\pgfqpoint{6.107950in}{1.690422in}}%
\pgfpathlineto{\pgfqpoint{6.111171in}{1.692193in}}%
\pgfpathlineto{\pgfqpoint{6.112245in}{1.690275in}}%
\pgfpathlineto{\pgfqpoint{6.113319in}{1.692930in}}%
\pgfpathlineto{\pgfqpoint{6.115466in}{1.692783in}}%
\pgfpathlineto{\pgfqpoint{6.118688in}{1.693742in}}%
\pgfpathlineto{\pgfqpoint{6.119762in}{1.692709in}}%
\pgfpathlineto{\pgfqpoint{6.120836in}{1.693447in}}%
\pgfpathlineto{\pgfqpoint{6.122983in}{1.679208in}}%
\pgfpathlineto{\pgfqpoint{6.126205in}{1.679282in}}%
\pgfpathlineto{\pgfqpoint{6.128353in}{1.675519in}}%
\pgfpathlineto{\pgfqpoint{6.129426in}{1.681421in}}%
\pgfpathlineto{\pgfqpoint{6.130500in}{1.679282in}}%
\pgfpathlineto{\pgfqpoint{6.133722in}{1.678544in}}%
\pgfpathlineto{\pgfqpoint{6.134796in}{1.675741in}}%
\pgfpathlineto{\pgfqpoint{6.135869in}{1.671019in}}%
\pgfpathlineto{\pgfqpoint{6.136943in}{1.670576in}}%
\pgfpathlineto{\pgfqpoint{6.138017in}{1.671904in}}%
\pgfpathlineto{\pgfqpoint{6.142312in}{1.675667in}}%
\pgfpathlineto{\pgfqpoint{6.143386in}{1.676995in}}%
\pgfpathlineto{\pgfqpoint{6.144460in}{1.669396in}}%
\pgfpathlineto{\pgfqpoint{6.145534in}{1.669396in}}%
\pgfpathlineto{\pgfqpoint{6.148755in}{1.666076in}}%
\pgfpathlineto{\pgfqpoint{6.149829in}{1.673823in}}%
\pgfpathlineto{\pgfqpoint{6.150903in}{1.677438in}}%
\pgfpathlineto{\pgfqpoint{6.151977in}{1.676847in}}%
\pgfpathlineto{\pgfqpoint{6.153051in}{1.678397in}}%
\pgfpathlineto{\pgfqpoint{6.156272in}{1.679946in}}%
\pgfpathlineto{\pgfqpoint{6.158420in}{1.692414in}}%
\pgfpathlineto{\pgfqpoint{6.160568in}{1.695291in}}%
\pgfpathlineto{\pgfqpoint{6.163789in}{1.698242in}}%
\pgfpathlineto{\pgfqpoint{6.164863in}{1.697209in}}%
\pgfpathlineto{\pgfqpoint{6.165937in}{1.690275in}}%
\pgfpathlineto{\pgfqpoint{6.171306in}{1.689315in}}%
\pgfpathlineto{\pgfqpoint{6.172380in}{1.693299in}}%
\pgfpathlineto{\pgfqpoint{6.173454in}{1.699939in}}%
\pgfpathlineto{\pgfqpoint{6.174528in}{1.698316in}}%
\pgfpathlineto{\pgfqpoint{6.175601in}{1.700677in}}%
\pgfpathlineto{\pgfqpoint{6.178823in}{1.702374in}}%
\pgfpathlineto{\pgfqpoint{6.179897in}{1.706284in}}%
\pgfpathlineto{\pgfqpoint{6.180971in}{1.701562in}}%
\pgfpathlineto{\pgfqpoint{6.182044in}{1.702743in}}%
\pgfpathlineto{\pgfqpoint{6.183118in}{1.705694in}}%
\pgfpathlineto{\pgfqpoint{6.187414in}{1.711227in}}%
\pgfpathlineto{\pgfqpoint{6.188487in}{1.710268in}}%
\pgfpathlineto{\pgfqpoint{6.189561in}{1.714694in}}%
\pgfpathlineto{\pgfqpoint{6.190635in}{1.714916in}}%
\pgfpathlineto{\pgfqpoint{6.194930in}{1.714620in}}%
\pgfpathlineto{\pgfqpoint{6.196004in}{1.713366in}}%
\pgfpathlineto{\pgfqpoint{6.197078in}{1.715284in}}%
\pgfpathlineto{\pgfqpoint{6.198152in}{1.712924in}}%
\pgfpathlineto{\pgfqpoint{6.202447in}{1.720744in}}%
\pgfpathlineto{\pgfqpoint{6.203521in}{1.720375in}}%
\pgfpathlineto{\pgfqpoint{6.204595in}{1.721186in}}%
\pgfpathlineto{\pgfqpoint{6.205669in}{1.715875in}}%
\pgfpathlineto{\pgfqpoint{6.208890in}{1.712038in}}%
\pgfpathlineto{\pgfqpoint{6.209964in}{1.712555in}}%
\pgfpathlineto{\pgfqpoint{6.211038in}{1.710637in}}%
\pgfpathlineto{\pgfqpoint{6.212112in}{1.712038in}}%
\pgfpathlineto{\pgfqpoint{6.213186in}{1.711227in}}%
\pgfpathlineto{\pgfqpoint{6.217481in}{1.712481in}}%
\pgfpathlineto{\pgfqpoint{6.218555in}{1.709235in}}%
\pgfpathlineto{\pgfqpoint{6.219629in}{1.709973in}}%
\pgfpathlineto{\pgfqpoint{6.220703in}{1.712333in}}%
\pgfpathlineto{\pgfqpoint{6.223924in}{1.710268in}}%
\pgfpathlineto{\pgfqpoint{6.224998in}{1.695217in}}%
\pgfpathlineto{\pgfqpoint{6.226072in}{1.692930in}}%
\pgfpathlineto{\pgfqpoint{6.227146in}{1.688651in}}%
\pgfpathlineto{\pgfqpoint{6.228219in}{1.691750in}}%
\pgfpathlineto{\pgfqpoint{6.231441in}{1.690275in}}%
\pgfpathlineto{\pgfqpoint{6.232515in}{1.687914in}}%
\pgfpathlineto{\pgfqpoint{6.233589in}{1.683856in}}%
\pgfpathlineto{\pgfqpoint{6.234663in}{1.683118in}}%
\pgfpathlineto{\pgfqpoint{6.235736in}{1.685110in}}%
\pgfpathlineto{\pgfqpoint{6.238958in}{1.681421in}}%
\pgfpathlineto{\pgfqpoint{6.240032in}{1.681495in}}%
\pgfpathlineto{\pgfqpoint{6.243253in}{1.687692in}}%
\pgfpathlineto{\pgfqpoint{6.246475in}{1.684889in}}%
\pgfpathlineto{\pgfqpoint{6.248622in}{1.681790in}}%
\pgfpathlineto{\pgfqpoint{6.249696in}{1.684299in}}%
\pgfpathlineto{\pgfqpoint{6.250770in}{1.689094in}}%
\pgfpathlineto{\pgfqpoint{6.255065in}{1.690643in}}%
\pgfpathlineto{\pgfqpoint{6.256139in}{1.692266in}}%
\pgfpathlineto{\pgfqpoint{6.257213in}{1.696472in}}%
\pgfpathlineto{\pgfqpoint{6.258287in}{1.698390in}}%
\pgfpathlineto{\pgfqpoint{6.262582in}{1.691971in}}%
\pgfpathlineto{\pgfqpoint{6.265804in}{1.694996in}}%
\pgfpathlineto{\pgfqpoint{6.269025in}{1.694627in}}%
\pgfpathlineto{\pgfqpoint{6.271173in}{1.688873in}}%
\pgfpathlineto{\pgfqpoint{6.273321in}{1.690127in}}%
\pgfpathlineto{\pgfqpoint{6.276542in}{1.690939in}}%
\pgfpathlineto{\pgfqpoint{6.277616in}{1.690201in}}%
\pgfpathlineto{\pgfqpoint{6.278690in}{1.695365in}}%
\pgfpathlineto{\pgfqpoint{6.279764in}{1.694701in}}%
\pgfpathlineto{\pgfqpoint{6.280838in}{1.696988in}}%
\pgfpathlineto{\pgfqpoint{6.285133in}{1.695439in}}%
\pgfpathlineto{\pgfqpoint{6.286207in}{1.692488in}}%
\pgfpathlineto{\pgfqpoint{6.287281in}{1.692045in}}%
\pgfpathlineto{\pgfqpoint{6.288354in}{1.692340in}}%
\pgfpathlineto{\pgfqpoint{6.291576in}{1.688725in}}%
\pgfpathlineto{\pgfqpoint{6.292650in}{1.689832in}}%
\pgfpathlineto{\pgfqpoint{6.295871in}{1.686217in}}%
\pgfpathlineto{\pgfqpoint{6.300167in}{1.689906in}}%
\pgfpathlineto{\pgfqpoint{6.301241in}{1.687987in}}%
\pgfpathlineto{\pgfqpoint{6.302314in}{1.687914in}}%
\pgfpathlineto{\pgfqpoint{6.303388in}{1.689389in}}%
\pgfpathlineto{\pgfqpoint{6.306610in}{1.688651in}}%
\pgfpathlineto{\pgfqpoint{6.308757in}{1.691160in}}%
\pgfpathlineto{\pgfqpoint{6.309831in}{1.689168in}}%
\pgfpathlineto{\pgfqpoint{6.314127in}{1.690422in}}%
\pgfpathlineto{\pgfqpoint{6.315200in}{1.693152in}}%
\pgfpathlineto{\pgfqpoint{6.316274in}{1.691307in}}%
\pgfpathlineto{\pgfqpoint{6.317348in}{1.687766in}}%
\pgfpathlineto{\pgfqpoint{6.318422in}{1.680020in}}%
\pgfpathlineto{\pgfqpoint{6.321643in}{1.678692in}}%
\pgfpathlineto{\pgfqpoint{6.322717in}{1.676331in}}%
\pgfpathlineto{\pgfqpoint{6.323791in}{1.680757in}}%
\pgfpathlineto{\pgfqpoint{6.325939in}{1.671093in}}%
\pgfpathlineto{\pgfqpoint{6.330234in}{1.671093in}}%
\pgfpathlineto{\pgfqpoint{6.331308in}{1.672716in}}%
\pgfpathlineto{\pgfqpoint{6.332382in}{1.670872in}}%
\pgfpathlineto{\pgfqpoint{6.333456in}{1.676331in}}%
\pgfpathlineto{\pgfqpoint{6.336677in}{1.675888in}}%
\pgfpathlineto{\pgfqpoint{6.337751in}{1.674487in}}%
\pgfpathlineto{\pgfqpoint{6.338825in}{1.674265in}}%
\pgfpathlineto{\pgfqpoint{6.340973in}{1.670650in}}%
\pgfpathlineto{\pgfqpoint{6.345268in}{1.667109in}}%
\pgfpathlineto{\pgfqpoint{6.346342in}{1.661576in}}%
\pgfpathlineto{\pgfqpoint{6.348489in}{1.667847in}}%
\pgfpathlineto{\pgfqpoint{6.352785in}{1.668216in}}%
\pgfpathlineto{\pgfqpoint{6.353859in}{1.665338in}}%
\pgfpathlineto{\pgfqpoint{6.354932in}{1.667109in}}%
\pgfpathlineto{\pgfqpoint{6.356006in}{1.667183in}}%
\pgfpathlineto{\pgfqpoint{6.360302in}{1.673011in}}%
\pgfpathlineto{\pgfqpoint{6.361375in}{1.675962in}}%
\pgfpathlineto{\pgfqpoint{6.363523in}{1.674634in}}%
\pgfpathlineto{\pgfqpoint{6.366745in}{1.674191in}}%
\pgfpathlineto{\pgfqpoint{6.368892in}{1.675003in}}%
\pgfpathlineto{\pgfqpoint{6.369966in}{1.672937in}}%
\pgfpathlineto{\pgfqpoint{6.371040in}{1.676479in}}%
\pgfpathlineto{\pgfqpoint{6.374262in}{1.679651in}}%
\pgfpathlineto{\pgfqpoint{6.375335in}{1.674782in}}%
\pgfpathlineto{\pgfqpoint{6.376409in}{1.676331in}}%
\pgfpathlineto{\pgfqpoint{6.378557in}{1.675888in}}%
\pgfpathlineto{\pgfqpoint{6.381778in}{1.675519in}}%
\pgfpathlineto{\pgfqpoint{6.383926in}{1.667773in}}%
\pgfpathlineto{\pgfqpoint{6.386074in}{1.667625in}}%
\pgfpathlineto{\pgfqpoint{6.389295in}{1.670060in}}%
\pgfpathlineto{\pgfqpoint{6.390369in}{1.664010in}}%
\pgfpathlineto{\pgfqpoint{6.391443in}{1.664010in}}%
\pgfpathlineto{\pgfqpoint{6.392517in}{1.661133in}}%
\pgfpathlineto{\pgfqpoint{6.393591in}{1.662756in}}%
\pgfpathlineto{\pgfqpoint{6.396812in}{1.665043in}}%
\pgfpathlineto{\pgfqpoint{6.398960in}{1.662682in}}%
\pgfpathlineto{\pgfqpoint{6.400034in}{1.658994in}}%
\pgfpathlineto{\pgfqpoint{6.401107in}{1.658699in}}%
\pgfpathlineto{\pgfqpoint{6.404329in}{1.656854in}}%
\pgfpathlineto{\pgfqpoint{6.405403in}{1.654715in}}%
\pgfpathlineto{\pgfqpoint{6.407551in}{1.658772in}}%
\pgfpathlineto{\pgfqpoint{6.408624in}{1.659289in}}%
\pgfpathlineto{\pgfqpoint{6.411846in}{1.659953in}}%
\pgfpathlineto{\pgfqpoint{6.412920in}{1.657887in}}%
\pgfpathlineto{\pgfqpoint{6.413994in}{1.658477in}}%
\pgfpathlineto{\pgfqpoint{6.415067in}{1.663715in}}%
\pgfpathlineto{\pgfqpoint{6.416141in}{1.663789in}}%
\pgfpathlineto{\pgfqpoint{6.419363in}{1.660248in}}%
\pgfpathlineto{\pgfqpoint{6.421510in}{1.664896in}}%
\pgfpathlineto{\pgfqpoint{6.422584in}{1.687914in}}%
\pgfpathlineto{\pgfqpoint{6.426880in}{1.691898in}}%
\pgfpathlineto{\pgfqpoint{6.427953in}{1.695217in}}%
\pgfpathlineto{\pgfqpoint{6.429027in}{1.690643in}}%
\pgfpathlineto{\pgfqpoint{6.431175in}{1.695365in}}%
\pgfpathlineto{\pgfqpoint{6.434396in}{1.694996in}}%
\pgfpathlineto{\pgfqpoint{6.437618in}{1.689315in}}%
\pgfpathlineto{\pgfqpoint{6.438692in}{1.689611in}}%
\pgfpathlineto{\pgfqpoint{6.441913in}{1.694480in}}%
\pgfpathlineto{\pgfqpoint{6.442987in}{1.692414in}}%
\pgfpathlineto{\pgfqpoint{6.444061in}{1.691971in}}%
\pgfpathlineto{\pgfqpoint{6.445135in}{1.688504in}}%
\pgfpathlineto{\pgfqpoint{6.446209in}{1.687102in}}%
\pgfpathlineto{\pgfqpoint{6.450504in}{1.691602in}}%
\pgfpathlineto{\pgfqpoint{6.451578in}{1.691086in}}%
\pgfpathlineto{\pgfqpoint{6.452652in}{1.691307in}}%
\pgfpathlineto{\pgfqpoint{6.453726in}{1.693816in}}%
\pgfpathlineto{\pgfqpoint{6.458021in}{1.692635in}}%
\pgfpathlineto{\pgfqpoint{6.459095in}{1.690053in}}%
\pgfpathlineto{\pgfqpoint{6.461242in}{1.688651in}}%
\pgfpathlineto{\pgfqpoint{6.465538in}{1.684889in}}%
\pgfpathlineto{\pgfqpoint{6.466612in}{1.681864in}}%
\pgfpathlineto{\pgfqpoint{6.467685in}{1.677069in}}%
\pgfpathlineto{\pgfqpoint{6.468759in}{1.676479in}}%
\pgfpathlineto{\pgfqpoint{6.471981in}{1.677733in}}%
\pgfpathlineto{\pgfqpoint{6.474129in}{1.684151in}}%
\pgfpathlineto{\pgfqpoint{6.475202in}{1.683709in}}%
\pgfpathlineto{\pgfqpoint{6.476276in}{1.688283in}}%
\pgfpathlineto{\pgfqpoint{6.479498in}{1.689832in}}%
\pgfpathlineto{\pgfqpoint{6.480572in}{1.698242in}}%
\pgfpathlineto{\pgfqpoint{6.481645in}{1.699201in}}%
\pgfpathlineto{\pgfqpoint{6.482719in}{1.695586in}}%
\pgfpathlineto{\pgfqpoint{6.483793in}{1.702005in}}%
\pgfpathlineto{\pgfqpoint{6.487015in}{1.702005in}}%
\pgfpathlineto{\pgfqpoint{6.488088in}{1.699423in}}%
\pgfpathlineto{\pgfqpoint{6.489162in}{1.699423in}}%
\pgfpathlineto{\pgfqpoint{6.490236in}{1.698759in}}%
\pgfpathlineto{\pgfqpoint{6.491310in}{1.699275in}}%
\pgfpathlineto{\pgfqpoint{6.494531in}{1.698390in}}%
\pgfpathlineto{\pgfqpoint{6.495605in}{1.701710in}}%
\pgfpathlineto{\pgfqpoint{6.496679in}{1.702005in}}%
\pgfpathlineto{\pgfqpoint{6.497753in}{1.701120in}}%
\pgfpathlineto{\pgfqpoint{6.498827in}{1.698611in}}%
\pgfpathlineto{\pgfqpoint{6.503122in}{1.700898in}}%
\pgfpathlineto{\pgfqpoint{6.504196in}{1.698980in}}%
\pgfpathlineto{\pgfqpoint{6.506344in}{1.692119in}}%
\pgfpathlineto{\pgfqpoint{6.509565in}{1.693742in}}%
\pgfpathlineto{\pgfqpoint{6.511713in}{1.697505in}}%
\pgfpathlineto{\pgfqpoint{6.513861in}{1.703554in}}%
\pgfpathlineto{\pgfqpoint{6.517082in}{1.699865in}}%
\pgfpathlineto{\pgfqpoint{6.518156in}{1.699496in}}%
\pgfpathlineto{\pgfqpoint{6.519230in}{1.697431in}}%
\pgfpathlineto{\pgfqpoint{6.520304in}{1.699201in}}%
\pgfpathlineto{\pgfqpoint{6.521377in}{1.699054in}}%
\pgfpathlineto{\pgfqpoint{6.524599in}{1.691971in}}%
\pgfpathlineto{\pgfqpoint{6.526747in}{1.691971in}}%
\pgfpathlineto{\pgfqpoint{6.527820in}{1.689389in}}%
\pgfpathlineto{\pgfqpoint{6.528894in}{1.689168in}}%
\pgfpathlineto{\pgfqpoint{6.532116in}{1.676257in}}%
\pgfpathlineto{\pgfqpoint{6.533190in}{1.676479in}}%
\pgfpathlineto{\pgfqpoint{6.534263in}{1.675741in}}%
\pgfpathlineto{\pgfqpoint{6.536411in}{1.671831in}}%
\pgfpathlineto{\pgfqpoint{6.539633in}{1.670872in}}%
\pgfpathlineto{\pgfqpoint{6.540706in}{1.667257in}}%
\pgfpathlineto{\pgfqpoint{6.541780in}{1.666519in}}%
\pgfpathlineto{\pgfqpoint{6.543928in}{1.675446in}}%
\pgfpathlineto{\pgfqpoint{6.547150in}{1.680831in}}%
\pgfpathlineto{\pgfqpoint{6.548223in}{1.680684in}}%
\pgfpathlineto{\pgfqpoint{6.549297in}{1.686955in}}%
\pgfpathlineto{\pgfqpoint{6.551445in}{1.686364in}}%
\pgfpathlineto{\pgfqpoint{6.554666in}{1.690791in}}%
\pgfpathlineto{\pgfqpoint{6.557888in}{1.712850in}}%
\pgfpathlineto{\pgfqpoint{6.558962in}{1.715284in}}%
\pgfpathlineto{\pgfqpoint{6.562183in}{1.718531in}}%
\pgfpathlineto{\pgfqpoint{6.563257in}{1.712997in}}%
\pgfpathlineto{\pgfqpoint{6.565405in}{1.709604in}}%
\pgfpathlineto{\pgfqpoint{6.566479in}{1.714178in}}%
\pgfpathlineto{\pgfqpoint{6.569700in}{1.719342in}}%
\pgfpathlineto{\pgfqpoint{6.570774in}{1.728564in}}%
\pgfpathlineto{\pgfqpoint{6.571848in}{1.726498in}}%
\pgfpathlineto{\pgfqpoint{6.572922in}{1.722736in}}%
\pgfpathlineto{\pgfqpoint{6.573995in}{1.724949in}}%
\pgfpathlineto{\pgfqpoint{6.577217in}{1.728933in}}%
\pgfpathlineto{\pgfqpoint{6.578291in}{1.726056in}}%
\pgfpathlineto{\pgfqpoint{6.579365in}{1.725687in}}%
\pgfpathlineto{\pgfqpoint{6.581512in}{1.728564in}}%
\pgfpathlineto{\pgfqpoint{6.586882in}{1.729154in}}%
\pgfpathlineto{\pgfqpoint{6.587955in}{1.730187in}}%
\pgfpathlineto{\pgfqpoint{6.589029in}{1.726793in}}%
\pgfpathlineto{\pgfqpoint{6.589029in}{1.726793in}}%
\pgfusepath{stroke}%
\end{pgfscope}%
\begin{pgfscope}%
\pgfpathrectangle{\pgfqpoint{4.123120in}{1.347524in}}{\pgfqpoint{2.583333in}{0.400885in}}%
\pgfusepath{clip}%
\pgfsetroundcap%
\pgfsetroundjoin%
\pgfsetlinewidth{1.505625pt}%
\definecolor{currentstroke}{rgb}{0.498039,0.498039,0.498039}%
\pgfsetstrokecolor{currentstroke}%
\pgfsetdash{}{0pt}%
\pgfpathmoveto{\pgfqpoint{4.240544in}{1.571496in}}%
\pgfpathlineto{\pgfqpoint{4.242692in}{1.567438in}}%
\pgfpathlineto{\pgfqpoint{4.243766in}{1.566848in}}%
\pgfpathlineto{\pgfqpoint{4.246987in}{1.566627in}}%
\pgfpathlineto{\pgfqpoint{4.250209in}{1.563919in}}%
\pgfpathlineto{\pgfqpoint{4.258800in}{1.564205in}}%
\pgfpathlineto{\pgfqpoint{4.262021in}{1.561270in}}%
\pgfpathlineto{\pgfqpoint{4.263095in}{1.558191in}}%
\pgfpathlineto{\pgfqpoint{4.264169in}{1.557618in}}%
\pgfpathlineto{\pgfqpoint{4.265243in}{1.555828in}}%
\pgfpathlineto{\pgfqpoint{4.269538in}{1.553107in}}%
\pgfpathlineto{\pgfqpoint{4.271686in}{1.554018in}}%
\pgfpathlineto{\pgfqpoint{4.273833in}{1.551426in}}%
\pgfpathlineto{\pgfqpoint{4.281350in}{1.547802in}}%
\pgfpathlineto{\pgfqpoint{4.287793in}{1.543107in}}%
\pgfpathlineto{\pgfqpoint{4.288867in}{1.541206in}}%
\pgfpathlineto{\pgfqpoint{4.293162in}{1.541334in}}%
\pgfpathlineto{\pgfqpoint{4.294236in}{1.540051in}}%
\pgfpathlineto{\pgfqpoint{4.296384in}{1.539730in}}%
\pgfpathlineto{\pgfqpoint{4.301753in}{1.539601in}}%
\pgfpathlineto{\pgfqpoint{4.303901in}{1.537053in}}%
\pgfpathlineto{\pgfqpoint{4.307122in}{1.535557in}}%
\pgfpathlineto{\pgfqpoint{4.308196in}{1.534210in}}%
\pgfpathlineto{\pgfqpoint{4.309270in}{1.535006in}}%
\pgfpathlineto{\pgfqpoint{4.310344in}{1.533536in}}%
\pgfpathlineto{\pgfqpoint{4.311418in}{1.533055in}}%
\pgfpathlineto{\pgfqpoint{4.318935in}{1.535041in}}%
\pgfpathlineto{\pgfqpoint{4.325378in}{1.535462in}}%
\pgfpathlineto{\pgfqpoint{4.326451in}{1.534439in}}%
\pgfpathlineto{\pgfqpoint{4.329673in}{1.534018in}}%
\pgfpathlineto{\pgfqpoint{4.331821in}{1.529443in}}%
\pgfpathlineto{\pgfqpoint{4.332894in}{1.528661in}}%
\pgfpathlineto{\pgfqpoint{4.333968in}{1.529323in}}%
\pgfpathlineto{\pgfqpoint{4.344707in}{1.525629in}}%
\pgfpathlineto{\pgfqpoint{4.346854in}{1.520413in}}%
\pgfpathlineto{\pgfqpoint{4.347928in}{1.521160in}}%
\pgfpathlineto{\pgfqpoint{4.349002in}{1.519953in}}%
\pgfpathlineto{\pgfqpoint{4.352224in}{1.520643in}}%
\pgfpathlineto{\pgfqpoint{4.353297in}{1.519379in}}%
\pgfpathlineto{\pgfqpoint{4.354371in}{1.519096in}}%
\pgfpathlineto{\pgfqpoint{4.356519in}{1.514803in}}%
\pgfpathlineto{\pgfqpoint{4.359740in}{1.514162in}}%
\pgfpathlineto{\pgfqpoint{4.360814in}{1.510527in}}%
\pgfpathlineto{\pgfqpoint{4.361888in}{1.510425in}}%
\pgfpathlineto{\pgfqpoint{4.362962in}{1.507982in}}%
\pgfpathlineto{\pgfqpoint{4.370479in}{1.509754in}}%
\pgfpathlineto{\pgfqpoint{4.371553in}{1.508376in}}%
\pgfpathlineto{\pgfqpoint{4.375848in}{1.509410in}}%
\pgfpathlineto{\pgfqpoint{4.376922in}{1.508376in}}%
\pgfpathlineto{\pgfqpoint{4.377996in}{1.509410in}}%
\pgfpathlineto{\pgfqpoint{4.379070in}{1.507244in}}%
\pgfpathlineto{\pgfqpoint{4.382291in}{1.506337in}}%
\pgfpathlineto{\pgfqpoint{4.383365in}{1.506862in}}%
\pgfpathlineto{\pgfqpoint{4.384439in}{1.506289in}}%
\pgfpathlineto{\pgfqpoint{4.385513in}{1.504570in}}%
\pgfpathlineto{\pgfqpoint{4.386586in}{1.505083in}}%
\pgfpathlineto{\pgfqpoint{4.391956in}{1.504244in}}%
\pgfpathlineto{\pgfqpoint{4.394103in}{1.504803in}}%
\pgfpathlineto{\pgfqpoint{4.398399in}{1.505828in}}%
\pgfpathlineto{\pgfqpoint{4.399472in}{1.504710in}}%
\pgfpathlineto{\pgfqpoint{4.400546in}{1.505455in}}%
\pgfpathlineto{\pgfqpoint{4.401620in}{1.503405in}}%
\pgfpathlineto{\pgfqpoint{4.404842in}{1.504430in}}%
\pgfpathlineto{\pgfqpoint{4.405916in}{1.503964in}}%
\pgfpathlineto{\pgfqpoint{4.406989in}{1.502053in}}%
\pgfpathlineto{\pgfqpoint{4.408063in}{1.501645in}}%
\pgfpathlineto{\pgfqpoint{4.409137in}{1.499016in}}%
\pgfpathlineto{\pgfqpoint{4.412359in}{1.499365in}}%
\pgfpathlineto{\pgfqpoint{4.413432in}{1.498144in}}%
\pgfpathlineto{\pgfqpoint{4.414506in}{1.498272in}}%
\pgfpathlineto{\pgfqpoint{4.415580in}{1.495916in}}%
\pgfpathlineto{\pgfqpoint{4.416654in}{1.495338in}}%
\pgfpathlineto{\pgfqpoint{4.420949in}{1.495875in}}%
\pgfpathlineto{\pgfqpoint{4.422023in}{1.494593in}}%
\pgfpathlineto{\pgfqpoint{4.423097in}{1.494676in}}%
\pgfpathlineto{\pgfqpoint{4.424171in}{1.492816in}}%
\pgfpathlineto{\pgfqpoint{4.427392in}{1.491932in}}%
\pgfpathlineto{\pgfqpoint{4.428466in}{1.492495in}}%
\pgfpathlineto{\pgfqpoint{4.429540in}{1.492334in}}%
\pgfpathlineto{\pgfqpoint{4.430614in}{1.492856in}}%
\pgfpathlineto{\pgfqpoint{4.431688in}{1.491490in}}%
\pgfpathlineto{\pgfqpoint{4.435983in}{1.490113in}}%
\pgfpathlineto{\pgfqpoint{4.439205in}{1.490037in}}%
\pgfpathlineto{\pgfqpoint{4.444574in}{1.491382in}}%
\pgfpathlineto{\pgfqpoint{4.445648in}{1.490729in}}%
\pgfpathlineto{\pgfqpoint{4.446721in}{1.489191in}}%
\pgfpathlineto{\pgfqpoint{4.449943in}{1.489341in}}%
\pgfpathlineto{\pgfqpoint{4.452091in}{1.491029in}}%
\pgfpathlineto{\pgfqpoint{4.453164in}{1.487353in}}%
\pgfpathlineto{\pgfqpoint{4.457460in}{1.486828in}}%
\pgfpathlineto{\pgfqpoint{4.458534in}{1.485215in}}%
\pgfpathlineto{\pgfqpoint{4.459607in}{1.485102in}}%
\pgfpathlineto{\pgfqpoint{4.461755in}{1.481916in}}%
\pgfpathlineto{\pgfqpoint{4.466050in}{1.482521in}}%
\pgfpathlineto{\pgfqpoint{4.467124in}{1.482734in}}%
\pgfpathlineto{\pgfqpoint{4.468198in}{1.481170in}}%
\pgfpathlineto{\pgfqpoint{4.469272in}{1.480778in}}%
\pgfpathlineto{\pgfqpoint{4.472494in}{1.481383in}}%
\pgfpathlineto{\pgfqpoint{4.474641in}{1.480032in}}%
\pgfpathlineto{\pgfqpoint{4.476789in}{1.481134in}}%
\pgfpathlineto{\pgfqpoint{4.483232in}{1.479890in}}%
\pgfpathlineto{\pgfqpoint{4.487527in}{1.478787in}}%
\pgfpathlineto{\pgfqpoint{4.488601in}{1.476690in}}%
\pgfpathlineto{\pgfqpoint{4.490749in}{1.475019in}}%
\pgfpathlineto{\pgfqpoint{4.491823in}{1.472601in}}%
\pgfpathlineto{\pgfqpoint{4.496118in}{1.471342in}}%
\pgfpathlineto{\pgfqpoint{4.498266in}{1.469487in}}%
\pgfpathlineto{\pgfqpoint{4.499339in}{1.469921in}}%
\pgfpathlineto{\pgfqpoint{4.503635in}{1.468086in}}%
\pgfpathlineto{\pgfqpoint{4.505782in}{1.469149in}}%
\pgfpathlineto{\pgfqpoint{4.506856in}{1.468151in}}%
\pgfpathlineto{\pgfqpoint{4.511152in}{1.469374in}}%
\pgfpathlineto{\pgfqpoint{4.514373in}{1.463931in}}%
\pgfpathlineto{\pgfqpoint{4.517595in}{1.464053in}}%
\pgfpathlineto{\pgfqpoint{4.521890in}{1.466387in}}%
\pgfpathlineto{\pgfqpoint{4.533702in}{1.466872in}}%
\pgfpathlineto{\pgfqpoint{4.534776in}{1.465932in}}%
\pgfpathlineto{\pgfqpoint{4.535850in}{1.463819in}}%
\pgfpathlineto{\pgfqpoint{4.536924in}{1.464475in}}%
\pgfpathlineto{\pgfqpoint{4.540145in}{1.463448in}}%
\pgfpathlineto{\pgfqpoint{4.544441in}{1.459339in}}%
\pgfpathlineto{\pgfqpoint{4.547662in}{1.459082in}}%
\pgfpathlineto{\pgfqpoint{4.549810in}{1.456828in}}%
\pgfpathlineto{\pgfqpoint{4.551958in}{1.453451in}}%
\pgfpathlineto{\pgfqpoint{4.555179in}{1.452684in}}%
\pgfpathlineto{\pgfqpoint{4.556253in}{1.451337in}}%
\pgfpathlineto{\pgfqpoint{4.558401in}{1.451680in}}%
\pgfpathlineto{\pgfqpoint{4.559474in}{1.450755in}}%
\pgfpathlineto{\pgfqpoint{4.564844in}{1.450600in}}%
\pgfpathlineto{\pgfqpoint{4.566991in}{1.448479in}}%
\pgfpathlineto{\pgfqpoint{4.571287in}{1.448075in}}%
\pgfpathlineto{\pgfqpoint{4.573434in}{1.444913in}}%
\pgfpathlineto{\pgfqpoint{4.578804in}{1.444787in}}%
\pgfpathlineto{\pgfqpoint{4.582025in}{1.442662in}}%
\pgfpathlineto{\pgfqpoint{4.585247in}{1.440032in}}%
\pgfpathlineto{\pgfqpoint{4.586320in}{1.440032in}}%
\pgfpathlineto{\pgfqpoint{4.587394in}{1.440646in}}%
\pgfpathlineto{\pgfqpoint{4.589542in}{1.439606in}}%
\pgfpathlineto{\pgfqpoint{4.593837in}{1.438273in}}%
\pgfpathlineto{\pgfqpoint{4.595985in}{1.436547in}}%
\pgfpathlineto{\pgfqpoint{4.600280in}{1.436700in}}%
\pgfpathlineto{\pgfqpoint{4.602428in}{1.435298in}}%
\pgfpathlineto{\pgfqpoint{4.604576in}{1.435792in}}%
\pgfpathlineto{\pgfqpoint{4.607797in}{1.435190in}}%
\pgfpathlineto{\pgfqpoint{4.609945in}{1.436393in}}%
\pgfpathlineto{\pgfqpoint{4.612093in}{1.435469in}}%
\pgfpathlineto{\pgfqpoint{4.616388in}{1.434932in}}%
\pgfpathlineto{\pgfqpoint{4.619609in}{1.433092in}}%
\pgfpathlineto{\pgfqpoint{4.626052in}{1.432945in}}%
\pgfpathlineto{\pgfqpoint{4.627126in}{1.432039in}}%
\pgfpathlineto{\pgfqpoint{4.630348in}{1.432629in}}%
\pgfpathlineto{\pgfqpoint{4.633569in}{1.430790in}}%
\pgfpathlineto{\pgfqpoint{4.637865in}{1.431715in}}%
\pgfpathlineto{\pgfqpoint{4.638938in}{1.430106in}}%
\pgfpathlineto{\pgfqpoint{4.640012in}{1.429965in}}%
\pgfpathlineto{\pgfqpoint{4.642160in}{1.428649in}}%
\pgfpathlineto{\pgfqpoint{4.646455in}{1.426712in}}%
\pgfpathlineto{\pgfqpoint{4.648603in}{1.425106in}}%
\pgfpathlineto{\pgfqpoint{4.649677in}{1.425676in}}%
\pgfpathlineto{\pgfqpoint{4.655046in}{1.426037in}}%
\pgfpathlineto{\pgfqpoint{4.657194in}{1.425866in}}%
\pgfpathlineto{\pgfqpoint{4.660415in}{1.426018in}}%
\pgfpathlineto{\pgfqpoint{4.661489in}{1.424993in}}%
\pgfpathlineto{\pgfqpoint{4.663637in}{1.425139in}}%
\pgfpathlineto{\pgfqpoint{4.664711in}{1.423836in}}%
\pgfpathlineto{\pgfqpoint{4.684040in}{1.423766in}}%
\pgfpathlineto{\pgfqpoint{4.687261in}{1.424943in}}%
\pgfpathlineto{\pgfqpoint{4.692630in}{1.426191in}}%
\pgfpathlineto{\pgfqpoint{4.698000in}{1.427193in}}%
\pgfpathlineto{\pgfqpoint{4.700147in}{1.425905in}}%
\pgfpathlineto{\pgfqpoint{4.701221in}{1.426163in}}%
\pgfpathlineto{\pgfqpoint{4.702295in}{1.425545in}}%
\pgfpathlineto{\pgfqpoint{4.705516in}{1.425360in}}%
\pgfpathlineto{\pgfqpoint{4.709812in}{1.422959in}}%
\pgfpathlineto{\pgfqpoint{4.715181in}{1.421921in}}%
\pgfpathlineto{\pgfqpoint{4.721624in}{1.422949in}}%
\pgfpathlineto{\pgfqpoint{4.722698in}{1.422310in}}%
\pgfpathlineto{\pgfqpoint{4.723772in}{1.422560in}}%
\pgfpathlineto{\pgfqpoint{4.732362in}{1.423043in}}%
\pgfpathlineto{\pgfqpoint{4.736658in}{1.423401in}}%
\pgfpathlineto{\pgfqpoint{4.738805in}{1.422061in}}%
\pgfpathlineto{\pgfqpoint{4.739879in}{1.422462in}}%
\pgfpathlineto{\pgfqpoint{4.744175in}{1.422001in}}%
\pgfpathlineto{\pgfqpoint{4.746322in}{1.419442in}}%
\pgfpathlineto{\pgfqpoint{4.747396in}{1.418021in}}%
\pgfpathlineto{\pgfqpoint{4.752765in}{1.417566in}}%
\pgfpathlineto{\pgfqpoint{4.753839in}{1.416124in}}%
\pgfpathlineto{\pgfqpoint{4.754913in}{1.416516in}}%
\pgfpathlineto{\pgfqpoint{4.759208in}{1.416781in}}%
\pgfpathlineto{\pgfqpoint{4.760282in}{1.415340in}}%
\pgfpathlineto{\pgfqpoint{4.762430in}{1.415606in}}%
\pgfpathlineto{\pgfqpoint{4.765651in}{1.414999in}}%
\pgfpathlineto{\pgfqpoint{4.766725in}{1.414165in}}%
\pgfpathlineto{\pgfqpoint{4.769947in}{1.414140in}}%
\pgfpathlineto{\pgfqpoint{4.780685in}{1.412854in}}%
\pgfpathlineto{\pgfqpoint{4.782833in}{1.411721in}}%
\pgfpathlineto{\pgfqpoint{4.783907in}{1.412091in}}%
\pgfpathlineto{\pgfqpoint{4.784981in}{1.411649in}}%
\pgfpathlineto{\pgfqpoint{4.789276in}{1.411136in}}%
\pgfpathlineto{\pgfqpoint{4.791424in}{1.409561in}}%
\pgfpathlineto{\pgfqpoint{4.792497in}{1.409023in}}%
\pgfpathlineto{\pgfqpoint{4.797867in}{1.408856in}}%
\pgfpathlineto{\pgfqpoint{4.798940in}{1.407353in}}%
\pgfpathlineto{\pgfqpoint{4.803236in}{1.407832in}}%
\pgfpathlineto{\pgfqpoint{4.807531in}{1.407019in}}%
\pgfpathlineto{\pgfqpoint{4.810753in}{1.406727in}}%
\pgfpathlineto{\pgfqpoint{4.812900in}{1.404844in}}%
\pgfpathlineto{\pgfqpoint{4.815048in}{1.403547in}}%
\pgfpathlineto{\pgfqpoint{4.818270in}{1.403241in}}%
\pgfpathlineto{\pgfqpoint{4.819343in}{1.402219in}}%
\pgfpathlineto{\pgfqpoint{4.822565in}{1.402141in}}%
\pgfpathlineto{\pgfqpoint{4.827934in}{1.402635in}}%
\pgfpathlineto{\pgfqpoint{4.834377in}{1.402974in}}%
\pgfpathlineto{\pgfqpoint{4.840820in}{1.401275in}}%
\pgfpathlineto{\pgfqpoint{4.842968in}{1.401831in}}%
\pgfpathlineto{\pgfqpoint{4.845115in}{1.401265in}}%
\pgfpathlineto{\pgfqpoint{4.856928in}{1.401605in}}%
\pgfpathlineto{\pgfqpoint{4.859075in}{1.400539in}}%
\pgfpathlineto{\pgfqpoint{4.863371in}{1.400677in}}%
\pgfpathlineto{\pgfqpoint{4.874109in}{1.399506in}}%
\pgfpathlineto{\pgfqpoint{4.875183in}{1.398915in}}%
\pgfpathlineto{\pgfqpoint{4.882700in}{1.398020in}}%
\pgfpathlineto{\pgfqpoint{4.888069in}{1.397300in}}%
\pgfpathlineto{\pgfqpoint{4.890217in}{1.396147in}}%
\pgfpathlineto{\pgfqpoint{4.904177in}{1.393392in}}%
\pgfpathlineto{\pgfqpoint{4.905250in}{1.393633in}}%
\pgfpathlineto{\pgfqpoint{4.912767in}{1.393019in}}%
\pgfpathlineto{\pgfqpoint{4.917063in}{1.392722in}}%
\pgfpathlineto{\pgfqpoint{4.925653in}{1.391622in}}%
\pgfpathlineto{\pgfqpoint{4.931023in}{1.391902in}}%
\pgfpathlineto{\pgfqpoint{4.934244in}{1.391209in}}%
\pgfpathlineto{\pgfqpoint{4.938539in}{1.391181in}}%
\pgfpathlineto{\pgfqpoint{4.940687in}{1.390394in}}%
\pgfpathlineto{\pgfqpoint{4.942835in}{1.388970in}}%
\pgfpathlineto{\pgfqpoint{4.965385in}{1.387504in}}%
\pgfpathlineto{\pgfqpoint{4.972902in}{1.387550in}}%
\pgfpathlineto{\pgfqpoint{4.980419in}{1.387510in}}%
\pgfpathlineto{\pgfqpoint{4.999748in}{1.386496in}}%
\pgfpathlineto{\pgfqpoint{5.002970in}{1.385976in}}%
\pgfpathlineto{\pgfqpoint{5.007265in}{1.385853in}}%
\pgfpathlineto{\pgfqpoint{5.009413in}{1.385446in}}%
\pgfpathlineto{\pgfqpoint{5.010487in}{1.385301in}}%
\pgfpathlineto{\pgfqpoint{5.024447in}{1.385683in}}%
\pgfpathlineto{\pgfqpoint{5.025520in}{1.385440in}}%
\pgfpathlineto{\pgfqpoint{5.030890in}{1.385190in}}%
\pgfpathlineto{\pgfqpoint{5.033037in}{1.384886in}}%
\pgfpathlineto{\pgfqpoint{5.054514in}{1.383464in}}%
\pgfpathlineto{\pgfqpoint{5.055588in}{1.383615in}}%
\pgfpathlineto{\pgfqpoint{5.060957in}{1.383108in}}%
\pgfpathlineto{\pgfqpoint{5.075991in}{1.382630in}}%
\pgfpathlineto{\pgfqpoint{5.078138in}{1.381751in}}%
\pgfpathlineto{\pgfqpoint{5.093172in}{1.380541in}}%
\pgfpathlineto{\pgfqpoint{5.113575in}{1.379945in}}%
\pgfpathlineto{\pgfqpoint{5.115723in}{1.379653in}}%
\pgfpathlineto{\pgfqpoint{5.133978in}{1.379393in}}%
\pgfpathlineto{\pgfqpoint{5.137200in}{1.379491in}}%
\pgfpathlineto{\pgfqpoint{5.143643in}{1.379442in}}%
\pgfpathlineto{\pgfqpoint{5.145790in}{1.378998in}}%
\pgfpathlineto{\pgfqpoint{5.153307in}{1.378952in}}%
\pgfpathlineto{\pgfqpoint{5.167267in}{1.378832in}}%
\pgfpathlineto{\pgfqpoint{5.168341in}{1.378522in}}%
\pgfpathlineto{\pgfqpoint{5.213442in}{1.378311in}}%
\pgfpathlineto{\pgfqpoint{5.235993in}{1.378076in}}%
\pgfpathlineto{\pgfqpoint{5.295054in}{1.377382in}}%
\pgfpathlineto{\pgfqpoint{5.299349in}{1.377271in}}%
\pgfpathlineto{\pgfqpoint{5.303645in}{1.377385in}}%
\pgfpathlineto{\pgfqpoint{5.344450in}{1.376378in}}%
\pgfpathlineto{\pgfqpoint{5.348746in}{1.376322in}}%
\pgfpathlineto{\pgfqpoint{5.371296in}{1.376062in}}%
\pgfpathlineto{\pgfqpoint{5.431431in}{1.373682in}}%
\pgfpathlineto{\pgfqpoint{5.446465in}{1.373287in}}%
\pgfpathlineto{\pgfqpoint{5.461499in}{1.372700in}}%
\pgfpathlineto{\pgfqpoint{5.625796in}{1.371397in}}%
\pgfpathlineto{\pgfqpoint{5.717072in}{1.369290in}}%
\pgfpathlineto{\pgfqpoint{5.753583in}{1.368893in}}%
\pgfpathlineto{\pgfqpoint{5.767543in}{1.368772in}}%
\pgfpathlineto{\pgfqpoint{5.790093in}{1.368488in}}%
\pgfpathlineto{\pgfqpoint{5.822309in}{1.368135in}}%
\pgfpathlineto{\pgfqpoint{5.841638in}{1.367738in}}%
\pgfpathlineto{\pgfqpoint{5.865262in}{1.367542in}}%
\pgfpathlineto{\pgfqpoint{5.904994in}{1.367450in}}%
\pgfpathlineto{\pgfqpoint{5.957612in}{1.367171in}}%
\pgfpathlineto{\pgfqpoint{6.112245in}{1.366813in}}%
\pgfpathlineto{\pgfqpoint{6.145534in}{1.366670in}}%
\pgfpathlineto{\pgfqpoint{6.408624in}{1.366061in}}%
\pgfpathlineto{\pgfqpoint{6.589029in}{1.365747in}}%
\pgfpathlineto{\pgfqpoint{6.589029in}{1.365747in}}%
\pgfusepath{stroke}%
\end{pgfscope}%
\begin{pgfscope}%
\pgfsetrectcap%
\pgfsetmiterjoin%
\pgfsetlinewidth{0.803000pt}%
\definecolor{currentstroke}{rgb}{1.000000,1.000000,1.000000}%
\pgfsetstrokecolor{currentstroke}%
\pgfsetdash{}{0pt}%
\pgfpathmoveto{\pgfqpoint{4.123120in}{1.347524in}}%
\pgfpathlineto{\pgfqpoint{4.123120in}{1.748409in}}%
\pgfusepath{stroke}%
\end{pgfscope}%
\begin{pgfscope}%
\pgfsetrectcap%
\pgfsetmiterjoin%
\pgfsetlinewidth{0.803000pt}%
\definecolor{currentstroke}{rgb}{1.000000,1.000000,1.000000}%
\pgfsetstrokecolor{currentstroke}%
\pgfsetdash{}{0pt}%
\pgfpathmoveto{\pgfqpoint{6.706453in}{1.347524in}}%
\pgfpathlineto{\pgfqpoint{6.706453in}{1.748409in}}%
\pgfusepath{stroke}%
\end{pgfscope}%
\begin{pgfscope}%
\pgfsetrectcap%
\pgfsetmiterjoin%
\pgfsetlinewidth{0.803000pt}%
\definecolor{currentstroke}{rgb}{1.000000,1.000000,1.000000}%
\pgfsetstrokecolor{currentstroke}%
\pgfsetdash{}{0pt}%
\pgfpathmoveto{\pgfqpoint{4.123120in}{1.347524in}}%
\pgfpathlineto{\pgfqpoint{6.706453in}{1.347524in}}%
\pgfusepath{stroke}%
\end{pgfscope}%
\begin{pgfscope}%
\pgfsetrectcap%
\pgfsetmiterjoin%
\pgfsetlinewidth{0.803000pt}%
\definecolor{currentstroke}{rgb}{1.000000,1.000000,1.000000}%
\pgfsetstrokecolor{currentstroke}%
\pgfsetdash{}{0pt}%
\pgfpathmoveto{\pgfqpoint{4.123120in}{1.748409in}}%
\pgfpathlineto{\pgfqpoint{6.706453in}{1.748409in}}%
\pgfusepath{stroke}%
\end{pgfscope}%
\begin{pgfscope}%
\definecolor{textcolor}{rgb}{0.150000,0.150000,0.150000}%
\pgfsetstrokecolor{textcolor}%
\pgfsetfillcolor{textcolor}%
\pgftext[x=5.414787in,y=1.831742in,,base]{\color{textcolor}\rmfamily\fontsize{16.800000}{20.160000}\selectfont VZ}%
\end{pgfscope}%
\begin{pgfscope}%
\pgfsetbuttcap%
\pgfsetmiterjoin%
\definecolor{currentfill}{rgb}{0.917647,0.917647,0.949020}%
\pgfsetfillcolor{currentfill}%
\pgfsetlinewidth{0.000000pt}%
\definecolor{currentstroke}{rgb}{0.000000,0.000000,0.000000}%
\pgfsetstrokecolor{currentstroke}%
\pgfsetstrokeopacity{0.000000}%
\pgfsetdash{}{0pt}%
\pgfpathmoveto{\pgfqpoint{0.506453in}{0.385400in}}%
\pgfpathlineto{\pgfqpoint{3.089787in}{0.385400in}}%
\pgfpathlineto{\pgfqpoint{3.089787in}{0.786285in}}%
\pgfpathlineto{\pgfqpoint{0.506453in}{0.786285in}}%
\pgfpathclose%
\pgfusepath{fill}%
\end{pgfscope}%
\begin{pgfscope}%
\pgfpathrectangle{\pgfqpoint{0.506453in}{0.385400in}}{\pgfqpoint{2.583333in}{0.400885in}}%
\pgfusepath{clip}%
\pgfsetroundcap%
\pgfsetroundjoin%
\pgfsetlinewidth{0.803000pt}%
\definecolor{currentstroke}{rgb}{1.000000,1.000000,1.000000}%
\pgfsetstrokecolor{currentstroke}%
\pgfsetdash{}{0pt}%
\pgfpathmoveto{\pgfqpoint{0.621730in}{0.385400in}}%
\pgfpathlineto{\pgfqpoint{0.621730in}{0.786285in}}%
\pgfusepath{stroke}%
\end{pgfscope}%
\begin{pgfscope}%
\definecolor{textcolor}{rgb}{0.150000,0.150000,0.150000}%
\pgfsetstrokecolor{textcolor}%
\pgfsetfillcolor{textcolor}%
\pgftext[x=0.621730in,y=0.288178in,,top]{\color{textcolor}\rmfamily\fontsize{14.000000}{16.800000}\selectfont 2012}%
\end{pgfscope}%
\begin{pgfscope}%
\pgfpathrectangle{\pgfqpoint{0.506453in}{0.385400in}}{\pgfqpoint{2.583333in}{0.400885in}}%
\pgfusepath{clip}%
\pgfsetroundcap%
\pgfsetroundjoin%
\pgfsetlinewidth{0.803000pt}%
\definecolor{currentstroke}{rgb}{1.000000,1.000000,1.000000}%
\pgfsetstrokecolor{currentstroke}%
\pgfsetdash{}{0pt}%
\pgfpathmoveto{\pgfqpoint{1.014755in}{0.385400in}}%
\pgfpathlineto{\pgfqpoint{1.014755in}{0.786285in}}%
\pgfusepath{stroke}%
\end{pgfscope}%
\begin{pgfscope}%
\definecolor{textcolor}{rgb}{0.150000,0.150000,0.150000}%
\pgfsetstrokecolor{textcolor}%
\pgfsetfillcolor{textcolor}%
\pgftext[x=1.014755in,y=0.288178in,,top]{\color{textcolor}\rmfamily\fontsize{14.000000}{16.800000}\selectfont 2013}%
\end{pgfscope}%
\begin{pgfscope}%
\pgfpathrectangle{\pgfqpoint{0.506453in}{0.385400in}}{\pgfqpoint{2.583333in}{0.400885in}}%
\pgfusepath{clip}%
\pgfsetroundcap%
\pgfsetroundjoin%
\pgfsetlinewidth{0.803000pt}%
\definecolor{currentstroke}{rgb}{1.000000,1.000000,1.000000}%
\pgfsetstrokecolor{currentstroke}%
\pgfsetdash{}{0pt}%
\pgfpathmoveto{\pgfqpoint{1.406706in}{0.385400in}}%
\pgfpathlineto{\pgfqpoint{1.406706in}{0.786285in}}%
\pgfusepath{stroke}%
\end{pgfscope}%
\begin{pgfscope}%
\definecolor{textcolor}{rgb}{0.150000,0.150000,0.150000}%
\pgfsetstrokecolor{textcolor}%
\pgfsetfillcolor{textcolor}%
\pgftext[x=1.406706in,y=0.288178in,,top]{\color{textcolor}\rmfamily\fontsize{14.000000}{16.800000}\selectfont 2014}%
\end{pgfscope}%
\begin{pgfscope}%
\pgfpathrectangle{\pgfqpoint{0.506453in}{0.385400in}}{\pgfqpoint{2.583333in}{0.400885in}}%
\pgfusepath{clip}%
\pgfsetroundcap%
\pgfsetroundjoin%
\pgfsetlinewidth{0.803000pt}%
\definecolor{currentstroke}{rgb}{1.000000,1.000000,1.000000}%
\pgfsetstrokecolor{currentstroke}%
\pgfsetdash{}{0pt}%
\pgfpathmoveto{\pgfqpoint{1.798657in}{0.385400in}}%
\pgfpathlineto{\pgfqpoint{1.798657in}{0.786285in}}%
\pgfusepath{stroke}%
\end{pgfscope}%
\begin{pgfscope}%
\definecolor{textcolor}{rgb}{0.150000,0.150000,0.150000}%
\pgfsetstrokecolor{textcolor}%
\pgfsetfillcolor{textcolor}%
\pgftext[x=1.798657in,y=0.288178in,,top]{\color{textcolor}\rmfamily\fontsize{14.000000}{16.800000}\selectfont 2015}%
\end{pgfscope}%
\begin{pgfscope}%
\pgfpathrectangle{\pgfqpoint{0.506453in}{0.385400in}}{\pgfqpoint{2.583333in}{0.400885in}}%
\pgfusepath{clip}%
\pgfsetroundcap%
\pgfsetroundjoin%
\pgfsetlinewidth{0.803000pt}%
\definecolor{currentstroke}{rgb}{1.000000,1.000000,1.000000}%
\pgfsetstrokecolor{currentstroke}%
\pgfsetdash{}{0pt}%
\pgfpathmoveto{\pgfqpoint{2.190608in}{0.385400in}}%
\pgfpathlineto{\pgfqpoint{2.190608in}{0.786285in}}%
\pgfusepath{stroke}%
\end{pgfscope}%
\begin{pgfscope}%
\definecolor{textcolor}{rgb}{0.150000,0.150000,0.150000}%
\pgfsetstrokecolor{textcolor}%
\pgfsetfillcolor{textcolor}%
\pgftext[x=2.190608in,y=0.288178in,,top]{\color{textcolor}\rmfamily\fontsize{14.000000}{16.800000}\selectfont 2016}%
\end{pgfscope}%
\begin{pgfscope}%
\pgfpathrectangle{\pgfqpoint{0.506453in}{0.385400in}}{\pgfqpoint{2.583333in}{0.400885in}}%
\pgfusepath{clip}%
\pgfsetroundcap%
\pgfsetroundjoin%
\pgfsetlinewidth{0.803000pt}%
\definecolor{currentstroke}{rgb}{1.000000,1.000000,1.000000}%
\pgfsetstrokecolor{currentstroke}%
\pgfsetdash{}{0pt}%
\pgfpathmoveto{\pgfqpoint{2.583633in}{0.385400in}}%
\pgfpathlineto{\pgfqpoint{2.583633in}{0.786285in}}%
\pgfusepath{stroke}%
\end{pgfscope}%
\begin{pgfscope}%
\definecolor{textcolor}{rgb}{0.150000,0.150000,0.150000}%
\pgfsetstrokecolor{textcolor}%
\pgfsetfillcolor{textcolor}%
\pgftext[x=2.583633in,y=0.288178in,,top]{\color{textcolor}\rmfamily\fontsize{14.000000}{16.800000}\selectfont 2017}%
\end{pgfscope}%
\begin{pgfscope}%
\pgfpathrectangle{\pgfqpoint{0.506453in}{0.385400in}}{\pgfqpoint{2.583333in}{0.400885in}}%
\pgfusepath{clip}%
\pgfsetroundcap%
\pgfsetroundjoin%
\pgfsetlinewidth{0.803000pt}%
\definecolor{currentstroke}{rgb}{1.000000,1.000000,1.000000}%
\pgfsetstrokecolor{currentstroke}%
\pgfsetdash{}{0pt}%
\pgfpathmoveto{\pgfqpoint{2.975584in}{0.385400in}}%
\pgfpathlineto{\pgfqpoint{2.975584in}{0.786285in}}%
\pgfusepath{stroke}%
\end{pgfscope}%
\begin{pgfscope}%
\definecolor{textcolor}{rgb}{0.150000,0.150000,0.150000}%
\pgfsetstrokecolor{textcolor}%
\pgfsetfillcolor{textcolor}%
\pgftext[x=2.975584in,y=0.288178in,,top]{\color{textcolor}\rmfamily\fontsize{14.000000}{16.800000}\selectfont 2018}%
\end{pgfscope}%
\begin{pgfscope}%
\pgfpathrectangle{\pgfqpoint{0.506453in}{0.385400in}}{\pgfqpoint{2.583333in}{0.400885in}}%
\pgfusepath{clip}%
\pgfsetroundcap%
\pgfsetroundjoin%
\pgfsetlinewidth{0.803000pt}%
\definecolor{currentstroke}{rgb}{1.000000,1.000000,1.000000}%
\pgfsetstrokecolor{currentstroke}%
\pgfsetdash{}{0pt}%
\pgfpathmoveto{\pgfqpoint{0.506453in}{0.403457in}}%
\pgfpathlineto{\pgfqpoint{3.089787in}{0.403457in}}%
\pgfusepath{stroke}%
\end{pgfscope}%
\begin{pgfscope}%
\definecolor{textcolor}{rgb}{0.150000,0.150000,0.150000}%
\pgfsetstrokecolor{textcolor}%
\pgfsetfillcolor{textcolor}%
\pgftext[x=0.285520in,y=0.329591in,left,base]{\color{textcolor}\rmfamily\fontsize{14.000000}{16.800000}\selectfont 0}%
\end{pgfscope}%
\begin{pgfscope}%
\pgfpathrectangle{\pgfqpoint{0.506453in}{0.385400in}}{\pgfqpoint{2.583333in}{0.400885in}}%
\pgfusepath{clip}%
\pgfsetroundcap%
\pgfsetroundjoin%
\pgfsetlinewidth{0.803000pt}%
\definecolor{currentstroke}{rgb}{1.000000,1.000000,1.000000}%
\pgfsetstrokecolor{currentstroke}%
\pgfsetdash{}{0pt}%
\pgfpathmoveto{\pgfqpoint{0.506453in}{0.766741in}}%
\pgfpathlineto{\pgfqpoint{3.089787in}{0.766741in}}%
\pgfusepath{stroke}%
\end{pgfscope}%
\begin{pgfscope}%
\definecolor{textcolor}{rgb}{0.150000,0.150000,0.150000}%
\pgfsetstrokecolor{textcolor}%
\pgfsetfillcolor{textcolor}%
\pgftext[x=0.285520in,y=0.692875in,left,base]{\color{textcolor}\rmfamily\fontsize{14.000000}{16.800000}\selectfont 5}%
\end{pgfscope}%
\begin{pgfscope}%
\pgfpathrectangle{\pgfqpoint{0.506453in}{0.385400in}}{\pgfqpoint{2.583333in}{0.400885in}}%
\pgfusepath{clip}%
\pgfsetroundcap%
\pgfsetroundjoin%
\pgfsetlinewidth{1.505625pt}%
\definecolor{currentstroke}{rgb}{0.000000,0.000000,0.000000}%
\pgfsetstrokecolor{currentstroke}%
\pgfsetdash{}{0pt}%
\pgfpathmoveto{\pgfqpoint{0.623878in}{0.476114in}}%
\pgfpathlineto{\pgfqpoint{0.624952in}{0.474824in}}%
\pgfpathlineto{\pgfqpoint{0.626025in}{0.475372in}}%
\pgfpathlineto{\pgfqpoint{0.627099in}{0.474534in}}%
\pgfpathlineto{\pgfqpoint{0.632468in}{0.473405in}}%
\pgfpathlineto{\pgfqpoint{0.633542in}{0.474953in}}%
\pgfpathlineto{\pgfqpoint{0.634616in}{0.474534in}}%
\pgfpathlineto{\pgfqpoint{0.638911in}{0.475791in}}%
\pgfpathlineto{\pgfqpoint{0.639985in}{0.476694in}}%
\pgfpathlineto{\pgfqpoint{0.642133in}{0.474437in}}%
\pgfpathlineto{\pgfqpoint{0.645354in}{0.473728in}}%
\pgfpathlineto{\pgfqpoint{0.646428in}{0.474727in}}%
\pgfpathlineto{\pgfqpoint{0.648576in}{0.474630in}}%
\pgfpathlineto{\pgfqpoint{0.649650in}{0.474759in}}%
\pgfpathlineto{\pgfqpoint{0.652871in}{0.473921in}}%
\pgfpathlineto{\pgfqpoint{0.653945in}{0.474469in}}%
\pgfpathlineto{\pgfqpoint{0.655019in}{0.475791in}}%
\pgfpathlineto{\pgfqpoint{0.656093in}{0.478307in}}%
\pgfpathlineto{\pgfqpoint{0.657167in}{0.478984in}}%
\pgfpathlineto{\pgfqpoint{0.661462in}{0.478952in}}%
\pgfpathlineto{\pgfqpoint{0.662536in}{0.479919in}}%
\pgfpathlineto{\pgfqpoint{0.663610in}{0.482789in}}%
\pgfpathlineto{\pgfqpoint{0.664684in}{0.483821in}}%
\pgfpathlineto{\pgfqpoint{0.667905in}{0.482983in}}%
\pgfpathlineto{\pgfqpoint{0.668979in}{0.484757in}}%
\pgfpathlineto{\pgfqpoint{0.670053in}{0.485305in}}%
\pgfpathlineto{\pgfqpoint{0.671127in}{0.484370in}}%
\pgfpathlineto{\pgfqpoint{0.672200in}{0.485240in}}%
\pgfpathlineto{\pgfqpoint{0.676496in}{0.484660in}}%
\pgfpathlineto{\pgfqpoint{0.677570in}{0.486208in}}%
\pgfpathlineto{\pgfqpoint{0.678643in}{0.486272in}}%
\pgfpathlineto{\pgfqpoint{0.679717in}{0.487046in}}%
\pgfpathlineto{\pgfqpoint{0.682939in}{0.486563in}}%
\pgfpathlineto{\pgfqpoint{0.684013in}{0.488046in}}%
\pgfpathlineto{\pgfqpoint{0.685087in}{0.486208in}}%
\pgfpathlineto{\pgfqpoint{0.686160in}{0.486788in}}%
\pgfpathlineto{\pgfqpoint{0.687234in}{0.486047in}}%
\pgfpathlineto{\pgfqpoint{0.690456in}{0.486111in}}%
\pgfpathlineto{\pgfqpoint{0.691530in}{0.485111in}}%
\pgfpathlineto{\pgfqpoint{0.692603in}{0.485692in}}%
\pgfpathlineto{\pgfqpoint{0.693677in}{0.487337in}}%
\pgfpathlineto{\pgfqpoint{0.694751in}{0.486756in}}%
\pgfpathlineto{\pgfqpoint{0.697973in}{0.486305in}}%
\pgfpathlineto{\pgfqpoint{0.699046in}{0.486853in}}%
\pgfpathlineto{\pgfqpoint{0.701194in}{0.486659in}}%
\pgfpathlineto{\pgfqpoint{0.702268in}{0.486434in}}%
\pgfpathlineto{\pgfqpoint{0.705489in}{0.487949in}}%
\pgfpathlineto{\pgfqpoint{0.706563in}{0.486305in}}%
\pgfpathlineto{\pgfqpoint{0.708711in}{0.486853in}}%
\pgfpathlineto{\pgfqpoint{0.709785in}{0.487917in}}%
\pgfpathlineto{\pgfqpoint{0.714080in}{0.488659in}}%
\pgfpathlineto{\pgfqpoint{0.717302in}{0.487369in}}%
\pgfpathlineto{\pgfqpoint{0.720523in}{0.488046in}}%
\pgfpathlineto{\pgfqpoint{0.721597in}{0.489046in}}%
\pgfpathlineto{\pgfqpoint{0.722671in}{0.488046in}}%
\pgfpathlineto{\pgfqpoint{0.723745in}{0.489497in}}%
\pgfpathlineto{\pgfqpoint{0.728040in}{0.488433in}}%
\pgfpathlineto{\pgfqpoint{0.729114in}{0.486466in}}%
\pgfpathlineto{\pgfqpoint{0.730188in}{0.486917in}}%
\pgfpathlineto{\pgfqpoint{0.732335in}{0.491013in}}%
\pgfpathlineto{\pgfqpoint{0.735557in}{0.489368in}}%
\pgfpathlineto{\pgfqpoint{0.736631in}{0.490271in}}%
\pgfpathlineto{\pgfqpoint{0.739852in}{0.489497in}}%
\pgfpathlineto{\pgfqpoint{0.743074in}{0.487433in}}%
\pgfpathlineto{\pgfqpoint{0.744148in}{0.488014in}}%
\pgfpathlineto{\pgfqpoint{0.746295in}{0.491045in}}%
\pgfpathlineto{\pgfqpoint{0.747369in}{0.491271in}}%
\pgfpathlineto{\pgfqpoint{0.752738in}{0.490336in}}%
\pgfpathlineto{\pgfqpoint{0.753812in}{0.486240in}}%
\pgfpathlineto{\pgfqpoint{0.754886in}{0.487208in}}%
\pgfpathlineto{\pgfqpoint{0.759181in}{0.487466in}}%
\pgfpathlineto{\pgfqpoint{0.760255in}{0.486950in}}%
\pgfpathlineto{\pgfqpoint{0.761329in}{0.487466in}}%
\pgfpathlineto{\pgfqpoint{0.766698in}{0.486369in}}%
\pgfpathlineto{\pgfqpoint{0.767772in}{0.487659in}}%
\pgfpathlineto{\pgfqpoint{0.769920in}{0.484144in}}%
\pgfpathlineto{\pgfqpoint{0.776363in}{0.489271in}}%
\pgfpathlineto{\pgfqpoint{0.777437in}{0.488981in}}%
\pgfpathlineto{\pgfqpoint{0.781732in}{0.489626in}}%
\pgfpathlineto{\pgfqpoint{0.784953in}{0.483886in}}%
\pgfpathlineto{\pgfqpoint{0.788175in}{0.485240in}}%
\pgfpathlineto{\pgfqpoint{0.789249in}{0.484982in}}%
\pgfpathlineto{\pgfqpoint{0.790323in}{0.486821in}}%
\pgfpathlineto{\pgfqpoint{0.792470in}{0.486982in}}%
\pgfpathlineto{\pgfqpoint{0.795692in}{0.486821in}}%
\pgfpathlineto{\pgfqpoint{0.796766in}{0.487401in}}%
\pgfpathlineto{\pgfqpoint{0.797840in}{0.485789in}}%
\pgfpathlineto{\pgfqpoint{0.799987in}{0.488272in}}%
\pgfpathlineto{\pgfqpoint{0.804283in}{0.490400in}}%
\pgfpathlineto{\pgfqpoint{0.805356in}{0.491271in}}%
\pgfpathlineto{\pgfqpoint{0.806430in}{0.489046in}}%
\pgfpathlineto{\pgfqpoint{0.807504in}{0.492980in}}%
\pgfpathlineto{\pgfqpoint{0.810726in}{0.490336in}}%
\pgfpathlineto{\pgfqpoint{0.811799in}{0.491690in}}%
\pgfpathlineto{\pgfqpoint{0.812873in}{0.491884in}}%
\pgfpathlineto{\pgfqpoint{0.813947in}{0.490529in}}%
\pgfpathlineto{\pgfqpoint{0.815021in}{0.492045in}}%
\pgfpathlineto{\pgfqpoint{0.818242in}{0.494173in}}%
\pgfpathlineto{\pgfqpoint{0.819316in}{0.493980in}}%
\pgfpathlineto{\pgfqpoint{0.821464in}{0.494464in}}%
\pgfpathlineto{\pgfqpoint{0.822538in}{0.493206in}}%
\pgfpathlineto{\pgfqpoint{0.826833in}{0.491077in}}%
\pgfpathlineto{\pgfqpoint{0.827907in}{0.489949in}}%
\pgfpathlineto{\pgfqpoint{0.830055in}{0.492367in}}%
\pgfpathlineto{\pgfqpoint{0.834350in}{0.495334in}}%
\pgfpathlineto{\pgfqpoint{0.835424in}{0.495044in}}%
\pgfpathlineto{\pgfqpoint{0.836498in}{0.493303in}}%
\pgfpathlineto{\pgfqpoint{0.837572in}{0.493561in}}%
\pgfpathlineto{\pgfqpoint{0.840793in}{0.492625in}}%
\pgfpathlineto{\pgfqpoint{0.841867in}{0.491335in}}%
\pgfpathlineto{\pgfqpoint{0.842941in}{0.491013in}}%
\pgfpathlineto{\pgfqpoint{0.845088in}{0.495979in}}%
\pgfpathlineto{\pgfqpoint{0.848310in}{0.497205in}}%
\pgfpathlineto{\pgfqpoint{0.850458in}{0.494786in}}%
\pgfpathlineto{\pgfqpoint{0.852605in}{0.497301in}}%
\pgfpathlineto{\pgfqpoint{0.857975in}{0.497430in}}%
\pgfpathlineto{\pgfqpoint{0.859048in}{0.495528in}}%
\pgfpathlineto{\pgfqpoint{0.860122in}{0.495947in}}%
\pgfpathlineto{\pgfqpoint{0.863344in}{0.495399in}}%
\pgfpathlineto{\pgfqpoint{0.864418in}{0.496527in}}%
\pgfpathlineto{\pgfqpoint{0.867639in}{0.496270in}}%
\pgfpathlineto{\pgfqpoint{0.874082in}{0.495173in}}%
\pgfpathlineto{\pgfqpoint{0.875156in}{0.494818in}}%
\pgfpathlineto{\pgfqpoint{0.880525in}{0.495786in}}%
\pgfpathlineto{\pgfqpoint{0.881599in}{0.494851in}}%
\pgfpathlineto{\pgfqpoint{0.882673in}{0.495979in}}%
\pgfpathlineto{\pgfqpoint{0.886968in}{0.496173in}}%
\pgfpathlineto{\pgfqpoint{0.888042in}{0.495496in}}%
\pgfpathlineto{\pgfqpoint{0.889116in}{0.496882in}}%
\pgfpathlineto{\pgfqpoint{0.890190in}{0.497011in}}%
\pgfpathlineto{\pgfqpoint{0.893411in}{0.496270in}}%
\pgfpathlineto{\pgfqpoint{0.895559in}{0.499720in}}%
\pgfpathlineto{\pgfqpoint{0.896633in}{0.500817in}}%
\pgfpathlineto{\pgfqpoint{0.897707in}{0.500301in}}%
\pgfpathlineto{\pgfqpoint{0.902002in}{0.499881in}}%
\pgfpathlineto{\pgfqpoint{0.903076in}{0.500688in}}%
\pgfpathlineto{\pgfqpoint{0.908445in}{0.499914in}}%
\pgfpathlineto{\pgfqpoint{0.909519in}{0.500526in}}%
\pgfpathlineto{\pgfqpoint{0.910593in}{0.499043in}}%
\pgfpathlineto{\pgfqpoint{0.912740in}{0.500333in}}%
\pgfpathlineto{\pgfqpoint{0.915962in}{0.502074in}}%
\pgfpathlineto{\pgfqpoint{0.917036in}{0.501558in}}%
\pgfpathlineto{\pgfqpoint{0.919183in}{0.504203in}}%
\pgfpathlineto{\pgfqpoint{0.920257in}{0.504622in}}%
\pgfpathlineto{\pgfqpoint{0.923479in}{0.503526in}}%
\pgfpathlineto{\pgfqpoint{0.924553in}{0.502107in}}%
\pgfpathlineto{\pgfqpoint{0.925626in}{0.502590in}}%
\pgfpathlineto{\pgfqpoint{0.926700in}{0.503751in}}%
\pgfpathlineto{\pgfqpoint{0.930996in}{0.504364in}}%
\pgfpathlineto{\pgfqpoint{0.933143in}{0.506299in}}%
\pgfpathlineto{\pgfqpoint{0.934217in}{0.505815in}}%
\pgfpathlineto{\pgfqpoint{0.935291in}{0.504428in}}%
\pgfpathlineto{\pgfqpoint{0.938512in}{0.503719in}}%
\pgfpathlineto{\pgfqpoint{0.939586in}{0.502010in}}%
\pgfpathlineto{\pgfqpoint{0.940660in}{0.501913in}}%
\pgfpathlineto{\pgfqpoint{0.942808in}{0.503235in}}%
\pgfpathlineto{\pgfqpoint{0.948177in}{0.503558in}}%
\pgfpathlineto{\pgfqpoint{0.949251in}{0.507234in}}%
\pgfpathlineto{\pgfqpoint{0.953546in}{0.505525in}}%
\pgfpathlineto{\pgfqpoint{0.954620in}{0.507234in}}%
\pgfpathlineto{\pgfqpoint{0.956768in}{0.505944in}}%
\pgfpathlineto{\pgfqpoint{0.957841in}{0.506557in}}%
\pgfpathlineto{\pgfqpoint{0.961063in}{0.506783in}}%
\pgfpathlineto{\pgfqpoint{0.964285in}{0.505396in}}%
\pgfpathlineto{\pgfqpoint{0.965358in}{0.507363in}}%
\pgfpathlineto{\pgfqpoint{0.969654in}{0.510008in}}%
\pgfpathlineto{\pgfqpoint{0.970728in}{0.510233in}}%
\pgfpathlineto{\pgfqpoint{0.972875in}{0.511297in}}%
\pgfpathlineto{\pgfqpoint{0.978244in}{0.510685in}}%
\pgfpathlineto{\pgfqpoint{0.980392in}{0.512458in}}%
\pgfpathlineto{\pgfqpoint{0.985761in}{0.511265in}}%
\pgfpathlineto{\pgfqpoint{0.987909in}{0.511620in}}%
\pgfpathlineto{\pgfqpoint{0.992204in}{0.511878in}}%
\pgfpathlineto{\pgfqpoint{0.994352in}{0.510620in}}%
\pgfpathlineto{\pgfqpoint{0.995426in}{0.510362in}}%
\pgfpathlineto{\pgfqpoint{0.999721in}{0.513200in}}%
\pgfpathlineto{\pgfqpoint{1.000795in}{0.512071in}}%
\pgfpathlineto{\pgfqpoint{1.001869in}{0.514458in}}%
\pgfpathlineto{\pgfqpoint{1.002943in}{0.513232in}}%
\pgfpathlineto{\pgfqpoint{1.006164in}{0.513523in}}%
\pgfpathlineto{\pgfqpoint{1.010460in}{0.511684in}}%
\pgfpathlineto{\pgfqpoint{1.013681in}{0.513813in}}%
\pgfpathlineto{\pgfqpoint{1.015829in}{0.516586in}}%
\pgfpathlineto{\pgfqpoint{1.016903in}{0.516683in}}%
\pgfpathlineto{\pgfqpoint{1.017976in}{0.517586in}}%
\pgfpathlineto{\pgfqpoint{1.021198in}{0.518424in}}%
\pgfpathlineto{\pgfqpoint{1.023346in}{0.521262in}}%
\pgfpathlineto{\pgfqpoint{1.024419in}{0.520327in}}%
\pgfpathlineto{\pgfqpoint{1.025493in}{0.520779in}}%
\pgfpathlineto{\pgfqpoint{1.031936in}{0.519876in}}%
\pgfpathlineto{\pgfqpoint{1.033010in}{0.518682in}}%
\pgfpathlineto{\pgfqpoint{1.040527in}{0.519843in}}%
\pgfpathlineto{\pgfqpoint{1.043749in}{0.517296in}}%
\pgfpathlineto{\pgfqpoint{1.044822in}{0.517522in}}%
\pgfpathlineto{\pgfqpoint{1.045896in}{0.516328in}}%
\pgfpathlineto{\pgfqpoint{1.046970in}{0.518424in}}%
\pgfpathlineto{\pgfqpoint{1.048044in}{0.518908in}}%
\pgfpathlineto{\pgfqpoint{1.051265in}{0.517522in}}%
\pgfpathlineto{\pgfqpoint{1.053413in}{0.520553in}}%
\pgfpathlineto{\pgfqpoint{1.054487in}{0.517812in}}%
\pgfpathlineto{\pgfqpoint{1.055561in}{0.518360in}}%
\pgfpathlineto{\pgfqpoint{1.058782in}{0.517264in}}%
\pgfpathlineto{\pgfqpoint{1.059856in}{0.517618in}}%
\pgfpathlineto{\pgfqpoint{1.060930in}{0.517102in}}%
\pgfpathlineto{\pgfqpoint{1.063078in}{0.519456in}}%
\pgfpathlineto{\pgfqpoint{1.067373in}{0.519198in}}%
\pgfpathlineto{\pgfqpoint{1.068447in}{0.517586in}}%
\pgfpathlineto{\pgfqpoint{1.070595in}{0.520521in}}%
\pgfpathlineto{\pgfqpoint{1.073816in}{0.517909in}}%
\pgfpathlineto{\pgfqpoint{1.075964in}{0.520843in}}%
\pgfpathlineto{\pgfqpoint{1.078111in}{0.519521in}}%
\pgfpathlineto{\pgfqpoint{1.085628in}{0.521424in}}%
\pgfpathlineto{\pgfqpoint{1.088850in}{0.521811in}}%
\pgfpathlineto{\pgfqpoint{1.090997in}{0.520488in}}%
\pgfpathlineto{\pgfqpoint{1.092071in}{0.521424in}}%
\pgfpathlineto{\pgfqpoint{1.093145in}{0.519876in}}%
\pgfpathlineto{\pgfqpoint{1.096367in}{0.519521in}}%
\pgfpathlineto{\pgfqpoint{1.097441in}{0.518005in}}%
\pgfpathlineto{\pgfqpoint{1.098514in}{0.520456in}}%
\pgfpathlineto{\pgfqpoint{1.099588in}{0.519327in}}%
\pgfpathlineto{\pgfqpoint{1.100662in}{0.521004in}}%
\pgfpathlineto{\pgfqpoint{1.103884in}{0.523875in}}%
\pgfpathlineto{\pgfqpoint{1.104957in}{0.526422in}}%
\pgfpathlineto{\pgfqpoint{1.107105in}{0.528164in}}%
\pgfpathlineto{\pgfqpoint{1.111400in}{0.526325in}}%
\pgfpathlineto{\pgfqpoint{1.112474in}{0.526777in}}%
\pgfpathlineto{\pgfqpoint{1.113548in}{0.524423in}}%
\pgfpathlineto{\pgfqpoint{1.114622in}{0.525680in}}%
\pgfpathlineto{\pgfqpoint{1.115696in}{0.524713in}}%
\pgfpathlineto{\pgfqpoint{1.118917in}{0.525551in}}%
\pgfpathlineto{\pgfqpoint{1.119991in}{0.524423in}}%
\pgfpathlineto{\pgfqpoint{1.121065in}{0.526067in}}%
\pgfpathlineto{\pgfqpoint{1.122139in}{0.526519in}}%
\pgfpathlineto{\pgfqpoint{1.126434in}{0.521875in}}%
\pgfpathlineto{\pgfqpoint{1.127508in}{0.524391in}}%
\pgfpathlineto{\pgfqpoint{1.128582in}{0.522585in}}%
\pgfpathlineto{\pgfqpoint{1.129656in}{0.521972in}}%
\pgfpathlineto{\pgfqpoint{1.130729in}{0.523842in}}%
\pgfpathlineto{\pgfqpoint{1.133951in}{0.523520in}}%
\pgfpathlineto{\pgfqpoint{1.137173in}{0.527357in}}%
\pgfpathlineto{\pgfqpoint{1.138246in}{0.526261in}}%
\pgfpathlineto{\pgfqpoint{1.142542in}{0.527164in}}%
\pgfpathlineto{\pgfqpoint{1.143616in}{0.525358in}}%
\pgfpathlineto{\pgfqpoint{1.144689in}{0.532259in}}%
\pgfpathlineto{\pgfqpoint{1.145763in}{0.535291in}}%
\pgfpathlineto{\pgfqpoint{1.148985in}{0.534807in}}%
\pgfpathlineto{\pgfqpoint{1.150059in}{0.535484in}}%
\pgfpathlineto{\pgfqpoint{1.153280in}{0.534710in}}%
\pgfpathlineto{\pgfqpoint{1.156502in}{0.534839in}}%
\pgfpathlineto{\pgfqpoint{1.157575in}{0.535871in}}%
\pgfpathlineto{\pgfqpoint{1.158649in}{0.537903in}}%
\pgfpathlineto{\pgfqpoint{1.159723in}{0.536452in}}%
\pgfpathlineto{\pgfqpoint{1.160797in}{0.539967in}}%
\pgfpathlineto{\pgfqpoint{1.164018in}{0.537774in}}%
\pgfpathlineto{\pgfqpoint{1.165092in}{0.537742in}}%
\pgfpathlineto{\pgfqpoint{1.167240in}{0.535194in}}%
\pgfpathlineto{\pgfqpoint{1.168314in}{0.536935in}}%
\pgfpathlineto{\pgfqpoint{1.172609in}{0.536613in}}%
\pgfpathlineto{\pgfqpoint{1.173683in}{0.535226in}}%
\pgfpathlineto{\pgfqpoint{1.174757in}{0.537355in}}%
\pgfpathlineto{\pgfqpoint{1.175831in}{0.535226in}}%
\pgfpathlineto{\pgfqpoint{1.180126in}{0.536742in}}%
\pgfpathlineto{\pgfqpoint{1.181200in}{0.534549in}}%
\pgfpathlineto{\pgfqpoint{1.182274in}{0.536129in}}%
\pgfpathlineto{\pgfqpoint{1.186569in}{0.538322in}}%
\pgfpathlineto{\pgfqpoint{1.187643in}{0.536355in}}%
\pgfpathlineto{\pgfqpoint{1.188717in}{0.535968in}}%
\pgfpathlineto{\pgfqpoint{1.189791in}{0.538258in}}%
\pgfpathlineto{\pgfqpoint{1.190864in}{0.537290in}}%
\pgfpathlineto{\pgfqpoint{1.194086in}{0.538322in}}%
\pgfpathlineto{\pgfqpoint{1.195160in}{0.539580in}}%
\pgfpathlineto{\pgfqpoint{1.196234in}{0.538548in}}%
\pgfpathlineto{\pgfqpoint{1.197307in}{0.535452in}}%
\pgfpathlineto{\pgfqpoint{1.198381in}{0.536226in}}%
\pgfpathlineto{\pgfqpoint{1.201603in}{0.535323in}}%
\pgfpathlineto{\pgfqpoint{1.204824in}{0.539741in}}%
\pgfpathlineto{\pgfqpoint{1.205898in}{0.538645in}}%
\pgfpathlineto{\pgfqpoint{1.211267in}{0.541805in}}%
\pgfpathlineto{\pgfqpoint{1.213415in}{0.544578in}}%
\pgfpathlineto{\pgfqpoint{1.218784in}{0.541547in}}%
\pgfpathlineto{\pgfqpoint{1.219858in}{0.543966in}}%
\pgfpathlineto{\pgfqpoint{1.220932in}{0.544514in}}%
\pgfpathlineto{\pgfqpoint{1.226301in}{0.543740in}}%
\pgfpathlineto{\pgfqpoint{1.227375in}{0.544740in}}%
\pgfpathlineto{\pgfqpoint{1.228449in}{0.543933in}}%
\pgfpathlineto{\pgfqpoint{1.231670in}{0.544901in}}%
\pgfpathlineto{\pgfqpoint{1.233818in}{0.541579in}}%
\pgfpathlineto{\pgfqpoint{1.234892in}{0.547416in}}%
\pgfpathlineto{\pgfqpoint{1.235966in}{0.546352in}}%
\pgfpathlineto{\pgfqpoint{1.240261in}{0.545062in}}%
\pgfpathlineto{\pgfqpoint{1.241335in}{0.534388in}}%
\pgfpathlineto{\pgfqpoint{1.242409in}{0.535968in}}%
\pgfpathlineto{\pgfqpoint{1.243483in}{0.539548in}}%
\pgfpathlineto{\pgfqpoint{1.246704in}{0.539967in}}%
\pgfpathlineto{\pgfqpoint{1.248852in}{0.537613in}}%
\pgfpathlineto{\pgfqpoint{1.250999in}{0.536452in}}%
\pgfpathlineto{\pgfqpoint{1.256369in}{0.536355in}}%
\pgfpathlineto{\pgfqpoint{1.257442in}{0.533065in}}%
\pgfpathlineto{\pgfqpoint{1.258516in}{0.532453in}}%
\pgfpathlineto{\pgfqpoint{1.261738in}{0.533839in}}%
\pgfpathlineto{\pgfqpoint{1.262812in}{0.532517in}}%
\pgfpathlineto{\pgfqpoint{1.263885in}{0.536387in}}%
\pgfpathlineto{\pgfqpoint{1.266033in}{0.536903in}}%
\pgfpathlineto{\pgfqpoint{1.270329in}{0.533227in}}%
\pgfpathlineto{\pgfqpoint{1.272476in}{0.534130in}}%
\pgfpathlineto{\pgfqpoint{1.273550in}{0.533420in}}%
\pgfpathlineto{\pgfqpoint{1.277845in}{0.535355in}}%
\pgfpathlineto{\pgfqpoint{1.278919in}{0.534742in}}%
\pgfpathlineto{\pgfqpoint{1.281067in}{0.535097in}}%
\pgfpathlineto{\pgfqpoint{1.284288in}{0.536516in}}%
\pgfpathlineto{\pgfqpoint{1.285362in}{0.540999in}}%
\pgfpathlineto{\pgfqpoint{1.286436in}{0.542289in}}%
\pgfpathlineto{\pgfqpoint{1.287510in}{0.541353in}}%
\pgfpathlineto{\pgfqpoint{1.288584in}{0.544288in}}%
\pgfpathlineto{\pgfqpoint{1.291805in}{0.544578in}}%
\pgfpathlineto{\pgfqpoint{1.296101in}{0.551609in}}%
\pgfpathlineto{\pgfqpoint{1.299322in}{0.549674in}}%
\pgfpathlineto{\pgfqpoint{1.301470in}{0.546191in}}%
\pgfpathlineto{\pgfqpoint{1.302544in}{0.547674in}}%
\pgfpathlineto{\pgfqpoint{1.306839in}{0.545868in}}%
\pgfpathlineto{\pgfqpoint{1.307913in}{0.547448in}}%
\pgfpathlineto{\pgfqpoint{1.308987in}{0.546384in}}%
\pgfpathlineto{\pgfqpoint{1.310061in}{0.544030in}}%
\pgfpathlineto{\pgfqpoint{1.311134in}{0.545385in}}%
\pgfpathlineto{\pgfqpoint{1.314356in}{0.542289in}}%
\pgfpathlineto{\pgfqpoint{1.315430in}{0.539483in}}%
\pgfpathlineto{\pgfqpoint{1.316504in}{0.540450in}}%
\pgfpathlineto{\pgfqpoint{1.318651in}{0.546675in}}%
\pgfpathlineto{\pgfqpoint{1.321873in}{0.547610in}}%
\pgfpathlineto{\pgfqpoint{1.322947in}{0.546062in}}%
\pgfpathlineto{\pgfqpoint{1.325094in}{0.551222in}}%
\pgfpathlineto{\pgfqpoint{1.326168in}{0.552834in}}%
\pgfpathlineto{\pgfqpoint{1.330463in}{0.552479in}}%
\pgfpathlineto{\pgfqpoint{1.331537in}{0.551673in}}%
\pgfpathlineto{\pgfqpoint{1.332611in}{0.554672in}}%
\pgfpathlineto{\pgfqpoint{1.339054in}{0.555349in}}%
\pgfpathlineto{\pgfqpoint{1.340128in}{0.549996in}}%
\pgfpathlineto{\pgfqpoint{1.341202in}{0.551867in}}%
\pgfpathlineto{\pgfqpoint{1.344423in}{0.549835in}}%
\pgfpathlineto{\pgfqpoint{1.346571in}{0.551770in}}%
\pgfpathlineto{\pgfqpoint{1.347645in}{0.549577in}}%
\pgfpathlineto{\pgfqpoint{1.348719in}{0.551222in}}%
\pgfpathlineto{\pgfqpoint{1.351940in}{0.551931in}}%
\pgfpathlineto{\pgfqpoint{1.353014in}{0.551189in}}%
\pgfpathlineto{\pgfqpoint{1.354088in}{0.553544in}}%
\pgfpathlineto{\pgfqpoint{1.355162in}{0.553802in}}%
\pgfpathlineto{\pgfqpoint{1.356236in}{0.555188in}}%
\pgfpathlineto{\pgfqpoint{1.359457in}{0.553640in}}%
\pgfpathlineto{\pgfqpoint{1.360531in}{0.551738in}}%
\pgfpathlineto{\pgfqpoint{1.361605in}{0.552286in}}%
\pgfpathlineto{\pgfqpoint{1.362679in}{0.554898in}}%
\pgfpathlineto{\pgfqpoint{1.363752in}{0.555285in}}%
\pgfpathlineto{\pgfqpoint{1.366974in}{0.555156in}}%
\pgfpathlineto{\pgfqpoint{1.368048in}{0.556349in}}%
\pgfpathlineto{\pgfqpoint{1.369122in}{0.556704in}}%
\pgfpathlineto{\pgfqpoint{1.371269in}{0.556285in}}%
\pgfpathlineto{\pgfqpoint{1.374491in}{0.557542in}}%
\pgfpathlineto{\pgfqpoint{1.375565in}{0.555027in}}%
\pgfpathlineto{\pgfqpoint{1.376639in}{0.555769in}}%
\pgfpathlineto{\pgfqpoint{1.377712in}{0.554995in}}%
\pgfpathlineto{\pgfqpoint{1.382008in}{0.554898in}}%
\pgfpathlineto{\pgfqpoint{1.383082in}{0.553286in}}%
\pgfpathlineto{\pgfqpoint{1.384155in}{0.557962in}}%
\pgfpathlineto{\pgfqpoint{1.385229in}{0.556285in}}%
\pgfpathlineto{\pgfqpoint{1.386303in}{0.559219in}}%
\pgfpathlineto{\pgfqpoint{1.389525in}{0.559510in}}%
\pgfpathlineto{\pgfqpoint{1.390598in}{0.563670in}}%
\pgfpathlineto{\pgfqpoint{1.391672in}{0.565218in}}%
\pgfpathlineto{\pgfqpoint{1.393820in}{0.565701in}}%
\pgfpathlineto{\pgfqpoint{1.400263in}{0.568862in}}%
\pgfpathlineto{\pgfqpoint{1.401337in}{0.568475in}}%
\pgfpathlineto{\pgfqpoint{1.404558in}{0.569410in}}%
\pgfpathlineto{\pgfqpoint{1.405632in}{0.570732in}}%
\pgfpathlineto{\pgfqpoint{1.407780in}{0.569474in}}%
\pgfpathlineto{\pgfqpoint{1.408854in}{0.569603in}}%
\pgfpathlineto{\pgfqpoint{1.412075in}{0.568604in}}%
\pgfpathlineto{\pgfqpoint{1.414223in}{0.570377in}}%
\pgfpathlineto{\pgfqpoint{1.416371in}{0.569571in}}%
\pgfpathlineto{\pgfqpoint{1.419592in}{0.567894in}}%
\pgfpathlineto{\pgfqpoint{1.420666in}{0.570732in}}%
\pgfpathlineto{\pgfqpoint{1.421740in}{0.571538in}}%
\pgfpathlineto{\pgfqpoint{1.422814in}{0.570055in}}%
\pgfpathlineto{\pgfqpoint{1.423887in}{0.577891in}}%
\pgfpathlineto{\pgfqpoint{1.428183in}{0.577698in}}%
\pgfpathlineto{\pgfqpoint{1.429257in}{0.578472in}}%
\pgfpathlineto{\pgfqpoint{1.431404in}{0.569668in}}%
\pgfpathlineto{\pgfqpoint{1.434626in}{0.565895in}}%
\pgfpathlineto{\pgfqpoint{1.435700in}{0.569442in}}%
\pgfpathlineto{\pgfqpoint{1.436773in}{0.566572in}}%
\pgfpathlineto{\pgfqpoint{1.437847in}{0.569378in}}%
\pgfpathlineto{\pgfqpoint{1.438921in}{0.565282in}}%
\pgfpathlineto{\pgfqpoint{1.442143in}{0.563831in}}%
\pgfpathlineto{\pgfqpoint{1.444290in}{0.565443in}}%
\pgfpathlineto{\pgfqpoint{1.446438in}{0.570055in}}%
\pgfpathlineto{\pgfqpoint{1.449660in}{0.569152in}}%
\pgfpathlineto{\pgfqpoint{1.450733in}{0.570474in}}%
\pgfpathlineto{\pgfqpoint{1.451807in}{0.573086in}}%
\pgfpathlineto{\pgfqpoint{1.452881in}{0.572990in}}%
\pgfpathlineto{\pgfqpoint{1.453955in}{0.574473in}}%
\pgfpathlineto{\pgfqpoint{1.458250in}{0.574505in}}%
\pgfpathlineto{\pgfqpoint{1.459324in}{0.572861in}}%
\pgfpathlineto{\pgfqpoint{1.461472in}{0.572474in}}%
\pgfpathlineto{\pgfqpoint{1.465767in}{0.575312in}}%
\pgfpathlineto{\pgfqpoint{1.466841in}{0.574538in}}%
\pgfpathlineto{\pgfqpoint{1.468989in}{0.574409in}}%
\pgfpathlineto{\pgfqpoint{1.472210in}{0.571022in}}%
\pgfpathlineto{\pgfqpoint{1.473284in}{0.574086in}}%
\pgfpathlineto{\pgfqpoint{1.474358in}{0.572054in}}%
\pgfpathlineto{\pgfqpoint{1.476506in}{0.574118in}}%
\pgfpathlineto{\pgfqpoint{1.479727in}{0.574086in}}%
\pgfpathlineto{\pgfqpoint{1.480801in}{0.575344in}}%
\pgfpathlineto{\pgfqpoint{1.481875in}{0.574538in}}%
\pgfpathlineto{\pgfqpoint{1.482949in}{0.570506in}}%
\pgfpathlineto{\pgfqpoint{1.484022in}{0.570506in}}%
\pgfpathlineto{\pgfqpoint{1.487244in}{0.572796in}}%
\pgfpathlineto{\pgfqpoint{1.488318in}{0.574763in}}%
\pgfpathlineto{\pgfqpoint{1.490465in}{0.571313in}}%
\pgfpathlineto{\pgfqpoint{1.491539in}{0.572474in}}%
\pgfpathlineto{\pgfqpoint{1.494761in}{0.570506in}}%
\pgfpathlineto{\pgfqpoint{1.496908in}{0.566669in}}%
\pgfpathlineto{\pgfqpoint{1.497982in}{0.566733in}}%
\pgfpathlineto{\pgfqpoint{1.499056in}{0.563928in}}%
\pgfpathlineto{\pgfqpoint{1.502278in}{0.566798in}}%
\pgfpathlineto{\pgfqpoint{1.503351in}{0.565927in}}%
\pgfpathlineto{\pgfqpoint{1.505499in}{0.566153in}}%
\pgfpathlineto{\pgfqpoint{1.506573in}{0.560606in}}%
\pgfpathlineto{\pgfqpoint{1.510868in}{0.556736in}}%
\pgfpathlineto{\pgfqpoint{1.511942in}{0.560509in}}%
\pgfpathlineto{\pgfqpoint{1.514090in}{0.552254in}}%
\pgfpathlineto{\pgfqpoint{1.517311in}{0.555543in}}%
\pgfpathlineto{\pgfqpoint{1.518385in}{0.557865in}}%
\pgfpathlineto{\pgfqpoint{1.519459in}{0.561864in}}%
\pgfpathlineto{\pgfqpoint{1.520533in}{0.560800in}}%
\pgfpathlineto{\pgfqpoint{1.525902in}{0.562347in}}%
\pgfpathlineto{\pgfqpoint{1.526976in}{0.561477in}}%
\pgfpathlineto{\pgfqpoint{1.528050in}{0.561896in}}%
\pgfpathlineto{\pgfqpoint{1.529124in}{0.553995in}}%
\pgfpathlineto{\pgfqpoint{1.534493in}{0.556768in}}%
\pgfpathlineto{\pgfqpoint{1.535567in}{0.559413in}}%
\pgfpathlineto{\pgfqpoint{1.536640in}{0.558123in}}%
\pgfpathlineto{\pgfqpoint{1.539862in}{0.560187in}}%
\pgfpathlineto{\pgfqpoint{1.540936in}{0.558832in}}%
\pgfpathlineto{\pgfqpoint{1.543083in}{0.563025in}}%
\pgfpathlineto{\pgfqpoint{1.544157in}{0.562960in}}%
\pgfpathlineto{\pgfqpoint{1.548453in}{0.563928in}}%
\pgfpathlineto{\pgfqpoint{1.549527in}{0.563444in}}%
\pgfpathlineto{\pgfqpoint{1.550600in}{0.561606in}}%
\pgfpathlineto{\pgfqpoint{1.551674in}{0.563412in}}%
\pgfpathlineto{\pgfqpoint{1.554896in}{0.563831in}}%
\pgfpathlineto{\pgfqpoint{1.555970in}{0.561993in}}%
\pgfpathlineto{\pgfqpoint{1.557043in}{0.563637in}}%
\pgfpathlineto{\pgfqpoint{1.558117in}{0.563089in}}%
\pgfpathlineto{\pgfqpoint{1.559191in}{0.565153in}}%
\pgfpathlineto{\pgfqpoint{1.563486in}{0.566991in}}%
\pgfpathlineto{\pgfqpoint{1.564560in}{0.566475in}}%
\pgfpathlineto{\pgfqpoint{1.566708in}{0.567249in}}%
\pgfpathlineto{\pgfqpoint{1.569929in}{0.566250in}}%
\pgfpathlineto{\pgfqpoint{1.571003in}{0.564573in}}%
\pgfpathlineto{\pgfqpoint{1.573151in}{0.565250in}}%
\pgfpathlineto{\pgfqpoint{1.574225in}{0.565863in}}%
\pgfpathlineto{\pgfqpoint{1.577446in}{0.565540in}}%
\pgfpathlineto{\pgfqpoint{1.578520in}{0.566798in}}%
\pgfpathlineto{\pgfqpoint{1.580668in}{0.564927in}}%
\pgfpathlineto{\pgfqpoint{1.584963in}{0.563734in}}%
\pgfpathlineto{\pgfqpoint{1.587111in}{0.564508in}}%
\pgfpathlineto{\pgfqpoint{1.589259in}{0.563186in}}%
\pgfpathlineto{\pgfqpoint{1.592480in}{0.563154in}}%
\pgfpathlineto{\pgfqpoint{1.593554in}{0.561896in}}%
\pgfpathlineto{\pgfqpoint{1.594628in}{0.562863in}}%
\pgfpathlineto{\pgfqpoint{1.596775in}{0.563025in}}%
\pgfpathlineto{\pgfqpoint{1.599997in}{0.564121in}}%
\pgfpathlineto{\pgfqpoint{1.601071in}{0.566798in}}%
\pgfpathlineto{\pgfqpoint{1.602145in}{0.567249in}}%
\pgfpathlineto{\pgfqpoint{1.603218in}{0.568539in}}%
\pgfpathlineto{\pgfqpoint{1.607514in}{0.568701in}}%
\pgfpathlineto{\pgfqpoint{1.608588in}{0.567572in}}%
\pgfpathlineto{\pgfqpoint{1.609661in}{0.568249in}}%
\pgfpathlineto{\pgfqpoint{1.610735in}{0.567798in}}%
\pgfpathlineto{\pgfqpoint{1.616105in}{0.572538in}}%
\pgfpathlineto{\pgfqpoint{1.617178in}{0.573215in}}%
\pgfpathlineto{\pgfqpoint{1.618252in}{0.569571in}}%
\pgfpathlineto{\pgfqpoint{1.619326in}{0.571345in}}%
\pgfpathlineto{\pgfqpoint{1.622548in}{0.570603in}}%
\pgfpathlineto{\pgfqpoint{1.623621in}{0.572151in}}%
\pgfpathlineto{\pgfqpoint{1.624695in}{0.572119in}}%
\pgfpathlineto{\pgfqpoint{1.625769in}{0.573280in}}%
\pgfpathlineto{\pgfqpoint{1.626843in}{0.567217in}}%
\pgfpathlineto{\pgfqpoint{1.632212in}{0.566669in}}%
\pgfpathlineto{\pgfqpoint{1.633286in}{0.564347in}}%
\pgfpathlineto{\pgfqpoint{1.634360in}{0.564960in}}%
\pgfpathlineto{\pgfqpoint{1.637581in}{0.565218in}}%
\pgfpathlineto{\pgfqpoint{1.638655in}{0.563992in}}%
\pgfpathlineto{\pgfqpoint{1.639729in}{0.564089in}}%
\pgfpathlineto{\pgfqpoint{1.640803in}{0.562831in}}%
\pgfpathlineto{\pgfqpoint{1.641877in}{0.563928in}}%
\pgfpathlineto{\pgfqpoint{1.646172in}{0.563799in}}%
\pgfpathlineto{\pgfqpoint{1.647246in}{0.565895in}}%
\pgfpathlineto{\pgfqpoint{1.648320in}{0.566733in}}%
\pgfpathlineto{\pgfqpoint{1.649394in}{0.564927in}}%
\pgfpathlineto{\pgfqpoint{1.654763in}{0.569571in}}%
\pgfpathlineto{\pgfqpoint{1.655837in}{0.569217in}}%
\pgfpathlineto{\pgfqpoint{1.656910in}{0.569474in}}%
\pgfpathlineto{\pgfqpoint{1.660132in}{0.569410in}}%
\pgfpathlineto{\pgfqpoint{1.662280in}{0.570281in}}%
\pgfpathlineto{\pgfqpoint{1.664427in}{0.566733in}}%
\pgfpathlineto{\pgfqpoint{1.669796in}{0.568572in}}%
\pgfpathlineto{\pgfqpoint{1.671944in}{0.568023in}}%
\pgfpathlineto{\pgfqpoint{1.675166in}{0.569217in}}%
\pgfpathlineto{\pgfqpoint{1.676239in}{0.568088in}}%
\pgfpathlineto{\pgfqpoint{1.677313in}{0.570087in}}%
\pgfpathlineto{\pgfqpoint{1.679461in}{0.567894in}}%
\pgfpathlineto{\pgfqpoint{1.682682in}{0.568346in}}%
\pgfpathlineto{\pgfqpoint{1.683756in}{0.570410in}}%
\pgfpathlineto{\pgfqpoint{1.684830in}{0.569055in}}%
\pgfpathlineto{\pgfqpoint{1.685904in}{0.569732in}}%
\pgfpathlineto{\pgfqpoint{1.686978in}{0.569603in}}%
\pgfpathlineto{\pgfqpoint{1.691273in}{0.566669in}}%
\pgfpathlineto{\pgfqpoint{1.692347in}{0.568185in}}%
\pgfpathlineto{\pgfqpoint{1.693421in}{0.565347in}}%
\pgfpathlineto{\pgfqpoint{1.694495in}{0.566282in}}%
\pgfpathlineto{\pgfqpoint{1.697716in}{0.565508in}}%
\pgfpathlineto{\pgfqpoint{1.698790in}{0.567378in}}%
\pgfpathlineto{\pgfqpoint{1.699864in}{0.564798in}}%
\pgfpathlineto{\pgfqpoint{1.700938in}{0.564573in}}%
\pgfpathlineto{\pgfqpoint{1.702012in}{0.566314in}}%
\pgfpathlineto{\pgfqpoint{1.705233in}{0.566153in}}%
\pgfpathlineto{\pgfqpoint{1.706307in}{0.563283in}}%
\pgfpathlineto{\pgfqpoint{1.707381in}{0.566604in}}%
\pgfpathlineto{\pgfqpoint{1.709528in}{0.560929in}}%
\pgfpathlineto{\pgfqpoint{1.712750in}{0.560413in}}%
\pgfpathlineto{\pgfqpoint{1.714898in}{0.557317in}}%
\pgfpathlineto{\pgfqpoint{1.717045in}{0.561735in}}%
\pgfpathlineto{\pgfqpoint{1.720267in}{0.563154in}}%
\pgfpathlineto{\pgfqpoint{1.721341in}{0.567346in}}%
\pgfpathlineto{\pgfqpoint{1.722415in}{0.565508in}}%
\pgfpathlineto{\pgfqpoint{1.723488in}{0.568088in}}%
\pgfpathlineto{\pgfqpoint{1.724562in}{0.567475in}}%
\pgfpathlineto{\pgfqpoint{1.727784in}{0.567411in}}%
\pgfpathlineto{\pgfqpoint{1.728858in}{0.569958in}}%
\pgfpathlineto{\pgfqpoint{1.729931in}{0.568378in}}%
\pgfpathlineto{\pgfqpoint{1.731005in}{0.585276in}}%
\pgfpathlineto{\pgfqpoint{1.732079in}{0.588953in}}%
\pgfpathlineto{\pgfqpoint{1.735301in}{0.589017in}}%
\pgfpathlineto{\pgfqpoint{1.736374in}{0.590114in}}%
\pgfpathlineto{\pgfqpoint{1.737448in}{0.595177in}}%
\pgfpathlineto{\pgfqpoint{1.738522in}{0.595596in}}%
\pgfpathlineto{\pgfqpoint{1.739596in}{0.597402in}}%
\pgfpathlineto{\pgfqpoint{1.743891in}{0.595306in}}%
\pgfpathlineto{\pgfqpoint{1.744965in}{0.598531in}}%
\pgfpathlineto{\pgfqpoint{1.746039in}{0.597725in}}%
\pgfpathlineto{\pgfqpoint{1.747113in}{0.596112in}}%
\pgfpathlineto{\pgfqpoint{1.752482in}{0.596918in}}%
\pgfpathlineto{\pgfqpoint{1.754630in}{0.600272in}}%
\pgfpathlineto{\pgfqpoint{1.757851in}{0.600595in}}%
\pgfpathlineto{\pgfqpoint{1.758925in}{0.602626in}}%
\pgfpathlineto{\pgfqpoint{1.759999in}{0.602626in}}%
\pgfpathlineto{\pgfqpoint{1.762147in}{0.603336in}}%
\pgfpathlineto{\pgfqpoint{1.765368in}{0.603336in}}%
\pgfpathlineto{\pgfqpoint{1.767516in}{0.605980in}}%
\pgfpathlineto{\pgfqpoint{1.768590in}{0.605626in}}%
\pgfpathlineto{\pgfqpoint{1.769663in}{0.607335in}}%
\pgfpathlineto{\pgfqpoint{1.772885in}{0.607141in}}%
\pgfpathlineto{\pgfqpoint{1.773959in}{0.607947in}}%
\pgfpathlineto{\pgfqpoint{1.775033in}{0.606077in}}%
\pgfpathlineto{\pgfqpoint{1.776106in}{0.607173in}}%
\pgfpathlineto{\pgfqpoint{1.777180in}{0.602272in}}%
\pgfpathlineto{\pgfqpoint{1.780402in}{0.602207in}}%
\pgfpathlineto{\pgfqpoint{1.781476in}{0.599659in}}%
\pgfpathlineto{\pgfqpoint{1.783623in}{0.607980in}}%
\pgfpathlineto{\pgfqpoint{1.784697in}{0.606045in}}%
\pgfpathlineto{\pgfqpoint{1.788993in}{0.608818in}}%
\pgfpathlineto{\pgfqpoint{1.790066in}{0.610656in}}%
\pgfpathlineto{\pgfqpoint{1.796509in}{0.608270in}}%
\pgfpathlineto{\pgfqpoint{1.797583in}{0.606464in}}%
\pgfpathlineto{\pgfqpoint{1.799731in}{0.608625in}}%
\pgfpathlineto{\pgfqpoint{1.804026in}{0.602820in}}%
\pgfpathlineto{\pgfqpoint{1.806174in}{0.608205in}}%
\pgfpathlineto{\pgfqpoint{1.807248in}{0.605174in}}%
\pgfpathlineto{\pgfqpoint{1.810469in}{0.604755in}}%
\pgfpathlineto{\pgfqpoint{1.811543in}{0.605368in}}%
\pgfpathlineto{\pgfqpoint{1.812617in}{0.601304in}}%
\pgfpathlineto{\pgfqpoint{1.813691in}{0.599434in}}%
\pgfpathlineto{\pgfqpoint{1.814765in}{0.600853in}}%
\pgfpathlineto{\pgfqpoint{1.822282in}{0.603433in}}%
\pgfpathlineto{\pgfqpoint{1.825503in}{0.602078in}}%
\pgfpathlineto{\pgfqpoint{1.827651in}{0.594177in}}%
\pgfpathlineto{\pgfqpoint{1.828725in}{0.595467in}}%
\pgfpathlineto{\pgfqpoint{1.829798in}{0.600820in}}%
\pgfpathlineto{\pgfqpoint{1.833020in}{0.601143in}}%
\pgfpathlineto{\pgfqpoint{1.835168in}{0.608528in}}%
\pgfpathlineto{\pgfqpoint{1.836241in}{0.613881in}}%
\pgfpathlineto{\pgfqpoint{1.837315in}{0.610495in}}%
\pgfpathlineto{\pgfqpoint{1.841611in}{0.608270in}}%
\pgfpathlineto{\pgfqpoint{1.843758in}{0.614720in}}%
\pgfpathlineto{\pgfqpoint{1.844832in}{0.613720in}}%
\pgfpathlineto{\pgfqpoint{1.849127in}{0.614687in}}%
\pgfpathlineto{\pgfqpoint{1.850201in}{0.613333in}}%
\pgfpathlineto{\pgfqpoint{1.851275in}{0.613333in}}%
\pgfpathlineto{\pgfqpoint{1.852349in}{0.616364in}}%
\pgfpathlineto{\pgfqpoint{1.857718in}{0.616364in}}%
\pgfpathlineto{\pgfqpoint{1.858792in}{0.616945in}}%
\pgfpathlineto{\pgfqpoint{1.859866in}{0.615042in}}%
\pgfpathlineto{\pgfqpoint{1.863087in}{0.620492in}}%
\pgfpathlineto{\pgfqpoint{1.865235in}{0.616945in}}%
\pgfpathlineto{\pgfqpoint{1.866309in}{0.617235in}}%
\pgfpathlineto{\pgfqpoint{1.867383in}{0.613494in}}%
\pgfpathlineto{\pgfqpoint{1.870604in}{0.615139in}}%
\pgfpathlineto{\pgfqpoint{1.871678in}{0.610269in}}%
\pgfpathlineto{\pgfqpoint{1.872752in}{0.609915in}}%
\pgfpathlineto{\pgfqpoint{1.873826in}{0.613688in}}%
\pgfpathlineto{\pgfqpoint{1.874900in}{0.610140in}}%
\pgfpathlineto{\pgfqpoint{1.878121in}{0.613268in}}%
\pgfpathlineto{\pgfqpoint{1.879195in}{0.609721in}}%
\pgfpathlineto{\pgfqpoint{1.880269in}{0.612204in}}%
\pgfpathlineto{\pgfqpoint{1.881343in}{0.611882in}}%
\pgfpathlineto{\pgfqpoint{1.882416in}{0.613752in}}%
\pgfpathlineto{\pgfqpoint{1.886712in}{0.612785in}}%
\pgfpathlineto{\pgfqpoint{1.887786in}{0.608496in}}%
\pgfpathlineto{\pgfqpoint{1.889933in}{0.607915in}}%
\pgfpathlineto{\pgfqpoint{1.893155in}{0.608270in}}%
\pgfpathlineto{\pgfqpoint{1.895303in}{0.606786in}}%
\pgfpathlineto{\pgfqpoint{1.896376in}{0.607141in}}%
\pgfpathlineto{\pgfqpoint{1.900672in}{0.606851in}}%
\pgfpathlineto{\pgfqpoint{1.902819in}{0.610914in}}%
\pgfpathlineto{\pgfqpoint{1.904967in}{0.610398in}}%
\pgfpathlineto{\pgfqpoint{1.909262in}{0.607883in}}%
\pgfpathlineto{\pgfqpoint{1.911410in}{0.608238in}}%
\pgfpathlineto{\pgfqpoint{1.912484in}{0.604723in}}%
\pgfpathlineto{\pgfqpoint{1.915705in}{0.605335in}}%
\pgfpathlineto{\pgfqpoint{1.916779in}{0.607302in}}%
\pgfpathlineto{\pgfqpoint{1.917853in}{0.615623in}}%
\pgfpathlineto{\pgfqpoint{1.920001in}{0.613946in}}%
\pgfpathlineto{\pgfqpoint{1.923222in}{0.612785in}}%
\pgfpathlineto{\pgfqpoint{1.924296in}{0.611753in}}%
\pgfpathlineto{\pgfqpoint{1.925370in}{0.613526in}}%
\pgfpathlineto{\pgfqpoint{1.926444in}{0.609495in}}%
\pgfpathlineto{\pgfqpoint{1.927518in}{0.608625in}}%
\pgfpathlineto{\pgfqpoint{1.930739in}{0.608044in}}%
\pgfpathlineto{\pgfqpoint{1.931813in}{0.609205in}}%
\pgfpathlineto{\pgfqpoint{1.932887in}{0.608302in}}%
\pgfpathlineto{\pgfqpoint{1.933961in}{0.611140in}}%
\pgfpathlineto{\pgfqpoint{1.935035in}{0.620170in}}%
\pgfpathlineto{\pgfqpoint{1.940404in}{0.617977in}}%
\pgfpathlineto{\pgfqpoint{1.941478in}{0.622201in}}%
\pgfpathlineto{\pgfqpoint{1.942551in}{0.620847in}}%
\pgfpathlineto{\pgfqpoint{1.946847in}{0.622717in}}%
\pgfpathlineto{\pgfqpoint{1.948994in}{0.620234in}}%
\pgfpathlineto{\pgfqpoint{1.950068in}{0.621008in}}%
\pgfpathlineto{\pgfqpoint{1.954364in}{0.617654in}}%
\pgfpathlineto{\pgfqpoint{1.955437in}{0.620621in}}%
\pgfpathlineto{\pgfqpoint{1.956511in}{0.620815in}}%
\pgfpathlineto{\pgfqpoint{1.957585in}{0.618074in}}%
\pgfpathlineto{\pgfqpoint{1.960807in}{0.619428in}}%
\pgfpathlineto{\pgfqpoint{1.962954in}{0.618912in}}%
\pgfpathlineto{\pgfqpoint{1.964028in}{0.616622in}}%
\pgfpathlineto{\pgfqpoint{1.965102in}{0.617106in}}%
\pgfpathlineto{\pgfqpoint{1.968324in}{0.614945in}}%
\pgfpathlineto{\pgfqpoint{1.969397in}{0.615784in}}%
\pgfpathlineto{\pgfqpoint{1.970471in}{0.621040in}}%
\pgfpathlineto{\pgfqpoint{1.971545in}{0.621073in}}%
\pgfpathlineto{\pgfqpoint{1.975840in}{0.617751in}}%
\pgfpathlineto{\pgfqpoint{1.976914in}{0.619106in}}%
\pgfpathlineto{\pgfqpoint{1.977988in}{0.618299in}}%
\pgfpathlineto{\pgfqpoint{1.979062in}{0.620653in}}%
\pgfpathlineto{\pgfqpoint{1.980136in}{0.618106in}}%
\pgfpathlineto{\pgfqpoint{1.983357in}{0.619364in}}%
\pgfpathlineto{\pgfqpoint{1.984431in}{0.620395in}}%
\pgfpathlineto{\pgfqpoint{1.986579in}{0.617880in}}%
\pgfpathlineto{\pgfqpoint{1.987653in}{0.618299in}}%
\pgfpathlineto{\pgfqpoint{1.990874in}{0.611946in}}%
\pgfpathlineto{\pgfqpoint{1.994096in}{0.616719in}}%
\pgfpathlineto{\pgfqpoint{1.998391in}{0.616332in}}%
\pgfpathlineto{\pgfqpoint{1.999465in}{0.615236in}}%
\pgfpathlineto{\pgfqpoint{2.000539in}{0.611979in}}%
\pgfpathlineto{\pgfqpoint{2.001613in}{0.612978in}}%
\pgfpathlineto{\pgfqpoint{2.002686in}{0.617267in}}%
\pgfpathlineto{\pgfqpoint{2.010203in}{0.624943in}}%
\pgfpathlineto{\pgfqpoint{2.013425in}{0.630651in}}%
\pgfpathlineto{\pgfqpoint{2.014499in}{0.628522in}}%
\pgfpathlineto{\pgfqpoint{2.016646in}{0.627684in}}%
\pgfpathlineto{\pgfqpoint{2.017720in}{0.637197in}}%
\pgfpathlineto{\pgfqpoint{2.020942in}{0.634263in}}%
\pgfpathlineto{\pgfqpoint{2.024163in}{0.642131in}}%
\pgfpathlineto{\pgfqpoint{2.025237in}{0.638906in}}%
\pgfpathlineto{\pgfqpoint{2.028459in}{0.640229in}}%
\pgfpathlineto{\pgfqpoint{2.030606in}{0.637778in}}%
\pgfpathlineto{\pgfqpoint{2.031680in}{0.633392in}}%
\pgfpathlineto{\pgfqpoint{2.032754in}{0.635359in}}%
\pgfpathlineto{\pgfqpoint{2.035975in}{0.635907in}}%
\pgfpathlineto{\pgfqpoint{2.037049in}{0.632715in}}%
\pgfpathlineto{\pgfqpoint{2.040271in}{0.635778in}}%
\pgfpathlineto{\pgfqpoint{2.045640in}{0.636326in}}%
\pgfpathlineto{\pgfqpoint{2.046714in}{0.634940in}}%
\pgfpathlineto{\pgfqpoint{2.047788in}{0.626297in}}%
\pgfpathlineto{\pgfqpoint{2.052083in}{0.613075in}}%
\pgfpathlineto{\pgfqpoint{2.053157in}{0.624717in}}%
\pgfpathlineto{\pgfqpoint{2.054231in}{0.630070in}}%
\pgfpathlineto{\pgfqpoint{2.055304in}{0.630264in}}%
\pgfpathlineto{\pgfqpoint{2.058526in}{0.626620in}}%
\pgfpathlineto{\pgfqpoint{2.059600in}{0.619299in}}%
\pgfpathlineto{\pgfqpoint{2.061747in}{0.623814in}}%
\pgfpathlineto{\pgfqpoint{2.062821in}{0.619944in}}%
\pgfpathlineto{\pgfqpoint{2.067117in}{0.624233in}}%
\pgfpathlineto{\pgfqpoint{2.068191in}{0.621298in}}%
\pgfpathlineto{\pgfqpoint{2.070338in}{0.624943in}}%
\pgfpathlineto{\pgfqpoint{2.073560in}{0.622588in}}%
\pgfpathlineto{\pgfqpoint{2.075707in}{0.625652in}}%
\pgfpathlineto{\pgfqpoint{2.076781in}{0.625555in}}%
\pgfpathlineto{\pgfqpoint{2.077855in}{0.621911in}}%
\pgfpathlineto{\pgfqpoint{2.081077in}{0.625007in}}%
\pgfpathlineto{\pgfqpoint{2.082150in}{0.623330in}}%
\pgfpathlineto{\pgfqpoint{2.083224in}{0.625523in}}%
\pgfpathlineto{\pgfqpoint{2.084298in}{0.623298in}}%
\pgfpathlineto{\pgfqpoint{2.085372in}{0.624717in}}%
\pgfpathlineto{\pgfqpoint{2.088593in}{0.613784in}}%
\pgfpathlineto{\pgfqpoint{2.090741in}{0.621492in}}%
\pgfpathlineto{\pgfqpoint{2.091815in}{0.622524in}}%
\pgfpathlineto{\pgfqpoint{2.092889in}{0.624652in}}%
\pgfpathlineto{\pgfqpoint{2.096110in}{0.629554in}}%
\pgfpathlineto{\pgfqpoint{2.097184in}{0.629070in}}%
\pgfpathlineto{\pgfqpoint{2.099332in}{0.634779in}}%
\pgfpathlineto{\pgfqpoint{2.100406in}{0.635036in}}%
\pgfpathlineto{\pgfqpoint{2.103627in}{0.638197in}}%
\pgfpathlineto{\pgfqpoint{2.104701in}{0.638229in}}%
\pgfpathlineto{\pgfqpoint{2.105775in}{0.635714in}}%
\pgfpathlineto{\pgfqpoint{2.107923in}{0.641357in}}%
\pgfpathlineto{\pgfqpoint{2.111144in}{0.644453in}}%
\pgfpathlineto{\pgfqpoint{2.113292in}{0.639648in}}%
\pgfpathlineto{\pgfqpoint{2.115439in}{0.644679in}}%
\pgfpathlineto{\pgfqpoint{2.118661in}{0.648162in}}%
\pgfpathlineto{\pgfqpoint{2.119735in}{0.646098in}}%
\pgfpathlineto{\pgfqpoint{2.120809in}{0.650322in}}%
\pgfpathlineto{\pgfqpoint{2.121882in}{0.649194in}}%
\pgfpathlineto{\pgfqpoint{2.126178in}{0.638906in}}%
\pgfpathlineto{\pgfqpoint{2.127252in}{0.647291in}}%
\pgfpathlineto{\pgfqpoint{2.128325in}{0.648710in}}%
\pgfpathlineto{\pgfqpoint{2.129399in}{0.651548in}}%
\pgfpathlineto{\pgfqpoint{2.130473in}{0.649935in}}%
\pgfpathlineto{\pgfqpoint{2.133695in}{0.647743in}}%
\pgfpathlineto{\pgfqpoint{2.134769in}{0.652677in}}%
\pgfpathlineto{\pgfqpoint{2.135842in}{0.651741in}}%
\pgfpathlineto{\pgfqpoint{2.136916in}{0.649000in}}%
\pgfpathlineto{\pgfqpoint{2.137990in}{0.648388in}}%
\pgfpathlineto{\pgfqpoint{2.141212in}{0.650871in}}%
\pgfpathlineto{\pgfqpoint{2.142285in}{0.650645in}}%
\pgfpathlineto{\pgfqpoint{2.143359in}{0.655740in}}%
\pgfpathlineto{\pgfqpoint{2.144433in}{0.654773in}}%
\pgfpathlineto{\pgfqpoint{2.145507in}{0.654902in}}%
\pgfpathlineto{\pgfqpoint{2.148728in}{0.654644in}}%
\pgfpathlineto{\pgfqpoint{2.150876in}{0.652935in}}%
\pgfpathlineto{\pgfqpoint{2.153024in}{0.653805in}}%
\pgfpathlineto{\pgfqpoint{2.156245in}{0.651193in}}%
\pgfpathlineto{\pgfqpoint{2.157319in}{0.654031in}}%
\pgfpathlineto{\pgfqpoint{2.159467in}{0.648936in}}%
\pgfpathlineto{\pgfqpoint{2.160541in}{0.655579in}}%
\pgfpathlineto{\pgfqpoint{2.164836in}{0.651387in}}%
\pgfpathlineto{\pgfqpoint{2.165910in}{0.647678in}}%
\pgfpathlineto{\pgfqpoint{2.166984in}{0.648420in}}%
\pgfpathlineto{\pgfqpoint{2.168058in}{0.642131in}}%
\pgfpathlineto{\pgfqpoint{2.171279in}{0.644485in}}%
\pgfpathlineto{\pgfqpoint{2.173427in}{0.653870in}}%
\pgfpathlineto{\pgfqpoint{2.174501in}{0.650193in}}%
\pgfpathlineto{\pgfqpoint{2.175574in}{0.642776in}}%
\pgfpathlineto{\pgfqpoint{2.179870in}{0.646098in}}%
\pgfpathlineto{\pgfqpoint{2.180944in}{0.649806in}}%
\pgfpathlineto{\pgfqpoint{2.182017in}{0.648871in}}%
\pgfpathlineto{\pgfqpoint{2.186313in}{0.649710in}}%
\pgfpathlineto{\pgfqpoint{2.187387in}{0.651838in}}%
\pgfpathlineto{\pgfqpoint{2.189534in}{0.646646in}}%
\pgfpathlineto{\pgfqpoint{2.193830in}{0.640841in}}%
\pgfpathlineto{\pgfqpoint{2.194903in}{0.642615in}}%
\pgfpathlineto{\pgfqpoint{2.198125in}{0.631973in}}%
\pgfpathlineto{\pgfqpoint{2.201347in}{0.635262in}}%
\pgfpathlineto{\pgfqpoint{2.202420in}{0.637874in}}%
\pgfpathlineto{\pgfqpoint{2.203494in}{0.632682in}}%
\pgfpathlineto{\pgfqpoint{2.204568in}{0.634875in}}%
\pgfpathlineto{\pgfqpoint{2.205642in}{0.628683in}}%
\pgfpathlineto{\pgfqpoint{2.209937in}{0.627329in}}%
\pgfpathlineto{\pgfqpoint{2.211011in}{0.625104in}}%
\pgfpathlineto{\pgfqpoint{2.213159in}{0.631425in}}%
\pgfpathlineto{\pgfqpoint{2.216380in}{0.628425in}}%
\pgfpathlineto{\pgfqpoint{2.217454in}{0.628845in}}%
\pgfpathlineto{\pgfqpoint{2.218528in}{0.625813in}}%
\pgfpathlineto{\pgfqpoint{2.219602in}{0.620847in}}%
\pgfpathlineto{\pgfqpoint{2.220676in}{0.637036in}}%
\pgfpathlineto{\pgfqpoint{2.223897in}{0.636681in}}%
\pgfpathlineto{\pgfqpoint{2.224971in}{0.633618in}}%
\pgfpathlineto{\pgfqpoint{2.226045in}{0.636681in}}%
\pgfpathlineto{\pgfqpoint{2.227119in}{0.634488in}}%
\pgfpathlineto{\pgfqpoint{2.228192in}{0.627780in}}%
\pgfpathlineto{\pgfqpoint{2.231414in}{0.615977in}}%
\pgfpathlineto{\pgfqpoint{2.232488in}{0.617719in}}%
\pgfpathlineto{\pgfqpoint{2.233562in}{0.623878in}}%
\pgfpathlineto{\pgfqpoint{2.234635in}{0.618590in}}%
\pgfpathlineto{\pgfqpoint{2.235709in}{0.624717in}}%
\pgfpathlineto{\pgfqpoint{2.240005in}{0.626878in}}%
\pgfpathlineto{\pgfqpoint{2.241079in}{0.630231in}}%
\pgfpathlineto{\pgfqpoint{2.242152in}{0.627748in}}%
\pgfpathlineto{\pgfqpoint{2.243226in}{0.628651in}}%
\pgfpathlineto{\pgfqpoint{2.246448in}{0.633489in}}%
\pgfpathlineto{\pgfqpoint{2.247522in}{0.630618in}}%
\pgfpathlineto{\pgfqpoint{2.248595in}{0.629651in}}%
\pgfpathlineto{\pgfqpoint{2.249669in}{0.634134in}}%
\pgfpathlineto{\pgfqpoint{2.250743in}{0.632424in}}%
\pgfpathlineto{\pgfqpoint{2.253965in}{0.631360in}}%
\pgfpathlineto{\pgfqpoint{2.255038in}{0.638455in}}%
\pgfpathlineto{\pgfqpoint{2.257186in}{0.636165in}}%
\pgfpathlineto{\pgfqpoint{2.258260in}{0.636133in}}%
\pgfpathlineto{\pgfqpoint{2.261481in}{0.629941in}}%
\pgfpathlineto{\pgfqpoint{2.262555in}{0.625781in}}%
\pgfpathlineto{\pgfqpoint{2.263629in}{0.625975in}}%
\pgfpathlineto{\pgfqpoint{2.264703in}{0.624523in}}%
\pgfpathlineto{\pgfqpoint{2.265777in}{0.628974in}}%
\pgfpathlineto{\pgfqpoint{2.268998in}{0.628522in}}%
\pgfpathlineto{\pgfqpoint{2.271146in}{0.631296in}}%
\pgfpathlineto{\pgfqpoint{2.273294in}{0.635746in}}%
\pgfpathlineto{\pgfqpoint{2.276515in}{0.635714in}}%
\pgfpathlineto{\pgfqpoint{2.277589in}{0.633166in}}%
\pgfpathlineto{\pgfqpoint{2.278663in}{0.636165in}}%
\pgfpathlineto{\pgfqpoint{2.279737in}{0.636875in}}%
\pgfpathlineto{\pgfqpoint{2.284032in}{0.636681in}}%
\pgfpathlineto{\pgfqpoint{2.286180in}{0.645195in}}%
\pgfpathlineto{\pgfqpoint{2.287254in}{0.644227in}}%
\pgfpathlineto{\pgfqpoint{2.288327in}{0.647743in}}%
\pgfpathlineto{\pgfqpoint{2.291549in}{0.648484in}}%
\pgfpathlineto{\pgfqpoint{2.292623in}{0.645775in}}%
\pgfpathlineto{\pgfqpoint{2.293697in}{0.649677in}}%
\pgfpathlineto{\pgfqpoint{2.294770in}{0.647710in}}%
\pgfpathlineto{\pgfqpoint{2.295844in}{0.649129in}}%
\pgfpathlineto{\pgfqpoint{2.299066in}{0.648452in}}%
\pgfpathlineto{\pgfqpoint{2.300140in}{0.650677in}}%
\pgfpathlineto{\pgfqpoint{2.302287in}{0.656353in}}%
\pgfpathlineto{\pgfqpoint{2.303361in}{0.655579in}}%
\pgfpathlineto{\pgfqpoint{2.306583in}{0.659900in}}%
\pgfpathlineto{\pgfqpoint{2.307657in}{0.657675in}}%
\pgfpathlineto{\pgfqpoint{2.308730in}{0.658933in}}%
\pgfpathlineto{\pgfqpoint{2.309804in}{0.657804in}}%
\pgfpathlineto{\pgfqpoint{2.310878in}{0.652515in}}%
\pgfpathlineto{\pgfqpoint{2.314100in}{0.649484in}}%
\pgfpathlineto{\pgfqpoint{2.316247in}{0.651419in}}%
\pgfpathlineto{\pgfqpoint{2.317321in}{0.648033in}}%
\pgfpathlineto{\pgfqpoint{2.318395in}{0.646646in}}%
\pgfpathlineto{\pgfqpoint{2.321616in}{0.650484in}}%
\pgfpathlineto{\pgfqpoint{2.322690in}{0.646485in}}%
\pgfpathlineto{\pgfqpoint{2.323764in}{0.646098in}}%
\pgfpathlineto{\pgfqpoint{2.325912in}{0.648162in}}%
\pgfpathlineto{\pgfqpoint{2.329133in}{0.649710in}}%
\pgfpathlineto{\pgfqpoint{2.330207in}{0.652806in}}%
\pgfpathlineto{\pgfqpoint{2.331281in}{0.647259in}}%
\pgfpathlineto{\pgfqpoint{2.332355in}{0.649129in}}%
\pgfpathlineto{\pgfqpoint{2.333429in}{0.645775in}}%
\pgfpathlineto{\pgfqpoint{2.336650in}{0.648839in}}%
\pgfpathlineto{\pgfqpoint{2.337724in}{0.645646in}}%
\pgfpathlineto{\pgfqpoint{2.338798in}{0.647678in}}%
\pgfpathlineto{\pgfqpoint{2.339872in}{0.645937in}}%
\pgfpathlineto{\pgfqpoint{2.340946in}{0.648420in}}%
\pgfpathlineto{\pgfqpoint{2.344167in}{0.646969in}}%
\pgfpathlineto{\pgfqpoint{2.345241in}{0.653805in}}%
\pgfpathlineto{\pgfqpoint{2.346315in}{0.652806in}}%
\pgfpathlineto{\pgfqpoint{2.347389in}{0.652612in}}%
\pgfpathlineto{\pgfqpoint{2.348462in}{0.654708in}}%
\pgfpathlineto{\pgfqpoint{2.352758in}{0.652419in}}%
\pgfpathlineto{\pgfqpoint{2.353832in}{0.653322in}}%
\pgfpathlineto{\pgfqpoint{2.354905in}{0.655611in}}%
\pgfpathlineto{\pgfqpoint{2.355979in}{0.655579in}}%
\pgfpathlineto{\pgfqpoint{2.360275in}{0.657675in}}%
\pgfpathlineto{\pgfqpoint{2.361348in}{0.660965in}}%
\pgfpathlineto{\pgfqpoint{2.362422in}{0.659739in}}%
\pgfpathlineto{\pgfqpoint{2.363496in}{0.656353in}}%
\pgfpathlineto{\pgfqpoint{2.366718in}{0.650709in}}%
\pgfpathlineto{\pgfqpoint{2.367791in}{0.651258in}}%
\pgfpathlineto{\pgfqpoint{2.368865in}{0.650000in}}%
\pgfpathlineto{\pgfqpoint{2.369939in}{0.650580in}}%
\pgfpathlineto{\pgfqpoint{2.371013in}{0.646291in}}%
\pgfpathlineto{\pgfqpoint{2.375308in}{0.647356in}}%
\pgfpathlineto{\pgfqpoint{2.376382in}{0.644840in}}%
\pgfpathlineto{\pgfqpoint{2.377456in}{0.650193in}}%
\pgfpathlineto{\pgfqpoint{2.378530in}{0.640164in}}%
\pgfpathlineto{\pgfqpoint{2.381751in}{0.634779in}}%
\pgfpathlineto{\pgfqpoint{2.383899in}{0.645485in}}%
\pgfpathlineto{\pgfqpoint{2.384973in}{0.637391in}}%
\pgfpathlineto{\pgfqpoint{2.386047in}{0.638358in}}%
\pgfpathlineto{\pgfqpoint{2.390342in}{0.638971in}}%
\pgfpathlineto{\pgfqpoint{2.391416in}{0.637036in}}%
\pgfpathlineto{\pgfqpoint{2.392490in}{0.638455in}}%
\pgfpathlineto{\pgfqpoint{2.393564in}{0.644485in}}%
\pgfpathlineto{\pgfqpoint{2.396785in}{0.644808in}}%
\pgfpathlineto{\pgfqpoint{2.397859in}{0.647807in}}%
\pgfpathlineto{\pgfqpoint{2.398933in}{0.647775in}}%
\pgfpathlineto{\pgfqpoint{2.400007in}{0.649903in}}%
\pgfpathlineto{\pgfqpoint{2.401080in}{0.650419in}}%
\pgfpathlineto{\pgfqpoint{2.404302in}{0.650451in}}%
\pgfpathlineto{\pgfqpoint{2.406450in}{0.653741in}}%
\pgfpathlineto{\pgfqpoint{2.407524in}{0.651967in}}%
\pgfpathlineto{\pgfqpoint{2.408597in}{0.655482in}}%
\pgfpathlineto{\pgfqpoint{2.413967in}{0.651129in}}%
\pgfpathlineto{\pgfqpoint{2.415040in}{0.653225in}}%
\pgfpathlineto{\pgfqpoint{2.416114in}{0.649613in}}%
\pgfpathlineto{\pgfqpoint{2.420410in}{0.650355in}}%
\pgfpathlineto{\pgfqpoint{2.421483in}{0.651709in}}%
\pgfpathlineto{\pgfqpoint{2.423631in}{0.656224in}}%
\pgfpathlineto{\pgfqpoint{2.429000in}{0.654773in}}%
\pgfpathlineto{\pgfqpoint{2.430074in}{0.656160in}}%
\pgfpathlineto{\pgfqpoint{2.431148in}{0.655837in}}%
\pgfpathlineto{\pgfqpoint{2.434369in}{0.658643in}}%
\pgfpathlineto{\pgfqpoint{2.435443in}{0.658320in}}%
\pgfpathlineto{\pgfqpoint{2.436517in}{0.658868in}}%
\pgfpathlineto{\pgfqpoint{2.437591in}{0.657062in}}%
\pgfpathlineto{\pgfqpoint{2.441886in}{0.659546in}}%
\pgfpathlineto{\pgfqpoint{2.442960in}{0.658739in}}%
\pgfpathlineto{\pgfqpoint{2.444034in}{0.657127in}}%
\pgfpathlineto{\pgfqpoint{2.445108in}{0.657224in}}%
\pgfpathlineto{\pgfqpoint{2.446182in}{0.657998in}}%
\pgfpathlineto{\pgfqpoint{2.449403in}{0.658965in}}%
\pgfpathlineto{\pgfqpoint{2.450477in}{0.659900in}}%
\pgfpathlineto{\pgfqpoint{2.451551in}{0.659062in}}%
\pgfpathlineto{\pgfqpoint{2.452625in}{0.660320in}}%
\pgfpathlineto{\pgfqpoint{2.453699in}{0.662513in}}%
\pgfpathlineto{\pgfqpoint{2.457994in}{0.664189in}}%
\pgfpathlineto{\pgfqpoint{2.459068in}{0.666447in}}%
\pgfpathlineto{\pgfqpoint{2.460142in}{0.665576in}}%
\pgfpathlineto{\pgfqpoint{2.461215in}{0.660416in}}%
\pgfpathlineto{\pgfqpoint{2.464437in}{0.665576in}}%
\pgfpathlineto{\pgfqpoint{2.465511in}{0.662190in}}%
\pgfpathlineto{\pgfqpoint{2.466585in}{0.660900in}}%
\pgfpathlineto{\pgfqpoint{2.467658in}{0.662545in}}%
\pgfpathlineto{\pgfqpoint{2.468732in}{0.662738in}}%
\pgfpathlineto{\pgfqpoint{2.473028in}{0.664028in}}%
\pgfpathlineto{\pgfqpoint{2.474101in}{0.666382in}}%
\pgfpathlineto{\pgfqpoint{2.475175in}{0.666834in}}%
\pgfpathlineto{\pgfqpoint{2.476249in}{0.664222in}}%
\pgfpathlineto{\pgfqpoint{2.479471in}{0.661932in}}%
\pgfpathlineto{\pgfqpoint{2.480545in}{0.663029in}}%
\pgfpathlineto{\pgfqpoint{2.481618in}{0.665576in}}%
\pgfpathlineto{\pgfqpoint{2.482692in}{0.662287in}}%
\pgfpathlineto{\pgfqpoint{2.483766in}{0.664738in}}%
\pgfpathlineto{\pgfqpoint{2.486988in}{0.665254in}}%
\pgfpathlineto{\pgfqpoint{2.488061in}{0.664834in}}%
\pgfpathlineto{\pgfqpoint{2.489135in}{0.666802in}}%
\pgfpathlineto{\pgfqpoint{2.490209in}{0.666834in}}%
\pgfpathlineto{\pgfqpoint{2.491283in}{0.665318in}}%
\pgfpathlineto{\pgfqpoint{2.494504in}{0.666060in}}%
\pgfpathlineto{\pgfqpoint{2.495578in}{0.662642in}}%
\pgfpathlineto{\pgfqpoint{2.496652in}{0.663319in}}%
\pgfpathlineto{\pgfqpoint{2.497726in}{0.662126in}}%
\pgfpathlineto{\pgfqpoint{2.498800in}{0.663964in}}%
\pgfpathlineto{\pgfqpoint{2.502021in}{0.662996in}}%
\pgfpathlineto{\pgfqpoint{2.503095in}{0.661190in}}%
\pgfpathlineto{\pgfqpoint{2.504169in}{0.665092in}}%
\pgfpathlineto{\pgfqpoint{2.506317in}{0.663641in}}%
\pgfpathlineto{\pgfqpoint{2.509538in}{0.666221in}}%
\pgfpathlineto{\pgfqpoint{2.510612in}{0.662609in}}%
\pgfpathlineto{\pgfqpoint{2.511686in}{0.661739in}}%
\pgfpathlineto{\pgfqpoint{2.517055in}{0.664125in}}%
\pgfpathlineto{\pgfqpoint{2.519203in}{0.658030in}}%
\pgfpathlineto{\pgfqpoint{2.520277in}{0.658256in}}%
\pgfpathlineto{\pgfqpoint{2.521350in}{0.657353in}}%
\pgfpathlineto{\pgfqpoint{2.525646in}{0.665318in}}%
\pgfpathlineto{\pgfqpoint{2.526720in}{0.666447in}}%
\pgfpathlineto{\pgfqpoint{2.527793in}{0.662126in}}%
\pgfpathlineto{\pgfqpoint{2.528867in}{0.662158in}}%
\pgfpathlineto{\pgfqpoint{2.532089in}{0.651096in}}%
\pgfpathlineto{\pgfqpoint{2.533163in}{0.651677in}}%
\pgfpathlineto{\pgfqpoint{2.535310in}{0.660126in}}%
\pgfpathlineto{\pgfqpoint{2.536384in}{0.659352in}}%
\pgfpathlineto{\pgfqpoint{2.539606in}{0.662093in}}%
\pgfpathlineto{\pgfqpoint{2.540679in}{0.656514in}}%
\pgfpathlineto{\pgfqpoint{2.541753in}{0.655386in}}%
\pgfpathlineto{\pgfqpoint{2.543901in}{0.657159in}}%
\pgfpathlineto{\pgfqpoint{2.547123in}{0.653838in}}%
\pgfpathlineto{\pgfqpoint{2.548196in}{0.654063in}}%
\pgfpathlineto{\pgfqpoint{2.550344in}{0.642260in}}%
\pgfpathlineto{\pgfqpoint{2.551418in}{0.643195in}}%
\pgfpathlineto{\pgfqpoint{2.554639in}{0.648226in}}%
\pgfpathlineto{\pgfqpoint{2.555713in}{0.647581in}}%
\pgfpathlineto{\pgfqpoint{2.556787in}{0.654547in}}%
\pgfpathlineto{\pgfqpoint{2.558935in}{0.654031in}}%
\pgfpathlineto{\pgfqpoint{2.562156in}{0.651999in}}%
\pgfpathlineto{\pgfqpoint{2.563230in}{0.654192in}}%
\pgfpathlineto{\pgfqpoint{2.564304in}{0.653999in}}%
\pgfpathlineto{\pgfqpoint{2.565378in}{0.655160in}}%
\pgfpathlineto{\pgfqpoint{2.566452in}{0.651516in}}%
\pgfpathlineto{\pgfqpoint{2.569673in}{0.650709in}}%
\pgfpathlineto{\pgfqpoint{2.570747in}{0.651548in}}%
\pgfpathlineto{\pgfqpoint{2.572895in}{0.650097in}}%
\pgfpathlineto{\pgfqpoint{2.573968in}{0.650871in}}%
\pgfpathlineto{\pgfqpoint{2.580412in}{0.651451in}}%
\pgfpathlineto{\pgfqpoint{2.581485in}{0.650484in}}%
\pgfpathlineto{\pgfqpoint{2.585781in}{0.655160in}}%
\pgfpathlineto{\pgfqpoint{2.587928in}{0.660191in}}%
\pgfpathlineto{\pgfqpoint{2.589002in}{0.663738in}}%
\pgfpathlineto{\pgfqpoint{2.592224in}{0.662287in}}%
\pgfpathlineto{\pgfqpoint{2.593298in}{0.660900in}}%
\pgfpathlineto{\pgfqpoint{2.594371in}{0.662448in}}%
\pgfpathlineto{\pgfqpoint{2.596519in}{0.660449in}}%
\pgfpathlineto{\pgfqpoint{2.600814in}{0.660739in}}%
\pgfpathlineto{\pgfqpoint{2.602962in}{0.662222in}}%
\pgfpathlineto{\pgfqpoint{2.607257in}{0.663545in}}%
\pgfpathlineto{\pgfqpoint{2.609405in}{0.669091in}}%
\pgfpathlineto{\pgfqpoint{2.610479in}{0.666995in}}%
\pgfpathlineto{\pgfqpoint{2.611553in}{0.668672in}}%
\pgfpathlineto{\pgfqpoint{2.614774in}{0.668446in}}%
\pgfpathlineto{\pgfqpoint{2.615848in}{0.665318in}}%
\pgfpathlineto{\pgfqpoint{2.617996in}{0.664028in}}%
\pgfpathlineto{\pgfqpoint{2.619070in}{0.675993in}}%
\pgfpathlineto{\pgfqpoint{2.623365in}{0.675025in}}%
\pgfpathlineto{\pgfqpoint{2.624439in}{0.672864in}}%
\pgfpathlineto{\pgfqpoint{2.626587in}{0.675412in}}%
\pgfpathlineto{\pgfqpoint{2.629808in}{0.677121in}}%
\pgfpathlineto{\pgfqpoint{2.630882in}{0.678444in}}%
\pgfpathlineto{\pgfqpoint{2.631956in}{0.681152in}}%
\pgfpathlineto{\pgfqpoint{2.634103in}{0.680894in}}%
\pgfpathlineto{\pgfqpoint{2.638399in}{0.682346in}}%
\pgfpathlineto{\pgfqpoint{2.639473in}{0.681959in}}%
\pgfpathlineto{\pgfqpoint{2.641620in}{0.683958in}}%
\pgfpathlineto{\pgfqpoint{2.645916in}{0.682410in}}%
\pgfpathlineto{\pgfqpoint{2.646989in}{0.685732in}}%
\pgfpathlineto{\pgfqpoint{2.648063in}{0.684281in}}%
\pgfpathlineto{\pgfqpoint{2.649137in}{0.685119in}}%
\pgfpathlineto{\pgfqpoint{2.655580in}{0.686119in}}%
\pgfpathlineto{\pgfqpoint{2.656654in}{0.688086in}}%
\pgfpathlineto{\pgfqpoint{2.659876in}{0.689279in}}%
\pgfpathlineto{\pgfqpoint{2.660949in}{0.687505in}}%
\pgfpathlineto{\pgfqpoint{2.663097in}{0.689086in}}%
\pgfpathlineto{\pgfqpoint{2.664171in}{0.689698in}}%
\pgfpathlineto{\pgfqpoint{2.667392in}{0.686344in}}%
\pgfpathlineto{\pgfqpoint{2.668466in}{0.682797in}}%
\pgfpathlineto{\pgfqpoint{2.671688in}{0.686377in}}%
\pgfpathlineto{\pgfqpoint{2.674909in}{0.685603in}}%
\pgfpathlineto{\pgfqpoint{2.677057in}{0.686344in}}%
\pgfpathlineto{\pgfqpoint{2.679205in}{0.685345in}}%
\pgfpathlineto{\pgfqpoint{2.682426in}{0.686925in}}%
\pgfpathlineto{\pgfqpoint{2.683500in}{0.685087in}}%
\pgfpathlineto{\pgfqpoint{2.685648in}{0.686054in}}%
\pgfpathlineto{\pgfqpoint{2.686722in}{0.684958in}}%
\pgfpathlineto{\pgfqpoint{2.691017in}{0.685280in}}%
\pgfpathlineto{\pgfqpoint{2.692091in}{0.684764in}}%
\pgfpathlineto{\pgfqpoint{2.693165in}{0.685345in}}%
\pgfpathlineto{\pgfqpoint{2.697460in}{0.688344in}}%
\pgfpathlineto{\pgfqpoint{2.699608in}{0.688150in}}%
\pgfpathlineto{\pgfqpoint{2.700681in}{0.692601in}}%
\pgfpathlineto{\pgfqpoint{2.701755in}{0.692601in}}%
\pgfpathlineto{\pgfqpoint{2.706051in}{0.695632in}}%
\pgfpathlineto{\pgfqpoint{2.709272in}{0.692827in}}%
\pgfpathlineto{\pgfqpoint{2.712494in}{0.692956in}}%
\pgfpathlineto{\pgfqpoint{2.713567in}{0.696987in}}%
\pgfpathlineto{\pgfqpoint{2.714641in}{0.696696in}}%
\pgfpathlineto{\pgfqpoint{2.715715in}{0.697245in}}%
\pgfpathlineto{\pgfqpoint{2.716789in}{0.695568in}}%
\pgfpathlineto{\pgfqpoint{2.721084in}{0.695277in}}%
\pgfpathlineto{\pgfqpoint{2.722158in}{0.696084in}}%
\pgfpathlineto{\pgfqpoint{2.723232in}{0.695600in}}%
\pgfpathlineto{\pgfqpoint{2.724306in}{0.697599in}}%
\pgfpathlineto{\pgfqpoint{2.727527in}{0.699180in}}%
\pgfpathlineto{\pgfqpoint{2.728601in}{0.698922in}}%
\pgfpathlineto{\pgfqpoint{2.729675in}{0.695052in}}%
\pgfpathlineto{\pgfqpoint{2.730749in}{0.694890in}}%
\pgfpathlineto{\pgfqpoint{2.731823in}{0.697341in}}%
\pgfpathlineto{\pgfqpoint{2.735044in}{0.699954in}}%
\pgfpathlineto{\pgfqpoint{2.736118in}{0.701727in}}%
\pgfpathlineto{\pgfqpoint{2.737192in}{0.704726in}}%
\pgfpathlineto{\pgfqpoint{2.738266in}{0.705468in}}%
\pgfpathlineto{\pgfqpoint{2.739340in}{0.704275in}}%
\pgfpathlineto{\pgfqpoint{2.743635in}{0.704436in}}%
\pgfpathlineto{\pgfqpoint{2.744709in}{0.706081in}}%
\pgfpathlineto{\pgfqpoint{2.745783in}{0.706597in}}%
\pgfpathlineto{\pgfqpoint{2.746856in}{0.708983in}}%
\pgfpathlineto{\pgfqpoint{2.750078in}{0.710273in}}%
\pgfpathlineto{\pgfqpoint{2.751152in}{0.707855in}}%
\pgfpathlineto{\pgfqpoint{2.752226in}{0.708790in}}%
\pgfpathlineto{\pgfqpoint{2.753300in}{0.708790in}}%
\pgfpathlineto{\pgfqpoint{2.754373in}{0.703952in}}%
\pgfpathlineto{\pgfqpoint{2.757595in}{0.700566in}}%
\pgfpathlineto{\pgfqpoint{2.758669in}{0.705597in}}%
\pgfpathlineto{\pgfqpoint{2.759743in}{0.706371in}}%
\pgfpathlineto{\pgfqpoint{2.760816in}{0.702695in}}%
\pgfpathlineto{\pgfqpoint{2.761890in}{0.702695in}}%
\pgfpathlineto{\pgfqpoint{2.765112in}{0.704662in}}%
\pgfpathlineto{\pgfqpoint{2.766186in}{0.703372in}}%
\pgfpathlineto{\pgfqpoint{2.767259in}{0.703888in}}%
\pgfpathlineto{\pgfqpoint{2.768333in}{0.702017in}}%
\pgfpathlineto{\pgfqpoint{2.769407in}{0.707177in}}%
\pgfpathlineto{\pgfqpoint{2.772629in}{0.706049in}}%
\pgfpathlineto{\pgfqpoint{2.773702in}{0.705017in}}%
\pgfpathlineto{\pgfqpoint{2.774776in}{0.709241in}}%
\pgfpathlineto{\pgfqpoint{2.775850in}{0.703501in}}%
\pgfpathlineto{\pgfqpoint{2.776924in}{0.701469in}}%
\pgfpathlineto{\pgfqpoint{2.780145in}{0.700083in}}%
\pgfpathlineto{\pgfqpoint{2.782293in}{0.702179in}}%
\pgfpathlineto{\pgfqpoint{2.783367in}{0.699792in}}%
\pgfpathlineto{\pgfqpoint{2.784441in}{0.701921in}}%
\pgfpathlineto{\pgfqpoint{2.788736in}{0.706629in}}%
\pgfpathlineto{\pgfqpoint{2.789810in}{0.709080in}}%
\pgfpathlineto{\pgfqpoint{2.790884in}{0.708338in}}%
\pgfpathlineto{\pgfqpoint{2.791958in}{0.711466in}}%
\pgfpathlineto{\pgfqpoint{2.795179in}{0.711144in}}%
\pgfpathlineto{\pgfqpoint{2.797327in}{0.715659in}}%
\pgfpathlineto{\pgfqpoint{2.798401in}{0.715207in}}%
\pgfpathlineto{\pgfqpoint{2.799475in}{0.719948in}}%
\pgfpathlineto{\pgfqpoint{2.802696in}{0.722399in}}%
\pgfpathlineto{\pgfqpoint{2.803770in}{0.721206in}}%
\pgfpathlineto{\pgfqpoint{2.804844in}{0.723914in}}%
\pgfpathlineto{\pgfqpoint{2.805918in}{0.719851in}}%
\pgfpathlineto{\pgfqpoint{2.806991in}{0.718529in}}%
\pgfpathlineto{\pgfqpoint{2.810213in}{0.719819in}}%
\pgfpathlineto{\pgfqpoint{2.811287in}{0.723979in}}%
\pgfpathlineto{\pgfqpoint{2.812361in}{0.725301in}}%
\pgfpathlineto{\pgfqpoint{2.813434in}{0.723108in}}%
\pgfpathlineto{\pgfqpoint{2.814508in}{0.724043in}}%
\pgfpathlineto{\pgfqpoint{2.817730in}{0.725946in}}%
\pgfpathlineto{\pgfqpoint{2.819877in}{0.724301in}}%
\pgfpathlineto{\pgfqpoint{2.820951in}{0.719593in}}%
\pgfpathlineto{\pgfqpoint{2.827394in}{0.732299in}}%
\pgfpathlineto{\pgfqpoint{2.828468in}{0.728623in}}%
\pgfpathlineto{\pgfqpoint{2.829542in}{0.729719in}}%
\pgfpathlineto{\pgfqpoint{2.832764in}{0.731977in}}%
\pgfpathlineto{\pgfqpoint{2.833837in}{0.734686in}}%
\pgfpathlineto{\pgfqpoint{2.834911in}{0.731848in}}%
\pgfpathlineto{\pgfqpoint{2.837059in}{0.732396in}}%
\pgfpathlineto{\pgfqpoint{2.841354in}{0.733750in}}%
\pgfpathlineto{\pgfqpoint{2.843502in}{0.732944in}}%
\pgfpathlineto{\pgfqpoint{2.844576in}{0.734137in}}%
\pgfpathlineto{\pgfqpoint{2.848871in}{0.731332in}}%
\pgfpathlineto{\pgfqpoint{2.849945in}{0.731848in}}%
\pgfpathlineto{\pgfqpoint{2.851019in}{0.736266in}}%
\pgfpathlineto{\pgfqpoint{2.852093in}{0.735847in}}%
\pgfpathlineto{\pgfqpoint{2.855314in}{0.741297in}}%
\pgfpathlineto{\pgfqpoint{2.856388in}{0.741490in}}%
\pgfpathlineto{\pgfqpoint{2.857462in}{0.740297in}}%
\pgfpathlineto{\pgfqpoint{2.858536in}{0.741103in}}%
\pgfpathlineto{\pgfqpoint{2.859610in}{0.738620in}}%
\pgfpathlineto{\pgfqpoint{2.862831in}{0.737104in}}%
\pgfpathlineto{\pgfqpoint{2.863905in}{0.738942in}}%
\pgfpathlineto{\pgfqpoint{2.864979in}{0.737556in}}%
\pgfpathlineto{\pgfqpoint{2.867126in}{0.739426in}}%
\pgfpathlineto{\pgfqpoint{2.870348in}{0.731364in}}%
\pgfpathlineto{\pgfqpoint{2.871422in}{0.731106in}}%
\pgfpathlineto{\pgfqpoint{2.873569in}{0.736330in}}%
\pgfpathlineto{\pgfqpoint{2.874643in}{0.738426in}}%
\pgfpathlineto{\pgfqpoint{2.878939in}{0.739523in}}%
\pgfpathlineto{\pgfqpoint{2.880012in}{0.738652in}}%
\pgfpathlineto{\pgfqpoint{2.882160in}{0.743167in}}%
\pgfpathlineto{\pgfqpoint{2.885382in}{0.744038in}}%
\pgfpathlineto{\pgfqpoint{2.886455in}{0.745005in}}%
\pgfpathlineto{\pgfqpoint{2.887529in}{0.748617in}}%
\pgfpathlineto{\pgfqpoint{2.888603in}{0.747553in}}%
\pgfpathlineto{\pgfqpoint{2.889677in}{0.749294in}}%
\pgfpathlineto{\pgfqpoint{2.892899in}{0.748166in}}%
\pgfpathlineto{\pgfqpoint{2.893972in}{0.745747in}}%
\pgfpathlineto{\pgfqpoint{2.895046in}{0.746553in}}%
\pgfpathlineto{\pgfqpoint{2.896120in}{0.744070in}}%
\pgfpathlineto{\pgfqpoint{2.897194in}{0.745779in}}%
\pgfpathlineto{\pgfqpoint{2.900415in}{0.745715in}}%
\pgfpathlineto{\pgfqpoint{2.903637in}{0.752938in}}%
\pgfpathlineto{\pgfqpoint{2.904711in}{0.752648in}}%
\pgfpathlineto{\pgfqpoint{2.909006in}{0.753487in}}%
\pgfpathlineto{\pgfqpoint{2.910080in}{0.756970in}}%
\pgfpathlineto{\pgfqpoint{2.911154in}{0.756679in}}%
\pgfpathlineto{\pgfqpoint{2.912228in}{0.757905in}}%
\pgfpathlineto{\pgfqpoint{2.917597in}{0.761420in}}%
\pgfpathlineto{\pgfqpoint{2.919744in}{0.759549in}}%
\pgfpathlineto{\pgfqpoint{2.922966in}{0.758034in}}%
\pgfpathlineto{\pgfqpoint{2.924040in}{0.759872in}}%
\pgfpathlineto{\pgfqpoint{2.925114in}{0.754357in}}%
\pgfpathlineto{\pgfqpoint{2.926188in}{0.757453in}}%
\pgfpathlineto{\pgfqpoint{2.927261in}{0.753616in}}%
\pgfpathlineto{\pgfqpoint{2.930483in}{0.754035in}}%
\pgfpathlineto{\pgfqpoint{2.931557in}{0.758808in}}%
\pgfpathlineto{\pgfqpoint{2.932631in}{0.756808in}}%
\pgfpathlineto{\pgfqpoint{2.934778in}{0.760452in}}%
\pgfpathlineto{\pgfqpoint{2.938000in}{0.761775in}}%
\pgfpathlineto{\pgfqpoint{2.939074in}{0.764903in}}%
\pgfpathlineto{\pgfqpoint{2.940147in}{0.753745in}}%
\pgfpathlineto{\pgfqpoint{2.941221in}{0.762452in}}%
\pgfpathlineto{\pgfqpoint{2.942295in}{0.756518in}}%
\pgfpathlineto{\pgfqpoint{2.945517in}{0.746005in}}%
\pgfpathlineto{\pgfqpoint{2.949812in}{0.762484in}}%
\pgfpathlineto{\pgfqpoint{2.953033in}{0.761742in}}%
\pgfpathlineto{\pgfqpoint{2.954107in}{0.765225in}}%
\pgfpathlineto{\pgfqpoint{2.955181in}{0.764742in}}%
\pgfpathlineto{\pgfqpoint{2.956255in}{0.763484in}}%
\pgfpathlineto{\pgfqpoint{2.957329in}{0.766354in}}%
\pgfpathlineto{\pgfqpoint{2.960550in}{0.765516in}}%
\pgfpathlineto{\pgfqpoint{2.961624in}{0.761001in}}%
\pgfpathlineto{\pgfqpoint{2.962698in}{0.760936in}}%
\pgfpathlineto{\pgfqpoint{2.964846in}{0.762774in}}%
\pgfpathlineto{\pgfqpoint{2.969141in}{0.763710in}}%
\pgfpathlineto{\pgfqpoint{2.970215in}{0.766999in}}%
\pgfpathlineto{\pgfqpoint{2.971289in}{0.768063in}}%
\pgfpathlineto{\pgfqpoint{2.972363in}{0.766999in}}%
\pgfpathlineto{\pgfqpoint{2.972363in}{0.766999in}}%
\pgfusepath{stroke}%
\end{pgfscope}%
\begin{pgfscope}%
\pgfpathrectangle{\pgfqpoint{0.506453in}{0.385400in}}{\pgfqpoint{2.583333in}{0.400885in}}%
\pgfusepath{clip}%
\pgfsetroundcap%
\pgfsetroundjoin%
\pgfsetlinewidth{1.505625pt}%
\definecolor{currentstroke}{rgb}{0.737255,0.741176,0.133333}%
\pgfsetstrokecolor{currentstroke}%
\pgfsetdash{}{0pt}%
\pgfpathmoveto{\pgfqpoint{0.623878in}{0.476114in}}%
\pgfpathlineto{\pgfqpoint{0.626025in}{0.474276in}}%
\pgfpathlineto{\pgfqpoint{0.627099in}{0.473450in}}%
\pgfpathlineto{\pgfqpoint{0.632468in}{0.472339in}}%
\pgfpathlineto{\pgfqpoint{0.633542in}{0.470814in}}%
\pgfpathlineto{\pgfqpoint{0.638911in}{0.469234in}}%
\pgfpathlineto{\pgfqpoint{0.639985in}{0.470055in}}%
\pgfpathlineto{\pgfqpoint{0.642133in}{0.468003in}}%
\pgfpathlineto{\pgfqpoint{0.645354in}{0.467358in}}%
\pgfpathlineto{\pgfqpoint{0.647502in}{0.466163in}}%
\pgfpathlineto{\pgfqpoint{0.649650in}{0.466477in}}%
\pgfpathlineto{\pgfqpoint{0.652871in}{0.465736in}}%
\pgfpathlineto{\pgfqpoint{0.653945in}{0.466220in}}%
\pgfpathlineto{\pgfqpoint{0.655019in}{0.465052in}}%
\pgfpathlineto{\pgfqpoint{0.656093in}{0.462910in}}%
\pgfpathlineto{\pgfqpoint{0.657167in}{0.463448in}}%
\pgfpathlineto{\pgfqpoint{0.662536in}{0.464191in}}%
\pgfpathlineto{\pgfqpoint{0.663610in}{0.461911in}}%
\pgfpathlineto{\pgfqpoint{0.664684in}{0.462671in}}%
\pgfpathlineto{\pgfqpoint{0.667905in}{0.462053in}}%
\pgfpathlineto{\pgfqpoint{0.668979in}{0.460747in}}%
\pgfpathlineto{\pgfqpoint{0.670053in}{0.461133in}}%
\pgfpathlineto{\pgfqpoint{0.671127in}{0.460474in}}%
\pgfpathlineto{\pgfqpoint{0.672200in}{0.461087in}}%
\pgfpathlineto{\pgfqpoint{0.676496in}{0.460678in}}%
\pgfpathlineto{\pgfqpoint{0.677570in}{0.459588in}}%
\pgfpathlineto{\pgfqpoint{0.682939in}{0.459828in}}%
\pgfpathlineto{\pgfqpoint{0.685087in}{0.457619in}}%
\pgfpathlineto{\pgfqpoint{0.686160in}{0.457999in}}%
\pgfpathlineto{\pgfqpoint{0.687234in}{0.457513in}}%
\pgfpathlineto{\pgfqpoint{0.690456in}{0.457556in}}%
\pgfpathlineto{\pgfqpoint{0.691530in}{0.456901in}}%
\pgfpathlineto{\pgfqpoint{0.692603in}{0.457281in}}%
\pgfpathlineto{\pgfqpoint{0.694751in}{0.455840in}}%
\pgfpathlineto{\pgfqpoint{0.705489in}{0.454684in}}%
\pgfpathlineto{\pgfqpoint{0.706563in}{0.453687in}}%
\pgfpathlineto{\pgfqpoint{0.708711in}{0.454019in}}%
\pgfpathlineto{\pgfqpoint{0.709785in}{0.454664in}}%
\pgfpathlineto{\pgfqpoint{0.714080in}{0.455114in}}%
\pgfpathlineto{\pgfqpoint{0.717302in}{0.454332in}}%
\pgfpathlineto{\pgfqpoint{0.722671in}{0.454742in}}%
\pgfpathlineto{\pgfqpoint{0.723745in}{0.453862in}}%
\pgfpathlineto{\pgfqpoint{0.728040in}{0.453239in}}%
\pgfpathlineto{\pgfqpoint{0.729114in}{0.452087in}}%
\pgfpathlineto{\pgfqpoint{0.730188in}{0.452351in}}%
\pgfpathlineto{\pgfqpoint{0.732335in}{0.450005in}}%
\pgfpathlineto{\pgfqpoint{0.735557in}{0.449131in}}%
\pgfpathlineto{\pgfqpoint{0.737705in}{0.449491in}}%
\pgfpathlineto{\pgfqpoint{0.744148in}{0.448411in}}%
\pgfpathlineto{\pgfqpoint{0.745221in}{0.447330in}}%
\pgfpathlineto{\pgfqpoint{0.747369in}{0.447951in}}%
\pgfpathlineto{\pgfqpoint{0.752738in}{0.447477in}}%
\pgfpathlineto{\pgfqpoint{0.753812in}{0.445402in}}%
\pgfpathlineto{\pgfqpoint{0.754886in}{0.445892in}}%
\pgfpathlineto{\pgfqpoint{0.759181in}{0.446023in}}%
\pgfpathlineto{\pgfqpoint{0.761329in}{0.446023in}}%
\pgfpathlineto{\pgfqpoint{0.766698in}{0.445468in}}%
\pgfpathlineto{\pgfqpoint{0.774215in}{0.441107in}}%
\pgfpathlineto{\pgfqpoint{0.777437in}{0.441422in}}%
\pgfpathlineto{\pgfqpoint{0.781732in}{0.441708in}}%
\pgfpathlineto{\pgfqpoint{0.784953in}{0.439160in}}%
\pgfpathlineto{\pgfqpoint{0.795692in}{0.437659in}}%
\pgfpathlineto{\pgfqpoint{0.796766in}{0.437897in}}%
\pgfpathlineto{\pgfqpoint{0.797840in}{0.437236in}}%
\pgfpathlineto{\pgfqpoint{0.798913in}{0.437699in}}%
\pgfpathlineto{\pgfqpoint{0.799987in}{0.437143in}}%
\pgfpathlineto{\pgfqpoint{0.805356in}{0.438334in}}%
\pgfpathlineto{\pgfqpoint{0.807504in}{0.435888in}}%
\pgfpathlineto{\pgfqpoint{0.810726in}{0.434930in}}%
\pgfpathlineto{\pgfqpoint{0.812873in}{0.435491in}}%
\pgfpathlineto{\pgfqpoint{0.815021in}{0.434451in}}%
\pgfpathlineto{\pgfqpoint{0.819316in}{0.433642in}}%
\pgfpathlineto{\pgfqpoint{0.822538in}{0.433384in}}%
\pgfpathlineto{\pgfqpoint{0.828981in}{0.432760in}}%
\pgfpathlineto{\pgfqpoint{0.830055in}{0.433104in}}%
\pgfpathlineto{\pgfqpoint{0.834350in}{0.432619in}}%
\pgfpathlineto{\pgfqpoint{0.840793in}{0.431759in}}%
\pgfpathlineto{\pgfqpoint{0.845088in}{0.429710in}}%
\pgfpathlineto{\pgfqpoint{0.849384in}{0.429691in}}%
\pgfpathlineto{\pgfqpoint{0.850458in}{0.429371in}}%
\pgfpathlineto{\pgfqpoint{0.852605in}{0.430085in}}%
\pgfpathlineto{\pgfqpoint{0.860122in}{0.429701in}}%
\pgfpathlineto{\pgfqpoint{0.865491in}{0.429874in}}%
\pgfpathlineto{\pgfqpoint{0.867639in}{0.429792in}}%
\pgfpathlineto{\pgfqpoint{0.882673in}{0.429710in}}%
\pgfpathlineto{\pgfqpoint{0.893411in}{0.429792in}}%
\pgfpathlineto{\pgfqpoint{0.894485in}{0.429216in}}%
\pgfpathlineto{\pgfqpoint{0.896633in}{0.429899in}}%
\pgfpathlineto{\pgfqpoint{0.902002in}{0.429645in}}%
\pgfpathlineto{\pgfqpoint{0.905223in}{0.429908in}}%
\pgfpathlineto{\pgfqpoint{0.925626in}{0.428491in}}%
\pgfpathlineto{\pgfqpoint{0.927774in}{0.428800in}}%
\pgfpathlineto{\pgfqpoint{0.938512in}{0.428776in}}%
\pgfpathlineto{\pgfqpoint{0.940660in}{0.428320in}}%
\pgfpathlineto{\pgfqpoint{0.942808in}{0.428654in}}%
\pgfpathlineto{\pgfqpoint{0.948177in}{0.428735in}}%
\pgfpathlineto{\pgfqpoint{0.949251in}{0.427807in}}%
\pgfpathlineto{\pgfqpoint{0.965358in}{0.426141in}}%
\pgfpathlineto{\pgfqpoint{0.970728in}{0.425832in}}%
\pgfpathlineto{\pgfqpoint{0.976097in}{0.425920in}}%
\pgfpathlineto{\pgfqpoint{0.979318in}{0.426062in}}%
\pgfpathlineto{\pgfqpoint{0.980392in}{0.426298in}}%
\pgfpathlineto{\pgfqpoint{1.013681in}{0.424077in}}%
\pgfpathlineto{\pgfqpoint{1.016903in}{0.423576in}}%
\pgfpathlineto{\pgfqpoint{1.022272in}{0.424074in}}%
\pgfpathlineto{\pgfqpoint{1.025493in}{0.423676in}}%
\pgfpathlineto{\pgfqpoint{1.045896in}{0.422909in}}%
\pgfpathlineto{\pgfqpoint{1.048044in}{0.422628in}}%
\pgfpathlineto{\pgfqpoint{1.063078in}{0.422024in}}%
\pgfpathlineto{\pgfqpoint{1.082407in}{0.422297in}}%
\pgfpathlineto{\pgfqpoint{1.088850in}{0.422401in}}%
\pgfpathlineto{\pgfqpoint{1.093145in}{0.422091in}}%
\pgfpathlineto{\pgfqpoint{1.097441in}{0.421791in}}%
\pgfpathlineto{\pgfqpoint{1.100662in}{0.420969in}}%
\pgfpathlineto{\pgfqpoint{1.107105in}{0.420417in}}%
\pgfpathlineto{\pgfqpoint{1.121065in}{0.420132in}}%
\pgfpathlineto{\pgfqpoint{1.123213in}{0.420009in}}%
\pgfpathlineto{\pgfqpoint{1.135025in}{0.418819in}}%
\pgfpathlineto{\pgfqpoint{1.138246in}{0.418966in}}%
\pgfpathlineto{\pgfqpoint{1.143616in}{0.418852in}}%
\pgfpathlineto{\pgfqpoint{1.145763in}{0.417638in}}%
\pgfpathlineto{\pgfqpoint{1.160797in}{0.416964in}}%
\pgfpathlineto{\pgfqpoint{1.179052in}{0.416226in}}%
\pgfpathlineto{\pgfqpoint{1.198381in}{0.415744in}}%
\pgfpathlineto{\pgfqpoint{1.205898in}{0.415968in}}%
\pgfpathlineto{\pgfqpoint{1.213415in}{0.416004in}}%
\pgfpathlineto{\pgfqpoint{1.228449in}{0.415517in}}%
\pgfpathlineto{\pgfqpoint{1.233818in}{0.415315in}}%
\pgfpathlineto{\pgfqpoint{1.235966in}{0.414730in}}%
\pgfpathlineto{\pgfqpoint{1.240261in}{0.414628in}}%
\pgfpathlineto{\pgfqpoint{1.241335in}{0.413786in}}%
\pgfpathlineto{\pgfqpoint{1.243483in}{0.413628in}}%
\pgfpathlineto{\pgfqpoint{1.256369in}{0.413389in}}%
\pgfpathlineto{\pgfqpoint{1.258516in}{0.413098in}}%
\pgfpathlineto{\pgfqpoint{1.266033in}{0.412850in}}%
\pgfpathlineto{\pgfqpoint{1.273550in}{0.412605in}}%
\pgfpathlineto{\pgfqpoint{1.287510in}{0.412530in}}%
\pgfpathlineto{\pgfqpoint{1.288584in}{0.412337in}}%
\pgfpathlineto{\pgfqpoint{1.311134in}{0.412036in}}%
\pgfpathlineto{\pgfqpoint{1.333685in}{0.411137in}}%
\pgfpathlineto{\pgfqpoint{1.382008in}{0.410646in}}%
\pgfpathlineto{\pgfqpoint{1.397041in}{0.410121in}}%
\pgfpathlineto{\pgfqpoint{1.421740in}{0.410064in}}%
\pgfpathlineto{\pgfqpoint{1.453955in}{0.408771in}}%
\pgfpathlineto{\pgfqpoint{1.499056in}{0.408265in}}%
\pgfpathlineto{\pgfqpoint{1.559191in}{0.407122in}}%
\pgfpathlineto{\pgfqpoint{1.609661in}{0.407070in}}%
\pgfpathlineto{\pgfqpoint{1.702012in}{0.406763in}}%
\pgfpathlineto{\pgfqpoint{1.729931in}{0.406235in}}%
\pgfpathlineto{\pgfqpoint{1.732079in}{0.405900in}}%
\pgfpathlineto{\pgfqpoint{1.836241in}{0.405402in}}%
\pgfpathlineto{\pgfqpoint{1.856644in}{0.405350in}}%
\pgfpathlineto{\pgfqpoint{1.916779in}{0.405070in}}%
\pgfpathlineto{\pgfqpoint{1.927518in}{0.404953in}}%
\pgfpathlineto{\pgfqpoint{2.166984in}{0.404088in}}%
\pgfpathlineto{\pgfqpoint{2.211011in}{0.403970in}}%
\pgfpathlineto{\pgfqpoint{2.273294in}{0.403789in}}%
\pgfpathlineto{\pgfqpoint{2.972363in}{0.403627in}}%
\pgfpathlineto{\pgfqpoint{2.972363in}{0.403627in}}%
\pgfusepath{stroke}%
\end{pgfscope}%
\begin{pgfscope}%
\pgfsetrectcap%
\pgfsetmiterjoin%
\pgfsetlinewidth{0.803000pt}%
\definecolor{currentstroke}{rgb}{1.000000,1.000000,1.000000}%
\pgfsetstrokecolor{currentstroke}%
\pgfsetdash{}{0pt}%
\pgfpathmoveto{\pgfqpoint{0.506453in}{0.385400in}}%
\pgfpathlineto{\pgfqpoint{0.506453in}{0.786285in}}%
\pgfusepath{stroke}%
\end{pgfscope}%
\begin{pgfscope}%
\pgfsetrectcap%
\pgfsetmiterjoin%
\pgfsetlinewidth{0.803000pt}%
\definecolor{currentstroke}{rgb}{1.000000,1.000000,1.000000}%
\pgfsetstrokecolor{currentstroke}%
\pgfsetdash{}{0pt}%
\pgfpathmoveto{\pgfqpoint{3.089787in}{0.385400in}}%
\pgfpathlineto{\pgfqpoint{3.089787in}{0.786285in}}%
\pgfusepath{stroke}%
\end{pgfscope}%
\begin{pgfscope}%
\pgfsetrectcap%
\pgfsetmiterjoin%
\pgfsetlinewidth{0.803000pt}%
\definecolor{currentstroke}{rgb}{1.000000,1.000000,1.000000}%
\pgfsetstrokecolor{currentstroke}%
\pgfsetdash{}{0pt}%
\pgfpathmoveto{\pgfqpoint{0.506453in}{0.385400in}}%
\pgfpathlineto{\pgfqpoint{3.089787in}{0.385400in}}%
\pgfusepath{stroke}%
\end{pgfscope}%
\begin{pgfscope}%
\pgfsetrectcap%
\pgfsetmiterjoin%
\pgfsetlinewidth{0.803000pt}%
\definecolor{currentstroke}{rgb}{1.000000,1.000000,1.000000}%
\pgfsetstrokecolor{currentstroke}%
\pgfsetdash{}{0pt}%
\pgfpathmoveto{\pgfqpoint{0.506453in}{0.786285in}}%
\pgfpathlineto{\pgfqpoint{3.089787in}{0.786285in}}%
\pgfusepath{stroke}%
\end{pgfscope}%
\begin{pgfscope}%
\definecolor{textcolor}{rgb}{0.150000,0.150000,0.150000}%
\pgfsetstrokecolor{textcolor}%
\pgfsetfillcolor{textcolor}%
\pgftext[x=1.798120in,y=0.869619in,,base]{\color{textcolor}\rmfamily\fontsize{16.800000}{20.160000}\selectfont V}%
\end{pgfscope}%
\begin{pgfscope}%
\pgfsetbuttcap%
\pgfsetmiterjoin%
\definecolor{currentfill}{rgb}{0.917647,0.917647,0.949020}%
\pgfsetfillcolor{currentfill}%
\pgfsetlinewidth{0.000000pt}%
\definecolor{currentstroke}{rgb}{0.000000,0.000000,0.000000}%
\pgfsetstrokecolor{currentstroke}%
\pgfsetstrokeopacity{0.000000}%
\pgfsetdash{}{0pt}%
\pgfpathmoveto{\pgfqpoint{4.123120in}{0.385400in}}%
\pgfpathlineto{\pgfqpoint{6.706453in}{0.385400in}}%
\pgfpathlineto{\pgfqpoint{6.706453in}{0.786285in}}%
\pgfpathlineto{\pgfqpoint{4.123120in}{0.786285in}}%
\pgfpathclose%
\pgfusepath{fill}%
\end{pgfscope}%
\begin{pgfscope}%
\pgfpathrectangle{\pgfqpoint{4.123120in}{0.385400in}}{\pgfqpoint{2.583333in}{0.400885in}}%
\pgfusepath{clip}%
\pgfsetroundcap%
\pgfsetroundjoin%
\pgfsetlinewidth{0.803000pt}%
\definecolor{currentstroke}{rgb}{1.000000,1.000000,1.000000}%
\pgfsetstrokecolor{currentstroke}%
\pgfsetdash{}{0pt}%
\pgfpathmoveto{\pgfqpoint{4.238397in}{0.385400in}}%
\pgfpathlineto{\pgfqpoint{4.238397in}{0.786285in}}%
\pgfusepath{stroke}%
\end{pgfscope}%
\begin{pgfscope}%
\definecolor{textcolor}{rgb}{0.150000,0.150000,0.150000}%
\pgfsetstrokecolor{textcolor}%
\pgfsetfillcolor{textcolor}%
\pgftext[x=4.238397in,y=0.288178in,,top]{\color{textcolor}\rmfamily\fontsize{14.000000}{16.800000}\selectfont 2012}%
\end{pgfscope}%
\begin{pgfscope}%
\pgfpathrectangle{\pgfqpoint{4.123120in}{0.385400in}}{\pgfqpoint{2.583333in}{0.400885in}}%
\pgfusepath{clip}%
\pgfsetroundcap%
\pgfsetroundjoin%
\pgfsetlinewidth{0.803000pt}%
\definecolor{currentstroke}{rgb}{1.000000,1.000000,1.000000}%
\pgfsetstrokecolor{currentstroke}%
\pgfsetdash{}{0pt}%
\pgfpathmoveto{\pgfqpoint{4.631422in}{0.385400in}}%
\pgfpathlineto{\pgfqpoint{4.631422in}{0.786285in}}%
\pgfusepath{stroke}%
\end{pgfscope}%
\begin{pgfscope}%
\definecolor{textcolor}{rgb}{0.150000,0.150000,0.150000}%
\pgfsetstrokecolor{textcolor}%
\pgfsetfillcolor{textcolor}%
\pgftext[x=4.631422in,y=0.288178in,,top]{\color{textcolor}\rmfamily\fontsize{14.000000}{16.800000}\selectfont 2013}%
\end{pgfscope}%
\begin{pgfscope}%
\pgfpathrectangle{\pgfqpoint{4.123120in}{0.385400in}}{\pgfqpoint{2.583333in}{0.400885in}}%
\pgfusepath{clip}%
\pgfsetroundcap%
\pgfsetroundjoin%
\pgfsetlinewidth{0.803000pt}%
\definecolor{currentstroke}{rgb}{1.000000,1.000000,1.000000}%
\pgfsetstrokecolor{currentstroke}%
\pgfsetdash{}{0pt}%
\pgfpathmoveto{\pgfqpoint{5.023373in}{0.385400in}}%
\pgfpathlineto{\pgfqpoint{5.023373in}{0.786285in}}%
\pgfusepath{stroke}%
\end{pgfscope}%
\begin{pgfscope}%
\definecolor{textcolor}{rgb}{0.150000,0.150000,0.150000}%
\pgfsetstrokecolor{textcolor}%
\pgfsetfillcolor{textcolor}%
\pgftext[x=5.023373in,y=0.288178in,,top]{\color{textcolor}\rmfamily\fontsize{14.000000}{16.800000}\selectfont 2014}%
\end{pgfscope}%
\begin{pgfscope}%
\pgfpathrectangle{\pgfqpoint{4.123120in}{0.385400in}}{\pgfqpoint{2.583333in}{0.400885in}}%
\pgfusepath{clip}%
\pgfsetroundcap%
\pgfsetroundjoin%
\pgfsetlinewidth{0.803000pt}%
\definecolor{currentstroke}{rgb}{1.000000,1.000000,1.000000}%
\pgfsetstrokecolor{currentstroke}%
\pgfsetdash{}{0pt}%
\pgfpathmoveto{\pgfqpoint{5.415324in}{0.385400in}}%
\pgfpathlineto{\pgfqpoint{5.415324in}{0.786285in}}%
\pgfusepath{stroke}%
\end{pgfscope}%
\begin{pgfscope}%
\definecolor{textcolor}{rgb}{0.150000,0.150000,0.150000}%
\pgfsetstrokecolor{textcolor}%
\pgfsetfillcolor{textcolor}%
\pgftext[x=5.415324in,y=0.288178in,,top]{\color{textcolor}\rmfamily\fontsize{14.000000}{16.800000}\selectfont 2015}%
\end{pgfscope}%
\begin{pgfscope}%
\pgfpathrectangle{\pgfqpoint{4.123120in}{0.385400in}}{\pgfqpoint{2.583333in}{0.400885in}}%
\pgfusepath{clip}%
\pgfsetroundcap%
\pgfsetroundjoin%
\pgfsetlinewidth{0.803000pt}%
\definecolor{currentstroke}{rgb}{1.000000,1.000000,1.000000}%
\pgfsetstrokecolor{currentstroke}%
\pgfsetdash{}{0pt}%
\pgfpathmoveto{\pgfqpoint{5.807275in}{0.385400in}}%
\pgfpathlineto{\pgfqpoint{5.807275in}{0.786285in}}%
\pgfusepath{stroke}%
\end{pgfscope}%
\begin{pgfscope}%
\definecolor{textcolor}{rgb}{0.150000,0.150000,0.150000}%
\pgfsetstrokecolor{textcolor}%
\pgfsetfillcolor{textcolor}%
\pgftext[x=5.807275in,y=0.288178in,,top]{\color{textcolor}\rmfamily\fontsize{14.000000}{16.800000}\selectfont 2016}%
\end{pgfscope}%
\begin{pgfscope}%
\pgfpathrectangle{\pgfqpoint{4.123120in}{0.385400in}}{\pgfqpoint{2.583333in}{0.400885in}}%
\pgfusepath{clip}%
\pgfsetroundcap%
\pgfsetroundjoin%
\pgfsetlinewidth{0.803000pt}%
\definecolor{currentstroke}{rgb}{1.000000,1.000000,1.000000}%
\pgfsetstrokecolor{currentstroke}%
\pgfsetdash{}{0pt}%
\pgfpathmoveto{\pgfqpoint{6.200300in}{0.385400in}}%
\pgfpathlineto{\pgfqpoint{6.200300in}{0.786285in}}%
\pgfusepath{stroke}%
\end{pgfscope}%
\begin{pgfscope}%
\definecolor{textcolor}{rgb}{0.150000,0.150000,0.150000}%
\pgfsetstrokecolor{textcolor}%
\pgfsetfillcolor{textcolor}%
\pgftext[x=6.200300in,y=0.288178in,,top]{\color{textcolor}\rmfamily\fontsize{14.000000}{16.800000}\selectfont 2017}%
\end{pgfscope}%
\begin{pgfscope}%
\pgfpathrectangle{\pgfqpoint{4.123120in}{0.385400in}}{\pgfqpoint{2.583333in}{0.400885in}}%
\pgfusepath{clip}%
\pgfsetroundcap%
\pgfsetroundjoin%
\pgfsetlinewidth{0.803000pt}%
\definecolor{currentstroke}{rgb}{1.000000,1.000000,1.000000}%
\pgfsetstrokecolor{currentstroke}%
\pgfsetdash{}{0pt}%
\pgfpathmoveto{\pgfqpoint{6.592251in}{0.385400in}}%
\pgfpathlineto{\pgfqpoint{6.592251in}{0.786285in}}%
\pgfusepath{stroke}%
\end{pgfscope}%
\begin{pgfscope}%
\definecolor{textcolor}{rgb}{0.150000,0.150000,0.150000}%
\pgfsetstrokecolor{textcolor}%
\pgfsetfillcolor{textcolor}%
\pgftext[x=6.592251in,y=0.288178in,,top]{\color{textcolor}\rmfamily\fontsize{14.000000}{16.800000}\selectfont 2018}%
\end{pgfscope}%
\begin{pgfscope}%
\pgfpathrectangle{\pgfqpoint{4.123120in}{0.385400in}}{\pgfqpoint{2.583333in}{0.400885in}}%
\pgfusepath{clip}%
\pgfsetroundcap%
\pgfsetroundjoin%
\pgfsetlinewidth{0.803000pt}%
\definecolor{currentstroke}{rgb}{1.000000,1.000000,1.000000}%
\pgfsetstrokecolor{currentstroke}%
\pgfsetdash{}{0pt}%
\pgfpathmoveto{\pgfqpoint{4.123120in}{0.403290in}}%
\pgfpathlineto{\pgfqpoint{6.706453in}{0.403290in}}%
\pgfusepath{stroke}%
\end{pgfscope}%
\begin{pgfscope}%
\definecolor{textcolor}{rgb}{0.150000,0.150000,0.150000}%
\pgfsetstrokecolor{textcolor}%
\pgfsetfillcolor{textcolor}%
\pgftext[x=3.716667in,y=0.329424in,left,base]{\color{textcolor}\rmfamily\fontsize{14.000000}{16.800000}\selectfont 0.0}%
\end{pgfscope}%
\begin{pgfscope}%
\pgfpathrectangle{\pgfqpoint{4.123120in}{0.385400in}}{\pgfqpoint{2.583333in}{0.400885in}}%
\pgfusepath{clip}%
\pgfsetroundcap%
\pgfsetroundjoin%
\pgfsetlinewidth{0.803000pt}%
\definecolor{currentstroke}{rgb}{1.000000,1.000000,1.000000}%
\pgfsetstrokecolor{currentstroke}%
\pgfsetdash{}{0pt}%
\pgfpathmoveto{\pgfqpoint{4.123120in}{0.677521in}}%
\pgfpathlineto{\pgfqpoint{6.706453in}{0.677521in}}%
\pgfusepath{stroke}%
\end{pgfscope}%
\begin{pgfscope}%
\definecolor{textcolor}{rgb}{0.150000,0.150000,0.150000}%
\pgfsetstrokecolor{textcolor}%
\pgfsetfillcolor{textcolor}%
\pgftext[x=3.716667in,y=0.603655in,left,base]{\color{textcolor}\rmfamily\fontsize{14.000000}{16.800000}\selectfont 2.5}%
\end{pgfscope}%
\begin{pgfscope}%
\pgfpathrectangle{\pgfqpoint{4.123120in}{0.385400in}}{\pgfqpoint{2.583333in}{0.400885in}}%
\pgfusepath{clip}%
\pgfsetroundcap%
\pgfsetroundjoin%
\pgfsetlinewidth{1.505625pt}%
\definecolor{currentstroke}{rgb}{0.000000,0.000000,0.000000}%
\pgfsetstrokecolor{currentstroke}%
\pgfsetdash{}{0pt}%
\pgfpathmoveto{\pgfqpoint{4.240544in}{0.512982in}}%
\pgfpathlineto{\pgfqpoint{4.243766in}{0.517560in}}%
\pgfpathlineto{\pgfqpoint{4.248061in}{0.516765in}}%
\pgfpathlineto{\pgfqpoint{4.249135in}{0.514095in}}%
\pgfpathlineto{\pgfqpoint{4.250209in}{0.514159in}}%
\pgfpathlineto{\pgfqpoint{4.251283in}{0.513237in}}%
\pgfpathlineto{\pgfqpoint{4.255578in}{0.513459in}}%
\pgfpathlineto{\pgfqpoint{4.257726in}{0.516193in}}%
\pgfpathlineto{\pgfqpoint{4.258800in}{0.515843in}}%
\pgfpathlineto{\pgfqpoint{4.263095in}{0.515652in}}%
\pgfpathlineto{\pgfqpoint{4.264169in}{0.516542in}}%
\pgfpathlineto{\pgfqpoint{4.266317in}{0.515652in}}%
\pgfpathlineto{\pgfqpoint{4.270612in}{0.514667in}}%
\pgfpathlineto{\pgfqpoint{4.271686in}{0.515875in}}%
\pgfpathlineto{\pgfqpoint{4.272760in}{0.514699in}}%
\pgfpathlineto{\pgfqpoint{4.273833in}{0.517814in}}%
\pgfpathlineto{\pgfqpoint{4.277055in}{0.519117in}}%
\pgfpathlineto{\pgfqpoint{4.279203in}{0.521438in}}%
\pgfpathlineto{\pgfqpoint{4.280276in}{0.522200in}}%
\pgfpathlineto{\pgfqpoint{4.281350in}{0.521946in}}%
\pgfpathlineto{\pgfqpoint{4.284572in}{0.522931in}}%
\pgfpathlineto{\pgfqpoint{4.286719in}{0.521374in}}%
\pgfpathlineto{\pgfqpoint{4.288867in}{0.522804in}}%
\pgfpathlineto{\pgfqpoint{4.293162in}{0.522296in}}%
\pgfpathlineto{\pgfqpoint{4.294236in}{0.521438in}}%
\pgfpathlineto{\pgfqpoint{4.295310in}{0.522041in}}%
\pgfpathlineto{\pgfqpoint{4.296384in}{0.521565in}}%
\pgfpathlineto{\pgfqpoint{4.301753in}{0.523504in}}%
\pgfpathlineto{\pgfqpoint{4.302827in}{0.524648in}}%
\pgfpathlineto{\pgfqpoint{4.303901in}{0.524553in}}%
\pgfpathlineto{\pgfqpoint{4.307122in}{0.525538in}}%
\pgfpathlineto{\pgfqpoint{4.308196in}{0.523535in}}%
\pgfpathlineto{\pgfqpoint{4.309270in}{0.522804in}}%
\pgfpathlineto{\pgfqpoint{4.311418in}{0.524235in}}%
\pgfpathlineto{\pgfqpoint{4.314639in}{0.524521in}}%
\pgfpathlineto{\pgfqpoint{4.315713in}{0.529289in}}%
\pgfpathlineto{\pgfqpoint{4.316787in}{0.527763in}}%
\pgfpathlineto{\pgfqpoint{4.317861in}{0.527731in}}%
\pgfpathlineto{\pgfqpoint{4.318935in}{0.526936in}}%
\pgfpathlineto{\pgfqpoint{4.322156in}{0.527668in}}%
\pgfpathlineto{\pgfqpoint{4.323230in}{0.527095in}}%
\pgfpathlineto{\pgfqpoint{4.325378in}{0.527223in}}%
\pgfpathlineto{\pgfqpoint{4.326451in}{0.528271in}}%
\pgfpathlineto{\pgfqpoint{4.329673in}{0.530338in}}%
\pgfpathlineto{\pgfqpoint{4.330747in}{0.529702in}}%
\pgfpathlineto{\pgfqpoint{4.332894in}{0.526428in}}%
\pgfpathlineto{\pgfqpoint{4.333968in}{0.528621in}}%
\pgfpathlineto{\pgfqpoint{4.337190in}{0.528812in}}%
\pgfpathlineto{\pgfqpoint{4.339338in}{0.526205in}}%
\pgfpathlineto{\pgfqpoint{4.340411in}{0.526619in}}%
\pgfpathlineto{\pgfqpoint{4.344707in}{0.523853in}}%
\pgfpathlineto{\pgfqpoint{4.345781in}{0.520643in}}%
\pgfpathlineto{\pgfqpoint{4.346854in}{0.521724in}}%
\pgfpathlineto{\pgfqpoint{4.347928in}{0.523949in}}%
\pgfpathlineto{\pgfqpoint{4.349002in}{0.523090in}}%
\pgfpathlineto{\pgfqpoint{4.352224in}{0.522550in}}%
\pgfpathlineto{\pgfqpoint{4.353297in}{0.525474in}}%
\pgfpathlineto{\pgfqpoint{4.354371in}{0.524934in}}%
\pgfpathlineto{\pgfqpoint{4.355445in}{0.523758in}}%
\pgfpathlineto{\pgfqpoint{4.356519in}{0.524521in}}%
\pgfpathlineto{\pgfqpoint{4.359740in}{0.523567in}}%
\pgfpathlineto{\pgfqpoint{4.360814in}{0.524044in}}%
\pgfpathlineto{\pgfqpoint{4.362962in}{0.527413in}}%
\pgfpathlineto{\pgfqpoint{4.364036in}{0.527413in}}%
\pgfpathlineto{\pgfqpoint{4.367257in}{0.526714in}}%
\pgfpathlineto{\pgfqpoint{4.368331in}{0.528653in}}%
\pgfpathlineto{\pgfqpoint{4.369405in}{0.527954in}}%
\pgfpathlineto{\pgfqpoint{4.370479in}{0.528716in}}%
\pgfpathlineto{\pgfqpoint{4.371553in}{0.526205in}}%
\pgfpathlineto{\pgfqpoint{4.375848in}{0.530115in}}%
\pgfpathlineto{\pgfqpoint{4.376922in}{0.532181in}}%
\pgfpathlineto{\pgfqpoint{4.379070in}{0.533739in}}%
\pgfpathlineto{\pgfqpoint{4.384439in}{0.532340in}}%
\pgfpathlineto{\pgfqpoint{4.386586in}{0.528716in}}%
\pgfpathlineto{\pgfqpoint{4.390882in}{0.530369in}}%
\pgfpathlineto{\pgfqpoint{4.391956in}{0.529861in}}%
\pgfpathlineto{\pgfqpoint{4.398399in}{0.533484in}}%
\pgfpathlineto{\pgfqpoint{4.399472in}{0.532690in}}%
\pgfpathlineto{\pgfqpoint{4.400546in}{0.534152in}}%
\pgfpathlineto{\pgfqpoint{4.401620in}{0.530401in}}%
\pgfpathlineto{\pgfqpoint{4.404842in}{0.530433in}}%
\pgfpathlineto{\pgfqpoint{4.409137in}{0.535678in}}%
\pgfpathlineto{\pgfqpoint{4.412359in}{0.534406in}}%
\pgfpathlineto{\pgfqpoint{4.413432in}{0.536059in}}%
\pgfpathlineto{\pgfqpoint{4.414506in}{0.535646in}}%
\pgfpathlineto{\pgfqpoint{4.415580in}{0.538379in}}%
\pgfpathlineto{\pgfqpoint{4.416654in}{0.538093in}}%
\pgfpathlineto{\pgfqpoint{4.419875in}{0.538125in}}%
\pgfpathlineto{\pgfqpoint{4.422023in}{0.539937in}}%
\pgfpathlineto{\pgfqpoint{4.423097in}{0.538983in}}%
\pgfpathlineto{\pgfqpoint{4.424171in}{0.539206in}}%
\pgfpathlineto{\pgfqpoint{4.427392in}{0.536981in}}%
\pgfpathlineto{\pgfqpoint{4.429540in}{0.540350in}}%
\pgfpathlineto{\pgfqpoint{4.430614in}{0.540191in}}%
\pgfpathlineto{\pgfqpoint{4.431688in}{0.542130in}}%
\pgfpathlineto{\pgfqpoint{4.435983in}{0.542416in}}%
\pgfpathlineto{\pgfqpoint{4.439205in}{0.540827in}}%
\pgfpathlineto{\pgfqpoint{4.442426in}{0.540700in}}%
\pgfpathlineto{\pgfqpoint{4.443500in}{0.538888in}}%
\pgfpathlineto{\pgfqpoint{4.444574in}{0.538634in}}%
\pgfpathlineto{\pgfqpoint{4.445648in}{0.539015in}}%
\pgfpathlineto{\pgfqpoint{4.446721in}{0.541272in}}%
\pgfpathlineto{\pgfqpoint{4.449943in}{0.540318in}}%
\pgfpathlineto{\pgfqpoint{4.451017in}{0.544578in}}%
\pgfpathlineto{\pgfqpoint{4.452091in}{0.544578in}}%
\pgfpathlineto{\pgfqpoint{4.454238in}{0.542416in}}%
\pgfpathlineto{\pgfqpoint{4.457460in}{0.540668in}}%
\pgfpathlineto{\pgfqpoint{4.459607in}{0.541558in}}%
\pgfpathlineto{\pgfqpoint{4.460681in}{0.545595in}}%
\pgfpathlineto{\pgfqpoint{4.461755in}{0.546262in}}%
\pgfpathlineto{\pgfqpoint{4.464977in}{0.545881in}}%
\pgfpathlineto{\pgfqpoint{4.467124in}{0.543052in}}%
\pgfpathlineto{\pgfqpoint{4.468198in}{0.543529in}}%
\pgfpathlineto{\pgfqpoint{4.469272in}{0.545785in}}%
\pgfpathlineto{\pgfqpoint{4.472494in}{0.545436in}}%
\pgfpathlineto{\pgfqpoint{4.473567in}{0.545881in}}%
\pgfpathlineto{\pgfqpoint{4.474641in}{0.547851in}}%
\pgfpathlineto{\pgfqpoint{4.476789in}{0.545436in}}%
\pgfpathlineto{\pgfqpoint{4.480010in}{0.546040in}}%
\pgfpathlineto{\pgfqpoint{4.481084in}{0.545563in}}%
\pgfpathlineto{\pgfqpoint{4.484306in}{0.547756in}}%
\pgfpathlineto{\pgfqpoint{4.487527in}{0.547724in}}%
\pgfpathlineto{\pgfqpoint{4.488601in}{0.545404in}}%
\pgfpathlineto{\pgfqpoint{4.489675in}{0.545468in}}%
\pgfpathlineto{\pgfqpoint{4.490749in}{0.544069in}}%
\pgfpathlineto{\pgfqpoint{4.491823in}{0.545181in}}%
\pgfpathlineto{\pgfqpoint{4.496118in}{0.545372in}}%
\pgfpathlineto{\pgfqpoint{4.497192in}{0.546548in}}%
\pgfpathlineto{\pgfqpoint{4.498266in}{0.544768in}}%
\pgfpathlineto{\pgfqpoint{4.503635in}{0.545468in}}%
\pgfpathlineto{\pgfqpoint{4.505782in}{0.551761in}}%
\pgfpathlineto{\pgfqpoint{4.510078in}{0.550776in}}%
\pgfpathlineto{\pgfqpoint{4.512226in}{0.551316in}}%
\pgfpathlineto{\pgfqpoint{4.513299in}{0.553891in}}%
\pgfpathlineto{\pgfqpoint{4.514373in}{0.553160in}}%
\pgfpathlineto{\pgfqpoint{4.518669in}{0.551888in}}%
\pgfpathlineto{\pgfqpoint{4.519742in}{0.554177in}}%
\pgfpathlineto{\pgfqpoint{4.521890in}{0.554272in}}%
\pgfpathlineto{\pgfqpoint{4.525112in}{0.554813in}}%
\pgfpathlineto{\pgfqpoint{4.527259in}{0.551984in}}%
\pgfpathlineto{\pgfqpoint{4.528333in}{0.553827in}}%
\pgfpathlineto{\pgfqpoint{4.529407in}{0.552969in}}%
\pgfpathlineto{\pgfqpoint{4.532628in}{0.552365in}}%
\pgfpathlineto{\pgfqpoint{4.533702in}{0.551125in}}%
\pgfpathlineto{\pgfqpoint{4.534776in}{0.553509in}}%
\pgfpathlineto{\pgfqpoint{4.536924in}{0.554940in}}%
\pgfpathlineto{\pgfqpoint{4.540145in}{0.553096in}}%
\pgfpathlineto{\pgfqpoint{4.541219in}{0.550776in}}%
\pgfpathlineto{\pgfqpoint{4.542293in}{0.549918in}}%
\pgfpathlineto{\pgfqpoint{4.543367in}{0.547406in}}%
\pgfpathlineto{\pgfqpoint{4.544441in}{0.548138in}}%
\pgfpathlineto{\pgfqpoint{4.547662in}{0.548710in}}%
\pgfpathlineto{\pgfqpoint{4.548736in}{0.549981in}}%
\pgfpathlineto{\pgfqpoint{4.549810in}{0.553001in}}%
\pgfpathlineto{\pgfqpoint{4.550884in}{0.553382in}}%
\pgfpathlineto{\pgfqpoint{4.551958in}{0.551888in}}%
\pgfpathlineto{\pgfqpoint{4.555179in}{0.551570in}}%
\pgfpathlineto{\pgfqpoint{4.556253in}{0.548614in}}%
\pgfpathlineto{\pgfqpoint{4.557327in}{0.548296in}}%
\pgfpathlineto{\pgfqpoint{4.559474in}{0.546675in}}%
\pgfpathlineto{\pgfqpoint{4.564844in}{0.543910in}}%
\pgfpathlineto{\pgfqpoint{4.565917in}{0.545817in}}%
\pgfpathlineto{\pgfqpoint{4.566991in}{0.546040in}}%
\pgfpathlineto{\pgfqpoint{4.571287in}{0.547788in}}%
\pgfpathlineto{\pgfqpoint{4.572360in}{0.546675in}}%
\pgfpathlineto{\pgfqpoint{4.573434in}{0.546548in}}%
\pgfpathlineto{\pgfqpoint{4.574508in}{0.538030in}}%
\pgfpathlineto{\pgfqpoint{4.577730in}{0.539142in}}%
\pgfpathlineto{\pgfqpoint{4.578804in}{0.540604in}}%
\pgfpathlineto{\pgfqpoint{4.579877in}{0.538348in}}%
\pgfpathlineto{\pgfqpoint{4.580951in}{0.539206in}}%
\pgfpathlineto{\pgfqpoint{4.582025in}{0.539047in}}%
\pgfpathlineto{\pgfqpoint{4.585247in}{0.540445in}}%
\pgfpathlineto{\pgfqpoint{4.587394in}{0.542670in}}%
\pgfpathlineto{\pgfqpoint{4.589542in}{0.544323in}}%
\pgfpathlineto{\pgfqpoint{4.592763in}{0.543656in}}%
\pgfpathlineto{\pgfqpoint{4.593837in}{0.542416in}}%
\pgfpathlineto{\pgfqpoint{4.595985in}{0.545626in}}%
\pgfpathlineto{\pgfqpoint{4.597059in}{0.545468in}}%
\pgfpathlineto{\pgfqpoint{4.601354in}{0.544419in}}%
\pgfpathlineto{\pgfqpoint{4.604576in}{0.546421in}}%
\pgfpathlineto{\pgfqpoint{4.608871in}{0.547120in}}%
\pgfpathlineto{\pgfqpoint{4.609945in}{0.547597in}}%
\pgfpathlineto{\pgfqpoint{4.612093in}{0.544768in}}%
\pgfpathlineto{\pgfqpoint{4.615314in}{0.546548in}}%
\pgfpathlineto{\pgfqpoint{4.616388in}{0.549282in}}%
\pgfpathlineto{\pgfqpoint{4.617462in}{0.548455in}}%
\pgfpathlineto{\pgfqpoint{4.618536in}{0.551348in}}%
\pgfpathlineto{\pgfqpoint{4.619609in}{0.548646in}}%
\pgfpathlineto{\pgfqpoint{4.624979in}{0.548201in}}%
\pgfpathlineto{\pgfqpoint{4.627126in}{0.546167in}}%
\pgfpathlineto{\pgfqpoint{4.630348in}{0.548010in}}%
\pgfpathlineto{\pgfqpoint{4.632495in}{0.551825in}}%
\pgfpathlineto{\pgfqpoint{4.633569in}{0.552143in}}%
\pgfpathlineto{\pgfqpoint{4.634643in}{0.555003in}}%
\pgfpathlineto{\pgfqpoint{4.638938in}{0.550839in}}%
\pgfpathlineto{\pgfqpoint{4.641086in}{0.550935in}}%
\pgfpathlineto{\pgfqpoint{4.642160in}{0.550331in}}%
\pgfpathlineto{\pgfqpoint{4.645382in}{0.550363in}}%
\pgfpathlineto{\pgfqpoint{4.647529in}{0.553096in}}%
\pgfpathlineto{\pgfqpoint{4.648603in}{0.555639in}}%
\pgfpathlineto{\pgfqpoint{4.649677in}{0.555448in}}%
\pgfpathlineto{\pgfqpoint{4.653972in}{0.556561in}}%
\pgfpathlineto{\pgfqpoint{4.655046in}{0.560121in}}%
\pgfpathlineto{\pgfqpoint{4.656120in}{0.560121in}}%
\pgfpathlineto{\pgfqpoint{4.657194in}{0.561360in}}%
\pgfpathlineto{\pgfqpoint{4.660415in}{0.561297in}}%
\pgfpathlineto{\pgfqpoint{4.662563in}{0.559644in}}%
\pgfpathlineto{\pgfqpoint{4.663637in}{0.559930in}}%
\pgfpathlineto{\pgfqpoint{4.664711in}{0.561964in}}%
\pgfpathlineto{\pgfqpoint{4.667932in}{0.559962in}}%
\pgfpathlineto{\pgfqpoint{4.670080in}{0.561774in}}%
\pgfpathlineto{\pgfqpoint{4.671154in}{0.561297in}}%
\pgfpathlineto{\pgfqpoint{4.672227in}{0.562187in}}%
\pgfpathlineto{\pgfqpoint{4.678670in}{0.562823in}}%
\pgfpathlineto{\pgfqpoint{4.679744in}{0.564952in}}%
\pgfpathlineto{\pgfqpoint{4.684040in}{0.565302in}}%
\pgfpathlineto{\pgfqpoint{4.685114in}{0.561996in}}%
\pgfpathlineto{\pgfqpoint{4.686187in}{0.560756in}}%
\pgfpathlineto{\pgfqpoint{4.687261in}{0.560979in}}%
\pgfpathlineto{\pgfqpoint{4.690483in}{0.559072in}}%
\pgfpathlineto{\pgfqpoint{4.691557in}{0.559962in}}%
\pgfpathlineto{\pgfqpoint{4.692630in}{0.561646in}}%
\pgfpathlineto{\pgfqpoint{4.693704in}{0.561964in}}%
\pgfpathlineto{\pgfqpoint{4.694778in}{0.564126in}}%
\pgfpathlineto{\pgfqpoint{4.698000in}{0.565493in}}%
\pgfpathlineto{\pgfqpoint{4.699073in}{0.567463in}}%
\pgfpathlineto{\pgfqpoint{4.701221in}{0.567018in}}%
\pgfpathlineto{\pgfqpoint{4.702295in}{0.570133in}}%
\pgfpathlineto{\pgfqpoint{4.705516in}{0.570896in}}%
\pgfpathlineto{\pgfqpoint{4.706590in}{0.569307in}}%
\pgfpathlineto{\pgfqpoint{4.709812in}{0.570674in}}%
\pgfpathlineto{\pgfqpoint{4.713033in}{0.568480in}}%
\pgfpathlineto{\pgfqpoint{4.714107in}{0.566986in}}%
\pgfpathlineto{\pgfqpoint{4.715181in}{0.568798in}}%
\pgfpathlineto{\pgfqpoint{4.716255in}{0.566986in}}%
\pgfpathlineto{\pgfqpoint{4.717329in}{0.568353in}}%
\pgfpathlineto{\pgfqpoint{4.720550in}{0.566700in}}%
\pgfpathlineto{\pgfqpoint{4.721624in}{0.567908in}}%
\pgfpathlineto{\pgfqpoint{4.722698in}{0.567431in}}%
\pgfpathlineto{\pgfqpoint{4.723772in}{0.568417in}}%
\pgfpathlineto{\pgfqpoint{4.728067in}{0.568099in}}%
\pgfpathlineto{\pgfqpoint{4.729141in}{0.570324in}}%
\pgfpathlineto{\pgfqpoint{4.730215in}{0.569720in}}%
\pgfpathlineto{\pgfqpoint{4.736658in}{0.575219in}}%
\pgfpathlineto{\pgfqpoint{4.738805in}{0.579319in}}%
\pgfpathlineto{\pgfqpoint{4.739879in}{0.579319in}}%
\pgfpathlineto{\pgfqpoint{4.743101in}{0.574456in}}%
\pgfpathlineto{\pgfqpoint{4.744175in}{0.579892in}}%
\pgfpathlineto{\pgfqpoint{4.745248in}{0.579669in}}%
\pgfpathlineto{\pgfqpoint{4.746322in}{0.577667in}}%
\pgfpathlineto{\pgfqpoint{4.747396in}{0.582244in}}%
\pgfpathlineto{\pgfqpoint{4.750618in}{0.583547in}}%
\pgfpathlineto{\pgfqpoint{4.751692in}{0.585232in}}%
\pgfpathlineto{\pgfqpoint{4.752765in}{0.583356in}}%
\pgfpathlineto{\pgfqpoint{4.754913in}{0.583134in}}%
\pgfpathlineto{\pgfqpoint{4.758135in}{0.586439in}}%
\pgfpathlineto{\pgfqpoint{4.759208in}{0.585963in}}%
\pgfpathlineto{\pgfqpoint{4.760282in}{0.587043in}}%
\pgfpathlineto{\pgfqpoint{4.762430in}{0.591652in}}%
\pgfpathlineto{\pgfqpoint{4.765651in}{0.592415in}}%
\pgfpathlineto{\pgfqpoint{4.766725in}{0.595339in}}%
\pgfpathlineto{\pgfqpoint{4.767799in}{0.595117in}}%
\pgfpathlineto{\pgfqpoint{4.769947in}{0.598645in}}%
\pgfpathlineto{\pgfqpoint{4.774242in}{0.599408in}}%
\pgfpathlineto{\pgfqpoint{4.775316in}{0.600012in}}%
\pgfpathlineto{\pgfqpoint{4.776390in}{0.596515in}}%
\pgfpathlineto{\pgfqpoint{4.777464in}{0.596833in}}%
\pgfpathlineto{\pgfqpoint{4.781759in}{0.594640in}}%
\pgfpathlineto{\pgfqpoint{4.783907in}{0.592924in}}%
\pgfpathlineto{\pgfqpoint{4.789276in}{0.597151in}}%
\pgfpathlineto{\pgfqpoint{4.790350in}{0.595912in}}%
\pgfpathlineto{\pgfqpoint{4.792497in}{0.586662in}}%
\pgfpathlineto{\pgfqpoint{4.795719in}{0.588760in}}%
\pgfpathlineto{\pgfqpoint{4.796793in}{0.590349in}}%
\pgfpathlineto{\pgfqpoint{4.797867in}{0.586789in}}%
\pgfpathlineto{\pgfqpoint{4.798940in}{0.586821in}}%
\pgfpathlineto{\pgfqpoint{4.800014in}{0.591811in}}%
\pgfpathlineto{\pgfqpoint{4.803236in}{0.588855in}}%
\pgfpathlineto{\pgfqpoint{4.804310in}{0.588760in}}%
\pgfpathlineto{\pgfqpoint{4.805383in}{0.586408in}}%
\pgfpathlineto{\pgfqpoint{4.806457in}{0.590190in}}%
\pgfpathlineto{\pgfqpoint{4.807531in}{0.588760in}}%
\pgfpathlineto{\pgfqpoint{4.810753in}{0.590762in}}%
\pgfpathlineto{\pgfqpoint{4.811826in}{0.593083in}}%
\pgfpathlineto{\pgfqpoint{4.812900in}{0.590285in}}%
\pgfpathlineto{\pgfqpoint{4.813974in}{0.583452in}}%
\pgfpathlineto{\pgfqpoint{4.815048in}{0.585645in}}%
\pgfpathlineto{\pgfqpoint{4.818270in}{0.584787in}}%
\pgfpathlineto{\pgfqpoint{4.819343in}{0.585168in}}%
\pgfpathlineto{\pgfqpoint{4.821491in}{0.588505in}}%
\pgfpathlineto{\pgfqpoint{4.822565in}{0.586853in}}%
\pgfpathlineto{\pgfqpoint{4.825786in}{0.589141in}}%
\pgfpathlineto{\pgfqpoint{4.826860in}{0.587170in}}%
\pgfpathlineto{\pgfqpoint{4.827934in}{0.588188in}}%
\pgfpathlineto{\pgfqpoint{4.830082in}{0.588823in}}%
\pgfpathlineto{\pgfqpoint{4.834377in}{0.592065in}}%
\pgfpathlineto{\pgfqpoint{4.835451in}{0.591970in}}%
\pgfpathlineto{\pgfqpoint{4.836525in}{0.596833in}}%
\pgfpathlineto{\pgfqpoint{4.837599in}{0.598009in}}%
\pgfpathlineto{\pgfqpoint{4.840820in}{0.594958in}}%
\pgfpathlineto{\pgfqpoint{4.841894in}{0.592320in}}%
\pgfpathlineto{\pgfqpoint{4.844042in}{0.594608in}}%
\pgfpathlineto{\pgfqpoint{4.845115in}{0.592701in}}%
\pgfpathlineto{\pgfqpoint{4.848337in}{0.590508in}}%
\pgfpathlineto{\pgfqpoint{4.851559in}{0.590921in}}%
\pgfpathlineto{\pgfqpoint{4.852632in}{0.592193in}}%
\pgfpathlineto{\pgfqpoint{4.855854in}{0.591112in}}%
\pgfpathlineto{\pgfqpoint{4.856928in}{0.589872in}}%
\pgfpathlineto{\pgfqpoint{4.859075in}{0.593273in}}%
\pgfpathlineto{\pgfqpoint{4.860149in}{0.596643in}}%
\pgfpathlineto{\pgfqpoint{4.863371in}{0.595212in}}%
\pgfpathlineto{\pgfqpoint{4.864445in}{0.598200in}}%
\pgfpathlineto{\pgfqpoint{4.865518in}{0.594894in}}%
\pgfpathlineto{\pgfqpoint{4.866592in}{0.594418in}}%
\pgfpathlineto{\pgfqpoint{4.867666in}{0.591462in}}%
\pgfpathlineto{\pgfqpoint{4.870888in}{0.589141in}}%
\pgfpathlineto{\pgfqpoint{4.873035in}{0.589205in}}%
\pgfpathlineto{\pgfqpoint{4.874109in}{0.584628in}}%
\pgfpathlineto{\pgfqpoint{4.875183in}{0.584024in}}%
\pgfpathlineto{\pgfqpoint{4.879478in}{0.583165in}}%
\pgfpathlineto{\pgfqpoint{4.880552in}{0.581004in}}%
\pgfpathlineto{\pgfqpoint{4.881626in}{0.582466in}}%
\pgfpathlineto{\pgfqpoint{4.882700in}{0.582720in}}%
\pgfpathlineto{\pgfqpoint{4.885921in}{0.581640in}}%
\pgfpathlineto{\pgfqpoint{4.886995in}{0.579701in}}%
\pgfpathlineto{\pgfqpoint{4.889143in}{0.580845in}}%
\pgfpathlineto{\pgfqpoint{4.890217in}{0.580114in}}%
\pgfpathlineto{\pgfqpoint{4.895586in}{0.580940in}}%
\pgfpathlineto{\pgfqpoint{4.897734in}{0.581735in}}%
\pgfpathlineto{\pgfqpoint{4.900955in}{0.582339in}}%
\pgfpathlineto{\pgfqpoint{4.905250in}{0.597151in}}%
\pgfpathlineto{\pgfqpoint{4.910620in}{0.598359in}}%
\pgfpathlineto{\pgfqpoint{4.912767in}{0.592256in}}%
\pgfpathlineto{\pgfqpoint{4.915989in}{0.591525in}}%
\pgfpathlineto{\pgfqpoint{4.917063in}{0.590254in}}%
\pgfpathlineto{\pgfqpoint{4.918136in}{0.590635in}}%
\pgfpathlineto{\pgfqpoint{4.919210in}{0.592924in}}%
\pgfpathlineto{\pgfqpoint{4.920284in}{0.592797in}}%
\pgfpathlineto{\pgfqpoint{4.923506in}{0.590762in}}%
\pgfpathlineto{\pgfqpoint{4.924580in}{0.591748in}}%
\pgfpathlineto{\pgfqpoint{4.925653in}{0.591907in}}%
\pgfpathlineto{\pgfqpoint{4.926727in}{0.589395in}}%
\pgfpathlineto{\pgfqpoint{4.927801in}{0.593114in}}%
\pgfpathlineto{\pgfqpoint{4.931023in}{0.591048in}}%
\pgfpathlineto{\pgfqpoint{4.933170in}{0.588156in}}%
\pgfpathlineto{\pgfqpoint{4.934244in}{0.593941in}}%
\pgfpathlineto{\pgfqpoint{4.935318in}{0.595753in}}%
\pgfpathlineto{\pgfqpoint{4.938539in}{0.597564in}}%
\pgfpathlineto{\pgfqpoint{4.939613in}{0.596420in}}%
\pgfpathlineto{\pgfqpoint{4.941761in}{0.596357in}}%
\pgfpathlineto{\pgfqpoint{4.942835in}{0.598486in}}%
\pgfpathlineto{\pgfqpoint{4.946056in}{0.599821in}}%
\pgfpathlineto{\pgfqpoint{4.947130in}{0.603858in}}%
\pgfpathlineto{\pgfqpoint{4.948204in}{0.601315in}}%
\pgfpathlineto{\pgfqpoint{4.949278in}{0.604017in}}%
\pgfpathlineto{\pgfqpoint{4.950352in}{0.604621in}}%
\pgfpathlineto{\pgfqpoint{4.954647in}{0.603635in}}%
\pgfpathlineto{\pgfqpoint{4.955721in}{0.602332in}}%
\pgfpathlineto{\pgfqpoint{4.956795in}{0.602682in}}%
\pgfpathlineto{\pgfqpoint{4.957869in}{0.603890in}}%
\pgfpathlineto{\pgfqpoint{4.962164in}{0.603445in}}%
\pgfpathlineto{\pgfqpoint{4.963238in}{0.603858in}}%
\pgfpathlineto{\pgfqpoint{4.964312in}{0.598486in}}%
\pgfpathlineto{\pgfqpoint{4.965385in}{0.602650in}}%
\pgfpathlineto{\pgfqpoint{4.968607in}{0.601951in}}%
\pgfpathlineto{\pgfqpoint{4.969681in}{0.600298in}}%
\pgfpathlineto{\pgfqpoint{4.971828in}{0.606846in}}%
\pgfpathlineto{\pgfqpoint{4.972902in}{0.606782in}}%
\pgfpathlineto{\pgfqpoint{4.976124in}{0.605320in}}%
\pgfpathlineto{\pgfqpoint{4.977198in}{0.604208in}}%
\pgfpathlineto{\pgfqpoint{4.978271in}{0.604525in}}%
\pgfpathlineto{\pgfqpoint{4.979345in}{0.606592in}}%
\pgfpathlineto{\pgfqpoint{4.980419in}{0.607354in}}%
\pgfpathlineto{\pgfqpoint{4.983641in}{0.606019in}}%
\pgfpathlineto{\pgfqpoint{4.984714in}{0.610215in}}%
\pgfpathlineto{\pgfqpoint{4.985788in}{0.609007in}}%
\pgfpathlineto{\pgfqpoint{4.987936in}{0.608340in}}%
\pgfpathlineto{\pgfqpoint{4.991158in}{0.609420in}}%
\pgfpathlineto{\pgfqpoint{4.992231in}{0.606496in}}%
\pgfpathlineto{\pgfqpoint{4.993305in}{0.606687in}}%
\pgfpathlineto{\pgfqpoint{4.994379in}{0.607450in}}%
\pgfpathlineto{\pgfqpoint{4.995453in}{0.611010in}}%
\pgfpathlineto{\pgfqpoint{4.998674in}{0.609993in}}%
\pgfpathlineto{\pgfqpoint{4.999748in}{0.611328in}}%
\pgfpathlineto{\pgfqpoint{5.000822in}{0.608276in}}%
\pgfpathlineto{\pgfqpoint{5.002970in}{0.608181in}}%
\pgfpathlineto{\pgfqpoint{5.007265in}{0.611232in}}%
\pgfpathlineto{\pgfqpoint{5.009413in}{0.618034in}}%
\pgfpathlineto{\pgfqpoint{5.010487in}{0.616350in}}%
\pgfpathlineto{\pgfqpoint{5.013708in}{0.618924in}}%
\pgfpathlineto{\pgfqpoint{5.014782in}{0.620609in}}%
\pgfpathlineto{\pgfqpoint{5.016930in}{0.622898in}}%
\pgfpathlineto{\pgfqpoint{5.018003in}{0.622103in}}%
\pgfpathlineto{\pgfqpoint{5.021225in}{0.627634in}}%
\pgfpathlineto{\pgfqpoint{5.022299in}{0.628111in}}%
\pgfpathlineto{\pgfqpoint{5.028742in}{0.626426in}}%
\pgfpathlineto{\pgfqpoint{5.029816in}{0.627952in}}%
\pgfpathlineto{\pgfqpoint{5.030890in}{0.624646in}}%
\pgfpathlineto{\pgfqpoint{5.031963in}{0.623692in}}%
\pgfpathlineto{\pgfqpoint{5.033037in}{0.625154in}}%
\pgfpathlineto{\pgfqpoint{5.036259in}{0.618924in}}%
\pgfpathlineto{\pgfqpoint{5.037333in}{0.622389in}}%
\pgfpathlineto{\pgfqpoint{5.040554in}{0.620990in}}%
\pgfpathlineto{\pgfqpoint{5.044849in}{0.621658in}}%
\pgfpathlineto{\pgfqpoint{5.045923in}{0.624900in}}%
\pgfpathlineto{\pgfqpoint{5.046997in}{0.623374in}}%
\pgfpathlineto{\pgfqpoint{5.048071in}{0.617303in}}%
\pgfpathlineto{\pgfqpoint{5.051292in}{0.615905in}}%
\pgfpathlineto{\pgfqpoint{5.052366in}{0.617748in}}%
\pgfpathlineto{\pgfqpoint{5.053440in}{0.613203in}}%
\pgfpathlineto{\pgfqpoint{5.054514in}{0.618765in}}%
\pgfpathlineto{\pgfqpoint{5.055588in}{0.616954in}}%
\pgfpathlineto{\pgfqpoint{5.058809in}{0.609262in}}%
\pgfpathlineto{\pgfqpoint{5.060957in}{0.614474in}}%
\pgfpathlineto{\pgfqpoint{5.062031in}{0.625663in}}%
\pgfpathlineto{\pgfqpoint{5.063105in}{0.625981in}}%
\pgfpathlineto{\pgfqpoint{5.068474in}{0.632561in}}%
\pgfpathlineto{\pgfqpoint{5.069548in}{0.632529in}}%
\pgfpathlineto{\pgfqpoint{5.070622in}{0.636438in}}%
\pgfpathlineto{\pgfqpoint{5.074917in}{0.637487in}}%
\pgfpathlineto{\pgfqpoint{5.075991in}{0.635389in}}%
\pgfpathlineto{\pgfqpoint{5.077065in}{0.636343in}}%
\pgfpathlineto{\pgfqpoint{5.078138in}{0.639108in}}%
\pgfpathlineto{\pgfqpoint{5.081360in}{0.640857in}}%
\pgfpathlineto{\pgfqpoint{5.082434in}{0.639331in}}%
\pgfpathlineto{\pgfqpoint{5.083508in}{0.638949in}}%
\pgfpathlineto{\pgfqpoint{5.085655in}{0.641111in}}%
\pgfpathlineto{\pgfqpoint{5.088877in}{0.637138in}}%
\pgfpathlineto{\pgfqpoint{5.089951in}{0.643749in}}%
\pgfpathlineto{\pgfqpoint{5.092098in}{0.648549in}}%
\pgfpathlineto{\pgfqpoint{5.093172in}{0.645211in}}%
\pgfpathlineto{\pgfqpoint{5.096394in}{0.644512in}}%
\pgfpathlineto{\pgfqpoint{5.097468in}{0.641810in}}%
\pgfpathlineto{\pgfqpoint{5.098541in}{0.642764in}}%
\pgfpathlineto{\pgfqpoint{5.099615in}{0.638504in}}%
\pgfpathlineto{\pgfqpoint{5.100689in}{0.638918in}}%
\pgfpathlineto{\pgfqpoint{5.104984in}{0.644576in}}%
\pgfpathlineto{\pgfqpoint{5.106058in}{0.640253in}}%
\pgfpathlineto{\pgfqpoint{5.107132in}{0.641111in}}%
\pgfpathlineto{\pgfqpoint{5.108206in}{0.639744in}}%
\pgfpathlineto{\pgfqpoint{5.111427in}{0.637201in}}%
\pgfpathlineto{\pgfqpoint{5.112501in}{0.637392in}}%
\pgfpathlineto{\pgfqpoint{5.113575in}{0.634658in}}%
\pgfpathlineto{\pgfqpoint{5.114649in}{0.634245in}}%
\pgfpathlineto{\pgfqpoint{5.115723in}{0.635739in}}%
\pgfpathlineto{\pgfqpoint{5.118944in}{0.638918in}}%
\pgfpathlineto{\pgfqpoint{5.120018in}{0.643336in}}%
\pgfpathlineto{\pgfqpoint{5.122166in}{0.643686in}}%
\pgfpathlineto{\pgfqpoint{5.123240in}{0.639967in}}%
\pgfpathlineto{\pgfqpoint{5.126461in}{0.636152in}}%
\pgfpathlineto{\pgfqpoint{5.127535in}{0.637456in}}%
\pgfpathlineto{\pgfqpoint{5.128609in}{0.640094in}}%
\pgfpathlineto{\pgfqpoint{5.129683in}{0.631384in}}%
\pgfpathlineto{\pgfqpoint{5.130757in}{0.629922in}}%
\pgfpathlineto{\pgfqpoint{5.135052in}{0.631829in}}%
\pgfpathlineto{\pgfqpoint{5.137200in}{0.638695in}}%
\pgfpathlineto{\pgfqpoint{5.141495in}{0.636089in}}%
\pgfpathlineto{\pgfqpoint{5.142569in}{0.637106in}}%
\pgfpathlineto{\pgfqpoint{5.144716in}{0.637519in}}%
\pgfpathlineto{\pgfqpoint{5.145790in}{0.633514in}}%
\pgfpathlineto{\pgfqpoint{5.149012in}{0.632179in}}%
\pgfpathlineto{\pgfqpoint{5.151159in}{0.636788in}}%
\pgfpathlineto{\pgfqpoint{5.152233in}{0.637424in}}%
\pgfpathlineto{\pgfqpoint{5.153307in}{0.639617in}}%
\pgfpathlineto{\pgfqpoint{5.156529in}{0.642319in}}%
\pgfpathlineto{\pgfqpoint{5.157602in}{0.641747in}}%
\pgfpathlineto{\pgfqpoint{5.158676in}{0.639553in}}%
\pgfpathlineto{\pgfqpoint{5.159750in}{0.643431in}}%
\pgfpathlineto{\pgfqpoint{5.160824in}{0.644448in}}%
\pgfpathlineto{\pgfqpoint{5.164046in}{0.645847in}}%
\pgfpathlineto{\pgfqpoint{5.165119in}{0.644830in}}%
\pgfpathlineto{\pgfqpoint{5.167267in}{0.639172in}}%
\pgfpathlineto{\pgfqpoint{5.171562in}{0.641810in}}%
\pgfpathlineto{\pgfqpoint{5.172636in}{0.641937in}}%
\pgfpathlineto{\pgfqpoint{5.173710in}{0.645148in}}%
\pgfpathlineto{\pgfqpoint{5.174784in}{0.645624in}}%
\pgfpathlineto{\pgfqpoint{5.175858in}{0.648485in}}%
\pgfpathlineto{\pgfqpoint{5.180153in}{0.649725in}}%
\pgfpathlineto{\pgfqpoint{5.181227in}{0.649312in}}%
\pgfpathlineto{\pgfqpoint{5.182301in}{0.650583in}}%
\pgfpathlineto{\pgfqpoint{5.183375in}{0.650519in}}%
\pgfpathlineto{\pgfqpoint{5.186596in}{0.651282in}}%
\pgfpathlineto{\pgfqpoint{5.187670in}{0.650138in}}%
\pgfpathlineto{\pgfqpoint{5.189818in}{0.652776in}}%
\pgfpathlineto{\pgfqpoint{5.190891in}{0.652268in}}%
\pgfpathlineto{\pgfqpoint{5.194113in}{0.654842in}}%
\pgfpathlineto{\pgfqpoint{5.196261in}{0.651409in}}%
\pgfpathlineto{\pgfqpoint{5.197335in}{0.646959in}}%
\pgfpathlineto{\pgfqpoint{5.198408in}{0.646959in}}%
\pgfpathlineto{\pgfqpoint{5.201630in}{0.648422in}}%
\pgfpathlineto{\pgfqpoint{5.202704in}{0.648008in}}%
\pgfpathlineto{\pgfqpoint{5.204851in}{0.649820in}}%
\pgfpathlineto{\pgfqpoint{5.205925in}{0.647023in}}%
\pgfpathlineto{\pgfqpoint{5.210221in}{0.646610in}}%
\pgfpathlineto{\pgfqpoint{5.211294in}{0.650202in}}%
\pgfpathlineto{\pgfqpoint{5.213442in}{0.654302in}}%
\pgfpathlineto{\pgfqpoint{5.216664in}{0.655605in}}%
\pgfpathlineto{\pgfqpoint{5.217737in}{0.657767in}}%
\pgfpathlineto{\pgfqpoint{5.218811in}{0.657671in}}%
\pgfpathlineto{\pgfqpoint{5.219885in}{0.658847in}}%
\pgfpathlineto{\pgfqpoint{5.224180in}{0.658116in}}%
\pgfpathlineto{\pgfqpoint{5.225254in}{0.655955in}}%
\pgfpathlineto{\pgfqpoint{5.226328in}{0.659960in}}%
\pgfpathlineto{\pgfqpoint{5.227402in}{0.658911in}}%
\pgfpathlineto{\pgfqpoint{5.228476in}{0.659006in}}%
\pgfpathlineto{\pgfqpoint{5.231697in}{0.658625in}}%
\pgfpathlineto{\pgfqpoint{5.234919in}{0.653507in}}%
\pgfpathlineto{\pgfqpoint{5.235993in}{0.655828in}}%
\pgfpathlineto{\pgfqpoint{5.239214in}{0.655605in}}%
\pgfpathlineto{\pgfqpoint{5.240288in}{0.657004in}}%
\pgfpathlineto{\pgfqpoint{5.241362in}{0.656495in}}%
\pgfpathlineto{\pgfqpoint{5.242436in}{0.658720in}}%
\pgfpathlineto{\pgfqpoint{5.243510in}{0.657036in}}%
\pgfpathlineto{\pgfqpoint{5.246731in}{0.659769in}}%
\pgfpathlineto{\pgfqpoint{5.247805in}{0.656972in}}%
\pgfpathlineto{\pgfqpoint{5.248879in}{0.659928in}}%
\pgfpathlineto{\pgfqpoint{5.249953in}{0.656018in}}%
\pgfpathlineto{\pgfqpoint{5.251026in}{0.654556in}}%
\pgfpathlineto{\pgfqpoint{5.254248in}{0.660023in}}%
\pgfpathlineto{\pgfqpoint{5.255322in}{0.658593in}}%
\pgfpathlineto{\pgfqpoint{5.256396in}{0.658116in}}%
\pgfpathlineto{\pgfqpoint{5.257469in}{0.654938in}}%
\pgfpathlineto{\pgfqpoint{5.258543in}{0.658879in}}%
\pgfpathlineto{\pgfqpoint{5.261765in}{0.660786in}}%
\pgfpathlineto{\pgfqpoint{5.262839in}{0.659928in}}%
\pgfpathlineto{\pgfqpoint{5.263912in}{0.661072in}}%
\pgfpathlineto{\pgfqpoint{5.266060in}{0.666031in}}%
\pgfpathlineto{\pgfqpoint{5.270356in}{0.668415in}}%
\pgfpathlineto{\pgfqpoint{5.271429in}{0.667684in}}%
\pgfpathlineto{\pgfqpoint{5.272503in}{0.669273in}}%
\pgfpathlineto{\pgfqpoint{5.273577in}{0.669591in}}%
\pgfpathlineto{\pgfqpoint{5.276799in}{0.669273in}}%
\pgfpathlineto{\pgfqpoint{5.277872in}{0.668192in}}%
\pgfpathlineto{\pgfqpoint{5.278946in}{0.669241in}}%
\pgfpathlineto{\pgfqpoint{5.280020in}{0.668828in}}%
\pgfpathlineto{\pgfqpoint{5.281094in}{0.667779in}}%
\pgfpathlineto{\pgfqpoint{5.286463in}{0.670894in}}%
\pgfpathlineto{\pgfqpoint{5.287537in}{0.668542in}}%
\pgfpathlineto{\pgfqpoint{5.288611in}{0.670894in}}%
\pgfpathlineto{\pgfqpoint{5.291832in}{0.669782in}}%
\pgfpathlineto{\pgfqpoint{5.292906in}{0.666953in}}%
\pgfpathlineto{\pgfqpoint{5.293980in}{0.666730in}}%
\pgfpathlineto{\pgfqpoint{5.295054in}{0.668065in}}%
\pgfpathlineto{\pgfqpoint{5.296128in}{0.667175in}}%
\pgfpathlineto{\pgfqpoint{5.303645in}{0.669591in}}%
\pgfpathlineto{\pgfqpoint{5.306866in}{0.666063in}}%
\pgfpathlineto{\pgfqpoint{5.307940in}{0.663170in}}%
\pgfpathlineto{\pgfqpoint{5.309014in}{0.666539in}}%
\pgfpathlineto{\pgfqpoint{5.310088in}{0.662471in}}%
\pgfpathlineto{\pgfqpoint{5.311161in}{0.664442in}}%
\pgfpathlineto{\pgfqpoint{5.315457in}{0.665300in}}%
\pgfpathlineto{\pgfqpoint{5.316531in}{0.660754in}}%
\pgfpathlineto{\pgfqpoint{5.317604in}{0.658688in}}%
\pgfpathlineto{\pgfqpoint{5.318678in}{0.663583in}}%
\pgfpathlineto{\pgfqpoint{5.321900in}{0.663901in}}%
\pgfpathlineto{\pgfqpoint{5.322974in}{0.659642in}}%
\pgfpathlineto{\pgfqpoint{5.324047in}{0.662566in}}%
\pgfpathlineto{\pgfqpoint{5.325121in}{0.655510in}}%
\pgfpathlineto{\pgfqpoint{5.326195in}{0.657163in}}%
\pgfpathlineto{\pgfqpoint{5.329417in}{0.650233in}}%
\pgfpathlineto{\pgfqpoint{5.330490in}{0.650901in}}%
\pgfpathlineto{\pgfqpoint{5.331564in}{0.644830in}}%
\pgfpathlineto{\pgfqpoint{5.332638in}{0.643844in}}%
\pgfpathlineto{\pgfqpoint{5.333712in}{0.649979in}}%
\pgfpathlineto{\pgfqpoint{5.336934in}{0.654969in}}%
\pgfpathlineto{\pgfqpoint{5.338007in}{0.660913in}}%
\pgfpathlineto{\pgfqpoint{5.339081in}{0.659610in}}%
\pgfpathlineto{\pgfqpoint{5.341229in}{0.664060in}}%
\pgfpathlineto{\pgfqpoint{5.344450in}{0.663583in}}%
\pgfpathlineto{\pgfqpoint{5.345524in}{0.667938in}}%
\pgfpathlineto{\pgfqpoint{5.346598in}{0.666762in}}%
\pgfpathlineto{\pgfqpoint{5.347672in}{0.668796in}}%
\pgfpathlineto{\pgfqpoint{5.348746in}{0.672197in}}%
\pgfpathlineto{\pgfqpoint{5.351967in}{0.673183in}}%
\pgfpathlineto{\pgfqpoint{5.353041in}{0.669178in}}%
\pgfpathlineto{\pgfqpoint{5.355189in}{0.674041in}}%
\pgfpathlineto{\pgfqpoint{5.356263in}{0.668129in}}%
\pgfpathlineto{\pgfqpoint{5.359484in}{0.667493in}}%
\pgfpathlineto{\pgfqpoint{5.360558in}{0.668097in}}%
\pgfpathlineto{\pgfqpoint{5.361632in}{0.667843in}}%
\pgfpathlineto{\pgfqpoint{5.363779in}{0.670513in}}%
\pgfpathlineto{\pgfqpoint{5.368075in}{0.668955in}}%
\pgfpathlineto{\pgfqpoint{5.369149in}{0.667620in}}%
\pgfpathlineto{\pgfqpoint{5.370223in}{0.664918in}}%
\pgfpathlineto{\pgfqpoint{5.371296in}{0.665077in}}%
\pgfpathlineto{\pgfqpoint{5.374518in}{0.669845in}}%
\pgfpathlineto{\pgfqpoint{5.375592in}{0.672992in}}%
\pgfpathlineto{\pgfqpoint{5.378813in}{0.675535in}}%
\pgfpathlineto{\pgfqpoint{5.382035in}{0.676075in}}%
\pgfpathlineto{\pgfqpoint{5.383109in}{0.678364in}}%
\pgfpathlineto{\pgfqpoint{5.384182in}{0.677283in}}%
\pgfpathlineto{\pgfqpoint{5.385256in}{0.677664in}}%
\pgfpathlineto{\pgfqpoint{5.386330in}{0.679222in}}%
\pgfpathlineto{\pgfqpoint{5.389552in}{0.679317in}}%
\pgfpathlineto{\pgfqpoint{5.391699in}{0.672928in}}%
\pgfpathlineto{\pgfqpoint{5.392773in}{0.676711in}}%
\pgfpathlineto{\pgfqpoint{5.397068in}{0.674200in}}%
\pgfpathlineto{\pgfqpoint{5.398142in}{0.672007in}}%
\pgfpathlineto{\pgfqpoint{5.400290in}{0.679286in}}%
\pgfpathlineto{\pgfqpoint{5.401364in}{0.680112in}}%
\pgfpathlineto{\pgfqpoint{5.405659in}{0.685484in}}%
\pgfpathlineto{\pgfqpoint{5.406733in}{0.684689in}}%
\pgfpathlineto{\pgfqpoint{5.408881in}{0.686501in}}%
\pgfpathlineto{\pgfqpoint{5.412102in}{0.687900in}}%
\pgfpathlineto{\pgfqpoint{5.414250in}{0.683990in}}%
\pgfpathlineto{\pgfqpoint{5.416398in}{0.682687in}}%
\pgfpathlineto{\pgfqpoint{5.420693in}{0.677156in}}%
\pgfpathlineto{\pgfqpoint{5.423914in}{0.684181in}}%
\pgfpathlineto{\pgfqpoint{5.427136in}{0.684816in}}%
\pgfpathlineto{\pgfqpoint{5.428210in}{0.686978in}}%
\pgfpathlineto{\pgfqpoint{5.429284in}{0.684117in}}%
\pgfpathlineto{\pgfqpoint{5.430357in}{0.684467in}}%
\pgfpathlineto{\pgfqpoint{5.431431in}{0.686946in}}%
\pgfpathlineto{\pgfqpoint{5.435727in}{0.685643in}}%
\pgfpathlineto{\pgfqpoint{5.436800in}{0.683926in}}%
\pgfpathlineto{\pgfqpoint{5.437874in}{0.686851in}}%
\pgfpathlineto{\pgfqpoint{5.438948in}{0.685579in}}%
\pgfpathlineto{\pgfqpoint{5.442170in}{0.686310in}}%
\pgfpathlineto{\pgfqpoint{5.444317in}{0.679476in}}%
\pgfpathlineto{\pgfqpoint{5.445391in}{0.681097in}}%
\pgfpathlineto{\pgfqpoint{5.446465in}{0.674359in}}%
\pgfpathlineto{\pgfqpoint{5.449687in}{0.677251in}}%
\pgfpathlineto{\pgfqpoint{5.450760in}{0.683736in}}%
\pgfpathlineto{\pgfqpoint{5.451834in}{0.705127in}}%
\pgfpathlineto{\pgfqpoint{5.452908in}{0.709164in}}%
\pgfpathlineto{\pgfqpoint{5.453982in}{0.707321in}}%
\pgfpathlineto{\pgfqpoint{5.457203in}{0.706462in}}%
\pgfpathlineto{\pgfqpoint{5.458277in}{0.707035in}}%
\pgfpathlineto{\pgfqpoint{5.459351in}{0.706876in}}%
\pgfpathlineto{\pgfqpoint{5.460425in}{0.711993in}}%
\pgfpathlineto{\pgfqpoint{5.461499in}{0.713741in}}%
\pgfpathlineto{\pgfqpoint{5.465794in}{0.713582in}}%
\pgfpathlineto{\pgfqpoint{5.466868in}{0.712820in}}%
\pgfpathlineto{\pgfqpoint{5.467942in}{0.712915in}}%
\pgfpathlineto{\pgfqpoint{5.469016in}{0.714886in}}%
\pgfpathlineto{\pgfqpoint{5.472237in}{0.716189in}}%
\pgfpathlineto{\pgfqpoint{5.473311in}{0.715235in}}%
\pgfpathlineto{\pgfqpoint{5.474385in}{0.717905in}}%
\pgfpathlineto{\pgfqpoint{5.476533in}{0.713487in}}%
\pgfpathlineto{\pgfqpoint{5.480828in}{0.720226in}}%
\pgfpathlineto{\pgfqpoint{5.484049in}{0.712692in}}%
\pgfpathlineto{\pgfqpoint{5.487271in}{0.716983in}}%
\pgfpathlineto{\pgfqpoint{5.488345in}{0.710531in}}%
\pgfpathlineto{\pgfqpoint{5.489419in}{0.709927in}}%
\pgfpathlineto{\pgfqpoint{5.490492in}{0.722673in}}%
\pgfpathlineto{\pgfqpoint{5.491566in}{0.720512in}}%
\pgfpathlineto{\pgfqpoint{5.494788in}{0.723277in}}%
\pgfpathlineto{\pgfqpoint{5.495862in}{0.722069in}}%
\pgfpathlineto{\pgfqpoint{5.496935in}{0.725057in}}%
\pgfpathlineto{\pgfqpoint{5.498009in}{0.723277in}}%
\pgfpathlineto{\pgfqpoint{5.499083in}{0.726424in}}%
\pgfpathlineto{\pgfqpoint{5.502305in}{0.725820in}}%
\pgfpathlineto{\pgfqpoint{5.503378in}{0.722514in}}%
\pgfpathlineto{\pgfqpoint{5.504452in}{0.716221in}}%
\pgfpathlineto{\pgfqpoint{5.509822in}{0.719558in}}%
\pgfpathlineto{\pgfqpoint{5.510895in}{0.715871in}}%
\pgfpathlineto{\pgfqpoint{5.513043in}{0.719208in}}%
\pgfpathlineto{\pgfqpoint{5.518412in}{0.717492in}}%
\pgfpathlineto{\pgfqpoint{5.520560in}{0.721497in}}%
\pgfpathlineto{\pgfqpoint{5.521634in}{0.722037in}}%
\pgfpathlineto{\pgfqpoint{5.524855in}{0.720702in}}%
\pgfpathlineto{\pgfqpoint{5.527003in}{0.722101in}}%
\pgfpathlineto{\pgfqpoint{5.528077in}{0.725438in}}%
\pgfpathlineto{\pgfqpoint{5.529151in}{0.721243in}}%
\pgfpathlineto{\pgfqpoint{5.532372in}{0.725820in}}%
\pgfpathlineto{\pgfqpoint{5.533446in}{0.724199in}}%
\pgfpathlineto{\pgfqpoint{5.534520in}{0.724962in}}%
\pgfpathlineto{\pgfqpoint{5.536667in}{0.729730in}}%
\pgfpathlineto{\pgfqpoint{5.539889in}{0.731605in}}%
\pgfpathlineto{\pgfqpoint{5.542037in}{0.730556in}}%
\pgfpathlineto{\pgfqpoint{5.543111in}{0.727314in}}%
\pgfpathlineto{\pgfqpoint{5.544184in}{0.732654in}}%
\pgfpathlineto{\pgfqpoint{5.547406in}{0.734180in}}%
\pgfpathlineto{\pgfqpoint{5.548480in}{0.733544in}}%
\pgfpathlineto{\pgfqpoint{5.549554in}{0.730270in}}%
\pgfpathlineto{\pgfqpoint{5.550627in}{0.728903in}}%
\pgfpathlineto{\pgfqpoint{5.551701in}{0.731446in}}%
\pgfpathlineto{\pgfqpoint{5.554923in}{0.726932in}}%
\pgfpathlineto{\pgfqpoint{5.555997in}{0.728840in}}%
\pgfpathlineto{\pgfqpoint{5.557070in}{0.728712in}}%
\pgfpathlineto{\pgfqpoint{5.559218in}{0.732018in}}%
\pgfpathlineto{\pgfqpoint{5.562440in}{0.732113in}}%
\pgfpathlineto{\pgfqpoint{5.563513in}{0.732781in}}%
\pgfpathlineto{\pgfqpoint{5.564587in}{0.731700in}}%
\pgfpathlineto{\pgfqpoint{5.565661in}{0.732272in}}%
\pgfpathlineto{\pgfqpoint{5.566735in}{0.731891in}}%
\pgfpathlineto{\pgfqpoint{5.571030in}{0.729443in}}%
\pgfpathlineto{\pgfqpoint{5.572104in}{0.732209in}}%
\pgfpathlineto{\pgfqpoint{5.573178in}{0.732686in}}%
\pgfpathlineto{\pgfqpoint{5.574252in}{0.732209in}}%
\pgfpathlineto{\pgfqpoint{5.577473in}{0.733989in}}%
\pgfpathlineto{\pgfqpoint{5.578547in}{0.733353in}}%
\pgfpathlineto{\pgfqpoint{5.579621in}{0.734593in}}%
\pgfpathlineto{\pgfqpoint{5.580695in}{0.732018in}}%
\pgfpathlineto{\pgfqpoint{5.581769in}{0.732018in}}%
\pgfpathlineto{\pgfqpoint{5.584990in}{0.728998in}}%
\pgfpathlineto{\pgfqpoint{5.586064in}{0.726710in}}%
\pgfpathlineto{\pgfqpoint{5.587138in}{0.731128in}}%
\pgfpathlineto{\pgfqpoint{5.588212in}{0.732972in}}%
\pgfpathlineto{\pgfqpoint{5.589286in}{0.730969in}}%
\pgfpathlineto{\pgfqpoint{5.592507in}{0.731637in}}%
\pgfpathlineto{\pgfqpoint{5.595729in}{0.740696in}}%
\pgfpathlineto{\pgfqpoint{5.596802in}{0.738916in}}%
\pgfpathlineto{\pgfqpoint{5.600024in}{0.741649in}}%
\pgfpathlineto{\pgfqpoint{5.601098in}{0.744256in}}%
\pgfpathlineto{\pgfqpoint{5.602172in}{0.742349in}}%
\pgfpathlineto{\pgfqpoint{5.604319in}{0.745972in}}%
\pgfpathlineto{\pgfqpoint{5.607541in}{0.740219in}}%
\pgfpathlineto{\pgfqpoint{5.609688in}{0.748388in}}%
\pgfpathlineto{\pgfqpoint{5.610762in}{0.747911in}}%
\pgfpathlineto{\pgfqpoint{5.615058in}{0.750104in}}%
\pgfpathlineto{\pgfqpoint{5.616132in}{0.754300in}}%
\pgfpathlineto{\pgfqpoint{5.617205in}{0.748579in}}%
\pgfpathlineto{\pgfqpoint{5.618279in}{0.749818in}}%
\pgfpathlineto{\pgfqpoint{5.619353in}{0.752329in}}%
\pgfpathlineto{\pgfqpoint{5.622575in}{0.757161in}}%
\pgfpathlineto{\pgfqpoint{5.623648in}{0.756557in}}%
\pgfpathlineto{\pgfqpoint{5.624722in}{0.757892in}}%
\pgfpathlineto{\pgfqpoint{5.625796in}{0.760212in}}%
\pgfpathlineto{\pgfqpoint{5.626870in}{0.759576in}}%
\pgfpathlineto{\pgfqpoint{5.630091in}{0.761738in}}%
\pgfpathlineto{\pgfqpoint{5.631165in}{0.760943in}}%
\pgfpathlineto{\pgfqpoint{5.632239in}{0.760975in}}%
\pgfpathlineto{\pgfqpoint{5.633313in}{0.759386in}}%
\pgfpathlineto{\pgfqpoint{5.634387in}{0.759735in}}%
\pgfpathlineto{\pgfqpoint{5.637608in}{0.757765in}}%
\pgfpathlineto{\pgfqpoint{5.638682in}{0.758369in}}%
\pgfpathlineto{\pgfqpoint{5.639756in}{0.762532in}}%
\pgfpathlineto{\pgfqpoint{5.641904in}{0.763009in}}%
\pgfpathlineto{\pgfqpoint{5.646199in}{0.768063in}}%
\pgfpathlineto{\pgfqpoint{5.647273in}{0.734625in}}%
\pgfpathlineto{\pgfqpoint{5.648347in}{0.728681in}}%
\pgfpathlineto{\pgfqpoint{5.649421in}{0.731065in}}%
\pgfpathlineto{\pgfqpoint{5.652642in}{0.736023in}}%
\pgfpathlineto{\pgfqpoint{5.653716in}{0.727028in}}%
\pgfpathlineto{\pgfqpoint{5.654790in}{0.724008in}}%
\pgfpathlineto{\pgfqpoint{5.655864in}{0.725597in}}%
\pgfpathlineto{\pgfqpoint{5.656937in}{0.724517in}}%
\pgfpathlineto{\pgfqpoint{5.660159in}{0.730175in}}%
\pgfpathlineto{\pgfqpoint{5.661233in}{0.723849in}}%
\pgfpathlineto{\pgfqpoint{5.662307in}{0.722387in}}%
\pgfpathlineto{\pgfqpoint{5.663380in}{0.703093in}}%
\pgfpathlineto{\pgfqpoint{5.667676in}{0.689139in}}%
\pgfpathlineto{\pgfqpoint{5.668750in}{0.690728in}}%
\pgfpathlineto{\pgfqpoint{5.670897in}{0.709546in}}%
\pgfpathlineto{\pgfqpoint{5.671971in}{0.710467in}}%
\pgfpathlineto{\pgfqpoint{5.675193in}{0.708687in}}%
\pgfpathlineto{\pgfqpoint{5.676266in}{0.701567in}}%
\pgfpathlineto{\pgfqpoint{5.677340in}{0.708719in}}%
\pgfpathlineto{\pgfqpoint{5.678414in}{0.709005in}}%
\pgfpathlineto{\pgfqpoint{5.679488in}{0.705954in}}%
\pgfpathlineto{\pgfqpoint{5.683783in}{0.715076in}}%
\pgfpathlineto{\pgfqpoint{5.684857in}{0.708783in}}%
\pgfpathlineto{\pgfqpoint{5.685931in}{0.710849in}}%
\pgfpathlineto{\pgfqpoint{5.687005in}{0.716475in}}%
\pgfpathlineto{\pgfqpoint{5.691300in}{0.713328in}}%
\pgfpathlineto{\pgfqpoint{5.693448in}{0.715648in}}%
\pgfpathlineto{\pgfqpoint{5.694522in}{0.711548in}}%
\pgfpathlineto{\pgfqpoint{5.697743in}{0.713265in}}%
\pgfpathlineto{\pgfqpoint{5.700965in}{0.704905in}}%
\pgfpathlineto{\pgfqpoint{5.702039in}{0.703951in}}%
\pgfpathlineto{\pgfqpoint{5.705260in}{0.698516in}}%
\pgfpathlineto{\pgfqpoint{5.706334in}{0.701313in}}%
\pgfpathlineto{\pgfqpoint{5.707408in}{0.709641in}}%
\pgfpathlineto{\pgfqpoint{5.709555in}{0.712025in}}%
\pgfpathlineto{\pgfqpoint{5.712777in}{0.714600in}}%
\pgfpathlineto{\pgfqpoint{5.713851in}{0.714345in}}%
\pgfpathlineto{\pgfqpoint{5.714925in}{0.713201in}}%
\pgfpathlineto{\pgfqpoint{5.717072in}{0.719717in}}%
\pgfpathlineto{\pgfqpoint{5.721368in}{0.722800in}}%
\pgfpathlineto{\pgfqpoint{5.722442in}{0.720226in}}%
\pgfpathlineto{\pgfqpoint{5.723515in}{0.726710in}}%
\pgfpathlineto{\pgfqpoint{5.729958in}{0.733290in}}%
\pgfpathlineto{\pgfqpoint{5.731032in}{0.742762in}}%
\pgfpathlineto{\pgfqpoint{5.732106in}{0.742285in}}%
\pgfpathlineto{\pgfqpoint{5.736401in}{0.744319in}}%
\pgfpathlineto{\pgfqpoint{5.738549in}{0.748134in}}%
\pgfpathlineto{\pgfqpoint{5.739623in}{0.744224in}}%
\pgfpathlineto{\pgfqpoint{5.743918in}{0.749627in}}%
\pgfpathlineto{\pgfqpoint{5.744992in}{0.742762in}}%
\pgfpathlineto{\pgfqpoint{5.746066in}{0.741999in}}%
\pgfpathlineto{\pgfqpoint{5.747140in}{0.750009in}}%
\pgfpathlineto{\pgfqpoint{5.750361in}{0.752266in}}%
\pgfpathlineto{\pgfqpoint{5.751435in}{0.755254in}}%
\pgfpathlineto{\pgfqpoint{5.752509in}{0.752552in}}%
\pgfpathlineto{\pgfqpoint{5.753583in}{0.751630in}}%
\pgfpathlineto{\pgfqpoint{5.754657in}{0.747530in}}%
\pgfpathlineto{\pgfqpoint{5.758952in}{0.751407in}}%
\pgfpathlineto{\pgfqpoint{5.760026in}{0.757415in}}%
\pgfpathlineto{\pgfqpoint{5.761100in}{0.759131in}}%
\pgfpathlineto{\pgfqpoint{5.762174in}{0.763200in}}%
\pgfpathlineto{\pgfqpoint{5.765395in}{0.761261in}}%
\pgfpathlineto{\pgfqpoint{5.766469in}{0.756843in}}%
\pgfpathlineto{\pgfqpoint{5.767543in}{0.759004in}}%
\pgfpathlineto{\pgfqpoint{5.769690in}{0.748388in}}%
\pgfpathlineto{\pgfqpoint{5.772912in}{0.743429in}}%
\pgfpathlineto{\pgfqpoint{5.773986in}{0.749182in}}%
\pgfpathlineto{\pgfqpoint{5.776133in}{0.738693in}}%
\pgfpathlineto{\pgfqpoint{5.777207in}{0.745718in}}%
\pgfpathlineto{\pgfqpoint{5.780429in}{0.744510in}}%
\pgfpathlineto{\pgfqpoint{5.782576in}{0.737422in}}%
\pgfpathlineto{\pgfqpoint{5.783650in}{0.737422in}}%
\pgfpathlineto{\pgfqpoint{5.784724in}{0.729221in}}%
\pgfpathlineto{\pgfqpoint{5.787946in}{0.733162in}}%
\pgfpathlineto{\pgfqpoint{5.790093in}{0.746576in}}%
\pgfpathlineto{\pgfqpoint{5.791167in}{0.741204in}}%
\pgfpathlineto{\pgfqpoint{5.792241in}{0.728267in}}%
\pgfpathlineto{\pgfqpoint{5.795463in}{0.724835in}}%
\pgfpathlineto{\pgfqpoint{5.796536in}{0.725311in}}%
\pgfpathlineto{\pgfqpoint{5.797610in}{0.721751in}}%
\pgfpathlineto{\pgfqpoint{5.802979in}{0.726837in}}%
\pgfpathlineto{\pgfqpoint{5.804053in}{0.726328in}}%
\pgfpathlineto{\pgfqpoint{5.805127in}{0.724103in}}%
\pgfpathlineto{\pgfqpoint{5.806201in}{0.720289in}}%
\pgfpathlineto{\pgfqpoint{5.810496in}{0.713964in}}%
\pgfpathlineto{\pgfqpoint{5.811570in}{0.707670in}}%
\pgfpathlineto{\pgfqpoint{5.814792in}{0.702712in}}%
\pgfpathlineto{\pgfqpoint{5.818013in}{0.704714in}}%
\pgfpathlineto{\pgfqpoint{5.819087in}{0.709387in}}%
\pgfpathlineto{\pgfqpoint{5.820161in}{0.700391in}}%
\pgfpathlineto{\pgfqpoint{5.821235in}{0.702267in}}%
\pgfpathlineto{\pgfqpoint{5.822309in}{0.686565in}}%
\pgfpathlineto{\pgfqpoint{5.826604in}{0.686787in}}%
\pgfpathlineto{\pgfqpoint{5.827678in}{0.682464in}}%
\pgfpathlineto{\pgfqpoint{5.828752in}{0.686914in}}%
\pgfpathlineto{\pgfqpoint{5.829825in}{0.695623in}}%
\pgfpathlineto{\pgfqpoint{5.833047in}{0.690760in}}%
\pgfpathlineto{\pgfqpoint{5.834121in}{0.693716in}}%
\pgfpathlineto{\pgfqpoint{5.835195in}{0.687836in}}%
\pgfpathlineto{\pgfqpoint{5.836268in}{0.685452in}}%
\pgfpathlineto{\pgfqpoint{5.837342in}{0.692350in}}%
\pgfpathlineto{\pgfqpoint{5.840564in}{0.690347in}}%
\pgfpathlineto{\pgfqpoint{5.841638in}{0.684212in}}%
\pgfpathlineto{\pgfqpoint{5.842711in}{0.690315in}}%
\pgfpathlineto{\pgfqpoint{5.843785in}{0.691173in}}%
\pgfpathlineto{\pgfqpoint{5.844859in}{0.686565in}}%
\pgfpathlineto{\pgfqpoint{5.848081in}{0.681193in}}%
\pgfpathlineto{\pgfqpoint{5.849154in}{0.681797in}}%
\pgfpathlineto{\pgfqpoint{5.850228in}{0.671339in}}%
\pgfpathlineto{\pgfqpoint{5.852376in}{0.678268in}}%
\pgfpathlineto{\pgfqpoint{5.856671in}{0.683577in}}%
\pgfpathlineto{\pgfqpoint{5.857745in}{0.691396in}}%
\pgfpathlineto{\pgfqpoint{5.859893in}{0.689902in}}%
\pgfpathlineto{\pgfqpoint{5.863114in}{0.694002in}}%
\pgfpathlineto{\pgfqpoint{5.864188in}{0.691015in}}%
\pgfpathlineto{\pgfqpoint{5.866336in}{0.691841in}}%
\pgfpathlineto{\pgfqpoint{5.867410in}{0.690824in}}%
\pgfpathlineto{\pgfqpoint{5.870631in}{0.691460in}}%
\pgfpathlineto{\pgfqpoint{5.871705in}{0.697880in}}%
\pgfpathlineto{\pgfqpoint{5.872779in}{0.695910in}}%
\pgfpathlineto{\pgfqpoint{5.873853in}{0.701408in}}%
\pgfpathlineto{\pgfqpoint{5.874927in}{0.700391in}}%
\pgfpathlineto{\pgfqpoint{5.878148in}{0.703125in}}%
\pgfpathlineto{\pgfqpoint{5.879222in}{0.698389in}}%
\pgfpathlineto{\pgfqpoint{5.880296in}{0.697912in}}%
\pgfpathlineto{\pgfqpoint{5.881370in}{0.696037in}}%
\pgfpathlineto{\pgfqpoint{5.882443in}{0.698738in}}%
\pgfpathlineto{\pgfqpoint{5.885665in}{0.701377in}}%
\pgfpathlineto{\pgfqpoint{5.886739in}{0.699660in}}%
\pgfpathlineto{\pgfqpoint{5.887813in}{0.700264in}}%
\pgfpathlineto{\pgfqpoint{5.888887in}{0.703761in}}%
\pgfpathlineto{\pgfqpoint{5.889960in}{0.702553in}}%
\pgfpathlineto{\pgfqpoint{5.893182in}{0.700328in}}%
\pgfpathlineto{\pgfqpoint{5.895330in}{0.695401in}}%
\pgfpathlineto{\pgfqpoint{5.896403in}{0.696577in}}%
\pgfpathlineto{\pgfqpoint{5.901773in}{0.699406in}}%
\pgfpathlineto{\pgfqpoint{5.903920in}{0.702871in}}%
\pgfpathlineto{\pgfqpoint{5.904994in}{0.702171in}}%
\pgfpathlineto{\pgfqpoint{5.908216in}{0.700995in}}%
\pgfpathlineto{\pgfqpoint{5.909289in}{0.695910in}}%
\pgfpathlineto{\pgfqpoint{5.910363in}{0.697372in}}%
\pgfpathlineto{\pgfqpoint{5.911437in}{0.693398in}}%
\pgfpathlineto{\pgfqpoint{5.912511in}{0.694161in}}%
\pgfpathlineto{\pgfqpoint{5.915732in}{0.693716in}}%
\pgfpathlineto{\pgfqpoint{5.916806in}{0.696958in}}%
\pgfpathlineto{\pgfqpoint{5.917880in}{0.703411in}}%
\pgfpathlineto{\pgfqpoint{5.918954in}{0.700836in}}%
\pgfpathlineto{\pgfqpoint{5.920028in}{0.700709in}}%
\pgfpathlineto{\pgfqpoint{5.925397in}{0.714822in}}%
\pgfpathlineto{\pgfqpoint{5.926471in}{0.713741in}}%
\pgfpathlineto{\pgfqpoint{5.927545in}{0.716348in}}%
\pgfpathlineto{\pgfqpoint{5.932914in}{0.720893in}}%
\pgfpathlineto{\pgfqpoint{5.935062in}{0.714790in}}%
\pgfpathlineto{\pgfqpoint{5.938283in}{0.718128in}}%
\pgfpathlineto{\pgfqpoint{5.939357in}{0.716380in}}%
\pgfpathlineto{\pgfqpoint{5.940431in}{0.716030in}}%
\pgfpathlineto{\pgfqpoint{5.942578in}{0.721688in}}%
\pgfpathlineto{\pgfqpoint{5.945800in}{0.721084in}}%
\pgfpathlineto{\pgfqpoint{5.946874in}{0.724866in}}%
\pgfpathlineto{\pgfqpoint{5.947948in}{0.711866in}}%
\pgfpathlineto{\pgfqpoint{5.949021in}{0.710118in}}%
\pgfpathlineto{\pgfqpoint{5.950095in}{0.706526in}}%
\pgfpathlineto{\pgfqpoint{5.953317in}{0.706049in}}%
\pgfpathlineto{\pgfqpoint{5.954391in}{0.704778in}}%
\pgfpathlineto{\pgfqpoint{5.956538in}{0.700169in}}%
\pgfpathlineto{\pgfqpoint{5.957612in}{0.704301in}}%
\pgfpathlineto{\pgfqpoint{5.960834in}{0.702489in}}%
\pgfpathlineto{\pgfqpoint{5.962981in}{0.704555in}}%
\pgfpathlineto{\pgfqpoint{5.964055in}{0.704396in}}%
\pgfpathlineto{\pgfqpoint{5.965129in}{0.705827in}}%
\pgfpathlineto{\pgfqpoint{5.969424in}{0.702616in}}%
\pgfpathlineto{\pgfqpoint{5.970498in}{0.700487in}}%
\pgfpathlineto{\pgfqpoint{5.972646in}{0.701186in}}%
\pgfpathlineto{\pgfqpoint{5.975867in}{0.701281in}}%
\pgfpathlineto{\pgfqpoint{5.978015in}{0.699056in}}%
\pgfpathlineto{\pgfqpoint{5.979089in}{0.698452in}}%
\pgfpathlineto{\pgfqpoint{5.980163in}{0.696927in}}%
\pgfpathlineto{\pgfqpoint{5.983384in}{0.697626in}}%
\pgfpathlineto{\pgfqpoint{5.984458in}{0.700137in}}%
\pgfpathlineto{\pgfqpoint{5.985532in}{0.699756in}}%
\pgfpathlineto{\pgfqpoint{5.986606in}{0.700073in}}%
\pgfpathlineto{\pgfqpoint{5.987680in}{0.701949in}}%
\pgfpathlineto{\pgfqpoint{5.990901in}{0.703665in}}%
\pgfpathlineto{\pgfqpoint{5.991975in}{0.701408in}}%
\pgfpathlineto{\pgfqpoint{5.993049in}{0.701313in}}%
\pgfpathlineto{\pgfqpoint{5.994123in}{0.702012in}}%
\pgfpathlineto{\pgfqpoint{5.995197in}{0.692063in}}%
\pgfpathlineto{\pgfqpoint{5.998418in}{0.688027in}}%
\pgfpathlineto{\pgfqpoint{6.000566in}{0.695846in}}%
\pgfpathlineto{\pgfqpoint{6.001640in}{0.698389in}}%
\pgfpathlineto{\pgfqpoint{6.002713in}{0.699025in}}%
\pgfpathlineto{\pgfqpoint{6.007009in}{0.697912in}}%
\pgfpathlineto{\pgfqpoint{6.010230in}{0.706017in}}%
\pgfpathlineto{\pgfqpoint{6.014526in}{0.707766in}}%
\pgfpathlineto{\pgfqpoint{6.015599in}{0.706780in}}%
\pgfpathlineto{\pgfqpoint{6.016673in}{0.707066in}}%
\pgfpathlineto{\pgfqpoint{6.017747in}{0.706558in}}%
\pgfpathlineto{\pgfqpoint{6.020969in}{0.707607in}}%
\pgfpathlineto{\pgfqpoint{6.022042in}{0.705541in}}%
\pgfpathlineto{\pgfqpoint{6.023116in}{0.701758in}}%
\pgfpathlineto{\pgfqpoint{6.028486in}{0.699215in}}%
\pgfpathlineto{\pgfqpoint{6.030633in}{0.696005in}}%
\pgfpathlineto{\pgfqpoint{6.031707in}{0.694733in}}%
\pgfpathlineto{\pgfqpoint{6.032781in}{0.694861in}}%
\pgfpathlineto{\pgfqpoint{6.036002in}{0.693621in}}%
\pgfpathlineto{\pgfqpoint{6.037076in}{0.692000in}}%
\pgfpathlineto{\pgfqpoint{6.038150in}{0.695274in}}%
\pgfpathlineto{\pgfqpoint{6.039224in}{0.692445in}}%
\pgfpathlineto{\pgfqpoint{6.040298in}{0.694479in}}%
\pgfpathlineto{\pgfqpoint{6.043519in}{0.694257in}}%
\pgfpathlineto{\pgfqpoint{6.045667in}{0.700646in}}%
\pgfpathlineto{\pgfqpoint{6.046741in}{0.700391in}}%
\pgfpathlineto{\pgfqpoint{6.047815in}{0.697562in}}%
\pgfpathlineto{\pgfqpoint{6.051036in}{0.698357in}}%
\pgfpathlineto{\pgfqpoint{6.055331in}{0.696196in}}%
\pgfpathlineto{\pgfqpoint{6.058553in}{0.694606in}}%
\pgfpathlineto{\pgfqpoint{6.059627in}{0.694924in}}%
\pgfpathlineto{\pgfqpoint{6.061775in}{0.693621in}}%
\pgfpathlineto{\pgfqpoint{6.062848in}{0.692604in}}%
\pgfpathlineto{\pgfqpoint{6.068218in}{0.690315in}}%
\pgfpathlineto{\pgfqpoint{6.069291in}{0.689711in}}%
\pgfpathlineto{\pgfqpoint{6.070365in}{0.690188in}}%
\pgfpathlineto{\pgfqpoint{6.075734in}{0.688058in}}%
\pgfpathlineto{\pgfqpoint{6.076808in}{0.689203in}}%
\pgfpathlineto{\pgfqpoint{6.077882in}{0.684117in}}%
\pgfpathlineto{\pgfqpoint{6.081104in}{0.687836in}}%
\pgfpathlineto{\pgfqpoint{6.083251in}{0.683640in}}%
\pgfpathlineto{\pgfqpoint{6.085399in}{0.684562in}}%
\pgfpathlineto{\pgfqpoint{6.088620in}{0.684753in}}%
\pgfpathlineto{\pgfqpoint{6.089694in}{0.685738in}}%
\pgfpathlineto{\pgfqpoint{6.090768in}{0.684022in}}%
\pgfpathlineto{\pgfqpoint{6.091842in}{0.687137in}}%
\pgfpathlineto{\pgfqpoint{6.092916in}{0.686723in}}%
\pgfpathlineto{\pgfqpoint{6.097211in}{0.681987in}}%
\pgfpathlineto{\pgfqpoint{6.098285in}{0.683450in}}%
\pgfpathlineto{\pgfqpoint{6.099359in}{0.682242in}}%
\pgfpathlineto{\pgfqpoint{6.100433in}{0.685452in}}%
\pgfpathlineto{\pgfqpoint{6.103654in}{0.684340in}}%
\pgfpathlineto{\pgfqpoint{6.104728in}{0.684657in}}%
\pgfpathlineto{\pgfqpoint{6.105802in}{0.684212in}}%
\pgfpathlineto{\pgfqpoint{6.106876in}{0.685357in}}%
\pgfpathlineto{\pgfqpoint{6.107950in}{0.684340in}}%
\pgfpathlineto{\pgfqpoint{6.111171in}{0.684340in}}%
\pgfpathlineto{\pgfqpoint{6.113319in}{0.681066in}}%
\pgfpathlineto{\pgfqpoint{6.114393in}{0.680176in}}%
\pgfpathlineto{\pgfqpoint{6.115466in}{0.680716in}}%
\pgfpathlineto{\pgfqpoint{6.118688in}{0.679286in}}%
\pgfpathlineto{\pgfqpoint{6.119762in}{0.680335in}}%
\pgfpathlineto{\pgfqpoint{6.120836in}{0.682623in}}%
\pgfpathlineto{\pgfqpoint{6.121909in}{0.682941in}}%
\pgfpathlineto{\pgfqpoint{6.122983in}{0.685992in}}%
\pgfpathlineto{\pgfqpoint{6.126205in}{0.687010in}}%
\pgfpathlineto{\pgfqpoint{6.127279in}{0.685134in}}%
\pgfpathlineto{\pgfqpoint{6.129426in}{0.688980in}}%
\pgfpathlineto{\pgfqpoint{6.130500in}{0.688472in}}%
\pgfpathlineto{\pgfqpoint{6.135869in}{0.682591in}}%
\pgfpathlineto{\pgfqpoint{6.136943in}{0.687010in}}%
\pgfpathlineto{\pgfqpoint{6.138017in}{0.684212in}}%
\pgfpathlineto{\pgfqpoint{6.141239in}{0.690220in}}%
\pgfpathlineto{\pgfqpoint{6.142312in}{0.690093in}}%
\pgfpathlineto{\pgfqpoint{6.144460in}{0.691841in}}%
\pgfpathlineto{\pgfqpoint{6.145534in}{0.700105in}}%
\pgfpathlineto{\pgfqpoint{6.148755in}{0.700836in}}%
\pgfpathlineto{\pgfqpoint{6.149829in}{0.700169in}}%
\pgfpathlineto{\pgfqpoint{6.150903in}{0.704492in}}%
\pgfpathlineto{\pgfqpoint{6.151977in}{0.705255in}}%
\pgfpathlineto{\pgfqpoint{6.153051in}{0.701822in}}%
\pgfpathlineto{\pgfqpoint{6.156272in}{0.699946in}}%
\pgfpathlineto{\pgfqpoint{6.157346in}{0.700201in}}%
\pgfpathlineto{\pgfqpoint{6.158420in}{0.701885in}}%
\pgfpathlineto{\pgfqpoint{6.160568in}{0.703570in}}%
\pgfpathlineto{\pgfqpoint{6.163789in}{0.704015in}}%
\pgfpathlineto{\pgfqpoint{6.164863in}{0.706145in}}%
\pgfpathlineto{\pgfqpoint{6.165937in}{0.704492in}}%
\pgfpathlineto{\pgfqpoint{6.167011in}{0.703951in}}%
\pgfpathlineto{\pgfqpoint{6.168085in}{0.702616in}}%
\pgfpathlineto{\pgfqpoint{6.172380in}{0.709164in}}%
\pgfpathlineto{\pgfqpoint{6.173454in}{0.713201in}}%
\pgfpathlineto{\pgfqpoint{6.175601in}{0.724390in}}%
\pgfpathlineto{\pgfqpoint{6.179897in}{0.721275in}}%
\pgfpathlineto{\pgfqpoint{6.182044in}{0.722927in}}%
\pgfpathlineto{\pgfqpoint{6.183118in}{0.721465in}}%
\pgfpathlineto{\pgfqpoint{6.187414in}{0.726233in}}%
\pgfpathlineto{\pgfqpoint{6.188487in}{0.726519in}}%
\pgfpathlineto{\pgfqpoint{6.190635in}{0.725280in}}%
\pgfpathlineto{\pgfqpoint{6.194930in}{0.725343in}}%
\pgfpathlineto{\pgfqpoint{6.196004in}{0.722673in}}%
\pgfpathlineto{\pgfqpoint{6.197078in}{0.723468in}}%
\pgfpathlineto{\pgfqpoint{6.198152in}{0.722419in}}%
\pgfpathlineto{\pgfqpoint{6.202447in}{0.728108in}}%
\pgfpathlineto{\pgfqpoint{6.203521in}{0.732272in}}%
\pgfpathlineto{\pgfqpoint{6.204595in}{0.732113in}}%
\pgfpathlineto{\pgfqpoint{6.205669in}{0.737009in}}%
\pgfpathlineto{\pgfqpoint{6.208890in}{0.735101in}}%
\pgfpathlineto{\pgfqpoint{6.209964in}{0.735165in}}%
\pgfpathlineto{\pgfqpoint{6.211038in}{0.738407in}}%
\pgfpathlineto{\pgfqpoint{6.212112in}{0.732558in}}%
\pgfpathlineto{\pgfqpoint{6.213186in}{0.734180in}}%
\pgfpathlineto{\pgfqpoint{6.217481in}{0.733894in}}%
\pgfpathlineto{\pgfqpoint{6.218555in}{0.734497in}}%
\pgfpathlineto{\pgfqpoint{6.219629in}{0.731796in}}%
\pgfpathlineto{\pgfqpoint{6.220703in}{0.732972in}}%
\pgfpathlineto{\pgfqpoint{6.223924in}{0.731287in}}%
\pgfpathlineto{\pgfqpoint{6.224998in}{0.733703in}}%
\pgfpathlineto{\pgfqpoint{6.227146in}{0.734180in}}%
\pgfpathlineto{\pgfqpoint{6.228219in}{0.737962in}}%
\pgfpathlineto{\pgfqpoint{6.231441in}{0.742984in}}%
\pgfpathlineto{\pgfqpoint{6.232515in}{0.742126in}}%
\pgfpathlineto{\pgfqpoint{6.233589in}{0.744097in}}%
\pgfpathlineto{\pgfqpoint{6.235736in}{0.741045in}}%
\pgfpathlineto{\pgfqpoint{6.238958in}{0.738820in}}%
\pgfpathlineto{\pgfqpoint{6.240032in}{0.737072in}}%
\pgfpathlineto{\pgfqpoint{6.241106in}{0.737072in}}%
\pgfpathlineto{\pgfqpoint{6.242179in}{0.738566in}}%
\pgfpathlineto{\pgfqpoint{6.243253in}{0.737867in}}%
\pgfpathlineto{\pgfqpoint{6.246475in}{0.739043in}}%
\pgfpathlineto{\pgfqpoint{6.247549in}{0.741141in}}%
\pgfpathlineto{\pgfqpoint{6.248622in}{0.740664in}}%
\pgfpathlineto{\pgfqpoint{6.249696in}{0.742285in}}%
\pgfpathlineto{\pgfqpoint{6.250770in}{0.740314in}}%
\pgfpathlineto{\pgfqpoint{6.256139in}{0.740473in}}%
\pgfpathlineto{\pgfqpoint{6.257213in}{0.739297in}}%
\pgfpathlineto{\pgfqpoint{6.258287in}{0.741109in}}%
\pgfpathlineto{\pgfqpoint{6.262582in}{0.740410in}}%
\pgfpathlineto{\pgfqpoint{6.263656in}{0.743302in}}%
\pgfpathlineto{\pgfqpoint{6.264730in}{0.741935in}}%
\pgfpathlineto{\pgfqpoint{6.265804in}{0.743906in}}%
\pgfpathlineto{\pgfqpoint{6.269025in}{0.742158in}}%
\pgfpathlineto{\pgfqpoint{6.270099in}{0.742762in}}%
\pgfpathlineto{\pgfqpoint{6.271173in}{0.742698in}}%
\pgfpathlineto{\pgfqpoint{6.272247in}{0.743270in}}%
\pgfpathlineto{\pgfqpoint{6.273321in}{0.742952in}}%
\pgfpathlineto{\pgfqpoint{6.276542in}{0.744764in}}%
\pgfpathlineto{\pgfqpoint{6.277616in}{0.747180in}}%
\pgfpathlineto{\pgfqpoint{6.279764in}{0.745368in}}%
\pgfpathlineto{\pgfqpoint{6.280838in}{0.745495in}}%
\pgfpathlineto{\pgfqpoint{6.284059in}{0.748420in}}%
\pgfpathlineto{\pgfqpoint{6.285133in}{0.745495in}}%
\pgfpathlineto{\pgfqpoint{6.287281in}{0.746989in}}%
\pgfpathlineto{\pgfqpoint{6.288354in}{0.746671in}}%
\pgfpathlineto{\pgfqpoint{6.291576in}{0.747402in}}%
\pgfpathlineto{\pgfqpoint{6.292650in}{0.749469in}}%
\pgfpathlineto{\pgfqpoint{6.293724in}{0.747879in}}%
\pgfpathlineto{\pgfqpoint{6.294797in}{0.749850in}}%
\pgfpathlineto{\pgfqpoint{6.295871in}{0.750517in}}%
\pgfpathlineto{\pgfqpoint{6.302314in}{0.749469in}}%
\pgfpathlineto{\pgfqpoint{6.303388in}{0.748006in}}%
\pgfpathlineto{\pgfqpoint{6.306610in}{0.747561in}}%
\pgfpathlineto{\pgfqpoint{6.307684in}{0.749532in}}%
\pgfpathlineto{\pgfqpoint{6.308757in}{0.749437in}}%
\pgfpathlineto{\pgfqpoint{6.314127in}{0.751694in}}%
\pgfpathlineto{\pgfqpoint{6.315200in}{0.752965in}}%
\pgfpathlineto{\pgfqpoint{6.316274in}{0.751535in}}%
\pgfpathlineto{\pgfqpoint{6.317348in}{0.754777in}}%
\pgfpathlineto{\pgfqpoint{6.318422in}{0.753728in}}%
\pgfpathlineto{\pgfqpoint{6.321643in}{0.751439in}}%
\pgfpathlineto{\pgfqpoint{6.322717in}{0.755985in}}%
\pgfpathlineto{\pgfqpoint{6.324865in}{0.758019in}}%
\pgfpathlineto{\pgfqpoint{6.329160in}{0.754618in}}%
\pgfpathlineto{\pgfqpoint{6.330234in}{0.753505in}}%
\pgfpathlineto{\pgfqpoint{6.331308in}{0.745082in}}%
\pgfpathlineto{\pgfqpoint{6.332382in}{0.743715in}}%
\pgfpathlineto{\pgfqpoint{6.333456in}{0.746226in}}%
\pgfpathlineto{\pgfqpoint{6.336677in}{0.744478in}}%
\pgfpathlineto{\pgfqpoint{6.337751in}{0.746449in}}%
\pgfpathlineto{\pgfqpoint{6.338825in}{0.739075in}}%
\pgfpathlineto{\pgfqpoint{6.339899in}{0.738820in}}%
\pgfpathlineto{\pgfqpoint{6.340973in}{0.739170in}}%
\pgfpathlineto{\pgfqpoint{6.344194in}{0.737454in}}%
\pgfpathlineto{\pgfqpoint{6.346342in}{0.728808in}}%
\pgfpathlineto{\pgfqpoint{6.347416in}{0.729952in}}%
\pgfpathlineto{\pgfqpoint{6.348489in}{0.732527in}}%
\pgfpathlineto{\pgfqpoint{6.351711in}{0.732876in}}%
\pgfpathlineto{\pgfqpoint{6.352785in}{0.731001in}}%
\pgfpathlineto{\pgfqpoint{6.353859in}{0.733099in}}%
\pgfpathlineto{\pgfqpoint{6.354932in}{0.731764in}}%
\pgfpathlineto{\pgfqpoint{6.356006in}{0.735260in}}%
\pgfpathlineto{\pgfqpoint{6.360302in}{0.735038in}}%
\pgfpathlineto{\pgfqpoint{6.361375in}{0.733798in}}%
\pgfpathlineto{\pgfqpoint{6.362449in}{0.734625in}}%
\pgfpathlineto{\pgfqpoint{6.363523in}{0.731478in}}%
\pgfpathlineto{\pgfqpoint{6.366745in}{0.729475in}}%
\pgfpathlineto{\pgfqpoint{6.367818in}{0.726328in}}%
\pgfpathlineto{\pgfqpoint{6.368892in}{0.727632in}}%
\pgfpathlineto{\pgfqpoint{6.369966in}{0.722737in}}%
\pgfpathlineto{\pgfqpoint{6.371040in}{0.726710in}}%
\pgfpathlineto{\pgfqpoint{6.374262in}{0.731065in}}%
\pgfpathlineto{\pgfqpoint{6.376409in}{0.728299in}}%
\pgfpathlineto{\pgfqpoint{6.377483in}{0.727822in}}%
\pgfpathlineto{\pgfqpoint{6.378557in}{0.726360in}}%
\pgfpathlineto{\pgfqpoint{6.381778in}{0.725947in}}%
\pgfpathlineto{\pgfqpoint{6.382852in}{0.721561in}}%
\pgfpathlineto{\pgfqpoint{6.383926in}{0.724199in}}%
\pgfpathlineto{\pgfqpoint{6.385000in}{0.722419in}}%
\pgfpathlineto{\pgfqpoint{6.386074in}{0.722864in}}%
\pgfpathlineto{\pgfqpoint{6.389295in}{0.726551in}}%
\pgfpathlineto{\pgfqpoint{6.390369in}{0.726074in}}%
\pgfpathlineto{\pgfqpoint{6.391443in}{0.730492in}}%
\pgfpathlineto{\pgfqpoint{6.392517in}{0.726996in}}%
\pgfpathlineto{\pgfqpoint{6.393591in}{0.728649in}}%
\pgfpathlineto{\pgfqpoint{6.396812in}{0.732336in}}%
\pgfpathlineto{\pgfqpoint{6.398960in}{0.726742in}}%
\pgfpathlineto{\pgfqpoint{6.400034in}{0.722101in}}%
\pgfpathlineto{\pgfqpoint{6.401107in}{0.722006in}}%
\pgfpathlineto{\pgfqpoint{6.405403in}{0.723627in}}%
\pgfpathlineto{\pgfqpoint{6.406477in}{0.725280in}}%
\pgfpathlineto{\pgfqpoint{6.407551in}{0.725025in}}%
\pgfpathlineto{\pgfqpoint{6.408624in}{0.727473in}}%
\pgfpathlineto{\pgfqpoint{6.411846in}{0.726551in}}%
\pgfpathlineto{\pgfqpoint{6.415067in}{0.734497in}}%
\pgfpathlineto{\pgfqpoint{6.416141in}{0.733639in}}%
\pgfpathlineto{\pgfqpoint{6.419363in}{0.733385in}}%
\pgfpathlineto{\pgfqpoint{6.420437in}{0.731573in}}%
\pgfpathlineto{\pgfqpoint{6.421510in}{0.733194in}}%
\pgfpathlineto{\pgfqpoint{6.422584in}{0.742635in}}%
\pgfpathlineto{\pgfqpoint{6.426880in}{0.742412in}}%
\pgfpathlineto{\pgfqpoint{6.427953in}{0.744510in}}%
\pgfpathlineto{\pgfqpoint{6.429027in}{0.738534in}}%
\pgfpathlineto{\pgfqpoint{6.430101in}{0.739901in}}%
\pgfpathlineto{\pgfqpoint{6.431175in}{0.735515in}}%
\pgfpathlineto{\pgfqpoint{6.434396in}{0.731382in}}%
\pgfpathlineto{\pgfqpoint{6.435470in}{0.733321in}}%
\pgfpathlineto{\pgfqpoint{6.436544in}{0.720512in}}%
\pgfpathlineto{\pgfqpoint{6.437618in}{0.715935in}}%
\pgfpathlineto{\pgfqpoint{6.438692in}{0.717905in}}%
\pgfpathlineto{\pgfqpoint{6.441913in}{0.716093in}}%
\pgfpathlineto{\pgfqpoint{6.442987in}{0.716443in}}%
\pgfpathlineto{\pgfqpoint{6.444061in}{0.718573in}}%
\pgfpathlineto{\pgfqpoint{6.446209in}{0.713932in}}%
\pgfpathlineto{\pgfqpoint{6.449430in}{0.715394in}}%
\pgfpathlineto{\pgfqpoint{6.450504in}{0.720321in}}%
\pgfpathlineto{\pgfqpoint{6.451578in}{0.716411in}}%
\pgfpathlineto{\pgfqpoint{6.452652in}{0.716475in}}%
\pgfpathlineto{\pgfqpoint{6.453726in}{0.719208in}}%
\pgfpathlineto{\pgfqpoint{6.458021in}{0.719717in}}%
\pgfpathlineto{\pgfqpoint{6.459095in}{0.720639in}}%
\pgfpathlineto{\pgfqpoint{6.460169in}{0.715490in}}%
\pgfpathlineto{\pgfqpoint{6.461242in}{0.716411in}}%
\pgfpathlineto{\pgfqpoint{6.466612in}{0.716411in}}%
\pgfpathlineto{\pgfqpoint{6.467685in}{0.702712in}}%
\pgfpathlineto{\pgfqpoint{6.471981in}{0.702807in}}%
\pgfpathlineto{\pgfqpoint{6.474129in}{0.708147in}}%
\pgfpathlineto{\pgfqpoint{6.475202in}{0.705286in}}%
\pgfpathlineto{\pgfqpoint{6.476276in}{0.707225in}}%
\pgfpathlineto{\pgfqpoint{6.479498in}{0.705922in}}%
\pgfpathlineto{\pgfqpoint{6.480572in}{0.706939in}}%
\pgfpathlineto{\pgfqpoint{6.481645in}{0.709355in}}%
\pgfpathlineto{\pgfqpoint{6.483793in}{0.707448in}}%
\pgfpathlineto{\pgfqpoint{6.487015in}{0.710467in}}%
\pgfpathlineto{\pgfqpoint{6.488088in}{0.707543in}}%
\pgfpathlineto{\pgfqpoint{6.489162in}{0.709418in}}%
\pgfpathlineto{\pgfqpoint{6.490236in}{0.705763in}}%
\pgfpathlineto{\pgfqpoint{6.495605in}{0.714218in}}%
\pgfpathlineto{\pgfqpoint{6.502048in}{0.710467in}}%
\pgfpathlineto{\pgfqpoint{6.503122in}{0.710467in}}%
\pgfpathlineto{\pgfqpoint{6.504196in}{0.707321in}}%
\pgfpathlineto{\pgfqpoint{6.505270in}{0.702298in}}%
\pgfpathlineto{\pgfqpoint{6.506344in}{0.703697in}}%
\pgfpathlineto{\pgfqpoint{6.510639in}{0.706717in}}%
\pgfpathlineto{\pgfqpoint{6.511713in}{0.706367in}}%
\pgfpathlineto{\pgfqpoint{6.513861in}{0.709927in}}%
\pgfpathlineto{\pgfqpoint{6.518156in}{0.706494in}}%
\pgfpathlineto{\pgfqpoint{6.519230in}{0.705000in}}%
\pgfpathlineto{\pgfqpoint{6.520304in}{0.707321in}}%
\pgfpathlineto{\pgfqpoint{6.521377in}{0.706558in}}%
\pgfpathlineto{\pgfqpoint{6.525673in}{0.705032in}}%
\pgfpathlineto{\pgfqpoint{6.526747in}{0.708783in}}%
\pgfpathlineto{\pgfqpoint{6.527820in}{0.706685in}}%
\pgfpathlineto{\pgfqpoint{6.528894in}{0.707575in}}%
\pgfpathlineto{\pgfqpoint{6.533190in}{0.716729in}}%
\pgfpathlineto{\pgfqpoint{6.534263in}{0.715426in}}%
\pgfpathlineto{\pgfqpoint{6.536411in}{0.726519in}}%
\pgfpathlineto{\pgfqpoint{6.539633in}{0.726392in}}%
\pgfpathlineto{\pgfqpoint{6.540706in}{0.721561in}}%
\pgfpathlineto{\pgfqpoint{6.541780in}{0.723150in}}%
\pgfpathlineto{\pgfqpoint{6.543928in}{0.722387in}}%
\pgfpathlineto{\pgfqpoint{6.547150in}{0.720257in}}%
\pgfpathlineto{\pgfqpoint{6.548223in}{0.721020in}}%
\pgfpathlineto{\pgfqpoint{6.549297in}{0.720226in}}%
\pgfpathlineto{\pgfqpoint{6.551445in}{0.719908in}}%
\pgfpathlineto{\pgfqpoint{6.554666in}{0.720385in}}%
\pgfpathlineto{\pgfqpoint{6.555740in}{0.722292in}}%
\pgfpathlineto{\pgfqpoint{6.556814in}{0.727950in}}%
\pgfpathlineto{\pgfqpoint{6.557888in}{0.726646in}}%
\pgfpathlineto{\pgfqpoint{6.558962in}{0.727981in}}%
\pgfpathlineto{\pgfqpoint{6.562183in}{0.743302in}}%
\pgfpathlineto{\pgfqpoint{6.564331in}{0.728617in}}%
\pgfpathlineto{\pgfqpoint{6.566479in}{0.727409in}}%
\pgfpathlineto{\pgfqpoint{6.570774in}{0.737358in}}%
\pgfpathlineto{\pgfqpoint{6.571848in}{0.737930in}}%
\pgfpathlineto{\pgfqpoint{6.572922in}{0.747116in}}%
\pgfpathlineto{\pgfqpoint{6.573995in}{0.749310in}}%
\pgfpathlineto{\pgfqpoint{6.577217in}{0.748547in}}%
\pgfpathlineto{\pgfqpoint{6.578291in}{0.750994in}}%
\pgfpathlineto{\pgfqpoint{6.579365in}{0.744383in}}%
\pgfpathlineto{\pgfqpoint{6.580439in}{0.744033in}}%
\pgfpathlineto{\pgfqpoint{6.581512in}{0.741236in}}%
\pgfpathlineto{\pgfqpoint{6.585808in}{0.739520in}}%
\pgfpathlineto{\pgfqpoint{6.586882in}{0.738026in}}%
\pgfpathlineto{\pgfqpoint{6.587955in}{0.738407in}}%
\pgfpathlineto{\pgfqpoint{6.589029in}{0.737612in}}%
\pgfpathlineto{\pgfqpoint{6.589029in}{0.737612in}}%
\pgfusepath{stroke}%
\end{pgfscope}%
\begin{pgfscope}%
\pgfpathrectangle{\pgfqpoint{4.123120in}{0.385400in}}{\pgfqpoint{2.583333in}{0.400885in}}%
\pgfusepath{clip}%
\pgfsetroundcap%
\pgfsetroundjoin%
\pgfsetlinewidth{1.505625pt}%
\definecolor{currentstroke}{rgb}{0.090196,0.745098,0.811765}%
\pgfsetstrokecolor{currentstroke}%
\pgfsetdash{}{0pt}%
\pgfpathmoveto{\pgfqpoint{4.240544in}{0.512982in}}%
\pgfpathlineto{\pgfqpoint{4.243766in}{0.517560in}}%
\pgfpathlineto{\pgfqpoint{4.248061in}{0.516765in}}%
\pgfpathlineto{\pgfqpoint{4.249135in}{0.514095in}}%
\pgfpathlineto{\pgfqpoint{4.250209in}{0.514159in}}%
\pgfpathlineto{\pgfqpoint{4.251283in}{0.513237in}}%
\pgfpathlineto{\pgfqpoint{4.255578in}{0.513459in}}%
\pgfpathlineto{\pgfqpoint{4.256652in}{0.511902in}}%
\pgfpathlineto{\pgfqpoint{4.257726in}{0.513045in}}%
\pgfpathlineto{\pgfqpoint{4.258800in}{0.512705in}}%
\pgfpathlineto{\pgfqpoint{4.263095in}{0.512520in}}%
\pgfpathlineto{\pgfqpoint{4.264169in}{0.513385in}}%
\pgfpathlineto{\pgfqpoint{4.266317in}{0.512520in}}%
\pgfpathlineto{\pgfqpoint{4.270612in}{0.511562in}}%
\pgfpathlineto{\pgfqpoint{4.272760in}{0.509269in}}%
\pgfpathlineto{\pgfqpoint{4.273833in}{0.506306in}}%
\pgfpathlineto{\pgfqpoint{4.277055in}{0.507478in}}%
\pgfpathlineto{\pgfqpoint{4.279203in}{0.509565in}}%
\pgfpathlineto{\pgfqpoint{4.280276in}{0.510251in}}%
\pgfpathlineto{\pgfqpoint{4.281350in}{0.510023in}}%
\pgfpathlineto{\pgfqpoint{4.284572in}{0.510909in}}%
\pgfpathlineto{\pgfqpoint{4.286719in}{0.509508in}}%
\pgfpathlineto{\pgfqpoint{4.288867in}{0.510795in}}%
\pgfpathlineto{\pgfqpoint{4.293162in}{0.510337in}}%
\pgfpathlineto{\pgfqpoint{4.294236in}{0.509565in}}%
\pgfpathlineto{\pgfqpoint{4.295310in}{0.510108in}}%
\pgfpathlineto{\pgfqpoint{4.296384in}{0.509680in}}%
\pgfpathlineto{\pgfqpoint{4.301753in}{0.511424in}}%
\pgfpathlineto{\pgfqpoint{4.302827in}{0.512453in}}%
\pgfpathlineto{\pgfqpoint{4.303901in}{0.512367in}}%
\pgfpathlineto{\pgfqpoint{4.307122in}{0.513254in}}%
\pgfpathlineto{\pgfqpoint{4.308196in}{0.511452in}}%
\pgfpathlineto{\pgfqpoint{4.309270in}{0.510795in}}%
\pgfpathlineto{\pgfqpoint{4.311418in}{0.512081in}}%
\pgfpathlineto{\pgfqpoint{4.314639in}{0.512339in}}%
\pgfpathlineto{\pgfqpoint{4.315713in}{0.508050in}}%
\pgfpathlineto{\pgfqpoint{4.316787in}{0.506781in}}%
\pgfpathlineto{\pgfqpoint{4.317861in}{0.506755in}}%
\pgfpathlineto{\pgfqpoint{4.318935in}{0.506094in}}%
\pgfpathlineto{\pgfqpoint{4.322156in}{0.506702in}}%
\pgfpathlineto{\pgfqpoint{4.324304in}{0.506279in}}%
\pgfpathlineto{\pgfqpoint{4.325378in}{0.506332in}}%
\pgfpathlineto{\pgfqpoint{4.326451in}{0.507204in}}%
\pgfpathlineto{\pgfqpoint{4.330747in}{0.504975in}}%
\pgfpathlineto{\pgfqpoint{4.333968in}{0.500577in}}%
\pgfpathlineto{\pgfqpoint{4.337190in}{0.500725in}}%
\pgfpathlineto{\pgfqpoint{4.339338in}{0.498702in}}%
\pgfpathlineto{\pgfqpoint{4.340411in}{0.499023in}}%
\pgfpathlineto{\pgfqpoint{4.344707in}{0.496876in}}%
\pgfpathlineto{\pgfqpoint{4.345781in}{0.494384in}}%
\pgfpathlineto{\pgfqpoint{4.346854in}{0.495223in}}%
\pgfpathlineto{\pgfqpoint{4.347928in}{0.493496in}}%
\pgfpathlineto{\pgfqpoint{4.349002in}{0.492854in}}%
\pgfpathlineto{\pgfqpoint{4.352224in}{0.492450in}}%
\pgfpathlineto{\pgfqpoint{4.353297in}{0.490264in}}%
\pgfpathlineto{\pgfqpoint{4.356519in}{0.489585in}}%
\pgfpathlineto{\pgfqpoint{4.360814in}{0.489246in}}%
\pgfpathlineto{\pgfqpoint{4.361888in}{0.490309in}}%
\pgfpathlineto{\pgfqpoint{4.362962in}{0.488974in}}%
\pgfpathlineto{\pgfqpoint{4.367257in}{0.488492in}}%
\pgfpathlineto{\pgfqpoint{4.369405in}{0.486685in}}%
\pgfpathlineto{\pgfqpoint{4.370479in}{0.487196in}}%
\pgfpathlineto{\pgfqpoint{4.371553in}{0.485516in}}%
\pgfpathlineto{\pgfqpoint{4.374774in}{0.483815in}}%
\pgfpathlineto{\pgfqpoint{4.375848in}{0.484692in}}%
\pgfpathlineto{\pgfqpoint{4.376922in}{0.483366in}}%
\pgfpathlineto{\pgfqpoint{4.379070in}{0.484334in}}%
\pgfpathlineto{\pgfqpoint{4.384439in}{0.483465in}}%
\pgfpathlineto{\pgfqpoint{4.386586in}{0.481214in}}%
\pgfpathlineto{\pgfqpoint{4.391956in}{0.479879in}}%
\pgfpathlineto{\pgfqpoint{4.394103in}{0.480379in}}%
\pgfpathlineto{\pgfqpoint{4.399472in}{0.478226in}}%
\pgfpathlineto{\pgfqpoint{4.400546in}{0.479073in}}%
\pgfpathlineto{\pgfqpoint{4.401620in}{0.476901in}}%
\pgfpathlineto{\pgfqpoint{4.404842in}{0.476919in}}%
\pgfpathlineto{\pgfqpoint{4.405916in}{0.477637in}}%
\pgfpathlineto{\pgfqpoint{4.406989in}{0.476496in}}%
\pgfpathlineto{\pgfqpoint{4.408063in}{0.476657in}}%
\pgfpathlineto{\pgfqpoint{4.409137in}{0.475675in}}%
\pgfpathlineto{\pgfqpoint{4.412359in}{0.474980in}}%
\pgfpathlineto{\pgfqpoint{4.413432in}{0.475883in}}%
\pgfpathlineto{\pgfqpoint{4.414506in}{0.475657in}}%
\pgfpathlineto{\pgfqpoint{4.415580in}{0.474163in}}%
\pgfpathlineto{\pgfqpoint{4.419875in}{0.474029in}}%
\pgfpathlineto{\pgfqpoint{4.422023in}{0.474980in}}%
\pgfpathlineto{\pgfqpoint{4.423097in}{0.474480in}}%
\pgfpathlineto{\pgfqpoint{4.424171in}{0.474596in}}%
\pgfpathlineto{\pgfqpoint{4.427392in}{0.473429in}}%
\pgfpathlineto{\pgfqpoint{4.429540in}{0.475197in}}%
\pgfpathlineto{\pgfqpoint{4.430614in}{0.475113in}}%
\pgfpathlineto{\pgfqpoint{4.431688in}{0.474096in}}%
\pgfpathlineto{\pgfqpoint{4.435983in}{0.474242in}}%
\pgfpathlineto{\pgfqpoint{4.439205in}{0.473432in}}%
\pgfpathlineto{\pgfqpoint{4.442426in}{0.473367in}}%
\pgfpathlineto{\pgfqpoint{4.443500in}{0.472443in}}%
\pgfpathlineto{\pgfqpoint{4.445648in}{0.472508in}}%
\pgfpathlineto{\pgfqpoint{4.446721in}{0.471357in}}%
\pgfpathlineto{\pgfqpoint{4.449943in}{0.470886in}}%
\pgfpathlineto{\pgfqpoint{4.451017in}{0.468785in}}%
\pgfpathlineto{\pgfqpoint{4.453164in}{0.468314in}}%
\pgfpathlineto{\pgfqpoint{4.454238in}{0.467783in}}%
\pgfpathlineto{\pgfqpoint{4.458534in}{0.467208in}}%
\pgfpathlineto{\pgfqpoint{4.459607in}{0.467385in}}%
\pgfpathlineto{\pgfqpoint{4.460681in}{0.465514in}}%
\pgfpathlineto{\pgfqpoint{4.461755in}{0.465806in}}%
\pgfpathlineto{\pgfqpoint{4.464977in}{0.465639in}}%
\pgfpathlineto{\pgfqpoint{4.467124in}{0.464402in}}%
\pgfpathlineto{\pgfqpoint{4.468198in}{0.464611in}}%
\pgfpathlineto{\pgfqpoint{4.469272in}{0.463624in}}%
\pgfpathlineto{\pgfqpoint{4.474641in}{0.462830in}}%
\pgfpathlineto{\pgfqpoint{4.476789in}{0.461835in}}%
\pgfpathlineto{\pgfqpoint{4.487527in}{0.462777in}}%
\pgfpathlineto{\pgfqpoint{4.488601in}{0.461822in}}%
\pgfpathlineto{\pgfqpoint{4.489675in}{0.461848in}}%
\pgfpathlineto{\pgfqpoint{4.490749in}{0.461272in}}%
\pgfpathlineto{\pgfqpoint{4.491823in}{0.461730in}}%
\pgfpathlineto{\pgfqpoint{4.503635in}{0.461848in}}%
\pgfpathlineto{\pgfqpoint{4.505782in}{0.459312in}}%
\pgfpathlineto{\pgfqpoint{4.511152in}{0.458988in}}%
\pgfpathlineto{\pgfqpoint{4.512226in}{0.459144in}}%
\pgfpathlineto{\pgfqpoint{4.514373in}{0.457906in}}%
\pgfpathlineto{\pgfqpoint{4.518669in}{0.457443in}}%
\pgfpathlineto{\pgfqpoint{4.519742in}{0.456609in}}%
\pgfpathlineto{\pgfqpoint{4.533702in}{0.455530in}}%
\pgfpathlineto{\pgfqpoint{4.534776in}{0.454688in}}%
\pgfpathlineto{\pgfqpoint{4.536924in}{0.455177in}}%
\pgfpathlineto{\pgfqpoint{4.540145in}{0.454547in}}%
\pgfpathlineto{\pgfqpoint{4.543367in}{0.452600in}}%
\pgfpathlineto{\pgfqpoint{4.548736in}{0.453481in}}%
\pgfpathlineto{\pgfqpoint{4.549810in}{0.452448in}}%
\pgfpathlineto{\pgfqpoint{4.550884in}{0.452573in}}%
\pgfpathlineto{\pgfqpoint{4.551958in}{0.452082in}}%
\pgfpathlineto{\pgfqpoint{4.555179in}{0.451978in}}%
\pgfpathlineto{\pgfqpoint{4.556253in}{0.451007in}}%
\pgfpathlineto{\pgfqpoint{4.559474in}{0.450371in}}%
\pgfpathlineto{\pgfqpoint{4.566991in}{0.448908in}}%
\pgfpathlineto{\pgfqpoint{4.572360in}{0.449111in}}%
\pgfpathlineto{\pgfqpoint{4.573434in}{0.449070in}}%
\pgfpathlineto{\pgfqpoint{4.574508in}{0.446348in}}%
\pgfpathlineto{\pgfqpoint{4.592763in}{0.448146in}}%
\pgfpathlineto{\pgfqpoint{4.593837in}{0.447750in}}%
\pgfpathlineto{\pgfqpoint{4.595985in}{0.448775in}}%
\pgfpathlineto{\pgfqpoint{4.603502in}{0.448867in}}%
\pgfpathlineto{\pgfqpoint{4.609945in}{0.449405in}}%
\pgfpathlineto{\pgfqpoint{4.612093in}{0.448501in}}%
\pgfpathlineto{\pgfqpoint{4.615314in}{0.447932in}}%
\pgfpathlineto{\pgfqpoint{4.619609in}{0.445186in}}%
\pgfpathlineto{\pgfqpoint{4.627126in}{0.444472in}}%
\pgfpathlineto{\pgfqpoint{4.630348in}{0.443940in}}%
\pgfpathlineto{\pgfqpoint{4.634643in}{0.442191in}}%
\pgfpathlineto{\pgfqpoint{4.640012in}{0.441140in}}%
\pgfpathlineto{\pgfqpoint{4.653972in}{0.441279in}}%
\pgfpathlineto{\pgfqpoint{4.655046in}{0.440396in}}%
\pgfpathlineto{\pgfqpoint{4.664711in}{0.439870in}}%
\pgfpathlineto{\pgfqpoint{4.669006in}{0.439672in}}%
\pgfpathlineto{\pgfqpoint{4.672227in}{0.439921in}}%
\pgfpathlineto{\pgfqpoint{4.684040in}{0.439655in}}%
\pgfpathlineto{\pgfqpoint{4.686187in}{0.438635in}}%
\pgfpathlineto{\pgfqpoint{4.687261in}{0.438685in}}%
\pgfpathlineto{\pgfqpoint{4.691557in}{0.438457in}}%
\pgfpathlineto{\pgfqpoint{4.693704in}{0.438906in}}%
\pgfpathlineto{\pgfqpoint{4.694778in}{0.438421in}}%
\pgfpathlineto{\pgfqpoint{4.701221in}{0.439053in}}%
\pgfpathlineto{\pgfqpoint{4.702295in}{0.438372in}}%
\pgfpathlineto{\pgfqpoint{4.709812in}{0.438486in}}%
\pgfpathlineto{\pgfqpoint{4.716255in}{0.437711in}}%
\pgfpathlineto{\pgfqpoint{4.717329in}{0.437998in}}%
\pgfpathlineto{\pgfqpoint{4.722698in}{0.437804in}}%
\pgfpathlineto{\pgfqpoint{4.723772in}{0.438012in}}%
\pgfpathlineto{\pgfqpoint{4.729141in}{0.437477in}}%
\pgfpathlineto{\pgfqpoint{4.738805in}{0.436842in}}%
\pgfpathlineto{\pgfqpoint{4.739879in}{0.436842in}}%
\pgfpathlineto{\pgfqpoint{4.743101in}{0.435915in}}%
\pgfpathlineto{\pgfqpoint{4.744175in}{0.434879in}}%
\pgfpathlineto{\pgfqpoint{4.746322in}{0.434481in}}%
\pgfpathlineto{\pgfqpoint{4.747396in}{0.433662in}}%
\pgfpathlineto{\pgfqpoint{4.754913in}{0.433813in}}%
\pgfpathlineto{\pgfqpoint{4.760282in}{0.433351in}}%
\pgfpathlineto{\pgfqpoint{4.761356in}{0.433668in}}%
\pgfpathlineto{\pgfqpoint{4.762430in}{0.433231in}}%
\pgfpathlineto{\pgfqpoint{4.766725in}{0.432888in}}%
\pgfpathlineto{\pgfqpoint{4.768873in}{0.433157in}}%
\pgfpathlineto{\pgfqpoint{4.769947in}{0.433397in}}%
\pgfpathlineto{\pgfqpoint{4.777464in}{0.433118in}}%
\pgfpathlineto{\pgfqpoint{4.790350in}{0.431905in}}%
\pgfpathlineto{\pgfqpoint{4.792497in}{0.430531in}}%
\pgfpathlineto{\pgfqpoint{4.797867in}{0.430550in}}%
\pgfpathlineto{\pgfqpoint{4.798940in}{0.430555in}}%
\pgfpathlineto{\pgfqpoint{4.800014in}{0.429813in}}%
\pgfpathlineto{\pgfqpoint{4.810753in}{0.428598in}}%
\pgfpathlineto{\pgfqpoint{4.811826in}{0.428911in}}%
\pgfpathlineto{\pgfqpoint{4.815048in}{0.427907in}}%
\pgfpathlineto{\pgfqpoint{4.820417in}{0.428096in}}%
\pgfpathlineto{\pgfqpoint{4.822565in}{0.428070in}}%
\pgfpathlineto{\pgfqpoint{4.830082in}{0.428336in}}%
\pgfpathlineto{\pgfqpoint{4.849411in}{0.426652in}}%
\pgfpathlineto{\pgfqpoint{4.855854in}{0.426712in}}%
\pgfpathlineto{\pgfqpoint{4.858002in}{0.426724in}}%
\pgfpathlineto{\pgfqpoint{4.859075in}{0.426981in}}%
\pgfpathlineto{\pgfqpoint{4.860149in}{0.426561in}}%
\pgfpathlineto{\pgfqpoint{4.864445in}{0.426029in}}%
\pgfpathlineto{\pgfqpoint{4.867666in}{0.425243in}}%
\pgfpathlineto{\pgfqpoint{4.888069in}{0.423912in}}%
\pgfpathlineto{\pgfqpoint{4.895586in}{0.424016in}}%
\pgfpathlineto{\pgfqpoint{4.900955in}{0.424179in}}%
\pgfpathlineto{\pgfqpoint{4.905250in}{0.422551in}}%
\pgfpathlineto{\pgfqpoint{4.911693in}{0.422270in}}%
\pgfpathlineto{\pgfqpoint{4.915989in}{0.421992in}}%
\pgfpathlineto{\pgfqpoint{4.919210in}{0.422131in}}%
\pgfpathlineto{\pgfqpoint{4.926727in}{0.421781in}}%
\pgfpathlineto{\pgfqpoint{4.927801in}{0.421411in}}%
\pgfpathlineto{\pgfqpoint{4.933170in}{0.420938in}}%
\pgfpathlineto{\pgfqpoint{4.934244in}{0.420386in}}%
\pgfpathlineto{\pgfqpoint{4.938539in}{0.420711in}}%
\pgfpathlineto{\pgfqpoint{4.942835in}{0.420793in}}%
\pgfpathlineto{\pgfqpoint{4.947130in}{0.420551in}}%
\pgfpathlineto{\pgfqpoint{4.950352in}{0.420150in}}%
\pgfpathlineto{\pgfqpoint{4.964312in}{0.419636in}}%
\pgfpathlineto{\pgfqpoint{4.965385in}{0.419288in}}%
\pgfpathlineto{\pgfqpoint{4.970755in}{0.418818in}}%
\pgfpathlineto{\pgfqpoint{4.972902in}{0.418577in}}%
\pgfpathlineto{\pgfqpoint{5.053440in}{0.415479in}}%
\pgfpathlineto{\pgfqpoint{5.055588in}{0.415056in}}%
\pgfpathlineto{\pgfqpoint{5.068474in}{0.413912in}}%
\pgfpathlineto{\pgfqpoint{5.098541in}{0.413423in}}%
\pgfpathlineto{\pgfqpoint{5.100689in}{0.413260in}}%
\pgfpathlineto{\pgfqpoint{5.141495in}{0.411705in}}%
\pgfpathlineto{\pgfqpoint{5.158676in}{0.411831in}}%
\pgfpathlineto{\pgfqpoint{5.160824in}{0.411726in}}%
\pgfpathlineto{\pgfqpoint{5.196261in}{0.411739in}}%
\pgfpathlineto{\pgfqpoint{5.198408in}{0.411588in}}%
\pgfpathlineto{\pgfqpoint{5.226328in}{0.411511in}}%
\pgfpathlineto{\pgfqpoint{5.243510in}{0.411418in}}%
\pgfpathlineto{\pgfqpoint{5.254248in}{0.411163in}}%
\pgfpathlineto{\pgfqpoint{5.266060in}{0.411099in}}%
\pgfpathlineto{\pgfqpoint{5.299349in}{0.411169in}}%
\pgfpathlineto{\pgfqpoint{5.315457in}{0.410878in}}%
\pgfpathlineto{\pgfqpoint{5.322974in}{0.410435in}}%
\pgfpathlineto{\pgfqpoint{5.326195in}{0.410365in}}%
\pgfpathlineto{\pgfqpoint{5.348746in}{0.409439in}}%
\pgfpathlineto{\pgfqpoint{5.391699in}{0.409236in}}%
\pgfpathlineto{\pgfqpoint{5.401364in}{0.408914in}}%
\pgfpathlineto{\pgfqpoint{5.423914in}{0.408838in}}%
\pgfpathlineto{\pgfqpoint{5.450760in}{0.408573in}}%
\pgfpathlineto{\pgfqpoint{5.452908in}{0.408105in}}%
\pgfpathlineto{\pgfqpoint{5.479754in}{0.407929in}}%
\pgfpathlineto{\pgfqpoint{5.489419in}{0.407674in}}%
\pgfpathlineto{\pgfqpoint{5.491566in}{0.407464in}}%
\pgfpathlineto{\pgfqpoint{5.540963in}{0.407478in}}%
\pgfpathlineto{\pgfqpoint{5.662307in}{0.406585in}}%
\pgfpathlineto{\pgfqpoint{5.664454in}{0.406350in}}%
\pgfpathlineto{\pgfqpoint{5.700965in}{0.405659in}}%
\pgfpathlineto{\pgfqpoint{5.993049in}{0.404171in}}%
\pgfpathlineto{\pgfqpoint{6.174528in}{0.403968in}}%
\pgfpathlineto{\pgfqpoint{6.243253in}{0.403891in}}%
\pgfpathlineto{\pgfqpoint{6.589029in}{0.403622in}}%
\pgfpathlineto{\pgfqpoint{6.589029in}{0.403622in}}%
\pgfusepath{stroke}%
\end{pgfscope}%
\begin{pgfscope}%
\pgfsetrectcap%
\pgfsetmiterjoin%
\pgfsetlinewidth{0.803000pt}%
\definecolor{currentstroke}{rgb}{1.000000,1.000000,1.000000}%
\pgfsetstrokecolor{currentstroke}%
\pgfsetdash{}{0pt}%
\pgfpathmoveto{\pgfqpoint{4.123120in}{0.385400in}}%
\pgfpathlineto{\pgfqpoint{4.123120in}{0.786285in}}%
\pgfusepath{stroke}%
\end{pgfscope}%
\begin{pgfscope}%
\pgfsetrectcap%
\pgfsetmiterjoin%
\pgfsetlinewidth{0.803000pt}%
\definecolor{currentstroke}{rgb}{1.000000,1.000000,1.000000}%
\pgfsetstrokecolor{currentstroke}%
\pgfsetdash{}{0pt}%
\pgfpathmoveto{\pgfqpoint{6.706453in}{0.385400in}}%
\pgfpathlineto{\pgfqpoint{6.706453in}{0.786285in}}%
\pgfusepath{stroke}%
\end{pgfscope}%
\begin{pgfscope}%
\pgfsetrectcap%
\pgfsetmiterjoin%
\pgfsetlinewidth{0.803000pt}%
\definecolor{currentstroke}{rgb}{1.000000,1.000000,1.000000}%
\pgfsetstrokecolor{currentstroke}%
\pgfsetdash{}{0pt}%
\pgfpathmoveto{\pgfqpoint{4.123120in}{0.385400in}}%
\pgfpathlineto{\pgfqpoint{6.706453in}{0.385400in}}%
\pgfusepath{stroke}%
\end{pgfscope}%
\begin{pgfscope}%
\pgfsetrectcap%
\pgfsetmiterjoin%
\pgfsetlinewidth{0.803000pt}%
\definecolor{currentstroke}{rgb}{1.000000,1.000000,1.000000}%
\pgfsetstrokecolor{currentstroke}%
\pgfsetdash{}{0pt}%
\pgfpathmoveto{\pgfqpoint{4.123120in}{0.786285in}}%
\pgfpathlineto{\pgfqpoint{6.706453in}{0.786285in}}%
\pgfusepath{stroke}%
\end{pgfscope}%
\begin{pgfscope}%
\definecolor{textcolor}{rgb}{0.150000,0.150000,0.150000}%
\pgfsetstrokecolor{textcolor}%
\pgfsetfillcolor{textcolor}%
\pgftext[x=5.414787in,y=0.869619in,,base]{\color{textcolor}\rmfamily\fontsize{16.800000}{20.160000}\selectfont DIS}%
\end{pgfscope}%
\begin{pgfscope}%
\definecolor{textcolor}{rgb}{0.150000,0.150000,0.150000}%
\pgfsetstrokecolor{textcolor}%
\pgfsetfillcolor{textcolor}%
\pgftext[x=3.506453in,y=5.834781in,,top]{\color{textcolor}\rmfamily\fontsize{16.800000}{20.160000}\selectfont Return Trading Strategy vs. Buy and Hold}%
\end{pgfscope}%
\end{pgfpicture}%
\makeatother%
\endgroup%

    \end{adjustbox}  
    \caption{Mean Reversion based on past returns for single stocks}
    \label{fig:mean_reversion_returns}
\end{figure}{}

\subsubsection{Comparison of 90d and 30d moving averages}
Idea: If 30d averages is above 90d average, then sell. And vice versa

\subsubsection{Mean Reversion Portfolio}
Idea: Look at entire portfolio. 




\subsection{Momentum Based Trading}


\subsection{Pairs Trading}



\section{Trading Strategies Based on Our Predictions}



\section{Hybrid Trading Strategies}
Take Mean of different predictions?


